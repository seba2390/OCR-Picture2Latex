% File tacl2021v1.tex
% Dec. 15, 2021

% The English content of this file was modified from various *ACL instructions
% by Lillian Lee and Kristina Toutanova
%
% LaTeXery is mostly all adapted from acl2018.sty.

\documentclass[11pt,a4paper]{article}
\usepackage{times,latexsym}
\usepackage{url}
\usepackage[T1]{fontenc}
%%%%%%%%%%%%%% citation 
%\usepackage[sorting=none, maxbibnames=1, maxcitenames=1, backend=biber]{biblatex} % load the package
%\usepackage[sorting=none, backend=biber, stylename=ieee, giveninits=true]{biblatex} % load the package
% \usepackage[sorting=none, backend=bibtex, stylename=ieee, giveninits=true]{biblatex} % load the package
% \addbibresource{references.bib} % add a bib-reference file

%% Package options:
%% Short version: "hyperref" and "submission" are the defaults.
%% More verbose version:
%% Most compact command to produce a submission version with hyperref enabled
%%    \usepackage[]{tacl2021v1}
%% Most compact command to produce a "camera-ready" version
   \usepackage[acceptedWithA]{tacl2021v1}
%% Most compact command to produce a double-spaced copy-editor's version
%%    \usepackage[acceptedWithA,copyedit]{tacl2021v1}
%
%% If you need to disable hyperref in any of the above settings (see Section
%% "LaTeX files") in the TACL instructions), add ",nohyperref" in the square
%% brackets. (The comma is a delimiter in case there are multiple options specified.)

\usepackage{tacl2021v1}
% \setlength\titlebox{10cm} % <- for Option 2 below

%%%% Material in this block is specific to generating TACL instructions
\usepackage{xspace,mfirstuc,tabulary}
\newcommand{\dateOfLastUpdate}{Dec. 15, 2021}
\newcommand{\styleFileVersion}{tacl2021v1}

\newcommand{\ex}[1]{{\sf #1}}

\newif\iftaclinstructions
\taclinstructionsfalse % AUTHORS: do NOT set this to true
\iftaclinstructions
\renewcommand{\confidential}{}
\renewcommand{\anonsubtext}{(No author info supplied here, for consistency with
TACL-submission anonymization requirements)}
\newcommand{\instr}
\fi

%
\iftaclpubformat % this "if" is set by the choice of options
\newcommand{\taclpaper}{final version\xspace}
\newcommand{\taclpapers}{final versions\xspace}
\newcommand{\Taclpaper}{Final version\xspace}
\newcommand{\Taclpapers}{Final versions\xspace}
\newcommand{\TaclPapers}{Final Versions\xspace}
\else
\newcommand{\taclpaper}{submission\xspace}
\newcommand{\taclpapers}{{\taclpaper}s\xspace}
\newcommand{\Taclpaper}{Submission\xspace}
\newcommand{\Taclpapers}{{\Taclpaper}s\xspace}
\newcommand{\TaclPapers}{Submissions\xspace}
\fi

\usepackage{graphicx}
\usepackage{amsmath,bm}
\usepackage{amsfonts}
\usepackage{multirow}
\usepackage{booktabs}
\usepackage{microtype}
\usepackage{adjustbox}
\usepackage{tikz}
\usepackage{pgf}

%%%% End TACL-instructions-specific macro block
%%%%

% \title{A Survey on the Interpretation of Large Language Models}
% \title{On the Interpretation of Emergent Abilities of Large Language Models: A Survey}
% \title{On the Interpretation of In-context Learning: A Survey}

%The Mystery and Fascination of LLMs: A Comprehensive Survey on the Interpretation and Analysis of Emergent Abilities
%Survey on the Interpretation and Analysis of Mysterious and Fascinating Emergent Abilities in LLMs
%Beyond Mystery: A Comprehensive Survey of Interpreting and Analyzing Emerge
%On the Interpretation of Emergent Abilities in Large Language Models: A Survey
%Interpreting Emergent Abilities of Large Language Models: A Survey
\title{The Mystery and Fascination of LLMs: A Comprehensive Survey \\
on the Interpretation and Analysis of Emergent Abilities}


% Author information does not appear in the pdf unless the "acceptedWithA" option is given

% The author block may be formatted in one of two ways:

% Option 1. Author’s address is underneath each name, centered.

% \author{
%   Template Author1\Thanks{The {\em actual} contributors to this instruction
%     document and corresponding template file are given in Section
%     \ref{sec:contributors}.} 
%   \\
%   Template Affiliation1/Address Line 1
%   \\
%   Template Affiliation1/Address Line 2
%   \\
%   Template Affiliation1/Address Line 2
%   \\
%   \texttt{template.email1example.com}
%   \And
%   Template Author2 
%   \\
%   Template Affiliation2/Address Line 1
%   \\
%   Template Affiliation2/Address Line 2
%   \\
%   Template Affiliation2/Address Line 2
%   \\
%   \texttt{template.email2@example.com}
% }

% % Option 2.  Author’s address is linked with superscript
% % characters to its name, author names are grouped, centered.

% \author{
%   Template Author1\Thanks{The {\em actual} contributors to this instruction
%     document and corresponding template file are given in Section
%     \ref{sec:contributors}.}$^\diamond$ 
%   \and
%   Template Author2$^\dagger$
%   \\
%   \ \\
%   $^\diamond$Template Affiliation1/Address Line 1
%   \\
%   Template Affiliation1/Address Line 2
%   \\
%   Template Affiliation1/Address Line 2
%   \\
%   \texttt{template.email1example.com}
%   \\
%   \ \\
%   \\
%   $^\dagger$Template Affiliation2/Address Line 1
%   \\
%   Template Affiliation2/Address Line 2
%   \\
%   Template Affiliation2/Address Line 2
%   \\
%   \texttt{template.email2@example.com}
% }
\author{Yuxiang Zhou$^1$, Jiazheng Li$^1$, Yanzheng Xiang$^{1}$, Hanqi Yan$^{1,2}$, Lin Gui$^{1}$, Yulan He$^{1,2,3}$ \\
$^{1}$King's College London, $^{2}$University of Warwick, $^{3}$The Alan Turing Institute\\
        \texttt{\{yuxiang.zhou,jiazheng.li,yanzheng.xiang\}@kcl.ac.uk}\\
       	\texttt{hanqi.yan@warwick.ac.uk}\\
        \texttt{\{lin.1.gui,yulan.he\}@kcl.ac.uk}}
\date{}

\begin{document}
\maketitle
% Briefly outline the purpose of the paper, its scope, key findings and implications.
% \begin{itemize}
% 	\item Background: Why interpretation of the emergent abilities (e.g., ICL) of LLMs is important.
% 	\item Existing works: How and what have they done to ICL interpretation.
% 	\item What we have done in this paper (high-level), and what we have done in detail.
% \end{itemize}
\begin{abstract}
Understanding emergent abilities, %~\cite{Wei2022EmergentAO}, 
such as in-context learning (ICL) and chain-of-thought (CoT) prompting 
in large language models (LLMs), is of utmost importance.
This importance stems not only from the better utilization of these capabilities across various tasks, but also from the proactive identification and mitigation of potential risks, including concerns of truthfulness, bias, and toxicity, that may arise alongside these capabilities.
In this paper, we present a thorough survey on the interpretation and analysis of emergent abilities of LLMs.
First, we provide a concise introduction to the background and definition of emergent abilities.
Then, we give an overview of advancements from two perspectives: 1) a macro perspective, emphasizing studies on the mechanistic interpretability and delving into the mathematical foundations behind emergent abilities; and 2) a micro-perspective, concerning studies that focus on empirical interpretability by examining factors associated with these abilities.
We conclude by highlighting the challenges encountered and suggesting potential avenues for future research.
We believe that our work establishes the basis for further exploration into the interpretation of emergent abilities. 
Additionally, we have created a repository\footnote{\url{www.github.xxx}} containing the resources referenced in our survey.
\end{abstract}

% While existing ICL surveys focus on comprehensive reviews, we narrow the scope to survey the analytical studies on ICL.
% More specifically, we target those works that attempt to interpret ICL from two different perspectives.
% \begin{itemize}
% \item  Background: Define ICL and highlight their growing importance in various fields. 
%  \item Motivation: Discuss the necessity for interpreting ICL, and emphasizing background in a more detail way.
%  \item Objective: State the aim of your paper, i.e., to review recent research on interpreting ICL.
%  \item Structure: Provide an overview of the rest of the paper.
% \end{itemize}
\section{Introduction}
Emergent abilities such as in-context learning (ICL) and chain-of-thought (CoT) prompting have become evident in large language models (LLMs) when they are scaled to certain levels~\cite{Wei2022EmergentAO}.
These capabilities are receiving heightened attention due to their remarkable adaptability and their parameter-free nature.
As shown in Figure~\ref{fig:intro}, LLMs such as LLaMA2~\cite{Touvron2023Llama2O} and GPT-4~\cite{OpenAI2023GPT4TR} have exhibited proficiency across various tasks, such as text classification, summarization, and question answering, with a minimal set of task-oriented examples or CoT demonstrations, all without the need for extensive re-training.
\begin{figure}
	\centering
	{\includegraphics[width=0.5\textwidth]{intro.fig.pdf}\label{vis(architure)}}
	\caption{Illustration of Emergent Abilities.}
 \label{fig:intro}
\end{figure}

The concept of emergent abilities within LLMs was originally introduced by~\citet{Wei2022EmergentAO}, defining them as \textit{capabilities that manifest in large-scale models but are absent in their smaller-scale counterparts}.
They further classified these abilities into two categories: 1)~\textit{few-shot prompting abilities}, referring to the capacity of LLMs to achieve significantly better results than random chance on certain tasks such as BIG-Bench~\cite{Srivastava2022BeyondTI} and TruthfulQA~\cite{Lin2021TruthfulQAMH}, when presented with only a small number of demonstration examples; % by few-shot prompting; 
2)~\textit{augmented prompting strategies}, where certain strategies produce less impressive outcomes compared to established baselines until applied to models of sufficient scale.
For example, this includes the chain-of-thought prompting strategy~\cite{Wei2022ChainOT,Suzgun2022ChallengingBT} and instruction tuning~\cite{Brown2020LanguageMA, Wei2021FinetunedLM,Chung2022ScalingIL}.

Many studies have investigated emergent abilities of LLMs. %falling under the first category.
\citet{Srivastava2022BeyondTI} introduced BIG-bench, encompassing 204 tasks designed to push the boundaries of what current LLMs can do. % capabilities of LLMs, 
They aim to systematically assess and extrapolate the emergent abilities of LLMs.
\citet{Bubeck2023SparksOA} presented an overview of the emergent abilities specifically related to GPT-4~\cite{OpenAI2023GPT4TR}.
%Another line of research has focused on abilities associated with the second category.
\citet{Dong2022ASO} and \citet{Chu2023ASO} summarized advancements in techniques related to ICL and CoT, respectively, within LLMs.
Beyond the scope of emergent abilities, several surveys have been conducted to consolidate research on other aspects of LLMs.
\citet{Liang2022HolisticEO}, \citet{Chang2023ASO}, and \citet{Srivastava2022BeyondTI} emphasized studies on the evaluation methodologies of LLMs.
Meanwhile, \citet{Huang2022TowardsRI} and \citet{Qiao2022ReasoningWL} conducted surveys on research addressing the reasoning abilities inherent in LLMs.
Furthermore, \citet{Cao2023ACS}, \citet{Zhou2023ACS}, and \citet{Yang2023HarnessingTP} provided summaries of various viewpoints regarding the interplay between the development of LLMs and ChatGPT~\cite{Ouyang2022TrainingLM}. 
\citet{Zhao2023ExplainabilityFL} conducted a survey on methods used to elucidate pre-trained LMs.

While these surveys serve the purpose of providing an overview of the progress in various facets of LLMs, they tend to be fragmented and place significant emphasis on studies that assess the effectiveness and performance of LLMs in specific tasks.
Furthermore, even though emergent abilities have demonstrated increasing success across various domains, our understanding of these abilities remains limited.
Recently, an increasing number of studies have attempted to interpret and analyze emergent abilities.
\citet{Garg2022WhatCT}, \citet{Dai2023WhyCG}, and \citet{Akyrek2023WHL} explained ICL through the lens of linear regression formulation.
\citet{Xie2021AnEO}, \citet{Wang2023LargeLM}, and \citet{Hahn2023ATO} provided an interpretation of ICL rooted in latent variable models.
Meanwhile, a distinct line of research has aimed to understand the influential factors affecting emergent abilities through empirical analyses.
\citet{Min2022RethinkingTR}, \citet{Wei2023LargerLM}, \citet{Wang2023LabelWA}, and \citet{Kim2022GroundTruthLM} demonstrated that the ICL performance is influenced by task-specific characteristics and multiple facets of ICL instances, including quantities, order, and flipped labels. %revise
Consequently, it is essential to systematically categorize and summarize these studies, not only for a deeper understanding and more effective utilization of emergent abilities across various tasks, but also to assist in anticipating and mitigating potential risks. These risks encompass concerns related to truthfulness, bias, and toxicity, that may arise alongside these capabilities.

In this paper, we present a thorough and organized survey of the research on the interpretation and analysis of emergent abilities.
First, we provide a brief introduction of the background and offer the definition of emergent abilities.
Then, we present a comprehensive overview of advancements, from two distinct viewpoints: 1) a macro perspective, encapsulating studies focused on mechanistic interpretability and theoretical investigations into the mathematical foundations of emergent abilities; 
and 2) a micro perspective, pertaining to studies that prioritize empirical interpretability by probing factors associated with these abilities. 
In conclusion, we highlight the existing challenges and suggest potential avenues for further research. % in this field.

\begin{table*}[t!]
  \centering
  \resizebox{\linewidth}{!}{
    \begin{tabular}{lllll}\toprule
    \textbf{Work} & \textbf{EA} & \textbf{Key Words} & \textbf{Models} & \textbf{Tasks} \\%\midrule
   		\midrule
   		\multicolumn{5}{c}{\textbf{Macro Perspective}}\\
    	\midrule 
        \citep{elhage2021mathematical} & ICL   & Mechanistic Interpretability & Transformer$^\dag$ & - \\
        \citep{Olsson2022IncontextLA} & ICL   & Mechanistic Interpretability & Transformer$^\dag$ & - \\
        \citep{Garg2022WhatCT} & ICL   & Regression Function Learning & Transformer$^\dag$ & Regression \\
        \citep{Li2023TransformersAA} & ICL  & Regression Function Learning & Transformer$^\dag$ & Regression \\
        \citep{Li2023TheCO} & ICL   & Regression Function Learning & Transformer$^\dag$ & Regression \\
        \citep{Akyrek2023WHL} & ICL   & Regression Function Learning & Transformer$^\dag$ & Regression \\
        \citep{Dai2023WhyCG} & ICL   & Gradient Descent, Meta-Optimization & GPT Family & Classification \\
        \citep{Oswald2022TransformersLI} & ICL   & Gradient Descent, Meta-Optimization & Transformer$^\dag$ & Regression \\
        \citep{Xie2021AnEO} & ICL   & Bayesian inference & Transformer$^\dag$, LSTM & Sythetic Generation  \\
        \citep{Wang2023LargeLM} & ICL   & Bayesian inference & GPT Family & Classification \\
        \citep{Jiang2023ALS} & ICL, CoT   & Bayesian inference & GPT$^\dag$&Sythetic Generation \\
        \citep{Zhang2023WhatAH} & ICL   & Bayesian inference & Transformer$^\dag$ & - \\
        
        % \citep{swaminathan2023schema} & ICL   & Bayesian inference & Clone-Structured Causal Graphs & Classification \\
    %\midrule
    	\midrule
    	\multicolumn{5}{c}{\textbf{Micro Perspective}}\\
    	\midrule 
            \citep{Shin2022OnTE} & ICL   &  \textsc{Data} Domain & GPT-3 & Classification, Translation \\
          \citep{Han2023UnderstandingIL} & ICL   &  \textsc{Data} Domain, \textsc{Data} Distribution & OPT Family & Classification \\
             \cite{Raventos2023PretrainingTD} & ICL & Task Diversity & GPT-2 & Regression \\
           \citep{Razeghi2022ImpactOP} & ICL   &  \textsc{Data} Term frequency & GPT Family & Reasoning \\
             \citep{Kandpal2022LargeLM} & ICL   &  \textsc{Data} Term frequency & BLOOM & QA \\
          \citep{Chan2022DataDP} & ICL   & \textsc{Data} Distribution & Transformer & Classification \\
          \citep{Wies2023TheLO} & ICL   & \textsc{Data} Distribution & GPT-2 & - \\
        \cite{Tay2022UnifyingLL} & ICL & \textsc{Data} Diversity & UL2, T5 ,GPT & Classification, QA, Reasoning \\
          \citep{Wei2022EmergentAO} & ICL, CoT   & Model Scale & GPT-3, Flan Family, LaMDA Family & Classification \\
          % \citep{Kirsch2022GeneralPurposeIL} & ICL   & Hidden state size & Transformer & Classification \\
          % \citep{Xie2021AnEO} & ICL   &  \textsc{Data} Distribution, Model Architecture & Transformer, LSTM & Classification \\
          % \citep{Berglund2023TheRC} & ICL   &  \textsc{Data} Term frequency & GPT Family, Llama-1 & QA \\
          \citep{Lu2021FantasticallyOP} & ICL   & Demonstration Order & GPT Family & Classification \\
          % \citep{an2023context} & ICL   & Order & GPT Family & Semantic Parsing \\
          % , Label Distribution
          % & \citep{Liu2023LostIT} & ICL   & Text Demonstration Order & QA \\
          \citep{Zhao2021CalibrateBU} & ICL   & Demonstration & GPT Family & Classification, Information Retrieval \\
          \citep{Liu2021WhatMG} & ICL   & Demonstration Order & GPT-3 & QA, Classification, Text Generation \\
          \citep{Min2022RethinkingTR} & ICL   & Input-Label Mapping & GPT-3, fairseq Family, etc. & Classification, Multi-choice Tasks \\
          \citep{Kossen2023InContextLI} & ICL   & Input-Label Mapping &  LLaMa Family, Falcon Family & Classification, QA \\
          \citep{Wei2023LargerLM} & ICL   & Input-Label Mapping & GPT-3 Family, PaLM Family, etc. & Classification \\
          \citep{Kim2022GroundTruthLM} & ICL   & Input-Label Mapping & GPT Family & Classification \\
          \citep{Tang2023LargeLM} & ICL   & Input-Label Mapping, Shortcuts & GPT Family, OPT Family & Classification \\
          \citep{Si2023MeasuringIB} & ICL   & Input-Label Mapping, Feature Bias & text-davinci-002, GPT-3 & Classification \\
          \citep{Wang2023LabelWA} & ICL   & Input-Label Mapping, Information Flow & GPT-J & Classification \\
          \citep{Wang2022TowardsUC} & CoT   & Demonstration & PaLM, text-davinci-003, etc. & Reasoning, QA \\
          \citep{Turpin2023LanguageMD} & CoT   & Demonstration & GPT-3.5, Claude 1.0 & Multi-choice Tasks  \\
    \bottomrule
    \end{tabular}}%`
\caption{Summary of research studies on the interpretation of emergent abilities in LLMs. EA is short for ``Emergent Abilities'' and QA stands for ``Question Answering''. \textsc{Data} refers to pre-training data. The symbol $^\dag$ denotes specifically designed models.}
  \label{tab:summ}%
\end{table*}%

% \begin{table*}[t!]
%   \centering
%   \resizebox{\linewidth}{!}{
%     \begin{tabular}{cp{0.3\linewidth}lll}\toprule
%     \textbf{Perspective} & \textbf{Work} & \textbf{Emergent Ability} & \textbf{\textcolor{white}{~~~~~~~~~~~~~~~~~~}Key Words} & \textbf{Tasks} \\\midrule
%     \multirow{9}[1]{*}{Macro} & \citep{elhage2021mathematical} & ICL   & Mechanistic Interpretability & - \\
%           & \citep{Olsson2022IncontextLA} & ICL   & Mechanistic Interpretability & - \\
%         & \citep{Garg2022WhatCT} & ICL   & Regression Function Learning & Regression \\
%         & \citep{Li2023TransformersAA} & ICL   & Regression Function Learning & Regression \\
%         & \citep{Li2023TheCO} & ICL   & Regression Function Learning & Regression \\
%         & \citep{Akyrek2023WHL} & ICL   & Regression Function Learning & Regression \\
%         & \citep{Dai2023WhyCG} & ICL   & Gradient Descent, Meta-Optimization & Classification \\
%           & \citep{Oswald2022TransformersLI} & ICL   & Gradient Descent, Meta-Optimization & Regression \\
%           & \citep{Xie2021AnEO} & ICL   & Bayesian inference & Classification \\
%           & \citep{swaminathan2023schema} & ICL   & Bayesian inference & Classification \\
%           & \citep{DBLP:journals/corr/abs-2304-09960} & ICL   & Bayesian inference & Classification \\
%     \midrule
%     \multirow{19}[2]{*}{Micro} & \citep{Chan2022DataDP} & ICL   & Pre-training Data Distribution, Model Architecture & Classification \\
%           & \citep{Razeghi2022ImpactOP} & ICL   & Training Data Term frequency  & Arithmetic Reasoning \\
%           & \citep{DBLP:conf/icml/KandpalDRWR23} & ICL   & Training Data Term frequency  & QA \\
%           & \citep{DBLP:conf/naacl/ShinLAKKKCLPHS22} & ICL   & Training Data Domain, Lexical Diversity & Classification, Translation \\
%           & \citep{Power2022GrokkingGB} & ICL   & Lexical Diversity, Training Data Quality & Arithmetic Reasoning \\
%           & \citep{Wei2022EmergentAO} & ICL   & Model Scale and Training Data Quality & Classification \\
%           & \citep{DBLP:journals/corr/abs-2212-04458} & ICL   & Hidden state size & Classification \\
%           & \citep{DBLP:conf/acl/HanSMTCW23} & ICL   & Training Data Domain, Pre-training Data Distribution & Classification \\
%           & \citep{Xie2021AnEO} & ICL   & Pre-training Data Distribution, Model Architecture & Classification \\
%           & \citep{Berglund2023TheRC} & ICL   & Training Data Term frequency  & QA \\
%           & \citep{Lu2021FantasticallyOP} & ICL   & Order, Label Distribution & Classification \\
%           % & \citep{Liu2023LostIT} & ICL   & Text Demonstration Order & QA \\
%           & \citep{Zhao2021CalibrateBU} & ICL   & Order & Classification \\
%           & \citep{Liu2021WhatMG} & ICL   & Order, Semantic Similarity & QA \\
%           & \citep{min-etal-2022-rethinking} & ICL   & Input-Label Mapping & Classification \\
%           & \citep{Kossen2023InContextLI} & ICL   & Input-Label Mapping & Classification \\
%           & \citep{Wei2023LargerLM} & ICL   & Input-Label Mapping & Classification \\
%           & \citep{Kim2022GroundTruthLM} & ICL   & Input-Label Mapping & Classification \\
%           & \citep{Tang2023LargeLM} & ICL   & Input-Label Mapping, Shortcuts within Demonstrations & Classification \\
%           & \citep{Si2023MeasuringIB} & ICL   & Input-Label Mapping, Feature Bias & Classification \\
%           & \citep{Wang2023LabelWA} & ICL   & Input-Label Mapping, Information Flow & Classification \\
%           & \citep{Wang2022TowardsUC} & CoT   & Demonstration & Arithmetic Reasoning, QA \\
%           & \citep{Turpin2023LanguageMD} & CoT   & Demonstration & Multiple-choice Tasks \\
%     \bottomrule
%     \end{tabular}}%`
% \caption{Summary of research studies on the interpretation of emergent abilities in LLMs.}
%   \label{tab:addlabel}%
% \end{table*}%
\section{Background and Notation}
\label{sec:bg}
%could use the term 'reframe' and give a brief rationle to explain the resaon... and highlight ICL can be category into our definitation:  refer to lu's paper are emergent ability xxx about the definition
We refine the definition in \citet{Wei2022EmergentAO}, and specify the emergent abilities as \textit{the abilities that LLMs can leverage to achieve \textit{satisfactory results}\footnote{Results that on par with or potentially superior to those achieved by LLMs fine-tuned for specific tasks or datasets.} across diverse tasks, with only a few examples or chain-of-thought demonstrations, and without the need for re-training.}
%This definition closely corresponds to the second category outlined in  \citet{Wei2022EmergentAO}'s original definition.
%our scope to survey the analytical studies centred on the interpretation of emergent abilities of LLMs.
%Following~\citet{Wei2022EmergentAO}, we define the emergent abilities of LLMs as \textit{the abilities that emerge when LLMs scale up to certain extents and cannot be anticipated only by extrapolating the performance improvements of smaller-scale models.}
%Need to revise: cannot be... to similar to the cited paper
%Specifically, with a few examples or chain-of-thought demonstrations, LLMs can produce \textit{satisfactory results}\footnote{Comparable or even better results than those achieved by LLMs fine-tuned on specifica tasks or datasets.} across various tasks without requiring continuous training or fine-tuning.

%D=\{(x_i,y_i)\};Q=(q_1,...,q_j);Y=(y_1,...,y_j)
Formally, let's define some key variables. $D \in \mathcal T_{train}$ represents a subset of demonstrations selected from the training set.
$Q\in \mathcal T_{test}$ is the query taken from the test set, and $Y$ stands for the label associated with each query. $\mathcal M$ represents the LLM with its parameters frozen as $\Theta$, and $\mathcal F$ denotes the evaluation metric function. 
For example, $\mathcal F$ is typically used to measure accuracy or F1 score in classification tasks, such as sentiment classification, and is often used to represent metrics like ROUGE~\cite{Lin2004ROUGEAP} or BLEU~\cite{Papineni2002BleuAM} in text generation tasks, such as summarization and machine translation.
The concept of emergent ability can be formally expressed using the equation:
\begin{equation}
	\mathcal F (Y,\mathcal M_\Theta(D,Q))
\end{equation}
where $\mathcal M$ is usually considered to exhibit emergent abilities if the computed value using $\mathcal F$ exceeds a pre-defined threshold. 
Under this definition, we can group similar concepts within the few-shot prompting paradigm. 
CoT can be viewed as a variant of ICL, with the primary distinction being the format of the demonstration. 
Specifically, ICL demonstrations typically rely on a standard prompt with optional demonstration examples, whereas  CoT prompting incorporates an additional textual reasoning process.

According to our definition, we organize existing literature (summarized in Table~\ref{tab:summ}) on interpreting emergent capabilities into macro and micro perspectives. 
Researchers in the macro category focus on factors such as overall loss or the model architecture. Their goal is to establish a connection between the outcome of $\mathcal F$ and the behavior of $\mathcal M$.
Conversely, those in the micro category primarily centre their attention on the relationship between the outcome of $\mathcal F$ and the characteristics of the demonstration set $D$.


% \begin{itemize}
% \item  Definition: Clearly define what ICL are and describe their architecture.
%  \item Emergent Abilities: Describe the unexpected abilities that emerge from these models, providing examples where relevant.
%  \item Importance and Uses: Discuss how and where ICL are currently being used, and why interpretation is crucial in these contexts.
% \end{itemize}
% \section{Background and Notation}
% Following~\citet{Wei2022EmergentAO}, we define the emergent abilities of LLMs as \textit{the abilities that emerge when LLMs scale up to certain extents and cannot be anticipated only by extrapolating the performance improvements of smaller-scale models.}
% %Need to revise: cannot be... to similar to the cited paper
% Specifically, with a few examples or chain-of-thought demonstrations, LLMs can produce \textit{satisfactory results}\footnote{Comparable or even better results than those achieved by LLMs fine-tuned on specifica tasks or datasets.} across various tasks without requiring continuous training or fine-tuning.

% %D=\{(x_i,y_i)\};Q=(q_1,...,q_j);Y=(y_1,...,y_j)
% Formally, define $D \in \mathcal T_{train}$ is a subset demonstration selected from the training set.
% $Q\in \mathcal T_{test}$ is the query in the test set, $Y$ is the label for each query, $\mathcal M$ is the LLM with frozen parameter $\Theta$, and $\mathcal F$ is the evaluation metric function. 
% For example, $\mathcal F$ usually represents accuracy or F1 score in classification tasks, such as sentiment analysis, and represents Rouge or BLEU~\cite{} in text generation tasks, such as summarization.
% The concept of emergent ability can be formalized with the equation:
% \begin{equation}
% 	\mathcal F (Y,\mathcal M_\Theta(D,Q))
% \end{equation}
% where $\mathcal M$ is usually considered to exhibit emergent abilities if the computed value from $\mathbb F$ exceeds a threshold. 
% Under this definition, we can group similar concepts within the few-shot prompting paradigm. 
% CoT can be viewed as a variant of ICL, with the primary distinction being the format of the demonstration. 
% Specifically, ICL demonstrations typically utilize a standard prompt, while CoT demonstrations incorporate an additional textual reasoning process.

% Based on our definition, we categorize existing literature on interpreting emergent capabilities into macro and micro perspectives. 
% Researchers in the macro category focus on general loss or the model architecture, aiming to find the correlation between the result of $\mathcal F$ and $\mathcal M$.
% Conversely, those in the micro category primarily centre their attention on the correlation between the result of $\mathcal F$ and $D$.

% \section{Interpreting Emergent Ability from Macro Perspective}
% Studies from the macro perspective centre on mechanistic interpretability, investigating factors related to $\mathcal M$, including model interal architectures, training mechanisms, general loss, active functions, and optimizations algorithms. 
% They aim to understand why, how, and when these factors affect the emergence of the abilities.
%%%%%% draft examples
% \subsection*{\textcolor{red}{Example}} To unravel the internal structures responsible for ICL, \citet{Olsson2022IncontextLA} designed 2-layer attention-only models.
% They tried to decompose the operations of transformers~\cite{}, studying a phase change that occurs early in traning in LLMs of every size.
% One striking discovery from their study was the potential role of induction heads~\cite{elhage2021mathematical} as a mechanistic source of ICL in a majority of LLMs.
% However, their conclusions  is only the beginnings and far from satisfactory reverse-engineering LLMs to address the safety isses, considering frontier LLMs that typically consist of hundreds of layers and boast billions or even trillions of parameters.
% On the other hand, as with any empirical or interventional study, there's the possibility of encountering numerous subtle confounders or alternative hypotheses.
% A plausible research direction might involve generalizing these findings to more complex models. 
% By empirically observing, perturbing, and delving deep into the learning process and structural formation, researchers can potentially piece together an indirect understanding of the mechanistic happenings within LLMs.

% \subsection{Model factors}
% \begin{itemize}
% 	\item architectures
%  \textcolor{blue}{3pages in total. HANQI:ICL: copy->linear->nonlinear}
% 	\item loss
% 	\item active function
% 	\item gradient \textcolor{blue}{HANQI: why ICL can approximate gradient}
% \end{itemize}

% \subsection{Traning factors}
% \begin{itemize}
% 	\item training data
%  \textcolor{blue}{HANQI:data distribution, training samples} 
% 	\item optimization algorithms
% 	\item grokking (Grokking works which aims to interpret LLMs by measuring)
% \end{itemize}

% \subsection{Model Architecture}
In this section, we will look into the transformer~\cite{DBLP:conf/nips/BrownMRSKDNSSAA20} components from the perspective of: 1) what makes transformers pay attention to their context, 2) how they can fit into different downstream tasks, and 3) under which circumstances, they come into play~\cite{von2023transformers}.

\subsection{Preliminary}
Before diving into the related works about how transformer components contribute to the core ability -- in context learning ability, we briefly give the formal notions of the transformer blocks, especially the Self-Attention Network (\textsl{SAN}). For each block, the input word index $t$ is firstly transformed by an Embedding layer $W_{E}$, then the \textsl{SAN} is essentially to incorporate the information from surrounding tokens via multiple heads, $h\in H$ and from itself. The resulting representation $x_1$ is finally transformed back to the output word index $T(t)$.

\begin{eqnarray}
    T(t)& =& W_{U}x_1,\nonumber  \\
    x_1& =& x_0+ \Sigma_{h\in H}h(x_0)\nonumber  \\
    x_0& =& W_E t
\end{eqnarray}

\citet{elhage2021mathematical} developed a mathematical framework for SAN decomposition and found that the OV circuit defined as $W_U W_E$ controls token embedding flows directly down its own path without intervening with its surrounding tokens' information. In contrast, the QK circuit in $h$-head, $A^{h} = \text{softmax}(t^{T}\cdot W_E^{T}W_{QK}^{h}W_{E}\cdot t)$ provides the attention scores between each query and key token, contributes to the so-called induction head, which only occurs in two-layer SAN, knows how the token was previously used and looks out for similar cases in the provided context, making the model adaptive to downstream tasks.

\subsection{Beyond Copy to Function Approximation}
The findings in \cite{elhage2021mathematical,olsson2022context} are limited to two-layer SAN (without MLPs) and copying mechanisms, although the empirical results in ~\citet{olsson2022context} show that induction heads are capable of more sophisticated tasks, for example, machine translation, and a specific type of abstract pattern matching in a 40-layer model with 13 billion parameters. 

Later, researchers give proof on how the SAN (without MLPs) can approximate linear algorithm~\cite{dai2023can,DBLP:conf/nips/0001TLV22,DBLP:conf/iclr/AkyurekSA0Z23} without gradient descent. In their settings, the ICL is framed as meta-learning and fed with a sequence of tasks $D$ and the ICL learner $\text{TF}_{\theta}$ is actually to find optimal parameter $\theta$ to approximate a certain algorithm $f$ (here, $f$ is linear algorithm): 
\begin{align*}
   D &=  \{x_i,f(x_{i}),...,f(x_{n}\}\\
\text{argmin}_{\theta} &= \sum_{i=1}^{n} L(f(x_{i}),\text{TF}_{\theta}(D))
\end{align*}



\citet{irie2022dual} presented that linear
layers optimized by gradient descent (GD) have a dual
form of linear self-attention network(LSN)-without softmax in attention weight calculation. The following works demonstrated that TF’s ability to implicitly execute gradient descent steps during inference could also be central to ICL, supporting their claims with empirical evidence. \citet{dai2023can} gave the dual form of linear model gradient descent and LSN output on in-context samples.~\citet{DBLP:conf/nips/0001TLV22} empirically demonstrated that TFs can learn basic function classes (e.g., linear functions, MLPs, and decision trees) via input
sample sequences.~\citet{DBLP:conf/iclr/AkyurekSA0Z23} consider the complete TF structures by incorporating the softmax activation, MLP with GeLU and empirically demonstrated that in computational level ICL can approximate various regression problems, such as KNN, One-pass stochastic gradient descent, One-step batch gradient descent and Ridge regression. Based on the prior results,~\citet{von2023transformers} further enables solving nonlinear regression tasks, i.e., sinWave fitting, within transformers by showing its equivalence to learning a linear model on deep representations. ~\citet{Xie2021AnEO,swaminathan2023schema,DBLP:journals/corr/abs-2304-09960} formulated the LLM as a latent variable model and demonstrated its ability of Bayesian inference.



% \subsection{Effects of Hyper-Parameters}
% \paragraph{Model Parameter}\citet{DBLP:journals/corr/abs-2212-04458} find that among factors determining the inductive
% bias of the model, \textsl{state-size} ((such as the hidden state size in a recurrent network) is a more crucial parameter than the model size for the emergence of in-context learning ability. Instead, more research results show that \textsl{model size} is the key factor for emergent ability, i.e., the scaling law~\cite{DBLP:journals/corr/abs-2303-03846,DBLP:journals/corr/abs-2001-08361,DBLP:journals/corr/abs-2203-15556}. For example, PaLM-540B are capable of overriding semantic priors in text classification tasks, while smaller counterparts are unable to do so~\cite{DBLP:journals/corr/abs-2303-03846}. 
% \paragraph{Prompt Parameter} In addition to the parameters in the model itself, the performance in ICL also heavily relies on prompt settings~\cite{DBLP:conf/acl/CaoLHL022,DBLP:conf/acl/LuBM0S22,DBLP:journals/csur/LiuYFJHN23,DBLP:journals/corr/abs-2212-04037,DBLP:conf/acl/SorensenRRSRDKF22}: such as \textit{wording}, \textit{perplexity}, \textit{order}, \textit{label distribution} and the \textit{mutual information} between the prompt and the
% language model’s output. More fine-grained, ~\citet{DBLP:journals/corr/abs-2307-03172} studied the how the position of the information affects the model's performance and came to the conclusion that the performance would be the highest if relevant information occurs at the beginning or end of the input context.  Most of the above studies are based on empirical experiments. In order to avoid the effects from confound, \citet{DBLP:conf/acl/CaoLHL022,DBLP:conf/acl/StolfoJSSS23} study the causal effects of prompt variable to model performances by backdoor criteria. On the other hand, some researchers explained the prompt variation from a cognitive perspective~\cite{DBLP:journals/corr/abs-2206-14576}. They designed the experiments to evaluate LLM's cognitive ability, including decision-making, information search, deliberation, and causal reasoning, by applying small perturbations to the prompts. On the other hand, researchers give theory proof of how the prompt parameters affect the performances based on simplified settings. For example, ~\citet{DBLP:journals/corr/abs-2305-19420,Xie2021AnEO} give theory proof that a larger length of the ICL input sequence, i.e., prompt (with possible demonstrations) length can help decrease the regret bound of ICL. 

\subsection{Strengthen by CoT.}
All the above works investigated the power of TFs from an expressively perspective, i.e., the function approximation. However, they pay little attention to the striking contributions from the carefully-designed prompt~\cite{DBLP:journals/tacl/JiangXAN20,DBLP:journals/csur/LiuYFJHN23,Wang2022TowardsUC}. In particular, the so-called Chain-of-Thought prompting (COT) plays an important role in complex reasoning tasks, e.g., mathematical arithmetic. By adding \textit{let's think step by step} to guide the model to generate intermediate output and derive final striking results based on it~\cite{Wei2022EmergentAO}. Empirically, researchers show the factors in CoT affecting the reasoning performances to better understand its working mechanism, such as relevance to the query, the reasoning order, etc~\cite{Wang2022TowardsUC}.~\citet{DBLP:journals/corr/abs-2307-13339} found that CoT would not increase the magnitude of the saliancy score of the important input tokens, but indeed enhance its robustness under different input perturbations. ~\citet{DBLP:journals/corr/abs-2305-15408} takes the first step towards theoretically answering how the complex reasoning tasks can be solved by CoT-assisted LLMs.

\subsection{Challenges and Future Work}
Research methods focusing on the problem of how the model structure and parameters contribute to the emergent abilities can be roughly divided into two directions: (1) ablating the research subject and other factors, and then empirically observing the differences in task performances. (2) Simplifying the real situations by either removing the obscure model components or evaluating with delicately designed synthetic experiments, in order to theoretically deduct the effects from principle elements to output. The findings of the first group of methods are often limited to specific evaluation cases, e.g., the variation of subjects in mathematics can affect LLM calculation
results~\cite{DBLP:conf/acl/StolfoJSSS23,shi2023large}. The conclusions from the second line of research hold true only in restricted situations, e.g., without noise data, nor regularisations~\cite{von2023transformers}. These issues are very common in other research areas, and what we can do to push the limits forward in this direction can be summarized as (a) proposing advanced TF structures with better computation efficiency based on the observation that the self-attention layer actually plays a similar role in optimization~\cite{zucchet2022beyond}. (b) Designing a systematic evaluation schema to evaluate the LLM's capability, one can refer to the cognitive literature for detailed ability measurements and implications. (c) Extending to more learning problems and incorporating more realistic settings in order to better approximate 
the real solutions. (d) Additionally, very limited studies theoretically illustrate the success of Chain-of-Thought, which is the key factor in understanding the LLM's reasoning ability.

\section{Training Data Matters}
Unlike prior work that explores implicit mechanisms behind ICL, some researchers study ICL via investigating how the pretraining data intrigue the emergent abilities.

% \subsection{Task Diversity}
% Although the previous studies~\cite{dai2023can,DBLP:conf/iclr/AkyurekSA0Z23,von2023transformers} have also shown that transformers can do linear regression in ICL. However, their assumptions are based on unlimited task diversity, that is the pretrained tasks are the same as true evaluation tasks.~\citet{wies2023learnability} show that unseen tasks can be efficiently learned via ICL
% in-context learning, when the pretraining distribution is a mixture of latent tasks. ~\cite{DBLP:journals/corr/abs-2212-04458,DBLP:journals/corr/abs-2306-15063} show the emergence of in-context learning with pretraining task diversity.~\cite{DBLP:journals/corr/abs-2212-04458} studies this issue in a more systematic way, they identify a task diversity threshold for the emergence of ICL by deriving the optimal estimator on linear regression problems. While lots of work~\cite{DBLP:journals/corr/abs-2306-15063} empirically observe improvement of MNIST classification performance with pretraining task diversity increased.~\citet{DBLP:journals/corr/abs-2301-07067} framed the ICL as a Multiple Task Learning(MTL) problem and identified the transfer learning risk on unseen tasks is governed by the task complexity and the number of MTL tasks. 

% \subsection{Pretraining Data Distribution}
% In addition to the abundance of pertaining tasks, some other important properties of the pertaining data also matter to the emergence of ICL. \citet{DBLP:conf/naacl/ShinLAKKKCLPHS22} discovered that ICL heavily relies on the \textsl{domain relevance}, but pretraining on a dataset related to a downstream task does not always reflect a competitive ICL performance. 
% ~\citet{DBLP:conf/nips/ChanSLWSRMH22} identify \textsl{training distribution—burstiness} and \textsl{occurrence of rare classes}—that are necessary for the emergence of ICL, although their simulation experiments are based on image data.~\citet{DBLP:conf/acl/HanSMTCW23} obtained the similar conclusions: (1) the supportive pretraining data do not necessarily have a higher domain overlap to downstream tasks, (2) the supportive
% pretraining data with \textsl{relative lower frequency} and 
% \textsl{long-tail tokens} have larger contributions. And they addressed the importance of \textsl{long-range} pertaining data.~\citet{DBLP:journals/corr/abs-2303-03846} study how ICL is affected by semantic priors and input-label mappings. Their empirical results show large improvements in flipped-labels presented in-context, implying that they are more capable of using in-context information to override prior semantic knowledge. ~\citet{gu-etal-2023-pre} enhance the language models’ in-context
% learning ability by pre-training the model on
% of intrinsic tasks in the contrastive training manner.~\citet{DBLP:journals/corr/abs-2305-19420} provide a generalisation bound given the number of token sequences and the length of each sequence in pretraining.

% Some studies focus on exploring the influences of particular training samples on the given test samples~\cite{Zhang2021CounterfactualMI,DBLP:conf/icml/KandpalDRWR23,Akyrek2022TowardsTF}, a.k.a, fact tracing. The key factor in such research line is to detect the influential training samples first, then exclude the identified samples in the training phase, so-called counterfactual memorization. More specifically, ~\citet{DBLP:conf/icml/KandpalDRWR23} observed the positive correlation between model's memorization ability and frequency of the pretraining samples. This conclusion is somewhat aligned with that the long-tail knowledge is essential to the difficult tasks in ICL mentioned before. 


\subsection{Challenges and Future Work}
The challenges mainly lie in how to select the relevant important training samples given the test samples and how to avoid retraining the model in a more effective way~\cite{}. Influence functions~\cite{hampel1974influence,DBLP:conf/icml/KohL17,DBLP:conf/nips/PruthiLKS20} are among the first methods to do this for neural networks, by estimating the marginal effect of a training example on the loss of a test example. Basically, there are divided into gradient-based and embedding-based method according to  their measurement units. However, these methods become less practical due to the in-feasibility to model parameters and intermediate output. Besides, the retrain methods that aim to calculate the prediction differences are time-costing. Therefore, how the evaluate the importance of particle training samples in a effective way can be a promising direction.

% For instance, \textcolor{red}{Forward-INF : Efficient Data Influence Estimation with Duality-based  Counterfactual Analysis.!!!fail to find citation, it is a workshop paper, waiting for author's response} 

% \input{sections/lin}
\section{Interpreting Emergent Abilities from Macro Perspective}
Studies from the macro perspective centred on \textit{mechanistic interpretability}~\cite{Olah2022MI}. It involves delving into the inner mechanism of emergent abilities through different theoretical conceptual lenses such as linear regression formulation, meta-learning, latent space theory, and Bayesian inference.
% with goal of  reverse-engineering frontier LLMs
\subsection{Mechanistic Interpretability}
With the goal of reverse-engineering components of frontier models into more understandable algorithms, \citet{elhage2021mathematical} developed a mathematical framework for decomposing operations within  transformers~\cite{Vaswani2017AttentionIA}. They initially introduced the concept of ``\textit{induction heads}'' in a two-layer attention-only model to explain the functioning of ICL within transformers with Circuits~\cite{cammarata2020thread}.
They found that one-layer attention-only models perform relatively basic ICL in a crude manner, whereas two-layer models perform very general ICL using very different algorithms.
Specifically, they discovered that one-layer models essentially function as an ensemble of bigram and “skip-trigram” models that can be accessed directly from the model weights without running the entire model. Most attention heads in these models allocate significant capacity to copying mechanisms, resulting in very simple ICL.
In contrast, the two-layer models manifest a significantly powerful mechanism that employs more advanced, qualitative algorithms at inference time, referred to as ``\emph{induction heads}''. 
This allows them to perform ICL in a manner that resembles a computer program executing an algorithm, rather than merely referencing skip-trigrams. 
Building on this foundation, \citet{Olsson2022IncontextLA} later investigated the internal structures responsible for ICL by extending the concept of ``\emph{induction head}''~\cite{elhage2021mathematical}. They implemented  circuits consist of two attention heads: the ``\emph{previous token head}'', which copies information from one token to its successor, and the actual ``\emph{induction head}'', which uses this information to target tokens that precede the current one.
Their study revealed a phase change occurring early in the training of LLMs of various sizes. This phase change involves circuits that perform ``fuzzy'' or ``nearest neighbor" pattern completion in a mechanism similar to the two-layer induction heads. These circuits play a crucial role in implementating most ICL in large models.
One pivotal insight from \cite{Olsson2022IncontextLA} presented six arguments supporting their hypothesis that induction heads may serve as the primary mechanistic source of ICL in a significant portion of LLMs, particularly those based on transformer architectures.

While \citet{elhage2021mathematical} and \citet{Olsson2022IncontextLA} contribute to our understanding of ICL by probing the internal architecture of LLMs, it is important to note that their findings represent initial steps towards the comprehensive reverse-engineering of LLMs. It becomes particularly intricate when dealing with LLMs characterized by complex structures comprising hundreds of layers and spanning billions to trillions of parameters. This complexity introduces significant challenges. %, particularly in the context of addressing safety concerns.
Moreover, a substantial portion of their conclusions relies primarily on empirical correlations, which might be susceptible to confounding from various factors, thereby introducing potential vulnerabilities into their findings.

\subsection{Regression Function Learning}
Several research studies posited that the emergence of LLMs' competence in ICL can be attributed to their intrinsic capability to approximate regression functions for a novel query~$Q$ based on the demonstrations $D$.
\citet{Garg2022WhatCT} first formally defined ICL as a problem of learning functions and explored whether LLMs can be trained from scratch to learn simple and well-defined function classes, such as linear regression functions.
To achieve this, they generated examples $D$ using these functions, and trained models to predict the function value for the corresponding query $Q$.
Their empirical findings revealed that trained Transformers exhibited ICL abilities, as they manifested to ``learn'' previously unseen linear functions from examples, achieving an average error comparable to that of the optimal least squares estimator.
Furthermore, \citet{Garg2022WhatCT} demonstrated that ICL can be applied to more complex function classes, including sparse linear functions, decision trees, and two-layer neural networks, and posited that the capability to learn a function class through ICL is an inherent property of the model $M_\Theta$, irrespective of its training methodology.

Later, \citet{Li2023TransformersAA} extended \citet{Garg2022WhatCT} to interpret ICL from a statistical perspective. They derived generalization bounds for ICL, considering two types of input examples: sequences that are independently and identically distributed (i.i.d.) and trajectories originating from a dynamical system.
They established a multitask generalization rate of $1 / \sqrt{n T}$ for both types of examples, addressing temporal dependencies by associating generalization to algorithmic stability, abstracting ICL as an algorithm learning problem. 
They found that transformers can indeed implement near-optimal algorithms on classical regression problems with both types of input example by ICL.
Furthermore, they provided theoretical proof highlighting that self-attention possesses favourable stability properties, established through a rigorous analysis quantifying the influence of one token over another.

At the same time, \citet{Li2023TheCO} took a further step from the work of~\cite{Garg2022WhatCT} to gain a deeper understanding of the role of the softmax unit within the attention mechanism of LLMs. They sought to mathematically interpret ICL based on the softmax regression formulation represented as $\min _x||\left\langle\exp (A x), \mathbf{1}_n\right\rangle^{-1} \exp (A x)-b||_2$.
Their analysis revealed that the upper bounds of data transformations, induced either by a singular self-attention layer or by the application of gradient descent on an $L_2$ regression loss, align with the softmax regression formulation. 
This suggests a noteworthy similarity between models learned through gradient descent and those learned by Transformers, especially when trained solely on fundamental regression tasks using self-attention.

Conversely, ~\citet{Akyrek2023WhatLA} took a different approach by delving into the process through which ICL learns linear functions, rather than analysing the types of functions that ICL can learn.
Through an examination of the inductive biases and algorithmic attributes inherent in transformer-based ICL, they discerned that ICL can be understood in algorithmic terms, and linear learners within the model may essentially rediscover standard estimation algorithms.
More specifically, \citet{Akyrek2023WhatLA} provided a theoretical proof to support the claim that transformers can implement learning algorithms for linear models using gradient descent and closed-form ridge regression. They also empirically demonstrated that trained ICLs closely align with the predictors derived from gradient descent, ridge regression, and precise least-squares regression. 
They also introduced preliminary findings suggesting that ICL exhibits algorithmic characteristics, with both predictors of learners’ late layers encoding weight vectors and moment matrices in a non-linear manner.

Although these studies have either provided theoretical proofs or showcased empirical evidence interpreting the ICL ability of LLMs as a problem of learning regression functions, their conclusions are limited to simplified model architectures and controlled synthetic experimental settings. These findings may not necessarily apply directly to real-world scenarios.

\subsection{Gradient Descent \& Meta-Optimization}
In the realm of gradient descent, \citet{Dai2023WhyCG} adopted a perspective of viewing LLMs as meta-optimizers and interpreting ICL as a form of implicit fine-tuning. 
They first conducted a qualitative analysis of Transformer attention, representing it in a relaxed linear attention form, and identified a dual relationship between it and gradient descent. 
Through a comparative analysis between ICL and explicit fine-tuning, \citet{Dai2023WhyCG} interpreted ICL as a meta-optimization process. They further provided evidence that the transformer attention head possesses a dual nature similar to gradient descent~\cite{Irie2022TheDF}, where the optimizer produces meta-gradients based on the provided examples for ICL through forward computation.
Concurrently, \citet{Oswald2022TransformersLI} also proposed a connection between the training of Transformers on auto-regressive objectives and gradient-based meta-learning formulations. 
They specifically examined how Transformers define a loss function based on the given examples and, subsequently, the mechanisms by which Transformers assimilate knowledge using the gradients of this loss function. 
Their findings suggest that ICL may manifest as an emergent property, approximating gradient-based few-shot learning within the forward pass of the model.

However, it is worth noting that both of these investigations only focused on ICL within Transformer architectures, without considering other architectural variations or emergent capabilities, such as CoT and instruction following.
In addition, their analyses predominantly rely on a simplifed form of linear attention for qualitative assessment. 
This poses a challenge since the operation of standard Transformer attention, without any approximation, may be intricate. Therefore, there is a need for  more nuanced explorations into this mechanism in future studies.

\subsection{Bayesian Inference}
In their work, \citet{Xie2021AnEO} first provided an interpretation of ICL through the lens of Bayesian inference, proposing that LLMs have the capability to perform implicit Bayesian inference via ICL.
Specifically, they synthesized a small-scale dataset to examine how ICL emerges in LSTM and Transformer models during pretraining on text with extended coherence. 
Their findings revealed that both models are capable of inferring latent concepts to generate coherent subsequent tokens during pretraining. Additionally, these models were shown to perform ICL by identifying a shared latent concept among examples during the inference process. 
Their theoretical analysis confirms that this phenomenon persists even when there is a distribution mismatch between the examples and the data used for pretraining,  particularly in settings where the pretraining distribution is derived from a mixture of Hidden Markov Models (HMMs)~\cite{Baum1966StatisticalIF}. 
Furthermore,~\citet{Xie2021AnEO} observed that the ICL error decreases as the length of each example increases, emphasizing the significance of the inherent information within inputs. This goes beyond mere input-label correlations and highlights the roles of intrinsic input characteristics in facilitating ICL.

Following on,~\citet{Wang2023LargeLM} expanded the investigation of ICL by relaxing the assumptions made by~\citet{Xie2021AnEO} and posited that ICL in LLMs essentially operates as a form of topic modeling that implicitly extracts task-relevant information from examples to aid in inference.
\citet{Wang2023LargeLM} grounded their theoretical analysis in a setting with a finite number of demonstrations, and under a more general language generation process. 
Specifically, they characterized the data generation process using a causal graph with three variables and imposed no constraints on the distribution or quantity of samples. Their empirical and theoretical investigations revealed that ICL can approximate the Bayes optimal predictor when a finite number of samples are chosen based on the latent concept variable.
Moreover, \citet{Wang2023LargeLM} devised an effective practical algorithm for demonstration selection tailored to real-world LLMs. 

At the same time, \citet{Jiang2023ALS} also introduced a novel latent space theory extending the idea of~\citet{Xie2021AnEO} to explain emergent abilities in LLMs.
Instead of focusing on specific data distributions generated by HMMs, they delved into general sparse data distributions and employed LLMs as a universal density approximator for the marginal distribution, allowing them to probe these sparse structures more broadly.
\citet{Jiang2023ALS} demonstrated that ICL, CoT, and instruction-following abilities in LLMs can be ascribed to Bayesian inference operating on the broader sparse joint distribution of languages.
To shed light on the significance of the attention mechanism for ICL from a Bayesian view,~\citet{Zhang2023WhatAH} defined ICL as the task of predicting a response that aligns with a given covariate based on examples derived from a latent variable model. 
They established that ICL implicitly implements the Bayesian Model Averaging (BMA) algorithm, which is approximated by the attention mechanism.
Furthermore, they demonstrated that certain attention mechanisms converge towards the conventional softmax attention as the number of examples goes to infinity. 
These attentions, due to their encoding of BMA within their structure, empower the Transformer model to perform ICL.

Although their conclusions are insightful, there is a room for improvement.
Their findings might be influenced by various factors, such as the formats of the  examples, the nature of tasks, and the choice of evaluation metrics. 
Additionally, many of these studies are based on analyses conducted using small synthetic datasets, potentially restricting their relevance and applicability to real-world scenarios.



% Studies from the micro perspective centre on empirical interpretability, probing the factors associated with $D$, such as type of tasks, and the quality, quantity, and sequence of demonstrations.
% They aim to explore how and why these factors are correlated with the result of $\mathbb F$.
\section{Interpreting Emergent Abilities from Micro Perspective}
From a micro perspective, research predominantly emphasizes empirical probing, focusing on the factors that influence the emergent abilities of LLMs.
 These factors encompass variations in results across downstream tasks, driven by aspects such as the quality of pre-training data \citep{Chan2022DataDP, Razeghi2022ImpactOP, Shin2022OnTE, Razeghi2022ImpactOP, Power2022GrokkingGB}, the quality of the provided examples \citep{Lu2021FantasticallyOP, Liu2021WhatMG, Wang2022TowardsUC, Turpin2023LanguageMD}, and mappings of demonstration labels \citep{Min2022RethinkingTR,Kossen2023InContextLI,Wei2023LargerLM,Kim2022GroundTruthLM}.
% \section{In-context Learning Applications}

% In-context learning applications of Large Language Models (LLMs) like GPT-3 have expanded our understanding and utilization of such models in varied real-world scenarios. This section will delve into the recent advancements and experiments done in the domain, shedding light on how LLMs are being applied, augmented, and furthered to handle few-shot prompted tasks, and enhance emergent abilities.

% \subsection{Few-Shot Prompted Tasks}

% \textbf{Augmentation:} As we move into the realm of harnessing the capabilities of LLMs, \citep{Wahle2022HowLL} stands out by proposing a unique technique using few-shot examples to prompt GPT-3. This enabled the generation of machine-paraphrase plagiarism data extracted from freely accessible scientific articles. Simultaneously, \citep{Agrawal2022LargeLM} undertakes an ambitious exploration of GPT-3's prowess in zero- and few-shot information extraction from clinical texts. What's noteworthy is that this approach sidesteps the need for domain-specific fine-tuning. Their research also extends to various NLP tasks demanding structured outputs, touching on areas like span identification, sequence classification at the token level, and relation extraction.

% \textbf{Reasoning:} The subtlety of reasoning is captured in \cite{Madaan2022LanguageMO}, where the focus is on improving the structural accuracy of commonsense reasoning, hinging on few-shot examples applied on code-generation LLMs. Intriguingly, \citep{Spiliopoulou2022EvEntSRE} tests the waters of LLMs' underlying and worldly knowledge. They probe the model's capability to reason event implications using two datasets while also evaluating the effects of varied prompting techniques on model performance. Meanwhile, \cite{Min2022RethinkingTR} demystifies the paradigm of in-context learning, emphasizing that for large language models, having ground truth demonstrations might not be mandatory for efficient in-context learning.

% \subsection{Augmented Prompting Strategies}

% \textbf{Provide External Knowledge:} The question of how LLMs utilize their vast internal knowledge has been the subject of extensive research. \cite{Li2022LargeLM} takes the lead by exploring LLMs' controllability and robustness, unveiling their shortcomings in leveraging knowledge from input contexts. Their proposed solution – a context-aware finetuning methodology – infuses counterfactual and irrelevant contexts into standard supervised datasets, addressing the challenge head-on. Similarly, \cite{Lal2022UsingCK} delves deeper into the sphere of commonsense knowledge, proposing techniques to supplement the model with additional context to better answer ``why'' questions.

% \textbf{Data Augmentation:} In an artful confluence of technology and literature, \citep{Chakrabarty2022HelpMW} introduced CoPoet, a system promoting collaborative writing with LLMs. On a different tangent, \citep{CallisonBurch2022DungeonsAD} presents a novel perspective by viewing textual-based games through the lens of a dialogue system, demonstrating LLMs' capability to pinpoint game stages via dialogue.

% \textbf{Chain-of-thought like methods:} Handling intricate queries that demand latent decisions is a formidable challenge, one that \cite{Dua2022SuccessivePF} tackles by iteratively simplifying a complex task. The process involves breaking down the task, resolving it, and repeating the cycle till a conclusion is reached. This approach is mirrored by \cite{Patel2022IsAQ} who propose question decomposition to aid LLMs in understanding and answering complex queries.

% \textbf{Reasoning:} The art of questioning is an avenue that \cite{Shridhar2022AutomaticGO} ventures into, focusing on sequential question generation to aid in solving mathematical word problems. In the broader context of understanding social dynamics and inferring mental states, \cite{Sap2022NeuralTO} exposes the current limitations of LLMs like GPT-3.

% \subsection{Methods to Enhance Emergent Ability}

% \textbf{Training Methods:} A novel approach to unsupervised masking strategy, InforMask, is introduced by \cite{Sadeq2022InforMaskUI}, leveraging Pointwise Mutual Information to pinpoint and mask the most informative tokens, showing immense potential in question-answer tasks. Meanwhile, \cite{Ke2022ContinualTO} puts forth the idea of perpetually extending LLMs via post-training with a sequence of domain-specific unlabeled corpora. Lastly, \cite{Zhang2022ActiveES} introduces a reinforcement learning algorithm, identifying policies that handpick generalizable demonstration examples, a feat that holds promise in tasks unseen during training.

\subsection{Pre-training Data}
% The investigation of the pre-training stage's influence on emergent abilities starts from \citep{DBLP:conf/nips/BrownMRSKDNSSAA20}, in which they introduced and delved into the capabilities of GPT-3. Their study shows that GPT-3 can perform various tasks without requiring task-specific fine-tuning. The researchers employed three popular ICL methods in their paper: zero-shot, one-shot, and few-shot. In this approach, instead of inundating the model with vast amounts of training data, GPT-3 was provided with just a few examples to help it understand or even without any examples in zero-shot cases and execute the task. \citet{DBLP:conf/nips/BrownMRSKDNSSAA20} comprehensively experimented on diverse datasets spanning nine categories of natural language tasks. Their finding suggests that zero-shot performance improves steadily with model size, few-shot performance increases more rapidly, demonstrating that larger models are more proficient at ICL. Besides, \citet{DBLP:conf/nips/BrownMRSKDNSSAA20} designed algorithms to detect the influence of ICL performance on contaminated pre-training data. Interestingly, they found that the language model is relatively insensitive to contamination. Still, they also observed it may bring some performance inflation on small parts of the dataset span tasks like reading comprehension, translation, etc.
%Several literature 
Some studies have suggested that factors related to pre-traning data such as data domain, data term frequency, and data distribution~\citep{Chan2022DataDP, Razeghi2022ImpactOP}, are crucial elements influencing the development of emergent abilities.
\paragraph{Data Domain} 
\citet{Shin2022OnTE} conducted a study to explore the variations of ICL performance concerning the \textit{domain source} and the \emph{size} of the pre-training corpus, focusing primarily on the Korean lexicon. 
They utilized seven subcorpora from the HyperCLOVA corpus \cite{Kim2021WhatCC} to pre-train various language models and evaluated these models on Korean downstream tasks. 
Interestingly, \citet{Shin2022OnTE} found that the size of the pre-training corpus does not always determine the emergence of ICL. 
Instead, the domain source of the corpus significantly influences ICL performance. 
For example, language models trained with subcorpora constructed from blog posts exhibited the best ICL capability. 
This phenomenon may be attributed to the greater token diversity presented in the blog posts corpus compared with other sources like news. Moreover, their experiments highlighted that combining multiple corpora can lead to the emergence of ICL, even if individual corpora did not produce such learning on their own. 
Surprisingly, \citet{Shin2022OnTE} also found that a language model pre-trained with a corpus related to a downstream task did not always guarantee competitive ICL performance. 
For instance, a model trained on a news-related dataset~\citep{Park2021KLUEKL} showed  superior performance in zero-shot news topic classification, but its few-shot performance was not superior.
In a similar vein, \citet{Han2023UnderstandingIL} reached similar conclusions and highlighted the importance of long-range pre-training data.
Their experimental observation can be summarized into two aspects: (1)  supportive pre-training data do not necessarily need to have a higher domain overlap with downstream tasks, and (2) supportive pre-training data with relatively lower frequency and long-tail tokens make larger contributions to the model's capabilities.
Furthermore, ~\citet{Raventos2023PretrainingTD} empirically observed improvements in MNIST classification performance as pre-training data task diversity increases. 
\vspace{-3mm}
\paragraph{Data Term Frequency}
A more comprehensive investigation into the factors affecting  emergence of the ICL based on the pre-training term frequencies has been proposed by \citet{Razeghi2022ImpactOP}. 
The authors focused particularly on a crucial type of reasoning in LLMs -- numerical reasoning in few-shot settings; and examined the extend to which the frequency of terms from the pre-training data correlates with model performance in these situations. 
Their analysis focused on the prevalence of numerical reasoning tasks within the training instances and established a connection between frequencies and reasoning performance. This connection is quantified  by introducing the ``performance gap'', which is defined as the accuracy of terms appearing more than 90\% of the time minus the accuracy of terms appearing less than 10\% of the time. They conducted their experiments using GPT-based language models trained on the Pile dataset \citep{Gao2020ThePA}, ranging in size from 1.3B to 6B parameters. Evaluation was carried out on 11 datasets spanning three types of mathematical reasoning tasks: Arithmetic, Operation Inference and Time-Unit Conversion. 
The findings consistently show that models perform better in instances where terms from the pre-training data are more prevalent \citep{Razeghi2022ImpactOP}. 
In some scenarios, there is a substantial performance gap of over 70\% between the most frequently occurring terms and the least frequently occurring terms. 
The significant performance difference raises questions about the actual generalization capabilities of these models beyond their pre-training data. 
\citet{Razeghi2022ImpactOP}'s observations suggest that the more prevalent content included in the pre-training data may exert an influence on the emergent abilities, and it is possible that these language models are not actually reasoning to solve arithmetic tasks. 
In line with this research, ~\citet{Kandpal2022LargeLM} also observed a positive correlation between model's memorisation ability and the frequency of the pre-training samples.
\vspace{-2mm}
\paragraph{Data Distribution} 
\citet{Chan2022DataDP} provided an interpretation of the emergent abilities of transformers from the perspective of training \textit{data distribution}. They conducted an experiment based on image classification to explore the emergent capabilities of language models, particularly in the context of performing few-shot learning without explicit training. 
They manipulated the distributional properties of the training data, such as burstiness and the presence of rare classes. They observed the impacts of these manipulations on ICL using models like transformers and recurrent networks. %Similar to \citet{DBLP:conf/naacl/ShinLAKKKCLPHS22}'s finding on the diversity of the lexical may benefit the emergence of ICL, 
\citet{Chan2022DataDP} also emphasised that ICL is more pronounced when item meanings or interpretations are dynamic rather than static. They highlighted that natural language and other naturalistic data sources exhibit these dynamic properties, which differ from the uniform distributions typically used in standard supervised learning. Through their experiments, they uncovered a trade-off in transformer models between ICL and in-weights learning, which relies on information stored through slow, gradient-based updates \citep{Chan2022DataDP}.
However, subsequent experiments showed that both ICL and in-weights learning could coexist in a model trained on data with a skewed Zipfian marginal distribution \citep{Zipf1949HumanBA}, a distribution commonly observed in the frequency of words in languages.
Interestingly, while transformers exhibited ICL when trained on specific data distributions, recurrent models like LSTMs and RNNs did not. Finally, they revealed that non-uniform training distributions can cause the induction of the emergence of new capabilities. Their work highlights the importance of architecture and training data distribution in the emergence of ICL in LLMs. 
To understand how data distribution affect the effectiveness of ICL, \citet{Wies2023TheLO} theoretically demonstrated that unseen tasks can be efficiently learned via ICL when the pre-training data distribution comprises a mixture of latent tasks.
% \citet{DBLP:journals/corr/abs-2301-07067} framed the ICL as a Multiple Task Learning(MTL) problem and identified the transfer learning risk on unseen tasks is governed by the task complexity and the number of MTL tasks.

% \citep{DBLP:conf/nips/BrownMRSKDNSSAA20, Wei2022EmergentAO} pointed out that LLMs' emergent abilities on a range of downstream tasks improve while increasing the scale of language models. 
% However, new line of studies discovered that the emergence of ICL are not limited by the \textit{model scale} factor. Other factors such as data distribution \citep{Chan2022DataDP, Razeghi2022ImpactOP}, source of data \citep{DBLP:conf/naacl/ShinLAKKKCLPHS22, Razeghi2022ImpactOP, Power2022GrokkingGB}, and model architecture \citep{Chan2022DataDP, Xie2021AnEO} may also plays an important role at the pre-training stage. Researchers proposed to investigate the emergent ability either theoretically \citep{Wei2022EmergentAO, Xie2021AnEO} or empirically \citep{Chan2022DataDP,DBLP:conf/naacl/ShinLAKKKCLPHS22,Razeghi2022ImpactOP,Power2022GrokkingGB}. %Interestingly, most of those interpretations focus on LLMs' reasoning capability. 

% In this section, we first summarise the existing approach to interpret the gain of LLMs' emergent abilities in the pre-training stage. Then, we discuss the existing challenges and barriers in current research. Finally, we present some possible future work directions for future interpretation of the emergent abilities.

\subsection{Pre-training Model}
\citet{Wei2022EmergentAO} embarked on an investigation into the emergent abilities of LLMs. 
Their approach to interpreting this phenomenon involved conducting a comprehensive survey of existing literature and analyzing the unpredictable nature of how certain abilities manifest as these models scale \citep{Brown2020LanguageMA}. 
\citet{Wei2022EmergentAO} emphasized that while \textit{model scale} has been correlated with LLM performance, it is not the sole determinant. Task-specific abilities can also be examined by considering a language model’s performance (perplexity) on general text corpora, such as WikiText103. 
Their experiments showed that, despite having fewer parameters, the PaLM 62B model outperformed LaMDA 137B and GPT-3 175B in certain tasks. This suggests that other factors, like \textit{high-quality data} and \textit{architectural differences}, also play a role.
Moreover, continued pre-training on different objectives, like the mixture-of-denoisers objective, has shown potential in enabling emergent abilities \citep{Tay2022UnifyingLL}. 
Research is also advancing to make these discovered abilities accessible for smaller-scale models.
For instance, instruction-based fine-tuning showed potential in smaller models with different architectures. 
Additionally, the emergence of syntactic rule-learning can be triggered by threshold frequencies in training data, similar to "aha" moments in human learning \citep{Abend2017BootstrappingLA, Zhang2020WhenDY}.
 While the majority of research agrees that model scale is a key factor for emergent abilities. 
 \citet{Kirsch2022GeneralPurposeIL} presented an interesting perspective. They found that among the factors determining the inductive bias of the model, the state-size (such as the hidden state size in a recurrent network) is a more crucial parameter than the overall model size for the emergence of ICL ability.


 % ~\citet{Kirsch2022GeneralPurposeIL} studies this issue in a more systematic way, they identify a task diversity threshold for the emergence of ICL by deriving the optimal estimator on linear regression problems. 

%%



% \citet{Chan2022TransformersGD}'s study also delves into the generalization behaviours of transformer models, focusing on the difference between in-weight learning and in-context learning. 
% More detailly, they want to understand whether transformer models generalize using rule-based methods (e.g. relying on minimal features) or exemplar-based methods (e.g. relying on the similarity to training instances). They designed their experiment on both syntactic and textual data with a ``partial exposure'' test \citep{Dasgupta2021DistinguishingRA} where stimuli have two features, and the model is trained on certain combinations and then evaluated on a held-out combination.
% \citet{Chan2022TransformersGD} find out that transformers display a distinct difference in their generalization from in-context vs. in-weights information.
% Specifically, they found that transformers lean towards exemplar-based generalization from in-context details, while they favour rule-based generalization from in-weights information.
% However, large transformer models pre-trained on language showed more substantial rule-based generalization from in-context information.

% A more theoretical framework to explain the emergence of ICL when pre-training with documents has long-range coherence has been developed by \citet{Xie2021AnEO}, especially when there's a distribution mismatch between prompts and pre-training data. The authors expanded on the probability of a given output based on a set of examples and a test input. Under their framework, They derived heuristics and made several assumptions related to the bounds on delimiter transitions, distribution shifts, and regularity. \citet{Xie2021AnEO} find out that language models tend to infer from a latent document-level concept to generate new tokens during pre-training, and language models tend to match this shared latent concept during the inference to emerge the ICL behaviour. Their theoretical investigation is grounded in a Bayesian inference perspective. \citet{Xie2021AnEO} carried out experiments on both transformers and LSTM-based models trained with a small-scale synthetic dataset (GINC) based on a uniform mixture of Hidden Markov Models (HMMs). During the model inference stage, they generate textual prompts from a sampled latent concept to validate their theoretical framework. Their quantitative and theoretical analysis highlights that the in-context predictor is optimal as the number of in-context examples increases. ICL tend to emerge when the pre-training data distribution is a mixture of HMMs.




\subsection{Demonstration Examples}
% With the scaling of model size and corpus size, LLMs have shown impressive performance on downstream tasks by conditioning on a prompt that contains a few input-label pairs from training data~(Demonstrations), which has been referred to as in-context learning~(ICL, \citet{Brown2020LanguageMA}).
Several recent  studies~\citep{Liu2021WhatMG,Min2022RethinkingTR,an2023context} have revealed the sensitivity of emergent abilities to alterations in order, format, and quantity of provided demonstrations.
\paragraph{Demonstration Order}
The order of the demonstrations has a significant impact on downstream task performance. \citet{Lu2021FantasticallyOP} showed that it be the deciding factor between achieving near state-of-the-art and random guessing. 
They designed demonstrations containing four samples with a balanced label distribution and conducted experiments involving all 24 possible permutations of sample orders.
The experimental results showed that the performance variations among different permutations exist across various model sizes, especially for smaller models.
Besides, is was observed that effective prompts are not transferrable across models, indicating that the optimal order is model-dependent, and what works well for one model does not guarantee good results for another model.

% Similar phenomena were discovered in .
\citet{Zhao2021CalibrateBU} identified a phenomenon that LLMs tend to repeat answers found at the end of  demonstrations, which they termed ``recency bias''.
Similarly, in multi-document question answering and key-value retrieval tasks,~\citet{Liu2023LostIT} made analogous observations.
These tasks involve identifying relevant information within lengthy input contexts.
The results showed that LLMs performed best when the relevant information is located at the beginning or end of their input contexts.
However, their performance degraded when they are forced to use information from the middle of their input.
In addition, they noted that model performance declines as the input context length increases, suggesting that current models struggle to effectively reason over their entire context window.
%\citet{Liu2023LostIT} studied the how the position of the information affects the model's performance and came to the conclusion that the performance would be the highest if relevant information occurs at the beginning or end of the input context.
Although these studies offer insights into how demonstration order influences emergent abilities, they do not delve into the underlying reasons of these obesrvations.
In an effort to investigate the impact of semantic similarity between ICL examples and test examples on downstream task, \citet{Liu2021WhatMG} proposed retrieving examples semantically similar to a test example for creating its demonstration.
They utilized the CLS embeddings from a pre-trained RoBERTa-large~\citep{Liu2019RoBERTaAR} model to represent sentences and assessed the semantic similarity between two sentences by computing the cosine similarity of their respective representations. 
For each test example, they identified the nearest $K$ neighbors from the training set and concatenated them in descending order of semantic similarity to create the demonstration. 
Their experiments on Web Questions~\cite{Berant2013SemanticPO} and Trivia Question Answering~\cite{Joshi2017TriviaQAAL} benchmarks showed that the default order performed slightly better than the reverse order.
However, the reverse order performed better on the Natural Questions \cite{Kwiatkowski2019NaturalQA} benchmark.
Consequently, the choice of order appears to be dependent on the specific dataset in use.
%is data-dependent.


\paragraph{Input-Label Mapping}
Some studies have been conducted to investigate how input-label mappings influence the performance of ICL.
\citet{Min2022RethinkingTR} revealed that substituting the correct labels of in-context examples in demonstrations with random labels only leads to a marginal decrease in performance across a variety of classification and multi-choice tasks.
They also conducted ablation experiments to investigate the impact of the number of correct labels on performance. Surprisingly, the results showed that the performance was not sensitive to the number of correct labels in demonstrations.
This led to the counter-intuitive conclusion that LLMs do not heavily rely on input-label mappings to perform tasks.

However,~\citet{Kim2022GroundTruthLM}, \citet{Wei2023LargerLM}, and \citet{Kossen2023InContextLI} disagreed with the claim put forth by \citet{Min2022RethinkingTR}.
\citet{Kim2022GroundTruthLM} pointed out that the claim exhibited over-generalization in two aspects: (1) Aggregating the mean performance across various datasets was found to be inadequate in capturing the insensitivity behavior observed within individual datasets. (2) The experimental setting lacked generalizability, and the results were sensitive to minor adjustments to the experimental setup.
To delve deeper into the topic of input-label mapping, \citet{Kim2022GroundTruthLM} introduced two novel metrics.
The first metric, Label-Correctness Sensitivity, quantifies the impact on downstream classification performance when a fixed amount of label corruption is introduced into the demonstration. 
The second metric, Ground-Truth Label Effect Ratio, assesses how much the presence of ground-truth labels improves the performance compared to a baseline with random labels. 
Their experimental results showed that the sensitivity exhibited significant variation across 17 datasets, with  the aggregate sensitivity considerably high. This indicated that label correctness does indeed affect downstream task performance.
Furthermore, \citet{Kim2022GroundTruthLM} suggested a strong correlation between sensitivity and task difficulty, revealing that LLMs displayed low sensitivity on challenging tasks.

\citet{Wei2023LargerLM} further explored how semantic priors and input-label mappings affect ICL.
They suggested that LLMs possess the ability to override semantic priors from pre-training in favour of input-label mappings from demonstrations. This explains why the performance of LLMs drops below random guessing when all the labels in the demonstrations are flipped.
They also found that smaller models experienced a less severe decline in performance because they lack the capacity to override semantic priors to the same extent.
More specifically, \citet{Wei2023LargerLM} conducted experiments where they replaced the labels with semantically unrelated labels. The results showed that the performance drop was more significant for small models compared to LLMs.
This led them to suggest that small models rely heavily on the semantic meanings of labels rather than learning the input-label mappings provided in the demonstrations.
\citet{Kossen2023InContextLI} also found that larger models are more sensitive to randomized labels, and they highlighted that LLMs can learn new input-label mappings from  demonstrations.
 
However, the ability to learn new input-label mappings can, at times, have adverse effect on performance.
\citet{Tang2023LargeLM} revealed that LLMs sometimes tend to exploit shortcuts within demonstrations for downstream tasks.
These shortcuts represent spurious correlations between in-context examples and their corresponding labels.
\citet{Tang2023LargeLM} designed several types of shortcuts,  and their experimental results showed that LLMs are ``lazy reasoners''. 
They relied heavily on the shortcuts within demonstrations to deduce the final answers. 
Furthermore, \citet{Si2023MeasuringIB} discovered that when presented with a set of non-specific demonstrations (For example, the labels are semantically unrelated), LLMs exhibited feature bias. 
This indicated that LLMs tend to favour one feature over another, even when both features are equally capable of predicting the label, as mentioned in the prompt.
For example, in a sentiment analysis setting, LLMs showed a significant bias towards predicting labels based on sentiment rather than shallow lexical features.
Nevertheless, feature bias has the potential to detrimentally affect performance when the model's feature bias does not align with the intended task.
\citet{Si2023MeasuringIB} suggested that certain interventions could help mitigate feature bias, such as employing natural-language instructions and incorporating label words that have semantic relevance to the intended feature.

To further investigate the underlying mechanism of how LLMs learn from input-label mappings, \citet{Wang2023LabelWA} conducted an extensive study into the workings of ICL from the perspective of information flow.
They computed saliency scores for each element within the attention matrix to unveil the significant token interactions.
The experimental results demonstrated that label words within demonstrations play a crucial role in this process. Specifically:
(1) During the processing of shallow computation layers, semantic information becomes concentrated within the representations of label words.
(2) The aggregated information contained within label words serves as a reference for the final predictions made by LLMs.
Their findings confirmed that label words can indeed have a substantial impact on the performance of the final task.

\paragraph{Chain-of-Thought Prompting}
% LLMs have shown superior performance on many downstream tasks by producing step-by-step reasoning, which is referred to as Chain-of-Thought (COT) reasoning.
% Chain-of-thought prompting is an approach that encourages LLMs to generate their intermediate rationales for solving a problem.
% It can considerably improve the multi-step reasoning ability of LLMs and is accomplished by providing the LLMs with some in-context examples in demonstrations where the reasoning process is explicitly outlined.
Some studies have focused on exploring the impact of COT prompting on LLM performance.
\citet{Wang2022TowardsUC} found that the validity of the reasoning process in demonstrations has only a minimal impact on performance. 
To assess this, they constructed invalid reasoning processes manually for all in-context examples. Surprisingly, the experimental results showed that LLMs can retain 80-90$\%$ of their performance even when presented with invalid reasoning steps in demonstrations.
They also found that the coherence of the reasoning process and its relevance to the query are significantly more crucial factors for the effectiveness of CoT.

Regarding the explanations generated by LLMs, \citet{Turpin2023LanguageMD} found that CoT explanations produced by LLMs can occasionally misrepresent the true underlying rationales behind their predictions.
They introduced two types of bias in the prompt design to investigate this phenomenon.
The first bias involves consistently reordering the multiple-choice options of in-context examples to make the answer 'A'.
The second bias entails including the suggested answers directly in the prompt.
The experimental results indicated that, in both bias scenarios, LLMs tend to provide answers aligned with stereotypes and generate explanations that do not faithfully support the answer.
Furthermore, there was a large drop in performance when comparing biased demonstrations to unbiased demonstrations.

% \subsubsection{Future direction}
% The current research is limited to specific closed-set tasks, which include classification and multiple-choice tasks.
% Additionally, it encompasses the examination of some template-based reasoning tasks.
% Future research endeavours could broaden their scope to encompass more intricate and open-set tasks, such as generation and tasks involving complex reasoning.





































%%%%%% draft examples
% \subsection*{\textcolor{red}{Example}}
% \citet{Lu2023AreEA} provided analysis of emergent abilities while accounting for various confounding factors that might lead to biased LLM.
% They controlled for various emergent abilities in an attempt to differentiate between those that are inherently emergent and those that appear as a result of specific prompting techniques.
% Surprisingly, \citet{Lu2023AreEA} found that emergent abilities were only evident in two of the 14 experimental tasks they administered, with the rest primarily demonstrating memorization. 
% However, their study did not control for CoT, which also identified by~\citet{Wei2022EmergentAO} as a crucial emergent ability of LLMs.
% A potential future direction might be to account for additional emergent abilities, such as CoT, while also considering confounding factors like data leakage and memorizability.


% \subsection{Demonstration factors}
% \begin{itemize}
% 	\item quality
% 	\item quantity
% 	\item order
% \end{itemize}

% \subsection{Task factors}
% \begin{itemize}
% 	\item type of task
% 	\item evaluation metrics
% 	\item 
% \end{itemize}



% % \section{Recent Research on ICL interpretation}
% % Summarize existing works used for interpreting ICL, highlighting the unique aspects of each. 
% % It can initially be categorized into the following two general categories, with some preliminary works listed under each.

% % \textbf{For each perspective and method, please first provide brief summaries and main findings of each work. Then, illustrate the insights drawn from these works and discuss the current challenges and limitations of interpreting ICL from these perspectives.}

% % \subsection{Macro Perspective on ICL Interpretation}
% % Analyzing and interpreting ICL from a macro perspective, focusing on general loss or the model architectures.
% % \begin{itemize}
% % 	\item\textbf{Hanqi:} 
% %  \\
% %  \textbf{notes: use semantic scholar for citations, 1month deadline, application track. models > 10 billion parameters? Papers after 2022. ICML papers }
% %  Other perspectives: (1)Causal explanation~\cite{Meng2022LocatingAE}, counterfactual interpretation~\cite{Li2023CounterfactualRT,HoelscherObermaier2023DetectingEF}
 
% %  (2)self-critic~\cite{Yin2023DoLL}, 
 
% %  (3)uncertainty estimation of LLMs' outputs, bias~\cite{Simmons2022MoralML}, 
 
% %  (4)LLMs access to tools~\cite{Sarti2023InseqAI}, etc~\cite{Wang2023ConstructingWS,Brinner2023ModelIA,Lam2023LargeLM,Riemenschneider2023ExploringLL}.
% %  	\item \textbf{Lin:} From the perspective of LLMs themselves. For example, interpretation of neurons~\cite{bills2023language,Gurnee2023FindingNI,Foote2023N2GAS}, structure, active functions~\cite{Yang2023LocalIO}, and optimization algorithms~\cite{Barak2022HiddenPI,Du2023GeneralizingBF} in LLMs. 
% % \end{itemize}
% % \subsection{Micro Perspective on ICL Interpretation}
% % Analyzing and interpreting ICL from a micro perspective, focusing more on specific tasks. 
% % \begin{itemize}
% % 	\item \textbf{Jiazheng:} From the perspective of ``the way how to use LLMs'' or ``the emergent abilities~\cite{Wei2022EmergentAO} of LLMs". Such as, In-Context-Learning (ICL)~\cite{Xie2021AnEO,Olsson2022IncontextLA,Dai2023WhyCG,Wang2023LargeLM,Sun2023HowDI,Bansal2022RethinkingTR,Tang2023LargeLM}, Chain-of-Thought (CoT), and prompt tuning~\cite{Wei2021WhyDP,Singh2022ExplainingPI,Ju2023IsCP}.
% % 	\item \textbf{Yanzheng:} From the perspective of different domains (NLP~\cite{Liu2023LostIT,Wicke2023LMsST}, CV, Physical, Theory of Mind~\cite{Mahowald2023DissociatingLA,Shapira2023HowWD}, etc ) and tasks (Coding~\cite{Bubeck2023SparksOA}, Mathematical~\cite{Saunshi2020AME,Bubeck2023SparksOA}, Reasoning~\cite{Prystawski2023WhyTS,Jung2022MaieuticPL,Lewkowycz2022SolvingQR}, Planning~\cite{Collins2022StructuredFA, Valmeekam2023OnTP}). 

% % \end{itemize}

% \section{Interpreting Emergent Abilities from Micro Perspective}
From a micro perspective, research predominantly emphasizes empirical probing, focusing on the factors that influence the emergent abilities of LLMs.
 These factors encompass variations in results across downstream tasks, driven by aspects such as the quality of pre-training data \citep{Chan2022DataDP, Razeghi2022ImpactOP, Shin2022OnTE, Razeghi2022ImpactOP, Power2022GrokkingGB}, the quality of the provided examples \citep{Lu2021FantasticallyOP, Liu2021WhatMG, Wang2022TowardsUC, Turpin2023LanguageMD}, and mappings of demonstration labels \citep{Min2022RethinkingTR,Kossen2023InContextLI,Wei2023LargerLM,Kim2022GroundTruthLM}.
% \section{In-context Learning Applications}

% In-context learning applications of Large Language Models (LLMs) like GPT-3 have expanded our understanding and utilization of such models in varied real-world scenarios. This section will delve into the recent advancements and experiments done in the domain, shedding light on how LLMs are being applied, augmented, and furthered to handle few-shot prompted tasks, and enhance emergent abilities.

% \subsection{Few-Shot Prompted Tasks}

% \textbf{Augmentation:} As we move into the realm of harnessing the capabilities of LLMs, \citep{Wahle2022HowLL} stands out by proposing a unique technique using few-shot examples to prompt GPT-3. This enabled the generation of machine-paraphrase plagiarism data extracted from freely accessible scientific articles. Simultaneously, \citep{Agrawal2022LargeLM} undertakes an ambitious exploration of GPT-3's prowess in zero- and few-shot information extraction from clinical texts. What's noteworthy is that this approach sidesteps the need for domain-specific fine-tuning. Their research also extends to various NLP tasks demanding structured outputs, touching on areas like span identification, sequence classification at the token level, and relation extraction.

% \textbf{Reasoning:} The subtlety of reasoning is captured in \cite{Madaan2022LanguageMO}, where the focus is on improving the structural accuracy of commonsense reasoning, hinging on few-shot examples applied on code-generation LLMs. Intriguingly, \citep{Spiliopoulou2022EvEntSRE} tests the waters of LLMs' underlying and worldly knowledge. They probe the model's capability to reason event implications using two datasets while also evaluating the effects of varied prompting techniques on model performance. Meanwhile, \cite{Min2022RethinkingTR} demystifies the paradigm of in-context learning, emphasizing that for large language models, having ground truth demonstrations might not be mandatory for efficient in-context learning.

% \subsection{Augmented Prompting Strategies}

% \textbf{Provide External Knowledge:} The question of how LLMs utilize their vast internal knowledge has been the subject of extensive research. \cite{Li2022LargeLM} takes the lead by exploring LLMs' controllability and robustness, unveiling their shortcomings in leveraging knowledge from input contexts. Their proposed solution – a context-aware finetuning methodology – infuses counterfactual and irrelevant contexts into standard supervised datasets, addressing the challenge head-on. Similarly, \cite{Lal2022UsingCK} delves deeper into the sphere of commonsense knowledge, proposing techniques to supplement the model with additional context to better answer ``why'' questions.

% \textbf{Data Augmentation:} In an artful confluence of technology and literature, \citep{Chakrabarty2022HelpMW} introduced CoPoet, a system promoting collaborative writing with LLMs. On a different tangent, \citep{CallisonBurch2022DungeonsAD} presents a novel perspective by viewing textual-based games through the lens of a dialogue system, demonstrating LLMs' capability to pinpoint game stages via dialogue.

% \textbf{Chain-of-thought like methods:} Handling intricate queries that demand latent decisions is a formidable challenge, one that \cite{Dua2022SuccessivePF} tackles by iteratively simplifying a complex task. The process involves breaking down the task, resolving it, and repeating the cycle till a conclusion is reached. This approach is mirrored by \cite{Patel2022IsAQ} who propose question decomposition to aid LLMs in understanding and answering complex queries.

% \textbf{Reasoning:} The art of questioning is an avenue that \cite{Shridhar2022AutomaticGO} ventures into, focusing on sequential question generation to aid in solving mathematical word problems. In the broader context of understanding social dynamics and inferring mental states, \cite{Sap2022NeuralTO} exposes the current limitations of LLMs like GPT-3.

% \subsection{Methods to Enhance Emergent Ability}

% \textbf{Training Methods:} A novel approach to unsupervised masking strategy, InforMask, is introduced by \cite{Sadeq2022InforMaskUI}, leveraging Pointwise Mutual Information to pinpoint and mask the most informative tokens, showing immense potential in question-answer tasks. Meanwhile, \cite{Ke2022ContinualTO} puts forth the idea of perpetually extending LLMs via post-training with a sequence of domain-specific unlabeled corpora. Lastly, \cite{Zhang2022ActiveES} introduces a reinforcement learning algorithm, identifying policies that handpick generalizable demonstration examples, a feat that holds promise in tasks unseen during training.

\subsection{Pre-training Data}
% The investigation of the pre-training stage's influence on emergent abilities starts from \citep{DBLP:conf/nips/BrownMRSKDNSSAA20}, in which they introduced and delved into the capabilities of GPT-3. Their study shows that GPT-3 can perform various tasks without requiring task-specific fine-tuning. The researchers employed three popular ICL methods in their paper: zero-shot, one-shot, and few-shot. In this approach, instead of inundating the model with vast amounts of training data, GPT-3 was provided with just a few examples to help it understand or even without any examples in zero-shot cases and execute the task. \citet{DBLP:conf/nips/BrownMRSKDNSSAA20} comprehensively experimented on diverse datasets spanning nine categories of natural language tasks. Their finding suggests that zero-shot performance improves steadily with model size, few-shot performance increases more rapidly, demonstrating that larger models are more proficient at ICL. Besides, \citet{DBLP:conf/nips/BrownMRSKDNSSAA20} designed algorithms to detect the influence of ICL performance on contaminated pre-training data. Interestingly, they found that the language model is relatively insensitive to contamination. Still, they also observed it may bring some performance inflation on small parts of the dataset span tasks like reading comprehension, translation, etc.
%Several literature 
Some studies have suggested that factors related to pre-traning data such as data domain, data term frequency, and data distribution~\citep{Chan2022DataDP, Razeghi2022ImpactOP}, are crucial elements influencing the development of emergent abilities.
\paragraph{Data Domain} 
\citet{Shin2022OnTE} conducted a study to explore the variations of ICL performance concerning the \textit{domain source} and the \emph{size} of the pre-training corpus, focusing primarily on the Korean lexicon. 
They utilized seven subcorpora from the HyperCLOVA corpus \cite{Kim2021WhatCC} to pre-train various language models and evaluated these models on Korean downstream tasks. 
Interestingly, \citet{Shin2022OnTE} found that the size of the pre-training corpus does not always determine the emergence of ICL. 
Instead, the domain source of the corpus significantly influences ICL performance. 
For example, language models trained with subcorpora constructed from blog posts exhibited the best ICL capability. 
This phenomenon may be attributed to the greater token diversity presented in the blog posts corpus compared with other sources like news. Moreover, their experiments highlighted that combining multiple corpora can lead to the emergence of ICL, even if individual corpora did not produce such learning on their own. 
Surprisingly, \citet{Shin2022OnTE} also found that a language model pre-trained with a corpus related to a downstream task did not always guarantee competitive ICL performance. 
For instance, a model trained on a news-related dataset~\citep{Park2021KLUEKL} showed  superior performance in zero-shot news topic classification, but its few-shot performance was not superior.
In a similar vein, \citet{Han2023UnderstandingIL} reached similar conclusions and highlighted the importance of long-range pre-training data.
Their experimental observation can be summarized into two aspects: (1)  supportive pre-training data do not necessarily need to have a higher domain overlap with downstream tasks, and (2) supportive pre-training data with relatively lower frequency and long-tail tokens make larger contributions to the model's capabilities.
Furthermore, ~\citet{Raventos2023PretrainingTD} empirically observed improvements in MNIST classification performance as pre-training data task diversity increases. 
\vspace{-3mm}
\paragraph{Data Term Frequency}
A more comprehensive investigation into the factors affecting  emergence of the ICL based on the pre-training term frequencies has been proposed by \citet{Razeghi2022ImpactOP}. 
The authors focused particularly on a crucial type of reasoning in LLMs -- numerical reasoning in few-shot settings; and examined the extend to which the frequency of terms from the pre-training data correlates with model performance in these situations. 
Their analysis focused on the prevalence of numerical reasoning tasks within the training instances and established a connection between frequencies and reasoning performance. This connection is quantified  by introducing the ``performance gap'', which is defined as the accuracy of terms appearing more than 90\% of the time minus the accuracy of terms appearing less than 10\% of the time. They conducted their experiments using GPT-based language models trained on the Pile dataset \citep{Gao2020ThePA}, ranging in size from 1.3B to 6B parameters. Evaluation was carried out on 11 datasets spanning three types of mathematical reasoning tasks: Arithmetic, Operation Inference and Time-Unit Conversion. 
The findings consistently show that models perform better in instances where terms from the pre-training data are more prevalent \citep{Razeghi2022ImpactOP}. 
In some scenarios, there is a substantial performance gap of over 70\% between the most frequently occurring terms and the least frequently occurring terms. 
The significant performance difference raises questions about the actual generalization capabilities of these models beyond their pre-training data. 
\citet{Razeghi2022ImpactOP}'s observations suggest that the more prevalent content included in the pre-training data may exert an influence on the emergent abilities, and it is possible that these language models are not actually reasoning to solve arithmetic tasks. 
In line with this research, ~\citet{Kandpal2022LargeLM} also observed a positive correlation between model's memorisation ability and the frequency of the pre-training samples.
\vspace{-2mm}
\paragraph{Data Distribution} 
\citet{Chan2022DataDP} provided an interpretation of the emergent abilities of transformers from the perspective of training \textit{data distribution}. They conducted an experiment based on image classification to explore the emergent capabilities of language models, particularly in the context of performing few-shot learning without explicit training. 
They manipulated the distributional properties of the training data, such as burstiness and the presence of rare classes. They observed the impacts of these manipulations on ICL using models like transformers and recurrent networks. %Similar to \citet{DBLP:conf/naacl/ShinLAKKKCLPHS22}'s finding on the diversity of the lexical may benefit the emergence of ICL, 
\citet{Chan2022DataDP} also emphasised that ICL is more pronounced when item meanings or interpretations are dynamic rather than static. They highlighted that natural language and other naturalistic data sources exhibit these dynamic properties, which differ from the uniform distributions typically used in standard supervised learning. Through their experiments, they uncovered a trade-off in transformer models between ICL and in-weights learning, which relies on information stored through slow, gradient-based updates \citep{Chan2022DataDP}.
However, subsequent experiments showed that both ICL and in-weights learning could coexist in a model trained on data with a skewed Zipfian marginal distribution \citep{Zipf1949HumanBA}, a distribution commonly observed in the frequency of words in languages.
Interestingly, while transformers exhibited ICL when trained on specific data distributions, recurrent models like LSTMs and RNNs did not. Finally, they revealed that non-uniform training distributions can cause the induction of the emergence of new capabilities. Their work highlights the importance of architecture and training data distribution in the emergence of ICL in LLMs. 
To understand how data distribution affect the effectiveness of ICL, \citet{Wies2023TheLO} theoretically demonstrated that unseen tasks can be efficiently learned via ICL when the pre-training data distribution comprises a mixture of latent tasks.
% \citet{DBLP:journals/corr/abs-2301-07067} framed the ICL as a Multiple Task Learning(MTL) problem and identified the transfer learning risk on unseen tasks is governed by the task complexity and the number of MTL tasks.

% \citep{DBLP:conf/nips/BrownMRSKDNSSAA20, Wei2022EmergentAO} pointed out that LLMs' emergent abilities on a range of downstream tasks improve while increasing the scale of language models. 
% However, new line of studies discovered that the emergence of ICL are not limited by the \textit{model scale} factor. Other factors such as data distribution \citep{Chan2022DataDP, Razeghi2022ImpactOP}, source of data \citep{DBLP:conf/naacl/ShinLAKKKCLPHS22, Razeghi2022ImpactOP, Power2022GrokkingGB}, and model architecture \citep{Chan2022DataDP, Xie2021AnEO} may also plays an important role at the pre-training stage. Researchers proposed to investigate the emergent ability either theoretically \citep{Wei2022EmergentAO, Xie2021AnEO} or empirically \citep{Chan2022DataDP,DBLP:conf/naacl/ShinLAKKKCLPHS22,Razeghi2022ImpactOP,Power2022GrokkingGB}. %Interestingly, most of those interpretations focus on LLMs' reasoning capability. 

% In this section, we first summarise the existing approach to interpret the gain of LLMs' emergent abilities in the pre-training stage. Then, we discuss the existing challenges and barriers in current research. Finally, we present some possible future work directions for future interpretation of the emergent abilities.

\subsection{Pre-training Model}
\citet{Wei2022EmergentAO} embarked on an investigation into the emergent abilities of LLMs. 
Their approach to interpreting this phenomenon involved conducting a comprehensive survey of existing literature and analyzing the unpredictable nature of how certain abilities manifest as these models scale \citep{Brown2020LanguageMA}. 
\citet{Wei2022EmergentAO} emphasized that while \textit{model scale} has been correlated with LLM performance, it is not the sole determinant. Task-specific abilities can also be examined by considering a language model’s performance (perplexity) on general text corpora, such as WikiText103. 
Their experiments showed that, despite having fewer parameters, the PaLM 62B model outperformed LaMDA 137B and GPT-3 175B in certain tasks. This suggests that other factors, like \textit{high-quality data} and \textit{architectural differences}, also play a role.
Moreover, continued pre-training on different objectives, like the mixture-of-denoisers objective, has shown potential in enabling emergent abilities \citep{Tay2022UnifyingLL}. 
Research is also advancing to make these discovered abilities accessible for smaller-scale models.
For instance, instruction-based fine-tuning showed potential in smaller models with different architectures. 
Additionally, the emergence of syntactic rule-learning can be triggered by threshold frequencies in training data, similar to "aha" moments in human learning \citep{Abend2017BootstrappingLA, Zhang2020WhenDY}.
 While the majority of research agrees that model scale is a key factor for emergent abilities. 
 \citet{Kirsch2022GeneralPurposeIL} presented an interesting perspective. They found that among the factors determining the inductive bias of the model, the state-size (such as the hidden state size in a recurrent network) is a more crucial parameter than the overall model size for the emergence of ICL ability.


 % ~\citet{Kirsch2022GeneralPurposeIL} studies this issue in a more systematic way, they identify a task diversity threshold for the emergence of ICL by deriving the optimal estimator on linear regression problems. 

%%



% \citet{Chan2022TransformersGD}'s study also delves into the generalization behaviours of transformer models, focusing on the difference between in-weight learning and in-context learning. 
% More detailly, they want to understand whether transformer models generalize using rule-based methods (e.g. relying on minimal features) or exemplar-based methods (e.g. relying on the similarity to training instances). They designed their experiment on both syntactic and textual data with a ``partial exposure'' test \citep{Dasgupta2021DistinguishingRA} where stimuli have two features, and the model is trained on certain combinations and then evaluated on a held-out combination.
% \citet{Chan2022TransformersGD} find out that transformers display a distinct difference in their generalization from in-context vs. in-weights information.
% Specifically, they found that transformers lean towards exemplar-based generalization from in-context details, while they favour rule-based generalization from in-weights information.
% However, large transformer models pre-trained on language showed more substantial rule-based generalization from in-context information.

% A more theoretical framework to explain the emergence of ICL when pre-training with documents has long-range coherence has been developed by \citet{Xie2021AnEO}, especially when there's a distribution mismatch between prompts and pre-training data. The authors expanded on the probability of a given output based on a set of examples and a test input. Under their framework, They derived heuristics and made several assumptions related to the bounds on delimiter transitions, distribution shifts, and regularity. \citet{Xie2021AnEO} find out that language models tend to infer from a latent document-level concept to generate new tokens during pre-training, and language models tend to match this shared latent concept during the inference to emerge the ICL behaviour. Their theoretical investigation is grounded in a Bayesian inference perspective. \citet{Xie2021AnEO} carried out experiments on both transformers and LSTM-based models trained with a small-scale synthetic dataset (GINC) based on a uniform mixture of Hidden Markov Models (HMMs). During the model inference stage, they generate textual prompts from a sampled latent concept to validate their theoretical framework. Their quantitative and theoretical analysis highlights that the in-context predictor is optimal as the number of in-context examples increases. ICL tend to emerge when the pre-training data distribution is a mixture of HMMs.




\subsection{Demonstration Examples}
% With the scaling of model size and corpus size, LLMs have shown impressive performance on downstream tasks by conditioning on a prompt that contains a few input-label pairs from training data~(Demonstrations), which has been referred to as in-context learning~(ICL, \citet{Brown2020LanguageMA}).
Several recent  studies~\citep{Liu2021WhatMG,Min2022RethinkingTR,an2023context} have revealed the sensitivity of emergent abilities to alterations in order, format, and quantity of provided demonstrations.
\paragraph{Demonstration Order}
The order of the demonstrations has a significant impact on downstream task performance. \citet{Lu2021FantasticallyOP} showed that it be the deciding factor between achieving near state-of-the-art and random guessing. 
They designed demonstrations containing four samples with a balanced label distribution and conducted experiments involving all 24 possible permutations of sample orders.
The experimental results showed that the performance variations among different permutations exist across various model sizes, especially for smaller models.
Besides, is was observed that effective prompts are not transferrable across models, indicating that the optimal order is model-dependent, and what works well for one model does not guarantee good results for another model.

% Similar phenomena were discovered in .
\citet{Zhao2021CalibrateBU} identified a phenomenon that LLMs tend to repeat answers found at the end of  demonstrations, which they termed ``recency bias''.
Similarly, in multi-document question answering and key-value retrieval tasks,~\citet{Liu2023LostIT} made analogous observations.
These tasks involve identifying relevant information within lengthy input contexts.
The results showed that LLMs performed best when the relevant information is located at the beginning or end of their input contexts.
However, their performance degraded when they are forced to use information from the middle of their input.
In addition, they noted that model performance declines as the input context length increases, suggesting that current models struggle to effectively reason over their entire context window.
%\citet{Liu2023LostIT} studied the how the position of the information affects the model's performance and came to the conclusion that the performance would be the highest if relevant information occurs at the beginning or end of the input context.
Although these studies offer insights into how demonstration order influences emergent abilities, they do not delve into the underlying reasons of these obesrvations.
In an effort to investigate the impact of semantic similarity between ICL examples and test examples on downstream task, \citet{Liu2021WhatMG} proposed retrieving examples semantically similar to a test example for creating its demonstration.
They utilized the CLS embeddings from a pre-trained RoBERTa-large~\citep{Liu2019RoBERTaAR} model to represent sentences and assessed the semantic similarity between two sentences by computing the cosine similarity of their respective representations. 
For each test example, they identified the nearest $K$ neighbors from the training set and concatenated them in descending order of semantic similarity to create the demonstration. 
Their experiments on Web Questions~\cite{Berant2013SemanticPO} and Trivia Question Answering~\cite{Joshi2017TriviaQAAL} benchmarks showed that the default order performed slightly better than the reverse order.
However, the reverse order performed better on the Natural Questions \cite{Kwiatkowski2019NaturalQA} benchmark.
Consequently, the choice of order appears to be dependent on the specific dataset in use.
%is data-dependent.


\paragraph{Input-Label Mapping}
Some studies have been conducted to investigate how input-label mappings influence the performance of ICL.
\citet{Min2022RethinkingTR} revealed that substituting the correct labels of in-context examples in demonstrations with random labels only leads to a marginal decrease in performance across a variety of classification and multi-choice tasks.
They also conducted ablation experiments to investigate the impact of the number of correct labels on performance. Surprisingly, the results showed that the performance was not sensitive to the number of correct labels in demonstrations.
This led to the counter-intuitive conclusion that LLMs do not heavily rely on input-label mappings to perform tasks.

However,~\citet{Kim2022GroundTruthLM}, \citet{Wei2023LargerLM}, and \citet{Kossen2023InContextLI} disagreed with the claim put forth by \citet{Min2022RethinkingTR}.
\citet{Kim2022GroundTruthLM} pointed out that the claim exhibited over-generalization in two aspects: (1) Aggregating the mean performance across various datasets was found to be inadequate in capturing the insensitivity behavior observed within individual datasets. (2) The experimental setting lacked generalizability, and the results were sensitive to minor adjustments to the experimental setup.
To delve deeper into the topic of input-label mapping, \citet{Kim2022GroundTruthLM} introduced two novel metrics.
The first metric, Label-Correctness Sensitivity, quantifies the impact on downstream classification performance when a fixed amount of label corruption is introduced into the demonstration. 
The second metric, Ground-Truth Label Effect Ratio, assesses how much the presence of ground-truth labels improves the performance compared to a baseline with random labels. 
Their experimental results showed that the sensitivity exhibited significant variation across 17 datasets, with  the aggregate sensitivity considerably high. This indicated that label correctness does indeed affect downstream task performance.
Furthermore, \citet{Kim2022GroundTruthLM} suggested a strong correlation between sensitivity and task difficulty, revealing that LLMs displayed low sensitivity on challenging tasks.

\citet{Wei2023LargerLM} further explored how semantic priors and input-label mappings affect ICL.
They suggested that LLMs possess the ability to override semantic priors from pre-training in favour of input-label mappings from demonstrations. This explains why the performance of LLMs drops below random guessing when all the labels in the demonstrations are flipped.
They also found that smaller models experienced a less severe decline in performance because they lack the capacity to override semantic priors to the same extent.
More specifically, \citet{Wei2023LargerLM} conducted experiments where they replaced the labels with semantically unrelated labels. The results showed that the performance drop was more significant for small models compared to LLMs.
This led them to suggest that small models rely heavily on the semantic meanings of labels rather than learning the input-label mappings provided in the demonstrations.
\citet{Kossen2023InContextLI} also found that larger models are more sensitive to randomized labels, and they highlighted that LLMs can learn new input-label mappings from  demonstrations.
 
However, the ability to learn new input-label mappings can, at times, have adverse effect on performance.
\citet{Tang2023LargeLM} revealed that LLMs sometimes tend to exploit shortcuts within demonstrations for downstream tasks.
These shortcuts represent spurious correlations between in-context examples and their corresponding labels.
\citet{Tang2023LargeLM} designed several types of shortcuts,  and their experimental results showed that LLMs are ``lazy reasoners''. 
They relied heavily on the shortcuts within demonstrations to deduce the final answers. 
Furthermore, \citet{Si2023MeasuringIB} discovered that when presented with a set of non-specific demonstrations (For example, the labels are semantically unrelated), LLMs exhibited feature bias. 
This indicated that LLMs tend to favour one feature over another, even when both features are equally capable of predicting the label, as mentioned in the prompt.
For example, in a sentiment analysis setting, LLMs showed a significant bias towards predicting labels based on sentiment rather than shallow lexical features.
Nevertheless, feature bias has the potential to detrimentally affect performance when the model's feature bias does not align with the intended task.
\citet{Si2023MeasuringIB} suggested that certain interventions could help mitigate feature bias, such as employing natural-language instructions and incorporating label words that have semantic relevance to the intended feature.

To further investigate the underlying mechanism of how LLMs learn from input-label mappings, \citet{Wang2023LabelWA} conducted an extensive study into the workings of ICL from the perspective of information flow.
They computed saliency scores for each element within the attention matrix to unveil the significant token interactions.
The experimental results demonstrated that label words within demonstrations play a crucial role in this process. Specifically:
(1) During the processing of shallow computation layers, semantic information becomes concentrated within the representations of label words.
(2) The aggregated information contained within label words serves as a reference for the final predictions made by LLMs.
Their findings confirmed that label words can indeed have a substantial impact on the performance of the final task.

\paragraph{Chain-of-Thought Prompting}
% LLMs have shown superior performance on many downstream tasks by producing step-by-step reasoning, which is referred to as Chain-of-Thought (COT) reasoning.
% Chain-of-thought prompting is an approach that encourages LLMs to generate their intermediate rationales for solving a problem.
% It can considerably improve the multi-step reasoning ability of LLMs and is accomplished by providing the LLMs with some in-context examples in demonstrations where the reasoning process is explicitly outlined.
Some studies have focused on exploring the impact of COT prompting on LLM performance.
\citet{Wang2022TowardsUC} found that the validity of the reasoning process in demonstrations has only a minimal impact on performance. 
To assess this, they constructed invalid reasoning processes manually for all in-context examples. Surprisingly, the experimental results showed that LLMs can retain 80-90$\%$ of their performance even when presented with invalid reasoning steps in demonstrations.
They also found that the coherence of the reasoning process and its relevance to the query are significantly more crucial factors for the effectiveness of CoT.

Regarding the explanations generated by LLMs, \citet{Turpin2023LanguageMD} found that CoT explanations produced by LLMs can occasionally misrepresent the true underlying rationales behind their predictions.
They introduced two types of bias in the prompt design to investigate this phenomenon.
The first bias involves consistently reordering the multiple-choice options of in-context examples to make the answer 'A'.
The second bias entails including the suggested answers directly in the prompt.
The experimental results indicated that, in both bias scenarios, LLMs tend to provide answers aligned with stereotypes and generate explanations that do not faithfully support the answer.
Furthermore, there was a large drop in performance when comparing biased demonstrations to unbiased demonstrations.

% \subsubsection{Future direction}
% The current research is limited to specific closed-set tasks, which include classification and multiple-choice tasks.
% Additionally, it encompasses the examination of some template-based reasoning tasks.
% Future research endeavours could broaden their scope to encompass more intricate and open-set tasks, such as generation and tasks involving complex reasoning.





































% % \subsection{Model Architecture}
In this section, we will look into the transformer~\cite{DBLP:conf/nips/BrownMRSKDNSSAA20} components from the perspective of: 1) what makes transformers pay attention to their context, 2) how they can fit into different downstream tasks, and 3) under which circumstances, they come into play~\cite{von2023transformers}.

\subsection{Preliminary}
Before diving into the related works about how transformer components contribute to the core ability -- in context learning ability, we briefly give the formal notions of the transformer blocks, especially the Self-Attention Network (\textsl{SAN}). For each block, the input word index $t$ is firstly transformed by an Embedding layer $W_{E}$, then the \textsl{SAN} is essentially to incorporate the information from surrounding tokens via multiple heads, $h\in H$ and from itself. The resulting representation $x_1$ is finally transformed back to the output word index $T(t)$.

\begin{eqnarray}
    T(t)& =& W_{U}x_1,\nonumber  \\
    x_1& =& x_0+ \Sigma_{h\in H}h(x_0)\nonumber  \\
    x_0& =& W_E t
\end{eqnarray}

\citet{elhage2021mathematical} developed a mathematical framework for SAN decomposition and found that the OV circuit defined as $W_U W_E$ controls token embedding flows directly down its own path without intervening with its surrounding tokens' information. In contrast, the QK circuit in $h$-head, $A^{h} = \text{softmax}(t^{T}\cdot W_E^{T}W_{QK}^{h}W_{E}\cdot t)$ provides the attention scores between each query and key token, contributes to the so-called induction head, which only occurs in two-layer SAN, knows how the token was previously used and looks out for similar cases in the provided context, making the model adaptive to downstream tasks.

\subsection{Beyond Copy to Function Approximation}
The findings in \cite{elhage2021mathematical,olsson2022context} are limited to two-layer SAN (without MLPs) and copying mechanisms, although the empirical results in ~\citet{olsson2022context} show that induction heads are capable of more sophisticated tasks, for example, machine translation, and a specific type of abstract pattern matching in a 40-layer model with 13 billion parameters. 

Later, researchers give proof on how the SAN (without MLPs) can approximate linear algorithm~\cite{dai2023can,DBLP:conf/nips/0001TLV22,DBLP:conf/iclr/AkyurekSA0Z23} without gradient descent. In their settings, the ICL is framed as meta-learning and fed with a sequence of tasks $D$ and the ICL learner $\text{TF}_{\theta}$ is actually to find optimal parameter $\theta$ to approximate a certain algorithm $f$ (here, $f$ is linear algorithm): 
\begin{align*}
   D &=  \{x_i,f(x_{i}),...,f(x_{n}\}\\
\text{argmin}_{\theta} &= \sum_{i=1}^{n} L(f(x_{i}),\text{TF}_{\theta}(D))
\end{align*}



\citet{irie2022dual} presented that linear
layers optimized by gradient descent (GD) have a dual
form of linear self-attention network(LSN)-without softmax in attention weight calculation. The following works demonstrated that TF’s ability to implicitly execute gradient descent steps during inference could also be central to ICL, supporting their claims with empirical evidence. \citet{dai2023can} gave the dual form of linear model gradient descent and LSN output on in-context samples.~\citet{DBLP:conf/nips/0001TLV22} empirically demonstrated that TFs can learn basic function classes (e.g., linear functions, MLPs, and decision trees) via input
sample sequences.~\citet{DBLP:conf/iclr/AkyurekSA0Z23} consider the complete TF structures by incorporating the softmax activation, MLP with GeLU and empirically demonstrated that in computational level ICL can approximate various regression problems, such as KNN, One-pass stochastic gradient descent, One-step batch gradient descent and Ridge regression. Based on the prior results,~\citet{von2023transformers} further enables solving nonlinear regression tasks, i.e., sinWave fitting, within transformers by showing its equivalence to learning a linear model on deep representations. ~\citet{Xie2021AnEO,swaminathan2023schema,DBLP:journals/corr/abs-2304-09960} formulated the LLM as a latent variable model and demonstrated its ability of Bayesian inference.



% \subsection{Effects of Hyper-Parameters}
% \paragraph{Model Parameter}\citet{DBLP:journals/corr/abs-2212-04458} find that among factors determining the inductive
% bias of the model, \textsl{state-size} ((such as the hidden state size in a recurrent network) is a more crucial parameter than the model size for the emergence of in-context learning ability. Instead, more research results show that \textsl{model size} is the key factor for emergent ability, i.e., the scaling law~\cite{DBLP:journals/corr/abs-2303-03846,DBLP:journals/corr/abs-2001-08361,DBLP:journals/corr/abs-2203-15556}. For example, PaLM-540B are capable of overriding semantic priors in text classification tasks, while smaller counterparts are unable to do so~\cite{DBLP:journals/corr/abs-2303-03846}. 
% \paragraph{Prompt Parameter} In addition to the parameters in the model itself, the performance in ICL also heavily relies on prompt settings~\cite{DBLP:conf/acl/CaoLHL022,DBLP:conf/acl/LuBM0S22,DBLP:journals/csur/LiuYFJHN23,DBLP:journals/corr/abs-2212-04037,DBLP:conf/acl/SorensenRRSRDKF22}: such as \textit{wording}, \textit{perplexity}, \textit{order}, \textit{label distribution} and the \textit{mutual information} between the prompt and the
% language model’s output. More fine-grained, ~\citet{DBLP:journals/corr/abs-2307-03172} studied the how the position of the information affects the model's performance and came to the conclusion that the performance would be the highest if relevant information occurs at the beginning or end of the input context.  Most of the above studies are based on empirical experiments. In order to avoid the effects from confound, \citet{DBLP:conf/acl/CaoLHL022,DBLP:conf/acl/StolfoJSSS23} study the causal effects of prompt variable to model performances by backdoor criteria. On the other hand, some researchers explained the prompt variation from a cognitive perspective~\cite{DBLP:journals/corr/abs-2206-14576}. They designed the experiments to evaluate LLM's cognitive ability, including decision-making, information search, deliberation, and causal reasoning, by applying small perturbations to the prompts. On the other hand, researchers give theory proof of how the prompt parameters affect the performances based on simplified settings. For example, ~\citet{DBLP:journals/corr/abs-2305-19420,Xie2021AnEO} give theory proof that a larger length of the ICL input sequence, i.e., prompt (with possible demonstrations) length can help decrease the regret bound of ICL. 

\subsection{Strengthen by CoT.}
All the above works investigated the power of TFs from an expressively perspective, i.e., the function approximation. However, they pay little attention to the striking contributions from the carefully-designed prompt~\cite{DBLP:journals/tacl/JiangXAN20,DBLP:journals/csur/LiuYFJHN23,Wang2022TowardsUC}. In particular, the so-called Chain-of-Thought prompting (COT) plays an important role in complex reasoning tasks, e.g., mathematical arithmetic. By adding \textit{let's think step by step} to guide the model to generate intermediate output and derive final striking results based on it~\cite{Wei2022EmergentAO}. Empirically, researchers show the factors in CoT affecting the reasoning performances to better understand its working mechanism, such as relevance to the query, the reasoning order, etc~\cite{Wang2022TowardsUC}.~\citet{DBLP:journals/corr/abs-2307-13339} found that CoT would not increase the magnitude of the saliancy score of the important input tokens, but indeed enhance its robustness under different input perturbations. ~\citet{DBLP:journals/corr/abs-2305-15408} takes the first step towards theoretically answering how the complex reasoning tasks can be solved by CoT-assisted LLMs.

\subsection{Challenges and Future Work}
Research methods focusing on the problem of how the model structure and parameters contribute to the emergent abilities can be roughly divided into two directions: (1) ablating the research subject and other factors, and then empirically observing the differences in task performances. (2) Simplifying the real situations by either removing the obscure model components or evaluating with delicately designed synthetic experiments, in order to theoretically deduct the effects from principle elements to output. The findings of the first group of methods are often limited to specific evaluation cases, e.g., the variation of subjects in mathematics can affect LLM calculation
results~\cite{DBLP:conf/acl/StolfoJSSS23,shi2023large}. The conclusions from the second line of research hold true only in restricted situations, e.g., without noise data, nor regularisations~\cite{von2023transformers}. These issues are very common in other research areas, and what we can do to push the limits forward in this direction can be summarized as (a) proposing advanced TF structures with better computation efficiency based on the observation that the self-attention layer actually plays a similar role in optimization~\cite{zucchet2022beyond}. (b) Designing a systematic evaluation schema to evaluate the LLM's capability, one can refer to the cognitive literature for detailed ability measurements and implications. (c) Extending to more learning problems and incorporating more realistic settings in order to better approximate 
the real solutions. (d) Additionally, very limited studies theoretically illustrate the success of Chain-of-Thought, which is the key factor in understanding the LLM's reasoning ability.

\section{Training Data Matters}
Unlike prior work that explores implicit mechanisms behind ICL, some researchers study ICL via investigating how the pretraining data intrigue the emergent abilities.

% \subsection{Task Diversity}
% Although the previous studies~\cite{dai2023can,DBLP:conf/iclr/AkyurekSA0Z23,von2023transformers} have also shown that transformers can do linear regression in ICL. However, their assumptions are based on unlimited task diversity, that is the pretrained tasks are the same as true evaluation tasks.~\citet{wies2023learnability} show that unseen tasks can be efficiently learned via ICL
% in-context learning, when the pretraining distribution is a mixture of latent tasks. ~\cite{DBLP:journals/corr/abs-2212-04458,DBLP:journals/corr/abs-2306-15063} show the emergence of in-context learning with pretraining task diversity.~\cite{DBLP:journals/corr/abs-2212-04458} studies this issue in a more systematic way, they identify a task diversity threshold for the emergence of ICL by deriving the optimal estimator on linear regression problems. While lots of work~\cite{DBLP:journals/corr/abs-2306-15063} empirically observe improvement of MNIST classification performance with pretraining task diversity increased.~\citet{DBLP:journals/corr/abs-2301-07067} framed the ICL as a Multiple Task Learning(MTL) problem and identified the transfer learning risk on unseen tasks is governed by the task complexity and the number of MTL tasks. 

% \subsection{Pretraining Data Distribution}
% In addition to the abundance of pertaining tasks, some other important properties of the pertaining data also matter to the emergence of ICL. \citet{DBLP:conf/naacl/ShinLAKKKCLPHS22} discovered that ICL heavily relies on the \textsl{domain relevance}, but pretraining on a dataset related to a downstream task does not always reflect a competitive ICL performance. 
% ~\citet{DBLP:conf/nips/ChanSLWSRMH22} identify \textsl{training distribution—burstiness} and \textsl{occurrence of rare classes}—that are necessary for the emergence of ICL, although their simulation experiments are based on image data.~\citet{DBLP:conf/acl/HanSMTCW23} obtained the similar conclusions: (1) the supportive pretraining data do not necessarily have a higher domain overlap to downstream tasks, (2) the supportive
% pretraining data with \textsl{relative lower frequency} and 
% \textsl{long-tail tokens} have larger contributions. And they addressed the importance of \textsl{long-range} pertaining data.~\citet{DBLP:journals/corr/abs-2303-03846} study how ICL is affected by semantic priors and input-label mappings. Their empirical results show large improvements in flipped-labels presented in-context, implying that they are more capable of using in-context information to override prior semantic knowledge. ~\citet{gu-etal-2023-pre} enhance the language models’ in-context
% learning ability by pre-training the model on
% of intrinsic tasks in the contrastive training manner.~\citet{DBLP:journals/corr/abs-2305-19420} provide a generalisation bound given the number of token sequences and the length of each sequence in pretraining.

% Some studies focus on exploring the influences of particular training samples on the given test samples~\cite{Zhang2021CounterfactualMI,DBLP:conf/icml/KandpalDRWR23,Akyrek2022TowardsTF}, a.k.a, fact tracing. The key factor in such research line is to detect the influential training samples first, then exclude the identified samples in the training phase, so-called counterfactual memorization. More specifically, ~\citet{DBLP:conf/icml/KandpalDRWR23} observed the positive correlation between model's memorization ability and frequency of the pretraining samples. This conclusion is somewhat aligned with that the long-tail knowledge is essential to the difficult tasks in ICL mentioned before. 


\subsection{Challenges and Future Work}
The challenges mainly lie in how to select the relevant important training samples given the test samples and how to avoid retraining the model in a more effective way~\cite{}. Influence functions~\cite{hampel1974influence,DBLP:conf/icml/KohL17,DBLP:conf/nips/PruthiLKS20} are among the first methods to do this for neural networks, by estimating the marginal effect of a training example on the loss of a test example. Basically, there are divided into gradient-based and embedding-based method according to  their measurement units. However, these methods become less practical due to the in-feasibility to model parameters and intermediate output. Besides, the retrain methods that aim to calculate the prediction differences are time-costing. Therefore, how the evaluate the importance of particle training samples in a effective way can be a promising direction.

% For instance, \textcolor{red}{Forward-INF : Efficient Data Influence Estimation with Duality-based  Counterfactual Analysis.!!!fail to find citation, it is a workshop paper, waiting for author's response} 

% % \section{Challenges and Insights}
% % \paragraph{Challenges} Discuss the current challenges and limitations of interpreting LLMs, such as the "black box" problem, difficulties in attributing specific outputs to specific inputs, etc.
% % Discuss the implications of these challenges, for example in terms of fairness, accountability, and transparency.


% % \paragraph{Insights} Categorize the key insights obtained from the interpretation of LLMs so far (e.g., how they handle long-term dependencies, how they generalize from limited data, how they invent narratives, etc.).
% % Discuss representative papers for each category, summarizing their methodologies, findings, and contributions.

% % \begin{itemize}
% %     \item Summarize the benefit of interpreting LLMs from the above perspectives. 
% % 	\item Given the findings and challenges discussed so far, speculate on possible future directions for research on interpreting LLMs.
% % 	\item Discuss how overcoming the current challenges might open up new applications for LLMs, or make current applications more effective, fair, or accountable.
% % \end{itemize}


% \begin{table*}[ht!]
% \centering
% \resizebox{\linewidth}{!}{
% \begin{tabular}{p{0.2\linewidth}p{0.4\linewidth}p{0.4\linewidth}}
% \toprule
% Perspective & Categories & Factors\\\midrule
% \multirow{2}{*}{Macro Perspective} & Linear regression formulation/meta learning & \citep{dai2023can, olsson2022context, von2023transformers, Akyrek2022TowardsTF}\\
%  & Latent space theory/ Bayesian inference & \citep{Xie2021AnEO, Wang2023LargeLM}\\\midrule
% \multirow{12}{*}{Micro-perspective} & \multirow{6}{*}{Pre-training Stage} & Pre-training data distribution\\
%  & & \citep{Chan2022DataDP, Razeghi2022ImpactOP}\\ 
%  & & Domain of pre-training data \\
%  & & \citep{DBLP:conf/naacl/ShinLAKKKCLPHS22, Razeghi2022ImpactOP, Power2022GrokkingGB} \\
%  & & Pre-training model architecture\\  
%  & & \citep{Chan2022DataDP, Xie2021AnEO} \\\cmidrule{2-3}
%  & \multirow{6}{*}{Inference Stage} & Order of demonstration \\
%  & & \citep{Zhao2021CalibrateBU, Liu2019RoBERTaAR, Zhao2021CalibrateBU, Liu2023LostIT}\\
%  & & Input-Label Mapping\\
%  & & \citep{min-etal-2022-rethinking,Kossen2023InContextLI,Kim2022GroundTruthLM, Tang2023LargeLM,Si2023MeasuringIB}\\
%  & & Chain-of-Thought Prompting\\
%  & & \citep{Wang2022TowardsUC,Turpin2023LanguageMD}\\\bottomrule
% \end{tabular}}
% \caption{Summary of factors that have a relatively strong influence on the emergent abilities.} 
% \end{table*}





\section{Challenges \& Future Directions} % for interpreting emergent abilities}
% \subsection{Challenge}
\subsection{Unified Framework}
There is currently no standardized framework available for understanding or interpreting emergent abilities. 
While researchers often investigate factors contributing to emergent abilities based on empirical insights, the resulting conclusion may not always be robust or broadly applicable to real-world applications.
The challenge lie in the multitude of factors that influence emergent abilities, many of which may not be directly modifiable with respect to the abilities themselves, as noted by~\citet{Wei2022EmergentAO}.
For instance, apart from the attention mechanism, ~\citet{Li2023TheCO} found that softmax unit plays a pivotal role in understanding ICL through function regression problems~\cite{Garg2022WhatCT,Akyrek2023WHL,Oswald2022TransformersLI}. 
From the micro-perspective, when examining how the extent of pre-training impacts emergent abilities, data quality serves a crucial role alongside factors like data scale and training time.
 
\subsection{Evaluation Metrics}
Current research efforts typically measure emergent abilities by assessing task performance or optimizing criteria such as gradient~\cite{Oswald2022TransformersLI} and token loss~\cite{Olsson2022IncontextLA} during the pre-training stage.
Another line of research~\citep{Shin2022OnTE, Razeghi2022ImpactOP} has discovered that the relationship between the evaluation measures of language models during training does not strongly correlate with the conventional evaluation metrics, such as F1-score, that have been used to measure performance of emergent abilities under most experimental setups.
However, a dedicated criterion explicitly designed for the assessment of emergent abilities is currently lacking.
In addition, assessing emergent abilities often becomes complicated due to the interwined emergence of other competencies~\cite{Lu2023AreEA}.
In this work, we postulate that the assessment of emergent ability can be based on its capability to produce \textit{satisfactory results} in comparison to a fine-tuned model. 
This approach provides a preliminary framework for devising evaluation criteria. 
However, it is important to note that this methodology is preliminary and not yet comprehensive or definitive. 
Further refinement and development of formal criteria are necessary to establish a robust and universally applicable evaluation metric for emergent ability itself.

\subsection{Cost and Computational Resources}
LLMs faced constraints related to their token capacity, which can lead to deficiency in coherence when dealing with longer demonstration examples or text generation. This limitation can result in challenges for emergent abilities, making it difficult to maintain a consistent and extended logical flow.
What's more, there are some experimental limitations that have hindered the exploration of this type of research. The pre-training stage of these models demands a huge amount of computational resources, which could become a barrier for researchers who lack the necessary resources \citep{Shin2022OnTE, Brown2020LanguageMA, Wei2022EmergentAO, Berglund2023TheRC}. This limitation has restricted investigations into the sources of emergent abilities in commonly used LLMs. Furthermore, the limited knowledge of the detailed lexical resources utilized during the pre-training stage adds complexity to the examination of their abilities \citep{Berglund2023TheRC}.
\vspace{-1mm}
\subsection{Transparency of Training Data}
Some studies \cite{Chan2022DataDP, Razeghi2022ImpactOP, Shin2022OnTE, Power2022GrokkingGB} have emphasized the connection between emergent abilities and the training data. 
It is clear that diverse, clearly structured pre-training data can facilitate the emergence of abilities, such as reasoning. Consequently, understanding how to better evaluate the data quality and how to construct high-quality training data may enable future research to better study the emergent abilities at the pre-training stage.
Hence, our community should refrain from treating the pre-training data of LLMs as black boxes.
Neglecting the role of pre-training data can lead to misinterpretations when assessing the emergent abilities.

\subsection{Causality rather than Correlation}
As demonstrated by \citet{Razeghi2022ImpactOP} and \citet{Power2022GrokkingGB}, intervening on the pre-training dataset, particularly with an emphasis on the emergence of reasoning abilities, offers a promising path for gaining deeper into the question of %interpretation thread to understand better whether the reasoning in the 
whether LLMs indeed possess reasoning abilities. Moreover, as highlighted by \citet{Chan2022DataDP}, delving into the intricacies of in-context and in-weights learning deserves further investigation, especially concerning how prior knowledge is signaled. It is crucial to make comparison between transformers and recurrent architectures, particularly in understanding their in-context learning capacities. What's more, there is a need for a more comprehensive interpretation of the impact of the ``Reversal Curse'' in extensive pre-training datasets for LLMs, considering the varying frequencies of reversed information.
% \subsubsection{Challenges}

% Despite the various progress made in interpreting the efficacy of the pre-training stage on the emergence of ICL, researchers also found some challenges: 

% The first challenge is the evaluation metrics for the pre-training stage. The relationship between the evaluation measurement of language models during the training does not strongly correlate with the performance of emergent abilities under most experiment setups. Therefore, commonly used evaluation metrics, such as perplexity, may hinder language models from maximizing their emergent capability.

% The second challenge is the pre-training data quality issue. Most works from this section \cite{Chan2022DataDP, Razeghi2022ImpactOP, Shin2022OnTE, Power2022GrokkingGB} have emphasized the connection between emergent abilities and the training data. It is clear that diverse, clearly structured pre-training data may urge the emergence of abilities, such as reasoning. Therefore, understanding how to better evaluate the data quality and how to construct high-quality training data may help future research to better study the emergent ability at the pre-training stage.

% What's more, there are some experimental limitations that have hindered the exploration of this type of research. The pre-training stage of these models requires a huge amount of computational resources, which could become a barrier for researchers who lack the necessary resources \citep{Shin2022OnTE, Brown2020LanguageMA, Wei2022EmergentAO, Berglund2023TheRC}. This has limited investigations into popularly used LLMs' source of emergent ability. Furthermore, the limited knowledge of the detailed lexical resources utilized during the pre-training stage makes it difficult to investigate their abilities \citep{Berglund2023TheRC}.

% \subsubsection{Future Work}

% As demonstrated by \citet{Razeghi2022ImpactOP, Power2022GrokkingGB}, intervention on the pre-training dataset focuses mainly on the emergent reasoning ability is a viable interpretation thread to understand better whether the reasoning in the LLM is actual. Moreover, as mentioned by \citet{Chan2022DataDP}, the nuances of in-context and in-weights learning warrant further exploration, especially concerning how prior knowledge is cued. Comparing transformers to recurrent architectures is essential, particularly in understanding their in-context learning capacities. What's more, the impact of the ``Reversal Curse'' in extensive pre-training datasets for LLMs needs further interpretation, given the varied frequency of reversed information.







% \begin{itemize}
% 	\item Summarize the main points of the paper, reiterating why the interpretation of LLMs is important and what the current state of research is.
% 	\item Restate the future directions suggested, and encourage other researchers to continue this line of inquiry.
% \end{itemize}
\section{Conclusion}
This paper has thoroughly reviewed the current research efforts aimed at interpretating and analyzing the emergent abilities of LLMs.
We have categorized these advancements into two main perspectives: 1) from a macro perspective, encompassing studies that focused on inner mechanism of emergent abilities through various theoretical frameworks, such as regression function learning, meta-optimization, and Bayesian inference;
2) from a micro perspective, highlighting studies that prioritize empirical interpretability by investigating factors associated with these abilities
We have identified the existing challenges and suggested potential avenues for further research in this area.
We believe that our work serves as a valuable resource for encouraging further exploration into the interpretation of emergent abilities of LLMs.


\bibliography{tacl2021}
\bibliographystyle{acl_natbib}
\end{document}







