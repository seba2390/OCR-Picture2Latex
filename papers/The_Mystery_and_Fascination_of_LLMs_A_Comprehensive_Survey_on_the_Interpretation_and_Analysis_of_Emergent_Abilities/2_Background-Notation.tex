\section{Background and Notation}
\label{sec:bg}
%could use the term 'reframe' and give a brief rationle to explain the resaon... and highlight ICL can be category into our definitation:  refer to lu's paper are emergent ability xxx about the definition
We refine the definition in \citet{Wei2022EmergentAO}, and specify the emergent abilities as \textit{the abilities that LLMs can leverage to achieve \textit{satisfactory results}\footnote{Results that on par with or potentially superior to those achieved by LLMs fine-tuned for specific tasks or datasets.} across diverse tasks, with only a few examples or chain-of-thought demonstrations, and without the need for re-training.}
%This definition closely corresponds to the second category outlined in  \citet{Wei2022EmergentAO}'s original definition.
%our scope to survey the analytical studies centred on the interpretation of emergent abilities of LLMs.
%Following~\citet{Wei2022EmergentAO}, we define the emergent abilities of LLMs as \textit{the abilities that emerge when LLMs scale up to certain extents and cannot be anticipated only by extrapolating the performance improvements of smaller-scale models.}
%Need to revise: cannot be... to similar to the cited paper
%Specifically, with a few examples or chain-of-thought demonstrations, LLMs can produce \textit{satisfactory results}\footnote{Comparable or even better results than those achieved by LLMs fine-tuned on specifica tasks or datasets.} across various tasks without requiring continuous training or fine-tuning.

%D=\{(x_i,y_i)\};Q=(q_1,...,q_j);Y=(y_1,...,y_j)
Formally, let's define some key variables. $D \in \mathcal T_{train}$ represents a subset of demonstrations selected from the training set.
$Q\in \mathcal T_{test}$ is the query taken from the test set, and $Y$ stands for the label associated with each query. $\mathcal M$ represents the LLM with its parameters frozen as $\Theta$, and $\mathcal F$ denotes the evaluation metric function. 
For example, $\mathcal F$ is typically used to measure accuracy or F1 score in classification tasks, such as sentiment classification, and is often used to represent metrics like ROUGE~\cite{Lin2004ROUGEAP} or BLEU~\cite{Papineni2002BleuAM} in text generation tasks, such as summarization and machine translation.
The concept of emergent ability can be formally expressed using the equation:
\begin{equation}
	\mathcal F (Y,\mathcal M_\Theta(D,Q))
\end{equation}
where $\mathcal M$ is usually considered to exhibit emergent abilities if the computed value using $\mathcal F$ exceeds a pre-defined threshold. 
Under this definition, we can group similar concepts within the few-shot prompting paradigm. 
CoT can be viewed as a variant of ICL, with the primary distinction being the format of the demonstration. 
Specifically, ICL demonstrations typically rely on a standard prompt with optional demonstration examples, whereas  CoT prompting incorporates an additional textual reasoning process.

According to our definition, we organize existing literature (summarized in Table~\ref{tab:summ}) on interpreting emergent capabilities into macro and micro perspectives. 
Researchers in the macro category focus on factors such as overall loss or the model architecture. Their goal is to establish a connection between the outcome of $\mathcal F$ and the behavior of $\mathcal M$.
Conversely, those in the micro category primarily centre their attention on the relationship between the outcome of $\mathcal F$ and the characteristics of the demonstration set $D$.
