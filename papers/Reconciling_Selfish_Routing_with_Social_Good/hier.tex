\section{Solution Concepts Hierarchy}\label{sec:hier}
In this section, we discuss the interrelation between the solution concepts defined in Section~\ref{sec:satisfaction}. 
For completeness, we first state the trivial relation between different $\theta$-flows of the same type.
\begin{proposition}\label{lemm:intra}
For $\theta' > \theta\geq 1$ and $\mc{F} \in \{\text{PNE, UNE, EF}\}$, $\theta$-$\mc{F}$ $\subseteq$ $\theta'$-$\mc{F}.$
\end{proposition}

In the rest of the section, we discuss the relations between different types of flows. 
\begin{lemma}\label{lemm:PNEvUNEvEF}
For any $\theta\geq 1$, we have the containment 
$$\theta\text{-PNE} \subseteq \theta\text{-UNE} \subseteq \theta\text{-EF}.$$
\end{lemma}
\begin{proof}
The first containment is due to the fact that any used path in a network $\mc{G}$ is a positive path in $\mc{G}$. Let $\bm{f}$ be a $\theta$-PNE, then for any commodity $k$, $$\max_{p\in \mc{P}^k_u} \ell_p(\bm{f}) \leq \max_{p\in \mc{P}^k_{+}} \ell_p(\bm{f}) \leq \theta\min_{p\in \mc{P}^k} \ell_p(\bm{f}).$$ Therefore, $\bm{f}$ is a $\theta$-UNE. 

Next, for the second containment let $\bm{f}$ be a $\theta$-UNE. We have for any commodity $k$, $$\max_{p\in \mc{P}^k_u} \ell_p(\bm{f}) \leq \theta\min_{p\in \mc{P}^k} \ell_p(\bm{f}) \leq \theta\min_{p\in \mc{P}^k_{u}} \ell_p(\bm{f}).$$ We conclude that $\bm{f}$ is a $\theta$-EF.
\end{proof}

\begin{proposition}\label{lemm:EFvUNE}
For any $\theta \geq 1$ there exists a network $\mc{G}$ s.t. $1\text{-EF} \not\subset \theta\text{-UNE}.$
\end{proposition}
\begin{proof}
Consider the two parallel link networks with constant latency $1$ in the lower link and latency $\ell(x)=x$ for the upper link. There is a commodity with demand one between the two nodes. Consider the flow using only the lower link. This is a $1\text{-EF}$ flow but as the minimum path has length equal to $0$ it can not be classified as $\theta\text{-UNE}$ for any $\theta$.
\end{proof}

Next, a more detailed relation between $\theta$-UNE and $\theta$-PNE is shown in Lemma~\ref{lemm:UNEvPNE}. 
\begin{lemma}\label{lemm:UNEvPNE}
	The following statements are true:
	\begin{enumerate}
		\item For any $\theta > 1$ and multi-commodity network $\mc{G}$ with $n$ nodes, $\theta\text{-UNE} \subset ((n-1)\theta)\text{-PNE}.$
		\item For any $\theta\geq 1.5$ there exists a single commodity network $\mc{G}$ with $n$ nodes such that $\theta\text{-UNE} \not\subset ((n-3)\theta/3)\text{-PNE}.$ 
		\item (Equivalence of $1$-PNE and $1$-UNE)  For any path flow $\bm{f}$, let $\bm{x}$ be the edge flow induced by $\bm{f}$.  Then $\bm{f} \in$ $1$-UNE if and only if $\bm{x} \in$ $1$-PNE.
		%$1\text{-PNE}=1\text{-UNE}$ and it is unique upto the edge cost.
	\end{enumerate}
\end{lemma}
\begin{proof}
First, consider a multi-commodity network with $n$ nodes and let  $k$ be any of its commodities. Let $\bm{f}$ be a $\theta\text{-UNE}$ with edge flow $\bm{x}$. We have 
\begin{align*}
\frac{\max_{p\in \mc{P}^k_{+}}\ell_p(\bm{f})}{\min_{p\in \mc{P}^k}\ell_p(\bm{f})} 
&\leq (n-1)\frac{\max_{e: x_e^k >0} \ell_e(\bm{x})}{\min_{p\in \mc{P}^k} \ell_p(\bm{f})} \\
&\leq (n-1)\frac{\max_{p\in \mc{P}^k_{u}}\ell_p(\bm{f})}{\min_{p\in \mc{P}^k}\ell_p(\bm{f})} \leq (n-1)\theta.
\end{align*} 

\begin{figure}
\centering
\includegraphics[width=0.57\linewidth]{UNEvPNE-SC}
\caption{$\theta$-UNE vs $\theta$-PNE in Lemma \ref{lemm:UNEvPNE}.}
\label{fig:UNEvPNE-SC}
\end{figure}

For the second part, consider the network in Fig.~\ref{fig:UNEvPNE-SC} with $n=k+2$ nodes. The edge flows are: $(s,u_1) = 1+\epsilon$, $(s,u_i)=1$ for $i=\{2,\dots, k-1\}$, $(u_k,t) = \epsilon$, $(u_i,t) = 1$ for  $i=\{1,\dots, k-1\}$ and $(u_i,u_{(i+1)}) = \epsilon$ for  $i=\{1,\dots, k-1\}$. Let, for  $i=\{1,\dots, k-1\}$, the edges $(u_i,u_{(i+1)})$ have latency $2(\theta - 1)$ and the remaining edges have latency $1$ under this flow.  The path decomposition $f(s,u_i,t)= 1$, $f(s,,u_i, u_{(i+1)},t)= \epsilon$ for  $i=\{1,\dots, k-1\}$ is a $\theta\text{-UNE}$ for this network. Whereas, the ratio of the minimum positive path,  $\ell((s,u_1,t))= 2$, and the maximum positive path $\ell((s,u_1,\dots, u_k, t))= 2+2(k-1)(\theta-1)$ is $\left(1+(k-1)(\theta -1)\right) > (n-3)\\theta/3$ for $\theta \geq 1.5$. The second part of the lemma follows.


For the third part, the \emph{if} direction follows from Lemma~\ref{lemm:PNEvUNEvEF}.  For the \emph{only if} direction, we first note that under UNE, every used path in the same commodity has the same length, which is the Nash length $L^k_{NE}$ (for commodity $k$).  It suffices to show that all positive paths in commodity $k$ have length $L^k_{NE}$ as well.  We prove this by contradiction.  Consider an instance $\mc{G}$ and a UNE path flow $\bm{f}$.  Assume there is a positive path $\pi$ in commodity $k$ with length not equal to $L_{NE}^k$. From the definition of UNE the length of path $\pi$ must be strictly greater than $L^k_{NE}$.  Let $e_1,\dots,e_r$ be the edges in $\pi$ in the order of traversal.  For each $e_i \in \pi$, we choose a path $\pi_i=\pi_i^s-e_i-\pi_i^t$ used by commodity $k$ that uses the edge $e_i$.  Note that under this notation $\pi_1^s=\emptyset$ and $\pi_r^t=\emptyset$. We can see that the sum of the lengths of these paths is $\sum_{i=1}^{r} \ell_{\pi_i}(\bm{f}) = rL^k_{NE}$. Now, consider the paths $\pi_1',\dots,\pi_{r-1}'$, where $\pi_i'=\pi_{i+1}^s-\pi_i^t$.  We can find that 
$$
rL^k_{NE}=\sum_{i=1}^{r} \ell_{\pi_i}(\bm{f}) = \sum_{i=1}^{r-1} \ell_{\pi_i'}(\bm{f}) + \sum_{i=1}^{r} \ell_{e_i}(\bm{f}) = \sum_{i=1}^{r-1} \ell_{\pi_i'}(\bm{f}) + \ell_{\pi}(\bm{f})
$$
According to the definition of UNE, we can see that for $i\in\{1,\dots,r-1\}$, $\ell_{\pi_i'}(\bm{f}) \ge L^k_{NE}$.  Therefore,  we have $\ell_{\pi}(\bm{f}) \le L^k_{NE}$, which contradicts the assumption. As a consequence, for each $k$, all positive paths of commodity $k$ have length equal to $L^k_{NE}$.

%Finally, note that a flow is a $1\text{-UNE}$ if and only if $\ell_p(\bm{f}) \leq \ell_q(\bm{f})$ for all paths $p$ and $q$ and  equality holds if $f_p>0$. But this is equivalent to the complementary slackness condition of the convex optimization problem $\min\left\{\sum_e \int_{0}^{x_e} \ell_e(x)dx: (\bm{x},\bm{f})\in \mc{F}\right\}$. But due to the edge space separation of the objective function in the previous optimization reduces to~\eqref{eq:1PNE}. Therefore, every $1\text{-UNE}$ induces an edge flow $\bm{x}$ which is a $1\text{-PNE}$,i.e. $1\text{-UNE}\subseteq 1\text{-PNE}$. As any valid \pathdecomp of $1\text{-PNE}$ is a $1\text{-UNE}$ they coincide.
   
\end{proof}

%Finally we try to relate the absolute satisfaction, i.e., the $\theta$-\inctv flow with the relative satisfaction notion.
%\begin{lemma}\label{lemm:ICvPNE}
%For any $\theta\geq 1$, $\theta\text{-PNE} \subseteq \left( \frac{d\theta PoA(\theta)}{d_{min}}\right)\text{-\inctv}$. Here $d$ is the total demand and $d_{min}$ is the minimum demand of a commodity. Moreover, $1\text{-\aIC}\not\subset \theta'\text{-EF}$ for any bounded $\theta'$. 
%\end{lemma}
%\begin{proof}
%For any $\theta\text{-PNE}$ flow $\bm{f} \in \mc{D}(\bm{x})$ and $k$-th commodity, let the longest path by $L^k_{max}$ and the average path by $L^k_{avg}$. Further let $\bm{o}$ be the SO flow, $L^k_{NE}$ be the Nash length of the $k$-th commodity and the maximum latency of any path in the MM flow be $L_{MM}$. We have
%\begin{align*}
%d_k L^k_{max} &\leq \theta d_k L^k_{avg} \leq \theta SC(\bm{x}) \leq \theta \text{PoA}(\theta) SC(\bm{o})\\
% &\leq d\theta \text{PoA}(\theta)  L_{MM} \leq d\theta \text{PoA}(\theta)L^k_{NE}.
%\end{align*} 
%Taking the minimum over all the commodities we obtain the bound.
%
%We consider the Pigou's network with two links in parallel with a single commodity with demand $1$. The top link has a latency which varies as $\ell_{t}(x)=x$ and the bottom link with latency $\ell_{b}= 1$. The Nash length of this network is $1$. Therefore, a \pathdecomp with $\epsilon$ flow going  through the top link and the remaining $(1-\epsilon)$ flow through the bottom link, for $\epsilon\in (0,1]$, is a $1\text{-\aIC}$ flow for this network. Further we can make $\epsilon$ arbitrary small so that the two used paths in the above \pathdecomp has a ratio $1/\epsilon > \theta'$, for any finite $\theta'$.
%
%\end{proof}
