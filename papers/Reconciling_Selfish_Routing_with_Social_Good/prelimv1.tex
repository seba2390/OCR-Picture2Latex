\section{Preliminaries}\label{sec:prelim}
\subsection{Network and Flows}
\textbf{Network.} We are given a directed graph $G(V,E)$ with vertex set $V$, edge set $E$, and a set of commodities % $ \mc{K}=\{(s_k, t_k)\}$. 
$\mc{K}=\{1,2,\dots,K\}$.  Each commodity $k \in \mc{K}$ is associated with a source $s_k$ and a sink $t_k$.  We denote $\mc{T} = \{(s_k,t_k)\}_{k \in \mc{K}}$ as the collection of the source-sink pairs for all commodities.
Also, for each commodity $k \in \mc{K}$, let $\mc{P}^k$ be the set of directed simple paths in $G$ from $s_k$ to $t_k$, and let $d_k > 0$ be the demand associated with commodity $k$. Define $\mc{P}:=\cup_{k \in \mc{K}}\mc{P}^k$ to be the set of paths over all commodities and $\bm{d}:=(d_k)_{k \in \mc{K}}$ to be the vector of the demands. Each edge $e \in E$ is given a load-dependent \emph{latency function} $\ell_e(x)$, assumed to be nonnegative, differentiable, and nondecreasing. Moreover, we assume $x \ell_e(x)$ is convex with respect to $x$.  We shall abbreviate an instance of the problem by the quadruple $\mc{G}=(G(V,E),\mc{T}, 
\{\ell_e\}_{e\in E}, \bm{d})$.

%We consider , namely an infinite set of users that are infinitesimally small, so that an individual user does not affect the delays experienced by other users in the network.

% should we define the flow for each commodity x_e^k?
\smallskip\noindent\textbf{Flows.} Given an instance $\mc{G}$, the collective decisions of users in commodity $k \in \mc{K}$ can be encoded in two ways, as a path flow $\bm{f}^k=(f_{\pi}^k)_{\pi \in \mc{P}}$ and as an edge flow $\bm{x}^k = (x_e^k)_{e\in E}$. These two representations are related as $x_e^k = \sum_{\pi \in \mathcal{P}^k:\pi \owns e} f_{\pi}^k$. We can also consider the collective decisions of users of all commodities together by defining the {\pathdecomp} $\bm{f} = \sum_{k \in \mc{K}} \bm{f}^k$ and the edge flow $\bm{x} = \sum_{k \in \mc{K}} \bm{x}^k$.
There may exist multiple {\pathdecomps} corresponding to an edge flow $\bm{x}$ and we denote the set of such decompositions as $\mc{D}_p(\bm{x})$. Denote the feasible edge flows by $\mathcal{D}_E.$\footnote{
For node $u\in V$, $E_u^+$ denote the set of its outgoing edges and $E_u^-$ denote the set of its incoming edges.  $\mc{D}_E$ is the set of vectors that satisfies the flow conservation equations:
\Scale[0.7]{
\mathcal{D}_E = \left\{\bm{x}: x_e=\sum_{k \in \mc{K}} x_e^k,
\sum_{e\in E_{u}^+} x_e^k -\sum_{e\in E_{u}^-} x_e^k = d_k \left(\mathbbm{1}_u (s_k) - \mathbbm{1}_u (t_k)\right),\forall e\in E,\\
\forall u\in V, \forall k \in \mc{K}\right\}.
}
}
%\edited{We can define the feasible region for path flow and  edge flow as $\mathcal{F} = \{(\bm{x},\bm{f}): \bm{x} \in \mathcal{D}_E, \bm{f} \in \mathcal{D}_P(\bm{x})\}$. } % should remove this?
We can define the feasible region for all possible path flows as $\mc{D}_p = \cup_{\bm{x} \in \mc{D}_E} \mc{D}_p(\bm{x})$.
  
We further differentiate a \emph{positive} path from a \emph{used} path in the following definitions.
 
\begin{definition}[Positive path]
For an edge flow vector $\bm{x}$, we call a path $\pi\in \mathcal{P}$ \emph{positive} for commodity $k \in \mc{K}$ if for all edges $e \in \pi$, $x_{e}^k > 0$.  
For each commodity $k \in \mc{K}$, we can define the set of \emph{positive} paths under edge flow $\bm{x}$ as $\mathcal{P}_{+}^k(\bm{x}) = \left\{p: p \in \mathcal{P}^k, \forall e \in p, x_e^k>0  \right\}$.  Further, the set of all positive paths for all commodities under edge flow $\bm{x}$ can be defined as $\mc{P}_{+}(\bm{x}) = \cup_{k \in \mc{K}} \mc{P}_{+}^k(\bm{x})$.
%Call the set of \emph{positive} paths under $\bm{x}$, $\mathcal{P}_{+}= \left\{p: p \in \mathcal{P}, \forall e \in p, x_e>0  \right\}$. For any commodity $k$ the positive paths are $\mc{P}^k_{+} =\mc{P}_{+} \cap \mc{P}^k$.
\end{definition}
     
\begin{definition}[Used path]
For a path flow $\bm{f}$, we call a path $\pi\in \mathcal{P}$ \emph{used} by commodity $k \in \mc{K}$ if $f_{\pi}^k > 0$ and \emph{unused} otherwise. For each commodity $k \in \mc{K}$, we can define the set of \emph{used} paths under path flow decomposition $\bm{f}$ as $\mathcal{P}_u^k(\bm{f}) = \{p: p\in \mathcal{P}, f_p^k >0\}$.  Further, the set of all used paths for all commodities under path flow decomposition $\bm{f}$ can be defined as $\mc{P}_u(\bm{f}) = \cup_{k \in \mc{K}} \mc{P}_u^k(\bm{f})$.
\end{definition}

\begin{remark}
Note that a used path is always positive but a positive path may be unused depending on the particular path flow decomposition.
\end{remark}

\subsection{Costs and Equilibria}
\textbf{Costs.} Under a path flow $\bm{f} \in \mc{D}_p$, the cost (latency) of a path $\pi$ is defined to be the sum of latencies of edges along the path: 
$\ell_{\pi}(\bm{f}) = \ell_{\pi}(\bm{x})=\sum_{e \in \pi}\ell_e(x_e)$ for $\bm{f} \in \mc{D}_p(\bm{x})$.

\begin{definition}[Social cost and socially optimal flow]
The \emph{social cost} (SC) of a flow $\bm{x} \in \mc{D}_E$ is the total latency in the network under the flow, $SC(\bm{x})=\sum_{e\in E}x_e \ell_{e}(x_e)$. The social cost of a path flow $\bm{f} \in \mc{D}_p$ is $SC(\bm{f})=SC(\bm{x_f})$, where $\bm{x_f}$ is the edge flow induced by $\bm{f}$.  We sometimes refer to the social cost simply as {\em cost}.
A flow with minimum social cost among all feasible flows is called a \emph{socially optimal} flow or simply, a \emph{social optimum}.  The set of socially optimal edge flows is denoted by 
$$SO_E = \{\bm{x} \in \arg\min SC(\bm{x}) \}.$$
Also, we denote the set of socially optimal path flows by
$$ SO_p= \{\bm{f} \in \arg\min SC(\bm{f}) \}.$$
\end{definition}

\smallskip\noindent 
\textbf{Equilibrium.} We assume that users are {\em nonatomic}, namely there are infinitely many users that are infinitesimally small.  As such, a single user controls an infinitesimally small fraction of flow and her routing choice does not unilaterally affect the costs experienced by other users.  This fact is captured by the definition of equilibrium below.

\begin{definition} (Nash Equilibrium)\footnote{The Nash equilibrium in nonatomic routing games is also commonly known as Wardrop equilibrium. }
A path flow $\bm{f}$ is a  \emph{Nash Equilibrium} if for any commodity $k\in \mathcal{K}$ and any used path $p\in \mathcal{P}_{u}^k(\bm{f})$ we have $\ell_p(\bm{f}) \leq  \ell_q(\bm{f})$, for all paths $q\in \mathcal{P}^k$. 
\end{definition}

Given a Nash equilibrium, we can measure its quality by comparing its cost with the cost of the socially optimal flow.  This idea is often formalized as the \emph{price of anarchy} and the \emph{price of stability} which we define below. Since our scope is to examine user oriented solution concepts other than  the Nash Equilibrium, we generalize the classic definitions of the price of anarchy and stability to apply to an arbitrary set of flows $\mc{F}$. If $\mc{F}$ is the set of Nash equilibria, we get the standard definition for the price of anarchy and price of stability.
%The $\theta$-PNE, $\theta$-UNE and $\theta$-EF flows are not unique and they cover a large range of social cost. Call them concisely as $\theta$-`F', with `F' $\in \{\text{PNE, UNE, EF} \}$,  to reduce notation. 

\begin{definition}[Price of Anarchy and Price of Stability]  Given an instance $\mc{G}$
and a set of (feasible) flows $\mc{F}$, we define the price of anarchy (PoA) as the ratio of the maximum social cost of any flow in $\mc{F}$  to the socially optimal cost. The price of stability (PoS) is the ratio of the minimum social cost of any  flow in $\mc{F}$ to the socially optimal cost.  The PoA and PoS are formally expressed as: 
\begin{flalign}\label{eq:PoA}
%PoA(\theta;\text{`F'})= \max_{\mc{G} } \max \left\{  \ \frac{C(\bm{f})}{C(\bm{x}^*)}: (\bm{x},\bm{f}) \text{ is a } \theta\text{-`F'} \text{in } \mc{G}\right\}.
PoA(\mc{F}) = \max \left\{  \ \frac{SC(\bm{f})}{SC(\bm{x}^*)}: \bm{f} \in \mc{F}, \bm{x}^* \in SO_E \right\}.
\end{flalign}
\begin{flalign}\label{eq:PoS}
%PoS(\theta; \text{`F'})= \max_{\mc{G}} \min \left\{  \ \frac{C(\bm{f})}{C(\bm{x}^*)}: (\bm{x},\bm{f}) \text{ is a } \theta\text{-`F'} \text{in } \mc{G}\right\}.
PoS(\mc{F}) = \min \left\{  \ \frac{SC(\bm{f})}{SC(\bm{x}^*)}: \bm{f} \in \mc{F}, \bm{x}^* \in SO_E \right\}.
\end{flalign}
We may define the PoA and the PoS over sets of instances. For a set of instances, its PoA and PoS equals the maximum  PoA and PoS among the instances in the set, respectively. We will use this definition when examining  instances with latency functions in class $\mc{L}$ (for some $\mc{L}$), and it will be clear from the context.
\end{definition}
 


The PoA and the PoS for the set of Nash equilibria coincide in nonatomic selfish routing instances, since there is only one Nash equilibrium (up to edge costs). In contrast, for the sets that we consider in this work and introduce in the following section (e.g., the set of approximate Nash equilibria)  the PoA and the PoS may get different values.

%\smallskip \noindent \textbf{Fairness.} \thl{[Totally commented out fairness, although initially i though it  was good here. Probably a remark should be put after the definition of $\theta$ flows of how they relate to fairness.]} 
%The fairness of a flow is a fundamental quantity that captures satisfaction among users and it has been studied in the literature under various forms. The unfairness of a flow has been defined as the ratio of the maximum used path in a path flow $\bm{f}$ and the Nash length of the network by Roughgarden in \cite{}. In their paper~\cite{} Correa et al. have used a different definition. They define unfairness to be the ratio between the maximum use' path and the minimum used path in a path flow $\bm{f}$. We use the pessimistic approach of Correa et al. and differentiate unfairness between an edge flow and a path flow. The unfairness in the two cases are with respect to the positive paths and the used paths, in the specific order.

%\thl{[Do we need/use these definitions? Maybe  (in the next section) just say that we may refer to all ($\theta$-PNE/UNE/EF) as $\theta$ fair flows. Does this suffice?]}  

%\begin{definition}[{\EFU}]
%Given an edge flow $\bm{x}$, the unfairness of the edge flow is given as the ratio between the maximum positive path and the minimum positive path under $\bm{x}$. We call this  {\efu}, $ U_E(\bm{x}) = \max_k\max\left\{\frac{\ell_p}{\ell_q}: p, q  \in \mc{P}_{+}^k(\bm{x})\right\}$.
%\end{definition}

%\begin{definition}[{\PFU}]
%Given a path flow $\bm{f}$, the unfairness is given as the ratio between the maximum used path and the minimum used path under $\bm{f}$. We call this  {\pfu}, $ U_P(\bm{f}) = \max_k\max\left\{\frac{\ell_p}{\ell_q}: p, q \in \mc{P}_{u}^k(\bm{f})\right\}$.
%\end{definition}

%We note that given $\bm{f}$ we can compute the corresponding edge flow $\bm{x}$ and compute its {\efu}. Similarly, given $\bm{x}$ we can construct a specific path decomposition and compute {\pfu}.
 



