\section{Solution Concepts}\label{sec:satisfaction} 
Here we give the formal definition of the solution concepts we introduced in Section~\ref{sec:intro}.  We also provide an example to illustrate their differences, and prove that each solution concept may correspond to a non-convex set of flows. %We begin with the relative satisfaction among users where we compare the latency of different paths under a given path flow.   

\begin{definition}[$\theta$-PNE]\label{def:ApproxPNE}
Given a network $\mc{G}$, an edge flow $\bm{x}$ is a $\theta$-Positive Nash Equilibrium ($\theta$-PNE) flow if for any commodity $k\in \mathcal{K}$ and any positive path $p\in \mathcal{P}_{+}^k(\bm{x})$ we have $\ell_p(\bm{x}) \leq \theta \ell_q(\bm{x})$, for all paths $q\in \mathcal{P}^k$. 
%Moreover, call a $\theta$-PNE flow $o$, a $\theta$-Positive Nash Optimal flow ($\theta$-PNO) if it has the minimum social cost among all other $\theta$-PNE. 
We may call a path flow $\bm{f}$  a $\theta$-Positive Nash Equilibrium, if $\bm{f}\in \mc{D}_p(\bm{x})$, for some $\theta$-Positive Nash Equilibrium edge flow $\bm{x}$.
\end{definition}

\begin{definition}[$\theta$-UNE]\label{def:ApproxUNE}
Given a network $\mc{G}$, a {\pathdecomp} $\bm{f}$ is a $\theta$-Used Nash Equilibrium ($\theta$-UNE) flow if for any commodity $k\in \mathcal{K}$ and any used path $p\in \mathcal{P}_{u}^k(\bm{f})$ we have $\ell_p(\bm{f}) \leq \theta \ell_q(\bm{f})$, for all paths $q\in \mathcal{P}^k$. %Moreover, call a $\theta$-UNE flow $o$, a $\theta$-Used Nash Optimal flow ($\theta$-UNO) if it has the minimum social cost among all other $\theta$-UNE.   
\end{definition}

The definition of $\theta$-UNE  corresponds to that of $\theta$-approximate Nash equilibrium used thus far in the literature.
%\noindent\textbf{Nash Length and $1$-PNE:}  
For $\theta=1$, $1$-UNE and $1$-PNE (or simply PNE) coincide, as we show in Lemma~\ref{lemm:UNEvPNE}, and they correspond to the Nash equilibrium, which has been studied extensively. %We simply abbreviated it as PNE.   
%Surprisingly, we will show in Lemma~\ref{lemm:UNEvPNE} that $1$-UNE coincides with $1$-PNE, which we simply abbreviated it as PNE. 
It turns out that every PNE of an instance solves the convex optimization problem  
%\begin{align}\label{eq:1PNE}
%\bm{x}_{\text{PNE}}\in \argmin
$\{ \sum_{e\in E}\int_{0}^{x_e} \ell_e(x)dx : \bm{x}\in \mc{D}_E\}$,
%\end{align}
which as a consequence yields the uniqueness of PNE up to edge costs, i.e. for all $e$ and any two PNE flows, $\bm{x}$, $\bm{x}'$, $\ell_e(x_e)=\ell_e(x'_e)$. This implies for any commodity $k$, all the positive paths have length $L^k_{NE}$ which is called the Nash length of that commodity.   

\begin{definition}[$\theta$-EF]\label{def:ApproxEF}
Given a network $\mc{G}$, a {\pathdecomp} $\bm{f}$ is $\theta$-Envy Free if for any commodity $k\in \mathcal{K}$ and any used path $p\in \mathcal{P}_u^k(\bm{f})$ we have $\ell_p(\bm{f}) \leq \theta \ell_q(\bm{f})$, for all used paths $q\in \mathcal{P}_u^k(\bm{f})$. 
%A $\theta$-EF flow is $\theta$-Envy optimal ($\theta$-EO) flow that minimizes the social cost among all the $\theta$-EF flows.
\end{definition} 

For an instance, we may use $\theta$-PNE, $\theta$-UNE or $\theta$-EF to describe the set of $\theta$-PNE, $\theta$-UNE or $\theta$-EF flows respectively, which will be clear from the context, and we may omit $\theta$ to represent $\theta=1$. Also, by \emph{incentive conditions} we refer to the conditions (inequalities) used to define these flows. Additionally, we may refer to all $\theta$-PNE, $\theta$-UNE and $\theta$-EF flows as $\theta$ fair flows.  The reason for that comes from the fact that their incentive  conditions have inherent the comparison of the maximum used path cost with the minimum (used) path cost, which in some sense describes how (un)fair the flow for players on the maximum cost paths compared to the cost of the lowest cost (used) paths is. Similar notions of (un)fairness have been examined in the past, e.g., by Rougharden \cite{roughgarden2002unfair} and Correa et al. \cite{correa2007fast}. 
 
%In a $\theta$-PNE, $\theta$-UNE or $\theta$-EF flow, the ratio of these costs depends strongly on $\theta$ while

%The fairness of a flow is a fundamental quantity that captures satisfaction among users and it has been studied in the literature under various forms. The unfairness of a flow has been defined as the ratio of the maximum used path in a path flow $\bm{f}$ and the Nash length of the network by Roughgarden in \cite{roughgarden2002unfair}. In their paper~\cite{correa2007fast} Correa et al. have used a different definition. They define unfairness to be the ratio between the maximum use' path and the minimum used path in a path flow $\bm{f}$. We use the pessimistic approach of Correa et al. and differentiate unfairness between an edge flow and a path flow. The unfairness in the two cases are with respect to the positive paths and the used paths, in the specific order.


%\thl{[TALK BRIEFLY ABOUT FAIRNESS HERE???]}
%
%We can mathematically represent a $\theta$-flow with a given relative satisfaction condition as follows. Different $\mc{P}_1$ and $\mc{P}_2$ in Table~\ref{t:flowDef} represents different satisfaction criteria.
%\begin{equation} \label{prog:thetaflows}
%\begin{split}
%\theta\text{-Flow} = \{ \bm{f} | \bm{f} \in \mc{D}_p, &\ell_{\pi_1}(\bm{f}) \le \theta \ell_{\pi_2}(\bm{f}), \forall k, \forall \pi_1 \in \mathcal{P}^k_{1}, \forall \pi_2 \in \mathcal{P}^k_2 \}. 
%\end{split}
%\end{equation} 
%\begin{table}
%\begin{center}
%\begin{tabular}{ |c|c|c| } 
% \hline
% Flow & $\mc{P}^k_1$ & $\mc{P}^k_2$ \\ 
% \hline
% $\theta$-PNE & $\mc{P}^k_{+}$  & $\mc{P}^k$ \\ 
% \hline
% $\theta$-UNE & $\mc{P}^k_{u}$  & $\mc{P}^k$ \\  
% \hline
% $\theta$-EF & $\mc{P}^k_{u}$  & $\mc{P}^k_{u}$ \\
% \hline
%\end{tabular}
%\end{center}
%\caption{Flow Definition}\label{t:flowDef} 
%\end{table}  
%
\begin{figure}
	\centering
	\begin{subfigure}[b]{0.45\linewidth}
		\centering
		\includegraphics[width=0.9\linewidth]{example_flows}
		\caption{Paths $\pi_1$ and $\pi_2$ have $1/2$ unit of flow.  This path flow assignment is a social optimum.}
		\label{fig:ex_flows1}
	\end{subfigure}\hspace{0.05\linewidth}
	\begin{subfigure}[b]{0.45\linewidth}
		\centering
		\includegraphics[width=0.9\linewidth]{example_flow1}
		\caption{The path flow assignment in Figure~\ref{fig:ex_flows1} is $1$-EF but not $1$-UNE.}
		\label{fig:ex_flows01}
	\end{subfigure}
	
	\begin{subfigure}[b]{0.45\linewidth}
		\centering
		\includegraphics[width=0.9\linewidth]{example_flow2}
		\caption{The path flow assignment in Figure~\ref{fig:ex_flows1} is $1.5$-UNE but not $1.5$-PNE.}
		\label{fig:ex_flows2}
	\end{subfigure}\hspace{0.05\linewidth}
	\begin{subfigure}[b]{0.45\linewidth}
		\centering
		\includegraphics[width=0.9\linewidth]{example_flow3}
		\caption{The path flow assignment in Figure~\ref{fig:ex_flows1} is $2$-PNE.}
		\label{fig:ex_flows3}
	\end{subfigure}
	\caption{Example illustrating the three solution concepts $\theta$-UNE, $\theta$-PNE and $\theta$-EF. }
	\label{fig:ex_flows}
\end{figure}
% 

To see how these concepts differ from each other, we give an example in Figure~\ref{fig:ex_flows}.  Suppose the instance is given as shown in Figure~\ref{fig:ex_flows1} and there is a single commodity that routes a unit demand  from $s$ to $t$.  We consider the path flow that routes $1/2$ of the demand through path $\pi_1$ and routes $1/2$ through path $\pi_2$.  It is easy to verify that this is indeed a socially optimal flow.  Next, let us find the appropriate sets that this path flow assignment belongs to with respect to the solution concepts we described above:
\begin{enumerate}
	\item \textbf{$1$-EF:} As shown in Figure~\ref{fig:ex_flows01}, it is easy to verify that this flow is a $1$-EF as each used path has the same length.
	\item \textbf{$1.5$-UNE:} As shown in Figure~\ref{fig:ex_flows2}, any used path has length $1.5$ and the shortest path in the graph has length $1$.  Hence the flow is a $1.5$-UNE as the length of each used path is within a factor $1.5$  of any path.
	\item \textbf{$2$-PNE:} As shown in Figure~\ref{fig:ex_flows3}, for a PNE, we have to take all positive path\evn{s} into account.  Since the longest positive path has length $2$, this flow assignment is not a $1.5$-PNE flow.  Instead, we can see that any positive path is within a factor $2$ of any path.  Hence, this flow is a $2$-PNE.
\end{enumerate}

 
Our goal is to examine the properties of $\theta$ fair flows and provide ways to obtain such flows with good social cost. Regarding the second direction,  in general, the sets of $\theta$-PNE, $\theta$-UNE, and $\theta$-EF flows may not be convex and may contain multiple path flows, which raises the level of difficulty for computing good or optimal such flows. Next we present an example that demonstrates the non-convexity of these sets.



\begin{proposition} There exists a network $\mc{G}$ and $\theta>1$ such that the sets $\theta$-PNE, $\theta$-UNE, and $\theta$-EF are not convex.
\end{proposition}
\begin{proof}
The instance in Figure~\ref{fig:ex_noncvx2} demonstrates that the sets $\theta$-PNE, $\theta$-UNE, and $\theta$-EF are all non convex.  Consider a commodity routing from $s$ to $t$ with unit demand. 
%Then, we are giving two assignment of flows that both satisfy these flow conditions.  
Then, consider the following two flow assignments.
The first one routes all demand along the path $s-u-v-t$.  In this case, path $s-u-v-t$ is the only positive path and has cost equal to $2$.  It is easy to verify that this is a $3/2$-PNE, $3/2$-UNE, and $3/2$-EF.  The second flow routes $2/3$ of the demand along  path $s-u-t$ and routes $1/3$ along path $s-v-t$.  Path $s-u-t$ has cost equal to $1$ and path $s-v-t$ has cost equal to $3/2$.  It is easy to verify that this is a $3/2$-PNE, $3/2$-UNE, and $3/2$-EF as well.  However, if we take the convex combination of these two assignments evenly, then we can find that the path $s-v-t$ has cost equal to $11/6$ and the path $s-u-v-t$ has cost equal to $7/6$, and thus their ratio is greater than $3/2$.  This shows that this combined flow is neither a $3/2$-PNE, a $3/2$-UNE, nor a $3/2$-EF, and hence they are not convex sets.
\end{proof}

\begin{figure}
	\centering
	\includegraphics[width=0.3\linewidth]{examples_noncvx2}
	\caption{Non-convexity of $\theta$-flows}
	\label{fig:ex_noncvx2}
\end{figure}

%The example for the non-convexity of $\theta$-I is demonstrated in Figure~\ref{fig:ex_noncvx1}.  Consider a commodity that routes unit demand from $s$ to $t$.  Note that the Nash length for this graph is $2$, when routing $8/9$ on the path $e1-e2$ and routing $1/9$ on the path $e3-e4$.  Similarly to the example above, we then consider two flow assignments.  The first one is to route half on $e1-e4$ and route half on $e3-e2$.  In this flow both path have length $11/2$.  We can easily verify that this is a $3$-I flow.  The second one is to route $2/3$ on the path $e1-e2$ and route $1/3$ on $e3-e4$.  The path $e1-e2$ has length $2$ and the path $e3-e4$ has length $6$.  However, if we mix them evenly, then the path $e3-e4$ has length $15/2$, and thus the combined flow is not $6$-I.  As a consequence, the set of $6$-I flows in this example is not a convex set.

