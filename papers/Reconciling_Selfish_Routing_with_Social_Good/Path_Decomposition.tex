% v2-acmsmall-sample.tex, dated March 6 2012
% This is a sample file for ACM small trim journals
%
% Compilation using 'acmsmall.cls' - version 1.3 (March 2012), Aptara Inc.
% (c) 2010 Association for Computing Machinery (ACM)
%
% Questions/Suggestions/Feedback should be addressed to => "acmtexsupport@aptaracorp.com".
% Users can also go through the FAQs available on the journal's submission webpage.
%
% Steps to compile: latex, bibtex, latex latex
%
% For tracking purposes => this is v1.3 - March 2012

% Conference specific section


\documentclass[11pt]{article}

%package for authors
\usepackage{authblk}


\title{Reconciling Selfish Routing with Social Good}
%\author{Soumya Basu\, Ger Yang\, Thanasis Lianeas\, Evdokia Nikolova  and  Yitao Chen\\
%Department of ECE, The University of Texas at Austin}

%\author[1]{ Soumya Basu \and Ger Yang \and Thanasis Lianeas \and Evdokia Nikolova  \and  Yitao Chen}

\author[]{Soumya Basu}
\author[]{Ger Yang}
\author[]{Thanasis Lianeas}
\author[]{Evdokia Nikolova}
\author[]{Yitao Chen}
\affil[]{Department of Electrical and Computer Engineering, \\ The University of Texas at Austin}

%  change " , and "  to  " and "
\renewcommand\Authands{ and }

\date{\vspace{-5ex}}
 
 


% Package to generate and customize Algorithm as per ACM style
\usepackage{algorithm}
\usepackage{algorithmic}
\usepackage{amsmath,amssymb, amsthm}
\usepackage{color}
\usepackage{graphicx}
\usepackage{subcaption}
\usepackage{epstopdf}
\usepackage{bbm}
\usepackage{bm}
\usepackage[makeroom]{cancel}
\usepackage[margin=1in]{geometry} 
\usepackage[titletoc,title]{appendix}
% bibiliography command


% Custom Commands

\newcommand{\edited}[1]{\textcolor{red}{#1}}
\newcommand{\mc}[1]{\mathcal{#1}}
\newcommand{\inctv}{Incentive }
\newcommand{\aIC}{I }
\newcommand{\pathdecomp}{path flow}
\newcommand{\pathdecomps}{path flows}
\newcommand{\approxMinMax}{Budgeted Min-Max}
\newcommand*{\Scale}[2][4]{\scalebox{#1}{$#2$}}%
\newcommand{\argmin}{\mathrm{argmin}}
\newcommand{\argmax}{\mathrm{argmax}}
\newcommand{\efu}{edge flow unfairness}
\newcommand{\EFU}{Edge Flow Unfairness}
\newcommand{\pfu}{path flow unfairness}
\newcommand{\PFU}{Path Flow Unfairness}

% Graphics path
\graphicspath {{fig/}}

% Theorems 
\theoremstyle{definition}
\newtheorem{theorem}{Theorem}
\theoremstyle{definition}
\newtheorem{lemma}{Lemma}
\theoremstyle{definition}
\newtheorem{proposition}{Proposition}
\theoremstyle{definition}
\newtheorem{claim}{Claim}
\theoremstyle{definition}
\newtheorem{corollary}{Corollary}
\theoremstyle{definition}
\newtheorem{definition}{Definition}
\theoremstyle{remark}
\newtheorem*{example}{Example}
\theoremstyle{remark}
\newtheorem*{remark}{Remark}




\usepackage{soul,xcolor}

\definecolor{purple}{rgb}{0.7, 0, 0.6}
\definecolor{dgreen}{rgb}{0, 0.8, 0}
\newcommand{\tgy}[1]{\textcolor{black}{#1}}
\newcommand{\smb}[1]{\textcolor{black}{#1}}
\newcommand{\evn}[1]{\textcolor{black}{#1}}
\newcommand{\thl}[1]{\textcolor{black}{#1}}
\newcommand{\fin}[1]{\textcolor{green}{#1}}




% Document starts
\begin{document}



\maketitle
\begin{abstract}
  In this paper, we explore the connection between secret key agreement and secure omniscience within the setting of the multiterminal source model with a wiretapper who has side information. While the secret key agreement problem considers the generation of a maximum-rate secret key through public discussion, the secure omniscience problem is concerned with communication protocols for omniscience that minimize the rate of information leakage to the wiretapper. The starting point of our work is a lower bound on the minimum leakage rate for omniscience, $\rl$, in terms of the wiretap secret key capacity, $\wskc$. Our interest is in identifying broad classes of sources for which this lower bound is met with equality, in which case we say that there is a duality between secure omniscience and secret key agreement. We show that this duality holds in the case of certain finite linear source (FLS) models, such as two-terminal FLS models and pairwise independent network models on trees with a linear wiretapper. Duality also holds for any FLS model in which $\wskc$ is achieved by a perfect linear secret key agreement scheme. We conjecture that the duality in fact holds unconditionally for any FLS model. On the negative side, we give an example of a (non-FLS) source model for which duality does not hold if we limit ourselves to communication-for-omniscience protocols with at most two (interactive) communications.  We also address the secure function computation problem and explore the connection between the minimum leakage rate for computing a function and the wiretap secret key capacity.
  
%   Finally, we demonstrate the usefulness of our lower bound on $\rl$ by using it to derive equivalent conditions for the positivity of $\wskc$ in the multiterminal model. This extends a recent result of Gohari, G\"{u}nl\"{u} and Kramer (2020) obtained for the two-user setting.
  
   
%   In this paper, we study the problem of secret key generation through an omniscience achieving communication that minimizes the 
%   leakage rate to a wiretapper who has side information in the setting of multiterminal source model.  We explore this problem by deriving a lower bound on the wiretap secret key capacity $\wskc$ in terms of the minimum leakage rate for omniscience, $\rl$. 
%   %The former quantity is defined to be the maximum secret key rate achievable, and the latter one is defined as the minimum possible leakage rate about the source through an omniscience scheme to a wiretapper. 
%   The main focus of our work is the characterization of the sources for which the lower bound holds with equality \textemdash it is referred to as a duality between secure omniscience and wiretap secret key agreement. For general source models, we show that duality need not hold if we limit to the communication protocols with at most two (interactive) communications. In the case when there is no restriction on the number of communications, whether the duality holds or not is still unknown. However, we resolve this question affirmatively for two-user finite linear sources (FLS) and pairwise independent networks (PIN) defined on trees, a subclass of FLS. Moreover, for these sources, we give a single-letter expression for $\wskc$. Furthermore, in the direction of proving the conjecture that duality holds for all FLS, we show that if $\wskc$ is achieved by a \emph{perfect} secret key agreement scheme for FLS then the duality must hold. All these results mount up the evidence in favor of the conjecture on FLS. Moreover, we demonstrate the usefulness of our lower bound on $\wskc$ in terms of $\rl$ by deriving some equivalent conditions on the positivity of secret key capacity for multiterminal source model. Our result indeed extends the work of Gohari, G\"{u}nl\"{u} and Kramer in two-user case.
%\begin{abstract}
%\begin{quote}
%We study the problem of giving a polynomial time computable path flow decomposition which induces low social cost and users feel the routing plan is ``fair''. There are two definitions for ``fair'', the first is that the length experienced by each user is within some constant times the length of the shortest path and the second is the length experienced by each user is bounded by a fixed length. Writing the problem into convex program according to these two definitions of ``fair'', that of the first definition of fair turns out to find a path decomposition of an approximate Nash Equilibrium with best social cost and that of the second definition of fair we call it length bounded minimum social cost problem. We proved both problems are NP-hard. We give a constant factor approximation algorithm for the former and an FPTAS for the latter. Finally, our social cost analysis for both cases show our approximation schemes is better than simply use the exact Nash Equilibrium to approximate the problem.
%\end{quote}
\end{abstract}


\section{Introduction}\label{sec:intro}

\subsection{Two sides of the coin: Social Welfare vs Selfishness} \label{subSect:SOFandFairF}
A fundamental problem arising in the management of road-traffic and communication networks is routing traffic to optimize network performance. In the setting of road-traffic networks the average delay incurred by a unit of flow quantifies the cost of a routing assignment.  From a collective perspective minimizing the average cost translates to maximizing the welfare obtained by society.  
%With complete knowledge of the network, we can efficiently describe any socially optimal (SO) flow, namely a flow that minimizes the average social cost, by quantifying the flow passing through each edge \cite{}. 
Starting from the seminal works of Wardrop \cite{wardrop1952some} and Beckman et al. \cite{beckmann1956studies}, the literature on network games has differentiated between 1) the objective of a central planner to minimize average cost and thus find a socially optimal (SO) flow, and 2) the selfish objectives of users minimizing their respective costs.  In the latter case,  
the network users acting in their own interest are assumed to converge to a Nash Equilibrium (NE) flow as further rerouting fails to improve their own objective.  

The tension between the central planner and individual objectives has been an object of intense study in the 
algorithmic game theory literature on congestion games. A central question arising in congestion games, ``how much does network performance suffer from selfish behavior?'', has been investigated extensively through the notions of Price of Anarchy (PoA) and Price of Stability (PoS), namely the ratio of the maximum cost among all Nash equilibria over the social optimum and the ratio of the minimum cost among all  Nash equilibria over the social optimum, respectively. Prior research shows that the Nash equilibrium flow may attain very poor social welfare compared to social optimum, i.e. we may get only poor bounds on the PoA or the PoS, with these bounds being tight for some classes of instances.  For an overview we refer the reader to the survey~\cite{roughgarden2002selfish}. %We note here that the PoA and the PoS coincide in nonatomic selfish routing instances (the case where users  are infinite yet infinitesimally small), but if we define the PoA and the PoS for other possible outcomes such as approximate Nash equilibria or the ones that will appear in the next paragraphs, then the PoA and the PoS may get different values.

This discrepancy between selfishness and social good calls for finding a middle ground between the two ends of the spectrum---the Nash flow and the socially optimal flow.  
To that end, previous research on mechanism design
has lead to theoretically appealing solutions such as 
toll placement and Stackelberg routing~\cite{swamy2012effectiveness}. Placing tolls on edges has been shown to improve the network performance up to the point of completely optimizing it when there are no restrictions on the tolls' values. Using Stackelberg strategies, where one assumes that a fraction of users is willing to cooperate and follow the routes dictated by the central planner, has also been theoretically shown to improve the network performance. In spite of the nice properties of these solutions that induce selfish users to act in a socially friendly way, these mechanisms have faced criticism in the real world in terms of their implementation and their fairness towards various users. 
 
To mitigate the tension between selfishness and social good in a way that is more fair to the users, we set out to explore the properties of alternative solution concepts 
where users under some reasonable incentive condition 
adopt a ``socially desirable" routing of traffic in between the Nash equilibrium (which has high social cost) and the social optimum (which may be undesirable/unfair to users on the longer paths) \cite{roughgarden2002unfair}.  
%Consider the scenario where some routing application gives route choices to the users instead of showing users the congestion state of the network and let users choose a route by themselves. Users may be incentivized to use this application under the condition that they are not forced to take ``relatively long routes''. Apparently, this scenario is close to reality, since  users are uninformed about the network's congestion and more often use their routing devices to travel to their destinations. 
The advent of routing applications and the growing dependence of users on these applications 
places us at an epoch when such new ideas in mechanism design may be more relevant and also more readily integrated to practice. 
Consider the scenario where some routing application presents the uninformed users with routes alongside the guarantees of ``relative fairness'' and ``reasonable delay'' and the users adopt the paths. This scenario is close to reality, since users unaware of the network congestion often use their routing devices to travel to their destinations or pick a path that has been presented to them before. 

This naturally brings forth the questions of whether there exist solutions (flows) where good social welfare is achieved under an appropriate incentive condition for the users %(e.g. guarantees for ``reasonable delay"), 
and if such solutions can be efficiently computed. 
%A crucial fact  is that users, when asking  routing applications, request for paths to their destinations and  a returned solution can be assumed to be a path flow decomposition  that provides guarantee only for the entire paths' costs. This differentiates from the standard Nash Equilibrium approach where local conditions for subpaths  should also be present. 
An example of such a solution could be enforcing a $\theta$-approximate Nash equilibrium of low social cost,  where users are guaranteed to get assigned a path of cost no greater than $\theta$ times the cost of the shortest path and as such, the solution is ``relatively fair".\evn{\footnote{We note that the concept of fairness has been considered in the literature of routing games in more than one ways. The two main approaches define fairness as: 1) the ratio of the maximum path delay in a given flow to the average delay under Nash equilibrium~\cite{roughgarden2002unfair} and 2) the ratio of the maximum path delay to the minimum path delay in a given flow~\cite{correa2007fast}.}} 
Yet, other solution concepts seem to arise naturally and are introduced below.%\footnotemark 

 
%A candidate solution concept is the $\theta$-approximate Nash equilibrium ($\theta$-NE) which admits diverse flows and specially flows with low social cost compared to Nash equilibrium or equivalently $1$-NE~\cite{}. Inducing a $\theta$-NE readily guarantees users ``relative fairness'' as by definition of $\theta$-NE the assigned path is no more than $\theta$ times the shortest path. Additionally, as the average social cost is low the users also have guarantees of ``reasonable delay''. Building on this idea we proceed to further generalize the Nash equilibrium concepts while keeping the above ``guarantees'' preserved.


% DO WE NEED THIS FOOTNOTE?
%\footnotetext{In the remaining part of the Section~\ref{sec:intro} for the ease of presentation we will contain ourselves to networks with a single commodity. All of the concepts discussed will be extended to the multi-commodity networks later.
%}       

\subsection{Selfishness and Envy}
In our quest to achieve the coveted middle ground between the social optimum and Nash equilibrium, by combining good social welfare with satisfied users, we present two notions related to: 1) Selfishness and 2) Envy. 
%We elaborate on these notions in an example single commodity network to convey the motivation and meaning to the reader. The ideas will be formalized in Section~\ref{sec:prelim} where we generalize them to the multi-commodity case. 

Firstly, we consider selfishness where users tend to selfishly improve their own cost whenever there exists some scope of improvement.  This conforms to the notion of Nash Equilibrium and a slight relaxation of  absolute selfishness leads us to the approximate Nash Equilibrium concept.  Specifically, we consider a multiplicative approximation consistent with the tradition in approximation algorithms: we refer to a $\theta$-Nash equilibrium flow as a flow in which the length of any {\em used} path in the network is less than or equal to $\theta$ times the length of any other path in the network, with $\theta \geq1$.   Note that for $\theta=1$ we have the Nash Equilibrium flow.

The existing literature in congestion games mainly regards a used path as a path that has positive flow in all of its edges, independent of the path flow decomposition that induces the edge flow. Here we make the distinction between \emph{positive} paths, i.e. paths that have positive flow in all of their edges, (note, this is independent of the path flow decomposition) %under the edge flow decomposition, 
 and \emph{used} paths, i.e. paths that appear in the path flow decomposition with positive flow. With these definitions we define a $\theta$-Positive Nash Equilibrium ($\theta$-PNE)\footnote{We remark that in the literature, PNE is typically used for abbreviating Pure Nash Equilibrium.  In this paper, we always use it to mean Positive Nash Equilibrium as we define it here.} 
 to be a flow in which the length of any {\em positive} path in the network is less than or equal to $\theta$ times the length of any other path, and  a $\theta$-Used Nash Equilibrium ($\theta$-UNE) to be a flow in which the length of any {\em used} path in the network is less than or equal to $\theta$ times the length of any other path. 
Specifically, the concept of UNE deals directly with the paths assigned to users whereas PNE deals with positive paths which may remain unused.  
 As we shall see, the set of $\theta$-PNE flows is a subset of the set of $\theta$-UNE flows and this inclusion might be strict, though for $\theta=1$ these sets coincide. The definition of $\theta$-approximate Nash equilibrium in the literature~\cite{christodoulou2011performance} corresponds to that of $\theta$-UNE. However, to the best of our knowledge, the significance of path flows in the definition of $\theta$-UNE has not been made explicit in any prior work.
 
Next, consider the notion of envy where for the same source and destination a user experiences envy against another user if the latter incurs smaller delay compared to the former under a given path flow. Similarly to the approximate Nash equilibrium flow we can consider a notion of approximately envy free flows where in a $\theta$-Envy Free ($\theta$-EF) flow, the ratio of any two {\em used} paths in the network is upper bounded by $\theta$, for some $\theta\geq 1$.  Note, the difference from the $\theta$-UNE definition is that a used path's cost is compared only to other used paths' costs. 
Envy free flows arise naturally as we consider the routing applications setup where users only collect information about the routes provided by the application. Thus, on the one hand, the possible costs for the current users in some sense compare to the costs of the users that have already used the network. On the other hand, routes for which there is no (sufficient) information potentially may never appear as an option. In other words, routes that have not been chosen in the past (sufficiently many times), i.e. ``unused routes'', do not arise in the comparison of the paths' costs. 

An example of how the concepts of $\theta$-PNE, $\theta$-UNE, and $\theta$-EF may differ from each other is illustrated in Figure~\ref{fig:ex_flows}. There, the optimal edge flow of the network is a $2$-PNE due to the presence of `positive' paths of length $2$ and $1$. However, considering path flows there exists a $1$-EF flow that induces  the optimal edge flow.  Also, the example has a path flow that is a $1.5$-UNE but it does not admit a path flow that is a $1$-UNE. More details  are discussed in Section~\ref{sec:satisfaction}, where these notions are formally introduced. 



% RELATED WORK
\section{Related Work}\label{sec:related}
 
The authors in \cite{humphreys2007noncontact} showed that it is possible to extract the PPG signal from the video using a complementary metal-oxide semiconductor camera by illuminating a region of tissue using through external light-emitting diodes at dual-wavelength (760nm and 880nm).  Further, the authors of  \cite{verkruysse2008remote} demonstrated that the PPG signal can be estimated by just using ambient light as a source of illumination along with a simple digital camera.  Further in \cite{poh2011advancements}, the PPG waveform was estimated from the videos recorded using a low-cost webcam. The red, green, and blue channels of the images were decomposed into independent sources using independent component analysis. One of the independent sources was selected to estimate PPG and further calculate HR, and HRV. All these works showed the possibility of extracting PPG signals from the videos and proved the similarity of this signal with the one obtained using a contact device. Further, the authors in \cite{10.1109/CVPR.2013.440} showed that heart rate can be extracted from features from the head as well by capturing the subtle head movements that happen due to blood flow.

%
The authors of \cite{kumar2015distanceppg} proposed a methodology that overcomes a challenge in extracting PPG for people with darker skin tones. The challenge due to slight movement and low lighting conditions during recording a video was also addressed. They implemented the method where PPG signal is extracted from different regions of the face and signal from each region is combined using their weighted average making weights different for different people depending on their skin color. 
%

There are other attempts where authors of \cite{6523142,6909939, 7410772, 7412627} have introduced different methodologies to make algorithms for estimating pulse rate robust to illumination variation and motion of the subjects. The paper \cite{6523142} introduces a chrominance-based method to reduce the effect of motion in estimating pulse rate. The authors of \cite{6909939} used a technique in which face tracking and normalized least square adaptive filtering is used to counter the effects of variations due to illumination and subject movement. 
The paper \cite{7410772} resolves the issue of subject movement by choosing the rectangular ROI's on the face relative to the facial landmarks and facial landmarks are tracked in the video using pose-free facial landmark fitting tracker discussed in \cite{yu2016face} followed by the removal of noise due to illumination to extract noise-free PPG signal for estimating pulse rate. 

Recently, the use of machine learning in the prediction of health parameters have gained attention. The paper \cite{osman2015supervised} used a supervised learning methodology to predict the pulse rate from the videos taken from any off-the-shelf camera. Their model showed the possibility of using machine learning methods to estimate the pulse rate. However, our method outperforms their results when the root mean squared error of the predicted pulse rate is compared. The authors in \cite{hsu2017deep} proposed a deep learning methodology to predict the pulse rate from the facial videos. The researchers trained a convolutional neural network (CNN) on the images generated using Short-Time Fourier Transform (STFT) applied on the R, G, \& B channels from the facial region of interests.
The authors of \cite{osman2015supervised, hsu2017deep} only predicted pulse rate, and we extended our work in predicting variance in the pulse rate measurements as well.

All the related work discussed above utilizes filtering and digital signal processing to extract PPG signals from the video which is further used to estimate the PR and PRV.  %
The method proposed in \cite{kumar2015distanceppg} is person dependent since the weights will be different for people with different skin tone. In contrast, we propose a deep learning model to predict the PR which is independent of the person who is being trained. Thus, the model would work even if there is no prior training model built for that individual and hence, making our model robust. 

%


% CONTRIBUTION
\subsection{Contribution}
The recent influx of technology in traffic routing, the scale of traffic networks and globalization bring about a definite shift in the well studied routing games. The incomplete knowledge of users creates a dependence on routing technologies, giving more freedom to a central planner to mitigate the inefficiency originating from the selfish routing of users in the full information setting. In this work, we show that the path flows in the network may play a key role in achieving the full potential of such route planning mechanisms.  In particular, we clearly differentiate path flows from edge flows through the introduction of `positive' paths and `used' paths. Recall, a `positive' path is a path with all edges carrying nonzero flows under a given edge flow. Whereas a `used' path is a path with nonzero flow under a specific path flow. From the inherent differences of `positive' and `used' paths,  two new concepts, used Nash equilibrium (UNE) and envy free (EF) flow, naturally emerge as generalizations of the Wardrop equilibrium. We call the classical Wardrop equilibrium positive Nash equilibrium (PNE) because it essentially deals with `positive' paths. To the best of our knowledge, this distinction between positive and used paths has not been made explicit despite the rich literature developed on this topic for over half a decade. We also define the respective approximate versions of all the three solution concepts, i.e. $\theta$-PNE, $\theta$-UNE and $\theta$-EF for $\theta> 1$, where the distinction plays a critical role. 


With the introduction of these three related concepts and their approximate versions, the first step in understanding them is to compare the flows against each other. We show that the $1$-UNE and the $1$-PNE are indeed identical and this helps in understanding why the `used' and the `positive' paths have not been explicitly differentiated before this work. But beyond this case the new concepts impose a hierarchical structure on the space of feasible flows. Specifically, we notice that $\theta$-PNE, $\theta$-UNE and $\theta$-EF flows are progressively larger sets, each containing the previous one, with promise of better tradeoff between the social welfare and fairness. In order to grasp the large separation between these concepts note that for some networks the $\theta$-UNE  is not contained in $\Omega(n\theta)$-PNE, where $n$ is the number of nodes in the network (Lemma~\ref{lemm:UNEvPNE} in Section~\ref{sec:hier}). 

Motivated from the classical study of the price of anarchy (PoA) of equilibrium flows we investigate the PoA of $\theta$-UNE and $\theta$-EF. In general we expect that as we move from the $\theta$-PNE to $\theta$-EF flows from a worst case perspective we will encounter flows with larger social cost. As a worst case example we show that the PoA can be unbounded for $1$-EF flows. However, we see that under the well used framework of variational inequality based PoA upper bounds~\cite{roughgarden2004bounding} both $\theta$-PNE and $\theta$-UNE admit the same bound on the PoA (Lemma~\ref{lemm:UNEvVI} in Section~\ref{sec:social}). Through a similar reasoning we show that the price of stability is non increasing from $\theta$-PNE to $\theta$-EF flows. 

%The implication of `used' and `positive' paths in a network brings about finer details in the \evn{definition of fairness}. We define edge flow fairness, which compares the `positive' paths in a network, and path flow fairness, which compares the `used' paths in the network. The importance of this differentiation becomes clear as one realizes that a path flow having a large `positive' path yet balanced `used' \evn{paths} should not be deemed unfair. 

Focusing on cost-efficient and fair flow design, the question of computing a $\theta$-PNE, a $\theta$-UNE or a $\theta$-EF flow with low social cost becomes one of the fundamental questions. We experience a temporary setback as the traditional convex optimization framework for computing the equilibrium and socially optimal flows fails here due to the non-convexity of the sets of $\theta$-PNE, $\theta$-UNE and $\theta$-EF flows for $\theta>1$. Formally, we prove (Theorem~\ref{thm:main_hardness} in Section~\ref{sec:complex}) that obtaining the best $\theta$-UNE or the best $\theta$-EF flow is NP-hard. Indeed given a socially optimal flow it is NP-hard to decide whether it admits a path flow decomposition which is $\theta$-UNE ($\theta$-EF). In a positive direction we show (Lemma~\ref{lemma:3Easy} in Section~\ref{sec:complex}) that for any `acylic' flow we can decide whether it is a $\theta$-PNE or not. As any `cyclic' flow can be made `acyclic' without increasing its social cost, the above result is sufficient for our design goal, i.e. balance social cost and fairness. However, we leave open the question of finding the best $\theta$-PNE flow ($\theta>1$). 

We further discuss how, at a conceptual level, the new ideas could be integrated with routing technologies (in Section~\ref{sec:design}). Drawing elements from different but related areas, we observe that minimization of modified latency functions can facilitate the calculation of a $\theta$-PNE flow with social cost guarantees. In particular, we use two techniques for bounding the social cost: 1) modified potential functions and 2) bounded tolls.  As a side note, following the ideas presented by Christodoulou et al.~\cite{christodoulou2011performance}, we explicitly articulate a technique to upper bound the price of stability for general functions and use it to extend the analysis of PoS for M/M/1 delay functions (Lemma~\ref{lemm:MM1} in Section~\ref{sec:design}).

In another direction, we deviate from the norm of deterministic flow design, and formalize the concept of randomization in flow design. We present (Theorem~\ref{thm:RR} in Section~\ref{sec:design}) an expression for the mean and a bound for the standard deviation of a path `used' by a typical user under this strategy.  The newly introduced concepts of $\theta$-UNE and $\theta$-EF flows play a crucial role in the variance reduction of this strategy.  The introduction of randomized routing in flow design may be of independent interest and we believe it can play an important role in emerging routing technologies. 




% Head 1
\section{Preliminaries}\label{sec:prelim}
\subsection{Network and Flows}
\textbf{Network.} We are given a directed graph $G(V,E)$ with vertex set $V$, edge set $E$, and a set of commodities % $ \mc{K}=\{(s_k, t_k)\}$. 
$\mc{K}=\{1,2,\dots,K\}$.  Each commodity $k \in \mc{K}$ is associated with a source $s_k$ and a sink $t_k$.  We denote $\mc{T} = \{(s_k,t_k)\}_{k \in \mc{K}}$ as the collection of the source-sink pairs for all commodities.
Also, for each commodity $k \in \mc{K}$, let $\mc{P}^k$ be the set of directed simple paths in $G$ from $s_k$ to $t_k$, and let $d_k > 0$ be the demand associated with commodity $k$. Define $\mc{P}:=\cup_{k \in \mc{K}}\mc{P}^k$ to be the set of paths over all commodities and $\bm{d}:=(d_k)_{k \in \mc{K}}$ to be the vector of the demands. Each edge $e \in E$ is given a load-dependent \emph{latency function} $\ell_e(x)$, assumed to be nonnegative, differentiable, and nondecreasing. Moreover, we assume $x \ell_e(x)$ is convex with respect to $x$.  We shall abbreviate an instance of the problem by the quadruple $\mc{G}=(G(V,E),\mc{T}, 
\{\ell_e\}_{e\in E}, \bm{d})$.

%We consider , namely an infinite set of users that are infinitesimally small, so that an individual user does not affect the delays experienced by other users in the network.

% should we define the flow for each commodity x_e^k?
\smallskip\noindent\textbf{Flows.} Given an instance $\mc{G}$, the collective decisions of users in commodity $k \in \mc{K}$ can be encoded in two ways, as a path flow $\bm{f}^k=(f_{\pi}^k)_{\pi \in \mc{P}}$ and as an edge flow $\bm{x}^k = (x_e^k)_{e\in E}$. These two representations are related as $x_e^k = \sum_{\pi \in \mathcal{P}^k:\pi \owns e} f_{\pi}^k$. We can also consider the collective decisions of users of all commodities together by defining the {\pathdecomp} $\bm{f} = \sum_{k \in \mc{K}} \bm{f}^k$ and the edge flow $\bm{x} = \sum_{k \in \mc{K}} \bm{x}^k$.
There may exist multiple {\pathdecomps} corresponding to an edge flow $\bm{x}$ and we denote the set of such decompositions as $\mc{D}_p(\bm{x})$. Denote the feasible edge flows by $\mathcal{D}_E.$\footnote{
For node $u\in V$, $E_u^+$ denote the set of its outgoing edges and $E_u^-$ denote the set of its incoming edges.  $\mc{D}_E$ is the set of vectors that satisfies the flow conservation equations:
\Scale[0.7]{
\mathcal{D}_E = \left\{\bm{x}: x_e=\sum_{k \in \mc{K}} x_e^k,
\sum_{e\in E_{u}^+} x_e^k -\sum_{e\in E_{u}^-} x_e^k = d_k \left(\mathbbm{1}_u (s_k) - \mathbbm{1}_u (t_k)\right),\forall e\in E,\\
\forall u\in V, \forall k \in \mc{K}\right\}.
}
}
%\edited{We can define the feasible region for path flow and  edge flow as $\mathcal{F} = \{(\bm{x},\bm{f}): \bm{x} \in \mathcal{D}_E, \bm{f} \in \mathcal{D}_P(\bm{x})\}$. } % should remove this?
We can define the feasible region for all possible path flows as $\mc{D}_p = \cup_{\bm{x} \in \mc{D}_E} \mc{D}_p(\bm{x})$.
  
We further differentiate a \emph{positive} path from a \emph{used} path in the following definitions.
 
\begin{definition}[Positive path]
For an edge flow vector $\bm{x}$, we call a path $\pi\in \mathcal{P}$ \emph{positive} for commodity $k \in \mc{K}$ if for all edges $e \in \pi$, $x_{e}^k > 0$.  
For each commodity $k \in \mc{K}$, we can define the set of \emph{positive} paths under edge flow $\bm{x}$ as $\mathcal{P}_{+}^k(\bm{x}) = \left\{p: p \in \mathcal{P}^k, \forall e \in p, x_e^k>0  \right\}$.  Further, the set of all positive paths for all commodities under edge flow $\bm{x}$ can be defined as $\mc{P}_{+}(\bm{x}) = \cup_{k \in \mc{K}} \mc{P}_{+}^k(\bm{x})$.
%Call the set of \emph{positive} paths under $\bm{x}$, $\mathcal{P}_{+}= \left\{p: p \in \mathcal{P}, \forall e \in p, x_e>0  \right\}$. For any commodity $k$ the positive paths are $\mc{P}^k_{+} =\mc{P}_{+} \cap \mc{P}^k$.
\end{definition}
     
\begin{definition}[Used path]
For a path flow $\bm{f}$, we call a path $\pi\in \mathcal{P}$ \emph{used} by commodity $k \in \mc{K}$ if $f_{\pi}^k > 0$ and \emph{unused} otherwise. For each commodity $k \in \mc{K}$, we can define the set of \emph{used} paths under path flow decomposition $\bm{f}$ as $\mathcal{P}_u^k(\bm{f}) = \{p: p\in \mathcal{P}, f_p^k >0\}$.  Further, the set of all used paths for all commodities under path flow decomposition $\bm{f}$ can be defined as $\mc{P}_u(\bm{f}) = \cup_{k \in \mc{K}} \mc{P}_u^k(\bm{f})$.
\end{definition}

\begin{remark}
Note that a used path is always positive but a positive path may be unused depending on the particular path flow decomposition.
\end{remark}

\subsection{Costs and Equilibria}
\textbf{Costs.} Under a path flow $\bm{f} \in \mc{D}_p$, the cost (latency) of a path $\pi$ is defined to be the sum of latencies of edges along the path: 
$\ell_{\pi}(\bm{f}) = \ell_{\pi}(\bm{x})=\sum_{e \in \pi}\ell_e(x_e)$ for $\bm{f} \in \mc{D}_p(\bm{x})$.

\begin{definition}[Social cost and socially optimal flow]
The \emph{social cost} (SC) of a flow $\bm{x} \in \mc{D}_E$ is the total latency in the network under the flow, $SC(\bm{x})=\sum_{e\in E}x_e \ell_{e}(x_e)$. The social cost of a path flow $\bm{f} \in \mc{D}_p$ is $SC(\bm{f})=SC(\bm{x_f})$, where $\bm{x_f}$ is the edge flow induced by $\bm{f}$.  We sometimes refer to the social cost simply as {\em cost}.
A flow with minimum social cost among all feasible flows is called a \emph{socially optimal} flow or simply, a \emph{social optimum}.  The set of socially optimal edge flows is denoted by 
$$SO_E = \{\bm{x} \in \arg\min SC(\bm{x}) \}.$$
Also, we denote the set of socially optimal path flows by
$$ SO_p= \{\bm{f} \in \arg\min SC(\bm{f}) \}.$$
\end{definition}

\smallskip\noindent 
\textbf{Equilibrium.} We assume that users are {\em nonatomic}, namely there are infinitely many users that are infinitesimally small.  As such, a single user controls an infinitesimally small fraction of flow and her routing choice does not unilaterally affect the costs experienced by other users.  This fact is captured by the definition of equilibrium below.

\begin{definition} (Nash Equilibrium)\footnote{The Nash equilibrium in nonatomic routing games is also commonly known as Wardrop equilibrium. }
A path flow $\bm{f}$ is a  \emph{Nash Equilibrium} if for any commodity $k\in \mathcal{K}$ and any used path $p\in \mathcal{P}_{u}^k(\bm{f})$ we have $\ell_p(\bm{f}) \leq  \ell_q(\bm{f})$, for all paths $q\in \mathcal{P}^k$. 
\end{definition}

Given a Nash equilibrium, we can measure its quality by comparing its cost with the cost of the socially optimal flow.  This idea is often formalized as the \emph{price of anarchy} and the \emph{price of stability} which we define below. Since our scope is to examine user oriented solution concepts other than  the Nash Equilibrium, we generalize the classic definitions of the price of anarchy and stability to apply to an arbitrary set of flows $\mc{F}$. If $\mc{F}$ is the set of Nash equilibria, we get the standard definition for the price of anarchy and price of stability.
%The $\theta$-PNE, $\theta$-UNE and $\theta$-EF flows are not unique and they cover a large range of social cost. Call them concisely as $\theta$-`F', with `F' $\in \{\text{PNE, UNE, EF} \}$,  to reduce notation. 

\begin{definition}[Price of Anarchy and Price of Stability]  Given an instance $\mc{G}$
and a set of (feasible) flows $\mc{F}$, we define the price of anarchy (PoA) as the ratio of the maximum social cost of any flow in $\mc{F}$  to the socially optimal cost. The price of stability (PoS) is the ratio of the minimum social cost of any  flow in $\mc{F}$ to the socially optimal cost.  The PoA and PoS are formally expressed as: 
\begin{flalign}\label{eq:PoA}
%PoA(\theta;\text{`F'})= \max_{\mc{G} } \max \left\{  \ \frac{C(\bm{f})}{C(\bm{x}^*)}: (\bm{x},\bm{f}) \text{ is a } \theta\text{-`F'} \text{in } \mc{G}\right\}.
PoA(\mc{F}) = \max \left\{  \ \frac{SC(\bm{f})}{SC(\bm{x}^*)}: \bm{f} \in \mc{F}, \bm{x}^* \in SO_E \right\}.
\end{flalign}
\begin{flalign}\label{eq:PoS}
%PoS(\theta; \text{`F'})= \max_{\mc{G}} \min \left\{  \ \frac{C(\bm{f})}{C(\bm{x}^*)}: (\bm{x},\bm{f}) \text{ is a } \theta\text{-`F'} \text{in } \mc{G}\right\}.
PoS(\mc{F}) = \min \left\{  \ \frac{SC(\bm{f})}{SC(\bm{x}^*)}: \bm{f} \in \mc{F}, \bm{x}^* \in SO_E \right\}.
\end{flalign}
We may define the PoA and the PoS over sets of instances. For a set of instances, its PoA and PoS equals the maximum  PoA and PoS among the instances in the set, respectively. We will use this definition when examining  instances with latency functions in class $\mc{L}$ (for some $\mc{L}$), and it will be clear from the context.
\end{definition}
 


The PoA and the PoS for the set of Nash equilibria coincide in nonatomic selfish routing instances, since there is only one Nash equilibrium (up to edge costs). In contrast, for the sets that we consider in this work and introduce in the following section (e.g., the set of approximate Nash equilibria)  the PoA and the PoS may get different values.

%\smallskip \noindent \textbf{Fairness.} \thl{[Totally commented out fairness, although initially i though it  was good here. Probably a remark should be put after the definition of $\theta$ flows of how they relate to fairness.]} 
%The fairness of a flow is a fundamental quantity that captures satisfaction among users and it has been studied in the literature under various forms. The unfairness of a flow has been defined as the ratio of the maximum used path in a path flow $\bm{f}$ and the Nash length of the network by Roughgarden in \cite{}. In their paper~\cite{} Correa et al. have used a different definition. They define unfairness to be the ratio between the maximum use' path and the minimum used path in a path flow $\bm{f}$. We use the pessimistic approach of Correa et al. and differentiate unfairness between an edge flow and a path flow. The unfairness in the two cases are with respect to the positive paths and the used paths, in the specific order.

%\thl{[Do we need/use these definitions? Maybe  (in the next section) just say that we may refer to all ($\theta$-PNE/UNE/EF) as $\theta$ fair flows. Does this suffice?]}  

%\begin{definition}[{\EFU}]
%Given an edge flow $\bm{x}$, the unfairness of the edge flow is given as the ratio between the maximum positive path and the minimum positive path under $\bm{x}$. We call this  {\efu}, $ U_E(\bm{x}) = \max_k\max\left\{\frac{\ell_p}{\ell_q}: p, q  \in \mc{P}_{+}^k(\bm{x})\right\}$.
%\end{definition}

%\begin{definition}[{\PFU}]
%Given a path flow $\bm{f}$, the unfairness is given as the ratio between the maximum used path and the minimum used path under $\bm{f}$. We call this  {\pfu}, $ U_P(\bm{f}) = \max_k\max\left\{\frac{\ell_p}{\ell_q}: p, q \in \mc{P}_{u}^k(\bm{f})\right\}$.
%\end{definition}

%We note that given $\bm{f}$ we can compute the corresponding edge flow $\bm{x}$ and compute its {\efu}. Similarly, given $\bm{x}$ we can construct a specific path decomposition and compute {\pfu}.
 




%%-------------------------------------------------------------------------
\section{Solution Concepts}\label{sec:satisfaction} 
Here we give the formal definition of the solution concepts we introduced in Section~\ref{sec:intro}.  We also provide an example to illustrate their differences, and prove that each solution concept may correspond to a non-convex set of flows. %We begin with the relative satisfaction among users where we compare the latency of different paths under a given path flow.   

\begin{definition}[$\theta$-PNE]\label{def:ApproxPNE}
Given a network $\mc{G}$, an edge flow $\bm{x}$ is a $\theta$-Positive Nash Equilibrium ($\theta$-PNE) flow if for any commodity $k\in \mathcal{K}$ and any positive path $p\in \mathcal{P}_{+}^k(\bm{x})$ we have $\ell_p(\bm{x}) \leq \theta \ell_q(\bm{x})$, for all paths $q\in \mathcal{P}^k$. 
%Moreover, call a $\theta$-PNE flow $o$, a $\theta$-Positive Nash Optimal flow ($\theta$-PNO) if it has the minimum social cost among all other $\theta$-PNE. 
We may call a path flow $\bm{f}$  a $\theta$-Positive Nash Equilibrium, if $\bm{f}\in \mc{D}_p(\bm{x})$, for some $\theta$-Positive Nash Equilibrium edge flow $\bm{x}$.
\end{definition}

\begin{definition}[$\theta$-UNE]\label{def:ApproxUNE}
Given a network $\mc{G}$, a {\pathdecomp} $\bm{f}$ is a $\theta$-Used Nash Equilibrium ($\theta$-UNE) flow if for any commodity $k\in \mathcal{K}$ and any used path $p\in \mathcal{P}_{u}^k(\bm{f})$ we have $\ell_p(\bm{f}) \leq \theta \ell_q(\bm{f})$, for all paths $q\in \mathcal{P}^k$. %Moreover, call a $\theta$-UNE flow $o$, a $\theta$-Used Nash Optimal flow ($\theta$-UNO) if it has the minimum social cost among all other $\theta$-UNE.   
\end{definition}

The definition of $\theta$-UNE  corresponds to that of $\theta$-approximate Nash equilibrium used thus far in the literature.
%\noindent\textbf{Nash Length and $1$-PNE:}  
For $\theta=1$, $1$-UNE and $1$-PNE (or simply PNE) coincide, as we show in Lemma~\ref{lemm:UNEvPNE}, and they correspond to the Nash equilibrium, which has been studied extensively. %We simply abbreviated it as PNE.   
%Surprisingly, we will show in Lemma~\ref{lemm:UNEvPNE} that $1$-UNE coincides with $1$-PNE, which we simply abbreviated it as PNE. 
It turns out that every PNE of an instance solves the convex optimization problem  
%\begin{align}\label{eq:1PNE}
%\bm{x}_{\text{PNE}}\in \argmin
$\{ \sum_{e\in E}\int_{0}^{x_e} \ell_e(x)dx : \bm{x}\in \mc{D}_E\}$,
%\end{align}
which as a consequence yields the uniqueness of PNE up to edge costs, i.e. for all $e$ and any two PNE flows, $\bm{x}$, $\bm{x}'$, $\ell_e(x_e)=\ell_e(x'_e)$. This implies for any commodity $k$, all the positive paths have length $L^k_{NE}$ which is called the Nash length of that commodity.   

\begin{definition}[$\theta$-EF]\label{def:ApproxEF}
Given a network $\mc{G}$, a {\pathdecomp} $\bm{f}$ is $\theta$-Envy Free if for any commodity $k\in \mathcal{K}$ and any used path $p\in \mathcal{P}_u^k(\bm{f})$ we have $\ell_p(\bm{f}) \leq \theta \ell_q(\bm{f})$, for all used paths $q\in \mathcal{P}_u^k(\bm{f})$. 
%A $\theta$-EF flow is $\theta$-Envy optimal ($\theta$-EO) flow that minimizes the social cost among all the $\theta$-EF flows.
\end{definition} 

For an instance, we may use $\theta$-PNE, $\theta$-UNE or $\theta$-EF to describe the set of $\theta$-PNE, $\theta$-UNE or $\theta$-EF flows respectively, which will be clear from the context, and we may omit $\theta$ to represent $\theta=1$. Also, by \emph{incentive conditions} we refer to the conditions (inequalities) used to define these flows. Additionally, we may refer to all $\theta$-PNE, $\theta$-UNE and $\theta$-EF flows as $\theta$ fair flows.  The reason for that comes from the fact that their incentive  conditions have inherent the comparison of the maximum used path cost with the minimum (used) path cost, which in some sense describes how (un)fair the flow for players on the maximum cost paths compared to the cost of the lowest cost (used) paths is. Similar notions of (un)fairness have been examined in the past, e.g., by Rougharden \cite{roughgarden2002unfair} and Correa et al. \cite{correa2007fast}. 
 
%In a $\theta$-PNE, $\theta$-UNE or $\theta$-EF flow, the ratio of these costs depends strongly on $\theta$ while

%The fairness of a flow is a fundamental quantity that captures satisfaction among users and it has been studied in the literature under various forms. The unfairness of a flow has been defined as the ratio of the maximum used path in a path flow $\bm{f}$ and the Nash length of the network by Roughgarden in \cite{roughgarden2002unfair}. In their paper~\cite{correa2007fast} Correa et al. have used a different definition. They define unfairness to be the ratio between the maximum use' path and the minimum used path in a path flow $\bm{f}$. We use the pessimistic approach of Correa et al. and differentiate unfairness between an edge flow and a path flow. The unfairness in the two cases are with respect to the positive paths and the used paths, in the specific order.


%\thl{[TALK BRIEFLY ABOUT FAIRNESS HERE???]}
%
%We can mathematically represent a $\theta$-flow with a given relative satisfaction condition as follows. Different $\mc{P}_1$ and $\mc{P}_2$ in Table~\ref{t:flowDef} represents different satisfaction criteria.
%\begin{equation} \label{prog:thetaflows}
%\begin{split}
%\theta\text{-Flow} = \{ \bm{f} | \bm{f} \in \mc{D}_p, &\ell_{\pi_1}(\bm{f}) \le \theta \ell_{\pi_2}(\bm{f}), \forall k, \forall \pi_1 \in \mathcal{P}^k_{1}, \forall \pi_2 \in \mathcal{P}^k_2 \}. 
%\end{split}
%\end{equation} 
%\begin{table}
%\begin{center}
%\begin{tabular}{ |c|c|c| } 
% \hline
% Flow & $\mc{P}^k_1$ & $\mc{P}^k_2$ \\ 
% \hline
% $\theta$-PNE & $\mc{P}^k_{+}$  & $\mc{P}^k$ \\ 
% \hline
% $\theta$-UNE & $\mc{P}^k_{u}$  & $\mc{P}^k$ \\  
% \hline
% $\theta$-EF & $\mc{P}^k_{u}$  & $\mc{P}^k_{u}$ \\
% \hline
%\end{tabular}
%\end{center}
%\caption{Flow Definition}\label{t:flowDef} 
%\end{table}  
%
\begin{figure}
	\centering
	\begin{subfigure}[b]{0.45\linewidth}
		\centering
		\includegraphics[width=0.9\linewidth]{example_flows}
		\caption{Paths $\pi_1$ and $\pi_2$ have $1/2$ unit of flow.  This path flow assignment is a social optimum.}
		\label{fig:ex_flows1}
	\end{subfigure}\hspace{0.05\linewidth}
	\begin{subfigure}[b]{0.45\linewidth}
		\centering
		\includegraphics[width=0.9\linewidth]{example_flow1}
		\caption{The path flow assignment in Figure~\ref{fig:ex_flows1} is $1$-EF but not $1$-UNE.}
		\label{fig:ex_flows01}
	\end{subfigure}
	
	\begin{subfigure}[b]{0.45\linewidth}
		\centering
		\includegraphics[width=0.9\linewidth]{example_flow2}
		\caption{The path flow assignment in Figure~\ref{fig:ex_flows1} is $1.5$-UNE but not $1.5$-PNE.}
		\label{fig:ex_flows2}
	\end{subfigure}\hspace{0.05\linewidth}
	\begin{subfigure}[b]{0.45\linewidth}
		\centering
		\includegraphics[width=0.9\linewidth]{example_flow3}
		\caption{The path flow assignment in Figure~\ref{fig:ex_flows1} is $2$-PNE.}
		\label{fig:ex_flows3}
	\end{subfigure}
	\caption{Example illustrating the three solution concepts $\theta$-UNE, $\theta$-PNE and $\theta$-EF. }
	\label{fig:ex_flows}
\end{figure}
% 

To see how these concepts differ from each other, we give an example in Figure~\ref{fig:ex_flows}.  Suppose the instance is given as shown in Figure~\ref{fig:ex_flows1} and there is a single commodity that routes a unit demand  from $s$ to $t$.  We consider the path flow that routes $1/2$ of the demand through path $\pi_1$ and routes $1/2$ through path $\pi_2$.  It is easy to verify that this is indeed a socially optimal flow.  Next, let us find the appropriate sets that this path flow assignment belongs to with respect to the solution concepts we described above:
\begin{enumerate}
	\item \textbf{$1$-EF:} As shown in Figure~\ref{fig:ex_flows01}, it is easy to verify that this flow is a $1$-EF as each used path has the same length.
	\item \textbf{$1.5$-UNE:} As shown in Figure~\ref{fig:ex_flows2}, any used path has length $1.5$ and the shortest path in the graph has length $1$.  Hence the flow is a $1.5$-UNE as the length of each used path is within a factor $1.5$  of any path.
	\item \textbf{$2$-PNE:} As shown in Figure~\ref{fig:ex_flows3}, for a PNE, we have to take all positive path\evn{s} into account.  Since the longest positive path has length $2$, this flow assignment is not a $1.5$-PNE flow.  Instead, we can see that any positive path is within a factor $2$ of any path.  Hence, this flow is a $2$-PNE.
\end{enumerate}

 
Our goal is to examine the properties of $\theta$ fair flows and provide ways to obtain such flows with good social cost. Regarding the second direction,  in general, the sets of $\theta$-PNE, $\theta$-UNE, and $\theta$-EF flows may not be convex and may contain multiple path flows, which raises the level of difficulty for computing good or optimal such flows. Next we present an example that demonstrates the non-convexity of these sets.



\begin{proposition} There exists a network $\mc{G}$ and $\theta>1$ such that the sets $\theta$-PNE, $\theta$-UNE, and $\theta$-EF are not convex.
\end{proposition}
\begin{proof}
The instance in Figure~\ref{fig:ex_noncvx2} demonstrates that the sets $\theta$-PNE, $\theta$-UNE, and $\theta$-EF are all non convex.  Consider a commodity routing from $s$ to $t$ with unit demand. 
%Then, we are giving two assignment of flows that both satisfy these flow conditions.  
Then, consider the following two flow assignments.
The first one routes all demand along the path $s-u-v-t$.  In this case, path $s-u-v-t$ is the only positive path and has cost equal to $2$.  It is easy to verify that this is a $3/2$-PNE, $3/2$-UNE, and $3/2$-EF.  The second flow routes $2/3$ of the demand along  path $s-u-t$ and routes $1/3$ along path $s-v-t$.  Path $s-u-t$ has cost equal to $1$ and path $s-v-t$ has cost equal to $3/2$.  It is easy to verify that this is a $3/2$-PNE, $3/2$-UNE, and $3/2$-EF as well.  However, if we take the convex combination of these two assignments evenly, then we can find that the path $s-v-t$ has cost equal to $11/6$ and the path $s-u-v-t$ has cost equal to $7/6$, and thus their ratio is greater than $3/2$.  This shows that this combined flow is neither a $3/2$-PNE, a $3/2$-UNE, nor a $3/2$-EF, and hence they are not convex sets.
\end{proof}

\begin{figure}
	\centering
	\includegraphics[width=0.3\linewidth]{examples_noncvx2}
	\caption{Non-convexity of $\theta$-flows}
	\label{fig:ex_noncvx2}
\end{figure}

%The example for the non-convexity of $\theta$-I is demonstrated in Figure~\ref{fig:ex_noncvx1}.  Consider a commodity that routes unit demand from $s$ to $t$.  Note that the Nash length for this graph is $2$, when routing $8/9$ on the path $e1-e2$ and routing $1/9$ on the path $e3-e4$.  Similarly to the example above, we then consider two flow assignments.  The first one is to route half on $e1-e4$ and route half on $e3-e2$.  In this flow both path have length $11/2$.  We can easily verify that this is a $3$-I flow.  The second one is to route $2/3$ on the path $e1-e2$ and route $1/3$ on $e3-e4$.  The path $e1-e2$ has length $2$ and the path $e3-e4$ has length $6$.  However, if we mix them evenly, then the path $e3-e4$ has length $15/2$, and thus the combined flow is not $6$-I.  As a consequence, the set of $6$-I flows in this example is not a convex set.


%%-------------------------------------------------------------------------
% Hierarchical matrices can be seen as algebraic generalizations of 
% the FMM where a matrix is compressed using both low-rank and sparse representations
% hierarchically ~\cite{bebendorf08}. 
% We consider a matrix $K\in \mathbb{R}^{N\times N}$ to be
% \emph{hierarchical} if it can be approximated as
% \begin{equation}
%   \label{e:partitioning}
%   \sk{K} =
% \begin{bmatrix}
% K_{\lc\lc} & 0 \\ 
% 0 & K_{\rc\rc} \\ 
% \end{bmatrix} + 
% \begin{bmatrix} 
%   0 & UV_{\lc\rc} \\ 
%   UV_{\rc\lc} & 0 \\ 
% \end{bmatrix} +
% \begin{bmatrix} 
% 0 & S_{\lc\rc} \\ 
% S_{\rc\lc} & 0 \\ 
% \end{bmatrix}
% \end{equation} 
% where the \emph{off-diagonal} blocks $K_{\lc\rc}$ and $K_{\rc\lc}$ 
% are \emph{approximated} by some low-rank factorizations
% $UV$ plus a sparse correction matrix $S$, and
% the \emph{on-diagonal} blocks $K_{\lc\lc}$ and $K_{\rc\rc}$ are
% themselves hierarchical.
% For example, on the right side of \figref{fig:tree}, 
% the sparsity of $S$ is in blue, and $UV$ contains
% many submatrices with different low-rank approximations in pink.
% %For simplicity, we write th approximation as $K\approx \sk{K}=D+UV+S$, 
% %where $UV$ are \emph{far} (approximation), $S$ is \emph{near} (no approximation)
% %and $D$ is again hierarchical.
% 
% %Note that the low-rank plus sparse structure is \emph{not
% %invariant} on permutations, it very strongly depends on the ordering
% %of the columns (or rows since the matrix is symmetric).
% 
% \begin{figure}[h]
%   \centering
%   \includegraphics[scale=.3]{figures/nearfar.pdf}
%   \caption{A hierarchical low-rank plus sparse matrix (right) and 
%     its tree representation.
%     The off-diagonal blocks are combinations of low-rank matrices (pink)
%     and sparse matrices (blue). The on-diagonal blocks are further
%     partitioned as two child nodes in the tree.
%     The $\bigstar$ symbols denote the entries (neighbors) that cannot
%     be approximated.
%     \algref{a:nearnear} computes the sparse pattern  in blue.
%     The low-rank pattern in pink is computed by \algref{a:nearfar}
%     and \algref{a:farfar} with a preorder and postorder traversal.
%     The solid edges in the tree show the traversing paths of
%     \texttt{NearFar($\beta$,0)}. The preorder traversal stops and add
%     $\alpha$ to $Far(\beta)$,
%     because $K_{\beta\alpha}$ does not contain any $\bigstar$.
%     \algref{a:farfar} merges common codes from two children's Far lists.
%     For example, after \texttt{NearFar(\lc,0)} and \texttt{NearFar(\rc,0)},
%     $Far(\lc)=\{\rc,4,4\}$ and $Far(\rc)=\{\lc,4,2\}$.
%     The common nodes will then be removed and merged to their parent such that 
%     $Near(\alpha)=\{4,2\}$, 
%     $Near(\lc)=\{\rc\}$ and
%     $Near(\rc)=\{\lc\}$.
%   }
%   \label{fig:tree}
% \end{figure}
% 
% 
% 
% 
% \textbf{Treecodes.}
% Note that the low-rank structure strongly depends on the ordering
% of the columns (or rows since the matrix is symmetric).
% Such ordering is typically exploited by a geometric tree structure, and
% equation \eqref{fig:tree} resembles the relation between a parent
% node $\alpha$ and the two children $\lc$ and $\rc$ in a tree.
% In \figref{fig:tree},
% a pair of treenodes $\alpha$ and $\beta$ correspond to a submatrix
% $K_{\beta\alpha} = \{K_{ij}\lvert i\in{\beta},j\in{\alpha}\}$.
% To create such approximation in \figref{fig:tree} for each
% $K_{\beta\alpha}$, we need to decide whether $\alpha$ and $\beta$
% are \emph{near} (in blue, cannot approximate) or \emph{far} 
% (in pink, prunable) from each other.
% %from the root to the leaf level.
% 
% \begin{algorithm}[!t]
% \caption{{} \texttt{NearNear}()}
% \begin{algorithmic}
%   \STATE \texttt{\bf for each} $\alpha$ and $\beta$ is leaf \texttt{\bf do}
%   \STATE \gap \texttt{\bf if} $\alpha \cap \MA{N}(\beta)$ \texttt{\bf then}
%            $Near(\beta) = Near(\beta) \cup \alpha$;
% \end{algorithmic}
% \label{a:nearnear}
% \end{algorithm}
% 
% \begin{algorithm}[!t]
% \caption{{} \texttt{NearFar}($\beta$, $\alpha$)}
% \begin{algorithmic}
%   \STATE \texttt{\bf if} $\alpha \cap Near(\beta)$ \texttt{\bf then}
%   \texttt{NearFar}($\beta$,\lc); \texttt{NearFar}($\beta$,\rc); 
%   \STATE \texttt{\bf else} $Far(\beta) = Far(\beta) \cup \alpha$;
% \end{algorithmic}
% \label{a:nearfar}
% \end{algorithm}
% 
% \begin{algorithm}[!t]
% \caption{{} \texttt{FarFar}($\alpha$)}
% \begin{algorithmic}
%   \STATE \texttt{FarFar}($\lc$); \texttt{FarFar}($\rc$);
%   %\STATE \texttt{\bf if} $\alpha$ is not leaf \texttt{\bf then}
%   \STATE $Far(\alpha) = Far(\lc) \cap Far(\rc)$; 
%   \STATE $Far(\lc) = Far(\lc) \backslash Far(\alpha)$; $Far(\rc) = Far(\rc) \backslash Far(\alpha)$;
% \end{algorithmic}
% \label{a:farfar}
% \end{algorithm}
% 
% \textbf{Near-far pruning.}
% Let $\MA{N}(\beta)$ be the set of \emph{neighbors} of $\beta$ ($\beta$ itself and indices 
% shown as $\bigstar$ in \figref{fig:tree}).
% We say $\beta$ and $\alpha$ are \emph{near} if $\alpha \cap \MA{N}(\beta)$ 
% is not empty (i.e., $K_{\beta\alpha}$ contains a least one $\bigstar$).
% \algref{a:nearnear} shows how we construct a list of leaf nodes
% $Near(\beta)$ to denote the non-prunable submatrices (blue) in
% \figref{fig:tree}. Here $Near(\beta) = \{\mu, \beta\}$, because $\mu$ 
% contains a neighbor of $\beta$.
% The prunable submatrices (pink) are
% identified with a preorder and postorder traversal in \algref{a:nearfar}
% and \algref{a:farfar} (see the caption of \figref{fig:tree} for examples). 
% For each leaf node $\beta$, \algref{a:nearfar}
% traverses the tree from the root. 
% If $\alpha \notin Near(\beta)$, then we add $\alpha$ into $Far(\beta)$.
% Otherwise, we recurse to the two children of $\alpha$ to see
% if we can add $\lc$ or $\rc$ to $Far(\beta)$.
% %In \figref{fig:tree}, the solid edges denote the path 
% %of \texttt{NearFar($\beta$,0)}.
% %The traversal stops at $\alpha$, 
% %because $\alpha$ is not in $Near(\beta)$. 
% %On the other hands, $Near(\beta) \in \beta \subset$ node 4; thus the
% %traversal continues after visiting node 4.
% Once we have a prunable list $Far(\beta)$ for each leaf node, 
% \algref{a:farfar} traverses bottom-up to merge the common nodes
% from two children lists $Far(\lc)$ and $Far(\rc)$.
% These common nodes are removed from the children and added to 
% their parent's prunable list $Far(\alpha)$, which
% creates larger pink submatrices.
% %For example, after \texttt{NearFar(\lc,0)} and \texttt{NearFar(\rc,0)},
% %$Near(\lc)=\{\rc,4,4\}$ and $Near(\rc)=\{\lc,4,2\}$.
% %The common nodes will then be removed and merged to their parent such that 
% %$Near(\alpha)=\{4,2\}$, 
% %$Near(\lc)=\{\rc\}$ and
% %$Near(\rc)=\{\lc\}$.
% The number of these blue and pink submatrices are usually linear in $N$.
% If the size (for blue) and the rank (for pink) of these submatrices 
% are also constant in $N$, then matrix-vector multiplication may only require
% $\MA{O(N)}$ work with the following scheme.
% 
% \textbf{Fast multiplication.}
% The key for FMM to reach linear time multiplication is to
% compress a pink $K_{\beta\alpha}$ on both sides 
% using nested column and row bases.
% Here 
% we use a two-sided nested Interpolative Decomposition (ID) to
% approximate $UV_{\lc\rc}$ as
% $P_{\sk{\lc}\lc}^{T}K_{\sk{\lc}\sk{\rc}}P_{\sk{\rc}\rc}$.
% Given $\sk{\lc} \subset \lc$
% and $\sk{\rc} \subset \rc$, $K_{\sk{\lc}\sk{\rc}}$ denotes a submatrix
% of $K_{\lc\rc}$, and $P_{\sk{\lc}\lc}$, $P_{\sk{\rc}\rc}$ are
% matrices of coefficients used to interpolate other entries.
% For an inner node $\alpha$, its skeleton
% $\sk{\alpha} \subset \sk{\lc} \cup \sk{\rc}$ is subselected from its
% children's skeletons. With this nested basis, the parent 
% coefficient matrix $P_{\sk{\alpha}\alpha}$ can be computed as a
% linear combination of two children's coefficients as shown in the box
% \begin{equation}
% \sk{w}_{\alpha} = 
% \framebox[1.1\width]{
% $P_{\sk{\alpha}\alpha}$
% }w_{\alpha}
% =
% \framebox[1.1\width]{$
% P_{\sk{\alpha}[\sk{\lc}\sk{\rc}]}
% \begin{bmatrix}
%   P_{\sk{\lc}\lc} & \\
%                   & P_{\sk{\rc}\rc} \\
% \end{bmatrix}
% $}
% \begin{bmatrix}
% w_{\lc} \\
% w_{\rc} \\
% \end{bmatrix}=
% P_{\sk{\alpha}[\sk{\lc}\sk{\rc}]}
% \begin{bmatrix}
%   \sk{w}_{\lc} \\
%   \sk{w}_{\rc} \\
% \end{bmatrix}
% \label{e:telescope}
% \end{equation}
% While multiplying weight $w$ from the right, \eqref{e:telescope}
% allows us to compute
% the skeleton weights $\sk{w}_{\alpha}$ for all $\alpha$ in 
% linear time with a postorder traversal (\emph{telescoping}). 
% If we assume there is only a linear number of submatrices $K_{\beta\alpha}$, then
% accumulating the approximate sum for each node $\beta$ as
% \begin{equation}
%   \sk{u}_{\beta} = 
%   \sum_{\alpha \in Far(\beta)} K_{\sk{\beta}\sk{\alpha}}
%   \sk{w}_{\alpha} = 
%   \sum_{\alpha \in Far(\beta)} K_{\sk{\beta}\sk{\alpha}}
%   P_{\sk{\alpha}\alpha}
% \end{equation}
% takes linear time. Similarly, we can apply the same \emph{telescoping} 
% relation from the left (preorder) such that all skeleton potentials 
% \begin{equation}
%   \begin{bmatrix}
%     \sk{u}_l \\
%     \sk{u}_r \\
%   \end{bmatrix} +=
% \framebox[1.1\width]{$
% \begin{bmatrix}
%   P_{\sk{\lc}\lc} & \\
%                   & P_{\sk{\rc}\rc} \\
% \end{bmatrix}^{T}
% P_{\sk{\beta}[\sk{\lc}\sk{\rc}]}^{T}
% $}
% \sk{u}_{\beta}
% \end{equation}
% can also be computed in linear time. Finally, we assume that
% the size of $Near(\beta)$ is constant for each leaf node.
% Then computing $u_{\beta} = \sk{u}_{\beta} + \sum_{\alpha \in
% Near(\beta)}K_{\beta\alpha}w_{\alpha}$ for all leaf node $\beta$
% can also be done in linear tme.
% 
% Overall, FMM assumes that $Near(\alpha)$ and $Far(\alpha)$ 
% are both $\MA{O}(1)$ and uses nested basis on both sides to achieve
% linear time complexity.
% For hierarchical partitioning directly derived from the geometric space, this 
% near-far pruning may be straight forward.
% However, so far we have not yet defined the neighbors 
% ($\bigstar$) and how to partition $K$ as a tree for an arbitrary 
% SPD matrix. 
% In \secref{s:seq}, we illustrate an idea that generalizes this
% near-far pruning to general SPD matrices using the Gramian vector space.

%%----------------------------------------------------------------------------------------------------
\section{Quality of $\theta$-Flows}\label{sec:social}
In this section, we analyze the cost and fairness of the solution concepts introduced in Section~\ref{sec:satisfaction}.
As noted in Section~\ref{sec:satisfaction}, the $\theta$ flows are not unique for $\theta>1$ and that implies that potentially under each solution concept we can have a range of attainable costs. We present the upper bounds on the PoS and PoA, defined respectively in \eqref{eq:PoA} and \eqref{eq:PoS},  for the flows under the three solution concepts. We compare the social cost of the $\theta$ flows with the socially optimal flow.
 
\textbf{Price of Anarchy.} Starting from Correa et al.~\cite{correa2008geometric},  there has been a unifying approach of bounding the PoA using the variational inequality formulation of the Nash equilibrium flow.  Christodoulou et al.~\cite{christodoulou2011performance}  extended the idea of using a variational inequality to formulate an approximate equilibrium, specifically a $\theta$-PNE, and to give new bounds for the price of anarchy of a $\theta$-PNE. We first show that the $\theta$ variational inequality encompasses both the $\theta$-PNE and $\theta$-UNE. Therefore, we can use the well established technique to give an upper bound for both types of flows.

\begin{lemma}\label{lemm:UNEvVI}
If a flow $\bm{f}$ is a $\theta$-UNE with edge flow $\bm{x}$, then it satisfies the following variational inequality for $\theta\geq 1$,
\begin{align}\label{eq:thetaVI}
\sum_e x_e\ell_e(x_e) \leq \theta\sum_e x'_e\ell_e(x_e), ~\forall \bm{x}'\in \mc{D}_E. 
\end{align}                                                                                                                           
Further, there exists a single commodity network and a flow $ \bm{f}'' \in \mc{D}_p(\bm{x}'')$ that satisfies the above inequality but is not a $\theta$-UNE.
\end{lemma}
\begin{proof}
The proof of the first part follows closely the proof of Theorem $1$ in Christodoulou et al~\cite{christodoulou2011performance}. Let $\bm{f} \in \mc{D}_p(\bm{x})$ be a $\theta$-UNE and  $\bm{f}' \in \mc{D}_p(\bm{x}')$ be any other feasible flow in the network. From the definition of $\theta$-UNE, for any commodity $k$, for any used path $p\in \mc{P}^k_u$ and for any other path $p'\in \mc{P}^k$ we have $\sum_{e\in p}\ell_e(x_e) \leq \theta \sum_{e\in p'}\ell_e(x_e).$  Further,  taking the summation of the flow weighted path latency over all pairs of paths in commodity $k$, we obtain the following:
\begin{align*}
\sum_{\substack{p\in \mc{P}^k_u(\bm{f})\\ p'\in \mc{P}^k(\bm{f}')}} f_p^k f_p^{k\prime}\sum_{e\in p}\ell_e(x_e) 
&\leq \theta\sum_{\substack{p\in \mc{P}^k_u(\bm{f})\\ p'\in \mc{P}^k(\bm{f}')}} f_p^k f_p^{k\prime}  \sum_{e\in p'}\ell_e(x_e), \\
\sum_{p'\in \mc{P}^k(\bm{f}')} f_p^{k\prime} \sum_{e\in E}x_e^k\ell_e(x_e) 
&\leq \theta \sum_{p\in \mc{P}^k_u(\bm{f})} f_p^k  \sum_{e\in E}x_e^{k\prime}\ell_e(x_e), \\
\sum_{e\in E}x_e^k\ell_e(x_e)&\leq \theta \sum_{e\in E}x_e^{k\prime}\ell_e(x_e).
\end{align*} 
The last inequality follows due to $\sum_{p'\in \mc{P}^k(\bm{f}')} f_p^{k\prime} = \sum_{p\in \mc{P}^k_u(\bm{f})} f_p^k = d_k>0$. Finally, taking summation over all commodities gives $\sum_e x_e\ell_e(x_e) \leq \theta\sum_e x'_e\ell_e(x_e)$.

Consider Pigou's network with demand $1$, top edge with latency $\ell_t(x) = \epsilon x$ and bottom edge with latency $\ell_b(x) = L$. For a given $\theta>1$, choose $\delta>0$ small (to be specified later).
% \in (0, \theta -1)
 Consider the flow $\bm{f}''$ equal to
 $(1-\delta\epsilon/L)$ in the top link and $\delta\epsilon/L$ in the bottom link. 
 The social cost of this flow is $\left(\delta\epsilon+\epsilon(1-\delta\epsilon/L)^2\right)$. The feasible flow minimizing the right hand side of Inequality \eqref{eq:thetaVI} is flow of $1$ through top link for $L>\epsilon$. Further, for any $\theta$, there is a $\delta$ small enough such that Inequality  \eqref{eq:thetaVI} holds, since, by the above, it suffices to have $\delta\epsilon+ \epsilon(1-\delta\epsilon/L)^2 \leq\theta \epsilon(1-\delta\epsilon/L)
\Leftrightarrow \theta\geq 1+ \frac{L^2\delta-L\delta \epsilon+\delta^2 \epsilon^2}{L(L-\delta\epsilon)} $. 
However, $\bm{f}''$ is not $(L/\epsilon)$-UNE and this approximation factor can be made arbitrarily large making $\epsilon$ small enough. 
\end{proof}

\begin{remark}
Here we have shown that the variational inequality is a sufficient condition for a flow to be $\theta$-UNE and the counterexample shows it is not necessary. Though due to Dafermos and Sparrow~\cite{dafermos1969traffic}, we know that for $\theta=1$ the variational inequality is the necessary and sufficient condition for $1$-PNE. This gives an alternative proof to the fact that $1$-UNE$=1$-PNE.
\end{remark}


For a fixed $\theta$, the flows satisfying Inequality~\eqref{eq:thetaVI} form a set. Call this set $\theta$-VI. %We define the PoA($\theta$, VI) similarly as in equation~\eqref{eq:PoA} over this set. 
The following lemmas characterize the price of anarchy for various flows under latency functions in class $\mc{L}$. We adopt the approach of Harks et al.~\cite{harks2007price} as it produces tighter PoA bounds for several latency functions compared to previous approaches~\cite{roughgarden2002selfish, correa2008geometric}. The result in~\cite{harks2007price} is for $\theta=1$ and here we state it for general $\theta$. 

We begin with a simple corollary to Lemma~\ref{lemm:UNEvVI} and omit the proof.
\begin{corollary}
For any multi-commodity instance $\mc{G}$ and any $\theta\geq 1$, the PoA values for the corresponding solution concepts are related as PoA($\theta$-PNE) $\leq $  PoA($\theta$-UNE) $\leq $  PoA($\theta$-VI).
\end{corollary}
  
We need the following definitions in order to bound PoA($\theta$-VI):
\begin{align*}
\omega(\mc{L}, \lambda) &= \sup_{\ell \in \mc{L}} \sup_{x,x'\geq 0}\frac{\left(\ell(x)-\lambda\ell(x')\right)x'}{x\ell(x)}.\\
\Lambda(\theta) &= \{\lambda\in \mathbb{R}^{+}: \omega(\mc{L}, \lambda)\leq  1/\theta \}.
\end{align*}


\begin{lemma}
For an instance $\mc{G}$ with latency functions in class $\mc{L}$, the PoA($\theta$-VI) is upper bounded by $ \inf_{\lambda \in \Lambda(\theta)} \theta\lambda(1-\theta\omega(\mc{L}, \lambda))^{-1}$. 
\end{lemma}
\begin{proof}
Let $\bm{x}$ be a $\theta$-VI flow satisfying Condition~\eqref{eq:thetaVI} and $\bm{y}\in SC_E$ be a socially optimal flow. Then, we have the following relations:
\begin{align*}
SC(\bm{x})= & \sum_e x_e\ell_e(x_e) \leq \theta\sum_e y_e\ell_e(x_e)\\
&\leq \theta\sum_e \left(y_e\ell_e(x_e)-\lambda y_e\ell_e(y_e) + \lambda y_e\ell_e(y_e)\right)\\
&\leq \theta  \omega(\mc{L}, \lambda) SC(\bm{x}) + \theta\lambda SC(\bm{y}).
\end{align*}
We obtain the desired bound by taking infimum over the set $\Lambda(\theta)$.
\end{proof}

\begin{example}
As an example consider the class of linear latency functions $\ell(x)= ax+b$.  For this class, we can obtain  $\omega(\mc{L},\lambda)\leq 1/4\lambda$ for $\lambda \geq 1$ and  $\omega(\mc{L},\lambda)>1$ otherwise. An upper bound on PoA($\theta$-VI) can be obtained through minimizing over the set $\Lambda(\theta)=\{ \lambda \geq \max\{1,\theta/4\}\}$. The exact bound obtained through this is $\max\{\theta^2, 4\theta/(4-\theta)\}$ and it matches the bounds given in~\cite{christodoulou2011performance}. Note that for $\theta=1$ it gives us the classical bound of $4/3$.
\end{example}

The following proposition further separates the envy free flows and the variational inequality characterization. 
\begin{proposition}
There exists a $1$-EF flow for which the variational inequality in \eqref{eq:thetaVI} does not hold for any bounded $\theta'$. Further, the PoA($1$-EF) is unbounded.
\end{proposition}
\begin{proof}
Consider the instance from Lemma~\ref{lemm:UNEvVI}, i.e. a Pigou network with demand $1$, top edge with latency $\ell_t(x) = \epsilon x$ and bottom edge with latency $\ell_b(x) = L$. The flow that routes $1$ unit through the bottom link (i.e., all the demand) is a $1$-EF flow but it does not satisfy Condition~\eqref{eq:thetaVI} for  any $\theta$, since under this flow the top link has cost 0.
For $L>2\epsilon$ the optimal flow routes all the flow through the upper link and thus the PoA of $1$-EF flows is $L/\epsilon$, which cannot be bounded as we can make $\epsilon$ arbitrarily small.
\end{proof}

\textbf{Price of Stability.} As discussed in the introduction, $\theta$-UNE and $\theta$-EF flows arise from the ability of a central planner to induce path flows in the network. The price of anarchy is motivated by the dynamics of users who can induce any worst case flow in the network under some given solution concept. In contrast, the price of stability is the quantity that is of special interest to the central planner, who wishes to induce the best (with respect to social cost) $\theta$-UNE or $\theta$-EF flow. Using already known techniques we can obtain bounds on the PoS but we defer this part to Section \ref{sec:designSubPoteFunc}, as these techniques will also be used to provide the central planner with good (with respect to social cost) $\theta$ fair flows, which is the scope of Section \ref{sec:design}.

\begin{lemma}\label{lemm:POSequality}
 For any multi-commodity network $\mc{G}$ and any $\theta\geq 1$, the PoS values of the corresponding flows are related as PoS($\theta$-EF) $\leq $  PoS($\theta$-UNE) $\leq $  PoS($\theta$-PNE). Moreover, there exists a network $\mc{G}$ such that for all $\theta\geq 1$ all the inequalities are tight.
\end{lemma}
\begin{proof}
The proof of the first part of the lemma follows due to Lemma~\ref{lemm:PNEvUNEvEF} and the fact that the infimum of a function over a set is less than or equal to the infimum of the same function over any subset of the set.

Consider the Pigou network with unit demand,  upper link having latency $\ell_u(x)= 1$ and  lower  link having latency $\ell_b(x) = x$. For this network and for any $\theta\geq 1$ the optimal $\theta$-PNE, $\theta$-UNE and $\theta$-EF are identical\footnote{For the special case of $\theta=1$, routing all the demand through the upper link is an optimal $1$-EF which is not a $1$-PNE or a $1$-UNE. Yet, the cost of this $1$-EF flow is the same as the $1$-PNE, $1$-UNE and $1$-EF flow that routes all the demand through the lower link. and given by $\max\{1/\theta, 0.5\}$ units of flow in the lower link and the remaining flow through the upper link. }
\end{proof}



\begin{remark}
%First of all, 
It is important to emphasize that the upper bound guarantees of PoS encompass all possible networks under a given latency class. Whereas, for a particular network the achievable social cost under $\theta$-fairness can be better compared to the bound dictated by PoS. As an example, consider the latency function class of polynomials of degree at most $p$ where the best possible upper bound for PoS($\theta$-PNE) is $\left(\theta\left(1-\frac{p\theta^{1/p}}{(p+1)^{(1+1/p)}}\right)\right)^{-1}$ for $\theta< p+1$~\cite{christodoulou2011performance}. On the other hand, as shown by Defarmos et al.~\cite{dafermos1969traffic}, if all the latency functions are monomials of degree $p$, the $1$-PNE is the socially optimal flow.
\end{remark}
 


%%----------------------------------------------------------------------------------------------------
\section{Existence and Complexity}\label{sec:complex}
In this section we discuss the computational issues surrounding the three types of $\theta$ fair flows. The existence of Pure Nash equilibrium in nonatomic routing games guarantees the existence of any $\theta$ fair flow for $\theta\geq 1$. The next question would be whether we can compute $\theta$ fair flows with good social cost.  In particular, we consider the following problems:
\begin{enumerate}
	\item[(P1)] Find a $\theta$-EF path flow with the minimal social cost.
	\item[(P2)] Find a $\theta$-UNE path flow with the minimal social cost.
	\item[(P3)] Find a $\theta$-PNE edge flow with the minimal social cost.
\end{enumerate}
We show that for large $\theta$, the socially optimal flow is guaranteed to be contained in those $\theta$-flows, and hence the optimal $\theta$-flows be computed efficiently.  However, for small $\theta$, we will show that solving Problem~(P1) and Problem~(P2) is NP-hard, while it remains open whether Problem~(P3) can be computed efficiently.  
More precisely, for a latency class $\mc{L}$, this particular threshold is $\gamma(\mc{L})= \min\{\gamma: \ell^{*}(x)\leq \gamma \ell(x), \forall \ell\in \mc{L}, \forall x\geq 0\}$, where $\ell^{*}(x)= \ell(x)+x\ell'(x)$.  The main result of this section is given as follows:
\begin{theorem}
	%\begin{enumerate}
	%\item \fin{
	For any multi commodity instance $\mc{G}$ 
	with latency functions in any class $\mc{L}$, there are polynomial time algorithms\footnotemark $\, $  for solving Problem~(P1)-(P3) 
	for $\theta \ge \gamma(\mc{L})$. %, for any class $\mc{L}$.
	%
	On the other hand, it is NP-hard to solve Problem~(P1) for $\theta \in [1, \gamma(\mc{L}))$ and Problem~(P2) for $\theta \in (1, \gamma(\mc{L}))$, for  arbitrary single commodity instances %$\mc{G}$ 
	with latency functions in an arbitrary class $\mc{L}$.
	%\end{enumerate}
	\label{thm:main_hardness}
\end{theorem} 

\footnotetext{The existence of polynomial time algorithms for our problem depends on the assumption that we can minimize separable convex functions with linear constraints in polynomial time; numerical issues for convex optimization are discussed in \cite{hochbaum1990convex,  nemirovski2004interior} and are beyond the scope of our work.}
%on the existence of polynomial time algorithms for minimizing separable convex functions with linear constraints upto arbitrary precision. See~\cite{hochbaum1990convex,  nemirovski2004interior}   for details, which are quite technical for the scope of this paper.}}





In the following sections, we first prove the first part of Theorem \ref{thm:main_hardness}. Right after we show that, for any $\theta$, from any $\theta$ fair flow we may get another $\theta$ fair flow, which uses only polynomially many paths. This, on the one hand, serves as a clarification that the difficulty of problems~(P1)-(P3) does not lie in the size of their solutions. On the other hand, it helps in showing that the decision version of these problems lies in NP, since for a YES instance, a non-deterministic machine will (non-deterministically) choose a path flow of polynomial size and in polynomial time check that it satisfies the conditions needed. Finally, the second part of Theorem \ref{thm:main_hardness} follows from an NP-hardness proof for a stronger version of the decision versions of problems~(P1) and (P2) (Theorem \ref{thm:12Hard}).


\subsection{When the Social Optimum is Guaranteed to be the Solution}\label{sec:complexity_so}
First, we show that Problems~(P1)-(P3) are easy for $\theta \ge \gamma(\mc{L})$ because the social optimum is the solution.
%For a latency class $\mc{L}$, define $\gamma(\mc{L})= \min\{\gamma: \ell^{*}(x)\leq \gamma \ell(x), \forall \ell\in \mc{L}, \forall x\geq 0\}$. Here $\ell^{*}(x)= \ell(x)+x\ell'(x)$ is the marginal latency.  
The following lemma, which is a direct extension of Theorem~4.2 in Correa et al~\cite{correa2007fast}, shows that any path decomposition of the socially optimal flow is a $\gamma(\mc{L})$ fair flow, provided that the latency functions are in class $\mc{L}$.  While in the proof of Theorem~4.2 in Correa et al~\cite{correa2007fast} they only conclude that the set of socially optimal path flows is $\gamma(\mc{L})$-EF, it is easy to see that the same argument holds for $\gamma(\mc{L})$-PNE.
\begin{lemma}(\cite{correa2007fast})
For a network $\mc{G}$ with latency functions  in class $\mc{L}$, any socially optimal path decomposition $o \in SO_p$ is $\gamma(\mc{L})$-PNE.
\end{lemma}
 
Since the social optimum can be computed using convex programming~\cite{roughgarden2002selfish}, it follows that %With this result, we are able to give polynomial time algorithms for 
Problems~(P1)-(P3) can be solved in polynomial time\footnotemark[6] for $\theta\ge\gamma(\mc{L})$.
\begin{proof}[Proof, first part of Theorem~\ref{thm:main_hardness}]
Note that given a path flow, which is $\gamma(\mc{L})$-PNE, it is $\gamma(\mc{L})$-UNE and $\gamma(\mc{L})$-EF as well.  This means that for all $\theta \geq \gamma(\mc{L})$, we can simply compute the socially optimal flow, and give any path decomposition as the $\theta$ fair flow. The socially optimal edge flow can be computed in time polynomial in the size of the network.  Further, a greedy path decomposition suffices. In the greedy algorithm, at every step we pick the current minimum path (among all commodities) and assign the maximum possible flow, under the social optimum, through this path. This can be computed in time $O(|\mc{K}|\times|E|)$.  Also, the output path flow can be represented with a sparse vector with $O(|\mc{K}|\times|E|)$ entries.
%linear in number of edges times the number of commodities ).
\end{proof}



\subsection{Existence of Polynomial-size Path Flow Solutions}
An observation to Problem~(P1) and (P2) is that the outputs of these two problems are path flow vectors, which are potentially of exponential size relative to the problem instances. 
In Section~\ref{sec:complexity_so} we showed a way to compute a path flow vector with polynomial support under the social optimum.  Here we ask whether we can do this for any edge flow.  In particular, we are interested in whether we can always find an answer to either Problem~(P1) or (P2) using only polynomially many paths.  If not, then there is no hope for us to find an efficient algorithm for these problems.  In this subsection, we show that the answer to this question is \emph{yes}.  To see this, we make a more general argument than Lemma~3.1 in Correa et al.~\cite{correa2007fast}, showing that given any path flow vector, we can always find another path flow assignment of polynomial support that preserves four important properties.
\begin{proposition}\label{lemma:correa1}
	Let $\bm{f}$ be a feasible flow for a multicommodity flow network with load-dependent edge latencies. Then, there
	exists another feasible flow $\bm{f}'$ such that 
	\begin{enumerate}
		\item $\bm{f}$ and $\bm{f}'$ have the same edge flow.
		\item The longest used path for commodity $k$ satisfies $\max_{\pi \in \mc{P}_u^k(\bm{f'})} l_{\pi}(\bm{f'}) \le \max_{\pi \in \mc{P}_u^k(\bm{f})} l_{\pi}(\bm{f})$.
		%$L_{max}(\bm{f}')\leq L_{max}(\bm{f}')$.
		\item The shortest used path for commodity $k$ satisfies $\min_{\pi \in \mc{P}_u^k(\bm{f})} l_{\pi}(\bm{f}) \le \min_{\pi \in \mc{P}_u^k(\bm{f'})} l_{\pi}(\bm{f'})$.
		\item The flow $\bm{f}'$ uses at most $|E|$ paths for each source-sink pair.
	\end{enumerate}
\end{proposition}
\begin{remark}
The proof of this proposition directly follows the proof of Lemma~3.1 in Correa et al.~\cite{correa2007fast}, although our lemma statement is more general. (Their Lemma only states part 2 of our Lemma statement.)
\end{remark}

%\begin{proof}
%	The proof basically follows the proof of Lemma~3.1 in Correa et al.~\cite{}.  They introduce a process that iteratively create a new path flow $\bm{f}'$ that has the same edge flow as $\bm{f}$ by moving the flow along one particular path to other used paths.  In their lemma, while they only conclude that the length of the longest path will not increase, similar argument will also hold for that the length of the shortest path will not decrease. 
	%Fix commodity $k$, we index the used paths in $\mc{P}_u^k(\bm{f})$ by $\pi_1, \pi_2, \dots, \pi_r$, where $r$ is the number of used paths in commodity $k$ under $\bm{f}$.  For each $\pi_i$, we define an edge incident vector $\bm{p}_i \in \{0,1\}^{|E|}$, where $p_{ie}=1$ if and only if $e \in \pi_i$.  If $r>|E|$, then the edge incident vectors are linearly dependent.  Then, we can reduce the number of used path with the following process:  First let $\lambda_1, \lambda_2, \dots, \lambda_r$ be such that $\sum_{i=1}^{r} \lambda_i \bm{p}_i = 0$.
%\end{proof}
With this proposition, we can make the following argument that given an edge flow $\bm{x}$, if there is at least one $\theta$-EF or $\theta$-UNE path flow decomposition, then we can always find one with %only linear
polynomial support:
\begin{lemma}\label{thm:npproblems}
	Given a $\theta$-EF path flow $\bm{f}_1$, there exists a $\theta$-EF path flow $\bm{f}'_1$ that uses at most $|E|$ paths for each source-sink pair and has the same edge flow as $\bm{f}_1$.  Similarly, given a $\theta$-UNE path flow $\bm{f}_2$, there exists a $\theta$-UNE path flow $\bm{f}'_2$ that uses at most $|E|$ paths for each source-sink pair and has the same edge flow as $\bm{f}_2$.
\end{lemma}
\begin{proof}
	For a $\theta$-EF path flow $\bm{f}_1$, by Proposition~\ref{lemma:correa1}, there exists a flow $\bm{f}'_1$ that has the same edge flow as $\bm{f}_1$ and the ratio of the longest used path to the shortest used path is bounded by
	$$
	\frac{\max_{\pi \in \mc{P}_u^k(\bm{f'}_1)} l_{\pi}(\bm{f'}_1)}{\min_{\pi \in \mc{P}_u^k(\bm{f'}_1)} l_{\pi}(\bm{f'}_1)} \le \frac{\max_{\pi \in \mc{P}_u^k(\bm{f}_1)} l_{\pi}(\bm{f}_1)}{\min_{\pi \in \mc{P}_u^k(\bm{f}_1)} l_{\pi}(\bm{f}_1)} \le \theta
	$$
	which indicates that $\bm{f}'$ is a $\theta$-EF path flow.  Similarly, given a $\theta$-UNE path flow $\bm{f}_2$, we can find a path flow $\bm{f}_2'$ that has the same edge flow as $\bm{f}_2$ and 
	$$
	\frac{\max_{\pi \in \mc{P}_u^k(\bm{f'}_2)} l_{\pi}(\bm{f'}_2)}{\min_{\pi \in \mc{P}^k} l_{\pi}(\bm{f'}_2)} \le \frac{\max_{\pi \in \mc{P}_u^k(\bm{f}_2)} l_{\pi}(\bm{f}_2)}{\min_{\pi \in \mc{P}^k} l_{\pi}(\bm{f}_2)} \le \theta
	$$
	from which we can conclude that $\bm{f}'_2$ is a $\theta$-UNE as well. 
\end{proof}
Now suppose $\bm{f}_1^*$ is the optimal solution to Problem~(P1).  According to Lemma~\ref{thm:npproblems}, we can see that there is an alternative path flow $\bm{f}_2^*$ that is also $\theta$-EF and has the same edge flow as $\bm{f}_1^*$.  Since the social cost only depends on the amount of the edge flow, $\bm{f}_1^*$ and $\bm{f}_2^*$ have the same social cost, from which we can conclude that $\bm{f}_2^*$ is an optimal solution to Problem~(P1) that uses only polynomially many paths.  A similar argument can be made for Problem~(P2) as well.

\subsection{Hardness Results}
In this section, we prove the second part of Theorem~\ref{thm:main_hardness} that it is NP-hard to solve Problem~(P1) and (P2) for small values of $\theta$.  More precisely, we consider the class of polynomial functions of degree at most $p$, which we denote as $\mc{L}_p$. We note that $\gamma(\mathcal{L}_p)=p+1$. We show that when the latency functions are in $\mc{L}_p$, then the related decision problems we state in Theorem~\ref{thm:12Hard} have polynomial-time reductions from the NP-complete problem PARTITION.  We state this result in the following theorem:

%Suppose we are given an instance of a single commodity flow network $\mc{G}$ 
%with latency functions in class $\mc{L}_p$ for $p \ge 1$.

\begin{theorem}
For an arbitrary single commodity instace $\mc{G}$ 
with latency functions in class $\mc{L}_p$ for $p \ge 1$, it is NP-hard to
\begin{enumerate}
\item decide whether a socially optimal flow has a $\theta$-UNE path flow decomposition for $\theta \in (1, p+1)$.
\item decide whether a socially optimal flow has a $\theta'$-EF path flow decomposition for $\theta' \in [1, p+1)$.
\end{enumerate}
\label{thm:12Hard}
\end{theorem}

We state the following corollary that readily follows from Theorem~\ref{thm:12Hard}.
\begin{corollary}
For  any finite $\theta> 1$, it is NP-hard to find the optimal $\theta$-UNE or $\theta$-EF flow of an arbitrary instance $\mathcal{G}$.
\end{corollary}
\begin{proof}
For a given $\theta$ pick any $p\in \mathbb{N}:\theta<p+1$. Since $p+1=\gamma(\mathcal{L}_p)$, we may use  Theorem~\ref{thm:12Hard} to get the result.
\end{proof}

The proof of Theorem~\ref{thm:12Hard} is composed of two parts.  For the first part, we show the NP-hardness for $1.5$-UNE and $1$-EF path flow decompositions under the social optimum  in Lemma~\ref{lemm:corehardness}, based on the construction in Theorem~3.3 in Correa et al.~\cite{correa2007fast}.  Then, in the second part, we propose a novel way to generalize the construction to the entire range of $\theta$ and $\theta'$ specified in Theorem~\ref{thm:12Hard}.

\begin{lemma}\label{lemm:corehardness}
For single commodity instances with linear latency functions it is NP-hard to decide whether a social optimum flow has a $1.5$-UNE flow decomposition or a $1$-EF flow decomposition.
\end{lemma}
\begin{proof}
We consider the PARTITION problem, where we are given a set of $n$ positive integer numbers $q_1,\ldots, q_n$, and we need to decide  \emph{is there a subset $I \subset \{1,\ldots,n\}$ such that $\sum_{i\in I} q_i = \sum_{i \notin I}q_i$?}
%-----------------------------------------------------------------------------------------
 \begin{figure}[!htb]
 \centering
 \includegraphics[width=0.6\linewidth]{reduction}
 \caption{An instance of congestion game constructed from a given instance of PARTITION}
 \label{Fig:instForProg4}
 \end{figure}
%-----------------------------------------------------------------------------------------

Consider the two link parallel network with the top link $e_{u}$ having latency $\ell_u(x)=q$ and the bottom link $e_b$ having latency $\ell_b(x)=qx$. The demand between the source and the destination is $1$. 
%In the equilibrium flow  the bottom link carries $\frac{1}{2}(1+k)^{\frac{1}{k}}$ flow and the Nash equilibrium length $L_{NE,1}=\frac{k+1}{2^k}$. 
The unique socially optimal flow splits the flow equally through the top and bottom link. Call this instance $G(q)$.

 
Given an instance of the PARTITION problem, $q_1,\ldots, q_n$, $\sum_{i=1}^{n} q_i=2B$, we now construct a single commodity network as the two link $n$ stage network $G$, as shown in Figure \ref{Fig:instForProg4}. In stage $i$ we connect $G(q_{i-1})$ to $G(q_{i})$ to the right for $i=2$ to $n$. A unit demand has to be routed from the source in $G(q_1)$ to the destination in $G(q_n)$.  For the graph $G$, the socially optimal flow $o$ routes $1/2$ flow through all top links and the remaining $1/2$ flow through each bottom link. We first observe that there is a one-to-one correspondence between the subsets $I\subseteq [n]$ and paths $p$ in $G$. Specifically, we can define the path corresponding to $I$ as $P_I=\left\{e_{u,i}: i\in I\right\} \cup \left\{e_{b,i}: i\notin I\right\}$. Further, the latency of the path is given by $\ell_I = \frac{1}{2}(\sum_{i\in [n]} q_i+\sum_{i\in I} q_i)$. 

In one direction, we observe that if the answer to the PARTITION problem is YES then there exists a subset $I^*$ such that $\sum_{i\in I^*} q_i = \sum_{i \notin I^*}q_i = B$. Consider the path flow under socially optimal flow $o$, with path $P_{I^*}$ carrying flow $1/2$ and path $P_{[n]\setminus I^*}$ carrying flow $1/2$.  The lengths of paths $P_{I^*}$ and $P_{[n]\setminus I^*}$ are both  equal to $\frac{3}{4}\sum_{i=1}^{n} q_i = 3B/2$. Whereas, the shortest path in the network is  $P_{\emptyset}$  with length $\frac{1}{2}\sum_{i=1}^{n} q_i = B$. Therefore, the socially optimal flow $o$ is a $3/2$-UNE flow and a $1$-EF flow, if $G$ comes from a YES instance of PARTITION. 



In the other direction, we first observe that if a path $P_{I}$ under edge flow $o$ has length $3B/2 = \frac{3}{4}\sum_{i=1}^{n} q_i$, then $\sum_{i\in I} q_i = \frac{1}{2}\sum_{i\in [n]} q_i$. This implies the given answer to the PARTITION problem is YES. Now assuming $o$ is a $3/2$-UNE, there exists a path flow with the maximum used path of length less or equal to $\frac{3}{4}\sum_{i=1}^{n} q_i$. But the average length of any used path under $o$ is equal to $\frac{3}{4}\sum_{i=1}^{n} q_i$. This implies that all the paths in the path flow must have length $\frac{3}{4}\sum_{i=1}^{n} q_i$.  Next we assume that $o$ is a $1$-EF flow. This implies that there exists a path flow for which all the used paths have equal length. But then any used path under this decomposition has length $\frac{3}{4}\sum_{i=1}^{n} q_i$. Therefore, if $o$ is a $3/2$-UNE or a $1$-EF then the PARTITION instance corresponding to $G$ is a YES instance.   
\end{proof}
 
\begin{proof}[Proof of Theorem~\ref{thm:12Hard}]
Consider $\theta \in (1,p+1)$ for a UNE flow and $\theta' \in [1,p+1)$ for an EF flow. Given a PARTITION instance, let $G'$ be a two link parallel network with latency of the top link $\ell_{u,(n+1)}(x) = ax^p+b$ and bottom link latency $\ell_{d,(n+1)}(x) = cx^p$. We set $a=\frac{\alpha B}{(1-3/8B)^p}$, $b=\beta B(p+1)$, and $c=\frac{(\alpha+\beta)B}{(3/8B)^p}$, where $\alpha,\beta>0$ are some parameters to be determined later.

Using the fact that the social optimum is an equilibrium of the instance with latencies modified to $(\ell(x) + x\ell'_e(x))$, we  get that the socially optimal flow in network $G'$ is $\frac{3}{8B}$ through the bottom link and $(1-\frac{3}{8B})$ through the top link. We also get that at the social optimum the latency function satisfies the following condition:
$$
c\bigg(\frac{3}{8B}\bigg)^p = a\bigg(1-\frac{3}{8B}\bigg)^p + \frac{b}{p+1} < a\bigg(1-\frac{3}{8B}\bigg)^p + b
$$
From the latter, we can see that the top link has larger cost than the bottom link.  We then combine in series the network $G$ of Lemma \ref{lemm:corehardness} with the network $G'$ to obtain network $H$. The unique socially optimal flow in  network $H$ is the union of the two unique socially optimal flows in $G$ and $G'$. Recall the notation from Lemma \ref{lemm:corehardness}.

Assume the PARTITION problem admits a solution $I$. Consider the path decomposition in $H$:
\begin{enumerate}
	\item Path $p = P_{I}-e_{u, (n+1)}$ carries $1/2$ flow (note that $3/8B < 1/2$).
	\item Path $q = P_{I^c}-e_{u, (n+1)}$ carries $(1/2 - 3/8B)$ flow.
	\item Path $r = P_{I^c}-e_{d, (n+1)}$ carries $3/8B$ flow.
\end{enumerate}
We can see that the path $s = P_{\emptyset}-e_{d, (n+1)}$ is the shortest path, with latency $\ell_{s} = B + c(3/8B)^p = (\alpha+\beta+1)B$.  The longest used path $q$ has latency $\ell_{q} = 3B/2 + a(1-3/8B)^p + b = (\alpha+\beta+\beta p+3/2)B$.  Letting $c_1 = \frac{\ell_{q}}{ \ell_{s}}=\bigg(\frac{\alpha+\beta+\beta p+3/2}{\alpha+\beta+1}\bigg)$, the social optimum flow in $H$ is a $c_1$-UNE flow.

We next consider a different path flow for the EF setting. In this path flow:
\begin{enumerate}
	\item Path $s' = P_{[n]}-e_{d, (n+1)}$ carries $\frac{3}{8B}$ flow.
	\item Path $p = P_{I}-e_{u, (n+1)}$ carries $(1/2 - 3/8B)$ flow.
	\item Path $q = P_{I^c}-e_{u, (n+1)}$ carries $(1/2 - 3/8B)$ flow.
	\item Path $r'= P_{\emptyset}-e_{u, (n+1)}$ carries $\frac{3}{8B}$ flow.
\end{enumerate}
We claim that path $s'$ is the shortest path if $\beta p>1$ as
$$
\ell_{s'} = 2B + c(3/8B)^p = (\alpha+\beta+2)B <
(\alpha + \beta + \beta p + 1)B = B + a\bigg(1-\frac{3}{8B}\bigg)^p + b = \ell_{r'}<\ell_p=\ell_q.
$$
In this setting, the minimum ratio of longest `used' path and shortest `used' path is  $c_2 = \frac{\ell_{q}}{ \ell_{s'}}=\bigg(\frac{\alpha+\beta+\beta p+3/2}{\alpha+\beta+2}\bigg)$ and the socially optimal flow is a $c_2$-EF flow.

Next, we need to show that if the answer to PARTITION is NO then the socially optimal flow is neither a $c_1$-UNE flow nor a $c_2$-EF flow. For this we need to ensure that for all possible path flows under the social optimum, there exists at least one used path which is obtained by concatenating a `long' positive subpath in $G$ with the upper edge in $G'$. The following claim lower bounds the flow through the longest path in $G$ for any valid path decomposition. 

\begin{claim}\label{lemm:flowlower}
If the answer to PARTITION is NO then in the sub-network $G$ any path decomposition of the socially optimal flow $o$ routes at least $\frac{1}{2B}$ amount of flow through paths of length strictly greater than $\frac{3}{2}B$. 
\end{claim}
\begin{proof}
Recall that if the given instance for the PARTITION problem is a NO instance then there is no path under $o$ which has length exactly $3B/2$.
Fix any path decomposition for $o$ and let $\delta$ be the flow passing through the paths of length strictly greater than $\frac{3}{2}B$.   Also let $\ell$ be the maximum length among the set of paths strictly smaller than $\frac{3}{2}B$. As $q_i$'s are integers  and the given instance of PARTITION is a NO instance, it is easy to observe that $\ell \leq \frac{3}{2}B-\frac{1}{2}$. Also $\ell\geq B$. Moreover, if we route $(1-\delta)$ flow through a path of length $\ell$ and  $\delta$ flow through the path of maximum length $2B$, then the cost of this routing is greater or equal to the socially optimal cost. This implies,
%\begin{align*}
$$\ell(1-\delta)+2\delta B \geq \frac{3}{2}B  \implies
\delta \geq \frac{3B/2-\ell}{2B-\ell} \geq \frac{3B/2-3B/2+1/2}{2B-B} \geq \frac{1}{2B}.$$
%\end{align*}  
\end{proof}   

From the above claim we see that the longest used path $q$ has length strictly greater than $\ell_{q}$ as the bottom link under $o$ has flow $3/8B < 1/2B$. The shortest path in the network has length $\ell_{s}$ as in the YES case. If the PARTITION instance is a NO instance, the optimal flow $o$ is not a $c_1$-UNE. Moreover, for the EF flow the best path flow again contains the path $s'$ as the shortest path but now the longest path is strictly greater than $\ell_{q}$. So it is not a $c_2$-EF flow. 

All that is left to show is that there are appropriate values of $\alpha$ and $\beta$ which make $c_1 = \theta$ or $c_2 = \theta'$, for any $\theta\in (1, p+1)$, and for any $\theta'\in (1, p+1)$.  This can be shown by observing that:
\begin{align*}
	c_1=\frac{\alpha+\beta+\beta p + \frac{3}{2}}{1+\alpha+\beta}&=1+\frac{\frac{1}{2}+\beta p}{1+\alpha+\beta} & c_2=\frac{\alpha+\beta+\beta p + \frac{3}{2}}{2+\alpha+\beta}&=1+\frac{-\frac{1}{2}+\beta p}{2+\alpha+\beta}
\end{align*}   
Combining this with what we have shown in Lemma~\ref{lemm:corehardness} for $1$-EF flows completes the proof.
\end{proof}
\begin{proof}[Proof, second part of Theorem~\ref{thm:main_hardness}]
The proof follows by constructing a reduction from the decision problems specified in Theorem~\ref{thm:12Hard} and recalling that $\gamma(\mathcal{L}_p)=p+1$.  The answer to each of the decision problem in Theorem~\ref{thm:12Hard} is YES if and only if the solution to Problem~(P1) or (P2) is a social optimum, the cost of which is known in advance, by construction.
\end{proof}




%\begin{remark}
%	The proof is inspired from the proof of the NP hardness of length bounded flow problem (Theorem~3.3) in Correa et al.~\cite{}. But the conclusions apply to the completely new setting of UNE and EF flow and we generalize it through novel constructions. 
%\end{remark}

For Problem~(P3), the proof technique in Theorem~\ref{thm:main_hardness} does not go through.  In fact, we show that the relevant decision problem related to Problem~(P3) is in P: % We consider the following problem:
\begin{enumerate}
\item[(P3')] Is there a socially optimal flow which is a $\theta$-PNE?
\end{enumerate}

To show that (P3') is in P, we first define an edge flow $\bm{x}$ to be \emph{acyclic} if for each commodity $k$, the subgraph $G_k$, induced by the edges $E_k(\bm{x}) = \{e: e\in E, x_e^k>0\} $ is a directed acyclic graph (DAG).
  
\begin{claim}
Given an instance of a multicommodity flow network $\mc{G}$ with standard latency functions, we can decide whether an `acyclic' edge flow $\bm{x}$ is in $\theta$-PNE in polynomial time.
\label{clm:Acyclic}
\end{claim}
\begin{proof}
We present the polynomial time algorithm which decides whether an `acylic' edge flow $\bm{x}$ is a $\theta$-PNE or not for some given $\theta$. For each commodity $k$ in $\mc{G}$, we construct the DAG induced by $E_k(\bm{x})$. Next, under the edge weights $w_e = \ell_e(x_e)$, we compute the costs of the shortest $(s_k, t_k)$ path in $G$ (call it $\ell_1$) and the longest $(s_k,t_k)$ path in $G_k$ (call it $\ell_2$).  Recall that shortest path computation and longest path computation in a DAG can both be done in polynomial time. Finally, we accept if $\ell_2 \leq \theta\ell_1$ and reject otherwise. 
\end{proof}

\begin{lemma}
Problem~(P3') can be solved in polynomial time.
\label{lemma:3Easy}
\end{lemma}
\begin{proof} 
We first claim that for any $k$, the set of edges that carry flow for commodity $k$ at the social optimum, $E_k(\bm{x}^*)$, has no positive loops.  This can be shown by contradiction.  Assume there is a positive loop in $E_k(\bm{x}^*)$, then, we can construct a new flow $\bm{x}'$ by removing some $\epsilon>0$ flow on the loop.  The flow $\bm{x}'$ can be kept feasible, and it has strictly smaller social cost due to the monotonicity and non-negativity of the latency functions, which contradicts the fact that $\bm{x}$ is the socially optimal flow.  Also, if there is a zero cost loop in $E_k(\bm{x}^*)$, we can safely remove the flow on that loop without changing the social cost. Therefore, the procedure in Lemma~\ref{clm:Acyclic} completes the proof. 
\end{proof}

%%----------------------------------------------------------------------------------------------------
\section{Designing a Watermark}
\label{sec:taxonomy}

After the quick overview of watermarking schemes in \cref{sec:background}, we now provide more details 
about the watermarking design space. We created a unifying taxonomy under which all previous schemes 
can be expressed. We first discuss the requirements then the building blocks of a text watermark. 
%
%We provide a modular implementation of all schemes, so any of the building blocks can be combined.
%
\cref{fig:design-figure} summarizes the current design space.

\subsection{Requirements}

A useful watermarking scheme must detect watermarked texts, without falsely flagging human-generated text and without impairing the original model's performance.
%
More precisely, we want watermarks to have the following properties.
% \begin{itemize}[leftmargin=\itemlm,itemsep=2pt]
\begin{enumerate}[leftmargin=\itemlm,itemsep=2pt]
    \item \textbf{High Recall}. $\Pr[\mathcal{V}_k(T) = \texttt{True}]$ is large if $T$ is a watermarked text generated using the marking procedure $\mathcal{W}$ and secret key $k$.
    %
    \item \textbf{High Precision}. For a random key $k$, $\Pr[\mathcal{V}_k(\Tilde{T}) = \texttt{False}]$ is large if $\Tilde{T}$ is a human-generated (\emph{non-watermarked}) text.
    %
    \item \textbf{Quality}. The watermarked model should perform similarly to the original model. 
    It should be useful for the same tasks and generate similar quality text.
    %
    \item \textbf{Robustness}. A good watermark should be robust to small changes to the watermarked text (potentially caused by an adversary), 
    meaning if a sample $T$ is watermarked with key $k$, then for any text $\Tilde{T}$ that is semantically close to $T$, $\mathcal{V}_k(\Tilde{T})$ should evaluate to \text{True}.
\end{enumerate}

\noindent
A desireable (but optional) property for watermarks is diversity. 
In some settings, such as creative tasks like story-telling, users might want the model to have the ability to generate 
multiple different outputs in response to the same prompt (so they can select their favorite).
We would like watermarked outputs to preserve this capability.
% \noindent
% In addition to these properties, another desirable property for a watermark is to 
% preserve a model's diversity. Language models tend to have diverse generated text distributions: 
% they are able to generate different responses to a same prompt. This is useful in many settings, 
% such as creative tasks like story telling, so the user can  their favorite output.

% The notion of \emph{undetectability} has been defined in previous work~\citep{christ_undetectable_2023}:
Another useful property is \emph{undetectability}, also called \emph{indistinguishability}:
%
no feasible adversary should be able to distinguish watermarked text from non-watermarked text, without knowledge of the secret key~\citep{christ_undetectable_2023}. 
%
A watermark is considered undetectable if the maximum advantage at distinguishing is very small.
%
This notion is appealing; for instance, undetectability implies that watermarking does not degrade the model's quality.
%
However, we find in practice that undetectability is not necessary and may be overly restrictive:
%
minor changes to the model's output distribution are not always detrimental to its quality.

In this paper we focus on symmetric-key watermarking, where both the watermarking and verification procedures share a secret key.
%
This is most suitable for proprietary language models that served via an API.
%
We imagine that the vendor would watermark all outputs, and also provide a second API to query the verification procedure.
%
Alternatively, one could publish the key, enabling anyone to run the verification procedure.
%
\begin{figure*}
    \begin{center}
    \begin{tikzpicture}
    
    \draw[draw=black] (0,15) rectangle ++(17.5,1) node[pos=0.5, align=center] {\Large{Watermarking Taxonomy}};
    \draw[draw=black] (0,12.75) rectangle ++(8.375,2) node[pos=0.5, align=left] 
    {\\
    \\
    \textbf{Parameters:} Key $k$, Sampling $\mathcal{C}$, Randomness $\mathcal{R}$\\
    \textbf{Inputs:} Probs $\mathcal{D}_n = \{\lambda^n_1,\, \cdots, \lambda^n_d\}$, Tokens $\{T_i\}_{i < n}$\\
    \textbf{Output:} Next token 
    $T_n \leftarrow \mathcal{C}(\mathcal{R}_k( \{T_i\}_{i < n}), \mathcal{D}_n)$};
    \draw[draw=black] (9.125,12.75) rectangle ++(8.375,2) node[pos=0.5, align=left] 
    {\\
    \\
    \textbf{Parameters:} Key $k$, Score $\mathcal{S}$, Threshold $p$\\
    \textbf{Inputs:} Text $T$\\
    \textbf{Output:} Decision $\mathcal{V} \leftarrow \text{P}_{0}\left( \mathcal{S} < \mathcal{S}_k(T)\right) < p$};
    \draw (8.75,13.75) circle (0.25) node {+};
    \draw[draw=none] (0,14.25) rectangle ++(8.375,.5) node[pos=0.5, align=left] {\large{Marking $\mathcal{W}$}};
    \draw[draw=none] (9.125,14.25) rectangle ++(8.375,.5) node[pos=0.5, align=left] {\large{Verification $\mathcal{V}$}};
    
    %%%
    
    \draw[draw=black,dashed] (0,8.75) rectangle ++(17.5,3.75);
    \draw[draw=none] (0,11) rectangle ++(17.5,1.75) node[pos=0.5, align=center] {\large{Randomness Source $\mathcal{R}$}\\
    \textbf{Inputs:} Tokens $\{T_i\}_{i < n}$\,
    \textbf{Output:} Random value $r_n = \mathcal{R}_k(\{T_i\}_{i < n})$};
    \draw[draw=black] (0.25,9) rectangle ++(11.25,2.35) node[pos=0, anchor=south west] {\textbf{Text-dependent.} Hash function $h$. Context length H};
    \draw[draw=black] (0.5,9.6) rectangle ++(10.75,0.625) node[anchor=north west] at (0.5, 10.225) {\textbf{(R2) Min Hash}} node[pos=1, anchor=north east, align=left] {
    $r_n = \text{min} \left( h\left( T_{n-1} \mathbin\Vert k\right), \, \cdots, h\left( T_{n-H} \mathbin\Vert k\right) \right)$\\
    };
    \draw[draw=black] (0.5,10.475) rectangle ++(10.75,0.625) node[anchor=north west] at (0.5, 11.1) {\textbf{(R1) Sliding Window}} node[pos=1, anchor=north east, align=left] {
    $r_n = h\left( T_{n-1} \mathbin\Vert \, \cdots \mathbin\Vert T_{n-H} \mathbin\Vert k\right)$\\
    };
    \draw[draw=black] (11.75,9) rectangle ++(5.5,2.35) node[pos=0, anchor=south west] {\textbf{(R3) Fixed}} node[pos=0.5, align=left] {Key length L. Expand $k$ to\\ pseudo-random sequence $\{r^k_i\}_{i<L}$.\\ 
    $r_n = r^k_{n \text{ (mod L)}}$ \\ \\ };
    
    %%%
    
    \draw[draw=black,dashed] (0,3.25) rectangle ++(17.5,5.25);
    \draw[draw=none] (0,6.85) rectangle ++(17.5,1.75) node[pos=0.5, align=center] {\large{Sampling algorithm $\mathcal{C}$ \& Per-token statistic $s$}\\
    \textbf{Inputs:} Random value $r_n = \mathcal{R}_k( \{T_i\}_{i < n})$, Probabilities $\mathcal{D}_n = \{\lambda^n_1,\, \cdots, \lambda^n_d\}$, Logits $\mathcal{L}_n = \{l^n_1,\,\cdots,l^n_d\}$\\};
    
    %
    
    \draw[draw=black] (11,4.75) rectangle ++(6.25,2.5) node[pos=0, anchor=south west] {\textbf{(C3) Binary}} node[pos=0.5, align=left] {Binary alphabet.\\ 
    $T_n \leftarrow 0$ if $r_n < \lambda^n_0$, else $1$. \\
    $s(T_n, r) = \begin{cases} -\log(r) \text{ if } T_n = 1\\
          -\log(1-r) \text{ if } T_n = 0\\\end{cases} $};
    
    \draw[draw=black] (5,4.75) rectangle ++(5.75,2.5) node[pos=0, anchor=south west] {\textbf{(C2) Inverse Transform}} node[pos=0.5, align=left] 
    {$\pi$ keyed permutation. $\eta$ scaling func.\\
    $T_n \leftarrow \pi_k \left( \min\limits_{ j \leq d } \sum\limits_{i=1}^j \lambda^n_{\pi_k (i)} \geq r_n \right)$ \\
    $s(T_n, r) = | r - \eta \left( \pi^{-1}_k(T_n) \right) | $\\};
    
    \draw[draw=black] (0.25,4.75) rectangle ++(4.5,2.5) node[pos=0, anchor=south west] {\textbf{(C1) Exponential}} node[pos=0.5, align=left] 
    {$h$ keyed hash function. \\
    $T_n \leftarrow \argmax\limits_{i \leq d} \left\{ \frac{\log \left( h_{r_n}\left( i \right) \right)}{\lambda^n_i} \right\}$ \\
    $s(T_n, r) = -\log(1 \! - \! h_r(T_n))$\\};
    
    % 
    
    \draw[draw=black] (0.25,3.5) rectangle ++(17,1) node[pos=0, anchor=south west] {\textbf{(C4) Distribution-shift}} node[pos=0.5, align=right] {Bias $\delta$, Greenlist size $\gamma$. Keyed permutation $\pi$. $T_n$ sampled from $\widetilde{\mathcal{L}}_n = \{l^n_i + \delta \text{ if } \pi_{r_n}(i) < \gamma d \text{ else } l^n_i\, , 1 \leq i \leq d\}$\\
    $s(T_n, r) = 1 \text{ if } \pi_{r}(T_n) < \gamma d \text{ else } 0$};
    
    %%% 
    
    \draw[draw=black,dashed] (0,0) rectangle ++(17.5,3);
    \draw[draw=none] (0,1.75) rectangle ++(17.5,1.25) node[pos=0.5, align=center] {\large{Score $\mathcal{S}$}\\
    \textbf{Inputs:} Per-token statistics $s_{i,j} = s(T_i, r_j)$, where $r_j = \mathcal{R}_k( \{T_l\}_{l < j}))$. \# Tokens $N$.};
    
    % 
    
    \draw[draw=black] (8.15,0.25) rectangle ++(9.1,1.5) node[pos=0, anchor=south west] {\textbf{(S3) Edit Score}}
    
    node[pos=0.5, align=left] {
    $\mathcal{S}_{\text{edit}}^\psi = s^\psi(N,N)$,
    $
        s^\psi (i,j) = \min \begin{cases}
          s^\psi (i-1, j-1) + s_{i,j}\\
          s^\psi (i-1, j) + \psi\\
          s^\psi (i, j-1) + \psi\\
        \end{cases} 
    $};
    \draw[draw=black] (0.25,0.25) rectangle ++(2.6,1.5) node[pos=0, anchor=south west] {\textbf{(S1) Sum Score}} node[pos=0.5, align=left] {$\mathcal{S}_{\text{sum}}\! = \! \sum_{i=1}^N s_{i,i}$ \\};
    \draw[draw=black] (3.1,0.25) rectangle ++(4.8,1.5) node[pos=0, anchor=south west] {\textbf{(S2) Align Score}} node[pos=0.5, align=left] {$\mathcal{S}_{\text{align}} \!= \!\min\limits_{0 \leq j < N} \sum\limits_{i=1}^N s_{i, (i+j) \text{ mod}(N)}$ \\ \\ };
    
    \end{tikzpicture}
    \caption{Watermarking design blocks. There are three main components: randomness source, sampling algorithm (and associated per-token statistics), and score function. Each solid box within each of these three components (dashed) denotes a design choice. The choice for each component is independent and offers different trade-offs.}\label{fig:design-figure}
    \end{center}
    \end{figure*}

\subsection{Watermark Design Space}
\label{sec:watermark-design}

Designing a good watermark is a balancing act.
% 
For instance, replacing every word of the output with [WATERMARK] would achieve high recall but destroy the utility of the model.
%
%Conversely, sampling from the original distribution preserves quality but makes it impossible to watermark. 

Existing proposals have cleverly crafted marking procedures that are meant to preserve quality, provide high precision and recall, and achieve a degree of robustness.
%
Despite their apparent differences, we realized they can all be expressed within a unified framework:

\begin{itemize}[leftmargin=\itemlm,itemsep=2pt]
    \item The marking procedure $\mathcal{W}$ contains a randomness source $\mathcal{R}$ and a sampling algorithm $\mathcal{C}$.
    %
    The randomness source $\mathcal{R}$ produces a (pseudo-random) value $r_n$ for each new token, based on the secret key $k$ and the previous tokens $T_0,\cdots,T_{n-1}$.
    %
    The sampling algorithm $\mathcal{C}$ uses $r_n$ and the model's next token distribution $\mathcal{D}$ to  a token.
    \item The verification procedure $\mathcal{V}$ is a one-tailed significance test that computes a $p$-value for the null hypothesis that the text is not watermarked.
    %
    The procedure compares this $p$-value to a threshold, which enables control over the watermark's precision and recall.
    %
    % This test is done using a \emph{score function} $\mathcal{S}$ based on a per-token variable that depends on the ed sampling algorithm.
    % We call the value of this per-token test statistic $s_n$, which only depends on the random value $r_n$ and the ed token $T_n$: $s_n = s(T_n, r_n)$.
    In particular, we compute a per-token score $s_{n,m} \coloneqq s(T_n, r_m)$ for each token $T_n$ and randomness $r_m$, aggregate them to obtain an overall score $\mathcal{S}$, and compute a $p$-value from this score.
    We consider all overlaps $s_{n,m}$ instead of only $s_{n,n}$ to support scores that consider misaligned randomness and text after perturbation. 
    %the test computes \emph{score function} $\mathcal{S}$ which takes as input per-token test statistics $s_{n,m} \coloneqq s(T_n, r_m)$ for a token $T_n$ and a random value $r_m$, $\forall n,m \in [N]$.
    %
    %$s_{n,m}$ depends on the sampling algorithm (see \cref{fig:design-figure} for examples).
    %
    % \dave{I believe $s(T_n, r_m)$ is incorrect and it should be $s(T_n, r_n)$.  Also I think the score should be $s_n$ rather than $s_{n,m}$.}
    % \jp{Depending on the alignment between the key string and the text, there are times we want to refer to the score for key at position m and token at poistion n (for instance, for both the align and edit scores). I'll add some explanation for this.}
    
\end{itemize}
% \dave{I find the sheer number of fonts inelegant (blackboard bold, mathcal, mathbf, typewritter, italics, bold, etc.). In some places, algorithms are denoted by mathcal (W,V), in other places by mathbf (R,C,S).  I suggest picking one and being consistent.  I prefer mathcal.  Lots of bold feels distracting to my eyes, as does lots of font changes.}
% \jp{I changed a bunch of fonts to make it more consistent, and removed bold fonts}

Next, we show how each scheme we consider falls within this framework, each with its own choices for $\mathcal{R},\mathcal{C},\mathcal{S}$.
%Given this template, previous work introduced their own variants of the building blocks, which we will now detail. 
% \chawin{I would have liked to see a summary of which design choices belong to which paper. Maybe we can add a shorthand notation denoting each paper in \cref{fig:design-figure} or have a separate table.}
% \jp{I agree that's a good idea. A table is probably the right way to represent this.}

\subsubsection{Randomness source $\mathcal{R}$}\label{ssec:randomness}
% \textbf{Randomness source $\mathcal{R}$.}
%
% \chawin{Maybe others?} \jp{Yeah but all the other papers i've seen seem to attribute it to one of these two.}
We distinguish two main ways of generating the random values $r_n$, \emph{text-dependent} (computed as a deterministic function of the prior tokens) vs \emph{fixed} (computed as a function of the token index).
Both approaches use the standard heuristic of applying a keyed function (typically, a PRF) to some data, to produce pseudorandom values that can be treated as effectively random but can also be reproduced by the verification procedure.

\citet{aaronson_watermarking_2022} and \citet{kirchenbauer_watermark_2023}
use text-dependent randomness: $r_n = f\left(T_0,\,\cdots,T_{n-1},k\right)$.
%
This scheme has two parameters: the length of the token context window (which we call the window size H) and the aggregation function $f$.
%
\citet{aaronson_watermarking_2022} proposed using the hash of the concatenation of previous tokens, $f := h\left( T_{n-1} \mathbin\Vert \, \cdots \mathbin\Vert T_{n-H} \mathbin\Vert k\right)$; we call this (R1) sliding window.
%
\citet{kirchenbauer_watermark_2023} used this with a window size of $ H = 1$ and also introduced an alternate aggregation function $f := \text{min} \left( h\left( T_{n-1} \mathbin\Vert k\right), \, \cdots, h\left( T_{n-H} \mathbin\Vert k\right) \right)$.
%
We call this last aggregation function (R2) min hash.
%
While these two schemes propose specific choices of $H$, other values are possible. 
We use \benchmarkname{} to evaluate a range of values of $H$ with each candidate aggregation function.

% \smallskip\noindent\textbf{(R3) Fixed}
\citet{kuditipudi_robust_2023} use fixed randomness:
$r_n = f_k(n)$, where $n$ is the index (position) of the token.
We call this (R3) fixed.
%
In practice, they propose using a fixed string of length $L$ (the key length), which is repeated across the generation.
% r_n = f_k(n \bmod L)$ where $L$ is the key length.
% \dave{I don't think we need this level of detail.  I suggest deleting the preceding sentence.}
% \jp{Since we look at the impact of the key length on generations we still need to introduce the idea that the key is repeated, but I canwrite that in english for it to be more digestable}
%
We test the choice of key length in ~\cref{ssec:param_tuning}
%
In the extreme case where $L=1$ or $H=0$, both sources are identical, as $r_n$ will be the same value for every token. \citet{zhao2023provable} explored this option using the same sampling algorithm as~\citet{kirchenbauer_watermark_2023}.

\label{ssec:binary}
\citet{christ_undetectable_2023} proposed setting a target entropy for the context window instead of fixing a window size.
%
This allows to set a lower bound on the security parameter for the model's undetectability.
%
However, setting a fixed entropy makes for a less efficient detector since all context window lengths must be tried in order to detect a watermark.
%
Furthermore, in practice, provable undetectability is not needed to achieve optimal quality: we chose to keep using a fixed-size window for increased efficiency.

\subsubsection{Sampling algorithm \(\mathcal{C}\)}\label{ssec:sampling}
% \textbf{sampling algorithm $\mathcal{C}$.}
%
\noindent
We now give more details about the four sampling algorithms initially presented in~\cref{tab:marking-algorithms}.

\smallskip\noindent\textbf{(C1) Exponential}.
%
Introduced by \citet{aaronson_watermarking_2022} and also used by \citet{kuditipudi_robust_2023}. It relies on the Gumbel-max trick.
%
Let $\mathcal{D}_n = \left\{\lambda^n_i\,, 1 \leq i \leq d\right\}$ be the distribution of the language model over the next token. %(obtained after passing the logits through a softmax and applying a temperature adjustment).
%
The exponential scheme will select the next token as:
\begin{align}
    T_{n} = \argmax\limits_{i \leq d}\left\{ \frac{\log \left( h_{r_n}\left( i \right) \right)}{\lambda^n_i} \right\}
\end{align}
where $h$ is a keyed hash function using $r_n$ as its key.
%
The per-token variable used in the statistical test is either $s_n = h_{r_n}(T_n)$ or $s_n = -\log \left( 1-h_{r_n}(T_n)\right)$.
%
\citet{aaronson_watermarking_2022} and \citet{kuditipudi_robust_2023} both use the latter quantity.
%
We argue the first variable provides the same results, and unlike the log-based variable, the distribution of watermarked variables can be expressed analytically (see~\cref{app:ssec:pseudorandom-proofs} for more details).
%
We align with previous work and use the $\log$ for \benchmarkname{}.

\smallskip\noindent\textbf{(C2) Inverse transform}.
%
\citet{kuditipudi_robust_2023} introduce inverse transform sampling.
%
They derive a random permutation using the secret key $\pi_k$. The next token is selected as follows:
\begin{align}
    T_{n} = \pi_k \left( \min\limits_{ j \leq d } \sum\limits_{i=1}^j \lambda^n_{\pi_k (i)} \geq r_n \right)
\end{align}
which is the smallest index in the inverse permutation such that the CDF of the next token distribution is at least $r_n$.
%
\citet{kuditipudi_robust_2023} propose to use $s_n = | r_n - \eta \left( \pi^{-1}_k(T_n) \right) |$ as a the test variable, where $\eta$ normalizes the token index to the $[0,1]$ range.
%
% We call this scheme the \textit{inverse transform} scheme.

\smallskip\noindent\textbf{(C3) Binary}.
%
\citet{christ_undetectable_2023} propose a different sampling scheme for binary token alphabets --- however, it can be applied to any model by using a bit encoding of the tokens.
%
In our implementation, we rely on a Huffman encoding of the token set, using frequencies derived from a large corpus of natural text.
%
In this case, the distribution over the next token reduces to a single probability $p_n$ that token ``0'' is ed next, and $1-p$ that ``1'' is ed.
%
The sampling rule s 0 if $r_n < p$, and 1 otherwise. The test variable for this case is $s_n = -\log \left( T_n r_n + (1-T_n) (1-r_n) \right)$.
%
% We call this scheme the \textit{binary} scheme.
%
% At first glance, it can seem like this scheme is identical to the exponential scheme. However, because it uses a binary alphabet, the distribution of the test variable is different for both schemes.
%
% However, we show in Appendix \jp{ref} that this is not the case: the distribution of the test variable is different for both schemes.
% %
% \jp{Maybe I'll remove this if I don't have time to show it.}

\smallskip\noindent\textbf{(C4) Distribution-shift}.
%
\citet{kirchenbauer_watermark_2023} propose the distribution-shift scheme. 
%
It produces a modified distribution $D_n$ from which the next token is sampled.
%
Let $\delta > 0$ and $\gamma \in [0,1]$ be two system parameters, and $d$ be the number of tokens.
%
The scheme constructs a permutation $\pi_{r_n}$, seeded by the random value $r_n$, which is used to define a ``green list,'' containing tokens $T$ such that $\pi_{r_n} (T) < \delta d$. It then adds $\delta$ to green-list logits.
%
This modified distribution is then used by the model to sample the next token. The test variable $s_n$ is a bit equal to ``1'' if $T_n$ is in the green list defined by $\pi_{r_n}$, and ``0'' if not.
%
% We call this scheme the \textit{distribution-shift} scheme.

The advantage of this last scheme over the others is that it preserves the model's diversity: 
for a given key, the model will still generate diverse outputs.
In contrast, for a given secret key and a given prompt, the first three sampling strategies 
will always produce the same result, since the randomness value $r_n$ will be the same.
\citet{kuditipudi_robust_2023} tackles this by randomly offseting the key sequence of 
fixed randomness for each generation. We introcude a skip probability $p$ for the 
same effect on text-dependent randomness. Each token is selected without the marking 
strategy with probability $p$. In the interest of space, we leave a detailed discussion 
of generation diversity in~\cref{app:ssec:diverse}.

Another advantage of the distribution-shift scheme is that it can also be used 
at any temperature, by applying the temperature scaling \emph{after} using the 
scheme to modify the logits. Other models apply temperature before watermarking.

However the distribution-shift scheme is not indistinguishable from the original model, 
as discussed earlier in~\cref{ssec:watermark-design}.

\subsubsection{Score Function $\mathcal{S}$}\label{ssec:score}

% \paragraph{Verification procedure $\mathcal{V}.$}

% The distribution of the per-token test statistic is different for watermarked text and non-watermarked text: this is what makes detection possible. Depending on the scheme, it is either higher or lower on average in the watermarked case. Without loss of generality, we assume it is always lower for this discussion.

To determine whether an $N$-token text is watermarked, we compute a score over per-token statistics.
%
This score is then subject to a one-tailed statistical test where the null hypothesis is that the text is not watermarked.
%
In other words, if its $p$-value is under a fixed threshold, the text is watermarked.
%
Different works propose different scores.

\smallskip\noindent\textbf{(S1) Sum score}.
%
\citet{aaronson_watermarking_2022} and \citet{kirchenbauer_watermark_2023} take the sum of all individual per-token statistics:
\begin{align}
    \mathcal{S}_{\text{sum}}=\sum_{i=1}^N s_i = \sum_{i=1}^N s(T_i, r_i).
\end{align}
%
This score requires the random values $r_i$ and the tokens $T_i$ to be aligned.
%
% \chawin{Maybe this goes into limitation or discussion or appendix}
This is not a problem when using text-dependent randomness, since the random values are directly obtained from the tokens.
%
However, this score is not suited for fixed randomness: removing one token at the start of the text will offset the values of $r_i$ for the rest of the text and remove the watermark.
%
The use of the randomness shift to increase diversity will have the same effect. 

\smallskip\noindent\textbf{(S2) Alignment score}.
Proposed by \citet{kuditipudi_robust_2023}, the alignment score aims to mitigate the misalignment issue mentioned earlier.
% \citet{kuditipudi_robust_2023} proposes two alternative scores to deal with this issue.
%
% In keeping with their work, we name these scores the alignment score and the edit score.
Given the sequence of random values $r_i$ and the sequence of tokens $T_i$, the verification process now computes different versions of the per-token test statistic for each possible overlap of both sequences $s_{i,j} = s(T_i, r_j)$.
%
The alignment score is defined as:
\begin{align}
   \mathcal{S}_{\text{align}}  = \min\limits_{0 \leq j < N} \sum\limits_{i=1}^N s_{i, (i+j) \text{ mod}(N)}
\end{align}

\smallskip\noindent\textbf{(S3) Edit score}.
Similar to the alignment score, \citet{kuditipudi_robust_2023} propose the edit score as an alternative for dealing with the misalignment issue.
%
It comes with an additional parameter $\psi$ and is defined as $\mathcal{S}_{\text{edit}}^\psi = s^\psi(N,N)$, where
\begin{align}
    s^\psi (i,j) &= \min \begin{cases}
      s^\psi (i-1, j-1) + s_{i,j}\\
      s^\psi (i-1, j) + \psi\\
      s^\psi (i, j-1) + \psi\\
    \end{cases} 
\end{align}

In all three cases, the average value of the score for watermarked text will be lower than for non-watermarked text.
%
% In the case of the sum score, we can often derive the distribution of the score under the null hypothesis, allowing us to use a $z$-test to determine if the text is watermarked.
In the case of the sum score, the previous works use the $z$-test on the score to determine whether the text is watermarked, but it is also possible, or even better in certain situations, to use a different statistical test according to \citet{fernandez_three_2023}.
%
When possible, we derive the exact distribution of the scores under the null hypothesis (see \cref{app:ssec:exact_dist}) which is more precise than the $z$-test. When it is not, we rely on an empirical T-test, as proposed by \citet{kuditipudi_robust_2023}
%
% This allows one to compute 

\subsection{Limitations of the Building Blocks}\label{ssec:limit_blocks}

While we design the blocks to be as independent as possible, some combinations of the scheme and specific parameters are obviously sub-optimal.
%
Here, we list a few of these subpar block combinations as a guide for practitioners.
% Even though any of the three scores can be used with any scheme and randomness source, in practice not all combinations are useful.
\begin{itemize}[leftmargin=\itemlm,itemsep=2pt]
    \item The sum score (S1) is not robust for fixed randomness (R3).
    \item The alignment score (S2) does not make sense for the text-dependent randomness (R1, R2) since misalignment is not an issue.
    \item The edit score (S3) has a robustness benefit since it can support local misalignment caused by token insertion, deletion, or swapping. However, using it with text-dependent randomness (R1, R2) only makes sense for a window size of 1: for longer window sizes, swapping, adding, or removing tokens would actually change the random values themselves, and not just misalign them.
    \item Finally, in the corner case when a window size of 0 for the text-dependent randomness (R1, R2) or when a random sequence length of 1 for the fixed randomness (R3), both the alignment score (S2) and the edit score (S3) are unnecessary since all random values are the same and misalignment is not possible.
\end{itemize}

In our experiments (\cref{sec:experiments}), we test all reasonable configurations of the randomness source, 
the sampling protocol, and the verification score, along with their parameters. 
We list the evaluated combinations in~\cref{tab:design_space_combinations}. 
The edit score is too inefficient 
to be run on all configurations, instead we rely on the sum and align scores.
%
We hope to not only fairly compare the prior works but also investigate previously unexplored combinations in the 
design space that can produce a better result.

% \chawin{We need a table or a tree that lists all the combinations we test.}\

\begin{table}[h!]
    \centering
    \caption{Tested combinations in the design space, using notations from~\cref{fig:design-figure}.\\
    We only tested the edit score {\bf S3} on a subset of watermarks.\\
    The distribution of non-watermarked scores is known for \textcolor{orange}{orange} configurations and 
    unknown for \textcolor{blue}{blue} configuration. We rely on empirical T-tests~\cite{kuditipudi_robust_2023} for blue configurations.
    }
    \label{tab:design_space_combinations}
    \normalsize
    \begin{tabular}{|l||c|c|c|c|} 
    \hline
     & \makecell[tc]{{\bf C4}\\{\small Distribution}\\{\small Shift}} & \makecell[tc]{{\bf C1}\\{\small Exponential}} & \makecell[tc]{{\bf C2}\\{\small Binary}} & \makecell[tc]{{\bf C3}\\{\small Inverse}\\{\small Transform}} \\
    \hline
    \hline
    \makecell{{\bf S1}+{\bf R1}}  & \textcolor{orange}{X} & \textcolor{orange}{X} & \textcolor{orange}{X} & \textcolor{blue}{X} \\
    \hline
    \makecell{{\bf S1}+{\bf R2}}  & \textcolor{orange}{X} & \textcolor{orange}{X} & \textcolor{orange}{X} & \textcolor{blue}{X} \\
    \hline
    \makecell{{\bf S2}+{\bf R3}}  & \textcolor{blue}{X} & \textcolor{blue}{X} & \textcolor{blue}{X} & \textcolor{blue}{X} \\
    \hline
    \makecell{{\bf S3}+{\bf R3}}  & \textcolor{blue}{X} &  &  &  \\
    \hline
    \end{tabular}
\end{table}
    

\subsection{Analysis of the edit score.} 
\label{ssec:editscore}
We analyzed the tamper-resistance of the edit score on a subset of watermarks 
(distribution-shift with $\delta=2.5$ at a temperature of 1, for key lengths between 1 and 1024). 
We tried various $\psi$ values between 0 and 1 for the edit distance, and compared the tamper-resistance 
and watermark size of the resulting verification procedures to the align score. 
Using an edit distance does improve tamper-resistance for key lengths under 32, but at a large efficiency cost: 
for key lengths above 8, the edit score size is at least twice that of the align score. 
We do not recommend using an edit score on low entropy models such as Llama-2 chat.



%%----------------------------------------------------------------------------------------------------
% \vspace{-0.5em}
\section{Conclusion}
% \vspace{-0.5em}
Recent advances in multimodal single-cell technology have enabled the simultaneous profiling of the transcriptome alongside other cellular modalities, leading to an increase in the availability of multimodal single-cell data. In this paper, we present \method{}, a multimodal transformer model for single-cell surface protein abundance from gene expression measurements. We combined the data with prior biological interaction knowledge from the STRING database into a richly connected heterogeneous graph and leveraged the transformer architectures to learn an accurate mapping between gene expression and surface protein abundance. Remarkably, \method{} achieves superior and more stable performance than other baselines on both 2021 and 2022 NeurIPS single-cell datasets.

\noindent\textbf{Future Work.}
% Our work is an extension of the model we implemented in the NeurIPS 2022 competition. 
Our framework of multimodal transformers with the cross-modality heterogeneous graph goes far beyond the specific downstream task of modality prediction, and there are lots of potentials to be further explored. Our graph contains three types of nodes. While the cell embeddings are used for predictions, the remaining protein embeddings and gene embeddings may be further interpreted for other tasks. The similarities between proteins may show data-specific protein-protein relationships, while the attention matrix of the gene transformer may help to identify marker genes of each cell type. Additionally, we may achieve gene interaction prediction using the attention mechanism.
% under adequate regulations. 
% We expect \method{} to be capable of much more than just modality prediction. Note that currently, we fuse information from different transformers with message-passing GNNs. 
To extend more on transformers, a potential next step is implementing cross-attention cross-modalities. Ideally, all three types of nodes, namely genes, proteins, and cells, would be jointly modeled using a large transformer that includes specific regulations for each modality. 

% insight of protein and gene embedding (diff task)

% all in one transformer

% \noindent\textbf{Limitations and future work}
% Despite the noticeable performance improvement by utilizing transformers with the cross-modality heterogeneous graph, there are still bottlenecks in the current settings. To begin with, we noticed that the performance variations of all methods are consistently higher in the ``CITE'' dataset compared to the ``GEX2ADT'' dataset. We hypothesized that the increased variability in ``CITE'' was due to both less number of training samples (43k vs. 66k cells) and a significantly more number of testing samples used (28k vs. 1k cells). One straightforward solution to alleviate the high variation issue is to include more training samples, which is not always possible given the training data availability. Nevertheless, publicly available single-cell datasets have been accumulated over the past decades and are still being collected on an ever-increasing scale. Taking advantage of these large-scale atlases is the key to a more stable and well-performing model, as some of the intra-cell variations could be common across different datasets. For example, reference-based methods are commonly used to identify the cell identity of a single cell, or cell-type compositions of a mixture of cells. (other examples for pretrained, e.g., scbert)


%\noindent\textbf{Future work.}
% Our work is an extension of the model we implemented in the NeurIPS 2022 competition. Now our framework of multimodal transformers with the cross-modality heterogeneous graph goes far beyond the specific downstream task of modality prediction, and there are lots of potentials to be further explored. Our graph contains three types of nodes. while the cell embeddings are used for predictions, the remaining protein embeddings and gene embeddings may be further interpreted for other tasks. The similarities between proteins may show data-specific protein-protein relationships, while the attention matrix of the gene transformer may help to identify marker genes of each cell type. Additionally, we may achieve gene interaction prediction using the attention mechanism under adequate regulations. We expect \method{} to be capable of much more than just modality prediction. Note that currently, we fuse information from different transformers with message-passing GNNs. To extend more on transformers, a potential next step is implementing cross-attention cross-modalities. Ideally, all three types of nodes, namely genes, proteins, and cells, would be jointly modeled using a large transformer that includes specific regulations for each modality. The self-attention within each modality would reconstruct the prior interaction network, while the cross-attention between modalities would be supervised by the data observations. Then, The attention matrix will provide insights into all the internal interactions and cross-relationships. With the linearized transformer, this idea would be both practical and versatile.

% \begin{acks}
% This research is supported by the National Science Foundation (NSF) and Johnson \& Johnson.
% \end{acks}

% Acknowledgments
\documentclass{article}

\begin{document}

\section{Acknowledgements}

The author would like to thank B. J. Hiley, M. Hajtanian, and D. Nellist for their insightful conversations and support.

\end{document}


% Bibliography 
\bibliographystyle{unsrt} 
\bibliography{mybib} 
%\medskip
% Appendix
%\begin{appendices}
%The choice of proper functions $\phi_e(\cdot)$ combined with the $\lambda$-$\mu$  smoothness  framework for latency functions~\cite{roughgarden2015intrinsic} enables us to strictly improve the social cost compared to that of $1$-PNE, thus bounding PoS($\theta$-PNE) away from PoA($1$-PNE). Following the ideas in~\cite{christodoulou2011performance}, we present a structured method to find good modified latency functions and extend the PoS bounds to the class of $M/M/1$ latency functions, which is commonly used in modeling congestion networks.


Given a standard latency function $\ell(\cdot)$ and a range $\mc{R}$, consider the class of functions $\mc{L}(\ell, \mc{R}) =\{ \phi(\cdot): \phi(\cdot) \text{ is standard}, \ell(x)/\theta \le \phi(x) \le l(x),~\forall x \in \mc{R}\}$.
Further, given a multi-commodity network $\mc{G}$ with total demand $d_{tot}$, define the class of new potential functions,
\begin{flalign}
\label{eq:newPotential}
\bm{\Phi}(\mc{G})=\left\{\sum_{e \in E} \int_{x=0}^{x_e}\phi_e(x) dx: \phi_e(\cdot)\in  \mc{L}(\ell, [0, d_{tot}])\right\}.
\end{flalign}

The following result from \cite{christodoulou2011performance} characterizes the $\theta$-PNE in $\mc{G}$.
\begin{proposition}\label{thm:NewNE}
Given a multi-commodity network $\mc{G}$, a feasible flow $\bm{x}$ is a $\theta$-PNE if it minimizes some potential function $\Phi(\bm{x}) \in \bm{\Phi}(\mc{G})$.
\end{proposition}

\smallskip\noindent\textbf{ PoS Upper Bounds for composite functions.}
Consider the class of latency functions represented as $\ell(x)= \sum_i a_i\ell_i(x)$ where $a_i\geq 0$ for all $i$. Let the total demand in the network be $d=\sum_k d_k$.  We can find an upper bound for PoS through the following procedure:

\begin{itemize}
\item[1.] For each $i$, guess a suitable form of function $\phi_i(x, \psi_i)$, where $\psi_i$ is an appropriately chosen parameter. Represent $\phi(x)= \sum_i \xi_i a_i\phi_i(x, \psi_i)$  for $\xi_i \in [1/\theta,1]$.
\item[2.] For each $i$, obtain the set 
\begin{align*}
\Psi_i(\theta,\xi_i) = \{\psi: \xi_i\phi_i(x, \psi) \in [\ell_i(x)/\theta, \ell_i(x)],~\forall x \in [0,d] \}.
\end{align*}
\item[3.] For each $i$, obtain the set 
\begin{align*}
\Lambda_i(\psi) = \{(\alpha, \beta): y \phi_i(x, \psi) \leq \alpha x \phi_i(x, \psi)  + \beta y \phi_i(y, \psi),~\forall x,y \in [0,d]\}.
\end{align*}
\item[4.] Solve the following optimization problem,
\begin{align*}
\label{eq:designPoS}
PoS(\theta)= \min \left\{\frac{\beta_p}{1-\alpha_p}:  \frac{1-\alpha_p}{1-\alpha_i} \leq \xi_i \leq \frac{\beta_p}{\beta_i},(\alpha_i,\beta_i) \in \Lambda_i(\psi_i), \psi_i \in \Psi_i(\theta,\xi_i), \xi_i \in [1/\theta,1], ~\forall i\right\}.
\end{align*}
\end{itemize}





\smallskip\noindent\textbf{$\bm{M/M/1}$ Delay functions.}
Consider latency functions in the class $\mathcal{D} =\{1/(u-x): u\geq u_{min}\}$, where $u$ is the capacity of the link and $x$ is the flow through the link. The term $u_{min}$ refers to the minimum capacity in the latency class. Further for each function the maximum load is given as $\rho = d/u$ and therefore, $\rho_{max} = d/ u_{min} < 1$ denotes the maximum possible load over the entire class. This is the class of $M/M/1$ delay functions which plays an important role in modeling congestion networks. 
  
\begin{lemma}
The PoS for the latency functions in class $\mathcal{D}$ for $\theta$-PNE,  $\theta \geq  1$, is upper bounded as
\begin{align*}
 \text{PoS}(\theta\text{-PNE};\mathcal{D})\leq \frac{1}{2}\left( 1+ \frac{1}{\sqrt{1- \rho_{\max}(\theta)}}\right),
\end{align*}
where $\rho_{\max}(\theta) = \max \{0, 1-\theta(1-\rho_{\max})\}$. Moreover, if $\theta \geq 1/(1-\rho_{max})$, the PoS of the network becomes $1$. 
\label{lemm:MM1}
\end{lemma}
\begin{proof}
Step 1: Consider the original function $\ell(x;u) = 1/(u-x)$ and the new functions $\phi(x;a,u) = 1/(u-ax)$ for some $a\in \mathbb{R}_{+}$.  Call the class of modified functions, $\mc{D}_{a} = \{\phi(x;a,u): u \geq u_{min}\}$.  

Step 2: Define the set $\Psi(\theta)= \{a:  a\rho\in [\max\{0,(1-\theta(1-\rho))\}, 1 ]\}.$ 

For $a\in \Psi(\theta)$, we have, $ \ell(x;u)/\theta \leq \phi(x;a,u)\leq \ell(x;u) $, for all $u$ and for all  $0\leq x\leq d$. 

The solution to the program that minimizes \eqref{eq:newPotential} is the Nash equilibrium under the latency functions $\phi_e(x;a_e,u_e)$. But, from Proposition~\ref{thm:NewNE}, the solution is a $\theta$-PNE for the original system with functions $\ell_e(x;u_e)$, if all $a_e\in \Psi(\theta)$. Therefore, the price of anarchy (PoA) under latency functions of the class $\mc{D}_{a}$ for all $a\in \Psi(\theta)$, gives upper bounds for the PoS for $\theta$-PNE. 

Step 3: 
The class of functions $\mc{D}_{\phi}$ assumes the same form but changes the maximum load on the system compared to $\mc{D}$. Specifically, for any fixed $a\in  \Psi(\theta) $, we can have the following relation true for $(\alpha,\beta)\in \Lambda(a)$,
\begin{align*}
y\phi(x; a, u) \le \alpha x\phi(x; a,u) + \beta y\phi(y; a, u),& \qquad \forall x,y \in [0,d].
\end{align*} 
Here through some basic calculus we can find out that if the inequality holds on the boundary of $[0,d]^2$ then it holds for the entire region. We obtain that for the boundary $y=0$ the inequality  is always true and for the boundaries $x = 0$ and  $y = d$, the  inequality holds for $\beta \geq 1$ (necessary and sufficient). Further, for $x=d$ the condition,  $4a\rho \alpha  \geq(1+a\rho\alpha -  \beta(1-a\rho))^2$, is  both necessary and sufficient. Therefore, we have $$\Lambda(a) = \{(\alpha,\beta): \alpha\in[0,1),  \beta\geq 1,  4a\rho \alpha  - (1+a\rho\alpha - \beta(1-a\rho))^2\geq 0 \}.$$

Step 4: We can now get the PoS upper bound after minimizing $$\min \{\beta/(1-\alpha): (\alpha,\beta) \in \Lambda(a), a \in \Psi(\theta)\}.$$ 

After the optimization, we get that the minimum is $\frac{1}{2}\left( 1+ \frac{1}{\sqrt{1- \rho(\theta)}}\right)$, where $\rho(\theta) = \max \{0, 1-\theta(1-\rho)\}$ and the choice of $a= \rho(\theta)/\rho$. Finally, taking maximum over all possible $\rho$ we obtain  $\text{PoS}(\theta\text{-PNE};\mathcal{D})\leq \frac{1}{2}\left( 1+ \frac{1}{\sqrt{1- \rho_{\max}(\theta)}}\right)$, where $\rho_{\max}(\theta) = \max \{0, 1-\theta(1-\rho_{\max})\}$.

\end{proof}




%\end{appendices}

\end{document}
% End of v2-acmsmall-sample.tex (March 2012) - Gerry Murray, ACM


