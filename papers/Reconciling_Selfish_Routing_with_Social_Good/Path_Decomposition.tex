% v2-acmsmall-sample.tex, dated March 6 2012
% This is a sample file for ACM small trim journals
%
% Compilation using 'acmsmall.cls' - version 1.3 (March 2012), Aptara Inc.
% (c) 2010 Association for Computing Machinery (ACM)
%
% Questions/Suggestions/Feedback should be addressed to => "acmtexsupport@aptaracorp.com".
% Users can also go through the FAQs available on the journal's submission webpage.
%
% Steps to compile: latex, bibtex, latex latex
%
% For tracking purposes => this is v1.3 - March 2012

% Conference specific section


\documentclass[11pt]{article}

%package for authors
\usepackage{authblk}


\title{Reconciling Selfish Routing with Social Good}
%\author{Soumya Basu\, Ger Yang\, Thanasis Lianeas\, Evdokia Nikolova  and  Yitao Chen\\
%Department of ECE, The University of Texas at Austin}

%\author[1]{ Soumya Basu \and Ger Yang \and Thanasis Lianeas \and Evdokia Nikolova  \and  Yitao Chen}

\author[]{Soumya Basu}
\author[]{Ger Yang}
\author[]{Thanasis Lianeas}
\author[]{Evdokia Nikolova}
\author[]{Yitao Chen}
\affil[]{Department of Electrical and Computer Engineering, \\ The University of Texas at Austin}

%  change " , and "  to  " and "
\renewcommand\Authands{ and }

\date{\vspace{-5ex}}
 
 


% Package to generate and customize Algorithm as per ACM style
\usepackage{algorithm}
\usepackage{algorithmic}
\usepackage{amsmath,amssymb, amsthm}
\usepackage{color}
\usepackage{graphicx}
\usepackage{subcaption}
\usepackage{epstopdf}
\usepackage{bbm}
\usepackage{bm}
\usepackage[makeroom]{cancel}
\usepackage[margin=1in]{geometry} 
\usepackage[titletoc,title]{appendix}
% bibiliography command


% Custom Commands

\newcommand{\edited}[1]{\textcolor{red}{#1}}
\newcommand{\mc}[1]{\mathcal{#1}}
\newcommand{\inctv}{Incentive }
\newcommand{\aIC}{I }
\newcommand{\pathdecomp}{path flow}
\newcommand{\pathdecomps}{path flows}
\newcommand{\approxMinMax}{Budgeted Min-Max}
\newcommand*{\Scale}[2][4]{\scalebox{#1}{$#2$}}%
\newcommand{\argmin}{\mathrm{argmin}}
\newcommand{\argmax}{\mathrm{argmax}}
\newcommand{\efu}{edge flow unfairness}
\newcommand{\EFU}{Edge Flow Unfairness}
\newcommand{\pfu}{path flow unfairness}
\newcommand{\PFU}{Path Flow Unfairness}

% Graphics path
\graphicspath {{fig/}}

% Theorems 
\theoremstyle{definition}
\newtheorem{theorem}{Theorem}
\theoremstyle{definition}
\newtheorem{lemma}{Lemma}
\theoremstyle{definition}
\newtheorem{proposition}{Proposition}
\theoremstyle{definition}
\newtheorem{claim}{Claim}
\theoremstyle{definition}
\newtheorem{corollary}{Corollary}
\theoremstyle{definition}
\newtheorem{definition}{Definition}
\theoremstyle{remark}
\newtheorem*{example}{Example}
\theoremstyle{remark}
\newtheorem*{remark}{Remark}




\usepackage{soul,xcolor}

\definecolor{purple}{rgb}{0.7, 0, 0.6}
\definecolor{dgreen}{rgb}{0, 0.8, 0}
\newcommand{\tgy}[1]{\textcolor{black}{#1}}
\newcommand{\smb}[1]{\textcolor{black}{#1}}
\newcommand{\evn}[1]{\textcolor{black}{#1}}
\newcommand{\thl}[1]{\textcolor{black}{#1}}
\newcommand{\fin}[1]{\textcolor{green}{#1}}




% Document starts
\begin{document}



\maketitle
\begin{abstract}
\begin{abstract}
\label{sec:abstract}

%% 1. what is the problem 
Scientific applications that run on leadership computing facilities often face the challenge 
of being unable to fit leading science cases onto accelerator devices due to memory constraints 
(memory-bound applications).
%
% 2. what is your solution 
In this work, the authors studied one such US Department of Energy mission-critical condensed matter 
physics application, Dynamical Cluster Approximation (DCA++), and this paper discusses how device memory-bound challenges were successfully reduced  by proposing an effective 
``all-to-all'' communication method---a ring communication algorithm. 
%
This implementation takes advantage of acceleration on GPUs and remote direct memory access (RDMA) for fast data exchange between GPUs. 
%
\\Additionally, the ring algorithm was optimized with sub-ring communicators
and multi-threaded support to further reduce communication overhead and 
expose more concurrency, respectively.
%
% 3. What's the cherry-picked evaluation result you want to mention
The computation and communication were also analyzed 
by using the Autonomic Performance Environment for Exascale 
(APEX) profiling tool,  and this paper further discusses the 
performance trade-off for the ring algorithm implementation. 
%
The memory analysis on the ring algorithm shows that the allocation size for the authors' most 
memory-intensive data structure per GPU is now reduced to $1/p$ of the original size, where $p$ is the number of GPUs in the ring communicator.
%
The communication analysis suggests that 
the distributed Quantum Monte Carlo execution time grows linearly as sub-ring size increases, and the cost of messages passing through the network interface connector could be a limiting factor.


%
% \todoRed{Ronnie: Next sentence needs rewrite, too much information about Green's function that no one knows in the abstract; recommend generalizing.} \emph {However, DCA++ is currently facing memory-bound challenge as 
% a larger device array $G_t$ is limited by device memory size, where
% $G_t$ is a two-particle Green's function that allows condensed matter
% scientists to explore larger and more complex (higher fidelity)
% physics cases.}

\end{abstract}

\keywords{DCA++, Quantum Monte Carlo, GPU Remote Direct Memory Access, memory-bound issue, exascale machines}

%\begin{abstract}
%\begin{quote}
%We study the problem of giving a polynomial time computable path flow decomposition which induces low social cost and users feel the routing plan is ``fair''. There are two definitions for ``fair'', the first is that the length experienced by each user is within some constant times the length of the shortest path and the second is the length experienced by each user is bounded by a fixed length. Writing the problem into convex program according to these two definitions of ``fair'', that of the first definition of fair turns out to find a path decomposition of an approximate Nash Equilibrium with best social cost and that of the second definition of fair we call it length bounded minimum social cost problem. We proved both problems are NP-hard. We give a constant factor approximation algorithm for the former and an FPTAS for the latter. Finally, our social cost analysis for both cases show our approximation schemes is better than simply use the exact Nash Equilibrium to approximate the problem.
%\end{quote}
\end{abstract}


\section{Introduction}\label{sec:intro}

\subsection{Two sides of the coin: Social Welfare vs Selfishness} \label{subSect:SOFandFairF}
A fundamental problem arising in the management of road-traffic and communication networks is routing traffic to optimize network performance. In the setting of road-traffic networks the average delay incurred by a unit of flow quantifies the cost of a routing assignment.  From a collective perspective minimizing the average cost translates to maximizing the welfare obtained by society.  
%With complete knowledge of the network, we can efficiently describe any socially optimal (SO) flow, namely a flow that minimizes the average social cost, by quantifying the flow passing through each edge \cite{}. 
Starting from the seminal works of Wardrop \cite{wardrop1952some} and Beckman et al. \cite{beckmann1956studies}, the literature on network games has differentiated between 1) the objective of a central planner to minimize average cost and thus find a socially optimal (SO) flow, and 2) the selfish objectives of users minimizing their respective costs.  In the latter case,  
the network users acting in their own interest are assumed to converge to a Nash Equilibrium (NE) flow as further rerouting fails to improve their own objective.  

The tension between the central planner and individual objectives has been an object of intense study in the 
algorithmic game theory literature on congestion games. A central question arising in congestion games, ``how much does network performance suffer from selfish behavior?'', has been investigated extensively through the notions of Price of Anarchy (PoA) and Price of Stability (PoS), namely the ratio of the maximum cost among all Nash equilibria over the social optimum and the ratio of the minimum cost among all  Nash equilibria over the social optimum, respectively. Prior research shows that the Nash equilibrium flow may attain very poor social welfare compared to social optimum, i.e. we may get only poor bounds on the PoA or the PoS, with these bounds being tight for some classes of instances.  For an overview we refer the reader to the survey~\cite{roughgarden2002selfish}. %We note here that the PoA and the PoS coincide in nonatomic selfish routing instances (the case where users  are infinite yet infinitesimally small), but if we define the PoA and the PoS for other possible outcomes such as approximate Nash equilibria or the ones that will appear in the next paragraphs, then the PoA and the PoS may get different values.

This discrepancy between selfishness and social good calls for finding a middle ground between the two ends of the spectrum---the Nash flow and the socially optimal flow.  
To that end, previous research on mechanism design
has lead to theoretically appealing solutions such as 
toll placement and Stackelberg routing~\cite{swamy2012effectiveness}. Placing tolls on edges has been shown to improve the network performance up to the point of completely optimizing it when there are no restrictions on the tolls' values. Using Stackelberg strategies, where one assumes that a fraction of users is willing to cooperate and follow the routes dictated by the central planner, has also been theoretically shown to improve the network performance. In spite of the nice properties of these solutions that induce selfish users to act in a socially friendly way, these mechanisms have faced criticism in the real world in terms of their implementation and their fairness towards various users. 
 
To mitigate the tension between selfishness and social good in a way that is more fair to the users, we set out to explore the properties of alternative solution concepts 
where users under some reasonable incentive condition 
adopt a ``socially desirable" routing of traffic in between the Nash equilibrium (which has high social cost) and the social optimum (which may be undesirable/unfair to users on the longer paths) \cite{roughgarden2002unfair}.  
%Consider the scenario where some routing application gives route choices to the users instead of showing users the congestion state of the network and let users choose a route by themselves. Users may be incentivized to use this application under the condition that they are not forced to take ``relatively long routes''. Apparently, this scenario is close to reality, since  users are uninformed about the network's congestion and more often use their routing devices to travel to their destinations. 
The advent of routing applications and the growing dependence of users on these applications 
places us at an epoch when such new ideas in mechanism design may be more relevant and also more readily integrated to practice. 
Consider the scenario where some routing application presents the uninformed users with routes alongside the guarantees of ``relative fairness'' and ``reasonable delay'' and the users adopt the paths. This scenario is close to reality, since users unaware of the network congestion often use their routing devices to travel to their destinations or pick a path that has been presented to them before. 

This naturally brings forth the questions of whether there exist solutions (flows) where good social welfare is achieved under an appropriate incentive condition for the users %(e.g. guarantees for ``reasonable delay"), 
and if such solutions can be efficiently computed. 
%A crucial fact  is that users, when asking  routing applications, request for paths to their destinations and  a returned solution can be assumed to be a path flow decomposition  that provides guarantee only for the entire paths' costs. This differentiates from the standard Nash Equilibrium approach where local conditions for subpaths  should also be present. 
An example of such a solution could be enforcing a $\theta$-approximate Nash equilibrium of low social cost,  where users are guaranteed to get assigned a path of cost no greater than $\theta$ times the cost of the shortest path and as such, the solution is ``relatively fair".\evn{\footnote{We note that the concept of fairness has been considered in the literature of routing games in more than one ways. The two main approaches define fairness as: 1) the ratio of the maximum path delay in a given flow to the average delay under Nash equilibrium~\cite{roughgarden2002unfair} and 2) the ratio of the maximum path delay to the minimum path delay in a given flow~\cite{correa2007fast}.}} 
Yet, other solution concepts seem to arise naturally and are introduced below.%\footnotemark 

 
%A candidate solution concept is the $\theta$-approximate Nash equilibrium ($\theta$-NE) which admits diverse flows and specially flows with low social cost compared to Nash equilibrium or equivalently $1$-NE~\cite{}. Inducing a $\theta$-NE readily guarantees users ``relative fairness'' as by definition of $\theta$-NE the assigned path is no more than $\theta$ times the shortest path. Additionally, as the average social cost is low the users also have guarantees of ``reasonable delay''. Building on this idea we proceed to further generalize the Nash equilibrium concepts while keeping the above ``guarantees'' preserved.


% DO WE NEED THIS FOOTNOTE?
%\footnotetext{In the remaining part of the Section~\ref{sec:intro} for the ease of presentation we will contain ourselves to networks with a single commodity. All of the concepts discussed will be extended to the multi-commodity networks later.
%}       

\subsection{Selfishness and Envy}
In our quest to achieve the coveted middle ground between the social optimum and Nash equilibrium, by combining good social welfare with satisfied users, we present two notions related to: 1) Selfishness and 2) Envy. 
%We elaborate on these notions in an example single commodity network to convey the motivation and meaning to the reader. The ideas will be formalized in Section~\ref{sec:prelim} where we generalize them to the multi-commodity case. 

Firstly, we consider selfishness where users tend to selfishly improve their own cost whenever there exists some scope of improvement.  This conforms to the notion of Nash Equilibrium and a slight relaxation of  absolute selfishness leads us to the approximate Nash Equilibrium concept.  Specifically, we consider a multiplicative approximation consistent with the tradition in approximation algorithms: we refer to a $\theta$-Nash equilibrium flow as a flow in which the length of any {\em used} path in the network is less than or equal to $\theta$ times the length of any other path in the network, with $\theta \geq1$.   Note that for $\theta=1$ we have the Nash Equilibrium flow.

The existing literature in congestion games mainly regards a used path as a path that has positive flow in all of its edges, independent of the path flow decomposition that induces the edge flow. Here we make the distinction between \emph{positive} paths, i.e. paths that have positive flow in all of their edges, (note, this is independent of the path flow decomposition) %under the edge flow decomposition, 
 and \emph{used} paths, i.e. paths that appear in the path flow decomposition with positive flow. With these definitions we define a $\theta$-Positive Nash Equilibrium ($\theta$-PNE)\footnote{We remark that in the literature, PNE is typically used for abbreviating Pure Nash Equilibrium.  In this paper, we always use it to mean Positive Nash Equilibrium as we define it here.} 
 to be a flow in which the length of any {\em positive} path in the network is less than or equal to $\theta$ times the length of any other path, and  a $\theta$-Used Nash Equilibrium ($\theta$-UNE) to be a flow in which the length of any {\em used} path in the network is less than or equal to $\theta$ times the length of any other path. 
Specifically, the concept of UNE deals directly with the paths assigned to users whereas PNE deals with positive paths which may remain unused.  
 As we shall see, the set of $\theta$-PNE flows is a subset of the set of $\theta$-UNE flows and this inclusion might be strict, though for $\theta=1$ these sets coincide. The definition of $\theta$-approximate Nash equilibrium in the literature~\cite{christodoulou2011performance} corresponds to that of $\theta$-UNE. However, to the best of our knowledge, the significance of path flows in the definition of $\theta$-UNE has not been made explicit in any prior work.
 
Next, consider the notion of envy where for the same source and destination a user experiences envy against another user if the latter incurs smaller delay compared to the former under a given path flow. Similarly to the approximate Nash equilibrium flow we can consider a notion of approximately envy free flows where in a $\theta$-Envy Free ($\theta$-EF) flow, the ratio of any two {\em used} paths in the network is upper bounded by $\theta$, for some $\theta\geq 1$.  Note, the difference from the $\theta$-UNE definition is that a used path's cost is compared only to other used paths' costs. 
Envy free flows arise naturally as we consider the routing applications setup where users only collect information about the routes provided by the application. Thus, on the one hand, the possible costs for the current users in some sense compare to the costs of the users that have already used the network. On the other hand, routes for which there is no (sufficient) information potentially may never appear as an option. In other words, routes that have not been chosen in the past (sufficiently many times), i.e. ``unused routes'', do not arise in the comparison of the paths' costs. 

An example of how the concepts of $\theta$-PNE, $\theta$-UNE, and $\theta$-EF may differ from each other is illustrated in Figure~\ref{fig:ex_flows}. There, the optimal edge flow of the network is a $2$-PNE due to the presence of `positive' paths of length $2$ and $1$. However, considering path flows there exists a $1$-EF flow that induces  the optimal edge flow.  Also, the example has a path flow that is a $1.5$-UNE but it does not admit a path flow that is a $1$-UNE. More details  are discussed in Section~\ref{sec:satisfaction}, where these notions are formally introduced. 



% RELATED WORK
\textbf{Related work}:
% Object detection related datasets/algo in non-medical domain
% Locally labeled CXR dataset
A few CXR datasets have localized abnormality annotations \cite{shih2019augmenting,filice2020crowdsourcing,jaeger2014two} that are curated manually. These are high quality gold standard ground truth datasets but tend to be smaller in scale (< 30,000 images) and have a narrow coverage, with typically only 1-2 labels. In addition, since most labeling efforts only have abnormality semantics attached, no direct relationships with the affected anatomical locations are available. 

%MEHDI: repeated concepts from above. I am removing the following: 

%The lack of anatomic semantics in the annotation is a limitation for complex multi-modal clinical reasoning work, e.g., differential diagnosis, since clinicians often integrate information along anatomical lines, and for downstream report generation tasks, which often requires describing not only the abnormality but also correctly communicate the location of the abnormalities (and medical devices) to the receiving clinicians. 

Two recent CXR datasets have labels for anatomies described in the reports. In \cite{datta2020dataset}, a small manually annotated dataset (2000 reports) included 10 abnormalities that are individually associated with 29 unique spatial locations (anatomies) at the report level. Another CXR dataset has automatically extracted abnormality and anatomy labels as disconnected concepts that are only correlated at the study level from  160,000 reports using a supervised NLP algorithm \cite{bustos2020padchest}. This was trained on a smaller set of manually annotated data. Neither datasets contain localized annotations for the associated CXR images, nor any comparison relation annotations between sequential exams, both of which are available in the Chest ImaGenome dataset. In Table \ref{tab:related}, we present a comparison of our Chest ImagGenome dataset with other datasets available in the literature.

% Table -- Kashyap

% MEdical imaging datasets to go here: Discussed that we will only focus on cxr datasets that are available for this paper. 
% \caption{\color{red} Kashyap, feel free to continue with the table. We should remove the questionmarks and add a line for our dataset (since all others are not graph). For longer text, using abbreviations and explaining them in the caption often works better. If fill in the values is not possible, it is better to remove the table altogether.}


\begin{table}[t!]
\caption{Summary of existing chest X-ray datasets}
\resizebox{\textwidth}{!}{%
\begin{tabular}{@{}lllllllll@{}}
\toprule
\textbf{Dataset} & \textbf{Annotation Level} & \textbf{Annotation Method} & \textbf{Num Labels} & \textbf{Anatomy Labeled} & \textbf{Graph} & \textbf{Dataset Size} & \textbf{Temporal Labels} & \textbf{Reports} \\ \midrule
SIIM-ACR Pneumothorax Segmentation \cite{filice2020crowdsourcing} & Segmentation & Manual + augmented & 1 & No & No & 12,047 & No & No \\
RSNA Pneumonia Detection Challenge   \cite{shih2019augmenting} & Bounding Boxes & Manual & 1 & No & No & 30,000 & No & No \\
Indiana University Chest X-ray collection \cite{demner2016preparing} & Global & Automated & 10 & No & No & 3,813 & No & Yes \\
NIH CXR dataset \cite{wang2017chestx} & Global & Automated & 14 & No & No & 112,120 & No & No \\
PLCO \cite{team2000prostate} & Global & Automated & 24 & Yes & No & 236,000 & Yes & No \\
Stanford CheXpert \cite{irvin2019chexpert} & Global & Automated & 14 & No & No & 224,316 & No & No \\
MIMIC-CXR \cite{johnson2019mimic} & Global & Automated & 14 & No & No & 377,110 & No & Yes \\
Dutta \cite{datta2020dataset} & Global & Manual & 10 & Yes & Yes & 2,000 & No & Yes \\
PadChest \cite{bustos2020padchest} & Global & Manual + automated & 297 & Yes & No & 160,868 & No & Yes \\
Montgomery County Chest X-ray   \cite{jaeger2014two} & Segmentation & Manual & 1 & Yes & No & 138 & No & No \\
Shenzen Hospital Chest X-ray   \cite{jaeger2014two} & Segmentation & Manual & 1 & Yes & No & 662 & No & No \\  \hline \hline
\textbf{Chest ImaGenome} & Bounding Boxes & Automated & 131 & Yes & Yes & 242,072 & Yes & Yes \\
\bottomrule
\end{tabular}%
}
\label{tab:related}
\vspace{-0.4cm}
\end{table}
% removed (Derived from MIMIC-CXR \cite{johnson2019mimic}) % makes table really small



% CONTRIBUTION
\subsection{Contribution}
The recent influx of technology in traffic routing, the scale of traffic networks and globalization bring about a definite shift in the well studied routing games. The incomplete knowledge of users creates a dependence on routing technologies, giving more freedom to a central planner to mitigate the inefficiency originating from the selfish routing of users in the full information setting. In this work, we show that the path flows in the network may play a key role in achieving the full potential of such route planning mechanisms.  In particular, we clearly differentiate path flows from edge flows through the introduction of `positive' paths and `used' paths. Recall, a `positive' path is a path with all edges carrying nonzero flows under a given edge flow. Whereas a `used' path is a path with nonzero flow under a specific path flow. From the inherent differences of `positive' and `used' paths,  two new concepts, used Nash equilibrium (UNE) and envy free (EF) flow, naturally emerge as generalizations of the Wardrop equilibrium. We call the classical Wardrop equilibrium positive Nash equilibrium (PNE) because it essentially deals with `positive' paths. To the best of our knowledge, this distinction between positive and used paths has not been made explicit despite the rich literature developed on this topic for over half a decade. We also define the respective approximate versions of all the three solution concepts, i.e. $\theta$-PNE, $\theta$-UNE and $\theta$-EF for $\theta> 1$, where the distinction plays a critical role. 


With the introduction of these three related concepts and their approximate versions, the first step in understanding them is to compare the flows against each other. We show that the $1$-UNE and the $1$-PNE are indeed identical and this helps in understanding why the `used' and the `positive' paths have not been explicitly differentiated before this work. But beyond this case the new concepts impose a hierarchical structure on the space of feasible flows. Specifically, we notice that $\theta$-PNE, $\theta$-UNE and $\theta$-EF flows are progressively larger sets, each containing the previous one, with promise of better tradeoff between the social welfare and fairness. In order to grasp the large separation between these concepts note that for some networks the $\theta$-UNE  is not contained in $\Omega(n\theta)$-PNE, where $n$ is the number of nodes in the network (Lemma~\ref{lemm:UNEvPNE} in Section~\ref{sec:hier}). 

Motivated from the classical study of the price of anarchy (PoA) of equilibrium flows we investigate the PoA of $\theta$-UNE and $\theta$-EF. In general we expect that as we move from the $\theta$-PNE to $\theta$-EF flows from a worst case perspective we will encounter flows with larger social cost. As a worst case example we show that the PoA can be unbounded for $1$-EF flows. However, we see that under the well used framework of variational inequality based PoA upper bounds~\cite{roughgarden2004bounding} both $\theta$-PNE and $\theta$-UNE admit the same bound on the PoA (Lemma~\ref{lemm:UNEvVI} in Section~\ref{sec:social}). Through a similar reasoning we show that the price of stability is non increasing from $\theta$-PNE to $\theta$-EF flows. 

%The implication of `used' and `positive' paths in a network brings about finer details in the \evn{definition of fairness}. We define edge flow fairness, which compares the `positive' paths in a network, and path flow fairness, which compares the `used' paths in the network. The importance of this differentiation becomes clear as one realizes that a path flow having a large `positive' path yet balanced `used' \evn{paths} should not be deemed unfair. 

Focusing on cost-efficient and fair flow design, the question of computing a $\theta$-PNE, a $\theta$-UNE or a $\theta$-EF flow with low social cost becomes one of the fundamental questions. We experience a temporary setback as the traditional convex optimization framework for computing the equilibrium and socially optimal flows fails here due to the non-convexity of the sets of $\theta$-PNE, $\theta$-UNE and $\theta$-EF flows for $\theta>1$. Formally, we prove (Theorem~\ref{thm:main_hardness} in Section~\ref{sec:complex}) that obtaining the best $\theta$-UNE or the best $\theta$-EF flow is NP-hard. Indeed given a socially optimal flow it is NP-hard to decide whether it admits a path flow decomposition which is $\theta$-UNE ($\theta$-EF). In a positive direction we show (Lemma~\ref{lemma:3Easy} in Section~\ref{sec:complex}) that for any `acylic' flow we can decide whether it is a $\theta$-PNE or not. As any `cyclic' flow can be made `acyclic' without increasing its social cost, the above result is sufficient for our design goal, i.e. balance social cost and fairness. However, we leave open the question of finding the best $\theta$-PNE flow ($\theta>1$). 

We further discuss how, at a conceptual level, the new ideas could be integrated with routing technologies (in Section~\ref{sec:design}). Drawing elements from different but related areas, we observe that minimization of modified latency functions can facilitate the calculation of a $\theta$-PNE flow with social cost guarantees. In particular, we use two techniques for bounding the social cost: 1) modified potential functions and 2) bounded tolls.  As a side note, following the ideas presented by Christodoulou et al.~\cite{christodoulou2011performance}, we explicitly articulate a technique to upper bound the price of stability for general functions and use it to extend the analysis of PoS for M/M/1 delay functions (Lemma~\ref{lemm:MM1} in Section~\ref{sec:design}).

In another direction, we deviate from the norm of deterministic flow design, and formalize the concept of randomization in flow design. We present (Theorem~\ref{thm:RR} in Section~\ref{sec:design}) an expression for the mean and a bound for the standard deviation of a path `used' by a typical user under this strategy.  The newly introduced concepts of $\theta$-UNE and $\theta$-EF flows play a crucial role in the variance reduction of this strategy.  The introduction of randomized routing in flow design may be of independent interest and we believe it can play an important role in emerging routing technologies. 




% Head 1
\section{Preliminaries}\label{sec:prelim}
\subsection{Network and Flows}
\textbf{Network.} We are given a directed graph $G(V,E)$ with vertex set $V$, edge set $E$, and a set of commodities % $ \mc{K}=\{(s_k, t_k)\}$. 
$\mc{K}=\{1,2,\dots,K\}$.  Each commodity $k \in \mc{K}$ is associated with a source $s_k$ and a sink $t_k$.  We denote $\mc{T} = \{(s_k,t_k)\}_{k \in \mc{K}}$ as the collection of the source-sink pairs for all commodities.
Also, for each commodity $k \in \mc{K}$, let $\mc{P}^k$ be the set of directed simple paths in $G$ from $s_k$ to $t_k$, and let $d_k > 0$ be the demand associated with commodity $k$. Define $\mc{P}:=\cup_{k \in \mc{K}}\mc{P}^k$ to be the set of paths over all commodities and $\bm{d}:=(d_k)_{k \in \mc{K}}$ to be the vector of the demands. Each edge $e \in E$ is given a load-dependent \emph{latency function} $\ell_e(x)$, assumed to be nonnegative, differentiable, and nondecreasing. Moreover, we assume $x \ell_e(x)$ is convex with respect to $x$.  We shall abbreviate an instance of the problem by the quadruple $\mc{G}=(G(V,E),\mc{T}, 
\{\ell_e\}_{e\in E}, \bm{d})$.

%We consider , namely an infinite set of users that are infinitesimally small, so that an individual user does not affect the delays experienced by other users in the network.

% should we define the flow for each commodity x_e^k?
\smallskip\noindent\textbf{Flows.} Given an instance $\mc{G}$, the collective decisions of users in commodity $k \in \mc{K}$ can be encoded in two ways, as a path flow $\bm{f}^k=(f_{\pi}^k)_{\pi \in \mc{P}}$ and as an edge flow $\bm{x}^k = (x_e^k)_{e\in E}$. These two representations are related as $x_e^k = \sum_{\pi \in \mathcal{P}^k:\pi \owns e} f_{\pi}^k$. We can also consider the collective decisions of users of all commodities together by defining the {\pathdecomp} $\bm{f} = \sum_{k \in \mc{K}} \bm{f}^k$ and the edge flow $\bm{x} = \sum_{k \in \mc{K}} \bm{x}^k$.
There may exist multiple {\pathdecomps} corresponding to an edge flow $\bm{x}$ and we denote the set of such decompositions as $\mc{D}_p(\bm{x})$. Denote the feasible edge flows by $\mathcal{D}_E.$\footnote{
For node $u\in V$, $E_u^+$ denote the set of its outgoing edges and $E_u^-$ denote the set of its incoming edges.  $\mc{D}_E$ is the set of vectors that satisfies the flow conservation equations:
\Scale[0.7]{
\mathcal{D}_E = \left\{\bm{x}: x_e=\sum_{k \in \mc{K}} x_e^k,
\sum_{e\in E_{u}^+} x_e^k -\sum_{e\in E_{u}^-} x_e^k = d_k \left(\mathbbm{1}_u (s_k) - \mathbbm{1}_u (t_k)\right),\forall e\in E,\\
\forall u\in V, \forall k \in \mc{K}\right\}.
}
}
%\edited{We can define the feasible region for path flow and  edge flow as $\mathcal{F} = \{(\bm{x},\bm{f}): \bm{x} \in \mathcal{D}_E, \bm{f} \in \mathcal{D}_P(\bm{x})\}$. } % should remove this?
We can define the feasible region for all possible path flows as $\mc{D}_p = \cup_{\bm{x} \in \mc{D}_E} \mc{D}_p(\bm{x})$.
  
We further differentiate a \emph{positive} path from a \emph{used} path in the following definitions.
 
\begin{definition}[Positive path]
For an edge flow vector $\bm{x}$, we call a path $\pi\in \mathcal{P}$ \emph{positive} for commodity $k \in \mc{K}$ if for all edges $e \in \pi$, $x_{e}^k > 0$.  
For each commodity $k \in \mc{K}$, we can define the set of \emph{positive} paths under edge flow $\bm{x}$ as $\mathcal{P}_{+}^k(\bm{x}) = \left\{p: p \in \mathcal{P}^k, \forall e \in p, x_e^k>0  \right\}$.  Further, the set of all positive paths for all commodities under edge flow $\bm{x}$ can be defined as $\mc{P}_{+}(\bm{x}) = \cup_{k \in \mc{K}} \mc{P}_{+}^k(\bm{x})$.
%Call the set of \emph{positive} paths under $\bm{x}$, $\mathcal{P}_{+}= \left\{p: p \in \mathcal{P}, \forall e \in p, x_e>0  \right\}$. For any commodity $k$ the positive paths are $\mc{P}^k_{+} =\mc{P}_{+} \cap \mc{P}^k$.
\end{definition}
     
\begin{definition}[Used path]
For a path flow $\bm{f}$, we call a path $\pi\in \mathcal{P}$ \emph{used} by commodity $k \in \mc{K}$ if $f_{\pi}^k > 0$ and \emph{unused} otherwise. For each commodity $k \in \mc{K}$, we can define the set of \emph{used} paths under path flow decomposition $\bm{f}$ as $\mathcal{P}_u^k(\bm{f}) = \{p: p\in \mathcal{P}, f_p^k >0\}$.  Further, the set of all used paths for all commodities under path flow decomposition $\bm{f}$ can be defined as $\mc{P}_u(\bm{f}) = \cup_{k \in \mc{K}} \mc{P}_u^k(\bm{f})$.
\end{definition}

\begin{remark}
Note that a used path is always positive but a positive path may be unused depending on the particular path flow decomposition.
\end{remark}

\subsection{Costs and Equilibria}
\textbf{Costs.} Under a path flow $\bm{f} \in \mc{D}_p$, the cost (latency) of a path $\pi$ is defined to be the sum of latencies of edges along the path: 
$\ell_{\pi}(\bm{f}) = \ell_{\pi}(\bm{x})=\sum_{e \in \pi}\ell_e(x_e)$ for $\bm{f} \in \mc{D}_p(\bm{x})$.

\begin{definition}[Social cost and socially optimal flow]
The \emph{social cost} (SC) of a flow $\bm{x} \in \mc{D}_E$ is the total latency in the network under the flow, $SC(\bm{x})=\sum_{e\in E}x_e \ell_{e}(x_e)$. The social cost of a path flow $\bm{f} \in \mc{D}_p$ is $SC(\bm{f})=SC(\bm{x_f})$, where $\bm{x_f}$ is the edge flow induced by $\bm{f}$.  We sometimes refer to the social cost simply as {\em cost}.
A flow with minimum social cost among all feasible flows is called a \emph{socially optimal} flow or simply, a \emph{social optimum}.  The set of socially optimal edge flows is denoted by 
$$SO_E = \{\bm{x} \in \arg\min SC(\bm{x}) \}.$$
Also, we denote the set of socially optimal path flows by
$$ SO_p= \{\bm{f} \in \arg\min SC(\bm{f}) \}.$$
\end{definition}

\smallskip\noindent 
\textbf{Equilibrium.} We assume that users are {\em nonatomic}, namely there are infinitely many users that are infinitesimally small.  As such, a single user controls an infinitesimally small fraction of flow and her routing choice does not unilaterally affect the costs experienced by other users.  This fact is captured by the definition of equilibrium below.

\begin{definition} (Nash Equilibrium)\footnote{The Nash equilibrium in nonatomic routing games is also commonly known as Wardrop equilibrium. }
A path flow $\bm{f}$ is a  \emph{Nash Equilibrium} if for any commodity $k\in \mathcal{K}$ and any used path $p\in \mathcal{P}_{u}^k(\bm{f})$ we have $\ell_p(\bm{f}) \leq  \ell_q(\bm{f})$, for all paths $q\in \mathcal{P}^k$. 
\end{definition}

Given a Nash equilibrium, we can measure its quality by comparing its cost with the cost of the socially optimal flow.  This idea is often formalized as the \emph{price of anarchy} and the \emph{price of stability} which we define below. Since our scope is to examine user oriented solution concepts other than  the Nash Equilibrium, we generalize the classic definitions of the price of anarchy and stability to apply to an arbitrary set of flows $\mc{F}$. If $\mc{F}$ is the set of Nash equilibria, we get the standard definition for the price of anarchy and price of stability.
%The $\theta$-PNE, $\theta$-UNE and $\theta$-EF flows are not unique and they cover a large range of social cost. Call them concisely as $\theta$-`F', with `F' $\in \{\text{PNE, UNE, EF} \}$,  to reduce notation. 

\begin{definition}[Price of Anarchy and Price of Stability]  Given an instance $\mc{G}$
and a set of (feasible) flows $\mc{F}$, we define the price of anarchy (PoA) as the ratio of the maximum social cost of any flow in $\mc{F}$  to the socially optimal cost. The price of stability (PoS) is the ratio of the minimum social cost of any  flow in $\mc{F}$ to the socially optimal cost.  The PoA and PoS are formally expressed as: 
\begin{flalign}\label{eq:PoA}
%PoA(\theta;\text{`F'})= \max_{\mc{G} } \max \left\{  \ \frac{C(\bm{f})}{C(\bm{x}^*)}: (\bm{x},\bm{f}) \text{ is a } \theta\text{-`F'} \text{in } \mc{G}\right\}.
PoA(\mc{F}) = \max \left\{  \ \frac{SC(\bm{f})}{SC(\bm{x}^*)}: \bm{f} \in \mc{F}, \bm{x}^* \in SO_E \right\}.
\end{flalign}
\begin{flalign}\label{eq:PoS}
%PoS(\theta; \text{`F'})= \max_{\mc{G}} \min \left\{  \ \frac{C(\bm{f})}{C(\bm{x}^*)}: (\bm{x},\bm{f}) \text{ is a } \theta\text{-`F'} \text{in } \mc{G}\right\}.
PoS(\mc{F}) = \min \left\{  \ \frac{SC(\bm{f})}{SC(\bm{x}^*)}: \bm{f} \in \mc{F}, \bm{x}^* \in SO_E \right\}.
\end{flalign}
We may define the PoA and the PoS over sets of instances. For a set of instances, its PoA and PoS equals the maximum  PoA and PoS among the instances in the set, respectively. We will use this definition when examining  instances with latency functions in class $\mc{L}$ (for some $\mc{L}$), and it will be clear from the context.
\end{definition}
 


The PoA and the PoS for the set of Nash equilibria coincide in nonatomic selfish routing instances, since there is only one Nash equilibrium (up to edge costs). In contrast, for the sets that we consider in this work and introduce in the following section (e.g., the set of approximate Nash equilibria)  the PoA and the PoS may get different values.

%\smallskip \noindent \textbf{Fairness.} \thl{[Totally commented out fairness, although initially i though it  was good here. Probably a remark should be put after the definition of $\theta$ flows of how they relate to fairness.]} 
%The fairness of a flow is a fundamental quantity that captures satisfaction among users and it has been studied in the literature under various forms. The unfairness of a flow has been defined as the ratio of the maximum used path in a path flow $\bm{f}$ and the Nash length of the network by Roughgarden in \cite{}. In their paper~\cite{} Correa et al. have used a different definition. They define unfairness to be the ratio between the maximum use' path and the minimum used path in a path flow $\bm{f}$. We use the pessimistic approach of Correa et al. and differentiate unfairness between an edge flow and a path flow. The unfairness in the two cases are with respect to the positive paths and the used paths, in the specific order.

%\thl{[Do we need/use these definitions? Maybe  (in the next section) just say that we may refer to all ($\theta$-PNE/UNE/EF) as $\theta$ fair flows. Does this suffice?]}  

%\begin{definition}[{\EFU}]
%Given an edge flow $\bm{x}$, the unfairness of the edge flow is given as the ratio between the maximum positive path and the minimum positive path under $\bm{x}$. We call this  {\efu}, $ U_E(\bm{x}) = \max_k\max\left\{\frac{\ell_p}{\ell_q}: p, q  \in \mc{P}_{+}^k(\bm{x})\right\}$.
%\end{definition}

%\begin{definition}[{\PFU}]
%Given a path flow $\bm{f}$, the unfairness is given as the ratio between the maximum used path and the minimum used path under $\bm{f}$. We call this  {\pfu}, $ U_P(\bm{f}) = \max_k\max\left\{\frac{\ell_p}{\ell_q}: p, q \in \mc{P}_{u}^k(\bm{f})\right\}$.
%\end{definition}

%We note that given $\bm{f}$ we can compute the corresponding edge flow $\bm{x}$ and compute its {\efu}. Similarly, given $\bm{x}$ we can construct a specific path decomposition and compute {\pfu}.
 




%%-------------------------------------------------------------------------
\section{Solution Concepts}\label{sec:satisfaction} 
Here we give the formal definition of the solution concepts we introduced in Section~\ref{sec:intro}.  We also provide an example to illustrate their differences, and prove that each solution concept may correspond to a non-convex set of flows. %We begin with the relative satisfaction among users where we compare the latency of different paths under a given path flow.   

\begin{definition}[$\theta$-PNE]\label{def:ApproxPNE}
Given a network $\mc{G}$, an edge flow $\bm{x}$ is a $\theta$-Positive Nash Equilibrium ($\theta$-PNE) flow if for any commodity $k\in \mathcal{K}$ and any positive path $p\in \mathcal{P}_{+}^k(\bm{x})$ we have $\ell_p(\bm{x}) \leq \theta \ell_q(\bm{x})$, for all paths $q\in \mathcal{P}^k$. 
%Moreover, call a $\theta$-PNE flow $o$, a $\theta$-Positive Nash Optimal flow ($\theta$-PNO) if it has the minimum social cost among all other $\theta$-PNE. 
We may call a path flow $\bm{f}$  a $\theta$-Positive Nash Equilibrium, if $\bm{f}\in \mc{D}_p(\bm{x})$, for some $\theta$-Positive Nash Equilibrium edge flow $\bm{x}$.
\end{definition}

\begin{definition}[$\theta$-UNE]\label{def:ApproxUNE}
Given a network $\mc{G}$, a {\pathdecomp} $\bm{f}$ is a $\theta$-Used Nash Equilibrium ($\theta$-UNE) flow if for any commodity $k\in \mathcal{K}$ and any used path $p\in \mathcal{P}_{u}^k(\bm{f})$ we have $\ell_p(\bm{f}) \leq \theta \ell_q(\bm{f})$, for all paths $q\in \mathcal{P}^k$. %Moreover, call a $\theta$-UNE flow $o$, a $\theta$-Used Nash Optimal flow ($\theta$-UNO) if it has the minimum social cost among all other $\theta$-UNE.   
\end{definition}

The definition of $\theta$-UNE  corresponds to that of $\theta$-approximate Nash equilibrium used thus far in the literature.
%\noindent\textbf{Nash Length and $1$-PNE:}  
For $\theta=1$, $1$-UNE and $1$-PNE (or simply PNE) coincide, as we show in Lemma~\ref{lemm:UNEvPNE}, and they correspond to the Nash equilibrium, which has been studied extensively. %We simply abbreviated it as PNE.   
%Surprisingly, we will show in Lemma~\ref{lemm:UNEvPNE} that $1$-UNE coincides with $1$-PNE, which we simply abbreviated it as PNE. 
It turns out that every PNE of an instance solves the convex optimization problem  
%\begin{align}\label{eq:1PNE}
%\bm{x}_{\text{PNE}}\in \argmin
$\{ \sum_{e\in E}\int_{0}^{x_e} \ell_e(x)dx : \bm{x}\in \mc{D}_E\}$,
%\end{align}
which as a consequence yields the uniqueness of PNE up to edge costs, i.e. for all $e$ and any two PNE flows, $\bm{x}$, $\bm{x}'$, $\ell_e(x_e)=\ell_e(x'_e)$. This implies for any commodity $k$, all the positive paths have length $L^k_{NE}$ which is called the Nash length of that commodity.   

\begin{definition}[$\theta$-EF]\label{def:ApproxEF}
Given a network $\mc{G}$, a {\pathdecomp} $\bm{f}$ is $\theta$-Envy Free if for any commodity $k\in \mathcal{K}$ and any used path $p\in \mathcal{P}_u^k(\bm{f})$ we have $\ell_p(\bm{f}) \leq \theta \ell_q(\bm{f})$, for all used paths $q\in \mathcal{P}_u^k(\bm{f})$. 
%A $\theta$-EF flow is $\theta$-Envy optimal ($\theta$-EO) flow that minimizes the social cost among all the $\theta$-EF flows.
\end{definition} 

For an instance, we may use $\theta$-PNE, $\theta$-UNE or $\theta$-EF to describe the set of $\theta$-PNE, $\theta$-UNE or $\theta$-EF flows respectively, which will be clear from the context, and we may omit $\theta$ to represent $\theta=1$. Also, by \emph{incentive conditions} we refer to the conditions (inequalities) used to define these flows. Additionally, we may refer to all $\theta$-PNE, $\theta$-UNE and $\theta$-EF flows as $\theta$ fair flows.  The reason for that comes from the fact that their incentive  conditions have inherent the comparison of the maximum used path cost with the minimum (used) path cost, which in some sense describes how (un)fair the flow for players on the maximum cost paths compared to the cost of the lowest cost (used) paths is. Similar notions of (un)fairness have been examined in the past, e.g., by Rougharden \cite{roughgarden2002unfair} and Correa et al. \cite{correa2007fast}. 
 
%In a $\theta$-PNE, $\theta$-UNE or $\theta$-EF flow, the ratio of these costs depends strongly on $\theta$ while

%The fairness of a flow is a fundamental quantity that captures satisfaction among users and it has been studied in the literature under various forms. The unfairness of a flow has been defined as the ratio of the maximum used path in a path flow $\bm{f}$ and the Nash length of the network by Roughgarden in \cite{roughgarden2002unfair}. In their paper~\cite{correa2007fast} Correa et al. have used a different definition. They define unfairness to be the ratio between the maximum use' path and the minimum used path in a path flow $\bm{f}$. We use the pessimistic approach of Correa et al. and differentiate unfairness between an edge flow and a path flow. The unfairness in the two cases are with respect to the positive paths and the used paths, in the specific order.


%\thl{[TALK BRIEFLY ABOUT FAIRNESS HERE???]}
%
%We can mathematically represent a $\theta$-flow with a given relative satisfaction condition as follows. Different $\mc{P}_1$ and $\mc{P}_2$ in Table~\ref{t:flowDef} represents different satisfaction criteria.
%\begin{equation} \label{prog:thetaflows}
%\begin{split}
%\theta\text{-Flow} = \{ \bm{f} | \bm{f} \in \mc{D}_p, &\ell_{\pi_1}(\bm{f}) \le \theta \ell_{\pi_2}(\bm{f}), \forall k, \forall \pi_1 \in \mathcal{P}^k_{1}, \forall \pi_2 \in \mathcal{P}^k_2 \}. 
%\end{split}
%\end{equation} 
%\begin{table}
%\begin{center}
%\begin{tabular}{ |c|c|c| } 
% \hline
% Flow & $\mc{P}^k_1$ & $\mc{P}^k_2$ \\ 
% \hline
% $\theta$-PNE & $\mc{P}^k_{+}$  & $\mc{P}^k$ \\ 
% \hline
% $\theta$-UNE & $\mc{P}^k_{u}$  & $\mc{P}^k$ \\  
% \hline
% $\theta$-EF & $\mc{P}^k_{u}$  & $\mc{P}^k_{u}$ \\
% \hline
%\end{tabular}
%\end{center}
%\caption{Flow Definition}\label{t:flowDef} 
%\end{table}  
%
\begin{figure}
	\centering
	\begin{subfigure}[b]{0.45\linewidth}
		\centering
		\includegraphics[width=0.9\linewidth]{example_flows}
		\caption{Paths $\pi_1$ and $\pi_2$ have $1/2$ unit of flow.  This path flow assignment is a social optimum.}
		\label{fig:ex_flows1}
	\end{subfigure}\hspace{0.05\linewidth}
	\begin{subfigure}[b]{0.45\linewidth}
		\centering
		\includegraphics[width=0.9\linewidth]{example_flow1}
		\caption{The path flow assignment in Figure~\ref{fig:ex_flows1} is $1$-EF but not $1$-UNE.}
		\label{fig:ex_flows01}
	\end{subfigure}
	
	\begin{subfigure}[b]{0.45\linewidth}
		\centering
		\includegraphics[width=0.9\linewidth]{example_flow2}
		\caption{The path flow assignment in Figure~\ref{fig:ex_flows1} is $1.5$-UNE but not $1.5$-PNE.}
		\label{fig:ex_flows2}
	\end{subfigure}\hspace{0.05\linewidth}
	\begin{subfigure}[b]{0.45\linewidth}
		\centering
		\includegraphics[width=0.9\linewidth]{example_flow3}
		\caption{The path flow assignment in Figure~\ref{fig:ex_flows1} is $2$-PNE.}
		\label{fig:ex_flows3}
	\end{subfigure}
	\caption{Example illustrating the three solution concepts $\theta$-UNE, $\theta$-PNE and $\theta$-EF. }
	\label{fig:ex_flows}
\end{figure}
% 

To see how these concepts differ from each other, we give an example in Figure~\ref{fig:ex_flows}.  Suppose the instance is given as shown in Figure~\ref{fig:ex_flows1} and there is a single commodity that routes a unit demand  from $s$ to $t$.  We consider the path flow that routes $1/2$ of the demand through path $\pi_1$ and routes $1/2$ through path $\pi_2$.  It is easy to verify that this is indeed a socially optimal flow.  Next, let us find the appropriate sets that this path flow assignment belongs to with respect to the solution concepts we described above:
\begin{enumerate}
	\item \textbf{$1$-EF:} As shown in Figure~\ref{fig:ex_flows01}, it is easy to verify that this flow is a $1$-EF as each used path has the same length.
	\item \textbf{$1.5$-UNE:} As shown in Figure~\ref{fig:ex_flows2}, any used path has length $1.5$ and the shortest path in the graph has length $1$.  Hence the flow is a $1.5$-UNE as the length of each used path is within a factor $1.5$  of any path.
	\item \textbf{$2$-PNE:} As shown in Figure~\ref{fig:ex_flows3}, for a PNE, we have to take all positive path\evn{s} into account.  Since the longest positive path has length $2$, this flow assignment is not a $1.5$-PNE flow.  Instead, we can see that any positive path is within a factor $2$ of any path.  Hence, this flow is a $2$-PNE.
\end{enumerate}

 
Our goal is to examine the properties of $\theta$ fair flows and provide ways to obtain such flows with good social cost. Regarding the second direction,  in general, the sets of $\theta$-PNE, $\theta$-UNE, and $\theta$-EF flows may not be convex and may contain multiple path flows, which raises the level of difficulty for computing good or optimal such flows. Next we present an example that demonstrates the non-convexity of these sets.



\begin{proposition} There exists a network $\mc{G}$ and $\theta>1$ such that the sets $\theta$-PNE, $\theta$-UNE, and $\theta$-EF are not convex.
\end{proposition}
\begin{proof}
The instance in Figure~\ref{fig:ex_noncvx2} demonstrates that the sets $\theta$-PNE, $\theta$-UNE, and $\theta$-EF are all non convex.  Consider a commodity routing from $s$ to $t$ with unit demand. 
%Then, we are giving two assignment of flows that both satisfy these flow conditions.  
Then, consider the following two flow assignments.
The first one routes all demand along the path $s-u-v-t$.  In this case, path $s-u-v-t$ is the only positive path and has cost equal to $2$.  It is easy to verify that this is a $3/2$-PNE, $3/2$-UNE, and $3/2$-EF.  The second flow routes $2/3$ of the demand along  path $s-u-t$ and routes $1/3$ along path $s-v-t$.  Path $s-u-t$ has cost equal to $1$ and path $s-v-t$ has cost equal to $3/2$.  It is easy to verify that this is a $3/2$-PNE, $3/2$-UNE, and $3/2$-EF as well.  However, if we take the convex combination of these two assignments evenly, then we can find that the path $s-v-t$ has cost equal to $11/6$ and the path $s-u-v-t$ has cost equal to $7/6$, and thus their ratio is greater than $3/2$.  This shows that this combined flow is neither a $3/2$-PNE, a $3/2$-UNE, nor a $3/2$-EF, and hence they are not convex sets.
\end{proof}

\begin{figure}
	\centering
	\includegraphics[width=0.3\linewidth]{examples_noncvx2}
	\caption{Non-convexity of $\theta$-flows}
	\label{fig:ex_noncvx2}
\end{figure}

%The example for the non-convexity of $\theta$-I is demonstrated in Figure~\ref{fig:ex_noncvx1}.  Consider a commodity that routes unit demand from $s$ to $t$.  Note that the Nash length for this graph is $2$, when routing $8/9$ on the path $e1-e2$ and routing $1/9$ on the path $e3-e4$.  Similarly to the example above, we then consider two flow assignments.  The first one is to route half on $e1-e4$ and route half on $e3-e2$.  In this flow both path have length $11/2$.  We can easily verify that this is a $3$-I flow.  The second one is to route $2/3$ on the path $e1-e2$ and route $1/3$ on $e3-e4$.  The path $e1-e2$ has length $2$ and the path $e3-e4$ has length $6$.  However, if we mix them evenly, then the path $e3-e4$ has length $15/2$, and thus the combined flow is not $6$-I.  As a consequence, the set of $6$-I flows in this example is not a convex set.


%%-------------------------------------------------------------------------
% Hierarchical matrices can be seen as algebraic generalizations of 
% the FMM where a matrix is compressed using both low-rank and sparse representations
% hierarchically ~\cite{bebendorf08}. 
% We consider a matrix $K\in \mathbb{R}^{N\times N}$ to be
% \emph{hierarchical} if it can be approximated as
% \begin{equation}
%   \label{e:partitioning}
%   \sk{K} =
% \begin{bmatrix}
% K_{\lc\lc} & 0 \\ 
% 0 & K_{\rc\rc} \\ 
% \end{bmatrix} + 
% \begin{bmatrix} 
%   0 & UV_{\lc\rc} \\ 
%   UV_{\rc\lc} & 0 \\ 
% \end{bmatrix} +
% \begin{bmatrix} 
% 0 & S_{\lc\rc} \\ 
% S_{\rc\lc} & 0 \\ 
% \end{bmatrix}
% \end{equation} 
% where the \emph{off-diagonal} blocks $K_{\lc\rc}$ and $K_{\rc\lc}$ 
% are \emph{approximated} by some low-rank factorizations
% $UV$ plus a sparse correction matrix $S$, and
% the \emph{on-diagonal} blocks $K_{\lc\lc}$ and $K_{\rc\rc}$ are
% themselves hierarchical.
% For example, on the right side of \figref{fig:tree}, 
% the sparsity of $S$ is in blue, and $UV$ contains
% many submatrices with different low-rank approximations in pink.
% %For simplicity, we write th approximation as $K\approx \sk{K}=D+UV+S$, 
% %where $UV$ are \emph{far} (approximation), $S$ is \emph{near} (no approximation)
% %and $D$ is again hierarchical.
% 
% %Note that the low-rank plus sparse structure is \emph{not
% %invariant} on permutations, it very strongly depends on the ordering
% %of the columns (or rows since the matrix is symmetric).
% 
% \begin{figure}[h]
%   \centering
%   \includegraphics[scale=.3]{figures/nearfar.pdf}
%   \caption{A hierarchical low-rank plus sparse matrix (right) and 
%     its tree representation.
%     The off-diagonal blocks are combinations of low-rank matrices (pink)
%     and sparse matrices (blue). The on-diagonal blocks are further
%     partitioned as two child nodes in the tree.
%     The $\bigstar$ symbols denote the entries (neighbors) that cannot
%     be approximated.
%     \algref{a:nearnear} computes the sparse pattern  in blue.
%     The low-rank pattern in pink is computed by \algref{a:nearfar}
%     and \algref{a:farfar} with a preorder and postorder traversal.
%     The solid edges in the tree show the traversing paths of
%     \texttt{NearFar($\beta$,0)}. The preorder traversal stops and add
%     $\alpha$ to $Far(\beta)$,
%     because $K_{\beta\alpha}$ does not contain any $\bigstar$.
%     \algref{a:farfar} merges common codes from two children's Far lists.
%     For example, after \texttt{NearFar(\lc,0)} and \texttt{NearFar(\rc,0)},
%     $Far(\lc)=\{\rc,4,4\}$ and $Far(\rc)=\{\lc,4,2\}$.
%     The common nodes will then be removed and merged to their parent such that 
%     $Near(\alpha)=\{4,2\}$, 
%     $Near(\lc)=\{\rc\}$ and
%     $Near(\rc)=\{\lc\}$.
%   }
%   \label{fig:tree}
% \end{figure}
% 
% 
% 
% 
% \textbf{Treecodes.}
% Note that the low-rank structure strongly depends on the ordering
% of the columns (or rows since the matrix is symmetric).
% Such ordering is typically exploited by a geometric tree structure, and
% equation \eqref{fig:tree} resembles the relation between a parent
% node $\alpha$ and the two children $\lc$ and $\rc$ in a tree.
% In \figref{fig:tree},
% a pair of treenodes $\alpha$ and $\beta$ correspond to a submatrix
% $K_{\beta\alpha} = \{K_{ij}\lvert i\in{\beta},j\in{\alpha}\}$.
% To create such approximation in \figref{fig:tree} for each
% $K_{\beta\alpha}$, we need to decide whether $\alpha$ and $\beta$
% are \emph{near} (in blue, cannot approximate) or \emph{far} 
% (in pink, prunable) from each other.
% %from the root to the leaf level.
% 
% \begin{algorithm}[!t]
% \caption{{} \texttt{NearNear}()}
% \begin{algorithmic}
%   \STATE \texttt{\bf for each} $\alpha$ and $\beta$ is leaf \texttt{\bf do}
%   \STATE \gap \texttt{\bf if} $\alpha \cap \MA{N}(\beta)$ \texttt{\bf then}
%            $Near(\beta) = Near(\beta) \cup \alpha$;
% \end{algorithmic}
% \label{a:nearnear}
% \end{algorithm}
% 
% \begin{algorithm}[!t]
% \caption{{} \texttt{NearFar}($\beta$, $\alpha$)}
% \begin{algorithmic}
%   \STATE \texttt{\bf if} $\alpha \cap Near(\beta)$ \texttt{\bf then}
%   \texttt{NearFar}($\beta$,\lc); \texttt{NearFar}($\beta$,\rc); 
%   \STATE \texttt{\bf else} $Far(\beta) = Far(\beta) \cup \alpha$;
% \end{algorithmic}
% \label{a:nearfar}
% \end{algorithm}
% 
% \begin{algorithm}[!t]
% \caption{{} \texttt{FarFar}($\alpha$)}
% \begin{algorithmic}
%   \STATE \texttt{FarFar}($\lc$); \texttt{FarFar}($\rc$);
%   %\STATE \texttt{\bf if} $\alpha$ is not leaf \texttt{\bf then}
%   \STATE $Far(\alpha) = Far(\lc) \cap Far(\rc)$; 
%   \STATE $Far(\lc) = Far(\lc) \backslash Far(\alpha)$; $Far(\rc) = Far(\rc) \backslash Far(\alpha)$;
% \end{algorithmic}
% \label{a:farfar}
% \end{algorithm}
% 
% \textbf{Near-far pruning.}
% Let $\MA{N}(\beta)$ be the set of \emph{neighbors} of $\beta$ ($\beta$ itself and indices 
% shown as $\bigstar$ in \figref{fig:tree}).
% We say $\beta$ and $\alpha$ are \emph{near} if $\alpha \cap \MA{N}(\beta)$ 
% is not empty (i.e., $K_{\beta\alpha}$ contains a least one $\bigstar$).
% \algref{a:nearnear} shows how we construct a list of leaf nodes
% $Near(\beta)$ to denote the non-prunable submatrices (blue) in
% \figref{fig:tree}. Here $Near(\beta) = \{\mu, \beta\}$, because $\mu$ 
% contains a neighbor of $\beta$.
% The prunable submatrices (pink) are
% identified with a preorder and postorder traversal in \algref{a:nearfar}
% and \algref{a:farfar} (see the caption of \figref{fig:tree} for examples). 
% For each leaf node $\beta$, \algref{a:nearfar}
% traverses the tree from the root. 
% If $\alpha \notin Near(\beta)$, then we add $\alpha$ into $Far(\beta)$.
% Otherwise, we recurse to the two children of $\alpha$ to see
% if we can add $\lc$ or $\rc$ to $Far(\beta)$.
% %In \figref{fig:tree}, the solid edges denote the path 
% %of \texttt{NearFar($\beta$,0)}.
% %The traversal stops at $\alpha$, 
% %because $\alpha$ is not in $Near(\beta)$. 
% %On the other hands, $Near(\beta) \in \beta \subset$ node 4; thus the
% %traversal continues after visiting node 4.
% Once we have a prunable list $Far(\beta)$ for each leaf node, 
% \algref{a:farfar} traverses bottom-up to merge the common nodes
% from two children lists $Far(\lc)$ and $Far(\rc)$.
% These common nodes are removed from the children and added to 
% their parent's prunable list $Far(\alpha)$, which
% creates larger pink submatrices.
% %For example, after \texttt{NearFar(\lc,0)} and \texttt{NearFar(\rc,0)},
% %$Near(\lc)=\{\rc,4,4\}$ and $Near(\rc)=\{\lc,4,2\}$.
% %The common nodes will then be removed and merged to their parent such that 
% %$Near(\alpha)=\{4,2\}$, 
% %$Near(\lc)=\{\rc\}$ and
% %$Near(\rc)=\{\lc\}$.
% The number of these blue and pink submatrices are usually linear in $N$.
% If the size (for blue) and the rank (for pink) of these submatrices 
% are also constant in $N$, then matrix-vector multiplication may only require
% $\MA{O(N)}$ work with the following scheme.
% 
% \textbf{Fast multiplication.}
% The key for FMM to reach linear time multiplication is to
% compress a pink $K_{\beta\alpha}$ on both sides 
% using nested column and row bases.
% Here 
% we use a two-sided nested Interpolative Decomposition (ID) to
% approximate $UV_{\lc\rc}$ as
% $P_{\sk{\lc}\lc}^{T}K_{\sk{\lc}\sk{\rc}}P_{\sk{\rc}\rc}$.
% Given $\sk{\lc} \subset \lc$
% and $\sk{\rc} \subset \rc$, $K_{\sk{\lc}\sk{\rc}}$ denotes a submatrix
% of $K_{\lc\rc}$, and $P_{\sk{\lc}\lc}$, $P_{\sk{\rc}\rc}$ are
% matrices of coefficients used to interpolate other entries.
% For an inner node $\alpha$, its skeleton
% $\sk{\alpha} \subset \sk{\lc} \cup \sk{\rc}$ is subselected from its
% children's skeletons. With this nested basis, the parent 
% coefficient matrix $P_{\sk{\alpha}\alpha}$ can be computed as a
% linear combination of two children's coefficients as shown in the box
% \begin{equation}
% \sk{w}_{\alpha} = 
% \framebox[1.1\width]{
% $P_{\sk{\alpha}\alpha}$
% }w_{\alpha}
% =
% \framebox[1.1\width]{$
% P_{\sk{\alpha}[\sk{\lc}\sk{\rc}]}
% \begin{bmatrix}
%   P_{\sk{\lc}\lc} & \\
%                   & P_{\sk{\rc}\rc} \\
% \end{bmatrix}
% $}
% \begin{bmatrix}
% w_{\lc} \\
% w_{\rc} \\
% \end{bmatrix}=
% P_{\sk{\alpha}[\sk{\lc}\sk{\rc}]}
% \begin{bmatrix}
%   \sk{w}_{\lc} \\
%   \sk{w}_{\rc} \\
% \end{bmatrix}
% \label{e:telescope}
% \end{equation}
% While multiplying weight $w$ from the right, \eqref{e:telescope}
% allows us to compute
% the skeleton weights $\sk{w}_{\alpha}$ for all $\alpha$ in 
% linear time with a postorder traversal (\emph{telescoping}). 
% If we assume there is only a linear number of submatrices $K_{\beta\alpha}$, then
% accumulating the approximate sum for each node $\beta$ as
% \begin{equation}
%   \sk{u}_{\beta} = 
%   \sum_{\alpha \in Far(\beta)} K_{\sk{\beta}\sk{\alpha}}
%   \sk{w}_{\alpha} = 
%   \sum_{\alpha \in Far(\beta)} K_{\sk{\beta}\sk{\alpha}}
%   P_{\sk{\alpha}\alpha}
% \end{equation}
% takes linear time. Similarly, we can apply the same \emph{telescoping} 
% relation from the left (preorder) such that all skeleton potentials 
% \begin{equation}
%   \begin{bmatrix}
%     \sk{u}_l \\
%     \sk{u}_r \\
%   \end{bmatrix} +=
% \framebox[1.1\width]{$
% \begin{bmatrix}
%   P_{\sk{\lc}\lc} & \\
%                   & P_{\sk{\rc}\rc} \\
% \end{bmatrix}^{T}
% P_{\sk{\beta}[\sk{\lc}\sk{\rc}]}^{T}
% $}
% \sk{u}_{\beta}
% \end{equation}
% can also be computed in linear time. Finally, we assume that
% the size of $Near(\beta)$ is constant for each leaf node.
% Then computing $u_{\beta} = \sk{u}_{\beta} + \sum_{\alpha \in
% Near(\beta)}K_{\beta\alpha}w_{\alpha}$ for all leaf node $\beta$
% can also be done in linear tme.
% 
% Overall, FMM assumes that $Near(\alpha)$ and $Far(\alpha)$ 
% are both $\MA{O}(1)$ and uses nested basis on both sides to achieve
% linear time complexity.
% For hierarchical partitioning directly derived from the geometric space, this 
% near-far pruning may be straight forward.
% However, so far we have not yet defined the neighbors 
% ($\bigstar$) and how to partition $K$ as a tree for an arbitrary 
% SPD matrix. 
% In \secref{s:seq}, we illustrate an idea that generalizes this
% near-far pruning to general SPD matrices using the Gramian vector space.

%%----------------------------------------------------------------------------------------------------
\section{Reddit administrative actions come loss}
Content moderation is seen a commodity rather than a necessity provided by the social media for its platform. It helps to maintain a welcoming and civil platform for new users to join. Moderation also creates a platform favourable for advertisers and investors. There are cases where \note{Cite YouTube apocalypse, sleeping giants, twitter (for inversters)} advertisers and investors withdrew their interest from a platform entirely because of the bad image of the platform due to inappropriate content and lack of content moderation. The lack of initiative in moderation from Reddit is apparent in how its content policy grew over the years, driven mainly by criticism from media and users. We hypothesize the delay or lack of urgency in performing administrative actions is caused by the cost of the interventions. In their work, et al. show how platforms profit greatly from toxicity and extremist ideologies. Following this pattern, we aim to measure the revenue these communities bring and the revenue lost from closing these communities. Establishing a high cost to administrative actions will enable us understand would enable us to understand the lack of moderation in newer platforms and the inconsistencies in Reddit’s moderation.

\subsection{Revenue Streams}
As a social media hosting user-generated content, platforms Reddit’s primary source of revenue is advertisement and then paid memberships. Advertisers can place their ad on Reddits sidebar or in the middle of feed. Being the 7th most popular website in the US Reddit sees a lot of traffic with 330 million monthly active users \note{cite}. Valued at \$6 billion with a revenue of \$226 million for 2021 \note{cite} Reddit’s ARPU (Average Revenue per User) is estimated to be around \$0.68. Compared to \$7 and \$9 ARPU of Facebook and Twitter Reddit’s is still growing its business model to catch up. Apart from advertising, a smaller part of Reddit’s revenue comes from selling in-site currency: \textit{Gold}, \textit{Silver} and \textit{Platinum} coins. Users can buy these coins and give them to submissions or comments as awards. Given these two revenue streams, in the next section we highlight how we design experiments to measure the cost of interventions.

\subsection{Methods}
In testing the cost of intervention, we break the analysis into two categories 1) \textbf{Before Intervention}: how much revenue/activity were the closed communities bringing and 2) \textbf{After Intervention}: how revenue was lost after closing the communities. In measuring the revenue brought by a closed community before their closure we measure three values: 1) total amount of activity on the particular community, 2) total amount of Reddit currency i.e. coins spent on the community and 3) the number of users brought to the platform by the community.  

\subsection{Method}
list of methods:

\textbf{How much revenue was being made:}
\begin{itemize}
    \item How many users came to reddit for this subreddit
    \item How much gold/reddit currency was distributed here.
    \item How much user activity was generated here daily \\
\end{itemize}

\textbf{How much revenue was lost:}
\begin{itemize}
    \item Qualitative analysis of migration (searching for other platforms names in these communities)
    \item Loss of number of users.
\end{itemize}

\note{talk about how generally moderations can be costly.}


Table:

\begin{center}
\begin{tabular}{ c | c |  c | c | c}
 & \textbf{Banned} & \textbf{Toxic} & \textbf{Control} & \textbf{All} \\
 \midrule
Activity 		& 150M	& XM	& YM 	& KM	\\
Gold	 		& 3.2K	& 2.6K	& 1.3K 	& 250K	\\
Gold per Community 	& 55	& 60	& 43 	& 49	\\
Incoming 		& 1M	& 550K	& 410K 	& -	\\
\end{tabular}
\end{center}

\note{literature review on user migration}
\note{List all the different platforms that users might have migrated to, search their names in all platforms and then look at their revenue}

%%----------------------------------------------------------------------------------------------------
\input{complexv2.tex}
%%----------------------------------------------------------------------------------------------------
\section{Scalable Representations for Communication Patterns}
\label{sec:design}

Using the lessons learned in our preliminary studies, along with existing case studies~\cite{isaacs2014combing, Isaacs2016} using idealized unit time, we design a set of strategies for representing communication patterns when there are too many PEs to draw distinct communication lines in Gantt charts. We first describe our design goals. Then, we present our designs. Finally, we discuss initial feedback from experts familiar trace analysis in HPC.


\subsection{Design Goals}

Our goal is to design a representation of communication in execution traces that (1) aids users in recognizing and understanding what communication is occurring in that temporal and logical position in the Gantt chart and (2) is agnostic to the number of processing elements, thereby scaling to larger traces. These goals are derived from usage and scalability limitations noted in prior work~\cite{isaacs2014combing}. 

We limit our focus to scaling in PEs (y-axis) rather than time. Traces are typically explored using a time window, so we focus on that case. Adapting a design or creating a new one for compressed time settings we leave for future work.

Based on our preliminary study (\autoref{sec:prelim}), we chose to focus on offset, ring, and exchange pattern types as stencils require more design consideration even at small scales. 

\subsection{Visualization Design}

Our design process began with open brainstorming on paper, which we include in the supplemental material. We tried a variety of strategies, including linked views and added channels to the traditional Gantt chart encoding rules. However, most of these retained scaling problems, leading us to focus on designs centering on glyphs.

In designing the representation, we considered the saliency of what was to be encoded (e.g., temporal range, pattern type, grouping, stride) and efficacy of available channels, taking into account that the design needs to be incorporated in a Gantt chart. For example, temporal range is set to a horizontal position matching where a pattern would be drawn in a full chart. See supplemental materials for a table containing discussion of channel considerations.

We prioritize the type of pattern before the grouping factor or stride. The rationale is that the pattern type is fixed by the source code while the grouping and stride are often computed from the problem size and number of resources. Therefore, a user will recognize pattern type first before considering other factors. \autoref{fig:abstract_designs} shows the resulting designs.


\begin{figure*}
    \centering
    \begin{subfigure}{0.18\textwidth}
         \centering
         \includegraphics[width=\textwidth]{figures/new-basic-offset.png}
         \caption{Continuous offset pattern}
         \label{fig:noc}
    \end{subfigure}
    \begin{subfigure}{0.18\textwidth}
         \centering
         \includegraphics[width=\textwidth]{figures/new-basic-offset-grouped.png}
         \caption{Grouped offset pattern}
         \label{fig:nog}
    \end{subfigure}
    \begin{subfigure}{0.18\textwidth}
         \centering
         \includegraphics[width=\textwidth]{figures/new-basic-ring.png}
         \caption{Continuous ring pattern}
         \label{fig:nrc}
    \end{subfigure}
    \begin{subfigure}{0.18\textwidth}
         \centering
         \includegraphics[width=\textwidth]{figures/new-basic-ring-grouped.png}
         \caption{Grouped ring pattern}
         \label{fig:nrg}
    \end{subfigure}
    \begin{subfigure}{0.18\textwidth}
         \centering
         \includegraphics[width=\textwidth]{figures/new-basic-exchange.png}
         \caption{Exchange pattern}
         \label{fig:neg}
    \end{subfigure}
    \caption{Examples of our designs for five communication patterns. They are reminiscent of the underlying communication pattern encoding, but not aligned to the underlying chart and agnostic to the number of rows the underlying pattern repeats over. 
    %Note that the angle for our ring pattern is slightly shallower than the angle for offset, this reflects the difference in stride between the two patterns. 
    Grouped representations fill the vertical space to indicate that the repetition continues from the top of row to the bottom.}
    \label{fig:abstract_designs}
\end{figure*}

\vspace{1ex}

\textbf{Encoding Pattern Type.} To encode the pattern type, we started with the overall shape of of the pattern when drawn at small scale with a small stride. Offsets are drawn with angled repeating lines forming a rhombus-like shape. We use a fixed distance between lines and draw as many will fit in the relevant area.

Rings add indicators of the ``wrap-around'' communication. However, unlike fully drawn rings, we only render the protruding segments at the ends of the shape. There are two main rationales for only drawing protruding segments: (1) we want to indicate this is an abstraction and (2) participants in our interviews found the crossing lines difficult to disambiguate. The number of protruding segments is proportional to the stride of a ring. 
%For a ring with a small stride, one segment is added. This increases to a max of four for a very large stride.

Exchange patterns are drawn as a series of symmetrical ``x'' shapes and avoid direct crossings for the same reason as rings. The number of lines in each cross is proportional to stride of the exchange. Short stride exchanges will exchange between only a few PEs, a long stride exchange spans many PEs. Our glyphs approximate this by increasing the number of crossing lines as stride increases.

\vspace{1ex}

\textbf{Encoding Grouping Factor.} To represent grouping, we partition the available area vertically and repeat the pattern type drawing in those partitions. More formally, the encoding rule to show ``grouping" is repetition and alignment on a non-common scale. The number of partitions is determined by the available vertical space in a chart.


\vspace{1ex}

\textbf{Encoding Stride.} We express the notion of stride through the angle of lines used in our pattern types. As people had difficulty with steeply angled lines in our preliminary study, we limit the angles to a range of 15 degrees to 60 degrees. Therefore, these do not match the encoding of a full view. Instead, they hint at the magnitude of distance over which communication is occurring. This allows users to see that there are differences in stride between glyphs, but not necessarily calculate the exact stride visually.

\vspace{1ex}

\textbf{Temporal range.} Rather than show the exact range, we place the glyphs on the x-axis so they are centered in their range. If two structures overlap, they are placed alongside one another. 

\vspace{1ex}

\textbf{Incorporation in Gantt Charts.} These are designed to be used in Gantt charts when exact lines would be too dense to be interpreted. The underlying interval rectangles will still be drawn. The color encoding of these intervals was shown to be a secondary indicator in our preliminary study, so we preserve them. We add a slight blur effect to the background as another signifier that the glyphs are an abstraction and should not be confused for exact lines.



\subsection{Expert Feedback}
\label{sec:expertfeedback}

We sent our designs to two HPC experts for feedback regarding both the designs themselves and the overall approach. Both experts were familiar with idealized unit time representations of traces. The first expert, E1, had previously collaborated on this strategy with the authors but was not involved in any of the work presented here. The second expert, E2, had managed an integration of the strategy into an HPC center's performance tools, referencing the open-source research code~\cite{isaacs2014combing} but using an alternate calculation method and front-end technology.

We sent both experts a short email with a PDF describing the visualizations with comparisons to fully drawn traces and showing how they might be applied in practice, including a few complicated examples such as zoomed-out time and idle processes. (See supplemental materials.) We asked if and how the strategy would be useful and if there were any suggestions or concerns. E1 responded the designs ``definitely look helpful,'' noted the trade off in exactness, and then pointed out figures which led to ambiguities in his view. He also identified a error where the mock-up did not match the underlying trace. E2 noted that stride is less important and wondered how the translation from data to glyph would be calculated. He suggested the strategy might also be helpful for collective communications (e.g., broadcasts, all-to-all, reductions), a set of patterns we did not consider in this work.

We interpreted these responses to suggest the designs were worth further study, particularly E1's ability to interpret well enough to detect an error and E2's interest in further patterns. However, there are design decisions in applying these glyphs in some scenarios, particularly in zoomed-out time, that require refinement. We leave these cases for future iterations and instead focus on how the base designs could be interpreted by a wider range of users in a controlled study.
%%----------------------------------------------------------------------------------------------------

\begin{comment}
\begin{figure}
\includegraphics[width=\linewidth]{figs/beyond_tss_lesion.pdf}
\caption[]{End-to-End runtime lesion study of the entire MNIST dataset and the FMA featurized music dataset. Each of DROP's contributions provides a runtime improvement.}
\label{fig:beyond_lesion}
\end{figure}
\end{comment}



\section{Conclusion}
\label{sec:conclusion}

Advanced data analytics techniques must scale to rising data volumes. 
DR techniques offer a powerful toolkit when processing these datasets, with PCA frequently outperforming popular techniques in exchange for high computational cost. 
In response, we propose DROP, a new dimensionality reduction optimizer. 
DROP combines progressive sampling, progress estimation, and online aggregation to identify high quality low dimensional bases via PCA without processing the entire dataset by balancing the runtime of downstream tasks and achieved dimensionality. 
Thus, DROP provides a first step in bridging the gap between quality and efficiency in end-to-end DR for downstream \red{analytics}. 

%We revisit canonical operators for time series dimensionality reduction and the measurement study of~\cite{keogh-study}, and show that PCA is more effective than popular alternatives in the data mining literature often by a margin of over $2\times$ on average on gold-standard time series benchmark data sets with respect to output data dimension. More surprisingly, we empirically demonstrate that a small number of samples are sufficient to accurately characterize directions of maximum variance and obtain a high-quality low-dimensional transformation.




% Acknowledgments
\section{Acknowledgements}

Luca Herranz-Celotti was supported by the Natural Sciences and Engineering Research Council of Canada through the Discovery Grant from professor Jean Rouat, and by CHIST-ERA IGLU. We thank Compute Canada for the clusters used to perform the experiments and NVIDIA for the donation of two GPUs. We thank Wolfgang Maass for the opportunity to visit the Institute of Theoretical Computer Science, Guillaume Bellec, Darjan Salaj and Franz Scherr, for their invaluable insights on learning with surrogate gradients, and Maryam Hosseini, Ahmad El Ferdaoussi and Guillaume Bellec for their feedback on the article.


% Bibliography 
\bibliographystyle{unsrt} 
\bibliography{mybib} 
%\medskip
% Appendix
%\begin{appendices}
%The choice of proper functions $\phi_e(\cdot)$ combined with the $\lambda$-$\mu$  smoothness  framework for latency functions~\cite{roughgarden2015intrinsic} enables us to strictly improve the social cost compared to that of $1$-PNE, thus bounding PoS($\theta$-PNE) away from PoA($1$-PNE). Following the ideas in~\cite{christodoulou2011performance}, we present a structured method to find good modified latency functions and extend the PoS bounds to the class of $M/M/1$ latency functions, which is commonly used in modeling congestion networks.


Given a standard latency function $\ell(\cdot)$ and a range $\mc{R}$, consider the class of functions $\mc{L}(\ell, \mc{R}) =\{ \phi(\cdot): \phi(\cdot) \text{ is standard}, \ell(x)/\theta \le \phi(x) \le l(x),~\forall x \in \mc{R}\}$.
Further, given a multi-commodity network $\mc{G}$ with total demand $d_{tot}$, define the class of new potential functions,
\begin{flalign}
\label{eq:newPotential}
\bm{\Phi}(\mc{G})=\left\{\sum_{e \in E} \int_{x=0}^{x_e}\phi_e(x) dx: \phi_e(\cdot)\in  \mc{L}(\ell, [0, d_{tot}])\right\}.
\end{flalign}

The following result from \cite{christodoulou2011performance} characterizes the $\theta$-PNE in $\mc{G}$.
\begin{proposition}\label{thm:NewNE}
Given a multi-commodity network $\mc{G}$, a feasible flow $\bm{x}$ is a $\theta$-PNE if it minimizes some potential function $\Phi(\bm{x}) \in \bm{\Phi}(\mc{G})$.
\end{proposition}

\smallskip\noindent\textbf{ PoS Upper Bounds for composite functions.}
Consider the class of latency functions represented as $\ell(x)= \sum_i a_i\ell_i(x)$ where $a_i\geq 0$ for all $i$. Let the total demand in the network be $d=\sum_k d_k$.  We can find an upper bound for PoS through the following procedure:

\begin{itemize}
\item[1.] For each $i$, guess a suitable form of function $\phi_i(x, \psi_i)$, where $\psi_i$ is an appropriately chosen parameter. Represent $\phi(x)= \sum_i \xi_i a_i\phi_i(x, \psi_i)$  for $\xi_i \in [1/\theta,1]$.
\item[2.] For each $i$, obtain the set 
\begin{align*}
\Psi_i(\theta,\xi_i) = \{\psi: \xi_i\phi_i(x, \psi) \in [\ell_i(x)/\theta, \ell_i(x)],~\forall x \in [0,d] \}.
\end{align*}
\item[3.] For each $i$, obtain the set 
\begin{align*}
\Lambda_i(\psi) = \{(\alpha, \beta): y \phi_i(x, \psi) \leq \alpha x \phi_i(x, \psi)  + \beta y \phi_i(y, \psi),~\forall x,y \in [0,d]\}.
\end{align*}
\item[4.] Solve the following optimization problem,
\begin{align*}
\label{eq:designPoS}
PoS(\theta)= \min \left\{\frac{\beta_p}{1-\alpha_p}:  \frac{1-\alpha_p}{1-\alpha_i} \leq \xi_i \leq \frac{\beta_p}{\beta_i},(\alpha_i,\beta_i) \in \Lambda_i(\psi_i), \psi_i \in \Psi_i(\theta,\xi_i), \xi_i \in [1/\theta,1], ~\forall i\right\}.
\end{align*}
\end{itemize}





\smallskip\noindent\textbf{$\bm{M/M/1}$ Delay functions.}
Consider latency functions in the class $\mathcal{D} =\{1/(u-x): u\geq u_{min}\}$, where $u$ is the capacity of the link and $x$ is the flow through the link. The term $u_{min}$ refers to the minimum capacity in the latency class. Further for each function the maximum load is given as $\rho = d/u$ and therefore, $\rho_{max} = d/ u_{min} < 1$ denotes the maximum possible load over the entire class. This is the class of $M/M/1$ delay functions which plays an important role in modeling congestion networks. 
  
\begin{lemma}
The PoS for the latency functions in class $\mathcal{D}$ for $\theta$-PNE,  $\theta \geq  1$, is upper bounded as
\begin{align*}
 \text{PoS}(\theta\text{-PNE};\mathcal{D})\leq \frac{1}{2}\left( 1+ \frac{1}{\sqrt{1- \rho_{\max}(\theta)}}\right),
\end{align*}
where $\rho_{\max}(\theta) = \max \{0, 1-\theta(1-\rho_{\max})\}$. Moreover, if $\theta \geq 1/(1-\rho_{max})$, the PoS of the network becomes $1$. 
\label{lemm:MM1}
\end{lemma}
\begin{proof}
Step 1: Consider the original function $\ell(x;u) = 1/(u-x)$ and the new functions $\phi(x;a,u) = 1/(u-ax)$ for some $a\in \mathbb{R}_{+}$.  Call the class of modified functions, $\mc{D}_{a} = \{\phi(x;a,u): u \geq u_{min}\}$.  

Step 2: Define the set $\Psi(\theta)= \{a:  a\rho\in [\max\{0,(1-\theta(1-\rho))\}, 1 ]\}.$ 

For $a\in \Psi(\theta)$, we have, $ \ell(x;u)/\theta \leq \phi(x;a,u)\leq \ell(x;u) $, for all $u$ and for all  $0\leq x\leq d$. 

The solution to the program that minimizes \eqref{eq:newPotential} is the Nash equilibrium under the latency functions $\phi_e(x;a_e,u_e)$. But, from Proposition~\ref{thm:NewNE}, the solution is a $\theta$-PNE for the original system with functions $\ell_e(x;u_e)$, if all $a_e\in \Psi(\theta)$. Therefore, the price of anarchy (PoA) under latency functions of the class $\mc{D}_{a}$ for all $a\in \Psi(\theta)$, gives upper bounds for the PoS for $\theta$-PNE. 

Step 3: 
The class of functions $\mc{D}_{\phi}$ assumes the same form but changes the maximum load on the system compared to $\mc{D}$. Specifically, for any fixed $a\in  \Psi(\theta) $, we can have the following relation true for $(\alpha,\beta)\in \Lambda(a)$,
\begin{align*}
y\phi(x; a, u) \le \alpha x\phi(x; a,u) + \beta y\phi(y; a, u),& \qquad \forall x,y \in [0,d].
\end{align*} 
Here through some basic calculus we can find out that if the inequality holds on the boundary of $[0,d]^2$ then it holds for the entire region. We obtain that for the boundary $y=0$ the inequality  is always true and for the boundaries $x = 0$ and  $y = d$, the  inequality holds for $\beta \geq 1$ (necessary and sufficient). Further, for $x=d$ the condition,  $4a\rho \alpha  \geq(1+a\rho\alpha -  \beta(1-a\rho))^2$, is  both necessary and sufficient. Therefore, we have $$\Lambda(a) = \{(\alpha,\beta): \alpha\in[0,1),  \beta\geq 1,  4a\rho \alpha  - (1+a\rho\alpha - \beta(1-a\rho))^2\geq 0 \}.$$

Step 4: We can now get the PoS upper bound after minimizing $$\min \{\beta/(1-\alpha): (\alpha,\beta) \in \Lambda(a), a \in \Psi(\theta)\}.$$ 

After the optimization, we get that the minimum is $\frac{1}{2}\left( 1+ \frac{1}{\sqrt{1- \rho(\theta)}}\right)$, where $\rho(\theta) = \max \{0, 1-\theta(1-\rho)\}$ and the choice of $a= \rho(\theta)/\rho$. Finally, taking maximum over all possible $\rho$ we obtain  $\text{PoS}(\theta\text{-PNE};\mathcal{D})\leq \frac{1}{2}\left( 1+ \frac{1}{\sqrt{1- \rho_{\max}(\theta)}}\right)$, where $\rho_{\max}(\theta) = \max \{0, 1-\theta(1-\rho_{\max})\}$.

\end{proof}




%\end{appendices}

\end{document}
% End of v2-acmsmall-sample.tex (March 2012) - Gerry Murray, ACM


