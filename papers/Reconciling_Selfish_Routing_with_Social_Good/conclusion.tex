\section{Conclusion}
In this article, we investigated the specific roles played by edge flows and path flows in achieving fairness in traffic routing without compromising on the social welfare too much. To this end we differentiated between `used' paths and `positive' paths. The former relates to paths with non zero flow under a given path flow, while the latter relates to paths with non zero flow on each edge under a given edge flow. The understanding of these two new flows led us to new solution concepts which generalize the classic Nash  equilibrium in routing games. Specifically, we defined positive Nash equilibrium (PNE) as an edge flow where the length of any `positive' path for any commodity is less than or equal to the length of any path of the same commodity.  Substituting `positive' paths with `used' paths in the definition of PNE gives us the concept of used Nash equilibrium (UNE). Relaxing the conditions further, we obtained envy free (EF) flows where for each commodity all `used' paths have equal length (in particular, this concept allows for the existence of unused paths of shorter length). In the spirit of approximate Nash equilibria, we considered the approximate versions of these solution concepts, $\theta$-PNE, $\theta$-UNE and $\theta$-EF, for some constant $\theta\geq 1$. Each of them yielded as a by-product a $\theta$-fair flow, under the fairness definition in Correa et al.~\cite{correa2007fast}.  However, we note that depending on the users' affinity towards selfishness and their knowledge of the network congestion one of these solution concepts might be more relevant than the others.  
  
We explored the interrelations among these flows building a reasonably complete hierarchy among them. Further, using the well developed framework of variational inequalities, we analyzed the price of anarchy (PoA) and price of stability (PoS) of $\theta$-PNE, $\theta$-UNE and $\theta$-EF flows. The results for PoA and PoS successfully encapsulate all possible instances of a multi-commodity network with latency functions from a given class. However, for a particular instance they fail to quantify the social cost efficiently.  
We then investigated the computational complexity of
%In a related direction of constrained minimization in routing games~\cite{jahn2005system} we tried to understand the computational complexity of 
finding $\theta$-PNE, $\theta$-UNE and $\theta$-EF flows with lowest social cost.  We proved that finding the $\theta$-UNE or $\theta$-EF with lowest social cost is NP-hard and remarked on the complexity of finding optimal $\theta$-PNE flows, which remains open.  
To circumvent these negative results, we then connected two existing approaches for related problems, namely, 1) bounded tolls~\cite{bonifaci2011efficiency} and  2) modified potential function~\cite{christodoulou2011performance}, to design edge flows which are $\theta$-PNE and have low social cost. Finally, we proposed a randomized routing where a central planner assigns a route to a user randomly and guarantees that the induced flow is $1$-EF flow `in expectation'. In fact, this technique can effectively induce the socially optimal flow as a $1$-EF flow `in expectation' but  possibly with large variance. The natural way to bound the variance is using a $\theta$-EF flow. Unfortunately, as of now we can not compute a $\theta$-EF or $\theta$-UNE flow directly and have to take recourse to $\theta$-PNE. We prove bounds on the variance of the length of the route assigned to the user by the randomized routing when starting from one of $\theta$-PNE, $\theta$-UNE or $\theta$-PNE.
         
We leave the following open problems and future directions, which can be used to design balanced flows with good social cost and fairness `in expectation': 
\begin{enumerate}
\item What is the computational complexity of (P3), i.e. calculating a $\theta$-PNE with the lowest social cost? Lemma~\ref{lemma:3Easy} shows how our technique fails to resolve the hardness in this case.
\item Can we design approximation algorithms to generate $\theta$-UNE or $\theta$-EF with near optimal social cost?
\item How can we formalize the notion of fairness in the presence of randomization in routing algorithms? 
\end{enumerate}   


