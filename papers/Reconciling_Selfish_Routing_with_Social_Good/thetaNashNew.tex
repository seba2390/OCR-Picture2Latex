The choice of proper functions $\phi_e(\cdot)$ combined with the $\lambda$-$\mu$  smoothness  framework for latency functions~\cite{roughgarden2015intrinsic} enables us to strictly improve the social cost compared to that of $1$-PNE, thus bounding PoS($\theta$-PNE) away from PoA($1$-PNE). Following the ideas in~\cite{christodoulou2011performance}, we present a structured method to find good modified latency functions and extend the PoS bounds to the class of $M/M/1$ latency functions, which is commonly used in modeling congestion networks.


Given a standard latency function $\ell(\cdot)$ and a range $\mc{R}$, consider the class of functions $\mc{L}(\ell, \mc{R}) =\{ \phi(\cdot): \phi(\cdot) \text{ is standard}, \ell(x)/\theta \le \phi(x) \le l(x),~\forall x \in \mc{R}\}$.
Further, given a multi-commodity network $\mc{G}$ with total demand $d_{tot}$, define the class of new potential functions,
\begin{flalign}
\label{eq:newPotential}
\bm{\Phi}(\mc{G})=\left\{\sum_{e \in E} \int_{x=0}^{x_e}\phi_e(x) dx: \phi_e(\cdot)\in  \mc{L}(\ell, [0, d_{tot}])\right\}.
\end{flalign}

The following result from \cite{christodoulou2011performance} characterizes the $\theta$-PNE in $\mc{G}$.
\begin{proposition}\label{thm:NewNE}
Given a multi-commodity network $\mc{G}$, a feasible flow $\bm{x}$ is a $\theta$-PNE if it minimizes some potential function $\Phi(\bm{x}) \in \bm{\Phi}(\mc{G})$.
\end{proposition}

\smallskip\noindent\textbf{ PoS Upper Bounds for composite functions.}
Consider the class of latency functions represented as $\ell(x)= \sum_i a_i\ell_i(x)$ where $a_i\geq 0$ for all $i$. Let the total demand in the network be $d=\sum_k d_k$.  We can find an upper bound for PoS through the following procedure:

\begin{itemize}
\item[1.] For each $i$, guess a suitable form of function $\phi_i(x, \psi_i)$, where $\psi_i$ is an appropriately chosen parameter. Represent $\phi(x)= \sum_i \xi_i a_i\phi_i(x, \psi_i)$  for $\xi_i \in [1/\theta,1]$.
\item[2.] For each $i$, obtain the set 
\begin{align*}
\Psi_i(\theta,\xi_i) = \{\psi: \xi_i\phi_i(x, \psi) \in [\ell_i(x)/\theta, \ell_i(x)],~\forall x \in [0,d] \}.
\end{align*}
\item[3.] For each $i$, obtain the set 
\begin{align*}
\Lambda_i(\psi) = \{(\alpha, \beta): y \phi_i(x, \psi) \leq \alpha x \phi_i(x, \psi)  + \beta y \phi_i(y, \psi),~\forall x,y \in [0,d]\}.
\end{align*}
\item[4.] Solve the following optimization problem,
\begin{align*}
\label{eq:designPoS}
PoS(\theta)= \min \left\{\frac{\beta_p}{1-\alpha_p}:  \frac{1-\alpha_p}{1-\alpha_i} \leq \xi_i \leq \frac{\beta_p}{\beta_i},(\alpha_i,\beta_i) \in \Lambda_i(\psi_i), \psi_i \in \Psi_i(\theta,\xi_i), \xi_i \in [1/\theta,1], ~\forall i\right\}.
\end{align*}
\end{itemize}





\smallskip\noindent\textbf{$\bm{M/M/1}$ Delay functions.}
Consider latency functions in the class $\mathcal{D} =\{1/(u-x): u\geq u_{min}\}$, where $u$ is the capacity of the link and $x$ is the flow through the link. The term $u_{min}$ refers to the minimum capacity in the latency class. Further for each function the maximum load is given as $\rho = d/u$ and therefore, $\rho_{max} = d/ u_{min} < 1$ denotes the maximum possible load over the entire class. This is the class of $M/M/1$ delay functions which plays an important role in modeling congestion networks. 
  
\begin{lemma}
The PoS for the latency functions in class $\mathcal{D}$ for $\theta$-PNE,  $\theta \geq  1$, is upper bounded as
\begin{align*}
 \text{PoS}(\theta\text{-PNE};\mathcal{D})\leq \frac{1}{2}\left( 1+ \frac{1}{\sqrt{1- \rho_{\max}(\theta)}}\right),
\end{align*}
where $\rho_{\max}(\theta) = \max \{0, 1-\theta(1-\rho_{\max})\}$. Moreover, if $\theta \geq 1/(1-\rho_{max})$, the PoS of the network becomes $1$. 
\label{lemm:MM1}
\end{lemma}
\begin{proof}
Step 1: Consider the original function $\ell(x;u) = 1/(u-x)$ and the new functions $\phi(x;a,u) = 1/(u-ax)$ for some $a\in \mathbb{R}_{+}$.  Call the class of modified functions, $\mc{D}_{a} = \{\phi(x;a,u): u \geq u_{min}\}$.  

Step 2: Define the set $\Psi(\theta)= \{a:  a\rho\in [\max\{0,(1-\theta(1-\rho))\}, 1 ]\}.$ 

For $a\in \Psi(\theta)$, we have, $ \ell(x;u)/\theta \leq \phi(x;a,u)\leq \ell(x;u) $, for all $u$ and for all  $0\leq x\leq d$. 

The solution to the program that minimizes \eqref{eq:newPotential} is the Nash equilibrium under the latency functions $\phi_e(x;a_e,u_e)$. But, from Proposition~\ref{thm:NewNE}, the solution is a $\theta$-PNE for the original system with functions $\ell_e(x;u_e)$, if all $a_e\in \Psi(\theta)$. Therefore, the price of anarchy (PoA) under latency functions of the class $\mc{D}_{a}$ for all $a\in \Psi(\theta)$, gives upper bounds for the PoS for $\theta$-PNE. 

Step 3: 
The class of functions $\mc{D}_{\phi}$ assumes the same form but changes the maximum load on the system compared to $\mc{D}$. Specifically, for any fixed $a\in  \Psi(\theta) $, we can have the following relation true for $(\alpha,\beta)\in \Lambda(a)$,
\begin{align*}
y\phi(x; a, u) \le \alpha x\phi(x; a,u) + \beta y\phi(y; a, u),& \qquad \forall x,y \in [0,d].
\end{align*} 
Here through some basic calculus we can find out that if the inequality holds on the boundary of $[0,d]^2$ then it holds for the entire region. We obtain that for the boundary $y=0$ the inequality  is always true and for the boundaries $x = 0$ and  $y = d$, the  inequality holds for $\beta \geq 1$ (necessary and sufficient). Further, for $x=d$ the condition,  $4a\rho \alpha  \geq(1+a\rho\alpha -  \beta(1-a\rho))^2$, is  both necessary and sufficient. Therefore, we have $$\Lambda(a) = \{(\alpha,\beta): \alpha\in[0,1),  \beta\geq 1,  4a\rho \alpha  - (1+a\rho\alpha - \beta(1-a\rho))^2\geq 0 \}.$$

Step 4: We can now get the PoS upper bound after minimizing $$\min \{\beta/(1-\alpha): (\alpha,\beta) \in \Lambda(a), a \in \Psi(\theta)\}.$$ 

After the optimization, we get that the minimum is $\frac{1}{2}\left( 1+ \frac{1}{\sqrt{1- \rho(\theta)}}\right)$, where $\rho(\theta) = \max \{0, 1-\theta(1-\rho)\}$ and the choice of $a= \rho(\theta)/\rho$. Finally, taking maximum over all possible $\rho$ we obtain  $\text{PoS}(\theta\text{-PNE};\mathcal{D})\leq \frac{1}{2}\left( 1+ \frac{1}{\sqrt{1- \rho_{\max}(\theta)}}\right)$, where $\rho_{\max}(\theta) = \max \{0, 1-\theta(1-\rho_{\max})\}$.

\end{proof}



