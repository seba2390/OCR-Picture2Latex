\subsection{Related Work}
The natural question of balancing the social welfare and user satisfaction is essential to practical traffic routing. Starting from the seminal work of Koutsoupias and Papadmitriou~\cite{koutsoupias1999worst}, quantifying the worst case inefficiency of various non-cooperative games, including routing games, quickly became an intense area of research. In a routing game with arbitrary latency functions the ratio between the cost of a Nash equilibrium (NE) flow to the cost of a socially optimal (SO) flow may grow unbounded, as shown by Roughgarden et al.~\cite{roughgarden2002bad}. A series of papers have focused on developing techniques for bounding the inefficiency of the NE flow~(e.g., \cite{correa2008geometric, roughgarden2002bad, harks2007price}). 
%\evn{DO WE NEED TO SAY ``WITH INCREMENTAL IMPROVEMENT"? SOUNDS DIMINISHING OF PRIOR WORK}
The next natural generalization led us to consider approximate NE flows with the hope that there exists some such flow which may improve the social welfare. The theoretical analysis by Caragiannis et al.~\cite{caragiannis2006tight} for linear latency functions and later by  Christodoulou et al.~\cite{christodoulou2011performance} for polynomial latency functions, corroborated this intuition. 
  
In a related thread of research, Jahn et al.~\cite{jahn2005system} formalized the notion of constrained system optimal, where additional constraints were added along with the flow feasibility constraints. The additional constraints were introduced to reduce the unfairness of the resulting flow.  Further, useful insights were obtained by Schulz et al.~\cite{schulz2006efficiency} about the social welfare and fairness of these constrained system optimal flows. Recently, there have been efforts~\cite{bertsimas2011price, bertsimas2012efficiency} in quantifying the inefficiency needed to guarantee fairness among users. The authors here define the `price of fairness' as the proportional decrease of utility under fair resource allocation. As mentioned earlier, in routing games the fairness of socially optimal flows under different but related definitions has been studied by Roughgarden~\cite{roughgarden2002unfair} and Correa et al.~\cite{correa2004computational}. Further, Correa et al.~\cite{correa2007fast} consider the fairness and  efficiency of min-max flows, where the objective is to minimize the maximum length of any used path in the network.  They note how different path flows affect the fairness in the network even when the induced edge flows are identical. 
 
In mechanism design with fully informed users various approaches for attaining better social welfare have been proposed and studied extensively. In a seminal work Beckman et al.~\cite{beckmann1956studies} showed that using marginal tolls one can induce SO as NE under tolled cost functions. Since then the idea of toll placement has been further generalized and studied under various practical settings, e.g. bounded tolls~\cite{hoefer2008taxing, bonifaci2011efficiency, jelinek2014computing} and heterogeneous users~\cite{cole2003pricing, fleischer2004tolls}. A Stackelberg equilibrium, where a fraction of users is willing to cooperate and follow the routes dictated by the central planner, and its variations~\cite{karakostas2009stackelberg, swamy2012effectiveness} have also been studied as an alternative. However, as new technologies play a crucial role in shifting in user behavior, various incomplete information models have been introduced.  Acemoglu et al.~\cite{acemoglu2016informational} has discussed an informational Nash equilibrium where users converge to an equilibrium with partial knowledge of the network structure. In a related setup where each user's information is limited to a common prior on the latency functions, Vasserman et al.~\cite{vasserman2015implementing} considered a mediated Bayesian Nash equilibrium (BNE). Under an incentive compatible mediation strategy they studied the cost of the BNE in a parallel arc network and showed it is bounded by the number of edges. Other work has also studied mediated games with tolls~\cite{rogers2015inducing} and without tolls~\cite{kearns2014mechanism} where the focus has been truthful mechanism design using differential privacy techniques.  
 
%In a seminal work O. Jahn et al. formalized the notion of constrained system optimal in \cite{}. The authors imposed various constraints motivated by the convenience of users and tried to optimize the social cost. They introduced the notion of ``normal length'' where the constraint is based upon a  measure of latency on each edge, possibly flow independent. They defined four notions of (un)fairness. Namely, 1) `Free-flow' (un)fairness where the latency is the latency in empty network, i.e. $\ell_e(0)$ for edge $e$, 2) `Normal' (un)fairness where the latency is some prior estimate of the latency, $\tau_e$ for edge $e$,  3) `Loaded' (un)fairness where the latency is equal to the latency under current flow, $f$, $l_e(f_e)$ for edge $e$ and 4)`User equilibrium' (un)fairness where the latency is the latency under NE flow, $f^{NE}$, $l_e(f^{NE}_e)$ for edge $e$. We re-interpret the `Loaded' fairness as the `Envy Free' flow. Whereas the `User equilibrium' fairness is expressed as the `Incentive Compatible' flow adhering to the new notion of incentive design. We note that the notion of `User-equilibrium' was introduced by T. Roughgarden et al. in \cite{} as a way of quantifying the unfairness in the social optimal flow. 

%The work by O. Jahn et al. \cite{} majorly addressed the problem of computing the such a constrained system optimal under the given (un)fairness constraints. In this paper they introduced the idea of using Frank-Wolfe algorithm \cite{} with the aid of already developed algorithms for constrained shortest paths \cite{}. A non-formal argument towards the NP hardness of this problem was presented whereas empirical evidence showed the effectiveness of the Frank-Wolfe heuristics. In the theoretical side useful insights were obtained by A. Schulz et al. about the Social welfare property \cite{} and the fairness property \cite{} of these constrained system optimal. In a related direction J. Correa et al. \cite{} highlighted the effect of path decomposition in the fairness of a given flow motivating to pivot our problem around the path decomposition description. The authors studied problems of obtaining path decomposition with `nice' properties in networks with fixed edge flows. The results uncovered the intricacies and diversities the power of path decomposition has to offer even when the network is fixed as viewed in terms of edge flow. As a side note the authors also discuss about the `Min-Max' flows where given a network the problem is to minimize the maximum used path in a feasible path decomposition.     


