Selfish routing is one of the %first problems considered in algorithmic game theory and has been extensively 
most studied problems in algorithmic game theory, with one of the principal applications being that of routing in road networks. The majority of related work, in the many variants of the problem,  deals with the inefficiency of equilibria to which users are assumed to converge.
Multiple mechanisms for improving the outcomes at equilibria have been considered, such as the use of tolls or the use of Stackelberg strategies, each with different caveats in terms of their applicability to real traffic routing.  
But the emergence of routing technologies and autonomous driving motivates new solution concepts that can be considered as outcomes of the game and may help in improving the network's performance. In reality, when users ask their routing devices for good origin to destination paths, they care about the end-to-end delay on their paths without (directly) caring about subpath optimality. This gives a central planner the ability, through routing devices, to provide path flow solutions that circumvent the local subpath optimality conditions imposed by (approximate) Nash equilibria, while they are acceptable to the players and potentially have good social cost.
%

Inspired by the above observation, we consider three possible outcomes for the game: (i) $\theta$-Positive Nash Equilibrium flow,  where every path that has non zero flow on all of its edges has cost no greater than $\theta$ times the cost of any other path, (ii) $\theta$-Used Nash Equilibrium flow, where every path that appears in the path flow decomposition has cost no greater than $\theta$ times the cost of any other path, and (iii) $\theta$-Envy Free flow, where every path that appears in the path flow decomposition has cost no greater than $\theta$ times the cost of any other path in the path flow decomposition.
We first examine the relations of these outcomes among each other and then measure their possible impact on the network's performance, through the notions of price of anarchy and price of stability. Afterwards, we examine the computational complexity of finding such flows of minimum social cost and give a range for $\theta$ for which this task is easy and a range for $\theta$ for which, for the newly introduced concepts of $\theta$-Used Nash Equilibrium flow and $\theta$-Envy Free flow, this task is NP-hard. Finally, we propose deterministic strategies which, in a worst case approach,  can be used by a central planner in order to provide good such flows, and also introduce a natural idea for randomly routing players after giving them specific guarantees about their costs in the randomized routing, as a tool for the central planner to implement a desired flow.
  
