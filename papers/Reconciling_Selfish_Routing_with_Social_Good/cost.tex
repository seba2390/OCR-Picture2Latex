\section{Quality of $\theta$-Flows}\label{sec:social}
In this section, we analyze the cost and fairness of the solution concepts introduced in Section~\ref{sec:satisfaction}.
As noted in Section~\ref{sec:satisfaction}, the $\theta$ flows are not unique for $\theta>1$ and that implies that potentially under each solution concept we can have a range of attainable costs. We present the upper bounds on the PoS and PoA, defined respectively in \eqref{eq:PoA} and \eqref{eq:PoS},  for the flows under the three solution concepts. We compare the social cost of the $\theta$ flows with the socially optimal flow.
 
\textbf{Price of Anarchy.} Starting from Correa et al.~\cite{correa2008geometric},  there has been a unifying approach of bounding the PoA using the variational inequality formulation of the Nash equilibrium flow.  Christodoulou et al.~\cite{christodoulou2011performance}  extended the idea of using a variational inequality to formulate an approximate equilibrium, specifically a $\theta$-PNE, and to give new bounds for the price of anarchy of a $\theta$-PNE. We first show that the $\theta$ variational inequality encompasses both the $\theta$-PNE and $\theta$-UNE. Therefore, we can use the well established technique to give an upper bound for both types of flows.

\begin{lemma}\label{lemm:UNEvVI}
If a flow $\bm{f}$ is a $\theta$-UNE with edge flow $\bm{x}$, then it satisfies the following variational inequality for $\theta\geq 1$,
\begin{align}\label{eq:thetaVI}
\sum_e x_e\ell_e(x_e) \leq \theta\sum_e x'_e\ell_e(x_e), ~\forall \bm{x}'\in \mc{D}_E. 
\end{align}                                                                                                                           
Further, there exists a single commodity network and a flow $ \bm{f}'' \in \mc{D}_p(\bm{x}'')$ that satisfies the above inequality but is not a $\theta$-UNE.
\end{lemma}
\begin{proof}
The proof of the first part follows closely the proof of Theorem $1$ in Christodoulou et al~\cite{christodoulou2011performance}. Let $\bm{f} \in \mc{D}_p(\bm{x})$ be a $\theta$-UNE and  $\bm{f}' \in \mc{D}_p(\bm{x}')$ be any other feasible flow in the network. From the definition of $\theta$-UNE, for any commodity $k$, for any used path $p\in \mc{P}^k_u$ and for any other path $p'\in \mc{P}^k$ we have $\sum_{e\in p}\ell_e(x_e) \leq \theta \sum_{e\in p'}\ell_e(x_e).$  Further,  taking the summation of the flow weighted path latency over all pairs of paths in commodity $k$, we obtain the following:
\begin{align*}
\sum_{\substack{p\in \mc{P}^k_u(\bm{f})\\ p'\in \mc{P}^k(\bm{f}')}} f_p^k f_p^{k\prime}\sum_{e\in p}\ell_e(x_e) 
&\leq \theta\sum_{\substack{p\in \mc{P}^k_u(\bm{f})\\ p'\in \mc{P}^k(\bm{f}')}} f_p^k f_p^{k\prime}  \sum_{e\in p'}\ell_e(x_e), \\
\sum_{p'\in \mc{P}^k(\bm{f}')} f_p^{k\prime} \sum_{e\in E}x_e^k\ell_e(x_e) 
&\leq \theta \sum_{p\in \mc{P}^k_u(\bm{f})} f_p^k  \sum_{e\in E}x_e^{k\prime}\ell_e(x_e), \\
\sum_{e\in E}x_e^k\ell_e(x_e)&\leq \theta \sum_{e\in E}x_e^{k\prime}\ell_e(x_e).
\end{align*} 
The last inequality follows due to $\sum_{p'\in \mc{P}^k(\bm{f}')} f_p^{k\prime} = \sum_{p\in \mc{P}^k_u(\bm{f})} f_p^k = d_k>0$. Finally, taking summation over all commodities gives $\sum_e x_e\ell_e(x_e) \leq \theta\sum_e x'_e\ell_e(x_e)$.

Consider Pigou's network with demand $1$, top edge with latency $\ell_t(x) = \epsilon x$ and bottom edge with latency $\ell_b(x) = L$. For a given $\theta>1$, choose $\delta>0$ small (to be specified later).
% \in (0, \theta -1)
 Consider the flow $\bm{f}''$ equal to
 $(1-\delta\epsilon/L)$ in the top link and $\delta\epsilon/L$ in the bottom link. 
 The social cost of this flow is $\left(\delta\epsilon+\epsilon(1-\delta\epsilon/L)^2\right)$. The feasible flow minimizing the right hand side of Inequality \eqref{eq:thetaVI} is flow of $1$ through top link for $L>\epsilon$. Further, for any $\theta$, there is a $\delta$ small enough such that Inequality  \eqref{eq:thetaVI} holds, since, by the above, it suffices to have $\delta\epsilon+ \epsilon(1-\delta\epsilon/L)^2 \leq\theta \epsilon(1-\delta\epsilon/L)
\Leftrightarrow \theta\geq 1+ \frac{L^2\delta-L\delta \epsilon+\delta^2 \epsilon^2}{L(L-\delta\epsilon)} $. 
However, $\bm{f}''$ is not $(L/\epsilon)$-UNE and this approximation factor can be made arbitrarily large making $\epsilon$ small enough. 
\end{proof}

\begin{remark}
Here we have shown that the variational inequality is a sufficient condition for a flow to be $\theta$-UNE and the counterexample shows it is not necessary. Though due to Dafermos and Sparrow~\cite{dafermos1969traffic}, we know that for $\theta=1$ the variational inequality is the necessary and sufficient condition for $1$-PNE. This gives an alternative proof to the fact that $1$-UNE$=1$-PNE.
\end{remark}


For a fixed $\theta$, the flows satisfying Inequality~\eqref{eq:thetaVI} form a set. Call this set $\theta$-VI. %We define the PoA($\theta$, VI) similarly as in equation~\eqref{eq:PoA} over this set. 
The following lemmas characterize the price of anarchy for various flows under latency functions in class $\mc{L}$. We adopt the approach of Harks et al.~\cite{harks2007price} as it produces tighter PoA bounds for several latency functions compared to previous approaches~\cite{roughgarden2002selfish, correa2008geometric}. The result in~\cite{harks2007price} is for $\theta=1$ and here we state it for general $\theta$. 

We begin with a simple corollary to Lemma~\ref{lemm:UNEvVI} and omit the proof.
\begin{corollary}
For any multi-commodity instance $\mc{G}$ and any $\theta\geq 1$, the PoA values for the corresponding solution concepts are related as PoA($\theta$-PNE) $\leq $  PoA($\theta$-UNE) $\leq $  PoA($\theta$-VI).
\end{corollary}
  
We need the following definitions in order to bound PoA($\theta$-VI):
\begin{align*}
\omega(\mc{L}, \lambda) &= \sup_{\ell \in \mc{L}} \sup_{x,x'\geq 0}\frac{\left(\ell(x)-\lambda\ell(x')\right)x'}{x\ell(x)}.\\
\Lambda(\theta) &= \{\lambda\in \mathbb{R}^{+}: \omega(\mc{L}, \lambda)\leq  1/\theta \}.
\end{align*}


\begin{lemma}
For an instance $\mc{G}$ with latency functions in class $\mc{L}$, the PoA($\theta$-VI) is upper bounded by $ \inf_{\lambda \in \Lambda(\theta)} \theta\lambda(1-\theta\omega(\mc{L}, \lambda))^{-1}$. 
\end{lemma}
\begin{proof}
Let $\bm{x}$ be a $\theta$-VI flow satisfying Condition~\eqref{eq:thetaVI} and $\bm{y}\in SC_E$ be a socially optimal flow. Then, we have the following relations:
\begin{align*}
SC(\bm{x})= & \sum_e x_e\ell_e(x_e) \leq \theta\sum_e y_e\ell_e(x_e)\\
&\leq \theta\sum_e \left(y_e\ell_e(x_e)-\lambda y_e\ell_e(y_e) + \lambda y_e\ell_e(y_e)\right)\\
&\leq \theta  \omega(\mc{L}, \lambda) SC(\bm{x}) + \theta\lambda SC(\bm{y}).
\end{align*}
We obtain the desired bound by taking infimum over the set $\Lambda(\theta)$.
\end{proof}

\begin{example}
As an example consider the class of linear latency functions $\ell(x)= ax+b$.  For this class, we can obtain  $\omega(\mc{L},\lambda)\leq 1/4\lambda$ for $\lambda \geq 1$ and  $\omega(\mc{L},\lambda)>1$ otherwise. An upper bound on PoA($\theta$-VI) can be obtained through minimizing over the set $\Lambda(\theta)=\{ \lambda \geq \max\{1,\theta/4\}\}$. The exact bound obtained through this is $\max\{\theta^2, 4\theta/(4-\theta)\}$ and it matches the bounds given in~\cite{christodoulou2011performance}. Note that for $\theta=1$ it gives us the classical bound of $4/3$.
\end{example}

The following proposition further separates the envy free flows and the variational inequality characterization. 
\begin{proposition}
There exists a $1$-EF flow for which the variational inequality in \eqref{eq:thetaVI} does not hold for any bounded $\theta'$. Further, the PoA($1$-EF) is unbounded.
\end{proposition}
\begin{proof}
Consider the instance from Lemma~\ref{lemm:UNEvVI}, i.e. a Pigou network with demand $1$, top edge with latency $\ell_t(x) = \epsilon x$ and bottom edge with latency $\ell_b(x) = L$. The flow that routes $1$ unit through the bottom link (i.e., all the demand) is a $1$-EF flow but it does not satisfy Condition~\eqref{eq:thetaVI} for  any $\theta$, since under this flow the top link has cost 0.
For $L>2\epsilon$ the optimal flow routes all the flow through the upper link and thus the PoA of $1$-EF flows is $L/\epsilon$, which cannot be bounded as we can make $\epsilon$ arbitrarily small.
\end{proof}

\textbf{Price of Stability.} As discussed in the introduction, $\theta$-UNE and $\theta$-EF flows arise from the ability of a central planner to induce path flows in the network. The price of anarchy is motivated by the dynamics of users who can induce any worst case flow in the network under some given solution concept. In contrast, the price of stability is the quantity that is of special interest to the central planner, who wishes to induce the best (with respect to social cost) $\theta$-UNE or $\theta$-EF flow. Using already known techniques we can obtain bounds on the PoS but we defer this part to Section \ref{sec:designSubPoteFunc}, as these techniques will also be used to provide the central planner with good (with respect to social cost) $\theta$ fair flows, which is the scope of Section \ref{sec:design}.

\begin{lemma}\label{lemm:POSequality}
 For any multi-commodity network $\mc{G}$ and any $\theta\geq 1$, the PoS values of the corresponding flows are related as PoS($\theta$-EF) $\leq $  PoS($\theta$-UNE) $\leq $  PoS($\theta$-PNE). Moreover, there exists a network $\mc{G}$ such that for all $\theta\geq 1$ all the inequalities are tight.
\end{lemma}
\begin{proof}
The proof of the first part of the lemma follows due to Lemma~\ref{lemm:PNEvUNEvEF} and the fact that the infimum of a function over a set is less than or equal to the infimum of the same function over any subset of the set.

Consider the Pigou network with unit demand,  upper link having latency $\ell_u(x)= 1$ and  lower  link having latency $\ell_b(x) = x$. For this network and for any $\theta\geq 1$ the optimal $\theta$-PNE, $\theta$-UNE and $\theta$-EF are identical\footnote{For the special case of $\theta=1$, routing all the demand through the upper link is an optimal $1$-EF which is not a $1$-PNE or a $1$-UNE. Yet, the cost of this $1$-EF flow is the same as the $1$-PNE, $1$-UNE and $1$-EF flow that routes all the demand through the lower link. and given by $\max\{1/\theta, 0.5\}$ units of flow in the lower link and the remaining flow through the upper link. }
\end{proof}



\begin{remark}
%First of all, 
It is important to emphasize that the upper bound guarantees of PoS encompass all possible networks under a given latency class. Whereas, for a particular network the achievable social cost under $\theta$-fairness can be better compared to the bound dictated by PoS. As an example, consider the latency function class of polynomials of degree at most $p$ where the best possible upper bound for PoS($\theta$-PNE) is $\left(\theta\left(1-\frac{p\theta^{1/p}}{(p+1)^{(1+1/p)}}\right)\right)^{-1}$ for $\theta< p+1$~\cite{christodoulou2011performance}. On the other hand, as shown by Defarmos et al.~\cite{dafermos1969traffic}, if all the latency functions are monomials of degree $p$, the $1$-PNE is the socially optimal flow.
\end{remark}
 

