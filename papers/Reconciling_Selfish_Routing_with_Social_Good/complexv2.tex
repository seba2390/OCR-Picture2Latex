\section{Existence and Complexity}\label{sec:complex}
In this section we discuss the computational issues surrounding the three types of $\theta$ fair flows. The existence of Pure Nash equilibrium in nonatomic routing games guarantees the existence of any $\theta$ fair flow for $\theta\geq 1$. The next question would be whether we can compute $\theta$ fair flows with good social cost.  In particular, we consider the following problems:
\begin{enumerate}
	\item[(P1)] Find a $\theta$-EF path flow with the minimal social cost.
	\item[(P2)] Find a $\theta$-UNE path flow with the minimal social cost.
	\item[(P3)] Find a $\theta$-PNE edge flow with the minimal social cost.
\end{enumerate}
We show that for large $\theta$, the socially optimal flow is guaranteed to be contained in those $\theta$-flows, and hence the optimal $\theta$-flows be computed efficiently.  However, for small $\theta$, we will show that solving Problem~(P1) and Problem~(P2) is NP-hard, while it remains open whether Problem~(P3) can be computed efficiently.  
More precisely, for a latency class $\mc{L}$, this particular threshold is $\gamma(\mc{L})= \min\{\gamma: \ell^{*}(x)\leq \gamma \ell(x), \forall \ell\in \mc{L}, \forall x\geq 0\}$, where $\ell^{*}(x)= \ell(x)+x\ell'(x)$.  The main result of this section is given as follows:
\begin{theorem}
	%\begin{enumerate}
	%\item \fin{
	For any multi commodity instance $\mc{G}$ 
	with latency functions in any class $\mc{L}$, there are polynomial time algorithms\footnotemark $\, $  for solving Problem~(P1)-(P3) 
	for $\theta \ge \gamma(\mc{L})$. %, for any class $\mc{L}$.
	%
	On the other hand, it is NP-hard to solve Problem~(P1) for $\theta \in [1, \gamma(\mc{L}))$ and Problem~(P2) for $\theta \in (1, \gamma(\mc{L}))$, for  arbitrary single commodity instances %$\mc{G}$ 
	with latency functions in an arbitrary class $\mc{L}$.
	%\end{enumerate}
	\label{thm:main_hardness}
\end{theorem} 

\footnotetext{The existence of polynomial time algorithms for our problem depends on the assumption that we can minimize separable convex functions with linear constraints in polynomial time; numerical issues for convex optimization are discussed in \cite{hochbaum1990convex,  nemirovski2004interior} and are beyond the scope of our work.}
%on the existence of polynomial time algorithms for minimizing separable convex functions with linear constraints upto arbitrary precision. See~\cite{hochbaum1990convex,  nemirovski2004interior}   for details, which are quite technical for the scope of this paper.}}





In the following sections, we first prove the first part of Theorem \ref{thm:main_hardness}. Right after we show that, for any $\theta$, from any $\theta$ fair flow we may get another $\theta$ fair flow, which uses only polynomially many paths. This, on the one hand, serves as a clarification that the difficulty of problems~(P1)-(P3) does not lie in the size of their solutions. On the other hand, it helps in showing that the decision version of these problems lies in NP, since for a YES instance, a non-deterministic machine will (non-deterministically) choose a path flow of polynomial size and in polynomial time check that it satisfies the conditions needed. Finally, the second part of Theorem \ref{thm:main_hardness} follows from an NP-hardness proof for a stronger version of the decision versions of problems~(P1) and (P2) (Theorem \ref{thm:12Hard}).


\subsection{When the Social Optimum is Guaranteed to be the Solution}\label{sec:complexity_so}
First, we show that Problems~(P1)-(P3) are easy for $\theta \ge \gamma(\mc{L})$ because the social optimum is the solution.
%For a latency class $\mc{L}$, define $\gamma(\mc{L})= \min\{\gamma: \ell^{*}(x)\leq \gamma \ell(x), \forall \ell\in \mc{L}, \forall x\geq 0\}$. Here $\ell^{*}(x)= \ell(x)+x\ell'(x)$ is the marginal latency.  
The following lemma, which is a direct extension of Theorem~4.2 in Correa et al~\cite{correa2007fast}, shows that any path decomposition of the socially optimal flow is a $\gamma(\mc{L})$ fair flow, provided that the latency functions are in class $\mc{L}$.  While in the proof of Theorem~4.2 in Correa et al~\cite{correa2007fast} they only conclude that the set of socially optimal path flows is $\gamma(\mc{L})$-EF, it is easy to see that the same argument holds for $\gamma(\mc{L})$-PNE.
\begin{lemma}(\cite{correa2007fast})
For a network $\mc{G}$ with latency functions  in class $\mc{L}$, any socially optimal path decomposition $o \in SO_p$ is $\gamma(\mc{L})$-PNE.
\end{lemma}
 
Since the social optimum can be computed using convex programming~\cite{roughgarden2002selfish}, it follows that %With this result, we are able to give polynomial time algorithms for 
Problems~(P1)-(P3) can be solved in polynomial time\footnotemark[6] for $\theta\ge\gamma(\mc{L})$.
\begin{proof}[Proof, first part of Theorem~\ref{thm:main_hardness}]
Note that given a path flow, which is $\gamma(\mc{L})$-PNE, it is $\gamma(\mc{L})$-UNE and $\gamma(\mc{L})$-EF as well.  This means that for all $\theta \geq \gamma(\mc{L})$, we can simply compute the socially optimal flow, and give any path decomposition as the $\theta$ fair flow. The socially optimal edge flow can be computed in time polynomial in the size of the network.  Further, a greedy path decomposition suffices. In the greedy algorithm, at every step we pick the current minimum path (among all commodities) and assign the maximum possible flow, under the social optimum, through this path. This can be computed in time $O(|\mc{K}|\times|E|)$.  Also, the output path flow can be represented with a sparse vector with $O(|\mc{K}|\times|E|)$ entries.
%linear in number of edges times the number of commodities ).
\end{proof}



\subsection{Existence of Polynomial-size Path Flow Solutions}
An observation to Problem~(P1) and (P2) is that the outputs of these two problems are path flow vectors, which are potentially of exponential size relative to the problem instances. 
In Section~\ref{sec:complexity_so} we showed a way to compute a path flow vector with polynomial support under the social optimum.  Here we ask whether we can do this for any edge flow.  In particular, we are interested in whether we can always find an answer to either Problem~(P1) or (P2) using only polynomially many paths.  If not, then there is no hope for us to find an efficient algorithm for these problems.  In this subsection, we show that the answer to this question is \emph{yes}.  To see this, we make a more general argument than Lemma~3.1 in Correa et al.~\cite{correa2007fast}, showing that given any path flow vector, we can always find another path flow assignment of polynomial support that preserves four important properties.
\begin{proposition}\label{lemma:correa1}
	Let $\bm{f}$ be a feasible flow for a multicommodity flow network with load-dependent edge latencies. Then, there
	exists another feasible flow $\bm{f}'$ such that 
	\begin{enumerate}
		\item $\bm{f}$ and $\bm{f}'$ have the same edge flow.
		\item The longest used path for commodity $k$ satisfies $\max_{\pi \in \mc{P}_u^k(\bm{f'})} l_{\pi}(\bm{f'}) \le \max_{\pi \in \mc{P}_u^k(\bm{f})} l_{\pi}(\bm{f})$.
		%$L_{max}(\bm{f}')\leq L_{max}(\bm{f}')$.
		\item The shortest used path for commodity $k$ satisfies $\min_{\pi \in \mc{P}_u^k(\bm{f})} l_{\pi}(\bm{f}) \le \min_{\pi \in \mc{P}_u^k(\bm{f'})} l_{\pi}(\bm{f'})$.
		\item The flow $\bm{f}'$ uses at most $|E|$ paths for each source-sink pair.
	\end{enumerate}
\end{proposition}
\begin{remark}
The proof of this proposition directly follows the proof of Lemma~3.1 in Correa et al.~\cite{correa2007fast}, although our lemma statement is more general. (Their Lemma only states part 2 of our Lemma statement.)
\end{remark}

%\begin{proof}
%	The proof basically follows the proof of Lemma~3.1 in Correa et al.~\cite{}.  They introduce a process that iteratively create a new path flow $\bm{f}'$ that has the same edge flow as $\bm{f}$ by moving the flow along one particular path to other used paths.  In their lemma, while they only conclude that the length of the longest path will not increase, similar argument will also hold for that the length of the shortest path will not decrease. 
	%Fix commodity $k$, we index the used paths in $\mc{P}_u^k(\bm{f})$ by $\pi_1, \pi_2, \dots, \pi_r$, where $r$ is the number of used paths in commodity $k$ under $\bm{f}$.  For each $\pi_i$, we define an edge incident vector $\bm{p}_i \in \{0,1\}^{|E|}$, where $p_{ie}=1$ if and only if $e \in \pi_i$.  If $r>|E|$, then the edge incident vectors are linearly dependent.  Then, we can reduce the number of used path with the following process:  First let $\lambda_1, \lambda_2, \dots, \lambda_r$ be such that $\sum_{i=1}^{r} \lambda_i \bm{p}_i = 0$.
%\end{proof}
With this proposition, we can make the following argument that given an edge flow $\bm{x}$, if there is at least one $\theta$-EF or $\theta$-UNE path flow decomposition, then we can always find one with %only linear
polynomial support:
\begin{lemma}\label{thm:npproblems}
	Given a $\theta$-EF path flow $\bm{f}_1$, there exists a $\theta$-EF path flow $\bm{f}'_1$ that uses at most $|E|$ paths for each source-sink pair and has the same edge flow as $\bm{f}_1$.  Similarly, given a $\theta$-UNE path flow $\bm{f}_2$, there exists a $\theta$-UNE path flow $\bm{f}'_2$ that uses at most $|E|$ paths for each source-sink pair and has the same edge flow as $\bm{f}_2$.
\end{lemma}
\begin{proof}
	For a $\theta$-EF path flow $\bm{f}_1$, by Proposition~\ref{lemma:correa1}, there exists a flow $\bm{f}'_1$ that has the same edge flow as $\bm{f}_1$ and the ratio of the longest used path to the shortest used path is bounded by
	$$
	\frac{\max_{\pi \in \mc{P}_u^k(\bm{f'}_1)} l_{\pi}(\bm{f'}_1)}{\min_{\pi \in \mc{P}_u^k(\bm{f'}_1)} l_{\pi}(\bm{f'}_1)} \le \frac{\max_{\pi \in \mc{P}_u^k(\bm{f}_1)} l_{\pi}(\bm{f}_1)}{\min_{\pi \in \mc{P}_u^k(\bm{f}_1)} l_{\pi}(\bm{f}_1)} \le \theta
	$$
	which indicates that $\bm{f}'$ is a $\theta$-EF path flow.  Similarly, given a $\theta$-UNE path flow $\bm{f}_2$, we can find a path flow $\bm{f}_2'$ that has the same edge flow as $\bm{f}_2$ and 
	$$
	\frac{\max_{\pi \in \mc{P}_u^k(\bm{f'}_2)} l_{\pi}(\bm{f'}_2)}{\min_{\pi \in \mc{P}^k} l_{\pi}(\bm{f'}_2)} \le \frac{\max_{\pi \in \mc{P}_u^k(\bm{f}_2)} l_{\pi}(\bm{f}_2)}{\min_{\pi \in \mc{P}^k} l_{\pi}(\bm{f}_2)} \le \theta
	$$
	from which we can conclude that $\bm{f}'_2$ is a $\theta$-UNE as well. 
\end{proof}
Now suppose $\bm{f}_1^*$ is the optimal solution to Problem~(P1).  According to Lemma~\ref{thm:npproblems}, we can see that there is an alternative path flow $\bm{f}_2^*$ that is also $\theta$-EF and has the same edge flow as $\bm{f}_1^*$.  Since the social cost only depends on the amount of the edge flow, $\bm{f}_1^*$ and $\bm{f}_2^*$ have the same social cost, from which we can conclude that $\bm{f}_2^*$ is an optimal solution to Problem~(P1) that uses only polynomially many paths.  A similar argument can be made for Problem~(P2) as well.

\subsection{Hardness Results}
In this section, we prove the second part of Theorem~\ref{thm:main_hardness} that it is NP-hard to solve Problem~(P1) and (P2) for small values of $\theta$.  More precisely, we consider the class of polynomial functions of degree at most $p$, which we denote as $\mc{L}_p$. We note that $\gamma(\mathcal{L}_p)=p+1$. We show that when the latency functions are in $\mc{L}_p$, then the related decision problems we state in Theorem~\ref{thm:12Hard} have polynomial-time reductions from the NP-complete problem PARTITION.  We state this result in the following theorem:

%Suppose we are given an instance of a single commodity flow network $\mc{G}$ 
%with latency functions in class $\mc{L}_p$ for $p \ge 1$.

\begin{theorem}
For an arbitrary single commodity instace $\mc{G}$ 
with latency functions in class $\mc{L}_p$ for $p \ge 1$, it is NP-hard to
\begin{enumerate}
\item decide whether a socially optimal flow has a $\theta$-UNE path flow decomposition for $\theta \in (1, p+1)$.
\item decide whether a socially optimal flow has a $\theta'$-EF path flow decomposition for $\theta' \in [1, p+1)$.
\end{enumerate}
\label{thm:12Hard}
\end{theorem}

We state the following corollary that readily follows from Theorem~\ref{thm:12Hard}.
\begin{corollary}
For  any finite $\theta> 1$, it is NP-hard to find the optimal $\theta$-UNE or $\theta$-EF flow of an arbitrary instance $\mathcal{G}$.
\end{corollary}
\begin{proof}
For a given $\theta$ pick any $p\in \mathbb{N}:\theta<p+1$. Since $p+1=\gamma(\mathcal{L}_p)$, we may use  Theorem~\ref{thm:12Hard} to get the result.
\end{proof}

The proof of Theorem~\ref{thm:12Hard} is composed of two parts.  For the first part, we show the NP-hardness for $1.5$-UNE and $1$-EF path flow decompositions under the social optimum  in Lemma~\ref{lemm:corehardness}, based on the construction in Theorem~3.3 in Correa et al.~\cite{correa2007fast}.  Then, in the second part, we propose a novel way to generalize the construction to the entire range of $\theta$ and $\theta'$ specified in Theorem~\ref{thm:12Hard}.

\begin{lemma}\label{lemm:corehardness}
For single commodity instances with linear latency functions it is NP-hard to decide whether a social optimum flow has a $1.5$-UNE flow decomposition or a $1$-EF flow decomposition.
\end{lemma}
\begin{proof}
We consider the PARTITION problem, where we are given a set of $n$ positive integer numbers $q_1,\ldots, q_n$, and we need to decide  \emph{is there a subset $I \subset \{1,\ldots,n\}$ such that $\sum_{i\in I} q_i = \sum_{i \notin I}q_i$?}
%-----------------------------------------------------------------------------------------
 \begin{figure}[!htb]
 \centering
 \includegraphics[width=0.6\linewidth]{reduction}
 \caption{An instance of congestion game constructed from a given instance of PARTITION}
 \label{Fig:instForProg4}
 \end{figure}
%-----------------------------------------------------------------------------------------

Consider the two link parallel network with the top link $e_{u}$ having latency $\ell_u(x)=q$ and the bottom link $e_b$ having latency $\ell_b(x)=qx$. The demand between the source and the destination is $1$. 
%In the equilibrium flow  the bottom link carries $\frac{1}{2}(1+k)^{\frac{1}{k}}$ flow and the Nash equilibrium length $L_{NE,1}=\frac{k+1}{2^k}$. 
The unique socially optimal flow splits the flow equally through the top and bottom link. Call this instance $G(q)$.

 
Given an instance of the PARTITION problem, $q_1,\ldots, q_n$, $\sum_{i=1}^{n} q_i=2B$, we now construct a single commodity network as the two link $n$ stage network $G$, as shown in Figure \ref{Fig:instForProg4}. In stage $i$ we connect $G(q_{i-1})$ to $G(q_{i})$ to the right for $i=2$ to $n$. A unit demand has to be routed from the source in $G(q_1)$ to the destination in $G(q_n)$.  For the graph $G$, the socially optimal flow $o$ routes $1/2$ flow through all top links and the remaining $1/2$ flow through each bottom link. We first observe that there is a one-to-one correspondence between the subsets $I\subseteq [n]$ and paths $p$ in $G$. Specifically, we can define the path corresponding to $I$ as $P_I=\left\{e_{u,i}: i\in I\right\} \cup \left\{e_{b,i}: i\notin I\right\}$. Further, the latency of the path is given by $\ell_I = \frac{1}{2}(\sum_{i\in [n]} q_i+\sum_{i\in I} q_i)$. 

In one direction, we observe that if the answer to the PARTITION problem is YES then there exists a subset $I^*$ such that $\sum_{i\in I^*} q_i = \sum_{i \notin I^*}q_i = B$. Consider the path flow under socially optimal flow $o$, with path $P_{I^*}$ carrying flow $1/2$ and path $P_{[n]\setminus I^*}$ carrying flow $1/2$.  The lengths of paths $P_{I^*}$ and $P_{[n]\setminus I^*}$ are both  equal to $\frac{3}{4}\sum_{i=1}^{n} q_i = 3B/2$. Whereas, the shortest path in the network is  $P_{\emptyset}$  with length $\frac{1}{2}\sum_{i=1}^{n} q_i = B$. Therefore, the socially optimal flow $o$ is a $3/2$-UNE flow and a $1$-EF flow, if $G$ comes from a YES instance of PARTITION. 



In the other direction, we first observe that if a path $P_{I}$ under edge flow $o$ has length $3B/2 = \frac{3}{4}\sum_{i=1}^{n} q_i$, then $\sum_{i\in I} q_i = \frac{1}{2}\sum_{i\in [n]} q_i$. This implies the given answer to the PARTITION problem is YES. Now assuming $o$ is a $3/2$-UNE, there exists a path flow with the maximum used path of length less or equal to $\frac{3}{4}\sum_{i=1}^{n} q_i$. But the average length of any used path under $o$ is equal to $\frac{3}{4}\sum_{i=1}^{n} q_i$. This implies that all the paths in the path flow must have length $\frac{3}{4}\sum_{i=1}^{n} q_i$.  Next we assume that $o$ is a $1$-EF flow. This implies that there exists a path flow for which all the used paths have equal length. But then any used path under this decomposition has length $\frac{3}{4}\sum_{i=1}^{n} q_i$. Therefore, if $o$ is a $3/2$-UNE or a $1$-EF then the PARTITION instance corresponding to $G$ is a YES instance.   
\end{proof}
 
\begin{proof}[Proof of Theorem~\ref{thm:12Hard}]
Consider $\theta \in (1,p+1)$ for a UNE flow and $\theta' \in [1,p+1)$ for an EF flow. Given a PARTITION instance, let $G'$ be a two link parallel network with latency of the top link $\ell_{u,(n+1)}(x) = ax^p+b$ and bottom link latency $\ell_{d,(n+1)}(x) = cx^p$. We set $a=\frac{\alpha B}{(1-3/8B)^p}$, $b=\beta B(p+1)$, and $c=\frac{(\alpha+\beta)B}{(3/8B)^p}$, where $\alpha,\beta>0$ are some parameters to be determined later.

Using the fact that the social optimum is an equilibrium of the instance with latencies modified to $(\ell(x) + x\ell'_e(x))$, we  get that the socially optimal flow in network $G'$ is $\frac{3}{8B}$ through the bottom link and $(1-\frac{3}{8B})$ through the top link. We also get that at the social optimum the latency function satisfies the following condition:
$$
c\bigg(\frac{3}{8B}\bigg)^p = a\bigg(1-\frac{3}{8B}\bigg)^p + \frac{b}{p+1} < a\bigg(1-\frac{3}{8B}\bigg)^p + b
$$
From the latter, we can see that the top link has larger cost than the bottom link.  We then combine in series the network $G$ of Lemma \ref{lemm:corehardness} with the network $G'$ to obtain network $H$. The unique socially optimal flow in  network $H$ is the union of the two unique socially optimal flows in $G$ and $G'$. Recall the notation from Lemma \ref{lemm:corehardness}.

Assume the PARTITION problem admits a solution $I$. Consider the path decomposition in $H$:
\begin{enumerate}
	\item Path $p = P_{I}-e_{u, (n+1)}$ carries $1/2$ flow (note that $3/8B < 1/2$).
	\item Path $q = P_{I^c}-e_{u, (n+1)}$ carries $(1/2 - 3/8B)$ flow.
	\item Path $r = P_{I^c}-e_{d, (n+1)}$ carries $3/8B$ flow.
\end{enumerate}
We can see that the path $s = P_{\emptyset}-e_{d, (n+1)}$ is the shortest path, with latency $\ell_{s} = B + c(3/8B)^p = (\alpha+\beta+1)B$.  The longest used path $q$ has latency $\ell_{q} = 3B/2 + a(1-3/8B)^p + b = (\alpha+\beta+\beta p+3/2)B$.  Letting $c_1 = \frac{\ell_{q}}{ \ell_{s}}=\bigg(\frac{\alpha+\beta+\beta p+3/2}{\alpha+\beta+1}\bigg)$, the social optimum flow in $H$ is a $c_1$-UNE flow.

We next consider a different path flow for the EF setting. In this path flow:
\begin{enumerate}
	\item Path $s' = P_{[n]}-e_{d, (n+1)}$ carries $\frac{3}{8B}$ flow.
	\item Path $p = P_{I}-e_{u, (n+1)}$ carries $(1/2 - 3/8B)$ flow.
	\item Path $q = P_{I^c}-e_{u, (n+1)}$ carries $(1/2 - 3/8B)$ flow.
	\item Path $r'= P_{\emptyset}-e_{u, (n+1)}$ carries $\frac{3}{8B}$ flow.
\end{enumerate}
We claim that path $s'$ is the shortest path if $\beta p>1$ as
$$
\ell_{s'} = 2B + c(3/8B)^p = (\alpha+\beta+2)B <
(\alpha + \beta + \beta p + 1)B = B + a\bigg(1-\frac{3}{8B}\bigg)^p + b = \ell_{r'}<\ell_p=\ell_q.
$$
In this setting, the minimum ratio of longest `used' path and shortest `used' path is  $c_2 = \frac{\ell_{q}}{ \ell_{s'}}=\bigg(\frac{\alpha+\beta+\beta p+3/2}{\alpha+\beta+2}\bigg)$ and the socially optimal flow is a $c_2$-EF flow.

Next, we need to show that if the answer to PARTITION is NO then the socially optimal flow is neither a $c_1$-UNE flow nor a $c_2$-EF flow. For this we need to ensure that for all possible path flows under the social optimum, there exists at least one used path which is obtained by concatenating a `long' positive subpath in $G$ with the upper edge in $G'$. The following claim lower bounds the flow through the longest path in $G$ for any valid path decomposition. 

\begin{claim}\label{lemm:flowlower}
If the answer to PARTITION is NO then in the sub-network $G$ any path decomposition of the socially optimal flow $o$ routes at least $\frac{1}{2B}$ amount of flow through paths of length strictly greater than $\frac{3}{2}B$. 
\end{claim}
\begin{proof}
Recall that if the given instance for the PARTITION problem is a NO instance then there is no path under $o$ which has length exactly $3B/2$.
Fix any path decomposition for $o$ and let $\delta$ be the flow passing through the paths of length strictly greater than $\frac{3}{2}B$.   Also let $\ell$ be the maximum length among the set of paths strictly smaller than $\frac{3}{2}B$. As $q_i$'s are integers  and the given instance of PARTITION is a NO instance, it is easy to observe that $\ell \leq \frac{3}{2}B-\frac{1}{2}$. Also $\ell\geq B$. Moreover, if we route $(1-\delta)$ flow through a path of length $\ell$ and  $\delta$ flow through the path of maximum length $2B$, then the cost of this routing is greater or equal to the socially optimal cost. This implies,
%\begin{align*}
$$\ell(1-\delta)+2\delta B \geq \frac{3}{2}B  \implies
\delta \geq \frac{3B/2-\ell}{2B-\ell} \geq \frac{3B/2-3B/2+1/2}{2B-B} \geq \frac{1}{2B}.$$
%\end{align*}  
\end{proof}   

From the above claim we see that the longest used path $q$ has length strictly greater than $\ell_{q}$ as the bottom link under $o$ has flow $3/8B < 1/2B$. The shortest path in the network has length $\ell_{s}$ as in the YES case. If the PARTITION instance is a NO instance, the optimal flow $o$ is not a $c_1$-UNE. Moreover, for the EF flow the best path flow again contains the path $s'$ as the shortest path but now the longest path is strictly greater than $\ell_{q}$. So it is not a $c_2$-EF flow. 

All that is left to show is that there are appropriate values of $\alpha$ and $\beta$ which make $c_1 = \theta$ or $c_2 = \theta'$, for any $\theta\in (1, p+1)$, and for any $\theta'\in (1, p+1)$.  This can be shown by observing that:
\begin{align*}
	c_1=\frac{\alpha+\beta+\beta p + \frac{3}{2}}{1+\alpha+\beta}&=1+\frac{\frac{1}{2}+\beta p}{1+\alpha+\beta} & c_2=\frac{\alpha+\beta+\beta p + \frac{3}{2}}{2+\alpha+\beta}&=1+\frac{-\frac{1}{2}+\beta p}{2+\alpha+\beta}
\end{align*}   
Combining this with what we have shown in Lemma~\ref{lemm:corehardness} for $1$-EF flows completes the proof.
\end{proof}
\begin{proof}[Proof, second part of Theorem~\ref{thm:main_hardness}]
The proof follows by constructing a reduction from the decision problems specified in Theorem~\ref{thm:12Hard} and recalling that $\gamma(\mathcal{L}_p)=p+1$.  The answer to each of the decision problem in Theorem~\ref{thm:12Hard} is YES if and only if the solution to Problem~(P1) or (P2) is a social optimum, the cost of which is known in advance, by construction.
\end{proof}




%\begin{remark}
%	The proof is inspired from the proof of the NP hardness of length bounded flow problem (Theorem~3.3) in Correa et al.~\cite{}. But the conclusions apply to the completely new setting of UNE and EF flow and we generalize it through novel constructions. 
%\end{remark}

For Problem~(P3), the proof technique in Theorem~\ref{thm:main_hardness} does not go through.  In fact, we show that the relevant decision problem related to Problem~(P3) is in P: % We consider the following problem:
\begin{enumerate}
\item[(P3')] Is there a socially optimal flow which is a $\theta$-PNE?
\end{enumerate}

To show that (P3') is in P, we first define an edge flow $\bm{x}$ to be \emph{acyclic} if for each commodity $k$, the subgraph $G_k$, induced by the edges $E_k(\bm{x}) = \{e: e\in E, x_e^k>0\} $ is a directed acyclic graph (DAG).
  
\begin{claim}
Given an instance of a multicommodity flow network $\mc{G}$ with standard latency functions, we can decide whether an `acyclic' edge flow $\bm{x}$ is in $\theta$-PNE in polynomial time.
\label{clm:Acyclic}
\end{claim}
\begin{proof}
We present the polynomial time algorithm which decides whether an `acylic' edge flow $\bm{x}$ is a $\theta$-PNE or not for some given $\theta$. For each commodity $k$ in $\mc{G}$, we construct the DAG induced by $E_k(\bm{x})$. Next, under the edge weights $w_e = \ell_e(x_e)$, we compute the costs of the shortest $(s_k, t_k)$ path in $G$ (call it $\ell_1$) and the longest $(s_k,t_k)$ path in $G_k$ (call it $\ell_2$).  Recall that shortest path computation and longest path computation in a DAG can both be done in polynomial time. Finally, we accept if $\ell_2 \leq \theta\ell_1$ and reject otherwise. 
\end{proof}

\begin{lemma}
Problem~(P3') can be solved in polynomial time.
\label{lemma:3Easy}
\end{lemma}
\begin{proof} 
We first claim that for any $k$, the set of edges that carry flow for commodity $k$ at the social optimum, $E_k(\bm{x}^*)$, has no positive loops.  This can be shown by contradiction.  Assume there is a positive loop in $E_k(\bm{x}^*)$, then, we can construct a new flow $\bm{x}'$ by removing some $\epsilon>0$ flow on the loop.  The flow $\bm{x}'$ can be kept feasible, and it has strictly smaller social cost due to the monotonicity and non-negativity of the latency functions, which contradicts the fact that $\bm{x}$ is the socially optimal flow.  Also, if there is a zero cost loop in $E_k(\bm{x}^*)$, we can safely remove the flow on that loop without changing the social cost. Therefore, the procedure in Lemma~\ref{clm:Acyclic} completes the proof. 
\end{proof}
