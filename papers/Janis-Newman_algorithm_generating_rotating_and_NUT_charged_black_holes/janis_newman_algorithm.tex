\documentclass[10pt, a4paper]{article}
\pdfoutput=1

\usepackage[utf8]{inputenc}
\usepackage[T1]{fontenc}
\usepackage{lmodern, textcomp}
\usepackage[british]{babel}
\usepackage{csquotes}

\usepackage[heightrounded, top=4cm, bottom=4cm, left=3.5cm, right=3.5cm]{geometry}

\usepackage{eurosym, xspace}
\usepackage[nottoc]{tocbibind}
\usepackage{graphicx, subfig, float}
\usepackage[multiple]{footmisc}
\usepackage{authblk}
\usepackage{fancyhdr}

\usepackage[usenames, dvipsnames]{xcolor}

\usepackage[backend=bibtex8, citestyle=numeric-comp, maxbibnames=99,
			sorting=none, sortcites=true, firstinits=true]{biblatex}
\documentclass[amsmath,amssymb,aps,pra,superscriptaddress,twocolumn]{revtex4-2}
\usepackage[subpreambles=true]{standalone}
\usepackage{lmodern}
\usepackage{stmaryrd}
\usepackage{graphicx,color}
\usepackage{subfigure}
\usepackage{units}
\usepackage{verbatim}
\usepackage{dsfont}
\usepackage{bm}
\usepackage{mathrsfs}
\usepackage{hyperref}
\hypersetup{colorlinks=true, linkcolor=black, citecolor=black, urlcolor=black}
\usepackage{amsmath,amssymb,amsthm,mathrsfs,bbm,bm,mathtools}
\usepackage{comment}
\usepackage{enumerate}
\usepackage{braket}
\usepackage{todonotes}
\usepackage[titletoc,title]{appendix}
\usepackage{xypic}
\usepackage{multirow}
\usepackage{soul}
\usepackage{bibunits}

\newcommand{\Red}[1]{\textcolor{red}{#1}}
\newcommand{\m}{\vspace{0.2in}}
\newcommand{\mm}{\vspace{0.09in}}
\newcommand{\missingcite}{{\color{magenta} [~]}}
\newtheorem{definition}{Definition}
\newtheorem{remark}{Remark}
\newtheorem{theorem}{Theorem}
\newtheorem{proposition}[theorem]{Proposition}
\newtheorem{lemma}[theorem]{Lemma}
\newtheorem{coro}[theorem]{Corollary}
\newtheorem{conjecture}[theorem]{Conjecture}
\newtheorem{innercustomlemma}{Lemma}
\newenvironment{customlemma}[1]
  {\renewcommand\theinnercustomlemma{#1}\innercustomlemma}
  {\endinnercustomlemma}

\newtheorem{innercustomcor}{Corollary}
\newenvironment{customcor}[1]
  {\renewcommand\theinnercustomcor{#1}\innercustomcor}
  {\endinnercustomcor}

\begin{document}
\title{Erratum: Observing a changing Hilbert-space inner product}
\author{Salini Karuvade}
\affiliation{Centre for Engineered Quantum Systems, School of Physics, 
University of Sydney, Sydney, New South Wales 2006, Australia}
\author{Abhijeet Alase }
\affiliation{Centre for Engineered Quantum Systems, School of Physics, 
University of Sydney, Sydney, New South Wales 2006, Australia}

\author{Jacob L.\ Barnett}
\affiliation{Perimeter Institute for Theoretical Physics, 31 Caroline Street North, Waterloo, Ontario N2J 2Y5, Canada}
\affiliation{Department of Physics \& Astronomy, University of Waterloo, Waterloo, Ontario
N2L 3G1, Canada}
\author{Barry C.\ Sanders}
\affiliation{Institute for Quantum Science and Technology, University of Calgary, Alberta T2N 1N4, Canada}

\maketitle

\onecolumngrid
% \begin{widetext}
Lemma~2 
in Appendix~A of our original paper,
which was
first presented in 1992~\cite{SGH92},
is incorrect as we explain in this erratum.
This lemma should be replaced by the following lemma and corollary.
\begin{customlemma}{2}
A bounded positive-definite operator, $\eta\in\mathcal{B}\left(\mathscr{H}\right)$, is invertible if and only if $\eta$ is surjective. 
\end{customlemma} 
\begin{customcor}{2.1}\label{cor}
Let $\eta\in\mathcal{B}\left(\mathscr{H}\right)$ 
be a surjective, positive-definite operator.
Then, $\eta^{-1}$ is bounded and positive-definite.
\end{customcor} 

\noindent Neither paper articulated a complete set of constraints required for the invertibility of the metric 
operator $\eta$.
Specifically, the inverse of a bounded positive-definite operator, $\eta\in\mathcal{B}\left(\mathscr{H}\right)$,
in the absence of the surjectivity constraint
is not necessarily bounded. 

A simple counter-example  to the original Lemma 2 is now presented.
Consider a separable Hilbert space $\mathscr{H}$ with a countably infinite orthonormal basis, $\{\ket{e_j}, j \in \mathbb{N}\}$. Consider the bounded positive-definite operator 
\begin{equation}\label{eq:cex}
    \eta: \ket{e_j} \mapsto \frac{1}{j}\ket{e_j} \quad \forall j \in \mathbb{N}.
\end{equation}
On the dense subspace of $\mathscr{H}$ spanned by finite linear combinations of 
$\{\ket{e_j}, j \in \mathbb{N}\}$ the inverse of $\eta$ is given
by 
\begin{equation}
    \eta^{-1}:\ket{e_j} \mapsto j\ket{e_j} \quad \forall j \in \mathbb{N},
\end{equation}
which is clearly not bounded
on the whole $\mathscr{H}$.



For $\eta$ to be invertible, we need to enforce an additional constraint,
namely that the range of $\eta$ coincides with $\mathscr{H}$, or in other words, $\eta$ is surjective.
Below we prove the corrected version of Lemma~2, which is stated above.
\begin{proof}
An operator is injective if and only if its kernel contains only the zero vector. With this fact in mind, a straightforward proof by contradiction demonstrates the injectivity of all positive-definite operators (if  $\ket{\psi}\in\ker(\eta) \setminus \bm{0}$, then 
$\braket{\psi|\eta|\psi}=0$, thereby violating the positive-definiteness of $\eta$). Together with the supposition that $\eta$ is surjective, we find that $\eta$ is a bijection. Thus, the bounded inverse theorem \cite[Thm.~14.5.1]{Narici2010} implies $\eta^{-1}\in\mathcal{B}\left(\mathscr{H}\right)$.
\end{proof}


Corollary~\ref{cor} now follows from
$\braket{\psi|\eta^{-1}|\psi} = (\bra{\psi}\eta^{-1})\eta(\eta^{-1}\ket{\psi}) > 0$ 
for any $\ket{\psi} \in \mathscr{H}$,
where we used the positive-definiteness of $\eta$ in the last inequality.
We remark that self-adjointness of the metric operator follows from 
its positive-definiteness by polarization identity~\cite{BB03},
and therefore need not be assumed separately as in the original 
version of Lemma~2. 
Similarly, if $\eta\in \mathcal{B}(\mathscr{H})$ is positive-definite and surjective,
then $\eta^{-1} \in \mathcal{B}(\mathscr{H})$ is positive-definite
by Corollary~\ref{cor} above, and therefore self-adjoint.


Invertibility of the metric operator also guarantees that $\mathscr{H}_\eta$ defined in
the original paper is a Hilbert space with respect to the modified inner product 
$\braket{\bullet|\bullet}_\eta$~\cite[Appendix A]{SGH92}. 
Subsequently, surjectivity of $\eta$,
in addition to boundedness and positive-definiteness, is required for defining 
the change in representation  $\mathcal{R}_\eta$ [Eq.~(2) of the original paper] and 
the inner-product changing operation $\mathcal{E}_{\eta}$Eq.~(4) of the original paper].


Our results concerning the simulation of PT-symmetric Hamiltonians hold without any
additional assumptions.
This is because we exclusively work with unbroken PT-symmetric Hamiltonians in finite dimensions. The rank-nullity theorem implies that all injective operators with finite-dimensional domains are additionally surjective. Therefore, Lemma~2 as originally stated in the paper is valid
for any finite-dimensional Hilbert space $\mathscr{H}$. 

We also note that for any unbroken PT-symmetric Hamiltonian in 
a separable, infinite dimensional Hilbert space $\mathscr{H}$, an $\eta \in \mathcal{B}(\mathscr{H})$
satisfying the quasi-Hermiticity condition 
that is both surjective and positive-definite can be constructed~\cite[Thm.~4.3]{Kar22}
(see also~\cite[Eq.~(23)]{BiOrthogonal}).
This is a consequence of the fact that unbroken PT-symmetric Hamiltonians in infinite dimensions are defined 
to have eigenvectors forming a Riesz basis, in addition to these vectors 
being invariant under the action of the PT operator~\cite{Mos10b}.


In conclusion, amendment to Lemma~2 mandates that the operations $\mathcal{R}_\eta$ [Eq.~(2) of the original paper] and $\mathcal{E}_\eta$ [Eq.~(4) of the original paper] are to be defined only under the additional constraint that $\eta$ is surjective. 
The rest of our original paper is correct and remains unchanged after these modifications.

\ 

\paragraph*{Author Contribution Statement:} JLB pointed out the error in Lemma 2 of the original paper and provided the counter example given in this erratum. SK and AA developed the correct formulation of Lemma 2. SK and AA wrote the draft of the erratum with feedback from JLB and BCS.

\pagebreak

\begin{center}
{\bf \large Observing a changing Hilbert space inner product} 

\ 

{Salini Karuvade, Abhijeet Alase, and Barry C. Sanders}

{\small \it Institute for Quantum Science and Technology, University of Calgary,\\
2500 University Drive NW, Calgary, Alberta T2N 1N4, Canada}

\ 

\setlength{\fboxsep}{0pt}%
\setlength{\fboxrule}{0pt}%

\noindent\fbox{%
    \parbox{0.8\textwidth}{%
\small        
In quantum mechanics,
physical states are represented by rays in Hilbert space~$\mathscr H$,
which is a vector space imbued by an inner product~$\langle\,|\,\rangle$,
whose physical meaning arises as the overlap~$\langle\phi|\psi\rangle$
for~$\ket\psi$ a pure state (description of preparation)
and~$\bra\phi$ a projective measurement. However,
current quantum theory does not formally address the consequences of 
a changing inner product during the interval between preparation and measurement.
We establish a theoretical framework for such a changing inner product,
which we show is consistent with standard quantum mechanics.
Furthermore, we show that this change is described by a quantum operation,
which is tomographically observable,
and we elucidate how our result is strongly related to the 
exploding topic of PT-symmetric quantum mechanics.
We explain how to realize experimentally a changing inner product for a 
qubit in terms of a qutrit protocol with a unitary channel.
    }%
}

\end{center}

\twocolumngrid
Hilbert-space inner product is fundamental to quantum mechanics (QM), and its physicality relates 
to norm through the Born interpretation and to fidelity and distinguishability through its complex angle~\cite{SN21}.
The uniqueness of the inner product associated to a quantum system has come under scrutiny 
following the advent of PT-symmetric QM.
PT-symmetric systems are described by non-Hermitian Hamiltonians invariant under the combined action of parity~(P) and time~(T) 
inversion symmetries~\cite{BB98,BBJ02,BBJ03,Ben05}, and they are predicted to exhibit novel physical phenomena which
have been simulated on a variety of experimental platforms~\cite{RME+10,SLZ+11,BDG+12,POS+14,ZZS+16,XZB+17,EMK+18,WLG+19,ZLW+20}. 
These phenomena have been explained by observing that non-Hermitian Hamiltonians with 
unbroken PT symmetry are Hermitian with respect to a different Hilbert-space inner product~\cite{BBJ02,Mos03a,MECM08,JMCN19}.
Changing Hilbert-space inner-product is valuable for certain quantum information processing (QIP) tasks~\cite{Cro15} 
such as non-orthogonal state discrimination~\cite{BBC+13}, cloning~\cite{ZWX+20} and quantum algorithms~\cite{BBJM07,Mos09},
but perfunctory applications have led to counter-factual conclusions~\cite{Cro15,Pat14,CCC14} 
including violation of the no-signalling principle~\cite{YHFL14}. 
Our aim is to prescribe the correct procedure for changing Hilbert-space inner product and
to devise an experiment to validate our prescription.



Consistency  of a changing Hilbert-space inner product
with standard QM  and the 
unobservability of such a change in closed systems 
have been investigated.
A C$^*$-algebraic approach shows that
a set of non-Hermitian operators comprises the
observables of a
quantum mechanical system if and only if the operators are Hermitian with respect to
a new Hilbert-space inner product~\cite{SGH92}.
Such a modified inner product is the key to proving the equivalence
of PT-symmetric QM with 
the Dirac-von Neumann formulation of QM in the case of closed systems i.e., systems in which 
every time evolution is a unitary operation~\cite{BBJ02,Mos03a,Mos10a,Mos10b,Zno15,Mos18,ZWG19,JMCN19}.
Furthermore, 
this equivalence implies that any change in inner product is unobservable in experiments on closed systems~\cite{Bro16}.
Therefore, the above proposals that use the inner-product change for QIP tasks
as well as the counter-factual claims are not applicable to closed systems.



Evolution generated by PT-symmetric Hamiltonians has been implemented experimentally
for applications including sensing~\cite{LZO+16,COZ+17,HHW+17}, cloaking~\cite{ZFZ+13,SFA15}
and unidirectional propagation~\cite{RKEC10}.
These experiments simulate PT-symmetric dynamics on classical~\cite{RME+10,SLZ+11,POS+14,ZZS+16} or
quantum~\cite{BDG+12,XZB+17,LHL+19,XQW+19} systems by balancing loss and gain.
Another way to simulate  PT-symmetric Hamiltonians with real spectra is by 
dilating the non-unitary propagator to a non-local unitary operator over multiple subsystems, which has been demonstrated 
on qubit systems~\cite{GS08a,GS08b,ZHL13,TWY+16,KAU17,HKW18,WLG+19,XWZ+19,GZL+21}.
However, none of these simulation strategies involve effecting a change 
of inner product.


PT-symmetric Hamiltonians and a changing Hilbert-space inner product
are known to be consistent with standard QM for closed systems,
but they are not yet known to be consistent for open systems.
To solve these outstanding problems, we 
construct an operational framework, consistent with the C$^*$-algebraic formulation of QM,
which accommodates a change in inner product between preparation and measurement.
Furthermore, neither PT symmetry nor a changing Hilbert-space inner product are observable in
closed systems, but could be observable in open systems~\cite{Bro16}.
We show our change in inner product is implemented by a quantum operation
(henceforth assumed to be completely positive and trace non-increasing),
which can be observed using tomography.
Next we connect our framework to the burgeoning topic of PT-symmetric QM
by explaining how an inner-product-changing quantum operation can be 
used to implement PT-symmetric dynamics in an open system.
Finally, at the empirical level, we describe a potential experimental simulation for changing the inner product 
of a qubit by subjecting a qutrit to unitary evolution and 
neglecting the third Hilbert-space dimension during preparation and measurement but not during evolution.
We also extend this simulation procedure to $d$-dimensional systems. 



To construct the operational framework for changing the inner product associated to a quantum system
between preparation and measurement,
we adopt the C$^*$-algebraic framework of QM~\cite{Str08},
which provides freedom in representing 
a given system on different Hilbert spaces following the Gel'fand-Naimark-Segal (GNS) construction~\cite{GN43,Seg47}.  
We employ this representation freedom 
first to construct representations of the C$^*$ algebra on a pair of Hilbert spaces 
whose inner products are related by a given metric operator~$\eta$. We then
define the change in inner product by~$\eta$ as the identity
isomorphism between the two Hilbert spaces.
To operationalize the change in inner product, we use commutative diagrams that connect 
this isomorphism to a quantum operation 
between the bounded operators on the two Hilbert spaces and 
finally observe that the quantum operation induces an observable physical transformation on the system.

In the operational approach, the operators of a quantum system form a unital C${}^*$ algebra
$\mathcal{A}=\{A\}$, which is equipped with a ${}^*$ operation that 
captures the notion of adjoint. The algebra~$\mathcal{A}$ is representable on a 
possibly infinite dimensional 
Hilbert space~$\mathscr H=(\mathscr{V},\langle\,\vert\,\rangle)$, 
comprising a complete vector space~$\mathscr V$
and an inner product~$\langle\,\vert\,\rangle$, which is a non-degenerate sesquilinear form.
In Fig.~\ref{fig:commutativediagram}(a),
observables are self-adjoint elements of~$\mathcal{A}$ and correspond to allowed measurements.
A representation of $\mathcal{A}$ is a product-preserving linear map 
\begin{equation}
\label{eq:pi*dagger}
\mathcal{A} \stackrel{\pi}{\to} \mathcal{B}(\mathscr{H}):
\pi(A^*) = (\pi(A))^\dagger,
\end{equation}
where~$\mathcal{B}(\mathscr H)$ denotes the space of bounded linear operators
acting on~$\mathscr H$ and $^\dagger$ denotes the Hermitian conjugate.
Such a representation can be obtained using the GNS construction~\cite{GN43,Seg47}.
Product preservation ensures that if $I$ is the identity operator in $\mathcal{A}$, then
$\pi(I)$ is the identity operator in~$\mathcal{B}(\mathscr H)$.
An operator $M\in \mathcal{B}(\mathscr H)$ satisfying $M^\dagger = M$ is called a self-adjoint or a Hermitian operator.
\begin{figure}
\begin{center}
\includegraphics[width = \columnwidth]{Fig1.pdf}
\end{center}
\caption{
(a)~Diagram illustrating the relation between the ${}^{*}$ operation of $\mathcal{A}$ and the 
$\dagger$ operation of $\mathcal{B}(\mathscr{H})$ under the representation $\pi$.
(b) Commutative diagram depicting the change of representation from $\pi$ to $\pi_{\eta}$ under $\mathcal{R}_\eta$. Operationally, 
$\mathcal{R}_\eta$ represents a trivial transformation, i.e., no change, in $S$. (c)~Commutative diagram illustrating the relation between 
the maps $\mathcal{F}_\eta$,
$\mathcal{E}_\eta$ and $\mathcal{I}_{\eta}$. }
\label{fig:commutativediagram}
\end{figure}

We now define states and explain how to represent states as operators on Hilbert space.
States correspond to allowed preparations of the system
(Fig.~\ref{fig:commutativediagram}(b)).
A state $\omega$ is a positive linear functional on $\mathcal{A}$
that is normalized, i.e. $\omega(I) = 1$.
This definition is extended to include any subnormalized positive linear functional,
i.e.\ $\omega(I)\le 1$, which corresponds to probabilistic preparation in the state 
$\omega/\omega(I)$ with probability $\omega(I)$~\cite{Gud79}.
Supernormalized positive linear functionals are not valid states
according to this probabilistic interpretation~\footnote{
We remark that supernormalized functionals are valid states in some other frameworks
for describing system evolution under non-Hermitian Hamiltonians~\cite{Mos07,GKN10,UGRM12}
}.
Let $\mathcal{D}(\mathscr{H}):= \{\rho:\rho \ge 0,\rho=\rho^\dagger,
\text{tr}(\rho)\leq 1\} \subset \mathcal{B}(\mathscr{H})$ denote the set of density operators
with $\rho\in \mathcal{D}(\mathscr H)$ representing a state~$\omega$ 
by~$\rho\stackrel{\!\!{}^{\#}\!\!\pi}{\mapsto}\omega$
if and only if the expectation value~tr$(\rho\pi(A)) = \omega(A)\,\forall A$.
As ${}^\#\pi$ is uniquely determined by $\pi$, we say
that $\omega$ is represented by $\rho\in \mathcal{D}(\mathscr H)$ under $\pi$. 
We denote by  $\mathcal{S} = \{\omega\}$, the set of all states that are represented by density operators under $\pi$.
For the special case of pure~$\omega$,
$\rho = \ket{\psi}\bra{\psi}$ for some $\ket{\psi}\in \mathscr H$ with $\braket{\psi | \psi} \le 1$.
The transformation $\ket{\psi}\stackrel{\operatorname{lift}}{\mapsto}\ket{\psi}\bra{\psi}$ relates Hilbert-space
vectors to the density operators in~$\mathcal{D}(\mathscr H)$~\footnote{
The map lift is defined to act only on the normalized and subnormalized vectors in $\mathscr{H}$.
The domain of lift in Fig.~\ref{fig:commutativediagram}(c) is shown to be $\mathscr{H}$ for simplicity, but is specified 
rigorously in Appendix~\ref{sec:liftmap}.
}.


Now that we have explained states and their representations,
we now discuss changing representation
to being over a different Hilbert space.
Given a self-adjoint positive-definite metric operator~$\eta\in\mathcal{B}(\mathscr{H})$,
a new Hilbert space~$\mathscr{H}_\eta = (\mathscr{V},\langle\,\vert\,\rangle_\eta)$ can be constructed~\cite{SGH92} 
such that the inner products of the two Hilbert spaces are related by
$\langle \bullet\,\vert\,\bullet\rangle_\eta := \langle\bullet\vert\,\eta\bullet\rangle$.
A representation $\pi_{\eta}$ on $\mathscr{H}_{\eta}$ can be constructed 
through~(see Appendix~\ref{sec:newHilbertspace})
\begin{equation}
\pi(\bullet)\stackrel{\mathcal{R}_{\eta}}{\mapsto}\pi_{\eta}(\bullet):=R\pi(\bullet)R^{-1},
\mathcal{R}_\eta:\mathcal{B}(\mathscr H)\to\mathcal{B}(\mathscr H_\eta),
\end{equation}
where $R = \eta^{\nicefrac{-1}{2}}$ is a linear isometry from $\mathscr H$ to $\mathscr H_\eta$.
We note that this isometry has been used to prove that Hamiltonians with unbroken PT symmetry 
are consistent with standard QM, in the case of closed systems~\cite{GS08a,Mos10a,Mos10b,Zno15,Cro15}.
We refer to the quantum channel $\mathcal{R}_{\eta}$ as a `change in representation' (Fig.~\ref{fig:commutativediagram}b).
In representation~$\pi_{\eta}$, the state $\omega\stackrel{\;{}^{\#}\!\!\pi_{\eta}}{\mapsfrom}\mathcal{R}_{\eta}(\rho)$ 
such that tr$(\mathcal{R}_{\eta}(\rho)\pi_{\eta}(A)) = \omega(A)\forall A$,
and $\mathcal{R}_{\eta}(\rho) \ne \rho$ in general. 
For any pure state~$\omega$, $\exists\ket\psi \in \mathscr{H}_{\eta}$ such that $\omega\stackrel{\;{}^{\#}\!\!\pi_{\eta}}{\mapsfrom}\ket{\psi}\bra{\psi}
\eta\stackrel{\operatorname{lift}_{\eta}}{\mapsfrom}\ket{\psi}$~(see Appendix~\ref{sec:liftmap}).
As 
\begin{equation}
\label{eq:differentreps}
    \text{tr}(\rho \pi(A)) = \text{tr}(\mathcal{R}_{\eta}(\rho) \pi_\eta(A)) \;\;\forall \omega,A,
\end{equation}
representations $\pi$ and~$\pi_{\eta}$ are physically,
i.e. observationally, indistinguishable. 
The right-hand side of Eq.~\eqref{eq:differentreps} 
can also be interpreted as preparation (state) described in $\pi$
followed by a change in representation from~$\pi$ to~$\pi_\eta$ 
effected by $\mathcal{R}_{\eta}$ and finally measurement (observable) described in~$\pi_\eta$.
Change in representation between preparation and measurement sets the stage for 
our definition of change in inner product. 

We define a change in inner product by $\eta$
to be the identity isomorphism~$\mathcal{I}_{\eta}:\mathscr H \to \mathscr{H}_{\eta}$ such that
every $\ket{\psi} \mapsto \ket{\psi}$.
For any pair $\ket{\psi},\ket{\phi}\in\mathscr{H}$,
the inner product between the pair of transformed vectors $\mathcal{I}_{\eta}\ket{\psi},\ \mathcal{I}_{\eta}\ket{\phi}$ 
is $\braket{\psi|\eta|\phi}$,
and the change is trivial if $\eta = \pi(I)$; i.e.\ 
for all pairs $\ket{\psi},\ket{\phi}$, $\braket{\psi|\eta|\phi} = \braket{\psi|\phi}$.
Our definition is motivated by proposals to effect PT-symmetric evolution and measurement by 
changing the Hilbert-space inner product~\cite{BBJM07,BBC+13} but without a 
prescription for making such changes operationally or mathematically.
Next we explain separately, for the cases $\eta \le \pi(I)$ and $\eta \nleq \pi(I)$,
how the isomorphism $\mathcal{I}_\eta$ can be physically realized as a quantum operation.


The change in inner product by $\eta \le \pi(I)$ is physically realizable 
via the operation 
\begin{equation}\label{eq:Etilde}
    \mathcal{E}_\eta:\mathcal{B}\left(\mathscr H\right) \to \mathcal{B}\left(\mathscr{H}_{\eta}\right):M \mapsto M\eta,
\end{equation}
which is not trace-preserving for $\eta \ne \pi(I)$~(see Appendix~\ref{sec:channelEeta}).
The operation  $\mathcal{E}_\eta$ mimics the action of $\mathcal{I}_{\eta}$ at the level of density operators,
because $\ket{\psi} \stackrel{\operatorname{lift}}{\mapsto}\ket{\psi}\bra{\psi}$
whereas~$\mathcal{I}_{\eta}\ket{\psi} \stackrel{\operatorname{lift}_{\eta}}{\mapsto}\ket{\psi}\bra{\psi}\eta = 
\mathcal{E}_\eta(\ket{\psi}\bra{\psi})$.
The operation $\mathcal{E}_\eta$ induces a linear map 
$\mathcal{F}_\eta:\mathcal{S} \to \mathcal{S}$
such that, for any pure $\omega\in \mathcal{S}$, both $\omega$ and $\mathcal{F}_\eta(\omega)$ are represented by the 
same $\ket{\psi}$ under the representations $\pi$ and $\pi_{\eta}$ respectively.
However, $\omega$ and $\mathcal{F}_\eta(\omega)$ are not necessarily the same state (Fig.~\ref{fig:commutativediagram}c).
Even in the special case where the two states differ by a scaling factor,
they are inequivalent in our setting.
The expectation value of $I$ with respect to 
$\mathcal{F}_\eta(\omega)$ gives the success probability
of the inner-product changing quantum operation on the state $\omega$.
$\mathcal{E}_\eta$ can be implemented experimentally 
by lossy purity-preserving operations,
i.e., operations that are not necessarily deterministic and 
transform the set of pure states into itself.
In the Heisenberg picture,
the operators transform according to the map
\begin{equation}\label{eq:Edual}
\mathcal{E}^{\rm op}_\eta:\mathcal{B}\left(\mathscr H\right) \to \mathcal{B}\left(\mathscr{H}_{\eta}\right):M \mapsto \eta M.
\end{equation}
This transformation $\mathcal{E}^{\rm op}_\eta$ could modify  
commutator relations as we show in Appendix~\ref{sec:commutation}.


In the case $\eta \nleq \pi(I)$,
$\mathcal{E}_\eta$
is completely positive
but trace-increasing for some $\rho\in\mathcal{B}(\mathscr{H})$ 
and hence not a quantum operation.
In such cases, a scaled version of change in inner product can be implemented in the following way:
choose $\kappa\in(0,1)$ such that $\kappa\eta\leq \pi(I)$ and observe that 
$\mathcal{E}_{\kappa\eta} = \kappa\mathcal{E}_\eta$ with 
$\mathcal{E}_{\kappa\eta}$ a quantum operation. 
Therefore, $\mathcal{E}_{\kappa\eta}$  implements change in inner product by $\eta \nleq \pi(I)$
up to a scaling factor $\kappa$.
Such a scaled version of change in inner product is useful to reverse the effect of operation 
$\mathcal{E}_\eta$ when $\eta \leq \pi(I)$.
In this case, the isomorphism $\mathcal{I}_{\eta^{-1}}:\mathscr{H}_\eta\to\mathscr{H}$
reverses the change in inner product and the corresponding $\mathcal{E}_{\eta^{-1}}$
is not a valid operation because  $\eta^{-1} \ge \pi_{\eta}(I)$.
Nevertheless, we can choose $\kappa = \nicefrac{1}{\|\eta^{-1}\|}$, 
where $\|\bullet\|$ denotes the operator norm~\cite{Con07}, 
and observe that $\mathcal{E}_{\kappa\eta^{-1}}\circ\mathcal{E}_{\eta}(\rho)=\kappa \rho$ 
for all $\rho\in \mathcal{B}(\mathscr{H})$.
Therefore, the operation   
$\mathcal{E}_{\kappa\eta^{-1}}:\mathcal{B}(\mathscr{H}_{\eta})\to\mathcal{B}(\mathscr{H})$ 
reverses, with probability $\kappa$, the change in inner product by $\eta$. 


The metric operator $\eta$ can be estimated via quantum process tomography~\cite{CN97} 
for $\eta\leq \pi(I)$,
or $\kappa \eta$ if otherwise.
The change in inner product by $\eta\leq \pi(I)$ is implemented via the operation 
(Fig.~\ref{fig:Etilde})
\begin{equation}
\label{eq:F}
\mathcal{E}_\eta= \mathcal{R}_{\eta}\circ \mathcal{G}_\eta,\,    
\mathcal{G}_\eta:\mathcal{B}(\mathscr H)\to \mathcal{B}(\mathscr H):M \mapsto \eta^{\nicefrac{1}{2}}M\eta^{\nicefrac{1}{2}},
\end{equation}
for the Kraus rank-1 operation $\mathcal{G}_\eta$.
Then the Kraus operator $\eta^{\nicefrac{1}{2}}$ and therefore $\eta$
can be estimated by quantum process tomography for trace non-increasing channels~\cite{BSS+10}.
In the other case $\eta\nleq \pi(I)$,
the change in inner product is implemented by the operation  
$\mathcal{E}_{\kappa\eta}$ from which $\kappa \eta$ is estimated similarly; however, 
the above procedure does not yield $\kappa$ and $\eta$ separately.
\begin{figure}
    \begin{center}
    \includegraphics[width = 0.7\columnwidth]{Fig2.pdf}
    \end{center}
    \caption{Commutative diagram showing the action of $\mathcal{F}_\eta$ decomposed in terms of $\mathcal{G}_\eta$ and $\mathcal{R}_\eta$.}
    \label{fig:Etilde}
\end{figure}

We now discuss how to implement dynamics generated by 
a diagonalizable Hamiltonian $H_{\rm PT}$ with unbroken PT symmetry in finite dimensions
over a time $t \ge 0$,
by building on our framework for changing inner product.
The dynamical transformation generated by~$H_{\rm PT}$ is
\begin{equation}
\label{eq:UPT}
    \rho \stackrel{\mathcal{U}_{\rm PT}}{\mapsto} \kappa U_{\rm{PT}}\rho U^\dagger_{\rm{PT}}, \; U_{\rm{PT}} := \text{e}^{-\text{i}H_{\rm {PT}}t/\hbar},
\end{equation}
for some $\kappa\in(0,1)$, where both $\rho$, $\mathcal{U}_{\rm PT}(\rho)\in\mathcal{D}(\mathscr{H})$ represent states under $\pi$.
We show that this dynamics can be implemented by first changing the inner product, 
then applying an appropriate unitary channel and finally reversing the change in inner product.
To explain this sequence, we consider an arbitrary pure state represented by $\ket{\psi}\in \mathscr{H}$,
which is to be transformed to that represented by $\sqrt{\kappa}U_{\rm PT}\ket{\psi}\in \mathscr{H}$ 
(lower row of Fig.~\ref{fig:PTdynamics}).
Next, we compute a metric operator $\eta\leq \pi(I)\in \mathcal{B}(\mathscr H)$
that satisfies the quasi-Hermiticity condition
\begin{equation}\label{eq:quasiHermitian}
H^\dagger_{\rm {PT}} = \eta H_{\rm {PT}} \eta^{-1};
\end{equation}
the existence of such an $\eta$ is guaranteed 
as $H_{\rm PT}$ has unbroken PT symmetry~\cite{Mos02}.
The Hamiltonian $H_{\rm PT}$ is self-adjoint with respect to the inner product of the new Hilbert space $\mathscr{H}_{\eta}$.
Therefore, $U_{\rm PT}$ represents unitary dynamics on $\mathscr{H}_{\eta}$,
which constitutes the second step of the sequence.
Prior to implementing~$U_{\rm PT}$, we transform $\ket{\psi}\in\mathscr{H}$ to $\ket{\psi}\in\mathscr{H}_{\eta}$ 
via a change in inner product using $\mathcal{I}_\eta$.
Finally, the transformation from $U_{\rm PT}\ket{\psi}\in\mathscr{H}_{\eta}$ to $\sqrt{\kappa}U_{\rm PT}\ket{\psi}\in\mathscr{H}$
is equivalent to reversing the change in inner product using  $\mathcal{I}_{\eta^{-1}}$ with probability $\kappa$. 
This sequence extends to general mixed states by the application of lift, lift${}_\eta$ maps 
and linearity (upper row of Fig.~\ref{fig:PTdynamics});
here  $ \widetilde{\mathcal{U}}_{\rm{PT}}\in\mathcal{B}\left(\mathscr{H}_\eta\right)$ 
is the unitary channel satisfying
$\operatorname{lift}_\eta\left(U_{\rm PT}\ket{\psi}\right) =  \widetilde{\mathcal{U}}_{\rm{PT}}\left(\operatorname{lift}_\eta\left(\ket{\psi}\right)\right)$,
for all $\ket{\psi}\in\mathscr{H}_\eta$.

\begin{figure}
\begin{center}
\includegraphics[width = \columnwidth]{Fig3.pdf}
\end{center}
\caption{Diagram showing implementation of a PT-symmetric dynamics using change in inner product and unitary dynamics.}
\label{fig:PTdynamics}
\end{figure}

PT-symmetric dynamics in Eq.~\eqref{eq:UPT} can be expressed as a sequence of channels 
acting exclusively on $\mathcal{B}(\mathscr{H})$, thereby paving the way for experimental simulation 
of PT-symmetric systems.
Following the upper row of Fig.~\ref{fig:PTdynamics}, 
we start by expressing $\mathcal{U}_{\rm{PT}}$ (Eq.~\eqref{eq:UPT})
as  $\mathcal{U}_{\rm{PT}}= \mathcal{E}_{\kappa \eta^{-1}} \circ  \widetilde{\mathcal{U}}_{\rm{PT}} \circ \mathcal{E}_\eta$.
Similar to~Eq.\ \eqref{eq:F},
we express the reverse change in inner product as 
$\mathcal{E}_{\kappa\eta^{-1}}= \mathcal{G}_{\kappa\eta^{-1}}\circ \mathcal{R}_{\kappa\eta^{-1}}$,
for  $\mathcal{G}_{\kappa\eta^{-1}}: \mathcal{B}(\mathscr{H})\to \mathcal{B}(\mathscr{H}): 
\mathcal{G}_{\kappa\eta^{-1}}(M) = \kappa \eta^{-\nicefrac{1}{2}}M\eta^{\nicefrac{1}{2}} $ and
$\mathcal{R}_{\kappa\eta^{-1}}: \mathcal{B}(\mathscr{H}_{\eta})\to \mathcal{B}(\mathscr{H})$ is the 
channel effecting the change in representation form $\pi_\eta$ to $\pi$.
We then rewrite~$\mathcal{U}_{\rm{PT}}$ as
\begin{equation}\label{eq:PTUnitaryoperational}
    \mathcal{U}_{\rm{PT}}= {\mathcal{G}}_{\kappa \eta^{-1}} \circ (\mathcal{R}_{\kappa\eta^{-1}} \circ \widetilde{\mathcal{U}}_{\rm{PT}} \circ \mathcal{R}_{\eta})\circ \mathcal{G}_\eta,
\end{equation}
which is the desired decomposition. The channels ${\mathcal{G}}_{\kappa \eta^{-1}}$, $ \mathcal{G}_\eta$ have 
single Kraus operators $\sqrt{\kappa}\eta^{-\nicefrac{1}{2}}$ and $\eta^{\nicefrac{1}{2}}$ respectively.
The maps $\mathcal{R}_{\eta}$,  $\mathcal{R}_{\kappa\eta^{-1}}$
only effect change in representation and operationally are equivalent to no change. 
Finally, the transformation $\mathcal{R}_{\kappa\eta^{-1}} \circ \widetilde{\mathcal{U}}_{\rm{PT}} \circ \mathcal{R}_{\eta}$
implements a channel 
corresponding to the unitary Kraus-operator $\eta^{\nicefrac{1}{2}}U_{\rm PT}\eta^{\nicefrac{-1}{2}}$ acting
on~$\mathscr{H}$, generated by the Hamiltonian 
\begin{equation}\label{eq:selfadjointHPT}
    h_{\rm PT} = \eta^{\nicefrac{1}{2}}H_{\rm PT}\eta^{\nicefrac{-1}{2}} \in \mathcal{B}(\mathscr{H}),
\end{equation}
which can be verified to be self-adjoint, i.e.\ $h_{\rm PT}^\dagger = h_{\rm PT}$, 
using the quasi-Hermiticity condition in Eq.~\eqref{eq:quasiHermitian}.

We now design a qutrit procedure for an agent to simulate successfully the change in inner product by $\eta \le \pi(I)$ 
of a qubit system with algebra $\mathcal{A}$, which is represented on a 
two-dimensional Hilbert space $\mathscr{H}_2$ by $\pi$.
Our procedure, which shall simulate the operation $\mathcal{G}_\eta$ (Fig.~\ref{fig:Etilde}),
uses a unitary operation on the three-dimensional Hilbert space $\mathscr{H}_3=\mathscr{H}_2 \oplus \mathscr{H}_1$ 
followed by a projective measurement on to $\mathscr{H}_2$ and postselection, as we now explain.
For any $\eta \le \pi(I)$, we first construct the metric operator
\begin{equation}
\label{eq:tildeeta}
\tilde{\eta}:=\frac{1}{\|\eta\|}\eta
\implies\mathcal{G}_\eta = \|\eta\|\mathcal{G}_{\tilde{\eta}},
\end{equation}
and the unitary operator $U_{\tilde{\eta}}\in \mathcal{B}(\mathscr{H}_3)$ that satisfies
\begin{equation}
\label{eq:simulatingFeta}
    \mathcal{G}_{\tilde{\eta}}(\rho)\oplus \bm{0} = PU_{\tilde{\eta}}\sigma U_{\tilde{\eta}}^\dagger P,\, \sigma:=\rho\oplus \bm{0},
    \;\forall \rho\in \mathcal{B}(\mathscr{H}_2),
\end{equation}
where $P$ is the orthogonal projector on~$\mathscr{H}_2$. 
The matrix representation of~$U_{\tilde{\eta}}$ is (see Appendix~\ref{sec:matrixUeta})
\begin{equation}
\label{eq:Ueta}
    \left[U_{\tilde{\eta}}\right]
    = \begin{pmatrix}
    \left[\tilde{\eta}\right]^{\frac12} &\bm{u} \\ -\text{e}^{\text{i} \theta}\bar{\bm{u}}^{\top} &\text{e}^{\text{i}\theta}r 
    \end{pmatrix},
    \text{spec}\left(\tilde{\eta}^{\nicefrac12}\right)=\{1,r\},
    \theta\in [0,2\pi),
\end{equation}
where~$[\;]$ denotes matrix representation,
$\bm{u}$ is the eigenvector of $[\tilde{\eta}]^{\nicefrac12}$ with eigenvalue $r$
and $\| \bm{u}\| = \sqrt{1-r^2}$.
Furthermore,
$\bar{\bm{u}}^{\top}$ is the Hermitian conjugate of the vector $\bm{u}$.
Both~$\theta$ and the global phase of $\bm{u}$ are free parameters.
The qutrit unitary operator $U_{\tilde{\eta}}$ is part of the overall 
simulation procedure (Eq.~\eqref{eq:simulatingFeta}).

Now we explain how an agent 
can sequentially apply each operator in Eq.~\eqref{eq:simulatingFeta}
to simulate $\mathcal{G}_{\eta}$ (Fig.~\ref{fig:flowchart}).
The agent is provided with a description of $2\times2$ matrix~$[\eta]$,
in the logical basis~$\{\ket0,\ket1\}$ 
and a quantum state $\sigma$ (Eq.~\eqref{eq:simulatingFeta}).
The task is to generate the state
$\left({\mathcal{G}_{\eta}(\rho)\oplus \bm{0}}\right)/{{\rm tr}\left(\mathcal{G}_{\eta}(\rho)\oplus \bm{0}\right)}$
with probability ${\rm tr}\left(\mathcal{G}_{\tilde{\eta}}(\rho)\oplus \bm{0}\right)$.
The agent first computes $\left[\tilde{\eta}\right]$ (Eq.~\eqref{eq:tildeeta}) and
$\left[U_{\tilde{\eta}}\right]$ (Eq.~\eqref{eq:Ueta})
and then applies physical operations corresponding to~$U_{\tilde{\eta}}$ on~$\sigma$ followed by projective measurement~$P$ (Eq.~\eqref{eq:simulatingFeta}).
For non-zero measurement outcome,
which occurs with probability ${\rm tr}\left(\mathcal{G}_{\tilde{\eta}}(\rho)\oplus \bm{0}\right)$,
the post-measurement state obtained is
(Eq.~\eqref{eq:tildeeta})
\begin{equation}
   \frac{\mathcal{G}_{\tilde{\eta}}(\rho)\oplus \bm{0}}{{\rm tr}(\mathcal{G}_{\tilde{\eta}}(\rho)\oplus \bm{0})} = \frac{\mathcal{G}_\eta(\rho)\oplus \bm{0}}{{\rm tr}(\mathcal{G}_\eta(\rho)\oplus \bm{0})}.
\end{equation}
The agent discards the state if the measurement outcome is zero.
This concludes the simulation procedure.
The agent may further estimate the success probability ${\rm tr}\left(\mathcal{G}_{\tilde{\eta}}(\rho)\oplus \bm{0}\right)$,
if required, by repeating the simulation procedure on a large number of copies
of $\sigma$ provided to them and then calculating the ratio of non-zero measurement outcomes to the total
number of copies used~\footnote{
The ratio of non-zero measurement outcomes to the total
number of copies approaches the success probability by the law of large numbers~\cite{DKLM05}.
Our setting assumes that multiple copies of the state $\sigma$
are provided to the agent by an external agent who has the knowledge of
$\sigma$, and not prepared by the agent implementing the inner-product changing channel, say, by cloning.}.


In Appendix~\ref{sec:qubitPTsymmetry}, we provide an explicit 
procedure to simulate the dynamics (Eq.~\eqref{eq:UPT}) 
of the qubit PT-symmetric Hamiltonian 
~\cite{BB98},
by sequentially applying the operators in Eq.~\eqref{eq:PTUnitaryoperational} and by using the qutrit 
simulation procedure to implement~$\mathcal{G}_{\eta},\mathcal{G}_{\kappa\eta^{-1}}$.
In Appendix~\ref{sec:dsimulation}, we design a simulation procedure, similar to our qutrit procedure given above, 
for changing the inner product of a $d$-dimensional 
system using a $2d$-dimensional system for any positive integer~$d$. Furthermore, we
use our procedure to simulate the dynamics of a $d$-dimensional PT-symmetric Hamiltonian
by using only $2d$ dimensions, instead of $d^3$ dimensions as required in the Stinespring dilation approach \cite{Sti95}.

\begin{figure}
\begin{center}
\includegraphics[width = \columnwidth]{Flowchart.pdf}
\end{center}
\vspace{-0.5cm}
\caption{Simulation of the application of $\mathcal{G}_{{\eta}}$ on $\rho$ with success probability
${\rm tr}\left(\mathcal{G}_{\tilde{\eta}}(\rho)\oplus \bm{0}\right)$}. 
\label{fig:flowchart}
\end{figure}

We also design a scheme to verify tomographically whether a prover
can perform an arbitrary change of inner product using our qutrit simulation procedure.
Input to the verification scheme is a threshold function $D_{\rm th}:\mathcal{B}(\mathscr{H}_2)\to (0,1)$ given as a black-box.
The output is `accept' if $\|\mathcal{G}_\eta\oplus \bm{0} - \hat{\mathcal{G}}_\eta\|_{1\to 1}\leq D_{\rm th}(\eta)$ 
or `reject' otherwise, where $\hat{\mathcal{G}}_\eta:\mathcal{B}(\mathscr{H}_3)\to \mathcal{B}(\mathscr{H}_3)$ 
represents a tomographic reconstruction of the qutrit process implemented by the prover, 
$\mathcal{G}_\eta\oplus \bm{0}$ extends the action of $\mathcal{G}_\eta$ to $B(\mathscr{H}_3)$
and $\|\bullet\|_{1 \to 1}$ is the induced Schatten $(1\to 1)$-norm~\cite{Pau03}.
The verifier supplies to the prover a randomly chosen valid $\eta$, a positive integer 
$N$ sufficiently large for the process tomography~\cite{STM11}
and copies of the quantum states $\sigma_i$ encoding $\rho_i$ on demand,
where $\{\rho_i\}$ is chosen based on the tomography procedure in use. 
The prover returns $N$ copies of the qutrit states on which the change of inner product is successful
 as well as the success ratios for each $\rho_i$,
both of which are used by the verifier to reconstruct $\hat{\mathcal{G}_\eta}$.
To ensure that the verifier does not accept the process 
performed by a dishonest prover implementing only
qubit-unitary channels and randomly discarding the system,
it suffices to set the threshold to 
$D_{\rm th}(\eta) = \nicefrac{1}{3}(\lambda_1-\lambda_2)$,
where $\lambda_1>\lambda_2>0$ are the eigenvalues of $\eta$ (see Appendix~\ref{sec:threshold}). 


In conclusion, we have three major results.
First, we have operationalized Hilbert-space inner-product change in a way that is both observable and fully compatible with axiomatic quantum mechanics. 
Physically we can understand this inner-product change as 
a lossy quantum operation effecting a change in norm.
This lossy operation
is reminiscent of how superluminality is reconciled by electromagnetic absorption~\cite{SKC93},
with loss in our case forbidding past counterfactual claims.
Consistency of our work is proven using C$^*$ algebra and representations.
Alternatively,
our claims can be verified experimentally by conducting two physically distinct experiments.
One experiment is for the lower-dimensional lossy quantum operation 
and the other experiment is for the higher-dimensional unitary channel with both realizations yielding the same success ratio and 
measurement statistics for a given task.
Our theory fully explains unbroken PT-symmetric quantum mechanics in all its forms as being about changing Hilbert-space inner product and observing its consequences.
Our scheme for simulating qubit PT-symmetric Hamiltonians 
only requires one extra Hilbert-space dimension and no interaction with the environment,
which eliminates the requirements for 
multiple subsystems and entangling operations used in existing schemes~\cite{GS08a,WLG+19,TWY+16,WLG+19,XWZ+19,GZL+21}.
We also show how to simulate $d$-dimensional ($d\geq 2$) PT-symmetric Hamiltonians using
$2d$ dimensions, as opposed to using $d^3$ dimensions in the Stinespring dilation approach.
Our results open possibilities for simulating PT-symmetric dynamics on new experimental platforms, 
such as transmons, where high fidelity qutrit-unitary operations have already been demonstrated~\cite{BRS+20,KYSL20}. 



\acknowledgements 
This project is supported by the Government of Alberta and 
by the Natural Sciences and Engineering Research Council of Canada (NSERC).
S.\ K.\ is grateful for a University of Calgary Eyes High International Doctoral Scholarship and an Alberta Innovates Graduate Student Scholarship.
A.\ A.\ acknowledges support through a Killam 2020 Postdoctoral Fellowship.

\begin{appendix}

\section{Constructing a representation of the C\texorpdfstring{$^{*}$}{*} algebra on the Hilbert space with a different inner product}\label{sec:newHilbertspace}
In this section, we show the construction of the Hilbert space $\mathscr{H}_\eta$ with inner product related 
to that of $\mathscr{H}$ by the metric
operator $\eta$,
the construction of a ${}^*$-representation of the C$^{*}$ algebra $\mathcal{A}$ on this new Hilbert space, 
and
finally the representation of states in $\mathcal{S}$ using density operators on $\mathscr{H}_\eta$.


\subsection{Constructing a new Hilbert space from the metric operator}
For a possibly infinte dimensional Hilbert space~$\mathscr{H}$, we denote by $\mathcal{L}(\mathscr{H})$ and $\mathcal{B}(\mathscr{H})$ the algebra of 
linear and bounded linear operators on~$\mathscr{H}$ respectively. We also denote by $\mathcal{D}\left(\mathscr{H}\right) := 
\{\rho \in \mathcal{B}(\mathscr{H}) : \rho \ge 0, \rho^\dagger = \rho, \text{tr}\rho\le 1 \}$ the set 
of density operators acting on~$\mathscr{H}_\eta$. 
\begin{definition}[\cite{Con07}]
The adjoint of an operator $A \in \mathcal{B}(\mathscr{H})$ is the unique operator $A^\dagger\in \mathcal{B}(\mathscr{H})$ satisfying
\begin{equation}
    \braket{\phi|A|\psi} = \overline{\braket{\psi|A^\dagger|\phi}} \;\; \forall \ket{\phi},\ket{\psi} \in \mathscr{H}.
\end{equation} 
The operator $A$ is self-adjoint if $A = A^\dagger$.
\end{definition}
\begin{definition}[\cite{Con07}]
An operator $A \in \mathcal{L}(\mathscr{H})$ is positive-definite if 
\begin{equation}
    \braket{\phi|A|\phi} >0  \;\; \forall \ket{\phi} \in \mathscr{H},\;\ket{\phi} \ne 0.
\end{equation} 
\end{definition}
The following theorem is adapted from the Appendix A of Ref.~\cite{SGH92}.
\begin{theorem}
\label{thm:Heta}
For any Hilbert space $\mathscr{H} = \left(\mathscr{V},\langle\,\vert\,\rangle\right)$ 
and a self-adjoint, positive-definite operator $\eta \in \mathcal{B}(\mathscr{H})$, 
\begin{enumerate}
    \item the sesquilinear form 
    \begin{equation}
    \label{eq:newip}
        \langle\bullet|\bullet \rangle_\eta := \langle\bullet|\eta|\bullet\rangle
    \end{equation}
    is non-degenerate, therefore an inner product on $\mathscr{V}$.
    \item The vector space $\mathscr{V}$ is complete with respect to the norm induced by the inner product 
    $\braket{\bullet|\bullet}'$, therefore $\mathscr{H}_\eta = \left(\mathscr{V},\langle\,\vert\,\rangle_\eta\right)$  is a Hilbert space.
\end{enumerate}
\end{theorem}

\subsection{Constructing a \texorpdfstring{${}^{*}$}{*}-representation on the new Hilbert space}

We now construct a ${}^*$-representation of the algebra $\mathcal{A}$ on the new Hilbert space $\mathscr{H}_\eta$ constructed in Theorem~\ref{thm:Heta}.
In the following, $\mathscr{H}$ and $\mathscr{H}_\eta$ are two Hilbert spaces with their inner product related by
the metric operator $\eta$ as in Theorem~\ref{thm:Heta}, $\mathcal{A}$ is a C${}^*$ algebra of operators and
$\pi:\mathcal{A} \to \mathcal{B}(\mathscr{H})$ is a ${}^*$-representation of $\mathcal{A}$.
We first establish some results required for constructing such a new representation. The following lemma,
which establishes the inverse of the metric operator $\eta$, is adapted from the Appendix A of Ref.~\cite{SGH92}.
\begin{lemma}
Any self-adjoint and positive-definite operator $\eta \in \mathcal{B}(\mathscr{H})$
is invertible. 
Furthermore, the inverse $\eta^{-1}\in \mathcal{B}(\mathscr{H})$ is self-adjoint and positive-definite.
\end{lemma}
We next show that the bounded operator spaces on $\mathscr{H}$ and $\mathscr{H}_\eta$ coincide.
\begin{lemma}\label{lem:boundedopspaces}
    $M\in\mathcal{B}\left(\mathscr{H}\right)$ if and only if $M\in\mathcal{B}\left(\mathscr{H}_\eta\right)$.
\end{lemma}
\begin{proof}
    Let $\|\bullet\|,\|\bullet\|_\eta$ respectively denote the operator norms in~$\mathscr{H}$, $\mathscr{H}_\eta$.
    From Eq.~\eqref{eq:newip}, 
    \begin{equation}\label{eq:twonorms}
        \left\|M\right\|_\eta = \left\|\eta^{\nicefrac{1}{2}}M\eta^{\nicefrac{-1}{2}}\right\|,\, \forall\, M\in\mathcal{B}(\mathscr{H}).
    \end{equation}
    If $M\in \mathcal{B}\left(\mathscr{H}\right)$, then
    \begin{equation}
        \left\|M\right\|_\eta \leq \left\|\eta^{\nicefrac{1}{2}}\right\|\cdot\left\|M\right\|\cdot\left\|\eta^{\nicefrac{-1}{2}}\right\|<\infty
    \end{equation}
    and therefore, $M\in \mathcal{B}\left(\mathscr{H}_\eta\right)$. 
    To verify the reverse implication, note that
    \begin{eqnarray}
        \left\|M\right\| &=& \left\|\eta^{\nicefrac{-1}{2}}\left(\eta^{\nicefrac{1}{2}}M\eta^{\nicefrac{-1}{2}}\right)\eta^{\nicefrac{1}{2}}\right\|\nonumber \\
        &\leq & \left\|\eta^{\nicefrac{-1}{2}}\right\|\cdot \left\|M\right\|_\eta\cdot  \left\|\eta^{\nicefrac{1}{2}}\right\| 
        <\infty,
    \end{eqnarray}
     for all $M\in\mathcal{B}\left(\mathscr{H}_\eta\right) $.
\end{proof}
The next lemma relates ${}^\dagger$ to ${}^\ddagger$, with the latter denoting the adjoint with respect to the inner product $\braket{\bullet|\bullet}'$ of $\mathscr{H}_\eta$.
\begin{lemma}
For any $M \in \mathcal{B}(\mathscr{H})$, $M^\ddagger = \eta^{-1}M^\dagger \eta$. Additionally, $\eta^\ddagger = \eta$.
\end{lemma}
\begin{proof}
By definition of $\ddagger$, we have 
\begin{eqnarray}
\label{eq:ddagger}
  \langle \phi|M|\psi\rangle_\eta = \overline{\langle \psi|M^\ddagger|\phi\rangle_\eta},\,
  &\forall&\, \ket{\psi},\ket{\phi}\in \mathscr{V}, \nonumber\\
  &\forall& \, M\in \mathcal{B}(\mathscr{H}).  
\end{eqnarray}
Using Eq.~\eqref{eq:newip}, 
\begin{eqnarray}
\label{eq:ddaggerformula}
    \langle \phi|M|\psi\rangle_\eta &=  \langle \phi|\eta M|\psi\rangle = \overline{\langle\psi |M^\dagger\eta|\phi\rangle} =  
    \overline{\langle\psi |\eta^{-1}M^\dagger\eta|\phi\rangle_\eta}, \nonumber\\
    &\forall\, \ket{\psi},\ket{\phi}\in \mathscr{V}, \, \forall \, M\in \mathcal{B}\left(\mathscr{H}\right).
\end{eqnarray}
Then $M^\ddagger = \eta^{-1}M^\dagger\eta$ follows from the comparison of Eq.~\eqref{eq:ddagger} and Eq.~\eqref{eq:ddaggerformula},
and $\eta^\ddagger = \eta$ can be obtained by substituting $M=\eta$ in this relation.
\end{proof}

We are now ready for the construction of a ${}^*$-representation of $\mathcal{A}$ on $\mathscr{H}_\eta$.
\begin{theorem}
\label{lem:pieta}
Let $\pi:\mathcal{A} \to \mathcal{B}(\mathscr{H})$ be a ${}^*$-representation of a C${}^*$ algebra $\mathcal{A}$.
Then $\pi_{\eta}: \mathcal{A} \to \mathcal{L}(\mathscr{H}_\eta): A\mapsto \eta^{\nicefrac{-1}{2}}\pi(A) \eta^{\nicefrac{1}{2}} $
is a ${}^*$-representation of $\mathcal{A}$ on $\mathscr{H}_\eta$.
\end{theorem}
\begin{proof}
The range of the map $\pi_{\eta}$ is $\mathcal{B}(\mathscr{H}_\eta)$, which follows from the fact that 
 $\eta^{1/2},\eta^{-1/2}, \pi(A) \in \mathcal{B}\left(\mathscr{H}_\eta\right)$, following Lemma~\ref{lem:boundedopspaces}
 and definition of $\pi$, for all $A\in\mathcal{A}$. 
The map $\pi_{\eta}$ is linear by construction, and it is product preserving
because $\pi_{\eta}(AB) = \eta^{\nicefrac{-1}{2}}\pi(AB)\eta^{\nicefrac{1}{2}} = \eta^{\nicefrac{-1}{2}}\pi(A)\left(\eta^{\nicefrac{1}{2}}\eta^{\nicefrac{-1}{2}}\right)\pi(B)\eta^{\nicefrac{1}{2}} = \pi_{\eta}(A)\pi_{\eta}(B)$. Therefore $\pi_\eta$ is a representation. 
The representation $\pi_{\eta}$ is also ${}^{*}$-preserving, as 
\begin{eqnarray}
    \pi_{\eta}(A^*) &=& \eta^{\nicefrac{-1}{2}}\pi\left(A^*\right)\eta^{\nicefrac{1}{2}} = \eta^{\nicefrac{-1}{2}}\pi(A)^\dagger \eta^{\nicefrac{1}{2}} 
    = \eta^{\nicefrac{1}{2}}\pi(A)^\ddagger \eta^{\nicefrac{-1}{2}} \nonumber \\
    &=&\left(\eta^{\nicefrac{-1}{2}}\pi(A) \eta^{\nicefrac{1}{2}}\right)^\ddagger = \pi_{\eta}(A)^\ddagger.  
\end{eqnarray}
Therefore $\pi_{\eta}$ is a  ${}^{*}$-representation of $\mathcal{A}$ on~$\mathscr{H}_\eta$.
\end{proof}


\subsection{Representing states on the new Hilbert space}
\label{sec:liftmap}
We now characterize the set of states represented by the set of density operators~$\mathcal{D}\left(\mathscr{H}_\eta\right)$
and construct vector representation of the pure states under $\pi_\eta$.
Recall that ${}^{\#}\pi:\rho \mapsto \omega$ such that $\omega(A) = \text{tr}(\rho \pi(A))\  \forall A \in \mathcal{A}$. 
The map ${}^{\#}\pi_\eta$ is defined analogously for the $\pi_\eta$ representation. 
In the following lemma
we show how the density operators acting on $\mathscr{H}$ and $\mathscr{H}_\eta$ are related,
which we further use to prove that the set of states represented under 
$\pi_\eta$ coincides with that represented under $\pi$, namely $\mathcal{S}$. 


\begin{lemma}
   An operator $\rho_\eta \in\mathcal{D}\left(\mathscr{H}_\eta\right) $ if and only if
   \begin{equation}\label{eq:rho_eta}
       \rho_\eta = \eta^{\nicefrac{-1}{2}}\rho \eta^{\nicefrac{1}{2}}
   \end{equation}
    for some  $\rho \in\mathcal{D}\left(\mathscr{H}\right)$.
    \end{lemma}
\begin{proof}
Let $\{\ket{e_i}\}$ be an orthonormal basis of $\mathscr{H}$. We first show that $\{\ket{f_i}:= \eta^{\nicefrac{-1}{2}}\ket{e_i}\}$
is an orthonormal basis of $\mathscr{H}_\eta$. The orthonormality of $\{\ket{f_i}\}$ follows from 
$\braket{f_i|f_j}' = \braket{e_i|\eta^{\nicefrac{-1}{2}}\eta\eta^{\nicefrac{-1}{2}} |e_j}' = \delta_{ij}$. 
We prove that $\{\ket{f_i}\}$ is a basis by showing that any $\ket{\phi} \in \mathscr{V}$ can be expressed
as $\ket{\phi} = \sum_i \braket{f_i|\phi}'\ket{f_i}$. Let $\ket{\psi} = \eta^{\nicefrac{1}{2}}\ket{\phi}$.
Then $\ket{\psi} = \sum_i \braket{e_i|\psi}\ket{e_i}$. Now premultiplying by $\eta^{\nicefrac{-1}{2}}$ and substituting
$\ket{f_i} = \eta^{\nicefrac{-1}{2}}\ket{e_i}$, $\ket{\psi} = \eta^{\nicefrac{1}{2}}\ket{\phi}$ yields 
the desired expression $\ket{\phi} = \sum_i \braket{f_i|\phi}'\ket{f_i}$. Note that this sum is convergent because
the set $\{\ket{f_i}\}$ is orthonormal~\cite{Con07}.

To prove the forward implication, we note that Hilbert Schmidt norm of any $\rho \in\mathcal{D}\left(\mathscr{H}\right)$
is finite, i.e.\ $\sum_{i\in B}\|\rho\ket{e_i}\|^2<\infty$, and
${\rm tr}(\rho) = \sum_{i\in B} \langle e_i|\rho |e_i\rangle<\infty$; 
both these properties follow from $\rho$ being a trace-class operator~\cite{Con07}.
For $\rho_\eta\in\mathcal{B}(\mathscr{H}_\eta)$ satisfying Eq.~\eqref{eq:rho_eta}, 
\begin{eqnarray}
    \sum_{i\in B_\eta}\left\|\rho_\eta\ket{f_i}\right\|_\eta^2 &=& 
    \sum_{i\in B}\left\|\eta^{\nicefrac{-1}{2}}\rho\ket{e_i}\right\|_\eta^2 \nonumber\\ 
    &=& \sum_{i\in B}\langle e_i| (\eta^{\nicefrac{-1}{2}}\rho)^\dagger \eta (\eta^{\nicefrac{-1}{2}}\rho)|e_i\rangle \nonumber\\
    &=& \sum_{i\in B}\left\|\rho\ket{e_i}\right\|^2 <\infty.
\end{eqnarray}
Therefore, $\rho_\eta$ is a Hilbert-Schmidt operator in $\mathcal{B}(\mathscr{H}_\eta)$. 
Furthermore, 
\begin{eqnarray}\label{eq:rho_etatrace}
    {\rm tr}(\rho_\eta) &=& \sum_{i\in B_\eta}\langle f_i|\rho_\eta|f_i\rangle_\eta 
 = \sum_{i\in B}\langle e_i|\eta^{\nicefrac{-1}{2}}\eta \rho_\eta\eta^{\nicefrac{-1}{2}}|e_i\rangle \nonumber\\
   & =& \sum_{i\in B}\langle e_i| \rho|e_i\rangle  = {\rm tr}(\rho).
\end{eqnarray}
Therefore,  $\rho_\eta$ is a trace-class operator with $ {\rm tr}(\rho_\eta)\leq 1$ and hence $\rho_\eta \in\mathcal{D}\left(\mathscr{H}_\eta\right) $.
Similarly, the reverse implication that for any $\rho_\eta \in\mathcal{D}\left(\mathscr{H}_\eta\right) $, the operator
$\eta^{\nicefrac{1}{2}}\rho_\eta \eta^{\nicefrac{-1}{2}} \in \mathcal{D}\left(\mathscr{H}\right)$ is proved by 
starting with an orthonormal basis $\{\ket{f_i}\}$ for $\mathscr{H}_\eta$ and observing that 
$\{\eta^{\nicefrac{1}{2}}\ket{f_i}\}$ is an orthornormal basis for $\mathscr{H}$.

\end{proof}

We now show that both $\mathcal{D}(\mathscr{H})$ and  $\mathcal{D}(\mathscr{H}_\eta)$ represent the same set of states, $\mathcal{S}$,
under the respective~$^*$-representations.
\begin{lemma}
    $\rho_\eta\stackrel{\!\!{}^{\#}\!\pi_\eta}{\to} \omega$ if and only if $\rho\stackrel{\!\!{}^{\#}\!\pi}{\to} \omega$, where $\rho,\rho_\eta$
are related by Eq.~\eqref{eq:rho_eta}.

\end{lemma}
\begin{proof}
To prove the forward implication, note $\omega(A) = \text{tr}(\rho \pi(A)) \ \forall A$ by the definition of ${}^\#\pi$. Then
$\text{tr}(\rho \pi(A)) = \text{tr}(\rho_\eta \eta^{\nicefrac{-1}{2}}\pi(A) \eta^{\nicefrac{1}{2}})
=\text{tr}(\rho_\eta\pi_\eta(A))$ using the cyclic property of trace and the definition of $\pi_\eta$ respectively.
Therefore $\omega(A) = \text{tr}(\rho_\eta\pi_\eta(A)) \ \forall A$, and therefore, $\rho_\eta\stackrel{\!\!{}^{\#}\!\pi_\eta}{\to} \omega$.
The reverse implication can be proved by following the same steps in reverse order.

\end{proof}

Finally, we represent pure states in $\mathcal{S}$ by vectors in the Hilbert space $\mathscr{H}_\eta$.
Recall that a state $\omega \in \mathcal{S}$ has a vector-representation $\ket{\psi} \in \mathscr{H}$ under $\pi$ if 
\begin{equation}
    \omega(A) = \braket{\psi | \pi(A)|\psi} \;\; 
    \forall A \in \mathcal{A}.
\end{equation}
We now extend this definition to the representation $\pi_\eta$.
\begin{definition}
\label{def:vecrep}
A state $\omega \in \mathcal{S}$ has a vector representation $\ket{\psi} \in \mathscr{H}_\eta$ under $\pi_\eta$ if 
\begin{equation}
    \omega(A) = \braket{\psi | \pi_\eta(A)|\psi}_\eta \;\; \forall A \in \mathcal{A}.
\end{equation}
\end{definition}

Let the ball $\overline{B_1}(\mathscr{H}) := \{\ket{\psi} \in \mathscr{H}: \sqrt{\braket{\psi|\psi}} \le 1\}$ 
denote the set of normalized and subnormalized vectors in $\mathscr{H}$. 
Recall that the map $\operatorname{lift}:\overline{B_1}(\mathscr{H}) \to \mathcal{D}(\mathscr{H}):
\ket{\psi} \mapsto \ket{\psi}\bra{\psi}$
connects the vector representation $\ket{\psi}$ of a pure state $\omega$ to its density operator representation
$\ket{\psi}\bra{\psi}$ under $\pi$. 
We now construct an analogous map $\operatorname{lift}_\eta: \overline{B_1}(\mathscr{H}_\eta) \to \mathcal{D}\left(\mathscr{H}_\eta\right)$
for the representation $\pi_\eta$, where the ball 
$\overline{B_1}(\mathscr{H}_\eta) := \{\ket{\psi} \in \mathscr{H}_\eta: \sqrt{\braket{\psi|\psi}}_\eta \le 1\}$. 
\begin{definition}
The map $\operatorname{lift}_\eta: \overline{B_1}(\mathscr{H}_\eta) \to \mathcal{D}(\mathscr{H}_\eta)$ is defined to be the map that 
satisfies the following condition: for any state $\omega \in \mathcal{S}$ with vector representation
$\ket{\psi} \in \mathscr{H}_\eta$ and density operator representation $\rho_\eta \in \mathcal{D}(\mathscr{H}_\eta)$,
$\operatorname{lift}_\eta:\ket{\psi} \mapsto \rho_\eta$.
\end{definition}
We now derive the explicit action of $\operatorname{lift}_\eta$.
\begin{lemma}
\label{lem:lifteta}
The map $\operatorname{lift}_\eta$ has action 
$\operatorname{lift}_\eta:\overline{B_1}(\mathscr{H}_\eta)\to \mathcal{D}(\mathscr{H}_\eta):\ket{\psi} \mapsto \ket{\psi}\bra{\psi}\eta$.
\end{lemma}
\begin{proof}
Let $\operatorname{lift}_\eta:\ket{\psi} \mapsto \rho_\eta$ and ${}^\#\pi_\eta:\rho_\eta \mapsto \omega$. 
Following the definition of ${}^\#\pi_\eta$ and  $\operatorname{lift}_\eta$,
\begin{equation}
    \braket{\psi|\pi_\eta(A)|\psi}_\eta = \text{tr}\left(\rho_\eta\pi_\eta(A)\right) = \omega(A) \;\; \forall A.
\end{equation}
As $\braket{\psi|\pi_\eta(I)|\psi}_\eta  = \omega(I)\le 1$, $\ket{\psi} \in \overline{B_1}(\mathscr{H}_\eta)$.
Using Eq.~\eqref{eq:newip}, we get $\braket{\psi|\pi_\eta(A)|\psi}_\eta = \braket{\psi|\eta\pi_\eta(A)|\psi}$. Then using
the cyclic property of the trace, this expectation value can be expressed as
\begin{equation}
    \braket{\psi|\eta\pi_\eta(A)|\psi} = \text{tr}\left(\ket{\psi}\bra{\psi}\eta\pi_\eta(A)\right) \;\; \forall A,
\end{equation}
therefore, $\rho_\eta = \ket{\psi}\bra{\psi}\eta \in \mathcal{D}(\mathscr{H}_\eta)$.
This leads to the desired action of $\operatorname{lift}_\eta$.

\end{proof}

\section{{{Quantum operation} for implementing the changing inner product}}\label{sec:channelEeta}
In this section, we construct the quantum operation that implements the change in inner product by $\eta \le \pi(I)$. 
Change in inner product is defined by the identity isomorphism $\mathcal{I}_\eta:\mathscr{H}\to\mathscr{H}_\eta$ 
(see Fig.~1c in main text).
We now show how  $\mathcal{I}_\eta$ is extended to $\mathcal{B}(\mathscr{H}_\eta)$ through the map $\mathcal{E}_\eta$ defined below.
\begin{theorem}
Let $\eta \le \pi(I)$ and $\mathcal{E}_\eta:\mathcal{B}(\mathscr{H})\to \mathcal{L}(\mathscr{H}_\eta):M \mapsto M\eta$.
Then
\begin{enumerate}
    \item $\operatorname{range}(\mathcal{E}_\eta) \subseteq \mathcal{B}(\mathscr{H}_\eta)$,
    \item $\mathcal{E}_\eta$ satisfies the following commutative diagram:
    \begin{equation}
    \xymatrix{
    \mathcal{D}(\mathscr{H}) \ar[r]^{\mathcal{E}_\eta} &  \mathcal{D}(\mathscr{H}_\eta)   \\
    \overline{B_1}(\mathscr{H}) \ar[u]^{\operatorname{lift}} \ar[r]^{\mathcal{I}_\eta} 
      & \overline{B_1}(\mathscr{H}_\eta) \ar[u]^{\operatorname{lift}_\eta} }
    \end{equation}
    \item $\mathcal{E}_\eta$ is a quantum operation.
\end{enumerate}
\end{theorem}
\begin{proof}
To prove Statement 1, note that for any $M \in \mathcal{B}\left(\mathscr{H}\right)$, the operator $M\eta \in  \mathcal{B}\left(\mathscr{H}_\eta\right)$ because
\begin{equation}
    \|M\eta\|_\eta = \|\eta^{\nicefrac{1}{2}}\left(M\eta\right)\eta^{\nicefrac{-1}{2}}\|\leq \|\eta^{\nicefrac{1}{2}}\|^2\cdot\|M\|\le \infty,
\end{equation}
where the first equality follows from Eq.~\eqref{eq:twonorms}. 
The commutative diagram in Statement 2 follows immediately from the action of lift${}_\eta$ map
in Lemma \ref{lem:lifteta}. 

We now show that $\mathcal{E}_\eta$ is a valid quantum operation,
i.e.\ a completely-positive, trace non-increasing map.
To prove the positivity of $\mathcal{E}_\eta$,
let $M \ge 0$, so that it can be expressed as $M = A A^\dagger$~\cite{Con07}. 
Then 
$\mathcal{E}_\eta(M) = A A^\dagger \eta$, which can expressed as
$\mathcal{E}_\eta(M) = BB^\ddagger$ with $B = A\eta^{\frac12}$. 
Therefore $\mathcal{E}_\eta(M) \in \mathcal{B}(\mathscr{H}_\eta)$  
is positive if $M$ is positive, which proves the positivity of $\mathcal{E}_\eta$.

Complete positivity of $\mathcal{E}_\eta$ can be proven by showing that the map 
$\mathcal{E}_\eta \otimes \mathscr{I}_k:\mathcal{B}\left(\mathscr{H}\right)\otimes\mathcal{B}(\mathbb{C}^k) \to\mathcal{B}\left(\mathscr{H}_\eta\right)\otimes\mathcal{B}(\mathbb{C}^k)  $ 
is positive,
for every positive integer $k$,
where $\mathscr{I}_k$ denotes the identity map on $\mathcal{B}(\mathbb{C}^k)$. 
The action of the new map is given by
$\left[\mathcal{E}_\eta \otimes \mathscr{I}_k\right]\left(N\right)= N \left(\eta \otimes I_k\right)$,
with $I_k \in \mathcal{B}(\mathbb{C}^k)$
the identity operator. 
Operator 
$\left[\mathcal{E}_\eta \otimes \mathscr{I}_k\right](N)\in\mathcal{B}\left(\mathscr{H}_\eta\right)\otimes\mathcal{B}(\mathbb{C}^k) $
because $\|\eta \otimes I_k\| = \|\eta\|_\eta\cdot\|I_k\| = \|\eta\|_\eta$. 
For proving positivity,
let $N\in \mathcal{B}(\mathscr{H})\otimes \mathcal{B}(\mathbb{C}^k)$ be a positive operator,
so that $N = C C^\dagger$. Then $\left[\mathcal{E}_\eta \otimes \mathscr{I}_k\right](N) = C C^\dagger (\eta \otimes I_k)$, 
which can be expressed as $\left[\mathcal{E}_\eta \otimes \mathscr{I}_k\right] (N) =D D^\ddagger$ with
$D = C(\eta^{\frac12}\otimes I_k)$,
thereby proving positivity of $\left[\mathcal{E}_\eta \otimes \mathscr{I}_k\right](N)$ and consequently positivity of 
$\mathcal{E}_\eta \otimes \mathscr{I}_k$.

To prove that $\mathcal{E}_\eta$ is trace non-increasing, let $\rho \in \mathcal{D}(\mathscr{H})$
and note that trace is independent of the inner product (Eq.~\eqref{eq:rho_etatrace}).
We have $\text{tr}(\rho\eta) = \text{tr}(\eta^{\nicefrac{1}{2}}\rho\eta^{\nicefrac{1}{2}})$
and 
\begin{equation}\label{eq:tracedecreasing}
    \text{tr}(\eta^{\nicefrac{1}{2}}\rho\eta^{\nicefrac{1}{2}}) = 
    \text{tr}|\eta^{\nicefrac{1}{2}}\rho\eta^{\nicefrac{1}{2}}|
    \le \|\eta^{\nicefrac{1}{2}}\|^2\text{tr}|\rho| \le \text{tr}(\rho),
\end{equation}
where $|M| = \sqrt{\left(M^\dagger M\right)}$ and we used $|M| = M$ for any $M \ge 0$. 
The first inequality in Eq.~\eqref{eq:tracedecreasing} is a property of the trace norm~\cite{Con07}, and the last inequality in Eq.~\eqref{eq:tracedecreasing} follows from the fact that $\eta\leq \pi(I)$ and therefore, $\left\|\eta^{\nicefrac{1}{2}}\right\|^2\leq 1$.

\end{proof}

\section{Transformation of the Operators under Changing the Inner Product}\label{sec:commutation}
An inner product changing channel could modify the 
commutation relations between the operators. 
In this section, we demonstrate such a change
with an explicit example of a qubit system undergoing an inner product change.
Consider a qubit system undergoing change in inner product by
\begin{equation}
    \eta = \frac{1}{1 + r\sin\phi} \begin{pmatrix}
1 & -\text{i}r\sin\phi \\
\text{i}r\sin\phi & 1
 \end{pmatrix}, \quad 0\le r<1.
\end{equation}
The Pauli operators $X,Y,Z \in \mathcal{B}(\mathscr{H})$ acting on the original
Hilbert space along with the identity operator $I_2\in \mathcal{B}(\mathscr{H})$ generate the $\mathfrak{u}(2)$
algebra.
These operators transform according to Eq.~(5) in the main text
following the inner product change by $\eta$.
This transformation is given by the map $\mathcal{E}^{\rm op}_\eta$.
The transformed operators satisfy the commutation relations
\begin{align}
    &\left[\mathcal{E}^{\rm op}_\eta\left(X\right),\mathcal{E}^{\rm op}_\eta\left(Y\right)\right] =  2\text{i}a\ \mathcal{E}^{\rm op}_\eta\left(Z\right),\nonumber \\ 
    &\left[ \mathcal{E}^{\rm op}_\eta\left(I_2\right), \mathcal{E}^{\rm op}_\eta\left(Z\right) \right] = 2\text{i}(1-a)\  \mathcal{E}^{\rm op}_\eta\left(X\right), \nonumber \\ 
    & \left[\mathcal{E}^{\rm op}_\eta\left(Y\right),\mathcal{E}^{\rm op}_\eta\left(Z\right)\right] =  2\text{i}a\ \mathcal{E}^{\rm op}_\eta\left(X\right),\nonumber \\ 
    &\left[ \mathcal{E}^{\rm op}_\eta\left(I_2\right), \mathcal{E}^{\rm op}_\eta\left(X\right) \right] = -2\text{i}(1-a)\ \mathcal{E}^{\rm op}_\eta\left(Z\right), 
    \nonumber \\
    &\left[\mathcal{E}^{\rm op}_\eta\left(Z\right),\mathcal{E}^{\rm op}_\eta\left(X\right)\right] =  -2\text{i}(1-a)\ \mathcal{E}^{\rm op}_\eta\left(I_2\right)+2\text{i}a\mathcal{E}^{\rm op}_\eta\left(Y\right),\nonumber \\ 
    &\left[\mathcal{E}^{\rm op}_\eta\left(I_2\right),\mathcal{E}^{\rm op}_\eta\left(Y\right)\right] = \bm{0},
\end{align}
where $a = 1/(1+r\sin \phi)$. These commutation relations are different from 
those of $\mathfrak{u}(2)$ algebra for $r \ne 0$, or equivalently $a \ne 1$.


\section{{Matrix representation of the qutrit unitary operator that simulates
 change in inner product of a qubit system}}\label{sec:matrixUeta}
In this section, we derive the matrix representation of the qutrit unitary operator $U_{\tilde{\eta}}$
(see Eq.~(12) in main text) employed in the 
simulation of the change in inner product of a qubit system. Equation~(10) in the main text
requires that $PU_{\tilde{\eta}}P = \tilde{\eta}^{\nicefrac{1}{2}}$, so that
$U_{\tilde{\eta}}$ can be expressed as
\begin{equation}\label{eq:Utildeeta}
    [U_{\tilde{\eta}}] = \begin{pmatrix}
    \left[\tilde{\eta}\right]^{\frac12} & \bm{u} \\ {\bar{\bm{v}}}^\top &r\text{e}^{\text{i}\theta} 
    \end{pmatrix}
\end{equation}
for some vectors $\bm{u},\bm{v} \in \mathbb{C}^2$, a number $r \in [0,1]$ and a phase $\theta \in [0,2\pi)$.
The unitarity conditions $U_{\tilde{\eta}}^\dagger U_{\tilde{\eta}} = U_{\tilde{\eta}} U_{\tilde{\eta}}^\dagger = I_3$ lead to
\begin{align}
    \left[\tilde{\eta}\right] + {\bm{u}}\bar{\bm{u}}^\top  = I_2,
    \label{eq:unitary1}\\
    \left[\tilde{\eta}\right]^{\frac12}\bm{u} + r\text{e}^{{\rm i}\theta}\bm{v}=0,\label{eq:unitary2}\\   
    \bar{\bm{u}}^\top{\bm{u}}+r^2 =  1 \implies \|\bm{u}\| = 1-r^2 \label{eq:unitary3}.
\end{align}
Postmultiplying Eq.~\eqref{eq:unitary1} by $\bm{u}$ and substituting Eq.~\eqref{eq:unitary3} yields
$\left[\tilde{\eta}\right]\bm{u} = r^2\bm{u}$, which implies that $\bm{u}$ is the eigenvector
of $\left[\tilde{\eta}\right]^{\nicefrac{1}{2}}$ with eigenvalue $r$. Then Eq.~\eqref{eq:unitary2}
yields $\bm{v} = -\text{e}^{-{\rm i}\theta}\bm{u}$ as desired, 
with the global phase of $\bm{u}$ and $\theta$ being the free parameters.



\section{{Simulation of a qubit PT-symmetric Hamiltonian using single qutrit}}\label{sec:qubitPTsymmetry}

We now design a qutrit procedure that simulates the dynamics of a qubit Hamiltonian with unbroken PT symmetry.
Our design is based on the qutrit procedure for simulating the change in inner product of a qubit system, 
provided in the main text  (Fig.~4). 
We illustrate our Hamiltonian-simulation procedure using the PT-symmetric Hamiltonian $H_{\rm PT}$ from~\cite{BBJ02}. 

The matrix form of  $H_{\rm PT}$ is
\begin{equation}
\label{eq:qubitHPT}
    \left[H_{\rm PT}\right] = \begin{pmatrix}
    r\text{e}^{{\rm i}\phi} & s \\ s & r\text{e}^{-{\rm i}\phi}\end{pmatrix}, \quad s>r\sin\phi\ge 0.
\end{equation}
Dynamics generated by $H_{\rm PT}$ is denoted by the operator $\mathcal{U}_{\rm PT}$ (Eq.~(6) in main text),
\begin{equation}
    \rho \stackrel{\mathcal{U}_{\rm PT}}{\mapsto} \kappa U_{\rm{PT}}\rho U^\dagger_{\rm{PT}}, \; U_{\rm{PT}} := \text{e}^{-\text{i}H_{\rm {PT}}t/\hbar},\, \kappa = \frac{1}{\|\eta_2^{-1}\|}.
\end{equation}
As proved in the main text, $\mathcal{U}_{\rm PT}$ can be expressed as a sequence of operations acting exclusively on 
$ \mathcal{B}(\mathscr{H}_2)$,
\begin{equation}~\label{eq:simulatingUPT}
     \mathcal{U}_{\rm{PT}}= {\mathcal{G}}_{\kappa \eta^{-1}_2} \circ (\mathcal{R}_{\kappa\eta^{-1}_2} \circ \widetilde{\mathcal{U}}_{\rm{PT}} \circ \mathcal{R}_{\eta_2})\circ \mathcal{G}_{\eta_2},
\end{equation}
where
$\mathcal{R}_{\kappa\eta^{-1}} \circ \widetilde{\mathcal{U}}_{\rm{PT}} \circ \mathcal{R}_{\eta}: \mathcal{B}(\mathscr{H}_2)\to \mathcal{B}(\mathscr{H}_2)$
is the channel with unitary Kraus operator $\eta^{\nicefrac{1}{2}}U_{\rm PT}\eta^{\nicefrac{-1}{2}}$ and
the Kraus operator is generated by self-adjoint Hamiltonian $ h_{\rm PT} = \eta^{\nicefrac{1}{2}}H_{\rm PT}\eta^{\nicefrac{-1}{2}}$.

Our qutrit procedure for simulating $\mathcal{U}_{\rm PT}$ involves implementing each operation in Eq.~\eqref{eq:simulatingUPT}
using qutrit unitaries and measurements, as we now explain through Steps 1-4.
The input to the simulation procedure is $\rho \in \mathcal{B}(\mathscr{H}_2)$ 
embedded as $\sigma:= \rho\oplus \bm{0} \in \mathcal{B}(\mathscr{H}_3)$ and a time $t>0$.
The output of the procedure is the state $\frac{U_{\rm{PT}}\rho U^\dagger_{\rm{PT}}}{{\rm tr}\left( U_{\rm{PT}}\rho U^\dagger_{\rm{PT}}\right)}$
with probability $\frac{1}{\|\eta_2^{-1}\|}{\rm tr}\left( U_{\rm{PT}}\rho U^\dagger_{\rm{PT}}\right).$
The simulation steps are
\begin{enumerate}
    \item Calculate the metric operator and its inverse:
    The agent calculates $\eta_2$, $\eta_2^{-1}$ satisfying the quasi-Hermiticity condition $H^\dagger_{\rm PT} = \eta_2H_{\rm PT}\eta^{-1}_2$.
    A choice of $\eta_2$ and, therefore $\eta_2^{-1}$, is
\begin{equation}
    \left[\eta_2\right] = \frac{1}{s+r\sin\phi}\begin{pmatrix}
    s & -{\rm i}r\sin\phi \\{\rm i}r\sin\phi & s \end{pmatrix},
\end{equation}
\begin{equation}
    \left[\eta_2^{-1}\right] = \frac{1}{s-r\sin\phi}\begin{pmatrix}
    s & {\rm i}r\sin\phi \\-{\rm i}r\sin\phi & s \end{pmatrix},
\end{equation}
with $\|\eta_2\|=1$ and $\|\eta_2^{-1}\| =(s+r\sin\phi)/(s-r\sin\phi)$.

\item Simulate change in inner product by $\eta_2$: 
Agent implements the qutrit procedure (Fig.~4) to simulate $\mathcal{G}_{{\eta}_2}$ 
by setting $\tilde{\eta} = \eta_2$ and for a single copy of $\sigma$.
A choice of the qutrit unitary~$U_{{\eta}_2}$ (see Eq.~\eqref{eq:Utildeeta})
simulating the action of  $\mathcal{G}_{\eta_2}$ is
\begin{equation}\label{eq:Ueta2}
    U_{{\eta}_2} = \begin{pmatrix}
    (1+q)/2 & -{\rm i}(1-q)/2 & p \\
    {\rm i}(1-q)/2 & (1+q)/2 & -{\rm i}p \\
    -p & -{\rm i}p & q
    \end{pmatrix},
\end{equation}
where
\begin{equation}
    \quad q = \sqrt{\frac{s-r\sin\phi}{s+r\sin\phi}} = \frac{1}{\sqrt{\|\eta_2^{-1}\|}}, \  p = \sqrt{\frac{r\sin\phi}{s+r\sin\phi}}.
\end{equation}
The output of this step is the qutrit state 
$\frac{\eta_2^{\nicefrac{1}{2}}\rho\eta_2^{\nicefrac{1}{2}}}{{\rm tr}\left(\eta_2^{\nicefrac{1}{2}}\rho\eta_2^{\nicefrac{1}{2}}\right)}\oplus\bm{0} $
with probability ${\rm tr}\left(\eta_2^{\nicefrac{1}{2}}\rho\eta_2^{\nicefrac{1}{2}}\right)$.

\item Simulate the unitary evolution generated by $h_{\rm PT}$:
Agent calculates $h_{\rm PT}$ embedded in $\mathcal{B}(\mathscr{H}_3)$,
\begin{equation}
    \qquad\; [h_{\rm PT} \oplus \bm{0}] = \begin{pmatrix}    
    r\cos \phi & \sqrt{s^2 - r^2\sin^2\phi} & 0\\
    \sqrt{s^2 - r^2\sin^2\phi} & r\cos \phi & 0\\
    0 & 0 & 0\end{pmatrix},
\end{equation}
and implements the qutrit unitary operator $e^{-{\rm i}(h_{\rm PT}\oplus \bm{0})t}$,
which is equivalent to simulating the channel $
(\mathcal{R}_{\kappa\eta^{-1}_2} \circ \widetilde{\mathcal{U}}_{\rm{PT}} \circ \mathcal{R}_{\eta_2})$
in Eq.~\eqref{eq:simulatingUPT}. 
The output of this deterministic step is the qutrit state $\left(e^{-{\rm i}h_{\rm PT}t} \frac{\left(\eta_2^{\nicefrac{1}{2}}\rho\eta_2^{\nicefrac{1}{2}}\right)}
{{\rm tr}\left(\eta_2^{\nicefrac{1}{2}}\rho\eta_2^{\nicefrac{1}{2}}\right)}e^{{\rm i}h_{\rm PT}t}\right)\oplus\bm{0} $, 
provided Step 2 is successful.
\item Simulate change in inner product by $\kappa\eta_2^{-1}$:
Agent applies the qutrit procedure (Fig.~4)
to simulate $\mathcal{G}_{\kappa\eta_2^{-1}}$,
by setting $\tilde{\eta} = \kappa\eta^{-1}_2$.
Note that we have $\kappa = \frac{1}{\|\eta_2^{-1}\|} = q^2$ (Eqs.~\eqref{eq:simulatingUPT},~\eqref{eq:Ueta2}).
A choice of $U_{\kappa\eta^{-1}_2} $ is
\begin{equation}
    U_{\kappa{\eta}_2^{-1}} = \begin{pmatrix}
    (1+q)/2 & {\rm i}(1-q)/2 & p \\
    -{\rm i}(1-q)/2 & (1+q)/2 & {\rm i}p \\
    -p & {\rm i}p & q
    \end{pmatrix}.
\end{equation}
The output of this procedure is the qutrit state 
$\frac{\left(\eta^{\nicefrac{-1}{2}}_2e^{-{\rm i}h_{\rm PT}t}\eta_2^{\nicefrac{1}{2}}\rho\eta_2^{\nicefrac{1}{2}}e^{{\rm i}h_{\rm PT}t}\eta^{\nicefrac{-1}{2}}_2\right)}
{{\rm tr}\left(\eta^{\nicefrac{-1}{2}}_2e^{-{\rm i}h_{\rm PT}t}\eta_2^{\nicefrac{1}{2}}\rho\eta_2^{\nicefrac{1}{2}}e^{{\rm i}h_{\rm PT}t}\eta^{\nicefrac{-1}{2}}_2\right)}\oplus\bm{0}  = \frac{U_{\rm PT} \rho U_{\rm PT}^\dagger}{{\rm tr}\left( U_{\rm PT} \rho U_{\rm PT}^\dagger\right)} \oplus \bm{0}$ 
with probability  $\frac{{\rm tr}\left( U_{\rm PT} \rho U_{\rm PT}^\dagger\right)}{\|\eta_2^{-1}\|{\rm tr} \left(\eta_2^{\nicefrac{1}{2}}\rho\eta_2^{\nicefrac{1}{2}}\right)}$.
\end{enumerate}
Therefore, the output of the simulation procedure is the state 
$\frac{U_{\rm PT} \rho U_{\rm PT}^\dagger}{{\rm tr}\left( U_{\rm PT} \rho U_{\rm PT}^\dagger\right)} \oplus \bm{0}$
with success probability given by the combined probability of success in Steps 2,4, which is equal to
$\frac{1}{\|\eta_2^{-1}\|}{{\rm tr}\left( U_{\rm PT} \rho U_{\rm PT}^\dagger\right)}$.



\section{Simulation of change in inner product and PT-symmetric dynamics 
of a \texorpdfstring{$d$}{d}-dimensional system}~\label{sec:dsimulation}
We first explain a simulation procedure
to change the inner product of a $d$-dimensional system using $2d$ dimensions.
We assume that the algebra $\mathcal{A}$ of the system is represented on a $d$-dimensional Hilbert space $\mathscr{H}_d^{(s)}$ by $\pi$.
Similar to the qutrit simulation procedure explained in main text,
the agent simulating $\mathcal{G}_\eta$ for $\eta\leq\pi(I)$ first 
constructs the metric operator $\tilde{\eta} =\frac{1}{\|\eta\|}\eta$
and the unitary operator $U_{\tilde{\eta}}\in\mathcal{B}(\mathscr{H}_d^{(s)} \oplus \mathscr{H}_d^{(a)} )$
satisfying 
\begin{equation}\label{eq:Getad}
    \mathcal{G}_{\tilde{\eta}}(\rho)\oplus \bm{0} = PU_{\tilde{\eta}}\sigma U_{\tilde{\eta}}^\dagger P,\, \sigma:=\rho\oplus \bm{0},
    \;\forall \rho\in \mathcal{B}(\mathscr{H}_d^{(s)}),
\end{equation}
where $\bm{0}$ denotes the zero operator in $\mathcal{B}(\mathscr{H}_d^{(a)})$.
The matrix representation of a choice of $U_{\tilde{\eta}}$ is
\begin{equation}
\label{eq:Uetad}
    \left[U_{\tilde{\eta}}\right]
    = \begin{pmatrix}
    \left[\tilde{\eta}\right]^{\frac12} & \left[1-\tilde{\eta}\right]^{\frac12} \\ \left[1-\tilde{\eta}\right]^{\frac12} & - \left[\tilde{\eta}\right]^{\frac12}
    \end{pmatrix}.
\end{equation}
Agent then implements  $U_{\tilde{\eta}}$
followed by projective measurement and postselection on to the subspace $\mathscr{H}_d^{(s)}$.
All steps of the simulation procedure are similar to the qutrit simulation procedure for changing inner product
explained in the main text.

We now discuss how this simulation procedure for changing the inner product 
can be used for simulating PT-symmetric dynamics in $d$ dimensions
using a $2d$-dimensional system. Similar to the $d=2$ case discussed in Sec.\,\ref{sec:qubitPTsymmetry}, 
the input to the simulation procedure is $\rho \in \mathcal{B}(\mathscr{H}_d^{(s)})$ 
embedded as $\sigma:= \rho\oplus \bm{0} \in \mathcal{B}(\mathscr{H}_d^{(s)}\oplus\mathscr{H}_d^{(a)})$ and a time $t>0$.
The simulation steps are as follows:
\begin{enumerate}
    \item The agent calculates $\eta$ satisfying the quasi-Hermiticity condition $H^\dagger_{\rm PT} = \eta H_{\rm PT}\eta^{-1}$ with $\|\eta\|=1$.
    \item The agent implements the procedure described above to simulate $\mathcal{G}_{\eta}$ 
    by setting $\tilde{\eta} = \eta$ and
    for a single copy of $\sigma$ (Eq.~\eqref{eq:Getad}). 
    \item The agent calculates $h_{\rm PT} = \eta^{\nicefrac{1}{2}}H_{\rm PT}\eta^{-\nicefrac{1}{2}}$ 
    embedded in $\mathcal{B}(\mathscr{H}_d^{(s)}\oplus\mathscr{H}_d^{(a)})$,
    and implements the unitary operator $e^{-{\rm i}(h_{\rm PT}\oplus \bm{0})t}$.
    \item The agent applies the procedure described above to simulate $\mathcal{G}_{\kappa\eta^{-1}}$, 
    by setting $\tilde{\eta} = \kappa\eta^{-1}$ such that $\kappa = \frac{1}{\|\eta^{-1}\|}$ (Eq.~\eqref{eq:Getad}).
\end{enumerate}
The output of this procedure is the state $\frac{U_{\rm{PT}}\rho U^\dagger_{\rm{PT}}}{{\rm tr}\left( U_{\rm{PT}}\rho U^\dagger_{\rm{PT}}\right)}$
with probability $\frac{1}{\|\eta_2^{-1}\|}{\rm tr}\left( U_{\rm{PT}}\rho U^\dagger_{\rm{PT}}\right).$



\section{{Additional details on the verification scheme}}\label{sec:threshold}
We now prove a threshold distance $D_{\rm th}$ for the tomographic verification scheme, for the 
qutrit procedure simulating the change in inner product by an arbitrary $\eta$, provided in the main text.
The scheme allows
a verifier to distinguish an honest prover implementing the operation $\mathcal{G}_\eta$ 
from a dishonest prover failing to implement the same.
We assume that the dishonest prover implements only unitary operations, on the qubit subspace, drawn from the set  
$\{U_j \oplus \bm{1}: U_j\in\mathcal{B}\left(\mathscr{H}_2\right) \}$, where each $U_j$ is selected with probability $p_j$, 
and the system is discarded with probability $p:=1-\sum_jp_j<1$.
The quantum operation implemented by the dishonest prover is given by 
\begin{equation}
    \hat{\mathcal{G}}_\eta(\bullet) = \sum_jp_j \left(U_j \oplus \bm{1}\right)^\dagger\bullet \left(U_j \oplus \bm{1}\right).
\end{equation}
We now derive a lower bound for the induced Schatten $(1\to 1)$-norm distance~\cite{Pau03} between the 
inner-product changing operation~$\mathcal{G}_\eta\oplus \bm{0}$  
and the implemented operation $\hat{\mathcal{G}}_\eta$.
Note that 
\begin{equation}
    \|\mathcal{G}_\eta\oplus\bm{0}-\hat{\mathcal{G}}_\eta\|_{1\to 1} = \max_{T\in\mathcal{B}\left(\mathscr{H}_3\right)}\frac{  \|\mathcal{G}_\eta\oplus\bm{1}(T)-\hat{\mathcal{G}}_\eta(T)\|_{\rm tr}}{\|T\|_{\rm tr}},
\end{equation}
 where  $\|T\|_{\rm tr} = {\rm tr}\left(\sqrt{T^\dagger T}\right)$.
 For $T = \frac{1}{3}I_3$, $\|T\|_{\rm tr} = 1$. 
 Therefore,
 \begin{align}
      \|\mathcal{G}_\eta\oplus\bm{0}-\hat{\mathcal{G}}_\eta\|_{1\to 1} 
      &\ge 
      \frac{1}{3}\|\mathcal{G}_\eta\oplus \bm{0}(I_3)-\hat{\mathcal{G}}_\eta(I_3)\|_{\rm tr} \nonumber\\
      &= 
      \frac{1}{3}\|\eta\oplus\bm{1}-\sum_jp_j(U_j\oplus\bm{1})^\dagger I_3(U_j\oplus\bm{1})\|_{\rm tr} \nonumber\\
      &= 
       \frac{1}{3}\|\eta\oplus\bm{1}-(1-p)I_3\|_{\rm tr}.
 \end{align}
 We assume that the eigenvalues of $\eta$ are denoted by $\lambda_1,\lambda_2$.
 The eigenvalues satisfy  $1\geq\lambda_1>\lambda_2>0$ for any non-trivial $\eta$, i.e.\ $\eta\neq I_3$.
 The trace distance $\|\eta\oplus\bm{1}-(1-p)I_3\|_{\rm tr} = \vert \lambda_1-(1-p)\vert + \vert\lambda_2-(1-p)\vert$.
 For any $p\in[0,1)$, it can be further verified that $\|\eta\oplus\bm{1}-(1-p)I_3\|_{\rm tr}\geq {(\lambda_1-\lambda_2)}.$
 Therefore, 
 \begin{equation}
     D_{\rm th} :=  \frac{(\lambda_1-\lambda_2)}{3} \leq \|\mathcal{G}_\eta\oplus\bm{0}-\hat{\mathcal{G}}_\eta\|_{1\to 1}.
 \end{equation}
  The above given value for $D_{\rm th}$ allows the verifier to distinguish an honest prover from a dishonest one, provided the honest prover 
  implements the operation $\mathcal{G}_{\eta}$ with error less than $D_{\rm th}$, where error is quantified by the induced Schatten $(1\to 1)$-norm.

\end{appendix}
\bibliography{ref.bib}
\end{document}


\usepackage{amsmath, amsfonts, amssymb}
\usepackage{cancel}


\usepackage{maths_min}

\usepackage[pdftex, breaklinks=true, linktocpage,
	colorlinks=true, urlcolor=blue, linkcolor=blue, citecolor=red,
	pdfauthor={Harold Erbin}
]{hyperref}
\usepackage[all]{hypcap}


\pagestyle{plain}
\graphicspath{{images/}}

\numberwithin{equation}{section}

\DeclareUnicodeCharacter{00A0}{~}
\DeclareUnicodeCharacter{202F}{~}

\renewcommand{\Affilfont}{\small}
\newcommand{\email}[1]{\thanks{\href{mailto:#1}{\nolinkurl{#1}}}}

\usepackage[sort&compress, english]{cleveref}


\addbibresource{janis_newman.bib}
\iffalse
\bibliography{janis_newman.bib}
\fi


\hypersetup{
	pdftitle={Janis-Newman algorithm: generating rotating and NUT charged black holes},
	% pdfkeywords={Mots clés PDF},
	% pdfsubject={Sujet PDF}
}

\title{Janis--Newman algorithm: generating rotating and NUT charged black holes}

\author[1]{Harold Erbin\email{erbin@lpt.ens.fr}}
\affil[1]{\textsc{Cnrs}, \textsc{Lptens}, École Normale Supérieure, F-75231 Paris, France}


\begin{document}

\maketitle


\begin{abstract} 
In this review we present the most general form of the Janis--Newman algorithm.
This extension allows to generate configurations which contain all bosonic fields with spin less than or equal to two (real and complex scalar fields, gauge fields, metric field) and with five of the six parameters of the Plebański--Demiański metric (mass, electric charge, magnetic charge, NUT charge and angular momentum).
Several examples are included to illustrate the algorithm.
We also discuss the extension of the algorithm to other dimensions.
\end{abstract}


\newpage

\hrule
\pdfbookmark[1]{\contentsname}{toc}
\tableofcontents
\bigskip
\hrule

\newpage


\section{Introduction}  \label{sec:introduction}

\newcommand\inexpIntro[3]{#1?(#2,#3).}
\newcommand\rinexpIntro[3]{*#1?(#2,#3).}
\newcommand\outexpIntro[3]{#1!(#2,#3).}
\newcommand\outatomIntro[3]{#1!(#2,#3)}

We propose a fully automated method for proving termination of \(\pi\)-calculus processes.
Although there have been a lot of studies on termination analysis for the \(\pi\)-calculus
and related calculi~\cite{Deng06IC,Demangeon07,SangiorgiTermination,KobayashiHybrid,Yoshida04IC,DBLP:journals/jlp/DemangeonHS10,Venet98SAS}, most of them have been rather theoretical,
and there have been surprisingly little efforts in developing  fully automated termination
verification methods and tools based on them. To our knowledge,
Kobayashi's \typical{}~\cite{TyPiCal,KobayashiHybrid} is the only exception that
can prove termination of \(\pi\)-calculus processes (extended with natural numbers)
fully automatically, but its termination analysis is quite limited (see Section~\ref{sec:relatedwork}).

Our method is based on a reduction to termination analysis for sequential programs:
we translate a \(\pi\)-calculus process \(P\) to a sequential program \(S_P\), so that
if \(S_P\) is terminating, so is \(P\). The reduction allows us to use
powerful, mature methods and tools
for termination analysis of sequential programs~\cite{heizmann2016ultimate,freqterm,DBLP:conf/lics/PodelskiR04,Kuwahara2014Termination,DBLP:journals/cacm/CookPR11}.

The idea of the translation is to convert a chain of communications on replicated input
channels to a chain of recursive function calls of the target sequential program.
Let us consider the following Fibonacci process:
\begin{align*}
    & \rinexpIntro{\fib}{n}{r}
        \ifexp{n<2}{ \soutatom{r}{1} \\ &\quad}
                   { \nuexp{s_1} \nuexp{s_2} (\outatomIntro{\fib}{n-1}{s_1} \PAR \outatomIntro{\fib}{n-2}{s_2} \PAR \sinexp{s_1}{x}\sinexp{s_2}{y}\soutatom{r}{x+y}) \\}
    & \PAR \outatomIntro{\fib}{m}{r}
\end{align*}
Here, the process
$\rinexpIntro{\fib}{n}{r} \ldots$ is a function server that computes the \(n\)-th Fibonacci number
in parallel and returns the result to \(r\),
and $\outatom{\fib}{m}{r}$ sends a request for computing the \(m\)-th Fibonacci number;
those who are not familiar with the syntax of the \(\pi\)-calculus may wish to consult
Section~\ref{sec:targetlanguage} first.
To prove that the process above is terminating for any integer \(m\),
it suffices to show that there is no infinite chain of communications on $\fib$:
\[
    \fib(m,r) \to \fib(m_1,r_1) \to \fib(m_2,r_2) \to \cdots.
\]
We convert the process above to the following program:\footnote{The actual translation
  given later is a little more complex.}
\begin{verbatim}
 let rec fib(n) = if n<2 then () else (fib(n-1) [] fib(n-2)) in
 fib(m)
\end{verbatim}
Here, \texttt{[]} represents the non-deterministic choice.
Note that, although the calculation of Fibonacci numbers is not preserved,
for each chain of communications on \texttt{fib}, there is a corresponding
sequence of recursive calls:
\[
\mathtt{fib}(m) \to \mathtt{fib}(m_1) \to \mathtt{fib}(m_2) \to \cdots.
\]
Thus, the termination of the sequential program above implies the termination of
the original process.
As shown in the example above, (i) each communication on a replicated input channel
is converted to a function call, (ii) each communication on a non-replicated input
channel is just removed (or, in the actual translation, replaced by a call of
a trivial function defined by \(f(\seq{x})=(\,)\)), and (iii) parallel composition
is replaced by a non-deterministic choice.
We formalize the translation outlined above and prove its correctness.

The basic translation sketched above sometimes loses too much information.
For example, consider the following process:
\begin{align*}
    & \rinexpIntro{\pre}{n}{r} \soutatom{r}{n-1} \\
    & \PAR \rinexpIntro{f}{n}{r} \ifexp{n<0}{ \soutatom{r}{1} }
                                       { \nuexp{s} (\outatomIntro{\pre}{n}{s} \PAR \sinexp{s}{x}\outatomIntro{f}{x}{r}) } \\
    & \PAR \outatomIntro{f}{m}{r}
\end{align*}
The translation sketched above would yield:
\begin{verbatim}
  let pred(n) = n-1 in
  let rec f(n) = if n<0 then () else (pred(n) [] f(*)) in
  f(m)
\end{verbatim}
Here, \texttt{*} represents a non-deterministic integer: since we have removed
the input $\sinatom{s}{x}$, we do not have information about the value of \( x \).
As a result, the sequential program above is non-terminating, although the original
process is terminating.
To remedy this problem, we also refine the basic translation above by using a refinement
type system for the \(\pi\)-calculus. Using the refinement type system,
we can infer that the value of \(x\) in the original process is less than \(n\),
so that we can refine the definition of \texttt{f} to:
\begin{verbatim}
 let rec f(n) = ... else (pred(n) [] let x=* in assume(x<n);f(x))
\end{verbatim}
The target program is now terminating, from which
we can deduce that the original process is also terminating.
We have implemented an automated tool based on the refined translation above.

The contributions of this paper are summarized as follows.
\begin{itemize}
\item The formalization of the basic translation from the \(\pi\)-calculus
  (extended with integers) to sequential programs, and a proof of its correctness.
\item The formalization of a refined translation based on a refinement type system.
\item An implementation of the refined translation, including automated refinement type
  inference based on CHC solving, and experiments to evaluate the effectiveness of
  our method.
\end{itemize}

The rest of this paper is structured as follows.
Section~\ref{sec:targetlanguage} introduces the source and target languages
of our translation.
Section~\ref{sec:approach} 
formalizes the basic translation, and proves its correctness.
Section~\ref{sec:refinement} refines the basic translation by using a refinement type system.
Section~\ref{sec:implementation} reports an implementation and experiments.
Section~\ref{sec:relatedwork} discusses related work,
and Section~\ref{sec:conclusion} concludes the paper.

\begin{algorithm}[!ht]
\begin{algorithmic}[1]
\Require Query workload $Q$, event stream $I$, \app\ graph $G$, hash table of snapshots $S$
\Ensure Hash table of results $R$ 
\State $G \leftarrow \emptyset$, $S, R \leftarrow$ empty hash tables
\ForAll {event $e \in I$ with $e.type=E$} 
    \State $//$ \textbf{\app\ graph construction}
    \ForAll {$q \in Q$ \text{ with event types }T}
        \ForAll {$E' \in T,\ E' \neq E$}
            \State $G_{E'} \leftarrow \mathit{getGraphlet}(G,E')$,
            $G_{E'}.\mathit{active} \leftarrow \mathit{false}$
        \EndFor
    \EndFor
    \If {\textbf{not} $G_E.\mathit{active}$}
        \State $G_E \leftarrow \mathit{createGraphlet()}$, $G_{E}.\mathit{active} \leftarrow \mathit{true}$,
        $G \leftarrow G \cup G_E$
        \If {$G_E.\mathit{shared}$ by $Q_E \subseteq Q$}
            $x \leftarrow \mathit{createSnapshot()}$ 
            \ForAll {$q \in Q_E$}
                \ForAll{$E' \in \mathit{pt}(E,q), E' \neq E$}
                    \State $G_{E'} \leftarrow \mathit{getGraphlet}(G,E')$
                    \State $S(x,q) \leftarrow S(x,q) + sum(G_{E'},q)$ \hspace{0.5cm}$//$ Eq.~5
                \EndFor
            \EndFor
        \EndIf    
    \EndIf
    \State insert $e$ into $G_E$
    \State $//$ \textbf{Trend count computation}
    \If {$G_E.\mathit{shared}$ by $Q_E \subseteq Q$}
        \If {$\forall q \in Q_E\ pe(e,q)$ are identical}
            \State $count(e,Q_E) \leftarrow count(e,q)$ \hspace{2.3cm}$//$ Eq.~2
        \Else\ $y \leftarrow \mathit{createSnapshot()}$, $count(e,Q_E) = y$
            \ForAll {$q \in Q_E$}
                $S(y,q) \leftarrow count(e,q)$ \hspace{0.2cm}$//$ Eq.~2
            \EndFor
          \EndIf
    \Else\ $count(e,q)$ \hspace{5.2cm}$//$ Eq.~2
    \EndIf
    \ForAll{$q \in Q$}
  	    \If {$E \in \mathit{end}(q)$} 
  		    $R(q) \leftarrow R(q) + count(e,q)$ $//$ Eq.~3
        \EndIf
    \EndFor
\EndFor
\State \Return $R$
\end{algorithmic}
\caption{\app\ shared online trend aggregation}
\label{algo:snapshot-propagation}
\end{algorithm}

\section{Extension through simple examples}
\label{sec:extension}


In this section we motivate through simple examples modifications to the original prescription for the transformation of the functions.


\subsection{Magnetic charges: dyonic Kerr--Newman}
\label{sec:extension:dyonic}


The dyonic Reissner--Nordström metric is obtained from the electric one \eqref{algo:eq:rn:functions} by the replacement~\cite[sec.~6.6]{Carroll:2004:SpacetimeGeometryIntroduction}
\begin{equation}
	q^2 \longrightarrow \abs{Z}^2 = q^2 + p^2
\end{equation} 
where $Z$ corresponds to the central charge
\begin{equation}
	Z = q + i p.
\end{equation} 
Then the metric function reads
\begin{equation}
	f = 1 - \frac{2m}{r} + \frac{\abs{Z}^2}{r^2}.
\end{equation} 
The gauge field receives a new $\phi$-component
\begin{equation}
	\label{ext:eq:static:vector}
	A = f_A\, \dd t - p \cos \theta\, \dd\phi
		= f_A\, \dd u - p \cos \theta\, \dd\phi
\end{equation}
(the last equality being valid after a gauge transformation) and
\begin{equation}
	f_A = \frac{q}{r}.
\end{equation} 

The transformation of the function $f$ under \eqref{algo:eq:change:complexification-ur} is straightforward and yields
\begin{equation}
	\tilde f = 1 - \frac{2m r' - \abs{Z}^2}{\rho^2}.
\end{equation} 
On the other hand transforming directly the $r$ inside $f_A$ according to \eqref{algo:eq:rules} does not yield the correct result.
Instead one needs to first rewrite the gauge field function as
\begin{equation}
	f_A = \Re\left(\frac{Z}{r}\right)
\end{equation} 
from which the transformation proceeds to
\begin{equation}
	\tilde f_A = \frac{\Re(Z \bar r)}{\abs{r}^2}
		= \frac{q r' - p a \cos \theta}{\rho^2}.
\end{equation} 
Note that it not useful to replace $p$ by $\Im Z$ in \eqref{ext:eq:static:vector} since it is not accompanied by any $r$ dependence.
Moreover it is natural that the factor $\abs{Z}^2$ appears in the metric and this explains why the charges there do not mix with the coordinates.

The gauge field in BL coordinates is finally
\begin{subequations}
\begin{align}
	A &= \frac{q r - p a \cos \theta}{\rho^2}\, \dd t
			+ \left(- \frac{q r}{\rho^2}\, a \sin^2 \theta + \frac{p(r^2 + a^2)}{\rho^2}\, \cos\theta \right) \dd\phi \\
		&= \frac{q r}{\rho^2} (\dd t - a \sin^2 \theta \dd\phi)
			+ \frac{p \cos \theta}{\rho^2} \left(a\, \dd t + (r^2 + a^2)\, \dd\phi \right).
\end{align}
\end{subequations}
The radial component has been removed thanks to a gauge transformation since it depends only on $r$
\begin{equation}
	\Delta \times A_r = - \frac{q r - p a \cos \theta}{\rho^2}\, \rho^2 - p a \cos \theta
		= - q r.
\end{equation} 

There is a coupling between the parameters $a$ and $p$ which can be interpreted from the fact that a rotating magnetic charge has an electric quadrupole moment.
This coupling is taken into account from the product of the imaginary parts which yield a real term.
In view of the form of the algorithm such contribution could not arise from any other place.
Moreover the combination $Z = q + i p$ appears naturally in the Plebański--Demiański solution~\cite{Plebanski:1975:ClassSolutionsEinsteinMaxwell, Plebanski:1976:RotatingChargedUniformly}.

The Yang--Mills Kerr--Newman black hole found by Perry~\cite{Perry:1977:BlackHolesAre} can also be derived in this way, starting from the seed
\begin{equation}
	A^I = \frac{q^I}{r}\, \dd t + p^I \cos \theta\, \dd\phi, \qquad
	\abs{Z}^2 = q^I q^I + p^I p^I
\end{equation} 
where $q^I$ and $p^I$ are constant elements of the Lie algebra.


\subsection[NUT charge, cosmological constant and topological horizon: (anti-)de Sitter Schwarzschild--NUT]
{NUT charge and cosmological constant and topological horizon: (anti-)de Sitter Schwarzschild--NUT}
\label{sec:extension:nut}


In this subsection we consider general topological horizons
\begin{equation}
	\dd \Omega^2 = \dd\theta^2 + H(\theta)^2\, \dd \phi^2, \qquad
	H(\theta) =
	\begin{cases}
		\sin \theta & \kappa = 1 \quad (S^2), \\
		\sinh \theta & \kappa = -1 \quad (H^2).
	\end{cases}
\end{equation} 
The cosmological constant is denoted by $\Lambda$.
We give only the main formulas to motivate the modification of the algorithm, leaving the details of the transformation for \cref{sec:general}.

The complex transformation that adds a NUT charge is
\begin{subequations}
\label{ext:eq:change:jna-nut}
\begin{gather}
	u = u' - 2 \kappa \ln H(\theta), \qquad
	r = r' + i n, \\
	m = m' + i \kappa n, \qquad
	\kappa = \kappa' - \frac{4\Lambda}{3}\, n^2.
\end{gather}
\end{subequations}
Note that it is $\kappa$ and not $\kappa'$ that appears in $m$.
After having shown

The metric derived from the seed \eqref{algo:eq:static:metric:tr} is
\begin{equation}
	\dd s^2 = - \tilde f\, (\dd t - 2 \kappa n H'(\theta)\, \dd\phi)^2
		+ \tilde f^{-1}\, \dd r^2
		+ \rho^2\, \dd\Omega^2,
\end{equation}
see \eqref{gen:eq:rotating:tr-F-cst}, where
\begin{equation}
	\rho^2 = r'^2 + n^2.
\end{equation} 

The function corresponding to the (a)dS--Schwarzschild metric is
\begin{equation}
	f = \kappa - \frac{2m}{r} - \frac{\Lambda}{3}\, r^2
		= \kappa - 2 \Re\left(\frac{m}{r}\right) - \frac{\Lambda}{3}\, r^2.
\end{equation} 
The transformation is
\begin{equation}
	\tilde f = \kappa
			- \frac{2 \Re(m \bar r)}{\abs{r}^2}
			- \frac{\Lambda}{3}\, \abs{r}^2
		= \kappa' - \frac{4\Lambda}{3}\, n^2
			- \frac{2 \left[ m' r' + \left( \kappa' - \frac{4\Lambda}{3}\, n^2 \right) n^2 \right]}{\rho^2}
			- \frac{\Lambda}{3}\, \rho^2
\end{equation} 
which after simplification gives
\begin{equation}
	\label{ext:eq:nut-tilde-f}
	\tilde f = \kappa' - \frac{2 m' r' + 2 \kappa' n^2}{\rho^2}
		- \frac{\Lambda}{3} (r'^2 + 5 n^2)
		+ \frac{8\Lambda}{3}\, \frac{n^4}{\rho^2}
\end{equation} 
which corresponds correctly to the function of (a)dS--Schwarzschild--NUT~\cite{AlonsoAlberca:2000:SupersymmetryTopologicalKerrNewmannTaubNUTaDS}.

Note that it is necessary to consider the general case of massive black hole with topological horizon (if $\Lambda \neq 0$ for the latter) even if one is ultimately interested in the $m = 0$ or $\kappa = 1$ cases.

The transformation \eqref{ext:eq:change:jna-nut} can be interpreted as follows.
In similarity with the case of the magnetic charge, writing the mass as a complex parameter is needed to take into account some couplings between the parameters that would not be found otherwise.
Moreover the shift of $\kappa$ is required because the curvature of the $(\theta, \phi)$ section should be normalized to $\kappa = \pm 1$ but the coupling of the NUT charge with the cosmological constant modifies the curvature: the new shift is necessary to balance this effect and to normalize the $(\theta, \phi)$ curvature to $\kappa' = \pm 1$ in the new metric.
The NUT charge in the Plebański--Demiański solution~\cite{Plebanski:1975:ClassSolutionsEinsteinMaxwell, Plebanski:1976:RotatingChargedUniformly} is
\begin{equation}
	\ell = n \left( 1 - \frac{4\Lambda}{3}\, n^2 \right)
\end{equation} 
so the natural complex combination is $m + i \ell$ and not $m + i \kappa n$ from this point of view, and similarly for the curvature~\cite[sec.~5.3]{Griffiths:2006:NewLookPlebanskiDemianski} (such relations appear when taking limit of the Plebański--Demiański solution to recover subcases).

Finally we conclude this section with two remarks to quote different contexts where the above expression appear naturally :
\begin{itemize}
	\item Embedding Einstein--Maxwell into $N = 2$ supergravity with a negative cosmological constant $\Lambda = - 3 g^2$, the solution is BPS if~\cite{AlonsoAlberca:2000:SupersymmetryTopologicalKerrNewmannTaubNUTaDS}
	\begin{equation}
		\kappa' = -1, \qquad
		n = \pm \frac{1}{2g},
	\end{equation} 
	in which case $\kappa' = \kappa$.
	
	\item The Euclidean NUT solution is obtained from the Wick rotation
	\begin{equation}
		t = - i \tau, \qquad
		n = i \nu.
	\end{equation}
	The condition for regularity is~\cite{Chamblin:1999:LargeNPhases, Johnson:2014:ThermodynamicVolumesAdSTaubNUT}
	\begin{equation}
		m = m' - \nu \left( \kappa + \frac{4\Lambda}{3}\, \nu^2 \right)
			= 0.
	\end{equation} 
\end{itemize}


\subsection{Complex scalar fields}


For a complex scalar field, or any pair of real fields that can be naturally gathered as a complex field, one should treat the full field as a single entity instead of looking at the real and imaginary parts independently.
In particular one should not impose any reality condition.
A typical case of such system is the axion--dilaton pair
\begin{equation}
	\tau = \e^{-2\phi} + i \sigma.
\end{equation} 

In order to demonstrate this principle consider the seed (for a complete example see \cref{sec:examples:rotating-T3})
\begin{equation}
	\tau = 1 + \frac{\mu}{r}
\end{equation} 
where only the dilaton is non-zero.
Then the transformation \eqref{algo:eq:change:complexification-ur} gives
\begin{equation}
	\tau' = 1 + \frac{\mu}{r}
		= 1 + \frac{\mu}{r' - i a \cos\theta}
		= 1 + \frac{\mu r'}{\rho^2} + i\, \frac{\mu a \cos\theta}{\rho^2}.
\end{equation} 
The transformation generates an imaginary part which cannot be obtained if $\Im \tau$ is treated separately: the algorithm does not change fields that vanish except if they are components of a larger field.
Note that both $\tau$ and $\tau'$ are harmonic functions.

\section{Complete algorithm}
\label{sec:general}


In this section we gather all the facts on the Janis--Newman algorithm and we explain how to apply it to a general setting.
We write the formulas corresponding to the most general configurations that can be obtained.
We insist again on the fact that all these results can also be derived from the tetrad formalism.


\subsection{Seed configuration}
\label{sec:general:seed}


We consider a general configuration with a metric $g_{\mu\nu}$, gauge fields $A_\mu^I$, complex scalar fields $\tau^i$ and real scalar fields $q^u$.
The initial parameters of the seed configuration are the mass $m$, electric charges $q^I$, magnetic charges $p^i$ and some other parameters $\lambda^A$ (such as the scalar charges).
The electric and magnetic charges are grouped in complex parameters
\begin{equation}
	Z^I = q^I + i p^I.
\end{equation} 
All indices run over some arbitrary ranges.

The seed configuration is spherically symmetric and in particular all the fields depend only on the radial direction $r$
\begin{subequations}
\label{gen:eq:static:tr}
\begin{gather}
	\label{gen:eq:static:metric:tr}
	\dd s^2 = - f_t(r)\, \dd t^2 + f_r(r)\, \dd r^2 + f_\Omega(r)\, \dd\Omega^2, \\
	A^I = f^I(r)\, \dd t + p^I H'(\theta)\, \dd\phi, \\
	\tau^i = \tau^i(r), \qquad
	q^u = q^u(r)
\end{gather}
\end{subequations}
where
\begin{equation}
	\dd \Omega^2 = \dd\theta^2 + H(\theta)^2\, \dd \phi^2, \qquad
	H(\theta) =
	\begin{cases}
		\sin \theta & \kappa = 1 \quad (S^2), \\
		\sinh \theta & \kappa = -1 \quad (H^2).
	\end{cases}
\end{equation} 
Note that only two functions in the metric are relevant since the last one can be fixed through a diffeomorphism.
All the real functions are denoted collectively by
\begin{equation}
	f_i = \{ f_t, f_r, f_\Omega, f^I, q^u \}.
\end{equation} 

The transformation to null coordinates is
\begin{equation}
	\label{gen:eq:change:null}
	\dd t = \dd u - \sqrt{\frac{f_r}{f_t}}\, \dd r
\end{equation} 
and yields
\begin{subequations}
\label{gen:eq:static:ur}
\begin{gather}
	\label{gen:eq:static:metric:ur}
	\dd s^2 = - f_t\, \dd u^2 - 2 \sqrt{f_t f_r}\, \dd r^2 + f_\Omega\, \dd\Omega^2, \\
	A^I = f^I\, \dd u + p^I H'\, \dd\phi
\end{gather}
\end{subequations}
where the radial component of the gauge field
\begin{equation}
	A^I_r = f^I \sqrt{\frac{f_r}{f_t}}
\end{equation} 
has been set to zero through a gauge transformation.


\subsection{Janis--Newman algorithm}
\label{sec:general:jna}


\subsubsection{Complex transformation}


One performs the complex change of coordinates
\begin{equation}
	\label{gen:eq:change:jna}
	r = r' + i\, F(\theta), \qquad
	u = u' + i\, G(\theta).
\end{equation}
In the case of topological horizons the Giampieri ansatz \eqref{algo:eq:giampieri-ansatz} generalizes to
\begin{equation}
	\label{gen:eq:giampieri-ansatz}
	i\, \dd \theta = H(\theta)\, \dd \phi
\end{equation} 
leading to the differentials
\begin{equation}
	\dd r = \dd r' + F'(\theta) H(\theta)\, \dd \phi, \qquad
	\dd u = \dd u' + G'(\theta) H(\theta)\, \dd \phi.
\end{equation} 
The ansatz \eqref{gen:eq:giampieri-ansatz} is a direct consequence of the fact that the $2$-dimensional slice $(\theta, \phi)$ is given by $\dd \Omega^2 = \dd\theta^2 + H(\theta)^2\, \dd \phi^2$, such that the function in the RHS of \eqref{gen:eq:giampieri-ansatz} corresponds to $\sqrt{g^\Omega_{\phi\phi}}$ (where $g$ is the static metric), as can be seen by doing the computation with $i\, \dd \theta = \mc H(\theta) \dd\phi$ and identifying $\mc H = H$ at the end.

The most general known transformation is
\begin{subequations}
\begin{gather}
	\label{gen:eq:change:jna-functions-FG}
	F(\theta) = n - a\, H'(\theta) + c \left( 1 + H'(\theta)\, \ln \frac{H(\theta/2)}{H'(\theta/2)} \right), \\
	G(\theta) = \kappa a\, H'(\theta)
		- 2 \kappa n \ln H(\theta)
		- \kappa c\, H'(\theta)\, \ln \frac{H(\theta/2)}{H'(\theta/2)}, \\
	m = m' + i \kappa n, \\
	\kappa = \kappa' - \frac{4\Lambda}{3}\, n^2,
\end{gather}
\end{subequations}
where $a, c \neq 0$ only if $\Lambda = 0$ (see \cref{sec:derivation} for the derivation).
The mass that is transformed is the physical mass: even if it written in terms of other parameters one should identify it and transform it.

The parameters $a$ and $n$ correspond respectively to the angular momentum and to the NUT charge.
On the other hand the constant $c$ did not receive any clear interpretation (see for example~\cites{Demianski:1972:NewKerrlikeSpacetime, Adamo:2014:KerrNewmanMetricReview}[sec.~5.3]{Krasinski:2006:InhomogeneousCosmologicalModels}).
It can be noted that the solution is of type II in Petrov classification (and thus the JN algorithm \emph{can} change the Petrov type) and it corresponds to a wire singularity on the rotation axis.
Moreover the BL transformation is not well-defined.


\subsubsection{Function transformation}
\label{sec:general:jna:functions}


All the real functions $f_i = f_i(r)$ must be modified to be kept real once $r \in \C$
\begin{equation}
	\label{gen:eq:complexification-functions}
	\tilde f_i = \tilde f_i(r, \bar r)
		= \tilde f_i\big(r', F(\theta) \big) \in \R.
\end{equation} 
The last equality means that $\tilde f_i$ can depend on $\theta$ only through $\Im r = F(\theta)$.
The condition that one recovers the seed for $\bar r = r = r'$ is
\begin{equation}
	\tilde f_i(r', 0) = f_i(r').
\end{equation} 

If all magnetic charges are vanishing or in terms without electromagnetic charges the rules for finding the $\tilde f_i$ are
\begin{subequations}
\label{gen:eq:rules}
\begin{align}
	\label{gen:eq:rules:r}
	r & \longrightarrow \frac{1}{2} (r + \bar r) = \Re r, \\
	\label{gen:eq:rules:1/r}
	\frac{1}{r} & \longrightarrow \frac{1}{2} \left(\frac{1}{r} + \frac{1}{\bar r}\right) = \frac{\Re r}{\abs{r}^2}, \\
	\label{gen:eq:rules:r2}
	r^2 & \longrightarrow \abs{r}^2.
\end{align}
\end{subequations}
Up to quadratic powers of $r$ and $r^{-1}$ these rules determine almost uniquely the result.
This is not anymore the case when the configurations involve higher power.
These can be dealt with by splitting it in lower powers: generically one should try to factorize the expression into at most quadratic pieces.
Some examples of this with natural guesses are
\begin{equation}
	r^4 - b^2 = (r^2 + b) (r^2 - b), \qquad
	r^4 + b = r^2 \left( r^2 + \frac{b}{r^2} \right).
\end{equation} 
Moreover the same power of $r$ can be transformed differently, for example
\begin{equation}
	\frac{1}{r^n} \longrightarrow \frac{1}{r^{n-2}}\, \frac{1}{\abs{r}^2}.
\end{equation} 

Denoting by $Q(r)$ and $P(r)$ collectively all functions that multiply $q^I$ and $p^I$ respectively, all such terms should be rewritten as
\begin{equation}
	\Big( q^I Q(r), p^I P(r) \Big) = \Big( \Re\big(Z^I Q(r)\big), \Im\big(Z^I P(r)\big) \Big)
\end{equation} 
before performing the transformation \eqref{gen:eq:change:jna}.
Note that in this case one does not use the rules \eqref{gen:eq:rules}.

Finally the transformed complex scalars are obtained by simply plugging \eqref{gen:eq:change:jna}
\begin{equation}
	\tau'^i(r', \theta) = \tau^i\big(r + i F(\theta)\big).
\end{equation} 


\subsubsection{Null coordinates}


Plugging the transformation \eqref{gen:eq:change:jna} inside the seed metric and gauge fields \eqref{gen:eq:static:ur} leads to\footnotemark{}%
\footnotetext{%
	We stress that at this stage these formula do not satisfy Einstein equations, they are just proxies to simplify later computations.
}
\begin{subequations}
\label{gen:eq:rotating:ur}
\begin{gather}
	\dd s^2 = - \tilde f_t\, (\dd u' + \alpha\, \dd r' + \omega H\, \dd\phi )^2
		+ 2 \beta\, \dd r' \dd \phi
		+ \tilde f_\Omega\, \big(\dd\theta^2 + \sigma^2 H^2\, \dd\phi^2 \big), \\
	A^I = \tilde f^I\, (\dd u' + G' H\, \dd \phi) + p^I H'\, \dd\phi
\end{gather}
\end{subequations}
where (one should not confuse the primes to indicate derivatives from the primes on the coordinates)
\begin{equation}
	\omega = G' + \sqrt{\frac{\tilde f_r}{\tilde f_t}}\, F', \qquad
	\sigma^2 = 1 + \frac{\tilde f_r}{\tilde f_\Omega}\, F'^2, \qquad
	\alpha = \sqrt{\frac{\tilde f_r}{\tilde f_t}}, \qquad
	\beta = \tilde f_r\, F' H.
\end{equation} 


\subsubsection{Boyer--Lindquist coordinates}


The Boyer--Lindquist transformation
\begin{equation}
	\label{gen:eq:change:bl}
	\dd u' = \dd t' - g(r') \dd r', \qquad
	\dd \phi = \dd \phi' - h(r') \dd r',
\end{equation} 
can be used to remove the off-diagonal $tr$ and $r\phi$ components of the metric
\begin{equation}
	g_{t'r'} = g_{r'\phi'} = 0.
\end{equation} 
The solution to these equations is
\begin{equation}
	\label{gen:eq:change:bl:solution-gh}
	g(r') = \frac{\sqrt{\big(\tilde f_t \tilde f_r \big)^{-1}}\, \tilde f_\Omega - F' G'}{\Delta}, \qquad
	h(r') = \frac{F'}{H \Delta}
\end{equation} 
where
\begin{equation}
	\label{gen:eq:change:bl:delta}
	\Delta = \frac{\tilde f_\Omega}{\tilde f_r}\, \sigma^2
		= \frac{\tilde f_\Omega}{\tilde f_r} + F'^2.
\end{equation} 
Remember that the changes of coordinate is valid only if $g$ and $h$ are functions of $r'$ only.

Inserting \eqref{gen:eq:change:bl:solution-gh} into \eqref{gen:eq:rotating:ur} yields
\begin{subequations}
\label{gen:eq:rotating:tr}
\begin{gather}
	\dd s^2 = - \tilde f_t\, (\dd t' + \omega H\, \dd\phi' )^2
		+ \frac{\tilde f_\Omega}{\Delta}\, \dd r'^2
		+ \tilde f_\Omega\, \big(\dd\theta^2 + \sigma^2 H^2\, \dd\phi'^2 \big), \\
	A^I = \tilde f^I\, \left(\dd t' - \frac{\tilde f_\Omega}{\Delta \sqrt{\tilde f_t \tilde f_r}}\, \dd r' + G' H\, \dd \phi' \right) + p^I H'\, \dd\phi'
\end{gather}
\end{subequations}
where we recall that
\begin{equation}
	\omega = G' + \sqrt{\frac{\tilde f_r}{\tilde f_t}}\, F', \qquad
	\sigma^2 = 1 + \frac{\tilde f_r}{\tilde f_\Omega}\, F'^2.
\end{equation} 
Generically one finds $A_r = A_r(r)$ which can be set to zero thanks to a gauge transformation.

Before closing this section we simplify the above formulas for few simple cases that are often used.


\paragraph{Degenerate Schwarzschild seed}

A degenerate seed (one unknown function) in Schwarzschild coordinates has
\begin{equation}
	f_r = f_t^{-1}, \qquad
	f_\Omega = r^2.
\end{equation} 
The above formulas for this case can be found in \cref{sec:derivation:ansatz}.


\paragraph{Degenerate isotropic seed}

A degenerate seed in isotropic coordinates has
\begin{equation}
	f_t = f^{-1}, \qquad
	f_r = f, \qquad
	f_\Omega = r^2 f.
\end{equation} 
In this case the above formulas reduced to
\begin{subequations}
\label{gen:eq:rotating:tr-degenerate-isotropic}
\begin{gather}
	\dd s^2 = - \tilde f^{-1}\, (\dd t + \omega H\, \dd\phi )^2
		+ \tilde f \rho^2 \left( \frac{\dd r^2}{\Delta}
			+ \dd\theta^2 + \sigma^2 H^2\, \dd\phi^2 \right), \\
	A^I = \tilde f^I\, \left(\dd t - \frac{\tilde f \rho^2}{\Delta}\, \dd r + G' H\, \dd \phi \right) + p^I H'\, \dd\phi
\end{gather}
\end{subequations}
where we recall that
\begin{equation}
	\omega = G' + \tilde f\, F', \qquad
	\sigma^2 = 1 + \frac{F'^2}{\rho^2}, \qquad
	\Delta = \tilde f \rho^2 + F'^2.
\end{equation} 

\paragraph{Constant $F$}

The expressions simplify greatly if $F' = 0$ (for example when $\Lambda \neq 0$).
First all functions depend only on $r$ since $F(\theta) = \cst$
\begin{equation}
	\tilde f_i(r, \theta) = \tilde f_i(r, 0).
\end{equation} 
As a consequence the Boyer--Lindquist transformation \eqref{gen:eq:change:bl:solution-gh}
\begin{equation}
	g(r') = \sqrt{\frac{\tilde f_r}{\tilde f_t}}, \qquad
	h(r') = 0
\end{equation} 
is always well-defined.
For the same reason it is always possible to perform a gauge transformation.
Finally the metric and gauge fields \eqref{gen:eq:rotating:tr} becomes
\begin{subequations}
\label{gen:eq:rotating:tr-F-cst}
\begin{gather}
	\dd s^2 = - \tilde f_t \big(\dd t + G' H\, \dd\phi \big)^2
		+ \tilde f_r\, \dd r^2
		+ \tilde f_\Omega\, \dd\Omega^2, \\
	A^I = \tilde f^I\, \left(\dd t' + G' H\, \dd \phi' \right) + p^I H'\, \dd\phi'.
\end{gather}
\end{subequations}


\subsection{Open questions}


The algorithm we have described help to work with five (four if $\Lambda \neq 0$) of the six parameters of the Plebański--Demiański (PD) solution.
It is tempting to conjecture that it can be extended to the full set of parameters by generalizing the ideas described in \cref{sec:extension:nut} (shifting $\kappa$, writing $a + i \alpha$…).
Indeed we have found that these operations were quite natural in the context of the  PD solution and inspiration could be found in~\cite{Griffiths:2006:NewLookPlebanskiDemianski}.

\section{Derivation of the transformations}
\label{sec:derivation}


The goal of this section is to derive the form \eqref{gen:eq:change:jna-functions-FG} of the possible complex transformations.
This method was first used by Demiański~\cite{Demianski:1972:NewKerrlikeSpacetime} and then generalized in~\cite{Erbin:2016:DecipheringGeneralizingDemianskiJanisNewman}.
The idea is to perform the algorithm in a simple setting (metric with one unknown function and one gauge field), leaving arbitrary the functions $F(\theta)$ and $G(\theta)$ in \eqref{gen:eq:change:jna} and $\tilde f_i$ before solving the equations of motion to determine them.
Then the result can be reinterpreted in terms of rules to get the functions $\tilde f_i$ from $f_i$ (this last part was not discussed in~\cite{Demianski:1972:NewKerrlikeSpacetime}).
This selects the possible complex transformations.
Then one can hope that these transformations will be the most general ones (under the assumptions that are made), and one can use these transformations in other cases without having to solve the equations.
The latter claim can be justified by looking at the equations of motions for more complex examples: even if one cannot find directly a solution, one finds that the same structure persists~\cite{Erbin:2016:DecipheringGeneralizingDemianskiJanisNewman} (this is also motivated by the solutions in~\cite{Krori:1981:ChargedDemianskiMetric, Patel:1988:RadiatingDemianskitypeMetrics}).
Another strength of this approach is to remove the ambiguity of the algorithm since the functions are found from the equations of motion, and this may help when one does not know how to perform precisely the algorithm (for example in higher dimensions, see \cref{sec:higher}).


Another goal of this section is to expose the full technical details of the computations: Demiański's paper~\cite{Demianski:1972:NewKerrlikeSpacetime} is short and results are extremely condensed.
In particular we uncover an underlying assumption on the form of the metric function and we show how this lead to an error an in his formula (14) (already pointed out in~\cite{Quevedo:1992:ComplexTransformationsCurvature}).
A generalization of this hypothesis leads to other equations that we could not solve analytically and which may lead to other complex transformations.

Finally this analysis shows the impossibility to derive the (a)dS--Kerr(--Newman) solutions from the JN algorithm.
As discussed in the previous section generalization of the ansatz may help to avoid this no-go theorem.


\subsection{Setting up the ansatz}
\label{sec:derivation:ansatz}


We first recall the action and equations of motion before describing the ansatz for the metric and gauge fields.
We refer to \cref{sec:general} for the general formulas from which the expressions in this section are derived.


\subsubsection{Action and equations of motion}


The action for Einstein--Maxwell gravity with cosmological constant $\Lambda$ reads
\begin{equation}
	\label{deriv:eq:einstein-maxwell-action}
	S = \int \dd^4 x\; \sqrt{- g} \left( \frac{1}{2 \varkappa^2} (R - 2 \Lambda) - \frac{1}{4}\, F^2 \right),
\end{equation} 
where $\varkappa^2 = 8 \pi G$ is the Einstein coupling constant, $g_{\mu\nu}$ is the metric with Ricci scalar $R$ and $F = \dd A$ is the field strength of the Maxwell field $A_\mu$.
In the rest of this section we will set $\varkappa = 1$.
The corresponding equations of motion (respectively Einstein and Maxwell) are
\begin{equation}
	\label{deriv:eq:einstein-maxwell-eom}
	G_{\mu\nu} + \Lambda g_{\mu\nu} = 2\, T_{\mu\nu}, \qquad
	\grad_\mu F^{\mu\nu} = 0,
\end{equation} 
where energy--momentum tensor for the electromagnetic gauge field $A_\mu$ is
\begin{equation}
	T_{\mu\nu} = F_{\mu\rho} \tens{F}{_\nu^\rho} - \frac{1}{4}\, g_{\mu\nu} F^2.
\end{equation} 


\subsubsection{Seed configuration}


We are interested in the subcase of \eqref{gen:eq:static:metric:tr} where
\begin{equation}
	\label{eq:static-ansatz-one-unknown}
	f_t = f, \qquad
	f_r = f^{-1}, \qquad
	f_\Omega = r^2.
\end{equation} 

The seed configuration is
\begin{subequations}
\label{deriv:eq:static:tr}
\begin{gather}
	\label{deriv:eq:static:metric:tr}
	\dd s^2 = - f(r)\, \dd t^2 + f(r)^{-1}\, \dd r^2 + r^2\, \dd\Omega^2, \\
	\label{deriv:eq:static:vector:tr}
	A = f_A(r)\, \dd t
\end{gather}
\end{subequations}
where we consider spherical and hyperbolic horizons
\begin{equation}
	\dd \Omega^2 = \dd\theta^2 + H(\theta)^2\, \dd \phi^2, \qquad
	H(\theta) =
	\begin{cases}
		\sin \theta & \kappa = 1, \\
		\sinh \theta & \kappa = -1.
	\end{cases}
\end{equation} 
In terms of null coordinates \eqref{gen:eq:change:null} the configuration reads
\begin{subequations}
\label{deriv:eq:static:ur}
\begin{gather}
	\label{deriv:eq:static:metric:ur}
	\dd s^2 = - f\, \dd u^2 - 2\, \dd u \dd r + r^2\, \dd\Omega^2, \\
	\label{deriv:eq:static:vector:ur}
	A = f_A\, \dd u.
\end{gather}
\end{subequations}


\subsubsection{Janis--Newman configuration}


The configuration obtained from the Janis--Newman algorithm with a general transformation \eqref{gen:eq:change:jna}
\begin{equation}
	r = r' + i\, F(\theta), \qquad
	u = u' + i\, G(\theta)
\end{equation}
corresponds to (we omit the primes on the coordinates)
\begin{subequations}
\label{deriv:eq:rotating:ur}
\begin{gather}
	\dd s^2 = - \tilde f\, (\dd u + \alpha\, \dd r + \omega H\, \dd\phi )^2
		+ 2 \beta\, \dd r \dd \phi
		+ \rho^2\, \big(\dd\theta^2 + \sigma^2 H^2\, \dd\phi^2 \big), \\
	A = \tilde f_A\, (\dd u + G' H\, \dd \phi)
\end{gather}
\end{subequations}
where
\begin{equation}
	\rho^2 = r^2 + F^2, \quad
	\omega = G' + \tilde f^{-1}\, F', \quad
	\sigma^2 = 1 + \frac{F'^2}{\tilde f \rho^2}, \quad
	\alpha = \tilde f^{-1}, \quad
	\beta = \tilde f^{-1}\, F' H.
\end{equation} 

The Boyer--Lindquist transformation \eqref{gen:eq:change:bl}
\begin{equation}
	\dd u = \dd t' - g(r) \dd r, \qquad
	\dd \phi = \dd \phi' - h(r) \dd r
\end{equation} 
with functions
\begin{equation}
	g(r) = \frac{\rho^2 - F' G'}{\Delta}, \qquad
	h(r) = \frac{F'}{H \Delta}, \qquad
	\Delta = \tilde f \rho^2\, \sigma^2
\end{equation} 
leads to (omitting the primes on the coordinates)
\begin{subequations}
\label{deriv:eq:rotating:tr}
\begin{gather}
	\dd s^2 = - \tilde f_t\, (\dd t + \omega H\, \dd\phi )^2
		+ \frac{\rho^2}{\Delta}\, \dd r^2
		+ \rho^2\, \big(\dd\theta^2 + \sigma^2 H^2\, \dd\phi^2 \big), \\
	A = \tilde f_A\, \left(\dd t - \frac{\rho^2}{\Delta}\, \dd r + G' H\, \dd \phi \right).
\end{gather}
\end{subequations}


\subsection{Static solution}


It is straightforward to solve the equations \eqref{deriv:eq:einstein-maxwell-eom} for the static configuration \eqref{deriv:eq:static:tr}.

Only the $(t)$ component of Maxwell equations is non trivial
\begin{equation}
	2 f'_A + r f''_A = 0,
\end{equation} 
the prime being a derivative with respect to $r$, and its solution is
\begin{equation}
	f_A(r) = \alpha + \frac{q}{r}
\end{equation} 
where $q$ is a constant of integration that is interpreted as the charge and $\alpha$ is an additional constant that can be removed by a gauge transformation.

The only relevant Einstein equation is
\begin{equation}
	\frac{q^2}{r^2} - \kappa + r^2 \Lambda + f + r f' = 0
\end{equation} 
whose solution reads
\begin{equation}
	\label{eq:topdown-1:static-f}
	f(r) = \kappa - \frac{2m}{r} + \frac{q^2}{r^2} - \frac{\Lambda}{3}\, r^2,
\end{equation} 
$m$ being a constant of integration that is identified to the mass.

We stress that we are just looking for solutions of Einstein equations and we are not concerned with regularity (in particular it is well-known that only $\kappa = 1$ is well-defined for $\Lambda = 0$).

The solution we will find in the next section should reduce to this one upon setting $F, G = 0$.


\subsection{Stationary solution}


Since Boyer--Lindquist imposes additional restrictions on the solutions we will solve the equations of motion \eqref{deriv:eq:einstein-maxwell-eom} for the configuration in null coordinates \eqref{deriv:eq:rotating:ur}.


\subsubsection{Simplifying the equations}
\label{sec:derivation:stationary:simplifying}


The components $(rr)$ and $(r\theta)$ give respectively the equation
\begin{subequations}
\begin{align}
	G'' + \frac{H'}{H}\, G' &= \pm 2 F, \\
	F' \left( G'' + \frac{H'}{H}\, G' \right) &= 2 F F'.
\end{align}
\end{subequations}
If $F' = 0$ then $F$ is an arbitrary constant and the sign of the first equation can be absorbed into its definition.\footnotemark{}%
\footnotetext{%
	In particular all expressions are quadratic in $F$, but only linear in $F'$.
}
On the other hand if $F' \neq 0$ one can simplify by the latter in the second equation and this fixes the sign of the first equation.
Then in both cases the relevant equation reduces to
\begin{equation}
	\label{eq:topdown-1-F-Gd-bis}
	G'' + \frac{H'}{H}\, G' = 2 F,
\end{equation} 
which depends only on $\theta$ and allows to solve for $G$ in terms of $F$.

Integrating the $r$-component of the Maxwell equation gives
\begin{equation}
	\tilde f_A = \frac{q\, r}{r^2 + F^2} + \alpha\, \frac{r^2 - F^2}{r^2 + F^2}.
\end{equation}
The $\theta$-equation reads
\begin{equation}
	\alpha\, F' = 0
\end{equation}
which implies $\alpha = 0$ if $F' \neq 0$.
The $\phi$- and $t$-equations follow from these two equations.
As seen above, $\alpha$ can be removed in the static limit $F \to 0$ and in the rest of this section we consider only the case\footnotemark{}%
\footnotetext{%
	We relax this assumption in \cref{sec:derivation:relaxing:gauge-fields}.
}
\begin{equation}
	\alpha = 0.
\end{equation} 

The $(tr)$ equation contains only $r$-derivatives of $\tilde f$ and can be integrated, giving\footnotemark{}%
\footnotetext{%
	In~\cite{Demianski:1972:NewKerrlikeSpacetime} the last term of $\tilde f$ is missing as pointed out in~\cite{Quevedo:1992:ComplexTransformationsCurvature}.
}
\begin{equation}
	\tilde f = \kappa - \frac{2m r - q^2 + 2 F (\kappa\, F + K)}{r^2 + F^2} - \frac{\Lambda}{3}\, (r^2 + F^2) - \frac{4 \Lambda}{3}\, F^2 + \frac{8 \Lambda}{3}\, \frac{F^4}{r^2 + F^2}
\end{equation} 
where again $m$ is a constant of integration interpreted as the mass and the function $K$ is defined by
\begin{equation}
	2 K = F'' + \frac{H'}{H}\, F'.
\end{equation} 
This implies the equations $(r\phi)$ and $(\theta\theta)$.

As explained below \eqref{gen:eq:complexification-functions} the $\theta$-dependence should be contain in $F(\theta)$ only.
The second term of the function $\tilde f$ contains some lonely $\theta$ from the $H(\theta)$ in the function $K$: this means that they should be compensated by the $F$, and we therefore ask that the sum $\kappa F + K$ be constant\footnotemark{}%
\footnotetext{%
	In \cref{sec:derivation:relaxing:metric-function} we relax this last assumption by allowing non-constant $\kappa F + K$.
	In this context the equations and the function $\tilde f$ are modified and this provides an explanation for the Demiański's error in $\tilde f$ in~\cite{Demianski:1972:NewKerrlikeSpacetime}.
}
\begin{equation}
	\kappa\, F' + K' = 0
	\quad \Longrightarrow \quad
	\kappa\, F + K = \kappa n.
\end{equation} 
The parameter $n$ is interpreted as the NUT charge.

The components $(t\theta)$ and $(\theta\phi)$ give the same equation
\begin{equation}
	\Lambda\, F' = 0.
\end{equation} 

Finally one can check that the last three equations $(tt), (t\phi)$ and $(\phi\phi)$ are satisfied.


\subsubsection{Summary of the equations}


The equations to be solved are
\begin{subequations}
\label{eq:topdown-1}
\begin{align}
	\label{eq:topdown-1-F-Gd}
	2 F &= G'' + \frac{H'}{H}\; G', \\
	\label{eq:topdown-1-Fd-Kd}
	\kappa\, n &= \kappa\, F + K, \\
	\label{eq:topdown-1-lambda}
	0 &= \Lambda F'
\end{align}
and the function $\tilde f$ is
\begin{equation}
	\label{eq:topdown-1-tilde-f}
	\tilde f = \kappa - \frac{2m r - q^2 + 2 F (\kappa\, F + K)}{r^2 + F^2} - \frac{\Lambda}{3}\, (r^2 + F^2) - \frac{4 \Lambda}{3}\, F^2 + \frac{8 \Lambda}{3}\, \frac{F^4}{r^2 + F^2}.
\end{equation}
We also defined
\begin{equation}
	\label{eq:topdown-1-K-Fd}
	2 K = F'' + \frac{H'}{H}\, F'.
\end{equation} 
\end{subequations}

As explained in the introduction the second step will be to explain \eqref{eq:topdown-1-tilde-f} in terms of new rules for the algorithm: they have been found in~\cite{Erbin:2016:DecipheringGeneralizingDemianskiJanisNewman} and this was the topic of \cref{sec:general:jna}.

In the next subsections we solve explicitly the equations \eqref{eq:topdown-1} in both cases $\Lambda \neq 0$ and $\Lambda = 0$.


\subsubsection{Solution for \texorpdfstring{$\Lambda \neq 0$}{non-vanishing cosmological constant}}


Equation \eqref{eq:topdown-1-lambda} implies that $F' = 0$, from which $K = 0$ follows by definition; then one obtains
\begin{equation}
	F(\theta) = n
\end{equation} 
by compatibility with \eqref{eq:topdown-1-Fd-Kd} and since $K(\theta) = 0$.

Solution to \eqref{eq:topdown-1-F-Gd} is
\begin{equation}
	G(\theta) = c_1 - 2 \kappa\, n \ln H(\theta) + c_2 \ln \frac{H(\theta/2)}{H'(\theta/2)}
\end{equation} 
where $c_1$ and $c_2$ are two constants of integration.
Since only $G'$ appears in the metric we can set $c_1 = 0$.
On the other hand the constant $c_2$ can be removed by the transformation
\begin{equation}
	\dd u = \dd u' - c_2\, \dd\phi
\end{equation} 
since one has
\begin{equation}
	\left( \ln \frac{H(\theta/2)}{H'(\theta/2)} \right)' = \frac{1}{H(\theta)}.
\end{equation} 

The solution to the system \eqref{eq:topdown-1} is thus
\begin{equation}
	F(\theta) = n, \qquad
	G(\theta) = - 2 \kappa\, n \ln H(\theta).
\end{equation} 
The function $\tilde f$ then takes the form
\begin{equation}
	\label{eq:topdown-1:tilde-f-lambda}
	\tilde f = \kappa - \frac{2m r - q^2 + 2 \kappa n^2}{r^2 + n^2} - \frac{\Lambda}{3}\,\frac{r^4 + 6 n^2 r^2 - 3 n^4}{r^2 + n^2}.
	% \kappa - \frac{2m r - q^2 + 2 \kappa n^2}{r^2 + n^2} - \frac{\Lambda}{3} (r^2 + 5 n^2) + \frac{8 \Lambda}{3}\, \frac{n^4}{r^2 + n^2}
\end{equation} 
This corresponds to the (a)dS--Schwarzschild--NUT solution: compare with \eqref{ext:eq:nut-tilde-f} and \eqref{gen:eq:rotating:tr-F-cst}.

The parameter $\Delta$ in the BL transformation \eqref{gen:eq:change:bl:delta} is
\begin{equation}
	\Delta = \kappa r^2 - 2 m r + q^2 + \Lambda n^4 - \frac{\Lambda}{3}\, r^4 - n^2 (\kappa + 2 \Lambda r^2 ).
\end{equation} 

As noted by Demiański the only parameters that appear are the mass and the NUT charge, and it is not possible to add angular momentum for non-vanishing cosmological constant.\footnotemark{}%
\footnotetext{%
	In~\cite{Leigh:2014:GerochGroupEinstein} Leigh et al.\ generalized Geroch's solution generating technique and also found that only the mass and the NUT charge appear when $\Lambda \neq 0$. We would like to thank D.\ Klemm for this remark.
}
As a consequence the JN algorithm cannot provide a derivation of the (a)dS--Kerr--Newman solution.


\subsubsection{Solution for \texorpdfstring{$\Lambda = 0$}{vanishing cosmological constant}}
\label{sec:derivation:stationary:solution-no-cosmo}


The solution to the differential equation \eqref{eq:topdown-1-Fd-Kd} is
\begin{equation}
	F(\theta) = n - a\, H'(\theta) + c \left( 1 + H'(\theta)\, \ln \frac{H(\theta/2)}{H'(\theta/2)} \right)
\end{equation}
where $a$ and $c$ denote two constants of integration.

We solve the equation \eqref{eq:topdown-1-F-Gd} for $G$
\begin{equation}
	\begin{aligned}
		G(\theta) = c_1 &+ \kappa\, a\, H'(\theta)
			- \kappa\, c\, H'(\theta)\, \ln \frac{H(\theta/2)}{H'(\theta/2)}
			- 2 \kappa\, n \ln H(\theta) \\
			&+ (a + c_2) \ln \frac{H(\theta/2)}{H'(\theta/2)}
	\end{aligned}
\end{equation} 
and $c_1, c_2$ are constants of integration.
Again since only $G'$ appears in the metric we can set $c_1 = 0$.
We can also remove the last term with the transformation
\begin{equation}
	\dd u = \dd u' - (c_2 + a) \dd\phi.
\end{equation} 
One finally gets
\begin{subequations}
\begin{align}
	F(\theta) &= n - a\, H'(\theta) + c \left( 1 + H'(\theta)\, \ln \frac{H(\theta/2)}{H'(\theta/2)} \right), \\
	G(\theta) &= \kappa\, a\, H'(\theta)
		- \kappa\, c\, H'(\theta)\, \ln \frac{H(\theta/2)}{H'(\theta/2)}
		- 2 \kappa\, n \ln H(\theta).
\end{align}
\end{subequations}

This solution was already found in~\cite{Krori:1981:ChargedDemianskiMetric} for the case $\kappa = 1$ by solving directly Einstein--Maxwell equations, starting with a metric ansatz of the form \eqref{deriv:eq:rotating:ur}.
Our aim was to show that the same solution can be obtained by applying Demiański's method to all the quantities, including the gauge field.

The BL transformation is well defined only for $c = 0$, in which case
\begin{equation}
	g = \frac{r^2 + a^2 + n^2}{\Delta}, \qquad
	h = \frac{\kappa a}{\Delta}, \qquad
	\Delta = \kappa r^2 - 2 m r + q^2 - \kappa n^2 + \kappa a^2.
\end{equation} 
The function $\tilde f$ reads
\begin{equation}
	\label{eq:topdown-1:tilde-f-no-Lambda-no-c}
	\tilde f = \kappa - \frac{2 m r - q^2}{\rho^2} + \frac{\kappa\, n (n - a H')}{\rho^2}, \qquad
	\rho^2 = r^2 + (n - a\, H')^2
\end{equation} 
and this corresponds to the Kerr--Newman--NUT solution~\cite[sec.~2.2]{AlonsoAlberca:2000:SupersymmetryTopologicalKerrNewmannTaubNUTaDS}.


\subsection{Relaxing assumptions}
\label{sec:derivation:relaxing}


In the derivation of \cref{sec:derivation:stationary:simplifying} we have made two assumptions in order to recover the simplest case.
The goal of this section is to show how these assumptions can be lifted, even if this does not lead to useful results: one cannot solve the equations in one case while in the other it is not clear how to recast the result in terms of a complex transformation.


\subsubsection{Metric function \texorpdfstring{$F$}{F}-dependence}
\label{sec:derivation:relaxing:metric-function}


In \cref{sec:derivation:stationary:simplifying} we obtained the equation \eqref{eq:topdown-1-Fd-Kd}
\begin{equation}
	\kappa\, F + K = \kappa\, n, \qquad
	2 K = F'' + \frac{H'}{H}\, F'
\end{equation}
by requiring that the function \eqref{eq:topdown-1-tilde-f}
\begin{equation}
	\tilde f = \kappa - \frac{2m r - q^2 + 2 F (\kappa\, F + K)}{r^2 + F^2} - \frac{\Lambda}{3}\, (r^2 + F^2) - \frac{4 \Lambda}{3}\, F^2 + \frac{8 \Lambda}{3}\, \frac{F^4}{r^2 + F^2}
\end{equation} 
depends on $\theta$ only through $F(\theta)$.
A more general assumption would be that $\kappa F + K$ is some function $\chi = \chi(F)$
\begin{equation}
	\label{eq:topdown-1-Fd-Kd-chi}
	\kappa\, F + K = \kappa\, \chi(F).
\end{equation} 
First if $F' = 0$ then $K = 0$ and the definition of $K$ implies
\begin{equation}
	\chi = F = n.
\end{equation} 
The $(t\theta)$- and $(\theta\phi)$-components give the equation
\begin{equation}
	4 \Lambda\, F^2 F' = F'\, \pd_F \chi.
\end{equation} 

If $\Lambda = 0$ we find that
\begin{equation}
	\pd_F \chi = 0
	\Longrightarrow
	\chi = n
\end{equation} 
which reduces to the case studied in \cref{sec:derivation:stationary:simplifying}, while if $F' = 0$ this equation does not provide anything.

On the other hand if $F' \neq 0$ and $\Lambda \neq 0$ then the previous equation becomes
\begin{equation}
	\pd_F \chi = 4 \Lambda F^2
\end{equation} 
which can be integrated to
\begin{equation}
	\label{top-down:eq:chi-F-solution}
	\chi(F) = n + \frac{4}{3}\, \Lambda F^3
\end{equation} 
(notice that the limit $\Lambda \to 0$ is coherent).
Plugging this function into equation \eqref{eq:topdown-1-Fd-Kd-chi} one obtains
\begin{equation}
	\label{eq:topdown-1-Fd-Kd-chi-replaced}
	\kappa\, F + K = \kappa \left(n + \frac{4}{3}\, \Lambda F^3 \right)
\end{equation} 
(remember that $F' \neq 0$).
This differential equation is non-linear and we were not able to find an analytical solution.
Despite that this provides a generalization of the algorithm with non-constant $F$ in the presence of a cosmological constant this is not sufficient for obtaining (a)dS--Kerr: the form of $g_{\theta\theta}$ given in \eqref{deriv:eq:rotating:tr} is not the required one.

Nonetheless by inserting the expression of $\chi$ in $\tilde f$ we see that the last term is killed
\begin{equation}
	\tilde f = \kappa - \frac{2m r - q^2 + 2 \kappa\, n\, F}{r^2 + F^2} - \frac{\Lambda}{3}\, (r^2 + F^2) - \frac{4 \Lambda}{3}\, F^2.
\end{equation} 
One can recognize the function given by Demiański~\cite{Demianski:1972:NewKerrlikeSpacetime} and may explain his error.


\subsubsection{Gauge field integration constant}
\label{sec:derivation:relaxing:gauge-fields}


In \cref{sec:derivation:stationary:simplifying} we obtained a second integration constant $\alpha$ in the expression of the gauge field
\begin{equation}
	\tilde f_A = \frac{q\, r}{r^2 + F^2} + \alpha\, \frac{r^2 - F^2}{r^2 + F^2}.
\end{equation}
One of the Maxwell equation gives $\alpha = 0$ if $F' \neq 0$, but otherwise no equation fixes its value.
For this reason we focus on the case $F' = 0$ or equivalently $\Lambda \neq 0$ through equation \eqref{eq:topdown-1-lambda}.

In this case the function $\tilde f$ is modified to
\begin{equation}
	\tilde f = \kappa - \frac{2m r - q^2 + 2 F (\kappa\, F + K) + 4 \alpha^2 F^2}{r^2 + F^2} - \frac{\Lambda}{3}\, (r^2 + F^2) - \frac{4 \Lambda}{3}\, F^2 + \frac{8 \Lambda}{3}\, \frac{F^4}{r^2 + F^2}.
\end{equation} 
Equation \eqref{eq:topdown-1-lambda} is modified but it is still solved by $F' = 0$ and all other equations are left unchanged (in particular $\kappa F + K$ is still given by the function $\chi(F)$ \eqref{top-down:eq:chi-F-solution}).
For $\chi(F) = n$ the configuration with $\alpha \neq 0$ provides another solution when $\Lambda \neq 0$ but it is not clear how to get it from a complexification of the function.






\section{SIMULATION RESULTS}
\label{sec:examples}
This section presents simulation results of the proposed method implemented on the unicycle model example.
Each semidefinite program was prepared using a custom software toolbox and the modeling tool YALMIP \cite{lofberg2004yalmip}.
The programs are run with commercial solver MOSEK on a machine with $1$ TB availabe memory. 

\subsection{FRS Computation}
We computed the FRS for a 3$^\text{rd}$ order Taylor-expanded Dubins car as the low-fidelity model $f_s$.
Trajectories produced by this model were tracked by the unicycle model from Equation \eqref{eq:big_dyn} as the high-fidelity model $f$.
The vehicle's representation as an initial distribution $X_0 \subset X_s$, was a rectangle of length $0.2$ [m] in $x$ and width $0.1$ [m] in $y$, at $0^\circ$ initial heading, and centered at $x=-0.75$ and $y=0$.
This is the same vehicle representation shown in all previous figures.

% The error function $g$, illustrated in Figure \ref{fig:error_dynamics}, was given by:
% \begin{equation}
% \label{eq:g_definition}
% g(t,x_s) = \begin{bmatrix}
% v_\text{err}\cdot(1 - \frac{1}{2}\theta^2)  \\
% v_\text{err}\cdot(\theta - \frac{1}{6}\theta^3) \\
% \dot{\theta}_\text{err}
% \end{bmatrix}
% \end{equation}
% where $v_\text{err} = (t-1)^2$ and $\dot{\theta}_\text{err} = (t-1)^4$.
We chose $\tau_\text{stop} = \tau_\text{plan} = 0.5$ [s], so $T = 1$ [s].
The stopping time can be seen in Figure \ref{fig:error_dynamics}. 
The FRS computation took 79 hours and used a maximum of 150 GB of memory 
%on a server with 1 TB of available memory and 18 processors each running at 1.2 GHz.

\subsection{Set Intersection and Trajectory Planning}

We used the precomputed FRS for safe trajectory planning in $1000$ simulated trials in MATLAB on the aforementioned machine.
For each trial, the vehicle began at the same initial location and heading, surrounded by $1-10$ randomized obstacles and a randomly-located goal to reach.
%If the planning time took more than $\tau_\text{plan}$, the simulation paused until the computation was complete. 
%In practice, if $\tau_\text{plan}$ was exceeded the vehicle could begin braking to ensure safety.
The vehicle's initial speed, and the desired speed to maintain for the duration of the trial, were randomly chosen between $0.25$ and $0.75$ [m/s].
% The trials ran in 12.7 hours.
% Prior to running these trials, several example trials were run on a laptop with a 2.3 GHz processor and 16 GB of RAM.
% The trials run on the server were individually no faster than running on the laptop, because the set intersection optimization is a single-core process that uses very little memory. 
% Therefore, the server did not provide any significant decrease in the implemented planning time.


Obstacles were represented as line segments between $0.1$ and $0.2$[m] in length, with random location and orientation.
The obstacles were always placed between the vehicle and the goal.
We checked for crashes conservatively for each trial, by inspecting if any obstacle was within a circle circumscribing the rectangular vehicle at any point of the vehicle's trajectory. 
Using this method, \emph{no crashes were detected in any trial}.
Out of all the trials, $82\%$ reached the goal, and $15\%$ performed an emergency braking maneuver (by setting $v_\text{des} = 0$). 
The remaining 3\% hit a simulation iteration limit.
Examples of the vehicle's path from a randomly-generated trial and from two constructed emergency braking cases are shown in Figure \ref{fig:example_trial}.


\begin{figure}
\centering
\includegraphics[width=1\columnwidth]{running_examples.pdf}
\caption{The top subplot shows an example result out of the $1000$ trials.
This trial used eight randomly-generated obstacles.
The vehicle begins on the left at $x = -0.75$ and reaches a randomly-generated goal near $(2.5, 0.5)$, plotted as a blue circle.
Every $\tau_\text{plan} = 0.5$[s], the vehicle replans its trajectory, shown by an asterisk plotted on the global trajectory in blue.
The bounding box of the vehicle at each planning step is shown as a grey rectangle. In the bottom-left subplot, an obstacle was constructed between the vehicle and the goal, forcing an emergency braking maneuver. In the bottom-right subplot, an obstacle was constructed with a hole that would allow the vehicle to pass, but the set intersection result is overly conservative, resulting in a braking maneuver.}
\label{fig:example_trial}
\end{figure}

Currently, our implementation cannot consistently achieve $\tau_\text{plan} = 0.5$ [s].
Consequently, instead of replanning and driving simultaneously, we pause time every 0.5 [s] of the simulation to guarantee that the vehicle can finish replanning.
In a physical implementation, if $\tau_\text{plan}$ is exceeded, then the vehicle must emergency brake; recall that a safe braking trajectory is always available.
As shown in Figure \ref{fig:planning_time_vs_Nobs}, $\tau_\text{plan}$ scales linearly with the number of obstacles.
%Methods for reducing the set intersection to meet $\tau_\text{plan}$ will be presented in future work.

\begin{figure}
\centering
\includegraphics[scale=0.45,trim={1cm 6cm 1cm 7cm},clip]{planning_time_vs_Nobs.pdf}
\caption{The mean set intersection time (top) and trajectory optimization time (bottom) versus the number of obstacles. Over the $1000$ trials, each number of obstacles from $1$ to $10$ was used for $100$ trials. Notice that set intersection takes up to $3$[s], and scales with the number of obstacles. On the other hand, the trajectory optimization takes around $80$ [ms] and has low correlation with number of obstacles.}
\label{fig:planning_time_vs_Nobs}
\end{figure}

% \begin{figure}
% \centering
% \includegraphics[scale=0.5,trim={1cm 8cm 1cm 8cm},clip]{example_trial_bluecar.pdf}
% \caption{An example result out of the 1000 trials.
% This trial used eight randomly-generated obstacles.
% The vehicle begins on the left at $x = -0.75$ and reaches a randomly-generated goal near $(2.5, 0.5)$, plotted as a blue circle.
% Every $\tau_\text{plan} = 0.5$ [s], the vehicle replans its trajectory, shown by an asterisk plotted on the global trajectory in blue.
% The bounding box of the vehicle at each planning step is shown as a grey rectangle.}
% \label{fig:example_trial}
% \end{figure}

% \begin{figure}
% \centering
% \includegraphics[scale=0.4,trim={1cm 7cm 1cm 7cm},clip]{example_emergency_brake.pdf}
% \caption{An example of a forced emergency braking situation. The vehicle cannot find a path to the desired location (plotted as a blue circle), so it brakes.}
% \label{fig:example_emergency_brake}
% \end{figure}

% \begin{figure}
% \centering
% \includegraphics[scale=0.4,trim={1cm 7cm 1cm 7cm},clip]{example_overly_conservative.pdf}
% \caption{An example of an unnecessary emergency braking situation. The vehicle cannot find a path to the desired location despite an obviously-safe path existing, because the FRS is overly conservative.}
% \label{fig:example_overly_conservative}
% \end{figure}

\section{Five dimensional algorithm}
\label{sec:five}


While in four dimensions we have at our disposal many theorems on the classification of solutions, this is not the case for higher dimensions and the bestiary for solutions is much wider and less understood~\cite{Emparan:2008:BlackHolesHigher, Adamo:2014:KerrNewmanMetricReview}.
Rotating solutions in higher dimensions are characterized by several angular momenta.
Important solutions have not yet been discovered, even in the simplest theories such as the charged rotating black holes with several angular momenta in pure Einstein--Maxwell gravity.

Generalizing the JN algorithm in other dimensions is challenging and only small steps have been taken in this direction.
For instance Xu recovered Myers--Perry solution with one angular momentum~\cite{Myers:1986:BlackHolesHigher} from the Schwarzschild--Tangherlini solution~\cite{Xu:1988:ExactSolutionsEinstein} (see also~\cite{Aliev:2006:RotatingBlackHoles}), and Kim showed how the rotating BTZ black hole~\cite{Banados:1992:BlackHoleThree} can be obtained from its static limit~\cite{Kim:1997:NotesSpinningAdS3, Kim:1999:SpinningBTZBlack}.
One of the difficulty is to be able to perform several successive transformations in order to introduce all the allowed angular momenta.

In this section we report the successful generalization of the JN algorithm to five dimensions where we recover two examples~\cite{Erbin:2015:FivedimensionalJanisNewmanAlgorithm}: the complete Myers--Perry black hole~\cite{Myers:1986:BlackHolesHigher} and the Breckenridge--Myers--Peet--Vafa (BMPV) extremal black hole~\cite{Breckenridge:1997:DbranesSpinningBlack}.
We give of proposal for extending this method to higher dimensions in the next section.

It appears that the two angular momenta can be added one after the other by performing two successive transformations, each using different rules for complexifying the functions.
These rules can be understood as transforming only the functions that appear in the part of the metric which describes the rotation plane associated to the angular momentum.
Our method makes use of the Giampieri prescription and we did not succeed in expressing it in terms of the Janis--Newman prescription.

A major application of our work would be to find the charged solution with two angular momenta of the $5d$ Einstein--Maxwell gravity.
This problem is highly non-trivial and there is few chances that this technique would work directly~\cite{Aliev:2006:RotatingBlackHoles}, but one can imagine that a generalization of Demiański's approach~\cite{Demianski:1972:NewKerrlikeSpacetime} (see \cref{sec:derivation}) could lead to new interesting solutions in five dimensions.
An intermediate step is represented by the CCLP metric~\cite{Chong:2005:GeneralNonExtremalRotating} which is a solution of the Einstein--Maxwell theory with a Chern--Simons term, but it cannot be derived from the JN algorithm and we give some intuition about this fact in the last subsection.

Finally one could seek for an extension of the algorithm to the derivation of black rings~\cite{Emparan:2002:RotatingBlackRing, Emparan:2008:BlackHolesHigher}.
Similarly it may be possible that such techniques could be used in $d = 4$ to derive multicentre solutions (for instance one could imagine adding rotation to both centres successively, changing coordinate system in-between to place the origin of the coordinates at each centre).


\subsection{Myers--Perry black hole}
\label{sec:higher-jna:5d:myers-perry}


In this section we show how to recover the Myers--Perry black hole in five dimensions through the Giampieri prescription.
This is a solution of $5$-dimensional pure Einstein theory which possesses two angular momenta and it generalizes the Kerr black hole.
The importance of this solution lies in the fact that it can be constructed in any dimension.

The seed metric is given by the five-dimensional Schwarzschild--Tangherlini metric
\begin{equation}
	\dd s^2 = - f(r)\, \dd t^2 + f(r)^{-1}\, \dd r^2 + r^2\, \dd \Omega_3^2
\end{equation}
where $\dd \Omega_3^2$ is the metric on $S^3$, which can be expressed in Hopf coordinates (see \cref{app:coord:5d:hopf})
\begin{equation}
	\label{higher-jna:eq:coord-S3-spherical}
	\dd \Omega_3^2 = \dd\theta^2 + \sin^2 \theta\, \dd\phi^2 + \cos^2 \theta\, \dd\psi^2,
\end{equation} 
and the function $f(r)$ is given by
\begin{equation}
	f(r) = 1 - \frac{m}{r^2}.
\end{equation}

An important feature of the JN algorithm is the fact that a given set of transformations in the $(r,\phi)$-plane generates rotation in the latter.
Generating several angular momenta in different 2-planes would then require successive applications of the JN algorithm on different hypersurfaces.
In order to do so, one has to identify what are the 2-planes which will be submitted to the algorithm.
In five dimensions, the two different planes that can be made rotating are the planes $(r,\phi)$ and $(r,\psi)$.
We claim that it is necessary to dissociate the radii of these 2-planes in order to apply separately the JN algorithm on each plane and hence to generate two distinct angular momenta.
In order to dissociate the parts of the metric that correspond to the rotating and non-rotating $2$-planes, one can protect the function $r^2$ to be transformed under complex transformations in the part of the metric defining the plane which will stay static.
We thus introduce the function
\begin{equation}
	R(r) = r
\end{equation} 
such that the metric in null coordinates reads
\begin{equation}
	\label{higher-jna:5d-jna:metric:static:general-ur}
	\dd s^2 = - \dd u\, (\dd u + 2 \dd r)
		+ (1 - f)\, \dd u^2
		+ r^2 (\dd\theta^2 + \sin^2 \theta\, \dd\phi^2) + R^2 \cos^2 \theta\, \dd\psi^2.
\end{equation} 
The first transformation -- hence concerning the $(r,\phi)$-plane -- is
\begin{equation}
	\label{higher-jna:eq:5d-ansatz-hopf-1}
	\begin{gathered}
		u = u' + i a \cos \chi_1, \qquad
		r = r' - i a \cos \chi_1, \\
		i\, \dd \chi_1 = \sin \chi_1\ \dd\phi, \qquad\text{~~with~~}\chi_1 = \theta, \\
		\dd u = \dd u' - a \sin^2 \theta\, \dd\phi, \qquad
		\dd r = \dd r' + a \sin^2 \theta\, \dd\phi,
	\end{gathered}
\end{equation}
and $f$ is replaced by $\tilde f^{\{1\}} = \tilde f^{\{1\}}(r, \theta)$.
Indeed one needs to keep track of the order of the transformation, since the function $f$ will be complexified twice consecutively.
On the other hand $R(r) = \Re(r)$ is transformed\footnotemark{} into $R' = r'$ and one finds (omitting the primes)%
\footnotetext{%
	Note that as a function this corresponds to the rule \eqref{gen:eq:rules:r} but we will see below that $R$ is better interpreted as a coordinate since below it will appear as $\dd R$.
}
\begin{equation}
	\begin{aligned}
	\dd s^2 = - \dd u^2 &- 2\, \dd u \dd r
		+ \big(1 - \tilde f^{\{1\}} \big) (\dd u - a \sin^2 \theta\, \dd \phi)^2
		+ 2 a \sin^2 \theta\, \dd r \dd \phi \\
		&+ (r^2 + a^2 \cos^2 \theta) \dd\theta^2
		+ (r^2 + a^2) \sin^2 \theta\, \dd\phi^2
		+ r^2 \cos^2 \theta\, \dd \psi^2.
	\end{aligned}
\end{equation} 
The function $\tilde f^{\{1\}}$ is
\begin{equation}
	\tilde f^{\{1\}} = 1 - \frac{m}{\abs{r}^2} = 1 - \frac{m}{r^2 + a^2 \cos^2 \theta}.
\end{equation} 
There is a cancellation between the $(u, r)$ and the $(\theta, \phi)$ parts of the metric
\begin{subequations}
\begin{align}
	\dd s_{u,r}^2 &= (1 - \tilde f^{\{1\}})\, (\dd u - a \sin^2 \theta\, \dd \phi)^2
		- \dd u (\dd u + 2 \dd r )
		+ 2 a \sin^2 \theta \, \dd r \dd \phi
		+ a^2 \sin^4 \theta\, \dd \phi^2, \\
	\dd s_{\theta,\phi}^2 &= (r^2 + a^2 \cos^2 \theta) \dd\theta^2
			+ \big(r^2 + a^2 (1 - \sin^2 \theta) \big) \sin^2 \theta\, \dd\phi^2.
\end{align}
\end{subequations}

In addition to the terms present in \eqref{higher-jna:5d-jna:metric:static:general-ur} one obtains new components corresponding to the rotation of the first plane $(r, \phi)$.
Since the structure is very similar one can perform a transformation\footnotemark{} in the second plane $(r, \psi)$%
\footnotetext{%
	The easiest justification for choosing the sinus here is by looking at the transformation in terms of direction cosines, see \cref{sec:higher-jna:examples:myers-perry-5d}.
	Otherwise this term can be guessed by looking at Myers--Perry non-diagonal terms.
}
\begin{equation}
	\label{higher-jna:eq:5d-ansatz-hopf-2}
	\begin{gathered}
		u = u' + i b\, \sin \chi_2, \qquad
		r = r' - i b\, \sin \chi_2, \\
		i\, \dd \chi_2 = - \cos \chi_2\, \dd\psi, \qquad \text{~~with~~}\chi_2 = \theta, \\
		\dd u = \dd u' - b \cos^2 \theta\, \dd\psi, \qquad
		\dd r = \dd r' + b \cos^2 \theta\, \dd\psi,
	\end{gathered}
\end{equation}
can be applied directly to the metric
\begin{equation}
	\begin{aligned}
	\dd s^2 = - \dd u^2 &- 2\, \dd u \dd r
		+ \big(1 - \tilde f^{\{1\}} \big) (\dd u - a \sin^2 \theta\, \dd \phi)^2
		+ 2 a \sin^2 \theta\, \dd R \dd \phi \\
		&+ \rho^2 \dd\theta^2
		+ (R^2 + a^2) \sin^2 \theta\, \dd\phi^2
		+ r^2 \cos^2 \theta\, \dd \psi^2
	\end{aligned}
\end{equation} 
where we introduced once again the function $R(r) = \Re(r)$ to protect the geometry of the first plane to be transformed under complex transformations.

The final result (using again $R = r'$ and omitting the primes) becomes
\begin{equation}
	\begin{aligned}
	\dd s^2 = - \dd u^2 &- 2\, \dd u \dd r
		+ \big(1 - \tilde f^{\{1, 2\}} \big) (\dd u - a \sin^2 \theta\, \dd \phi - b \cos^2 \theta\, \dd \psi)^2
		\\
		&+ 2 a \sin^2 \theta\, \dd r \dd \phi
		+ 2 b \cos^2 \theta\, \dd r\dd \psi \\
		&+ \rho^2 \dd\theta^2
		+ (r^2 + a^2) \sin^2 \theta\, \dd\phi^2
		+ (r^2 + b^2) \cos^2 \theta\, \dd \psi^2
	\end{aligned}
\end{equation} 
where
\begin{equation}
	\rho^2 = r^2 + a^2 \cos^2 \theta + b^2 \sin^2 \theta.
\end{equation} 
Furthermore, the function $\tilde f^{\{1\}}$ has been complexified as
\begin{equation}
 	\tilde f^{\{1,2\}} = 1 - \frac{m}{\abs{r}^2 + a^2 \cos^2 \theta}
		= 1 - \frac{m}{r'^2 + a^2 \cos^2 \theta + b^2 \sin^2 \theta}
		= 1 - \frac{m}{\rho^2}.
\end{equation}

The metric can then be transformed into the Boyer--Lindquist (BL) using
\begin{equation}
	\label{higher:change:5d-bl}
	\dd u = \dd t - g(r)\, \dd r, \qquad
	\dd\phi = \dd\phi' - h_\phi(r)\, \dd r, \qquad
	\dd\psi = \dd\psi' - h_\psi(r)\, \dd r.
\end{equation} 
Defining the parameters\footnotemark{}%
\footnotetext{%
	See \eqref{higher-jna:metric:rotating:result-jna-bl-parameters} for a definition of $\Delta$ in terms of $\tilde f$.
}
\begin{equation}
	\Pi = (r^2 + a^2) (r^2 + b^2), \qquad
	\Delta = r^4 + r^2 (a^2 + b^2- m) + a^2 b^2,
\end{equation}
the functions can be written
\begin{equation}
	\label{higher:change:myers-perry:bl-g-h}
	g(r) = \frac{\Pi}{\Delta}, \qquad
	h_\phi(r) = \frac{\Pi}{\Delta}\, \frac{a}{r^2 + a^2}, \qquad
	h_\psi(r) = \frac{\Pi}{\Delta}\, \frac{b}{r^2 + b^2}.
\end{equation} 
Finally one gets
\begin{equation}
	\label{higher-jna:metric:rotating:5d-2-moments-bl}
	\begin{aligned}
		\dd s^2 = - \dd t^2
			&+ \big(1 - \tilde f^{\{1, 2\}} \big) (\dd t - a \sin^2 \theta\, \dd \phi - b \cos^2 \theta\, \dd \psi)^2
			+ \frac{r^2 \rho^2}{\Delta}\, \dd r^2 \\
			&+ \rho^2 \dd\theta^2
			+ (r^2 + a^2) \sin^2 \theta\, \dd\phi^2
			+ (r^2 + b^2) \cos^2 \theta\, \dd \psi^2.
	\end{aligned}
\end{equation} 
One recovers here the five dimensional Myers--Perry black hole with two angular momenta~\cite{Myers:1986:BlackHolesHigher}.


\subsection{BMPV black hole}
\label{sec:higher-jna:5d:bmpv}




\subsubsection{Few properties and seed metric}


In this section we focus on another example in five dimensions, which is the BMPV black hole~\cite{Breckenridge:1997:DbranesSpinningBlack, Gauntlett:1999:BlackHolesD5}.
This solution possesses many interesting properties, in particular it can be proven that it is the only asymptotically flat rotating BPS black hole in five dimensions with the corresponding near-horizon geometry~\cites[sec.~7.2.2, 8.5]{Emparan:2008:BlackHolesHigher}{Reall:2003:HigherDimensionalBlack}.\footnotemark{}%
\footnotetext{%
	Other possible near-horizon geometries are $S^1 \times S^2$ (for black rings) and $T^3$, even if the latter does not seem really physical.
	BMPV horizon corresponds to the squashed $S^3$.
}
It is interesting to notice that even if this extremal solution is a slowly rotating metric, it is an exact solution (whereas Einstein equations need to be truncated for consistency of usual slow rotation).

For a rotating black hole the BPS and extremal limits do not coincide~\cites[sec.~7.2]{Emparan:2008:BlackHolesHigher}[sec.~1]{Gauntlett:1999:BlackHolesD5}: the first implies that the mass is related to the electric charge,\footnote{It is a consequence from the BPS bound $m \ge \sqrt{3}/2\, \abs{q}$.} while extremality\footnotemark{}%
\footnotetext{%
	Regularity is given by a bound, which is saturated for extremal black holes.
}
implies that one linear combination of the angular momenta vanishes, and for this reason we set $a = b$ from the beginning.\footnotemark{}%
\footnotetext{%
	If we had kept $a \neq b$ we would have discovered later that one cannot transform the metric to Boyer--Lindquist coordinates without setting $a = b$.
}
Thus two independent parameters are left and are taken to be the mass and one angular momentum.

In the non-rotating limit BMPV black hole reduces to the charged extremal Schwarz\-schild--Tangherlini (with equal mass and charge) written in isotropic coordinates.
For non-rotating black hole the extremal and BPS limit are equivalent.

Both the charged extremal Schwarzschild--Tangherlini and BMPV black holes are solutions of minimal ($N = 2$) $d = 5$ supergravity (Einstein--Maxwell plus Chern--Simons) whose bosonic action is~\cites[sec.~1]{Gauntlett:1999:BlackHolesD5}[sec.~2]{Aliev:2014:SuperradianceBlackHole}[sec.~2]{Gauntlett:2003:AllSupersymmetricSolutions}
\begin{equation}
	\label{higher-jna:higher-jna:action:N=2-d=5-sugra}
	S = - \frac{1}{16\pi G} \int \left(R\, \hodge{1} + F \wedge \hodge{F} + \frac{2\lambda}{3 \sqrt{3}}\, F \wedge F \wedge A \right),
\end{equation} 
where supersymmetry imposes $\lambda = 1$.

Since extremal limits are different for static and rotating black holes we can guess that the black hole obtained from the algorithm will not be a solution of the equations of motion and that it will be necessary to take some limit.

The charged extremal Schwarzschild--Tangherlini black hole is taken as a seed metric~\cites[sec.~3.2]{Gauntlett:2003:AllSupersymmetricSolutions}[sec.~4]{Gibbons:1994:SupersymmetricSelfGravitatingSolitons}[sec.~1.3.1]{Puhm:2013:BlackHolesString}
\begin{equation}
	\label{higher-jna:metric:5d-bmpv}
	\dd s^2 = - H^{-2}\, \dd t^2 + H\, (\dd r^2 + r^2\, \dd\Omega_3^2 )
\end{equation} 
where $\dd\Omega_3^2$ is the metric of the $3$-sphere written in
\eqref{higher-jna:eq:coord-S3-spherical}.
The function $H$ is harmonic
\begin{equation}
	H(r) = 1 + \frac{m}{r^2},
\end{equation} 
and the electromagnetic field reads
\begin{equation}
	\label{higher-jna:pot:5d-bmpv}
	A = \frac{\sqrt{3}}{2 \lambda}\, \frac{m}{r^2}\, \dd t
		= (H - 1)\, \dd t.
\end{equation} 

In the next subsections we apply successively the transformations \eqref{higher-jna:eq:5d-ansatz-hopf-1} and \eqref{higher-jna:eq:5d-ansatz-hopf-2} with $a = b$ in the case $\lambda = 1$.


\subsubsection{Transforming the metric}


The transformation to $(u, r)$ coordinates of the seed metric \eqref{higher-jna:metric:5d-bmpv}
\begin{equation}
	\dd t = \dd u + H^{3/2}\, \dd r
\end{equation} 
gives
\begin{subequations}
\begin{align}
	\dd s^2 &= - H^{-2}\, \dd u^2 - 2 H^{-1/2}\, \dd u \dd r + H r^2\, \dd\Omega_3^2 \\
		&= - H^{-2}\, \big(\dd u - 2 H^{3/2}\, \dd r \big)\, \dd u + H r^2\, \dd\Omega_3^2.
\end{align}
\end{subequations}

For transforming the above metric one should follow the recipe of the previous section: the transformations \eqref{higher-jna:eq:5d-ansatz-hopf-1}
\begin{equation}
	u = u' + i a \cos \theta, \qquad
	\dd u = \dd u' - a \sin^2 \theta\, \dd\phi,
\end{equation}
and \eqref{higher-jna:eq:5d-ansatz-hopf-2}
\begin{equation}
	u = u' + i a\, \sin \theta, \qquad
	\dd u = \dd u' - a \cos^2 \theta\, \dd\psi
\end{equation} 
are performed one after another, transforming each time only the terms that pertain to the corresponding rotation plane.\footnotemark{}%
\footnotetext{%
	For another approach see \cref{sec:higher-jna:5d:bmpv-second-approach}.
}
In order to preserve the isotropic form of the metric the function $H$ is complexified everywhere (even when it multiplies terms that belong to the other plane).

Since the procedure is exactly similar to the Myers--Perry case we give only the final result in $(u, r)$ coordinates
\begin{equation}
	\label{higher-jna:metric:5d-bmpv:ur-before-limit}
	\begin{aligned}
		\dd s^2 = &- \tilde H^{-2} \big(\dd u
				- a (1 - \tilde H^{3/2}) (\sin^2 \theta\, \dd\phi + \cos^2 \theta\, \dd\psi) \big)^2 \\
			&- 2 \tilde H^{-1/2} \big(\dd u - a (1 - \tilde H^{3/2})\, (\sin^2 \theta\, \dd\phi + \cos^2 \theta\, \dd\psi) \big)\, \dd r \\
			&+ 2 a \tilde H\, (\sin^2 \theta\, \dd\phi + \cos^2 \theta\, \dd\psi)\, \dd r
			- 2 a^2 \tilde H \cos^2 \theta \sin^2 \theta\, \dd\phi \dd\psi
			\\
			&+ \tilde H\, \Big(
				(r^2 + a^2) (\dd \theta^2 + \sin^2 \theta\, \dd\phi^2 + \cos^2 \theta\, \dd\psi^2)
				+ a^2 (\sin^2 \theta\, \dd\phi + \cos^2 \theta\, \dd\psi)^2 \Big).
	\end{aligned}
\end{equation} 
After both transformations the resulting function $\tilde H$ is
\begin{equation}
	\label{higher-jna:eq:5d-bmpv:tilde-H}
	\tilde H = 1 + \frac{m}{r^2 + a^2 \cos^2\theta + a^2 \sin^2\theta}
		= 1 + \frac{m}{r^2 + a^2}
\end{equation}
which does not depend on $\theta$.

It is easy to check that the Boyer--Lindquist transformation \eqref{higher:change:5d-bl}
\begin{equation}
	\dd u = \dd t - g(r)\, \dd r, \qquad
	\dd\phi = \dd\phi' - h_\phi(r)\, \dd r, \qquad
	\dd\psi = \dd\psi' - h_\psi(r)\, \dd r
\end{equation} 
is ill-defined because the functions depend on $\theta$.
The way out is to take the extremal limit alluded above.

Following the prescription of \cite{Breckenridge:1997:DbranesSpinningBlack, Gauntlett:1999:BlackHolesD5} and taking the extremal limit
\begin{equation}
	\label{higher-jna:eq:5d-bmpv-extremal-limit}
	a, m \longrightarrow 0, \qquad
	\text{imposing} \qquad
	\frac{m}{a^2} = \cst ,
\end{equation}
one gets at leading order
\begin{equation}
	\tilde H(r) = 1 + \frac{m}{r^2} = H(r), \qquad
	a\, (1 - \tilde H^{3/2}) = - \frac{3\, m a}{2\, r^2}
\end{equation} 
which translate into the metric
\begin{equation}
	\begin{aligned}
		\dd s^2 = - H^{-2}\, & \left(\dd u
				+ \frac{3\, m a}{2\, r^2}\, (\sin^2 \theta\, \dd\phi + \cos^2 \theta\, \dd\psi) \right)^2 \\
			&- 2 H^{-1/2} \left( \dd u + \frac{3\, m a}{2\, r^2}\, (\sin^2 \theta\, \dd\phi + \cos^2 \theta\, \dd\psi) \right) \dd r \\
			&+ H\, r^2 (\dd \theta^2 + \sin^2 \theta\, \dd\phi^2 + \cos^2 \theta\, \dd\psi^2).
	\end{aligned}
\end{equation} 
Then Boyer--Lindquist functions are
\begin{equation}
	\label{higher:change:bmpv:g-h}
	g(r) = H(r)^{3/2}, \qquad
	h_\phi(r) = h_\psi(r) = 0
\end{equation} 
and one gets the metric in $(t, r)$ coordinates
\begin{equation}
	\label{higher-jna:metric:5d-bmpv:bmpv-metric}
	\begin{aligned}
		\dd s^2 = &- \tilde H^{-2} \left(\dd t
				+ \frac{3\, m a}{2\, r^2}\, (\sin^2 \theta\, \dd\phi + \cos^2 \theta\, \dd\psi) \right)^2 \\
			&+ \tilde H\, \Big(\dd r^2 + r^2 \big( \dd \theta^2 + \sin^2 \theta\, \dd\phi^2 + \cos^2 \theta\, \dd\psi^2 \big) \Big).
	\end{aligned}
\end{equation} 
One can recognize the BMPV solution~\cites[p.~4]{Breckenridge:1997:DbranesSpinningBlack}[p.~16]{Gauntlett:1999:BlackHolesD5}.
The fact that this solution has only one rotation parameter can be seen more easily in Euler angle coordinates~\cites[sec.~3]{Gauntlett:1999:BlackHolesD5}[sec.~2]{Gibbons:1999:SupersymmetricRotatingBlack} or by looking at the conserved charges in the $\phi$- and $\psi$-planes~\cite[sec.~3]{Breckenridge:1997:DbranesSpinningBlack}.


\subsubsection{Transforming the Maxwell potential}


The seed gauge field \eqref{higher-jna:pot:5d-bmpv} in the $(u, r)$ coordinates is
\begin{equation}
	A = \frac{\sqrt{3}}{2}\, (H - 1)\, \dd u,
\end{equation} 
since the $A_r(r)$ component can be removed by a gauge transformation.
One can apply the two JN transformations \eqref{higher-jna:eq:5d-ansatz-hopf-1} and \eqref{higher-jna:eq:5d-ansatz-hopf-2} with $b = a$ to obtain
\begin{equation}
	A = \frac{\sqrt{3}}{2}\, (\tilde H - 1) \Big( \dd u - a\, (\sin^2 \theta\, \dd\phi + \cos^2 \theta\, \dd\psi) \Big).
\end{equation} 

Then going into BL coordinates with \eqref{higher:change:5d-bl} and \eqref{higher:change:bmpv:g-h} provides
\begin{equation}
	A = \frac{\sqrt{3}}{2}\, (\tilde H - 1) \Big( \dd t - a\, (\sin^2 \theta\, \dd\phi + \cos^2 \theta\, \dd\psi) \Big) + A_r(r)\, \dd r.
\end{equation} 
Again $A_r$ depends only on $r$ and can be removed by a gauge transformation.
Applying the extremal limit \eqref{higher-jna:eq:5d-bmpv-extremal-limit} finally gives
\begin{equation}
	A = \frac{\sqrt{3}}{2}\, \frac{m}{r^2} \Big( \dd t - a\, (\sin^2 \theta\, \dd\phi + \cos^2 \theta\, \dd\psi) \Big),
\end{equation}
which is again the result presented in~\cite[p. 5]{Breckenridge:1997:DbranesSpinningBlack}.

Despite the fact that the seed metric \eqref{higher-jna:metric:5d-bmpv} together with the gauge field \eqref{higher-jna:pot:5d-bmpv} solves the equations of motion for any value of $\lambda$, the resulting rotating metric solves the equations only for $\lambda = 1$ (see~\cite[sec.~7]{Gauntlett:1999:BlackHolesD5} for a discussion).
An explanation in this reduction can be found in the limit \eqref{higher-jna:eq:5d-bmpv-extremal-limit} that was needed for transforming the metric to Boyer--Lindquist coordinates and which gives a supersymmetric black hole -- which necessarily has $\lambda = 1$.



\subsection{Another approach to BMPV}
\label{sec:higher-jna:5d:bmpv-second-approach}


In \cref{sec:higher-jna:5d:bmpv} we applied the same recipe given in \cref{sec:higher-jna:5d:myers-perry} which, according to our claim, is the standard procedure in five dimensions.

There is another way to derive BMPV black hole.
Indeed, by considering that terms quadratic in the angular momentum do not survive in the extremal limit, they can be added to the metric without modifying the final result.
Hence we can decide to transform all the terms of the metric\footnotemark{} since the additional terms will be subleading.%
\footnotetext{%
	In opposition to our initial recipe, but this is done in a controlled way.
}
As a result the BL transformation is directly well defined and overall formulas are simpler, but we need to take the extremal limit before the end (this could be done either in $(u, r)$ or $(t, r)$ coordinates).
This section shows that both approaches give the same result.

Applying the two transformations
\begin{subequations}
\begin{gather}
	u = u' + i a \cos \theta, \qquad
	\dd u = \dd u' - a \sin^2 \theta\, \dd\phi, \\
	u = u' + i a\, \sin \theta, \qquad
	\dd u = \dd u' - a \cos^2 \theta\, \dd\psi
\end{gather}
\end{subequations}
successively on all the terms one obtains the metric
\begin{equation}
	\begin{aligned}
		\dd s^2 = &- \tilde H^{-2} \big(\dd u
				- a (1 - \tilde H^{3/2}) (\sin^2 \theta\, \dd\phi + \cos^2 \theta\, \dd\psi) \big)^2 \\
			&- 2 \tilde H^{-1/2} \big(\dd u - a (\sin^2 \theta\, \dd\phi + \cos^2 \theta\, \dd\psi) \big)\, \dd r \\
			&+ \tilde H\, \Big(
				(r^2 + a^2) (\dd \theta^2 + \sin^2 \theta\, \dd\phi^2 + \cos^2 \theta\, \dd\psi^2)
				+ a^2 (\sin^2 \theta\, \dd\phi + \cos^2 \theta\, \dd\psi)^2 \Big),
	\end{aligned}
\end{equation} 
where again $\tilde H$ is given by \eqref{higher-jna:eq:5d-bmpv:tilde-H}
\begin{equation}
	\tilde H = 1 + \frac{m}{r^2 + a^2}.
\end{equation} 
Only one term is different when comparing with \eqref{higher-jna:metric:5d-bmpv:ur-before-limit}.

The BL transformation \eqref{higher:change:5d-bl} is well-defined and the corresponding functions are
\begin{equation}
	\label{higher-jna:change:bmpv-2:bl-gh}
	g(r) = \frac{a^2 + (r^2 + a^2) \tilde H(r)}{r^2 + 2 a^2}, \qquad
	h_\phi(r) = h_\psi(r) = \frac{a}{r^2 + 2 a^2}
\end{equation} 
which do not depend on $\theta$.
They lead to the metric
\begin{equation}
	\begin{aligned}
		\dd s^2 = &- \tilde H^{-2} \big(\dd t
				- a (1 - \tilde H^{3/2}) (\sin^2 \theta\, \dd\phi + \cos^2 \theta\, \dd\psi) \big)^2 \\
			&+ \tilde H\, \bigg[
				(r^2 + a^2) \left(\frac{\dd r^2}{r^2 + 2 a^2} + \dd \theta^2 + \sin^2 \theta\, \dd\phi^2 + \cos^2 \theta\, \dd\psi^2 \right) \\
				&\qquad\quad+ a^2 (\sin^2 \theta\, \dd\phi + \cos^2 \theta\, \dd\psi)^2 \bigg].
	\end{aligned}
\end{equation} 

At this point it is straightforward to check that this solution does not satisfy Einstein equations and we need to take the extremal limit \eqref{higher-jna:eq:5d-bmpv-extremal-limit}
\begin{equation}
	a, m \longrightarrow 0, \qquad
	\text{imposing} \qquad
	\frac{m}{a^2} = \cst
\end{equation}
in order to get the BMPV solution \eqref{higher-jna:metric:5d-bmpv:bmpv-metric}
\begin{equation}
	\begin{aligned}
		\dd s^2 = &- \tilde H^{-2} \left(\dd t
				+ \frac{3\, m a}{2\, r^2}\, (\sin^2 \theta\, \dd\phi + \cos^2 \theta\, \dd\psi) \right)^2 \\
			&+ \tilde H\, \Big(\dd r^2 + r^2 \big( \dd \theta^2 + \sin^2 \theta\, \dd\phi^2 + \cos^2 \theta\, \dd\psi^2 \big) \Big).
	\end{aligned}
\end{equation} 

It is surprising that the BL transformation is simpler in this case.
Another point that is worth stressing is that we did not need to take the extremal limit at an intermediate stage, whereas in \cref{sec:higher-jna:5d:bmpv} we had to in order to get a well-defined BL transformation.


\subsection{CCLP black hole}
\label{sec:higher-jna:5d:cclp}


The CCLP black hole~\cite{Chong:2005:GeneralNonExtremalRotating} (see also~\cite[sec.~2]{Aliev:2014:SuperradianceBlackHole}) corresponds to the non-extremal generalization of the BMPV solution and it possesses four independent charges: two angular momenta $a$ and $b$, an electric charge $q$ and the mass $m$.
It is a solution of $d = 5$ minimal supergravity \eqref{higher-jna:higher-jna:action:N=2-d=5-sugra}.

The solution reads
\begin{subequations}
\begin{gather}
	\label{higher-jna:higher-jna:metric:cclp}
	\begin{aligned}
		\dd s^2 = - \dd t^2
			&+ (1 - \tilde f) (\dd t - a \sin^2 \theta\, \dd \phi - b \cos^2 \theta\, \dd \psi)^2
			+ \frac{r^2 \rho^2}{\Delta_r}\, \dd r^2 \\
			&+ \rho^2 \dd\theta^2
			+ (r^2 + a^2) \sin^2 \theta\, \dd\phi^2
			+ (r^2 + b^2) \cos^2 \theta\, \dd \psi^2 \\
			&- \frac{2 q}{\rho^2}\, (b \sin^2 \theta\, \dd \phi + a \cos^2 \theta\, \dd \psi) (\dd t - a \sin^2 \theta\, \dd \phi - b \cos^2 \theta\, \dd \psi),
	\end{aligned} \\
	A = \frac{\sqrt{3}}{2}\, \frac{q}{\rho^2} (\dd t - a \sin^2 \theta\, \dd \phi - b \cos^2 \theta\, \dd \psi),
\end{gather}
\end{subequations}
where the function are given by
\begin{subequations}
\begin{align}
	\rho^2 &= r^2 + a^2 \cos^2 \theta + b^2 \sin^2 \theta, \\
	\tilde f &= 1 - \frac{2 m}{\rho^2} + \frac{q^2}{\rho^4}, \\
	\Delta_r &= \Pi + 2 a b q + q^2 - 2 m r^2.
\end{align}
\end{subequations}

Yet, using our prescription, it appears that the metric of this black hole cannot entirely be recovered.
Indeed while the gauge field can be found straightforwardly, all the terms of the metric but one are generated by our algorithm.
The missing term (corresponding to the last one in \eqref{higher-jna:higher-jna:metric:cclp}) is proportional to the electric charge and the current prescription cannot generate it since the latter can only appear in $\tilde f$ (or in the gauge field); moreover the algorithm cannot explain the first term in parenthesis since $a$ and $b$ always appear with $\dd\phi$ and $\dd\psi$ respectively.

This issue may be related to the fact that the CCLP solution cannot be written as a Kerr--Schild metric but rather as an extended Kerr--Schild one~\cite{Aliev:2009:NoteRotatingCharged, Ett:2010:ExtendedKerrSchildAnsatz, Malek:2014:ExtendedKerrSchildSpacetimes}, which includes an additional term proportional to a spacelike vector.
It appears that the missing term corresponds precisely to this additional term in the extended Kerr--Schild metric and it is well-known that the JN algorithm works mostly for Kerr--Schild metrics.
Moreover the $\Delta$ computed from \eqref{higher-jna:metric:rotating:result-jna-bl-parameters} depends on $\theta$ and the BL transformation would not be well-defined if the additional term is not present to modify $\Delta$ to $\Delta_r$.


\section{Algorithm in any dimension}
\label{sec:higher}


Following the same prescription in dimensions higher than five does not lead as nicely to the exact Myers--Perry solution.
Indeed we show in this section that, while the transformation of the metric can be done along the same line, the -- major -- obstacle comes from the function $f$ that cannot be transformed as expected.
Finding the correct complexification seems very challenging and it may be necessary to use a different complex coordinate transformation in order to perform a completely general transformation in any dimension.
It might be possible to gain insight into this problem by computing the transformation within the framework of the tetrad formalism.
One may think that a possible solution would be to replace complex numbers by quaternions, assigning one angular momentum to each complex direction but it is straightforward to check that this approach is not working.

The key element to perform the algorithm on the metric is to parametrize the metric on the sphere by direction cosines since these coordinates are totally symmetric under permutation of angular momenta (at the opposite of the spherical coordinates).
We are able to derive the general form of a rotating metric with the maximal number of angular momenta it can have in $d$ dimensions, but we are not able to apply this result to any specific example for $d \ge 6$, except if all momenta but one are vanishing.
Nonetheless this provides a unified view of the JN algorithm in any $d \ge 3$.
We conclude this section by few examples, including the singly-rotating Myers--Perry solution in any dimension and the rotating BTZ black hole.

It would be very desirable to derive the general $d$-dimensional Myers--Perry solution~\cite{Myers:1986:BlackHolesHigher}, or at least to understand why only the metric can be found, and not the function inside.


\subsection{Metric transformation}
\label{sec:higher-jna:any-dimension}


We consider the JN algorithm applied to a general static $d$-dimension metric and show how the tensor structure can be transformed.
In the following the dimension is taken to be odd in order to simplify the computations but the final result holds also for $d$ even.



\subsubsection{Seed metric and discussion}


Consider the $d$-dimensional static metric (notations are defined in \cref{app:coord:general-d})
\begin{equation}
	\dd s^2 = - f\, \dd t^2 + f^{-1}\, \dd r^2 + r^2\, \dd \Omega_{d-2}^2
\end{equation} 
where $\dd \Omega_{d-2}^2$ is the metric on $S^{d-2}$
\begin{equation}
	\dd \Omega_{d-2}^2 = \dd\theta_{d-2} + \sin^2 \theta_{d-2}\, \dd \Omega_{d-3}^2
		= \sum_{i=1}^n \big( \dd\mu_i^2 + \mu_i^2 \dd\phi_i^2).
\end{equation} 
The number $n = (d-1) / 2$ counts the independent $2$-spheres.

In Eddington--Finkelstein coordinates the metric reads
\begin{equation}
	\label{higher-jna:higher-jna:metric:static-seed}
	\dd s^2 = (1 - f)\, \dd u^2 - \dd u\, (\dd u + 2 \dd r)
			+ r^2 \sum_i \Big(\dd \mu_i^2 + \mu_i^2\, \dd \phi_i^2 \Big).
\end{equation} 

The metric looks like a $2$-dimensional space $(t, r)$ with a certain number of additional $2$-spheres $(\mu_i, \phi_i)$ which are independent from one another.
Then we can consider only the piece $(u, r, \mu_i, \phi_i)$ (for fixed $i$) which will transform like a $4$-dimensional spacetime, while the other part of the metric $(\mu_j, \phi_j)$ for all $j \neq i$ will be unchanged.
After the first transformation we can move to another $2$-sphere.
We can thus imagine to put in rotation only one of these spheres.
Then we will apply again and again the algorithm until all the spheres have angular momentum: the whole complexification will thus be a $n$-steps process.
Moreover if these $2$-spheres are taken to be independent this implies that we should not complexify the functions that are not associated with the plane we are putting in rotation.

To match these demands the metric is rewritten as
\begin{equation}
	\label{higher-jna:higher-jna:higher-jna:metric:static-seed-ur}
	\dd s^2 = (1 - f)\, \dd u^2 - \dd u\, (\dd u + 2 \dd r_{i_1})
		+ r_{i_1}^2 (\dd\mu_{i_1}^2 + \mu_{i_1}^2 \dd\phi_{i_1}^2)
		+ \sum_{i \neq i_1} \Big(r_{i_1}^2 \dd \mu_i^2 + R^2 \mu_i^2\, \dd \phi_i^2 \Big).
\end{equation} 
where we introduced the following two functions of $r$
\begin{equation}
	r_{i_1}(r) = r, \qquad R(r) = r.
\end{equation} 
This allows to choose different complexifications for the different terms in the metric.
It may be surprising to note that the factors in front of $\dd \mu_i^2$ have been chosen to be $r_{i_1}^2$ and not $R^2$, but the reason is that the $\mu_i$ are all linked by the constraint
\begin{equation}
	\sum_i \mu_i^2 = 1
\end{equation} 
and the transformation of one $i_1$-th $2$-sphere will change the corresponding $\mu_{i_1}$, but also all the others, as it is clear from the formula \eqref{coord:eq:spherical-to-oblate-mu} with all the $a_i$ vanishing but one (this can also be observed in $5d$ where both $\mu_i$ are gathered into $\theta$).


\subsubsection{First transformation}


The transformation is chosen to be
\begin{subequations}
\label{higher-jna:higher-jna:change:jna-1}
\begin{equation}
	r_{i_1} = r'_{i_1} - i\, a_{i_1} \sqrt{1 - \mu_{i_1}^2}, \qquad
	u = u' + i\, a_{i_1} \sqrt{1 - \mu_{i_1}^2}
\end{equation} 
which, together with the ansatz
\begin{equation}
	i\, \frac{\dd \mu_{i_1}}{\sqrt{1 - \mu_{i_1}^2}} = \mu_{i_1}\, \dd \phi_{i_1},
\end{equation} 
gives the differentials
\begin{equation}
	\dd r_{i_1} = \dd r'_{i_1} + a_{i_1} \mu_{i_1}^2\, \dd \phi_{i_1}, \qquad
	\dd u = \dd u' - a_{i_1} \mu_{i_1}^2\, \dd \phi_{i_1}.
\end{equation} 
\end{subequations}
It is easy to check that this transformation reproduces the one given in four and five dimensions.
The complexified version of $f$ is written as $\tilde f^{\{i_1\}}$: we need to keep track of the order in which we gave angular momentum since the function $\tilde f$ will be transformed at each step.

We consider separately the transformation of the $(u, r)$ and $\{ \mu_i, \phi_i \}$ parts.
Inserting the transformations \eqref{higher-jna:higher-jna:change:jna-1} in \eqref{higher-jna:higher-jna:metric:static-seed} results in
\begin{subequations}
\begin{align*}
	\dd s_{u,r}^2 &= (1 - \tilde f^{\{i_1\}})\, \Big(\dd u - a_{i_1} \mu_{i_1}^2\, \dd \phi_{i_1} \Big)^2
		- \dd u\, (\dd u + 2 \dd r_{i_1})
		+ 2 a_{i_1} \mu_{i_1}^2\, \dd r_{i_1} \dd \phi_{i_1}
		+ a_{i_1}^2 \mu_{i_1}^4\, \dd \phi_{i_1}^2, \\
	%
	\dd s_{\mu,\phi}^2 &= \big( r_{i_1}^2 + a_{i_1}^2 \big) (\dd\mu_{i_1}^2 + \mu_{i_1}^2 \dd\phi_{i_1}^2)
		+ \sum_{i \neq i_1} \big( r_{i_1}^2 \dd \mu_i^2 + R^2 \mu_i^2\, \dd \phi_i^2 \big) - a_{i_1}^2 \mu_{i_1}^4\, \dd \phi_{i_1}^2 \\
		&\qquad + a_{i_1}^2 \bigg[- \mu_{i_1}^2 \dd \mu_{i_1}^2 + (1 - \mu_{i_1}^2) \sum_{i \neq i_1} \dd \mu_i^2 \bigg].
\end{align*}
\end{subequations}

The term in the last bracket vanishes as can be seen by using the differential of the constraint
\begin{equation}
	\sum_i \mu_i^2 = 1 \Longrightarrow
	\sum_i \mu_i \dd\mu_i = 0.
\end{equation} 
Since this step is very important and non-trivial we expose the details
\begin{align*}
	[\cdots] &= \mu_{i_1}^2 \dd \mu_{i_1}^2 - (1 - \mu_{i_1}^2) \sum_{i \neq i_1} \dd \mu_i^2
		= \left(\sum_{i \neq i_1} \mu_i \dd\mu_i \right)^2 - \sum_{j \neq i_1} \mu_j^2 \sum_{i \neq i_1} \dd \mu_i^2 \\
		&= \sum_{i,j \neq i_1} \big(\mu_i \mu_j \dd\mu_i \dd\mu_j - \mu_j^2 \dd \mu_i^2 \big)
		= \sum_{i,j \neq i_1} \mu_j \big(\mu_i \dd\mu_j - \mu_j \dd \mu_i \big) \dd\mu_i
		= 0
\end{align*}
by antisymmetry.

Setting $r_{i_1} = R = r$ one obtains the metric
\begin{equation}
\begin{aligned}
	\dd s^2 &= (1 - \tilde f^{\{i_1\}})\, \Big(\dd u - a_{i_1} \mu_{i_1}^2\, \dd \phi_{i_1} \Big)^2
		- \dd u\, (\dd u + 2 \dd r)
		+ 2 a_{i_1} \mu_{i_1}^2\, \dd r \dd \phi_{i_1} \\
		&\qquad+ \big( r^2 + a_{i_1}^2 \big) (\dd\mu_{i_1}^2 + \mu_{i_1}^2 \dd\phi_{i_1}^2)
		+ \sum_{i \neq i_1} r^2 \big( \dd \mu_i^2 + \mu_i^2\, \dd \phi_i^2 \big).
\end{aligned}
\end{equation}
It corresponds to Myers--Perry metric in $d$ dimensions with one non-vanishing angular momentum.
We recover the same structure as in \eqref{higher-jna:higher-jna:higher-jna:metric:static-seed-ur} with some extra terms that are specific to the $i_1$-th $2$-sphere.


\subsubsection{Iteration and final result}


We should now split again $r$ in functions $(r_{i_2}, R)$.
Very similarly to the first time we have
\begin{equation}
\begin{aligned}
	\dd s^2 &= (1 - \tilde f^{\{i_1\}})\, \Big(\dd u - a_{i_1} \mu_{i_1}^2\, \dd \phi_{i_1} \Big)^2
		- \dd u\, (\dd u + 2 \dd r_{i_2})
		+ 2 a_{i_1} \mu_{i_1}^2\, \dd R \dd \phi_{i_1} \\
		&\qquad+ \big( r_{i_2}^2 + a_{i_1}^2 \big) \dd\mu_{i_1}^2
		+ \big( R^2 + a_{i_1}^2 \big) \mu_{i_1}^2 \dd\phi_{i_1}^2
		+ r_{i_2}^2 ( \dd\mu_{i_2}^2 + \mu_{i_2}^2 \dd\phi_{i_2}^2 ) \\
		&\qquad+ \sum_{i \neq i_1, i_2} \Big(r_{i_2}^2 \dd \mu_i^2 + R^2 \mu_i^2\, \dd \phi_i^2 \Big).
\end{aligned}
\end{equation}
We can now complexify as
\begin{equation}
	r_{i_2} = r'_{i_2} - i a_{i_2} \sqrt{1 - \mu_{i_2}^2}, \qquad
	u = u' + i\, a_{i_1} \sqrt{1 - \mu_{i_2}^2}.
\end{equation} 
The steps are exactly the same as before, except that we have some inert terms.
The complexified functions is now $\tilde f^{\{i_1, i_2\}}$.

Repeating the procedure $n$ times we arrive at
\begin{equation}
	\label{higher-jna:metric:rotating:result-jna-ur}
	\begin{aligned}
		\dd s^2 = &- \dd u^2 - 2 \dd u \dd r
			+ \sum_i (r^2 + a_i^2) (\dd \mu_i^2 + \mu_i^2 \dd \phi_i^2)
			- 2 \sum_i a_i \mu_i^2 \, \dd r \dd \phi_i \\
			&+ \Big(1 - \tilde f^{\{i_1, \ldots, i_n\}} \Big) \left(\dd u + \sum_i a_i \mu_i^2 \dd \phi_i \right)^2.
	\end{aligned}
\end{equation} 
One recognizes the general form of the $d$-dimensional metric with $n$ angular momenta~\cite{Myers:1986:BlackHolesHigher}.

Let's quote the metric in Boyer--Lindquist coordinates (omitting the indices on $\tilde f$)~\cite{Myers:1986:BlackHolesHigher}
\begin{equation}
	\label{higher-jna:metric:rotating:result-jna-bl}
	\dd s^2 = - \dd t^2
		+ (1 - \tilde f) \left(\dd t - \sum_i a_i \mu_i^2 \dd \phi_i \right)^2
		+ \frac{r^2 \rho^2}{\Delta}\, \dd r^2
		+ \sum_i (r^2 + a_i^2) \Big(\dd \mu_i^2 + \mu_i^2\, \dd \phi_i^2 \Big)
\end{equation} 
which is obtained from the transformation
\begin{equation}
	\dd u = \dd t - g\, \dd r, \qquad
	\dd \phi_i = \dd \phi'_i - h_i\, \dd r
\end{equation} 
with functions
\begin{equation}
\label{higher-jna:change:rotating:higher-dim-func-gh}
	g = \frac{\Pi}{\Delta}
		= \frac{1}{1 - F (1 - \tilde f)}, \qquad
	h_i = \frac{\Pi}{\Delta} \, \frac{a_i}{r^2 + a_i^2},
\end{equation}
and where the various quantities involved are (see \cref{app:coord:general-d:oblate-cosines})
\begin{equation}
	\label{higher-jna:metric:rotating:result-jna-bl-parameters}
	\begin{gathered}
		\Pi = \prod_i (r^2 + a_i^2), \qquad
		F = 1 - \sum_i \frac{a_i^2 \mu_i^2}{r^2 + a_i^2} = r^2 \sum_i \frac{\mu_i^2}{r^2 + a_i^2}, \\
		r^2 \rho^2 = \Pi F, \qquad
		\Delta = \tilde f\, r^2 \rho^2 + \Pi (1 - F).
	\end{gathered}
\end{equation}

Before ending this section, we comment the case of even dimensions: the term $\varepsilon'\, r^2 \dd \alpha^2$ is complexified as $\varepsilon'\, r_{i_1}^2 \dd \alpha^2$, since it contributes to the sum
\begin{equation}
	\sum_i \mu_i^2 + \alpha^2 = 1.
\end{equation} 
This can be seen more clearly by defining $\mu_{n+1} = \alpha$ (we can also define $\phi_{n+1} = 0$), in which case the index $i$ runs from $1$ to $n+\varepsilon$, and all the previous computations are still valid.


\subsection{Examples in various dimensions}
\label{sec:higher-jna:examples}


\subsubsection{Flat space}


A first and trivial example is to take $f = 1$.
In this case one recovers Minkowski metric in spheroidal coordinates with direction cosines (\cref{app:coord:general-d:oblate-cosines})
\begin{equation}
	\dd s^2 = - \dd t^2 + F\, \dd \bar r^2 + \sum_i (\bar r^2 + a_i^2) \Big(\dd \bar \mu_i^2 + \bar \mu_i^2\, \dd \bar \phi_i^2 \Big) + \varepsilon'\, r^2 \dd \alpha^2.
\end{equation}
In this case the JN algorithm is equivalent to a (true) change of coordinates and there is no intrinsic rotation.
The presence of a non-trivial function $f$ then deforms the algorithm.


\subsubsection{Myers--Perry black hole with one angular momentum}


The derivation of the Myers--Perry metric with one non-vanishing angular momentum has been found by Xu~\cite{Xu:1988:ExactSolutionsEinstein}.

The transformation is taken to be in the first plane
\begin{equation}
	r = r' - i a \sqrt{1 - \mu^2}
\end{equation} 
where $\mu \equiv \mu_1$.
The transformation to the mixed spherical--spheroidal system (\cref{app:coord:general-d:oblate-spherical} is obtained by setting
\begin{equation}
	\mu = \sin \theta, \qquad
	\phi_1 = \phi.
\end{equation} 
In these coordinates the transformation reads
\begin{equation}
	r = r' - i a \cos \theta.
\end{equation} 
We will use the quantity
\begin{equation}
	\rho^2 = r^2 + a^2 (1 - \mu^2)
		= r^2 + a^2 \cos^2 \theta.
\end{equation} 

The Schwarzschild--Tangherlini metric is~\cite{Tangherlini:1963:SchwarzschildFieldDimensions}
\begin{equation}
	\dd s^2 = - f\, \dd t^2 + f^{-1}\, \dd r^2 + r^2\, \dd \Omega_{d-2}^2, \qquad
	f = 1 - \frac{m}{r^{d-3}}.
\end{equation} 

Applying the previous transformation results in
\begin{equation}
\begin{aligned}
	\dd s^2 &= (1 - \tilde f)\, \Big(\dd u - a \mu^2\, \dd \phi \Big)^2
		- \dd u\, (\dd u + 2 \dd r)
		+ 2 a \mu^2\, \dd r \dd \phi \\
		&\qquad+ \big( r^2 + a^2 \big) (\dd\mu^2 + \mu^2 \dd\phi^2)
		+ \sum_{i \neq 1} r^2 \big( \dd \mu_i^2 + \mu_i^2\, \dd \phi_i^2 \big).
\end{aligned}
\end{equation}
where $f$ has been complexified as
\begin{equation}
	\tilde f = 1 - \frac{m}{\rho^2 r^{d-5}}.
\end{equation} 

In the mixed coordinate system one has~\cite{Xu:1988:ExactSolutionsEinstein, Aliev:2006:RotatingBlackHoles}
\begin{equation}
	\begin{aligned}
		\dd s^2 = &- \tilde f\, \dd t^2
			+ 2 a (1 - \tilde f) \sin^2 \theta\, \dd t \dd\phi
			+ \frac{r^{d-3} \rho^2}{\Delta}\, \dd r^2 + \rho^2 \dd\theta^2 \\
			&+ \frac{\Sigma^2}{\rho^2}\, \sin^2 \theta\, \dd\phi^2
			+ r^2 \cos^2 \theta^2\, \dd\Omega_{d-4}^2.
	\end{aligned}
\end{equation} 
where we defined as usual
\begin{equation}
	\Delta = \tilde f \rho^2 + a^2 \sin^2 \theta, \qquad
	\frac{\Sigma^2}{\rho^2} = r^2 + a^2 + a g_{t\phi}.
\end{equation} 
This last expression explains why the transformation is straightforward with one angular momentum: the transformation is exactly the one for $d = 4$ and the extraneous dimensions are just spectators.

We have not been able to generalize this result for several non-vanishing momenta for $d \ge 6$, even for the case with equal momenta .


\subsubsection{Five-dimensional Myers--Perry}
\label{sec:higher-jna:examples:myers-perry-5d}


We take a new look at the five-dimensional Myers--Perry solution in order to derive it in spheroidal coordinates because it is instructive.

The function
\begin{equation}
	1 - f = \frac{m}{r^2}
\end{equation} 
is first complexified as
\begin{equation}
	1 - \tilde f^{\{1\}} = \frac{m}{\abs{r_1}^2}
		= \frac{m}{r^2 + a^2 (1 - \mu^2)}
\end{equation}
and then as 
\begin{equation}
	1 - \tilde f^{\{1, 2\}} = \frac{m}{\abs{r_2}^2 + a^2 (1 - \mu^2)}
		= \frac{m}{r^2 + a^2 (1 - \mu^2) + b^2 (1 - \nu^2)}.
\end{equation}
after the two transformations
\begin{equation}
	r_1 = r_1' - i a \sqrt{1 - \mu^2}, \qquad
	r_2 = r_2' - i b \sqrt{1 - \nu^2}.
\end{equation} 
For $\mu = \sin \theta$ and $\nu = \cos \theta$ one recovers the transformations from \cref{sec:higher-jna:5d:myers-perry,sec:higher-jna:5d:bmpv}.

Let's denote the denominator by $\rho^2$ and compute
\begin{align*}
	\frac{\rho^2}{r^2} &= r^2 + a^2 (1 - \mu^2) + b^2 (1 - \nu^2)
		= (\mu^2 + \nu^2) r^2 + \nu^2 a^2 + \mu^2 b^2 \\
		&= \mu^2 (r^2 + b^2) + \nu^2 (r^2 + a^2)
		= (r^2 + b^2) (r^2 + a^2) \left( \frac{\mu^2}{r^2 + a^2} + \frac{\nu^2}{r^2 + b^2} \right).
\end{align*}
and thus
\begin{equation}
	r^2 \rho^2 = \Pi F.
\end{equation} 
Plugging this into $\tilde f^{\{1, 2\}}$ we have~\cite{Myers:1986:BlackHolesHigher}
\begin{equation}
	1 - \tilde f^{\{1, 2\}} = \frac{m r^2}{\Pi F}.
\end{equation} 


\subsubsection{Three dimensions: BTZ black hole}
\label{sec:higher-jna:examples:btz}


As another application we show how to derive the $d = 3$ rotating BTZ black hole from its static version~\cite{Banados:1992:BlackHoleThree}
\begin{equation}
	\dd s^2 = - f\, \dd t^2 + f^{-1}\, \dd r^2 + r^2 \dd\phi^2, \qquad
	f(r) = - M + \frac{r^2}{\ell^2}.
\end{equation} 

In three dimensions the metric on $S^1$ in spherical coordinates is given by
\begin{equation}
	\dd\Omega_1^2 = \dd\phi^2.
\end{equation} 
Introducing the coordinate $\mu$ we can write it in oblate spheroidal coordinates
\begin{equation}
	\dd\Omega_1^2 = \dd\mu^2 + \mu^2 \dd\phi^2
\end{equation} 
with the constraint
\begin{equation}
	\mu^2 = 1.
\end{equation} 

Application of the transformation
\begin{equation}
	u = u' + i a \sqrt{1 - \mu^2}, \qquad
	r = r' - i a \sqrt{1 - \mu^2}
\end{equation} 
gives from \eqref{higher-jna:metric:rotating:result-jna-ur}
\begin{equation}
	\begin{aligned}
		\dd s^2 = &- \dd u^2 - 2 \dd u \dd r
			+ (r^2 + a^2) (\dd \mu^2 + \mu^2 \dd \phi^2)
			- 2 a \mu^2 \, \dd r \dd \phi \\
			&+ (1 - \tilde f) (\dd u + a \mu^2 \dd \phi )^2.
	\end{aligned}
\end{equation} 
The transformation of $f$ is
\begin{equation}
	\tilde f = - m + \frac{\rho^2}{\ell^2}, \qquad
	\rho^2 = r^2 + a^2 (1 - \mu^2).
\end{equation} 

The transformation \eqref{higher-jna:change:rotating:higher-dim-func-gh}
\begin{equation}
	g = \frac{\rho^2 (1 - \tilde f)}{\Delta}, \qquad
	h = \frac{a}{\Delta}, \qquad
	\Delta = r^2 + a^2 + (\tilde f - 1) \rho^2
\end{equation}
to Boyer--Lindquist coordinates leads to the metric \eqref{higher-jna:metric:rotating:result-jna-bl}
\begin{equation}
	\dd s^2 = - \dd t^2
		+ (1 - \tilde f) (\dd t + a \mu^2 \dd \phi )^2
		+ \frac{\rho^2}{\Delta}\, \dd r^2
		+ (r^2 + a^2) (\dd \mu^2 + \mu^2\, \dd \phi^2 ).
\end{equation} 

Finally the constraint $\mu^2 = 1$ can be used to remove the $\mu$.
In this case one finds
\begin{equation}
	\rho^2 = r^2, \qquad
	\Delta = a^2 + \tilde f r^2
\end{equation}
and the metric simplifies to
\begin{equation}
	\dd s^2 = - \dd t^2
		+ (1 - \tilde f) (\dd t + a \dd \phi )^2
		+ \frac{r^2}{a^2 + r^2 \tilde f}\, \dd r^2
		+ (r^2 + a^2) \dd \phi^2.
\end{equation} 

We define the function
\begin{equation}
	N^2 = \tilde f + \frac{a^2}{r^2} = - M + \frac{r^2}{\ell^2} + \frac{a^2}{r^2}.
\end{equation} 
Then redefining the time variable as~\cite{Kim:1997:NotesSpinningAdS3, Kim:1999:SpinningBTZBlack}
\begin{equation}
	t = t' - a \phi
\end{equation} 
we get (omitting the prime)
\begin{equation}
	\dd s^2 = - N^2 \dd t^2 + N^{-2}\, \dd r^2 + r^2 (N^\phi \dd t + \dd \phi)^2
\end{equation} 
with the angular shift
\begin{equation}
	N^\phi(r) = \frac{a}{r^2}.
\end{equation} 
This is the solution given in~\cite{Banados:1992:BlackHoleThree} with $J = -2a$.

It has already been showed by Kim that the rotating BTZ black hole can be derived through the JN algorithm in a different settings~\cite{Kim:1997:NotesSpinningAdS3, Kim:1999:SpinningBTZBlack}: he views the $d = 3$ solution as the slice $\theta = \pi/2$ of the $d = 4$ solution.
Obviously this is equivalent to our approach: we have seen that $\mu = \sin \theta$ in $d = 4$ (\cref{app:coord:4d}), and the constraint $\mu^2 = 1$ is solved by $\theta = \pi/2$.
Nonetheless our approach is more direct since the result just follows from a suitable choice of coordinates and there are no need for advanced justification.

Starting from the charged BTZ black hole
\begin{equation}
       f(r) = - M + \frac{r^2}{\ell^2} - Q^2 \ln r^2, \qquad
       A = - \frac{Q}{2}\, \ln r^2,
\end{equation} 
it is not possible to find the charged rotating BTZ black hole from~\cites{Clement:1993:ClassicalSolutionsThreedimensional, Clement:1996:SpinningChargedBTZ}[sec.~4.2]{Martinez:2000:ChargedRotatingBlack}: the solution solves Einstein equations, but not the Maxwell ones.
This has been already remarked using another technique in~\cite[app.~B]{Lambert:2014:ConformalSymmetriesGravity}.
It may be possible that a more general ansatz is necessary, following \cref{sec:general} but in $d = 3$.



\section*{Acknowledgments}


I am particularly grateful and indebted to Lucien Heurtier for our collaboration and our many discussions on this project.
I thank also Nick Halmagyi and Dietmar Klemm for interesting discussions, and I am grateful to the latter and Marco Rabbiosi for allowing me to reproduce an unpublished example of application.
Finally I wish to thank the members of the Harish--Chandra Research Institute (Allahabad, India) for organizing the set of lectures that helped me to transform my thesis in the current review.


\appendix


\section{Coordinate systems}
\label{app:coord}

This appendix is partly based on~\cite{Tangherlini:1963:SchwarzschildFieldDimensions, Myers:1986:BlackHolesHigher, Gibbons:2005:GeneralKerrdeSitter}.
We present formulas for any dimension before summarizing them for $4$ and $5$ dimensions.


\subsection{\texorpdfstring{$d$}{d}-dimensional}
\label{app:coord:general-d}


Let's consider $d = N + 1$ dimensional Minkowski space whose metric is denoted by
\begin{equation}
	\dd s^2 = \eta_{\mu\nu}\; \dd x^\mu \dd x^\nu, \qquad
	\mu = 0, \ldots, N.
\end{equation} 
In all the following coordinates systems the time direction can separated from the spatial (positive definite) metric as
\begin{equation}
	\dd s^2 = - \dd t^2 + \dd \Sigma^2, \qquad
	\dd \Sigma^2 = \gamma_{ab}\; \dd x^a \dd x^b, \qquad
	a = 1, \ldots, N,
\end{equation} 
where $x^0 = t$.

One defines by $n$ the number of independent $2$-planes of rotation
\begin{equation}
	n = \floor{\frac{N}{2}}
\end{equation} 
such that
\begin{equation}
	\label{coord:eq:d-dim-epsilon}
	d + \varepsilon = 2n + 2, \qquad
	N + \varepsilon = 2n + 1, \qquad
	\varepsilon' = 1 - \varepsilon
\end{equation} 
where
\begin{equation}
	\varepsilon = \frac{1}{2} (1 - (-1)^d ) =
	\begin{cases}
		0 & \text{$d$ even (or $N$ odd)} \\
		1 & \text{$d$ odd (or $N$ even)},
	\end{cases}
\end{equation} 
and conversely for $\varepsilon'$.


\subsubsection{Cartesian system}


The usual Cartesian metric is
\begin{equation}
	\dd \Sigma^2 = \delta_{ab} \dd x^a \dd x^b
		= \dd x^a \dd x^a
		= \dd \vec x^2.
\end{equation} 


\subsubsection{Spherical}


Introducing a radial coordinate $r$, the flat space metric can be written as a $(N-1)$-sphere of radius $r$
\begin{equation}
	\label{coord:metric:flat-d:spherical}
	\dd \Sigma^2 = \dd r^2 + r^2 \dd \Omega_{N-1}^2.
\end{equation} 
The term $\dd \Omega_{N-1}^2$ corresponds the metric on the unit $(N-1)$-sphere $S^{N-1}$, which is parame\-trized by $(N-1)$ angles $\theta_i$ and is defined recursively as
\begin{equation}
	\dd \Omega_{N-1}^2 = \dd \theta_{N-1}^2 + \sin^2 \theta_{N-1} \; \dd \Omega_{N-2}^2.
\end{equation} 

This surface can be embedded in $N$-dimensional flat space with coordinates $X^a$ constrained by
\begin{equation}
	\label{coord:eq:spherical-embedding}
	X^a X^a = 1.
\end{equation} 


\subsubsection{Spherical with direction cosines}


In $d$-dimensions there are $n$ orthogonal $2$-planes,\footnotemark{} thus we can pair $2n$ of the embedding coordinates $X^a$ \eqref{coord:eq:spherical-embedding} as $(X_i, Y_i)$ which are parametrized as%
\footnotetext{%
	Note that this is linked to the fact that the little group of massive representation in $D$ dimension is $\group{SO}(N)$, which possess $n$ Casimir invariants~\cite{Myers:1986:BlackHolesHigher}.
}
\begin{equation}
	X_i + i Y_i = \mu_i \e^{i\phi_i}, \qquad
	i = 1, \ldots n.
\end{equation} 
For $d$ even there is an extra unpaired coordinate that is taken to be
\begin{equation}
	X^N = \alpha.
\end{equation}

Each pair parametrizes a $2$-sphere of radius $\mu_i$.
The $\mu_i$ are called the \emph{direction cosines} and satisfy
\begin{equation}
	\sum_i \mu_i^2 + \varepsilon' \alpha^2 = 1
\end{equation} 
since there is one superfluous coordinate from the embedding.
Finally the metric is
\begin{equation}
	\dd \Omega_{N-1}^2 = \sum_i \Big(\dd \mu_i^2 + \mu_i^2\; \dd \phi_i^2 \Big) + \varepsilon'\, \dd \alpha^2.
\end{equation} 

The interest of these coordinates is that all rotational directions are symmetric.


\subsubsection{Spheroidal with direction cosines}
\label{app:coord:general-d:oblate-cosines}

From the previous system we can define the spheroidal $(\bar r, \bar\mu_i, \bar\phi_i)$ system – adapted when some of the $2$-spheres are deformed to ellipses – by introducing parameters $a_i$ such that (for $d$ odd)
\begin{equation}
	\label{coord:eq:spherical-to-oblate-mu}
	r^2 \mu_i^2 = (\bar r^2 + a_i^2) \bar \mu_i^2, \qquad
	\sum_i \bar \mu_i^2 = 1.
\end{equation} 
This last condition implies that
\begin{equation}
	r^2 = \sum_i (\bar r^2 + a_i^2) \bar \mu_i^2
		= \bar r^2 + \sum_i a_i^2 \bar \mu_i^2.
\end{equation} 

In these coordinates the metric reads
\begin{equation}
	\label{coord:metric:flat-d:spheroidal}
	\dd \Sigma^2 = F\; \dd \bar r^2 + \sum_i (\bar r^2 + a_i^2) \Big(\dd \bar \mu_i^2 + \bar \mu_i^2\; \dd \bar \phi_i^2 \Big) + \varepsilon'\, r^2 \dd \alpha^2
\end{equation} 
and we defined
\begin{equation}
	\label{coord:eq:flat-d:spheroidal:F}
	F = 1 - \sum_i \frac{a_i^2 \bar \mu_i^2}{\bar r^2 + a_i^2} = \sum_i \frac{\bar r^2 \bar \mu_i^2}{\bar r^2 + a_i^2}.
\end{equation} 

Here the $a_i$ are just introduced as parameters in the transformation, but in the main text they are interpreted as "true" rotation parameters, i.e.
angular momenta (per unit of mass) of a black hole.
They all appear on the same footing.

Another quantity of interest is
\begin{equation}
	\label{coord:eq:flat-d:spheroidal:Pi}
	\Pi = \prod_i (\bar r^2 + a_i^2).
\end{equation} 


\subsubsection{Mixed spherical–spheroidal}
\label{app:coord:general-d:oblate-spherical}

We consider the deformation of the spherical metric where one of the $2$-sphere is replaced by an ellipse~\cite[sec.~3]{Aliev:2006:RotatingBlackHoles}.

To shorten the notation let's define
\begin{equation}
	\theta = \theta_{N-1}, \qquad
	\phi = \theta_{N-2}.
\end{equation} 
Doing the change of coordinates
\begin{equation}
	\sin^2 \theta \sin^2 \phi = \cos^2 \theta.
\end{equation}
the metric becomes
\begin{equation}
	\dd \Sigma^2 = \frac{\rho^2}{r^2 + a^2}\, \dd r^2
		+ \rho^2 \dd\theta^2 \\
		+ (r^2 + a^2)\, \sin^2 \theta\, \dd\phi^2
		+ r^2 \cos^2 \theta^2\, \dd\Omega_{d-4}^2
\end{equation} 
where as usual
\begin{equation}
	\rho^2 = r^2 + a^2 \cos^2 \theta.
\end{equation} 
Except for the last term one recognizes $4$-dimensional oblate spheroidal coordinates \eqref{coord:metric:4d:spheroidal}.


\subsection{4-dimensional}
\label{app:coord:4d}


In this section one considers
\begin{equation}
	d = 4, \quad
	N = 3, \quad
	n = 1.
\end{equation} 


\subsubsection{Cartesian system}

\begin{equation}
	\dd \Sigma^2 = \dd x^2 + \dd y^2 + \dd z^2.
\end{equation} 


\subsubsection{Spherical}

\begin{subequations}
\begin{gather}
	\dd \Sigma^2 = \dd r^2 + r^2 \dd \Omega^2, \\
	\dd \Omega^2 = \dd \theta^2 + \sin^2 \theta\; \dd \phi^2,
\end{gather}
\end{subequations}
where $\dd \Omega^2 \equiv \dd \Omega_2^2$.


\subsubsection{Spherical with direction cosines}

\begin{subequations}
\begin{gather}
	\dd \Omega^2 = \dd \mu^2 + \mu^2\; \dd \phi^2 + \dd \alpha^2, \\
	\mu^2 + \alpha^2 = 1,
\end{gather}
\end{subequations}
where
\begin{equation}
	x + iy = r \mu\, \e^{i\phi}, \qquad
	z = r \alpha,
\end{equation} 

Using the constraint one can rewrite
\begin{equation}
	\dd \Omega^2 = \frac{1}{1 - \mu^2}\; \dd \mu^2 + \mu^2\; \dd \phi^2.
\end{equation} 
Finally the change of coordinates
\begin{equation}
	\alpha = \cos \theta, \qquad
	\mu = \sin \theta.
\end{equation} 
solves the constraint and gives back the spherical coordinates.


\subsubsection{Spheroidal with direction cosines}

The oblate spheroidal coordinates from the Cartesian ones are~\cite[p.~15]{Visser:2009:KerrSpacetimeBrief}
\begin{equation}
	x + i y = \sqrt{r^2 + a^2}\, \sin \theta\, \e^{i\phi}, \qquad
	z = r \cos\theta,
\end{equation} 
and the metric is
\begin{equation}
	\label{coord:metric:4d:spheroidal}
	\dd \Sigma^2 = \frac{\rho^2}{r^2 + a^2}\; \dd r^2 + \rho^2 \dd\theta^2 + (r^2 + a^2) \sin^2 \theta\; \dd \phi^2, \qquad
	\rho^2 = r^2 + a^2 \cos^2 \theta.
\end{equation} 

In terms of direction cosines one has
\begin{equation}
	\dd \Sigma^2 = \left(1 - \frac{r^2 \mu^2}{r^2 + a^2} \right)\; \dd r^2 + (r^2 + a^2) \Big(\dd \mu^2 + \mu^2\; \dd \phi^2 \Big) + r^2 \dd \alpha^2.
\end{equation} 


\subsection{5-dimensional}
\label{app:coord:5d}


In this section one considers
\begin{equation}
	d = 4, \quad
	N = 3, \quad
	n = 1.
\end{equation} 


\subsubsection{Spherical with direction cosines}

\begin{equation}
	\label{coord:metric:5d:spherical}
	\dd\Omega_3^2 = \dd \mu^2 + \mu^2\, \dd\phi^2 + \dd \nu^2 + \nu^2\, \dd\psi^2, \qquad
	\mu^2 + \nu^2 = 1
\end{equation} 
where for simplicity
\begin{equation}
	\mu = \mu_1, \qquad
	\mu = \mu_2, \qquad
	\phi = \phi_1, \qquad
	\psi = \phi_2.
\end{equation} 


\subsubsection{Hopf coordinates}
\label{app:coord:5d:hopf}

The constraint \eqref{coord:metric:5d:spherical} can be solved by
\begin{equation}
	\mu = \sin \theta, \qquad
	\nu = \cos \theta
\end{equation} 
and this gives the metric in Hopf coordinates
\begin{equation}
	\label{coord:metric:5d:hopf}
	\dd \Omega_3^2 = \dd\theta^2 + \sin^2 \theta\, \dd\phi^2 + \cos^2 \theta\, \dd\psi^2.
\end{equation} 

\section{Review of \texorpdfstring{$N=2$}{N = 2} ungauged supergravity}
\label{app:N=2-sugra}


In order for this review to be self-contained we recall the basic elements of $N = 2$ supergravity without hypermultiplets -- we refer the reader to the standard references for more details~\cite{Freedman:2012:Supergravity, Andrianopoli:1996:GeneralMatterCoupled, Andrianopoli:1997:N2SupergravityN2}.

The gravity multiplet contains the metric and the graviphoton
\begin{equation}
	\{ g_{\mu\nu}, A^0 \}
\end{equation} 
while each of the vector multiplets contains a gauge field and a complex scalar field
\begin{equation}
	\{ A^i, \tau^i \}, \qquad i = 1, \ldots, n_v.
\end{equation} 
The scalar fields $\tau^i$ (the conjugate fields $\conj{(\tau^i)}$ are denoted by $\bar\tau^{\bar\imath}$) parametrize a special Kähler manifold with metric $g_{i\bar\jmath}$.
This manifold is uniquely determined by an holomorphic function called the prepotential $F$.
The latter is better defined using the homogeneous (or projective) coordinates $X^\Lambda$ such that
\begin{equation}
	\tau^i = \frac{X^i}{X^0}.
\end{equation} 
The first derivative of the prepotential with respect to $X^\Lambda$ is denoted by
\begin{equation}
	F_\Lambda = \frac{\pd F}{\pd X^\Lambda}.
\end{equation} 
Finally it makes sense to regroup the gauge fields into one single vector
\begin{equation}
	A^\Lambda = (A^0, A^i).
\end{equation} 

One needs to introduce two more quantities, respectively the Kähler potential and the Kähler connection
\begin{equation}
	K = - \ln i (\bar X^\Lambda F_\Lambda - X^\Lambda \bar F^\Lambda), \qquad
	\mc A_\mu = - \frac{i}{2} (\pd_i K\, \pd_\mu \tau^i - \pd_{\bar\imath} K\, \pd_\mu \bar\tau^{\bar\imath}).
\end{equation} 

The Lagrangian for the theory without gauge group is given by
\begin{equation}
	\mc L = - \frac{R}{2}
		+ g_{i\bar\jmath}(\tau, \bar \tau)\, \pd_\mu \tau^i \pd^\nu \bar\tau^{\bar\imath}
		+ \mc I_{\Lambda\Sigma}(\tau, \bar \tau)\, F^\Lambda_{\mu\nu} F^{\Sigma\,\mu\nu}
		- \mc R_{\Lambda\Sigma}(\tau, \bar \tau)\, F^\Lambda_{\mu\nu} \hodge{F}^{\Sigma\,\mu\nu}
\end{equation} 
where $R$ is the Ricci scalar and $\hodge{F}^\Lambda$ is the Hodge dual of $F^\Lambda$.
The matrix
\begin{equation}
	\mc N = \mc R + i\, \mc I
\end{equation} 
can be expressed in terms of $F$.
From this Lagrangian one can introduce the symplectic dual of $F^\Lambda$
\begin{equation}
	G_\Lambda = \frac{\var \mc L}{\var F^\Lambda} = \mc R_{\Lambda\Sigma} F^\Sigma - \mc I_{\Lambda\Sigma} \hodge{F}^\Sigma.
\end{equation} 

\section{Properties of Block Palindromes}
In this section, we investigate the properties of block palindromes.
We assume that $\T$ is an input string of length $\NN$ in the rest of the paper.

Since there are $O(2^\NN)$ factorization of $\T$ and block palindromes are symmetric, there are $O(2^{\NN/2})$ block palindromes of $\T$.
Moreover, there is a tight example that $\T$ consists of only the same characters.

Although there are a huge number of block palindromes of a string, they are very redundant.
To look for more essential properties of block palindromes, we define
the \emph{largest block palindrome} which is a representative of other block palindromes.
A block palindrome $\ff=\ff_{-\nn} \cdots \ff_{\nn}$ of $\T$ that has
the largest number of blocks among all block palindromes of $\T$ is called the largest block palindrome.
Note that each block $\ff_\ii$ for $0 \leq \ii \leq \nn$ is an unbordered substring and $\ff_\ii$ for $0 < \ii \leq \nn$ is  the shortest border of $\T[\kk + 1 \ldots \NN - \kk]$, where $\kk=0$ if $\ii=\nn$ and $\kk=|\ff_{\ii+1} \cdots \ff_\nn|$ otherwise.
So, the largest block palindrome of $\T$ is unique.
The largest block palindrome is a representative of all block palindromes in the sense that all block palindromes can be represented as block palindromes of $\ff$.

A natural and prompt question would be about how to efficiently compute the largest block palindrome of $\T$.
The following theorem answers this question.

\begin{theorem}\label{theorem:largest}
	The largest block palindrome $\ff_{-\nn} \cdots \ff_{\nn}$ of $\T$ can be computed in $O(\NN)$ time.
\end{theorem}
\begin{proof}
	We construct a data structure in $O(\NN)$ time that can answer any LCE query in constant time.

	We greedily compute the blocks from outside $\ff_{\nn}$ to inner $\ff_{1}$ by LCE queries.
	We assume that we compute the shortest border $\ff_{\ii}$ of $\T[\bb \ldots \ee]$.
	For $\kk=1$ to $\floor{(\ee - \bb + 1)/2}$, we check whether $\T[\bb \ldots \bb + \kk - 1]$ is the border of $\T[\bb \ldots \ee]$ or not by checking whether $\LCE(\bb, \ee-\kk+1) \geq \kk$ or not.
	If $\T[\bb \ldots \ee]$ does not have any border, we obtain $\ff_0 = \T[\bb \ldots \ee]$.
	Otherwise, we obtain the shortest border $\ff_\ii=\T[\bb \ldots \bb + \kk-1]$ of $\T[\bb \ldots \ee]$, and compute the more inner blocks for $\T[\bb + \kk \ldots \ee - \kk]$.
	Since the number of LCE queries is $O(\NN)$ and each LCE query takes constant time, the largest block palindrome of $\T$ can be computed in $O(\NN)$ time.
\qed
\end{proof}

So far, we have considered only block palindromes that are equal to $\T$ itself.
Next, we consider block palindromes that appear as substrings in $\T$.
We define a \emph{maximal block palindrome} which is a representative of some block palindromes in $\T$, and study how many maximal block palindromes can appear in $\T$.

For a half-position $1 \leq \cc \leq \NN$ and an integer $1 \leq \dd \leq \NN/2$, let $\FF_\T(\cc, \dd)=\{\ff | \ff=\ff_{-\nn} \cdots \ff_0 \cdots \ff_{\nn} \mbox{ is the largest block palindrome}, \ff_0=\T[\cc - \dd + 1 \ldots \cc + \dd-1], \ff=\T[\cc - \dd - \kk+1 \ldots \cc + \dd + \kk-1], \kk = |\ff_1 \cdots \ff_\nn| \}$ be the set of largest block palindromes whose center positions are the same and whose center blocks appear at $\T[\cc-\dd+1 \ldots \cc+\dd-1]$.
When context is clear, we denote $\FF_{\T}$ by $\FF$.
For a string $\T$, a largest block palindrome $\ff \in \FF(\cc, \dd)$ such that $|\ff|$ is the longest, namely the number of blocks are maximal among all largest block palindromes of $\FF(\cc, \dd)$, is called a \textit{maximal block palindrome}.


We remark that the maximal block palindrome of $\FF(\cc, \dd)$ is a representative of all the largest block palindromes of $\FF(\cc, \dd)$.

\begin{remarkx}
	\label{lem:maximal-palindrome-is-representative}
  For a half-position $1 \leq \cc \leq \NN$ and an integer $1 \leq \dd \leq \NN/2$, any largest block palindrome $\ff=\ff_{-\nn} \cdots \ff_{\nn} \in \FF(\cc, \dd)$ is a sub-factorization of the maximal block palindrome $\vg = \vg_{-\mm} \cdots \vg_{\mm} \in \FF(\cc, \dd)$, that is, $\nn \leq \mm$ and $\ff_{\ii}=\vg_{\ii}$ for $0 \leq \ii \leq \nn$.
\end{remarkx}
\begin{proof}
	We assume that the statement does not hold.
	Let $\ff_\jj$ be a block that $\ff_{\jj} \neq \vg_{\jj}$, and $\jj=0$ or $\ff_{\ii}=\vg_{\ii}$ for $0 \leq \ii < \jj \leq \nn$.
	If $|\ff_{\jj}| < |\vg_{\jj}|$, $\ff_{\jj}$ is a border of $\vg_{\jj}$ and it contradicts that $\vg_{\jj}$ is the largest block palindrome.
	We can say the same things for the case $|\ff_{\jj}| > |\vg_{\jj}|$.
	Therefore, such $\ff_\jj$ and $\vg_\jj$ do not exist and this statement holds.
	\qed
\end{proof}

We are interested in how many maximal block palindromes can appear in $\T$.
It is trivially upper bounded by $O(\NN^2)$ since there are $O(\NN^2)$ substrings which can be center substrings.
If there is no limitation on the size of maximal block palindromes, we can easily see that it is tight.
For a string $\T$ of length $\NN$ in which the characters are all distinct, any substring $\ww$ is unbordered, and there is at least one maximal block palindrome that contains $\ww$ as a center block.
Thus, $\T$ can contain $\Theta(\NN^2)$ maximal block palindromes.
The following example says that the number of maximal block palindromes having three blocks has also the same tight upper bound.

\begin{examplex}
	\label{lem:bound-of-num-maximal-palindromes}
	The number of maximal block palindromes in $\T=\mathtt{a}^n\mathtt{b}^n\mathtt{a}\mathtt{b}\mathtt{a}^n\mathtt{b}^n$ that have at least three blocks is $\Theta(\NN^2)$, where $\cc^\nn$ for a character $\cc$ denotes run of $\cc$ of length $\nn$ , and $\nn=(\NN-2)/4$.
\end{examplex}
For convenience, we denote $\T$ by $\T=\vA_0 \vB_1 \vA_1 \vB_2 \vA_2 \vB_3$, where $\vA_0$, $\vB_1$, $\vA_1$, $\vB_2$, $\vA_2$, and $\vB_3$ are strings $\mathtt{a}^n$, $\mathtt{b}^n$, $\mathtt{a}$, $\mathtt{b}$, $\mathtt{a}^n$, and $\mathtt{b}^n$, respectively.
There are maximal block palindromes of size three that, for $1<\ii \leq \nn$, $1<\jj \leq \nn$, $\T[\nn-\jj+1 \ldots \NN-\nn+\ii-1]$=$(\vA_0[\nn-\jj+1 \ldots \nn] \vB_1[1..\ii-1])(\vB_1[\ii \ldots \nn]\vA_1 \vB_2 \vA_2[1 \ldots \jj])(\vA_2[\nn-\jj+1 \ldots \nn]\vB_3[1 \ldots \ii-1])$ and they are unbordered, where the parentheses indicate blocks.


%% If we consider only maximal block palindromes of even size, whose center block must be empty,
%% the number of occurrences of center blocks are at most $\NN-1$,
%% and thus, $\T$ can contain $\Theta(\NN)$ maximal block palindromes of even size.

Remark that the upper bound is reduced to $O(\NN)$ if we impose a limitation on the lengths of center blocks.
\begin{remarkx}
	\label{lem:limited-center-block}
  For any constant $\kk \ge 0$, a string of length $\NN$ can contain $\Theta(\NN)$
  maximal block palindromes whose center blocks are of length $\le \kk$ because there are $O(\NN)$ possible center blocks.
  In particular, a string contains at most $\NN - 1$ maximal block palindromes of even size (i.e., the center blocks must be empty)
  because the number of occurrences of center blocks are at most $\NN-1$.
\end{remarkx}



The following lemma shows an interesting property of maximal block palindromes, and this property can be used for the proof of Lemma~\ref{lem:size-maximal-block-palindrome}.
\begin{lemma}
	\label{lem:there-is-no-same-starting-positions-of-factors}
  For a half-position $1 \leq \cc \leq \NN$ and two integers $1 \leq \dd<\dd^\prime \leq \NN/2$, two largest block palindromes $\ff=\ff_{-\nn} \cdots \ff_{\nn} \in \FF(\cc, \dd)$ and $\vg = \vg_{-\mm} \cdots \vg_{\mm} \in \FF(\cc, \dd^\prime)$ do not share the block boundaries, namely, the ending positions of blocks $\kk_\ii$ and $\kk^\prime_\ii$ such that $\kk_\ii=\cc + \dd -1 + |\ff_1 \cdots \ff_\ii|$ and $\kk^\prime_\ii=\cc + \dd^\prime -1 + |\vg_1 \cdots \vg_\jj|$ do not equal for any $1 \leq \ii \leq \nn$ and $1 \leq \jj \leq \mm$.
\end{lemma}
\begin{proof}
  Similar to Remark~\ref{lem:maximal-palindrome-is-representative}, if we assume that this lemma does not hold, a block of $\ff$ or $\vg$ must have a border and it contradicts that $\ff$ and $\vg$ are the largest block palindromes.
	\qed
\end{proof}

Let $\|\MBP(\T)\|$ denote the sum of the sizes of all maximal block palindromes in $\T$.
\begin{lemma}
  \label{lem:size-maximal-block-palindrome}
  For any string $\T$ of length $\NN$, $\|\MBP(\T)\| \le \NN(2\NN-1)$.
\end{lemma}
\begin{proof}
  From Lemma~\ref{lem:there-is-no-same-starting-positions-of-factors}, any two largest block palindromes, whose center positions
  are same but center blocks are different, do not share the block boundaries.
  This implies that, for a half-position $\cc$, the number of blocks of maximal block palindromes whose center position is $\cc$ is up to $\NN$.
  Since there are $2\NN-1$ center positions, we have $\|\MBP(\T)\| \le \NN(2\NN-1)$.
	\qed
\end{proof}



\printbibliography[heading=bibintoc]


\end{document}

