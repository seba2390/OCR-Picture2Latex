\documentclass[10pt, a4paper]{article}
\pdfoutput=1

\usepackage[utf8]{inputenc}
\usepackage[T1]{fontenc}
\usepackage{lmodern, textcomp}
\usepackage[british]{babel}
\usepackage{csquotes}

\usepackage[heightrounded, top=4cm, bottom=4cm, left=3.5cm, right=3.5cm]{geometry}

\usepackage{eurosym, xspace}
\usepackage[nottoc]{tocbibind}
\usepackage{graphicx, subfig, float}
\usepackage[multiple]{footmisc}
\usepackage{authblk}
\usepackage{fancyhdr}

\usepackage[usenames, dvipsnames]{xcolor}

\usepackage[backend=bibtex8, citestyle=numeric-comp, maxbibnames=99,
			sorting=none, sortcites=true, firstinits=true]{biblatex}
\documentclass[12pt]{article}
\usepackage{amsmath}
\usepackage{graphicx}
\usepackage{enumerate}
\usepackage{natbib}
\usepackage{url} % not crucial - just used below for the URL 
\usepackage{pgf,tikz}  % MIQUEL
\usetikzlibrary{positioning}  % MIQUEL
\usepackage{comment}  % MIQUEL
\usepackage[linesnumbered,ruled,vlined]{algorithm2e}  % MIQUEL

%\pdfminorversion=4
% NOTE: To produce blinded version, replace "0" with "1" below.
\newcommand{\blind}{1}

% DON'T change margins - should be 1 inch all around.
\addtolength{\oddsidemargin}{-.5in}%
\addtolength{\evensidemargin}{-.5in}%
\addtolength{\textwidth}{1in}%
\addtolength{\textheight}{-.3in}%
\addtolength{\topmargin}{-.8in}%

% MIQUEL
\usepackage{amsmath}
\usepackage{amssymb}
\usepackage{bm}
\usepackage{xcolor}
\usepackage[ruled,vlined]{algorithm2e}
\usepackage{xr}

%\externaldocument{draft_jasacs_v9}

\def\cred{\textcolor{red}}
\def\cblue{\textcolor{blue}}
\def\cteal{\textcolor{teal}}

\newcommand{\E}{\mbox{\textup{E}}}
\newcommand{\mb}[1]{\mathbf{#1}}
\newcommand{\tr}[1]{#1^{\intercal}}
\newcommand{\norm}[1]{\lVert#1\rVert}
\newcommand{\cind}{\mathrel{\text{\scalebox{1.07}{$\perp\mkern-10mu\perp$}}}}
\newcommand{\indf}[1]{\textup{I}\left( #1 \right)}
\DeclareMathOperator*{\argmax}{arg\,max}
\DeclareMathOperator*{\argmin}{arg\,min}

\newcommand{\by}{{\mb{y}}}
\newcommand{\bd}{{\mb{d}}}
\newcommand{\bx}{{\mb{x}}}
\newcommand{\bX}{{\mb{X}}}
\newcommand{\bD}{{\mb{D}}}
\newcommand{\bI}{{\mb{I}}}
\newcommand{\bw}{{\mb{w}}}
\newcommand{\balpha}{{\bm{\alpha}}}
\newcommand{\bbeta}{{\bm{\beta}}}
\newcommand{\bgamma}{{\bm{\gamma}}}
\newcommand{\bdelta}{{\bm{\delta}}}
\newcommand{\blambda}{{\bm{\lambda}}}
\newcommand{\bpsi}{{\bm{\psi}}}
\newcommand{\brho}{{\bm{\rho}}}
\newcommand{\btheta}{{\bm{\theta}}}
\newcommand{\bmu}{{\bm{\mu}}}
\newcommand{\bomega}{{\bm{\omega}}}
\newcommand{\bpi}{{\bm{\pi}}}

\newcommand{\bthetay}{\btheta_{\text{y}}}
\newcommand{\bthetaD}{\btheta_{\textsc{d}}}
\newcommand{\hatbthetaD}{\hat{\btheta}_{\textsc{d}}}

\newcommand{\bthetaeb}{\btheta^{\textsc{eb}}}
\newcommand{\bthetaep}{\btheta^{\textsc{ep}}}

\newcommand{\py}{p}
\newcommand{\pdt}{p}
%\newcommand{\py}{p_{\textup{y}}}
%\newcommand{\pdt}{p_{\textup{d}^{t}}}

\newcommand{\hlambeb}{\hat{\lambda}^{\textsc{eb}}}
\newcommand{\hlambep}{\hat{\lambda}^{\textsc{ep}}}
\newcommand{\tlambep}{\tilde{\lambda}^{\textsc{ep}}}
\newcommand{\hthetaeb}{\hat{\btheta}^{\textsc{eb}}}
\newcommand{\hthetaep}{\hat{\btheta}^{\textsc{ep}}}
\newcommand{\tthetaep}{\tilde{\btheta}^{\textsc{ep}}}

\newtheorem{result}{Result}
\newtheorem{thm}{Theorem}
\newtheorem{lemma}[thm]{Lemma}
\newtheorem{corollary}[thm]{Corollary}
\newtheorem{prop}[thm]{Proposition}

% Load supplement file
%\externaldocument{supplement}
% END MIQUEL

\newcommand{\omcom}[1]{ {\color{blue} #1} }
\newcommand{\davidcom}[1]{{\color{red} [DR. #1]} }

\begin{document}

%\bibliographystyle{natbib}

\def\spacingset#1{\renewcommand{\baselinestretch}%
{#1}\small\normalsize} \spacingset{1}


%%%%%%%%%%%%%%%%%%%%%%%%%%%%%%%%%%%%%%%%%%%%%%%%%%%%%%%%%%%%%%%%%%%%%%%%%%%%%%

\if1\blind
{
  \title{ Confounder importance learning for treatment effect inference }
  \author{
Miquel Torrens i Dinar\`{e}s\\
    Department of Economics \& Business, Universitat Pompeu Fabra\\
    and \\
    Omiros Papaspiliopoulos \\
    Department of Decision Sciences, Universit\`{a} Bocconi\\
    and \\
    David Rossell\thanks{
    DR gratefully acknowledges support from Spanish Government grants Europa Excelencia EUR2020-112096, RYC-2015-18544, PGC2018-101643-B-I00.}\hspace{.2cm}\\
    Department of Economics \& Business, Universitat Pompeu Fabra \\ Data Science Center, Barcelona School of Economics
}
  \maketitle
} \fi

\if0\blind
{
  \bigskip
  \bigskip
  \bigskip
  \begin{center}
    {\LARGE\bf  Confounder importance learning for treatment effect inference }
\end{center}
  \medskip
} \fi

\bigskip
\begin{abstract}

We address applied and computational issues for the problem of multiple treatment effect inference under many potential confounders. While there is abundant literature on the harmful effects of omitting relevant controls (under-selection), we show that over-selection can be comparably problematic, introducing substantial variance and a bias related to the non-random over-inclusion controls. We provide a novel empirical Bayes framework to mitigate both under-selection problems in standard high-dimensional methods and over-selection issues in recent proposals, by learning whether each control's inclusion should be encouraged or discouraged. We develop efficient gradient-based and Expectation-Propagation model-fitting algorithms to render the approach practical for a wide class of models. A motivating problem is to estimate the salary gap evolution in recent years in relation to potentially discriminatory characteristics such as gender, race, ethnicity and place of birth. We found that, albeit smaller, some wage differences remained for female and black individuals. A similar trend is observed when analyzing the joint contribution of these factors to deviations from the average salary. Our methodology also showed remarkable estimation robustness to the addition of spurious artificial controls, relative to existing proposals.

%We address the problem of treatment effect inference in the presence of a high-dimensional set of potential confounders, and possibly multiple simultaneous treatments. While a lot of focus has been put on the effects of omitting relevant controls, we show that over-selection problems can become as severe in different contexts.
%problems related to over-selection have been partially overlooked. Here we show that over-selection problems can become as severe as under-selection

%The text of your abstract. 200 or fewer words.
\end{abstract}

\noindent%
{\it Keywords:} multiple treatments, Bayesian model averaging, empirical Bayes, double machine learning, sparsity, variable selection, variational approximation
\vfill

\newpage
\spacingset{1.5} % DON'T change the spacing!

\section{Introduction} \label{sec:intro}

This article addresses a problem of fundamental importance in applied research, that of evaluating the joint effect, if any, of multiple treatments on a response variable while controlling for a large number of covariates, many of which might be correlated with the treatments. We refer to the covariates as \textit{controls}, and when correlated to both the response and the treatments we call them \textit{confounders}.
A motivating application is to study the association between salary and multiple treatments such as gender, race and country of birth, after accounting for differential access to education, position at the work place, economic sector, region in the country, and many other potential controls. Studying which treatments are associated to salary, after accounting for said controls, helps quantify salary variation potentially due to discrimination, and how this evolved over time.

We model the dependence of the response $y_{i} \sim p(y_{i}; \eta_{i}, \phi)$ on $t=1,\ldots,T$ treatments $d_{i,t}$ and $j=1,\ldots,J$ controls $x_{i,j}$, via 
\begin{align}
\eta_{i} = \sum_{t=1}^{T} \alpha_{t} d_{i,t} + \sum_{j=1}^{J} \beta_{j} x_{i,j}, i=1,\ldots, n \label{eq:y_eq}
\end{align}
where $p(y_i; \eta_i, \phi)$ defines a generalized linear model with linear predictor $\eta_i$ and dispersion parameter $\phi$.
The distinction between treatments and controls is that we are primarily interested in inference for the former, i.e, for the set of $\alpha_t$'s in \eqref{eq:y_eq}, whereas the latter are included in the model to avoid omitted variable biases. When the controls are plenty, a shrinkage and selection estimation framework is required, either in terms of penalized likelihood or Bayesian model averaging (BMA). 
%However, these frameworks are geared towards prediction and consequently 
However, a key observation is that their naive application can yield imprecise estimates, especially when there are many confounders and/or the treatment effects are not very strong.
%, see for example Figure \ref{fig:intro2} later in this section. For example, 
For example, Figure \ref{fig:intro2} (described later in this section) shows that LASSO and BMA suffer from very high root mean squared error (RMSE) when all controls affecting the response $(\beta_j \neq 0)$ are confounders.
A further issue for LASSO and other likelihood penalties is that the treatment variables might not be selected, in which case the powerful post-selection inference framework of \cite{lee_jason:2016}, %implemented in the \texttt{R} package \texttt{selectiveInference}, 
is not applicable.
An alternative is to use methods based on de-biasing the LASSO \citep{vdgeer14, javanmard14}, which provide inference for all parameters, however these incur a loss of shrinkage that typically results in low power to detect anything but strong effects.
For example, in linear regression with orthogonal design matrix the debiased LASSO of \cite{vdgeer14} recovers the least-squares estimate, i.e. offers no shrinkage.

%\davidcom{I shuffled the order of Fig 1 and 2, to match the order in which they are presented in the text.}

To address this issue a number of penalized likelihood and Bayesian methods have been introduced. Section \ref{sec:other_methods} provides an overview. A popular one is the Double Machine Learning (DML) approach of \cite{Belloni14b}. %, which at a second stage fits a model like \eqref{eq:y_eq} by using OLS on a set of controls that have been selected at a first stage by, say, separate LASSO algorithms  that regress the response $y$  but also the treatments $d_t$ on the controls. 
In a first step, DML regresses separately the outcome and the treatment on the controls via penalized likelihood, and in a second step it fits a model like \eqref{eq:y_eq} via maximum likelihood on the set of controls selected in either regression in the first stage.
In a similar spirit, Bayesian Adjustment for Confounding (BAC)  in \cite{Wang12} models jointly the response and the treatment and uses a prior that encourages controls to be simultaneously selected in the two regression models. 
The key idea behind these approaches and related work is to encourage the inclusion of covariates that are associated with the treatments to the regression model for the response. %, as we detail in Section \ref{sec:other_methods}).

In this article we highlight an overlooked weakness that is relevant for the application of such methods, % for treatment effect estimation in presence of many controls, 
which relates to over-selection of controls. Adding controls to the regression model for $y_{i}$ that are correlated to the treatments, but are not conditionally related to the outcome, has two effects. The first one is an obvious \emph{over-selection variance}, an inflation of the standard errors of the treatment effects that leads to a reduced power to detect weaker effects. The second one is more subtle, and we call it \emph{control over-selection bias}. The inclusion of controls in \eqref{eq:y_eq} which were screened out to be correlated with the treatments leads to biased inference for the treatment effects. % that basic LASSO and Bayesian model averaging also suffer from, as we discussed above. 
Said over-selection bias and variance worsen as the number of treatments increases and as the level of confounding (the proportion of controls that are relevant both for the response and the treatments) decreases. In Figure \ref{fig:intro2} we consider a single treatment  simulated  according to linear regression on 6 active controls, and we varied the number of controls active in both models from 0 (no confounding) to 6  (full confounding, i.e., the same controls are used for generating the response and the treatment). While LASSO and BMA perform worse the stronger the confounding due to control under-selection, DML and BAC perform well in the presence of strong confounding but poorly in the lack of it. % is little overlap in the sets of controls that are active for response and treatment. On the other hand, CIL achieves good performance for the whole spectrum. 
In Figure \ref{fig:multitreat_uncounfounded} (described later) we show that these effects can be exacerbated in the presence of an increasing number of treatments.

We propose a new approach, Confounder Importance Learning (CIL), which can be easily implemented using existing software, that deals successfully with both over- and under-selection, in both high and low confounding situations, and in the presence of multiple simultaneous treatments. A first illustration of the merits of CIL is given in Figure \ref{fig:intro2}, where it achieves good performance across the spectrum. %Figures \ref{fig:intro1} and \ref{fig:intro2}. 
\begin{figure}[htbp]
\centering
\begin{tabular}{ccc}
$\alpha = 1$ & $\alpha = 1/3$ & $\alpha = 0$ \\
\includegraphics[scale=0.66]{fig1Ap1.pdf} &
\includegraphics[scale=0.66]{fig1Ap2.pdf} &
\includegraphics[scale=0.66]{fig1Ap3.pdf}
\end{tabular}
\caption{Parameter RMSE relative to an oracle OLS,  for a single treatment effect ($T=1$)  averaged over 250 simulated datasets, considering strong ($\alpha=1$), weak $(\alpha=1/3)$ and  no effect $(\alpha=0)$.  In all panels, $n=100$, $J=49$ and the response and treatment are simulated from a linear regression model based on 6 active controls each. The overlap between the two sets of active controls varies from 0 (no confounding) to 6 (full confounding). DML is double machine learning, BMA is Bayesian model averaging, BAC is Bayesian Adjustment for Confounding and CIL is confounder importance learning.} %\omcom{I have asked this before:  remove EP from label. }}
%, i.e., the same controls are used for generating both the response and the treatment). }%,  \omcom{Om2M: remove EP from label, remove LASSO from DML, remove inf from BAC, remove NLP from BMA and remove PCR since it is too early to also talk about this in the intro. we provide mode detail later}}
\label{fig:intro2}
\end{figure}

In Figure \ref{fig:intro1} we analyze data from the Current Population Survey, which has $T=204$ treatments and $J=278$ controls (see Section \ref{sec:cps}). % on the data, the methods and the conclusions from this analysis.
The former include four average treatment effects of particular interest, Figure  \ref{fig:intro1} focuses on two of them, a gender and a black race indicators, %the sex of the individual being female and their race being black, 
and compares a number of methods.  All methods show that the average log-salary is reduced for women, although this gap is less pronounced in 2019 relative to 2010. However, the methods differ in their conclusions for the black race. To understand better what drives these differences, we added 100 and 200 simulated controls that are dependent on the treatments but conditionally independent of the response. The figure shows a marked robustness of CIL to the addition of said controls, whereas other methods lose their ability to detect the weaker effects (e.g. gender in 2019 and race in 2010). %and its power to estimate weaker but existent effects. 
%Figure \ref{fig:intro2} gives further insights into the over-selection bias and variance, and the ability of CIL to perform well in a range of scenarios. Here we consider response and a single treatment simulated according to linear regressions on 6 active controls, but where we vary the number of controls active in both models from 0 (no confounding) to 6  (full confounding, i.e., the same controls are used for generating both the response and the treatment). As argued earlier, LASSO and basic Bayesian model averaging that do not account for the difference between treatments and controls perform the worse the stronger the confounding, which is due to under-selection. On the other hand, tailored existing methods, such as double machine learning and Bayesian adjustment for confounding, perform well in presence of strong confounding but poorly when there is little overlap in the sets of controls that are active for response and treatment. On the other hand, CIL achieves good performance for the whole spectrum. 

\begin{figure}[htbp]
\centering
%\includegraphics[scale=0.5]{figApp1}
\begin{tabular}{cc}
\includegraphics[width=0.48\textwidth]{figApp1top.pdf} &
\includegraphics[width=0.48\textwidth]{figApp1bot.pdf}
\end{tabular}
\caption{Inference for treatments ``female'' (left) and ``black'' (right) in 2010 and 2019; see Section \ref{sec:cps}. We analyze Current Population Survey data with $J=278$ controls (left black point and bar in each panel) but also adding 100 (middle) and 200 (right) artificial controls correlated with the treatments and conditionally independent of  the response. Names of methods as in the caption of Figure \ref{fig:intro2}.}
\label{fig:intro1}
\end{figure}

The remainder of the introduction outlines our main ideas and the paper's structure.
Section \ref{sec:method} details our proposed approach and provides a deeper discussion of related literature. Our main contribution is a BMA framework where the prior inclusion probabilities, $\pi_j = P(\beta_j \neq 0)$, vary with $j$ in a manner that is learned from data. We build a model 
\begin{align}
\pi_{j}(\btheta) = \rho  + (1 - 2 \rho) \left( 1 + \exp \left\{ - \theta_{0} - \sum_{t=1}^{T} \theta_{t} f_{j,t} \right\} \right)^{-1} \label{eq:cilprior}
\end{align}
which expresses these probabilities in terms of \emph{features} $f_{j,t} \geq 0$ extracted from the treatment and control data, and hyperparameters $\rho \in (0, 1/2)$ and  $\btheta = (\theta_{0}, \theta_{1}, \dots, \theta_{T})$. The role of $\rho$ is to bound the probabilities away from 0 and 1 and, as we discuss in Section \ref{sec:model},  we propose the default choice $\rho = 1/(1+J^2)$. Our method relies on a good choice of the features $f_{j,t}$ %, one for each control and each treatment, 
and the hyperparameters $\theta_t$. %, one for each treatment variable. 
Section \ref{sec:model} describes high-dimensional approaches to obtain the former, and the main idea is to obtain rough estimates of the relative impact of each control to predict each treatment. Regarding the latter, when $\theta_t=0$ for $t=1,\ldots,T$ our approach assigns equal inclusion probabilities, when $\theta_t>0$ controls found to predict treatment $t$ are encouraged to be included in the response model, and discouraged when  $\theta_t<0$. 
This is in contrast to methods such as DML and BAC that encourage the inclusion of any control associated with any treatment, i.e. with large $f_{j,t}$.  

Section \ref{sec:comput} describes our computational methods, which are another main contribution. \color{black}
We design an empirical Bayes choice for $\btheta$ based on optimizing the marginal likelihood % using stochastic gradient descent. %, as shown in Section \ref{sec:ml}, but 
and a much cheaper alternative based on an expectation-propagation approximation. % variational approximation, % developed in Section \ref{sec:ep}, 
%which we used throughout the article.
%Under suitable conditions, there are consistency results available for BMS when hyperparameters such as $\btheta$ are estimated via empirical Bayes. Given the application-oriented nature of this paper we refer the interested reader to \cite{petrone:2014}.
In summary, our approach is based on \emph{learning} the importance of each possible confounder via \eqref{eq:cilprior}. %, while treating differentially treatments and controls. 
In doing so, it helps prevent both under-selection bias and over-selection bias and variance.
%Additionally, our approach scales well with the number of treatments, as it is not based on jointly modelling of treatments and response. 

Section \ref{sec:cps} applies the methodology to our motivating salary application. As discussed, our CIL detects negative salary gaps associated to the black race that might go otherwise unnoticed. 
We also illustrate how considering multiple treatments allows to portray, via posterior predictive inference, a measure of joint salary variation due to potentially discriminatory factors. Our results suggest that in 2019 said variation decreased nation-wide (from 5.4\% in 2010 to 1.5\% in 2019) and state-wide, with lesser disparities across states than in 2010.
%We also illustrate how, via posterior predictive inference, the framework can characterize functionals such as the average salary gap associated to joint variation in multiple treatments. For example, in 2010 the average salary gap associated to variation in the discriminatory treatments was 5.4\%, whereas in 2019 it was 1.5\%.
Finally, it also shows simulations that help understand practical differences between methods, in particular in terms of the amount of confounding present in the data.
All proofs and additional empirical results are in the supplementary material.
%\davidcom{Miquel, please prepare the R files so that we can submit them with the paper, and comply with JASA's blinded requirement. The unblinded version can link to GitHub}.
%THE LINK IS: https://github.com/mtorrens/rcil
R code to reproduce our results is available at \url{https://github.com/mtorrens/rcil}.

%\omcom{Om2D: here a paragraph summarizing the main findings from our analysis of salary data. I think for an applied oriented paper we should at least give some priority to this. maybe here even show the figure on posterior predictive}

%%%%%%%%%%%%%%%%%%%%%%%%%%%%%%%%%%%%%%%%%%%%%%%%%%%%%%%%%%%%%%%%%%%%%%
\section{Modelling Framework} \label{sec:method}

\subsection{Sparse Treatment and Control Selection}
\label{sec:model}

We model the dependence of the response $y_{i}$ on treatments $\mb{d}_{i} = (d_{i,1}, \ldots, d_{i,T})$ and controls $\mb{x}_{i} = (x_{i,1}, \dots, x_{i,J})$, according to \eqref{eq:y_eq}. %, where $\phi >0$ is a dispersion parameter. 
We are primarily interested in inference for $\balpha = (\alpha_{1}, \dots, \alpha_{T})$, i.e.  the \textit{treatment effects}. We call \textit{single treatment} to the special case $T=1$, while for $T>1$ we have  \textit{multiple treatments}.
%For concreteness, we focus on the normal linear model $y_{i} \sim \text{N}(eta_{i}, \phi)$, where $\phi$ represents the error variance. We postpone discussion of other choices to Section \ref{sec:discuss}.

We adopt a Bayesian framework where we   introduce variable inclusion indicators   $\gamma_{j} = \text{I}(\beta_{j} \neq 0)$ and $\delta_{t} = \text{I}(\alpha_{t} \neq 0)$, and define a model prior 
\begin{align}
p(\balpha, \bbeta, \bdelta, \bgamma, \phi \mid \btheta) = p(\balpha, \bbeta \mid \bdelta, \bgamma, \phi) p(\bgamma \mid \btheta) p(\bdelta) p(\phi), \label{eq:probmodel}
\end{align}
where $\btheta $ are the hyper-parameters in \eqref{eq:cilprior}, and $p(\phi)$ is dropped for models with known dispersion parameter (e.g. logistic or Poisson regression). For the regression coefficients, we assume prior independence, %Figure \ref{fig:dag1} includes a graphical model detailing the relation between the different variables and parameters in the model. For the linear coefficients, we assume prior independence,
\begin{align*}
p(\balpha, \bbeta \mid \bdelta, \bgamma, \phi) := \prod_{t=1}^{T} p(\alpha_{t} \mid \delta_{t}, \phi) \prod_{j=1}^{J} p(\beta_{j} \mid \gamma_{j}, \phi),
\end{align*}
and adopt the so-called product moment (pMOM) non-local prior of \cite{Johnson12}, according to which $\alpha_{t} = 0$ almost surely if $\delta_{t} = 0$, and
\begin{align*}
p(\alpha_{t} \mid \delta_{t} = 1, \phi) = \frac{\alpha_{t}^{2}}{\tau \phi} \text{N}(\alpha_{t}; 0, \tau \phi),
\end{align*}
with the analogous setting for every $\beta_{j}$. Figure \ref{fig:nlpmom} illustrates its density.
This prior involves a hyperparameter $\tau > 0$, that we set to $\tau = 0.348$, following \cite{Johnson10}, so that the prior signal-to-noise ratio $|\alpha_t|/\sqrt{\phi}$ is greater than 0.2 with probability 0.99. Non-local priors achieve model selection consistency on a range of high-dimensional linear and generalized linear regression models and play an important role in helping discard spurious predictors \citep{Johnson12, wu2016nonlocal, Shin18, Rossell21}. As for the dispersion parameter, where unknown,  we place a standard $\phi \sim \text{IGam}(a_{\phi} = 0.01, b_{\phi} = 0.01)$ prior, %commonly used for the Gaussian linear model, 
see e.g. \cite{Gelman06}.

For the inclusion indicators, we also assume prior independence, and set
\begin{eqnarray}
p(\bdelta) &=& \prod_{t=1}^{T} \text{Bern}(\delta_{t}; 1/2), \label{eq:pdelta} \\
p(\bgamma \mid \btheta) &=& \prod_{j=1}^{J}  \text{Bern}(\gamma_{j}; \pi_{j}(\btheta)). \label{eq:pgamma}
\end{eqnarray}
All treatments get a fixed marginal prior inclusion probability $P(\delta_t=1)=1/2$, as we do not want to favor their exclusion a priori, considering that there is at least some suspicion that any given treatment has an effect. %Such prior treatment inclusion probability leads to a uniform prior on $\bdelta$. 
This choice is a practical default when the number of treatments $T$ is not too large, else one may set $P(\delta_t=1)<1/2$ to avoid false positive inflation due to multiple hypothesis testing \citep{Scott10, Rossell21}.
%\cred{For example, setting $P(\delta_{t} = 1) = 1/?$ leads to prior inclusion probabilities of the same order as the $\text{Beta-Binomial}(1, 1)$, shown by \cite{Scott10} to prevent false positive inflation, see also \cite{Rossell21}.} \cblue{[\textbf{Miquel:} this comment confused me because the marginal for BBin(1,1) is 1/2 as well]} \davidcom{Maybe refer to $P(\delta_t=1)= 1/p$, which I think defines a Complexity prior?} 
Our software allows the user to set any desired $P(\delta_{t} = 1)$.

The main modelling novelty in this article is the choice of $\pi_{j}(\btheta)$, which we set according to \eqref{eq:cilprior}. A key part of the construction is the choice of features $f_{j,t}$. Our generic approach is to take $f_{j,t} = |w_{j,t}|$, where $\bw_t = (w_{1,t},\ldots, w_{J,t})$ are regression coefficients obtained via a high-dimensional regression of $\bd_t$ on the controls. We highlight two possibilities. First, a LASSO regression,
\begin{align}
\bw_{t} := \argmin_{(v_{t,1}, \dots, v_{t,J})} \left\{ \sum_{i=1}^{n} \log p\left( d_{i,t}; \sum_{j=1}^{J} x_{i,j} v_{t,j} \right) + \lambda \sum_{j=1}^{J} |v_{t,j}| \right\}, \label{eq:lasso}
\end{align}
%\mb{f}_{t}^{(\lambda)} := \arg \min_{(v_{t,1}, \dots, v_{t,J}) \in \mathbb{R}^{J}} \left\{ \sum_{n=1}^{N} \log p\left( d_{n,t}; \sum_{j=1}^{J} x_{n,j} v_{t,j} \right) + \lambda \sum_{j=1}^{J} |v_{t,j}| \right\}, \label{eq:lasso}
where $\lambda > 0$ is a regularization parameter, which we set by minimizing the BIC (we obtained similar results when using cross-validation). 
%, intuitively the reason is that values $\pi_j(\btheta)$ are mainly driven by the relative magnitudes of $f_{j,t}$, rather than their exact value. 
The choice in \eqref{eq:lasso} balances speed with reasonable point estimate precision, and is the option that we used in all our examples.
A second option, available when dealing with continuous treatments, is to use the minimum norm ridge regression,
\begin{align}
%\mb{f}_{t} := \tr{\bX} \left( \bX \tr{\bX} \right)^{-1} \bd_{t},
\mb{w}_{t} =  \left( \tr{\bX} \bX \right)^{+} \bd_{t}, \label{eq:pseudoinv}
\end{align}
where $\left( \tr{\bX} \bX \right)^{+}$ is the Moore-Penrose pseudo-inverse, and $\bX$ the $n \times J$ design matrix. For $J<n$ this is the familiar OLS estimator, but \eqref{eq:pseudoinv} is also well-defined when $J>n$, and it has been recently investigated in terms of the so-called benign overfitting property in \cite{Bartlett20}. \cite{Wang16} showed that when $J>n$, \eqref{eq:pseudoinv} ranks the coefficients consistently under theoretical conditions slightly broader than the LASSO.
This is appealing in our context, since $\pi_j(\btheta)$ are mainly driven by the relative magnitudes of $f_{j,t}$, rather than their exact value. 

The scalar $\rho$ in \eqref{eq:cilprior} ensures that prior probabilities are bounded away from 0 and 1. %, and we propose to set it according to general Bayesian variable selection asymptotic theory. 
In particular, we set it to $\rho = 1/(J^2 + 1)$. 
This sets a lower-bound $P(\beta_j \neq 0) \geq 1/(J^2+1)$ that is of the same order as $J$ grows as the Complexity priors in \citep{Castillo12}, which are sufficiently sparse to discard spurious predictors and attain minimax estimation rates \citep{Castillo15,Rossell21}.
%This contraction rate is sufficiently strong to attain to optimal minimax rates based on complexity prior theory \citep{Castillo12, Castillo15} using $c=1$ as the complexity parameter, although \cite{Rossell21} shows that lesser sparsity can in fact improve performance in smaller samples. 

The final element in \eqref{eq:cilprior} are the hyper-parameters $\theta_t$, which can encourage the inclusion or exclusion of controls associated to the treatment $t$. 
Figure \ref{fig:theta1} illustrates $\pi_{j}(\btheta)$ for three different values of $\theta_{1}$. Setting $\btheta$ is critical for the performance of our inferential paradigm, and in Section \ref{sec:comput} we introduce data-driven criteria and algorithms for its choice. 

\begin{figure}[h]
\centering
\includegraphics[scale=0.7]{cil_shape.pdf} 
\caption{Prior inclusion probability \eqref{eq:cilprior} as a function of $f_{j,1}$, a feature measuring correlation between control $j$ and treatment $t=1$, for $\theta_{0}=-1$, $\rho = (J^2 + 1)^{-1}$, and $J=99$ controls. Top and bottom dotted lines show the upper and lower bounds, $1 - \rho$ and $\rho$, respectively. The dotted line in the middle corresponds to $\theta_{1} = 0$.} %\omcom{Remove the absolute from f in the xaxis}}
\label{fig:theta1}
\end{figure}

\subsection{Connections to the literature}
\label{sec:other_methods}

%\davidcom{Miquel, many refs are missing Bibtex entries and proper cite commands, can you please add?}

We discuss approaches for treatment effect inference in the presence of many controls.
The main idea in both frequentist and Bayesian literatures is to encourage the inclusion of confounders in \eqref{eq:y_eq} to mitigate under-selection bias. 
\cite{Farrell15} adapted the DML framework of \cite{Belloni14b} by using a robust estimator to safeguard from mistakes in the double selection step, 
\cite{Shortreed17} employed a two-step adaptive LASSO approach,
\cite{Antonelli18} used propensity matching,
and \cite{Victor18} extended DML by introducing a de-biasing step, and cross-fitting to ameliorate false positive inclusion of controls. 
%See also \cred{Ghosh et al (2015)} with shared and difference Lasso for similar multi-step contributions. 
An an alternative to these two-step approaches, \cite{Ertefaie15} used a joint likelihood $L_1$ penalization on the outcome and treatment regressions.
%\davidcom{To discuss. Do \cite{Victor18} address any of the over-selection issues that we describe, or some other type of over-fitting?}
%Miquel: not really, the idea is to split the sample into, say, two halves and work out the parameter estimates of the "first" stage with one of them, and plug them in on the "second" stage, which is estimated with the second half. The cross-fitting basically flips the two halves and re-estimates everything (round 2), then averages the estimates of round 1 and round 2 ("avoiding" loss of efficiency).
%\cred{Ma et al. (2019)} combine regularization with sufficient dimension reduction into an estimator that achieves good asymptotic properties without requiring model selection consistency. 
%Yet, even if some of these proposals do allow for inference, most of them are essentially designed with sufficient control selection in mind and tackle point estimation only after selection has been conducted, generally without a focus on quantification of uncertainty. 
%Other proposals exist focused on inference for PL-related methods in this particular context, see \cred{Athey et al. (2016)} in the context of binary treatments, see also \cred{Vansteelandt, Bekaert and Claeskens (2012)}, as well as more recent work in \cred{Dukes and Vansteelandt (2019)}. Generally, PL-based proposals are heavily focused on asymptotic results regarding estimation efficiency and their distributional properties, which follow from attaining guarantees of sufficient control selection, i.e. avoiding detection errors of relevant controls connected to either outcome or treatment. In contrast, they are not too concerned with excess inclusion of any other spurious variable. This can be problematic based on the aforementioned reasons, which in turn often limit their oracle performance to a super-model of the true outcome model, inflated in size with unnecessary controls that may relate to the treatment only. 
%It is also worth mentioning that a number of these proposals are not designed for continuous or even multi-valued treatments, or sometimes outcomes.

Within a Bayesian framework, a natural approach is to build a joint model
\begin{align}
p(y_{i}, \bd_{i} \mid \bx_{i}) = p(y_{i} \mid \bd_{i}, \bx_{i}) p(\bd_{i} \mid \bx_{i}),
\label{eq:joint_bms}
\end{align}
where $p(y_i \mid \bd_i, \bx_i)$ is as in \eqref{eq:y_eq} and $p(\bd_i \mid \bx_i)$ adds $T \times J$ inclusion indicators $\xi_{tj}$ describing the dependence between each treatment $t$ and control $j$. 
%In the single treatment model, for example, this approach involves modelling a third set of inclusion indicators $\bm{\xi} \in \{ 0,1 \}^{J}$, with elements $\xi_{j} = \text{I}(v_{1,j} \neq 0)$ that capture the existence of a conditional relation between the treatment and every control. 
BAC \citep{Wang12} considers this approach only for $T=1$, setting
% This method uses a tuning hyperparameter $\omega \geq 1$ that regulates the dependence of treatment effect indicators $\bgamma$ on the controls, and forces the inclusion of the treatment in \eqref{eq:y_eq}. 
a prior for $\gamma_j$ where each control has two potential prior inclusion probabilities. %, depending on whether the associated $\xi_{tj}$ is 0 or 1. 
If a control $j$ is associated to the single treatment $t=1$ ($\xi_{tj}=1$), the prior inclusion probability $P(\gamma_j=1)$ increases by a factor determined by a hyperparameter $\omega$
%Posterior parameter estimation of the model is then conducted with standard BMA. 
that is be set by the user. % and, for large $\omega$, it can lead to over-selection problems, see Figures \ref{fig:fig1} through \ref{fig:fig3}. 
\cite{Lefebvre14} and \cite{Wang15} provided some theoretical support and proposals to set $\omega$,
and \cite{Wilson18} proposed a multiple treatment extension of BAC.
\cite{Talbot15} introduced Bayesian causal effect estimation, which incorporates informative priors to deter excess control inclusion, 
and \cite{Antonelli17} generalized BAC %\textit{guided} BAC 
to address treatment effect heterogeneity.
From a practical point of view, \eqref{eq:joint_bms} multiplies the size of the model space by a factor of $2^{JT}$, rendering the framework impractical even for moderate $T$.

%Similar methods have also been explored in propensity score analysis, as in \cred{Cefalu et al. (2015)}, but more generally around model uncertainty as well, see e.g. \cred{Ziegler and Dominici (2014)} and \cred{An (2010)}. See also \cred{Jacobi et al. (2016)} for methodological adaptations to dynamic effects in panel data. 
In a different thread, \cite{Hahn18} proposed a shrinkage prior framework based on re-parameterizing a joint outcome and treatment regression, designed to improve estimation biases,
%preserve the joint modelling approach but move away from model averaging, by addressing BAC using hierarchical priors, in an attempt to give some prior flexibility and ease computational difficulties via posterior sampling. This is a reparametrization technique designed to achieve debiased point-estimates using regularization priors. 
and \cite{Hahn20} considered non-parametric Bayesian causal forests.
\cite{Antonelli19} proposed a spike-and-slab Laplace prior on the controls that shrinks less those controls that are associated to the treatment, % places low shrinkage to controls with any association to the treatment (only encouragement is possible), 
and an empirical Bayes algorithm for hyper-parameter setting. %This method formulates model estimation as a penalized likelihood problem to recover the posterior mode of the model with its corresponding coefficient estimates, without recurring to a model averaging approach.

Our main contributions are of an applied, but relevant, nature: replacing the joint model \eqref{eq:joint_bms} by extracting features derived from $p(\bd_i \mid \bx_i)$ to render computations practical, and learning from data whether confounder inclusion should be encouraged, discouraged, or neither, to avoid over-selection issues.
%It also comes at sizeable computational cost derived from the joint modelling strategy, relative to our framework, which replaces an explicit model for $p(\bd_i \mid \bx_i)$ by features extracted from observations from this conditional distribution.
Another contribution is considering the multiple treatments problem ($T>1$), which has been less studied.
%The Penalized Credible Regions method 
%\cite{Wilson14} used $L_{1}$ penalization to find the simplest model among those in a given posterior credible region of the full model,
%and can consider multiple treatments, but provide point estimation rather than parameter inference.
%%Posterior quantities are generally computed using a flat prior on the coefficients and an inverse-gamma prior on the error variance. In $p>n$ situations, the flat prior is replaced by a Gaussian prior $\bbeta \sim \text{N}(\mb{0}, \phi/\tau \mb{I})$, adding a gamma prior on $\tau$. 
%%A posterior credible region for parameters of the outcome model is built with the estimates of the posterior mean and covariance, and then an $L_{1}$-type penalty is applied to find a sparse coefficient vector within such region, penalizing less those controls with a strong relation to any given treatment, i.e. restricting to positively discriminating them in terms of prior inclusion probability. 
%%For a fixed tuning parameter $\lambda \geq 0$, the resulting estimator is
%%\begin{equation}
%%\hat{\bbeta}^{\textsc{pcr}} := \arg \min_{\bbeta \in \mathbb{R}^{J}} \tr{(\bbeta - \hat{\bbeta})} \hat{\Sigma}^{-1} (\bbeta - \hat{\bbeta}) + \lambda \sum_{j=1}^{J} \frac{|\beta_{j}|}{|\hat{\beta}_{j}| + \sum_{t=1}^{T} |\hat{v}_{t,j}|}, \nonumber
%%\end{equation}
%%where $\hat{\bbeta}$ and $\hat{\Sigma}$ denote the posterior mean of $\bbeta$ and $\Sigma$, and $\hat{v}_{t,j}$ that of $\hat{v}_{t,j}$, representing the model coefficients of regressing $\mb{d}_{t}$ on $\bX$. 
%\cite{Wilson18} %proposed ACPME, which 
%incorporated correlations between controls and multiple treatments %, this time without imposing sudden jumps in prior probabilities. 
%to tilt prior probabilities to encourage confounder inclusion in \eqref{eq:y_eq}.
%Besides the discussed over-selection bias and variance issues, marginal prior inclusion probabilities are bounded below at 1/2, which does not help reduce false positive inclusion.


%CARVALHO
%ANTONELLI
%ACPME

%%%%%%%%%%%%%%%%%%%%%%%%%%%%%%%%%%%%%%%%%%%%%%%%%%%%%%%%%%%%%%%%%%%%%%
\section{Computational Methodology} \label{sec:comput}

\subsection{Bayesian model averaging} \label{ssec:bma}

All expressions in this section are conditional on the observed $(\bx_i,\bd_i)$, we drop them from the notation for simplicity.
Inference for our approach relies on
%the density of the response given variable inclusion indicators with the all other unknown parameters integrated out, i.e., 
posterior model probabilities
\begin{align}
  %p(\bgamma, \bdelta \mid \by, \btheta) = \frac{p(\by \mid \bgamma, \bdelta) p(\bgamma \mid \btheta) p(\bdelta)}{p(\by \mid \btheta)},
\nonumber
p(\bgamma, \bdelta \mid \by, \btheta) \propto p(\by \mid \bgamma, \bdelta) p(\bgamma \mid \btheta) p(\bdelta),
\nonumber
\end{align}
where
\begin{align}
p(\by \mid \bgamma, \bdelta) = \int p(\by \mid \balpha, \bbeta, \phi, \bdelta, \bgamma) p(\balpha, \bbeta \mid \bdelta, \bgamma, \phi) p(\phi) \text{d}\balpha \text{d}\bbeta \text{d}\phi \label{eq:marglik2}
\end{align}
is the marginal likelihood of model $(\bgamma,\bdelta)$.
We set the hyperparameter $\btheta$ to a point estimate $\hat{\btheta}$ described in the next section.
Conditional on $\btheta$, our model prior $p(\bgamma \mid \btheta)$ is a product of independent Bernouilli's with asymmetric success probabilities defined by \eqref{eq:cilprior}. As a simple variation of standard BMA, one can exploit existing computational algorithms, which we outline next.

%\begin{equation}
%\begin{aligned}
%& p(y_i \mid \bx_i,\bd_i,\bgamma, \bdelta)  = \\ & \int p(y_i \mid \bx_i,\bd_i,\bgamma, \bdelta, \balpha, \bbeta, \phi ) p(\balpha \mid  \bdelta,  \phi) p(\bbeta \mid \bgamma,\phi) p(\phi) \text{d}\balpha \text{d}\bbeta \text{d}\phi.
%\end{aligned} \label{eq:marglik2}
%\end{equation}
Outside particular cases such as Gaussian regression under Gaussian priors, \eqref{eq:marglik2} does not have a closed-form expression.
To estimate \eqref{eq:marglik2} under our pMOM prior we adopt the approximate Laplace approximations of \cite{Rossell20a}, %using Proposition 1 in \cite{Rossell17}, 
see Section \ref{subsec:nlp_approx}.

We obtain point estimates using BMA,
\begin{align}
\hat{\balpha} := \sum_{\bgamma, \bdelta}  \text{E}(\balpha \mid \by, \bgamma, \bdelta) p(\bgamma, \bdelta \mid \by, \btheta), \label{eq:teffest}
\end{align}
and similarly employ the BMA posterior density $p(\balpha \mid \by, \bgamma, \bdelta, \btheta)$ to provide posterior credible intervals.
To this end we use posterior samples from this density  using a latent truncation representation described by \cite{Rossell17}. 
%\omcom{I think we need a supplement section with more details on how we do all this for reproducibility}
%\davidcom{Maybe we can just provide the R code to ensure reproducibility, for an applications paper we already have lots of computational details}
Expression \eqref{eq:teffest} is a sum across $2^{T+J}$ models, when it is unfeasible we use Markov Chain Monte Carlo to explore the posterior distribution $p(\bgamma,\bdelta \mid \by,\btheta)$,
%The computational methods we employ and introduce rely on samples from the posterior density
%$$p(\bgamma,\bdelta \mid \by,\btheta) \propto p(\by \mid \bgamma, \bdelta) p(\bgamma) p(\bdelta \mid \btheta).$$
see e.g. \cite{Clyde12} for a review. 

We used all the algorithms described above as implemented by the \texttt{modelSelection} function in \texttt{R} package \texttt{mombf} \citep{Rossell20}.

%\omcom{Om2D: Here is where I have stopped editing. I created the two subsections below and I think these two should be the only other thing in this section, I have left the material from the past below in case you want to copy and paste stuff easily.}

\subsection{Confounder importance learning via marginal likelihood}
\label{sec:ml}

Our main computational contribution is a strategy to learn the hyperparameter $\btheta$, which plays a critical role by determining prior inclusion probabilities. We devised an empirical Bayes approach maximizing the marginal likelihood, with
\begin{align}
\bthetaeb := \argmax_{\btheta \in \mathbb{R}^{T+1}} p(\by \mid \btheta)=
% \label{eq:eb_eq}
%\argmax_{\btheta \in \mathbb{R}^{T+1}} \sum_{\bgamma, \bdelta} p(\by \mid \bgamma, \bdelta) p(\bgamma) p(\bdelta \mid \btheta). 
\argmax_{\btheta \in \mathbb{R}^{T+1}} \sum_{(\bdelta, \bgamma)} p_{u}(\bdelta, \bgamma \mid \by) p(\bdelta, \bgamma \mid \btheta)
\label{eq:marglik1}
\end{align}
where the right-hand side follows easily, denoting by $p_{u}(\bdelta, \bgamma \mid \by)$ the posterior probabilities under a uniform model prior $p_{u}(\bdelta, \bgamma) \propto 1$.
The use of empirical Bayes for hyperparameter learning in variable selection has been well-studied, see \cite{George00, Scott10, petrone:2014}.
%\begin{align}
%p(\by \mid \btheta) = \sum_{(\bdelta, \bgamma)} p(\by \mid \bgamma, \bdelta) p(\bdelta, \bgamma \mid \btheta), \label{eq:marglik1}
%\end{align}

The main challenge is that one must evaluate the costly sum in \eqref{eq:marglik1} for each value of $\btheta$ considered by an optimization algorithm. 
%A second challenge is that evaluating \eqref{eq:marglik2} under non-local priors, as with our MOM prior, can be computationally cumbersome. 
Note that $p(\by \mid \bgamma,\bdelta)$ does not depend on $\btheta$, and hence can be re-used to evaluate \eqref{eq:marglik1} for any number of $\btheta$ values. 
%Section \ref{subsec:comput1} discusses  strategies to evaluate \eqref{eq:marglik2} and obtain posterior inference, given $\bthetaeb$. Section \ref{subsec:sgdeb} presents a strategy to evaluate the gradient of \eqref{eq:marglik1}, which can serve as the basis of standard optimization algorithms, and discusses issues related to the existence of local maxima. Section \ref{subsec:varapprox} proposes a fast Expectation Propagation mean-field approximation to $\bthetaeb$, which we denote $\bthetaep$. The latter provided practically indistinguishable results from $\bthetaeb$ in our numerical experiments, at a significantly lower computational cost. Alternatively, $\bthetaep$ can be used to initialize $\bthetaeb$ when optimizing \eqref{eq:marglik1}, ameliorating issues related to finding low-quality local maxima.
In fact, by Proposition \ref{prop:one} below, this provides grounds to use stochastic gradient methodology to maximize \eqref{eq:marglik1}.
\begin{prop} \label{prop:one}
If $p(\by \mid \bgamma, \bdelta, \btheta) = p(\by \mid \bgamma, \bdelta)$, then
\begin{align*}
\nabla_{\btheta} \log p(\by \mid \btheta) = \sum_{(\bdelta, \bgamma)} p(\bgamma, \bdelta \mid \by, \btheta) \nabla_{\btheta} \log p(\bgamma, \bdelta \mid \btheta).
\end{align*}
If, additionally, the model prior is separable such that
\begin{align*}
p(\bgamma, \bdelta \mid \btheta) = \prod_{t=1}^{T} p(\delta_{t}) \prod_{j=1}^{J} p(\gamma_{j} \mid \btheta),
\end{align*}
then
\begin{align}
\nabla_{\btheta} \log p(\by \mid \btheta) = \sum_{j=1}^{J} \E\left[ \nabla_{\btheta} \log p(\gamma_{j} \mid \btheta) \mid \by \right]. \label{eq:prop_one}
\end{align}
\end{prop}

\begin{corollary} \label{prop:two}
Under the model prior in \eqref{eq:pdelta} and \eqref{eq:pgamma}, and with $\pi_{j}(\btheta)$ as defined by \eqref{eq:cilprior},
\begin{align}
\nabla_{\btheta} \log p(\by \mid \btheta) = (1-2\rho) \sum_{j=1}^{J} \mb{f}_{j} \left[P(\gamma_{j} = 1 \mid \by, \btheta) - \pi_{j}(\btheta)\right], \label{eq:prop_two}
\end{align}
where $\mb{f}_{j} = \tr{(1, f_{j,1}, \dots, f_{j,T})}$.
\end{corollary}
%\omcom{Remove absolute values above and fix any proofs in Supplement}
Expressions \eqref{eq:prop_one} and \eqref{eq:prop_two} evaluate the gradient with a sum of $J$ terms, relative to the $2^{J+T}$ terms in \eqref{eq:marglik1}. Further, \eqref{eq:prop_two} only depends on $\by$ via marginal inclusion probabilities $P(\gamma_j = 1 \mid \by, \btheta)$, which can typically be estimated more accurately than the joint model probabilities in \eqref{eq:marglik1}. 
%The Gibbs sampling scheme introduced in Section \ref{subsec:comput1} provides estimates for the marginal posterior inclusion probabilities needed in \eqref{eq:prop_two}, which are Rao-Blackwellized to increase accuracy. Importantly, the result in Corollary \ref{prop:two} allows for the use of stochastic Newton-type optimization algorithms to approximate $\bthetaeb$. 
However, two problems remain unaddressed. First, one must compute $P(\gamma_j = 1 \mid \by, \btheta)$ for every considered $\btheta$, which is cumbersome when the optimization requires more than a few iterations. Second, $\log p(\by \mid \btheta)$ can be non-convex. Hence, standard algorithms may converge to low-quality local optima if $\btheta$ is poorly initialized. Figure \ref{fig:thEPEB} (left) shows an example of a multi-modal $p(\by \mid \btheta)$.
We next describe an Expectation Propagation approximation which, as illustrated in Figure \ref{fig:thEPEB}, typically provides a good approximation to the global mode.



\subsection{Confounder importance learning by Expectation-Propagation}
\label{sec:ep}

The use of Expectation Propagation \citep{Minka01a, Minka01b} is common in Bayesian machine learning, including in variable selection \citep{Seeger07, HdezLobato13, Xu14}. 
We propose a computationally tractable approximation to \eqref{eq:marglik1}, which can also serve as an initialization point for an algorithm to solve \eqref{eq:marglik1} exactly, if so desired.

We consider a mean-field approximation to the posterior probabilities in \eqref{eq:marglik1},
\begin{align}
\hat{p}_{u}(\bdelta, \bgamma \mid \by) = \prod_{t=1}^{T} \text{Bern}(\delta_{t}; s_{t}) \prod_{j=1}^{J} \text{Bern}(\gamma_{j}; q_{j}). \label{eq:mfapp}
\end{align}
where $\mb{s}=(s_1,\ldots,s_T)$ and $\mb{q}=(q_1,\ldots,q_J)$ are given in Proposition \ref{prop:three} to optimally approximate $p(\bdelta, \bgamma \mid \by)$. 
By Proposition \ref{prop:three} below, \eqref{eq:mfapp} leads to replacing \eqref{eq:marglik1} by a new objective function
\eqref{eq:ep_eq0} that only requires an inexpensive product across $J$ terms. These only depend on $\by$ via posterior inclusion probabilities $q_{j} = P(\gamma_{j} = 1 \mid \by, \btheta = \mb{0})$ that can be easily pre-computed prior to conducting the optimization exercise.

\begin{prop}  \label{prop:three}
Let $s_t$, $q_j$ and $\hat{p}_u(\bdelta,\bgamma \mid \by)$ be as defined in \eqref{eq:mfapp}. Then, $s_{t}^{\textsc{ep}} = P(\delta_{t} = 1 \mid \by, \btheta = \mb{0})$ and $q_{j}^{\textsc{ep}} = P(\gamma_{j} = 1 \mid \by, \btheta = \mb{0})$ minimize Kullback-Leibler divergence from $p_{u}(\bdelta, \bgamma \mid \by)$ to $\hat{p}_u(\bdelta,\bgamma \mid \by)$. Further
\begin{eqnarray}
\bthetaep_{u} &:=& \argmax_{\btheta \in \mathbb{R}^{T+1}} \sum_{\bdelta, \bgamma} \hat{p}_{u}(\bdelta, \bgamma \mid \by) p(\bdelta, \bgamma \mid \btheta) \nonumber \\
&=& \argmax_{\btheta \in \mathbb{R}^{T+1}} \sum_{j=1}^{J} \log \left( q_{j}^{\textsc{ep}} \pi_j(\btheta) + (1-q_{j}^{\textsc{ep}}) (1-\pi_j(\btheta)) \right). \label{eq:ep_eq0}
\end{eqnarray}
\end{prop}
%\omcom{There are two issues here. One is notation, there is both $q_j$ and $q_j^{EP}$. Other is that the $s_t$ do not show up anywhere which is confusing. if they play no role in this latter optimization why do we make the approximation in the first place? we could just use the actual marginal of the deltas. Anyway, Miquel please talk about this with David because I do not understand this. And fix the typos}
%Note that \eqref{eq:ep_eq0} has no closed form solution, but critically its objective function only depends on $\btheta$ through deterministic terms: its only stochastic quantities $\hat{\mb{q}}$ are independent of $\btheta$ and can be accurately estimated upfront with a single model space search under $\btheta = \mb{0}$. As a result, evaluating the expression in \eqref{eq:ep_eq0} becomes much faster to evaluate compared to \eqref{eq:eb_eq}, and it can be efficiently optimized deterministically for a given $\hat{\mb{q}}$ using a gradient descent method, as 
The gradient of the objective function \eqref{eq:ep_eq0} is in Section \ref{sec:gradEP}.
Since this function may not be concave, we conduct an initial grid search and subsequently use a quasi-Newton BFGS algorithm. See Section \ref{sec:algorithm} and Algorithm \ref{alg:one} therein for a full description of our algorithm to obtain $\hthetaeb$ and $\hthetaep$.
In most our examples $\hthetaeb$ and $\hthetaep$ provided virtually indistinguishable inference, the latter incurring a significantly lower computational cost, but the exact $\hthetaeb$ did provide slight advantages in some high-dimensional settings (see Section \ref{subsec:singleT}).
%\cred{While the parameter $\hthetaep$ itself might suggest a value of $\btheta$ that can be readily implemented in \eqref{eq:cilprior}, it also provides an initialization point to optimize \eqref{eq:eb_eq} that is fast to compute. A proper initialization point can help us escape local optima, as well as minimize the number of optimization steps.}


\section{Results} \label{sec:cps}

We studied the association between belonging to certain social groups and the hourly salary, and its evolution over the last decade (prior to the \textsc{covid}-19 pandemic), to assess progress in wage discrimination. We analyzed the USA Current Population Survey (CPS) microdata \citep{ipums}, which records many social, economic and job-related factors. %Surveys are administered monthly by the U.S. Bureau of the Census to over 65,000 households. %The microdata is made freely available by the Integrated Public Use Microdata Series (IPUMS) website.
The outcome is the individual log-hourly wage, rescaled by the consumer price index of 1999, and we considered four treatments: gender, black race, Hispanic ethnicity and Latin America as place of birth. These treatments are highly correlated to sociodemographic and job characteristics that can impact salary, i.e. there are many potential confounders.

Section \ref{sec:cps_data} describes the data and Section \ref{sec:cps_results} contains results on the treatment effects, both individually and in terms of a composite score measuring their joint association with salary.
These results support that methods designed for treatment effect inference may run into over-selection, whereas naive methods may run into under-selection.
To provide further insight, Section \ref{subsec:singleT} shows simulation studies, with particular attention on how the presence/absence of confounders affects each method.
%We design a set of simulations to illustrate the various effects discussed along this article, some of which will also be apparent in the real application presented in Section \ref{subsec:realapp}. In these synthetic datasets we compare the performance of a few of the methods discussed in a variety of situations. We divide them in two subsets: one for single treatment estimation, and one for multiple treatments.


In Section \ref{sec:cps_results} we compare our CIL (under the EP approximation) to three methods: OLS under the full model ($J=278$ controls), DML based on the LASSO \citep{Belloni14b}, and standard BMA with a $\text{Beta-Binomial}(1,1)$ model prior and the pMOM prior in Section \ref{sec:model}.
In Section \ref{subsec:singleT} using simulated data we also compare to BAC \citep{Wang12}, which was computationally unfeasible to apply to the salary data. We set its hyperparameter to $\omega = +\infty$, which encourages the inclusion of confounders relative to %$\omega = 1$, which corresponds to 
standard BMA. For completeness, we also considered a standard LASSO regression on the outcome equation \eqref{eq:y_eq}, setting the penalization parameter via cross-validation. 
%Finally, we report the Penalized Credible Regions approach (PCR) by \cite{Wilson14} on a number of figures in the Supplementary material, as despite not providing inference it exhibited good point estimation in some situations, setting the penalty parameter to minimize the Bayesian information criterion. 
%Debiased LASSO provides no shrinkage and thus was outside of our present scope. 
%\davidcom{I removed the mention to PCR, it was just confusing, please also remove from the suppl figures, PCR just makes them harder to read}
We compared these methods to the oracle OLS, i.e. based on the subset of controls truly featuring in \eqref{eq:y_eq}. 
%Regarding standard BMS, we used the same pMOM prior as in our CIL and set a Binomial model prior akin to \eqref{eq:pdelta}--\eqref{eq:pgamma}. We set the prior inclusion probabilities to the true proportion of active covariates, to assess the advantages of CIL over BMS in a most favorable setting for the latter.
These methods are implemented in \texttt{R} packages \texttt{glmnet} \citep{glmnet} for LASSO, \texttt{mombf} for BMA, \texttt{hdm} \citep{hdm} for DML and \texttt{BACprior} \citep{BACprior} for BAC. %, and \texttt{BayesPen} \citep{BayesPen} for PCR.


\subsection{Data}
\label{sec:cps_data}

We downloaded data %of 03-2010 and 03-2019, which include data from the Annual Social and Economic Supplement, 
from 2010 and 2019 and analyzed each year separately. %, which include data from the Annual Social and Economic Supplement. 
We selected individuals aged over 18, with a yearly income over \$1,000 and working 20-60 hours per week, giving $n=64,380$ and $n=58,885$ in 2010 and 2019, respectively. 
The controls included characteristics of the place of residence, education, labor force status, migration status, household composition, housing type, health status, financial and tax records, reception of subsidies, and sources of income (beyond wage). 
Overall, there were $J=278$ controls, after adding 50 binary indicators for state. 

Since every state has its own regulatory, sociodemographic and political framework, we captured state effects by adding interactions for each pair of treatment and state. On these interactions, we applied a sum to zero constraint, so that the coefficients associated to the four treatments remain interpretable as average effects across the USA, and the interactions as deviation from the average. Hence, overall, we have $T = 4+4 \times 50 = 204$ treatments, our main interest being in the first four. 
To simplify computation in our CIL prior we assumed a common $\theta_t$
%From a computational perspective, to deal with this amount of treatments, which include interactions, we make an amendment to our CIL prior: we assume that for a main treatment $d_{i,t}$ and every interaction with it we have one common $\theta_{t}$, i.e. $\theta_t$ is
shared between each main treatment and all its interactions with state, so that $\mbox{dim}(\btheta)=5$. %Thus, we limit the dimensionality of $\btheta$ to $4+1$.
A full list of the extracted variables can be found as a supplementary file.%\omcom{add the exact pointer here}.
%A full list of the extracted variables can be found in Section SXXX \davidcom{State what section, I did not see this info in the supplement}.
%\davidcom{Miquel, provide a list of the finally used variables in a supplementary file, and the R script doing the data pre-processing.} 
%We collected a total of 232 initial available control variables.  The number of controls is reduced to 162 after removing variables with insufficient variability or redundancies. After converting categorical variables into dummy variables, the number of control variables increases to 228, including the intercept. We added 50 dummy variables reporting which U.S. State the observation belongs to. 

%Our objective is two-fold. First, we want
To study issues related to under- and over-selection, we analyzed the original data and two augmented datasets where we added artificial controls correlated with the treatments but not the outcome. The augmented data incorporated 100 artificial controls in the first scenario, and 200 in the second one, which were split into four subsets of size 25 and 50, respectively. Each of these subsets was designed to correlate to one of the four main treatments. Section \ref{sec:fakepreds_supp} summarizes the simulation procedure.
The resulting average correlation between gender and its associated artificial variables was of $0.83$, and analogously of $0.69$, $0.76$ and $0.67$ for black race, ethnicity and place of birth with their corresponding correlated variables, respectively. %\omcom{can't follow the notation here and the meaning either. First, how does this construction give correlation with EACH of the treatments? the math suggest that it is with one of them (and then maybe implicitly with the others if they are between them correlated). second, why is bold z? Cannot follow this at all. can you please write carefully since it is important and is unclear}


\subsection{Salary survey results}
\label{sec:cps_results}

%\begin{figure}[h]
%\centering
%\includegraphics[scale=0.5]{figApp1.pdf} 
%\caption{Inference on treatment effects of gender and race for selected methods, for 2010 and 2019. Vertical axes show the size of the effect on the log hourly wage for the respective treatment. A negative sign corresponds to the existence of a negative wage discrimination, and vice versa. Dots and squares correspond to point estimates, while braces show the corresponding 95\% confidence band. Dots represent significant point estimates, chosen as $p\text{-value} < 0.05$ for the frequentist methods, or marginal posterior inclusion probability above 0.5 for Bayesian methods. Results in black correspond the estimation under the true data, while the grey results correspond to the estimation for the artificially enhanced datasets with 100 and 200 fake predictors, sequentially.}
%\label{fig:fig6}
%\end{figure}

None of the methods pointed to an association between salary and ethnicity nor place of birth, see Figure \ref{fig:figApp1B}. Figure \ref{fig:intro1} reports the results for gender and race. The treatment effect for gender is picked up by all methods in both years with similar point estimates. All methods found a significant decrease of this effect in 2019. 
When adding the artificial controls, the confidence intervals for OLS and DML became notably wider, which in 2019 led to a loss of statistical significance. This points towards a relevant loss in power due to over-selecting irrelevant controls, with the associated variance inflation.
The CIL results were particularly robust to the artificial controls.

As for race, we obtained more diverse results. In 2010, DML, BMA and CIL found a negative association between black race and salary. However, in 2019 DML and BMA found no significant association, OLS found a small positive association, and CIL was the only method finding a negative association in both years.
Once we introduce the artificial controls, we observe that OLS and DML suffer a large variance inflation, and in 2010 BMA suffers a significant loss of power, failing to detect an effect that was confidently picked up in the original data. 
On the other end, CIL experiences no perceptible change to adding the artificial controls. These results seem to suggest that CIL has sufficient power to detect the difference that other methods miss in the original data in 2019. 

The full scope of our proposed approach is materialized when considering more complex functions of the parameters. We study a measure of overall treatment contribution to deviations from the average salary. For a new observation $n+1$, with observed treatments $\mb{d}_{n+1}$ and controls $\mb{x}_{n+1}$, let
\begin{align}
  & h_{n+1}(\mb{d}_{n+1}, \balpha, \mb{x}_{n+1}) = \nonumber \\
  & | \E(y_{n+1} \mid \mb{d}_{n+1}, \mb{x}_{n+1}, \balpha, \bbeta) - \E(y_{n+1} \mid \mb{x}_{n+1}, \balpha, \bbeta) |
%\nonumber \\
%=(\tr{\mb{d}_{n+1}} \balpha + \tr{\mb{x}_{n+1}} \bbeta) - (\tr{\bar{\mb{d}}_{i}} \balpha + \tr{\mb{x}_{n+1}} \bbeta) 
= | \tr{[\mb{d}_{n+1} - \E(\mb{d}_{n+1} \mid \mb{x}_{n+1})]} \balpha | \label{eq:app2}
\end{align}
be its expected salary minus the expected salary averaged over possible $\mb{d}_{n+1}$, given equal control values $\mb{x}_{n+1}$.
%Naturally, this quantity also depends on $\mb{x}_{n+1}$, which we drop from the notation for simplicity.
Since $y_{n+1}$ is a log-salary, we examine the posterior predictive distribution of $\exp \left\{ h_{n+1}(\mb{d}_{n+1}, \balpha, \mb{x}_{n+1}) \right\}$ as a measure of salary variation associated to the treatments. A value of 1 indicates no deviation from the average salary, relative to another individual with the same controls $\mb{x}_{n+1}$.


\begin{figure}[h]
\centering
\includegraphics[scale=0.6]{figApp2.pdf} 
\includegraphics[scale=0.6]{figApp2B.pdf}
%\includegraphics[scale=0.6]{figApp2B_old.pdf}
\caption{The left panel shows the posterior predictive distribution of deviations from average salary as given by $\exp \left\{ h_{n+1}(\mb{d}_{n+1}, \balpha, \mb{x}_{n+1}) \right\}$ in \eqref{eq:app2}, for 2010 and 2019. The gray boxes represent 50\% posterior intervals and the black lines are 90\% intervals. The black dot is the posterior median. The right panel shows the posterior median of these deviations for every U.S. state in 2010 and 2019 on the horizontal axis, ordered by their value in 2019, with the corresponding 50\% posterior intervals for both years.}
\label{fig:salary_variation}
\end{figure}

To evaluate the posterior predictive distribution of \eqref{eq:app2} given $\by$, the observed $\bd$ and the set of controls, we obtained posterior samples from the model averaged posterior $p(\balpha \mid \by)$ associated to CIL (Section \ref{ssec:bma}). Given that we do not have an explicit model for $(\bd_{n+1}, \mb{x}_{n+1})$, we sampled pairs $(\mb{d}_{n+1}, \mb{x}_{n+1})$ from their empirical distribution, and estimated $\E(\mb{d}_{n+1} \mid \mb{x}_{n+1})$ from a logistic regression of $\bd$ on the set of controls.
Figure \ref{fig:salary_variation} shows the results. In 2010, joint variation in the treatments was associated to an average 5.4\% salary variation (90\% predictive interval [0.1\%, 18.3\%]). The posterior mean in 2019 dropped to 1.5\% and the 90\% predictive interval was [0\%, 4.9\%]. That is, the treatments not only played a smaller role in the 2019 average, but there was also a strong improvement in inequality, e.g. individuals whose salary was furthest from the average. %, conditional on their control values $\mb{x}_{n+1}$.

It is also of interest to study differences between states. This is possible in our model, which features 200 interaction terms for the 4 treatments and 50 states. Figure \ref{fig:salary_variation} (right) shows the results. The most salient feature is a lower heterogeneity across states in 2019 relative to 2010.
%At the state level, in 2010 we observe Wyoming (2.7\%), Maine (2.9\%) and South Dakota (3.1\%); and Pennsylvania (6.4\%), New York (6.0\%) and Ohio (6.0\%), as those states with smallest and largest posterior average deviations, respectively. In 2019, the states with smallest average deviations were the same three (at roughly 0.8\%), and those with largest ones were Pennsylvania (1.9\%), Illinois (1.7\%) and Kentucky (1.7\%). 
The states whose median improved the most were among the lowest ranking states in 2010: Pennsylvania (reducing it by 3.7\%), Indiana (3.7\%) and Ohio (3.6\%), while those improving the least (Nebraska aside, whose median is 0\% on both years) were Maine (0.2\%), Wyoming (0.2\%) and South Dakota (0.3\%), which were already among the top-ranking states in 2019. This points towards a gradual catch-up effect across U.S. states, although the intervals still show some variability within states.
%\omcom{There are still many issues with this part, apart from being clearly more poorly written than any of the previous ones. Please talk to David and in between the two of you polish this. I do not need to see this again but please try and improve. 1. I thought David was not using an average of d anymore but some typical values. clearly this changed again to average? just check with him. 2. we actually have a model for d given x, the one we use for CIL. why are we not using directly that? 3. you say use sample the x's from empirical but you are also the d's and it is confusing you don't write this so. 4. it is confusing why h is written as a function of alpha and d and not x. 5. Is there some reason why there is enormous variability in 2010 and so much less in 2019? are you sure you are not doing something wrong? }

%%%%%%%%%%%%%%%%%%%%%%%%%%%%%%%%%%%%%%%%%%%%%%%%%%%%%%%%%%%%%%%%%%%%%%
\subsection{Simulation Studies} \label{subsec:singleT}
%\subsection{Single Treatment Model} \label{subsec:singleT}

To illustrate issues associated to under- and over-selection of controls, a key factor we focus on is the \textit{level of confounding}.
Our scenarios range from no confounding (no controls affect both $\by$ and $\bd$) to complete confounding (any control affecting $\by$ also affects $\bd$, and vice versa). We also considered the effect of dimensionality, treatment effect sizes $\alpha$, and true level of sparsity (number of active controls).

We considered a Gaussian outcome and a single treatment $(T=1)$, and an error variance $\phi = 1$. 
The controls were obtained as independent Gaussian draws $\bx_i \sim \text{N}(\mb{0}, \mb{I})$, and any active control had a coefficient $\beta_{j} = 1$.
The treatment $\bd$ was also Gaussian with its mean depending linearly on the controls, unit error variance,  and any control having an effect on $\bd$ had unit regression coefficient.
Throughout, the number of controls that truly had affect $\bd$ was set equal to the number of controls that affect the outcome $\by$.
We measured the RMSE of the estimated $\hat{\alpha}$. 
%A key factor we focus on is the \textit{level of confounding}, i.e. how many controls truly affect both the treatment and the outcome. 


Figures \ref{fig:intro2}, \ref{fig:singletreat_growingdim}, \ref{fig:fig1b} and \ref{fig:multitreat_uncounfounded} summarize the results.
%\begin{figure}[h]
%\centering
%\includegraphics[scale=0.6]{fig1A.pdf} 
%\caption{Single treatment parameter RMSE (relative to Oracle OLS) based on $R=250$ simulated datasets for each level of confounding. On each $x$-axis we plot increasing levels of confounding from left (no confounding) to right (complete confounding). Left panel shows results for $\alpha = 1$ (strong signal), center for $\alpha = 1/3$ (weak signal), and right for $\alpha^{*} = 0$ (lack of effect). In all panels, $n=100$, $J+T=50$ and $\norm{\bgamma}_{0} = 6$.}
%\label{fig:fig1}
%\end{figure}
Figure \ref{fig:intro2} shows that the RMSE of BMA and LASSO worsens as confounding increases, this was due to a lower power to select all relevant confounders (see Figure \ref{fig:fig1b} for model selection diagnostics), i.e. an omitted variable bias. 
%We observe that generic methods perform as one might expect: as confounding increases they have more difficulties to pick up every correlated active control, and thus tend to under-select relevant controls, running into omitted variable bias. BMA suffers less than LASSO, in part because it has the advantage of knowing the true underlying sparsity via the model prior. 
These effects have been well studied. % and occur regardless of the different treatment effect strengths. 
Methods such as DML and BAC were designed to prevent omitted variables, but as shown in Figure \ref{fig:intro2} they can run into over-selection when there truly are few confounders.
%Specific methodology developed to address them, such as DL and BAC, are strongly focused on not losing relevant variables by encouraging inclusion of controls that correlate to the treatment. This is highly beneficial in high-confounding situations, however, with low-confounding it can lead to non-random over-selection of controls, causing a notable inflation of the variance of the estimator, and leading to a different source of bias. 
%Over-selection bias arises from the inclusion of covariates strongly correlated to an active treatment that are truly inactive on the outcome. Issues related to over-selection are often overlooked as generally it is preferred to err on the side of model inflation but, importantly, here we observe that the magnitude of these effects can be of similar order to those related to control under-selection, and hence should be taken equally into account. 
%\cred{PCR attained a quite satisfactory and steady performance in point estimation throughout.}
In contrast, our CIL performed well at all levels of confounding.
%When treatment size is weak, however, it can suffer more to detect it with precision, in the general line of Bayesian variable selection based methods, although it still performs close to competing methods. 
The Empirical Bayes and the Expectation Propagation versions of CIL provide nearly indistinguishable results (not shown).
It is worth noting that, when the treatment truly had no effect $(\alpha = 0)$, CIL provided a strong shrinkage that delivered a significantly lower RMSE than other methods.
%  shows the partition of the methods between those forcing treatment inclusion and those not forcing it. The latter methods are capable of achieving results better than oracle OLS in this scenario, thanks to detecting that the treatment is inactive with a precise point estimate at zero.


Figure \ref{fig:singletreat_growingdim} extends the analysis to consider a growing number of covariates, under a strong treatment effect ($\alpha=1$). As dimensionality grew, standard LASSO and BMA incurred a significantly higher RMSE under strong confounding. Our CIL generally provided marked improvements over BMA, except for the larger $J+T=200$.
%Figure \ref{fig:singletreat_growingdim} shows the results for different design sizes under a strong treatment effect. Overall results exhibited a similar trend to Figure \ref{fig:intro2}, with exacerbating under- and over-selection effects as dimensionality grows.
%\cred{PCR suffered the strongest deterioration in higher dimensions, with strong spikes in RMSE as $J$ grows relative to $n$. Their authors anticipate that PCR requires $n$ to grow at sufficiently fast rates relative to $J$ to achieve good performance.}
%Generally, CIL continued to perform very well for all confounding levels under $J \leq N$. As for the case with $J > N$, no method achieved satisfactory performance consistently. Our method suffered not to over-select as confounding levels decrease, although it had the smallest spike in RMSE relative to the rest of methods. 
Here we observed the only perceptible differences between the EB and EP approximations, with the former attaining better results, pointing to advantages of the EB approach in higher dimension. Figure \ref{fig:fig3} further extends the analysis to less sparse settings, with $\norm{\gamma}_{0}=6$, 12 and 18 active parameters. Overall, the results were similar to Figures \ref{fig:intro2} and \ref{fig:singletreat_growingdim}.
%Figure \ref{fig:singletreat_growingdim}  examines the situation in which the total amount of confounders varies, with very similar relative results as those exhibited in Figures \ref{fig:intro2} and \ref{fig:singletreat_growingdim}.


A focus in this paper is to understand over-selection issues in multiple treatment inference.
%To this end, we added a second treatment that was truly uncounfounded with any covariate. 
%Additionally, we extended some of these simulations to $T=2$, and hence $\balpha = (\alpha_{1}, \alpha_{2})$. We restricted confounding to the first treatment, where the subsets of controls associated to each treatment were disjoint and of equal size. The rest of elements in the design were kept equal to the design for $T=1$. 
To this end, we added a multiple treatments design with an increasing number of treatments, with a maximum of $T=5$. There, every present treatment was active, setting $\alpha_t = 1$ on all treatments. For all levels of $T$, we set $\beta_j = 1$ for $j = 1, \dots, 20$, denoting the set of active controls by $\bx_{1:20}$, and $\beta_j = 0$ for the rest of controls $\bx_{21:J}$. Regarding the association between treatments and controls, $\bx_{1:20}$ were divided into five disjoint subsets with four variables each, and each of these subsets was associated to a different treatment. When existing, the treatments were linearly dependent on every control of its associated subset. Additionally, each treatment also depended on a further subset of controls in the set $\bx_{21:J}$. In this case, the size of such subset was increasing by four with each added treatment: treatment 1 was associated to $\bx_{21:24}$, treatment 2 was associated to $\bx_{21:28}$, etc., up to treatment 5, which correlated to $\bx_{21:40}$. All controls affecting a treatment had a unit linear coefficient. The rest of the design, including the DGP of the controls and the error variances, is akin to that in Figure \ref{fig:intro2}.
We also replaced BAC with the ACPME method %the Adjustment for Confounding in the Presence of Multivariate Exposures method 
of \citep{Wilson18}, an extension of BAC for multiple treatments.

\begin{figure}[h]
\centering
%\begin{tabular}{cc}
%Treatment 1 & Average across Treatments 1 to 5 \\
%\includegraphics[scale=0.75]{figXp1.pdf} &
%\includegraphics[scale=0.75]{figXp2.pdf} 
%\end{tabular}
\includegraphics[scale=0.8]{figXp2.pdf} 
\caption{Treatment parameter RMSE (relative to oracle least-squares) based on $R=250$ simulated datasets at every value of $T$, for $n=100$, $J=95$, and $T \in \{2, 3, 4, 5 \}$. For every $T$ ($x$-axis), we show the average RMSE across Treatments $1,\ldots,T$.
%, i.e. at $T=2$ we show the average of two RMSE measures (one per treatment), at $T=3$ across three of them, etc. 
%Here the $x$-axis indicates the number of treatments in the model.
}%In both diagrams, the $x$-axis indicates the number of treatments in the model. The left panel shows the results for Treatment 1, while the right panel shows the average across present treatments at each level of $T$, i.e. at $T=2$ it shows the average RMSE across two treatments, at $T=3$ across three treatments, etc.}
\label{fig:multitreat_uncounfounded}
\end{figure}

Figure \ref{fig:multitreat_uncounfounded} shows the estimation results on $\balpha$ for the different values of $T$, akin to Figure \ref{fig:intro2}. We observe similar trends as before. Methods prone to over-selection recovered more inactive controls as $T$ increased, i.e. they included too many controls affecting the inference of those treatments they correlated to. Some of these controls were increasingly influential with $T$ as they were associated to more treatments, and so they became harder to discard. Despite that, under-selection (here suffered by BMA) was also problematic, as for larger $T$ the model became highly confounded, as a subset of the controls accounted for a larger proportion of the variance in the outcome, as well as for that of the treatment(s). This led to BMA discarding with high probability active but highly correlated variables between treatments and confounders. Additionally, we also observed stable but improving performance of ACPME, although in its best performing scenario its average RMSE still more than doubled that of oracle OLS. On the other end, our CIL proposal was able to achieve oracle-type performance for every examined level of $T$.

%Figure \ref{fig:multitreat_uncounfounded} shows that over-selection issues related to Treatment 1 carry over to Treatment 2: DML and ACPME incurred a larger RMSE, roughly the same as that for Treatment 1 under no confounding. These results illustrate how over-selection issues may propagate over treatments.
%We report the results on a set of strong treatment effects for a maximum of six active confounders in Figure \ref{fig:multitreat_uncounfounded}. \cred{As for the confounded treatment, we observed almost no change with respect to its counterpart in the single treatment case, although there was a drop in performance of our method only in the complete confounding case, where it fell short of LASSO and ACPME, despite still beating regular BMA. ACMPE showed very similar performance for $T=2$ relative to BAC in $T=1$. 
%On the second treatment, results are generally consistent through levels of confounding, as these levels refer only to the first treatment. Here our proposed method achieves practically oracle performance, alongside BMA. For neither treatment do we appreciate differences between EB and EP versions.}

%As for the confounded treatment, we see a general drop in performance of all methods with respect to oracle OLS, although the relative performances experience little variation, and all of the effects described before remain present. The ACPME method improves substantially the performance of BAC on a single treatment, in Section \ref{subsec:singleT}. All in all, our approach achieves better results than every compared method in most situations, and performs very close to the top performer in the remaining scenarios. On the second treatment, results are generally consistent through levels of confounding, as these levels refer only to the first treatment. We observe a similar relative performance among tested methods: here our proposed method achieves practically oracle performance, alongside BMA. For neither effect we appreciate differences between EB and EP versions. The problems encountered by \cred{DL} are very explicit here since the effect of the regressors dragged into the model by the first treatment propagate to harm inference on the effect of the second one, causing a sizeable RMSE inflation throughout the panels.

%\davidcom{Maybe remove Figure \ref{fig:fig4or} and Figure \ref{fig:fig5}? If we keep them we should explain why did we do the exercise, and explain the findings better, I couldn't make sense of the explanation. Frankly, it feels like the simulation section is long enough already without them.}
%Figure \ref{fig:fig5} examines the situation where the treatment effects were weak or not present, akin to panels in Figure \ref{fig:intro2}, with similar relative results.

%%%%%%%%%%%%%%%%%%%%%%%%%%%%%%%%%%%%%%%%%%%%%%%%%%%%%%%%%%%%%%%%%%%%%%

%\subsection{Related Literature}
% 
%Literature on single treatment effect estimation has grown considerably over the last decade, with the main objective of addressing pitfalls in parameter estimation that arise under general variable selection methods. Frequentist approaches particular to the single treatment problem concentrate around varied forms of penalized likelihood (PL), which make them computationally very appealing and suitable for high-dimensional settings. One of the most popular is the Double Machine Learning (DML) approach, presented in \cred{Belloni et al. (2014)}, a two-step procedure by which one analizes the output and exposure equations separately with the objective of identifying which controls affect either output, to then fits a regular model \textit{ex-post} using the union set of selected controls. Its main concern is to avoid estimation bias through fully adjusting for confounding, that is, including into the model at least any control relevant to the treatment or the outcome. \cred{Farrell (2015)} builds upon similar ideas with a doubly-robust estimator attempting to safeguard from model selection mistakes after the double selection step, in the context of multi-valued treatments, resonating also with \cred{Antonelli et al. (2018)}, who use matching on propensity and prognostic scores on a similar direction. \cred{Shortreed and Ertefaie (2017)} employ a two-step approach as well, in this case using adaptive Lasso on the exposure equation, showing improvement on confounder-selection results. See also \cred{Ghosh et al (2015)} with shared and difference Lasso for similar multi-step contributions. \cred{Ertefaie et al. (2015)} use joint likelihood $L_1$ penalization of both equations instead, in order to accommodate for the information shared between the two equations. \cred{Ma et al. (2019)} combine regularization with sufficient dimension reduction into an estimator that achieves good asymptotic properties without requiring model selection consistency. Yet, even if some of these proposals do allow for inference, most of them are essentially designed with sufficient control selection in mind and tackle point estimation only after selection has been conducted, generally without a focus on quantification of uncertainty. \cred{Chernozhukov et al. (2018)} elaborate on the DML technique with a more explicit attempt at inference seeking dependence reduction on the selection step, introducing an additional debiasing operation combining Neyman-orthogonal scores on the first equation, with cross-fitting to address overfitting concerns. Other proposals exist focused on inference for PL-related methods in this particular context, see \cred{Athey et al. (2016)} in the context of binary treatments, see also \cred{Vansteelandt, Bekaert and Claeskens (2012)}, as well as more recent work in \cred{Dukes and Vansteelandt (2019)}. Generally, PL-based proposals are heavily focused on asymptotic results regarding estimation efficiency and their distributional properties, which follow from attaining guarantees of sufficient control selection, i.e. avoiding detection errors of relevant controls connected to either outcome or treatment. In contrast, they are not too concerned with excess inclusion of any other spurious variable. This can be problematic based on the aforementioned reasons, which in turn often limit their oracle performance to a super-model of the true outcome model, inflated in size with unnecessary controls that may relate to the treatment only. It is also worth mentioning that a number of these proposals are not designed for continuous or even multi-valued treatments, or sometimes outcomes.
% 
%From a Bayesian perspective, a number of different proposals also exist. A widely referenced method is Bayesian Adjustment for Confounding (BAC) as presented in \cred{Wang, Parmigiani and Dominici (2012)}, a joint-modelling approach which essentially employs a Bayesian Model Averaging (BMA) approach under a specifically designed model space prior. This prior is separable across controls, and is governed by a hyperparameter $\omega \in [1, \infty)$ that represents the odds of including a control in the outcome model conditional on that same control being included in the exposure model. This is quite helpful since for any finite $\omega$ control inclusion on the outcome model is only encouraged and not forced, allowing for some model flexibility that combines with the accounting of model uncertainty in the averaged point estimate. This entanglement across equations poses a series of questions, however, mainly related as to how one should set such a sensitive hyperparameter, combined with the fact that resulting marginal prior inclusion probabilities are generally high, hence encouraging model size inflation. This can hinder performance notably in high-dimensional settings as over-selection problems may enter into play. Additionally, computational demands arising from joint equation modelling can quickly become insurmountable. Further contributions to BAC include \cred{Lefebvre et al. (2014)} and \cred{Wang et al. (2015)}, which provide some theoretical support as well as further proposals on how to set $\omega$. Other articles build on the approach based on control selection with model averaging: \cred{Talbot et al. (2015)} introduce Bayesian causal effect estimation, a similar method to BAC that incorporates informative priors aimed at deterring excess control inclusion; \cred{Antonelli et al. (2017)} contribute with \textit{guided} BAC as a generalized framework on BAC addressing treatment effect heterogeneity, as well as additional technical questions. Similar methods have also been explored in propensity score analysis, as in \cred{Cefalu et al. (2015)}, but more generally around model uncertainty as well, see e.g. \cred{Ziegler and Dominici (2014)} and \cred{An (2010)}. See also \cred{Jacobi et al. (2016)} for methodological adaptations to dynamic effects in panel data. More recently, \cred{Hahn et al. (2018)} preserve the joint modelling approach but move away from model averaging, by addressing BAC using hierarchical priors, in an attempt to give some prior flexibility and ease computational difficulties via posterior sampling. This is a reparametrization technique designed to achieve debiased point-estimates using regularization priors. \cred{Hahn et al. (2019)} also extend this notion to non-parametric setups with Bayesian Causal Forests. \cred{Antonelli et al. (2019)} propose a spike-and-slab prior formulation with a prior distribution that places low shrinkage to controls associated with the treatment, combined with an Empirical Bayes algorithm for hyperparameter setting. Addressing computational concerns in high-dimensional setups, \cred{Wilson and Reich (2014)} propose Penalized Credible Regions (PCR), stemming from \cred{Bondell and Reich (2012)}. This is a decision theoretic approach that can be formulated as a PL method. It essentially uses the posterior credible region of the outcome regression parameters to form a set of possible models to choose from, and then apply $L_1$-type penalization with lighter penalties to those covariates that are associated to the treatment, as a function of their strength. This relies strongly on a the quality of the posterior mean, and the fact that the treatment is always included casts some difficulty on its single parameter estimation ability as it might sacrifice precision on one parameter to favor models that aggregately perform better predictively. PCR tends to be conservative as well as to dropping variables, depending on the penalty parameter, whose setting is an open end itself. It also introduces the notion of strength of relation between controls and treatment as a determinant of relevance on the outcome equation. Additionally, it naturally allows for the inclusion of multiple simultaneous treatments, which is absent in previously reviewed methods. On the other hand, its PL nature disallows PCR as a method of uncertainty quantification. This proposal has strong ties to the Adaptive Lasso \cred{(Zou, 2006)}, as well as relation to Bayesian Lasso strategies employing shrinkage priors \cred{(Park and Casella, 2008; Hans, 2010)}. All in all, it is worh noting that most current Bayesian proposals are also heavily concerned with omission of relevant controls, and despite extra flexibility they can run into similar problems related to over-selection, as those described under the frequentist paradigm.
% 
%As for literature related to the multiple treatments problem, to the best of our knowledge, the list of available methodology is short. Beyond PCR, perhaps the most notable recent contribution is ACPME by \cred{Wilson et al. (2018)}, a method with strong ties to BAC. This is also a BMA-based algorithm with a specific model prior that incorporates, for each feature, a measure of correlation strength between each control and the set of treatments, this time without imposing sudden jumps in prior probabilities. Similarly to BAC, the philosophy of ACPME is to tilt prior probabilities towards models fully adjusting for confounding. Again, this can put a good fraction of prior mass on super-models of the true outcome model, as feature inclusion can only be encouraged and, hence, is not designed to overcome over-selection problems. More so taking into account that marginal prior inclusion probabilities are capped below at 1/2 by construction. It is also unclear whether it can perform well in high dimensions or with a large number of treatments. Finally, in the context of treatment heterogeneity, a mention to recent work on the Debiased Orthogonal Lasso by \cred{Semenova et al. (2020)} that can potentially extend the frequentist DML scheme to the multiple treatments, although it might not be the focus of its current formulation.


%%%%%%%%%%%%%%%%%%%%%%%%%%%%%%%%%%%%%%%%%%%%%%%%%%%%%%%%%%%%%%%%%%%%%%

%%%%%%%%%%%%%%%%%%%%%%%%%%%%%%%%%%%%%%%%%%%%%%%%%%%%%%%%%%%%%%%%%%%%%%
\section{Discussion} \label{sec:discuss}
%\davidcom{I added a bit of discussion, this needs to be refined, do not refrain from butchering at your leisure.}

The two main ingredients in our proposal are learning from data whether and to what extent control inclusion/exclusion should be encouraged to improve multiple treatment inference, and a convenient computational strategy to render the approach practical.
This is in contrast to current literature, which encourages the inclusion of certain controls to avoid under-selection biases but can run into serious over-selection bias and variance, as we have illustrated.
By learning the relative importance of potential confounders, as in our CIL framework, one may bypass this problem.

These issues are practically relevant, e.g. in the salary data we showed that one may fail to detect a negative association between the black race and salary. Further, the proposed Bayesian framework naturally allows for posterior predictive inference on functions that depend on multiple parameters, such as the variation in salary jointly associated with multiple treatments. Interestingly, our analyses revealed a reduced association between salary and potentially discriminatory factors such as gender or race in 2019 relative to 2010, as well as a lesser heterogeneity across states.
These results are conditional on controls that include education, employment and other characteristics that affect salary. That is, our results reveal lower salary discrepancies in 2019 between races/genders, provided that two individuals have the same characteristics (and that they were hired in the first place).
This analysis offers a complementary view to analyses that are unadjusted by controls, and which may reveal equally interesting information.
For example, if females migrated towards lower-paying occupational sections in 2019 and received a lower salary as a consequence, this would not be detected by our analysis, but would be revealed by an unadjusted analysis.

To keep our exposition simple, we focused our discussion on linear treatment effects, but it is possible to extend our framework to non-linear effects and interactions between treatments, e.g. via splines or other suitable non-linear bases.


%%%%%%%%%%%%%%%%%%%%%%%%%%%%%%%%%%%%%%%%%%%%%%%%%%%%%%%%%%%%%%%%%%%%%%%
%\section{Appendix}
%
%Text.

%%%%%%%%%%%%%%%%%%%%%%%%%%%%%%%%%%%%%%%%%%%%%%%%%%%%%%%%%%%%%%%%%%%%%%
%%\bigskip
%%\begin{center}
%%{\large\bf SUPPLEMENTARY MATERIAL}
%%\end{center}
%% 
%%\begin{description}
%% 
%%\item[Title: \texttt{supplement}.] Supplementary material, including proofs, computational methodology, and further complementary numerical results. (.pdf file)
%% 
%%\item[Title: \texttt{cps\_00004.cbk}.] List and description of variables employed in the CPS salary data. (.txt file)
%% 
%%\item[\texttt{R}-code for CPS dataset and numerical experiments:] \texttt{R}-code to reproduce the results in this article, and to obtain the necessary CPS data. (.zip file, available upon request)
%% 
%%%\item[\texttt{R}-code to implement CIL:] available at \texttt{https://github.com/mtorrens/rcil}. %R-package MYNEW containing code to perform the diagnostic methods described in the article. The package also contains all datasets used as examples in the article. (GNU zipped tar file)
%% 
%%%\item[CPS data set:] \cred{Format TBD.} (.\cred{XXX} file) %Data set used in the illustration of MYNEW method in Section~ 3.2. (.txt file)
%% 
%%\end{description}


\newpage
\section*{Supplementary material}



\renewcommand{\thethm}{S\arabic{theorem}}
\renewcommand{\theprop}{S\arabic{prop}}
\renewcommand{\thelemma}{S\arabic{lemma}}
\renewcommand{\thefigure}{S\arabic{figure}}
\renewcommand{\thetable}{S\arabic{table}}
\renewcommand{\thesection}{S\arabic{section}}
\renewcommand{\theequation}{S\arabic{equation}}

\section{Proofs}

\subsection{Proof of Proposition \ref{prop:one}} \label{sec:proof_prop1}

Let $p(\by \mid \bgamma, \bdelta, \btheta) = p(\by \mid \bgamma, \bdelta)$, then
\begin{eqnarray}
\nabla_{\btheta} \log p(\by \mid \btheta) &=& \frac{\nabla_{\btheta} p(\by \mid \btheta)}{p(\by \mid \btheta)}  \nonumber \\
&=& \frac{\nabla_{\btheta} \sum_{(\bgamma, \bdelta)} p(\by \mid \bgamma, \bdelta, \btheta) p(\bgamma, \bdelta \mid \btheta)}{p(\by \mid \btheta)} \nonumber \\
&=& \frac{\sum_{(\bgamma, \bdelta)} p(\by \mid \bgamma, \bdelta) \nabla_{\btheta}  p(\bgamma, \bdelta \mid \btheta)}{p(\by \mid \btheta)} \nonumber \\
&=& \sum_{(\bgamma, \bdelta)} \frac{p(\by \mid \bgamma, \bdelta, \btheta)}{p(\by \mid \btheta)} \frac{p(\bgamma, \bdelta \mid \btheta)}{p(\bgamma, \bdelta \mid \btheta)} \nabla_{\btheta} p(\bgamma, \bdelta \mid \btheta) \nonumber \\
&=& \sum_{(\bgamma, \bdelta)} \frac{\nabla_{\btheta} p(\bgamma, \bdelta \mid \btheta)}{p(\bgamma, \bdelta \mid \btheta)} p(\bgamma, \bdelta \mid \by, \btheta) \nonumber \\
&=& \sum_{(\bgamma, \bdelta)} p(\bgamma, \bdelta \mid \by, \btheta) \nabla_{\btheta} \log p(\bgamma, \bdelta \mid \btheta).
\end{eqnarray}
If, further, the model prior satisfies $p(\bgamma, \bdelta \mid \btheta) = \prod_{t=1}^{T} p(\delta_{t}) \prod_{j=1}^{J} p(\gamma_{j} \mid \btheta)$, then
\begin{align*}
\nabla_{\btheta} \log p(\bgamma, \bdelta \mid \btheta) = \sum_{j=1}^{J} \nabla_{\btheta} \log p(\gamma_{j} \mid \btheta),
\end{align*}
and so
\begin{align*}
\nabla_{\btheta} \log p(\by \mid \btheta) = \sum_{j=1}^{J} \sum_{(\bgamma, \bdelta)} \nabla_{\btheta} \log p(\gamma_{j} \mid \btheta) p(\bgamma, \bdelta \mid \by, \btheta) = \sum_{j=1}^{J} \E\left[ \nabla_{\btheta} \log p(\gamma_{j} \mid \btheta) \mid \by, \btheta \right].
\end{align*}
\hfill $\blacksquare$

\subsection{Proof of Corollary \ref{prop:two}} \label{sec:proof_prop2}

The empirical Bayes estimate defined by \eqref{eq:marglik1} writes
\begin{align*}
\bthetaeb = \argmax_{\btheta \in \mathbb{R}^{T+1}} \log p(\by \mid \btheta) = \argmax_{\btheta \in \mathbb{R}^{T+1}} \log \sum_{(\bgamma, \bdelta)} p(\by \mid \bgamma, \bdelta) p(\bgamma, \bdelta \mid \btheta).
\end{align*}
For short, denote $H(\btheta) = p(\by \mid \btheta)$ and $h_{j}(\btheta) = p_{j}(\gamma_j \mid \btheta)$, where generically $\nabla_{\btheta} \log H(\btheta) = \nabla_{\btheta} H(\btheta) / H(\btheta)$. Under the assumptions of Corollary \ref{prop:two}
\begin{eqnarray}
\nabla_{\btheta} H(\btheta) &=& \sum_{(\bgamma, \bdelta)} p(\by \mid \bgamma, \bdelta) p(\bdelta) \nabla_{\btheta} \prod_{j=1}^{J} h_{j}(\btheta) \nonumber \\
&=& \sum_{(\bgamma, \bdelta)} p(\by \mid \bgamma, \bdelta) p(\bdelta) \sum_{j=1}^{J} \left( \nabla_{\btheta} h_{j}(\btheta) \prod_{j \neq l} h_{l}(\btheta) \right).  \label{eq:prop2_eq1}
\end{eqnarray}
Denoting $\mb{f}_{j} = \tr{(1, f_{j,1}, \dots, f_{j,T})} $, direct algebra gives
\begin{align}
\nabla_{\btheta} h_{j}(\btheta) = \nabla_{\btheta}\left\{ \pi_{j}(\btheta)^{\gamma_j} (1-\pi_{j}(\btheta))^{1-\gamma_j} \right\} = (1-2\rho) \mb{f}_{j} (\gamma_j - \pi_{j}(\btheta)) h_{j}(\btheta), \label{eq:prop2_eq2}
\end{align}
since
\begin{align}
\nabla_{\btheta} \pi_{j}(\btheta) = (1-2\rho) \mb{f}_{j}  \pi_{j}(\btheta) (1-\pi_{j}(\btheta)). \label{eq:prop2_eq3}
\end{align}
Then, replacing \eqref{eq:prop2_eq2} into \eqref{eq:prop2_eq1}
\begin{eqnarray}
\nabla_{\btheta} H(\btheta) &=& \sum_{(\bgamma, \bdelta)} p(\by \mid \bgamma, \bdelta) p(\bdelta) \sum_{j=1}^{J} (1-2\rho) \mb{f}_{j} (\gamma_j - \pi_{j}(\btheta)) \prod_{j=1}^{J} f_{j}(\btheta) \nonumber \\
%&=& \sum_{j=1}^{J} (1-2\rho) \mb{f}_{j} (\gamma_j - \pi_{j}(\btheta))  \sum_{(\bgamma, \bdelta)} p(\by \mid \bgamma, \bdelta) p(\bdelta) p(\bgamma \mid \btheta) \nonumber \\
&=& \sum_{j=1}^{J} (1-2\rho) \mb{f}_{j} \sum_{(\bgamma, \bdelta)} (\gamma_j - \pi_{j}(\btheta)) p(\by \mid \bgamma, \bdelta) p(\bdelta, \bgamma \mid \btheta) \nonumber \\
&=& \sum_{j=1}^{J} (1-2\rho) \mb{f}_{j} \left[ (1-\pi_{j}(\btheta)) \sum_{(\bgamma, \bdelta): \gamma_{j}=1} p(\by, \bdelta, \bgamma \mid \btheta) - \pi_{j}(\btheta) \sum_{(\bgamma, \bdelta): \gamma_{j}=0} p(\by, \bdelta, \bgamma \mid \btheta) \right]. \nonumber
\end{eqnarray}
Finally
\begin{eqnarray}
\nabla_{\btheta} \log H(\btheta) &=& \frac{\nabla_{\btheta} H(\btheta)}{H(\btheta)} \nonumber \\
&=& \sum_{j=1}^{J} (1-2\rho) \mb{f}_{j} \left[ (1-\pi_{j}(\btheta)) \frac{\sum_{(\bgamma, \bdelta): \gamma_{j}=1} p(\by, \bdelta, \bgamma \mid \btheta)}{\sum_{(\bgamma, \bdelta)} p(\by, \bdelta, \bgamma \mid \btheta)} - \pi_{j}(\btheta) \frac{\sum_{(\bgamma, \bdelta): \gamma_{j}=0} p(\by, \bdelta, \bgamma \mid \btheta)}{\sum_{(\bgamma, \bdelta)} p(\by, \bdelta, \bgamma \mid \btheta)} \right] \nonumber \\
&=& \sum_{j=1}^{J} (1-2\rho) \mb{f}_{j} \left[ (1-\pi_{j}(\btheta)) P(\gamma_{j} = 1 \mid \by, \btheta) - \pi_{j}(\btheta) (1 - P(\gamma_{j} = 1 \mid \by, \btheta)) \right] \nonumber \\
&=& (1-2\rho) \sum_{j=1}^{J} \mb{f}_{j} \left[ P(\gamma_{j} = 1 \mid \by, \btheta) - \pi_{j}(\btheta) \right]. \nonumber
\end{eqnarray}
\hfill $\blacksquare$

\subsection{Proof of Proposition \ref{prop:three}}  \label{sec:proof_prop3}

Consider the right-hand side in \eqref{eq:marglik1},
\begin{align}
\argmax_{\btheta \in \mathbb{R}^{T+1}} \sum_{(\bdelta, \bgamma)} p_{u}(\bdelta, \bgamma \mid \by) p(\bdelta, \bgamma \mid \btheta)
\label{eq:l1eq1}
\end{align}
where $p_{u}(\bdelta, \bgamma \mid \by)$ are the posterior probabilities under a uniform prior $p_{u}(\bdelta, \bgamma) \propto 1$.

We seek to set the parameters $s_t$ and $q_j$ in the approximation
\begin{align}
\hat{p}_{u}(\bdelta, \bgamma \mid \by) = \prod_{t=1}^{T} \text{Bern}(\delta_{t}; s_{t}) \prod_{j=1}^{J} \text{Bern}(\gamma_{j}; q_{j})
\nonumber
\end{align}
using Expectation Propagation. That is, setting and $\mb{q} = (q_1, \dots, q_J)$ such that
\begin{align*}
\mb{q}^{\textsc{ep}} = \argmax_{\mb{q} \in [0,1]^{J}} \sum_{(\bgamma, \bdelta)} p_{u}(\bdelta, \bgamma \mid \by) \log \left(\prod_{t=1}^{T} s_{t}^{\delta_{t}} (1-s_{t})^{1-\delta_{t}} \prod_{j=1}^{J} q_{j}^{\gamma_{j}} (1-q_{j})^{1-\gamma_{j}}  \right).
\end{align*}
and analogously for $\mb{s} = (s_1, \dots, s_T)$. Proceeding elementwise, we derive
\begin{eqnarray}
q_{j}^{\textsc{ep}} &:=& \argmax_{q_j \in [0,1]} \sum_{(\bgamma, \bdelta)} p_{u}(\bdelta, \bgamma \mid \by) \times \nonumber \\
&& \times \left( \sum_{j=1}^{J} [\gamma_j \log q_{j} + (1 - \gamma_{j}) \log(1-q_{j})] + \sum_{t=1}^{T} [\delta_t \log s_{j} + (1 - \delta_{t}) \log(1-s_{t})] \right) \nonumber \\
&=& \argmax_{q_j \in [0,1]} \sum_{(\bgamma, \bdelta)} p_{u}(\bdelta, \bgamma \mid \by) \left( \sum_{j=1}^{J} [\gamma_j \log q_{j} + (1 - \gamma_{j}) \log(1-q_{j})] \right) \nonumber \\
&=& \arg \max_{q_j \in [0,1]} \sum_{j=1}^{J} \sum_{(\bgamma, \bdelta)} p_{u}(\bdelta, \bgamma \mid \by) \left[ \gamma_j \log q_j + (1 - \gamma_j) \log (1 - q_j) \right]. \nonumber
\end{eqnarray}
Optimizing this expression yields
\begin{eqnarray}
\frac{\partial}{\partial q_j} = 0 &\Leftrightarrow& \sum_{(\bgamma, \bdelta)} p_{u}(\bdelta, \bgamma \mid \by) \left( \frac{\gamma_j}{q_{j}^{\textsc{ep}}} - \frac{1 - \gamma_j}{1 - q_{j}^{\textsc{ep}}} \right) = 0 \nonumber \\
&\Leftrightarrow& \frac{1}{q_{j}^{\textsc{ep}}} \sum_{(\bgamma, \bdelta): \gamma_j = 1} p_{u}(\bdelta, \bgamma \mid \by) - \frac{1}{1 - q_{j}^{\textsc{ep}}} \sum_{(\bgamma, \bdelta): \gamma_j = 0} p_{u}(\bdelta, \bgamma \mid \by) = 0 \nonumber \\
&\Leftrightarrow& \frac{P_{u}(\gamma_j = 1 \mid \by)}{q_{j}^{\textsc{ep}}} - \frac{P_{u}(\gamma_j = 0 \mid \by)}{1 - q_{j}^{\textsc{ep}}} = 0 \nonumber \\
&\Leftrightarrow& q_{j}^{\textsc{ep}} = P_{u}(\gamma_j = 1 \mid \by) = P(\gamma_j = 1 \mid \by, \btheta = \mb{0}). \label{eq:prop3_eq2}
\end{eqnarray}
With the same exact procedure one analogously obtains $s_{t}^{\textsc{ep}} = P_{u}(\delta_t = 1 \mid \by)$. Let
\begin{align*}
h(\bdelta) := \prod_{T=1}^{T} \text{Bern}(\delta_t; s_{t}^{\textsc{ep}}) \prod_{j=1}^{J} \text{Bern}(\delta_t; \pi_{t}) = \prod_{t=1}^{T} \left[ s_{t}^{\textsc{ep}} \pi_t \right]^{\delta_t} \left[ (1-s_{t}^{\textsc{ep}}) (1-\pi_t) \right]^{1-\delta_t},
\end{align*}
which is independent of $\btheta$, and where $\pi_{t}$ is the marginal prior inclusion probability within our framework. Then, implementing the approximation \eqref{eq:prop3_eq2} into \eqref{eq:l1eq1} gives
\begin{eqnarray}
\bthetaep &:=& \argmax_{\btheta \in \mathbb{R}^{T+1}} \sum_{(\bgamma, \bdelta)} h(\bdelta) \prod_{j=1}^{J} \text{Bern}(\gamma_j; q_{j}^{\textsc{ep}}) \prod_{j=1}^{J} \text{Bern}(\gamma_j; \pi_j(\btheta)) \nonumber \\
&=& \argmax_{\btheta \in \mathbb{R}^{T+1}} \sum_{(\bgamma, \bdelta)} h(\bdelta) \prod_{j=1}^{J} \left[ q_{j}^{\textsc{ep}} \pi_j(\btheta) \right]^{\gamma_j} \left[ (1-q_{j}^{\textsc{ep}}) (1-\pi_j(\btheta)) \right]^{1-\gamma_j}. \label{eq:prop3_eq3}
\end{eqnarray}
Note that the product in the RHS of \eqref{eq:prop3_eq3} defines a probability distribution on $(\delta_1, \dots, \delta_{T}, \gamma_1, \dots, \gamma_{J})$ with independent components, hence the sum is the normalizing constant of such distribution. Thus, this constant is just the product of the univariate normalizing constants. The univariate normalizing constant of each Bernouilli is then
\begin{align*}
q_{j}^{\textsc{ep}} \pi_j(\btheta) + (1-q_{j}^{\textsc{ep}}) (1-\pi_j(\btheta))
\end{align*}
for every $q_{j}$, and similarly $s_{t}^{\textsc{ep}} \pi_t + (1-s_{t}^{\textsc{ep}}) (1-\pi_t)$ for every $s_{t}$.
Hence, replacing into \eqref{eq:prop3_eq3} we obtain
\begin{eqnarray}
\bthetaep &:=& \argmax_{\btheta \in \mathbb{R}^{T+1}} \prod_{j=1}^{J} \left\{ q_{j}^{\textsc{ep}} \pi_j(\btheta) + (1-q_{j}^{\textsc{ep}}) (1-\pi_j(\btheta)) \right\} \prod_{t=1}^{T} \left\{ s_{t}^{\textsc{ep}} \pi_{t} + (1-s_{t}^{\textsc{ep}}) (1-\pi_{t}) \right\}. \nonumber \\
&=& \argmax_{\btheta \in \mathbb{R}^{T+1}} \sum_{j=1}^{J} \log \left( q_{j}^{\textsc{ep}} \pi_j(\btheta) + (1-q_{j}^{\textsc{ep}}) (1-\pi_j(\btheta)) \right). \nonumber
\end{eqnarray}
\hfill $\blacksquare$

%\subsection{Gradient of the log of function optimized in \eqref{eq:ep_eq0} in Proposition \ref{prop:three}} \label{sec:gradEP}
\subsection{Gradient of the function optimized in \eqref{eq:ep_eq0} in Proposition \ref{prop:three}} \label{sec:gradEP}

From \eqref{eq:ep_eq0}, for a given set of $q_{j}$ we have
%\begin{eqnarray}
\begin{align}
\bthetaep = \argmax_{\btheta \in \mathbb{R}^{T+1}} \sum_{j=1}^{J} \log \left( q_{j}^{\textsc{ep}} \pi_j(\btheta) + (1-q_{j}^{\textsc{ep}}) (1-\pi_j(\btheta)) \right). \label{eq:prop3b_eq1}
\end{align}
%\bthetaep &=& \argmax_{\btheta \in \mathbb{R}^{T+1}} \prod_{j=1}^{J} \left[ q_{j} \pi_{j}(\btheta) + (1-q_{j}) (1-\pi_{j}(\btheta)) \right] \nonumber \\
%&=& \argmax_{\btheta \in \mathbb{R}^{T+1}} \sum_{j=1}^{J} \log \left( q_{j} \pi_{j}(\btheta) + (1-q_{j}) (1-\pi_{j}(\btheta)) \right). \nonumber \label{eq:prop3_eq1}
%\end{eqnarray}
We are interested in computing the gradient of the function being optimized in \eqref{eq:prop3b_eq1}. Denote $h_{j}(\btheta) := q_{j} \pi_{j}(\btheta) + (1-q_{j}) (1-\pi_{j}(\btheta))$ for short. Simple algebra provides
\begin{align*}
\nabla_{\btheta} h_{j}(\btheta) = (2 q_j - 1) \nabla_{\btheta} \pi_{j}(\btheta).
\end{align*}
From \eqref{eq:prop2_eq3} we recover the remaining gradient in the last expression and derive
\begin{align*}
\nabla_{\btheta} \log h_{j}(\btheta) = \frac{\nabla_{\btheta} h_{j}(\btheta)}{h_{j}(\btheta)} = \frac{2 q_{j} - 1}{h_{j}(\btheta)} \left[ (1-2\rho) \mb{f}_{j}  \pi_{j}(\btheta) (1-\pi_{j}(\btheta)) \right],
\end{align*}
where $\mb{f}_{j} = \tr{(1, f_{j,1}, \dots, f_{j,T})}$, and so the gradient for the expression in \eqref{eq:prop3b_eq1} is simply
\begin{align*}
\nabla_{\btheta} \sum_{j=1}^{J} \log h_{j}(\btheta) = (1-2\rho) \sum_{j=1}^{J} \mb{f}_{j} \frac{\pi_{j}(\btheta) (1-\pi_{j}(\btheta))}{h_{j}(\btheta)}.
\end{align*}
\hfill $\blacksquare$

\section{Computational methods}

\subsection{Product MOM non-local prior}

Figure \ref{fig:nlpmom} illustrates the density of the product MOM non-local prior of \cite{Johnson12}.

\begin{figure}[h]
\centering
\includegraphics[scale=0.85]{nlpmom.pdf} 
\caption{Prior density $p(\alpha_{t} \mid \delta_{t} = 1, \phi = 1)$ of the MOM non-local prior, with $\tau = 0.348$.}
\label{fig:nlpmom}
\end{figure}

\subsection{Numerical computation of the marginal likelihood for non-local priors} \label{subsec:nlp_approx}

Briefly, denote by $p^\textsc{n}(\alpha_t \mid \delta_t=1, \phi) = \text{N}(\alpha_t; 0, \tau \phi)$ independent Gaussian priors for $t=1,..,T$, and similarly $p^\textsc{n}(\beta_j \mid \gamma_j=1, \phi) = \text{N}(\beta_j; 0, \tau \phi)$ for $j=1,\ldots, J$. Proposition 1 in \cite{Rossell17} shows that the following identity holds exactly
\begin{align*}
p(\by \mid \bgamma, \bdelta)= p^\textsc{n}(\by \mid \bgamma, \bdelta) \E^\textsc{n} \left[ \prod_{t=1}^{T} \frac{\alpha_t^2}{\tau\phi} \prod_{j=1}^{J} \frac{\beta_j^2}{\tau\phi} \mid \by, \bgamma, \bdelta \right]
\end{align*}
where $p^\textsc{n}(\by \mid \bgamma, \bdelta)$ is the integrated likelihood under $p^\textsc{n}(\balpha, \bbeta)$, and $\E^\textsc{n}[\cdot]$ denotes the posterior expectation under $p^\textsc{n}(\balpha, \bbeta \mid \by, \bgamma, \bdelta)$.
%For certain models this is analytically tractable. The most common tractable case is when $y_i \mid \eta_i,\phi \sim N(\eta_i,\phi)$  and Gaussian prior is used for $\balpha$ and $\bbeta$ conditionally on being non-zero \omcom{add some reference for where this formula is to be found}. For the examples in this article, we are interested in the case of Gaussian likelihood but non-Gaussian priors, in particular the pMOM prior discussed in Section \ref{sec:model}, for which the expression is intractable \omcom{Om2D: I think it is expensive not entirely intractable right? please edit to make this correct}
To estimate $p^\textsc{n}(\by \mid \bgamma, \bdelta)$ for non-Gaussian outcomes we use a Laplace approximation. Regarding the second term, we approximate it by a product of expectations, which \cite{Rossell20a} showed leads to the same asymptotic properties and typically enjoys better finite-$n$ properties than a Laplace approximation.

\subsection{Numerical optimization in search of $\hat{\theta}^{\text{EB}}$ and $\hat{\theta}^{\text{EP}}$} \label{sec:algorithm}

%\davidcom{Miquel, please move the algorithm below and rest of this section to the supplement. Also, indicate the grid size in Step 3 of the algorithm}
Algorithm \ref{alg:one} describes our method to estimate $\hthetaep$ and $\hthetaeb$. We employ the quasi-Newton BFGS algorithm to optimize the objective function. For $\hthetaeb$, we use the gradients from Corollary \ref{prop:two}, while the Hessian is evaluated numerically using line search, with the \texttt{R} function \texttt{nlminb}. Note, however, that obtaining $\hthetaeb$ requires sampling models from their posterior distribution for each $\btheta$, which is impractical, to then obtain posterior inclusion probabilities required by \eqref{eq:prop_two}. Instead, we restrict attention to the models $M$ sampled for either $\btheta = \mb{0}$ or $\btheta = \hthetaep$ in order to avoid successive MCMC runs at every step, relying on the relative regional proximity between the starting point $\hthetaep$ and $\hthetaeb$. This proximity would ensure that $M$ contains the large majority of models with non-negligible posterior probability under $\hthetaeb$. For $\hthetaep$, we use employ the same BFGS strategy using gradient computed in \ref{sec:gradEP}, with numerical evaluation of the Hessian. This computation requires only one MCMC run at $\btheta = \mb{0}$, which allows us to use grid search to avoid local optima. As for the size of the grid, we let the user specify what points are evaluated. For $K$ points in the grid one must evaluate the log objective function $K^{T+1}$ times, so we recommend to reduce the grid density as $T$ grows. By default, we evaluate every integer in the grid assuming $T$ is not large, but preferably we avoid coordinates greater than 10 in absolute value, as in our experiments it is very unlikely that any global posterior mode far from zero is isolated, i.e. not reachable by BFGS by starting to its closest point in the grid. Additionally, even if that were the case, numerically it makes no practical difference, considering that marginal inclusion probabilities are bounded away from zero and one regardless.

\begin{algorithm}[H] \label{alg:one}
%\KwIn{a,b,c,d}
\KwOut{$\hthetaep$ and $\hthetaeb$}
Obtain $B$ posterior samples $(\bgamma, \bdelta)^{(b)} \sim p(\bgamma, \bdelta \mid \by, \btheta = \mb{0})$ for $b = 1,\ldots,B$. Denote by $M^{(0)}$ the corresponding set of unique models.\\

Compute $s_{t} = P(\delta_{t} = 1 \mid \by, \btheta = \mb{0})$ and $q_{j} = P(\gamma_{j} = 1 \mid \by, \btheta = \mb{0})$.\\

%Compute estimates $\hat{q}_{j}$ using $M^{(0)}$ as a model dictionary.\\

Condut a grid search for $\hthetaep$ around $\btheta={\bf 0}$. Optimize \eqref{eq:ep_eq0} with the BFGS algorithm initialized at the grid's optimum.\\

Obtain $B$ posterior samples $(\bgamma, \bdelta)^{(b)} \sim p(\bgamma, \bdelta \mid \by, \btheta = \hthetaep)$. Denote by $M^{(1)}$ the corresponding set of unique models. Set $M = M^{(0)} \cup M^{(1)}$.\\

Initialize search for $\hthetaeb$ at $\hthetaep$. Use the BFGS algorithm to optimize \eqref{eq:marglik1}, restricting the sum to $(\bdelta,\bgamma) \in M$.
\caption{Obtaining $\bthetaep$ and $\bthetaeb$}
\end{algorithm}

\begin{comment}

% Algorithm example
%\begin{algorithm}
\begin{algorithm}[H] \label{alg:one}
\SetAlgoLined
\KwResult{$\hthetaep$ and/or $\hthetaeb$}
 \STATE Set $\btheta = \mb{0}$ and obtain $B$ posterior samples $(\gamma,\delta)^{(b)} \sim p(\bgamma, \bdelta \mid \by, \btheta = \mb{0})$ for $b = 1,\ldots,B$. Denote by $M^{(0)}$ the corresponding set of unique models. $M^{(0)}$\;
 Compute estimates $\hat{q}_{j} = \sum_{(\bdelta, \bgamma): \gamma_j = 1; (\bdelta, \bgamma) \in M^{(0)}} p(\bdelta, \bgamma \mid \by, \btheta = \mb{0})$\;
 Approximate the optimum of (log) objective function in \eqref{eq:ep_eq0} $\longrightarrow \hthetaep$\;
 Conduct new model search conditional on $\btheta = \hthetaep$, visiting models $M^{(1)}$\;
 Set $M = M^{(0)} \cup M^{(1)}$, precompute and store $p(\by \mid \bdelta, \bgamma)$ for every $(\bdelta, \bgamma) \in M$\;
 Apply BFGS method on (log) objective function initialised at $\btheta = \hthetaep$\;
 %Return the optimum found by BFGS $\longrightarrow \hat{\btheta} := \hthetaeb$\;
 Return the optimum found by BFGS $\longrightarrow \hthetaeb$\;
 \caption{Quick approximation to $\bthetaeb$}
\end{algorithm}

\end{comment}

\section{Supplementary Results}


\subsection{Illustration of the EB and EP objective functions}

\begin{figure}[h]
\centering
\includegraphics[scale=0.45]{thEB.pdf} 
\includegraphics[scale=0.45]{thEP.pdf} 
\caption{Empirical Bayes (left) and Expectation-Propagation (right) objective functions \eqref{eq:marglik1} and \eqref{eq:ep_eq0} in the single treatment case ($T=1$). Here, $\hthetaeb = (-2.43, 3.19)$ and $\hthetaep = (-2.34, 3.09)$, for $n=100$ and $J=49$, for the first data realization for the simulation design displayed in the center-left panel of Figure \ref{fig:intro2} with three confounders. See Section \ref{subsec:singleT} for further details.}
\label{fig:thEPEB}
\end{figure}


Figure \ref{fig:thEPEB} shows the Empirical Bayes objective function in \eqref{eq:marglik1} and \eqref{eq:ep_eq0} in a simulated dataset with a single treatment. A bimodality is appreciated in the left panel.

\subsection{Salary survey: generation of augmented datasets} \label{sec:fakepreds_supp}

For both amounts $K_1=100$ and $K_2=200$ of artificial predictors, the simulation protocol was the same. Every artificial control $\mb{z}_{k} \in \mathbb{R}^{n}$, for $k=1,\dots,100$ or $k=1,\dots,200$ respectively, was simulated to correlate to one individual treatment, according to which subset said control was assigned to, correlating only indirectly to the rest of treatments. In particular, we drew elements of $\mb{z}_{k}$ from $z_{i,k} \mid d_{i,t} = 1 \sim \text{N}(1.5, 1)$, and $z_{i,k} \mid d_{i,t} = 0 \sim \text{N}(-1.5, 1)$, where $\bd_{t}$ denotes the corresponding column in the treatment matrix associated to the given $\mb{z}_{k}$. 

\subsection{Further results on salary survey} \label{sec:salary_supp}

\begin{figure}[h]
\centering
\includegraphics[scale=0.75]{figApp1Btop.pdf}\\
\includegraphics[scale=0.75]{figApp1Bbot.pdf}
\caption{Inference for treatment variables ``hispanic'' (top) and ``born in Latin America'' (bottom) in 2010 and 2019; see Section \ref{sec:cps}. Read caption to Figure \ref{fig:intro1} to read this figure.}
\label{fig:figApp1B}
\end{figure}

Figure \ref{fig:figApp1B} follows Figure \ref{fig:intro1} by showing the results for the other two treatments: Hispanic ethnicity, and birthplace in Latin America.


\subsection{Model selection results in simulation study}

\begin{figure}[h]
\centering
\begin{tabular}{ccc}
$\alpha = 1$ & $\alpha = 1/3$ & $\alpha = 0$ \\
\includegraphics[scale=0.57]{fig1Bp1.pdf} &
\includegraphics[scale=0.57]{fig1Bp2.pdf} &
\includegraphics[scale=0.57]{fig1Bp3.pdf} \\
\includegraphics[scale=0.57]{fig1Bp4.pdf} &
\includegraphics[scale=0.57]{fig1Bp5.pdf} &
\includegraphics[scale=0.57]{fig1Bp6.pdf}
\end{tabular}
\caption{To be read vertically in relation to Fig. \ref{fig:intro2}. The top panels show the average outcome model size across levels of confounding, divided by the true model size (i.e. 1 indicates that it matches the true model size). The bottom panels show the probability of selecting the treatment using a 0.05 P-value cut-off for DML, and for Bayesian methods the treatment is included when marginal posterior inclusion probability is $>$1/2. LASSO does not appear in these panels as its not designed for inference.}
\label{fig:fig1b}
\end{figure}

Figure \ref{fig:fig1b} summarizes model selection results for the simulations described in Figure \ref{fig:intro2}.


\subsection{Simulations under growing dimensionality ($T=1$)}

\begin{figure}[h]
\centering
\begin{tabular}{ccc}
$n = 50$; $J+T = 25$ & $n = 100$; $J+T = 100$ & $n = 100$; $J+T = 200$ \\
\includegraphics[scale=0.66]{fig2Ap1.pdf} &
\includegraphics[scale=0.66]{fig2Ap2.pdf} &
\includegraphics[scale=0.66]{fig2Ap3.pdf} 
\end{tabular}
\caption{Single treatment parameter RMSE (relative to Oracle OLS) based on $R=250$ simulated datasets for each level of confounding.
%Here, leftmost panel shows results for $\{ N=50, J+T=25 \}$, center-left for $\{ N=100, J+T=50 \}$, center-right for $\{ N=100, J+T=100 \}$, and rightmost for $\{ N=100, J+T=200 \}$. 
In all panels, $\alpha = 1$ and $\norm{\bgamma}_{0} = 6$. We show the empirical Bayes version CIL only in the right panel, for the other panels results are undistinguishable relative to EP.}
\label{fig:singletreat_growingdim}
\end{figure}


Figure \ref{fig:singletreat_growingdim} studies the effect of growing number of covariates on inference, specifically for $J+T=25$, 100 and 200.


\subsection{Testing CIL to different amounts of confounders for $T=1$}

\begin{figure}[h]
\centering
\begin{tabular}{ccc}
$\norm{\bgamma}_{0} = 6$ & $\norm{\bgamma}_{0} = 12$ & $\norm{\bgamma}_{0} = 18$ \\
\includegraphics[scale=0.66]{fig3Ap1.pdf} &
\includegraphics[scale=0.66]{fig3Ap2.pdf} &
\includegraphics[scale=0.66]{fig3Ap3.pdf}
\end{tabular}
\caption{Single treatment parameter RMSE (relative to Oracle OLS) based on $R=250$ simulated datasets for each level of confounding reported, as described in Figure \ref{fig:intro2}. In all panels, $n=100$, $J+T=100$ and $\alpha = 1$. Sudden general improvement at the right end of center and right panels is due to a sharper deterioration of oracle OLS RMSE at complete confounding relative to other methods.}
\label{fig:fig3}
\end{figure}

Figure \ref{fig:fig3} shows the effect of having various amounts of active confounders. The results look consistent to the effects reported in Figures \ref{fig:intro2} and \ref{fig:singletreat_growingdim}, which were magnified for large amounts of active confounders. These are really challenging situations to tackle since the tested methods aim at model sparsity, while the true model size is relatively large. Although our method still performed at oracle rates in low-confounding scenarios, its relative performance was compromised for the highest levels of confounding. This occurred in part because accurate point estimation in \eqref{eq:lasso} became increasingly harder as the correlation between covariates strengthened, which in turn influenced the ability of the algorithm to calibrate $\btheta$ reliably. Even in these hard cases, however, its performance was not excessively far to the best competing method, while it clearly outperformed BMA on all of them.




%%%%%%%%%%%%%%%%%%%%%%%%%%%%%%%%%%%%%%%%%%%%%%%%%%%%%%%%%%%%%%%%%%%%%%
%\section{References}

%We hope you've chosen to use BibTeX!\ If you have, please feel free to use the package natbib with any bibliography style you're comfortable with. The .bst file agsm has been included here for your convenience. 

%\bibliographystyle{Chicago}
\bibliographystyle{agsm}

%\bibliography{Bibliography-MM-MC}
\bibliography{references}
\end{document}


\usepackage{amsmath, amsfonts, amssymb}
\usepackage{cancel}


\usepackage{maths_min}

\usepackage[pdftex, breaklinks=true, linktocpage,
	colorlinks=true, urlcolor=blue, linkcolor=blue, citecolor=red,
	pdfauthor={Harold Erbin}
]{hyperref}
\usepackage[all]{hypcap}


\pagestyle{plain}
\graphicspath{{images/}}

\numberwithin{equation}{section}

\DeclareUnicodeCharacter{00A0}{~}
\DeclareUnicodeCharacter{202F}{~}

\renewcommand{\Affilfont}{\small}
\newcommand{\email}[1]{\thanks{\href{mailto:#1}{\nolinkurl{#1}}}}

\usepackage[sort&compress, english]{cleveref}


\addbibresource{janis_newman.bib}
\iffalse
\bibliography{janis_newman.bib}
\fi


\hypersetup{
	pdftitle={Janis-Newman algorithm: generating rotating and NUT charged black holes},
	% pdfkeywords={Mots clés PDF},
	% pdfsubject={Sujet PDF}
}

\title{Janis--Newman algorithm: generating rotating and NUT charged black holes}

\author[1]{Harold Erbin\email{erbin@lpt.ens.fr}}
\affil[1]{\textsc{Cnrs}, \textsc{Lptens}, École Normale Supérieure, F-75231 Paris, France}


\begin{document}

\maketitle


\begin{abstract} 
In this review we present the most general form of the Janis--Newman algorithm.
This extension allows to generate configurations which contain all bosonic fields with spin less than or equal to two (real and complex scalar fields, gauge fields, metric field) and with five of the six parameters of the Plebański--Demiański metric (mass, electric charge, magnetic charge, NUT charge and angular momentum).
Several examples are included to illustrate the algorithm.
We also discuss the extension of the algorithm to other dimensions.
\end{abstract}


\newpage

\hrule
\pdfbookmark[1]{\contentsname}{toc}
\tableofcontents
\bigskip
\hrule

\newpage


% \leavevmode
% \\
% \\
% \\
% \\
% \\
\section{Introduction}
\label{introduction}

AutoML is the process by which machine learning models are built automatically for a new dataset. Given a dataset, AutoML systems perform a search over valid data transformations and learners, along with hyper-parameter optimization for each learner~\cite{VolcanoML}. Choosing the transformations and learners over which to search is our focus.
A significant number of systems mine from prior runs of pipelines over a set of datasets to choose transformers and learners that are effective with different types of datasets (e.g. \cite{NEURIPS2018_b59a51a3}, \cite{10.14778/3415478.3415542}, \cite{autosklearn}). Thus, they build a database by actually running different pipelines with a diverse set of datasets to estimate the accuracy of potential pipelines. Hence, they can be used to effectively reduce the search space. A new dataset, based on a set of features (meta-features) is then matched to this database to find the most plausible candidates for both learner selection and hyper-parameter tuning. This process of choosing starting points in the search space is called meta-learning for the cold start problem.  

Other meta-learning approaches include mining existing data science code and their associated datasets to learn from human expertise. The AL~\cite{al} system mined existing Kaggle notebooks using dynamic analysis, i.e., actually running the scripts, and showed that such a system has promise.  However, this meta-learning approach does not scale because it is onerous to execute a large number of pipeline scripts on datasets, preprocessing datasets is never trivial, and older scripts cease to run at all as software evolves. It is not surprising that AL therefore performed dynamic analysis on just nine datasets.

Our system, {\sysname}, provides a scalable meta-learning approach to leverage human expertise, using static analysis to mine pipelines from large repositories of scripts. Static analysis has the advantage of scaling to thousands or millions of scripts \cite{graph4code} easily, but lacks the performance data gathered by dynamic analysis. The {\sysname} meta-learning approach guides the learning process by a scalable dataset similarity search, based on dataset embeddings, to find the most similar datasets and the semantics of ML pipelines applied on them.  Many existing systems, such as Auto-Sklearn \cite{autosklearn} and AL \cite{al}, compute a set of meta-features for each dataset. We developed a deep neural network model to generate embeddings at the granularity of a dataset, e.g., a table or CSV file, to capture similarity at the level of an entire dataset rather than relying on a set of meta-features.
 
Because we use static analysis to capture the semantics of the meta-learning process, we have no mechanism to choose the \textbf{best} pipeline from many seen pipelines, unlike the dynamic execution case where one can rely on runtime to choose the best performing pipeline.  Observing that pipelines are basically workflow graphs, we use graph generator neural models to succinctly capture the statically-observed pipelines for a single dataset. In {\sysname}, we formulate learner selection as a graph generation problem to predict optimized pipelines based on pipelines seen in actual notebooks.

%. This formulation enables {\sysname} for effective pruning of the AutoML search space to predict optimized pipelines based on pipelines seen in actual notebooks.}
%We note that increasingly, state-of-the-art performance in AutoML systems is being generated by more complex pipelines such as Directed Acyclic Graphs (DAGs) \cite{piper} rather than the linear pipelines used in earlier systems.  
 
{\sysname} does learner and transformation selection, and hence is a component of an AutoML systems. To evaluate this component, we integrated it into two existing AutoML systems, FLAML \cite{flaml} and Auto-Sklearn \cite{autosklearn}.  
% We evaluate each system with and without {\sysname}.  
We chose FLAML because it does not yet have any meta-learning component for the cold start problem and instead allows user selection of learners and transformers. The authors of FLAML explicitly pointed to the fact that FLAML might benefit from a meta-learning component and pointed to it as a possibility for future work. For FLAML, if mining historical pipelines provides an advantage, we should improve its performance. We also picked Auto-Sklearn as it does have a learner selection component based on meta-features, as described earlier~\cite{autosklearn2}. For Auto-Sklearn, we should at least match performance if our static mining of pipelines can match their extensive database. For context, we also compared {\sysname} with the recent VolcanoML~\cite{VolcanoML}, which provides an efficient decomposition and execution strategy for the AutoML search space. In contrast, {\sysname} prunes the search space using our meta-learning model to perform hyperparameter optimization only for the most promising candidates. 

The contributions of this paper are the following:
\begin{itemize}
    \item Section ~\ref{sec:mining} defines a scalable meta-learning approach based on representation learning of mined ML pipeline semantics and datasets for over 100 datasets and ~11K Python scripts.  
    \newline
    \item Sections~\ref{sec:kgpipGen} formulates AutoML pipeline generation as a graph generation problem. {\sysname} predicts efficiently an optimized ML pipeline for an unseen dataset based on our meta-learning model.  To the best of our knowledge, {\sysname} is the first approach to formulate  AutoML pipeline generation in such a way.
    \newline
    \item Section~\ref{sec:eval} presents a comprehensive evaluation using a large collection of 121 datasets from major AutoML benchmarks and Kaggle. Our experimental results show that {\sysname} outperforms all existing AutoML systems and achieves state-of-the-art results on the majority of these datasets. {\sysname} significantly improves the performance of both FLAML and Auto-Sklearn in classification and regression tasks. We also outperformed AL in 75 out of 77 datasets and VolcanoML in 75  out of 121 datasets, including 44 datasets used only by VolcanoML~\cite{VolcanoML}.  On average, {\sysname} achieves scores that are statistically better than the means of all other systems. 
\end{itemize}


%This approach does not need to apply cleaning or transformation methods to handle different variances among datasets. Moreover, we do not need to deal with complex analysis, such as dynamic code analysis. Thus, our approach proved to be scalable, as discussed in Sections~\ref{sec:mining}.
%% This declares a command \Comment
%% The argument will be surrounded by /* ... */
\SetKwComment{Comment}{/* }{ */}

\begin{algorithm}[t]
\caption{Training Scheduler}\label{alg:TS}
% \KwData{$n \geq 0$}
% \KwResult{$y = x^n$}
\LinesNumbered
\KwIn{Training data $\mathcal{D}_{train}=\{(q_i, a_i, p_i^+)\}_{i=1}^m$, \\
\qquad \quad Iteration number $L$.}
\KwOut{A set of optimal model parameters.}

\For{$l=1,\cdots, L$}{
    Sample a batch of questions $Q^{(l)}$\\
    \For{$q_i\in Q^{(l)}$}{
        $\mathcal{P}_{i}^{(l)} \gets \mathrm{arg\,max}_{p_{i,j}}(\mathrm{sim}(q_i^{en},p_{i,j}),K)$\\
        $\mathcal{P}_{Gi}^{(l)} \gets \mathcal{P}_{i}^{(l)}\cup\{p^+_i\}$\\
        Compute $\mathcal{L}^i_{retriever}$, $\mathcal{L}^i_{postranker}$, $\mathcal{L}^i_{reader}$\\ according to Eq.\ref{eq:retriever}, Eq.\ref{eq:rerank}, Eq.\ref{eq:reader}\\
    }
    % $\mathcal{L}^{(l)}_{retriever} \gets \frac{1}{|Q^{(l)}|}\sum_i\mathcal{L}^i_{retriever}$\\
    % $\mathcal{L}^{(l)}_{retriever} \gets \mathrm{Avg}(\mathcal{L}^i_{retriever})$,
    % $\mathcal{L}^{(l)}_{rerank} \gets \mathrm{Avg}(\mathcal{L}^i_{rerank})$,
    % $\mathcal{L}^{(l)}_{reader} \gets \mathrm{Avg}(\mathcal{L}^i_{reader})$\\
    % Compute $\mathcal{L}^{(l)}_{retriever}$, $\mathcal{L}^{(l)}_{rerank}$, and $\mathcal{L}^{(l)}_{reader}$ by averaging over $Q^{(l)}$\\
    $\mathcal{L}^{(l)} \gets \frac{1}{|Q^{(l)}|}\sum_i(\mathcal{L}^{i}_{retriever} + \mathcal{L}^{i}_{postranker}+ \mathcal{L}^{i}_{reader})$\\
    $\mathcal{P}^{(l)}_K\gets\{\mathcal{P}^{(l)}_i|q_i\in Q^{(l)}\}$,\quad $\mathcal{P}^{(l)}_{KG}\gets\{\mathcal{P}^{(l)}_{Gi}|q_i\in Q^{(l)}\}$\\
    Compute the coefficient $v^{(l)}$ according to Eq.~\ref{eq:v}\\
  \eIf{$ v^{(l)}=1$}{
    $\mathcal{L}^{(l)}_{final} \gets \mathcal{L}^{(l)}(\mathcal{P}_{KG}^{(l)})$\\
  }{
      $\mathcal{L}^{(l)}_{final} \gets \mathcal{L}^{(l)}(\mathcal{P}^{(l)}_{K}),$\\
    }
    Optimize $\mathcal{L}^{(l)}_{final}$
}
\end{algorithm}


%  \eIf{$ \mathcal{L}^{(l-1)}_{retriever}<\lambda$}{
%     $\mathcal{L}^{(l)}_{final} \gets \mathcal{L}^{(l)}(\mathcal{P}_K^{(l)})$\\
%   }{
%       $\mathcal{L}^{(l)}_{final} \gets \mathcal{L}^{(l)}(\mathcal{P}^{(l)}_{KG}),$\\
%     }
\section{Extension through simple examples}
\label{sec:extension}


In this section we motivate through simple examples modifications to the original prescription for the transformation of the functions.


\subsection{Magnetic charges: dyonic Kerr--Newman}
\label{sec:extension:dyonic}


The dyonic Reissner--Nordström metric is obtained from the electric one \eqref{algo:eq:rn:functions} by the replacement~\cite[sec.~6.6]{Carroll:2004:SpacetimeGeometryIntroduction}
\begin{equation}
	q^2 \longrightarrow \abs{Z}^2 = q^2 + p^2
\end{equation} 
where $Z$ corresponds to the central charge
\begin{equation}
	Z = q + i p.
\end{equation} 
Then the metric function reads
\begin{equation}
	f = 1 - \frac{2m}{r} + \frac{\abs{Z}^2}{r^2}.
\end{equation} 
The gauge field receives a new $\phi$-component
\begin{equation}
	\label{ext:eq:static:vector}
	A = f_A\, \dd t - p \cos \theta\, \dd\phi
		= f_A\, \dd u - p \cos \theta\, \dd\phi
\end{equation}
(the last equality being valid after a gauge transformation) and
\begin{equation}
	f_A = \frac{q}{r}.
\end{equation} 

The transformation of the function $f$ under \eqref{algo:eq:change:complexification-ur} is straightforward and yields
\begin{equation}
	\tilde f = 1 - \frac{2m r' - \abs{Z}^2}{\rho^2}.
\end{equation} 
On the other hand transforming directly the $r$ inside $f_A$ according to \eqref{algo:eq:rules} does not yield the correct result.
Instead one needs to first rewrite the gauge field function as
\begin{equation}
	f_A = \Re\left(\frac{Z}{r}\right)
\end{equation} 
from which the transformation proceeds to
\begin{equation}
	\tilde f_A = \frac{\Re(Z \bar r)}{\abs{r}^2}
		= \frac{q r' - p a \cos \theta}{\rho^2}.
\end{equation} 
Note that it not useful to replace $p$ by $\Im Z$ in \eqref{ext:eq:static:vector} since it is not accompanied by any $r$ dependence.
Moreover it is natural that the factor $\abs{Z}^2$ appears in the metric and this explains why the charges there do not mix with the coordinates.

The gauge field in BL coordinates is finally
\begin{subequations}
\begin{align}
	A &= \frac{q r - p a \cos \theta}{\rho^2}\, \dd t
			+ \left(- \frac{q r}{\rho^2}\, a \sin^2 \theta + \frac{p(r^2 + a^2)}{\rho^2}\, \cos\theta \right) \dd\phi \\
		&= \frac{q r}{\rho^2} (\dd t - a \sin^2 \theta \dd\phi)
			+ \frac{p \cos \theta}{\rho^2} \left(a\, \dd t + (r^2 + a^2)\, \dd\phi \right).
\end{align}
\end{subequations}
The radial component has been removed thanks to a gauge transformation since it depends only on $r$
\begin{equation}
	\Delta \times A_r = - \frac{q r - p a \cos \theta}{\rho^2}\, \rho^2 - p a \cos \theta
		= - q r.
\end{equation} 

There is a coupling between the parameters $a$ and $p$ which can be interpreted from the fact that a rotating magnetic charge has an electric quadrupole moment.
This coupling is taken into account from the product of the imaginary parts which yield a real term.
In view of the form of the algorithm such contribution could not arise from any other place.
Moreover the combination $Z = q + i p$ appears naturally in the Plebański--Demiański solution~\cite{Plebanski:1975:ClassSolutionsEinsteinMaxwell, Plebanski:1976:RotatingChargedUniformly}.

The Yang--Mills Kerr--Newman black hole found by Perry~\cite{Perry:1977:BlackHolesAre} can also be derived in this way, starting from the seed
\begin{equation}
	A^I = \frac{q^I}{r}\, \dd t + p^I \cos \theta\, \dd\phi, \qquad
	\abs{Z}^2 = q^I q^I + p^I p^I
\end{equation} 
where $q^I$ and $p^I$ are constant elements of the Lie algebra.


\subsection[NUT charge, cosmological constant and topological horizon: (anti-)de Sitter Schwarzschild--NUT]
{NUT charge and cosmological constant and topological horizon: (anti-)de Sitter Schwarzschild--NUT}
\label{sec:extension:nut}


In this subsection we consider general topological horizons
\begin{equation}
	\dd \Omega^2 = \dd\theta^2 + H(\theta)^2\, \dd \phi^2, \qquad
	H(\theta) =
	\begin{cases}
		\sin \theta & \kappa = 1 \quad (S^2), \\
		\sinh \theta & \kappa = -1 \quad (H^2).
	\end{cases}
\end{equation} 
The cosmological constant is denoted by $\Lambda$.
We give only the main formulas to motivate the modification of the algorithm, leaving the details of the transformation for \cref{sec:general}.

The complex transformation that adds a NUT charge is
\begin{subequations}
\label{ext:eq:change:jna-nut}
\begin{gather}
	u = u' - 2 \kappa \ln H(\theta), \qquad
	r = r' + i n, \\
	m = m' + i \kappa n, \qquad
	\kappa = \kappa' - \frac{4\Lambda}{3}\, n^2.
\end{gather}
\end{subequations}
Note that it is $\kappa$ and not $\kappa'$ that appears in $m$.
After having shown

The metric derived from the seed \eqref{algo:eq:static:metric:tr} is
\begin{equation}
	\dd s^2 = - \tilde f\, (\dd t - 2 \kappa n H'(\theta)\, \dd\phi)^2
		+ \tilde f^{-1}\, \dd r^2
		+ \rho^2\, \dd\Omega^2,
\end{equation}
see \eqref{gen:eq:rotating:tr-F-cst}, where
\begin{equation}
	\rho^2 = r'^2 + n^2.
\end{equation} 

The function corresponding to the (a)dS--Schwarzschild metric is
\begin{equation}
	f = \kappa - \frac{2m}{r} - \frac{\Lambda}{3}\, r^2
		= \kappa - 2 \Re\left(\frac{m}{r}\right) - \frac{\Lambda}{3}\, r^2.
\end{equation} 
The transformation is
\begin{equation}
	\tilde f = \kappa
			- \frac{2 \Re(m \bar r)}{\abs{r}^2}
			- \frac{\Lambda}{3}\, \abs{r}^2
		= \kappa' - \frac{4\Lambda}{3}\, n^2
			- \frac{2 \left[ m' r' + \left( \kappa' - \frac{4\Lambda}{3}\, n^2 \right) n^2 \right]}{\rho^2}
			- \frac{\Lambda}{3}\, \rho^2
\end{equation} 
which after simplification gives
\begin{equation}
	\label{ext:eq:nut-tilde-f}
	\tilde f = \kappa' - \frac{2 m' r' + 2 \kappa' n^2}{\rho^2}
		- \frac{\Lambda}{3} (r'^2 + 5 n^2)
		+ \frac{8\Lambda}{3}\, \frac{n^4}{\rho^2}
\end{equation} 
which corresponds correctly to the function of (a)dS--Schwarzschild--NUT~\cite{AlonsoAlberca:2000:SupersymmetryTopologicalKerrNewmannTaubNUTaDS}.

Note that it is necessary to consider the general case of massive black hole with topological horizon (if $\Lambda \neq 0$ for the latter) even if one is ultimately interested in the $m = 0$ or $\kappa = 1$ cases.

The transformation \eqref{ext:eq:change:jna-nut} can be interpreted as follows.
In similarity with the case of the magnetic charge, writing the mass as a complex parameter is needed to take into account some couplings between the parameters that would not be found otherwise.
Moreover the shift of $\kappa$ is required because the curvature of the $(\theta, \phi)$ section should be normalized to $\kappa = \pm 1$ but the coupling of the NUT charge with the cosmological constant modifies the curvature: the new shift is necessary to balance this effect and to normalize the $(\theta, \phi)$ curvature to $\kappa' = \pm 1$ in the new metric.
The NUT charge in the Plebański--Demiański solution~\cite{Plebanski:1975:ClassSolutionsEinsteinMaxwell, Plebanski:1976:RotatingChargedUniformly} is
\begin{equation}
	\ell = n \left( 1 - \frac{4\Lambda}{3}\, n^2 \right)
\end{equation} 
so the natural complex combination is $m + i \ell$ and not $m + i \kappa n$ from this point of view, and similarly for the curvature~\cite[sec.~5.3]{Griffiths:2006:NewLookPlebanskiDemianski} (such relations appear when taking limit of the Plebański--Demiański solution to recover subcases).

Finally we conclude this section with two remarks to quote different contexts where the above expression appear naturally :
\begin{itemize}
	\item Embedding Einstein--Maxwell into $N = 2$ supergravity with a negative cosmological constant $\Lambda = - 3 g^2$, the solution is BPS if~\cite{AlonsoAlberca:2000:SupersymmetryTopologicalKerrNewmannTaubNUTaDS}
	\begin{equation}
		\kappa' = -1, \qquad
		n = \pm \frac{1}{2g},
	\end{equation} 
	in which case $\kappa' = \kappa$.
	
	\item The Euclidean NUT solution is obtained from the Wick rotation
	\begin{equation}
		t = - i \tau, \qquad
		n = i \nu.
	\end{equation}
	The condition for regularity is~\cite{Chamblin:1999:LargeNPhases, Johnson:2014:ThermodynamicVolumesAdSTaubNUT}
	\begin{equation}
		m = m' - \nu \left( \kappa + \frac{4\Lambda}{3}\, \nu^2 \right)
			= 0.
	\end{equation} 
\end{itemize}


\subsection{Complex scalar fields}


For a complex scalar field, or any pair of real fields that can be naturally gathered as a complex field, one should treat the full field as a single entity instead of looking at the real and imaginary parts independently.
In particular one should not impose any reality condition.
A typical case of such system is the axion--dilaton pair
\begin{equation}
	\tau = \e^{-2\phi} + i \sigma.
\end{equation} 

In order to demonstrate this principle consider the seed (for a complete example see \cref{sec:examples:rotating-T3})
\begin{equation}
	\tau = 1 + \frac{\mu}{r}
\end{equation} 
where only the dilaton is non-zero.
Then the transformation \eqref{algo:eq:change:complexification-ur} gives
\begin{equation}
	\tau' = 1 + \frac{\mu}{r}
		= 1 + \frac{\mu}{r' - i a \cos\theta}
		= 1 + \frac{\mu r'}{\rho^2} + i\, \frac{\mu a \cos\theta}{\rho^2}.
\end{equation} 
The transformation generates an imaginary part which cannot be obtained if $\Im \tau$ is treated separately: the algorithm does not change fields that vanish except if they are components of a larger field.
Note that both $\tau$ and $\tau'$ are harmonic functions.

\section{Complete algorithm}
\label{sec:general}


In this section we gather all the facts on the Janis--Newman algorithm and we explain how to apply it to a general setting.
We write the formulas corresponding to the most general configurations that can be obtained.
We insist again on the fact that all these results can also be derived from the tetrad formalism.


\subsection{Seed configuration}
\label{sec:general:seed}


We consider a general configuration with a metric $g_{\mu\nu}$, gauge fields $A_\mu^I$, complex scalar fields $\tau^i$ and real scalar fields $q^u$.
The initial parameters of the seed configuration are the mass $m$, electric charges $q^I$, magnetic charges $p^i$ and some other parameters $\lambda^A$ (such as the scalar charges).
The electric and magnetic charges are grouped in complex parameters
\begin{equation}
	Z^I = q^I + i p^I.
\end{equation} 
All indices run over some arbitrary ranges.

The seed configuration is spherically symmetric and in particular all the fields depend only on the radial direction $r$
\begin{subequations}
\label{gen:eq:static:tr}
\begin{gather}
	\label{gen:eq:static:metric:tr}
	\dd s^2 = - f_t(r)\, \dd t^2 + f_r(r)\, \dd r^2 + f_\Omega(r)\, \dd\Omega^2, \\
	A^I = f^I(r)\, \dd t + p^I H'(\theta)\, \dd\phi, \\
	\tau^i = \tau^i(r), \qquad
	q^u = q^u(r)
\end{gather}
\end{subequations}
where
\begin{equation}
	\dd \Omega^2 = \dd\theta^2 + H(\theta)^2\, \dd \phi^2, \qquad
	H(\theta) =
	\begin{cases}
		\sin \theta & \kappa = 1 \quad (S^2), \\
		\sinh \theta & \kappa = -1 \quad (H^2).
	\end{cases}
\end{equation} 
Note that only two functions in the metric are relevant since the last one can be fixed through a diffeomorphism.
All the real functions are denoted collectively by
\begin{equation}
	f_i = \{ f_t, f_r, f_\Omega, f^I, q^u \}.
\end{equation} 

The transformation to null coordinates is
\begin{equation}
	\label{gen:eq:change:null}
	\dd t = \dd u - \sqrt{\frac{f_r}{f_t}}\, \dd r
\end{equation} 
and yields
\begin{subequations}
\label{gen:eq:static:ur}
\begin{gather}
	\label{gen:eq:static:metric:ur}
	\dd s^2 = - f_t\, \dd u^2 - 2 \sqrt{f_t f_r}\, \dd r^2 + f_\Omega\, \dd\Omega^2, \\
	A^I = f^I\, \dd u + p^I H'\, \dd\phi
\end{gather}
\end{subequations}
where the radial component of the gauge field
\begin{equation}
	A^I_r = f^I \sqrt{\frac{f_r}{f_t}}
\end{equation} 
has been set to zero through a gauge transformation.


\subsection{Janis--Newman algorithm}
\label{sec:general:jna}


\subsubsection{Complex transformation}


One performs the complex change of coordinates
\begin{equation}
	\label{gen:eq:change:jna}
	r = r' + i\, F(\theta), \qquad
	u = u' + i\, G(\theta).
\end{equation}
In the case of topological horizons the Giampieri ansatz \eqref{algo:eq:giampieri-ansatz} generalizes to
\begin{equation}
	\label{gen:eq:giampieri-ansatz}
	i\, \dd \theta = H(\theta)\, \dd \phi
\end{equation} 
leading to the differentials
\begin{equation}
	\dd r = \dd r' + F'(\theta) H(\theta)\, \dd \phi, \qquad
	\dd u = \dd u' + G'(\theta) H(\theta)\, \dd \phi.
\end{equation} 
The ansatz \eqref{gen:eq:giampieri-ansatz} is a direct consequence of the fact that the $2$-dimensional slice $(\theta, \phi)$ is given by $\dd \Omega^2 = \dd\theta^2 + H(\theta)^2\, \dd \phi^2$, such that the function in the RHS of \eqref{gen:eq:giampieri-ansatz} corresponds to $\sqrt{g^\Omega_{\phi\phi}}$ (where $g$ is the static metric), as can be seen by doing the computation with $i\, \dd \theta = \mc H(\theta) \dd\phi$ and identifying $\mc H = H$ at the end.

The most general known transformation is
\begin{subequations}
\begin{gather}
	\label{gen:eq:change:jna-functions-FG}
	F(\theta) = n - a\, H'(\theta) + c \left( 1 + H'(\theta)\, \ln \frac{H(\theta/2)}{H'(\theta/2)} \right), \\
	G(\theta) = \kappa a\, H'(\theta)
		- 2 \kappa n \ln H(\theta)
		- \kappa c\, H'(\theta)\, \ln \frac{H(\theta/2)}{H'(\theta/2)}, \\
	m = m' + i \kappa n, \\
	\kappa = \kappa' - \frac{4\Lambda}{3}\, n^2,
\end{gather}
\end{subequations}
where $a, c \neq 0$ only if $\Lambda = 0$ (see \cref{sec:derivation} for the derivation).
The mass that is transformed is the physical mass: even if it written in terms of other parameters one should identify it and transform it.

The parameters $a$ and $n$ correspond respectively to the angular momentum and to the NUT charge.
On the other hand the constant $c$ did not receive any clear interpretation (see for example~\cites{Demianski:1972:NewKerrlikeSpacetime, Adamo:2014:KerrNewmanMetricReview}[sec.~5.3]{Krasinski:2006:InhomogeneousCosmologicalModels}).
It can be noted that the solution is of type II in Petrov classification (and thus the JN algorithm \emph{can} change the Petrov type) and it corresponds to a wire singularity on the rotation axis.
Moreover the BL transformation is not well-defined.


\subsubsection{Function transformation}
\label{sec:general:jna:functions}


All the real functions $f_i = f_i(r)$ must be modified to be kept real once $r \in \C$
\begin{equation}
	\label{gen:eq:complexification-functions}
	\tilde f_i = \tilde f_i(r, \bar r)
		= \tilde f_i\big(r', F(\theta) \big) \in \R.
\end{equation} 
The last equality means that $\tilde f_i$ can depend on $\theta$ only through $\Im r = F(\theta)$.
The condition that one recovers the seed for $\bar r = r = r'$ is
\begin{equation}
	\tilde f_i(r', 0) = f_i(r').
\end{equation} 

If all magnetic charges are vanishing or in terms without electromagnetic charges the rules for finding the $\tilde f_i$ are
\begin{subequations}
\label{gen:eq:rules}
\begin{align}
	\label{gen:eq:rules:r}
	r & \longrightarrow \frac{1}{2} (r + \bar r) = \Re r, \\
	\label{gen:eq:rules:1/r}
	\frac{1}{r} & \longrightarrow \frac{1}{2} \left(\frac{1}{r} + \frac{1}{\bar r}\right) = \frac{\Re r}{\abs{r}^2}, \\
	\label{gen:eq:rules:r2}
	r^2 & \longrightarrow \abs{r}^2.
\end{align}
\end{subequations}
Up to quadratic powers of $r$ and $r^{-1}$ these rules determine almost uniquely the result.
This is not anymore the case when the configurations involve higher power.
These can be dealt with by splitting it in lower powers: generically one should try to factorize the expression into at most quadratic pieces.
Some examples of this with natural guesses are
\begin{equation}
	r^4 - b^2 = (r^2 + b) (r^2 - b), \qquad
	r^4 + b = r^2 \left( r^2 + \frac{b}{r^2} \right).
\end{equation} 
Moreover the same power of $r$ can be transformed differently, for example
\begin{equation}
	\frac{1}{r^n} \longrightarrow \frac{1}{r^{n-2}}\, \frac{1}{\abs{r}^2}.
\end{equation} 

Denoting by $Q(r)$ and $P(r)$ collectively all functions that multiply $q^I$ and $p^I$ respectively, all such terms should be rewritten as
\begin{equation}
	\Big( q^I Q(r), p^I P(r) \Big) = \Big( \Re\big(Z^I Q(r)\big), \Im\big(Z^I P(r)\big) \Big)
\end{equation} 
before performing the transformation \eqref{gen:eq:change:jna}.
Note that in this case one does not use the rules \eqref{gen:eq:rules}.

Finally the transformed complex scalars are obtained by simply plugging \eqref{gen:eq:change:jna}
\begin{equation}
	\tau'^i(r', \theta) = \tau^i\big(r + i F(\theta)\big).
\end{equation} 


\subsubsection{Null coordinates}


Plugging the transformation \eqref{gen:eq:change:jna} inside the seed metric and gauge fields \eqref{gen:eq:static:ur} leads to\footnotemark{}%
\footnotetext{%
	We stress that at this stage these formula do not satisfy Einstein equations, they are just proxies to simplify later computations.
}
\begin{subequations}
\label{gen:eq:rotating:ur}
\begin{gather}
	\dd s^2 = - \tilde f_t\, (\dd u' + \alpha\, \dd r' + \omega H\, \dd\phi )^2
		+ 2 \beta\, \dd r' \dd \phi
		+ \tilde f_\Omega\, \big(\dd\theta^2 + \sigma^2 H^2\, \dd\phi^2 \big), \\
	A^I = \tilde f^I\, (\dd u' + G' H\, \dd \phi) + p^I H'\, \dd\phi
\end{gather}
\end{subequations}
where (one should not confuse the primes to indicate derivatives from the primes on the coordinates)
\begin{equation}
	\omega = G' + \sqrt{\frac{\tilde f_r}{\tilde f_t}}\, F', \qquad
	\sigma^2 = 1 + \frac{\tilde f_r}{\tilde f_\Omega}\, F'^2, \qquad
	\alpha = \sqrt{\frac{\tilde f_r}{\tilde f_t}}, \qquad
	\beta = \tilde f_r\, F' H.
\end{equation} 


\subsubsection{Boyer--Lindquist coordinates}


The Boyer--Lindquist transformation
\begin{equation}
	\label{gen:eq:change:bl}
	\dd u' = \dd t' - g(r') \dd r', \qquad
	\dd \phi = \dd \phi' - h(r') \dd r',
\end{equation} 
can be used to remove the off-diagonal $tr$ and $r\phi$ components of the metric
\begin{equation}
	g_{t'r'} = g_{r'\phi'} = 0.
\end{equation} 
The solution to these equations is
\begin{equation}
	\label{gen:eq:change:bl:solution-gh}
	g(r') = \frac{\sqrt{\big(\tilde f_t \tilde f_r \big)^{-1}}\, \tilde f_\Omega - F' G'}{\Delta}, \qquad
	h(r') = \frac{F'}{H \Delta}
\end{equation} 
where
\begin{equation}
	\label{gen:eq:change:bl:delta}
	\Delta = \frac{\tilde f_\Omega}{\tilde f_r}\, \sigma^2
		= \frac{\tilde f_\Omega}{\tilde f_r} + F'^2.
\end{equation} 
Remember that the changes of coordinate is valid only if $g$ and $h$ are functions of $r'$ only.

Inserting \eqref{gen:eq:change:bl:solution-gh} into \eqref{gen:eq:rotating:ur} yields
\begin{subequations}
\label{gen:eq:rotating:tr}
\begin{gather}
	\dd s^2 = - \tilde f_t\, (\dd t' + \omega H\, \dd\phi' )^2
		+ \frac{\tilde f_\Omega}{\Delta}\, \dd r'^2
		+ \tilde f_\Omega\, \big(\dd\theta^2 + \sigma^2 H^2\, \dd\phi'^2 \big), \\
	A^I = \tilde f^I\, \left(\dd t' - \frac{\tilde f_\Omega}{\Delta \sqrt{\tilde f_t \tilde f_r}}\, \dd r' + G' H\, \dd \phi' \right) + p^I H'\, \dd\phi'
\end{gather}
\end{subequations}
where we recall that
\begin{equation}
	\omega = G' + \sqrt{\frac{\tilde f_r}{\tilde f_t}}\, F', \qquad
	\sigma^2 = 1 + \frac{\tilde f_r}{\tilde f_\Omega}\, F'^2.
\end{equation} 
Generically one finds $A_r = A_r(r)$ which can be set to zero thanks to a gauge transformation.

Before closing this section we simplify the above formulas for few simple cases that are often used.


\paragraph{Degenerate Schwarzschild seed}

A degenerate seed (one unknown function) in Schwarzschild coordinates has
\begin{equation}
	f_r = f_t^{-1}, \qquad
	f_\Omega = r^2.
\end{equation} 
The above formulas for this case can be found in \cref{sec:derivation:ansatz}.


\paragraph{Degenerate isotropic seed}

A degenerate seed in isotropic coordinates has
\begin{equation}
	f_t = f^{-1}, \qquad
	f_r = f, \qquad
	f_\Omega = r^2 f.
\end{equation} 
In this case the above formulas reduced to
\begin{subequations}
\label{gen:eq:rotating:tr-degenerate-isotropic}
\begin{gather}
	\dd s^2 = - \tilde f^{-1}\, (\dd t + \omega H\, \dd\phi )^2
		+ \tilde f \rho^2 \left( \frac{\dd r^2}{\Delta}
			+ \dd\theta^2 + \sigma^2 H^2\, \dd\phi^2 \right), \\
	A^I = \tilde f^I\, \left(\dd t - \frac{\tilde f \rho^2}{\Delta}\, \dd r + G' H\, \dd \phi \right) + p^I H'\, \dd\phi
\end{gather}
\end{subequations}
where we recall that
\begin{equation}
	\omega = G' + \tilde f\, F', \qquad
	\sigma^2 = 1 + \frac{F'^2}{\rho^2}, \qquad
	\Delta = \tilde f \rho^2 + F'^2.
\end{equation} 

\paragraph{Constant $F$}

The expressions simplify greatly if $F' = 0$ (for example when $\Lambda \neq 0$).
First all functions depend only on $r$ since $F(\theta) = \cst$
\begin{equation}
	\tilde f_i(r, \theta) = \tilde f_i(r, 0).
\end{equation} 
As a consequence the Boyer--Lindquist transformation \eqref{gen:eq:change:bl:solution-gh}
\begin{equation}
	g(r') = \sqrt{\frac{\tilde f_r}{\tilde f_t}}, \qquad
	h(r') = 0
\end{equation} 
is always well-defined.
For the same reason it is always possible to perform a gauge transformation.
Finally the metric and gauge fields \eqref{gen:eq:rotating:tr} becomes
\begin{subequations}
\label{gen:eq:rotating:tr-F-cst}
\begin{gather}
	\dd s^2 = - \tilde f_t \big(\dd t + G' H\, \dd\phi \big)^2
		+ \tilde f_r\, \dd r^2
		+ \tilde f_\Omega\, \dd\Omega^2, \\
	A^I = \tilde f^I\, \left(\dd t' + G' H\, \dd \phi' \right) + p^I H'\, \dd\phi'.
\end{gather}
\end{subequations}


\subsection{Open questions}


The algorithm we have described help to work with five (four if $\Lambda \neq 0$) of the six parameters of the Plebański--Demiański (PD) solution.
It is tempting to conjecture that it can be extended to the full set of parameters by generalizing the ideas described in \cref{sec:extension:nut} (shifting $\kappa$, writing $a + i \alpha$…).
Indeed we have found that these operations were quite natural in the context of the  PD solution and inspiration could be found in~\cite{Griffiths:2006:NewLookPlebanskiDemianski}.

\section{Derivation of the transformations}
\label{sec:derivation}


The goal of this section is to derive the form \eqref{gen:eq:change:jna-functions-FG} of the possible complex transformations.
This method was first used by Demiański~\cite{Demianski:1972:NewKerrlikeSpacetime} and then generalized in~\cite{Erbin:2016:DecipheringGeneralizingDemianskiJanisNewman}.
The idea is to perform the algorithm in a simple setting (metric with one unknown function and one gauge field), leaving arbitrary the functions $F(\theta)$ and $G(\theta)$ in \eqref{gen:eq:change:jna} and $\tilde f_i$ before solving the equations of motion to determine them.
Then the result can be reinterpreted in terms of rules to get the functions $\tilde f_i$ from $f_i$ (this last part was not discussed in~\cite{Demianski:1972:NewKerrlikeSpacetime}).
This selects the possible complex transformations.
Then one can hope that these transformations will be the most general ones (under the assumptions that are made), and one can use these transformations in other cases without having to solve the equations.
The latter claim can be justified by looking at the equations of motions for more complex examples: even if one cannot find directly a solution, one finds that the same structure persists~\cite{Erbin:2016:DecipheringGeneralizingDemianskiJanisNewman} (this is also motivated by the solutions in~\cite{Krori:1981:ChargedDemianskiMetric, Patel:1988:RadiatingDemianskitypeMetrics}).
Another strength of this approach is to remove the ambiguity of the algorithm since the functions are found from the equations of motion, and this may help when one does not know how to perform precisely the algorithm (for example in higher dimensions, see \cref{sec:higher}).


Another goal of this section is to expose the full technical details of the computations: Demiański's paper~\cite{Demianski:1972:NewKerrlikeSpacetime} is short and results are extremely condensed.
In particular we uncover an underlying assumption on the form of the metric function and we show how this lead to an error an in his formula (14) (already pointed out in~\cite{Quevedo:1992:ComplexTransformationsCurvature}).
A generalization of this hypothesis leads to other equations that we could not solve analytically and which may lead to other complex transformations.

Finally this analysis shows the impossibility to derive the (a)dS--Kerr(--Newman) solutions from the JN algorithm.
As discussed in the previous section generalization of the ansatz may help to avoid this no-go theorem.


\subsection{Setting up the ansatz}
\label{sec:derivation:ansatz}


We first recall the action and equations of motion before describing the ansatz for the metric and gauge fields.
We refer to \cref{sec:general} for the general formulas from which the expressions in this section are derived.


\subsubsection{Action and equations of motion}


The action for Einstein--Maxwell gravity with cosmological constant $\Lambda$ reads
\begin{equation}
	\label{deriv:eq:einstein-maxwell-action}
	S = \int \dd^4 x\; \sqrt{- g} \left( \frac{1}{2 \varkappa^2} (R - 2 \Lambda) - \frac{1}{4}\, F^2 \right),
\end{equation} 
where $\varkappa^2 = 8 \pi G$ is the Einstein coupling constant, $g_{\mu\nu}$ is the metric with Ricci scalar $R$ and $F = \dd A$ is the field strength of the Maxwell field $A_\mu$.
In the rest of this section we will set $\varkappa = 1$.
The corresponding equations of motion (respectively Einstein and Maxwell) are
\begin{equation}
	\label{deriv:eq:einstein-maxwell-eom}
	G_{\mu\nu} + \Lambda g_{\mu\nu} = 2\, T_{\mu\nu}, \qquad
	\grad_\mu F^{\mu\nu} = 0,
\end{equation} 
where energy--momentum tensor for the electromagnetic gauge field $A_\mu$ is
\begin{equation}
	T_{\mu\nu} = F_{\mu\rho} \tens{F}{_\nu^\rho} - \frac{1}{4}\, g_{\mu\nu} F^2.
\end{equation} 


\subsubsection{Seed configuration}


We are interested in the subcase of \eqref{gen:eq:static:metric:tr} where
\begin{equation}
	\label{eq:static-ansatz-one-unknown}
	f_t = f, \qquad
	f_r = f^{-1}, \qquad
	f_\Omega = r^2.
\end{equation} 

The seed configuration is
\begin{subequations}
\label{deriv:eq:static:tr}
\begin{gather}
	\label{deriv:eq:static:metric:tr}
	\dd s^2 = - f(r)\, \dd t^2 + f(r)^{-1}\, \dd r^2 + r^2\, \dd\Omega^2, \\
	\label{deriv:eq:static:vector:tr}
	A = f_A(r)\, \dd t
\end{gather}
\end{subequations}
where we consider spherical and hyperbolic horizons
\begin{equation}
	\dd \Omega^2 = \dd\theta^2 + H(\theta)^2\, \dd \phi^2, \qquad
	H(\theta) =
	\begin{cases}
		\sin \theta & \kappa = 1, \\
		\sinh \theta & \kappa = -1.
	\end{cases}
\end{equation} 
In terms of null coordinates \eqref{gen:eq:change:null} the configuration reads
\begin{subequations}
\label{deriv:eq:static:ur}
\begin{gather}
	\label{deriv:eq:static:metric:ur}
	\dd s^2 = - f\, \dd u^2 - 2\, \dd u \dd r + r^2\, \dd\Omega^2, \\
	\label{deriv:eq:static:vector:ur}
	A = f_A\, \dd u.
\end{gather}
\end{subequations}


\subsubsection{Janis--Newman configuration}


The configuration obtained from the Janis--Newman algorithm with a general transformation \eqref{gen:eq:change:jna}
\begin{equation}
	r = r' + i\, F(\theta), \qquad
	u = u' + i\, G(\theta)
\end{equation}
corresponds to (we omit the primes on the coordinates)
\begin{subequations}
\label{deriv:eq:rotating:ur}
\begin{gather}
	\dd s^2 = - \tilde f\, (\dd u + \alpha\, \dd r + \omega H\, \dd\phi )^2
		+ 2 \beta\, \dd r \dd \phi
		+ \rho^2\, \big(\dd\theta^2 + \sigma^2 H^2\, \dd\phi^2 \big), \\
	A = \tilde f_A\, (\dd u + G' H\, \dd \phi)
\end{gather}
\end{subequations}
where
\begin{equation}
	\rho^2 = r^2 + F^2, \quad
	\omega = G' + \tilde f^{-1}\, F', \quad
	\sigma^2 = 1 + \frac{F'^2}{\tilde f \rho^2}, \quad
	\alpha = \tilde f^{-1}, \quad
	\beta = \tilde f^{-1}\, F' H.
\end{equation} 

The Boyer--Lindquist transformation \eqref{gen:eq:change:bl}
\begin{equation}
	\dd u = \dd t' - g(r) \dd r, \qquad
	\dd \phi = \dd \phi' - h(r) \dd r
\end{equation} 
with functions
\begin{equation}
	g(r) = \frac{\rho^2 - F' G'}{\Delta}, \qquad
	h(r) = \frac{F'}{H \Delta}, \qquad
	\Delta = \tilde f \rho^2\, \sigma^2
\end{equation} 
leads to (omitting the primes on the coordinates)
\begin{subequations}
\label{deriv:eq:rotating:tr}
\begin{gather}
	\dd s^2 = - \tilde f_t\, (\dd t + \omega H\, \dd\phi )^2
		+ \frac{\rho^2}{\Delta}\, \dd r^2
		+ \rho^2\, \big(\dd\theta^2 + \sigma^2 H^2\, \dd\phi^2 \big), \\
	A = \tilde f_A\, \left(\dd t - \frac{\rho^2}{\Delta}\, \dd r + G' H\, \dd \phi \right).
\end{gather}
\end{subequations}


\subsection{Static solution}


It is straightforward to solve the equations \eqref{deriv:eq:einstein-maxwell-eom} for the static configuration \eqref{deriv:eq:static:tr}.

Only the $(t)$ component of Maxwell equations is non trivial
\begin{equation}
	2 f'_A + r f''_A = 0,
\end{equation} 
the prime being a derivative with respect to $r$, and its solution is
\begin{equation}
	f_A(r) = \alpha + \frac{q}{r}
\end{equation} 
where $q$ is a constant of integration that is interpreted as the charge and $\alpha$ is an additional constant that can be removed by a gauge transformation.

The only relevant Einstein equation is
\begin{equation}
	\frac{q^2}{r^2} - \kappa + r^2 \Lambda + f + r f' = 0
\end{equation} 
whose solution reads
\begin{equation}
	\label{eq:topdown-1:static-f}
	f(r) = \kappa - \frac{2m}{r} + \frac{q^2}{r^2} - \frac{\Lambda}{3}\, r^2,
\end{equation} 
$m$ being a constant of integration that is identified to the mass.

We stress that we are just looking for solutions of Einstein equations and we are not concerned with regularity (in particular it is well-known that only $\kappa = 1$ is well-defined for $\Lambda = 0$).

The solution we will find in the next section should reduce to this one upon setting $F, G = 0$.


\subsection{Stationary solution}


Since Boyer--Lindquist imposes additional restrictions on the solutions we will solve the equations of motion \eqref{deriv:eq:einstein-maxwell-eom} for the configuration in null coordinates \eqref{deriv:eq:rotating:ur}.


\subsubsection{Simplifying the equations}
\label{sec:derivation:stationary:simplifying}


The components $(rr)$ and $(r\theta)$ give respectively the equation
\begin{subequations}
\begin{align}
	G'' + \frac{H'}{H}\, G' &= \pm 2 F, \\
	F' \left( G'' + \frac{H'}{H}\, G' \right) &= 2 F F'.
\end{align}
\end{subequations}
If $F' = 0$ then $F$ is an arbitrary constant and the sign of the first equation can be absorbed into its definition.\footnotemark{}%
\footnotetext{%
	In particular all expressions are quadratic in $F$, but only linear in $F'$.
}
On the other hand if $F' \neq 0$ one can simplify by the latter in the second equation and this fixes the sign of the first equation.
Then in both cases the relevant equation reduces to
\begin{equation}
	\label{eq:topdown-1-F-Gd-bis}
	G'' + \frac{H'}{H}\, G' = 2 F,
\end{equation} 
which depends only on $\theta$ and allows to solve for $G$ in terms of $F$.

Integrating the $r$-component of the Maxwell equation gives
\begin{equation}
	\tilde f_A = \frac{q\, r}{r^2 + F^2} + \alpha\, \frac{r^2 - F^2}{r^2 + F^2}.
\end{equation}
The $\theta$-equation reads
\begin{equation}
	\alpha\, F' = 0
\end{equation}
which implies $\alpha = 0$ if $F' \neq 0$.
The $\phi$- and $t$-equations follow from these two equations.
As seen above, $\alpha$ can be removed in the static limit $F \to 0$ and in the rest of this section we consider only the case\footnotemark{}%
\footnotetext{%
	We relax this assumption in \cref{sec:derivation:relaxing:gauge-fields}.
}
\begin{equation}
	\alpha = 0.
\end{equation} 

The $(tr)$ equation contains only $r$-derivatives of $\tilde f$ and can be integrated, giving\footnotemark{}%
\footnotetext{%
	In~\cite{Demianski:1972:NewKerrlikeSpacetime} the last term of $\tilde f$ is missing as pointed out in~\cite{Quevedo:1992:ComplexTransformationsCurvature}.
}
\begin{equation}
	\tilde f = \kappa - \frac{2m r - q^2 + 2 F (\kappa\, F + K)}{r^2 + F^2} - \frac{\Lambda}{3}\, (r^2 + F^2) - \frac{4 \Lambda}{3}\, F^2 + \frac{8 \Lambda}{3}\, \frac{F^4}{r^2 + F^2}
\end{equation} 
where again $m$ is a constant of integration interpreted as the mass and the function $K$ is defined by
\begin{equation}
	2 K = F'' + \frac{H'}{H}\, F'.
\end{equation} 
This implies the equations $(r\phi)$ and $(\theta\theta)$.

As explained below \eqref{gen:eq:complexification-functions} the $\theta$-dependence should be contain in $F(\theta)$ only.
The second term of the function $\tilde f$ contains some lonely $\theta$ from the $H(\theta)$ in the function $K$: this means that they should be compensated by the $F$, and we therefore ask that the sum $\kappa F + K$ be constant\footnotemark{}%
\footnotetext{%
	In \cref{sec:derivation:relaxing:metric-function} we relax this last assumption by allowing non-constant $\kappa F + K$.
	In this context the equations and the function $\tilde f$ are modified and this provides an explanation for the Demiański's error in $\tilde f$ in~\cite{Demianski:1972:NewKerrlikeSpacetime}.
}
\begin{equation}
	\kappa\, F' + K' = 0
	\quad \Longrightarrow \quad
	\kappa\, F + K = \kappa n.
\end{equation} 
The parameter $n$ is interpreted as the NUT charge.

The components $(t\theta)$ and $(\theta\phi)$ give the same equation
\begin{equation}
	\Lambda\, F' = 0.
\end{equation} 

Finally one can check that the last three equations $(tt), (t\phi)$ and $(\phi\phi)$ are satisfied.


\subsubsection{Summary of the equations}


The equations to be solved are
\begin{subequations}
\label{eq:topdown-1}
\begin{align}
	\label{eq:topdown-1-F-Gd}
	2 F &= G'' + \frac{H'}{H}\; G', \\
	\label{eq:topdown-1-Fd-Kd}
	\kappa\, n &= \kappa\, F + K, \\
	\label{eq:topdown-1-lambda}
	0 &= \Lambda F'
\end{align}
and the function $\tilde f$ is
\begin{equation}
	\label{eq:topdown-1-tilde-f}
	\tilde f = \kappa - \frac{2m r - q^2 + 2 F (\kappa\, F + K)}{r^2 + F^2} - \frac{\Lambda}{3}\, (r^2 + F^2) - \frac{4 \Lambda}{3}\, F^2 + \frac{8 \Lambda}{3}\, \frac{F^4}{r^2 + F^2}.
\end{equation}
We also defined
\begin{equation}
	\label{eq:topdown-1-K-Fd}
	2 K = F'' + \frac{H'}{H}\, F'.
\end{equation} 
\end{subequations}

As explained in the introduction the second step will be to explain \eqref{eq:topdown-1-tilde-f} in terms of new rules for the algorithm: they have been found in~\cite{Erbin:2016:DecipheringGeneralizingDemianskiJanisNewman} and this was the topic of \cref{sec:general:jna}.

In the next subsections we solve explicitly the equations \eqref{eq:topdown-1} in both cases $\Lambda \neq 0$ and $\Lambda = 0$.


\subsubsection{Solution for \texorpdfstring{$\Lambda \neq 0$}{non-vanishing cosmological constant}}


Equation \eqref{eq:topdown-1-lambda} implies that $F' = 0$, from which $K = 0$ follows by definition; then one obtains
\begin{equation}
	F(\theta) = n
\end{equation} 
by compatibility with \eqref{eq:topdown-1-Fd-Kd} and since $K(\theta) = 0$.

Solution to \eqref{eq:topdown-1-F-Gd} is
\begin{equation}
	G(\theta) = c_1 - 2 \kappa\, n \ln H(\theta) + c_2 \ln \frac{H(\theta/2)}{H'(\theta/2)}
\end{equation} 
where $c_1$ and $c_2$ are two constants of integration.
Since only $G'$ appears in the metric we can set $c_1 = 0$.
On the other hand the constant $c_2$ can be removed by the transformation
\begin{equation}
	\dd u = \dd u' - c_2\, \dd\phi
\end{equation} 
since one has
\begin{equation}
	\left( \ln \frac{H(\theta/2)}{H'(\theta/2)} \right)' = \frac{1}{H(\theta)}.
\end{equation} 

The solution to the system \eqref{eq:topdown-1} is thus
\begin{equation}
	F(\theta) = n, \qquad
	G(\theta) = - 2 \kappa\, n \ln H(\theta).
\end{equation} 
The function $\tilde f$ then takes the form
\begin{equation}
	\label{eq:topdown-1:tilde-f-lambda}
	\tilde f = \kappa - \frac{2m r - q^2 + 2 \kappa n^2}{r^2 + n^2} - \frac{\Lambda}{3}\,\frac{r^4 + 6 n^2 r^2 - 3 n^4}{r^2 + n^2}.
	% \kappa - \frac{2m r - q^2 + 2 \kappa n^2}{r^2 + n^2} - \frac{\Lambda}{3} (r^2 + 5 n^2) + \frac{8 \Lambda}{3}\, \frac{n^4}{r^2 + n^2}
\end{equation} 
This corresponds to the (a)dS--Schwarzschild--NUT solution: compare with \eqref{ext:eq:nut-tilde-f} and \eqref{gen:eq:rotating:tr-F-cst}.

The parameter $\Delta$ in the BL transformation \eqref{gen:eq:change:bl:delta} is
\begin{equation}
	\Delta = \kappa r^2 - 2 m r + q^2 + \Lambda n^4 - \frac{\Lambda}{3}\, r^4 - n^2 (\kappa + 2 \Lambda r^2 ).
\end{equation} 

As noted by Demiański the only parameters that appear are the mass and the NUT charge, and it is not possible to add angular momentum for non-vanishing cosmological constant.\footnotemark{}%
\footnotetext{%
	In~\cite{Leigh:2014:GerochGroupEinstein} Leigh et al.\ generalized Geroch's solution generating technique and also found that only the mass and the NUT charge appear when $\Lambda \neq 0$. We would like to thank D.\ Klemm for this remark.
}
As a consequence the JN algorithm cannot provide a derivation of the (a)dS--Kerr--Newman solution.


\subsubsection{Solution for \texorpdfstring{$\Lambda = 0$}{vanishing cosmological constant}}
\label{sec:derivation:stationary:solution-no-cosmo}


The solution to the differential equation \eqref{eq:topdown-1-Fd-Kd} is
\begin{equation}
	F(\theta) = n - a\, H'(\theta) + c \left( 1 + H'(\theta)\, \ln \frac{H(\theta/2)}{H'(\theta/2)} \right)
\end{equation}
where $a$ and $c$ denote two constants of integration.

We solve the equation \eqref{eq:topdown-1-F-Gd} for $G$
\begin{equation}
	\begin{aligned}
		G(\theta) = c_1 &+ \kappa\, a\, H'(\theta)
			- \kappa\, c\, H'(\theta)\, \ln \frac{H(\theta/2)}{H'(\theta/2)}
			- 2 \kappa\, n \ln H(\theta) \\
			&+ (a + c_2) \ln \frac{H(\theta/2)}{H'(\theta/2)}
	\end{aligned}
\end{equation} 
and $c_1, c_2$ are constants of integration.
Again since only $G'$ appears in the metric we can set $c_1 = 0$.
We can also remove the last term with the transformation
\begin{equation}
	\dd u = \dd u' - (c_2 + a) \dd\phi.
\end{equation} 
One finally gets
\begin{subequations}
\begin{align}
	F(\theta) &= n - a\, H'(\theta) + c \left( 1 + H'(\theta)\, \ln \frac{H(\theta/2)}{H'(\theta/2)} \right), \\
	G(\theta) &= \kappa\, a\, H'(\theta)
		- \kappa\, c\, H'(\theta)\, \ln \frac{H(\theta/2)}{H'(\theta/2)}
		- 2 \kappa\, n \ln H(\theta).
\end{align}
\end{subequations}

This solution was already found in~\cite{Krori:1981:ChargedDemianskiMetric} for the case $\kappa = 1$ by solving directly Einstein--Maxwell equations, starting with a metric ansatz of the form \eqref{deriv:eq:rotating:ur}.
Our aim was to show that the same solution can be obtained by applying Demiański's method to all the quantities, including the gauge field.

The BL transformation is well defined only for $c = 0$, in which case
\begin{equation}
	g = \frac{r^2 + a^2 + n^2}{\Delta}, \qquad
	h = \frac{\kappa a}{\Delta}, \qquad
	\Delta = \kappa r^2 - 2 m r + q^2 - \kappa n^2 + \kappa a^2.
\end{equation} 
The function $\tilde f$ reads
\begin{equation}
	\label{eq:topdown-1:tilde-f-no-Lambda-no-c}
	\tilde f = \kappa - \frac{2 m r - q^2}{\rho^2} + \frac{\kappa\, n (n - a H')}{\rho^2}, \qquad
	\rho^2 = r^2 + (n - a\, H')^2
\end{equation} 
and this corresponds to the Kerr--Newman--NUT solution~\cite[sec.~2.2]{AlonsoAlberca:2000:SupersymmetryTopologicalKerrNewmannTaubNUTaDS}.


\subsection{Relaxing assumptions}
\label{sec:derivation:relaxing}


In the derivation of \cref{sec:derivation:stationary:simplifying} we have made two assumptions in order to recover the simplest case.
The goal of this section is to show how these assumptions can be lifted, even if this does not lead to useful results: one cannot solve the equations in one case while in the other it is not clear how to recast the result in terms of a complex transformation.


\subsubsection{Metric function \texorpdfstring{$F$}{F}-dependence}
\label{sec:derivation:relaxing:metric-function}


In \cref{sec:derivation:stationary:simplifying} we obtained the equation \eqref{eq:topdown-1-Fd-Kd}
\begin{equation}
	\kappa\, F + K = \kappa\, n, \qquad
	2 K = F'' + \frac{H'}{H}\, F'
\end{equation}
by requiring that the function \eqref{eq:topdown-1-tilde-f}
\begin{equation}
	\tilde f = \kappa - \frac{2m r - q^2 + 2 F (\kappa\, F + K)}{r^2 + F^2} - \frac{\Lambda}{3}\, (r^2 + F^2) - \frac{4 \Lambda}{3}\, F^2 + \frac{8 \Lambda}{3}\, \frac{F^4}{r^2 + F^2}
\end{equation} 
depends on $\theta$ only through $F(\theta)$.
A more general assumption would be that $\kappa F + K$ is some function $\chi = \chi(F)$
\begin{equation}
	\label{eq:topdown-1-Fd-Kd-chi}
	\kappa\, F + K = \kappa\, \chi(F).
\end{equation} 
First if $F' = 0$ then $K = 0$ and the definition of $K$ implies
\begin{equation}
	\chi = F = n.
\end{equation} 
The $(t\theta)$- and $(\theta\phi)$-components give the equation
\begin{equation}
	4 \Lambda\, F^2 F' = F'\, \pd_F \chi.
\end{equation} 

If $\Lambda = 0$ we find that
\begin{equation}
	\pd_F \chi = 0
	\Longrightarrow
	\chi = n
\end{equation} 
which reduces to the case studied in \cref{sec:derivation:stationary:simplifying}, while if $F' = 0$ this equation does not provide anything.

On the other hand if $F' \neq 0$ and $\Lambda \neq 0$ then the previous equation becomes
\begin{equation}
	\pd_F \chi = 4 \Lambda F^2
\end{equation} 
which can be integrated to
\begin{equation}
	\label{top-down:eq:chi-F-solution}
	\chi(F) = n + \frac{4}{3}\, \Lambda F^3
\end{equation} 
(notice that the limit $\Lambda \to 0$ is coherent).
Plugging this function into equation \eqref{eq:topdown-1-Fd-Kd-chi} one obtains
\begin{equation}
	\label{eq:topdown-1-Fd-Kd-chi-replaced}
	\kappa\, F + K = \kappa \left(n + \frac{4}{3}\, \Lambda F^3 \right)
\end{equation} 
(remember that $F' \neq 0$).
This differential equation is non-linear and we were not able to find an analytical solution.
Despite that this provides a generalization of the algorithm with non-constant $F$ in the presence of a cosmological constant this is not sufficient for obtaining (a)dS--Kerr: the form of $g_{\theta\theta}$ given in \eqref{deriv:eq:rotating:tr} is not the required one.

Nonetheless by inserting the expression of $\chi$ in $\tilde f$ we see that the last term is killed
\begin{equation}
	\tilde f = \kappa - \frac{2m r - q^2 + 2 \kappa\, n\, F}{r^2 + F^2} - \frac{\Lambda}{3}\, (r^2 + F^2) - \frac{4 \Lambda}{3}\, F^2.
\end{equation} 
One can recognize the function given by Demiański~\cite{Demianski:1972:NewKerrlikeSpacetime} and may explain his error.


\subsubsection{Gauge field integration constant}
\label{sec:derivation:relaxing:gauge-fields}


In \cref{sec:derivation:stationary:simplifying} we obtained a second integration constant $\alpha$ in the expression of the gauge field
\begin{equation}
	\tilde f_A = \frac{q\, r}{r^2 + F^2} + \alpha\, \frac{r^2 - F^2}{r^2 + F^2}.
\end{equation}
One of the Maxwell equation gives $\alpha = 0$ if $F' \neq 0$, but otherwise no equation fixes its value.
For this reason we focus on the case $F' = 0$ or equivalently $\Lambda \neq 0$ through equation \eqref{eq:topdown-1-lambda}.

In this case the function $\tilde f$ is modified to
\begin{equation}
	\tilde f = \kappa - \frac{2m r - q^2 + 2 F (\kappa\, F + K) + 4 \alpha^2 F^2}{r^2 + F^2} - \frac{\Lambda}{3}\, (r^2 + F^2) - \frac{4 \Lambda}{3}\, F^2 + \frac{8 \Lambda}{3}\, \frac{F^4}{r^2 + F^2}.
\end{equation} 
Equation \eqref{eq:topdown-1-lambda} is modified but it is still solved by $F' = 0$ and all other equations are left unchanged (in particular $\kappa F + K$ is still given by the function $\chi(F)$ \eqref{top-down:eq:chi-F-solution}).
For $\chi(F) = n$ the configuration with $\alpha \neq 0$ provides another solution when $\Lambda \neq 0$ but it is not clear how to get it from a complexification of the function.






\section{Discussion on Approximation \textit{vs} Stability and Recovery}\label{sec:approx-stability}


In the world of approximation algorithms, for a maximization problem for which an algorithm outputs $S$ and the optimum is $S^*$, what we typically try to prove is that
$w(S)\ge w(S^*)/\alpha$, even in the worst case; this \textit{approximation inequality} means that the algorithm at hand is an $\alpha$-approximation, so it is a \textit{good} algorithm. Though one might be quick to say that recovery of $\alpha$-stable instances immediately follows from the approximation inequality, this is not true because of the intersection $S\cap S^*$; if we have no intersection, then recovery indeed follows. 

What the research on stability and exact recovery suggests, is that we should try to understand if some of our already known approximation algorithms have the stronger property $w(S\setminus S^*)\ge w(S^*\setminus S)/\alpha$ or at least if they have it on stable instances. We refer to the latter as the \textit{recovery inequality}. This would directly imply an exact recovery result for $\alpha$-stable instances because we could $\alpha$-perturb only the $S\setminus S^*$ part of the input and get: 
\[
\noindent w(S\setminus S^*)\ge w(S^*\setminus S)/\alpha \implies \alpha\cdot w(S\setminus S^*) +w(S\cap S^*) \ge w(S^*\setminus S) +w(S\cap S^*) = w(S^*)
\] thus violating the fact we were given an $\alpha$-stable instance, unless $S\setminus S^* = \emptyset$.

This would mean that the algorithm successfully retrieved $S^*$ and could potentially explain why many approximation algorithms behave far better in practice than in theory. Furthermore, from a theory perspective, it would mean that many results from the well-studied area of approximation algorithms could be translated in terms of stability and recovery.

As a concluding remark, we want to point out that even though one might think that an $\alpha$-approximation algorithm needs at least $\alpha$-stability for recovery, this is not true as the somewhat counterintuitive result from \cite{balcan2015k} tells us: asymmetric $k$-center cannot be approximated to any constant factor, but can be solved optimally on 2-stable instances. This was the
first problem that is hard to approximate to any constant factor in the worst case, yet can be optimally
solved in polynomial time for 2-stable instances. The other direction (having an $\alpha$-approximation algorithm that cannot recover arbitrarily stable instances) is also true. These findings suggest that there are interesting connections between stability, exact recovery and approximation.

\section{Five dimensional algorithm}
\label{sec:five}


While in four dimensions we have at our disposal many theorems on the classification of solutions, this is not the case for higher dimensions and the bestiary for solutions is much wider and less understood~\cite{Emparan:2008:BlackHolesHigher, Adamo:2014:KerrNewmanMetricReview}.
Rotating solutions in higher dimensions are characterized by several angular momenta.
Important solutions have not yet been discovered, even in the simplest theories such as the charged rotating black holes with several angular momenta in pure Einstein--Maxwell gravity.

Generalizing the JN algorithm in other dimensions is challenging and only small steps have been taken in this direction.
For instance Xu recovered Myers--Perry solution with one angular momentum~\cite{Myers:1986:BlackHolesHigher} from the Schwarzschild--Tangherlini solution~\cite{Xu:1988:ExactSolutionsEinstein} (see also~\cite{Aliev:2006:RotatingBlackHoles}), and Kim showed how the rotating BTZ black hole~\cite{Banados:1992:BlackHoleThree} can be obtained from its static limit~\cite{Kim:1997:NotesSpinningAdS3, Kim:1999:SpinningBTZBlack}.
One of the difficulty is to be able to perform several successive transformations in order to introduce all the allowed angular momenta.

In this section we report the successful generalization of the JN algorithm to five dimensions where we recover two examples~\cite{Erbin:2015:FivedimensionalJanisNewmanAlgorithm}: the complete Myers--Perry black hole~\cite{Myers:1986:BlackHolesHigher} and the Breckenridge--Myers--Peet--Vafa (BMPV) extremal black hole~\cite{Breckenridge:1997:DbranesSpinningBlack}.
We give of proposal for extending this method to higher dimensions in the next section.

It appears that the two angular momenta can be added one after the other by performing two successive transformations, each using different rules for complexifying the functions.
These rules can be understood as transforming only the functions that appear in the part of the metric which describes the rotation plane associated to the angular momentum.
Our method makes use of the Giampieri prescription and we did not succeed in expressing it in terms of the Janis--Newman prescription.

A major application of our work would be to find the charged solution with two angular momenta of the $5d$ Einstein--Maxwell gravity.
This problem is highly non-trivial and there is few chances that this technique would work directly~\cite{Aliev:2006:RotatingBlackHoles}, but one can imagine that a generalization of Demiański's approach~\cite{Demianski:1972:NewKerrlikeSpacetime} (see \cref{sec:derivation}) could lead to new interesting solutions in five dimensions.
An intermediate step is represented by the CCLP metric~\cite{Chong:2005:GeneralNonExtremalRotating} which is a solution of the Einstein--Maxwell theory with a Chern--Simons term, but it cannot be derived from the JN algorithm and we give some intuition about this fact in the last subsection.

Finally one could seek for an extension of the algorithm to the derivation of black rings~\cite{Emparan:2002:RotatingBlackRing, Emparan:2008:BlackHolesHigher}.
Similarly it may be possible that such techniques could be used in $d = 4$ to derive multicentre solutions (for instance one could imagine adding rotation to both centres successively, changing coordinate system in-between to place the origin of the coordinates at each centre).


\subsection{Myers--Perry black hole}
\label{sec:higher-jna:5d:myers-perry}


In this section we show how to recover the Myers--Perry black hole in five dimensions through the Giampieri prescription.
This is a solution of $5$-dimensional pure Einstein theory which possesses two angular momenta and it generalizes the Kerr black hole.
The importance of this solution lies in the fact that it can be constructed in any dimension.

The seed metric is given by the five-dimensional Schwarzschild--Tangherlini metric
\begin{equation}
	\dd s^2 = - f(r)\, \dd t^2 + f(r)^{-1}\, \dd r^2 + r^2\, \dd \Omega_3^2
\end{equation}
where $\dd \Omega_3^2$ is the metric on $S^3$, which can be expressed in Hopf coordinates (see \cref{app:coord:5d:hopf})
\begin{equation}
	\label{higher-jna:eq:coord-S3-spherical}
	\dd \Omega_3^2 = \dd\theta^2 + \sin^2 \theta\, \dd\phi^2 + \cos^2 \theta\, \dd\psi^2,
\end{equation} 
and the function $f(r)$ is given by
\begin{equation}
	f(r) = 1 - \frac{m}{r^2}.
\end{equation}

An important feature of the JN algorithm is the fact that a given set of transformations in the $(r,\phi)$-plane generates rotation in the latter.
Generating several angular momenta in different 2-planes would then require successive applications of the JN algorithm on different hypersurfaces.
In order to do so, one has to identify what are the 2-planes which will be submitted to the algorithm.
In five dimensions, the two different planes that can be made rotating are the planes $(r,\phi)$ and $(r,\psi)$.
We claim that it is necessary to dissociate the radii of these 2-planes in order to apply separately the JN algorithm on each plane and hence to generate two distinct angular momenta.
In order to dissociate the parts of the metric that correspond to the rotating and non-rotating $2$-planes, one can protect the function $r^2$ to be transformed under complex transformations in the part of the metric defining the plane which will stay static.
We thus introduce the function
\begin{equation}
	R(r) = r
\end{equation} 
such that the metric in null coordinates reads
\begin{equation}
	\label{higher-jna:5d-jna:metric:static:general-ur}
	\dd s^2 = - \dd u\, (\dd u + 2 \dd r)
		+ (1 - f)\, \dd u^2
		+ r^2 (\dd\theta^2 + \sin^2 \theta\, \dd\phi^2) + R^2 \cos^2 \theta\, \dd\psi^2.
\end{equation} 
The first transformation -- hence concerning the $(r,\phi)$-plane -- is
\begin{equation}
	\label{higher-jna:eq:5d-ansatz-hopf-1}
	\begin{gathered}
		u = u' + i a \cos \chi_1, \qquad
		r = r' - i a \cos \chi_1, \\
		i\, \dd \chi_1 = \sin \chi_1\ \dd\phi, \qquad\text{~~with~~}\chi_1 = \theta, \\
		\dd u = \dd u' - a \sin^2 \theta\, \dd\phi, \qquad
		\dd r = \dd r' + a \sin^2 \theta\, \dd\phi,
	\end{gathered}
\end{equation}
and $f$ is replaced by $\tilde f^{\{1\}} = \tilde f^{\{1\}}(r, \theta)$.
Indeed one needs to keep track of the order of the transformation, since the function $f$ will be complexified twice consecutively.
On the other hand $R(r) = \Re(r)$ is transformed\footnotemark{} into $R' = r'$ and one finds (omitting the primes)%
\footnotetext{%
	Note that as a function this corresponds to the rule \eqref{gen:eq:rules:r} but we will see below that $R$ is better interpreted as a coordinate since below it will appear as $\dd R$.
}
\begin{equation}
	\begin{aligned}
	\dd s^2 = - \dd u^2 &- 2\, \dd u \dd r
		+ \big(1 - \tilde f^{\{1\}} \big) (\dd u - a \sin^2 \theta\, \dd \phi)^2
		+ 2 a \sin^2 \theta\, \dd r \dd \phi \\
		&+ (r^2 + a^2 \cos^2 \theta) \dd\theta^2
		+ (r^2 + a^2) \sin^2 \theta\, \dd\phi^2
		+ r^2 \cos^2 \theta\, \dd \psi^2.
	\end{aligned}
\end{equation} 
The function $\tilde f^{\{1\}}$ is
\begin{equation}
	\tilde f^{\{1\}} = 1 - \frac{m}{\abs{r}^2} = 1 - \frac{m}{r^2 + a^2 \cos^2 \theta}.
\end{equation} 
There is a cancellation between the $(u, r)$ and the $(\theta, \phi)$ parts of the metric
\begin{subequations}
\begin{align}
	\dd s_{u,r}^2 &= (1 - \tilde f^{\{1\}})\, (\dd u - a \sin^2 \theta\, \dd \phi)^2
		- \dd u (\dd u + 2 \dd r )
		+ 2 a \sin^2 \theta \, \dd r \dd \phi
		+ a^2 \sin^4 \theta\, \dd \phi^2, \\
	\dd s_{\theta,\phi}^2 &= (r^2 + a^2 \cos^2 \theta) \dd\theta^2
			+ \big(r^2 + a^2 (1 - \sin^2 \theta) \big) \sin^2 \theta\, \dd\phi^2.
\end{align}
\end{subequations}

In addition to the terms present in \eqref{higher-jna:5d-jna:metric:static:general-ur} one obtains new components corresponding to the rotation of the first plane $(r, \phi)$.
Since the structure is very similar one can perform a transformation\footnotemark{} in the second plane $(r, \psi)$%
\footnotetext{%
	The easiest justification for choosing the sinus here is by looking at the transformation in terms of direction cosines, see \cref{sec:higher-jna:examples:myers-perry-5d}.
	Otherwise this term can be guessed by looking at Myers--Perry non-diagonal terms.
}
\begin{equation}
	\label{higher-jna:eq:5d-ansatz-hopf-2}
	\begin{gathered}
		u = u' + i b\, \sin \chi_2, \qquad
		r = r' - i b\, \sin \chi_2, \\
		i\, \dd \chi_2 = - \cos \chi_2\, \dd\psi, \qquad \text{~~with~~}\chi_2 = \theta, \\
		\dd u = \dd u' - b \cos^2 \theta\, \dd\psi, \qquad
		\dd r = \dd r' + b \cos^2 \theta\, \dd\psi,
	\end{gathered}
\end{equation}
can be applied directly to the metric
\begin{equation}
	\begin{aligned}
	\dd s^2 = - \dd u^2 &- 2\, \dd u \dd r
		+ \big(1 - \tilde f^{\{1\}} \big) (\dd u - a \sin^2 \theta\, \dd \phi)^2
		+ 2 a \sin^2 \theta\, \dd R \dd \phi \\
		&+ \rho^2 \dd\theta^2
		+ (R^2 + a^2) \sin^2 \theta\, \dd\phi^2
		+ r^2 \cos^2 \theta\, \dd \psi^2
	\end{aligned}
\end{equation} 
where we introduced once again the function $R(r) = \Re(r)$ to protect the geometry of the first plane to be transformed under complex transformations.

The final result (using again $R = r'$ and omitting the primes) becomes
\begin{equation}
	\begin{aligned}
	\dd s^2 = - \dd u^2 &- 2\, \dd u \dd r
		+ \big(1 - \tilde f^{\{1, 2\}} \big) (\dd u - a \sin^2 \theta\, \dd \phi - b \cos^2 \theta\, \dd \psi)^2
		\\
		&+ 2 a \sin^2 \theta\, \dd r \dd \phi
		+ 2 b \cos^2 \theta\, \dd r\dd \psi \\
		&+ \rho^2 \dd\theta^2
		+ (r^2 + a^2) \sin^2 \theta\, \dd\phi^2
		+ (r^2 + b^2) \cos^2 \theta\, \dd \psi^2
	\end{aligned}
\end{equation} 
where
\begin{equation}
	\rho^2 = r^2 + a^2 \cos^2 \theta + b^2 \sin^2 \theta.
\end{equation} 
Furthermore, the function $\tilde f^{\{1\}}$ has been complexified as
\begin{equation}
 	\tilde f^{\{1,2\}} = 1 - \frac{m}{\abs{r}^2 + a^2 \cos^2 \theta}
		= 1 - \frac{m}{r'^2 + a^2 \cos^2 \theta + b^2 \sin^2 \theta}
		= 1 - \frac{m}{\rho^2}.
\end{equation}

The metric can then be transformed into the Boyer--Lindquist (BL) using
\begin{equation}
	\label{higher:change:5d-bl}
	\dd u = \dd t - g(r)\, \dd r, \qquad
	\dd\phi = \dd\phi' - h_\phi(r)\, \dd r, \qquad
	\dd\psi = \dd\psi' - h_\psi(r)\, \dd r.
\end{equation} 
Defining the parameters\footnotemark{}%
\footnotetext{%
	See \eqref{higher-jna:metric:rotating:result-jna-bl-parameters} for a definition of $\Delta$ in terms of $\tilde f$.
}
\begin{equation}
	\Pi = (r^2 + a^2) (r^2 + b^2), \qquad
	\Delta = r^4 + r^2 (a^2 + b^2- m) + a^2 b^2,
\end{equation}
the functions can be written
\begin{equation}
	\label{higher:change:myers-perry:bl-g-h}
	g(r) = \frac{\Pi}{\Delta}, \qquad
	h_\phi(r) = \frac{\Pi}{\Delta}\, \frac{a}{r^2 + a^2}, \qquad
	h_\psi(r) = \frac{\Pi}{\Delta}\, \frac{b}{r^2 + b^2}.
\end{equation} 
Finally one gets
\begin{equation}
	\label{higher-jna:metric:rotating:5d-2-moments-bl}
	\begin{aligned}
		\dd s^2 = - \dd t^2
			&+ \big(1 - \tilde f^{\{1, 2\}} \big) (\dd t - a \sin^2 \theta\, \dd \phi - b \cos^2 \theta\, \dd \psi)^2
			+ \frac{r^2 \rho^2}{\Delta}\, \dd r^2 \\
			&+ \rho^2 \dd\theta^2
			+ (r^2 + a^2) \sin^2 \theta\, \dd\phi^2
			+ (r^2 + b^2) \cos^2 \theta\, \dd \psi^2.
	\end{aligned}
\end{equation} 
One recovers here the five dimensional Myers--Perry black hole with two angular momenta~\cite{Myers:1986:BlackHolesHigher}.


\subsection{BMPV black hole}
\label{sec:higher-jna:5d:bmpv}




\subsubsection{Few properties and seed metric}


In this section we focus on another example in five dimensions, which is the BMPV black hole~\cite{Breckenridge:1997:DbranesSpinningBlack, Gauntlett:1999:BlackHolesD5}.
This solution possesses many interesting properties, in particular it can be proven that it is the only asymptotically flat rotating BPS black hole in five dimensions with the corresponding near-horizon geometry~\cites[sec.~7.2.2, 8.5]{Emparan:2008:BlackHolesHigher}{Reall:2003:HigherDimensionalBlack}.\footnotemark{}%
\footnotetext{%
	Other possible near-horizon geometries are $S^1 \times S^2$ (for black rings) and $T^3$, even if the latter does not seem really physical.
	BMPV horizon corresponds to the squashed $S^3$.
}
It is interesting to notice that even if this extremal solution is a slowly rotating metric, it is an exact solution (whereas Einstein equations need to be truncated for consistency of usual slow rotation).

For a rotating black hole the BPS and extremal limits do not coincide~\cites[sec.~7.2]{Emparan:2008:BlackHolesHigher}[sec.~1]{Gauntlett:1999:BlackHolesD5}: the first implies that the mass is related to the electric charge,\footnote{It is a consequence from the BPS bound $m \ge \sqrt{3}/2\, \abs{q}$.} while extremality\footnotemark{}%
\footnotetext{%
	Regularity is given by a bound, which is saturated for extremal black holes.
}
implies that one linear combination of the angular momenta vanishes, and for this reason we set $a = b$ from the beginning.\footnotemark{}%
\footnotetext{%
	If we had kept $a \neq b$ we would have discovered later that one cannot transform the metric to Boyer--Lindquist coordinates without setting $a = b$.
}
Thus two independent parameters are left and are taken to be the mass and one angular momentum.

In the non-rotating limit BMPV black hole reduces to the charged extremal Schwarz\-schild--Tangherlini (with equal mass and charge) written in isotropic coordinates.
For non-rotating black hole the extremal and BPS limit are equivalent.

Both the charged extremal Schwarzschild--Tangherlini and BMPV black holes are solutions of minimal ($N = 2$) $d = 5$ supergravity (Einstein--Maxwell plus Chern--Simons) whose bosonic action is~\cites[sec.~1]{Gauntlett:1999:BlackHolesD5}[sec.~2]{Aliev:2014:SuperradianceBlackHole}[sec.~2]{Gauntlett:2003:AllSupersymmetricSolutions}
\begin{equation}
	\label{higher-jna:higher-jna:action:N=2-d=5-sugra}
	S = - \frac{1}{16\pi G} \int \left(R\, \hodge{1} + F \wedge \hodge{F} + \frac{2\lambda}{3 \sqrt{3}}\, F \wedge F \wedge A \right),
\end{equation} 
where supersymmetry imposes $\lambda = 1$.

Since extremal limits are different for static and rotating black holes we can guess that the black hole obtained from the algorithm will not be a solution of the equations of motion and that it will be necessary to take some limit.

The charged extremal Schwarzschild--Tangherlini black hole is taken as a seed metric~\cites[sec.~3.2]{Gauntlett:2003:AllSupersymmetricSolutions}[sec.~4]{Gibbons:1994:SupersymmetricSelfGravitatingSolitons}[sec.~1.3.1]{Puhm:2013:BlackHolesString}
\begin{equation}
	\label{higher-jna:metric:5d-bmpv}
	\dd s^2 = - H^{-2}\, \dd t^2 + H\, (\dd r^2 + r^2\, \dd\Omega_3^2 )
\end{equation} 
where $\dd\Omega_3^2$ is the metric of the $3$-sphere written in
\eqref{higher-jna:eq:coord-S3-spherical}.
The function $H$ is harmonic
\begin{equation}
	H(r) = 1 + \frac{m}{r^2},
\end{equation} 
and the electromagnetic field reads
\begin{equation}
	\label{higher-jna:pot:5d-bmpv}
	A = \frac{\sqrt{3}}{2 \lambda}\, \frac{m}{r^2}\, \dd t
		= (H - 1)\, \dd t.
\end{equation} 

In the next subsections we apply successively the transformations \eqref{higher-jna:eq:5d-ansatz-hopf-1} and \eqref{higher-jna:eq:5d-ansatz-hopf-2} with $a = b$ in the case $\lambda = 1$.


\subsubsection{Transforming the metric}


The transformation to $(u, r)$ coordinates of the seed metric \eqref{higher-jna:metric:5d-bmpv}
\begin{equation}
	\dd t = \dd u + H^{3/2}\, \dd r
\end{equation} 
gives
\begin{subequations}
\begin{align}
	\dd s^2 &= - H^{-2}\, \dd u^2 - 2 H^{-1/2}\, \dd u \dd r + H r^2\, \dd\Omega_3^2 \\
		&= - H^{-2}\, \big(\dd u - 2 H^{3/2}\, \dd r \big)\, \dd u + H r^2\, \dd\Omega_3^2.
\end{align}
\end{subequations}

For transforming the above metric one should follow the recipe of the previous section: the transformations \eqref{higher-jna:eq:5d-ansatz-hopf-1}
\begin{equation}
	u = u' + i a \cos \theta, \qquad
	\dd u = \dd u' - a \sin^2 \theta\, \dd\phi,
\end{equation}
and \eqref{higher-jna:eq:5d-ansatz-hopf-2}
\begin{equation}
	u = u' + i a\, \sin \theta, \qquad
	\dd u = \dd u' - a \cos^2 \theta\, \dd\psi
\end{equation} 
are performed one after another, transforming each time only the terms that pertain to the corresponding rotation plane.\footnotemark{}%
\footnotetext{%
	For another approach see \cref{sec:higher-jna:5d:bmpv-second-approach}.
}
In order to preserve the isotropic form of the metric the function $H$ is complexified everywhere (even when it multiplies terms that belong to the other plane).

Since the procedure is exactly similar to the Myers--Perry case we give only the final result in $(u, r)$ coordinates
\begin{equation}
	\label{higher-jna:metric:5d-bmpv:ur-before-limit}
	\begin{aligned}
		\dd s^2 = &- \tilde H^{-2} \big(\dd u
				- a (1 - \tilde H^{3/2}) (\sin^2 \theta\, \dd\phi + \cos^2 \theta\, \dd\psi) \big)^2 \\
			&- 2 \tilde H^{-1/2} \big(\dd u - a (1 - \tilde H^{3/2})\, (\sin^2 \theta\, \dd\phi + \cos^2 \theta\, \dd\psi) \big)\, \dd r \\
			&+ 2 a \tilde H\, (\sin^2 \theta\, \dd\phi + \cos^2 \theta\, \dd\psi)\, \dd r
			- 2 a^2 \tilde H \cos^2 \theta \sin^2 \theta\, \dd\phi \dd\psi
			\\
			&+ \tilde H\, \Big(
				(r^2 + a^2) (\dd \theta^2 + \sin^2 \theta\, \dd\phi^2 + \cos^2 \theta\, \dd\psi^2)
				+ a^2 (\sin^2 \theta\, \dd\phi + \cos^2 \theta\, \dd\psi)^2 \Big).
	\end{aligned}
\end{equation} 
After both transformations the resulting function $\tilde H$ is
\begin{equation}
	\label{higher-jna:eq:5d-bmpv:tilde-H}
	\tilde H = 1 + \frac{m}{r^2 + a^2 \cos^2\theta + a^2 \sin^2\theta}
		= 1 + \frac{m}{r^2 + a^2}
\end{equation}
which does not depend on $\theta$.

It is easy to check that the Boyer--Lindquist transformation \eqref{higher:change:5d-bl}
\begin{equation}
	\dd u = \dd t - g(r)\, \dd r, \qquad
	\dd\phi = \dd\phi' - h_\phi(r)\, \dd r, \qquad
	\dd\psi = \dd\psi' - h_\psi(r)\, \dd r
\end{equation} 
is ill-defined because the functions depend on $\theta$.
The way out is to take the extremal limit alluded above.

Following the prescription of \cite{Breckenridge:1997:DbranesSpinningBlack, Gauntlett:1999:BlackHolesD5} and taking the extremal limit
\begin{equation}
	\label{higher-jna:eq:5d-bmpv-extremal-limit}
	a, m \longrightarrow 0, \qquad
	\text{imposing} \qquad
	\frac{m}{a^2} = \cst ,
\end{equation}
one gets at leading order
\begin{equation}
	\tilde H(r) = 1 + \frac{m}{r^2} = H(r), \qquad
	a\, (1 - \tilde H^{3/2}) = - \frac{3\, m a}{2\, r^2}
\end{equation} 
which translate into the metric
\begin{equation}
	\begin{aligned}
		\dd s^2 = - H^{-2}\, & \left(\dd u
				+ \frac{3\, m a}{2\, r^2}\, (\sin^2 \theta\, \dd\phi + \cos^2 \theta\, \dd\psi) \right)^2 \\
			&- 2 H^{-1/2} \left( \dd u + \frac{3\, m a}{2\, r^2}\, (\sin^2 \theta\, \dd\phi + \cos^2 \theta\, \dd\psi) \right) \dd r \\
			&+ H\, r^2 (\dd \theta^2 + \sin^2 \theta\, \dd\phi^2 + \cos^2 \theta\, \dd\psi^2).
	\end{aligned}
\end{equation} 
Then Boyer--Lindquist functions are
\begin{equation}
	\label{higher:change:bmpv:g-h}
	g(r) = H(r)^{3/2}, \qquad
	h_\phi(r) = h_\psi(r) = 0
\end{equation} 
and one gets the metric in $(t, r)$ coordinates
\begin{equation}
	\label{higher-jna:metric:5d-bmpv:bmpv-metric}
	\begin{aligned}
		\dd s^2 = &- \tilde H^{-2} \left(\dd t
				+ \frac{3\, m a}{2\, r^2}\, (\sin^2 \theta\, \dd\phi + \cos^2 \theta\, \dd\psi) \right)^2 \\
			&+ \tilde H\, \Big(\dd r^2 + r^2 \big( \dd \theta^2 + \sin^2 \theta\, \dd\phi^2 + \cos^2 \theta\, \dd\psi^2 \big) \Big).
	\end{aligned}
\end{equation} 
One can recognize the BMPV solution~\cites[p.~4]{Breckenridge:1997:DbranesSpinningBlack}[p.~16]{Gauntlett:1999:BlackHolesD5}.
The fact that this solution has only one rotation parameter can be seen more easily in Euler angle coordinates~\cites[sec.~3]{Gauntlett:1999:BlackHolesD5}[sec.~2]{Gibbons:1999:SupersymmetricRotatingBlack} or by looking at the conserved charges in the $\phi$- and $\psi$-planes~\cite[sec.~3]{Breckenridge:1997:DbranesSpinningBlack}.


\subsubsection{Transforming the Maxwell potential}


The seed gauge field \eqref{higher-jna:pot:5d-bmpv} in the $(u, r)$ coordinates is
\begin{equation}
	A = \frac{\sqrt{3}}{2}\, (H - 1)\, \dd u,
\end{equation} 
since the $A_r(r)$ component can be removed by a gauge transformation.
One can apply the two JN transformations \eqref{higher-jna:eq:5d-ansatz-hopf-1} and \eqref{higher-jna:eq:5d-ansatz-hopf-2} with $b = a$ to obtain
\begin{equation}
	A = \frac{\sqrt{3}}{2}\, (\tilde H - 1) \Big( \dd u - a\, (\sin^2 \theta\, \dd\phi + \cos^2 \theta\, \dd\psi) \Big).
\end{equation} 

Then going into BL coordinates with \eqref{higher:change:5d-bl} and \eqref{higher:change:bmpv:g-h} provides
\begin{equation}
	A = \frac{\sqrt{3}}{2}\, (\tilde H - 1) \Big( \dd t - a\, (\sin^2 \theta\, \dd\phi + \cos^2 \theta\, \dd\psi) \Big) + A_r(r)\, \dd r.
\end{equation} 
Again $A_r$ depends only on $r$ and can be removed by a gauge transformation.
Applying the extremal limit \eqref{higher-jna:eq:5d-bmpv-extremal-limit} finally gives
\begin{equation}
	A = \frac{\sqrt{3}}{2}\, \frac{m}{r^2} \Big( \dd t - a\, (\sin^2 \theta\, \dd\phi + \cos^2 \theta\, \dd\psi) \Big),
\end{equation}
which is again the result presented in~\cite[p. 5]{Breckenridge:1997:DbranesSpinningBlack}.

Despite the fact that the seed metric \eqref{higher-jna:metric:5d-bmpv} together with the gauge field \eqref{higher-jna:pot:5d-bmpv} solves the equations of motion for any value of $\lambda$, the resulting rotating metric solves the equations only for $\lambda = 1$ (see~\cite[sec.~7]{Gauntlett:1999:BlackHolesD5} for a discussion).
An explanation in this reduction can be found in the limit \eqref{higher-jna:eq:5d-bmpv-extremal-limit} that was needed for transforming the metric to Boyer--Lindquist coordinates and which gives a supersymmetric black hole -- which necessarily has $\lambda = 1$.



\subsection{Another approach to BMPV}
\label{sec:higher-jna:5d:bmpv-second-approach}


In \cref{sec:higher-jna:5d:bmpv} we applied the same recipe given in \cref{sec:higher-jna:5d:myers-perry} which, according to our claim, is the standard procedure in five dimensions.

There is another way to derive BMPV black hole.
Indeed, by considering that terms quadratic in the angular momentum do not survive in the extremal limit, they can be added to the metric without modifying the final result.
Hence we can decide to transform all the terms of the metric\footnotemark{} since the additional terms will be subleading.%
\footnotetext{%
	In opposition to our initial recipe, but this is done in a controlled way.
}
As a result the BL transformation is directly well defined and overall formulas are simpler, but we need to take the extremal limit before the end (this could be done either in $(u, r)$ or $(t, r)$ coordinates).
This section shows that both approaches give the same result.

Applying the two transformations
\begin{subequations}
\begin{gather}
	u = u' + i a \cos \theta, \qquad
	\dd u = \dd u' - a \sin^2 \theta\, \dd\phi, \\
	u = u' + i a\, \sin \theta, \qquad
	\dd u = \dd u' - a \cos^2 \theta\, \dd\psi
\end{gather}
\end{subequations}
successively on all the terms one obtains the metric
\begin{equation}
	\begin{aligned}
		\dd s^2 = &- \tilde H^{-2} \big(\dd u
				- a (1 - \tilde H^{3/2}) (\sin^2 \theta\, \dd\phi + \cos^2 \theta\, \dd\psi) \big)^2 \\
			&- 2 \tilde H^{-1/2} \big(\dd u - a (\sin^2 \theta\, \dd\phi + \cos^2 \theta\, \dd\psi) \big)\, \dd r \\
			&+ \tilde H\, \Big(
				(r^2 + a^2) (\dd \theta^2 + \sin^2 \theta\, \dd\phi^2 + \cos^2 \theta\, \dd\psi^2)
				+ a^2 (\sin^2 \theta\, \dd\phi + \cos^2 \theta\, \dd\psi)^2 \Big),
	\end{aligned}
\end{equation} 
where again $\tilde H$ is given by \eqref{higher-jna:eq:5d-bmpv:tilde-H}
\begin{equation}
	\tilde H = 1 + \frac{m}{r^2 + a^2}.
\end{equation} 
Only one term is different when comparing with \eqref{higher-jna:metric:5d-bmpv:ur-before-limit}.

The BL transformation \eqref{higher:change:5d-bl} is well-defined and the corresponding functions are
\begin{equation}
	\label{higher-jna:change:bmpv-2:bl-gh}
	g(r) = \frac{a^2 + (r^2 + a^2) \tilde H(r)}{r^2 + 2 a^2}, \qquad
	h_\phi(r) = h_\psi(r) = \frac{a}{r^2 + 2 a^2}
\end{equation} 
which do not depend on $\theta$.
They lead to the metric
\begin{equation}
	\begin{aligned}
		\dd s^2 = &- \tilde H^{-2} \big(\dd t
				- a (1 - \tilde H^{3/2}) (\sin^2 \theta\, \dd\phi + \cos^2 \theta\, \dd\psi) \big)^2 \\
			&+ \tilde H\, \bigg[
				(r^2 + a^2) \left(\frac{\dd r^2}{r^2 + 2 a^2} + \dd \theta^2 + \sin^2 \theta\, \dd\phi^2 + \cos^2 \theta\, \dd\psi^2 \right) \\
				&\qquad\quad+ a^2 (\sin^2 \theta\, \dd\phi + \cos^2 \theta\, \dd\psi)^2 \bigg].
	\end{aligned}
\end{equation} 

At this point it is straightforward to check that this solution does not satisfy Einstein equations and we need to take the extremal limit \eqref{higher-jna:eq:5d-bmpv-extremal-limit}
\begin{equation}
	a, m \longrightarrow 0, \qquad
	\text{imposing} \qquad
	\frac{m}{a^2} = \cst
\end{equation}
in order to get the BMPV solution \eqref{higher-jna:metric:5d-bmpv:bmpv-metric}
\begin{equation}
	\begin{aligned}
		\dd s^2 = &- \tilde H^{-2} \left(\dd t
				+ \frac{3\, m a}{2\, r^2}\, (\sin^2 \theta\, \dd\phi + \cos^2 \theta\, \dd\psi) \right)^2 \\
			&+ \tilde H\, \Big(\dd r^2 + r^2 \big( \dd \theta^2 + \sin^2 \theta\, \dd\phi^2 + \cos^2 \theta\, \dd\psi^2 \big) \Big).
	\end{aligned}
\end{equation} 

It is surprising that the BL transformation is simpler in this case.
Another point that is worth stressing is that we did not need to take the extremal limit at an intermediate stage, whereas in \cref{sec:higher-jna:5d:bmpv} we had to in order to get a well-defined BL transformation.


\subsection{CCLP black hole}
\label{sec:higher-jna:5d:cclp}


The CCLP black hole~\cite{Chong:2005:GeneralNonExtremalRotating} (see also~\cite[sec.~2]{Aliev:2014:SuperradianceBlackHole}) corresponds to the non-extremal generalization of the BMPV solution and it possesses four independent charges: two angular momenta $a$ and $b$, an electric charge $q$ and the mass $m$.
It is a solution of $d = 5$ minimal supergravity \eqref{higher-jna:higher-jna:action:N=2-d=5-sugra}.

The solution reads
\begin{subequations}
\begin{gather}
	\label{higher-jna:higher-jna:metric:cclp}
	\begin{aligned}
		\dd s^2 = - \dd t^2
			&+ (1 - \tilde f) (\dd t - a \sin^2 \theta\, \dd \phi - b \cos^2 \theta\, \dd \psi)^2
			+ \frac{r^2 \rho^2}{\Delta_r}\, \dd r^2 \\
			&+ \rho^2 \dd\theta^2
			+ (r^2 + a^2) \sin^2 \theta\, \dd\phi^2
			+ (r^2 + b^2) \cos^2 \theta\, \dd \psi^2 \\
			&- \frac{2 q}{\rho^2}\, (b \sin^2 \theta\, \dd \phi + a \cos^2 \theta\, \dd \psi) (\dd t - a \sin^2 \theta\, \dd \phi - b \cos^2 \theta\, \dd \psi),
	\end{aligned} \\
	A = \frac{\sqrt{3}}{2}\, \frac{q}{\rho^2} (\dd t - a \sin^2 \theta\, \dd \phi - b \cos^2 \theta\, \dd \psi),
\end{gather}
\end{subequations}
where the function are given by
\begin{subequations}
\begin{align}
	\rho^2 &= r^2 + a^2 \cos^2 \theta + b^2 \sin^2 \theta, \\
	\tilde f &= 1 - \frac{2 m}{\rho^2} + \frac{q^2}{\rho^4}, \\
	\Delta_r &= \Pi + 2 a b q + q^2 - 2 m r^2.
\end{align}
\end{subequations}

Yet, using our prescription, it appears that the metric of this black hole cannot entirely be recovered.
Indeed while the gauge field can be found straightforwardly, all the terms of the metric but one are generated by our algorithm.
The missing term (corresponding to the last one in \eqref{higher-jna:higher-jna:metric:cclp}) is proportional to the electric charge and the current prescription cannot generate it since the latter can only appear in $\tilde f$ (or in the gauge field); moreover the algorithm cannot explain the first term in parenthesis since $a$ and $b$ always appear with $\dd\phi$ and $\dd\psi$ respectively.

This issue may be related to the fact that the CCLP solution cannot be written as a Kerr--Schild metric but rather as an extended Kerr--Schild one~\cite{Aliev:2009:NoteRotatingCharged, Ett:2010:ExtendedKerrSchildAnsatz, Malek:2014:ExtendedKerrSchildSpacetimes}, which includes an additional term proportional to a spacelike vector.
It appears that the missing term corresponds precisely to this additional term in the extended Kerr--Schild metric and it is well-known that the JN algorithm works mostly for Kerr--Schild metrics.
Moreover the $\Delta$ computed from \eqref{higher-jna:metric:rotating:result-jna-bl-parameters} depends on $\theta$ and the BL transformation would not be well-defined if the additional term is not present to modify $\Delta$ to $\Delta_r$.


\section{Algorithm in any dimension}
\label{sec:higher}


Following the same prescription in dimensions higher than five does not lead as nicely to the exact Myers--Perry solution.
Indeed we show in this section that, while the transformation of the metric can be done along the same line, the -- major -- obstacle comes from the function $f$ that cannot be transformed as expected.
Finding the correct complexification seems very challenging and it may be necessary to use a different complex coordinate transformation in order to perform a completely general transformation in any dimension.
It might be possible to gain insight into this problem by computing the transformation within the framework of the tetrad formalism.
One may think that a possible solution would be to replace complex numbers by quaternions, assigning one angular momentum to each complex direction but it is straightforward to check that this approach is not working.

The key element to perform the algorithm on the metric is to parametrize the metric on the sphere by direction cosines since these coordinates are totally symmetric under permutation of angular momenta (at the opposite of the spherical coordinates).
We are able to derive the general form of a rotating metric with the maximal number of angular momenta it can have in $d$ dimensions, but we are not able to apply this result to any specific example for $d \ge 6$, except if all momenta but one are vanishing.
Nonetheless this provides a unified view of the JN algorithm in any $d \ge 3$.
We conclude this section by few examples, including the singly-rotating Myers--Perry solution in any dimension and the rotating BTZ black hole.

It would be very desirable to derive the general $d$-dimensional Myers--Perry solution~\cite{Myers:1986:BlackHolesHigher}, or at least to understand why only the metric can be found, and not the function inside.


\subsection{Metric transformation}
\label{sec:higher-jna:any-dimension}


We consider the JN algorithm applied to a general static $d$-dimension metric and show how the tensor structure can be transformed.
In the following the dimension is taken to be odd in order to simplify the computations but the final result holds also for $d$ even.



\subsubsection{Seed metric and discussion}


Consider the $d$-dimensional static metric (notations are defined in \cref{app:coord:general-d})
\begin{equation}
	\dd s^2 = - f\, \dd t^2 + f^{-1}\, \dd r^2 + r^2\, \dd \Omega_{d-2}^2
\end{equation} 
where $\dd \Omega_{d-2}^2$ is the metric on $S^{d-2}$
\begin{equation}
	\dd \Omega_{d-2}^2 = \dd\theta_{d-2} + \sin^2 \theta_{d-2}\, \dd \Omega_{d-3}^2
		= \sum_{i=1}^n \big( \dd\mu_i^2 + \mu_i^2 \dd\phi_i^2).
\end{equation} 
The number $n = (d-1) / 2$ counts the independent $2$-spheres.

In Eddington--Finkelstein coordinates the metric reads
\begin{equation}
	\label{higher-jna:higher-jna:metric:static-seed}
	\dd s^2 = (1 - f)\, \dd u^2 - \dd u\, (\dd u + 2 \dd r)
			+ r^2 \sum_i \Big(\dd \mu_i^2 + \mu_i^2\, \dd \phi_i^2 \Big).
\end{equation} 

The metric looks like a $2$-dimensional space $(t, r)$ with a certain number of additional $2$-spheres $(\mu_i, \phi_i)$ which are independent from one another.
Then we can consider only the piece $(u, r, \mu_i, \phi_i)$ (for fixed $i$) which will transform like a $4$-dimensional spacetime, while the other part of the metric $(\mu_j, \phi_j)$ for all $j \neq i$ will be unchanged.
After the first transformation we can move to another $2$-sphere.
We can thus imagine to put in rotation only one of these spheres.
Then we will apply again and again the algorithm until all the spheres have angular momentum: the whole complexification will thus be a $n$-steps process.
Moreover if these $2$-spheres are taken to be independent this implies that we should not complexify the functions that are not associated with the plane we are putting in rotation.

To match these demands the metric is rewritten as
\begin{equation}
	\label{higher-jna:higher-jna:higher-jna:metric:static-seed-ur}
	\dd s^2 = (1 - f)\, \dd u^2 - \dd u\, (\dd u + 2 \dd r_{i_1})
		+ r_{i_1}^2 (\dd\mu_{i_1}^2 + \mu_{i_1}^2 \dd\phi_{i_1}^2)
		+ \sum_{i \neq i_1} \Big(r_{i_1}^2 \dd \mu_i^2 + R^2 \mu_i^2\, \dd \phi_i^2 \Big).
\end{equation} 
where we introduced the following two functions of $r$
\begin{equation}
	r_{i_1}(r) = r, \qquad R(r) = r.
\end{equation} 
This allows to choose different complexifications for the different terms in the metric.
It may be surprising to note that the factors in front of $\dd \mu_i^2$ have been chosen to be $r_{i_1}^2$ and not $R^2$, but the reason is that the $\mu_i$ are all linked by the constraint
\begin{equation}
	\sum_i \mu_i^2 = 1
\end{equation} 
and the transformation of one $i_1$-th $2$-sphere will change the corresponding $\mu_{i_1}$, but also all the others, as it is clear from the formula \eqref{coord:eq:spherical-to-oblate-mu} with all the $a_i$ vanishing but one (this can also be observed in $5d$ where both $\mu_i$ are gathered into $\theta$).


\subsubsection{First transformation}


The transformation is chosen to be
\begin{subequations}
\label{higher-jna:higher-jna:change:jna-1}
\begin{equation}
	r_{i_1} = r'_{i_1} - i\, a_{i_1} \sqrt{1 - \mu_{i_1}^2}, \qquad
	u = u' + i\, a_{i_1} \sqrt{1 - \mu_{i_1}^2}
\end{equation} 
which, together with the ansatz
\begin{equation}
	i\, \frac{\dd \mu_{i_1}}{\sqrt{1 - \mu_{i_1}^2}} = \mu_{i_1}\, \dd \phi_{i_1},
\end{equation} 
gives the differentials
\begin{equation}
	\dd r_{i_1} = \dd r'_{i_1} + a_{i_1} \mu_{i_1}^2\, \dd \phi_{i_1}, \qquad
	\dd u = \dd u' - a_{i_1} \mu_{i_1}^2\, \dd \phi_{i_1}.
\end{equation} 
\end{subequations}
It is easy to check that this transformation reproduces the one given in four and five dimensions.
The complexified version of $f$ is written as $\tilde f^{\{i_1\}}$: we need to keep track of the order in which we gave angular momentum since the function $\tilde f$ will be transformed at each step.

We consider separately the transformation of the $(u, r)$ and $\{ \mu_i, \phi_i \}$ parts.
Inserting the transformations \eqref{higher-jna:higher-jna:change:jna-1} in \eqref{higher-jna:higher-jna:metric:static-seed} results in
\begin{subequations}
\begin{align*}
	\dd s_{u,r}^2 &= (1 - \tilde f^{\{i_1\}})\, \Big(\dd u - a_{i_1} \mu_{i_1}^2\, \dd \phi_{i_1} \Big)^2
		- \dd u\, (\dd u + 2 \dd r_{i_1})
		+ 2 a_{i_1} \mu_{i_1}^2\, \dd r_{i_1} \dd \phi_{i_1}
		+ a_{i_1}^2 \mu_{i_1}^4\, \dd \phi_{i_1}^2, \\
	%
	\dd s_{\mu,\phi}^2 &= \big( r_{i_1}^2 + a_{i_1}^2 \big) (\dd\mu_{i_1}^2 + \mu_{i_1}^2 \dd\phi_{i_1}^2)
		+ \sum_{i \neq i_1} \big( r_{i_1}^2 \dd \mu_i^2 + R^2 \mu_i^2\, \dd \phi_i^2 \big) - a_{i_1}^2 \mu_{i_1}^4\, \dd \phi_{i_1}^2 \\
		&\qquad + a_{i_1}^2 \bigg[- \mu_{i_1}^2 \dd \mu_{i_1}^2 + (1 - \mu_{i_1}^2) \sum_{i \neq i_1} \dd \mu_i^2 \bigg].
\end{align*}
\end{subequations}

The term in the last bracket vanishes as can be seen by using the differential of the constraint
\begin{equation}
	\sum_i \mu_i^2 = 1 \Longrightarrow
	\sum_i \mu_i \dd\mu_i = 0.
\end{equation} 
Since this step is very important and non-trivial we expose the details
\begin{align*}
	[\cdots] &= \mu_{i_1}^2 \dd \mu_{i_1}^2 - (1 - \mu_{i_1}^2) \sum_{i \neq i_1} \dd \mu_i^2
		= \left(\sum_{i \neq i_1} \mu_i \dd\mu_i \right)^2 - \sum_{j \neq i_1} \mu_j^2 \sum_{i \neq i_1} \dd \mu_i^2 \\
		&= \sum_{i,j \neq i_1} \big(\mu_i \mu_j \dd\mu_i \dd\mu_j - \mu_j^2 \dd \mu_i^2 \big)
		= \sum_{i,j \neq i_1} \mu_j \big(\mu_i \dd\mu_j - \mu_j \dd \mu_i \big) \dd\mu_i
		= 0
\end{align*}
by antisymmetry.

Setting $r_{i_1} = R = r$ one obtains the metric
\begin{equation}
\begin{aligned}
	\dd s^2 &= (1 - \tilde f^{\{i_1\}})\, \Big(\dd u - a_{i_1} \mu_{i_1}^2\, \dd \phi_{i_1} \Big)^2
		- \dd u\, (\dd u + 2 \dd r)
		+ 2 a_{i_1} \mu_{i_1}^2\, \dd r \dd \phi_{i_1} \\
		&\qquad+ \big( r^2 + a_{i_1}^2 \big) (\dd\mu_{i_1}^2 + \mu_{i_1}^2 \dd\phi_{i_1}^2)
		+ \sum_{i \neq i_1} r^2 \big( \dd \mu_i^2 + \mu_i^2\, \dd \phi_i^2 \big).
\end{aligned}
\end{equation}
It corresponds to Myers--Perry metric in $d$ dimensions with one non-vanishing angular momentum.
We recover the same structure as in \eqref{higher-jna:higher-jna:higher-jna:metric:static-seed-ur} with some extra terms that are specific to the $i_1$-th $2$-sphere.


\subsubsection{Iteration and final result}


We should now split again $r$ in functions $(r_{i_2}, R)$.
Very similarly to the first time we have
\begin{equation}
\begin{aligned}
	\dd s^2 &= (1 - \tilde f^{\{i_1\}})\, \Big(\dd u - a_{i_1} \mu_{i_1}^2\, \dd \phi_{i_1} \Big)^2
		- \dd u\, (\dd u + 2 \dd r_{i_2})
		+ 2 a_{i_1} \mu_{i_1}^2\, \dd R \dd \phi_{i_1} \\
		&\qquad+ \big( r_{i_2}^2 + a_{i_1}^2 \big) \dd\mu_{i_1}^2
		+ \big( R^2 + a_{i_1}^2 \big) \mu_{i_1}^2 \dd\phi_{i_1}^2
		+ r_{i_2}^2 ( \dd\mu_{i_2}^2 + \mu_{i_2}^2 \dd\phi_{i_2}^2 ) \\
		&\qquad+ \sum_{i \neq i_1, i_2} \Big(r_{i_2}^2 \dd \mu_i^2 + R^2 \mu_i^2\, \dd \phi_i^2 \Big).
\end{aligned}
\end{equation}
We can now complexify as
\begin{equation}
	r_{i_2} = r'_{i_2} - i a_{i_2} \sqrt{1 - \mu_{i_2}^2}, \qquad
	u = u' + i\, a_{i_1} \sqrt{1 - \mu_{i_2}^2}.
\end{equation} 
The steps are exactly the same as before, except that we have some inert terms.
The complexified functions is now $\tilde f^{\{i_1, i_2\}}$.

Repeating the procedure $n$ times we arrive at
\begin{equation}
	\label{higher-jna:metric:rotating:result-jna-ur}
	\begin{aligned}
		\dd s^2 = &- \dd u^2 - 2 \dd u \dd r
			+ \sum_i (r^2 + a_i^2) (\dd \mu_i^2 + \mu_i^2 \dd \phi_i^2)
			- 2 \sum_i a_i \mu_i^2 \, \dd r \dd \phi_i \\
			&+ \Big(1 - \tilde f^{\{i_1, \ldots, i_n\}} \Big) \left(\dd u + \sum_i a_i \mu_i^2 \dd \phi_i \right)^2.
	\end{aligned}
\end{equation} 
One recognizes the general form of the $d$-dimensional metric with $n$ angular momenta~\cite{Myers:1986:BlackHolesHigher}.

Let's quote the metric in Boyer--Lindquist coordinates (omitting the indices on $\tilde f$)~\cite{Myers:1986:BlackHolesHigher}
\begin{equation}
	\label{higher-jna:metric:rotating:result-jna-bl}
	\dd s^2 = - \dd t^2
		+ (1 - \tilde f) \left(\dd t - \sum_i a_i \mu_i^2 \dd \phi_i \right)^2
		+ \frac{r^2 \rho^2}{\Delta}\, \dd r^2
		+ \sum_i (r^2 + a_i^2) \Big(\dd \mu_i^2 + \mu_i^2\, \dd \phi_i^2 \Big)
\end{equation} 
which is obtained from the transformation
\begin{equation}
	\dd u = \dd t - g\, \dd r, \qquad
	\dd \phi_i = \dd \phi'_i - h_i\, \dd r
\end{equation} 
with functions
\begin{equation}
\label{higher-jna:change:rotating:higher-dim-func-gh}
	g = \frac{\Pi}{\Delta}
		= \frac{1}{1 - F (1 - \tilde f)}, \qquad
	h_i = \frac{\Pi}{\Delta} \, \frac{a_i}{r^2 + a_i^2},
\end{equation}
and where the various quantities involved are (see \cref{app:coord:general-d:oblate-cosines})
\begin{equation}
	\label{higher-jna:metric:rotating:result-jna-bl-parameters}
	\begin{gathered}
		\Pi = \prod_i (r^2 + a_i^2), \qquad
		F = 1 - \sum_i \frac{a_i^2 \mu_i^2}{r^2 + a_i^2} = r^2 \sum_i \frac{\mu_i^2}{r^2 + a_i^2}, \\
		r^2 \rho^2 = \Pi F, \qquad
		\Delta = \tilde f\, r^2 \rho^2 + \Pi (1 - F).
	\end{gathered}
\end{equation}

Before ending this section, we comment the case of even dimensions: the term $\varepsilon'\, r^2 \dd \alpha^2$ is complexified as $\varepsilon'\, r_{i_1}^2 \dd \alpha^2$, since it contributes to the sum
\begin{equation}
	\sum_i \mu_i^2 + \alpha^2 = 1.
\end{equation} 
This can be seen more clearly by defining $\mu_{n+1} = \alpha$ (we can also define $\phi_{n+1} = 0$), in which case the index $i$ runs from $1$ to $n+\varepsilon$, and all the previous computations are still valid.


\subsection{Examples in various dimensions}
\label{sec:higher-jna:examples}


\subsubsection{Flat space}


A first and trivial example is to take $f = 1$.
In this case one recovers Minkowski metric in spheroidal coordinates with direction cosines (\cref{app:coord:general-d:oblate-cosines})
\begin{equation}
	\dd s^2 = - \dd t^2 + F\, \dd \bar r^2 + \sum_i (\bar r^2 + a_i^2) \Big(\dd \bar \mu_i^2 + \bar \mu_i^2\, \dd \bar \phi_i^2 \Big) + \varepsilon'\, r^2 \dd \alpha^2.
\end{equation}
In this case the JN algorithm is equivalent to a (true) change of coordinates and there is no intrinsic rotation.
The presence of a non-trivial function $f$ then deforms the algorithm.


\subsubsection{Myers--Perry black hole with one angular momentum}


The derivation of the Myers--Perry metric with one non-vanishing angular momentum has been found by Xu~\cite{Xu:1988:ExactSolutionsEinstein}.

The transformation is taken to be in the first plane
\begin{equation}
	r = r' - i a \sqrt{1 - \mu^2}
\end{equation} 
where $\mu \equiv \mu_1$.
The transformation to the mixed spherical--spheroidal system (\cref{app:coord:general-d:oblate-spherical} is obtained by setting
\begin{equation}
	\mu = \sin \theta, \qquad
	\phi_1 = \phi.
\end{equation} 
In these coordinates the transformation reads
\begin{equation}
	r = r' - i a \cos \theta.
\end{equation} 
We will use the quantity
\begin{equation}
	\rho^2 = r^2 + a^2 (1 - \mu^2)
		= r^2 + a^2 \cos^2 \theta.
\end{equation} 

The Schwarzschild--Tangherlini metric is~\cite{Tangherlini:1963:SchwarzschildFieldDimensions}
\begin{equation}
	\dd s^2 = - f\, \dd t^2 + f^{-1}\, \dd r^2 + r^2\, \dd \Omega_{d-2}^2, \qquad
	f = 1 - \frac{m}{r^{d-3}}.
\end{equation} 

Applying the previous transformation results in
\begin{equation}
\begin{aligned}
	\dd s^2 &= (1 - \tilde f)\, \Big(\dd u - a \mu^2\, \dd \phi \Big)^2
		- \dd u\, (\dd u + 2 \dd r)
		+ 2 a \mu^2\, \dd r \dd \phi \\
		&\qquad+ \big( r^2 + a^2 \big) (\dd\mu^2 + \mu^2 \dd\phi^2)
		+ \sum_{i \neq 1} r^2 \big( \dd \mu_i^2 + \mu_i^2\, \dd \phi_i^2 \big).
\end{aligned}
\end{equation}
where $f$ has been complexified as
\begin{equation}
	\tilde f = 1 - \frac{m}{\rho^2 r^{d-5}}.
\end{equation} 

In the mixed coordinate system one has~\cite{Xu:1988:ExactSolutionsEinstein, Aliev:2006:RotatingBlackHoles}
\begin{equation}
	\begin{aligned}
		\dd s^2 = &- \tilde f\, \dd t^2
			+ 2 a (1 - \tilde f) \sin^2 \theta\, \dd t \dd\phi
			+ \frac{r^{d-3} \rho^2}{\Delta}\, \dd r^2 + \rho^2 \dd\theta^2 \\
			&+ \frac{\Sigma^2}{\rho^2}\, \sin^2 \theta\, \dd\phi^2
			+ r^2 \cos^2 \theta^2\, \dd\Omega_{d-4}^2.
	\end{aligned}
\end{equation} 
where we defined as usual
\begin{equation}
	\Delta = \tilde f \rho^2 + a^2 \sin^2 \theta, \qquad
	\frac{\Sigma^2}{\rho^2} = r^2 + a^2 + a g_{t\phi}.
\end{equation} 
This last expression explains why the transformation is straightforward with one angular momentum: the transformation is exactly the one for $d = 4$ and the extraneous dimensions are just spectators.

We have not been able to generalize this result for several non-vanishing momenta for $d \ge 6$, even for the case with equal momenta .


\subsubsection{Five-dimensional Myers--Perry}
\label{sec:higher-jna:examples:myers-perry-5d}


We take a new look at the five-dimensional Myers--Perry solution in order to derive it in spheroidal coordinates because it is instructive.

The function
\begin{equation}
	1 - f = \frac{m}{r^2}
\end{equation} 
is first complexified as
\begin{equation}
	1 - \tilde f^{\{1\}} = \frac{m}{\abs{r_1}^2}
		= \frac{m}{r^2 + a^2 (1 - \mu^2)}
\end{equation}
and then as 
\begin{equation}
	1 - \tilde f^{\{1, 2\}} = \frac{m}{\abs{r_2}^2 + a^2 (1 - \mu^2)}
		= \frac{m}{r^2 + a^2 (1 - \mu^2) + b^2 (1 - \nu^2)}.
\end{equation}
after the two transformations
\begin{equation}
	r_1 = r_1' - i a \sqrt{1 - \mu^2}, \qquad
	r_2 = r_2' - i b \sqrt{1 - \nu^2}.
\end{equation} 
For $\mu = \sin \theta$ and $\nu = \cos \theta$ one recovers the transformations from \cref{sec:higher-jna:5d:myers-perry,sec:higher-jna:5d:bmpv}.

Let's denote the denominator by $\rho^2$ and compute
\begin{align*}
	\frac{\rho^2}{r^2} &= r^2 + a^2 (1 - \mu^2) + b^2 (1 - \nu^2)
		= (\mu^2 + \nu^2) r^2 + \nu^2 a^2 + \mu^2 b^2 \\
		&= \mu^2 (r^2 + b^2) + \nu^2 (r^2 + a^2)
		= (r^2 + b^2) (r^2 + a^2) \left( \frac{\mu^2}{r^2 + a^2} + \frac{\nu^2}{r^2 + b^2} \right).
\end{align*}
and thus
\begin{equation}
	r^2 \rho^2 = \Pi F.
\end{equation} 
Plugging this into $\tilde f^{\{1, 2\}}$ we have~\cite{Myers:1986:BlackHolesHigher}
\begin{equation}
	1 - \tilde f^{\{1, 2\}} = \frac{m r^2}{\Pi F}.
\end{equation} 


\subsubsection{Three dimensions: BTZ black hole}
\label{sec:higher-jna:examples:btz}


As another application we show how to derive the $d = 3$ rotating BTZ black hole from its static version~\cite{Banados:1992:BlackHoleThree}
\begin{equation}
	\dd s^2 = - f\, \dd t^2 + f^{-1}\, \dd r^2 + r^2 \dd\phi^2, \qquad
	f(r) = - M + \frac{r^2}{\ell^2}.
\end{equation} 

In three dimensions the metric on $S^1$ in spherical coordinates is given by
\begin{equation}
	\dd\Omega_1^2 = \dd\phi^2.
\end{equation} 
Introducing the coordinate $\mu$ we can write it in oblate spheroidal coordinates
\begin{equation}
	\dd\Omega_1^2 = \dd\mu^2 + \mu^2 \dd\phi^2
\end{equation} 
with the constraint
\begin{equation}
	\mu^2 = 1.
\end{equation} 

Application of the transformation
\begin{equation}
	u = u' + i a \sqrt{1 - \mu^2}, \qquad
	r = r' - i a \sqrt{1 - \mu^2}
\end{equation} 
gives from \eqref{higher-jna:metric:rotating:result-jna-ur}
\begin{equation}
	\begin{aligned}
		\dd s^2 = &- \dd u^2 - 2 \dd u \dd r
			+ (r^2 + a^2) (\dd \mu^2 + \mu^2 \dd \phi^2)
			- 2 a \mu^2 \, \dd r \dd \phi \\
			&+ (1 - \tilde f) (\dd u + a \mu^2 \dd \phi )^2.
	\end{aligned}
\end{equation} 
The transformation of $f$ is
\begin{equation}
	\tilde f = - m + \frac{\rho^2}{\ell^2}, \qquad
	\rho^2 = r^2 + a^2 (1 - \mu^2).
\end{equation} 

The transformation \eqref{higher-jna:change:rotating:higher-dim-func-gh}
\begin{equation}
	g = \frac{\rho^2 (1 - \tilde f)}{\Delta}, \qquad
	h = \frac{a}{\Delta}, \qquad
	\Delta = r^2 + a^2 + (\tilde f - 1) \rho^2
\end{equation}
to Boyer--Lindquist coordinates leads to the metric \eqref{higher-jna:metric:rotating:result-jna-bl}
\begin{equation}
	\dd s^2 = - \dd t^2
		+ (1 - \tilde f) (\dd t + a \mu^2 \dd \phi )^2
		+ \frac{\rho^2}{\Delta}\, \dd r^2
		+ (r^2 + a^2) (\dd \mu^2 + \mu^2\, \dd \phi^2 ).
\end{equation} 

Finally the constraint $\mu^2 = 1$ can be used to remove the $\mu$.
In this case one finds
\begin{equation}
	\rho^2 = r^2, \qquad
	\Delta = a^2 + \tilde f r^2
\end{equation}
and the metric simplifies to
\begin{equation}
	\dd s^2 = - \dd t^2
		+ (1 - \tilde f) (\dd t + a \dd \phi )^2
		+ \frac{r^2}{a^2 + r^2 \tilde f}\, \dd r^2
		+ (r^2 + a^2) \dd \phi^2.
\end{equation} 

We define the function
\begin{equation}
	N^2 = \tilde f + \frac{a^2}{r^2} = - M + \frac{r^2}{\ell^2} + \frac{a^2}{r^2}.
\end{equation} 
Then redefining the time variable as~\cite{Kim:1997:NotesSpinningAdS3, Kim:1999:SpinningBTZBlack}
\begin{equation}
	t = t' - a \phi
\end{equation} 
we get (omitting the prime)
\begin{equation}
	\dd s^2 = - N^2 \dd t^2 + N^{-2}\, \dd r^2 + r^2 (N^\phi \dd t + \dd \phi)^2
\end{equation} 
with the angular shift
\begin{equation}
	N^\phi(r) = \frac{a}{r^2}.
\end{equation} 
This is the solution given in~\cite{Banados:1992:BlackHoleThree} with $J = -2a$.

It has already been showed by Kim that the rotating BTZ black hole can be derived through the JN algorithm in a different settings~\cite{Kim:1997:NotesSpinningAdS3, Kim:1999:SpinningBTZBlack}: he views the $d = 3$ solution as the slice $\theta = \pi/2$ of the $d = 4$ solution.
Obviously this is equivalent to our approach: we have seen that $\mu = \sin \theta$ in $d = 4$ (\cref{app:coord:4d}), and the constraint $\mu^2 = 1$ is solved by $\theta = \pi/2$.
Nonetheless our approach is more direct since the result just follows from a suitable choice of coordinates and there are no need for advanced justification.

Starting from the charged BTZ black hole
\begin{equation}
       f(r) = - M + \frac{r^2}{\ell^2} - Q^2 \ln r^2, \qquad
       A = - \frac{Q}{2}\, \ln r^2,
\end{equation} 
it is not possible to find the charged rotating BTZ black hole from~\cites{Clement:1993:ClassicalSolutionsThreedimensional, Clement:1996:SpinningChargedBTZ}[sec.~4.2]{Martinez:2000:ChargedRotatingBlack}: the solution solves Einstein equations, but not the Maxwell ones.
This has been already remarked using another technique in~\cite[app.~B]{Lambert:2014:ConformalSymmetriesGravity}.
It may be possible that a more general ansatz is necessary, following \cref{sec:general} but in $d = 3$.



\section*{Acknowledgments}


I am particularly grateful and indebted to Lucien Heurtier for our collaboration and our many discussions on this project.
I thank also Nick Halmagyi and Dietmar Klemm for interesting discussions, and I am grateful to the latter and Marco Rabbiosi for allowing me to reproduce an unpublished example of application.
Finally I wish to thank the members of the Harish--Chandra Research Institute (Allahabad, India) for organizing the set of lectures that helped me to transform my thesis in the current review.


\appendix


\section{Coordinate systems}
\label{app:coord}

This appendix is partly based on~\cite{Tangherlini:1963:SchwarzschildFieldDimensions, Myers:1986:BlackHolesHigher, Gibbons:2005:GeneralKerrdeSitter}.
We present formulas for any dimension before summarizing them for $4$ and $5$ dimensions.


\subsection{\texorpdfstring{$d$}{d}-dimensional}
\label{app:coord:general-d}


Let's consider $d = N + 1$ dimensional Minkowski space whose metric is denoted by
\begin{equation}
	\dd s^2 = \eta_{\mu\nu}\; \dd x^\mu \dd x^\nu, \qquad
	\mu = 0, \ldots, N.
\end{equation} 
In all the following coordinates systems the time direction can separated from the spatial (positive definite) metric as
\begin{equation}
	\dd s^2 = - \dd t^2 + \dd \Sigma^2, \qquad
	\dd \Sigma^2 = \gamma_{ab}\; \dd x^a \dd x^b, \qquad
	a = 1, \ldots, N,
\end{equation} 
where $x^0 = t$.

One defines by $n$ the number of independent $2$-planes of rotation
\begin{equation}
	n = \floor{\frac{N}{2}}
\end{equation} 
such that
\begin{equation}
	\label{coord:eq:d-dim-epsilon}
	d + \varepsilon = 2n + 2, \qquad
	N + \varepsilon = 2n + 1, \qquad
	\varepsilon' = 1 - \varepsilon
\end{equation} 
where
\begin{equation}
	\varepsilon = \frac{1}{2} (1 - (-1)^d ) =
	\begin{cases}
		0 & \text{$d$ even (or $N$ odd)} \\
		1 & \text{$d$ odd (or $N$ even)},
	\end{cases}
\end{equation} 
and conversely for $\varepsilon'$.


\subsubsection{Cartesian system}


The usual Cartesian metric is
\begin{equation}
	\dd \Sigma^2 = \delta_{ab} \dd x^a \dd x^b
		= \dd x^a \dd x^a
		= \dd \vec x^2.
\end{equation} 


\subsubsection{Spherical}


Introducing a radial coordinate $r$, the flat space metric can be written as a $(N-1)$-sphere of radius $r$
\begin{equation}
	\label{coord:metric:flat-d:spherical}
	\dd \Sigma^2 = \dd r^2 + r^2 \dd \Omega_{N-1}^2.
\end{equation} 
The term $\dd \Omega_{N-1}^2$ corresponds the metric on the unit $(N-1)$-sphere $S^{N-1}$, which is parame\-trized by $(N-1)$ angles $\theta_i$ and is defined recursively as
\begin{equation}
	\dd \Omega_{N-1}^2 = \dd \theta_{N-1}^2 + \sin^2 \theta_{N-1} \; \dd \Omega_{N-2}^2.
\end{equation} 

This surface can be embedded in $N$-dimensional flat space with coordinates $X^a$ constrained by
\begin{equation}
	\label{coord:eq:spherical-embedding}
	X^a X^a = 1.
\end{equation} 


\subsubsection{Spherical with direction cosines}


In $d$-dimensions there are $n$ orthogonal $2$-planes,\footnotemark{} thus we can pair $2n$ of the embedding coordinates $X^a$ \eqref{coord:eq:spherical-embedding} as $(X_i, Y_i)$ which are parametrized as%
\footnotetext{%
	Note that this is linked to the fact that the little group of massive representation in $D$ dimension is $\group{SO}(N)$, which possess $n$ Casimir invariants~\cite{Myers:1986:BlackHolesHigher}.
}
\begin{equation}
	X_i + i Y_i = \mu_i \e^{i\phi_i}, \qquad
	i = 1, \ldots n.
\end{equation} 
For $d$ even there is an extra unpaired coordinate that is taken to be
\begin{equation}
	X^N = \alpha.
\end{equation}

Each pair parametrizes a $2$-sphere of radius $\mu_i$.
The $\mu_i$ are called the \emph{direction cosines} and satisfy
\begin{equation}
	\sum_i \mu_i^2 + \varepsilon' \alpha^2 = 1
\end{equation} 
since there is one superfluous coordinate from the embedding.
Finally the metric is
\begin{equation}
	\dd \Omega_{N-1}^2 = \sum_i \Big(\dd \mu_i^2 + \mu_i^2\; \dd \phi_i^2 \Big) + \varepsilon'\, \dd \alpha^2.
\end{equation} 

The interest of these coordinates is that all rotational directions are symmetric.


\subsubsection{Spheroidal with direction cosines}
\label{app:coord:general-d:oblate-cosines}

From the previous system we can define the spheroidal $(\bar r, \bar\mu_i, \bar\phi_i)$ system – adapted when some of the $2$-spheres are deformed to ellipses – by introducing parameters $a_i$ such that (for $d$ odd)
\begin{equation}
	\label{coord:eq:spherical-to-oblate-mu}
	r^2 \mu_i^2 = (\bar r^2 + a_i^2) \bar \mu_i^2, \qquad
	\sum_i \bar \mu_i^2 = 1.
\end{equation} 
This last condition implies that
\begin{equation}
	r^2 = \sum_i (\bar r^2 + a_i^2) \bar \mu_i^2
		= \bar r^2 + \sum_i a_i^2 \bar \mu_i^2.
\end{equation} 

In these coordinates the metric reads
\begin{equation}
	\label{coord:metric:flat-d:spheroidal}
	\dd \Sigma^2 = F\; \dd \bar r^2 + \sum_i (\bar r^2 + a_i^2) \Big(\dd \bar \mu_i^2 + \bar \mu_i^2\; \dd \bar \phi_i^2 \Big) + \varepsilon'\, r^2 \dd \alpha^2
\end{equation} 
and we defined
\begin{equation}
	\label{coord:eq:flat-d:spheroidal:F}
	F = 1 - \sum_i \frac{a_i^2 \bar \mu_i^2}{\bar r^2 + a_i^2} = \sum_i \frac{\bar r^2 \bar \mu_i^2}{\bar r^2 + a_i^2}.
\end{equation} 

Here the $a_i$ are just introduced as parameters in the transformation, but in the main text they are interpreted as "true" rotation parameters, i.e.
angular momenta (per unit of mass) of a black hole.
They all appear on the same footing.

Another quantity of interest is
\begin{equation}
	\label{coord:eq:flat-d:spheroidal:Pi}
	\Pi = \prod_i (\bar r^2 + a_i^2).
\end{equation} 


\subsubsection{Mixed spherical–spheroidal}
\label{app:coord:general-d:oblate-spherical}

We consider the deformation of the spherical metric where one of the $2$-sphere is replaced by an ellipse~\cite[sec.~3]{Aliev:2006:RotatingBlackHoles}.

To shorten the notation let's define
\begin{equation}
	\theta = \theta_{N-1}, \qquad
	\phi = \theta_{N-2}.
\end{equation} 
Doing the change of coordinates
\begin{equation}
	\sin^2 \theta \sin^2 \phi = \cos^2 \theta.
\end{equation}
the metric becomes
\begin{equation}
	\dd \Sigma^2 = \frac{\rho^2}{r^2 + a^2}\, \dd r^2
		+ \rho^2 \dd\theta^2 \\
		+ (r^2 + a^2)\, \sin^2 \theta\, \dd\phi^2
		+ r^2 \cos^2 \theta^2\, \dd\Omega_{d-4}^2
\end{equation} 
where as usual
\begin{equation}
	\rho^2 = r^2 + a^2 \cos^2 \theta.
\end{equation} 
Except for the last term one recognizes $4$-dimensional oblate spheroidal coordinates \eqref{coord:metric:4d:spheroidal}.


\subsection{4-dimensional}
\label{app:coord:4d}


In this section one considers
\begin{equation}
	d = 4, \quad
	N = 3, \quad
	n = 1.
\end{equation} 


\subsubsection{Cartesian system}

\begin{equation}
	\dd \Sigma^2 = \dd x^2 + \dd y^2 + \dd z^2.
\end{equation} 


\subsubsection{Spherical}

\begin{subequations}
\begin{gather}
	\dd \Sigma^2 = \dd r^2 + r^2 \dd \Omega^2, \\
	\dd \Omega^2 = \dd \theta^2 + \sin^2 \theta\; \dd \phi^2,
\end{gather}
\end{subequations}
where $\dd \Omega^2 \equiv \dd \Omega_2^2$.


\subsubsection{Spherical with direction cosines}

\begin{subequations}
\begin{gather}
	\dd \Omega^2 = \dd \mu^2 + \mu^2\; \dd \phi^2 + \dd \alpha^2, \\
	\mu^2 + \alpha^2 = 1,
\end{gather}
\end{subequations}
where
\begin{equation}
	x + iy = r \mu\, \e^{i\phi}, \qquad
	z = r \alpha,
\end{equation} 

Using the constraint one can rewrite
\begin{equation}
	\dd \Omega^2 = \frac{1}{1 - \mu^2}\; \dd \mu^2 + \mu^2\; \dd \phi^2.
\end{equation} 
Finally the change of coordinates
\begin{equation}
	\alpha = \cos \theta, \qquad
	\mu = \sin \theta.
\end{equation} 
solves the constraint and gives back the spherical coordinates.


\subsubsection{Spheroidal with direction cosines}

The oblate spheroidal coordinates from the Cartesian ones are~\cite[p.~15]{Visser:2009:KerrSpacetimeBrief}
\begin{equation}
	x + i y = \sqrt{r^2 + a^2}\, \sin \theta\, \e^{i\phi}, \qquad
	z = r \cos\theta,
\end{equation} 
and the metric is
\begin{equation}
	\label{coord:metric:4d:spheroidal}
	\dd \Sigma^2 = \frac{\rho^2}{r^2 + a^2}\; \dd r^2 + \rho^2 \dd\theta^2 + (r^2 + a^2) \sin^2 \theta\; \dd \phi^2, \qquad
	\rho^2 = r^2 + a^2 \cos^2 \theta.
\end{equation} 

In terms of direction cosines one has
\begin{equation}
	\dd \Sigma^2 = \left(1 - \frac{r^2 \mu^2}{r^2 + a^2} \right)\; \dd r^2 + (r^2 + a^2) \Big(\dd \mu^2 + \mu^2\; \dd \phi^2 \Big) + r^2 \dd \alpha^2.
\end{equation} 


\subsection{5-dimensional}
\label{app:coord:5d}


In this section one considers
\begin{equation}
	d = 4, \quad
	N = 3, \quad
	n = 1.
\end{equation} 


\subsubsection{Spherical with direction cosines}

\begin{equation}
	\label{coord:metric:5d:spherical}
	\dd\Omega_3^2 = \dd \mu^2 + \mu^2\, \dd\phi^2 + \dd \nu^2 + \nu^2\, \dd\psi^2, \qquad
	\mu^2 + \nu^2 = 1
\end{equation} 
where for simplicity
\begin{equation}
	\mu = \mu_1, \qquad
	\mu = \mu_2, \qquad
	\phi = \phi_1, \qquad
	\psi = \phi_2.
\end{equation} 


\subsubsection{Hopf coordinates}
\label{app:coord:5d:hopf}

The constraint \eqref{coord:metric:5d:spherical} can be solved by
\begin{equation}
	\mu = \sin \theta, \qquad
	\nu = \cos \theta
\end{equation} 
and this gives the metric in Hopf coordinates
\begin{equation}
	\label{coord:metric:5d:hopf}
	\dd \Omega_3^2 = \dd\theta^2 + \sin^2 \theta\, \dd\phi^2 + \cos^2 \theta\, \dd\psi^2.
\end{equation} 

\section{Review of \texorpdfstring{$N=2$}{N = 2} ungauged supergravity}
\label{app:N=2-sugra}


In order for this review to be self-contained we recall the basic elements of $N = 2$ supergravity without hypermultiplets -- we refer the reader to the standard references for more details~\cite{Freedman:2012:Supergravity, Andrianopoli:1996:GeneralMatterCoupled, Andrianopoli:1997:N2SupergravityN2}.

The gravity multiplet contains the metric and the graviphoton
\begin{equation}
	\{ g_{\mu\nu}, A^0 \}
\end{equation} 
while each of the vector multiplets contains a gauge field and a complex scalar field
\begin{equation}
	\{ A^i, \tau^i \}, \qquad i = 1, \ldots, n_v.
\end{equation} 
The scalar fields $\tau^i$ (the conjugate fields $\conj{(\tau^i)}$ are denoted by $\bar\tau^{\bar\imath}$) parametrize a special Kähler manifold with metric $g_{i\bar\jmath}$.
This manifold is uniquely determined by an holomorphic function called the prepotential $F$.
The latter is better defined using the homogeneous (or projective) coordinates $X^\Lambda$ such that
\begin{equation}
	\tau^i = \frac{X^i}{X^0}.
\end{equation} 
The first derivative of the prepotential with respect to $X^\Lambda$ is denoted by
\begin{equation}
	F_\Lambda = \frac{\pd F}{\pd X^\Lambda}.
\end{equation} 
Finally it makes sense to regroup the gauge fields into one single vector
\begin{equation}
	A^\Lambda = (A^0, A^i).
\end{equation} 

One needs to introduce two more quantities, respectively the Kähler potential and the Kähler connection
\begin{equation}
	K = - \ln i (\bar X^\Lambda F_\Lambda - X^\Lambda \bar F^\Lambda), \qquad
	\mc A_\mu = - \frac{i}{2} (\pd_i K\, \pd_\mu \tau^i - \pd_{\bar\imath} K\, \pd_\mu \bar\tau^{\bar\imath}).
\end{equation} 

The Lagrangian for the theory without gauge group is given by
\begin{equation}
	\mc L = - \frac{R}{2}
		+ g_{i\bar\jmath}(\tau, \bar \tau)\, \pd_\mu \tau^i \pd^\nu \bar\tau^{\bar\imath}
		+ \mc I_{\Lambda\Sigma}(\tau, \bar \tau)\, F^\Lambda_{\mu\nu} F^{\Sigma\,\mu\nu}
		- \mc R_{\Lambda\Sigma}(\tau, \bar \tau)\, F^\Lambda_{\mu\nu} \hodge{F}^{\Sigma\,\mu\nu}
\end{equation} 
where $R$ is the Ricci scalar and $\hodge{F}^\Lambda$ is the Hodge dual of $F^\Lambda$.
The matrix
\begin{equation}
	\mc N = \mc R + i\, \mc I
\end{equation} 
can be expressed in terms of $F$.
From this Lagrangian one can introduce the symplectic dual of $F^\Lambda$
\begin{equation}
	G_\Lambda = \frac{\var \mc L}{\var F^\Lambda} = \mc R_{\Lambda\Sigma} F^\Sigma - \mc I_{\Lambda\Sigma} \hodge{F}^\Sigma.
\end{equation} 

\section{Technical properties}
\label{app:technical-properties}


In this chapter we describe few technical properties of the algorithm.
We comment on the group properties that some of the JN transformations possess~\cite{Erbin:2016:DecipheringGeneralizingDemianskiJanisNewman}.
Another useful property of Giampieri's prescription is to allow to chain all coordinate transformation, making computations easier~\cite{Erbin:2015:JanisNewmanAlgorithmSimplifications}.
Then finally we discuss the fact that not all the rules \eqref{gen:eq:rules} are independent and several choices of complexification are equivalent~\cite{Erbin:2015:JanisNewmanAlgorithmSimplifications}, contrary to what is commonly believed.


\subsection{Group properties}
\label{app:group-properties}


We want to study the JN transformations that form a group: the main motivation is to state clearly the effect of chaining several transformations.
This observation can be useful for chaining several transformations, therefore adding charges to a solution that is already non-static (for example adding rotation to a solution that already contains a NUT charge).
More importantly this provides a setting where the algorithm has good chances to preserve Einstein equations.

We will make the assumptions that the functions $F(\theta)$ and $G(\theta)$ are linear in some parameters $\pi^A$ (implicit sum over $A$)
\begin{equation}
	F(\theta) = \pi^A F_A(\theta), \qquad
	G(\theta) = \pi^A G_A(\theta),
\end{equation} 
where $\{ F_A(\theta) \}$ and $\{ G_A(\theta) \}$ are the functions associated to the parameters and $A$ runs over the dimension of this space.
Mathematically the functions are member of an additive group $\mc G$ with elements in\footnotemark{} $\mc F \times \mc F$ ($\mc F$ being the space of functions with second derivatives) with generators $\big( F_A(\theta), G_A(\theta) \big)$, $A = 1, \ldots, \dim \mc V$ since there is an identity element $0$ and each element with coefficients $\pi^A$ possesses an inverse given by $- \pi^A$.%
\footnotetext{%
	For simplicity we consider the case where $F$ and $G$ are expanded over the same parameters, but this is not necessarily the case.
}
Adding the multiplication by a scalar turns this group into a vector space but we do not need this extra structure.
As a consequence the sum of two functions $F_1 = \pi_1^A F_A$ and $F_2 = \pi_2^A F_A$ gives another function $F_3 = \pi_3^A F_A$ with $\pi_3^A = \pi_1^A + \pi_2^A$.
These assumptions are motivated by the results of \cref{sec:derivation} where $F$ and $G$ were solutions of (non-homogeneous) second order linear differential equations where the $\pi^A$ are the integration constants.

After a first transformation
\begin{equation}
	r = r' + i\, F_1, \qquad
	u = u' + i\, G_1
\end{equation} 
one obtains the metric (omitting the primes)
\begin{equation}
	\begin{aligned}
		\dd s^2 = &- \tilde f^{\{1\}}_t (\dd u + H G_1'\, \dd\phi)^2
			+ \tilde f^{\{1\}}_\Omega (\dd\theta^2 + H^2 \dd\phi^2) \\
			&- 2 \sqrt{\tilde f^{\{1\}}_t \tilde f^{\{1\}}_r} (\dd u + G_1' H\, \dd\phi) (\dd r + F_1' H\, \dd\phi)
	\end{aligned}
\end{equation} 
where
\begin{equation}
	\tilde f^{\{1\}}_i = \tilde f^{\{1\}}_i(r, F_1).
\end{equation} 
Performing a second transformation
\begin{equation}
	r = r' + i\, F_2, \qquad
	u = u' + i\, G_2
\end{equation} 
the previous metric becomes (omitting the primes)
\begin{equation}
	\label{topdown:eq:metric-two-transf}
	\begin{aligned}
		\dd s^2 = &- \tilde f^{\{1,2\}}_t \big( \dd u + H (G_1' + G_2')\, \dd\phi \big)^2
			+ \tilde f^{\{1,2\}}_\Omega (\dd\theta^2 + H^2 \dd\phi^2) \\
			&- 2 \sqrt{\tilde f^{\{1,2\}}_t \tilde f^{\{1,2\}}_r} \big( \dd u + (G_1' + G_2') H\, \dd\phi \big) \big( \dd r + (F_1' + F_2') H\, \dd\phi \big)
	\end{aligned}
\end{equation} 
where
\begin{equation}
	\tilde f^{\{1,2\}}_i = \tilde f^{\{1,2\}}_i(r, F_1, F_2).
\end{equation}
This function is required to satisfy the following conditions (omitting the primes)
\begin{equation}
	\tilde f^{\{1,2\}}_i(r, F_1, 0) = \tilde f^{\{1\}}_i(r, F_1), \qquad
	\tilde f^{\{1,2\}}_i(r, F_1, F_2) = \tilde f^{\{2,1\}}_i(r, F_2, F_1).
\end{equation} 
The second condition means that the order of the transformations should not matter because we want to obtain the same solution given identical seed metric and parameters.

The metric \eqref{topdown:eq:metric-two-transf} is obviously equivalent to the one we would get with a unique transformation\footnotemark{}%
\footnotetext{%
	This breaks down when the metric is transformed with more complicated rules, such as in higher dimensions~\cite{Erbin:2015:FivedimensionalJanisNewmanAlgorithm}.
}
\begin{equation}
	r = r' + i\, (F_1 + F_2), \qquad
	u = u' + i\, (G_1 + G_2).
\end{equation} 
Then, for the transformations which are such that
\begin{equation}
	\label{topdown:eq:fi-sum-F}
	\tilde f^{\{1,2\}}_i(r, F_1, F_2) = \tilde f^{\{1\}}_i(r, F_1 + F_2),
\end{equation} 
the DJN transformations form an Abelian group thanks to the group properties of the function space.
This structure implies that we can first add one parameter, and later another one (say first the NUT charge, and then an angular momentum).
Said another way this group \emph{preserves Einstein equations} when the seed metric is a known (stationary) solution.
But note that it may be very difficult to do it as soon as one begins to replace the $F$ in the functions by their expression, because it obscures the original function – in one word we can not find $\tilde f_i(r, F)$ from $\tilde f_i(r, \theta)$.

Another point worth to mention is that not all DJN transformation are in this group since the condition \eqref{topdown:eq:fi-sum-F} may not satisfied: we recall that imposing or not the latter is a choice that one is doing when performing the algorithm.
A simple example is provided by
\begin{equation}
	f(r) = r^2,
\end{equation} 
which can be transformed under the two successive transformations
\begin{equation}
	r = r' + i F_1, \qquad
	r' = r'' + i F_2
\end{equation} 
in two ways:
\begin{subequations}
\begin{align}
	1.& \qquad
		\tilde f^{\{1\}} = \abs{r}^2
			= r'^2 + F_1^2, \qquad
		\tilde f^{\{1,2\}} = \abs{r'}^2 + F_1^2
			= r''^2 + F_1^2 + F_2^2, \\
	2.& \qquad
		\tilde f^{\{1\}} = \abs{r}^2
			= \abs{r' + i F_1}^2, \qquad
		\begin{aligned}
			\tilde f^{\{1,2\}} &= \abs{r'' + i (F_1 + F_2)}^2 \\
				&= r''^2 + F_1^2 + F_2^2 + 2 F_1 F_2.
		\end{aligned}
\end{align}
\end{subequations}
Only the second option satisfy the property \eqref{topdown:eq:fi-sum-F} that leads to a group.
Such an example is provided in $5d$ where the function $f_\Omega(r) = r^2$ is successively transformed as~\cite{Erbin:2015:FivedimensionalJanisNewmanAlgorithm}
\begin{equation}
	r^2 \longrightarrow \abs{r}^2 = r^2 + a^2 \cos^2 \theta \longrightarrow \abs{r}^2 + a^2 \cos^2 \theta = r^2 + a^2 \cos^2 \theta + b^2 \sin^2 \theta,
\end{equation} 
with the functions
\begin{equation}
	F_1 = a \cos \theta, \qquad
	F_2 = b \sin \theta.
\end{equation} 
The condition \eqref{topdown:eq:fi-sum-F} is clearly not satisfied.


\subsection{Chaining transformations}
\label{sec:jna-prop:chaining}



The JN algorithm is summarized by the following table
\begin{equation}
	\begin{array}{cccccccccc}
 		t             & \to & u & \to & u \in \C    & \to & u' & \to & t'    \\
		r             &     &   & \to & r \in \C    & \to & r' &     &      \\
		\phi          &     &   &     &             &     &    & \to & \phi' \\
		f             &     &   & \to & \tilde f    &     &    &     &
	\end{array}
\end{equation}
where the arrows correspond to the different steps of the algorithm.

A major advantage of Giampieri's prescription is that one can chain all these transformations since it involves only substitutions and no tensor operations.
For this reason it is much easier to implement on a computer algebra system such as Mathematica.
It is then possible to perform a unique change of variables that leads directly from the static metric to the rotating metric in any system defined by the function $(g, h)$.
For example in the case of rotation for a metric with a single function one finds
\begin{subequations}
\begin{align}
	\dd t &= \dd t' + \big(a h \sin^2 \theta\, (1 - \tilde f^{-1}) - g + \tilde f^{-1} \big)\, \dd r'
		+ a \sin^2 \theta\, (\tilde f^{-1} - 1)\, \dd \phi', \\
	\dd r &= (1 - a h \sin^2 \theta)\, \dd r' + a \sin^2 \theta\, \dd \phi', \\
	\dd \phi &= \dd \phi' - h\, \dd r',
\end{align}
\end{subequations}
where the complexification of the metric function $f$ can be made at the end.
It is impressive that the algorithm from section~\ref{sec:algo} can be written in such a compact way.


\subsection{Arbitrariness of the transformation}
\label{sec:jna-prop:arbitrariness}


We provide a short comment on the arbitrariness of the complexification rules \eqref{gen:eq:rules}.
In particular let's consider the functions
\begin{equation}
	f_1(r) = \frac{1}{r}, \qquad
	f_2(r) = \frac{1}{r^2}.
\end{equation} 

The usual rule is to complexify these two functions as
\begin{equation}
	\label{prop:eq:arb-usual-rules}
	\tilde f_1(r) = \frac{\Re r}{\abs{r}^2}, \qquad
	\tilde f_2(r) = \frac{1}{\abs{r}^2}
\end{equation} 
using respectively the rules \eqref{gen:eq:rules:1/r} and \eqref{gen:eq:rules:r2} (in the denominator).

But it is possible to arrive at the same result with a different combinations of rules.
In fact the functions can be rewritten as
\begin{equation}
	f_1(r) = \frac{r}{r^2}, \qquad
	f_2(r) = \frac{1}{r}\, \frac{1}{r}.
\end{equation} 
The following set of rules results again in \eqref{prop:eq:arb-usual-rules}:
\begin{itemize}
	\item $f_1$: \eqref{gen:eq:rules:r} (numerator) and \eqref{gen:eq:rules:r2} (denominator);
	\item $f_2$: \eqref{gen:eq:rules:r} (first fraction) and \eqref{gen:eq:rules:1/r} (second fraction).
\end{itemize}




\printbibliography[heading=bibintoc]


\end{document}

