\section{Coordinate systems}
\label{app:coord}

This appendix is partly based on~\cite{Tangherlini:1963:SchwarzschildFieldDimensions, Myers:1986:BlackHolesHigher, Gibbons:2005:GeneralKerrdeSitter}.
We present formulas for any dimension before summarizing them for $4$ and $5$ dimensions.


\subsection{\texorpdfstring{$d$}{d}-dimensional}
\label{app:coord:general-d}


Let's consider $d = N + 1$ dimensional Minkowski space whose metric is denoted by
\begin{equation}
	\dd s^2 = \eta_{\mu\nu}\; \dd x^\mu \dd x^\nu, \qquad
	\mu = 0, \ldots, N.
\end{equation} 
In all the following coordinates systems the time direction can separated from the spatial (positive definite) metric as
\begin{equation}
	\dd s^2 = - \dd t^2 + \dd \Sigma^2, \qquad
	\dd \Sigma^2 = \gamma_{ab}\; \dd x^a \dd x^b, \qquad
	a = 1, \ldots, N,
\end{equation} 
where $x^0 = t$.

One defines by $n$ the number of independent $2$-planes of rotation
\begin{equation}
	n = \floor{\frac{N}{2}}
\end{equation} 
such that
\begin{equation}
	\label{coord:eq:d-dim-epsilon}
	d + \varepsilon = 2n + 2, \qquad
	N + \varepsilon = 2n + 1, \qquad
	\varepsilon' = 1 - \varepsilon
\end{equation} 
where
\begin{equation}
	\varepsilon = \frac{1}{2} (1 - (-1)^d ) =
	\begin{cases}
		0 & \text{$d$ even (or $N$ odd)} \\
		1 & \text{$d$ odd (or $N$ even)},
	\end{cases}
\end{equation} 
and conversely for $\varepsilon'$.


\subsubsection{Cartesian system}


The usual Cartesian metric is
\begin{equation}
	\dd \Sigma^2 = \delta_{ab} \dd x^a \dd x^b
		= \dd x^a \dd x^a
		= \dd \vec x^2.
\end{equation} 


\subsubsection{Spherical}


Introducing a radial coordinate $r$, the flat space metric can be written as a $(N-1)$-sphere of radius $r$
\begin{equation}
	\label{coord:metric:flat-d:spherical}
	\dd \Sigma^2 = \dd r^2 + r^2 \dd \Omega_{N-1}^2.
\end{equation} 
The term $\dd \Omega_{N-1}^2$ corresponds the metric on the unit $(N-1)$-sphere $S^{N-1}$, which is parame\-trized by $(N-1)$ angles $\theta_i$ and is defined recursively as
\begin{equation}
	\dd \Omega_{N-1}^2 = \dd \theta_{N-1}^2 + \sin^2 \theta_{N-1} \; \dd \Omega_{N-2}^2.
\end{equation} 

This surface can be embedded in $N$-dimensional flat space with coordinates $X^a$ constrained by
\begin{equation}
	\label{coord:eq:spherical-embedding}
	X^a X^a = 1.
\end{equation} 


\subsubsection{Spherical with direction cosines}


In $d$-dimensions there are $n$ orthogonal $2$-planes,\footnotemark{} thus we can pair $2n$ of the embedding coordinates $X^a$ \eqref{coord:eq:spherical-embedding} as $(X_i, Y_i)$ which are parametrized as%
\footnotetext{%
	Note that this is linked to the fact that the little group of massive representation in $D$ dimension is $\group{SO}(N)$, which possess $n$ Casimir invariants~\cite{Myers:1986:BlackHolesHigher}.
}
\begin{equation}
	X_i + i Y_i = \mu_i \e^{i\phi_i}, \qquad
	i = 1, \ldots n.
\end{equation} 
For $d$ even there is an extra unpaired coordinate that is taken to be
\begin{equation}
	X^N = \alpha.
\end{equation}

Each pair parametrizes a $2$-sphere of radius $\mu_i$.
The $\mu_i$ are called the \emph{direction cosines} and satisfy
\begin{equation}
	\sum_i \mu_i^2 + \varepsilon' \alpha^2 = 1
\end{equation} 
since there is one superfluous coordinate from the embedding.
Finally the metric is
\begin{equation}
	\dd \Omega_{N-1}^2 = \sum_i \Big(\dd \mu_i^2 + \mu_i^2\; \dd \phi_i^2 \Big) + \varepsilon'\, \dd \alpha^2.
\end{equation} 

The interest of these coordinates is that all rotational directions are symmetric.


\subsubsection{Spheroidal with direction cosines}
\label{app:coord:general-d:oblate-cosines}

From the previous system we can define the spheroidal $(\bar r, \bar\mu_i, \bar\phi_i)$ system – adapted when some of the $2$-spheres are deformed to ellipses – by introducing parameters $a_i$ such that (for $d$ odd)
\begin{equation}
	\label{coord:eq:spherical-to-oblate-mu}
	r^2 \mu_i^2 = (\bar r^2 + a_i^2) \bar \mu_i^2, \qquad
	\sum_i \bar \mu_i^2 = 1.
\end{equation} 
This last condition implies that
\begin{equation}
	r^2 = \sum_i (\bar r^2 + a_i^2) \bar \mu_i^2
		= \bar r^2 + \sum_i a_i^2 \bar \mu_i^2.
\end{equation} 

In these coordinates the metric reads
\begin{equation}
	\label{coord:metric:flat-d:spheroidal}
	\dd \Sigma^2 = F\; \dd \bar r^2 + \sum_i (\bar r^2 + a_i^2) \Big(\dd \bar \mu_i^2 + \bar \mu_i^2\; \dd \bar \phi_i^2 \Big) + \varepsilon'\, r^2 \dd \alpha^2
\end{equation} 
and we defined
\begin{equation}
	\label{coord:eq:flat-d:spheroidal:F}
	F = 1 - \sum_i \frac{a_i^2 \bar \mu_i^2}{\bar r^2 + a_i^2} = \sum_i \frac{\bar r^2 \bar \mu_i^2}{\bar r^2 + a_i^2}.
\end{equation} 

Here the $a_i$ are just introduced as parameters in the transformation, but in the main text they are interpreted as "true" rotation parameters, i.e.
angular momenta (per unit of mass) of a black hole.
They all appear on the same footing.

Another quantity of interest is
\begin{equation}
	\label{coord:eq:flat-d:spheroidal:Pi}
	\Pi = \prod_i (\bar r^2 + a_i^2).
\end{equation} 


\subsubsection{Mixed spherical–spheroidal}
\label{app:coord:general-d:oblate-spherical}

We consider the deformation of the spherical metric where one of the $2$-sphere is replaced by an ellipse~\cite[sec.~3]{Aliev:2006:RotatingBlackHoles}.

To shorten the notation let's define
\begin{equation}
	\theta = \theta_{N-1}, \qquad
	\phi = \theta_{N-2}.
\end{equation} 
Doing the change of coordinates
\begin{equation}
	\sin^2 \theta \sin^2 \phi = \cos^2 \theta.
\end{equation}
the metric becomes
\begin{equation}
	\dd \Sigma^2 = \frac{\rho^2}{r^2 + a^2}\, \dd r^2
		+ \rho^2 \dd\theta^2 \\
		+ (r^2 + a^2)\, \sin^2 \theta\, \dd\phi^2
		+ r^2 \cos^2 \theta^2\, \dd\Omega_{d-4}^2
\end{equation} 
where as usual
\begin{equation}
	\rho^2 = r^2 + a^2 \cos^2 \theta.
\end{equation} 
Except for the last term one recognizes $4$-dimensional oblate spheroidal coordinates \eqref{coord:metric:4d:spheroidal}.


\subsection{4-dimensional}
\label{app:coord:4d}


In this section one considers
\begin{equation}
	d = 4, \quad
	N = 3, \quad
	n = 1.
\end{equation} 


\subsubsection{Cartesian system}

\begin{equation}
	\dd \Sigma^2 = \dd x^2 + \dd y^2 + \dd z^2.
\end{equation} 


\subsubsection{Spherical}

\begin{subequations}
\begin{gather}
	\dd \Sigma^2 = \dd r^2 + r^2 \dd \Omega^2, \\
	\dd \Omega^2 = \dd \theta^2 + \sin^2 \theta\; \dd \phi^2,
\end{gather}
\end{subequations}
where $\dd \Omega^2 \equiv \dd \Omega_2^2$.


\subsubsection{Spherical with direction cosines}

\begin{subequations}
\begin{gather}
	\dd \Omega^2 = \dd \mu^2 + \mu^2\; \dd \phi^2 + \dd \alpha^2, \\
	\mu^2 + \alpha^2 = 1,
\end{gather}
\end{subequations}
where
\begin{equation}
	x + iy = r \mu\, \e^{i\phi}, \qquad
	z = r \alpha,
\end{equation} 

Using the constraint one can rewrite
\begin{equation}
	\dd \Omega^2 = \frac{1}{1 - \mu^2}\; \dd \mu^2 + \mu^2\; \dd \phi^2.
\end{equation} 
Finally the change of coordinates
\begin{equation}
	\alpha = \cos \theta, \qquad
	\mu = \sin \theta.
\end{equation} 
solves the constraint and gives back the spherical coordinates.


\subsubsection{Spheroidal with direction cosines}

The oblate spheroidal coordinates from the Cartesian ones are~\cite[p.~15]{Visser:2009:KerrSpacetimeBrief}
\begin{equation}
	x + i y = \sqrt{r^2 + a^2}\, \sin \theta\, \e^{i\phi}, \qquad
	z = r \cos\theta,
\end{equation} 
and the metric is
\begin{equation}
	\label{coord:metric:4d:spheroidal}
	\dd \Sigma^2 = \frac{\rho^2}{r^2 + a^2}\; \dd r^2 + \rho^2 \dd\theta^2 + (r^2 + a^2) \sin^2 \theta\; \dd \phi^2, \qquad
	\rho^2 = r^2 + a^2 \cos^2 \theta.
\end{equation} 

In terms of direction cosines one has
\begin{equation}
	\dd \Sigma^2 = \left(1 - \frac{r^2 \mu^2}{r^2 + a^2} \right)\; \dd r^2 + (r^2 + a^2) \Big(\dd \mu^2 + \mu^2\; \dd \phi^2 \Big) + r^2 \dd \alpha^2.
\end{equation} 


\subsection{5-dimensional}
\label{app:coord:5d}


In this section one considers
\begin{equation}
	d = 4, \quad
	N = 3, \quad
	n = 1.
\end{equation} 


\subsubsection{Spherical with direction cosines}

\begin{equation}
	\label{coord:metric:5d:spherical}
	\dd\Omega_3^2 = \dd \mu^2 + \mu^2\, \dd\phi^2 + \dd \nu^2 + \nu^2\, \dd\psi^2, \qquad
	\mu^2 + \nu^2 = 1
\end{equation} 
where for simplicity
\begin{equation}
	\mu = \mu_1, \qquad
	\mu = \mu_2, \qquad
	\phi = \phi_1, \qquad
	\psi = \phi_2.
\end{equation} 


\subsubsection{Hopf coordinates}
\label{app:coord:5d:hopf}

The constraint \eqref{coord:metric:5d:spherical} can be solved by
\begin{equation}
	\mu = \sin \theta, \qquad
	\nu = \cos \theta
\end{equation} 
and this gives the metric in Hopf coordinates
\begin{equation}
	\label{coord:metric:5d:hopf}
	\dd \Omega_3^2 = \dd\theta^2 + \sin^2 \theta\, \dd\phi^2 + \cos^2 \theta\, \dd\psi^2.
\end{equation} 
