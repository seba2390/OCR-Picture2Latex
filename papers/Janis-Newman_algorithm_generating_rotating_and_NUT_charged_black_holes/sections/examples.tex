\section{Examples}
\label{sec:examples}


In this section we list several examples that can be derived from the JN algorithm described in \cref{sec:general}.
Other examples were described previously: Kerr--Newman in \cref{sec:algorithm:kerr-newman}, dyonic Kerr--Newman and Yang--Mills Kerr--Newman in \cref{sec:extension:dyonic}.
For simplicity we will always consider the case $\kappa = 1$ except when $\Lambda \neq 0$.

The first two examples are the Kerr--Newmann--NUT solution (already derived by another path in \cref{sec:derivation:stationary:solution-no-cosmo}) and the charged (a)dS--BBMB--NUT solution in conformal gravity.
We will also give examples from ungauged $N = 2$ supergravity coupled to $n_v = 0, 1, 3$ vector multiplets (pure supergravity, T$^3$ model and STU model): this theory is reviewed in \cref{app:N=2-sugra}.



\subsection{Kerr--Newman--NUT}


The Reissner--Nordström metric and gauge fields are given by
\begin{subequations}
\label{matter:eq:reissner-nordstrom}
\begin{gather}
	\label{matter:metric:reissner-nordstrom:tr}
	\dd s^2 = - f\, \dd t^2 + f^{-1}\, \dd r^2 + r^2 \dd \Omega^2, \qquad
	f = 1 - \frac{2m}{r} + \frac{q^2}{r^2}, \\
	A = f_A\, \dd t, \qquad
	f_A = \frac{q}{r},
\end{gather} 
\end{subequations}
$m$ and $q$ being the mass and the electric charge.

The two functions are complexified as
\begin{equation}
	\tilde f = 1 - \frac{2 \Re(m \bar r) + q^2}{\abs{r}^2}, \qquad
	\tilde f_A = \frac{q \Re r}{\abs{r}^2}.
\end{equation} 
Performing the transformation
\begin{equation}
	u = u' + \big( a \cos\theta - 2 n \ln \sin\theta \big), \qquad
	r = r' + i \big( n - a \cos\theta \big), \qquad
	m = m' + i n
\end{equation} 
gives (omitting the primes)
\begin{equation}
	\tilde f = 1 - \frac{2 m r + 2 n ( n - a \cos\theta) - q^2}{\rho^2}, \qquad
	\rho^2 = r^2 + (n - a \cos\theta)^2.
\end{equation}
The metric and the gauge fields in BL coordinates are
\begin{subequations}
\begin{gather}
	\dd s^2 = - \tilde f\, (\dd t + \Omega\, \dd\phi )^2
		+ \frac{\rho^2}{\Delta}\, \dd r^2
		+ \rho^2 (\dd\theta^2 + \sigma^2 H^2 \dd\phi^2), \\
	A = \frac{q r}{\rho^2}\, \Big( \dd t - (a \sin^2 \theta + 2 n \cos \theta) \dd \phi \Big) + A_r\, \dd r
\end{gather}
\end{subequations}
where
\begin{equation}
	\begin{gathered}
		\Omega = - 2 n \cos \theta - (1 - \tilde f^{-1})\, a \sin^2 \theta, \qquad
		\sigma^2 = \frac{\Delta}{\tilde f \rho^2}, \\
		\Delta = r^2 - 2 m r + a^2 + q^2 - n^2.
	\end{gathered}
\end{equation} 
This corresponds to the Kerr--Newman--NUT solution~\cite{AlonsoAlberca:2000:SupersymmetryTopologicalKerrNewmannTaubNUTaDS}.

One can check that $A_r$ is a function of $r$ only
\begin{equation}
	A_r = - \frac{q r}{\Delta}
\end{equation} 
and it can be removed by a gauge transformation.


\subsection{Charged (a)dS--BBMB--NUT}
\label{sec:examples:bbmb}


The action of Einstein--Maxwell theory with cosmological constant conformally coupled to a scalar field is~\cite{Bardoux:2013:IntegrabilityConformallyCoupled}
\begin{equation}
	S = \frac{1}{2} \int \dd^4 x\, \sqrt{-g} \left( R - 2 \Lambda - \frac{1}{6}\, R \phi^2 - (\pd \phi)^2 - 2 \alpha \phi^4 - F^2 \right),
\end{equation} 
where $\alpha$ is a coupling constant, and we have set $8\pi G = 1$.

For $F, \alpha, \Lambda = 0$, the Bocharova--Bronnikov--Melnikov--Bekenstein (BBMB) solution~\cite{Bekenstein:1974:ExactSolutionsEinsteinconformal, Bocharova:1970:ExactSolutionSystem} is static and spherically symmetric -- it can be seen as the equivalent of the Schwarzschild black hole in conformal gravity.

The general static charged solution with cosmological constant and quartic coupling reads
\begin{subequations}
\begin{gather}
	\dd s^2 = - f\, \dd t^2 + f^{-1}\, \dd r^2 + r^2\, \dd\Omega^2, \\
	A = \frac{q}{r}\, \dd t, \qquad
	\phi = \sqrt{- \frac{\Lambda}{6 \alpha}}\; \frac{m}{r - m}, \\
	f = - \frac{\Lambda}{3}\, r^2 + \kappa\, \frac{(r - m)^2}{r^2},
\end{gather}
\end{subequations}
where the horizon can be spherical or hyperbolic.
There is one constraint among the parameters
\begin{equation}
	\label{matter:eq:bbmb:constraint}
	q^2 = \kappa m^2 \left( 1 + \frac{\Lambda}{36 \alpha} \right)
\end{equation} 
and one has $\alpha \Lambda < 0$ in order for $\phi$ to be real.

In order to add a NUT charge one performs the JN transformation\footnotemark{}%
\footnotetext{%
	Due to the convention of~\cite{Bardoux:2013:IntegrabilityConformallyCoupled} there is no $\kappa$ in the transformations.
}
\begin{equation}
	u = u' - 2 n \ln H(\theta), \qquad
	r = r' + i n, \qquad
	m = m' + i n, \qquad
	\kappa = \kappa' - \frac{4\Lambda}{3}\, n^2.
\end{equation} 
One obtains the metric (omitting the primes)
\begin{equation}
	\dd s^2 = - \tilde f \big(\dd t - 2 n H'\, \dd\phi \big)^2
		+ \tilde f^{-1}\, \dd r^2
		+ (r^2 + n^2)\, \dd\Omega^2
\end{equation}
where the function $\tilde f$ is
\begin{equation}
	\tilde f = - \frac{\Lambda}{3}\, (r^2 + n^2) + \left( \kappa - \frac{4\Lambda}{3}\, n^2 \right)\, \frac{(r - m)^2}{r^2 + n^2}.
\end{equation} 
Note that the term $(r - m)$ is invariant.
Similarly one obtains the scalar field
\begin{equation}
	\phi = \sqrt{- \frac{\Lambda}{6 \alpha}}\; \frac{\sqrt{m^2 + n^2}}{r - m}
\end{equation} 
where the $m$ in the numerator as been complexified as $\abs{m}$.
Finally it is trivial to find the gauge field
\begin{equation}
	A = \frac{q}{r^2 + n^2}\, \big( \dd t - 2 n \cos \theta\, \dd \phi \big)
\end{equation} 
and the constraint \eqref{matter:eq:bbmb:constraint} becomes
\begin{equation}
	q^2 = \left( \kappa - \frac{4\Lambda}{3}\, n^2 \right) (m^2 + n^2) \left( 1 + \frac{\Lambda}{36 \alpha} \right).
\end{equation} 

An interesting point is that the radial coordinate is redefined in~\cite{Bardoux:2013:IntegrabilityConformallyCoupled} when obtaining the stationary solution from the static one.

Note that the BBMB solution and its NUT version are obtained from the limit
\begin{equation}
	\Lambda, \alpha \longrightarrow 0, \qquad
	\text{with} \qquad
	- \frac{\Lambda}{36 \alpha} \longrightarrow 1,
\end{equation} 
which also implies $q = 0$ from the constraint \eqref{matter:eq:bbmb:constraint}.
Since no other modifications are needed, the derivation from the JN algorithm also holds in this case.


\subsection{Ungauged \texorpdfstring{$N = 2$}{N = 2} BPS solutions}
\label{sec:examples:N=2-bps}


A BPS solution is a classical solution which preserves a part of the supersymmetry.
The BPS equations are obtained by setting to zero the variations of the fermionic partners under a supersymmetric transformation.
These equations are first order and under some conditions their solutions also solve the equations of motion.

In~\cite[sec.~3.1]{Behrndt:1998:StationarySolutionsN2} (see also~\cite[sec.~2.2]{Hristov:2010:BPSBlackHoles} for a summary), Behrndt, Lüst and Sabra obtained the most general stationary BPS solution for $N = 2$ ungauged supergravity.
The metric for this class of solutions reads
\begin{equation}
	\label{matter:metric:sugra:static-N=2}
	\dd s^2 = f^{-1} (\dd t + \Omega\, \dd\phi)^2 + f\, \dd \Sigma^2,
\end{equation} 
with the $3$-dimensional spatial metric given in spherical or spheroidal coordinates
\begin{subequations}
\label{matter:metric:flat-spatial}
\begin{align}
	\dd \Sigma^2 &= h_{ij}\, \dd x^i \dd x^j
		= \dd r^2 + r^2 (\dd\theta^2 + \sin^2 \theta\, \dd\phi^2) \\
		&= \frac{\rho^2}{r^2 + a^2}\; \dd r^2 + \rho^2 \dd\theta^2 + (r^2 + a^2) \sin^2 \theta\; \dd \phi^2,
\end{align}
\end{subequations}
where $i, j, k$ are flat spatial indices (which should not be confused with the indices of the scalar fields).
The functions $f$ and $\Omega$ depend on $r$ and $\theta$ only.

Then the solution is entirely given in terms of two sets of (real) harmonic functions\footnotemark{} $\{ H^\Lambda, H_\Lambda \}$%
\footnotetext{%
	We omit the tilde that is present in~\cite{Behrndt:1998:StationarySolutionsN2} to avoid the confusion with the quantities that are transformed by the JNA.
	No confusion is possible since the index position will always indicate which function we are using.
}
\begin{subequations}
\label{matter:eq:N=2-bps-equations}
\begin{gather}
	f = \e^{-K} = i (\bar X^\Lambda F_\Lambda - X^\Lambda \bar F_\Lambda), \\
	\levi{_{ijk}} \pd_j \Omega_k = 2 \e^{-K} \mc A_i = (H_\Lambda \pd_i H^\Lambda - H^\Lambda \pd_i H_\Lambda), \\
	F^\Lambda_{ij} = \frac{1}{2}\, \levi{_{ijk}} \pd_k H^\Lambda, \qquad
	G_{\Lambda\,ij} = \frac{1}{2}\, \levi{_{ijk}} \pd_k H_\Lambda, \\
	i (X^\Lambda - \bar X^\Lambda) = H^\Lambda, \qquad
	i (F_\Lambda - \bar F_\Lambda) = H_\Lambda.
\end{gather}
\end{subequations}
The object $\Omega_i$ is the connection of the line bundle corresponding to the fibration of time over the spatial manifold (its curl is related to the Kähler connection).
Its only non-vanishing component is $\Omega_\phi \equiv \Omega = \omega H$.


Starting from the metric \eqref{matter:metric:sugra:static-N=2} in spherical coordinates with $\Omega = 0$, one can use the JN algorithm of \cref{sec:general} with
\begin{equation}
	f_t = f^{-1}, \qquad
	f_r = f, \qquad
	f_\Omega = r^2 f,
\end{equation} 
leading to the formula \eqref{gen:eq:rotating:tr-degenerate-isotropic}.
The function $\Omega$ reads
\begin{equation}
	\Omega = \omega H
		= a (1 - \tilde f) \sin^2 \theta + 2 n \cos\theta.
\end{equation} 

Then one needs only to find the complexification of $f$ and to check that it gives the correct $\omega$, as would be found from the equations \eqref{matter:eq:N=2-bps-equations}.
However it appears that one cannot complexify directly $f$ since it should be viewed as a composite object made of complex functions.
Therefore one needs to complexify first the harmonic functions $H_\Lambda$ and $H^\Lambda$ (or equivalently $X^\Lambda$), and then to reconstruct the other quantities.
Nonetheless, equations \eqref{matter:eq:N=2-bps-equations} ensure that finding the correct harmonic functions gives a solution, thus it is not necessary to check these equations for all the other quantities.

In the next subsections we provide two examples,\footnotemark{} one for pure supergravity as an appetizer, and then one with $n_v = 3$ multiplets (STU model).%
\footnotetext{%
	They correspond to singular solutions, but we are not concerned with regularity here.
}


\subsubsection{Pure supergravity}
\label{sec:examples:N=2-bps:pure}


As a first example we consider pure (or minimal) supergravity, i.e. $n_v = 0$~\cite[sec.~4.2]{Behrndt:1998:StationarySolutionsN2}.
The prepotential reads
\begin{equation}
	F = - \frac{i}{4}\, (X^0)^2.
\end{equation} 
The function $H_0$ and $H^0$ are related to the real and imaginary parts of the scalar $X^0$
\begin{equation}
	H_0 = \frac{1}{2} (X^0 + \bar X^0) = \Re X^0, \qquad
	\bar H^0 = i (X^0 - \bar X^0) = - 2 \Im X^0,
\end{equation} 
while the Kähler potential is given by
\begin{equation}
	f = \e^{-K} = X^0 \bar X^0.
\end{equation} 
The static solution corresponds to
\begin{equation}
	\label{matter:eq:pure-sugra-static-X0}
	H_0 = X^0 = 1 + \frac{m}{r}
\end{equation} 

Performing the JN transformation for the angular momentum gives
\begin{equation}
	\tilde X^0 = 1 + \frac{m (r + i a \cos\theta)}{\rho^2}.
\end{equation}
This corresponds to the second solution of which is stationary with
\begin{equation}
	\omega = \frac{m (2r + m)}{\rho^2}\, a \sin^2 \theta.
\end{equation} 

Alternatively one can use the JN algorithm to add a NUT charge.
In this case using the rule 
\begin{equation}
	r \longrightarrow \frac{1}{2}\, (r + \bar r) = \Re r = r'
\end{equation} 
must be use for transforming $f$ and $r^2$ (in front of $\dd\Omega$), leading to
\begin{equation}
	\label{matter:eq:pure-sugra-static-X0-nut}
	X^0 = 1 + \frac{m + i n}{r}.
\end{equation} 
Note that it gives
\begin{equation}
	\tilde f = \left(1 + \frac{m}{r}\right)^2 + \frac{n^2}{r^2}.
\end{equation} 
It is slightly puzzling that the above rule should be used instead of the two others in \eqref{gen:eq:rules}.
One possible explanation is the following: in the seed solution shift the radial coordinate such that $r = R - m$ and apply the JN transformation in this coordinate system.
It is clear that every function of $r$ is left unchanged while the tensor structure transforms identically since $\dd r = \dd R$.
After the transformation one can undo the coordinate transformation.
As we mentioned earlier the algorithm is very sensible to the coordinate system and to the parametrization (but it is still not clear why the $R$-coordinate is the natural one).
This kind of difficulty will reappear in the SWIP solution (\cref{sec:examples:swip}).


\subsubsection{STU model}


We now consider the STU model $n_v = 3$ with prepotential~\cite[sec.~3]{Behrndt:1998:StationarySolutionsN2}
\begin{equation}
	F = - \frac{X^1 X^2 X^3}{X^0}.
\end{equation} 
The expressions for the Kähler potential and the scalar fields in terms of the harmonic functions are complicated and will not be needed (see ~\cite[sec.~3]{Behrndt:1998:StationarySolutionsN2} for the expressions).
Various choices for the functions will give different solutions.

A class of static black hole-like solutions are given by the harmonic functions~\cite[sec.~4.4]{Behrndt:1998:StationarySolutionsN2}
\begin{equation}
	\label{matter:eq:stu-static-functions}
	H_0 = h_0 + \frac{q_0}{r}, \qquad
	H^i = h^i + \frac{p^i}{r}, \qquad
	H^0 = H_i = 0.
\end{equation} 
These solutions carry three magnetic $p^i$ and one electric $q_0$ charges.

Let's form the complex harmonic functions
\begin{equation}
	\mc H_0 = H_0 + i\, H^0, \qquad
	\mc H_i = H^i + i\, H_i.
\end{equation} 
Then the rule for complex function leads to
\begin{equation}
	\mc H_0 = h_0 + \frac{q_0 (r + i a \cos\theta)}{\rho^2}, \qquad
	\mc H_i = h^i + \frac{p^i (r + i a \cos\theta)}{\rho^2},
\end{equation} 
for which the various harmonic functions read explicitly
\begin{equation}
	H_0 = h_0 + \frac{q_0 r}{\rho^2}, \qquad
	H^i = h^i + \frac{p^i r}{\rho^2}, \qquad
	H^0 = \frac{q_0 a \cos\theta}{\rho^2}, \qquad
	H_i = \frac{p^i a \cos\theta}{\rho^2}.
\end{equation}
This set of functions corresponds to the stationary solution of~\cite[sec.~4.4]{Behrndt:1998:StationarySolutionsN2} where the magnetic and electric dipole momenta are not independent parameters but obtained from the magnetic and electric charges instead.


\subsection{Non-extremal rotating solution in \texorpdfstring{$T^3$}{T3} model}
\label{sec:examples:rotating-T3}



The $T^3$ model under consideration corresponds to Einstein--Maxwell gravity coupled to an axion $\sigma$ and a dilaton $\phi$ (with specific coupling constants) and the action is given by \eqref{ex:eq:action:swip} with $M = 1$.
This model can be embedded in $N = 2$ ungauged supergravity with $n_v = 1$, equal gauge fields $A \equiv A^0 = A^1$ and prepotential\footnotemark{}%
\footnotetext{%
	This model can be obtained from the STU model by setting the sections pairwise equal $X^2 = X^0$ and $X^3 = X^1$~\cite{Chow:2014:BlackHolesN8}.
	It is also a truncation of pure $N = 4$ supergravity.
}
\begin{equation}
	F = - i\, X^0 X^1,
\end{equation} 
The dilaton and the axion corresponds to the complex scalar field
\begin{equation}
	\tau = \e^{-2\phi} + i\, \sigma.
\end{equation} 
Sen derived the rotating black hole for this theory using the fact that it can be embedded in heterotic string theory~\cite{Sen:1992:RotatingChargedBlack}.

The static metric, gauge field and the complex field read respectively
\begin{subequations}
\begin{align}
	\dd s^2 &= - \frac{f_1}{f_2}\, \dd t^2 + f_2 \Big(f_1^{-1}\, \dd r^2 + r^2\, \dd\Omega^2 \Big), \\
	A &= \frac{f_A}{f_2}\, \dd t, \\
	\tau &= \e^{-2\phi} = f_2
\end{align}
\end{subequations}
where
\begin{equation}
	f_1 = 1 - \frac{r_1}{r}, \qquad
	f_2 = 1 + \frac{r_2}{r}, \qquad
	f_A = \frac{q}{r}.
\end{equation} 
The radii $r_1$ and $r_2$ are related to the mass $m$ and the charge $q$ by
\begin{equation}
	r_1 + r_2 = 2 m, \qquad
	r_2 = \frac{q^2}{m}.
\end{equation} 

Applying the Janis--Newman algorithm with rotation, the two functions $f_1$ and $f_2$ are complexified as
\begin{equation}
	\tilde f_1 = 1 - \frac{r_1 r}{\rho^2}, \qquad
	\tilde f_2 = 1 + \frac{r_2 r}{\rho^2}.
\end{equation} 
The final metric in BL coordinates is given by
\begin{equation}
	\dd s^2 = - \frac{\tilde f_1}{\tilde f_2} \left[ \dd t - a \left(1 - \frac{\tilde f_2}{\tilde f_1}\right) \sin^2 \theta\, \dd\phi \right]^2
		+ \tilde f_2 \left( \frac{\rho^2 \dd r^2}{\Delta} + \rho^2 \dd\theta^2 + \frac{\Delta}{\tilde f_1}\, \sin^2 \theta\, \dd\phi^2 \right)
\end{equation}
for which the BL functions are
\begin{equation}
	g(r) = \frac{\hat \Delta}{\Delta}, \qquad
	h(r) = \frac{a}{\Delta}
\end{equation} 
with
\begin{equation}
	\label{matter:eq:sen-bh-delta}
	\Delta = \tilde f_1 \rho^2 + a^2 \sin^2 \theta, \qquad
	\hat \Delta = \tilde f_2 \rho^2 + a^2 \sin^2 \theta.
\end{equation} 

Once $f_A$ has been complexified as
\begin{equation}
	\tilde f_A = \frac{q r}{\rho^2}
\end{equation} 
the transformation of the gauge field is straightforward
\begin{equation}
	A = \frac{\tilde f_A}{\tilde f_2}\, (\dd t - a \sin^2 \theta\, \dd\phi )
		- \frac{q r}{\Delta}\, \dd r.
\end{equation} 
The $A_r$ depending solely on $r$ can again be removed thanks to a gauge transformation.

Finally the scalar field is complex and is transformed as
\begin{equation}
	\tau = 1 + \frac{r_2 \bar r}{\rho^2}.
\end{equation} 
The explicit values for the dilaton and axion are then
\begin{equation}
	\e^{-2\phi} = \tilde f_2, \qquad
	\sigma = \frac{r_2 a \cos \theta}{\rho^2}.
\end{equation} 

This reproduces Sen's solution and it completes the computation from~\cite{Yazadjiev:2000:NewmanJanisMethodRotating} which could not derive the gauge field nor the axion.
It is interesting to note that for another value of the dilaton coupling we cannot use the transformation~\cite{Horne:1992:RotatingDilatonBlack, Pirogov:2013:RotatingScalarvacuumBlack}.\footnotemark{}%
\footnotetext{%
	The authors of~\cite{Hansen:2013:ApplicabilityNewmanJanisAlgorithm} report incorrectly that~\cite{Pirogov:2013:RotatingScalarvacuumBlack} is excluding all dilatonic solutions.
}


\subsection{SWIP solutions}
\label{sec:examples:swip}


Let's consider the action~\cites{Bergshoeff:1996:StationaryAxionDilatonSolutions}[sec.~12.2]{Ortin:2004:GravityStrings}
\begin{equation}
	\label{ex:eq:action:swip}
	S = \frac{1}{16 \pi} \int \dd^4 x\, \sqrt{\abs{g}}\, \left( R
		- 2 (\pd \phi)^2 - \frac{1}{2}\, \e^{4\phi}\, (\pd \sigma)^2
		- \e^{-2\phi} F^i_{\mu\nu} F^{i\mu\nu} + \sigma\, F^i_{\mu\nu} \tilde{F}^{i\mu\nu} \right)
\end{equation} 
where $i = 1, \ldots, M$.
When $M = 2$ and $M = 6$ this action corresponds respectively to $N = 2$ supergravity with one vector multiplet and to $N = 4$ pure supergravity, but we keep $M$ arbitrary.
The axion $\sigma$ and the dilaton $\phi$ are naturally paired into a complex scalar
\begin{equation}
	\tau = \sigma + i \e^{-2\phi}.
\end{equation} 

In order to avoid redundancy we first provide the general metric with $a, n \neq 0$, and we explain how to find it from the restricted case $a = n = 0$.
The stationary Israel--Wilson--Perjés (SWIP) solutions correspond to
\begin{subequations}
\label{matter:eq:swip:solution}
\begin{gather}
	\dd s^2 = - \e^{2U} W (\dd t + A_\phi\, \dd \phi)^2 + \e^{-2U} W^{-1} \dd \Sigma^2, \\
	A^i_t = 2 \e^{2U} \Re(k^i H_2), \qquad
	\tilde A^i_t = 2 \e^{2U} \Re(k^i H_1), \qquad
	\tau = \frac{H_1}{H_2}, \\
	A_\phi = 2 n \cos \theta - a \sin^2 \theta (\e^{-2U} W^{-1} - 1), \\
	\e^{-2U} = 2 \Im(H_1 \bar H_2), \qquad
	W = 1 - \frac{r_0^2}{\rho^2}.
\end{gather}
\end{subequations}
This solution is entirely determined by the two harmonic functions
\begin{equation}
	\label{matter:eq:swip:harmonic-functions}
	H_1 = \frac{1}{\sqrt{2}}\, \e^{\phi_0} \left( \tau_0 + \frac{\tau_0 \mc M + \bar \tau_0 \Upsilon}{r - i a \cos \theta} \right), \qquad
	H_2 = \frac{1}{\sqrt{2}}\, \e^{\phi_0} \left( 1 + \frac{\mc M + \Upsilon}{r - i a \cos \theta} \right).
\end{equation} 
The spatial $3$-dimensional metric $\dd \Sigma^2$ reads
\begin{equation}
	\label{matter:metric:swip:flat-spatial}
	\dd\Sigma^2 = h_{ij}\, \dd x^i \dd x^j
		= \frac{\rho^2 - r_0^2}{r^2 + a^2 - r_0^2}\; \dd r^2 + (\rho^2 - r_0^2) \dd\theta^2 + (r^2 + a^2 - r_0^2) \sin^2 \theta\; \dd \phi^2.
\end{equation} 

Finally, $r_0$ corresponds to
\begin{equation}
	r_0^2 = \abs{\mc M}^2 + \abs{\Upsilon}^2 - \sum_i \abs{\Gamma^i}^2
\end{equation} 
where the complex parameters are
\begin{equation}
	\mc M = m + i n, \qquad
	\Gamma^i = q^i + i p^i,
\end{equation} 
$m$ being the mass, $n$ the NUT charge, $q^i$ the electric charges and $p^i$ the magnetic charges, while the axion--dilaton charge $\Upsilon$ takes the form
\begin{equation}
	\Upsilon = - \frac{1}{2} \sum_i \frac{(\bar \Gamma^i)^2}{\mc M}.
\end{equation} 
The latter together with the asymptotic values $\tau_0$ are defined by
\begin{equation}
	\tau \sim \tau_0 - i \e^{-2\phi_0} \frac{2 \Upsilon}{r}.
\end{equation} 
The complex constant $k^i$ are determined by
\begin{equation}
	k^i = - \frac{1}{\sqrt{2}}\, \frac{\mc M \Gamma^i + \bar \Upsilon \bar \Gamma^i}{\abs{\mc M}^2 - \abs{\Upsilon}^2}.
\end{equation} 

As discussed in the previous section, the transformation of scalar fields is different depending on one is turning on a NUT charge or an angular momentum.
For this reason, starting from the case $a = n = 0$, one needs to perform the two successive transformations
\begin{subequations}
\begin{gather}
	\label{matter:eq:swip:djn-nut}
	u = u' - 2 i n \ln \sin \theta, \qquad
	r = r' + i n, \qquad
	m = m' + i n, \\
	\label{matter:eq:swip:djn-rot}
	u = u' + i a \cos \theta, \qquad
	r = r' - i a \cos \theta,
\end{gather}
\end{subequations}
the order being irrelevant (for definiteness we choose to add the NUT charge first), the reason being that the transformations of the functions are different in both cases (as in \cref{sec:examples:N=2-bps:pure}).
As explained in \cref{app:group-properties}, group properties of the JN algorithm ensure that the metric will be transformed as if only one transformation was performed.
Then the metric and the gauge fields are directly obtained, which ensures that the general form of the solution \eqref{matter:eq:swip:solution} is correct.
For that one needs to shift $r^2$ by $r_0^2$ in order to bring the metric \eqref{matter:metric:swip:flat-spatial} to the form \eqref{matter:metric:flat-spatial}.
This modifies the function but one does not need this fact to obtain the general form. Then one can shift by $- r_0^2$ before dealing with the complexification of the functions.
See~\cite[p.~17]{Bergshoeff:1996:StationaryAxionDilatonSolutions} and \cref{sec:examples:N=2-bps:pure} for discussions about the changes of coordinates.
Since all the functions and the parameters depend only on $\mc M$, $H_1$ and $H_2$, it is sufficient to explain their complexification.

The function $W$ is transformed as a real function.
On the other hand $H_1$ and $H_2$ are complex harmonic functions and should be transformed accordingly.
For the NUT charge one should use the rule
\begin{equation}
	r \longrightarrow \Re r.
\end{equation} 
Then one can perform the second transformation \eqref{matter:eq:swip:djn-rot} in order to add the angular momentum by applying the usual rules \eqref{gen:eq:rules}.
On can see that it yields the correct result.



Finally let's note that it seems possible to also start from $p^i = 0$ and to turn them on using the transformation
\begin{equation}
	q^i = q'^i = q^i + i p^i,
\end{equation} 
using different rules for complexifying the various terms (depending whether one is dealing with a real or a complex function/parameter).


\subsection{Gauged \texorpdfstring{$N = 2$}{N = 2} non-extremal solution}
\label{sec:examples:gauged-N=2}


The simplest deformation of $N = 2$ supergravity with $n_v$ vector multiplets consists in the so-called Fayet--Iliopoulos (FI) gauging.
It amounts to gauging $(n_v + 1)$ times the diagonal $\group{U}(1)$ group of the $\group{SU}(2)$ part of the R-symmetry group (automorphism of the supersymmetry algebra).
The potential can be entirely written in terms of the quantities defined in \cref{app:N=2-sugra} and of the $(n_v + 1)$ coupling constants $g_I$, where $I = 0, \ldots, n_v$.

We consider the model with prepotential (see also \cref{sec:examples:rotating-T3})
\begin{equation}
	F = - i\, X^0 X^1.
\end{equation} 
for which the potential generated by the FI gauging is
\begin{equation}
	V(\tau, \bar \tau) = - \frac{4}{\tau + \bar \tau} \big(g_0^2 + g_0 g_1 (\tau + \bar\tau) + g_1^2 \abs{\tau}^2 \big).
\end{equation} 
The goal of this section is to derive the NUT charged black hole from~\cite{Gnecchi:2014:RotatingBlackHoles} using the JN algorithm.\footnotemark{}%
\footnotetext{%
	The original derivation is due to D.\ Klemm and M.\ Rabbiosi and has not been published.
	I am grateful to them for allowing me to reproduce it here.
}

The seed solution is taken to be eq.\ (4.22) from~\cite{Gnecchi:2014:RotatingBlackHoles} with $j = N = 0$
\begin{subequations}
\begin{gather}
	f_t = \kappa
		- \frac{2 m r - 2 \ell^2 \sum_I g_I \abs{Z^I}^2}{f_\Omega}
		+ \frac{f_\Omega}{\ell^2}, \\
	f_\Omega = r^2 - \Delta^2 - \delta^2, \\
	f^I = \frac{(r - \Delta) Q^I - \delta\, P^I}{f_\Omega}, \\
	\tau = \frac{g_0}{g_1}\, \frac{r + \Delta - i \delta}{r - \Delta + i \delta}.
\end{gather}
\end{subequations}
where the following quantities have been defined
\begin{subequations}
\begin{gather}
	m = \frac{\ell^2 P^0}{\Delta}\; \frac{g_1^2 \big[- (P^1)^2 P^0 + (Q^1)^2 P^0 - 2 Q^0 Q^1 P^1 \big] + g_0^2 P^0 \abs{Z^0}^2}{\abs{Z^0}^2}, \\
	\delta = - \Delta\, \frac{Q^0}{P^0}.
\end{gather}
\end{subequations}
The independent parameters are given by $Q^I$ (electric charges), $P^I$ (magnetic charges), $g_\Lambda$ (FI gaugings), $\Delta$ (scalar charge) and $\Lambda = - 3 / \ell^2$ (the cosmological constant).

In order to perform the complexification the functions are first rewritten as
\begin{subequations}
\begin{gather}
	f_t = \kappa
		- \frac{2 \Re(m \bar r) - 2 \ell^2 \sum_I g_I \abs{Z^I}^2}{f_\Omega}
		+ \frac{f_\Omega}{\ell^2}, \\
	f_\Omega = \abs{r}^2 - \Delta^2 - \delta^2
		= \abs{r}^2 - \frac{\Delta^2 \abs{Z^1}^2}{\Im(Z^1)^2}, \\
	f^I
		= \frac{\Re(Q^I \bar r) \Im Z^1 - \Delta \Im(Z^I Z^1)}{\Im Z^1\, f_\Omega}, \\
	%\tau = \frac{g_0}{g_1}\, \frac{P^1\, r + \Delta (P^1 + i Q^1)}{P^1\, r - \Delta (P^1 + i Q^1)}
	%	= \frac{g_0}{g_1}\, \frac{\Im(Z^1 \bar r) + i \Delta \bar Z^1}{\Im(Z^1 \bar r) - i \Delta \bar Z^1}.
	\tau = \frac{g_0}{g_1}\, \frac{\bar r + \Delta - i \delta}{\bar r - \Delta + i \delta}.
\end{gather}
\end{subequations}
Applying the transformations \eqref{gen:eq:change:jna} with \eqref{gen:eq:change:jna-functions-FG} gives (omitting the primes)
\begin{subequations}
\begin{gather}
	\tilde f_t = \kappa + \frac{4 n^2}{\ell^2}
		- \frac{2 m r + 2 \left(\kappa + 4 n^2 / \ell^2 \right) n^2 - 2 \ell^2 \sum_I g_I \abs{Z^I}^2}{\tilde f_\Omega}
		+ \frac{\tilde f_\Omega}{\ell^2}, \\
	\tilde f_\Omega = r^2 + n^2 - \Delta^2 - \delta^2, \\
	\tilde f^I = \frac{(Q^I r + P^I n) \Im Z^1 - \Delta \Im(Z^I Z^1)}{\Im Z^1\, \tilde f_\Omega}, \\
	\tilde \tau = \frac{g_0}{g_1}\, \frac{r + \Delta - i (\delta + n)}{r - \Delta + i (\delta - n)}.
\end{gather}
\end{subequations}
The last step is to simplify these expressions
\begin{subequations}
\begin{gather}
	\tilde f_t = \kappa + \frac{4 n^2}{\ell^2}
		- \frac{2 m r + 2 \kappa n^2 + 8 n^4 / \ell^2 - 2 \ell^2 \sum_I g_I \abs{Z^I}^2}{\tilde f_\Omega}
		+ \frac{\tilde f_\Omega}{\ell^2}, \\
	\tilde f_\Omega = r^2 + n^2 - \Delta^2 - \delta^2, \\
	\tilde f^I = \frac{Q^I (r - \Delta) + P^I (n - \delta)}{\tilde f_\Omega}, \\
	\tilde \tau = \frac{g_0}{g_1}\, \frac{r + \Delta - i (\delta + n)}{r - \Delta + i (\delta - n)}.
\end{gather}
\end{subequations}
It is straightforward to check that the form of the metric and gauge fields are correctly reproduced by the algorithm given in \cref{sec:general} for the tensor structure.
In total this reproduces the eq.\ (4.22) and formulas below in~\cite{Gnecchi:2014:RotatingBlackHoles} with $j = 0$.


An important thing that we learn here is that the mass parameter needs to be transformed as if it was not composed of other parameters.
