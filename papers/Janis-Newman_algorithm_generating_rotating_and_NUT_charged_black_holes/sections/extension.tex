\section{Extension through simple examples}
\label{sec:extension}


In this section we motivate through simple examples modifications to the original prescription for the transformation of the functions.


\subsection{Magnetic charges: dyonic Kerr--Newman}
\label{sec:extension:dyonic}


The dyonic Reissner--Nordström metric is obtained from the electric one \eqref{algo:eq:rn:functions} by the replacement~\cite[sec.~6.6]{Carroll:2004:SpacetimeGeometryIntroduction}
\begin{equation}
	q^2 \longrightarrow \abs{Z}^2 = q^2 + p^2
\end{equation} 
where $Z$ corresponds to the central charge
\begin{equation}
	Z = q + i p.
\end{equation} 
Then the metric function reads
\begin{equation}
	f = 1 - \frac{2m}{r} + \frac{\abs{Z}^2}{r^2}.
\end{equation} 
The gauge field receives a new $\phi$-component
\begin{equation}
	\label{ext:eq:static:vector}
	A = f_A\, \dd t - p \cos \theta\, \dd\phi
		= f_A\, \dd u - p \cos \theta\, \dd\phi
\end{equation}
(the last equality being valid after a gauge transformation) and
\begin{equation}
	f_A = \frac{q}{r}.
\end{equation} 

The transformation of the function $f$ under \eqref{algo:eq:change:complexification-ur} is straightforward and yields
\begin{equation}
	\tilde f = 1 - \frac{2m r' - \abs{Z}^2}{\rho^2}.
\end{equation} 
On the other hand transforming directly the $r$ inside $f_A$ according to \eqref{algo:eq:rules} does not yield the correct result.
Instead one needs to first rewrite the gauge field function as
\begin{equation}
	f_A = \Re\left(\frac{Z}{r}\right)
\end{equation} 
from which the transformation proceeds to
\begin{equation}
	\tilde f_A = \frac{\Re(Z \bar r)}{\abs{r}^2}
		= \frac{q r' - p a \cos \theta}{\rho^2}.
\end{equation} 
Note that it not useful to replace $p$ by $\Im Z$ in \eqref{ext:eq:static:vector} since it is not accompanied by any $r$ dependence.
Moreover it is natural that the factor $\abs{Z}^2$ appears in the metric and this explains why the charges there do not mix with the coordinates.

The gauge field in BL coordinates is finally
\begin{subequations}
\begin{align}
	A &= \frac{q r - p a \cos \theta}{\rho^2}\, \dd t
			+ \left(- \frac{q r}{\rho^2}\, a \sin^2 \theta + \frac{p(r^2 + a^2)}{\rho^2}\, \cos\theta \right) \dd\phi \\
		&= \frac{q r}{\rho^2} (\dd t - a \sin^2 \theta \dd\phi)
			+ \frac{p \cos \theta}{\rho^2} \left(a\, \dd t + (r^2 + a^2)\, \dd\phi \right).
\end{align}
\end{subequations}
The radial component has been removed thanks to a gauge transformation since it depends only on $r$
\begin{equation}
	\Delta \times A_r = - \frac{q r - p a \cos \theta}{\rho^2}\, \rho^2 - p a \cos \theta
		= - q r.
\end{equation} 

There is a coupling between the parameters $a$ and $p$ which can be interpreted from the fact that a rotating magnetic charge has an electric quadrupole moment.
This coupling is taken into account from the product of the imaginary parts which yield a real term.
In view of the form of the algorithm such contribution could not arise from any other place.
Moreover the combination $Z = q + i p$ appears naturally in the Plebański--Demiański solution~\cite{Plebanski:1975:ClassSolutionsEinsteinMaxwell, Plebanski:1976:RotatingChargedUniformly}.

The Yang--Mills Kerr--Newman black hole found by Perry~\cite{Perry:1977:BlackHolesAre} can also be derived in this way, starting from the seed
\begin{equation}
	A^I = \frac{q^I}{r}\, \dd t + p^I \cos \theta\, \dd\phi, \qquad
	\abs{Z}^2 = q^I q^I + p^I p^I
\end{equation} 
where $q^I$ and $p^I$ are constant elements of the Lie algebra.


\subsection[NUT charge, cosmological constant and topological horizon: (anti-)de Sitter Schwarzschild--NUT]
{NUT charge and cosmological constant and topological horizon: (anti-)de Sitter Schwarzschild--NUT}
\label{sec:extension:nut}


In this subsection we consider general topological horizons
\begin{equation}
	\dd \Omega^2 = \dd\theta^2 + H(\theta)^2\, \dd \phi^2, \qquad
	H(\theta) =
	\begin{cases}
		\sin \theta & \kappa = 1 \quad (S^2), \\
		\sinh \theta & \kappa = -1 \quad (H^2).
	\end{cases}
\end{equation} 
The cosmological constant is denoted by $\Lambda$.
We give only the main formulas to motivate the modification of the algorithm, leaving the details of the transformation for \cref{sec:general}.

The complex transformation that adds a NUT charge is
\begin{subequations}
\label{ext:eq:change:jna-nut}
\begin{gather}
	u = u' - 2 \kappa \ln H(\theta), \qquad
	r = r' + i n, \\
	m = m' + i \kappa n, \qquad
	\kappa = \kappa' - \frac{4\Lambda}{3}\, n^2.
\end{gather}
\end{subequations}
Note that it is $\kappa$ and not $\kappa'$ that appears in $m$.
After having shown

The metric derived from the seed \eqref{algo:eq:static:metric:tr} is
\begin{equation}
	\dd s^2 = - \tilde f\, (\dd t - 2 \kappa n H'(\theta)\, \dd\phi)^2
		+ \tilde f^{-1}\, \dd r^2
		+ \rho^2\, \dd\Omega^2,
\end{equation}
see \eqref{gen:eq:rotating:tr-F-cst}, where
\begin{equation}
	\rho^2 = r'^2 + n^2.
\end{equation} 

The function corresponding to the (a)dS--Schwarzschild metric is
\begin{equation}
	f = \kappa - \frac{2m}{r} - \frac{\Lambda}{3}\, r^2
		= \kappa - 2 \Re\left(\frac{m}{r}\right) - \frac{\Lambda}{3}\, r^2.
\end{equation} 
The transformation is
\begin{equation}
	\tilde f = \kappa
			- \frac{2 \Re(m \bar r)}{\abs{r}^2}
			- \frac{\Lambda}{3}\, \abs{r}^2
		= \kappa' - \frac{4\Lambda}{3}\, n^2
			- \frac{2 \left[ m' r' + \left( \kappa' - \frac{4\Lambda}{3}\, n^2 \right) n^2 \right]}{\rho^2}
			- \frac{\Lambda}{3}\, \rho^2
\end{equation} 
which after simplification gives
\begin{equation}
	\label{ext:eq:nut-tilde-f}
	\tilde f = \kappa' - \frac{2 m' r' + 2 \kappa' n^2}{\rho^2}
		- \frac{\Lambda}{3} (r'^2 + 5 n^2)
		+ \frac{8\Lambda}{3}\, \frac{n^4}{\rho^2}
\end{equation} 
which corresponds correctly to the function of (a)dS--Schwarzschild--NUT~\cite{AlonsoAlberca:2000:SupersymmetryTopologicalKerrNewmannTaubNUTaDS}.

Note that it is necessary to consider the general case of massive black hole with topological horizon (if $\Lambda \neq 0$ for the latter) even if one is ultimately interested in the $m = 0$ or $\kappa = 1$ cases.

The transformation \eqref{ext:eq:change:jna-nut} can be interpreted as follows.
In similarity with the case of the magnetic charge, writing the mass as a complex parameter is needed to take into account some couplings between the parameters that would not be found otherwise.
Moreover the shift of $\kappa$ is required because the curvature of the $(\theta, \phi)$ section should be normalized to $\kappa = \pm 1$ but the coupling of the NUT charge with the cosmological constant modifies the curvature: the new shift is necessary to balance this effect and to normalize the $(\theta, \phi)$ curvature to $\kappa' = \pm 1$ in the new metric.
The NUT charge in the Plebański--Demiański solution~\cite{Plebanski:1975:ClassSolutionsEinsteinMaxwell, Plebanski:1976:RotatingChargedUniformly} is
\begin{equation}
	\ell = n \left( 1 - \frac{4\Lambda}{3}\, n^2 \right)
\end{equation} 
so the natural complex combination is $m + i \ell$ and not $m + i \kappa n$ from this point of view, and similarly for the curvature~\cite[sec.~5.3]{Griffiths:2006:NewLookPlebanskiDemianski} (such relations appear when taking limit of the Plebański--Demiański solution to recover subcases).

Finally we conclude this section with two remarks to quote different contexts where the above expression appear naturally :
\begin{itemize}
	\item Embedding Einstein--Maxwell into $N = 2$ supergravity with a negative cosmological constant $\Lambda = - 3 g^2$, the solution is BPS if~\cite{AlonsoAlberca:2000:SupersymmetryTopologicalKerrNewmannTaubNUTaDS}
	\begin{equation}
		\kappa' = -1, \qquad
		n = \pm \frac{1}{2g},
	\end{equation} 
	in which case $\kappa' = \kappa$.
	
	\item The Euclidean NUT solution is obtained from the Wick rotation
	\begin{equation}
		t = - i \tau, \qquad
		n = i \nu.
	\end{equation}
	The condition for regularity is~\cite{Chamblin:1999:LargeNPhases, Johnson:2014:ThermodynamicVolumesAdSTaubNUT}
	\begin{equation}
		m = m' - \nu \left( \kappa + \frac{4\Lambda}{3}\, \nu^2 \right)
			= 0.
	\end{equation} 
\end{itemize}


\subsection{Complex scalar fields}


For a complex scalar field, or any pair of real fields that can be naturally gathered as a complex field, one should treat the full field as a single entity instead of looking at the real and imaginary parts independently.
In particular one should not impose any reality condition.
A typical case of such system is the axion--dilaton pair
\begin{equation}
	\tau = \e^{-2\phi} + i \sigma.
\end{equation} 

In order to demonstrate this principle consider the seed (for a complete example see \cref{sec:examples:rotating-T3})
\begin{equation}
	\tau = 1 + \frac{\mu}{r}
\end{equation} 
where only the dilaton is non-zero.
Then the transformation \eqref{algo:eq:change:complexification-ur} gives
\begin{equation}
	\tau' = 1 + \frac{\mu}{r}
		= 1 + \frac{\mu}{r' - i a \cos\theta}
		= 1 + \frac{\mu r'}{\rho^2} + i\, \frac{\mu a \cos\theta}{\rho^2}.
\end{equation} 
The transformation generates an imaginary part which cannot be obtained if $\Im \tau$ is treated separately: the algorithm does not change fields that vanish except if they are components of a larger field.
Note that both $\tau$ and $\tau'$ are harmonic functions.
