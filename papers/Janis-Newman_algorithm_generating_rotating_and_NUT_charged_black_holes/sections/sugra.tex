\section{Review of \texorpdfstring{$N=2$}{N = 2} ungauged supergravity}
\label{app:N=2-sugra}


In order for this review to be self-contained we recall the basic elements of $N = 2$ supergravity without hypermultiplets -- we refer the reader to the standard references for more details~\cite{Freedman:2012:Supergravity, Andrianopoli:1996:GeneralMatterCoupled, Andrianopoli:1997:N2SupergravityN2}.

The gravity multiplet contains the metric and the graviphoton
\begin{equation}
	\{ g_{\mu\nu}, A^0 \}
\end{equation} 
while each of the vector multiplets contains a gauge field and a complex scalar field
\begin{equation}
	\{ A^i, \tau^i \}, \qquad i = 1, \ldots, n_v.
\end{equation} 
The scalar fields $\tau^i$ (the conjugate fields $\conj{(\tau^i)}$ are denoted by $\bar\tau^{\bar\imath}$) parametrize a special Kähler manifold with metric $g_{i\bar\jmath}$.
This manifold is uniquely determined by an holomorphic function called the prepotential $F$.
The latter is better defined using the homogeneous (or projective) coordinates $X^\Lambda$ such that
\begin{equation}
	\tau^i = \frac{X^i}{X^0}.
\end{equation} 
The first derivative of the prepotential with respect to $X^\Lambda$ is denoted by
\begin{equation}
	F_\Lambda = \frac{\pd F}{\pd X^\Lambda}.
\end{equation} 
Finally it makes sense to regroup the gauge fields into one single vector
\begin{equation}
	A^\Lambda = (A^0, A^i).
\end{equation} 

One needs to introduce two more quantities, respectively the Kähler potential and the Kähler connection
\begin{equation}
	K = - \ln i (\bar X^\Lambda F_\Lambda - X^\Lambda \bar F^\Lambda), \qquad
	\mc A_\mu = - \frac{i}{2} (\pd_i K\, \pd_\mu \tau^i - \pd_{\bar\imath} K\, \pd_\mu \bar\tau^{\bar\imath}).
\end{equation} 

The Lagrangian for the theory without gauge group is given by
\begin{equation}
	\mc L = - \frac{R}{2}
		+ g_{i\bar\jmath}(\tau, \bar \tau)\, \pd_\mu \tau^i \pd^\nu \bar\tau^{\bar\imath}
		+ \mc I_{\Lambda\Sigma}(\tau, \bar \tau)\, F^\Lambda_{\mu\nu} F^{\Sigma\,\mu\nu}
		- \mc R_{\Lambda\Sigma}(\tau, \bar \tau)\, F^\Lambda_{\mu\nu} \hodge{F}^{\Sigma\,\mu\nu}
\end{equation} 
where $R$ is the Ricci scalar and $\hodge{F}^\Lambda$ is the Hodge dual of $F^\Lambda$.
The matrix
\begin{equation}
	\mc N = \mc R + i\, \mc I
\end{equation} 
can be expressed in terms of $F$.
From this Lagrangian one can introduce the symplectic dual of $F^\Lambda$
\begin{equation}
	G_\Lambda = \frac{\var \mc L}{\var F^\Lambda} = \mc R_{\Lambda\Sigma} F^\Sigma - \mc I_{\Lambda\Sigma} \hodge{F}^\Sigma.
\end{equation} 
