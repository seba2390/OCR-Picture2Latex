\section{Algorithm: main ideas}
\label{sec:algo}



In this section we summarize the original algorithm together with its extension to gauge fields.
We will see that the algorithm involves the transformations of two different objects (the tensor structure and the coordinate-dependent functions of the fields) which can be taken care of separately.
The transformation of the tensor structure is simple and no new idea (for $d = 4$) will be needed after this section since we will be dealing with the two most general tensor structures for bosonic fields of spin less than or equal to two (the metric and vector fields).
On the other hand the transformation of the functions is more involved and we will introduce new concepts through simple examples in the next section before giving the most general formulation in~\cref{sec:general}.
We review the two different prescriptions for the transformation and we illustrate the algorithm with two basic examples: the flat space and the Kerr--Newman metrics.


\subsection{Summary}
\label{sec:algo:summary}


The general procedure for the Janis--Newman algorithm can be summarized as follows:
\begin{enumerate}
	\item Perform a change of coordinates $(t, r)$ to $(u, r)$ and a gauge transformation such that $g_{rr} = 0$ and $A_r = 0$.
	
	\item Take $u, r \in \C$ and replace the functions $f_i(r)$ inside the real fields by new real-valued functions $\tilde f_i(r, \bar r)$ (there is a set of “empirical” rules).
	
	\item Perform a complex change of coordinates and transform accordingly:
	\begin{enumerate}
		\item the tensor structure, i.e.\ the $\dd x^\mu$ (two prescriptions: Janis--Newman~\cite{Newman:1965:NoteKerrSpinningParticle} and Giampieri~\cite{Giampieri:1990:IntroducingAngularMomentum});
		\label{algo:list:procedure-tensor}
		
		\item the functions $\tilde f_i(r, \bar r)$.
		\label{algo:list:procedure-functions}
	\end{enumerate}
	
	\item Perform a change of coordinates to simplify the metric (for example to Boyer--Lindquist system).
	If the transformation is infinitesimal then one should check that it is a valid diffeomorphism, i.e. that it is integrable.
\end{enumerate}
Note that in the last point the operations (a) and (b) are independent.
In practice one is performing the algorithm for a generic class of configurations with unspecified $f_i(r)$ in order to obtain general formulas.
One leaves point 2 and (3b) implicit since the other steps are independent of the form of the functions.
Then given a specific configuration one can perform 2 and (3b).

Throughout the review we will not be interested in showing that the examples discussed are indeed solutions but merely to explain how to extend the algorithm.
All examples we are discussing have been shown to be solutions of the theory under concerned and we refer the reader to the original literature for more details.
For this reason we will rarely mention the action or the equations of motion and just discussed the fields and their expressions.

One could add a fifth point to the list: checking the equations of motion.
We stress again that the algorithm is \emph{off-shell} and there is no guarantee (except in some specific cases~\cite{Adamo:2014:KerrNewmanMetricReview}) that a solution is mapped to a solution.


\subsection{Algorithm}
\label{sec:algo:algorithm}


We present the algorithm for a metric $g_{\mu\nu}$ and a gauge field $A_\mu$ associated with a $\group{U}(1)$ gauge symmetry.
This simple case is sufficient to illustrate the main features of the algorithm.

As already mentioned in the introduction, the authors of~\cite{Newman:1965:MetricRotatingCharged} failed to derive the field strength of the Kerr--Newman black hole from the Reissner--Nordström one.
In the null tetrad formalism it is natural to write the field strength in terms of its Newman--Penrose coefficients, but a problem arises when one tries to generate the rotating solution since one of the coefficients is zero in the case of Reissner--Nordström, but non-zero for Kerr--Newman.
Three different prescriptions have been proposed: two works in the Newman--Penrose formalism -- one with the field strength~\cite{Keane:2014:ExtensionNewmanJanisAlgorithm} and one with the gauge field~\cite{Erbin:2015:JanisNewmanAlgorithmSimplifications} -- while the third extends Giampieri's approach to the gauge field~\cite{Erbin:2015:JanisNewmanAlgorithmSimplifications}.
Since the proposals from~\cite{Erbin:2015:JanisNewmanAlgorithmSimplifications} fit more directly (and parallel each other) inside the prescriptions of Janis--Newman and Giampieri, we will focus on them.
It is also more convenient to work with the gauge fields since any other quantity can be easily computed from them.


\subsubsection{Seed metric and gauge fields}


The seed metric and gauge field take the form
\begin{subequations}
\label{algo:eq:static:tr}
\begin{gather}
	\label{algo:eq:static:metric:tr}
	\dd s^2 = - f(r)\, \dd t^2 + f(r)^{-1}\, \dd r^2 + r^2 \dd \Omega^2, \qquad
	\dd \Omega^2 = \dd\theta^2 + H(\theta)^2\, \dd \phi^2, \\
	\label{algo:eq:static:vector:tr}
	A = f_A(r)\, \dd t.
\end{gather}
\end{subequations}
The normalized curvature of the $(\theta, \phi)$ sections (or equivalently of the horizon) is denoted by $\kappa$
\begin{equation}
	\kappa =
	\begin{cases}
		+1 & S^2, \\
		-1 & H^2
	\end{cases}
\end{equation} 
where $S^2$ and $H^2$ are respectively the sphere and the hyperboloid,\footnotemark{} and one has%
\footnotetext{%
	We leave aside the case of the plane $\R^2$ with $\kappa = 0$.
	The formulas can easily be extended to this case.
}
\begin{equation}
	H(\theta) =
	\begin{cases}
		\sin \theta & \kappa = 1, \\
		\sinh \theta & \kappa = -1.
	\end{cases}
\end{equation} 
In all this section we will consider the case of spherical horizon with $\kappa = 1$.


Introduce Eddington--Finkelstein coordinates $(u, r)$
\begin{equation}
	\dd u = \dd t - f^{-1} \dd r
\end{equation} 
in order to remove the $g_{rr}$ term of the metric~\cite{Newman:1965:NoteKerrSpinningParticle}.
Under this transformation the gauge field becomes
\begin{equation}
	A = f_A\, (\dd u + f^{-1} \dd r).
\end{equation} 
The changes of coordinate has introduced an $A_r$ component but since it depends only on the radial coordinate $A_r = A_r(r)$ it can be removed by a gauge transformation.

At the end the metric and gauge fields are
\begin{subequations}
\label{algo:eq:static:ur}
\begin{gather}
	\label{algo:eq:static:metric:ur}
	\dd s^2 = - f\, \dd t^2 + 2 \dd u \dd r + r^2 \dd \Omega^2, \\
	\label{algo:eq:static:vector:ur}
	A = f_A\, \dd u.
\end{gather} 
\end{subequations}

This last step was missing in~\cite{Newman:1965:MetricRotatingCharged} and explains why they could not derive the full solution from the algorithm.
The lesson to draw is that the validity of the algorithm depends a lot on the coordinate basis\footnotemark{} and of the parametrization of the fields, although guiding principle founded on all known examples seems that one needs to have
\footnotetext{%
	The canonical example being that the Kerr metric in quasi-isotropic coordinates cannot be derived from the Schwarzschild metric in isotropic coordinates while it can be derived in the usual coordinates (see \cref{sec:algorithm:kerr-newman}).
}
\begin{equation}
	g_{rr} = 0, \qquad
	A_r = 0.
\end{equation} 


\subsubsection{Janis--Newman prescription: Newman--Penrose formalism}


The Janis--Newman prescription for transforming the tensor structure relies on the Newman--Penrose formalism~\cite{Newman:1965:NoteKerrSpinningParticle, Newman:1965:MetricRotatingCharged, Adamo:2014:KerrNewmanMetricReview}.


First one needs to obtain the contravariant expressions of the metric and of the gauge field
\begin{subequations}
\begin{gather}
	\frac{\pd^2}{\pd s^2} = g^{\mu\nu} \pd_\mu \pd_\nu
		= f\, \pd r^2 - 2\, \pd u \pd r + \frac{1}{r^2} \left( \pd_\theta^2 + \frac{\pd_\phi^2}{\sin^2 \theta} \right), \\
	A = - f_A\, \pd_r.
\end{gather}
\end{subequations}
Then one introduces null complex tetrads
\begin{equation}
	Z_a^\mu = \{ \ell^\mu, n^\mu, m^\mu, \bar m^\mu \}
\end{equation} 
with flat metric
\begin{equation}
	\eta^{ab} =
		\begin{pmatrix}
			0 & -1 & 0 & 0 \\
			-1 & 0 & 0 & 0 \\
			0 &  0 & 0 & 1 \\
			0 &  0 & 1 & 0 \\
		\end{pmatrix}
\end{equation} 
such that
\begin{equation}
	g^{\mu\nu} = \eta^{ab} Z^\mu_a Z^\nu_b
		= - \ell^\mu n^\nu - \ell^\nu n^\mu + m^\mu \bar m^\nu + m^\nu \bar m^\mu.
\end{equation} 
The explicit tetrad expressions are
\begin{equation}
	\label{algo:eq:static:tetrads}
	\ell^\mu = \delta_r^\mu, \qquad
	n^\mu = \delta_u^\mu -\frac{f}{2}\; \delta_r^\mu, \qquad
	m^\mu = \frac{1}{\sqrt{2} \bar r} \left(\delta_\theta^\mu + \frac{i}{\sin \theta}\; \delta_\phi^\mu \right)
\end{equation}
and the gauge field is
\begin{equation}
	A^\mu = -f_A\, \ell^\mu.
\end{equation} 
Note that without the gauge transformation there would be an additional term and the expression of $A^\mu$ in terms of the tetrads would be ambiguous.

At this point $u$ and $r$ are allowed to take complex values but keeping $\ell^\mu$ and $n^\mu$ real and $\conj{(m^\mu)} = \bar m^\mu$ and replacing
\begin{equation}
	f(r) \longrightarrow \tilde f(r, \bar r) \in \R, \qquad
	f_A(r) \longrightarrow \tilde f_A(r, \bar r) \in \R.
\end{equation} 
Consistency implies that one recovers the seed for $\bar r = r$ and $\bar u = u$.

Finally one can perform a complex change of coordinates
\begin{equation}
	\label{algo:eq:change:complexification-ur}
	u = u' + i a \cos \theta, \qquad
	r = r' - i a \cos \theta
\end{equation} 
where $a$ is a parameter (to be interpreted as the angular momentum per unit of mass) and $r', u' \in \R$.
While this transformation seems arbitrary we will show later (\cref{sec:general,sec:derivation}) how to extend it and that general consistency limits severely the possibilities.
The tetrads transform as vectors
\begin{equation}
	Z'^\mu_a = \frac{\pd x'^\mu}{\pd x^\nu}\, Z^\nu_a
\end{equation} 
and this lead to the expressions
\begin{equation}
\begin{gathered}
	\label{algo:eq:rotating:tetrads}
	\ell'^\mu = \delta_r^\mu, \qquad
	n'^\mu = \delta_u^\mu - \frac{\tilde f}{2}\, \delta_r^\mu, \\
	m'^\mu = \frac{1}{\sqrt{2} (r' + i a \cos \theta)} \left(\delta_\theta^\mu + \frac{i}{\sin \theta}\, \delta_\phi^\mu - i a \sin \theta\, (\delta_u^\mu - \delta_r^\mu) \right).
\end{gathered}
\end{equation} 
After inverting the contravariant form of the metric and the gauge field one is lead to the final expressions
\begin{subequations}
\label{algo:eq:rotating:ur}
\begin{gather}
	\label{algo:metric:rotating:ur}
		\dd s'^2  = - \tilde f\, (\dd u' - a \sin^2 \theta\, \dd\phi)^2
			- 2\, (\dd u' - a \sin^2 \theta\, \dd\phi) (\dd r' + a \sin^2 \theta\, \dd\phi)
			+ \rho^2 \dd \Omega^2, \\
	A' = \tilde f_A\, (\dd u' - a \sin^2 \theta\, \dd \phi).
\end{gather}
\end{subequations}
where
\begin{equation}
	\label{algo:eq:rotating:metric:rho}
	\rho^2 = \abs{r}^2
		= r'^2 + a^2 \cos^2 \theta.
\end{equation} 
The coordinate dependence of the functions can be written as
\begin{equation}
	\tilde f = \tilde f(r, \bar r)
		= \tilde f(r', \theta)
\end{equation} 
in the new coordinates (and similarly for $\tilde f_A$), but note that the $\theta$ dependence is not arbitrary and comes solely from $\Im r$.



\subsubsection{Giampieri prescription}


The net effect of the transformation \eqref{algo:eq:change:complexification-ur} on the tensor structure amounts to the replacements
\begin{equation}
	\label{algo:eq:replacement-diff}
	\dd u \longrightarrow \dd u' - a \sin^2 \theta\, \dd \phi, \qquad
	\dd r \longrightarrow \dd r' + a \sin^2 \theta\, \dd \phi
\end{equation} 
by comparing \eqref{algo:eq:static:ur} and \eqref{algo:eq:rotating:ur}, up to the $r^2 \to \rho^2$ in front of $\dd\Omega^2$.
Is it possible to obtain the same effect by avoiding the Newman--Penrose formalism and all the computations associated to changing from covariant to contravariant expressions?
Inspecting the infinitesimal form of \eqref{algo:eq:change:complexification-ur}
\begin{equation}
	\dd u = \dd u' - i a \sin \theta\, \dd \theta, \qquad
	\dd r = \dd r' + i a \sin \theta\, \dd \theta,
\end{equation} 
one sees that \eqref{algo:eq:replacement-diff} can be recovered if one sets~\cite{Giampieri:1990:IntroducingAngularMomentum}
\begin{equation}
	\label{algo:eq:giampieri-ansatz}
	i \dd \theta = \sin \theta\, \dd\phi.
\end{equation} 
Note that it should be done only in the infinitesimal transformation and not elsewhere in the metric.
Although some authors~\cite{Ibohal:2005:RotatingMetricsAdmitting, Ferraro:2014:UntanglingNewmanJanisAlgorithm} mentioned the equivalence between the tetrad computation and \eqref{algo:eq:replacement-diff}, it is surprising that this direction has not been followed further.

While this new prescription is not rigorous and there is no known way to derive \eqref{algo:eq:giampieri-ansatz}, it continues to hold for the most general seed (\cref{sec:general}) and it gives systematically the same results as the Janis--Newman prescription, as can be seen by simple inspection.
In particular this approach is not adding nor removing any of the ambiguities due to the function transformations that are already present and well-known in JN algorithm.
Since this prescription is much simpler we will continue to use it throughout the rest of this review (we will show in \cref{sec:general} how it is modified for topological horizons).

Finally the comparison of the two prescriptions clearly shows that the $r^2$ factor in front of $\dd\Omega^2$ should be considered as a function instead of a part of the tensor structure: the replacement $r^2 \to \rho^2$ is dictated by the rules given in the next section.
We did not want to enter into these subtleties here but this will become evident in \cref{sec:general}.



\subsubsection{Transforming the functions}


The transformation of the functions is common to both the Janis--Newman and Giampieri prescriptions since they are independent of the tensor structure.
This step is the main weakness of the Janis--Newman algorithm because there is no unique way to perform the replacement and for this reason the final result contains some part of arbitrariness.
This provides another incentive for checking systematically if the equations of motion are satisfied.
Nonetheless examples have provided a small set of rules~\cite{Newman:1965:NoteKerrSpinningParticle, Newman:1965:MetricRotatingCharged, Drake:2000:UniquenessNewmanJanisAlgorithm, Erbin:2015:JanisNewmanAlgorithmSimplifications}
\begin{subequations}
\label{algo:eq:rules}
\begin{align}
	r & \longrightarrow \frac{1}{2} (r + \bar r) = \Re r, \\
	\frac{1}{r} & \longrightarrow \frac{1}{2} \left(\frac{1}{r} + \frac{1}{\bar r}\right) = \frac{\Re r}{\abs{r}^2}, \\
	r^2 & \longrightarrow \abs{r}^2.
\end{align}
\end{subequations}
The idea is to use geometric or arithmetic means.
All other functions can be reduced to a combination of them, for example $1 / r^2$ is complexified as $1 / \abs{r}^2$.

Every known configuration which does not involve a magnetic, a NUT charge, complex scalar fields or powers higher of $r$than quadratic can be derived with these rules (these cases will be dealt with in \cref{sec:extension,sec:general}).
Hence despite the fact that there is some arbitrariness, it is ultimately quite limited and very few options are possible in most cases.


\subsubsection{Boyer--Lindquist coordinates}


Boyer--Lindquist coordinates are defined to be those with the minimal number of non-zero off-diagonal components in the metric.
Performing the transformation (the primes in \eqref{algo:eq:rotating:ur} are now omitted)
\begin{equation}
	\label{algo:eq:change:bl}
	\dd u' = \dd t' - g(r) \dd r', \qquad
	\dd \phi = \dd \phi' - h(r) \dd r,
\end{equation} 
the conditions $g_{tr} = g_{r\phi'} = 0$ are solved for
\begin{equation}
	\label{algo:eq:change:bl:solution-gh}
	g(r) = \frac{r^2 + a^2}{\Delta}, \qquad
	h(r) = \frac{a}{\Delta}
\end{equation} 
where we have defined
\begin{equation}
	\label{algo:eq:rotating:tr-delta}
	\Delta(r) = \tilde f \rho^2 + a^2 \sin^2 \theta.
\end{equation} 
As indicated by the $r$-dependence this change of variables is integrable provided that $g$ and $h$ are functions of $r$ only.
However $\Delta$ as given in \eqref{algo:eq:rotating:tr-delta} for a generic configuration contains a $\theta$ dependence: one should check that this dependence cancels once restricted to the example of interest.
Otherwise one is not allowed to perform this change of coordinates (but other systems may still be found).

Given \eqref{algo:eq:change:bl:solution-gh} one gets the metric and gauge fields (deleting the prime)
\begin{subequations}
\label{algo:eq:rotating:tr}
\begin{gather}
	\label{algo:eq:rotating:metric:tr}
	\dd s^2 = - \tilde f\, \dd t^2
		+ \frac{\rho^2}{\Delta}\, \dd r^2
		+ \rho^2 \dd\theta^2
		+ \frac{\Sigma^2}{\rho^2} \sin^2 \theta\, \dd\phi^2
		+ 2 a (\tilde f - 1) \sin^2 \theta\, \dd t \dd\phi, \\
	\label{algo:eq:rotating:vector:tr}
	A = \tilde f_A\, \left(\dd t - \frac{\rho^2}{\Delta}\, \dd r - a \sin^2 \theta\, \dd \phi \right)
\end{gather}
\end{subequations}
with
\begin{equation}
	\label{algo:eq:rotating:tr-sigma}
	\frac{\Sigma^2}{\rho^2} = r^2 + a^2 + a g_{t\phi}.
\end{equation} 
The $rr$-term has been computed from
\begin{equation}
	\label{algo:eq:g-plus-a-sin-h}
	g - a \sin^2 \theta\, h = \frac{\rho^2}{\Delta}.
\end{equation} 
Generically the radial component of the gauge field depends only on radial coordinate $A_r = A_r(r)$ ($\theta$-dependence of the function $\tilde f_A$ sits in a factor $1 / \rho^2$ which cancels the one in front of $\dd r$) and one can perform a gauge transformation in order to set it to zero, leaving
\begin{equation}
	A = \tilde f_A\, \left(\dd t - a \sin^2 \theta\, \dd \phi \right).
\end{equation} 



\subsection{Examples}
\label{sec:algo:examples}


\subsubsection{Flat space}
\label{sec:algorithm:flat}


It is straightforward to check that the algorithm applied to the Minkowski metric -- which has $f = 1$, leading to $\tilde f = 1$ -- in spherical coordinates
\begin{equation}
	\dd s^2 = - \dd t^2 + \dd r^2 + r^2 \big( \dd\theta^2 + \sin^2 \theta\; \dd \phi^2 \big)
\end{equation} 
gives again the Minkowski metric but in spheroidal coordinates \eqref{coord:metric:4d:spheroidal} (after a Boyer--Lindquist transformation)
\begin{equation}
	\dd s^2 = - \dd t^2 + \frac{\rho^2}{r^2 + a^2}\; \dd r^2 + \rho^2 \dd\theta^2 + (r^2 + a^2) \sin^2 \theta\; \dd \phi^2,
\end{equation} 
recalling that $\rho^2 = r^2 + a^2 \cos^2 \theta$.
The metric is exactly diagonal because $g_{t\phi} = 0$ for $\tilde f = 1$ from \eqref{algo:eq:rotating:metric:tr}.

Hence for flat space the JN algorithm reduces to a change of coordinates, from spherical to (oblate) spheroidal coordinates: the $2$-spheres foliating the space in the radial direction are deformed to ellipses with semi-major axis $a$.

This fact is an important consistency check that will be useful when extending the algorithm to higher dimensions (\cref{sec:higher}) or to other coordinate systems (such as one with direction cosines).
Moreover in this case one can forget about the time direction and consider only the transformation of the radial coordinate.


\subsubsection{Kerr--Newman}
\label{sec:algorithm:kerr-newman}


The seed function is the Reissner--Nordström for which the metric and gauge field are
\begin{equation}
	\label{algo:eq:rn:functions}
	f(r) = 1 - \frac{2m}{r} + \frac{q^2}{r^2}, \qquad
	f_A = \frac{q}{r}.
\end{equation} 
Applications of the rules \eqref{algo:eq:rules} leads to
\begin{subequations}
\begin{gather}
	\tilde f = 1 - \frac{2 m \Re r}{\abs{r}^2} + \frac{q^2}{\abs{r}^2}
		= 1 + \frac{q^2 - 2 m r'}{\rho^2}, \\
	\tilde f_A = \frac{q \Re r}{\abs{r}^2}
		= \frac{q r'}{\rho^2}.
\end{gather}
\end{subequations}
These functions together with \eqref{algo:eq:rotating:tr} describe correctly the Kerr--Newman solution~\cite{Visser:2009:KerrSpacetimeBrief, Adamo:2014:KerrNewmanMetricReview}.
For completeness we spell out the expressions of the quantities appearing in the metric
\begin{subequations}
\begin{gather}
	\frac{\Sigma^2}{\rho^2} = r^2 + a^2 - \frac{q^2 - 2 m r}{\rho^2}\, a^2 \sin^2 \theta, \\
	\Delta = r^2 - 2 m r + a^2 + q^2.
\end{gather}
\end{subequations}
In particular $\Delta$ does not contain any $\theta$ dependence and the BL transformation is well defined.
Moreover the radial component of the gauge field is
\begin{equation}
	A_r = - \frac{\tilde f_A \rho^2}{\Delta}
		= \frac{q r}{\Delta}
\end{equation} 
and it is independent of $\theta$.
