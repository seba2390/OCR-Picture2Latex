\section{Five dimensional algorithm}
\label{sec:five}


While in four dimensions we have at our disposal many theorems on the classification of solutions, this is not the case for higher dimensions and the bestiary for solutions is much wider and less understood~\cite{Emparan:2008:BlackHolesHigher, Adamo:2014:KerrNewmanMetricReview}.
Rotating solutions in higher dimensions are characterized by several angular momenta.
Important solutions have not yet been discovered, even in the simplest theories such as the charged rotating black holes with several angular momenta in pure Einstein--Maxwell gravity.

Generalizing the JN algorithm in other dimensions is challenging and only small steps have been taken in this direction.
For instance Xu recovered Myers--Perry solution with one angular momentum~\cite{Myers:1986:BlackHolesHigher} from the Schwarzschild--Tangherlini solution~\cite{Xu:1988:ExactSolutionsEinstein} (see also~\cite{Aliev:2006:RotatingBlackHoles}), and Kim showed how the rotating BTZ black hole~\cite{Banados:1992:BlackHoleThree} can be obtained from its static limit~\cite{Kim:1997:NotesSpinningAdS3, Kim:1999:SpinningBTZBlack}.
One of the difficulty is to be able to perform several successive transformations in order to introduce all the allowed angular momenta.

In this section we report the successful generalization of the JN algorithm to five dimensions where we recover two examples~\cite{Erbin:2015:FivedimensionalJanisNewmanAlgorithm}: the complete Myers--Perry black hole~\cite{Myers:1986:BlackHolesHigher} and the Breckenridge--Myers--Peet--Vafa (BMPV) extremal black hole~\cite{Breckenridge:1997:DbranesSpinningBlack}.
We give of proposal for extending this method to higher dimensions in the next section.

It appears that the two angular momenta can be added one after the other by performing two successive transformations, each using different rules for complexifying the functions.
These rules can be understood as transforming only the functions that appear in the part of the metric which describes the rotation plane associated to the angular momentum.
Our method makes use of the Giampieri prescription and we did not succeed in expressing it in terms of the Janis--Newman prescription.

A major application of our work would be to find the charged solution with two angular momenta of the $5d$ Einstein--Maxwell gravity.
This problem is highly non-trivial and there is few chances that this technique would work directly~\cite{Aliev:2006:RotatingBlackHoles}, but one can imagine that a generalization of Demiański's approach~\cite{Demianski:1972:NewKerrlikeSpacetime} (see \cref{sec:derivation}) could lead to new interesting solutions in five dimensions.
An intermediate step is represented by the CCLP metric~\cite{Chong:2005:GeneralNonExtremalRotating} which is a solution of the Einstein--Maxwell theory with a Chern--Simons term, but it cannot be derived from the JN algorithm and we give some intuition about this fact in the last subsection.

Finally one could seek for an extension of the algorithm to the derivation of black rings~\cite{Emparan:2002:RotatingBlackRing, Emparan:2008:BlackHolesHigher}.
Similarly it may be possible that such techniques could be used in $d = 4$ to derive multicentre solutions (for instance one could imagine adding rotation to both centres successively, changing coordinate system in-between to place the origin of the coordinates at each centre).


\subsection{Myers--Perry black hole}
\label{sec:higher-jna:5d:myers-perry}


In this section we show how to recover the Myers--Perry black hole in five dimensions through the Giampieri prescription.
This is a solution of $5$-dimensional pure Einstein theory which possesses two angular momenta and it generalizes the Kerr black hole.
The importance of this solution lies in the fact that it can be constructed in any dimension.

The seed metric is given by the five-dimensional Schwarzschild--Tangherlini metric
\begin{equation}
	\dd s^2 = - f(r)\, \dd t^2 + f(r)^{-1}\, \dd r^2 + r^2\, \dd \Omega_3^2
\end{equation}
where $\dd \Omega_3^2$ is the metric on $S^3$, which can be expressed in Hopf coordinates (see \cref{app:coord:5d:hopf})
\begin{equation}
	\label{higher-jna:eq:coord-S3-spherical}
	\dd \Omega_3^2 = \dd\theta^2 + \sin^2 \theta\, \dd\phi^2 + \cos^2 \theta\, \dd\psi^2,
\end{equation} 
and the function $f(r)$ is given by
\begin{equation}
	f(r) = 1 - \frac{m}{r^2}.
\end{equation}

An important feature of the JN algorithm is the fact that a given set of transformations in the $(r,\phi)$-plane generates rotation in the latter.
Generating several angular momenta in different 2-planes would then require successive applications of the JN algorithm on different hypersurfaces.
In order to do so, one has to identify what are the 2-planes which will be submitted to the algorithm.
In five dimensions, the two different planes that can be made rotating are the planes $(r,\phi)$ and $(r,\psi)$.
We claim that it is necessary to dissociate the radii of these 2-planes in order to apply separately the JN algorithm on each plane and hence to generate two distinct angular momenta.
In order to dissociate the parts of the metric that correspond to the rotating and non-rotating $2$-planes, one can protect the function $r^2$ to be transformed under complex transformations in the part of the metric defining the plane which will stay static.
We thus introduce the function
\begin{equation}
	R(r) = r
\end{equation} 
such that the metric in null coordinates reads
\begin{equation}
	\label{higher-jna:5d-jna:metric:static:general-ur}
	\dd s^2 = - \dd u\, (\dd u + 2 \dd r)
		+ (1 - f)\, \dd u^2
		+ r^2 (\dd\theta^2 + \sin^2 \theta\, \dd\phi^2) + R^2 \cos^2 \theta\, \dd\psi^2.
\end{equation} 
The first transformation -- hence concerning the $(r,\phi)$-plane -- is
\begin{equation}
	\label{higher-jna:eq:5d-ansatz-hopf-1}
	\begin{gathered}
		u = u' + i a \cos \chi_1, \qquad
		r = r' - i a \cos \chi_1, \\
		i\, \dd \chi_1 = \sin \chi_1\ \dd\phi, \qquad\text{~~with~~}\chi_1 = \theta, \\
		\dd u = \dd u' - a \sin^2 \theta\, \dd\phi, \qquad
		\dd r = \dd r' + a \sin^2 \theta\, \dd\phi,
	\end{gathered}
\end{equation}
and $f$ is replaced by $\tilde f^{\{1\}} = \tilde f^{\{1\}}(r, \theta)$.
Indeed one needs to keep track of the order of the transformation, since the function $f$ will be complexified twice consecutively.
On the other hand $R(r) = \Re(r)$ is transformed\footnotemark{} into $R' = r'$ and one finds (omitting the primes)%
\footnotetext{%
	Note that as a function this corresponds to the rule \eqref{gen:eq:rules:r} but we will see below that $R$ is better interpreted as a coordinate since below it will appear as $\dd R$.
}
\begin{equation}
	\begin{aligned}
	\dd s^2 = - \dd u^2 &- 2\, \dd u \dd r
		+ \big(1 - \tilde f^{\{1\}} \big) (\dd u - a \sin^2 \theta\, \dd \phi)^2
		+ 2 a \sin^2 \theta\, \dd r \dd \phi \\
		&+ (r^2 + a^2 \cos^2 \theta) \dd\theta^2
		+ (r^2 + a^2) \sin^2 \theta\, \dd\phi^2
		+ r^2 \cos^2 \theta\, \dd \psi^2.
	\end{aligned}
\end{equation} 
The function $\tilde f^{\{1\}}$ is
\begin{equation}
	\tilde f^{\{1\}} = 1 - \frac{m}{\abs{r}^2} = 1 - \frac{m}{r^2 + a^2 \cos^2 \theta}.
\end{equation} 
There is a cancellation between the $(u, r)$ and the $(\theta, \phi)$ parts of the metric
\begin{subequations}
\begin{align}
	\dd s_{u,r}^2 &= (1 - \tilde f^{\{1\}})\, (\dd u - a \sin^2 \theta\, \dd \phi)^2
		- \dd u (\dd u + 2 \dd r )
		+ 2 a \sin^2 \theta \, \dd r \dd \phi
		+ a^2 \sin^4 \theta\, \dd \phi^2, \\
	\dd s_{\theta,\phi}^2 &= (r^2 + a^2 \cos^2 \theta) \dd\theta^2
			+ \big(r^2 + a^2 (1 - \sin^2 \theta) \big) \sin^2 \theta\, \dd\phi^2.
\end{align}
\end{subequations}

In addition to the terms present in \eqref{higher-jna:5d-jna:metric:static:general-ur} one obtains new components corresponding to the rotation of the first plane $(r, \phi)$.
Since the structure is very similar one can perform a transformation\footnotemark{} in the second plane $(r, \psi)$%
\footnotetext{%
	The easiest justification for choosing the sinus here is by looking at the transformation in terms of direction cosines, see \cref{sec:higher-jna:examples:myers-perry-5d}.
	Otherwise this term can be guessed by looking at Myers--Perry non-diagonal terms.
}
\begin{equation}
	\label{higher-jna:eq:5d-ansatz-hopf-2}
	\begin{gathered}
		u = u' + i b\, \sin \chi_2, \qquad
		r = r' - i b\, \sin \chi_2, \\
		i\, \dd \chi_2 = - \cos \chi_2\, \dd\psi, \qquad \text{~~with~~}\chi_2 = \theta, \\
		\dd u = \dd u' - b \cos^2 \theta\, \dd\psi, \qquad
		\dd r = \dd r' + b \cos^2 \theta\, \dd\psi,
	\end{gathered}
\end{equation}
can be applied directly to the metric
\begin{equation}
	\begin{aligned}
	\dd s^2 = - \dd u^2 &- 2\, \dd u \dd r
		+ \big(1 - \tilde f^{\{1\}} \big) (\dd u - a \sin^2 \theta\, \dd \phi)^2
		+ 2 a \sin^2 \theta\, \dd R \dd \phi \\
		&+ \rho^2 \dd\theta^2
		+ (R^2 + a^2) \sin^2 \theta\, \dd\phi^2
		+ r^2 \cos^2 \theta\, \dd \psi^2
	\end{aligned}
\end{equation} 
where we introduced once again the function $R(r) = \Re(r)$ to protect the geometry of the first plane to be transformed under complex transformations.

The final result (using again $R = r'$ and omitting the primes) becomes
\begin{equation}
	\begin{aligned}
	\dd s^2 = - \dd u^2 &- 2\, \dd u \dd r
		+ \big(1 - \tilde f^{\{1, 2\}} \big) (\dd u - a \sin^2 \theta\, \dd \phi - b \cos^2 \theta\, \dd \psi)^2
		\\
		&+ 2 a \sin^2 \theta\, \dd r \dd \phi
		+ 2 b \cos^2 \theta\, \dd r\dd \psi \\
		&+ \rho^2 \dd\theta^2
		+ (r^2 + a^2) \sin^2 \theta\, \dd\phi^2
		+ (r^2 + b^2) \cos^2 \theta\, \dd \psi^2
	\end{aligned}
\end{equation} 
where
\begin{equation}
	\rho^2 = r^2 + a^2 \cos^2 \theta + b^2 \sin^2 \theta.
\end{equation} 
Furthermore, the function $\tilde f^{\{1\}}$ has been complexified as
\begin{equation}
 	\tilde f^{\{1,2\}} = 1 - \frac{m}{\abs{r}^2 + a^2 \cos^2 \theta}
		= 1 - \frac{m}{r'^2 + a^2 \cos^2 \theta + b^2 \sin^2 \theta}
		= 1 - \frac{m}{\rho^2}.
\end{equation}

The metric can then be transformed into the Boyer--Lindquist (BL) using
\begin{equation}
	\label{higher:change:5d-bl}
	\dd u = \dd t - g(r)\, \dd r, \qquad
	\dd\phi = \dd\phi' - h_\phi(r)\, \dd r, \qquad
	\dd\psi = \dd\psi' - h_\psi(r)\, \dd r.
\end{equation} 
Defining the parameters\footnotemark{}%
\footnotetext{%
	See \eqref{higher-jna:metric:rotating:result-jna-bl-parameters} for a definition of $\Delta$ in terms of $\tilde f$.
}
\begin{equation}
	\Pi = (r^2 + a^2) (r^2 + b^2), \qquad
	\Delta = r^4 + r^2 (a^2 + b^2- m) + a^2 b^2,
\end{equation}
the functions can be written
\begin{equation}
	\label{higher:change:myers-perry:bl-g-h}
	g(r) = \frac{\Pi}{\Delta}, \qquad
	h_\phi(r) = \frac{\Pi}{\Delta}\, \frac{a}{r^2 + a^2}, \qquad
	h_\psi(r) = \frac{\Pi}{\Delta}\, \frac{b}{r^2 + b^2}.
\end{equation} 
Finally one gets
\begin{equation}
	\label{higher-jna:metric:rotating:5d-2-moments-bl}
	\begin{aligned}
		\dd s^2 = - \dd t^2
			&+ \big(1 - \tilde f^{\{1, 2\}} \big) (\dd t - a \sin^2 \theta\, \dd \phi - b \cos^2 \theta\, \dd \psi)^2
			+ \frac{r^2 \rho^2}{\Delta}\, \dd r^2 \\
			&+ \rho^2 \dd\theta^2
			+ (r^2 + a^2) \sin^2 \theta\, \dd\phi^2
			+ (r^2 + b^2) \cos^2 \theta\, \dd \psi^2.
	\end{aligned}
\end{equation} 
One recovers here the five dimensional Myers--Perry black hole with two angular momenta~\cite{Myers:1986:BlackHolesHigher}.


\subsection{BMPV black hole}
\label{sec:higher-jna:5d:bmpv}




\subsubsection{Few properties and seed metric}


In this section we focus on another example in five dimensions, which is the BMPV black hole~\cite{Breckenridge:1997:DbranesSpinningBlack, Gauntlett:1999:BlackHolesD5}.
This solution possesses many interesting properties, in particular it can be proven that it is the only asymptotically flat rotating BPS black hole in five dimensions with the corresponding near-horizon geometry~\cites[sec.~7.2.2, 8.5]{Emparan:2008:BlackHolesHigher}{Reall:2003:HigherDimensionalBlack}.\footnotemark{}%
\footnotetext{%
	Other possible near-horizon geometries are $S^1 \times S^2$ (for black rings) and $T^3$, even if the latter does not seem really physical.
	BMPV horizon corresponds to the squashed $S^3$.
}
It is interesting to notice that even if this extremal solution is a slowly rotating metric, it is an exact solution (whereas Einstein equations need to be truncated for consistency of usual slow rotation).

For a rotating black hole the BPS and extremal limits do not coincide~\cites[sec.~7.2]{Emparan:2008:BlackHolesHigher}[sec.~1]{Gauntlett:1999:BlackHolesD5}: the first implies that the mass is related to the electric charge,\footnote{It is a consequence from the BPS bound $m \ge \sqrt{3}/2\, \abs{q}$.} while extremality\footnotemark{}%
\footnotetext{%
	Regularity is given by a bound, which is saturated for extremal black holes.
}
implies that one linear combination of the angular momenta vanishes, and for this reason we set $a = b$ from the beginning.\footnotemark{}%
\footnotetext{%
	If we had kept $a \neq b$ we would have discovered later that one cannot transform the metric to Boyer--Lindquist coordinates without setting $a = b$.
}
Thus two independent parameters are left and are taken to be the mass and one angular momentum.

In the non-rotating limit BMPV black hole reduces to the charged extremal Schwarz\-schild--Tangherlini (with equal mass and charge) written in isotropic coordinates.
For non-rotating black hole the extremal and BPS limit are equivalent.

Both the charged extremal Schwarzschild--Tangherlini and BMPV black holes are solutions of minimal ($N = 2$) $d = 5$ supergravity (Einstein--Maxwell plus Chern--Simons) whose bosonic action is~\cites[sec.~1]{Gauntlett:1999:BlackHolesD5}[sec.~2]{Aliev:2014:SuperradianceBlackHole}[sec.~2]{Gauntlett:2003:AllSupersymmetricSolutions}
\begin{equation}
	\label{higher-jna:higher-jna:action:N=2-d=5-sugra}
	S = - \frac{1}{16\pi G} \int \left(R\, \hodge{1} + F \wedge \hodge{F} + \frac{2\lambda}{3 \sqrt{3}}\, F \wedge F \wedge A \right),
\end{equation} 
where supersymmetry imposes $\lambda = 1$.

Since extremal limits are different for static and rotating black holes we can guess that the black hole obtained from the algorithm will not be a solution of the equations of motion and that it will be necessary to take some limit.

The charged extremal Schwarzschild--Tangherlini black hole is taken as a seed metric~\cites[sec.~3.2]{Gauntlett:2003:AllSupersymmetricSolutions}[sec.~4]{Gibbons:1994:SupersymmetricSelfGravitatingSolitons}[sec.~1.3.1]{Puhm:2013:BlackHolesString}
\begin{equation}
	\label{higher-jna:metric:5d-bmpv}
	\dd s^2 = - H^{-2}\, \dd t^2 + H\, (\dd r^2 + r^2\, \dd\Omega_3^2 )
\end{equation} 
where $\dd\Omega_3^2$ is the metric of the $3$-sphere written in
\eqref{higher-jna:eq:coord-S3-spherical}.
The function $H$ is harmonic
\begin{equation}
	H(r) = 1 + \frac{m}{r^2},
\end{equation} 
and the electromagnetic field reads
\begin{equation}
	\label{higher-jna:pot:5d-bmpv}
	A = \frac{\sqrt{3}}{2 \lambda}\, \frac{m}{r^2}\, \dd t
		= (H - 1)\, \dd t.
\end{equation} 

In the next subsections we apply successively the transformations \eqref{higher-jna:eq:5d-ansatz-hopf-1} and \eqref{higher-jna:eq:5d-ansatz-hopf-2} with $a = b$ in the case $\lambda = 1$.


\subsubsection{Transforming the metric}


The transformation to $(u, r)$ coordinates of the seed metric \eqref{higher-jna:metric:5d-bmpv}
\begin{equation}
	\dd t = \dd u + H^{3/2}\, \dd r
\end{equation} 
gives
\begin{subequations}
\begin{align}
	\dd s^2 &= - H^{-2}\, \dd u^2 - 2 H^{-1/2}\, \dd u \dd r + H r^2\, \dd\Omega_3^2 \\
		&= - H^{-2}\, \big(\dd u - 2 H^{3/2}\, \dd r \big)\, \dd u + H r^2\, \dd\Omega_3^2.
\end{align}
\end{subequations}

For transforming the above metric one should follow the recipe of the previous section: the transformations \eqref{higher-jna:eq:5d-ansatz-hopf-1}
\begin{equation}
	u = u' + i a \cos \theta, \qquad
	\dd u = \dd u' - a \sin^2 \theta\, \dd\phi,
\end{equation}
and \eqref{higher-jna:eq:5d-ansatz-hopf-2}
\begin{equation}
	u = u' + i a\, \sin \theta, \qquad
	\dd u = \dd u' - a \cos^2 \theta\, \dd\psi
\end{equation} 
are performed one after another, transforming each time only the terms that pertain to the corresponding rotation plane.\footnotemark{}%
\footnotetext{%
	For another approach see \cref{sec:higher-jna:5d:bmpv-second-approach}.
}
In order to preserve the isotropic form of the metric the function $H$ is complexified everywhere (even when it multiplies terms that belong to the other plane).

Since the procedure is exactly similar to the Myers--Perry case we give only the final result in $(u, r)$ coordinates
\begin{equation}
	\label{higher-jna:metric:5d-bmpv:ur-before-limit}
	\begin{aligned}
		\dd s^2 = &- \tilde H^{-2} \big(\dd u
				- a (1 - \tilde H^{3/2}) (\sin^2 \theta\, \dd\phi + \cos^2 \theta\, \dd\psi) \big)^2 \\
			&- 2 \tilde H^{-1/2} \big(\dd u - a (1 - \tilde H^{3/2})\, (\sin^2 \theta\, \dd\phi + \cos^2 \theta\, \dd\psi) \big)\, \dd r \\
			&+ 2 a \tilde H\, (\sin^2 \theta\, \dd\phi + \cos^2 \theta\, \dd\psi)\, \dd r
			- 2 a^2 \tilde H \cos^2 \theta \sin^2 \theta\, \dd\phi \dd\psi
			\\
			&+ \tilde H\, \Big(
				(r^2 + a^2) (\dd \theta^2 + \sin^2 \theta\, \dd\phi^2 + \cos^2 \theta\, \dd\psi^2)
				+ a^2 (\sin^2 \theta\, \dd\phi + \cos^2 \theta\, \dd\psi)^2 \Big).
	\end{aligned}
\end{equation} 
After both transformations the resulting function $\tilde H$ is
\begin{equation}
	\label{higher-jna:eq:5d-bmpv:tilde-H}
	\tilde H = 1 + \frac{m}{r^2 + a^2 \cos^2\theta + a^2 \sin^2\theta}
		= 1 + \frac{m}{r^2 + a^2}
\end{equation}
which does not depend on $\theta$.

It is easy to check that the Boyer--Lindquist transformation \eqref{higher:change:5d-bl}
\begin{equation}
	\dd u = \dd t - g(r)\, \dd r, \qquad
	\dd\phi = \dd\phi' - h_\phi(r)\, \dd r, \qquad
	\dd\psi = \dd\psi' - h_\psi(r)\, \dd r
\end{equation} 
is ill-defined because the functions depend on $\theta$.
The way out is to take the extremal limit alluded above.

Following the prescription of \cite{Breckenridge:1997:DbranesSpinningBlack, Gauntlett:1999:BlackHolesD5} and taking the extremal limit
\begin{equation}
	\label{higher-jna:eq:5d-bmpv-extremal-limit}
	a, m \longrightarrow 0, \qquad
	\text{imposing} \qquad
	\frac{m}{a^2} = \cst ,
\end{equation}
one gets at leading order
\begin{equation}
	\tilde H(r) = 1 + \frac{m}{r^2} = H(r), \qquad
	a\, (1 - \tilde H^{3/2}) = - \frac{3\, m a}{2\, r^2}
\end{equation} 
which translate into the metric
\begin{equation}
	\begin{aligned}
		\dd s^2 = - H^{-2}\, & \left(\dd u
				+ \frac{3\, m a}{2\, r^2}\, (\sin^2 \theta\, \dd\phi + \cos^2 \theta\, \dd\psi) \right)^2 \\
			&- 2 H^{-1/2} \left( \dd u + \frac{3\, m a}{2\, r^2}\, (\sin^2 \theta\, \dd\phi + \cos^2 \theta\, \dd\psi) \right) \dd r \\
			&+ H\, r^2 (\dd \theta^2 + \sin^2 \theta\, \dd\phi^2 + \cos^2 \theta\, \dd\psi^2).
	\end{aligned}
\end{equation} 
Then Boyer--Lindquist functions are
\begin{equation}
	\label{higher:change:bmpv:g-h}
	g(r) = H(r)^{3/2}, \qquad
	h_\phi(r) = h_\psi(r) = 0
\end{equation} 
and one gets the metric in $(t, r)$ coordinates
\begin{equation}
	\label{higher-jna:metric:5d-bmpv:bmpv-metric}
	\begin{aligned}
		\dd s^2 = &- \tilde H^{-2} \left(\dd t
				+ \frac{3\, m a}{2\, r^2}\, (\sin^2 \theta\, \dd\phi + \cos^2 \theta\, \dd\psi) \right)^2 \\
			&+ \tilde H\, \Big(\dd r^2 + r^2 \big( \dd \theta^2 + \sin^2 \theta\, \dd\phi^2 + \cos^2 \theta\, \dd\psi^2 \big) \Big).
	\end{aligned}
\end{equation} 
One can recognize the BMPV solution~\cites[p.~4]{Breckenridge:1997:DbranesSpinningBlack}[p.~16]{Gauntlett:1999:BlackHolesD5}.
The fact that this solution has only one rotation parameter can be seen more easily in Euler angle coordinates~\cites[sec.~3]{Gauntlett:1999:BlackHolesD5}[sec.~2]{Gibbons:1999:SupersymmetricRotatingBlack} or by looking at the conserved charges in the $\phi$- and $\psi$-planes~\cite[sec.~3]{Breckenridge:1997:DbranesSpinningBlack}.


\subsubsection{Transforming the Maxwell potential}


The seed gauge field \eqref{higher-jna:pot:5d-bmpv} in the $(u, r)$ coordinates is
\begin{equation}
	A = \frac{\sqrt{3}}{2}\, (H - 1)\, \dd u,
\end{equation} 
since the $A_r(r)$ component can be removed by a gauge transformation.
One can apply the two JN transformations \eqref{higher-jna:eq:5d-ansatz-hopf-1} and \eqref{higher-jna:eq:5d-ansatz-hopf-2} with $b = a$ to obtain
\begin{equation}
	A = \frac{\sqrt{3}}{2}\, (\tilde H - 1) \Big( \dd u - a\, (\sin^2 \theta\, \dd\phi + \cos^2 \theta\, \dd\psi) \Big).
\end{equation} 

Then going into BL coordinates with \eqref{higher:change:5d-bl} and \eqref{higher:change:bmpv:g-h} provides
\begin{equation}
	A = \frac{\sqrt{3}}{2}\, (\tilde H - 1) \Big( \dd t - a\, (\sin^2 \theta\, \dd\phi + \cos^2 \theta\, \dd\psi) \Big) + A_r(r)\, \dd r.
\end{equation} 
Again $A_r$ depends only on $r$ and can be removed by a gauge transformation.
Applying the extremal limit \eqref{higher-jna:eq:5d-bmpv-extremal-limit} finally gives
\begin{equation}
	A = \frac{\sqrt{3}}{2}\, \frac{m}{r^2} \Big( \dd t - a\, (\sin^2 \theta\, \dd\phi + \cos^2 \theta\, \dd\psi) \Big),
\end{equation}
which is again the result presented in~\cite[p. 5]{Breckenridge:1997:DbranesSpinningBlack}.

Despite the fact that the seed metric \eqref{higher-jna:metric:5d-bmpv} together with the gauge field \eqref{higher-jna:pot:5d-bmpv} solves the equations of motion for any value of $\lambda$, the resulting rotating metric solves the equations only for $\lambda = 1$ (see~\cite[sec.~7]{Gauntlett:1999:BlackHolesD5} for a discussion).
An explanation in this reduction can be found in the limit \eqref{higher-jna:eq:5d-bmpv-extremal-limit} that was needed for transforming the metric to Boyer--Lindquist coordinates and which gives a supersymmetric black hole -- which necessarily has $\lambda = 1$.



\subsection{Another approach to BMPV}
\label{sec:higher-jna:5d:bmpv-second-approach}


In \cref{sec:higher-jna:5d:bmpv} we applied the same recipe given in \cref{sec:higher-jna:5d:myers-perry} which, according to our claim, is the standard procedure in five dimensions.

There is another way to derive BMPV black hole.
Indeed, by considering that terms quadratic in the angular momentum do not survive in the extremal limit, they can be added to the metric without modifying the final result.
Hence we can decide to transform all the terms of the metric\footnotemark{} since the additional terms will be subleading.%
\footnotetext{%
	In opposition to our initial recipe, but this is done in a controlled way.
}
As a result the BL transformation is directly well defined and overall formulas are simpler, but we need to take the extremal limit before the end (this could be done either in $(u, r)$ or $(t, r)$ coordinates).
This section shows that both approaches give the same result.

Applying the two transformations
\begin{subequations}
\begin{gather}
	u = u' + i a \cos \theta, \qquad
	\dd u = \dd u' - a \sin^2 \theta\, \dd\phi, \\
	u = u' + i a\, \sin \theta, \qquad
	\dd u = \dd u' - a \cos^2 \theta\, \dd\psi
\end{gather}
\end{subequations}
successively on all the terms one obtains the metric
\begin{equation}
	\begin{aligned}
		\dd s^2 = &- \tilde H^{-2} \big(\dd u
				- a (1 - \tilde H^{3/2}) (\sin^2 \theta\, \dd\phi + \cos^2 \theta\, \dd\psi) \big)^2 \\
			&- 2 \tilde H^{-1/2} \big(\dd u - a (\sin^2 \theta\, \dd\phi + \cos^2 \theta\, \dd\psi) \big)\, \dd r \\
			&+ \tilde H\, \Big(
				(r^2 + a^2) (\dd \theta^2 + \sin^2 \theta\, \dd\phi^2 + \cos^2 \theta\, \dd\psi^2)
				+ a^2 (\sin^2 \theta\, \dd\phi + \cos^2 \theta\, \dd\psi)^2 \Big),
	\end{aligned}
\end{equation} 
where again $\tilde H$ is given by \eqref{higher-jna:eq:5d-bmpv:tilde-H}
\begin{equation}
	\tilde H = 1 + \frac{m}{r^2 + a^2}.
\end{equation} 
Only one term is different when comparing with \eqref{higher-jna:metric:5d-bmpv:ur-before-limit}.

The BL transformation \eqref{higher:change:5d-bl} is well-defined and the corresponding functions are
\begin{equation}
	\label{higher-jna:change:bmpv-2:bl-gh}
	g(r) = \frac{a^2 + (r^2 + a^2) \tilde H(r)}{r^2 + 2 a^2}, \qquad
	h_\phi(r) = h_\psi(r) = \frac{a}{r^2 + 2 a^2}
\end{equation} 
which do not depend on $\theta$.
They lead to the metric
\begin{equation}
	\begin{aligned}
		\dd s^2 = &- \tilde H^{-2} \big(\dd t
				- a (1 - \tilde H^{3/2}) (\sin^2 \theta\, \dd\phi + \cos^2 \theta\, \dd\psi) \big)^2 \\
			&+ \tilde H\, \bigg[
				(r^2 + a^2) \left(\frac{\dd r^2}{r^2 + 2 a^2} + \dd \theta^2 + \sin^2 \theta\, \dd\phi^2 + \cos^2 \theta\, \dd\psi^2 \right) \\
				&\qquad\quad+ a^2 (\sin^2 \theta\, \dd\phi + \cos^2 \theta\, \dd\psi)^2 \bigg].
	\end{aligned}
\end{equation} 

At this point it is straightforward to check that this solution does not satisfy Einstein equations and we need to take the extremal limit \eqref{higher-jna:eq:5d-bmpv-extremal-limit}
\begin{equation}
	a, m \longrightarrow 0, \qquad
	\text{imposing} \qquad
	\frac{m}{a^2} = \cst
\end{equation}
in order to get the BMPV solution \eqref{higher-jna:metric:5d-bmpv:bmpv-metric}
\begin{equation}
	\begin{aligned}
		\dd s^2 = &- \tilde H^{-2} \left(\dd t
				+ \frac{3\, m a}{2\, r^2}\, (\sin^2 \theta\, \dd\phi + \cos^2 \theta\, \dd\psi) \right)^2 \\
			&+ \tilde H\, \Big(\dd r^2 + r^2 \big( \dd \theta^2 + \sin^2 \theta\, \dd\phi^2 + \cos^2 \theta\, \dd\psi^2 \big) \Big).
	\end{aligned}
\end{equation} 

It is surprising that the BL transformation is simpler in this case.
Another point that is worth stressing is that we did not need to take the extremal limit at an intermediate stage, whereas in \cref{sec:higher-jna:5d:bmpv} we had to in order to get a well-defined BL transformation.


\subsection{CCLP black hole}
\label{sec:higher-jna:5d:cclp}


The CCLP black hole~\cite{Chong:2005:GeneralNonExtremalRotating} (see also~\cite[sec.~2]{Aliev:2014:SuperradianceBlackHole}) corresponds to the non-extremal generalization of the BMPV solution and it possesses four independent charges: two angular momenta $a$ and $b$, an electric charge $q$ and the mass $m$.
It is a solution of $d = 5$ minimal supergravity \eqref{higher-jna:higher-jna:action:N=2-d=5-sugra}.

The solution reads
\begin{subequations}
\begin{gather}
	\label{higher-jna:higher-jna:metric:cclp}
	\begin{aligned}
		\dd s^2 = - \dd t^2
			&+ (1 - \tilde f) (\dd t - a \sin^2 \theta\, \dd \phi - b \cos^2 \theta\, \dd \psi)^2
			+ \frac{r^2 \rho^2}{\Delta_r}\, \dd r^2 \\
			&+ \rho^2 \dd\theta^2
			+ (r^2 + a^2) \sin^2 \theta\, \dd\phi^2
			+ (r^2 + b^2) \cos^2 \theta\, \dd \psi^2 \\
			&- \frac{2 q}{\rho^2}\, (b \sin^2 \theta\, \dd \phi + a \cos^2 \theta\, \dd \psi) (\dd t - a \sin^2 \theta\, \dd \phi - b \cos^2 \theta\, \dd \psi),
	\end{aligned} \\
	A = \frac{\sqrt{3}}{2}\, \frac{q}{\rho^2} (\dd t - a \sin^2 \theta\, \dd \phi - b \cos^2 \theta\, \dd \psi),
\end{gather}
\end{subequations}
where the function are given by
\begin{subequations}
\begin{align}
	\rho^2 &= r^2 + a^2 \cos^2 \theta + b^2 \sin^2 \theta, \\
	\tilde f &= 1 - \frac{2 m}{\rho^2} + \frac{q^2}{\rho^4}, \\
	\Delta_r &= \Pi + 2 a b q + q^2 - 2 m r^2.
\end{align}
\end{subequations}

Yet, using our prescription, it appears that the metric of this black hole cannot entirely be recovered.
Indeed while the gauge field can be found straightforwardly, all the terms of the metric but one are generated by our algorithm.
The missing term (corresponding to the last one in \eqref{higher-jna:higher-jna:metric:cclp}) is proportional to the electric charge and the current prescription cannot generate it since the latter can only appear in $\tilde f$ (or in the gauge field); moreover the algorithm cannot explain the first term in parenthesis since $a$ and $b$ always appear with $\dd\phi$ and $\dd\psi$ respectively.

This issue may be related to the fact that the CCLP solution cannot be written as a Kerr--Schild metric but rather as an extended Kerr--Schild one~\cite{Aliev:2009:NoteRotatingCharged, Ett:2010:ExtendedKerrSchildAnsatz, Malek:2014:ExtendedKerrSchildSpacetimes}, which includes an additional term proportional to a spacelike vector.
It appears that the missing term corresponds precisely to this additional term in the extended Kerr--Schild metric and it is well-known that the JN algorithm works mostly for Kerr--Schild metrics.
Moreover the $\Delta$ computed from \eqref{higher-jna:metric:rotating:result-jna-bl-parameters} depends on $\theta$ and the BL transformation would not be well-defined if the additional term is not present to modify $\Delta$ to $\Delta_r$.

