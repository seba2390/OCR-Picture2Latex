\section{Introduction}
\label{sec:intro}


\subsection{Motivations}


General relativity is the theory of gravitational phenomena.
It describes the dynamical evolution of spacetime through the Einstein--Hilbert action that leads to Einstein equations.
The latter are highly non-linear differential equations and finding exact solutions is a notoriously difficult problem.

There are different types of solutions but this review will cover only black-hole-like solutions (type-D in the Petrov classification) which can be described as particle-like objects that carry some charges, such as a mass or an electric charge.

Black holes are important objects in any theory of gravity for the insight they provide into the quantum gravity realm.
For this reason it is a key step, in any theory, to obtain all possible black holes solutions.
Rotating black holes are the most relevant subcases for astrophysics as it is believed that most astrophysical black holes are rotating.
These solutions may also provide exterior metric for rotating stars.

The most general solution of this type in pure Einstein--Maxwell gravity with a cosmological constant $\Lambda$ is the Plebański--Demiański metric~\cite{Plebanski:1975:ClassSolutionsEinsteinMaxwell, Plebanski:1976:RotatingChargedUniformly}: it possesses six charges: mass $m$, NUT charge $n$, electric charge $q$, magnetic charge $p$, spin $a$ and acceleration $\alpha$.
A challenging work is to generalize this solution to more complex Lagrangians, involving scalar fields and other gauge fields with non-minimal interactions, as is typically the case in supergravity.
As the complexity of the equations of motion increase, it is harder to find exact analytical solutions, and one often consider specific types of solutions (extremal, BPS), truncations (some fields are constant, equal or vanishing) or solutions with restricted number of charges.
For this reason it is interesting to find solution generating algorithms -- procedures which transform a seed configuration to another configuration with a greater complexity (for example with a higher number of charges).

An \emph{on-shell} algorithm is very precious because one is sure to obtain a solution if one starts with a seed configuration which solves the equations of motion.
On the other hand \emph{off-shell} algorithms do not necessarily preserve the equations of motion but they are nonetheless very useful: they provide a motivated ansatz, and it is always easier to check if an ansatz satisfy the equations than solving them from scratch.
Even if in practice this kind of solution generating technique does not provide so many new solutions, it can help to understand better the underlying theory (which can be general relativity, modified gravities or even supergravity) and it may shed light on the structure of gravitational solutions.


\subsection{The Janis--Newman algorithm}


The Janis--Newman (JN) algorithm is one of these (off-shell) solution generating techniques, which -- in its original formulation -- generates rotating metrics from static ones.
It was found by Janis and Newman as an alternative derivation of the Kerr metric~\cite{Newman:1965:NoteKerrSpinningParticle}, while shortly after it has been used again to discover the Kerr--Newman metric~\cite{Newman:1965:MetricRotatingCharged}.

This algorithm provides a way to generate axisymmetric metrics from a spherically symmetric seed metric through a particular complexification of radial and (null) time coordinates, followed by a complex coordinate transformation.
Often one performs eventually a change of coordinates to write the result in Boyer--Lindquist coordinates.

The original prescription uses the Newman--Penrose tetrad formalism, which appears to be very tedious since it requires to invert the metric, to find a null tetrad basis where the transformation can be applied, and lastly to invert again the metric.
In~\cite{Giampieri:1990:IntroducingAngularMomentum} Giampieri introduced another formulation of the JN algorithm which avoids gymnastics with null tetrads and which appears to be very useful for extending the procedure to more complicated solutions (such as higher dimensional ones).
However it has been so far totally ignored in the literature.
We stress that \emph{all} results are totally equivalent in both approaches, and every computation that can be done with the Giampieri prescription can be done with the other.
Finally~\cite{Nawarajan:2016:CartesianKerrSchildVariation} provides an alternative view on the algorithm.

In order for the metric to be still real, the radial functions inside the metric must be transformed such that reality is preserved.\footnotemark{}%
\footnotetext{%
	For simplifying, we will say that we complexify the functions inside the metric when we perform this transformation, even if in practice we "realify" them.
}
Despite that there is \emph{no} rigorous statement concerning the possible complexification of these functions, some general features have been worked out in the last decades and a set of rules has been established.
Note that this step is the same in both prescriptions.
In particular these rules can be obtained by solving the equations of motion for some examples and by identifying the terms in the solution~\cite{Demianski:1972:NewKerrlikeSpacetime}.
Another approach consists in expressing the metric functions in terms of the Boyer--Lindquist functions -- that appear in the change of coordinates and which are real --, the latter being then determined from the equations of motion~\cite{Drake:2000:UniquenessNewmanJanisAlgorithm, AzregAinou:2014:StaticRotatingConformal}.

It is widely believed that the JN algorithm is just a trick without any physical or mathematical basis, which is not accurate.
Indeed it was proved by Talbot~\cite{Talbot:1969:NewmanPenroseApproachTwisting} shortly after its discovery why this transformation was well-defined, and he characterizes under which conditions the algorithm is on-shell for a subclass of Kerr--Schild (KS) metrics (see also~\cite{Gurses:1975:LorentzCovariantTreatment}).\footnotemark{}%
\footnotetext{%
	It has not been proved that the KS condition is necessary, but all known examples seem to fit in this category.
}
KS metrics admit a very natural formulation in terms of complex functions for which (some) complex change of coordinates can be defined.
Note that KS metrics are physically interesting as they contain solutions of Petrov type II and D.
Another way to understand this algorithm has been provided by Schiffer et al.~\cite{Schiffer:1973:KerrGeometryComplexified} (see also~\cite{Finkelstein:1975:GeneralRelativisticFields}) who showed that some KS metrics can be written in terms of a unique complex generating function, from which other solutions can be obtained through a complex change of coordinates.
In various papers, Newman shows that the imaginary part of complex coordinates may be interpreted as an angular momentum, and there are similar correspondences for other charges (magnetic…)~\cite{Newman:1973:ComplexCoordinateTransformations, Newman:1974:CuriosityConcerningAngular, Newman:1976:HeavenItsProperties}.
More recently Ferraro shed a new light on the JN algorithm using Cartan formalism~\cite{Ferraro:2014:UntanglingNewmanJanisAlgorithm}.
Uniqueness results for the case of pure Einstein theory have been derived in~\cite{Drake:2000:UniquenessNewmanJanisAlgorithm}.
A recent account on these different points can be found in~\cite{Adamo:2014:KerrNewmanMetricReview}.

In its current form the algorithm is independent of the gravity theory under consideration since it operates independently at the level of each field in order to generate an ansatz, and the equations of motion are introduced only at the end to check if the configuration is a genuine solution.
We believe that a better understanding of the algorithm would lead to an on-shell formulation where the algorithm would be interpreted as some kind of symmetry or geometric property.
One intuition is that every configuration found with the JN algorithm and solving the equations of motion is derived from a seed that also solves the equations of motion (in particular no useful ansatz has been generated from an off-shell seed configuration).

Other solution generating algorithms rely on a complex formulation of general relativity which allows complex changes of coordinates.
This is the case of the Ernst potential formulation~\cite{Ernst:1968:NewFormulationAxially-1, Ernst:1968:NewFormulationAxially-2} or of Quevedo's formalism who decomposes the Riemann tensor in irreducible representations of $\group{SO}(3,\C) \sim \group{SO}(3,1)$ and then uses the symmetry group to generate new solutions~\cite{Quevedo:1992:ComplexTransformationsCurvature, Quevedo:1992:DeterminationMetricCurvature}.



Despite its long history the Janis--Newman algorithm has not produced any new rotating solution for non-fluid configurations (which excludes radiating and interior solutions) beside the Kerr--Newman metric~\cite{Newman:1965:MetricRotatingCharged}, and very few known examples have been reproduced~\cite{Newman:1965:NoteKerrSpinningParticle, Xu:1988:ExactSolutionsEinstein, Kim:1997:NotesSpinningAdS3, Kim:1999:SpinningBTZBlack, Yazadjiev:2000:NewmanJanisMethodRotating}.
Generically the application the Janis--Newman algorithm to interior and radiating systems~\cite{Herrera:1982:ComplexificationNonrotatingSphere, Drake:1997:ApplicationNewmanJanisAlgorithm, Glass:2004:KottlerLambdaKerrSpacetime, Ibohal:2005:RotatingMetricsAdmitting, AzregAinou:2014:GeneratingRotatingRegular, AzregAinou:2014:StaticRotatingConformal} consist in deriving a configuration that do not solve the equations of motion by itself and to interpret the mismatch as a fluid (whose properties can be studied) -- in this review we will not be interested by this kind of applications.
Moreover the only solutions that have been fully derived using the algorithm are the original Kerr metric~\cite{Newman:1965:NoteKerrSpinningParticle}, the $d = 3$ BTZ black hole~\cite{Kim:1997:NotesSpinningAdS3, Kim:1999:SpinningBTZBlack} and the $d$-dimensional Myers--Perry metric with a single angular momentum~\cite{Xu:1988:ExactSolutionsEinstein}: only the metric was found in the other cases~\cite{Newman:1965:MetricRotatingCharged, Yazadjiev:2000:NewmanJanisMethodRotating} and the other fields had to be obtained using the equations of motion.

A first explanation is that there is no real understanding of the algorithm in its most general form (as reviewed above it is understood in some cases): there is no geometric or symmetry-related interpretation.
Another reason is that the algorithm has been defined only for the metric (and real scalar fields) and no extension to the other types of fields was known until recently.
It has also been understood that the algorithm could not be applied in the presence of a cosmological constant~\cite{Demianski:1972:NewKerrlikeSpacetime}: in particular the (a)dS--Kerr(--Newman) metrics~\cite{Carter:1968:HamiltonJacobiSchrodingerSeparable} (see also~\cite{Plebanski:1976:RotatingChargedUniformly, Plebanski:1975:ClassSolutionsEinsteinMaxwell, Gibbons:1977:CosmologicalEventHorizons, Klemm:1997:RotatingTopologicalBlack}) cannot be derived in this way despite various erroneous claims~\cite{Ibohal:2005:RotatingMetricsAdmitting, deUrreta:2015:ExtendedNewmanJanisAlgorithm}.
Finally many works~\cite{Mallett:1988:MetricRotatingRadiating, Viaggiu:2006:InteriorKerrSolutions, Whisker:2008:BraneworldBlackHoles, Lessner:2008:ComplexTrickFivedimensional, Capozziello:2010:AxiallySymmetricSolutions, Caravelli:2010:SpinningLoopBlack, Dadhich:2013:RotatingBlackHole, Ghosh:2013:SpinningHigherDimensional, Ghosh:2015:RotatingBlackHole} (to cite only few) are (at least partly) incorrect or not reliable because they do not check the equations of motion or they perform non-integrable Boyer--Lindquist changes of coordinates~\cite{AzregAinou:2011:CommentSpinningLoop, AzregAinou:2014:GeneratingRotatingRegular, Xu:1998:RadiatingMetricRetarded}.

The algorithm was later shown to be generalizable by Demiański and Newman who demonstrated by writing a general ansatz and solving the equations of motion that other parameters can be added~\cite{Demianski:1966:CombinedKerrNUTSolution, Demianski:1972:NewKerrlikeSpacetime}, even in the presence of a cosmological constant.
While one parameter corresponds to the NUT charge, the other one did not receive any interpretation until now.\footnotemark{}%
\footnotetext{%
	Demiański's metric has been generalized in~\cite{Patel:1978:RadiatingDemianskitypeSpacetimes, Krori:1981:ChargedDemianskiMetric, Patel:1988:RadiatingDemianskitypeMetrics}.
}
Unfortunately Demiański did not express his result to a concrete algorithm (the normal prescription fails in the presence of the NUT charge and of the cosmological constant) which may explain why this work did not receive any further attention.
Note that the algorithm also failed in the presence of magnetic charges.

A way to avoid problems in defining the changes of coordinates to the Boyer--Lindquist system and to find the complexification of the functions has been proposed in~\cite{Drake:2000:UniquenessNewmanJanisAlgorithm} and extended in~\cite{AzregAinou:2014:GeneratingRotatingRegular}: the method consists in writing the unknown complexified function in terms of the functions of the coordinate transformation.
This philosophy is particularly well-suited for providing an ansatz which does not relies on a static seed solution.

More recently it has been investigated whether the JN algorithm can be applied in modified theories of gravity.
Pirogov put forward that rotating metrics obtained from the JN algorithm in Brans--Dicke theory are not solutions if $\alpha \neq 1$~\cite{Pirogov:2013:RotatingScalarvacuumBlack}.
Similarly Hansen and Yunes have shown a similar result in quadratic modified gravity (which includes Gauss--Bonnet)~\cite{Hansen:2013:ApplicabilityNewmanJanisAlgorithm}.\footnotemark{}%
\footnotetext{%
	There are some errors in the introduction of~\cite{Hansen:2013:ApplicabilityNewmanJanisAlgorithm}: they report incorrectly that the result from~\cite{Pirogov:2013:RotatingScalarvacuumBlack} implies that Sen's black hole cannot be derived from the JN algorithm, as was done by Yazadjiev~\cite{Yazadjiev:2000:NewmanJanisMethodRotating}.
	But this black hole corresponds to $\alpha = 1$ and as reported above there is no problem in this case (see~\cite{Horne:1992:RotatingDilatonBlack} for comparison).
	Moreover they argue that several works published before 2013 did not take into account the results of Pirogov~\cite{Pirogov:2013:RotatingScalarvacuumBlack}, published in 2013…
}
These do not include Sen's dilaton--axion black hole for which $\alpha = 1$ (\cref{sec:examples:rotating-T3}), nor the BBMB black hole from conformal gravity (\cref{sec:examples:bbmb}).
Finally it was proved in~\cite{CiriloLombardo:2006:NewmanJanisAlgorithmRotating} that it does not work either for Einstein--Born--Infled theories.\footnotemark{}%
\footnotetext{%
	It may be possible to circumvent the result of~\cite{CiriloLombardo:2006:NewmanJanisAlgorithmRotating} by using the results described in this review since several tools were not known by its author.
}
We note that all these no-go theorem have been found by assuming a transformation with only rotation.


Previous reviews of the JN algorithm can be found in~\cites{Adamo:2014:KerrNewmanMetricReview}[chap.~19]{DInverno:1992:IntroducingEinsteinsRelativity}{Drake:2000:UniquenessNewmanJanisAlgorithm}[sec.~5.4]{Whisker:2008:BraneworldBlackHoles} (see also~\cite{Reed:1974:ImaginaryTetradtransformationsEinstein}).


\subsection{Summary}



The goal of the current work is to review a series of recent papers~\cite{Erbin:2015:JanisNewmanAlgorithmSimplifications, Erbin:2015:FivedimensionalJanisNewmanAlgorithm, Erbin:2016:DecipheringGeneralizingDemianskiJanisNewman, Erbin:2015:SupergravityComplexParameters} in which the JN algorithm has been extended in several directions, opening the doors to many new applications.
This review evolved from the thesis of the author~\cite{Erbin:2015:BlackHolesN}, which presented the material from a slightly different perspective, and from lectures given at \textsc{Hri} (Allahabad, India).

As explained in the previous section, the JN algorithm was formulated only for the metric and all other fields had to be found using the equations of motion (with or without using an ansatz).
For example neither the Kerr--Newman gauge field or its associated field strength could be derived in~\cite{Newman:1965:MetricRotatingCharged}.
The solution to this problem is to perform a gauge transformation in order to remove the radial component of the gauge field in null coordinates~\cite{Erbin:2015:JanisNewmanAlgorithmSimplifications}.
It is then straightforward to apply the JN algorithm in either prescription.\footnotemark{}%
\footnotetext{%
	Another solution has been proposed by Keane~\cite{Keane:2014:ExtensionNewmanJanisAlgorithm} but it is applicable only to the Newman--Penrose coefficients of the field strength.
	Our proposal requires less computations and yields directly the gauge field from which all relevant quantities can easily be derived.
}
Another problem was exemplified by the derivation of Sen's axion--dilaton rotating black hole~\cite{Sen:1992:RotatingChargedBlack} by Yazadjiev~\cite{Yazadjiev:2000:NewmanJanisMethodRotating}, who could find the metric and the dilaton, but not the axion (nor the gauge field).
The reason is that while the JN algorithm applies directly to real scalar fields, it does not for complex scalar fields (or for a pair of real fields that can naturally be gathered into a complex scalar).
Then it is necessary to consider the complex scalar as a unique object and to perform the transformation without trying to keep it real~\cite{Erbin:2015:SupergravityComplexParameters}.
Hence this completes the JN algorithm for all bosonic fields with spin less than or equal to two.

A second aspect for which the original form of the algorithm was deficient is configuration with magnetic and NUT charges and in presence of a cosmological constant.
The issue corresponds to finding how one should complexify the functions: the usual rules do not work and if there were no way to obtain the functions by complexification then the JN algorithm would be of limited interest as it could not be exported to other cases (except if one is willing to solve equations of motion, which is not the goal of a solution generating technique).
We have found that to reproduce Demiański's result~\cite{Demianski:1972:NewKerrlikeSpacetime} it is necessary to complexify also the mass and to consider the complex parameter $m + i n$~\cite{Erbin:2016:DecipheringGeneralizingDemianskiJanisNewman, Erbin:2015:SupergravityComplexParameters} and to shift the curvature of the spherical horizon.
Similarly for configurations with magnetic charges one needs to consider the complex charge $q + i p$~\cite{Erbin:2015:SupergravityComplexParameters}.
Such complex combinations are quite natural from the point of view of the Plebański--Demiański solution~\cite{Plebanski:1975:ClassSolutionsEinsteinMaxwell, Plebanski:1976:RotatingChargedUniformly} described previously.
It is to notice that the appearance of complex coordinate transformations mixed with complex parameter transformations was a feature of Quevedo's solution generating technique~\cite{Quevedo:1992:ComplexTransformationsCurvature, Quevedo:1992:DeterminationMetricCurvature}, yet it is unclear what the link with our approach really is.
Hence the final metric obtained from the JN algorithm may contain (for vanishing cosmological constant) five of the six Plebański--Demiański parameters~\cite{Plebanski:1975:ClassSolutionsEinsteinMaxwell, Plebanski:1976:RotatingChargedUniformly} along with Demiański's parameter.

An interesting fact is that the previous argument works in the presence of the cosmological constant only if one considers the possibility of having a generic topological horizons (flat, hyperbolic or spherical) and for this reason we have provided an extension of the formalism to this case~\cite{Erbin:2016:DecipheringGeneralizingDemianskiJanisNewman}.

We also propose a generalization of the algorithm to any dimension~\cite{Erbin:2015:FivedimensionalJanisNewmanAlgorithm}, but while new examples could be found for $d = 5$ the program could not be carried to the end for $d > 5$.


All these results provide a complete framework for most of the theories of gravity that are commonly used.
As a conclusion we summarize the features of our new results:
\begin{itemize}
	\item all bosonic fields with spin $\le 2$;
	\item topological horizons;
	\item charges $m, n, q, p, a$ (with $a$ only for $\Lambda = 0$);
	\item extend to $d = 3, 5$ dimensions (and proposal for higher).
\end{itemize}
We have written a general \emph{Mathematica} package for the JN algorithm in Einstein--Maxwell theory.\footnotemark{}%
\footnotetext{
	Available at \url{http://www.lpthe.jussieu.fr/~erbin/}.
}
Here is a list of new examples that have been completely derived using the previous results (all in $4d$ except when said explicitly):
\begin{itemize}
	\item Kerr--Newman--NUT;
	\item dyonic Kerr--Newman;
	\item Yang--Mills Kerr--Newman black hole~\cite{Perry:1977:BlackHolesAre};
	\item adS--NUT Schwarzschild;
	\item Demiański's solution~\cite{Demianski:1972:NewKerrlikeSpacetime};
	\item ungauged $N = 2$ BPS solutions~\cite{Behrndt:1998:StationarySolutionsN2};
	\item non-extremal solution in $T^3$ model~\cite{Sen:1992:RotatingChargedBlack} (partly derived in~\cite{Yazadjiev:2000:NewmanJanisMethodRotating});
	\item SWIP solutions~\cite{Bergshoeff:1996:StationaryAxionDilatonSolutions};
	\item (a)dS--charged Taub--NUT--BBMB~\cite{Bardoux:2013:IntegrabilityConformallyCoupled};
	\item $5d$ Myers--Perry~\cite{Myers:1986:BlackHolesHigher};
	\item $5d$ BMPV~\cite{Breckenridge:1997:DbranesSpinningBlack};
	\item NUT charged black hole\footnotemark{} in gauged $N = 2$ sugra with $F = - i\, X^0 X^1$~\cite{Gnecchi:2014:RotatingBlackHoles}.
\end{itemize}
\footnotetext{%
	Derived by D.\ Klemm and M.\ Rabbiosi, unpublished work.
}
We also found a more direct derivation of the rotating BTZ black hole (derived in another way by Kim~\cite{Kim:1997:NotesSpinningAdS3, Kim:1999:SpinningBTZBlack}).


\subsection{Outlook}



A major playground for this modified Janis--Newman (JN) algorithm is (gauged) supergravity -- where many interesting solutions remain to be discovered -- since all the necessary ingredients are now present.
Moreover important solutions are still missing in higher-dimensional Einstein--Maxwell (in particular the charged Myers--Perry solution) and one can hope that understanding the JN algorithm in higher dimensions would shed light on this problem.
Another open case is whether black rings can also be derived using the algorithm.

A major question about the JN algorithm is whether it is possible to include rotation for non-vanishing cosmological constant.
A possible related problem concerns the addition of acceleration $\alpha$, which is the only missing parameter when $\Lambda = 0$.
It is indeed puzzling that one could get all Plebański--Demiański parameters but the acceleration, which appears in the combination $a + i \alpha$.
Both problems are linked to the fact that the JN algorithm -- in its current form -- does not take into account various couplings between the parameters (such as the spin with the cosmological constant or the acceleration with the mass in the simplest cases).
On the other hand it does not mean that it is impossible to find a generalization of the algorithm: philosophically the problem is identical to the ones of adding NUT and magnetic charges.


In any case the meaning and a rigorous derivation of the JN algorithm -- perhaps elevating it to the status of a true solution generating algorithm -- are still to be found.
It is also interesting to note that almost all of the examples quoted in the previous section can be embedded into $N = 2$ supergravity.
This calls for a possible interpretation of the algorithm in terms of some hidden symmetry of supergravity, or even of string theory.

We hope that our new extension of the algorithm will help to bring it outside the shadow where it stayed since its creation and to establish it as a standard tool for deriving new solutions in the various theories of gravity.



\subsection{Overview}


In \cref{sec:algo} we review the original Janis--Newman algorithm and its alternative form due to Giampieri before illustrating it with some examples.
\Cref{sec:extension} shows how to extend the algorithm to more complicated set of fields (complex scalars, gauge fields) and parameters (magnetic and NUT charges, topological horizon).
Then \cref{sec:general} provides a general description of the algorithm in its most general form.
The complex transformation described in the previous section are derived in \cref{sec:derivation}.
\Cref{sec:examples} describes several examples.
Finally \cref{sec:five} extends first the algorithm to five dimensions and \cref{sec:higher} generalizes these ideas to any dimension.

\Cref{app:coord} gathers useful formulas on coordinate systems in various numbers of dimensions.
\Cref{app:N=2-sugra} reviews briefly the main properties of $N = 2$ supergravity.
Finally \cref{app:technical-properties} discusses some additional properties of the JN algorithm.

In our conventions the spacetime signature is mostly plus.
