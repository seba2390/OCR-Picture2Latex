\section{Algorithm in any dimension}
\label{sec:higher}


Following the same prescription in dimensions higher than five does not lead as nicely to the exact Myers--Perry solution.
Indeed we show in this section that, while the transformation of the metric can be done along the same line, the -- major -- obstacle comes from the function $f$ that cannot be transformed as expected.
Finding the correct complexification seems very challenging and it may be necessary to use a different complex coordinate transformation in order to perform a completely general transformation in any dimension.
It might be possible to gain insight into this problem by computing the transformation within the framework of the tetrad formalism.
One may think that a possible solution would be to replace complex numbers by quaternions, assigning one angular momentum to each complex direction but it is straightforward to check that this approach is not working.

The key element to perform the algorithm on the metric is to parametrize the metric on the sphere by direction cosines since these coordinates are totally symmetric under permutation of angular momenta (at the opposite of the spherical coordinates).
We are able to derive the general form of a rotating metric with the maximal number of angular momenta it can have in $d$ dimensions, but we are not able to apply this result to any specific example for $d \ge 6$, except if all momenta but one are vanishing.
Nonetheless this provides a unified view of the JN algorithm in any $d \ge 3$.
We conclude this section by few examples, including the singly-rotating Myers--Perry solution in any dimension and the rotating BTZ black hole.

It would be very desirable to derive the general $d$-dimensional Myers--Perry solution~\cite{Myers:1986:BlackHolesHigher}, or at least to understand why only the metric can be found, and not the function inside.


\subsection{Metric transformation}
\label{sec:higher-jna:any-dimension}


We consider the JN algorithm applied to a general static $d$-dimension metric and show how the tensor structure can be transformed.
In the following the dimension is taken to be odd in order to simplify the computations but the final result holds also for $d$ even.



\subsubsection{Seed metric and discussion}


Consider the $d$-dimensional static metric (notations are defined in \cref{app:coord:general-d})
\begin{equation}
	\dd s^2 = - f\, \dd t^2 + f^{-1}\, \dd r^2 + r^2\, \dd \Omega_{d-2}^2
\end{equation} 
where $\dd \Omega_{d-2}^2$ is the metric on $S^{d-2}$
\begin{equation}
	\dd \Omega_{d-2}^2 = \dd\theta_{d-2} + \sin^2 \theta_{d-2}\, \dd \Omega_{d-3}^2
		= \sum_{i=1}^n \big( \dd\mu_i^2 + \mu_i^2 \dd\phi_i^2).
\end{equation} 
The number $n = (d-1) / 2$ counts the independent $2$-spheres.

In Eddington--Finkelstein coordinates the metric reads
\begin{equation}
	\label{higher-jna:higher-jna:metric:static-seed}
	\dd s^2 = (1 - f)\, \dd u^2 - \dd u\, (\dd u + 2 \dd r)
			+ r^2 \sum_i \Big(\dd \mu_i^2 + \mu_i^2\, \dd \phi_i^2 \Big).
\end{equation} 

The metric looks like a $2$-dimensional space $(t, r)$ with a certain number of additional $2$-spheres $(\mu_i, \phi_i)$ which are independent from one another.
Then we can consider only the piece $(u, r, \mu_i, \phi_i)$ (for fixed $i$) which will transform like a $4$-dimensional spacetime, while the other part of the metric $(\mu_j, \phi_j)$ for all $j \neq i$ will be unchanged.
After the first transformation we can move to another $2$-sphere.
We can thus imagine to put in rotation only one of these spheres.
Then we will apply again and again the algorithm until all the spheres have angular momentum: the whole complexification will thus be a $n$-steps process.
Moreover if these $2$-spheres are taken to be independent this implies that we should not complexify the functions that are not associated with the plane we are putting in rotation.

To match these demands the metric is rewritten as
\begin{equation}
	\label{higher-jna:higher-jna:higher-jna:metric:static-seed-ur}
	\dd s^2 = (1 - f)\, \dd u^2 - \dd u\, (\dd u + 2 \dd r_{i_1})
		+ r_{i_1}^2 (\dd\mu_{i_1}^2 + \mu_{i_1}^2 \dd\phi_{i_1}^2)
		+ \sum_{i \neq i_1} \Big(r_{i_1}^2 \dd \mu_i^2 + R^2 \mu_i^2\, \dd \phi_i^2 \Big).
\end{equation} 
where we introduced the following two functions of $r$
\begin{equation}
	r_{i_1}(r) = r, \qquad R(r) = r.
\end{equation} 
This allows to choose different complexifications for the different terms in the metric.
It may be surprising to note that the factors in front of $\dd \mu_i^2$ have been chosen to be $r_{i_1}^2$ and not $R^2$, but the reason is that the $\mu_i$ are all linked by the constraint
\begin{equation}
	\sum_i \mu_i^2 = 1
\end{equation} 
and the transformation of one $i_1$-th $2$-sphere will change the corresponding $\mu_{i_1}$, but also all the others, as it is clear from the formula \eqref{coord:eq:spherical-to-oblate-mu} with all the $a_i$ vanishing but one (this can also be observed in $5d$ where both $\mu_i$ are gathered into $\theta$).


\subsubsection{First transformation}


The transformation is chosen to be
\begin{subequations}
\label{higher-jna:higher-jna:change:jna-1}
\begin{equation}
	r_{i_1} = r'_{i_1} - i\, a_{i_1} \sqrt{1 - \mu_{i_1}^2}, \qquad
	u = u' + i\, a_{i_1} \sqrt{1 - \mu_{i_1}^2}
\end{equation} 
which, together with the ansatz
\begin{equation}
	i\, \frac{\dd \mu_{i_1}}{\sqrt{1 - \mu_{i_1}^2}} = \mu_{i_1}\, \dd \phi_{i_1},
\end{equation} 
gives the differentials
\begin{equation}
	\dd r_{i_1} = \dd r'_{i_1} + a_{i_1} \mu_{i_1}^2\, \dd \phi_{i_1}, \qquad
	\dd u = \dd u' - a_{i_1} \mu_{i_1}^2\, \dd \phi_{i_1}.
\end{equation} 
\end{subequations}
It is easy to check that this transformation reproduces the one given in four and five dimensions.
The complexified version of $f$ is written as $\tilde f^{\{i_1\}}$: we need to keep track of the order in which we gave angular momentum since the function $\tilde f$ will be transformed at each step.

We consider separately the transformation of the $(u, r)$ and $\{ \mu_i, \phi_i \}$ parts.
Inserting the transformations \eqref{higher-jna:higher-jna:change:jna-1} in \eqref{higher-jna:higher-jna:metric:static-seed} results in
\begin{subequations}
\begin{align*}
	\dd s_{u,r}^2 &= (1 - \tilde f^{\{i_1\}})\, \Big(\dd u - a_{i_1} \mu_{i_1}^2\, \dd \phi_{i_1} \Big)^2
		- \dd u\, (\dd u + 2 \dd r_{i_1})
		+ 2 a_{i_1} \mu_{i_1}^2\, \dd r_{i_1} \dd \phi_{i_1}
		+ a_{i_1}^2 \mu_{i_1}^4\, \dd \phi_{i_1}^2, \\
	%
	\dd s_{\mu,\phi}^2 &= \big( r_{i_1}^2 + a_{i_1}^2 \big) (\dd\mu_{i_1}^2 + \mu_{i_1}^2 \dd\phi_{i_1}^2)
		+ \sum_{i \neq i_1} \big( r_{i_1}^2 \dd \mu_i^2 + R^2 \mu_i^2\, \dd \phi_i^2 \big) - a_{i_1}^2 \mu_{i_1}^4\, \dd \phi_{i_1}^2 \\
		&\qquad + a_{i_1}^2 \bigg[- \mu_{i_1}^2 \dd \mu_{i_1}^2 + (1 - \mu_{i_1}^2) \sum_{i \neq i_1} \dd \mu_i^2 \bigg].
\end{align*}
\end{subequations}

The term in the last bracket vanishes as can be seen by using the differential of the constraint
\begin{equation}
	\sum_i \mu_i^2 = 1 \Longrightarrow
	\sum_i \mu_i \dd\mu_i = 0.
\end{equation} 
Since this step is very important and non-trivial we expose the details
\begin{align*}
	[\cdots] &= \mu_{i_1}^2 \dd \mu_{i_1}^2 - (1 - \mu_{i_1}^2) \sum_{i \neq i_1} \dd \mu_i^2
		= \left(\sum_{i \neq i_1} \mu_i \dd\mu_i \right)^2 - \sum_{j \neq i_1} \mu_j^2 \sum_{i \neq i_1} \dd \mu_i^2 \\
		&= \sum_{i,j \neq i_1} \big(\mu_i \mu_j \dd\mu_i \dd\mu_j - \mu_j^2 \dd \mu_i^2 \big)
		= \sum_{i,j \neq i_1} \mu_j \big(\mu_i \dd\mu_j - \mu_j \dd \mu_i \big) \dd\mu_i
		= 0
\end{align*}
by antisymmetry.

Setting $r_{i_1} = R = r$ one obtains the metric
\begin{equation}
\begin{aligned}
	\dd s^2 &= (1 - \tilde f^{\{i_1\}})\, \Big(\dd u - a_{i_1} \mu_{i_1}^2\, \dd \phi_{i_1} \Big)^2
		- \dd u\, (\dd u + 2 \dd r)
		+ 2 a_{i_1} \mu_{i_1}^2\, \dd r \dd \phi_{i_1} \\
		&\qquad+ \big( r^2 + a_{i_1}^2 \big) (\dd\mu_{i_1}^2 + \mu_{i_1}^2 \dd\phi_{i_1}^2)
		+ \sum_{i \neq i_1} r^2 \big( \dd \mu_i^2 + \mu_i^2\, \dd \phi_i^2 \big).
\end{aligned}
\end{equation}
It corresponds to Myers--Perry metric in $d$ dimensions with one non-vanishing angular momentum.
We recover the same structure as in \eqref{higher-jna:higher-jna:higher-jna:metric:static-seed-ur} with some extra terms that are specific to the $i_1$-th $2$-sphere.


\subsubsection{Iteration and final result}


We should now split again $r$ in functions $(r_{i_2}, R)$.
Very similarly to the first time we have
\begin{equation}
\begin{aligned}
	\dd s^2 &= (1 - \tilde f^{\{i_1\}})\, \Big(\dd u - a_{i_1} \mu_{i_1}^2\, \dd \phi_{i_1} \Big)^2
		- \dd u\, (\dd u + 2 \dd r_{i_2})
		+ 2 a_{i_1} \mu_{i_1}^2\, \dd R \dd \phi_{i_1} \\
		&\qquad+ \big( r_{i_2}^2 + a_{i_1}^2 \big) \dd\mu_{i_1}^2
		+ \big( R^2 + a_{i_1}^2 \big) \mu_{i_1}^2 \dd\phi_{i_1}^2
		+ r_{i_2}^2 ( \dd\mu_{i_2}^2 + \mu_{i_2}^2 \dd\phi_{i_2}^2 ) \\
		&\qquad+ \sum_{i \neq i_1, i_2} \Big(r_{i_2}^2 \dd \mu_i^2 + R^2 \mu_i^2\, \dd \phi_i^2 \Big).
\end{aligned}
\end{equation}
We can now complexify as
\begin{equation}
	r_{i_2} = r'_{i_2} - i a_{i_2} \sqrt{1 - \mu_{i_2}^2}, \qquad
	u = u' + i\, a_{i_1} \sqrt{1 - \mu_{i_2}^2}.
\end{equation} 
The steps are exactly the same as before, except that we have some inert terms.
The complexified functions is now $\tilde f^{\{i_1, i_2\}}$.

Repeating the procedure $n$ times we arrive at
\begin{equation}
	\label{higher-jna:metric:rotating:result-jna-ur}
	\begin{aligned}
		\dd s^2 = &- \dd u^2 - 2 \dd u \dd r
			+ \sum_i (r^2 + a_i^2) (\dd \mu_i^2 + \mu_i^2 \dd \phi_i^2)
			- 2 \sum_i a_i \mu_i^2 \, \dd r \dd \phi_i \\
			&+ \Big(1 - \tilde f^{\{i_1, \ldots, i_n\}} \Big) \left(\dd u + \sum_i a_i \mu_i^2 \dd \phi_i \right)^2.
	\end{aligned}
\end{equation} 
One recognizes the general form of the $d$-dimensional metric with $n$ angular momenta~\cite{Myers:1986:BlackHolesHigher}.

Let's quote the metric in Boyer--Lindquist coordinates (omitting the indices on $\tilde f$)~\cite{Myers:1986:BlackHolesHigher}
\begin{equation}
	\label{higher-jna:metric:rotating:result-jna-bl}
	\dd s^2 = - \dd t^2
		+ (1 - \tilde f) \left(\dd t - \sum_i a_i \mu_i^2 \dd \phi_i \right)^2
		+ \frac{r^2 \rho^2}{\Delta}\, \dd r^2
		+ \sum_i (r^2 + a_i^2) \Big(\dd \mu_i^2 + \mu_i^2\, \dd \phi_i^2 \Big)
\end{equation} 
which is obtained from the transformation
\begin{equation}
	\dd u = \dd t - g\, \dd r, \qquad
	\dd \phi_i = \dd \phi'_i - h_i\, \dd r
\end{equation} 
with functions
\begin{equation}
\label{higher-jna:change:rotating:higher-dim-func-gh}
	g = \frac{\Pi}{\Delta}
		= \frac{1}{1 - F (1 - \tilde f)}, \qquad
	h_i = \frac{\Pi}{\Delta} \, \frac{a_i}{r^2 + a_i^2},
\end{equation}
and where the various quantities involved are (see \cref{app:coord:general-d:oblate-cosines})
\begin{equation}
	\label{higher-jna:metric:rotating:result-jna-bl-parameters}
	\begin{gathered}
		\Pi = \prod_i (r^2 + a_i^2), \qquad
		F = 1 - \sum_i \frac{a_i^2 \mu_i^2}{r^2 + a_i^2} = r^2 \sum_i \frac{\mu_i^2}{r^2 + a_i^2}, \\
		r^2 \rho^2 = \Pi F, \qquad
		\Delta = \tilde f\, r^2 \rho^2 + \Pi (1 - F).
	\end{gathered}
\end{equation}

Before ending this section, we comment the case of even dimensions: the term $\varepsilon'\, r^2 \dd \alpha^2$ is complexified as $\varepsilon'\, r_{i_1}^2 \dd \alpha^2$, since it contributes to the sum
\begin{equation}
	\sum_i \mu_i^2 + \alpha^2 = 1.
\end{equation} 
This can be seen more clearly by defining $\mu_{n+1} = \alpha$ (we can also define $\phi_{n+1} = 0$), in which case the index $i$ runs from $1$ to $n+\varepsilon$, and all the previous computations are still valid.


\subsection{Examples in various dimensions}
\label{sec:higher-jna:examples}


\subsubsection{Flat space}


A first and trivial example is to take $f = 1$.
In this case one recovers Minkowski metric in spheroidal coordinates with direction cosines (\cref{app:coord:general-d:oblate-cosines})
\begin{equation}
	\dd s^2 = - \dd t^2 + F\, \dd \bar r^2 + \sum_i (\bar r^2 + a_i^2) \Big(\dd \bar \mu_i^2 + \bar \mu_i^2\, \dd \bar \phi_i^2 \Big) + \varepsilon'\, r^2 \dd \alpha^2.
\end{equation}
In this case the JN algorithm is equivalent to a (true) change of coordinates and there is no intrinsic rotation.
The presence of a non-trivial function $f$ then deforms the algorithm.


\subsubsection{Myers--Perry black hole with one angular momentum}


The derivation of the Myers--Perry metric with one non-vanishing angular momentum has been found by Xu~\cite{Xu:1988:ExactSolutionsEinstein}.

The transformation is taken to be in the first plane
\begin{equation}
	r = r' - i a \sqrt{1 - \mu^2}
\end{equation} 
where $\mu \equiv \mu_1$.
The transformation to the mixed spherical--spheroidal system (\cref{app:coord:general-d:oblate-spherical} is obtained by setting
\begin{equation}
	\mu = \sin \theta, \qquad
	\phi_1 = \phi.
\end{equation} 
In these coordinates the transformation reads
\begin{equation}
	r = r' - i a \cos \theta.
\end{equation} 
We will use the quantity
\begin{equation}
	\rho^2 = r^2 + a^2 (1 - \mu^2)
		= r^2 + a^2 \cos^2 \theta.
\end{equation} 

The Schwarzschild--Tangherlini metric is~\cite{Tangherlini:1963:SchwarzschildFieldDimensions}
\begin{equation}
	\dd s^2 = - f\, \dd t^2 + f^{-1}\, \dd r^2 + r^2\, \dd \Omega_{d-2}^2, \qquad
	f = 1 - \frac{m}{r^{d-3}}.
\end{equation} 

Applying the previous transformation results in
\begin{equation}
\begin{aligned}
	\dd s^2 &= (1 - \tilde f)\, \Big(\dd u - a \mu^2\, \dd \phi \Big)^2
		- \dd u\, (\dd u + 2 \dd r)
		+ 2 a \mu^2\, \dd r \dd \phi \\
		&\qquad+ \big( r^2 + a^2 \big) (\dd\mu^2 + \mu^2 \dd\phi^2)
		+ \sum_{i \neq 1} r^2 \big( \dd \mu_i^2 + \mu_i^2\, \dd \phi_i^2 \big).
\end{aligned}
\end{equation}
where $f$ has been complexified as
\begin{equation}
	\tilde f = 1 - \frac{m}{\rho^2 r^{d-5}}.
\end{equation} 

In the mixed coordinate system one has~\cite{Xu:1988:ExactSolutionsEinstein, Aliev:2006:RotatingBlackHoles}
\begin{equation}
	\begin{aligned}
		\dd s^2 = &- \tilde f\, \dd t^2
			+ 2 a (1 - \tilde f) \sin^2 \theta\, \dd t \dd\phi
			+ \frac{r^{d-3} \rho^2}{\Delta}\, \dd r^2 + \rho^2 \dd\theta^2 \\
			&+ \frac{\Sigma^2}{\rho^2}\, \sin^2 \theta\, \dd\phi^2
			+ r^2 \cos^2 \theta^2\, \dd\Omega_{d-4}^2.
	\end{aligned}
\end{equation} 
where we defined as usual
\begin{equation}
	\Delta = \tilde f \rho^2 + a^2 \sin^2 \theta, \qquad
	\frac{\Sigma^2}{\rho^2} = r^2 + a^2 + a g_{t\phi}.
\end{equation} 
This last expression explains why the transformation is straightforward with one angular momentum: the transformation is exactly the one for $d = 4$ and the extraneous dimensions are just spectators.

We have not been able to generalize this result for several non-vanishing momenta for $d \ge 6$, even for the case with equal momenta .


\subsubsection{Five-dimensional Myers--Perry}
\label{sec:higher-jna:examples:myers-perry-5d}


We take a new look at the five-dimensional Myers--Perry solution in order to derive it in spheroidal coordinates because it is instructive.

The function
\begin{equation}
	1 - f = \frac{m}{r^2}
\end{equation} 
is first complexified as
\begin{equation}
	1 - \tilde f^{\{1\}} = \frac{m}{\abs{r_1}^2}
		= \frac{m}{r^2 + a^2 (1 - \mu^2)}
\end{equation}
and then as 
\begin{equation}
	1 - \tilde f^{\{1, 2\}} = \frac{m}{\abs{r_2}^2 + a^2 (1 - \mu^2)}
		= \frac{m}{r^2 + a^2 (1 - \mu^2) + b^2 (1 - \nu^2)}.
\end{equation}
after the two transformations
\begin{equation}
	r_1 = r_1' - i a \sqrt{1 - \mu^2}, \qquad
	r_2 = r_2' - i b \sqrt{1 - \nu^2}.
\end{equation} 
For $\mu = \sin \theta$ and $\nu = \cos \theta$ one recovers the transformations from \cref{sec:higher-jna:5d:myers-perry,sec:higher-jna:5d:bmpv}.

Let's denote the denominator by $\rho^2$ and compute
\begin{align*}
	\frac{\rho^2}{r^2} &= r^2 + a^2 (1 - \mu^2) + b^2 (1 - \nu^2)
		= (\mu^2 + \nu^2) r^2 + \nu^2 a^2 + \mu^2 b^2 \\
		&= \mu^2 (r^2 + b^2) + \nu^2 (r^2 + a^2)
		= (r^2 + b^2) (r^2 + a^2) \left( \frac{\mu^2}{r^2 + a^2} + \frac{\nu^2}{r^2 + b^2} \right).
\end{align*}
and thus
\begin{equation}
	r^2 \rho^2 = \Pi F.
\end{equation} 
Plugging this into $\tilde f^{\{1, 2\}}$ we have~\cite{Myers:1986:BlackHolesHigher}
\begin{equation}
	1 - \tilde f^{\{1, 2\}} = \frac{m r^2}{\Pi F}.
\end{equation} 


\subsubsection{Three dimensions: BTZ black hole}
\label{sec:higher-jna:examples:btz}


As another application we show how to derive the $d = 3$ rotating BTZ black hole from its static version~\cite{Banados:1992:BlackHoleThree}
\begin{equation}
	\dd s^2 = - f\, \dd t^2 + f^{-1}\, \dd r^2 + r^2 \dd\phi^2, \qquad
	f(r) = - M + \frac{r^2}{\ell^2}.
\end{equation} 

In three dimensions the metric on $S^1$ in spherical coordinates is given by
\begin{equation}
	\dd\Omega_1^2 = \dd\phi^2.
\end{equation} 
Introducing the coordinate $\mu$ we can write it in oblate spheroidal coordinates
\begin{equation}
	\dd\Omega_1^2 = \dd\mu^2 + \mu^2 \dd\phi^2
\end{equation} 
with the constraint
\begin{equation}
	\mu^2 = 1.
\end{equation} 

Application of the transformation
\begin{equation}
	u = u' + i a \sqrt{1 - \mu^2}, \qquad
	r = r' - i a \sqrt{1 - \mu^2}
\end{equation} 
gives from \eqref{higher-jna:metric:rotating:result-jna-ur}
\begin{equation}
	\begin{aligned}
		\dd s^2 = &- \dd u^2 - 2 \dd u \dd r
			+ (r^2 + a^2) (\dd \mu^2 + \mu^2 \dd \phi^2)
			- 2 a \mu^2 \, \dd r \dd \phi \\
			&+ (1 - \tilde f) (\dd u + a \mu^2 \dd \phi )^2.
	\end{aligned}
\end{equation} 
The transformation of $f$ is
\begin{equation}
	\tilde f = - m + \frac{\rho^2}{\ell^2}, \qquad
	\rho^2 = r^2 + a^2 (1 - \mu^2).
\end{equation} 

The transformation \eqref{higher-jna:change:rotating:higher-dim-func-gh}
\begin{equation}
	g = \frac{\rho^2 (1 - \tilde f)}{\Delta}, \qquad
	h = \frac{a}{\Delta}, \qquad
	\Delta = r^2 + a^2 + (\tilde f - 1) \rho^2
\end{equation}
to Boyer--Lindquist coordinates leads to the metric \eqref{higher-jna:metric:rotating:result-jna-bl}
\begin{equation}
	\dd s^2 = - \dd t^2
		+ (1 - \tilde f) (\dd t + a \mu^2 \dd \phi )^2
		+ \frac{\rho^2}{\Delta}\, \dd r^2
		+ (r^2 + a^2) (\dd \mu^2 + \mu^2\, \dd \phi^2 ).
\end{equation} 

Finally the constraint $\mu^2 = 1$ can be used to remove the $\mu$.
In this case one finds
\begin{equation}
	\rho^2 = r^2, \qquad
	\Delta = a^2 + \tilde f r^2
\end{equation}
and the metric simplifies to
\begin{equation}
	\dd s^2 = - \dd t^2
		+ (1 - \tilde f) (\dd t + a \dd \phi )^2
		+ \frac{r^2}{a^2 + r^2 \tilde f}\, \dd r^2
		+ (r^2 + a^2) \dd \phi^2.
\end{equation} 

We define the function
\begin{equation}
	N^2 = \tilde f + \frac{a^2}{r^2} = - M + \frac{r^2}{\ell^2} + \frac{a^2}{r^2}.
\end{equation} 
Then redefining the time variable as~\cite{Kim:1997:NotesSpinningAdS3, Kim:1999:SpinningBTZBlack}
\begin{equation}
	t = t' - a \phi
\end{equation} 
we get (omitting the prime)
\begin{equation}
	\dd s^2 = - N^2 \dd t^2 + N^{-2}\, \dd r^2 + r^2 (N^\phi \dd t + \dd \phi)^2
\end{equation} 
with the angular shift
\begin{equation}
	N^\phi(r) = \frac{a}{r^2}.
\end{equation} 
This is the solution given in~\cite{Banados:1992:BlackHoleThree} with $J = -2a$.

It has already been showed by Kim that the rotating BTZ black hole can be derived through the JN algorithm in a different settings~\cite{Kim:1997:NotesSpinningAdS3, Kim:1999:SpinningBTZBlack}: he views the $d = 3$ solution as the slice $\theta = \pi/2$ of the $d = 4$ solution.
Obviously this is equivalent to our approach: we have seen that $\mu = \sin \theta$ in $d = 4$ (\cref{app:coord:4d}), and the constraint $\mu^2 = 1$ is solved by $\theta = \pi/2$.
Nonetheless our approach is more direct since the result just follows from a suitable choice of coordinates and there are no need for advanced justification.

Starting from the charged BTZ black hole
\begin{equation}
       f(r) = - M + \frac{r^2}{\ell^2} - Q^2 \ln r^2, \qquad
       A = - \frac{Q}{2}\, \ln r^2,
\end{equation} 
it is not possible to find the charged rotating BTZ black hole from~\cites{Clement:1993:ClassicalSolutionsThreedimensional, Clement:1996:SpinningChargedBTZ}[sec.~4.2]{Martinez:2000:ChargedRotatingBlack}: the solution solves Einstein equations, but not the Maxwell ones.
This has been already remarked using another technique in~\cite[app.~B]{Lambert:2014:ConformalSymmetriesGravity}.
It may be possible that a more general ansatz is necessary, following \cref{sec:general} but in $d = 3$.
