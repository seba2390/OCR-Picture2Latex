\section{Derivation of the transformations}
\label{sec:derivation}


The goal of this section is to derive the form \eqref{gen:eq:change:jna-functions-FG} of the possible complex transformations.
This method was first used by Demiański~\cite{Demianski:1972:NewKerrlikeSpacetime} and then generalized in~\cite{Erbin:2016:DecipheringGeneralizingDemianskiJanisNewman}.
The idea is to perform the algorithm in a simple setting (metric with one unknown function and one gauge field), leaving arbitrary the functions $F(\theta)$ and $G(\theta)$ in \eqref{gen:eq:change:jna} and $\tilde f_i$ before solving the equations of motion to determine them.
Then the result can be reinterpreted in terms of rules to get the functions $\tilde f_i$ from $f_i$ (this last part was not discussed in~\cite{Demianski:1972:NewKerrlikeSpacetime}).
This selects the possible complex transformations.
Then one can hope that these transformations will be the most general ones (under the assumptions that are made), and one can use these transformations in other cases without having to solve the equations.
The latter claim can be justified by looking at the equations of motions for more complex examples: even if one cannot find directly a solution, one finds that the same structure persists~\cite{Erbin:2016:DecipheringGeneralizingDemianskiJanisNewman} (this is also motivated by the solutions in~\cite{Krori:1981:ChargedDemianskiMetric, Patel:1988:RadiatingDemianskitypeMetrics}).
Another strength of this approach is to remove the ambiguity of the algorithm since the functions are found from the equations of motion, and this may help when one does not know how to perform precisely the algorithm (for example in higher dimensions, see \cref{sec:higher}).


Another goal of this section is to expose the full technical details of the computations: Demiański's paper~\cite{Demianski:1972:NewKerrlikeSpacetime} is short and results are extremely condensed.
In particular we uncover an underlying assumption on the form of the metric function and we show how this lead to an error an in his formula (14) (already pointed out in~\cite{Quevedo:1992:ComplexTransformationsCurvature}).
A generalization of this hypothesis leads to other equations that we could not solve analytically and which may lead to other complex transformations.

Finally this analysis shows the impossibility to derive the (a)dS--Kerr(--Newman) solutions from the JN algorithm.
As discussed in the previous section generalization of the ansatz may help to avoid this no-go theorem.


\subsection{Setting up the ansatz}
\label{sec:derivation:ansatz}


We first recall the action and equations of motion before describing the ansatz for the metric and gauge fields.
We refer to \cref{sec:general} for the general formulas from which the expressions in this section are derived.


\subsubsection{Action and equations of motion}


The action for Einstein--Maxwell gravity with cosmological constant $\Lambda$ reads
\begin{equation}
	\label{deriv:eq:einstein-maxwell-action}
	S = \int \dd^4 x\; \sqrt{- g} \left( \frac{1}{2 \varkappa^2} (R - 2 \Lambda) - \frac{1}{4}\, F^2 \right),
\end{equation} 
where $\varkappa^2 = 8 \pi G$ is the Einstein coupling constant, $g_{\mu\nu}$ is the metric with Ricci scalar $R$ and $F = \dd A$ is the field strength of the Maxwell field $A_\mu$.
In the rest of this section we will set $\varkappa = 1$.
The corresponding equations of motion (respectively Einstein and Maxwell) are
\begin{equation}
	\label{deriv:eq:einstein-maxwell-eom}
	G_{\mu\nu} + \Lambda g_{\mu\nu} = 2\, T_{\mu\nu}, \qquad
	\grad_\mu F^{\mu\nu} = 0,
\end{equation} 
where energy--momentum tensor for the electromagnetic gauge field $A_\mu$ is
\begin{equation}
	T_{\mu\nu} = F_{\mu\rho} \tens{F}{_\nu^\rho} - \frac{1}{4}\, g_{\mu\nu} F^2.
\end{equation} 


\subsubsection{Seed configuration}


We are interested in the subcase of \eqref{gen:eq:static:metric:tr} where
\begin{equation}
	\label{eq:static-ansatz-one-unknown}
	f_t = f, \qquad
	f_r = f^{-1}, \qquad
	f_\Omega = r^2.
\end{equation} 

The seed configuration is
\begin{subequations}
\label{deriv:eq:static:tr}
\begin{gather}
	\label{deriv:eq:static:metric:tr}
	\dd s^2 = - f(r)\, \dd t^2 + f(r)^{-1}\, \dd r^2 + r^2\, \dd\Omega^2, \\
	\label{deriv:eq:static:vector:tr}
	A = f_A(r)\, \dd t
\end{gather}
\end{subequations}
where we consider spherical and hyperbolic horizons
\begin{equation}
	\dd \Omega^2 = \dd\theta^2 + H(\theta)^2\, \dd \phi^2, \qquad
	H(\theta) =
	\begin{cases}
		\sin \theta & \kappa = 1, \\
		\sinh \theta & \kappa = -1.
	\end{cases}
\end{equation} 
In terms of null coordinates \eqref{gen:eq:change:null} the configuration reads
\begin{subequations}
\label{deriv:eq:static:ur}
\begin{gather}
	\label{deriv:eq:static:metric:ur}
	\dd s^2 = - f\, \dd u^2 - 2\, \dd u \dd r + r^2\, \dd\Omega^2, \\
	\label{deriv:eq:static:vector:ur}
	A = f_A\, \dd u.
\end{gather}
\end{subequations}


\subsubsection{Janis--Newman configuration}


The configuration obtained from the Janis--Newman algorithm with a general transformation \eqref{gen:eq:change:jna}
\begin{equation}
	r = r' + i\, F(\theta), \qquad
	u = u' + i\, G(\theta)
\end{equation}
corresponds to (we omit the primes on the coordinates)
\begin{subequations}
\label{deriv:eq:rotating:ur}
\begin{gather}
	\dd s^2 = - \tilde f\, (\dd u + \alpha\, \dd r + \omega H\, \dd\phi )^2
		+ 2 \beta\, \dd r \dd \phi
		+ \rho^2\, \big(\dd\theta^2 + \sigma^2 H^2\, \dd\phi^2 \big), \\
	A = \tilde f_A\, (\dd u + G' H\, \dd \phi)
\end{gather}
\end{subequations}
where
\begin{equation}
	\rho^2 = r^2 + F^2, \quad
	\omega = G' + \tilde f^{-1}\, F', \quad
	\sigma^2 = 1 + \frac{F'^2}{\tilde f \rho^2}, \quad
	\alpha = \tilde f^{-1}, \quad
	\beta = \tilde f^{-1}\, F' H.
\end{equation} 

The Boyer--Lindquist transformation \eqref{gen:eq:change:bl}
\begin{equation}
	\dd u = \dd t' - g(r) \dd r, \qquad
	\dd \phi = \dd \phi' - h(r) \dd r
\end{equation} 
with functions
\begin{equation}
	g(r) = \frac{\rho^2 - F' G'}{\Delta}, \qquad
	h(r) = \frac{F'}{H \Delta}, \qquad
	\Delta = \tilde f \rho^2\, \sigma^2
\end{equation} 
leads to (omitting the primes on the coordinates)
\begin{subequations}
\label{deriv:eq:rotating:tr}
\begin{gather}
	\dd s^2 = - \tilde f_t\, (\dd t + \omega H\, \dd\phi )^2
		+ \frac{\rho^2}{\Delta}\, \dd r^2
		+ \rho^2\, \big(\dd\theta^2 + \sigma^2 H^2\, \dd\phi^2 \big), \\
	A = \tilde f_A\, \left(\dd t - \frac{\rho^2}{\Delta}\, \dd r + G' H\, \dd \phi \right).
\end{gather}
\end{subequations}


\subsection{Static solution}


It is straightforward to solve the equations \eqref{deriv:eq:einstein-maxwell-eom} for the static configuration \eqref{deriv:eq:static:tr}.

Only the $(t)$ component of Maxwell equations is non trivial
\begin{equation}
	2 f'_A + r f''_A = 0,
\end{equation} 
the prime being a derivative with respect to $r$, and its solution is
\begin{equation}
	f_A(r) = \alpha + \frac{q}{r}
\end{equation} 
where $q$ is a constant of integration that is interpreted as the charge and $\alpha$ is an additional constant that can be removed by a gauge transformation.

The only relevant Einstein equation is
\begin{equation}
	\frac{q^2}{r^2} - \kappa + r^2 \Lambda + f + r f' = 0
\end{equation} 
whose solution reads
\begin{equation}
	\label{eq:topdown-1:static-f}
	f(r) = \kappa - \frac{2m}{r} + \frac{q^2}{r^2} - \frac{\Lambda}{3}\, r^2,
\end{equation} 
$m$ being a constant of integration that is identified to the mass.

We stress that we are just looking for solutions of Einstein equations and we are not concerned with regularity (in particular it is well-known that only $\kappa = 1$ is well-defined for $\Lambda = 0$).

The solution we will find in the next section should reduce to this one upon setting $F, G = 0$.


\subsection{Stationary solution}


Since Boyer--Lindquist imposes additional restrictions on the solutions we will solve the equations of motion \eqref{deriv:eq:einstein-maxwell-eom} for the configuration in null coordinates \eqref{deriv:eq:rotating:ur}.


\subsubsection{Simplifying the equations}
\label{sec:derivation:stationary:simplifying}


The components $(rr)$ and $(r\theta)$ give respectively the equation
\begin{subequations}
\begin{align}
	G'' + \frac{H'}{H}\, G' &= \pm 2 F, \\
	F' \left( G'' + \frac{H'}{H}\, G' \right) &= 2 F F'.
\end{align}
\end{subequations}
If $F' = 0$ then $F$ is an arbitrary constant and the sign of the first equation can be absorbed into its definition.\footnotemark{}%
\footnotetext{%
	In particular all expressions are quadratic in $F$, but only linear in $F'$.
}
On the other hand if $F' \neq 0$ one can simplify by the latter in the second equation and this fixes the sign of the first equation.
Then in both cases the relevant equation reduces to
\begin{equation}
	\label{eq:topdown-1-F-Gd-bis}
	G'' + \frac{H'}{H}\, G' = 2 F,
\end{equation} 
which depends only on $\theta$ and allows to solve for $G$ in terms of $F$.

Integrating the $r$-component of the Maxwell equation gives
\begin{equation}
	\tilde f_A = \frac{q\, r}{r^2 + F^2} + \alpha\, \frac{r^2 - F^2}{r^2 + F^2}.
\end{equation}
The $\theta$-equation reads
\begin{equation}
	\alpha\, F' = 0
\end{equation}
which implies $\alpha = 0$ if $F' \neq 0$.
The $\phi$- and $t$-equations follow from these two equations.
As seen above, $\alpha$ can be removed in the static limit $F \to 0$ and in the rest of this section we consider only the case\footnotemark{}%
\footnotetext{%
	We relax this assumption in \cref{sec:derivation:relaxing:gauge-fields}.
}
\begin{equation}
	\alpha = 0.
\end{equation} 

The $(tr)$ equation contains only $r$-derivatives of $\tilde f$ and can be integrated, giving\footnotemark{}%
\footnotetext{%
	In~\cite{Demianski:1972:NewKerrlikeSpacetime} the last term of $\tilde f$ is missing as pointed out in~\cite{Quevedo:1992:ComplexTransformationsCurvature}.
}
\begin{equation}
	\tilde f = \kappa - \frac{2m r - q^2 + 2 F (\kappa\, F + K)}{r^2 + F^2} - \frac{\Lambda}{3}\, (r^2 + F^2) - \frac{4 \Lambda}{3}\, F^2 + \frac{8 \Lambda}{3}\, \frac{F^4}{r^2 + F^2}
\end{equation} 
where again $m$ is a constant of integration interpreted as the mass and the function $K$ is defined by
\begin{equation}
	2 K = F'' + \frac{H'}{H}\, F'.
\end{equation} 
This implies the equations $(r\phi)$ and $(\theta\theta)$.

As explained below \eqref{gen:eq:complexification-functions} the $\theta$-dependence should be contain in $F(\theta)$ only.
The second term of the function $\tilde f$ contains some lonely $\theta$ from the $H(\theta)$ in the function $K$: this means that they should be compensated by the $F$, and we therefore ask that the sum $\kappa F + K$ be constant\footnotemark{}%
\footnotetext{%
	In \cref{sec:derivation:relaxing:metric-function} we relax this last assumption by allowing non-constant $\kappa F + K$.
	In this context the equations and the function $\tilde f$ are modified and this provides an explanation for the Demiański's error in $\tilde f$ in~\cite{Demianski:1972:NewKerrlikeSpacetime}.
}
\begin{equation}
	\kappa\, F' + K' = 0
	\quad \Longrightarrow \quad
	\kappa\, F + K = \kappa n.
\end{equation} 
The parameter $n$ is interpreted as the NUT charge.

The components $(t\theta)$ and $(\theta\phi)$ give the same equation
\begin{equation}
	\Lambda\, F' = 0.
\end{equation} 

Finally one can check that the last three equations $(tt), (t\phi)$ and $(\phi\phi)$ are satisfied.


\subsubsection{Summary of the equations}


The equations to be solved are
\begin{subequations}
\label{eq:topdown-1}
\begin{align}
	\label{eq:topdown-1-F-Gd}
	2 F &= G'' + \frac{H'}{H}\; G', \\
	\label{eq:topdown-1-Fd-Kd}
	\kappa\, n &= \kappa\, F + K, \\
	\label{eq:topdown-1-lambda}
	0 &= \Lambda F'
\end{align}
and the function $\tilde f$ is
\begin{equation}
	\label{eq:topdown-1-tilde-f}
	\tilde f = \kappa - \frac{2m r - q^2 + 2 F (\kappa\, F + K)}{r^2 + F^2} - \frac{\Lambda}{3}\, (r^2 + F^2) - \frac{4 \Lambda}{3}\, F^2 + \frac{8 \Lambda}{3}\, \frac{F^4}{r^2 + F^2}.
\end{equation}
We also defined
\begin{equation}
	\label{eq:topdown-1-K-Fd}
	2 K = F'' + \frac{H'}{H}\, F'.
\end{equation} 
\end{subequations}

As explained in the introduction the second step will be to explain \eqref{eq:topdown-1-tilde-f} in terms of new rules for the algorithm: they have been found in~\cite{Erbin:2016:DecipheringGeneralizingDemianskiJanisNewman} and this was the topic of \cref{sec:general:jna}.

In the next subsections we solve explicitly the equations \eqref{eq:topdown-1} in both cases $\Lambda \neq 0$ and $\Lambda = 0$.


\subsubsection{Solution for \texorpdfstring{$\Lambda \neq 0$}{non-vanishing cosmological constant}}


Equation \eqref{eq:topdown-1-lambda} implies that $F' = 0$, from which $K = 0$ follows by definition; then one obtains
\begin{equation}
	F(\theta) = n
\end{equation} 
by compatibility with \eqref{eq:topdown-1-Fd-Kd} and since $K(\theta) = 0$.

Solution to \eqref{eq:topdown-1-F-Gd} is
\begin{equation}
	G(\theta) = c_1 - 2 \kappa\, n \ln H(\theta) + c_2 \ln \frac{H(\theta/2)}{H'(\theta/2)}
\end{equation} 
where $c_1$ and $c_2$ are two constants of integration.
Since only $G'$ appears in the metric we can set $c_1 = 0$.
On the other hand the constant $c_2$ can be removed by the transformation
\begin{equation}
	\dd u = \dd u' - c_2\, \dd\phi
\end{equation} 
since one has
\begin{equation}
	\left( \ln \frac{H(\theta/2)}{H'(\theta/2)} \right)' = \frac{1}{H(\theta)}.
\end{equation} 

The solution to the system \eqref{eq:topdown-1} is thus
\begin{equation}
	F(\theta) = n, \qquad
	G(\theta) = - 2 \kappa\, n \ln H(\theta).
\end{equation} 
The function $\tilde f$ then takes the form
\begin{equation}
	\label{eq:topdown-1:tilde-f-lambda}
	\tilde f = \kappa - \frac{2m r - q^2 + 2 \kappa n^2}{r^2 + n^2} - \frac{\Lambda}{3}\,\frac{r^4 + 6 n^2 r^2 - 3 n^4}{r^2 + n^2}.
	% \kappa - \frac{2m r - q^2 + 2 \kappa n^2}{r^2 + n^2} - \frac{\Lambda}{3} (r^2 + 5 n^2) + \frac{8 \Lambda}{3}\, \frac{n^4}{r^2 + n^2}
\end{equation} 
This corresponds to the (a)dS--Schwarzschild--NUT solution: compare with \eqref{ext:eq:nut-tilde-f} and \eqref{gen:eq:rotating:tr-F-cst}.

The parameter $\Delta$ in the BL transformation \eqref{gen:eq:change:bl:delta} is
\begin{equation}
	\Delta = \kappa r^2 - 2 m r + q^2 + \Lambda n^4 - \frac{\Lambda}{3}\, r^4 - n^2 (\kappa + 2 \Lambda r^2 ).
\end{equation} 

As noted by Demiański the only parameters that appear are the mass and the NUT charge, and it is not possible to add angular momentum for non-vanishing cosmological constant.\footnotemark{}%
\footnotetext{%
	In~\cite{Leigh:2014:GerochGroupEinstein} Leigh et al.\ generalized Geroch's solution generating technique and also found that only the mass and the NUT charge appear when $\Lambda \neq 0$. We would like to thank D.\ Klemm for this remark.
}
As a consequence the JN algorithm cannot provide a derivation of the (a)dS--Kerr--Newman solution.


\subsubsection{Solution for \texorpdfstring{$\Lambda = 0$}{vanishing cosmological constant}}
\label{sec:derivation:stationary:solution-no-cosmo}


The solution to the differential equation \eqref{eq:topdown-1-Fd-Kd} is
\begin{equation}
	F(\theta) = n - a\, H'(\theta) + c \left( 1 + H'(\theta)\, \ln \frac{H(\theta/2)}{H'(\theta/2)} \right)
\end{equation}
where $a$ and $c$ denote two constants of integration.

We solve the equation \eqref{eq:topdown-1-F-Gd} for $G$
\begin{equation}
	\begin{aligned}
		G(\theta) = c_1 &+ \kappa\, a\, H'(\theta)
			- \kappa\, c\, H'(\theta)\, \ln \frac{H(\theta/2)}{H'(\theta/2)}
			- 2 \kappa\, n \ln H(\theta) \\
			&+ (a + c_2) \ln \frac{H(\theta/2)}{H'(\theta/2)}
	\end{aligned}
\end{equation} 
and $c_1, c_2$ are constants of integration.
Again since only $G'$ appears in the metric we can set $c_1 = 0$.
We can also remove the last term with the transformation
\begin{equation}
	\dd u = \dd u' - (c_2 + a) \dd\phi.
\end{equation} 
One finally gets
\begin{subequations}
\begin{align}
	F(\theta) &= n - a\, H'(\theta) + c \left( 1 + H'(\theta)\, \ln \frac{H(\theta/2)}{H'(\theta/2)} \right), \\
	G(\theta) &= \kappa\, a\, H'(\theta)
		- \kappa\, c\, H'(\theta)\, \ln \frac{H(\theta/2)}{H'(\theta/2)}
		- 2 \kappa\, n \ln H(\theta).
\end{align}
\end{subequations}

This solution was already found in~\cite{Krori:1981:ChargedDemianskiMetric} for the case $\kappa = 1$ by solving directly Einstein--Maxwell equations, starting with a metric ansatz of the form \eqref{deriv:eq:rotating:ur}.
Our aim was to show that the same solution can be obtained by applying Demiański's method to all the quantities, including the gauge field.

The BL transformation is well defined only for $c = 0$, in which case
\begin{equation}
	g = \frac{r^2 + a^2 + n^2}{\Delta}, \qquad
	h = \frac{\kappa a}{\Delta}, \qquad
	\Delta = \kappa r^2 - 2 m r + q^2 - \kappa n^2 + \kappa a^2.
\end{equation} 
The function $\tilde f$ reads
\begin{equation}
	\label{eq:topdown-1:tilde-f-no-Lambda-no-c}
	\tilde f = \kappa - \frac{2 m r - q^2}{\rho^2} + \frac{\kappa\, n (n - a H')}{\rho^2}, \qquad
	\rho^2 = r^2 + (n - a\, H')^2
\end{equation} 
and this corresponds to the Kerr--Newman--NUT solution~\cite[sec.~2.2]{AlonsoAlberca:2000:SupersymmetryTopologicalKerrNewmannTaubNUTaDS}.


\subsection{Relaxing assumptions}
\label{sec:derivation:relaxing}


In the derivation of \cref{sec:derivation:stationary:simplifying} we have made two assumptions in order to recover the simplest case.
The goal of this section is to show how these assumptions can be lifted, even if this does not lead to useful results: one cannot solve the equations in one case while in the other it is not clear how to recast the result in terms of a complex transformation.


\subsubsection{Metric function \texorpdfstring{$F$}{F}-dependence}
\label{sec:derivation:relaxing:metric-function}


In \cref{sec:derivation:stationary:simplifying} we obtained the equation \eqref{eq:topdown-1-Fd-Kd}
\begin{equation}
	\kappa\, F + K = \kappa\, n, \qquad
	2 K = F'' + \frac{H'}{H}\, F'
\end{equation}
by requiring that the function \eqref{eq:topdown-1-tilde-f}
\begin{equation}
	\tilde f = \kappa - \frac{2m r - q^2 + 2 F (\kappa\, F + K)}{r^2 + F^2} - \frac{\Lambda}{3}\, (r^2 + F^2) - \frac{4 \Lambda}{3}\, F^2 + \frac{8 \Lambda}{3}\, \frac{F^4}{r^2 + F^2}
\end{equation} 
depends on $\theta$ only through $F(\theta)$.
A more general assumption would be that $\kappa F + K$ is some function $\chi = \chi(F)$
\begin{equation}
	\label{eq:topdown-1-Fd-Kd-chi}
	\kappa\, F + K = \kappa\, \chi(F).
\end{equation} 
First if $F' = 0$ then $K = 0$ and the definition of $K$ implies
\begin{equation}
	\chi = F = n.
\end{equation} 
The $(t\theta)$- and $(\theta\phi)$-components give the equation
\begin{equation}
	4 \Lambda\, F^2 F' = F'\, \pd_F \chi.
\end{equation} 

If $\Lambda = 0$ we find that
\begin{equation}
	\pd_F \chi = 0
	\Longrightarrow
	\chi = n
\end{equation} 
which reduces to the case studied in \cref{sec:derivation:stationary:simplifying}, while if $F' = 0$ this equation does not provide anything.

On the other hand if $F' \neq 0$ and $\Lambda \neq 0$ then the previous equation becomes
\begin{equation}
	\pd_F \chi = 4 \Lambda F^2
\end{equation} 
which can be integrated to
\begin{equation}
	\label{top-down:eq:chi-F-solution}
	\chi(F) = n + \frac{4}{3}\, \Lambda F^3
\end{equation} 
(notice that the limit $\Lambda \to 0$ is coherent).
Plugging this function into equation \eqref{eq:topdown-1-Fd-Kd-chi} one obtains
\begin{equation}
	\label{eq:topdown-1-Fd-Kd-chi-replaced}
	\kappa\, F + K = \kappa \left(n + \frac{4}{3}\, \Lambda F^3 \right)
\end{equation} 
(remember that $F' \neq 0$).
This differential equation is non-linear and we were not able to find an analytical solution.
Despite that this provides a generalization of the algorithm with non-constant $F$ in the presence of a cosmological constant this is not sufficient for obtaining (a)dS--Kerr: the form of $g_{\theta\theta}$ given in \eqref{deriv:eq:rotating:tr} is not the required one.

Nonetheless by inserting the expression of $\chi$ in $\tilde f$ we see that the last term is killed
\begin{equation}
	\tilde f = \kappa - \frac{2m r - q^2 + 2 \kappa\, n\, F}{r^2 + F^2} - \frac{\Lambda}{3}\, (r^2 + F^2) - \frac{4 \Lambda}{3}\, F^2.
\end{equation} 
One can recognize the function given by Demiański~\cite{Demianski:1972:NewKerrlikeSpacetime} and may explain his error.


\subsubsection{Gauge field integration constant}
\label{sec:derivation:relaxing:gauge-fields}


In \cref{sec:derivation:stationary:simplifying} we obtained a second integration constant $\alpha$ in the expression of the gauge field
\begin{equation}
	\tilde f_A = \frac{q\, r}{r^2 + F^2} + \alpha\, \frac{r^2 - F^2}{r^2 + F^2}.
\end{equation}
One of the Maxwell equation gives $\alpha = 0$ if $F' \neq 0$, but otherwise no equation fixes its value.
For this reason we focus on the case $F' = 0$ or equivalently $\Lambda \neq 0$ through equation \eqref{eq:topdown-1-lambda}.

In this case the function $\tilde f$ is modified to
\begin{equation}
	\tilde f = \kappa - \frac{2m r - q^2 + 2 F (\kappa\, F + K) + 4 \alpha^2 F^2}{r^2 + F^2} - \frac{\Lambda}{3}\, (r^2 + F^2) - \frac{4 \Lambda}{3}\, F^2 + \frac{8 \Lambda}{3}\, \frac{F^4}{r^2 + F^2}.
\end{equation} 
Equation \eqref{eq:topdown-1-lambda} is modified but it is still solved by $F' = 0$ and all other equations are left unchanged (in particular $\kappa F + K$ is still given by the function $\chi(F)$ \eqref{top-down:eq:chi-F-solution}).
For $\chi(F) = n$ the configuration with $\alpha \neq 0$ provides another solution when $\Lambda \neq 0$ but it is not clear how to get it from a complexification of the function.





