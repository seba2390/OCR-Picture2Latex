\section{Complete algorithm}
\label{sec:general}


In this section we gather all the facts on the Janis--Newman algorithm and we explain how to apply it to a general setting.
We write the formulas corresponding to the most general configurations that can be obtained.
We insist again on the fact that all these results can also be derived from the tetrad formalism.


\subsection{Seed configuration}
\label{sec:general:seed}


We consider a general configuration with a metric $g_{\mu\nu}$, gauge fields $A_\mu^I$, complex scalar fields $\tau^i$ and real scalar fields $q^u$.
The initial parameters of the seed configuration are the mass $m$, electric charges $q^I$, magnetic charges $p^i$ and some other parameters $\lambda^A$ (such as the scalar charges).
The electric and magnetic charges are grouped in complex parameters
\begin{equation}
	Z^I = q^I + i p^I.
\end{equation} 
All indices run over some arbitrary ranges.

The seed configuration is spherically symmetric and in particular all the fields depend only on the radial direction $r$
\begin{subequations}
\label{gen:eq:static:tr}
\begin{gather}
	\label{gen:eq:static:metric:tr}
	\dd s^2 = - f_t(r)\, \dd t^2 + f_r(r)\, \dd r^2 + f_\Omega(r)\, \dd\Omega^2, \\
	A^I = f^I(r)\, \dd t + p^I H'(\theta)\, \dd\phi, \\
	\tau^i = \tau^i(r), \qquad
	q^u = q^u(r)
\end{gather}
\end{subequations}
where
\begin{equation}
	\dd \Omega^2 = \dd\theta^2 + H(\theta)^2\, \dd \phi^2, \qquad
	H(\theta) =
	\begin{cases}
		\sin \theta & \kappa = 1 \quad (S^2), \\
		\sinh \theta & \kappa = -1 \quad (H^2).
	\end{cases}
\end{equation} 
Note that only two functions in the metric are relevant since the last one can be fixed through a diffeomorphism.
All the real functions are denoted collectively by
\begin{equation}
	f_i = \{ f_t, f_r, f_\Omega, f^I, q^u \}.
\end{equation} 

The transformation to null coordinates is
\begin{equation}
	\label{gen:eq:change:null}
	\dd t = \dd u - \sqrt{\frac{f_r}{f_t}}\, \dd r
\end{equation} 
and yields
\begin{subequations}
\label{gen:eq:static:ur}
\begin{gather}
	\label{gen:eq:static:metric:ur}
	\dd s^2 = - f_t\, \dd u^2 - 2 \sqrt{f_t f_r}\, \dd r^2 + f_\Omega\, \dd\Omega^2, \\
	A^I = f^I\, \dd u + p^I H'\, \dd\phi
\end{gather}
\end{subequations}
where the radial component of the gauge field
\begin{equation}
	A^I_r = f^I \sqrt{\frac{f_r}{f_t}}
\end{equation} 
has been set to zero through a gauge transformation.


\subsection{Janis--Newman algorithm}
\label{sec:general:jna}


\subsubsection{Complex transformation}


One performs the complex change of coordinates
\begin{equation}
	\label{gen:eq:change:jna}
	r = r' + i\, F(\theta), \qquad
	u = u' + i\, G(\theta).
\end{equation}
In the case of topological horizons the Giampieri ansatz \eqref{algo:eq:giampieri-ansatz} generalizes to
\begin{equation}
	\label{gen:eq:giampieri-ansatz}
	i\, \dd \theta = H(\theta)\, \dd \phi
\end{equation} 
leading to the differentials
\begin{equation}
	\dd r = \dd r' + F'(\theta) H(\theta)\, \dd \phi, \qquad
	\dd u = \dd u' + G'(\theta) H(\theta)\, \dd \phi.
\end{equation} 
The ansatz \eqref{gen:eq:giampieri-ansatz} is a direct consequence of the fact that the $2$-dimensional slice $(\theta, \phi)$ is given by $\dd \Omega^2 = \dd\theta^2 + H(\theta)^2\, \dd \phi^2$, such that the function in the RHS of \eqref{gen:eq:giampieri-ansatz} corresponds to $\sqrt{g^\Omega_{\phi\phi}}$ (where $g$ is the static metric), as can be seen by doing the computation with $i\, \dd \theta = \mc H(\theta) \dd\phi$ and identifying $\mc H = H$ at the end.

The most general known transformation is
\begin{subequations}
\begin{gather}
	\label{gen:eq:change:jna-functions-FG}
	F(\theta) = n - a\, H'(\theta) + c \left( 1 + H'(\theta)\, \ln \frac{H(\theta/2)}{H'(\theta/2)} \right), \\
	G(\theta) = \kappa a\, H'(\theta)
		- 2 \kappa n \ln H(\theta)
		- \kappa c\, H'(\theta)\, \ln \frac{H(\theta/2)}{H'(\theta/2)}, \\
	m = m' + i \kappa n, \\
	\kappa = \kappa' - \frac{4\Lambda}{3}\, n^2,
\end{gather}
\end{subequations}
where $a, c \neq 0$ only if $\Lambda = 0$ (see \cref{sec:derivation} for the derivation).
The mass that is transformed is the physical mass: even if it written in terms of other parameters one should identify it and transform it.

The parameters $a$ and $n$ correspond respectively to the angular momentum and to the NUT charge.
On the other hand the constant $c$ did not receive any clear interpretation (see for example~\cites{Demianski:1972:NewKerrlikeSpacetime, Adamo:2014:KerrNewmanMetricReview}[sec.~5.3]{Krasinski:2006:InhomogeneousCosmologicalModels}).
It can be noted that the solution is of type II in Petrov classification (and thus the JN algorithm \emph{can} change the Petrov type) and it corresponds to a wire singularity on the rotation axis.
Moreover the BL transformation is not well-defined.


\subsubsection{Function transformation}
\label{sec:general:jna:functions}


All the real functions $f_i = f_i(r)$ must be modified to be kept real once $r \in \C$
\begin{equation}
	\label{gen:eq:complexification-functions}
	\tilde f_i = \tilde f_i(r, \bar r)
		= \tilde f_i\big(r', F(\theta) \big) \in \R.
\end{equation} 
The last equality means that $\tilde f_i$ can depend on $\theta$ only through $\Im r = F(\theta)$.
The condition that one recovers the seed for $\bar r = r = r'$ is
\begin{equation}
	\tilde f_i(r', 0) = f_i(r').
\end{equation} 

If all magnetic charges are vanishing or in terms without electromagnetic charges the rules for finding the $\tilde f_i$ are
\begin{subequations}
\label{gen:eq:rules}
\begin{align}
	\label{gen:eq:rules:r}
	r & \longrightarrow \frac{1}{2} (r + \bar r) = \Re r, \\
	\label{gen:eq:rules:1/r}
	\frac{1}{r} & \longrightarrow \frac{1}{2} \left(\frac{1}{r} + \frac{1}{\bar r}\right) = \frac{\Re r}{\abs{r}^2}, \\
	\label{gen:eq:rules:r2}
	r^2 & \longrightarrow \abs{r}^2.
\end{align}
\end{subequations}
Up to quadratic powers of $r$ and $r^{-1}$ these rules determine almost uniquely the result.
This is not anymore the case when the configurations involve higher power.
These can be dealt with by splitting it in lower powers: generically one should try to factorize the expression into at most quadratic pieces.
Some examples of this with natural guesses are
\begin{equation}
	r^4 - b^2 = (r^2 + b) (r^2 - b), \qquad
	r^4 + b = r^2 \left( r^2 + \frac{b}{r^2} \right).
\end{equation} 
Moreover the same power of $r$ can be transformed differently, for example
\begin{equation}
	\frac{1}{r^n} \longrightarrow \frac{1}{r^{n-2}}\, \frac{1}{\abs{r}^2}.
\end{equation} 

Denoting by $Q(r)$ and $P(r)$ collectively all functions that multiply $q^I$ and $p^I$ respectively, all such terms should be rewritten as
\begin{equation}
	\Big( q^I Q(r), p^I P(r) \Big) = \Big( \Re\big(Z^I Q(r)\big), \Im\big(Z^I P(r)\big) \Big)
\end{equation} 
before performing the transformation \eqref{gen:eq:change:jna}.
Note that in this case one does not use the rules \eqref{gen:eq:rules}.

Finally the transformed complex scalars are obtained by simply plugging \eqref{gen:eq:change:jna}
\begin{equation}
	\tau'^i(r', \theta) = \tau^i\big(r + i F(\theta)\big).
\end{equation} 


\subsubsection{Null coordinates}


Plugging the transformation \eqref{gen:eq:change:jna} inside the seed metric and gauge fields \eqref{gen:eq:static:ur} leads to\footnotemark{}%
\footnotetext{%
	We stress that at this stage these formula do not satisfy Einstein equations, they are just proxies to simplify later computations.
}
\begin{subequations}
\label{gen:eq:rotating:ur}
\begin{gather}
	\dd s^2 = - \tilde f_t\, (\dd u' + \alpha\, \dd r' + \omega H\, \dd\phi )^2
		+ 2 \beta\, \dd r' \dd \phi
		+ \tilde f_\Omega\, \big(\dd\theta^2 + \sigma^2 H^2\, \dd\phi^2 \big), \\
	A^I = \tilde f^I\, (\dd u' + G' H\, \dd \phi) + p^I H'\, \dd\phi
\end{gather}
\end{subequations}
where (one should not confuse the primes to indicate derivatives from the primes on the coordinates)
\begin{equation}
	\omega = G' + \sqrt{\frac{\tilde f_r}{\tilde f_t}}\, F', \qquad
	\sigma^2 = 1 + \frac{\tilde f_r}{\tilde f_\Omega}\, F'^2, \qquad
	\alpha = \sqrt{\frac{\tilde f_r}{\tilde f_t}}, \qquad
	\beta = \tilde f_r\, F' H.
\end{equation} 


\subsubsection{Boyer--Lindquist coordinates}


The Boyer--Lindquist transformation
\begin{equation}
	\label{gen:eq:change:bl}
	\dd u' = \dd t' - g(r') \dd r', \qquad
	\dd \phi = \dd \phi' - h(r') \dd r',
\end{equation} 
can be used to remove the off-diagonal $tr$ and $r\phi$ components of the metric
\begin{equation}
	g_{t'r'} = g_{r'\phi'} = 0.
\end{equation} 
The solution to these equations is
\begin{equation}
	\label{gen:eq:change:bl:solution-gh}
	g(r') = \frac{\sqrt{\big(\tilde f_t \tilde f_r \big)^{-1}}\, \tilde f_\Omega - F' G'}{\Delta}, \qquad
	h(r') = \frac{F'}{H \Delta}
\end{equation} 
where
\begin{equation}
	\label{gen:eq:change:bl:delta}
	\Delta = \frac{\tilde f_\Omega}{\tilde f_r}\, \sigma^2
		= \frac{\tilde f_\Omega}{\tilde f_r} + F'^2.
\end{equation} 
Remember that the changes of coordinate is valid only if $g$ and $h$ are functions of $r'$ only.

Inserting \eqref{gen:eq:change:bl:solution-gh} into \eqref{gen:eq:rotating:ur} yields
\begin{subequations}
\label{gen:eq:rotating:tr}
\begin{gather}
	\dd s^2 = - \tilde f_t\, (\dd t' + \omega H\, \dd\phi' )^2
		+ \frac{\tilde f_\Omega}{\Delta}\, \dd r'^2
		+ \tilde f_\Omega\, \big(\dd\theta^2 + \sigma^2 H^2\, \dd\phi'^2 \big), \\
	A^I = \tilde f^I\, \left(\dd t' - \frac{\tilde f_\Omega}{\Delta \sqrt{\tilde f_t \tilde f_r}}\, \dd r' + G' H\, \dd \phi' \right) + p^I H'\, \dd\phi'
\end{gather}
\end{subequations}
where we recall that
\begin{equation}
	\omega = G' + \sqrt{\frac{\tilde f_r}{\tilde f_t}}\, F', \qquad
	\sigma^2 = 1 + \frac{\tilde f_r}{\tilde f_\Omega}\, F'^2.
\end{equation} 
Generically one finds $A_r = A_r(r)$ which can be set to zero thanks to a gauge transformation.

Before closing this section we simplify the above formulas for few simple cases that are often used.


\paragraph{Degenerate Schwarzschild seed}

A degenerate seed (one unknown function) in Schwarzschild coordinates has
\begin{equation}
	f_r = f_t^{-1}, \qquad
	f_\Omega = r^2.
\end{equation} 
The above formulas for this case can be found in \cref{sec:derivation:ansatz}.


\paragraph{Degenerate isotropic seed}

A degenerate seed in isotropic coordinates has
\begin{equation}
	f_t = f^{-1}, \qquad
	f_r = f, \qquad
	f_\Omega = r^2 f.
\end{equation} 
In this case the above formulas reduced to
\begin{subequations}
\label{gen:eq:rotating:tr-degenerate-isotropic}
\begin{gather}
	\dd s^2 = - \tilde f^{-1}\, (\dd t + \omega H\, \dd\phi )^2
		+ \tilde f \rho^2 \left( \frac{\dd r^2}{\Delta}
			+ \dd\theta^2 + \sigma^2 H^2\, \dd\phi^2 \right), \\
	A^I = \tilde f^I\, \left(\dd t - \frac{\tilde f \rho^2}{\Delta}\, \dd r + G' H\, \dd \phi \right) + p^I H'\, \dd\phi
\end{gather}
\end{subequations}
where we recall that
\begin{equation}
	\omega = G' + \tilde f\, F', \qquad
	\sigma^2 = 1 + \frac{F'^2}{\rho^2}, \qquad
	\Delta = \tilde f \rho^2 + F'^2.
\end{equation} 

\paragraph{Constant $F$}

The expressions simplify greatly if $F' = 0$ (for example when $\Lambda \neq 0$).
First all functions depend only on $r$ since $F(\theta) = \cst$
\begin{equation}
	\tilde f_i(r, \theta) = \tilde f_i(r, 0).
\end{equation} 
As a consequence the Boyer--Lindquist transformation \eqref{gen:eq:change:bl:solution-gh}
\begin{equation}
	g(r') = \sqrt{\frac{\tilde f_r}{\tilde f_t}}, \qquad
	h(r') = 0
\end{equation} 
is always well-defined.
For the same reason it is always possible to perform a gauge transformation.
Finally the metric and gauge fields \eqref{gen:eq:rotating:tr} becomes
\begin{subequations}
\label{gen:eq:rotating:tr-F-cst}
\begin{gather}
	\dd s^2 = - \tilde f_t \big(\dd t + G' H\, \dd\phi \big)^2
		+ \tilde f_r\, \dd r^2
		+ \tilde f_\Omega\, \dd\Omega^2, \\
	A^I = \tilde f^I\, \left(\dd t' + G' H\, \dd \phi' \right) + p^I H'\, \dd\phi'.
\end{gather}
\end{subequations}


\subsection{Open questions}


The algorithm we have described help to work with five (four if $\Lambda \neq 0$) of the six parameters of the Plebański--Demiański (PD) solution.
It is tempting to conjecture that it can be extended to the full set of parameters by generalizing the ideas described in \cref{sec:extension:nut} (shifting $\kappa$, writing $a + i \alpha$…).
Indeed we have found that these operations were quite natural in the context of the  PD solution and inspiration could be found in~\cite{Griffiths:2006:NewLookPlebanskiDemianski}.
