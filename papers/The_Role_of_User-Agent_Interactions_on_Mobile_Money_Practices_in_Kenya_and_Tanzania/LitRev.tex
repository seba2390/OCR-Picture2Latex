\section{Related Work}
\label{Sec:related work}
We discuss two strands of prior work related to this study: adoption and use of MoMo, and privacy and security of digital financial services.  

\subsection{Adoption of mobile money}
Some studies have identified the barriers to the use of MoMo, most of which mirror the top barriers to account ownership at a formal financial institution. These include cost, unreliable mobile networks and power grids, preference for cash, lack of documentation for KYC requirements, and low literacy---including digital literacy skills \cite{awanis2022state} \cite{demirgucc2022global}. These studies also suggest that although perceived trust of MoMo contributes to its use among users, lack of trust is one of the main barriers cited by non-users \cite{akinyemi2020determinants} \cite{awanis2022state}.

Similarly, technology adoption studies (e.g.,\cite{baganzi2017examining},\cite{abdul2019customers}, \cite{okello2020trust}, \cite{noreen2021impact}, \cite{osakwe2022trust})  have investigated the factors that predict uptake and use of MoMo. Most of these studies are  quantitative investigations that aim to establish antecedents to adoption. The findings in these studies unanimously emphasize the importance of trust as a significant predictor of adoption. Osakwe et al., \cite{osakwe2022trust} further explore factors that can increase trust in MoMo and offer four specific  institutional measures that can support initial trust building. One of the factors is perceived reputation of both the MoMo agents and the MoMo provider. While these studies have been useful in elucidating adoption, none of them explore the interactions that arise from trust between the agents and users, either pre-adoption or post-adoption
\cite{benbasat2005trust} \cite{sowon2020trust}. This trust often influences continued use.

Post adoption, users tend to develop technology coping strategies in response to technology challenges. Other researchers have studied this in relation to technologies like mobile apps, mobile health, and tablets \cite{inal2019users} \cite{sowon2022information} \cite{zamani2021accommodating}. In general, these studies show that users can cope with technology by abandoning it (the flight response) or accommodating if (the fight response). The concept of workarounds that we use to describe alternative workflows in this study can be considered a fight response.
  
\subsection{Mobile financial system privacy and security}
The increased uptake of MoMo has been associated with an increase in fraud that takes advantage of security vulnerabilities in MoMo technologies \cite{reaves2017mo}. Some of the fraud cases center on the role of agents. In 2020, Uganda suspended its MoMo transactions on its network after hackers breached the payment system, resulting in a loss of over \$3 million  \cite{kafeero_2020}. The attack was perpetrated by fraudsters who worked in collaboration with MoMo agents. Researchers have also analyzed the security of MoMo in Uganda and identified the potential sources of security risks \cite{ali2020evaluation}. 
While our study touches on the role of fraud in MoMo users' decision-making,  it is only a small component of our study. More broadly, we want to understand how user beliefs and perceptions affect their interactions with agents and the MoMo system at large, whether those perceptions relate to fraud or not.

Similarly, several research studies have explored MoMo privacy issues, reviewing privacy policies and relevant regulations   \cite{bowers2017regulators} \cite{harris2012privacy}. MoMo is a data-rich environment, and prior work has noted various aspects of the transaction process that may raise privacy concerns \cite{makulilo2015privacy}. These include the identification of subscribers through SIM registration, KYC performed by agents, and the multiple roles that agents play that make them data depositors. The interaction between various stakeholders also introduces layers of complexity for privacy and security in MoMo environments \cite{mogaji2022dark}. Other studies have focused on privacy concerns with related services such as mobile loan apps \cite{munyendo2022desperate} and some have explored the role of human intermediaries on access to digital services. For example, in Kenya,  cybercafe owners sometimes performed services for their patrons such as logging in~\cite{munyendoeighty}.  While addressing privacy, these studies do not specifically evaluate the user-agent interaction. 

In the work most closely related to ours, Mogaji and Nguyen \cite{mogaji2022dark} explore the ``dark side" of MoMo interactions between stakeholders in Nigeria. 
Their study touches on user-agent relationships, from which they observe several similar patterns to those in our work, including unofficial fees from agents and third-party data sharing. 
The most important difference between their work and ours is that we study the user-agent relationship in depth from the perspective of users in Kenya and Tanzania, whereas their work considers multiple stakeholders (including users and agents), and focuses mainly on the downsides of MoMo for financially-vulnerable customers in Nigeria.
As a result, their study does not uncover the same breadth or depth of workarounds or practices that we observe, nor does it probe the underlying causes for these workarounds in as much depth, if at all. 