\section{Results}
\label {Sec:Results}

We present our results for each of the three research questions posed in the introduction. In most cases we found similar trends in Tanzania and Kenya so we report merged findings. We highlight areas where we observed different trends in the two countries. While we often provide the number of responses corresponding to a particular code to give readers a sense of the frequency of each theme in our interviews, we caution that this is a qualitative study and these numbers should not be interpreted as representative of frequencies of beliefs and behaviors in the population.  
\begin {table} [htbp]
\caption{A summary of workarounds by users and agents}
\centering
\begin {tabular} {|p{3cm}|p{2.7cm}|p{1.8cm}|}
\hline
\multicolumn{2}{|l|}{\textbf{Transaction Execution Workarounds}}& \textbf{Changing Component}\\
\hline
\multirow[c]{2}{*}{\textbf{Alternative execution}}& Agent executes transfer& Agent \\ \cline{2-3} & Transact at a distance & User \& Agent\\
\hline
\multirow[c]{2}{*}{\textbf{Alternative services}}& Advance transactions& \multirow[c]{2}{*}{Agent} \\ \cline{2-2} & Credit/Loan services & \\
\hline
\multirow[c]{3}{3cm}{\textbf{Change transaction characteristics}}& Location& \multirow[c]{3}{*}{Transaction} \\ \cline{2-2} & Size & \\ \cline{2-2} & Channel & \\
\hline
\multicolumn{2}{|l|}{\textbf{KYC Workarounds}}& \\
\hline
\multirow[c]{2}{*}{\textbf{Agent-adjusted KYC}}& Relationship-based& User \& Agent \\ \cline{2-3} & Agent complaisance & Agent\\
\hline
\multirow[c]{3}{3cm}{\textbf{Alternative identity proofing}}& Third-party ID& \multirow[c]{3}{*}{User} \\ \cline{2-2} & Reduced KYC & \\ \cline{2-2} & Human recommender & \\
\hline
\multicolumn{2}{|l|}{\textbf{Confirmation Workarounds}}& \\
\hline
\multirow[c]{2}{*}{\textbf{Alternative confirmation}}& Agent confirmation& User \& Agent \\ \cline{2-3} & Recipient confirmation & User\\
\hline
\multicolumn{2}{|l|}{\textbf{Transaction Reversal Workarounds}}& \\
\hline
\textbf{Alternative reversal}& Agent-assisted reversal& Agent\\
\hline
 \end{tabular}
\label{table:tab1workaroundsContinuum}
\end{table}


\subsection{Q1: Practices facilitated by the user-agent relationship, motivations, and privacy/security concerns}
\label{sec:habitsandpractices} 
The practices we observed can be roughly divided into two categories: 1) practices surrounding how mobile money users \textit{choose} an agent, and 2) practices around how mobile money users \textit{use} an agent. 
The latter category was characterized by surprising examples of usage that we term \textit{workarounds}---workflows that are not intended by the MoMo provider, but that clients and agents, either independently or sometimes in a concerted way, work out between themselves. We identify these workarounds (Table \ref{table:tab1workarounds}) according to the stage of the transaction continuum in which they took place:  during the transaction execution, KYC, transaction confirmation, or the optional reversal phase (Table \ref{table:tab1workaroundsContinuum}). In the following sections, we discuss these workarounds and their motivations (Table \ref{table:tab2workaroundmotivations}). Workarounds can additionally (roughly) be categorized by what component of the MoMo environment they modify; \textit{the user}, \textit{the agent}, or \textit{the transaction} itself.  

\begin {table} [htbp]
\caption{Workarounds through the transaction stages}
\centering
\begin {tabular} {|l|l|l|}
\hline
\textbf{Workaround}& \textbf{Ke (n)}& \textbf{Tz (n)}\\
 \hline
 \hline
\multicolumn{3}{|c|}{Transaction Execution Stage}\\
\hline
Proxy, remote and direct transactions& 23& 18\\
Leaving money with the agent& 18& 20\\
Modify transaction size and/or location& 10& 9\\
Advance transactions& 5& 10\\
System circumvention& 7& 6\\
Agent-enabled credit& 2& 5\\
\hline
\multicolumn{3}{|c|}{KYC Stage} \\
\hline
Relationship-based KYC & 20& 1\\
Reduced KYC using ID number only& 19& 0\\
Third-party ID use& 12& 0\\
Complaisant agents& 7& 0\\
\hline
\multicolumn{3}{|c|}{Transaction Confirmation Stage}\\
\hline
Reliance on agent confirmation & 4& 14\\
Get recipient's verbal confirmation& 1& 18\\
Alternative confirmation (Check balance)& & \\
\hline
\multicolumn{3}{|c|}{Post-transaction Execution} \\
\hline
Agent-assisted reversal & 7& 8\\
Multiple transaction  attempts& 1& 3\\
\hline
\multicolumn{3}{|c|}{\textbf{(n): number of respondents who mentioned}} \\
\hline
 \end{tabular}
\label{table:tab1workarounds}
\end{table}

\begin {table*} [htbp]
\caption{Top seven motivations for pursuing workarounds}
\centering
\begin {tabular} {|p{5cm}|l|l|p{9cm}|}
\hline
\textbf{Motivation}& \textbf{Ke (n)}& \textbf{Tz (n)}& \textbf{Sample Quote}\\
 \hline
 \hline
Convenience and time-saving& 22& 26&  ``I had to leave [the money]. I didn’t want to wait because I was in a hurry" (KE19) \\
\hline
Navigating MoMo costs&15 & 21& ``I was trying skip the transaction charges because when you put it into your phone you will be charged the fees to complete that transaction" (TZ20)  \\
\hline
Unavailable service (Network downtime and Insufficient float)& 10& 14& ``I went to an agent and he didn’t have float and so I left money and the [phone] number on a paper and deposited for me later." (TZ11) \\
\hline
Surveillance concerns& 16& 16& ``There is an issue with the withdrawal. I cannot go to someone [an agent] I know because they can send someone to follow me. If you want to withdraw such amounts, you do it in town where nobody knows you" (KE19) \\
\hline
Physical ID challenges& 18& 1& ``[The requirement to have and ID] is an issue because sometimes you forget" (KE31) \\
\hline
Navigating KYC denial of service& 15& 2& ``I do not have an ID, they[the agent] know I do not have an ID. They do not ask me a lot of questions. If the agent sees me, they just deposit the money without asking" (KE32) \\
\hline
Usability issues (Data entry, complex interfaces etc)&15 & 16& ``I was withdrawing money but unfortunately I entered the wrong agent number and so the money went to another agent." (TZ24) \\
\hline
 \end{tabular}
\label{table:tab2workaroundmotivations}
\end{table*}

\subsubsection{Considerations in choice of agents}
While factors like distance from their home, float, and size of transaction were important considerations in choosing an agent, users also considered the perceived security based on the agent's physical location (n=23), and indicated that they preferred to use specific agents regularly (n=58). The top reason for this preference was trust (n=47).  For most, this trust stemmed from knowing the agent and perceiving them as honest. 
``\textit{If you know each other, that means you have built the trust}" (TZ13). 
Other reasons for trust were characterized by a utilitarian perspective---what the user could do with their agent---and were similar to the other reasons for using specific agents regularly like easier recourse, traceability, being assured of float, informal transaction arrangements and perceived confidentiality by the agent (Table \ref{table:regularagents}).
Overall, we observed that users choose agents largely based on security and/or convenience. Unlike the study by Chamboko et al, \cite{chamboko2021role} we did not observe any major considerations around gender. When participants mentioned a preference for agents of a particular gender (n=10) it was generally due to their sense of which agents were more personable.

\begin {table*} [htbp]
\caption{Reasons why users use agents regularly}
\centering
\begin {tabular} {|p{2.7cm}|p{1.5cm}|p{10.8cm}|}
\hline
\textbf{Reason}& \textbf{(n)}& \textbf{Excerpt}\\
\hline
Trust& 47& ``When we are dealing with something like money, the word trust has to be there." (KE07)\\
\hline
Being known& 34& ``When you call them at that time or any other time, they'll just accept your request because they already know that you are their customer" (KE02)\\
\hline
Informal agreements& 25& ``If I have a money issue and I do not have money, at times they lend me." (KE20)

\medskip
``Because sometimes I have to withdraw funds when I am not present to give to someone who is present to keep things going" (TZ12)\\
\hline
Easier recourse& 12& ``Even if I made a mistake, that means I can go back to him and its easy for him to understand me." (TZ11)\\
\hline
Traceability/permanence& 10& ``I know where to find them. Yeah, like not necessarily at their work, [but also] their home" (KE08)

\medskip
``Other agents might do something bad and when you come back the next time you won't find them... Because they just have a small place arranged for this particular service, and they might even move elsewhere. I trust those ones that I can go and buy products from them as well...somebody with a shop and he is doing other transactions" (TZ27)\\
\hline
Easier KYC& 7& ``Because I do not have an ID for this place [Kenyan ID]. So I will go to the one who is further because we know each other." (KE24)\\
\hline
Loyalty& 7& ``There is an agent I prefer because they are faithful to you and you are faithful to them" (KE25)\\
\hline
Assured of float& 6& ``I prefer them because they are reliable with their work. Others would tell you to leave your number to send it later. That is what I don’t like." (TZ31)\\
\hline
Confidentiality& 5& ``The agent is supposed to be someone who can keep secrets because you don’t know what intentions one might have with you if they share information about your transactions" (TZ16)\\
\hline
\multicolumn{3}{|c|}{\textbf{*(n): number of respondents who mentioned}}\\
\hline
\end{tabular}
\label{table:regularagents}
\end{table*}

\subsubsection{Workarounds in the transaction execution phase}
% \mbox{}
% \medskip
Among workarounds that happen during transaction execution, we categorize them as follows: (1) Alternative transaction executions in which either the agent or client does not execute their intended role in the transaction, %(Fig. \ref{figDeposit} and \ref{figWithdraw}), 
(2) agents provide alternative services that are not endorsed by MoMo, and (3) clients modify the transaction characteristics (e.g., size, location).
\mypara{\textbf{Alternative execution with changes in agent/client role}}
\label{lbl:alternativeTxExec}
This first category of workarounds at the transaction execution phase entailed users working with agents to navigate various challenges by having agents either complete the transaction on their behalf, or having them act as transaction proxies. For example, to manage their transaction charges, users completed transactions through direct deposits. In this case the agent directly deposited into the recipient's account (Avoiding steps 1-3 in Figure \ref{P2P}). TZ27 said,  ``\textit{When [the agent] deposits it into my account and I send it from my account, I will be charged, and that’s why sometimes I just decide to ask the agent to do the transaction straight away.}" 
In addition to saving on cost, this workaround transferred to the agent the responsibility for potential mistakes. The transferred responsibility was motivated by the complexities of the reversal process, which we discuss further in Section \ref{workaroundpostexec}. 

Some users also reported leaving money with the agent to execute the transaction later. This was mostly motivated by convenience either because they were in a hurry and/or the MoMo network was unavailable, or when the agent did not have sufficient float (Section \ref{sec:floatdef}). ``\textit{If there is a delay... I will just have to leave the money with my details and phone number}” (KE01). Other times, insufficient float necessitated partial fulfillment of transactions where the user collected part of the cash and got the rest later. Because of the burden of liquidity management, which falls squarely on agents, \cite{eijkman2010bridges} leaving money or getting money later benefits the agent as well, but may be problematic too when users cannot cash out their money \cite{kenya2009mobile}. Some of those who experienced difficulty in using MoMo reported having an agent completing a transaction on their behalf.   


For P2P transactions, agents served as proxies as users made them cash collection points for the intended recipient. In addition to being useful when a MoMo user wanted to transact with a non-registered individual, or when the recipient did not have the required identification---such as when using a SIM card registered under another person, agents acting as proxies was also motivated by the need to save on transaction charges. This workaround circumvented steps 2-4 in Figure \ref{P2P}, by allowing the user to cash out at recipient's agent instead. “\textit{Maybe you have a friend, who needs like a hundred shillings, instead of sending money to the friend, to avoid that transaction cost, you just withdraw in a certain agent then you tell your friend to go and collect}” (KE17).

While the intended design of MoMo involved transacting in person to facilitate KYC, users modified this by sending a third party to complete the transaction on their behalf. Although this sometimes entailed sending the person to make a deposit, more participants indicated that they would send the proxy to collect physical cash once they had remotely initiated the cash-out. Sometimes, the individual who was sent was the intended final recipient. Many participants stated that they completed transactions without being physically present, mostly for convenience's sake.
\mypara{\textit{Privacy and security issues in alternative execution}}Using direct transactions means that the agent received all the transaction notifications and that the user had no means of recourse in the event of a problem as they had no proof of the completed transaction. The alternative execution workarounds also exacerbated the need to share more personal information with both agents and other parties acting as proxies. For example, in direct transactions, Alice has to share with her agent Bob's personal details, and when leaving money with the agent to transact later, users reported leaving their personal details such as their ID, phone numbers, and names with the agent. These were often written ``on a piece of paper.'' Alternatively, users gave a proxy some or all of the following details to share with the agent for purposes of verifying the transaction: their names, ID and phone number, amount transacted, and the transaction confirmation message that they received from the MoMo provider. Remote and proxy transactions therefore thwarted the KYC measures put in place by MoMo providers that require the customer to show up in person at the agent to facilitate identification and authentication (Figure \ref{figDeposit} and \ref{figWithdraw}). As a result, some participants reported that the agent kept their personal details such as their phone number and ID number to facilitate such transactions when they were not physically present. This is similar to a previous finding on cybercafe managers keeping their patrons' passwords \cite{munyendoeighty}.
\mypara{\textbf{Alternative services: agent loans and advance transactions}}
For this second category of workarounds at the transaction execution phase, the findings show that in addition to facilitating CICO services, agents offered loans and advance transactions to users. These interactions were not accompanied by any formal agreements. TZ07 indicated that he gets a loan from the agent to settle employee wages when ``\textit{working at the site and I need to pay the workers and I have shortage of money.}" KE09 even reported that she can get ``\textit{a certain amount of money, and return it at the end of the month.}"

On the other hand, advance transactions allowed the agent to disburse e-money or physical cash before the actual MoMo transaction. ``\textit{When you are in a hurry, you can even go to the agent and take some cash, then withdraw later}" (KE04). KE16 also said that she would call the agent and ``\textit{tell him to deposit some money for me. `When I come back, I will give it to you'.}"  Instead of getting cash, TZ02 shared that they can get an airtime advance.  ``\textit{I can go and tell her to send an airtime bundle to me, she does so and then I will give her the money later.}"
\mypara{\textbf{Modifying transaction characteristics}}
\label{sec:TxModification}
In this third category of workarounds, depending on what challenge they were trying to address, users modified transaction characteristics altering the size, location, or channel of transaction execution. Users reported modifying the transaction size to save on transaction fees. For example, we observed that users in Kenya split their transaction to smaller payments that did not attract any charges, as opposed to sending the whole amount as a lump sum. ``\textit{[Say] I want to send you Ksh 300. If I send you like a hundred separately three times, that will save on cost because sending Ksh 100 is free}" (KE08). In Tanzania the providers imposed fees on all P2P transactions, even of the smallest value. However, we still observed a different way of saving on cost through the use of alternative transaction channels that circumvented the MoMo system. This was enabled by the use of person-to-business (P2B) payment channels instead of the normal withdrawal channels through the agent's store number. In this case, the consumer did not incur any charges---even for large amounts. ``\textit{if you want to withdraw money and you don’t want to be charged a lot they tell you to use \textit{LIPA Number}\footnote{LIPA number is a merchant-specific number that is reserved for settling payment for other goods and services that the agent may provide. The consumer pays no service charge.}}" (TZ18). Users also modified the transaction size and location for privacy and security reasons, which we discuss separately below. 
\medskip
\mypara{{\textit{Privacy/security issues in modification of transaction characteristics}}} Many users were apprehensive about agents knowing their transactions details. ``\textit{I don’t feel good because you can’t know someone’s intentions. [The agent] can know and tell someone else and then you are followed up and robbed}" (TZ32). In prior work, users in Nigeria expressed similar sentiments, reporting the fear of being robbed at the point of cash withdrawal \cite{mogaji2022dark}. 
Participants therefore indicated that they sometimes changed how much they transacted and this afforded them some perceived security/privacy. For KE08, ``\textit{if the transaction is very large, I don’t have to transact it all at once. Maybe I can divide it into two, maybe three times. Maybe do one transaction here, then another in another agent like that. You can do that for security reasons.}" While many users alluded to some fear of robbery, others reported social concerns as a reason for desiring privacy from the agent. For example KE16 said, ``\textit{when you are transacting a small amount, there is shame [especially] when I am a male and I go to a female agent.}"

In addition to changing the transaction size, some users reported changing the location where they transacted to achieve more security/privacy. TZ33 shared that, ``\textit{If for instance this week I have withdrawn a large amount from [one agent], and if next week I am withdrawing almost the same amount of money then, I would not withdraw from [the same agent] this time.}" Some of those who reported feeling  embarrassed about sharing their transaction amount with the agent said that they would travel further to a place where they are not known.  TZ07 explained, ``\textit{I might feel shy to withdraw TShs 6,000\footnote{One US dollar was approximately equivalent to 2340 Tanzanian Shillings at the time of writing this paper.} and would say, `I don’t want my nearest agent to know'.  [So] I will go very far where no one knows about me and withdraw.}" This same participant brought up the benefit of ATMs when compared to transacting at an agent. ``\textit{If I go to withdraw 10,000 at the ATM, it is my secret, or 200,000, it's my secret.}"
\subsubsection{Workarounds in the KYC phase}
\label{Sec:KYCworkarounds}
\mbox{}
\\
KYC practices in Kenya and Tanzania were implemented differently at the time of our study.  In Kenya, MoMo users were required to show their physical ID to the agent whenever they transacted. Tanzania did not have similar requirements at the time of the study. We therefore skipped the questions addressing user perceptions about using  IDs to transact in Tanzania. Thus any sentiments that arose emerged organically as seen in (Table \ref{table:tab1workarounds}). When asked how they felt about the requirement for an ID to transact, KE06 said, ``\textit{It's good to provide IDs because most people are fraudsters.}"  Participants from Tanzania indicated that physical ID was a requirement in the past, but these checks happened rarely at the time of the study, depending on the size of the transaction and the agent. However, they too were concerned about the risks this might present: ``\textit{I am using a SIM card which is not mine... and nobody asks about it. What if I stole that SIM card? Therefore, [the ID] is very important}" \footnote{Up to 18\% of SIM cards in LMICs are registered under a third-party's ID, usually because the people have no ID of their own.}(TZ07). These responses suggest users' beliefs regarding the importance of KYC for security. TZ16 shared a negative experience stemming from lack of identification. ``\textit{I told you of how money was withdrawn from my account. If they required to be shown ID that would not have happened because the SIM card was registered with my name.}"

While users expressed beliefs about the importance of KYC, many of them also felt that KYC was ``cumbersome," ``frustrating," and ``not good, because they require the ID every time." Further analysis of the data revealed that KYC using a physical ID card presented multiple challenges that we summarized as being forgotten, lost, or simply not owning one. As a result, individuals adopted workarounds to navigate the physical ID challenges and the potential denial of service from falling short of requirements. We discuss the two categories of KYC workarounds that our findings surfaced.
\\
\mypara{\textbf{Agent-adjusted KYC}}
By transacting at agents they already used regularly, participants said they avoided  having to show an ID because the agent knew them. KE19 said, ``\textit{It's not a must if the agent knows you... I carry my ID when I want to withdraw money in a place where I am not known, maybe in town or places that are far from home.}" Some agents instead acted out of goodwill, or users looked for agents with lax KYC practices. KE06 urgently needed some money because they were taking their child to school ``...\textit{I went to an agent who refused to help with the transaction because I had no ID. I was upset...I went to another agent to whom I explained the situation to and told him/her that I had memorized the ID number. The agent asked me if I was sure the ID was mine about three times. I said yes...The agent allowed me to transact}." 
\mypara{\textbf{Alternative identity proofing}}
% \medskip
In place of the standard KYC procedures, users and agents sometimes adopted alternative identity proofing in which users: (a) gave their ID number instead of producing the physical ID card, (b) used a third-party ID, or (c) relied on third-party recommenders. KE08 remarked that,  ``\textit{...some agents don’t even bother asking for the physical ID. They just ask for the number.}” Some of those who did not use a SIM card registered in their own name presented the ID of the person under whom the SIM card was registered. ``\textit{I used my cousin’s ID because I had already registered the line with that ID, so I had to show his ID because the name that came was not my name}" (KE15).
When users could not present the third party's physical ID, they either provided the ID number or a copy of the  ID. ``\textit{I was using my friend's ID. I had his photocopy}" (KE16). Two participants from Kenya mentioned ``identifying themselves through other means" like ``\textit{a friend who can confirm that this so and so}" (KE12).
\mypara{{\textit{Privacy/security issues in KYC workarounds}}}
 In addition to agents keeping customer information for purposes of transacting in their absence, as we discussed earlier (\ref{lbl:alternativeTxExec}, the KYC workarounds modified what was provided and how it was provided and these were divergent from the KYC processes as stipulated. Agent-adjusted KYC and using a third-party ID changed what was provided. By using third-party recommenders and an ID number in place of the physical card, users and agents modified how KYC was achieved. The purpose of showing a physical ID in Kenya was to facilitate authentication by the agent, who compared the name on the ID with the name appearing on the agent's transaction summary. This was essentially the name under which the SIM card was registered. Lax KYC procedures by agents exacerbated the privacy and security risks arising from KYC workarounds. The findings show that not all agents implemented KYC in a standard manner. Whether an agent asks or does not ask you for an ID \textit{``depends on the place you usually go to deposit or withdraw money}" (KE05). For example \textit{``in case I am in Kayole,\footnote{Kayole is a low-income neighborhood in Nairobi, Kenya.} I have to have an ID, and when I am in Kibera,\footnote{Kibera is a low-income and largely informal settlement neighborhood in Nairobi, Kenya.} it’s not a must}" (KE20). MoMo users also felt that agents were not always vigilant with regards to implementing KYC. ``\textit{They are not keen}" (KE16). ``\textit{Sometimes I am a boy and I have gone with a girl’s ID}" (KE15) or ``\textit{they would ask `what is your Name?’ I tell them ‘XXX YYY.' They can see I am a male, but I mentioned a female name and nobody cares. Whether I am a thief, I stole her SIM card or whatever, nobody cares and it is actually very concerning}" (TZ07).

\subsubsection{Workarounds at transaction confirmation}
\label{sec:tx_confirmation_workarounds}
\mbox{}
\\
When asked about how they knew that a transaction was complete, the two most common ways users mentioned were the transaction confirmation message by the MoMo provider (n=61) and checking their balance (n=31). However, due to network downtime, ``\textit{...the message sometimes delayed}" (TZ31). Sometimes, only the balance checking option was available. Unfortunately in addition to challenges with timeliness of the message, these two methods were not always dependable. ``\textit{Some messages might be fake}" (TZ27) (We discuss these fraud-related concerns further in Section \ref{sec:fraud}) and ``\textit{to check balance, they charge some amount. It is not free}" (TZ12). As a result, users designed other transaction confirmation practices namely: reliance on agent confirmation and getting the recipient's verbal confirmation.
\mypara{\textbf{Reliance on agent for confirmation}} Users relied on agent communication in many instances. When they were transacting at a distance, some users depended on the agent to let them know ``\textit{if there is a problem.''}  KE05 explained, \textit{``...that’s why I prefer to go to one agent who is close}" (KE05). In remote and direct transactions where the user did not receive the transaction notifications, users waited until the agent told them the name that had appeared on their side of the transaction summary. ``\textit{When I give [the agent] the money to send to someone, when he mentions the recipient’s name and it matches the one that I am sending to, then I know it has gone through}" (TZ06). As an additional precaution, TZ24 said, ``\textit{what I usually do is that I don’t tell the agent the recipient’s name that will appear, I would wait until the agent tells me ‘is it so and so name?’}"
\mypara{\textbf{Getting recipient's verbal confirmation}}In addition to getting confirmation from the agent, some participants also called the recipient to confirm that they had received the money. This was especially common when the user asked the agent to directly top up the recipient's balance (see Section \ref{lbl:alternativeTxExec}). ``\textit{I would call the person I am intending to send money to before [to let them know], that I will be sending money through an agent and not through my number}" (TZ13). %"\textit{when the person who I am sending the money confirms 'Yes, I have seen [the money' then I leave}" (TZ01).
These multiple confirmation methods seemed to be necessary ``...\textit{because there are times that when you go to transfer money to someone else, the name comes up but when you ask the person whether they received it, they say ‘no, I have not yet received it'}" (TZ17).

\subsubsection{Workarounds at the optional transaction reversal phase}
\label{workaroundpostexec}
At the post-execution stage, the main workaround we observed was related to transaction reversal. The need for reversal mostly arose due to challenges in data-entry. The most common mistakes that users cited were entering the wrong agent number and sending to the wrong recipient. For such erroneous transactions, MoMo providers provided a self-serve reversal process. However, these reversal processes were not always clear. ``\textit{I had a challenge reversing it. So I went to the agent and asked if they could help.}" (KE1). They also did not always result in users getting back their money since success depended on the cooperation and honesty of the unintended recipient. If the latter had already withdrawn the money or transferred it out of their mobile wallet, a user could not get it back. Even in instances when the reversal process was successful, users did not get the money back immediately.``\textit{I had to wait for seven days}" (TZ09). In general, the cost of making a mistake was high for users as aptly indicated by KE15, ``\textit{I saw that when you make a mistake, you will lose money so you have to now be keen.}"



\subsection{Q2: Dealing with reduced privacy/security}
\label{sec:enablers}
As seen through the findings in Section \ref{sec:habitsandpractices}, the user workarounds were often accompanied by more risks and reduced security/privacy. As a result users took further measures to mitigate such risks. Based on user responses, we identified two categories of measures:  using a familiar agent, and measures to limit personal risks. 

\subsubsection{Using a familiar agent} 
Close to half of the participants (n=36) believed that the agent agreed to help because they were a customer. Assent by the agent was deemed necessary for ``courtesy" and ``building friendship" because some of these agents were neighbors: ``\textit{...we live close to each other}" (KE24), and because \textit{``If he helped me it’s easy to send other people to his place for transactions}" (TZ09).

While the perception that success was based on a ``customer being king'' philosophy, a majority of participants (n=51) mentioned that knowing the agent gave a sense of security. 
``\textit{I felt it was not good [to leave the agent the money] but, I just trusted anyway...because I usually use that same agent}" (KE01). This concept of knowing the agent as an enabler was also evident in the way users reported ``doing the safe thing'' when they did not know the agent. For example, with regards to waiting to see the transaction confirmation message, TZ09  said, ``\textit{If the agent is a stranger, I wait until I receive it.}"

\subsubsection{Measures to limit personal risk}Participants spoke about actions they took to limit personal risk. Most were similar to those already discussed as workarounds during transaction execution: i.e, changing the transaction size or location depending on the perceived risk.
TZ16 said, ``\textit{...I always look at the amount, I wouldn’t have left one million shillings [for the agent to deposit later] because I would have got sick, if it got lost.}" People who lacked technical literacy and relied on agent assistance mentioned withdrawing all the money from their wallet. While they could not navigate the interface to know their balance they could count the money to ensure it was the full sum of what the sender had indicated. In proxy and remote transactions, users and agents designed alternative ways of limiting risk including calling the agent ahead of the proxy's arrival to authorize them to disburse money to a proxy, or the agent would call the sender. Some users even went ahead and introduced their elected proxies to the agent, and this made authentication in future proxy transactions unnecessary. Users also limited their risk by involving proxies less in transactions that would required them to share information such as their MoMo PIN or to send the proxy with the phone. As a result, most users reported using proxies to collect the cash after initiating the transaction remotely. ``\textit{I sent my brother...because I was just making a deposit and not a withdrawal that would require me to give them the PIN. If I gave him the PIN maybe he can go and withdraw everything}" (KE25). Users also seemed to exercise caution in who they choose as a proxy. More often than not, the proxy was a trusted or known individual.

\subsection{Q3: Implications and challenges of user-agent interactions and  practices} 
\label{sec:risksandchallenges}
Some of the risks and challenges that arise from using MoMo relate to privacy and security. Others stem directly from workarounds, while others are a result of the current MoMo system structure (as noted by \cite{mogaji2022dark}).

\subsubsection{Unofficial and agent-determined fees}
When users worked with the agent to circumvent official transaction costs, they were often subject to ad-hoc agent-determined charges. ``\textit{You may find that he tells you “Please give me Tshs 500 because I have gained nothing from this transaction.” You have to do it because if you do not, next time you won’t be helped}" (TZ01). For some users, there was no other option because the lack of interoperability made the workaround necessary. ``\textit{When you go to send money in another mobile network, the agent asks you to add a certain amount.... They claim that to send money in a different network is a little bit costly}" (TZ34). This risk of being overcharged has been noted before \cite{martin2019mobile,mogaji2022dark}.  

\subsubsection{Reduced data privacy}
As highlighted in previous sections, the alternative ways of transacting such as proxy,  remote, and direct transactions resulted in more sharing of personal and private transaction data (often in unsafe ways such as writing on paper) with multiple parties including agents and proxies to facilitate the various workarounds. 

 \subsubsection{Concerns about disclosed data}We asked participants about how their data is recorded and processed and how they felt about giving personal details at the agents. Close to half (n=29) were concerned about disclosed data and indicated feeling ``insecure." 
 When we asked why they felt this way, most users cited the possibility of agent misconduct including sharing it with third parties like ``politicians" and ``random strangers." ``\textit{I think the agents are using those details wrongly. Maybe that’s why people receive scamming messages}" (TZ15). KE25 added,  ``\textit{...agents can use your ID to register other people.}" TZ27 had experienced the same, ``\textit{I discovered when I was going to register for my SIM cards that there were two other SIM cards that were registered under my ID}." Some of these concerns with agents have been noted before \cite{martin2019mobile}.
For others, the hesitation was purely a matter of privacy. Agents got to know the full names of their customers and some were not happy with this for reasons like not wanting to be called by their full name. 
\mypara{Willingness to share data} Despite these concerns, many participants were willing to share their data (n=45); half of these had previously indicated being concerned about disclosed data. The participants gave various reasons for why they shared their data. Most said that ``it was a requirement" for service access and they ``had no other way." There were those who felt that the information they left including the transaction data trails were important for security so that ``\textit{if an issue happens...a person cannot deny you because you have written [your details] there}" (TZ26).
In contrast to those who identified with the need for security, there are those who said they were not worried because ``the money is theirs" and they ``have not stolen it" or because they believed they were not sharing any sensitive information, where sensitive mostly referred to their PIN number. ``\textit{I get scared of sharing my PIN number but as for such other details like contact number and names, no, I don’t worry.}" (TZ06). The belief that agents were trustworthy and the belief that as users they were anonymous to the agent also made some users confident to give their information. 
``\textit{There is no problem when they know...because I withdraw and leave. They don’t know me}" (KE36).

\subsubsection{Risks from MoMo-related fraud}
\label{sec:fraud}
Some users were concerned about MoMo-related fraud. The types of fraud described by users  were similar to previously-documented fraud \cite{buku2017fraud} \cite{akomea2019control}; many of these were enabled by  social engineering, such as fake transaction reversals and masquerading as a customer care representative or agent. An example of the former is where fraudsters tricked the target user to believing they had sent money erroneously and asked them to send it back. Common reasons that the masquerading caller gave for the call involved resolving account anomalies. As the user complied by giving the required information, the fraudster was able to execute fraud. 
Participants also acknowledged that access to personal information as well as transaction trails facilitated fraud like SIM swaps and other forms of identity theft. For example, users in Kenya strongly associated identity theft frauds with the loss or unauthorized access of a person's ID. ``\textit{If you lose your ID, somebody can replace your SIM card, if that person knows your number, your last withdrawal, and balance, he might call customer care and give these details. After that they will withdraw, and later when you replace your line, you find there is nothing [in your wallet]}" (KE17). 
