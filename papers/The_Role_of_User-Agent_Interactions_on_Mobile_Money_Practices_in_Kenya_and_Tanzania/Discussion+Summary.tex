\section{Discussion}
\label {Sec:discussion}
% \subsection {Role of agents}
Our study joins several others in highlighting the centrality of agents to the MoMo ecosystem \cite{mogaji2022dark}, e.g., by ensuring the inclusion of users with limited technical capabilities. However, our study enriches this understanding by also illustrating the ways in which agents and users modify intended MoMo procedures (workarounds) to better meet users' needs, and the underlying causes for these workarounds.
In this section, we list some takeaway messages that emerged from the results in Section \ref{Sec:Results}. 
\mypara{Takeaway 1: User workarounds involving agents are often motivated by efforts to limit risk or uncertainty (e.g., due to environmental factors or user interface limitations)} Agents play a complex role in the MoMo ecosystem, mitigating risk in some cases, and introducing risk in others. For instance, agents regularly help users revert transactions sent to the wrong recipient, despite user interface tools meant to prevent such events. These challenges may be exacerbated in LMICs where users often use devices with small screens and that support technologies such as USSD and STK instead of apps. Users therefore have to manually type the recipient number instead of selecting from a recipient list as is common with mobile payment apps in the US. These challenges motivate some workarounds, such as people writing numbers on a paper (to enable easier transfer), or asking the agent to enter the number,
thus transferring responsibility for data entry risks to the agent. 
 
Treating agent interactions as a risk-management measure is plausible given the terms of service in most MoMo implementations, which seldom protect the user from losses related to the use of the service \cite{bowers2017regulators}. This is unlike the United States where users of mobile payment services like Venmo and Paypal are protected by regulations and policies that limit liability for loss \cite{bowers2017regulators}.

Although agents often mitigate risks for users, they can also introduce new risks. 
Agents regularly handle sensitive and diverse user data. Concerns about agent surveillance, agent maleficence in perpetrating fraud, and inappropriate sharing/commodification of user information are all issues raised by our study participants (see next takeaway). 
\mypara{Takeaway 2: Users' stated preferences for security measures are inconsistent with their (workaround) behavior}

Our findings suggest that people are largely aware of  security threats related to the use of MoMo, and view security measures (like KYC and not sharing one's PIN) as important. 
Some of this awareness stems from campaigns by MNOs that caution users to keep their PINs secret \cite{mckee2015doing}. On the flip side, users often perceive themselves as safe as long as they have not shared their PIN; other personal data is often viewed as non-security-sensitive. While several study subjects detailed the importance of process security, e.g., for curbing fraud, they also engaged in contradictory behaviour, especially with regards to KYC. 
We found prevalent use of third-party IDs, as well as other workarounds to the expected KYC process. 
We hypothesize that this is a result of burdensome KYC practices that does not appear to meet the expectations and needs of the target population of MoMo users.   
\mypara{Takeaway 3: Users have varied perceptions on the role and impact of data sharing with agents} A significant number of users in our study reported consented data sharing,  mainly because giving personal information was a mandatory prerequisite for MoMo service access. Users also reported concerns over third-party information sharing although they had no way of ascertaining that this was actually happening. This is drastically different from many countries with data use restrictions; for instance,  US banks and mobile payment services are required by the Federal Deposit Insurance Corporation (FDIC) to have privacy notices where users can opt out of data collection and sharing \cite{bowers2017regulators}. 
These regulations also mandate that such services clearly state the purpose for collecting  user data and third-party data sharing. To the contrary, most MoMo services do not have privacy policies, and when they do, they are difficult for users to understand \cite{bowers2017regulators,munyendo2022desperate}, a challenge which has equally been noted in the US~\cite{schaub2015design}.

Other subjects were willing to freely share data with  agents because they ``had nothing to hide" or because they trusted the agent. 
Similar attitudes have been reported in prior work from other contexts, e.g., on the (lack of) need for privacy for `normal people' \cite{gaw2006secrecy} and the link between trust and willingness to share data   \cite{das2014effect,yisa2023investigating}.
Some felt that data sharing with agents was important to prevent fraud. Our study does not directly show evidence of risks from agents inappropriately sharing user data with third parties; however, we postulate that lax data sharing with agents could pose security vulnerabilities. 

Our observations regarding attitudes towards data sharing are particularly interesting in light of prior work (e.g., \cite{martin2019mobile,acquisti2015privacy}), which showed that the perception of being monitored plays an important role in people's behaviour. 
In fact, in Zambia, people migrated to MoMo from traditional banking services because of concerns about surveillance posed by their Taxpayer Identification Number \cite{phiri_2018}.  The bottom line, therefore, is for governments, regulators and technology providers to balance between protecting user privacy while having visibility. However, since governments can also misuse visibility to suppress and subdue citizens, this recommendation should be adopted with caution within a strong policy and regulatory environment that has the necessary checks and balances in place. 

\subsection {Remaining barriers to financial inclusion}
While MoMo has inarguably extended access and reach of financial services, 
the workarounds documented in this work suggest that there is still room for improvement.

Access continues to depend heavily on local and regional regulations. For almost a decade, most African countries have continued to mandate SIM card registration using some form of identification like biometrics or national identification numbers \cite{privacyinternational_2019} \cite{jentzsch2012implications} \cite{donovan2014therise}. Access to such ID is known to be challenging in many locations and contexts \cite{gsmmobile} \cite{clark2022id4d} \cite{demirgucc2022global} and we discuss some of these challenges in Section \ref{Sec:KYCworkarounds}. Once users gain access to MoMo, the cost-related workarounds documented in this work suggest that affordability may still be a barrier for many. 
This is expected to worsen as governments increasingly impose e-levies on MoMo transactions; 
e.g., MoMo use in many African countries dropped when  governments introduced more taxes \cite{clifford2020causes}. 

Finally, our study also suggests that usability challenges pose a barrier to the flexible  use of MoMo (Sections \ref{workaroundpostexec} and \ref{sec:tx_confirmation_workarounds}), and these challenges go hand-in-hand with  \emph{systemic} float- and network-related limitations (Sections \ref{lbl:alternativeTxExec} and \ref{workaroundpostexec}). 
For instance, while money transfer apps in the US allow for interoperability across banks, interoperability of MoMo services in Africa is still nascent \cite{naji2020tracking}. Some users cited these challenges as motivations for workarounds like direct withdrawals (Section \ref{lbl:alternativeTxExec}) which allowed the recipient to collect cash. Other times, agents who were interoperable (i.e, providing MoMo services for multiple MoMo providers) offered the transfer services through direct deposits but charged the user extra fees for offering this. Such barriers may reinforce reliance on cash. 

\section {Recommendations and Future Directions}
\label{sec:summary}
Nan et al., \cite{nan2021we} called for qualitative studies to better uncover the less understood themes of MoMo. 
Our study offers an in-depth understanding of the MoMo user-agent interaction which has largely remained anecdotal to date. Our findings show that users and agents work together to design alternative workflows, or workarounds. We offer insights on the challenges that users face, thus motivating these workarounds. By understanding and addressing these challenges, MoMo can be a more effective tool for digital financial inclusion. 
While we make no claims of generalizability, we hypothesize that the insights gained may be relevant to other African countries, since MoMo operations and regulatory environments have more similarities than differences \cite{bowers2017regulators, nan2021we}. 
Investigating whether MoMo usage patterns hold more broadly in Africa is an interesting question for future work. We provide some recommendations. 
 \\
 \noindent \textbf{Recommendation 1: Study how to improve the MoMo user experience for reverting and confirming transactions.}
 Our findings suggest that many users in both Kenya and Tanzania rely on agents (and/or workarounds) to revert or confirm transactions.
 This is despite the fact that in Kenya, the dominant MoMo's (M-PESA's) user interface includes an option to revert transactions within 25 seconds after a transaction is placed.
Hence, research is needed first to understand why users struggle with transaction reversal. 
Similar questions arise for transaction confirmation, though our study subjects generally attributed failures to receive transaction confirmation to network outages.
A better understanding of both problems, would pave way for the design of new interfaces that are compatible with mobile devices common in LMICs (e.g., feature phones) to improve this aspect of the user experience. 

\noindent \textbf{Recommendation 2:  Measure the effects of user privacy and security concerns regarding data sharing with agents}. 
Privacy and security concerns were a strong undercurrent in our results, with many users reluctant to share data with agents for various reasons.
We envision that semi-automated solutions could help address this issue, and reduce the prevalence of several workarounds. 
However, to know whether such solutions are worth the investment, we hypothesize that a deeper understanding is needed of user data sharing concerns with MoMo agents. 
For example, we would like to understand (a) what fraction of users would prefer to share less data with agents, and (b) what fraction of users actively change their MoMo behaviors (e.g. by breaking up transactions into smaller ones) as a result of privacy or security concerns with regards to agents. 

\noindent \textbf{Recommendation 3: Develop registration and KYC mechanisms for users who lack formal ID}. 
Stricter KYC requirements (both for onboarding and during cash-in or cash-out transactions) do not necessarily improve system security. 
They can instead lead to workarounds such as third-party SIM cards, especially for the many who lack formal ID. 
Although the root problem is lack of access to ID (a problem that should be a high priority to address, but will likely require changes to laws and government processes), we conjecture that as a stopgap measure, there is value to designing mechanisms that allow users to access MoMo without formal ID. For example, this could include biometric-based registration. 
Clearly, such methods will present tradeoffs (e.g., privacy/surveillance, operating costs), which should be explored systematically.
