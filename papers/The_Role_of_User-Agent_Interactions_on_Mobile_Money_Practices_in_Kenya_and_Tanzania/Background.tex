\section{Background on Mobile Money}
\label {Sec:background}
MoMo differs from other mobile-based financial services, such as mobile banking or mobile payments in developed countries, in three key ways \cite{nan2019mobile}. First, MoMo initiatives are specifically designed to cater to the financial needs of individuals who are excluded from formal financial services (the un-banked). Second, users do not need a formal bank account since the user's phone acts as an account. Finally, MoMo relies on widely-available transactional agents. 
These features make MoMo an effective digital tool to extend basic financial services to individuals who would otherwise be financially excluded. 
 
\subsection{Understanding mobile money operations} 
Assuming the role of a financial service provider, the MoMo providers work with agents who are mostly small businesses within the community, and therefore extend financial services in a non-intimidating and familiar environment \cite{mckee2015doing}. With a MoMo account, individuals can use their mobile phones to execute financial transactions such as cashing in, cashing out, or person-to-person (P2P) transfers (Figures 2-4). They may also pay their bills and pay for other goods and services \cite{osakwe2022trust}. However, loan facilities are not provided through agents. The mobile wallet is often protected by a PIN, and most MoMo services adopt some form of additional authentication to identify users. Both Kenya and Tanzania, like most African countries, mandated SIM card registration with some proof of identity \cite{theodorou2021access}. Both countries require national IDs for this process, with Tanzania also collecting biometrics. In addition to these SIM registration requirements to identify users,  Kenya requires MoMo users to show their ID card to the agent when they transact. These authentication or KYC procedures are meant to protect against fraud and money laundering. 

\begin{figure}[t]
\includegraphics[width=\linewidth]{MoMoDepoFree.png}
\caption{Standard scenario for Cashing-In (Deposit)}
\label{figDeposit}
\end{figure}

\begin{figure}[t]
\includegraphics[width=\linewidth]{MoMoWithdFree2.png}
\caption{Standard scenario for Cashing-out (Withdraw)}
\label{figWithdraw}
\end{figure}

\begin{figure}[t]
\includegraphics[width=\linewidth]{MoMoP2PFree.png}
\caption{Standard scenario for a P2P transaction}
\label{P2P}
\end{figure}

\subsubsection{Transaction life cycles}
\label{sec:floatdef}
The MoMo operators provide agent outlets with special e-wallets that have higher maximum account balances than user wallets.
These e-wallets provide an e-money store \textit{float} against which the agent balances their customers' cash-in/cash-out (CICO) transactions. If the agent conducts too many cash-in transactions, they will eventually exhaust their e-float, and if they perform too many cash-out transactions, they will run out of cash. To restore balance, the agent needs to convert excess e-float into cash or vice versa, at a higher cash distribution entity such as a bank or a super agent. Here we illustrate the three main types of transactions that a MoMo user can accomplish. 
\mypara{Cashing-in (Depositing money, also known as topping up)}
Suppose Alice needs some money in her mobile wallet  (Figure \ref{figDeposit}). Alice goes to the MoMo agent and hands over her physical cash in exchange for e-money. As the agent transfers the e-money equivalent to Alice, the agent's float will reduce.
\mypara{Cashing-out (Withdrawing money)}
Suppose Alice has some e-money in her wallet and needs physical cash (Figure \ref{figWithdraw}). Alice goes to the MoMo agent where the cash-out transaction will be initiated either by Alice or by the agent. The transaction reduces Alice's balance but increases the agent's float. The agent  hands  the cash equivalent to Alice.
\mypara{Person-to-person transactions (P2P)}Suppose Alice wants to send money to Bob  (Figure \ref{P2P}). 
Alice goes to the MoMo agent (Agent 1) and tops up her MoMo mobile account (assuming she does not already have any balance in her MoMo wallet). Alice then accesses the MoMo service on her phone.  
Alice sends the money to Bob by entering the amount, and Bob's mobile number that acts as his account, and (in some cases) confirms by entering her PIN. Bob can go to the nearest MoMo agent (Agent 2) and withdraw the money from his account using the previously described cash-out process.
