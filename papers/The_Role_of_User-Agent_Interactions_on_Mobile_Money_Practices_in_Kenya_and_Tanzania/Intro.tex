\section{Introduction}
\label{Sec:Intro}
The pervasive adoption of mobile phones in many low-and-middle-income countries (LMICs) has led to the widespread use of mobile phones for delivery of financial services. One such digital financial service is mobile money (MoMo) which makes financial services available on the mobile phone through one or a combination of the following technologies: unstructured supplementary service data (USSD), short message service (SMS), SIM Toolkit (STK), and smartphone apps. The use of SMS, STK, and USSD ensures that the service can be provided on basic phones, which a larger majority in LMICs have access to \cite{silver_johnson_2018}.  

Mobile money has helped LMICs overcome many traditional banking challenges and has spurred financial inclusion and driven economic growth \cite{demirgucc2022global}. At the center of the definition of financial inclusion is the idea of availability, use, and accessibility, which is measured by the reach of financial services to individuals including the underserved \cite{demirguc2018global}. While there are still close to 1.5 billion people in the world, mostly in Africa, who lack access to basic financial services 
\cite{demirguc2018global} \cite{demirgucc2022global}, MoMo has been instrumental in making financial services more widely available. In 2020, there were approximately 1.2 billion registered mobile money accounts globally, and over 50\% of these were from sub-Saharan Africa (SSA) \cite{andersson2021state}. The use of MoMo contributed to narrowing the gap of un-banked people from 2.5 billion in 2011 down to 1.7 billion people in 2017 \cite{demirguc2018global}. Accessibility is achieved by having an extensive  network of agents (Figure \ref{figAgent}). These are third parties acting on behalf of the MoMo service providers---mostly mobile network operators (MNO)---to facilitate digitization of money. This network is the backbone of MoMo services since it acts as a distribution channel and provides a trusted mechanism for people to transact.
With a far extended reach of about 55 times that of banks and 26 times that of ATMs \cite{jakachira_andersson_2020}, agents act as an alternative to bank branches, and provide ``last-mile access,'' hence overcoming the distance-related barriers of owning and managing an account at a formal financial institution. Agents digitize more than \$715 million per day \cite{awanis2022state}. Overall, MoMo contributes enormously to the economy with transactions representing up to 75\% of the country’s gross domestic product (GDP) in  countries like Kenya, the home of one of the earliest implementations of MoMo, M-PESA \cite{creemers2020five}.

\begin{figure}[htbp]
\includegraphics[width=\linewidth,scale=0.5]{MPesaShop.png}
\caption{An M-PESA shop in Kenya (credit: Karen Sowon)
}
\label{figAgent}
\end{figure}

While MoMo agents and users are central actors to the ecosystem, little is known about the intricacies of their interaction and the overall impact this has on the use of MoMo. Although low trust is one of the hindrances to the use of formal financial products, we know from MoMo adoption literature that trust is an important antecedent to its use \cite{noreen2021impact} \cite{okello2020trust} \cite{osakwe2022trust}, considering that one of the first interactions that users will have entails handing over physical cash to the agent \cite{davidson2011driving} \cite{naghavi2019state}. In addition to facilitating cashing-in and cashing-out services (CICO), agents also provide additional services such as on-boarding new users and offering technical assistance \cite{andersson2021state}. Given the essential role that agents play, interactions between users and agents can have serious privacy and security implications for users that are not currently well-understood. We therefore study users' habits qualitatively, to get a rich understanding of the ways that users interact with agents, and how this can affect the privacy and security of the MoMo ecosystem at large. Our study investigates these research questions:
\begin{itemize}
\item RQ1a: What practices do the relationship between agents and users facilitate and why do users engage in these practices? 
\item RQ1b: What privacy and security challenges contribute to or arise from these practices? 
\item RQ2: How do users navigate these privacy and security challenges? 
\item RQ3: What are the implications and challenges of MoMo user-agent interactions and practices? 
\end{itemize}
Our study presents the results of an exploratory qualitative study to understand the practices of MoMo users in their interaction with agents in two East African countries with high MoMo adoption rates: Kenya and Tanzania with 73.12 million and 40.9 million registered accounts respectively as of 2022  \cite{tcraquarterly2022}\cite{kenyamobilemoney}. 
We conducted structured interviews between July and September 2022 with MoMo users between the ages of 18 and 65. We used inductive thematic analysis to synthesize the data.

Due to rapid population growth in LMICs, the demand for MoMo services may remain high for a long time, especially among the millions who are marginalized \cite{awanis2022state}. Thus, it is crucial to understand the interaction between users and agents to safeguard the opportunities for financial inclusion.

Our findings in this work highlight three important, previously-undocumented findings related to the interaction between clients and MoMo agents:
\begin{itemize} %[align=left]
    \item First, we find that users have a range of reasons for choosing a particular agent over another. These reasons are often related to trust and/or convenience, as well as lax know-your-customer (KYC) practices. The reasons for choosing a particular agent impact how users interact with that agent.
    \item Second, we identify and categorize a number of challenges that users encounter in their interactions with MoMo systems in general, and agents in particular. These challenges relate to complying with KYC requirements, privacy and security concerns surrounding information sharing with agents, and interface usability. Our data help to explain why these challenges arise; for instance, many challenges arise due to cost and a lack of sufficiently convenient access to agents. 
    \item Third, our results illustrate a wide range of workarounds or practices that both users and agents engage in to circumvent these challenges. These workarounds often introduce increased  risk to users, thwarting the very measures that MoMo providers have put in place to facilitate privacy and security. Some of these workarounds involve changing the location or intended process of a transaction, while others actually involve agents breaking KYC protocol, as well as offering services that are not officially supported by the MoMo provider such as loans and advance transactions. 

\end{itemize}

Overall, our results suggest that agents---thanks to their willingness to flexibly navigate or modify intended MoMo procedures---play a critical role in making MoMo accessible and convenient for end users. At the same time, the challenges users experience allow us to offer concrete recommendations for MNOs, researchers, and regulators. 

First, we identify a need to improve the MoMo user experience for confirming and reverting transactions. 
Second, our results suggest that users are often hesitant to share sensitive data such as transaction amounts (especially if very large or very small) with agents. We recommend further research to understand how and how often user privacy concerns are affecting MoMo usage, as well research on solutions that would limit agents' data access during interactions. 
Third, we suggest developing registration and KYC mechanisms for users who lack formal ID. Many of the workarounds we observe are necessitated by regulations requiring SIM cards to be registered to a formal ID, which causes significant obstacles to those without ID. The recommendations are not entirely prescriptive because solutions to improve MoMo would need to be accompanied by studies to assess their feasibility and usability.
