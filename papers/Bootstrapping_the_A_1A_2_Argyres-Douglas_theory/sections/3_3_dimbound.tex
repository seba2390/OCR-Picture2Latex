%!TEX root = ../H0bootstrap.tex
%%%%%%%%%%%%%%%%%%%%%%%%%%%%%%%%%%%%%%%%%%%%%
\subsection{Dimensions of unprotected operators}
\label{sec:dimensions}
%%%%%%%%%%%%%%%%%%%%%%%%%%%%%%%%%%%%%%%%%%%%%

\begin{figure}[htbp!]
\begin{center}
	\begin{subfigure}[t]{0.5\textwidth}        
	\includegraphics[scale=0.365]{figures/Firstlongnonchiral.pdf}
	\caption{Dimension of first long multiplet in the non-chiral channel arising from different bounds.}
	\label{Fig:firstnonchirallong}
	\end{subfigure}~
	\begin{subfigure}[t]{0.5\textwidth}
	\includegraphics[scale=0.365]{figures/Firstlongchiral.pdf}
	\caption{Dimension of first long multiplet in the chiral channel arising from different bounds.}
	\label{Fig:firstchirallong}
	\end{subfigure}
\end{center}
\caption{Numerical estimates for the first scalar long operator in the non-chiral (a) and chiral (b) channels obtained from the functionals of figures \ref{Fig:c_min}, \ref{Fig:Bbound}, \ref{Fig:Ebound}, and \ref{Fig:Cl24}. The data points are color-coded according to the bound the extremal functional was extracted from, and in the cases where the bounds are plotted as a function of $c$, the functional for $c=\tfrac{11}{30}$ was used. The lines give an estimate of the extrapolation to infinitely many derivatives.}
\end{figure}

Finally, we estimate dimensions of unprotected long operators. In \cite{Beem:2014zpa,Lemos:2015awa} numerical upper bounds on the dimensions of the first long in the non-chiral and chiral channels were obtained, for various values of $c$ and  $r_0$. The best bound obtained in \cite{Beem:2014zpa} for the dimension of the first scalar long operator in the  non-chiral channel reads $\Delta_{\phi \bar{\phi}} \leqslant 2.68$, for $\Lambda=18$, $r_0=\frac65$ and $c\leqslant\tfrac{11}{30}$. On the other hand, the bound obtained for the first scalar long operator in the chiral channel was very weak and converged too slowly without further assumptions (figure~2 of \cite{Lemos:2015awa}). Removing the $\BB_{1,\frac75(0,0)}$ multiplet this bound improved to $\Delta_{\phi \phi} \leqslant 4.93 $ for $\Lambda=20$, $r_0=\frac65$, and did not appear to depend on $c$ \cite{Lemos:2015awa}.

Here, instead, we extract the dimensions of the first long $\AA_{0,0,\ell}^{\Delta_{\phi\bar{\phi}}}$ and $\AA_{0,\frac25,\ell}^{\Delta_{\phi\phi}}$ multiplets in the approximate solutions to crossing saturating the various bounds presented above. 
The results for $\ell=0$, and various values of the cutoff $\Lambda$, are given in figure \ref{Fig:firstnonchirallong} for  $\Delta_{\phi \bar{\phi}}$, and in figure \ref{Fig:firstchirallong} for $\Delta_{\phi \phi}$.
The dimensions were extracted from the extremal functionals \cite{ElShowk:2012hu} of figures \ref{Fig:Bbound}--\ref{Fig:Cl24} for $c=\tfrac{11}{30}$, and from figure \ref{Fig:c_min}, and are color coded according to the bound they came from. Note that these are the dimensions of the first long operator present in each of the extremal solutions, and are not rigorous upper bounds, which would require a different extremization problem. The lines in the figures show various extrapolations of the dimensions to $\Lambda \to \infty$.


The estimates of the dimensions of the lowest-lying long multiplet in the non-chiral channel appear all consistent with each other, even at finite $\Lambda$. This implies that, even if the various extremization problems are solved by different solutions to the crossing equations, these solutions do not differ by much as far as $\Delta_{\phi \bar{\phi}}$ is concerned, and we can take the spread of the values as an estimate for the uncertainty in the value of this dimension. Then, conservative extrapolations for $\Lambda \to \infty$ of the values coming from the various functionals give
\be 
2.56 \lesssim \Delta_{\phi \bar{\phi}} \lesssim 2.68 \,, \qquad \text{from the extrapolations as } \Lambda \to \infty\,,
\label{eq:dimextrapol}
\ee
for $\Delta_{\phi \bar{\phi}}$ in the $(A_1,A_2)$ theory.
Similarly, the dimensions of the leading twist non-chiral operators with spin $\ell >0$, obtained from the various extremal functionals of figures \ref{Fig:c_min}--\ref{Fig:Cl24}, are shown in figure \ref{Fig:anomCH} for $\Lambda=34$. We will comment on these results in section \ref{sec:nonchiral_inv}.

Less coherent are the results for $\Delta_{\phi \phi}$, the values  extracted from the extremal functionals of figures \ref{Fig:c_min}--\ref{Fig:Cl24} look very different for finite $\Lambda$, and the extrapolations are not conclusive. This is shown in  figure \ref{Fig:firstchirallong}.
Since the dimensions obtained are so disparate, it is not clear we can get any meaningful estimate for this operator in the $(A_1,A_2)$ theory. 




