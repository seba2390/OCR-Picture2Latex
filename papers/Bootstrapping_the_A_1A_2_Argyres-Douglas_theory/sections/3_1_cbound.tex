%!TEX root = ../H0bootstrap.tex
%%%%%%%%%%%%%%%%%%%%%%%%%%%%%%%%%%%%%%%%%%%%%
\subsection{Central charge bound}
\label{sec:cbound}
%%%%%%%%%%%%%%%%%%%%%%%%%%%%%%%%%%%%%%%%%%%%%

\begin{figure}[htbp!]
             \begin{center}           
              \includegraphics[scale=0.35]{figures/cmin.pdf}
              \caption{Numerical lower bound (black dots) on the central charge of theories with an $\NN=2$ chiral operator of dimension $r_{0}=\frac65$ as a function of the inverse cutoff $\Lambda$. The lines correspond to various extrapolations to infinitely many derivatives, and the horizontal dashed line marks $c=\tfrac{11}{30}$ -- the central charge of the $(A_1,A_2)$ SCFT.}
              \label{Fig:c_min}
            \end{center}
\end{figure}
%
Our first task is to obtain numerical lower bounds on the central charge $c_{min}(\Lambda)$, for fixed $r_0=\frac65$, as a function of the cutoff $\Lambda$.
The resulting bound $c_{min}(\Lambda)$ is shown in figure \ref{Fig:c_min}, together with various different extrapolations to $\Lambda \to \infty$. 
While the results are consistent with the bound converging to $c_{min}=\tfrac{11}{30}$ (the dashed line in figure \ref{Fig:c_min}), they are still not conclusive enough. In what follows we will be agnostic about the $\Lambda \to \infty$ fate of the $c$-bound, and concentrate on a region around $c\sim\tfrac{11}{30}$ in an attempt to estimate the CFT data of the $(A_1,A_2)$ theory.


