%!TEX root = ../H0bootstrap.tex
%%%%%%%%%%%%%%%%%%%%%%%%%%%%%%%%%%%%%%%%%%%%%

%%%%%%%%%%%%%%%%%%%%%%%%%%%%%%%%%%%%%%%%%%%%%
\subsection{The Superconformal Index}
%%%%%%%%%%%%%%%%%%%%%%%%%%%%%%%%%%%%%%%%%%%%%

In the non-chiral channel the OPE coefficients of all short multiplets can be fixed, as they correspond either to the stress-tensor multiplet $\hat{\CC}_{0,0}$, which contributes as \eqref{eq:STOPEcoeff}, or to multiplets containing conserved currents of spin greater than two  $\hat{\CC}_{0,\ell >0}$ which are absent in interacting theories \cite{Maldacena:2011jn,Alba:2013yda}.  
In the chiral OPE selection rules \eqref{eq:selruleschiral}, there appear short multiplets whose dimensions are fixed, but whose OPE coefficients are not known, and could even be absent in the $(A_1,A_2)$ theory. 
To access the spectrum of short operators, a useful quantity is the superconformal index \cite{Kinney:2005ej,Gadde:2011uv,Rastelli:2014jja}. The index is the most general invariant that counts, with signs, short multiplets up to multiplets that can recombine to form a long multiplet. Because short multiplets that have the right quantum numbers to recombine in a long multiplet give zero contribution, the index has some intrinsic ambiguities. Relevant to us are the following recombination rules
\be 
\begin{split}
\AA^{2R+r+2 j_2 +2}_{R,r(j_1,j_2)} &\to \CC_{R,r(j_1,j_2)} \oplus \CC_{R+\frac{1}{2},r+\frac{1}{2}(j_1-\frac{1}{2},j_2)}\,,\\
\AA^{2R+r+2 j_2 +2}_{R,r(0,j_2)} &\to \CC_{R,r(0,j_2)} \oplus \BB_{R+1,r+\frac{1}{2}(0,j_2)}\,,
\end{split}
\label{eq:recombination}
\ee
where the latter can be seen as a special case of the former with the identification $\CC_{R,r(-\frac{1}{2},j_2)} = \BB_{R+\frac{1}{2},r(0,j_2)}$.
This means that, while the multiplets $\BB_{1, 2r-1 (0,0)}$, $\CC_{0, 2r- 1 (j-1,j) }$, and $\CC_{\frac12, 2r- \frac32 (j-\frac12,j)}$ appearing in \eqref{eq:selruleschiral} contribute to the superconformal index, we can only see if these multiplets appear modulo pairs of the type \eqref{eq:recombination}. Note that the $\EE_r$, $\bar{\EE}_{-r}$ multiplets themselves can never recombine.
 
The \emph{full} superconformal index of the $(A_1,A_2)$  theory has been computed in \cite{Maruyoshi:2016tqk} using a $4d$ $\NN=1$ Lagrangian theory which flows to the $(A_1,A_2)$ theory in the IR.\footnote{Various limits of the superconformal index of Argyres-Douglas theories had been  obtained before in \cite{Buican:2015ina,Buican:2015tda,Cordova:2015nma,Song:2015wta}.}
The final expression for the index is given in integral form in equation (16) of \cite{Maruyoshi:2016tqk}, which we use to gather information about the spectrum of short multiplets in the theory. Expanding said expression we find, unsurprisingly, that the Coulomb branch chiral ring operators $\EE_{k \frac65}$, with integer $k\geqslant1$ are present (since we explored the index in an expansion this was only checked for low values of $k$). In section \ref{sec:OPEbounds} we shall bound, from above and from below, the OPE coefficient of the $\EE_{\frac{12}5}$ operator appearing in the $\EE_{\frac65}$ self OPE (see \eqref{eq:Ebound}).
From the index we also find that the operator $\CC_{0,\frac75(0,1)}$  is present in the spectrum, which corresponds to the leading-twist contribution of spin two in \eqref{eq:chiralblocks}. We shall recover this result in section \ref{sec:OPEbounds}, since we find a non-zero lower bound for its OPE coefficient (see \eqref{eq:Clbound}).
Finally, we find no contributions arising from a  $\BB_{1, \frac75(0,0)}$ multiplet implying that, modulo 
the recombination ambiguities of \eqref{eq:recombination}, it is absent in the $(A_1,A_2)$ theory. This is consistent with the expectation that such multiplets may be identified with the existence of a mixed branch \cite{Argyres:2015ffa}. Although this is not conclusive evidence for the absence of this multiplet, we take it as an indication that if there are multiple solutions to the crossing equations for $r_0=\frac65$ and $c=\tfrac{11}{30}$, then the one corresponding to the $(A_1,A_2)$ theory likely has the $\BB_{1, \frac75(0,0)}$ multiplet absent.