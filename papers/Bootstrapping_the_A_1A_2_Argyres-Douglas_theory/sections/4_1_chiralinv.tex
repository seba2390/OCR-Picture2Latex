%!TEX root = ../H0bootstrap.tex
%%%%%%%%%%%%%%%%%%%%%%%%%%%%%%%%%%%%%%%%%%%%%
\subsection{Inverting the chiral OPE}
\label{sec:chiral_inv}
%%%%%%%%%%%%%%%%%%%%%%%%%%%%%%%%%%%%%%%%%%%%%

The Lorentzian inversion formula obtained in \cite{Caron-Huot:2017vep} can be directly applied to invert the $s-$channel OPE of the correlator \eqref{eq:chiral},
\be 
\langle \phi(x_1) \phi(x_2) \bar{\phi}(x_3) \bar{\phi}(x_4) \rangle = \frac{1}{x_{12}^{2 \Delta_\phi}x_{34}^{2 \Delta_\phi}} \sum\limits_{\OO_{\Delta,\ell}} |\lambda_{\phi \phi \OO_{\Delta,\ell}}|^2 g_{\Delta,\ell}^{0,0}(u,v)\,,
\label{eq:chiralcorr}
\ee 
as it is exactly of the form \eqref{eq:CHfourpoint}, with $\GG(z,\zb)$ admitting a decomposition in bosonic blocks, with $\Delta_{12}=\Delta_{34}=0$. We can thus apply \eqref{eq:ctandcu} directly, with the $t-$ and $u-$channel decompositions as dictated by crossing symmetry of \eqref{eq:chiralcorr}. Since the operators at point one and two are identical, the $u-$ and $t-$ channels give identical contributions, and thus only even spins appear in the $s-$channel OPE of \eqref{eq:chiralcorr}, precisely in agreement with Bose symmetry.

We now want to make use of the generating functional \eqref{eq:CHgeneratingbos}, to obtain the dimensions and OPE coefficients of the $s-$channel operators (at least for large enough spin) by providing information about the $t-$channel decomposition.
For large spin (that is large $\beta$) the leading contributions in \eqref{eq:CHgeneratingbos} come from the $\zb \to 1$ limit of the integrand, with the leading contribution corresponding to the lowest twist operators exchanged in the $t-$channel.
The $t-$channel decompositions are given by the non-chiral OPE, as follows from \eqref{eq:chiralnonchiral}.
From \eqref{eq:selrulesnonchiral} we see that, after the identity, the leading contribution comes from the superconformal multiplet of the stress tensor $\hat{\CC}_{0,0}$, and, since we are interested on interacting theories, there is no other contribution with the same twist. The next contributions will arise from long multiplets, for which we only currently have the numerical estimates for their dimensions  obtained in section \ref{sec:dimensions}. At large spin the contributions of one of these operators of twist $\tau$ behaves as $\ell^{-\tau}$ \cite{Fitzpatrick:2012yx,  Caron-Huot:2017vep}.\footnote{Here we are using the bosonic results of \cite{Fitzpatrick:2012yx,  Caron-Huot:2017vep}, while we have a superblock contribution at twist $\tau$. However, decomposing the superblock in bosonic blocks we find a finite number of bosonic blocks with twist $\tau$ together with a finite number of higher twist, and so the presence  of the superblock will only modify the coefficient of the leading behavior for large $\ell$, which is unimportant for our point here.}
The leading twist operators have been estimated numerically from the various extremal functionals obtained in section \ref{sec:numericsh0}. The leading spin zero operator could have twist as low as $\tau \sim 2.5$, while the higher spin operators (figure \ref{Fig:anomCH}) all appear to have twists close to $2r_0=2.4$, with small corrections depending on the spin. This is to be compared with the contribution of the stress tensor with exactly $\tau=2$.
For sufficiently large spin the contributions of long multiplets are subleading,
so in what follows we shall consider only the stress-tensor and identity exchanges in the $t-$ and $u-$channels.

The identity and stress tensor contribute to \eqref{eq:CHgeneratingbos} according to the crossing equation \eqref{eq:chiralnonchiral}, with the identity contributing as $|\lambda_{\phi \bar{\phi} 1}|^2 \tilde{\GG}_{0,0}(u,v)=1$. The stress-tensor superblock is given by \eqref{eq:superblockbraid} with $\Delta=2$, $\ell=0$, and with OPE coefficient given by \eqref{eq:STOPEcoeff}, and we  find
%
\be 
C^t(z,\beta) \supset \int\limits_0^1 \frac{\d \zb}{\zb^2} \kappa_{\beta} k_\beta^{\Delta_{12}, \Delta_{34}}(\zb) \dDisc\left[\frac{(z \zb)^{r_0}}{((1-z) (1-\zb))^{r_0}} \left( 1+ |\lambda_{\phi \bar{\phi} \hat{\CC}_{0,0}}|^2 \tilde{\GG}_{2,0}(1-z,1-\zb)\right)\right]\,.
\label{eq:nonchiral_lead}
\ee
%

From the identity contribution, which is the leading one for large spin, we recover the existence of double-twist operators $[\phi\phi]_{m,\ell}$ (see for example section 4.2 of \cite{Caron-Huot:2017vep}), namely operators with dimensions approaching
%
\be 
\Delta_{[\phi\phi]_{m,\ell}} \underset{\ell \gg 1}{\longrightarrow} 2 r_0+2m + \ell\,, \qquad \ell \; \mathrm{even}\,,
\ee
%
and with OPE coefficients approaching those of generalized free field theory,
%
\be 
\lambda_{\mathrm{gfft}}^2 = \frac{\left((-1)^{\ell }+1\right) \left(\left(r_0-1\right)_m\right){}^2 \left(\left(r_0\right)_{m+\ell }\right){}^2}{m! \ell ! \left(m+2 r_0-3\right)_m (\ell +2)_m \left(m+\ell +2
   r_0-2\right)_m \left(2 m+\ell +2 r_0-1\right)_{\ell }}\,,
 \label{eq:gfftOPEcoeff}
\ee
%
at large spin. In \eqref{eq:gfftOPEcoeff} $(a)_b$ denotes the Pochhammer symbol.

To compute the leading correction to these dimensions and OPE coefficients at large spin we take into account the contribution of the stress-tensor multiplet to the OPE. 
To do so we take the $z \to 0 $ limit of \eqref{eq:nonchiral_lead}; as pointed in \cite{Caron-Huot:2017vep}, the correct procedure should be to subtract a known sum, such that the limit $z \to 0 $ commutes with the infinite sum over $t-$channel primaries.
However, when anomalous dimensions are small this procedure gives small corrections to the naive one of taking a series expansion in $z$ and extracting anomalous dimensions from the terms proportional to $\log z$  (the generating function should have $z^{\gamma/2} \approx 1+ \tfrac{1}{2} \gamma \, \log z + \ldots $) and corrections to OPE coefficients from the terms without $\log z$. For the case considered below the situation is even better as the anomalous dimensions of the operators we are interested in vanish.
%
Taking the small $z$ limit, the first observation is that anomalous dimensions, \ie, log-terms, only come with a power of $z^{\Delta_\phi+2}$, and thus only the operators $[\phi\phi]_{m \geqslant 2,\ell}$ acquire an anomalous dimension. This is consistent with the fact that from the block decomposition \eqref{eq:chiralblocks} we identify the double-twist operators with $m=0,1$ as short multiplets, $\CC_{0, 2r_0- 1 (\frac{\ell}2-1,\frac{\ell}2) } $ and $\CC_{\frac12, 2r_0- \frac32 (\frac{\ell}2-\frac12,\frac{\ell}2)}$ respectively, whose dimensions are protected.\footnote{We assume $\ell\geqslant 2$ here since the inversion formula is not guaranteed to converge for $\ell=0$.}

\begin{figure}[htb!]
             \begin{center}           
              \includegraphics[scale=0.35]{figures/ClCH.pdf} \; \includegraphics[scale=0.35]{figures/ClCH2.pdf}
              \caption{Comparison between the numerical bounds on the OPE coefficient squared of the leading twist operators in the chiral channel and the results from  the inversion formula \eqref{eq:nonchiral_lead}, for $c=\tfrac{11}{30}$ and external dimension $r_0=\frac65$. The black boxes mark the numerically allowed range for the squared OPE coefficients of the $\mathcal{C}_{0,\frac75,\left(\frac{\ell }{2}-1,\frac{\ell }{2}\right)}$ operators for different values of $\ell$ (the $\ell=0$ operator should be interpreted as $\EE_{\frac{12}5}$) and with $\Lambda=36$. The dashed line shows the result of equation \eqref{eq:nonchiral_lead}, where we considered only the contribution of the identity and stress-tensor operators in the non-chiral channel, and thus is an approximate result for sufficiently large spin. The formula \eqref{eq:nonchiral_lead} is not guaranteed to be valid for $\ell=0$, and the results here are just shown as an illustration.}
              \label{Fig:ClCH}
            \end{center}
\end{figure}
%

We can now compute corrections to the OPE coefficient of the $\mathcal{C}_{0,2 r_0-1,\left(\frac{\ell }{2}-1,\frac{\ell }{2}\right)}$ operators, for $r_0=\frac65$ and $c=\tfrac{11}{30}$, from \eqref{eq:nonchiral_lead}.
The result is plotted in figure \ref{Fig:ClCH}, where we performed the integral in \eqref{eq:nonchiral_lead} numerically (after taking the leading $z$ term), together with the numerical upper and lower bonds on the OPE coefficients, obtained in section \ref{sec:OPEbounds}.\footnote{Note that by the usual lightcone methods \cite{Fitzpatrick:2012yx,Komargodski:2012ek,Alday:2015ewa}, we could obtain an asymptotic expansion in $\tfrac{1}{\ell}$ of the correction to the generalized free field theory OPE coefficients \eqref{eq:gfftOPEcoeff} arising from the stress tensor exchange.  By considering the contributions of the stress tensor superblock to \eqref{eq:CHgeneratingbos} we are effectively re-summing the lightcone expansion to all orders.}
The results for $\ell=0$ (where the multiplet becomes an $\EE_{2r_0}$) are also shown even though the formula is only guaranteed to be valid for $\ell \geqslant 2$.\footnote{The formula could only be valid for $\ell=0$ if, for some reason, the growth of the four-point function of the $(A_1,A_2)$ theory, in the limit relevant for the dropping of arcs of integration along the derivation of the inversion formula, was better than the generic growth expected in any CFT and derived in \cite{Caron-Huot:2017vep}.}



We point out that the only input was the leading $t-$ and $u-$channels contributions and thus the resulting OPE coefficients are an approximation for sufficient large spin. Indeed, the neglected contribution of the long multiplets should behave like $\ell^{-\tau}$ for large spin, with $\tau \sim 2.4$, while the stress tensor contributes as $\tau=2$. Nevertheless, we see in figure \ref{Fig:ClCH} that starting from $\ell=4$ the analytical result is already inside the numerically allowed range for the OPE coefficient. This is shown clearly in figure \ref{Fig:Cl4} where the result of \eqref{eq:nonchiral_lead} for $\ell=4$ is shown as a dashed blue line, together with the numerically allowed range. For $\ell=2$, however, the result of \eqref{eq:nonchiral_lead} (blue dashed line in figure \ref{Fig:Cl2}) is clearly insufficient, as it is outside the numerically allowed region.
Note that the numerical results are not optimal yet, \ie, while they provide true bounds they have not yet converged, and the optimal bounds will be more restrictive. Thus, the fact that the $\ell=4$ estimate was inside the numerical bounds should not be taken to mean the subleading contributions are negligible for such a low spin. What is in fact surprising is that the estimates from \eqref{eq:nonchiral_lead} are so close to the numerically obtained ranges for such low values of the spin. These results leave us optimistic that better estimates can be obtained by providing a few of the subleading contributions, as was done in \cite{Simmons-Duffin:2016wlq} for the $3d$ Ising model.
The computation used to obtain figure \ref{Fig:ClCH} could be easily extended to obtain estimates for the OPE coefficients of the 
$\CC_{\frac12, 2r_0- \frac32 (\frac{\ell}2-\frac12,\frac{\ell}2)}$  multiplets, and also the dimensions and OPE coefficients of the remaining operators in \eqref{eq:selruleschiral}. One particularly interesting multiplet would be $\BB_{1, \frac75 (0,0)}$ since, as discussed before, we expect it to be absent in the $(A_1,A_2)$ theory. However, this corresponds to a spin zero contribution and thus convergence of the inversion formula is not guaranteed.
