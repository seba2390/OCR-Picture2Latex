%!TEX root = ../H0bootstrap.tex
%%%%%%%%%%%%%%%%%%%%%%%%%%%%%%%%%%%%%%%%%%%%%
\section{Introduction and summary}
\label{sec:intro}
%%%%%%%%%%%%%%%%%%%%%%%%%%%%%%%%%%%%%%%%%%%%%

The revival of the conformal bootstrap program \cite{Rattazzi:2008pe}, has provided new tools to study non-perturbative physics. The numerical techniques introduced in \cite{Rattazzi:2008pe} have given a wealth of results, with the most impressive being the high-precision estimates of the critical exponents of the $3d$ Ising model \cite{ElShowk:2012ht,El-Showk:2014dwa,Kos:2014bka,Simmons-Duffin:2015qma,Kos:2016ysd}. 
In a parallel line of development, analytic approaches to the bootstrap have also been explored, and recent progress has given access to the spectrum of conformal field theories (CFTs) at large spin by means of the lightcone limit \cite{Fitzpatrick:2012yx,Komargodski:2012ek}.
These two methods were combined in \cite{Alday:2015ota,Simmons-Duffin:2016wlq}, where knowledge of
operator dimensions and operator product expansion (OPE) coefficients, obtained numerically for the Ising model, was used to derive analytic approximations for the CFT data at large spin. Remarkably, the analytic results obtained matched the numerical data down to spin two.


The \emph{superconformal} bootstrap has also seen substantial progress. Apart from numerical explorations of crossing symmetry \cite{Poland:2010wg,Poland:2011ey,Poland:2015mta,Li:2017ddj,Berkooz:2014yda,
Beem:2014zpa,Beem:2013qxa,Alday:2013opa,Alday:2014qfa,
Chester:2014fya,Chester:2014mea,Lemos:2015awa,Chester:2015qca,
Bobev:2015vsa,Bobev:2015jxa,Bae:2016jpi,Beem:2015aoa,Lin:2015wcg,Lin:2016gcl,Beem:2016wfs,Lemos:2016xke,
Cornagliotto:2017dup,Chang:2017xmr,Chang:2017cdx}, the bootstrap line of thinking helped uncover a solvable subsector in four-dimensional superconformal theories \cite{Beem:2013sza}.\footnote{See also \cite{Beem:2014kka} and \cite{Chester:2014mea,Beem:2016cbd} for similar results in six and three dimensions.} More precisely, the results of \cite{Beem:2013sza} imply that any $4d$ $\NN \geqslant 2$ superconformal field theory (SCFT) contains a closed subsector isomorphic to a $2d$ chiral algebra.

The subsector captures certain protected quantities, and in order to access non-protected data the numerical bootstrap is still a necessary tool. In some cases, similarly to the $3d$ Ising model and $O(N)$ models, known supersymmetric theories appear on special points, ``kinks'', of the numerically produced exclusion curves, and the numerical machinery of \cite{Rattazzi:2008pe} can be applied in order to extract the CFT data.
%
However, kinks seem to be scarce, in particular, while the numerical bounds for $4d$ $\NN=2$ SCFTs obtained in \cite{Beem:2014zpa,Lemos:2015awa} put strong constraints on the landscape of theories, they did not single out any particular solution to crossing. 

\medskip

In this work we focus on the ``simplest'' four-dimensional $\NN=2$ Argyres-Douglas SCFT: the $(A_1,A_2)$ (or $H_0$) theory \cite{Argyres:1995jj,Argyres:1995xn}. It has the lowest possible $c$-anomaly coefficient among interacting SCFTs \cite{Liendo:2015ofa}, and the lowest $a$-anomaly coefficient among the known ones. The $(A_1,A_2)$ SCFT can be realized by going to a special point on the Coulomb branch of an $\NN=2$ supersymmetric gauge theory, with gauge group $\SU(3)$, where electric and magnetic particles become simultaneously massless \cite{Argyres:1995jj,Argyres:1995xn}.
It is an isolated $\NN=2$ SCFT, with no exactly marginal deformations, and thus no weak-coupling description. As such, despite being known for a very long time, little is known about the spectrum of this theory.
Known data includes the scaling dimension, $\Delta_{\phi}$, of the single generator of the Coulomb branch chiral ring, whose vev parametrizes the Coulomb branch, and the $a$- and $c$-anomaly coefficients \cite{Aharony:2016kai}:
\be 
\Delta_{\phi} = \frac{6}{5}\, , \qquad c = \frac{11}{30}\, , \qquad a = \frac{43}{120}\, .
\label{eq:acanomalies_dimension}
\ee
The full superconformal index \cite{Kinney:2005ej,Gadde:2011uv,Rastelli:2014jja} was recently computed using an $\NN=1$ Lagrangian that flows to the $(A_1,A_2)$ SCFT in the IR \cite{Maruyoshi:2016tqk}. 
The chiral algebra of this theory is conjectured to be the Yang-Lee minimal  model \cite{rastelli_harvard,Beem:2017ooy}, which gives access to the spectrum of a particular class of short operators, dubbed ``Schur'' operators.
%
However, the chiral algebra is insensitive to the Coulomb branch data of the theory, and even though the dimensions of the  operators parameterizing the Coulomb branch chiral ring are known, not much is known about the values of the corresponding three-point functions.\footnote{See \cite{Hellerman:2017sur} for a recent computation of the two-point function (in normalizations where the OPE coefficients are one) of a Coulomb branch chiral ring operator, for theories with a single chiral ring generator, in the limit of large $\U(1)_r$ charge.}


The relatively low values of its central charge and of the dimension of its Coulomb branch chiral ring generator make the $(A_1,A_2)$ Argyres-Douglas theory amenable to numerical bootstrap techniques. 
In fact, one could argue that this is the $\NN=2$ SCFT with the best chance to be ``solved'' numerically.
We approach this theory based on the existing Coulomb branch data, by considering four-point functions of $\NN=2$ chiral operators, whose superconformal primaries are identified with the elements of the Coulomb branch chiral ring.\footnote{Another natural operator to consider in the correlation functions would be the $\NN=2$ stress-tensor multiplet, however, the superconformal blocks for this multiplet are not known, and we leave this for future work.}
While the values of $c$ and $\Delta_\phi$ in \eqref{eq:acanomalies_dimension} are not selected by the numerical bootstrap, thanks to supersymmetry they are exactly known and thus we can use them as input in our analysis.
We note however, that nothing is known about the spectrum of 
non-supersymmetry preserving
relevant deformations of the $(A_1,A_2)$ theory, and this type of information was essential to corner the $3d$ Ising model to a small ``island'' \cite{Kos:2014bka}. 

\bigskip

The results we find are encouraging, and provide the first estimates for unprotected quantities in this theory.
We start by obtaining a lower bound on the central charge valid for any $\NN=2$ theory with a Coulomb branch chiral ring operator of dimension $\Delta_{\phi} = \frac65$. This bound appears to be converging to a value \emph{close} to $c=\tfrac{11}{30}$, however the numerics are not conclusive enough. If the bound on $c$ converges to $\tfrac{11}{30}$, then there is a unique solution to the crossing equations at $\Delta_{\phi} = \frac65$ that corresponds to the $(A_1,A_2)$ theory.
If the numerical bound falls short of $\tfrac{11}{30}$, we present evidence, in the form of valid bounds on OPE coefficients and estimates on operator dimensions, that the various solutions around $c \sim \tfrac{11}{30}$ do not look so different, as far as certain observables are concerned. 
While the results we obtain are not at the level of the precision numerics of the $3d$ Ising model, 
we are able to provide estimates for the CFT data of this theory.
For example, we constrain the OPE coefficient of 
the square of the Coulomb branch chiral ring generator (after unit normalizing its two-point function) to lie in the interval
\be
2.1418 \leqslant \lambda_{\EE_{\frac{12}5}}^2 \leqslant 2.1672 \,.
\label{eq:Ebound_intro}
\ee
While this is a true bound, due to slow convergence it is still far from being optimal, and will improve as more of the constraints of the crossing equations are taken into account. In section \ref{sec:OPEbounds} we present estimates for the optimal range, based on conservative extrapolations of the bounds.
Similarly, we constrain the OPE coefficients of a family of semi-short multiplets,
 appearing in the self-OPE of $\NN=2$ chiral operators, to lie in a narrow range, quoted in \eqref{eq:Clbound} for $\ell=2,4$, and in figure \ref{Fig:ClCH} for even spins up to $\ell=20$.

We also provide in \eqref{eq:dimextrapol} the first estimate of the dimension of the lowest-lying unprotected scalar appearing in the  OPE of the $\NN=2$ chiral operator with its conjugate. This operator corresponds to a long multiplet that is a singlet under $\SU(2)_R$ symmetry, and neutral under $U(1)_r$, and we find it is relevant.
These estimates are obtained from the extremal functionals \cite{ElShowk:2012hu} that gave rise to the aforementioned OPE coefficient bounds. From these extremal functionals we also obtain rough estimates for the dimensions of the lowest-twist long operator for higher values of the spin, shown in figure \ref{Fig:anomCH}. Surprisingly, for spin greater than zero these operators are very close to being double-twist operators, \ie, $\Delta = 2 \Delta_\phi + \ell$.

Finally, we make use of the inversion formula of \cite{Caron-Huot:2017vep} to obtain large-spin estimates of the CFT data. As our numerical results are much further away from convergence than \cite{Simmons-Duffin:2016wlq}, we refrain from using them as input in the inversion formula. As such the only input we provide is the identity and stress-tensor supermultiplet exchange (with the appropriate central charge). Interestingly, we find that this input already provides a reasonable estimate of the numerically-bounded quantities for small spin. 

A hybrid approach, combining both the numerical bootstrap and the inversion formula seems to be the most promising way to proceed, perhaps along the lines of the one suggested in \cite{Simmons-Duffin:2016wlq}. The results of this paper are a first step in this direction, and give us hope that a large amount of CFT data can be bootstrapped for the $(A_1,A_2)$ theory.