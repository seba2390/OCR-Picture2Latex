%!TEX root = ../H0bootstrap.tex
%%%%%%%%%%%%%%%%%%%%%%%%%%%%%%%%%%%%%%%%%%%%%
\section{Inverting the OPEs}
\label{sec:anlytical}
%%%%%%%%%%%%%%%%%%%%%%%%%%%%%%%%%%%%%%%%%%%%%

In recent years, starting from \cite{Fitzpatrick:2012yx,Komargodski:2012ek}, there has been much progress in understanding the large spin spectrum of CFTs by studying analytically the crossing equations in a Lorentzian limit.
In $d >2$, by studying 
the four-point function $\langle \phi_1(x_1) \phi_1(x_2) \phi_2(x_3) \phi_2(x_4) \rangle$ in the lightcone limit, the authors of \cite{Fitzpatrick:2012yx,Komargodski:2012ek} found that in the $t-$channel there must be exchanged, in a distributional sense, double-twist operators, \ie, operators whose dimensions approach $\Delta_1 + \Delta_2 + \ell$ for $\ell \to \infty$, and whose OPE coefficients tend to the values of generalized free field theory. Corrections to these dimensions and OPE coefficients, or rather weighted averages of these quantities, in a large spin expansion can be obtained in terms of the leading twist operators exchanged in the $s-$channel.\footnote{Some of the steps taken in the derivations \cite{Fitzpatrick:2012yx,Komargodski:2012ek} rely on intuitive assumptions, some of which have started to be put on a firm footing in \cite{Qiao:2017xif}.} 
Assuming the existence of individual operators close to the average values, this procedure was set up to systematically compute the OPE coefficients and dimensions of the double-twist operators in an asymptotic expansion in the inverse spin \cite{Alday:2015ewa}.
The lightcone limit of the crossing equations has been used to constrain the large spin spectrum of various CFTs with different global symmetries and supersymmetries \cite{Fitzpatrick:2012yx,Komargodski:2012ek,Alday:2013cwa,Fitzpatrick:2014vua,Vos:2014pqa,Fitzpatrick:2015qma,Kaviraj:2015cxa,Alday:2015eya,Kaviraj:2015xsa,Alday:2015ota,Beem:2015aoa,1612.03891,1603.05150,1611.01500,1612.00696,1510.08091,1602.04928,1511.08025}.
Remarkably, this has resulted in predictions for OPE coefficients and anomalous dimensions of operators that match the numerical results down to spin two \cite{Simmons-Duffin:2016wlq, Alday:2015ota}.

Recently, the work of \cite{Caron-Huot:2017vep} has explained this agreement, by showing that the spectrum organizes in analytic families.\footnote{We thank Marco Meineri for many discussions on \cite{Caron-Huot:2017vep}.} There, a ``Lorentzian'' inversion formula for the $s-$channel OPE of a given correlator was obtained, with the crucial feature that the result of the inversion is a function that is analytic in spin (a function valid only for spin greater than one). 
This established, for sufficiently large spin, the existence of each individual double-twist operator.
The inversion formula explained the organization of the spectrum, and allows one to compute individual OPE coefficients and anomalous dimensions, avoiding the asymptotic expansions, and obtaining the coefficients instead of averages.


\medskip


Motivated by the success of \cite{Simmons-Duffin:2016wlq}, we take the first steps towards a systematic analysis of the $(A_1,A_2)$ theory for large spin, considering both crossing equations \eqref{eq:crossingeqs}.
We apply the inversion formula obtained in \cite{Caron-Huot:2017vep}  to invert the chiral \eqref{eq:chiral} and non-chiral \eqref{eq:nonchiral} OPEs. The block decomposition of the former happens to be simply a decomposition in bosonic blocks \eqref{eq:chiralblocks}, and thus the inversion formula directly applies. The latter has a decomposition in superblocks, but as we shall see, the formula can still be applied, although we must work as if we had a correlator of unequal external operators. The only required modifications will be on the spin down to which the formula holds, and on what the crossed-channel decompositions are.
Since the numerical results in supersymmetric theories are not yet at the level of accuracy of the $3d$ Ising model, we refrain from using numerical data as input to the analysis, and instead compare the large-spin estimates coming from the inversion formula with the numerical results. The only input we give is the exchange of the identity and stress-tensor supermultiplet in the non-chiral OPE, and thus find results that are good estimates for sufficiently large spin. We find a reasonable agreement between the OPE coefficients of the leading-twist short operators exchanged in the chiral channel, and the analytical estimate, already for low spin, see figure \ref{Fig:ClCH}. Our analysis also shows that the anomalous dimensions of the double-twist operators in the non-chiral channel, arising from the stress tensor exchange, are small, and we confirm this by matching to numerical estimates of these dimensions obtained from the bounds of section \ref{sec:numericsh0} (see figure \ref{Fig:anomCH}).



We start with a brief summary of the results of \cite{Caron-Huot:2017vep} relevant for our purposes and refer the reader to that reference for further details. 
Starting from  the four-point function of unequal scalar operators,
\be 
\langle \OO_1(x_1)\OO_2(x_2)\OO_3(x_3)\OO_4(x_4) \rangle =\frac{1}{|x_{12}|^{\Delta_1+\Delta_2}|x_{34}|^{\Delta_3+\Delta_4}} \left|\frac{x_{14}}{x_{24}}\right|^{\Delta_{21}}  \left|\frac{x_{14}}{x_{13}}\right|^{\Delta_{34}} \GG(z, \zb)\,,
\label{eq:CHfourpoint}
\ee
the main result is a ``Lorentzian'' inversion formula for the $s-$channel decomposition of $\GG(z,\zb)$ in conformal blocks,
\be 
\GG(z, \zb) = \sum\limits_{\Delta,\ell} \lambda_{\Delta,\ell}^2 g^{\Delta_{12},\Delta_{34}}_{\Delta,\ell}(z, \zb)\,.
\label{eq:Gdecbos}
\ee
The OPE coefficients $\lambda_{\Delta,\ell}^2$ (for $\ell >1$) of the above decomposition \eqref{eq:Gdecbos} are then encoded in the residues of a function $c(\ell,\Delta)$  that is analytic in spin, contrasting with the one that can be obtained from a Euclidean inversion of the OPE. The condition $\ell >1$ arises during the contour manipulations needed to go from the  Euclidean inversion of the OPE, valid only for integer $\ell$, to the ``Lorentzian'' formula of \cite{Caron-Huot:2017vep}. This condition requires looking at the $t-$ and $u-$ channels to bound the growth of $\GG(z,\zb)$  in a particular region, and it is valid for any unitary CFT.
The function $c(\ell, \Delta) $ receives contributions from the $t-$ and  $u-$channels, with the even and odd spin operators defining two independent trajectories, as 
\be 
c(\ell, \Delta) = c^t (\ell, \Delta) + (-1)^\ell c^u(\ell, \Delta)\,,
\label{eq:ctandcu}
\ee
where $c^t$ and $c^u$ are defined in (3.20) of \cite{Caron-Huot:2017vep}.

The poles of $c(\ell, \Delta)$ in $\Delta$, at fixed $\ell$, encode the dimensions of the operators in the theory, with the residues giving the  OPE coefficients.\footnote{In some cases the residues need to be corrected as discussed in (3.9) of \cite{Caron-Huot:2017vep}, but for the computations carried out in this section this correction is not needed.} As described in section 3.2 of \cite{Caron-Huot:2017vep}, if one is interested only in getting the poles and residues, the inversion formula can be written as
\be
\begin{split}
c^t(\ell,\Delta)\big\vert_{\mathrm{poles}} &= \int\limits_0^1 \frac{\d z}{2z} z^{\frac{\ell-\Delta}{2}} \left( \sum\limits_{m=0}^{\infty} z^m \sum\limits_{k=-m}^{m} B_{\ell,\Delta}^{(m,k)} C^t(z, \ell+\Delta+2k)\right)\,, \\
C^t(z, \beta)&= \int\limits_z^1 \d \zb \frac{(1-\zb)^{\frac{\Delta_{21}+\Delta_{34}}{2}}}{\zb^2} \kappa_{\beta} k_\beta^{\Delta_{12}, \Delta_{34}}(\zb) \dDisc\left[\GG(z,\zb)\right]\,,
\end{split}
\label{eq:CHgeneratingbos}
\ee
and similarly for $c^u(\ell,\Delta)$.
Here  $\dDisc$ denotes the double-discontinuity of the function, 
$k_\beta^{\Delta_{12}, \Delta_{34}}(\zb)$ is defined in equation \eqref{eq:bosblock}, and 
\be 
\kappa_{\beta}= \frac{\Gamma \left(\frac{\beta -\Delta_{21}}{2}\right) \Gamma \left(\frac{\Delta_{21}+\beta }{2}\right) \Gamma \left(\frac{\beta-\Delta_{34} }{2}\right) \Gamma \left(\frac{\Delta_{34} + \beta }{2}\right)}{2 \pi ^2 \Gamma (\beta -1) \Gamma (\beta )} \,.
\ee
The $z \to 0$ limit of the block in \eqref{eq:CHgeneratingbos} gave the collinear block $k_\beta^{\Delta_{12}, \Delta_{34}}(\zb)$ which does not take into account all descendants, these are instead taken into account by the functions $B_{\ell,\Delta}^{(m,k)}$, as discussed in \cite{Caron-Huot:2017vep}. Since we shall focus only on leading twist operators we do not need to subtract descendants and thus do not need these functions, apart from $B_{\ell,\Delta}^{(0,0)}=1$.
A term $z^{\frac{\tau(\Delta+\ell)}{2}}$ in the bracketed term in \eqref{eq:CHgeneratingbos}  implies there exists a pole at $\Delta-\ell = \tau(\Delta+\ell)$, with its residue, taken at fixed $\ell$, providing the OPE coefficient;  see \cite{Caron-Huot:2017vep} for more details.

%!TEX root = ../H0bootstrap.tex
%%%%%%%%%%%%%%%%%%%%%%%%%%%%%%%%%%%%%%%%%%%%%
\subsection{Inverting the chiral OPE}
\label{sec:chiral_inv}
%%%%%%%%%%%%%%%%%%%%%%%%%%%%%%%%%%%%%%%%%%%%%

The Lorentzian inversion formula obtained in \cite{Caron-Huot:2017vep} can be directly applied to invert the $s-$channel OPE of the correlator \eqref{eq:chiral},
\be 
\langle \phi(x_1) \phi(x_2) \bar{\phi}(x_3) \bar{\phi}(x_4) \rangle = \frac{1}{x_{12}^{2 \Delta_\phi}x_{34}^{2 \Delta_\phi}} \sum\limits_{\OO_{\Delta,\ell}} |\lambda_{\phi \phi \OO_{\Delta,\ell}}|^2 g_{\Delta,\ell}^{0,0}(u,v)\,,
\label{eq:chiralcorr}
\ee 
as it is exactly of the form \eqref{eq:CHfourpoint}, with $\GG(z,\zb)$ admitting a decomposition in bosonic blocks, with $\Delta_{12}=\Delta_{34}=0$. We can thus apply \eqref{eq:ctandcu} directly, with the $t-$ and $u-$channel decompositions as dictated by crossing symmetry of \eqref{eq:chiralcorr}. Since the operators at point one and two are identical, the $u-$ and $t-$ channels give identical contributions, and thus only even spins appear in the $s-$channel OPE of \eqref{eq:chiralcorr}, precisely in agreement with Bose symmetry.

We now want to make use of the generating functional \eqref{eq:CHgeneratingbos}, to obtain the dimensions and OPE coefficients of the $s-$channel operators (at least for large enough spin) by providing information about the $t-$channel decomposition.
For large spin (that is large $\beta$) the leading contributions in \eqref{eq:CHgeneratingbos} come from the $\zb \to 1$ limit of the integrand, with the leading contribution corresponding to the lowest twist operators exchanged in the $t-$channel.
The $t-$channel decompositions are given by the non-chiral OPE, as follows from \eqref{eq:chiralnonchiral}.
From \eqref{eq:selrulesnonchiral} we see that, after the identity, the leading contribution comes from the superconformal multiplet of the stress tensor $\hat{\CC}_{0,0}$, and, since we are interested on interacting theories, there is no other contribution with the same twist. The next contributions will arise from long multiplets, for which we only currently have the numerical estimates for their dimensions  obtained in section \ref{sec:dimensions}. At large spin the contributions of one of these operators of twist $\tau$ behaves as $\ell^{-\tau}$ \cite{Fitzpatrick:2012yx,  Caron-Huot:2017vep}.\footnote{Here we are using the bosonic results of \cite{Fitzpatrick:2012yx,  Caron-Huot:2017vep}, while we have a superblock contribution at twist $\tau$. However, decomposing the superblock in bosonic blocks we find a finite number of bosonic blocks with twist $\tau$ together with a finite number of higher twist, and so the presence  of the superblock will only modify the coefficient of the leading behavior for large $\ell$, which is unimportant for our point here.}
The leading twist operators have been estimated numerically from the various extremal functionals obtained in section \ref{sec:numericsh0}. The leading spin zero operator could have twist as low as $\tau \sim 2.5$, while the higher spin operators (figure \ref{Fig:anomCH}) all appear to have twists close to $2r_0=2.4$, with small corrections depending on the spin. This is to be compared with the contribution of the stress tensor with exactly $\tau=2$.
For sufficiently large spin the contributions of long multiplets are subleading,
so in what follows we shall consider only the stress-tensor and identity exchanges in the $t-$ and $u-$channels.

The identity and stress tensor contribute to \eqref{eq:CHgeneratingbos} according to the crossing equation \eqref{eq:chiralnonchiral}, with the identity contributing as $|\lambda_{\phi \bar{\phi} 1}|^2 \tilde{\GG}_{0,0}(u,v)=1$. The stress-tensor superblock is given by \eqref{eq:superblockbraid} with $\Delta=2$, $\ell=0$, and with OPE coefficient given by \eqref{eq:STOPEcoeff}, and we  find
%
\be 
C^t(z,\beta) \supset \int\limits_0^1 \frac{\d \zb}{\zb^2} \kappa_{\beta} k_\beta^{\Delta_{12}, \Delta_{34}}(\zb) \dDisc\left[\frac{(z \zb)^{r_0}}{((1-z) (1-\zb))^{r_0}} \left( 1+ |\lambda_{\phi \bar{\phi} \hat{\CC}_{0,0}}|^2 \tilde{\GG}_{2,0}(1-z,1-\zb)\right)\right]\,.
\label{eq:nonchiral_lead}
\ee
%

From the identity contribution, which is the leading one for large spin, we recover the existence of double-twist operators $[\phi\phi]_{m,\ell}$ (see for example section 4.2 of \cite{Caron-Huot:2017vep}), namely operators with dimensions approaching
%
\be 
\Delta_{[\phi\phi]_{m,\ell}} \underset{\ell \gg 1}{\longrightarrow} 2 r_0+2m + \ell\,, \qquad \ell \; \mathrm{even}\,,
\ee
%
and with OPE coefficients approaching those of generalized free field theory,
%
\be 
\lambda_{\mathrm{gfft}}^2 = \frac{\left((-1)^{\ell }+1\right) \left(\left(r_0-1\right)_m\right){}^2 \left(\left(r_0\right)_{m+\ell }\right){}^2}{m! \ell ! \left(m+2 r_0-3\right)_m (\ell +2)_m \left(m+\ell +2
   r_0-2\right)_m \left(2 m+\ell +2 r_0-1\right)_{\ell }}\,,
 \label{eq:gfftOPEcoeff}
\ee
%
at large spin. In \eqref{eq:gfftOPEcoeff} $(a)_b$ denotes the Pochhammer symbol.

To compute the leading correction to these dimensions and OPE coefficients at large spin we take into account the contribution of the stress-tensor multiplet to the OPE. 
To do so we take the $z \to 0 $ limit of \eqref{eq:nonchiral_lead}; as pointed in \cite{Caron-Huot:2017vep}, the correct procedure should be to subtract a known sum, such that the limit $z \to 0 $ commutes with the infinite sum over $t-$channel primaries.
However, when anomalous dimensions are small this procedure gives small corrections to the naive one of taking a series expansion in $z$ and extracting anomalous dimensions from the terms proportional to $\log z$  (the generating function should have $z^{\gamma/2} \approx 1+ \tfrac{1}{2} \gamma \, \log z + \ldots $) and corrections to OPE coefficients from the terms without $\log z$. For the case considered below the situation is even better as the anomalous dimensions of the operators we are interested in vanish.
%
Taking the small $z$ limit, the first observation is that anomalous dimensions, \ie, log-terms, only come with a power of $z^{\Delta_\phi+2}$, and thus only the operators $[\phi\phi]_{m \geqslant 2,\ell}$ acquire an anomalous dimension. This is consistent with the fact that from the block decomposition \eqref{eq:chiralblocks} we identify the double-twist operators with $m=0,1$ as short multiplets, $\CC_{0, 2r_0- 1 (\frac{\ell}2-1,\frac{\ell}2) } $ and $\CC_{\frac12, 2r_0- \frac32 (\frac{\ell}2-\frac12,\frac{\ell}2)}$ respectively, whose dimensions are protected.\footnote{We assume $\ell\geqslant 2$ here since the inversion formula is not guaranteed to converge for $\ell=0$.}

\begin{figure}[htb!]
             \begin{center}           
              \includegraphics[scale=0.35]{figures/ClCH.pdf} \; \includegraphics[scale=0.35]{figures/ClCH2.pdf}
              \caption{Comparison between the numerical bounds on the OPE coefficient squared of the leading twist operators in the chiral channel and the results from  the inversion formula \eqref{eq:nonchiral_lead}, for $c=\tfrac{11}{30}$ and external dimension $r_0=\frac65$. The black boxes mark the numerically allowed range for the squared OPE coefficients of the $\mathcal{C}_{0,\frac75,\left(\frac{\ell }{2}-1,\frac{\ell }{2}\right)}$ operators for different values of $\ell$ (the $\ell=0$ operator should be interpreted as $\EE_{\frac{12}5}$) and with $\Lambda=36$. The dashed line shows the result of equation \eqref{eq:nonchiral_lead}, where we considered only the contribution of the identity and stress-tensor operators in the non-chiral channel, and thus is an approximate result for sufficiently large spin. The formula \eqref{eq:nonchiral_lead} is not guaranteed to be valid for $\ell=0$, and the results here are just shown as an illustration.}
              \label{Fig:ClCH}
            \end{center}
\end{figure}
%

We can now compute corrections to the OPE coefficient of the $\mathcal{C}_{0,2 r_0-1,\left(\frac{\ell }{2}-1,\frac{\ell }{2}\right)}$ operators, for $r_0=\frac65$ and $c=\tfrac{11}{30}$, from \eqref{eq:nonchiral_lead}.
The result is plotted in figure \ref{Fig:ClCH}, where we performed the integral in \eqref{eq:nonchiral_lead} numerically (after taking the leading $z$ term), together with the numerical upper and lower bonds on the OPE coefficients, obtained in section \ref{sec:OPEbounds}.\footnote{Note that by the usual lightcone methods \cite{Fitzpatrick:2012yx,Komargodski:2012ek,Alday:2015ewa}, we could obtain an asymptotic expansion in $\tfrac{1}{\ell}$ of the correction to the generalized free field theory OPE coefficients \eqref{eq:gfftOPEcoeff} arising from the stress tensor exchange.  By considering the contributions of the stress tensor superblock to \eqref{eq:CHgeneratingbos} we are effectively re-summing the lightcone expansion to all orders.}
The results for $\ell=0$ (where the multiplet becomes an $\EE_{2r_0}$) are also shown even though the formula is only guaranteed to be valid for $\ell \geqslant 2$.\footnote{The formula could only be valid for $\ell=0$ if, for some reason, the growth of the four-point function of the $(A_1,A_2)$ theory, in the limit relevant for the dropping of arcs of integration along the derivation of the inversion formula, was better than the generic growth expected in any CFT and derived in \cite{Caron-Huot:2017vep}.}



We point out that the only input was the leading $t-$ and $u-$channels contributions and thus the resulting OPE coefficients are an approximation for sufficient large spin. Indeed, the neglected contribution of the long multiplets should behave like $\ell^{-\tau}$ for large spin, with $\tau \sim 2.4$, while the stress tensor contributes as $\tau=2$. Nevertheless, we see in figure \ref{Fig:ClCH} that starting from $\ell=4$ the analytical result is already inside the numerically allowed range for the OPE coefficient. This is shown clearly in figure \ref{Fig:Cl4} where the result of \eqref{eq:nonchiral_lead} for $\ell=4$ is shown as a dashed blue line, together with the numerically allowed range. For $\ell=2$, however, the result of \eqref{eq:nonchiral_lead} (blue dashed line in figure \ref{Fig:Cl2}) is clearly insufficient, as it is outside the numerically allowed region.
Note that the numerical results are not optimal yet, \ie, while they provide true bounds they have not yet converged, and the optimal bounds will be more restrictive. Thus, the fact that the $\ell=4$ estimate was inside the numerical bounds should not be taken to mean the subleading contributions are negligible for such a low spin. What is in fact surprising is that the estimates from \eqref{eq:nonchiral_lead} are so close to the numerically obtained ranges for such low values of the spin. These results leave us optimistic that better estimates can be obtained by providing a few of the subleading contributions, as was done in \cite{Simmons-Duffin:2016wlq} for the $3d$ Ising model.
The computation used to obtain figure \ref{Fig:ClCH} could be easily extended to obtain estimates for the OPE coefficients of the 
$\CC_{\frac12, 2r_0- \frac32 (\frac{\ell}2-\frac12,\frac{\ell}2)}$  multiplets, and also the dimensions and OPE coefficients of the remaining operators in \eqref{eq:selruleschiral}. One particularly interesting multiplet would be $\BB_{1, \frac75 (0,0)}$ since, as discussed before, we expect it to be absent in the $(A_1,A_2)$ theory. However, this corresponds to a spin zero contribution and thus convergence of the inversion formula is not guaranteed.

%!TEX root = ../H0bootstrap.tex
%%%%%%%%%%%%%%%%%%%%%%%%%%%%%%%%%%%%%%%%%%%%%
\subsection{Inverting the non-chiral OPE}
\label{sec:nonchiral_inv}
%%%%%%%%%%%%%%%%%%%%%%%%%%%%%%%%%%%%%%%%%%%%%

Next we turn to the non-chiral channel, where we have a decomposition in superconformal blocks, and so we must obtain a supersymmetric version of the inversion formula of \cite{Caron-Huot:2017vep}.
We consider the inversion of the $s-$channel OPE of \eqref{eq:nonchiral}, with the superblocks given by \eqref{eq:superblock},
\be 
\langle \phi(x_1) \bar{\phi}(x_2)  \phi(x_3) \bar{\phi}(x_4) \rangle = \frac{(z \zb)^{-\frac{\NN}{2}}}{x_{12}^{2 \Delta_\phi} x_{34}^{2 \Delta_\phi}} \left(\sum\limits_{\OO_{\Delta,\ell}} |\lambda_{\phi \bar{\phi} \OO_{\Delta,\ell}}|^2 g_{\Delta+ \NN,\ell}^{\NN,\NN}(z, \zb) \right)\,,
\label{eq:nonchiralforCH}
\ee 
where we are interested in taking $\NN=2$, but the same equation is also valid for $\NN=1$, and so all that follows generalizes easily to that case. Fortunately, the fact that, up to the overall prefactor $(z \zb)^{-\NN/2}$ in \eqref{eq:nonchiralforCH}, the blocks relevant for the $s-$channel decomposition are identical to bosonic blocks of operators with unequal dimensions makes the task of obtaining an inversion formula very easy.
We can use the results of \cite{Caron-Huot:2017vep} with small modifications: The Lorentzian inversion formula applies to the term between brackets in \eqref{eq:nonchiralforCH}, and the fact that the pre-factor is not the correct one for operators of unequal dimension plays a small role in the derivation of \cite{Caron-Huot:2017vep}. The only time the prefactor is considered is when bounding the growth of the correlator, needed to show the inversion formula is valid for spin greater than one. The modified prefactor here seems to ameliorate the growth: we are inverting $(z \zb)^{\tfrac{\NN}{2}}$ times a CFT correlator whose growth is bounded as discussed in \cite{Caron-Huot:2017vep}.
The condition $\ell > 1$ on the inversion formula \eqref{eq:CHgeneratingbos} came from the need to have $\ell$ large such that one could drop the arcs at infinity during the derivation of \cite{Caron-Huot:2017vep}. The prefactor's behavior in this limit means the inversion formula will be valid for all $\ell > 1 - \NN$, and the results we obtain for $\NN=2$ should be valid for all spins.
Apart from this, the prefactor will only play a role when representing the correlator by its $t-$ and $u-$channel OPEs.
As such we apply \eqref{eq:CHgeneratingbos} with 
\be 
\GG(z,\zb)= \sum\limits_{\OO_{\Delta,\ell}} |\lambda_{\phi \bar{\phi} \OO_{\Delta,\ell}}|^2 g_{\Delta+ \NN,\ell}^{\NN,\NN}(z, \zb) \,.
\ee
The $t-$ and $u-$channels of the correlator \eqref{eq:nonchiralforCH} are given by a non-chiral and chiral OPE respectively.
Using the crossing equation \eqref{eq:nonchiralnonchiral} we see that the $t-$channel expansion of $\GG(z,\zb)$ is
\be 
\GG(z,\zb) =  (z \zb)^{\frac{\NN}{2}} \left( \frac{z \zb }{(1-z)(1-\zb ) } \right)^{r_0} \sum\limits_{\Delta,\ell} |\lambda_{\phi \bar{\phi} \OO_{\Delta,\ell}}|^2  \GG_{\Delta,\ell}(1-z,1-\zb)\,,
\label{eq:Gtchan_nonchiral}
\ee
with the superblock given by \eqref{eq:superblock}. While the $u-$channel is given by
\be 
\GG(z,\zb) = (z \zb)^{r_0+\frac{\NN}{2}} \sum\limits_{\Delta,\ell} |\lambda_{\phi \phi \OO}|^2 g_{\Delta, \ell}\left(\frac{1}{z},\frac{1}{\zb}\right)\,.
\label{eq:Guchan_nonchiral}
\ee

Once again, the leading contributions to the $s-$channel spectrum at large spin, \ie, the leading contributions for  $\zb \to 1$ in \eqref{eq:Gtchan_nonchiral}, are from the  $t-$channel identity and stress-tensor multiplet. The subleading contributions in the $t-$channel come from long multiplets with $\Delta > \ell + 2$. On the other hand, the leading twist contribution in the $u-$ channel arises from the $\EE_{2r_0}$ and $\mathcal{C}_{0,2 r_0-1,\left(\frac{\ell }{2}-1,\frac{\ell }{2}\right)}$ multiplets, whose twists are all exactly $2r_0$, and so one should consider the infinite sum over $\ell$. 
From a lightcone computation, \eg, \cite{Li:2015rfa}, we expect an individual chiral operator of twist $\tau_c$ to contribute to the anomalous dimensions of the non-chiral operators at large $\ell$ as $\frac{(-1)^\ell}{\ell^{\tau_c}}$. Similarly, a non-chiral operator of twist  $\tau$ contributes to the same anomalous dimension at large $\ell$ as $\frac{1}{\ell^{\tau}}$. In the case at hand, $\tau=2$ for the stress-tensor multiplet and $\tau_c=2.4$ for each of the infinite number of leading operators in the chiral channel. The contribution of an individual chiral operator in the $u-$channel is thus subleading for sufficiently large spin. This is similar to what happened in section \ref{sec:chiral_inv}, and while in this case the dimensions of the operators are protected, their OPE coefficients are not. Indeed, the value of these OPE coefficients remains elusive, and the best estimate we have to go on comes from the numerically obtained bounds for the operators with $\ell \leqslant 20$ presented in figure \ref{Fig:ClCH}. An interesting possibility would be to attempt to combine the numerical ranges for low spin with the estimate for the large spin OPE coefficients obtained from \eqref{eq:nonchiral_lead}. The numerical bounds on the OPE coefficients would turn into an estimate, in the form of an interval, for the anomalous dimension; we leave this exploration for future work. Here we apply the inversion formula \eqref{eq:CHgeneratingbos} only to the exchange of the identity and stress-tensor multiplets
\be 
C^t(z,\beta) \supset \int\limits_0^1 \frac{\d \zb}{\zb^2} \kappa_{\beta} k_\beta^{1,1}(\zb) \dDisc\left[  \frac{(z \zb)^{r_0+1} }{((1-z)(1-\zb ))^{r_0} } \left( 1+ |\lambda_{\phi \bar{\phi} \hat{\CC}_{0,0}}|^2 \GG_{2,0}(1-z,1-\zb)\right)\right]\,,
\label{eq:CHgeneratingsuper}
\ee
where one should recall that $\Delta_{12}=\Delta_{34}= \frac{\NN}{2}=1$ when taking the double-discontinuity.


Like before, the exchange of the identity in \eqref{eq:CHgeneratingsuper} gives the existence of double-twist operators $\left[\phi \bar{\phi}\right]_{m,\ell}$, with dimensions 
\be
\Delta_{\left[\phi \bar{\phi}\right]_{m,\ell}} \underset{\ell \gg 1}{\longrightarrow} 2 r_0+2m + \ell \,.
\label{eq:nonchiraldoubletwist}
\ee
Computing the OPE coefficients from the identity exchange we find, for the leading twist operators,
\be
|\lambda_{\phi \bar{\phi} \left[\phi \bar{\phi}\right]_{0,\ell}}|^2 \underset{\ell \gg 1}{\longrightarrow} \frac{4^{2-\ell } r_0  (r_0)_{\ell -2} (2 r_0+1)_{\ell -2}}{(1)_{\ell -2} (r_0+\ell -2) \left(r_0+\frac{1}{2}\right)_{\ell -2}} \,,
\label{eq:OPEdoubletwistsuper}
\ee
which are precisely the OPE coefficients of generalized free field theory, now decomposed in superblocks instead of bosonic blocks.

\begin{figure}[htb!]
             \begin{center}           
              \includegraphics[scale=0.35]{figures/anomCH.pdf}
              \caption{Anomalous dimension ($\gamma_\ell = \Delta_\ell - (2 \Delta_\phi + \ell)$) of the first spin $\ell$ long multiplet in the non-chiral channel. The colored dots are the dimension estimates extracted from the extremal functionals of the various bounds (figures \ref{Fig:c_min} and \ref{Fig:Bbound}-\ref{Fig:Cl24} for $c=\tfrac{11}{30}$) as indicated by their colors, with $\Lambda=34$. The dashed line corresponds to the result from the inversion formula \eqref{eq:CHgeneratingsuper}, for $c=\tfrac{11}{30}$ and external dimension $r_0=\frac65$, taking into account only the exchange of the identity and stress tensor in the $t-$channel, and is thus an approximate result for sufficiently large spin.
              }
              \label{Fig:anomCH}
            \end{center}
\end{figure}

The stress-tensor exchange provides corrections to these dimensions and OPE coefficients. As an illustration we computed its contribution to the anomalous dimensions of the leading twist operators $\left[\phi \bar{\phi}\right]_{0,\ell}$, $\gamma_\ell= \Delta_\ell - (2 \Delta_\phi + \ell)$. From the numerical estimates (see figure \ref{Fig:anomCH}) we see the anomalous dimensions starting at spin one are rather small, and so we simply take the zeroth order of the procedure outlined in \cite{Caron-Huot:2017vep} to commute the $z \to 0$ limit with the sum over primaries in \eqref{eq:CHgeneratingsuper}. These results are also shown in figure \ref{Fig:anomCH} for $\ell \geqslant1$ as a dashed blue line, together with estimates for these values arising from the various extremal functionals of section \ref{sec:numericsh0}, color coded according to which bound they came from.\footnote{We omitted two spin seven dimensions, as we could not accurately estimate them from the functionals. The two points that appear to be outlying in spin $6$ and $7$ correspond to cases where there were two zeros of the functional very close to one another, and we extracted the dimension of the first. We expect that higher derivative orders would fix both situations.} We are plotting the results starting from spin $\ell=1$. 
The leading $\ell=0$ operator is the stress tensor itself, which was not present in the generalized free field theory solution. As such the dimension of $\left[\phi \bar{\phi}\right]_{0,0}$ must come down from $2r_0=2.4$ to exactly $2$. The value of the anomalous dimensions coming from \eqref{eq:CHgeneratingsuper} is still insufficient for  this to happen, as clear from figure \ref{Fig:anomCH}.
For $\ell \geqslant1$, however, the numerical estimates of leading twist operators' dimensions are very close the values of double-twist operators \eqref{eq:nonchiraldoubletwist}. Indeed, the maximum anomalous dimension in figure \ref{Fig:anomCH}, ignoring the two out-lying points, is of the order of $\gamma_1 \sim 0.04$, in a dimension that is close to $2r_0+1 =3.4$.
The anomalous dimensions obtained from \eqref{eq:CHgeneratingsuper} (dashed blue line in \eqref{Fig:anomCH}) are close to the numerically obtained values starting from  $\ell=2$, despite the fact that our results are only valid for sufficiently large spin, as we have only considered the identity and stress tensor contributions in the $t-$channel, and completely disregarded any $u-$channel contribution.  In particular, for spin $\ell \gtrsim 8$ the numerical estimates arising from the different extremization problems of section \ref{sec:numericsh0} are all cluttered, approaching the value \eqref{eq:nonchiraldoubletwist}, and close to the values coming from \eqref{eq:CHgeneratingsuper}.

\bigskip

All in all, we have seen that both in the chiral and non-chiral channels the estimates coming from applying the inversion formula, and providing only the leading twist operators (identity plus stress-tensor supermultiplet), come very close to the numerically obtained bounds/estimates. This leaves us optimistic that the spectrum of the $(A_1,A_2)$ can be bootstrapped, similarly to the $3d$ Ising model. The numerical results for $\NN=2$ theories suffer from slow convergence and thus the estimates for OPE coefficients and anomalous dimensions we obtain are not yet with the precision of those of the $3d$ Ising model. By using this data as input to the inversion formulas they would in turn produce ranges for the various quantities appearing in the chiral and non-chiral OPEs.
Finally, another direction corresponds to using the output of each inversion formula as input for the other to obtain better estimates. We leave these two directions for future work.