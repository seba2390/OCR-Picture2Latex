%!TEX root = ../H0bootstrap.tex
%%%%%%%%%%%%%%%%%%%%%%%%%%%%%%%%%%%%%%%%%%%%%
\subsection{OPE coefficient bounds}
\label{sec:OPEbounds}
%%%%%%%%%%%%%%%%%%%%%%%%%%%%%%%%%%%%%%%%%%%%%

We now concentrate on OPE-coefficient bounds for the different short multiplets appearing in the chiral channel, for varying $c \geqslant c_{min}(\Lambda)$, and external dimension fixed to $r_0=\frac65$. In particular, we obtain an upper bound for the OPE coefficient of the $\BB_{1,\frac75(0,0)}$ multiplet, 
and both \emph{lower and upper} bounds (see discussion in subsection \ref{subsubsec:implementation}) for the coefficients of the $\EE_{\frac{12}5}$ and $\mathcal{C}_{0,\frac75\left(\frac{\ell }{2}-1,\frac{\ell }{2}\right)}$ multiplets.

For fixed $\Lambda$, there is a unique solution to the truncated crossing equations \eqref{eq:crossingeqs} at $c=c_{min}(\Lambda)$ \cite{Poland:2010wg,ElShowk:2012hu}, and indeed we will see below that upper and lower bounds (when available) coincide. As already discussed, it is plausible that $c_{min}(\Lambda) \to \tfrac{11}{30}$ as $\Lambda \to \infty$, and so in this limit the meeting point of upper and lower bounds would be at $c=\frac{11}{30} \simeq 0.367$.


An important subtlety in all the plots that follow is that we cannot fix the central charge exactly: each time we quote a value of $c$, the corresponding plot captures values \textit{less or equal} than the given number. This follows from the fact that we allow for a continuum of long multiplets with dimensions consistent with the unitarity bounds; for the non-chiral channel this means long multiplets with $\Delta > \ell +2$ \eqref{eq:selrulesnonchiral}. However, as is clear from the superconformal blocks \eqref{eq:superblock}, the contribution of a long multiplet at the unitarity bound mimics the contribution of a conserved current $\hat{\CC}_{0,\ell}$. This has two important consequences. First, we cannot restrict ourselves to interacting theories, because it is not possible to set to zero the OPE coefficient of the conserved currents of spin greater than two ($\hat{\CC}_{0,\ell \geqslant 1}$), without imposing a gap on the spectrum of all long multiplets. Second, even if we fix the central charge according to \eqref{eq:STOPEcoeff}, a long multiplet at the unitarity bound with an arbitrary (positive) coefficient, will increase the value of the $\lambda_{\phi \bar{\phi} \OO_{\Delta=2, \ell=0}}$ coefficient, which means that we are really allowing for all central charges smaller than the fixed value. This implies that a given bound can only get weaker as $c$ is increased, and explains the flatness of some of the bounds presented below.



\subsubsection*{OPE coefficient bound for $\BB_{1,\frac75(0,0)}$}

\begin{figure}[htb!]
             \begin{center}           
              \includegraphics[scale=0.363]{figures/Bvsc.pdf}
               \includegraphics[scale=0.362]{figures/Bextrapol.pdf}
              \caption{Numerical upper bound on the OPE coefficient squared of the operator $\BB_{1,\frac75(0,0)}$ appearing in the chiral channel for $\Lambda=26,\ldots,40$, and external dimension $r_0=\frac65$. Left: Upper bound on the OPE coefficient for different values of the central charge, with the strongest bound corresponding to $\Lambda=40$; the dashed lines mark the minimum central charge as extracted from figure \ref{Fig:c_min} for each cutoff $\Lambda$, and the solid line marks $c=\tfrac{11}{30}$. Right: Bound on the OPE coefficient for $c=\tfrac{11}{30}$ as a function of the inverse cutoff $\Lambda$, together with various extrapolations to infinitely many derivatives.}
              \label{Fig:Bbound}
            \end{center}
\end{figure}

Let us first consider the OPE coefficient squared of $\BB_{1,\frac75(0,0)}$. A numerical upper bound, as a function of the central charge, and for fixed external dimension $r_0=\frac65$, is shown on the left-hand side of figure~\ref{Fig:Bbound} for various values of the cutoff $\Lambda$. For each value of $\Lambda$, the upper bound vanishes for $c=c_{min}(\Lambda)$ (marked by the dashed vertical lines in the figure), and becomes negative for $c<c_{min}(\Lambda)$, implying there is no unitary solution to the crossing equations. This is consistent with what was found for $\Lambda=12$ in \cite{Lemos:2015awa}, and suggests this operator is responsible for the existence of the central charge bound.
Since such a multiplet is associated with the mixed branch, and the $(A_1,A_2)$ theory has no mixed branch, it 
would be natural to expect its absence to be a feature of the four-point function of the $(A_1,A_2)$ theory.
However, as can be seen on the right-hand side of figure \ref{Fig:Bbound},
the numerical results appear to leave room for solutions to crossing with a small value of this OPE coefficient, as it is not clear if the upper bound will converge to zero as $\Lambda \to \infty$.
If there is more than one solution, it is plausible that the one corresponding to the $(A_1,A_2)$ theory is one in which $\BB_{1,\frac75(0,0)}$ has zero OPE coefficient. We should point out though, that the absence of a mixed branch does not guarantee that the aforementioned multiplet is absent, as it is possible that such a multiplet is present, but one cannot give it a vev and thus no mixed branch exists \cite{Tachikawa:2013kta,Argyres:2015ffa}. 


\subsubsection*{OPE coefficient bound for \texorpdfstring{$\EE_{\frac{12}5}$}{E2r0 OPE coefficient bound}}

Turning to the OPE coefficient of the Coulomb branch chiral ring operator $\EE_{\frac{12}5}$, we can now place upper and lower bounds  as a function of the central charge. We present the results on the left-hand side of figure~\ref{Fig:Ebound} for several values of $\Lambda$.

\begin{figure}[htb!]
             \begin{center}           
              \includegraphics[scale=0.363]{figures/E_mult.pdf}
              \includegraphics[scale=0.362]{figures/E_extrapolation.pdf}
              \caption{Numerical upper and lower bounds on the OPE coefficient squared of the chiral operator $\EE_{\frac{12}5}$ for increasing number of derivatives and external dimension $r_0=\frac65$. Left: Bounds on the OPE coefficient for different values of the central charge, with cutoffs $\Lambda=26,\ldots,40$, the vertical line marks $c=\tfrac{11}{30}$. Right: Various extrapolations of the lower and upper bounds at $c=\tfrac{11}{30}$ for infinite $\Lambda$.}
              \label{Fig:Ebound}
            \end{center}
\end{figure}

As already discussed, the plots in this section allow for all central charges $c\geqslant c_{\mathrm{fixed}}$, since a gap in the spectrum of spin zero long multiplets is not imposed. This explains the flatness of the upper bound: solutions to crossing saturating it can effectively have central charges equal to $c_{min}(\Lambda)$.\footnote{A natural solution would be to impose small gaps in the spectrum of long multiplets, this removes the conserved currents of spin greater than two and fixes the central charge. However, we have no intuition on the size of these gaps, not even for spin zero, as there is no understanding of the number of non-supersymmetry preserving relevant deformations. We 
experimented imposing that the spectrum of long multiplets obeys $\Delta \geqslant 2+\epsilon + \ell$ for various small values of $\epsilon$, and although the upper bound gets stronger than that of figure \ref{Fig:Ebound}, it varies smoothly with $\epsilon$ and thus there is no justification to pick any specific value. The lower bound, on the other hand, shows a much smaller dependence on $\epsilon$.} 
The lower bound, however, must be saturated by theories with central charge equal to the fixed value.
At $c_{min}(\Lambda)$ the upper and lower bounds coincide, fixing a unique value of the OPE coefficient, and as $c$ is increased a wider range of values, and distinct solutions to crossing, are allowed. We show the allowed range, for $c=\tfrac{11}{30}$, as a function of $1/\Lambda$ on the right-hand side of figure~\ref{Fig:Ebound}. The lines correspond to different extrapolations through (subsets) of the data points, and the shaded region aims to give an idea of where the bounds are converging to. If $c_{min}(\Lambda \to \infty) = \tfrac{11}{30}$, then the upper and lower bound should converge to the same value, which is not ruled out by the extrapolations. In any case, our results indicate that the OPE coefficient of $\EE_{\frac{12}5}$ is constrained to a narrow range.


We have thus obtained the following 
\emph{rigorous} bounds for the value of this OPE coefficient in the $(A_1,A_2)$ theory:
\be
2.1418 \leqslant \lambda_{\EE_{\frac{12}5}}^2 \leqslant 2.1672 \,, \qquad \mathrm{for}\; \Lambda=40\,.
\label{eq:Ebound}
\ee
Furthermore, the most conservative of the extrapolations presented in figure \ref{Fig:Ebound} gives
\be
2.146 \lesssim \lambda_{\EE_{\frac{12}5}}^2 \lesssim 2.159 \,, \qquad \text{extrapolated for}\; \Lambda\to \infty\,.
\label{eq:Eboundextrapol}
\ee


\subsubsection*{OPE coefficient bounds for $\CC_{0,\frac75(0,1)}$  and $\CC_{0,\frac75(1,2)}$}

\begin{figure}[htbp!]
\begin{center}      
	\begin{subfigure}[t]{0.5\textwidth}
	\centering        
	\includegraphics[scale=0.362]{figures/Cl2plot.pdf}
	\caption{$\CC_{0,\frac75(0,1)}$}
    \label{Fig:Cl2}
    \end{subfigure}~
    \begin{subfigure}[t]{0.5\textwidth}
    \centering        
    \includegraphics[scale=0.362]{figures/Cl4plot.pdf}
    \caption{$\CC_{0,\frac75(1,2)}$}
    \label{Fig:Cl4}
    \end{subfigure}
\end{center}
\caption{Numerical upper and lower bounds on the OPE coefficient squared of the chiral channel multiplet $\mathcal{C}_{0,\frac75\left(\frac{\ell }{2}-1,\frac{\ell }{2}\right)}$, for $\ell=2,4$, and external dimension $r_0=\frac65$. The bounds were obtained for cutoffs $\Lambda=26,\ldots,34$ and the vertical line marks $c=\tfrac{11}{30}$. The dashed line corresponds to the value obtained from the Lorentzian inversion formula of \cite{Caron-Huot:2017vep} applied to the chiral channel and using as input only the exchange of the identity and stress-tensor superblocks in the non-chiral channel, and thus valid for sufficiently large $\ell$ (see section \ref{sec:chiral_inv} for more details).}
\label{Fig:Cl24}
\end{figure}

Let us now focus on the $\mathcal{C}_{0,\frac75\left(\frac{\ell }{2}-1,\frac{\ell }{2}\right)}$ family of multiplets. Like in the $\EE_{\frac{12}5}$ case, upper and lower bounds are possible thanks to the gap that separates these $\CC$-type multiplets from the continuum of long operators. The bounds for $\ell=2,4$,  as a function of $c$, are shown in figures \ref{Fig:Cl2} and \ref{Fig:Cl4} respectively, while bounds for higher values of $\ell$ can be found in figure \ref{Fig:ClCH}, for fixed $c=\tfrac{11}{30}$.
The dashed lines in figure \ref{Fig:Cl24} are estimates of the OPE coefficient valid for sufficiently large $\ell$, that will be discussed in detail in section \ref{sec:chiral_inv}.
Similarly to the $\EE_{\frac{12}5}$ multiplet, the OPE coefficients of these multiplets in  the $(A_1,A_2)$ theory are constrained to lie in a narrow range:
\be
0.46831 \leqslant \lambda_{\CC_{0,\frac75, (0,1)}}^2 \leqslant 0.46901 \,, \qquad 
0.048919 \leqslant \lambda_{\CC_{0,\frac75, (1,2)}}^2 \leqslant 0.048945 \,,  \qquad
 \text{for}\; \Lambda=34\,.
 \label{eq:Clbound}
\ee

The upper bounds in figure \ref{Fig:Cl24} now show a mild dependence on the central charge, and so we can compare the extrapolations of the upper and lower bounds at $c=\tfrac{11}{30}$ with the extrapolation of the value of the OPE coefficient for the unique solution at $c_{min}(\Lambda)$. Like  before, the extrapolations (not shown) do not rule out that $c_{min} \to \tfrac{11}{30}$ as $\Lambda \to \infty$. As visible in figure \ref{Fig:Cl24}, the value of the OPE coefficient at $c_{min}(\Lambda)$ (the meeting point) shows a very mild dependence on $\Lambda$,  unlike the OPE coefficient of $\EE_{\frac{12}5}$, we can therefore obtain the following estimates 
\be 
\begin{split}
0.4687 & \lesssim \lambda_{\CC_{0,\frac75(0,1)}}^2 (c= c_{min}(\Lambda)) \lesssim 0.4688 \,, \\
0.04892 & \lesssim \lambda_{\CC_{0,\frac75(1,2)}}^2 (c= c_{min}(\Lambda))   \lesssim 0.04894\,, \\
\end{split}\qquad
 \text{extrapolated for }\; \Lambda\to \infty\,.
 \label{eq:Clboundextrapol}
\ee

