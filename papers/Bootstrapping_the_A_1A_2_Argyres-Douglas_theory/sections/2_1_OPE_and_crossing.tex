%!TEX root = ../H0bootstrap.tex
%%%%%%%%%%%%%%%%%%%%%%%%%%%%%%%%%%%%%%%%%%%%%

%%%%%%%%%%%%%%%%%%%%%%%%%%%%%%%%%%%%%%%%%%%%%
\subsection{OPE decomposition and crossing symmetry}
%%%%%%%%%%%%%%%%%%%%%%%%%%%%%%%%%%%%%%%%%%%%%

As we just discussed, our angle to attack the $(A_1,A_2)$ theory is through its Coulomb branch, and thus we are interested in the  $\NN=2$ chiral and anti-chiral operators, respectively $\EE_{r}$ and $\bar{\EE}_{r}$ multiplets, using the naming conventions of \cite{Dolan:2002zh}. 
These are short representations of the superconformal algebra that are half-BPS, where the superconformal primary  is annihilated by all supercharges of one chirality. We denote the superconformal primary of the chiral (anti-chiral) multiplets $\EE_{r}$  ($\bar{\EE}_{r}$) by  $\phi_{r}$ ($\bar{\phi}_{-r}$), where $r$ is the $U(1)_r$ charge of the superconformal primary, with unitarity requiring $r \geqslant 1$ ($-r \geqslant 1$).
The dimensions of the superconformal primaries $\phi_{r}$ ($\bar{\phi}_{-r}$) are fixed in terms of their $U(1)_r$ charges by $\Delta_\phi= r$ ($\Delta_{\bar{\phi}}=-r$). We refer the reader to, \eg, \cite{Dolan:2002zh}, for more on representation theory of the $\NN=2$ superconformal algebra.


The numerical bootstrap program applied to chiral correlators was considered in \cite{Beem:2014zpa,Lemos:2015awa} for the case of two identical operators, and their conjugates, and in \cite{Lemos:2015awa} for two distinct operators, and their conjugates.
Here we briefly review the setup for two identical operators $\EE_{r}$, and conjugates, and refer the reader to  \cite{Beem:2014zpa,Lemos:2015awa} for a more detailed account.
Considering all four-point functions involving the superconformal primaries of these multiplets, we write down the OPE selection rules and conformal block decompositions for all of the channels, and the crossing equations to be studied in sections \ref{sec:numericsh0} and \ref{sec:anlytical}. 
In this work we are only concerned with the $(A_1,A_2)$ theory and thus we fix $r$ to $r_0=\frac65$, according to \eqref{eq:acanomalies_dimension}.


%%%%%%%%%%%%%%%%%%%%%%%%%%%%%%%%%%%%%%%%%%%%%
\subsubsection{Non-chiral channel}
%%%%%%%%%%%%%%%%%%%%%%%%%%%%%%%%%%%%%%%%%%%%%
The OPE selection rules in the non-chiral channel are \cite{Beem:2014zpa}
\be
\phi_{r} \times \bar{\phi}_{-r}  \sim \mathbf{1} + \hat{\CC}_{0 (j,j)} + \AA^{\Delta > 2j+2}_{0, 0 (j,j)}\, .
\label{eq:selrulesnonchiral}
\ee
Here the $\hat{\CC}_{0 (j,j)}$ multiplets include conserved currents of spin $2j+2$, which for $j >0$ are absent in interacting theories \cite{Maldacena:2011jn,Alba:2013yda} and thus we will set them to zero. The multiplet $\hat{\CC}_{0(0,0)}$ corresponds to the superconformal multiplet that contains the stress tensor. By an abuse of notation we will often replace the subscript $(j,j)$ by $\ell$, with $\ell=2j$.
The superconformal block decomposition in this channel can be written as
\be
\langle \phi_{r}(x_1) \bar{\phi}_{-r}(x_2) \phi_{r}(x_3) \bar{\phi}_{-r}(x_4)  \rangle = 
\frac{1}{x_{12}^{2\Delta_{\phi}}x_{34}^{2\Delta_{\phi}}} \sum_{\OO_{\Delta,\ell}} |\lambda_{\phi \bar{\phi} \OO}|^2 \GG_{\Delta, \ell}(z,\bar{z})\, ,
\label{eq:nonchiral}
\ee
where the superblocks $\GG_{\Delta, \ell}(z,\bar{z})$, capturing the supersymmetric multiplets being exchanged in \eqref{eq:selrulesnonchiral}, were computed in \cite{Fitzpatrick:2014oza},
\be 
\label{eq:superblock}
\GG_{\Delta, \ell}(z,\bar{z}) = (z \bar{z})^{-\frac{\NN}{2}}g_{\Delta + \NN,\ell}^{\NN,\NN}(z,\bar{z})\, .
\ee
Here we wrote the blocks for $\NN=1,2$ chiral operators, since both cases can be treated almost simultaneously \cite{Fitzpatrick:2014oza,Lemos:2015awa}, but hereafter we focus only on the case $\NN=2$. 
The function $g_{\Delta,\ell}^{\Delta_{12},\Delta_{34}}(z,\bar{z})$ is the standard bosonic block for the decomposition of a correlation function with four distinct operators, defined in \eqref{eq:bosblock}. Although not immediately obvious, the bosonic block with shifted arguments in \eqref{eq:superblock} can be written as a finite sum of $g_{\Delta,\ell}^{0,0}(z,\bar{z})$ blocks, as expected from supersymmetry. The block reduces to $1$ for the identity exchange, \ie, $\Delta=\ell=0$.

The stress-tensor multiplet $\hat{\CC}_{0 (0,0)}$ corresponds to $\Delta=2$, $\ell=0$ in \eqref{eq:superblock}, and its OPE coefficient can be fixed using the Ward identities (see for example \cite{Beem:2014zpa}):
\be
\left|\lambda_{\phi \bar{\phi} \OO_{\Delta=2,\ell=0}}\right|^2 = \frac{\Delta_\phi^2 }{6 c}\,,
\label{eq:STOPEcoeff}
\ee
while long multiplets $\AA^{\Delta > \ell +2}_{0, 0, \ell}$ contribute as \eqref{eq:superblock} with $\Delta > \ell +2$.

When writing the crossing equations it will be useful to have the block expansion with a slightly different ordering
\be 
\langle \bar{\phi}_{-r}(x_1)  \phi_{r}(x_2) \phi_{r}(x_3) \bar{\phi}_{-r}(x_4)  \rangle = 
\frac{1}{x_{12}^{2\Delta_{\phi}} x_{34}^{2\Delta_{\phi}}} \sum_{\OO_{\Delta,\ell}} (-1)^{\ell} |\lambda_{ \phi  \bar{\phi} \OO}|^2\tilde{\GG}_{\Delta, \ell}(z,\bar{z})\,,
\label{eq:nonchiralbraid}
\ee
where the function $\tilde{\GG}_{\Delta, \ell}(z,\bar{z})(z,\bar{z})$ is defined as
\be 
\label{eq:superblockbraid}
\tilde{\GG}_{\Delta, \ell}(z,\bar{z}) =  (z \bar{z})^{-\frac{\NN}{2}} g_{\Delta + \NN,\ell}^{\NN,-\NN}(z,\bar{z})\, ,
\ee
and again we are only interested in the case $\NN=2$.


%%%%%%%%%%%%%%%%%%%%%%%%%%%%%%%%%%%%%%%%%%%%%
\subsubsection{Chiral channel}
%%%%%%%%%%%%%%%%%%%%%%%%%%%%%%%%%%%%%%%%%%%%%

The OPE selection rules of two identical $\NN=2$ chiral primary operators read \cite{Beem:2014zpa}
\be 
\phi_{r} \times \phi_{r} \sim   \EE_{2r}    + \CC_{0, 2r- 1 (j-1,j) } 
 + \BB_{1, 2r-1 (0,0)} + \CC_{\frac12, 2r- \frac32 (j-\frac12,j)} + \AA_{0, 2r-2 (j,j)}^{\Delta > 2+2r+2j}   \,,
\label{eq:selruleschiral}
\ee
where we already imposed Bose symmetry, and we assumed $\phi_r$ to be above the unitarity bound, \ie, $r >1$. If one considers different operators, or if $r=1$, additional multiplets are allowed to appear (see \eg, \cite{Lemos:2015awa}).
Chirality of $\phi_{r}$ requires each supermultiplet contributes with a single conformal family, and therefore the superblock decomposition contains only bosonic blocks:
\be
\label{eq:chiral}
\langle \phi_r(x_1) \phi_r(x_2) \bar{\phi}_{-r}(x_3) \bar{\phi}_{-r}(x_4) \rangle =  \sum_{\Delta,\ell} |\lambda_{\phi \phi \OO_{\Delta, \ell}}|^2 g^{0,0}_{\Delta,\ell}(z,\bar{z})\,.
\ee
Since we are considering the OPE between two identical $\phi_r$ multiplets, Bose symmetry requires 
the above sum to include only even $\ell$.
The precise contribution from each of the multiplets appearing in \eqref{eq:selruleschiral} is the following
\begin{equation}
\begin{alignedat}{4}
&{\AA_{0, 2r-2 (j,j)}}							&:\qquad &	g^{0,0}_{\Delta,\ell}\,,				\qquad &\Delta > 2+2r+\ell\,, \; \ell \; \mathrm{even}\,,\\
&{\CC_{\tfrac12, 2r-\tfrac32 (j-\tfrac12,j) }}	&:\qquad &	g^{0,0}_{\Delta=2r+\ell+2,\ell}\,,		\qquad & \ell \geqslant 2 \,,\;\ell\; \mathrm{even}\,, \\
&{\BB_{1, 2r-1 (0,0)}}							&:\qquad &	g^{0,0}_{\Delta=2r+2,\ell=0}\,,		\qquad &~\\
&{\CC_{0, 2r- 1 (j-1,j) }}						&:\qquad &	g^{0,0}_{\Delta=2r+\ell,\ell}\,,		\qquad & \ell \geqslant 2\,,\;\ell\; \mathrm{even}\,,\\
&{\EE_{2r}}										&:\qquad &	g^{0,0}_{\Delta=2r,\ell=0}\,,			\qquad &~ 
\end{alignedat}
\label{eq:chiralblocks}
\end{equation}
where $\ell= 2j$ is even (see \cite{Lemos:2015awa} for the contribution in the case of different operators).
While the short multiplets being exchanged in this channel have their dimensions fixed by supersymmetry, their OPE coefficients are not known. In fact, it is not even guaranteed all these multiplets are present as their physical meaning is not as clear as the short multiplets exchanged in the non-chiral channel. The $\EE_{2r}$ multiplet corresponds to an operator in the Coulomb branch chiral ring and therefore must be present in the $(A_1,A_2)$ theory, although the value of its OPE coefficient is not known. 
The scalar primary of the $\BB_{1, 2r-1 (0,0)}$ multiplet may be identified with a mixed branch chiral ring operator \cite{Argyres:2015ffa}, and since the $(A_1,A_2)$ has no mixed branch, one might expect this multiplet to be absent. We must point out, however, that the identification of this multiplet with the mixed branch is conjectural, it could be that the multiplet is present but it does not correspond to a flat direction \cite{Tachikawa:2013kta,Argyres:2015ffa}. 
Note that the contribution of the two short operators $\BB_{1, 2r-1 (0,0)}$ and $\CC_{\frac12, 2r- \frac32 (j-\frac12,j)}$ is identical to that of a long multiplet saturating the unitarity bound $\Delta= 2 + 2r + \ell$, as follows directly from the decomposition of the long multiplet when hitting the unitarity bound \cite{Dolan:2002zh}. On the other hand, the contribution of the short multiplets $\EE_{2r}$ and $\CC_{0 , 2r- 1 (j-1,j) } $ is isolated from the continuous spectrum of long operators by a gap; this will be relevant for the numerical analysis of section \ref{sec:numericsh0}.

%%%%%%%%%%%%%%%%%%%%%%%%%%%%%%%%%%%%%%%%%%%%%
\subsubsection{Crossing symmetry}
%%%%%%%%%%%%%%%%%%%%%%%%%%%%%%%%%%%%%%%%%%%%%

We are now ready to quote the crossing equations to be studied in the subsequent sections, where we recall only even spins are allowed in the $\phi \phi$ OPE,
%
\begin{subequations}
\be 
(z \zb)^{\Delta_\phi} \sum\limits_{\OO \in \phi \phi} |\lambda_{\phi \phi \OO}|^2 g^{0,0}_{\Delta_\OO, \ell_\OO} (1-z,1-\zb) = ((1-z)(1-\zb))^{\Delta_\phi }\sum\limits_{\OO \in \phi \bar\phi} |\lambda_{\phi \bar \phi \OO}|^2 (-1)^\ell \tilde{\GG}_{\Delta_\OO, \ell_\OO}(z,\zb) \,,
\label{eq:chiralnonchiral}
\ee   
\be 
((1-z)(1-\zb))^{\Delta_\phi} \sum\limits_{\OO \in \phi \bar\phi} |\lambda_{\phi \bar \phi \OO}|^2 \GG_{\Delta_\OO, \ell_\OO} (z,\zb) 
= (z \zb)^{\Delta_\phi} \sum\limits_{\OO \in \phi \bar\phi} |\lambda_{\phi \bar \phi \OO}|^2 \GG_{\Delta_\OO, \ell_\OO}(1-z,1-\zb)\,.
\label{eq:nonchiralnonchiral}
\ee 
\label{eq:crossingeqs}
\end{subequations}
%
The full system of equations comprises \eqref{eq:crossingeqs}  together with equation \eqref{eq:chiralnonchiral} with $z \to 1-z$ and $\zb \to 1-\zb$. These are collected in a form suitable for the numerical implementation in \eqref{eq:cross_numerics}. 


%%%%%%%%%%%%%%%%%%%%%%%%%%%%%%%%%%%%%%%%%%%%%
\subsubsection{Numerical bootstrap}
\label{subsubsec:implementation}
%%%%%%%%%%%%%%%%%%%%%%%%%%%%%%%%%%%%%%%%%%%%%

In this short section we give details of the numerical implementation that will be necessary to understand the results of subsequent sections. Schematically, the final form of the crossing equations given in \eqref{eq:cross_numerics} is
\be 
|\lambda_{\OO^\star}|^2 \vec{V}_{\OO^\star} + \sum\limits_{\OO} |\lambda_{\OO}|^2 \vec{V}_{\OO} +\vec{V}_{\mathrm{fixed}} = 0\,.
\ee
Here $\OO^\star$ is a superconformal multiplet whose OPE coefficient we would like to bound numerically. 
The term $\vec{V}_{\mathrm{fixed}} $ encodes the contribution of the identity, or of the identity and stress tensor if we fix the central charge $c$, and is given in \eqref{eq:idandst}. 
OPE coefficient bounds are obtained using the SDPB solver of \cite{Simmons-Duffin:2015qma} to solve the following optimization problem
\be
\begin{split}
&\vec{\Psi}\cdot \vec{V}_{\OO} \geqslant 0 \,, \qquad \forall \; \OO\in \{\text{trial spectrum}\}\,,\\
&\vec{\Psi}\cdot \vec{V}_{\OO^\star} = \pm 1\,,  \\
&\mathrm{Maximize}\left(\vec{\Psi}\cdot \vec{V}_{\mathrm{fixed}}\right)\,,
\end{split}
\label{eq:optimiz}
\ee
where the minus sign in the second line can be consistently imposed at the same time as the first line, \textit{only} when the contribution of  $\OO^\star$ is isolated from the contribution of the remaining $\OO\in \{\text{trial spectrum}\}$ \cite{Poland:2011ey}.
As is standard in the bootstrap literature, we truncate the infinite-dimensional  functional as
\be  
\vec{\Psi} = \sum\limits_{m,n}^{m+n \leqslant\Lambda} \vec{\Psi}_{m,n} \partial_z^m \partial_{\zb}^n \Big\vert_{z= \zb=\frac12}\,.
\ee
The result of the extremization problem \eqref{eq:optimiz} provides a bound on the OPE coefficient of $\OO^\star$ as
\be 
\pm |\lambda_{\OO^\star}|^2 \leqslant -\mathrm{Max}\left(\Psi\cdot \vec{V}_{\mathrm{fixed}}\right)\,.
\ee
When the bound is saturated, there is a unique solution to the (truncated) crossing equations \cite{Poland:2010wg,ElShowk:2012hu}, with different extremization problems possibly leading to different solutions.\footnote{See, however, \cite{Behan:2017rca} for subtitles that arise when considering systems of mixed correlators.} 
At finite $\Lambda$, this corresponds to an approximate solution to the full crossing system, with the spectrum encoded in the extremal functional \cite{ElShowk:2012hu}.
We refer the reader to, \eg, \cite{Poland:2011ey,Simmons-Duffin:2015qma,Simmons-Duffin:2016gjk} for more technical details pertaining the numerical bootstrap.