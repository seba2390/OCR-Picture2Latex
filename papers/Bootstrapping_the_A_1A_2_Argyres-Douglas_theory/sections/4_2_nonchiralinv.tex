%!TEX root = ../H0bootstrap.tex
%%%%%%%%%%%%%%%%%%%%%%%%%%%%%%%%%%%%%%%%%%%%%
\subsection{Inverting the non-chiral OPE}
\label{sec:nonchiral_inv}
%%%%%%%%%%%%%%%%%%%%%%%%%%%%%%%%%%%%%%%%%%%%%

Next we turn to the non-chiral channel, where we have a decomposition in superconformal blocks, and so we must obtain a supersymmetric version of the inversion formula of \cite{Caron-Huot:2017vep}.
We consider the inversion of the $s-$channel OPE of \eqref{eq:nonchiral}, with the superblocks given by \eqref{eq:superblock},
\be 
\langle \phi(x_1) \bar{\phi}(x_2)  \phi(x_3) \bar{\phi}(x_4) \rangle = \frac{(z \zb)^{-\frac{\NN}{2}}}{x_{12}^{2 \Delta_\phi} x_{34}^{2 \Delta_\phi}} \left(\sum\limits_{\OO_{\Delta,\ell}} |\lambda_{\phi \bar{\phi} \OO_{\Delta,\ell}}|^2 g_{\Delta+ \NN,\ell}^{\NN,\NN}(z, \zb) \right)\,,
\label{eq:nonchiralforCH}
\ee 
where we are interested in taking $\NN=2$, but the same equation is also valid for $\NN=1$, and so all that follows generalizes easily to that case. Fortunately, the fact that, up to the overall prefactor $(z \zb)^{-\NN/2}$ in \eqref{eq:nonchiralforCH}, the blocks relevant for the $s-$channel decomposition are identical to bosonic blocks of operators with unequal dimensions makes the task of obtaining an inversion formula very easy.
We can use the results of \cite{Caron-Huot:2017vep} with small modifications: The Lorentzian inversion formula applies to the term between brackets in \eqref{eq:nonchiralforCH}, and the fact that the pre-factor is not the correct one for operators of unequal dimension plays a small role in the derivation of \cite{Caron-Huot:2017vep}. The only time the prefactor is considered is when bounding the growth of the correlator, needed to show the inversion formula is valid for spin greater than one. The modified prefactor here seems to ameliorate the growth: we are inverting $(z \zb)^{\tfrac{\NN}{2}}$ times a CFT correlator whose growth is bounded as discussed in \cite{Caron-Huot:2017vep}.
The condition $\ell > 1$ on the inversion formula \eqref{eq:CHgeneratingbos} came from the need to have $\ell$ large such that one could drop the arcs at infinity during the derivation of \cite{Caron-Huot:2017vep}. The prefactor's behavior in this limit means the inversion formula will be valid for all $\ell > 1 - \NN$, and the results we obtain for $\NN=2$ should be valid for all spins.
Apart from this, the prefactor will only play a role when representing the correlator by its $t-$ and $u-$channel OPEs.
As such we apply \eqref{eq:CHgeneratingbos} with 
\be 
\GG(z,\zb)= \sum\limits_{\OO_{\Delta,\ell}} |\lambda_{\phi \bar{\phi} \OO_{\Delta,\ell}}|^2 g_{\Delta+ \NN,\ell}^{\NN,\NN}(z, \zb) \,.
\ee
The $t-$ and $u-$channels of the correlator \eqref{eq:nonchiralforCH} are given by a non-chiral and chiral OPE respectively.
Using the crossing equation \eqref{eq:nonchiralnonchiral} we see that the $t-$channel expansion of $\GG(z,\zb)$ is
\be 
\GG(z,\zb) =  (z \zb)^{\frac{\NN}{2}} \left( \frac{z \zb }{(1-z)(1-\zb ) } \right)^{r_0} \sum\limits_{\Delta,\ell} |\lambda_{\phi \bar{\phi} \OO_{\Delta,\ell}}|^2  \GG_{\Delta,\ell}(1-z,1-\zb)\,,
\label{eq:Gtchan_nonchiral}
\ee
with the superblock given by \eqref{eq:superblock}. While the $u-$channel is given by
\be 
\GG(z,\zb) = (z \zb)^{r_0+\frac{\NN}{2}} \sum\limits_{\Delta,\ell} |\lambda_{\phi \phi \OO}|^2 g_{\Delta, \ell}\left(\frac{1}{z},\frac{1}{\zb}\right)\,.
\label{eq:Guchan_nonchiral}
\ee

Once again, the leading contributions to the $s-$channel spectrum at large spin, \ie, the leading contributions for  $\zb \to 1$ in \eqref{eq:Gtchan_nonchiral}, are from the  $t-$channel identity and stress-tensor multiplet. The subleading contributions in the $t-$channel come from long multiplets with $\Delta > \ell + 2$. On the other hand, the leading twist contribution in the $u-$ channel arises from the $\EE_{2r_0}$ and $\mathcal{C}_{0,2 r_0-1,\left(\frac{\ell }{2}-1,\frac{\ell }{2}\right)}$ multiplets, whose twists are all exactly $2r_0$, and so one should consider the infinite sum over $\ell$. 
From a lightcone computation, \eg, \cite{Li:2015rfa}, we expect an individual chiral operator of twist $\tau_c$ to contribute to the anomalous dimensions of the non-chiral operators at large $\ell$ as $\frac{(-1)^\ell}{\ell^{\tau_c}}$. Similarly, a non-chiral operator of twist  $\tau$ contributes to the same anomalous dimension at large $\ell$ as $\frac{1}{\ell^{\tau}}$. In the case at hand, $\tau=2$ for the stress-tensor multiplet and $\tau_c=2.4$ for each of the infinite number of leading operators in the chiral channel. The contribution of an individual chiral operator in the $u-$channel is thus subleading for sufficiently large spin. This is similar to what happened in section \ref{sec:chiral_inv}, and while in this case the dimensions of the operators are protected, their OPE coefficients are not. Indeed, the value of these OPE coefficients remains elusive, and the best estimate we have to go on comes from the numerically obtained bounds for the operators with $\ell \leqslant 20$ presented in figure \ref{Fig:ClCH}. An interesting possibility would be to attempt to combine the numerical ranges for low spin with the estimate for the large spin OPE coefficients obtained from \eqref{eq:nonchiral_lead}. The numerical bounds on the OPE coefficients would turn into an estimate, in the form of an interval, for the anomalous dimension; we leave this exploration for future work. Here we apply the inversion formula \eqref{eq:CHgeneratingbos} only to the exchange of the identity and stress-tensor multiplets
\be 
C^t(z,\beta) \supset \int\limits_0^1 \frac{\d \zb}{\zb^2} \kappa_{\beta} k_\beta^{1,1}(\zb) \dDisc\left[  \frac{(z \zb)^{r_0+1} }{((1-z)(1-\zb ))^{r_0} } \left( 1+ |\lambda_{\phi \bar{\phi} \hat{\CC}_{0,0}}|^2 \GG_{2,0}(1-z,1-\zb)\right)\right]\,,
\label{eq:CHgeneratingsuper}
\ee
where one should recall that $\Delta_{12}=\Delta_{34}= \frac{\NN}{2}=1$ when taking the double-discontinuity.


Like before, the exchange of the identity in \eqref{eq:CHgeneratingsuper} gives the existence of double-twist operators $\left[\phi \bar{\phi}\right]_{m,\ell}$, with dimensions 
\be
\Delta_{\left[\phi \bar{\phi}\right]_{m,\ell}} \underset{\ell \gg 1}{\longrightarrow} 2 r_0+2m + \ell \,.
\label{eq:nonchiraldoubletwist}
\ee
Computing the OPE coefficients from the identity exchange we find, for the leading twist operators,
\be
|\lambda_{\phi \bar{\phi} \left[\phi \bar{\phi}\right]_{0,\ell}}|^2 \underset{\ell \gg 1}{\longrightarrow} \frac{4^{2-\ell } r_0  (r_0)_{\ell -2} (2 r_0+1)_{\ell -2}}{(1)_{\ell -2} (r_0+\ell -2) \left(r_0+\frac{1}{2}\right)_{\ell -2}} \,,
\label{eq:OPEdoubletwistsuper}
\ee
which are precisely the OPE coefficients of generalized free field theory, now decomposed in superblocks instead of bosonic blocks.

\begin{figure}[htb!]
             \begin{center}           
              \includegraphics[scale=0.35]{figures/anomCH.pdf}
              \caption{Anomalous dimension ($\gamma_\ell = \Delta_\ell - (2 \Delta_\phi + \ell)$) of the first spin $\ell$ long multiplet in the non-chiral channel. The colored dots are the dimension estimates extracted from the extremal functionals of the various bounds (figures \ref{Fig:c_min} and \ref{Fig:Bbound}-\ref{Fig:Cl24} for $c=\tfrac{11}{30}$) as indicated by their colors, with $\Lambda=34$. The dashed line corresponds to the result from the inversion formula \eqref{eq:CHgeneratingsuper}, for $c=\tfrac{11}{30}$ and external dimension $r_0=\frac65$, taking into account only the exchange of the identity and stress tensor in the $t-$channel, and is thus an approximate result for sufficiently large spin.
              }
              \label{Fig:anomCH}
            \end{center}
\end{figure}

The stress-tensor exchange provides corrections to these dimensions and OPE coefficients. As an illustration we computed its contribution to the anomalous dimensions of the leading twist operators $\left[\phi \bar{\phi}\right]_{0,\ell}$, $\gamma_\ell= \Delta_\ell - (2 \Delta_\phi + \ell)$. From the numerical estimates (see figure \ref{Fig:anomCH}) we see the anomalous dimensions starting at spin one are rather small, and so we simply take the zeroth order of the procedure outlined in \cite{Caron-Huot:2017vep} to commute the $z \to 0$ limit with the sum over primaries in \eqref{eq:CHgeneratingsuper}. These results are also shown in figure \ref{Fig:anomCH} for $\ell \geqslant1$ as a dashed blue line, together with estimates for these values arising from the various extremal functionals of section \ref{sec:numericsh0}, color coded according to which bound they came from.\footnote{We omitted two spin seven dimensions, as we could not accurately estimate them from the functionals. The two points that appear to be outlying in spin $6$ and $7$ correspond to cases where there were two zeros of the functional very close to one another, and we extracted the dimension of the first. We expect that higher derivative orders would fix both situations.} We are plotting the results starting from spin $\ell=1$. 
The leading $\ell=0$ operator is the stress tensor itself, which was not present in the generalized free field theory solution. As such the dimension of $\left[\phi \bar{\phi}\right]_{0,0}$ must come down from $2r_0=2.4$ to exactly $2$. The value of the anomalous dimensions coming from \eqref{eq:CHgeneratingsuper} is still insufficient for  this to happen, as clear from figure \ref{Fig:anomCH}.
For $\ell \geqslant1$, however, the numerical estimates of leading twist operators' dimensions are very close the values of double-twist operators \eqref{eq:nonchiraldoubletwist}. Indeed, the maximum anomalous dimension in figure \ref{Fig:anomCH}, ignoring the two out-lying points, is of the order of $\gamma_1 \sim 0.04$, in a dimension that is close to $2r_0+1 =3.4$.
The anomalous dimensions obtained from \eqref{eq:CHgeneratingsuper} (dashed blue line in \eqref{Fig:anomCH}) are close to the numerically obtained values starting from  $\ell=2$, despite the fact that our results are only valid for sufficiently large spin, as we have only considered the identity and stress tensor contributions in the $t-$channel, and completely disregarded any $u-$channel contribution.  In particular, for spin $\ell \gtrsim 8$ the numerical estimates arising from the different extremization problems of section \ref{sec:numericsh0} are all cluttered, approaching the value \eqref{eq:nonchiraldoubletwist}, and close to the values coming from \eqref{eq:CHgeneratingsuper}.

\bigskip

All in all, we have seen that both in the chiral and non-chiral channels the estimates coming from applying the inversion formula, and providing only the leading twist operators (identity plus stress-tensor supermultiplet), come very close to the numerically obtained bounds/estimates. This leaves us optimistic that the spectrum of the $(A_1,A_2)$ can be bootstrapped, similarly to the $3d$ Ising model. The numerical results for $\NN=2$ theories suffer from slow convergence and thus the estimates for OPE coefficients and anomalous dimensions we obtain are not yet with the precision of those of the $3d$ Ising model. By using this data as input to the inversion formulas they would in turn produce ranges for the various quantities appearing in the chiral and non-chiral OPEs.
Finally, another direction corresponds to using the output of each inversion formula as input for the other to obtain better estimates. We leave these two directions for future work.