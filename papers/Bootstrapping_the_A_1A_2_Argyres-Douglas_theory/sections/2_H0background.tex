%!TEX root = ../H0bootstrap.tex
%%%%%%%%%%%%%%%%%%%%%%%%%%%%%%%%%%%%%%%%%%%%%
\section{The \texorpdfstring{$(A_1,A_2)$}{(A1,A2)} Argyres-Douglas theory}
\label{sec:h0background}
%%%%%%%%%%%%%%%%%%%%%%%%%%%%%%%%%%%%%%%%%%%%%


Argyres-Douglas theories \cite{Argyres:1995jj,Argyres:1995xn} were first obtained by going to a special point on the Coulomb branch of an $\NN=2$ theory in which several BPS particles, with mutually non-local charges, become massless simultaneously. 
Among the various Argyres-Douglas models a particular class appears to be the ``simplest'', that is, the $(A_1,A_{2n})$ theories obtained in \cite{Eguchi:1996vu}. 
They are rank $n$ theories, 
\ie, their Coulomb branches have complex dimension $n$, and have trivial Higgs branches \cite{Argyres:2012fu}. 
The  chiral algebras associated to the $(A_1,A_{2n})$ theories have been conjectured to be the non-unitary series of Virasoro minimal models $\MM_{2,2n+3}$ \cite{rastelli_harvard,Beem:2017ooy}, and in this sense the theories could be argued to be ``simple''. 

In this paper we focus on the $n=1$ case of the $(A_1,A_{2n})$ Argyres-Douglas family, which is of rank one and thus the simplest in this class. In fact, among all interacting rank one SCFTs obtained through the systematic classification of \cite{Argyres:2015ffa,Argyres:2015gha,Argyres:2016xua,Argyres:2016xmc,Argyres:2016yzz}, it corresponds to the theory with the smallest $a$-anomaly coefficient,\footnote{This assumes the standard lore that the Coulomb branch chiral ring is freely generated. While this is true for all known SCFTs, there is no proof of this fact in a generic SCFT. See \cite{Argyres:2017tmj} for an exploration of theories which could have relations on the Coulomb branch.} which provides a measure of degrees of freedom in CFT \cite{Komargodski:2011vj}.
This theory was originally obtained on the Coulomb branch of a pure $\SU(3)$ gauge theory, or alternatively from an $\SU(2)$ gauge theory with a single hypermultiplet \cite{Argyres:1995jj,Argyres:1995xn}.
There is no standard nomenclature for this model, and in this paper we follow the $(A_1,A_2)$ naming convention based on its BPS quiver \cite{Cecotti:2010fi}. To emphasize its original construction it was also named $AD_{N_f=0}(\SU(3))$ and $AD_{N_f=1}(\SU(2))$  in \cite{Tachikawa:2013kta}. Finally, it can also be realized in  F-theory, on a single D3-brane probing a codimension one singularity of type $H_0$ where the dilaton is constant \cite{Dasgupta:1996ij,Aharony:1998xz}. 
For this reason the theory is often referred to as the $H_0$ theory.


The $(A_1,A_2)$ theory is an intrinsically interacting isolated fixed point with no marginal coupling: it does not have a conformal manifold nor a weak-coupling expansion. Recently, there has been progress in obtaining RG flows from $\NN=1$ Lagrangian theories that end on Argyres-Douglas SCFTs in the IR \cite{Maruyoshi:2016aim,Maruyoshi:2016tqk,Agarwal:2016pjo,Agarwal:2017roi,Benvenuti:2017bpg}, and in particular the $(A_1,A_2)$ theory can be obtained starting from a deformation of $\SU(2)$  $\NN=2$ superconformal QCD. This allows for the computation of some information about the theory, such as the superconformal index. 
As quoted in \eqref{eq:acanomalies_dimension} the values of the $a$- and $c$-anomaly coefficients are known, first obtained through a holographic computation in \cite{Aharony:2016kai},
and the dimension of the single generator of the Coulomb branch chiral ring is also known, and given in~\eqref{eq:acanomalies_dimension}.
Coulomb branch chiral ring operators can be associated with the scalar primaries of $\NN=2$ chiral operators, $\EE_{r}$ multiplets in the notation of \cite{Dolan:2002zh}, and this implies this SCFT must contain an operator  with $r_0=\Delta_{\phi}=\frac65$, as well as its conjugate.


The chiral algebra of the $(A_1,A_2)$ theory is conjectured to be the $\MM_{2,5}$ minimal model \cite{rastelli_harvard,Beem:2017ooy}, also known as the Yang-Lee edge singularity.
The first indication of this conjecture comes from the central charge. The basic chiral algebra dictionary states that $4d$ and $2d$ central charges are related by $c_{2d} = -12c_{4d}$; for the $(A_1,A_2)$ theory this gives $c_{2d} = -\frac{22}{5}$, which is indeed the correct value for the Yang-Lee model. 
Thanks to the interplay between $2d$ and $4d$ descriptions one can actually prove that $c_{4d} \geqslant \frac{11}{30}$ for any interacting $\NN=2$ SCFTs \cite{Liendo:2015ofa}.\footnote{Similar bounds can be obtained for $\NN=3$ \cite{Cornagliotto:2017dup} and $\NN=4$ \cite{Beem:2013qxa,Beem:2016wfs} theories, and also for $\NN=2$ theories with flavor symmetries  \cite{Beem:2013sza,Lemos:2015orc,Beem:2017ooy}.} This bound is saturated by the  $(A_1,A_2)$ theory, which in some sense sits at the origin of the $\NN=2$ theory space, as all other interacting SCFTs must have higher values of the $c$-central charge.
Another entry of the chiral algebra dictionary states that the Schur limit of the superconformal index \cite{Kinney:2005ej,Gadde:2011uv,Rastelli:2014jja} should match the $2d$ vacuum character. For the Yang-Lee minimal model the vacuum character seems to match the expression for the Schur index proposed in \cite{Cordova:2015nma}, while the character of the non-vacuum module has been matched to the index in the presence of a surface defect \cite{Cordova:2017mhb}.
Using the Yang-Lee model we can compute three-point functions of Schur operators, \ie, the operators captured by the chiral algebra,
modulo ambiguities when lifting operators from the $2d$ chiral algebra to representations of the four-dimensional superconformal algebra. A conjectured prescription on how to lift these ambiguities for the $(A_1,A_2)$ Argyres-Douglas theory has been put forward in \cite{Song:2016yfd}. Coulomb branch chiral ring operators, however, are not captured by the chiral algebra.


The features described above suggest that the $(A_1,A_2)$ theory might be the simplest $\NN=2$ interacting SCFT. Despite this, apart from the aforementioned quantities not much is known about the CFT data of this theory.
The fact that the theory has a Coulomb branch operator of relatively low dimension, $r_0=\frac65$, and a very low $c$ central charge, makes it well suited for the bootstrap program. 
Hence, the goal of this paper is to use modern bootstrap tools in order to access \emph{non-protected} dynamical data. 
The natural first step
is to study the two operators that are guaranteed to be present:
the stress tensor, and the $\NN=2$ chiral operator that parametrizes the Coulomb branch. Since the superconformal blocks of the former remain elusive we focus on the latter.
A preliminary analysis of chiral correlators was already started in \cite{Beem:2014zpa,Lemos:2015awa}, however the main goal of those papers was the exploration of the landscape of $\NN=2$ SCFTs through their Coulomb branch data. 
In the following sections we instead focus exclusively on the $(A_1,A_2)$ theory, and attempt to ``zoom in'' on it by studying an $\NN=2$ chiral operator of fixed dimension $r_0=\frac65$.\footnote{There are other known SCFTs with a Coulomb branch chiral ring operator of dimension $r_0=\frac65$, in particular higher rank theories whose lowest dimensional Coulomb branch generator has this dimension are obtained in F-theory by probing a singularity of type $H_0$ with $N$ D3-branes \cite{Dasgupta:1996ij,Aharony:1998xz}. However, these theories have larger values of the $c$-anomaly coefficient \cite{Aharony:2016kai}, and by fixing the central charge we can focus on the $(A_1,A_2)$ theory.}


%!TEX root = ../H0bootstrap.tex
%%%%%%%%%%%%%%%%%%%%%%%%%%%%%%%%%%%%%%%%%%%%%

%%%%%%%%%%%%%%%%%%%%%%%%%%%%%%%%%%%%%%%%%%%%%
\subsection{OPE decomposition and crossing symmetry}
%%%%%%%%%%%%%%%%%%%%%%%%%%%%%%%%%%%%%%%%%%%%%

As we just discussed, our angle to attack the $(A_1,A_2)$ theory is through its Coulomb branch, and thus we are interested in the  $\NN=2$ chiral and anti-chiral operators, respectively $\EE_{r}$ and $\bar{\EE}_{r}$ multiplets, using the naming conventions of \cite{Dolan:2002zh}. 
These are short representations of the superconformal algebra that are half-BPS, where the superconformal primary  is annihilated by all supercharges of one chirality. We denote the superconformal primary of the chiral (anti-chiral) multiplets $\EE_{r}$  ($\bar{\EE}_{r}$) by  $\phi_{r}$ ($\bar{\phi}_{-r}$), where $r$ is the $U(1)_r$ charge of the superconformal primary, with unitarity requiring $r \geqslant 1$ ($-r \geqslant 1$).
The dimensions of the superconformal primaries $\phi_{r}$ ($\bar{\phi}_{-r}$) are fixed in terms of their $U(1)_r$ charges by $\Delta_\phi= r$ ($\Delta_{\bar{\phi}}=-r$). We refer the reader to, \eg, \cite{Dolan:2002zh}, for more on representation theory of the $\NN=2$ superconformal algebra.


The numerical bootstrap program applied to chiral correlators was considered in \cite{Beem:2014zpa,Lemos:2015awa} for the case of two identical operators, and their conjugates, and in \cite{Lemos:2015awa} for two distinct operators, and their conjugates.
Here we briefly review the setup for two identical operators $\EE_{r}$, and conjugates, and refer the reader to  \cite{Beem:2014zpa,Lemos:2015awa} for a more detailed account.
Considering all four-point functions involving the superconformal primaries of these multiplets, we write down the OPE selection rules and conformal block decompositions for all of the channels, and the crossing equations to be studied in sections \ref{sec:numericsh0} and \ref{sec:anlytical}. 
In this work we are only concerned with the $(A_1,A_2)$ theory and thus we fix $r$ to $r_0=\frac65$, according to \eqref{eq:acanomalies_dimension}.


%%%%%%%%%%%%%%%%%%%%%%%%%%%%%%%%%%%%%%%%%%%%%
\subsubsection{Non-chiral channel}
%%%%%%%%%%%%%%%%%%%%%%%%%%%%%%%%%%%%%%%%%%%%%
The OPE selection rules in the non-chiral channel are \cite{Beem:2014zpa}
\be
\phi_{r} \times \bar{\phi}_{-r}  \sim \mathbf{1} + \hat{\CC}_{0 (j,j)} + \AA^{\Delta > 2j+2}_{0, 0 (j,j)}\, .
\label{eq:selrulesnonchiral}
\ee
Here the $\hat{\CC}_{0 (j,j)}$ multiplets include conserved currents of spin $2j+2$, which for $j >0$ are absent in interacting theories \cite{Maldacena:2011jn,Alba:2013yda} and thus we will set them to zero. The multiplet $\hat{\CC}_{0(0,0)}$ corresponds to the superconformal multiplet that contains the stress tensor. By an abuse of notation we will often replace the subscript $(j,j)$ by $\ell$, with $\ell=2j$.
The superconformal block decomposition in this channel can be written as
\be
\langle \phi_{r}(x_1) \bar{\phi}_{-r}(x_2) \phi_{r}(x_3) \bar{\phi}_{-r}(x_4)  \rangle = 
\frac{1}{x_{12}^{2\Delta_{\phi}}x_{34}^{2\Delta_{\phi}}} \sum_{\OO_{\Delta,\ell}} |\lambda_{\phi \bar{\phi} \OO}|^2 \GG_{\Delta, \ell}(z,\bar{z})\, ,
\label{eq:nonchiral}
\ee
where the superblocks $\GG_{\Delta, \ell}(z,\bar{z})$, capturing the supersymmetric multiplets being exchanged in \eqref{eq:selrulesnonchiral}, were computed in \cite{Fitzpatrick:2014oza},
\be 
\label{eq:superblock}
\GG_{\Delta, \ell}(z,\bar{z}) = (z \bar{z})^{-\frac{\NN}{2}}g_{\Delta + \NN,\ell}^{\NN,\NN}(z,\bar{z})\, .
\ee
Here we wrote the blocks for $\NN=1,2$ chiral operators, since both cases can be treated almost simultaneously \cite{Fitzpatrick:2014oza,Lemos:2015awa}, but hereafter we focus only on the case $\NN=2$. 
The function $g_{\Delta,\ell}^{\Delta_{12},\Delta_{34}}(z,\bar{z})$ is the standard bosonic block for the decomposition of a correlation function with four distinct operators, defined in \eqref{eq:bosblock}. Although not immediately obvious, the bosonic block with shifted arguments in \eqref{eq:superblock} can be written as a finite sum of $g_{\Delta,\ell}^{0,0}(z,\bar{z})$ blocks, as expected from supersymmetry. The block reduces to $1$ for the identity exchange, \ie, $\Delta=\ell=0$.

The stress-tensor multiplet $\hat{\CC}_{0 (0,0)}$ corresponds to $\Delta=2$, $\ell=0$ in \eqref{eq:superblock}, and its OPE coefficient can be fixed using the Ward identities (see for example \cite{Beem:2014zpa}):
\be
\left|\lambda_{\phi \bar{\phi} \OO_{\Delta=2,\ell=0}}\right|^2 = \frac{\Delta_\phi^2 }{6 c}\,,
\label{eq:STOPEcoeff}
\ee
while long multiplets $\AA^{\Delta > \ell +2}_{0, 0, \ell}$ contribute as \eqref{eq:superblock} with $\Delta > \ell +2$.

When writing the crossing equations it will be useful to have the block expansion with a slightly different ordering
\be 
\langle \bar{\phi}_{-r}(x_1)  \phi_{r}(x_2) \phi_{r}(x_3) \bar{\phi}_{-r}(x_4)  \rangle = 
\frac{1}{x_{12}^{2\Delta_{\phi}} x_{34}^{2\Delta_{\phi}}} \sum_{\OO_{\Delta,\ell}} (-1)^{\ell} |\lambda_{ \phi  \bar{\phi} \OO}|^2\tilde{\GG}_{\Delta, \ell}(z,\bar{z})\,,
\label{eq:nonchiralbraid}
\ee
where the function $\tilde{\GG}_{\Delta, \ell}(z,\bar{z})(z,\bar{z})$ is defined as
\be 
\label{eq:superblockbraid}
\tilde{\GG}_{\Delta, \ell}(z,\bar{z}) =  (z \bar{z})^{-\frac{\NN}{2}} g_{\Delta + \NN,\ell}^{\NN,-\NN}(z,\bar{z})\, ,
\ee
and again we are only interested in the case $\NN=2$.


%%%%%%%%%%%%%%%%%%%%%%%%%%%%%%%%%%%%%%%%%%%%%
\subsubsection{Chiral channel}
%%%%%%%%%%%%%%%%%%%%%%%%%%%%%%%%%%%%%%%%%%%%%

The OPE selection rules of two identical $\NN=2$ chiral primary operators read \cite{Beem:2014zpa}
\be 
\phi_{r} \times \phi_{r} \sim   \EE_{2r}    + \CC_{0, 2r- 1 (j-1,j) } 
 + \BB_{1, 2r-1 (0,0)} + \CC_{\frac12, 2r- \frac32 (j-\frac12,j)} + \AA_{0, 2r-2 (j,j)}^{\Delta > 2+2r+2j}   \,,
\label{eq:selruleschiral}
\ee
where we already imposed Bose symmetry, and we assumed $\phi_r$ to be above the unitarity bound, \ie, $r >1$. If one considers different operators, or if $r=1$, additional multiplets are allowed to appear (see \eg, \cite{Lemos:2015awa}).
Chirality of $\phi_{r}$ requires each supermultiplet contributes with a single conformal family, and therefore the superblock decomposition contains only bosonic blocks:
\be
\label{eq:chiral}
\langle \phi_r(x_1) \phi_r(x_2) \bar{\phi}_{-r}(x_3) \bar{\phi}_{-r}(x_4) \rangle =  \sum_{\Delta,\ell} |\lambda_{\phi \phi \OO_{\Delta, \ell}}|^2 g^{0,0}_{\Delta,\ell}(z,\bar{z})\,.
\ee
Since we are considering the OPE between two identical $\phi_r$ multiplets, Bose symmetry requires 
the above sum to include only even $\ell$.
The precise contribution from each of the multiplets appearing in \eqref{eq:selruleschiral} is the following
\begin{equation}
\begin{alignedat}{4}
&{\AA_{0, 2r-2 (j,j)}}							&:\qquad &	g^{0,0}_{\Delta,\ell}\,,				\qquad &\Delta > 2+2r+\ell\,, \; \ell \; \mathrm{even}\,,\\
&{\CC_{\tfrac12, 2r-\tfrac32 (j-\tfrac12,j) }}	&:\qquad &	g^{0,0}_{\Delta=2r+\ell+2,\ell}\,,		\qquad & \ell \geqslant 2 \,,\;\ell\; \mathrm{even}\,, \\
&{\BB_{1, 2r-1 (0,0)}}							&:\qquad &	g^{0,0}_{\Delta=2r+2,\ell=0}\,,		\qquad &~\\
&{\CC_{0, 2r- 1 (j-1,j) }}						&:\qquad &	g^{0,0}_{\Delta=2r+\ell,\ell}\,,		\qquad & \ell \geqslant 2\,,\;\ell\; \mathrm{even}\,,\\
&{\EE_{2r}}										&:\qquad &	g^{0,0}_{\Delta=2r,\ell=0}\,,			\qquad &~ 
\end{alignedat}
\label{eq:chiralblocks}
\end{equation}
where $\ell= 2j$ is even (see \cite{Lemos:2015awa} for the contribution in the case of different operators).
While the short multiplets being exchanged in this channel have their dimensions fixed by supersymmetry, their OPE coefficients are not known. In fact, it is not even guaranteed all these multiplets are present as their physical meaning is not as clear as the short multiplets exchanged in the non-chiral channel. The $\EE_{2r}$ multiplet corresponds to an operator in the Coulomb branch chiral ring and therefore must be present in the $(A_1,A_2)$ theory, although the value of its OPE coefficient is not known. 
The scalar primary of the $\BB_{1, 2r-1 (0,0)}$ multiplet may be identified with a mixed branch chiral ring operator \cite{Argyres:2015ffa}, and since the $(A_1,A_2)$ has no mixed branch, one might expect this multiplet to be absent. We must point out, however, that the identification of this multiplet with the mixed branch is conjectural, it could be that the multiplet is present but it does not correspond to a flat direction \cite{Tachikawa:2013kta,Argyres:2015ffa}. 
Note that the contribution of the two short operators $\BB_{1, 2r-1 (0,0)}$ and $\CC_{\frac12, 2r- \frac32 (j-\frac12,j)}$ is identical to that of a long multiplet saturating the unitarity bound $\Delta= 2 + 2r + \ell$, as follows directly from the decomposition of the long multiplet when hitting the unitarity bound \cite{Dolan:2002zh}. On the other hand, the contribution of the short multiplets $\EE_{2r}$ and $\CC_{0 , 2r- 1 (j-1,j) } $ is isolated from the continuous spectrum of long operators by a gap; this will be relevant for the numerical analysis of section \ref{sec:numericsh0}.

%%%%%%%%%%%%%%%%%%%%%%%%%%%%%%%%%%%%%%%%%%%%%
\subsubsection{Crossing symmetry}
%%%%%%%%%%%%%%%%%%%%%%%%%%%%%%%%%%%%%%%%%%%%%

We are now ready to quote the crossing equations to be studied in the subsequent sections, where we recall only even spins are allowed in the $\phi \phi$ OPE,
%
\begin{subequations}
\be 
(z \zb)^{\Delta_\phi} \sum\limits_{\OO \in \phi \phi} |\lambda_{\phi \phi \OO}|^2 g^{0,0}_{\Delta_\OO, \ell_\OO} (1-z,1-\zb) = ((1-z)(1-\zb))^{\Delta_\phi }\sum\limits_{\OO \in \phi \bar\phi} |\lambda_{\phi \bar \phi \OO}|^2 (-1)^\ell \tilde{\GG}_{\Delta_\OO, \ell_\OO}(z,\zb) \,,
\label{eq:chiralnonchiral}
\ee   
\be 
((1-z)(1-\zb))^{\Delta_\phi} \sum\limits_{\OO \in \phi \bar\phi} |\lambda_{\phi \bar \phi \OO}|^2 \GG_{\Delta_\OO, \ell_\OO} (z,\zb) 
= (z \zb)^{\Delta_\phi} \sum\limits_{\OO \in \phi \bar\phi} |\lambda_{\phi \bar \phi \OO}|^2 \GG_{\Delta_\OO, \ell_\OO}(1-z,1-\zb)\,.
\label{eq:nonchiralnonchiral}
\ee 
\label{eq:crossingeqs}
\end{subequations}
%
The full system of equations comprises \eqref{eq:crossingeqs}  together with equation \eqref{eq:chiralnonchiral} with $z \to 1-z$ and $\zb \to 1-\zb$. These are collected in a form suitable for the numerical implementation in \eqref{eq:cross_numerics}. 


%%%%%%%%%%%%%%%%%%%%%%%%%%%%%%%%%%%%%%%%%%%%%
\subsubsection{Numerical bootstrap}
\label{subsubsec:implementation}
%%%%%%%%%%%%%%%%%%%%%%%%%%%%%%%%%%%%%%%%%%%%%

In this short section we give details of the numerical implementation that will be necessary to understand the results of subsequent sections. Schematically, the final form of the crossing equations given in \eqref{eq:cross_numerics} is
\be 
|\lambda_{\OO^\star}|^2 \vec{V}_{\OO^\star} + \sum\limits_{\OO} |\lambda_{\OO}|^2 \vec{V}_{\OO} +\vec{V}_{\mathrm{fixed}} = 0\,.
\ee
Here $\OO^\star$ is a superconformal multiplet whose OPE coefficient we would like to bound numerically. 
The term $\vec{V}_{\mathrm{fixed}} $ encodes the contribution of the identity, or of the identity and stress tensor if we fix the central charge $c$, and is given in \eqref{eq:idandst}. 
OPE coefficient bounds are obtained using the SDPB solver of \cite{Simmons-Duffin:2015qma} to solve the following optimization problem
\be
\begin{split}
&\vec{\Psi}\cdot \vec{V}_{\OO} \geqslant 0 \,, \qquad \forall \; \OO\in \{\text{trial spectrum}\}\,,\\
&\vec{\Psi}\cdot \vec{V}_{\OO^\star} = \pm 1\,,  \\
&\mathrm{Maximize}\left(\vec{\Psi}\cdot \vec{V}_{\mathrm{fixed}}\right)\,,
\end{split}
\label{eq:optimiz}
\ee
where the minus sign in the second line can be consistently imposed at the same time as the first line, \textit{only} when the contribution of  $\OO^\star$ is isolated from the contribution of the remaining $\OO\in \{\text{trial spectrum}\}$ \cite{Poland:2011ey}.
As is standard in the bootstrap literature, we truncate the infinite-dimensional  functional as
\be  
\vec{\Psi} = \sum\limits_{m,n}^{m+n \leqslant\Lambda} \vec{\Psi}_{m,n} \partial_z^m \partial_{\zb}^n \Big\vert_{z= \zb=\frac12}\,.
\ee
The result of the extremization problem \eqref{eq:optimiz} provides a bound on the OPE coefficient of $\OO^\star$ as
\be 
\pm |\lambda_{\OO^\star}|^2 \leqslant -\mathrm{Max}\left(\Psi\cdot \vec{V}_{\mathrm{fixed}}\right)\,.
\ee
When the bound is saturated, there is a unique solution to the (truncated) crossing equations \cite{Poland:2010wg,ElShowk:2012hu}, with different extremization problems possibly leading to different solutions.\footnote{See, however, \cite{Behan:2017rca} for subtitles that arise when considering systems of mixed correlators.} 
At finite $\Lambda$, this corresponds to an approximate solution to the full crossing system, with the spectrum encoded in the extremal functional \cite{ElShowk:2012hu}.
We refer the reader to, \eg, \cite{Poland:2011ey,Simmons-Duffin:2015qma,Simmons-Duffin:2016gjk} for more technical details pertaining the numerical bootstrap.
%!TEX root = ../H0bootstrap.tex
%%%%%%%%%%%%%%%%%%%%%%%%%%%%%%%%%%%%%%%%%%%%%

%%%%%%%%%%%%%%%%%%%%%%%%%%%%%%%%%%%%%%%%%%%%%
\subsection{The Superconformal Index}
%%%%%%%%%%%%%%%%%%%%%%%%%%%%%%%%%%%%%%%%%%%%%

In the non-chiral channel the OPE coefficients of all short multiplets can be fixed, as they correspond either to the stress-tensor multiplet $\hat{\CC}_{0,0}$, which contributes as \eqref{eq:STOPEcoeff}, or to multiplets containing conserved currents of spin greater than two  $\hat{\CC}_{0,\ell >0}$ which are absent in interacting theories \cite{Maldacena:2011jn,Alba:2013yda}.  
In the chiral OPE selection rules \eqref{eq:selruleschiral}, there appear short multiplets whose dimensions are fixed, but whose OPE coefficients are not known, and could even be absent in the $(A_1,A_2)$ theory. 
To access the spectrum of short operators, a useful quantity is the superconformal index \cite{Kinney:2005ej,Gadde:2011uv,Rastelli:2014jja}. The index is the most general invariant that counts, with signs, short multiplets up to multiplets that can recombine to form a long multiplet. Because short multiplets that have the right quantum numbers to recombine in a long multiplet give zero contribution, the index has some intrinsic ambiguities. Relevant to us are the following recombination rules
\be 
\begin{split}
\AA^{2R+r+2 j_2 +2}_{R,r(j_1,j_2)} &\to \CC_{R,r(j_1,j_2)} \oplus \CC_{R+\frac{1}{2},r+\frac{1}{2}(j_1-\frac{1}{2},j_2)}\,,\\
\AA^{2R+r+2 j_2 +2}_{R,r(0,j_2)} &\to \CC_{R,r(0,j_2)} \oplus \BB_{R+1,r+\frac{1}{2}(0,j_2)}\,,
\end{split}
\label{eq:recombination}
\ee
where the latter can be seen as a special case of the former with the identification $\CC_{R,r(-\frac{1}{2},j_2)} = \BB_{R+\frac{1}{2},r(0,j_2)}$.
This means that, while the multiplets $\BB_{1, 2r-1 (0,0)}$, $\CC_{0, 2r- 1 (j-1,j) }$, and $\CC_{\frac12, 2r- \frac32 (j-\frac12,j)}$ appearing in \eqref{eq:selruleschiral} contribute to the superconformal index, we can only see if these multiplets appear modulo pairs of the type \eqref{eq:recombination}. Note that the $\EE_r$, $\bar{\EE}_{-r}$ multiplets themselves can never recombine.
 
The \emph{full} superconformal index of the $(A_1,A_2)$  theory has been computed in \cite{Maruyoshi:2016tqk} using a $4d$ $\NN=1$ Lagrangian theory which flows to the $(A_1,A_2)$ theory in the IR.\footnote{Various limits of the superconformal index of Argyres-Douglas theories had been  obtained before in \cite{Buican:2015ina,Buican:2015tda,Cordova:2015nma,Song:2015wta}.}
The final expression for the index is given in integral form in equation (16) of \cite{Maruyoshi:2016tqk}, which we use to gather information about the spectrum of short multiplets in the theory. Expanding said expression we find, unsurprisingly, that the Coulomb branch chiral ring operators $\EE_{k \frac65}$, with integer $k\geqslant1$ are present (since we explored the index in an expansion this was only checked for low values of $k$). In section \ref{sec:OPEbounds} we shall bound, from above and from below, the OPE coefficient of the $\EE_{\frac{12}5}$ operator appearing in the $\EE_{\frac65}$ self OPE (see \eqref{eq:Ebound}).
From the index we also find that the operator $\CC_{0,\frac75(0,1)}$  is present in the spectrum, which corresponds to the leading-twist contribution of spin two in \eqref{eq:chiralblocks}. We shall recover this result in section \ref{sec:OPEbounds}, since we find a non-zero lower bound for its OPE coefficient (see \eqref{eq:Clbound}).
Finally, we find no contributions arising from a  $\BB_{1, \frac75(0,0)}$ multiplet implying that, modulo 
the recombination ambiguities of \eqref{eq:recombination}, it is absent in the $(A_1,A_2)$ theory. This is consistent with the expectation that such multiplets may be identified with the existence of a mixed branch \cite{Argyres:2015ffa}. Although this is not conclusive evidence for the absence of this multiplet, we take it as an indication that if there are multiple solutions to the crossing equations for $r_0=\frac65$ and $c=\tfrac{11}{30}$, then the one corresponding to the $(A_1,A_2)$ theory likely has the $\BB_{1, \frac75(0,0)}$ multiplet absent.