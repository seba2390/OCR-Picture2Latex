 \section{Introduction}
The spectrum of the Hodge Laplacian is a fundamental and well studied geometric invariant of Riemannian manifolds. The Hodge theorem partitions the positive spectrum into exact and coexact eigenvalues. For differential forms of degree one, the exact eigenvalues contain exactly the data of the Laplacian acting on functions and are well understood. The coexact spectrum however is considerably more mysterious. Recently, the first coexact eigenvalue of the Hodge Laplacian of a closed hyperbolic 3-manifold has been related to other aspects of its geometry and topology. For the function Laplacian, the first eigenvalue is known to be comparable to the square of the isoperimetric Cheeger constant. In this paper, we derive a similar estimate for the first coexact eigenvalue, building on work of Lipnowski and Stern in \cite{LS} motivated by torsion growth in finite covers. We use this new estimate to construct the first examples of hyperbolic 3-manifolds with coexact eigenvalues exponentially small compared to volume.

Given a hyperbolic 3-manifold $M$, it is natural to try to extract information about $M$ from its finite covers. A deep and interesting conjecture of Bergeron-Venkatesh, L{\^{e}, and L\"uck (see \cite{BV}, \cite{thang}, and \cite{wolf}) asks in part whether the volume of $M$ can be found by studying the torsion in the homology of a family of finite covers of $M$. In studying this question, Bergeron, \c Seng\"un, and Venkatesh in \cite{BSV} relate the growth rate of the cardinality of the torsion in the first homology of a tower of covers of a closed \textit{arithmetic} hyperbolic 3-manifold to the spectrum of the Laplacian on 1-forms. In particular, they prove the following theorem, where the technical definitions are given below.

\begin{thm} (\cite{BSV}) If a sequence $M_n\to M_0$ of congruence covers of an arithmetic hyperbolic 3-manifold $M_0$ satisfies the few small eigenvalues, small Betti numbers, and simple cycles conditions, then the log torsion growth rate is proportional to the volume. In particular, one has \[\lim\limits_{n\to\infty} \frac{\log\left|H_1(M_n;\Z)_{\emph{torsion}}\right|}{\vol(M_n)} = \frac{1}{6\pi}.\]
\end{thm}


The conditions appearing in Theorem 1.1. are:

\begin{enumerate}
    \item (Few small eigenvalues) For all $\e>0$ there is a $c>0$ such that $$\limsup_{n\to\infty} \frac{1}{\vol(M_n)}\sum\limits_{0<\lambda<c}|\log\lambda| \leq \e,$$
where $\lambda$ runs over the eigenvalues of the 1-form Laplacian on $M_n$.

    \item  (Small Betti numbers) $b_1(M_n) = o\left(\frac{\vol(M_n)}{\log \vol(M_n)}\right)$.

    \item (Simple cycles) There exists a constant $C$ depending on $M_0$ such that $H_2(M_n;\Z)$ admits a basis of surfaces $[S_i]$ with bounded Thurston norm $||[S_i]||_{Th}\ll \vol(M)^C$.
\end{enumerate}

Bergeron, \c{S}eng\"un, and Venkatesh’s theorem provides a version of L\"uck approximation for the limiting $L^2$-torsion of sequences of manifolds that satisfy these conditions. However, no examples of such a sequence are known to exist. Encouraged by the Betti number approximation theorem of \cite{samurai}, one would hope the above conditions also imply the convergence of $L^2$-torsion for families of manifolds that Benjamini-Schramm converge to $\H^3$. Examples of Brock and Dunfield show that Benjamini-Schramm convergence itself is insufficient \cite{BDE}.

Previous work has focused on the simple cycles condition. In their paper, Bergeron, \c Seng\"un, and Venkatesh conjecture the simple cycles condition is satisfied for all arithmetic congruence covers, and ask in contrast if there are families of closed hyperbolic 3-manifolds with injectivity radius bounded below and volume going to infinity that do not satisfy it. This question was answered by Brock and Dunfield in \cite{BD}, where they construct a sequence of closed hyperbolic 3-manifolds $W_n$ with injectivity radius bounded below and volume going to infinity such that $H_2(W_n;\Z)\cong \Z$ for every $n$ and for which the Thurston norm of the generator grows exponentially in the volume.

In this paper, we continue the study of the few small eigenvalues condition initiated by Lipnowski and Stern in \cite{LS}. There, it is shown that for a family of covers of a triangulated closed hyperbolic $n$-manifold, the first positive eigenvalue of the Laplacian acting on 1-forms is comparable to a certain isoperimetric ratio, where the comparison constants depend on the volume of the cover, the geometry of the base, and the specific triangulation. In this paper, we prove that for closed hyperbolic 3-manifolds satisfying a uniform lower bound on injectivity radius, such a comparison can be done with universal constants and polynomial dependence on volume (Theorems A and B, with only Theorem A requiring the restriction to dimension 3). We then leverage the comparison in Theorem A to show that a specific family of closed hyperbolic 3-manifolds have first positive eigenvalue of the 1-form Laplacian vanishing exponentially fast in the volume (Theorem C). The manifolds in Theorem C are not covers of a fixed base, so the estimates of \cite{LS} do not apply.

Another motivation for this work comes from recent work of Lin and Lipnowski in \cite{LL}, where for closed rational homology 3-spheres they leverage a relationship between the first eigenvalue of the Hodge Laplacian acting on coexact 1-forms and irreducible solutions to the Seiberg-Witten equations to determine if certain spaces are $L$-spaces. In particular, if a closed rational homology 3-sphere $M$ is not an $L$-space, then the first eigenvalue $\lambda$ of the Hodge Laplacian acting on coexact 1-forms satisfies $\lambda\leq 2$. Using a version of the Selberg trace formula, they relate this to the complex length spectrum of $M$. Numerical methods can then be used to verify if $\lambda \leq 2$. While the constants in our eigenvalue estimates are rather opaque, so that the isoperimetric ratio we study is not, at least presently, capable of certifying that $\lambda\leq 2$, the relationship between the stable isoperimetric ratio and whether a space is an $L$-space remains tantalizing.

\subsection{Results}
The isoperimetric ratio we study relates the topological complexity of a surface with boundary to the geometric length of its boundary. One can view the extremal value of this ratio as an analogue of the two-dimensional Cheeger constant. The topological complexity measure is given by stable commutator length. The stable commutator length of a nullhomologous loop $\gamma$ in a manifold $M$, denoted $\scl(\gamma)$, is defined to be \[\scl(\gamma) = \inf\limits_{m\geq 1}\frac{{\tt{cl}}(\gamma ^{m})}{m},\] where ${\tt{cl}}(\gamma)$ is the word length of $\gamma$ in the commutator subgroup of $\pi_1M$ with generating set all commutators. Topologically, stable commutator length corresponds to the stable complexity of a surface bounding a nullhomologous loop. Denote the subgroup of rationally nullhomologous loops by $\Gamma_{\Q}’ = \ker\left(\pi_1M\to H_1(M;\Q)\right)$.
Then, for $\gamma\in \Gamma’_{\Q}$, one has $$\scl(\gamma) = \inf\left\{\frac{\chi_-(S)}{2m}~:~S \text{ with }\d S = \gamma^m,~S\text{ is connected}\right\},$$ where for connected surfaces $\chi_-(S) = \max\{0,-\chi(S)\}$. One can think of stable commutator length as a relative version of the Thurston norm. See the monograph \cite{Calegari} for a detailed exposition of $\scl$.
Stable commutator length is closely related to area, by Gauss-Bonnet, providing some justification for the following nomenclature. Define the stable isoperimetric constant of $M$ to be \[\rho(M) = \inf\limits_{\gamma\in \Gamma’_\Q\setminus\{1\}}\frac{|\gamma|}{\scl(\gamma)}.\]

One can also define the stable area of a rationally nullhomologous loop $\gamma\in \Gamma’_{\Q}$ to be the infimal normalized area of a surface bounding a power of $\gamma$: $$\text{sArea}(\gamma) = \inf\limits_{\d S = \gamma^n}\frac{\text{Area}(S)}{n}.$$ This leads to another notion of stabilized isoperimetric ratio using stable area in place of stable commutator length: $$\rho_{\text{Area}}(M) = \inf\limits_{\gamma\in\Gamma’_{\Q}}\frac{|\gamma|}{\text{sArea}(\gamma)}.$$
Stable area is related to stable commutator length by $$\text{sArea}(\gamma) \leq 4\pi\scl(\gamma).$$ Thus, $$\rho(M) \leq 4\pi \rho_{\text{Area}}(M).$$
Very little is known generally about how geodesic length relates to stable commutator length in hyperbolic manifolds, though estimates relating length, stable area, and stable commutator length for short curves in hyperbolic manifolds have been obtained by Calegari in \cite{length}.

Our first theorem relates the coexact spectral gap to stable isoperimetric ratios of arbitrary nullhomologous curves.

\begin{bigthm}\label{thm:A}
Let $M$ be a closed hyperbolic 3-manifold with injectivity radius bounded below by $\e>0$ and let $\lambda$ denote the first eigenvalue of the Hodge Laplacian acting on coexact 1-forms. Then there is a constant $A = A(\e)$ such that for any nontrivial element $\gamma \in \Gamma’_\Q$, one has $$\sqrt{\lambda} \leq A \vol(M) \frac{|\gamma|}{ \scl( \gamma)},$$ where $|\gamma|$ denotes the geodesic length of $\gamma$.
\end{bigthm}

Theorem A has the following obvious corollary:


\begin{cor}
  Let $M$ be a hyperbolic $3$-manifold with injectivity radius bounded below by $\e>0$ and let $\lambda$ be the first eigenvalue of the Hodge Laplacian acting on coexact 1-forms. Then for the constant $A = A(\e)$ from Theorem A, $$ \sqrt{\lambda} \leq A \vol(M)\rho(M).$$
\end{cor}

The analogue of Theorem A in \cite{LS} studies a cochain version of the Hodge Laplacian introduced by Dodziuk in \cite{Dodziuk} for triangulated manifolds. This chochain Laplacian is called the Whitney Laplacian, and is induced by the Hodge Laplacian by embedding cochains into the $L^2$-de Rham complex.

\begin{thm} (\cite{LS} Theorem 1.4) Let $M_0$ be a closed hyperbolic $n$-manifold. Let $K_0$ be a sufficiently fine triangulation. Let $M$ be a finite cover of $M_0$. Let $\lambda_W(M)_{d^*}$ be the first coexact eigenvalue for the Whitney cochain Laplacian associated to the pullback of the triangulation $K_0$ to $M$. Then if some multiple of $\gamma\in\pi_1(M)$ bounds a surface, then $$\left(\frac{\scl(\gamma)}{|\gamma|}\right)^2 \leq W_{M_0}\frac{\vol(M)}{\lambda_W(M)_{d^*}},$$
for a constant $W_{M_0}$ depending on the triangulation $K_0$.
\end{thm}


Under a well behaved sequence of subdivisions, Dodziuk and Patodi in \cite{dP} showed the spectrum of the Whitney Laplacian converges to the spectrum of the Hodge Laplacian. For a fixed but very fine triangulation, the eigenvalue comparison is somewhat delicate. In this setting, Lipnowski and Stern relate the Whitney Laplacian’s first coexact eigenvalue to the Hodge Laplacian’s first coexact eigenvalue in the following way:

\begin{thm} (\cite{LS} Theorem 1.5)	 Let $M_0$ be a closed hyperbolic $n$-manifold. Let $K_0$ be a sufficiently fine triangulation of $M_0$. Let $M$ be a finite cover of $M_0.$ Let $\lambda_W(M)_{d^*}$ be the first coexact eigenvalue for the Whitney cochain Laplacian associated to the pullback of the triangulation $K_0$ to $M$. Then, $$\frac{1}{\lambda_W(M)_{d^*}}\leq \max\left\{\frac{4G_{M_0}^2\vol(M)}{\lambda_{d^*}(M)}, G_{M_0}^2C_{M_0}^2\vol(M)\right\}.$$
The constants $C_{M_0}$ and $G_{M_0}$ depend only on $K_0$.
\end{thm}

In the course of proving Theorem A, we too require a comparison of this sort. By using a smoothed version of the Whitney embedding of cochains into the de Rham complex and triangulations with uniformly controlled geometry (these are called deeply embedded triangulations and are introduced in Section 2), we prove the following Whitney-Hodge eigenvalue comparison.

\begin{thm} Let $M$ be a closed hyperbolic 3-manifold with $\inj(M)>\e$. Let $\lambda$ denote the first coexact eigenvalue for the Hodge Laplacian acting on $1$-forms and let $\lambda_W$ denote the first coexact eigenvalue for the smoothened Whitney Laplacian on 1-cochains associated to a deeply embedded triangulation. There is a constant $G = G(\e)$ such that $$\lambda \leq G \vol(M)\lambda_W.$$
\end{thm}

The smoothened version of the Whitney map needed for the above proposition is obtained by replacing the barycentric coordinates associated to a triangulation with certain smooth partitions of unity indexed by the vertices of a triangulation (an idea of Dodziuk’s \cite{dodzuik2}), which we call barycentric partitions of unity. One can then show that the forms built from these pieces have well behaved Hodge decompositions. This analysis leads to the above eigenvalue comparison for the smoothened Whitney Laplacian constructed from a barycentric partition of unity.

Our second theorem uses the isoperimetric constant $\rho(M)$ to provide a lower bound on the first coexact eigenvalue of the 1-form Laplacian.

\begin{bigthm} Let $M$ be a closed hyperbolic $n$-manifold with $\inj(M) > \e$. Let $\lambda$ be the first positive eigenvalue for the Hodge Laplacian acting on coexact 1-forms and let $H > \lambda$. Then there is a constant $P(H,\e,n)>0$ such that $$ \frac{P\rho(M)}{\vol(M)^{7/2+1/n}}\leq\sqrt{\lambda}.$$
\end{bigthm}

Theorem B corresponds to Theorem 1.5 below from \cite{LS}, which uses stable area in place of stable commutator length. Note that Theorem B above remains true if one replaces $\rho(M)$ with $\rho_{\text{Area}}(M)$.

\begin{thm}(\cite{LS} Theorem 1.3 )  Let $M_0$ be a closed hyperbolic $n$-manifold and let $M$ be a finite cover of $M_0.$ Then there are constants $A_0$ and $C$, where $A_0$ is depends only on $M_0$ and $C$ is a constant that is uniformly bounded when the injectivity radius of $M$ is bounded below and $\lambda_1^1(M)$ is bounded above, for which $$\frac{1}{\lambda_1^1(M)_{d^*}}\leq A_0C^2\vol(M)^{3/2}\diam(M)^2 \left(1+\rho_{\emph{Area}}(M)^{-1}\right).$$
\end{thm}

Our approach to proving Theorems A and B is grounded in the following dual characterizations of $\scl$. By Bavard duality, stable commutator length is related to the defect norm for quasimorphisms and the Gersten filling norm for singular 1-chains. The Gersten filling norm for a nullhomologous loop $\gamma$ is given by the infimal $\ell^1$-norm of a singular 2-chain whose boundary is a fundamental cycle for $\gamma^m$, normalized by $m$. A quasimorphism for a group $\Gamma$ is a map from $\Gamma\to \R$ that is nearly a homomorphism in the sense that its coboundary is a bounded map on $\Gamma^2$. The defect of a quasimorphism is the sup norm of its coboundary. Bavard duality says $$\scl(\gamma) = 4\fill(\gamma) = \frac{1}{2}\sup\limits_q \frac{q(\gamma)}{D(q)},$$ where the supremum is over all quasimorphisms $q$ and $D(q)$ is the defect of $q$.

One can therefore use the characterization of $\scl$ as a filling norm when bounding it from above and, similarly, the quasimorphism point of view when bounding it from below. For Theorem A, we relate the filling norm to the spectrum of the Hodge Laplacian via the Whitney Laplacian and Poincar\'e duality (which forces us to restrict to dimension 3). For Theorem B, we use de Rham quasimorphisms, which are given by integrating coclosed forms over geodesics. Studying the de Rham quasimorphism of a coexact eigenform gives the connection to the spectrum of the Hodge Laplacian.

Our methods primarily differ from \cite{LS} in that instead of studying covers of a fixed manifold with a specific triangulation, we use that closed hyperbolic manifolds with injectivity radius greater than some $\e>0$ can all be triangulated so that the simplices come from a compact collection, and instead of using an $L^2$ discretization of the eigenvalue problem, we use a smooth discretization. The local structure of these triangulations can then be compared in a uniform way, thereby allowing us to relate various combinatorial and geometric norms. By working in the smooth setting instead of the $L^2$ setting, we are able to make use of geometric estimates that require higher regularity. This leads to the more direct Whitney-Hodge eigenvalue comparison of Proposition 1.4.

As an application of the spectral gap estimate of Theorem A, we modify the construction in \cite{BD} to construct a family of closed hyperbolic manifolds for which we have control over the stable isoperimetric constant and which have injectivity radius uniformly bounded below.

\begin{bigthm}\label{thm:C} There is a family $\{W_n\}$ of closed hyperbolic 3-manifolds with injectivity radius bounded below by some $\e>0$ and volume growing linearly in $n$ such that the 1-form Laplacian spectral gap vanishes exponentially fast in relation to volume: $$\sqrt{\lambda(W_n)}\leq B \vol(W_n)e^{-r\vol(W_n)}$$ where $r$ and $B$ are positive constants and $\lambda(W_n)$ is the first positive eigenvalue of the 1-form Laplacian on $W_n$.
\end{bigthm}

The manifolds $W_n$ in Theorem $C$ are obtained by taking a hyperbolic 3-manifold with totally geodesic boundary and gluing it to itself using a particular psuedo-Anosov with several useful properties. By \cite{BMNS}, this family has geometry that up to bounded error can be understood in terms of a simple model family and therefore has the desired injectivity radius lower bound and linear volume growth.

Using this model family, we show that one can find curves with uniformly bounded length whose stable commutator length grows exponentially in the volume. This task is rather delicate, as it is quite difficult to ensure that the \emph{stable} commutator length of a given loop is large--- one typically does not know if passing to a power of the loop causes a drastic simplification in commutator length. To overcome this, we require an algebraic condition on the gluing map to control which surfaces bound the loops we study. We then use the following homological isoperimetric estimate to control stable commutator length.

 \begin{thm} Let $M$ be a compact oriented hyperbolic 3-manifold with totally geodesic boundary $\d M = S$. Let $\gamma$ be a geodesic multicurve in $S$ that is rationally nullhomologous in $M$. Let $||\cdot||_{s,S}$ be the stable norm on $H_1(S)$ induced by the Riemannian metric on $S.$ Then there is a constant $D> 0$ depending only on $M$ such that $$||[\gamma]||_{s,S}\leq D\scl_M(\gamma).$$ \end{thm}


 Theorem A then implies the first positive eigenvalue vanishes exponentially fast. We stress that while the manifolds $W_n$ are constructed from the same basic building blocks, they are not otherwise related to one another. In particular, they are not finite covers of some fixed base manifold, so the estimates of \cite{LS} do not apply; it is therefore vital that the constants appearing in Theorem A depend only on an injectivity radius lower bound.

\subsection{Brief Outline} In \hyperref[sec:2]{Section 2}, we show existence and study the local properties of the triangulations we use throughout the paper. We also introduce the smooth Whitney cochain map used to define the approximation of the Laplacian. In \hyperref[sec:3]{Section 3}, we relate various chain and cochain norms and compare these to various geometric norms.  In \hyperref[sec:4]{Section 4}, we compare the eigenvalues of the approximation of the Laplacian to the genuine eigenvalues of the Laplacian, then use this comparison and the estimates from \hyperref[sec:3]{Section 3} to  prove Theorem A. In \hyperref[sec:5]{Section 5}, we prove Theorem B. Finally, in \hyperref[sec:6]{Section 6}, we compare the homological length and stable commutator length of certain curves and construct the example of Theorem C. \hyperref[sec:6]{Section 6} makes use of Theorem A, but is otherwise independent from the rest of the paper.

\begin{remark}
Throughout the paper, numerous constants are used. Constants defined inside proofs have no meaning outside the local setting of the proof. The letter $C$ is repeatedly reused in Sections 2 and 3 to denote a constant coming from a Sobolev type estimate and can at any time be taken to be the maximum among all constants denoted by $C$. Such constants depend only on injectivity radius and local choices of things like bump functions, unless specifically noted.
\end{remark}
