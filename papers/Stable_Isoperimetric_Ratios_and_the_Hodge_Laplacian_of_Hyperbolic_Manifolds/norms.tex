
\section{Norm estimates}
\label{sec:3}

In this section, we use deeply embedded triangulations and the Whitney maps described in the previous section to compare various geometric and combinatorial norms on forms and cochains. Throughout, let be $M$ a closed hyperbolic manifold of dimension $n>2$ with injectivity radius bounded below by $\e>0$ and a fixed deeply embedded triangulation $K$. Let $\beta$ be the smooth barycentric partition of unity associated to $K$.

We require various comparisons of the following norms on cochain and chain complexes associated to $M$ and $K$. The relevant norms are:


\begin{enumerate}
    \item The combinatorial Gromov norm $||\cdot||_G$ on any chain or cochain complex given by $||\sum a_i\sigma_i||_G = \sum|a_i|$ and $||\sum a_i\delta_{\sigma_i}||_G = \sum|a_i|$.
    \item The combinatorial Euclidean norm $||\cdot||_E$, which is the usual $\ell^2$ norm on chains and cochains given by $ ||\sum a_i\sigma_i||_E = \sqrt{\sum|a_i|^2}$ and  $ ||\sum a_i\delta_{\sigma_i}||_E = \sqrt{\sum|a_i|^2}$.
     \item The combinatorial max norm $||\cdot||_{\max}$ on any chain or cochain complex given by $||\sum a_i\sigma_i||_{\max} = \max|a_i|$.

    \item The Whitney induced $L^2$-norm $||~\cdot~||_2$ on the cochain complex $C^{\bullet}(K)$, given by $||f||_2 = \sqrt{\int_MW_{\beta}(f)\wedge\star W_{\beta}(f)}$.

    \item The Whitney induced $L^{\infty}$-norm $||\cdot||_{\infty}$ on the cochain complex $C^{\bullet}(K)$, given by taking the essential supremum of the pointwise Riemannian metric operator norms $||f||_{\infty} = \esssup\limits_{p\in M} ||W_{\beta}(f)_p||_{\infty}.$
\end{enumerate}

Given a norm $||\cdot||$ on the cochain complex $C^{\bullet}(K)$, let $||\cdot||^*$ denote the dual norm on the linear dual chain complex $C_{\bullet}(K)$ induced by the integration pairing: $$||a||^* = \sup\limits_{\substack{||f||=1\\ f\in C^{\bullet}(K)}} \int_a W_{\beta}(f).$$

\begin{prop} \label{prop: 3.1} There is a constant $B =  B(\e)>0$ such that the norms $||\cdot||_G$ and $||\cdot||_2$ on $C^{\bullet}(K)$ satisfy $$||\cdot||_G\leq B\sqrt{\vol(M)}||\cdot||_2,$$ and the norms $||\cdot||_G$ and $||\cdot||_2^*$ on $C_{\bullet}(K)$ satisfy  $$||\cdot||_G\leq B \sqrt{\vol(M)}||\cdot||_2^{*}.$$
\end{prop}
\begin{proof}

 Let $f = \sum\limits_Fa_F\delta_F$ be a cochain.
 Then for any $n$-simplex $\sigma$, $f|_{\sigma} = \sum\limits_{F\subset \sigma}a_F\delta_F$ and $$||f||^2_2 = \sum\limits_{\sigma \in K^{(n)}} || W_{\beta}(f|_{\sigma})|_{\sigma}||_2^2 =\sum\limits_{\sigma \in K^{(n)}} ||W_{\beta}(f)|_{\sigma}||_2^2 .$$
Apply Proposition \ref{prop: smooth comparison} to obtain $D = \mathcal B(\e,||\cdot||_G)$. This gives $||f|_{\sigma}||_G\leq D ||f|_{\sigma}||_{2,\sigma},$ where $||\cdot||_{2,\sigma}$ is the $L^2$-norm on the simplex $\sigma$ associated to the smooth barycentric coordinate $\beta$.
Then, by applying the Euclidean $\ell^1$-$\ell^2$-comparison to the cochain complex and using the fact there is a constant $T$ such that the number of $n$-simplices in $K$ is less than $T\vol(M)$, we find
\begin{align*}
||f||_G  &\leq \sum\limits_{\sigma \in K^{(n)}} ||f|_{\sigma}||_G\\
		&\leq \sum\limits_{\sigma \in K^{(n)}}  D||f|_{\sigma}||_{2,\sigma}\\
	   &\leq D\sqrt{T\vol(M)\sum\limits_{\sigma \in K^{(n)}}||W_{\beta}(f)|_{\sigma}||_2^2}\\
	   &\leq D\sqrt{T\vol(M)}||f||_2.
\end{align*}
For the second inequality, notice $||\cdot||_G$ is the usual $\ell^1$-norm on a finite dimensional vector space, so its dual norm is the max norm $||\cdot||_{\max}$.
\color{black}

Apply Proposition \ref{prop: smooth comparison} and set $ D’ =\mathcal B(\e, ||\cdot||_{\max}),$ so that if $\sigma$ is the simplex in which $||f||_{\infty}$ is realized, then $$||f||_{\infty} = ||f|_{\sigma}||_{\infty}\leq  D’||f|_{\sigma}||_{\max}\leq D’||f||_{\max}.$$
Then, $$||f||_2 \leq\sqrt{\vol(M)} ||f||_{\infty} \leq D’ \sqrt{\vol(M)}||f||_{\max}.$$
Dualizing gives $$  ||\cdot||_G = ||\cdot||_{\max}^*\leq  D’ \sqrt{\vol(M)}||\cdot||_2^*,$$
since
\begin{align*}
	||a||_G &= ||a||_{\max}^*\\
		   &= \sup\limits_{||f||_{\max} \leq 1}\int_a W_{\beta}(f)\\
		   &=\sup\limits_{|| D’\sqrt{\vol(M)}f||_{\max} \leq 1}\int_{a} D’\sqrt{\vol(M)} W_{\beta} (f)\\
		   &\leq \sup\limits_{||f||_2\leq1}\int_{a} D’\sqrt{\vol(M)} W_{\beta} (f)\\
		   &=  D’\sqrt{\vol(M)} \sup\limits_{||f||_2\leq 1} \int_a W_{\beta} (f)\\
		   &= D’\sqrt{\vol(M)}||a||_2^*.
\end{align*}
Set $B = \max\{D\sqrt{T}, D’\}$ to obtain the claim.
 \end{proof}

Recall from Section 2 that there is a polyhedral celluation $K^*$ dual to $K$ that can be canonically subdivided into a triangulation $\tau(K)$. Equipping these dual complexes with the Gromov norm, we have the following two propositions relating these norms by the Poincar\'e duality and subdivision maps.

\begin{prop} \label{prop: 3.2} 
There is a constant $D = D(\e)$ such that for any $\bullet$-cochain $f\in C^{\bullet}(K)$ one has $||f||_2 \leq D||f||_G$.
\end{prop}

\begin{proof} Let $f = \sum a_i\delta_{\sigma_i}$ be a $\bullet$-cochain. Then $||\omega||_G = \sum|a_i|$ and $||f||_2 \leq \sum |a_i| ||\delta_{\sigma_i}||_2.$ Then, for any fixed $\bullet$-simplex $\sigma$ that is a face of an $n$-simplex from $\mathcal G_e$, using the $L^2$-change of variables formula and Proposition \ref{prop: smooth L2 comparison} we can take $D = A\mathcal L^{n/2}||W(\delta_{\sigma})||_2$, so that $||\delta_{\sigma_i}||_2 \leq D.$

The comparison \[||f||_2\leq\sum |a_i| ||\delta_{\sigma_i}||_2 \leq D\sum |a_i| = D||f||_G \] then follows.
\end{proof}


\begin{prop} \label{prop: 3.3} 
The Poincar\'e duality map $\Phi:C^{\bullet}(K)\to C_{n-\bullet}(K^*)$ preserves the Gromov norm $$||f||_G= ||\Phi(f)||_G.$$
\end{prop}

\begin{proof}
  Let $f = \sum a_i\delta_{\sigma_i}$, then $\Phi(f) = \sum a_i (\sigma_i)^*$.
\end{proof}

\begin{prop} \label{prop: 3.4} 
Let $N$ be the constant from Proposition \ref{prop: star bound}, which bounds the number of simplices in the star of a simplex in a deeply embedded triangulation. Then the subdivision map $\tau: C_{2}(K^*)\to C_{2}(T)$ satisfies $$ || \tau(c)||_G \leq N ||c||_G.$$
\end{prop}

\begin{proof}
  The number of sides of a $2$-cell in $K^*$ dual to a $(n-2)$-simplex $\sigma$ in $K$ corresponds to the number of $n$-simplices in $K$ that contain $\sigma$. The number of such simplices is bounded by $N$.
\end{proof}

The following estimates are essential in comparing the first eigenvalue of the Whitney Laplacian to the genuine first eigenvalue. For this, we need to work with various Sobolev spaces to control the orthogonal projection of a Whitney form onto its coexact part. This discussion is the reason we use the smoothed Whitney forms in place of the standard ones.

We will require the following version of the Gaffney inequality, which follows from Lemma 2.4.10 in \cite{schwarz}. To simplify the following discussion, for a smooth manifold $Y$ possibly with boundary, we introduce an alternative Sobolev norm on $H^{k+1}_{\nabla}\Omega^{\bullet}(Y)$: $$||\omega||_{A^{k}(Y)}:= ||\omega||_{H^k_{\nabla}(Y)} + ||d\omega||_{H^k_{\nabla}(Y)} + ||d^*\omega||_{H^k_{\nabla}(Y)}.$$

Since $d$ and $d^*$ are bounded operators $H^{k+1}\Omega^{\bullet}_{\nabla}(Y)\to H^{k}\Omega^{\bullet\pm 1}_{\nabla}(Y)$, we immediately have that there is a constant $C$ such that $||\omega||_{A^{k}(Y)}\leq  C ||\omega||_{H^{k+1}_{\nabla}(Y)}.$

Recall that the marking $\mathring H$ denotes the subspace of given Sobolev space that is the closure of smooth functions supported in the interior.

\begin{lem}  \label{lem: 3.5} (Gaffney inequality) Let $Y$ be a smooth manifold with boundary. Let $\omega\in \mathring{H}^k_{\nabla} \Omega^{1}(Y)$. Then there is a constant $C = C(Y)>0$ such that $||\omega||_{\mathring{H}^k_{\nabla}(Y)}\leq C ||\omega||_{A^{k-1}(Y)}$.
\end{lem}

\begin{lem} \label{lem: 3.6} 
Let $B_0 = B_r(p)$ and $B_1= B_{r+\delta}(p)$ be a pair of concentric balls in $\H^n$ and let $\phi$ be a smooth bump function that is identically 1 on $B_0$ and vanishes in a neighborhood of $\d B_1$. There is a constant $C = C(\phi,k)$ that depends only on the norm of the covariant derivatives of $\phi$ up to order $k+1$ such that if $\omega \in \Omega^{1}(\H^n)$, then $$||\phi\omega||_{A^k(B_1)}\leq C ||\omega||_{A^k(B_1)}.$$
\end{lem}

\begin{proof}
Notice that $d\phi\omega = d\phi\wedge\omega + \phi d \omega$ and $d^*\phi\omega = \phi d^*\omega + g(\nabla \phi, X_{\omega}),$ where $X_{\omega}$ is the vector field dual to $\omega$.
As a result, the triangle inequality yields \begin{align*}
||\phi\omega||_{A^k(B_1)} &\leq ||\phi d\omega||_{H^k_{\nabla}(B_1)} + ||\phi d^*\omega||_{H^k_{\nabla}(B_1)}+ ||\phi\omega||_{H^k_{\nabla}(B_1)} \\ &+ ||d\phi\wedge\omega||_{H^{k}_{\nabla}(B_1)} + ||g(X_{\omega},\nabla\phi)||_{H^k_{\nabla}(B_1)}.
\end{align*}

The estimate $||\alpha\wedge\beta||_2\leq ||\alpha||_{\infty}||\beta||_2$ implies $$||d\phi\wedge\omega||_{H^k_{\nabla}(B_1)} \leq C||\omega||_{H^k_{\nabla}(B_1)},$$ where the constant $C$ is given by the sum of the $||\cdot||_{\infty}$-norms of the covariant derivatives of the bump function $\phi$.
The same argument gives for any form $\xi$ that $||\phi\wedge\xi||_{H^k_{\nabla}(B_1)}\leq C ||\xi||_{H^k_{\nabla}(B_1)}$. Applying this estimate to $\phi d\omega, \phi d^*\omega$, and $\phi \omega$ handles all terms in the comparison save for $||g(X_{\omega},\nabla\phi)||_{H^{k}_{\nabla}(B_1)}$.
For this term, notice that $\nabla g(X_{\omega},\nabla\phi) = g(\nabla X_{\omega},\nabla \phi) + g( X_{\omega}, \nabla^2\phi).$ We therefore have the pointwise estimate

\begin{align*}
|\nabla g(X_{\omega},\nabla\phi) | &\leq |g(\nabla X_{\omega},\nabla \phi)| + |g( X_{\omega}, \nabla^2\phi)| \text{ by the triangle inequality,} \\
&\leq |\nabla \omega|^2|\nabla\phi|^2 + |\omega|^2|\nabla^2\phi|^2,
\end{align*}

by applying Cauchy-Schwarz and the musical isomorphism.
Integrating then gives the corresponding inequality for the Sobolev norm $H^1_{\nabla}$. Repeating this calculation for higher order covariant derivatives completes the proof.
\end{proof}

The two previous lemmas combine to give the following statement.
\begin{prop} \label{prop: 3.7} 
Let $B_0 = B_r(p)$ and $B_1= B_{r+\delta}(p)$ be a pair of concentric balls in $\H^n$ Then there is a constant $C = C(r,\delta)$ such that for any $\omega \in H^k_{\nabla}(B_1)$ one has $$||\omega||_{H^k_{\nabla}(B_0)}\leq C||\omega||_{A^{k-1}(B_1)}.$$
\end{prop}
\begin{proof}
Let $\sqrt{C}$ be the maximum of the constants from Gaffney’s inequality and Lemma 3.6 with bump function $\phi$. Then for a form $\omega \in H^k_{\nabla}(B_1)$, one has $$||\omega||_{H^k_{\nabla}(B_0)}\leq ||\phi\omega||_{H^k_{\nabla}(B_1)},$$ since $\phi\omega|_{B_0} = \omega.$
Since $\phi\omega$ vanishes on $\d B_1$, Gaffney’s inequality and Lemma 3.6 give $$||\phi\omega||_{H^k_{\nabla}(B_1)} = ||\phi\omega||_{\mathring{H}^k_{\nabla}(B_1)}\leq \sqrt{C} ||\phi\omega||_{A^{k-1}(B_1)}\leq C||\omega||_{A^{k-1}(B_1)}.$$ Combining these two estimates gives the proposition.
\end{proof}

\begin{prop}\label{prop: 3.8} 
Let $M$ be a hyperbolic $n$-manifold with injectivity radius greater than $\e>0$. There is a constant $H(\e) = H >0$ depending only on $\e$ such that if the $L^2$-Hodge decomposition of a smooth 1-form $\omega\in \Omega^1$ has coexact part $\alpha$, then for any point $p\in M$ there is a ball $B\subset M$ centered at $p$ of radius determined by $\e$ such that for $k\geq n$ one has $$|
\nabla \alpha(p)|\leq H\left(||\omega||_{H^k_{\nabla}(B)} + ||\alpha||_{2,B}\right)$$ and consequently $$|
\nabla \alpha(p)|\leq H||\omega||_{H^k_{\nabla}(M)}.$$
\end{prop}

\begin{proof}

The $L^2$-Hodge decomposition of $M$ determines an orthogonal decomposition of $\omega= \alpha + \eta + h$, where $\alpha = d^*A$, $\eta = db$, and $h$ harmonic. Cantor’s estimate (Theorem \ref{thm: Cantor Sobolev} implies at any point $p$ of $M$, $$|\nabla \alpha (p)| \leq C(r)||\alpha||_{H^k_{\nabla}(B_r(p))}$$ so long as $r<\inj(M)$ and $k > n/2 + 1$. Since we assume dimension $n>2$, any $k\geq n$ suffices.
Take a concentric family of balls $B_i = B_{r + i\delta}(p)$ where $r = 6\e$, and $k\delta + r < 10\e < \inj(M)$; note that $B_0$ contains the 0-star of $\sigma$. Let $\phi_i$ be a bump function that is identically one on $B_i$ and vanishes on $\d B_{i+1}$.

Letting $C_i$ be the maximum of the constant from Cantor’s estimate on the ball $B_i$ and the constant from Proposition \ref{prop: 3.7} for the balls $B_i\subset B_{i+1}$, we have
$$|\nabla \alpha (p)| \leq C_i||\alpha||_{H^k_{\nabla}(B_i)}
   \leq  C_i^2||\alpha||_{A^{k-1}(B_{i+1})}.$$
Since $\alpha$ is coexact, $$||\alpha||_{A^{k-1}(B_{i+1})} = ||\alpha||_{H^{k-1}_{\nabla}(B_{i+1})} + ||d\alpha||_{H^{k-1}_{\nabla}(B_{i+1})}.$$ Notice that $d\alpha = d\omega$, and that $d:H^k_{\nabla}(B_i)\to H^{k-1}_{\nabla}(B_i)$ is a bounded operator, say with operator norm $C$. Thus, we have $$||d\alpha||_{H^{k-1}_{\nabla}(B_{i+1})} \leq C||\omega||_{H^k_{\nabla}}.$$ As a result, we get the estimate $$||\alpha||_{A^{k-1}(B_{i+1})} \leq ||\alpha||_{H^{k-1}_{\nabla}(B_{i+1})} + C||\omega||_{H^{k}_{\nabla}(B_{i+1})}$$
Combining this with the estimate of $\nabla \alpha$ above gives $$|\nabla \alpha(p)|\leq C_i\left(||\omega||_{H^{k}_{\nabla}(B_{i+1})}+||\alpha||_{H^{k-1}_{\nabla}(B_{i+1})}\right).$$
We can repeat argument this using the family of balls $B_i$ to reduce the order of the Sobolev norm of $\alpha$ on the right-hand side until we obtain $$|\nabla \alpha(p)|\leq H\left(||\omega||_{H^k_{\nabla}(B_n)} + ||\alpha||_{2,B_n}\right),$$ where $H$ is obtained by combining all the constants appearing in the iterated calculation. Orthogonality of the Hodge decomposition implies $||\alpha||_2\leq ||\omega||_2$, and we clearly have $||\omega||_{2, B_n}\leq ||\omega||_{H^k_{\nabla}(M)},$ so we are done after increasing $H$ by 1.
\end{proof}

When the form in the previous proposition is a Whitney form, we can make the following refinement.

\begin{prop}\label{prop: 3.9} 
Let $M$ be a hyperbolic $n$-dimensional manifold with a deeply embedded triangulation $K$ and let $f\in C^1(K)$. There is a constant $H(\e) = H >0$ depending only on $\e$ such that if the $L^2$-Hodge decomposition of the generalized Whitney form $W_{\beta}(f)$ has coexact part $\alpha$,  then  $$
||\nabla \alpha||_{\infty}\leq H||W_{\beta}(f)||_2.$$
\end{prop}
\begin{proof}
Let $\omega = W_{\beta}(f)$ be a smooth Whitney form. We will apply Proposition \ref{prop: 3.8} at a point $p$ at which $||\nabla\alpha||_{\infty}$ is realized. Proposition \ref{prop: 3.8} gives a ball $B$ about $p$ of radius depending only on $\e$ such that $$|
\nabla \alpha(p)|\leq H\left(||\omega||_{H^k_{\nabla}(B)} + ||\alpha||_{2,B}\right),$$ for a constant $H$ depending only on $\e$.
The ball $B$ intersects some uniformly bounded collection of $n$-simplices $\sigma’$ from $K$ where the constant depends only on $\e$; let $T = T(\e)$ be this bound. We can therefore estimate the norm of the Whitney form term by $$||\omega||_{H^k_{\nabla}(B)}\leq \sum\limits_{\sigma’\cap B \neq \emptyset} ||\omega|_{\sigma’}||_{H^k_{\nabla}(\sigma’)}.$$
Applying Lemma 2.13 to each summand in the previous estimate then gives
$$ ||\omega||_{H^k_{\nabla}(B)} \leq \sum\limits_{\sigma’\cap B_i \neq \emptyset} ||\omega|_{\sigma’}||_{H^k_{\nabla}(\sigma’)}\leq R\sum\limits_{\sigma’\cap B \neq \emptyset} ||\omega|_{\sigma’}||_2\leq R\sqrt{T}||\omega||_2.$$
Combining the above then gives that $$|\nabla \alpha(p)|\leq HR\sqrt{T}||\omega||_2 + H||\alpha||_{2,B}.$$ Clearly $$H||\alpha||_{2,B}\leq H||\alpha||_2 \leq H||\omega||_2,$$ so that after increasing $H$ to absorb the $R\sqrt{T}$ term we are done.
\end{proof}

\begin{prop}\label{prop: 3.10} 
Let $M$ be a hyperbolic $n$-manifold. Let $\omega\in \Omega^1(M).$ Assume there exists a constant $H$ such that $|\nabla \omega| \leq H$. Then there is a constant $C(H,\e)$ such that $||\omega||_{\infty}\leq C(H,\e) ||\omega||_2.$
\end{prop}

\begin{proof} Assume $||\omega||_{\infty} = 1$ and is realized at the point $p$. By Kato’s inequality and the hypothesis, away from the zeros of $\omega$ one has $|\nabla|\omega||\leq |\nabla \omega| \leq H$. Fix a normal coordinate frame $x_0,\dots,x_{n-1}$ at $p$ of radius $2\e$. Define the function $\phi$ on this normal coordinate neighborhood by $\phi(x) = 1- Hd(x,p)$ for $ d(x,p) < 1/H$ and extend by zero.
Then $||\phi||_{\infty} = ||\omega||_{\infty}$ and $||\phi||_2 \leq ||\omega||_2$.
The claim then holds for $C(H,\e) = 1/||\phi||_2$ by scaling the unit norm case.
\end{proof}
\begin{prop}\label{prop: 3.11} 
There is a constant $C = C(\e)$ such that if $f \in C^1(K)$ and $\omega = W_{\beta}(f)  = \alpha + \eta$ where $\alpha$ is $L^2$-coexact and $\eta$ is closed, then $$||\alpha||_{\infty} \leq C ||\alpha||_2.$$
\end{prop}

\begin{proof}
Assume that $||f||_2 = 1$. By Proposition \ref{prop: 3.9}, $||\nabla \alpha ||_{\infty} \leq H||f||_2 = H$.  Proposition \ref{prop: 3.10} gives a constant $C = C(H(\e))$ (so this really just depends on $\e$) such that $||\alpha||_{\infty}\leq C||\alpha||_2$. If $f$ does not have unit $L^2$-norm, then either $f=0$, in which case the result is trivial, or else $f = \lambda f’$ for some unit $L^2$-norm cochain $f’$ and positive number $\lambda$.
The coexact part of $W_{\beta}(f’)$ is $\alpha ‘ = \alpha/\lambda$. Hence, $||\alpha’||_{\infty}\leq C||\alpha’||_2$, and the result follows. \end{proof}
