
\section{The upper bound}
\label{sec:4}
In this section we prove Theorem A, which states that in a closed hyperbolic 3-manifold $M$ the first positive eigenvalue of the Hodge Laplacian acting on coexact 1-forms is bounded above by a multiple of the stable isoperimetric ratio $\rho(M)$. The background results of this section all hold in any dimension greater than 2, however the proof of Theorem A makes use of Poincar\'e duality to relate 1-forms and surfaces, this forces us to restrict Theorem A to the 3-dimensional setting.


The cochain results of the previous section are connected to spectral geometry via the inner product induced by the Whitney map associated to a triangulation and barycentric coordinate: $$\langle f, g\rangle = \int_M W_{\beta}(f)\wedge\star W_{\beta}(g),$$ which along with the corresponding norm $||\cdot||_2$, determine a Hodge theory for the cochain complex $C^{\bullet}(K)$. This inner product determines a codifferential $$d^*_W:C^{\bullet}(K)\to C^{\bullet-1}(K)$$ which, as the adjoint of the standard differential, satisfies $\langle d f, g\rangle = \langle f , d_W^* g \rangle.$
The corresponding Whitney Laplacian $\Delta_W:C^{\bullet}(K)\to C^{\bullet}(K)$ is then given by the standard formula $\Delta_W = dd_W^*+d_W^*d.$ This inner product was introduced using the standard barycentric coordinates in \cite{Dodziuk}.

This Laplacian decomposes the space $C^{\bullet}(K)$ into harmonic, exact, and coexact components: $$C^{\bullet}(K)\cong H^{\bullet}(M) \oplus dC^{\bullet-1}(K) \oplus d_W^* C^{\bullet+1}(K).$$ This combinatorial Hodge decomposition serves as a good approximation of the $L^2$-Hodge decomposition of $M$, though it does not capture the $L^2$-Hodge decomposition exactly. In particular, the Whitney coexact chains may not be $L^2$-coexact.

We begin by relating the Whitney and the Riemannian coexact eigenvalues.
\begin{lem} \label{lem:4.1}
Let $M$ be a closed Riemannian $n$-manifold with triangulation $K$ and an associated barycentric partition of unity $\beta$. Give the cochain complex the Whitney $L^2$-norm induced by the Whitney map determined by $\beta.$ Likewise, give the chain complex the dual norm $||\cdot||_2^*$ determined by the integration pairing. Then for every coexact cochain $f\in d_W^*C^2(K)$, there is an exact chain $a\in \d C_2(K)$ of unit norm such that $||f||_2 = \int_aW_{\beta}(f).$
\end{lem}
\begin{proof}

The cochain Hodge decomposition from the Whitney inner product gives the orthogonal decomposition $$C^1(K) = H^1(M) \oplus d_W^*C^2(K) \oplus dC_0(K).$$ Let $Z^1(K) = H^1(M)\oplus d C_0(K)$. Identify $C_1(K)$ with $C^1(K)^*$ via the integration pairing. The composition of the inclusion and quotient map determines an isomorphism $d_W^*C^2(K) \to C^1(K)/Z^1(K) $ that allows us to identify these spaces. If $\text{Ann}$ assigns to a subspace its annihalator, then there is also an isomorphism $(C^1(K)/Z^1(K))^*\to \text{Ann}(Z^1(K)).$ By Stokes’ theorem and dimension counting, $\text{Ann}(Z^1(K)) = \d C_2(K)$.
Thus, the dual of $d_W^*C^2(K)$ is exactly $\d C_2(K)$. The dual norm of an element $a \in \d C_2(K)$ is given by $$||a||_2^* = \sup\limits_{\substack{f\in C^1(K)\\||f||_2\leq 1}}\int_a W_{\beta}(f).$$
If $f$ has unit $L^2$-norm and $f = g + h$ where $g\in d_W^*C^2(K)$ and $h\in Z^1(K)$, then orthogonality implies $||g||_2\leq 1.$ Whence,
$$||a||_2^* = \sup\limits_{\substack{f = g + h\in C^1(K)\\||f||_2\leq 1}}\int_a W_{\beta}(g) = \sup\limits_{\substack{g\in d^*_WC^2(K)\\||g||_2\leq 1}}\int_a W_{\beta}(g).$$ The isometric identification of $(d^*_WC^2(K),||\cdot||_2)$ with its double dual therefore implies we can compute the norm of an element $f\in d^*_WC^2(K)$ via the integration pairing integrating only against chains in $\d C_2(K)$:
$$||f||_2= \sup\limits_{\substack{a \in \d C_2(K)\\||a||_2^* = 1}}\int_aW_{\beta}(f).$$ In particular, for any coexact cochain $f\in d^*_WC^1(K)$, there exists an exact chain $a$ with $||a||^*_2=1$ and $$\int_aW_{\beta}(f) = ||f||_2.$$
\end{proof}



\begin{prop} \label{prop:4.2}
Let $\lambda$ denote the first eigenvalue for the Hodge Laplacian acting on coexact $1$-forms and let $\lambda_W$ denote the first eigenvalue for the Whitney Laplacian acting on coexact 1-cochains associated to a deeply embedded triangulation $K$ and barycentric partition of unity $\beta$. There is a constant $G = G(\e)$ such that $$\lambda \leq G \vol(M)\lambda_W.$$
\end{prop}
\begin{proof}

The main issue here is that a Whitney coexact cochain will not generally map to an $L^2$-coexact form. This potentially adds a closed term to the denominator in the Whitney Rayleigh quotient, causing the Whitney Rayleigh quotient to be smaller than the Riemannian Rayleigh quotient. However, this failure can be controlled.

Let $f$ be a coexact eigen-cochain with eigenvalue $\lambda_W$. Set $\omega = W_{\beta}(f)\in\Omega^1(M)$, so that $\frac{||d\omega||_2^2}{||\omega||_2^2} = \lambda_W$. Let $p:\Omega^1(M)\to \Omega^1(M)$ be the $L^2$-orthogonal projection onto coexact forms. Let $a\in C_1(K)$ be the unit norm exact chain that realizes the norm of $f$ by integration given by Lemma 4.1. Then using that $d\omega = d(p(\omega))$ and the fact $a$ is exact, we obtain $$||f||_2 = ||\omega||_2 = \int_{a}\omega = \int_{a}p(\omega).$$

Using 3.11, we have $||p(\omega)||_{\infty}\leq C||f||_2.$ Hence, $||f||_2\leq C\text{len}(a)||p(\omega)||_2.$ We can therefore obtain a lower bound on $||p(\omega)||_2$ by bounding $\text{len}(a)$ from above. Applying Proposition \ref{prop: 3.1} to the chain $a$ above gives $$\mathcal ||a||_G\leq B \sqrt{\vol(M)} ||a||_2^* =  B \sqrt{\vol(M)}.$$ Since the lengths of the edges in the triangulation are bounded, we conclude the length of the support of $a$ is bounded.
Take $E$ to be the length of the largest edge possible in a deeply embedded triangulation, so that $\text{len}(a)\leq BE\sqrt{\vol(M)}$
Then we have obtained $$||\omega||_2 = \int_{a} \omega = \int_{a}p(\omega) \leq ||p(\omega)||_2  BCE\sqrt{\vol(M)}.$$ Setting $G = ( BCE)^2$ and using that $\omega$ is a Whitney eigenform, we obtain the result by the following short computation:
\[
 \lambda \leq \frac{||d\omega||_2^2}{||p(\omega)||_2^2}
 \leq G\vol(M)\frac{||d\omega||_2^2}{||\omega||_2^2}
 = G\vol(M)\lambda_W.
\]
\end{proof}

\begin{remark}
Note that the above estimate in fact holds for the first positive eigenvalue since the first positive eigenvalue $\lambda$ is the minimium of the first  eigenvalue of the Laplacian acting on functions and the first eigenvalue of the Laplacian acting on coexact 1-forms. The first eigenvalue $\lambda_f$ for the Laplacian acting on functions automatically satisfies the comparison $\lambda_f \leq \lambda_W$, as can be seen by studying the Rayleigh quotient and noticing that the estimate above controlling the projection in the denominator is immaterial in the function case.
\end{remark}


We are now ready to introduce stable commutator length, a thorough reference for which is \cite{Calegari}. For a group  $\G$, let  $\Gamma’$ denote the commutator subgroup and define the rational commutator subgroup to be $$\G_{\Q}’ = \text{Ker}(\Gamma\to \Gamma^{ab}\otimes \Q).$$ Note that when $\G$ is the fundamental group of a manifold, these subgroups correspond to the integrally nullhomologous and rationally nullhomologous loops respectively. The commutator length of an element $\gamma\in \Gamma’$, denoted ${\tt{cl}}(\gamma)$ is the shortest word length of $\gamma$ with respect to the generating set of all commutators.
The stable commutator length for $\gamma\in \G’_{\Q}$ is then defined to be \[\scl(\gamma) = \inf\limits_{m\geq 1}\frac{{\tt{cl}}(\gamma^m)}{m}.\] Topologically, stable commutator length corresponds to the stable complexity of a surface bounding a nullhomologous curve. In particular, for $\gamma\in \Gamma’_{\Q}$, one has $$\scl(\gamma) = \inf\left\{\frac{\chi_-(S)}{2m}~:~S \text{ with }\d S = \gamma^m \text{ and $S$ with no closed components}\right\},$$
where for a connected surface $S$ we define $\chi_-(S) = \max\{0,-\chi(S)\}$, and extend this additively to disconnected surfaces.
There is another natural complexity measure for loops in $\G’_\Q$, the Gersten filling norm. For a loop $\gamma\in \G’_\Q$, $\fill(\gamma)$ is the infimum of the Gromov norm $\frac{||A||_G}{m}$ for all singular 2-chains $A$ bounding a 1-cycle representing a singular fundamental class of $\gamma^m$. A fundamental theorem of Bavard relates the filling norm to the stable commutator length.

\begin{thm} \label{thm:4.3} (\cite{Bavard}) For any group element $\gamma$, there is an equality: \[\scl(\gamma) = 4\fill(\gamma).\]
\end{thm}
For proof, see for instance Lemma 2.69 in \cite{Calegari}.

\begin{remark} Let $B(\Gamma)$ be the $\R$-vector space of 1-boundaries. Then stable commutator length can be extended to a psuedo-norm on $B(\Gamma)$. After identifying chains with vanishing psuedo-norm, Bavard duality, which relates the filling norm to quasimorphisms and their defect norm, becomes a genuine functional analytic duality theorem. One could define the stable isoperimetric ratio in this chain setting, and the results of this paper would go through for that (smaller) ratio as well.
\end{remark}

We can now prove the main theorem of this section.

\begin{mainthm}\label{thm:A} Let $M$ be a closed hyperbolic 3-manifold  with injectivity radius bound below by $\e$.  There is a constant $A = A(\e)$ that only depends on $\e$ such that for any nontrivial boundary $\gamma \in \Gamma’_{\Q}$, one has
$$\sqrt{\lambda} \leq A \vol(M)\frac{|\gamma|}{\scl(\gamma)},$$
 where $\lambda$ is the first coexact eigenvalue of the Hodge Laplacian on $\Omega^1(M)$.
\end{mainthm}


\begin{proof}
First note that since stable commutator length and geodesic length are both multiplicative under powers, it suffices to show the claim for an integrally nullhomologous loop $\gamma$.

Fix a deeply embedded triangulation $K$ of $M$ and denote by $\lambda_W$ the first eigenvalue of the Whitney Laplacian $\Delta_W$ acting on $d^*_WC^2(K)$ associated to a smooth barycentric partition of unity. Notice that the Hodge decomposition ensures that zero is not an eigenvalue of this operator. Let $c:S^1\to M$ be a cellular path in the 1-skeleton of $K^*$ representing the loop $\gamma$, constructed as in Proposition  \ref{prop: length comparison}.
Let $T$ be a triangulation of $K^*$. Let $a\in C_1(K^*)$ be the fundamental cycle for $\gamma$ corresponding to the path $c$ in $C_1(K^*)\subset C_1(T)\subset C_1^{\text{sing}}(M)$.
If $\Phi:C^2(K)\to C_1(K^*)$ is the Poincar\'e duality map, then $\Phi^{-1}(a)$ is an exact 2-cochain. We can therefore choose $\omega\in d^*_WC^2(K)$ with $d\omega = \Phi^{-1}(a)$. Setting $A = \Phi(\omega)$ in $C_2(K^*)$, we have $\d A = a$ and $||A||_G = ||\omega||_G.$
Since $\lambda_W$ is nonzero, we have $||\omega||_2 \leq \frac{||d\omega||_2}{\sqrt{\lambda_W}}$.  Proposition \ref{prop:4.2} implies  \[||\omega||_2\leq \frac{\sqrt G\sqrt{\vol(M)} ||d\omega||_2}{\sqrt{\lambda}}.\] By Bavard’s theorem relating the filling norm to stable commutator length (Theorem 4.3), our choice of $A$, and Proposition \ref{prop: 3.4} we find that
$$\scl(\gamma) = 4\fill(\gamma) \leq 4||\tau(A)||_G \leq 4N||A||_G = 4N||\omega||_G,$$ where, as in Proposition \ref{prop: 3.4}, $\tau$ is the triangulation map relating the cellular chain $A$ to the subdivided simplicial chain in $C_2(T)$. Consequently,
\begin{align*}
                  \scl(\gamma) &\leq 4N ||\omega||_G\\
                 &\leq 4NB\sqrt{\vol(M)}||\omega||_2 ~\text{by Proposition \ref{prop: 3.1},}\\
                 &\leq  4NB\sqrt G \vol(M)\frac{||d\omega||_2}{\sqrt{\lambda}} ~\text{by above computation,}\\
                 &\leq 4NB\sqrt G \vol(M)\frac{D||d\omega||_G}{\sqrt{\lambda}}~\text{by Proposition  \ref{prop: 3.2},} \\
                 &=4NB\sqrt G \vol(M)\frac{D||\d  A||_G}{\sqrt{\lambda}}~\text{by Proposition  \ref{prop: 3.3},} \\
                 &=  4NB\sqrt G \vol(M)\frac{D||c||_G}{\sqrt{\lambda}} \text{ by construction of $\d A$,} \\
                 &\leq 4NB\sqrt G \vol(M)\frac{DL|\gamma|}{\sqrt{\lambda}}~\text{by Proposition  \ref{prop: length comparison},} \\
                 &= 4NB\sqrt G DL\vol(M)\frac{|\gamma|}{\sqrt{\lambda}}.
\end{align*}
Setting $A=4NB\sqrt G DL$ and rearranging, we are done.
\end{proof}
