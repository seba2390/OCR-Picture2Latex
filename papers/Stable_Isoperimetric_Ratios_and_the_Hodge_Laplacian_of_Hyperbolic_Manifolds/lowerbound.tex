

\section{The lower bound}
\label{sec:5}

We now turn to proving the lower bound on the first coexact eigenvalue of the 1-form Laplacian that constitutes Theorem B. Unlike Theorem A, we prove this eigenvalue comparison without a dimension constraint. The line of proof follows that of Theorem 1.3 in \cite{LS}.

In \cite{LS}, the authors obtain the following estimate controlling the $L^2$-norm of coclosed forms.  Note that this estimate does not depend on the fundamental domain coming from a deeply embedded triangulation.

\begin{prop} \label{prop:5.1} (Proposition 5.4 in \cite{LS})
Let $\eta$ be a 1-form on $M$ and $\mathcal D\subset \H^n$ any fundamental domain. Then,
$$||\eta||_2^2 \leq \emph{Area}(\d \mathcal D)||\eta||_{\infty}\left(3\pi||d\eta||_{\infty} + \max_i\left|\int_{\gamma_i}\eta\right|\right) + \frac{1}{2}||d\eta||_{\infty}||\eta||_2\sqrt{\vol(M)},$$
where the $\gamma_i$ are the geodesics in the homotopy class of the loops representing the side pairing transformations of the fundamental domain $\mathcal D$.
\end{prop}


Studying the terms in the estimate of Proposition \ref{prop:5.1} for a coexact $\lambda$-eigenform provides a lower bound on $\lambda$ given later as Theorem B. The essential idea is that after applying an $L^2$-$L^{\infty}$ norm comparison, all but one summand on the right-hand side (the integral term), has a $||d\eta||_2$ term. In particular, if $\eta$ is a unit norm eigenform, the right-hand side almost has a $\sqrt{\lambda}$ term in every summand. Our aim, then, is to replace the integral term with something that looks like $||d\eta||_{\infty} (\rho(M)^{-1}+\text{stuff})$, where the stuff is polynomial in the volume of $M$ with constants that depend only on the lower bound on injectivity radius.

\begin{lem} \label{lem:5.2}Let $a$ be the lift to $\H^n$ of the cellular approximation of a geodesic loop and let $\gamma$ be the (oriented) lift of the same geodesic loop. If $\tilde \eta$ is the pullback to $\H^n$ of a 1-form $\eta\in \Omega^1(M)$, then $$\left|\int_{a}\tilde\eta - \int_{\gamma}\tilde \eta\right| \leq \pi||d\eta||_{\infty}||a||_G.$$ \end{lem}
\begin{proof}
Let $x$ be the starting point of $\gamma$, let $a_i$ be the geodesic arcs of $a$ (so that the word corresponding to the cellular path $a$ is the word $a_1\cdots a_{||a||_G}$), and let $y$ be the end point of $a_{||a||_G}$. Let $Q$ be the piecewise geodesic $(||a||_G+3)$-gon obtained by taking the union of the triangles $\text{convex hull}(a_i,x)$ and the triangle $\text{convex hull}(\gamma,y)$. Since $Q$ is the union of $(||a||_G + 1)$ geodesic triangles of area bounded by $\pi$, we have the upper bound of $(||a||_G+1)\pi$ for the area of $Q$.
Then, since  $$
\left|\int_a\tilde\eta - \int_{\gamma}\tilde\eta\right| = \left|\int_{\d Q} \tilde\eta\right|  \leq \pi||d\eta||_{\infty}(||a||_G+1),$$ where the first equality follows from the deck transformation invariance of $\tilde \eta$, which makes the integral over $\d Q\setminus(a\cup\gamma)$ vanish.
If one then considers the cellular
s $ma$ and the geodesic $\gamma^m$ integrated over the form $\frac{1}{m}\tilde\eta$, one gets
$$ \left|\int_a\tilde\eta - \int_{\gamma}\tilde\eta\right| = \frac{1}{m}\left|\int_{ma}\tilde\eta - \int_{\gamma^m}\tilde \eta\right|\leq \frac{1}{m}\pi||d\eta||_{\infty}(||ma||_G+1).$$
Taking the limit as $m\to\infty$ then results in  $$\left|\int_{a}\tilde\eta - \int_{\gamma} \tilde\eta\right|\leq\pi||d\eta||_{\infty}||a||_G,$$ proving the lemma.
\end{proof}

\begin{lem}\label{lem:5.3}
There is a constant $B_0 = B_0(\e)$ such that $\diam(M)\leq B_0\vol(M)$.
\end{lem}
\begin{proof}
	First note that there is a constant $T = T(\e)$ such that the number of simplices in $M$ is bounded by $T\vol(M)$ and that each simplex from a deeply embedded triangulation has bounded diameter, say bounded by $C_0$. With $B_0 = C_0T$, one has that $B_0\vol(M)$ bounds the diameter of $M$, as desired.
\end{proof}

\begin{lem}\label{lem:5.4}
Let $\eta\in\Omega^1(M)$ be a 1-form and $\gamma$ a rationally nullhomologous loop in $M$. Then integrating over the geodesic in the free homotopy class satisfies $$\left|\int_{\gamma}\eta \right|\leq 2 \pi||d\eta||_{\infty}\scl(\gamma).$$
\end{lem}
\begin{proof}

This follows from Bavard duality and the fact that $\int_{\gamma}\eta$, where the integral is over the geodesic in the free homotopy class of $\gamma$, is a quasimorphism with defect bounded by $\pi||d\eta||_{\infty}$ (see \cite{Calegari}, page 21).
\end{proof}

The key estimate allowing us to replace the integral term with one involving the stable isoperimetric constant $\rho(M)$ is the following proposition; compare with Proposition 5.24 in \cite{LS}.

\begin{prop} \label{prop:5.5} Let $\eta$ be a 1-form on $M$. Then there is a harmonic form $h$ and a constant $L_0 = L_0(\e)>0$ such that for every closed geodesic $\alpha$ in $M$, one has $$\left|\int_{\alpha}(\eta - h)\right|\leq |\alpha| L_0\vol(M)^{3/2} ||d\eta||_{\infty}\left(\rho(M)^{-1} + 1 \right).$$
\end{prop}
\begin{proof}

If $M$ is a $\Q$-homology sphere, $\alpha$ is rationally nullhomologous and $h$ can only be 0. Lemma \ref{lem:5.4} gives $$\left|\int_{\alpha}\eta\right|\leq 2\pi\scl(\alpha)||d\eta||_{\infty}.$$
By multiplying the right-hand side by $|\alpha|/|\alpha|$, this becomes $$ \left|\int_{\alpha}\eta\right|\leq 2\pi |\alpha|\frac{\scl(\alpha)}{|\alpha|}||d\eta||_{\infty}\leq 2\pi|\alpha|\rho(M)^{-1}||d\eta||_{\infty}.$$
If $W$ is the minimal volume hyperbolic $n$-manifold (in dimension 3, this is the Weeks manifold, see\cite{minvol}, more generally it is know that in dimension $n\geq4$ that the set of hyperbolic volumes is discrete in $\R$, see \cite{minvol2}), the claim follows with $L_0 = \frac{2\pi}{\vol(W)^{3/2}}$.

Thus, we assume $M$ has nontrivial real homology classes. Fix a basepoint $x_0\in M$. Take a basis $c_1,\dots, c_n$ of harmonic 1-chains for  $C_1(K^*)$ using the Euclidean inner product on $C_1(K^*)$. Harmonic chains are norm minimizing for the induced $\ell^2$-norm. Let $||\cdot||_E$ denote this norm. Let $h$ be the (unique) harmonic form that satisfies $\int_{c_i}\eta - h = 0$ for each $i$. Let $a$ be a cellular path in $K^*$ approximating $\alpha$, as in Proposition \ref{prop: length comparison}, so $||a||_G \leq L|\alpha|$ and identify the cellular path $a$ with the chain it represents.

Then, using the Hodge decomposition induced by the Euclidean inner product, we get $a = a_h^E + \d S,$ where $a_h^E$ is harmonic with respect to the Euclidean inner product on the chain complex $C_1(K^*)$ and $S$ is some 2-chain. Since $\d$ and the Euclidean adjoint $\d^*_E$ have integral bases, and since $a$ is integral, $\d S$ is a rational 2-chain. Recall that there is a constant $T=T(\e)$ such that the number of 2-simplices in $K^*$ is bounded by $T\vol(M)$. A short computation then shows,
\begin{align*}
||\d S||_G &= ||a - a_h^E||_G\\
		&\leq ||a||_G+||a_h^E||_G\\
		&\leq ||a||_G+\sqrt{T\vol(M)}||a_h^E||_E, \text{~by the Euclidean $\ell^1$-$\ell^2$ norm comparison,}\\
		&\leq ||a||_G + \sqrt{T\vol(M)}||a||_E, \text{~ as $a_h^E$ is $\ell^2$-norm minimizing in its class},\\
		&\leq ||a||_G + \sqrt{T\vol(M)}||a||_G, \text{~by the Euclidean $\ell^1$-$\ell^2$ norm comparison},\\
		&= (\sqrt{T\vol(M)}+1 )||a||_G.
\end{align*}
Because there is a minimal volume hyperbolic 3-manifold, we can increase $T$ so that we can write the above as $||\d S||_G\leq T\sqrt{\vol(M)}||a||_G$. Additionally, since there is a universal upper bound on the length of an edge in $K^*$, there is a constant $E>0$, such that the geodesic length $|\gamma|$ of a loop $\gamma$ satisfies $|\gamma|\leq E\len(c)$ for any cellular path $c$ in $K^*$ homotopic to $\gamma$.

Since $\d S$ is a rational cycle, take $N>0$ to be an integer so that $N\d S$ is integral. Then one can glue together oriented copies of the edges on which $\d S$ is supported along their boundaries to obtain a (non unique) collection of closed cellular loops $b_1,\dots,b_m$ whose union represents the cycle $N\d S$.  Fix a vertex $v_i$ in each loop $b_i$. Note that by construction, $\len(b_i) = ||b_i||_G$ for each $i$. Let $\tau_i$ be the geodesic arc connecting the basepoint $x_0$ to $v_i$ and $\tau_i^{-1}$ the oppositely oriented geodesic arc. Define the curve $b$ to be the path $$\tau_1b_1\tau_1^{-1}\tau_2b_2\tau_2^{-1}\cdots \tau_mb_m\tau_m^{-1}.$$
Let $\beta$ be the geodesic loop through $x_0$ homotopic to $b$. Notice $\sum_i ||b_i||_G = ||b||_G$, where $||b||_G$ is meant in the sense of the norm on singular chains, where the $\tau^{\pm 1}$ terms cancel. This gives a possibly trivial element of $\Gamma_\Q’$ whose length is bounded as follows:

\begin{align*}|\beta| \leq |b| &= \sum_i (2|\tau_i| + |b_i|) \\ &\leq 2\diam(M)m + E||b||_G \\ &\leq (2\diam(M) + E)||b||_G \\ &\leq (2B_0\vol(M) + E)||b||_G,
\end{align*}
where we use that $m \leq ||b||_G$, the diameter bound of Lemma \ref{lem:5.3}, along with the remarks in the above discussion.

Since $$\frac{1}{N}||b||_G = ||\d S||_G \leq T\sqrt{\vol(M)}||a||_G,$$ and  $||a||_G\leq L|\alpha|$, we obtain $$\frac{||b||_G}{N} \leq TL\sqrt{\vol(M)}|\alpha|.$$ As a result, $$\frac{|\beta|}{N}\leq TL(2B_0\vol(M)+E)\sqrt{\vol(M)}|\alpha|.$$
We compute,
\begin{align*}
\left|\int_{\alpha}\eta-h\right| &= \left|\int_{\alpha-a_h} \eta-h \right|\text{~since $a_h$ is in the span of the $c_i$, and $\int_{c_i}\eta-h$ = 0,}\\
&\leq\left| \int_{\d S}\eta- h\right| + \left|\left(\int_{\alpha} \eta - h\right) - \left(\int_{a}\eta-h\right)\right| \\
&\leq\left| \int_{\d S}\eta- h\right| + \pi||d\eta||_{\infty}||a||_G,\text{~by Lemma \ref{lem:5.2},} \\
&= \frac{1}{N}\left|\int_{N\d S}\eta-h\right| + \pi||d\eta||_{\infty}||a||_G\\
&= \frac{1}{N}\left|\int_{b}\eta-h\right| + \pi||d\eta||_{\infty}||a||_G,\text{~since $b$ abelianizes to $N\d S$,}\\
&\leq \frac{1}{N}\left(\left|\int_{\beta}\eta-h\right| + \left|\int_b(\eta-h) - \int_{\beta}(\eta-h)\right|\right) + \pi||d\eta||_{\infty}||a||_G\\
&\leq \frac{1}{N}\left|\int_{\beta}\eta-h\right| + \frac{1}{N}\pi||d\eta||_{\infty}||b||_G + \pi||d\eta||_{\infty}||a||_G,\text{~by Lemma \ref{lem:5.2}}.
\end{align*}

If $\beta$ is trivial, then the integral term $|\int_{\beta}\eta-h| $ vanishes, and we can replace that term with $$TL\pi |\alpha| ||d\eta||_{\infty}\sqrt{\vol(M)}\rho(M)^{-1} $$ to obtain (after using our estimate for $||b||_G/N$ and $||a||_G \leq L|\alpha|$)

\begin{align*}
\left|\int_{\alpha}\eta-h\right| &\leq TL\pi|\alpha|||d\eta||_{\infty}\sqrt{\vol(M)}\left(\rho(M)^{-1}+1\right) + \pi L ||d\eta||_{\infty}|\alpha|\\
&\leq TL\pi|\alpha|||d\eta||_{\infty}\sqrt{\vol(M)}\left(\rho(M)^{-1}+1\right) + \frac{\vol(M)^{3/2}}{\vol(W)^{3/2}}\pi L ||d\eta||_{\infty}||\alpha|\\
&\leq TL\pi|\alpha|||d\eta||_{\infty}\frac{\vol(M)^{3/2}}{\vol(W)}\left(\rho(M)^{-1}+1\right) + \frac{\vol(M)^{3/2}}{\vol(W)^{3/2}}\pi L ||d\eta||_{\infty}|\alpha|,
\end{align*}
where in the last line we again use the minimal volume hyperbolic 3-manifold $W$ to replace $\sqrt{\vol(M)}$ with $\vol(M)^{3/2}.$
Setting $$L_0 = 2\max\{\frac{2\pi BDL}{\vol(W)},\frac{\pi L}{\vol(W)^{3/2}}\}$$ and factoring gives the result.

Assume now that $\beta$ is nontrivial. Combining the above estimates yields

\begin{align*}
\left|\int_{\alpha}\eta-h\right| &\leq \frac{|\beta|}{N} \frac{1}{|\beta|} \left|\int_{\beta}\eta-h\right| + \frac{1}{N}\pi||d\eta||_{\infty}||b||_G + \pi L ||d\eta||_{\infty}|\alpha| \\
&\leq TL(2B_0\vol(M)+E)\sqrt{\vol(M)}|\alpha| \frac{1}{|\beta|} \left|\int_{\beta}\eta-h\right| \\
&~~~~+ \pi||d\eta||_{\infty} TL\sqrt{\vol(M)}|\alpha | + \pi L ||d\eta||_{\infty}|\alpha|
 \\
&= TL\sqrt{\vol(M)}|\alpha|\left((2B_0\vol(M) + E)\frac{1}{|\beta|}\left|\int_{\beta}\eta-h\right|  + \pi||d\eta||_{\infty}\right) \\
&~~~~+ \pi L ||d\eta||_{\infty}|\alpha|.
\end{align*}

Since the geodesic $\beta$ is nullhomologous, Lemma \ref{lem:5.4} implies $$\left|\int_{\beta}(\eta-h) \right|\leq 2\pi||d\eta||_{\infty}\scl(\beta).$$ Replacing the integral term with this estimate and using that $\frac{\scl(\beta)}{|\beta|}\leq\rho(M)^{-1}$ gives

\begin{align*}
\left|\int_{\alpha}\eta-h\right| \leq |\alpha| TL\sqrt{\vol(M)} ||d\eta||_{\infty}\left(2\pi(B_0\vol(M) + E)\rho(M)^{-1}\right)+ \pi) \\
+ \pi L ||d \eta||_{\infty}|\alpha|.
\end{align*}

Again using the existence of a minimal volume hyperbolic $n$-manifold, one can replace $B_0$ with the constant $B_1 = 2B_0 + E/\vol(W)$ since $B_1\vol(M) > 2B_0\vol(M) + E$. Then, after combining constants in the first summand (and using that $2\pi > \pi$ to pull out the terms containing $\pi$) into a single constant $L_1$, one obtains:

$$\left|\int_{\alpha}(\eta - h)\right|\leq |\alpha| L_1\vol(M)^{3/2} ||d\eta||_{\infty}\left(\rho(M)^{-1} + 1 \right) + \pi L_1 ||d\eta||_{\infty}|\alpha|. $$

Set $L_0 =2\max\{L_1,\frac{\pi L}{\vol(W)^{3/2}}\}$ and multiply the second summand by $\vol(M)^{3/2}$ to obtain the claim.

\end{proof}

\begin{lem}  \label{lem:5.6} Let $M$ have deeply embedded triangulation $K$ and let $\tilde K$ be the pullback of this triangulation to $\H^n$.
Then there is a fundamental domain $\mathcal D\subset \H^n$ for $M$ that is a union of simplices from $\tilde K$ such that the diameter of $\mathcal D$ satisfies $\diam(\mathcal D) \leq 3\diam(M).$ \end{lem}
\begin{proof}
Fix a top dimensional simplex $\sigma_0\in K^{(n)}$ and let $\tilde \sigma_0$ be a lifted copy in $\H^n$. Let $\tilde x_0$ be the barycenter of $\tilde \sigma_0$. For every other top dimensional simplex $\sigma$ in $K^{(n)}$ there is a lift 	$\tilde \sigma$ whose barycenter $\tilde x_{\sigma}$ is within $\diam(M)$ of $\tilde x_0$. Choose one such lift for every $\sigma$ in such a way the resulting fundamental domain $\mathcal D$ is connected. Then the diameter of the fundamental domain satisfies $\diam(\mathcal D)\leq \diam(M) + 2e$, where $e$ is the maximum distance from the barycenter of a simplex in a deeply embedded triangulation to its boundary. Clearly $e<\diam(M)$, so the lemma immediately follows.
\end{proof}


Now, assume $M$ has a fixed deeply embedded triangulation and let $\mathcal D$ be a fundamental domain as in Lemma \ref{lem:5.6}. Let $\gamma_i$ be the geodesics in the free homotopy class of the side pairing transformations of the fundamental domain $\mathcal D$, and notice by construction $|\gamma_i|\leq 3\diam(M)$. With this, we modify the estimate in Proposition \ref{prop:5.1}to obtain the following.

\begin{prop}\label{prop:5.7}
Let $\eta$ be a coclosed 1-form on $M$. Let $h$ be the harmonic form of Proposition \ref{prop:5.5} associated to $\eta$. Then for a constant $A_0 = A_0(\e)>0$, the following holds:
\begin{align*}
||\eta-h||_2^2 \leq A_0\vol(M)||\eta-h||_{\infty}\left(3\pi||d\eta||_{\infty} + 3L_0B_0\vol(M)^{5/2} ||d\eta||_{\infty} \left(\rho(M)^{-1} + 1\right)\right) \\ + \frac{1}{2}||d\eta||_{\infty}||\eta-h||_2\sqrt{\vol(M)}.
\end{align*}

\end{prop}
\begin{proof}

Let $\gamma_i$ realize the maximum among the integrals $\int_{\gamma_i}\eta$.
Substitute the estimate of Proposition \ref{prop:5.5} for the integral term in Proposition \ref{prop:5.1} applied to the coclosed form $\eta-h$ and the fundamental domain $\mathcal D$ to obtain
\begin{align*}
    ||\eta-h||_2^2 \leq \text{Area}(\d \mathcal D)||\eta-h||_{\infty}
    \left(3\pi||d\eta||_{\infty} +L_0\vol(M)^{3/2} ||d\eta||_{\infty} |\gamma_i| \left(\rho(M)^{-1} +1 \right)\right) \\ + \frac{1}{2}||d\eta||_{\infty}||\eta-h||_2\sqrt{\vol(M)}.
\end{align*}


Then, replace $|\gamma_i|$ with $3B_0\vol(M)$, using Lemma \ref{lem:5.6} and Lemma \ref{lem:5.3}.
Lastly, since there is an upper bound on the area of a face of any simplex in $\mathcal G_\e$ (a consequence of the bounds on the dihedral angles), the total area of the boundary of a complex made from no more than $T\vol(M)$ simplices from $\mathcal G_\e$ is bounded by $A_0\vol(M)$ for a constant $A_0$ depending on $\e$. Substituting this estimate for the $\text{Area}(\d \mathcal D)$ term completes the proof.
\end{proof}

\begin{prop}\label{prop:5.8} (Proposition 2.2 of \cite{LS}) Let $M$ be a closed hyperbolic n-manifold with $\inj(M)>\e$. Assume the first positive eigenvalue $\lambda$ of the Laplacian acting on coexact 1-forms is less than some fixed constant $H>0$. Then there is a constant $C(H,\e)>0$ such that for a coexact $\lambda$-eigenform $\omega$, one has $$||\omega||_{\infty}\leq C(H,\e)||\omega||_2.$$
\end{prop}

\begin{prop}\label{prop:5.9}
Let $M$ be a closed hyperbolic n-manifold with $\inj(M)> \e$. Let $\lambda < H$ be the first positive eigenvalue for the Hodge Laplacian acting on coexact 1-cochains. Then the following holds:
\begin{align*}\frac{1}{\sqrt{\lambda}} \leq A_0\vol(M)C(H,\e)^2
    \left(3\pi+ 3L_0B_0\vol(M)^{5/2}\left(\rho(M)^{-1} + 1\right)\right) \\
    + \frac{C(H,\e)}{2}\sqrt{\vol(M)}.
\end{align*}
\end{prop}

\begin{proof}Let $\eta$ be a $\lambda$ coexact eigenform.
Applying the Sobolev type estimate of Proposition \ref{prop:5.8} to each instance of the sup norm in Proposition \ref{prop:5.7} and using that $||d\eta||_2=\sqrt{\lambda}||\eta||_2 \leq \sqrt{\lambda}||\eta - h||_2$, where the inequality follows from the orthogonality of the Hodge decomposition, gives

\begin{align*}||\eta-h||_2^2 \leq A_0\vol(M) C(H,\e) ^2 \sqrt{\lambda} ||\eta-h||_{2}^2 \left(3\pi + 3L_0B_0\vol(M)^{5/2}\left(\rho(M)^{-1}
    + 1\right)\right) \\ + \frac{C(H,\e)}{2}\sqrt{\lambda} ||\eta-h||_2^2\sqrt{\vol(M)}.
\end{align*}

Dividing both sides by $\sqrt{\lambda} ||\eta-h||_2^2$ then gives \begin{align*} \frac{1}{\sqrt{\lambda}}\leq A_0\vol(M) C(H,\e) ^2\left(3\pi+ 3L_0B_0\vol(M)^{5/2}\left(\rho(M)^{-1} + 1\right)\right) \\ + \frac{C(H,\e)}{2}\sqrt{\vol(M)}.\end{align*}\end{proof}

Rearanging the terms and combining constants (which again requires the existence of a minimal volume hyperbolic $n$-manifold) in the previous proposition and applying a geometric estimate of Calegari and a systolic inequality due to Sabourau leads to the main theorem of this section.

\begin{mainthm}\label{thm:B}
Let $M$ be a closed hyperbolic $n$-manifold with $\inj(M) > \e$. Let $\lambda$ be the first positive eigenvalue for the Laplacian acting on coexact 1-forms and let $H > \lambda$. Then there is a constant $P(H,\e)>0$ such that $$ \frac{P\rho(M)}{\vol(M)^{7/2+1/n}}\leq\sqrt{\lambda}.$$
\end{mainthm}
\begin{proof}
First we rearrange the previous proposition and combine constants into one constant $P$ to get the estimate $$\frac{P\rho(M)}{(1+\rho(M))\vol(M)^{7/2}}\leq\sqrt{\lambda}.$$ We need an estimate of Calegari’s (see the proof of Theorem 3.9 in \cite{Calegari}, the estimate at the bottom of page 58) which gives that for a genus $g$ surface $S$ with boundary $\d S = \gamma^m$,
one has $$\frac{m|\gamma|}{12g-6}\leq 4\mu + \frac{2\pi}{3\mu} + 2|\gamma|,$$ where $\mu$ depends only on the dimension $n$. Since $\chi_-(S)\geq2g-1,$ we get $$\frac{2m|\gamma|}{\chi_-(S)}\leq 24\left(4\mu + \frac{2\pi}{3\mu} + 2|\gamma|\right).$$
Since this is true for any surface $S$ bounding a power of $\gamma$, we obtain $$\frac{|\gamma|}{\scl(\gamma)}\leq 24\left(4\mu + \frac{2\pi}{3\mu} + 2|\gamma|\right).$$
We also have the commutator systolic inequality of Sabourau from Theorem 1.4 in \cite{Sab}, which bounds the shortest nontrivial integrally nullhomologous loop $\gamma\in \Gamma’$ by $$|\gamma|\leq c\vol(M)^{1/n},$$ for a dimensional constant $c$.

Both the inequality of Calegari and the systolic inequality involve a dimensional constant; let $\mu$ be the maximum of these constants in dimension $n$ and write Calegari’s inequality as $\frac{|\gamma|}{\scl(\gamma)} \leq \mu(1+|\gamma|)$. Then we get $$ \rho(M)\leq \frac{|\gamma|}{\scl(\gamma)}\leq \mu(1 + |\gamma|) \leq \mu (1+ \mu\vol(M)^{1/n}).$$
Inserting this upper bound into the denominator of the above rearranged estimate above gives $$\frac{P\rho(M)}{(1 + \mu(1+\mu\vol(M)^{1/n})\vol(M)^{7/2}}\leq \frac{P\rho(M)}{(1+\rho(M))\vol(M)^{7/2}} \leq \sqrt{\lambda}.$$ We can then increase $P$ to allow us to pull out the volume term and absorb $\mu$, thereby obtaining the desired estimate.
\end{proof}
