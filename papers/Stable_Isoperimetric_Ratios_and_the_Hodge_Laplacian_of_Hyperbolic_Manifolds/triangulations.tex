\section{Triangulations and Whitney forms}
\label{sec:2}

The purpose of this section is to outline the basic properties of the triangulations we use in this paper and how the Whitney map relates these triangulations to the de Rham complex.

\subsection{Deeply embedded triangulations}

In this section we study certain triangulations, called deeply embedded triangulations, of hyperbolic manifolds with injectivity radius bounded below that enjoy useful combinatorial and geometric properties that will facilitate the estimates in sections 3 through 5. While we focus on the hyperbolic setting, we give an account motivated by potential generalizations to the variable negative curvature setting.

The triangulations we use are generally obtained via Delaunay complexes associated to collections of points. To obtain a Delaunay complex in a Riemannian manifold $M$, take a finite collection of points $P\subset M$ and consider the Voronoi celluation consisting of cells $$V_p = \{x\in M~:~ d(x,p) \leq d(x,q)~\text{for all } p \neq q \in P\}$$ for $p\in P.$ Dual to the Voronoi celluation is the Delaunay complex. The cells of the Delauney complex are the convex hulls of tuples of points whose corresponding Voronoi cells have nonempty intersections. In \cite{Bois}, it is shown that if the collection of points $P$ satisfies certain density and separation conditions, then there is a quantifiably small perturbation of the point set $P$ whose Delaunay complex is a triangulation. The simplices of this triangulation are geodesic. The precise conditions are as follows.

Let $M$ be a closed Riemannian manifold with distance function $d$. Given a pair $1\geq\mu>0,~\e>0$, a $(\mu,\e)$-net is a collection $P$ of points in $M$ for which the following hold:

\begin{enumerate}
    \item ($P$ is $\e$-dense) For all $x\in M$, there is a $p\in P$ such that $d(x,p)<\e$.

    \item ($P$ is $\mu$-separated) All distinct $p,q\in P$ satisfy $d(p,q) \geq \mu\e$.

\end{enumerate}

Theorem 3 of \cite{Bois} says that if $\mu$ and $\e$ satisfy several inequalities relating to the injectivity radius and sectional curvatures, a $(\mu,\e)$-net can be perturbed to $(\mu’,\e’)$-net such that the resulting Delaunay complex is indeed a triangulation. This theorem, specialized to closed hyperbolic $n$-manifolds, becomes:
\begin{thm}

\cite{Bois} Let $M$ be a closed hyperbolic $n$-manifold and $P$ a $(\mu,\e)$-net such that \[\e \leq \min\left\{\frac{\inj(M)}{4},~\Psi(\mu)\right\},\] where $\inj(M)$ is the injectivity radius of $M$ and $\Psi$ is a function of the net parameter $\mu$. The function $\Psi$ is described in \cite{Bois} and is independent of the manifold $M$. Then there is a point set $P’$ that is a $(\mu’,\e’)$-net with resulting Delaunay complex a triangulation. Moreover, $\mu’$ and $\e’$ satisfy the following (see equation (2) in \cite{Bois}):
: \begin{align*}
     \e &\leq \e’ \leq \frac{5}{4}\e, \\
    \frac{2}{5}\mu &\leq \mu’ \leq \mu.
\end{align*}
\end{thm}

The separation and density conditions of a $(\mu’,\e’)$-net ensure that the resulting Delaunay triangulation has edge lengths lying in the closed interval $[\frac{2}{5}\mu \e ,2\e]$. We now specialize to the case $\mu=1$. Note that when $n=3$, $\Psi(1) = 2\times 3^{121.5}\times 5^{-81}$, which is roughly 45.15.

Set $\epsilon_0 = \min\{\e/10,\Psi(1)\}$ and define $\mathcal{G}_{\e}$ to be the space of hyperbolic $n$-simplices with edge lengths in the interval $[\frac{2\epsilon_0}{5} ,2\epsilon_0]$. Because the space of all hyperbolic $n$-simplices is parametrized by edge lengths, the space $\mathcal{G}_{\e}$ is compact. Proposition \ref{prop: existence} below establishes that every closed hyperbolic manifold $M$ with injectivity radius bounded below by $\e$ admits a triangulation whose simplices are isometric to those in $\mathcal{G}_\e$.

In a triangulation $K$, the combinatorial 1-neighborhood of a simplex is the union of the stars of the vertices of that simplex. A triangulation $K$ of a hyperbolic manifold $M$ is an \textbf{$\e$-deeply embedded triangulation} if:

\begin{enumerate}
    \item Every simplex is geodesic and contained in $\mathcal G_\e$,
    \item The combinatorial 1-neighborhood of every simplex lifts isometrically to $\H^n$.
\end{enumerate}

Throughout, when referring to deeply embedded triangulations, we generally suppress reference to some fixed $\e$.

\begin{prop} \label{prop: existence}
Let $M$ be a closed hyperbolic $n$-manifold and let $0<\e < \inj(M)$. Then there is a deeply embedded triangulation $K$ of $M$. That is, there is a geodesic triangulation $K$ of $M$ whose simplices come from $\mathcal G_{\e}$ such that the combinatorial 1-neighborhood of every simplex isometrically embeds in $\H^n$.
\end{prop}

\begin{proof} Set $\epsilon_0 = \min\{\e/10,\Psi(1)\}$.
 Take a maximal collection of points $P\subset M$ such that the balls $B_{\epsilon_0/2}(p)$ for $p\in P$ are all disjoint. By maximality, the $B_{\epsilon_0}(p)$ balls then cover $M$.
 Since the $B_{\epsilon_0/2}(p)$ balls are all disjoint, we have that for all $p,q\in P$, $d(p,q) \geq \epsilon_0 = \mu {\epsilon_0} $, so $P$ is $\mu$-separated. Since the $B_{\epsilon_0}(p)$ balls cover, every point $x\in M$ is ${\epsilon_0} $-close to some point $p\in P$, so $P$ is ${\epsilon} $-dense.
 The collection $P$ therefore is a $(\mu, {\epsilon_0})$-net, with $\mu = 1$ and ${\epsilon_0} $ satisfying the hypotheses of Theorem 2.1. Consequently, there is a perturbation of $P$ that is a $(\mu’,{\epsilon_0}’)$-net whose Delaunay complex is a triangulation. Since, as remarked above, the edge lengths of simplices in this Delaunay triangulation lies in the interval $[\frac{2}{5} {\epsilon_0} ,2 {\epsilon_0}]$, the simplices come from $\mathcal G_{\e} $.
 The edge length bound along with the fact ${\epsilon_0} < \inj(M)/10$ ensures the diameter of any vertex star (which will be less than 3 times the length of the longest edge of a simplex) will be less than $2 {\epsilon_0}$. Thus, the star of every vertex embeds isometrically in $\H^n$ via the local inverse of the exponential map. \end{proof}

We will also use cell complexes that are dual to deeply embedded triangulations. Every simplicial triangulation $K$ of a closed Riemannian manifold admits a dual celluation $K^*$ comprised of cells $\sigma^*$ dual to the simplices $\sigma$ of the triangulation $K$ in the following sense (for a reference, see \cite{bredon} chapter VI.6): Take the first barycentric subdivision $\tau(K)$ of $K$, then the $n$-cells of the dual celluation $K^*$ are the closed stars of the vertices of the original triangulation $K$ in the barycentric subdivision. This celluation is naturally triangulated by the the barycentric subdivision triangulation $\tau(K)$. Like $K$, the dual celluation can be uniformly controlled.

The controlled geometry of deeply embedded triangulations and their dual celluations primarily manifests in the compactness of $\mathcal G_\e$ and the following local bound.

\begin{prop} \label{prop: star bound}Let $M$ be a closed hyperbolic $n$-manifold of injectivity radius $\inj(M)>\e$ with a deeply embedded triangulation $K$. Then there is a positive constant $N = N(\e)$ such that the number of $k$-simplices in the star of a $j$-simplex is less than $N$.
\end{prop}

\begin{proof}

The edge length bounds provide a lower bound on the angle between two edges meeting at a vertex that span a face via the hyperbolic law of cosines. This implies that the number of $n$-simplices meeting at a vertex $v$ is bounded uniformly, since for any ball around the vertex, there is a uniform lower bound on the volume of the intersection of an $n$-simplex containing the vertex $v$ and the ball. It follows that there is an $N$ such that the number of $k$-simplices in the star of a vertex is less than $N$ for $k = 0,\dots, n$.\end{proof}


Proposition \ref{prop: star bound} and the compactness of the space $\mathcal G_\e$ of simplices together imply that the geometry and combinatorics of the 1-neighborhood of a simplex in a deeply embedded triangulation is uniformly controlled.
In particular, let $K$ be a deeply embedded triangulation of $M$. By Proposition \ref{prop: star bound}, any vertex in $K$ is contained in at most $N$ simplices. Therefore, there are finitely many possible finite simplicial complexes that appear as the combinatorial 1-neighborhood of a simplex in a deeply embedded triangulation. Let $\mathcal C$ be the finite set of such possible complexes.

For any complex $a \in \mathcal C$, say with $|a|$ many $n$-simplices, a hyperbolic structure on $a$ is given by identifying each $n$-simplex in $a$ with a model simplex in $\mathcal G_{\e}$ so that the face gluing maps are isometries. The possible geometric structures on $a$ are parametrized by a subspace $\mathcal S_\e(a)$ of $\mathcal G_{\e}^{|a|}$. Because the gluing conditions are closed, and because $\mathcal G_\e$ is compact, the space $\mathcal S_\e(a)$ is compact.
By taking the disjoint union over the finite list of possible complexes $a\in\mathcal C$, there is a compact space $\mathcal S_{\e}$ that parametrizes the geometry of the combinatorial 1-neighborhood of a simplex in a deeply embedded triangulation.

We now turn to relating closed geodesics in $M$ to cellular paths in $K$.

\begin{prop} \label{prop: loop multiplicity}
Let $M$ be a closed hyperbolic $n$-manifold with injectivity radius $\inj(M)>\e$ with a deeply embedded triangulation $K$. Let $\gamma$ be a closed geodesic curve in $M$. Then there is a constant $J = J(\e)$ such that the number $v$ of cells in the dual cell complex $K^*$ that $\gamma$ intersects (counted with multiplicity) satisfies \[v\leq J|\gamma|.\] \end{prop}

\begin{proof}

Suppose $\gamma$ moves from an $n$-cell $\sigma$ to an $n$-cell $\sigma’$, intersecting the ($n-1)$-skeleton of $K^*$ at a point $p\in \sigma\cap \sigma’$. Consider the closed radius $\e$-ball at the point $p$, $V = \bar B_{\e}(p).$ Let $x$ be the point at which $\gamma$ enters $V$ and let $y$ be the point at which it exits $V$.
Then the geodesic subarc of $\gamma$ running from $x$ to $y$ has length $2\e$. Since the triangulation $K$ has simplices from $\mathcal G_{\e}$, the restrained combinatorics of the dual celluation ensures that the ball $V$ intersects a universally bounded number of dual cells. Let $R(\e)$ denote this bound.

Consider the sequence $x_n,y_n$ of points such that $x_1 = x$ and $y_1 = y$ from above for the first simplex crossing, and $x_n$ is obtained by taking the simplex crossing that happens after $y_n$. Then each pair $x_n,y_n$ corresponds to a geodesic sub arc of $\gamma$ that intersects at most $R(\e)$ simplices. Thus, $\nu \leq (\frac{|\gamma|}{2\e} +1)R(\e)$.
It therefore follows that $\frac{v}{R(\e)} 2\e \leq |\gamma| + 2\e.$ Since $\e \leq |\gamma|$, we have $|\gamma|+2\e\leq 3|\gamma|$, and the stated linear bound follows with $J = \frac{3R(\e)}{2\e}$. \end{proof}

The next result compares the lengths of closed geodesics in $M$ to approximating paths in the 1-skeleton of dual celluation $K^*$. To measure the complexity of paths in $K^*$, let $||\cdot||_G$ be the $\ell^1$-norm on chains and $\len(\cdot)$ the word length of the cellular path. For a cellular path $c$, let $||c||_G$ be the $\ell^1$-norm of the corresponding chain.

\begin{prop} \label{prop: length comparison} There is a constant $L = L(\e)>0$ such that for any closed geodesic curve $\gamma$ in $M$, there is a cellular path $c$ in $K^*$ homotopic to $\gamma$ such that $||c||_G\leq \len(c) \leq L|\gamma|$.
\end{prop}

\begin{proof}

Fix a base point and orientation for $\gamma$ such that the base point lies on a face of a top dimensional cell. The curve $\gamma$ can be replaced by a homotopic curve whose length is bounded by a constant times the geodesic length of $\gamma$ and which intersects the boundary of every simplex at vertices. This follows from Proposition \ref{prop: loop multiplicity} which gives that there is a bound on the number of simplices $\gamma$ intersects (counting these intersections with multiplicity) that depends linearly on the length of $\gamma$ and the fact the simplices of the triangulation have bounded diameter. Using the orientation and basepoint, we obtain a sequence of vertices with line segments between them that lie entirely in a cell. We can further modify $\gamma$ by replacing these curve segments with curves that lie in the 1-skeleton by traversing the 1-simplex joining the two boundary vertices. Since the edges in the celluation $K^*$ have bounded length, this again adds bounded length to the curve. Let $c$ denote the cellular path we have constructed. By the previous considerations, there is a constant $L$ depending only on $\e$ giving the comparison $\len(c)\leq L|\gamma|$. The inequality $||c||_G\leq \len(c)$ is trivial.

\end{proof}


The geometry of geodesic simplices in hyperbolic manifolds can also be understood using barycentric coordinates via Thurston’s straightening map (see \cite{thurstonbook} page 124). Identify $\H^n$ with the upper sheet of the hyperboloid in Minkowski space $\R^{n,1}$ with quadratic form $Q = x_0^2 + x_1^2 + \cdots + x_{n-1}^2 - x_n^2$ and consider a singular simplex $\sigma:\Delta\to \H^n$.
Let $b_0, \dots, b_n:\Delta\to [0,1]$ be the barycentric coordinates on the standard Euclidean simplex $\Delta$ with vertices $e_0,\dots, e_n$. Then for $v\in \Delta$, write $v = \sum b_i(v)e_i$. The straightening of $\sigma$ is the singular simplex in $\H^n$ given by centrally projecting the affine simplex $\sum b_i(v)\sigma(v_i)$ from the origin to the upper sheet of the hyperboloid. This process endows each geodesic simplex with a natural barycentric coordinate.

If $\pi:\H^n\to M$ is the projection map and $\sigma$ is a singular simplex in $M$, let $\st(\sigma):\Delta\to M$ be the composition of the straightening of $\sigma$ applied to some lift of $\sigma$ and the projection map.
This is well-defined and independent of the lift because the isometry group of $\H^n$ acts linearly on $\R^{n,1}$ preserving the quadratic form $Q$.

For a geodesic simplex $\sigma$ in $\H^n$, let $V_{\sigma}:\sigma \to \Delta$ be the map from $\sigma$ to the standard simplex given by the barycentric coordinates induced by straightening. Geodesic simplices $\sigma,\sigma'$ can be compared using the composition $V_{\sigma’}^{-1}\circ V_{\sigma}$.
Using the straightening construction and the barycentric coordinates, one sees that the maps $V_{\sigma}$ depend continuously on $\sigma$ in the sense that if $\sigma$ is a straight simplex in $\H^n$ and $\sigma’$ is obtained by perturbing the vertices of $\sigma$, then the composition map $V_{\sigma’}^{-1}\circ V_{\sigma}$ is almost an isometry, where the failure to be an isometry is controlled by the size of the vertex perturbation.


\begin{prop}\label{prop: simplices biLipschitz} Let $\sigma$ and $\sigma’$ be geodesic simplices from $\mathcal G_\e$ embedded in $\H^n$. Then the map $V_{\sigma’}^{-1}\circ V_{\sigma}$ is $\kappa$-biLipschitz for some $\kappa = \kappa(\e)>0$ that does not depend on $\sigma$ and $\sigma’$.
\end{prop}
\begin{proof}
The biLipschitz constant for the comparison map between any given simplex and the Euclidean simplex depends continuously on the simplex. The result then follows from the compactness of $\mathcal G_\e$.
\end{proof}


We also have uniform control over the geometry of the combinatorial 1-neighborhood of a simplex.

\begin{prop}\label{prop: stars biLipschitz} There is a constant $\mathcal L = \mathcal L(\e)$ such that  if $s$ and $s’$ are two complexes of the same combinatorial type in $\mathcal S_\e$, then $s$ and $s’$ are $\mathcal L$-biLipschitz equivalent.
\end{prop}
\begin{proof}
    Define maps from the combinatorial 1-neighborhood $s$ of a simplex $\sigma$ to an abstract Euclidean model by gluing the maps $V_{\sigma'}$ together according to the combinatorics of $s$. Since the gluing maps are isometries, this is well defined. Since the map restricted to each simplex $\sigma'$ is uniformly biLipschitz equivalent to the model simplex, and since there are a uniformly bounded number of simplices in $s$, it follows that $s$ is uniformly biLipschitz equivalent to the Euclidean model.
\end{proof}

Lastly, we note that the lower bound on volumes of simplices in deeply embedded triangulations implies the following.

\begin{prop} \label{prop: volume}There is a constant $T=T(\e)$ such that if $M$ is a closed hyperbolic $n$-manifold with $\inj(M)>\e$ and $K$ is a deeply embedded triangulation of $M$, then the number $V_K$ of simplices in $K$ satisfies $$V_K\leq T\vol(M).$$
\end{prop}



\subsection{Whitney forms}

The combinatorial geometry of the triangulation $K$ is related to the Riemannian geometry of $M$ by way of the Whitney form map $W:C^{\bullet}(K)\to L^2\Omega^{\bullet}(M)$ relating the cochain complex $C^{\bullet}(K)$, with $\R$ coefficients, to the $L^2$-de Rham complex. It will be useful to view $C^{\bullet}(K)$ as a subcomplex of the singular cochain complex. With this in mind, we often identify singular simplices with their images.

The Whitney map is readily defined using the basis for $C^{\bullet}(K)$ dual to the basis of simplices. This basis consists of the cochains $\delta_{\sigma}$ that take the value 1 on the simplex $\sigma$ and zero on all other simplices.
The Whitney form $W(\delta_{\sigma})$ associated to the cochain $\delta_{\sigma}$ dual to an  oriented simplex $\sigma = [v_0,\dots,v_q]$ is given by
$$W(\delta_{\sigma}) = q! \sum_{k=0}^q(-1)^kb_{k}db_0\wedge \cdots \wedge db_{k-1}\wedge db_{k+1}\wedge \cdots \wedge d b_q,$$
where $b_k:M\to [0,1]$ is the barycentric coordinate associated to the vertex $v_k$. See \cite{Dodziuk} for more details.

An $L^2$-form in the image of $W$  is called a Whitney form. The support of a Whitney form $W(\delta_{\sigma})$ is contained in the closed star of the simplex $\sigma$. The barycentric coordinates used to define the Whitney forms are not smooth, however, they are smooth in the compliment of the $(n-1)$-skeleton of $K$. One can define the exterior derivative of a Whitney form in a weak sense, which yields a differential that is well defined as an $L^2$-form. With this exterior derivative, the Whitney map becomes a chain map. For any cochain $f$ and simplex $\sigma$, the restriction of the Whitney cochain $\omega= W(f)$ to $\sigma$, denoted $\omega|_{\sigma}$, can be uniquely extended to a smooth form on the boundary of $\sigma$. This extension however is not unique when $\sigma$ lies in the boundary of multiple simplices. In addition to the restriction of Whitney forms, we have the restriction for cochains.
If $f= \sum a_i \delta_{\sigma_i}$ is a cochain and $\sigma$ is a simplex, then $f|_{\sigma} = \sum\limits_{\sigma_i\subset\sigma}a_i\delta_{\sigma_i}.$ This cochain restriction satisfies $\omega|_{\sigma} = W(f|_{\sigma})|_{\sigma}$.

A Whitney form associated to a geodesic simplex $\sigma$ in $M$ is the corresponding Whitney form on the standard Euclidean simplex pulled back to $\sigma$ via the map $V_\sigma$. Geometric norms on cochains determined by the Whitney map can be compared with various combinatorial norms. This is done using the $L^p$-change of variables formula for $k$-forms, see \cite{stern}.

\begin{prop} \label{prop: L2 comparisons}
Let $s\in\mathcal S_\e\cup\mathcal G_\e$. Let $W(f)$ be the Whitney form associated to a cochain $f\in C^{k}(s;\R)$. Let $||\cdot||$ be some fixed norm on the real vector space $C^{k}(s;\R)$ and let $||\cdot||_{p,s}$ be the $p$-norm associated to $s$ on  $\Omega^k(s)$, where $p = {\infty}$ or $p=2$.
Then there is a constant $\mathcal A = \mathcal A (\e,||\cdot||)>0$ such that $$\mathcal A^{-1}||W(f)||_{p,{s}}\leq  \mathcal ||f|| \leq \mathcal A||W(f)||_{p,s}.$$ The constant $\mathcal A$ only depends on the chosen norm and combinatorial type of $s,$ not on the geometric structure of $s$.
\end{prop}
\begin{proof}
By Proposition \ref{prop: stars biLipschitz}, there is a constant $\mathcal L$ (note $\kappa\leq \mathcal L$, so that if we’re working with simplices, $\mathcal L$ works as Lipschitz constant) such that any pair $s,s’\in \mathcal S_\e\cup\mathcal G_\e$ in the same combinatorial type are $\mathcal L$-biLipschitz equivalent.
For each combinatorial type, fix a model $s_a\in \mathcal S_\e\cup\mathcal G_\e$ and let $\mu:s\to s_a$ be the $\mathcal L$-biLipschitz comparison map described above. Then, because Whitney forms are obtained by pulling back the standard Whitney forms via the model map, we can apply the $L^2$-change of variables formula for $k$-forms and use the biLipschitz comparison to get $$\frac{||W(f)||_{2,{s}}}{||f||}\leq \mathcal L^{n/2}\frac{||W(f)||_{2,s_a}}{||f||}.$$ Similarly, applying the $L^{\infty}$-change of variables formula for $k$-forms, gives
$$\frac{||W(f)||_{{\infty},s}}{||f||}\leq \mathcal L^{k}\frac{||W(f)||_{\infty,s_a}}{||f||}.$$

For $p=2$, set $$\mathcal A_2 =  \mathcal L^{n/2}\max\limits_a \sup\limits_{g\in C^{k}(s_a)} \frac{||W(g)||_{2,s_a}}{||g||}$$ and for $p = \infty$, set $$\mathcal A_{\infty} = \mathcal L^{n} \max\limits_a\sup\limits_{g\in C^{k}(s_a)} \frac{||W(g)||_{2,s_a}}{||g||},$$
where the maximum runs over all combinatorial types $a$.
Then, $A = \max\{\mathcal A_2,\mathcal A_{\infty}\}$ gives the first inequality.
The second inequality is obtained from an identical argument via the lower bound in biLipschitz comparison. Let $\mathcal A$ be the maximum of these two constants.
\end{proof}

This estimate is enough to reproduce the statements in \cite{LS} for deeply embedded triangulations. However, to obtain the cleaner discrete-smooth eigenvalue comparison in Proposition \ref{prop:4.2}, we need to study a smooth analogue of the Whitney map. The smooth Whitney map was introduced by Dodzuik in \cite{dodzuik2}. The map is defined by replacing the barycentric coordinates with a smooth partition of unity indexed by the vertices of a triangulation. The particular partition of unity is provided by the following proposition.

\begin{prop} \label{prop: partition existence} (\cite{dodzuik2}, Lemma 2.11)
If $M$ admits a deeply embedded triangulation $K$, then there exists a $C^{\infty}$ partition of unity $\beta_i$ indexed by the vertices of $K$ and subordinate to the covering of $M$ by open stars of vertices of $K$ (indeed, compactly supported in each open star). Moreover, each $\beta_i$ has covariant derivatives satisfying the pointwise bound $|\nabla^k\beta_i|< C$ for some constant $C = C(\e)$, for $k\leq n$.
\end{prop}

\begin{proof}
Let $s\in\mathcal S_\e$ be the combinatorial 1-neighborhood of a simplex. Denote the vertices of $s$ by $v_0,\dots, v_n$ and let $b_i$ be the standard barycentric coordinates associated to the vertex $v_i$. Define

\[\bar b_i(x) =  \begin{cases}
       0 & b_i(x)\leq 1/(n+2), \\
       \frac{(n+2)b_i(x)-1}{n+1} & b_i(x) \geq 1/(n+2). \\

   \end{cases}
\] Observe that $\sum\limits_i \bar b_i(y) \geq \frac{1}{(n+2)}$. Define $$\delta(s) = \inf\limits_{\substack{x\in\supp(\bar b_i)\\y\in \d \s(v_i)}}d(x,y),$$ where $d$ is the distance function induced by the Riemannian metric.  Set $\delta = \frac{1}{2}\inf\limits_{s\in \mathcal S_\e} \delta(s)$ and notice $\delta>0$. For any point $x\notin \s(v_i)$, the ball $B_{\delta}(x)$ is disjoint from the support of $\bar b_i$.
Let $\eta$ be a smooth cutoff function such that $\eta(r) = 1$ when $|r|<\delta/2$ and $\eta(r) = 0$ when $|r|>3\delta/4$. The function $\eta(d(x,y))$ is smooth on the open star of a vertex, so the operator given by integrating against $\eta(d(x,y))$ is smoothing. Therefore, if we define $$\tilde b_i(x) = \int_{B_{3\delta/4}(x)}\eta(d(x,y))\bar b_i(y)dy,$$ the result is a smooth function.  Notice $\tilde b_i$ is supported in the interior of the star of $v_i$ by virtue of our choice of $\delta$. We now define smoothed barycentric partitions of unity for a smooth manifold with deeply embedded triangulation $K$ by normalizing the functions $\tilde b_i$ associated to the vertices of $K$:
$$\beta_i(x) = \left(\sum\limits_j \tilde b_j (x)\right)^{-1} \tilde b_i (x).$$ Notice that if $\tilde b_i(x)\neq 0$, then this normalizing sum really just runs over the vertices of $\s(v_i)$. This normalizing constant can be bounded from below:
\begin{align*}
    \sum\limits_j \tilde b_j(x) &\geq \sum\limits_j \int_{B_{\delta/2}(x)}\bar b_j(y)dy \\
    &= \int_{B_{\delta/2}(x)} \sum\limits_j\bar b_j(y)dy \\
    &\geq \vol(B_{\delta/2}(x))\frac{1}{(n+2)(n+1)}.
\end{align*}

The covariant derivative bound follows from repeated application of the quotient rule and the corresponding bounds for $\bar b_i$, which  depends only on the derivatives of cutoff function $\eta$ and the covariant derivatives of the metric. The choice of $\delta$ ensures each function $\beta_i$ is compactly supported in the star of $v_i$.
\end{proof}

Our aim now is to establish a version of Proposition \ref{prop: L2 comparisons} for these partitions of unity that will allow us to relate the geometric norms induced by the smooth Whitney map to combinatorial norms.


Let $M$ be a hyperbolic $n$-manifold with a deeply embedded triangulation $K$. Denote by $\s_0(\sigma)$ the combinatorial 1-neighborhood of an $n$-simplex $\sigma$ in $K$.
Fix some $n$-simplex $\sigma$ in $K$ and set $s = \s_0(\sigma)$. Because $K$ is deeply embedded, for any point $p\in s$, the ball $B = B_\e(p)$ contains $s$ and lifts isometrically to $\H^n$. Identify $s$ with such a lift. The functions $\beta_i$ associated to the vertices of $\sigma$ are supported in $s$ and their value in any simplex $\sigma$ depends only on the geometry of $s$. This enables us to isolate the local properties of the barycentric partition of unity functions.
By perturbing the vertices of $s$ in $\H^n$ and modifying the various simplices making up the complex $s$ accordingly, we can then see how these functions relate to the geometry of combinatorial 1-neighborhoods as encoded by the space $\mathcal S_\e$.

\begin{lem} \label{lem: partition of unity continuous}
Let $\sigma$ be an $n$-simplex with vertices $v_0,\dots,v_n$ from $\mathcal G_\e$ contained in combinatorial 1-neighborhood $s = \s_0(\sigma) \in \mathcal S_\e$ with additional vertices $v_{n+1},\dots,v_m$.
The functions $\beta_i$ associated to the vertices of $\sigma$ constructed in Proposition \ref{prop: partition existence} and their covariant derivatives $\nabla \beta_i$ vary continuously in $L^2(\H^n)$ when the perturbation of $s$ in $\mathcal S_\e$ is realized as above in $\H^n$.
\end{lem}
\begin{proof}

As above, we can embed the combinatorial 1-neighborhood $s$ in $\H^n$ and the functions $\tilde b_i$ for each vertex $v_i$ of $\sigma$ described in Proposition \ref{prop: partition existence} are well defined smooth functions on $\H^n$ supported on $s$. To define the barycentric partition of unity, we also need the functions $\tilde b_k$ for vertices $v_k$ in $s$ that are not in $\sigma$ to be well defined on the support of the functions $\tilde b_i$ for vertices $v_i$ of $\sigma$.
For a vertex $v_k$ in $s$  that is not part of $\sigma$ and a point $x$ in the support of $\tilde b_i$ for $v_i$ a vertex of $\sigma$, we have that the ball $B_{3\delta/4}$ used in the definition of $\tilde b_k$ is contained in $s$, so $\tilde b_k(x)$ only depends on $s$.

From the definition, one sees that each function $\tilde b_i$ varies continuously in $L^2(\H^n)$ as the complex $s$ in $\H^n$ is varied by perturbing the vertices and modifying the various simplices making up the complex $s$ accordingly. One similarly can see from the definition of $\tilde b_i(x)$ that $\nabla \tilde b_i(x)$ varies continuously as $s$ is varied as above.
The functions $\beta_i$ in the barycentric partition of unity are defined by normalizing the functions $\tilde b_i$: $$\beta_i(x) = \left(\sum\limits_{v_k\in s^{(0)}} \tilde b_k (x)\right)^{-1} \tilde b_i (x) .$$ By the remark above, $\tilde b_k$ is well defined on the support of $\tilde b_i$ for every vertex $v_k$ of $s$. Since each $\tilde b_j$ and $\nabla \tilde b_j$ varies continuously with $s$, the same holds for $\beta_j$ and $\nabla \beta_j$.
\end{proof}

To a partition of unity indexed by the vertices of a triangulation and subordinate to the covering by open stars of vertices, one can define a generalized Whitney mapping, given by the same formula as the standard Whitney map but with the smooth barycentric partitions of unity in place of the standard barycentric coordinates. Let $\beta = (\beta_i)$ be the barycentric partition of unity. The Whitney form $W_{\beta}(\delta_{\sigma})$ associated to the cochain $\delta_{\sigma}$ dual to an  oriented simplex $\sigma = [v_0,\dots,v_q]$ is given by $$W_{\beta}(\delta_{\sigma}) = q! \sum_{k=0}^q(-1)^k\beta_{k}d\beta_0\wedge \cdots \wedge d \beta_{k-1}\wedge d \beta_{k+1}\wedge \cdots \wedge d \beta_q,$$ Like the standard Whitney map, these generalized Whitney maps satisfy:
\begin{enumerate}
	\item For a chain $a$ and cochain $f$ of the same degree, $\int_aW_{\beta}(f) = f(a).$
	\item For any cochain $f$, $dW_{\beta}(f) = W_{\beta}(df)$.
	\item If $p$ is contained in the interior of an  $n$-simplex $\sigma$, and any cochain $f$, $W_{\beta}(f)_p = W_{\beta}(f|_{\sigma})_p$.
\end{enumerate}

\begin{lem} \label{lem: forms continuous} Fix a simplex $\sigma\in\mathcal G_\e$ contained in its combinatorial 1-neighborhood $s\in \mathcal S_\e$.
The Whitney map $W_{\beta}: C^{\bullet}(\sigma)\to L^2\Omega^{\bullet}(\H^n)$ varies continuously as the geometry of the star $s$ varies in $\H^n$ as described above. Consequently, $||W_{\beta}(f)||_{2,s}$ and $|| W_{\beta}(f) ||_{2,\sigma}$ vary continuously with the geometric structure on $s$ in $\mathcal S_\e$.
 \end{lem}
\begin{proof} We work with the combinatorial 1-neighborhood $s$ embedded in a ball $B\subset \H^n$ as above. Because $\beta_i$ and $\nabla \beta_i$ vary continuously with $s$ and since $||\nabla\beta_i||_{2,B} $ is comparable to $||d\beta_i||_{2,B}$, it suffices to show that exterior products of $d\beta_i$ vary continuously in the $L^2$-norm. The degree 0 case and  degree 1 case are immediate from the continuity in Lemma \ref{lem: partition of unity continuous}. We treat only the degree 2 case as the other higher degree cases are handled similarly. Assume $||\beta_i - \beta_i’||_{2,B} < \epsilon$ and $||d\beta_i - d\beta_i’||_{2,B}<\e$. Then we have \begin{align*}
||d\beta_0\wedge d\beta_1 - d\beta’_0\wedge d\beta_1’||_{2,B} &=||d\beta_0\wedge d\beta_1 - d\beta’_0\wedge d\beta_1’ + d\beta_0 \wedge d\beta_1’ - d\beta_0 \wedge d\beta_1’||_{2,B} \\
&\leq ||d\beta_0\wedge d\beta_1  - d\beta_0 \wedge d\beta_1’||_{2,B} + || d\beta_0’ \wedge d\beta_1’ - d\beta_0\wedge d\beta_1’||_{2,B} \\
&= ||d\beta_0\wedge d(\beta_1 - \beta_1’)||_{2,B} + ||d(\beta_1’ - \beta_1)\wedge d\beta_1’||_{2,B}\\
&\leq C||d\beta_0||_{2,B} ||d(\beta_1-\beta_1’)||_{2,B} + C||d\beta_1’
||_{2,B} ||d(\beta_1’-\beta_1)||_{2,B},\\
&\leq 2C\epsilon, \text{~after increasing $C$.}
\end{align*}
This immediately gives that $||W_{\beta}(f)||_{2,s}$ varies continously with $s$. Let $\xi_{\sigma}$ be the characteristic function of $\sigma$ in $\H^n$. Then $\xi_{\sigma}$ varies continuously in $L^2$ when $\sigma$ is varied by perturbing its vertices. The norm $||W_{\beta}(f)||_{2,\sigma} = ||W_{\beta}(f)\xi_{\sigma}||_{2,B}$ therefore also varies continuously when $s$ is varied in $\mathcal S_\e$.
\end{proof}

\begin{prop} \label{prop: smooth L2 comparison} There is a constant $A=A(\e)>0$ such that if $\sigma\in \mathcal G_\e$ has combinatorial 1-neighborhood $s\in\mathcal S_\e$ and $f\in C^{\bullet}(\sigma)$ is any cochain, then there are comparisons $$ A^{-1}||W(f)||_{2,\sigma}\leq ||W_{\beta}(f)||_{2,\sigma}\leq A||W(f)||_{2,\sigma}.$$
\end{prop}
\begin{proof}
    We embed $s$ in a ball $B$ in $\H^n$ as above. The $L^2$-norm induced by $\beta$ for a fixed geometric structure on $\sigma$ is continuous on the vector space of cochains $C^{\bullet}(\sigma)$. For any cochain $f\in C^{\bullet}(\sigma)$, the form $W_{\beta}(f)$ varies continuously in $L^2\Omega^{\bullet}(B)$ as the combinatorial 1-neighborhood $s$ varies in $\mathcal S_\e$.
    It follows that $||W_{\beta}(f)||_{2,\sigma}$ is continuous as a function on the component of $\mathcal S_\e \times C^{\bullet}(\sigma)$ corresponding to the combinatorial type of the combinatorial 1-neighborhood $s$.

    Since $W_{\beta}$ sends nonzero cochains to nonzero forms, for any $f\neq 0$, one has $0<||W_{\beta}(f)||_{2,\sigma}$. Let $||\cdot||_{E}$ be the usual $\ell^2$ norm on $C^{\bullet}(\sigma)$.
    The norm $\norm{\cdot}_E$ is fixed as the geometric structure on $\sigma$ varies. For $s’\in \mathcal S_\e$, let $\beta’$ be the corresponding  barycentric partition of unity defined by the combinatorial 1-neighborhood $s’$. By definition, each $s’\in\mathcal S_\e$ is the combinatorial 1-neighborhood of some simplex, let $\sigma’$ be this simplex.

    Using the continuity in Lemma \ref{lem: partition of unity continuous} and the compactness of $\mathcal S_\e$, we conclude the constants
    $$A_{\bullet} = \inf\limits_{s’\in \mathcal S_\e}\inf\limits _{\substack{f\in C^{\bullet}(\sigma’)\\ ||f||_{E} = 1}} ||W_{\beta’}(f)||_{2, \sigma’}$$
    and
    $$B_{\bullet}= \sup\limits_{s’\in \mathcal S_\e}\sup\limits_{\substack{f\in C^{\bullet}(\sigma’)\\ ||f||_{E} = 1}} ||W_{\beta’}(f)||_{2, \sigma’},$$
    are strictly positive real numbers independent of the particular geometric structure on $s$ or $\sigma$ giving the desired comparison between $||f||_E$ and $||W_{\beta}(f)||_{2,\sigma}$.

    We can then compare $||W(f)||_{2,\sigma}$ and $||f||_E$ using Proposition \ref{prop: L2 comparisons}. For $f\in C^{\bullet}(\sigma)$, Proposition \ref{prop: L2 comparisons} gives a constant $\mathcal A$ such that $\mathcal A^{-1}||W(f)||_{2,\sigma} \leq ||f||_{E}\leq \mathcal A||W(f)||_{2,\sigma}$.
    Combining these comparisons with the comparison of $||W_{\beta}(f)||_{2,\sigma} $ and $||f||_{E}$ above gives the claimed result for $||W_{\beta}(f)||_{2,\sigma} $.
\end{proof}

The upshot of this is that the smooth barycentric partition of unity induces a norm on the cochain complex of a simplex that locally is uniformly comparable to the $L^2$-norm induced by the standard barycentric partition of unity.  This gives an analogue of Proposition \ref{prop: L2 comparisons} for the $L^2$-norm induced by the barycentric partition of unity. We also require such a comparison for the $L^{\infty}$-norm. The upgraded version of Proposition \ref{prop: L2 comparisons} appears below as Proposition \ref{prop: smooth comparison}.

For a smooth manifold $Y$, possibly with boundary, we denote the Sobolev spaces of differential $\bullet$-forms by $H^k_{\nabla}\Omega^{\bullet}(Y)$ and their norms by $||\cdot||_{H^k_{\nabla}(Y)}$, where $$||\omega||_{H^k_{\nabla}} = \sum\limits_{i=0}^k||\nabla^i \omega||_2.$$ When $\bullet = 0$, we drop $\Omega^0$ from this notation.
When $Y$ has boundary, the marking $\mathring{H}^k_{\nabla}$ denotes the subspace of forms that are approximated by smooth forms supported in the interior of $Y$.

\begin{lem} \label{lem: constant R}
    There is a constant $R(\e)>0$ such that for any $n$-simplex $\sigma\in \mathcal G_\e$ with combinatorial 1-neighborhood $s\in\mathcal S_\e$, the map $W_{\beta}:C^{\bullet}(\sigma)\to H^{n}_{\nabla}\Omega^{\bullet}(B)$ satisfies
    $$||W_{\beta}(f)|_{\sigma}||_{H^n_{\nabla}(\sigma)} \leq ||W_{\beta}(f)||_{H^n_{\nabla}(B)} \leq R||W_{\beta}(f)||_{2,\sigma},$$ where we have identified the combinatorial 1-neighborhood $s$ isometrically with a domain $D$ in $\H^n$ and $B$ is a ball of radius $\e$ based at $p\in \sigma$.
\end{lem}

\begin{proof}
The first inequality follows from the definition of the Sobolov norm.

We now observe that the covariant derivative bounds for a smooth barycentric partition of unity imply that $||W_{\beta}(f)||_{H^n_{\nabla}(B)}$ is bounded by some constant times $||f||_{G,\sigma}$, where $||\cdot||_{G,\sigma}$ is the $\ell^1$-norm on $C^{\bullet}(\sigma)$. For a cochain $f = \sum a_i \delta_{\sigma_i}$, let $\omega_i = W_{\beta}(\delta_{\sigma_i})$ so that $W_{\beta}(f) = \sum a_i \omega_i$. We can then compute,
\begin{align*}
	||W_{\beta}(f)||_{H^k_{\nabla}(B)} &= ||\sum a_i\omega_i||_{H^k_{\nabla}(B)}\\
	&\leq \sum_j||\nabla^j \sum_i  a_i\omega_i||_{2,B}\\
	&\leq  \sum_j\sum_i|a_i|||\nabla^j \omega_i||_{2,B}.
\end{align*}
Each summand above satisfies $||\nabla^j\omega_i||_{2,B}<C$ for a constant $C$ depending on the covariant derivative bounds of the barycentric partition of unity. There is a constant $T$ such that the number of $\bullet$-faces of an $n$-simplex is less than $T$.
Thus, \begin{align*}
 ||W_{\beta}(f)||_{H^k_{\nabla}(B)} &\leq \sum_j\sum_i|a_i|||\nabla^j \omega_i||_{2,B} \\
&\leq \sum_jC\sum_i|a_i|\\
&\leq TC\sum_i|a_i| \\
&= TC||f||_{G,\sigma}.
\end{align*}
 For a single simplex, this combinatorial $\ell^1$-norm is comparable to the $\beta$-induced $L^2$-norm by Proposition \ref{prop: L2 comparisons} and Proposition \ref{prop: smooth L2 comparison}. Thus, $$||W_{\beta}(f)||_{H^n_{\nabla}(B)}\leq R||W_{\beta}(f)||_{2,\sigma}.$$
\end{proof}

The following version of the Sobolev inequality, a consequence of Theorem 1 in \cite{cantor}, ensures that control over certain Sobolev norms implies pointwise norm control. Let $|\cdot|$ be the pointwise norm induced by the Riemannian metric.

\begin{thm} \label{thm: Cantor Sobolev}(Cantor-Sobolev)
Suppose $M$ is a hyperbolic $n$-manifold with injectivity radius bounded below by $\e$. Let $r<\e$, and $l\geq 0,~k\geq 0$ be such that $l + n/2 < k$. Then if $\omega$ is in the Sobolev space $H^k_{\nabla} \Omega^{\bullet}(M)$, there is a constant $C = C(r)$ such that for every $p\in M$, $$|\nabla^l\omega(p)|\leq C||\omega||_{H_{\nabla}^k(B_r(p))}.$$
\end{thm}


This discussion provides us with the following upgraded version of Proposition \ref{prop: L2 comparisons}.

\begin{prop} \label{prop: smooth comparison}
Let $s\in\mathcal S_\e$ be the combinatorial 1-neighborhood of an $n$-simplex $\sigma \in\mathcal G_\e$. Let $\beta$ be the smooth barycentric partition of unity in Proposition \ref{prop: smooth L2 comparison} or the standard barycentric coordinate on $s$. Let $W_{\beta}(f)$ be the resulting generalized Whitney form associated to a cochain $f\in C^{\bullet}(\sigma;\R)$. Let $||\cdot||$ be some fixed norm on the real vector space $C^{\bullet}(\sigma;\R)$ and let $||\cdot||_{p,s}$ be the $p$-norm associated to $s$ on $\Omega^{\bullet}(s)$ and likewise for $||\cdot||_{p,\sigma}$, where $p = {\infty}$ or $p=2$.
Then there is a constant $\mathcal B = \mathcal B (\e,||\cdot||)>0$ (independent of $\sigma$ and $s$) such that $$\mathcal B ^{-1}|| W_{\beta}(f)||_{2,{\sigma}}\leq   ||f||\leq \mathcal B|| W_{\beta}(f)||_{2,\sigma}$$ and
$$\mathcal B^{-1}|| W_{\beta}(f) ||_{\infty, \sigma}\leq \mathcal B^{-1} || W_{\beta}(f)||_{\infty,s} \leq ||f||\leq \mathcal B|| W_{\beta}(f)||_{\infty,s}.$$
\end{prop}

\begin{proof}
    Identify $s$ with a domain $D$ in $\H^n$ and let $B$ be a ball of fixed radius $r = r(\e)$ that contains $s$ as in Lemma 2.11 and assume the basepoint of the ball is the point at which $||W_{\beta}(f)||_{\infty, s}$ is realized.

    From the Sobolev inequality, Propositions \ref{prop: L2 comparisons} and \ref{prop: smooth L2 comparison}, and Lemma \ref{lem: constant R}, for any cochain $f$,
        \begin{align*}
        ||W_{\beta}(f)||_{\infty,\sigma}&\leq ||W_{\beta}(f)||_{\infty,s}\\
        &\leq C||W_{\beta}(f)||_{H^n_{\nabla}(B)}\\
        &\leq CR||f||_{G,\sigma} \\
        &\leq \mathcal ACR||W_{\beta}(f)||_{2,\sigma}.
        \end{align*}

    Then the comparison $$||W_{\beta}(f)||_{2,\sigma}\leq ||W_{\beta}(f)||_{2,s}\leq\sqrt{\vol(s)} ||W_{\beta}(f)||_{\infty,s}$$ implies the smooth barycentric partition of unity induced $L^2$-norm $||W_{\beta}(f)||_{2,\sigma} $ and $L^{\infty}$-norm $||W_{\beta}(f)||_{\infty,s}$ are comparable for any simplex $\sigma\in\mathcal G_\e$ with combinatorial 1-neighborhood
    $s\in \mathcal S_\e$.

    The only constant appearing in the comparison depending on $s$ is $\sqrt{\vol(s)}$, which can be uniformly bounded by a constant depending only on $\e$ by compactness of $\mathcal S_\e$. As remarked after the proof of Proposition \ref{prop: smooth L2 comparison}, the $L^2$-norm version of the desired estimate follows from Proposition \ref{prop: L2 comparisons} and Proposition \ref{prop: smooth L2 comparison}. The above comparisons of the $L^2$-norm and the $L^{\infty}$-norm imply the claim.
\end{proof}
