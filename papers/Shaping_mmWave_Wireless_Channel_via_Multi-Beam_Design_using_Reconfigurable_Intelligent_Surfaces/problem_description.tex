\section{System model} 
\label{sec:desc}
\begin{figure}
    \centering
%    \includegraphics[width=\linewidth]{figures/RIS_Structure.jpg}
    \includegraphics[width=0.8\linewidth]{figures/RIS_Structure.eps}
    \caption{System Model}
    \label{fig:RIS}
\end{figure}


\subsection{Channel Model}
Consider a communication system consisting of a multi-antenna BS with $M_t$ antenna elements as a transmitter and a multi-antenna receiver with $M_r$ antenna elements. The MIMO system is aided by a multi-element RIS consisting of $M$ elements arranged in $M_h \times M_v$ grid in the form of UPA  as shown in figure~\ref{fig:system} where $M_h$ and $M_v$ are the number of elements in the horizontal and vertical directions, respectively. The received signal $\y \in \mathbb{C}^{M_r}$ as a function of the transmitted signal $\x \in \mathbb{C}^{M_t}$ can be written as,
\begin{align}
    \y = (\H_r\boldsymbol\Theta\H_t)\x + \z \label{channel}
\end{align}

\noindent where $\z$ is the noise vector, with each element of $\z$ drawn from a complex Gaussian distribution $\mathcal{C N}\left(0, \sigma_n^{2}\right)$, $\H_t \in \mathbb{C}^{M\times M_t}$ and $\H_r \in \mathbb{C}^{M_r\times M}$ are the channel matrices between each party and the RIS. We assume that the RIS consists of elements for which both the phase $\theta_m$ and the gain $\beta_m$ (in form of attenuation of the reflected signal) of each element, say $m$, may be controlled and $\boldsymbol\Theta \in \mathbb{C}^{M\times M}$ is a diagonal matrix where the element $(m,m)$ denotes the coefficient $\beta_m e^{j \theta_m}$ of the $m^{th}$ element of the RIS. Assuming LoS channel model both between the transmitter and the RIS and between the RIS and the receiver and using the directivity vectors at the transmitter, the RIS, and the receiver, the effective channel matrices can be written as,
% \begin{align}
%     & \H_t = \d_{M_t}\{\Omega_t\}\rho_{t}\d^{H}_{M}\{\Omega_{1}\}\label{channel_t} \\
%     & \H_r = \d_{M}\{\Omega_{2}\}\rho_{ r}\d^{H}_{M_r}\{\Omega_{r}\} \label{channel_r}
% \end{align}
\begin{align}
    & \H_r = \a_{M_r}(\Omega_r)\rho_{r}\a^{H}_{M}(\Omega_{2})\label{channel_r} \\
    & \H_t = \a_{M}(\Omega_{1})\rho_{t}\a^{H}_{M_t}(\Omega_{t}) \label{channel_t}
\end{align}

\noindent
% where $\Omega_t$ and $\Omega_2$ are the solid angle of departure (AoD) of the transmitted beams from transmitter and the RIS and $\Omega_1$ and $\Omega_r$ are the solid angle of arrival (AoA) of the received beams at the RIS and the receiver, respectively. 
where $\a_M(\Omega)$ is the array response vector of an RIS with elements in a UPA structure (RIS-UPA), $\Omega_t$ and $\Omega_2$ are the solid angles of departure (AoD) of the transmitted beams from transmitter and the RIS and $\Omega_1$ and $\Omega_r$ are the solid angle of arrival (AoA) of the received beams at the RIS and the receiver, respectively. 
% The directivity RIS-UPA can be found in similar way to that of a UPA. For a solid angle $\Omega = (\phi, \theta)$ where $-\frac{\pi}{2} \leq \phi \leq \frac{\pi}{2}$ is the elevation angle and $-\pi \leq \theta \leq \pi$ is the azimuth angle, the gain of a UPA with $M$ elements is given by
% \begin{align}
%     G (\c, \Omega) = \left| \sum_{m=0}^{M_t-1} c_{m} e^{j \frac{2 \pi}{\lambda}[ \cos \phi \cos\theta, \cos\phi \sin \theta, \sin\phi]  \r_m} \right|^2  \label{init_gain} 
% \end{align}
% where the element $m$ is located at $\r_m$ and is excited by coefficients $c_m$. Hence, The directivity vector for an UPA at the solid angle $\Omega$ is defined as
% \begin{align}
%     \a_{M}( \Omega ) = \left[1, e^{j \frac{2 \pi}{\lambda}[ \cos \phi \cos\theta, \cos\phi \sin \theta, \sin\phi]  \r_1}, \ldots, \right. \nonumber \\
%     \left. e^{j \frac{2 \pi}{\lambda}[ \cos \phi \cos\theta, \cos\phi \sin \theta, \sin\phi]  \r_{M-1}} \right]^{T} \in \mathbb{C}^{M}
% \end{align}
% \amir{Please note that an array response vector has similar gain in the same direction in three dimensional space. However, the receive and transmit directions differ by $\pi$, hence, the array response vector for the transmitter and receiver are written as $\a^{H}_{M}(\Omega)$, and $\a_{M}(\Omega)$, respectively.  
% }
The gain of the LoS paths from the transmitter to the RIS and from the RIS to the receiver are denoted by $\rho_t$ and $\rho_r$, respectively. Note that the solid angle $\Omega_a$ specifies a pair of elevation and azimuth angles i.e. $(\phi_a, \theta_a)$, $a \in \{1,2, t, r\}$. Further, assuming no pairing between the RIS elements, $\boldsymbol\Theta$ will be a diagonal matrix specified as 
\begin{equation}
    \boldsymbol\Theta = \mbox{diag}\{[\beta_1 e^{j \theta_1}, \ldots, \beta_M e^{j \theta_M}]\}
\end{equation}

\noindent where $\beta_i \in [0,1]$ and $\theta_i \in [0, 2\pi]$.


%where $\Omega_t$, $\Omega_1$, $\Omega_2$, and $\Omega_r$ are the angle between the direction of the transmitted or received beam with respect to their respective antenna array or RIS. 



% Define 
% \begin{equation}
%     \h_i = [1, \alpha, \alpha^2, \ldots, \alpha^{(M-1)}]^T, \alpha = e^{\frac{-j2 \pi d \phi_i}{\lambda}}, i = 1,2
% \end{equation}
% \begin{equation}
%     \h_G = [1, \alpha, \alpha^2, \ldots, \alpha^{(M_G-1)}]^T, \alpha = e^{\frac{-j2 \pi d_G \phi_G}{\lambda}}
% \end{equation}
% \begin{equation}
%     \h_U = [1, \alpha, \alpha^2, \ldots, \alpha^{(M_U-1)}]^T, \alpha = e^{\frac{-j2 \pi d_U \phi_U}{\lambda}}
% \end{equation}
% \begin{equation}
%     \Theta = \mbox{diag}\{[\beta_1 e^{j \theta_1}, \ldots, \beta_M e^{j \theta_M}]\}
% \end{equation}


% The received signal $\y$ at G as a function of the transmitted signal $\x$ from U is given by

% \begin{equation}
%     \y = \h_G  \rho_G \h_2^H \Theta \h_1  \rho_U \h_U^H
%  %       \y = \h_G \otimes \rho_G \h_2^H \Theta \h_1 \otimes \rho_U \h_U^H
% \end{equation}



\subsection{RIS Model}

Suppose an RIS consisting of $M_h \times M_v$ antenna elements forming a UPA structure is placed at the $x$-$z$ plane, where $M = M_h M_v$ and $z$ axis corresponds to horizon. Let $d_z$, and $d_x$ denote the distance between the antennas elements in $z$ and $x$ axis, respectively. 
The directivity of a RIS-UPA can be found in similar way to that of a UPA. At a solid angle $\Omega = (\phi, \theta)$, we have,
\begin{align}
    \a_{M}( \Omega ) = \left[1, e^{j \frac{2 \pi}{\lambda}\r_{\Omega}  \r_1}, \ldots,  e^{j \frac{2 \pi}{\lambda}\r_{\Omega}  \r_{M-1}} \right]^{T} \in \mathbb{C}^{M} \label{directivity_first}
\end{align}
\noindent where respectively,  $\r_{\Omega} = [ \cos \phi \cos\theta, \cos\phi \sin \theta, \sin\phi]$, and $\r_m = (m_hd_x, 0, m_vd_z)$  denote the direction corresponding to the solid angle $\Omega$ and the location of the $m$-th RIS element corresponding to the antenna placed at the position $(m_v, m_h)$. Further, we define a transformation of variables as follows. For a solid angle $\Omega = [\phi, \theta]$, define $\psi = [\xi, \zeta]$ as follows,
\begin{align}
    \xi=\frac{2 \pi d_{z}}{\lambda} \sin \phi \text {,  }\quad \zeta=\frac{2 \pi d_{x}}{\lambda} \sin \theta \cos \phi \label{transformation}
\end{align} Introducing the new variables into equation~\eqref{directivity_first}, it is straightforward to write, 
\begin{align}
    \a_M(\Omega) = \mathbf{d}_{M}\left(\xi, \zeta\right) =
    \mathbf{d}_{M_{v}}\left(\xi\right) \otimes
    \mathbf{d}_{M_{h}}\left(\zeta\right)  \in \mathbb{C}^{M}
\end{align}

where we define for $a \in \{v,h\} $ the  directivity vectors $\d_{M_a}$ as follows, and denote by $\d_M$ the directivity vector corresponding to the RIS. 
% \begin{align}
% \mathbf{d}_{M_{a}}\left(\psi_{a}\right) = \left[1, e^{j \psi_{a}} \cdots e^{j\left(M_{a}-1\right) \psi_{a}}\right]^{T} \in \mathbb{C}^{M_{a}}
% \end{align}
\begin{align}
&\mathbf{d}_{M_{v}}\left(\xi\right) = \left[1, e^{j \xi} \cdots e^{j\left(M_{v}-1\right) \xi}\right]^{T} \in \mathbb{C}^{M_{v}}\nonumber\\
&\mathbf{d}_{M_{h}}\left(\zeta\right) = \left[1, e^{j \zeta} \cdots e^{j\left(M_{h}-1\right) \zeta}\right]^{T} \in \mathbb{C}^{M_{h}}\label{directivity_final}
\end{align}

% The array response vector, i.e., directivity vector, of the RIS for a solid angle $\Omega$, i.e., $\d_M\{\Omega\}$ as a function of $\xi$ and $\zeta$ is denoted by $\mathbf{d}_{M}\left(\xi, \zeta\right)$ and is given by 

% \begin{align}
%     \mathbf{d}_{M}\left(\xi, \zeta\right) =
%     \mathbf{d}_{M_{v}}\left(\xi\right) \otimes
%     \mathbf{d}_{M_{h}}\left(\zeta\right)  \in \mathbb{C}^{M}
% \end{align}
% where, $\zeta=\frac{2 \pi d_{x}}{\lambda} \sin \theta \cos \phi \text { and } \xi=\frac{2 \pi d_{z}}{\lambda} \sin \phi$ for a solid angle $\Omega = (\phi, \psi)$. 

Let $\mathcal{B}$ be the angular range under cover defined as

\begin{equation}
    \mathcal{B} \doteq \left[-\phi^{\mathrm{B}}, \phi^{\mathrm{B}}\right)
    \times \left[-\theta^{\mathrm{B}}, \theta^{\mathrm{B}}\right) \label{range_angle}
\end{equation}

We note that there is a one-to-one correspondence between the solid angle $\Omega = (\phi, \psi)$ and its representation after change of variable as $(\zeta, \xi)$. Accordingly, let $\mathcal{B}^{\psi}$ be the angular range under cover in the $(\zeta, \xi)$ domain given by 

\begin{equation}
    \mathcal{B}^\psi \doteq\left[-\xi^{\mathrm{B}}, \xi^{\mathrm{B}}\right) \times\left[-\zeta^{\mathrm{B}}, \zeta^{\mathrm{B}}\right)
\end{equation}
In this paper, we set $d_x = d_z = \frac{\lambda}{2}$, $\phi^{\mathrm{B}} = \frac{\pi}{4}$, and $\theta^{\mathrm{B}} = \frac{\pi}{2}$, hence $\xi \in [-\pi\frac{\sqrt{2}}{2}, \pi\frac{\sqrt{2}}{2})$, and $\zeta \in [-\pi, \pi)$. Note that, the dependence between variables $\xi$ and $\zeta$ can be resolved using the approximation in \cite{Song17}.
Let us uniformly divide $\mathcal{B}^{\psi}$  into $Q=Q_{v} Q_{h}$ subregions, where $Q_h$ and $Q_v$ are the number of division in horizontal and vertical directions, respectively. A subregion is denoted by  
$$ \mathcal{B}^{\psi}_{ p, q} \doteq \nu_{v}^{p, q} \times \nu_{h}^ {p, q}$$
\noindent where $\nu_{v}^{p} = [\xi^{p-1}, \xi^{p}]$, and $\nu_{h}^{q} = [\zeta^{q-1}, \zeta^{q}]$ defining,
\begin{align}
    & \xi^{p} = -\xi^{\mathrm{B}} + p\delta_v, \quad \zeta^{q} = -\zeta^{\mathrm{B}} + q\delta_h
\end{align}
%Note that, the dependence between variables $\xi$ and $\zeta$ can be resolved using the approximation in \cite{Song17}. 
\noindent with $\delta_v = \frac{2\xi^{\mathrm{B}}}{Q_v}$, and $\delta_h = \frac{2\zeta^{\mathrm{B}}}{Q_h}$. In the next section, we define the multi-beamforming design problem as the core of our proposed RIS structure. 


% One can define for the solid angle $\Omega = (\phi, \psi)$,

% \begin{align}
% \mathbf{d}_{M_{a}}\left(\psi_{a}\right) = \left[1, e^{j \psi_{a}} \cdots e^{j\left(M_{a}-1\right) \psi_{a}}\right]^{T} \in \mathbb{C}^{M_{a}}
% \end{align}
% where, $\zeta=\frac{2 \pi d_{x}}{\lambda} \sin \theta \cos \phi \text { and } \xi=\frac{2 \pi d_{z}}{\lambda} \sin \phi$, and $a\in\{v, h\}$.  
% The array response vector is then defined as 

% \begin{align}
%     \mathbf{d}_{M}\left(\xi, \zeta\right) =
%     \mathbf{d}_{M_{v}}\left(\xi\right) \otimes
%     \mathbf{d}_{M_{h}}\left(\zeta\right)  \in \mathbb{C}^{M}
% \end{align}
