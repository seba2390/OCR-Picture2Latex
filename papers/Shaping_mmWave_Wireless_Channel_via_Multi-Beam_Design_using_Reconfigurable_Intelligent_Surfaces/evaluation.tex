\section{Performance Evaluation}
\label{sec:evaluation}


\begin{figure*} 
    \centering
    % \subfloat[3D UPA Pattern \label{one-sided32}]{%4
    %         \includegraphics[width=0.26\linewidth]{figures/UPA_3D_24M = 1024.png}}
        \subfloat[Dual-beam 3D UPA Pattern \label{one-sided64}]{%4
            \includegraphics[width=0.26\linewidth]{figures/UPA_3D_2M_1024.eps}}
        \subfloat[UPA Pattern Cut at $\phi_c$ \label{TULA_H_64}]{%10
            \includegraphics[width=0.25\linewidth]{figures/crossed_phi.eps}}
        \subfloat[UPA Pattern Cut at $\theta_c$ \label{fig:1024omp_comp}]{
        \includegraphics[width=0.25\linewidth]{figures/crossed_theta.eps}}
        \subfloat[3D UPA Pattern \label{one-sided32}]{%4
            \includegraphics[width=0.26\linewidth]{figures/UPA_3D_24M_1024.eps}}
  \caption{Beam pattern for UPA structure}
  \label{fig:UPA_pattern} 
\end{figure*}


% \begin{figure} 
%     \centering
%     \subfloat[3D UPA Pattern \label{3d_one-sided32}]{%4
%             \includegraphics[width=0.49\linewidth]{figures/UPA_3DM = 1024_Abig.png}}
%         \subfloat[Dual-beam 3D UPA Pattern \label{3d_one-sided64}]{%4
%             \includegraphics[width=0.49\linewidth]{figures/UPA_3DM = 1024.png}}
%   \caption{3D Beam pattern for UPA structure}
%   \label{fig:UPA_pattern_3D} 
% \end{figure}



In this section, we evaluate the performance of our multi-beam design framework. We aim to design a dual-beam  which comprises of two lobes with centers at $(-8\pi/32, -5 \pi /32)$ and $(7\pi/32, \pi /32)$ in the $(\phi, \theta)$ domain, and with the beam-width equal to $\pi /16$. We divide both the $\psi_h$, and the $\psi_v$ range uniformly into $Q_h = 16$, and $Q_v = 16$ regions resulting in $Q = 256$ equally-shaped units in $(\psi_v, \psi_h)$ domain. We cover each desired beam with the smallest number of the designed units to provide uniform gain at the desired angular regions.
Figures~\ref{fig:UPA_pattern}(a)-(c) depict the beam pattern of the dual beam obtained through our design where all angles are measured in radians. 
Figure~\ref{fig:UPA_pattern}(a), shows the heat-map corresponding to the gain of the reflected beam from RIS for the designed dual-beam. The gains are computed in dB. It can be seen that the designed beamformer generates two disjoint beams with an almost uniform gain over the desired ACIs. It is also observed that the beams sharply drop outside the desired ACIs and effectively suppress the gain everywhere outside ACI. In order to quantify the suppression we depict the cross-section of the gain pattern at a fixed elevation angle $\phi_c$ for two values of $\phi_c \in \{-8\pi/32, 7\pi/32\}$ located inside the two lobes of the designed dual beam in Figure~\ref{fig:UPA_pattern}(b). Similarly, Figure~\ref{fig:UPA_pattern}(c) shows the cross-section of the beam pattern at a fixed azimuth angle $\theta_c$ for two values of $\theta_c \in \{ -5 \pi /32, \pi /32\}$. Both Figures~\ref{fig:UPA_pattern}(b)~and~(c) confirm the sharpness of both lobes of the designed dual-beam and can be used to find the beamwidth of each lobes at an arbitrary fraction from its maximum values, e.g,, the 3dB beamwidth or 10dB beamwidth. Indeed, there is negligible difference between 3dB and 10dB beamwidth which clarifies the sharpness of the beams. From Figures~\ref{fig:UPA_pattern}(b)~and(c), it is also observed that the gain within the ACI is almost uniform. Nonetheless, we should emphasize the fact that the shape of the lobes of the beam that are centered at different solid angle may suffer from slight deformation as seen by Figure~\ref{fig:UPA_pattern}(a). This phenomenon worsens as the corresponding lobes of the beams get too close to the plane of the RIS. 
% Fig. ~\ref{TULA_H_64} depicts a section of the gain pattern crossed at a fixed elevation angle $\phi_c$ for two values of $\phi_c$ each located inside one of the ACI's. Fig. ~\ref{fig:1024omp_comp} shows similar profile when the beam pattern is cut at a given $\theta_c$ for two values of $\theta_c$. In agreement with the 3D beam pattern in fig. \ref{one-sided64}, it is observed that our designed beamformer is able to support the desired beams with uniform gain.

% Fig.~\ref{3d_one-sided32} shows the 3D beam pattern of the RIS designed for a single beam containing  both the desired ACI's. It is observed that our design uniformly covers the designated single beam with a high sharp gain.  Fig. ~\ref{3d_one-sided64} shows the 3D reference gain profile over the angular space when the beamformer is designed to only cover the desired dual-beam and suppress everywhere else.
Finally, in order to compare the performance of our multi-beam design to a single-beam design, we consider a beam with single lobe which is capable of covering the same two regions as in the dual-beam design. Figure~\ref{fig:UPA_pattern}(d), shows the heat-map corresponding to the gain of the reflected beam from RIS for the corresponding single beam that is optimized based on our design. As it was the case for multi-beam, this figure also shows that for a single beam our design generates an almost uniform and fairly sharp beam. However, comparing Figures~\ref{fig:UPA_pattern}(a)~and(d), we observe that in the desired ACI, the multi-beamforming procedure, enhances the gain by about $20$ dB over the beams with optimized single lobe.  















