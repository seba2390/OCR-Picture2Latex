\section{Introduction}


Next generation of wireless communication systems aims to address the ever-increasing demand for high throughput, low latency, better quality of service and ubiquitous coverage. The abundance of bandwidth available at the mmWave frequency range, i.e., $[20, 100]$ Ghz, is considered as a key enabler towards the realization of the promises of next generation wireless communication systems. However, communication in mmWave suffers from high path-loss, and poor scattering. Since the channel in mmWave is mostly LoS, i.e., a strong LoS path and very few and much weaker secondary components, the mmWave coverage map includes \emph{blind spots} as a result of shadowing and blockage. Beamforming is primarily used to address the high attenuation in the channel. In addition to beamforming, relaying can potentially be designed to generate constructive superposition and enhance the received signals at the receiving nodes. 
% Importance of mmWave in 5G and beyond systems.
% Channel in mmWave is mostly LOS. Blockage and shadowing generate blind spots.
% High attentuation in mmwave should be addressed. In addition to beamforming, relaying can generate constructive superposition and enhance the received signals for the users.
% Contyributions are as follows.
Reconfigurable intelligent surface (RIS)\cite{Huang19}\cite{Liaskos18}\cite{Basar19} is a new paradigm with a great potential for stretching the coverage and enhancing the capacity of next-generation communication systems. Indeed, it is possible to shape the wireless channel by using RIS, e.g., by covering blind spots or providing diversity reception at a receiving node. In particular, passive RIS provide not only an energy-efficient solution but also a cost-effective one both in terms of the initial deployment cost and the operational costs. RIS are promising to be deployed in a wide range of communications scenarios and use-cases, such as high throughput MIMO communications\cite{Huang20}\cite{Nadeem20}, ad-hoc networks, e.g., UAV communications\cite{Li20}, physical layer security\cite{Maka20}, etc. % A comprehensive survey is conducted in \cite{Liu21} that \amir{enumerates} the fundamentals, opportunities and challenges of integrating RIS into wireless communication environments. 
Apart from the works focusing on theoretical performance analysis of RIS-enabled systems  \cite{Han19}\cite{Nadeem20}\cite{Jung19}, considerable amount of work has been dedicated to optimizing such an integration, mostly focusing on the phase optimization of RIS elements \cite{Abey20}\cite{Guo20}\cite{Di20}\cite{Ata20} to achieve various goals such as maximum received signal strength, maximum spectral efficiency,  etc. For more information on RIS, we refer the interested readers to \cite{Liu21} and the references therein.


% Importance of RIS in shaping the wireless channel, generating opportunities for coherent combining through relay, enhancing the coverage by acting like a deformable mirror to cover blind spots.

% In this work we provide novel approach to design RIS such that we can simultaneously optimize the received power at RIS and the propagated (reflected) signal off of RIS in arbitrary 3D (Asimuth and elevation) angles. The dual-beam design allows for optimizing both receive and transmit direction. Moreover, it allows to control the beamwidth and hence the gain of the receive and transmit beamforming vectors. The work can be easily extended to cover multiple transmit and receive directions in multi-beam instead of dual-beam.

\begin{figure}
    \centering
    \includegraphics[width=0.75\linewidth]{figures/scene2.eps}
    \caption{Filling the coverage gap in mmWave communications by utilizing Reconfigurable Intelligent Surfaces enabled by multi-beamforming }
    \label{fig:system}
\end{figure}

In this paper, we consider a communication scenario between a transmitter, e.g., the base station (BS), and terrestrial end-users through a passive RIS that reflects the received signal from the transmitter towards the users. Hence, the users that are otherwise in blind spots of network coverage, become capable of communicating with the base station through the RIS that is serving as a passive reflector (passive relay) maintaining  communication links to the BS and to the users. Given the geo-spatial variance among the locations of the end-users served by the same wireless system, the RIS may have to accommodate users that lie in distant angular intervals simultaneously, with satisfactory quality of service (QoS). In what we refer to as \emph{multi-beamforming}, we particularly address the design of beams consisting of multiple disjoint lobes using RIS in order to cover different blind spots using sharp and effective beam patterns. In the following, we summaries the main contributions of this paper: 
%The contributions of this paper are as follows:
\begin{itemize} 
    \item We design the parameters of the RIS to achieve multiple disjoint beams covering various ranges of solid angle. The designed beams are fairly sharp, have almost uniform gain in the desired angular coverage interval (ACI), and have negligible power transmitted outside the ACI.
    \item We formulate the multi-beamforming design as an optimization problem for which we derive the optimal solution.
    \item Thanks to the derived analytical closed form solutions for the optimal multi-beamforming design, the proposed solution bears very low computational complexity even for RIS with massive array size.
%    \amir{\item (iv) The proposed design can be used with arrays that are passive (whether it is only phased-controlled or both phase-and gain controlled) as well as RIS with active elements (that are capable of amplifying the reflected signal).}
%    \item (v) We identify the fact that if a passive RIS is capable of controlling the gain of its element (e.g., through attenuation), it can provide smoother gain in the desired ACI and also boost up the beamforming gain. The latter could be counter intuitive as the power radiated from every single element is not maximized due to attenuation, but in fact, to shape the beams it is essential to use controlled attenuation for signal reflected from different array elements of RIS in the superposition of the signals emitted by each RIS elements.
%    \item (vi) The multi-beamforming design inherently depends on the solid angle (say $\Omega_1$ in Fig.~\ref{fig:RIS}) at which the incident wave activates the RIS elements. The proposed beamforming design easily adapts to changes in $\Omega_1$ and we provide a visualization as how the beam would change in response to change in $\Omega_1$.
    \item Through numerical evaluation we show that by using passive RIS, multi-beamforming can simultaneously cover multiple ACIs. Moreover, multi-beamforming provides tens of dB power boost w.r.t. single-beam RIS design even when the single beam is designed optimally. 
\end{itemize}

\textbf{Notation} Throughout this paper, $\mathbb{C}$, $\mathbb{R}$, and $\mathbb{Z}$ denote the set of complex, real, and integer numbers, respectively,  $\mathcal{C N}\left(m, \sigma^{2}\right)$ denotes the circularly symmetric complex normal distribution with mean $m$ and variance $\sigma^{2}$, $[a, b]$ is the closed interval between $a$ and $b, \mathbf{1}_{a, b}$ is the $a \times b$ all ones matrix, $\mathbf{I}_{N}$ is the $N \times N$ identity matrix, $\mathds{1}_{[a, b)}$ is the indicator function, $\|\cdot\|$ is the $2$ -norm, $\|\cdot\|_{\infty}$ is the infinity-norm, $|\cdot|$ may denote cardinality if applied to a set and $1$-norm if applied to a vector, $\odot$ is the Hadamard product, $\otimes$ is the Kronecker product, $\mathbf{A}^{H},$ and $\mathbf{A}_{a, b}$ denote conjugate transpose, and $(a, b)^{t h}$ entry of $\A$ respectively.





The remainder of the paper is organized as follows. Section~\ref{sec:desc} describes the system model. In Section~\ref{sec:problem} we formulate the multi-beamforming design problem and propose our solutions in Section ~\ref{sec:proposed}. Section ~\ref{sec:evaluation} presents our evaluation results, and finally, we conclude in Section ~\ref{sec:conclusions}. %, we highlight our conclusions and discuss directions for future 


%In dual beam design, we jointly optimize a receive beam steering (beamforming) covering the 3D angular location of the base stations and a transmit beam steering covering the 3D angular location of the user (users) of interest.

%Shaping wireless channel: through relaying or covering the blind spots.

