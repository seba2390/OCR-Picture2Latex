\section{Problem Formulation}
\label{sec:problem}

% \subsection{Composite Codebook Design Problem}

Let us define a \textit{composite beam} $\omega_k$ as a union of multiple disjoint, possibly non-neighboring beams $\nu_{q} \text { for } q \in \mathcal{Q} = \left\{1, \cdots, Q\right\}$. Let $\mathcal{C'}=\left\{\mathbf{c}_{1}, \cdots, \mathbf{c}_{K}\right\}$ be the codebook corresponding to the composite problem.

% $$\bigcup_{k = 1}^{2^{B'}}{\omega_k} = [-\pi, \pi], \quad \text{and} \quad \omega_k \cap \omega_l = \emptyset, \quad \forall k\neq l.$$

Moreover, define for each composite beam $\omega_k$ the set $\mathcal{W}_k \subseteq \mathcal{Q}$ to be the set of indices of the single beams that form $\omega_k$. i.e. $\mathcal{W}_k = \{q \in \mathcal{Q}: v_q \subseteq \omega_k\}$. 
For any codeword $\c$, it holds that, 

\begin{equation}
    \int_{-\pi}^{\pi} G(\psi, \mathbf{c}) d \psi=2 \pi\|\mathbf{c}\|^{2}=2 \pi
\end{equation}
We have then for the ideal gain corresponding to  each codeword $\textbf{c}_k$, 

\begin{align}
&\int_{-\pi}^{\pi} G_{\text {ideal }, k}(\psi) d \psi =\int_{\omega_{k}} t d \psi+\int_{[-\pi, \pi] \backslash \omega_{k}} 0 d \psi \nonumber\\
&= \sum_{q \in {\mathcal{W}_k}}{\int_{\nu_{q}} t d \psi} = \sum_{q \in \mathcal{W}_k}\delta_q t=2 \pi \label{composite}
\end{align}

Therefore, $t = \frac{2 \pi}{\Delta_k}$, where $\Delta_k = \sum_{q \in \mathcal{W}_k}\delta_q$. It follows that, 

\begin{equation}
    G_{\text {ideal }, k}(\psi)=\frac{2 \pi}{\Delta_k} \mathds{1}_{\omega_{k}}(\psi), \quad \psi \in[-\pi, \pi] \label{composite_ideal_gain}
\end{equation}

The new MSE problem for the $k$-th codeword can therefore be written as follows.



\begin{align}
\c_k^{opt} &=\underset{\c, \|\c\|=1}{\arg \min } \int_{-\pi}^{\pi}\left|G_{\text {ideal }, k}(\psi)-G(\psi, \c)\right| d \psi 
\end{align}




By uniformly sampling on the range of   we can rewrite
the optimization problem as follows,

\begin{align}
    \c_k^{opt} &= \nonumber \\&\underset{\c, \|\c\|=1}{\arg \min }\left[\lim _{L \rightarrow \infty} \sum_{p=1}^{Q} \delta_p\sum_{\ell=1}^{L} \frac{\left|G_{\text {ideal }, k}\left(\psi_{p, \ell}\right)-G\left(\psi_{p, \ell}, \c\right)\right|}{L}\right]\label{composite_MSE}
\end{align}

 where, for $q = 1 \ldots Q$, 
 $$\delta_q = \psi_{q}-\psi_{q-1}, \quad \psi_{q, \ell}=\psi_{q-1}+\frac{\delta_q(\ell-0.5)}{L} $$

We can rewrite equation \eqref{composite_MSE} as follows, 

\begin{align}
 &\c_k^{opt} = \underset{\c,  \|\c\|=1}{\arg \min } \lim_{L\rightarrow \infty}\frac{1}{L}\left\|\mathbf{G}_{\text {ideal }, k}-\mathbf{G}(\c)\right\| \label{init_opt}
 \end{align}
 
 where,  $$\mathbf{G}(\c)=\left[{\delta_1} G\left(\psi_{1,1}, \c\right) \ldots {\delta_{Q}}G\left(\psi_{Q, L}, \c\right)\right]^{T} \in \mathbb{Z}^{L Q}$$

and, 
$$ \mathbf{G}_{\text {ideal }, k}=\left[{\delta_1} G_{\text {ideal }, k}\left(\psi_{1,1}\right) \ldots {\delta_{Q}}G_{\text {ideal }, k}\left(\psi_{Q, L}\right)\right]^{T} \in \mathbb{Z}^{L Q}$$


Note that it holds that 

\begin{equation}
    \mathbf{G}_{\text {ideal },k}=\sum_{q \in \mathcal{W}_k}\delta_q\frac{2\pi}{\Delta_k}\left(\mathbf{e}_{q} \otimes \mathbf{1}_{L, 1}\right) \label{ideal}
\end{equation}

with $\mathbf{e}_{q} \in \mathbb{Z}^{Q}$ being the standard basis vector for the $q$-th axis among $Q$ ones. Now, note that $\mathbf{1}_{L, 1}=\mathbf{g} \odot \mathbf{g}^{*}$ for any equal gain $\mathbf{g} \in \mathbb{C}^L$. Therefore, for a suitable choice of $\g$ we can write: 

% $$\mathcal{G}_{L}=\left\{\mathbf{g} \in \mathbb{C}^{L}:\left(\mathbf{g} \mathbf{g}^{H}\right)_{\ell, \ell}=1,1 \leq \ell \leq L\right\}$$


\begin{align}
\mathbf{G}_{\text {ideal }, k} &= \sum_{q \in \mathcal{W}_k}\gamma_q\left(\mathbf{e}_{q} \otimes\left(\mathbf{g} \odot \mathbf{g}^{*}\right)\right) \nonumber\\
&=\sum_{q \in \mathcal{W}_k}\left(\sqrt{\gamma_q}\left(\mathbf{e}_{q} \otimes \mathbf{g}\right)\right) \odot\left(\sqrt{{\gamma_q}}\left(\mathbf{e}_{q} \otimes \mathbf{g}\right)\right)^{*} \nonumber\\
&=\left(\sum_{q \in \mathcal{W}_k}\sqrt{\gamma_q}\left(\mathbf{e}_{q} \otimes \mathbf{g}_q\right)\right)  \odot \left(\sum_{q \in \mathcal{W}_k}\sqrt{\gamma_q}\left(\mathbf{e}_{q} \otimes \mathbf{g}_q\right)\right)^{*} \label{final_gik}
\end{align}




















% \begin{align}
% \left(\mathbf{F}_{k}, \mathbf{v}_{k}\right) &=\underset{\mathbf{F}, \mathbf{v}}{\arg \min } \sum_{p=1}^{Q} \sum_{\ell=1}^{L}\left|G_{\text {ideal }, k}\left(\psi_{p, \ell}\right)-G\left(\psi_{p, \ell}, \mathbf{F} \mathbf{v}\right)\right|^{2} \nonumber\\
% &=\underset{\mathbf{F}, \mathbf{v}}{\arg \min }\lim_{L\longrightarrow \infty} \frac{1}{L}\left\|\mathbf{G}_{\text {ideal }, k}-\mathbf{G}(\mathbf{F} \mathbf{v})\right\|_{2}^{2} \label{init_opt}
% \end{align}

% where,  $$\mathbf{G}(\mathbf{F} \mathbf{v})=\left[G\left(\psi_{1,1}, \mathbf{F} \mathbf{v}\right) \cdots G\left(\psi_{Q, L}, \mathbf{F} \mathbf{v}\right)\right]^{T} \in \mathbb{Z}^{L Q}$$

% and, 
% $$ \mathbf{G}_{\text {ideal }, k}=\left[G_{\text {ideal }, k}\left(\psi_{1,1}\right) \cdots G_{\text {ideal }, k}\left(\psi_{Q, L}\right)\right]^{T} \in \mathbb{Z}^{L Q}$$

% Note that we can write using the definition of $\mathcal{W}_k$ that, 

% $$G_{\text {ideal }, k}\left(\psi_{p, \ell}\right)=\left\{\begin{array}{ll}
% \frac{Q}{|\mathcal{W}_k| M_{t}}, & p \in \mathcal{W}_k \\
% 0, & p \notin \mathcal{W}_k 
% \end{array},\right.$$

% Or in the matrix form, 

% \begin{align}
%     &\mathbf{G}_{\text {ideal }, k}=\frac{Q}{|\mathcal{W}_k| M_{t}}\left((\bigoplus_{q \in \mathcal{W}_k}{\mathbf{e}_{q}}) \otimes \mathbf{1}_{L, 1}\right)  \nonumber \\ 
%     & = \sum_{q \in \mathcal{W}_k}{\frac{Q}{|\mathcal{W}_k| M_{t}}\left(\mathbf{e}_{q} \otimes \mathbf{1}_{L, 1}\right)} \label{G_ideal_k}
% \end{align}

% with $\mathbf{e}_{q} \in \mathbb{Z}^{Q}$ being the standard basis vector for the $q$-th axis among $Q$ ones. Now, note that $\mathbf{1}_{L, 1}=\mathbf{g} \odot \mathbf{g}^{*}$ for any $\mathbf{g} \in \mathcal{G}_L$, where, 

% $$\mathcal{G}_{L}=\left\{\mathbf{g} \in \mathbb{C}^{L}:\left(\mathbf{g} \mathbf{g}^{H}\right)_{\ell, \ell}=1,1 \leq \ell \leq L\right\}$$



% Therefore, equation \eqref{G_ideal_k} can be rewritten as:
% \begin{align}
% &\mathbf{G}_{\text {ideal }, k} = \sum_{q \in \mathcal{W}_k}\frac{Q}{|\mathcal{W}_k| M_{t}}\left(\mathbf{e}_{q} \otimes\left(\mathbf{g}_q \odot \mathbf{g}_q^{*}\right)\right) \nonumber\\
% &=\sum_{q \in \mathcal{W}_k}\left(\sigma\left(\mathbf{e}_{q} \otimes \mathbf{g}_q\right)\right) \odot\left(\sigma\left(\mathbf{e}_{q} \otimes \mathbf{g}_q\right)\right)^{*} \nonumber\\
% &=\left(\sum_{q \in \mathcal{W}_k}\sigma\left(\mathbf{e}_{q} \otimes \mathbf{g}_q\right)\right)  \odot \left(\sum_{q \in \mathcal{W}_k}\sigma\left(\mathbf{e}_{q} \otimes \mathbf{g}_q\right)\right)^{*} \label{final_gik}
% \end{align}

% where $\sigma = \sqrt{\frac{Q}{|\mathcal{W}_k|M_{t}}}$. 

Similarly, it is straightforward to observe,


\begin{align}
\mathbf{G}(\c) &=\left(\mathbf{D}^{H} \c\right) \odot\left(\mathbf{D}^{H} \c\right)^{*} \label{dc}
\end{align}
where $\mathbf{D} =\left[\sqrt{\delta_1}\mathbf{D}_{1} \cdots \sqrt{\delta_{Q}}\mathbf{D}_{Q} \right] \in \mathbb{C}^{M_{t} \times L Q}$, and 
$\mathbf{D}_{q}=\left[\mathbf{d}_{M_{t}}\left(\psi_{q, 1}\right) \cdots \mathbf{d}_{M_{t}}\left(\psi_{q, L}\right)\right] \in \mathbb{C}^{M_{t} \times L}.$
 







% On the other hand note that $\mathbf{G}(\mathbf{F} \mathbf{v})$ can be written as,


% \begin{equation}
%     \mathbf{G}(\mathbf{F} \mathbf{v}) =\left(\mathbf{D}^{H} \mathbf{F} \mathbf{v}\right) \odot\left(\mathbf{D}^{H} \mathbf{F} \mathbf{v}\right)^{*}\label{dfv_decomp}
% \end{equation}

% where, 
% $$\mathbf{D} =\left[\mathbf{D}_{1} \cdots \mathbf{D}_{Q}\right] \in \mathbb{C}^{M_{t} \times L Q}$$

% and, $\mathbf{D}_{q}=\left[\mathbf{d}_{M_{t}}\left(\psi_{q, 1}\right) \cdots \mathbf{d}_{M_{t}}\left(\psi_{q, L}\right)\right] \in \mathbb{C}^{M_{t} \times L}$. 

Comparing the expressions \eqref{init_opt}, \eqref{final_gik}, and \eqref{dc}, one can show that the optimal choice of $\c_q$ in \eqref{init_opt} is the solution to the following optimization problem for a proper choice of $\g_q$. 
\begin{problem}
Given an equal-gain vector $\g_q \in \mathbb{C}^L$, $q \in \{1, \cdots, Q\}$, find vector $\c_q \in \mathbb{C}^{M_t}$ such that
\begin{align}
&\c_q=\underset{\c, \|c\|=1}{\arg \min } \lim_{L\rightarrow \infty} \left\|\sum_{q \in \mathcal{W}_k}\sqrt{\gamma_q}\left(\mathbf{e}_{q} \otimes \mathbf{g}_q\right)- \mathbf{D}^{H} \c\right\|^{2} \label{obj_func}
\end{align}
\label{main_problem}
\end{problem}


However, we now need to find the optimal choice of $\g_q$ that minimizes the objective in \eqref{init_opt}. Using \eqref{final_gik}, and \eqref{dc}, we have the following optimization problem.
\begin{problem}
Find a set of equal-gain $\g_q \in \mathbb{C}^L$, i.e. $\mathcal{G}_k$ such that
\begin{equation}
 \mathcal{G}_k = \underset{\mathcal{G}}{\arg\min }\left\| abs(\D^H \c_q)- abs(\sum_{q \in \mathcal{W}_k}\sqrt{\gamma_q}(\mathbf{e}_{q} \otimes \mathbf{g}_q))\right\|^{2} \label{g_final_eq}
\end{equation} 
where $\mathcal{G}_k = \{\g_q| q \in \mathcal{W}_k\}$, and $abs(.)$ denotes the element-wise absolute value of a vector.
\label{g_problem}
\end{problem}

Hence, the codebook design for a system with full-digital beamforming capability is found by solving Problem~\ref{main_problem} for proper choice of $\g_q$ obtained as a solution to Problem ~\ref{g_problem}. The codebook for Hybrid beamforming, is then found as
\begin{align}
    \underset{\mathbf{F}_{q},\mathbf{v}_{q}}{\arg\min } \|\mathbf{F}_{q}\mathbf{v}_{q} - \c_q\|^2
\end{align}
where the columns of $\F_q \in \mathbb{C}^{M_t \times N_{RF}}$ are equal-gain vectors and $\v_q \in \mathbb{C}^{N_{RF}}$. The solution may be obtained using simple, yet effective suboptimal algorithm such as orthogonal matching pursuit (OMP) \cite{love15}\cite{noh17}\cite{Hussain17}. In the next section we continue with the solution of Problem~\ref{main_problem}, and~\ref{g_problem}.





% Comparing the expressions in equations \eqref{final_gik}, and, \eqref{dc}, a perfect solution to the optimization problem in \eqref{init_opt}, could be a pair $\textbf{(F, v)}$, for which the following equality holds.  



% \begin{align}
%     \mathbf{D}^{H} \mathbf{F} \mathbf{v}=\sum_{q \in \mathcal{W}_k}\sigma\left(\mathbf{e}_{q} \otimes \mathbf{g}_q\right) \label{potential_answer}
% \end{align}

% However, we have that $M_t > N_{RF}$ and therefore, the matrix $\textbf{D}^H \textbf{F}$ may not be full-rank, resulting in the RHS of equation \eqref{potential_answer} being in the null space of LHS. Therefore, given some $\g_q \in \mathcal{G}_L$ we formulate the following two-step optimization problem to search for $\textbf{(F, v)}$ that gets as close as possible to the desired value. 

% \begin{problem}

% a) For all $q \in \{1, \cdots, Q\}$, given $\g_q \in \mathcal{G}_L$, find the unit-norm vector $\c_q \in \mathbb{C}^{M_t}$ such that
% \begin{align}
% &\c^{|\g_q}_q=\underset{\c}{\arg \min } \lim_{L\longrightarrow \infty} \frac{1}{L}\left\|\left(\sum_{q \in \mathcal{W}_k}\sigma\left(\mathbf{e}_{q} \otimes \mathbf{g}_q\right)\right)- \mathbf{D}^{H} \c\right\|_{2}^{2} \label{obj_func}
% \end{align}

% \noindent
% b) For every given $\c_q$, find $\F_q \in \mathbb{C}^{M_t \times N_{RF}}$ , and $\v_q \in \mathbb{C}^{N_RF}$ that satisfy $\mathbf{F}_{q}\mathbf{v}_{q} = \c_q$ the condition $\f_n \in \mathcal{B}_{M_t}$ as in equation \eqref{f_constant_gain}. 

% \begin{align}
%     \mathbf{F}_{\mid \g_q}\mathbf{v}_{\mid \g_q} = \c_q
% \end{align}

% \end{problem}



Note that the solution to problem \ref{main_problem} is the limit of the sequence of solutions to a least-square optimization problem as $L$ goes to infinity. For each $L$ we find that,
 \begin{align}
& {\c}^{(L)}_k = \sum_{q \in \mathcal{W}_k}\sqrt{\gamma_q}(\D \D^H)^{-1} \D  \left(\mathbf{e}_q \otimes \mathbf{g}_q\right) \\
& {\c}^{(L)}_k = \sum_{q \in \mathcal{W}_k}\sigma_q \D_q\g_q \label{c_final_eq}
\end{align}
where $\sigma_q = \frac{\sqrt{ 2\pi \delta_q \frac{\delta_q}{\Delta_k}}}{L \sum_{p=1}^{2^B}\delta_p} = \frac{\delta_q}{L\sqrt{2\pi\Delta_k}}$, noting that it holds that, 

% $$\mathbf{D D}^{H}=\left({L \sum_{p=1}^{2^B}\delta_p}\right) \mathbf{I}_{M_{t}}$$.
% % Dividing by $\|\Tilde{\c}^{(L)}_q\|$ and taking the limit as L goes to infinity we will find the optimal $\c_q$. i.e. 


Using \eqref{c_final_eq}, equation \eqref{g_final_eq} can now be rewritten as, 

\begin{align}
 &\mathcal{G}_k = \underset{\mathcal{G}}{\arg\min } \nonumber
 \\&\left\| abs(\D^H \D\sum_{q \in \mathcal{W}_k}\sigma_q(\mathbf{e}_{q} \otimes \mathbf{g}_q))-abs(\sum_{q \in \mathcal{W}_k}\sqrt{\gamma_q}(\mathbf{e}_{q} \otimes \mathbf{g}_q))\right\|^{2} \label{g_simple_final_eq}
\end{align} 

% where $\sigma' = \frac{1}{2\pi L}$. We will state the following theorem without formal proof.
\begin{proposition}
The maximizer of \eqref{g_simple_final_eq} is in the form $\g_q = \left[{\begin{array}{cccc} 1& \alpha^\eta &\cdots & \alpha^{\eta (L -1)} \end{array}}\right]^T$ for some $\eta$ where $\alpha = e^{j(\frac{\delta_q}{L})}$. \label{proposiiton_g}
\end{proposition}

An analytical closed form solution for $\c_q$ can be found as follows. We have, 
 \begin{align}
     {\c_k}^{(L)} & = \sum_{q \in \mathcal{W}_k} \sigma_q \left(\sum_{l=1}^{L}g_{q, l}\mathbf{d}_{M_t}(\psi_{q, l})\right)  \nonumber\\
     & = \sum_{q \in \mathcal{W}_k} \left(\sum_{l=1}^L \sigma_qg_q^{(l)}\left[{\begin{array}{ccc}
     1 &\cdots & e^{j(M_t-1)\psi_{q, l}}\\
     \end{array}}\right]^T \right)  \nonumber\\
     & = \sum_{q \in \mathcal{W}_k} \sigma_q \left[{\begin{array}{ccc}
     \sum_{l=1}^L g_{q, l}& \cdots & \sum_{l=1}^L g_{q, l}e^{j(M_t-1)\psi_{q, l}}\\
     \end{array}}\right]^T 
\end{align}

 
Let us write, 

\begin{equation}
    \c_k^{(L)} = \sum_{q \in \mathcal{W}_k} \c_q^{(L)}
\end{equation}

where, 

\begin{equation}
{\c_q}^{(L)}  = \sigma_q \sum_{l=1}^{L}g_{q, l}\mathbf{d}_{M_t}(\psi_{q, l})
\end{equation}


Choosing $\g_q$ as in proposition \ref{proposiiton_g},  the $m$-th element of the vector ${\c}_q = \lim_{L\rightarrow \infty}{\c_q}^{(L)} $, i.e. $c_{q,m}$, is given by 

\begin{equation}
    c_{q,m} = \frac{\delta_q}{\sqrt{2\pi}\Delta_k}\lim_{L\rightarrow \infty} \frac{1}{L}\sum_{l=0}^{L-1} g_{q, l} e^{j(m\psi_{q, l+1})}
\end{equation}

% With a choice of $\g_q = \left[{\begin{array}{cccc} 1& \alpha^\eta &\cdots & \alpha^{\eta (L -1)} \end{array}}\right]^T$ where we set 
% % $\alpha = e^j(\frac{2\pi}{L2^B})$
% $\alpha = e^{j(\frac{\delta_q}{L})}$ and suitable $\eta$ (to be determined later), we can write


% \begin{equation}
%     \Tilde{c}_q^{(m)} = \lim_{L\longrightarrow \infty} \frac{1}{L}\sum_{l=0}^{L-1} g_l e^{jm(\psi_{q-1}+\frac{2 \pi(\ell+0.5)}{L2^B})}
% \end{equation}

\begin{equation}
    c_{q,m} = \frac{\delta_q}{\sqrt{2\pi}\Delta_k}\lim_{L\rightarrow \infty} \frac{1}{L}\sum_{l=0}^{L-1} g_{q, l} e^{jm(\psi_{q-1}+\frac{\delta_q(\ell+0.5)}{L})}
\end{equation}

% After some basic manipulations we get, 
\begin{equation}
    c_{q,m} = \frac{\delta_q}{\sqrt{2\pi}\Delta_k}\lim_{L\rightarrow \infty} \frac{1}{L}\sum_{l=0}^{L-1} \alpha^{(\eta+m) l} e^{jm(\psi_{q-1}+\frac{0.5\delta_q}{L})} 
\end{equation}


\begin{equation}
    c_{q,m} = \frac{\delta_q}{\sqrt{2\pi}\Delta_k}e^{jm(\psi_{q-1})}\lim_{L\longrightarrow \infty} \frac{1}{L}\sum_{l=0}^{L-1} \alpha^{(\eta+m) l}   
\end{equation}

\begin{equation}
    c_{q,m} = \frac{\delta_q}{\sqrt{2\pi}\Delta_k}e^{jm(\psi_{q-1})} \int_{0}^{1} \alpha^{(\eta + m)Lx}dx
\end{equation}


\begin{equation}
    c_{q,m} = \frac{\delta_q}{\sqrt{2\pi}\Delta_k}e^{jm(\psi_{q-1})} \int_{0}^{1} e^{j\frac{2\pi(\eta + m)L}{L2^B}x}dx
\end{equation}


\begin{align}
 c_{q,m} &= \frac{\delta_q}{\sqrt{2\pi}\Delta_k}e^{jm\psi_{q-1}} \int_{0}^{1} e^{j\xi x}dx \nonumber \\
 &= \frac{\delta_q}{\sqrt{2\pi}\Delta_k}e^{j(m\psi_{q-1} + \frac{\xi}{2})} sinc(\frac{\xi}{2\pi}) \label{final_g}
\end{align}

where $\xi = \delta_q (\eta +m )$.


% \begin{equation}
%     c_{q,m} = e^{jm(\psi_{q-1})} \frac{e^{j\xi}-1}{j\xi}
% \end{equation}

% \begin{equation}
%     c_{q,m} = e^{j(m\psi_{q-1} + \frac{\xi}{2})} \frac{e^{j\xi/2}-e^{-j\xi/2}}{j\xi/2}
% \end{equation}

% \begin{equation}
%     c_{q,m} = e^{j(m\psi_{q-1} + \frac{\xi}{2})} sinc(\frac{\xi}{2\pi}) \label{final_g}
% \end{equation}






% Note that the solution to part \textit{(a)} is the limit of the sequence of solutions to a least-square optimization problem as $L$ goes to infinity.
%  \begin{align}
% & \Tilde{\c}^{(L)}_q = (\D \D^H)^{-1} \D \sigma \left(\mathbf{e}_q \otimes \mathbf{g}_q\right) \\
% & \Tilde{\c}^{(L)}_q =  \sigma' \D \left(\mathbf{e}_q \otimes \mathbf{g}_q\right) = \sigma' \D_q\g_q
% \end{align}
% Dividing by $\|\Tilde{\c}^{(L)}_q\|$ and taking the limit as L goes to infinity we will find the optimal $\c_q$. i.e. 

% \begin{equation}
%     \c_q = \lim_{L\longrightarrow \infty} \frac{\Tilde{\c}^{(L)}_q}{L\|\Tilde{\c}^{(L)}_q\|}
% \end{equation}

% We aim to find $\g_q$ such that

% \begin{align}
% & \left\| abs(\D^H \gamma' \D_q \mathbf{g}_q)- abs(\gamma \left(\mathbf{e}_{q} \otimes \mathbf{g}_q\right)) \right\|_{2}^{2}
% \end{align}

% is minimized.

% % Is it possible to show that $\g_q$ is independent of $q$ (assume regular scenario)? If true, i.e., $\g_q = \g$ we have

% We will provide a suboptimal structure for the choice of $\g$. Let us assume $\g_q = \g, $  for all $ q \in \{1, \cdots, Q\}$. Then we can write
% \begin{align}
%      &\Tilde{\c_q}^{(L)} = D_q\mathbf{g} = \sum_{l=1}^{L}g_l\mathbf{d}_{M_t}(\psi_{q, l})  \nonumber\\
%      & = \sum_{l=1}^L g_l\left[{\begin{array}{cccc}
%      1& e^{j\psi_{q, l}} &\cdots & e^{j(M_t-1)\psi_{q, l}}\\
%      \end{array}}\right]^T  = \nonumber\\
%      & \left[{\begin{array}{cccc}
%      \sum_{l=1}^L g_l& \sum_{l=1}^L g_le^{j\psi_{q, l}} &\cdots & \sum_{l=1}^L g_le^{j(M_t-1)\psi_{q, l}}\\
%      \end{array}}\right]^T
%  \end{align}
 
 

%  We have for the $m$-th element of the vector $\Tilde{\c}_q$, i.e. $\Tilde{c}^{(m)}_q$  for $ 0 \leq m \leq M_t-1$

% \begin{equation}
%     \Tilde{c}_q^{(m)} = \lim_{L\longrightarrow \infty} \frac{1}{L}\sum_{l=0}^{L-1} g_l e^{j(m\psi_{q, l+1})}
% \end{equation}

% With a choice of $\g = \left[{\begin{array}{cccc} 1& \alpha^\eta &\cdots & \alpha^{\eta (L -1)} \end{array}}\right]^T$ where we set 
% % $\alpha = e^j(\frac{2\pi}{LQ})$
% $\alpha = e^j(\frac{\psi_{q}-\psi_{q-1}}{L})$ and suitable $\eta$ (to be determined later), we can write


% % \begin{equation}
% %     \Tilde{c}_q^{(m)} = \lim_{L\longrightarrow \infty} \frac{1}{L}\sum_{l=0}^{L-1} g_l e^{jm(\psi_{q-1}+\frac{2 \pi(\ell+0.5)}{LQ})}
% % \end{equation}

% \begin{equation}
%     \Tilde{c}_q^{(m)} = \lim_{L\longrightarrow \infty} \frac{1}{L}\sum_{l=0}^{L-1} g_l e^{jm(\psi_{q-1}+\frac{(\psi_{q}-\psi_{q-1})(\ell+0.5)}{L})}
% \end{equation}

% \begin{equation}
%     c_q^{(m)} = \lim_{L\longrightarrow \infty} \frac{1}{L}\sum_{l=0}^{L-1} \alpha^{(\eta+m) l} e^{jm(\psi_{q-1}+\frac{0.5(\psi_{q}-\psi_{q-1})}{L})} 
% \end{equation}


% \begin{equation}
%     c_q^{(m)} = e^{jm(\psi_{q-1})}\lim_{L\longrightarrow \infty} \frac{1}{L}\sum_{l=0}^{L-1} \alpha^{(\eta+m) l}   
% \end{equation}

% \begin{equation}
%     c_q^{(m)} = e^{jm(\psi_{q-1})} \int_{0}^{1} \alpha^{(\eta + m)Lx}dx
% \end{equation}


% % \begin{equation}
% %     c_q^{(m)} = e^{jm(\psi_{q-1})} \int_{0}^{1} e^{j\frac{2\pi(\eta + m)L}{LQ}x}dx
% % \end{equation}


% \begin{equation}
%     c_q^{(m)} = e^{jm(\psi_{q-1})} \int_{0}^{1} e^{j\xi x}dx
% \end{equation}

% where $\xi = (\psi_{q}-\psi_{q-1}) (\eta +m )$.


% \begin{equation}
%     c_q^{(m)} = e^{jm(\psi_{q-1})} \frac{e^{j\xi}-1}{j\xi}
% \end{equation}

% \begin{equation}
%     c_q^{(m)} = e^{j(m\psi_{q-1} + \frac{\xi}{2})} \frac{e^{j\xi/2}-e^{-j\xi/2}}{j\xi/2}
% \end{equation}

% \begin{equation}
%     c_q^{(m)} = e^{j(m\psi_{q-1} + \frac{\xi}{2})} sinc(\frac{\xi}{2\pi})
% \end{equation}

% where, $sinc(t) = \frac{sin(\pi t)}{\pi t}$.
















% We first optimize for $\alpha$ by setting the derivative of \eqref{obj_func} to zero, to get: 

% $$\hat{\alpha}=\frac{\sum_{q \in \mathcal{W}_k}\sqrt{\frac{Q}{M_{t}}}(\mathbf{F} \mathbf{v})^{H} \mathbf{D}_{q} \mathbf{g}_q}{\left\|\mathbf{D}^{H} \mathbf{F} \mathbf{v}\right\|_{2}^{2}}$$

% Replacing the value of $\alpha$ as above we get to the following optimization problem. 

% \begin{align}
% \left(\mathbf{F}_{\mid \mathbf{g}}, \mathbf{v}_{\mid \mathbf{g}}\right) &=\underset{\mathbf{F}, \mathbf{v}}{\arg \max } \frac{\left|\sum_{q \in \mathcal{W}_k}\sqrt{\frac{2^{\mathbf{B}}}{M_{t}}}\left(\mathbf{D}_{q} \mathbf{g}_q\right)^{H} \mathbf{F} \mathbf{v}\right|^{2}}{\left\|\mathbf{D}^{H} \mathbf{F} \mathbf{v}\right\|_{2}^{2}} \nonumber \\
% &=\underset{\mathbf{F}, \mathbf{v}}{\arg \max }\left|\sum_{q \in \mathcal{W}_k}\left(\mathbf{D}_{q} \mathbf{g}\right)^{H} \mathbf{F} \mathbf{v}\right|^{2}
% \end{align}


% \section{Codebook Design Problem Formulation}
% \label{sec: formulation}

% % Let $\mathbf{c} \doteq \mathbf{F} \mathbf{v} \in \mathbb{C}^{M_{t}}$ be the unit-norm transmit beam-forming codeword, where analog beamsteering matrix $\mathbf{F}=\left[\mathbf{f}_{1}, \cdots, \mathbf{f}_{N_{R F}}\right] \in \mathbb{C}^{M_{t} \times N_{R F}}$ contains $N_{RF}$ equal-gain analog beamsteering vectors $\mathbf{f}_{n}$ and the baseband beamforming vector $\mathbf{v} \in \mathbb{C}^{N_{R F}}$ is such that $\|\c\|_{2}=1$ holds. The equal-gain constraint holds if $\mathbf{f}_{n} \in \mathcal{B}_{M_{t}}$ for 

% % \begin{equation}
% %     \mathcal{B}_{M_{t}}=\left\{\mathbf{w} \in \mathbb{C}^{M_{t}}:\left(\mathbf{w} \mathbf{w}^{H}\right)_{\ell, \ell}=1 / M_{t}, 1 \leq \ell \leq M_{t}\right\}
% %     \label{f_constant_gain}
% % \end{equation}

% % Let $\mathcal{C}=\left\{\mathbf{c}_{1}, \cdots, \mathbf{c}_{Q}\right\}$ be the codebook with codewords taking value as $\mathbf{c}_{q}=\mathbf{F}_{q} \mathbf{v}_{q}$. Consider a half-wavelength uniform linear array (ULA) of antennas as in the previous subsection, i.e. $d = \frac{\lambda}{2}$, with an array response vector as in \eqref{array_factor}.

% % % %$$\mathbf{d}_{M_{t}}(\psi)=\frac{1}{\sqrt{M_{t}}}\left[1 e^{j \psi} \cdots e^{j\left(M_{t}-1\right) \psi}\right]^{T} \in \mathbb{C}^{M_{t}}$$
% % % $$\mathbf{d}_{M_{t}}(\psi)=\left[1, e^{j \psi}, \cdots, e^{j\left(M_{t}-1\right) \psi}\right]^{T} \in \mathbb{C}^{M_{t}}$$

%  For convenience, in this section, we will drop the index \emph{ula} from the expression of the array factor $\d_{ula, M_t}(\psi)$ and gain $G^{ula}(\psi, \mathbf{c})$.  Before presenting the formulation of the codebook design problem, we introduce some new notations. We divide the angular range of $\theta \in [-\pi, 0]$  into equal-length beams $\omega_{q} \text { for } q \in\left\{1, \cdots, Q\right\}$. i.e. 
% $$
% \begin{aligned}
% \omega_{q} &=\left[\theta_{q-1}, \theta_{q}\right) ,& \theta_{q}= -\pi+ \frac{ \pi}{Q} q
% \end{aligned}
% $$

% % Further we introduce the change of variable $\psi = \frac{2 \pi d}{\lambda} \cos(\theta)$ where $\psi \in [-\pi, \pi]$. For $d = \frac{\lambda}{2}$,

% Corresponding to $\omega_q$ intervals, there exists $\nu_q$ ranges with respect to $\psi$ such that, 

% $$
% \begin{aligned}
% \nu_{q} &=\left[\psi_{q-1}, \psi_{q}\right), & \psi_q = -\pi\cos(\frac{\pi}{Q}q)
% \end{aligned}
% $$


% Under the reference gain as in \eqref{reference gain} and using Parseval's theorem \cite{parseval}, we will have:
% % and assuming $|\c| = 1$
% \begin{equation}
% \int_{-\pi}^{\pi} G(\psi, \mathbf{c}) d \psi=2 \pi\left\| \mathbf{c}\right\|^{2}=2 \pi\label{parseval}
% \end{equation}

% Let  $G_{\text {ideal }, q}(\psi)$ denote the desired ideal gain which is supposed to be constant on $\nu_q$ and zero otherwise. It must hold that, 
% % \begin{equation}
% % \int_{-\pi}^{\pi} G(\psi, \mathbf{c}) d \psi=2 \pi\left\|\frac{1}{\sqrt{M_{t}}} \mathbf{c}\right\|_{2}^{2}=\frac{2 \pi}{M_{t}}    \label{parseval}
% % \end{equation}





% \begin{align}
% &\int_{-\pi}^{\pi} G_{\text {ideal }, q}(\psi) d \psi =\int_{\nu_{q}} t d \psi+\int_{[-\pi, \pi] \backslash \nu_{q}} 0 d \psi \nonumber\\
% &=(\psi_q - \psi_{q-1}) t={2 \pi} \label{plain}
% \end{align}

% which in turn will give: 

% \begin{equation}
%     G_{\text {ideal }, q}(\psi)=\frac{2\pi}{(\psi_q - \psi_{q-1})} \mathds{1}_{\nu_{q}}(\psi), \quad \psi \in[-\pi, \pi] \label{ideal_gain}
% \end{equation}

% We aim to design the codebooks so as to mimic the ideal gain computed in equation \eqref{ideal_gain}. Therefore, the plain codebook design problem is formulated as a minimization of a MSE as follows: 


% % \begin{align}
% % &\left(\mathbf{F}_{\text {opt }, q}, \mathbf{v}_{\mathrm{opt}, q}\right) \nonumber \\&=\underset{\mathbf{F}, \mathbf{v}}{\arg \min } \int_{-\pi}^{\pi}\left|G_{\text {ideal }, q}(\psi)-G(\psi, \mathbf{F} \mathbf{v})\right|^{2} d \psi
% % \end{align}

% \begin{align}
% &\c_q^{opt}=\underset{\c, \|\c\|=1}{\arg \min } \int_{-\pi}^{\pi}\left|G_{\text {ideal }, q}(\psi)-G(\psi, \c)\right| d \psi
% \label{composite_MSE}
% \end{align}

% By uniformly sampling on the range of $\psi$ we can rewrite the optimization problem as follows, 
% \begin{align}
%     &\underset{\c, \|\c\|=1}{\arg \min }\left[\lim _{L \rightarrow \infty} \sum_{p=1}^{Q} \delta_p \sum_{\ell=1}^{L} \frac{\left|G_{\text {ideal }, q}\left(\psi_{p, \ell}\right)-G\left(\psi_{p, \ell}, \c \right)\right|}{L}\right]\label{equivalent_MSE}
% \end{align}

%  where for $q = 1 \ldots Q$, 
%  $$\delta_q = \psi_{q}-\psi_{q-1}, \quad \psi_{q, \ell}=\psi_{q-1}+\frac{\delta_q(\ell-0.5)}{L} $$

 
%  We can write equation \eqref{equivalent_MSE}, as 
%  \begin{align}
%  &\c_q^{opt} = \underset{\c,  \|\c\|=1}{\arg \min } \lim_{L\rightarrow \infty}\frac{1}{L}\left\|\mathbf{G}_{\text {ideal }, q}-\mathbf{G}(\c)\right\| \label{init_opt}
%  \end{align}
 
%  where,  $$\mathbf{G}(\c)=\left[{\delta_1} G\left(\psi_{1,1}, \c\right) \ldots {\delta_{Q}}G\left(\psi_{Q, L}, \c\right)\right]^{T} \in \mathbb{Z}^{L Q}$$

% and, 
% $$ \mathbf{G}_{\text {ideal }, q}=\left[{\delta_1} G_{\text {ideal }, q}\left(\psi_{1,1}\right) \ldots {\delta_{Q}}G_{\text {ideal }, q}\left(\psi_{Q, L}\right)\right]^{T} \in \mathbb{Z}^{L Q}$$

% Note that it holds that 

% \begin{equation}
%     \mathbf{G}_{\text {ideal }, q}={2\pi}\left(\mathbf{e}_{q} \otimes \mathbf{1}_{L, 1}\right) \label{ideal}
% \end{equation}

% with $\mathbf{e}_{q} \in \mathbb{Z}^{Q}$ being the standard basis vector for the $q$-th axis among $Q$ ones. Now, note that $\mathbf{1}_{L, 1}=\mathbf{g} \odot \mathbf{g}^{*}$ for any equal gain $\mathbf{g} \in \mathbb{C}^L$. Therefore, for a suitable choice of $\g$ we can write: 

% % $$\mathcal{G}_{L}=\left\{\mathbf{g} \in \mathbb{C}^{L}:\left(\mathbf{g} \mathbf{g}^{H}\right)_{\ell, \ell}=1,1 \leq \ell \leq L\right\}$$


% \begin{align}
% \mathbf{G}_{\text {ideal }, q} &={2\pi}\left(\mathbf{e}_{q} \otimes\left(\mathbf{g} \odot \mathbf{g}^{*}\right)\right) \nonumber\\
% &=\left(\sqrt{2\pi}\left(\mathbf{e}_{q} \otimes \mathbf{g}\right)\right) \odot\left(\sqrt{{2\pi}}\left(\mathbf{e}_{q} \otimes \mathbf{g}\right)\right)^{*} \label{g_id_q_equivalent}
% \end{align}


% Similarly, it is straightforward to observe,


% \begin{align}
% \mathbf{G}(\c) &=\left(\mathbf{D}^{H} \c\right) \odot\left(\mathbf{D}^{H} \c\right)^{*} \label{dc}
% \end{align}
% where $\mathbf{D} =\left[\sqrt{\delta_1}\mathbf{D}_{1} \cdots \sqrt{\delta_{Q}}\mathbf{D}_{Q} \right] \in \mathbb{C}^{M_{t} \times L Q}$, and 
% $\mathbf{D}_{q}=\left[\mathbf{d}_{M_{t}}\left(\psi_{q, 1}\right) \cdots \mathbf{d}_{M_{t}}\left(\psi_{q, L}\right)\right] \in \mathbb{C}^{M_{t} \times L}.$
 

% % Comparing the expressions in equations \eqref{g_id_q_equivalent} and \eqref{dc},  we formulate the following two-step optimization for the codebook design problem for hybrid beamforming, presented below as \textit{Problem \ref{main_problem}}. Given a suitable $\g_q$, in the first step, we search for a unit-norm codeword $\c_q$ for fully-digital beamforming, and in the second step we find a suitable pair $(\F_q, \v_q)$ to complete the hybrid beamforming codebook design.  

 

% % \begin{problem}

% % a) For all $q \in \{1, \cdots, Q\}$, given equal-gain $\g_q \in \mathbb{C}^L$, find vector $\c_q \in \mathbb{C}^{M_t}$ such that
% % \begin{align}
% % &\c_q=\underset{\c, \|c\|=1}{\arg \min } \lim_{L\longrightarrow \infty} \left\|\sqrt{2\pi}\left(\mathbf{e}_{q} \otimes \mathbf{g}_q\right)- \mathbf{D}^{H} \c\right\|^{2} \label{obj_func}
% % \end{align}

% % \vspace{2mm}
% % \noindent
% % b) Find $\F_q \in \mathbb{C}^{M_t \times N_{RF}}$ , and $\v_q \in \mathbb{C}^{N_{RF}}$ that solve $\mathbf{F}_{q}\mathbf{v}_{q} = \c_q$ subject to the equal-gain condition on the columns of $\F_q$. 

% % % \begin{align}
% % %     \mathbf{F}_{\mid \g_q}\mathbf{v}_{\mid \g_q} = \c_q
% % % \end{align}
% % \label{main_problem}
% % \end{problem}

% % The solution to the last problem provides the optimal codebook designed for hybrid beamforming under a ULA structure, given the choice of $\g_q$. Note that we were initially trying to solve the optimization problem in equation \eqref{init_opt}. In light of modifications \eqref{g_id_q_equivalent}, and \eqref{dc}, the quantity to be minimized in the limit of equation \eqref{init_opt} can be written as follows,

% % \begin{align}
% %     & \left\|\mathbf{G}_{\text {ideal }, q}-\mathbf{G}(\c)\right\| = \left\| abs(\D^H \c_q)- \sqrt{2\pi} abs(\mathbf{e}_{q} \otimes \mathbf{g})\right\| \label{g_init}
% % \end{align}
% % where $abs(.)$ denotes the element-wise absolute value of a vector.

% % Equation \eqref{g_init} reveals the central role of the vector $\g_q$ in minimizing the MSE in equation \eqref{init_opt}. In fact, the choice of $\g_q$ has to be optimized in a way that norm in \eqref{g_init} is minimized; i.e. the result gain designed by codebook $\c_q$ gets as close as possible to the ideal gain. To optimize the choice of $\g_q$, we formulate the following optimization problem, presented as \textit{Problem \ref{g_problem}}.

% % \begin{problem}
% % For all $q = 1, \ldots, Q$, find equal-gain $\g_q \in \mathbb{C}^L$ such that

% % \begin{equation}
% %  \g_q = \underset{\g}{\arg\min }\left\| abs(\D^H \c_q)- \sqrt{2\pi} abs(\mathbf{e}_{q} \otimes \mathbf{g})\right\|^{2} \label{g_final_eq}
% % \end{equation} 

% % \label{g_problem}
% % \end{problem}

% % In the next session we propose our approach to solve problems \ref{main_problem} and \ref{g_problem}.

% %%%%%%%%%%%%%%%%%%%%%%%%%%%%%%%
% Comparing the expressions \eqref{init_opt}, \eqref{g_id_q_equivalent}, and \eqref{dc}, one can show that the optimal choice of $\c_q$ in \eqref{init_opt} is the solution to the following optimization problem for a proper choice of $\g_q$. 
% \begin{problem}
% Given an equal-gain vector $\g_q \in \mathbb{C}^L$, $q \in \{1, \cdots, Q\}$, find vector $\c_q \in \mathbb{C}^{M_t}$ such that
% \begin{align}
% &\c_q=\underset{\c, \|c\|=1}{\arg \min } \lim_{L\longrightarrow \infty} \left\|\sqrt{2\pi}\left(\mathbf{e}_{q} \otimes \mathbf{g}_q\right)- \mathbf{D}^{H} \c\right\|^{2} \label{obj_func}
% \end{align}
% \label{main_problem}
% \end{problem}
% However, we now need to find the optimal choice of $\g_q$ that minimizes the objective in \eqref{init_opt}. Using \eqref{g_id_q_equivalent}, and \eqref{dc}, we have the following optimization problem.
% \begin{problem}
% Find equal-gain $\g_q \in \mathbb{C}^L$, $q = 1, \ldots, Q$, such that
% \begin{equation}
%  \g_q = \underset{\g}{\arg\min }\left\| abs(\D^H \c_q)- \sqrt{2\pi} abs(\mathbf{e}_{q} \otimes \mathbf{g})\right\|^{2} \label{g_final_eq}
% \end{equation} 
% where $abs(.)$ denotes the element-wise absolute value of a vector.
% \label{g_problem}
% \end{problem}

% Hence, the codebook design for a system with full-digital beamforming capability is found by solving Problem~\ref{main_problem} for proper choice of $\g_q$ obtained as a solution to Problem ~\ref{g_problem}. The codebook for Hybrid beamforming, is then found as
% \begin{align}
%     \underset{\mathbf{F}_{q},\mathbf{v}_{q}}{\arg\min } \|\mathbf{F}_{q}\mathbf{v}_{q} - \c_q\|^2
% \end{align}
% where the columns of $\F_q \in \mathbb{C}^{M_t \times N_{RF}}$ are equal-gain vectors and $\v_q \in \mathbb{C}^{N_{RF}}$. The solution may be obtained using simple, yet effective suboptimal algorithm such as orthogonal matching pursuit (OMP) \cite{love15}\cite{noh17}\cite{Hussain17}. In the next section we continue with the solution of Problem~\ref{main_problem}, and~\ref{g_problem}.


% %%%%%%%%%%%%%%%%%%%%%%%%%%%%%%%

% % Note that the solution to part \textit{(a)} is the limit of the sequence of solutions to a least-square optimization problem as $L$ goes to infinity.
% %  \begin{align}
% % & \Tilde{\c}^{(L)}_q = (\D \D^H)^{-1} \D \sigma \left(\mathbf{e}_q \otimes \mathbf{g}_q\right) \\
% % & \Tilde{\c}^{(L)}_q =  \sigma' \D \left(\mathbf{e}_q \otimes \mathbf{g}_q\right) = \sigma' \D_q\g_q
% % \end{align}
% % Dividing by $\|\Tilde{\c}^{(L)}_q\|$ and taking the limit as L goes to infinity we will find the optimal $\c_q$. i.e. 

% % \begin{equation}
% %     \c_q = \lim_{L\longrightarrow \infty} \frac{\Tilde{\c}^{(L)}_q}{L\|\Tilde{\c}^{(L)}_q\|}
% % \end{equation}

% % We aim to find $\g_q$ such that

% % \begin{align}
% % & \left\| abs(\D^H \gamma' \D_q \mathbf{g}_q)- abs(\gamma \left(\mathbf{e}_{q} \otimes \mathbf{g}_q\right)) \right\|_{2}^{2}
% % \end{align}

% % is minimized.

% % % Is it possible to show that $\g_q$ is independent of $q$ (assume regular scenario)? If true, i.e., $\g_q = \g$ we have

% % We will provide a suboptimal structure for the choice of $\g$. Let us assume $\g_q = \g, $  for all $ q \in \{1, \cdots, Q\}$. Then we can write
% % \begin{align}
% %      &\Tilde{\c_q}^{(L)} = D_q\mathbf{g} = \sum_{l=1}^{L}g_l\mathbf{d}_{M_t}(\psi_{q, l})  \nonumber\\
% %      & = \sum_{l=1}^L g_l\left[{\begin{array}{cccc}
% %      1& e^{j\psi_{q, l}} &\cdots & e^{j(M_t-1)\psi_{q, l}}\\
% %      \end{array}}\right]^T  = \nonumber\\
% %      & \left[{\begin{array}{cccc}
% %      \sum_{l=1}^L g_l& \sum_{l=1}^L g_le^{j\psi_{q, l}} &\cdots & \sum_{l=1}^L g_le^{j(M_t-1)\psi_{q, l}}\\
% %      \end{array}}\right]^T
% %  \end{align}
 
 

% %  We have for the $m$-th element of the vector $\Tilde{\c}_q$, i.e. $\Tilde{c}^{(m)}_q$  for $ 0 \leq m \leq M_t-1$

% % \begin{equation}
% %     \Tilde{c}_q^{(m)} = \lim_{L\longrightarrow \infty} \frac{1}{L}\sum_{l=0}^{L-1} g_l e^{j(m\psi_{q, l+1})}
% % \end{equation}

% % With a choice of $\g = \left[{\begin{array}{cccc} 1& \alpha^\eta &\cdots & \alpha^{\eta (L -1)} \end{array}}\right]^T$ where we set 
% % % $\alpha = e^j(\frac{2\pi}{LQ})$
% % $\alpha = e^{j(\frac{\psi_{q}-\psi_{q-1}}{L})}$ and suitable $\eta$ (to be determined later), we can write


% % % \begin{equation}
% % %     \Tilde{c}_q^{(m)} = \lim_{L\longrightarrow \infty} \frac{1}{L}\sum_{l=0}^{L-1} g_l e^{jm(\psi_{q-1}+\frac{2 \pi(\ell+0.5)}{LQ})}
% % % \end{equation}

% % \begin{equation}
% %     \Tilde{c}_q^{(m)} = \lim_{L\longrightarrow \infty} \frac{1}{L}\sum_{l=0}^{L-1} g_l e^{jm(\psi_{q-1}+\frac{(\psi_{q}-\psi_{q-1})(\ell+0.5)}{L})}
% % \end{equation}

% % \begin{equation}
% %     c_q^{(m)} = \lim_{L\longrightarrow \infty} \frac{1}{L}\sum_{l=0}^{L-1} \alpha^{(\eta+m) l} e^{jm(\psi_{q-1}+\frac{0.5(\psi_{q}-\psi_{q-1})}{L})} 
% % \end{equation}


% % \begin{equation}
% %     c_q^{(m)} = e^{jm(\psi_{q-1})}\lim_{L\longrightarrow \infty} \frac{1}{L}\sum_{l=0}^{L-1} \alpha^{(\eta+m) l}   
% % \end{equation}

% % \begin{equation}
% %     c_q^{(m)} = e^{jm(\psi_{q-1})} \int_{0}^{1} \alpha^{(\eta + m)Lx}dx
% % \end{equation}


% % % \begin{equation}
% % %     c_q^{(m)} = e^{jm(\psi_{q-1})} \int_{0}^{1} e^{j\frac{2\pi(\eta + m)L}{LQ}x}dx
% % % \end{equation}


% % \begin{equation}
% %     c_q^{(m)} = e^{jm(\psi_{q-1})} \int_{0}^{1} e^{j\xi x}dx
% % \end{equation}

% % where $\xi = (\psi_{q}-\psi_{q-1}) (\eta +m )$.


% % \begin{equation}
% %     c_q^{(m)} = e^{jm(\psi_{q-1})} \frac{e^{j\xi}-1}{j\xi}
% % \end{equation}

% % \begin{equation}
% %     c_q^{(m)} = e^{j(m\psi_{q-1} + \frac{\xi}{2})} \frac{e^{j\xi/2}-e^{-j\xi/2}}{j\xi/2}
% % \end{equation}

% % \begin{equation}
% %     c_q^{(m)} = e^{j(m\psi_{q-1} + \frac{\xi}{2})} sinc(\frac{\xi}{2\pi})
% % \end{equation}

% % where, $sinc(t) = \frac{sin(\pi t)}{\pi t}$.

% % \subsection{Twin-ULA Setting}

% % We start from problem \eqref{main_problem}, and particularly, equation \eqref{obj_func}. Let $\{\s_m\}_{m =0}^{M_t -1}$ denote the columns of $D^H$. We can write: 

% % \begin{align}
% % &\s_m c_q^{(m)} + \s_{m+M_t/2} c_q^{(m+M_t/2)} \nonumber\\
% % &= \s_m (c_q^{(m)} + e^{-j \frac{2\pi d_y}{\lambda} \sin(\theta)} c_q^{(m+M_t/2)}) \nonumber\\
% % & \approx \s_m \Tilde{c}_q^{(m)}, \quad  m = 0 \ldots \frac{M_t}{2}-1\end{align}

% % where we define 
% % \begin{equation}
% %     \Tilde{c}_q^{(m)} = c_q^{(m)} + e^{-j \frac{2\pi d_y}{\lambda} \sin(\theta^{(m)}_q)} c_q^{(m+M_t/2)} \label{c_defn}
% % \end{equation}

% % with $\theta_{q-1}\leq\theta^{(m)}_q \leq \theta_{q}$ being a constant value.  

% % Therefore, the solution to part $(a)$ of problem \eqref{main_problem}, under the twin-ULA setup can be expressed as:

% %  \begin{align}
% % & \Tilde{\c}_q = (\Tilde{\D} \Tilde{\D}^H)^{-1} \Tilde{\D} \sigma \left(\mathbf{e}_q \otimes \mathbf{g}_q\right) \\
% % & \Tilde{\c}_q =  \sigma' \Tilde{\D} \left(\mathbf{e}_q \otimes \mathbf{g}_q\right) = \sigma' \Tilde{\D}_q\g_q
% % \end{align}

% % where it holds that $\Tilde{\D} \Tilde{\D}^H = \I_{\frac{M_t}{2}}$, and


% % $$\Tilde{\D}_{q}=\left[\d_{\frac{M_{t}}{2}}\left(\psi_{q, 1}\right) \cdots \d_{\frac{M_{t}}{2}}\left(\psi_{q, L}\right)\right] \in \mathbb{C}^{\frac{M_{t}}{2} \times L}$$

% % In other words, one can approximately think of the codebook design problem over a twin ULA of $M_t$ antennas, as such a problem over a ULA of size $\frac{M_t}{2}$.

% % Following the approach in the previous subsection, we can infer, 

% % \begin{equation}
% %     \Tilde{c}_q^{(m)} = e^{j(m\psi_{q-1} + \frac{\xi}{2})} sinc(\frac{\xi}{2\pi}), \quad m = 0 \ldots \frac{M_t}{2} -1
% %     \label{twin_c_result}
% % \end{equation}

% % Having $d_y = \frac{\lambda}{3}$, equating equations \eqref{c_defn}, and \eqref{twin_c_result}, by further assuming $c_q^{(m + \frac{M_t}{2})} = e^{j\beta_m} c_q^{(m)}$, after basic operations we get: 

% % \begin{equation}
% %     c_q^{(m)} = \frac{e^{j(m\psi_{q-1} + \frac{\xi}{2})} sinc(\frac{\xi}{2\pi})}{1+ e^{-j(\frac{2\pi}{3}\sin(\theta_q^{(m)}) + \beta_m)}}
% % \end{equation}

% % % \begin{equation}
% % %     c_q^{(m)} = \frac{e^{j(m\psi_{q-1} + \frac{\xi}{2} +\frac{\pi}{3}\sin(\theta_q^{(m)})+\frac{\beta_m}{2}) sinc(\frac{\xi}{2\pi})}}{2\cos(\frac{\pi}{3}\sin(\theta_q^{(m)}) + \frac{\beta_m}{2})}
% % % \end{equation}

% % Therefore, for each codeword entry $c_q^{(m)}$, it remains to pick $\beta_m$ and $\theta_q^{(m)}$. We relax this decision by setting $\beta_m = \beta$, and $\theta_q^{(m)} = \theta_q$, $ m = 1 \ldots M_t$. Next we explicitly right the expression for the reference gain as $G(\theta, \c_{twin}) = \left|\d_{twin, M_{t}}^{H}(\theta) \c_{twin}\right|^{2}$ where 

% % \begin{align}
% %     &\d_{twin, M_{t}}(\theta) = \left[\d_{\frac{M_{t}}{2}}^{T}(\theta), e^{j(\frac{2\pi}{3}\sin(\theta))}\d_{\frac{M_{t}}{2}}^{T}(\theta)\right]^T\\
% %     & \c_{twin} = \left[\c^T_{twin, \frac{M_t}{2}}, e^{j\beta}\c^T_{twin, \frac{M_t}{2}}\right]^T
% % \end{align}

% % We can  therefore write, 

% % \begin{align}
% %     G(\theta, \c_{twin}) &= \left|\d_{\frac{M_{t}}{2}}^{H}(\theta)\c_{twin, \frac{M_t}{2}} (1 + e^{j(\beta-\frac{2\pi}{3}\sin(\theta))})\right|^2\nonumber\\
% %     & = \left|\d_{\frac{M_{t}}{2}}^{H}(\theta)\c\right|^2 \left| \frac{1 + e^{j(\beta-\frac{2\pi}{3}\sin(\theta))}}{1 + e^{-j(\beta+\frac{2\pi}{3}\sin(\theta))}}\right|^2
% % \end{align}

% % Further define 
% % $$ L(\theta) = \left| \frac{1 + e^{j(\beta-\frac{2\pi}{3}\sin(\theta))}}{1 + e^{-j(\beta+\frac{2\pi}{3}\sin(\theta))}}\right|$$
% % The last equation consists of two parts, the first part being the gain corresponding to a ULA of size $\frac{M_t}{2}$ and the second part being  $L^2(\theta)$. For any $\omega_q = \left[\theta_{q-1}, \theta_{q}\right)$, there exist a $\omega'_q \doteq \left(- \theta_{q}, -\theta_{q-1}\right]$. The gain of a ULA is symmetric over $\omega_q$, and $\omega'_q$ by definition. However utilizing a twin ULA we wish to suppress the gain over $\omega'_q$ as much as we can and contribute to the gain over $\omega_q$ for each codeword $c_q, \quad q = 1\ldots Q$. To this end, we define the \emph{isolation factor} $\mu$ as follows,

% % \begin{align}
% %     \mu = \underset{\omega_q}{\int}{\frac{L(-\theta)}{L(\theta)}} d\theta
% % \end{align}
% %  to denote the level of isolation between each $\omega_q$ and its counterpart.
 
 















% % % \subsection{Composite Codebook Design Problem}

% % % Let us define a \textit{composite beam} $\omega_k$ as a union of multiple disjoint, possibly non-neighboring beams $\nu_{q} \text { for } q \in \mathcal{Q} = \left\{1, \cdots, Q\right\}$. Let $\mathcal{C'}=\left\{\mathbf{c}_{1}, \cdots, \mathbf{c}_{K}\right\}$ be the codebook corresponding to the composite problem.

% % % % $$\bigcup_{k = 1}^{2^{B'}}{\omega_k} = [-\pi, \pi], \quad \text{and} \quad \omega_k \cap \omega_l = \emptyset, \quad \forall k\neq l.$$

% % % Moreover, define for each composite beam $\omega_k$ the set $\mathcal{W}_k \subseteq \mathcal{Q}$ to be the set of indices of the single beams that form $\omega_k$. i.e. $\mathcal{W}_k = \{q \in \mathcal{Q}: v_q \subseteq \omega_k\}$. 
% % % We have then for the new ideal gain for each new codeword $\textbf{c}_k$, 

% % % \begin{align}
% % % &\int_{-\pi}^{\pi} G_{\text {ideal }, k}(\psi) d \psi =\int_{\omega_{k}} t' d \psi+\int_{[-\pi, \pi] \backslash \omega_{k}} 0 d \psi \nonumber\\
% % % &= \sum_{q \in {\mathcal{W}_k}}{\int_{\nu_{q}} t' d \psi} = |\mathcal{W}_k|\frac{2 \pi}{Q} t'=\frac{2 \pi}{M_{t}} \label{composite}
% % % \end{align}

% % % Therefore, $t' = \frac{Q}{|\mathcal{W}_k| M_t}$. It follows that, 

% % % \begin{equation}
% % %     G_{\text {ideal }, k}(\psi)=\frac{Q}{|\mathcal{W}_k| M_{t}} \mathds{1}_{\omega_{k}}(\psi), \quad \psi \in[-\pi, \pi] \label{composite_ideal_gain}
% % % \end{equation}

% % % The new MSE problem for the $k$-th codeword can therefore be written as follows.

% % % \begin{align}
% % % &\left(\mathbf{F}_{\text {opt }, k}, \mathbf{v}_{\mathrm{opt}, k}\right) \nonumber \\&=\underset{\mathbf{F}, \mathbf{v}}{\arg \min } \int_{-\pi}^{\pi}\left|G_{\text {ideal }, k}(\psi)-G(\psi, \mathbf{F} \mathbf{v})\right|^{2} d \psi \nonumber \\&= \underset{\mathbf{F}, \mathbf{v}}{\arg \min }\left[\lim _{L \rightarrow \infty} \sum_{p=1}^{Q} \sum_{\ell=1}^{L} \frac{\left|G_{\text {ideal }, k}\left(\psi_{p, \ell}\right)-G\left(\psi_{p, \ell}, \mathbf{F} \mathbf{v}\right)\right|^{2}}{L Q / 2 \pi}\right]\label{composite_MSE}
% % % \end{align}


% % % \begin{align}
% % % \left(\mathbf{F}_{k}, \mathbf{v}_{k}\right) &=\underset{\mathbf{F}, \mathbf{v}}{\arg \min } \sum_{p=1}^{Q} \sum_{\ell=1}^{L}\left|G_{\text {ideal }, k}\left(\psi_{p, \ell}\right)-G\left(\psi_{p, \ell}, \mathbf{F} \mathbf{v}\right)\right|^{2} \nonumber\\
% % % &=\underset{\mathbf{F}, \mathbf{v}}{\arg \min }\lim_{L\longrightarrow \infty} \frac{1}{L}\left\|\mathbf{G}_{\text {ideal }, k}-\mathbf{G}(\mathbf{F} \mathbf{v})\right\|_{2}^{2} \label{init_opt}
% % % \end{align}

% % % where,  $$\mathbf{G}(\mathbf{F} \mathbf{v})=\left[G\left(\psi_{1,1}, \mathbf{F} \mathbf{v}\right) \cdots G\left(\psi_{Q, L}, \mathbf{F} \mathbf{v}\right)\right]^{T} \in \mathbb{Z}^{L Q}$$

% % % and, 
% % % $$ \mathbf{G}_{\text {ideal }, k}=\left[G_{\text {ideal }, k}\left(\psi_{1,1}\right) \cdots G_{\text {ideal }, k}\left(\psi_{Q, L}\right)\right]^{T} \in \mathbb{Z}^{L Q}$$

% % % Note that we can write using the definition of $\mathcal{W}_k$ that, 

% % % $$G_{\text {ideal }, k}\left(\psi_{p, \ell}\right)=\left\{\begin{array}{ll}
% % % \frac{Q}{|\mathcal{W}_k| M_{t}}, & p \in \mathcal{W}_k \\
% % % 0, & p \notin \mathcal{W}_k 
% % % \end{array},\right.$$

% % % Or in the matrix form, 

% % % \begin{align}
% % %     &\mathbf{G}_{\text {ideal }, k}=\frac{Q}{|\mathcal{W}_k| M_{t}}\left((\bigoplus_{q \in \mathcal{W}_k}{\mathbf{e}_{q}}) \otimes \mathbf{1}_{L, 1}\right)  \nonumber \\ 
% % %     & = \sum_{q \in \mathcal{W}_k}{\frac{Q}{|\mathcal{W}_k| M_{t}}\left(\mathbf{e}_{q} \otimes \mathbf{1}_{L, 1}\right)} \label{G_ideal_k}
% % % \end{align}

% % % with $\mathbf{e}_{q} \in \mathbb{Z}^{Q}$ being the standard basis vector for the $q$-th axis among $Q$ ones. Now, note that $\mathbf{1}_{L, 1}=\mathbf{g} \odot \mathbf{g}^{*}$ for any $\mathbf{g} \in \mathcal{G}_L$, where, 

% % % $$\mathcal{G}_{L}=\left\{\mathbf{g} \in \mathbb{C}^{L}:\left(\mathbf{g} \mathbf{g}^{H}\right)_{\ell, \ell}=1,1 \leq \ell \leq L\right\}$$



% % % Therefore, equation \eqref{G_ideal_k} can be rewritten as:
% % % \begin{align}
% % % &\mathbf{G}_{\text {ideal }, k} = \sum_{q \in \mathcal{W}_k}\frac{Q}{|\mathcal{W}_k| M_{t}}\left(\mathbf{e}_{q} \otimes\left(\mathbf{g}_q \odot \mathbf{g}_q^{*}\right)\right) \nonumber\\
% % % &=\sum_{q \in \mathcal{W}_k}\left(\sigma\left(\mathbf{e}_{q} \otimes \mathbf{g}_q\right)\right) \odot\left(\sigma\left(\mathbf{e}_{q} \otimes \mathbf{g}_q\right)\right)^{*} \nonumber\\
% % % &=\left(\sum_{q \in \mathcal{W}_k}\sigma\left(\mathbf{e}_{q} \otimes \mathbf{g}_q\right)\right)  \odot \left(\sum_{q \in \mathcal{W}_k}\sigma\left(\mathbf{e}_{q} \otimes \mathbf{g}_q\right)\right)^{*} \label{final_gik}
% % % \end{align}

% % % where $\sigma = \sqrt{\frac{Q}{|\mathcal{W}_k|M_{t}}}$. 
% % % On the other hand note that $\mathbf{G}(\mathbf{F} \mathbf{v})$ can be written as,


% % % \begin{equation}
% % %     \mathbf{G}(\mathbf{F} \mathbf{v}) =\left(\mathbf{D}^{H} \mathbf{F} \mathbf{v}\right) \odot\left(\mathbf{D}^{H} \mathbf{F} \mathbf{v}\right)^{*}\label{dfv_decomp}
% % % \end{equation}

% % % where, 
% % % $$\mathbf{D} =\left[\mathbf{D}_{1} \cdots \mathbf{D}_{Q}\right] \in \mathbb{C}^{M_{t} \times L Q}$$

% % % and, $\mathbf{D}_{q}=\left[\mathbf{d}_{M_{t}}\left(\psi_{q, 1}\right) \cdots \mathbf{d}_{M_{t}}\left(\psi_{q, L}\right)\right] \in \mathbb{C}^{M_{t} \times L}$. 

% % % Comparing the expressions in equations \eqref{final_gik}, and, \eqref{dfv_decomp}, a perfect solution to the optimization problem in \eqref{init_opt}, could be a pair $\textbf{(F, v)}$, for which the following equality holds.  



% % % \begin{align}
% % %     \mathbf{D}^{H} \mathbf{F} \mathbf{v}=\sum_{q \in \mathcal{W}_k}\sigma\left(\mathbf{e}_{q} \otimes \mathbf{g}_q\right) \label{potential_answer}
% % % \end{align}

% % % However, we have that $M_t > N_{RF}$ and therefore, the matrix $\textbf{D}^H \textbf{F}$ may not be full-rank, resulting in the RHS of equation \eqref{potential_answer} being in the null space of LHS. Therefore, given some $\g_q \in \mathcal{G}_L$ we formulate the following two-step optimization problem to search for $\textbf{(F, v)}$ that gets as close as possible to the desired value. 

% % % \begin{problem}

% % % a) For all $q \in \{1, \cdots, Q\}$, given $\g_q \in \mathcal{G}_L$, find the unit-norm vector $\c_q \in \mathbb{C}^{M_t}$ such that
% % % \begin{align}
% % % &\c^{|\g_q}_q=\underset{\c}{\arg \min } \lim_{L\longrightarrow \infty} \frac{1}{L}\left\|\left(\sum_{q \in \mathcal{W}_k}\sigma\left(\mathbf{e}_{q} \otimes \mathbf{g}_q\right)\right)- \mathbf{D}^{H} \c\right\|_{2}^{2} \label{obj_func}
% % % \end{align}

% % % \noindent
% % % b) For every given $\c_q$, find $\F_q \in \mathbb{C}^{M_t \times N_{RF}}$ , and $\v_q \in \mathbb{C}^{N_RF}$ that satisfy $\mathbf{F}_{q}\mathbf{v}_{q} = \c_q$ the condition $\f_n \in \mathcal{B}_{M_t}$ as in equation \eqref{f_constant_gain}. 

% % % % \begin{align}
% % % %     \mathbf{F}_{\mid \g_q}\mathbf{v}_{\mid \g_q} = \c_q
% % % % \end{align}

% % % \end{problem}

% % % Note that the solution to part \textit{(a)} is the limit of the sequence of solutions to a least-square optimization problem as $L$ goes to infinity.
% % %  \begin{align}
% % % & \Tilde{\c}^{(L)}_q = (\D \D^H)^{-1} \D \sigma \left(\mathbf{e}_q \otimes \mathbf{g}_q\right) \\
% % % & \Tilde{\c}^{(L)}_q =  \sigma' \D \left(\mathbf{e}_q \otimes \mathbf{g}_q\right) = \sigma' \D_q\g_q
% % % \end{align}
% % % Dividing by $\|\Tilde{\c}^{(L)}_q\|$ and taking the limit as L goes to infinity we will find the optimal $\c_q$. i.e. 

% % % \begin{equation}
% % %     \c_q = \lim_{L\longrightarrow \infty} \frac{\Tilde{\c}^{(L)}_q}{L\|\Tilde{\c}^{(L)}_q\|}
% % % \end{equation}

% % % We aim to find $\g_q$ such that

% % % \begin{align}
% % % & \left\| abs(\D^H \gamma' \D_q \mathbf{g}_q)- abs(\gamma \left(\mathbf{e}_{q} \otimes \mathbf{g}_q\right)) \right\|_{2}^{2}
% % % \end{align}

% % % is minimized.

% % % % Is it possible to show that $\g_q$ is independent of $q$ (assume regular scenario)? If true, i.e., $\g_q = \g$ we have

% % % We will provide a suboptimal structure for the choice of $\g$. Let us assume $\g_q = \g, $  for all $ q \in \{1, \cdots, Q\}$. Then we can write
% % % \begin{align}
% % %      &\Tilde{\c_q}^{(L)} = D_q\mathbf{g} = \sum_{l=1}^{L}g_l\mathbf{d}_{M_t}(\psi_{q, l})  \nonumber\\
% % %      & = \sum_{l=1}^L g_l\left[{\begin{array}{cccc}
% % %      1& e^{j\psi_{q, l}} &\cdots & e^{j(M_t-1)\psi_{q, l}}\\
% % %      \end{array}}\right]^T  = \nonumber\\
% % %      & \left[{\begin{array}{cccc}
% % %      \sum_{l=1}^L g_l& \sum_{l=1}^L g_le^{j\psi_{q, l}} &\cdots & \sum_{l=1}^L g_le^{j(M_t-1)\psi_{q, l}}\\
% % %      \end{array}}\right]^T
% % %  \end{align}
 
 

% % %  We have for the $m$-th element of the vector $\Tilde{\c}_q$, i.e. $\Tilde{c}^{(m)}_q$  for $ 0 \leq m \leq M_t-1$

% % % \begin{equation}
% % %     \Tilde{c}_q^{(m)} = \lim_{L\longrightarrow \infty} \frac{1}{L}\sum_{l=0}^{L-1} g_l e^{j(m\psi_{q, l+1})}
% % % \end{equation}

% % % With a choice of $\g = \left[{\begin{array}{cccc} 1& \alpha^\eta &\cdots & \alpha^{\eta (L -1)} \end{array}}\right]^T$ where we set 
% % % % $\alpha = e^j(\frac{2\pi}{LQ})$
% % % $\alpha = e^j(\frac{\psi_{q}-\psi_{q-1}}{L})$ and suitable $\eta$ (to be determined later), we can write


% % % % \begin{equation}
% % % %     \Tilde{c}_q^{(m)} = \lim_{L\longrightarrow \infty} \frac{1}{L}\sum_{l=0}^{L-1} g_l e^{jm(\psi_{q-1}+\frac{2 \pi(\ell+0.5)}{LQ})}
% % % % \end{equation}

% % % \begin{equation}
% % %     \Tilde{c}_q^{(m)} = \lim_{L\longrightarrow \infty} \frac{1}{L}\sum_{l=0}^{L-1} g_l e^{jm(\psi_{q-1}+\frac{(\psi_{q}-\psi_{q-1})(\ell+0.5)}{L})}
% % % \end{equation}

% % % \begin{equation}
% % %     c_q^{(m)} = \lim_{L\longrightarrow \infty} \frac{1}{L}\sum_{l=0}^{L-1} \alpha^{(\eta+m) l} e^{jm(\psi_{q-1}+\frac{0.5(\psi_{q}-\psi_{q-1})}{L})} 
% % % \end{equation}


% % % \begin{equation}
% % %     c_q^{(m)} = e^{jm(\psi_{q-1})}\lim_{L\longrightarrow \infty} \frac{1}{L}\sum_{l=0}^{L-1} \alpha^{(\eta+m) l}   
% % % \end{equation}

% % % \begin{equation}
% % %     c_q^{(m)} = e^{jm(\psi_{q-1})} \int_{0}^{1} \alpha^{(\eta + m)Lx}dx
% % % \end{equation}


% % % % \begin{equation}
% % % %     c_q^{(m)} = e^{jm(\psi_{q-1})} \int_{0}^{1} e^{j\frac{2\pi(\eta + m)L}{LQ}x}dx
% % % % \end{equation}


% % % \begin{equation}
% % %     c_q^{(m)} = e^{jm(\psi_{q-1})} \int_{0}^{1} e^{j\xi x}dx
% % % \end{equation}

% % % where $\xi = (\psi_{q}-\psi_{q-1}) (\eta +m )$.


% % % \begin{equation}
% % %     c_q^{(m)} = e^{jm(\psi_{q-1})} \frac{e^{j\xi}-1}{j\xi}
% % % \end{equation}

% % % \begin{equation}
% % %     c_q^{(m)} = e^{j(m\psi_{q-1} + \frac{\xi}{2})} \frac{e^{j\xi/2}-e^{-j\xi/2}}{j\xi/2}
% % % \end{equation}

% % % \begin{equation}
% % %     c_q^{(m)} = e^{j(m\psi_{q-1} + \frac{\xi}{2})} sinc(\frac{\xi}{2\pi})
% % % \end{equation}

% % % where, $sinc(t) = \frac{sin(\pi t)}{\pi t}$.
















% % % % We first optimize for $\alpha$ by setting the derivative of \eqref{obj_func} to zero, to get: 

% % % % $$\hat{\alpha}=\frac{\sum_{q \in \mathcal{W}_k}\sqrt{\frac{Q}{M_{t}}}(\mathbf{F} \mathbf{v})^{H} \mathbf{D}_{q} \mathbf{g}_q}{\left\|\mathbf{D}^{H} \mathbf{F} \mathbf{v}\right\|_{2}^{2}}$$

% % % % Replacing the value of $\alpha$ as above we get to the following optimization problem. 

% % % % \begin{align}
% % % % \left(\mathbf{F}_{\mid \mathbf{g}}, \mathbf{v}_{\mid \mathbf{g}}\right) &=\underset{\mathbf{F}, \mathbf{v}}{\arg \max } \frac{\left|\sum_{q \in \mathcal{W}_k}\sqrt{\frac{2^{\mathbf{B}}}{M_{t}}}\left(\mathbf{D}_{q} \mathbf{g}_q\right)^{H} \mathbf{F} \mathbf{v}\right|^{2}}{\left\|\mathbf{D}^{H} \mathbf{F} \mathbf{v}\right\|_{2}^{2}} \nonumber \\
% % % % &=\underset{\mathbf{F}, \mathbf{v}}{\arg \max }\left|\sum_{q \in \mathcal{W}_k}\left(\mathbf{D}_{q} \mathbf{g}\right)^{H} \mathbf{F} \mathbf{v}\right|^{2}
% % % % \end{align}


% \subsection{UPA Codebook Design Prolem}

% Suppose an $M_h \times M_v$ UPA is placed at the $x-z$ plane, and let $M = M_h M_v$. Let the distance between the antennas that are placed parallel to $z$ axis, be $d_z$, and $d_x$ respectively. One can define,

% \begin{align}
% \mathbf{d}_{M_{a}}\left(\psi_{a}\right) = \left[1, e^{j \psi_{a}} \cdots e^{j\left(M_{a}-1\right) \psi_{a}}\right]^{T} \in \mathbb{C}^{M_{a}}
% \end{align}
% where, $\psi_{h}=\frac{2 \pi d_{x}}{\lambda} \sin \theta \cos \phi \text { and } \psi_{v}=\frac{2 \pi d_{z}}{\lambda} \sin \phi$, and $a$ takes value in the set $\{v, h\}$.  
% The array response vector is then defined as 

% \begin{align}
%     \mathbf{d}_{M}\left(\psi_{v}, \psi_{h}\right) =
%     \mathbf{d}_{M_{v}}\left(\psi_{v}\right) \otimes
%     \mathbf{d}_{M_{h}}\left(\psi_{h}\right)  \in \mathbb{C}^{M}
% \end{align}
% Let $\mathcal{B}_s$ be the angular range under cover. We have, 

% \begin{equation}
%     \mathcal{B}_{s} \doteq \left[-\phi^{\mathrm{B}}, \phi^{\mathrm{B}}\right)
%     \times \left[-\theta^{\mathrm{B}}, \theta^{\mathrm{B}}\right) \label{range_angle}
% \end{equation}

% Notation $\psi^{p,q}_{h}[1:L_h,1 :L_v], \psi^{p,q}_{v}[1:L_v]$

% Let us uniformly divide the angular range  into $Q=Q_{v} Q_{h}$ subregions. For each subregion we have, 

% $$ \mathcal{B}_{p, q} =  \omega_{\phi, p} \times \omega_{\theta, q}$$

% Let $\Delta_{\theta} =  2 \theta^{\mathrm{B}} / Q_{v}$, and $\Delta_\phi =  2 \phi^{\mathrm{B}} / Q_{h}$ be the length of each sub-interval along axis $\theta$, and $\phi$. We have $\omega_{\theta, q} = [\theta_{q-1}, \theta_q)$, and, $\omega_{\phi, p} = [\phi_{p-1},\phi_p)$ where  $\theta_q = -\theta^B + q\Delta_\theta$, and $\phi_p = -\phi^B + p\Delta_\phi$. 


% In the $(\psi_h, \psi_v)$ domain, we can rewrite equation \eqref{range_angle} as, 

% \begin{equation}
%     \mathcal{B}^\psi_{s} \doteq\left[-\psi_h^{\mathrm{B}}, \psi_h^{\mathrm{B}}\right) \times\left[-\psi_v^{\mathrm{B}}, \psi_v^{\mathrm{B}}\right)
% \end{equation}

% and each sub-region similarly can be rewritten in the $\psi$ domain as

% $$ \mathcal{B}^\psi_{ p, q} \doteq \nu_{v}^{p, q} \times \nu_{h}^ {p, q}$$

% % We can then write for each $v_{a, b}$, i.e the $b$-th sub-interval along axis $a$,

% % \begin{equation}
% %     \nu_{v, p} =-\psi_{a}^{\mathrm{B}}+(b-1)\Delta_{a}+\left[0, \Delta_{a}\right)
% % \end{equation}

% Define the reference gain at $(\psi_{h}, \psi_{v}) $ as, 

% \begin{equation}
%     G\left(\psi_{v}, \psi_{h}, \mathbf{c}\right) =\left|\left(\mathbf{d}_{M_{v}}\left(\psi_{v}\right)\otimes\mathbf{d}_{M_{h}}\left(\psi_{h}\right)  \right)^{H} \mathbf{c}\right|^{2}
% \end{equation}

% It is straightforward to show that, 

% \begin{equation}
%     \int_{-\pi}^{\pi} \int_{-\pi}^{\pi} G\left(\psi_{v}, \psi_{h}, \mathbf{c}\right) d \psi_{v} d \psi_{h}={(2 \pi)^{2}}
% \end{equation}

% Therefore, for every interval $\mathcal{B}_{ p, q}$ we can derive the ideal gain expression as, 

% \begin{equation}
%     G_{p, q}^{\text {ideal }}\left(\psi_{v}, \psi_{h}\right)=G_{p,q} \mathds{1}_{\mathcal{B}^\psi_{p, q}}\left(\psi_{v}, \psi_{h}\right)
% \end{equation}

% where $G_{q,p} = \frac{(2\pi)^2}{\delta_{p,q}}$, where $\delta_{p,q}$ is the area of the interval $\mathcal{B}^\psi_{p,q}$. We define the codebook design problem under the UPA structure as follows. 
% % where $G_{q,p} = \frac{(2\pi)^2}{\delta^{p,q}_h\delta^p_v}$, $\delta^p_v = \psi^p_v - \psi^{p-1}_v $, and $\delta^{q, p}_h = \psi^{q,p}_h - \psi^{q-1, p}_h $. Also, let us define $\delta_{q,p} = \delta^{q, p}_h\delta^p_v$. 



% \begin{align}
% & \c^{opt}_{p, q} = \nonumber\\&\underset{\c, \|\c\|=1}{\arg \min } \iint_{\mathcal{B}_{s}}\left|G_{p, q}^{\text {ideal }}\left(\psi_{v}, \psi_{h}\right)-G\left(\psi_{v}, \psi_{h}, \c\right)\right| d \psi_{v} d \psi_{h} \label{init_opt}
% \end{align}

% By partitioning the range of $(\psi_h, \psi_v)$ into the pre-defined intervals, we can rewrite the optimization problem as follows, 

% \begin{align}
% & \c^{opt}_{p, q} = \underset{\c, \|\c\|=1}{\arg \min }\nonumber\\& \sum_{r=1}^{Q_v}\sum_{s=1}^{Q_h}\iint_{\mathcal{B}^{\psi}_{r,s}}\left|G_{p, q}^{\text {ideal }}\left(\psi_{v}, \psi_{h}\right)-G\left(\psi_{v}, \psi_{h}, \c\right)\right| d \psi_{v} d \psi_{h} \nonumber\\&
%  = \underset{L_h, L_v \rightarrow \infty}{\lim}\sum_{r=1}^{Q_v}\sum_{s=1}^{Q_h}\sum_{l_v =1}^{L_v}\sum_{l_h =1}^{L_h}\frac{\delta^r_v\delta^{r,s,l_v}_h}{L_hL_v}\nonumber\\&
% \left|G_{p, q}^{\text {ideal }}\left(\psi^{r,s}_{v}[l_v], \psi^{r,s}_{h}[l_v][l_h]\right)-G\left(\psi^{r,s}_{v}[l_v], \psi^{r,s}_{h}[l_v][l_h], \c\right)\right| \label{alternative_opt}
% \end{align}
% where, 

% \begin{align}
%     &\psi^{r,s}_v[l_v] = \psi^{r-1, s}_v + l_v\frac{\delta_v^r}{L_v}\nonumber\\
%     &\psi^{r,s}_h[l_v][l_h] = \psi^{r, s-1}_h[l_v] + l_h\frac{\delta_h^{r,s, l_v}}{L_h}
% \end{align}

% We can rewrite the optimization problem \eqref{alternative_opt} as, 

% \begin{align}
%     \c_{p,q}^{opt}=\arg \min _{\c, \|\c\|=1} \underset{L_h, L_v \rightarrow \infty}{\lim}\frac{1}{L_hL_v}\left\|\mathbf{G}^{\text {ideal }}_{p,q}-\mathbf{G}(\c)\right\|
% \end{align}


% where, 

% \begin{align}
%     &\mathbf{G}(\c) = \nonumber\\&\left[\delta^1_v\delta^{1,1,1}_hG\left(\psi_{v}^{1,1}[1], \psi_{h}^{1,1}[1][1], \c\right) \cdots \right. \nonumber\\ &\left. \delta^{Q_v}_v\delta^{Q_v,Q_h,L_v}_hG\left(\psi_{v}^{Q_{v}, Q_{h}}\left[L_{v}\right], \psi_{h}^{Q_{v}, Q_{h}}\left[L_{v}\right]\left[L_{h}\right], \c\right)\right]^{T} 
% \end{align}

% and, 

% \begin{align}
%     &\mathbf{G}^{\text {ideal }}_{ p, q} = \nonumber\\&\left[\delta^1_v\delta^{1,1,1}_hG^{\text {ideal }}_{ p, q}\left(\psi_{v}^{1,1}[1], \psi_{h}^{1,1}[1][1] \right) \cdots \right. \nonumber\\ &\left. \delta^{Q_v}_v\delta^{Q_v,Q_h,L_v}_hG^{\text {ideal }}_{ p, q}\left(\psi_{v}^{Q_{v}, Q_{h}}\left[L_{v}\right], \psi_{h}^{Q_{v}, Q_{h}}\left[L_{v}\right]\left[L_{h}\right]\right)\right]^{T} 
% \end{align}


% % \begin{align}
% %      &\mathbf{G}^{\text {ideal }}_{ q, p}= \nonumber\\&\left[{\delta_{1,1}} G^{\text {ideal }} _{q, p}\left(\psi^{1}_{h}[1], \psi^{1}_{v}[1], \c\right) \ldots {\delta_{Q_h, Q_v}}G^{\text {ideal }} _{q, p}\left(\psi^{Q_h}_{h}[L_h], \psi^{Q_v}_{v}[L_v], \c\right)\right]^{T} 
% % \end{align}


% Note that we can write, 

% \begin{align}
%     \mathbf{G}^{\text{ideal }}_{p,q} = \frac{{(2\pi)}^2}{\delta_{p,q}} \left(\e_{p,q} \otimes (\boldsymbol{\delta}_{p,q}\odot\mathbf{1}_{L, 1})\right)
% \end{align}

% where, $\boldsymbol\delta_{p,q} = $
% with $\mathbf{e}_{p,q} \in \mathbb{Z}^{Q}$ being the standard basis vector for the $(p,q)$-th axis among $Q$ ones. 








% Now, note that $\mathbf{1}_{L, 1}=\mathbf{g} \odot \mathbf{g}^{*}$ for any equal gain $\mathbf{g} \in \mathbb{C}^L$. Therefore, we can write: 

% % $$\mathcal{G}_{L}=\left\{\mathbf{g} \in \mathbb{C}^{L}:\left(\mathbf{g} \mathbf{g}^{H}\right)_{\ell, \ell}=1,1 \leq \ell \leq L\right\}$$


% \begin{align}
% \mathbf{G}^{\text {ideal }}_{p,q} &=\frac{{2\pi}^2}{\delta_{p,q}}\left(\mathbf{e}_{p,q} \otimes\left((\boldsymbol\delta^{(\frac{1}{2})}_{p,q}\odot\boldsymbol\delta^{(\frac{1}{2})}_{p,q})\odot(\mathbf{g} \odot \mathbf{g}^{*})\right)\right) \nonumber\\
% &=\left(\frac{{2\pi}}{\sqrt{\delta_{p,q}}}\left(\mathbf{e}_{p,q} \otimes (\boldsymbol\delta^{(\frac{1}{2})}_{p,q}\odot\mathbf{g})\right)\right) \nonumber\\&\odot\left(\frac{{(2\pi)}}{\sqrt{\delta_{p,q}}}\left(\mathbf{e}_{p,q} \otimes (\boldsymbol\delta^{(\frac{1}{2})}_{p,q}\odot\mathbf{g})\right)\right)^{*} \label{g_id_q_equivalent}
% \end{align}

% Also, note that we can write, 

% % \begin{align}
% %     \mathbf{G}(\c)=\left(\left(\mathbf{D}_{h}^{H} \otimes \mathbf{D}_{v}^{H}\right) \c\right) \odot\left(\left(\mathbf{D}_{h}^{H} \otimes \mathbf{D}_{v}^{H}\right) \c\right)^{*}
% % \end{align}
% \begin{align}
%     \mathbf{G}(\c)=\left(\D^H \c\right) \odot\left(\D^H \c\right)^{*}
% \end{align}

% where 

% \begin{align}
%     \D = \left[D^{1,1}_{vh}, \cdots, D^{Q_v,Q_h}_{vh}\right] \in \mathbb{C}^{M_vM_h\times L_vL_hQ_vQ_h}
% \end{align}

% % \begin{align}
% % % &\mathbf{D}_{v} =\left[\mathbf{D}_{v, 1}, \cdots, \mathbf{D}_{v, Q_{v}}\right] \in \mathbb{C}^{M_{v} \times L_{v} Q_{v}} \\
% % % &\mathbf{D}_{h} =\left[\mathbf{D}_{h, 1}, \cdots, \mathbf{D}_{h, Q_{h}}\right] \in \mathbb{C}^{M_{h} \times L_{h} Q_{h}} \\
% % % &\mathbf{D}_{v, p} =\left[\mathbf{d}_{M_{v}}\left(\psi_{v}^{p}[1]\right), \cdots, \mathbf{d}_{M_{v}}\left(\psi_{v}^{p}\left[L_{v}\right]\right)\right] \in \mathbb{C}^{M_{v} \times L_{v}}\\
% % % &\mathbf{D}_{h, q} =\left[\mathbf{d}_{M_{h}}\left(\psi_{h}^{q}[1]\right), \cdots, \mathbf{d}_{M_{h}}\left(\psi_{h}^{q}\left[L_{h}\right]\right)\right] \in \mathbb{C}^{M_{h} \times L_{h}}\\
% % &\mathbf{D}_{h, q}(\chi) =\left[\mathbf{d}_{M_{h}}\left(\psi_{h}^{q}[1], \chi \right), \cdots, \mathbf{d}_{M_{h}}\left(\psi_{h}^{q}\left[L_{h}\right], \chi \right)\right] \in \mathbb{C}^{M_{h} \times L_{h}}\\
% % &\mathbf{D}_{vh, p,q} =\left[\mathbf{d}_{M_{v}}\left(\psi_{v}^{p}[1]\right) \otimes \mathbf{D}_{h, q}(\psi_{v}^{p}[1]), \cdots, \mathbf{d}_{M_{v}}\left(\psi_{v}^{p}\left[L_{v}\right]\right) \otimes \mathbf{D}_{h, q}(\psi_{v}^{p}\left[L_{v}\right]) \right] \in \mathbb{C}^{M_{v}M_{h} \times L_{v}L_{h}}
% % \end{align}

% \begin{align}
% % &\mathbf{D}_{v} =\left[\mathbf{D}_{v, 1}, \cdots, \mathbf{D}_{v, Q_{v}}\right] \in \mathbb{C}^{M_{v} \times L_{v} Q_{v}} \\
% % &\mathbf{D}_{h} =\left[\mathbf{D}_{h, 1}, \cdots, \mathbf{D}_{h, Q_{h}}\right] \in \mathbb{C}^{M_{h} \times L_{h} Q_{h}} \\
% % &\mathbf{D}_{v, p} =\left[\mathbf{d}_{M_{v}}\left(\psi_{v}^{p}[1]\right), \cdots, \mathbf{d}_{M_{v}}\left(\psi_{v}^{p}\left[L_{v}\right]\right)\right] \in \mathbb{C}^{M_{v} \times L_{v}}\\
% % &\mathbf{D}_{h, q} =\left[\mathbf{d}_{M_{h}}\left(\psi_{h}^{q}[1]\right), \cdots, \mathbf{d}_{M_{h}}\left(\psi_{h}^{q}\left[L_{h}\right]\right)\right] \in \mathbb{C}^{M_{h} \times L_{h}}\\
% \mathbf{D}^{p,q}_{h}(\ell) =&\sqrt{\delta_v^{p}\delta^{p,q,\ell}_h}\left[\mathbf{d}_{M_{h}}\left(\psi_{h}^{p,q}[\ell][1] \right), \cdots, \right.\nonumber\\&\left. \mathbf{d}_{M_{h}}\left(\psi_{h}^{p,q}\left[\ell] [L_{h} \right] \right)\right] \in \mathbb{C}^{M_{h} \times L_{h}}, 
% \end{align}

% \begin{align}
%     \mathbf{D}_{vh}^{p,q} =&\left[\mathbf{d}_{M_{v}}\left(\psi_{v}^{p,q}[1]\right) \otimes \mathbf{D}_{h}^{p,q}(1), \cdots, \right.\nonumber\\&\left. \mathbf{d}_{M_{v}}\left(\psi_{v}^{p,q}\left[L_{v}\right]\right) \otimes \mathbf{D}_{h}^{p,q}(L_{v}) \right] \in \mathbb{C}^{M_{v}M_{h} \times L_{v}L_{h}}
% \end{align}
% \begin{problem}
% Given an equal-gain vector $\g_{p,q} \in \mathbb{C}^L$, $(p,q) \in \{(1,1), \cdots, (Q_v, Q_h)\}$, find vector $\c_{p,q} \in \mathbb{C}^{M_t}$ such that
% \begin{align}
% &\c_{p,q}=\underset{\c, \|c\|=1}{\arg \min } \lim_{L\rightarrow \infty} \left\|\frac{{2\pi}}{\sqrt{\delta_{p,q}}}\left(\mathbf{e}_{p,q} \otimes (\boldsymbol\delta^{(\frac{1}{2})}_{L,1}\odot\mathbf{g}_{p,q})\right)- \D^H \c\right\|^{2} \label{obj_func}
% \end{align}
% \label{main_problem_UPA}
% \end{problem}


% Note that the solution to problem \ref{main_problem_UPA} is the limit of the sequence of solutions to a least-square optimization problem as $L$ goes to infinity. For each $L$ we find that,
%  \begin{align}
%  {\c}^{(L)}_{p,q} &= {\frac{{2\pi}}{\sqrt{\delta_{p,q}}}}(\D \D^H)^{-1} \D  \left(\mathbf{e}_{p,q} \otimes (\boldsymbol\delta^{(\frac{1}{2})}_{L,1}\odot\mathbf{g}_{p,q})\right) \nonumber\\& =\sigma \D^{p,q}(\boldsymbol\delta^{(\frac{1}{2})}_{L,1}\odot\mathbf{g}_{p,q})
% \end{align}
% where $\sigma = \frac{2\pi \sqrt{\delta_{p,q}}}{L\sum_{(q,p)= (1,1)}^{(Q_h, Q_v)} \delta_{p,q}}$, noting that it holds that, 

% Noting that it holds that, 

% \begin{align}
%     (\D\D^H) = \left(L\sum_{(p,q)= (1,1)}^{(Q_h, Q_v)}\sum_{(l_v,l_h)= (1,1)}^{(L_h, L_v)} \delta_{v}^p\delta^{p,q,l_v}_{h}\right)\I_{M_t}
% \end{align}

% \begin{problem}
% Find equal-gain $\g_{p,q} \in \mathbb{C}^L$, $(q,p) = (1,1), \ldots, (Q_h, Q_v)$, such that
% \begin{equation}
%  \g_{q,p} = \underset{\g}{\arg\min }\left\| abs(\D^H \c_{q,p})- {2\pi} abs(\mathbf{e}_{q,p} \otimes \mathbf{g})\right\|^{2} \label{g_final_eq}
% \end{equation} 
% where $abs(.)$ denotes the element-wise absolute value of a vector.
% \label{g_problem}
% \end{problem}


% \begin{proposition}
% The minimizer of \eqref{g_simple_final_eq} is in the form $\g_q = \left[{\begin{array}{cccc} 1& \alpha^\eta &\cdots & \alpha_h^{\eta (L_h -1)}\alpha_v^{\eta (L_v -1)} \end{array}}\right]^T$ for some $\eta$ where $\alpha_h = e^{j(\frac{\delta^{q,p}_h}{L_h})}$ and $\alpha_v = e^{j(\frac{\delta^p_v}{L_v})}$. \label{proposiiton_g}
% \end{proposition}

%  \begin{align}
%      {\c_{p,q}}^{(L)} & = \sigma \sum_{(l_v, l_h)=(1,1)}^{(L_v, L_h)}g_{p,q, l}\mathbf{d}_{M_t}\left(\psi^{p,q}_{v}[l_v], \psi^{p,q}_{h}[l_v][l_h]\right)  \nonumber\\
%      & = \sigma \sum_{l=1}^L g_{q,p,l}\left[{\begin{array}{ccc}
%      1 &\cdots & e^{j\left( (M_v-1)\psi^{p,q}_{v}[l_v] + (M_h-1)\psi^{p,q}_{h}[l_v][l_h]\right)}\\
%      \end{array}}\right]^T   
%     %  \nonumber\\& = \left[{\begin{array}{ccc}
%     %  \sigma\sum_{l=1}^L g_{q, p, l}& \cdots & \sigma\sum_{l=1}^L g_{q,p, l}e^{j\left((M_h-1)\psi^q_{h}[l] + (M_v-1)\psi^p_{v}[l]\right)}\\
%     %  \end{array}}\right]^T
% \end{align}

% We can then write 

% \begin{align}
%     c_{p,q, m_v, m_h} &= \lim_{L_h, L_v\rightarrow \infty} \frac{1}{L_hL_v}\sum_{(l_h, l_v)=(1,1)}^{(L_h, L_v)} \nonumber \\&\delta^p_v\delta_h^{p,q,l_v}g_{p,q, l_v, l_h}e^{j\left(m_v\psi^{p,q}_{v}[l_v]+ m_h\psi^{p,q}_{h}[l_v][l_h] \right)}
% \end{align}


% \begin{align}
%     c_{p,q, m_v, m_h} &=  \nonumber\\&\lim_{L_h, L_v\rightarrow \infty} \frac{1}{L_hL_v}\sum_{(l_h, l_v)=(1,1)}^{(L_h, L_v)} \delta^p_v\delta_h^{p,q,l_v}g_{p,q, l_v, l_h}\nonumber \\&e^{j\left(m_v(\psi^{p-1,q}_{v}+l_v\frac{\delta^p_v}{L_v})+ m_h(\psi^{p,q-1}_{h}[l_v] + l_h\frac{\delta^{p,q,l_v}_h}{L_h}) \right)}
% \end{align}


% \begin{align}
%     c_{p,q, m_v, m_h} &=  \lim_{ L_v\rightarrow \infty} \frac{\delta^p_v}{L_v}\sum_{l_v=1}^{L_v} e^{j(m_v(\psi^{p-1,q}_{v}+l_v\frac{\delta^p_v}{L_v}) + x\eta\frac{l_v}{L_v})} S^{(l_v)}
% \end{align}

% where, 

% \begin{align}
%     S^{(l_v)} &= \lim_{ L_h\rightarrow \infty}\frac{\delta^{p,q,l_v}_h}{L_h}\sum_{l_h=1}^{L_h}e^{j(m_h(\psi^{p,q-1}_{h}[l_v] + l_h\frac{\delta^{p,q,l_v}_h}{L_h}) + y\gamma \frac{l_h}{L_h})} \nonumber\\& = \delta^{p,q,l_v}_h e^{jm_h\psi^{p,q-1}_{h}[l_v]}sinc(\frac{\xi}{2\pi})
% \end{align}

% with $\xi = {\delta^{p,q,l_v}_h}{m} + y\gamma$. Therefore, it. holds that, 

% \begin{align}
%     c_{p,q, m_v, m_h} &=  \lim_{ L_v\rightarrow \infty} \frac{\delta^p_v}{L_v}\sum_{l_v=1}^{L_v} e^{j(m_v(\psi^{p-1,q}_{v}+l_v\frac{\delta^p_v}{L_v}) + x\eta\frac{l_v}{L_v})} \delta^{p,q,l_v}_h e^{jm_h\psi^{p,q-1}_{h}[l_v]}sinc(\frac{\xi}{2\pi})
% \end{align}
% \subsection{UPA Codebook Design Prolem}

Suppose an $M_h \times M_v$ UPA is placed at the $x-z$ plane, and let $M = M_h M_v$. Let the distance between the antennas that are placed parallel to $z$ axis, be $d_z$, and $d_x$ respectively. One can define,

\begin{align}
\mathbf{d}_{M_{a}}\left(\psi_{a}\right) = \left[1, e^{j \psi_{a}} \cdots e^{j\left(M_{a}-1\right) \psi_{a}}\right]^{T} \in \mathbb{C}^{M_{a}}
\end{align}
where, $\psi_{h}=\frac{2 \pi d_{x}}{\lambda} \sin \theta \cos \phi \text { and } \psi_{v}=\frac{2 \pi d_{z}}{\lambda} \sin \phi$, and $a$ takes value in the set $\{v, h\}$.  
The array response vector is then defined as 

\begin{align}
    \mathbf{d}_{M}\left(\psi_{v}, \psi_{h}\right) =
    \mathbf{d}_{M_{v}}\left(\psi_{v}\right) \otimes
    \mathbf{d}_{M_{h}}\left(\psi_{h}\right)  \in \mathbb{C}^{M}
\end{align}
Let $\mathcal{B}_s$ be the angular range under cover in the $(\psi_h, \psi_v)$ domain. We have, 

\begin{equation}
    \mathcal{B}^\psi_{s} \doteq\left[-\psi_h^{\mathrm{B}}, \psi_h^{\mathrm{B}}\right) \times\left[-\psi_v^{\mathrm{B}}, \psi_v^{\mathrm{B}}\right)
\end{equation}

% \begin{equation}
%     \mathcal{B}_{s} \doteq \left[-\phi^{\mathrm{B}}, \phi^{\mathrm{B}}\right)
%     \times \left[-\theta^{\mathrm{B}}, \theta^{\mathrm{B}}\right) \label{range_angle}
% \end{equation}

% Notation $\psi^{p,q}_{h}[1:L_h,1 :L_v], \psi^{p,q}_{v}[1:L_v]$

Let us uniformly divide the angular range  into $Q=Q_{v} Q_{h}$ subregions. For each subregion we have, 

% $$ \mathcal{B}_{p, q} =  \omega_{\phi, p} \times \omega_{\theta, q}$$

% Let $\Delta_{\theta} =  2 \theta^{\mathrm{B}} / Q_{v}$, and $\Delta_\phi =  2 \phi^{\mathrm{B}} / Q_{h}$ be the length of each sub-interval along axis $\theta$, and $\phi$. We have $\omega_{\theta, q} = [\theta_{q-1}, \theta_q)$, and, $\omega_{\phi, p} = [\phi_{p-1},\phi_p)$ where  $\theta_q = -\theta^B + q\Delta_\theta$, and $\phi_p = -\phi^B + p\Delta_\phi$. 


$$ \mathcal{B}_{ p, q} \doteq \nu_{v}^{p, q} \times \nu_{h}^ {p, q}$$

% We can then write for each $v_{a, b}$, i.e the $b$-th sub-interval along axis $a$,

% \begin{equation}
%     \nu_{v, p} =-\psi_{a}^{\mathrm{B}}+(b-1)\Delta_{a}+\left[0, \Delta_{a}\right)
% \end{equation}

Define the reference gain at $(\psi_{h}, \psi_{v}) $ as, 

\begin{equation}
    G\left(\psi_{v}, \psi_{h}, \mathbf{c}\right) =\left|\left(\mathbf{d}_{M_{v}}\left(\psi_{v}\right)\otimes\mathbf{d}_{M_{h}}\left(\psi_{h}\right)  \right)^{H} \mathbf{c}\right|^{2}
\end{equation}

It is straightforward to show that, 

\begin{equation}
    \int_{-\pi}^{\pi} \int_{-\pi}^{\pi} G\left(\psi_{v}, \psi_{h}, \mathbf{c}\right) d \psi_{v} d \psi_{h}={(2 \pi)^{2}}
\end{equation}

Therefore, for every interval $\mathcal{B}_{ p, q}$ we can derive the ideal gain expression as, 

\begin{equation}
    G_{p, q}^{\text {ideal }}\left(\psi_{v}, \psi_{h}\right)=G_{p,q} \mathds{1}_{\mathcal{B}^\psi_{p, q}}\left(\psi_{v}, \psi_{h}\right)
\end{equation}

where $G_{q,p} = \frac{(2\pi)^2}{\delta_{p,q}}$, where $\delta_{p,q}$ is the area of the interval $\mathcal{B}^\psi_{p,q}$. We can write $\delta_{p,q} = \delta_v\delta_h = (\frac{2\psi^B_v}{Q_v})(\frac{2\psi^B_h}{Q_h})$. Therefore, We define the codebook design problem under the UPA structure as follows. 
% where $G_{q,p} = \frac{(2\pi)^2}{\delta^{p,q}_h\delta^p_v}$, $\delta^p_v = \psi^p_v - \psi^{p-1}_v $, and $\delta^{q, p}_h = \psi^{q,p}_h - \psi^{q-1, p}_h $. Also, let us define $\delta_{q,p} = \delta^{q, p}_h\delta^p_v$. 



\begin{align}
& \c^{opt}_{p, q} = \nonumber\\&\underset{\c, \|\c\|=1}{\arg \min } \iint_{\mathcal{B}_{s}}\left|G_{p, q}^{\text {ideal }}\left(\psi_{v}, \psi_{h}\right)-G\left(\psi_{v}, \psi_{h}, \c\right)\right| d \psi_{v} d \psi_{h} \label{init_opt}
\end{align}

By partitioning the range of $(\psi_h, \psi_v)$ into the pre-defined intervals, we can rewrite the optimization problem as follows, 

\begin{align}
& \c^{opt}_{p, q} = \underset{\c, \|\c\|=1}{\arg \min }\nonumber\\& \sum_{r=1}^{Q_v}\sum_{s=1}^{Q_h}\iint_{\mathcal{B}^{\psi}_{r,s}}\left|G_{p, q}^{\text {ideal }}\left(\psi_{v}, \psi_{h}\right)-G\left(\psi_{v}, \psi_{h}, \c\right)\right| d \psi_{v} d \psi_{h} \nonumber\\&
 = \underset{L_h, L_v \rightarrow \infty}{\lim}\sum_{r=1}^{Q_v}\sum_{s=1}^{Q_h}\sum_{l_v =1}^{L_v}\sum_{l_h =1}^{L_h}\frac{\delta_v\delta_h}{L_hL_v}\nonumber\\&
\left|G_{p, q}^{\text {ideal }}\left(\psi^{r,s}_{v}[l_v], \psi^{r,s}_{h}[l_h]\right)-G\left(\psi^{r,s}_{v}[l_v], \psi^{r,s}_{h}[l_h], \c\right)\right| \label{alternative_opt}
\end{align}
where, 

\begin{align}
    &\psi^{r,s}_v[l_v] = \psi^{r-1, s}_v + l_v\frac{\delta_v}{L_v}\nonumber\\
    &\psi^{r,s}_h[l_h] = \psi^{r, s-1}_h + l_h\frac{\delta_h}{L_h}
\end{align}

We can rewrite the optimization problem \eqref{alternative_opt} as, 

\begin{align}
    \c_{p,q}^{opt}=\arg \min _{\c, \|\c\|=1} \underset{L_h, L_v \rightarrow \infty}{\lim}\frac{1}{L_hL_v}\left\|\mathbf{G}^{\text {ideal }}_{p,q}-\mathbf{G}(\c)\right\|
\end{align}


where, 

\begin{align}
    \mathbf{G}(\c) =& \delta_v\delta_h\left[G\left(\psi_{v}^{1,1}[1], \psi_{h}^{1,1}[1], \c\right) \cdots \right. \nonumber\\ &\left. G\left(\psi_{v}^{Q_{v}, Q_{h}}\left[L_{v}\right], \psi_{h}^{Q_{v}, Q_{h}}\left[L_{h}\right], \c\right)\right]^{T} 
\end{align}

and, 

\begin{align}
    \mathbf{G}^{\text {ideal }}_{ p, q} =& \delta_v\delta_h\left[G^{\text {ideal }}_{ p, q}\left(\psi_{v}^{1,1}[1], \psi_{h}^{1,1}[1] \right) \cdots \right. \nonumber\\ &\left. G^{\text {ideal }}_{ p, q}\left(\psi_{v}^{Q_{v}, Q_{h}}\left[L_{v}\right], \psi_{h}^{Q_{v}, Q_{h}}\left[L_{h}\right]\right)\right]^{T} 
\end{align}


% \begin{align}
%      &\mathbf{G}^{\text {ideal }}_{ q, p}= \nonumber\\&\left[{\delta_{1,1}} G^{\text {ideal }} _{q, p}\left(\psi^{1}_{h}[1], \psi^{1}_{v}[1], \c\right) \ldots {\delta_{Q_h, Q_v}}G^{\text {ideal }} _{q, p}\left(\psi^{Q_h}_{h}[L_h], \psi^{Q_v}_{v}[L_v], \c\right)\right]^{T} 
% \end{align}


Note that we can write, 

\begin{align}
    \mathbf{G}^{\text{ideal }}_{p,q} = \delta_v\delta_h\frac{{(2\pi)}^2}{\delta_v\delta_h} \left(\e_{p,q} \otimes \mathbf{1}_{L, 1}\right)
\end{align}


with $\mathbf{e}_{p,q} \in \mathbb{Z}^{Q}$ being the standard basis vector for the $(p,q)$-th axis among $Q$ ones. 








Now, note that $\mathbf{1}_{L, 1}=\mathbf{g} \odot \mathbf{g}^{*}$ for any equal gain $\mathbf{g} \in \mathbb{C}^L$. Therefore, we can write: 

% $$\mathcal{G}_{L}=\left\{\mathbf{g} \in \mathbb{C}^{L}:\left(\mathbf{g} \mathbf{g}^{H}\right)_{\ell, \ell}=1,1 \leq \ell \leq L\right\}$$


\begin{align}
\mathbf{G}^{\text {ideal }}_{p,q} &={{(2\pi)}^2}\left(\mathbf{e}_{p,q} \otimes(\mathbf{g} \odot \mathbf{g}^{*})\right) \nonumber\\
&=\left({{2\pi}}\left(\mathbf{e}_{p,q} \otimes \mathbf{g}\right)\right) \odot \left({{2\pi}}\left(\mathbf{e}_{p,q} \otimes \mathbf{g}\right)\right)^* \label{g_id_q_equivalent}
\end{align}

Also, note that we can write, 

% \begin{align}
%     \mathbf{G}(\c)=\left(\left(\mathbf{D}_{h}^{H} \otimes \mathbf{D}_{v}^{H}\right) \c\right) \odot\left(\left(\mathbf{D}_{h}^{H} \otimes \mathbf{D}_{v}^{H}\right) \c\right)^{*}
% \end{align}
\begin{align}
    \mathbf{G}(\c)=\left(\D^H \c\right) \odot\left(\D^H \c\right)^{*}
\end{align}

\noindent where, $\D^H = \sqrt{\delta_v\delta_h}(\mathbf{D}_{v}^{H} \otimes \mathbf{D}_{h}^{H})$, and for $a \in \{v,h\}$, we have, 

\begin{align}
\mathbf{D}_{a} &=\left[\mathbf{D}_{a, 1}, \cdots, \mathbf{D}_{a, Q_{a}}\right] \in \mathbb{C}^{M_{a} \times L_{a} Q_{a}} \\
\mathbf{D}_{a, b} &=\left[\mathbf{d}_{M_{a}}\left(\psi_{a}^{b}[1]\right), \cdots, \mathbf{d}_{M_{a}}\left(\psi_{a}^{b}\left[L_{a}\right]\right)\right] \in \mathbb{C}^{M_{a} \times L_{a}}
\end{align}




% where 

% \begin{align}
%     \D = \left[D^{1,1}_{vh}, \cdots, D^{Q_v,Q_h}_{vh}\right] \in \mathbb{C}^{M_vM_h\times L_vL_hQ_vQ_h}
% \end{align}

% \begin{align}
% % &\mathbf{D}_{v} =\left[\mathbf{D}_{v, 1}, \cdots, \mathbf{D}_{v, Q_{v}}\right] \in \mathbb{C}^{M_{v} \times L_{v} Q_{v}} \\
% % &\mathbf{D}_{h} =\left[\mathbf{D}_{h, 1}, \cdots, \mathbf{D}_{h, Q_{h}}\right] \in \mathbb{C}^{M_{h} \times L_{h} Q_{h}} \\
% % &\mathbf{D}_{v, p} =\left[\mathbf{d}_{M_{v}}\left(\psi_{v}^{p}[1]\right), \cdots, \mathbf{d}_{M_{v}}\left(\psi_{v}^{p}\left[L_{v}\right]\right)\right] \in \mathbb{C}^{M_{v} \times L_{v}}\\
% % &\mathbf{D}_{h, q} =\left[\mathbf{d}_{M_{h}}\left(\psi_{h}^{q}[1]\right), \cdots, \mathbf{d}_{M_{h}}\left(\psi_{h}^{q}\left[L_{h}\right]\right)\right] \in \mathbb{C}^{M_{h} \times L_{h}}\\
% &\mathbf{D}_{h, q}(\chi) =\left[\mathbf{d}_{M_{h}}\left(\psi_{h}^{q}[1], \chi \right), \cdots, \mathbf{d}_{M_{h}}\left(\psi_{h}^{q}\left[L_{h}\right], \chi \right)\right] \in \mathbb{C}^{M_{h} \times L_{h}}\\
% &\mathbf{D}_{vh, p,q} =\left[\mathbf{d}_{M_{v}}\left(\psi_{v}^{p}[1]\right) \otimes \mathbf{D}_{h, q}(\psi_{v}^{p}[1]), \cdots, \mathbf{d}_{M_{v}}\left(\psi_{v}^{p}\left[L_{v}\right]\right) \otimes \mathbf{D}_{h, q}(\psi_{v}^{p}\left[L_{v}\right]) \right] \in \mathbb{C}^{M_{v}M_{h} \times L_{v}L_{h}}
% \end{align}

% \begin{align}
% % &\mathbf{D}_{v} =\left[\mathbf{D}_{v, 1}, \cdots, \mathbf{D}_{v, Q_{v}}\right] \in \mathbb{C}^{M_{v} \times L_{v} Q_{v}} \\
% % &\mathbf{D}_{h} =\left[\mathbf{D}_{h, 1}, \cdots, \mathbf{D}_{h, Q_{h}}\right] \in \mathbb{C}^{M_{h} \times L_{h} Q_{h}} \\
% % &\mathbf{D}_{v, p} =\left[\mathbf{d}_{M_{v}}\left(\psi_{v}^{p}[1]\right), \cdots, \mathbf{d}_{M_{v}}\left(\psi_{v}^{p}\left[L_{v}\right]\right)\right] \in \mathbb{C}^{M_{v} \times L_{v}}\\
% % &\mathbf{D}_{h, q} =\left[\mathbf{d}_{M_{h}}\left(\psi_{h}^{q}[1]\right), \cdots, \mathbf{d}_{M_{h}}\left(\psi_{h}^{q}\left[L_{h}\right]\right)\right] \in \mathbb{C}^{M_{h} \times L_{h}}\\
% \mathbf{D}^{p,q}_{h}(\ell) =&\sqrt{\delta_v^{p}\delta^{p,q,\ell}_h}\left[\mathbf{d}_{M_{h}}\left(\psi_{h}^{p,q}[\ell][1] \right), \cdots, \right.\nonumber\\&\left. \mathbf{d}_{M_{h}}\left(\psi_{h}^{p,q}\left[\ell] [L_{h} \right] \right)\right] \in \mathbb{C}^{M_{h} \times L_{h}}, 
% \end{align}

% \begin{align}
%     \mathbf{D}_{vh}^{p,q} =&\left[\mathbf{d}_{M_{v}}\left(\psi_{v}^{p,q}[1]\right) \otimes \mathbf{D}_{h}^{p,q}(1), \cdots, \right.\nonumber\\&\left. \mathbf{d}_{M_{v}}\left(\psi_{v}^{p,q}\left[L_{v}\right]\right) \otimes \mathbf{D}_{h}^{p,q}(L_{v}) \right] \in \mathbb{C}^{M_{v}M_{h} \times L_{v}L_{h}}
% \end{align}

\begin{problem}
Given an equal-gain vector $\g_{p,q} \in \mathbb{C}^L$, $(p,q) \in \{(1,1), \cdots, (Q_v, Q_h)\}$, find vector $\c_{p,q} \in \mathbb{C}^{M_t}$ such that
\begin{align}
&\c_{p,q}=\underset{\c, \|c\|=1}{\arg \min } \lim_{L\rightarrow \infty} \left\|{{2\pi}}\left(\mathbf{e}_{p,q} \otimes \mathbf{g}\right)- \D^H \c\right\|^{2} \label{obj_func}
\end{align}
\label{main_problem_UPA}
\end{problem}


Note that the solution to problem \ref{main_problem_UPA} is the limit of the sequence of solutions to a least-square optimization problem as $L$ goes to infinity. For each $L$ we find that,
 \begin{align}
 {\c}^{(L)}_{p,q} &= {{{2\pi}}}(\D \D^H)^{-1} \D  \left(\mathbf{e}_{p,q} \otimes \mathbf{g}_{p,q}\right) \nonumber\\& =\sigma \D_{p,q}\mathbf{g}_{p,q}
\end{align}
where $\sigma = \frac{2\pi \sqrt{\delta_{v}\delta_{h}}}{LQ\delta_{v}\delta_{h}} = \frac{2\pi}{LQ\sqrt{\delta_v\delta_h}}$, noting that it holds that, 

Noting that it holds that, 

\begin{align}
    (\D\D^H) = \left(L\sum_{(p,q)= (1,1)}^{(Q_h, Q_v)}\sum_{(l_v,l_h)= (1,1)}^{(L_h, L_v)} \delta_{v}\delta_{h}\right)\I_{M_t}
\end{align}

\begin{problem}
Find equal-gain $\g_{p,q} \in \mathbb{C}^L$, $(p,q) = (1,1), \ldots, (Q_h, Q_v)$, such that
\begin{equation}
 \g_{q,p} = \underset{\g}{\arg\min }\left\| abs(\D^H \c_{p,q})- {2\pi} abs(\mathbf{e}_{p,q} \otimes \mathbf{g})\right\|^{2} \label{g_final_eq}
\end{equation} 
where $abs(.)$ denotes the element-wise absolute value of a vector.
\label{g_problem}
\end{problem}


\begin{proposition}
The minimizer of \eqref{g_simple_final_eq} is in the form $\g_q = \left[{\begin{array}{cccc} 1& \alpha^\eta &\cdots & \alpha_v^{\eta (L_v -1)}\alpha_h^{\eta (L_h -1)} \end{array}}\right]^T$ for some $\eta$ where $\alpha_v = e^{j(\frac{\eta_v}{L_v})}$ and $\alpha_h = e^{j(\frac{\eta_h}{L_h})}$. \label{proposiiton_g}
\end{proposition}

 \begin{align}
     {\c_{p,q}}^{(L)} & = \sigma \sum_{(l_v, l_h)=(1,1)}^{(L_v, L_h)}g_{p,q, l}\mathbf{d}_{M_t}\left(\psi^{p}_{v}[l_v], \psi^{q}_{h}[l_h]\right)  \nonumber\\
     & = \sigma \sum_{l=1}^L g_{q,p,l}\left[{\begin{array}{ccc}
     1 &\cdots & e^{j\phi_{p,q}^{M_v-1, M_h-1}}\\
     \end{array}}\right]^T   
    %  \nonumber\\& = \left[{\begin{array}{ccc}
    %  \sigma\sum_{l=1}^L g_{q, p, l}& \cdots & \sigma\sum_{l=1}^L g_{q,p, l}e^{j\left((M_h-1)\psi^q_{h}[l] + (M_v-1)\psi^p_{v}[l]\right)}\\
    %  \end{array}}\right]^T
\end{align}

where $ \phi_{p,q}^{m_v, m_h} = \left( m_v\psi^{p}_{v}[l_v] + m_h\psi^{q}_{h}[l_h]\right)$. We can then write 

\begin{align}
    c_{p,q, m_v, m_h} &= \nonumber\\& \lim_{L_h, L_v\rightarrow \infty} \frac{1}{L_hL_v}\sum_{(l_h, l_v)=(1,1)}^{(L_h, L_v)} g_{p,q, l_v, l_h}e^{j\phi_{p,q}^{M_v-1, M_h-1}}
\end{align}


\begin{align}
    c_{p,q, m_v, m_h} &=  \lim_{L_h, L_v\rightarrow \infty} \frac{1}{L_hL_v}\sum_{(l_h, l_v)=(1,1)}^{(L_h, L_v)} g_{p,q, l_v, l_h}\nonumber \\&e^{j\left(m_v(\psi^{p-1}_{v}+l_v\frac{\delta_v}{L_v})+ m_h(\psi^{q-1}_{h} + l_h\frac{\delta_h}{L_h}) \right)}
\end{align}


\begin{align}
    c_{p,q, m_v, m_h} =&  \frac{2\pi}{Q}e^{j\phi_{p-1, q-1}^{m_v, m_h}}
    \left(\frac{1}{L_v}\lim_{ L_v\rightarrow \infty} \sum_{l_v=1}^{L_v} e^{j\frac{\eta_v+ m_v\delta_v}{L_v} l_v}\right) \nonumber\\&
    \left(\frac{1}{L_h}\lim_{ L_h\rightarrow \infty} \sum_{l_h=1}^{L_h} e^{j\frac{\eta_h+ m_h\delta_h}{L_h} l_h}\right)
\end{align}

\begin{align}
    c_{p,q, m_v, m_h} &=  \frac{2\pi}{Q}e^{j\phi_{p-1, q-1}^{m_v, m_h}}
    \int_{0}^{1} e^{j\xi_v x}dx\int_{0}^{1} e^{j\xi_h x}dx \nonumber\\&
    = \frac{2\pi}{Q}e^{j(\phi_{p-1, q-1}^{m_v, m_h}+ \frac{\xi_v+\xi_h}{2})} sinc(\frac{\xi_v}{2\pi})sinc(\frac{\xi_h}{2\pi})
\end{align}

with $\xi_a = {\delta_a}{m_a} + \eta_a$, where $a\in \{v,h\}$. 
% Therefore, it. holds that, 

% \begin{align}
%     c_{p,q, m_v, m_h} &=  \lim_{ L_v\rightarrow \infty} \frac{\delta^p_v}{L_v}\sum_{l_v=1}^{L_v} e^{j(m_v(\psi^{p-1,q}_{v}+l_v\frac{\delta^p_v}{L_v}) + x\eta\frac{l_v}{L_v})} \delta^{p,q,l_v}_h e^{jm_h\psi^{p,q-1}_{h}[l_v]}sinc(\frac{\xi}{2\pi}) 
% \end{align}

% \subsection{UPA Dual Beamforming}

% Suppose an $M_h \times M_v$ UPA is placed at the $x-z$ plane, and let $M = M_h M_v$. Let the distance between the antennas that are placed parallel to $z$ axis, be $d_z$, and $d_x$ respectively. One can define,

% \begin{align}
% \mathbf{d}_{M_{a}}\left(\psi_{a}\right) = \left[1, e^{j \psi_{a}} \cdots e^{j\left(M_{a}-1\right) \psi_{a}}\right]^{T} \in \mathbb{C}^{M_{a}}
% \end{align}
% where, $\zeta=\frac{2 \pi d_{x}}{\lambda} \sin \theta \cos \phi \text { and } \xi=\frac{2 \pi d_{z}}{\lambda} \sin \phi$, and $a$ takes value in the set $\{v, h\}$.  
% The array response vector is then defined as 

% \begin{align}
%     \mathbf{d}_{M}\left(\xi, \zeta\right) =
%     \mathbf{d}_{M_{v}}\left(\xi\right) \otimes
%     \mathbf{d}_{M_{h}}\left(\zeta\right)  \in \mathbb{C}^{M}
% \end{align}
% Define the reference gain at $(\zeta, \xi) $ as, 

% \begin{equation}
%     G\left(\xi, \zeta, \mathbf{c}\right) =\left|\left(\mathbf{d}_{M_{v}}\left(\xi\right)\otimes\mathbf{d}_{M_{h}}\left(\zeta\right)  \right)^{H} \mathbf{c}\right|^{2}
% \end{equation}



Prior to formulating the multi-beamforming design problem, we proceed with a few preliminary definitions. Let us define the \emph{multi-beam}  $\mathcal{D} =(\mathcal{D}_1, \ldots \mathcal{D}_k)$ as collection of $k$ \emph{compound beams} $\mathcal{D}_i, i = 1,\ldots, k$ where $\mathcal{D}_i \subseteq \mathcal{B}^{\psi}$ and $\mathcal{D}_i = {\bigcup}_{{(p,q) \in \mathcal{A}_i}} \mathcal{B}^{\psi}_{p,q}$, with $\mathcal{A}_i$ being the set of all pairs $(p,q)$ that all beams $\mathcal{B}^{\psi}_{p,q}$ cover $\mathcal{D}_i$. The union of $\mathcal{B}^{\psi}_{p,q}$ is in fact approximating the shape of the solid angle for the desired compound beam corresponding to $\mathcal{D}_i$. By using larger number of division, i.e., finer beams, one can make the approximation better. We have 
\begin{align}
    &\mathcal{A}_i = \underset{\{\mathcal{\hat{A}}|\mathcal{D}_i \subseteq \underset{(p,q) \in \mathcal{A}}{\bigcup} \mathcal{B}_{p,q}\}}{\arg\min} |\mathcal{\hat{A}}|
\end{align}
% where, $i\in \{1,2\}$ and define further $\mathcal{A} = (\mathcal{A}_1, \mathcal{A}_2)$. 

Further define $\mathcal{A} = {{\bigcup}_{i=1}^k}\mathcal{A}_i$. 
%Let $\c$ denote the configuration of RIS (a.k.a beamformer) to control the gain and the phase of the exchanged signals, i.e. $c_{m_v, m_h} = \beta_{m_v, m_h}e^{j\theta_{m_v, m_h}}$, $m_v = 1\ldots M_v$, $m_h = 1\ldots M_h$. 
%
% Let $\c = \mbox{diag}( \Theta )$ denote a vector of length $M$ consisting of the diagonal elements of the matrix $\Theta$. For antenna element located at $(m_v, m_h)$ in the ULA grid, we define $c_{m_v, m_h} = \beta_{m_v, m_h}e^{j\theta_{m_v, m_h}}$, $m_v = 0, \ldots,  M_v-1$, $m_h = 0, \ldots, M_h-1$, and hence the vector $\c$ is given by
% \begin{equation}
%     \c = [c_{0,0}, \ldots, c_{0,M_h-1}, c_{1,0}, \ldots, c_{M_v-1, M_h-1}]
% \end{equation}
%
We aim to design a beamforming vector $\c$ such that the multi-beam $\mathcal{D}$ is covered when the RIS is excited by an incident wave received at solid angle $\Omega_1$. Using (\ref{channel})-(\ref{channel_t}), the contribution of the RIS in the channel matrix for a receiver at the solid angle $\Omega_2$ is given by
% \begin{align}
%     \Gamma = \d^{H}_{M}\{\Omega_{1}\} \Theta \d_{M}\{\Omega_{2}\} = \boldsymbol\lambda^H \d_{M}\{\Omega_{2}\} 
% \end{align}
\begin{align}
    \Gamma = \a^{H}_{M}(\Omega_2) \Theta \a_{M}(\Omega_1) = \d^H_M(\Omega_2) \boldsymbol{\lambda}  
\end{align}
where  $\boldsymbol{\lambda} \in \mathbb{C}^M$  is defined as follows. For antenna element located at position $(m_v, m_h)$ in the UPA grid,  we have 
\begin{align}
    \lambda_{m_v, m_h} = \beta_{m_v, m_h}e^{-j(\theta_{m_v, m_h}- m_v\xi_{1} - m_h\zeta_{1})}
\end{align} 
where $(\xi_{1}, \zeta_{1})$ is the representation of $\Omega_1$ in the $\psi$-domain, and hence the vector $\boldsymbol{\lambda}$ is given by
\begin{equation}
    \boldsymbol{\lambda} = [\lambda_{0,0}, \ldots, \lambda_{0,M_h-1}, \lambda_{1,0}, \ldots, \lambda_{M_v-1, M_h-1}]
\end{equation}

We note that $\boldsymbol{\lambda}$ depends on the AoA of the incident beams at the RIS, i.e., $\Omega_1$, as well as the RIS parameters. The reference gain of RIS in direction $(\zeta, \xi) $ in terms of $\boldsymbol{\lambda}$ is given by 
\begin{align}
    G\left(\xi, \zeta, \boldsymbol{\lambda} \right) =\left|\left(\mathbf{d}_{M_{v}}\left(\xi\right)\otimes\mathbf{d}_{M_{h}}\left(\zeta\right)  \right)^{H} \boldsymbol{\lambda} \right|^{2}
\end{align}

% The RIS parameters, i.e., the phase shift $\theta_m$ and the attenuation value $\beta_m$ for the $m^{th}$ elements of the RIS, are obtained by by using the corresponding coefficients of the vector $\boldsymbol{\lambda}$ and the directivity vector $\a_{M}(\Omega_{1})$ for any receive incident solid angle $\Omega_1$. To design $\boldsymbol{\lambda}$, we first define and work on the normalized beamforming vector $\c = \frac{\boldsymbol{\lambda}}{\|\boldsymbol{\lambda}\|}$, and then we compute $\boldsymbol{\lambda} = \frac{\c}{\|\c\|_{\infty}}$ based on the obtained value for $\c$. 

On the other hand, the gain of UPA antenna with the feed coefficients $\c$ is given by
\begin{equation}
    G\left(\xi, \zeta, \c \right) =\left|\left(\mathbf{d}_{M_{v}}\left(\xi\right)\otimes\mathbf{d}_{M_{h}}\left(\zeta\right)  \right)^{H} \c \right|^{2} \label{UPA_beamforming_c}
\end{equation}
that has a clear similarity.
%where $\|\c\| \leq 1$.
This means that to design the RIS-UPA for the STMR problem with receive zone $\mathcal{D}$ we can use the multi-beamforming design framework to cover the ACI's included in $\mathcal{D}$ for the UPA antenna. In particular, a RIS-UPA with parameters $\boldsymbol{\lambda}$ and a UPA-antenna with beamforming parameters $\c$ have the same beamforming gain pattern if UPA structures are the same and $\boldsymbol{\lambda}=\c$. Hence, a RIS-UPA which is excited from the solid angle $\Omega_1$ has the same beamforming gain as its UPA antenna counterpart if $\boldsymbol{\Theta}= \mbox{diag} \{\c^T \odot \a_M^H(\Omega_1)\}$.
%\amir{Please note that in the equation of the reference gain, $\d$ and $\c$ are both normalized with respect to the maximum values of the gain to be one. This means that the norm of every element of the directivity vector $\d$ is one. Also, the gain of every element of the beamforming coefficient is also less than or equal to one since the RIS is assumed to be passive.}
For any normalized beamforming vector $\c$, it is straightforward to show that, 
% Let us define the \emph{dual beam}  $ \mathcal{D} =(\mathcal{D}_1, \mathcal{D}_2) , \text{ given } \mathcal{D}_1 , \mathcal{D}_2\subseteq \mathcal{B}_s$. Let $\mathcal{A}_1$, and $\mathcal{A}_2$ denote the smallest set of index pairs $(p,q)$ the corresponding beams to which collectively cover the area marked by the desired beams $\mathcal{D}_1$ and $\mathcal{D}_2$ respectively. More precisely, we can write 
% where $c^{\mathrm{D}}_1 = (\phi^{\mathrm{D}}_1, \theta^{\mathrm{D}}_1)$ and $c^{\mathrm{D}}_2 = (\phi^{\mathrm{D}}_2, \theta^{\mathrm{D}}_2)$ are the directions of the centers of the two beams respectively.
% \begin{align}
%     &\mathcal{A}_i = \underset{\{\mathcal{A}|\mathcal{D}_i \subseteq \underset{(p,q) \in \mathcal{A}}{\bigcup} \mathcal{B}_{p,q}\}}{\arg\min} Card(\mathcal{A})
% \end{align}
% where, $i\in \{1,2\}$ and define further $\mathcal{A} = (\mathcal{A}_1, \mathcal{A}_2)$. 
\begin{equation}
    \int_{-\pi}^{\pi} \int_{-\pi}^{\pi} G\left(\xi, \zeta, \mathbf{c}\right) d \xi d \zeta={(2 \pi)^{2}}
\end{equation}

We wish to design beamformers that provide high, sharp, and constant gain within the desired ACI's and zero gain everywhere else. We have then for the ideal gain corresponding to such beamformer $\c$ that,
\begin{align}
&\iint_{\mathcal{B}^{\psi}} G^\text {ideal }_{\mathcal{D}}(\xi, \zeta) d \xi d \zeta =\sum_{i=1}^k\iint_{\mathcal{D}_i} t d \xi d \zeta \nonumber\\
&= \sum_{(p,q) \in \mathcal{A} }{\iint_{\mathcal{B}^\psi_{p,q}} t d \xi d \zeta} = \sum_{(p,q) \in \mathcal{A}}\delta_{p,q} t=(2 \pi)^2 \label{composite}
\end{align}

where $\delta_{p,q}$ denotes the area of the $(p,q)$-th beam in the $(\xi, \zeta)$ domain. Therefore, we can derive $t= \frac{(2\pi)^2}{|\mathcal{A}|\delta_{p,q}}$. It holds that, 
\begin{equation}
    G^{\text {ideal }}_{ \mathcal{D}}\left(\xi, \zeta\right)=\frac{(2\pi)^2}{|\mathcal{A}|\delta_{p,q}} \mathds{1}_{\mathcal{D}}\left(\xi, \zeta\right)\label{ideal_compound}
\end{equation}

Using the beamformer $\c$ we wish to mimic the deal gain in equation \eqref{ideal_compound}. Therefore, we formulate the following optimization problem, 
\begin{align}
& \c^{opt}_{\mathcal{D}} = \underset{\c, \|\c\|=1}{\arg \min } \underset{\mathcal{B}^{\psi}}{\iint}\left|G^{\text {ideal }}_{\mathcal{D}}\left(\xi, \zeta\right)-G\left(\xi, \zeta, \c\right)\right| d \xi d \zeta \label{init_opt}
\end{align}

By partitioning the range of $(\xi, \zeta)$ into the pre-defined intervals, and then uniformly sampling with the rate $(L_v, L_h)$ per interval along both axis,  we can rewrite the optimization problem as follows, 
% \begin{align}
% & \c^{opt}_{\mathcal{D}} = \underset{\c, \|\c\|=1}{\arg \min } \sum_{r=1}^{Q_v}\sum_{s=1}^{Q_h}\iint_{\mathcal{B}^{\psi}_{r,s}}\left|G_{\mathcal{D}}^{\text {ideal }}\left(\xi, \zeta\right)-G\left(\xi, \zeta, \c\right)\right| d \xi d \zeta \nonumber\\&
%  = \underset{L_h, L_v \rightarrow \infty}{\lim}\sum_{r=1}^{Q_v}\sum_{s=1}^{Q_h}\sum_{l_v =1}^{L_v}\sum_{l_h =1}^{L_h}\frac{\delta_v\delta_h}{L_hL_v}\nonumber\\&
% \left|G_{\mathcal{D}}^{\text {ideal }}\left(\psi^{r,s}_{v}[l_v], \psi^{r,s}_{h}[l_h]\right)-G\left(\psi^{r,s}_{v}[l_v], \psi^{r,s}_{h}[l_h], \c\right)\right| \label{alternative_opt}
% \end{align}
% where, 
% \begin{align}
%     & \psi^{r,s}_v = -\psi^{\mathrm{B}}_v + r\delta_v, \quad \psi^{r,s}_h = -\psi^{\mathrm{B}}_h + s\delta_h \\
%     &\psi^{r,s}_v[l_v] = \psi^{r-1, s}_v + l_v\frac{\delta_v}{L_v},  \quad \psi^{r,s}_h[l_h] = \psi^{r, s-1}_h + l_h\frac{\delta_h}{L_h} \label{l_vl_h}
% \end{align}
\begin{align}
 \c^{opt}_{\mathcal{D}} &= \underset{{\c, \|\c\|=1}}{{\arg \min }} \sum_{r=1}^{Q_v}\sum_{s=1}^{Q_h}\iint_{\mathcal{B}^{\psi}_{r,s}}\left|G_{\mathcal{D}}^{\text {ideal }}\left(\xi, \zeta \right)-G\left(\xi, \zeta, \c\right)\right| d \xi d \zeta \nonumber\\&
 = \underset{L_h, L_v \rightarrow \infty}{\lim}\sum_{r=1}^{Q_v}\sum_{s=1}^{Q_h}\sum_{l_v =1}^{L_v}\sum_{l_h =1}^{L_h}\nonumber\\&\frac{\delta_v\delta_h}{L_hL_v}
%\left|G_{\mathcal{D}}^{\text {ideal }}\left(\psi^{r,s}_{v}[l_v], \psi^{r,s}_{h}[l_h]\right)-G\left(\psi^{r,s}_{v}[l_v], \psi^{r,s}_{h}[l_h], \c\right)\right|
\left|G_{\mathcal{D}}^{\text {ideal }}\left(\xi_{r, l_v}, \zeta_{s,l_h}\right)-G\left(\xi_{r, l_v}, \zeta_{s,l_h}, \c\right)\right|\label{alternative_opt}
\end{align}
where, 
\begin{align}
    &\xi_{r,l_v} = \xi^{r-1} + l_v\frac{\delta_v}{L_v}, \quad \zeta_{s, l_h} = \zeta^{s-1} + l_h\frac{\delta_h}{L_h} \label{l_v_l_h}
\end{align}




\noindent with $\delta_a = \frac{2\psi_a^{\mathrm{B}}}{Q_a}$, for $a \in \{v, h\}$. Note that it holds for all $(p,q)$ pairs that, $\delta_{p,q} = \delta_v\delta_h$. 
We can rewrite equation \eqref{alternative_opt} as, 
\begin{align}
    \c_{\mathcal{D}}^{opt}=\arg \min _{\c, \|\c\|=1} \underset{L_h, L_v \rightarrow \infty}{\lim}\frac{1}{L_hL_v}\left|\mathbf{G}^{\text {ideal }}_{\mathcal{D}}-\mathbf{G}(\c)\right|\label{init_normed_opt}
\end{align}
where, 
\begin{align}
    \mathbf{G}(\c) =&  \delta_{p,q}\left[G\left(\xi_{1,1}, \zeta_{1,1}, \c\right) \cdots G\left(\xi_{Q_{v}, L_{v}}, \zeta_{Q_{h}, L_{h}}, \c\right)\right]^{T}  
\end{align}
and,
\begin{align}
    \mathbf{G}^{\text {ideal }}_{ \mathcal{D}} =& \delta_{p,q}\left[G^{\text {ideal }}_{ \mathcal{D}}\left(\xi_{1,1}, \zeta_{1,1} \right) \cdots  G^{\text {ideal }}_{ \mathcal{D}}\left(\xi_{Q_{v}, L_v}, \zeta_{ Q_{h}, L_{h}}\right)\right]^{T} 
\end{align}

% \begin{align}
%     \mathbf{G}(\c) =&  \delta_v\delta_h\left[G\left(\xi^{1}[1], \zeta^{1}[1], \c\right) \cdots \right. \nonumber\\ &\left. G\left(\xi^{Q_{v}}\left[L_{v}\right], \zeta^{Q_{h}}\left[L_{h}\right], \c\right)\right]^{T} 
% \end{align}

% and, 
% \begin{align}
%     \mathbf{G}^{\text {ideal }}_{ \mathcal{D}} =& \delta_v\delta_h\left[G^{\text {ideal }}_{ \mathcal{D}}\left(\xi^{1}[1], \zeta^{1}[1] \right) \cdots \right. \nonumber\\ &\left. G^{\text {ideal }}_{ \mathcal{D}}\left(\xi^{Q_{v}}\left[L_{v}\right], \zeta^{ Q_{h}}\left[L_{h}\right]\right)\right]^{T} 
% \end{align}


Unfortunately, the optimization problem in \eqref{init_normed_opt} does not admit an optimal closed-form solution as is, due to the absolute values of the complex numbers existing in the formulation. However, note that, 
\begin{align}
    \mathbf{G}^{\text {ideal }}_{\mathcal{D}}&=\sum_{(p,q) \in \mathcal{A}}\delta_{p,q}\frac{(2\pi)^2}{|\mathcal{A}|\delta_{p,q}}\left(\mathbf{e}_{p,q} \otimes \mathbf{1}_{L, 1}\right) \nonumber\\&= \frac{(2\pi)^2}{|\mathcal{A}|}\sum_{(p,q) \in \mathcal{A}}{\mathbf{e}_{p,q} \otimes \mathbf{1}_{L, 1}}
    \label{ideal}
\end{align}
with $\mathbf{e}_{p,q} \in \mathbb{Z}^{Q}$ being the standard basis vector for the $(p,q)$-th axis among $(Q_v, Q_h)$ pairs. Now, note that $\mathbf{1}_{L, 1}=\mathbf{g} \odot \mathbf{g}^{*}$ for any equal gain $\mathbf{g} \in \mathbb{C}^L$ where $L = L_hL_v$. An equal-gain  vector $\g \in \mathbb{C}^L$ is a vector where all elements have equal absolute values (in this case, equal to $1$). Therefore, we can write: 
% $$\mathcal{G}_{L}=\left\{\mathbf{g} \in \mathbb{C}^{L}:\left(\mathbf{g} \mathbf{g}^{H}\right)_{\ell, \ell}=1,1 \leq \ell \leq L\right\}$$
\begin{align}
\mathbf{G}^{\text {ideal }}_{\mathcal{D}} &= \sum_{(p,q) \in \mathcal{A}}\frac{(2\pi)^2}{|\mathcal{A}|}\left(\mathbf{e}_{p,q} \otimes\left(\mathbf{g} \odot \mathbf{g}^{*}\right)\right) \nonumber\\
&=\frac{(2\pi)^2}{|\mathcal{A}|}\sum_{(p,q) \in \mathcal{A}}\left(\mathbf{e}_{p,q} \otimes \mathbf{g}\right) \odot\left(\mathbf{e}_{p,q} \otimes \mathbf{g}\right)^{*} \nonumber\\
&=\left(\sum_{(p,q) \in \mathcal{A}}\frac{2\pi}{\sqrt{|\mathcal{A}|}}\left(\mathbf{e}_{p,q} \otimes \mathbf{g}\right)\right)  \nonumber\\&\odot \left(\sum_{(p,q) \in \mathcal{A}}\frac{2\pi}{\sqrt{|\mathcal{A}|}}\left(\mathbf{e}_{p,q} \otimes \mathbf{g}\right)\right)^* \label{final_gik}
\end{align}

Also, it is straightforward to write, 
% \begin{align}
%     \mathbf{G}(\c)=\left(\left(\mathbf{D}_{h}^{H} \otimes \mathbf{D}_{v}^{H}\right) \c\right) \odot\left(\left(\mathbf{D}_{h}^{H} \otimes \mathbf{D}_{v}^{H}\right) \c\right)^{*}
% \end{align}
\begin{align}
    \mathbf{G}(\c)=\left(\D^H \c\right) \odot\left(\D^H \c\right)^{*}\label{dc}
\end{align}

\noindent where, $\D^H = \sqrt{\delta_v\delta_h}(\mathbf{D}_{v}^{H} \otimes \mathbf{D}_{h}^{H})$, and for $a \in \{v,h\}$, and $b= 1\ldots Q_a$ we have, 
\begin{align}
\mathbf{D}_{a} &=\left[\mathbf{D}_{a, 1}, \cdots, \mathbf{D}_{a, Q_{a}}\right] \in \mathbb{C}^{M_{a} \times L_{a} Q_{a}}
% \mathbf{D}_{a, b} &=\left[\mathbf{d}_{M_{a}}\left(\psi_{a}^{b}[1]\right), \cdots, \mathbf{d}_{M_{a}}\left(\psi_{a}^{b}\left[L_{a}\right]\right)\right] \in \mathbb{C}^{M_{a} \times L_{a}}
\end{align}
where, 
\begin{align}
    &\mathbf{D}_{v, b} =\left[\mathbf{d}_{M_{v}}\left(\xi_{b,1}\right), \cdots, \mathbf{d}_{M_{v}}\left(\xi_{b, L_v}\right)\right] \in \mathbb{C}^{M_{v} \times L_{v}} \\
    &\mathbf{D}_{h, b} =\left[\mathbf{d}_{M_{h}}\left(\zeta_{b,1}\right), \cdots, \mathbf{d}_{M_{h}}\left(\zeta_{b, L_h}\right)\right] \in \mathbb{C}^{M_{h} \times L_{h}}
\end{align}

Comparing the expressions \eqref{init_normed_opt}, \eqref{final_gik}, and \eqref{dc}, one can show that the optimal choice of $\c_\mathcal{D}$ in \eqref{init_opt} is the solution to the following optimization problem for proper choices of $\g_{p,q}$. 

\begin{problem}
Given equal-gain vectors $\g_{p,q} \in \mathbb{C}^L$, for $(p,q) \in \mathcal{A}$  find vector $\c_{\mathcal{D}} \in \mathbb{C}^{M}$ such that
\begin{align}
\c_{\mathcal{D}}=&{\arg \min }_{\c, \|\c\|=1}\nonumber\\& \lim_{L\rightarrow \infty} \left\|\sum_{(p,q) \in \mathcal{A}}\frac{2\pi}{\sqrt{|\mathcal{A}|}}\left(\mathbf{e}_{p,q} \otimes \mathbf{g}_{p,q}\right)- \D^H \c\right\|^{2} \label{obj_func}
\end{align}
\label{main_problem_UPA}
\end{problem}

However, we now need to find the optimal choices of $\g_{p,q}$ that minimize the objective in \eqref{init_normed_opt}. Using \eqref{final_gik}, and \eqref{dc}, we have the following optimization problem.

\begin{problem}
Find equal-gain vectors $\g^*_{p,q} \in \mathbb{C}^L$, $(p,q) \in \mathcal{A}$ such that
\begin{align}
 &<\g^*_{p,q}>_{(p,q)\in\mathcal{A}} = \underset{<\g_{p,q}>_{(p,q)\in\mathcal{A}}}{\arg\min }\nonumber\\
 &\left\| abs(\D^H \c_{\mathcal{D}})- \frac{{2\pi}}{\sqrt{|\mathcal{A}|}} abs(\sum_{(p,q)\in \mathcal{A}}\mathbf{e}_{p,q} \otimes \mathbf{g}_{p,q})\right\|^{2} \label{g_final_eq}
\end{align} 
where $abs(.)$ denotes the element-wise absolute value of a vector.
\label{g_problem}
\end{problem}

Next, we continue with the solution of Problems ~\ref{main_problem_UPA}, and~\ref{g_problem}.
