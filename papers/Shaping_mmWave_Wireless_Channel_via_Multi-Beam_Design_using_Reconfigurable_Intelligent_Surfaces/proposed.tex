\section{Proposed Multi-beam Design}
\label{sec:proposed}

Note that the solution to problem \ref{main_problem_UPA} is the limit of the sequence of solutions to a least-square optimization problem as $L$ goes to infinity. For each $L$ we find that,
 \begin{align}
& {\c}^{(L)}_{\mathcal{D}} = \sum_{(p,q) \in \mathcal{A}}\frac{2\pi}{\sqrt{|\mathcal{A}|}}(\D \D^H)^{-1} \D  \left(\mathbf{e}_{p,q} \otimes \mathbf{g}_{p,q}\right) \\
& {\c}^{(L)}_{\mathcal{D}} = \sum_{(p,q) \in \mathcal{A}}\sigma \D_{p,q}\g_{p,q} \label{c_final_eq}
\end{align}
where $\sigma = \frac{{ 2\pi \sqrt{\delta_{v}\delta_{h}} }}{L Q\delta_{v}\delta_{h}\sqrt{|\mathcal{A}|}} = \frac{2\pi}{LQ\sqrt{\delta_{v}\delta_{h}|\mathcal{A}|}}$, noting that it holds that, 
\begin{align}
\D\D^H &= \delta_v\delta_h(\D_v \otimes\D_h)(\D_v^H \otimes \D_h^H) = \delta_v\delta_hLQ
\end{align}



% To solve the second problem, we present the following proposition without the formal proof. 
% \begin{proposition}
% The minimizer of \eqref{g_final_eq} is in the form of 
% $$\g^*_{p,q} = \left[{\begin{array}{cccc} 1& \alpha^\eta &\cdots & \alpha_v^{\eta (L_v -1)}\alpha_h^{\eta (L_h -1)} \end{array}}\right]^T, (p,q) \in \mathcal{A}$$ for some $\eta$ where $\alpha_a = e^{j(\frac{\eta_a}{L_a})}$, $a \in \{v, h\}$. \label{proposiiton_g}
% \end{proposition}

Even though Problem~\ref{main_problem_UPA} admits a nice analytical closed form solution, doing so for the Problem~\ref{g_problem} is not a trivial task, especially due to the fact that the objective function is not convex. However, the convexification of the objective problem \eqref{g_final_eq} in the form of 
\begin{align}
 %&<\g^*_{p,q}>_{(p,q)\in\mathcal{A}} = 
 \underset{<\g_{p,q}>_{(p,q)\in\mathcal{A}}}{\arg\min }
 \left\| \D^H \c_{\mathcal{D}}- \frac{{2\pi}}{\sqrt{|\mathcal{A}|}} \sum_{(p,q)\in \mathcal{A}}\mathbf{e}_{p,q} \otimes \mathbf{g}_{p,q} \right\|^{2} \label{g_final_eq_convexified}
\end{align} 
% \begin{align}
%  & <\g^*_{p,q}>_{(p,q)\in\mathcal{A}}=
% \nonumber\\
%  &\underset{<\g_{p,q}>_{(p,q)\in\mathcal{A}}}{\arg\min }\left\| \left(\kappa\D^H \D- \I_{LQ}\right) \sum_{(p,q)\in \mathcal{A}}\mathbf{e}_{p,q} \otimes \mathbf{g}_{p,q}\right\|^{2} \label{g_final_eq_convexified}
% \end{align} 
and using $\c_{\mathcal{D}}$ from \eqref{c_final_eq} leads to an effective solution for the original problem.
Indeed, it can be verified by solving the optimization problem \eqref{g_final_eq_convexified} numerically that a close-to-optimal solution admits the following form.
%Indeed, it can be verified by solving the optimization problem \eqref{g_final_eq_convexified} numerically that the solution is in the form \eqref{g_conjecture} which leads us to the following conjecture. 
% \begin{conjecture}
% The minimizer of \eqref{g_final_eq_convexified} is in the form of 
\begin{align}
    \g^*_{p,q} = \left[{ 1,  \alpha_v \alpha_h, \cdots,  \alpha_v^{ (L_v -1)}\alpha_h^{ (L_h -1)} }\right]^T, (p,q) \in \mathcal{A} \label{g_conjecture}
\end{align}
for some $\eta_v$, $\eta_h$  where $\alpha_a = e^{j(\frac{\eta_a}{L_a})}$, $a \in \{v, h\}$. \label{proposiiton_g}
% \end{conjecture}
%
% Except for some special cases, we have not been able to analytically prove this conjecture in its entirety. 
In the following, we use the the analytical form \eqref{g_conjecture} for $\g^*_{p,q}$ for the rest of our derivations. This solution would not be the optimal solution for the original problem \eqref{g_final_eq}.  However, it provides a near optimal solution with added benefits of allowing to (i) find the limit of the solution as $L$ goes to infinity, and (ii) express the beamforming vectors in closed form, as it will be revealed in the following discussion. An analytical closed form solution for $\c_\mathcal{D}$ can be found as follows. It holds that, 
%  \begin{align}
%      {\c_{\mathcal{D}}}^{(L)} & = \sigma \sum_{(p,q)\in \mathcal{A}}\left(\sum_{(l_v, l_h)=(1,1)}^{(L_v, L_h)}g_{p,q, l}\mathbf{d}_{M_t}\left(\psi^{p}_{v}[l_v], \psi^{q}_{h}[l_h]\right) \right) \nonumber\\
%      & = \sigma \sum_{(p,q)\in \mathcal{A}}\left(\sum_{l=1}^L g_{q,p,l}\left[{\begin{array}{ccc}
%      1 &\cdots & e^{j\mu_{p,q}^{M_v-1, M_h-1}}\\
%      \end{array}}\right]^T  \right) 
% \end{align}
\begin{align}
     &{\c_{\mathcal{D}}}^{(L)}  =  \sum_{(p,q)\in \mathcal{A}}\left(\sum_{(l_v, l_h)=(1,1)}^{(L_v, L_h)}\sigma g_{p,q, l_v, l_h}\mathbf{d}_{M_t}\left(\xi_{p,l_v}, \zeta_{q,l_h}\right) \right) \nonumber\\
     & = \sum_{(p,q)\in \mathcal{A}}\left(\sum_{(l_v, l_h)=(1,1)}^{(L_v, L_h)}  \sigma g_{q,p,l_v, l_h}\left[1, \cdots,  e^{j\mu_{p,q, l_v, l_h}^{M_v-1, M_h-1}}\right]^T  \right) 
\end{align}

\noindent where $ \mu_{p,q, l_v, l_h}^{m_v, m_h} = \left( m_v\xi_{p,l_v} + m_h\zeta_{q,l_h}\right)$. We can then write for the $(m_v, m_h)^{th}$ component of the beamformer $\c_{\mathcal{D}}$, 
% \begin{align}
%     &c_{p,q, m_v, m_h} = \nonumber\\& \lim_{L_h, L_v\rightarrow \infty} \frac{1}{L_hL_v}\sum_{(p,q)\in \mathcal{A}}\sum_{(l_h, l_v)=(1,1)}^{(L_h, L_v)} g_{p,q, l_v, l_h}e^{j\mu_{p,q}^{M_v-1, M_h-1}} \label{c_middle}
% \end{align}
\begin{align}
    &c_{p,q, m_v, m_h} \nonumber\\&=  \lim_{L_h, L_v\rightarrow \infty} \frac{1}{L_hL_v}\sum_{(p,q)\in \mathcal{A}}\sum_{(l_h, l_v)=(1,1)}^{(L_h, L_v)} g_{p,q, l_v, l_h}e^{j\mu_{p,q, l_v, l_h}^{M_v-1, M_h-1}} \label{c_middle}
\end{align}

Using equation \eqref{l_v_l_h}, we can rewrite \eqref{c_middle} as,
% \begin{align}
%     c_{p,q, m_v, m_h} &=  \lim_{L_h, L_v\rightarrow \infty} \frac{1}{L_hL_v}\sum_{(l_h, l_v)=(1,1)}^{(L_h, L_v)} g_{p,q, l_v, l_h}\nonumber \\&e^{j\left(m_v(\psi^{p-1}_{v}+l_v\frac{\delta_v}{L_v})+ m_h(\psi^{q-1}_{h} + l_h\frac{\delta_h}{L_h}) \right)}
% \end{align}
\begin{align}
    c_{p,q, m_v, m_h} =&  \frac{2\pi}{Q}e^{j\chi_{p-1, q-1}^{m_v, m_h}}
    \left(\frac{1}{L_v}\lim_{ L_v\rightarrow \infty} \sum_{l_v=1}^{L_v} e^{j\frac{\eta_v+ m_v\delta_v}{L_v} l_v}\right) \nonumber\\&
    \left(\frac{1}{L_h}\lim_{ L_h\rightarrow \infty} \sum_{l_h=1}^{L_h} e^{j\frac{\eta_h+ m_h\delta_h}{L_h} l_h}\right)
\end{align}

to get, 
\begin{align}
    c_{\mathcal{D}, m_v, m_h} &=  \sum_{(p,q) \in \mathcal{A}}\frac{2\pi}{Q}e^{j\chi_{p-1, q-1}^{m_v, m_h}}
    \int_{0}^{1} e^{j\xi_v x}dx\int_{0}^{1} e^{j\xi_h x}dx \nonumber\\&
    = \sum_{(p,q) \in \mathcal{A}}\frac{2\pi}{Q}e^{j(\zeta_{p-1, q-1}^{m_v, m_h}+ \frac{\xi_v+\xi_h}{2})} sinc(\frac{\xi_v}{2\pi})sinc(\frac{\xi_h}{2\pi})
\end{align}

\noindent with $ \chi_{p,q}^{m_v, m_h} = \left( m_v\xi^{p} + m_h\zeta^{q}\right)$, and $\xi_a = {\delta_a}{m_a} + \eta_a$,  for $a\in \{v,h\}$. Now that the closed-form expression for $\c_{\mathcal{D}}$, and therefore, $\boldsymbol{\lambda}$  is known, the RIS parameters at the antenna placed at location $(m_v, m_h)$ can be easily computed. More precisely, we get, 
\begin{align}
    &\beta_{m_v, m_h} = {|\c_{\mathcal{D}, m_v, m_h}|}\\
    & \theta_{m_v, m_h} = \phase{\c_{\mathcal{D}, m_v, m_h}} + m_v\psi_{v,1} + m_h\psi_{h,1}
\end{align}

In the case that gain control (attenuation) at the RIS elements is not feasible, $\beta_{m_v, m_h} = 1$ will be replaced by the derivation for the absolute value of the RIS parameters. Next, we verify the effectiveness of our multi-beamforming design approach by means of numerical experiments. 

% \amir{In order to find the RIS parameters, 
% element  at $(m_v, m_h)$ when $\c_\mathcal{D}$ is normalized such that the maximum gain among the elements is equal to one, given that both phase control and gain control at the passive RIS is allowed. Under the condition that controlling the gains of the RIS elements is not feasible, we only take $\phase{c_{\mathcal{D}, m_v, m_h}}$ as the beamformer configuration. Under the active mode, the gains of each RIS element can be amplified.}


% \begin{figure}
%     \centering
%     \includegraphics[width=\linewidth]{figures/ULA_2_Beam.jpg}
%     \caption{.}
%     \label{fig:ULA}
% \end{figure}