%% bare_conf.tex
%% V1.4a
%% 2014/09/17
%% by Michael Shell
%% See:
%% http://www.michaelshell.org/
%% for current contact information.
%%
%% This is a skeleton file demonstrating the use of IEEEtran.cls
%% (requires IEEEtran.cls version 1.8a or later) with an IEEE
%% conference paper.
%%
%% Support sites:
%% http://www.michaelshell.org/tex/ieeetran/
%% http://www.ctan.org/tex-archive/macros/latex/contrib/IEEEtran/
%% and
%% http://www.ieee.org/

%%*************************************************************************
%% Legal Notice:
%% This code is offered as-is without any warranty either expressed or
%% implied; without even the implied warranty of MERCHANTABILITY or
%% FITNESS FOR A PARTICULAR PURPOSE! 
%% User assumes all risk.
%% In no event shall IEEE or any contributor to this code be liable for
%% any damages or losses, including, but not limited to, incidental,
%% consequential, or any other damages, resulting from the use or misuse
%% of any information contained here.
%%
%% All comments are the opinions of their respective authors and are not
%% necessarily endorsed by the IEEE.
%%
%% This work is distributed under the LaTeX Project Public License (LPPL)
%% ( http://www.latex-project.org/ ) version 1.3, and may be freely used,
%% distributed and modified. A copy of the LPPL, version 1.3, is included
%% in the base LaTeX documentation of all distributions of LaTeX released
%% 2003/12/01 or later.
%% Retain all contribution notices and credits.
%% ** Modified files should be clearly indicated as such, including  **
%% ** renaming them and changing author support contact information. **
%%
%% File list of work: IEEEtran.cls, IEEEtran_HOWTO.pdf, bare_adv.tex,
%%                    bare_conf.tex, bare_jrnl.tex, bare_conf_compsoc.tex,
%%                    bare_jrnl_compsoc.tex, bare_jrnl_transmag.tex
%%*************************************************************************


% *** Authors should verify (and, if needed, correct) their LaTeX system  ***
% *** with the testflow diagnostic prior to trusting their LaTeX platform ***
% *** with production work. IEEE's font choices and paper sizes can       ***
% *** trigger bugs that do not appear when using other class files.       ***                          ***
% The testflow support page is at:
% http://www.michaelshell.org/tex/testflow/



\documentclass[conference]{IEEEtran}
\usepackage[utf8]{inputenc}
\usepackage{times,graphics, dsfont, epsfig,amsmath,xspace,endnotes,pifont,multirow,rotating,listings,amssymb,algorithmic,color,caption,nicefrac,adjustbox,tabularx,mathtools, algorithmic,algorithm}
\usepackage{graphicx}
% modifications for tabular
\usepackage{tabularx}

\newcolumntype{L}[1]{>{\raggedright\arraybackslash}p{#1}} 
\newcolumntype{C}[1]{>{\centering\arraybackslash}p{#1}} 
\newcolumntype{R}[1]{>{\raggedleft\arraybackslash}p{#1}} %
\usepackage{steinmetz}
\usepackage{color}
\usepackage[dvipsnames]{xcolor}
\usepackage{amsmath}
\usepackage{amssymb}
\usepackage{float}
\ifCLASSOPTIONcompsoc
    \usepackage[caption=false, font=normalsize, labelfont=sf, textfont=sf]{subfig}
\else
\usepackage[caption=false, font=footnotesize]{subfig}
\fi
\usepackage{graphicx}
\usepackage[letterpaper, left=0.625in, right=0.625in, bottom=1in, top=0.75in]{geometry}

\newcommand{\vect}[1]{\boldsymbol{#1}}

\usepackage{tikz}
\usepackage{amsthm}
\newtheorem{thm}{Theorem}
\newtheorem{lemma}[thm]{Lemma}
\newtheorem{pf}{Proof}
\newtheorem{defn}[thm]{Definition}
\newtheorem{problem}[thm]{Problem}
\newtheorem{conjecture}[thm]{Conjecture}
\newtheorem{proposition}[thm]{Proposition}


\renewcommand{\H} {{\bf{H}}}
\newcommand{\A} {{\bf{A}}}
\newcommand{\B} {{\bf{B}}}
\newcommand{\F} {{\bf{F}}}
\newcommand{\D} {{\bf{D}}}
\newcommand{\I} {{\bf{I}}}
\renewcommand{\b} {{\bf{b}}}
\renewcommand{\c} {{\bf{c}}}
\newcommand{\g} {{\bf{g}}}
\newcommand{\f} {{\bf{f}}}
\renewcommand{\v} {{\bf{v}}}
\newcommand{\e} {{\bf{e}}}
\newcommand{\h} {{\bf{h}}}
\renewcommand{\a} {{\bf{a}}}
\renewcommand{\r} {{\bf{r}}}
\renewcommand{\d} {{\bf{d}}}
\newcommand{\s} {{\bf{s}}}
\newcommand{\y} {{\bf{y}}}
\newcommand{\x} {{\bf{x}}}
\newcommand{\z} {{\bf{z}}}
\def\E{\mathbb E}

\newcommand{\nariman}[1]{\textcolor{red}{#1}}
%\newcommand{\amir}[1]{\textcolor{blue}{#1}}
\newcommand{\amir}[1]{\textcolor{blue}{#1}}


% author names and affiliations
% use a multiple column layout for up to three different
% affiliations
\author{
\IEEEauthorblockN{Nariman Torkzaban}
\IEEEauthorblockA{\textit{University of Maryland, College Park}\\
College Park, MD\\
narimant@umd.edu}
\and
\IEEEauthorblockN{Mohammad A. (Amir) Khojastepour}
\IEEEauthorblockA{\textit{NEC Laboratories,
America}\\
Princeton, NJ\\
amir@nec-labs.com}
% \and
% \IEEEauthorblockN{Nariman Torkzaban}
% \IEEEauthorblockA{\textit{University of Maryland, College Park}\\
% College Park, MD\\
% narimant@umd.edu}
}


% use for special paper notices
%\IEEEspecialpapernotice{(Invited Paper)}













\def\BibTeX{{\rm B\kern-.05em{\sc i\kern-.025em b}\kern-.08em
    T\kern-.1667em\lower.7ex\hbox{E}\kern-.125emX}}

\newcommand\copyrighttext{%
  \footnotesize \textcopyright 2021 IEEE. Personal use of this material is permitted.
  Permission from IEEE must be obtained for all other uses, in any current or future
  media, including reprinting/republishing this material for advertising or promotional
  purposes, creating new collective works, for resale or redistribution to servers or
  lists, or reuse of any copyrighted component of this work in other works.
%  DOI: \href{<http://tex.stackexchange.com>}{<DOI No.>}
  }
\newcommand\copyrightnotice{%
\begin{tikzpicture}[remember picture,overlay]
\node[anchor=south,yshift=10pt] at (current page.south) {\fbox{\parbox{\dimexpr\textwidth-\fboxsep-\fboxrule\relax}{\copyrighttext}}};
\end{tikzpicture}%
}
\begin{document}

\title{Shaping mmWave Wireless Channel via Multi-Beam Design using Reconfigurable Intelligent Surfaces}

%\title{RIS-enabled Coverage by Dual-beam design in mmWave communication}
%\title{Shaping mmWave Wireless Channel via Composite Beamforming Design}
%\title{RIS Discussion: Dual-beam design for covering blind spots in mmWave communication using RIS }
%\title{Virtualized Network Function Placement\\for Cellular Network Slicing}



% author names and affiliations
% use a multiple column layout for up to three different
% affiliations



% use for special paper notices
%\IEEEspecialpapernotice{(Invited Paper)}




% make the title area
\maketitle
\copyrightnotice
% As a general rule, do not put math, special symbols or citations
% in the abstract
\begin{abstract}
Millimeter-wave (mmWave) communications is considered as a key enabler towards the realization of next-generation wireless networks, due to the abundance of available spectrum at mmWave frequencies. However, mmWave suffers from high free-space path-loss and poor scattering resulting in mostly line-of-sight (LoS) channels which result in a lack of coverage. 
Reconfigurable intelligent surfaces (RIS), as a new paradigm, have the potential to fill the coverage holes by shaping the wireless channel. 
In this paper, we propose a novel approach for designing RIS with elements arranged in a uniform planar array (UPA) structure. In what we refer to as multi-beamforming, We propose and design RIS such that the reflected beam comprises multiple disjoint lobes. Moreover, the beams have optimized gain within the desired angular coverage with fairly sharp edges 
avoiding power leakage to other regions. We provide a closed-form low-complexity solution for the multi-beamforming design. We confirm our theoretical results by numerical analysis.  

%In this paper, we study the codebook design problem for hybrid beamforming. Considering a uniform linear array (ULA) of antennas, We first propose the structure of optimal corresponding beamformers. We show the ULA structure suffers from intrinsic inefficiencies limiting its performance in practical scenarios. To deal with these inefficiencies, we propose for the first time a twin ULA (TULA) structure of antennas that not only solves all the problems related to ULA, but also provides higher gains. We further extend our findings to develop more sophisticated antenna structures to boost the system performance. We both discuss the theory behind our approach, and show the effectiveness of our proposed codebooks by means of extensive simulations. 

%With the emergence of network slicing as a key enabler for the multitenancy paradigm, the corresponding resource allocation problems become of paramount importance. 


%Leveraging on Network Function Virtualization (NFV) and Software Defined Networking (SDN), we introduce exact methods for LTE network function (NF) placement. Towards this end we: (i) jointly optimize the virtualized RAN and EPC functions placement; and (ii) investigate trade-offs between optimizations tailored to different operator resource allocation policies.

\end{abstract}




\begin{IEEEkeywords}
 Beamforming, Reconfigurable Intelligent Surface (RIS), Uniform Planar Array (UPA), Blind-spot, MIMO
\end{IEEEkeywords}
% no keywords




% For peer review papers, you can put extra information on the cover
% page as needed:
% \ifCLASSOPTIONpeerreview
% \begin{center} \bfseries EDICS Category: 3-BBND \end{center}
% \fi
%
% For peerreview papers, this IEEEtran command inserts a page break and
% creates the second title. It will be ignored for other modes.
\IEEEpeerreviewmaketitle

\section{Introduction}


Next generation of wireless communication systems aims to address the ever-increasing demand for high throughput, low latency, better quality of service and ubiquitous coverage. The abundance of bandwidth available at the mmWave frequency range, i.e., $[20, 100]$ Ghz, is considered as a key enabler towards the realization of the promises of next generation wireless communication systems. However, communication in mmWave suffers from high path-loss, and poor scattering. Since the channel in mmWave is mostly LoS, i.e., a strong LoS path and very few and much weaker secondary components, the mmWave coverage map includes \emph{blind spots} as a result of shadowing and blockage. Beamforming is primarily used to address the high attenuation in the channel. In addition to beamforming, relaying can potentially be designed to generate constructive superposition and enhance the received signals at the receiving nodes. 
% Importance of mmWave in 5G and beyond systems.
% Channel in mmWave is mostly LOS. Blockage and shadowing generate blind spots.
% High attentuation in mmwave should be addressed. In addition to beamforming, relaying can generate constructive superposition and enhance the received signals for the users.
% Contyributions are as follows.
Reconfigurable intelligent surface (RIS)\cite{Huang19}\cite{Liaskos18}\cite{Basar19} is a new paradigm with a great potential for stretching the coverage and enhancing the capacity of next-generation communication systems. Indeed, it is possible to shape the wireless channel by using RIS, e.g., by covering blind spots or providing diversity reception at a receiving node. In particular, passive RIS provide not only an energy-efficient solution but also a cost-effective one both in terms of the initial deployment cost and the operational costs. RIS are promising to be deployed in a wide range of communications scenarios and use-cases, such as high throughput MIMO communications\cite{Huang20}\cite{Nadeem20}, ad-hoc networks, e.g., UAV communications\cite{Li20}, physical layer security\cite{Maka20}, etc. % A comprehensive survey is conducted in \cite{Liu21} that \amir{enumerates} the fundamentals, opportunities and challenges of integrating RIS into wireless communication environments. 
Apart from the works focusing on theoretical performance analysis of RIS-enabled systems  \cite{Han19}\cite{Nadeem20}\cite{Jung19}, considerable amount of work has been dedicated to optimizing such an integration, mostly focusing on the phase optimization of RIS elements \cite{Abey20}\cite{Guo20}\cite{Di20}\cite{Ata20} to achieve various goals such as maximum received signal strength, maximum spectral efficiency,  etc. For more information on RIS, we refer the interested readers to \cite{Liu21} and the references therein.


% Importance of RIS in shaping the wireless channel, generating opportunities for coherent combining through relay, enhancing the coverage by acting like a deformable mirror to cover blind spots.

% In this work we provide novel approach to design RIS such that we can simultaneously optimize the received power at RIS and the propagated (reflected) signal off of RIS in arbitrary 3D (Asimuth and elevation) angles. The dual-beam design allows for optimizing both receive and transmit direction. Moreover, it allows to control the beamwidth and hence the gain of the receive and transmit beamforming vectors. The work can be easily extended to cover multiple transmit and receive directions in multi-beam instead of dual-beam.

\begin{figure}
    \centering
    \includegraphics[width=0.75\linewidth]{figures/scene2.eps}
    \caption{Filling the coverage gap in mmWave communications by utilizing Reconfigurable Intelligent Surfaces enabled by multi-beamforming }
    \label{fig:system}
\end{figure}

In this paper, we consider a communication scenario between a transmitter, e.g., the base station (BS), and terrestrial end-users through a passive RIS that reflects the received signal from the transmitter towards the users. Hence, the users that are otherwise in blind spots of network coverage, become capable of communicating with the base station through the RIS that is serving as a passive reflector (passive relay) maintaining  communication links to the BS and to the users. Given the geo-spatial variance among the locations of the end-users served by the same wireless system, the RIS may have to accommodate users that lie in distant angular intervals simultaneously, with satisfactory quality of service (QoS). In what we refer to as \emph{multi-beamforming}, we particularly address the design of beams consisting of multiple disjoint lobes using RIS in order to cover different blind spots using sharp and effective beam patterns. In the following, we summaries the main contributions of this paper: 
%The contributions of this paper are as follows:
\begin{itemize} 
    \item We design the parameters of the RIS to achieve multiple disjoint beams covering various ranges of solid angle. The designed beams are fairly sharp, have almost uniform gain in the desired angular coverage interval (ACI), and have negligible power transmitted outside the ACI.
    \item We formulate the multi-beamforming design as an optimization problem for which we derive the optimal solution.
    \item Thanks to the derived analytical closed form solutions for the optimal multi-beamforming design, the proposed solution bears very low computational complexity even for RIS with massive array size.
%    \amir{\item (iv) The proposed design can be used with arrays that are passive (whether it is only phased-controlled or both phase-and gain controlled) as well as RIS with active elements (that are capable of amplifying the reflected signal).}
%    \item (v) We identify the fact that if a passive RIS is capable of controlling the gain of its element (e.g., through attenuation), it can provide smoother gain in the desired ACI and also boost up the beamforming gain. The latter could be counter intuitive as the power radiated from every single element is not maximized due to attenuation, but in fact, to shape the beams it is essential to use controlled attenuation for signal reflected from different array elements of RIS in the superposition of the signals emitted by each RIS elements.
%    \item (vi) The multi-beamforming design inherently depends on the solid angle (say $\Omega_1$ in Fig.~\ref{fig:RIS}) at which the incident wave activates the RIS elements. The proposed beamforming design easily adapts to changes in $\Omega_1$ and we provide a visualization as how the beam would change in response to change in $\Omega_1$.
    \item Through numerical evaluation we show that by using passive RIS, multi-beamforming can simultaneously cover multiple ACIs. Moreover, multi-beamforming provides tens of dB power boost w.r.t. single-beam RIS design even when the single beam is designed optimally. 
\end{itemize}

\textbf{Notation} Throughout this paper, $\mathbb{C}$, $\mathbb{R}$, and $\mathbb{Z}$ denote the set of complex, real, and integer numbers, respectively,  $\mathcal{C N}\left(m, \sigma^{2}\right)$ denotes the circularly symmetric complex normal distribution with mean $m$ and variance $\sigma^{2}$, $[a, b]$ is the closed interval between $a$ and $b, \mathbf{1}_{a, b}$ is the $a \times b$ all ones matrix, $\mathbf{I}_{N}$ is the $N \times N$ identity matrix, $\mathds{1}_{[a, b)}$ is the indicator function, $\|\cdot\|$ is the $2$ -norm, $\|\cdot\|_{\infty}$ is the infinity-norm, $|\cdot|$ may denote cardinality if applied to a set and $1$-norm if applied to a vector, $\odot$ is the Hadamard product, $\otimes$ is the Kronecker product, $\mathbf{A}^{H},$ and $\mathbf{A}_{a, b}$ denote conjugate transpose, and $(a, b)^{t h}$ entry of $\A$ respectively.





The remainder of the paper is organized as follows. Section~\ref{sec:desc} describes the system model. In Section~\ref{sec:problem} we formulate the multi-beamforming design problem and propose our solutions in Section ~\ref{sec:proposed}. Section ~\ref{sec:evaluation} presents our evaluation results, and finally, we conclude in Section ~\ref{sec:conclusions}. %, we highlight our conclusions and discuss directions for future 


%In dual beam design, we jointly optimize a receive beam steering (beamforming) covering the 3D angular location of the base stations and a transmit beam steering covering the 3D angular location of the user (users) of interest.

%Shaping wireless channel: through relaying or covering the blind spots.



\section{System model} 
\label{sec:desc}
\begin{figure}
    \centering
%    \includegraphics[width=\linewidth]{figures/RIS_Structure.jpg}
    \includegraphics[width=0.8\linewidth]{figures/RIS_Structure.eps}
    \caption{System Model}
    \label{fig:RIS}
\end{figure}


\subsection{Channel Model}
Consider a communication system consisting of a multi-antenna BS with $M_t$ antenna elements as a transmitter and a multi-antenna receiver with $M_r$ antenna elements. The MIMO system is aided by a multi-element RIS consisting of $M$ elements arranged in $M_h \times M_v$ grid in the form of UPA  as shown in figure~\ref{fig:system} where $M_h$ and $M_v$ are the number of elements in the horizontal and vertical directions, respectively. The received signal $\y \in \mathbb{C}^{M_r}$ as a function of the transmitted signal $\x \in \mathbb{C}^{M_t}$ can be written as,
\begin{align}
    \y = (\H_r\boldsymbol\Theta\H_t)\x + \z \label{channel}
\end{align}

\noindent where $\z$ is the noise vector, with each element of $\z$ drawn from a complex Gaussian distribution $\mathcal{C N}\left(0, \sigma_n^{2}\right)$, $\H_t \in \mathbb{C}^{M\times M_t}$ and $\H_r \in \mathbb{C}^{M_r\times M}$ are the channel matrices between each party and the RIS. We assume that the RIS consists of elements for which both the phase $\theta_m$ and the gain $\beta_m$ (in form of attenuation of the reflected signal) of each element, say $m$, may be controlled and $\boldsymbol\Theta \in \mathbb{C}^{M\times M}$ is a diagonal matrix where the element $(m,m)$ denotes the coefficient $\beta_m e^{j \theta_m}$ of the $m^{th}$ element of the RIS. Assuming LoS channel model both between the transmitter and the RIS and between the RIS and the receiver and using the directivity vectors at the transmitter, the RIS, and the receiver, the effective channel matrices can be written as,
% \begin{align}
%     & \H_t = \d_{M_t}\{\Omega_t\}\rho_{t}\d^{H}_{M}\{\Omega_{1}\}\label{channel_t} \\
%     & \H_r = \d_{M}\{\Omega_{2}\}\rho_{ r}\d^{H}_{M_r}\{\Omega_{r}\} \label{channel_r}
% \end{align}
\begin{align}
    & \H_r = \a_{M_r}(\Omega_r)\rho_{r}\a^{H}_{M}(\Omega_{2})\label{channel_r} \\
    & \H_t = \a_{M}(\Omega_{1})\rho_{t}\a^{H}_{M_t}(\Omega_{t}) \label{channel_t}
\end{align}

\noindent
% where $\Omega_t$ and $\Omega_2$ are the solid angle of departure (AoD) of the transmitted beams from transmitter and the RIS and $\Omega_1$ and $\Omega_r$ are the solid angle of arrival (AoA) of the received beams at the RIS and the receiver, respectively. 
where $\a_M(\Omega)$ is the array response vector of an RIS with elements in a UPA structure (RIS-UPA), $\Omega_t$ and $\Omega_2$ are the solid angles of departure (AoD) of the transmitted beams from transmitter and the RIS and $\Omega_1$ and $\Omega_r$ are the solid angle of arrival (AoA) of the received beams at the RIS and the receiver, respectively. 
% The directivity RIS-UPA can be found in similar way to that of a UPA. For a solid angle $\Omega = (\phi, \theta)$ where $-\frac{\pi}{2} \leq \phi \leq \frac{\pi}{2}$ is the elevation angle and $-\pi \leq \theta \leq \pi$ is the azimuth angle, the gain of a UPA with $M$ elements is given by
% \begin{align}
%     G (\c, \Omega) = \left| \sum_{m=0}^{M_t-1} c_{m} e^{j \frac{2 \pi}{\lambda}[ \cos \phi \cos\theta, \cos\phi \sin \theta, \sin\phi]  \r_m} \right|^2  \label{init_gain} 
% \end{align}
% where the element $m$ is located at $\r_m$ and is excited by coefficients $c_m$. Hence, The directivity vector for an UPA at the solid angle $\Omega$ is defined as
% \begin{align}
%     \a_{M}( \Omega ) = \left[1, e^{j \frac{2 \pi}{\lambda}[ \cos \phi \cos\theta, \cos\phi \sin \theta, \sin\phi]  \r_1}, \ldots, \right. \nonumber \\
%     \left. e^{j \frac{2 \pi}{\lambda}[ \cos \phi \cos\theta, \cos\phi \sin \theta, \sin\phi]  \r_{M-1}} \right]^{T} \in \mathbb{C}^{M}
% \end{align}
% \amir{Please note that an array response vector has similar gain in the same direction in three dimensional space. However, the receive and transmit directions differ by $\pi$, hence, the array response vector for the transmitter and receiver are written as $\a^{H}_{M}(\Omega)$, and $\a_{M}(\Omega)$, respectively.  
% }
The gain of the LoS paths from the transmitter to the RIS and from the RIS to the receiver are denoted by $\rho_t$ and $\rho_r$, respectively. Note that the solid angle $\Omega_a$ specifies a pair of elevation and azimuth angles i.e. $(\phi_a, \theta_a)$, $a \in \{1,2, t, r\}$. Further, assuming no pairing between the RIS elements, $\boldsymbol\Theta$ will be a diagonal matrix specified as 
\begin{equation}
    \boldsymbol\Theta = \mbox{diag}\{[\beta_1 e^{j \theta_1}, \ldots, \beta_M e^{j \theta_M}]\}
\end{equation}

\noindent where $\beta_i \in [0,1]$ and $\theta_i \in [0, 2\pi]$.


%where $\Omega_t$, $\Omega_1$, $\Omega_2$, and $\Omega_r$ are the angle between the direction of the transmitted or received beam with respect to their respective antenna array or RIS. 



% Define 
% \begin{equation}
%     \h_i = [1, \alpha, \alpha^2, \ldots, \alpha^{(M-1)}]^T, \alpha = e^{\frac{-j2 \pi d \phi_i}{\lambda}}, i = 1,2
% \end{equation}
% \begin{equation}
%     \h_G = [1, \alpha, \alpha^2, \ldots, \alpha^{(M_G-1)}]^T, \alpha = e^{\frac{-j2 \pi d_G \phi_G}{\lambda}}
% \end{equation}
% \begin{equation}
%     \h_U = [1, \alpha, \alpha^2, \ldots, \alpha^{(M_U-1)}]^T, \alpha = e^{\frac{-j2 \pi d_U \phi_U}{\lambda}}
% \end{equation}
% \begin{equation}
%     \Theta = \mbox{diag}\{[\beta_1 e^{j \theta_1}, \ldots, \beta_M e^{j \theta_M}]\}
% \end{equation}


% The received signal $\y$ at G as a function of the transmitted signal $\x$ from U is given by

% \begin{equation}
%     \y = \h_G  \rho_G \h_2^H \Theta \h_1  \rho_U \h_U^H
%  %       \y = \h_G \otimes \rho_G \h_2^H \Theta \h_1 \otimes \rho_U \h_U^H
% \end{equation}



\subsection{RIS Model}

Suppose an RIS consisting of $M_h \times M_v$ antenna elements forming a UPA structure is placed at the $x$-$z$ plane, where $M = M_h M_v$ and $z$ axis corresponds to horizon. Let $d_z$, and $d_x$ denote the distance between the antennas elements in $z$ and $x$ axis, respectively. 
The directivity of a RIS-UPA can be found in similar way to that of a UPA. At a solid angle $\Omega = (\phi, \theta)$, we have,
\begin{align}
    \a_{M}( \Omega ) = \left[1, e^{j \frac{2 \pi}{\lambda}\r_{\Omega}  \r_1}, \ldots,  e^{j \frac{2 \pi}{\lambda}\r_{\Omega}  \r_{M-1}} \right]^{T} \in \mathbb{C}^{M} \label{directivity_first}
\end{align}
\noindent where respectively,  $\r_{\Omega} = [ \cos \phi \cos\theta, \cos\phi \sin \theta, \sin\phi]$, and $\r_m = (m_hd_x, 0, m_vd_z)$  denote the direction corresponding to the solid angle $\Omega$ and the location of the $m$-th RIS element corresponding to the antenna placed at the position $(m_v, m_h)$. Further, we define a transformation of variables as follows. For a solid angle $\Omega = [\phi, \theta]$, define $\psi = [\xi, \zeta]$ as follows,
\begin{align}
    \xi=\frac{2 \pi d_{z}}{\lambda} \sin \phi \text {,  }\quad \zeta=\frac{2 \pi d_{x}}{\lambda} \sin \theta \cos \phi \label{transformation}
\end{align} Introducing the new variables into equation~\eqref{directivity_first}, it is straightforward to write, 
\begin{align}
    \a_M(\Omega) = \mathbf{d}_{M}\left(\xi, \zeta\right) =
    \mathbf{d}_{M_{v}}\left(\xi\right) \otimes
    \mathbf{d}_{M_{h}}\left(\zeta\right)  \in \mathbb{C}^{M}
\end{align}

where we define for $a \in \{v,h\} $ the  directivity vectors $\d_{M_a}$ as follows, and denote by $\d_M$ the directivity vector corresponding to the RIS. 
% \begin{align}
% \mathbf{d}_{M_{a}}\left(\psi_{a}\right) = \left[1, e^{j \psi_{a}} \cdots e^{j\left(M_{a}-1\right) \psi_{a}}\right]^{T} \in \mathbb{C}^{M_{a}}
% \end{align}
\begin{align}
&\mathbf{d}_{M_{v}}\left(\xi\right) = \left[1, e^{j \xi} \cdots e^{j\left(M_{v}-1\right) \xi}\right]^{T} \in \mathbb{C}^{M_{v}}\nonumber\\
&\mathbf{d}_{M_{h}}\left(\zeta\right) = \left[1, e^{j \zeta} \cdots e^{j\left(M_{h}-1\right) \zeta}\right]^{T} \in \mathbb{C}^{M_{h}}\label{directivity_final}
\end{align}

% The array response vector, i.e., directivity vector, of the RIS for a solid angle $\Omega$, i.e., $\d_M\{\Omega\}$ as a function of $\xi$ and $\zeta$ is denoted by $\mathbf{d}_{M}\left(\xi, \zeta\right)$ and is given by 

% \begin{align}
%     \mathbf{d}_{M}\left(\xi, \zeta\right) =
%     \mathbf{d}_{M_{v}}\left(\xi\right) \otimes
%     \mathbf{d}_{M_{h}}\left(\zeta\right)  \in \mathbb{C}^{M}
% \end{align}
% where, $\zeta=\frac{2 \pi d_{x}}{\lambda} \sin \theta \cos \phi \text { and } \xi=\frac{2 \pi d_{z}}{\lambda} \sin \phi$ for a solid angle $\Omega = (\phi, \psi)$. 

Let $\mathcal{B}$ be the angular range under cover defined as

\begin{equation}
    \mathcal{B} \doteq \left[-\phi^{\mathrm{B}}, \phi^{\mathrm{B}}\right)
    \times \left[-\theta^{\mathrm{B}}, \theta^{\mathrm{B}}\right) \label{range_angle}
\end{equation}

We note that there is a one-to-one correspondence between the solid angle $\Omega = (\phi, \psi)$ and its representation after change of variable as $(\zeta, \xi)$. Accordingly, let $\mathcal{B}^{\psi}$ be the angular range under cover in the $(\zeta, \xi)$ domain given by 

\begin{equation}
    \mathcal{B}^\psi \doteq\left[-\xi^{\mathrm{B}}, \xi^{\mathrm{B}}\right) \times\left[-\zeta^{\mathrm{B}}, \zeta^{\mathrm{B}}\right)
\end{equation}
In this paper, we set $d_x = d_z = \frac{\lambda}{2}$, $\phi^{\mathrm{B}} = \frac{\pi}{4}$, and $\theta^{\mathrm{B}} = \frac{\pi}{2}$, hence $\xi \in [-\pi\frac{\sqrt{2}}{2}, \pi\frac{\sqrt{2}}{2})$, and $\zeta \in [-\pi, \pi)$. Note that, the dependence between variables $\xi$ and $\zeta$ can be resolved using the approximation in \cite{Song17}.
Let us uniformly divide $\mathcal{B}^{\psi}$  into $Q=Q_{v} Q_{h}$ subregions, where $Q_h$ and $Q_v$ are the number of division in horizontal and vertical directions, respectively. A subregion is denoted by  
$$ \mathcal{B}^{\psi}_{ p, q} \doteq \nu_{v}^{p, q} \times \nu_{h}^ {p, q}$$
\noindent where $\nu_{v}^{p} = [\xi^{p-1}, \xi^{p}]$, and $\nu_{h}^{q} = [\zeta^{q-1}, \zeta^{q}]$ defining,
\begin{align}
    & \xi^{p} = -\xi^{\mathrm{B}} + p\delta_v, \quad \zeta^{q} = -\zeta^{\mathrm{B}} + q\delta_h
\end{align}
%Note that, the dependence between variables $\xi$ and $\zeta$ can be resolved using the approximation in \cite{Song17}. 
\noindent with $\delta_v = \frac{2\xi^{\mathrm{B}}}{Q_v}$, and $\delta_h = \frac{2\zeta^{\mathrm{B}}}{Q_h}$. In the next section, we define the multi-beamforming design problem as the core of our proposed RIS structure. 


% One can define for the solid angle $\Omega = (\phi, \psi)$,

% \begin{align}
% \mathbf{d}_{M_{a}}\left(\psi_{a}\right) = \left[1, e^{j \psi_{a}} \cdots e^{j\left(M_{a}-1\right) \psi_{a}}\right]^{T} \in \mathbb{C}^{M_{a}}
% \end{align}
% where, $\zeta=\frac{2 \pi d_{x}}{\lambda} \sin \theta \cos \phi \text { and } \xi=\frac{2 \pi d_{z}}{\lambda} \sin \phi$, and $a\in\{v, h\}$.  
% The array response vector is then defined as 

% \begin{align}
%     \mathbf{d}_{M}\left(\xi, \zeta\right) =
%     \mathbf{d}_{M_{v}}\left(\xi\right) \otimes
%     \mathbf{d}_{M_{h}}\left(\zeta\right)  \in \mathbb{C}^{M}
% \end{align}

\section{Problem Formulation}
\label{sec:problem}

% \subsection{Composite Codebook Design Problem}

Let us define a \textit{composite beam} $\omega_k$ as a union of multiple disjoint, possibly non-neighboring beams $\nu_{q} \text { for } q \in \mathcal{Q} = \left\{1, \cdots, Q\right\}$. Let $\mathcal{C'}=\left\{\mathbf{c}_{1}, \cdots, \mathbf{c}_{K}\right\}$ be the codebook corresponding to the composite problem.

% $$\bigcup_{k = 1}^{2^{B'}}{\omega_k} = [-\pi, \pi], \quad \text{and} \quad \omega_k \cap \omega_l = \emptyset, \quad \forall k\neq l.$$

Moreover, define for each composite beam $\omega_k$ the set $\mathcal{W}_k \subseteq \mathcal{Q}$ to be the set of indices of the single beams that form $\omega_k$. i.e. $\mathcal{W}_k = \{q \in \mathcal{Q}: v_q \subseteq \omega_k\}$. 
For any codeword $\c$, it holds that, 

\begin{equation}
    \int_{-\pi}^{\pi} G(\psi, \mathbf{c}) d \psi=2 \pi\|\mathbf{c}\|^{2}=2 \pi
\end{equation}
We have then for the ideal gain corresponding to  each codeword $\textbf{c}_k$, 

\begin{align}
&\int_{-\pi}^{\pi} G_{\text {ideal }, k}(\psi) d \psi =\int_{\omega_{k}} t d \psi+\int_{[-\pi, \pi] \backslash \omega_{k}} 0 d \psi \nonumber\\
&= \sum_{q \in {\mathcal{W}_k}}{\int_{\nu_{q}} t d \psi} = \sum_{q \in \mathcal{W}_k}\delta_q t=2 \pi \label{composite}
\end{align}

Therefore, $t = \frac{2 \pi}{\Delta_k}$, where $\Delta_k = \sum_{q \in \mathcal{W}_k}\delta_q$. It follows that, 

\begin{equation}
    G_{\text {ideal }, k}(\psi)=\frac{2 \pi}{\Delta_k} \mathds{1}_{\omega_{k}}(\psi), \quad \psi \in[-\pi, \pi] \label{composite_ideal_gain}
\end{equation}

The new MSE problem for the $k$-th codeword can therefore be written as follows.



\begin{align}
\c_k^{opt} &=\underset{\c, \|\c\|=1}{\arg \min } \int_{-\pi}^{\pi}\left|G_{\text {ideal }, k}(\psi)-G(\psi, \c)\right| d \psi 
\end{align}




By uniformly sampling on the range of   we can rewrite
the optimization problem as follows,

\begin{align}
    \c_k^{opt} &= \nonumber \\&\underset{\c, \|\c\|=1}{\arg \min }\left[\lim _{L \rightarrow \infty} \sum_{p=1}^{Q} \delta_p\sum_{\ell=1}^{L} \frac{\left|G_{\text {ideal }, k}\left(\psi_{p, \ell}\right)-G\left(\psi_{p, \ell}, \c\right)\right|}{L}\right]\label{composite_MSE}
\end{align}

 where, for $q = 1 \ldots Q$, 
 $$\delta_q = \psi_{q}-\psi_{q-1}, \quad \psi_{q, \ell}=\psi_{q-1}+\frac{\delta_q(\ell-0.5)}{L} $$

We can rewrite equation \eqref{composite_MSE} as follows, 

\begin{align}
 &\c_k^{opt} = \underset{\c,  \|\c\|=1}{\arg \min } \lim_{L\rightarrow \infty}\frac{1}{L}\left\|\mathbf{G}_{\text {ideal }, k}-\mathbf{G}(\c)\right\| \label{init_opt}
 \end{align}
 
 where,  $$\mathbf{G}(\c)=\left[{\delta_1} G\left(\psi_{1,1}, \c\right) \ldots {\delta_{Q}}G\left(\psi_{Q, L}, \c\right)\right]^{T} \in \mathbb{Z}^{L Q}$$

and, 
$$ \mathbf{G}_{\text {ideal }, k}=\left[{\delta_1} G_{\text {ideal }, k}\left(\psi_{1,1}\right) \ldots {\delta_{Q}}G_{\text {ideal }, k}\left(\psi_{Q, L}\right)\right]^{T} \in \mathbb{Z}^{L Q}$$


Note that it holds that 

\begin{equation}
    \mathbf{G}_{\text {ideal },k}=\sum_{q \in \mathcal{W}_k}\delta_q\frac{2\pi}{\Delta_k}\left(\mathbf{e}_{q} \otimes \mathbf{1}_{L, 1}\right) \label{ideal}
\end{equation}

with $\mathbf{e}_{q} \in \mathbb{Z}^{Q}$ being the standard basis vector for the $q$-th axis among $Q$ ones. Now, note that $\mathbf{1}_{L, 1}=\mathbf{g} \odot \mathbf{g}^{*}$ for any equal gain $\mathbf{g} \in \mathbb{C}^L$. Therefore, for a suitable choice of $\g$ we can write: 

% $$\mathcal{G}_{L}=\left\{\mathbf{g} \in \mathbb{C}^{L}:\left(\mathbf{g} \mathbf{g}^{H}\right)_{\ell, \ell}=1,1 \leq \ell \leq L\right\}$$


\begin{align}
\mathbf{G}_{\text {ideal }, k} &= \sum_{q \in \mathcal{W}_k}\gamma_q\left(\mathbf{e}_{q} \otimes\left(\mathbf{g} \odot \mathbf{g}^{*}\right)\right) \nonumber\\
&=\sum_{q \in \mathcal{W}_k}\left(\sqrt{\gamma_q}\left(\mathbf{e}_{q} \otimes \mathbf{g}\right)\right) \odot\left(\sqrt{{\gamma_q}}\left(\mathbf{e}_{q} \otimes \mathbf{g}\right)\right)^{*} \nonumber\\
&=\left(\sum_{q \in \mathcal{W}_k}\sqrt{\gamma_q}\left(\mathbf{e}_{q} \otimes \mathbf{g}_q\right)\right)  \odot \left(\sum_{q \in \mathcal{W}_k}\sqrt{\gamma_q}\left(\mathbf{e}_{q} \otimes \mathbf{g}_q\right)\right)^{*} \label{final_gik}
\end{align}




















% \begin{align}
% \left(\mathbf{F}_{k}, \mathbf{v}_{k}\right) &=\underset{\mathbf{F}, \mathbf{v}}{\arg \min } \sum_{p=1}^{Q} \sum_{\ell=1}^{L}\left|G_{\text {ideal }, k}\left(\psi_{p, \ell}\right)-G\left(\psi_{p, \ell}, \mathbf{F} \mathbf{v}\right)\right|^{2} \nonumber\\
% &=\underset{\mathbf{F}, \mathbf{v}}{\arg \min }\lim_{L\longrightarrow \infty} \frac{1}{L}\left\|\mathbf{G}_{\text {ideal }, k}-\mathbf{G}(\mathbf{F} \mathbf{v})\right\|_{2}^{2} \label{init_opt}
% \end{align}

% where,  $$\mathbf{G}(\mathbf{F} \mathbf{v})=\left[G\left(\psi_{1,1}, \mathbf{F} \mathbf{v}\right) \cdots G\left(\psi_{Q, L}, \mathbf{F} \mathbf{v}\right)\right]^{T} \in \mathbb{Z}^{L Q}$$

% and, 
% $$ \mathbf{G}_{\text {ideal }, k}=\left[G_{\text {ideal }, k}\left(\psi_{1,1}\right) \cdots G_{\text {ideal }, k}\left(\psi_{Q, L}\right)\right]^{T} \in \mathbb{Z}^{L Q}$$

% Note that we can write using the definition of $\mathcal{W}_k$ that, 

% $$G_{\text {ideal }, k}\left(\psi_{p, \ell}\right)=\left\{\begin{array}{ll}
% \frac{Q}{|\mathcal{W}_k| M_{t}}, & p \in \mathcal{W}_k \\
% 0, & p \notin \mathcal{W}_k 
% \end{array},\right.$$

% Or in the matrix form, 

% \begin{align}
%     &\mathbf{G}_{\text {ideal }, k}=\frac{Q}{|\mathcal{W}_k| M_{t}}\left((\bigoplus_{q \in \mathcal{W}_k}{\mathbf{e}_{q}}) \otimes \mathbf{1}_{L, 1}\right)  \nonumber \\ 
%     & = \sum_{q \in \mathcal{W}_k}{\frac{Q}{|\mathcal{W}_k| M_{t}}\left(\mathbf{e}_{q} \otimes \mathbf{1}_{L, 1}\right)} \label{G_ideal_k}
% \end{align}

% with $\mathbf{e}_{q} \in \mathbb{Z}^{Q}$ being the standard basis vector for the $q$-th axis among $Q$ ones. Now, note that $\mathbf{1}_{L, 1}=\mathbf{g} \odot \mathbf{g}^{*}$ for any $\mathbf{g} \in \mathcal{G}_L$, where, 

% $$\mathcal{G}_{L}=\left\{\mathbf{g} \in \mathbb{C}^{L}:\left(\mathbf{g} \mathbf{g}^{H}\right)_{\ell, \ell}=1,1 \leq \ell \leq L\right\}$$



% Therefore, equation \eqref{G_ideal_k} can be rewritten as:
% \begin{align}
% &\mathbf{G}_{\text {ideal }, k} = \sum_{q \in \mathcal{W}_k}\frac{Q}{|\mathcal{W}_k| M_{t}}\left(\mathbf{e}_{q} \otimes\left(\mathbf{g}_q \odot \mathbf{g}_q^{*}\right)\right) \nonumber\\
% &=\sum_{q \in \mathcal{W}_k}\left(\sigma\left(\mathbf{e}_{q} \otimes \mathbf{g}_q\right)\right) \odot\left(\sigma\left(\mathbf{e}_{q} \otimes \mathbf{g}_q\right)\right)^{*} \nonumber\\
% &=\left(\sum_{q \in \mathcal{W}_k}\sigma\left(\mathbf{e}_{q} \otimes \mathbf{g}_q\right)\right)  \odot \left(\sum_{q \in \mathcal{W}_k}\sigma\left(\mathbf{e}_{q} \otimes \mathbf{g}_q\right)\right)^{*} \label{final_gik}
% \end{align}

% where $\sigma = \sqrt{\frac{Q}{|\mathcal{W}_k|M_{t}}}$. 

Similarly, it is straightforward to observe,


\begin{align}
\mathbf{G}(\c) &=\left(\mathbf{D}^{H} \c\right) \odot\left(\mathbf{D}^{H} \c\right)^{*} \label{dc}
\end{align}
where $\mathbf{D} =\left[\sqrt{\delta_1}\mathbf{D}_{1} \cdots \sqrt{\delta_{Q}}\mathbf{D}_{Q} \right] \in \mathbb{C}^{M_{t} \times L Q}$, and 
$\mathbf{D}_{q}=\left[\mathbf{d}_{M_{t}}\left(\psi_{q, 1}\right) \cdots \mathbf{d}_{M_{t}}\left(\psi_{q, L}\right)\right] \in \mathbb{C}^{M_{t} \times L}.$
 







% On the other hand note that $\mathbf{G}(\mathbf{F} \mathbf{v})$ can be written as,


% \begin{equation}
%     \mathbf{G}(\mathbf{F} \mathbf{v}) =\left(\mathbf{D}^{H} \mathbf{F} \mathbf{v}\right) \odot\left(\mathbf{D}^{H} \mathbf{F} \mathbf{v}\right)^{*}\label{dfv_decomp}
% \end{equation}

% where, 
% $$\mathbf{D} =\left[\mathbf{D}_{1} \cdots \mathbf{D}_{Q}\right] \in \mathbb{C}^{M_{t} \times L Q}$$

% and, $\mathbf{D}_{q}=\left[\mathbf{d}_{M_{t}}\left(\psi_{q, 1}\right) \cdots \mathbf{d}_{M_{t}}\left(\psi_{q, L}\right)\right] \in \mathbb{C}^{M_{t} \times L}$. 

Comparing the expressions \eqref{init_opt}, \eqref{final_gik}, and \eqref{dc}, one can show that the optimal choice of $\c_q$ in \eqref{init_opt} is the solution to the following optimization problem for a proper choice of $\g_q$. 
\begin{problem}
Given an equal-gain vector $\g_q \in \mathbb{C}^L$, $q \in \{1, \cdots, Q\}$, find vector $\c_q \in \mathbb{C}^{M_t}$ such that
\begin{align}
&\c_q=\underset{\c, \|c\|=1}{\arg \min } \lim_{L\rightarrow \infty} \left\|\sum_{q \in \mathcal{W}_k}\sqrt{\gamma_q}\left(\mathbf{e}_{q} \otimes \mathbf{g}_q\right)- \mathbf{D}^{H} \c\right\|^{2} \label{obj_func}
\end{align}
\label{main_problem}
\end{problem}


However, we now need to find the optimal choice of $\g_q$ that minimizes the objective in \eqref{init_opt}. Using \eqref{final_gik}, and \eqref{dc}, we have the following optimization problem.
\begin{problem}
Find a set of equal-gain $\g_q \in \mathbb{C}^L$, i.e. $\mathcal{G}_k$ such that
\begin{equation}
 \mathcal{G}_k = \underset{\mathcal{G}}{\arg\min }\left\| abs(\D^H \c_q)- abs(\sum_{q \in \mathcal{W}_k}\sqrt{\gamma_q}(\mathbf{e}_{q} \otimes \mathbf{g}_q))\right\|^{2} \label{g_final_eq}
\end{equation} 
where $\mathcal{G}_k = \{\g_q| q \in \mathcal{W}_k\}$, and $abs(.)$ denotes the element-wise absolute value of a vector.
\label{g_problem}
\end{problem}

Hence, the codebook design for a system with full-digital beamforming capability is found by solving Problem~\ref{main_problem} for proper choice of $\g_q$ obtained as a solution to Problem ~\ref{g_problem}. The codebook for Hybrid beamforming, is then found as
\begin{align}
    \underset{\mathbf{F}_{q},\mathbf{v}_{q}}{\arg\min } \|\mathbf{F}_{q}\mathbf{v}_{q} - \c_q\|^2
\end{align}
where the columns of $\F_q \in \mathbb{C}^{M_t \times N_{RF}}$ are equal-gain vectors and $\v_q \in \mathbb{C}^{N_{RF}}$. The solution may be obtained using simple, yet effective suboptimal algorithm such as orthogonal matching pursuit (OMP) \cite{love15}\cite{noh17}\cite{Hussain17}. In the next section we continue with the solution of Problem~\ref{main_problem}, and~\ref{g_problem}.





% Comparing the expressions in equations \eqref{final_gik}, and, \eqref{dc}, a perfect solution to the optimization problem in \eqref{init_opt}, could be a pair $\textbf{(F, v)}$, for which the following equality holds.  



% \begin{align}
%     \mathbf{D}^{H} \mathbf{F} \mathbf{v}=\sum_{q \in \mathcal{W}_k}\sigma\left(\mathbf{e}_{q} \otimes \mathbf{g}_q\right) \label{potential_answer}
% \end{align}

% However, we have that $M_t > N_{RF}$ and therefore, the matrix $\textbf{D}^H \textbf{F}$ may not be full-rank, resulting in the RHS of equation \eqref{potential_answer} being in the null space of LHS. Therefore, given some $\g_q \in \mathcal{G}_L$ we formulate the following two-step optimization problem to search for $\textbf{(F, v)}$ that gets as close as possible to the desired value. 

% \begin{problem}

% a) For all $q \in \{1, \cdots, Q\}$, given $\g_q \in \mathcal{G}_L$, find the unit-norm vector $\c_q \in \mathbb{C}^{M_t}$ such that
% \begin{align}
% &\c^{|\g_q}_q=\underset{\c}{\arg \min } \lim_{L\longrightarrow \infty} \frac{1}{L}\left\|\left(\sum_{q \in \mathcal{W}_k}\sigma\left(\mathbf{e}_{q} \otimes \mathbf{g}_q\right)\right)- \mathbf{D}^{H} \c\right\|_{2}^{2} \label{obj_func}
% \end{align}

% \noindent
% b) For every given $\c_q$, find $\F_q \in \mathbb{C}^{M_t \times N_{RF}}$ , and $\v_q \in \mathbb{C}^{N_RF}$ that satisfy $\mathbf{F}_{q}\mathbf{v}_{q} = \c_q$ the condition $\f_n \in \mathcal{B}_{M_t}$ as in equation \eqref{f_constant_gain}. 

% \begin{align}
%     \mathbf{F}_{\mid \g_q}\mathbf{v}_{\mid \g_q} = \c_q
% \end{align}

% \end{problem}



Note that the solution to problem \ref{main_problem} is the limit of the sequence of solutions to a least-square optimization problem as $L$ goes to infinity. For each $L$ we find that,
 \begin{align}
& {\c}^{(L)}_k = \sum_{q \in \mathcal{W}_k}\sqrt{\gamma_q}(\D \D^H)^{-1} \D  \left(\mathbf{e}_q \otimes \mathbf{g}_q\right) \\
& {\c}^{(L)}_k = \sum_{q \in \mathcal{W}_k}\sigma_q \D_q\g_q \label{c_final_eq}
\end{align}
where $\sigma_q = \frac{\sqrt{ 2\pi \delta_q \frac{\delta_q}{\Delta_k}}}{L \sum_{p=1}^{2^B}\delta_p} = \frac{\delta_q}{L\sqrt{2\pi\Delta_k}}$, noting that it holds that, 

% $$\mathbf{D D}^{H}=\left({L \sum_{p=1}^{2^B}\delta_p}\right) \mathbf{I}_{M_{t}}$$.
% % Dividing by $\|\Tilde{\c}^{(L)}_q\|$ and taking the limit as L goes to infinity we will find the optimal $\c_q$. i.e. 


Using \eqref{c_final_eq}, equation \eqref{g_final_eq} can now be rewritten as, 

\begin{align}
 &\mathcal{G}_k = \underset{\mathcal{G}}{\arg\min } \nonumber
 \\&\left\| abs(\D^H \D\sum_{q \in \mathcal{W}_k}\sigma_q(\mathbf{e}_{q} \otimes \mathbf{g}_q))-abs(\sum_{q \in \mathcal{W}_k}\sqrt{\gamma_q}(\mathbf{e}_{q} \otimes \mathbf{g}_q))\right\|^{2} \label{g_simple_final_eq}
\end{align} 

% where $\sigma' = \frac{1}{2\pi L}$. We will state the following theorem without formal proof.
\begin{proposition}
The maximizer of \eqref{g_simple_final_eq} is in the form $\g_q = \left[{\begin{array}{cccc} 1& \alpha^\eta &\cdots & \alpha^{\eta (L -1)} \end{array}}\right]^T$ for some $\eta$ where $\alpha = e^{j(\frac{\delta_q}{L})}$. \label{proposiiton_g}
\end{proposition}

An analytical closed form solution for $\c_q$ can be found as follows. We have, 
 \begin{align}
     {\c_k}^{(L)} & = \sum_{q \in \mathcal{W}_k} \sigma_q \left(\sum_{l=1}^{L}g_{q, l}\mathbf{d}_{M_t}(\psi_{q, l})\right)  \nonumber\\
     & = \sum_{q \in \mathcal{W}_k} \left(\sum_{l=1}^L \sigma_qg_q^{(l)}\left[{\begin{array}{ccc}
     1 &\cdots & e^{j(M_t-1)\psi_{q, l}}\\
     \end{array}}\right]^T \right)  \nonumber\\
     & = \sum_{q \in \mathcal{W}_k} \sigma_q \left[{\begin{array}{ccc}
     \sum_{l=1}^L g_{q, l}& \cdots & \sum_{l=1}^L g_{q, l}e^{j(M_t-1)\psi_{q, l}}\\
     \end{array}}\right]^T 
\end{align}

 
Let us write, 

\begin{equation}
    \c_k^{(L)} = \sum_{q \in \mathcal{W}_k} \c_q^{(L)}
\end{equation}

where, 

\begin{equation}
{\c_q}^{(L)}  = \sigma_q \sum_{l=1}^{L}g_{q, l}\mathbf{d}_{M_t}(\psi_{q, l})
\end{equation}


Choosing $\g_q$ as in proposition \ref{proposiiton_g},  the $m$-th element of the vector ${\c}_q = \lim_{L\rightarrow \infty}{\c_q}^{(L)} $, i.e. $c_{q,m}$, is given by 

\begin{equation}
    c_{q,m} = \frac{\delta_q}{\sqrt{2\pi}\Delta_k}\lim_{L\rightarrow \infty} \frac{1}{L}\sum_{l=0}^{L-1} g_{q, l} e^{j(m\psi_{q, l+1})}
\end{equation}

% With a choice of $\g_q = \left[{\begin{array}{cccc} 1& \alpha^\eta &\cdots & \alpha^{\eta (L -1)} \end{array}}\right]^T$ where we set 
% % $\alpha = e^j(\frac{2\pi}{L2^B})$
% $\alpha = e^{j(\frac{\delta_q}{L})}$ and suitable $\eta$ (to be determined later), we can write


% \begin{equation}
%     \Tilde{c}_q^{(m)} = \lim_{L\longrightarrow \infty} \frac{1}{L}\sum_{l=0}^{L-1} g_l e^{jm(\psi_{q-1}+\frac{2 \pi(\ell+0.5)}{L2^B})}
% \end{equation}

\begin{equation}
    c_{q,m} = \frac{\delta_q}{\sqrt{2\pi}\Delta_k}\lim_{L\rightarrow \infty} \frac{1}{L}\sum_{l=0}^{L-1} g_{q, l} e^{jm(\psi_{q-1}+\frac{\delta_q(\ell+0.5)}{L})}
\end{equation}

% After some basic manipulations we get, 
\begin{equation}
    c_{q,m} = \frac{\delta_q}{\sqrt{2\pi}\Delta_k}\lim_{L\rightarrow \infty} \frac{1}{L}\sum_{l=0}^{L-1} \alpha^{(\eta+m) l} e^{jm(\psi_{q-1}+\frac{0.5\delta_q}{L})} 
\end{equation}


\begin{equation}
    c_{q,m} = \frac{\delta_q}{\sqrt{2\pi}\Delta_k}e^{jm(\psi_{q-1})}\lim_{L\longrightarrow \infty} \frac{1}{L}\sum_{l=0}^{L-1} \alpha^{(\eta+m) l}   
\end{equation}

\begin{equation}
    c_{q,m} = \frac{\delta_q}{\sqrt{2\pi}\Delta_k}e^{jm(\psi_{q-1})} \int_{0}^{1} \alpha^{(\eta + m)Lx}dx
\end{equation}


\begin{equation}
    c_{q,m} = \frac{\delta_q}{\sqrt{2\pi}\Delta_k}e^{jm(\psi_{q-1})} \int_{0}^{1} e^{j\frac{2\pi(\eta + m)L}{L2^B}x}dx
\end{equation}


\begin{align}
 c_{q,m} &= \frac{\delta_q}{\sqrt{2\pi}\Delta_k}e^{jm\psi_{q-1}} \int_{0}^{1} e^{j\xi x}dx \nonumber \\
 &= \frac{\delta_q}{\sqrt{2\pi}\Delta_k}e^{j(m\psi_{q-1} + \frac{\xi}{2})} sinc(\frac{\xi}{2\pi}) \label{final_g}
\end{align}

where $\xi = \delta_q (\eta +m )$.


% \begin{equation}
%     c_{q,m} = e^{jm(\psi_{q-1})} \frac{e^{j\xi}-1}{j\xi}
% \end{equation}

% \begin{equation}
%     c_{q,m} = e^{j(m\psi_{q-1} + \frac{\xi}{2})} \frac{e^{j\xi/2}-e^{-j\xi/2}}{j\xi/2}
% \end{equation}

% \begin{equation}
%     c_{q,m} = e^{j(m\psi_{q-1} + \frac{\xi}{2})} sinc(\frac{\xi}{2\pi}) \label{final_g}
% \end{equation}






% Note that the solution to part \textit{(a)} is the limit of the sequence of solutions to a least-square optimization problem as $L$ goes to infinity.
%  \begin{align}
% & \Tilde{\c}^{(L)}_q = (\D \D^H)^{-1} \D \sigma \left(\mathbf{e}_q \otimes \mathbf{g}_q\right) \\
% & \Tilde{\c}^{(L)}_q =  \sigma' \D \left(\mathbf{e}_q \otimes \mathbf{g}_q\right) = \sigma' \D_q\g_q
% \end{align}
% Dividing by $\|\Tilde{\c}^{(L)}_q\|$ and taking the limit as L goes to infinity we will find the optimal $\c_q$. i.e. 

% \begin{equation}
%     \c_q = \lim_{L\longrightarrow \infty} \frac{\Tilde{\c}^{(L)}_q}{L\|\Tilde{\c}^{(L)}_q\|}
% \end{equation}

% We aim to find $\g_q$ such that

% \begin{align}
% & \left\| abs(\D^H \gamma' \D_q \mathbf{g}_q)- abs(\gamma \left(\mathbf{e}_{q} \otimes \mathbf{g}_q\right)) \right\|_{2}^{2}
% \end{align}

% is minimized.

% % Is it possible to show that $\g_q$ is independent of $q$ (assume regular scenario)? If true, i.e., $\g_q = \g$ we have

% We will provide a suboptimal structure for the choice of $\g$. Let us assume $\g_q = \g, $  for all $ q \in \{1, \cdots, Q\}$. Then we can write
% \begin{align}
%      &\Tilde{\c_q}^{(L)} = D_q\mathbf{g} = \sum_{l=1}^{L}g_l\mathbf{d}_{M_t}(\psi_{q, l})  \nonumber\\
%      & = \sum_{l=1}^L g_l\left[{\begin{array}{cccc}
%      1& e^{j\psi_{q, l}} &\cdots & e^{j(M_t-1)\psi_{q, l}}\\
%      \end{array}}\right]^T  = \nonumber\\
%      & \left[{\begin{array}{cccc}
%      \sum_{l=1}^L g_l& \sum_{l=1}^L g_le^{j\psi_{q, l}} &\cdots & \sum_{l=1}^L g_le^{j(M_t-1)\psi_{q, l}}\\
%      \end{array}}\right]^T
%  \end{align}
 
 

%  We have for the $m$-th element of the vector $\Tilde{\c}_q$, i.e. $\Tilde{c}^{(m)}_q$  for $ 0 \leq m \leq M_t-1$

% \begin{equation}
%     \Tilde{c}_q^{(m)} = \lim_{L\longrightarrow \infty} \frac{1}{L}\sum_{l=0}^{L-1} g_l e^{j(m\psi_{q, l+1})}
% \end{equation}

% With a choice of $\g = \left[{\begin{array}{cccc} 1& \alpha^\eta &\cdots & \alpha^{\eta (L -1)} \end{array}}\right]^T$ where we set 
% % $\alpha = e^j(\frac{2\pi}{LQ})$
% $\alpha = e^j(\frac{\psi_{q}-\psi_{q-1}}{L})$ and suitable $\eta$ (to be determined later), we can write


% % \begin{equation}
% %     \Tilde{c}_q^{(m)} = \lim_{L\longrightarrow \infty} \frac{1}{L}\sum_{l=0}^{L-1} g_l e^{jm(\psi_{q-1}+\frac{2 \pi(\ell+0.5)}{LQ})}
% % \end{equation}

% \begin{equation}
%     \Tilde{c}_q^{(m)} = \lim_{L\longrightarrow \infty} \frac{1}{L}\sum_{l=0}^{L-1} g_l e^{jm(\psi_{q-1}+\frac{(\psi_{q}-\psi_{q-1})(\ell+0.5)}{L})}
% \end{equation}

% \begin{equation}
%     c_q^{(m)} = \lim_{L\longrightarrow \infty} \frac{1}{L}\sum_{l=0}^{L-1} \alpha^{(\eta+m) l} e^{jm(\psi_{q-1}+\frac{0.5(\psi_{q}-\psi_{q-1})}{L})} 
% \end{equation}


% \begin{equation}
%     c_q^{(m)} = e^{jm(\psi_{q-1})}\lim_{L\longrightarrow \infty} \frac{1}{L}\sum_{l=0}^{L-1} \alpha^{(\eta+m) l}   
% \end{equation}

% \begin{equation}
%     c_q^{(m)} = e^{jm(\psi_{q-1})} \int_{0}^{1} \alpha^{(\eta + m)Lx}dx
% \end{equation}


% % \begin{equation}
% %     c_q^{(m)} = e^{jm(\psi_{q-1})} \int_{0}^{1} e^{j\frac{2\pi(\eta + m)L}{LQ}x}dx
% % \end{equation}


% \begin{equation}
%     c_q^{(m)} = e^{jm(\psi_{q-1})} \int_{0}^{1} e^{j\xi x}dx
% \end{equation}

% where $\xi = (\psi_{q}-\psi_{q-1}) (\eta +m )$.


% \begin{equation}
%     c_q^{(m)} = e^{jm(\psi_{q-1})} \frac{e^{j\xi}-1}{j\xi}
% \end{equation}

% \begin{equation}
%     c_q^{(m)} = e^{j(m\psi_{q-1} + \frac{\xi}{2})} \frac{e^{j\xi/2}-e^{-j\xi/2}}{j\xi/2}
% \end{equation}

% \begin{equation}
%     c_q^{(m)} = e^{j(m\psi_{q-1} + \frac{\xi}{2})} sinc(\frac{\xi}{2\pi})
% \end{equation}

% where, $sinc(t) = \frac{sin(\pi t)}{\pi t}$.
















% We first optimize for $\alpha$ by setting the derivative of \eqref{obj_func} to zero, to get: 

% $$\hat{\alpha}=\frac{\sum_{q \in \mathcal{W}_k}\sqrt{\frac{Q}{M_{t}}}(\mathbf{F} \mathbf{v})^{H} \mathbf{D}_{q} \mathbf{g}_q}{\left\|\mathbf{D}^{H} \mathbf{F} \mathbf{v}\right\|_{2}^{2}}$$

% Replacing the value of $\alpha$ as above we get to the following optimization problem. 

% \begin{align}
% \left(\mathbf{F}_{\mid \mathbf{g}}, \mathbf{v}_{\mid \mathbf{g}}\right) &=\underset{\mathbf{F}, \mathbf{v}}{\arg \max } \frac{\left|\sum_{q \in \mathcal{W}_k}\sqrt{\frac{2^{\mathbf{B}}}{M_{t}}}\left(\mathbf{D}_{q} \mathbf{g}_q\right)^{H} \mathbf{F} \mathbf{v}\right|^{2}}{\left\|\mathbf{D}^{H} \mathbf{F} \mathbf{v}\right\|_{2}^{2}} \nonumber \\
% &=\underset{\mathbf{F}, \mathbf{v}}{\arg \max }\left|\sum_{q \in \mathcal{W}_k}\left(\mathbf{D}_{q} \mathbf{g}\right)^{H} \mathbf{F} \mathbf{v}\right|^{2}
% \end{align}


% \section{Codebook Design Problem Formulation}
% \label{sec: formulation}

% % Let $\mathbf{c} \doteq \mathbf{F} \mathbf{v} \in \mathbb{C}^{M_{t}}$ be the unit-norm transmit beam-forming codeword, where analog beamsteering matrix $\mathbf{F}=\left[\mathbf{f}_{1}, \cdots, \mathbf{f}_{N_{R F}}\right] \in \mathbb{C}^{M_{t} \times N_{R F}}$ contains $N_{RF}$ equal-gain analog beamsteering vectors $\mathbf{f}_{n}$ and the baseband beamforming vector $\mathbf{v} \in \mathbb{C}^{N_{R F}}$ is such that $\|\c\|_{2}=1$ holds. The equal-gain constraint holds if $\mathbf{f}_{n} \in \mathcal{B}_{M_{t}}$ for 

% % \begin{equation}
% %     \mathcal{B}_{M_{t}}=\left\{\mathbf{w} \in \mathbb{C}^{M_{t}}:\left(\mathbf{w} \mathbf{w}^{H}\right)_{\ell, \ell}=1 / M_{t}, 1 \leq \ell \leq M_{t}\right\}
% %     \label{f_constant_gain}
% % \end{equation}

% % Let $\mathcal{C}=\left\{\mathbf{c}_{1}, \cdots, \mathbf{c}_{Q}\right\}$ be the codebook with codewords taking value as $\mathbf{c}_{q}=\mathbf{F}_{q} \mathbf{v}_{q}$. Consider a half-wavelength uniform linear array (ULA) of antennas as in the previous subsection, i.e. $d = \frac{\lambda}{2}$, with an array response vector as in \eqref{array_factor}.

% % % %$$\mathbf{d}_{M_{t}}(\psi)=\frac{1}{\sqrt{M_{t}}}\left[1 e^{j \psi} \cdots e^{j\left(M_{t}-1\right) \psi}\right]^{T} \in \mathbb{C}^{M_{t}}$$
% % % $$\mathbf{d}_{M_{t}}(\psi)=\left[1, e^{j \psi}, \cdots, e^{j\left(M_{t}-1\right) \psi}\right]^{T} \in \mathbb{C}^{M_{t}}$$

%  For convenience, in this section, we will drop the index \emph{ula} from the expression of the array factor $\d_{ula, M_t}(\psi)$ and gain $G^{ula}(\psi, \mathbf{c})$.  Before presenting the formulation of the codebook design problem, we introduce some new notations. We divide the angular range of $\theta \in [-\pi, 0]$  into equal-length beams $\omega_{q} \text { for } q \in\left\{1, \cdots, Q\right\}$. i.e. 
% $$
% \begin{aligned}
% \omega_{q} &=\left[\theta_{q-1}, \theta_{q}\right) ,& \theta_{q}= -\pi+ \frac{ \pi}{Q} q
% \end{aligned}
% $$

% % Further we introduce the change of variable $\psi = \frac{2 \pi d}{\lambda} \cos(\theta)$ where $\psi \in [-\pi, \pi]$. For $d = \frac{\lambda}{2}$,

% Corresponding to $\omega_q$ intervals, there exists $\nu_q$ ranges with respect to $\psi$ such that, 

% $$
% \begin{aligned}
% \nu_{q} &=\left[\psi_{q-1}, \psi_{q}\right), & \psi_q = -\pi\cos(\frac{\pi}{Q}q)
% \end{aligned}
% $$


% Under the reference gain as in \eqref{reference gain} and using Parseval's theorem \cite{parseval}, we will have:
% % and assuming $|\c| = 1$
% \begin{equation}
% \int_{-\pi}^{\pi} G(\psi, \mathbf{c}) d \psi=2 \pi\left\| \mathbf{c}\right\|^{2}=2 \pi\label{parseval}
% \end{equation}

% Let  $G_{\text {ideal }, q}(\psi)$ denote the desired ideal gain which is supposed to be constant on $\nu_q$ and zero otherwise. It must hold that, 
% % \begin{equation}
% % \int_{-\pi}^{\pi} G(\psi, \mathbf{c}) d \psi=2 \pi\left\|\frac{1}{\sqrt{M_{t}}} \mathbf{c}\right\|_{2}^{2}=\frac{2 \pi}{M_{t}}    \label{parseval}
% % \end{equation}





% \begin{align}
% &\int_{-\pi}^{\pi} G_{\text {ideal }, q}(\psi) d \psi =\int_{\nu_{q}} t d \psi+\int_{[-\pi, \pi] \backslash \nu_{q}} 0 d \psi \nonumber\\
% &=(\psi_q - \psi_{q-1}) t={2 \pi} \label{plain}
% \end{align}

% which in turn will give: 

% \begin{equation}
%     G_{\text {ideal }, q}(\psi)=\frac{2\pi}{(\psi_q - \psi_{q-1})} \mathds{1}_{\nu_{q}}(\psi), \quad \psi \in[-\pi, \pi] \label{ideal_gain}
% \end{equation}

% We aim to design the codebooks so as to mimic the ideal gain computed in equation \eqref{ideal_gain}. Therefore, the plain codebook design problem is formulated as a minimization of a MSE as follows: 


% % \begin{align}
% % &\left(\mathbf{F}_{\text {opt }, q}, \mathbf{v}_{\mathrm{opt}, q}\right) \nonumber \\&=\underset{\mathbf{F}, \mathbf{v}}{\arg \min } \int_{-\pi}^{\pi}\left|G_{\text {ideal }, q}(\psi)-G(\psi, \mathbf{F} \mathbf{v})\right|^{2} d \psi
% % \end{align}

% \begin{align}
% &\c_q^{opt}=\underset{\c, \|\c\|=1}{\arg \min } \int_{-\pi}^{\pi}\left|G_{\text {ideal }, q}(\psi)-G(\psi, \c)\right| d \psi
% \label{composite_MSE}
% \end{align}

% By uniformly sampling on the range of $\psi$ we can rewrite the optimization problem as follows, 
% \begin{align}
%     &\underset{\c, \|\c\|=1}{\arg \min }\left[\lim _{L \rightarrow \infty} \sum_{p=1}^{Q} \delta_p \sum_{\ell=1}^{L} \frac{\left|G_{\text {ideal }, q}\left(\psi_{p, \ell}\right)-G\left(\psi_{p, \ell}, \c \right)\right|}{L}\right]\label{equivalent_MSE}
% \end{align}

%  where for $q = 1 \ldots Q$, 
%  $$\delta_q = \psi_{q}-\psi_{q-1}, \quad \psi_{q, \ell}=\psi_{q-1}+\frac{\delta_q(\ell-0.5)}{L} $$

 
%  We can write equation \eqref{equivalent_MSE}, as 
%  \begin{align}
%  &\c_q^{opt} = \underset{\c,  \|\c\|=1}{\arg \min } \lim_{L\rightarrow \infty}\frac{1}{L}\left\|\mathbf{G}_{\text {ideal }, q}-\mathbf{G}(\c)\right\| \label{init_opt}
%  \end{align}
 
%  where,  $$\mathbf{G}(\c)=\left[{\delta_1} G\left(\psi_{1,1}, \c\right) \ldots {\delta_{Q}}G\left(\psi_{Q, L}, \c\right)\right]^{T} \in \mathbb{Z}^{L Q}$$

% and, 
% $$ \mathbf{G}_{\text {ideal }, q}=\left[{\delta_1} G_{\text {ideal }, q}\left(\psi_{1,1}\right) \ldots {\delta_{Q}}G_{\text {ideal }, q}\left(\psi_{Q, L}\right)\right]^{T} \in \mathbb{Z}^{L Q}$$

% Note that it holds that 

% \begin{equation}
%     \mathbf{G}_{\text {ideal }, q}={2\pi}\left(\mathbf{e}_{q} \otimes \mathbf{1}_{L, 1}\right) \label{ideal}
% \end{equation}

% with $\mathbf{e}_{q} \in \mathbb{Z}^{Q}$ being the standard basis vector for the $q$-th axis among $Q$ ones. Now, note that $\mathbf{1}_{L, 1}=\mathbf{g} \odot \mathbf{g}^{*}$ for any equal gain $\mathbf{g} \in \mathbb{C}^L$. Therefore, for a suitable choice of $\g$ we can write: 

% % $$\mathcal{G}_{L}=\left\{\mathbf{g} \in \mathbb{C}^{L}:\left(\mathbf{g} \mathbf{g}^{H}\right)_{\ell, \ell}=1,1 \leq \ell \leq L\right\}$$


% \begin{align}
% \mathbf{G}_{\text {ideal }, q} &={2\pi}\left(\mathbf{e}_{q} \otimes\left(\mathbf{g} \odot \mathbf{g}^{*}\right)\right) \nonumber\\
% &=\left(\sqrt{2\pi}\left(\mathbf{e}_{q} \otimes \mathbf{g}\right)\right) \odot\left(\sqrt{{2\pi}}\left(\mathbf{e}_{q} \otimes \mathbf{g}\right)\right)^{*} \label{g_id_q_equivalent}
% \end{align}


% Similarly, it is straightforward to observe,


% \begin{align}
% \mathbf{G}(\c) &=\left(\mathbf{D}^{H} \c\right) \odot\left(\mathbf{D}^{H} \c\right)^{*} \label{dc}
% \end{align}
% where $\mathbf{D} =\left[\sqrt{\delta_1}\mathbf{D}_{1} \cdots \sqrt{\delta_{Q}}\mathbf{D}_{Q} \right] \in \mathbb{C}^{M_{t} \times L Q}$, and 
% $\mathbf{D}_{q}=\left[\mathbf{d}_{M_{t}}\left(\psi_{q, 1}\right) \cdots \mathbf{d}_{M_{t}}\left(\psi_{q, L}\right)\right] \in \mathbb{C}^{M_{t} \times L}.$
 

% % Comparing the expressions in equations \eqref{g_id_q_equivalent} and \eqref{dc},  we formulate the following two-step optimization for the codebook design problem for hybrid beamforming, presented below as \textit{Problem \ref{main_problem}}. Given a suitable $\g_q$, in the first step, we search for a unit-norm codeword $\c_q$ for fully-digital beamforming, and in the second step we find a suitable pair $(\F_q, \v_q)$ to complete the hybrid beamforming codebook design.  

 

% % \begin{problem}

% % a) For all $q \in \{1, \cdots, Q\}$, given equal-gain $\g_q \in \mathbb{C}^L$, find vector $\c_q \in \mathbb{C}^{M_t}$ such that
% % \begin{align}
% % &\c_q=\underset{\c, \|c\|=1}{\arg \min } \lim_{L\longrightarrow \infty} \left\|\sqrt{2\pi}\left(\mathbf{e}_{q} \otimes \mathbf{g}_q\right)- \mathbf{D}^{H} \c\right\|^{2} \label{obj_func}
% % \end{align}

% % \vspace{2mm}
% % \noindent
% % b) Find $\F_q \in \mathbb{C}^{M_t \times N_{RF}}$ , and $\v_q \in \mathbb{C}^{N_{RF}}$ that solve $\mathbf{F}_{q}\mathbf{v}_{q} = \c_q$ subject to the equal-gain condition on the columns of $\F_q$. 

% % % \begin{align}
% % %     \mathbf{F}_{\mid \g_q}\mathbf{v}_{\mid \g_q} = \c_q
% % % \end{align}
% % \label{main_problem}
% % \end{problem}

% % The solution to the last problem provides the optimal codebook designed for hybrid beamforming under a ULA structure, given the choice of $\g_q$. Note that we were initially trying to solve the optimization problem in equation \eqref{init_opt}. In light of modifications \eqref{g_id_q_equivalent}, and \eqref{dc}, the quantity to be minimized in the limit of equation \eqref{init_opt} can be written as follows,

% % \begin{align}
% %     & \left\|\mathbf{G}_{\text {ideal }, q}-\mathbf{G}(\c)\right\| = \left\| abs(\D^H \c_q)- \sqrt{2\pi} abs(\mathbf{e}_{q} \otimes \mathbf{g})\right\| \label{g_init}
% % \end{align}
% % where $abs(.)$ denotes the element-wise absolute value of a vector.

% % Equation \eqref{g_init} reveals the central role of the vector $\g_q$ in minimizing the MSE in equation \eqref{init_opt}. In fact, the choice of $\g_q$ has to be optimized in a way that norm in \eqref{g_init} is minimized; i.e. the result gain designed by codebook $\c_q$ gets as close as possible to the ideal gain. To optimize the choice of $\g_q$, we formulate the following optimization problem, presented as \textit{Problem \ref{g_problem}}.

% % \begin{problem}
% % For all $q = 1, \ldots, Q$, find equal-gain $\g_q \in \mathbb{C}^L$ such that

% % \begin{equation}
% %  \g_q = \underset{\g}{\arg\min }\left\| abs(\D^H \c_q)- \sqrt{2\pi} abs(\mathbf{e}_{q} \otimes \mathbf{g})\right\|^{2} \label{g_final_eq}
% % \end{equation} 

% % \label{g_problem}
% % \end{problem}

% % In the next session we propose our approach to solve problems \ref{main_problem} and \ref{g_problem}.

% %%%%%%%%%%%%%%%%%%%%%%%%%%%%%%%
% Comparing the expressions \eqref{init_opt}, \eqref{g_id_q_equivalent}, and \eqref{dc}, one can show that the optimal choice of $\c_q$ in \eqref{init_opt} is the solution to the following optimization problem for a proper choice of $\g_q$. 
% \begin{problem}
% Given an equal-gain vector $\g_q \in \mathbb{C}^L$, $q \in \{1, \cdots, Q\}$, find vector $\c_q \in \mathbb{C}^{M_t}$ such that
% \begin{align}
% &\c_q=\underset{\c, \|c\|=1}{\arg \min } \lim_{L\longrightarrow \infty} \left\|\sqrt{2\pi}\left(\mathbf{e}_{q} \otimes \mathbf{g}_q\right)- \mathbf{D}^{H} \c\right\|^{2} \label{obj_func}
% \end{align}
% \label{main_problem}
% \end{problem}
% However, we now need to find the optimal choice of $\g_q$ that minimizes the objective in \eqref{init_opt}. Using \eqref{g_id_q_equivalent}, and \eqref{dc}, we have the following optimization problem.
% \begin{problem}
% Find equal-gain $\g_q \in \mathbb{C}^L$, $q = 1, \ldots, Q$, such that
% \begin{equation}
%  \g_q = \underset{\g}{\arg\min }\left\| abs(\D^H \c_q)- \sqrt{2\pi} abs(\mathbf{e}_{q} \otimes \mathbf{g})\right\|^{2} \label{g_final_eq}
% \end{equation} 
% where $abs(.)$ denotes the element-wise absolute value of a vector.
% \label{g_problem}
% \end{problem}

% Hence, the codebook design for a system with full-digital beamforming capability is found by solving Problem~\ref{main_problem} for proper choice of $\g_q$ obtained as a solution to Problem ~\ref{g_problem}. The codebook for Hybrid beamforming, is then found as
% \begin{align}
%     \underset{\mathbf{F}_{q},\mathbf{v}_{q}}{\arg\min } \|\mathbf{F}_{q}\mathbf{v}_{q} - \c_q\|^2
% \end{align}
% where the columns of $\F_q \in \mathbb{C}^{M_t \times N_{RF}}$ are equal-gain vectors and $\v_q \in \mathbb{C}^{N_{RF}}$. The solution may be obtained using simple, yet effective suboptimal algorithm such as orthogonal matching pursuit (OMP) \cite{love15}\cite{noh17}\cite{Hussain17}. In the next section we continue with the solution of Problem~\ref{main_problem}, and~\ref{g_problem}.


% %%%%%%%%%%%%%%%%%%%%%%%%%%%%%%%

% % Note that the solution to part \textit{(a)} is the limit of the sequence of solutions to a least-square optimization problem as $L$ goes to infinity.
% %  \begin{align}
% % & \Tilde{\c}^{(L)}_q = (\D \D^H)^{-1} \D \sigma \left(\mathbf{e}_q \otimes \mathbf{g}_q\right) \\
% % & \Tilde{\c}^{(L)}_q =  \sigma' \D \left(\mathbf{e}_q \otimes \mathbf{g}_q\right) = \sigma' \D_q\g_q
% % \end{align}
% % Dividing by $\|\Tilde{\c}^{(L)}_q\|$ and taking the limit as L goes to infinity we will find the optimal $\c_q$. i.e. 

% % \begin{equation}
% %     \c_q = \lim_{L\longrightarrow \infty} \frac{\Tilde{\c}^{(L)}_q}{L\|\Tilde{\c}^{(L)}_q\|}
% % \end{equation}

% % We aim to find $\g_q$ such that

% % \begin{align}
% % & \left\| abs(\D^H \gamma' \D_q \mathbf{g}_q)- abs(\gamma \left(\mathbf{e}_{q} \otimes \mathbf{g}_q\right)) \right\|_{2}^{2}
% % \end{align}

% % is minimized.

% % % Is it possible to show that $\g_q$ is independent of $q$ (assume regular scenario)? If true, i.e., $\g_q = \g$ we have

% % We will provide a suboptimal structure for the choice of $\g$. Let us assume $\g_q = \g, $  for all $ q \in \{1, \cdots, Q\}$. Then we can write
% % \begin{align}
% %      &\Tilde{\c_q}^{(L)} = D_q\mathbf{g} = \sum_{l=1}^{L}g_l\mathbf{d}_{M_t}(\psi_{q, l})  \nonumber\\
% %      & = \sum_{l=1}^L g_l\left[{\begin{array}{cccc}
% %      1& e^{j\psi_{q, l}} &\cdots & e^{j(M_t-1)\psi_{q, l}}\\
% %      \end{array}}\right]^T  = \nonumber\\
% %      & \left[{\begin{array}{cccc}
% %      \sum_{l=1}^L g_l& \sum_{l=1}^L g_le^{j\psi_{q, l}} &\cdots & \sum_{l=1}^L g_le^{j(M_t-1)\psi_{q, l}}\\
% %      \end{array}}\right]^T
% %  \end{align}
 
 

% %  We have for the $m$-th element of the vector $\Tilde{\c}_q$, i.e. $\Tilde{c}^{(m)}_q$  for $ 0 \leq m \leq M_t-1$

% % \begin{equation}
% %     \Tilde{c}_q^{(m)} = \lim_{L\longrightarrow \infty} \frac{1}{L}\sum_{l=0}^{L-1} g_l e^{j(m\psi_{q, l+1})}
% % \end{equation}

% % With a choice of $\g = \left[{\begin{array}{cccc} 1& \alpha^\eta &\cdots & \alpha^{\eta (L -1)} \end{array}}\right]^T$ where we set 
% % % $\alpha = e^j(\frac{2\pi}{LQ})$
% % $\alpha = e^{j(\frac{\psi_{q}-\psi_{q-1}}{L})}$ and suitable $\eta$ (to be determined later), we can write


% % % \begin{equation}
% % %     \Tilde{c}_q^{(m)} = \lim_{L\longrightarrow \infty} \frac{1}{L}\sum_{l=0}^{L-1} g_l e^{jm(\psi_{q-1}+\frac{2 \pi(\ell+0.5)}{LQ})}
% % % \end{equation}

% % \begin{equation}
% %     \Tilde{c}_q^{(m)} = \lim_{L\longrightarrow \infty} \frac{1}{L}\sum_{l=0}^{L-1} g_l e^{jm(\psi_{q-1}+\frac{(\psi_{q}-\psi_{q-1})(\ell+0.5)}{L})}
% % \end{equation}

% % \begin{equation}
% %     c_q^{(m)} = \lim_{L\longrightarrow \infty} \frac{1}{L}\sum_{l=0}^{L-1} \alpha^{(\eta+m) l} e^{jm(\psi_{q-1}+\frac{0.5(\psi_{q}-\psi_{q-1})}{L})} 
% % \end{equation}


% % \begin{equation}
% %     c_q^{(m)} = e^{jm(\psi_{q-1})}\lim_{L\longrightarrow \infty} \frac{1}{L}\sum_{l=0}^{L-1} \alpha^{(\eta+m) l}   
% % \end{equation}

% % \begin{equation}
% %     c_q^{(m)} = e^{jm(\psi_{q-1})} \int_{0}^{1} \alpha^{(\eta + m)Lx}dx
% % \end{equation}


% % % \begin{equation}
% % %     c_q^{(m)} = e^{jm(\psi_{q-1})} \int_{0}^{1} e^{j\frac{2\pi(\eta + m)L}{LQ}x}dx
% % % \end{equation}


% % \begin{equation}
% %     c_q^{(m)} = e^{jm(\psi_{q-1})} \int_{0}^{1} e^{j\xi x}dx
% % \end{equation}

% % where $\xi = (\psi_{q}-\psi_{q-1}) (\eta +m )$.


% % \begin{equation}
% %     c_q^{(m)} = e^{jm(\psi_{q-1})} \frac{e^{j\xi}-1}{j\xi}
% % \end{equation}

% % \begin{equation}
% %     c_q^{(m)} = e^{j(m\psi_{q-1} + \frac{\xi}{2})} \frac{e^{j\xi/2}-e^{-j\xi/2}}{j\xi/2}
% % \end{equation}

% % \begin{equation}
% %     c_q^{(m)} = e^{j(m\psi_{q-1} + \frac{\xi}{2})} sinc(\frac{\xi}{2\pi})
% % \end{equation}

% % where, $sinc(t) = \frac{sin(\pi t)}{\pi t}$.

% % \subsection{Twin-ULA Setting}

% % We start from problem \eqref{main_problem}, and particularly, equation \eqref{obj_func}. Let $\{\s_m\}_{m =0}^{M_t -1}$ denote the columns of $D^H$. We can write: 

% % \begin{align}
% % &\s_m c_q^{(m)} + \s_{m+M_t/2} c_q^{(m+M_t/2)} \nonumber\\
% % &= \s_m (c_q^{(m)} + e^{-j \frac{2\pi d_y}{\lambda} \sin(\theta)} c_q^{(m+M_t/2)}) \nonumber\\
% % & \approx \s_m \Tilde{c}_q^{(m)}, \quad  m = 0 \ldots \frac{M_t}{2}-1\end{align}

% % where we define 
% % \begin{equation}
% %     \Tilde{c}_q^{(m)} = c_q^{(m)} + e^{-j \frac{2\pi d_y}{\lambda} \sin(\theta^{(m)}_q)} c_q^{(m+M_t/2)} \label{c_defn}
% % \end{equation}

% % with $\theta_{q-1}\leq\theta^{(m)}_q \leq \theta_{q}$ being a constant value.  

% % Therefore, the solution to part $(a)$ of problem \eqref{main_problem}, under the twin-ULA setup can be expressed as:

% %  \begin{align}
% % & \Tilde{\c}_q = (\Tilde{\D} \Tilde{\D}^H)^{-1} \Tilde{\D} \sigma \left(\mathbf{e}_q \otimes \mathbf{g}_q\right) \\
% % & \Tilde{\c}_q =  \sigma' \Tilde{\D} \left(\mathbf{e}_q \otimes \mathbf{g}_q\right) = \sigma' \Tilde{\D}_q\g_q
% % \end{align}

% % where it holds that $\Tilde{\D} \Tilde{\D}^H = \I_{\frac{M_t}{2}}$, and


% % $$\Tilde{\D}_{q}=\left[\d_{\frac{M_{t}}{2}}\left(\psi_{q, 1}\right) \cdots \d_{\frac{M_{t}}{2}}\left(\psi_{q, L}\right)\right] \in \mathbb{C}^{\frac{M_{t}}{2} \times L}$$

% % In other words, one can approximately think of the codebook design problem over a twin ULA of $M_t$ antennas, as such a problem over a ULA of size $\frac{M_t}{2}$.

% % Following the approach in the previous subsection, we can infer, 

% % \begin{equation}
% %     \Tilde{c}_q^{(m)} = e^{j(m\psi_{q-1} + \frac{\xi}{2})} sinc(\frac{\xi}{2\pi}), \quad m = 0 \ldots \frac{M_t}{2} -1
% %     \label{twin_c_result}
% % \end{equation}

% % Having $d_y = \frac{\lambda}{3}$, equating equations \eqref{c_defn}, and \eqref{twin_c_result}, by further assuming $c_q^{(m + \frac{M_t}{2})} = e^{j\beta_m} c_q^{(m)}$, after basic operations we get: 

% % \begin{equation}
% %     c_q^{(m)} = \frac{e^{j(m\psi_{q-1} + \frac{\xi}{2})} sinc(\frac{\xi}{2\pi})}{1+ e^{-j(\frac{2\pi}{3}\sin(\theta_q^{(m)}) + \beta_m)}}
% % \end{equation}

% % % \begin{equation}
% % %     c_q^{(m)} = \frac{e^{j(m\psi_{q-1} + \frac{\xi}{2} +\frac{\pi}{3}\sin(\theta_q^{(m)})+\frac{\beta_m}{2}) sinc(\frac{\xi}{2\pi})}}{2\cos(\frac{\pi}{3}\sin(\theta_q^{(m)}) + \frac{\beta_m}{2})}
% % % \end{equation}

% % Therefore, for each codeword entry $c_q^{(m)}$, it remains to pick $\beta_m$ and $\theta_q^{(m)}$. We relax this decision by setting $\beta_m = \beta$, and $\theta_q^{(m)} = \theta_q$, $ m = 1 \ldots M_t$. Next we explicitly right the expression for the reference gain as $G(\theta, \c_{twin}) = \left|\d_{twin, M_{t}}^{H}(\theta) \c_{twin}\right|^{2}$ where 

% % \begin{align}
% %     &\d_{twin, M_{t}}(\theta) = \left[\d_{\frac{M_{t}}{2}}^{T}(\theta), e^{j(\frac{2\pi}{3}\sin(\theta))}\d_{\frac{M_{t}}{2}}^{T}(\theta)\right]^T\\
% %     & \c_{twin} = \left[\c^T_{twin, \frac{M_t}{2}}, e^{j\beta}\c^T_{twin, \frac{M_t}{2}}\right]^T
% % \end{align}

% % We can  therefore write, 

% % \begin{align}
% %     G(\theta, \c_{twin}) &= \left|\d_{\frac{M_{t}}{2}}^{H}(\theta)\c_{twin, \frac{M_t}{2}} (1 + e^{j(\beta-\frac{2\pi}{3}\sin(\theta))})\right|^2\nonumber\\
% %     & = \left|\d_{\frac{M_{t}}{2}}^{H}(\theta)\c\right|^2 \left| \frac{1 + e^{j(\beta-\frac{2\pi}{3}\sin(\theta))}}{1 + e^{-j(\beta+\frac{2\pi}{3}\sin(\theta))}}\right|^2
% % \end{align}

% % Further define 
% % $$ L(\theta) = \left| \frac{1 + e^{j(\beta-\frac{2\pi}{3}\sin(\theta))}}{1 + e^{-j(\beta+\frac{2\pi}{3}\sin(\theta))}}\right|$$
% % The last equation consists of two parts, the first part being the gain corresponding to a ULA of size $\frac{M_t}{2}$ and the second part being  $L^2(\theta)$. For any $\omega_q = \left[\theta_{q-1}, \theta_{q}\right)$, there exist a $\omega'_q \doteq \left(- \theta_{q}, -\theta_{q-1}\right]$. The gain of a ULA is symmetric over $\omega_q$, and $\omega'_q$ by definition. However utilizing a twin ULA we wish to suppress the gain over $\omega'_q$ as much as we can and contribute to the gain over $\omega_q$ for each codeword $c_q, \quad q = 1\ldots Q$. To this end, we define the \emph{isolation factor} $\mu$ as follows,

% % \begin{align}
% %     \mu = \underset{\omega_q}{\int}{\frac{L(-\theta)}{L(\theta)}} d\theta
% % \end{align}
% %  to denote the level of isolation between each $\omega_q$ and its counterpart.
 
 















% % % \subsection{Composite Codebook Design Problem}

% % % Let us define a \textit{composite beam} $\omega_k$ as a union of multiple disjoint, possibly non-neighboring beams $\nu_{q} \text { for } q \in \mathcal{Q} = \left\{1, \cdots, Q\right\}$. Let $\mathcal{C'}=\left\{\mathbf{c}_{1}, \cdots, \mathbf{c}_{K}\right\}$ be the codebook corresponding to the composite problem.

% % % % $$\bigcup_{k = 1}^{2^{B'}}{\omega_k} = [-\pi, \pi], \quad \text{and} \quad \omega_k \cap \omega_l = \emptyset, \quad \forall k\neq l.$$

% % % Moreover, define for each composite beam $\omega_k$ the set $\mathcal{W}_k \subseteq \mathcal{Q}$ to be the set of indices of the single beams that form $\omega_k$. i.e. $\mathcal{W}_k = \{q \in \mathcal{Q}: v_q \subseteq \omega_k\}$. 
% % % We have then for the new ideal gain for each new codeword $\textbf{c}_k$, 

% % % \begin{align}
% % % &\int_{-\pi}^{\pi} G_{\text {ideal }, k}(\psi) d \psi =\int_{\omega_{k}} t' d \psi+\int_{[-\pi, \pi] \backslash \omega_{k}} 0 d \psi \nonumber\\
% % % &= \sum_{q \in {\mathcal{W}_k}}{\int_{\nu_{q}} t' d \psi} = |\mathcal{W}_k|\frac{2 \pi}{Q} t'=\frac{2 \pi}{M_{t}} \label{composite}
% % % \end{align}

% % % Therefore, $t' = \frac{Q}{|\mathcal{W}_k| M_t}$. It follows that, 

% % % \begin{equation}
% % %     G_{\text {ideal }, k}(\psi)=\frac{Q}{|\mathcal{W}_k| M_{t}} \mathds{1}_{\omega_{k}}(\psi), \quad \psi \in[-\pi, \pi] \label{composite_ideal_gain}
% % % \end{equation}

% % % The new MSE problem for the $k$-th codeword can therefore be written as follows.

% % % \begin{align}
% % % &\left(\mathbf{F}_{\text {opt }, k}, \mathbf{v}_{\mathrm{opt}, k}\right) \nonumber \\&=\underset{\mathbf{F}, \mathbf{v}}{\arg \min } \int_{-\pi}^{\pi}\left|G_{\text {ideal }, k}(\psi)-G(\psi, \mathbf{F} \mathbf{v})\right|^{2} d \psi \nonumber \\&= \underset{\mathbf{F}, \mathbf{v}}{\arg \min }\left[\lim _{L \rightarrow \infty} \sum_{p=1}^{Q} \sum_{\ell=1}^{L} \frac{\left|G_{\text {ideal }, k}\left(\psi_{p, \ell}\right)-G\left(\psi_{p, \ell}, \mathbf{F} \mathbf{v}\right)\right|^{2}}{L Q / 2 \pi}\right]\label{composite_MSE}
% % % \end{align}


% % % \begin{align}
% % % \left(\mathbf{F}_{k}, \mathbf{v}_{k}\right) &=\underset{\mathbf{F}, \mathbf{v}}{\arg \min } \sum_{p=1}^{Q} \sum_{\ell=1}^{L}\left|G_{\text {ideal }, k}\left(\psi_{p, \ell}\right)-G\left(\psi_{p, \ell}, \mathbf{F} \mathbf{v}\right)\right|^{2} \nonumber\\
% % % &=\underset{\mathbf{F}, \mathbf{v}}{\arg \min }\lim_{L\longrightarrow \infty} \frac{1}{L}\left\|\mathbf{G}_{\text {ideal }, k}-\mathbf{G}(\mathbf{F} \mathbf{v})\right\|_{2}^{2} \label{init_opt}
% % % \end{align}

% % % where,  $$\mathbf{G}(\mathbf{F} \mathbf{v})=\left[G\left(\psi_{1,1}, \mathbf{F} \mathbf{v}\right) \cdots G\left(\psi_{Q, L}, \mathbf{F} \mathbf{v}\right)\right]^{T} \in \mathbb{Z}^{L Q}$$

% % % and, 
% % % $$ \mathbf{G}_{\text {ideal }, k}=\left[G_{\text {ideal }, k}\left(\psi_{1,1}\right) \cdots G_{\text {ideal }, k}\left(\psi_{Q, L}\right)\right]^{T} \in \mathbb{Z}^{L Q}$$

% % % Note that we can write using the definition of $\mathcal{W}_k$ that, 

% % % $$G_{\text {ideal }, k}\left(\psi_{p, \ell}\right)=\left\{\begin{array}{ll}
% % % \frac{Q}{|\mathcal{W}_k| M_{t}}, & p \in \mathcal{W}_k \\
% % % 0, & p \notin \mathcal{W}_k 
% % % \end{array},\right.$$

% % % Or in the matrix form, 

% % % \begin{align}
% % %     &\mathbf{G}_{\text {ideal }, k}=\frac{Q}{|\mathcal{W}_k| M_{t}}\left((\bigoplus_{q \in \mathcal{W}_k}{\mathbf{e}_{q}}) \otimes \mathbf{1}_{L, 1}\right)  \nonumber \\ 
% % %     & = \sum_{q \in \mathcal{W}_k}{\frac{Q}{|\mathcal{W}_k| M_{t}}\left(\mathbf{e}_{q} \otimes \mathbf{1}_{L, 1}\right)} \label{G_ideal_k}
% % % \end{align}

% % % with $\mathbf{e}_{q} \in \mathbb{Z}^{Q}$ being the standard basis vector for the $q$-th axis among $Q$ ones. Now, note that $\mathbf{1}_{L, 1}=\mathbf{g} \odot \mathbf{g}^{*}$ for any $\mathbf{g} \in \mathcal{G}_L$, where, 

% % % $$\mathcal{G}_{L}=\left\{\mathbf{g} \in \mathbb{C}^{L}:\left(\mathbf{g} \mathbf{g}^{H}\right)_{\ell, \ell}=1,1 \leq \ell \leq L\right\}$$



% % % Therefore, equation \eqref{G_ideal_k} can be rewritten as:
% % % \begin{align}
% % % &\mathbf{G}_{\text {ideal }, k} = \sum_{q \in \mathcal{W}_k}\frac{Q}{|\mathcal{W}_k| M_{t}}\left(\mathbf{e}_{q} \otimes\left(\mathbf{g}_q \odot \mathbf{g}_q^{*}\right)\right) \nonumber\\
% % % &=\sum_{q \in \mathcal{W}_k}\left(\sigma\left(\mathbf{e}_{q} \otimes \mathbf{g}_q\right)\right) \odot\left(\sigma\left(\mathbf{e}_{q} \otimes \mathbf{g}_q\right)\right)^{*} \nonumber\\
% % % &=\left(\sum_{q \in \mathcal{W}_k}\sigma\left(\mathbf{e}_{q} \otimes \mathbf{g}_q\right)\right)  \odot \left(\sum_{q \in \mathcal{W}_k}\sigma\left(\mathbf{e}_{q} \otimes \mathbf{g}_q\right)\right)^{*} \label{final_gik}
% % % \end{align}

% % % where $\sigma = \sqrt{\frac{Q}{|\mathcal{W}_k|M_{t}}}$. 
% % % On the other hand note that $\mathbf{G}(\mathbf{F} \mathbf{v})$ can be written as,


% % % \begin{equation}
% % %     \mathbf{G}(\mathbf{F} \mathbf{v}) =\left(\mathbf{D}^{H} \mathbf{F} \mathbf{v}\right) \odot\left(\mathbf{D}^{H} \mathbf{F} \mathbf{v}\right)^{*}\label{dfv_decomp}
% % % \end{equation}

% % % where, 
% % % $$\mathbf{D} =\left[\mathbf{D}_{1} \cdots \mathbf{D}_{Q}\right] \in \mathbb{C}^{M_{t} \times L Q}$$

% % % and, $\mathbf{D}_{q}=\left[\mathbf{d}_{M_{t}}\left(\psi_{q, 1}\right) \cdots \mathbf{d}_{M_{t}}\left(\psi_{q, L}\right)\right] \in \mathbb{C}^{M_{t} \times L}$. 

% % % Comparing the expressions in equations \eqref{final_gik}, and, \eqref{dfv_decomp}, a perfect solution to the optimization problem in \eqref{init_opt}, could be a pair $\textbf{(F, v)}$, for which the following equality holds.  



% % % \begin{align}
% % %     \mathbf{D}^{H} \mathbf{F} \mathbf{v}=\sum_{q \in \mathcal{W}_k}\sigma\left(\mathbf{e}_{q} \otimes \mathbf{g}_q\right) \label{potential_answer}
% % % \end{align}

% % % However, we have that $M_t > N_{RF}$ and therefore, the matrix $\textbf{D}^H \textbf{F}$ may not be full-rank, resulting in the RHS of equation \eqref{potential_answer} being in the null space of LHS. Therefore, given some $\g_q \in \mathcal{G}_L$ we formulate the following two-step optimization problem to search for $\textbf{(F, v)}$ that gets as close as possible to the desired value. 

% % % \begin{problem}

% % % a) For all $q \in \{1, \cdots, Q\}$, given $\g_q \in \mathcal{G}_L$, find the unit-norm vector $\c_q \in \mathbb{C}^{M_t}$ such that
% % % \begin{align}
% % % &\c^{|\g_q}_q=\underset{\c}{\arg \min } \lim_{L\longrightarrow \infty} \frac{1}{L}\left\|\left(\sum_{q \in \mathcal{W}_k}\sigma\left(\mathbf{e}_{q} \otimes \mathbf{g}_q\right)\right)- \mathbf{D}^{H} \c\right\|_{2}^{2} \label{obj_func}
% % % \end{align}

% % % \noindent
% % % b) For every given $\c_q$, find $\F_q \in \mathbb{C}^{M_t \times N_{RF}}$ , and $\v_q \in \mathbb{C}^{N_RF}$ that satisfy $\mathbf{F}_{q}\mathbf{v}_{q} = \c_q$ the condition $\f_n \in \mathcal{B}_{M_t}$ as in equation \eqref{f_constant_gain}. 

% % % % \begin{align}
% % % %     \mathbf{F}_{\mid \g_q}\mathbf{v}_{\mid \g_q} = \c_q
% % % % \end{align}

% % % \end{problem}

% % % Note that the solution to part \textit{(a)} is the limit of the sequence of solutions to a least-square optimization problem as $L$ goes to infinity.
% % %  \begin{align}
% % % & \Tilde{\c}^{(L)}_q = (\D \D^H)^{-1} \D \sigma \left(\mathbf{e}_q \otimes \mathbf{g}_q\right) \\
% % % & \Tilde{\c}^{(L)}_q =  \sigma' \D \left(\mathbf{e}_q \otimes \mathbf{g}_q\right) = \sigma' \D_q\g_q
% % % \end{align}
% % % Dividing by $\|\Tilde{\c}^{(L)}_q\|$ and taking the limit as L goes to infinity we will find the optimal $\c_q$. i.e. 

% % % \begin{equation}
% % %     \c_q = \lim_{L\longrightarrow \infty} \frac{\Tilde{\c}^{(L)}_q}{L\|\Tilde{\c}^{(L)}_q\|}
% % % \end{equation}

% % % We aim to find $\g_q$ such that

% % % \begin{align}
% % % & \left\| abs(\D^H \gamma' \D_q \mathbf{g}_q)- abs(\gamma \left(\mathbf{e}_{q} \otimes \mathbf{g}_q\right)) \right\|_{2}^{2}
% % % \end{align}

% % % is minimized.

% % % % Is it possible to show that $\g_q$ is independent of $q$ (assume regular scenario)? If true, i.e., $\g_q = \g$ we have

% % % We will provide a suboptimal structure for the choice of $\g$. Let us assume $\g_q = \g, $  for all $ q \in \{1, \cdots, Q\}$. Then we can write
% % % \begin{align}
% % %      &\Tilde{\c_q}^{(L)} = D_q\mathbf{g} = \sum_{l=1}^{L}g_l\mathbf{d}_{M_t}(\psi_{q, l})  \nonumber\\
% % %      & = \sum_{l=1}^L g_l\left[{\begin{array}{cccc}
% % %      1& e^{j\psi_{q, l}} &\cdots & e^{j(M_t-1)\psi_{q, l}}\\
% % %      \end{array}}\right]^T  = \nonumber\\
% % %      & \left[{\begin{array}{cccc}
% % %      \sum_{l=1}^L g_l& \sum_{l=1}^L g_le^{j\psi_{q, l}} &\cdots & \sum_{l=1}^L g_le^{j(M_t-1)\psi_{q, l}}\\
% % %      \end{array}}\right]^T
% % %  \end{align}
 
 

% % %  We have for the $m$-th element of the vector $\Tilde{\c}_q$, i.e. $\Tilde{c}^{(m)}_q$  for $ 0 \leq m \leq M_t-1$

% % % \begin{equation}
% % %     \Tilde{c}_q^{(m)} = \lim_{L\longrightarrow \infty} \frac{1}{L}\sum_{l=0}^{L-1} g_l e^{j(m\psi_{q, l+1})}
% % % \end{equation}

% % % With a choice of $\g = \left[{\begin{array}{cccc} 1& \alpha^\eta &\cdots & \alpha^{\eta (L -1)} \end{array}}\right]^T$ where we set 
% % % % $\alpha = e^j(\frac{2\pi}{LQ})$
% % % $\alpha = e^j(\frac{\psi_{q}-\psi_{q-1}}{L})$ and suitable $\eta$ (to be determined later), we can write


% % % % \begin{equation}
% % % %     \Tilde{c}_q^{(m)} = \lim_{L\longrightarrow \infty} \frac{1}{L}\sum_{l=0}^{L-1} g_l e^{jm(\psi_{q-1}+\frac{2 \pi(\ell+0.5)}{LQ})}
% % % % \end{equation}

% % % \begin{equation}
% % %     \Tilde{c}_q^{(m)} = \lim_{L\longrightarrow \infty} \frac{1}{L}\sum_{l=0}^{L-1} g_l e^{jm(\psi_{q-1}+\frac{(\psi_{q}-\psi_{q-1})(\ell+0.5)}{L})}
% % % \end{equation}

% % % \begin{equation}
% % %     c_q^{(m)} = \lim_{L\longrightarrow \infty} \frac{1}{L}\sum_{l=0}^{L-1} \alpha^{(\eta+m) l} e^{jm(\psi_{q-1}+\frac{0.5(\psi_{q}-\psi_{q-1})}{L})} 
% % % \end{equation}


% % % \begin{equation}
% % %     c_q^{(m)} = e^{jm(\psi_{q-1})}\lim_{L\longrightarrow \infty} \frac{1}{L}\sum_{l=0}^{L-1} \alpha^{(\eta+m) l}   
% % % \end{equation}

% % % \begin{equation}
% % %     c_q^{(m)} = e^{jm(\psi_{q-1})} \int_{0}^{1} \alpha^{(\eta + m)Lx}dx
% % % \end{equation}


% % % % \begin{equation}
% % % %     c_q^{(m)} = e^{jm(\psi_{q-1})} \int_{0}^{1} e^{j\frac{2\pi(\eta + m)L}{LQ}x}dx
% % % % \end{equation}


% % % \begin{equation}
% % %     c_q^{(m)} = e^{jm(\psi_{q-1})} \int_{0}^{1} e^{j\xi x}dx
% % % \end{equation}

% % % where $\xi = (\psi_{q}-\psi_{q-1}) (\eta +m )$.


% % % \begin{equation}
% % %     c_q^{(m)} = e^{jm(\psi_{q-1})} \frac{e^{j\xi}-1}{j\xi}
% % % \end{equation}

% % % \begin{equation}
% % %     c_q^{(m)} = e^{j(m\psi_{q-1} + \frac{\xi}{2})} \frac{e^{j\xi/2}-e^{-j\xi/2}}{j\xi/2}
% % % \end{equation}

% % % \begin{equation}
% % %     c_q^{(m)} = e^{j(m\psi_{q-1} + \frac{\xi}{2})} sinc(\frac{\xi}{2\pi})
% % % \end{equation}

% % % where, $sinc(t) = \frac{sin(\pi t)}{\pi t}$.
















% % % % We first optimize for $\alpha$ by setting the derivative of \eqref{obj_func} to zero, to get: 

% % % % $$\hat{\alpha}=\frac{\sum_{q \in \mathcal{W}_k}\sqrt{\frac{Q}{M_{t}}}(\mathbf{F} \mathbf{v})^{H} \mathbf{D}_{q} \mathbf{g}_q}{\left\|\mathbf{D}^{H} \mathbf{F} \mathbf{v}\right\|_{2}^{2}}$$

% % % % Replacing the value of $\alpha$ as above we get to the following optimization problem. 

% % % % \begin{align}
% % % % \left(\mathbf{F}_{\mid \mathbf{g}}, \mathbf{v}_{\mid \mathbf{g}}\right) &=\underset{\mathbf{F}, \mathbf{v}}{\arg \max } \frac{\left|\sum_{q \in \mathcal{W}_k}\sqrt{\frac{2^{\mathbf{B}}}{M_{t}}}\left(\mathbf{D}_{q} \mathbf{g}_q\right)^{H} \mathbf{F} \mathbf{v}\right|^{2}}{\left\|\mathbf{D}^{H} \mathbf{F} \mathbf{v}\right\|_{2}^{2}} \nonumber \\
% % % % &=\underset{\mathbf{F}, \mathbf{v}}{\arg \max }\left|\sum_{q \in \mathcal{W}_k}\left(\mathbf{D}_{q} \mathbf{g}\right)^{H} \mathbf{F} \mathbf{v}\right|^{2}
% % % % \end{align}


% \subsection{UPA Codebook Design Prolem}

% Suppose an $M_h \times M_v$ UPA is placed at the $x-z$ plane, and let $M = M_h M_v$. Let the distance between the antennas that are placed parallel to $z$ axis, be $d_z$, and $d_x$ respectively. One can define,

% \begin{align}
% \mathbf{d}_{M_{a}}\left(\psi_{a}\right) = \left[1, e^{j \psi_{a}} \cdots e^{j\left(M_{a}-1\right) \psi_{a}}\right]^{T} \in \mathbb{C}^{M_{a}}
% \end{align}
% where, $\psi_{h}=\frac{2 \pi d_{x}}{\lambda} \sin \theta \cos \phi \text { and } \psi_{v}=\frac{2 \pi d_{z}}{\lambda} \sin \phi$, and $a$ takes value in the set $\{v, h\}$.  
% The array response vector is then defined as 

% \begin{align}
%     \mathbf{d}_{M}\left(\psi_{v}, \psi_{h}\right) =
%     \mathbf{d}_{M_{v}}\left(\psi_{v}\right) \otimes
%     \mathbf{d}_{M_{h}}\left(\psi_{h}\right)  \in \mathbb{C}^{M}
% \end{align}
% Let $\mathcal{B}_s$ be the angular range under cover. We have, 

% \begin{equation}
%     \mathcal{B}_{s} \doteq \left[-\phi^{\mathrm{B}}, \phi^{\mathrm{B}}\right)
%     \times \left[-\theta^{\mathrm{B}}, \theta^{\mathrm{B}}\right) \label{range_angle}
% \end{equation}

% Notation $\psi^{p,q}_{h}[1:L_h,1 :L_v], \psi^{p,q}_{v}[1:L_v]$

% Let us uniformly divide the angular range  into $Q=Q_{v} Q_{h}$ subregions. For each subregion we have, 

% $$ \mathcal{B}_{p, q} =  \omega_{\phi, p} \times \omega_{\theta, q}$$

% Let $\Delta_{\theta} =  2 \theta^{\mathrm{B}} / Q_{v}$, and $\Delta_\phi =  2 \phi^{\mathrm{B}} / Q_{h}$ be the length of each sub-interval along axis $\theta$, and $\phi$. We have $\omega_{\theta, q} = [\theta_{q-1}, \theta_q)$, and, $\omega_{\phi, p} = [\phi_{p-1},\phi_p)$ where  $\theta_q = -\theta^B + q\Delta_\theta$, and $\phi_p = -\phi^B + p\Delta_\phi$. 


% In the $(\psi_h, \psi_v)$ domain, we can rewrite equation \eqref{range_angle} as, 

% \begin{equation}
%     \mathcal{B}^\psi_{s} \doteq\left[-\psi_h^{\mathrm{B}}, \psi_h^{\mathrm{B}}\right) \times\left[-\psi_v^{\mathrm{B}}, \psi_v^{\mathrm{B}}\right)
% \end{equation}

% and each sub-region similarly can be rewritten in the $\psi$ domain as

% $$ \mathcal{B}^\psi_{ p, q} \doteq \nu_{v}^{p, q} \times \nu_{h}^ {p, q}$$

% % We can then write for each $v_{a, b}$, i.e the $b$-th sub-interval along axis $a$,

% % \begin{equation}
% %     \nu_{v, p} =-\psi_{a}^{\mathrm{B}}+(b-1)\Delta_{a}+\left[0, \Delta_{a}\right)
% % \end{equation}

% Define the reference gain at $(\psi_{h}, \psi_{v}) $ as, 

% \begin{equation}
%     G\left(\psi_{v}, \psi_{h}, \mathbf{c}\right) =\left|\left(\mathbf{d}_{M_{v}}\left(\psi_{v}\right)\otimes\mathbf{d}_{M_{h}}\left(\psi_{h}\right)  \right)^{H} \mathbf{c}\right|^{2}
% \end{equation}

% It is straightforward to show that, 

% \begin{equation}
%     \int_{-\pi}^{\pi} \int_{-\pi}^{\pi} G\left(\psi_{v}, \psi_{h}, \mathbf{c}\right) d \psi_{v} d \psi_{h}={(2 \pi)^{2}}
% \end{equation}

% Therefore, for every interval $\mathcal{B}_{ p, q}$ we can derive the ideal gain expression as, 

% \begin{equation}
%     G_{p, q}^{\text {ideal }}\left(\psi_{v}, \psi_{h}\right)=G_{p,q} \mathds{1}_{\mathcal{B}^\psi_{p, q}}\left(\psi_{v}, \psi_{h}\right)
% \end{equation}

% where $G_{q,p} = \frac{(2\pi)^2}{\delta_{p,q}}$, where $\delta_{p,q}$ is the area of the interval $\mathcal{B}^\psi_{p,q}$. We define the codebook design problem under the UPA structure as follows. 
% % where $G_{q,p} = \frac{(2\pi)^2}{\delta^{p,q}_h\delta^p_v}$, $\delta^p_v = \psi^p_v - \psi^{p-1}_v $, and $\delta^{q, p}_h = \psi^{q,p}_h - \psi^{q-1, p}_h $. Also, let us define $\delta_{q,p} = \delta^{q, p}_h\delta^p_v$. 



% \begin{align}
% & \c^{opt}_{p, q} = \nonumber\\&\underset{\c, \|\c\|=1}{\arg \min } \iint_{\mathcal{B}_{s}}\left|G_{p, q}^{\text {ideal }}\left(\psi_{v}, \psi_{h}\right)-G\left(\psi_{v}, \psi_{h}, \c\right)\right| d \psi_{v} d \psi_{h} \label{init_opt}
% \end{align}

% By partitioning the range of $(\psi_h, \psi_v)$ into the pre-defined intervals, we can rewrite the optimization problem as follows, 

% \begin{align}
% & \c^{opt}_{p, q} = \underset{\c, \|\c\|=1}{\arg \min }\nonumber\\& \sum_{r=1}^{Q_v}\sum_{s=1}^{Q_h}\iint_{\mathcal{B}^{\psi}_{r,s}}\left|G_{p, q}^{\text {ideal }}\left(\psi_{v}, \psi_{h}\right)-G\left(\psi_{v}, \psi_{h}, \c\right)\right| d \psi_{v} d \psi_{h} \nonumber\\&
%  = \underset{L_h, L_v \rightarrow \infty}{\lim}\sum_{r=1}^{Q_v}\sum_{s=1}^{Q_h}\sum_{l_v =1}^{L_v}\sum_{l_h =1}^{L_h}\frac{\delta^r_v\delta^{r,s,l_v}_h}{L_hL_v}\nonumber\\&
% \left|G_{p, q}^{\text {ideal }}\left(\psi^{r,s}_{v}[l_v], \psi^{r,s}_{h}[l_v][l_h]\right)-G\left(\psi^{r,s}_{v}[l_v], \psi^{r,s}_{h}[l_v][l_h], \c\right)\right| \label{alternative_opt}
% \end{align}
% where, 

% \begin{align}
%     &\psi^{r,s}_v[l_v] = \psi^{r-1, s}_v + l_v\frac{\delta_v^r}{L_v}\nonumber\\
%     &\psi^{r,s}_h[l_v][l_h] = \psi^{r, s-1}_h[l_v] + l_h\frac{\delta_h^{r,s, l_v}}{L_h}
% \end{align}

% We can rewrite the optimization problem \eqref{alternative_opt} as, 

% \begin{align}
%     \c_{p,q}^{opt}=\arg \min _{\c, \|\c\|=1} \underset{L_h, L_v \rightarrow \infty}{\lim}\frac{1}{L_hL_v}\left\|\mathbf{G}^{\text {ideal }}_{p,q}-\mathbf{G}(\c)\right\|
% \end{align}


% where, 

% \begin{align}
%     &\mathbf{G}(\c) = \nonumber\\&\left[\delta^1_v\delta^{1,1,1}_hG\left(\psi_{v}^{1,1}[1], \psi_{h}^{1,1}[1][1], \c\right) \cdots \right. \nonumber\\ &\left. \delta^{Q_v}_v\delta^{Q_v,Q_h,L_v}_hG\left(\psi_{v}^{Q_{v}, Q_{h}}\left[L_{v}\right], \psi_{h}^{Q_{v}, Q_{h}}\left[L_{v}\right]\left[L_{h}\right], \c\right)\right]^{T} 
% \end{align}

% and, 

% \begin{align}
%     &\mathbf{G}^{\text {ideal }}_{ p, q} = \nonumber\\&\left[\delta^1_v\delta^{1,1,1}_hG^{\text {ideal }}_{ p, q}\left(\psi_{v}^{1,1}[1], \psi_{h}^{1,1}[1][1] \right) \cdots \right. \nonumber\\ &\left. \delta^{Q_v}_v\delta^{Q_v,Q_h,L_v}_hG^{\text {ideal }}_{ p, q}\left(\psi_{v}^{Q_{v}, Q_{h}}\left[L_{v}\right], \psi_{h}^{Q_{v}, Q_{h}}\left[L_{v}\right]\left[L_{h}\right]\right)\right]^{T} 
% \end{align}


% % \begin{align}
% %      &\mathbf{G}^{\text {ideal }}_{ q, p}= \nonumber\\&\left[{\delta_{1,1}} G^{\text {ideal }} _{q, p}\left(\psi^{1}_{h}[1], \psi^{1}_{v}[1], \c\right) \ldots {\delta_{Q_h, Q_v}}G^{\text {ideal }} _{q, p}\left(\psi^{Q_h}_{h}[L_h], \psi^{Q_v}_{v}[L_v], \c\right)\right]^{T} 
% % \end{align}


% Note that we can write, 

% \begin{align}
%     \mathbf{G}^{\text{ideal }}_{p,q} = \frac{{(2\pi)}^2}{\delta_{p,q}} \left(\e_{p,q} \otimes (\boldsymbol{\delta}_{p,q}\odot\mathbf{1}_{L, 1})\right)
% \end{align}

% where, $\boldsymbol\delta_{p,q} = $
% with $\mathbf{e}_{p,q} \in \mathbb{Z}^{Q}$ being the standard basis vector for the $(p,q)$-th axis among $Q$ ones. 








% Now, note that $\mathbf{1}_{L, 1}=\mathbf{g} \odot \mathbf{g}^{*}$ for any equal gain $\mathbf{g} \in \mathbb{C}^L$. Therefore, we can write: 

% % $$\mathcal{G}_{L}=\left\{\mathbf{g} \in \mathbb{C}^{L}:\left(\mathbf{g} \mathbf{g}^{H}\right)_{\ell, \ell}=1,1 \leq \ell \leq L\right\}$$


% \begin{align}
% \mathbf{G}^{\text {ideal }}_{p,q} &=\frac{{2\pi}^2}{\delta_{p,q}}\left(\mathbf{e}_{p,q} \otimes\left((\boldsymbol\delta^{(\frac{1}{2})}_{p,q}\odot\boldsymbol\delta^{(\frac{1}{2})}_{p,q})\odot(\mathbf{g} \odot \mathbf{g}^{*})\right)\right) \nonumber\\
% &=\left(\frac{{2\pi}}{\sqrt{\delta_{p,q}}}\left(\mathbf{e}_{p,q} \otimes (\boldsymbol\delta^{(\frac{1}{2})}_{p,q}\odot\mathbf{g})\right)\right) \nonumber\\&\odot\left(\frac{{(2\pi)}}{\sqrt{\delta_{p,q}}}\left(\mathbf{e}_{p,q} \otimes (\boldsymbol\delta^{(\frac{1}{2})}_{p,q}\odot\mathbf{g})\right)\right)^{*} \label{g_id_q_equivalent}
% \end{align}

% Also, note that we can write, 

% % \begin{align}
% %     \mathbf{G}(\c)=\left(\left(\mathbf{D}_{h}^{H} \otimes \mathbf{D}_{v}^{H}\right) \c\right) \odot\left(\left(\mathbf{D}_{h}^{H} \otimes \mathbf{D}_{v}^{H}\right) \c\right)^{*}
% % \end{align}
% \begin{align}
%     \mathbf{G}(\c)=\left(\D^H \c\right) \odot\left(\D^H \c\right)^{*}
% \end{align}

% where 

% \begin{align}
%     \D = \left[D^{1,1}_{vh}, \cdots, D^{Q_v,Q_h}_{vh}\right] \in \mathbb{C}^{M_vM_h\times L_vL_hQ_vQ_h}
% \end{align}

% % \begin{align}
% % % &\mathbf{D}_{v} =\left[\mathbf{D}_{v, 1}, \cdots, \mathbf{D}_{v, Q_{v}}\right] \in \mathbb{C}^{M_{v} \times L_{v} Q_{v}} \\
% % % &\mathbf{D}_{h} =\left[\mathbf{D}_{h, 1}, \cdots, \mathbf{D}_{h, Q_{h}}\right] \in \mathbb{C}^{M_{h} \times L_{h} Q_{h}} \\
% % % &\mathbf{D}_{v, p} =\left[\mathbf{d}_{M_{v}}\left(\psi_{v}^{p}[1]\right), \cdots, \mathbf{d}_{M_{v}}\left(\psi_{v}^{p}\left[L_{v}\right]\right)\right] \in \mathbb{C}^{M_{v} \times L_{v}}\\
% % % &\mathbf{D}_{h, q} =\left[\mathbf{d}_{M_{h}}\left(\psi_{h}^{q}[1]\right), \cdots, \mathbf{d}_{M_{h}}\left(\psi_{h}^{q}\left[L_{h}\right]\right)\right] \in \mathbb{C}^{M_{h} \times L_{h}}\\
% % &\mathbf{D}_{h, q}(\chi) =\left[\mathbf{d}_{M_{h}}\left(\psi_{h}^{q}[1], \chi \right), \cdots, \mathbf{d}_{M_{h}}\left(\psi_{h}^{q}\left[L_{h}\right], \chi \right)\right] \in \mathbb{C}^{M_{h} \times L_{h}}\\
% % &\mathbf{D}_{vh, p,q} =\left[\mathbf{d}_{M_{v}}\left(\psi_{v}^{p}[1]\right) \otimes \mathbf{D}_{h, q}(\psi_{v}^{p}[1]), \cdots, \mathbf{d}_{M_{v}}\left(\psi_{v}^{p}\left[L_{v}\right]\right) \otimes \mathbf{D}_{h, q}(\psi_{v}^{p}\left[L_{v}\right]) \right] \in \mathbb{C}^{M_{v}M_{h} \times L_{v}L_{h}}
% % \end{align}

% \begin{align}
% % &\mathbf{D}_{v} =\left[\mathbf{D}_{v, 1}, \cdots, \mathbf{D}_{v, Q_{v}}\right] \in \mathbb{C}^{M_{v} \times L_{v} Q_{v}} \\
% % &\mathbf{D}_{h} =\left[\mathbf{D}_{h, 1}, \cdots, \mathbf{D}_{h, Q_{h}}\right] \in \mathbb{C}^{M_{h} \times L_{h} Q_{h}} \\
% % &\mathbf{D}_{v, p} =\left[\mathbf{d}_{M_{v}}\left(\psi_{v}^{p}[1]\right), \cdots, \mathbf{d}_{M_{v}}\left(\psi_{v}^{p}\left[L_{v}\right]\right)\right] \in \mathbb{C}^{M_{v} \times L_{v}}\\
% % &\mathbf{D}_{h, q} =\left[\mathbf{d}_{M_{h}}\left(\psi_{h}^{q}[1]\right), \cdots, \mathbf{d}_{M_{h}}\left(\psi_{h}^{q}\left[L_{h}\right]\right)\right] \in \mathbb{C}^{M_{h} \times L_{h}}\\
% \mathbf{D}^{p,q}_{h}(\ell) =&\sqrt{\delta_v^{p}\delta^{p,q,\ell}_h}\left[\mathbf{d}_{M_{h}}\left(\psi_{h}^{p,q}[\ell][1] \right), \cdots, \right.\nonumber\\&\left. \mathbf{d}_{M_{h}}\left(\psi_{h}^{p,q}\left[\ell] [L_{h} \right] \right)\right] \in \mathbb{C}^{M_{h} \times L_{h}}, 
% \end{align}

% \begin{align}
%     \mathbf{D}_{vh}^{p,q} =&\left[\mathbf{d}_{M_{v}}\left(\psi_{v}^{p,q}[1]\right) \otimes \mathbf{D}_{h}^{p,q}(1), \cdots, \right.\nonumber\\&\left. \mathbf{d}_{M_{v}}\left(\psi_{v}^{p,q}\left[L_{v}\right]\right) \otimes \mathbf{D}_{h}^{p,q}(L_{v}) \right] \in \mathbb{C}^{M_{v}M_{h} \times L_{v}L_{h}}
% \end{align}
% \begin{problem}
% Given an equal-gain vector $\g_{p,q} \in \mathbb{C}^L$, $(p,q) \in \{(1,1), \cdots, (Q_v, Q_h)\}$, find vector $\c_{p,q} \in \mathbb{C}^{M_t}$ such that
% \begin{align}
% &\c_{p,q}=\underset{\c, \|c\|=1}{\arg \min } \lim_{L\rightarrow \infty} \left\|\frac{{2\pi}}{\sqrt{\delta_{p,q}}}\left(\mathbf{e}_{p,q} \otimes (\boldsymbol\delta^{(\frac{1}{2})}_{L,1}\odot\mathbf{g}_{p,q})\right)- \D^H \c\right\|^{2} \label{obj_func}
% \end{align}
% \label{main_problem_UPA}
% \end{problem}


% Note that the solution to problem \ref{main_problem_UPA} is the limit of the sequence of solutions to a least-square optimization problem as $L$ goes to infinity. For each $L$ we find that,
%  \begin{align}
%  {\c}^{(L)}_{p,q} &= {\frac{{2\pi}}{\sqrt{\delta_{p,q}}}}(\D \D^H)^{-1} \D  \left(\mathbf{e}_{p,q} \otimes (\boldsymbol\delta^{(\frac{1}{2})}_{L,1}\odot\mathbf{g}_{p,q})\right) \nonumber\\& =\sigma \D^{p,q}(\boldsymbol\delta^{(\frac{1}{2})}_{L,1}\odot\mathbf{g}_{p,q})
% \end{align}
% where $\sigma = \frac{2\pi \sqrt{\delta_{p,q}}}{L\sum_{(q,p)= (1,1)}^{(Q_h, Q_v)} \delta_{p,q}}$, noting that it holds that, 

% Noting that it holds that, 

% \begin{align}
%     (\D\D^H) = \left(L\sum_{(p,q)= (1,1)}^{(Q_h, Q_v)}\sum_{(l_v,l_h)= (1,1)}^{(L_h, L_v)} \delta_{v}^p\delta^{p,q,l_v}_{h}\right)\I_{M_t}
% \end{align}

% \begin{problem}
% Find equal-gain $\g_{p,q} \in \mathbb{C}^L$, $(q,p) = (1,1), \ldots, (Q_h, Q_v)$, such that
% \begin{equation}
%  \g_{q,p} = \underset{\g}{\arg\min }\left\| abs(\D^H \c_{q,p})- {2\pi} abs(\mathbf{e}_{q,p} \otimes \mathbf{g})\right\|^{2} \label{g_final_eq}
% \end{equation} 
% where $abs(.)$ denotes the element-wise absolute value of a vector.
% \label{g_problem}
% \end{problem}


% \begin{proposition}
% The minimizer of \eqref{g_simple_final_eq} is in the form $\g_q = \left[{\begin{array}{cccc} 1& \alpha^\eta &\cdots & \alpha_h^{\eta (L_h -1)}\alpha_v^{\eta (L_v -1)} \end{array}}\right]^T$ for some $\eta$ where $\alpha_h = e^{j(\frac{\delta^{q,p}_h}{L_h})}$ and $\alpha_v = e^{j(\frac{\delta^p_v}{L_v})}$. \label{proposiiton_g}
% \end{proposition}

%  \begin{align}
%      {\c_{p,q}}^{(L)} & = \sigma \sum_{(l_v, l_h)=(1,1)}^{(L_v, L_h)}g_{p,q, l}\mathbf{d}_{M_t}\left(\psi^{p,q}_{v}[l_v], \psi^{p,q}_{h}[l_v][l_h]\right)  \nonumber\\
%      & = \sigma \sum_{l=1}^L g_{q,p,l}\left[{\begin{array}{ccc}
%      1 &\cdots & e^{j\left( (M_v-1)\psi^{p,q}_{v}[l_v] + (M_h-1)\psi^{p,q}_{h}[l_v][l_h]\right)}\\
%      \end{array}}\right]^T   
%     %  \nonumber\\& = \left[{\begin{array}{ccc}
%     %  \sigma\sum_{l=1}^L g_{q, p, l}& \cdots & \sigma\sum_{l=1}^L g_{q,p, l}e^{j\left((M_h-1)\psi^q_{h}[l] + (M_v-1)\psi^p_{v}[l]\right)}\\
%     %  \end{array}}\right]^T
% \end{align}

% We can then write 

% \begin{align}
%     c_{p,q, m_v, m_h} &= \lim_{L_h, L_v\rightarrow \infty} \frac{1}{L_hL_v}\sum_{(l_h, l_v)=(1,1)}^{(L_h, L_v)} \nonumber \\&\delta^p_v\delta_h^{p,q,l_v}g_{p,q, l_v, l_h}e^{j\left(m_v\psi^{p,q}_{v}[l_v]+ m_h\psi^{p,q}_{h}[l_v][l_h] \right)}
% \end{align}


% \begin{align}
%     c_{p,q, m_v, m_h} &=  \nonumber\\&\lim_{L_h, L_v\rightarrow \infty} \frac{1}{L_hL_v}\sum_{(l_h, l_v)=(1,1)}^{(L_h, L_v)} \delta^p_v\delta_h^{p,q,l_v}g_{p,q, l_v, l_h}\nonumber \\&e^{j\left(m_v(\psi^{p-1,q}_{v}+l_v\frac{\delta^p_v}{L_v})+ m_h(\psi^{p,q-1}_{h}[l_v] + l_h\frac{\delta^{p,q,l_v}_h}{L_h}) \right)}
% \end{align}


% \begin{align}
%     c_{p,q, m_v, m_h} &=  \lim_{ L_v\rightarrow \infty} \frac{\delta^p_v}{L_v}\sum_{l_v=1}^{L_v} e^{j(m_v(\psi^{p-1,q}_{v}+l_v\frac{\delta^p_v}{L_v}) + x\eta\frac{l_v}{L_v})} S^{(l_v)}
% \end{align}

% where, 

% \begin{align}
%     S^{(l_v)} &= \lim_{ L_h\rightarrow \infty}\frac{\delta^{p,q,l_v}_h}{L_h}\sum_{l_h=1}^{L_h}e^{j(m_h(\psi^{p,q-1}_{h}[l_v] + l_h\frac{\delta^{p,q,l_v}_h}{L_h}) + y\gamma \frac{l_h}{L_h})} \nonumber\\& = \delta^{p,q,l_v}_h e^{jm_h\psi^{p,q-1}_{h}[l_v]}sinc(\frac{\xi}{2\pi})
% \end{align}

% with $\xi = {\delta^{p,q,l_v}_h}{m} + y\gamma$. Therefore, it. holds that, 

% \begin{align}
%     c_{p,q, m_v, m_h} &=  \lim_{ L_v\rightarrow \infty} \frac{\delta^p_v}{L_v}\sum_{l_v=1}^{L_v} e^{j(m_v(\psi^{p-1,q}_{v}+l_v\frac{\delta^p_v}{L_v}) + x\eta\frac{l_v}{L_v})} \delta^{p,q,l_v}_h e^{jm_h\psi^{p,q-1}_{h}[l_v]}sinc(\frac{\xi}{2\pi})
% \end{align}
% \subsection{UPA Codebook Design Prolem}

Suppose an $M_h \times M_v$ UPA is placed at the $x-z$ plane, and let $M = M_h M_v$. Let the distance between the antennas that are placed parallel to $z$ axis, be $d_z$, and $d_x$ respectively. One can define,

\begin{align}
\mathbf{d}_{M_{a}}\left(\psi_{a}\right) = \left[1, e^{j \psi_{a}} \cdots e^{j\left(M_{a}-1\right) \psi_{a}}\right]^{T} \in \mathbb{C}^{M_{a}}
\end{align}
where, $\psi_{h}=\frac{2 \pi d_{x}}{\lambda} \sin \theta \cos \phi \text { and } \psi_{v}=\frac{2 \pi d_{z}}{\lambda} \sin \phi$, and $a$ takes value in the set $\{v, h\}$.  
The array response vector is then defined as 

\begin{align}
    \mathbf{d}_{M}\left(\psi_{v}, \psi_{h}\right) =
    \mathbf{d}_{M_{v}}\left(\psi_{v}\right) \otimes
    \mathbf{d}_{M_{h}}\left(\psi_{h}\right)  \in \mathbb{C}^{M}
\end{align}
Let $\mathcal{B}_s$ be the angular range under cover in the $(\psi_h, \psi_v)$ domain. We have, 

\begin{equation}
    \mathcal{B}^\psi_{s} \doteq\left[-\psi_h^{\mathrm{B}}, \psi_h^{\mathrm{B}}\right) \times\left[-\psi_v^{\mathrm{B}}, \psi_v^{\mathrm{B}}\right)
\end{equation}

% \begin{equation}
%     \mathcal{B}_{s} \doteq \left[-\phi^{\mathrm{B}}, \phi^{\mathrm{B}}\right)
%     \times \left[-\theta^{\mathrm{B}}, \theta^{\mathrm{B}}\right) \label{range_angle}
% \end{equation}

% Notation $\psi^{p,q}_{h}[1:L_h,1 :L_v], \psi^{p,q}_{v}[1:L_v]$

Let us uniformly divide the angular range  into $Q=Q_{v} Q_{h}$ subregions. For each subregion we have, 

% $$ \mathcal{B}_{p, q} =  \omega_{\phi, p} \times \omega_{\theta, q}$$

% Let $\Delta_{\theta} =  2 \theta^{\mathrm{B}} / Q_{v}$, and $\Delta_\phi =  2 \phi^{\mathrm{B}} / Q_{h}$ be the length of each sub-interval along axis $\theta$, and $\phi$. We have $\omega_{\theta, q} = [\theta_{q-1}, \theta_q)$, and, $\omega_{\phi, p} = [\phi_{p-1},\phi_p)$ where  $\theta_q = -\theta^B + q\Delta_\theta$, and $\phi_p = -\phi^B + p\Delta_\phi$. 


$$ \mathcal{B}_{ p, q} \doteq \nu_{v}^{p, q} \times \nu_{h}^ {p, q}$$

% We can then write for each $v_{a, b}$, i.e the $b$-th sub-interval along axis $a$,

% \begin{equation}
%     \nu_{v, p} =-\psi_{a}^{\mathrm{B}}+(b-1)\Delta_{a}+\left[0, \Delta_{a}\right)
% \end{equation}

Define the reference gain at $(\psi_{h}, \psi_{v}) $ as, 

\begin{equation}
    G\left(\psi_{v}, \psi_{h}, \mathbf{c}\right) =\left|\left(\mathbf{d}_{M_{v}}\left(\psi_{v}\right)\otimes\mathbf{d}_{M_{h}}\left(\psi_{h}\right)  \right)^{H} \mathbf{c}\right|^{2}
\end{equation}

It is straightforward to show that, 

\begin{equation}
    \int_{-\pi}^{\pi} \int_{-\pi}^{\pi} G\left(\psi_{v}, \psi_{h}, \mathbf{c}\right) d \psi_{v} d \psi_{h}={(2 \pi)^{2}}
\end{equation}

Therefore, for every interval $\mathcal{B}_{ p, q}$ we can derive the ideal gain expression as, 

\begin{equation}
    G_{p, q}^{\text {ideal }}\left(\psi_{v}, \psi_{h}\right)=G_{p,q} \mathds{1}_{\mathcal{B}^\psi_{p, q}}\left(\psi_{v}, \psi_{h}\right)
\end{equation}

where $G_{q,p} = \frac{(2\pi)^2}{\delta_{p,q}}$, where $\delta_{p,q}$ is the area of the interval $\mathcal{B}^\psi_{p,q}$. We can write $\delta_{p,q} = \delta_v\delta_h = (\frac{2\psi^B_v}{Q_v})(\frac{2\psi^B_h}{Q_h})$. Therefore, We define the codebook design problem under the UPA structure as follows. 
% where $G_{q,p} = \frac{(2\pi)^2}{\delta^{p,q}_h\delta^p_v}$, $\delta^p_v = \psi^p_v - \psi^{p-1}_v $, and $\delta^{q, p}_h = \psi^{q,p}_h - \psi^{q-1, p}_h $. Also, let us define $\delta_{q,p} = \delta^{q, p}_h\delta^p_v$. 



\begin{align}
& \c^{opt}_{p, q} = \nonumber\\&\underset{\c, \|\c\|=1}{\arg \min } \iint_{\mathcal{B}_{s}}\left|G_{p, q}^{\text {ideal }}\left(\psi_{v}, \psi_{h}\right)-G\left(\psi_{v}, \psi_{h}, \c\right)\right| d \psi_{v} d \psi_{h} \label{init_opt}
\end{align}

By partitioning the range of $(\psi_h, \psi_v)$ into the pre-defined intervals, we can rewrite the optimization problem as follows, 

\begin{align}
& \c^{opt}_{p, q} = \underset{\c, \|\c\|=1}{\arg \min }\nonumber\\& \sum_{r=1}^{Q_v}\sum_{s=1}^{Q_h}\iint_{\mathcal{B}^{\psi}_{r,s}}\left|G_{p, q}^{\text {ideal }}\left(\psi_{v}, \psi_{h}\right)-G\left(\psi_{v}, \psi_{h}, \c\right)\right| d \psi_{v} d \psi_{h} \nonumber\\&
 = \underset{L_h, L_v \rightarrow \infty}{\lim}\sum_{r=1}^{Q_v}\sum_{s=1}^{Q_h}\sum_{l_v =1}^{L_v}\sum_{l_h =1}^{L_h}\frac{\delta_v\delta_h}{L_hL_v}\nonumber\\&
\left|G_{p, q}^{\text {ideal }}\left(\psi^{r,s}_{v}[l_v], \psi^{r,s}_{h}[l_h]\right)-G\left(\psi^{r,s}_{v}[l_v], \psi^{r,s}_{h}[l_h], \c\right)\right| \label{alternative_opt}
\end{align}
where, 

\begin{align}
    &\psi^{r,s}_v[l_v] = \psi^{r-1, s}_v + l_v\frac{\delta_v}{L_v}\nonumber\\
    &\psi^{r,s}_h[l_h] = \psi^{r, s-1}_h + l_h\frac{\delta_h}{L_h}
\end{align}

We can rewrite the optimization problem \eqref{alternative_opt} as, 

\begin{align}
    \c_{p,q}^{opt}=\arg \min _{\c, \|\c\|=1} \underset{L_h, L_v \rightarrow \infty}{\lim}\frac{1}{L_hL_v}\left\|\mathbf{G}^{\text {ideal }}_{p,q}-\mathbf{G}(\c)\right\|
\end{align}


where, 

\begin{align}
    \mathbf{G}(\c) =& \delta_v\delta_h\left[G\left(\psi_{v}^{1,1}[1], \psi_{h}^{1,1}[1], \c\right) \cdots \right. \nonumber\\ &\left. G\left(\psi_{v}^{Q_{v}, Q_{h}}\left[L_{v}\right], \psi_{h}^{Q_{v}, Q_{h}}\left[L_{h}\right], \c\right)\right]^{T} 
\end{align}

and, 

\begin{align}
    \mathbf{G}^{\text {ideal }}_{ p, q} =& \delta_v\delta_h\left[G^{\text {ideal }}_{ p, q}\left(\psi_{v}^{1,1}[1], \psi_{h}^{1,1}[1] \right) \cdots \right. \nonumber\\ &\left. G^{\text {ideal }}_{ p, q}\left(\psi_{v}^{Q_{v}, Q_{h}}\left[L_{v}\right], \psi_{h}^{Q_{v}, Q_{h}}\left[L_{h}\right]\right)\right]^{T} 
\end{align}


% \begin{align}
%      &\mathbf{G}^{\text {ideal }}_{ q, p}= \nonumber\\&\left[{\delta_{1,1}} G^{\text {ideal }} _{q, p}\left(\psi^{1}_{h}[1], \psi^{1}_{v}[1], \c\right) \ldots {\delta_{Q_h, Q_v}}G^{\text {ideal }} _{q, p}\left(\psi^{Q_h}_{h}[L_h], \psi^{Q_v}_{v}[L_v], \c\right)\right]^{T} 
% \end{align}


Note that we can write, 

\begin{align}
    \mathbf{G}^{\text{ideal }}_{p,q} = \delta_v\delta_h\frac{{(2\pi)}^2}{\delta_v\delta_h} \left(\e_{p,q} \otimes \mathbf{1}_{L, 1}\right)
\end{align}


with $\mathbf{e}_{p,q} \in \mathbb{Z}^{Q}$ being the standard basis vector for the $(p,q)$-th axis among $Q$ ones. 








Now, note that $\mathbf{1}_{L, 1}=\mathbf{g} \odot \mathbf{g}^{*}$ for any equal gain $\mathbf{g} \in \mathbb{C}^L$. Therefore, we can write: 

% $$\mathcal{G}_{L}=\left\{\mathbf{g} \in \mathbb{C}^{L}:\left(\mathbf{g} \mathbf{g}^{H}\right)_{\ell, \ell}=1,1 \leq \ell \leq L\right\}$$


\begin{align}
\mathbf{G}^{\text {ideal }}_{p,q} &={{(2\pi)}^2}\left(\mathbf{e}_{p,q} \otimes(\mathbf{g} \odot \mathbf{g}^{*})\right) \nonumber\\
&=\left({{2\pi}}\left(\mathbf{e}_{p,q} \otimes \mathbf{g}\right)\right) \odot \left({{2\pi}}\left(\mathbf{e}_{p,q} \otimes \mathbf{g}\right)\right)^* \label{g_id_q_equivalent}
\end{align}

Also, note that we can write, 

% \begin{align}
%     \mathbf{G}(\c)=\left(\left(\mathbf{D}_{h}^{H} \otimes \mathbf{D}_{v}^{H}\right) \c\right) \odot\left(\left(\mathbf{D}_{h}^{H} \otimes \mathbf{D}_{v}^{H}\right) \c\right)^{*}
% \end{align}
\begin{align}
    \mathbf{G}(\c)=\left(\D^H \c\right) \odot\left(\D^H \c\right)^{*}
\end{align}

\noindent where, $\D^H = \sqrt{\delta_v\delta_h}(\mathbf{D}_{v}^{H} \otimes \mathbf{D}_{h}^{H})$, and for $a \in \{v,h\}$, we have, 

\begin{align}
\mathbf{D}_{a} &=\left[\mathbf{D}_{a, 1}, \cdots, \mathbf{D}_{a, Q_{a}}\right] \in \mathbb{C}^{M_{a} \times L_{a} Q_{a}} \\
\mathbf{D}_{a, b} &=\left[\mathbf{d}_{M_{a}}\left(\psi_{a}^{b}[1]\right), \cdots, \mathbf{d}_{M_{a}}\left(\psi_{a}^{b}\left[L_{a}\right]\right)\right] \in \mathbb{C}^{M_{a} \times L_{a}}
\end{align}




% where 

% \begin{align}
%     \D = \left[D^{1,1}_{vh}, \cdots, D^{Q_v,Q_h}_{vh}\right] \in \mathbb{C}^{M_vM_h\times L_vL_hQ_vQ_h}
% \end{align}

% \begin{align}
% % &\mathbf{D}_{v} =\left[\mathbf{D}_{v, 1}, \cdots, \mathbf{D}_{v, Q_{v}}\right] \in \mathbb{C}^{M_{v} \times L_{v} Q_{v}} \\
% % &\mathbf{D}_{h} =\left[\mathbf{D}_{h, 1}, \cdots, \mathbf{D}_{h, Q_{h}}\right] \in \mathbb{C}^{M_{h} \times L_{h} Q_{h}} \\
% % &\mathbf{D}_{v, p} =\left[\mathbf{d}_{M_{v}}\left(\psi_{v}^{p}[1]\right), \cdots, \mathbf{d}_{M_{v}}\left(\psi_{v}^{p}\left[L_{v}\right]\right)\right] \in \mathbb{C}^{M_{v} \times L_{v}}\\
% % &\mathbf{D}_{h, q} =\left[\mathbf{d}_{M_{h}}\left(\psi_{h}^{q}[1]\right), \cdots, \mathbf{d}_{M_{h}}\left(\psi_{h}^{q}\left[L_{h}\right]\right)\right] \in \mathbb{C}^{M_{h} \times L_{h}}\\
% &\mathbf{D}_{h, q}(\chi) =\left[\mathbf{d}_{M_{h}}\left(\psi_{h}^{q}[1], \chi \right), \cdots, \mathbf{d}_{M_{h}}\left(\psi_{h}^{q}\left[L_{h}\right], \chi \right)\right] \in \mathbb{C}^{M_{h} \times L_{h}}\\
% &\mathbf{D}_{vh, p,q} =\left[\mathbf{d}_{M_{v}}\left(\psi_{v}^{p}[1]\right) \otimes \mathbf{D}_{h, q}(\psi_{v}^{p}[1]), \cdots, \mathbf{d}_{M_{v}}\left(\psi_{v}^{p}\left[L_{v}\right]\right) \otimes \mathbf{D}_{h, q}(\psi_{v}^{p}\left[L_{v}\right]) \right] \in \mathbb{C}^{M_{v}M_{h} \times L_{v}L_{h}}
% \end{align}

% \begin{align}
% % &\mathbf{D}_{v} =\left[\mathbf{D}_{v, 1}, \cdots, \mathbf{D}_{v, Q_{v}}\right] \in \mathbb{C}^{M_{v} \times L_{v} Q_{v}} \\
% % &\mathbf{D}_{h} =\left[\mathbf{D}_{h, 1}, \cdots, \mathbf{D}_{h, Q_{h}}\right] \in \mathbb{C}^{M_{h} \times L_{h} Q_{h}} \\
% % &\mathbf{D}_{v, p} =\left[\mathbf{d}_{M_{v}}\left(\psi_{v}^{p}[1]\right), \cdots, \mathbf{d}_{M_{v}}\left(\psi_{v}^{p}\left[L_{v}\right]\right)\right] \in \mathbb{C}^{M_{v} \times L_{v}}\\
% % &\mathbf{D}_{h, q} =\left[\mathbf{d}_{M_{h}}\left(\psi_{h}^{q}[1]\right), \cdots, \mathbf{d}_{M_{h}}\left(\psi_{h}^{q}\left[L_{h}\right]\right)\right] \in \mathbb{C}^{M_{h} \times L_{h}}\\
% \mathbf{D}^{p,q}_{h}(\ell) =&\sqrt{\delta_v^{p}\delta^{p,q,\ell}_h}\left[\mathbf{d}_{M_{h}}\left(\psi_{h}^{p,q}[\ell][1] \right), \cdots, \right.\nonumber\\&\left. \mathbf{d}_{M_{h}}\left(\psi_{h}^{p,q}\left[\ell] [L_{h} \right] \right)\right] \in \mathbb{C}^{M_{h} \times L_{h}}, 
% \end{align}

% \begin{align}
%     \mathbf{D}_{vh}^{p,q} =&\left[\mathbf{d}_{M_{v}}\left(\psi_{v}^{p,q}[1]\right) \otimes \mathbf{D}_{h}^{p,q}(1), \cdots, \right.\nonumber\\&\left. \mathbf{d}_{M_{v}}\left(\psi_{v}^{p,q}\left[L_{v}\right]\right) \otimes \mathbf{D}_{h}^{p,q}(L_{v}) \right] \in \mathbb{C}^{M_{v}M_{h} \times L_{v}L_{h}}
% \end{align}

\begin{problem}
Given an equal-gain vector $\g_{p,q} \in \mathbb{C}^L$, $(p,q) \in \{(1,1), \cdots, (Q_v, Q_h)\}$, find vector $\c_{p,q} \in \mathbb{C}^{M_t}$ such that
\begin{align}
&\c_{p,q}=\underset{\c, \|c\|=1}{\arg \min } \lim_{L\rightarrow \infty} \left\|{{2\pi}}\left(\mathbf{e}_{p,q} \otimes \mathbf{g}\right)- \D^H \c\right\|^{2} \label{obj_func}
\end{align}
\label{main_problem_UPA}
\end{problem}


Note that the solution to problem \ref{main_problem_UPA} is the limit of the sequence of solutions to a least-square optimization problem as $L$ goes to infinity. For each $L$ we find that,
 \begin{align}
 {\c}^{(L)}_{p,q} &= {{{2\pi}}}(\D \D^H)^{-1} \D  \left(\mathbf{e}_{p,q} \otimes \mathbf{g}_{p,q}\right) \nonumber\\& =\sigma \D_{p,q}\mathbf{g}_{p,q}
\end{align}
where $\sigma = \frac{2\pi \sqrt{\delta_{v}\delta_{h}}}{LQ\delta_{v}\delta_{h}} = \frac{2\pi}{LQ\sqrt{\delta_v\delta_h}}$, noting that it holds that, 

Noting that it holds that, 

\begin{align}
    (\D\D^H) = \left(L\sum_{(p,q)= (1,1)}^{(Q_h, Q_v)}\sum_{(l_v,l_h)= (1,1)}^{(L_h, L_v)} \delta_{v}\delta_{h}\right)\I_{M_t}
\end{align}

\begin{problem}
Find equal-gain $\g_{p,q} \in \mathbb{C}^L$, $(p,q) = (1,1), \ldots, (Q_h, Q_v)$, such that
\begin{equation}
 \g_{q,p} = \underset{\g}{\arg\min }\left\| abs(\D^H \c_{p,q})- {2\pi} abs(\mathbf{e}_{p,q} \otimes \mathbf{g})\right\|^{2} \label{g_final_eq}
\end{equation} 
where $abs(.)$ denotes the element-wise absolute value of a vector.
\label{g_problem}
\end{problem}


\begin{proposition}
The minimizer of \eqref{g_simple_final_eq} is in the form $\g_q = \left[{\begin{array}{cccc} 1& \alpha^\eta &\cdots & \alpha_v^{\eta (L_v -1)}\alpha_h^{\eta (L_h -1)} \end{array}}\right]^T$ for some $\eta$ where $\alpha_v = e^{j(\frac{\eta_v}{L_v})}$ and $\alpha_h = e^{j(\frac{\eta_h}{L_h})}$. \label{proposiiton_g}
\end{proposition}

 \begin{align}
     {\c_{p,q}}^{(L)} & = \sigma \sum_{(l_v, l_h)=(1,1)}^{(L_v, L_h)}g_{p,q, l}\mathbf{d}_{M_t}\left(\psi^{p}_{v}[l_v], \psi^{q}_{h}[l_h]\right)  \nonumber\\
     & = \sigma \sum_{l=1}^L g_{q,p,l}\left[{\begin{array}{ccc}
     1 &\cdots & e^{j\phi_{p,q}^{M_v-1, M_h-1}}\\
     \end{array}}\right]^T   
    %  \nonumber\\& = \left[{\begin{array}{ccc}
    %  \sigma\sum_{l=1}^L g_{q, p, l}& \cdots & \sigma\sum_{l=1}^L g_{q,p, l}e^{j\left((M_h-1)\psi^q_{h}[l] + (M_v-1)\psi^p_{v}[l]\right)}\\
    %  \end{array}}\right]^T
\end{align}

where $ \phi_{p,q}^{m_v, m_h} = \left( m_v\psi^{p}_{v}[l_v] + m_h\psi^{q}_{h}[l_h]\right)$. We can then write 

\begin{align}
    c_{p,q, m_v, m_h} &= \nonumber\\& \lim_{L_h, L_v\rightarrow \infty} \frac{1}{L_hL_v}\sum_{(l_h, l_v)=(1,1)}^{(L_h, L_v)} g_{p,q, l_v, l_h}e^{j\phi_{p,q}^{M_v-1, M_h-1}}
\end{align}


\begin{align}
    c_{p,q, m_v, m_h} &=  \lim_{L_h, L_v\rightarrow \infty} \frac{1}{L_hL_v}\sum_{(l_h, l_v)=(1,1)}^{(L_h, L_v)} g_{p,q, l_v, l_h}\nonumber \\&e^{j\left(m_v(\psi^{p-1}_{v}+l_v\frac{\delta_v}{L_v})+ m_h(\psi^{q-1}_{h} + l_h\frac{\delta_h}{L_h}) \right)}
\end{align}


\begin{align}
    c_{p,q, m_v, m_h} =&  \frac{2\pi}{Q}e^{j\phi_{p-1, q-1}^{m_v, m_h}}
    \left(\frac{1}{L_v}\lim_{ L_v\rightarrow \infty} \sum_{l_v=1}^{L_v} e^{j\frac{\eta_v+ m_v\delta_v}{L_v} l_v}\right) \nonumber\\&
    \left(\frac{1}{L_h}\lim_{ L_h\rightarrow \infty} \sum_{l_h=1}^{L_h} e^{j\frac{\eta_h+ m_h\delta_h}{L_h} l_h}\right)
\end{align}

\begin{align}
    c_{p,q, m_v, m_h} &=  \frac{2\pi}{Q}e^{j\phi_{p-1, q-1}^{m_v, m_h}}
    \int_{0}^{1} e^{j\xi_v x}dx\int_{0}^{1} e^{j\xi_h x}dx \nonumber\\&
    = \frac{2\pi}{Q}e^{j(\phi_{p-1, q-1}^{m_v, m_h}+ \frac{\xi_v+\xi_h}{2})} sinc(\frac{\xi_v}{2\pi})sinc(\frac{\xi_h}{2\pi})
\end{align}

with $\xi_a = {\delta_a}{m_a} + \eta_a$, where $a\in \{v,h\}$. 
% Therefore, it. holds that, 

% \begin{align}
%     c_{p,q, m_v, m_h} &=  \lim_{ L_v\rightarrow \infty} \frac{\delta^p_v}{L_v}\sum_{l_v=1}^{L_v} e^{j(m_v(\psi^{p-1,q}_{v}+l_v\frac{\delta^p_v}{L_v}) + x\eta\frac{l_v}{L_v})} \delta^{p,q,l_v}_h e^{jm_h\psi^{p,q-1}_{h}[l_v]}sinc(\frac{\xi}{2\pi}) 
% \end{align}

% \subsection{UPA Dual Beamforming}

% Suppose an $M_h \times M_v$ UPA is placed at the $x-z$ plane, and let $M = M_h M_v$. Let the distance between the antennas that are placed parallel to $z$ axis, be $d_z$, and $d_x$ respectively. One can define,

% \begin{align}
% \mathbf{d}_{M_{a}}\left(\psi_{a}\right) = \left[1, e^{j \psi_{a}} \cdots e^{j\left(M_{a}-1\right) \psi_{a}}\right]^{T} \in \mathbb{C}^{M_{a}}
% \end{align}
% where, $\zeta=\frac{2 \pi d_{x}}{\lambda} \sin \theta \cos \phi \text { and } \xi=\frac{2 \pi d_{z}}{\lambda} \sin \phi$, and $a$ takes value in the set $\{v, h\}$.  
% The array response vector is then defined as 

% \begin{align}
%     \mathbf{d}_{M}\left(\xi, \zeta\right) =
%     \mathbf{d}_{M_{v}}\left(\xi\right) \otimes
%     \mathbf{d}_{M_{h}}\left(\zeta\right)  \in \mathbb{C}^{M}
% \end{align}
% Define the reference gain at $(\zeta, \xi) $ as, 

% \begin{equation}
%     G\left(\xi, \zeta, \mathbf{c}\right) =\left|\left(\mathbf{d}_{M_{v}}\left(\xi\right)\otimes\mathbf{d}_{M_{h}}\left(\zeta\right)  \right)^{H} \mathbf{c}\right|^{2}
% \end{equation}



Prior to formulating the multi-beamforming design problem, we proceed with a few preliminary definitions. Let us define the \emph{multi-beam}  $\mathcal{D} =(\mathcal{D}_1, \ldots \mathcal{D}_k)$ as collection of $k$ \emph{compound beams} $\mathcal{D}_i, i = 1,\ldots, k$ where $\mathcal{D}_i \subseteq \mathcal{B}^{\psi}$ and $\mathcal{D}_i = {\bigcup}_{{(p,q) \in \mathcal{A}_i}} \mathcal{B}^{\psi}_{p,q}$, with $\mathcal{A}_i$ being the set of all pairs $(p,q)$ that all beams $\mathcal{B}^{\psi}_{p,q}$ cover $\mathcal{D}_i$. The union of $\mathcal{B}^{\psi}_{p,q}$ is in fact approximating the shape of the solid angle for the desired compound beam corresponding to $\mathcal{D}_i$. By using larger number of division, i.e., finer beams, one can make the approximation better. We have 
\begin{align}
    &\mathcal{A}_i = \underset{\{\mathcal{\hat{A}}|\mathcal{D}_i \subseteq \underset{(p,q) \in \mathcal{A}}{\bigcup} \mathcal{B}_{p,q}\}}{\arg\min} |\mathcal{\hat{A}}|
\end{align}
% where, $i\in \{1,2\}$ and define further $\mathcal{A} = (\mathcal{A}_1, \mathcal{A}_2)$. 

Further define $\mathcal{A} = {{\bigcup}_{i=1}^k}\mathcal{A}_i$. 
%Let $\c$ denote the configuration of RIS (a.k.a beamformer) to control the gain and the phase of the exchanged signals, i.e. $c_{m_v, m_h} = \beta_{m_v, m_h}e^{j\theta_{m_v, m_h}}$, $m_v = 1\ldots M_v$, $m_h = 1\ldots M_h$. 
%
% Let $\c = \mbox{diag}( \Theta )$ denote a vector of length $M$ consisting of the diagonal elements of the matrix $\Theta$. For antenna element located at $(m_v, m_h)$ in the ULA grid, we define $c_{m_v, m_h} = \beta_{m_v, m_h}e^{j\theta_{m_v, m_h}}$, $m_v = 0, \ldots,  M_v-1$, $m_h = 0, \ldots, M_h-1$, and hence the vector $\c$ is given by
% \begin{equation}
%     \c = [c_{0,0}, \ldots, c_{0,M_h-1}, c_{1,0}, \ldots, c_{M_v-1, M_h-1}]
% \end{equation}
%
We aim to design a beamforming vector $\c$ such that the multi-beam $\mathcal{D}$ is covered when the RIS is excited by an incident wave received at solid angle $\Omega_1$. Using (\ref{channel})-(\ref{channel_t}), the contribution of the RIS in the channel matrix for a receiver at the solid angle $\Omega_2$ is given by
% \begin{align}
%     \Gamma = \d^{H}_{M}\{\Omega_{1}\} \Theta \d_{M}\{\Omega_{2}\} = \boldsymbol\lambda^H \d_{M}\{\Omega_{2}\} 
% \end{align}
\begin{align}
    \Gamma = \a^{H}_{M}(\Omega_2) \Theta \a_{M}(\Omega_1) = \d^H_M(\Omega_2) \boldsymbol{\lambda}  
\end{align}
where  $\boldsymbol{\lambda} \in \mathbb{C}^M$  is defined as follows. For antenna element located at position $(m_v, m_h)$ in the UPA grid,  we have 
\begin{align}
    \lambda_{m_v, m_h} = \beta_{m_v, m_h}e^{-j(\theta_{m_v, m_h}- m_v\xi_{1} - m_h\zeta_{1})}
\end{align} 
where $(\xi_{1}, \zeta_{1})$ is the representation of $\Omega_1$ in the $\psi$-domain, and hence the vector $\boldsymbol{\lambda}$ is given by
\begin{equation}
    \boldsymbol{\lambda} = [\lambda_{0,0}, \ldots, \lambda_{0,M_h-1}, \lambda_{1,0}, \ldots, \lambda_{M_v-1, M_h-1}]
\end{equation}

We note that $\boldsymbol{\lambda}$ depends on the AoA of the incident beams at the RIS, i.e., $\Omega_1$, as well as the RIS parameters. The reference gain of RIS in direction $(\zeta, \xi) $ in terms of $\boldsymbol{\lambda}$ is given by 
\begin{align}
    G\left(\xi, \zeta, \boldsymbol{\lambda} \right) =\left|\left(\mathbf{d}_{M_{v}}\left(\xi\right)\otimes\mathbf{d}_{M_{h}}\left(\zeta\right)  \right)^{H} \boldsymbol{\lambda} \right|^{2}
\end{align}

% The RIS parameters, i.e., the phase shift $\theta_m$ and the attenuation value $\beta_m$ for the $m^{th}$ elements of the RIS, are obtained by by using the corresponding coefficients of the vector $\boldsymbol{\lambda}$ and the directivity vector $\a_{M}(\Omega_{1})$ for any receive incident solid angle $\Omega_1$. To design $\boldsymbol{\lambda}$, we first define and work on the normalized beamforming vector $\c = \frac{\boldsymbol{\lambda}}{\|\boldsymbol{\lambda}\|}$, and then we compute $\boldsymbol{\lambda} = \frac{\c}{\|\c\|_{\infty}}$ based on the obtained value for $\c$. 

On the other hand, the gain of UPA antenna with the feed coefficients $\c$ is given by
\begin{equation}
    G\left(\xi, \zeta, \c \right) =\left|\left(\mathbf{d}_{M_{v}}\left(\xi\right)\otimes\mathbf{d}_{M_{h}}\left(\zeta\right)  \right)^{H} \c \right|^{2} \label{UPA_beamforming_c}
\end{equation}
that has a clear similarity.
%where $\|\c\| \leq 1$.
This means that to design the RIS-UPA for the STMR problem with receive zone $\mathcal{D}$ we can use the multi-beamforming design framework to cover the ACI's included in $\mathcal{D}$ for the UPA antenna. In particular, a RIS-UPA with parameters $\boldsymbol{\lambda}$ and a UPA-antenna with beamforming parameters $\c$ have the same beamforming gain pattern if UPA structures are the same and $\boldsymbol{\lambda}=\c$. Hence, a RIS-UPA which is excited from the solid angle $\Omega_1$ has the same beamforming gain as its UPA antenna counterpart if $\boldsymbol{\Theta}= \mbox{diag} \{\c^T \odot \a_M^H(\Omega_1)\}$.
%\amir{Please note that in the equation of the reference gain, $\d$ and $\c$ are both normalized with respect to the maximum values of the gain to be one. This means that the norm of every element of the directivity vector $\d$ is one. Also, the gain of every element of the beamforming coefficient is also less than or equal to one since the RIS is assumed to be passive.}
For any normalized beamforming vector $\c$, it is straightforward to show that, 
% Let us define the \emph{dual beam}  $ \mathcal{D} =(\mathcal{D}_1, \mathcal{D}_2) , \text{ given } \mathcal{D}_1 , \mathcal{D}_2\subseteq \mathcal{B}_s$. Let $\mathcal{A}_1$, and $\mathcal{A}_2$ denote the smallest set of index pairs $(p,q)$ the corresponding beams to which collectively cover the area marked by the desired beams $\mathcal{D}_1$ and $\mathcal{D}_2$ respectively. More precisely, we can write 
% where $c^{\mathrm{D}}_1 = (\phi^{\mathrm{D}}_1, \theta^{\mathrm{D}}_1)$ and $c^{\mathrm{D}}_2 = (\phi^{\mathrm{D}}_2, \theta^{\mathrm{D}}_2)$ are the directions of the centers of the two beams respectively.
% \begin{align}
%     &\mathcal{A}_i = \underset{\{\mathcal{A}|\mathcal{D}_i \subseteq \underset{(p,q) \in \mathcal{A}}{\bigcup} \mathcal{B}_{p,q}\}}{\arg\min} Card(\mathcal{A})
% \end{align}
% where, $i\in \{1,2\}$ and define further $\mathcal{A} = (\mathcal{A}_1, \mathcal{A}_2)$. 
\begin{equation}
    \int_{-\pi}^{\pi} \int_{-\pi}^{\pi} G\left(\xi, \zeta, \mathbf{c}\right) d \xi d \zeta={(2 \pi)^{2}}
\end{equation}

We wish to design beamformers that provide high, sharp, and constant gain within the desired ACI's and zero gain everywhere else. We have then for the ideal gain corresponding to such beamformer $\c$ that,
\begin{align}
&\iint_{\mathcal{B}^{\psi}} G^\text {ideal }_{\mathcal{D}}(\xi, \zeta) d \xi d \zeta =\sum_{i=1}^k\iint_{\mathcal{D}_i} t d \xi d \zeta \nonumber\\
&= \sum_{(p,q) \in \mathcal{A} }{\iint_{\mathcal{B}^\psi_{p,q}} t d \xi d \zeta} = \sum_{(p,q) \in \mathcal{A}}\delta_{p,q} t=(2 \pi)^2 \label{composite}
\end{align}

where $\delta_{p,q}$ denotes the area of the $(p,q)$-th beam in the $(\xi, \zeta)$ domain. Therefore, we can derive $t= \frac{(2\pi)^2}{|\mathcal{A}|\delta_{p,q}}$. It holds that, 
\begin{equation}
    G^{\text {ideal }}_{ \mathcal{D}}\left(\xi, \zeta\right)=\frac{(2\pi)^2}{|\mathcal{A}|\delta_{p,q}} \mathds{1}_{\mathcal{D}}\left(\xi, \zeta\right)\label{ideal_compound}
\end{equation}

Using the beamformer $\c$ we wish to mimic the deal gain in equation \eqref{ideal_compound}. Therefore, we formulate the following optimization problem, 
\begin{align}
& \c^{opt}_{\mathcal{D}} = \underset{\c, \|\c\|=1}{\arg \min } \underset{\mathcal{B}^{\psi}}{\iint}\left|G^{\text {ideal }}_{\mathcal{D}}\left(\xi, \zeta\right)-G\left(\xi, \zeta, \c\right)\right| d \xi d \zeta \label{init_opt}
\end{align}

By partitioning the range of $(\xi, \zeta)$ into the pre-defined intervals, and then uniformly sampling with the rate $(L_v, L_h)$ per interval along both axis,  we can rewrite the optimization problem as follows, 
% \begin{align}
% & \c^{opt}_{\mathcal{D}} = \underset{\c, \|\c\|=1}{\arg \min } \sum_{r=1}^{Q_v}\sum_{s=1}^{Q_h}\iint_{\mathcal{B}^{\psi}_{r,s}}\left|G_{\mathcal{D}}^{\text {ideal }}\left(\xi, \zeta\right)-G\left(\xi, \zeta, \c\right)\right| d \xi d \zeta \nonumber\\&
%  = \underset{L_h, L_v \rightarrow \infty}{\lim}\sum_{r=1}^{Q_v}\sum_{s=1}^{Q_h}\sum_{l_v =1}^{L_v}\sum_{l_h =1}^{L_h}\frac{\delta_v\delta_h}{L_hL_v}\nonumber\\&
% \left|G_{\mathcal{D}}^{\text {ideal }}\left(\psi^{r,s}_{v}[l_v], \psi^{r,s}_{h}[l_h]\right)-G\left(\psi^{r,s}_{v}[l_v], \psi^{r,s}_{h}[l_h], \c\right)\right| \label{alternative_opt}
% \end{align}
% where, 
% \begin{align}
%     & \psi^{r,s}_v = -\psi^{\mathrm{B}}_v + r\delta_v, \quad \psi^{r,s}_h = -\psi^{\mathrm{B}}_h + s\delta_h \\
%     &\psi^{r,s}_v[l_v] = \psi^{r-1, s}_v + l_v\frac{\delta_v}{L_v},  \quad \psi^{r,s}_h[l_h] = \psi^{r, s-1}_h + l_h\frac{\delta_h}{L_h} \label{l_vl_h}
% \end{align}
\begin{align}
 \c^{opt}_{\mathcal{D}} &= \underset{{\c, \|\c\|=1}}{{\arg \min }} \sum_{r=1}^{Q_v}\sum_{s=1}^{Q_h}\iint_{\mathcal{B}^{\psi}_{r,s}}\left|G_{\mathcal{D}}^{\text {ideal }}\left(\xi, \zeta \right)-G\left(\xi, \zeta, \c\right)\right| d \xi d \zeta \nonumber\\&
 = \underset{L_h, L_v \rightarrow \infty}{\lim}\sum_{r=1}^{Q_v}\sum_{s=1}^{Q_h}\sum_{l_v =1}^{L_v}\sum_{l_h =1}^{L_h}\nonumber\\&\frac{\delta_v\delta_h}{L_hL_v}
%\left|G_{\mathcal{D}}^{\text {ideal }}\left(\psi^{r,s}_{v}[l_v], \psi^{r,s}_{h}[l_h]\right)-G\left(\psi^{r,s}_{v}[l_v], \psi^{r,s}_{h}[l_h], \c\right)\right|
\left|G_{\mathcal{D}}^{\text {ideal }}\left(\xi_{r, l_v}, \zeta_{s,l_h}\right)-G\left(\xi_{r, l_v}, \zeta_{s,l_h}, \c\right)\right|\label{alternative_opt}
\end{align}
where, 
\begin{align}
    &\xi_{r,l_v} = \xi^{r-1} + l_v\frac{\delta_v}{L_v}, \quad \zeta_{s, l_h} = \zeta^{s-1} + l_h\frac{\delta_h}{L_h} \label{l_v_l_h}
\end{align}




\noindent with $\delta_a = \frac{2\psi_a^{\mathrm{B}}}{Q_a}$, for $a \in \{v, h\}$. Note that it holds for all $(p,q)$ pairs that, $\delta_{p,q} = \delta_v\delta_h$. 
We can rewrite equation \eqref{alternative_opt} as, 
\begin{align}
    \c_{\mathcal{D}}^{opt}=\arg \min _{\c, \|\c\|=1} \underset{L_h, L_v \rightarrow \infty}{\lim}\frac{1}{L_hL_v}\left|\mathbf{G}^{\text {ideal }}_{\mathcal{D}}-\mathbf{G}(\c)\right|\label{init_normed_opt}
\end{align}
where, 
\begin{align}
    \mathbf{G}(\c) =&  \delta_{p,q}\left[G\left(\xi_{1,1}, \zeta_{1,1}, \c\right) \cdots G\left(\xi_{Q_{v}, L_{v}}, \zeta_{Q_{h}, L_{h}}, \c\right)\right]^{T}  
\end{align}
and,
\begin{align}
    \mathbf{G}^{\text {ideal }}_{ \mathcal{D}} =& \delta_{p,q}\left[G^{\text {ideal }}_{ \mathcal{D}}\left(\xi_{1,1}, \zeta_{1,1} \right) \cdots  G^{\text {ideal }}_{ \mathcal{D}}\left(\xi_{Q_{v}, L_v}, \zeta_{ Q_{h}, L_{h}}\right)\right]^{T} 
\end{align}

% \begin{align}
%     \mathbf{G}(\c) =&  \delta_v\delta_h\left[G\left(\xi^{1}[1], \zeta^{1}[1], \c\right) \cdots \right. \nonumber\\ &\left. G\left(\xi^{Q_{v}}\left[L_{v}\right], \zeta^{Q_{h}}\left[L_{h}\right], \c\right)\right]^{T} 
% \end{align}

% and, 
% \begin{align}
%     \mathbf{G}^{\text {ideal }}_{ \mathcal{D}} =& \delta_v\delta_h\left[G^{\text {ideal }}_{ \mathcal{D}}\left(\xi^{1}[1], \zeta^{1}[1] \right) \cdots \right. \nonumber\\ &\left. G^{\text {ideal }}_{ \mathcal{D}}\left(\xi^{Q_{v}}\left[L_{v}\right], \zeta^{ Q_{h}}\left[L_{h}\right]\right)\right]^{T} 
% \end{align}


Unfortunately, the optimization problem in \eqref{init_normed_opt} does not admit an optimal closed-form solution as is, due to the absolute values of the complex numbers existing in the formulation. However, note that, 
\begin{align}
    \mathbf{G}^{\text {ideal }}_{\mathcal{D}}&=\sum_{(p,q) \in \mathcal{A}}\delta_{p,q}\frac{(2\pi)^2}{|\mathcal{A}|\delta_{p,q}}\left(\mathbf{e}_{p,q} \otimes \mathbf{1}_{L, 1}\right) \nonumber\\&= \frac{(2\pi)^2}{|\mathcal{A}|}\sum_{(p,q) \in \mathcal{A}}{\mathbf{e}_{p,q} \otimes \mathbf{1}_{L, 1}}
    \label{ideal}
\end{align}
with $\mathbf{e}_{p,q} \in \mathbb{Z}^{Q}$ being the standard basis vector for the $(p,q)$-th axis among $(Q_v, Q_h)$ pairs. Now, note that $\mathbf{1}_{L, 1}=\mathbf{g} \odot \mathbf{g}^{*}$ for any equal gain $\mathbf{g} \in \mathbb{C}^L$ where $L = L_hL_v$. An equal-gain  vector $\g \in \mathbb{C}^L$ is a vector where all elements have equal absolute values (in this case, equal to $1$). Therefore, we can write: 
% $$\mathcal{G}_{L}=\left\{\mathbf{g} \in \mathbb{C}^{L}:\left(\mathbf{g} \mathbf{g}^{H}\right)_{\ell, \ell}=1,1 \leq \ell \leq L\right\}$$
\begin{align}
\mathbf{G}^{\text {ideal }}_{\mathcal{D}} &= \sum_{(p,q) \in \mathcal{A}}\frac{(2\pi)^2}{|\mathcal{A}|}\left(\mathbf{e}_{p,q} \otimes\left(\mathbf{g} \odot \mathbf{g}^{*}\right)\right) \nonumber\\
&=\frac{(2\pi)^2}{|\mathcal{A}|}\sum_{(p,q) \in \mathcal{A}}\left(\mathbf{e}_{p,q} \otimes \mathbf{g}\right) \odot\left(\mathbf{e}_{p,q} \otimes \mathbf{g}\right)^{*} \nonumber\\
&=\left(\sum_{(p,q) \in \mathcal{A}}\frac{2\pi}{\sqrt{|\mathcal{A}|}}\left(\mathbf{e}_{p,q} \otimes \mathbf{g}\right)\right)  \nonumber\\&\odot \left(\sum_{(p,q) \in \mathcal{A}}\frac{2\pi}{\sqrt{|\mathcal{A}|}}\left(\mathbf{e}_{p,q} \otimes \mathbf{g}\right)\right)^* \label{final_gik}
\end{align}

Also, it is straightforward to write, 
% \begin{align}
%     \mathbf{G}(\c)=\left(\left(\mathbf{D}_{h}^{H} \otimes \mathbf{D}_{v}^{H}\right) \c\right) \odot\left(\left(\mathbf{D}_{h}^{H} \otimes \mathbf{D}_{v}^{H}\right) \c\right)^{*}
% \end{align}
\begin{align}
    \mathbf{G}(\c)=\left(\D^H \c\right) \odot\left(\D^H \c\right)^{*}\label{dc}
\end{align}

\noindent where, $\D^H = \sqrt{\delta_v\delta_h}(\mathbf{D}_{v}^{H} \otimes \mathbf{D}_{h}^{H})$, and for $a \in \{v,h\}$, and $b= 1\ldots Q_a$ we have, 
\begin{align}
\mathbf{D}_{a} &=\left[\mathbf{D}_{a, 1}, \cdots, \mathbf{D}_{a, Q_{a}}\right] \in \mathbb{C}^{M_{a} \times L_{a} Q_{a}}
% \mathbf{D}_{a, b} &=\left[\mathbf{d}_{M_{a}}\left(\psi_{a}^{b}[1]\right), \cdots, \mathbf{d}_{M_{a}}\left(\psi_{a}^{b}\left[L_{a}\right]\right)\right] \in \mathbb{C}^{M_{a} \times L_{a}}
\end{align}
where, 
\begin{align}
    &\mathbf{D}_{v, b} =\left[\mathbf{d}_{M_{v}}\left(\xi_{b,1}\right), \cdots, \mathbf{d}_{M_{v}}\left(\xi_{b, L_v}\right)\right] \in \mathbb{C}^{M_{v} \times L_{v}} \\
    &\mathbf{D}_{h, b} =\left[\mathbf{d}_{M_{h}}\left(\zeta_{b,1}\right), \cdots, \mathbf{d}_{M_{h}}\left(\zeta_{b, L_h}\right)\right] \in \mathbb{C}^{M_{h} \times L_{h}}
\end{align}

Comparing the expressions \eqref{init_normed_opt}, \eqref{final_gik}, and \eqref{dc}, one can show that the optimal choice of $\c_\mathcal{D}$ in \eqref{init_opt} is the solution to the following optimization problem for proper choices of $\g_{p,q}$. 

\begin{problem}
Given equal-gain vectors $\g_{p,q} \in \mathbb{C}^L$, for $(p,q) \in \mathcal{A}$  find vector $\c_{\mathcal{D}} \in \mathbb{C}^{M}$ such that
\begin{align}
\c_{\mathcal{D}}=&{\arg \min }_{\c, \|\c\|=1}\nonumber\\& \lim_{L\rightarrow \infty} \left\|\sum_{(p,q) \in \mathcal{A}}\frac{2\pi}{\sqrt{|\mathcal{A}|}}\left(\mathbf{e}_{p,q} \otimes \mathbf{g}_{p,q}\right)- \D^H \c\right\|^{2} \label{obj_func}
\end{align}
\label{main_problem_UPA}
\end{problem}

However, we now need to find the optimal choices of $\g_{p,q}$ that minimize the objective in \eqref{init_normed_opt}. Using \eqref{final_gik}, and \eqref{dc}, we have the following optimization problem.

\begin{problem}
Find equal-gain vectors $\g^*_{p,q} \in \mathbb{C}^L$, $(p,q) \in \mathcal{A}$ such that
\begin{align}
 &<\g^*_{p,q}>_{(p,q)\in\mathcal{A}} = \underset{<\g_{p,q}>_{(p,q)\in\mathcal{A}}}{\arg\min }\nonumber\\
 &\left\| abs(\D^H \c_{\mathcal{D}})- \frac{{2\pi}}{\sqrt{|\mathcal{A}|}} abs(\sum_{(p,q)\in \mathcal{A}}\mathbf{e}_{p,q} \otimes \mathbf{g}_{p,q})\right\|^{2} \label{g_final_eq}
\end{align} 
where $abs(.)$ denotes the element-wise absolute value of a vector.
\label{g_problem}
\end{problem}

Next, we continue with the solution of Problems ~\ref{main_problem_UPA}, and~\ref{g_problem}.


\section{Flow-Packet Hybrid Traffic Classification}
\label{sec:proposed}

We propose FPHTC for a router that needs to conduct class-aware traffic processing. In this section, we provide a detailed description of our scheme. A diagram illustrating the overall framework of FPHTC is given in Fig.~\ref{fig:scheme}.


\subsection{Core Components of FPHTC}
\subsubsection{Router}
The router accepts an incoming stream of packets and processes them according to their service classes using the routing policy. The basic structure and function of such a routing policy are well-defined in prior works on packet classification \cite{Gupta99, Gupta01}. Throughout our work, we focus on how to generate routing
policy rules by training a machine learning model for packet-based traffic classification, where the chosen header fields of each packet are its features, i.e., the inputs into the learning model, and the packet is classified by the learning model to determine its CoS. For example, the chosen header fields may be the source IP address, destination IP address, source port number, and destination port number, among others, and the possible actions may be to route a packet as delay sensitive, delay moderate, or delay tolerant.

\subsubsection{Flow-based Traffic Classifier}
The flow-based traffic classifier resides outside the router, in some powerful equipment that can handle the heavy computation required by sophisticated machine learning techniques. It is a complex and highly accurate machine learning model that can classify a traffic flow in terms of CoS for all of its packets. It is trained using a number of bidirectional TCP flows with a set of flow-level statistical features extracted from the raw dataset.

Various methods are possible to generate the training dataset for the flow-based traffic classifier. In this work, since we are ultimately interested in online classification to handle changing traffic pattern over time, we propose to use a continuously updated recording of the past traffic. Specifically, we use a traffic mirror and a traffic selector, as shown in Fig.~\ref{fig:scheme}, to separate a selected small portion of the incoming traffic flows. The selected flows are then labeled using a Deep Packet Inspection (DPI) module according to their CoS. The true CoS labels obtained by DPI are used to train the flow-based classifier. We note that DPI cannot be used to replace the role of the flow-based classifier for all flows, due to its prohibitive cost and delay for common encrypted traffic. 

The role of the flow-based traffic classifier designer includes data preprocessing, hyperparameter selection, and finally, training the flow-based classifier. Once the flow-based classifier is trained, we use it to infer the CoS labels of all incoming flows captured by the traffic mirror. Then all packets belonging to a flow can be tagged by CoS label of the flow. We note that the CoS labels generated in this way, by a flow-based classifier, are too late to be used in the \textit{routing} of the labeled packets. However, what this achieves is to create a packet-level dataset for \textit{training} the packet-based routing policy as explained below.

\subsubsection{Packet-based Routing Policy Designer}

The packet-based routing policy designer takes labeled packets from the flow-based classifier as input, and it outputs a routing policy for the router. Specifically,  the routing policy designer trains a packet-based classifier using the labeled packets as the training dataset. 

In this work, we use the binary decision tree learning model for the packet-based classifier. In the decision tree, each path from the root to a node is a routing policy rule. Thus, to obtain routing policy rules that can be used in the router, the routing policy designer only needs to train a decision tree on the packet-level dataset. Furthermore, we note that the number of routing policy rules equals the number of leaf nodes in the decision tree. This provides an easy way to control the size of the routing policy, i.e., the routing policy designer can limit the maximum number of leaf nodes while training the decision tree.

\subsection{Construction of Routing Policy}

The construction of the routing policy in FPHTC involves transferring learned knowledge from the flow-based classifier to the routing policy designer. In the machine learning literature, knowledge distillation \cite{Hinton15, Vapnik16} is a technique where a simple student model is trained on the predictions supplied by a highly accurate and complex teacher model. In FPHTC, we train a decision tree at the routing policy designer using the predictions from the flow-based classifier as training targets. In essence, the routing policy designer tries to approximate the performance of the flow-based classifier. 

The flow-based classifier is trained with flow-level statistical features whereas the routing policy designer uses only some features that can be read directly from the packet header. Therefore, it is clear that the learned routing policy will perform worse than the flow-based classifier given the same traffic data for training. However, since there are unlabeled training data available, i.e., those that have not been labeled by DPI, we can label those data samples using our flow-based classifier to substantially enlarge the training dataset for the routing policy designer. Since the decision tree at the routing policy designer is trained on a much larger dataset than that of the flow-based classifier, the performance of the routing policy can be close to that of the flow-based classifier. More importantly, since the routing policy created by FPHTC utilizes information learned from a more powerful flow-based classifier, it can substantially outperform a regular packet-based classifier trained using only the small amount of labels generated by DPI.


\subsection{Routing Policy Update Procedure in Online Setting}

In a practical system, the data pattern of the incoming traffic changes over time, e.g., due to new applications appearing in the network, or changing user behavior. Therefore, we design FPHTC to dynamically update the routing policy over time.

In Fig.~\ref{fig:online}, we illustrate how the modules sequentially function over a continuous stream of traffic. At any given time slot, we collect and label a small portion of the incoming traffic flows using DPI to train the flow-based classifier. Meanwhile, we continue to collect flows to be used in the training of the routing policy. Once the flow-based classifier is trained, we use it to label those collected flows not labeled by DPI. Then, the routing policy designer trains a decision tree to generate the routing policy, which is then updated to the router. 

One important question is whether we should repeat these steps and update the routing policy at each time slot. If the traffic data pattern does not change too frequently, routing policy update at every time slot would be a waste of resources. To re-train the flow-based classifier, the labeling cost using DPI would also be expensive. A cost-effective solution is to update the routing policy only when the traffic pattern has altered significantly. This can be inferred by measuring the performance deterioration at the router. A feedback signal can be generated, for example, based on the increase in packet drop or congestion, to indicate that a routing policy update is necessary. We demonstrate the adaptiveness of FPHTC in the online setting in Section \ref{sec:results}.

\begin{figure}[t]
	\centering
	\includegraphics[width=9cm]{"figures/online".pdf}
	\caption{FPHTC in online setting.}
	\label{fig:online}
\end{figure}
% \begin{algorithm}[!ht]
\begin{algorithmic}[1]
\Require Query workload $Q$, event stream $I$, \app\ graph $G$, hash table of snapshots $S$
\Ensure Hash table of results $R$ 
\State $G \leftarrow \emptyset$, $S, R \leftarrow$ empty hash tables
\ForAll {event $e \in I$ with $e.type=E$} 
    \State $//$ \textbf{\app\ graph construction}
    \ForAll {$q \in Q$ \text{ with event types }T}
        \ForAll {$E' \in T,\ E' \neq E$}
            \State $G_{E'} \leftarrow \mathit{getGraphlet}(G,E')$,
            $G_{E'}.\mathit{active} \leftarrow \mathit{false}$
        \EndFor
    \EndFor
    \If {\textbf{not} $G_E.\mathit{active}$}
        \State $G_E \leftarrow \mathit{createGraphlet()}$, $G_{E}.\mathit{active} \leftarrow \mathit{true}$,
        $G \leftarrow G \cup G_E$
        \If {$G_E.\mathit{shared}$ by $Q_E \subseteq Q$}
            $x \leftarrow \mathit{createSnapshot()}$ 
            \ForAll {$q \in Q_E$}
                \ForAll{$E' \in \mathit{pt}(E,q), E' \neq E$}
                    \State $G_{E'} \leftarrow \mathit{getGraphlet}(G,E')$
                    \State $S(x,q) \leftarrow S(x,q) + sum(G_{E'},q)$ \hspace{0.5cm}$//$ Eq.~5
                \EndFor
            \EndFor
        \EndIf    
    \EndIf
    \State insert $e$ into $G_E$
    \State $//$ \textbf{Trend count computation}
    \If {$G_E.\mathit{shared}$ by $Q_E \subseteq Q$}
        \If {$\forall q \in Q_E\ pe(e,q)$ are identical}
            \State $count(e,Q_E) \leftarrow count(e,q)$ \hspace{2.3cm}$//$ Eq.~2
        \Else\ $y \leftarrow \mathit{createSnapshot()}$, $count(e,Q_E) = y$
            \ForAll {$q \in Q_E$}
                $S(y,q) \leftarrow count(e,q)$ \hspace{0.2cm}$//$ Eq.~2
            \EndFor
          \EndIf
    \Else\ $count(e,q)$ \hspace{5.2cm}$//$ Eq.~2
    \EndIf
    \ForAll{$q \in Q$}
  	    \If {$E \in \mathit{end}(q)$} 
  		    $R(q) \leftarrow R(q) + count(e,q)$ $//$ Eq.~3
        \EndIf
    \EndFor
\EndFor
\State \Return $R$
\end{algorithmic}
\caption{\app\ shared online trend aggregation}
\label{algo:snapshot-propagation}
\end{algorithm}

%!TEX root = main.tex
\section{Evaluation}
\label{sec:eval}

In this section, we evaluate the performance of our unsupervised Ordered Word Mover's Distance metric and supervised Multi-scale Sentence Matching model with factorized sentences as input. We apply our algorithms to semantic textual similarity estimation tasks and sentence pair paraphrase identification tasks, based on four datasets: STSbenchmark, SICK, MSRP and MSRvid. 

\subsection{Experimental Setup}
\label{subsec:setup}


\begin{table}[tb]
  \caption{Description of evaluation datasets.}
  \label{tab:datasets}
  \begin{tabular}{lllll}
    \toprule
    Dataset & Task & Train & Dev & Test\\
    \midrule
    STSbenchmark & Similarity scoring & $5748$ & $1500$ & $1378$ \\
    SICK & Similarity scoring & $4500$ & $500$ & $4927$ \\
    MSRP & Paraphrase identification & $4076$ & - & $1725$ \\
    MSRvid & Similarity scoring & $750$ & - & $750$ \\
    \bottomrule
  \end{tabular}
  \vspace{-2mm}
\end{table}

We will start with a brief description for each dataset:
\begin{itemize}
\item \textbf{STSbenchmark}\cite{cer2017semeval}: it is a dataset for semantic textual similarity (STS) estimation. The task is to assign a similarity score to each sentence pair on a scale of 0.0 to 5.0, with 5.0 being the most similar.

\item \textbf{SICK}\cite{marelli2014sick}: it is another STS dataset from the SemEval 2014 task 1. It has the same scoring mechanism as STSbenchmark, where 0.0 represents the least amount of relatedness and 5.0 represents the most.

\item \textbf{MSRvid}: the Microsoft Research Video Description Corpus contains 1500 sentences that are concise summaries on the content of a short video. Each pair of sentences is also assigned a semantic similarity score between 0.0 and 5.0. 

\item \textbf{MSRP}\cite{quirk2004monolingual}: the Microsoft Research Paraphrase Corpus is a set of 5800 sentence pairs collected from news articles on the Internet. Each sentence pair is labeled 0 or 1, with 1 indicating that the two sentences are paraphrases of each other.
\end{itemize}

Table \ref{tab:datasets} shows a detailed breakdown of the datasets used in evaluation.
For STSbenchmark dataset we use the provided train/dev/test split.
The SICK dataset does not provide development set out of the box, so we extracted 500 instances from the training set as the development set.
For MSRP and MSRvid, since their sizes are relatively small to begin with, we did not create any development set for them.

One metric we used to evaluate the performance of our proposed models on the task of semantic textual similarity estimation is the Pearson Correlation coefficient, commonly denoted by $r$. Pearson Correlation is defined as:
\begin{equation}
\label{eq:pearson}
 r = cov(X,Y) /( \sigma_X \sigma_Y),
\end{equation}
where $cov(X,Y)$ is the co-variance between distributions X and Y, and $\sigma_X$, $\sigma_Y$ are the standard deviations of X and Y.
The Pearson Correlation coefficient can be thought as a measure of how well two distributions fit on a straight line. Its value has range [-1, 1], where a value of 1 indicates that data points from two distribution lie on the same line with a positive slope.
% Due to this unique property, we believe the Pearson Correlation coefficient is a strong indicator of the performance of our metric. 

Another metric we utilized is the Spearman's Rank Correlation coefficient. Commonly denoted by $r_s$, the Spearman's Rank Correlation coefficient shares a similar mathematical expression with the Pearson Correlation coefficient, but it is applied to ranked variables.
Formally it is defined as \cite{wiki:spearman}:
\begin{equation}
\label{eq:spearman}
 \rho = cov(rg_X, rg_Y) / (\sigma_{rg_X} \sigma_{rg_Y}),
\end{equation}
where $rg_X$, $rg_Y$ denotes the ranked variables derived from $X$ and $Y$. $cov(rg_X,rg_Y)$, $\sigma_{rg_X}$, $\sigma_{rg_Y}$ corresponds to the co-variance and standard deviations of the rank variables. The term ranked simply means that each instance in X is ranked higher or lower against every other instances in X and the same for Y. We then compare the rank values of X and Y with \ref{eq:spearman}. Like the Pearson Correlation coefficient, the Spearman's Rank Correlation coefficient has an output range of [-1, 1], and it measures the monotonic relationship between X and Y. A Spearman's Rank Correlation value of 1 implies that as X increases, Y is guaranteed to increase as well.
The Spearman's Rank Correlation is also less sensitive to noise created by outliers compared to the Pearson Correlation.

For the task of paraphrase identification, the classification accuracy of label $1$ and the F1 score are used as metrics. 

In the supervised learning portion, we conduct the experiments on the aforementioned four datasets. We use training sets to train the models, development set to tune the hyper-parameters and each test set is only used once in the final evaluation. For datasets without any development set, we will use cross-validation in the training process to prevent overfitting, that is, use $10\%$ of the training data for validation and the rest is used in training. For each model, we carry out training for 10 epochs. We then choose the model with the best validation performance to be evaluated on the test set.  


\subsection{Unsupervised Matching with OWMD}
\label{subsec:eval-owmd}

To evaluate the effectiveness of our Ordered Word Mover's Distance metric, we first take an unsupervised approach towards the similarity estimation task on the STSbenchmark, SICK and MSRvid datasets. Using the distance metrics listed in Table \ref{tab:compare-pearson} and \ref{tab:compare-spearman}, we first computed the distance between two sentences, then calculated the Pearson Correlation coefficients and the Spearman's Rank Correlation coefficients between all pair's distances and their labeled scores. We did not use the MSRP dataset since it is a binary classification problem.


In our proposed Ordered Word Mover's Distance metric, distance between two sentences is calculated using the order preserving Word Mover's Distance algorithm. For all three datasets, we performed hyper-parameter tuning using the training set and calculated the Pearson Correlation coefficients on the test and development set. We found that for the STSbenchmark dataset, setting $\lambda_1=10$, $\lambda_2=0.03$ produces the most optimal result. For the SICK dataset, a combination of $\lambda_1=3.5$, $\lambda_2=0.015$ works best. And for the MSRvid dataset, the highest Pearson Correlation is attained when $\lambda_1=0.01$, $\lambda_2=0.02$.
We maintain a max iteration of 20 since in our experiments we found that it is sufficient for the correlation result to converge.
During hyper-parameter tuning we discovered that using the Euclidean metric along with $\sigma=10$ produces better results, so all OWMD results summarized in Table \ref{tab:compare-pearson} and \ref{tab:compare-spearman} are acquired under these parameter settings. Finally, it is worth mentioning that our OWMD metric calculates the distances using factorized versions of sentences, while all other metrics use the original sentences. Sentence factorization is a necessary preprocessing step for the OWMD metric.


We compared the performance of Ordered Word Mover's Distance metric with the following methods:

\begin{itemize}
\item \textbf{Bag-of-Words (BoW)}: in the Bag-of-Words metric, distance between two sentences is computed as the cosine similarity between the word counts of the sentences.

\item \textbf{LexVec}~\cite{salle2016enhancing}: calculate the cosine similarity between the  averaged 300-dimensional LexVec word embedding of the two sentences. 

\item \textbf{GloVe}~\cite{pennington2014glove}: calculate the cosine similarity between the averaged 300-dimensional GloVe 6B word embedding of the two sentences. 

\item \textbf{Fastext}~\cite{joulin2016bag}: calculate the cosine similarity between the averaged 300-dimensional Fastext word embedding of the two sentences. 

\item \textbf{Word2vec}~\cite{mikolov2013efficient}: calculate the cosine similarity between the averaged 300-dimensional Word2vec word embedding of the two sentences.

\item \textbf{Word Mover's Distance (WMD)}~\cite{kusner2015word}: estimating the semantic distance between two sentences by WMD introduced in Sec.~\ref{sec:owmd}.
\end{itemize} 


\begin{table}[tb]
  \caption{Pearson Correlation results on different distance metrics.}
  \label{tab:compare-pearson}
  \begin{tabular}{c|cc|cc|c}
    \toprule
    \multirow{2}{*}{Algorithm} & \multicolumn{2}{c}{STSbenchmark} & \multicolumn{2}{c}{SICK} & MSRvid\\ 
     & Test & Dev & Test & Dev & Test\\
    \midrule
    BoW & $0.5705$ & $0.6561$ & $0.6114$ & $0.6087$ & $0.5044$ \\
    LexVec & $0.5759$ & $0.6852$ & $0.6948$ & $\mathbf{0.6811}$ & $0.7318$\\
    GloVe & $0.4064$ & $0.5207$ & $0.6297$ & $0.5892$  & $0.5481$ \\
    Fastext & $0.5079$ & $0.6247$ & $0.6517$ & $0.6421$  & $0.5517$  \\
    Word2vec & $0.5550$ & $0.6911$ & $\mathbf{0.7021}$ & $0.6730$  & $0.7209$  \\
    WMD & $0.4241$ & $0.5679$ & $0.5962$ & $0.5953$  & $0.3430$  \\
    OWMD & $\mathbf{0.6144}$ & $\mathbf{0.7240}$ & $0.6797$ & $0.6772$  & $\mathbf{0.7519}$  \\
    \bottomrule
  \end{tabular}
  \vspace{-4mm}
\end{table}

\begin{table}[tb]
  \caption{Spearman's Rank Correlation results on different distance metrics.}
  \label{tab:compare-spearman}
  \begin{tabular}{c|cc|cc|c}
    \toprule
    \multirow{2}{*}{Algorithm} & \multicolumn{2}{c}{STSbenchmark} & \multicolumn{2}{c}{SICK} & MSRvid\\ 
     & Test & Dev & Test & Dev & Test\\
    \midrule
    BoW & $0.5592$ & $0.6572$ & $0.5727$ & $0.5894$ & $0.5233$ \\
    LexVec & $0.5472$ & $0.7032$ & $0.5872$ & $0.5879$ & $0.7311$\\
    GloVe & $0.4268$ & $0.5862$ & $0.5505$ & $0.5490$  & $0.5828$ \\
    Fastext & $0.4874$ & $0.6424$ & $0.5739$ & $0.5941$  & $0.5634$  \\
    Word2vec & $0.5184$ & $0.7021$ & $0.6082$ & $0.6056$  & $0.7175$  \\
    WMD & $0.4270$ & $0.5781$ & $0.5488$ & $0.5612$  & $0.3699$  \\
    OWMD & $\mathbf{0.5855}$ & $\mathbf{0.7253}$ & $\mathbf{0.6133}$ & $\mathbf{0.6188}$  & $\mathbf{0.7543}$  \\
    \bottomrule
  \end{tabular}
  \vspace{-2mm}
\end{table}


Table \ref{tab:compare-pearson} and Table \ref{tab:compare-spearman} compare the performance of different metrics in terms of the Pearson Correlation coefficients and the Spearman's Rank Correlation coefficients.
We can see that the result of our OWMD metric achieves the best performance on all the datasets in terms of the Spearman's Rank Correlation coefficients.
It also produced the best Pearson Correlation results on the STSbenchmark and the MSRvid dataset, while the performance on SICK datasets are close to the best.
This can be attributed to the two characteristics of OWMD. First, the input sentence is re-organized into a predicate-argument structure using the sentence factorization tree. Therefore, corresponding semantic units in the two sentences will be aligned roughly in order. Second, our OWMD metric takes word positions into consideration and penalizes disordered matches. Therefore, it will produce less mismatches compared with the WMD metric.

% On the SICK dataset, although the result of our metric falls slightly behind Word2vec, LexVec on the test set and Word2vec on the development set, we still believe that it is a superior metric because it produced competitive results across multiple datasets. 

% Table \ref{tab:compare-spearman} presents the Spearman's Rank Correlation coefficients acquired with the same distance metrics. We can observe that our OWMD metric achieves the highest correlation scores on all three datasets. Which proves once again that OWMD is a better distance metric for the task of semantic similarity detection.

\subsection{Supervised Multi-scale Semantic Matching}
\label{subsec:eval-multilayer}

\begin{table*}[tb]
  \caption{A comparison among different supervised learning models in terms of accuracy, F1 score, Pearson's $r$ and Spearman's $\rho$ on various test sets.}
  \label{tab:sts}
  \begin{tabular}{c|cc|cc|cc|cc}
    \toprule
    \multirow{2}{*}{Model} & \multicolumn{2}{c}{MSRP} & \multicolumn{2}{c}{SICK} & \multicolumn{2}{c}{MSRvid} & \multicolumn{2}{c}{STSbenchmark}\\ 
     & Acc.(\%) & F1(\%) & $r$ & $\rho$ & $r$ & $\rho$ & $r$ & $\rho$ \\
    \midrule
    MaLSTM & $66.95$ & $73.95$ & $0.7824$ & $0.71843$ & $0.7325$ & $0.7193$ & $0.5739$ & $0.5558$\\
    Multi-scale MaLSTM & $\mathbf{74.09}$ & $\mathbf{82.18}$ & $\mathbf{0.8168}$ & $\mathbf{0.74226}$ & $\mathbf{0.8236}$ & $\mathbf{0.8188}$ & $\mathbf{0.6839}$ & $\mathbf{0.6575}$\\
    \midrule
    HCTI & $73.80$ & $80.85$ & $0.8408$ & $0.7698$ & $\mathbf{0.8848}$ & $\mathbf{0.8763}$  & $\mathbf{0.7697}$ & $\mathbf{0.7549}$ \\
    Multi-scale HCTI & $\mathbf{74.03}$ & $\mathbf{81.76}$ & $\mathbf{0.8437}$ & $\mathbf{0.7729}$ & $0.8763$ & $0.8686$  & $0.7269$ & $0.7033$  \\
    \bottomrule
  \end{tabular}
  \vspace{-2mm}
\end{table*}

The use of sentence factorization can improve both existing unsupervised metrics and existing supervised models. 
% We extend the normal Siamese model to Fig. \ref{fig:network} to take advantage of different level of information in the factorized sentence. 
To evaluate how the performance of existing Siamese neural networks can be improved by our sentence factorization technique and the multi-scale Siamese architecture, we implemented two types of Siamese sentence matching models, HCTI \cite{mueller2016siamese} and MaLSTM \cite{shao2017hcti}. HCTI is a Convolutional Neural Network (CNN) based Siamese model, which achieves the best Pearson Correlation coefficient on STSbenchmark dataset in SemEval2017 competition (compared with all the other neural network models). MaLSTM is a Siamese adaptation of the Long Short-Term Memory (LSTM) network for learning sentence similarity. As the source code of HCTI is not released in public, we implemented it according to \cite{shao2017hcti} by Keras \cite{chollet2015keras}. With the same parameter settings listed in paper \cite{shao2017hcti} and tried our best to optimize the model, we got a Pearson correlation of 0.7697 (0.7833 in paper \cite{shao2017hcti}) in STSbencmark test dataset.

We extended HCTI and MaLSTM to our proposed Siamese architecture in Fig. \ref{fig:network}, namely the Multi-scale MaLSTM and the Multi-scale HCTI. To evaluate the performance of our models, the experiment is conducted on two tasks: 1) semantic textual similarity estimation based on the STSbenchmark, MSRvid, and SICK2014 datasets; 2) paraphrase identification based on the MSRP dataset.

Table \ref{tab:sts} shows the results of HCTI, MaLSTM and our multi-scale models on different datasets. Compared with the original models, our models with multi-scale semantic units of the input sentences as network inputs significantly improved the performance on most datasets. 
Furthermore, the improvements on different tasks and datasets also proved the general applicability of our proposed architecture.

Compared with MaLSTM, our multi-scaled Siamese models with factorized sentences as input perform much better on each dataset. For MSRvid and STSbenmark dataset, both Pearson's $r$ and Spearman's $\rho$ increase about $10\%$ with Multi-scale MaLSTM. Moreover, the Multi-scale MaLSTM achieves the highest accuracy and F1 score on the MSRP dataset compared with other models listed in Table \ref{tab:sts}.

There are two reasons why our Multi-scale MaLSTM significantly outperforms MaLSTM model. First, for an input sentence pair, 
we explicitly model their semantic units with the factorization algorithm.
%we explicitly model the different scales of semantics of them with the semantic units produced by our sentence factorization algorithm. 
Second, our multi-scaled network architecture is 
specifically designed
%specially adapted to 
for multi-scaled sentences representations. Therefore, it is able to explicitly match a pair of sentences at different granularities.

We also report the results of HCTI and Multi-scale HCTI in Table \ref{tab:sts}. For the paraphrase identification task, our model shows better accuracy and F1 score on MSRP dataset. For the semantic textual similarity estimation task, the performance varies across datasets. On the SICK dataset, the performance of Multi-scale HCTI is close to HCTI with slightly better Pearson' $r$ and Spearman's $\rho$. However, the Multi-scale HCTI is not able to outperform HCTI on MSRvid and STSbenchmark. HCTI is still the best neural network model on the STSbenchmark dataset, and the MSRvid dataset is a subset of STSbenchmark.
Although HCTI has strong performance on these two datasets, it performs worse than our model on other datasets.
% Overall, the experimental results demonstrated the superior applicability and generalizability of our proposed models.
Overall, the experimental results demonstrated the general applicability of our proposed model architecture, which performs well on various semantic matching tasks.

% \begin{table}[tb]
%   \caption{Results of Accuracy and F1 score on MSRP test dataset.}
%   \label{tab:MSRP result}
%   \begin{tabular}{lllll}
%     \toprule
%     Model & Acc.(\%) & F1(\%)  \\
%     \midrule
%     MaLSTM & $66.95$ & $73.95$ \\
%     Factorized MaLSTM & $\mathbf{74.09}$ & $\mathbf{82.18}$ \\
%     HCTI & $73.80$ & $80.85$ \\
%     Factorized HCTI & $\mathbf{74.03}$ & $\mathbf{81.76}$ \\
%     \bottomrule
%   \end{tabular}
%   \vspace{0mm}
% \end{table}


% \begin{table}[tb]
%   \caption{Results of Pearson's $r$ and Spearman's $\rho$ on SICK test dataset.}
%   \label{tab:SICK result}
%   \begin{tabular}{lllll}
%     \toprule
%     Model & r & $\rho$ \\
%     \midrule
%     MaLSTM & $0.7824$ & $0.71843$ \\
%     Factorized MaLSTM & $\mathbf{0.8168}$ & $\mathbf{0.74226}$ \\
%     HCTI & $0.8408$ & $\mathbf{0.7698}$ \\
%     Factorized HCTI & $\mathbf{0.8429}$ & $0.7676$ \\
%     \bottomrule
%   \end{tabular}
%   \vspace{0mm}
% \end{table}


% \begin{table}[tb]
%   \caption{Results of Pearson's $r$ and Spearman's $\rho$ on MSRvid test dataset.}
%   \label{tab:MSRvid result}
%   \begin{tabular}{lll}
%     \toprule
%     Model & r & $\rho$  \\
%     \midrule
%     MaLSTM & $0.7325$ & $0.7193$ \\
%     Factorized MaLSTM & $\mathbf{0.8236}$ & $\mathbf{0.8188}$ \\
%     HCTI & $\mathbf{0.8848}$ & $\mathbf{0.8763}$ \\
%     Factorized HCTI & $0.8763$ & $0.8686$ \\
%     \bottomrule
%   \end{tabular}
%   \vspace{0mm}
% \end{table}



% \begin{table}[tb]
%   \caption{Results of Pearson's $r$ and Spearman's $\rho$ on STSbenchmark test dataset.}
%   \label{tab:STSbenchmark result}
%   \begin{tabular}{lllll}
%     \toprule
%     Model & r & $\rho$ \\
%     \midrule
%     MaLSTM & $0.5739$ & $0.5558$ \\
%     Factorized MaLSTM & $\mathbf{0.6839}$ & $\mathbf{0.6575}$ \\
%     HCTI & $\mathbf{0.7697}$ & $\mathbf{0.7549}$ \\
%     Factorized HCTI & $0.7269$ & $0.7033$ \\
%     \bottomrule
%   \end{tabular}
%   \vspace{0mm}
% \end{table}




% \section{Alternative modeling of QKD}
\label{sec:alternative}

So far we discussed QKD as protocols that start with an insecure
quantum channel and an authentic classical channel and generate, as
the desired ideal resource, a key of fixed length. In this section we
discuss other variants of QKD protocols, where these resources are
chosen differently. In \secref{sec:alternative.adaptive} we
consider an ideal key resource with adaptive key length. In
\secref{sec:alternative.entanglement} we discuss protocols which
use a source of entanglement instead of an insecure quantum
channel. In \secref{sec:alternative.randomness} we show how to
model a situation in which no perfect randomness is available.  In
\secref{sec:alternative.di} we model device\-/independent
QKD. Relaxations of this known as semi\-/device\-/independence are
discussed in \secref{sec:alternative.semi}. Finally, in
\secref{sec:alternative.memoryless} we consider adversaries that have
no quantum memory.


\subsection{Adaptive key length}
\label{sec:alternative.adaptive}

For a protocol to construct the shared secret key resource of \figref{fig:qkd.resource.switch}, it must either abort or produce a key of a fixed length. A more practical protocol could adapt the secret key length to the noise level of the quantum channel. This provides the adversary with the functionality to control the key length (not only whether it gets generated or not), and can be modeled by allowing the key length to be input at Eve's interface of the ideal key resource, as illustrated in \figref{fig:qkd.resource.adaptive}.

\begin{figure}[tb]


\begin{tikzpicture}[
sArrow/.style={->,>=stealth,thick},
largeResource/.style={draw,thick,minimum width=1.618*2cm,minimum height=2cm}]

\small

\def\u{.236} %2/1.618-1

\node[largeResource] (keyBox) at (0,0) {};
\node (alice) at (-2.5,\u) {Alice};
\node (bob) at (2.5,\u) {Bob};
\node (eve) at (0,-1.7) {Eve};
\node[draw] (key) at (0,-.5) {key};

\draw[sArrow,<->] (alice) to node[pos=.22,auto] {$k(m)$} node[pos=.78,auto] {$k(m)$} (bob);
\draw[thick] (0,\u) to (key);
\draw[sArrow] (eve) to node[pos=.3,auto] {$m$} (key);

\end{tikzpicture}

\caption[Secret key resource with adaptive
length]{\label{fig:qkd.resource.adaptive}A secret key resource with   adaptive key length. This resources allows Eve to choose the length $m$   of the final key $k$, which is then output at Alice's and Bob's interfaces.}
\end{figure}

Such an ideal resource has been considered in \textcite{BHLMO05,HT12}. The reduction from the corresponding security definition in AC to a trace distance criterion still goes through. But instead of \eqnref{eq:d}, we get 
\begin{equation} \label{eq:adpative.trace.d}
\sum_m p_m D \left( \rho^m_{KE},\tau^m_K \otimes \rho^m_E \right) \leq
\eps,
\end{equation}
where $p_m$ is the probability of obtaining a key of length $m$, $\rho^m_{KE}$ is the joint state of the key and Eve's system conditioned on the key having length $m$,  and $\tau^m_K$ is a fully mixed state of dimension $2^m$.


\subsection{Source of entanglement}
\label{sec:alternative.entanglement}

In contrast to \emph{prepare\-/and\-/measure} protocols,
\emph{entanglement\-/based} protocols, e.g., \textcite{Eke91,BBM92},
use a source of entanglement, instead of a quantum communication
channel. It is also pretty standard in security proofs to first
transform a given prepare\-/and\-/measure protocol into an
entanglement\-/based one, and then prove the security of the
latter~\cite{SP00}. In \figref{fig:qkd.real.ent} we draw the system
consisting of a QKD protocol $(\pi^{\qkd}_A,\pi^{\qkd}_B)$, the
authentic channel $\aA$ and a source  $\aE$ of entangled states, which may be
controlled by Eve. To specify the completeness property, we also consider
a source of entanglement $\aE'$ that produces a fixed bipartite
entangled state instead of allowing Eve to decide.

\begin{figure}[tb]


\begin{tikzpicture}[
sArrow/.style={->,>=stealth,thick},
thinResource/.style={draw,thick,minimum width=2.4cm,minimum height=1cm},
protocol/.style={draw,rounded corners,thick,minimum width=1.2cm,minimum height=2.5cm},
pnode/.style={minimum width=.8cm,minimum height=.5cm}]

\small

\def\t{4} %.6+1.2+.4+1.2*3/2
\def\u{2.8} %1.2/2+.4+1.2*3/2
\def\v{.75}
\def\w{.6} %1.2/2

\node[pnode] (a1) at (-\u,\v) {};
\node[pnode] (a2) at (-\u,0) {};
\node[pnode] (a3) at (-\u,-\v) {};
\node[protocol] (a) at (-\u,0) {};
\node[yshift=-2,above right] at (a.north west) {\footnotesize
  $\pi^{\qkd}_A$};
\node (alice) at (-\t,0) {};

\node[pnode] (b1) at (\u,\v) {};
\node[pnode] (b2) at (\u,0) {};
\node[pnode] (b3) at (\u,-\v) {};
\node[protocol] (b) at (\u,0) {};
\node[yshift=-2,above right] at (b.north west) {\footnotesize $\pi^{\qkd}_B$};
\node (bob) at (\t,0) {};

\node[thinResource] (cch) at (\w,\v) {};
\node[yshift=-2,above right] at (cch.north west) {\footnotesize
  Authentic ch.~$\aA$};
\node[thinResource] (qch) at (-\w,-\v) {};
\node[yshift=-1.5,above right] at (qch.north west) {\footnotesize
  Source of states $\aE$};
\node (eveq1) at (-\w-.4,-1.75) {};
\node (junc1) at (eveq1 |- a3) {};
\node (eveq2) at (-\w+.4,-1.75) {};
\node (junc2) at (eveq2 |- a3) {};
\node (evec) at (\w+\w,-1.75) {};
\node (junc3) at (evec |- b1) {};

\draw[sArrow,<->] (a1) to node[auto,pos=.08] {$t$} node[auto,pos=.92] {$t$}  (b1);
\draw[sArrow] (junc3.center) to node[auto,pos=.9] {$t$} (evec.center);

\draw[sArrow] (a2) to node[auto,pos=.75,swap] {$k_{A},\bot$} (alice.center);
\draw[sArrow] (b2) to node[auto,pos=.75] {$k_{B},\bot$} (bob.center);

\draw[sArrow,<-] (a3) to (junc1.center) to node[pos=.8,auto,swap] {$\rho_A$} (eveq1.center);
\draw[sArrow] (eveq2.center) to node[pos=.2,auto,swap] {$\rho_B$} (junc2.center) to (b3);

\end{tikzpicture}


\caption[QKD system with source of
entanglement]{\label{fig:qkd.real.ent}A real QKD system that uses a
  source of entangled states. Instead of having access to an insecure
  channel as in \figref{fig:qkd.real.adv}, Alice and Bob use a source
  of entanglement $\aE$ that is controlled by Eve. This means that Eve may
  generate an arbitrary state $\rho_{ABE}$ of which the $A$ register goes to Alice and
  the $B$ register to Bob.}
\end{figure}

The reduction from the AC security definition to the trace distance criterion described in \secref{sec:security} works here, too, with the source of entanglement replacing the insecure channel, resulting in the same conditions for $\eps$\=/secrecy and $\eps$\=/correctness.

One can also show that any protocol designed for a distributed source
of entanglement can be transformed into one where a state is prepared
locally and sent over an (insecure) channel. To explain this, we first
decompose Alice's QKD protocol in two parts.  In the first she carries
out a subprotocol $\alpha$ that performs a measurement
$\bM^a = \{M^a_x\}_x$ on the state received from the source of
entangled states, where $\bM^a$ is chosen with some probability $p_a$
from a set $\{\bM^a\}_a$. The second part consists of the rest of her
QKD protocol. We illustrate this in \figref{fig:qkd.ent.protocol}.

\begin{figure}[tb]


\begin{tikzpicture}[
sArrow/.style={->,>=stealth,thick},
thinResource/.style={draw,thick,minimum width=1.618*2cm,minimum height=1cm},
protocol/.style={draw,rounded corners,thick,minimum width=1.545cm,minimum height=2.5cm},
pnode/.style={minimum width=1cm,minimum height=.5cm},
sqResource/.style={draw,rounded corners,thick,minimum width=1cm,minimum height=1cm}]

\small

\def\t{5.92} %.75+1.545+.5+1+.5+1.618
\def\u{4.39} %1.545/2+.5+1+.5+1.618
\def\um{2.5} %1/2+.5+1.618-.1
\def\ub{2.37} %1.618+.75
\def\v{.75}


\node[pnode] (a1) at (-\u,\v) {};
\node[pnode] (a2) at (-\u,0) {};
\node[pnode] (a3) at (-\u,-\v) {};
\node[protocol] (a) at (-\u,0) {};
\node[yshift=-2,above right] at (a.north west) {\footnotesize
  $\pi^{\qkd}_A$};
\node (alice) at (-\t,0) {};

\node (b1) at (-.4,0 |- a1) {};
\node (b3) at (\ub,-\v) {};

\node[sqResource] (m) at (-\um,-\v) {};
\node[inner sep=1] (mInner) at (-\um,-\v) {$M^a_x$};
\node[yshift=-2,above right] at (m.north west) {\footnotesize
  $\alpha$};

\node[thinResource] (qch) at (0,-\v) {};
\node[yshift=-1.5,above right] at (qch.north west) {\footnotesize
  Source of states $\aE$};
\node (eveq1) at (-.4,-1.75) {};
\node (junc1) at (eveq1 |- a3) {};
\node (eveq2) at (.4,-1.75) {};
\node (junc2) at (eveq2 |- a3) {};

\draw[sArrow] (b1) to node[auto,pos=.85,swap] {$t$}  (a1);

\draw[sArrow] (a2) to node[auto,pos=.75,swap] {$k_{A},\bot$} (alice.center);

\draw[sArrow] (mInner) to node[swap,auto,pos=.48] {$a,x$} (a3);
\draw[sArrow,<-] (mInner) to (junc1.center) to node[pos=.8,auto,swap] {$\rho_A$} (eveq1.center);
\draw[sArrow] (eveq2.center) to node[pos=.264,auto,swap] {$\rho_B$} (junc2.center) to (b3);

\end{tikzpicture}


\caption[Entanglement based QKD
protocol]{\label{fig:qkd.ent.protocol}We split Alice's part of an
  entanglement\-/based QKD protocol in two parts, the measurement of
  the incoming states (denoted by $\alpha$) and the rest of the
  protocol (denoted by $\pi^{\qkd}_A$).}
\end{figure}

We now need to argue that there exists a converter $\gamma$ which constructs
$\alpha \aE$ from an insecure channel $\aQ$ and $\alpha \aE'$ from a
noiseless channel $\aQ'$. For this, we must
establish the two following conditions.
\begin{enumerate}[label=(\roman*), ref=\roman*]
\item \label{eq:ent.sec} There exists a simulator $\sigma_E$ such that
  \[ 
   \gamma \aQ = \alpha \aE \sigma_E.
\]
\item \label{eq:ent.cor} The following equality holds,
  \[\gamma \aQ' = \alpha \aE'.\] 
\end{enumerate}
Once we have established these conditions, it follows immediately from
the composition theorem of the AC framework~\cite{MR11} that any QKD
protocol which is sound when using $\alpha\aE$ and complete when using
$\alpha\aE'$ is also sound and complete when using $\gamma\aQ$ and
$\gamma\aQ'$, respectively.

Let $\rho_{AB}$ be the bipartite entangled state that is generated by
$\aE'$. Let
$\tilde{\varphi}^{x,a}_B \coloneqq \trace[A]{M^a_x \rho_{AB}
  \hconj{\left(M^a_x\right)}}$,
$p_{x|a} \coloneqq \tr \tilde{\varphi}^{x,a}_B$ and
$\varphi^{x,a}_B \coloneqq \tilde{\varphi}^{x,a}_B/p_{x|a}$. We define
the converter $\gamma$ to prepare the state $\varphi^{x,a}_B$ with
probability $p_ap_{x|a}$, which it sends on the insecure
channel. Furthermore, we define the simulator $\sigma_E$ to prepare
$\rho_{AB}$, input the $A$\=/part on the entanglement resource for
Alice and output the $B$\=/part at the outer interface. It is then
straightforward to check from \figref{fig:qkd.ent} that this satisfies
the conditions \eqref{eq:ent.sec} and \eqref{eq:ent.cor} described above.


\begin{figure*}[htb]
\centering
\subfloat[Soundness][\label{fig:qkd.ent.soundness}When modeling
  soundness, the adversary can modify the messages on the insecure
  channel $\aQ$. The simulator $\sigma_E$ generates the entangled state
  $\rho_{AB}$ that is expected from of a non\-/adversarial source of
  entangled states, and outputs the $B$ part at the outer interface,
  making the two systems on the left and right indistinguishable.]{
\begin{tikzpicture}[
sArrow/.style={->,>=stealth,thick},
thinResource/.style={draw,thick,minimum width=1.618*2cm,minimum height=1cm},
sqResource/.style={draw,rounded corners,thick,minimum width=1cm,minimum height=1cm}]

\small

\def\ua{3.85} %.75+1+.5+1.618
\def\um{2.6} %1/2+.5+1.618
\def\ub{2.37} %1.618+.75
\def\t{2.5}

\node (a) at (-\ua,0) {};
\node (b) at (\ub,0) {};

\node[sqResource] (m) at (-\um,0) {};
\node[inner sep=1] (mInner) at (-\um,0) {$\varphi^{x,a}$};
\node[yshift=-2,above right] at (m.north west) {\footnotesize
  $\gamma$};

\node[thinResource] (qch) at (0,0) {};
\node[yshift=-1.5,above right] at (qch.north west) {\footnotesize
  Insecure channel $\aQ$};

\node (le) at (-.4,-\t) {};
\node (re) at (.4,-\t) {};

\node (junc1) at (le |- a) {};
\node (junc2) at (re |- a) {};

\draw[sArrow] (mInner) to node[swap,auto,pos=.6] {$a,x$} (a);

\draw[sArrow] (mInner) to node[auto,pos=.2] {$\varphi$} (junc1.center) to (le);
\draw[sArrow] (re) to (junc2.center) to (b);

\end{tikzpicture}  \hspace{2cm}
\begin{tikzpicture}[
sArrow/.style={->,>=stealth,thick},
thinResource/.style={draw,thick,minimum width=1.618*2cm,minimum height=1cm},
sqResource/.style={draw,rounded corners,thick,minimum width=1cm,minimum height=1cm}]

\small

\def\ua{3.85} %.75+1+.5+1.618
\def\um{2.6} %1/2+.5+1.618
\def\ub{2.37} %1.618+.75
\def\v{.75}
\def\t{2.5}

\node (a) at (-\ua,0) {};
\node (b) at (\ub,0) {};

\node[sqResource] (m) at (-\um,0) {};
\node[inner sep=1] (mInner) at (-\um,0) {$M^a_x$};
\node[yshift=-2,above right] at (m.north west) {\footnotesize
  $\alpha$};

\node[thinResource] (qch) at (0,0) {};
\node[yshift=-1.5,above right] at (qch.north west) {\footnotesize
  Source of states $\aE$};

\node[sqResource] (sim) at (-.4,-2*\v) {};
\node[inner sep=5] (simInner) at (-.4,-2*\v) {$\rho_{AB}$};
\node[xshift=1,below left] at (sim.north west) {\footnotesize $\sigma_E$};

\node (le) at (-.4,-\t) {};
\node (re) at (.4,-\t) {};

\node (junc1) at (le |- a) {};
\node (junc2) at (re |- a) {};

\draw[sArrow] (mInner) to node[swap,auto,pos=.6] {$a,x$} (a);

\draw[sArrow,<-] (mInner) to (junc1.center) to (simInner);
\draw[sArrow] (simInner) to (le);
\draw[sArrow] (re) to (junc2.center) to (b);

\end{tikzpicture}}

\vspace{12pt}

\subfloat[Completeness][\label{fig:qkd.ent.completeness}When modeling
completeness, the source of entanglement $\aE'$ prepares the state
$\rho_{AB}$. The systems on the left and right are
indistinguishable.]{
\begin{tikzpicture}[
sArrow/.style={->,>=stealth,thick},
thinResource/.style={draw,thick,minimum width=1.618*2cm,minimum height=1cm},
% filter/.style={draw,thick,minimum width=1.618cm,minimum height=1cm},
sqResource/.style={draw,rounded corners,thick,minimum width=1cm,minimum height=1cm}]

\small

\def\ua{3.85} %.75+1+.5+1.618
\def\um{2.6} %1/2+.5+1.618
\def\ub{2.37} %1.618+.75
\def\v{.75}

\node (a) at (-\ua,0) {};
\node (b) at (\ub,0) {};

\node[sqResource] (m) at (-\um,0) {};
\node[inner sep=1] (mInner) at (-\um,0) {$\varphi^{x,a}$};
\node[yshift=-2,above right] at (m.north west) {\footnotesize
  $\gamma$};

\node[thinResource] (qch) at (0,0) {};
\node[yshift=-1.5,above right] at (qch.north west) {\footnotesize
  Noiseless channel $\aQ'$};

% \node[filter] (qchf) at (0,-2*\v) {};
% \node[xshift=2,below left] at (qchf.north west) {\footnotesize $\sharp_E$};

% \node[xshift=-.4cm] (qchfl) at (qchf.center) {};
% \node[xshift=.4cm] (qchfr) at (qchf.center) {};
% \node (qchl) at (qchfl |- qchf.north) {};
% \node (qchr) at (qchfr |- qchf.north) {};
% \node (junc1) at (qchfl |- a) {};
% \node (junc2) at (qchfr |- a) {};

\draw[sArrow] (mInner) to node[swap,auto,pos=.6] {$a,x$} (a);

\draw[sArrow] (mInner) to node[auto,pos=.93] {$\varphi$} (b);

% \draw[sArrow] (qchl.center) to (qchfl.center) to (qchfr.center) to (qchr.center);

\end{tikzpicture} \hspace{2cm}
\begin{tikzpicture}[
sArrow/.style={->,>=stealth,thick},
thinResource/.style={draw,thick,minimum width=1.618*2cm,minimum height=1cm},
% filter/.style={draw,thick,minimum width=1.618cm,minimum height=1cm},
sqResource/.style={draw,rounded corners,thick,minimum width=1cm,minimum height=1cm}]

\small

\def\ua{3.85} %.75+1+.5+1.618
\def\um{2.6} %1/2+.5+1.618
\def\ub{2.37} %1.618+.75
\def\v{.75}

\node (a) at (-\ua,0) {};
\node (b) at (\ub,0) {};

\node[sqResource] (m) at (-\um,0) {};
\node[inner sep=1] (mInner) at (-\um,0) {$M^a_x$};
\node[yshift=-2,above right] at (m.north west) {\footnotesize
  $\alpha$};

\node[thinResource] (qch) at (0,0) {};
\node[yshift=-1.5,above right] at (qch.north west) {\footnotesize
  Source of states $\aE'$};

% \node[filter] (qchf) at (0,-2*\v) {};
% \node[xshift=2,below left] at (qchf.north west) {\footnotesize $\lozenge_E$};
\node[draw] (boxState) at (0,0) {$\rho_{AB}$};

% \node[xshift=-.4cm] (qchfl) at (qchf.center) {};
% \node[xshift=.4cm] (qchfr) at (qchf.center) {};
% \node (qchl) at (qchfl |- qchf.north) {};
% \node (qchr) at (qchfr |- qchf.north) {};
% \node (junc1) at (qchfl |- a) {};
% \node (junc2) at (qchfr |- a) {};

\draw[sArrow] (mInner) to node[swap,auto,pos=.6] {$a,x$} (a);

\draw[sArrow] (boxState) to (mInner);
\draw[sArrow] (boxState) to (b);

% \draw[sArrow,<->] (qchl.center) to (qchfl.center) to (qchfr.center) to (qchr.center);

\end{tikzpicture}}

\caption[Using an entanglement protocol with an insecure
channel]{\label{fig:qkd.ent}Pictorial proof for the security of the
  construction of $\alpha\aE$ from $\aQ$ and $\alpha \aE'$ from
  $\aQ'$. Any protocol designed to run with a source of entangled
  states $\aE$ and which measures the incoming states on Alice's side
  as does $\alpha$ can be equivalently used with an insecure channel
  $\aQ$ and a converter $\gamma$ that generates the states to be sent
  on the channel.}
\end{figure*}

\subsection{Imperfect randomness}
\label{sec:alternative.randomness}

QKD protocols usually assume that the honest parties have (arbitrary) access to perfect random numbers. This is however never the case in practice. A more realistic model of a QKD system would consider randomness as a resource that is available in limited and imperfect quantities to Alice and Bob. The real QKD setting drawn in \figref{fig:qkd.real} needs to be changed to take this into account. In \figref{fig:qkd.real.randomness} we depict a QKD protocol that \--- additionally to the insecure quantum channel and authentic classical channel \--- has access to resources producing (local) randomness, $\aR_A$ and $\aR_B$, at Alice's and Bob's interfaces, respectively. A different model of randomness resources might also provide some partial (quantum) information about the randomness to the eavesdropper. For simplicity, however, we chose to draw the simpler case in which $\aR_A$ and $\aR_B$ have an empty interface for the dishonest party.

\begin{figure}[tb]


\begin{tikzpicture}[
sArrow/.style={->,>=stealth,thick},
thinResource/.style={draw,thick,minimum width=2.4cm,minimum height=1cm},
pnode/.style={minimum width=.8cm,minimum height=.5cm},
sqResource/.style={draw,thick,minimum width=1cm,minimum height=1cm},
longProtocol/.style={draw,rounded corners,thick,minimum width=1.2cm,minimum height=4cm}]


\small

\def\t{4.1} %.6+1.2+.5+1.2*3/2
\def\u{2.9} %1.2/2+.5+1.2*3/2
\def\v{1.5}
\def\w{.6} %1.2/2
\def\x{.8} %1.2-.4


\node[pnode] (a1) at (-\u,\v) {};
\node[pnode] (a2) at (-\u,0) {};
\node[pnode] (a3) at (-\u,-\v) {};
\node[longProtocol] (a) at (-\u,0) {};
\node[yshift=-2,above right] at (a.north west) {\footnotesize
  $\pi^{\qkd}_A$};
\node (alice) at (-\t,0) {};

\node[pnode] (b1) at (\u,\v) {};
\node[pnode] (b2) at (\u,0) {};
\node[pnode] (b3) at (\u,-\v) {};
\node[longProtocol] (b) at (\u,0) {};
\node[yshift=-2,above right] at (b.north west) {\footnotesize $\pi^{\qkd}_B$};
\node (bob) at (\t,0) {};

\node[sqResource] (ra) at (-\x-\w,\v) {};
\node[yshift=-2,above right] at (ra.north west) {\footnotesize $\aR_A$};
\node[sqResource] (rb) at (\x+\w,\v) {};
\node[yshift=-2,above right] at (rb.north west) {\footnotesize $\aR_B$};
\node[thinResource] (cch) at (\w,0) {};
\node[yshift=-2,above right] at (cch.north west) {\footnotesize
  Authentic ch.~$\aA$};
\node[thinResource] (qch) at (-\w,-\v) {};
\node[yshift=-1.5,above right] at (qch.north west) {\footnotesize
  Insecure ch.~$\aQ$};
\node (eveq1) at (-\w-.4,-\v-1) {};
\node (junc1) at (eveq1 |- a3) {};
\node (eveq2) at (-\w+.4,-\v-1) {};
\node (junc2) at (eveq2 |- a3) {};
\node (evec) at (\w+\w,-\v-1) {};
\node (junc3) at (evec |- b2) {};

\draw[sArrow] (ra.center) to node[auto,swap,pos=.65] {$r_A$} (a1);
\draw[sArrow] (rb.center) to node[auto,pos=.65] {$r_B$}  (b1);

\draw[sArrow,<->] (a2) to node[auto,pos=.08] {$t$} node[auto,pos=.92] {$t$}  (b2);
\draw[sArrow] (junc3.center) to node[auto,pos=.9] {$t$} (evec.center);

\draw[sArrow] (a2) to node[auto,pos=.75,swap] {$k_{A},\bot$} (alice.center);
\draw[sArrow] (b2) to node[auto,pos=.75] {$k_{B},\bot$} (bob.center);

\draw[sArrow] (a3) to (junc1.center) to node[pos=.8,auto,swap] {$\rho$} (eveq1.center);
\draw[sArrow] (eveq2.center) to node[pos=.264,auto,swap] {$\rho'$} (junc2.center) to (b3);

\end{tikzpicture}



\caption[QKD system with explicit randomness
resource]{\label{fig:qkd.real.randomness} A real QKD system with a
  deterministic protocol $(\pi^{\qkd}_A,\pi^{\qkd}_B)$ and explicit
  sources of randomness $\aR_A$ and $\aR_B$.}
\end{figure}

In such a setting, the converters $\pi^{\qkd}_A$ and $\pi^{\qkd}_B$
are deterministic systems. A QKD protocol would then construct an
ideal key resource given access to these three resources. It remains
an open problem to minimize the assumptions on the sources of
randomness in QKD. Recent results on device\-/independent randomness
amplification~\cite{CR12} show that under certain minimal assumptions\footnote{One
  generally has to assume that no messages leave or enter the quantum
  devices unless authorized by the protocol. Some papers make
  additional assumptions to simplify the protocols and proofs.} about
the workings of an unknown quantum system, one can transform a single
(public) weak source of randomness into a fully (private) random
source~\cite{CSW14,BRGHHHSW16,KAF20}. Alternatively, if two (or more)
sources of weak randomness are available to a player (under certain
strict conditions on the correlations between these different
sources), these can be combined to obtain (approximately) uniform
randomness \cite{CLW14,AFPS16}. Composing this with a standard QKD
protocol would allow secret keys to be distributed when only weak
randomness is available to the honest parties.


\subsection{Device-independent QKD}
\label{sec:alternative.di}

In this review we have so far always considered scenarios for which it
is assumed that the players have trusted quantum devices, which work
exactly according to their specifications. For instance, if a player
instructs the device to generate a $\zero$ state, then it is assumed
that the device generates precisely this state. This assumption is
however not met in any actual implementation with realistic devices,
as these are never perfect. Indeed, there have been numerous
demonstrations of successful attacks against implementations of
quantum cryptographic protocols that exploited deviations of the
devices' functionality from the specifications, as discussed in
\secref{sec:attacks}. Crucially, this problem cannot be solved only by
a more careful design of the devices, for it appears to be impossible
to guarantee their perfect working under all possible environmental
conditions.

A theoretical solution to this problem is to devise protocols whose
security does not rely on the assumption that devices are perfect.
Ideally, they should provide security guarantees even if the devices
are untrusted, meaning that their behavior may deviate arbitrarily
from the specification. Remarkably, using quantum devices, this is
possible (with certain caveats described below). The idea is to use a
phenomenon called \emph{(Bell) non\-/locality} \cite{Bell64} \--- see
also \textcite{Sca13,BCPSW14} for review articles on the topic. The
subfield of cryptography that studies the use of non\-/locality to
design protocols that work with untrusted devices is termed
\emph{device\-/independent cryptography}.
  
In a nutshell, a Bell inequality is a bound on the probability of
observing certain values in an experiment involving measurements of two isolated (and hence non-communicating) systems. The bound characterises classical locality: it cannot be violated if
the two isolated systems are described by classical physics. However, the bound can be violated by measurements on entangled quantum systems.   One of the most commonly used Bell inequalities is the
CHSH inequality \cite{CHSH69}. It states that, if two players each
hold non\-/communicating systems, and each performs one out of two
binary measurements chosen uniformly at random on their respective
system, where the choice of the measurement is given by
$x,y \in \{0,1\}$ and the outcome is given by $a,b \in \{0,1\}$,
respectively, then the probability that $xy = a \xor b$ should be less than
or equal to $3/4$.\footnote{An alternative formulation of the
  inequality is
  $\left| E(0,0) + E(0,1) + E(1,0) - E(1,1) \right| \leq 2$, where
  $E(x,y)$ is the expected value of the product of the outcomes of the
  systems when measured with settings $x$ and $y$, respectively, and
  the outcomes are values in $\{-1,+1\}$.} But if the systems are
quantum, it is possible to observe this outcome with probability up to
$\approx .85$ \--- this is achieved if the systems are in a perfectly
entangled state and the players perform an optimal measurement.

An observation of a violation of a Bell inequality implies that the
measurement outcomes contain some genuine randomness
\cite{Col06,PAM10,AMP12,CR12}, even conditioned on the knowledge of
the person who set up and programmed the devices used in the
experiment \--- the only assumptions being that no information other
than the measurement result leaves the devices, and that these devices
never fall in the hands of an adversary, since their internal memory
may contain a copy of the measurement outcomes. This randomness may
then be used to generate uniform random numbers
\cite{VV12,CSW14,MS14,BRGHHHSW16,KAF20} or a shared secret key
\cite{BHK05,PABGMS09,VV14,ADFRV18,AFRV19}.

For a review of different results and techniques in
device\-/independent cryptography, we refer to \textcite{ER14}. In
this section we show how to model device\-/independent quantum key
distribution (DI-QKD) in the AC framework. It then follows from the
composition theorem of AC, that the resulting key can be safely used
in applications requiring one.

The converters $\pi^\qkd_A$ and $\pi^\qkd_B$ modeling Alice's and
Bob's parts of the protocol in \secref{sec:qkd} are systems which
generate quantum states and perform measurements. In DI-QKD, exactly
these operations cannot be trusted. So instead, the DI protocol
$(\pi^\diqkd_A,\pi^\diqkd_B)$ will only involve \emph{classical}
operations. Everything \emph{quantum} is moved into a resource, a
device $\aD$. The honest players can send bits to these devices, and
receive bits back from them \--- this corresponds to choosing a
measurement $x,y \in \{0,1\}$ and receiving the outcome
$a,b \in \{0,1\}$, described a few paragraphs earlier. The adversary is
permitted to ``program'' these devices by providing some initial state
$\rho$ as input \--- depending on the model, Eve may be allowed to
provide further inputs to the device at some later point, e.g., to
provide more EPR pairs so that it may continue running. The
corresponding real world is drawn in
\figref{fig:alternatives.diqkd}. The ideal world will be identical to
that of standard QKD, since we wish to construct the same key
resource, i.e., \figref{fig:qkd.resource.sim}.


\begin{figure}[tb]

\begin{tikzpicture}[
sArrow/.style={->,>=stealth,thick},
thinResource/.style={draw,thick,minimum width=2.4cm,minimum height=1cm},
sqResource/.style={draw,thick,minimum width=1cm,minimum height=1cm},
protocol/.style={draw,rounded corners,thick,minimum width=1.2cm,minimum height=2.5cm},
pnode/.style={minimum width=.6cm,minimum height=.5cm}]

\small

\def\t{4} %.6+1.2+.4+1.2*3/2
\def\u{2.8} %1.2/2+.4+1.2*3/2
\def\v{.75}
\def\w{1.3} 

\node[pnode] (a1) at (-\u,\v) {};
\node[pnode] (a2) at (-\u,0) {};
\node[pnode] (a3) at (-\u,-\v) {};
\node[protocol] (a) at (-\u,0) {};
\node[yshift=-2,above right] at (a.north west) {\footnotesize
  $\pi^{\diqkd}_A$};
\node (alice) at (-\t,0) {};

\node[pnode] (b1) at (\u,\v) {};
\node[pnode] (b2) at (\u,0) {};
\node[pnode] (b3) at (\u,-\v) {};
\node[protocol] (b) at (\u,0) {};
\node[yshift=-2,above right] at (b.north west) {\footnotesize $\pi^{\diqkd}_B$};
\node (bob) at (\t,0) {};

\node[thinResource] (cch) at (0,\v) {};
\node[yshift=-2,above right] at (cch.north west) {\footnotesize
  Authentic ch.~$\aA$};
\node[sqResource] (da) at (-\w,-\v) {$\aD_A$};
\node[yshift=-1.5,above] at (da.north) {\footnotesize
  Device};
\node[pnode] (dan) at (-\w,-\v) {};
\node[sqResource] (db) at (\w,-\v) {$\aD_B$};
\node[yshift=-1.5,above] at (db.north) {\footnotesize
  Device};
\node[pnode] (dbn) at (\w,-\v) {};

\node (eveq1) at (-\w,-1.75) {};
\node (eveq2) at (\w,-1.75) {};
\node (evec) at (0,-1.75) {};
\node (junc3) at (evec |- b1) {};

\draw[sArrow,<->] (a1) to node[auto,pos=.2] {$t$} node[auto,pos=.8] {$t$}  (b1);
\draw[sArrow] (junc3.center) to node[auto,pos=.9] {$t$} (evec.center);

\draw[sArrow] (a2) to node[auto,pos=.75,swap] {$k_{A},\bot$} (alice.center);
\draw[sArrow] (b2) to node[auto,pos=.75] {$k_{B},\bot$} (bob.center);

\draw[sArrow] (eveq1.center) to node[pos=.3,auto,swap] {$\rho_A$} (dan);
\draw[sArrow] (eveq2.center) to node[pos=.3,auto,swap] {$\rho_B$} (dbn);

\draw[sArrow,bend left] (a3) to node[pos=.5,auto] {$x$} (dan);
\draw[sArrow,bend left] (dan) to node[pos=.5,auto] {$a$} (a3);
\draw[sArrow,bend right] (b3) to node[pos=.5,auto,swap] {$y$} (dbn);
\draw[sArrow,bend right] (dbn) to node[pos=.5,auto,swap] {$b$} (b3);

\end{tikzpicture}


\caption[DI-QKD]{\label{fig:alternatives.diqkd}The real world setting
  for  a DI-QKD protocol. Eve can program the devices $\aD$, but cannot
  receive any output from them.}
\end{figure}

Applying \defref{def:security}, this means that the protocol
$(\pi^\diqkd_A,\pi^\diqkd_B)$ constructs $\aK$ from $\aA$, $\aD_A$ and
$\aD_B$ within $\eps$ if
\begin{equation} \label{eq:diqkd}
  \exists \sigma_E, \quad \pi_A^{\diqkd}\pi_B^{\diqkd}(\aD_A \| \aD_B \| \aA)
  \close{\eps} \aK \sigma_E.
\end{equation}
Note that we have not specified the behaviors of the devices $\aD_A$
and $\aD_B$ at all. In fact, we need \eqnref{eq:diqkd} to hold for all
devices $\aD_A$ and $\aD_B$.\footnote{The simulator may depend on
  these devices, i.e., $\forall \aD_A, \aD_B, \exists \sigma_E$ such
  that \eqnref{eq:diqkd} holds.} This is exactly the
\emph{device\-/independant} guarantee, namely that security holds
regardless of how the (quantum) devices work. Alternatively one can
consider fixed devices $\aD_A$ and $\aD_B$ that are universal
computers, and have their program be part of the inputs at the $E$ interface.

As usual, completeness is captured by specific devices $\aD'_A$ and
$\aD'_B$ that work honestly \--- e.g., they share perfectly entangled
states and perform the correct measurements as specified by the
protocol \--- as well as the same honest resources $\aA'$ and $\aK'$
as in \secref{sec:qkd}. Additionally to
\eqnref{eq:diqkd}, we also need
\begin{equation*} %\label{eq:qkd.robust}
  \pi_A^{\diqkd}\pi_B^{\diqkd}(\aD'_A \| \aD'_B
  \| \aA') \close{\eps'} \aK'.
\end{equation*}

The same reduction as for (normal) QKD goes through, and one can show
that \eqnref{eq:diqkd} is satisfied if for all behaviors of the
devices (and their inputs), \eqnsref{eq:qkd.cor} and \eqref{eq:qkd.sec}
hold.

Note however that the construction outlined in this section only
allows the devices $\aD_A$ and $\aD_B$ to be accessed during the
protocol. No access is granted after the protocol ends, meaning that
we make no security statement about what happens if the devices are
reused. It is an open question how to reuse devices in DI
cryptography, which we discuss in \secref{sec:open.di}.

Proving security of device-independent QKD is more challenging than in
the device-dependent case. One of the difficulties is that the
measurement operators that describe Alice and Bob's measurement can be
arbitrary. In particular it cannot be assumed, for instance, that two
subsequent measurement outcomes by Bob are obtained by two separate
measurement processes. While some of the techniques described
in~\secref{sec:securityproofs}, such as entropy accumulation, are
still applicable to the device-independent setting, others, like de
Finetti-type theorems, are not or must be adapted,
[see~\textcite{ArnonThesis} for details].

\subsection{Semi-device-independent QKD}
\label{sec:alternative.semi}

The only assumption made about the devices in DI-QKD is that no
information leaves these devices unless allowed by the protocols (see
\secref{sec:alternative.di}). But achieving the violation of Bell
inequalities needed for this is challenging because it requires high
detector efficiency and tolerates only low noise on the
channel~\cite{BCPSW14}. Protocols that are easier to implement can be
achieved by making additional assumptions about the quantum devices
used by Alice and Bob. These are generally called
\emph{semi\-/device\-/independent} (semi-DI).

Many different assumptions may be labeled semi-DI. For example, in a
one-sided device-independent setting the protocol is DI for Bob but
not for Alice \cite{BCWSW12}. One may also assume dimension bounds on
the quantum systems generated by untrusted devices as in
\textcite{PB11}. Alternatively, \textcite{LPTRG13} assume that each
use of the devices are causally independent \--- i.e., the states
generated and measurements performed are in product form \--- to
analyze a protocol where the Bell violation is measured locally in
Alice's lab, thus avoiding the noise introduced by the channel between
Alice and Bob. Similar ideas have been used for other protocols than
QKD, e.g., semi-DI quantum money~\cite{HS20,BDG19}.

One of the most promising forms of semi-DI QKD, which has already been
implemented over large distances
\cite{Liu13,Tang2014,Pirandola2015,Yin2016} is
\emph{measurement\-/device\-/independent} (MDI) QKD
\cite{LCQ12,BP12,MR12,CXCLTL14}. Here, one assumes that players can
generate the states they desire, but one does not trust measurement
devices at all. This model is motivated by the attacks on the
detectors, e.g., the time-shift attacks or detector blinding attacks
discussed in \secref{sec:attacks:hacking}.

To understand how such protocols work, we will start from an
entanglement based protocol as in
\secref{sec:alternative.entanglement}, then modify it step by
step, until we achieve a prepare\-/and\-/measure protocol, in which
all measurements are performed by the adversary. Since the final
protocol is as secure as the original one, and the original one is
secure for all adversaries, the final MDI QKD protocol is secure
for all adversaries as well. In particular, it is secure for
adversaries that completely control the measurement apparatus.

In an entanglement based protocol, Alice and Bob receive the $A$ and
$B$ parts of a state $\psi_{ABR}$, and measure these systems in either
the computational or diagonal basis, obtaining a raw key. This key is
then processed as in a prepare\-/and\-/measure protocol (see
\secref{sec:qkd.protocol} and \secref{sec:securityproofs}).  If the
source gave them a state which is (close to) a tensor product of EPR
pairs, such a protocol is guaranteed to terminate with a shared secret
key. Equivalently, the source could generate any of the Bell states,
and notify Alice and Bob which one it gave them. They then perform bit
or phase flips to change it to an EPR pair.

Instead of the source distributing an entangled state, Alice and Bob
could each generate an EPR pair $AA'$ and $BB'$, respectively. They
then send $A'$ and $B'$ to a third party, Charlie, who measures this
in the Bell basis, and tells them the measurement outcome. If
performed correctly, the $AB$ system will be in a Bell state, and the
measurement outcome will tell them which one. By flipping bits or
phases, Alice and Bob can turn this into an EPR pair, and continue
with the protocol as above. Crucially, if Charlie does not perform the
correct measurement, then Alice and Bob will end up holding the $A$
and $B$ parts of some unknown state $\psi_{ABR}$. But this does not
compromise security: if it is too far from the expected state, the
protocol will just abort.

Instead of first performing a bit or phase flip, and then measuring,
Alice and Bob could first measure their systems $A$ and $B$, and then
flip the value of the measurement outcome if needed. And instead of
generating EPR pairs $AA'$ and $BB'$, then measuring $A$ and $B$, they
could pick the measurement outcome at random, then generate the
corresponding reduced state in $A'$ and $B'$ respectively, send these
to Charlie, and when they obtain the measurement outcome from Charlie,
they flip their bits if needed.

The only (trusted) quantum operations that Alice and Bob need to
perform in the protocol described in the paragraph above are
generating the states in the systems $A'$ and $B'$. All measurements
have now been delegated to Charlie, who may deviate arbitrarily from
the protocol without compromising security.

The real world for such a MDI-QKD protocol is drawn in
\figref{fig:alternatives.mdi}, where one can see that the converters
$\pi^{\mdi}_A$ and $\pi^{\mdi}_B$ do not have any incoming quantum
states, i.e., they do not need to perform any measurement.

Security proofs for MDI-QKD protocols can be based on the same techniques as those for fully device-independent protocols, as discussed in \secref{sec:alternative.di}. The comments on security proofs made in that section thus also apply here. 

\begin{figure}[tb]

\begin{tikzpicture}[
sArrow/.style={->,>=stealth,thick},
thinResource/.style={draw,thick,minimum width=2.4cm,minimum height=1cm},
sqResource/.style={draw,thick,minimum width=1.4cm,minimum height=1cm},
protocol/.style={draw,rounded corners,thick,minimum width=1.2cm,minimum height=2.5cm},
pnode/.style={minimum width=.6cm,minimum height=.5cm}]

\small

\def\t{4} %.6+1.2+.4+1.2*3/2
\def\u{2.8} %1.2/2+.4+1.2*3/2
\def\v{.75}
\def\w{1.1} 
\def\a{.3}

\node[pnode] (a1) at (-\u,\v) {};
\node[pnode] (a2) at (-\u,0) {};
\node[pnode] (a3) at (-\u,-\v) {};
\node[protocol] (a) at (-\u,0) {};
\node[yshift=-2,above right] at (a.north west) {\footnotesize
  $\pi^{\mdi}_A$};
\node (alice) at (-\t,0) {};

\node[pnode] (b1) at (\u,\v) {};
\node[pnode] (b2) at (\u,0) {};
\node[pnode] (b3) at (\u,-\v) {};
\node[protocol] (b) at (\u,0) {};
\node[yshift=-2,above right] at (b.north west) {\footnotesize $\pi^{\mdi}_B$};
\node (bob) at (\t,0) {};

\node[thinResource] (cch) at (0,\v) {};
\node[yshift=-2,above right] at (cch.north west) {\footnotesize
  Authentic ch.~$\aA$};
\node[sqResource] (da) at (-\w,-\v) {};
\node[yshift=-1.5,above] at (da.north) {\footnotesize
  Channel $\aC_A$};
\node[pnode] (dan) at (-\w,-\v) {};
\node[sqResource] (db) at (\w,-\v) {};
\node[yshift=-1.5,above] at (db.north) {\footnotesize
  Channel $\aC_B$};
\node[pnode] (dbn) at (\w,-\v) {};

\node (eveq11) at (-\w-\a,-1.75) {};
\node (eveq12) at (-\w+\a,-1.75) {};
\node (eveq21) at (\w-\a,-1.75) {};
\node (eveq22) at (\w+\a,-1.75) {};
\node (evec) at (0,-1.75) {};
\node (junc3) at (evec |- b1) {};

\draw[sArrow,<->] (a1) to node[auto,pos=.2] {$t$} node[auto,pos=.8] {$t$}  (b1);
\draw[sArrow] (junc3.center) to node[auto,pos=.9] {$t$} (evec.center);

\draw[sArrow] (a2) to node[auto,pos=.75,swap] {$k_{A},\bot$} (alice.center);
\draw[sArrow] (b2) to node[auto,pos=.75] {$k_{B},\bot$} (bob.center);

\draw[sArrow] (a3.south east) to (eveq11 |- a3.south east) to node[pos=.8,auto,swap] {$\rho_{A'}$} (eveq11.center);
\draw[sArrow] (b3.south west) to (eveq22 |- b3.south west) to node[pos=.8,auto] {$\rho_{B'}$} (eveq22.center);
\draw[sArrow] (eveq12.center) to node[pos=.2,auto,swap,xshift=-1] {$b$} (eveq12 |- a3.north east) to (a3.north east);
\draw[sArrow] (eveq21.center) to node[pos=.2,auto,xshift=1] {$b'$} (eveq21 |- b3.north west) to (b3.north west);

\end{tikzpicture}


\caption[MDI-QKD]{\label{fig:alternatives.mdi}The real world setting
  for a MDI-QKD protocol. The only quantum operations performed by
  $\pi^{\mdi}_A$ and $\pi^{\mdi}_B$ are the generation of quantum
  states. The communication resources $\aC$ send quantum systems from
  Alice or Bob to Eve, and classical bits from Eve to Alice and Bob..}
\end{figure}

\subsection{Memoryless adversaries}
\label{sec:alternative.memoryless}

The previous sections analyzed different models of QKD, in which we
changed the capabilities and resources of the honest players running
the protocol. Similar techniques may also be used to model
limitations on adversaries. In this last section we consider as example QKD
protocol with an adversary that has no (long-term) quantum memory, and is
forced to measure the quantum states exchanged between Alice and Bob
during the QKD protocol and store the classical information.

The insecure channel resource, $\aQ$, modeled as part of the real QKD
system in \figref{fig:qkd.real.adv} gives complete control over the
states sent on this channel to the adversary. Since this may include
storing them and measuring them at a later point, we need to limit the
adversary's access to this channel as part of the insecure channel
resource. We may thus define a different channel $\tilde{\aQ}$, which
requires Eve to input some measurement specification and then obtains
the measurement outcome at her interface. The resulting
post\-/measurement state is then output at Bob's interface.

The model of $\tilde{\aQ}$ described above is just one possible way
one may imagine limiting Eve's access to the states sent during
QKD. The result is a change in the requirements of the QKD
protocol. Instead of constructing a secure key $\aK$ from an authentic
channel $\aA$ and an insecure channel $\aQ$, it is now sufficient if
$\aK$ can be constructed from $\aA$ and $\tilde{\aQ}$. It is not hard
to see that, since the accessible information (see
\secref{sec:qkd.other.ai}) measures the information an adversary has
\emph{after} measuring their quantum states, a QKD protocol with low
accessible information would satisfy such a construction \--- namely,
$\aA \| \tilde{\aQ} \rightarrow \aK$. The accessible information
security measure is thus a composable criterion under the assumption
that the adversary has such a physical limit on their memory.

Since QKD protocols are secure against general adversaries as modeled
in \secref{sec:qkd}, there does not seem to be much incentive to
consider adversaries with limitations on their memory (unlike for some
two-party protocols discussed in \secref{sec:open.other}). It is
however noteworthy that, as already mentioned in
\secref{sec:qkd.other.ac}, by explicitly limiting the adversary's
capabilities we capture weaker security notions that appeared in the literature.


%%% Local Variables:
%%% TeX-master: "main.tex"
%%% End:

% The industry standard for pose edition is to create rigs, a collection of pieces of software designed to manipulate a character's skeleton. The rig describes the skeleton's bones, how they relate to each other, are constrained in their possible motion and are deformed. These rules are loosely specified and creating a good rig requires a detailed understanding of physics and anatomy, as well as technical and artistic skills. Rigging is thus a time consuming task even for experienced animators, and even more so in large scale productions which often require a different in-depth rig for each character in the cast.
Previous work has helped alleviate this difficulty by providing efficient tools to speed up/and or ease the rigging process, relying on inverse kinematics or data-driven methods.
\subsection{Character pose design}
\subsubsection{Inverse Kinematics (IK)}
IK solvers are a family of methods commonly used in robotics, engineering and computer graphics, in which the parameterization of a kinematic chain is determined from the position of its end effector.
They are a staple tool in pose design software, ensuring the respect of elementary constraints during pose edition. Their de-facto role is to guarantee the length of the limbs, and in some cases to enforce the orientation angle range of a joint.
Many IK solutions have been studied over the years \cite{aristidou_inverse_2018}; usually revolving around approximated linearizations or heuristics. 

Numerical methods require a set of iterations to achieve a satisfactory solution formulated by a cost function to be minimized.
IK solutions can generally be divided into three sub-categories: Jacobian \cite{Siciliano_Handbook_Robot_2007}, Newtonians \cite{cohen_ik_1996} and Heuristics. Most software implement heuristic methods such as Cyclic Coordinate Descent (CCD) \cite{wang_ccd_1991} or 
Forward-Backward Reaching IK (FABRIK) \cite{aristidou_fabrik:_2011} due to their simplicity and extensibility. 

The main drawback of 
these solvers is that they manipulate kinematic chains without taking into account many morphological aspects that make a pose more or less plausible. They offer a first level of help to users but are not sufficient to guarantee a realistic pose. Many joints constraints are dependent on each other and require subjective, human-made approximations.

\subsubsection{Data-driven pose edition}
Data-driven methods offer promising opportunities to solve these approximations. Using real-life data can help in modelling the complex inter-dependencies of skeletons and providing users with smarter edition tools.
While it is still an early field of research, some solutions have been studied. Wu \etal \cite{wu_posing_2009} propose a method for natural character posing from a large motion database. It employs adaptive KD-clustering to select a representative frame from a database and sparse approximations to accelerate training and posing. 
Huang \etal in \cite{Huang_IK_MGDM_2017} present a method based on the formulation of multi-variate Gaussian distribution models (MGDMs), which learn the joint constraints of a kinematic skeleton from motion capture data. 

Some work has also been dedicated to finding new editing interfaces. \modify{}{Instead of the usual setup manipulating joints directly, Guay \etal \cite{guay_line_2013} articulate a framework based on the conceptual "line of action" which describes the overall pose dynamics. They provide a mathematical definition of the line of action, and a interface in which the software modifies the pose to follow a user-provided line. In the same line of though} Garcia \etal \cite{garcia_sketching_2019} propose \modify{a method transforming doodle of trajectories (position and orientation over time) }{a virtual reality-based interface where the user's hands motion (position and orientation over time) are transformed} into sequences of actions and then into detailed character animations using a dataset of parametrized motion clips automatically fitted to the trajectory. 

% ==> DL et Latent Space. 
\subsection{Neural modelling of human motion}
Neural networks have received a great amount of attention over the last decade and shown impressive result in modelling complex data. Human motion has not been spared and deep learning methods have proven their capability of generating realistic motion in a number of difficult cases. 

The literature in neural-based animation include example in user-controlled character navigation \cite{Holden2017} and interactions with the environment \cite{starke_neural_2019}. 
Holden \etal \cite{Holden2020} also show that neural networks can be used to replace parts of existing data-driven methods, improving their scalability potential.
More recently, some work has also focused on improving smaller parts of the animation pipeline rather than replacing it completely. Berson et al. \cite{berson_intuitive_2020} leverage neural networks to provide an interactive system to edit facial animation. 

% Wrap up
Data-driven IK and pose editing can relieve animators from time-consuming, back-and-forth pose adjustments by applying constraints extracted from real-world data. Recently, neural-network-based approaches have demonstrated their ability to model the intricacies of human motion while scaling to large amount of data and retaining a fast inference time. In this paper we seek to take advantage of these properties to create an efficient posing tool, intuitively usable even by a inexperienced user.

\begin{comment}
\begin{figure}
\includegraphics[width=\linewidth]{figs/beyond_tss_lesion.pdf}
\caption[]{End-to-End runtime lesion study of the entire MNIST dataset and the FMA featurized music dataset. Each of DROP's contributions provides a runtime improvement.}
\label{fig:beyond_lesion}
\end{figure}
\end{comment}



\section{Conclusion}
\label{sec:conclusion}

Advanced data analytics techniques must scale to rising data volumes. 
DR techniques offer a powerful toolkit when processing these datasets, with PCA frequently outperforming popular techniques in exchange for high computational cost. 
In response, we propose DROP, a new dimensionality reduction optimizer. 
DROP combines progressive sampling, progress estimation, and online aggregation to identify high quality low dimensional bases via PCA without processing the entire dataset by balancing the runtime of downstream tasks and achieved dimensionality. 
Thus, DROP provides a first step in bridging the gap between quality and efficiency in end-to-end DR for downstream \red{analytics}. 

%We revisit canonical operators for time series dimensionality reduction and the measurement study of~\cite{keogh-study}, and show that PCA is more effective than popular alternatives in the data mining literature often by a margin of over $2\times$ on average on gold-standard time series benchmark data sets with respect to output data dimension. More surprisingly, we empirically demonstrate that a small number of samples are sufficient to accurately characterize directions of maximum variance and obtain a high-quality low-dimensional transformation.




% \input{L_free_formulation.tex}

% The authors would like to thank...

\renewcommand{\nariman}[1]{\textcolor{red}{#1}}
\renewcommand{\amir}[1]{\textcolor{blue}{#1}}




% trigger a \newpage just before the given reference
% number - used to balance the columns on the last page
% adjust value as needed - may need to be readjusted if
% the document is modified later
%\IEEEtriggeratref{8}
% The "triggered" command can be changed if desired:
%\IEEEtriggercmd{\enlargethispage{-5in}}

% references section

% can use a bibliography generated by BibTeX as a .bbl file
% BibTeX documentation can be easily obtained at:
% http://www.ctan.org/tex-archive/biblio/bibtex/contrib/doc/
% The IEEEtran BibTeX style support page is at:
% http://www.michaelshell.org/tex/ieeetran/bibtex/
%\bibliographystyle{IEEEtran}
% argument is your BibTeX string definitions and bibliography database(s)
%\bibliography{IEEEabrv,../bib/paper}
%
% <OR> manually copy in the resultant .bbl file
% set second argument of \begin to the number of references
% (used to reserve space for the reference number labels box)

\bibliographystyle{IEEEtran}
\bibliography{bibliography}

% \begin{thebibliography}{1}
% \bibitem{love15}
% J. Song, J. Choi and D. J. Love, "Codebook design for hybrid beamforming in millimeter wave systems," 2015 IEEE International Conference on Communications (ICC), 2015, pp. 1298-1303, doi: 10.1109/ICC.2015.7248502.

% \bibitem{ayach14}
% O. E. Ayach, S. Rajagopal, S. Abu-Surra, Z. Pi and R. W. Heath, "Spatially Sparse Precoding in Millimeter Wave MIMO Systems," in IEEE Transactions on Wireless Communications, vol. 13, no. 3, pp. 1499-1513, March 2014, doi: 10.1109/TWC.2014.011714.130846.

% \bibitem{song15}
% J. Song, J. Choi, S. G. Larew, D. J. Love, T. A. Thomas and A. A. Ghosh, "Adaptive Millimeter Wave Beam Alignment for Dual-Polarized MIMO Systems," in IEEE Transactions on Wireless Communications, vol. 14, no. 11, pp. 6283-6296, Nov. 2015, doi: 10.1109/TWC.2015.2452263.

% \bibitem{noh17}
% S. Noh, M. D. Zoltowski and D. J. Love, "Multi-Resolution Codebook and Adaptive Beamforming Sequence Design for Millimeter Wave Beam Alignment," in IEEE Transactions on Wireless Communications, vol. 16, no. 9, pp. 5689-5701, Sept. 2017, doi: 10.1109/TWC.2017.2713357.

% \bibitem{Hussain17}
%  M. Hussain, D. J. Love, and N. Michelusi, “Neyman-pearson codebook
% design for beam alignment in millimeter-wave networks,” in Proc. 1st
% ACM Workshop Millim.-Wave Netw. Sens. Syst., Oct. 2017, pp. 17–22.

% \bibitem{parseval}
% S. Hur, T. Kim, D. J. Love, J. V. Krogmeier, T. A. Thomas and A. Ghosh, "Millimeter Wave Beamforming for Wireless Backhaul and Access in Small Cell Networks," in IEEE Transactions on Communications, vol. 61, no. 10, pp. 4391-4403, October 2013, doi: 10.1109/TCOMM.2013.090513.120848.

% \bibitem{khateeb14}
% A. Alkhateeb, O. El Ayach, G. Leus and R. W. Heath, "Channel Estimation and Hybrid Precoding for Millimeter Wave Cellular Systems," in IEEE Journal of Selected Topics in Signal Processing, vol. 8, no. 5, pp. 831-846, Oct. 2014, doi: 10.1109/JSTSP.2014.2334278.

% \bibitem{vlachos19}
% E. Vlachos, G. C. Alexandropoulos and J. Thompson, "Wideband MIMO Channel Estimation for Hybrid Beamforming Millimeter Wave Systems via Random Spatial Sampling," in IEEE Journal of Selected Topics in Signal Processing, vol. 13, no. 5, pp. 1136-1150, Sept. 2019, doi: 10.1109/JSTSP.2019.2937633.

% \bibitem{noh16}
% S. Noh, M. D. Zoltowski and D. J. Love, "Training Sequence Design for Feedback Assisted Hybrid Beamforming in Massive MIMO Systems," in IEEE Transactions on Communications, vol. 64, no. 1, pp. 187-200, Jan. 2016, doi: 10.1109/TCOMM.2015.2498184.

% \bibitem{vlachos18}
% E. Vlachos, G. C. Alexandropoulos and J. Thompson, "Massive MIMO Channel Estimation for Millimeter Wave Systems via Matrix Completion," in IEEE Signal Processing Letters, vol. 25, no. 11, pp. 1675-1679, Nov. 2018, doi: 10.1109/LSP.2018.2870533.

% \bibitem{qiao17}
% D. Qiao, H. Qian and G. Y. Li, "Multi-resolution codebook design for two-stage precoding in FDD massive MIMO networks," 2017 IEEE 18th International Workshop on Signal Processing Advances in Wireless Communications (SPAWC), 2017, pp. 1-5, doi: 10.1109/SPAWC.2017.8227756.

% \end{thebibliography}

% that's all folks
\end{document}