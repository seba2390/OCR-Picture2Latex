%% bare_conf.tex
%% V1.4a
%% 2014/09/17
%% by Michael Shell
%% See:
%% http://www.michaelshell.org/
%% for current contact information.
%%
%% This is a skeleton file demonstrating the use of IEEEtran.cls
%% (requires IEEEtran.cls version 1.8a or later) with an IEEE
%% conference paper.
%%
%% Support sites:
%% http://www.michaelshell.org/tex/ieeetran/
%% http://www.ctan.org/tex-archive/macros/latex/contrib/IEEEtran/
%% and
%% http://www.ieee.org/

%%*************************************************************************
%% Legal Notice:
%% This code is offered as-is without any warranty either expressed or
%% implied; without even the implied warranty of MERCHANTABILITY or
%% FITNESS FOR A PARTICULAR PURPOSE! 
%% User assumes all risk.
%% In no event shall IEEE or any contributor to this code be liable for
%% any damages or losses, including, but not limited to, incidental,
%% consequential, or any other damages, resulting from the use or misuse
%% of any information contained here.
%%
%% All comments are the opinions of their respective authors and are not
%% necessarily endorsed by the IEEE.
%%
%% This work is distributed under the LaTeX Project Public License (LPPL)
%% ( http://www.latex-project.org/ ) version 1.3, and may be freely used,
%% distributed and modified. A copy of the LPPL, version 1.3, is included
%% in the base LaTeX documentation of all distributions of LaTeX released
%% 2003/12/01 or later.
%% Retain all contribution notices and credits.
%% ** Modified files should be clearly indicated as such, including  **
%% ** renaming them and changing author support contact information. **
%%
%% File list of work: IEEEtran.cls, IEEEtran_HOWTO.pdf, bare_adv.tex,
%%                    bare_conf.tex, bare_jrnl.tex, bare_conf_compsoc.tex,
%%                    bare_jrnl_compsoc.tex, bare_jrnl_transmag.tex
%%*************************************************************************


% *** Authors should verify (and, if needed, correct) their LaTeX system  ***
% *** with the testflow diagnostic prior to trusting their LaTeX platform ***
% *** with production work. IEEE's font choices and paper sizes can       ***
% *** trigger bugs that do not appear when using other class files.       ***                          ***
% The testflow support page is at:
% http://www.michaelshell.org/tex/testflow/



\documentclass[conference]{IEEEtran}
\usepackage[utf8]{inputenc}
\usepackage{times,graphics, dsfont, epsfig,amsmath,xspace,endnotes,pifont,multirow,rotating,listings,amssymb,algorithmic,color,caption,nicefrac,adjustbox,tabularx,mathtools, algorithmic,algorithm}
\usepackage{graphicx}
% modifications for tabular
\usepackage{tabularx}

\newcolumntype{L}[1]{>{\raggedright\arraybackslash}p{#1}} 
\newcolumntype{C}[1]{>{\centering\arraybackslash}p{#1}} 
\newcolumntype{R}[1]{>{\raggedleft\arraybackslash}p{#1}} %
\usepackage{steinmetz}
\usepackage{color}
\usepackage[dvipsnames]{xcolor}
\usepackage{amsmath}
\usepackage{amssymb}
\usepackage{float}
\ifCLASSOPTIONcompsoc
    \usepackage[caption=false, font=normalsize, labelfont=sf, textfont=sf]{subfig}
\else
\usepackage[caption=false, font=footnotesize]{subfig}
\fi
\usepackage{graphicx}
\usepackage[letterpaper, left=0.625in, right=0.625in, bottom=1in, top=0.75in]{geometry}

\newcommand{\vect}[1]{\boldsymbol{#1}}

\usepackage{tikz}
\usepackage{amsthm}
\newtheorem{thm}{Theorem}
\newtheorem{lemma}[thm]{Lemma}
\newtheorem{pf}{Proof}
\newtheorem{defn}[thm]{Definition}
\newtheorem{problem}[thm]{Problem}
\newtheorem{conjecture}[thm]{Conjecture}
\newtheorem{proposition}[thm]{Proposition}


\renewcommand{\H} {{\bf{H}}}
\newcommand{\A} {{\bf{A}}}
\newcommand{\B} {{\bf{B}}}
\newcommand{\F} {{\bf{F}}}
\newcommand{\D} {{\bf{D}}}
\newcommand{\I} {{\bf{I}}}
\renewcommand{\b} {{\bf{b}}}
\renewcommand{\c} {{\bf{c}}}
\newcommand{\g} {{\bf{g}}}
\newcommand{\f} {{\bf{f}}}
\renewcommand{\v} {{\bf{v}}}
\newcommand{\e} {{\bf{e}}}
\newcommand{\h} {{\bf{h}}}
\renewcommand{\a} {{\bf{a}}}
\renewcommand{\r} {{\bf{r}}}
\renewcommand{\d} {{\bf{d}}}
\newcommand{\s} {{\bf{s}}}
\newcommand{\y} {{\bf{y}}}
\newcommand{\x} {{\bf{x}}}
\newcommand{\z} {{\bf{z}}}
\def\E{\mathbb E}

\newcommand{\nariman}[1]{\textcolor{red}{#1}}
%\newcommand{\amir}[1]{\textcolor{blue}{#1}}
\newcommand{\amir}[1]{\textcolor{blue}{#1}}


% author names and affiliations
% use a multiple column layout for up to three different
% affiliations
\author{
\IEEEauthorblockN{Nariman Torkzaban}
\IEEEauthorblockA{\textit{University of Maryland, College Park}\\
College Park, MD\\
narimant@umd.edu}
\and
\IEEEauthorblockN{Mohammad A. (Amir) Khojastepour}
\IEEEauthorblockA{\textit{NEC Laboratories,
America}\\
Princeton, NJ\\
amir@nec-labs.com}
% \and
% \IEEEauthorblockN{Nariman Torkzaban}
% \IEEEauthorblockA{\textit{University of Maryland, College Park}\\
% College Park, MD\\
% narimant@umd.edu}
}


% use for special paper notices
%\IEEEspecialpapernotice{(Invited Paper)}













\def\BibTeX{{\rm B\kern-.05em{\sc i\kern-.025em b}\kern-.08em
    T\kern-.1667em\lower.7ex\hbox{E}\kern-.125emX}}

\newcommand\copyrighttext{%
  \footnotesize \textcopyright 2021 IEEE. Personal use of this material is permitted.
  Permission from IEEE must be obtained for all other uses, in any current or future
  media, including reprinting/republishing this material for advertising or promotional
  purposes, creating new collective works, for resale or redistribution to servers or
  lists, or reuse of any copyrighted component of this work in other works.
%  DOI: \href{<http://tex.stackexchange.com>}{<DOI No.>}
  }
\newcommand\copyrightnotice{%
\begin{tikzpicture}[remember picture,overlay]
\node[anchor=south,yshift=10pt] at (current page.south) {\fbox{\parbox{\dimexpr\textwidth-\fboxsep-\fboxrule\relax}{\copyrighttext}}};
\end{tikzpicture}%
}
\begin{document}

\title{Shaping mmWave Wireless Channel via Multi-Beam Design using Reconfigurable Intelligent Surfaces}

%\title{RIS-enabled Coverage by Dual-beam design in mmWave communication}
%\title{Shaping mmWave Wireless Channel via Composite Beamforming Design}
%\title{RIS Discussion: Dual-beam design for covering blind spots in mmWave communication using RIS }
%\title{Virtualized Network Function Placement\\for Cellular Network Slicing}



% author names and affiliations
% use a multiple column layout for up to three different
% affiliations



% use for special paper notices
%\IEEEspecialpapernotice{(Invited Paper)}




% make the title area
\maketitle
\copyrightnotice
% As a general rule, do not put math, special symbols or citations
% in the abstract
\begin{abstract}
Millimeter-wave (mmWave) communications is considered as a key enabler towards the realization of next-generation wireless networks, due to the abundance of available spectrum at mmWave frequencies. However, mmWave suffers from high free-space path-loss and poor scattering resulting in mostly line-of-sight (LoS) channels which result in a lack of coverage. 
Reconfigurable intelligent surfaces (RIS), as a new paradigm, have the potential to fill the coverage holes by shaping the wireless channel. 
In this paper, we propose a novel approach for designing RIS with elements arranged in a uniform planar array (UPA) structure. In what we refer to as multi-beamforming, We propose and design RIS such that the reflected beam comprises multiple disjoint lobes. Moreover, the beams have optimized gain within the desired angular coverage with fairly sharp edges 
avoiding power leakage to other regions. We provide a closed-form low-complexity solution for the multi-beamforming design. We confirm our theoretical results by numerical analysis.  

%In this paper, we study the codebook design problem for hybrid beamforming. Considering a uniform linear array (ULA) of antennas, We first propose the structure of optimal corresponding beamformers. We show the ULA structure suffers from intrinsic inefficiencies limiting its performance in practical scenarios. To deal with these inefficiencies, we propose for the first time a twin ULA (TULA) structure of antennas that not only solves all the problems related to ULA, but also provides higher gains. We further extend our findings to develop more sophisticated antenna structures to boost the system performance. We both discuss the theory behind our approach, and show the effectiveness of our proposed codebooks by means of extensive simulations. 

%With the emergence of network slicing as a key enabler for the multitenancy paradigm, the corresponding resource allocation problems become of paramount importance. 


%Leveraging on Network Function Virtualization (NFV) and Software Defined Networking (SDN), we introduce exact methods for LTE network function (NF) placement. Towards this end we: (i) jointly optimize the virtualized RAN and EPC functions placement; and (ii) investigate trade-offs between optimizations tailored to different operator resource allocation policies.

\end{abstract}




\begin{IEEEkeywords}
 Beamforming, Reconfigurable Intelligent Surface (RIS), Uniform Planar Array (UPA), Blind-spot, MIMO
\end{IEEEkeywords}
% no keywords




% For peer review papers, you can put extra information on the cover
% page as needed:
% \ifCLASSOPTIONpeerreview
% \begin{center} \bfseries EDICS Category: 3-BBND \end{center}
% \fi
%
% For peerreview papers, this IEEEtran command inserts a page break and
% creates the second title. It will be ignored for other modes.
\IEEEpeerreviewmaketitle

\section{Introduction}


Next generation of wireless communication systems aims to address the ever-increasing demand for high throughput, low latency, better quality of service and ubiquitous coverage. The abundance of bandwidth available at the mmWave frequency range, i.e., $[20, 100]$ Ghz, is considered as a key enabler towards the realization of the promises of next generation wireless communication systems. However, communication in mmWave suffers from high path-loss, and poor scattering. Since the channel in mmWave is mostly LoS, i.e., a strong LoS path and very few and much weaker secondary components, the mmWave coverage map includes \emph{blind spots} as a result of shadowing and blockage. Beamforming is primarily used to address the high attenuation in the channel. In addition to beamforming, relaying can potentially be designed to generate constructive superposition and enhance the received signals at the receiving nodes. 
% Importance of mmWave in 5G and beyond systems.
% Channel in mmWave is mostly LOS. Blockage and shadowing generate blind spots.
% High attentuation in mmwave should be addressed. In addition to beamforming, relaying can generate constructive superposition and enhance the received signals for the users.
% Contyributions are as follows.
Reconfigurable intelligent surface (RIS)\cite{Huang19}\cite{Liaskos18}\cite{Basar19} is a new paradigm with a great potential for stretching the coverage and enhancing the capacity of next-generation communication systems. Indeed, it is possible to shape the wireless channel by using RIS, e.g., by covering blind spots or providing diversity reception at a receiving node. In particular, passive RIS provide not only an energy-efficient solution but also a cost-effective one both in terms of the initial deployment cost and the operational costs. RIS are promising to be deployed in a wide range of communications scenarios and use-cases, such as high throughput MIMO communications\cite{Huang20}\cite{Nadeem20}, ad-hoc networks, e.g., UAV communications\cite{Li20}, physical layer security\cite{Maka20}, etc. % A comprehensive survey is conducted in \cite{Liu21} that \amir{enumerates} the fundamentals, opportunities and challenges of integrating RIS into wireless communication environments. 
Apart from the works focusing on theoretical performance analysis of RIS-enabled systems  \cite{Han19}\cite{Nadeem20}\cite{Jung19}, considerable amount of work has been dedicated to optimizing such an integration, mostly focusing on the phase optimization of RIS elements \cite{Abey20}\cite{Guo20}\cite{Di20}\cite{Ata20} to achieve various goals such as maximum received signal strength, maximum spectral efficiency,  etc. For more information on RIS, we refer the interested readers to \cite{Liu21} and the references therein.


% Importance of RIS in shaping the wireless channel, generating opportunities for coherent combining through relay, enhancing the coverage by acting like a deformable mirror to cover blind spots.

% In this work we provide novel approach to design RIS such that we can simultaneously optimize the received power at RIS and the propagated (reflected) signal off of RIS in arbitrary 3D (Asimuth and elevation) angles. The dual-beam design allows for optimizing both receive and transmit direction. Moreover, it allows to control the beamwidth and hence the gain of the receive and transmit beamforming vectors. The work can be easily extended to cover multiple transmit and receive directions in multi-beam instead of dual-beam.

\begin{figure}
    \centering
    \includegraphics[width=0.75\linewidth]{figures/scene2.eps}
    \caption{Filling the coverage gap in mmWave communications by utilizing Reconfigurable Intelligent Surfaces enabled by multi-beamforming }
    \label{fig:system}
\end{figure}

In this paper, we consider a communication scenario between a transmitter, e.g., the base station (BS), and terrestrial end-users through a passive RIS that reflects the received signal from the transmitter towards the users. Hence, the users that are otherwise in blind spots of network coverage, become capable of communicating with the base station through the RIS that is serving as a passive reflector (passive relay) maintaining  communication links to the BS and to the users. Given the geo-spatial variance among the locations of the end-users served by the same wireless system, the RIS may have to accommodate users that lie in distant angular intervals simultaneously, with satisfactory quality of service (QoS). In what we refer to as \emph{multi-beamforming}, we particularly address the design of beams consisting of multiple disjoint lobes using RIS in order to cover different blind spots using sharp and effective beam patterns. In the following, we summaries the main contributions of this paper: 
%The contributions of this paper are as follows:
\begin{itemize} 
    \item We design the parameters of the RIS to achieve multiple disjoint beams covering various ranges of solid angle. The designed beams are fairly sharp, have almost uniform gain in the desired angular coverage interval (ACI), and have negligible power transmitted outside the ACI.
    \item We formulate the multi-beamforming design as an optimization problem for which we derive the optimal solution.
    \item Thanks to the derived analytical closed form solutions for the optimal multi-beamforming design, the proposed solution bears very low computational complexity even for RIS with massive array size.
%    \amir{\item (iv) The proposed design can be used with arrays that are passive (whether it is only phased-controlled or both phase-and gain controlled) as well as RIS with active elements (that are capable of amplifying the reflected signal).}
%    \item (v) We identify the fact that if a passive RIS is capable of controlling the gain of its element (e.g., through attenuation), it can provide smoother gain in the desired ACI and also boost up the beamforming gain. The latter could be counter intuitive as the power radiated from every single element is not maximized due to attenuation, but in fact, to shape the beams it is essential to use controlled attenuation for signal reflected from different array elements of RIS in the superposition of the signals emitted by each RIS elements.
%    \item (vi) The multi-beamforming design inherently depends on the solid angle (say $\Omega_1$ in Fig.~\ref{fig:RIS}) at which the incident wave activates the RIS elements. The proposed beamforming design easily adapts to changes in $\Omega_1$ and we provide a visualization as how the beam would change in response to change in $\Omega_1$.
    \item Through numerical evaluation we show that by using passive RIS, multi-beamforming can simultaneously cover multiple ACIs. Moreover, multi-beamforming provides tens of dB power boost w.r.t. single-beam RIS design even when the single beam is designed optimally. 
\end{itemize}

\textbf{Notation} Throughout this paper, $\mathbb{C}$, $\mathbb{R}$, and $\mathbb{Z}$ denote the set of complex, real, and integer numbers, respectively,  $\mathcal{C N}\left(m, \sigma^{2}\right)$ denotes the circularly symmetric complex normal distribution with mean $m$ and variance $\sigma^{2}$, $[a, b]$ is the closed interval between $a$ and $b, \mathbf{1}_{a, b}$ is the $a \times b$ all ones matrix, $\mathbf{I}_{N}$ is the $N \times N$ identity matrix, $\mathds{1}_{[a, b)}$ is the indicator function, $\|\cdot\|$ is the $2$ -norm, $\|\cdot\|_{\infty}$ is the infinity-norm, $|\cdot|$ may denote cardinality if applied to a set and $1$-norm if applied to a vector, $\odot$ is the Hadamard product, $\otimes$ is the Kronecker product, $\mathbf{A}^{H},$ and $\mathbf{A}_{a, b}$ denote conjugate transpose, and $(a, b)^{t h}$ entry of $\A$ respectively.





The remainder of the paper is organized as follows. Section~\ref{sec:desc} describes the system model. In Section~\ref{sec:problem} we formulate the multi-beamforming design problem and propose our solutions in Section ~\ref{sec:proposed}. Section ~\ref{sec:evaluation} presents our evaluation results, and finally, we conclude in Section ~\ref{sec:conclusions}. %, we highlight our conclusions and discuss directions for future 


%In dual beam design, we jointly optimize a receive beam steering (beamforming) covering the 3D angular location of the base stations and a transmit beam steering covering the 3D angular location of the user (users) of interest.

%Shaping wireless channel: through relaying or covering the blind spots.



\section{System model} 
\label{sec:desc}
\begin{figure}
    \centering
%    \includegraphics[width=\linewidth]{figures/RIS_Structure.jpg}
    \includegraphics[width=0.8\linewidth]{figures/RIS_Structure.eps}
    \caption{System Model}
    \label{fig:RIS}
\end{figure}


\subsection{Channel Model}
Consider a communication system consisting of a multi-antenna BS with $M_t$ antenna elements as a transmitter and a multi-antenna receiver with $M_r$ antenna elements. The MIMO system is aided by a multi-element RIS consisting of $M$ elements arranged in $M_h \times M_v$ grid in the form of UPA  as shown in figure~\ref{fig:system} where $M_h$ and $M_v$ are the number of elements in the horizontal and vertical directions, respectively. The received signal $\y \in \mathbb{C}^{M_r}$ as a function of the transmitted signal $\x \in \mathbb{C}^{M_t}$ can be written as,
\begin{align}
    \y = (\H_r\boldsymbol\Theta\H_t)\x + \z \label{channel}
\end{align}

\noindent where $\z$ is the noise vector, with each element of $\z$ drawn from a complex Gaussian distribution $\mathcal{C N}\left(0, \sigma_n^{2}\right)$, $\H_t \in \mathbb{C}^{M\times M_t}$ and $\H_r \in \mathbb{C}^{M_r\times M}$ are the channel matrices between each party and the RIS. We assume that the RIS consists of elements for which both the phase $\theta_m$ and the gain $\beta_m$ (in form of attenuation of the reflected signal) of each element, say $m$, may be controlled and $\boldsymbol\Theta \in \mathbb{C}^{M\times M}$ is a diagonal matrix where the element $(m,m)$ denotes the coefficient $\beta_m e^{j \theta_m}$ of the $m^{th}$ element of the RIS. Assuming LoS channel model both between the transmitter and the RIS and between the RIS and the receiver and using the directivity vectors at the transmitter, the RIS, and the receiver, the effective channel matrices can be written as,
% \begin{align}
%     & \H_t = \d_{M_t}\{\Omega_t\}\rho_{t}\d^{H}_{M}\{\Omega_{1}\}\label{channel_t} \\
%     & \H_r = \d_{M}\{\Omega_{2}\}\rho_{ r}\d^{H}_{M_r}\{\Omega_{r}\} \label{channel_r}
% \end{align}
\begin{align}
    & \H_r = \a_{M_r}(\Omega_r)\rho_{r}\a^{H}_{M}(\Omega_{2})\label{channel_r} \\
    & \H_t = \a_{M}(\Omega_{1})\rho_{t}\a^{H}_{M_t}(\Omega_{t}) \label{channel_t}
\end{align}

\noindent
% where $\Omega_t$ and $\Omega_2$ are the solid angle of departure (AoD) of the transmitted beams from transmitter and the RIS and $\Omega_1$ and $\Omega_r$ are the solid angle of arrival (AoA) of the received beams at the RIS and the receiver, respectively. 
where $\a_M(\Omega)$ is the array response vector of an RIS with elements in a UPA structure (RIS-UPA), $\Omega_t$ and $\Omega_2$ are the solid angles of departure (AoD) of the transmitted beams from transmitter and the RIS and $\Omega_1$ and $\Omega_r$ are the solid angle of arrival (AoA) of the received beams at the RIS and the receiver, respectively. 
% The directivity RIS-UPA can be found in similar way to that of a UPA. For a solid angle $\Omega = (\phi, \theta)$ where $-\frac{\pi}{2} \leq \phi \leq \frac{\pi}{2}$ is the elevation angle and $-\pi \leq \theta \leq \pi$ is the azimuth angle, the gain of a UPA with $M$ elements is given by
% \begin{align}
%     G (\c, \Omega) = \left| \sum_{m=0}^{M_t-1} c_{m} e^{j \frac{2 \pi}{\lambda}[ \cos \phi \cos\theta, \cos\phi \sin \theta, \sin\phi]  \r_m} \right|^2  \label{init_gain} 
% \end{align}
% where the element $m$ is located at $\r_m$ and is excited by coefficients $c_m$. Hence, The directivity vector for an UPA at the solid angle $\Omega$ is defined as
% \begin{align}
%     \a_{M}( \Omega ) = \left[1, e^{j \frac{2 \pi}{\lambda}[ \cos \phi \cos\theta, \cos\phi \sin \theta, \sin\phi]  \r_1}, \ldots, \right. \nonumber \\
%     \left. e^{j \frac{2 \pi}{\lambda}[ \cos \phi \cos\theta, \cos\phi \sin \theta, \sin\phi]  \r_{M-1}} \right]^{T} \in \mathbb{C}^{M}
% \end{align}
% \amir{Please note that an array response vector has similar gain in the same direction in three dimensional space. However, the receive and transmit directions differ by $\pi$, hence, the array response vector for the transmitter and receiver are written as $\a^{H}_{M}(\Omega)$, and $\a_{M}(\Omega)$, respectively.  
% }
The gain of the LoS paths from the transmitter to the RIS and from the RIS to the receiver are denoted by $\rho_t$ and $\rho_r$, respectively. Note that the solid angle $\Omega_a$ specifies a pair of elevation and azimuth angles i.e. $(\phi_a, \theta_a)$, $a \in \{1,2, t, r\}$. Further, assuming no pairing between the RIS elements, $\boldsymbol\Theta$ will be a diagonal matrix specified as 
\begin{equation}
    \boldsymbol\Theta = \mbox{diag}\{[\beta_1 e^{j \theta_1}, \ldots, \beta_M e^{j \theta_M}]\}
\end{equation}

\noindent where $\beta_i \in [0,1]$ and $\theta_i \in [0, 2\pi]$.


%where $\Omega_t$, $\Omega_1$, $\Omega_2$, and $\Omega_r$ are the angle between the direction of the transmitted or received beam with respect to their respective antenna array or RIS. 



% Define 
% \begin{equation}
%     \h_i = [1, \alpha, \alpha^2, \ldots, \alpha^{(M-1)}]^T, \alpha = e^{\frac{-j2 \pi d \phi_i}{\lambda}}, i = 1,2
% \end{equation}
% \begin{equation}
%     \h_G = [1, \alpha, \alpha^2, \ldots, \alpha^{(M_G-1)}]^T, \alpha = e^{\frac{-j2 \pi d_G \phi_G}{\lambda}}
% \end{equation}
% \begin{equation}
%     \h_U = [1, \alpha, \alpha^2, \ldots, \alpha^{(M_U-1)}]^T, \alpha = e^{\frac{-j2 \pi d_U \phi_U}{\lambda}}
% \end{equation}
% \begin{equation}
%     \Theta = \mbox{diag}\{[\beta_1 e^{j \theta_1}, \ldots, \beta_M e^{j \theta_M}]\}
% \end{equation}


% The received signal $\y$ at G as a function of the transmitted signal $\x$ from U is given by

% \begin{equation}
%     \y = \h_G  \rho_G \h_2^H \Theta \h_1  \rho_U \h_U^H
%  %       \y = \h_G \otimes \rho_G \h_2^H \Theta \h_1 \otimes \rho_U \h_U^H
% \end{equation}



\subsection{RIS Model}

Suppose an RIS consisting of $M_h \times M_v$ antenna elements forming a UPA structure is placed at the $x$-$z$ plane, where $M = M_h M_v$ and $z$ axis corresponds to horizon. Let $d_z$, and $d_x$ denote the distance between the antennas elements in $z$ and $x$ axis, respectively. 
The directivity of a RIS-UPA can be found in similar way to that of a UPA. At a solid angle $\Omega = (\phi, \theta)$, we have,
\begin{align}
    \a_{M}( \Omega ) = \left[1, e^{j \frac{2 \pi}{\lambda}\r_{\Omega}  \r_1}, \ldots,  e^{j \frac{2 \pi}{\lambda}\r_{\Omega}  \r_{M-1}} \right]^{T} \in \mathbb{C}^{M} \label{directivity_first}
\end{align}
\noindent where respectively,  $\r_{\Omega} = [ \cos \phi \cos\theta, \cos\phi \sin \theta, \sin\phi]$, and $\r_m = (m_hd_x, 0, m_vd_z)$  denote the direction corresponding to the solid angle $\Omega$ and the location of the $m$-th RIS element corresponding to the antenna placed at the position $(m_v, m_h)$. Further, we define a transformation of variables as follows. For a solid angle $\Omega = [\phi, \theta]$, define $\psi = [\xi, \zeta]$ as follows,
\begin{align}
    \xi=\frac{2 \pi d_{z}}{\lambda} \sin \phi \text {,  }\quad \zeta=\frac{2 \pi d_{x}}{\lambda} \sin \theta \cos \phi \label{transformation}
\end{align} Introducing the new variables into equation~\eqref{directivity_first}, it is straightforward to write, 
\begin{align}
    \a_M(\Omega) = \mathbf{d}_{M}\left(\xi, \zeta\right) =
    \mathbf{d}_{M_{v}}\left(\xi\right) \otimes
    \mathbf{d}_{M_{h}}\left(\zeta\right)  \in \mathbb{C}^{M}
\end{align}

where we define for $a \in \{v,h\} $ the  directivity vectors $\d_{M_a}$ as follows, and denote by $\d_M$ the directivity vector corresponding to the RIS. 
% \begin{align}
% \mathbf{d}_{M_{a}}\left(\psi_{a}\right) = \left[1, e^{j \psi_{a}} \cdots e^{j\left(M_{a}-1\right) \psi_{a}}\right]^{T} \in \mathbb{C}^{M_{a}}
% \end{align}
\begin{align}
&\mathbf{d}_{M_{v}}\left(\xi\right) = \left[1, e^{j \xi} \cdots e^{j\left(M_{v}-1\right) \xi}\right]^{T} \in \mathbb{C}^{M_{v}}\nonumber\\
&\mathbf{d}_{M_{h}}\left(\zeta\right) = \left[1, e^{j \zeta} \cdots e^{j\left(M_{h}-1\right) \zeta}\right]^{T} \in \mathbb{C}^{M_{h}}\label{directivity_final}
\end{align}

% The array response vector, i.e., directivity vector, of the RIS for a solid angle $\Omega$, i.e., $\d_M\{\Omega\}$ as a function of $\xi$ and $\zeta$ is denoted by $\mathbf{d}_{M}\left(\xi, \zeta\right)$ and is given by 

% \begin{align}
%     \mathbf{d}_{M}\left(\xi, \zeta\right) =
%     \mathbf{d}_{M_{v}}\left(\xi\right) \otimes
%     \mathbf{d}_{M_{h}}\left(\zeta\right)  \in \mathbb{C}^{M}
% \end{align}
% where, $\zeta=\frac{2 \pi d_{x}}{\lambda} \sin \theta \cos \phi \text { and } \xi=\frac{2 \pi d_{z}}{\lambda} \sin \phi$ for a solid angle $\Omega = (\phi, \psi)$. 

Let $\mathcal{B}$ be the angular range under cover defined as

\begin{equation}
    \mathcal{B} \doteq \left[-\phi^{\mathrm{B}}, \phi^{\mathrm{B}}\right)
    \times \left[-\theta^{\mathrm{B}}, \theta^{\mathrm{B}}\right) \label{range_angle}
\end{equation}

We note that there is a one-to-one correspondence between the solid angle $\Omega = (\phi, \psi)$ and its representation after change of variable as $(\zeta, \xi)$. Accordingly, let $\mathcal{B}^{\psi}$ be the angular range under cover in the $(\zeta, \xi)$ domain given by 

\begin{equation}
    \mathcal{B}^\psi \doteq\left[-\xi^{\mathrm{B}}, \xi^{\mathrm{B}}\right) \times\left[-\zeta^{\mathrm{B}}, \zeta^{\mathrm{B}}\right)
\end{equation}
In this paper, we set $d_x = d_z = \frac{\lambda}{2}$, $\phi^{\mathrm{B}} = \frac{\pi}{4}$, and $\theta^{\mathrm{B}} = \frac{\pi}{2}$, hence $\xi \in [-\pi\frac{\sqrt{2}}{2}, \pi\frac{\sqrt{2}}{2})$, and $\zeta \in [-\pi, \pi)$. Note that, the dependence between variables $\xi$ and $\zeta$ can be resolved using the approximation in \cite{Song17}.
Let us uniformly divide $\mathcal{B}^{\psi}$  into $Q=Q_{v} Q_{h}$ subregions, where $Q_h$ and $Q_v$ are the number of division in horizontal and vertical directions, respectively. A subregion is denoted by  
$$ \mathcal{B}^{\psi}_{ p, q} \doteq \nu_{v}^{p, q} \times \nu_{h}^ {p, q}$$
\noindent where $\nu_{v}^{p} = [\xi^{p-1}, \xi^{p}]$, and $\nu_{h}^{q} = [\zeta^{q-1}, \zeta^{q}]$ defining,
\begin{align}
    & \xi^{p} = -\xi^{\mathrm{B}} + p\delta_v, \quad \zeta^{q} = -\zeta^{\mathrm{B}} + q\delta_h
\end{align}
%Note that, the dependence between variables $\xi$ and $\zeta$ can be resolved using the approximation in \cite{Song17}. 
\noindent with $\delta_v = \frac{2\xi^{\mathrm{B}}}{Q_v}$, and $\delta_h = \frac{2\zeta^{\mathrm{B}}}{Q_h}$. In the next section, we define the multi-beamforming design problem as the core of our proposed RIS structure. 


% One can define for the solid angle $\Omega = (\phi, \psi)$,

% \begin{align}
% \mathbf{d}_{M_{a}}\left(\psi_{a}\right) = \left[1, e^{j \psi_{a}} \cdots e^{j\left(M_{a}-1\right) \psi_{a}}\right]^{T} \in \mathbb{C}^{M_{a}}
% \end{align}
% where, $\zeta=\frac{2 \pi d_{x}}{\lambda} \sin \theta \cos \phi \text { and } \xi=\frac{2 \pi d_{z}}{\lambda} \sin \phi$, and $a\in\{v, h\}$.  
% The array response vector is then defined as 

% \begin{align}
%     \mathbf{d}_{M}\left(\xi, \zeta\right) =
%     \mathbf{d}_{M_{v}}\left(\xi\right) \otimes
%     \mathbf{d}_{M_{h}}\left(\zeta\right)  \in \mathbb{C}^{M}
% \end{align}

\section{Problem Formulation}
\label{sec:problem}

% \subsection{Composite Codebook Design Problem}

Let us define a \textit{composite beam} $\omega_k$ as a union of multiple disjoint, possibly non-neighboring beams $\nu_{q} \text { for } q \in \mathcal{Q} = \left\{1, \cdots, Q\right\}$. Let $\mathcal{C'}=\left\{\mathbf{c}_{1}, \cdots, \mathbf{c}_{K}\right\}$ be the codebook corresponding to the composite problem.

% $$\bigcup_{k = 1}^{2^{B'}}{\omega_k} = [-\pi, \pi], \quad \text{and} \quad \omega_k \cap \omega_l = \emptyset, \quad \forall k\neq l.$$

Moreover, define for each composite beam $\omega_k$ the set $\mathcal{W}_k \subseteq \mathcal{Q}$ to be the set of indices of the single beams that form $\omega_k$. i.e. $\mathcal{W}_k = \{q \in \mathcal{Q}: v_q \subseteq \omega_k\}$. 
For any codeword $\c$, it holds that, 

\begin{equation}
    \int_{-\pi}^{\pi} G(\psi, \mathbf{c}) d \psi=2 \pi\|\mathbf{c}\|^{2}=2 \pi
\end{equation}
We have then for the ideal gain corresponding to  each codeword $\textbf{c}_k$, 

\begin{align}
&\int_{-\pi}^{\pi} G_{\text {ideal }, k}(\psi) d \psi =\int_{\omega_{k}} t d \psi+\int_{[-\pi, \pi] \backslash \omega_{k}} 0 d \psi \nonumber\\
&= \sum_{q \in {\mathcal{W}_k}}{\int_{\nu_{q}} t d \psi} = \sum_{q \in \mathcal{W}_k}\delta_q t=2 \pi \label{composite}
\end{align}

Therefore, $t = \frac{2 \pi}{\Delta_k}$, where $\Delta_k = \sum_{q \in \mathcal{W}_k}\delta_q$. It follows that, 

\begin{equation}
    G_{\text {ideal }, k}(\psi)=\frac{2 \pi}{\Delta_k} \mathds{1}_{\omega_{k}}(\psi), \quad \psi \in[-\pi, \pi] \label{composite_ideal_gain}
\end{equation}

The new MSE problem for the $k$-th codeword can therefore be written as follows.



\begin{align}
\c_k^{opt} &=\underset{\c, \|\c\|=1}{\arg \min } \int_{-\pi}^{\pi}\left|G_{\text {ideal }, k}(\psi)-G(\psi, \c)\right| d \psi 
\end{align}




By uniformly sampling on the range of   we can rewrite
the optimization problem as follows,

\begin{align}
    \c_k^{opt} &= \nonumber \\&\underset{\c, \|\c\|=1}{\arg \min }\left[\lim _{L \rightarrow \infty} \sum_{p=1}^{Q} \delta_p\sum_{\ell=1}^{L} \frac{\left|G_{\text {ideal }, k}\left(\psi_{p, \ell}\right)-G\left(\psi_{p, \ell}, \c\right)\right|}{L}\right]\label{composite_MSE}
\end{align}

 where, for $q = 1 \ldots Q$, 
 $$\delta_q = \psi_{q}-\psi_{q-1}, \quad \psi_{q, \ell}=\psi_{q-1}+\frac{\delta_q(\ell-0.5)}{L} $$

We can rewrite equation \eqref{composite_MSE} as follows, 

\begin{align}
 &\c_k^{opt} = \underset{\c,  \|\c\|=1}{\arg \min } \lim_{L\rightarrow \infty}\frac{1}{L}\left\|\mathbf{G}_{\text {ideal }, k}-\mathbf{G}(\c)\right\| \label{init_opt}
 \end{align}
 
 where,  $$\mathbf{G}(\c)=\left[{\delta_1} G\left(\psi_{1,1}, \c\right) \ldots {\delta_{Q}}G\left(\psi_{Q, L}, \c\right)\right]^{T} \in \mathbb{Z}^{L Q}$$

and, 
$$ \mathbf{G}_{\text {ideal }, k}=\left[{\delta_1} G_{\text {ideal }, k}\left(\psi_{1,1}\right) \ldots {\delta_{Q}}G_{\text {ideal }, k}\left(\psi_{Q, L}\right)\right]^{T} \in \mathbb{Z}^{L Q}$$


Note that it holds that 

\begin{equation}
    \mathbf{G}_{\text {ideal },k}=\sum_{q \in \mathcal{W}_k}\delta_q\frac{2\pi}{\Delta_k}\left(\mathbf{e}_{q} \otimes \mathbf{1}_{L, 1}\right) \label{ideal}
\end{equation}

with $\mathbf{e}_{q} \in \mathbb{Z}^{Q}$ being the standard basis vector for the $q$-th axis among $Q$ ones. Now, note that $\mathbf{1}_{L, 1}=\mathbf{g} \odot \mathbf{g}^{*}$ for any equal gain $\mathbf{g} \in \mathbb{C}^L$. Therefore, for a suitable choice of $\g$ we can write: 

% $$\mathcal{G}_{L}=\left\{\mathbf{g} \in \mathbb{C}^{L}:\left(\mathbf{g} \mathbf{g}^{H}\right)_{\ell, \ell}=1,1 \leq \ell \leq L\right\}$$


\begin{align}
\mathbf{G}_{\text {ideal }, k} &= \sum_{q \in \mathcal{W}_k}\gamma_q\left(\mathbf{e}_{q} \otimes\left(\mathbf{g} \odot \mathbf{g}^{*}\right)\right) \nonumber\\
&=\sum_{q \in \mathcal{W}_k}\left(\sqrt{\gamma_q}\left(\mathbf{e}_{q} \otimes \mathbf{g}\right)\right) \odot\left(\sqrt{{\gamma_q}}\left(\mathbf{e}_{q} \otimes \mathbf{g}\right)\right)^{*} \nonumber\\
&=\left(\sum_{q \in \mathcal{W}_k}\sqrt{\gamma_q}\left(\mathbf{e}_{q} \otimes \mathbf{g}_q\right)\right)  \odot \left(\sum_{q \in \mathcal{W}_k}\sqrt{\gamma_q}\left(\mathbf{e}_{q} \otimes \mathbf{g}_q\right)\right)^{*} \label{final_gik}
\end{align}




















% \begin{align}
% \left(\mathbf{F}_{k}, \mathbf{v}_{k}\right) &=\underset{\mathbf{F}, \mathbf{v}}{\arg \min } \sum_{p=1}^{Q} \sum_{\ell=1}^{L}\left|G_{\text {ideal }, k}\left(\psi_{p, \ell}\right)-G\left(\psi_{p, \ell}, \mathbf{F} \mathbf{v}\right)\right|^{2} \nonumber\\
% &=\underset{\mathbf{F}, \mathbf{v}}{\arg \min }\lim_{L\longrightarrow \infty} \frac{1}{L}\left\|\mathbf{G}_{\text {ideal }, k}-\mathbf{G}(\mathbf{F} \mathbf{v})\right\|_{2}^{2} \label{init_opt}
% \end{align}

% where,  $$\mathbf{G}(\mathbf{F} \mathbf{v})=\left[G\left(\psi_{1,1}, \mathbf{F} \mathbf{v}\right) \cdots G\left(\psi_{Q, L}, \mathbf{F} \mathbf{v}\right)\right]^{T} \in \mathbb{Z}^{L Q}$$

% and, 
% $$ \mathbf{G}_{\text {ideal }, k}=\left[G_{\text {ideal }, k}\left(\psi_{1,1}\right) \cdots G_{\text {ideal }, k}\left(\psi_{Q, L}\right)\right]^{T} \in \mathbb{Z}^{L Q}$$

% Note that we can write using the definition of $\mathcal{W}_k$ that, 

% $$G_{\text {ideal }, k}\left(\psi_{p, \ell}\right)=\left\{\begin{array}{ll}
% \frac{Q}{|\mathcal{W}_k| M_{t}}, & p \in \mathcal{W}_k \\
% 0, & p \notin \mathcal{W}_k 
% \end{array},\right.$$

% Or in the matrix form, 

% \begin{align}
%     &\mathbf{G}_{\text {ideal }, k}=\frac{Q}{|\mathcal{W}_k| M_{t}}\left((\bigoplus_{q \in \mathcal{W}_k}{\mathbf{e}_{q}}) \otimes \mathbf{1}_{L, 1}\right)  \nonumber \\ 
%     & = \sum_{q \in \mathcal{W}_k}{\frac{Q}{|\mathcal{W}_k| M_{t}}\left(\mathbf{e}_{q} \otimes \mathbf{1}_{L, 1}\right)} \label{G_ideal_k}
% \end{align}

% with $\mathbf{e}_{q} \in \mathbb{Z}^{Q}$ being the standard basis vector for the $q$-th axis among $Q$ ones. Now, note that $\mathbf{1}_{L, 1}=\mathbf{g} \odot \mathbf{g}^{*}$ for any $\mathbf{g} \in \mathcal{G}_L$, where, 

% $$\mathcal{G}_{L}=\left\{\mathbf{g} \in \mathbb{C}^{L}:\left(\mathbf{g} \mathbf{g}^{H}\right)_{\ell, \ell}=1,1 \leq \ell \leq L\right\}$$



% Therefore, equation \eqref{G_ideal_k} can be rewritten as:
% \begin{align}
% &\mathbf{G}_{\text {ideal }, k} = \sum_{q \in \mathcal{W}_k}\frac{Q}{|\mathcal{W}_k| M_{t}}\left(\mathbf{e}_{q} \otimes\left(\mathbf{g}_q \odot \mathbf{g}_q^{*}\right)\right) \nonumber\\
% &=\sum_{q \in \mathcal{W}_k}\left(\sigma\left(\mathbf{e}_{q} \otimes \mathbf{g}_q\right)\right) \odot\left(\sigma\left(\mathbf{e}_{q} \otimes \mathbf{g}_q\right)\right)^{*} \nonumber\\
% &=\left(\sum_{q \in \mathcal{W}_k}\sigma\left(\mathbf{e}_{q} \otimes \mathbf{g}_q\right)\right)  \odot \left(\sum_{q \in \mathcal{W}_k}\sigma\left(\mathbf{e}_{q} \otimes \mathbf{g}_q\right)\right)^{*} \label{final_gik}
% \end{align}

% where $\sigma = \sqrt{\frac{Q}{|\mathcal{W}_k|M_{t}}}$. 

Similarly, it is straightforward to observe,


\begin{align}
\mathbf{G}(\c) &=\left(\mathbf{D}^{H} \c\right) \odot\left(\mathbf{D}^{H} \c\right)^{*} \label{dc}
\end{align}
where $\mathbf{D} =\left[\sqrt{\delta_1}\mathbf{D}_{1} \cdots \sqrt{\delta_{Q}}\mathbf{D}_{Q} \right] \in \mathbb{C}^{M_{t} \times L Q}$, and 
$\mathbf{D}_{q}=\left[\mathbf{d}_{M_{t}}\left(\psi_{q, 1}\right) \cdots \mathbf{d}_{M_{t}}\left(\psi_{q, L}\right)\right] \in \mathbb{C}^{M_{t} \times L}.$
 







% On the other hand note that $\mathbf{G}(\mathbf{F} \mathbf{v})$ can be written as,


% \begin{equation}
%     \mathbf{G}(\mathbf{F} \mathbf{v}) =\left(\mathbf{D}^{H} \mathbf{F} \mathbf{v}\right) \odot\left(\mathbf{D}^{H} \mathbf{F} \mathbf{v}\right)^{*}\label{dfv_decomp}
% \end{equation}

% where, 
% $$\mathbf{D} =\left[\mathbf{D}_{1} \cdots \mathbf{D}_{Q}\right] \in \mathbb{C}^{M_{t} \times L Q}$$

% and, $\mathbf{D}_{q}=\left[\mathbf{d}_{M_{t}}\left(\psi_{q, 1}\right) \cdots \mathbf{d}_{M_{t}}\left(\psi_{q, L}\right)\right] \in \mathbb{C}^{M_{t} \times L}$. 

Comparing the expressions \eqref{init_opt}, \eqref{final_gik}, and \eqref{dc}, one can show that the optimal choice of $\c_q$ in \eqref{init_opt} is the solution to the following optimization problem for a proper choice of $\g_q$. 
\begin{problem}
Given an equal-gain vector $\g_q \in \mathbb{C}^L$, $q \in \{1, \cdots, Q\}$, find vector $\c_q \in \mathbb{C}^{M_t}$ such that
\begin{align}
&\c_q=\underset{\c, \|c\|=1}{\arg \min } \lim_{L\rightarrow \infty} \left\|\sum_{q \in \mathcal{W}_k}\sqrt{\gamma_q}\left(\mathbf{e}_{q} \otimes \mathbf{g}_q\right)- \mathbf{D}^{H} \c\right\|^{2} \label{obj_func}
\end{align}
\label{main_problem}
\end{problem}


However, we now need to find the optimal choice of $\g_q$ that minimizes the objective in \eqref{init_opt}. Using \eqref{final_gik}, and \eqref{dc}, we have the following optimization problem.
\begin{problem}
Find a set of equal-gain $\g_q \in \mathbb{C}^L$, i.e. $\mathcal{G}_k$ such that
\begin{equation}
 \mathcal{G}_k = \underset{\mathcal{G}}{\arg\min }\left\| abs(\D^H \c_q)- abs(\sum_{q \in \mathcal{W}_k}\sqrt{\gamma_q}(\mathbf{e}_{q} \otimes \mathbf{g}_q))\right\|^{2} \label{g_final_eq}
\end{equation} 
where $\mathcal{G}_k = \{\g_q| q \in \mathcal{W}_k\}$, and $abs(.)$ denotes the element-wise absolute value of a vector.
\label{g_problem}
\end{problem}

Hence, the codebook design for a system with full-digital beamforming capability is found by solving Problem~\ref{main_problem} for proper choice of $\g_q$ obtained as a solution to Problem ~\ref{g_problem}. The codebook for Hybrid beamforming, is then found as
\begin{align}
    \underset{\mathbf{F}_{q},\mathbf{v}_{q}}{\arg\min } \|\mathbf{F}_{q}\mathbf{v}_{q} - \c_q\|^2
\end{align}
where the columns of $\F_q \in \mathbb{C}^{M_t \times N_{RF}}$ are equal-gain vectors and $\v_q \in \mathbb{C}^{N_{RF}}$. The solution may be obtained using simple, yet effective suboptimal algorithm such as orthogonal matching pursuit (OMP) \cite{love15}\cite{noh17}\cite{Hussain17}. In the next section we continue with the solution of Problem~\ref{main_problem}, and~\ref{g_problem}.





% Comparing the expressions in equations \eqref{final_gik}, and, \eqref{dc}, a perfect solution to the optimization problem in \eqref{init_opt}, could be a pair $\textbf{(F, v)}$, for which the following equality holds.  



% \begin{align}
%     \mathbf{D}^{H} \mathbf{F} \mathbf{v}=\sum_{q \in \mathcal{W}_k}\sigma\left(\mathbf{e}_{q} \otimes \mathbf{g}_q\right) \label{potential_answer}
% \end{align}

% However, we have that $M_t > N_{RF}$ and therefore, the matrix $\textbf{D}^H \textbf{F}$ may not be full-rank, resulting in the RHS of equation \eqref{potential_answer} being in the null space of LHS. Therefore, given some $\g_q \in \mathcal{G}_L$ we formulate the following two-step optimization problem to search for $\textbf{(F, v)}$ that gets as close as possible to the desired value. 

% \begin{problem}

% a) For all $q \in \{1, \cdots, Q\}$, given $\g_q \in \mathcal{G}_L$, find the unit-norm vector $\c_q \in \mathbb{C}^{M_t}$ such that
% \begin{align}
% &\c^{|\g_q}_q=\underset{\c}{\arg \min } \lim_{L\longrightarrow \infty} \frac{1}{L}\left\|\left(\sum_{q \in \mathcal{W}_k}\sigma\left(\mathbf{e}_{q} \otimes \mathbf{g}_q\right)\right)- \mathbf{D}^{H} \c\right\|_{2}^{2} \label{obj_func}
% \end{align}

% \noindent
% b) For every given $\c_q$, find $\F_q \in \mathbb{C}^{M_t \times N_{RF}}$ , and $\v_q \in \mathbb{C}^{N_RF}$ that satisfy $\mathbf{F}_{q}\mathbf{v}_{q} = \c_q$ the condition $\f_n \in \mathcal{B}_{M_t}$ as in equation \eqref{f_constant_gain}. 

% \begin{align}
%     \mathbf{F}_{\mid \g_q}\mathbf{v}_{\mid \g_q} = \c_q
% \end{align}

% \end{problem}



Note that the solution to problem \ref{main_problem} is the limit of the sequence of solutions to a least-square optimization problem as $L$ goes to infinity. For each $L$ we find that,
 \begin{align}
& {\c}^{(L)}_k = \sum_{q \in \mathcal{W}_k}\sqrt{\gamma_q}(\D \D^H)^{-1} \D  \left(\mathbf{e}_q \otimes \mathbf{g}_q\right) \\
& {\c}^{(L)}_k = \sum_{q \in \mathcal{W}_k}\sigma_q \D_q\g_q \label{c_final_eq}
\end{align}
where $\sigma_q = \frac{\sqrt{ 2\pi \delta_q \frac{\delta_q}{\Delta_k}}}{L \sum_{p=1}^{2^B}\delta_p} = \frac{\delta_q}{L\sqrt{2\pi\Delta_k}}$, noting that it holds that, 

% $$\mathbf{D D}^{H}=\left({L \sum_{p=1}^{2^B}\delta_p}\right) \mathbf{I}_{M_{t}}$$.
% % Dividing by $\|\Tilde{\c}^{(L)}_q\|$ and taking the limit as L goes to infinity we will find the optimal $\c_q$. i.e. 


Using \eqref{c_final_eq}, equation \eqref{g_final_eq} can now be rewritten as, 

\begin{align}
 &\mathcal{G}_k = \underset{\mathcal{G}}{\arg\min } \nonumber
 \\&\left\| abs(\D^H \D\sum_{q \in \mathcal{W}_k}\sigma_q(\mathbf{e}_{q} \otimes \mathbf{g}_q))-abs(\sum_{q \in \mathcal{W}_k}\sqrt{\gamma_q}(\mathbf{e}_{q} \otimes \mathbf{g}_q))\right\|^{2} \label{g_simple_final_eq}
\end{align} 

% where $\sigma' = \frac{1}{2\pi L}$. We will state the following theorem without formal proof.
\begin{proposition}
The maximizer of \eqref{g_simple_final_eq} is in the form $\g_q = \left[{\begin{array}{cccc} 1& \alpha^\eta &\cdots & \alpha^{\eta (L -1)} \end{array}}\right]^T$ for some $\eta$ where $\alpha = e^{j(\frac{\delta_q}{L})}$. \label{proposiiton_g}
\end{proposition}

An analytical closed form solution for $\c_q$ can be found as follows. We have, 
 \begin{align}
     {\c_k}^{(L)} & = \sum_{q \in \mathcal{W}_k} \sigma_q \left(\sum_{l=1}^{L}g_{q, l}\mathbf{d}_{M_t}(\psi_{q, l})\right)  \nonumber\\
     & = \sum_{q \in \mathcal{W}_k} \left(\sum_{l=1}^L \sigma_qg_q^{(l)}\left[{\begin{array}{ccc}
     1 &\cdots & e^{j(M_t-1)\psi_{q, l}}\\
     \end{array}}\right]^T \right)  \nonumber\\
     & = \sum_{q \in \mathcal{W}_k} \sigma_q \left[{\begin{array}{ccc}
     \sum_{l=1}^L g_{q, l}& \cdots & \sum_{l=1}^L g_{q, l}e^{j(M_t-1)\psi_{q, l}}\\
     \end{array}}\right]^T 
\end{align}

 
Let us write, 

\begin{equation}
    \c_k^{(L)} = \sum_{q \in \mathcal{W}_k} \c_q^{(L)}
\end{equation}

where, 

\begin{equation}
{\c_q}^{(L)}  = \sigma_q \sum_{l=1}^{L}g_{q, l}\mathbf{d}_{M_t}(\psi_{q, l})
\end{equation}


Choosing $\g_q$ as in proposition \ref{proposiiton_g},  the $m$-th element of the vector ${\c}_q = \lim_{L\rightarrow \infty}{\c_q}^{(L)} $, i.e. $c_{q,m}$, is given by 

\begin{equation}
    c_{q,m} = \frac{\delta_q}{\sqrt{2\pi}\Delta_k}\lim_{L\rightarrow \infty} \frac{1}{L}\sum_{l=0}^{L-1} g_{q, l} e^{j(m\psi_{q, l+1})}
\end{equation}

% With a choice of $\g_q = \left[{\begin{array}{cccc} 1& \alpha^\eta &\cdots & \alpha^{\eta (L -1)} \end{array}}\right]^T$ where we set 
% % $\alpha = e^j(\frac{2\pi}{L2^B})$
% $\alpha = e^{j(\frac{\delta_q}{L})}$ and suitable $\eta$ (to be determined later), we can write


% \begin{equation}
%     \Tilde{c}_q^{(m)} = \lim_{L\longrightarrow \infty} \frac{1}{L}\sum_{l=0}^{L-1} g_l e^{jm(\psi_{q-1}+\frac{2 \pi(\ell+0.5)}{L2^B})}
% \end{equation}

\begin{equation}
    c_{q,m} = \frac{\delta_q}{\sqrt{2\pi}\Delta_k}\lim_{L\rightarrow \infty} \frac{1}{L}\sum_{l=0}^{L-1} g_{q, l} e^{jm(\psi_{q-1}+\frac{\delta_q(\ell+0.5)}{L})}
\end{equation}

% After some basic manipulations we get, 
\begin{equation}
    c_{q,m} = \frac{\delta_q}{\sqrt{2\pi}\Delta_k}\lim_{L\rightarrow \infty} \frac{1}{L}\sum_{l=0}^{L-1} \alpha^{(\eta+m) l} e^{jm(\psi_{q-1}+\frac{0.5\delta_q}{L})} 
\end{equation}


\begin{equation}
    c_{q,m} = \frac{\delta_q}{\sqrt{2\pi}\Delta_k}e^{jm(\psi_{q-1})}\lim_{L\longrightarrow \infty} \frac{1}{L}\sum_{l=0}^{L-1} \alpha^{(\eta+m) l}   
\end{equation}

\begin{equation}
    c_{q,m} = \frac{\delta_q}{\sqrt{2\pi}\Delta_k}e^{jm(\psi_{q-1})} \int_{0}^{1} \alpha^{(\eta + m)Lx}dx
\end{equation}


\begin{equation}
    c_{q,m} = \frac{\delta_q}{\sqrt{2\pi}\Delta_k}e^{jm(\psi_{q-1})} \int_{0}^{1} e^{j\frac{2\pi(\eta + m)L}{L2^B}x}dx
\end{equation}


\begin{align}
 c_{q,m} &= \frac{\delta_q}{\sqrt{2\pi}\Delta_k}e^{jm\psi_{q-1}} \int_{0}^{1} e^{j\xi x}dx \nonumber \\
 &= \frac{\delta_q}{\sqrt{2\pi}\Delta_k}e^{j(m\psi_{q-1} + \frac{\xi}{2})} sinc(\frac{\xi}{2\pi}) \label{final_g}
\end{align}

where $\xi = \delta_q (\eta +m )$.


% \begin{equation}
%     c_{q,m} = e^{jm(\psi_{q-1})} \frac{e^{j\xi}-1}{j\xi}
% \end{equation}

% \begin{equation}
%     c_{q,m} = e^{j(m\psi_{q-1} + \frac{\xi}{2})} \frac{e^{j\xi/2}-e^{-j\xi/2}}{j\xi/2}
% \end{equation}

% \begin{equation}
%     c_{q,m} = e^{j(m\psi_{q-1} + \frac{\xi}{2})} sinc(\frac{\xi}{2\pi}) \label{final_g}
% \end{equation}






% Note that the solution to part \textit{(a)} is the limit of the sequence of solutions to a least-square optimization problem as $L$ goes to infinity.
%  \begin{align}
% & \Tilde{\c}^{(L)}_q = (\D \D^H)^{-1} \D \sigma \left(\mathbf{e}_q \otimes \mathbf{g}_q\right) \\
% & \Tilde{\c}^{(L)}_q =  \sigma' \D \left(\mathbf{e}_q \otimes \mathbf{g}_q\right) = \sigma' \D_q\g_q
% \end{align}
% Dividing by $\|\Tilde{\c}^{(L)}_q\|$ and taking the limit as L goes to infinity we will find the optimal $\c_q$. i.e. 

% \begin{equation}
%     \c_q = \lim_{L\longrightarrow \infty} \frac{\Tilde{\c}^{(L)}_q}{L\|\Tilde{\c}^{(L)}_q\|}
% \end{equation}

% We aim to find $\g_q$ such that

% \begin{align}
% & \left\| abs(\D^H \gamma' \D_q \mathbf{g}_q)- abs(\gamma \left(\mathbf{e}_{q} \otimes \mathbf{g}_q\right)) \right\|_{2}^{2}
% \end{align}

% is minimized.

% % Is it possible to show that $\g_q$ is independent of $q$ (assume regular scenario)? If true, i.e., $\g_q = \g$ we have

% We will provide a suboptimal structure for the choice of $\g$. Let us assume $\g_q = \g, $  for all $ q \in \{1, \cdots, Q\}$. Then we can write
% \begin{align}
%      &\Tilde{\c_q}^{(L)} = D_q\mathbf{g} = \sum_{l=1}^{L}g_l\mathbf{d}_{M_t}(\psi_{q, l})  \nonumber\\
%      & = \sum_{l=1}^L g_l\left[{\begin{array}{cccc}
%      1& e^{j\psi_{q, l}} &\cdots & e^{j(M_t-1)\psi_{q, l}}\\
%      \end{array}}\right]^T  = \nonumber\\
%      & \left[{\begin{array}{cccc}
%      \sum_{l=1}^L g_l& \sum_{l=1}^L g_le^{j\psi_{q, l}} &\cdots & \sum_{l=1}^L g_le^{j(M_t-1)\psi_{q, l}}\\
%      \end{array}}\right]^T
%  \end{align}
 
 

%  We have for the $m$-th element of the vector $\Tilde{\c}_q$, i.e. $\Tilde{c}^{(m)}_q$  for $ 0 \leq m \leq M_t-1$

% \begin{equation}
%     \Tilde{c}_q^{(m)} = \lim_{L\longrightarrow \infty} \frac{1}{L}\sum_{l=0}^{L-1} g_l e^{j(m\psi_{q, l+1})}
% \end{equation}

% With a choice of $\g = \left[{\begin{array}{cccc} 1& \alpha^\eta &\cdots & \alpha^{\eta (L -1)} \end{array}}\right]^T$ where we set 
% % $\alpha = e^j(\frac{2\pi}{LQ})$
% $\alpha = e^j(\frac{\psi_{q}-\psi_{q-1}}{L})$ and suitable $\eta$ (to be determined later), we can write


% % \begin{equation}
% %     \Tilde{c}_q^{(m)} = \lim_{L\longrightarrow \infty} \frac{1}{L}\sum_{l=0}^{L-1} g_l e^{jm(\psi_{q-1}+\frac{2 \pi(\ell+0.5)}{LQ})}
% % \end{equation}

% \begin{equation}
%     \Tilde{c}_q^{(m)} = \lim_{L\longrightarrow \infty} \frac{1}{L}\sum_{l=0}^{L-1} g_l e^{jm(\psi_{q-1}+\frac{(\psi_{q}-\psi_{q-1})(\ell+0.5)}{L})}
% \end{equation}

% \begin{equation}
%     c_q^{(m)} = \lim_{L\longrightarrow \infty} \frac{1}{L}\sum_{l=0}^{L-1} \alpha^{(\eta+m) l} e^{jm(\psi_{q-1}+\frac{0.5(\psi_{q}-\psi_{q-1})}{L})} 
% \end{equation}


% \begin{equation}
%     c_q^{(m)} = e^{jm(\psi_{q-1})}\lim_{L\longrightarrow \infty} \frac{1}{L}\sum_{l=0}^{L-1} \alpha^{(\eta+m) l}   
% \end{equation}

% \begin{equation}
%     c_q^{(m)} = e^{jm(\psi_{q-1})} \int_{0}^{1} \alpha^{(\eta + m)Lx}dx
% \end{equation}


% % \begin{equation}
% %     c_q^{(m)} = e^{jm(\psi_{q-1})} \int_{0}^{1} e^{j\frac{2\pi(\eta + m)L}{LQ}x}dx
% % \end{equation}


% \begin{equation}
%     c_q^{(m)} = e^{jm(\psi_{q-1})} \int_{0}^{1} e^{j\xi x}dx
% \end{equation}

% where $\xi = (\psi_{q}-\psi_{q-1}) (\eta +m )$.


% \begin{equation}
%     c_q^{(m)} = e^{jm(\psi_{q-1})} \frac{e^{j\xi}-1}{j\xi}
% \end{equation}

% \begin{equation}
%     c_q^{(m)} = e^{j(m\psi_{q-1} + \frac{\xi}{2})} \frac{e^{j\xi/2}-e^{-j\xi/2}}{j\xi/2}
% \end{equation}

% \begin{equation}
%     c_q^{(m)} = e^{j(m\psi_{q-1} + \frac{\xi}{2})} sinc(\frac{\xi}{2\pi})
% \end{equation}

% where, $sinc(t) = \frac{sin(\pi t)}{\pi t}$.
















% We first optimize for $\alpha$ by setting the derivative of \eqref{obj_func} to zero, to get: 

% $$\hat{\alpha}=\frac{\sum_{q \in \mathcal{W}_k}\sqrt{\frac{Q}{M_{t}}}(\mathbf{F} \mathbf{v})^{H} \mathbf{D}_{q} \mathbf{g}_q}{\left\|\mathbf{D}^{H} \mathbf{F} \mathbf{v}\right\|_{2}^{2}}$$

% Replacing the value of $\alpha$ as above we get to the following optimization problem. 

% \begin{align}
% \left(\mathbf{F}_{\mid \mathbf{g}}, \mathbf{v}_{\mid \mathbf{g}}\right) &=\underset{\mathbf{F}, \mathbf{v}}{\arg \max } \frac{\left|\sum_{q \in \mathcal{W}_k}\sqrt{\frac{2^{\mathbf{B}}}{M_{t}}}\left(\mathbf{D}_{q} \mathbf{g}_q\right)^{H} \mathbf{F} \mathbf{v}\right|^{2}}{\left\|\mathbf{D}^{H} \mathbf{F} \mathbf{v}\right\|_{2}^{2}} \nonumber \\
% &=\underset{\mathbf{F}, \mathbf{v}}{\arg \max }\left|\sum_{q \in \mathcal{W}_k}\left(\mathbf{D}_{q} \mathbf{g}\right)^{H} \mathbf{F} \mathbf{v}\right|^{2}
% \end{align}


% \section{Codebook Design Problem Formulation}
% \label{sec: formulation}

% % Let $\mathbf{c} \doteq \mathbf{F} \mathbf{v} \in \mathbb{C}^{M_{t}}$ be the unit-norm transmit beam-forming codeword, where analog beamsteering matrix $\mathbf{F}=\left[\mathbf{f}_{1}, \cdots, \mathbf{f}_{N_{R F}}\right] \in \mathbb{C}^{M_{t} \times N_{R F}}$ contains $N_{RF}$ equal-gain analog beamsteering vectors $\mathbf{f}_{n}$ and the baseband beamforming vector $\mathbf{v} \in \mathbb{C}^{N_{R F}}$ is such that $\|\c\|_{2}=1$ holds. The equal-gain constraint holds if $\mathbf{f}_{n} \in \mathcal{B}_{M_{t}}$ for 

% % \begin{equation}
% %     \mathcal{B}_{M_{t}}=\left\{\mathbf{w} \in \mathbb{C}^{M_{t}}:\left(\mathbf{w} \mathbf{w}^{H}\right)_{\ell, \ell}=1 / M_{t}, 1 \leq \ell \leq M_{t}\right\}
% %     \label{f_constant_gain}
% % \end{equation}

% % Let $\mathcal{C}=\left\{\mathbf{c}_{1}, \cdots, \mathbf{c}_{Q}\right\}$ be the codebook with codewords taking value as $\mathbf{c}_{q}=\mathbf{F}_{q} \mathbf{v}_{q}$. Consider a half-wavelength uniform linear array (ULA) of antennas as in the previous subsection, i.e. $d = \frac{\lambda}{2}$, with an array response vector as in \eqref{array_factor}.

% % % %$$\mathbf{d}_{M_{t}}(\psi)=\frac{1}{\sqrt{M_{t}}}\left[1 e^{j \psi} \cdots e^{j\left(M_{t}-1\right) \psi}\right]^{T} \in \mathbb{C}^{M_{t}}$$
% % % $$\mathbf{d}_{M_{t}}(\psi)=\left[1, e^{j \psi}, \cdots, e^{j\left(M_{t}-1\right) \psi}\right]^{T} \in \mathbb{C}^{M_{t}}$$

%  For convenience, in this section, we will drop the index \emph{ula} from the expression of the array factor $\d_{ula, M_t}(\psi)$ and gain $G^{ula}(\psi, \mathbf{c})$.  Before presenting the formulation of the codebook design problem, we introduce some new notations. We divide the angular range of $\theta \in [-\pi, 0]$  into equal-length beams $\omega_{q} \text { for } q \in\left\{1, \cdots, Q\right\}$. i.e. 
% $$
% \begin{aligned}
% \omega_{q} &=\left[\theta_{q-1}, \theta_{q}\right) ,& \theta_{q}= -\pi+ \frac{ \pi}{Q} q
% \end{aligned}
% $$

% % Further we introduce the change of variable $\psi = \frac{2 \pi d}{\lambda} \cos(\theta)$ where $\psi \in [-\pi, \pi]$. For $d = \frac{\lambda}{2}$,

% Corresponding to $\omega_q$ intervals, there exists $\nu_q$ ranges with respect to $\psi$ such that, 

% $$
% \begin{aligned}
% \nu_{q} &=\left[\psi_{q-1}, \psi_{q}\right), & \psi_q = -\pi\cos(\frac{\pi}{Q}q)
% \end{aligned}
% $$


% Under the reference gain as in \eqref{reference gain} and using Parseval's theorem \cite{parseval}, we will have:
% % and assuming $|\c| = 1$
% \begin{equation}
% \int_{-\pi}^{\pi} G(\psi, \mathbf{c}) d \psi=2 \pi\left\| \mathbf{c}\right\|^{2}=2 \pi\label{parseval}
% \end{equation}

% Let  $G_{\text {ideal }, q}(\psi)$ denote the desired ideal gain which is supposed to be constant on $\nu_q$ and zero otherwise. It must hold that, 
% % \begin{equation}
% % \int_{-\pi}^{\pi} G(\psi, \mathbf{c}) d \psi=2 \pi\left\|\frac{1}{\sqrt{M_{t}}} \mathbf{c}\right\|_{2}^{2}=\frac{2 \pi}{M_{t}}    \label{parseval}
% % \end{equation}





% \begin{align}
% &\int_{-\pi}^{\pi} G_{\text {ideal }, q}(\psi) d \psi =\int_{\nu_{q}} t d \psi+\int_{[-\pi, \pi] \backslash \nu_{q}} 0 d \psi \nonumber\\
% &=(\psi_q - \psi_{q-1}) t={2 \pi} \label{plain}
% \end{align}

% which in turn will give: 

% \begin{equation}
%     G_{\text {ideal }, q}(\psi)=\frac{2\pi}{(\psi_q - \psi_{q-1})} \mathds{1}_{\nu_{q}}(\psi), \quad \psi \in[-\pi, \pi] \label{ideal_gain}
% \end{equation}

% We aim to design the codebooks so as to mimic the ideal gain computed in equation \eqref{ideal_gain}. Therefore, the plain codebook design problem is formulated as a minimization of a MSE as follows: 


% % \begin{align}
% % &\left(\mathbf{F}_{\text {opt }, q}, \mathbf{v}_{\mathrm{opt}, q}\right) \nonumber \\&=\underset{\mathbf{F}, \mathbf{v}}{\arg \min } \int_{-\pi}^{\pi}\left|G_{\text {ideal }, q}(\psi)-G(\psi, \mathbf{F} \mathbf{v})\right|^{2} d \psi
% % \end{align}

% \begin{align}
% &\c_q^{opt}=\underset{\c, \|\c\|=1}{\arg \min } \int_{-\pi}^{\pi}\left|G_{\text {ideal }, q}(\psi)-G(\psi, \c)\right| d \psi
% \label{composite_MSE}
% \end{align}

% By uniformly sampling on the range of $\psi$ we can rewrite the optimization problem as follows, 
% \begin{align}
%     &\underset{\c, \|\c\|=1}{\arg \min }\left[\lim _{L \rightarrow \infty} \sum_{p=1}^{Q} \delta_p \sum_{\ell=1}^{L} \frac{\left|G_{\text {ideal }, q}\left(\psi_{p, \ell}\right)-G\left(\psi_{p, \ell}, \c \right)\right|}{L}\right]\label{equivalent_MSE}
% \end{align}

%  where for $q = 1 \ldots Q$, 
%  $$\delta_q = \psi_{q}-\psi_{q-1}, \quad \psi_{q, \ell}=\psi_{q-1}+\frac{\delta_q(\ell-0.5)}{L} $$

 
%  We can write equation \eqref{equivalent_MSE}, as 
%  \begin{align}
%  &\c_q^{opt} = \underset{\c,  \|\c\|=1}{\arg \min } \lim_{L\rightarrow \infty}\frac{1}{L}\left\|\mathbf{G}_{\text {ideal }, q}-\mathbf{G}(\c)\right\| \label{init_opt}
%  \end{align}
 
%  where,  $$\mathbf{G}(\c)=\left[{\delta_1} G\left(\psi_{1,1}, \c\right) \ldots {\delta_{Q}}G\left(\psi_{Q, L}, \c\right)\right]^{T} \in \mathbb{Z}^{L Q}$$

% and, 
% $$ \mathbf{G}_{\text {ideal }, q}=\left[{\delta_1} G_{\text {ideal }, q}\left(\psi_{1,1}\right) \ldots {\delta_{Q}}G_{\text {ideal }, q}\left(\psi_{Q, L}\right)\right]^{T} \in \mathbb{Z}^{L Q}$$

% Note that it holds that 

% \begin{equation}
%     \mathbf{G}_{\text {ideal }, q}={2\pi}\left(\mathbf{e}_{q} \otimes \mathbf{1}_{L, 1}\right) \label{ideal}
% \end{equation}

% with $\mathbf{e}_{q} \in \mathbb{Z}^{Q}$ being the standard basis vector for the $q$-th axis among $Q$ ones. Now, note that $\mathbf{1}_{L, 1}=\mathbf{g} \odot \mathbf{g}^{*}$ for any equal gain $\mathbf{g} \in \mathbb{C}^L$. Therefore, for a suitable choice of $\g$ we can write: 

% % $$\mathcal{G}_{L}=\left\{\mathbf{g} \in \mathbb{C}^{L}:\left(\mathbf{g} \mathbf{g}^{H}\right)_{\ell, \ell}=1,1 \leq \ell \leq L\right\}$$


% \begin{align}
% \mathbf{G}_{\text {ideal }, q} &={2\pi}\left(\mathbf{e}_{q} \otimes\left(\mathbf{g} \odot \mathbf{g}^{*}\right)\right) \nonumber\\
% &=\left(\sqrt{2\pi}\left(\mathbf{e}_{q} \otimes \mathbf{g}\right)\right) \odot\left(\sqrt{{2\pi}}\left(\mathbf{e}_{q} \otimes \mathbf{g}\right)\right)^{*} \label{g_id_q_equivalent}
% \end{align}


% Similarly, it is straightforward to observe,


% \begin{align}
% \mathbf{G}(\c) &=\left(\mathbf{D}^{H} \c\right) \odot\left(\mathbf{D}^{H} \c\right)^{*} \label{dc}
% \end{align}
% where $\mathbf{D} =\left[\sqrt{\delta_1}\mathbf{D}_{1} \cdots \sqrt{\delta_{Q}}\mathbf{D}_{Q} \right] \in \mathbb{C}^{M_{t} \times L Q}$, and 
% $\mathbf{D}_{q}=\left[\mathbf{d}_{M_{t}}\left(\psi_{q, 1}\right) \cdots \mathbf{d}_{M_{t}}\left(\psi_{q, L}\right)\right] \in \mathbb{C}^{M_{t} \times L}.$
 

% % Comparing the expressions in equations \eqref{g_id_q_equivalent} and \eqref{dc},  we formulate the following two-step optimization for the codebook design problem for hybrid beamforming, presented below as \textit{Problem \ref{main_problem}}. Given a suitable $\g_q$, in the first step, we search for a unit-norm codeword $\c_q$ for fully-digital beamforming, and in the second step we find a suitable pair $(\F_q, \v_q)$ to complete the hybrid beamforming codebook design.  

 

% % \begin{problem}

% % a) For all $q \in \{1, \cdots, Q\}$, given equal-gain $\g_q \in \mathbb{C}^L$, find vector $\c_q \in \mathbb{C}^{M_t}$ such that
% % \begin{align}
% % &\c_q=\underset{\c, \|c\|=1}{\arg \min } \lim_{L\longrightarrow \infty} \left\|\sqrt{2\pi}\left(\mathbf{e}_{q} \otimes \mathbf{g}_q\right)- \mathbf{D}^{H} \c\right\|^{2} \label{obj_func}
% % \end{align}

% % \vspace{2mm}
% % \noindent
% % b) Find $\F_q \in \mathbb{C}^{M_t \times N_{RF}}$ , and $\v_q \in \mathbb{C}^{N_{RF}}$ that solve $\mathbf{F}_{q}\mathbf{v}_{q} = \c_q$ subject to the equal-gain condition on the columns of $\F_q$. 

% % % \begin{align}
% % %     \mathbf{F}_{\mid \g_q}\mathbf{v}_{\mid \g_q} = \c_q
% % % \end{align}
% % \label{main_problem}
% % \end{problem}

% % The solution to the last problem provides the optimal codebook designed for hybrid beamforming under a ULA structure, given the choice of $\g_q$. Note that we were initially trying to solve the optimization problem in equation \eqref{init_opt}. In light of modifications \eqref{g_id_q_equivalent}, and \eqref{dc}, the quantity to be minimized in the limit of equation \eqref{init_opt} can be written as follows,

% % \begin{align}
% %     & \left\|\mathbf{G}_{\text {ideal }, q}-\mathbf{G}(\c)\right\| = \left\| abs(\D^H \c_q)- \sqrt{2\pi} abs(\mathbf{e}_{q} \otimes \mathbf{g})\right\| \label{g_init}
% % \end{align}
% % where $abs(.)$ denotes the element-wise absolute value of a vector.

% % Equation \eqref{g_init} reveals the central role of the vector $\g_q$ in minimizing the MSE in equation \eqref{init_opt}. In fact, the choice of $\g_q$ has to be optimized in a way that norm in \eqref{g_init} is minimized; i.e. the result gain designed by codebook $\c_q$ gets as close as possible to the ideal gain. To optimize the choice of $\g_q$, we formulate the following optimization problem, presented as \textit{Problem \ref{g_problem}}.

% % \begin{problem}
% % For all $q = 1, \ldots, Q$, find equal-gain $\g_q \in \mathbb{C}^L$ such that

% % \begin{equation}
% %  \g_q = \underset{\g}{\arg\min }\left\| abs(\D^H \c_q)- \sqrt{2\pi} abs(\mathbf{e}_{q} \otimes \mathbf{g})\right\|^{2} \label{g_final_eq}
% % \end{equation} 

% % \label{g_problem}
% % \end{problem}

% % In the next session we propose our approach to solve problems \ref{main_problem} and \ref{g_problem}.

% %%%%%%%%%%%%%%%%%%%%%%%%%%%%%%%
% Comparing the expressions \eqref{init_opt}, \eqref{g_id_q_equivalent}, and \eqref{dc}, one can show that the optimal choice of $\c_q$ in \eqref{init_opt} is the solution to the following optimization problem for a proper choice of $\g_q$. 
% \begin{problem}
% Given an equal-gain vector $\g_q \in \mathbb{C}^L$, $q \in \{1, \cdots, Q\}$, find vector $\c_q \in \mathbb{C}^{M_t}$ such that
% \begin{align}
% &\c_q=\underset{\c, \|c\|=1}{\arg \min } \lim_{L\longrightarrow \infty} \left\|\sqrt{2\pi}\left(\mathbf{e}_{q} \otimes \mathbf{g}_q\right)- \mathbf{D}^{H} \c\right\|^{2} \label{obj_func}
% \end{align}
% \label{main_problem}
% \end{problem}
% However, we now need to find the optimal choice of $\g_q$ that minimizes the objective in \eqref{init_opt}. Using \eqref{g_id_q_equivalent}, and \eqref{dc}, we have the following optimization problem.
% \begin{problem}
% Find equal-gain $\g_q \in \mathbb{C}^L$, $q = 1, \ldots, Q$, such that
% \begin{equation}
%  \g_q = \underset{\g}{\arg\min }\left\| abs(\D^H \c_q)- \sqrt{2\pi} abs(\mathbf{e}_{q} \otimes \mathbf{g})\right\|^{2} \label{g_final_eq}
% \end{equation} 
% where $abs(.)$ denotes the element-wise absolute value of a vector.
% \label{g_problem}
% \end{problem}

% Hence, the codebook design for a system with full-digital beamforming capability is found by solving Problem~\ref{main_problem} for proper choice of $\g_q$ obtained as a solution to Problem ~\ref{g_problem}. The codebook for Hybrid beamforming, is then found as
% \begin{align}
%     \underset{\mathbf{F}_{q},\mathbf{v}_{q}}{\arg\min } \|\mathbf{F}_{q}\mathbf{v}_{q} - \c_q\|^2
% \end{align}
% where the columns of $\F_q \in \mathbb{C}^{M_t \times N_{RF}}$ are equal-gain vectors and $\v_q \in \mathbb{C}^{N_{RF}}$. The solution may be obtained using simple, yet effective suboptimal algorithm such as orthogonal matching pursuit (OMP) \cite{love15}\cite{noh17}\cite{Hussain17}. In the next section we continue with the solution of Problem~\ref{main_problem}, and~\ref{g_problem}.


% %%%%%%%%%%%%%%%%%%%%%%%%%%%%%%%

% % Note that the solution to part \textit{(a)} is the limit of the sequence of solutions to a least-square optimization problem as $L$ goes to infinity.
% %  \begin{align}
% % & \Tilde{\c}^{(L)}_q = (\D \D^H)^{-1} \D \sigma \left(\mathbf{e}_q \otimes \mathbf{g}_q\right) \\
% % & \Tilde{\c}^{(L)}_q =  \sigma' \D \left(\mathbf{e}_q \otimes \mathbf{g}_q\right) = \sigma' \D_q\g_q
% % \end{align}
% % Dividing by $\|\Tilde{\c}^{(L)}_q\|$ and taking the limit as L goes to infinity we will find the optimal $\c_q$. i.e. 

% % \begin{equation}
% %     \c_q = \lim_{L\longrightarrow \infty} \frac{\Tilde{\c}^{(L)}_q}{L\|\Tilde{\c}^{(L)}_q\|}
% % \end{equation}

% % We aim to find $\g_q$ such that

% % \begin{align}
% % & \left\| abs(\D^H \gamma' \D_q \mathbf{g}_q)- abs(\gamma \left(\mathbf{e}_{q} \otimes \mathbf{g}_q\right)) \right\|_{2}^{2}
% % \end{align}

% % is minimized.

% % % Is it possible to show that $\g_q$ is independent of $q$ (assume regular scenario)? If true, i.e., $\g_q = \g$ we have

% % We will provide a suboptimal structure for the choice of $\g$. Let us assume $\g_q = \g, $  for all $ q \in \{1, \cdots, Q\}$. Then we can write
% % \begin{align}
% %      &\Tilde{\c_q}^{(L)} = D_q\mathbf{g} = \sum_{l=1}^{L}g_l\mathbf{d}_{M_t}(\psi_{q, l})  \nonumber\\
% %      & = \sum_{l=1}^L g_l\left[{\begin{array}{cccc}
% %      1& e^{j\psi_{q, l}} &\cdots & e^{j(M_t-1)\psi_{q, l}}\\
% %      \end{array}}\right]^T  = \nonumber\\
% %      & \left[{\begin{array}{cccc}
% %      \sum_{l=1}^L g_l& \sum_{l=1}^L g_le^{j\psi_{q, l}} &\cdots & \sum_{l=1}^L g_le^{j(M_t-1)\psi_{q, l}}\\
% %      \end{array}}\right]^T
% %  \end{align}
 
 

% %  We have for the $m$-th element of the vector $\Tilde{\c}_q$, i.e. $\Tilde{c}^{(m)}_q$  for $ 0 \leq m \leq M_t-1$

% % \begin{equation}
% %     \Tilde{c}_q^{(m)} = \lim_{L\longrightarrow \infty} \frac{1}{L}\sum_{l=0}^{L-1} g_l e^{j(m\psi_{q, l+1})}
% % \end{equation}

% % With a choice of $\g = \left[{\begin{array}{cccc} 1& \alpha^\eta &\cdots & \alpha^{\eta (L -1)} \end{array}}\right]^T$ where we set 
% % % $\alpha = e^j(\frac{2\pi}{LQ})$
% % $\alpha = e^{j(\frac{\psi_{q}-\psi_{q-1}}{L})}$ and suitable $\eta$ (to be determined later), we can write


% % % \begin{equation}
% % %     \Tilde{c}_q^{(m)} = \lim_{L\longrightarrow \infty} \frac{1}{L}\sum_{l=0}^{L-1} g_l e^{jm(\psi_{q-1}+\frac{2 \pi(\ell+0.5)}{LQ})}
% % % \end{equation}

% % \begin{equation}
% %     \Tilde{c}_q^{(m)} = \lim_{L\longrightarrow \infty} \frac{1}{L}\sum_{l=0}^{L-1} g_l e^{jm(\psi_{q-1}+\frac{(\psi_{q}-\psi_{q-1})(\ell+0.5)}{L})}
% % \end{equation}

% % \begin{equation}
% %     c_q^{(m)} = \lim_{L\longrightarrow \infty} \frac{1}{L}\sum_{l=0}^{L-1} \alpha^{(\eta+m) l} e^{jm(\psi_{q-1}+\frac{0.5(\psi_{q}-\psi_{q-1})}{L})} 
% % \end{equation}


% % \begin{equation}
% %     c_q^{(m)} = e^{jm(\psi_{q-1})}\lim_{L\longrightarrow \infty} \frac{1}{L}\sum_{l=0}^{L-1} \alpha^{(\eta+m) l}   
% % \end{equation}

% % \begin{equation}
% %     c_q^{(m)} = e^{jm(\psi_{q-1})} \int_{0}^{1} \alpha^{(\eta + m)Lx}dx
% % \end{equation}


% % % \begin{equation}
% % %     c_q^{(m)} = e^{jm(\psi_{q-1})} \int_{0}^{1} e^{j\frac{2\pi(\eta + m)L}{LQ}x}dx
% % % \end{equation}


% % \begin{equation}
% %     c_q^{(m)} = e^{jm(\psi_{q-1})} \int_{0}^{1} e^{j\xi x}dx
% % \end{equation}

% % where $\xi = (\psi_{q}-\psi_{q-1}) (\eta +m )$.


% % \begin{equation}
% %     c_q^{(m)} = e^{jm(\psi_{q-1})} \frac{e^{j\xi}-1}{j\xi}
% % \end{equation}

% % \begin{equation}
% %     c_q^{(m)} = e^{j(m\psi_{q-1} + \frac{\xi}{2})} \frac{e^{j\xi/2}-e^{-j\xi/2}}{j\xi/2}
% % \end{equation}

% % \begin{equation}
% %     c_q^{(m)} = e^{j(m\psi_{q-1} + \frac{\xi}{2})} sinc(\frac{\xi}{2\pi})
% % \end{equation}

% % where, $sinc(t) = \frac{sin(\pi t)}{\pi t}$.

% % \subsection{Twin-ULA Setting}

% % We start from problem \eqref{main_problem}, and particularly, equation \eqref{obj_func}. Let $\{\s_m\}_{m =0}^{M_t -1}$ denote the columns of $D^H$. We can write: 

% % \begin{align}
% % &\s_m c_q^{(m)} + \s_{m+M_t/2} c_q^{(m+M_t/2)} \nonumber\\
% % &= \s_m (c_q^{(m)} + e^{-j \frac{2\pi d_y}{\lambda} \sin(\theta)} c_q^{(m+M_t/2)}) \nonumber\\
% % & \approx \s_m \Tilde{c}_q^{(m)}, \quad  m = 0 \ldots \frac{M_t}{2}-1\end{align}

% % where we define 
% % \begin{equation}
% %     \Tilde{c}_q^{(m)} = c_q^{(m)} + e^{-j \frac{2\pi d_y}{\lambda} \sin(\theta^{(m)}_q)} c_q^{(m+M_t/2)} \label{c_defn}
% % \end{equation}

% % with $\theta_{q-1}\leq\theta^{(m)}_q \leq \theta_{q}$ being a constant value.  

% % Therefore, the solution to part $(a)$ of problem \eqref{main_problem}, under the twin-ULA setup can be expressed as:

% %  \begin{align}
% % & \Tilde{\c}_q = (\Tilde{\D} \Tilde{\D}^H)^{-1} \Tilde{\D} \sigma \left(\mathbf{e}_q \otimes \mathbf{g}_q\right) \\
% % & \Tilde{\c}_q =  \sigma' \Tilde{\D} \left(\mathbf{e}_q \otimes \mathbf{g}_q\right) = \sigma' \Tilde{\D}_q\g_q
% % \end{align}

% % where it holds that $\Tilde{\D} \Tilde{\D}^H = \I_{\frac{M_t}{2}}$, and


% % $$\Tilde{\D}_{q}=\left[\d_{\frac{M_{t}}{2}}\left(\psi_{q, 1}\right) \cdots \d_{\frac{M_{t}}{2}}\left(\psi_{q, L}\right)\right] \in \mathbb{C}^{\frac{M_{t}}{2} \times L}$$

% % In other words, one can approximately think of the codebook design problem over a twin ULA of $M_t$ antennas, as such a problem over a ULA of size $\frac{M_t}{2}$.

% % Following the approach in the previous subsection, we can infer, 

% % \begin{equation}
% %     \Tilde{c}_q^{(m)} = e^{j(m\psi_{q-1} + \frac{\xi}{2})} sinc(\frac{\xi}{2\pi}), \quad m = 0 \ldots \frac{M_t}{2} -1
% %     \label{twin_c_result}
% % \end{equation}

% % Having $d_y = \frac{\lambda}{3}$, equating equations \eqref{c_defn}, and \eqref{twin_c_result}, by further assuming $c_q^{(m + \frac{M_t}{2})} = e^{j\beta_m} c_q^{(m)}$, after basic operations we get: 

% % \begin{equation}
% %     c_q^{(m)} = \frac{e^{j(m\psi_{q-1} + \frac{\xi}{2})} sinc(\frac{\xi}{2\pi})}{1+ e^{-j(\frac{2\pi}{3}\sin(\theta_q^{(m)}) + \beta_m)}}
% % \end{equation}

% % % \begin{equation}
% % %     c_q^{(m)} = \frac{e^{j(m\psi_{q-1} + \frac{\xi}{2} +\frac{\pi}{3}\sin(\theta_q^{(m)})+\frac{\beta_m}{2}) sinc(\frac{\xi}{2\pi})}}{2\cos(\frac{\pi}{3}\sin(\theta_q^{(m)}) + \frac{\beta_m}{2})}
% % % \end{equation}

% % Therefore, for each codeword entry $c_q^{(m)}$, it remains to pick $\beta_m$ and $\theta_q^{(m)}$. We relax this decision by setting $\beta_m = \beta$, and $\theta_q^{(m)} = \theta_q$, $ m = 1 \ldots M_t$. Next we explicitly right the expression for the reference gain as $G(\theta, \c_{twin}) = \left|\d_{twin, M_{t}}^{H}(\theta) \c_{twin}\right|^{2}$ where 

% % \begin{align}
% %     &\d_{twin, M_{t}}(\theta) = \left[\d_{\frac{M_{t}}{2}}^{T}(\theta), e^{j(\frac{2\pi}{3}\sin(\theta))}\d_{\frac{M_{t}}{2}}^{T}(\theta)\right]^T\\
% %     & \c_{twin} = \left[\c^T_{twin, \frac{M_t}{2}}, e^{j\beta}\c^T_{twin, \frac{M_t}{2}}\right]^T
% % \end{align}

% % We can  therefore write, 

% % \begin{align}
% %     G(\theta, \c_{twin}) &= \left|\d_{\frac{M_{t}}{2}}^{H}(\theta)\c_{twin, \frac{M_t}{2}} (1 + e^{j(\beta-\frac{2\pi}{3}\sin(\theta))})\right|^2\nonumber\\
% %     & = \left|\d_{\frac{M_{t}}{2}}^{H}(\theta)\c\right|^2 \left| \frac{1 + e^{j(\beta-\frac{2\pi}{3}\sin(\theta))}}{1 + e^{-j(\beta+\frac{2\pi}{3}\sin(\theta))}}\right|^2
% % \end{align}

% % Further define 
% % $$ L(\theta) = \left| \frac{1 + e^{j(\beta-\frac{2\pi}{3}\sin(\theta))}}{1 + e^{-j(\beta+\frac{2\pi}{3}\sin(\theta))}}\right|$$
% % The last equation consists of two parts, the first part being the gain corresponding to a ULA of size $\frac{M_t}{2}$ and the second part being  $L^2(\theta)$. For any $\omega_q = \left[\theta_{q-1}, \theta_{q}\right)$, there exist a $\omega'_q \doteq \left(- \theta_{q}, -\theta_{q-1}\right]$. The gain of a ULA is symmetric over $\omega_q$, and $\omega'_q$ by definition. However utilizing a twin ULA we wish to suppress the gain over $\omega'_q$ as much as we can and contribute to the gain over $\omega_q$ for each codeword $c_q, \quad q = 1\ldots Q$. To this end, we define the \emph{isolation factor} $\mu$ as follows,

% % \begin{align}
% %     \mu = \underset{\omega_q}{\int}{\frac{L(-\theta)}{L(\theta)}} d\theta
% % \end{align}
% %  to denote the level of isolation between each $\omega_q$ and its counterpart.
 
 















% % % \subsection{Composite Codebook Design Problem}

% % % Let us define a \textit{composite beam} $\omega_k$ as a union of multiple disjoint, possibly non-neighboring beams $\nu_{q} \text { for } q \in \mathcal{Q} = \left\{1, \cdots, Q\right\}$. Let $\mathcal{C'}=\left\{\mathbf{c}_{1}, \cdots, \mathbf{c}_{K}\right\}$ be the codebook corresponding to the composite problem.

% % % % $$\bigcup_{k = 1}^{2^{B'}}{\omega_k} = [-\pi, \pi], \quad \text{and} \quad \omega_k \cap \omega_l = \emptyset, \quad \forall k\neq l.$$

% % % Moreover, define for each composite beam $\omega_k$ the set $\mathcal{W}_k \subseteq \mathcal{Q}$ to be the set of indices of the single beams that form $\omega_k$. i.e. $\mathcal{W}_k = \{q \in \mathcal{Q}: v_q \subseteq \omega_k\}$. 
% % % We have then for the new ideal gain for each new codeword $\textbf{c}_k$, 

% % % \begin{align}
% % % &\int_{-\pi}^{\pi} G_{\text {ideal }, k}(\psi) d \psi =\int_{\omega_{k}} t' d \psi+\int_{[-\pi, \pi] \backslash \omega_{k}} 0 d \psi \nonumber\\
% % % &= \sum_{q \in {\mathcal{W}_k}}{\int_{\nu_{q}} t' d \psi} = |\mathcal{W}_k|\frac{2 \pi}{Q} t'=\frac{2 \pi}{M_{t}} \label{composite}
% % % \end{align}

% % % Therefore, $t' = \frac{Q}{|\mathcal{W}_k| M_t}$. It follows that, 

% % % \begin{equation}
% % %     G_{\text {ideal }, k}(\psi)=\frac{Q}{|\mathcal{W}_k| M_{t}} \mathds{1}_{\omega_{k}}(\psi), \quad \psi \in[-\pi, \pi] \label{composite_ideal_gain}
% % % \end{equation}

% % % The new MSE problem for the $k$-th codeword can therefore be written as follows.

% % % \begin{align}
% % % &\left(\mathbf{F}_{\text {opt }, k}, \mathbf{v}_{\mathrm{opt}, k}\right) \nonumber \\&=\underset{\mathbf{F}, \mathbf{v}}{\arg \min } \int_{-\pi}^{\pi}\left|G_{\text {ideal }, k}(\psi)-G(\psi, \mathbf{F} \mathbf{v})\right|^{2} d \psi \nonumber \\&= \underset{\mathbf{F}, \mathbf{v}}{\arg \min }\left[\lim _{L \rightarrow \infty} \sum_{p=1}^{Q} \sum_{\ell=1}^{L} \frac{\left|G_{\text {ideal }, k}\left(\psi_{p, \ell}\right)-G\left(\psi_{p, \ell}, \mathbf{F} \mathbf{v}\right)\right|^{2}}{L Q / 2 \pi}\right]\label{composite_MSE}
% % % \end{align}


% % % \begin{align}
% % % \left(\mathbf{F}_{k}, \mathbf{v}_{k}\right) &=\underset{\mathbf{F}, \mathbf{v}}{\arg \min } \sum_{p=1}^{Q} \sum_{\ell=1}^{L}\left|G_{\text {ideal }, k}\left(\psi_{p, \ell}\right)-G\left(\psi_{p, \ell}, \mathbf{F} \mathbf{v}\right)\right|^{2} \nonumber\\
% % % &=\underset{\mathbf{F}, \mathbf{v}}{\arg \min }\lim_{L\longrightarrow \infty} \frac{1}{L}\left\|\mathbf{G}_{\text {ideal }, k}-\mathbf{G}(\mathbf{F} \mathbf{v})\right\|_{2}^{2} \label{init_opt}
% % % \end{align}

% % % where,  $$\mathbf{G}(\mathbf{F} \mathbf{v})=\left[G\left(\psi_{1,1}, \mathbf{F} \mathbf{v}\right) \cdots G\left(\psi_{Q, L}, \mathbf{F} \mathbf{v}\right)\right]^{T} \in \mathbb{Z}^{L Q}$$

% % % and, 
% % % $$ \mathbf{G}_{\text {ideal }, k}=\left[G_{\text {ideal }, k}\left(\psi_{1,1}\right) \cdots G_{\text {ideal }, k}\left(\psi_{Q, L}\right)\right]^{T} \in \mathbb{Z}^{L Q}$$

% % % Note that we can write using the definition of $\mathcal{W}_k$ that, 

% % % $$G_{\text {ideal }, k}\left(\psi_{p, \ell}\right)=\left\{\begin{array}{ll}
% % % \frac{Q}{|\mathcal{W}_k| M_{t}}, & p \in \mathcal{W}_k \\
% % % 0, & p \notin \mathcal{W}_k 
% % % \end{array},\right.$$

% % % Or in the matrix form, 

% % % \begin{align}
% % %     &\mathbf{G}_{\text {ideal }, k}=\frac{Q}{|\mathcal{W}_k| M_{t}}\left((\bigoplus_{q \in \mathcal{W}_k}{\mathbf{e}_{q}}) \otimes \mathbf{1}_{L, 1}\right)  \nonumber \\ 
% % %     & = \sum_{q \in \mathcal{W}_k}{\frac{Q}{|\mathcal{W}_k| M_{t}}\left(\mathbf{e}_{q} \otimes \mathbf{1}_{L, 1}\right)} \label{G_ideal_k}
% % % \end{align}

% % % with $\mathbf{e}_{q} \in \mathbb{Z}^{Q}$ being the standard basis vector for the $q$-th axis among $Q$ ones. Now, note that $\mathbf{1}_{L, 1}=\mathbf{g} \odot \mathbf{g}^{*}$ for any $\mathbf{g} \in \mathcal{G}_L$, where, 

% % % $$\mathcal{G}_{L}=\left\{\mathbf{g} \in \mathbb{C}^{L}:\left(\mathbf{g} \mathbf{g}^{H}\right)_{\ell, \ell}=1,1 \leq \ell \leq L\right\}$$



% % % Therefore, equation \eqref{G_ideal_k} can be rewritten as:
% % % \begin{align}
% % % &\mathbf{G}_{\text {ideal }, k} = \sum_{q \in \mathcal{W}_k}\frac{Q}{|\mathcal{W}_k| M_{t}}\left(\mathbf{e}_{q} \otimes\left(\mathbf{g}_q \odot \mathbf{g}_q^{*}\right)\right) \nonumber\\
% % % &=\sum_{q \in \mathcal{W}_k}\left(\sigma\left(\mathbf{e}_{q} \otimes \mathbf{g}_q\right)\right) \odot\left(\sigma\left(\mathbf{e}_{q} \otimes \mathbf{g}_q\right)\right)^{*} \nonumber\\
% % % &=\left(\sum_{q \in \mathcal{W}_k}\sigma\left(\mathbf{e}_{q} \otimes \mathbf{g}_q\right)\right)  \odot \left(\sum_{q \in \mathcal{W}_k}\sigma\left(\mathbf{e}_{q} \otimes \mathbf{g}_q\right)\right)^{*} \label{final_gik}
% % % \end{align}

% % % where $\sigma = \sqrt{\frac{Q}{|\mathcal{W}_k|M_{t}}}$. 
% % % On the other hand note that $\mathbf{G}(\mathbf{F} \mathbf{v})$ can be written as,


% % % \begin{equation}
% % %     \mathbf{G}(\mathbf{F} \mathbf{v}) =\left(\mathbf{D}^{H} \mathbf{F} \mathbf{v}\right) \odot\left(\mathbf{D}^{H} \mathbf{F} \mathbf{v}\right)^{*}\label{dfv_decomp}
% % % \end{equation}

% % % where, 
% % % $$\mathbf{D} =\left[\mathbf{D}_{1} \cdots \mathbf{D}_{Q}\right] \in \mathbb{C}^{M_{t} \times L Q}$$

% % % and, $\mathbf{D}_{q}=\left[\mathbf{d}_{M_{t}}\left(\psi_{q, 1}\right) \cdots \mathbf{d}_{M_{t}}\left(\psi_{q, L}\right)\right] \in \mathbb{C}^{M_{t} \times L}$. 

% % % Comparing the expressions in equations \eqref{final_gik}, and, \eqref{dfv_decomp}, a perfect solution to the optimization problem in \eqref{init_opt}, could be a pair $\textbf{(F, v)}$, for which the following equality holds.  



% % % \begin{align}
% % %     \mathbf{D}^{H} \mathbf{F} \mathbf{v}=\sum_{q \in \mathcal{W}_k}\sigma\left(\mathbf{e}_{q} \otimes \mathbf{g}_q\right) \label{potential_answer}
% % % \end{align}

% % % However, we have that $M_t > N_{RF}$ and therefore, the matrix $\textbf{D}^H \textbf{F}$ may not be full-rank, resulting in the RHS of equation \eqref{potential_answer} being in the null space of LHS. Therefore, given some $\g_q \in \mathcal{G}_L$ we formulate the following two-step optimization problem to search for $\textbf{(F, v)}$ that gets as close as possible to the desired value. 

% % % \begin{problem}

% % % a) For all $q \in \{1, \cdots, Q\}$, given $\g_q \in \mathcal{G}_L$, find the unit-norm vector $\c_q \in \mathbb{C}^{M_t}$ such that
% % % \begin{align}
% % % &\c^{|\g_q}_q=\underset{\c}{\arg \min } \lim_{L\longrightarrow \infty} \frac{1}{L}\left\|\left(\sum_{q \in \mathcal{W}_k}\sigma\left(\mathbf{e}_{q} \otimes \mathbf{g}_q\right)\right)- \mathbf{D}^{H} \c\right\|_{2}^{2} \label{obj_func}
% % % \end{align}

% % % \noindent
% % % b) For every given $\c_q$, find $\F_q \in \mathbb{C}^{M_t \times N_{RF}}$ , and $\v_q \in \mathbb{C}^{N_RF}$ that satisfy $\mathbf{F}_{q}\mathbf{v}_{q} = \c_q$ the condition $\f_n \in \mathcal{B}_{M_t}$ as in equation \eqref{f_constant_gain}. 

% % % % \begin{align}
% % % %     \mathbf{F}_{\mid \g_q}\mathbf{v}_{\mid \g_q} = \c_q
% % % % \end{align}

% % % \end{problem}

% % % Note that the solution to part \textit{(a)} is the limit of the sequence of solutions to a least-square optimization problem as $L$ goes to infinity.
% % %  \begin{align}
% % % & \Tilde{\c}^{(L)}_q = (\D \D^H)^{-1} \D \sigma \left(\mathbf{e}_q \otimes \mathbf{g}_q\right) \\
% % % & \Tilde{\c}^{(L)}_q =  \sigma' \D \left(\mathbf{e}_q \otimes \mathbf{g}_q\right) = \sigma' \D_q\g_q
% % % \end{align}
% % % Dividing by $\|\Tilde{\c}^{(L)}_q\|$ and taking the limit as L goes to infinity we will find the optimal $\c_q$. i.e. 

% % % \begin{equation}
% % %     \c_q = \lim_{L\longrightarrow \infty} \frac{\Tilde{\c}^{(L)}_q}{L\|\Tilde{\c}^{(L)}_q\|}
% % % \end{equation}

% % % We aim to find $\g_q$ such that

% % % \begin{align}
% % % & \left\| abs(\D^H \gamma' \D_q \mathbf{g}_q)- abs(\gamma \left(\mathbf{e}_{q} \otimes \mathbf{g}_q\right)) \right\|_{2}^{2}
% % % \end{align}

% % % is minimized.

% % % % Is it possible to show that $\g_q$ is independent of $q$ (assume regular scenario)? If true, i.e., $\g_q = \g$ we have

% % % We will provide a suboptimal structure for the choice of $\g$. Let us assume $\g_q = \g, $  for all $ q \in \{1, \cdots, Q\}$. Then we can write
% % % \begin{align}
% % %      &\Tilde{\c_q}^{(L)} = D_q\mathbf{g} = \sum_{l=1}^{L}g_l\mathbf{d}_{M_t}(\psi_{q, l})  \nonumber\\
% % %      & = \sum_{l=1}^L g_l\left[{\begin{array}{cccc}
% % %      1& e^{j\psi_{q, l}} &\cdots & e^{j(M_t-1)\psi_{q, l}}\\
% % %      \end{array}}\right]^T  = \nonumber\\
% % %      & \left[{\begin{array}{cccc}
% % %      \sum_{l=1}^L g_l& \sum_{l=1}^L g_le^{j\psi_{q, l}} &\cdots & \sum_{l=1}^L g_le^{j(M_t-1)\psi_{q, l}}\\
% % %      \end{array}}\right]^T
% % %  \end{align}
 
 

% % %  We have for the $m$-th element of the vector $\Tilde{\c}_q$, i.e. $\Tilde{c}^{(m)}_q$  for $ 0 \leq m \leq M_t-1$

% % % \begin{equation}
% % %     \Tilde{c}_q^{(m)} = \lim_{L\longrightarrow \infty} \frac{1}{L}\sum_{l=0}^{L-1} g_l e^{j(m\psi_{q, l+1})}
% % % \end{equation}

% % % With a choice of $\g = \left[{\begin{array}{cccc} 1& \alpha^\eta &\cdots & \alpha^{\eta (L -1)} \end{array}}\right]^T$ where we set 
% % % % $\alpha = e^j(\frac{2\pi}{LQ})$
% % % $\alpha = e^j(\frac{\psi_{q}-\psi_{q-1}}{L})$ and suitable $\eta$ (to be determined later), we can write


% % % % \begin{equation}
% % % %     \Tilde{c}_q^{(m)} = \lim_{L\longrightarrow \infty} \frac{1}{L}\sum_{l=0}^{L-1} g_l e^{jm(\psi_{q-1}+\frac{2 \pi(\ell+0.5)}{LQ})}
% % % % \end{equation}

% % % \begin{equation}
% % %     \Tilde{c}_q^{(m)} = \lim_{L\longrightarrow \infty} \frac{1}{L}\sum_{l=0}^{L-1} g_l e^{jm(\psi_{q-1}+\frac{(\psi_{q}-\psi_{q-1})(\ell+0.5)}{L})}
% % % \end{equation}

% % % \begin{equation}
% % %     c_q^{(m)} = \lim_{L\longrightarrow \infty} \frac{1}{L}\sum_{l=0}^{L-1} \alpha^{(\eta+m) l} e^{jm(\psi_{q-1}+\frac{0.5(\psi_{q}-\psi_{q-1})}{L})} 
% % % \end{equation}


% % % \begin{equation}
% % %     c_q^{(m)} = e^{jm(\psi_{q-1})}\lim_{L\longrightarrow \infty} \frac{1}{L}\sum_{l=0}^{L-1} \alpha^{(\eta+m) l}   
% % % \end{equation}

% % % \begin{equation}
% % %     c_q^{(m)} = e^{jm(\psi_{q-1})} \int_{0}^{1} \alpha^{(\eta + m)Lx}dx
% % % \end{equation}


% % % % \begin{equation}
% % % %     c_q^{(m)} = e^{jm(\psi_{q-1})} \int_{0}^{1} e^{j\frac{2\pi(\eta + m)L}{LQ}x}dx
% % % % \end{equation}


% % % \begin{equation}
% % %     c_q^{(m)} = e^{jm(\psi_{q-1})} \int_{0}^{1} e^{j\xi x}dx
% % % \end{equation}

% % % where $\xi = (\psi_{q}-\psi_{q-1}) (\eta +m )$.


% % % \begin{equation}
% % %     c_q^{(m)} = e^{jm(\psi_{q-1})} \frac{e^{j\xi}-1}{j\xi}
% % % \end{equation}

% % % \begin{equation}
% % %     c_q^{(m)} = e^{j(m\psi_{q-1} + \frac{\xi}{2})} \frac{e^{j\xi/2}-e^{-j\xi/2}}{j\xi/2}
% % % \end{equation}

% % % \begin{equation}
% % %     c_q^{(m)} = e^{j(m\psi_{q-1} + \frac{\xi}{2})} sinc(\frac{\xi}{2\pi})
% % % \end{equation}

% % % where, $sinc(t) = \frac{sin(\pi t)}{\pi t}$.
















% % % % We first optimize for $\alpha$ by setting the derivative of \eqref{obj_func} to zero, to get: 

% % % % $$\hat{\alpha}=\frac{\sum_{q \in \mathcal{W}_k}\sqrt{\frac{Q}{M_{t}}}(\mathbf{F} \mathbf{v})^{H} \mathbf{D}_{q} \mathbf{g}_q}{\left\|\mathbf{D}^{H} \mathbf{F} \mathbf{v}\right\|_{2}^{2}}$$

% % % % Replacing the value of $\alpha$ as above we get to the following optimization problem. 

% % % % \begin{align}
% % % % \left(\mathbf{F}_{\mid \mathbf{g}}, \mathbf{v}_{\mid \mathbf{g}}\right) &=\underset{\mathbf{F}, \mathbf{v}}{\arg \max } \frac{\left|\sum_{q \in \mathcal{W}_k}\sqrt{\frac{2^{\mathbf{B}}}{M_{t}}}\left(\mathbf{D}_{q} \mathbf{g}_q\right)^{H} \mathbf{F} \mathbf{v}\right|^{2}}{\left\|\mathbf{D}^{H} \mathbf{F} \mathbf{v}\right\|_{2}^{2}} \nonumber \\
% % % % &=\underset{\mathbf{F}, \mathbf{v}}{\arg \max }\left|\sum_{q \in \mathcal{W}_k}\left(\mathbf{D}_{q} \mathbf{g}\right)^{H} \mathbf{F} \mathbf{v}\right|^{2}
% % % % \end{align}


% \subsection{UPA Codebook Design Prolem}

% Suppose an $M_h \times M_v$ UPA is placed at the $x-z$ plane, and let $M = M_h M_v$. Let the distance between the antennas that are placed parallel to $z$ axis, be $d_z$, and $d_x$ respectively. One can define,

% \begin{align}
% \mathbf{d}_{M_{a}}\left(\psi_{a}\right) = \left[1, e^{j \psi_{a}} \cdots e^{j\left(M_{a}-1\right) \psi_{a}}\right]^{T} \in \mathbb{C}^{M_{a}}
% \end{align}
% where, $\psi_{h}=\frac{2 \pi d_{x}}{\lambda} \sin \theta \cos \phi \text { and } \psi_{v}=\frac{2 \pi d_{z}}{\lambda} \sin \phi$, and $a$ takes value in the set $\{v, h\}$.  
% The array response vector is then defined as 

% \begin{align}
%     \mathbf{d}_{M}\left(\psi_{v}, \psi_{h}\right) =
%     \mathbf{d}_{M_{v}}\left(\psi_{v}\right) \otimes
%     \mathbf{d}_{M_{h}}\left(\psi_{h}\right)  \in \mathbb{C}^{M}
% \end{align}
% Let $\mathcal{B}_s$ be the angular range under cover. We have, 

% \begin{equation}
%     \mathcal{B}_{s} \doteq \left[-\phi^{\mathrm{B}}, \phi^{\mathrm{B}}\right)
%     \times \left[-\theta^{\mathrm{B}}, \theta^{\mathrm{B}}\right) \label{range_angle}
% \end{equation}

% Notation $\psi^{p,q}_{h}[1:L_h,1 :L_v], \psi^{p,q}_{v}[1:L_v]$

% Let us uniformly divide the angular range  into $Q=Q_{v} Q_{h}$ subregions. For each subregion we have, 

% $$ \mathcal{B}_{p, q} =  \omega_{\phi, p} \times \omega_{\theta, q}$$

% Let $\Delta_{\theta} =  2 \theta^{\mathrm{B}} / Q_{v}$, and $\Delta_\phi =  2 \phi^{\mathrm{B}} / Q_{h}$ be the length of each sub-interval along axis $\theta$, and $\phi$. We have $\omega_{\theta, q} = [\theta_{q-1}, \theta_q)$, and, $\omega_{\phi, p} = [\phi_{p-1},\phi_p)$ where  $\theta_q = -\theta^B + q\Delta_\theta$, and $\phi_p = -\phi^B + p\Delta_\phi$. 


% In the $(\psi_h, \psi_v)$ domain, we can rewrite equation \eqref{range_angle} as, 

% \begin{equation}
%     \mathcal{B}^\psi_{s} \doteq\left[-\psi_h^{\mathrm{B}}, \psi_h^{\mathrm{B}}\right) \times\left[-\psi_v^{\mathrm{B}}, \psi_v^{\mathrm{B}}\right)
% \end{equation}

% and each sub-region similarly can be rewritten in the $\psi$ domain as

% $$ \mathcal{B}^\psi_{ p, q} \doteq \nu_{v}^{p, q} \times \nu_{h}^ {p, q}$$

% % We can then write for each $v_{a, b}$, i.e the $b$-th sub-interval along axis $a$,

% % \begin{equation}
% %     \nu_{v, p} =-\psi_{a}^{\mathrm{B}}+(b-1)\Delta_{a}+\left[0, \Delta_{a}\right)
% % \end{equation}

% Define the reference gain at $(\psi_{h}, \psi_{v}) $ as, 

% \begin{equation}
%     G\left(\psi_{v}, \psi_{h}, \mathbf{c}\right) =\left|\left(\mathbf{d}_{M_{v}}\left(\psi_{v}\right)\otimes\mathbf{d}_{M_{h}}\left(\psi_{h}\right)  \right)^{H} \mathbf{c}\right|^{2}
% \end{equation}

% It is straightforward to show that, 

% \begin{equation}
%     \int_{-\pi}^{\pi} \int_{-\pi}^{\pi} G\left(\psi_{v}, \psi_{h}, \mathbf{c}\right) d \psi_{v} d \psi_{h}={(2 \pi)^{2}}
% \end{equation}

% Therefore, for every interval $\mathcal{B}_{ p, q}$ we can derive the ideal gain expression as, 

% \begin{equation}
%     G_{p, q}^{\text {ideal }}\left(\psi_{v}, \psi_{h}\right)=G_{p,q} \mathds{1}_{\mathcal{B}^\psi_{p, q}}\left(\psi_{v}, \psi_{h}\right)
% \end{equation}

% where $G_{q,p} = \frac{(2\pi)^2}{\delta_{p,q}}$, where $\delta_{p,q}$ is the area of the interval $\mathcal{B}^\psi_{p,q}$. We define the codebook design problem under the UPA structure as follows. 
% % where $G_{q,p} = \frac{(2\pi)^2}{\delta^{p,q}_h\delta^p_v}$, $\delta^p_v = \psi^p_v - \psi^{p-1}_v $, and $\delta^{q, p}_h = \psi^{q,p}_h - \psi^{q-1, p}_h $. Also, let us define $\delta_{q,p} = \delta^{q, p}_h\delta^p_v$. 



% \begin{align}
% & \c^{opt}_{p, q} = \nonumber\\&\underset{\c, \|\c\|=1}{\arg \min } \iint_{\mathcal{B}_{s}}\left|G_{p, q}^{\text {ideal }}\left(\psi_{v}, \psi_{h}\right)-G\left(\psi_{v}, \psi_{h}, \c\right)\right| d \psi_{v} d \psi_{h} \label{init_opt}
% \end{align}

% By partitioning the range of $(\psi_h, \psi_v)$ into the pre-defined intervals, we can rewrite the optimization problem as follows, 

% \begin{align}
% & \c^{opt}_{p, q} = \underset{\c, \|\c\|=1}{\arg \min }\nonumber\\& \sum_{r=1}^{Q_v}\sum_{s=1}^{Q_h}\iint_{\mathcal{B}^{\psi}_{r,s}}\left|G_{p, q}^{\text {ideal }}\left(\psi_{v}, \psi_{h}\right)-G\left(\psi_{v}, \psi_{h}, \c\right)\right| d \psi_{v} d \psi_{h} \nonumber\\&
%  = \underset{L_h, L_v \rightarrow \infty}{\lim}\sum_{r=1}^{Q_v}\sum_{s=1}^{Q_h}\sum_{l_v =1}^{L_v}\sum_{l_h =1}^{L_h}\frac{\delta^r_v\delta^{r,s,l_v}_h}{L_hL_v}\nonumber\\&
% \left|G_{p, q}^{\text {ideal }}\left(\psi^{r,s}_{v}[l_v], \psi^{r,s}_{h}[l_v][l_h]\right)-G\left(\psi^{r,s}_{v}[l_v], \psi^{r,s}_{h}[l_v][l_h], \c\right)\right| \label{alternative_opt}
% \end{align}
% where, 

% \begin{align}
%     &\psi^{r,s}_v[l_v] = \psi^{r-1, s}_v + l_v\frac{\delta_v^r}{L_v}\nonumber\\
%     &\psi^{r,s}_h[l_v][l_h] = \psi^{r, s-1}_h[l_v] + l_h\frac{\delta_h^{r,s, l_v}}{L_h}
% \end{align}

% We can rewrite the optimization problem \eqref{alternative_opt} as, 

% \begin{align}
%     \c_{p,q}^{opt}=\arg \min _{\c, \|\c\|=1} \underset{L_h, L_v \rightarrow \infty}{\lim}\frac{1}{L_hL_v}\left\|\mathbf{G}^{\text {ideal }}_{p,q}-\mathbf{G}(\c)\right\|
% \end{align}


% where, 

% \begin{align}
%     &\mathbf{G}(\c) = \nonumber\\&\left[\delta^1_v\delta^{1,1,1}_hG\left(\psi_{v}^{1,1}[1], \psi_{h}^{1,1}[1][1], \c\right) \cdots \right. \nonumber\\ &\left. \delta^{Q_v}_v\delta^{Q_v,Q_h,L_v}_hG\left(\psi_{v}^{Q_{v}, Q_{h}}\left[L_{v}\right], \psi_{h}^{Q_{v}, Q_{h}}\left[L_{v}\right]\left[L_{h}\right], \c\right)\right]^{T} 
% \end{align}

% and, 

% \begin{align}
%     &\mathbf{G}^{\text {ideal }}_{ p, q} = \nonumber\\&\left[\delta^1_v\delta^{1,1,1}_hG^{\text {ideal }}_{ p, q}\left(\psi_{v}^{1,1}[1], \psi_{h}^{1,1}[1][1] \right) \cdots \right. \nonumber\\ &\left. \delta^{Q_v}_v\delta^{Q_v,Q_h,L_v}_hG^{\text {ideal }}_{ p, q}\left(\psi_{v}^{Q_{v}, Q_{h}}\left[L_{v}\right], \psi_{h}^{Q_{v}, Q_{h}}\left[L_{v}\right]\left[L_{h}\right]\right)\right]^{T} 
% \end{align}


% % \begin{align}
% %      &\mathbf{G}^{\text {ideal }}_{ q, p}= \nonumber\\&\left[{\delta_{1,1}} G^{\text {ideal }} _{q, p}\left(\psi^{1}_{h}[1], \psi^{1}_{v}[1], \c\right) \ldots {\delta_{Q_h, Q_v}}G^{\text {ideal }} _{q, p}\left(\psi^{Q_h}_{h}[L_h], \psi^{Q_v}_{v}[L_v], \c\right)\right]^{T} 
% % \end{align}


% Note that we can write, 

% \begin{align}
%     \mathbf{G}^{\text{ideal }}_{p,q} = \frac{{(2\pi)}^2}{\delta_{p,q}} \left(\e_{p,q} \otimes (\boldsymbol{\delta}_{p,q}\odot\mathbf{1}_{L, 1})\right)
% \end{align}

% where, $\boldsymbol\delta_{p,q} = $
% with $\mathbf{e}_{p,q} \in \mathbb{Z}^{Q}$ being the standard basis vector for the $(p,q)$-th axis among $Q$ ones. 








% Now, note that $\mathbf{1}_{L, 1}=\mathbf{g} \odot \mathbf{g}^{*}$ for any equal gain $\mathbf{g} \in \mathbb{C}^L$. Therefore, we can write: 

% % $$\mathcal{G}_{L}=\left\{\mathbf{g} \in \mathbb{C}^{L}:\left(\mathbf{g} \mathbf{g}^{H}\right)_{\ell, \ell}=1,1 \leq \ell \leq L\right\}$$


% \begin{align}
% \mathbf{G}^{\text {ideal }}_{p,q} &=\frac{{2\pi}^2}{\delta_{p,q}}\left(\mathbf{e}_{p,q} \otimes\left((\boldsymbol\delta^{(\frac{1}{2})}_{p,q}\odot\boldsymbol\delta^{(\frac{1}{2})}_{p,q})\odot(\mathbf{g} \odot \mathbf{g}^{*})\right)\right) \nonumber\\
% &=\left(\frac{{2\pi}}{\sqrt{\delta_{p,q}}}\left(\mathbf{e}_{p,q} \otimes (\boldsymbol\delta^{(\frac{1}{2})}_{p,q}\odot\mathbf{g})\right)\right) \nonumber\\&\odot\left(\frac{{(2\pi)}}{\sqrt{\delta_{p,q}}}\left(\mathbf{e}_{p,q} \otimes (\boldsymbol\delta^{(\frac{1}{2})}_{p,q}\odot\mathbf{g})\right)\right)^{*} \label{g_id_q_equivalent}
% \end{align}

% Also, note that we can write, 

% % \begin{align}
% %     \mathbf{G}(\c)=\left(\left(\mathbf{D}_{h}^{H} \otimes \mathbf{D}_{v}^{H}\right) \c\right) \odot\left(\left(\mathbf{D}_{h}^{H} \otimes \mathbf{D}_{v}^{H}\right) \c\right)^{*}
% % \end{align}
% \begin{align}
%     \mathbf{G}(\c)=\left(\D^H \c\right) \odot\left(\D^H \c\right)^{*}
% \end{align}

% where 

% \begin{align}
%     \D = \left[D^{1,1}_{vh}, \cdots, D^{Q_v,Q_h}_{vh}\right] \in \mathbb{C}^{M_vM_h\times L_vL_hQ_vQ_h}
% \end{align}

% % \begin{align}
% % % &\mathbf{D}_{v} =\left[\mathbf{D}_{v, 1}, \cdots, \mathbf{D}_{v, Q_{v}}\right] \in \mathbb{C}^{M_{v} \times L_{v} Q_{v}} \\
% % % &\mathbf{D}_{h} =\left[\mathbf{D}_{h, 1}, \cdots, \mathbf{D}_{h, Q_{h}}\right] \in \mathbb{C}^{M_{h} \times L_{h} Q_{h}} \\
% % % &\mathbf{D}_{v, p} =\left[\mathbf{d}_{M_{v}}\left(\psi_{v}^{p}[1]\right), \cdots, \mathbf{d}_{M_{v}}\left(\psi_{v}^{p}\left[L_{v}\right]\right)\right] \in \mathbb{C}^{M_{v} \times L_{v}}\\
% % % &\mathbf{D}_{h, q} =\left[\mathbf{d}_{M_{h}}\left(\psi_{h}^{q}[1]\right), \cdots, \mathbf{d}_{M_{h}}\left(\psi_{h}^{q}\left[L_{h}\right]\right)\right] \in \mathbb{C}^{M_{h} \times L_{h}}\\
% % &\mathbf{D}_{h, q}(\chi) =\left[\mathbf{d}_{M_{h}}\left(\psi_{h}^{q}[1], \chi \right), \cdots, \mathbf{d}_{M_{h}}\left(\psi_{h}^{q}\left[L_{h}\right], \chi \right)\right] \in \mathbb{C}^{M_{h} \times L_{h}}\\
% % &\mathbf{D}_{vh, p,q} =\left[\mathbf{d}_{M_{v}}\left(\psi_{v}^{p}[1]\right) \otimes \mathbf{D}_{h, q}(\psi_{v}^{p}[1]), \cdots, \mathbf{d}_{M_{v}}\left(\psi_{v}^{p}\left[L_{v}\right]\right) \otimes \mathbf{D}_{h, q}(\psi_{v}^{p}\left[L_{v}\right]) \right] \in \mathbb{C}^{M_{v}M_{h} \times L_{v}L_{h}}
% % \end{align}

% \begin{align}
% % &\mathbf{D}_{v} =\left[\mathbf{D}_{v, 1}, \cdots, \mathbf{D}_{v, Q_{v}}\right] \in \mathbb{C}^{M_{v} \times L_{v} Q_{v}} \\
% % &\mathbf{D}_{h} =\left[\mathbf{D}_{h, 1}, \cdots, \mathbf{D}_{h, Q_{h}}\right] \in \mathbb{C}^{M_{h} \times L_{h} Q_{h}} \\
% % &\mathbf{D}_{v, p} =\left[\mathbf{d}_{M_{v}}\left(\psi_{v}^{p}[1]\right), \cdots, \mathbf{d}_{M_{v}}\left(\psi_{v}^{p}\left[L_{v}\right]\right)\right] \in \mathbb{C}^{M_{v} \times L_{v}}\\
% % &\mathbf{D}_{h, q} =\left[\mathbf{d}_{M_{h}}\left(\psi_{h}^{q}[1]\right), \cdots, \mathbf{d}_{M_{h}}\left(\psi_{h}^{q}\left[L_{h}\right]\right)\right] \in \mathbb{C}^{M_{h} \times L_{h}}\\
% \mathbf{D}^{p,q}_{h}(\ell) =&\sqrt{\delta_v^{p}\delta^{p,q,\ell}_h}\left[\mathbf{d}_{M_{h}}\left(\psi_{h}^{p,q}[\ell][1] \right), \cdots, \right.\nonumber\\&\left. \mathbf{d}_{M_{h}}\left(\psi_{h}^{p,q}\left[\ell] [L_{h} \right] \right)\right] \in \mathbb{C}^{M_{h} \times L_{h}}, 
% \end{align}

% \begin{align}
%     \mathbf{D}_{vh}^{p,q} =&\left[\mathbf{d}_{M_{v}}\left(\psi_{v}^{p,q}[1]\right) \otimes \mathbf{D}_{h}^{p,q}(1), \cdots, \right.\nonumber\\&\left. \mathbf{d}_{M_{v}}\left(\psi_{v}^{p,q}\left[L_{v}\right]\right) \otimes \mathbf{D}_{h}^{p,q}(L_{v}) \right] \in \mathbb{C}^{M_{v}M_{h} \times L_{v}L_{h}}
% \end{align}
% \begin{problem}
% Given an equal-gain vector $\g_{p,q} \in \mathbb{C}^L$, $(p,q) \in \{(1,1), \cdots, (Q_v, Q_h)\}$, find vector $\c_{p,q} \in \mathbb{C}^{M_t}$ such that
% \begin{align}
% &\c_{p,q}=\underset{\c, \|c\|=1}{\arg \min } \lim_{L\rightarrow \infty} \left\|\frac{{2\pi}}{\sqrt{\delta_{p,q}}}\left(\mathbf{e}_{p,q} \otimes (\boldsymbol\delta^{(\frac{1}{2})}_{L,1}\odot\mathbf{g}_{p,q})\right)- \D^H \c\right\|^{2} \label{obj_func}
% \end{align}
% \label{main_problem_UPA}
% \end{problem}


% Note that the solution to problem \ref{main_problem_UPA} is the limit of the sequence of solutions to a least-square optimization problem as $L$ goes to infinity. For each $L$ we find that,
%  \begin{align}
%  {\c}^{(L)}_{p,q} &= {\frac{{2\pi}}{\sqrt{\delta_{p,q}}}}(\D \D^H)^{-1} \D  \left(\mathbf{e}_{p,q} \otimes (\boldsymbol\delta^{(\frac{1}{2})}_{L,1}\odot\mathbf{g}_{p,q})\right) \nonumber\\& =\sigma \D^{p,q}(\boldsymbol\delta^{(\frac{1}{2})}_{L,1}\odot\mathbf{g}_{p,q})
% \end{align}
% where $\sigma = \frac{2\pi \sqrt{\delta_{p,q}}}{L\sum_{(q,p)= (1,1)}^{(Q_h, Q_v)} \delta_{p,q}}$, noting that it holds that, 

% Noting that it holds that, 

% \begin{align}
%     (\D\D^H) = \left(L\sum_{(p,q)= (1,1)}^{(Q_h, Q_v)}\sum_{(l_v,l_h)= (1,1)}^{(L_h, L_v)} \delta_{v}^p\delta^{p,q,l_v}_{h}\right)\I_{M_t}
% \end{align}

% \begin{problem}
% Find equal-gain $\g_{p,q} \in \mathbb{C}^L$, $(q,p) = (1,1), \ldots, (Q_h, Q_v)$, such that
% \begin{equation}
%  \g_{q,p} = \underset{\g}{\arg\min }\left\| abs(\D^H \c_{q,p})- {2\pi} abs(\mathbf{e}_{q,p} \otimes \mathbf{g})\right\|^{2} \label{g_final_eq}
% \end{equation} 
% where $abs(.)$ denotes the element-wise absolute value of a vector.
% \label{g_problem}
% \end{problem}


% \begin{proposition}
% The minimizer of \eqref{g_simple_final_eq} is in the form $\g_q = \left[{\begin{array}{cccc} 1& \alpha^\eta &\cdots & \alpha_h^{\eta (L_h -1)}\alpha_v^{\eta (L_v -1)} \end{array}}\right]^T$ for some $\eta$ where $\alpha_h = e^{j(\frac{\delta^{q,p}_h}{L_h})}$ and $\alpha_v = e^{j(\frac{\delta^p_v}{L_v})}$. \label{proposiiton_g}
% \end{proposition}

%  \begin{align}
%      {\c_{p,q}}^{(L)} & = \sigma \sum_{(l_v, l_h)=(1,1)}^{(L_v, L_h)}g_{p,q, l}\mathbf{d}_{M_t}\left(\psi^{p,q}_{v}[l_v], \psi^{p,q}_{h}[l_v][l_h]\right)  \nonumber\\
%      & = \sigma \sum_{l=1}^L g_{q,p,l}\left[{\begin{array}{ccc}
%      1 &\cdots & e^{j\left( (M_v-1)\psi^{p,q}_{v}[l_v] + (M_h-1)\psi^{p,q}_{h}[l_v][l_h]\right)}\\
%      \end{array}}\right]^T   
%     %  \nonumber\\& = \left[{\begin{array}{ccc}
%     %  \sigma\sum_{l=1}^L g_{q, p, l}& \cdots & \sigma\sum_{l=1}^L g_{q,p, l}e^{j\left((M_h-1)\psi^q_{h}[l] + (M_v-1)\psi^p_{v}[l]\right)}\\
%     %  \end{array}}\right]^T
% \end{align}

% We can then write 

% \begin{align}
%     c_{p,q, m_v, m_h} &= \lim_{L_h, L_v\rightarrow \infty} \frac{1}{L_hL_v}\sum_{(l_h, l_v)=(1,1)}^{(L_h, L_v)} \nonumber \\&\delta^p_v\delta_h^{p,q,l_v}g_{p,q, l_v, l_h}e^{j\left(m_v\psi^{p,q}_{v}[l_v]+ m_h\psi^{p,q}_{h}[l_v][l_h] \right)}
% \end{align}


% \begin{align}
%     c_{p,q, m_v, m_h} &=  \nonumber\\&\lim_{L_h, L_v\rightarrow \infty} \frac{1}{L_hL_v}\sum_{(l_h, l_v)=(1,1)}^{(L_h, L_v)} \delta^p_v\delta_h^{p,q,l_v}g_{p,q, l_v, l_h}\nonumber \\&e^{j\left(m_v(\psi^{p-1,q}_{v}+l_v\frac{\delta^p_v}{L_v})+ m_h(\psi^{p,q-1}_{h}[l_v] + l_h\frac{\delta^{p,q,l_v}_h}{L_h}) \right)}
% \end{align}


% \begin{align}
%     c_{p,q, m_v, m_h} &=  \lim_{ L_v\rightarrow \infty} \frac{\delta^p_v}{L_v}\sum_{l_v=1}^{L_v} e^{j(m_v(\psi^{p-1,q}_{v}+l_v\frac{\delta^p_v}{L_v}) + x\eta\frac{l_v}{L_v})} S^{(l_v)}
% \end{align}

% where, 

% \begin{align}
%     S^{(l_v)} &= \lim_{ L_h\rightarrow \infty}\frac{\delta^{p,q,l_v}_h}{L_h}\sum_{l_h=1}^{L_h}e^{j(m_h(\psi^{p,q-1}_{h}[l_v] + l_h\frac{\delta^{p,q,l_v}_h}{L_h}) + y\gamma \frac{l_h}{L_h})} \nonumber\\& = \delta^{p,q,l_v}_h e^{jm_h\psi^{p,q-1}_{h}[l_v]}sinc(\frac{\xi}{2\pi})
% \end{align}

% with $\xi = {\delta^{p,q,l_v}_h}{m} + y\gamma$. Therefore, it. holds that, 

% \begin{align}
%     c_{p,q, m_v, m_h} &=  \lim_{ L_v\rightarrow \infty} \frac{\delta^p_v}{L_v}\sum_{l_v=1}^{L_v} e^{j(m_v(\psi^{p-1,q}_{v}+l_v\frac{\delta^p_v}{L_v}) + x\eta\frac{l_v}{L_v})} \delta^{p,q,l_v}_h e^{jm_h\psi^{p,q-1}_{h}[l_v]}sinc(\frac{\xi}{2\pi})
% \end{align}
% \subsection{UPA Codebook Design Prolem}

Suppose an $M_h \times M_v$ UPA is placed at the $x-z$ plane, and let $M = M_h M_v$. Let the distance between the antennas that are placed parallel to $z$ axis, be $d_z$, and $d_x$ respectively. One can define,

\begin{align}
\mathbf{d}_{M_{a}}\left(\psi_{a}\right) = \left[1, e^{j \psi_{a}} \cdots e^{j\left(M_{a}-1\right) \psi_{a}}\right]^{T} \in \mathbb{C}^{M_{a}}
\end{align}
where, $\psi_{h}=\frac{2 \pi d_{x}}{\lambda} \sin \theta \cos \phi \text { and } \psi_{v}=\frac{2 \pi d_{z}}{\lambda} \sin \phi$, and $a$ takes value in the set $\{v, h\}$.  
The array response vector is then defined as 

\begin{align}
    \mathbf{d}_{M}\left(\psi_{v}, \psi_{h}\right) =
    \mathbf{d}_{M_{v}}\left(\psi_{v}\right) \otimes
    \mathbf{d}_{M_{h}}\left(\psi_{h}\right)  \in \mathbb{C}^{M}
\end{align}
Let $\mathcal{B}_s$ be the angular range under cover in the $(\psi_h, \psi_v)$ domain. We have, 

\begin{equation}
    \mathcal{B}^\psi_{s} \doteq\left[-\psi_h^{\mathrm{B}}, \psi_h^{\mathrm{B}}\right) \times\left[-\psi_v^{\mathrm{B}}, \psi_v^{\mathrm{B}}\right)
\end{equation}

% \begin{equation}
%     \mathcal{B}_{s} \doteq \left[-\phi^{\mathrm{B}}, \phi^{\mathrm{B}}\right)
%     \times \left[-\theta^{\mathrm{B}}, \theta^{\mathrm{B}}\right) \label{range_angle}
% \end{equation}

% Notation $\psi^{p,q}_{h}[1:L_h,1 :L_v], \psi^{p,q}_{v}[1:L_v]$

Let us uniformly divide the angular range  into $Q=Q_{v} Q_{h}$ subregions. For each subregion we have, 

% $$ \mathcal{B}_{p, q} =  \omega_{\phi, p} \times \omega_{\theta, q}$$

% Let $\Delta_{\theta} =  2 \theta^{\mathrm{B}} / Q_{v}$, and $\Delta_\phi =  2 \phi^{\mathrm{B}} / Q_{h}$ be the length of each sub-interval along axis $\theta$, and $\phi$. We have $\omega_{\theta, q} = [\theta_{q-1}, \theta_q)$, and, $\omega_{\phi, p} = [\phi_{p-1},\phi_p)$ where  $\theta_q = -\theta^B + q\Delta_\theta$, and $\phi_p = -\phi^B + p\Delta_\phi$. 


$$ \mathcal{B}_{ p, q} \doteq \nu_{v}^{p, q} \times \nu_{h}^ {p, q}$$

% We can then write for each $v_{a, b}$, i.e the $b$-th sub-interval along axis $a$,

% \begin{equation}
%     \nu_{v, p} =-\psi_{a}^{\mathrm{B}}+(b-1)\Delta_{a}+\left[0, \Delta_{a}\right)
% \end{equation}

Define the reference gain at $(\psi_{h}, \psi_{v}) $ as, 

\begin{equation}
    G\left(\psi_{v}, \psi_{h}, \mathbf{c}\right) =\left|\left(\mathbf{d}_{M_{v}}\left(\psi_{v}\right)\otimes\mathbf{d}_{M_{h}}\left(\psi_{h}\right)  \right)^{H} \mathbf{c}\right|^{2}
\end{equation}

It is straightforward to show that, 

\begin{equation}
    \int_{-\pi}^{\pi} \int_{-\pi}^{\pi} G\left(\psi_{v}, \psi_{h}, \mathbf{c}\right) d \psi_{v} d \psi_{h}={(2 \pi)^{2}}
\end{equation}

Therefore, for every interval $\mathcal{B}_{ p, q}$ we can derive the ideal gain expression as, 

\begin{equation}
    G_{p, q}^{\text {ideal }}\left(\psi_{v}, \psi_{h}\right)=G_{p,q} \mathds{1}_{\mathcal{B}^\psi_{p, q}}\left(\psi_{v}, \psi_{h}\right)
\end{equation}

where $G_{q,p} = \frac{(2\pi)^2}{\delta_{p,q}}$, where $\delta_{p,q}$ is the area of the interval $\mathcal{B}^\psi_{p,q}$. We can write $\delta_{p,q} = \delta_v\delta_h = (\frac{2\psi^B_v}{Q_v})(\frac{2\psi^B_h}{Q_h})$. Therefore, We define the codebook design problem under the UPA structure as follows. 
% where $G_{q,p} = \frac{(2\pi)^2}{\delta^{p,q}_h\delta^p_v}$, $\delta^p_v = \psi^p_v - \psi^{p-1}_v $, and $\delta^{q, p}_h = \psi^{q,p}_h - \psi^{q-1, p}_h $. Also, let us define $\delta_{q,p} = \delta^{q, p}_h\delta^p_v$. 



\begin{align}
& \c^{opt}_{p, q} = \nonumber\\&\underset{\c, \|\c\|=1}{\arg \min } \iint_{\mathcal{B}_{s}}\left|G_{p, q}^{\text {ideal }}\left(\psi_{v}, \psi_{h}\right)-G\left(\psi_{v}, \psi_{h}, \c\right)\right| d \psi_{v} d \psi_{h} \label{init_opt}
\end{align}

By partitioning the range of $(\psi_h, \psi_v)$ into the pre-defined intervals, we can rewrite the optimization problem as follows, 

\begin{align}
& \c^{opt}_{p, q} = \underset{\c, \|\c\|=1}{\arg \min }\nonumber\\& \sum_{r=1}^{Q_v}\sum_{s=1}^{Q_h}\iint_{\mathcal{B}^{\psi}_{r,s}}\left|G_{p, q}^{\text {ideal }}\left(\psi_{v}, \psi_{h}\right)-G\left(\psi_{v}, \psi_{h}, \c\right)\right| d \psi_{v} d \psi_{h} \nonumber\\&
 = \underset{L_h, L_v \rightarrow \infty}{\lim}\sum_{r=1}^{Q_v}\sum_{s=1}^{Q_h}\sum_{l_v =1}^{L_v}\sum_{l_h =1}^{L_h}\frac{\delta_v\delta_h}{L_hL_v}\nonumber\\&
\left|G_{p, q}^{\text {ideal }}\left(\psi^{r,s}_{v}[l_v], \psi^{r,s}_{h}[l_h]\right)-G\left(\psi^{r,s}_{v}[l_v], \psi^{r,s}_{h}[l_h], \c\right)\right| \label{alternative_opt}
\end{align}
where, 

\begin{align}
    &\psi^{r,s}_v[l_v] = \psi^{r-1, s}_v + l_v\frac{\delta_v}{L_v}\nonumber\\
    &\psi^{r,s}_h[l_h] = \psi^{r, s-1}_h + l_h\frac{\delta_h}{L_h}
\end{align}

We can rewrite the optimization problem \eqref{alternative_opt} as, 

\begin{align}
    \c_{p,q}^{opt}=\arg \min _{\c, \|\c\|=1} \underset{L_h, L_v \rightarrow \infty}{\lim}\frac{1}{L_hL_v}\left\|\mathbf{G}^{\text {ideal }}_{p,q}-\mathbf{G}(\c)\right\|
\end{align}


where, 

\begin{align}
    \mathbf{G}(\c) =& \delta_v\delta_h\left[G\left(\psi_{v}^{1,1}[1], \psi_{h}^{1,1}[1], \c\right) \cdots \right. \nonumber\\ &\left. G\left(\psi_{v}^{Q_{v}, Q_{h}}\left[L_{v}\right], \psi_{h}^{Q_{v}, Q_{h}}\left[L_{h}\right], \c\right)\right]^{T} 
\end{align}

and, 

\begin{align}
    \mathbf{G}^{\text {ideal }}_{ p, q} =& \delta_v\delta_h\left[G^{\text {ideal }}_{ p, q}\left(\psi_{v}^{1,1}[1], \psi_{h}^{1,1}[1] \right) \cdots \right. \nonumber\\ &\left. G^{\text {ideal }}_{ p, q}\left(\psi_{v}^{Q_{v}, Q_{h}}\left[L_{v}\right], \psi_{h}^{Q_{v}, Q_{h}}\left[L_{h}\right]\right)\right]^{T} 
\end{align}


% \begin{align}
%      &\mathbf{G}^{\text {ideal }}_{ q, p}= \nonumber\\&\left[{\delta_{1,1}} G^{\text {ideal }} _{q, p}\left(\psi^{1}_{h}[1], \psi^{1}_{v}[1], \c\right) \ldots {\delta_{Q_h, Q_v}}G^{\text {ideal }} _{q, p}\left(\psi^{Q_h}_{h}[L_h], \psi^{Q_v}_{v}[L_v], \c\right)\right]^{T} 
% \end{align}


Note that we can write, 

\begin{align}
    \mathbf{G}^{\text{ideal }}_{p,q} = \delta_v\delta_h\frac{{(2\pi)}^2}{\delta_v\delta_h} \left(\e_{p,q} \otimes \mathbf{1}_{L, 1}\right)
\end{align}


with $\mathbf{e}_{p,q} \in \mathbb{Z}^{Q}$ being the standard basis vector for the $(p,q)$-th axis among $Q$ ones. 








Now, note that $\mathbf{1}_{L, 1}=\mathbf{g} \odot \mathbf{g}^{*}$ for any equal gain $\mathbf{g} \in \mathbb{C}^L$. Therefore, we can write: 

% $$\mathcal{G}_{L}=\left\{\mathbf{g} \in \mathbb{C}^{L}:\left(\mathbf{g} \mathbf{g}^{H}\right)_{\ell, \ell}=1,1 \leq \ell \leq L\right\}$$


\begin{align}
\mathbf{G}^{\text {ideal }}_{p,q} &={{(2\pi)}^2}\left(\mathbf{e}_{p,q} \otimes(\mathbf{g} \odot \mathbf{g}^{*})\right) \nonumber\\
&=\left({{2\pi}}\left(\mathbf{e}_{p,q} \otimes \mathbf{g}\right)\right) \odot \left({{2\pi}}\left(\mathbf{e}_{p,q} \otimes \mathbf{g}\right)\right)^* \label{g_id_q_equivalent}
\end{align}

Also, note that we can write, 

% \begin{align}
%     \mathbf{G}(\c)=\left(\left(\mathbf{D}_{h}^{H} \otimes \mathbf{D}_{v}^{H}\right) \c\right) \odot\left(\left(\mathbf{D}_{h}^{H} \otimes \mathbf{D}_{v}^{H}\right) \c\right)^{*}
% \end{align}
\begin{align}
    \mathbf{G}(\c)=\left(\D^H \c\right) \odot\left(\D^H \c\right)^{*}
\end{align}

\noindent where, $\D^H = \sqrt{\delta_v\delta_h}(\mathbf{D}_{v}^{H} \otimes \mathbf{D}_{h}^{H})$, and for $a \in \{v,h\}$, we have, 

\begin{align}
\mathbf{D}_{a} &=\left[\mathbf{D}_{a, 1}, \cdots, \mathbf{D}_{a, Q_{a}}\right] \in \mathbb{C}^{M_{a} \times L_{a} Q_{a}} \\
\mathbf{D}_{a, b} &=\left[\mathbf{d}_{M_{a}}\left(\psi_{a}^{b}[1]\right), \cdots, \mathbf{d}_{M_{a}}\left(\psi_{a}^{b}\left[L_{a}\right]\right)\right] \in \mathbb{C}^{M_{a} \times L_{a}}
\end{align}




% where 

% \begin{align}
%     \D = \left[D^{1,1}_{vh}, \cdots, D^{Q_v,Q_h}_{vh}\right] \in \mathbb{C}^{M_vM_h\times L_vL_hQ_vQ_h}
% \end{align}

% \begin{align}
% % &\mathbf{D}_{v} =\left[\mathbf{D}_{v, 1}, \cdots, \mathbf{D}_{v, Q_{v}}\right] \in \mathbb{C}^{M_{v} \times L_{v} Q_{v}} \\
% % &\mathbf{D}_{h} =\left[\mathbf{D}_{h, 1}, \cdots, \mathbf{D}_{h, Q_{h}}\right] \in \mathbb{C}^{M_{h} \times L_{h} Q_{h}} \\
% % &\mathbf{D}_{v, p} =\left[\mathbf{d}_{M_{v}}\left(\psi_{v}^{p}[1]\right), \cdots, \mathbf{d}_{M_{v}}\left(\psi_{v}^{p}\left[L_{v}\right]\right)\right] \in \mathbb{C}^{M_{v} \times L_{v}}\\
% % &\mathbf{D}_{h, q} =\left[\mathbf{d}_{M_{h}}\left(\psi_{h}^{q}[1]\right), \cdots, \mathbf{d}_{M_{h}}\left(\psi_{h}^{q}\left[L_{h}\right]\right)\right] \in \mathbb{C}^{M_{h} \times L_{h}}\\
% &\mathbf{D}_{h, q}(\chi) =\left[\mathbf{d}_{M_{h}}\left(\psi_{h}^{q}[1], \chi \right), \cdots, \mathbf{d}_{M_{h}}\left(\psi_{h}^{q}\left[L_{h}\right], \chi \right)\right] \in \mathbb{C}^{M_{h} \times L_{h}}\\
% &\mathbf{D}_{vh, p,q} =\left[\mathbf{d}_{M_{v}}\left(\psi_{v}^{p}[1]\right) \otimes \mathbf{D}_{h, q}(\psi_{v}^{p}[1]), \cdots, \mathbf{d}_{M_{v}}\left(\psi_{v}^{p}\left[L_{v}\right]\right) \otimes \mathbf{D}_{h, q}(\psi_{v}^{p}\left[L_{v}\right]) \right] \in \mathbb{C}^{M_{v}M_{h} \times L_{v}L_{h}}
% \end{align}

% \begin{align}
% % &\mathbf{D}_{v} =\left[\mathbf{D}_{v, 1}, \cdots, \mathbf{D}_{v, Q_{v}}\right] \in \mathbb{C}^{M_{v} \times L_{v} Q_{v}} \\
% % &\mathbf{D}_{h} =\left[\mathbf{D}_{h, 1}, \cdots, \mathbf{D}_{h, Q_{h}}\right] \in \mathbb{C}^{M_{h} \times L_{h} Q_{h}} \\
% % &\mathbf{D}_{v, p} =\left[\mathbf{d}_{M_{v}}\left(\psi_{v}^{p}[1]\right), \cdots, \mathbf{d}_{M_{v}}\left(\psi_{v}^{p}\left[L_{v}\right]\right)\right] \in \mathbb{C}^{M_{v} \times L_{v}}\\
% % &\mathbf{D}_{h, q} =\left[\mathbf{d}_{M_{h}}\left(\psi_{h}^{q}[1]\right), \cdots, \mathbf{d}_{M_{h}}\left(\psi_{h}^{q}\left[L_{h}\right]\right)\right] \in \mathbb{C}^{M_{h} \times L_{h}}\\
% \mathbf{D}^{p,q}_{h}(\ell) =&\sqrt{\delta_v^{p}\delta^{p,q,\ell}_h}\left[\mathbf{d}_{M_{h}}\left(\psi_{h}^{p,q}[\ell][1] \right), \cdots, \right.\nonumber\\&\left. \mathbf{d}_{M_{h}}\left(\psi_{h}^{p,q}\left[\ell] [L_{h} \right] \right)\right] \in \mathbb{C}^{M_{h} \times L_{h}}, 
% \end{align}

% \begin{align}
%     \mathbf{D}_{vh}^{p,q} =&\left[\mathbf{d}_{M_{v}}\left(\psi_{v}^{p,q}[1]\right) \otimes \mathbf{D}_{h}^{p,q}(1), \cdots, \right.\nonumber\\&\left. \mathbf{d}_{M_{v}}\left(\psi_{v}^{p,q}\left[L_{v}\right]\right) \otimes \mathbf{D}_{h}^{p,q}(L_{v}) \right] \in \mathbb{C}^{M_{v}M_{h} \times L_{v}L_{h}}
% \end{align}

\begin{problem}
Given an equal-gain vector $\g_{p,q} \in \mathbb{C}^L$, $(p,q) \in \{(1,1), \cdots, (Q_v, Q_h)\}$, find vector $\c_{p,q} \in \mathbb{C}^{M_t}$ such that
\begin{align}
&\c_{p,q}=\underset{\c, \|c\|=1}{\arg \min } \lim_{L\rightarrow \infty} \left\|{{2\pi}}\left(\mathbf{e}_{p,q} \otimes \mathbf{g}\right)- \D^H \c\right\|^{2} \label{obj_func}
\end{align}
\label{main_problem_UPA}
\end{problem}


Note that the solution to problem \ref{main_problem_UPA} is the limit of the sequence of solutions to a least-square optimization problem as $L$ goes to infinity. For each $L$ we find that,
 \begin{align}
 {\c}^{(L)}_{p,q} &= {{{2\pi}}}(\D \D^H)^{-1} \D  \left(\mathbf{e}_{p,q} \otimes \mathbf{g}_{p,q}\right) \nonumber\\& =\sigma \D_{p,q}\mathbf{g}_{p,q}
\end{align}
where $\sigma = \frac{2\pi \sqrt{\delta_{v}\delta_{h}}}{LQ\delta_{v}\delta_{h}} = \frac{2\pi}{LQ\sqrt{\delta_v\delta_h}}$, noting that it holds that, 

Noting that it holds that, 

\begin{align}
    (\D\D^H) = \left(L\sum_{(p,q)= (1,1)}^{(Q_h, Q_v)}\sum_{(l_v,l_h)= (1,1)}^{(L_h, L_v)} \delta_{v}\delta_{h}\right)\I_{M_t}
\end{align}

\begin{problem}
Find equal-gain $\g_{p,q} \in \mathbb{C}^L$, $(p,q) = (1,1), \ldots, (Q_h, Q_v)$, such that
\begin{equation}
 \g_{q,p} = \underset{\g}{\arg\min }\left\| abs(\D^H \c_{p,q})- {2\pi} abs(\mathbf{e}_{p,q} \otimes \mathbf{g})\right\|^{2} \label{g_final_eq}
\end{equation} 
where $abs(.)$ denotes the element-wise absolute value of a vector.
\label{g_problem}
\end{problem}


\begin{proposition}
The minimizer of \eqref{g_simple_final_eq} is in the form $\g_q = \left[{\begin{array}{cccc} 1& \alpha^\eta &\cdots & \alpha_v^{\eta (L_v -1)}\alpha_h^{\eta (L_h -1)} \end{array}}\right]^T$ for some $\eta$ where $\alpha_v = e^{j(\frac{\eta_v}{L_v})}$ and $\alpha_h = e^{j(\frac{\eta_h}{L_h})}$. \label{proposiiton_g}
\end{proposition}

 \begin{align}
     {\c_{p,q}}^{(L)} & = \sigma \sum_{(l_v, l_h)=(1,1)}^{(L_v, L_h)}g_{p,q, l}\mathbf{d}_{M_t}\left(\psi^{p}_{v}[l_v], \psi^{q}_{h}[l_h]\right)  \nonumber\\
     & = \sigma \sum_{l=1}^L g_{q,p,l}\left[{\begin{array}{ccc}
     1 &\cdots & e^{j\phi_{p,q}^{M_v-1, M_h-1}}\\
     \end{array}}\right]^T   
    %  \nonumber\\& = \left[{\begin{array}{ccc}
    %  \sigma\sum_{l=1}^L g_{q, p, l}& \cdots & \sigma\sum_{l=1}^L g_{q,p, l}e^{j\left((M_h-1)\psi^q_{h}[l] + (M_v-1)\psi^p_{v}[l]\right)}\\
    %  \end{array}}\right]^T
\end{align}

where $ \phi_{p,q}^{m_v, m_h} = \left( m_v\psi^{p}_{v}[l_v] + m_h\psi^{q}_{h}[l_h]\right)$. We can then write 

\begin{align}
    c_{p,q, m_v, m_h} &= \nonumber\\& \lim_{L_h, L_v\rightarrow \infty} \frac{1}{L_hL_v}\sum_{(l_h, l_v)=(1,1)}^{(L_h, L_v)} g_{p,q, l_v, l_h}e^{j\phi_{p,q}^{M_v-1, M_h-1}}
\end{align}


\begin{align}
    c_{p,q, m_v, m_h} &=  \lim_{L_h, L_v\rightarrow \infty} \frac{1}{L_hL_v}\sum_{(l_h, l_v)=(1,1)}^{(L_h, L_v)} g_{p,q, l_v, l_h}\nonumber \\&e^{j\left(m_v(\psi^{p-1}_{v}+l_v\frac{\delta_v}{L_v})+ m_h(\psi^{q-1}_{h} + l_h\frac{\delta_h}{L_h}) \right)}
\end{align}


\begin{align}
    c_{p,q, m_v, m_h} =&  \frac{2\pi}{Q}e^{j\phi_{p-1, q-1}^{m_v, m_h}}
    \left(\frac{1}{L_v}\lim_{ L_v\rightarrow \infty} \sum_{l_v=1}^{L_v} e^{j\frac{\eta_v+ m_v\delta_v}{L_v} l_v}\right) \nonumber\\&
    \left(\frac{1}{L_h}\lim_{ L_h\rightarrow \infty} \sum_{l_h=1}^{L_h} e^{j\frac{\eta_h+ m_h\delta_h}{L_h} l_h}\right)
\end{align}

\begin{align}
    c_{p,q, m_v, m_h} &=  \frac{2\pi}{Q}e^{j\phi_{p-1, q-1}^{m_v, m_h}}
    \int_{0}^{1} e^{j\xi_v x}dx\int_{0}^{1} e^{j\xi_h x}dx \nonumber\\&
    = \frac{2\pi}{Q}e^{j(\phi_{p-1, q-1}^{m_v, m_h}+ \frac{\xi_v+\xi_h}{2})} sinc(\frac{\xi_v}{2\pi})sinc(\frac{\xi_h}{2\pi})
\end{align}

with $\xi_a = {\delta_a}{m_a} + \eta_a$, where $a\in \{v,h\}$. 
% Therefore, it. holds that, 

% \begin{align}
%     c_{p,q, m_v, m_h} &=  \lim_{ L_v\rightarrow \infty} \frac{\delta^p_v}{L_v}\sum_{l_v=1}^{L_v} e^{j(m_v(\psi^{p-1,q}_{v}+l_v\frac{\delta^p_v}{L_v}) + x\eta\frac{l_v}{L_v})} \delta^{p,q,l_v}_h e^{jm_h\psi^{p,q-1}_{h}[l_v]}sinc(\frac{\xi}{2\pi}) 
% \end{align}

% \subsection{UPA Dual Beamforming}

% Suppose an $M_h \times M_v$ UPA is placed at the $x-z$ plane, and let $M = M_h M_v$. Let the distance between the antennas that are placed parallel to $z$ axis, be $d_z$, and $d_x$ respectively. One can define,

% \begin{align}
% \mathbf{d}_{M_{a}}\left(\psi_{a}\right) = \left[1, e^{j \psi_{a}} \cdots e^{j\left(M_{a}-1\right) \psi_{a}}\right]^{T} \in \mathbb{C}^{M_{a}}
% \end{align}
% where, $\zeta=\frac{2 \pi d_{x}}{\lambda} \sin \theta \cos \phi \text { and } \xi=\frac{2 \pi d_{z}}{\lambda} \sin \phi$, and $a$ takes value in the set $\{v, h\}$.  
% The array response vector is then defined as 

% \begin{align}
%     \mathbf{d}_{M}\left(\xi, \zeta\right) =
%     \mathbf{d}_{M_{v}}\left(\xi\right) \otimes
%     \mathbf{d}_{M_{h}}\left(\zeta\right)  \in \mathbb{C}^{M}
% \end{align}
% Define the reference gain at $(\zeta, \xi) $ as, 

% \begin{equation}
%     G\left(\xi, \zeta, \mathbf{c}\right) =\left|\left(\mathbf{d}_{M_{v}}\left(\xi\right)\otimes\mathbf{d}_{M_{h}}\left(\zeta\right)  \right)^{H} \mathbf{c}\right|^{2}
% \end{equation}



Prior to formulating the multi-beamforming design problem, we proceed with a few preliminary definitions. Let us define the \emph{multi-beam}  $\mathcal{D} =(\mathcal{D}_1, \ldots \mathcal{D}_k)$ as collection of $k$ \emph{compound beams} $\mathcal{D}_i, i = 1,\ldots, k$ where $\mathcal{D}_i \subseteq \mathcal{B}^{\psi}$ and $\mathcal{D}_i = {\bigcup}_{{(p,q) \in \mathcal{A}_i}} \mathcal{B}^{\psi}_{p,q}$, with $\mathcal{A}_i$ being the set of all pairs $(p,q)$ that all beams $\mathcal{B}^{\psi}_{p,q}$ cover $\mathcal{D}_i$. The union of $\mathcal{B}^{\psi}_{p,q}$ is in fact approximating the shape of the solid angle for the desired compound beam corresponding to $\mathcal{D}_i$. By using larger number of division, i.e., finer beams, one can make the approximation better. We have 
\begin{align}
    &\mathcal{A}_i = \underset{\{\mathcal{\hat{A}}|\mathcal{D}_i \subseteq \underset{(p,q) \in \mathcal{A}}{\bigcup} \mathcal{B}_{p,q}\}}{\arg\min} |\mathcal{\hat{A}}|
\end{align}
% where, $i\in \{1,2\}$ and define further $\mathcal{A} = (\mathcal{A}_1, \mathcal{A}_2)$. 

Further define $\mathcal{A} = {{\bigcup}_{i=1}^k}\mathcal{A}_i$. 
%Let $\c$ denote the configuration of RIS (a.k.a beamformer) to control the gain and the phase of the exchanged signals, i.e. $c_{m_v, m_h} = \beta_{m_v, m_h}e^{j\theta_{m_v, m_h}}$, $m_v = 1\ldots M_v$, $m_h = 1\ldots M_h$. 
%
% Let $\c = \mbox{diag}( \Theta )$ denote a vector of length $M$ consisting of the diagonal elements of the matrix $\Theta$. For antenna element located at $(m_v, m_h)$ in the ULA grid, we define $c_{m_v, m_h} = \beta_{m_v, m_h}e^{j\theta_{m_v, m_h}}$, $m_v = 0, \ldots,  M_v-1$, $m_h = 0, \ldots, M_h-1$, and hence the vector $\c$ is given by
% \begin{equation}
%     \c = [c_{0,0}, \ldots, c_{0,M_h-1}, c_{1,0}, \ldots, c_{M_v-1, M_h-1}]
% \end{equation}
%
We aim to design a beamforming vector $\c$ such that the multi-beam $\mathcal{D}$ is covered when the RIS is excited by an incident wave received at solid angle $\Omega_1$. Using (\ref{channel})-(\ref{channel_t}), the contribution of the RIS in the channel matrix for a receiver at the solid angle $\Omega_2$ is given by
% \begin{align}
%     \Gamma = \d^{H}_{M}\{\Omega_{1}\} \Theta \d_{M}\{\Omega_{2}\} = \boldsymbol\lambda^H \d_{M}\{\Omega_{2}\} 
% \end{align}
\begin{align}
    \Gamma = \a^{H}_{M}(\Omega_2) \Theta \a_{M}(\Omega_1) = \d^H_M(\Omega_2) \boldsymbol{\lambda}  
\end{align}
where  $\boldsymbol{\lambda} \in \mathbb{C}^M$  is defined as follows. For antenna element located at position $(m_v, m_h)$ in the UPA grid,  we have 
\begin{align}
    \lambda_{m_v, m_h} = \beta_{m_v, m_h}e^{-j(\theta_{m_v, m_h}- m_v\xi_{1} - m_h\zeta_{1})}
\end{align} 
where $(\xi_{1}, \zeta_{1})$ is the representation of $\Omega_1$ in the $\psi$-domain, and hence the vector $\boldsymbol{\lambda}$ is given by
\begin{equation}
    \boldsymbol{\lambda} = [\lambda_{0,0}, \ldots, \lambda_{0,M_h-1}, \lambda_{1,0}, \ldots, \lambda_{M_v-1, M_h-1}]
\end{equation}

We note that $\boldsymbol{\lambda}$ depends on the AoA of the incident beams at the RIS, i.e., $\Omega_1$, as well as the RIS parameters. The reference gain of RIS in direction $(\zeta, \xi) $ in terms of $\boldsymbol{\lambda}$ is given by 
\begin{align}
    G\left(\xi, \zeta, \boldsymbol{\lambda} \right) =\left|\left(\mathbf{d}_{M_{v}}\left(\xi\right)\otimes\mathbf{d}_{M_{h}}\left(\zeta\right)  \right)^{H} \boldsymbol{\lambda} \right|^{2}
\end{align}

% The RIS parameters, i.e., the phase shift $\theta_m$ and the attenuation value $\beta_m$ for the $m^{th}$ elements of the RIS, are obtained by by using the corresponding coefficients of the vector $\boldsymbol{\lambda}$ and the directivity vector $\a_{M}(\Omega_{1})$ for any receive incident solid angle $\Omega_1$. To design $\boldsymbol{\lambda}$, we first define and work on the normalized beamforming vector $\c = \frac{\boldsymbol{\lambda}}{\|\boldsymbol{\lambda}\|}$, and then we compute $\boldsymbol{\lambda} = \frac{\c}{\|\c\|_{\infty}}$ based on the obtained value for $\c$. 

On the other hand, the gain of UPA antenna with the feed coefficients $\c$ is given by
\begin{equation}
    G\left(\xi, \zeta, \c \right) =\left|\left(\mathbf{d}_{M_{v}}\left(\xi\right)\otimes\mathbf{d}_{M_{h}}\left(\zeta\right)  \right)^{H} \c \right|^{2} \label{UPA_beamforming_c}
\end{equation}
that has a clear similarity.
%where $\|\c\| \leq 1$.
This means that to design the RIS-UPA for the STMR problem with receive zone $\mathcal{D}$ we can use the multi-beamforming design framework to cover the ACI's included in $\mathcal{D}$ for the UPA antenna. In particular, a RIS-UPA with parameters $\boldsymbol{\lambda}$ and a UPA-antenna with beamforming parameters $\c$ have the same beamforming gain pattern if UPA structures are the same and $\boldsymbol{\lambda}=\c$. Hence, a RIS-UPA which is excited from the solid angle $\Omega_1$ has the same beamforming gain as its UPA antenna counterpart if $\boldsymbol{\Theta}= \mbox{diag} \{\c^T \odot \a_M^H(\Omega_1)\}$.
%\amir{Please note that in the equation of the reference gain, $\d$ and $\c$ are both normalized with respect to the maximum values of the gain to be one. This means that the norm of every element of the directivity vector $\d$ is one. Also, the gain of every element of the beamforming coefficient is also less than or equal to one since the RIS is assumed to be passive.}
For any normalized beamforming vector $\c$, it is straightforward to show that, 
% Let us define the \emph{dual beam}  $ \mathcal{D} =(\mathcal{D}_1, \mathcal{D}_2) , \text{ given } \mathcal{D}_1 , \mathcal{D}_2\subseteq \mathcal{B}_s$. Let $\mathcal{A}_1$, and $\mathcal{A}_2$ denote the smallest set of index pairs $(p,q)$ the corresponding beams to which collectively cover the area marked by the desired beams $\mathcal{D}_1$ and $\mathcal{D}_2$ respectively. More precisely, we can write 
% where $c^{\mathrm{D}}_1 = (\phi^{\mathrm{D}}_1, \theta^{\mathrm{D}}_1)$ and $c^{\mathrm{D}}_2 = (\phi^{\mathrm{D}}_2, \theta^{\mathrm{D}}_2)$ are the directions of the centers of the two beams respectively.
% \begin{align}
%     &\mathcal{A}_i = \underset{\{\mathcal{A}|\mathcal{D}_i \subseteq \underset{(p,q) \in \mathcal{A}}{\bigcup} \mathcal{B}_{p,q}\}}{\arg\min} Card(\mathcal{A})
% \end{align}
% where, $i\in \{1,2\}$ and define further $\mathcal{A} = (\mathcal{A}_1, \mathcal{A}_2)$. 
\begin{equation}
    \int_{-\pi}^{\pi} \int_{-\pi}^{\pi} G\left(\xi, \zeta, \mathbf{c}\right) d \xi d \zeta={(2 \pi)^{2}}
\end{equation}

We wish to design beamformers that provide high, sharp, and constant gain within the desired ACI's and zero gain everywhere else. We have then for the ideal gain corresponding to such beamformer $\c$ that,
\begin{align}
&\iint_{\mathcal{B}^{\psi}} G^\text {ideal }_{\mathcal{D}}(\xi, \zeta) d \xi d \zeta =\sum_{i=1}^k\iint_{\mathcal{D}_i} t d \xi d \zeta \nonumber\\
&= \sum_{(p,q) \in \mathcal{A} }{\iint_{\mathcal{B}^\psi_{p,q}} t d \xi d \zeta} = \sum_{(p,q) \in \mathcal{A}}\delta_{p,q} t=(2 \pi)^2 \label{composite}
\end{align}

where $\delta_{p,q}$ denotes the area of the $(p,q)$-th beam in the $(\xi, \zeta)$ domain. Therefore, we can derive $t= \frac{(2\pi)^2}{|\mathcal{A}|\delta_{p,q}}$. It holds that, 
\begin{equation}
    G^{\text {ideal }}_{ \mathcal{D}}\left(\xi, \zeta\right)=\frac{(2\pi)^2}{|\mathcal{A}|\delta_{p,q}} \mathds{1}_{\mathcal{D}}\left(\xi, \zeta\right)\label{ideal_compound}
\end{equation}

Using the beamformer $\c$ we wish to mimic the deal gain in equation \eqref{ideal_compound}. Therefore, we formulate the following optimization problem, 
\begin{align}
& \c^{opt}_{\mathcal{D}} = \underset{\c, \|\c\|=1}{\arg \min } \underset{\mathcal{B}^{\psi}}{\iint}\left|G^{\text {ideal }}_{\mathcal{D}}\left(\xi, \zeta\right)-G\left(\xi, \zeta, \c\right)\right| d \xi d \zeta \label{init_opt}
\end{align}

By partitioning the range of $(\xi, \zeta)$ into the pre-defined intervals, and then uniformly sampling with the rate $(L_v, L_h)$ per interval along both axis,  we can rewrite the optimization problem as follows, 
% \begin{align}
% & \c^{opt}_{\mathcal{D}} = \underset{\c, \|\c\|=1}{\arg \min } \sum_{r=1}^{Q_v}\sum_{s=1}^{Q_h}\iint_{\mathcal{B}^{\psi}_{r,s}}\left|G_{\mathcal{D}}^{\text {ideal }}\left(\xi, \zeta\right)-G\left(\xi, \zeta, \c\right)\right| d \xi d \zeta \nonumber\\&
%  = \underset{L_h, L_v \rightarrow \infty}{\lim}\sum_{r=1}^{Q_v}\sum_{s=1}^{Q_h}\sum_{l_v =1}^{L_v}\sum_{l_h =1}^{L_h}\frac{\delta_v\delta_h}{L_hL_v}\nonumber\\&
% \left|G_{\mathcal{D}}^{\text {ideal }}\left(\psi^{r,s}_{v}[l_v], \psi^{r,s}_{h}[l_h]\right)-G\left(\psi^{r,s}_{v}[l_v], \psi^{r,s}_{h}[l_h], \c\right)\right| \label{alternative_opt}
% \end{align}
% where, 
% \begin{align}
%     & \psi^{r,s}_v = -\psi^{\mathrm{B}}_v + r\delta_v, \quad \psi^{r,s}_h = -\psi^{\mathrm{B}}_h + s\delta_h \\
%     &\psi^{r,s}_v[l_v] = \psi^{r-1, s}_v + l_v\frac{\delta_v}{L_v},  \quad \psi^{r,s}_h[l_h] = \psi^{r, s-1}_h + l_h\frac{\delta_h}{L_h} \label{l_vl_h}
% \end{align}
\begin{align}
 \c^{opt}_{\mathcal{D}} &= \underset{{\c, \|\c\|=1}}{{\arg \min }} \sum_{r=1}^{Q_v}\sum_{s=1}^{Q_h}\iint_{\mathcal{B}^{\psi}_{r,s}}\left|G_{\mathcal{D}}^{\text {ideal }}\left(\xi, \zeta \right)-G\left(\xi, \zeta, \c\right)\right| d \xi d \zeta \nonumber\\&
 = \underset{L_h, L_v \rightarrow \infty}{\lim}\sum_{r=1}^{Q_v}\sum_{s=1}^{Q_h}\sum_{l_v =1}^{L_v}\sum_{l_h =1}^{L_h}\nonumber\\&\frac{\delta_v\delta_h}{L_hL_v}
%\left|G_{\mathcal{D}}^{\text {ideal }}\left(\psi^{r,s}_{v}[l_v], \psi^{r,s}_{h}[l_h]\right)-G\left(\psi^{r,s}_{v}[l_v], \psi^{r,s}_{h}[l_h], \c\right)\right|
\left|G_{\mathcal{D}}^{\text {ideal }}\left(\xi_{r, l_v}, \zeta_{s,l_h}\right)-G\left(\xi_{r, l_v}, \zeta_{s,l_h}, \c\right)\right|\label{alternative_opt}
\end{align}
where, 
\begin{align}
    &\xi_{r,l_v} = \xi^{r-1} + l_v\frac{\delta_v}{L_v}, \quad \zeta_{s, l_h} = \zeta^{s-1} + l_h\frac{\delta_h}{L_h} \label{l_v_l_h}
\end{align}




\noindent with $\delta_a = \frac{2\psi_a^{\mathrm{B}}}{Q_a}$, for $a \in \{v, h\}$. Note that it holds for all $(p,q)$ pairs that, $\delta_{p,q} = \delta_v\delta_h$. 
We can rewrite equation \eqref{alternative_opt} as, 
\begin{align}
    \c_{\mathcal{D}}^{opt}=\arg \min _{\c, \|\c\|=1} \underset{L_h, L_v \rightarrow \infty}{\lim}\frac{1}{L_hL_v}\left|\mathbf{G}^{\text {ideal }}_{\mathcal{D}}-\mathbf{G}(\c)\right|\label{init_normed_opt}
\end{align}
where, 
\begin{align}
    \mathbf{G}(\c) =&  \delta_{p,q}\left[G\left(\xi_{1,1}, \zeta_{1,1}, \c\right) \cdots G\left(\xi_{Q_{v}, L_{v}}, \zeta_{Q_{h}, L_{h}}, \c\right)\right]^{T}  
\end{align}
and,
\begin{align}
    \mathbf{G}^{\text {ideal }}_{ \mathcal{D}} =& \delta_{p,q}\left[G^{\text {ideal }}_{ \mathcal{D}}\left(\xi_{1,1}, \zeta_{1,1} \right) \cdots  G^{\text {ideal }}_{ \mathcal{D}}\left(\xi_{Q_{v}, L_v}, \zeta_{ Q_{h}, L_{h}}\right)\right]^{T} 
\end{align}

% \begin{align}
%     \mathbf{G}(\c) =&  \delta_v\delta_h\left[G\left(\xi^{1}[1], \zeta^{1}[1], \c\right) \cdots \right. \nonumber\\ &\left. G\left(\xi^{Q_{v}}\left[L_{v}\right], \zeta^{Q_{h}}\left[L_{h}\right], \c\right)\right]^{T} 
% \end{align}

% and, 
% \begin{align}
%     \mathbf{G}^{\text {ideal }}_{ \mathcal{D}} =& \delta_v\delta_h\left[G^{\text {ideal }}_{ \mathcal{D}}\left(\xi^{1}[1], \zeta^{1}[1] \right) \cdots \right. \nonumber\\ &\left. G^{\text {ideal }}_{ \mathcal{D}}\left(\xi^{Q_{v}}\left[L_{v}\right], \zeta^{ Q_{h}}\left[L_{h}\right]\right)\right]^{T} 
% \end{align}


Unfortunately, the optimization problem in \eqref{init_normed_opt} does not admit an optimal closed-form solution as is, due to the absolute values of the complex numbers existing in the formulation. However, note that, 
\begin{align}
    \mathbf{G}^{\text {ideal }}_{\mathcal{D}}&=\sum_{(p,q) \in \mathcal{A}}\delta_{p,q}\frac{(2\pi)^2}{|\mathcal{A}|\delta_{p,q}}\left(\mathbf{e}_{p,q} \otimes \mathbf{1}_{L, 1}\right) \nonumber\\&= \frac{(2\pi)^2}{|\mathcal{A}|}\sum_{(p,q) \in \mathcal{A}}{\mathbf{e}_{p,q} \otimes \mathbf{1}_{L, 1}}
    \label{ideal}
\end{align}
with $\mathbf{e}_{p,q} \in \mathbb{Z}^{Q}$ being the standard basis vector for the $(p,q)$-th axis among $(Q_v, Q_h)$ pairs. Now, note that $\mathbf{1}_{L, 1}=\mathbf{g} \odot \mathbf{g}^{*}$ for any equal gain $\mathbf{g} \in \mathbb{C}^L$ where $L = L_hL_v$. An equal-gain  vector $\g \in \mathbb{C}^L$ is a vector where all elements have equal absolute values (in this case, equal to $1$). Therefore, we can write: 
% $$\mathcal{G}_{L}=\left\{\mathbf{g} \in \mathbb{C}^{L}:\left(\mathbf{g} \mathbf{g}^{H}\right)_{\ell, \ell}=1,1 \leq \ell \leq L\right\}$$
\begin{align}
\mathbf{G}^{\text {ideal }}_{\mathcal{D}} &= \sum_{(p,q) \in \mathcal{A}}\frac{(2\pi)^2}{|\mathcal{A}|}\left(\mathbf{e}_{p,q} \otimes\left(\mathbf{g} \odot \mathbf{g}^{*}\right)\right) \nonumber\\
&=\frac{(2\pi)^2}{|\mathcal{A}|}\sum_{(p,q) \in \mathcal{A}}\left(\mathbf{e}_{p,q} \otimes \mathbf{g}\right) \odot\left(\mathbf{e}_{p,q} \otimes \mathbf{g}\right)^{*} \nonumber\\
&=\left(\sum_{(p,q) \in \mathcal{A}}\frac{2\pi}{\sqrt{|\mathcal{A}|}}\left(\mathbf{e}_{p,q} \otimes \mathbf{g}\right)\right)  \nonumber\\&\odot \left(\sum_{(p,q) \in \mathcal{A}}\frac{2\pi}{\sqrt{|\mathcal{A}|}}\left(\mathbf{e}_{p,q} \otimes \mathbf{g}\right)\right)^* \label{final_gik}
\end{align}

Also, it is straightforward to write, 
% \begin{align}
%     \mathbf{G}(\c)=\left(\left(\mathbf{D}_{h}^{H} \otimes \mathbf{D}_{v}^{H}\right) \c\right) \odot\left(\left(\mathbf{D}_{h}^{H} \otimes \mathbf{D}_{v}^{H}\right) \c\right)^{*}
% \end{align}
\begin{align}
    \mathbf{G}(\c)=\left(\D^H \c\right) \odot\left(\D^H \c\right)^{*}\label{dc}
\end{align}

\noindent where, $\D^H = \sqrt{\delta_v\delta_h}(\mathbf{D}_{v}^{H} \otimes \mathbf{D}_{h}^{H})$, and for $a \in \{v,h\}$, and $b= 1\ldots Q_a$ we have, 
\begin{align}
\mathbf{D}_{a} &=\left[\mathbf{D}_{a, 1}, \cdots, \mathbf{D}_{a, Q_{a}}\right] \in \mathbb{C}^{M_{a} \times L_{a} Q_{a}}
% \mathbf{D}_{a, b} &=\left[\mathbf{d}_{M_{a}}\left(\psi_{a}^{b}[1]\right), \cdots, \mathbf{d}_{M_{a}}\left(\psi_{a}^{b}\left[L_{a}\right]\right)\right] \in \mathbb{C}^{M_{a} \times L_{a}}
\end{align}
where, 
\begin{align}
    &\mathbf{D}_{v, b} =\left[\mathbf{d}_{M_{v}}\left(\xi_{b,1}\right), \cdots, \mathbf{d}_{M_{v}}\left(\xi_{b, L_v}\right)\right] \in \mathbb{C}^{M_{v} \times L_{v}} \\
    &\mathbf{D}_{h, b} =\left[\mathbf{d}_{M_{h}}\left(\zeta_{b,1}\right), \cdots, \mathbf{d}_{M_{h}}\left(\zeta_{b, L_h}\right)\right] \in \mathbb{C}^{M_{h} \times L_{h}}
\end{align}

Comparing the expressions \eqref{init_normed_opt}, \eqref{final_gik}, and \eqref{dc}, one can show that the optimal choice of $\c_\mathcal{D}$ in \eqref{init_opt} is the solution to the following optimization problem for proper choices of $\g_{p,q}$. 

\begin{problem}
Given equal-gain vectors $\g_{p,q} \in \mathbb{C}^L$, for $(p,q) \in \mathcal{A}$  find vector $\c_{\mathcal{D}} \in \mathbb{C}^{M}$ such that
\begin{align}
\c_{\mathcal{D}}=&{\arg \min }_{\c, \|\c\|=1}\nonumber\\& \lim_{L\rightarrow \infty} \left\|\sum_{(p,q) \in \mathcal{A}}\frac{2\pi}{\sqrt{|\mathcal{A}|}}\left(\mathbf{e}_{p,q} \otimes \mathbf{g}_{p,q}\right)- \D^H \c\right\|^{2} \label{obj_func}
\end{align}
\label{main_problem_UPA}
\end{problem}

However, we now need to find the optimal choices of $\g_{p,q}$ that minimize the objective in \eqref{init_normed_opt}. Using \eqref{final_gik}, and \eqref{dc}, we have the following optimization problem.

\begin{problem}
Find equal-gain vectors $\g^*_{p,q} \in \mathbb{C}^L$, $(p,q) \in \mathcal{A}$ such that
\begin{align}
 &<\g^*_{p,q}>_{(p,q)\in\mathcal{A}} = \underset{<\g_{p,q}>_{(p,q)\in\mathcal{A}}}{\arg\min }\nonumber\\
 &\left\| abs(\D^H \c_{\mathcal{D}})- \frac{{2\pi}}{\sqrt{|\mathcal{A}|}} abs(\sum_{(p,q)\in \mathcal{A}}\mathbf{e}_{p,q} \otimes \mathbf{g}_{p,q})\right\|^{2} \label{g_final_eq}
\end{align} 
where $abs(.)$ denotes the element-wise absolute value of a vector.
\label{g_problem}
\end{problem}

Next, we continue with the solution of Problems ~\ref{main_problem_UPA}, and~\ref{g_problem}.

\label{sec:propose}

\paragraph{Correlation with listwise ground-truth}
Before describing our new QPP evaluation framework \proposed, we begin by introducing the required notation. Formally, a QPP estimate is a function of the form $\phi(Q, M_k(Q)) \mapsto \mathbb{R}$, where $M_k(Q)$ is the set of top-$k$ ranked documents retrieved by an IR model $M$ for a query $Q \in \mathcal{Q}$, a benchmark set of queries.

For the purpose of listwise evaluation, for each $Q\in \mathcal{Q}$, we first compute the value of a target IR evaluation metric, $\mu(Q)$ that reflects the quality of the retrieved list $M_k(Q)$. The next step uses these $\mu(Q)$ scores to induce a \textit{ground-truth ranking} of the set $\mathcal{Q}$, or in other words, arrange the queries by their decreasing (or increasing) $\mu(Q)$ values, i.e., 
\begin{equation}
\mathcal{Q}_\mu = \{Q_i \in \mathcal{Q}: \mu(Q_i) > \mu(Q_{i+1}),
\, \forall i=1,\ldots,|\mathcal{Q}|-1\}  \}
\end{equation}
Similarly, the evaluation framework also yields a \emph{predicted ranking} of the queries, where this time the queries are sorted by the QPP estimated scores, i.e.,
\begin{equation}
\mathcal{Q}_\phi = \{Q_i \in \mathcal{Q}: \phi(Q_i) > \phi(Q_{i+1}),
\, \forall i=1,\ldots,|\mathcal{Q}|-1 \} 
\label{qpp_listwise_pred}
\end{equation}
A listwise evaluation framework then computes the rank correlation between these two ordered sets
$\gamma(\mathcal{Q}_\mu, \mathcal{Q}_\phi),\,\,\text{where}\,\, \gamma: \mathbb{R}^{|\mathcal{Q}|}\times\mathbb{R}^{|\mathcal{Q}|} \mapsto [0,1]$ is a correlation measure, such as Kendall's $\tau$.

\paragraph{Individual ground-truth}
In contrast to listwise evaluations, where the ground-truth takes the form of an ordered set of queries, pointwise QPP evaluation involves making $|\mathcal{Q}|$ \textit{independent comparisons}. Each comparison is made between a query $Q$'s predicted QPP score $\phi(Q)$ and its retrieval effectiveness measure $\mu(Q)$, i.e.,
\begin{equation}
\eta(\mathcal{Q}, \mu, \phi) \defas \frac{1}{|\mathcal{Q}|}\sum_{Q \in \mathcal{Q}}\eta(\mu(Q), \phi(Q))
\label{eq:pwcorr}  
\end{equation}
Unlike the rank correlation $\gamma$, 
here $\eta$ is a pointwise correlation function of the form $\eta:\mathbb{R}\times \mathbb{R}\mapsto\mathbb{R}$.
It is often convenient to think of $\eta$ as the inverse of a \emph{distance} function that measures the extent to which a predicted value deviates from the corresponding true value.
In contrast to ground-truth evaluation metrics, most QPP estimates (e.g., NQC, WIG etc.) are not bounded within $[0, 1]$. Therefore, to employ a distance measure, each QPP estimate $\phi(Q)$ must be normalized to the unit interval. Subsequently, $\eta$ can be defined as
$\eta(\mu(Q), \phi(Q)) \defas 1-|\mu(Q) - \phi(Q)/\aleph|$,
where $\aleph$ is a normalization constant which is sufficiently large to ensure that the denominator is positive.

\paragraph{Selecting an IR metric for pointwise QPP evaluation}

In general, an unsupervised QPP estimator will be agnostic with respect to the target IR metric $\mu$. For instance, NQC scores can be seen as being approximations of AP@100 values, but can also be interpreted as approximating any other metric, such as nDCG@20 or P@10. Therefore, a question arises around which metric should be used to compute the individual correlations in Equation \ref{eq:pwcorr}. Of course, the results can differ substantially for different choices of $\mu$, e.g., AP or nDCG. This is also the case for listwise QPP evaluation, as reported in \cite{dg22ecir}. To reduce the effect of such variations, we now propose a simple yet effective solution.

\paragraph{Metric-agnostic pointwise QPP evaluation}
For a set of evaluation functions
$\mu \in \mathcal{M}$ (e.g., $\mathcal{M} = \{\text{AP@100}, \text{nDCG@20},\ldots\}$), we employ an aggregation function to compute the overall pointwise correlation (Equation \ref{eq:pwcorr}) of a QPP estimate with respect to each metric.
Formally,
\begin{equation}
\eta(Q,\mathcal{M},\phi) = \Sigma_{\mu \in \mathcal{M}} 
(1-|\mu(Q) - \phi(Q)/\aleph|), \label{eq:avgpwcorr}
\end{equation}
where $\Sigma$ denotes an aggregation function (it does not indicate summation). In particular, we use the most commonly-used such functions as choices for $\Sigma$: `minimum', `maximum', and `average' -- i.e., $\Sigma \in \{\text{avg}, \text{min}, \text{max}\}$.
Next, we find the average over these values computed for a given set of queries $\mathcal{Q}$, i.e., we substitute $\eta(Q,\mathcal{M},\phi)$ from Equation \ref{eq:avgpwcorr} into the summation of Equation \ref{eq:pwcorr}.
% %% This declares a command \Comment
%% The argument will be surrounded by /* ... */
\SetKwComment{Comment}{/* }{ */}

\begin{algorithm}[t]
\caption{Training Scheduler}\label{alg:TS}
% \KwData{$n \geq 0$}
% \KwResult{$y = x^n$}
\LinesNumbered
\KwIn{Training data $\mathcal{D}_{train}=\{(q_i, a_i, p_i^+)\}_{i=1}^m$, \\
\qquad \quad Iteration number $L$.}
\KwOut{A set of optimal model parameters.}

\For{$l=1,\cdots, L$}{
    Sample a batch of questions $Q^{(l)}$\\
    \For{$q_i\in Q^{(l)}$}{
        $\mathcal{P}_{i}^{(l)} \gets \mathrm{arg\,max}_{p_{i,j}}(\mathrm{sim}(q_i^{en},p_{i,j}),K)$\\
        $\mathcal{P}_{Gi}^{(l)} \gets \mathcal{P}_{i}^{(l)}\cup\{p^+_i\}$\\
        Compute $\mathcal{L}^i_{retriever}$, $\mathcal{L}^i_{postranker}$, $\mathcal{L}^i_{reader}$\\ according to Eq.\ref{eq:retriever}, Eq.\ref{eq:rerank}, Eq.\ref{eq:reader}\\
    }
    % $\mathcal{L}^{(l)}_{retriever} \gets \frac{1}{|Q^{(l)}|}\sum_i\mathcal{L}^i_{retriever}$\\
    % $\mathcal{L}^{(l)}_{retriever} \gets \mathrm{Avg}(\mathcal{L}^i_{retriever})$,
    % $\mathcal{L}^{(l)}_{rerank} \gets \mathrm{Avg}(\mathcal{L}^i_{rerank})$,
    % $\mathcal{L}^{(l)}_{reader} \gets \mathrm{Avg}(\mathcal{L}^i_{reader})$\\
    % Compute $\mathcal{L}^{(l)}_{retriever}$, $\mathcal{L}^{(l)}_{rerank}$, and $\mathcal{L}^{(l)}_{reader}$ by averaging over $Q^{(l)}$\\
    $\mathcal{L}^{(l)} \gets \frac{1}{|Q^{(l)}|}\sum_i(\mathcal{L}^{i}_{retriever} + \mathcal{L}^{i}_{postranker}+ \mathcal{L}^{i}_{reader})$\\
    $\mathcal{P}^{(l)}_K\gets\{\mathcal{P}^{(l)}_i|q_i\in Q^{(l)}\}$,\quad $\mathcal{P}^{(l)}_{KG}\gets\{\mathcal{P}^{(l)}_{Gi}|q_i\in Q^{(l)}\}$\\
    Compute the coefficient $v^{(l)}$ according to Eq.~\ref{eq:v}\\
  \eIf{$ v^{(l)}=1$}{
    $\mathcal{L}^{(l)}_{final} \gets \mathcal{L}^{(l)}(\mathcal{P}_{KG}^{(l)})$\\
  }{
      $\mathcal{L}^{(l)}_{final} \gets \mathcal{L}^{(l)}(\mathcal{P}^{(l)}_{K}),$\\
    }
    Optimize $\mathcal{L}^{(l)}_{final}$
}
\end{algorithm}


%  \eIf{$ \mathcal{L}^{(l-1)}_{retriever}<\lambda$}{
%     $\mathcal{L}^{(l)}_{final} \gets \mathcal{L}^{(l)}(\mathcal{P}_K^{(l)})$\\
%   }{
%       $\mathcal{L}^{(l)}_{final} \gets \mathcal{L}^{(l)}(\mathcal{P}^{(l)}_{KG}),$\\
%     }
\section{Evaluation}
\label{sec:evaluation}
\begin{table*}[!t]
\begin{center}
%\small
\caption {Benchmarks and applications for the study of the application-level resilience}
\vspace{-5pt}
\label{tab:benchmark}
\tiny
\begin{tabular}{|p{1.7cm}|p{7.5cm}|p{4cm}|p{2.5cm}|}
\hline
\textbf{Name} 	& \textbf{Benchmark description} 		& \textbf{Execution phase for evaluation}  			& \textbf{Target data objects}             \\ \hline \hline
CG (NPB)             & Conjugate Gradient, irregular memory access (input class S)   & The routine conj\_grad in the main computation loop  & The arrays $r$ and $colidx$     \\\hline
MG (NPB)    	       & Multi-Grid on a sequence of meshes (input class S)             & The routine mg3P in the main computation loop & The arrays $u$ and $r$ 	\\ \hline
FT (NPB)             & Discrete 3D fast Fourier Transform (input class S)            & The routine fftXYZ in the main computation loop  & The arrays $plane$ and $exp1$    \\ \hline
BT (NPB)             & Block Tri-diagonal solver (input class S)         		& The routine x\_solve in the main computation loop & The arrays $grid\_points$ and $u$	\\ \hline
SP (NPB)             & Scalar Penta-diagonal solver (input class S)         		& The routine x\_solve in the main computation loop & The arrays $rhoi$ and $grid\_points$  \\ \hline
LU (NPB)            & Lower-Upper Gauss-Seidel solver (input class S)        	& The routine ssor 	& The arrays $u$ and $rsd$ \\ \hline \hline
LULESH~\cite{IPDPS13:LULESH} & Unstructured Lagrangian explicit shock hydrodynamics (input 5x5x5) & 
The routine CalcMonotonicQRegionForElems 
& The arrays $m\_elemBC$ and $m\_delv\_zeta$ \\ \hline
AMG2013~\cite{anm02:amg} & An algebraic multigrid solver for linear systems arising from problems on unstructured grids (we use  GMRES(10) with AMG preconditioner). We use a compact version from LLNL with input matrix $aniso$. & The routine hypre\_GMRESSolve & The arrays $ipiv$ and $A$   \\ \hline
%$hierarchy.levels[0].R.V$ \\ \hline
\end{tabular}
\end{center}
\vspace{-5pt}
\end{table*}

%We evaluate the effectiveness of ARAT, and 
%We use ARAT to study the application-level resilience.
%The goal is to demonstrate 
%that aDVF can be a very useful metric to quantify the resilience of data objects
%at the application level. 
We study 12 data objects from six benchmarks of the NAS parallel benchmark (NPB) suite (we use SNU\_NPB-1.0.3) and 4 data objects from two scientific applications. 
%which is a c version of NPB 3.3, but ARAT can work for Fortran.
Those data objects are chosen to be representative: they have various data access patterns and participate in various execution phases.  
%For the benchmarks, we use CLASS S as the input problems and use the default compiler options of NPB.
For those benchmarks and applications, we use their default compiler options, and use gcc 4.7.3 and LLVM 3.4.2 for trace generation.
To count the algorithm-level fault masking, we use the default convergence thresholds (or the fault tolerance levels) for those benchmarks.
Table~\ref{tab:benchmark} gives 
%for->on by anzheng
detailed information on the benchmarks and applications.
The maximum fault propagation path for aDVF analysis is set to 10 by default.
%the value shadowing threshold is set as 0.01 (except for BT, we use $1 \times 10^{-6}$).
%These value shadowing thresholds are chosen such that any error corruption
%that results in the operand's value variance less than 1\% (for the threshold 0.01) or 0.0001\% (for the threshold $1 \times 10^{-6}$) during the 
%trace analysis does not impact the outcome correctness of six benchmarks.
%LU: check the newton-iteration residuals against the tolerance levels
%SP: check the newton-iteration residuals against the tolerance levels
%BT: check the newton-iteration residuals against the tolerance levels

\subsection{Resilience Modeling Results}
%We use ARAT to calculate aDVF values of 16 data objects. 
Figure~\ref{fig:aDVF_3tiers_profiling}
shows the aDVF results and breaks them down into the three levels 
(i.e., the operation-level, fault propagation level, and algorithm-level).
Figure~\ref{fig:aDVF_3classes_profiling} shows the 
%for->of by anzheng
results for the analyses at the levels of the operation and fault propagation,
and further breaks down the results into 
the three classes (i.e., the value overwriting, logical and comparison operations,
and value shadowing). %based on the reasons of the fault masking.
We have multiple interesting findings from the results.

\begin{figure*}
	\centering
        \includegraphics[width=0.8\textwidth]{three_tiers_gray.pdf}
% * <azguolu@gmail.com> 2017-03-23T03:20:28.808Z:
%
% ^.
        \vspace{-5pt}
        \caption{The breakdown of aDVF results based on the three level analysis. The $x$ axis is the data object name.}
        \vspace{-8pt}
        \label{fig:aDVF_3tiers_profiling}
\end{figure*}


\begin{figure*}
	\centering
	\includegraphics[width=0.8\textwidth]{three_types_gray.pdf}
	\vspace{-5pt}
	\caption{The breakdown of aDVF results based on the three classes of fault masking. The $x$ axis is the data object name. \textit{zeta} and \textit{elemBC} in LULESH are \textit{m\_delv\_zeta} and \textit{m\_elemBC} respectively.} % Anzheng
	\vspace{-5pt}
	\label{fig:aDVF_3classes_profiling}
    %\vspace{-5pt}
\end{figure*}

(1) Fault masking is common across benchmarks and applications.
Several data objects (e.g., $r$ in CG, and $exp1$ and $plane$ in FT)
have aDVF values close to 1 in Figure~\ref{fig:aDVF_3tiers_profiling}, 
which indicates that most of operations working on these data objects
have fault masking.
However, a couple of data objects have much less intensive fault masking.
For example, the aDVF value of $colidx$ in CG is 0.28 (Figure~\ref{fig:aDVF_3tiers_profiling}). 
Further study reveals that $colidx$ is an array to store column indexes of sparse matrices, and there is few operation-level or fault propagation-level fault masking  (Figure~\ref{fig:aDVF_3classes_profiling}).
The corruption of it can easily cause segmentation fault caught by the
algorithm-level analysis. 
$grid\_points$ in SP and BT also have a relatively small aDVF value (0.14 and 0.38 for SP and BT respectively in Figure~\ref{fig:aDVF_3tiers_profiling}).
Further study reveals that $grid\_points$ defines input problems for SP and BT. 
A small corruption of $grid\_points$ 
%change->changes by anzheng
can easily cause major changes in computation
caught by the fault propagation analysis. 

The data object $u$ in BT also has a relatively small aDVF value (0.82 in Figure~\ref{fig:aDVF_3tiers_profiling}).
Further study reveals that $u$ is read-only in our target code region
for matrix factorization and Jacobian, neither of which is friendly
for fault masking.
Furthermore, the major fault masking for $u$ comes from value shadowing,
and value shadowing only happens in a couple of the least significant bits 
of the operands that reference $u$, which further reduces the value of aDVF.
%also reduces fault masking.

(2) The data type is strongly correlated with fault masking.
Figure~\ref{fig:aDVF_3tiers_profiling} reveals that the integer data objects ($colidx$ in CG, $grid\_points$ in BT and SP, $m\_elemBC$ in LULESH) appear to be 
more sensitive to faults than the floating point data objects 
($u$ and $r$ in MG, $exp1$ and $plane$ in FT, $u$ and $rsd$ in LU, $m\_delv\_zeta$ in LULESH, and $rhoi$ in SP).
In HPC applications, the integer data objects are commonly employed to
define input problems and bound computation boundaries (e.g., $colidx$ in CG and $grid\_points$ in BT), 
or track computation status (e.g., $m\_elemBC$ in LULESH). Their corruption 
%these integer data objects
is very detrimental to the application correctness. 

(3) Operation-level fault masking is very common.
For many data objects, the operation-level fault masking contributes 
more than 70\% of the aDVF values. For $r$ in CG, $exp1$ in FT, and $rhoi$ in SP,
the contribution of the operation-level fault masking is close to 99\% (Figure~\ref{fig:aDVF_3tiers_profiling}).

Furthermore, the value shadowing is a very common operation level fault masking,
especially for floating point data objects (e.g., $u$ and $r$ in BT, $m\_delv\_zeta$ in LULESH, and $rhoi$ in SP in Figure~\ref{fig:aDVF_3classes_profiling}).
This finding has a very important indication for studying the application resilience.
In particular, the values of a data object can be different across different input problems. If the values of the data object are different, 
then the number of fault masking events due to the value shadowing will be different. 
Hence, we deduce that the application resilience
can be correlated with the input problems,
because of the correlation between the value shadowing and input problems. 
We must consider the input problems when studying the application resilience.
This conclusion is consistent with a very recent work~\cite{sc16:guo}.

(4) The contribution of the algorithm-level fault masking to the application resilience can be nontrivial.
For example, the algorithm-level fault masking contributes 19\% of the aDVF value for $u$ in MG and 27\% for $plane$ in FT (Figure~\ref{fig:aDVF_3tiers_profiling}).
The large contribution of algorithm-level fault masking in MG is consistent with
the results of existing work~\cite{mg_ics12}. 
For FT (particularly 3D FFT), the large contribution of algorithm-level fault masking in $plane$ (Figure~\ref{fig:aDVF_3tiers_profiling})
comes from frequent transpose and 1D FFT computations that average out 
or overwrite the data corruption.
CG, as an iterative solver, is known to have the algorithm-level fault masking
because of the iterative nature~\cite{2-shantharam2011characterizing}.
Interestingly, the algorithm-level fault masking in CG contributes most to the resilience of $colidx$ which is a vulnerable integer data object (Figure~\ref{fig:aDVF_3tiers_profiling}).

%Our study reveals the algorithm-level fault masking of CG from
%two perspectives. First, $a$ in CG, which is an array for intermediate results,
%has few algorithm-level fault masking (0.008\%);
%Second, $x$ in CG, which is a result vector, has 5.4\% of the aDVF value coming from the algorithm-level fault masking.
%This result indicates that the effects of the algorithm-level fault masking
%are not uniform across data objects. 

(5) Fault masking at the fault propagation level is small.
For all data objects, the contribution of the fault masking at the level of fault propagation is less than 5\% (Figure~\ref{fig:aDVF_3tiers_profiling}).
For 6 data objects ($r$ and $colidx$ in CG, $grid\_points$ and $u$ in BT, and 
$grid\_points$ and $rhoi$ in SP),  there is no fault masking at the level of fault propagation.
In combination with the finding 4, we conclude that once the fault
is propagated, it is difficult to mask it because of the contamination of
more data objects after fault propagation, and only the algorithm semantics can tolerate  propagated faults well. 
%This finding is consistent with our sensitivity analysis. 

(6) Fault masking by logical and comparison operations is small,
%For all data objects, the fault masking contributions due to logical and comparison operations are very small, 
comparing with the contributions of value shadowing and overwriting (Figure~\ref{fig:aDVF_3classes_profiling}). 
Among all data objects, 
the logical and comparison operations in $grid\_points$ in BT contribute the most (25\% contribution in Figure~\ref{fig:aDVF_fine_profiling}), 
because of intensive ICmp operations (integer comparison). %logical OR and SHL (left shifting).


(7) The resilience varies across data objects. %within the same application.
This fact is especially pronounced in two data objects $colidx$ and $r$ in CG (Figure~\ref{fig:aDVF_3tiers_profiling}).
 $colidx$ has aDVF much smaller than $r$, which means $colidx$ is much less resilient than $r$ (see finding 1 for a detailed analysis on $colidx$). 
Furthermore, $colidx$ and $r$ have different algorithm-level
fault masking (see finding 4 for a detailed analysis).

\begin{comment}
\textbf{Finding 7: The resilience of the same data objects varies across different applications.}
This fact is especially pronounced in BT and SP.
BT and SP address the same numerical problem but with different algorithms.
BT and SP have the same data objects, $qs$ and $rhoi$, but
$qs$ manifests different resilience in BT and SP.
This result is interesting, because it indicates that by using
different algorithms, we have opportunities to
improve the resilience of data objects.
\end{comment}

To further investigate the reasons for fault masking, 
we break down the aDVF results at the granularity of LLVM instructions,
based on the analyses at the levels of operation and fault propagation.
The results are shown in Figure~\ref{fig:aDVF_fine_profiling}.
%Because of the space limitation, 
%we only show one data object per benchmark, but each selected data object has the most diverse fault masking events within the corresponding benchmark.
%Based on Figure~\ref{fig:aDVF_fine_profiling}, we have another interesting finding.

(8) Arithmetic operations make a lot of contributions to fault masking.
%For $r$ in CG, $r$ in MG, $exp1$ in FT, $u$ in BT, $qs$ in SP, and $u$ in LU,
%the arithmetic operations, FMul (100\%), Add (16\%), FMul (85\%), 
%FMul (94\%), FMul (28\%), and FAdd (50\%)
For $r$ in CG, $u$ in BT, $plane$ and $exp1$ in FT, $m\_elemBC$ in LULESH, 
arithmetic operations (addition, multiplication, and division) contribute to almost 100\% of the fault masking (Figure~\ref{fig:aDVF_fine_profiling}).  
%(at the operation level and the fault propagation level).
%For $qs$ in SP and $u$ in LU, the store operation also makes
%important contributions as the arithmetic operations because of value overwriting.

\begin{figure*}
	\centering
	\includegraphics[width=0.77\textheight, height=0.23\textheight]{pie_chart.pdf}
	\vspace{-10pt}
	\caption{Breakdown of the aDVF results based on the analyses at the levels of operation and fault propagation}
    \vspace{-10pt}
	\label{fig:aDVF_fine_profiling}
\end{figure*}


\subsection{Sensitivity Study}
\label{sec:eval_sen}
%\textbf{change the fault propagation threshold and study the sensitivity of analysis to the threshold}
ARAT uses 10 as the default fault propagation analysis threshold. 
The fault propagation analysis will not go beyond 10 operations. Instead,
we will use deterministic fault injection after 10 operations. 
In this section, we study the impact of this threshold on the modeling accuracy. We use a range of threshold values and examine how the aDVF value varies and whether
the identification of fault masking varies. 
Figure~\ref{fig:sensitivity_error_propagation} shows the results for 
%add , after BT by anzheng
multiple data objects in CG, BT, and SP.
We perform the sensitivity study for all 16 data objects.
%in six benchmarks and two applications.
Due to the page space limitation, we only show the results for three data objects,
but we summarize the sensitivity study results for all data objects in this section.
%but other data objects in all benchmarks have the same trend.

Our results reveal that the identification of fault masking by tracking fault propagation is not significantly 
affected by the fault propagation analysis threshold. Even if we use a rather large threshold (50), 
the variation of aDVF values is 4.48\% on average among all data objects,
and the variation at each of the three levels of analysis (the operation level, fault propagation level,  and algorithm level) is less than 5.2\% on average. 
In fact, using a threshold value of 5 is sufficiently accurate in most of the cases (14 out of 16 data objects).
This result is consistent with our finding 5 (i.e., fault masking at the fault propagation level is small). %in most benchmarks).
However, we do find a data object ($m\_elementBC$ in LULESH) %and $exp1$ in FT) 
showing relatively high-sensitive (up to 15\% variation) to the threshold. For this uncommon data object, using 50 as the fault propagation path is sufficient. 

%In other words, even though using a larger threshold value can identify more error masking by tracking error 
%propagation, the implicit error masking induced by the error propagation is very limited.

\begin{figure}
		\begin{center}
		\includegraphics[width=0.48\textwidth,height=0.11\textheight]{sensi_study_gray.pdf}
		\vspace{-15pt}
		\caption{Sensitivity study for fault propagation threshold}
		\label{fig:sensitivity_error_propagation}
		\end{center}
\vspace{-15pt}
\end{figure}


\begin{comment}
\subsection{Comparison with the Traditional Random Fault Injection}
%\textbf{compare with the traditional fault injection to verify accuracy}
To show the effectiveness of our resilience modeling, we compare traditional random fault injection
and our analytical modeling. Figure~\ref{fig:comparison_fi} and Table~\ref{tab:comparison} show the results.
The figure shows the success rate of all random fault injection. The ``success'' means the application
outcome is verified successfully by the benchmarks and the execution does not have any segfault. The success rate is used as a metric
to evaluate the application resilience.

We use a data-oriented approach to perform random fault injection.
In particular, given a data object, for each fault injection test we trigger a bit flip
in an operand of a random instruction, and this operand must be a reference to the
target data object. We develop a tool based on PIN~\cite{pintool} to implement the above fault injection functionality.
For each data object, we conduct five sets of random fault injection tests, 
and each set has 200 tests (in total 1000 tests per data object). 
We show the results for CG and FT in this section, but we find that
the conclusions we draw from CG and FT are also valid for the other four benchmarks.


%\begin{table*}
%\label{tab:success_rate}
%\begin{centering}
%\renewcommand\arraystretch{1.1}
%\begin{tabular}{|c|c|c|c|c|c|c|}
%\hline 
%Success Rate (Difference) & Test set 1 & Test set 2 & Test set 3 & Test set 4 & Test set 5 & Average\tabularnewline
%\hline 
%\hline 
%CG-a & 66.1\% (11.7\%) & 68.5\% (15.7\%) & 56.7\% (4.21\%) & 61.3\% (3.57\%) & 43.3\% (26.8\%) & 59.2\%\tabularnewline
%\hline 
%CG-x & 99.2\% (2.2\%) & 98.6\% (1.5\%) & 96.5\% (0.63\%) & 97.8\% (0.64\%) & 93.6\% (3.7\%) & 97.1\%\tabularnewline
%\hline 
%CG-colidx & 36.8\% (12.7\%) & 49.6\% (17.8\%) & 40.2\% (4.6\%) & 52.6\% (24.9\%) & 31.4\% (25.4\%) & 42.1\%\tabularnewline
%\hline 
%FT-exp1 & 52.7\% (1.4\%) & 22.6\% (56.5\%) & 78.5\% (51.0\%) & 60.7\% (16.7\%) & 45.4\% (12.7\%) & 51.9\%\tabularnewline
%\hline 
%FT-plane & 82.1\% (2.5\%) & 79.3\% (5.6\%) & 99.5\% (18.2\%) & 93.2\% (10.7\%) & 66.8\% (20.6\%) & 84.2\%\tabularnewline
%\hline 
%\end{tabular}
%\par\end{centering}
%\caption{XXXXX}
%\end{table*}


\begin{table*}
\begin{centering}
\caption{\small The results for random fault injection. The numbers in parentheses for each set of tests (200 tests per set) are the success rate difference from the average success rate of 1000 fault injection tests.}
\label{tab:comparison}
\renewcommand\arraystretch{1.1}
\begin{tabular}{|c|p{2.2cm}|p{2.2cm}|p{2.2cm}|p{2.2cm}|p{2.2cm}|p{1.8cm}|}
\hline 
       %& Test set 1 & Test set 2 & Test set 3 & Test set 4 & Test set 5 & Average\tabularnewline
       & \hspace{13pt} Test set 1 \hspace{1pt}/  & \hspace{13pt} Test set 2 \hspace{1pt}/ & \hspace{13pt} Test set 3 \hspace{1pt}/ & \hspace{13pt} Test set 4 \hspace{1pt}/ & \hspace{13pt} Test set 5 \hspace{1pt}/ & Ave. of all test / \\
       & success rate (diff.) & success rate (diff.) & success rate (diff.) & success rate (diff.) & success rate (diff.) & \hspace{5pt} success rate \\
\hline 
\hline 
CG-a & 66.1\% (6.9\%) & 68.5\% (9.3\%) & 56.7\% (-2.5\%) & 61.3\% (2.1\%) & 43.3\% (-15.9\%) & 59.2\%\tabularnewline
\hline 
CG-x & 99.2\% (2.1\%) & 98.6\% (1.5\%) & 96.5\% (-0.6\%) & 97.8\% (0.7\%) & 93.6\% (-3.5\%) & 97.1\%\tabularnewline
\hline 
CG-colidx & 36.8\% (-5.3\%) & 49.6\% (7.5\%) & 40.2\% (-2.0\%) & 52.6\% (10.5\%) & 31.4\% (-10.7\%) & 42.1\%\tabularnewline
\hline 
FT-exp1 & 52.7\% (0.8\%) & 22.6\% (-29.3\%) & 78.5\% (26.6\%) & 60.7\% (8.8\%) & 45.4\% (-6.5\%) & 51.9\%\tabularnewline
\hline 
FT-plane & 82.1\% (-2.1\%) & 79.3\% (-4.9\%) & 99.5\% (15.3\%) & 93.2\% (9.0\%) & 66.8\% (-17.4\%) & 84.2\%\tabularnewline
\hline 
\end{tabular}
\par\end{centering}
\vspace{-0.4cm}
\end{table*}

\begin{figure}
	\begin{center}
		\includegraphics[width=0.48\textwidth,keepaspectratio]{verifi-study.png}
		\caption{The traditional random fault injection vs. ARAT}
		\label{fig:comparison_fi}
	\end{center}
\vspace{-0.7cm}
\end{figure}


We first notice from Table~\ref{tab:comparison} that 
%across 5 sets of random fault injection tests, there are big variances (up to 55.9\% in $exp1$ of FT) in terms of the success rate. 
the results of 5 test sets can be quite different from each other and from 1000 random fault inject tests (up to 29.3\%).
1000 fault injection tests provide better statistical significance than 200 fault injection tests.
We expect 1000 fault injection tests potentially provide higher accuracy to quantify the application resilience.
The above result difference is clearly an indication to the randomness of fault injection, and there
is no guarantee on the random fault injection accuracy.

%In Figure~\ref{fig:comparison_fi}, 
We compare the success rate of 1000 fault inject tests with the aDVF value (Fig.~\ref{fig:comparison_fi}). 
We find that the order of the success rate of the three data objects in CG (i.e., $colidx < a < x$) and the two data objects in FT 
(i.e., $exp1 < plane$) is the same as the order of the aDVF values of these data objects. 
%In fact, 1000 fault injection tests
%account for \textcolor{blue}{\textbf{xxx\%}} of total memory references to the data object,
%and provide better resilience quantification than 200 fault injection tests.
The same order (or the same resilience trend)
%between our approach and the random fault injection based on a large number of tests 
is a demonstration of the effectiveness of our approach.
Note that the values of the aDVF and success rate %for a data object
cannot be exactly the same (even if we have sufficiently large numbers of random fault injection), 
because aDVF and random fault injection quantify
the resilience based on different metrics.
Also, the random fault injection can miss some fault masking events that can be captured by our approach.

\end{comment}
% In this section, we show that it is possible to formulate an alternative \textit{rate-constrained Maximum Likelihood (ML)} problem or \textit{unconstrained Maximum a Posterior (MAP)} problem by combining the prior term and the rate term into a \textit{super rate} term denoted by $R_s(\hat{\vec{x}})$ or a \textit{super prior} denoted by $P_s(\hat{\vec{x}})$ respectively.
\red{I don' get this super rate and super prior. why do we need this definitions? this needs more explanation why these combos are reasonable.}
The corresponding formulations are as follows:
\begin{equation}
\label{eq:alternative_lagrangian_objective}
\underset{\hat{\vec{x}}\in  \mathcal{S}}{\min}\ -\log P(\vec{y}|\hat{\vec{x}})+\lambda_s R_s(\hat{\vec{x}})
\end{equation}
\begin{equation}
\label{eq:alternative_MAP}
\underset{\hat{\vec{x}}\in  \mathcal{S}}{\max}\ P(\vec{y}|\hat{\vec{x}})\cdot P_s(\hat{\vec{x}})
\end{equation}
Different from (\ref{eq:lagrangian_objective}), (\ref{eq:alternative_lagrangian_objective}) performs the rate-constrained ML estimation, \textit{i.e.}, find $\hat{\vec{x}} \in  \mathcal{S}$ that maximize likelihood $P(\vec{y}|\hat{\vec{x}})$ subject to $R_s(\hat{\vec{x}})\leq R_{\max}$.
In contrast, (\ref{eq:alternative_MAP}) performs the unconstrained MAP estimation, \textit{i.e.}, find $\hat{\vec{x}} \in  \mathcal{S}$ that maximizes the product of the likelihood $P(\vec{y}|\hat{\vec{x}})$ and the super prior $P_s(\hat{\vec{x}})$.

In (\ref{eq:objective}), both the prior term and the rate term can be regarded as a modelling of the contour signal.
Specifically, the prior models the underlying statistics of the contour by assuming that contours are more likely to be straight than curvy, which is independent of the signal observations. 
In contrast, the rate term models the symbol probabilities from the training data, which is a data-driven statistical model.
It is possible to combine the two terms into one term as in (\ref{eq:alternative_lagrangian_objective}) and (\ref{eq:alternative_MAP}), but in general they are capturing distinct distributions of the contour.
For example, if the rate constraint is loose and the signal is corrupted by heavy noise, then the reconstructed contours obtained by solving (\ref{eq:objective}) without the prior term are not likely straight.
Similarly, after removing the rate term in (\ref{eq:objective}), the reconstructed contours may not have a small entropy (rate).
\red{point being made here is not crystal clear.}

To find out the relationship between the rate-constrained MAP, rate-constrained ML and unconstrained MAP problems, we rewrite (\ref{eq:alternative_MAP}) as follows:
\begin{equation}
\underset{\hat{\vec{x}}\in  \mathcal{S}}{\min}\ -\log P(\vec{y}|\hat{\vec{x}}) - \beta_s \log P_s(\hat{\vec{x}})
\end{equation}
When the combined super rate term and super prior term are both equal to the sum of the prior term and rate term in the original formulation (\ref{eq:objective}), \textit{i.e.}, 
\begin{equation}
\lambda_s R_s(\hat{\vec{x}}) = -\beta_s \log P_s(\hat{\vec{x}}) = \lambda R(\hat{\vec{x}}) - \beta \log P(\hat{\vec{x}}),
\end{equation}
all the three formulations lead to the same solution.
This provides additional insight into the relationship between the rate-constrained MAP, rate-constrained ML and unconstrained MAP problems.
\red{Idon't get this. why do we have this discussion?}
% \section{Related Work}
%\mz{We lack a comparison to this paper: https://arxiv.org/abs/2305.14877}
%\anirudh{refine to be more on-topic?}
\iffalse
\paragraph{In-Context Learning} As language models have scaled, the ability to learn in-context, without any weight updates, has emerged. \cite{brown2020language}. While other families of large language models have emerged, in-context learning remains ubiquitous \cite{llama, bloom, gptneo, opt}. Although such as HELM \cite{helm} have arisen for systematic evaluation of \emph{models}, there is no systematic framework to our knowledge for evaluating \emph{prompting methods}, and validating prompt engineering heuristics. The test-suite we propose will ensure that progress in the field of prompt-engineering is structured and objectively evaluated. 

\paragraph{Prompt Engineering Methods} Researchers are interested in the automatic design of high performing instructions for downstream tasks. Some focus on simple heuristics, such as selecting instructions that have the lowest perplexity \cite{lowperplexityprompts}. Other methods try to use large language models to induce an instruction when provided with a few input-output pairs \cite{ape}. Researchers have also used RL objectives to create discrete token sequences that can serve as instructions \cite{rlprompt}. Since the datasets and models used in these works have very little intersection, it is impossible to compare these methods objectively and glean insights. In our work, we evaluate these three methods on a diverse set of tasks and models, and analyze their relative performance. Additionally, we recognize that there are many other interesting angles of prompting that are not covered by instruction engineering \cite{weichain, react, selfconsistency}, but we leave these to future work.

\paragraph{Analysis of Prompting Methods} While most prompt engineering methods focus on accuracy, there are many other interesting dimensions of performance as well. For instance, researchers have found that for most tasks, the selection of demonstrations plays a large role in few-shot accuracy \cite{whatmakesgoodicexamples, selectionmachinetranslation, knnprompting}. Additionally, many researchers have found that even permuting the ordering of a fixed set of demonstrations has a significant effect on downstream accuracy \cite{fantasticallyorderedprompts}. Prompts that are sensitive to the permutation of demonstrations have been shown to also have lower accuracies \cite{relationsensitivityaccuracy}. Especially in low-resource domains, which includes the large public usage of in-context learning, these large swings in accuracy make prompting less dependable. In our test-suite we include sensitivity metrics that go beyond accuracy and allow us to find methods that are not only performant but reliable.

\paragraph{Existing Benchmarks} We recognize that other holistic in-context learning benchmarks exist. BigBench is a large benchmark of 204 tasks that are beyond the capabilities of current LLMs. BigBench seeks to evaluate the few-shot abilities of state of the art large language models, focusing on performance metrics such as accuracy \cite{bigbench}. Similarly, HELM is another benchmark for language model in-context learning ability. Rather than only focusing on performance, HELM branches out and considers many other metrics such as robustness and bias \cite{helm}. Both BigBench and HELM focus on ranking different language model, while fix a generic instruction and prompt format. We instead choose to evaluate instruction induction / selection methods over a fixed set of models. We are the first ever evaluation script that compares different prompt-engineering methods head to head. 
\fi

\paragraph{In-Context Learning and Existing Benchmarks} As language models have scaled, in-context learning has emerged as a popular paradigm and remains ubiquitous among several autoregressive LLM families \cite{brown2020language, llama, bloom, gptneo, opt}. Benchmarks like BigBench \cite{bigbench} and HELM \cite{helm} have been created for the holistic evaluation of these models. BigBench focuses on few-shot abilities of state-of-the-art large language models, while HELM extends to consider metrics like robustness and bias. However, these benchmarks focus on evaluating and ranking \emph{language models}, and do not address the systematic evaluation of \emph{prompting methods}. Although contemporary work by \citet{yang2023improving} also aims to perform a similar systematic analysis of prompting methods, they focus on simple probability-based prompt selection while we evaluate a broader range of methods including trivial instruction baselines, curated manually selected instructions, and sophisticated automated instruction selection.

\paragraph{Automated Prompt Engineering Methods} There has been interest in performing automated prompt-engineering for target downstream tasks within ICL. This has led to the exploration of various prompting methods, ranging from simple heuristics such as selecting instructions with the lowest perplexity \cite{lowperplexityprompts}, inducing instructions from large language models using a few annotated input-output pairs \cite{ape}, to utilizing RL objectives to create discrete token sequences as prompts \cite{rlprompt}. However, these works restrict their evaluation to small sets of models and tasks with little intersection, hindering their objective comparison. %\mz{For paragraphs that only have one work in the last line, try to shorten the paragraph to squeeze in context.}

\paragraph{Understanding in-context learning} There has been much recent work attempting to understand the mechanisms that drive in-context learning. Studies have found that the selection of demonstrations included in prompts significantly impacts few-shot accuracy across most tasks \cite{whatmakesgoodicexamples, selectionmachinetranslation, knnprompting}. Works like \cite{fantasticallyorderedprompts} also show that altering the ordering of a fixed set of demonstrations can affect downstream accuracy. Prompts sensitive to demonstration permutation often exhibit lower accuracies \cite{relationsensitivityaccuracy}, making them less reliable, particularly in low-resource domains.

Our work aims to bridge these gaps by systematically evaluating the efficacy of popular instruction selection approaches over a diverse set of tasks and models, facilitating objective comparison. We evaluate these methods not only on accuracy metrics, but also on sensitivity metrics to glean additional insights. We recognize that other facets of prompting not covered by instruction engineering exist \cite{weichain, react, selfconsistency}, and defer these explorations to future work. 
% \vspace{-0.5em}
\section{Conclusion}
% \vspace{-0.5em}
Recent advances in multimodal single-cell technology have enabled the simultaneous profiling of the transcriptome alongside other cellular modalities, leading to an increase in the availability of multimodal single-cell data. In this paper, we present \method{}, a multimodal transformer model for single-cell surface protein abundance from gene expression measurements. We combined the data with prior biological interaction knowledge from the STRING database into a richly connected heterogeneous graph and leveraged the transformer architectures to learn an accurate mapping between gene expression and surface protein abundance. Remarkably, \method{} achieves superior and more stable performance than other baselines on both 2021 and 2022 NeurIPS single-cell datasets.

\noindent\textbf{Future Work.}
% Our work is an extension of the model we implemented in the NeurIPS 2022 competition. 
Our framework of multimodal transformers with the cross-modality heterogeneous graph goes far beyond the specific downstream task of modality prediction, and there are lots of potentials to be further explored. Our graph contains three types of nodes. While the cell embeddings are used for predictions, the remaining protein embeddings and gene embeddings may be further interpreted for other tasks. The similarities between proteins may show data-specific protein-protein relationships, while the attention matrix of the gene transformer may help to identify marker genes of each cell type. Additionally, we may achieve gene interaction prediction using the attention mechanism.
% under adequate regulations. 
% We expect \method{} to be capable of much more than just modality prediction. Note that currently, we fuse information from different transformers with message-passing GNNs. 
To extend more on transformers, a potential next step is implementing cross-attention cross-modalities. Ideally, all three types of nodes, namely genes, proteins, and cells, would be jointly modeled using a large transformer that includes specific regulations for each modality. 

% insight of protein and gene embedding (diff task)

% all in one transformer

% \noindent\textbf{Limitations and future work}
% Despite the noticeable performance improvement by utilizing transformers with the cross-modality heterogeneous graph, there are still bottlenecks in the current settings. To begin with, we noticed that the performance variations of all methods are consistently higher in the ``CITE'' dataset compared to the ``GEX2ADT'' dataset. We hypothesized that the increased variability in ``CITE'' was due to both less number of training samples (43k vs. 66k cells) and a significantly more number of testing samples used (28k vs. 1k cells). One straightforward solution to alleviate the high variation issue is to include more training samples, which is not always possible given the training data availability. Nevertheless, publicly available single-cell datasets have been accumulated over the past decades and are still being collected on an ever-increasing scale. Taking advantage of these large-scale atlases is the key to a more stable and well-performing model, as some of the intra-cell variations could be common across different datasets. For example, reference-based methods are commonly used to identify the cell identity of a single cell, or cell-type compositions of a mixture of cells. (other examples for pretrained, e.g., scbert)


%\noindent\textbf{Future work.}
% Our work is an extension of the model we implemented in the NeurIPS 2022 competition. Now our framework of multimodal transformers with the cross-modality heterogeneous graph goes far beyond the specific downstream task of modality prediction, and there are lots of potentials to be further explored. Our graph contains three types of nodes. while the cell embeddings are used for predictions, the remaining protein embeddings and gene embeddings may be further interpreted for other tasks. The similarities between proteins may show data-specific protein-protein relationships, while the attention matrix of the gene transformer may help to identify marker genes of each cell type. Additionally, we may achieve gene interaction prediction using the attention mechanism under adequate regulations. We expect \method{} to be capable of much more than just modality prediction. Note that currently, we fuse information from different transformers with message-passing GNNs. To extend more on transformers, a potential next step is implementing cross-attention cross-modalities. Ideally, all three types of nodes, namely genes, proteins, and cells, would be jointly modeled using a large transformer that includes specific regulations for each modality. The self-attention within each modality would reconstruct the prior interaction network, while the cross-attention between modalities would be supervised by the data observations. Then, The attention matrix will provide insights into all the internal interactions and cross-relationships. With the linearized transformer, this idea would be both practical and versatile.

% \begin{acks}
% This research is supported by the National Science Foundation (NSF) and Johnson \& Johnson.
% \end{acks}

% \input{L_free_formulation.tex}

% The authors would like to thank...

\renewcommand{\nariman}[1]{\textcolor{red}{#1}}
\renewcommand{\amir}[1]{\textcolor{blue}{#1}}




% trigger a \newpage just before the given reference
% number - used to balance the columns on the last page
% adjust value as needed - may need to be readjusted if
% the document is modified later
%\IEEEtriggeratref{8}
% The "triggered" command can be changed if desired:
%\IEEEtriggercmd{\enlargethispage{-5in}}

% references section

% can use a bibliography generated by BibTeX as a .bbl file
% BibTeX documentation can be easily obtained at:
% http://www.ctan.org/tex-archive/biblio/bibtex/contrib/doc/
% The IEEEtran BibTeX style support page is at:
% http://www.michaelshell.org/tex/ieeetran/bibtex/
%\bibliographystyle{IEEEtran}
% argument is your BibTeX string definitions and bibliography database(s)
%\bibliography{IEEEabrv,../bib/paper}
%
% <OR> manually copy in the resultant .bbl file
% set second argument of \begin to the number of references
% (used to reserve space for the reference number labels box)

\bibliographystyle{IEEEtran}
\bibliography{bibliography}

% \begin{thebibliography}{1}
% \bibitem{love15}
% J. Song, J. Choi and D. J. Love, "Codebook design for hybrid beamforming in millimeter wave systems," 2015 IEEE International Conference on Communications (ICC), 2015, pp. 1298-1303, doi: 10.1109/ICC.2015.7248502.

% \bibitem{ayach14}
% O. E. Ayach, S. Rajagopal, S. Abu-Surra, Z. Pi and R. W. Heath, "Spatially Sparse Precoding in Millimeter Wave MIMO Systems," in IEEE Transactions on Wireless Communications, vol. 13, no. 3, pp. 1499-1513, March 2014, doi: 10.1109/TWC.2014.011714.130846.

% \bibitem{song15}
% J. Song, J. Choi, S. G. Larew, D. J. Love, T. A. Thomas and A. A. Ghosh, "Adaptive Millimeter Wave Beam Alignment for Dual-Polarized MIMO Systems," in IEEE Transactions on Wireless Communications, vol. 14, no. 11, pp. 6283-6296, Nov. 2015, doi: 10.1109/TWC.2015.2452263.

% \bibitem{noh17}
% S. Noh, M. D. Zoltowski and D. J. Love, "Multi-Resolution Codebook and Adaptive Beamforming Sequence Design for Millimeter Wave Beam Alignment," in IEEE Transactions on Wireless Communications, vol. 16, no. 9, pp. 5689-5701, Sept. 2017, doi: 10.1109/TWC.2017.2713357.

% \bibitem{Hussain17}
%  M. Hussain, D. J. Love, and N. Michelusi, “Neyman-pearson codebook
% design for beam alignment in millimeter-wave networks,” in Proc. 1st
% ACM Workshop Millim.-Wave Netw. Sens. Syst., Oct. 2017, pp. 17–22.

% \bibitem{parseval}
% S. Hur, T. Kim, D. J. Love, J. V. Krogmeier, T. A. Thomas and A. Ghosh, "Millimeter Wave Beamforming for Wireless Backhaul and Access in Small Cell Networks," in IEEE Transactions on Communications, vol. 61, no. 10, pp. 4391-4403, October 2013, doi: 10.1109/TCOMM.2013.090513.120848.

% \bibitem{khateeb14}
% A. Alkhateeb, O. El Ayach, G. Leus and R. W. Heath, "Channel Estimation and Hybrid Precoding for Millimeter Wave Cellular Systems," in IEEE Journal of Selected Topics in Signal Processing, vol. 8, no. 5, pp. 831-846, Oct. 2014, doi: 10.1109/JSTSP.2014.2334278.

% \bibitem{vlachos19}
% E. Vlachos, G. C. Alexandropoulos and J. Thompson, "Wideband MIMO Channel Estimation for Hybrid Beamforming Millimeter Wave Systems via Random Spatial Sampling," in IEEE Journal of Selected Topics in Signal Processing, vol. 13, no. 5, pp. 1136-1150, Sept. 2019, doi: 10.1109/JSTSP.2019.2937633.

% \bibitem{noh16}
% S. Noh, M. D. Zoltowski and D. J. Love, "Training Sequence Design for Feedback Assisted Hybrid Beamforming in Massive MIMO Systems," in IEEE Transactions on Communications, vol. 64, no. 1, pp. 187-200, Jan. 2016, doi: 10.1109/TCOMM.2015.2498184.

% \bibitem{vlachos18}
% E. Vlachos, G. C. Alexandropoulos and J. Thompson, "Massive MIMO Channel Estimation for Millimeter Wave Systems via Matrix Completion," in IEEE Signal Processing Letters, vol. 25, no. 11, pp. 1675-1679, Nov. 2018, doi: 10.1109/LSP.2018.2870533.

% \bibitem{qiao17}
% D. Qiao, H. Qian and G. Y. Li, "Multi-resolution codebook design for two-stage precoding in FDD massive MIMO networks," 2017 IEEE 18th International Workshop on Signal Processing Advances in Wireless Communications (SPAWC), 2017, pp. 1-5, doi: 10.1109/SPAWC.2017.8227756.

% \end{thebibliography}

% that's all folks
\end{document}