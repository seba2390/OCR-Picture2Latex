 \documentclass[journal]{IEEEtran}




% *** CITATION PACKAGES ***
%

% *** GRAPHICS and FIGURES RELATED PACKAGES ***
%
\usepackage{graphicx}
\usepackage{wrapfig}
\usepackage{graphicx}
\usepackage{subfigure}
\usepackage{float}
\usepackage{adjustbox}

\usepackage[justification=centering]{caption}
%TABLE
\usepackage{tabularx}
\usepackage{multirow}
\usepackage[belowskip=-5pt,aboveskip=0pt]{caption}
\usepackage{array}
\newcolumntype{P}[1]{>{\centering\arraybackslash}p{#1}}
\newcolumntype{M}[1]{>{\centering\arraybackslash}m{#1}}
\newcommand{\PreserveBackslash}[1]{\let\temp=\\#1\let\\=\temp}
\newcolumntype{R}[1]{>{\PreserveBackslash\raggedleft}p{#1}}
\usepackage{tabulary,booktabs}
\usepackage{lipsum}
\usepackage{blindtext}
% *** MATH PACKAGES ***
%
\usepackage{amsmath,amssymb}

% Links
\usepackage{hyperref}

% *** SPECIALIZED LIST PACKAGES ***
%
\usepackage{enumitem}


% *** ALIGNMENT PACKAGES ***
%
%\addtolength{\parskip}{-0.25mm}
% Example definitions.
% --------------------
\def\x{{\mathbf x}}
\def\L{{\cal L}}

% *** ALGORITHMS PACKAGES ***
%
\usepackage[linesnumbered, algo2e, ruled,norelsize]{algorithm2e}
\usepackage{xcolor}
\usepackage{cite}
\newcommand\mycommfont[1]{\small\ttfamily\textcolor{darkgray}{#1}}
\SetCommentSty{mycommfont}

% specific symbol
\usepackage{gensymb}
%
% The IEEEtran \ifCLASSOPTIONcaptionsoff conditional can also be used
% later in the document, say, to conditionally put the References on a 
% page by themselves.




% correct bad hyphenation here
\hyphenation{op-tical net-works semi-conduc-tor}


\begin{document}
%
% paper title
% Titles are generally capitalized except for words such as a, an, and, as,
% at, but, by, for, in, nor, of, on, or, the, to and up, which are usually
% not capitalized unless they are the first or last word of the title.
% Linebreaks \\ can be used within to get better formatting as desired.
% Do not put math or special symbols in the title.
\title{Lossless Coding of Point Cloud Geometry using a Deep Generative Model}
%
%
% author names and IEEE memberships
% note positions of commas and nonbreaking spaces ( ~ ) LaTeX will not break
% a structure at a ~ so this keeps an author's name from being broken across
% two lines.
% use \thanks{} to gain access to the first footnote area
% a separate \thanks must be used for each paragraph as LaTeX2e's \thanks
% was not built to handle multiple paragraphs
%

\author{Dat Thanh Nguyen, Maurice Quach,~\IEEEmembership{Student Member,~IEEE,} Giuseppe Valenzise,~\IEEEmembership{Senior Member,~IEEE}, Pierre Duhamel,~\IEEEmembership{Life Fellow, IEEE}% <-this % stops a space
\thanks{D. T. Nguyen, M. Quach, G. Valenzise and P. Duhamel are with the Université Paris-Saclay, CNRS, CentraleSupelec, Laboratoire des Signaux et Systèmes (UMR 8506), 91190 Gif-sur-Yvette, France (email: \href{mailto:doandat.nguyen@gmail.com}{thanh-dat.nguyen@centralesupelec.fr}; \href{mailto:maurice.quach@l2s.centralesupelec.fr}{maurice.quach@l2s.centralesupelec.fr}; \href{mailto:giuseppe.valenzise@l2s.centralesupelec.fr}{giuseppe.valenzise@l2s.centralesupelec.fr}; \href{mailto:pierre.duhamel@l2s.centralesupelec.fr}{pierre.duhamel@l2s.centralesupelec.fr}).}}% <-this % stops a space
%\thanks{J. Doe and J. Doe are with Anonymous University.}% <-this % stops a space
%\thanks{Manuscript received April 19, 2005; revised August 26, 2015.}}



% The paper headers
%\markboth{Journal of \LaTeX\ Class Files,~Vol.~14, No.~8, August~2015}%
%{Shell \MakeLowercase{\textit{et al.}}: Bare Demo of IEEEtran.cls for IEEE Journals}


% make the title area
\maketitle

% As a general rule, do not put math, special symbols or citations
% in the abstract or keywords.
\begin{abstract}
\begin{abstract}
\label{sec:abstract}

%% 1. what is the problem 
Scientific applications that run on leadership computing facilities often face the challenge 
of being unable to fit leading science cases onto accelerator devices due to memory constraints 
(memory-bound applications).
%
% 2. what is your solution 
In this work, the authors studied one such US Department of Energy mission-critical condensed matter 
physics application, Dynamical Cluster Approximation (DCA++), and this paper discusses how device memory-bound challenges were successfully reduced  by proposing an effective 
``all-to-all'' communication method---a ring communication algorithm. 
%
This implementation takes advantage of acceleration on GPUs and remote direct memory access (RDMA) for fast data exchange between GPUs. 
%
\\Additionally, the ring algorithm was optimized with sub-ring communicators
and multi-threaded support to further reduce communication overhead and 
expose more concurrency, respectively.
%
% 3. What's the cherry-picked evaluation result you want to mention
The computation and communication were also analyzed 
by using the Autonomic Performance Environment for Exascale 
(APEX) profiling tool,  and this paper further discusses the 
performance trade-off for the ring algorithm implementation. 
%
The memory analysis on the ring algorithm shows that the allocation size for the authors' most 
memory-intensive data structure per GPU is now reduced to $1/p$ of the original size, where $p$ is the number of GPUs in the ring communicator.
%
The communication analysis suggests that 
the distributed Quantum Monte Carlo execution time grows linearly as sub-ring size increases, and the cost of messages passing through the network interface connector could be a limiting factor.


%
% \todoRed{Ronnie: Next sentence needs rewrite, too much information about Green's function that no one knows in the abstract; recommend generalizing.} \emph {However, DCA++ is currently facing memory-bound challenge as 
% a larger device array $G_t$ is limited by device memory size, where
% $G_t$ is a two-particle Green's function that allows condensed matter
% scientists to explore larger and more complex (higher fidelity)
% physics cases.}

\end{abstract}

\keywords{DCA++, Quantum Monte Carlo, GPU Remote Direct Memory Access, memory-bound issue, exascale machines}

\end{abstract}

% Note that keywords are not normally used for peerreview papers.
\begin{IEEEkeywords}
Point Cloud Coding, VoxelDNN, Deep Learning, G-PCC, context model, arithmetic coding.
\end{IEEEkeywords}


%
\section{Introduction}
% The very first letter is a 2 line initial drop letter followed
% by the rest of the first word in caps.
% 
% form to use if the first word consists of a single letter:
% \IEEEPARstart{A}{demo} file is ....
% 
% form to use if you need the single drop letter followed by
% normal text (unknown if ever used by the IEEE):
% \IEEEPARstart{A}{}demo file is ....
% 
% Some journals put the first two words in caps:
% \IEEEPARstart{T}{his demo} file is ....
% 
% Here we have the typical use of a "T" for an initial drop letter
% and "HIS" in caps to complete the first word.


\label{sec:intro}
\section{Introduction}  \label{sec:introduction}

\newcommand\inexpIntro[3]{#1?(#2,#3).}
\newcommand\rinexpIntro[3]{*#1?(#2,#3).}
\newcommand\outexpIntro[3]{#1!(#2,#3).}
\newcommand\outatomIntro[3]{#1!(#2,#3)}

We propose a fully automated method for proving termination of \(\pi\)-calculus processes.
Although there have been a lot of studies on termination analysis for the \(\pi\)-calculus
and related calculi~\cite{Deng06IC,Demangeon07,SangiorgiTermination,KobayashiHybrid,Yoshida04IC,DBLP:journals/jlp/DemangeonHS10,Venet98SAS}, most of them have been rather theoretical,
and there have been surprisingly little efforts in developing  fully automated termination
verification methods and tools based on them. To our knowledge,
Kobayashi's \typical{}~\cite{TyPiCal,KobayashiHybrid} is the only exception that
can prove termination of \(\pi\)-calculus processes (extended with natural numbers)
fully automatically, but its termination analysis is quite limited (see Section~\ref{sec:relatedwork}).

Our method is based on a reduction to termination analysis for sequential programs:
we translate a \(\pi\)-calculus process \(P\) to a sequential program \(S_P\), so that
if \(S_P\) is terminating, so is \(P\). The reduction allows us to use
powerful, mature methods and tools
for termination analysis of sequential programs~\cite{heizmann2016ultimate,freqterm,DBLP:conf/lics/PodelskiR04,Kuwahara2014Termination,DBLP:journals/cacm/CookPR11}.

The idea of the translation is to convert a chain of communications on replicated input
channels to a chain of recursive function calls of the target sequential program.
Let us consider the following Fibonacci process:
\begin{align*}
    & \rinexpIntro{\fib}{n}{r}
        \ifexp{n<2}{ \soutatom{r}{1} \\ &\quad}
                   { \nuexp{s_1} \nuexp{s_2} (\outatomIntro{\fib}{n-1}{s_1} \PAR \outatomIntro{\fib}{n-2}{s_2} \PAR \sinexp{s_1}{x}\sinexp{s_2}{y}\soutatom{r}{x+y}) \\}
    & \PAR \outatomIntro{\fib}{m}{r}
\end{align*}
Here, the process
$\rinexpIntro{\fib}{n}{r} \ldots$ is a function server that computes the \(n\)-th Fibonacci number
in parallel and returns the result to \(r\),
and $\outatom{\fib}{m}{r}$ sends a request for computing the \(m\)-th Fibonacci number;
those who are not familiar with the syntax of the \(\pi\)-calculus may wish to consult
Section~\ref{sec:targetlanguage} first.
To prove that the process above is terminating for any integer \(m\),
it suffices to show that there is no infinite chain of communications on $\fib$:
\[
    \fib(m,r) \to \fib(m_1,r_1) \to \fib(m_2,r_2) \to \cdots.
\]
We convert the process above to the following program:\footnote{The actual translation
  given later is a little more complex.}
\begin{verbatim}
 let rec fib(n) = if n<2 then () else (fib(n-1) [] fib(n-2)) in
 fib(m)
\end{verbatim}
Here, \texttt{[]} represents the non-deterministic choice.
Note that, although the calculation of Fibonacci numbers is not preserved,
for each chain of communications on \texttt{fib}, there is a corresponding
sequence of recursive calls:
\[
\mathtt{fib}(m) \to \mathtt{fib}(m_1) \to \mathtt{fib}(m_2) \to \cdots.
\]
Thus, the termination of the sequential program above implies the termination of
the original process.
As shown in the example above, (i) each communication on a replicated input channel
is converted to a function call, (ii) each communication on a non-replicated input
channel is just removed (or, in the actual translation, replaced by a call of
a trivial function defined by \(f(\seq{x})=(\,)\)), and (iii) parallel composition
is replaced by a non-deterministic choice.
We formalize the translation outlined above and prove its correctness.

The basic translation sketched above sometimes loses too much information.
For example, consider the following process:
\begin{align*}
    & \rinexpIntro{\pre}{n}{r} \soutatom{r}{n-1} \\
    & \PAR \rinexpIntro{f}{n}{r} \ifexp{n<0}{ \soutatom{r}{1} }
                                       { \nuexp{s} (\outatomIntro{\pre}{n}{s} \PAR \sinexp{s}{x}\outatomIntro{f}{x}{r}) } \\
    & \PAR \outatomIntro{f}{m}{r}
\end{align*}
The translation sketched above would yield:
\begin{verbatim}
  let pred(n) = n-1 in
  let rec f(n) = if n<0 then () else (pred(n) [] f(*)) in
  f(m)
\end{verbatim}
Here, \texttt{*} represents a non-deterministic integer: since we have removed
the input $\sinatom{s}{x}$, we do not have information about the value of \( x \).
As a result, the sequential program above is non-terminating, although the original
process is terminating.
To remedy this problem, we also refine the basic translation above by using a refinement
type system for the \(\pi\)-calculus. Using the refinement type system,
we can infer that the value of \(x\) in the original process is less than \(n\),
so that we can refine the definition of \texttt{f} to:
\begin{verbatim}
 let rec f(n) = ... else (pred(n) [] let x=* in assume(x<n);f(x))
\end{verbatim}
The target program is now terminating, from which
we can deduce that the original process is also terminating.
We have implemented an automated tool based on the refined translation above.

The contributions of this paper are summarized as follows.
\begin{itemize}
\item The formalization of the basic translation from the \(\pi\)-calculus
  (extended with integers) to sequential programs, and a proof of its correctness.
\item The formalization of a refined translation based on a refinement type system.
\item An implementation of the refined translation, including automated refinement type
  inference based on CHC solving, and experiments to evaluate the effectiveness of
  our method.
\end{itemize}

The rest of this paper is structured as follows.
Section~\ref{sec:targetlanguage} introduces the source and target languages
of our translation.
Section~\ref{sec:approach} 
formalizes the basic translation, and proves its correctness.
Section~\ref{sec:refinement} refines the basic translation by using a refinement type system.
Section~\ref{sec:implementation} reports an implementation and experiments.
Section~\ref{sec:relatedwork} discusses related work,
and Section~\ref{sec:conclusion} concludes the paper.





\section{Related work}
\label{sec:stateoftheart}


\par Relevant work related to this paper includes state-of-the-art PC geometry coding and learning-based methods in image and point cloud compression.

\subsection{MPEG G-PCC and Conventional Lossless Codecs}
\par Most existing methods that compress point cloud geometry, including MPEG G-PCC, use octree coding \cite{schnabel2006octree,7434610,6224647, garcia2017context, garcia2018intra, garcia2019geometry,huang2020octsqueeze, biswas2020muscle} and local approximations called ``triangle soups'' (trisoup)~\cite{schnabel2006octree,dricot2019adaptive}. 
\par In the G-PCC geometry coder, points are first transformed and voxelized into an axis-aligned bounding box before geometry analysis using trisoup or octree scheme. In the trisoup coder, geometry can be represented by a pruned octree plus a surface model. This model approximates the surface in each leaf of the pruned octree using 1 to 10 triangles. 
% This technique is known as triangle soup. 
In contrast, the octree coder partitions voxelized blocks until sub-cubes of dimension one are reached. First, the coordinates of isolated points are independently encoded to avoid "polluting" the octree coding (Direct Coding Mode - DCM) \cite{dcm}. To encode the occupancy pattern of each octree node, G-PCC introduces many methods to exploit local geometry information and obtain an accurate context for arithmetic coding, such as Neighbour-Dependent Entropy Context \cite{neighbor}, intra prediction \cite{intracodinggpcc}, planar/angular coding mode \cite{planarcodingmode,angularcodingmode}, etc. The lossless geometry coding mode of G-PCC is based on octree coding only.

\par In order to deal with the irregular point space, many octree-based lossless PCC methods have been proposed. In \cite{schnabel2006octree}, the authors proposed an octree-based method which aims at reducing entropy by employing prediction techniques based on local surface approximations to predict occupancy patterns. Recently, more context modeling based approaches are proposed \cite{garcia2017context, garcia2018intra, garcia2019geometry}. For example, the intra-frame compression method P(PNI) proposed in \cite{garcia2019geometry} builds a reference octree by propagating the parent octet to all children nodes, thus providing 255 contexts to encode the current octant. Octree coding allows for a progressive representation of point clouds since each level of the octree is a downsampled version of the point cloud. However, a drawback of octree representation is that, at the first levels of the tree, it produces ``blocky'' scenes, and geometry information of point clouds (i.e., curve, plane) is lost. The authors of \cite{8122226} proposed a binary tree based method which analyzes the point cloud geometry using binary tree structure and realizes an intra prediction via the extended Travelling Salesman Problem (TSP) within each leaf node. Instead, in this paper, we employ a hybrid octree/voxel representation to better exploit the geometry information. Besides,  the methods in \cite{garcia2017context, garcia2018intra, garcia2019geometry} produce frequency tables which are collected from the coarser level or the previous frame and must be transmitted to the decoder. Our method predicts voxel distributions in a sequential manner at the decoder side, thus avoiding the extra cost of transmitting large frequency tables.
%  The lossless geometry compression method of \cite{6224647} is based on predictive coding with inter-frame prediction
\subsection{Generative Models and Learning-based Compression}
\par 
% Learning probabilistic models that return explicit probability densities from training data is the central problem in unsupervised learning.
Estimating the data distribution from a training dataset is the main objective of generative models, and is a central problem in unsupervised learning.
It has a number of applications, from image generation \cite{theis2015generative,gregor2015draw, oord2016pixel,salimans2017pixelcnn++}, to image compression \cite{491334, balle2016end,mentzer2018conditional} and denoising \cite{chen2018image}. Among the several types of generative models proposed in the literature \cite{10.5555/2969033.2969125}, auto-regressive models such as PixelCNN \cite{oord2016pixel,salimans2017pixelcnn++} are particularly relevant for our purpose as they allow to compute the exact likelihood of the data and to generate realistic images, although with a high computational cost. Specifically, PixelCNN factorizes the likelihood of a picture by modeling the conditional distribution of a given pixel's color given all previously generated pixels. These conditional distributions only depend on the possible pixel values with respect to the scanned context, which imposes a \textit{causality} constraint. PixelCNN models the distribution using a neural network and the causality constraint is enforced using masked filters in each convolutional layer. Recently, this approach has also been employed in image compression to yield accurate and learnable entropy models~\cite{mentzer2018conditional}. Our paper explores the potential of this approach for point cloud geometry compression by adopting and extending conditional image modeling and masking filters into the 3D voxel domain.

\par Inspired by the success in learning-based image compression, deep learning has been widely adopted in point cloud coding  both in the octree domain \cite{huang2020octsqueeze,biswas2020muscle},  voxel domain \cite{8954537,9191021,quach2019learning,quach2020improved,wang2019learned,guarda2020point}  and point domain \cite{yan2019deep,huang20193d, wang2020multiscale}. Recently, the authors of \cite{huang2020octsqueeze} proposed an octree-based entropy model that models the probability distributions of the octree symbols based on the contextual information from octree structure. This method only targets static LiDAR point cloud compression. The extension version for intensity-valued LiDAR streaming data using spatio-temporal relations is proposed in \cite{biswas2020muscle}. However, these methods target dynamically acquired point clouds, while in this paper we mainly focus on dense static point clouds.
\par Working in the voxel domain enables to easily extend most 2D tools, such as convolutions, to the 3D space. 
Many recent 3D convolution based autoencoder approaches for lossy coding \cite{quach2019learning,quach2020improved,wang2019learned,guarda2020point} compress 3D voxelized blocks into latent representations and cast the reconstruction as a binary classification problem. The authors of \cite{yan2019deep} proposed a pointnet-based auto-encoder method which directly takes points as input rather than voxelized point cloud.  To handle sparse point clouds, recent methods leverage advances in sparse convolution \cite{choy20194d,graham2017submanifold} to allow point-based approaches  \cite{huang20193d, wang2020multiscale}. For example, the proposed lossy compression method in \cite{wang2020multiscale} progressively downscale the point cloud into multiple scales using sparse convolutional transforms. Then, at the bottleneck, the geometry of scaled point cloud is encoded using an octree codec and the attributes are compressed using a learning-based context model.  In contrast, in this paper, we focus on dense voxelized point clouds and losslessly encode each voxel using  the learned distribution from its 3D context. In addition, we apply this approach in a block-based fashion, which has been successfully employed in traditional image and video coding.
%and tackle the sparsity by representing point cloud in hybrid octree/voxel domain


\section{Proposed method}
\label{proposedmethod}

\section{Flow-Packet Hybrid Traffic Classification}
\label{sec:proposed}

We propose FPHTC for a router that needs to conduct class-aware traffic processing. In this section, we provide a detailed description of our scheme. A diagram illustrating the overall framework of FPHTC is given in Fig.~\ref{fig:scheme}.


\subsection{Core Components of FPHTC}
\subsubsection{Router}
The router accepts an incoming stream of packets and processes them according to their service classes using the routing policy. The basic structure and function of such a routing policy are well-defined in prior works on packet classification \cite{Gupta99, Gupta01}. Throughout our work, we focus on how to generate routing
policy rules by training a machine learning model for packet-based traffic classification, where the chosen header fields of each packet are its features, i.e., the inputs into the learning model, and the packet is classified by the learning model to determine its CoS. For example, the chosen header fields may be the source IP address, destination IP address, source port number, and destination port number, among others, and the possible actions may be to route a packet as delay sensitive, delay moderate, or delay tolerant.

\subsubsection{Flow-based Traffic Classifier}
The flow-based traffic classifier resides outside the router, in some powerful equipment that can handle the heavy computation required by sophisticated machine learning techniques. It is a complex and highly accurate machine learning model that can classify a traffic flow in terms of CoS for all of its packets. It is trained using a number of bidirectional TCP flows with a set of flow-level statistical features extracted from the raw dataset.

Various methods are possible to generate the training dataset for the flow-based traffic classifier. In this work, since we are ultimately interested in online classification to handle changing traffic pattern over time, we propose to use a continuously updated recording of the past traffic. Specifically, we use a traffic mirror and a traffic selector, as shown in Fig.~\ref{fig:scheme}, to separate a selected small portion of the incoming traffic flows. The selected flows are then labeled using a Deep Packet Inspection (DPI) module according to their CoS. The true CoS labels obtained by DPI are used to train the flow-based classifier. We note that DPI cannot be used to replace the role of the flow-based classifier for all flows, due to its prohibitive cost and delay for common encrypted traffic. 

The role of the flow-based traffic classifier designer includes data preprocessing, hyperparameter selection, and finally, training the flow-based classifier. Once the flow-based classifier is trained, we use it to infer the CoS labels of all incoming flows captured by the traffic mirror. Then all packets belonging to a flow can be tagged by CoS label of the flow. We note that the CoS labels generated in this way, by a flow-based classifier, are too late to be used in the \textit{routing} of the labeled packets. However, what this achieves is to create a packet-level dataset for \textit{training} the packet-based routing policy as explained below.

\subsubsection{Packet-based Routing Policy Designer}

The packet-based routing policy designer takes labeled packets from the flow-based classifier as input, and it outputs a routing policy for the router. Specifically,  the routing policy designer trains a packet-based classifier using the labeled packets as the training dataset. 

In this work, we use the binary decision tree learning model for the packet-based classifier. In the decision tree, each path from the root to a node is a routing policy rule. Thus, to obtain routing policy rules that can be used in the router, the routing policy designer only needs to train a decision tree on the packet-level dataset. Furthermore, we note that the number of routing policy rules equals the number of leaf nodes in the decision tree. This provides an easy way to control the size of the routing policy, i.e., the routing policy designer can limit the maximum number of leaf nodes while training the decision tree.

\subsection{Construction of Routing Policy}

The construction of the routing policy in FPHTC involves transferring learned knowledge from the flow-based classifier to the routing policy designer. In the machine learning literature, knowledge distillation \cite{Hinton15, Vapnik16} is a technique where a simple student model is trained on the predictions supplied by a highly accurate and complex teacher model. In FPHTC, we train a decision tree at the routing policy designer using the predictions from the flow-based classifier as training targets. In essence, the routing policy designer tries to approximate the performance of the flow-based classifier. 

The flow-based classifier is trained with flow-level statistical features whereas the routing policy designer uses only some features that can be read directly from the packet header. Therefore, it is clear that the learned routing policy will perform worse than the flow-based classifier given the same traffic data for training. However, since there are unlabeled training data available, i.e., those that have not been labeled by DPI, we can label those data samples using our flow-based classifier to substantially enlarge the training dataset for the routing policy designer. Since the decision tree at the routing policy designer is trained on a much larger dataset than that of the flow-based classifier, the performance of the routing policy can be close to that of the flow-based classifier. More importantly, since the routing policy created by FPHTC utilizes information learned from a more powerful flow-based classifier, it can substantially outperform a regular packet-based classifier trained using only the small amount of labels generated by DPI.


\subsection{Routing Policy Update Procedure in Online Setting}

In a practical system, the data pattern of the incoming traffic changes over time, e.g., due to new applications appearing in the network, or changing user behavior. Therefore, we design FPHTC to dynamically update the routing policy over time.

In Fig.~\ref{fig:online}, we illustrate how the modules sequentially function over a continuous stream of traffic. At any given time slot, we collect and label a small portion of the incoming traffic flows using DPI to train the flow-based classifier. Meanwhile, we continue to collect flows to be used in the training of the routing policy. Once the flow-based classifier is trained, we use it to label those collected flows not labeled by DPI. Then, the routing policy designer trains a decision tree to generate the routing policy, which is then updated to the router. 

One important question is whether we should repeat these steps and update the routing policy at each time slot. If the traffic data pattern does not change too frequently, routing policy update at every time slot would be a waste of resources. To re-train the flow-based classifier, the labeling cost using DPI would also be expensive. A cost-effective solution is to update the routing policy only when the traffic pattern has altered significantly. This can be inferred by measuring the performance deterioration at the router. A feedback signal can be generated, for example, based on the increase in packet drop or congestion, to indicate that a routing policy update is necessary. We demonstrate the adaptiveness of FPHTC in the online setting in Section \ref{sec:results}.

\begin{figure}[t]
	\centering
	\includegraphics[width=9cm]{"figures/online".pdf}
	\caption{FPHTC in online setting.}
	\label{fig:online}
\end{figure}


\section{Experimental Results}
\label{performanceeval}
\newcommand{\twomoons}{{\tt Twomoons}}
\newcommand{\gauss}{{\tt Gauss}}
\newcommand{\sculpture}{{\tt Sculpture}}
\newcommand{\baseline}{{\tt Baseline}}
\newcommand{\MM}{{\tt MsgPassing}}
\newcommand{\blackboard}{{\tt Blackboard}}
\newcommand{\ncut}{\text{ncut}}
\newcommand{\chensays}[2][]{\textcolor{blue} {\textsc{Jiecao #1:} \emph{#2}}}

\section{Experiments}
In this section we present experimental results for  graph clustering in the message passing and blackboard models. We will compare the following three algorithms. (1) \baseline: each site sends all the data to the coordinator directly; (2) \MM: our algorithm in the message passing model (Section~\ref{sec:gcmessage}); (3) 
\blackboard: our algorithm in  the blackboard model (Section~\ref{sec:bb}).


%Since both of our algorithms are crucially based on the use of spectral scarification, our main focus in the experiments is to investigate to what extend the quality of the spectral clustering algorithms will be affected by using spectral sparsification, the saving of communication costs by using spectral sparsificaion, ...
%
%
%The goal of this experiment is not to demonstrate the effectiveness of the spectral clustering algorithm. We mainly want to investigate the following, 
%\begin{itemize}
%\item to what extend the quality of clustered results will be affected by using spectral sparsification.
%\item saving of communication costs by using spectral sparsifier.
%\item the affect of constants in algorithms of the message passing/blackboard model.
%\end{itemize}
%
%
%\subsection{The Setup}
%\paragraph{Reference Algorithms}
%We compare different algorithms in our experiment.

%Note that we can also run \MM~ in the blackboard model.

Besides giving the visualized results of these algorithms on various datasets, we also measure the qualities of the results via the {\em normalized cut}, defined as 
\[
\ncut(A_1, \ldots, A_{k}) = \frac{1}{2}\sum_{i\in[k]}\frac{w(A_i, V\backslash A_i)}{\vol(A_i)},
\]
 which is a standard objective function to be minimized for spectral clustering algorithms. 
%We will compare the communication costs of these algorithms in different settings.

%We also compare the total communication costs of different algorithms/models. As the unit does not matter in our case, we normalize all communication costs by the cost of \baseline.  Whenever possible, we will visualize the clustered results.

We implemented the algorithms using multiple languages, including Matlab, Python and C++. Our experiments were conducted on an IBM NeXtScale nx360 M4 server, which is equipped with 2 Intel Xeon E5-2652 v2 8-core processors, 32GB RAM and 250GB local storage.


\subsection{Datasets.}
We test the algorithms in the following real and synthetic datasets, which is visualized in \figref{visualization}.


\begin{figure}[h]
     \centering
     \subfigure[\twomoons]{\includegraphics[width=0.23\textwidth]{twomoons-14000-original.png}\label{fig:twomoons}}
     ~~
     \subfigure[\gauss]{\includegraphics[width=0.23\textwidth]{gauss-10000-original.png}\label{fig:gauss}}
     ~~
     \subfigure[\sculpture]{\includegraphics[width=0.13\textwidth,height=0.16\textwidth]{sculpture-11680-original.jpg}\label{fig:sculpture}}
     \caption{Visualization of the datasets for our experiments.}
     \label{fig:visualization}
\end{figure}



\vspace{-1mm}
\begin{itemize}
\item \twomoons : this dataset contains $n=14,000$ coordinates in $\mathbb{R}^2$. We consider each point to be a vertex. For any two vertices $u, v$, we add an edge with weight $w(u,v) = \exp\{-\|u-v\|_2^2/\sigma^2\}$ with $\sigma = 0.1$ when one vertex is among the $7000$-nearest points of the other.  This construction results in a graph with about $110,000,000$ edges.

\item  \gauss : this dataset contains $n = 10,000$ points in $\mathbb{R}^2$. There are $4$ clusters in this dataset, each generated using a Gaussian distribution. We construct a complete graph as the similarity graph.  For any two vertices $u, v$, we define the weight $w(u,v) = \exp\{-\|u-v\|_2^2/\sigma^2\}$ with $\sigma = 1$. The resulting graph has about $100,000,000$ edges.

\item \sculpture : a photo of \textit{The Greek Slave}~\footnote{Available in e.g., \url{http://artgallery.yale.edu/collections/objects/14794}}. We use an $80\times 150$ version of this photo where each pixel is viewed as a vertex. To construct a similarity graph, we map each pixel to a point in $\mathbb{R}^5$, i.e., $(x, y, r, g, b)$, where the latter three coordinates are the RGB values. For any two vertices $u, v$, we  put an edge between $u, v$ with weight $w(u,v) = \exp\{-\|u-v\|_2^2/\sigma^2\}$ with $\sigma = 0.5$ if one of $u, v$ is among the $5000$-nearest points of the other. This results in a graph with about $70,000,000$ edges.
\end{itemize}
\vspace{-1mm}
In the distributed model edges are randomly partitioned across $s$ sites. 

%\vspace{-1.5mm}



\subsection{Results on clustering quality}
%{\em Quality.} \
\begin{figure*}[ht]
     \centering
     \subfigure[\baseline]{\includegraphics[width=0.2\textwidth]{twomoons-14000-original-clustered.png}\label{fig:twomoons-clustered-original}}
     \subfigure[\MM]{\includegraphics[width=0.2\textwidth]{twomoons-14000-sparsify-clustered-15.png}\label{fig:twomoons-clustered-sparsify}}
     \subfigure[\blackboard]{\includegraphics[width=0.2\textwidth]{twomoons-14000-chain-clustered.png}\label{fig:twomoons-clustered-chain}}
     \caption*{\twomoons, $k = 2$;}

\subfigure[\baseline]{\includegraphics[width=0.2\textwidth]{gauss-10000-original-clustered.png}\label{fig:gauss-clustered-original}}
     \subfigure[\MM]{\includegraphics[width=0.2\textwidth]{gauss-10000-sparsify-clustered-15.png}\label{fig:gauss-clustered-sparsify}}
     \subfigure[\blackboard]{\includegraphics[width=0.2\textwidth]{gauss-10000-chain-clustered.png}\label{fig:gauss-clustered-chain}}
     \caption*{\gauss, $k = 4$}


     \subfigure[\baseline]{\includegraphics[width=0.2\textwidth,height=0.2\textwidth]{sculpture-11680-original-clustered.png}\label{fig:sculpture-clustered-original}}  
     \subfigure[\MM]{\includegraphics[width=0.2\textwidth,height=0.2\textwidth]{sculpture-11680-sparsify-clustered-15.png}\label{fig:sculpture-clustered-sparsify}}
     \subfigure[\blackboard]{\includegraphics[width=0.2\textwidth,height=0.2\textwidth]{sculpture-11680-chain-clustered.png}\label{fig:sculpture-clustered-chain}}
     \caption*{\sculpture, $k = 3$. }


     
     \caption{Visualization of the results on \twomoons, \gauss\ and \sculpture. In the message passing model each site samples $5 n$ edges; in the blackboard model all sites jointly sample $10n$ edges (in \twomoons~ and \gauss) or $20n$ edges (in \sculpture) and the chain has length $18$. $s = 15$.}
     \label{fig:quality-1}
\end{figure*}

We visualize the clustered results for 
the \twomoons, \gauss\ and \sculpture\ in Figure~\ref{fig:quality-1}.
% and visualize the clustered results for \gauss\ and \sculpture in Figure~\ref{fig:quality-2}.
It can be seen that \baseline, \MM\ and \blackboard\ give results of very similar qualities.  For simplicity, here we only present the visualization for $s=15$. Similar results were observed when we varied the values of $s$.  
%\he{To Qin: Do you plan to have two titles (Results \& Quality)?}


% \begin{figure*}[h]
%      \centering
% \subfigure[\baseline]{\includegraphics[width=0.3\textwidth]{gauss-10000-original-clustered.png}\label{fig:gauss-clustered-original}}
%      \subfigure[\MM]{\includegraphics[width=0.3\textwidth]{gauss-10000-sparsify-clustered-15.png}\label{fig:gauss-clustered-sparsify}}
%      \subfigure[\blackboard]{\includegraphics[width=0.3\textwidth]{gauss-10000-chain-clustered.png}\label{fig:gauss-clustered-chain}}
%      \caption*{\gauss, $k = 4$}


%      \subfigure[\baseline]{\includegraphics[width=0.2\textwidth]{sculpture-11680-original-clustered.png}\label{fig:sculpture-clustered-original}}  
%      \subfigure[\MM]{\includegraphics[width=0.2\textwidth]{sculpture-11680-sparsify-clustered-15.png}\label{fig:sculpture-clustered-sparsify}}
%      \subfigure[\blackboard]{\includegraphics[width=0.2\textwidth]{sculpture-11680-chain-clustered.png}\label{fig:sculpture-clustered-chain}}
%      \caption*{\sculpture, $k = 3$. }

%      \caption{Visualization of results on \gauss\ and \sculpture; in the message passing model each site samples $5 n$ edges; in the blackboard model all sites jointly sample $10n$ (in \gauss) or $20n$ (in \sculpture) edges and the chain has length $18$.}
%      \label{fig:quality-2}
% \end{figure*}


We also compare the normalized cut (ncut) values of the clustering results of different algorithms.  The results are presented in Figure \ref{fig:quality}. In all datasets, the ncut values of different algorithms are very close. The ncut value of \MM\ slightly decreases when we increase the value of $s$, while the ncut value of \blackboard\ is independent of $s$.
%We comment that in general, it is difficult to compare \MM\ and \blackboard\ directly because they are affected by different parameters.


\begin{figure*}[!ht]
  \centering
  \subfigure[\twomoons]{\includegraphics[width=0.33\textwidth]{twomoons-14000-ncut.png}\label{fig:twomoons-quality}}\hspace*{-1.1em}
  \subfigure[\gauss]{\includegraphics[width=0.31\textwidth]{gauss-10000-ncut.png}\label{fig:gauss-quality}}\hspace*{-1.1em}
  \subfigure[\sculpture]{\includegraphics[width=0.31\textwidth]{sculpture-11680-ncut.png}\label{fig:sculpture-quality}}\hspace*{-1.1em}
  \subfigure{\includegraphics[width=0.14\textwidth]{legend.png}}
     \caption{Comparisons on normalized cuts. In the message passing model, each site samples $5n$ edges; in each round of the algorithm in the blackboard model, all sites jointly sample $10n$ edges (in \twomoons~and \gauss) or $20n$ edges (in \sculpture) edges and the chain has length $18$.}
     \label{fig:quality}
\end{figure*}

%\textcolor{red}{To Jiecao: Can you put the color lines indicating baseline, message passing, and blackboard within one row in Pic 2? Withthis we can save some space.}

%\vspace{-1.5mm}

\subsection{Results on communication costs} 
\begin{figure*}[!ht]
     \centering
     \subfigure[\twomoons]{\includegraphics[width=0.3\textwidth]{twomoons-14000-communication.png}\label{fig:twomoons-communication}}
     \subfigure[\gauss]{\includegraphics[width=0.3\textwidth]{gauss-10000-communication.png}\label{fig:gauss-communication}}
     \subfigure[\sculpture]{\includegraphics[width=0.3\textwidth]{sculpture-11680-communication.png}\label{fig:sculpture-communication}}


     \subfigure[\twomoons]{\includegraphics[width=0.32\textwidth]{twomoons-14000-communication-2.png}\label{fig:twomoons-communication-2}}
     \subfigure[\gauss]{\includegraphics[width=0.32\textwidth]{gauss-10000-communication-2.png}\label{fig:gauss-communication-2}}
     \subfigure[\sculpture]{\includegraphics[width=0.32\textwidth]{sculpture-11680-communication-2.png}\label{fig:sculpture-communication-2}}
     \caption{Comparisons on communication costs. In the message passing model, each site samples $5n$ edges; in each round of the algorithm in the blackboard model, all sites jointly sample $10n$ (in \twomoons~and \gauss) or $20n$ (in \sculpture) edges and the chain has length $18$. }
     \label{fig:communication}
\end{figure*}

We compare the communication costs of different algorithms in Figure \ref{fig:communication}. We observe that while achieving similar clustering qualities as \baseline, both \MM\ and \blackboard\ are significantly more communication-efficient (by one or two orders of magnitudes in our experiments). We also notice that the value of $s$ does not affect the communication cost of \blackboard, while the communication cost of \MM\ grows almost linearly with $s$; when $s$ is large, \MM\ uses significantly more communication than \blackboard. These confirm our theory.  %In Figure~\ref{fig:mm-const} and Figure~\ref{fig:blackboard-const}   in Appendix~\ref{sec:parameters} we present how the performance of \MM\ and \blackboard\ are affected by their parameters.

%
%
%\vspace{-1.5mm}
%\paragraph{Summary.}  From our experimental results we conclude that \MM\ and \blackboard\ achieve similar clustering quality as the native algorithm \baseline, while significantly reduce the communication cost.  When the number of sites is large, \blackboard\ is more communication efficient than \MM, as predicted by our theory.



\subsection{Parameters in \MM\ and \blackboard}
\label{sec:parameters}

Figure \ref{fig:mm-const} shows in \MM how the value of ncut is affected by the number of sites and the number of edges sampled in each site. 
Here, each site samples $cn$ edges. 
When $c=3$ and $s=1$, the ncut value diverges in all datasets. This is because with such a small $c$, the algorithm does not generate a valid sparsifier. In general, increasing $c$ or $s$ will slightly decrease the ncut value. But once they are above some thresholds, the ncut values of \MM\ and \baseline\ become very close.

Figure \ref{fig:blackboard-const} shows in \blackboard  how the ncut value is affected by the number of iterations and the number of edges sampled. When the number of iterations is set to be $5$, ncut values diverge in all datasets. This is because we cannot expect to generate a valid sparsifier by using such few iterations. It can be seen from \ref{fig:bb-gauss-constant} that for a fixed $c$, performing more iterations will help to reduce ncut values. From the same figure, one can also conclude that for fixed iterations, increasing $c$ also helps to reduce the ncut values.



\begin{figure*}[h!t]
     \centering
     \subfigure[\twomoons]{\includegraphics[width=0.3\textwidth]{twomoons-c.png}\label{fig:mm-twomoons-constant}}
     \subfigure[\gauss~dataset]{\includegraphics[width=0.3\textwidth]{gauss-c.png}\label{fig:mm-gauss-constant}}
     \subfigure[\sculpture]{\includegraphics[width=0.3\textwidth]{sculpture-c.png}\label{fig:mm-sculpture-constant}}
     \caption{The pictures above show the $\ncut$ values with respect to the values of $c$ and $s$ for the \MM\ algorithm. Here  
 each site samples $c n$ edges.}
     \label{fig:mm-const}
\end{figure*}


\begin{figure*}[h!t]
     \centering
     \subfigure[\twomoons]{\includegraphics[width=0.3\textwidth]{twomoons-iter.png}\label{fig:bb-twomoons-constant}}
     \subfigure[\gauss]{\includegraphics[width=0.3\textwidth]{gauss-iter.png}\label{fig:bb-gauss-constant}}
     \subfigure[\sculpture]{\includegraphics[width=0.3\textwidth]{sculpture-iter.png}\label{fig:bb-sculpture-constant}}
     \caption{The pictures above show how the $\ncut$ values are affected by the number of iterations and the value of $c$ for the \blackboard\ algorithm. Here 
all sites jointly sample $c n$ edges. }
     \label{fig:blackboard-const}
\end{figure*}







\section{Conclusions and future work}
\label{conclusion}

\begin{comment}
\begin{figure}
\includegraphics[width=\linewidth]{figs/beyond_tss_lesion.pdf}
\caption[]{End-to-End runtime lesion study of the entire MNIST dataset and the FMA featurized music dataset. Each of DROP's contributions provides a runtime improvement.}
\label{fig:beyond_lesion}
\end{figure}
\end{comment}



\section{Conclusion}
\label{sec:conclusion}

Advanced data analytics techniques must scale to rising data volumes. 
DR techniques offer a powerful toolkit when processing these datasets, with PCA frequently outperforming popular techniques in exchange for high computational cost. 
In response, we propose DROP, a new dimensionality reduction optimizer. 
DROP combines progressive sampling, progress estimation, and online aggregation to identify high quality low dimensional bases via PCA without processing the entire dataset by balancing the runtime of downstream tasks and achieved dimensionality. 
Thus, DROP provides a first step in bridging the gap between quality and efficiency in end-to-end DR for downstream \red{analytics}. 

%We revisit canonical operators for time series dimensionality reduction and the measurement study of~\cite{keogh-study}, and show that PCA is more effective than popular alternatives in the data mining literature often by a margin of over $2\times$ on average on gold-standard time series benchmark data sets with respect to output data dimension. More surprisingly, we empirically demonstrate that a small number of samples are sufficient to accurately characterize directions of maximum variance and obtain a high-quality low-dimensional transformation.







% Can use something like this to put references on a page
% by themselves when using endfloat and the captionsoff option.
\ifCLASSOPTIONcaptionsoff
  \newpage
\fi




\bibliographystyle{./IEEEtran}
\bibliography{./IEEEabrv,./refs}

\end{document}


