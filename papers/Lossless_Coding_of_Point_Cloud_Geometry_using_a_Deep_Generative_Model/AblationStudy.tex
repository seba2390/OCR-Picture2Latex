%\par \textbf{Single-resolution encoder}: In order to show the effectiveness of the partitioning scheme, we perform the encoding experiments with fixed block size. Each $64 \times 64 \times 64$ block is divided into blocks of size 64, 32, 16 or 8 and a single bit is used to indicate the occupancy of each sub-block. In addition, occupied blocks will be encoded using VoxelDNN. Table \ref{table:singleres} shows the $bpov$ of each single-resolution encoder compared with multi-resolution VoxelDNN and G-PCC version 10. We observe that encoders with block size $32$ and $64$ outperform other single-resolution encoders with the gain over G-PCC more than $20\%$ on both MVUB and MPEG 8i. This can be explained by the fact that less contexts are provided to VoxelDNN at a lower resolution, and we thus do not have a good probability estimation. Besides, as we shown in the multi-res VoxelDNN, we should only partition sparse blocks into small blocks to eliminate the redundancy instead of dividing all blocks.
\input{table_bits_spent_on_block}
\begin{center}
\centering
\begin{table}[H]
\caption{Average bits percentage in total bitstream and number of occupied voxel for each block size with four partitioning levels }
\resizebox{\linewidth}{!}{ \begin{tabular}{M{1.7cm} P{3cm} P{3cm}}


Block size&\% in bitstream & Avg occupied voxels\\
\hline
64&71.5\%&7374\\

32&24.3\%&1224\\

16&4.0\%&119\\

8&0.2\%&13\\
\hline
\end{tabular}}
\label{table:bitpercentage}

\end{table}

\end{center}
\vspace{-1cm}
%Table \ref{table:bitpercentage} shows the bits percentage in the bitstream and the number of occupied voxel for each block size with four partitioning level encoder. It can be seen that most of the bits are spent for block 64 and 32, while block 16 and 8 only account for $4.2\%$ in bitstream. This explains why we only have a small gain when adding block 16 and 8 into the partitioner.