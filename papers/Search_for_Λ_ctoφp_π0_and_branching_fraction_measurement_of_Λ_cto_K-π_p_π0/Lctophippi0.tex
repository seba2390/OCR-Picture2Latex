% %%%%%%%%%%%%%%%%%%%%%%%%%%%%%%%%%%%%%%%%%%%%%%%%%%%%
%%% Paper:    Search for $\Lambda_c^+\to\phi p \pi^0$ and measurement of  $\Lambda_c^+\to K^-\pi^+ p \pi^0$
%%% Journal:  Phys. ReV. D (RC)
%%% Contacts: Dr. Bilas Pal 
%%%                 University of Cincinnati
%%%                email:  palbs@ucmail.uc.edu, bkpal.syr@gmail.com
%%%                phone: 513-556-3928, 315-706-3888
% %%               fax:   513-556-3425
%%% %%%%%%%%%%%%%%%%%%%%%%%%%%%%%%%%%%%%%%%%%%%%%%%%%%

\RequirePackage{lineno}

\documentclass[aps,prl,twocolumn,superscriptaddress,showpacs,preprintnumbers,amsmath,amssymb]{revtex4-1}
%\documentclass[aps,prl,tightenlines,superscriptaddress,showpacs,preprintnumbers,amsmath,amssymb]{revtex4-1}

% Some other (several out of many) possibilities
%\documentclass[preprint,aps]{revtex4}
%\documentclass[preprint,aps,draft]{revtex4}

\usepackage{graphicx} % Include figure files
\usepackage{dcolumn}  % Align table columns on decimal point
\usepackage{color} % 
%\usepackage[hypertex]{hyperref} % Make link for reference for dvi file
\RequirePackage{xspace}
\usepackage{relsize}

\graphicspath{{ps}}

%\setpagewiselinenumbers
%\linenumbers
\begin{document}


%\preprint{\vbox{ \hbox{   }
%		     \hbox{\textbf  {Belle Preprint  2017-12}}
  %                           \hbox{\textbf {KEK Preprint  2017-08}}
 %                           \hbox{\textbf  {UCHEP Preprint  2017-06}}
%}}

%\vskip -1.0cm
\title{ \quad\\[1.0cm] 
Search for \mbox{\boldmath$\Lambda_c^+\to\phi p \pi^0$} and branching fraction measurement of  \mbox{\boldmath$\Lambda_c^+\to K^-\pi^+ p \pi^0$} }
%%%%%%%%%%%%%%%%%%%%%%%%%%%%%%%%%%%%%%%%%%%%%%%%
%\collaboration{The Belle Collaboration}

%%%%%%%%%%%%%%%%%%%%%%%%%%%%%%%%%%%%%%%%%%%%%%%%%

%\section*{Author Biographies}

\textbf{\uppercase{Hao Feng}} is a PhD Student at Aarhus University specializing in the fundamental techniques that make up Digital Twins, including monitoring. His PhD combines software and robotics.

\textbf{\uppercase{CLÁUDIO GOMES}} is a postdoc researcher at Aarhus University (Denmark). He received his PhD at the University of Antwerp, for his work on the foundations of co-simulation, one of the fundamental techniques enabling the DT. His web address is \url{ http://pure.au.dk/portal/en/claudio.gomes@eng.au.dk}.

% \textbf{\uppercase{Alexandros Iosifidis}} is an associate professor at Aarhus University.  His research interests include topics of neural networks, deep/representation learning and statistical machine learning.

\textbf{\uppercase{Alexandros Iosifidis}} is an associate professor at Aarhus University.  His research interests include topics of neural networks, deep/representation learning and statistical machine learning.


\textbf{\uppercase{Peter Gorm Larsen}} is a professor at Aarhus University.
He has been involved with >40 research projects and he currently leads the AU DIGIT Centre of Digitalisation, Big Data and Data Analytics, the AU Centre for Digital Twins, as well as the research group for Cyber-Physical Systems.
He has published more than 200 articles and three books on modelling of cyber-physical systems and digital twins.

%\author{B. Pal, A. Schwartz}\affiliation{University of Cincinnati, OH-45221, USA}
%\collaboration{The Belle Collaboration}
%\noaffiliation
%%% Paper:    Lambda_c+ -> phi p pi0- and K- pi+ p pi0
%%% Journal:  Physical Review D (Rapid Communication)
%%% Contacts: B. Pal (bkpal.syr@gmail.com)
%%%           A. Schwartz (alan.j.schwartz@uc.edu)
%%% Non-responding authors or those who said NO are commented out.
%%% ====================================================================
%%% Click the RELOAD button on your web browser to see the updated file.
%%% ====================================================================
%%% Use \section*{Author Biographies}

\textbf{\uppercase{Hao Feng}} is a PhD Student at Aarhus University specializing in the fundamental techniques that make up Digital Twins, including monitoring. His PhD combines software and robotics.

\textbf{\uppercase{CLÁUDIO GOMES}} is a postdoc researcher at Aarhus University (Denmark). He received his PhD at the University of Antwerp, for his work on the foundations of co-simulation, one of the fundamental techniques enabling the DT. His web address is \url{ http://pure.au.dk/portal/en/claudio.gomes@eng.au.dk}.

% \textbf{\uppercase{Alexandros Iosifidis}} is an associate professor at Aarhus University.  His research interests include topics of neural networks, deep/representation learning and statistical machine learning.

\textbf{\uppercase{Alexandros Iosifidis}} is an associate professor at Aarhus University.  His research interests include topics of neural networks, deep/representation learning and statistical machine learning.


\textbf{\uppercase{Peter Gorm Larsen}} is a professor at Aarhus University.
He has been involved with >40 research projects and he currently leads the AU DIGIT Centre of Digitalisation, Big Data and Data Analytics, the AU Centre for Digital Twins, as well as the research group for Cyber-Physical Systems.
He has published more than 200 articles and three books on modelling of cyber-physical systems and digital twins.
 to insert this material into your latex file.
%%%%% Force institutions to appear in alphabetical order when typeset.
\noaffiliation
%%%\affiliation{Aligarh Muslim University, Aligarh 202002}
\affiliation{University of the Basque Country UPV/EHU, 48080 Bilbao}
\affiliation{Beihang University, Beijing 100191}
%%%\affiliation{University of Bonn, 53115 Bonn}
\affiliation{Budker Institute of Nuclear Physics SB RAS, Novosibirsk 630090}
\affiliation{Faculty of Mathematics and Physics, Charles University, 121 16 Prague}
%%%\affiliation{Chiba University, Chiba 263-8522}
\affiliation{Chonnam National University, Kwangju 660-701}
\affiliation{University of Cincinnati, Cincinnati, Ohio 45221}
\affiliation{Deutsches Elektronen--Synchrotron, 22607 Hamburg}
\affiliation{University of Florida, Gainesville, Florida 32611}
%%%\affiliation{Department of Physics, Fu Jen Catholic University, Taipei 24205}
%%%\affiliation{Justus-Liebig-Universit\"at Gie\ss{}en, 35392 Gie\ss{}en}
\affiliation{Gifu University, Gifu 501-1193}
%%%\affiliation{II. Physikalisches Institut, Georg-August-Universit\"at G\"ottingen, 37073 G\"ottingen}
\affiliation{SOKENDAI (The Graduate University for Advanced Studies), Hayama 240-0193}
\affiliation{Gyeongsang National University, Chinju 660-701}
\affiliation{Hanyang University, Seoul 133-791}
\affiliation{University of Hawaii, Honolulu, Hawaii 96822}
\affiliation{High Energy Accelerator Research Organization (KEK), Tsukuba 305-0801}
\affiliation{J-PARC Branch, KEK Theory Center, High Energy Accelerator Research Organization (KEK), Tsukuba 305-0801}
%%%\affiliation{Hiroshima Institute of Technology, Hiroshima 731-5193}
\affiliation{IKERBASQUE, Basque Foundation for Science, 48013 Bilbao}
%%%\affiliation{University of Illinois at Urbana-Champaign, Urbana, Illinois 61801}
%%%\affiliation{Indian Institute of Science Education and Research Mohali, SAS Nagar, 140306}
\affiliation{Indian Institute of Technology Bhubaneswar, Satya Nagar 751007}
\affiliation{Indian Institute of Technology Guwahati, Assam 781039}
\affiliation{Indian Institute of Technology Madras, Chennai 600036}
\affiliation{Indiana University, Bloomington, Indiana 47408}
\affiliation{Institute of High Energy Physics, Chinese Academy of Sciences, Beijing 100049}
\affiliation{Institute of High Energy Physics, Vienna 1050}
\affiliation{Institute for High Energy Physics, Protvino 142281}
%%%\affiliation{Institute of Mathematical Sciences, Chennai 600113}
\affiliation{INFN - Sezione di Napoli, 80126 Napoli}
\affiliation{INFN - Sezione di Torino, 10125 Torino}
\affiliation{Advanced Science Research Center, Japan Atomic Energy Agency, Naka 319-1195}
\affiliation{J. Stefan Institute, 1000 Ljubljana}
\affiliation{Kanagawa University, Yokohama 221-8686}
\affiliation{Institut f\"ur Experimentelle Kernphysik, Karlsruher Institut f\"ur Technologie, 76131 Karlsruhe}
%%%\affiliation{Kavli Institute for the Physics and Mathematics of the Universe (WPI), University of Tokyo, Kashiwa 277-8583}
\affiliation{Kennesaw State University, Kennesaw, Georgia 30144}
\affiliation{King Abdulaziz City for Science and Technology, Riyadh 11442}
\affiliation{Department of Physics, Faculty of Science, King Abdulaziz University, Jeddah 21589}
\affiliation{Korea Institute of Science and Technology Information, Daejeon 305-806}
\affiliation{Korea University, Seoul 136-713}
\affiliation{Kyoto University, Kyoto 606-8502}
\affiliation{Kyungpook National University, Daegu 702-701}
\affiliation{\'Ecole Polytechnique F\'ed\'erale de Lausanne (EPFL), Lausanne 1015}
\affiliation{P.N. Lebedev Physical Institute of the Russian Academy of Sciences, Moscow 119991}
%%%\affiliation{Faculty of Mathematics and Physics, University of Ljubljana, 1000 Ljubljana}
\affiliation{Ludwig Maximilians University, 80539 Munich}
\affiliation{Luther College, Decorah, Iowa 52101}
%%%\affiliation{University of Malaya, 50603 Kuala Lumpur}
\affiliation{University of Maribor, 2000 Maribor}
\affiliation{Max-Planck-Institut f\"ur Physik, 80805 M\"unchen}
\affiliation{School of Physics, University of Melbourne, Victoria 3010}
%%%\affiliation{Middle East Technical University, 06531 Ankara}
%%%\affiliation{University of Miyazaki, Miyazaki 889-2192}
\affiliation{Moscow Physical Engineering Institute, Moscow 115409}
\affiliation{Moscow Institute of Physics and Technology, Moscow Region 141700}
\affiliation{Graduate School of Science, Nagoya University, Nagoya 464-8602}
%%%\affiliation{Kobayashi-Maskawa Institute, Nagoya University, Nagoya 464-8602}
%%%\affiliation{Nara University of Education, Nara 630-8528}
\affiliation{Nara Women's University, Nara 630-8506}
\affiliation{National Central University, Chung-li 32054}
\affiliation{National United University, Miao Li 36003}
\affiliation{Department of Physics, National Taiwan University, Taipei 10617}
\affiliation{H. Niewodniczanski Institute of Nuclear Physics, Krakow 31-342}
\affiliation{Nippon Dental University, Niigata 951-8580}
\affiliation{Niigata University, Niigata 950-2181}
%%%\affiliation{University of Nova Gorica, 5000 Nova Gorica}
\affiliation{Novosibirsk State University, Novosibirsk 630090}
\affiliation{Osaka City University, Osaka 558-8585}
%%%\affiliation{Osaka University, Osaka 565-0871}
\affiliation{Pacific Northwest National Laboratory, Richland, Washington 99352}
%%%\affiliation{Panjab University, Chandigarh 160014}
%%%\affiliation{Peking University, Beijing 100871}
\affiliation{University of Pittsburgh, Pittsburgh, Pennsylvania 15260}
%%%\affiliation{Punjab Agricultural University, Ludhiana 141004}
%%%\affiliation{Research Center for Electron Photon Science, Tohoku University, Sendai 980-8578}
%%%\affiliation{Research Center for Nuclear Physics, Osaka University, Osaka 567-0047}
\affiliation{Theoretical Research Division, Nishina Center, RIKEN, Saitama 351-0198}
%%%\affiliation{RIKEN BNL Research Center, Upton, New York 11973}
%%%\affiliation{Saga University, Saga 840-8502}
\affiliation{University of Science and Technology of China, Hefei 230026}
%%%\affiliation{Seoul National University, Seoul 151-742}
%%%\affiliation{Shinshu University, Nagano 390-8621}
\affiliation{Showa Pharmaceutical University, Tokyo 194-8543}
\affiliation{Soongsil University, Seoul 156-743}
%%%\affiliation{University of South Carolina, Columbia, South Carolina 29208}
\affiliation{Stefan Meyer Institute for Subatomic Physics, Vienna 1090}
\affiliation{Sungkyunkwan University, Suwon 440-746}
\affiliation{School of Physics, University of Sydney, New South Wales 2006}
\affiliation{Department of Physics, Faculty of Science, University of Tabuk, Tabuk 71451}
\affiliation{Tata Institute of Fundamental Research, Mumbai 400005}
\affiliation{Excellence Cluster Universe, Technische Universit\"at M\"unchen, 85748 Garching}
\affiliation{Department of Physics, Technische Universit\"at M\"unchen, 85748 Garching}
\affiliation{Toho University, Funabashi 274-8510}
%%%\affiliation{Tohoku Gakuin University, Tagajo 985-8537}
\affiliation{Department of Physics, Tohoku University, Sendai 980-8578}
%%%\affiliation{Earthquake Research Institute, University of Tokyo, Tokyo 113-0032}
\affiliation{Department of Physics, University of Tokyo, Tokyo 113-0033}
\affiliation{Tokyo Institute of Technology, Tokyo 152-8550}
\affiliation{Tokyo Metropolitan University, Tokyo 192-0397}
%%%\affiliation{Tokyo University of Agriculture and Technology, Tokyo 184-8588}
\affiliation{University of Torino, 10124 Torino}
%%%\affiliation{Toyama National College of Maritime Technology, Toyama 933-0293}
%%%\affiliation{Utkal University, Bhubaneswar 751004}
\affiliation{Virginia Polytechnic Institute and State University, Blacksburg, Virginia 24061}
\affiliation{Wayne State University, Detroit, Michigan 48202}
\affiliation{Yamagata University, Yamagata 990-8560}
\affiliation{Yonsei University, Seoul 120-749}
 \author{B.~Pal}\affiliation{University of Cincinnati, Cincinnati, Ohio 45221} % Cincinnati
 \author{A.~J.~Schwartz}\affiliation{University of Cincinnati, Cincinnati, Ohio 45221} % Cincinnati
% \author{A.~Abdesselam}\affiliation{Department of Physics, Faculty of Science, University of Tabuk, Tabuk 71451} % Tabuk
  \author{I.~Adachi}\affiliation{High Energy Accelerator Research Organization (KEK), Tsukuba 305-0801}\affiliation{SOKENDAI (The Graduate University for Advanced Studies), Hayama 240-0193} % KEK
% \author{K.~Adamczyk}\affiliation{H. Niewodniczanski Institute of Nuclear Physics, Krakow 31-342} % Krakow
  \author{H.~Aihara}\affiliation{Department of Physics, University of Tokyo, Tokyo 113-0033} % Tokyo
  \author{S.~Al~Said}\affiliation{Department of Physics, Faculty of Science, University of Tabuk, Tabuk 71451}\affiliation{Department of Physics, Faculty of Science, King Abdulaziz University, Jeddah 21589} % Tabuk
% \author{K.~Arinstein}\affiliation{Budker Institute of Nuclear Physics SB RAS, Novosibirsk 630090}\affiliation{Novosibirsk State University, Novosibirsk 630090} % BINP
% \author{Y.~Arita}\affiliation{Graduate School of Science, Nagoya University, Nagoya 464-8602} % Nagoya
  \author{D.~M.~Asner}\affiliation{Pacific Northwest National Laboratory, Richland, Washington 99352} % PNNL
% \author{T.~Aso}\affiliation{Toyama National College of Maritime Technology, Toyama 933-0293} % Toyama
% \author{H.~Atmacan}\affiliation{University of South Carolina, Columbia, South Carolina 29208} % SouthCarolina
% \author{V.~Aulchenko}\affiliation{Budker Institute of Nuclear Physics SB RAS, Novosibirsk 630090}\affiliation{Novosibirsk State University, Novosibirsk 630090} % BINP
 \author{T.~Aushev}\affiliation{Moscow Institute of Physics and Technology, Moscow Region 141700} % MIPT
 \author{R.~Ayad}\affiliation{Department of Physics, Faculty of Science, University of Tabuk, Tabuk 71451} % Tabuk
% \author{T.~Aziz}\affiliation{Tata Institute of Fundamental Research, Mumbai 400005} % Tata
% \author{V.~Babu}\affiliation{Tata Institute of Fundamental Research, Mumbai 400005} % Tata
  \author{I.~Badhrees}\affiliation{Department of Physics, Faculty of Science, University of Tabuk, Tabuk 71451}\affiliation{King Abdulaziz City for Science and Technology, Riyadh 11442} % Tabuk
% \author{S.~Bahinipati}\affiliation{Indian Institute of Technology Bhubaneswar, Satya Nagar 751007} % IITB
  \author{A.~M.~Bakich}\affiliation{School of Physics, University of Sydney, New South Wales 2006} % Sydney
% \author{A.~Bala}\affiliation{Panjab University, Chandigarh 160014} % Panjab
% \author{Y.~Ban}\affiliation{Peking University, Beijing 100871} % Peking
  \author{V.~Bansal}\affiliation{Pacific Northwest National Laboratory, Richland, Washington 99352} % PNNL
% \author{E.~Barberio}\affiliation{School of Physics, University of Melbourne, Victoria 3010} % Melbourne
% \author{M.~Barrett}\affiliation{University of Hawaii, Honolulu, Hawaii 96822} % Hawaii
% \author{W.~Bartel}\affiliation{Deutsches Elektronen--Synchrotron, 22607 Hamburg} % DESY
% \author{A.~Bay}\affiliation{\'Ecole Polytechnique F\'ed\'erale de Lausanne (EPFL), Lausanne 1015} % Lausanne
  \author{P.~Behera}\affiliation{Indian Institute of Technology Madras, Chennai 600036} % IITM
% \author{M.~Belhorn}\affiliation{University of Cincinnati, Cincinnati, Ohio 45221} % Cincinnati
% \author{K.~Belous}\affiliation{Institute for High Energy Physics, Protvino 142281} % Protvino
  \author{M.~Berger}\affiliation{Stefan Meyer Institute for Subatomic Physics, Vienna 1090} % Vienna
% \author{F.~Bernlochner}\affiliation{University of Bonn, 53115 Bonn} % Bonn
% \author{D.~Besson}\affiliation{Moscow Physical Engineering Institute, Moscow 115409} % MEPhI
 \author{V.~Bhardwaj}\affiliation{Indian Institute of Science Education and Research Mohali, SAS Nagar, 140306} % IISERM
% \author{B.~Bhuyan}\affiliation{Indian Institute of Technology Guwahati, Assam 781039} % IITG
  \author{J.~Biswal}\affiliation{J. Stefan Institute, 1000 Ljubljana} % Ljubljana
% \author{T.~Bloomfield}\affiliation{School of Physics, University of Melbourne, Victoria 3010} % Melbourne
% \author{S.~Blyth}\affiliation{National United University, Miao Li 36003} % NUU
  \author{A.~Bobrov}\affiliation{Budker Institute of Nuclear Physics SB RAS, Novosibirsk 630090}\affiliation{Novosibirsk State University, Novosibirsk 630090} % BINP
% \author{A.~Bondar}\affiliation{Budker Institute of Nuclear Physics SB RAS, Novosibirsk 630090}\affiliation{Novosibirsk State University, Novosibirsk 630090} % BINP
% \author{G.~Bonvicini}\affiliation{Wayne State University, Detroit, Michigan 48202} % WayneState
% \author{C.~Bookwalter}\affiliation{Pacific Northwest National Laboratory, Richland, Washington 99352} % PNNL
% \author{C.~Boulahouache}\affiliation{Department of Physics, Faculty of Science, University of Tabuk, Tabuk 71451} % Tabuk
  \author{A.~Bozek}\affiliation{H. Niewodniczanski Institute of Nuclear Physics, Krakow 31-342} % Krakow
  \author{M.~Bra\v{c}ko}\affiliation{University of Maribor, 2000 Maribor}\affiliation{J. Stefan Institute, 1000 Ljubljana} % Ljubljana
% \author{N.~Braun}\affiliation{Institut f\"ur Experimentelle Kernphysik, Karlsruher Institut f\"ur Technologie, 76131 Karlsruhe} % Karlsruhe
% \author{F.~Breibeck}\affiliation{Institute of High Energy Physics, Vienna 1050} % Vienna
% \author{J.~Brodzicka}\affiliation{H. Niewodniczanski Institute of Nuclear Physics, Krakow 31-342} % Krakow
  \author{T.~E.~Browder}\affiliation{University of Hawaii, Honolulu, Hawaii 96822} % Hawaii
% \author{G.~Caria}\affiliation{School of Physics, University of Melbourne, Victoria 3010} % Melbourne
  \author{D.~\v{C}ervenkov}\affiliation{Faculty of Mathematics and Physics, Charles University, 121 16 Prague} % Charles
% \author{M.-C.~Chang}\affiliation{Department of Physics, Fu Jen Catholic University, Taipei 24205} % FuJen
% \author{P.~Chang}\affiliation{Department of Physics, National Taiwan University, Taipei 10617} % Taiwan
% \author{Y.~Chao}\affiliation{Department of Physics, National Taiwan University, Taipei 10617} % Taiwan
  \author{V.~Chekelian}\affiliation{Max-Planck-Institut f\"ur Physik, 80805 M\"unchen} % MPI
  \author{A.~Chen}\affiliation{National Central University, Chung-li 32054} % NCU
% \author{K.-F.~Chen}\affiliation{Department of Physics, National Taiwan University, Taipei 10617} % Taiwan
% \author{P.~Chen}\affiliation{Department of Physics, National Taiwan University, Taipei 10617} % Taiwan
  \author{B.~G.~Cheon}\affiliation{Hanyang University, Seoul 133-791} % Hanyang
  \author{K.~Chilikin}\affiliation{P.N. Lebedev Physical Institute of the Russian Academy of Sciences, Moscow 119991}\affiliation{Moscow Physical Engineering Institute, Moscow 115409} % Lebedev
% \author{R.~Chistov}\affiliation{P.N. Lebedev Physical Institute of the Russian Academy of Sciences, Moscow 119991}\affiliation{Moscow Physical Engineering Institute, Moscow 115409} % Lebedev
  \author{K.~Cho}\affiliation{Korea Institute of Science and Technology Information, Daejeon 305-806} % KISTI
% \author{V.~Chobanova}\affiliation{Max-Planck-Institut f\"ur Physik, 80805 M\"unchen} % MPI
  \author{S.-K.~Choi}\affiliation{Gyeongsang National University, Chinju 660-701} % Gyeongsang
  \author{Y.~Choi}\affiliation{Sungkyunkwan University, Suwon 440-746} % Sungkyunkwan
  \author{D.~Cinabro}\affiliation{Wayne State University, Detroit, Michigan 48202} % WayneState
% \author{J.~Crnkovic}\affiliation{University of Illinois at Urbana-Champaign, Urbana, Illinois 61801} % UIUC
% \author{J.~Dalseno}\affiliation{Max-Planck-Institut f\"ur Physik, 80805 M\"unchen}\affiliation{Excellence Cluster Universe, Technische Universit\"at M\"unchen, 85748 Garching} % MPI
% \author{M.~Danilov}\affiliation{Moscow Physical Engineering Institute, Moscow 115409}\affiliation{P.N. Lebedev Physical Institute of the Russian Academy of Sciences, Moscow 119991} % Lebedev
  \author{N.~Dash}\affiliation{Indian Institute of Technology Bhubaneswar, Satya Nagar 751007} % IITB
  \author{S.~Di~Carlo}\affiliation{Wayne State University, Detroit, Michigan 48202} % WayneState
% \author{J.~Dingfelder}\affiliation{University of Bonn, 53115 Bonn} % Bonn
  \author{Z.~Dole\v{z}al}\affiliation{Faculty of Mathematics and Physics, Charles University, 121 16 Prague} % Charles
% \author{D.~Dossett}\affiliation{School of Physics, University of Melbourne, Victoria 3010} % Melbourne
  \author{Z.~Dr\'asal}\affiliation{Faculty of Mathematics and Physics, Charles University, 121 16 Prague} % Charles
% \author{A.~Drutskoy}\affiliation{P.N. Lebedev Physical Institute of the Russian Academy of Sciences, Moscow 119991}\affiliation{Moscow Physical Engineering Institute, Moscow 115409} % Lebedev
% \author{S.~Dubey}\affiliation{University of Hawaii, Honolulu, Hawaii 96822} % Hawaii
% \author{D.~Dutta}\affiliation{Tata Institute of Fundamental Research, Mumbai 400005} % Tata
% \author{K.~Dutta}\affiliation{Indian Institute of Technology Guwahati, Assam 781039} % IITG
  \author{S.~Eidelman}\affiliation{Budker Institute of Nuclear Physics SB RAS, Novosibirsk 630090}\affiliation{Novosibirsk State University, Novosibirsk 630090} % BINP
% \author{D.~Epifanov}\affiliation{Budker Institute of Nuclear Physics SB RAS, Novosibirsk 630090}\affiliation{Novosibirsk State University, Novosibirsk 630090} % BINP
% \author{H.~Farhat}\affiliation{Wayne State University, Detroit, Michigan 48202} % WayneState
% \author{J.~E.~Fast}\affiliation{Pacific Northwest National Laboratory, Richland, Washington 99352} % PNNL
% \author{M.~Feindt}\affiliation{Institut f\"ur Experimentelle Kernphysik, Karlsruher Institut f\"ur Technologie, 76131 Karlsruhe} % Karlsruhe
  \author{T.~Ferber}\affiliation{Deutsches Elektronen--Synchrotron, 22607 Hamburg} % DESY
% \author{A.~Frey}\affiliation{II. Physikalisches Institut, Georg-August-Universit\"at G\"ottingen, 37073 G\"ottingen} % Goettingen
% \author{O.~Frost}\affiliation{Deutsches Elektronen--Synchrotron, 22607 Hamburg} % DESY
  \author{B.~G.~Fulsom}\affiliation{Pacific Northwest National Laboratory, Richland, Washington 99352} % PNNL
  \author{V.~Gaur}\affiliation{Virginia Polytechnic Institute and State University, Blacksburg, Virginia 24061} % VPI
  \author{N.~Gabyshev}\affiliation{Budker Institute of Nuclear Physics SB RAS, Novosibirsk 630090}\affiliation{Novosibirsk State University, Novosibirsk 630090} % BINP
% \author{S.~Ganguly}\affiliation{Wayne State University, Detroit, Michigan 48202} % WayneState
  \author{A.~Garmash}\affiliation{Budker Institute of Nuclear Physics SB RAS, Novosibirsk 630090}\affiliation{Novosibirsk State University, Novosibirsk 630090} % BINP
% \author{M.~Gelb}\affiliation{Institut f\"ur Experimentelle Kernphysik, Karlsruher Institut f\"ur Technologie, 76131 Karlsruhe} % Karlsruhe
% \author{J.~Gemmler}\affiliation{Institut f\"ur Experimentelle Kernphysik, Karlsruher Institut f\"ur Technologie, 76131 Karlsruhe} % Karlsruhe
% \author{D.~Getzkow}\affiliation{Justus-Liebig-Universit\"at Gie\ss{}en, 35392 Gie\ss{}en} % Giessen
% \author{R.~Gillard}\affiliation{Wayne State University, Detroit, Michigan 48202} % WayneState
% \author{F.~Giordano}\affiliation{University of Illinois at Urbana-Champaign, Urbana, Illinois 61801} % UIUC
% \author{R.~Glattauer}\affiliation{Institute of High Energy Physics, Vienna 1050} % Vienna
% \author{Y.~M.~Goh}\affiliation{Hanyang University, Seoul 133-791} % Hanyang
  \author{P.~Goldenzweig}\affiliation{Institut f\"ur Experimentelle Kernphysik, Karlsruher Institut f\"ur Technologie, 76131 Karlsruhe} % Karlsruhe
% \author{B.~Golob}\affiliation{Faculty of Mathematics and Physics, University of Ljubljana, 1000 Ljubljana}\affiliation{J. Stefan Institute, 1000 Ljubljana} % Ljubljana
% \author{D.~Greenwald}\affiliation{Department of Physics, Technische Universit\"at M\"unchen, 85748 Garching} % TUM
% \author{M.~Grosse~Perdekamp}\affiliation{University of Illinois at Urbana-Champaign, Urbana, Illinois 61801}\affiliation{RIKEN BNL Research Center, Upton, New York 11973} % UIUC
% \author{J.~Grygier}\affiliation{Institut f\"ur Experimentelle Kernphysik, Karlsruher Institut f\"ur Technologie, 76131 Karlsruhe} % Karlsruhe
% \author{O.~Grzymkowska}\affiliation{H. Niewodniczanski Institute of Nuclear Physics, Krakow 31-342} % Krakow
% \author{Y.~Guan}\affiliation{Indiana University, Bloomington, Indiana 47408}\affiliation{High Energy Accelerator Research Organization (KEK), Tsukuba 305-0801} % Indiana
  \author{E.~Guido}\affiliation{INFN - Sezione di Torino, 10125 Torino} % Torino
% \author{H.~Guo}\affiliation{University of Science and Technology of China, Hefei 230026} % USTC
% \author{J.~Haba}\affiliation{High Energy Accelerator Research Organization (KEK), Tsukuba 305-0801}\affiliation{SOKENDAI (The Graduate University for Advanced Studies), Hayama 240-0193} % KEK
% \author{P.~Hamer}\affiliation{II. Physikalisches Institut, Georg-August-Universit\"at G\"ottingen, 37073 G\"ottingen} % Goettingen
% \author{Y.~L.~Han}\affiliation{Institute of High Energy Physics, Chinese Academy of Sciences, Beijing 100049} % IHEP
% \author{K.~Hara}\affiliation{High Energy Accelerator Research Organization (KEK), Tsukuba 305-0801} % KEK
  \author{T.~Hara}\affiliation{High Energy Accelerator Research Organization (KEK), Tsukuba 305-0801}\affiliation{SOKENDAI (The Graduate University for Advanced Studies), Hayama 240-0193} % KEK
% \author{Y.~Hasegawa}\affiliation{Shinshu University, Nagano 390-8621} % Shinshu
% \author{J.~Hasenbusch}\affiliation{University of Bonn, 53115 Bonn} % Bonn
  \author{K.~Hayasaka}\affiliation{Niigata University, Niigata 950-2181} % Niigata
  \author{H.~Hayashii}\affiliation{Nara Women's University, Nara 630-8506} % Nara
% \author{X.~H.~He}\affiliation{Peking University, Beijing 100871} % Peking
% \author{M.~Heck}\affiliation{Institut f\"ur Experimentelle Kernphysik, Karlsruher Institut f\"ur Technologie, 76131 Karlsruhe} % Karlsruhe
% \author{M.~T.~Hedges}\affiliation{University of Hawaii, Honolulu, Hawaii 96822} % Hawaii
% \author{D.~Heffernan}\affiliation{Osaka University, Osaka 565-0871} % Osaka
% \author{M.~Heider}\affiliation{Institut f\"ur Experimentelle Kernphysik, Karlsruher Institut f\"ur Technologie, 76131 Karlsruhe} % Karlsruhe
% \author{A.~Heller}\affiliation{Institut f\"ur Experimentelle Kernphysik, Karlsruher Institut f\"ur Technologie, 76131 Karlsruhe} % Karlsruhe
% \author{T.~Higuchi}\affiliation{Kavli Institute for the Physics and Mathematics of the Universe (WPI), University of Tokyo, Kashiwa 277-8583} % IPMU
% \author{S.~Himori}\affiliation{Department of Physics, Tohoku University, Sendai 980-8578} % Tohoku
% \author{S.~Hirose}\affiliation{Graduate School of Science, Nagoya University, Nagoya 464-8602} % Nagoya
% \author{T.~Horiguchi}\affiliation{Department of Physics, Tohoku University, Sendai 980-8578} % Tohoku
% \author{Y.~Hoshi}\affiliation{Tohoku Gakuin University, Tagajo 985-8537} % TohokuGakuin
% \author{K.~Hoshina}\affiliation{Tokyo University of Agriculture and Technology, Tokyo 184-8588} % TUAT
  \author{W.-S.~Hou}\affiliation{Department of Physics, National Taiwan University, Taipei 10617} % Taiwan
% \author{Y.~B.~Hsiung}\affiliation{Department of Physics, National Taiwan University, Taipei 10617} % Taiwan
  \author{C.-L.~Hsu}\affiliation{School of Physics, University of Melbourne, Victoria 3010} % Melbourne
% \author{M.~Huschle}\affiliation{Institut f\"ur Experimentelle Kernphysik, Karlsruher Institut f\"ur Technologie, 76131 Karlsruhe} % Karlsruhe
% \author{H.~J.~Hyun}\affiliation{Kyungpook National University, Daegu 702-701} % Kyungpook
% \author{Y.~Igarashi}\affiliation{High Energy Accelerator Research Organization (KEK), Tsukuba 305-0801} % KEK
% \author{T.~Iijima}\affiliation{Kobayashi-Maskawa Institute, Nagoya University, Nagoya 464-8602}\affiliation{Graduate School of Science, Nagoya University, Nagoya 464-8602} % Nagoya
% \author{M.~Imamura}\affiliation{Graduate School of Science, Nagoya University, Nagoya 464-8602} % Nagoya
  \author{K.~Inami}\affiliation{Graduate School of Science, Nagoya University, Nagoya 464-8602} % Nagoya
% \author{G.~Inguglia}\affiliation{Deutsches Elektronen--Synchrotron, 22607 Hamburg} % DESY
  \author{A.~Ishikawa}\affiliation{Department of Physics, Tohoku University, Sendai 980-8578} % Tohoku
% \author{K.~Itagaki}\affiliation{Department of Physics, Tohoku University, Sendai 980-8578} % Tohoku
  \author{R.~Itoh}\affiliation{High Energy Accelerator Research Organization (KEK), Tsukuba 305-0801}\affiliation{SOKENDAI (The Graduate University for Advanced Studies), Hayama 240-0193} % KEK
% \author{M.~Iwabuchi}\affiliation{Yonsei University, Seoul 120-749} % Yonsei
% \author{M.~Iwasaki}\affiliation{Department of Physics, University of Tokyo, Tokyo 113-0033} % Tokyo
% \author{Y.~Iwasaki}\affiliation{High Energy Accelerator Research Organization (KEK), Tsukuba 305-0801} % KEK
% \author{S.~Iwata}\affiliation{Tokyo Metropolitan University, Tokyo 192-0397} % TMU
  \author{W.~W.~Jacobs}\affiliation{Indiana University, Bloomington, Indiana 47408} % Indiana
  \author{I.~Jaegle}\affiliation{University of Florida, Gainesville, Florida 32611} % Florida
  \author{H.~B.~Jeon}\affiliation{Kyungpook National University, Daegu 702-701} % Kyungpook
  \author{S.~Jia}\affiliation{Beihang University, Beijing 100191} % Beihang
  \author{Y.~Jin}\affiliation{Department of Physics, University of Tokyo, Tokyo 113-0033} % Tokyo
  \author{D.~Joffe}\affiliation{Kennesaw State University, Kennesaw, Georgia 30144} % Kennesaw
% \author{M.~Jones}\affiliation{University of Hawaii, Honolulu, Hawaii 96822} % Hawaii
  \author{K.~K.~Joo}\affiliation{Chonnam National University, Kwangju 660-701} % Chonnam
% \author{T.~Julius}\affiliation{School of Physics, University of Melbourne, Victoria 3010} % Melbourne
% \author{J.~Kahn}\affiliation{Ludwig Maximilians University, 80539 Munich} % LMU
% \author{H.~Kakuno}\affiliation{Tokyo Metropolitan University, Tokyo 192-0397} % TMU
% \author{A.~B.~Kaliyar}\affiliation{Indian Institute of Technology Madras, Chennai 600036} % IITM
% \author{J.~H.~Kang}\affiliation{Yonsei University, Seoul 120-749} % Yonsei
% \author{K.~H.~Kang}\affiliation{Kyungpook National University, Daegu 702-701} % Kyungpook
% \author{P.~Kapusta}\affiliation{H. Niewodniczanski Institute of Nuclear Physics, Krakow 31-342} % Krakow
  \author{G.~Karyan}\affiliation{Deutsches Elektronen--Synchrotron, 22607 Hamburg} % DESY
% \author{S.~U.~Kataoka}\affiliation{Nara University of Education, Nara 630-8528} % NUE
% \author{E.~Kato}\affiliation{Department of Physics, Tohoku University, Sendai 980-8578} % Tohoku
  \author{Y.~Kato}\affiliation{Graduate School of Science, Nagoya University, Nagoya 464-8602} % Nagoya
% \author{P.~Katrenko}\affiliation{Moscow Institute of Physics and Technology, Moscow Region 141700}\affiliation{P.N. Lebedev Physical Institute of the Russian Academy of Sciences, Moscow 119991} % Lebedev
% \author{H.~Kawai}\affiliation{Chiba University, Chiba 263-8522} % Chiba
% \author{T.~Kawasaki}\affiliation{Niigata University, Niigata 950-2181} % Niigata
% \author{T.~Keck}\affiliation{Institut f\"ur Experimentelle Kernphysik, Karlsruher Institut f\"ur Technologie, 76131 Karlsruhe} % Karlsruhe
% \author{H.~Kichimi}\affiliation{High Energy Accelerator Research Organization (KEK), Tsukuba 305-0801} % KEK
% \author{C.~Kiesling}\affiliation{Max-Planck-Institut f\"ur Physik, 80805 M\"unchen} % MPI
% \author{B.~H.~Kim}\affiliation{Seoul National University, Seoul 151-742} % Seoul
  \author{D.~Y.~Kim}\affiliation{Soongsil University, Seoul 156-743} % Soongsil
% \author{H.~J.~Kim}\affiliation{Kyungpook National University, Daegu 702-701} % Kyungpook
% \author{H.-J.~Kim}\affiliation{Yonsei University, Seoul 120-749} % Yonsei
  \author{J.~B.~Kim}\affiliation{Korea University, Seoul 136-713} % Korea
% \author{J.~H.~Kim}\affiliation{Korea Institute of Science and Technology Information, Daejeon 305-806} % KISTI
  \author{K.~T.~Kim}\affiliation{Korea University, Seoul 136-713} % Korea
% \author{M.~J.~Kim}\affiliation{Kyungpook National University, Daegu 702-701} % Kyungpook
  \author{S.~H.~Kim}\affiliation{Hanyang University, Seoul 133-791} % Hanyang
% \author{S.~K.~Kim}\affiliation{Seoul National University, Seoul 151-742} % Seoul
  \author{Y.~J.~Kim}\affiliation{Korea Institute of Science and Technology Information, Daejeon 305-806} % KISTI
  \author{K.~Kinoshita}\affiliation{University of Cincinnati, Cincinnati, Ohio 45221} % Cincinnati
% \author{C.~Kleinwort}\affiliation{Deutsches Elektronen--Synchrotron, 22607 Hamburg} % DESY
% \author{J.~Klucar}\affiliation{J. Stefan Institute, 1000 Ljubljana} % Ljubljana
% \author{B.~R.~Ko}\affiliation{Korea University, Seoul 136-713} % Korea
% \author{N.~Kobayashi}\affiliation{Tokyo Institute of Technology, Tokyo 152-8550} % NPC
% \author{S.~Koblitz}\affiliation{Max-Planck-Institut f\"ur Physik, 80805 M\"unchen} % MPI 
  \author{P.~Kody\v{s}}\affiliation{Faculty of Mathematics and Physics, Charles University, 121 16 Prague} % Charles
% \author{Y.~Koga}\affiliation{Graduate School of Science, Nagoya University, Nagoya 464-8602} % Nagoya
  \author{S.~Korpar}\affiliation{University of Maribor, 2000 Maribor}\affiliation{J. Stefan Institute, 1000 Ljubljana} % Ljubljana
  \author{D.~Kotchetkov}\affiliation{University of Hawaii, Honolulu, Hawaii 96822} % Hawaii
% \author{R.~T.~Kouzes}\affiliation{Pacific Northwest National Laboratory, Richland, Washington 99352} % PNNL
% \author{P.~Kri\v{z}an}\affiliation{Faculty of Mathematics and Physics, University of Ljubljana, 1000 Ljubljana}\affiliation{J. Stefan Institute, 1000 Ljubljana} % Ljubljana
  \author{P.~Krokovny}\affiliation{Budker Institute of Nuclear Physics SB RAS, Novosibirsk 630090}\affiliation{Novosibirsk State University, Novosibirsk 630090} % BINP
% \author{B.~Kronenbitter}\affiliation{Institut f\"ur Experimentelle Kernphysik, Karlsruher Institut f\"ur Technologie, 76131 Karlsruhe} % Karlsruhe
  \author{T.~Kuhr}\affiliation{Ludwig Maximilians University, 80539 Munich} % LMU
  \author{R.~Kulasiri}\affiliation{Kennesaw State University, Kennesaw, Georgia 30144} % Kennesaw
% \author{R.~Kumar}\affiliation{Punjab Agricultural University, Ludhiana 141004} % Punjab
% \author{T.~Kumita}\affiliation{Tokyo Metropolitan University, Tokyo 192-0397} % TMU
% \author{E.~Kurihara}\affiliation{Chiba University, Chiba 263-8522} % Chiba
% \author{Y.~Kuroki}\affiliation{Osaka University, Osaka 565-0871} % Osaka
% \author{A.~Kuzmin}\affiliation{Budker Institute of Nuclear Physics SB RAS, Novosibirsk 630090}\affiliation{Novosibirsk State University, Novosibirsk 630090} % BINP
% \author{P.~Kvasni\v{c}ka}\affiliation{Faculty of Mathematics and Physics, Charles University, 121 16 Prague} % Charles
  \author{Y.-J.~Kwon}\affiliation{Yonsei University, Seoul 120-749} % Yonsei
% \author{Y.-T.~Lai}\affiliation{Department of Physics, National Taiwan University, Taipei 10617} % Taiwan
% \author{J.~S.~Lange}\affiliation{Justus-Liebig-Universit\"at Gie\ss{}en, 35392 Gie\ss{}en} % Giessen
% \author{D.~H.~Lee}\affiliation{Korea University, Seoul 136-713} % Korea
% \author{I.~S.~Lee}\affiliation{Hanyang University, Seoul 133-791} % Hanyang
% \author{S.-H.~Lee}\affiliation{Korea University, Seoul 136-713} % Korea
% \author{M.~Leitgab}\affiliation{University of Illinois at Urbana-Champaign, Urbana, Illinois 61801}\affiliation{RIKEN BNL Research Center, Upton, New York 11973} % UIUC
% \author{R.~Leitner}\affiliation{Faculty of Mathematics and Physics, Charles University, 121 16 Prague} % Charles
% \author{D.~Levit}\affiliation{Department of Physics, Technische Universit\"at M\"unchen, 85748 Garching} % TUM
% \author{P.~Lewis}\affiliation{University of Hawaii, Honolulu, Hawaii 96822} % Hawaii
  \author{C.~H.~Li}\affiliation{School of Physics, University of Melbourne, Victoria 3010} % Melbourne
% \author{H.~Li}\affiliation{Indiana University, Bloomington, Indiana 47408} % Indiana
% \author{J.~Li}\affiliation{Seoul National University, Seoul 151-742} % Seoul
  \author{L.~Li}\affiliation{University of Science and Technology of China, Hefei 230026} % USTC
% \author{X.~Li}\affiliation{Seoul National University, Seoul 151-742} % Seoul
  \author{Y.~Li}\affiliation{Virginia Polytechnic Institute and State University, Blacksburg, Virginia 24061} % VPI
  \author{L.~Li~Gioi}\affiliation{Max-Planck-Institut f\"ur Physik, 80805 M\"unchen} % MPI
  \author{J.~Libby}\affiliation{Indian Institute of Technology Madras, Chennai 600036} % IITM
% \author{A.~Limosani}\affiliation{School of Physics, University of Melbourne, Victoria 3010} % Melbourne
% \author{C.~Liu}\affiliation{University of Science and Technology of China, Hefei 230026} % USTC
% \author{Y.~Liu}\affiliation{University of Cincinnati, Cincinnati, Ohio 45221} % Cincinnati
% \author{Z.~Q.~Liu}\affiliation{Institute of High Energy Physics, Chinese Academy of Sciences, Beijing 100049} % IHEP
  \author{D.~Liventsev}\affiliation{Virginia Polytechnic Institute and State University, Blacksburg, Virginia 24061}\affiliation{High Energy Accelerator Research Organization (KEK), Tsukuba 305-0801} % VPI
% \author{A.~Loos}\affiliation{University of South Carolina, Columbia, South Carolina 29208} % SouthCarolina
% \author{R.~Louvot}\affiliation{\'Ecole Polytechnique F\'ed\'erale de Lausanne (EPFL), Lausanne 1015} % Lausanne
  \author{M.~Lubej}\affiliation{J. Stefan Institute, 1000 Ljubljana} % Ljubljana
% \author{P.~Lukin}\affiliation{Budker Institute of Nuclear Physics SB RAS, Novosibirsk 630090}\affiliation{Novosibirsk State University, Novosibirsk 630090} % BINP
  \author{T.~Luo}\affiliation{University of Pittsburgh, Pittsburgh, Pennsylvania 15260} % Pittsburgh
  \author{J.~MacNaughton}\affiliation{High Energy Accelerator Research Organization (KEK), Tsukuba 305-0801} % KEK
% \author{C.~MacQueen}\affiliation{School of Physics, University of Melbourne, Victoria 3010} % Melbourne
% \author{M.~Masuda}\affiliation{Earthquake Research Institute, University of Tokyo, Tokyo 113-0032} % NPC
% \author{T.~Matsuda}\affiliation{University of Miyazaki, Miyazaki 889-2192} % NPC
  \author{D.~Matvienko}\affiliation{Budker Institute of Nuclear Physics SB RAS, Novosibirsk 630090}\affiliation{Novosibirsk State University, Novosibirsk 630090} % BINP
% \author{A.~Matyja}\affiliation{H. Niewodniczanski Institute of Nuclear Physics, Krakow 31-342} % Krakow
% \author{S.~McOnie}\affiliation{School of Physics, University of Sydney, New South Wales 2006} % Sydney
  \author{M.~Merola}\affiliation{INFN - Sezione di Napoli, 80126 Napoli} % Napoli
% \author{F.~Metzner}\affiliation{Institut f\"ur Experimentelle Kernphysik, Karlsruher Institut f\"ur Technologie, 76131 Karlsruhe} % Karlsruhe
% \author{Y.~Mikami}\affiliation{Department of Physics, Tohoku University, Sendai 980-8578} % Tohoku
  \author{K.~Miyabayashi}\affiliation{Nara Women's University, Nara 630-8506} % Nara
% \author{Y.~Miyachi}\affiliation{Yamagata University, Yamagata 990-8560} % NPC
% \author{H.~Miyake}\affiliation{High Energy Accelerator Research Organization (KEK), Tsukuba 305-0801}\affiliation{SOKENDAI (The Graduate University for Advanced Studies), Hayama 240-0193} % KEK
  \author{H.~Miyata}\affiliation{Niigata University, Niigata 950-2181} % Niigata
% \author{Y.~Miyazaki}\affiliation{Graduate School of Science, Nagoya University, Nagoya 464-8602} % Nagoya
  \author{R.~Mizuk}\affiliation{P.N. Lebedev Physical Institute of the Russian Academy of Sciences, Moscow 119991}\affiliation{Moscow Physical Engineering Institute, Moscow 115409}\affiliation{Moscow Institute of Physics and Technology, Moscow Region 141700} % Lebedev
  \author{G.~B.~Mohanty}\affiliation{Tata Institute of Fundamental Research, Mumbai 400005} % Tata
% \author{S.~Mohanty}\affiliation{Tata Institute of Fundamental Research, Mumbai 400005}\affiliation{Utkal University, Bhubaneswar 751004} % Tata
% \author{D.~Mohapatra}\affiliation{Pacific Northwest National Laboratory, Richland, Washington 99352} % PNNL
% \author{A.~Moll}\affiliation{Max-Planck-Institut f\"ur Physik, 80805 M\"unchen}\affiliation{Excellence Cluster Universe, Technische Universit\"at M\"unchen, 85748 Garching} % MPI
  \author{H.~K.~Moon}\affiliation{Korea University, Seoul 136-713} % Korea
  \author{T.~Mori}\affiliation{Graduate School of Science, Nagoya University, Nagoya 464-8602} % Nagoya
% \author{T.~Morii}\affiliation{Kavli Institute for the Physics and Mathematics of the Universe (WPI), University of Tokyo, Kashiwa 277-8583} % IPMU
% \author{H.-G.~Moser}\affiliation{Max-Planck-Institut f\"ur Physik, 80805 M\"unchen} % MPI
% \author{M.~Mrvar}\affiliation{J. Stefan Institute, 1000 Ljubljana} % Ljubljana
% \author{T.~M\"uller}\affiliation{Institut f\"ur Experimentelle Kernphysik, Karlsruher Institut f\"ur Technologie, 76131 Karlsruhe} % Karlsruhe
% \author{N.~Muramatsu}\affiliation{Research Center for Electron Photon Science, Tohoku University, Sendai 980-8578} % NPC
  \author{R.~Mussa}\affiliation{INFN - Sezione di Torino, 10125 Torino} % Torino
% \author{T.~Nagamine}\affiliation{Department of Physics, Tohoku University, Sendai 980-8578} % Tohoku
% \author{Y.~Nagasaka}\affiliation{Hiroshima Institute of Technology, Hiroshima 731-5193} % Hiroshima
% \author{Y.~Nakahama}\affiliation{Department of Physics, University of Tokyo, Tokyo 113-0033} % Tokyo
% \author{I.~Nakamura}\affiliation{High Energy Accelerator Research Organization (KEK), Tsukuba 305-0801}\affiliation{SOKENDAI (The Graduate University for Advanced Studies), Hayama 240-0193} % KEK
% \author{K.~R.~Nakamura}\affiliation{High Energy Accelerator Research Organization (KEK), Tsukuba 305-0801} % KEK
  \author{E.~Nakano}\affiliation{Osaka City University, Osaka 558-8585} % OsakaCity
% \author{H.~Nakano}\affiliation{Department of Physics, Tohoku University, Sendai 980-8578} % Tohoku
% \author{T.~Nakano}\affiliation{Research Center for Nuclear Physics, Osaka University, Osaka 567-0047} % NPC
  \author{M.~Nakao}\affiliation{High Energy Accelerator Research Organization (KEK), Tsukuba 305-0801}\affiliation{SOKENDAI (The Graduate University for Advanced Studies), Hayama 240-0193} % KEK
% \author{H.~Nakayama}\affiliation{High Energy Accelerator Research Organization (KEK), Tsukuba 305-0801}\affiliation{SOKENDAI (The Graduate University for Advanced Studies), Hayama 240-0193} % KEK
% \author{H.~Nakazawa}\affiliation{National Central University, Chung-li 32054} % NCU
  \author{T.~Nanut}\affiliation{J. Stefan Institute, 1000 Ljubljana} % Ljubljana
  \author{K.~J.~Nath}\affiliation{Indian Institute of Technology Guwahati, Assam 781039} % IITG
% \author{Z.~Natkaniec}\affiliation{H. Niewodniczanski Institute of Nuclear Physics, Krakow 31-342} % Krakow
  \author{M.~Nayak}\affiliation{Wayne State University, Detroit, Michigan 48202}\affiliation{High Energy Accelerator Research Organization (KEK), Tsukuba 305-0801} % WayneState
% \author{E.~Nedelkovska}\affiliation{Max-Planck-Institut f\"ur Physik, 80805 M\"unchen} % MPI 
% \author{K.~Negishi}\affiliation{Department of Physics, Tohoku University, Sendai 980-8578} % Tohoku
% \author{K.~Neichi}\affiliation{Tohoku Gakuin University, Tagajo 985-8537} % TohokuGakuin
% \author{C.~Ng}\affiliation{Department of Physics, University of Tokyo, Tokyo 113-0033} % Tokyo
% \author{C.~Niebuhr}\affiliation{Deutsches Elektronen--Synchrotron, 22607 Hamburg} % DESY
  \author{M.~Niiyama}\affiliation{Kyoto University, Kyoto 606-8502} % NPC
% \author{N.~K.~Nisar}\affiliation{University of Pittsburgh, Pittsburgh, Pennsylvania 15260} % Pittsburgh
 \author{S.~Nishida}\affiliation{High Energy Accelerator Research Organization (KEK), Tsukuba 305-0801}\affiliation{SOKENDAI (The Graduate University for Advanced Studies), Hayama 240-0193} % KEK
% \author{K.~Nishimura}\affiliation{University of Hawaii, Honolulu, Hawaii 96822} % Hawaii
% \author{O.~Nitoh}\affiliation{Tokyo University of Agriculture and Technology, Tokyo 184-8588} % TUAT
% \author{T.~Nozaki}\affiliation{High Energy Accelerator Research Organization (KEK), Tsukuba 305-0801} % KEK
% \author{A.~Ogawa}\affiliation{RIKEN BNL Research Center, Upton, New York 11973} % RIKEN
  \author{S.~Ogawa}\affiliation{Toho University, Funabashi 274-8510} % Toho
% \author{T.~Ohshima}\affiliation{Graduate School of Science, Nagoya University, Nagoya 464-8602} % Nagoya
% \author{S.~Okuno}\affiliation{Kanagawa University, Yokohama 221-8686} % Kanagawa
% \author{S.~L.~Olsen}\affiliation{Seoul National University, Seoul 151-742} % Seoul
  \author{H.~Ono}\affiliation{Nippon Dental University, Niigata 951-8580}\affiliation{Niigata University, Niigata 950-2181} % NihonDental
% \author{Y.~Ono}\affiliation{Department of Physics, Tohoku University, Sendai 980-8578} % Tohoku
% \author{Y.~Onuki}\affiliation{Department of Physics, University of Tokyo, Tokyo 113-0033} % Tokyo
% \author{W.~Ostrowicz}\affiliation{H. Niewodniczanski Institute of Nuclear Physics, Krakow 31-342} % Krakow
% \author{C.~Oswald}\affiliation{University of Bonn, 53115 Bonn} % Bonn
% \author{H.~Ozaki}\affiliation{High Energy Accelerator Research Organization (KEK), Tsukuba 305-0801}\affiliation{SOKENDAI (The Graduate University for Advanced Studies), Hayama 240-0193} % KEK
  \author{P.~Pakhlov}\affiliation{P.N. Lebedev Physical Institute of the Russian Academy of Sciences, Moscow 119991}\affiliation{Moscow Physical Engineering Institute, Moscow 115409} % Lebedev
  \author{G.~Pakhlova}\affiliation{P.N. Lebedev Physical Institute of the Russian Academy of Sciences, Moscow 119991}\affiliation{Moscow Institute of Physics and Technology, Moscow Region 141700} % Lebedev
 % \author{B.~Pal}\affiliation{University of Cincinnati, Cincinnati, Ohio 45221} % Cincinnati
% \author{H.~Palka}\affiliation{H. Niewodniczanski Institute of Nuclear Physics, Krakow 31-342} % Krakow
% \author{E.~Panzenb\"ock}\affiliation{II. Physikalisches Institut, Georg-August-Universit\"at G\"ottingen, 37073 G\"ottingen}\affiliation{Nara Women's University, Nara 630-8506} % Goettingen
  \author{S.~Pardi}\affiliation{INFN - Sezione di Napoli, 80126 Napoli} % Napoli
  \author{C.-S.~Park}\affiliation{Yonsei University, Seoul 120-749} % Yonsei
% \author{C.~W.~Park}\affiliation{Sungkyunkwan University, Suwon 440-746} % Sungkyunkwan
% \author{H.~Park}\affiliation{Kyungpook National University, Daegu 702-701} % Kyungpook
% \author{K.~S.~Park}\affiliation{Sungkyunkwan University, Suwon 440-746} % Sungkyunkwan
  \author{S.~Paul}\affiliation{Department of Physics, Technische Universit\"at M\"unchen, 85748 Garching} % TUM
% \author{L.~S.~Peak}\affiliation{School of Physics, University of Sydney, New South Wales 2006} % Sydney
  \author{T.~K.~Pedlar}\affiliation{Luther College, Decorah, Iowa 52101} % Luther
% \author{T.~Peng}\affiliation{University of Science and Technology of China, Hefei 230026} % USTC
% \author{L.~Pes\'{a}ntez}\affiliation{University of Bonn, 53115 Bonn} % Bonn
  \author{R.~Pestotnik}\affiliation{J. Stefan Institute, 1000 Ljubljana} % Ljubljana
% \author{M.~Peters}\affiliation{University of Hawaii, Honolulu, Hawaii 96822} % Hawaii
% \author{M.~Petri\v{c}}\affiliation{J. Stefan Institute, 1000 Ljubljana} % Ljubljana
  \author{L.~E.~Piilonen}\affiliation{Virginia Polytechnic Institute and State University, Blacksburg, Virginia 24061} % VPI
% \author{A.~Poluektov}\affiliation{Budker Institute of Nuclear Physics SB RAS, Novosibirsk 630090}\affiliation{Novosibirsk State University, Novosibirsk 630090} % BINP
% \author{K.~Prasanth}\affiliation{Indian Institute of Technology Madras, Chennai 600036} % IITM
% \author{M.~Prim}\affiliation{Institut f\"ur Experimentelle Kernphysik, Karlsruher Institut f\"ur Technologie, 76131 Karlsruhe} % Karlsruhe
% \author{K.~Prothmann}\affiliation{Max-Planck-Institut f\"ur Physik, 80805 M\"unchen}\affiliation{Excellence Cluster Universe, Technische Universit\"at M\"unchen, 85748 Garching} % MPI
% \author{C.~Pulvermacher}\affiliation{High Energy Accelerator Research Organization (KEK), Tsukuba 305-0801} % KEK
% \author{M.~V.~Purohit}\affiliation{University of South Carolina, Columbia, South Carolina 29208} % SouthCarolina
% \author{J.~Rauch}\affiliation{Department of Physics, Technische Universit\"at M\"unchen, 85748 Garching} % TUM
% \author{B.~Reisert}\affiliation{Max-Planck-Institut f\"ur Physik, 80805 M\"unchen} % MPI
% \author{E.~Ribe\v{z}l}\affiliation{J. Stefan Institute, 1000 Ljubljana} % Ljubljana
  \author{M.~Ritter}\affiliation{Ludwig Maximilians University, 80539 Munich} % LMU
% \author{J.~Rorie}\affiliation{University of Hawaii, Honolulu, Hawaii 96822} % Hawaii
  \author{A.~Rostomyan}\affiliation{Deutsches Elektronen--Synchrotron, 22607 Hamburg} % DESY
% \author{M.~Rozanska}\affiliation{H. Niewodniczanski Institute of Nuclear Physics, Krakow 31-342} % Krakow
% \author{S.~Rummel}\affiliation{Ludwig Maximilians University, 80539 Munich} % LMU
% \author{S.~Ryu}\affiliation{Seoul National University, Seoul 151-742} % Seoul
% \author{H.~Sahoo}\affiliation{University of Hawaii, Honolulu, Hawaii 96822} % Hawaii
% \author{T.~Saito}\affiliation{Department of Physics, Tohoku University, Sendai 980-8578} % Tohoku
% \author{K.~Sakai}\affiliation{High Energy Accelerator Research Organization (KEK), Tsukuba 305-0801} % KEK
  \author{Y.~Sakai}\affiliation{High Energy Accelerator Research Organization (KEK), Tsukuba 305-0801}\affiliation{SOKENDAI (The Graduate University for Advanced Studies), Hayama 240-0193} % KEK
% \author{M.~Salehi}\affiliation{University of Malaya, 50603 Kuala Lumpur}\affiliation{Ludwig Maximilians University, 80539 Munich} % Malaya
  \author{S.~Sandilya}\affiliation{University of Cincinnati, Cincinnati, Ohio 45221} % Cincinnati
% \author{D.~Santel}\affiliation{University of Cincinnati, Cincinnati, Ohio 45221} % Cincinnati
  \author{L.~Santelj}\affiliation{High Energy Accelerator Research Organization (KEK), Tsukuba 305-0801} % KEK
  \author{T.~Sanuki}\affiliation{Department of Physics, Tohoku University, Sendai 980-8578} % Tohoku
% \author{J.~Sasaki}\affiliation{Department of Physics, University of Tokyo, Tokyo 113-0033} % Tokyo
% \author{N.~Sasao}\affiliation{Kyoto University, Kyoto 606-8502} % Kyoto
% \author{Y.~Sato}\affiliation{Graduate School of Science, Nagoya University, Nagoya 464-8602} % Nagoya
  \author{V.~Savinov}\affiliation{University of Pittsburgh, Pittsburgh, Pennsylvania 15260} % Pittsburgh
% \author{T.~Schl\"{u}ter}\affiliation{Ludwig Maximilians University, 80539 Munich} % LMU
  \author{O.~Schneider}\affiliation{\'Ecole Polytechnique F\'ed\'erale de Lausanne (EPFL), Lausanne 1015} % Lausanne
  \author{G.~Schnell}\affiliation{University of the Basque Country UPV/EHU, 48080 Bilbao}\affiliation{IKERBASQUE, Basque Foundation for Science, 48013 Bilbao} % Bilbao
% \author{P.~Sch\"onmeier}\affiliation{Department of Physics, Tohoku University, Sendai 980-8578} % Tohoku
% \author{M.~Schram}\affiliation{Pacific Northwest National Laboratory, Richland, Washington 99352} % PNNL
  \author{C.~Schwanda}\affiliation{Institute of High Energy Physics, Vienna 1050} % Vienna
% \author{A.~J.~Schwartz}\affiliation{University of Cincinnati, Cincinnati, Ohio 45221} % Cincinnati
% \author{B.~Schwenker}\affiliation{II. Physikalisches Institut, Georg-August-Universit\"at G\"ottingen, 37073 G\"ottingen} % Goettingen
% \author{R.~Seidl}\affiliation{RIKEN BNL Research Center, Upton, New York 11973} % RIKEN
  \author{Y.~Seino}\affiliation{Niigata University, Niigata 950-2181} % Niigata
% \author{D.~Semmler}\affiliation{Justus-Liebig-Universit\"at Gie\ss{}en, 35392 Gie\ss{}en} % Giessen
  \author{K.~Senyo}\affiliation{Yamagata University, Yamagata 990-8560} % Yamagata
% \author{O.~Seon}\affiliation{Graduate School of Science, Nagoya University, Nagoya 464-8602} % Nagoya
% \author{I.~S.~Seong}\affiliation{University of Hawaii, Honolulu, Hawaii 96822} % Hawaii
  \author{M.~E.~Sevior}\affiliation{School of Physics, University of Melbourne, Victoria 3010} % Melbourne
% \author{L.~Shang}\affiliation{Institute of High Energy Physics, Chinese Academy of Sciences, Beijing 100049} % IHEP
% \author{M.~Shapkin}\affiliation{Institute for High Energy Physics, Protvino 142281} % Protvino
% \author{V.~Shebalin}\affiliation{Budker Institute of Nuclear Physics SB RAS, Novosibirsk 630090}\affiliation{Novosibirsk State University, Novosibirsk 630090} % BINP
% \author{C.~P.~Shen}\affiliation{Beihang University, Beijing 100191} % Beihang
  \author{T.-A.~Shibata}\affiliation{Tokyo Institute of Technology, Tokyo 152-8550} % NPC
% \author{H.~Shibuya}\affiliation{Toho University, Funabashi 274-8510} % Toho
% \author{N.~Shimizu}\affiliation{Department of Physics, University of Tokyo, Tokyo 113-0033} % Tokyo
% \author{S.~Shinomiya}\affiliation{Osaka University, Osaka 565-0871} % Osaka
  \author{J.-G.~Shiu}\affiliation{Department of Physics, National Taiwan University, Taipei 10617} % Taiwan
% \author{B.~Shwartz}\affiliation{Budker Institute of Nuclear Physics SB RAS, Novosibirsk 630090}\affiliation{Novosibirsk State University, Novosibirsk 630090} % BINP
% \author{A.~Sibidanov}\affiliation{School of Physics, University of Sydney, New South Wales 2006} % Sydney
  \author{F.~Simon}\affiliation{Max-Planck-Institut f\"ur Physik, 80805 M\"unchen}\affiliation{Excellence Cluster Universe, Technische Universit\"at M\"unchen, 85748 Garching} % MPI
% \author{J.~B.~Singh}\affiliation{Panjab University, Chandigarh 160014} % Panjab
% \author{R.~Sinha}\affiliation{Institute of Mathematical Sciences, Chennai 600113} % IMSC
% \author{P.~Smerkol}\affiliation{J. Stefan Institute, 1000 Ljubljana} % Ljubljana
% \author{Y.-S.~Sohn}\affiliation{Yonsei University, Seoul 120-749} % Yonsei
  \author{A.~Sokolov}\affiliation{Institute for High Energy Physics, Protvino 142281} % Protvino
% \author{Y.~Soloviev}\affiliation{Deutsches Elektronen--Synchrotron, 22607 Hamburg} % DESY
  \author{E.~Solovieva}\affiliation{P.N. Lebedev Physical Institute of the Russian Academy of Sciences, Moscow 119991}\affiliation{Moscow Institute of Physics and Technology, Moscow Region 141700} % Lebedev
% \author{S.~Stani\v{c}}\affiliation{University of Nova Gorica, 5000 Nova Gorica} % NovaGorica
  \author{M.~Stari\v{c}}\affiliation{J. Stefan Institute, 1000 Ljubljana} % Ljubljana
% \author{M.~Steder}\affiliation{Deutsches Elektronen--Synchrotron, 22607 Hamburg} % DESY
  \author{J.~F.~Strube}\affiliation{Pacific Northwest National Laboratory, Richland, Washington 99352} % PNNL
% \author{J.~Stypula}\affiliation{H. Niewodniczanski Institute of Nuclear Physics, Krakow 31-342} % Krakow
% \author{S.~Sugihara}\affiliation{Department of Physics, University of Tokyo, Tokyo 113-0033} % Tokyo
% \author{A.~Sugiyama}\affiliation{Saga University, Saga 840-8502} % Saga
  \author{M.~Sumihama}\affiliation{Gifu University, Gifu 501-1193} % NPC
% \author{K.~Sumisawa}\affiliation{High Energy Accelerator Research Organization (KEK), Tsukuba 305-0801}\affiliation{SOKENDAI (The Graduate University for Advanced Studies), Hayama 240-0193} % KEK
  \author{T.~Sumiyoshi}\affiliation{Tokyo Metropolitan University, Tokyo 192-0397} % TMU
% \author{K.~Suzuki}\affiliation{Graduate School of Science, Nagoya University, Nagoya 464-8602} % Nagoya
% \author{K.~Suzuki}\affiliation{Stefan Meyer Institute for Subatomic Physics, Vienna 1090} % Vienna
% \author{S.~Suzuki}\affiliation{Saga University, Saga 840-8502} % Saga
% \author{S.~Y.~Suzuki}\affiliation{High Energy Accelerator Research Organization (KEK), Tsukuba 305-0801} % KEK
% \author{Z.~Suzuki}\affiliation{Department of Physics, Tohoku University, Sendai 980-8578} % Tohoku
% \author{H.~Takeichi}\affiliation{Graduate School of Science, Nagoya University, Nagoya 464-8602} % Nagoya
  \author{M.~Takizawa}\affiliation{Showa Pharmaceutical University, Tokyo 194-8543}\affiliation{J-PARC Branch, KEK Theory Center, High Energy Accelerator Research Organization (KEK), Tsukuba 305-0801}\affiliation{Theoretical Research Division, Nishina Center, RIKEN, Saitama 351-0198} % NPC
  \author{U.~Tamponi}\affiliation{INFN - Sezione di Torino, 10125 Torino}\affiliation{University of Torino, 10124 Torino} % Torino
% \author{M.~Tanaka}\affiliation{High Energy Accelerator Research Organization (KEK), Tsukuba 305-0801}\affiliation{SOKENDAI (The Graduate University for Advanced Studies), Hayama 240-0193} % KEK
% \author{S.~Tanaka}\affiliation{High Energy Accelerator Research Organization (KEK), Tsukuba 305-0801}\affiliation{SOKENDAI (The Graduate University for Advanced Studies), Hayama 240-0193} % KEK
  \author{K.~Tanida}\affiliation{Advanced Science Research Center, Japan Atomic Energy Agency, Naka 319-1195} % NPC
% \author{N.~Taniguchi}\affiliation{High Energy Accelerator Research Organization (KEK), Tsukuba 305-0801} % KEK
% \author{G.~N.~Taylor}\affiliation{School of Physics, University of Melbourne, Victoria 3010} % Melbourne
  \author{F.~Tenchini}\affiliation{School of Physics, University of Melbourne, Victoria 3010} % Melbourne
% \author{Y.~Teramoto}\affiliation{Osaka City University, Osaka 558-8585} % OsakaCity
% \author{I.~Tikhomirov}\affiliation{Moscow Physical Engineering Institute, Moscow 115409} % MEPhI
% \author{K.~Trabelsi}\affiliation{High Energy Accelerator Research Organization (KEK), Tsukuba 305-0801}\affiliation{SOKENDAI (The Graduate University for Advanced Studies), Hayama 240-0193} % KEK
% \author{T.~Tsuboyama}\affiliation{High Energy Accelerator Research Organization (KEK), Tsukuba 305-0801}\affiliation{SOKENDAI (The Graduate University for Advanced Studies), Hayama 240-0193} % KEK
  \author{M.~Uchida}\affiliation{Tokyo Institute of Technology, Tokyo 152-8550} % NPC
% \author{T.~Uchida}\affiliation{High Energy Accelerator Research Organization (KEK), Tsukuba 305-0801} % KEK
% \author{I.~Ueda}\affiliation{High Energy Accelerator Research Organization (KEK), Tsukuba 305-0801} % KEK
% \author{S.~Uehara}\affiliation{High Energy Accelerator Research Organization (KEK), Tsukuba 305-0801}\affiliation{SOKENDAI (The Graduate University for Advanced Studies), Hayama 240-0193} % KEK
% \author{K.~Ueno}\affiliation{Department of Physics, National Taiwan University, Taipei 10617} % Taiwan
  \author{T.~Uglov}\affiliation{P.N. Lebedev Physical Institute of the Russian Academy of Sciences, Moscow 119991}\affiliation{Moscow Institute of Physics and Technology, Moscow Region 141700} % Lebedev
% \author{Y.~Unno}\affiliation{Hanyang University, Seoul 133-791} % Hanyang
  \author{S.~Uno}\affiliation{High Energy Accelerator Research Organization (KEK), Tsukuba 305-0801}\affiliation{SOKENDAI (The Graduate University for Advanced Studies), Hayama 240-0193} % KEK
% \author{S.~Uozumi}\affiliation{Kyungpook National University, Daegu 702-701} % Kyungpook
% \author{P.~Urquijo}\affiliation{School of Physics, University of Melbourne, Victoria 3010} % Melbourne
% \author{Y.~Ushiroda}\affiliation{High Energy Accelerator Research Organization (KEK), Tsukuba 305-0801}\affiliation{SOKENDAI (The Graduate University for Advanced Studies), Hayama 240-0193} % KEK
% \author{Y.~Usov}\affiliation{Budker Institute of Nuclear Physics SB RAS, Novosibirsk 630090}\affiliation{Novosibirsk State University, Novosibirsk 630090} % BINP
% \author{S.~E.~Vahsen}\affiliation{University of Hawaii, Honolulu, Hawaii 96822} % Hawaii
  \author{C.~Van~Hulse}\affiliation{University of the Basque Country UPV/EHU, 48080 Bilbao} % Bilbao
% \author{P.~Vanhoefer}\affiliation{Max-Planck-Institut f\"ur Physik, 80805 M\"unchen} % MPI 
  \author{G.~Varner}\affiliation{University of Hawaii, Honolulu, Hawaii 96822} % Hawaii
  \author{K.~E.~Varvell}\affiliation{School of Physics, University of Sydney, New South Wales 2006} % Sydney
% \author{K.~Vervink}\affiliation{\'Ecole Polytechnique F\'ed\'erale de Lausanne (EPFL), Lausanne 1015} % Lausanne
  \author{A.~Vinokurova}\affiliation{Budker Institute of Nuclear Physics SB RAS, Novosibirsk 630090}\affiliation{Novosibirsk State University, Novosibirsk 630090} % BINP
  \author{V.~Vorobyev}\affiliation{Budker Institute of Nuclear Physics SB RAS, Novosibirsk 630090}\affiliation{Novosibirsk State University, Novosibirsk 630090} % BINP
% \author{A.~Vossen}\affiliation{Indiana University, Bloomington, Indiana 47408} % Indiana
% \author{M.~N.~Wagner}\affiliation{Justus-Liebig-Universit\"at Gie\ss{}en, 35392 Gie\ss{}en} % Giessen
% \author{E.~Waheed}\affiliation{School of Physics, University of Melbourne, Victoria 3010} % Melbourne
% \author{B.~Wang}\affiliation{University of Cincinnati, Cincinnati, Ohio 45221} % Cincinnati
  \author{C.~H.~Wang}\affiliation{National United University, Miao Li 36003} % NUU
% \author{J.~Wang}\affiliation{Peking University, Beijing 100871} % Peking
  \author{M.-Z.~Wang}\affiliation{Department of Physics, National Taiwan University, Taipei 10617} % Taiwan
% \author{P.~Wang}\affiliation{Institute of High Energy Physics, Chinese Academy of Sciences, Beijing 100049} % IHEP
\author{X.~L.~Wang}\affiliation{Pacific Northwest National Laboratory, Richland, Washington 99352}\affiliation{High Energy Accelerator Research Organization (KEK), Tsukuba 305-0801} % PNNL
  \author{M.~Watanabe}\affiliation{Niigata University, Niigata 950-2181} % Niigata
  \author{Y.~Watanabe}\affiliation{Kanagawa University, Yokohama 221-8686} % Kanagawa
  \author{S.~Watanuki}\affiliation{Department of Physics, Tohoku University, Sendai 980-8578} % Tohoku
% \author{R.~Wedd}\affiliation{School of Physics, University of Melbourne, Victoria 3010} % Melbourne
% \author{S.~Wehle}\affiliation{Deutsches Elektronen--Synchrotron, 22607 Hamburg} % DESY
% \author{E.~White}\affiliation{University of Cincinnati, Cincinnati, Ohio 45221} % Cincinnati
  \author{E.~Widmann}\affiliation{Stefan Meyer Institute for Subatomic Physics, Vienna 1090} % Vienna
% \author{J.~Wiechczynski}\affiliation{H. Niewodniczanski Institute of Nuclear Physics, Krakow 31-342} % Krakow
% \author{K.~M.~Williams}\affiliation{Virginia Polytechnic Institute and State University, Blacksburg, Virginia 24061} % VPI
  \author{E.~Won}\affiliation{Korea University, Seoul 136-713} % Korea
% \author{B.~D.~Yabsley}\affiliation{School of Physics, University of Sydney, New South Wales 2006} % Sydney
% \author{S.~Yamada}\affiliation{High Energy Accelerator Research Organization (KEK), Tsukuba 305-0801} % KEK
% \author{H.~Yamamoto}\affiliation{Department of Physics, Tohoku University, Sendai 980-8578} % Tohoku
% \author{J.~Yamaoka}\affiliation{Pacific Northwest National Laboratory, Richland, Washington 99352} % PNNL
  \author{Y.~Yamashita}\affiliation{Nippon Dental University, Niigata 951-8580} % NihonDental
% \author{M.~Yamauchi}\affiliation{High Energy Accelerator Research Organization (KEK), Tsukuba 305-0801}\affiliation{SOKENDAI (The Graduate University for Advanced Studies), Hayama 240-0193} % KEK
% \author{S.~Yashchenko}\affiliation{Deutsches Elektronen--Synchrotron, 22607 Hamburg} % DESY
  \author{H.~Ye}\affiliation{Deutsches Elektronen--Synchrotron, 22607 Hamburg} % DESY
  \author{J.~Yelton}\affiliation{University of Florida, Gainesville, Florida 32611} % Florida
% \author{Y.~Yook}\affiliation{Yonsei University, Seoul 120-749} % Yonsei
  \author{C.~Z.~Yuan}\affiliation{Institute of High Energy Physics, Chinese Academy of Sciences, Beijing 100049} % IHEP
% \author{Y.~Yusa}\affiliation{Niigata University, Niigata 950-2181} % Niigata
% \author{C.~C.~Zhang}\affiliation{Institute of High Energy Physics, Chinese Academy of Sciences, Beijing 100049} % IHEP
% \author{L.~M.~Zhang}\affiliation{University of Science and Technology of China, Hefei 230026} % USTC
  \author{Z.~P.~Zhang}\affiliation{University of Science and Technology of China, Hefei 230026} % USTC
% \author{L.~Zhao}\affiliation{University of Science and Technology of China, Hefei 230026} % USTC
  \author{V.~Zhilich}\affiliation{Budker Institute of Nuclear Physics SB RAS, Novosibirsk 630090}\affiliation{Novosibirsk State University, Novosibirsk 630090} % BINP
  \author{V.~Zhukova}\affiliation{Moscow Physical Engineering Institute, Moscow 115409} % MEPhI
  \author{V.~Zhulanov}\affiliation{Budker Institute of Nuclear Physics SB RAS, Novosibirsk 630090}\affiliation{Novosibirsk State University, Novosibirsk 630090} % BINP
% \author{M.~Ziegler}\affiliation{Institut f\"ur Experimentelle Kernphysik, Karlsruher Institut f\"ur Technologie, 76131 Karlsruhe} % Karlsruhe
% \author{T.~Zivko}\affiliation{J. Stefan Institute, 1000 Ljubljana} % Ljubljana
 \author{A.~Zupanc}\affiliation{Faculty of Mathematics and Physics, University of Ljubljana, 1000 Ljubljana}\affiliation{J. Stefan Institute, 1000 Ljubljana} % Ljubljana
% \author{N.~Zwahlen}\affiliation{\'Ecole Polytechnique F\'ed\'erale de Lausanne (EPFL), Lausanne 1015} % Lausanne
\collaboration{The Belle Collaboration}



\begin{abstract}
\noindent
We have searched for the Cabibbo-suppressed decay $\Lambda_c^+\to\phi p\pi^0$ in $e^+e^-$ collisions using a data sample corresponding to
an integrated luminosity of 915 $\rm fb^{-1}$. The data were collected by the Belle experiment  at the 
KEKB $e^+e^-$ asymmetric-energy collider running at or near the $\Upsilon(4S)$ and $\Upsilon(5S)$ resonances. No significant signal is observed, and we set an upper limit on the branching fraction of $\mathcal{B}(\Lambda_c^+\to \phi p\pi^0) <15.3\times10^{-5}$ at 90\% confidence level. %We see no evidence of a hidden-strangeness pentaquark decay $P_s^+\to\phi p$ and set an upper limit on the product branching fraction of $\mathcal{B}(\Lambda_c^+\to P_s^+\pi^0) \times \mathcal{B}(P_s^+\to \phi p)<9.3\times10^{-5}$ at 90\% confidence level.
The contribution of nonresonant $\Lambda_c^+\to K^+K^- p\pi^0$ decays is found to be consistent with zero, and the corresponding upper limit on its  branching fraction is set to be $\mathcal{B}(\Lambda_c^+\to K^+K^-p\pi^0)_{\rm NR} <6.3\times10^{-5} $ at 90\% confidence level. We also search for an intermediate 
hidden-strangeness pentaquark decay $P^+_s\to\phi p$. 
We see no evidence for this intermediate decay and set 
an upper limit on the product branching fraction of
${\cal B}(\Lambda_c^+\to P^+_s \pi^0)\times {\cal B}(P^+_s\to\phi p)
<8.3\times 10^{-5}$ at 90\% confidence level. Finally, 
we  measure the branching fraction for the Cabibbo-favored decay $\Lambda_c^+\to K^-\pi^+p\pi^0$;  the result is  $\mathcal{B}(\Lambda_c^+\to K^-\pi^+p\pi^0)= (4.42\pm0.05\, (\rm stat.) \pm 0.12\, (\rm syst.) \pm 0.16\, (norm.))\%$, which is the most precise measurement to date.
\end{abstract}

\pacs{13.30.Eg, 14.20.Lq, 14.20.pt}

\maketitle

\tighten
The story of exotic hadron  spectroscopy   begins  with the  discovery  of  the
$X(3872)$ by the Belle collaboration in 2003~\cite{Choi:2003ue}. Since then, many exotic $X\!Y\!Z$ states have been reported by Belle and other experiments~\cite{Rev_xyz}. Recent observations of two hidden-charm pentaquark  states $P_c^+(4380)$ and $P_c^+(4450)$ by the LHCb collaboration in the $J/\psi p$ invariant mass spectrum of the  $\Lambda_b^0\to J/\psi pK^- $ process~\cite{Aaij:2015tga} raises the question of whether a hidden-strangeness pentaquark  $P_s^+$, where the $c\bar{c}$ pair  in  $P_c^+$  is replaced by  an $s\bar{s}$ pair, exists~\cite{Kopeliovich:2015vqa, Zhu:2015bba, Lebed:2015dca}. The strange-flavor analogue of the $P_c^+$ discovery channel is the decay $\Lambda_c^+\to\phi p\pi^0$~\cite{Kopeliovich:2015vqa, Lebed:2015dca}, shown in Fig.~\ref{fig:Feynman} (a)~\cite{charge-conjugate}. The detection of a hidden-strangeness pentaquark could be possible through  the $\phi p$ invariant mass spectrum within this channel [see Fig.~\ref{fig:Feynman} (b)],
if the underlying mechanism creating the $P_c^+$ states also holds for $P_s^+$, independent of the flavor~\cite{Lebed:2015dca}, and only if  the mass of $P_s^+$ is less than $M_{\Lambda_c^+}-M_{\pi^0}$.  
%This proposal is not at all guaranteed success, since the $s$ quark is not truly heavy ($i.e.$, the current quark mass $m_s$ is smaller than $\Lambda_{QCD}$), and it may well turn out that exotics of the type thus far seen absolutely require
%the presence of two heavy quarks~\cite{Lebed:2015dca}.  
In an analogous $s\bar{s}$ process of $\phi$ photoproduction $(\gamma p\to\phi p)$,  a forward-angle  bump structure at $\sqrt{s}\approx2.0$ GeV 
has been observed by  the LEPS~\cite{Mibe:2005er} and CLAS collaborations~\cite{Dey:2014tfa}.
% was reported by the CLAS collaboration~\cite{Dey:2014tfa, Dey:2014npa}.  Similar indications were also seen in results from the LEPS %collaboration~\cite{Mibe:2005er}. 
However, this structure appears only at the most forward angles, %(in which $\phi$ lies in the same direction as $\gamma$), 
which is  not   expected for  the decay of a resonance~\cite{Lebed:2015fpa}.
%almost certainly does not indicate a resonance.
%, meaning that the bump is only a ``would-be" pentaquark. 
\begin{figure}[htb]
%\vskip -0.3cm
\centering
\includegraphics[width=0.245\textwidth]{Fig1a}%
\includegraphics[width=0.245\textwidth]{Fig1b}
\caption{\small Feynman diagram for the decay (a) $\Lambda_c^+\to\phi p\pi^0$ and (b) $\Lambda_c^+\to P_s^+\pi^0$.}
\label{fig:Feynman}
\end{figure}

Previously, the decay $\Lambda_c^+\to\phi p\pi^0$ has not been studied by any experiment. In this paper, we report a search for this decay  using a data set corresponding to an integrated luminosity of 915 $\rm fb^{-1}$ collected with the Belle detector~\cite{Belle} recorded at or near the $\Upsilon(4S)$ and $\Upsilon(5S)$ resonances at the KEKB asymmetric-energy $e^+e^-$ (3.5 on 8.0~GeV) collider~\cite{KEKB}. %In addition, we study the $p\pi^0$ and $\phi p$ mass spectra of the $\Lambda_c^+\to\phi p\pi^0$ process. 
In addition, we search for the nonresonant decay $\Lambda_c^+\to K^+K^-p\pi^0$ and measure the branching fraction of the  Cabibbo-favored decay $\Lambda_c^+\to K^-\pi^+p\pi^0$.

The Belle detector is %a large-solid-angle magnetic spectrometer that consists of a silicon vertex detector (SVD), a 50-layer central drift chamber (CDC), an array of aerogel threshold Cherenkov counters (ACC), a barrel-like arrangement of time-of-flight scintillation counters (TOF), and an electromagnetic calorimeter (ECL) composed of CsI(Tl) crystals located inside a superconducting solenoid coil that provides a 1.5 T magnetic field. An iron flux-return located outside of the coil is instrumented to detect $K^0_L$ mesons and to identify muons. The detector is 
described in detail elsewhere~\cite{Belle}. To calculate the detector acceptance and reconstruction efficiencies and to study background, we use Monte Carlo (MC) simulated events. The MC events are generated uniformly in phase space  with
{\mbox{\textsc{EvtGen}}\xspace}~\cite{Lange:2001uf} and {\mbox{\textsc{JetSet}}\xspace}~\cite{Sjostrand:1993yb}; the detector response is modeled using {\mbox{\textsc{Geant3}}\xspace}~\cite{geant3}. Final-state radiation is taken into account using the {\mbox{\textsc{Photos}}\xspace}~\cite{Barberio:1993qi} package.

The reconstruction of   $\Lambda_c^+\to \phi p\pi^0$ (and  $\Lambda_c^+\to K^- \pi^+ p\pi^0$) decays proceeds by first reconstructing $\pi^0\to\gamma\gamma$ candidates. %We begin reconstruction of   $\Lambda_c^+\to h^+K^- p\pi^0$ final states, where $h$ is either a  kaon or pion by first reconstructing $\pi^0\to\gamma\gamma$ decays.
An electromagnetic calorimeter (ECL) cluster not matched to any track is identified as a photon candidate. Such candidates are required to have an energy greater than 50~MeV in the barrel region and 100~MeV in the endcap regions, where the barrel region covers the polar angle range $32^{\circ} < \theta < 130^{\circ}$, and the endcap regions cover the ranges $12^{\circ} < \theta < 32^{\circ}$ and $130^{\circ}<\theta <157^{\circ}$.  To reject showers produced by neutral hadrons, the photon energy deposited in the $3\times3$ array of ECL crystals centered on the crystal with the highest energy must exceed 80\% of the energy deposited in the corresponding $5\times5$ array of crystals.  %We reconstruct $\pi^0\to\gamma\gamma$ decays by pairing together photon candidates and requiring that the $\gamma\gamma$ invariant mass be in the range 0.115--0.155 GeV/$c^2$. 
We require that the $\gamma\gamma$ invariant mass be within  0.020 GeV/$c^2$ (about $3.5\sigma$ in resolution)  of  the known $\pi^0$ mass~\cite{PDG}.
% the range $m(\gamma\gamma)\in(0.115-0.155)$~GeV/$c^2$. 
%This corresponds to $\pm3.5\sigma$ in mass resolution around the nominal $\pi^0$ mass~\cite{PDG}. 
To improve the $\pi^0$ momentum resolution, we perform a mass-constrained fit and require that the resulting $\chi^2$ be less than 30. In addition, the momentum of the $\pi^0$ candidates in the center-of-mass (CM) frame is required to be higher than 0.30 GeV/$c$.

We subsequently combine $\pi^0$ candidates with three charged tracks.
Such tracks  are identified using  requirements on the distance of closest approach with respect to the interaction point  along the $z$ axis (antiparallel to the $e^+$ beam) of $|dz|< 1.0$ cm, and in the transverse plane of $dr<0.1$  cm. In addition, charged tracks are required to have a minimum number of hits in the vertex detector ($>1$ in both  the  $z$ and transverse directions). Information obtained  from the central drift chamber, the time-of-flight scintillation counters, and the aerogel threshold Cherenkov counters is combined  to form a likelihood $\mathcal{L}$ for hadron identification. A charged track with the likelihood ratios of $\mathcal{L}_K/(\mathcal{L}_{\pi}+\mathcal{L}_K)> 0.9$ and $\mathcal{L}_K/(\mathcal{L}_{p}+\mathcal{L}_K) > 0.6$;  $\mathcal{L}_K/(\mathcal{L}_{\pi}+\mathcal{L}_K)<0.6$ and $\mathcal{L}_{\pi}/(\mathcal{L}_{p}+\mathcal{L}_{\pi}) > 0.6$; and  $\mathcal{L}_p/(\mathcal{L}_{p}+\mathcal{L}_K) > 0.9$ and $\mathcal{L}_p/(\mathcal{L}_{p}+\mathcal{L}_{\pi})> 0.9$ is regarded as kaon, pion and  proton, respectively. The  efficiencies of these requirements for kaons, pions, and protons are 77\%, 97\%, and 75\%, respectively. The probabilities for a kaon, pion, or proton to be misidentified  are $\mathcal{P}(K\to\pi)\approx10\%$, $\mathcal{P}(K\to p)\approx1\%$; $\mathcal{P}(\pi\to K)\approx1\%$, $\mathcal{P}(\pi\to p)<1\%$; and $\mathcal{P}(p\to K)\approx7\%$, $\mathcal{P}(p\to \pi)\approx1\%$.  Candidate $\phi$ mesons are formed from two oppositely charged tracks that have been identified as kaons. %The invariant mass of the $K^+K^-$ pair, $m(K^+K^-)$, is required to  be within 0.020 GeV/$c^2$ of the nominal  $\phi$ meson mass~\cite{PDG}. 
We accept events in the wide $K^+K^-$ mass range  $m(K^+K^-)\in(0.99,~1.13)$~GeV/$c^2$.
To suppress combinatorial background, especially from $B$ meson decays, we require that the scaled momentum $(x_p=Pc/\sqrt{E^2_{\rm CM}/4-M^2c^4})$  be greater than 0.45, where $E_{\rm CM}$ is the total CM energy, and $P$ and $M$ are the momentum and invariant mass of the $\Lambda_c^+$ candidates. 
A vertex fit is performed to the charged tracks to form a $\Lambda_c^+$ vertex, and we require that the $\chi^2$ from the fit be less than 50. The decay $\Lambda_c^+\to\Sigma^+\phi$ has the same final state as  the signal decay and is Cabibbo-favored. To avoid contamination from this decay, we reject candidates in which the $p\pi^0$ system has an invariant mass within 0.010 GeV/$c^2$ of the known $\Sigma^+$ mass~\cite{PDG}.
%All quantities are evaluated in the CM frame. We retain events that satisfy $M\in (2.22-2.35) ~{\rm GeV}/c^2$. 
We extract the $\Lambda_c^+$ yield in a signal region that spans  $2.5\sigma$ in resolution %$\pm 0.010~{\rm GeV}/c^2$ 
around the $\Lambda_c^+$  mass~\cite{PDG}; this range corresponds to $\pm0.015$~GeV/$c^2$ for $\Lambda_c\to K^-\pi^+p\pi^0$  and approximately $\pm0.010$~GeV/$c^2$ for the other decays studied.

After applying all these selection criteria, about 16\% of events  in the signal region have multiple $\Lambda_c^+$ candidates.
For these events, we retain the candidate having the smallest sum of $\chi^2$ values obtained from the $\pi^0$ mass-constrained fit and the $\Lambda_c^+$ vertex fit. According to MC simulation, this criterion selects the correct $\Lambda_c^+$ candidate in 72\% of multiple-candidate events. 

In order to extract the signal yield, we perform a two-dimensional (2D) unbinned extended maximum likelihood fit to the variables $m (K^+K^-p\pi^0)$ and  $m(K^+K^-)$.  Our likelihood function  accounts for three components: $\phi p\pi^0$ signal, $K^+K^-p\pi^0$ nonresonant events, and  combinatorial background. The likelihood  function is defined as
\begin{linenomath}
\begin{equation}
e^{-\sum_j Y_j}\prod_i^N\left( \sum_j Y_j\mathcal{P}_j\big[m^i(K^+K^- p\pi^0),m^i(K^+K^-)\big]\right),
\end{equation}
\end{linenomath}
where $N$ is the total number of events, $\mathcal{P}_j\big[m^i(K^+K^- p\pi^0),m^i(K^+K^-)\big]$  is the probability density function (PDF) of signal or background component $j$ for event $i$, and $j$ runs over all signal and background components. The parameter $Y_j$ is the yield of component $j$. %Our likelihood function  consists of three components: $\phi p\pi^0$ signal, $K^+K^-p\pi^0$ nonresonant contribution and  combinatorial background. 
The $m(K^+K^-p\pi^0)$ for signal and nonresonant contributions are modeled with the sum of two Crystal Ball (CB) functions~\cite{Skwarnicki:1986xj} having a common mean, whereas for the combinatorial background, a second-order Chebyshev polynomial  is used. The peak positions and resolutions of the CB functions are
adjusted according to data-MC differences observed in the high statistics  sample of $\Lambda_c^+\to K^-\pi^+p\pi^0$ decays. The $m(K^+K^-)$ of signal is modeled with a relativistic Breit-Wigner  function convolved with a Gaussian resolution function ($\rm RBW\otimes G$), with the mass and width of the resonance $\phi$  fixed to their nominal values~\cite{PDG}. The width of the Gaussian resolution function is fixed to the value obtained from the MC simulation. The $m(K^+K^-)$ of nonresonant  background is modeled with a one-dimensional nonparametric PDF~\cite{Cranmer:2000du}.  The $m(K^+K^-)$ of combinatorial background  is modeled with the sum of  a third-order Chebyshev polynomial and the same ${\rm RBW\otimes G}$ function as used to model the signal.
The floated parameters are the component yields $Y_j$ and, for the combinatorial background, the coefficients of the Chebyshev polynomials and the fraction of the RBW. All other parameters are fixed in the fit to the  values obtained from the MC simulation.
%All parameters, except the component yields $Y_j$,  the coefficients and the fraction (in $m(K^+K^-)$ PDF)  of polynomial functions  in the combinatorial background, are fixed in the fit to the  values obtained from the MC simulation.  
Projections of the fit result are shown in Fig.~\ref{fig:2dfit}.
%\begin{figure*}[htp]
%\begin{center}
%    \includegraphics[width=0.475\textwidth]{Fig2a}%
  %     \includegraphics[width=0.475\textwidth]{Fig2b}
%\end{center}
%\vskip -0.5cm
%\caption{\small Projections of the 2D fit: (a) $m(\phi p\pi^0)$ in  $\pm 0.008~{\rm GeV}/c^2$ region around the  nominal %$\phi$ mass and (b) $m(K^+K^-)$  in  $\pm 0.01~{\rm GeV}/c^2$ region around the nominal $\Lambda_c^+$ mass. The %points with the error bars are the  data,  the (red) dotted, (green) dashed  and (brown) dot-dashed curves represent the %combinatorial,  signal and nonresonant candidates, respectively, and (blue) curves represent the total PDF. %The %normalized residuals (pull) in each bin are shown below, defined as the difference between the data and the fit divided by %the uncertainties on the data.
%}
%\label{fig:2dfit}
%\end{figure*}
\begin{figure*}[h!tp]
\begin{center}
    \includegraphics[width=0.495\textwidth]{Fig2a}%
       \includegraphics[width=0.495\textwidth]{Fig2b}
\end{center}
\vskip -0.5cm
\caption{\small Projections of the 2D fit: (a) $m(K^+K^- p\pi^0)$ and (b) $m(K^+K^-)$. The points with the error bars are the  data, and the (red) dotted, (green) dashed and (brown) dot-dashed curves represent the combinatorial,  signal and nonresonant candidates, respectively, and (blue) solid curves represent the total PDF. The solid curve in (b) completely overlaps the curve for the combinatorial background.}
\label{fig:2dfit}
\end{figure*}
From the fit, we extract $148.4\pm61.8$ signal events, $75.9\pm84.8$ nonresonant events, 
and $7158.4\pm36.4$ combinatorial background events in the $\Lambda_c^+$ signal region.  The statistical significance is evaluated as $\sqrt{-2\ln(\mathcal{L}_0/\mathcal{L}_{\rm max})}$, where $\mathcal{L}_0$ is the likelihood value when the signal yield is fixed to zero, and $\mathcal{L}_{\rm max}$ is  the nominal likelihood value. The statistical significances are found to be 2.4 and 1.0 standard deviations for $\Lambda_c^+\to\phi p \pi^0$ and nonresonant $\Lambda_c^+\to K^+K^- p \pi^0$ decays, respectively.

%To measure the nonresonant $\Lambda_c^+\to K^+K^-p\pi^0$ branching fraction, we relax 
%the $K^+K^-$ mass requirement and refit. All  PDFs are the same as above except  $m(K^+K^-)$ of the combinatorial %background, which is now modeled with a sum of RBW and a third-order Chebyshev polynomial function. We obtain %$75.9\pm84.8$ nonresonant events in  $m(K^+K^-)\in(0.99,~1.13)$~GeV/$c^2$  with a statistical significance of 1.0 %standard deviation.

We use the well-established decay $\Lambda_c^+\to p K^-\pi^+$~\cite{PDG} as the normalization channel for the branching fraction measurements. 
The track, particle identification,  and vertex selection criteria are similar to those used for the signal decays. If there are multiple candidates present in an event, we select the candidate  having the smallest value of $\chi^2$ from the $\Lambda_c^+$ vertex fit.
%$\phi p\pi^0$ final state. 
The resulting invariant mass distribution of the $pK^-\pi^+$ candidates is shown in Fig.~\ref{fig:invmass_control_data}. The signal is modeled with the sum of three Gaussian functions, and the combinatorial background is modeled with a linear function. There are  $1\,468\,435\pm 4816$ signal candidates and $567\,855\pm815$ background candidates in the $\Lambda_c^+$ signal region.
\begin{figure}[htb]
%\vskip -0.3cm
\centering
\includegraphics[width=0.5\textwidth]{Fig3}
\caption{\small  Fit to the invariant mass distribution of $p K^-\pi^+$. The points with the error bars are the  data,  the (red) dotted and (green) dashed curves  represent the combinatorial and  signal  candidates, respectively, and (blue) curve represents the total PDF.}
\label{fig:invmass_control_data}
\end{figure}

The ratio of branching fractions is calculated as
\begin{linenomath}
\begin{eqnarray}
\frac{\mathcal{B}(\Lambda_c^+\to {\rm final~state})}{\mathcal{B}(\Lambda_c^+\to pK^-\pi^+)} &=& \frac{Y_{\rm Sig}/\varepsilon_{\rm Sig}}{Y_{\rm Norm}/\varepsilon_{\rm Norm}} ,\label{eq:br}
\end{eqnarray}
\end{linenomath}
where $Y$ represents the observed yield in the signal region of the decay of interest and $\varepsilon$ corresponds to the reconstruction efficiency as obtained from the MC simulation. For the $\phi p\pi^0$ final state, we include  $\mathcal{B}(\phi\to K^+K^-)=(48.9\,\pm0.5)\%$~\cite{PDG} in $\varepsilon_{\rm sig}$ of Eq.~(\ref{eq:br}).
% The daughter branching fraction $\mathcal{B}(\phi\to K^+K^-)=(48.9\,\pm0.5)\%$ is also used for the $\phi p\pi^0$ final state. 
The reconstruction efficiencies are $(2.165\,\pm 0.007)\%$, $(2.291\,\pm0.008)\%$, and $(16.564\,\pm0.023)\%$  for $\phi p\pi^0$, nonresonant $K^+K^-p\pi^0$, and  $p K^-\pi^+$ final states, respectively, where the errors are due to MC statistics only.  The ratio $\varepsilon_{\rm Sig}/\varepsilon_{\rm Norm}$  is corrected by a factor $1.028 \pm 0.018$ to  account for   small differences in particle identification efficiencies between data and simulation.  This correction is estimated from a sample of $D^{*+}\to D^0(\to K^-\pi^+)\pi^+$ decays. 
For the $\phi p\pi^0$ final state, the ratio  is 
\begin{linenomath}
\begin{eqnarray}
\frac{\mathcal{B}(\Lambda_c^+\to\phi p\pi^0)}{\mathcal{B}(\Lambda_c^+\to pK^-\pi^+)}=(1.538\pm0.641^{+0.077}_{-0.100})\times10^{-3}.\nonumber 
\end{eqnarray}
\end{linenomath}
%where the uncertainties are statistical and systematic (discussed below), respectively. 
Whenever two or more uncertainties are quoted, the first is  statistical and the second is systematic.
Using  $\mathcal{B}(\Lambda_c^+\to pK^-\pi^+)=(6.46\pm0.24)\%$~~\cite{Amhis:2016xyh}, we obtain
\begin{linenomath}
\begin{eqnarray}
\mathcal{B}(\Lambda_c^+\to\phi p\pi^0)=(9.94\pm4.14^{+0.50}_{-0.65}\pm0.37)\times10^{-3},\nonumber 
\end{eqnarray}
\end{linenomath}
where the third uncertainty is that due to the branching fraction $\mathcal{B}(\Lambda_c^+\to p K^-\pi^+)$. 


Since the significances are below 3.0 standard deviations for both $\phi p\pi^0$ signal and $K^+K^-p\pi^0$ nonresonant decays, we set upper limits on their branching fractions  at 90\% confidence level (C.L.) using a Bayesian approach. The limit is obtained by integrating the
likelihood function from zero to infinity; the value that
corresponds to 90\% of this total area is taken as the
90\% C.L. upper limit.  We include the systematic uncertainty  in the calculation by
convolving the likelihood distribution with a Gaussian function
whose width is set equal to the total systematic uncertainty. %that
%affects the signal yield. 
The results are
\begin{linenomath}
\begin{eqnarray*}
\mathcal{B}(\Lambda_c^+\to \phi p\pi^0) &<& 15.3\times10^{-5} ,\\
\mathcal{B}(\Lambda_c^+\to K^+K^-p\pi^0)_{\rm NR} &<&6.3\times10^{-5} ,
\end{eqnarray*}
\end{linenomath}
which are the first limits on these branching fractions. 

To search for a putative $P_s^+\to\phi p$ decay, we select $\Lambda_c^+\to K^+K^-p\pi^0$ candidates in which $m(K^+K^-)$ is within 0.020~GeV/$c^2$ of the   $\phi$ meson mass~\cite{PDG} 
%To view the spectrum of $\phi p$ invariant mass in the $\phi p\pi^0$ final state, we select events within 0.020 GeV/$c^2$ of the nominal  $\phi$ meson %mass~\cite{PDG}  
and plot the  background-subtracted $m(\phi p)$ distribution (Fig.~\ref{fig:bkg-sub_dis}). This distribution is obtained by performing 2D fits as discussed above in bins of $m(\phi p)$. %where we  require that $\phi p\pi^0$ events be in the $\Lambda_c^+$ signal region. 
The data shows no clear evidence for a $P_s^+$ state. 
We set an upper limit on the product branching fraction 
$\mathcal{B}(\Lambda_c^+\to P_s^+\pi^0) \times \mathcal{B}(P_s^+\to \phi p)$ by fitting the  distribution of Fig.~\ref{fig:bkg-sub_dis}
to the sum of a RBW function and a phase space 
distribution determined from a sample of simulated $\Lambda^+_c\to\phi p\pi^0$ 
decays. We obtain $77.6\pm28.1$ $P_s^+$ events from the fit, which gives an upper limit of 
\begin{eqnarray*}
\mathcal{B}(\Lambda_c^+\to P_s^+\pi^0) \times 
\mathcal{B}(P_s^+\to \phi p) & < & 8.3\times 10^{-5}
\end{eqnarray*}
at 90\% C.L. This limit is calculated using the same procedure as that used for our limit on ${\cal B}(\Lambda_c^+\rightarrow \phi p \pi^0)$. The
systematic uncertainties for the two cases are essentially 
identical except for that due to the size of the MC sample 
used to calculate the reconstruction efficiency.
The  efficiency used here
[$\varepsilon=(2.438\pm 0.026)$\%] 
corresponds to the fitted values
$M_{P_s^+}=(2.025\pm 0.005)$~GeV/$c^2$ and 
$\Gamma_{P_s^+}=(0.022\pm 0.012)$~GeV. 

%In order to set an upper limit on product branching fraction $\mathcal{B}(\Lambda_c^+\to P_s^+\pi^0) \times \mathcal{B}(P_s^+\to \phi p)$, a binned-$\chi^2$
%fit is performed to the background-subtracted $m(\phi p)$ spectrum. A BW function with floating mass and width is used to model the $P_s^+$ component and a  polynomial function for non-$P_s^+$ component. The normalization of the polynomial function is set to zero in order to avoid the negative non-$P_s^+$ contribution. The mass fit gives $101.5\pm39.8$ $P_s^+$ events. We set an upper limit of $\mathcal{B}(\Lambda_c^+\to P_s^+\pi^0) \times \mathcal{B}(P_s^+\to \phi p)<9.3\times10^{-5}$ at 90\% C.L.; this is the first such limit. The reconstruction efficiency of $\Lambda_c^+\to P_s^+\pi^0$ is  $(2.438\pm0.026)\%$, where we use $M_{P_s^+}=2.024$~GeV/$c^2$ and $\Gamma_{P_s^+}=0.014$~GeV that are the central vales obtained from the fit. 
%All the systematic uncertainties are identical to that of the $\Lambda_c^+\to\phi p\pi^0$ decay mode, except the uncertainty due to the MC sample size that are carefully taken into account in the calculation of the upper limit.

%Figure~\ref{fig:bkg-sub_dis} shows the background-subtracted distribution of $m(\phi p)$, where the  $K^+K^-$ %invariant mass      is required to  be within 0.020 GeV/$c^2$ of the nominal  $\phi$ meson mass~\cite{PDG}. No clear %evidence of an exotic state is observed with the current statistics.
%The statistics is not enough to analyse the $\phi p$ intermediate state in search for the hidden-strangeness pentaquark. 
\begin{figure}[htb]
%\vskip -0.3cm
\centering
\includegraphics[width=0.5\textwidth]{Fig4}
\caption{\small   The background-subtracted distribution of $m(\phi p)$ in the $\phi p\pi^0$ final state. The points with error bars are data, and   the (blue) solid line shows the total PDF. The (red) dotted curve shows the fitted phase space component (which has fluctuated negative).}
\label{fig:bkg-sub_dis}
\end{figure}

For the $\Lambda_c^+\to K^-\pi^+p\pi^0$ sample, %(used to calibrate data-MC differences), 
the mass distribution is plotted in Fig.~\ref{fig:invmass_control2_data}. We fit this distribution to obtain the signal yield. We model the signal  with a sum of two CB functions having a common mean, and the combinatorial background  with a linear function.
%The invariant mass distribution of  $K^-\pi^+p\pi^0$ candidates in the control decay mode $\Lambda_c^+\to K^-\pi^+p\pi^0$  is shown in Fig.~\ref{fig:invmass_control2_data}, where the signal is modeled with a sum of two CB functions having a common mean, and the combinatorial background is modeled with a linear function. 
We find $242\,039\pm \,2342$ signal candidates and $472\,729\pm\,467$ background candidates in the $\Lambda_c^+$ signal region.
%within $\pm0.01$ GeV/$c^2$ of the $\Lambda_c^+$ nominal mass.
The corresponding signal efficiency is $(3.988\pm0.009)\%$, obtained from  MC simulation. We measure the ratio of branching fractions
\begin{linenomath}
\begin{equation*}
\frac{\mathcal{B}(\Lambda_c^+\to K^-\pi^+p\pi^0)}{\mathcal{B}(\Lambda_c^+\to K^-\pi^+p)}=(0.685\pm0.007\pm 0.018),
\end{equation*}
\end{linenomath}
%where the first uncertainty is statistical and the second is systematic. 
%Multiplying this ratio by the world average value of $\mathcal{B}(\Lambda_c^+\to K^-\pi^+p)=(6.46\pm0.24)\%$~\cite{Amhis:2016xyh}, we obtain
which results in a branching fraction
\begin{linenomath}
\begin{equation*}
\mathcal{B}(\Lambda_c^+\to K^-\pi^+p\pi^0)=(4.42\pm0.05\pm 0.12\pm0.16)\%.
\end{equation*}
\end{linenomath}
%where the first uncertainty is statistical, the second is systematic, and the third reflects the uncertainty due to the branching fraction of the normalization decay mode ($\mathcal{B}_{\rm Norm}$). 
This is the most precise measurement of $\mathcal{B}(\Lambda_c^+\to K^-\pi^+p\pi^0)$ to date and is consistent with the recently measured value $\mathcal{B}(\Lambda_c^+\to K^-\pi^+p\pi^0)=(4.53\pm0.23\pm0.30)\%$ by the BESIII collaboration~\cite{Ablikim:2015flg}.
\begin{figure}[h!tb]
\centering
\includegraphics[width=0.5\textwidth]{Fig5}
\caption{\small  Fit to the invariant mass distribution of $m(K^-\pi^+p\pi^0)$. The points with the error bars are the  data,  the (red) dotted and (green) dashed curves  represent the combinatorial and  signal  candidates, respectively, and (blue) curve represents the total PDF.  The $\chi^2/$ (number of bins) of the fit is 1.43, which indicate that the fit gives a good description of the data.}
\label{fig:invmass_control2_data}
\end{figure}

The systematic uncertainties on all branching fractions are listed in Table~\ref{sys1}. The uncertainties due to fixed parameters in the PDF shape are estimated by varying the parameters individually according to their statistical uncertainties. For each variation, the branching fraction is recalculated, and the difference with the nominal value is taken as the systematic uncertainty associated with that parameter. In order to determine the systematic uncertainty due to the $m(K^+K^-)$ PDF of nonresonant $K^+K^-p\pi^0$, we  replace the nonparametric PDF by a fourth-order polynomial  and refit the data.  For the $\phi p\pi^0$ final state, we also try including a separate PDF for an $f_0(980)$ intermediate state. The differences in the fit results are included as systematic uncertainties. We add  all uncertainties in quadrature to obtain the overall uncertainty due to PDF parametrization. The uncertainties due to errors in the calibration factors used to account for small data-MC differences in the signal PDF are evaluated separately but in a similar manner. 
%\begin{sidewaystable}
\begin{table*}[h!tb]
\renewcommand{\arraystretch}{1.2}
\caption{\small Systematic uncertainties (\%) on $\mathcal{B}(\Lambda_c^+\to \phi p\pi^0)$, $\mathcal{B}(\Lambda_c^+\to K^+K^-p\pi^0)_{\rm NR}$, and $\mathcal{B}(\Lambda_c^+\to K^-\pi^+p\pi^0)$.}
%Those listed in the upper section are associated with fitting for the 
%signal yields and are included in the  upper limit calculations.} % of $\mathcal{B}(\Lambda_c^+\to \phi p\pi^0)$, and $\mathcal{B}(\Lambda_c^+\to K^+K^-p\pi^0)$. }
\label{sys1}
\centering
\begin{tabular}{l|cccc}
\hline \hline
Source & $\mathcal{B}(\Lambda_c^+\to \phi p\pi^0)$ & $\mathcal{B}(\Lambda_c^+\to K^+K^-p\pi^0)_{\rm NR}$ & $\mathcal{B}(\Lambda_c^+\to K^-\pi^+p\pi^0)$  \\
\hline
%PDF parametrization & $^{+1.09}_{-1.29}$ & $^{+18.21}_{-12.59}$&- \\
%Calibration factor &$^{+3.80}_{-5.41}$  & $^{+30.69}_{-16.03}$&-\\
PDF parametrization & $^{+1.0}_{-1.9}$ & $^{+1.9}_{-1.5}$&- \\
Calibration factor &$^{+3.8}_{-5.2}$  & $^{+2.8}_{-1.5}$&-\\
Choice of $m(K^+K^-)$ range & $^{+0.0}_{-1.2}$ &- &-\\
%\hline
Best candidate selection &2.1 & 2.1 & 2.1  \\
MC sample size &  0.4 & 0.4 & 0.3 \\
$\pi^0$ reconstruction & 1.5 & 1.5 & 1.5 \\
Particle identification & 1.8 & 1.8 & - \\
$\mathcal{B}(\phi\to K^+K^-)$ & 1.0 & - & - \\
\hline
Total (without $\mathcal{B}_{\rm Norm}$) & $^{+5.0}_{-6.5}$ &$^{+4.6}_{-3.8}$ &$2.6$\\
\hline
$\mathcal{B}_{\rm Norm}$ & 3.7 &  3.7 &  3.7 \\
\hline\hline
\end{tabular}
\end{table*}
%\end{sidewaystable}
A  systematic uncertainty of $-1.2\%$ is assigned to account for changes associated with the choice of the $m(K^+K^-)$ range in $\mathcal{B}(\Lambda_c^+\to \phi p\pi^0)$. 
A 2.1\% systematic uncertainty is assigned due to the  best candidate selection. This is evaluated  by analyzing the decay channel $\Lambda_c^+\to\Sigma^+\phi$, which has much higher purity than the signal channels analyzed. % is relatively pure than other decays involved in this analysis having a $\pi^0$ in the final state.  
We determine this by applying an alternative best candidate selection, $i.e.$,  the deviations of the candidate $\phi$ and $\Sigma^+$ masses from their nominal values.  %We reanalyze the $\Lambda_c^+\to\Sigma^+\phi$ sample by changing the nominal best candidate selection criterion to the $\chi^2$ obtained from the deviations of the reconstructed $\phi$ and $\Sigma^+$ masses from their nominal values~\cite{PDG}. 
The difference in the branching fraction due to the two methods of the best candidate selection is taken as the systematic uncertainty.
%We measure the branching fraction with and without the best candidate selection. The difference is assigned as a systematic uncertainty. 
We assign a 1.5\% systematic uncertainty due to $\pi^0$ reconstruction; this is determined from a study of $\tau^-\to\pi^-\pi^0\nu_{\tau}$ decays. Since the branching fractions are measured with respect to the normalization channel $\Lambda_c^+\to p K^-\pi^+$, which has an identical number of charged tracks, the systematic uncertainty due to differences in  tracking performance between signal and normalization modes is negligible. There is a 1.8\% systematic uncertainty assigned for the  particle identification efficiencies in the $\phi p\pi^0$ and nonresonant $K^+K^- p\pi^0$ final states relative to the $pK^-\pi^+$ normalization channel.  The uncertainty in acceptance due to possible resonance substructure in the decay is found to be negligible. The total of the above systematic uncertainties is calculated as their sum in quadrature. In addition, there is a 3.7\% uncertainty due to the branching fraction of the normalization mode. As this large uncertainty does not arise from our
analysis and  will decrease with future measurements of $\Lambda_c^+\to pK^-\pi^+$, we quote it separately. 

In summary,  we have searched for the decays $\Lambda_c^+\to\phi p\pi^0$ and nonresonant $\Lambda_c^+\to K^+K^- p\pi^0$. No significant signal is observed for either decay mode and we set 90\% C.L. upper limits on their branching fractions, which are
$\mathcal{B}(\Lambda_c^+\to \phi p\pi^0) < 15.3\times10^{-5}$ and 
$\mathcal{B}(\Lambda_c^+\to K^+K^-p\pi^0)_{\rm NR} <6.3\times10^{-5}$.
We see no evidence for a hidden-strangeness pentaquark decay $P_s^+\to \phi p$ and set an upper limit on the product branching fraction of  $\mathcal{B}(\Lambda_c^+\to P_s^+\pi^0) \times \mathcal{B}(P_s^+\to \phi p)<8.3\times10^{-5}$ at 90\% C.L. This limit is a factor of six higher than the product branching fraction measured 
by LHCb for an analogous hidden-charm pentaquark state: 
$\mathcal{B}(\Lambda_b^0\to P_c(4450)^+ K^-) \times 
\mathcal{B}(P_c(4450)^+\to J/\psi\,p)=(1.3\pm 0.4)\times10^{-5}$~\cite{Aaij:2015tga}.
We also measure 
$\mathcal{B}(\Lambda_c^+\to K^-\pi^+p\pi^0)=(4.42\pm0.05\pm 0.12\pm0.16)\%$.
%to a significantly better precision than the previous measurement~\cite{Ablikim:2015flg}.
This is the world's most precise measurement of this branching fraction.

\begin{center}
\textbf{ACKNOWLEDGMENTS}
\end{center}
%----------- Long version, for most papers ----------- 
We thank the KEKB group for the excellent operation of the
accelerator; the KEK cryogenics group for the efficient
operation of the solenoid; and the KEK computer group,
the National Institute of Informatics, and the 
PNNL/EMSL computing group for valuable computing
and SINET5 network support.  We acknowledge support from
the Ministry of Education, Culture, Sports, Science, and
Technology (MEXT) of Japan, the Japan Society for the 
Promotion of Science (JSPS), and the Tau-Lepton Physics 
Research Center of Nagoya University; 
the Australian Research Council;
Austrian Science Fund under Grant No.~P 26794-N20;
the National Natural Science Foundation of China under Contracts 
No.~10575109, No.~10775142, No.~10875115, No.~11175187, No.~11475187, 
No.~11521505 and No.~11575017;
the Chinese Academy of Science Center for Excellence in Particle Physics; 
the Ministry of Education, Youth and Sports of the Czech
Republic under Contract No.~LTT17020;
the Carl Zeiss Foundation, the Deutsche Forschungsgemeinschaft, the
Excellence Cluster Universe, and the VolkswagenStiftung;
the Department of Science and Technology of India; 
the Istituto Nazionale di Fisica Nucleare of Italy; 
the WCU program of the Ministry of Education, National Research Foundation (NRF)
of Korea Grants No.~2011-0029457, No.~2012-0008143,
No.~2014R1A2A2A01005286,
No.~2014R1A2A2A01002734, No.~2015R1A2A2A01003280,
No.~2015H1A2A1033649, No.~2016R1D1A1B01010135, No.~2016K1A3A7A09005603, No.~2016K1A3A7A09005604, No.~2016R1D1A1B02012900,
No.~2016K1A3A7A09005606, No.~NRF-2013K1A3A7A06056592;
the Brain Korea 21-Plus program, Radiation Science Research Institute, Foreign Large-size Research Facility Application Supporting project and the Global Science Experimental Data Hub Center of the Korea Institute of Science and Technology Information;
the Polish Ministry of Science and Higher Education and 
the National Science Center;
the Ministry of Education and Science of the Russian Federation and
the Russian Foundation for Basic Research;
the Slovenian Research Agency;
Ikerbasque, Basque Foundation for Science and
MINECO (Juan de la Cierva), Spain;
the Swiss National Science Foundation; 
the Ministry of Education and the Ministry of Science and Technology of Taiwan;
and the U.S.\ Department of Energy and the National Science Foundation.
\begin{thebibliography}{99}

%\cite{Choi:2003ue}
\bibitem{Choi:2003ue} 
  S.~K.~Choi {\it et al.} (Belle Collaboration),
  Observation of a narrow charmoniumlike state in exclusive $B^{\pm} \to K^{\pm} \pi^+ \pi^- J/\psi$ decays,
  Phys.\ Rev.\ Lett.\  {\bf 91}, 262001 (2003).
  %doi:10.1103/PhysRevLett.91.262001
  %[hep-ex/0309032].
  %%CITATION = doi:10.1103/PhysRevLett.91.262001;%%
  %1119 citations counted in INSPIRE as of 09 Dec 2015


\bibitem{Rev_xyz} 
  C.~Patrignani {\it et al.}  (Particle Data Group),
 Review of particle physics, Spectroscopy of mesons containing two heavy quarks,
  Chin.\ Phys.\ C {\bf 40}, 100001 (2016).



%\cite{Aaij:2015tga}
\bibitem{Aaij:2015tga} 
  R.~Aaij {\it et al.} (LHCb Collaboration),
  Observation of $J/\psi p$ resonances consistent with pentaquark states in $\Lambda_b^0\to J/\psi K^-p$ decays,
  Phys.\ Rev.\ Lett.\  {\bf 115}, 072001 (2015).
 % doi:10.1103/PhysRevLett.115.072001
 % [arXiv:1507.03414 [hep-ex]].
  %%CITATION = doi:10.1103/PhysRevLett.115.072001;%%
  %95 citations counted in INSPIRE as of 09 Dec 2015

%\cite{Zhu:2015bba}
\bibitem{Zhu:2015bba} 
  R.~Zhu and C.~F.~Qiao,
  Pentaquark states in a diquark–triquark model,
  Phys.\ Lett.\ B {\bf 756}, 259 (2016).
 % doi:10.1016/j.physletb.2016.03.022
 % [arXiv:1510.08693 [hep-ph]].
  %%CITATION = doi:10.1016/j.physletb.2016.03.022;%%
  %20 citations counted in INSPIRE as of 02 Aug 2016

%\cite{Kopeliovich:2015vqa}
\bibitem{Kopeliovich:2015vqa} 
  V.~Kopeliovich and I.~Potashnikova,
 Simple estimates of the hidden beauty pentaquarks masses,
  arXiv:1510.05958 [hep-ph].
  %%CITATION = ARXIV:1510.05958;%%

%\cite{Lebed:2015dca}
\bibitem{Lebed:2015dca} 
  R.~F.~Lebed,
  Do the $P_c^+$ pentaquarks have strange siblings?,
  Phys.\ Rev.\ D {\bf 92}, 114030 (2015).
  %doi:10.1103/PhysRevD.92.114030
  %[arXiv:1510.06648 [hep-ph]].
  %%CITATION = doi:10.1103/PhysRevD.92.114030;%%
  %18 citations counted in INSPIRE as of 02 Mar 2017



\bibitem{charge-conjugate}
Unless stated otherwise, charge-conjugate modes are implicitly included.
%mention of a particular decay mode implies the additional use of the charge-conjugate mode.

%\cite{Mibe:2005er}
\bibitem{Mibe:2005er} 
  T.~Mibe {\it et al.} (LEPS Collaboration),
  Diffractive $\phi$-meson photoproduction on proton near threshold,
  Phys.\ Rev.\ Lett.\  {\bf 95}, 182001 (2005).
  %doi:10.1103/PhysRevLett.95.182001
 % [nucl-ex/0506015].
  %%CITATION = doi:10.1103/PhysRevLett.95.182001;%%
  %72 citations counted in INSPIRE as of 10 Jun 2016


%\cite{Dey:2014tfa}
\bibitem{Dey:2014tfa} 
  B.~Dey {\it et al.} (CLAS Collaboration),
  Data analysis techniques, differential cross sections, and spin density matrix elements for the reaction $\gamma p \rightarrow \phi p$,
  Phys.\ Rev.\ C {\bf 89}, 055208 (2014);
Phys.\ Rev.\ C {\bf 90}, 019901 (2014).
 % doi:10.1103/PhysRevC.90.019901, 10.1103/PhysRevC.89.055208
 % arXiv:1403.2110 [nucl-ex].
  %%CITATION = doi:10.1103/PhysRevC.90.019901, 10.1103/PhysRevC.89.055208;%%
  %10 citations counted in INSPIRE as of 10 May 2016

%\cite{Dey:2014npa}
%\bibitem{Dey:2014npa} 
  %B.~Dey,
  %``Phenomenology of $\phi$ photoproduction from recent CLAS data at Jefferson Lab,''
  %arXiv:1403.3730 [hep-ex].
  %%CITATION = ARXIV:1403.3730;%%
  %2 citations counted in INSPIRE as of 10 May 2016

%\cite{Lebed:2015fpa}
\bibitem{Lebed:2015fpa} 
  R.~F.~Lebed,
  Diquark substructure in $\phi$ photoproduction,
  Phys.\ Rev.\ D {\bf 92},  114006 (2015).
  %doi:10.1103/PhysRevD.92.114006
  %[arXiv:1510.01412 [hep-ph]].
  %%CITATION = doi:10.1103/PhysRevD.92.114006;%%
  %6 citations counted in INSPIRE as of 24 Jul 2017

\bibitem{Belle}
A.~Abashian {\it et al.} (Belle Collaboration), 
The Belle Detector,
Nucl. Instrum. Methods
 Phys. Res., Sect. A {\bf 479}, 117 (2002); also see the detector section in
 J.~Brodzicka {\it et al.}, 
Physics achievements from the Belle Experiment,
Prog. Theor. Exp. Phys. {\bf 2012}, 04D001 (2012). 

%\bibitem{svd2}
 % Z.~Natkaniec {\it et al.} (Belle SVD2 Group),
%``Status of the Belle silicon vertex detector,''
% Nucl. Instrum. Methods Phys. Res., Sect. A  {\bf 560}, 1 (2006).
%Y. Ushiroda (Belle SVD2 Group), Nucl. Instr. and Meth.A {\bf 511} 6 (2003). 


\bibitem{KEKB}
S.~Kurokawa and E.~Kikutani, 
Overview of the KEKB accelerators,
Nucl. Instrum. Methods Phys. Res. Sect.
 A {\bf 499}, 1 (2003), and other papers included in this volume.
T.Abe {\it et al.}, 
Achievements of KEKB,
Prog. Theor. Exp. Phys. {\bf 2013}, 03A001 (2013)
 and references therein.

 \bibitem{Lange:2001uf} 
  D.~J.~Lange,
  The {\mbox{\textsc{EvtGen}}\xspace} particle decay simulation package,
 Nucl. Instrum. Methods Phys. Res., Sect. A  {\bf 462}, 152 (2001).
  %%CITATION = NUIMA,A462,152;%%
  %1075 citations counted in INSPIRE as of 17 Nov 2014
  
%\cite{Sjostrand:1993yb}
\bibitem{Sjostrand:1993yb}
  T.~Sj\"{o}strand,
 High-energy physics event generation with {\mbox{\textsc{Pythia}}\xspace} 5.7 and {\mbox{\textsc{JetSet}}\xspace} 7.4,
  Comput.\ Phys.\ Commun.\  {\bf 82}  74 (1994).
  %doi:10.1016/0010-4655(94)90132-5
  %%CITATION = doi:10.1016/0010-4655(94)90132-5;%%
  %3534 citations counted in INSPIRE as of 23 Dec 2015

  \bibitem{geant3}
    R.~Brun {\em et al.}, {\mbox{\textsc{Geant}}\xspace} 3.21, CERN Report DD/EE/84-1, 1984.
   % {\em {GEANT3} Users Guide, {CERN} Program Library
    %  {W5013}} (1994).

%\cite{Barberio:1993qi}
\bibitem{Barberio:1993qi} 
  E.~Barberio and Z.~W\c{a}s,
 {\mbox{\textsc{Photos}}\xspace}: A universal Monte Carlo for QED radiative corrections. Version 2.0,
  Comput.\ Phys.\ Commun.\  {\bf 79}, 291 (1994).
  %doi:10.1016/0010-4655(94)90074-4
  %%CITATION = doi:10.1016/0010-4655(94)90074-4;%%
  %670 citations counted in INSPIRE as of 23 Dec 2015
 
 \bibitem{PDG}
  C.~Patrignani {\it et al.}  (Particle Data Group),
 Review of particle physics,
  Chin.\ Phys.\ C {\bf 40}, 100001 (2016).

%\cite{Skwarnicki:1986xj}
\bibitem{Skwarnicki:1986xj} 
  T.~Skwarnicki,
 A study of the radiative CASCADE transitions between the Upsilon-prime and Upsilon resonances,
  DESY-F31-86-02.
  %%CITATION = DESY-F31-86-02;%%
  %361 citations counted in INSPIRE as of 25 Dec 2015

%\cite{Cranmer:2000du}
\bibitem{Cranmer:2000du} 
  K.~S.~Cranmer,
  Kernel estimation in high-energy physics,
  Comput.\ Phys.\ Commun.\  {\bf 136}, 198 (2001).
  %doi:10.1016/S0010-4655(00)00243-5,
%  [hep-ex/0011057].
  %%CITATION = doi:10.1016/S0010-4655(00)00243-5;%%
  %269 citations counted in INSPIRE as of 04 Mar 2016

%\cite{Amhis:2016xyh}
\bibitem{Amhis:2016xyh} 
  Y.~Amhis {\it et al.} (Heavy Flavor Averaging Group),
 Averages of $b$-hadron, $c$-hadron, and $\tau$-lepton properties as of summer 2016,
  arXiv:1612.07233 [hep-ex].
  %%CITATION = ARXIV:1612.07233;%%
  %3 citations counted in INSPIRE as of 25 Jan 2017

%\cite{Ablikim:2015flg}
\bibitem{Ablikim:2015flg} 
  M.~Ablikim {\it et al.} (BESIII Collaboration),
  Measurements of absolute hadronic branching fractions of the $\Lambda_{c}^{+}$ baryon,
  Phys.\ Rev.\ Lett.\  {\bf 116}, 052001 (2016).
  %doi:10.1103/PhysRevLett.116.052001
 % [arXiv:1511.08380 [hep-ex]].
  %%CITATION = doi:10.1103/PhysRevLett.116.052001;%%
  %8 citations counted in INSPIRE as of 20 Oct 2016


\end{thebibliography}

\end{document}
