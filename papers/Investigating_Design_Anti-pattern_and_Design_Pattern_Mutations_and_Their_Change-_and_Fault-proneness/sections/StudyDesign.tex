\section{Experimental Setup}
\label{sec:expSetup}

We now describe the systems and patterns making our experimental setup.



\subsection{Subject Systems} 
\label{ssec:section4.1}

We consider seven Java-based open-source systems for our study, summarised in Table \ref{tab:StudySystems}. We select these systems based on diversity in code size, application domains, and evolution histories. The number of lines of code of these systems range from hundred of thousand to several millions. They belong to different domains, from IDE to database. Some were used in previous studies, which allows some comparisons. These systems have evolved over the years and have many commits/versions to provide a dataset for analyzing pattern mutations and fault-proneness.

However, the choice of systems is inherently a threat to the conclusion and generalisability of any empirical study, which we acknowledge in Section \ref{sec:Treats to Validity}. 

We now briefly describe the subject systems.

\begin{table*} [ht]
\centering
\caption{Analyzed systems}
\scalebox{0.8}{
\begin{tabular}{|l|l|l|r|r|r|}
\hline
\textbf{System} & \textbf{Applicaion domain} &\textbf{\# Commits} & \textbf{LOC} & \textbf{Issue Tracker}\\
\hline \hline
Eclipse for Java & IDE& 281,396 & 9,064,794 & Bugzilla \\
\hline
Nuxeo Platform& Colaboration management& 265,380 & 5,741,131 & Jira \\
\hline
oVirt & Visualization platform & 149,128 & 2,764,655 & Bugzilla \\
\hline
Matsim & Transportation management & 44,200 & 1,602,877 & Atlassian \\
\hline
Apache Solr& Search server& 30,995 & 658,711 & Jira\\
\hline
Apache Ignite&Distributed DB platform& 24,104 & 1,471,036 & Jira\\
\hline
Mule Community Edition&Integration platform & 22,891 & 309,616 & Jira \\
\hline
\end{tabular}
}
\label{tab:StudySystems}
\end{table*}

\noindent\textbf{Eclipse IDE for Java} is an IDE for Java developers. The IDE offers the Java Development Tools (JDT) to develop Java systems. It contains also CVS, SVN, and Git clients. It also includes an XML editor, Mylyn as a task management system, build supports for Maven and WindowBuilder, etc.

\noindent\textbf{Nuxeo}, also called Nuxeo Platform, is an open-source context management and collaboration platform, which provides different information management solutions for developers to build business applications.

\noindent\textbf{oVirt} is a visualization management platform in Java. It provides a centralized management of resources, storage, and virtual machines, which allows managing enterprise infrastructure.

\noindent\textbf{Matism} is a framework to build large-scale transport simulations. Its development team provides a comprehensive documentation for users and developers to ease usability and maintainability. 

\noindent\textbf{Apache Solr} from the Apache Lucene project is an open-source Java search server for Web sites, databases, and files. It is popular and fast, using Lucene Java search library at its core. It runs as a standalone full-text search server. 

\noindent\textbf{Apache Ignite} is a in-memory computing platform used as database and caching system. It helps solving problems related to speed and scalability and can be used to speed up relational and NoSQL databases.

\noindent\textbf{Mule} is the run-time engine of a Java-based enterprise service bus (ESB) and integration platform. Developers can connect applications quickly and easily to exchange data. It allows service creation and hosting, service mediation, message routing, and data transformation.



\subsection{Analyzed Patterns}

\subsubsection{Anti-patterns}

We select thirteen anti-patterns in our study. These anti-patterns introduced by Brown \etal \cite{brown1998antipatterns} express problems with data, complexity, size, and the features related to classes \cite{khomh2012exploratory}. They have been studied in previous work \cite{khomh2012exploratory}. We summarize their definitions below, details are available elsewhere \cite{romano2012analyzing}:

\begin{itemize}
\item AntiSingleton (AS): A class that provides mutable class variables, which could be used as global variables.

\item Blob (Bl) or God Class (GC): A class that is too large and not cohesive enough, which monopolizes most of the system processing, takes most of the decisions, and is associated to data classes.

\item ClassDataShouldBePrivate (CS): A class that exposes its fields, thus violating the principle of encapsulation.

\item ComplexClass (CC): A class that has (at least) one large method and complex method, in terms of cyclomatic complexity and line of codes LOCs.

\item LargeClass (LC): A class that has (at least) one large method, in terms of LOCs.

\item LazyClass (LZC): A class that has few fields and methods that are complex.

\item LongMethod (LM): A class that has (at least) one method that is overly long, in terms of LOCs.

\item LongParameterList (LP): A class that has (at least) one method with a long list of parameters with respect to the average numbers of parameters per methods.

\item MessageChain (MCh): A class that uses a long chain of method invocations to realize one of its functionality.

\item RefusedParentBequest (RP): A class that overrides methods using empty bodies.

\item SpaghettiCode (SC): A class declaring long methods which do not have any parameters. These methods are complex, with a complicated control flow. The class does not use polymorphism and--or inheritance.

\item SpeculativeGenerality (SG): A class that is defined as abstract but that has very few children, which do not make use of its methods.

\item SwissArmyKnife (SA): A class whose methods can be divided into disjoint sets, providing different, unrelated functionalities.
\end{itemize}


\subsubsection{Design Patterns}

We consider eight design patterns presented in Gamma \etal \cite{gamma1995design}, which we select due to their popularity and because previous works also studied them \cite{tsantalis2006design,vlissides1995design}. Their complete definitions and specifications are available in \cite{vlissides1995design,khomh2009playing}:

\begin{itemize}
\item Builder (Bu): A pattern to separate the construction of a complex object from its representation.

\item Command (Cm): A pattern to encapsulate a request as an object.

\item Composite (Cp): A pattern that composes objects into tree structures to represent part-whole hierarchies. Composite lets clients treat individual objects and compositions of objects uniformly.

\item Decorator (De): A pattern that attaches additional responsibilities to an object dynamically. Decorator provides a flexible alternative to sub-classing for extending functionality.

\item Factory Method (FM): A pattern that defines an API for object creation in which subclasses choose the class to instantiate.

\item Observer (Ob): A pattern that defines a one-to-many dependency between objects to notify all the object dependent on one object when it changes.

\item Prototype (Pt): A pattern that specifies the kind of objects to create using a prototypical instance.

\item Singleton (Si): A pattern that restricts the instantiation of a class to one object to coordinate actions across a system.
\end{itemize}