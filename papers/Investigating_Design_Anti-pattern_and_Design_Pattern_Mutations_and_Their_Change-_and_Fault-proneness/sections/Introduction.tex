\section{Introduction}
\label{sec:introduction}

\subsection{Background and Motivation}

Quality assurance is one of the most critical challenges in software development and evolution \cite{gamma1995design}. Under tight deadlines and other business constraints, developers may take poor design or coding decisions and may follow bad practices. These poor design decisions and bad coding practices are collectively called ``smells'' \cite{stamelos2002code,khomh2007perception}.

Smells are divided into several different categories \cite{fowler1999refactoring}, such as code smells, design smells, lexical smells, etc. Code smells include low-level problems in the source code that may be symptoms of poor coding practices \cite{van2002java,mantyla2003taxonomy}. Design smells include design anti-patterns that describe poor solutions to recurring design problems. Design anti-patterns have been reported to make software development and evolution difficult \cite{khomh2012exploratory}. They affect program comprehension \cite{abbes2011empirical} and increase change- and fault-proneness \cite{khomh2012exploratory}. 

Opposite to design anti-patterns, design patterns are good solutions to recurring design problems, which promote code reuse and increase reliability, readability, and flexibility \cite{gamma1995design}. Gamma \etal \cite{gamma1995design} suggested using specific design patterns to ease evolution and increase reuse and flexibility. Studies showed that design patterns often increase the quality of software systems (\eg{} \cite{ampatzoglou2012methodology}). Yet, a few studies showed that some design patterns can decrease some quality attributes \cite{khomh2008design}. 

Recent studies \cite{jaafar2013analysing,vokavc2004defect,aversano2007empirical} showed the existence of relationships between design patterns and design anti-patterns. Such relationships can help developers to understand their systems better and simplify development and evolution \cite{jaafar2013analysing}. Yet, no study considered that design patterns sometimes (d)evolve into design anti-patterns and analysed the impact of such mutations on software quality. 

Because design anti-patterns negatively affect software quality while design patterns improve it, understanding the evolution of design patterns and design anti-patterns into one another could help developers identify the riskiest design anti-patterns, avoid introducing design anti-patterns, and--or avoid evolving design patterns into such design anti-patterns.



\subsection{Research problem}

In previous works \cite{jaafar2013analysing,jaafar2014anti}, we investigated the static relationships between design anti-patterns and design patterns and how these relationships evolve in time. We studied the relationships between classes playing roles in design patterns and design anti-patterns and reported that static relationships between design patterns and design anti-patterns exist but they are temporary. We also showed that classes playing roles in design anti-patterns and having relationships with design patterns are more change-prone but are less fault-prone than other classes.

In another previous work, \cite{khomh2008design}, we also showed that the design of systems degrades over time, presumably due to the removal (or lack of use) of design patterns and the introduction of design anti-patterns. 

Thus, these studies (and others) considered design patterns and design anti-patterns as unique, atomic entities in each releases of the studied systems. Yet, during software evolution, design (anti)patterns \emph{do} evolve, appearing, disappearing, and mutating into one another. Understanding this evolution of design patterns and design anti-patterns across releases, and in particular their mutations into one another, could help developers avoid mutations that negatively impact software quality while promoting those that improve it.



\subsection{Contributions}

This study is a quasi-replication of a previous study by Jaafar \etal \cite{jaafar2014anti}. Some of the research questions in this paper are similar to those in the previous study \cite{jaafar2014anti}, which pertain to:

\begin{itemize}
\item Computing the probability of design anti-patterns mutations using Markov models.

\item Comparing classes with and without design anti-patterns and their fault-proneness. 
\end{itemize}

 \noindent Moreover, in this study, we consider both design patterns and design anti-pattern mutations during software evolution. We examine the impacts of design patterns and design anti-patterns mutations on change- and fault-proneness. We investigate seven open-source Java software systems to answer to the the following five research contributions:

 \begin{enumerate}
 \item We study how design patterns and design anti-patterns mutate over time using Markov models.

 \item We study the types of changes that occur during design anti-patterns mutations.

 \item We study the impact of these mutations on fault-proneness.

 \item We study the types of changes that lead to design patterns and design anti-patterns mutations.

 \item We study the most fault-prone transitions between design patterns and design anti-patterns.
 \end{enumerate}

We use seven different open-source systems of different sizes and from different application domains: Apache Ignite\footnote{\url{https://ignite.apache.org/}}, Apache Solr \footnote{\url{http://lucene.apache.org/solr/}}, Eclipse IDE \footnote{\url{https://www.eclipse.org/}}, Matsim \footnote{\url{https://matsim.org/}}, Mule \footnote{\url{http://www.mulesoft.org/}}, Nuxeo \footnote{\url{https://www.nuxeo.com/}}, and Ovirt \footnote{\url{https://www.ovirt.org/}}.

We consider thirteen design anti-patterns from \cite{brown1998antipatterns} and eight design patterns \cite{gamma1995design}. For the detection of design anti-patterns and design patterns, we use DECOR \cite{moha2010decor} and DeMIMA \cite{gueheneuc2008demima}. We first detect the occurrences of design anti-patterns and design patterns in all the studied releases of the systems and then investigate the types of mutations: persistent, deleted, introduced, and changed between these snapshots.

Second, we build Markov models \cite{meyn2012markov} to compute the probability values of such mutations. We build one Markov model per studied system. In the models, design anti-patterns and design patterns are nodes while the probabilities of their mutations into one another label the edges between nodes. We compute the probability values by analyzing all the releases of the software systems during the considered period of time.

Third, we use the SZZ algorithm  \cite{sliwerski2005changes} to find fault-inducing commits and investigate the impact of design patterns and--or design anti-patterns mutations on the fault-proneness of classes. 

Fourth, we define thirteen types of change and study them to discover the kinds of changes leading to mutations between design patterns and design anti-patterns. We also study the effects of the change types on fault-proneness. 



\subsection{Research Questions} \label{RQuestions}

% As we mentioned before, this study is a quasi-replication of a previous work \cite{jaafar2014anti}. Addition to that work, we also consider design patterns mutations during the evolution. In particular, we examine the impacts of design patterns and design anti-patterns mutation on the change- and fault-proneness of classes. 

We use the seven open-source Java software systems to answer the following four research questions: 

\begin{itemize}
\item {\bf RQ1:} \emph{\RQOne} We build Markov models \cite{meyn2012markov} showing which design patterns and--or design anti-patterns mutate into one another during a studied period of evolution. We consider both appearance and disappearance of design patterns and design anti-patterns. We observe that both design patterns and design anti-patterns mutate in the systems. We compute the probabilities of all possible mutations using the Markov models. We also present the most frequent mutations of design patterns and design anti-patterns along the following four mutation types:

\begin{enumerate}
\item Design anti-patterns to some other design anti-patterns;
\item Design anti-patterns to design patterns;
\item Design patterns to design anti-patterns;
\item Design patterns to some other design patterns.
\end{enumerate}

\item {\bf RQ2:} \emph{\RQTwo} Design patterns and design anti-patterns evolve through different types of changes as the system evolves. We define thirteen change types and investigate classes experiencing these change types and participating in design patterns and--or design anti-patterns. We see that different types of changes occur during the evolution of software systems and lead to different mutations. We study the impact of the types of changes on mutations between design patterns and--or design anti-patterns. We present the most prevalent type of changes leading to mutations. 

\item {\bf RQ3:} \emph{\RQThree} Design patterns and design anti-patterns may frequently mutate in other types of patterns during the evolution process. We study whether such mutations are risky regarding fault-proneness. We also present the riskiest transitions among design patterns and design anti-patterns. We observe that classes participating in mutated design anti-patterns are more fault-prone than classes involved in mutated design patterns. We also see that mutations between design anti-patterns and design patterns are more faulty than other mutations. 

\item {\bf RQ4:} \emph{\RQFour} We investigate whether the types of changes to design patterns and design anti-patterns impact fault-proneness. We examine faulty-classes and check whether a mutation occurred during the evolution of these classes. We also examine all changes experienced by the classes during the evolution of the systems. We observe that some of the change types make the systems more fault-prone. We study whether specific types of changes that cause the mutations between design patterns and design anti-patterns are more fault-prone. 
\end{itemize}

The results of these four research questions show that there is a high probability for some design patterns and design anti-patterns to mutate to others types of design patterns and design anti-patterns. The changes that lead to the mutations are mostly structural changes, in particular the addition of large number of attributes or long methods. Results also show that some mutations increase the fault-proneness of the analysed software systems.



\subsection{Organization}

The rest of the paper is organized as follows: Section \ref{sec:Related Works} describes the related work. Section \ref{sec:Methodology} presents the methodology of our study. Section \ref{sec:expSetup} reports its experimental setup. Section \ref{sec:Study Results} presents its results. Section \ref{sec:Discussion} discusses the results and Section \ref{sec:Treats to Validity} threats to their validity. Finally, Section \ref{sec:Conclusion and Future Work} concludes the paper with future work.