\section{Conclusion and Future Work}
\label{sec:Conclusion and Future Work}

We investigated the evolution and impacts of design patterns and design anti-patterns in terms of change- and fault-proneness during software evolution. We built Markov models to analyse the mutations of design patterns and design anti-patterns in seven open-source Java systems: Apache Ignite, Apache Solr, Eclipse, Matsim, Mule, Nuxeo, and Ovirt. We identified the change types that led to mutations and we calculated the probabilities of all possible mutations. Finally, we reported which patterns are mostly mutated into which other patterns (including appearance and disappearance) as well as change types.

Results showed that design patterns and design anti-patterns mutate into one another during software evolution and that these mutations impact the fault-proneness of classes participating in these patterns. Generally, when a mutation led to the introduction of a design anti-pattern, quality in terms of fault-proneness decreased; when a design anti-pattern was removed or mutated into a design pattern,  quality increased.

Using this information, developers can focus on the design patterns that are most likely to mutate into design anti-patterns and--or to have more faults. Thus, this information can help evolution and quality assurance by focusing refactoring efforts on classes with design (anti-)patterns that could mutate into patterns with higher fault-proneness.

In future work, we intend to apply our study on more systems written in different programming languages and also consider more design (anti-)patterns. We will also attempt to identify the root causes of mutations by studying manually and qualitatively the changes leading to mutations.
