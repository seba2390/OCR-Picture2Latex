% SIAM Shared Information Template
% This is information that is shared between the main document and any
% supplement. If no supplement is required, then this information can
% be included directly in the main document.


% Packages and macros go here
\usepackage{lipsum}
\usepackage{amsfonts}
\usepackage{epstopdf}
\ifpdf
  \DeclareGraphicsExtensions{.eps,.pdf,.png,.jpg}
\else
  \DeclareGraphicsExtensions{.eps}
\fi

% date 
\usepackage{datetime}
\newdateformat{monthyeardate}{%
  \monthname[\THEMONTH] \THEDAY, \THEYEAR}

% ORCID 
\usepackage{academicons}
\usepackage{xcolor}
\renewcommand{\orcid}[1]{\href{https://orcid.org/#1}{\textcolor[HTML]{A6CE39}{orcid.org/#1}}}

% math packages
\usepackage{amsmath}
\allowdisplaybreaks
\usepackage{amssymb}
\usepackage{commath}
\usepackage{mathtools}
\usepackage{bbm}

% figures
\usepackage{color}
\usepackage{graphicx}
\usepackage[small]{caption}
\usepackage{subcaption}

% Pseudocode, tables, and tikz 
\usepackage{relsize}
\usepackage{adjustbox}
\usepackage{algorithm}
\usepackage[noend]{algpseudocode}
\usepackage{booktabs}
\usepackage{tikz}

% For comment blocks 
\usepackage{verbatim}
% for strikethough
\usepackage{ulem}

% Itemize with squares 
\renewcommand{\labelitemi}{\tiny$\blacksquare$}

% Prevent itemized lists from running into the left margin inside theorems and proofs
\usepackage{enumitem}
\setlist[enumerate]{leftmargin=.5in}
\setlist[itemize]{leftmargin=.5in}

% Add a serial/Oxford comma by default.
\newcommand{\creflastconjunction}{, and~}

% Used for creating new theorem and remark environments 
\newsiamthm{problem}{Problem}
\newsiamremark{remark}{Remark}
\newsiamremark{hypothesis}{Hypothesis} 
\crefname{hypothesis}{Hypothesis}{Hypotheses}
\newsiamremark{example}{Example}
\newsiamthm{claim}{Claim}
\newsiamthm{conjecture}{Conjecture}

% Sets running headers as well as PDF title and authors
\headers{Oscillation free high-order FV}{ D. Hillebrand, S.-Ch. Klein,  and P. \"Offner}

% Title. If the supplement option is on, then "Supplementary Material"
% is automatically inserted before the title.
\title{Comparison to control oscillations in  high-order Finite Volume schemes via  physical constraint limiters, neural networks and polynomial annihilation\thanks{
\monthyeardate\today 
\corresponding{Philipp \"Offfner}\funding{This work was partially supported by the German Science Foundation (DFG) under Grant SO 363/15-1 (Hillebrand), Grant SO 363/14-1 (Klein) and the Gutenberg Research College, JGU Mainz (\"Offner). }
}}

 %Authors: full names plus addresses.
\author{
Dorian Hillebrand\thanks{Institute of Mathematics, Technical University Brunswick, Brunswick, Germany,\email{d.hillebrand@tu-braunschweig.de}, 
}
\and 
Simon-Christian Klein\thanks{Institute of Mathematics, Technical University Brunswick, Brunswick, Germany, (\email{simon-christian.klein@tu-braunschweig.de}, \orcid{0000-0002-8710-9089})}
\and 
Philipp \"Offner\thanks{Institute of Mathematics, Johannes Gutenberg University, Mainz, Germany, (\email{poeffner@uni-mainz.de}, \orcid{0000-0002-1367-1917})} 
}

%\usepackage{amsopn}

% commands 
\newcommand{\todo}[1]{{\Large \textcolor{red}{#1}}}
\newcommand{\revA}[1]{{\color{red}#1}}
\newcommand{\revB}[1]{{\color{blue}#1}}
\newcommand{\revC}[1]{{\color{cyan}#1}}
\newcommand{\JN}[1]{{\color{red}#1}}
\newcommand{\SK}[1]{{\color{blue}#1}}
\newcommand{\PO}[1]{{\color{cyan}#1}}


% definitions 
\newenvironment{eq}{\begin{equation}}{\end{equation}} 
\DeclareMathOperator{\rank}{rank}
\DeclareMathOperator{\diag}{diag}
\DeclareMathOperator*{\argmin}{arg\,min} 
\newcommand{\scp}[2]{\left\langle{#1, #2}\right\rangle} 
\newcommand{\Span}{\mathrm{span}}
\newcommand{\LS}{\mathrm{LS}}
\renewcommand{\dim}{\mathrm{dim} \,}
\newcommand{\vd}{\mathrm{d}}
\newcommand{\intd}{\, \mathrm{d}}
\newcommand{\N}{\mathbb{N}}
\newcommand{\Z}{\mathbb{Z}}
\newcommand{\R}{\mathbb{R}} 
\renewcommand{\epsilon}{\varepsilon}

\renewcommand{\div}{\operatorname{div}}
\newcommand{\bu}{\mathbf{u}}
\newcommand{\bx}{\mathbf{x}}
\newcommand{\fnum}{f^{\mathrm{num}}}
\newcommand{\fprec}{f^{n,\mathrm{precise}}}
\newcommand{\fn}{f^{ n}}
\newcommand{\fnn}{f^{n, \mathrm{num}}}
\newcommand{\fnnnn}{f^{n, \mathrm{neural}}}
\newcommand{\Fnum}{F^{\mathrm{num}}}
\newcommand{\Fnn}{F^n}
\newcommand{\uinit}{\mathcal{I}}
\newcommand{\meshT}{\mathcal{T}}
\newcommand{\nquad}{\mathrm{I}}


\newcommand{\of}[1]{\left (#1 \right)}
\newcommand{\derive}[2] {\frac{\partial {#1} }{\partial {#2}}}
\newcommand{\derd}[2]{\frac{\vd {#1}}{\vd {#2}}}

\DeclareMathOperator{\ch}{conv}
\DeclareMathOperator{\PR}{P}
\DeclareMathOperator{\ran}{ran}

%%% Local Variables: 
%%% mode:latex
%%% TeX-master: "ex_article"
%%% End: 
