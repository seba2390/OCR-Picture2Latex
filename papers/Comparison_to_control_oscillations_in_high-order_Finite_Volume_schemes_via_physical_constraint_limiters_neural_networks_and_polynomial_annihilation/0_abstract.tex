% 1. Why should people care? 
The construction of high-order structure-preserving numerical schemes to solve hyperbolic conservation laws has attracted a lot of attention in the last decades and various different ansatzes exist. 
% 2. What is the specific research question?
%One idea in this  context is the usage of a combination of an high-order scheme and a low order method with structure preserving properties. 
%The high-order methods are used in smooth regions where the structure-preserving methods are  applied in regions around shocks.
%To detect the problematic regions accurately is  essential since it controls, in turns, the scheme selection 
%and so the properties of the total underlying scheme. Various different ansatzes  can be found in the literature to mark the regions based e.g. on physical constraint limiters but  also lately 
%on machine learning techniques, and naturally  the questions rises what are the advantages or disadvantages of the individual techniques. 
% 3. What is done here? 
In this paper, we compare  three completely different approaches, i.e. physical constraint limiting, deep neural networks and the application of polynomial annihilation to 
construct high-order oscillation free Finite Volume (FV) blending schemes. We further analyze their analytical and numerical properties. 
% 4. What are the key findings? 
We demonstrate that all techniques can be used and yield highly efficient FV methods but also come with some additional drawbacks which we point out. 
% 5. What are the implications of these findings? 
Our investigation of the different blending strategies should lead to a better understanding of those techniques and can be transferred
 to other numerical methods as well which use similar ideas. 

