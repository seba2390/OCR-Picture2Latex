\section{Introduction} 
\label{sec:introduction} 

Hyperbolic conservation laws play a fundamental role within mathematical models for various physical processes, including fluid mechanics, electromagnetism and wave phenomena. However, since especially nonlinear conservation laws cannot be solved analytically, numerical methods have to be applied.
Starting already in 1950 with first-order finite difference methods (FD), the development has dramatically increased over the last decades including finite volume (FV) and finite element (FE) ansatzes \cite{richtmyer1994difference, du2016handbook, abgrall2017handbook}. To use modern computer power efficiently, high-order methods are nowadays constructed  which are used to obtain accurate solutions in a fast way. However, the drawback of high-order methods is that they suffer from stability issues, in particular after the developments of discontinuities which is a natural feature of hyperbolic conservation laws/balance laws. 
Here, first-order methods are favorable since their natural amount of high dissipation results in robust methods. In addition, many first-order methods have also the property that they preserve  other physical constraints like the positivity of density or pressure in the context of the Euler equations of gas dynamics.
In contrast, high-order approaches need additional techniques like positivity preserving limiters, etc. \cite{zhang2011maximum}.
Due to those reasons, researchers have combined low-order methods with high-order approaches to obtain schemes with favorable properties. The high-order accuracy of the method in smooth regions is kept, while also the excellent  stability conditions  and the  preservation of physical constraints   of the low order methods  near the discontinuities remain.  
Techniques in such context are e.g. Multi-dimensional Optimal Order Detection (MOOD) \cite{bacigaluppi2019posteriori, clain2011high}, subcell FV methods \cite{sonntag2014shock, hennemann2021provably} or limiting \cite{guermond2019invariant,kuzmin2020monolithic, kuzmin2021Limiter} strategies to name some.
In the last two approaches, free parameters are selected/determined which mark the problematic cells where the discontinuity may live. Here, the low order method is used whereas in the unmarked cells the high-order scheme still remains. 
To select those parameters, one uses either  shock sensors \cite{persson2006sub, offner2015zweidimensionale} or constraints on physical quantities  (entropy inequality, the positivity of density and pressure, etc.). 
As an alternative to those classical ansatzes, the application of machine learning (ML) techniques as shock sensors and to control oscillations have recently driven a lot of attention \cite{abgrall2020neural, beck2020neural, discacciati2020controlling, zeifang2021data}.  ML can be used for function approximation, classification and regression  \cite{Cybenko1989}. In this manuscript, we will extend those investigations in various ways. \\
In  \cite{klein2021using}, the author has proposed a simple blending scheme that combines a high-order entropy conservative numerical flux with the low-order Godunov-type flux in a convex combination. The approach is somehow  related to convex limiting. The convex parameter is selected by a
 predictor step automatically to enforce that the underlying method satisfies the Dafermos entropy condition numerically.  We focus on this scheme and extend the investigation from  \cite{klein2021using} in various ways.
 First, we propose a novel selection criteria not only based on Dafermos entropy criteria  \cite{dafermos1973entropy}  but rather on the preservation of other physical constraints, e.g. the positivity of density and pressure. As an alternative ansatz, we  further investigate the application of forward neural networks (NN) to specify the convex  parameter. As the last approach, we apply polynomial annihilation (PA) operators described in \cite{glaubitz2019high}. 
Our investigation of the different limiting strategies should lead to a better understanding of those techniques and can be transferred to 
alternative approaches based on similar ideas. Finally, all of our extensions will lead to highly efficient numerical methods for solving hyperbolic conservations laws.  The rest of the paper is organized as follows:\\
In  \cref{se_numerical_method}, we present the one-dimensional blending scheme from 
\cite{klein2021using}, introduce the notation and repeat its basic properties.  We further demonstrate that also a fully discrete entropy inequality will be satisfied locally under additional constraints on the blending parameter. In \cref{se_phys_cons}, we further specify the parameter selection not only taking the entropy condition into account but also other physical constraints. 
Here, we concentrate on the Euler equation of gas dynamics and demand the positivity of pressure and density as well. In \cref{se_neunet}, we repeat forward NN and how we apply them to determine the convex parameter in the extended blending scheme to obtain an oscillation free numerical scheme. In \cref{se_regul}, the polynomial annihilation  operators are finally 
explained and how they are  used in our framework to select the blending parameter.  In \cref{se_numerics}, we test all presented methods and limiting strategies and compare the results with each other. We discuss the advantages and disadvantages of all presented methods and give finally a summary with a conclusion.  


 
