\section{Numerical Experiments}\label{se_numerics}

<<<<<<< HEAD
\PO{We should specify the test we are considering here... Euler equations with the shu osher modelll
}

%\subsection{Neural Network based Scheme}
	\subsubsection*{Calculation of Training Data}
=======
	\subsection{Calculation of training Data}
>>>>>>> c4ac145f40e5bd7908dd68b7d1c882073d73414e
		As written before one starts by calculation of suitable sets of training data. We used initial data
	\[
		\begin{pmatrix}\tilde \rho_0(x, t)\\ \tilde v_0(x, t) \\ \tilde p_0(x, t) \end{pmatrix} = \sum_{k=1}^{N} a_k \sin(2\pi k x) + b_k\cos(2 \pi k x)
	\]
	with $N = 100$ and the vectors $a, b \in \R^{3 \times N}$ were selected by first selecting random $\tilde a, \tilde b \in [0, 1]^{3 \times N}$ and afterwards scaling them by
	\[
		\hat a_k = \frac {\tilde a_k}{k^{c}}  \quad \hat b_k = \frac{\tilde b_k}{k^c}.
	\]
<<<<<<< HEAD
	This option is based on the fact that the Fourier coefficients $a_k, b_k$ function $f \in C^l$ satisfies $a_k, b_k \in \mathcal O(1/k^l)$. We used therefore $c =2.5$ to generate random function with high regularity. \todo{Explain pressure/density regulatization}
	The generated initial data was afterwards solved by an second order MUSCL scheme (\PO{Or our high-order scheme with more grids)} using the local Lax- Friedrichs flux and the SSPRK(3,3) method in time on a fine grid of $4000$ points \todo{Give exact numbers after tests on gene were completed with higher accuracy}.
	\subsubsection*{Training of the Network}
\PO{What is here missing}
	\subsubsection*{Testing of the Schemes}
=======
	This scaling is based on the fact that the fourier coefficients $a_k, b_k$ function $f \in C^l$ satisfie $a_k, b_k \in \mathcal O(1/k^l)$ and we used therefore $c =2.0$ to generate random functions with suitable regularity. As we would like to ensure positive pressure and density with a suitable amplitude to give a representative picture of the dynamics of the system but without a unsuitably small the following modifications were done 
	\[
		\begin{aligned}
			a_{1, k} &= (0.2 + A_\rho)\hat a_{1, k}, \quad \hat  a_{2, k} &= A_v a_{2, k}, \quad \hat a_{3, k} &= (0.3 + A_p) \\
			\rho_0(x, t) &= \tilde \rho_0(x, t) - \min_x(0, \tilde \rho_0(x, t)), \quad v_0(x, t) &= \tilde v_0(x, t) + B_v, \quad p_0(x, t) &= \tilde p_0(x, t) - \min_x(p_0(x, t), 0)
		\end{aligned}
	\]
	to ensure positivity. The values $A_\rho, A_v, B_v, A_p$ were randomly selected to satisfy $A_\rho \in [0, 2], A_v \in [0, 2], B_v \in [-2, 2], A_p \in [0, 4]$.  
	The generated initial data was afterwards solved by an second order MUSCL scheme using the local lax friedrichs flux and the SSPRK(3,3) method in time on a fine Grid of $4000$ points 
	\subsection{Layout and Training of the Network}
	\begin{table}
		\begin{tabular}{ccccccc}
			Layer 				& input & 2 	& 3 	& 4 	& 5 	& output\\
			Activation 			& elu 	& elu 	& elu 	& elu 	& elu 	& $x \to x$\\
			Number of Neurons 	& 40 	& 80 	& 80 	& 80 	& 80 	& 1\\
		\end{tabular}
	\caption{Used networkstructure}
	\label{tab:net}
	\end{table}
	We used a Neuonral Network built out of 6 Layers whose dimensions are given in table \ref{tab:net}. In all but the last layer the $elu$ activation function was used. As inputs were the values of the conserved variables and the pressure 5 cells left and right of the cell boundary where $\alpha$ is to be determined used.
	Our network for the prediction of $\alpha$ was trained using the ADAM optimizer \cite{kingma2017adam} with parameters scheduled as given by table \ref{tab:sched}.
	\begin{table}
		\begin{tabular}{cc cc cc cc}
			Section 	& 1 	& 2 	& 3 	& 4 	& 5 	& 6		& 7 \\
			Epochs 		& 25 	& 25 	& 25 	& 25 	& 25 	& 25 	& 25\\
			Batchsize 	& 32 	& 256 	& 1024 	& 4096 	& 4096 	& 4096  & 4096\\
			Stepsize  	&0.001 	& 0.001 & 0.001	& 0.001	& 0.0001& 0.00001& 0.000001\\
		\end{tabular}
		\caption{overview of used training parameters}
		\label{tab:sched}
	\end{table}
	\begin{figure}
	\includegraphics{pics/trainloss}
		\caption{Training loss of the neuonal network.}
		\label{fig:loss}
		\end{figure}
	The resulting losscurve is given in figure \ref{fig:loss} training took about 20 minutes on 8 Cores of a AMD Ryzen Threadripper 5900X at 3.7 Ghz. 
	\subsection{Testing of the Schemes}
>>>>>>> c4ac145f40e5bd7908dd68b7d1c882073d73414e
	\begin{figure}
		\begin{subfigure}{0.48\textwidth}
			\includegraphics[width=\textwidth]{pics/shuosh6de400}
		\end{subfigure}
		\begin{subfigure}{0.48\textwidth}
			\includegraphics[width=\textwidth]{pics/shuosh6pp400}
		\end{subfigure}
		\begin{subfigure}{0.48\textwidth}
			\includegraphics[width=\textwidth]{pics/shuosh6dd400}
		\end{subfigure}
		\begin{subfigure}{0.48\textwidth}
			\includegraphics[width=\textwidth]{pics/shuosh6hoes400}
		\end{subfigure}
		\begin{subfigure}{0.48\textwidth}
			\includegraphics[width=\textwidth]{pics/shuosh6ddEntro400}
		\end{subfigure}
		\begin{subfigure}{0.48\textwidth}
			\includegraphics[width=\textwidth]{pics/shuosh6hoesEntro400}
		\end{subfigure}
		\caption{Shu-Osher Test 6 at $T = 1.8$}
		\label{fig:SO6}
	\end{figure}
	Our numerical tests were carried out to answer the three questions
	\begin{itemize}
		\item Are the schemes able to capture shocks and is oscillation free?
		\item Does the discrete entropy inequality \ref{eq:eie} also hold for the Hoes and Data driven selection of $\alpha$? \PO{Actually, for all tests or not?}
		\item Which order of convergence can be expected from our different schemes. 
	\end{itemize}
	We will investigate four different schemes with various combinations.	\begin{enumerate}
<<<<<<< HEAD
		\item The provably discretely locally entropy stable scheme with $\alpha$ dictated by \emph{Condition F} using the a fourth order entropy conservative flux with SSPRK(3, 3) time integration and the local Lax-Friedrichs flux (LLF) with forward Euler  method.
		\item A provably positivity preserving scheme with $\alpha = \max(\alpha_\rho, \alpha_p)$ determined by Condition $\rho$ and Condition $P$. This scheme is using the fourth order entropy conservative flux and LLF flux with uniform SSPRK(3, 3) time Integration.
		\item A data driven scheme using $\alpha$ determined using the neuronal net described before. The high -order flux is again fourth order entropy conservative  combined with SSPRK33 quadrature while the low order entropy stable flux is given by the  LLF scheme using the SSPRK22 quadrature in time.
		\item A polynomial annihilation based scheme where $\alpha$ is determined as described before and the high-order flux is the fourth order entropy conservative flux with SSPRK33 quadrature while the low order entropy stable flux is given by the Lax-Friedrichs scheme using the SSPRK22 quadrature in time. The PA operators used are of fourth order.
=======
		\item The provably discreteley locally entropy stable scheme with $\alpha$ dictated by \emph{Condition F} using the a second order entropy conservative flux with SSPRK(3, 3) time integration and the Local-Lax-Friedrichs flux with forward euler steps.
		\item A provably positivity preserving scheme with $\alpha = \max(\alpha_\rho, \alpha_p)$ determined by Condition $\rho$ and Condition $P$. This scheme is using the 4-th order entropy stable flux and LLF flux with uniform SSPRK(3, 3) time Integration.
		\item A data driven scheme using $\alpha$ determined using the neuronal net described before. The High Order flux is the 4-th order entropy stable flux with SSPRK33 quadrature while the low order entropy stable flux is given by the Lax-Friedrichs scheme using the SSPRK22 quadrature in time.
		\item A Polynomial Annealing based scheme were $\alpha$ is determined as described before and the High order flux is the 4-th order entropy stable flux with SSPRK33 quadrature while the low order entropy stable flux is given by the Lax-Friedrichs scheme using the SSPRK22 quadrature in time. The PA operators used are of fourth order.
>>>>>>> c4ac145f40e5bd7908dd68b7d1c882073d73414e
	\end{enumerate}
Calculations for the Shu-Osher Problem number 6 \cite{SO1988, SO1989} given by the following initial condition 
	\begin{align*}
		\rho_0(x, 0) = \begin{cases}3.857153  \\ 1 + \epsilon \sin(5 x)  \end{cases} 
		\quad 
		v_0(x, 0) = \begin{cases} 2.629  \\ 0  \end{cases}
		p_0(x, 0) = \begin{cases} 10.333 & x < 1 \\ 1 & x \geq 1 \end{cases}
	\end{align*}
<<<<<<< HEAD
	for the Euler equations of gas dynamics \ref{eq:Euler} were carried out on the domain $\Omega = [0, 10]$. The parameter $\epsilon = 0.2$ was set to the canonical value of $0.2$ and the adiabatic exponent was set to $\gamma=\frac 7 5$ for an ideal gas. The results are depicted in \ref{fig:SO6} and show overall a good performance of all but the only positivity preserving scheme. While the results using \textbf{Condition $\rho$} look meaningless. However, the positivity of the solution is still ensured and the calculation could be carried on up to $t = 1.8$. The other three schemes are able to resolve the strong shock without unnecessary oscillations. The amount of points needed for the transition is small and the wave structure trailing the shock is resolved at an acceptable level. The high gradient areas at $x=2.5, 3.5, 4.5$ lead so some oscillations for the data driven and PA based schemes but are not troublesome. Figure \ref{fig:deie} shows the discrete entropy productions 
	\[
		\frac{U(u^{n+1}_k) - U(u^n_k)}{\Delta t} + \frac{F^n_{\alpha_{k + \frac 1 2}} - F^n_{\alpha_{k - \frac 1 2}}}{\Delta x}
	\]
	for the data driven and PA scheme using the entropy flux from theorem  \ref{thm:deie} in conjunction with the $\alpha$ from these schemes, underlining that these schemes seem to be entropy stable as the entropy inequality is fulfilled by them. The small oscillations still visible underline the view of the author given already in \cite{klein2021using} that the fulfillment of the entropy inequality is not enough to guarantee oscillation free solutions whereas the Dafermos criterion could be a key feature in this case. 
=======
	for the Euler equations of gas dynamics \ref{eq:Euler} were carried out on the Domain $\Omega = [0, 10]$. The parameter $\epsilon = 0.2$ was set to the canoncial value of $0.2$ and the adiabatic exponent was set to $\gamma=\frac 7 5$ for an ideal gas. The results are depicted in \ref{fig:SO6} and show overall a good performance of all but the only positivity preserving scheme. While the results for the only positivity preserving scheme look desastruous the positivity of the solution is still given and the calculation could be carried on up to $t = 1.8$. The other 3 schemes show are able to resolve the strong shock without unnecessary oscillations. The amount of points needed for the transition is small and the wave structure trailing the shock is resolved at an acceptable level. The high gradient areas at $x=2.5, 3.5, 4.5$ lead so some oscillations for the Datadriven and PA based schemes but are not troublesome. Figure \ref{fig:SO6} shows the discrete entropy productions 
	\[
		\frac{U(u^{n+1}_k) - U(u^n_k)}{\Delta t} + \frac{F^n_{\alpha_{k + \frac 1 2}} - F^n_{\alpha_{k - \frac 1 2}}}{\Delta x}
	\]
	for the data driven and PA scheme using the entropy flux from theorem  \ref{thm:deie} in conjunctin with the $\alpha$ from these schemes, underlining that these schemes seem to be entropy stable as the entropy inequality is fullfilled by them at least in this experiment. The small oscillations still visible underline the view of the author given already in \cite{klein2021using} that the fullfillment of the entropy inequality is not enough to guarantee oscillation free solutions, whereas the Dafermos criterion could be a key ingredient in this case. 
>>>>>>> c4ac145f40e5bd7908dd68b7d1c882073d73414e
	\begin{figure}
		\includegraphics[width=\textwidth]{pics/convana}
		\caption{Convergence analysis of the schemes.}
		\label{fig:CA}
	\end{figure}
	The next test carried out is the smooth transport of a density variation under pressure equilibrium already used in \cite{klein2021using} to determine experimental orders of convergence
	\[
	\rho_0(x, 0) = 3.857153 + \epsilon \sin(2 x)  \quad v_0(x, 0) =  2.0 \quad p_0(x, 0) = 10.33333.
	\]
<<<<<<< HEAD
	The $\L1$ errors of the schemes is depicted in Figure \ref{fig:CA}. One clearly sees that the positivity preserving scheme, being unsuitable for the discontinuous testcase, is in fact the most accurate scheme in this case and converges with third order. The HOES and data driven schemes also converge with nearly third order while the provably entropy stable scheme only converges with superlinear but not second order convergence rate. In case of the data driven and polynomial annealing bases schemes the only third but not fourth order of convergence could be based on the fact that only a third order accurate time integration was used.
	\PO{KÖnnen wir umschreiben und ich würde auf jedenfall positivity scheme aus der Konvergenz rausnehmen...}
=======
	The $\L1$ errors of the schemes is depicted in figure \ref{fig:CA}. The reference was calculated by the ENO2 method \cite {ENOIII} on a grid with $2^{14}$ cells. Please note that we calculated for a given grid $T_N$ with $2^N$ cells the errors between the mean values that were calulcated by our schemes and mean values of the reference solution over the cells of the $T_N$ grid. The values $\tilde u_k$ of the reference solution calculated downwards to a grid with $2^{N-1}$cells are therefore calculated from the reference solution $u_k$ with $2^N$ cells by $\tilde u_k = 0.5(u_{2k} + u_{2k-1})$. One clearly sees that the Positivity preserving scheme, beeing unsuitable for the disconinuous testcase, is in fact the mos accurate scheme in this case and converges with third order. The HOES and data driven schemes also converge with nearly third order while the provably entropy stable scheme only converges with second order convergence rate. Higher orders of convergence by using higher order time integration and entropy conservative fluxes could not be demonstrated for scheme based on condition $F$, indicating that condition $F$ gives to big values for $\alpha$ to reach third order convergence. In case of the data driven and polynomial annealing bases schemes the only third but not fourth order of convergence could be based on the fact that only a third order accurate time integration was used.
	
>>>>>>> c4ac145f40e5bd7908dd68b7d1c882073d73414e
	
	