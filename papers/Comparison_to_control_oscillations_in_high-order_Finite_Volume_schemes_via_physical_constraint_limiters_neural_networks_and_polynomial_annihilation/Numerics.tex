\section{Numerical Experiments}\label{se_numerics}
In the following part, we investigate and compare the approaches described before in our blending schemes \ref{def_Blending}.  Especially, we
focus on the following questions: 
\begin{itemize}
	\item Are all schemes  able to capture shocks and are they oscillation free?
	\item Does the discrete entropy inequality \ref{eq:eie} also hold for the PA and data-driven selection of $\alpha$? Are the schemes violating other physical constraints like the positivity of density and pressure?
	\item Which order of convergence can we expect from our schemes?
\end{itemize}
We test our schemes on the Euler equations of gas dynamics  \eqref{eq:Euler}  and use the mathematical entropy   \eqref{eq:EulerPE}. To test the high-order accuracy of the proposed schemes, we consider a smooth connected density variation from  \cite{klein2021using}. To test and compare our schemes for problems incorporating
strong shocks, we focus on the famous Shu-Osher problem (Test Number 6 from  \cite{SO1988, SO1989}).
The Shu-Osher test case is nowadays working as  a benchmark problem in computational fluid dynamics. 
We compare four different schemes with each other and also tested the Dafermos criterion based scheme from \cite{klein2021using}. We consider in detail: %
%Tests of the Shu-Osher Test number 6 \cite{SO1988, SO1989} and a convected density variation  were used as we can thereby study the behavior of the schemes for problems incooperating strong shocks as well as smooth structures. The second problem is also well suited to determine experimental orders of convergence. In total four different schemes were tested.
\begin{enumerate}
	\item The discrete local entropy stable scheme with $\alpha$ dictated by \textbf{Condition F} using a second-order entropy conservative flux with SSPRK(2, 2) time integration and the local Lax-Friedrichs flux (LLF) with forward Euler method. We denote the scheme in the following with DELFT.
	\item A positivity preserving scheme with $\alpha = \max(\alpha_\rho, \alpha_p)$ determined by \textbf{Condition $\rho$} and \textbf{Condition $P$}. This scheme is using the fourth-order entropy conservative flux and  the LLF flux with SSPRK(3, 3) time integration. The scheme is called PPLFT. 
	\item A data-driven scheme, $\alpha$ is determined using  neuronal nets. The high-order flux is again fourth-order entropy conservative combined with an SSPRK(3,3) while the low order entropy stable flux is given by the  LLF flux together with an SSPRK(2,2) method in time.
The scheme is denoted with DDLFT. 
	\item A polynomial annihilation based scheme where $\alpha$ is determined through the technique described in \cref{se_regul}. The high-order flux is the fourth-order entropy conservative flux with SSPRK(3,3) quadrature while the low order entropy stable flux is given by the Lax-Friedrichs scheme using the SSPRK(2,2) quadrature in time. The PA operators used are of fourth-order. We denote the scheme with PALFT.
	\item The Dafermos criterion based method developed and proposed in \cite{klein2021using}.
\end{enumerate}
The different decisions concerning the fluxes and their time quadrature are rooted in the following observations. The stencil of the fourth-order in space, SSPRK(3,3) in time flux is 12  cells wide, whereas the second-order in space and second-order in time flux has only  4 points wide stencils. Therefore, a discontinuity results in bigger values for $\alpha$ using \textbf{Condition $F$} if the base stencil is wider because even a distant discontinuity impacts the selection of $\alpha$. As a small $\alpha$ implies a more accurate method, we therefore use the smaller stencils for the provably discretely entropy dissipative method.
To avoid the decrease in accuracy for the high-order methods even in smooth regions, one can use subcell techniques \cite{Harten1989ENO, rueda2022subcell} instead.  \\
The PPLFT method uses uniform SSPRK time integration on the other hand as we expect no gains from splitting the time integration in this case. The recalculation of $\alpha$ was therefore carried out in every sub step of the RK method for PPLFT method, whereas it was carried out only once in every time step for the last two methods (PALFT and DDLFT). The splitting of the time integration therefore allows us to make use of a speed improvement by only calling the neuronal network once every time step. A second reason to use the quadrature based definition of high-order time integration lies in the fact that the $\alpha$ used as training data for the network is the least-squares solution described in section \ref{se_neunet}. A more accurate flux allows the value of $\alpha$ to be shifted towards zero in smooth regions in the training, therefore lowering the numerical dissipation. The same combination of fluxes are also used for the last scheme for comparison reasons. Before comparing our schemes, we explain how we generate our training data for DDLFT.
	\subsection{Calculation of Training Data}
		We used initial data
	\[
		\begin{pmatrix}\tilde \rho_0(x, t)\\ \tilde v_0(x, t) \\ \tilde p_0(x, t) \end{pmatrix} = \sum_{k=1}^{N} a_k \sin(2\pi k x) + b_k\cos(2 \pi k x)
	\]
	with $N = 100$ and the vectors $a, b \in \R^{3 \times N}$ were selected by selecting random $\tilde a, \tilde b \in [0, 1]^{3 \times N}$ and afterwards scaling them by $
		\hat a_k = \frac {\tilde a_k}{k^{c}}$ and $ \hat b_k = \frac{\tilde b_k}{k^c}.$
	This scaling is based on the fact that the Fourier coefficients $a_k, b_k$ of functions $f \in C^l$ satisfy $a_k, b_k \in \mathcal O(1/k^l)$ and we used therefore $c =2.0$ to generate random functions with suitable regularity. We would like to ensure positive pressure and density with a suitable amplitude to give a representative picture of the dynamics of the system but without an unsuitably small CFL restriction on the time step. Therefore, the following modifications were done 
	\[
		\begin{aligned}
			 a_{1, k} &= (0.2 + A_\rho)\hat a_{1, k}, \quad  \hat  a_{2, k} = A_v a_{2, k},\quad  \hat a_{3, k} = (0.3 + A_p), \\
			\rho_0(x, t) & = \tilde \rho_0(x, t) - \min_x(0, \tilde \rho_0(x, t)), \quad  v_0(x, t) = \tilde v_0(x, t) + B_v, \\  p_0(x, t)  &= \tilde p_0(x, t) - \min_x(p_0(x, t), 0)
		\end{aligned}
	\]
	to ensure positivity. The values $A_\rho, A_v, B_v, A_p$ were randomly selected to satisfy $A_\rho \in [0, 2], A_v \in [0, 2], B_v \in [-2, 2], A_p \in [0, 4]$.  
	The generated initial data was afterwards solved  by a second-order MUSCL scheme and SSPRK(3,3)  with $4000$ cells. The MUSCL scheme was selected because it is a highly robust method. In total $32$ different initial conditions were generated, solved up to $t = 1$ and sampled at every cell interface of the coarse grid at $100$ equally spaced times in the available interval.
	\subsubsection*{Layout and Training of the Network}
	\begin{table}
		\begin{tabular}{ccccccc}
			Layer 				& input & 2 	& 3 	& 4 	& 5 	& output\\
			Activation 			& ELU 	& ELU 	& ELU 	& ELU 	& ELU 	& $x \to x$\\
			Number of Neurons 	& 40 	& 80 	& 80 	& 80 	& 80 	& 1\\
		\end{tabular}
	\caption{Used network structure}
	\label{tab:net}
	\end{table}
	We use a neural  network built out of six layers whose dimensions are given in table \ref{tab:net}. In all, but the last layer, the $ELU$ activation function is applied. The inputs are  the values of the conserved variables and the pressure of five cells left and right to the cell boundary where $\alpha$ has to be determined.
	Our network for the prediction of $\alpha$ was trained using the ADAM optimizer \cite{kingma2017adam} with parameters scheduled as given in  \cref{tab:sched}.
		The resulting loss curve is printed in \cref{fig:loss}. The training took circa 20 minutes on 8 cores of a AMD Ryzen Threadripper 5900X at 3.7 Ghz. 
	\begin{table}
		\begin{center}
		\begin{tabular}{cc cc cc cc}
			Section 	& 1 	& 2 	& 3 	& 4 	& 5 	& 6		& 7 \\
			Epochs 		& 25 	& 25 	& 25 	& 25 	& 25 	& 25 	& 25\\
			Batchsize 	& 32 	& 256 	& 1024 	& 4096 	& 4096 	& 4096  & 4096\\
			Stepsize  	&0.001 	& 0.001 & 0.001	& 0.001	& 0.0001& 0.00001& 0.000001\\
		\end{tabular}
		\caption{Overview of used training parameters}
		\label{tab:sched}
		\end{center}
	\end{table}	
	\begin{figure}
	\begin{center}
	\includegraphics[width=0.5\textwidth]{pics/trainloss}
		\caption{Training loss of the neuronal network.}
		\label{fig:loss}
		\end{center}
		\end{figure}
	\subsection{Testing of the Schemes}
	\begin{figure}
		\centering
		\begin{subfigure}{0.48\textwidth}
			\includegraphics[width=\textwidth]{pics/shuosh6de400}
		\end{subfigure}
		\begin{subfigure}{0.48\textwidth}
			\includegraphics[width=\textwidth]{pics/shuosh6pp400}
		\end{subfigure}
		\begin{subfigure}{0.48\textwidth}
			\includegraphics[width=\textwidth]{pics/shuosh6dd400}
		\end{subfigure}
		\begin{subfigure}{0.48\textwidth}
			\includegraphics[width=\textwidth]{pics/shuosh6hoes400}
		\end{subfigure}
		\begin{subfigure}{0.48\textwidth}
			\includegraphics[width=\textwidth]{pics/shuoshrho400}
			\subcaption{Dafermos Criterion scheme from \cite{klein2021using}}
		\end{subfigure}
%		\begin{subfigure}{0.48\textwidth}
%			\includegraphics[width=\textwidth]{pics/shuosh6hoesEntro400}
%		\end{subfigure}
		\caption{Density profile  $T = 1.8$}
		\label{fig:SO6}
	\end{figure}
The initial conditions of the Shu-Osther test are given by
	\begin{align*}
		\rho_0(x, 0) = \begin{cases}3.857153  \\ 1 + \epsilon \sin(5 x)  \end{cases} 
		\quad 
		v_0(x, 0) = \begin{cases} 2.629  \\ 0  \end{cases}
		p_0(x, 0) = \begin{cases} 10.333 & x < 1 \\ 1 & x \geq 1 \end{cases}
	\end{align*}
	in the domain $\Omega = [0, 10]$. The parameter $\epsilon = 0.2$ was set to the canonical  value of $0.2$ and the adiabatic exponent was set to $\gamma=\frac 7 5$ for an ideal gas. The density profiles  are printed in \cref{fig:SO6}. The numerical solutions are describing in nearly all cases the reference solutions   except for the positivity preserving scheme (PPLFT). Here, the calculated solution is obviously meaningless even if the positivity of the solutions is still ensured and the calculation could be carried on up to $T= 1.8$. The other three schemes  are able to resolve the strong shock without nonphysical oscillations. The amount of points needed for the transition is small and the wave structure trailing the shock is resolved accurate. The high gradient areas at $x=2.5, 3.5, 4.5$ results in small oscillations for the data-driven scheme   and PA based scheme. However, these oscillations are nearly not visible, especially for PALFT.  The dotted lines give also the $\alpha$ coefficients in the convex combination of our blending schemes. We  realize that for PALFT 
the $\alpha$s distinguish essentially from zero around the shock where for $DDLFT$ the lower-order method is also activated in smooth regions (i.e. $\alpha>0$).
	
	\begin{figure}
		\begin{subfigure}{0.32\textwidth}
			\includegraphics[width=\textwidth]{pics/shuosh6dd6Entro400}
			\caption{DDLFT}
		\end{subfigure}
		\begin{subfigure}{0.32\textwidth}
			\includegraphics[width=\textwidth]{pics/shuosh6hoes6Entro400}
			\caption{PALFT}
		\end{subfigure}
		\begin{subfigure}{0.32\textwidth}
			\includegraphics[width=\textwidth]{pics/shuosh6de6Entro400}
			\caption{DELFT}
		\end{subfigure}
		\begin{subfigure}{0.32\textwidth}
			\includegraphics[width=\textwidth]{pics/shuosh6dd12Entro400}
			\caption{DDLFT}
		\end{subfigure}
		\begin{subfigure}{0.32\textwidth}
			\includegraphics[width=\textwidth]{pics/shuosh6hoes12Entro400}
			\caption{PALFT}
		\end{subfigure}
		\begin{subfigure}{0.32\textwidth}
			\includegraphics[width=\textwidth]{pics/shuosh6de12Entro400}
			\caption{DELFT}
		\end{subfigure}
		\begin{subfigure}{0.32\textwidth}
			\includegraphics[width=\textwidth]{pics/shuosh6dd18Entro400}
			\caption{DDLFT}
		\end{subfigure}
		\begin{subfigure}{0.32\textwidth}
			\includegraphics[width=\textwidth]{pics/shuosh6hoes18Entro400}
			\caption{PALFT}
		\end{subfigure}
		\begin{subfigure}{0.32\textwidth}
			\includegraphics[width=\textwidth]{pics/shuosh6de18Entro400}
			\caption{DELFT}
		\end{subfigure}
		
		\caption{Entropy production at $T = 0.6$ (first row), $T = 1.2$ (second row), $T = 1.8$ (third row)}
		\label{fig:SO7}
	\end{figure}	
	
	
	 \cref{fig:SO7} shows the discrete entropy productions over the cells for the schemes 
	  DDLFT (left), PALFT(central) and DELFT(right) as snapshots at $T \in \{0.6, 1.2, 1.8\}$. %	\[
%		\frac{U(u^{n+1}_k) - U(u^n_k)}{\Delta t} + \frac{F^n_{\alpha_{k + \frac 1 2}} - F^n_{\alpha_{k - \frac 1 2}}}{\Delta x} \PO{Das ist keine Produktion}
%	\]
%	for the data-driven and PA scheme using the entropy flux from theorem  \ref{thm:deie} in conjunction with the $\alpha$ from these schemes,
	As printed, the schemes fulfill also locally the entropy inequality and are entropy dissipative (at least for this experiment). %  be entropy dissipative  as the entropy inequality is fulfilled by them at least in this experiment. 
The small oscillations inside the numerical solution may be further cancelled out using additionally the Dafermos criterion \cref{eq_Dafermos} as mentioned also in \cite{klein2021using}.
Finally, we stress out that we have in all of our  simulations  no violations of positivity of density and pressure recognized. 
%that the  fulfillment  of the entropy inequality is not enough to guarantee oscillation free solutions, whereas the Dafermos criterion could be a key ingredient in this case. 
	\begin{figure}
	\begin{center}
		\includegraphics[width=0.7\textwidth]{pics/convana}
		\caption{Convergence analysis of the schemes.}
		\label{fig:CA}
		\end{center}
	\end{figure}
	
	To determine the experimental order of convergence, we simulate the smooth transport of a density variation under pressure equilibrium already used in \cite{klein2021using}:
	\[
	\rho_0(x, 0) = 3.857153 + \epsilon \sin(2 x),  \quad v_0(x, 0) =  2.0, \quad p_0(x, 0) = 10.33333.
	\]
	The $\mathrm{L}^1$-errors of the schemes are shown in  \cref{fig:CA}. The reference was calculated by the ENO2 method  on a grid with $2^{14}$ cells. Please note that we calculated for a given grid $T_N$ with $2^N$ cells  the errors between the mean values of our approximated solution and the mean values of the reference solution using $2^{14}$ cells. 
	The values $\tilde u_k$ of a solution with $2^N$ cells is scaled  downwards to a grid with $2^{N-1}$ cells by the following procedure. We use the solution $\tilde u_k$ with $2^N$ cells and apply $ u_k = \frac{\tilde u_{2k} + \tilde u_{2k-1}}{2}$ to find the solution $u_k$ on the grid with $2^{N-1}$ cells. This procedure can be applied several times to find consistent mean values for any grid having a power of two cells. 
	The PA and data-driven schemes converge with third-order while the provable entropy stable scheme only converges with second-order convergence rate. Higher orders of convergence by using higher-order time integration and entropy conservative fluxes could not be demonstrated for schemes based on \textbf{Condition $F$} since  \textbf{Condition $F$} gives to big values for $\alpha$ to reach third-order convergence. The reason for this is that condition $F$ is only a sufficient condition for a satisfied fully discrete entropy inequality. A fully discrete entropy inequality can be also satisfied by smaller values of $\alpha$ as could be seen in \cref{fig:SO7}. A deeper analysis of sharper lower bounds on $\alpha$ will be part of future publications considering also subcell techniques. We further see a slide decrease of order for the DDLFT for fine grids. This is due to the fact that the training  data and the neural nets can not keep up with the DDLFT scheme itself on fine grids. 
	Finally,  we mark that third-order accuracy is only reached due to the time-integration method. 
%	 In case of the data-driven and polynomial annealing bases schemes the only third but not fourth-order of convergence could be based on the fact that only a third-order accurate time integration was used.

	
	