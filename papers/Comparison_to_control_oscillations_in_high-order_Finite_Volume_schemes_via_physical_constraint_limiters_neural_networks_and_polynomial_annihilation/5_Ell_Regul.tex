\section{Polynomial Annihilation Based Scheme} \label{se_regul}

In this chapter, we want to propose another possibility to approximate blending parameter $\alpha$. Therefore, the construction of polynomial annihilation operators in one spatial dimension is visited at first. Since these operators approximate the jump function of a given sensing variable, we use them in a second step to propose a choice of $\alpha$.

\subsection{Polynomial Annihilation-Basic Framework}

The general idea of polynomial annihilation (PA) operators proposed in \cite{archibald2005polynomial} is to approximate the jump function
\begin{equation}
 [s](x) = s(x^+)-s(x^-)
\end{equation}
for a given $s:\Omega\rightarrow\mathbb{R}$ referred to as the sensing variable. Therefore, we want to construct an operator $L_m[s](\xi)$ which gives an approximation of $[s](\xi)$ of $m$-th order. 

For a given $\xi\in\Omega$, we first choose a stencil of $m+1$ grid points around $\xi$
\begin{equation}
 S_\xi = (x_k,\ldots, x_{k+m})\hspace{1em}\text{with}\hspace{1em} x_k\leq \xi\leq x_{k+m}.
\end{equation}
The name giving polynomial annihilation is performed in the next step by defining the annihilation coefficients $c_j$ implicitly by
\begin{equation}
 \sum_{x_j\in S_\xi} c_j p_l(x_j) = p_l^{(m)}(\xi)
\end{equation}
where $\{p_l\}_{l=0}^m$ is any basis of the space of all polynomials with degree $\leq m$. Note that the coefficients $c_j$ only depend on the choice of the stencil $S_\xi$ since the $m$-th derivative $p_l^{(m)}$ of a polynomial with degree $\leq m$ is constant. Expanding this approach to higher spatial dimensions would cause the $c_j$ to directly depend on $\xi$.
Finally, we also need a normalization factor $q_m$ calculated by
\begin{equation}
 q_m = \sum_{x_j\in S_\xi^+} c_j,
\end{equation}
where $S_\xi^+ = \{x_j\in S_\xi| x_j\geq\xi\}$. The normalization factor $q_m$ is constant for fixed choice of $S_\xi$. We define the PA operator of order $m$ now by
\begin{equation}
 L_m[s](\xi) = \frac{1}{q_m} \sum_{x_j\in S_\xi} c_j s(x_j).
\end{equation}
In \cite{archibald2005polynomial} it was shown that
\begin{equation}
 L_m[s](\xi) =\begin{cases}
 			   [s](\tilde{x}) + \mathcal{O}\left(\tilde{h}(\xi) \right), &\text{if } x_{j-1}\leq\xi,\tilde{x}\leq x_j,\\
 			   \mathcal{O}\left( (\tilde{h}(\xi))^{\min(m,l) }\right) , & \text{if } s\in C^l([x_k,x_{k+m}]).
 			  \end{cases}
\end{equation}
Here, $\tilde{x}$ denotes a jump discontinuity of $s$ and $\tilde{h}(\xi):=\max\{|x_i-x_{i-1}|~|~x_i,x_{i-1}\in S_\xi \}$.

\subsection{Scheme based on polynomial annihilation}

Based on the above-presented framework, we now construct the convex parameter $\alpha$. Since entropy could be dissipated in cells where discontinuities occur, the parameter is supposed to be $0$ in regions where the solution is smooth and $1$ in discontinuity containing cells. This can be achieved by PA operators which are not constructed to give the location of a discontinuity but to approximate the height of the jump at that location. Hence, we need to normalize the operator by a factor approximating the height of a typical jump, i.e. $\frac{1}{z}L_{2p}[s]$ with a normalization factor $ z\approx[s](\tilde{x})$. Here, the PA operator is used on the four-point stencil $(x_{k-p+1},\ldots,x_{k+p})$ and the corresponding mean values $(u_{k-p+1}^n,\ldots,u_{k+p}^n)$ for  a given $n$. 
This normalization factor $z$ is also provided by a PA operator. Therefore, we apply $L_{2p}$ to the idealized values 
\begin{equation*}
 (u_{\max}^n,\ldots,u_{\max}^n,u_{\min}^n,\ldots,u_{\min}^n)
\end{equation*}
based on the same four-point stencil where 
\begin{equation*}
 u_{\max}^n = \max\{u_{k-p+1}^n,\ldots,u_{k+p}^n\},\hspace{1em} u_{\min}^n = \min\{u_{k-p+1}^n,\ldots,u_{k+p}^n\}.
\end{equation*}
With this normalization factor, the natural selection of $\alpha$ would be
\begin{equation*}
 \alpha = \frac{L_{2p}[u]}{z}.
\end{equation*}
This choice does not fulfill the before mentioned recommended property since the normalization gives a much more accurate approximation of the jump height. Actually, the jump function is approximated flatter in case of using $L_{2p}[u]$ on $(u_{k-p+1}^n,\ldots,u_{k+p}^n)$. Another occurring problem is the normalization factor $z$ equal to zero. That is whenever $u_{\max}^n= u_{\min}^n$ holds. A solution can be obtained by a simple regularization. Considering both these issues, we choose
\begin{equation}
 \alpha^n = \frac{c_1 L_{2p}[u]}{z + c_2},
\end{equation} 
where $c_2>0$. Experiments showed that $c_1=10$ is an appropriate choice to compensate the difference between the accuracies of the approximations. The regularization is picked as $c_2 = \|\bu\|_1$, where
\begin{equation*}
 \|\bu^n\|_1 = \sum_{i=1}^N \frac{|u_i^n|}{N}\mu(\Omega)
\end{equation*}
is the discrete $L^1$-norm. The order of the applied PA operators is selected as $p=4$ in this work. In a last step, we apply a sup-mollification to define the predictor
\begin{equation}
 \tilde{\alpha}^n = \alpha^n\circledast \max\left \{1-\frac{1}{3}\norm{\frac{x}{\Delta x}},0 \right\}.
\end{equation}







