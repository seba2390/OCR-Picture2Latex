\section{Appendix: Notation} \label{se_not}

%A multitude of fluxes need to be named in our work. We therefore use $f:\R^p \to \R^p$ to refer to the given flux of a conservation law or system of conservation laws
%\[
%	\derive{\bu}{t} + \derive{f \circ \bu}{x} = 0
%\]. The variable $u: \R \times \R \to \R$ refers to a single conserved quantitiy while $\bu: \R \times \R \to \R^p$ is a vector of conserved quantities. Numerical fluxes are designated using letters following $f$ in the alphabet and have $2$ or more arguments. As we ware using entropy stable low order fluxes like the godunov flux are these designated $g: \R^p \times \R^p \to \R^p$ while entropy conservative and possibly high order numerical fluxes should be named $h: \R^p \times \R^p \to \R^p$. If a numerical flux can be from either of both categories we will use $f$ and write down it's arguments in the form $f(u_{k-p+1}, \dots, u_{k+p})$ to remove ambiguities with analytical fluxes. 
%Entropy-Entropyflux pairs $(U, F)$ are designated using uppercase letters and numerical entropy fluxes use the uppercase letters following $f$. It is therefore highly favorable to name the numerical entropy flux $G: \R^p \times \R^p \to \R^p$ associated with a dissipative numerical flux $g$ with the corresponding uppercase letter. Calculations involving numerical fluxes can involve lengthy argument list and therefore the well known short forms
%\[
%	g(u(x_k, t), u(x_{k+1}, t)) = g(u_{k}(t), u_{k+1}(t)) = g_{k+\frac 1 2}(t)
%\]
% generalizing $x_k$ as a way of refering to the center of cell $x_k$ and $x_{k+\frac 1 2}$ to refer to the right cell boundary are used in this case. The same procedure can be also used to refer to grid points in time
% \[
%  g(u(x_k, t_n), u(x_{k+1}, t_n)) = g(u^n_{k}, u^n_{k+1}) = g_{k+\frac 1 2}^n.
% \] 
% Please note that a $2p$ point numerical flux at position $k+ \frac 1 2$ therefore uses the points $u_{k-p+1}, \dots,u_{k+p}$
% \[
%  h(u^n_{k-1}, u^n_{k}, u^n_{k+1}, u^n_{k+2}) = h^n_{k+\frac 1 2}.
% \] Convex combined fluxes are written as 
% \[
% f_{\alpha_{k+\frac 1 2}}^n = \alpha_{k + \frac 1 2} g^n_{k+\frac 1 2} + (1-\alpha_{k+\frac 1 2}) h^n_{k+\frac 1 2}.
% \]
% and use the letter $f$ as they can be, depending on $\alpha$, conservative or dissipative. The position for $\alpha$ also denotes the cell boundary and $\alpha$ is used to differentiate between analytical and convex combination numerical fluxes.
% It is sometimes possible to refer to analytic fluxes in the middle of a cell. We will assume that the solution in the cell is constant in space at a certain time and we will therefore use the short form
%\[
%	f^n_k = f(u^n_k) = f(u(x_k, t_n))
%\]
%at this instance. The same short forms will be also used for numerical entropy fluxes. Sometimes cells are cutted in half at position $x_k$ as in figure \ref{fig:celldivision}. Therefore exist the cell interface at $x_{k-1}, x_{k-\frac 1 2}, x_k, x_{k + \frac 1 2}$ and $x_{k+1}$. The middlepoints are therfore $x_{k-\frac 3 4}, x_{k-\frac 1 4}, x_{k+\frac 1 4}, x_{k+\frac 3 4}$.