\section{Summary}\label{se_summary}

In this work, we compared three different ways to control oscillations in a high-order finite volume scheme. After giving an introduction and an overview over the underlying numerical flux based on a convex combination, some physical constraints were concerned. To be more specific, we gave conditions that will assure the fully discrete entropy inequality or  pressure and density of the numerical solution of the Euler equations to be positive. A second possibility was further constructed using a feedforward neural network. Here, the network was trained by data which were calculated by a reference scheme. We provided afterwards a choice of the convex parameter based on polynomial annihilation operators after giving a brief introduction to their basic framework. In a last step, the resulting schemes were tested and compared by numerical experiments on the Euler equations. 
We have recognized that  our FV schemes except the PPLFT  were able to handle 
strong shocks and are mainly oscillation free. In respect to oscillations, we recognized the best performance for the DELFT scheme which is not suprising since it ensures the fully discrete entropy inequality. The drawback of this scheme was that only second order could be reached in our tests due to the selection of $\alpha$. However, subcell techniques can be used to solve this issue and will be investigated in the future. 
As we also recognized in our simulations, the selection of $\alpha$ using \textbf{Condition F} is sufficient but not necessary. By analyzing the remaining schemes, we have not recognized any violation of the entropy inequality even not for the data driven scheme.  We like to point out again that all of our considered approaches show promising results and can be used. \\
In the future, we plan to continue our investigation and consider two-dimensional problems using unstructured grids. Here, additional techniques are needed and we will also consider more advanced benchmark problems.  Extensions to multiphase flows are as well planned. 
Finally, our high-order FV blending schemes can be also the starting point of a convergence analysis for the Euler equations via dissipative measure-valued solutions  \cite{feireisl2019convergence, lukavcova2022convergence} which is already work in progress.