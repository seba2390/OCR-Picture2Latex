\section{Background}

\subsection{Coloring Problem }

 % Distance-1 graph coloring is the typical instance.
While there exist many definitions of the ``graph coloring problem,'' we specifically consider variants of distance-1 and distance-2 coloring.
Consider graph $G = (V,E)$ with vertex set $V$ and edge set $E$.
\emph{Distance-1 coloring} assigns to each vertex $v \in V$ a color $C(v)$ such that $\forall (u,v) \in E, C(u) \neq C(v)$.
In \emph{distance-2 coloring}, colors are assigned such that 
$\forall (u,v),(v,w) \in E, C(u) \neq C(v) \neq C(w)$; 
i.e., all vertices within two hops of each other have different colors.
\emph{Partial Distance-2 coloring} is a special case of distance-2 coloring in which
$\forall (u,v),(v,w) \in E, C(u) \neq C(w)$; it is typically applied to bipartite graphs
in which only one set of the vertices is given colors (thus, the designation ``partial'').
Partial distance-2 coloring is used to color sparse Jacobian matrices~\cite{GebremedhinMannePothen}.
When a coloring satisfies one of the above constraints, it is called \emph{proper}.
The goal is to find proper colorings of $G$ such that the total number of different colors used is minimized.


%\note{updated some of the definitions. Changed c$\rightarrow$C for consistency with algorithm below. I also made Distance-1 and Distance-2 as lowercase, since I don't think they should be proper.}

%For these problems, the total number of colors needed to 


%is a variant where each vertex \textit{v} must have a different color than any vertex at most two edges away from \textit{v}.



\subsection{Coloring Background}

While minimizing the number of colors is NP-hard, serial coloring algorithms using greedy heuristics have been effective for many applications~\cite{IAB:gebremedhin2000scalable}.
The serial greedy algorithm in Algorithm~\ref{IAB:alg:serialgreed} colors vertices one at a time.
%with implementations often using some heuristic to control the order in which the vertices get colored.
Colors are represented by integers, and the smallest usable color is assigned as a vertex's color.
Most serial and parallel coloring algorithms use some variation of greedy coloring, with algorithmic differences usually involving the processing order of vertices or, in parallel, the handling of conflicts and communication.

\begin{algorithm}
  \caption{Serial greedy coloring algorithm}
  \label{IAB:alg:serialgreed}
  \begin{algorithmic}
    \Procedure{SerialGreedy}{Graph $G=(V,E)$}
      \State $C(\forall v \in V) \gets 0$ \Comment{Initialize all colors as null}
      \ForAll{$v \in V$ in some order}
        \State $c \gets$ the \emph{smallest} color not used by a neighbor of $v$
        \State $C(v) \gets c$
      \EndFor
    \EndProcedure
  \end{algorithmic}
\end{algorithm}

\emph{Conflicts} in a coloring are edges that violate the color-assignment criterion; for example, in distance-1 coloring, a conflict is an edge with both endpoints sharing the same color.
Colorings that contain conflicts are not proper colorings, and are referred to as \emph{pseudo-colorings}.
Pseudo-colorings arise only in parallel coloring, as conflicts arise only when two vertices are colored concurrently.
%The number of colors used determines 
A coloring's ``quality'' refers to the number of colors used; higher quality colorings of a graph $G$ use fewer colors, while lower quality colorings of $G$ use more colors.

%\note{updated algorithm. Note that I use \$V\$ instead of italic text simply for consistency with everything in algorithms, definitions, and the text. Also, I changed it to use arrows for assignment instead of equal signs, as it's generally better to save those for explicit comparison in pseudo-code. I avoid language/code-specific notation like ``=='' or ``!='', but that's mostly just preference.}

It has been observed that the order vertices are visited affects the number of colors needed.
Popular vertex orderings include largest-degree-first, smallest-degree-last, and saturation degree~\cite{Besta20}. 
These orderings are highly sequential and do not allow much parallelism. However, relaxations of those orderings
can allow some parallelism \cite{Hasenplaugh14}.

%\todo{probably don't need this subsection}

%Ways this problem has been solved
\subsection{Parallel Coloring Algorithms}

There are two popular approaches to parallel graph coloring.
The first concurrently finds independent sets of vertices and concurrently colors all of the vertices in each set. This approach was used by 
Jones and Plassmann~\cite{IAB:jones1993parallel}.
Osama et al.~\cite{osama19} found independent sets on a single GPU and explored the impact of varying the baseline independent set algorithm.

The second approach, referred to as ``speculate and iterate''~\cite{IAB:gebremedhin2000scalable,IAB:ccatalyurek2012graph}, colors as many vertices as possible in parallel and then iteratively fixes conflicts in the resulting pseudo-coloring until no conflicts remain.
Gebremedhin et al.~\cite{IAB:gebremedhin2000scalable}, {\c{C}}ataly{\"{u}}rek et al.~\cite{IAB:ccatalyurek2012graph} and Rokos et al.~\cite{IAB:rokos2015fast} present shared-memory implementations based on the speculate and iterate approach.
Deveci et al.~\cite{IAB:deveci2016parallel} present implementations based on the speculate and iterate approach that are scalable on a single GPU.
Distributed-memory algorithms such as those in~\cite{IAB:bozdaug2008framework,IAB:sariyuce2012scalable} use the speculate and iterate approach. 
Grosset et al.~\cite{IAB:grosset2011evaluating} present a hybrid speculate and iterate approach that splits computations between the CPU and a single GPU, 
but does not operate on multiple GPUs in a distributed memory context.
%Osama et al.~\cite{osama19} compare different algorithms on single GPU.
Sallinen et al.~\cite{Sallinen16} demonstrated how to color very large, dynamic graphs efficiently.
Besta et al. \cite{Besta20} developed shared memory coloring algorithms and analyzed their performance.
They compared to both Jones-Plassman and speculative methods, but only on multicore CPU.

Bozda{\u{g}} et al.~\cite{IAB:bozdaug2008framework} showed that, in distributed memory, the speculative approach is more scalable than methods based on the independent set approach of Jones and Plassmann.
Therefore, we choose a speculative and iterative approach with our algorithms.

\subsection{Distributed Coloring}

%\note{updated the definition for local graph below}

In a typical distributed memory setting, an input graph is split into subgraphs that are assigned to separate processes.
A process's \emph{local graph} $G_l = \{V_l+V_g, E_l+E_g\}$ is the subgraph assigned to the process.
Its vertex set $V_l$ contains \emph{local vertices}, and a process is said to \emph{own} its local vertices.  The intersection of all processes' $V_l$ 
is null, and the union equals $V$.
The local graph also has non-local vertex set $V_g$, with such non-local vertices commonly referred to as \emph{ghost vertices}; these vertices are copies of 
vertices owned by other processes.
To ensure a proper coloring, each process needs to store color state information for both local vertices and ghost vertices; typically, ghost vertices are treated as read-only.
The local graph contains edge set $E_l$, edges between local vertices, and $E_g$, edges containing at least one ghost vertex as an endpoint.
% Its edge set $E_l$ typically contains all edges among to local vertices; as will be discussed in our methods, we also consider $E_l$ containing all edges within two-hops of local vertices.
% This set can include edges that connect to or among non-local vertices.
% each process needs copies of non-local vertices that share edges with its local vertices.
% These vertex copies are called \emph{ghosts}; typically, they are treated as read-only.
Bozda{\u{g}} et al.~\cite{IAB:bozdaug2008framework} also defines two subsets of local vertices: \emph{boundary vertices} and \emph{interior vertices}.
Boundary vertices are locally owned vertices that share an edge with at least one ghost; interior vertices are locally owned vertices that do not neighbor ghosts.
For processes to communicate colors associated with their local vertices, each vertex has a unique global identifier (GID).
%KDD Not needed; we don't refer to LIDs elsewhere in paper.
%Local identifiers (LIDs) that each process uses for vertices is likely different from these GIDs, so each process must have a way of mapping their LIDs to GIDs.
%\subsection{Distributed Coloring Framework}

%\todo{This and the following subsections are probably better off in the methods. The text is pretty much repeated in the methods, so just consolidate everything there.}

%Our general approach builds on the framework presented by Bozda{\u{g}} et al.
%Their framework groups vertices on each process into a group of vertices that do not neighbor any vertices on another process, called ``interior vertices'',
%and vertices that do neighbor vertices on another process, called ``boundary vertices''.
%Each process's set of interior vertices can then be colored independently, without creating any conflicts and without requiring any communication.
%Boundary vertices are colored in rounds in order to reduce the chance of conflicts occurring, which in turn reduces the amount of communication necessary to color the boundary vertices.
%By contrast, our approach does not draw this distinction between the interior and the boundary explicitly.
%Because we do not explicitly have a set of ``boundary vertices'', we do not color them in rounds.
%Instead, we opt to initially color all vertices locally then fix global conflicts after communicating the updated colors. 

%\subsection{Local Distance-1 Coloring Algorithms}
%To initially color the local graph in our Distance-1 algorithms, we use an algorithm proposed by Deveci et. al.~\cite{IAB:deveci2016parallel}.
%We use their edge-based coloring, as they show that this approach is more scalable on GPUs than their vertex-based coloring algorithm.
%Vertex-based approaches have inherent load balance issues, as graphs typically have wildly imbalanced degree distributions.
%However, edge-based methods require looping through every edge, even for verifying a correct coloring.
%Because of this, we use their vertex-based algorithm for recoloring since it only needs to loop through the vertices a single time to verify a correct coloring.
%These algorithms are both implemented in KokkosKernels~\cite{IAB:edwards2014kokkos}, so we are able to use them on both CPU and GPU architectures.
%Both of these algorithms are parallel and optimistic, so it is necessary to check the global coloring for validity and fix the conflicts until the coloring is valid.

%\subsection{Local Distance-2 Coloring Algorithms}
%Our Distance-2 implementation uses the NB\_BIT algorithm implemented in KokkosKernels. This algorithm is similar to the one described by Ta{\c{s}} et. al~\cite{IAB:tacs2017greed}.
%NB\_BIT is a Net-Based Distance-2 coloring algorithm that parallelizes over Distance-2 neighborhoods, rather than individual vertices.
%This approach requires looking through each neighborhood for every iteration, so using this approach for recoloring at each iteration is expensive when the number of vertices to recolor is relatively small.
%For recoloring, we use a Vertex-Based approach implemented in KokkosKernels, as it has less overhead when the number of vertices to color is small. 
%Vertex-Based approaches visit each uncolored vertex in parallel and pick a color that does not conflict with any vertex in its Distance-2 neighborhood. 
%This optimistic coloring can lead to a pseudo-coloring, so it is necessary to check for conflicts and recolor until a valid coloring is created.

