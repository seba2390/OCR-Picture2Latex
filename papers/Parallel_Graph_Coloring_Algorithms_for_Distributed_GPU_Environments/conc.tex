\section{Future work}

We have presented new multi-GPU distributed memory implementations of distance-1, distance-2 and
partial distance-2 graph coloring.  These methods enable parallel graph coloring for graphs too large
to fit into a single GPUs memory; weak-scaling results demonstrate coloring of a graph with
12.8 billion vertices and 76.7 billion edges in less than two seconds.  
We introduced a new recoloring heuristic based on vertex degrees 
that reduces the amount of recoloring needed in parallel coloring methods.
We showed that our approaches are scalable to 128 GPUs and produce colorings with quality
similar to or better than Zoltan's distributed memory coloring algorithms.

Because our coloring algorithms use the Kokkos and KokkosKernels library for 
on-node performance portability, our MPI+X methods can also run on distributed-memory 
computers with multicore (CPU-based) nodes.
In this work, we focused on GPU architectures; exploring multicore performance 
is future work.

We are currently integrating this code into the Zoltan2 package of Trilinos.
Our goal is to deliver a complete suite of MPI+X algorithms for distance-1, distance-2, and partial distance-2 coloring in Zoltan2. 
We will modify PD2 to allow it to color only vertices of interest to the application as Zoltan does.

%This work's target application is the optimization of the computation of sparse Jacobian~\cite{GebremedhinMannePothen,IAB:rostami2020preconditioning} and Hessian matrices~\cite{IAB:gebremedhin2020introduction},
%both of which are used in automatic differentiation and other computational problems~\cite{IAB:bozdaug2010distributed}.
%Future work will include interfacing with these applications, and benchmarking their speedup due to our colorings.

We also will investigate further optimizations to increase performance.
There are optimizations present in Zoltan's implementation that are not directly applicable to our implementation, but can inform optimizations that
reduce the overall recoloring workload and minimize communication.
These changes could increase performance for D1, D2, and PD2, 
as well as make D2 and PD2 more scalable on skewed graphs.

%\includegraphics{New_DOE_Logo_Black.png}
%\includegraphics{NNSA Logo_Black.jpg}
%\includegraphics{SNL_Stacked_Black_Blue.jpg}

