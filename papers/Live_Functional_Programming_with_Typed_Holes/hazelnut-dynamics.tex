\PassOptionsToPackage{svgnames,dvipsnames,svgnames}{xcolor}
\newif\ifarxiv
\arxivtrue
% \arxivfalse
\ifarxiv
\documentclass[acmsmall,screen,nonacm]{acmart}
\else
%%% Note: arxiv does not want line numbers (they are detected somehow, and are not allowed)
\documentclass[acmsmall,screen]{acmart}
% \settopmatter{printfolios=false,printccs=false,printacmref=false}
% \settopmatter{printccs=false,printacmref=false}
\fi

\ifarxiv
\newcommand{\appendixName}{appendix}
\else
\newcommand{\appendixName}{extended appendix}
\fi

%% For single-blind review submission
%\documentclass[acmlarge,review]{acmart}\settopmatter{printfolios=true}
%% For final camera-ready submission
%\documentclass[acmlarge]{acmart}\settopmatter{}

%% Note: Authors migrating a paper from PACMPL format to traditional
%% SIGPLAN proceedings format should change 'acmlarge' to
%% 'sigplan,10pt'.

% \bibliographystyle{ACM-Reference-Format}


%% Some recommended packages.
\usepackage{booktabs}   %% For formal tables:
                        %% http://ctan.org/pkg/booktabs
\usepackage{subcaption} %% For complex figures with subfigures/subcaptions
                        %% http://ctan.org/pkg/subcaption

%% Cyrus packages
\usepackage{microtype}
\usepackage{mdframed}
\usepackage{colortab}
\usepackage{mathpartir}
\usepackage{enumitem}
\usepackage{bbm}
\usepackage{stmaryrd}
\usepackage{mathtools}
\usepackage{leftidx}
\usepackage{todonotes}
\usepackage{xspace}
\usepackage{wrapfig}
\usepackage{extarrows}
% \usepackage[subtle]{savetrees}

\usepackage{listings}%
\lstloadlanguages{ML}
\lstset{tabsize=2, 
basicstyle=\footnotesize\ttfamily, 
% keywordstyle=\sffamily,
commentstyle=\itshape\ttfamily\color{gray}, 
stringstyle=\ttfamily\color{purple},
mathescape=false,escapechar=\#,
numbers=left, numberstyle=\scriptsize\color{gray}\ttfamily, language=ML, showspaces=false,showstringspaces=false,xleftmargin=15pt, 
morekeywords={string, float, int, bool},
classoffset=0,belowskip=\smallskipamount, aboveskip=\smallskipamount,
moredelim=**[is][\color{red}]{SSTR}{ESTR}
}
\newcommand{\li}[1]{\lstinline[basicstyle=\ttfamily\fontsize{9pt}{1em}\selectfont]{#1}}
\newcommand{\lismall}[1]{\lstinline[basicstyle=\ttfamily\fontsize{9pt}{1em}\selectfont]{#1}}

%% Joshua Dunfield macros
\def\OPTIONConf{1}%
\usepackage{joshuadunfield}

%% Can remove this eventually
\usepackage{blindtext}

\usepackage{enumitem}

%%%%%%%%%%%%%%%%%%%%%%%%%%%%%%%%%%%%%%%%%%%%%%%%%%%%%%%%%%%%%%%%%%%%%%%%%%%%%
%% Matt says: Cyrus, this package `adjustbox` seems directly related
%% to the `clipbox` error; To get rid of the error, I moved it last
%% (after other usepackages) and I added the line just above it, which
%% permits it to redefine `clipbox` (apparently also defined in
%% `pstricks`, and due to latex's complete lack of namespace
%% management, these would otherwise names clash).
\let\clipbox\relax
\usepackage[export]{adjustbox}% http://ctan.org/pkg/adjustbox
%%%%%%%%%%%%%%%%%%%%%%%%%%%%%%%%%%%%%%%%%%%%%%%%%%%%%%%%%%%%%%%%%%%%%%%%%%%%%%%%%


%%%%%%%%%%%%%%%%%%%%%%%%%%%%%%%%%%%%%%%%%%%%%%%%%%%%%%%%%%%%%%%%%%%%%%%%%%%%%%%%%
%\usepackage{draftwatermark}
%\SetWatermarkText{DRAFT}
%\SetWatermarkScale{1}
%%%%%%%%%%%%%%%%%%%%%%%%%%%%%%%%%%%%%%%%%%%%%%%%%%%%%%%%%%%%%%%%%%%%%%%%%%%%%%%%%


% A macro for the name of the system being described by ``this paper''
\newcommand{\HazelnutLive}{\textsf{Hazelnut Live}\xspace}
\newcommand{\Hazelnut}{\textsf{Hazelnut}\xspace}
% The mockup, work-in-progress system.
\newcommand{\Hazel}{\textsf{Hazel}\xspace}

% \newtheorem{theorem}{Theorem}[chapter]
% \newtheorem{lemma}[theorem]{Lemma}
% \newtheorem{corollary}[theorem]{Corollary}
% \newtheorem{definition}[theorem]{Definition}
% \newtheorem{assumption}[theorem]{Assumption}
% \newtheorem{condition}[theorem]{Condition}

\newtheoremstyle{slplain}% name
  {.15\baselineskip\@plus.1\baselineskip\@minus.1\baselineskip}% Space above
  {.15\baselineskip\@plus.1\baselineskip\@minus.1\baselineskip}% Space below
  {\slshape}% Body font
  {\parindent}%Indent amount (empty = no indent, \parindent = para indent)
  {\bfseries}%  Thm head font
  {.}%       Punctuation after thm head
  { }%      Space after thm head: " " = normal interword space;
        %       \newline = linebreak
  {}%       Thm head spec
\theoremstyle{slplain}
\newtheorem{thm}{Theorem}  % Numbered with the equation counter
\numberwithin{thm}{section}
\newtheorem{defn}[thm]{Definition}
\newtheorem{lem}[thm]{Lemma}
\newtheorem{prop}[thm]{Proposition}
\newtheorem{corol}[thm]{Corollary}
% \newtheorem{cor}[section]{Corollary}     
% \newtheorem{lem}[section]{Lemma}         
% \newtheorem{prop}[section]{Proposition}  

% \setlength{\abovedisplayskip}{0pt}
% \setlength{\belowdisplayskip}{0pt}
% \setlength{\abovedisplayshortskip}{0pt}
% \setlength{\belowdisplayshortskip}{0pt}


\ifarxiv
\setcopyright{rightsretained} 
\acmJournal{PACMPL}
\acmYear{2019} \acmVolume{3} \acmNumber{POPL} \ifarxiv \acmArticle{1} \else \acmArticle{14} \fi \acmMonth{1} \acmPrice{}\acmDOI{10.1145/3290327}
\copyrightyear{2019}
\else
%%% The following is specific to POPL '19 and the paper
%%% 'Live Functional Programming with Typed Holes'
%%% by Cyrus Omar, Ian Voysey, Ravi Chugh, and Matthew A. Hammer.
%%%
\setcopyright{rightsretained}
\acmPrice{}
\acmDOI{10.1145/3290327}
\acmYear{2019}
\copyrightyear{2019}
\acmJournal{PACMPL}
\acmVolume{3}
\acmNumber{POPL}
\acmArticle{14}
\acmMonth{1}
\fi


% \fancyfoot{} % suppresses the footer (also need \thispagestyle{empty} after \maketitle below)


%% Bibliography style
\bibliographystyle{ACM-Reference-Format}
%% Citation style
%% Note: author/year citations are required for papers published as an
%% issue of PACMPL.
\citestyle{acmauthoryear}   %% For author/year citations

\usepackage{booktabs} 
\usepackage{amsmath,url}
\let\Bbbk\relax
\usepackage{amssymb}
 
\usepackage{amsfonts}
% \usepackage{ctable}
\usepackage{multirow}
% \usepackage{algorithm}
% \usepackage{algpseudocode}
% \usepackage{pifont}
\usepackage{color}
% \usepackage{bbm}
\usepackage{enumitem}
\usepackage{dsfont}
\usepackage{graphicx}
\usepackage{subcaption}
% \let\comment\undefined
% \usepackage[commentmarkup=margin]{changes}
% NOTE: I have to undefined \comment since I want to use the \comment environment
% provided by the verbatim package




\newcommand{\todo}[1]{\textcolor{blue}{\bf #1}}
\newcommand{\fixme}[1]{\textcolor{red}{\bf #1}}

\newcommand{\mc}[3]{\multicolumn{#1}{#2}{#3}}
\newcommand{\mr}[2]{\multirow{#1}{0.10\textheight}{#2}}
% \newcommand{\he}[1]{{\textsf{\textcolor{red}{[From He: #1]}}}}

\newcommand{\mylistbegin}{
  \begin{list}{$\bullet$}
   {
     \setlength{\itemsep}{-2pt}
     \setlength{\leftmargin}{1em}
     \setlength{\labelwidth}{1em}
     \setlength{\labelsep}{0.5em} } }
\newcommand{\mylistend}{
   \end{list}  }

\newcommand{\eg}{\textit{e.g.}}
\newcommand{\xeg}{\textit{E.g.}}
\newcommand{\ie}{\textit{i.e.}}
\newcommand{\etc}{\textit{etc}}
\newcommand{\etal}{\textit{et al.}}
\newcommand{\wrt}{\textit{w.r.t.~}}
\newcommand{\header}[1]{{\vspace{+1mm}\flushleft \textbf{#1}}}
\newcommand{\sheader}[1]{{\flushleft \textit{#1}}}
\newcommand{\CGIR}{\textit{CGIR}}

\newcommand{\floor}[1]{\lfloor #1 \rfloor}
\newcommand{\ceil}[1]{\lceil #1 \rceil}

\newcommand{\bx}{\boldsymbol{x}}
\newcommand{\by}{\boldsymbol{y}}
\newcommand{\ba}{\boldsymbol{a}}
\newcommand{\bw}{\boldsymbol{w}}
\newcommand{\bW}{\boldsymbol{W}}
\newcommand{\bfn}{\boldsymbol{f}}
\newcommand{\blambda}{\boldsymbol{\lambda}}
\newcommand{\btheta}{\boldsymbol{\theta}}

\newcommand{\mcW}{\mathcal{W}}
\newcommand{\mcY}{\mathcal{Y}}
\newcommand{\mcS}{\mathcal{S}}
\newcommand{\mcA}{\mathcal{A}}
\newcommand{\mcV}{\mathcal{V}}
\newcommand{\mcE}{\mathcal{E}}
\newcommand{\mcG}{\mathcal{G}}

\DeclareMathOperator*{\argmax}{arg\,max}
\DeclareMathOperator*{\argmin}{arg\,min}

\let\comment\undefined
%TODO
\newcommand{\todo}[1]{{\color{red}{\bf [TODO]:~{#1}}}}

%THEOREMS
\newtheorem{theorem}{Theorem}
\newtheorem{corollary}{Corollary}
\newtheorem{lemma}{Lemma}
\newtheorem{proposition}{Proposition}
\newtheorem{problem}{Problem}
\newtheorem{definition}{Definition}
\newtheorem{remark}{Remark}
\newtheorem{example}{Example}
\newtheorem{assumption}{Assumption}

%HANS' CONVENIENCES
\newcommand{\define}[1]{\textit{#1}}
\newcommand{\join}{\vee}
\newcommand{\meet}{\wedge}
\newcommand{\bigjoin}{\bigvee}
\newcommand{\bigmeet}{\bigwedge}
\newcommand{\jointimes}{\boxplus}
\newcommand{\meettimes}{\boxplus'}
\newcommand{\bigjoinplus}{\bigjoin}
\newcommand{\bigmeetplus}{\bigmeet}
\newcommand{\joinplus}{\join}
\newcommand{\meetplus}{\meet}
\newcommand{\lattice}[1]{\mathbf{#1}}
\newcommand{\semimod}{\mathcal{S}}
\newcommand{\graph}{\mathcal{G}}
\newcommand{\nodes}{\mathcal{V}}
\newcommand{\agents}{\{1,2,\dots,N\}}
\newcommand{\edges}{\mathcal{E}}
\newcommand{\neighbors}{\mathcal{N}}
\newcommand{\Weights}{\mathcal{A}}
\renewcommand{\leq}{\leqslant}
\renewcommand{\geq}{\geqslant}
\renewcommand{\preceq}{\preccurlyeq}
\renewcommand{\succeq}{\succcurlyeq}
\newcommand{\Rmax}{\mathbb{R}_{\mathrm{max}}}
\newcommand{\Rmin}{\mathbb{R}_{\mathrm{min}}}
\newcommand{\Rext}{\overline{\mathbb{R}}}
\newcommand{\R}{\mathbb{R}}
\newcommand{\N}{\mathbb{N}}
\newcommand{\A}{\mathbf{A}}
\newcommand{\B}{\mathbf{B}}
\newcommand{\x}{\mathbf{x}}
\newcommand{\e}{\mathbf{e}}
\newcommand{\X}{\mathbf{X}}
\newcommand{\W}{\mathbf{W}}
\newcommand{\weights}{\mathcal{W}}
\newcommand{\alternatives}{\mathcal{X}}
\newcommand{\xsol}{\bar{\mathbf{x}}}
\newcommand{\y}{\mathbf{y}}
\newcommand{\Y}{\mathbf{Y}}
\newcommand{\z}{\mathbf{z}}
\newcommand{\Z}{\mathbf{Z}}
\renewcommand{\a}{\mathbf{a}}
\renewcommand{\b}{\mathbf{b}}
\newcommand{\I}{\mathbf{I}}
\DeclareMathOperator{\supp}{supp}
\newcommand{\Par}[2]{\mathcal{P}_{{#1} \to {#2}}}
\newcommand{\Laplacian}{\mathcal{L}}
\newcommand{\F}{\mathcal{F}}
\newcommand{\inv}[1]{{#1}^{\sharp}}
\newcommand{\energy}{Q}
\newcommand{\err}{\mathrm{err}}
\newcommand{\argmin}{\mathrm{argmin}}
\newcommand{\argmax}{\mathrm{argmax}}

\setlength{\abovecaptionskip}{4pt plus 3pt minus 2pt} % Chosen fairly arbitrarily
\setlength{\belowcaptionskip}{-4pt plus 3pt minus 2pt} % Chosen fairly arbitrarily


\begin{document}

%% Title information
\title{Live Functional Programming with Typed Holes}         %% [Short Title] is optional;
\ifarxiv
\subtitle{Extended Version}
\subtitlenote{The original version of this article was published in the POPL 2019 edition of PACMPL \cite{HazelnutLive}. This extended version includes an additional appendix.}
\fi
                                        %% when present, will be used in

                                        %% header instead of Full Title.
% \titlenote{with title note}             %% \titlenote is optional;
                                        %% can be repeated if necessary;
                                        %% contents suppressed with 'anonymous'
% \subtitle{Subtitle}                     %% \subtitle is optional
% \subtitlenote{with subtitle note}       %% \subtitlenote is optional;
                                        %% can be repeated if necessary;
                                        %% contents suppressed with 'anonymous'


%% Author information
%% Contents and number of authors suppressed with 'anonymous'.
%% Each author should be introduced by \author, followed by
%% \authornote (optional), \orcid (optional), \affiliation, and
%% \email.
%% An author may have multiple affiliations and/or emails; repeat the
%% appropriate command.
%% Many elements are not rendered, but should be provided for metadata
%% extraction tools.

%% Author with single affiliation.
\author{Cyrus Omar}
% \authornote{with author1 note}          %% \authornote is optional;
                                        %% can be repeated if necessary
% \orcid{nnnn-nnnn-nnnn-nnnn}             %% \orcid is optional
\affiliation{
  % \position{Position1}
  % \department{Department1}              %% \department is recommended
  \institution{University of Chicago, USA}            %% \institution is required
  % \streetaddress{Street1 Address1}
  % \city{City1}
  % \state{State1}
  % \postcode{Post-Code1}
  % \country{Country1}
}
\email{comar@cs.uchicago.edu}          %% \email is recommended

\author{Ian Voysey}
% \authornote{with author1 note}          %% \authornote is optional;
                                        %% can be repeated if necessary
% \orcid{nnnn-nnnn-nnnn-nnnn}             %% \orcid is optional
\affiliation{
  % \position{Position1}
  % \department{Department1}              %% \department is recommended
  \institution{Carnegie Mellon University, USA}            %% \institution is required
  % \streetaddress{Street1 Address1}
  % \city{City1}
  % \state{State1}
  % \postcode{Post-Code1}
  % \country{Country1}
}
\email{iev@cs.cmu.edu}          %% \email is recommended

\author{Ravi Chugh}
% \authornote{with author1 note}          %% \authornote is optional;
                                        %% can be repeated if necessary
% \orcid{nnnn-nnnn-nnnn-nnnn}             %% \orcid is optional
\affiliation{
  % \position{Position1}
  % \department{Department1}              %% \department is recommended
  \institution{University of Chicago, USA}            %% \institution is required
  % \streetaddress{Street1 Address1}
  % \city{City1}
  % \state{State1}
  % \postcode{Post-Code1}
  % \country{Country1}
}
\email{rchugh@cs.uchicago.edu}          %% \email is recommended

\author{Matthew A. Hammer}
% \authornote{with author1 note}          %% \authornote is optional;
                                        %% can be repeated if necessary
% \orcid{nnnn-nnnn-nnnn-nnnn}             %% \orcid is optional
\affiliation{
  % \position{Position1}
  % \department{Department1}              %% \department is recommended
  \institution{University of Colorado Boulder, USA}            %% \institution is required
  % \streetaddress{Street1 Address1}
  % \city{City1}
  % \state{State1}
  % \postcode{Post-Code1}
  % \country{Country1}
}
\email{matthew.hammer@colorado.edu}          %% \email is recommended


% %% Author with two affiliations and emails.
% \author{First2 Last2}
% \authornote{with author2 note}          %% \authornote is optional;
%                                         %% can be repeated if necessary
% \orcid{nnnn-nnnn-nnnn-nnnn}             %% \orcid is optional
% \affiliation{
%   \position{Position2a}
%   \department{Department2a}             %% \department is recommended
%   \institution{Institution2a}           %% \institution is required
%   \streetaddress{Street2a Address2a}
%   \city{City2a}
%   \state{State2a}
%   \postcode{Post-Code2a}
%   \country{Country2a}
% }
% \email{first2.last2@inst2a.com}         %% \email is recommended
% \affiliation{
%   \position{Position2b}
%   \department{Department2b}             %% \department is recommended
%   \institution{Institution2b}           %% \institution is required
%   \streetaddress{Street3b Address2b}
%   \city{City2b}
%   \state{State2b}
%   \postcode{Post-Code2b}
%   \country{Country2b}
% }
% \email{first2.last2@inst2b.org}         %% \email is recommended


%% Paper note
%% The \thanks command may be used to create a "paper note" ---
%% similar to a title note or an author note, but not explicitly
%% associated with a particular element.  It will appear immediately
%% above the permission/copyright statement.
% \thanks{with paper note}                %% \thanks is optional
                                        %% can be repeated if necesary
                                        %% contents suppressed with 'anonymous'


%% Abstract
%% Note: \begin{abstract}...\end{abstract} environment must come
%% before \maketitle command
  In this paper, we explore the connection between secret key agreement and secure omniscience within the setting of the multiterminal source model with a wiretapper who has side information. While the secret key agreement problem considers the generation of a maximum-rate secret key through public discussion, the secure omniscience problem is concerned with communication protocols for omniscience that minimize the rate of information leakage to the wiretapper. The starting point of our work is a lower bound on the minimum leakage rate for omniscience, $\rl$, in terms of the wiretap secret key capacity, $\wskc$. Our interest is in identifying broad classes of sources for which this lower bound is met with equality, in which case we say that there is a duality between secure omniscience and secret key agreement. We show that this duality holds in the case of certain finite linear source (FLS) models, such as two-terminal FLS models and pairwise independent network models on trees with a linear wiretapper. Duality also holds for any FLS model in which $\wskc$ is achieved by a perfect linear secret key agreement scheme. We conjecture that the duality in fact holds unconditionally for any FLS model. On the negative side, we give an example of a (non-FLS) source model for which duality does not hold if we limit ourselves to communication-for-omniscience protocols with at most two (interactive) communications.  We also address the secure function computation problem and explore the connection between the minimum leakage rate for computing a function and the wiretap secret key capacity.
  
%   Finally, we demonstrate the usefulness of our lower bound on $\rl$ by using it to derive equivalent conditions for the positivity of $\wskc$ in the multiterminal model. This extends a recent result of Gohari, G\"{u}nl\"{u} and Kramer (2020) obtained for the two-user setting.
  
   
%   In this paper, we study the problem of secret key generation through an omniscience achieving communication that minimizes the 
%   leakage rate to a wiretapper who has side information in the setting of multiterminal source model.  We explore this problem by deriving a lower bound on the wiretap secret key capacity $\wskc$ in terms of the minimum leakage rate for omniscience, $\rl$. 
%   %The former quantity is defined to be the maximum secret key rate achievable, and the latter one is defined as the minimum possible leakage rate about the source through an omniscience scheme to a wiretapper. 
%   The main focus of our work is the characterization of the sources for which the lower bound holds with equality \textemdash it is referred to as a duality between secure omniscience and wiretap secret key agreement. For general source models, we show that duality need not hold if we limit to the communication protocols with at most two (interactive) communications. In the case when there is no restriction on the number of communications, whether the duality holds or not is still unknown. However, we resolve this question affirmatively for two-user finite linear sources (FLS) and pairwise independent networks (PIN) defined on trees, a subclass of FLS. Moreover, for these sources, we give a single-letter expression for $\wskc$. Furthermore, in the direction of proving the conjecture that duality holds for all FLS, we show that if $\wskc$ is achieved by a \emph{perfect} secret key agreement scheme for FLS then the duality must hold. All these results mount up the evidence in favor of the conjecture on FLS. Moreover, we demonstrate the usefulness of our lower bound on $\wskc$ in terms of $\rl$ by deriving some equivalent conditions on the positivity of secret key capacity for multiterminal source model. Our result indeed extends the work of Gohari, G\"{u}nl\"{u} and Kramer in two-user case.


%% 2012 ACM Computing Classification System (CSS) concepts
%% Generate at 'http://dl.acm.org/ccs/ccs.cfm'.
% \begin{CCSXML}
% <ccs2012>
% <concept>
% <concept_id>10011007.10011006.10011008</concept_id>
% <concept_desc>Software and its engineering~General programming languages</concept_desc>
% <concept_significance>500</concept_significance>
% </concept>
% <concept>
% <concept_id>10003456.10003457.10003521.10003525</concept_id>
% <concept_desc>Social and professional topics~History of programming languages</concept_desc>
% <concept_significance>300</concept_significance>
% </concept>
% </ccs2012>
% \end{CCSXML}
\begin{CCSXML}
<ccs2012>
<concept>
<concept_id>10011007.10011006.10011008.10011009.10011012</concept_id>
<concept_desc>Software and its engineering~Functional languages</concept_desc>
<concept_significance>500</concept_significance>
</concept>
% <concept>
% <concept_id>10003752.10010124.10010131.10010134</concept_id>
% <concept_desc>Theory of computation~Operational semantics</concept_desc>
% <concept_significance>500</concept_significance>
% </concept>
</ccs2012>
\end{CCSXML}

\ccsdesc[500]{Software and its engineering~Functional languages}
% \ccsdesc[500]{Theory of computation~Operational semantics}
%% End of generated code


%% Keywords
%% comma separated list
\keywords{live programming, gradual typing, contextual modal type theory, typed holes, structured editing}  %% \keywords is optional


%% \maketitle
%% Note: \maketitle command must come after title commands, author
%% commands, abstract environment, Computing Classification System
%% environment and commands, and keywords command.
\maketitle
% \thispagestyle{empty} % suppresses the footer

% !TEX root = ../arxiv.tex

Unsupervised domain adaptation (UDA) is a variant of semi-supervised learning \cite{blum1998combining}, where the available unlabelled data comes from a different distribution than the annotated dataset \cite{Ben-DavidBCP06}.
A case in point is to exploit synthetic data, where annotation is more accessible compared to the costly labelling of real-world images \cite{RichterVRK16,RosSMVL16}.
Along with some success in addressing UDA for semantic segmentation \cite{TsaiHSS0C18,VuJBCP19,0001S20,ZouYKW18}, the developed methods are growing increasingly sophisticated and often combine style transfer networks, adversarial training or network ensembles \cite{KimB20a,LiYV19,TsaiSSC19,Yang_2020_ECCV}.
This increase in model complexity impedes reproducibility, potentially slowing further progress.

In this work, we propose a UDA framework reaching state-of-the-art segmentation accuracy (measured by the Intersection-over-Union, IoU) without incurring substantial training efforts.
Toward this goal, we adopt a simple semi-supervised approach, \emph{self-training} \cite{ChenWB11,lee2013pseudo,ZouYKW18}, used in recent works only in conjunction with adversarial training or network ensembles \cite{ChoiKK19,KimB20a,Mei_2020_ECCV,Wang_2020_ECCV,0001S20,Zheng_2020_IJCV,ZhengY20}.
By contrast, we use self-training \emph{standalone}.
Compared to previous self-training methods \cite{ChenLCCCZAS20,Li_2020_ECCV,subhani2020learning,ZouYKW18,ZouYLKW19}, our approach also sidesteps the inconvenience of multiple training rounds, as they often require expert intervention between consecutive rounds.
We train our model using co-evolving pseudo labels end-to-end without such need.

\begin{figure}[t]%
    \centering
    \def\svgwidth{\linewidth}
    \input{figures/preview/bars.pdf_tex}
    \caption{\textbf{Results preview.} Unlike much recent work that combines multiple training paradigms, such as adversarial training and style transfer, our approach retains the modest single-round training complexity of self-training, yet improves the state of the art for adapting semantic segmentation by a significant margin.}
    \label{fig:preview}
\end{figure}

Our method leverages the ubiquitous \emph{data augmentation} techniques from fully supervised learning \cite{deeplabv3plus2018,ZhaoSQWJ17}: photometric jitter, flipping and multi-scale cropping.
We enforce \emph{consistency} of the semantic maps produced by the model across these image perturbations.
The following assumption formalises the key premise:

\myparagraph{Assumption 1.}
Let $f: \mathcal{I} \rightarrow \mathcal{M}$ represent a pixelwise mapping from images $\mathcal{I}$ to semantic output $\mathcal{M}$.
Denote $\rho_{\bm{\epsilon}}: \mathcal{I} \rightarrow \mathcal{I}$ a photometric image transform and, similarly, $\tau_{\bm{\epsilon}'}: \mathcal{I} \rightarrow \mathcal{I}$ a spatial similarity transformation, where $\bm{\epsilon},\bm{\epsilon}'\sim p(\cdot)$ are control variables following some pre-defined density (\eg, $p \equiv \mathcal{N}(0, 1)$).
Then, for any image $I \in \mathcal{I}$, $f$ is \emph{invariant} under $\rho_{\bm{\epsilon}}$ and \emph{equivariant} under $\tau_{\bm{\epsilon}'}$, \ie~$f(\rho_{\bm{\epsilon}}(I)) = f(I)$ and $f(\tau_{\bm{\epsilon}'}(I)) = \tau_{\bm{\epsilon}'}(f(I))$.

\smallskip
\noindent Next, we introduce a training framework using a \emph{momentum network} -- a slowly advancing copy of the original model.
The momentum network provides stable, yet recent targets for model updates, as opposed to the fixed supervision in model distillation \cite{Chen0G18,Zheng_2020_IJCV,ZhengY20}.
We also re-visit the problem of long-tail recognition in the context of generating pseudo labels for self-supervision.
In particular, we maintain an \emph{exponentially moving class prior} used to discount the confidence thresholds for those classes with few samples and increase their relative contribution to the training loss.
Our framework is simple to train, adds moderate computational overhead compared to a fully supervised setup, yet sets a new state of the art on established benchmarks (\cf \cref{fig:preview}).

\section{Discussion on Approximation \textit{vs} Stability and Recovery}\label{sec:approx-stability}


In the world of approximation algorithms, for a maximization problem for which an algorithm outputs $S$ and the optimum is $S^*$, what we typically try to prove is that
$w(S)\ge w(S^*)/\alpha$, even in the worst case; this \textit{approximation inequality} means that the algorithm at hand is an $\alpha$-approximation, so it is a \textit{good} algorithm. Though one might be quick to say that recovery of $\alpha$-stable instances immediately follows from the approximation inequality, this is not true because of the intersection $S\cap S^*$; if we have no intersection, then recovery indeed follows. 

What the research on stability and exact recovery suggests, is that we should try to understand if some of our already known approximation algorithms have the stronger property $w(S\setminus S^*)\ge w(S^*\setminus S)/\alpha$ or at least if they have it on stable instances. We refer to the latter as the \textit{recovery inequality}. This would directly imply an exact recovery result for $\alpha$-stable instances because we could $\alpha$-perturb only the $S\setminus S^*$ part of the input and get: 
\[
\noindent w(S\setminus S^*)\ge w(S^*\setminus S)/\alpha \implies \alpha\cdot w(S\setminus S^*) +w(S\cap S^*) \ge w(S^*\setminus S) +w(S\cap S^*) = w(S^*)
\] thus violating the fact we were given an $\alpha$-stable instance, unless $S\setminus S^* = \emptyset$.

This would mean that the algorithm successfully retrieved $S^*$ and could potentially explain why many approximation algorithms behave far better in practice than in theory. Furthermore, from a theory perspective, it would mean that many results from the well-studied area of approximation algorithms could be translated in terms of stability and recovery.

As a concluding remark, we want to point out that even though one might think that an $\alpha$-approximation algorithm needs at least $\alpha$-stability for recovery, this is not true as the somewhat counterintuitive result from \cite{balcan2015k} tells us: asymmetric $k$-center cannot be approximated to any constant factor, but can be solved optimally on 2-stable instances. This was the
first problem that is hard to approximate to any constant factor in the worst case, yet can be optimally
solved in polynomial time for 2-stable instances. The other direction (having an $\alpha$-approximation algorithm that cannot recover arbitrarily stable instances) is also true. These findings suggest that there are interesting connections between stability, exact recovery and approximation.

% !TEX root = hazelnut-dynamics.tex
\newcommand{\calculusSec}{Hazelnut Live}
\section{\calculusSec}
\label{sec:calculus}
% \def \TirNameStyle \Vtexttt{#1}{{#1}}



% !TEX root = hazelnut-dynamics.tex
\begin{figure}[t]
$\arraycolsep=4pt\begin{array}{rllllll}
\mathsf{HTyp} & \htau & ::= &
  b ~\vert~
  \tarr{\htau}{\htau} ~\vert~
  % \tprod{\htau}{\htau} ~\vert~
  % \tsum{\htau}{\htau} ~\vert~
  \tehole\\
\mathsf{HExp} & \hexp & ::= &
  c ~\vert~
  x ~\vert~
  \halam{x}{\htau}{\hexp} ~\vert~
      {\hlam{x}{\hexp}} ~\vert~
  \hap{\hexp}{\hexp} ~\vert~
  % \hpair{\hexp}{\hexp} ~\vert~
  % \hprj{i}{\hexp} ~\vert~
  % \hinj{i}{\hexp} ~\vert~
  % \hcase{\hexp}{x}{\hexp}{x}{\hexp} ~\vert~
  % \hadd{\hexp}{\hexp} ~\vert~
  \hehole{u} ~\vert~
  \hhole{\hexp}{u} ~\vert~
  \hexp : \htau\\
% \mathsf{Mark} & \markname{} & ::= &
%   \evaled{} ~\vert~  \unevaled{}\\
 \mathsf{IHExp} & \dexp  & ::= &
  c ~\vert~
  x ~\vert~
  {\halam{x}{\htau}{\dexp}} ~\vert~
  \hap{\dexp}{\dexp} ~\vert~
  % \hpair{\dexp}{\dexp} ~\vert~
  % \hprj{i}{\dexp} ~\vert~
  % \hinj{i}{\dexp} ~\vert~
  % \hcase{\dexp}{x}{\dexp}{x}{\dexp} ~\vert~
  % \hadd{\dexp}{\dexp} ~\vert~
  \dehole{\mvar}{\subst}{} ~\vert~
  \dhole{\dexp}{\mvar}{\subst}{} ~\vert~
  \dcasttwo{\dexp}{\htau}{\htau} ~\vert~
  \dcastfail{\dexp}{\htau}{\htau}\\
\end{array}$
$$
\dcastthree{\dexp}{\htau_1}{\htau_2}{\htau_3} \defeq
  \dcasttwo{\dcasttwo{\dexp}{\htau_1}{\htau_2}}{\htau_2}{\htau_3}
$$
\vspace{-12px}
\CaptionLabel{Syntax of types, $\htau$, external expressions, $\hexp$, and internal expressions, $\dexp$.
We write $x$ to range over variables,
$u$ over hole names, and
$\sigma$ over finite substitutions (i.e., environments) 
which map variables to internal expressions, written $d_1/x_1, ~\cdots, d_n/x_n$ for $n \geq 0$.}{fig:hazelnut-live-syntax}
\label{fig:HTyp}
\label{fig:HExp}
\end{figure}


We will now make the intuitions developed in the previous section formally
precise by specifying a core calculus, which we call \HazelnutLive, and
characterizing its metatheory.

\noindent
\parahead{Overview} The syntax of the core calculus given in
Fig.~\ref{fig:hazelnut-live-syntax} consists of types and expressions with
holes.  We distinguish between {external} expressions, $e$, and {internal}
expressions, $d$.  External expressions correspond to programs as entered
by the programmer (see Sec.~\ref{sec:intro} for discussion of implicit,
manual, semi-automated and fully automated hole entry methods).  Each
well-typed external expression (see Sec.~\ref{sec:external-statics} below)
elaborates to a well-typed internal expression (see
Sec.~\ref{sec:elaboration}) before it is evaluated (see
Sec.~\ref{sec:evaluation}).  We take this approach, notably also taken in
the ``redefinition'' of Standard ML by \citet{Harper00atype-theoretic},
because (1) the external language supports type inference and explicit type
ascriptions, $\hexp : \htau$, but it is formally simpler to eliminate
ascriptions and specify a type assignment system when defining the dynamic
semantics; and (2) we need additional syntactic machinery during evaluation
for tracking hole closures and dynamic type casts.  This machinery is
inserted by elaboration, rather than entered explicitly by the programmer.
In this regard, the internal language is analogous to the cast calculus in
the gradually typed lambda calculus
\cite{DBLP:conf/snapl/SiekVCB15,Siek06a}, though as we will see the
\HazelnutLive internal language goes beyond the cast calculus in several
respects. We have mechanized these formal developments using the Agda proof
assistant \cite{norell:thesis,norell2009dependently} (see
Sec.~\ref{sec:agda-mechanization}). Rule names in this section,
e.g. \rulename{SVar}, correspond to variables from the
mechanization. The \Hazel implementation substantially follows the formal
specification of \Hazelnut (for the editor) and \HazelnutLive
(for the evaluator). We can formally state a continuity invariant for a
putative combined calculus (see Sec.~\ref{sec:implementation}).

% \rkc{this syntactic sugar is used in four places: ITCastSucceed, ITCastFail,
% ITGround, and ITExpand. that's not many, and those rules don't look much more
% cluttered without the sugar, so consider eliminating it. if so, just toggle the
% definition of the dcastthree macro to the unsugared option.}

\subsection{Static Semantics of the External Language}
\label{sec:external-statics}

% !TEX root = hazelnut-dynamics.tex

\begin{figure}[t]
\judgbox{\hsyn{\hGamma}{\hexp}{\htau}}{$\hexp$ synthesizes type $\htau$}
\vspace{-10px}
\begin{mathpar}
\inferrule[SConst]{ }{
  \hsyn{\hGamma}{c}{b}
}

\inferrule[SVar]{
  x : \htau \in \hGamma
}{
  \hsyn{\hGamma}{x}{\htau}
}

\inferrule[SLam]{
  \hsyn{\hGamma, x : \htau_1}{\hexp}{\htau_2}
}{
  \hsyn{\hGamma}{\halam{x}{\htau_1}{\hexp}}{\tarr{\htau_1}{\htau_2}}
}

\inferrule[SAp]{
    \hsyn{\hGamma}{\hexp_1}{\htau_1}    \\
    \arrmatch{\htau_1}{\tarr{\htau_2}{\htau}}\\\\
        \hana{\hGamma}{\hexp_2}{\htau_2}
}{
  \hsyn{\hGamma}{\hap{\hexp_1}{\hexp_2}}{\htau}
}

\inferrule[SEHole]{ }{
  \hsyn{\hGamma}{\hehole{u}}{\tehole}
}

\inferrule[SNEHole]{
  \hsyn{\hGamma}{\hexp}{\htau}
}{
  \hsyn{\hGamma}{\hhole{\hexp}{u}}{\tehole}
}

\inferrule[SAsc]{
  \hana{\hGamma}{\hexp}{\htau}
}{
  \hsyn{\hGamma}{\hexp : \htau}{\htau}
}
\end{mathpar}

\vsepRule

\judgbox{\hana{\hGamma}{\hexp}{\htau}}{$\hexp$ analyzes against type $\htau$}
\vspace{-4px}
\begin{mathpar}
\inferrule[ALam]{
  \arrmatch{\htau}{\tarr{\htau_1}{\htau_2}}\\
  \hana{\hGamma, x : \htau_1}{\hexp}{\htau_2}
}{
  \hana{\hGamma}{\hlam{x}{\hexp}}{\htau}
}

\inferrule[ASubsume]{
  \hsyn{\hGamma}{\hexp}{\htau}\\
  \tconsistent{\htau}{\htau'}
}{
  \hana{\hGamma}{\hexp}{\htau'}
}
\end{mathpar}
\vspace{-2px}
\CaptionLabel{Bidirectional Typing of External Expressions}{fig:bidirectional-typing}
\vspace{-2px}
\end{figure}


We start with the type system of the \HazelnutLive external language,
which closely follows the \Hazelnut type system \cite{popl-paper}; we summarize the minor differences as they come up.

% except that (1) holes in \HazelnutLive each have a \emph{unique name};
% (2) for brevity of exposition, we removed numbers in favor of a simpler base type, $b$, with one value, $c$ (i.e. $b$ is the unit type); and
% (3) we include both unannotated lambdas, $\hlam{x}{\hexp}$, and half-annotated lambdas, $\halam{x}{\htau}{\hexp}$.  which we discuss
% (along with other systematic extensions) in
% Appendix~\ref{sec:extensions}.

\Figref{fig:bidirectional-typing} defines the type system in the \emph{bidirectional} style
%
with two mutually defined judgements \cite{Pierce:2000ve,bidi-tutorial,DBLP:conf/icfp/DunfieldK13,Chlipala:2005da}. The type synthesis
judgement~$\hsyn{\hGamma}{\hexp}{\htau}$ synthesizes a type~$\htau$
for external expression~$\hexp$ under typing context $\hGamma$, which tracks typing
assumptions of the form $x : \htau$ in the usual
manner \cite{pfpl,tapl}.
%
The type analysis judgement~$\hana{\hGamma}{\hexp}{\htau}$ checks
expression~$\hexp$ against a given type~$\htau$.
%
Algorithmically, analysis accepts a type as input, and synthesis gives
a type as output.
%
We start with synthesis for the programmer's top level external
expression.

% Algorithmically, the type is an output of type synthesis but an input of type analysis.

The primary benefit of specifying the \HazelnutLive external language
bidirectionally is that the programmer need not annotate each hole with a type.
%
An empty hole is
written simply $\hehole{u}$, where $u$ is the hole name, which we tacitly assume is unique
(holes in \Hazelnut were not named).
Rule \rulename{SEHole} specifies that an empty hole synthesizes hole type, written $\tehole$.
%
If an empty hole appears where an expression of some other type is
expected, e.g. under an explicit ascription (governed by Rule \rulename{SAsc})
or in the argument position of a function application (governed by
Rule \rulename{SAp}, discussed below), we apply the \emph{subsumption rule},
Rule \rulename{ASubsume}, which specifies that if an expression $e$ synthesizes
type $\htau$, then it may be checked against any \emph{consistent}
type, $\htau'$.

% !TEX root = hazelnut-dynamics.tex

\begin{figure}[t]
\judgbox{\tconsistent{\htau_1}{\htau_2}}{$\htau_1$ is consistent with $\htau_2$}
\vspace{-5px}
\begin{mathpar}
\inferrule[TCHole1]{ }{
  \tconsistent{\tehole}{\htau}
}

\inferrule[TCHole2]{ }{
  \tconsistent{\htau}{\tehole}
}

\inferrule[TCRefl]{ }{
  \tconsistent{\htau}{\htau}
}

\inferrule[TCArr]{
  \tconsistent{\htau_1}{\htau_1'}\\
  \tconsistent{\htau_2}{\htau_2'}
}{
  \tconsistent{\tarr{\htau_1}{\htau_2}}{\tarr{\htau_1'}{\htau_2'}}
}
%
% \inferrule{
%   \tconsistent{\htau_1}{\htau_1'}\\
%   \tconsistent{\htau_2}{\htau_2'}
% }{
%   \tconsistent{\tprod{\htau_1}{\htau_2}}{\tprod{\htau_1'}{\htau_2'}}
% }
%
% \inferrule{
%   \tconsistent{\htau_1}{\htau_1'}\\
%   \tconsistent{\htau_2}{\htau_2'}
% }{
%   \tconsistent{\tsum{\htau_1}{\htau_2}}{\tsum{\htau_1'}{\htau_2'}}
% }
\end{mathpar}

% \vsepRule

% \judgbox{\tinconsistent{\htau_1}{\htau_2}}{$\htau_1$ is inconsistent with $\htau_2$}
% \begin{mathpar}
%     \inferrule[ICBaseArr1]{ }{
%       \tinconsistent{\tb}{\tarr{\htau_1}{\htau_2}}
%     }

%     \inferrule[ICBaseArr2]{ }{
%       \tinconsistent{\tarr{\htau_1}{\htau_2}}{\tb}
%     }

%     \inferrule[ICArr1]{
%       \tinconsistent{\htau_1}{\htau_3}
%     }{
%       \tinconsistent{\tarr{\htau_1}{\htau_2}}{\tarr{\htau_3}{\htau_4}}
%     }

%     \inferrule[ICArr2]{
%       \tinconsistent{\htau_2}{\htau_4}
%     }{
%       \tinconsistent{\tarr{\htau_1}{\htau_2}}{\tarr{\htau_3}{\htau_4}}
%     }
% \end{mathpar}

\vsepRule

\judgbox{\arrmatch{\htau}{\tarr{\htau_1}{\htau_2}}}{$\htau$ has matched arrow type $\tarr{\htau_1}{\htau_2}$}
\vspace{-4px}
\begin{mathpar}
\inferrule[MAHole]{ }{
  \arrmatch{\tehole}{\tarr{\tehole}{\tehole}}
}

\inferrule[MAArr]{ }{
  \arrmatch{\tarr{\htau_1}{\htau_2}}{\tarr{\htau_1}{\htau_2}}
}
\end{mathpar}

% \judgbox{\prodmatch{\htau}{\tprod{\htau_1}{\htau_2}}}{$\htau$ has matched product type $\tprod{\htau_1}{\htau_2}$}
% \begin{mathpar}
% \inferrule{ }{
%   \prodmatch{\tehole}{\tprod{\tehole}{\tehole}}
% }

% \inferrule{ }{
%   \prodmatch{\tprod{\htau_1}{\htau_2}}{\tprod{\htau_1}{\htau_2}}
% }
% \end{mathpar}

% \judgbox{\summatch{\htau}{\tsum{\htau_1}{\htau_2}}}{$\htau$ has matched sum type $\tsum{\htau_1}{\htau_2}$}
% \begin{mathpar}
% \inferrule{ }{
%   \summatch{\tehole}{\tsum{\tehole}{\tehole}}
% }

% \inferrule{ }{
%   \summatch{\tsum{\htau_1}{\htau_2}}{\tsum{\htau_1}{\htau_2}}
% }
% \end{mathpar}
\CaptionLabel{Type Consistency and Matching}{fig:tconsistent}
\label{fig:arrmatch}
\vspace{-4px}
\end{figure}


Fig.~\ref{fig:tconsistent} specifies the type consistency relation, written $\tconsistent{\htau}{\htau'}$, which specifies that two types are consistent if they differ only up to type holes in corresponding positions.
%
The hole type is consistent with every type, and so, by the subsumption rule, expression holes may appear where an expression of any type is expected. The type consistency relation here coincides with the type consistency relation from gradual type theory by identifying the hole type with the unknown type~\cite{Siek06a}.
%
Type consistency is reflexive and symmetric, but it is \emph{not} transitive.
%
This stands in contrast to subtyping, which is anti-symmetric and transitive; subtyping may be integrated into a gradual type system following \citet{Siek:2007qy}.

Non-empty expression holes, written $\hhole{\hexp}{u}$, behave similarly to empty holes.
%
Rule \rulename{SNEHole} specifies that a non-empty expression hole also synthesizes hole type as long as the expression inside the hole, $\hexp$, synthesizes some (arbitrary) type.
%
Non-empty expression holes therefore internalize the ``red underline/outline'' that many editors display around type inconsistencies in a program.

For the familiar forms of the lambda calculus, the rules again follow prior work.
%
For simplicity, the core calculus includes only a single base type~$b$ with a single constant~$c$, governed by Rule \rulename{SConst} (i.e. $b$ is the unit type).
%
%\matt{Extraneous and uninteresting:}
By contrast, \citet{popl-paper} instead defined a number type with a single operation. That paper also defined sum types as an extension to the core calculus. We follow suit on both counts in \ifarxiv Appendix \ref{sec:extensions}\else the \appendixName\fi.%Appendix~\ref{sec:extensions}.
%

Rule \rulename{SVar} synthesizes the corresponding type from $\hGamma$.
For the sake of exposition, \HazelnutLive includes ``half-annotated'' lambdas, $\halam{x}{\htau}{\hexp}$, in addition to the unannotated lambdas, $\hlam{x}{\hexp}$, from \Hazelnut.
%
Half-annotated lambdas may appear in synthetic position according to Rule \rulename{SLam}, which is standard \cite{Chlipala:2005da}.
%
Unannotated lambdas may only appear where the expected type is known to be either an arrow type or the hole type, which is treated as if it were $\tarr{\tehole}{\tehole}$.\footnote{A system supporting ML-style type reconstruction \cite{damas1982principal} might also include a synthetic rule for unannotated lambdas, e.g. as outlined by \citet{DBLP:conf/icfp/DunfieldK13}, but we stick to this simpler ``Scala-style'' local type inference scheme in this paper \cite{Pierce:2000ve,Odersky:2001lb}.} 
%
To avoid the need for separate rules for these two cases, Rule \rulename{ALam} uses the matching relation $\arrmatch{\htau}{\tarr{\htau_1}{\htau_2}}$ defined in \Figref{fig:arrmatch}, which produces the matched arrow type $\tarr{\tehole}{\tehole}$ given the hole type, and operates as the identity on arrow types \cite{DBLP:conf/snapl/SiekVCB15,DBLP:conf/popl/GarciaC15}. The rule governing function application, Rule \rulename{SAp}, similarly treats an expression of hole type in function position as if it were of type $\tarr{\tehole}{\tehole}$ using the same matching relation.
%
% \Secref{sec:related} dicusses how \HazelnutLive might be enriched with
% with ML-style type reconstruction~\cite{damas1982principal}, perhaps via
% the approach outlined by~\citet{DBLP:conf/icfp/DunfieldK13}.
%
%%%%%%%%%%%%%%%%%%%%%%%%%%%%%%%%%%%%%%%%%%%%%%%%%%%%%%%%%%%%%%%%%%%%%%%%%%%%%%%%%%%%%%%%%%%%%%%%%%%%%%%%%%%%%%%%%%%%%
% To Cyrus from Matt:
%
% Why say the following here? --- It's discussing a different design that we didn't pursue here.  We should move such discussion to related work.
%
%Note that
%%


We do not formally need an explicit fixpoint operator because this calculus supports general recursion due to type holes, e.g. we can express the Y combinator as $(\halam{x}{\tehole}{x(x)}) (\halam{x}{\tehole}{x(x)})$. More generally, the untyped lambda calculus can be embedded as described by \citet{Siek06a}.
%As such, we omit an explicit fixpoint operator for concision.\todo{fixpoint?}{}

\vspace{-4px}
\subsection{Elaboration}
\label{sec:elaboration}
\vspace{-1px}

% !TEX root = hazelnut-dynamics.tex

\begin{figure}[p]
\judgbox
  {\elabSyn{\hGamma}{\hexp}{\htau}{\dexp}{\Delta}}
  {$\hexp$ synthesizes type $\htau$ and elaborates to $\dexp$}
\begin{mathpar}
\inferrule[ESConst]{ }{
  \elabSyn{\hGamma}{c}{b}{c}{\emptyset}
}
\and
\inferrule[ESVar]{
  x : \htau \in \hGamma
}{
  \elabSyn{\hGamma}{x}{\htau}{x}{\emptyset}
}
\and
\inferrule[ESLam]{
  \elabSyn{\hGamma, x : \htau_1}{\hexp}{\htau_2}{\dexp}{\Delta}
}{
  \elabSyn{\hGamma}{\halam{x}{\htau_1}{\hexp}}{\tarr{\htau_1}{\htau_2}}{\halam{x}{\htau_1}{\dexp}}{\Delta}
}
\and
\inferrule[ESAp]{
  \hsyn{\hGamma}{\hexp_1}{\htau_1}\\
  \arrmatch{\htau_1}{\tarr{\htau_2}{\htau}}
  \\\\
  \elabAna{\hGamma}{\hexp_1}{\tarr{\htau_2}{\htau}}{\dexp_1}{\htau_1'}{\Delta_1}\\
  \elabAna{\hGamma}{\hexp_2}{\htau_2}{\dexp_2}{\htau_2'}{\Delta_2}
}{
  \elabSyn
    {\hGamma}
    {\hap{\hexp_1}{\hexp_2}}
    {\htau}
    {\hap{(\dcasttwo{\dexp_1}{\htau_1'}{\tarr{\htau_2}{\htau}})}
         {\dcasttwo{\dexp_2}{\htau_2'}{\htau_2}}}
    {\Dunion{\Delta_1}{\Delta_2}}
}
%
%% \inferrule[ESAp1]{
%%   \hsyn{\hGamma}{\hexp_1}{\tehole}\\
%%   \elabAna{\hGamma}{\hexp_1}{\tarr{\htau_2}{\tehole}}{\dexp_1}{\htau_1}{\Delta_1}\\
%%   \elabAna{\hGamma}{\hexp_2}{\tehole}{\dexp_2}{\htau_2}{\Delta_2}
%% }{
%%   \elabSyn{\hGamma}{\hap{\hexp_1}{\hexp_2}}{\tehole}{\hap{(\dcast{\tarr{\htau_2}{\tehole}}{\dexp_1})}{\dexp_2}}{\Dunion{\Delta_1}{\Delta_2}}
%% }
%%
%% \inferrule[ESAp2]{
%%   \elabSyn{\hGamma}{\hexp_1}{\tarr{\htau_2}{\htau}}{\dexp_1}{\Delta_1}\\
%%   \elabAna{\hGamma}{\hexp_2}{\htau_2}{\dexp_2}{\htau'_2}{\Delta_2}\\
%%   \htau_2 \neq \htau'_2
%% }{
%%   \elabSyn{\hGamma}{\hap{\hexp_1}{\hexp_2}}{\htau}{\hap{\dexp_1}{\dcast{\htau_2}{\dexp_2}}}{\Dunion{\Delta_1}{\Delta_2}}
%% }
%%
%% \inferrule[ESAp3]{
%%   \elabSyn{\hGamma}{\hexp_1}{\tarr{\htau_2}{\htau}}{\dexp_1}{\Delta_1}\\
%%   \elabAna{\hGamma}{\hexp_2}{\htau_2}{\dexp_2}{\htau_2}{\Delta_2}
%% }{
%%   \elabSyn{\hGamma}{\hap{\hexp_1}{\hexp_2}}{\htau}{\hap{\dexp_1}{\dexp_2}}{\Dunion{\Delta_1}{\Delta_2}}
%% }\\
%
%
% \inferrule[expand-pair]{
%   \elabSyn{\hGamma}{\hexp_1}{\htau_1}{\dexp_1}{\Delta_1}\\
%   \elabSyn{\hGamma}{\hexp_2}{\htau_2}{\dexp_2}{\Delta_2}
% }{
%   \elabSyn{\hGamma}{\hpair{\hexp_1}{\hexp_2}}{\tprod{\htau_1}{\htau_2}}{\hpair{\dexp_1}{\dexp_2}}{\Dunion{\Delta_1}{\Delta_2}}
% }
%
% \inferrule[expand-prj]{
%   a
% }{
%   b
% }
%
% (inj)
%
%
% \inferrule[expand-plus]{ }{
%   \elabSyn{\hGamma}{\hadd{\hexp_1}{\hexp_2}}{\tnum}{\hadd{\dexp_1}{\dexp_2}}{\Dunion{\Delta_1}{\Delta_2}}
% }
\and
\inferrule[ESEHole]{ }{
  \elabSyn{\hGamma}{\hehole{u}}{\tehole}{\dehole{u}{\idof{\hGamma}}{}}{\Dbinding{u}{\hGamma}{\tehole}}
}
\and
\inferrule[ESNEHole]{
  \elabSyn{\hGamma}{\hexp}{\htau}{\dexp}{\Delta}
}{
  \elabSyn{\hGamma}{\hhole{\hexp}{u}}{\tehole}{\dhole{\dexp}{u}{\idof{\hGamma}}{}}{\Delta, \Dbinding{u}{\hGamma}{\tehole}}
}
\and
\inferrule[ESAsc]{
  \elabAna{\hGamma}{\hexp}{\htau}{\dexp}{\htau'}{\Delta}
}{
  \elabSyn{\hGamma}{\hexp : \htau}{\htau}{\dcasttwo{\dexp}{\htau'}{\htau}}{\Delta}
}
%% \inferrule[ESAsc1]{
%%   \elabAna{\hGamma}{\hexp}{\htau}{\dexp}{\htau'}{\Delta}\\
%%   \htau \neq \htau'
%% }{
%%   \elabSyn{\hGamma}{\hexp : \htau}{\htau}{\dcast{\htau}{\dexp}}{\Delta}
%% }
%%
%% \inferrule[ESAsc2]{
%%   \elabAna{\hGamma}{\hexp}{\htau}{\dexp}{\htau}{\Delta}
%% }{
%%   \elabSyn{\hGamma}{\hexp : \htau}{\htau}{\dexp}{\Delta}
%% }
\end{mathpar}

\vsepRule

\judgbox
  {\elabAna{\hGamma}{\hexp}{\htau_1}{\dexp}{\htau_2}{\Delta}}
  {$\hexp$ analyzes against type $\htau_1$ and
   elaborates to $\dexp$ of consistent type $\htau_2$}
\begin{mathpar}
\inferrule[EALam]{
  \arrmatch{\htau}{\tarr{\htau_1}{\htau_2}}\\
  \elabAna{\hGamma, x : \htau_1}{\hexp}{\htau_2}{\dexp}{\htau'_2}{\Delta}
}{
  \elabAna{\hGamma}{\hlam{x}{\hexp}}{\htau}{\halam{x}{\htau_1}{\dexp}}{\tarr{\htau_1}{\htau_2'}}{\Delta}
}

%% \inferrule[EALam]{
%%   \elabAna{\hGamma, x : \htau_1}{\hexp}{\htau_2}{\dexp}{\htau'_2}{\Delta}
%% }{
%%   \elabAna{\hGamma}{\hlam{x}{\hexp}}{\tarr{\htau_1}{\htau_2}}{\halam{x}{\htau_1}{\dexp}}{\tarr{\htau_1}{\htau_2'}}{\Delta}
%% }
%%
%% \inferrule[EALamHole]{
%%   \elabAna{\hGamma, x : \tehole}{\hexp}{\tehole}{\dexp}{\htau}{\Delta}
%% }{
%%   \elabAna{\hGamma}{\hlam{x}{\hexp}}{\tehole}{\halam{x}{\tehole}{\dexp}}{\tarr{\tehole}{\htau}}{\Delta}
%% }
%%
\inferrule[EASubsume]{
  \hexp \neq \hehole{u}\\
  \hexp \neq \hhole{\hexp'}{u}\\\\
  \elabSyn{\hGamma}{\hexp}{\htau'}{\dexp}{\Delta}\\
  \tconsistent{\htau}{\htau'}
}{
  \elabAna{\hGamma}{\hexp}{\htau}{\dexp}{\htau'}{\Delta}
}

\inferrule[EAEHole]{ }{
  \elabAna{\hGamma}{\hehole{u}}{\htau}{\dehole{u}{\idof{\hGamma}}{}}{\htau}{\Dbinding{u}{\hGamma}{\htau}}
}

\inferrule[EANEHole]{
  \elabSyn{\hGamma}{\hexp}{\htau'}{\dexp}{\Delta}\\
}{
  \elabAna{\hGamma}{\hhole{\hexp}{u}}{\htau}{\dhole{\dexp}{u}{\idof{\hGamma}}{}}{\htau}{\Delta, \Dbinding{u}{\hGamma}{\htau}}
}
\end{mathpar}
\CaptionLabel{Elaboration}{fig:elaboration}
\label{fig:expandSyn}
\label{fig:expandAna}
\end{figure}

% !TEX root = hazelnut-dynamics.tex

\begin{figure}[p]
\judgbox{\hasType{\Delta}{\hGamma}{\dexp}{\htau}}{$\dexp$ is assigned type $\htau$}
\begin{mathpar}
\inferrule[TAConst]{ }{
  \hasType{\Delta}{\hGamma}{c}{b}
}

\inferrule[TAVar]{
  x : \htau \in \hGamma
}{
	\hasType{\Delta}{\hGamma}{x}{\htau}
}

\inferrule[TALam]{
  \hasType{\Delta}{\hGamma, x : \htau_1}{\dexp}{\htau_2}
}{
  \hasType{\Delta}{\hGamma}{\halam{x}{\htau_1}{\dexp}}{\tarr{\htau_1}{\htau_2}}
}

\inferrule[TAAp]{
  \hasType{\Delta}{\hGamma}{\dexp_1}{\tarr{\htau_2}{\htau}}\\
  \hasType{\Delta}{\hGamma}{\dexp_2}{\htau_2}
}{
  \hasType{\Delta}{\hGamma}{\hap{\dexp_1}{\dexp_2}}{\htau}
}

\inferrule[TAEHole]{
  \Dbinding{u}{\hGamma'}{\htau} \in \Delta\\
  \hasType{\Delta}{\hGamma}{\sigma}{\hGamma'}
}{
  \hasType{\Delta}{\hGamma}{\dehole{u}{\sigma}{}}{\htau}
}

\inferrule[TANEHole]{
  \hasType{\Delta}{\hGamma}{\dexp}{\htau'}\\\\
  \Dbinding{u}{\hGamma'}{\htau} \in \Delta\\
  \hasType{\Delta}{\hGamma}{\sigma}{\hGamma'}
}{
  \hasType{\Delta}{\hGamma}{\dhole{\dexp}{u}{\sigma}{}}{\htau}
}

\inferrule[TACast]{
  \hasType{\Delta}{\Gamma}{\dexp}{\htau_1}\\
  \tconsistent{\htau_1}{\htau_2}
}{
  \hasType{\Delta}{\hGamma}{\dcasttwo{\dexp}{\htau_1}{\htau_2}}{\htau_2}
}

\inferrule[TAFailedCast]{
  \hasType{\Delta}{\Gamma}{\dexp}{\htau_1}\\
  \isGround{\htau_1}\\
  \isGround{\htau_2}\\
  \htau_1\neq\htau_2
}{
  \hasType{\Delta}{\hGamma}{\dcastfail{\dexp}{\htau_1}{\htau_2}}{\htau_2}
}
\end{mathpar}
\CaptionLabel{Type Assignment for Internal Expressions}{fig:hasType}
\end{figure}



Each well-typed external expression~$e$ elaborates to a well-typed internal
expression~$d$, for evaluation.
%
\Figref{fig:elaboration} specifies elaboration, and \Figref{fig:hasType}
specifies type assignment for internal expressions.
%
% \Secref{sec:evaluation} discusses internal expression evaluation.

As with the type system for the external language (above), we specify
elaboration bidirectionally \cite{DBLP:conf/ppdp/FerreiraP14}.
%
The synthetic elaboration
judgement~$\elabSyn{\hGamma}{\hexp}{\htau}{\dexp}{\Delta}$
%
produces an elaboration~$d$ and a hole context~$\hDelta$ when synthesizing
type $\htau$ for $\hexp$.
%
%We say more about hole contexts, which are used in the type assignment judgement, $\hasType{\Delta}{\hGamma}{d}{\htau}$, below.
We describe hole contexts, which serve as ``inputs'' to the type assignment
judgement~$\hasType{\Delta}{\hGamma}{d}{\htau}$, further below.
%
The analytic elaboration
judgement~$\elabAna{\hGamma}{\hexp}{\htau}{\dexp}{\htau'}{\Delta}$,
produces an elaboration~$d$ of type~$\htau'$, and a hole context~$\hDelta$,
when checking~$\hexp$ against~$\htau$.
%
The following theorem establishes that elaborations are well-typed and in
the analytic case that the assigned type, $\htau'$, is consistent with
provided type, $\htau$.
%
\begin{thm}[Typed Elaboration]\label{thm:typed-elaboration} ~
  \begin{enumerate}[nolistsep]
    \item
      If $\elabSyn{\hGamma}{\hexp}{\htau}{\dexp}{\Delta}$
      then $\hasType{\Delta}{\hGamma}{\dexp}{\htau}$.
    \item
      If $\elabAna{\hGamma}{\hexp}{\htau}{\dexp}{\htau'}{\Delta}$ then
      $\tconsistent{\htau}{\htau'}$ and
      $\hasType{\Delta}{\hGamma}{\dexp}{\htau'}$.
  \end{enumerate}
\end{thm}
\noindent
%
%The reason analytic expansion produces an expansion of consistent
%type is because the subsumption rule, as previously discussed, allows
%us to check an external expression against any type consistent with
%the type the expression actually synthesizes, whereas every internal
%expression can be assigned at most one type, i.e. the following
%standard unicity property holds of the type assignment system.
%
The reason that $\htau'$ is only consistent with the provided type $\htau$ is because
%
the subsumption rule permits us to check an external expression against any
type consistent with the type that the expression \emph{actually}
synthesizes, whereas every internal expression can be assigned at most one
type, i.e. the following standard unicity property holds of the type
assignment system.
%
\begin{thm}[Type Assignment Unicity]
  If $\hasType{\Delta}{\hGamma}{\dexp}{\htau}$
  and $\hasType{\Delta}{\hGamma}{\dexp}{\htau'}$
  then $\htau=\htau'$.
\end{thm}
\noindent
Consequently, analytic elaboration reports the type actually assigned to
the elaboration it produces.
%
For example, we can derive that
$\elabAna{\hGamma}{c}{\tehole}{c}{b}{\emptyset}$.
% where $\emptyset$ is the empty hole context.

Before describing the rules in detail, let us state two other guiding
theorems.  The following theorem establishes that every well-typed external
expression can be elaborated.
 \begin{thm}[Elaborability] \label{thm:elaborability}~
  \begin{enumerate}[nolistsep]
    \item
      If $\hsyn{\hGamma}{\hexp}{\htau}$
      then $\elabSyn{\hGamma}{\hexp}{\htau}{\dexp}{\Delta}$
      for some $\dexp$ and $\Delta$.
    \item
      If $\hana{\hGamma}{\hexp}{\htau}$
      then $\elabAna{\hGamma}{\hexp}{\htau}{\dexp}{\htau'}{\Delta}$
      for some $\dexp$ and $\htau'$ and $\Delta$.
  \end{enumerate}
 \end{thm}

% \noindent
% The following theorem establishes that when an expansion exists, it is unique.
% \begin{thm}[Expansion Unicity] \label{thm:expansion-unicity}~
%   % \begin{enumerate}[nolistsep]
%   %   \item
%       If $\elabSyn{\hGamma}{\hexp}{\htau}{\dexp}{\Delta}$
%       and $\elabSyn{\hGamma}{\hexp}{\htau'}{\dexp'}{\Delta'}$
%       then $\htau=\htau'$ and $\dexp=\dexp'$ and $\Delta=\Delta'$.
%   %   \item
%   %     If $\elabAna{\hGamma}{\hexp}{\htau_1}{\dexp}{\htau_2}{\Delta}$
%   %     and $\elabAna{\hGamma}{\hexp}{\htau_1}{\dexp'}{\htau_2'}{\Delta'}$
%   %     then $\dexp=\dexp'$ and $\htau_2=\htau_2'$ and $\Delta=\Delta'$.
%   % \end{enumerate}
% \end{thm}
\noindent
The following theorem establishes that elaboration generalizes external
typing.
\begin{thm}[Elaboration Generality] \label{thm:elaboration-generality}~
  \begin{enumerate}[nolistsep]
    \item
      If $\elabSyn{\hGamma}{\hexp}{\htau}{\dexp}{\Delta}$
      then $\hsyn{\hGamma}{\hexp}{\htau}$.
    \item
      If $\elabAna{\hGamma}{\hexp}{\htau}{\dexp}{\htau'}{\Delta}$
      then $\hana{\hGamma}{\hexp}{\htau}$.
  \end{enumerate}
\end{thm}

We also establish that the elaboration produces unique results.
\begin{thm}[Elaboration Unicity] \label{thm:expansion-unicity}~
  \begin{enumerate}[nolistsep]
  \item If $\elabSyn{\hGamma}{\hexp}{\htau_1}{\dexp{}_1}{\Delta_1}$ and
    $\elabSyn{\hGamma}{\hexp}{\htau_2}{\dexp{}_2}{\Delta_2}$, then
    $\htau_1 = \htau_2$ and $\dexp{}_1 = \dexp{}_2$ and $\Delta_1 =
    \Delta_2$.
  \item If
    $\elabAna{\hGamma}{\hexp}{\htau}{\dexp{}_1}{\htau_1}{\Delta_1}$ and
    $\elabAna{\hGamma}{\hexp}{\htau}{\dexp{}_2}{\htau_2}{\Delta_2}$, then
    $\htau_1 = \htau_2$ and $\dexp{}_1 = \dexp{}_2$ and $\Delta_1 =
    \Delta_2$
  \end{enumerate}
\end{thm}

The rules governing elaboration of constants, variables and lambda
expressions---Rules \rulename{ESConst}, \rulename{ESVar}, \rulename{ESLam}
and \rulename{EALam}---mirror the corresponding type assignment rules---
Rules \rulename{TAConst}, \rulename{TAVar} and \rulename{TALam}---and in
turn, the corresponding bidirectional typing rules from
Fig.~\ref{fig:bidirectional-typing}.
%
% Consequently, the corresponding cases of
% Theorem~\ref{thm:typed-expansion}, Theorem~\ref{thm:expandability} and
% Theorem~\ref{thm:expansion-generality} are straightforward.
%
To support type assignment, all lambdas in the internal language are
half-annotated---Rule \rulename{EALam} inserts the annotation when
elaborating an unannotated external lambda based on the given type.
%
The rules governing hole elaboration, and the rules that perform \emph{cast
  insertion}---those governing function application and type
ascription---are more interesting. Let us consider each of these two groups
of rules in turn in Sec.~\ref{sec:hole-elaboration} and
Sec.~\ref{sec:cast-insertion}, respectively.

\subsubsection{Hole Elaboration}\label{sec:hole-elaboration}
Rules \rulename{ESEHole}, \rulename{ESNEHole}, \rulename{EAEHole} and
\rulename{EANEHole} govern the elaboration of empty and non-empty
expression holes to empty and non-empty \emph{hole closures},
$\dehole{u}{\sigma}{}$ and $\dhole{\dexp}{u}{\sigma}{}$.
%
The hole name~$u$ on a hole closure identifies the external hole to which
the hole closure corresponds.
%
While we assume each hole name to be unique in the external language, once
evaluation begins, there may be multiple hole closures with the same name
due to substitution.
%
For example, the result from Fig.~\ref{fig:grades-example} shows three
closures for the hole named 1.
%
There, we numbered each hole closure for a given hole sequentially,
\li{1:1}, \li{1:2} and \li{1:3}, but this is strictly for the sake of
presentation, so we omit hole closure numbers from the core calculus.

%
For each hole, $u$, in an external expression, the hole context generated
by elaboration, $\Delta$, contains a hypothesis of the
form~$\Dbinding{u}{\hGamma}{\htau}$, which records the hole's type, $\tau$,
and the typing context, $\Gamma$, from where it appears in the original
expression.\footnote{ We use a hole context, rather than recording the
  typing context and type directly on each hole closure, to ensure that all
  closures for a hole name have the same typing context and type.}
%
We borrow this hole context notation from contextual modal type theory
(CMTT) \cite{Nanevski2008}, identifying hole names with metavariables and
hole contexts with modal contexts (we say more about the connection with
CMTT below).
%
% Each hole expansion rule records the ``current'' typing context under which the hole is expanded.
%
In the synthetic hole elaboration rules~\rulename{ESEHole}
and~\rulename{ESNEHole}, the generated hole context assigns the hole
type~$\tehole$ to hole name~$u$, as in the external typing rules.
%
However, the first two premises of the elaboration subsumption
rule~\rulename{EASubsume} disallow the use of subsumption for holes in
analytic position.
%
Instead, we employ separate analytic rules~\rulename{EAEHole}
and~\rulename{EANEHole}, which each record the checked type~$\tau$ in the
hole context.
%
Consequently, we can use type assignment for the internal language --- the
type assignment rules \rulename{TAEHole} and \rulename{TANEHole} in
Fig.~\ref{fig:hasType} assign a hole closure for hole name~$u$ the
corresponding type from the hole context.

Each hole closure also has an associated environment~$\sigma$ which
consists of a finite substitution of the form $[d_1/x_1, ~\cdots, d_n/x_n]$
for $n \geq 0$.
%
The closure environment keeps a record of the substitutions that occur around the hole as evaluation occurs.
%
Initially, when no evaluation has yet occurred, the hole elaboration rules
generate the identity substitution for the typing context associated with
hole name~$u$ in hole context~$\Delta$, which we notate $\idof{\hGamma}$,
and define as follows.
%
\begin{defn}[Identity Substitution] $\idof{x_1 : \tau_1, ~\cdots, x_n : \tau_n} = [x_1/x_1, ~\cdots, x_n/x_n]$
\end{defn}
\noindent
The type assignment rules for hole closures,~\rulename{TAEHole} and \rulename{TANEHole}, each require that the hole closure environment~$\sigma$ be consistent with the corresponding typing context, written as $\hasType{\Delta}{\hGamma}{\sigma}{\hGamma'}$.
%
Formally, we define this relation in terms of type assignment as follows:
\begin{defn}[Substitution Typing]
$\hasType{\Delta}{\hGamma}{\sigma}{\hGamma'}$ iff $\domof{\sigma} = \domof{\hGamma'}$ and for each $x : \htau \in \hGamma'$ we have that $d/x \in \sigma$ and $\hasType{\Delta}{\hGamma}{d}{\tau}$.
\end{defn}
\noindent
It is easy to verify that the identity substitution satisfies this requirement, i.e. that $\hasType{\Delta}{\hGamma}{\idof{\hGamma}}{\hGamma}$.

Empty hole closures, $\dehole{u}{\sigma}{}$,  correspond to the metavariable closures (a.k.a. deferred substitutions) from CMTT, $\cmttclo{u}{\sigma}$.
%
\Secref{sec:evaluation} defines how these closure environments evolve during evaluation.
%
Non-empty hole closures~$\dhole{d}{u}{\sigma}{}$ have no direct correspondence with a notion from CMTT (see Sec.~\ref{sec:resumption}).

\subsubsection{Cast Insertion}\label{sec:cast-insertion}
%
% USE A TOPIC SENTENCE SO THAT THE READER IS GUIDED A LITTLE MORE, e.g.,
%
Holes in types require us to defer certain structural checks to run time.
%-----------------------------------------------------------------------------------------------
%
To see why this is necessary, consider the following
example: $\hap{(\halam{x}{\tehole}{\hap{x}{c}})}{c}$.
%
Viewed as an external expression, this example synthesizes type
$\tehole$, since the hole type annotation on variable~$x$ permits
applying~$x$ as a function of type~$\tarr{\tehole}{\tehole}$, and base
constant~$c$ may be checked against type~$\tehole$, by subsumption.
%
However, viewed as an internal expression, this example is not
well-typed---the type assignment system defined in
\Figref{fig:hasType} lacks subsumption.
%
Indeed, it would violate type safety if we could assign a type to this
example in the internal language, because beta reduction of this
example viewed as an internal expression would result in $c(c)$, which
is clearly not well-typed.
%
The difficulty arises because leaving the argument type unknown also leaves unknown how
the argument is being used (in this case, as a function).\footnote{In a system where type reconstruction is first used
to try to fill in type holes, we could express a similar example by
using $x$ at two or more different types, thereby causing type
reconstruction to fail.
%
% On the other hand, if it is acceptable to arbitrarily choose one of
%the possible types, and type reconstruction is complete, then type
%holes will never appear in the internal language and the cast
%insertion machinery described in this section can be omitted
%entirely, leaving only the hole closure machinery described
%previously.
}
By our interpretation of hole types as unknown types from gradual type
theory, we can address the problem by performing cast insertion.
%

The cast form in \HazelnutLive is $\dcasttwo{\dexp}{\htau_1}{\htau_2}$.
%
This form serves to ``box'' an expression of type $\htau_1$ for
treatment as an expression of a consistent type $\htau_2$
(Rule~\rulename{TACast} in \Figref{fig:hasType}).%
\footnote{
In the earliest work on gradual type theory, the cast form only gave
the target type~$\htau_2$ \cite{Siek06a}, but it simplifies the dynamic semantics substantially
to include the assigned type~$\htau_1$ in the syntax \cite{DBLP:conf/snapl/SiekVCB15}.
}

Elaboration inserts casts at function applications and ascriptions.
%
The latter is more straightforward: Rule~\rulename{ESAsc}
in \Figref{fig:expandSyn} inserts a cast from the assigned type to the
ascribed type.
%
Theorem~\ref{thm:typed-elaboration} inductively ensures that the two types
are consistent.
%
We include ascription for expository purposes---this form is derivable
by using application together with the half-annotated identity, $e
: \tau = \hap{(\halam{x}{\htau}{x})}{e}$; as such, application
elaboration, discussed below, is more general.

Rule~\rulename{ESAp} elaborates function applications.
%
To understand the rule, consider the elaboration of the example
discussed above, $\hap{(\halam{x}{\tehole}{\hap{x}{c}})}{c}$:
\[
        \hap{\dcasttwo{
        (\halam{x}{\tehole}{\underbrace{
                \hap{\dcasttwo{x}{\tehole}{\tarr{\tehole}{\tehole}}}
                {\dcasttwo{c}{b}{\tehole}}
                }_{\textrm{elaboration of function body}~x(c)}
        }
        )}{\tarr{\tehole}{\tehole}}{
           \tarr{\tehole}{\tehole}}
           }
           {\dcasttwo{c}{b}{\tehole}}
\]
Consider the (indicated) function body,
%
where elaboration inserts a cast on both the function expression~$x$ and
its argument~$c$.
%
Together, these casts for~$x$ and~$c$ permit assigning a type to the
function body according to the rules in \Figref{fig:hasType}, where we
could not do so under the same context without casts.
%
We separately consider the elaborations of~$x$ and of~$c$.

First, consider the function position of this application, here variable~$x$.
%
Without any cast, the type of variable~$x$ is the hole type~$\tehole$;
however, the inserted cast on~$x$ permits treating it as though it has
arrow type $\tarr{\tehole}{\tehole}$.
%
The first three premises of Rule~\rulename{ESAp} accomplish this
%
by first synthesizing a type for the function expression, here
$\tehole$, then
%
by determining the matched arrow type~$\tarr{\tehole}{\tehole}$, and
finally,
%
by performing analytic elaboration on the function expression with this
matched arrow type.
%
The resulting elaboration has some type~$\tau_1'$ consistent with the
matched arrow type.
%
In this case, because the subexpression~$x$ is a variable, analytic
elaboration goes through subsumption so that type~$\tau_1'$ is
simply~$\tehole$.
%
The conclusion of the rule inserts the corresponding cast.
%
We go through type synthesis, \emph{then} analytic elaboration, so that the
hole context records the matched arrow type for holes in function position,
rather than the type~$\tehole$ for all such holes, as would be the case in
a variant of this rule using synthetic elaboration for the function
expression.

Next, consider the application's argument, here constant~$c$.
%
The conclusion of Rule~\rulename{ESAp} inserts the cast on the argument's
elaboration, from the type it is assigned by the final premise of the
rule~(type~$b$), to the argument type of the matched arrow type of the
function expression~(type~$\tehole$).

The example's second, outermost application goes through the same
application elaboration rule.
%
In this case, the cast on the function is the identity cast for
$\tarr{\tehole}{\tehole}$.
%
For simplicity, we do not attempt to avoid the insertion of identity
casts in the core calculus; these will simply never fail during
evaluation.
%
However, it is safe in practice to eliminate such identity casts during
elaboration, and some formal accounts of gradual typing do so by defining
three application elaboration rules, including the original account of
\citet{Siek06a}.

\subsection{Dynamic Semantics}
\label{sec:evaluation}

To recap, the result of elaboration is a well-typed internal expression
with appropriately initialized hole closures and casts.  This section
specifies the dynamic semantics of \HazelnutLive as a ``small-step''
transition system over internal expressions equipped with a meaningful
notion of type safety even for incomplete programs, i.e. expressions typed
under a non-empty hole context, $\Delta$.  We establish that evaluation
does not stop immediately when it encounters a hole, nor when a cast fails,
by precisely characterizing when evaluation \emph{does} stop. In the case
of complete programs, we also recover the familiar statements of preservation
and progress for the simply typed lambda calculus.

% It is perhaps worth stating at the outset that a dynamic semantics equipped
% with these properties does not simply ``fall out'' from the observations
% made above that (1) empty hole closures correspond to metavariable closures
% from CMTT \cite{Nanevski2008} and (2) casts also arise in gradual type
% theory \cite{DBLP:conf/snapl/SiekVCB15}.
%
% We say more in Sec.~\ref{sec:relatedWork}.
%

%  specify the dynamic semantics of \HazelnutLive.
% %
% The dynamic semantics is capable of running incomplete programs, i.e. those with hole closures, without aborting at holes, or when a cast fails.
% %




% !TEX root = hazelnut-dynamics.tex

\begin{figure}
\begin{subfigure}[t]{0.5\textwidth}
\judgbox{\isGround{\htau}}{$\htau$ is a ground type}
\begin{mathpar}
\inferrule[GBase]{ }{
  \isGround{b}
}

\inferrule[GHole]{ }{
  \isGround{\tarr{\tehole}{\tehole}}
}
\end{mathpar}
\end{subfigure}
\hfill
\begin{subfigure}[t]{0.46\textwidth}
\judgbox{\groundmatch{\htau}{\htau'}}{$\htau$ has matched ground type $\htau'$}
\begin{mathpar}
\inferrule[MGArr]{
  \tarr{\htau_1}{\htau_2}\neq\tarr{\tehole}{\tehole}
}{
  \groundmatch{\tarr{\htau_1}{\htau_2}}{\tarr{\tehole}{\tehole}}
}
\end{mathpar}
\end{subfigure}
\CaptionLabel{Ground Types}{fig:isGround}
\label{fig:groundmatch}
\end{figure}

% !TEX root = hazelnut-dynamics.tex
\begin{figure}

\begin{tabular}[t]{cc}

\begin{minipage}{0.5\textwidth}
\judgbox{\isFinal{\dexp}}{$\dexp$ is final}
\begin{mathpar}
%% \inferrule[FVal]
%% {\isValue{\dexp}}{\isFinal{\dexp}}
\inferrule[FBoxedVal]
{\isBoxedValue{\dexp}}{\isFinal{\dexp}}
\and
\inferrule[FIndet]
{\isIndet{\dexp}}{\isFinal{\dexp}}
\end{mathpar}
\end{minipage}

&

\begin{minipage}{0.5\textwidth}

\judgbox{\isValue{\dexp}}{$\dexp$ is a value}
\begin{mathpar}
\inferrule[VConst]{ }{
  \isValue{c}
}

\inferrule[VLam]{ }{
  \isValue{\halam{x}{\htau}{\dexp}}
}
\end{mathpar}
\end{minipage}

\end{tabular}

\vsepRule

\judgbox{\isBoxedValue{\dexp}}{$\dexp$ is a boxed value}
\begin{mathpar}
\inferrule[BVVal]{
  \isValue{\dexp}
}{
  \isBoxedValue{\dexp}
}

\inferrule[BVArrCast]{
  \tarr{\htau_1}{\htau_2} \neq \tarr{\htau_3}{\htau_4}\\
  \isBoxedValue{\dexp}
}{
  \isBoxedValue{\dcasttwo{\dexp}{\tarr{\htau_1}{\htau_2}}{\tarr{\htau_3}{\htau_4}}}
}

\inferrule[BVHoleCast]{
  \isBoxedValue{\dexp}\\
  \isGround{\htau}
}{
  \isBoxedValue{\dcasttwo{\dexp}{\htau}{\tehole}}
}
\end{mathpar}

\vsepRule

\judgbox{\isIndet{\dexp}}{$\dexp$ is indeterminate}
\begin{mathpar}
\inferrule[IEHole]
{ }
{\isIndet{\dehole{\mvar}{\subst}{}}}

\inferrule[INEHole]
{\isFinal{\dexp}}
{\isIndet{\dhole{\dexp}{\mvar}{\subst}{}}}

\inferrule[IAp]
{\dexp_1\neq
   \dcasttwo{\dexp_1'}
            {\tarr{\htau_1}{\htau_2}}
            {\tarr{\htau_3}{\htau_4}}\\
 \isIndet{\dexp_1}\\
% \isFinal{\dexp_2}~\text{\cy{??}}}
 \isFinal{\dexp_2}}
{\isIndet{\dap{\dexp_1}{\dexp_2}}}

\inferrule[ICastGroundHole] {
  \isIndet{\dexp}\\
  \isGround{\htau}
}{
  \isIndet{\dcasttwo{\dexp}{\htau}{\tehole}}
}

\inferrule[ICastHoleGround] {
  \dexp\neq\dcasttwo{\dexp'}{\htau'}{\tehole}\\
  \isIndet{\dexp}\\
  \isGround{\htau}
}{
  \isIndet{\dcasttwo{\dexp}{\tehole}{\htau}}
}

\inferrule[ICastArr]{
  \tarr{\htau_1}{\htau_2} \neq \tarr{\htau_3}{\htau_4}\\
  \isIndet{\dexp}
}{
  \isIndet{\dcasttwo{\dexp}{\tarr{\htau_1}{\htau_2}}{\tarr{\htau_3}{\htau_4}}}
}

\inferrule[IFailedCast] {
  \isFinal{\dexp}\\
  \isGround{\htau_1}\\
  \isGround{\htau_2}\\
  \htau_1\neq\htau_2
}{
  \isIndet{\dcastfail{\dexp}{\htau_1}{\htau_2}}
}

%% \inferrule[ICast]
%% {\isIndet{\dexp}}
%% {\isIndet{\dcast{\htau}{\dexp}}}

\end{mathpar}

%\vsepRule

\CaptionLabel{Final Forms}{fig:isFinal}
\label{fig:isValue}
\label{fig:isIndet}
\end{figure}




Figures~\ref{fig:isGround}-\ref{fig:step} define the dynamic semantics.
%
Most of the cast-related machinery closely follows the cast calculus from
the ``refined'' account of the gradually typed lambda calculus
by \citet{DBLP:conf/snapl/SiekVCB15}, which is known to be
theoretically well-behaved.
%
In particular, \Figref{fig:isGround} defines the judgement
$\isGround{\htau}$, which distinguishes the base type~$b$ and the
least specific arrow type~$\tarr{\tehole}{\tehole}$ as \emph{ground
types}; this judgement helps simplify the treatment of function casts, discussed below.

%
\Figref{fig:isFinal} defines the judgement $\isFinal{d}$, which
distinguishes the final, i.e. irreducible, forms of the transition system.
%
The two rules distinguish two classes of final forms: (possibly-)boxed values and
indeterminate forms.%
%
\footnote{
        Most accounts of the cast calculus distinguish ground types and values
        with separate grammars together with an
        implicit identification convention.
        %
        Our judgemental formulation is more faithful to the mechanization and
        cleaner for our purposes, because we are distinguishing several
        classes of final forms.
}
The judgement $\isBoxedValue{d}$ defines (possibly-)boxed values as either
ordinary values~(Rule~\rulename{BVVal}), or one of two cast forms: casts
between unequal function types and casts from a ground type to the hole
type. In each case, the cast must appear inductively on a boxed value.
These forms are irreducible because they represent values that have been
boxed but have never flowed into a corresponding ``unboxing'' cast,
discussed below.
%
%, which correspond to the values from the cast calculus and include
%the classic values from the lambda calculus, distingui%shed by
%$\isValue{d}$,
%
The judgement $\isIndet{d}$ defines \emph{indeterminate} forms, so named
because they are rooted at expression holes and failed casts, and so,
conceptually, their ultimate value awaits programmer action (see
Sec.~\ref{sec:resumption}). Note that no term is both complete, i.e. has no
holes, and indeterminate.
%
The first two rules specify that {empty} hole closures are always
indeterminate, and that {non}-empty hole closures are indeterminate when
they consist of a {final} inner expression.
%
Below, we describe failed casts and the remaining indeterminate forms simultaneously
with the corresponding transition rules.

Figures~\ref{fig:instruction-transitions}-\ref{fig:step} define the transition rules.
%
Top-level transitions are \emph{steps}, $\stepsToD{}{d}{d'}$, governed by Rule~\rulename{Step} in \Figref{fig:step}, which
%
(1) decomposes $d$ into an evaluation context, $\evalctx$, and a selected sub-term, $d_0$;
%
(2) takes an \emph{instruction transition}, $\reducesE{}{d_0}{d_0'}$, as specified in \Figref{fig:instruction-transitions};
%
and (3) places $d_0'$ back at the selected position, indicated in
the evaluation context by the \emph{mark}, $\evalhole$, to obtain $d'$.%
\footnote{
        We say ``mark'', rather than the more conventional ``hole'', to avoid confusion with the (orthogonal) holes of \HazelnutLive.
        %
        %the form $\evalhole$ in the grammar of
        %evaluation contexts is referred to as the \emph{hole}, but
        %this hole is a technical device entirely orthogonal to the
        %holes of this paper, so we use the term ``mark'' instead.
        }
%
%% \matt{Next paragraph can be dropped entirely for space}
%% %
This approach was originally developed in the reduction semantics of \citet{DBLP:journals/tcs/FelleisenH92} and is the predominant style of operational semantics in the literature on gradual typing.
Because we distinguish final forms judgementally, rather than syntactically, we use a judgemental formulation of this approach called a \emph{contextual dynamics} by \citet{pfpl}.
It would be straightforward to construct an equivalent structural operational semantics \cite{DBLP:journals/jlp/Plotkin04a} by using search rules instead of evaluation contexts (\citet{pfpl} relates the two approaches).

The rules maintain the property that final expressions truly cannot take a step.%
\begin{thm}[Finality] There does not exist $d$ such that both $\isFinal{d}$ and $\stepsToD{}{d}{d'}$ for some $d'$.
\end{thm}

%% \begin{figure}[t]
\judgbox{\stepsToD{\Delta}{\dexp_1}{\dexp_2}}{$\dexp_1$
steps to $\dexp_2$}
\begin{mathpar}
\inferrule[STEHoleEvaled]
{ }
{\stepsToD{\Delta}{\dehole{\mvar}{\subst}{\unevaled}}{\dehole{\mvar}{\subst}{\evaled}}}

\inferrule[STNEHoleStep]
{\stepsToD{\Delta}{\dexp_1}{\dexp_2} }
{\stepsToD{\Delta}{\dhole{\dexp_1}{\mvar}{\subst}{\unevaled}}{\dhole{\dexp_2}{\mvar}{\subst}{\unevaled}}}

\inferrule[STNEHoleEvaled]
{\isFinal{\dexp}}
{\stepsToD{\Delta}{\dhole{\dexp}{\mvar}{\subst}{\unevaled}}{\dhole{\dexp}{\mvar}{\subst}{\evaled}}}

\inferrule[STCast]
{
\isValue{\dexp}\\
\hasType{\Delta}{\emptyset}{\dexp}{\htau_2} \\
\tconsistent{\tau_1}{\tau_2}}
{\stepsToD{\Delta}{\dcast{\htau_1}{\dexp}}{\dexp}}

\inferrule[STApStep1]
{\stepsToD{\Delta}{\dexp_1}{\dexp_1'}}
{\stepsToD{\Delta}{\dap{\dexp_1}{\dexp_2}}{\dap{\dexp_1'}{\dexp_2}}}

\inferrule[STApStep2]
{ \isFinal{\dexp_1} \\ \stepsToD{\Delta}{\dexp_2}{\dexp_2'}}
{\stepsToD{\Delta}{\dap{\dexp_1}{\dexp_2}}{\dap{\dexp_1}{\dexp_2'}}}

\inferrule[STApBeta]
{ \isFinal{\dexp_2} }
{\stepsToD{\Delta}{\dapP{\dlam{x}{\htau}{\dexp_1}}{\dexp_2}}{ [\dexp_2/x]\dexp_1 }}
\end{mathpar}
\caption{Structural Dynamics}
\label{fig:stepsTo}
\end{figure}


\subsubsection{Application and Substitution}
% !TEX root = hazelnut-dynamics.tex
\begin{comment}
\begin{figure}[t]

\begin{comment}
\vsepRule

\judgbox{\isevalctx{\evalctx}}{$\evalctx$ is an evaluation context}
\begin{mathpar}
\inferrule[ECDot]{ }{
  \isevalctx{\evalhole}
}

%% \inferrule[ECLam]{
%%   \isevalctx{\evalctx}
%% }{
%%   \isevalctx{\halam{x}{\htau}{\evalctx}}
%% }

\inferrule[ECAp1]{
  \isevalctx{\evalctx}
}{
  \isevalctx{\hap{\evalctx}{\dexp}}
}

\inferrule[ECAp2]{
  \maybePremise{\isFinal{\dexp}}\\
  \isevalctx{\evalctx}
}{
  \isevalctx{\hap{\dexp}{\evalctx}}
}

\inferrule[ECNEHole]{
  \isevalctx{\evalctx}
}{
  \isevalctx{\dhole{\evalctx}{\mvar}{\subst}{}}
}

\inferrule[ECCast]{
  \isevalctx{\evalctx}
}{
  \isevalctx{\dcasttwo{\evalctx}{\htau_1}{\htau_2}}
}

\inferrule[ECFailedCast]{
  \isevalctx{\evalctx}
}{
  \isevalctx{\dcastfail{\evalctx}{\htau_1}{\htau_2}}
}
\end{mathpar}
% \end{comment}
\vsepRule


\caption{Evaluation Contexts}
\label{fig:eval-contexts}
\end{figure}
\end{comment}

%% \vsepRule

\begin{figure}
\judgbox{\reducesE{}{\dexp}{\dexp'}}{$\dexp$ takes an instruction transition to $\dexp'$}
\begin{mathpar}
\inferrule[ITLam]{
  \maybePremise{\isFinal{\dexp_2}}
}{
  \reducesE{}{\hap{(\halam{x}{\htau}{\dexp_1})}{\dexp_2}}{[\dexp_2/x]\dexp_1}
}

\inferrule[ITApCast]{
  \maybePremise{\isFinal{\dexp_1}}\\
  \maybePremise{\isFinal{\dexp_2}}\\
  \tarr{\htau_1}{\htau_2} \neq \tarr{\htau_1'}{\htau_2'}
}{
  \reducesE{}
    {\hap{\dcasttwo{\dexp_1}{\tarr{\htau_1}{\htau_2}}{\tarr{\htau_1'}{\htau_2'}}}{\dexp_2}}
    {\dcasttwo{(\hap{\dexp_1}{\dcasttwo{\dexp_2}{\htau_1'}{\htau_1}})}{\htau_2}{\htau_2'}}
}

\inferrule[ITCastId]{
  \maybePremise{\isFinal{\dexp}}
}{
  \reducesE{}{\dcasttwo{\dexp}{\htau}{\htau}}{\dexp}
}

\inferrule[ITCastSucceed]{
  \maybePremise{\isFinal{\dexp}}\\
  \isGround{\htau}
}{
  \reducesE{}{\dcastthree{\dexp}{\htau}{\tehole}{\htau}}{\dexp}
}

\inferrule[ITCastFail]{
  \maybePremise{\isFinal{\dexp}}\\
  \htau_1\neq\htau_2\\\\
  \isGround{\htau_1}\\
  \isGround{\htau_2}
}{
  \reducesE{}
    {\dcastthree{\dexp}{\htau_1}{\tehole}{\htau_2}}
    {\dcastfail{\dexp}{\htau_1}{\htau_2}}
}

\inferrule[ITGround]{
  \maybePremise{\isFinal{\dexp}}\\
  \groundmatch{\htau}{\htau'}
}{
  \reducesE{}
    {\dcasttwo{\dexp}{\htau}{\tehole}}
    {\dcastthree{\dexp}{\htau}{\htau'}{\tehole}}
}

\inferrule[ITExpand]{
  \maybePremise{\isFinal{\dexp}}\\
  \groundmatch{\htau}{\htau'}
}{
  \reducesE{}
    {\dcasttwo{\dexp}{\tehole}{\htau}}
    {\dcastthree{\dexp}{\tehole}{\htau'}{\htau}}
}

%% \inferrule[ITCast]{
%%   \isFinal{d}\\
%%   \hasType{\Delta}{\emptyset}{d}{\tau_2}\\
%%   \tconsistent{\tau_1}{\tau_2}
%% }{
%%   \reducesE{\Delta}{\dcast{\htau_1}{d}}{d}
%% }
%%
%% \inferrule[ITEHole]{ }{
%%   \reducesE{\Delta}{\dehole{\mvar}{\subst}{\unevaled}}{\dehole{\mvar}{\subst}{\evaled}}
%% }
%%
%% \inferrule[ITNEHole]{
%%   \isFinal{d}
%% }{
%%   \reducesE{\Delta}{\dhole{d}{\mvar}{\subst}{\unevaled}}{\dhole{d}{\mvar}{\subst}{\evaled}}
%% }
\end{mathpar}
\CaptionLabel{Instruction Transitions}{fig:instruction-transitions}
\end{figure}

\begin{figure}
$\arraycolsep=4pt\begin{array}{rllllll}
\mathsf{EvalCtx} & \evalctx & ::= &
  \evalhole ~\vert~
  \hap{\evalctx}{\dexp} ~\vert~
  \hap{\dexp}{\evalctx} ~\vert~
  \dhole{\evalctx}{\mvar}{\subst}{} ~\vert~
  \dcasttwo{\evalctx}{\htau}{\htau} ~\vert~
  \dcastfail{\evalctx}{\htau}{\htau}
\end{array}$

\vsepRule

\judgbox{\selectEvalCtx{\dexp}{\evalctx}{\dexp'}}{$\dexp$ is obtained by placing $\dexp'$ at the mark in $\evalctx$}
\begin{mathpar}
\vspace{-3px}
\inferrule[FHOuter]{ }{
  \selectEvalCtx{\dexp}{\evalhole}{\dexp}
}

%% \inferrule[FLam]{
%%   \selectEvalCtx{d}{\evalctx}{d'}
%% }{
%%   \selectEvalCtx{\halam{x}{\htau}{d}}{\halam{x}{\htau}{\evalctx}}{d'}
%% }

\inferrule[FHAp1]{
  \selectEvalCtx{\dexp_1}{\evalctx}{\dexp_1'}
}{
  \selectEvalCtx{\hap{\dexp_1}{\dexp_2}}{\hap{\evalctx}{\dexp_2}}{\dexp_1'}
}

\inferrule[FHAp2]{
  \maybePremise{\isFinal{\dexp_1}}\\
  \selectEvalCtx{\dexp_2}{\evalctx}{\dexp_2'}
}{
  \selectEvalCtx{\hap{\dexp_1}{\dexp_2}}{\hap{\dexp_1}{\evalctx}}{\dexp_2'}
}

%% \inferrule[FHEHole]{ }{
%%   \selectEvalCtx{\dehole{\mvar}{\subst}{}}{\evalhole}{\dehole{\mvar}{\subst}{}}
%% }
%%
%% \inferrule[FHNEHoleEvaled]{ }{
%%   \selectEvalCtx{\dhole{d}{\mvar}{\subst}{\evaled}}{\evalhole}{\dhole{d}{\mvar}{\subst}{\evaled}}
%% }

\inferrule[FHNEHoleInside]{
  \selectEvalCtx{\dexp}{\evalctx}{\dexp'}
}{
  \selectEvalCtx{\dhole{\dexp}{\mvar}{\subst}{}}{\dhole{\evalctx}{\mvar}{\subst}{}}{\dexp'}
}

%% \inferrule[FHNEHoleFinal]{
%%   \isFinal{d}
%% }{
%%   \selectEvalCtx{\dhole{d}{\mvar}{\subst}{\unevaled}}{\evalhole}{\dhole{d}{\mvar}{\subst}{\unevaled}}
%% }

\inferrule[FHCastInside]{
  \selectEvalCtx{\dexp}{\evalctx}{\dexp'}
}{
  \selectEvalCtx{\dcasttwo{\dexp}{\htau_1}{\htau_2}}
                {\dcasttwo{\evalctx}{\htau_1}{\htau_2}}
                {\dexp'}
}

\inferrule[FHFailedCast]{
  \selectEvalCtx{\dexp}{\evalctx}{\dexp'}
}{
  \selectEvalCtx{\dcastfail{\dexp}{\htau_1}{\htau_2}}
                {\dcastfail{\evalctx}{\htau_1}{\htau_2}}
                {\dexp'}
}

%% \inferrule[FHCastFinal]{
%%   \isFinal{d}
%% }{
%%   \selectEvalCtx{\dcast{\htau}{d}}{\evalhole}{\dcast{\htau}{d}}
%% }
\end{mathpar}

\vsepRule

\judgbox{\stepsToD{}{\dexp}{\dexp'}}{$\dexp$ steps to $\dexp'$}
\vspace{-20px}
\begin{mathpar}
\inferrule[Step]{
  \selectEvalCtx{d}{\evalctx}{\dexp_0}\\
  \reducesE{}{\dexp_0}{\dexp_0'}\\
  \selectEvalCtx{\dexp'}{\evalctx}{\dexp_0'}
}{
  \stepsToD{}{\dexp}{\dexp'}
}
\end{mathpar}
\vspace{-6px}
\CaptionLabel{Evaluation Contexts and Steps}{fig:step}
\vspace{-4px}
\end{figure}

%
Rule \rulename{ITLam} in Fig.~\ref{fig:instruction-transitions} defines the
standard beta reduction transition.
%
The bracketed premises of the form $\maybePremise{\isFinal{\dexp}}$ in
Fig.~\ref{fig:instruction-transitions}-\ref{fig:step} may be \emph{included}
to specify an eager, left-to-right evaluation strategy, or \emph{excluded} to
leave the evaluation strategy and order unspecified.
%
In our metatheory, we exclude these premises, both for
the sake of generality, and to support the specification of the fill-and-resume operation~(see \Secref{sec:resumption}).



Substitution, written $[d/x]d'$, operates in the standard capture-avoiding manner~\cite{pfpl} (see \ifarxiv Appendix \ref{sec:substitution} \else the \appendixName~\fi for the full definition).
% Appendix~\ref{sec:additional-defns}).
%
The only cases of special interest arise when substitution reaches a hole closure:
\[
\begin{array}{rcl}
  [d/x]\dehole{u}{\sigma}{} & = & \dehole{u}{[d/x]\sigma}{} \\%
  \substitute{d}{x}{\dhole{d'}{u}{\sigma}{}} & = & \dhole{[d/x]d'}{u}{[d/x]\sigma}{}
\end{array}
\]
In both cases, we write~$[d/x]\sigma$ to perform substitution on each expression in the hole environment~$\sigma$, i.e. the environment ``records'' the substitution.
%
For example, $\stepsToD{}
    {\hap{(\halam{x}{b}{\halam{y}{b}{\dehole{u}{[x/x, y/y]}{}}})}{c}}
    {\halam{y}{b}{\dehole{u}{[c/x, y/y]}{}}}$.
%
Beta reduction can duplicate hole closures.
%
Consequently, the environments of different closures with the same hole name may differ,
e.g., when a reduction applies a function with a hole closure body multiple times as in Fig.~\ref{fig:grades-example}.
Hole closures may also appear within the environments of other hole
closures, giving rise to the closure paths described in
Sec.~\ref{sec:paths}.



The \rulename{ITLam} rule is not the only rule we need to handle function
application, because lambdas are not the only final form of arrow type.
%
Two other situations may also arise.

First, the expression in function position might be a cast between
arrow types, in which case we apply the arrow cast conversion rule,
Rule \rulename{ITApCast}, to rewrite the application form, obtaining an
equivalent application where the expression~$d_1$ under the function
cast is exposed.
%
We know from inverting the typing rules that~$d_1$ has type
$\tarr{\htau_1}{\htau_2}$, and that~$d_2$ has type~$\htau_1'$, where
$\tconsistent{\htau_1}{\htau_1'}$.
Consequently, we maintain type
safety by placing a cast on~$d_2$ from~$\htau_1'$ to~$\htau_1$.
%
The result of this application has type $\htau_2$, but the
original cast promised that the result would have consistent type
$\htau_2'$, so we also need a cast on the result from $\htau_2$ to
$\htau_2'$.

Second, the expression in function position may be indeterminate,
where arrow cast conversion is not applicable,
e.g. $\hap{(\dehole{u}{\sigma}{})}{c}$.
%
In this case, the application is indeterminate (Rule~\rulename{IAp}
in \Figref{fig:isFinal}), and the application reduces no
further.


\subsubsection{Casts}
Rule \rulename{ITCastId} strips identity casts. The remaining instruction
transition rules assign meaning to non-identity casts.
%
As discussed in Sec.~\ref{sec:cast-insertion}, the structure of a term
cast \emph{to} hole type is statically obscure,
%
so we must await a \emph{use} of the term at some other type, via a
cast \emph{away} from hole type, to detect the type error dynamically.
%
Rules \rulename{ITCastSucceed} and \rulename{ITCastFail} handle this situation when the
two types involved are ground types (Fig.~\ref{fig:isGround}).
%
If the two ground types are equal, then the cast succeeds and the cast
may be dropped.
%
If they are not equal, then the cast fails and the failed cast form,
$\dcastfail{\dexp}{\htau_1}{\htau_2}$, arises.
%
Rule \rulename{TAFailedCast} specifies that a failed cast is well-typed exactly
when $d$ has ground type $\tau_1$ and $\tau_2$ is a ground type
not equal to $\tau_1$.
%
Rule \rulename{IFailedCast} specifies that a failed cast operates as an
indeterminate form (once $d$ is final), i.e. evaluation does not stop. For simplicity, we do not include blame labels as found in some accounts of gradual typing \cite{DBLP:conf/esop/WadlerF09,DBLP:conf/snapl/SiekVCB15}, but it would be straightforward to do so by recording the blame labels from the two constituent casts on the two arrows of the failed cast.

% Rules \rulename{ITCastSucceed} and \rulename{ITCastFail} only operate at ground type.
%
The two remaining instruction transition rules, Rule~\rulename{ITGround}
and~\rulename{ITExpand}, insert intermediate casts from non-ground type to a
consistent ground type, and \emph{vice versa}.
%
These rules serve as technical devices, permitting us to restrict our
interest exclusively to casts involving ground types and type holes elsewhere.
%
Here, the only non-ground types are the arrow types, so the grounding
judgement~$\groundmatch{\tau_1}{\htau_2}$ (\Figref{fig:groundmatch}),
produces the ground arrow type~$\tarr{\tehole}{\tehole}$.
%
More generally, the following invariant governs this judgement.
\begin{lem}[Grounding]
  If $\groundmatch{\htau_1}{\htau_2}$
  then $\isGround{\htau_2}$
  and $\tconsistent{\htau_1}{\htau_2}$
  and $\htau_1\neq\htau_2$.
\end{lem}

In all other cases, casts evaluate either to boxed values or to
indeterminate forms according to the remaining rules
in \Figref{fig:isFinal}.
%
Of note, Rule \rulename{ICastHoleGround} handles casts from hole to
ground type that are not of
the form $\dcastthree{\dexp}{\htau_1}{\tehole}{\htau_2}$.
%

\subsubsection{Type Safety}
%
The purpose of establishing type safety is to ensure that the static and dynamic semantics of a
language cohere.
%
We follow the approach developed by \citet{wright94:_type_soundness},
now standard \cite{pfpl}, which distinguishes two type safety
properties, preservation and progress.
%
To permit the evaluation of incomplete programs, we establish these
properties for terms typed under arbitrary hole context $\Delta$.
%
We assume an empty typing context, $\hGamma$; to run open programs, the
system may treat free variables as empty holes with a corresponding
name.

The preservation theorem establishes that transitions preserve type
assignment, i.e. that the type of an expression accurately predicts
the type of the result of reducing that expression.

\begin{thm}[Preservation]
  If $\hasType{\Delta}{\emptyset}{\dexp}{\htau}$ and
  $\stepsToD{\Delta}{\dexp}{\dexp'}$ then
  $\hasType{\Delta}{\emptyset}{\dexp'}{\htau}$.
\end{thm}
\noindent
%
The proof relies on an analogous preservation lemma for instruction
transitions and a standard substitution lemma stated in \ifarxiv Appendix \ref{sec:substitution}\else the \appendixName\fi.
% Appendix~\ref{sec:additional-defns}.
%
Hole closures can disappear during evaluation, so we must have structural weakening 
of $\Delta$.

The progress theorem establishes that the dynamic semantics accounts
for every well-typed term, i.e. that we have not forgotten some
necessary rules or premises.
%
\begin{thm}[Progress]
  If $\hasType{\Delta}{\emptyset}{\dexp}{\htau}$ then either
  (a) there exists $\dexp'$ such that $\stepsToD{}{\dexp}{\dexp'}$ or
  (b) $\isBoxedValue{\dexp}$ or
  (c) $\isIndet{\dexp}$.
\end{thm}
\noindent
The key to establishing the progress theorem under a non-empty hole
context is to explicitly account for indeterminate forms,
i.e. those rooted at either a hole closure or a failed cast.
%
The proof relies on canonical forms lemmas stated in \ifarxiv Appendix~\ref{sec:canonical-forms}\else the \appendixName\fi.
% Appendix~\ref{sec:additional-defns}.

\subsubsection{Complete Programs}
%
Although this paper focuses on running \emph{incomplete} programs, it helps
to know that the necessary machinery does not interfere with running
\emph{complete} programs, i.e. those with no type or expression holes.
%
% Appendix~\ref{sec:additional-defns}
\ifarxiv Appendix~\ref{sec:complete-programs} \else The \appendixName{} \fi defines the predicates~$\isComplete{\htau}$,
$\isComplete{\hexp}$, $\isComplete{\dexp}$ and~$\isComplete{\hGamma}$.
%
Of note, failed casts cannot appear in complete internal expressions.
%
The following theorem establishes that elaboration preserves program
completeness.

\begin{thm}[Complete Elaboration]
% ~
  % \begin{enumerate}[nolistsep]
      If $\isComplete{\hGamma}$ and $\isComplete{\hexp}$
      and $\elabSyn{\hGamma}{\hexp}{\htau}{\dexp}{\Delta}$
      then $\isComplete{\htau}$ and $\isComplete{\dexp}$ and $\Delta = \emptyset$.
%     \item
%       If $\isComplete{\hGamma}$ and $\isComplete{\hexp}$
% and $\isComplete{\htau}$
%       and $\elabAna{\hGamma}{\hexp}{\htau}{\dexp}{\htau'}{\Delta}$
%       then $\isComplete{\dexp}$ and $\isComplete{\htau'}$ and $\Delta=\emptyset$
  % \end{enumerate}
\end{thm}

The following preservation theorem establishes that stepping preserves
program completeness.
\begin{thm}[Complete Preservation]
  If $\hasType{\hDelta}{\emptyset}{\dexp}{\htau}$
  and $\isComplete{\dexp}$
  and $\stepsToD{}{\dexp}{\dexp'}$
  then $\hasType{\hDelta}{\emptyset}{\dexp'}{\htau}$
  and $\isComplete{\dexp'}$.
\end{thm}

The following progress theorem establishes that evaluating a complete
program always results in classic values, not boxed values nor
indeterminate forms.
%
\begin{thm}[Complete Progress]
  If $\hasType{\hDelta}{\emptyset}{\dexp}{\htau}$ and $\isComplete{\dexp}$
  then either there exists a $\dexp'$ such that
  $\stepsToD{}{\dexp}{\dexp'}$, or $\isValue{\dexp}$.
\end{thm}

%% \begin{figure}[!ht]
%%   \begin{definition}
%%     $\hasType{\Delta}{\hGamma}{\sigma}{\hGamma'}$ iff for each $\dexp/x \in \sigma$, we have $x : \htau \in \hGamma'$ and $\hasType{\Delta}{\hGamma}{\dexp}{\htau}$.
%%   \end{definition}
%%   \caption{substitution type assignment}
%%   \label{fig:subassign}
%% \end{figure}


%% \begin{figure}[!ht]
%%   \caption{substitution type assignment}
%% \end{figure}

\vspace{-3px}
\subsection{Agda Mechanization}
\label{sec:agda-mechanization}
\vspace{-2px}

The archived artifact includes our Agda
mechanization  \cite{norell2009dependently,norell:thesis,Aydemir:2005fk}
of the semantics and metatheory of \HazelnutLive,
%including proofs of all of the
including all of the theorems stated above and necessary lemmas.
%
%We choose the
%(as did the mechanization of \Hazelnut by \citet{popl-paper}, though only a few definitions are common).
%
%Agda is a good choice because it is designed to explicitly communicate a proof's structure, as is our goal, rather than relying on proof automation.
%
%Agda itself was also an inspiration for this work because it supports holes, albeit in a more limited form than described here (cf. Sec.~\ref{sec:intro}).
%
%
Our approach is standard: we model judgements as
inductive datatypes, and rules as dependently typed constructors of these judgements.
%
We adopt Barendregt's convention for bound variables \cite{urban,barendregt84:_lambda_calculus} and hole names, and avoid certain other complications related to substitution by enforcing the requirement that all bound variables in a term are unique when convenient (this requirement can always be discharged by alpha-variation). We encode typing
contexts and hole contexts using metafunctions.
To support this encoding choice, we postulate function extensionality (which is independent of Agda's axioms) \cite{awodey2012inductive}. We encode finite substitutions as an inductive datatype with a base case representing the identity substitution and an inductive case that records a single substitution. Every finite substitution can be represented this way. This makes it easier for Agda to see why certain inductions that we perform are well-founded.  
The documentation provided with the mechanization has more details.

\vspace{-3px}
\subsection{Implementation and Continuity}\label{sec:implementation}
\vspace{-2px}

The \Hazel implementation described in Sec.~\ref{sec:examples}
includes an unoptimized interpreter, written in OCaml, that implements the semantics as described
in this section, with some simple extensions. As with many full-scale systems, there is not currently a formal
specification for the full \Hazel language, but \ifarxiv Appendix~\ref{sec:extensions} \else the \appendixName~\fi 
discusses how the standard approach for deriving a ``gradualized'' version of a
language construct provides most of the necessary scaffolding \cite{DBLP:conf/popl/CiminiS16}, and provides some examples (sum types and numbers).

% The supplemental material includes a browser-based implementation
% of \HazelnutLive.
% % The implementation supports the core calculus of this section but with the notional base type $b$ replaced by the $\tnum$ type from Appendix~\ref{sec:extensions}. The implementation also includes the sum types extension from Appendix~\ref{sec:extensions} and a few minor  conveniences, e.g. \li{let} binding.
% All of the live programming features from \Secref{sec:examples} are available in the implementation essentially as shown, but for this more austere language (with some minor conveniences, notably let binding). Appendix~\ref{sec:impl-screenshots} provides screenshots of the full user interface.

%  % which is a functional reactive program \cite{Elliott:1997jh,DBLP:conf/pldi/CzaplickiC13}
% %
% % Functional Reactive Animation (Hudak 1997) is the canonical FRP cite, not Elm, if you want one; we don't need one, I think. --Matt
% %
% %\cite{DBLP:conf/pldi/CzaplickiC13}

% The implementation is written with the Reason toolchain for OCaml \cite{reason-what,leroy03:_ocaml}
% together with the OCaml \lismall{React} library \cite{OcamlReact}
% and the \lismall{js_of_ocaml} compiler and its associated libraries \cite{vouillon2014bytecode}. This follows the implementation of \Hazel, which, although tracking toward an Elm-like semantics, is implemented in OCaml because the Elm compiler is not yet self-hosted. The implementation of the dynamic semantics consists of a simple evaluator that closely follows the rules specified in this section.

The editor component of the \Hazel implementation is derived
from the structure editor calculus of~\Hazelnut, but with support for more natural cursor-based movement and infix operator sequences (the details of which are beyond the scope of this paper). It exposes a language of structured
edit actions that automatically insert empty and non-empty holes as necessary
to guarantee that every edit state has some (possibly incomplete) type. This corresponds to the top-level Sensibility invariant established for the \Hazelnut calculus by \citet{popl-paper}, reproduced below:
\begin{prop}[Sensibility]
  \label{thrm:sensibility}
  If $\hsyn{\Gamma}{\removeSel{\zexp}}{\htau}$ and
    $\performSyn{\Gamma}{\zexp}{\htau}{\alpha}{\zexp'}{\tau'}$ then
    $\hsyn{\hGamma}{\removeSel{\zexp'}}{\htau'}$.
  % \item If $\hana{\hGamma}{\removeSel{\zexp}}{\htau}$ and
  %   $\performAna{\hGamma}{\zexp}{\htau}{\alpha}{\zexp'}$ then
  %   $\hana{\hGamma}{\removeSel{\zexp'}}{\htau}$.
\end{prop}
\noindent
Here, $\zexp$ is an editor state (an expression with a cursor), and $\removeSel{\zexp}$ drops the cursor, producing an expression ($e$ in this paper). So in words, ``if, ignoring the cursor, the editor state, $\removeSel{\zexp}$, initially has type $\htau$ and we perform an edit action $\alpha$ on it, then the resulting editor state, $\removeSel{\zexp'}$, will have type $\htau'$''.


By composing this Sensibility property with the \Property{Elaborability},
\Property{Typed Elaboration}, \Property{Progress} and
\Property{Preservation} properties from this section, we establish a
uniquely powerful Continuity invariant:
\begin{corol}[Continuity]
  \label{thrm:continuity}
  If $\hsyn{\emptyset}{\removeSel{\zexp}}{\htau}$ and
    $\performSyn{\emptyset}{\zexp}{\htau}{\alpha}{\zexp'}{\tau'}$ then
    $\elabSyn{\emptyset}{\removeSel{\zexp'}}{\htau'}{\dexp}{\Delta}$
      for some $\dexp$ and $\Delta$ such that
$\hasType{\Delta}{\emptyset}{\dexp}{\htau'}$
and either
  (a) $\stepsToD{}{\dexp}{\dexp'}$ for some $d'$ such that $\hasType{\Delta}{\emptyset}{\dexp'}{\htau'}$; or
  (b) $\isBoxedValue{\dexp}$ or
  (c) $\isIndet{\dexp}$.
  % \item If $\hana{\hGamma}{\removeSel{\zexp}}{\htau}$ and
  %   $\performAna{\hGamma}{\zexp}{\htau}{\alpha}{\zexp'}$ then
  %   $\hana{\hGamma}{\removeSel{\zexp'}}{\htau}$.
\end{corol}

This addresses the gap problem: \emph{every} editor state has a {static
meaning} (so editor services like the type inspector from Fig.~\ref{fig:qsort-type-inspector} are always available) and a non-trivial {dynamic meaning} (a result is always available, evaluation does not stop when a hole or cast failure is encountered, and editor services that rely on hole closures, like the live context inspector from Fig.~\ref{fig:grades-sidebar}, are always available).

In settings where the editor does not maintain this Sensibility invariant, but where programmers can manually insert holes, our approach still helps to reduce the severity of the gap problem, i.e. \emph{more} editor states are dynamically meaningful, even if not \emph{all} of them are.% We can formally state this end-to-end continuity corollary as follows:
%
% We make no further claims about the usability or practicality of the implementation (and indeed, there remain many important and decidedly unresolved questions along these lines, which we leave beyond the scope of this paper).
% We also reiterate that the essential ideas developed from type-theoretic first principles in this paper do not require that
% the editor component of the programming environment be implemented as a structure editor (see Sec.~\ref{sec:intro}).
 %,  to scale up these ideas to a ``real-world'' language and programming environment.

%, and provides to be of use to researchers studying the calculus as presented in this section.
% Consistent with this goal, it closely follows the theoretical account in this section, rather than including advanced language features.

%%%%%%%%%%%%%%%%%%%%%%%%%%%%%%%%%%%%%%%%%%%%%%%%%%%%%%%%%%%%%%%%%%%%%%%%%%%%%%%%%%%%%%%%%%%%%%%%%%%

%% \matt{I'd drop everything else in this sub-section;
%% it leaves us open to attack from a hostile reviewer
%% who is cranky that the implementation is not ``complete'' yet; also, we'd save the space}

%%   \halam{x}{\htau}{\evalctx} ~\vert~

%%%%%%%%%%%%%%%%%%%%%%%%%%%%%%%%%%%%%%%%%%%%%%%%%%%%%%%%%%%%%%%%%%%%%%%%%%%

\newcommand{\commutativitySec}{A Contextual Modal Interpretation of Fill-and-Resume}
\section{\commutativitySec}
\label{sec:resumption}


%The result of evaluation is a final internal expression with hole closures, each with an associated hole environment, $\sigma$. These hole environments can be reported directly to the programmer, e.g. via the sidebar shown in Fig.~X\todo{fig}, to help them as they think about how to fill in the corresponding hole in the external expression. Hole environments might also be useful indirectly, e.g. by informing an edit action synthesis and suggestion system. In any case,
When the programmer performs one or more edit actions to fill in an expression hole in the program, a new result must be computed, ideally quickly \cite{DBLP:conf/icse/Tanimoto13,DBLP:journals/vlc/Tanimoto90}. Na\"ively, the system would need to compute the result ``from scratch'' on each such edit. For small exploratory programming tasks, recomputation is acceptable, but in cases where a large amount of computation might occur, e.g. in data science tasks, a more efficient approach is to resume evaluation from where it left off after an edit that amounts to hole filling. This section develops a foundational account of this feature, which we call \emph{fill-and-resume}. This approach is complementary to, but distinct from, incremental computing (which is concerned with changes in input, not code insertions)~\cite{Hammer2014}.

%%%%%%%%% New par 
Formally,
%
the key idea is to interpret hole environments as \emph{delayed substitutions}. This is the same interpretation suggested for metavariable closures in contextual modal
type theory (CMTT) by \citet{Nanevski2008}.
%
%In practice, it may be useful to cache results from several previous
%expansions, e.g., by employing off-the-shelf programming language
%abstractions for incremental computation~\cite{Hammer14,Hammer15}.
%
%%%%%%%%% New par
\Figref{fig:substitution} defines the hole filling operation~$\instantiate{d}{u}{d'}$
based on the contextual substitution operation of~CMTT.
%
Unlike usual notions of capture-avoiding substitution,
hole filling imposes no condition on the binder when passing into the
body of a lambda expression---the expression that fills a hole can, of
course, refer to variables in scope where the hole appears.
%
When hole filling encounters an empty closure for the hole being
instantiated, $\instantiate{d}{u}{\dehole{u}{\sigma}{}}$, the result
is $[\instantiate{d}{u}{\sigma}]d$.
%
That is, we apply the delayed substitution to the fill expression~$d$
after first recursively filling any instances of hole~$u$ in~$\sigma$.
%
Hole filling for non-empty closures is analogous, where it discards
the previously-enveloped expression.
%
%
% \matt{Will any reader actually wonder this? Does this thought connect to any other statement elsewhere in the paper?}
%
This case shows why we cannot interpret a non-empty hole as an empty
hole of arrow type applied to the enveloped expression---the hole
filling operation would not operate as expected under this
interpretation.


% !TEX root = hazelnut-dynamics.tex

\begin{figure}
\small
%% TODO use instantiate macro
\judgbox
  {\instantiate{\dexp}{u}{\dexp'} = \dexp''}
  {$\dexp''$ is obtained by filling hole $u$ with $\dexp$ in $\dexp'$}

\vsepRule

\judgbox
  {\instantiate{\dexp}{u}{\sigma} = \sigma'}
  {$\sigma'$ is obtained by filling hole $u$ with $\dexp$ in $\sigma$}
  %% {$\dexp''$ is the result of substituting $\dexp$ for $u$ in $\dexp'$}
\[
\begin{array}{lcll}
\instantiate{\dexp}{u}{c}
&=&
c\\
{\instantiate{\dexp}{u}{x}}
&=&
x\\
%% {[\dexp_1 / x] y}
%% &=&
%% y & (y \neq x)\\
{\instantiate{\dexp}{u}{\halam{x}{\htau}{\dexp'}}}
&=&
{\halam{x}{\htau}{\instantiate{\dexp}{u}{\dexp'}}}\\
{\instantiate{\dexp}{u}{\dap{d_1}{d_2}}}
&=&
{\dapP{\instantiate{\dexp}{u}{d_1}}{\instantiate{\dexp}{u}{d_2}}}
\\
% {[\dexp_1 / x] \dinj{i}{e_2}}
% &=&
% {\dinj{i}{[\dexp_1 / x] e_2}}
% \\
% {[\dexp_1 / x] \dcase{e}{x}{e_x}{y}{e_y}}
% &=&
% {\dcase{[\dexp_1 / x]e}{x}{[\dexp_1 / x]e_x}{y}{[\dexp_1 / x]e_y}}
% \\
\instantiate{\dexp}{u}{\dehole{u}{\subst}{}}
&=&
[\instantiate{\dexp}{u}{\subst}]\dexp
\\
\instantiate{\dexp}{u}{\dehole{v}{\subst}{}}
&=&
\dehole{v}{\instantiate{\dexp}{u}{\subst}}{}
& \text{when $u \neq v$}
\\
\instantiate{\dexp}{u}{\dhole{\dexp'}{u}{\subst}{}}
&=&
[\instantiate{\dexp}{u}{\subst}]\dexp
\\
\instantiate{\dexp}{u}{\dhole{\dexp'}{v}{\subst}{}}
&=&
\dhole{\instantiate{\dexp}{u}{\dexp'}}{v}{\instantiate{\dexp}{u}{\subst}}{}
& \text{when $u \neq v$}
\\
%\end{array}
%\]
%\[
%\begin{array}{lcll}
%
\instantiate{\dexp}{u}{\dcasttwo{\dexp'}{\htau}{\htau'}}
&=&
\dcasttwo{(\instantiate{\dexp}{u}{\dexp'})}{\htau}{\htau'}
\\
\instantiate{\dexp}{u}{\dcastfail{\dexp'}{\htau_1}{\htau_2}}
&=&
\dcastfail{(\instantiate{\dexp}{u}{\dexp'})}{\htau_1}{\htau_2}\\
\instantiate{\dexp}{u}{\cdot} 
&=&
\cdot\\
\instantiate{\dexp}{u}{\sigma, \dexp'/x} 
&=&
\instantiate{\dexp}{u}{\sigma}, \instantiate{\dexp}{u}\dexp'/x
%% {[\dexp_1 / x] \dcast{\htau}{\dexp}}
%% &=&
%% \dcast{\htau}{[\dexp_1 / x] \dexp}
\end{array}
\]
\CaptionLabel{Hole Filling}{fig:substitution}
\vspace{-8px}
\end{figure}


%%%%%%%%% New par
The following theorem characterizes the static behavior of hole filling.
\begin{thm}[Filling]
  If $\hasType{\hDelta, \Dbinding{u}{\hGamma'}{\htau'}}{\hGamma}{\dexp}{\tau}$
  and $\hasType{\hDelta}{\hGamma'}{\dexp'}{\htau'}$
  then $\hasType{\hDelta}{\hGamma}{\instantiate{\dexp'}{u}{\dexp}}{\tau}$.
\end{thm}

Dynamically, the correctness of fill-and-resume depends on
the following \emph{commutativity} property: if there is some sequence of
steps that go from $d_1$ to $d_2$, then one can fill a hole in these
terms at \emph{either} the beginning or at the end of that step
sequence.
%
We write $\multiStepsTo{\dexp_1}{\dexp_2}$ for the reflexive,
transitive closure of stepping (see \ifarxiv Appendix~\ref{sec:multi-step}\else the \appendixName\fi).
% Appendix~\ref{sec:additional-defns}).
%
\begin{thm}[Commutativity]
  If $\hasType{\hDelta, \Dbinding{u}{\hGamma'}{\htau'}}{\emptyset}{\dexp_1}{\tau}$
  and $\hasType{\hDelta}{\hGamma'}{\dexp'}{\htau'}$ and $\multiStepsTo{\dexp_1}{\dexp_2}$
  then $\multiStepsTo{\instantiate{\dexp'}{u}{\dexp_1}}
                     {\instantiate{\dexp'}{u}{\dexp_2}}$.
\end{thm}
%
The key idea is that we can resume evaluation by replaying the substitutions that were recorded in the closures and then taking the necessary ``catch up'' steps that evaluate the hole fillings that now appear.
%
The caveat is that resuming from $\instantiate{\dexp'}{u}{d_2}$ will not
reduce sub-expressions in the same order as a ``fresh'' eager left-to-right
reduction sequence starting from $\instantiate{\dexp'}{u}{d_1}$ so 
filling commutes with reduction only for 
languages where evaluation order ``does not matter'', e.g. pure functional languages like \HazelnutLive.\footnote{There are various standard ways to formalize this intuition, e.g. by stating a suitable confluence property. 
For the sake of space, we review confluence in \ifarxiv Appendix \ref{sec:confluence}\else the \appendixName\fi.} Languages with non-commutative effects do not enjoy this property.
% Appendix~\ref{sec:confluence}.)

%% Everyone knows what confluence means, or they should.  We don't need to cite Church :)
%%

We describe the proof, which is straightforward but involves a
number of lemmas and definitions, in \ifarxiv Appendix~\ref{sec:hole-filling}\else the \appendixName\fi.
%Appendix~\ref{sec:hole-filling}.
%
In particular, care is needed to handle the situation where a
now-filled non-empty hole had taken a step in the original evaluation trace.

% \matt{We could (should?) move the next paragraph to the appendix,
% before/after the proofs mentioned above;
% its not really adding much to the discussion above,
% but may be interesting to the very studious reader/prover; also, leaving it here also gives a lazy/cranky reviewer a nice weapon to hit us with}

We do not separately define hole filling in the external language (i.e. we
consider a change to an external expression to be a hole filling if the new
elaboration differs from the previous elaboration up to hole filling).  In
practice, it may be useful to cache more than one recent edit state to take
full advantage of hole filling.  As an example, consider two edits, the
first filling a hole~$u$ with the number~$2$, and the next applying
operator~$+$, resulting in $2 + \dehole{v}{\sigma}{}$.
%
This second edit is not a hole filling edit with respect to the
immediately preceding edit state, $2$, but it can be understood as filling
hole $u$ from two states back with $2 + \dehole{v}{\sigma}{}$.

%%%%%%%%%%%%%%%%%%%%%%%%%%%%%%%%%%%%%%%%%%%%%%%%%%%%%%%%%%%%%%%%%%%%%%%%%%%5

Hole filling also allows us to give a contextual modal interpretation
to lab notebook cells  like those of
Jupyter/IPython \cite{PER-GRA:2007} (and read-eval-print loops as a restricted case
where edits to previous cells are impossible).
%
Each cell can be understood as a series of \li{let}
bindings ending implicitly in a hole, which is filled by the next cell.
The live environment in the subsequent cell is exactly the hole environment of this implicit trailing hole.
%
Hole filling when a subsequent cell changes avoids recomputing the
environment from preceding cells, without relying on mutable state. Commutativity provides a reproducibility guarantee
missing from Jupyter/IPython, where editing and executing previous cells can cause the state to differ substantially from the state that would result when attempting to run the notebook from the top.
% TODO citation


\begin{comment}

\begin{theorem}[Maximum Informativity]
If the elaboration produces $t1$, and there exists another possible type
choice $t2$, then $t1 \sim t2$ and $t1 JOIN t2 = t1$
\end{theorem}\footnote{idea is that special casing the holes in EANEHole gives you ``the
most descriptive hole types'' for some sense of what that means -- they'd
all just be hole other wise. from Matt:
\begin{quote}
It sounds like we need a something akin to an abstract domain (a lattice),
where hole has the least information, and a fully-defined type (without
holes) has the most information.  You can imagine that this lattice really
expands the existing definition we have of type consistency, which is
merely the predicate that says whether two types are comparable
(“join-able”) in this lattice.  lattice join is the operation that goes
through the structure of two (consistent) types, and chooses the structure
that is more defined (i.e., non-hole, if given the choice between hole and
non-hole).

The rule choosenonhole below is the expansion of this consistency rule that
we already have (hole consistent with everything)
\end{quote}}

\begin{verbatim}
t not hole
-------------------- :: choose-non-hole
hole JOIN t  = t
\end{verbatim}
\begin{verbatim}
------------ :: hole-consistent-with-everything
hole ~ t
\end{verbatim}

\end{comment}

% \section{Implementation: Ring Abstraction}
\label{sec:implement}
\subsection{Distributed \mbox{$G_t$} in QMC Solver}
\label{distributedG4}
Before introducing the communication phase of the ring abstraction layer,
it is important to understand how the authors distributed the large device array $G_t$ across MPI ranks.
%
Original $G_t$ was compared, and $G^d_t$ versions were distributed
(Figure~\ref{fig:compare_original_distributed_g4}). 


In the original $G_t$ implementation, the measurements---which were computed by matrix-matrix multiplication---are distributed statically and independently over the MPI ranks to avoid
inter-node communications. Each MPI rank keeps its partial copy of $G_{t,i}$ to accumulate 
measurements within a rank, where $i$ is the rank index. 
After all the measurements are finished, a reduction step is 
taken to accumulate $G_{t,i}$ across all MPI ranks into a final and complete
$G_t$ in the root MPI rank. The size of the $G_{t,i}$ in each rank is 
the same size as the final and complete $G_t$. 

With the distributed $G^d_t$ implementation, this large device array 
$G_t$ was evenly partitioned across all MPI ranks; each portion of it is local to each MPI rank.
%
Instead of keeping its partial copy of $G_t$, 
each rank now keeps an instance of $G^d_{t,i}$ to accumulate measurements 
of a portion or sub-slice of the final and complete $G_t$, where the notation
$d$ in $G^d_t$  refers to the distributed version, and $i$ means the $i$-th rank.
%
The $G^d_{t,i}$ size in each rank is 
reduced to $1/p$ of the size of the final and complete $G_t$, comparing the same configuration 
in original $G_t$ implementation, where $p$ is the number of MPI ranks used. 
%
For example, in Figure~\ref{fig:distributed_g4}, there are four ranks, and rank $i$
now only keeps $G^d_{t,i}$, which is one-fourth the size of the original $G_t$ array size.
% and contains values indexing from range of $[0, ..., N/4)$ in original $G_t$ array where $N$ is the 
% number of entries in  $G_t$  when viewed as a one-dimensional array.

To compute the final and complete $G^d_{t,i}$ for the distributed $G^d_t$ implementation, 
each rank must see every $G_{\sigma,i}$ from all ranks. 
%
In other words, each rank must broadcast the
locally generated $G_{\sigma,i}$ to the remainder of the other ranks at every measurement step. 
%
To efficiently perform this ``all-to-all'' broadcast, a ring abstraction layer was built (Section. \ref{section:ring_algorithm}), which circulates
all $G_{\sigma,i}$ across all ranks.

% over high-speed GPUs interconnect (GPUDirect RDMA) to facilitate the communication phase.

% \begin{figure}
% \centering
% \subfloat[Original $G_t$ implementation.]
%     {\includegraphics[width=\columnwidth]{original_g4.pdf}}\label{fig:original_g4}

% \subfloat[Distributed $G_t$ implementation.]
%     {\includegraphics[width=0.99\columnwidth]{distributed_g4.pdf} \label{fig:distributed_g4}}

% \caption{Comparison of the original $G_t$ vs. the distributed $G^d_t$ implementation. Each rank contains one GPU resource.}
% \label{fig:compare_original_distributed_g4} 
% \end{figure} 

\begin{figure}
\centering
     \begin{subfigure}[b]{\columnwidth}
         \centering
         \includegraphics[width=\textwidth]{images/original_g4.pdf}
         \caption{Original $G_t$ implementation.}
         \label{fig:original_g4}
     \end{subfigure}
     
    \begin{subfigure}[b]{\columnwidth}
         \centering
         \includegraphics[width=\textwidth]{images/distributed_g4.pdf}
         \caption{Distributed $G_t$ implementation.}
         \label{fig:distributed_g4}
     \end{subfigure}
     
\caption{Comparison of the original $G_t$ vs. the distributed $G^d_t$ implementation. Each rank contains one GPU resource.}
\label{fig:compare_original_distributed_g4}
\end{figure}

\subsection{Pipeline Ring Algorithm}
\label{section:ring_algorithm}
A pipeline ring algorithm was implemented that broadcasts the $G_{\sigma}$ 
array circularly during every measurement. 
%
The algorithm (Algorithm \ref{alg:ring_algorithm_code}) is 
visualized in Figure~\ref{fig:ring_algorithm_figure}.

\begin{algorithm}
\SetAlgoLined
    generateGSigma(gSigmaBuf)\; \label{lst:line:generateG2}
    updateG4(gSigmaBuf)\;       \label{lst:line:updateG4}
    %\texttt{\\}
    {$i\leftarrow 0$}\;         \label{lst:line:initStart}
    {$myRank \leftarrow worldRank$}\;          \label{lst:line:initRankId}
    {$ringSize \leftarrow mpiWorldSize$}\;      \label{lst:line:initRingSize}
    {$leftRank \leftarrow (myRank - 1 + ringSize) \: \% \: ringSize $}\;
    {$rightRank \leftarrow (myRank + 1 + ringSize) \: \% \: ringSize $}\;
    sendBuf.swap(gSigmaBuf)\;           \label{lst:line:initEnd}
    \While{$i< ringSize$}{
        MPI\_Irecv(recvBuf, source=leftRank, tag = recvTag, recvRequest)\; \label{lst:line:Irecv}
        MPI\_Isend(sendBuf, source=rightRank, tag = sendTag, sendRequest)\; \label{lst:line:Isend}
        MPI\_Wait(recvRequest)\;        \label{lst:line:recevBuffWait}
        
        updateG4(recvBuf)\;             \label{lst:line:updateG4_loop}
        
        MPI\_Wait(sendRequest)\;        \label{lst:line:sendBuffWait}
        
        sendBuf.swap(recvBuf)\;         \label{lst:line:swapBuff}
        i++\;
    }
\caption{Pipeline ring algorithm}
\label{alg:ring_algorithm_code}
\end{algorithm}

\begin{figure}
	\centering
	\includegraphics[width=\columnwidth, trim=0 5cm 0 0, clip]{images/ring_algorithm.pdf}
	\caption{Workflow of ring algorithm per iteration. }
	\label{fig:ring_algorithm_figure}
\end{figure}

At the start of every new measurement, a single-particle Green's function $G_{\sigma}$
 (Line~\ref{lst:line:generateG2}) is generated 
and then used to update $G^d_{t,i}$ (Line~\ref{lst:line:updateG4})
via the formula in Eq.~(\ref{eq:G4}).
%
% Different from original method that performs 
% full matrix-matrix multiplication (Equation~(\ref{eq:G4})), the current ring algorithm only performs partial
% matrix-matrix multiplication that contributes to $G^d_{t,i}$, a subslice of the final and complete $G_t$.
%
Between Lines \ref{lst:line:initStart} to \ref{lst:line:initEnd}, the algorithm 
initializes the indices
of left and right neighbors and prepares the sending message buffer from the
previously generated $G_{\sigma}$ buffer. 
%
The processes are organized as a ring so that the first and last rank are considered to be neighbors to each other. 
%
A \textit{swap} operation is used to avoid unnecessary memory copies for \textit{sendBuf} preparation.
%
A walker-accumulator thread allocates an additional \textit{recvBuf} buffer of the same size 
as \textit{gSigmaBuf} to hold incoming 
\textit{gSigmaBuf} buffer from \textit{leftRank}. 

The \textit{while} loop is the core part of the pipeline ring algorithm. 
%
For every iteration, each thread in a rank 
receives a $G_{\sigma}$ buffer from its left neighbor rank and sends a $G_{\sigma}$ buffer to its right neighbor rank. 
A synchronization step (Line~\ref{lst:line:recevBuffWait}) is performed
afterward to ensure that each rank receives a new buffer to update the 
local $G^d_{t,i}$ (Line~\ref{lst:line:updateG4_loop}). 
%
Another synchronization step
follows to ensure that all send requests are finalized 
(Line~\ref{lst:line:sendBuffWait}). Lastly, another \textit{swap} operation is used to exchange
content pointers between \textit{sendBuf} and \textit{recvBuf} to avoid unnecessary memory copy and prepare
for the next iteration of communication.
%
In the multi-threaded version (Section~\ref{subsec:multi-thread}), the thread of index, \textit{i}, only communicates with
	threads of index, \textit{i}, in neighbor ranks, and each thread allocates two buffers: \textit{sendBuff} and \textit{recvBuff}.

The \textit{while} loop will be terminated after $\mbox{\textit{ringSize}} - 1$ steps. By that time, 
each locally generated $G_{\sigma,i}$ will have traveled across all MPI ranks and
updated $G^d_{t,i}$ in all ranks. Eventually, each $G_{\sigma,i}$ reaches
to the left neighbor of its birth rank. For example, $G_{\sigma,0}$ generated from rank $0$ will end 
in last rank in the ring communicator.

Additionally, if the $G_t$ is too large to be stored in one node, 
it is optional to accumulate all $G^d_{t,i}$
at the end of all measurements. 
%
Instead, a parallel write into the file system could be taken.

\subsubsection{Sub-Ring Optimization.}

A sub-ring optimization strategy is further proposed to reduce message communication
times if the large device array $G_t$ can fit in fewer devices. 
%
The sub-ring algorithm is visualized in Figure~\ref{fig:subring_algorithm_figure}.

For the ring algorithm (Section~\ref{section:ring_algorithm}), the size of the ring communicator
(\textit{mpiWorldSize}) is set to the same size of the global \mbox{\texttt{MPI\_COMM\_WORLD}}, and thus the size of $G_t$ is equally 
distributed across all MPI ranks.

However, to complete the update to $G^d_{t,i}$ in one measurement, 
one $G_{\sigma,i}$
must travel \textit{mpiWorldSize} ranks. In total, 
there are \textit{mpiWorldSize} numbers of $G_{\sigma,i}$
being sent and received concurrently in one measurement 
in the global
\mbox{\texttt{MPI\_COMM\_WORLD}} 
communicator. If the size of $G^d_{t,i}$ is relatively small per rank, then this will cause high communication overhead.

If $G_t$ can be distributed and fitted in fewer devices, then a shorter travel distance is required 
for $G_{\sigma,i}$, thus reducing the communication overhead. One reduction
step was performed at the end of all measurements to accumulate $G^d_{t,s_i}$, 
where $s_i$ means $i$-th rank on the $s$-th sub-ring.

At the beginning of MPI initialization, the global \mbox{\texttt{MPI\_COMM\_WORLD}} was partitioned  into several new sub-ring communicators by using \mbox{\texttt{MPI\_Comm\_split}}. 
% where each new communicator represents conceptually a subring. 
The new
communicator information was passed to the DCA++ concurrency class by substituting the original global 
\mbox{\texttt{MPI\_COMM\_WORLD}} with this new communicator. Now, only a few minor modifications
are needed to transform the ring algorithm (Algorithm~\ref{alg:ring_algorithm_code})
to sub-ring Algorithm~\ref{alg:sub_ring_algorithm}. In Line~\ref{lst:line:initRankId}, \textit{myRank} is 
initialized to \textit{subRingRank} instead of \textit{worldRank}, where 
\textit{subRingRank} is the rank index in the local sub-ring communicator. 
%
In Line~\ref{lst:line:initRingSize}, \textit{ringSize} is initialized to \textit{subRingSize}
instead of \textit{mpiWorldSize}, where \textit{subRingSize} is the
size of the new communicator.
%
The general ring algorithm is a special case for the sub-ring algorithm because the
\textit{subRingSize} of the general ring algorithm is equal to \textit{mpiWorldSize}, and
there is only one sub-ring group throughout all MPI ranks.


\LinesNumberedHidden
\begin{algorithm}
    {$\mbox{\textit{myRank}} \leftarrow \mbox{\textit{subRingRank}}$}\;         
    {$\mbox{\textit{ringSize}} \leftarrow \mbox{\textit{subRingSize}}$}\;      
\caption{Modified ring algorithm to support sub-ring communication}
\label{alg:sub_ring_algorithm}
\end{algorithm}


\begin{figure}
	\centering
	\includegraphics[width=\columnwidth, trim=0 5cm 0 0, clip]{images/subring_alg.pdf}
	\caption{Workflow of sub-ring algorithm per iteration. Every consecutive $S$ rank forms a sub-ring communicator, 
	and no communication occurs between sub-ring communicators until all measurements are finished. Here, $S$ is the number of ranks in a sub-ring.}
	\label{fig:subring_algorithm_figure}
\end{figure}

\subsubsection{Multi-Threaded Ring Communication.}
\label{subsec:multi-thread}
To take advantage of the multi-threaded QMC model already in DCA++, 
multi-threaded ring communication support was further implemented in the ring algorithm.
%
Figure~\ref{fig:dca_overview} shows that in the original DCA++ method,
each walker-accumulator
thread in a rank is independent of each other, and all the threads in a 
rank synchronize only after all rank-local measurements are finished.
%
Moreover, during every measurement, each walker-accumulator thread
generates its own thread-private $G_{\sigma, i}$ to update $G_t$. 
%

The multi-threaded ring algorithm now allows concurrent message exchange so that threads of same rank-local thread index exchange their thread-private $G_{\sigma, i}$. 
%
Conceptually, there are $k$ parallel and independent rings, where $k$ 
is number of threads per rank, because threads of the same local thread ID
form a closed ring. 
%
For example, a thread of index $0$ in rank $0$ will send its $G_\sigma$ to 
the thread of index $0$ in rank $1$ and receive another $G_\sigma$ from thread index of $0$ 
from last rank in the ring algorithm.
%

The only changes in the ring algorithm are offsetting the tag values 
(\texttt{recvTag} and \texttt{sendTag}) by the thread index value. For example,
Lines~\ref{lst:line:Irecv} and ~\ref{lst:line:Isend} from 
Algorithm~\ref{alg:ring_algorithm_code} are modified to Algorithm~\ref{alg:multi_threaded_ring}.

\LinesNumberedHidden
\begin{algorithm}
        MPI\_Irecv(recvBuf, source=leftRank, tag = recvTag + threadId, recvRequest)\; 
        MPI\_Isend(sendBuf, source=rightRank, tag = sendTag + threadId, sendRequest)\;
\caption{Modified ring algorithm to support multi-threaded ring}
\label{alg:multi_threaded_ring}
\end{algorithm}

To efficiently send and receive $G_\sigma$, each thread
will allocate one additional \textit{recvBuff} to hold incoming 
\textit{gSigmaBuf} buffer from \textit{leftRank} and perform send/receive efficiently.
%
In the original DCA++ method, there are $k$ numbers of buffers of $G_\sigma$ 
size per rank, and in the multi-threaded ring method, there are $2k$
numbers of buffers of $G_\sigma$ size per rank, where $k$ is number of 
threads per rank.

\section{Related Work}
We focus on the setting of \emph{information-giving} for open-domain
human-robot collaboration. Information-giving was first explored
algorithmically by McCarthy~\cite{programswithcommonsense} in a
seminal 1959 paper on advice-takers. Much work in open-domain
collaboration focuses on the robot understanding goals given by the
human~\cite{talamadupula1}, whereas we focus on understanding
information given by the human, as in work on
advice-giving~\cite{odom2015learning} and commonsense
reasoning~\cite{corpp}.

\subsection{Adaptive Belief Representations}
Our work explores belief state representations that adapt to the
structure of the observations received by the robot. The work perhaps
most similar to ours is that of Lison et
al.~\cite{beliefsituationaware}, who acknowledge the importance of
information fusion and abstraction in uncertain environments. Building
on Markov Logic Networks~\cite{mln}, they describe a method for belief
refinement that 1) groups percepts likely to be referring to the same
object, 2) fuses information about objects based on these percept
groups, and 3) dynamically evolves the belief over time by combining
it with those in the past and future. They focus on beliefs about each
object in the environment, whereas our work focuses on combining
information about multiple objects, based on the structure of the
observations.

The notion of adaptive belief representations has also been explored
in domains outside robotics. For instance, Sleep~\cite{adaptivegrid}
applies this idea to an acoustic target-tracking setting. The belief
representation, which tracks potential locations of targets, can
expand to store additional information about the targets that may be
important for locating them, such as their acoustic power. The belief
can also contract to remove information that is deemed no longer
necessary. It would be difficult to update this factoring with
information linking multiple targets, whereas our method is
well-suited to incorporating such relational constraints.

\subsection{Factored Belief Representations for \pomdp s}
The more general problem of finding efficient belief representations
for \pomdp s is very well-studied. Boyen and Koller~\cite{boyenkoller}
were the first to provide a tractable method for belief propagation
and inference in a hidden Markov model or dynamic Bayesian
network. Their basic strategy is to first pick a computationally
tractable approximate belief representation (such as a factored one),
then after a belief update, fold the newly obtained belief into the
chosen approximate representation. This technique is a specific
application of the more general principle of \emph{assumed density
  filtering}~\cite{maybeck1982stochastic}. Although it seems that the
approximation error will build up and propagate, the authors show that
actually, the error remains bounded under reasonable assumptions. Our
work adopts a fluid notion of a factored belief representation, not
committing to one factoring.

Bonet and Geffner~\cite{bonet2014belief} introduce the idea of beam tracking for belief
tracking in a \pomdp. Their method leverages the fact that a problem
can be decomposed into projected subproblems, allowing the joint
distribution over state variables to be represented as a product over
factors. Then, the authors introduce an alternative decomposition over
beams, subsets of variables that are causally relevant to each
observable variable. This work shares with ours the idea of having the
structure of the belief representation not be fixed ahead of time. A
difference, however, is that the decomposition used in beam tracking
is informed by the structure of the underlying model, while ours is
informed by the structure of the observations and can thus efficiently
incorporate arbitrary relational constraints on the world state.

% \subsection{Markov Logic Networks}
% Markov Logic Networks (MLNs)~\cite{mln} are a sound and practical mechanism for
% combining probabilistic inference and first-order logic into a unified
% representation. An MLN is a template for a ground Markov network,
% consisting of a knowledge base (KB), a weight for each formula in the
% KB, and a set of constants. Thus, an MLN can be viewed as a soft
% version of a KB in which each formula has an associated measure of
% importance, and each world state has a corresponding likelihood based
% on the total importance of the formulas it satisfies.

% A key difference between MLNs and our work is that the structure of
% the MLN is fixed and defined by the formulas and constants. On the
% other hand, our system works online, changing the structure of the
% underlying representation dynamically based on new formulas and
% constraints that arise while the robot is acting in the world. This
% type of dynamic sensitivity is crucial for real-world systems, in
% which it would be difficult to find a fixed structure that could
% properly capture the full space of incoming information. Another
% difference is that our work folds the provided first-order logic
% observations into the belief representation dynamically rather than
% storing them explicitly (as is needed for inference with MLNs), making
% inference easier at the cost of slightly more expensive updates.
\mySection{Related Works and Discussion}{}
\label{chap3:sec:discussion}

In this section we briefly discuss the similarities and differences of the model presented in this chapter, comparing it with some related work presented earlier (Chapter \ref{chap1:artifact-centric-bpm}). We will mention a few related studies and discuss directly; a more formal comparative study using qualitative and quantitative metrics should be the subject of future work.

Hull et al. \citeyearpar{hull2009facilitating} provide an interoperation framework in which, data are hosted on central infrastructures named \textit{artifact-centric hubs}. As in the work presented in this chapter, they propose mechanisms (including user views) for controlling access to these data. Compared to choreography-like approach as the one presented in this chapter, their settings has the advantage of providing a conceptual rendezvous point to exchange status information. The same purpose can be replicated in this chapter's approach by introducing a new type of agent called "\textit{monitor}", which will serve as a rendezvous point; the behaviour of the agents will therefore have to be slightly adapted to take into account the monitor and to preserve as much as possible the autonomy of agents.

Lohmann and Wolf \citeyearpar{lohmann2010artifact} abandon the concept of having a single artifact hub \cite{hull2009facilitating} and they introduce the idea of having several agents which operate on artifacts. Some of those artifacts are mobile; thus, the authors provide a systematic approach for modelling artifact location and its impact on the accessibility of actions using a Petri net. Even though we also manipulate mobile artifacts, we do not model artifact location; rather, our agents are equipped with capabilities that allow them to manipulate the artifacts appropriately (taking into account their location). Moreover, our approach considers that artifacts can not be remotely accessed, this increases the autonomy of agents.

The process design approach presented in this chapter, has some conceptual similarities with the concept of \textit{proclets} proposed by Wil M. P. van der Aalst et al. \citeyearpar{van2001proclets, van2009workflow}: they both split the process when designing it. In the model presented in this chapter, the process is split into execution scenarios and its specification consists in the diagramming of each of them. Proclets \cite{van2001proclets, van2009workflow} uses the concept of \textit{proclet-class} to model different levels of granularity and cardinality of processes. Additionally, proclets act like agents and are autonomous enough to decide how to interact with each other.

The model presented in this chapter uses an attributed grammar as its mathematical foundation. This is also the case of the AWGAG model by Badouel et al. \citeyearpar{badouel14, badouel2015active}. However, their model puts stress on modelling process data and users as first class citizens and it is designed for Adaptive Case Management.

To summarise, the proposed approach in this chapter allows the modelling and decentralized execution of administrative processes using autonomous agents. In it, process management is very simply done in two steps. The designer only needs to focus on modelling the artifacts in the form of task trees and the rest is easily deduced. Moreover, we propose a simple but powerful mechanism for securing data based on the notion of accreditation; this mechanism is perfectly composed with that of artifacts. The main strengths of our model are therefore : 
\begin{itemize}
	\item The simplicity of its syntax (process specification language), which moreover (well helped by the accreditation model), is suitable for administrative processes;
	\item The simplicity of its execution model; the latter is very close to the blockchain's execution model \cite{hull2017blockchain, mendling2018blockchains}. On condition of a formal study, the latter could possess the same qualities (fault tolerance, distributivity, security, peer autonomy, etc.) that emanate from the blockchain;
	\item Its formal character, which makes it verifiable using appropriate mathematical tools;
	\item The conformity of its execution model with the agent paradigm and service technology.
\end{itemize}
In view of all these benefits, we can say that the objectives set for this thesis have indeed been achieved. However, the proposed model is perfectible. For example, it can be modified to permit agents to respond incrementally to incoming requests as soon as any prefix of the extension of a bud is produced. This makes it possible to avoid the situation observed on figure \ref{chap3:fig:execution-figure-4} where the associated editor is informed of the evolution of the subtree resulting from $C$ only when this one is closed. All the criticisms we can make of the proposed model in particular, and of this thesis in general, have been introduced in the general conclusion (page \pageref{chap5:general-conclusion}) of this manuscript.





%\clearpage
\bibliography{all.short}

\ifarxiv
\clearpage
\appendix
% !TEX root = hazelnut-dynamics.tex

\newcommand{\additionalDefnsSec}{Additional Definitions for Hazelnut Live}
\section{\protect\additionalDefnsSec} % don't like the all-caps thing that the template does, so protecting it from that
\label{sec:additional-defns}

\subsection{Substitution}
\label{sec:substitution}
\judgbox
  {[\dexp/x]\dexp' = \dexp''}
  {$\dexp''$ is obtained by substituting $\dexp$ for $x$ in $\dexp'$}
  %% {$\dexp''$ is the result of substituting $\dexp$ for $u$ in $\dexp'$}

\vspace{5pt}
\judgbox
  {[\dexp/x]\sigma = \sigma'}
  {$\sigma'$ is obtained by substituting $\dexp$ for $x$ in $\sigma$}
  %% {$\dexp''$ is the result of substituting $\dexp$ for $u$ in $\dexp'$}
\[
\begin{array}{lcll}
\substitute{\dexp}{x}{c}
&=&
c\\
\substitute{\dexp}{x}{x}
&=&
\dexp
\\%
\substitute{\dexp}{x}{y}
&=&
y
& \text{when $x \neq y$}
\\
\substitute{\dexp}{x}{\halam{x}{\htau}{\dexp'}}
&=&
\halam{x}{\htau}{\dexp'}\\
\substitute{\dexp}{x}{\halam{y}{\htau}{\dexp'}}
&=&
\halam{y}{\htau}{\substitute{\dexp}{x}{\dexp'}}
& \text{when $x \neq y$ and $y \notin \fvof{d}$}
\\
\substitute{\dexp}{x}{\dap{\dexp_1}{\dexp_2}}
&=&
\dap{(\substitute{\dexp}{x}{\dexp_1})}{\substitute{\dexp}{x}{\dexp_2}}
\\
\substitute{\dexp}{x}{\dehole{u}{\subst}{}}
&=&
\dehole{u}{\substitute{\dexp}{x}{\subst}}{}
\\
\substitute{\dexp}{x}{\dhole{\dexp'}{u}{\subst}{}}
&=&
\dhole{\substitute{\dexp}{x}{\dexp'}}{u}{\substitute{\dexp}{x}{\subst}}{}
\\
\substitute{\dexp}{x}{\dcasttwo{\dexp'}{\htau_1}{\htau_2}}
&=&
\dcasttwo{(\substitute{\dexp}{x}{\dexp'})}{\htau_1}{\htau_2}
\\
\substitute{\dexp}{x}{\dcastfail{\dexp'}{\htau_1}{\htau_2}}
&=&
\dcastfail{(\substitute{\dexp}{x}{\dexp'})}{\htau_1}{\htau_2}
\\[6px]
\substitute{\dexp}{x}{\cdot}
&=&
\cdot\\
\substitute{\dexp}{x}{\sigma, d/y}
&=&
\substitute{\dexp}{x}{\sigma}, \substitute{\dexp}{x}{d}/y
%% {[\dexp_1 / x] \dcast{\htau}{\dexp}}
%% &=&
%% \dcast{\htau}{[\dexp_1 / x] \dexp}
\end{array}
\]

\vspace{5pt}
\begin{lem}[Substitution] \label{thm:substitution}~
\begin{enumerate}[nolistsep]
\item If $\hasType{\hDelta}{\hGamma, x : \htau'}{d}{\htau}$ and $\hasType{\hDelta}{\hGamma}{d'}{\htau'}$ then $\hasType{\hDelta}{\hGamma}{[d'/x]d}{\htau}$.
\item If $\hasType{\hDelta}{\hGamma, x : \htau'}{\sigma}{\hGamma'}$ and $\hasType{\hDelta}{\hGamma}{d'}{\htau'}$ then $\hasType{\hDelta}{\hGamma}{[d'/x]\sigma}{\hGamma'}$.
\end{enumerate}
\end{lem}
% \begin{proof} By rule induction on the first assumption in each case. The conclusion follows from the definition of substitution in each case. \end{proof}

\subsection{Canonical Forms}
\label{sec:canonical-forms}

\begin{lem}[Canonical Value Forms]\label{thm:canonincal-value-forms}
  If $\hasType{\hDelta}{\emptyset}{\dexp}{\htau}$ and $\isValue{\dexp}$
  then $\htau\neq\tehole$ and
  \begin{enumerate}[nolistsep]
    \item If $\htau=b$ then $\dexp=c$.
    \item If $\htau=\tarr{\htau_1}{\htau_2}$
          then $\dexp=\halam{x}{\htau_1}{\dexp'}$
          where $\hasType{\hDelta}{x : \htau_1}{\dexp'}{\htau_2}$.
  \end{enumerate}
\end{lem}

\begin{lem}[Canonical Boxed Forms]\label{thm:canonical-boxed-forms}
  If $\hasType{\hDelta}{\emptyset}{\dexp}{\htau}$ and $\isBoxedValue{\dexp}$
  then
  \begin{enumerate}[nolistsep]
    \item If $\htau=b$ then $\dexp=c$.
    \item If $\htau=\tarr{\htau_1}{\htau_2}$ then either
      \begin{enumerate}
        \item
          $\dexp=\halam{x}{\htau_1}{\dexp'}$
          where $\hasType{\hDelta}{x : \htau_1}{\dexp'}{\htau_2}$, or
        \item
          $\dexp=\dcasttwo{\dexp'}{\tarr{\htau_1'}{\htau_2'}}{\tarr{\htau_1}{\htau_2}}$
          where $\tarr{\htau_1'}{\htau_2'}\neq\tarr{\htau_1}{\htau_2}$
          and $\hasType{\hDelta}{\emptyset}{\dexp'}{\tarr{\htau_1'}{\htau_2'}}$.
      \end{enumerate}
    \item If $\htau=\tehole$
          then $\dexp=\dcasttwo{\dexp'}{\htau'}{\tehole}$
          where $\isGround{\htau'}$
          and $\hasType{\hDelta}{\emptyset}{\dexp'}{\htau'}$.
  \end{enumerate}
\end{lem}

\begin{lem}[Canonical Indeterminate Forms]
  If $\hasType{\hDelta}{\emptyset}{\dexp}{\htau}$
  and $\isIndet{\dexp}$
  then
  \begin{enumerate}[nolistsep]
    \item 
      If $\htau = b$ then either
        \begin{enumerate}
          \item $\dexp = \dehole{u}{\subst}{}$ where $\Dbinding{u}{\Gamma}{b} \in \hDelta$ and $\hasType{\hDelta}{\emptyset}{\subst}{\hGamma}$, or
          \item $\dexp = \dhole{\dexp'}{u}{\subst}{}$ where $\hasType{\hDelta}{\emptyset}{\dexp'}{\htau'}$ and $\isFinal{\dexp'}$ and $\Dbinding{u}{\Gamma}{b} \in \hDelta$ and $\hasType{\hDelta}{\emptyset}{\subst}{\hGamma}$, or
          \item $\dexp = \dap{\dexp_1}{\dexp_2}$ where $\hasType{\hDelta}{\emptyset}{\dexp_1}{\tarr{\htau_2}{b}}$ and $\hasType{\hDelta}{\emptyset}{\dexp_2}{\htau_2}$ and $\isIndet{\dexp_1}$ and $\isFinal{\dexp_2}$ and $\dexp_1 \neq \dcasttwo{\dexp_1'}{\tarr{\htau_3}{\htau_4}}{\tarr{\htau_3'}{\htau_4'}}$ for any $\dexp_1', \htau_3, \htau_4, \htau_3', \htau_4'$, or 
          \item $\dexp = \dcasttwo{\dexp'}{\tehole}{b}$ where $\hasType{\hDelta}{\emptyset}{\dexp'}{\tehole}$ and $\isIndet{\dexp'}$ and $\dexp' \neq \dcasttwo{\dexp''}{\htau'}{\tehole}$ for any $\dexp'', \htau'$, or 
          \item $\dexp = \dcastfail{\dexp'}{\htau'}{b}$ where $\hasType{\hDelta}{\emptyset}{\dexp'}{\htau'}$ and $\isGround{\htau'}$ and $\htau' \neq b$.
        \end{enumerate}
    \item 
      If $\htau = \tarr{\htau_1}{\htau_2}$ then either 
        \begin{enumerate}
          \item $\dexp = \dehole{u}{\subst}{}$ where $\Dbinding{u}{\Gamma}{\tarr{\htau_1}{\htau_2}} \in \hDelta$ and $\hasType{\hDelta}{\emptyset}{\subst}{\hGamma}$, or
          \item $\dexp = \dhole{\dexp'}{u}{\subst}{}$ where $\hasType{\hDelta}{\emptyset}{\dexp'}{\htau'}$ and $\isFinal{\dexp'}$ and $\Dbinding{u}{\Gamma}{\tarr{\htau_1}{\htau_2}} \in \hDelta$ and $\hasType{\hDelta}{\emptyset}{\subst}{\hGamma}$, or
          \item $\dexp = \dap{\dexp_1}{\dexp_2}$ where $\hasType{\hDelta}{\emptyset}{\dexp_1}{\tarr{\htau_2'}{\tarr{\htau_1}{\htau_2}}}$ and $\hasType{\hDelta}{\emptyset}{\dexp_2}{\htau_2'}$ and $\isIndet{\dexp_1}$ and $\isFinal{\dexp_2}$ and $\dexp_1 \neq \dcasttwo{\dexp_1'}{\tarr{\htau_3}{\htau_4}}{\tarr{\htau_3'}{\htau_4'}}$ for any $\dexp_1', \htau_3, \htau_4, \htau_3', \htau_4'$, or 
          \item $\dexp = \dcasttwo{\dexp'}{\tarr{\htau_1'}{\htau_2'}}{\tarr{\htau_1}{\htau_2}}$ where $\hasType{\hDelta}{\emptyset}{\dexp'}{\tarr{\htau_1'}{\htau_2'}}$ and $\isIndet{\dexp'}$ and $\tarr{\htau_1'}{\htau_2'} \neq \tarr{\htau_1}{\htau_2}$, or 
          \item $\dexp = \dcasttwo{\dexp'}{\tehole}{\tarr{\tehole}{\tehole}}$ and $\htau_1 = \tehole$ and $\htau_2 = \tehole$ where $\hasType{\hDelta}{\emptyset}{\dexp'}{\tehole}$ and $\isIndet{\dexp'}$ and $\dexp' \neq \dcasttwo{\dexp''}{\htau'}{\tehole}$ for any $\dexp'', \htau'$, or 
          \item $\dexp = \dcastfail{\dexp'}{\htau'}{\tarr{\tehole}{\tehole}}$ and $\htau_1 = \tehole$ and $\htau_2 = \tehole$ where $\hasType{\hDelta}{\emptyset}{\dexp'}{\htau'}$ and $\isGround{\htau'}$ and $\htau' \neq \tarr{\tehole}{\tehole}$.
        \end{enumerate}
    \item 
      If $\htau = \tehole$ then either 
        \begin{enumerate}
          \item $\dexp = \dehole{u}{\subst}{}$ where $\Dbinding{u}{\Gamma}{\tehole} \in \hDelta$ and $\hasType{\hDelta}{\emptyset}{\subst}{\hGamma}$, or
          \item $\dexp = \dhole{\dexp'}{u}{\subst}{}$ where $\hasType{\hDelta}{\emptyset}{\dexp'}{\htau'}$ and $\isFinal{\dexp'}$ and $\Dbinding{u}{\Gamma}{\tehole} \in \hDelta$ and $\hasType{\hDelta}{\emptyset}{\subst}{\hGamma}$, or
          \item $\dexp = \dap{\dexp_1}{\dexp_2}$ and $\hasType{\hDelta}{\emptyset}{\dexp_1}{\tarr{\htau_2}{\tehole}}$ and $\hasType{\hDelta}{\emptyset}{\dexp_2}{\htau_2}$ and $\isIndet{\dexp_1}$ and $\isFinal{\dexp_2}$ and $\dexp_1 \neq \dcasttwo{\dexp_1'}{\tarr{\htau_3}{\htau_4}}{\tarr{\htau_3'}{\htau_4'}}$ for any $\dexp_1', \htau_3, \htau_4, \htau_3', \htau_4'$, or 
          \item $\dexp = \dcasttwo{\dexp'}{\htau'}{\tehole}$ where $\hasType{\hDelta}{\emptyset}{\dexp'}{\htau'}$ and $\isGround{\htau'}$ and $\isIndet{\dexp'}$.
        \end{enumerate}
  \end{enumerate}
  % \begin{enumerate}[nolistsep]
  %   \item
  %     $\dexp=\dehole{u}{\subst}{}$
  %     and $\Dbinding{u}{\Gamma'}{\htau}\in\hDelta$, or
  %   \item
  %     $\dexp=\dhole{\dexp'}{u}{\subst}{}$
  %     and $\isFinal{\dexp'}$
  %     and $\hasType{\hDelta}{\emptyset}{\dexp'}{\htau'}$
  %     and $\Dbinding{u}{\Gamma'}{\htau}\in\hDelta$, or
  %   \item
  %     $\dexp=\dap{\dexp_1}{\dexp_2}$
  %     and $\hasType{\hDelta}{\emptyset}{\dexp_1}{\tarr{\htau_2}{\htau}}$
  %     and $\hasType{\hDelta}{\emptyset}{\dexp_2}{\htau_2}$
  %     and $\isIndet{\dexp_1}$
  %     and $\isFinal{\dexp_2}$
  %     and $\dexp_1\neq\dcasttwo{\dexp_1}{\tarr{\htau_3}{\htau_4}}
  %                                       {\tarr{\htau_3'}{\htau_4'}}$, or
  %   %% \item
  %   %%   \begin{enumerate}
  %   %%     \item blah
  %   %%     \item blah
  %   %%     \item blah
  %   %%   \end{enumerate}
  %   \item
  %     $\htau=b$
  %     and $\dexp=\dcasttwo{\dexp'}{\tehole}{b}$
  %     and $\isIndet{\dexp'}$
  %     and $\dexp'\neq\dcasttwo{\dexp''}{\htau'}{\tehole}$, or
  %   \item
  %     $\htau=b$
  %     and $\dexp=\dcastfail{\dexp'}{\htau'}{b}$
  %     and $\isGround{\htau'}$
  %     and $\htau'\neq{b}$
  %     and $\hasType{\hDelta}{\emptyset}{\dexp'}{\htau'}$, or
  %   \item
  %     $\htau=\tarr{\htau_{11}}{\htau_{12}}$
  %     and $\dexp=\dcasttwo{\dexp'}{\tarr{\htau_1}{\htau_2}}
  %                                 {\tarr{\htau_{11}}{\htau_{12}}}$
  %     and $\isIndet{\dexp'}$
  %     and $\tarr{\htau_1}{\htau_2}\neq\tarr{\htau_{11}}{\htau_{12}}$, or
  %   \item
  %     $\htau=\tarr{\tehole}{\tehole}$
  %     %% $\htau=\tarr{\htau_{11}}{\htau_{12}}$
  %     %% and $\htau_{11}=\tehole$
  %     %% and $\htau_{12}=\tehole$
  %     and $\dexp=\dcastthree{\dexp'}{\tehole}{\tehole}{\tehole}$
  %     and $\isIndet{\dexp'}$
  %     and $\dexp'\neq\dcasttwo{\dexp''}{\htau'}{\tehole}$, or
  %   \item
  %     $\htau=\tarr{\tehole}{\tehole}$
  %     %% $\htau=\tarr{\htau_{11}}{\htau_{12}}$
  %     %% and $\htau_{11}=\tehole$
  %     %% and $\htau_{12}=\tehole$
  %     and $\dexp=\dcastfail{\dexp'}{\htau'}{\tarr{\tehole}{\tehole}}$
  %     %% and $\dexp=\dcastfail{\dexp'}{\htau'}{\tarr{\htau_{11}}{\htau_{12}}}$
  %     and $\htau'\neq\htau$
  %     and $\isGround{\htau'}$
  %     and $\isIndet{\dexp'}$
  %     and $\hasType{\hDelta}{\emptyset}{\dexp'}{\htau'}$, or
  %   \item
  %     $\htau=\tehole$
  %     and $\dexp=\dcasttwo{\dexp'}{\htau'}{\tehole}$
  %     and $\isGround{\htau'}$
  %     and $\isIndet{\dexp'}$.
  % \end{enumerate}
\end{lem}

% The proofs for all three of these theorems follow by straightforward rule induction.

% No weakening for Gammas in Delta:
% If $\hasType{\Delta, \Dbinding{u}{\hGamma}{\tau}}{\hGamma'}{d}{\tau'}$ then $\hasType{\Delta, \Dbinding{u}{\hGamma, x : \tau''}{\tau}}{\hGamma'}{d}{\tau'}$.

\subsection{Complete Programs}
\label{sec:complete-programs}

% !TEX root = hazelnut-dynamics.tex
\begin{figure}[h]
\judgbox{\isComplete{\htau}}{$\htau$ is complete}
\begin{mathpar}
\inferrule[TCBase]{ }{
  \isComplete{b}
}

\inferrule[TCArr]{
  \isComplete{\htau_1}\\
  \isComplete{\htau_2}
}{
  \isComplete{\tarr{\htau_1}{\htau_2}}
}
\end{mathpar}

\vsepRule

\judgbox{\isComplete{\hexp}}{$\hexp$ is complete}
\begin{mathpar}
\inferrule[ECVar]{ }{
  \isComplete{x}
}

\inferrule[ECConst]{ }{
  \isComplete{c}
}

\inferrule[ECLam1]{
  \isComplete{\htau}\\
  \isComplete{\hexp}
}{
  \isComplete{\halam{x}{\htau}{\hexp}}
}

\inferrule[ECLam2]{
  \isComplete{\hexp}
}{
  \isComplete{\hlam{x}{\hexp}}
}

\inferrule[ECAp]{
  \isComplete{\hexp_1}\\
  \isComplete{\hexp_2}
}{
  \isComplete{\hap{\hexp_1}{\hexp_2}}
}

\inferrule[ECAsc]{
  \isComplete{\hexp}\\
  \isComplete{\htau}
}{
  \isComplete{\hexp : \htau}
}
\end{mathpar}

\vsepRule


\judgbox{\isComplete{\dexp}}{$\dexp$ is complete}
\begin{mathpar}
\inferrule[DCVar]{ }{
  \isComplete{x}
}

\inferrule[DCConst]{ }{
  \isComplete{c}
}

\inferrule[DCLam]{
  \isComplete{\htau}\\
  \isComplete{\dexp}
}{
  \isComplete{\dlam{x}{\htau}{\dexp}}
}

\inferrule[DCAp]{
  \isComplete{\dexp_1}\\
  \isComplete{\dexp_2}
}{
  \isComplete{\dap{\dexp_1}{\dexp_2}}
}

\inferrule[DCCast]{
  \isComplete{\dexp}\\
  \isComplete{\htau_1}\\
  \isComplete{\htau_2}
}{
  \isComplete{\dcasttwo{\dexp}{\htau_1}{\htau_2}}
}
\end{mathpar}

\caption{Complete types, external expressions, and internal expressions}
\label{fig:complete}
\end{figure}

We define $\isComplete{\hGamma}$ as follows.
\begin{defn}[Typing Context Completeness]
$\isComplete{\hGamma}$ iff for each $x : \htau \in \hGamma$, we have $\isComplete{\htau}$.
\end{defn}

When two types are complete and consistent, they are equal.

\begin{lem}[Complete Consistency]\label{lem:complete-consistency} If $\tconsistent{\htau_1}{\htau_2}$ and $\isComplete{\htau_1}$ and $\isComplete{\htau_2}$ then $\htau_1 = \htau_2$.
\end{lem}
\begin{proof} By straightforward rule induction. \end{proof}

This implies that in a well-typed and complete internal expression, every cast is 
an identity cast.

\begin{lem}[Complete Casts] If $\hasType{\hGamma}{\hDelta}{\dcasttwo{\dexp}{\htau_1}{\htau_2}}{\htau_2}$ and $\isComplete{\dcasttwo{\dexp}{\htau_1}{\htau_2}}$ then $\htau_1 = \htau_2$. \end{lem}
\begin{proof} By straightforward rule induction and Lemma~\ref{lem:complete-consistency}. \end{proof}

\subsection{Multiple Steps}
\label{sec:multi-step}

\begin{figure}[h]
\judgbox{\multiStepsTo{\dexp}{\dexp'}}{$\dexp$ multi-steps to $\dexp'$}
\begin{mathpar}
\inferrule[MultiStepRefl]{~}{
  \multiStepsTo{\dexp}{\dexp}
}

\inferrule[MultiStepSteps]{
  \stepsToD{}{\dexp}{\dexp'}\\
  \multiStepsTo{\dexp'}{\dexp''}
}{
  \multiStepsTo{\dexp}{\dexp''}
}
\end{mathpar}
\CaptionLabel{Multi-Step Transitions}{fig:multi-step}
\end{figure}


\subsection{Hole Filling}\label{sec:hole-filling}
\begin{lem}[Filling] ~
  \begin{enumerate}[nolistsep]
  \item If $\hasType{\hDelta, \Dbinding{u}{\hGamma'}{\htau'}}{\hGamma}{\dexp}{\tau}$
  and $\hasType{\hDelta}{\hGamma'}{\dexp'}{\htau'}$
  then $\hasType{\hDelta}{\hGamma}{\instantiate{\dexp'}{u}{\dexp}}{\tau}$.
  \item If $\hasType{\hDelta, \Dbinding{u}{\hGamma'}{\htau'}}{\hGamma}{\sigma}{\hGamma''}$
  and $\hasType{\hDelta}{\hGamma'}{\dexp'}{\htau'}$
  then $\hasType{\hDelta}{\hGamma}{\instantiate{\dexp'}{u}{\sigma}}{\hGamma''}$.
  \end{enumerate}
\end{lem}
\begin{proof}
In each case, we proceed by rule induction on the first assumption, appealing to the Substitution Lemma as necessary.
\end{proof}

To prove the Commutativity theorem, we need the auxiliary definitions in Fig.~\ref{fig:evalctx-filling}, which lift hole filling to evaluation contexts taking care to consider the special situation where the mark is inside the hole that is being filled.
% !TEX root = hazelnut-dynamics.tex

\begin{figure}[h]
\judgbox{\inhole{u}{\evalctx}}{The mark in $\evalctx$ is inside non-empty hole closure $u$}
\begin{mathpar}
\inferrule[InHoleNEHole]{~}{
  \inhole{u}{\dhole{\evalctx}{u}{\subst}{}}
}

\inferrule[InHoleAp1]{
  \inhole{u}{\evalctx}
}{
  \inhole{u}{\hap{\evalctx}{\dexp}}
}

\inferrule[InHoleAp2]{
  \inhole{u}{\evalctx}
}{
  \inhole{u}{\hap{\dexp}{\evalctx}}
}
\\
\inferrule[InHoleCast]{
  \inhole{u}{\evalctx}
}{
  \inhole{u}{\dcasttwo{\evalctx}{\htau_1}{\htau_2}}
}

\inferrule[InHoleFailedCast]{
  \inhole{u}{\evalctx}
}{
  \inhole{u}{\dcastfail{\evalctx}{\htau_1}{\htau_2}}
}
\end{mathpar}

\vsepRule

\judgbox{\instantiate{d}{u}{\evalctx} = \evalctx'}{$\evalctx'$ is obtained by filling hole $u$ in $\evalctx$ with $d$}
\begin{mathpar}
\inferrule[EFillMark]{~}{
  \instantiate{d}{u}{\evalhole} = \evalhole
}

\inferrule[EFillAp1]{
  \instantiate{d}{u}{\evalctx} = \evalctx'
}{
  \instantiate{d}{u}{\hap{\evalctx}{d_2}} = \hap{\evalctx'}{
  	\instantiate{d}{u}{d_2}}
}

\inferrule[EFillAp2]{
  \instantiate{d}{u}{\evalctx} = \evalctx'
}{
  \instantiate{d}{u}{\hap{d_1}{\evalctx}} = \hap{(
  	\instantiate{d}{u}{d_1})}{\evalctx'}
}

\inferrule[EFillNEHole]{
  u \neq v\\
  \instantiate{d}{u}{\evalctx}={\evalctx'}
}{
  \instantiate{d}{u}{\dhole{\evalctx}{v}{\sigma}{}} = \dhole{\evalctx'}{v}{\instantiate{d}{u}{\sigma}{}}{}
}

\inferrule[EFillCast]{
	\instantiate{d}{u}{\evalctx} = \evalctx'
}{
	\instantiate{d}{u}{\dcasttwo{\evalctx}{\htau_1}{\htau_2}} = 
	\dcasttwo{\evalctx'}{\htau_1}{\htau_2}
}

\inferrule[EFillFailedCast]{
	\instantiate{d}{u}{\evalctx} = \evalctx'
}{
	\instantiate{d}{u}{\dcastfail{\evalctx}{\htau_1}{\htau_2}} = 
	\dcastfail{\evalctx'}{\htau_1}{\htau_2}
}
\end{mathpar}
\CaptionLabel{Evaluation Context Filling}{fig:evalctx-filling}
\end{figure}


We also need the following lemmas, which characterize how hole filling interacts with substitution and instruction transitions. 
\begin{lem}[Substitution Commutativity]
  If
  \begin{enumerate}[nolistsep]
  	\item $\hasType{\hDelta, \Dbinding{u}{\hGamma'}{\htau'}}{x : \htau_2}{\dexp_1}{\tau}$ and
  	\item $\hasType{\hDelta, \Dbinding{u}{\hGamma'}{\htau'}}{\emptyset}{\dexp_2}{\htau_2}$ and
  	\item $\hasType{\hDelta}{\hGamma'}{\dexp'}{\htau'}$
  \end{enumerate}

  then  $\instantiate{d'}{u}{\substitute{d_2}{x}{d_1}} = \substitute{\instantiate{d'}{u}{d_2}}{x}{\instantiate{d'}{u}{d_1}}$.
\end{lem}
\begin{proof}
We proceed by structural induction on $d_1$ and rule induction on the typing premises, which serve to ensure that the free
variables in $d'$ are accounted for by every closure for $u$.
\end{proof}

\begin{lem}[Instruction Commutativity]
  If
  \begin{enumerate}[nolistsep]
  	\item $\hasType{\hDelta, \Dbinding{u}{\hGamma'}{\htau'}}{\emptyset}{\dexp_1}{\tau}$ and
  	\item $\hasType{\hDelta}{\hGamma'}{\dexp'}{\htau'}$ and
  	\item $\reducesE{}{\dexp_1}{\dexp_2}$
  \end{enumerate}

  then $\reducesE{}{\instantiate{\dexp'}{u}{\dexp_1}}
                     {\instantiate{\dexp'}{u}{\dexp_2}}$.
\end{lem}
\begin{proof}
We proceed by cases on the instruction transition assumption (no induction is needed). For Rule \rulename{ITLam}, we defer to the Substitution Commutativity lemma above. For the remaining cases, the conclusion follows from the definition of hole filling.
\end{proof}

\begin{lem}[Filling Totality]
Either $\inhole{u}{\evalctx}$ or $\instantiate{d}{u}{\evalctx}=\evalctx'$ for some $\evalctx'$.
\end{lem}
\begin{proof} We proceed by structural induction on $\evalctx$. Every case is handled by one of the two judgements. \end{proof}

\begin{lem}[Discarding] If
	\begin{enumerate}[nolistsep]
	\item $\selectEvalCtx{d_1}{\evalctx}{\dexp_1'}$ and
	\item $\selectEvalCtx{d_2}{\evalctx}{\dexp_2'}$ and
	\item $\inhole{u}{\evalctx}$
	\end{enumerate}

	then $\instantiate{d}{u}{d_1} = \instantiate{d}{u}{d_2}$.
\end{lem}
\begin{proof} We proceed by structural induction on $\evalctx$ and rule induction on all three assumptions. Each case follows from the definition of instruction selection and hole filling. \end{proof}


\begin{lem}[Filling Distribution] If
	$\selectEvalCtx{d_1}{\evalctx}{d_1'}$ and $\instantiate{d}{u}{\evalctx}=\evalctx'$ then $\selectEvalCtx{\instantiate{d}{u}{d_1}}{\evalctx'}{\instantiate{d}{u}{d_1'}}$.
\end{lem}
\begin{proof} We proceed by rule induction on both assumptions. Each case follows from the definition of instruction selection and hole filling. \end{proof}


\begin{thm}[Commutativity]
  If
  \begin{enumerate}[nolistsep]
  \item $\hasType{\hDelta, \Dbinding{u}{\hGamma'}{\htau'}}{\emptyset}{\dexp_1}{\tau}$ and
  \item $\hasType{\hDelta}{\hGamma'}{\dexp'}{\htau'}$ and
  \item $\multiStepsTo{\dexp_1}{\dexp_2}$
\end{enumerate}

  then $\multiStepsTo{\instantiate{\dexp'}{u}{\dexp_1}}
                     {\instantiate{\dexp'}{u}{\dexp_2}}$.
\end{thm}
\begin{proof}
By rule induction on assumption (3). The reflexive case is immediate. In the inductive case, we proceed by rule induction on the stepping premise. There is one rule, Rule~\rulename{Step}. By Filling Totality, either $\inhole{u}{\evalctx}$ or $\instantiate{d}{u}{\evalctx} = \evalctx'$. In the former case, by Discarding, we can conclude by \rulename{MultiStepRefl}. In the latter case, by Instruction Commutativity and Filling Distribution we can take a \rulename{Step}, and we can conclude via \rulename{MultiStepSteps} by applying Filling, Preservation and then the induction hypothesis.
\end{proof}

% We excluded these proofs and definitions from the Agda mechanization
% for two reasons.
% %
% First, fill-and-resume is merely an optimization, and unlike the meta
% theory of \Secref{sec:calculus}, these properties are generally not
% conserved by certain reasonable extensions of the core
% calculus~(e.g., reference cells and other non-commuting effects).
% %
% Second, to properly encode the hole filling operation, such a
% mechanization requires a more complex representation of
% hole environments; unfortunately, Agda cannot be easily convinced that
% the definition of hole filling is well-founded (\citet{Nanevski2008}
% establish that it is in fact well-founded).
% %
% By contrast, the developments in \Secref{sec:calculus} do not require
% these more complex representations. 


\subsection{Confluence and Resumption}\label{sec:confluence}
There are various ways to encode the intuition that ``evaluation order does not matter''. One way to do so is
by establishing a confluence property (which is closely related to
the Church-Rosser property \cite{church1936some}).

The most general confluence property does not hold for the dynamic
semantics in Sec.~\ref{sec:calculus} for the usual reason: we do not
reduce under binders (\citet{DBLP:conf/birthday/BlancLM05} discuss the
standard counterexample).
%
We could recover confluence by specifying reduction under binders,
either generally or in a more restricted form where only closed
sub-expressions are
reduced \cite{DBLP:journals/tcs/CagmanH98,DBLP:conf/birthday/BlancLM05,levy1999explicit}.
%
However, reduction under binders conflicts with the standard implementation approaches
for most programming languages \cite{DBLP:conf/birthday/BlancLM05}.
%
A more satisfying approach considers confluence modulo equality \cite{Huet:1980ng}.
%
The simplest such approach restricts our interest to top-level expressions
of base type that result in values, in which case the following
special case of confluence does hold (trivially when the only base
type has a single value, but also more generally for other base
types).
\begin{lem}[Base Confluence]
  If $\hasType{\Delta}{\emptyset}{\dexp}{b}$ and
  $\multiStepsTo{\dexp}{\dexp_1}$
  and $\isValue{\dexp_1}$
  and $\multiStepsTo{\dexp}{\dexp_2}$
  then $\multiStepsTo{\dexp_2}{\dexp_1}$.
\end{lem}
We can then prove the following property, which establishes that fill-and-resume is sound.
\begin{thm}[Resumption]
  If $\hasType{\hDelta, \Dbinding{u}{\hGamma'}{\htau'}}{\emptyset}{\dexp}{b}$
  and $\hasType{\hDelta}{\hGamma'}{\dexp'}{\htau'}$
  and $\multiStepsTo{\dexp}{\dexp_1}$
  and $\multiStepsTo{\instantiate{\dexp'}{u}{\dexp}}{\dexp_2}$
  and $\isValue{\dexp_2}$
  then $\multiStepsTo{\instantiate{\dexp'}{u}{\dexp_1}}{\dexp_2}$.
  \begin{proof}
    By Commutativity,
    $\multiStepsTo{\instantiate{\dexp'}{u}{\dexp}}
                  {\instantiate{\dexp'}{u}{\dexp_1}}$.
    By Base Confluence, we can conclude.
  \end{proof}
\end{thm}

% !TEX root = ../CRCBook.tex

\chapter{Extensions of Behavior Trees}
\label{ch:extensions}
\graphicspath{{extensions/}}

As the concept of BT has spread in the AI and robotics communities, a number of extensions have been proposed.
Many of them revolve around the Fallback node, and the observation that the ordering of a Fallback node is often somewhat arbitrary.
In the nominal case, the children of a Fallback node are different ways of achieving the same outcome,
which makes the ordering itself unimportant, but note that this is not the case when Fallbacks are used to increase reactivity with implicit sequences, as described in Section~\ref{sec:implicit_sequences}. 

In this chapter, we will describe a number of extensions of the BT concept that have been proposed.


\begin{figure}[h]
\centering
  \includegraphics[width=8cm]{burglar_utility}
\caption{The result of adding a utility Fallback in the BT controlling a burglar game character in Figure~\ref{design:fig:burglar_safety}.
Note how the Utility node enables a reactive re-ordering of the actions \emph{Escape} and \emph{Fight Cops}.}
\label{design:fig:burglar_utility}
\end{figure}

\section{Utility BTs}
%\label{sec:explicit_conditions}
Utility theory is the basic notion that if we can measure the utility of all potential decisions, it would make sense to 
choose the most useful one. In \cite{merrill2014building} it was suggested that a utility Fallback node would address what was described as the biggest drawback of BTs, i.e. having fixed priorities in the children of Fallback nodes.

A simple example can be seen in the burglar BT of Figure~\ref{design:fig:burglar_utility}.
How do we know that escaping is always better than fighting?
This is highly dependent on the circumstances, do we have a getaway vehicle, do we have a weapon, how many opponents are there, and what are their vehicles and weapons?

By letting the children of a utility Fallback node return their expected utility, the Fallback node can start with the node of highest utility. Enabling the burglar to escape when a getaway car is available, and fight when having a superior weapon at hand. 
In \cite{merrill2014building} it is suggested that all values are normalized to the interval $[0,1]$ to allow comparison between different actions.

Working with utilities is however not entirely straightforward. One of the core strengths of BTs is the modularity, how single actions are handled in the same way as a large tree.
But how do we compute utility for a tree?
Two possible solutions exist,
either we add Decorators computing utility below every utility Fallback node, 
or we add a utility estimate in all actions, and create  a way to propagate utility up the tree, passing both Fallbacks and Sequences. 
The former is a bit ad-hoc, while the latter presents some theoretical difficulties.

It is unclear how to aggregate and propagate utility in the tree. It is suggested in~\cite{merrill2014building} to use the max value in both Fallbacks and Sequences.
This is reasonable for Fallbacks, as the utility Fallback will prioritize the max utility child and execute it first, but one might also argue that a second Fallback child of almost as high utility should increase overall utility for the Fallback.
The max rule is less clear in the Sequence case, as there is no re-ordering, and a high utility child might not be executed due to a failure of another child before it.
These difficulties brings us to the next extension, the Stochastic BTs.


%\section{Fallback adaptation BTs}
%\label{sec:explicit_conditions}

\section{Stochastic BTs}
%\label{sec:explicit_conditions}
\label{sec:stochastic_extension}
A natural variation of the idea of utilities above is to consider success probabilities, as suggested in  \cite{Colledanchise14,hannaford2016simulation}.
If something needs to be done, the action with the highest success probability might be a good candidate.
Before going into details, we note that both costs, execution times, and possible undesired outcomes also matters,
but defer this discussion to a later time.

One advantage of considering success probabilities is that the aggregation across both Sequences and Fallbacks
is theoretically straightforward. Let  $P^s_i$ be the success probability of a given tree, then
the probabilities can be aggregated  as follows~\cite{hannaford2016simulation}:
\begin{equation}
 P^s_{\mbox{Sequence}} = \Pi_i P^s_i, \quad P^s_{\mbox{Fallback}} = 1- \Pi_i (1- P^s_i),
\end{equation}
since Sequences need all children to succeed, while Fallbacks need only one, with probability equal to the complement of all failing.
This is theoretically appealing, but relies on the implicit assumption that each action is only tried once. In a reactive BT for a robot picking and placing items,
you could imagine the robot first picking an item, then accidentally dropping it halfway, and then picking it up again. 
Note that the formulas above do not account for this kind of events.

Now the question comes to how we compute or estimate $P^s_i$ for the individual actions. A natural idea
is to learn this from experience~\cite{hannaford2016simulation}.
It is reasonable to assume that the success probability of an action, $P^s_i$, is  a function of the world state, so it would make sense to try to learn
the success probability as a function of state. Ideally we can classify situations such that one action is known to work in some situations,
and another is known to work in others. The continuous maximization of success probabilities in a Fallback node would then make
the BT choose the correct action depending on the situation at hand.

There might still be some randomness to the outcomes, and then the following estimate is reasonable
\begin{equation}
  P^s_i = \frac{\mbox{\# successes}}{\mbox{\# trials}}. 
\end{equation}
However, this leads to a exploit/explore problem \cite{hannaford2016simulation}. What if both available actions of a Fallback have high success probability?
Initially we try one that works, yielding a good estimate for that action. Then the optimization might continue to favor (exploit) that action,
never trying (explore) the other one that might be even better. For the estimates to converge for all actions, even the ones with
lower success estimates needs to be executed sometimes. One can also note that having multiple similar  robots connected to a cloud service
enables much faster learning of both forms of success estimates described above.

It was mentioned above that it might also be relevant to 
 include costs and execution times in the decision of what tree to execute. A formal treatment of both success probabilities and execution times can be found in 
 Chapter~\ref{ch:stochastic}. A combination of cost and success probabilities might result in a utility system, as described above, but finding the right combination of all three is still an open problem. 
 

\section{Temporary Modification of BTs}
Both in robotics and gaming there is sometimes a need to temporary modify the behavior of a BT.
In many robotics applications there is  an operator or collaborator that might want to temporarily 
influence the actions or priorities of a robot. For instance, convincing a service robot to set the table 
before doing the dirty dishes, or making a delivery drone complete the final mission even though
the battery is low enough to motivate an immediate recharge in normal circumstances.
In computer games, the AI is influenced by both level designers, responsible for the player experience,
and AI engineers, responsible for agents behaving rationally.
Thus, the level designers need a way of making some behaviors more likely, without causing
irrational side effects ruining the game experience.

\begin{figure}[h]
\centering
  \includegraphics[width=8cm]{burglar_agressive}
\caption{The \emph{agressive burglar} style, resulting from disabling  \emph{Escape} in the BT controlling a burglar game character in Figure~\ref{design:fig:burglar_safety}.}
\label{design:fig:burglar_agressive}
\end{figure}

This problem was discussed in one of the first papers on BTs \cite{isla2005handling}, 
with the proposed solutions being \emph{styles},
with each style corresponding to disabling a subset of the BT. For instance, the style \emph{agressive burglar}
might simply have the actions \emph{Escape} disabled, making it disregard injuries and attack until defeated, see Figure~\ref{design:fig:burglar_agressive}.
Similarly, the  \emph{Fight} action can be disabled in the \emph{pacifist burglar} style, as shown in Figure~\ref{design:fig:burglar_pacifist}.
A more elaborate solution to the same problem can be found in the Hinted BTs described below.

\begin{figure}[h]
\centering

  \includegraphics[width=8cm]{burglar_pacifist}
\caption{The \emph{pacifist burglar} style, resulting from disabling  \emph{Fight} in the BT controlling a burglar game character in Figure~\ref{design:fig:burglar_safety}.}
\label{design:fig:burglar_pacifist}
\end{figure}






%\subsection{Hinted BTs}
Hinted BTs were first introduced in \cite{ocio2010dynamic, ocio2012adapting}.
The key idea is to have an external entity, either human or machine, 
giving suggestions, so-called \emph{hints}, regarding actions to execute, to a  BT.
In robotics, the external entity might be an operator or user suggesting something, and in a computer game
it might be the level designer wanting to influence the behavior of a character  without having to edit the actual BT.

The hints can be both positive (+), in terms of suggested actions, and negative (-), actions to avoid,
and a somewhat complex example can be found in Figure~\ref{design:fig:burglar_hinted}.
 Multiple hints can be active simultaneously,
 each influencing the BT in one, or both, of two different ways.
  First they can effect the ordering of Fallback nodes. Actions or trees with positive hints are moved to the left, and ones with negative hints are moved to the right.
 Second, the BT is extended with additional conditions, checking if a specific hint is given.


\begin{figure}[h!]
\centering
  \includegraphics[width=\textwidth]{burglar_hinted}
\caption{The result of providing the hints \emph{Fight Cops+},  \emph{Brake Door Open+} and  \emph{Spend Money-} to the BT in Figure~\ref{design:fig:burglar_combined}. The dashed arrows indicated changes in the BT.}
\label{design:fig:burglar_hinted}
\end{figure}

In the BT of Figure~\ref{design:fig:burglar_hinted}, the following hints were given: \emph{Fight Cops+},  \emph{Brake Door Open+} and  \emph{Spend Money-}.
\emph{Fight Cops+} makes the burglar first considering the fight option, and only escaping when fighting fails.
 \emph{Brake Door Open+} makes the burglar try to brake the door, before seeing if it is open or not,
 and the new corresponding condition makes it ignore the requirements of having a weak door and a crowbar before attempting to brake the door.
Finally,   \emph{Spend Money-} makes the burglar prefer to drive around looking for promising houses rather than spending money.


\section{Other extensions of BTs}
In this section we will briefly describe a number of additional suggested extensions of BTs.
\subsection{Dynamic Expansion of BTs}
The concept of Dynamic Expansions was suggested in \cite{florez2008dynamic}.
Here, the basic idea is to let the BT designer leave some details of the BT to a run-time search.
To enable that search, some desired features of the action needed are specified, these include the
category, given a proposed behavior taxonomy, including \emph{Attack}, \emph{Defend}, \emph{Hunt}, and \emph{Move}. 
The benefit of the proposed approach is that newly created actions can be used in BTs that were created before the actions, 
as long as the BTs have specified the desired features that the new action should have.





\else

\fi
% \clearpage
\section{Misc}

\subsection{Misc Raw Material for Intro}

\cy{intro from LIVE 2017 paper is below}

Broadly speaking, live programming environments are those that granularly interleave editing and evaluation \cite{DBLP:conf/icse/Tanimoto13,DBLP:journals/vlc/Tanimoto90,McDirmid:2007:LUL:1297027.1297073,burckhardt2013s}. 
In the words of \citet{burckhardt2013s}, live programming environments 
``promise to narrow the temporal and perceptive gap 
between program development and code execution''. Examples of live programming environments include {lab notebook environments},
e.g. the popular IPython/Jupyter~\cite{PER-GRA:2007}, which allow the
programmer to interactively edit and evaluate program fragments organized into a
sequence of cells (an extension of the read-eval-print loop (REPL)); spreadsheets; {live graphics programming environments} like SuperGlue \cite{McDirmid:2007:LUL:1297027.1297073}, Sketch-n-Sketch \cite{DBLP:conf/pldi/ChughHSA16} and the tools demonstrated by Bret Victor in his lectures \cite{victor2012inventing}; the TouchDevelop live UI framework \cite{burckhardt2013s}; live mobile application development systems like Flutter \cite{flutter}; and live visual and auditory dataflow languages \cite{DBLP:conf/vl/BurnettAW98}, to name a few prominent examples.


The problem that has motivated much of our recent work is that most  
programming environments, live programming environments included,  provide feedback via various editor services only once the program being edited is syntactically well-formed and, when relevant, well-typed. This leaves a ``temporal and perceptive gap'', because programmers sometimes leave a program malformed or ill-typed for extended periods of time, e.g. as they think about what to enter at the cursor, or as they work on a different part of the program.

In view of this general problem, we recently developed a \emph{structure editor calculus} called Hazelnut where every edit state consists of a well-formed and statically meaningful, i.e. well-typed, incomplete expression, which we take to mean an expression with holes \cite{popl-paper}. This calculus addressed fundamental questions relevant to editor services that operate statically, but there was no solution in that paper to the problems faced by editor services that also require knowledge of the dynamic meaning of an incomplete program, as would be relevant to a live programming environment. For example, consider a \emph{stepper} (a.k.a. a \emph{stepwise debugger}), like that available to Haskell programmers in the GHCi system \cite{GHC-stepper} and other systems \cite{DBLP:conf/haskell/MarlowIPG07,kar13566}, to OCaml programmers with recent work on \texttt{ocamli} by \citet{ocaml-stepper} and to Standard ML programmers \cite{DBLP:journals/jfp/TolmachA95}. A stepper requires that the expression being stepped be assigned dynamic meaning according to a small-step operational semantics \cite{DBLP:journals/jlp/Plotkin04a,pfpl}, but no such semantics was defined for incomplete expressions that arise when using Hazelnut, or any of these other systems. Defining such a semantics was left as future work in the Hazelnut paper, and in a subsequent ``vision paper'' \cite{snapl17-paper}. The purpose of this paper is to sketch out our progress toward a theoretically well-grounded solution.

\matt{below are misc points that may be worth mentioning in section 1
  (somehow, somewhere), and weaving into the continuation of the story
  above:}

\textbf{Type holes vs expression holes:}

The dynamics of type holes is gradual typing, including run-time cast
insertion and execution.
%
The dynamics of expression holes consists of defining
\emph{indeterminate} expressions, as placeholders for subterms that
traditionally would be deemed ``stuck'' .

In live programming, we expect that programs will generally have holes
in both terms and types (CITE?); however, to understand the mechanisms
of \HazelnutLive, it is instructive to imagine situations where the
only holes are in terms, or are in types.
%
When we only use expression holes and no type holes, we do not need
type casts, or the other dynamic mechanisms of gradual typing.
% 
When we only use type holes, we recover gradual typing and do not use
indeterminate forms until and unless cast errors arise.

\textbf{Cast errors as indeterminate forms:}

Instead of being ``stuck'', cast errors in \HazelnutLive are
fully-reduced indeterminate terms.


\textbf{Indeterminate forms as exceptions:}

TODO --- transfer thoughts from 2016 grant writing


\textbf{Holes as breakpoints:}

\begin{enumerate}

\item 
  \texttt{map somefun list}

Suppose the programmer wishes to map with \texttt{somefun} that they
either did not author themselves, or they forgot how the function
works:
  
\item 
  \textt{map ($\lambda \texttt{x}. \hhole{\texttt{x}}{a}$) list}

First, the programmer can intercept the function's inputs.  These
hole~$a$ above functions like a ``breakpoint'', and shows the
programmer me all of the inputs to \texttt{somefun}.

TODO -- show the result of running above.

\item
  \textt{map ($\lambda \texttt{x}. \hhole{\texttt{somefun x}}{a}$) list}

  Next, suppose the programmer wishes to see all of the the outputs of
  \texttt{somefun}.  To do so, they put that (single static)
  expression into a hole, as above.

\item
  \textt{map ($\lambda \texttt{x}. \hhole{(x, \texttt{somefun x})}{a}$) list}

  Finally, the programmer can see input-output pairs without breaking
  the typing derivation of their program globally, since holes are
  permitted to go \emph{anywhere}.  (They may have also exploited this
  fact to show the inputs, in the first step).
  
\end{enumerate}


\cy{intro from grant section is below}

Live programming environments granularly interleave editing and evaluation,
``promis[ing] to narrow the temporal and perceptive gap 
between program development and code execution''~\cite{burckhardt2013s}.
Examples of live programming environments include {lab notebook environments},
like the popular IPython/Jupyter~\cite{PER-GRA:2007}, which allow the
programmer to interactively edit and evaluate program fragments organized into a
sequence of cells (an extension of the ubiquitous read-eval-print loop (REPL)); spreadsheets; {live graphics programming environments} like SuperGlue \cite{McDirmid:2007}, \sns{}~\cite{sns-pldi,sns-uist} and the tools demonstrated by Bret Victor in his lectures \cite{victor2012inventing}; the TouchDevelop live UI framework \cite{burckhardt2013s}; live mobile application development systems like Flutter \cite{flutter}; and live visual and auditory dataflow languages \cite{DBLP:conf/vl/BurnettAW98}, to name a few prominent examples. \cy{find somewhere to cite these tanimoto papers \cite{DBLP:conf/icse/Tanimoto13,DBLP:journals/vlc/Tanimoto90}}


Live programming environments typically cannot run incomplete programs, or they simply halt or raise an exception as soon as they encounter a hole during evaluation. This leaves a ``temporal and perceptive gap'' because the results of even those computations that do not depend critically on the hole are not available to the programmer. Our second track of research will address this gap by defining a dynamic semantics for incomplete programs that proceeds as far as possible around holes. 
Every program assigned static meaning by the semantics of Track 1 will also be assigned dynamic meaning by the semantics of Track 2.

For example, in \autoref{fig:intro-example} we saw that mapping the incomplete function \li{weighted_average} over the \li{grades} list produced an incomplete result. If holes simply raised exceptions, then the system would have aborted evaluation as soon as \li{weighted_average} was applied for the first time by \li{map}, and the programmer would not have been able to see that the result was a list of two elements, nor see the value of the sub-expression \li{10.0 *. g.hw1} for each element \li{g} of the \li{grades} list. Even programs with type inconsistencies can be evaluated under our proposed dynamics, because, as discussed in Track 1, non-empty holes operate as membranes around type inconsistencies. Taken together, the results from Tracks 1 and 2 may help to address a common complaint: that a static type discipline makes it difficult to do exploratory programming. 

Our proposed dynamics goes beyond staged evaluation (\eg{}~\cite{Taha:1999}), partial evaluation (\eg{}~\cite{Jones:1993uq}) and symbolic evaluation (\eg{}~\cite{King:1976}) in that we track the dynamic environment (\ie{}~the substitutions performed) around each instance of an expression hole in the result. This allows the programmer to see, in the sidebar of \autoref{fig:intro-example}, the actual values that the variables in scope take on everywhere that the hole they are filling ends up during evaluation. This provides a hole-oriented specification of a standard feature of debuggers: inspection of the environment at designated points in the program. Finally, this environmental information is useful (though not required) for the edit action suggestion service of Track 3 and the direct manipulation programming service of Track 4. 

\subsection{Raw Material for Related Work}

\begin{itemize}
	\item Hazelnut paper
	\item simply typed underdeterminism paper
	\item work on partial evaluation and staging (incl. connections to modal logic -- ``modal analysis of staged computation'')
	\item symbolic evaluation
	\item full beta reduction
	\item Agda
	\item Idris
	\item GHC holes
	\item CMTT
	\item gradual typing (and dynamic typing)
		\begin{itemize}
			\item siek and taha paper
			\item snapl15 paper
			\item gradualizer paper
			\item maybe other things, e.g. several papers by ron garcia
		\end{itemize}
	\item DuctileJ stuff -- \url{https://homes.cs.washington.edu/~mernst/pubs/ductile-icse2011.pdf}
	\item OLEG from McBride's thesis
	\item Visual Studio (and others) support for edit-and-resume
	\item Scratch lets you just skip over statement holes
	\item prior work on confluence for the lambda calculus
	\item work on debuggers that allow you to inspect environments
		\begin{itemize}
		\item might be something relevant in the paper ``A Debugger for Standard ML'' 
		\item "Visualizing the evaluation of functional programs for debugging" by Whitington and Ridge
		\item "A lightweight interactive debugger for Haskell'' and ``Multiple-View Tracing for Haskell: a New Hat'' might be relevant
		\item ocamli -- \url{https://github.com/johnwhitington/ocamli}
		\item Better supporting debugging aids learning a novel programming language. -- Scaffidi at VLHCC 2017
		\item quote from Wadler in ``Why no one uses functional languages'':
			\begin{quote}
			“...there are few debuggers or
profilers for strict [functional] languages, perhaps because constructing them is not considered
research. This is a shame, since such tools are sorely needed, and there remains much of
interest to learn about their construction and use.
\end{quote}
		\end{itemize}
	\item papers that show up in a search for ``typed holes'' -- \url{https://scholar.google.com/scholar?hl=en&as_sdt=0%2C39&q=%22typed+holes%22&btnG=}
	\item maybe also search for ``partial programs'' 
	\item roly pererra work on program slicing
	\item should say something about how holes show up in program synthesis under various names (what?) 
	\item ``Achieving flexibility in direct-manipulation programming environments by relaxing the edit-time grammar'' -- \url{http://ieeexplore.ieee.org/document/1509511/}
	\item ``Call-by-value is dual to call-by-name'' might be relevant? \url{http://homepages.inf.ed.ac.uk/wadler/papers/dual/dual.pdf}
	\item mention how unspecified evaluation order is something that people do when talking about parallelism?
\end{itemize}

% % !TEX root = hazelnut-dynamics.tex

\section{Implementation Screenshots}
\label{sec:impl-screenshots}

\begin{figure}[h]
\includegraphics[width=\textwidth]{images/screenshot1.png}
\caption{This screenshot demonstrates both empty and non-empty expression holes as well as the live context inspector.}
\end{figure}

\begin{figure}[h]
\includegraphics[width=\textwidth]{images/screenshot2.png}
\caption{This screenshot demonstrates dynamic cast failure.}
\end{figure}


\end{document}
