\appendix
\section{Experiments on CIFAR-10}
\label{sec:cifar}

We detail below the experimental details for the observations made in Section~\ref{sec:results}. If it's not explicitly mention otherwise, a ResNet-18 model was trained on CIFAR 10 with batches of 128 examples, using an Adam optimiser with a constant learning rate of 0.001. The optimiser's statistics were reset between the warm up and the tuning phase. In the first stage (during warm up) the model was trained for 350 epochs on half of data. In the second stage the model is trained for 500 epochs on all training data. We report here the average test performance over all seeds during the last 100 training epochs. In all plots the error bars measure standard deviation.

Due to space constraints we don't show learning curves, but if it's not otherwise specified it's implied that accuracy is 100\% on the data used in training for both stages as in Figure~\ref{fig:demo}.

\subsection{Different optimisers}

In addition to Figure~\ref{fig:adam} in Section~\ref{sec:results} we show here how generalisation performance varies with the number of pretraining epochs on half of data for three additional optimisers: RMSprop in Figure~\ref{fig:rmsprop}, Stochastic Gradient Descent (SGD) in Figure~\ref{fig:sgd}, and SGD with a constant momentum (0.9) in Figure~\ref{fig:msgd}.

\begin{figure}[h!tb]
    \centering%
    \includegraphics[width=\linewidth]{RMSprop_.001.pdf}
    \caption{Final performance after warming up the model for different numbers of epochs using RMSProp with a constant learning rate for both phases.}
    \label{fig:rmsprop}
\end{figure}

\begin{figure}[h!tb]
    \centering%
    \includegraphics[width=\linewidth]{SGD_.001.pdf}
    \caption{Final performance after warming up the model for different numbers of epochs using SGD with a constant learning rate for both phases.}
    \label{fig:sgd}
\end{figure}

\begin{figure}[h!tb]
    \centering%
    \includegraphics[width=\linewidth]{mSGD_.001_.9.pdf}
    \caption{Final performance after warming up the model for different numbers of epochs using SGD with a constant learning rate and momentum for both phases.}
    \label{fig:msgd}
\end{figure}

\subsection{Smooth transition between distributions.}
\label{sec:blending}

In the experiments presented in Figure~\ref{fig:blending} we trained models in a single stage of 500 epochs.
In this case we called an epoch a sequence of 390 update steps (which is the equivalent of 1 pass through the training data with a batch size of 128). Note that such an epoch is not a permutation of the data. Each example from each batch is individually sampled with probability $p$ from the full training set, and with probability $1-p$ from the pretrainig set.

\subsection{Class imbalance in the multiple stage setup}
\label{sec:splits}

Given (i) a training set $\mathcal{D}$ with examples from a set of classes $\mathcal{C}$, (ii) a number of pretraining stages $n$ and (iii) a number $0 \le r \le 1$ (the ratio of data from all classes) we constructed $n$ subsets of $\mathcal{D}$: $\left\lbrace\mathcal{D}_i\right\rbrace_{1 \le i \le n}$ to be used for optimisation during the $n$ pretraining stages. In doing this we applied the following methodology:
\begin{enumerate}
    \item We created a partition of the all classes: $\left\lbrace \mathcal{C}_1, \ldots \mathcal{C}_{n + 1} \right\rbrace$ such that $\mathcal{C}_i \cap \mathcal{C}_j = \emptyset, \forall i, j$, and $\mathcal{C} = \bigcup_{i=1}^{n+1} \mathcal{C}_{i}$.
    \item We randomly split the full training set $\mathcal{D}$ in two: $\mathcal{D}_{c}$, $\mathcal{D}_{u}$ such that $\frac{\vert\mathcal{D}_{u}\vert}{\vert\mathcal{D}\vert} = r$. (of course: $\mathcal{D}_{u} \cap \mathcal{D}_{c} = \emptyset$, and $\mathcal{D}_{u} \cup \mathcal{D}_{c} = \mathcal{D}$).
    \item We partition $\mathcal{D}_{u}$ into $n+1$ subsets: $\left\lbrace D_{u,1}, \ldots \mathcal{D}_{u, n+1} \right\rbrace$.
    \item We now define the data sets used to optimise the model in each stage (and considering $\mathcal{D}_{0}=\emptyset$):
    $$ \mathcal{D}_i = \mathcal{D}_{i-1} \cup \mathcal{D}_{u, i} \cup \left\lbrace \left(x, c\right) \in \mathcal{D}_{c} \mid c \in \mathcal{C}_i \right\rbrace$$
\end{enumerate}

The $n+1$-th dataset $\mathcal{D}_{n+1} \equiv \mathcal{D}$ corresponds to the final tuning phase on the full data set.

\subsection{Residual networks of various depths and widths}
\label{sec:resnets}

In the experiment with residual neural networks of different widths and depths we changed the architecture of ResNet-18 \cite{He_2016_CVPR} as follows.

Apart from the first convolution and the fully connected layer at the output, ResNet-18 consists of four modules, each made up of $d=2$ residual blocks with the same number of output channels. Each module doubles the number of channels and halves the height and the width of the feature maps. The first module receives a volume with $w=64$ channels, the second operates on $2w=128$, and so on.

In our experiments we uniformly changed the depth $d$ of the four modules, and/or scaled the number of channels in all modules ($w, 2w, 4w, 8w$).

Note that this is not the standard way in which people design deeper residual architectures such as ResNet-32, or ResNet-55. Deeper ResNets increase the depth of the modules differently, and use bottleneck blocks to avoid an explosion in the number of parameters.

\subsection{Resetting the layers of the model}
\label{sec:reset_details}

In the experiments presented in Figure~\ref{fig:reset} from Section~\ref{sec:results} we reset subsets of the model's parameters. We reset entire modules referring with 1 to the first convolution, with numbers from 2 to 5 to the four modules (each consisting of 2 residual blocks), and naming 6 the last fully connected layer.

As Figure~\ref{fig:reset} shows, resetting the last 4, or 5 modules seems to recover the original performance of a model trained from random parameters. Therefore we asked whether keeping the pretrained parameters of the first 1 or 2 modules comes with any advantage in terms of training speed. As Figure~\ref{fig:advantage350} shows, in our setup there seems to be no benefit from preserving parameters from the pretraining stage.


\begin{figure}[h!tb]
    \centering%
    \includegraphics[width=\linewidth]{reset_advantage_350.pdf}
    \caption{Here we show the learning curves for three models pretrained for 350 epochs on half of data. For two of them we keep the first 1 or 2 modules, reinitialise the rest and tune for 500 epochs.}
    \label{fig:advantage350}
\end{figure}





\section{Supporting evidence for the two phases of learning hyothesis}
\label{sec:twophase_literature}

A couple of works identify critical differences between the early stage and the late stage of training, offering empirical evidence for the two phases of learning hypothesis.

\cite{achille2018critical} identifies an initial \textit{memorisation} phase when data information is absorbed into the network's weights, followed by a \textit{reorganisation} stage where unimportant connections are pruned and information decreases while being redistributed among layers for efficiency. \citeauthor{achille2018critical} used the Fisher Information Matrix to approximate the amount of information stored in the weights. The FIM is also a curvature matrix, therefore the observed regimes support the view that learning changes basins of attraction of different minima until it lands in one with low curvature, corresponding to a flat minimum. \citeauthor{achille2018critical} also point out that if data statistics change after the initial phase, the network would remain trapped in the valley the memorisation phase guided it into.

\cite{golatkar2019time} empirically shows that regularisation has an impact on final generalisation performance only in the early stages of training. Applying weight decay or data augmentation only after this initial phase, or stopping regularisation after that point would not affect generalisation. The experiments using data augmentation later in training offer additional evidence for the generalisation gap -- if one thinks about that stage as tuning on more data from the same distribution.

\cite{gur2018gradient} shows that after an early training stage the gradients reside in a small subspace that remains constant for the rest of training. This reiterates the importance of the data used in the first steps of training.

\cite{li2019towards} shows that in overparametrized networks the volume of good minima dominates the volume of poor minima and underlines the importance of a high learning rate to land in the basin of attraction of a well generalising minimum of the loss function. \cite{jastrzebski2020break} extends the observation with the importance of using batches to induce the noise needed to escape poorly generalising minima. Precisely, \citeauthor{jastrzebski2020break} point out that the ratio between learning rate and batch size determines the flatness of the minumum.

\cite{ghorbani2019investigation} computes the full spectrum of the Hessian showing that \citeauthor{jastrzebski2020break}'s claims about smaller learning rates guiding the network into sharper minima doesn't hold empirically. A possible explanation is that the network is already trapped around some minima, and the slow learning rate just reaches an even flatter region closer to the critical point.