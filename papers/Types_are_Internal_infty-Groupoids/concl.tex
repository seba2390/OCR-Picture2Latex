\section{Conclusion}
\label{sec:concl}

We have presented an approach to defining higher coherent structures
in homotopy type theory by equipping type theory with a primitive set
of structures collected into a universe $\MM$ of polynomial monads,
and demonstrated that this approach can be used to prove non-trivial
theorems about these structures.  In this brief final section, we
compare some related approaches and survey some of the possible
directions and applications.

\subsection{Future Directions}
\label{sec:future-directions}

\subsubsection{Symmetric Structures}

A natural class of structures which escapes the capabilities of our
current approach is that of \emph{symmetric structures}, that is,
those which incorporate higher analogs of commutativity.  Examples
would include $\mathbb{E}_{\infty}$ groups and monoids, symmetric
monoidal categories, and general $\infty$-operads and their algebras.

\subsubsection{Higher Category Theory}

As we have seen, one higher structure which \emph{is} amenable to
treatment by our methods is that of an $\infty$-category.  An obvious
point to follow up on, then, is how much of the well developed theory
of $\infty$-categories can be formalized in this manner.

\subsubsection{A General Theory of Structures}

As we have mentioned in the introduction, we see the present work as a
first step towards a general theory of types and structures.  And
though we feel certain that at least some of the ideas of the present
work will carry over to such a theory, a complete picture of the basic
principles remains to be understood.  Moreover, a careful
investigation of the interaction of our techniques with univalent
implementations of type theory (such as \emph{cubical} type theory)
also remains for future work.

Accompanying such a general theory, we anticipate a deeper
investigation of the meta-theoretic properties of our proposed
approach.  For example, the Agda implementation is limited by the
expressivity of rewrite rules, and complicated by the explicit
universe construction, while a proper extension of MLTT would allow
for the investigation of meta-theoretic properties like decidability
of type checking and strong normalization using techniques like
normalization-by-evaluation (and potentially settling the conjecture
of \ref{sec:mnd-struct}). Furthermore, we have not touched at all on
the potential models of our system, topic which deserves we feel
deserves careful attention.


%%% Local Variables:
%%% mode: latex
%%% TeX-master: "types-are-grpds-ext"
%%% End:
