\section{Opetopic Types}
\label{sec:opetopic-types}

In this section, we show how to use the universes introduced above in
order to implement Baez and Dolan's definition of \emph{opetopic type}
\cite{BD98}.  We go on to explain how to use this definition to capture
the notion of \emph{weak $M$-algebra}, and finish with some examples.

\begin{defn}
  An \textbf{opetopic type over a monad $M$} is defined coninductively
  as follow:
  \begin{align*}
    &\texttt{record}\hspace{1ex} \OpType (M : \MM) : \UU_1 \hspace{1ex} \texttt{where} \\
    &\hspace{.5cm}\Car : \Idx M \to \UU \\
    &\hspace{.5cm}\Rel : \OpType (\Slice (\Pb M\, \Car))
  \end{align*}
\end{defn}
We see from the definition that an opetopic type consists of
an infinite sequence of dependent families
\[ \Car X \,, \Car (\Rel X) \,, \Car (\Rel (\Rel X)) \,, \dots \]
whose domain is the set of indices of a monad whose definition
incorporates all the previous families in the sequence.  Given an
opetopic type $X : \OpType M$, we will often denote this sequence of
dependent types more succinctly as just $X_0, X_1, X_2, \dots$ since
the destructor notation quickly becomes quite heavy.  We will use a
similar convention for the series of monads $M = M_0, M_1, M_2 \dots$
generated by the definition.  That is, we have:
\begin{equation}
  \label{eq:unfold}
  \begin{aligned}
    &M_0 = M && X_0 = \Car X : \Idx M \to \UU \\
    &M_1 = \Slice (\Pb M_0\, X_0) && X_1 = \Car (\Rel X) : \Idx M_1 \to \UU \\
    &M_2 = \Slice (\Pb M_1\, X_1) && X_2 = \Car (\Rel (\Rel X)) : \\
    &&&\hspace{2cm} \Idx M_2 \to \UU \\
    &\hspace{1cm} \vdots && \hspace{1cm} \vdots
  \end{aligned}
\end{equation}

Before describing the connection between opetopic types and weak
$M$-algebras, let us give some examples of how to think of the
resulting dependent families as ``fillers'' for a collection of
``shapes'' generated by the monad $M$.  For concreteness, we will fix
$M = \Id$ in our examples.  Given $X : \OpType \Id$, we can define the
type of \emph{objects} of $X$ as simply
\begin{equation}
  \label{eq:obj-defn}
  \begin{aligned}
    &\mathsf{Obj} : \UU \\
    &\mathsf{Obj} = \Car X\, \ttt
  \end{aligned}
\end{equation}
Next, after a single slice, $X$ provides us with a type of
\emph{arrows} between any two objects which can be defined as follows:
\begin{equation}
  \label{eq:arrow-defn}
  \begin{aligned}
    &\mathsf{Arrow} : (x\, y : \mathsf{Obj}) \to \UU \\
    &\mathsf{Arrow}\, x\, y = \Car (\Rel X)\, \\
    &\hspace{2cm} ((\ttt \,, y) \,, (\ttt , \etadec \Id\, (\Car X)\, x))
  \end{aligned}
\end{equation}
Furthermore, for a \emph{loop} $f$ in $X$, that is, an arrow with the
same domain and codomain, $X$ includes a family whose elements can be
thought of as ``null-homotopies of $f$'', and which is defined by
\begin{align*}
  &\mathsf{Null} : (x : \mathsf{Obj})\, (f : \mathsf{Arrow}\, x\, x) \to \UU \\
  &\mathsf{Null}\, x\, f = \Car (\Rel (\Rel X)) \\
  & \hspace{2cm} ((((\ttt \,, x) \,, (\ttt , \etadec \Id\, (\Car X)\, x)) \,, f) \,, \\
  & \hspace{2.5cm} \lf\, (\ttt \,, x) \,, \bot\mhyphen\mathsf{elim})
\end{align*}
More examples of shapes and filling families may be found in the
development.

\subsection{Weak Algebras and Fibrant Opetopic Types}
\label{sec:weak-alg}

We now wish to describe how an opetopic type $X : \OpType M$ encodes
the structure of a weak $M$-algebra.  Before we begin, it will be
convenient to adopt the following convention: recall that $X$ consists
of an infinite sequence of dependent types following the form of
Equation \ref{eq:unfold}.  In the discussion which follows, instead of
working with a fixed opetopic type $X$, we will rather just work with
abstract type families $X_0 , X_1 , \dots$ over monads
$M = M_0, M_1, \dots$ following the same pattern of dependencies. We
will then freely add new families of the form $X_i$ to our hypotheses
as they become necessary.  The advantage of working this way is that
our definitions are parameterized over just that portion of the
opetopic type which is necessary, as opposed to depending on the
entire opetopic type $X$ itself, and consequently, we will be able to
reuse our definitions and constructions starting at any point in the
infinite sequence generated by $X$.

We recall that for $M$ a polynomial monad, an \emph{$M$-algebra}
consists of a \emph{carrier family} $C : \Idx M \to \UU$ together with
a map
\[ \alpha : \{i : \Idx M\} \to [ M ]\, C\, i \to C\, i \] which
satisfies some equations expressing the compatibility of $\alpha$ with
the multiplication of $M$.  Indeed, it is the need for a complete
description of these equations in all dimensions which motivates the
present work.  Now, clearly the first dependent type
$X_0 : \Idx M \to \UU$ may serve as a carrier for an $M$-algebra
structure.  Let us now see what else this sequence of families
provides us with.

After one iteration, we obtain a type family $X_1 : \Idx M_1 \to \UU$,
and unfolding the definition of the indices of the slice and pullback
monads, we find that the domain of $X_1$ takes the form
\begin{align*}
  \sum_{(i : \Idx M)} \sum_{(x : X_0 i)}
  \sum_{(c : \Cns M\ i)} (p : \Pos M\, c) \to X_0\, (\Typ M\, c\, p)
\end{align*}
The elements of this type are 4-tuples $(i \,, x \,, c \,, \nu)$, and
we now observe that the three elements $i$, $c$ and $\nu$ are typed
such that they are exactly the arguments of the hypothetical algebra
map $\alpha$ introduced above.  We may regard the family $X_1$,
therefore, as a relation between triples $(i \,, c \,, \nu)$ and
elements $x : X_0\, i$, and in order to define a map $\alpha$, we only
need to impose that this relation is functional in the sense that
there is a \emph{unique} $x$ determined by any such triple.  When this
is the case, we will say that the family $X_1$ is
\emph{multiplicative}.  That is, we define:
\begin{align*}
  &\ismult : \{X_0 : \Idx M_0 \to \UU\}\, (X_1 : \Idx M_1 \to \UU) \to \UU \\ 
  &\ismult \{X_0\}\, X_1 = \{i : \Idx M\}\, (c : \Cns M\, i) \\
  &\hspace{1cm} \to (\nu : (p : \Pos M\, c) \to X_0\, (\Typ M\, c\, p))\\
  &\hspace{1cm} \to \iscontr (\sum_{x : X_0\, i} X_1\, (i \,, x \,, c \,, \nu))
\end{align*}
Supposing we are given a proof $m_1 : \ismult X_1$, we can define an
algebra map $\alpha$ as above by
\[\alpha\, (c \,, \nu) = \fst (\ctr (m_1\, c\, \nu)) \]
Furthermore, we will write
\[\alpha\wit\, (c \,, \nu) = \snd (\ctr (m_1\, c\, \nu)) \]
for the associated element of the relation
$X_1 (i \,, \alpha\, (c \,, \nu) \,, c \,, \nu)$ which witnesses
this multiplication.

Let us now suppose that our sequence extends one step further, that
is, that we are given a type family $ X_2 : \Idx M_1 \to \UU $ and a
proof $m_2 : \ismult X_2$.  We now show how to use this further
structure to derive some of the expected \emph{laws} for the algebra
map $\alpha$ we have just defined.  As a first example, we expect
$\alpha$ to satisfy a unit law: decorating a unit constructor with
some element $x$ and then applying $\alpha$ should return the element
$x$ itself.  In other words, we expect to be able to prove
\begin{align*}
  &\alpha^{\eta}\coh : \{i : \Idx M\} (x : X_0\, i) \\
  &\hspace{.3cm} \to \alpha\, (\upeta\, M\, i \,, \etadec M\, X_0\, x)\, \equiv x
\end{align*}
To prove this equality, let us define the following function:
\begin{align*}
  &\upeta\alg_{m_2} : \{i : \Idx M\} (x : X_0\, i) \\
  % &\upeta\alg_{m_2} : \{i : \Idx M\} (x : X_0\, i) \\
  &\hspace{.3cm} \to X_1\, ((i \,, x) \,, (\upeta\, M\, i , \etadec M\, X_0\, x)) \\
  &\upeta\alg_{m_2} = \fst (\ctr (m_2\, (\lf (i \,, x))\, \botelim)) 
\end{align*}
Now we simply notice that the pairs
\[ \ctr (m_1\, (\upeta\, M\, i)\, (\etadec M\, X_0\, x)) \equiv (x \,,
  \upeta\alg\, x) \] must be equal as indicated, since they inhabit a
contractible space.  Projecting on the first factor gives exactly the
desired equation.

We also expect our algebra map $\alpha$ to satisfy an equation
expressing its compatibility with multiplication of the following
form:
\begin{align*}
  &\alpha^{\upmu}\coh : \{i' : \Idx M\}\, (c' : \Cns M\, i)\\
  &\hspace{.2cm} \to (\delta' : (p : \Pos M\, c) \to \Cns M (\Typ M\, c\, p))\\
  &\hspace{.2cm} \to (\nu' : (p : \Pos M\, c') (q : \Pos M\, (\delta'\, p)) \\
  &\hspace{3cm} \to X_0\, (\Typ M\ (\delta'\, p)\, q)) \\
  &\hspace{.2cm} \to \alpha\, (\upmu\, M\, c' \, \delta')\, (\lambda p \to \nu'\, (\muposfst p)\, (\mupossnd p)) \equiv\\
  &\hspace{.7cm} \alpha\, c'\, (\lambda p \to \alpha\, (\delta'\, p)\, (\nu'\, p))
\end{align*}
We note that this equation is simply the type theoretic translation of
the familiar commutative diagram
\[
  \begin{tikzcd}
    {[ M ]\, [ M ]\, X_0} \ar[r,"\upmu_{X_0}"] \ar[d,"{[ M ]\, \alpha}"'] &
    {[ M ]\, X_0} \ar[d,"\alpha"] \\
    {[ M ]\, X_0} \ar[r,"\alpha"'] & X_0
  \end{tikzcd}
\]
To prove this axiom, we use $m_2$ to define the following
multiplication operation on elements of the family $X_1$:
\begin{align*}
  &\upmu\alg_{m_2} : \{i : \Idx M\}\, (c : \Cns M\, i)\\
  &\hspace{.3cm}\to (\nu : (p : \Pos M c) \to X_0\, (\Typ M\, c\, p)) \\
  % &\hspace{.3cm}\to (\delta : \Decor (c \,, \nu)\, (\Cns (\Pb M\, X_0))) \\
  &\hspace{.3cm}\to (\delta : (p : \Pos M c)\\
  &\hspace{2cm} \to  \Cns (\Pb M\, X_0) (\Typ (\Pb M\, X_0)\, (c, \nu)\, p)) \\
  &\hspace{.3cm}\to (x_0 : X_0\, i)\, (x_1 : X_1\, (i \,, x_0 \,, c \,, \nu)) \\
  &\hspace{.3cm}\to (\bar{x} : (p : \Pos M\, c) \to X_1 (\Typ (\Pb M\, X_0) (c \,, \nu) \,, \delta\, p)) \\
  &\hspace{.3cm}\to X_1 (i \,, x_0 \,, \upmu\, (\Pb M\, X_0)\, (c \,, \nu)\, \delta) \\
  &\upmu\alg_{m_2} = \fst (\ctr (m_2\, \sigma\, \theta))
\end{align*}
where
\begin{align*}
  &\sigma = \nd (c \,, \nu)\, \delta\, (\lambda p \to \upeta\, M_1\, ((\Typ M\, c\, p \,, \nu\, p) \,, \delta\, p)) \\
\end{align*}
is the two-level tree consisting of a base node $(c , \nu)$, as well
as a second level of constructors specified by the decoration
$\delta$, and $\theta$ is the decoration of the nodes
of $\sigma$ by elements of $X_1$ defined by:
\begin{align*}
  &\theta\, (\inl \ttt) = x_1 \\
  &\theta\, (\inr (p \,, \inl \ttt)) = \bar{x}\, p \\
\end{align*}
Now instantiating our function $\upmu\alg_{m_2}$ with arguments
\begin{align*}
  &c = c' &&x_1 = \alpha\mhyphen\mathsf{wit}\, (c \,, \nu) \\
  &\nu\, p = \alpha\, (\delta'\, p \,, \nu'\, p) &&\delta\, p = (\delta'\, p \,, \nu' \, p) \\
  &x_0 = \alpha\, (c \,, \nu) &&\bar{x}\, p = \alpha\mhyphen\mathsf{wit}\, (\delta'\, p \,, \nu'\, p) 
\end{align*}
we find that the pairs
\begin{align*}
  &\ctr (m_1\, (\upmu\, M\, c'\, \delta')\, (\lambda p \to \nu'\, (\muposfst p)\, (\mupossnd p))) \equiv \\
  &\hspace{.3cm} (\alpha\, (c \,, \nu) \,,  \upmu\alg_{m_2}\, c\, \nu\, x_0\, x_1\, \delta\, \bar{x})
\end{align*}
again inhabit a contractible space, whereby their first components are
equal, giving the desired equation.

We may think of the functions $\upeta\alg_{m_2}$ and $\upmu\alg_{m_2}$
defined above as the nullary and binary cases of a multiplicative
operation on the \emph{relations} of our algebra structure.  The key
insight, as we have seen, is that this multiplicative structure
encodes exactly the \emph{laws} for the algebra map $\alpha$ defined
one level lower.  Similarly, if we are able to extend our sequence on
\emph{further} step to a family $X_3$ which is itself multiplicative,
then we will be able to show that the operations $\upeta\alg_{m_2}$
and $\upmu\alg_{m_2}$ \emph{themselves satisfy unit and associativity
  laws}, and this in turn encodes the ``2-associativity'' and
``2-unitality'' of the algebra map $\alpha$.  This motivates
the following definition:

\begin{defn}
  An opetopic type $X$ over a monad $M$ is said to be \textbf{fibrant}
  if we are given an element of the following coinductively defined
  property:
  \begin{align*}
    &\texttt{record}\hspace{1ex} \isfibrant \{M : \MM\}\, (X : \OpType M) : \UU\\
    &\hspace{.3cm} \texttt{where} \\
    % &\hspace{.3cm} (X : \OpType M) : \UU \hspace{1ex} \texttt{where} \\
    &\hspace{.5cm} \carmult : \ismult M\, (\Car (\Rel X)) \\
    &\hspace{.5cm} \relfib : \isfibrant (\Rel X)
  \end{align*}
\end{defn}
Fibrant opetopic types, therefore, are our definition of infinitely
coherent $M$-algebras, with the multiplicativity of the relations
further in the sequence witnessing the higher dimensional laws
satisfied by the structure earlier in the sequence.

\subsection{Higher structures}
\label{sec:higher-structures}

We now use the preceding notions to define a number of coherent
algebraic structures.  A first example is that we obtain an internal
definition of the notion of $\infty$-groupoid as follows: 
\begin{defn}
  An \textbf{$\infty$-groupoid} is a fibrant opetopic type over
  the identity monad.  That is,
\[ \ooGrp = \dsum{X : \OpType \Id} \isfibrant X \] 
\end{defn}
\noindent We will attempt to justify the correctness of this
definition in the sections which follow.

Next, it happens that the monad $\Slice \Id$ is in fact the monad
whose algebras are monoids, and consequently, our setup leads
naturally to the definition of an $\mathbb{A}_\infty$-type, that is, a
type with a coherently associative binary operation.

\begin{defn}
  An \textbf{$\mathbb{A}_\infty$-type} is a fibrant opetopic type
  over the first slice of the identity monad.
  \[ \mathbb{A}_\infty\mhyphen\mathsf{type} = \dsum{X : \OpType
      (\Slice \Id)} \isfibrant X \]
\end{defn}
Furthermore, the notion of $\mathbb{A}_\infty$-group can now be
defined by imposing an invertibility axiom.  A classical theorem of
homotopy theory asserts that the type of $\mathbb{A}_\infty$-groups is
equivalent to the type of pointed, connected spaces via the loop-space
construction.  It would be interesting to see if the techniques of
this article lead to a proof of this fact in type theory.

The notion of $\infty$-category can also be defined using this setup.
Recall that an opetopic type over the identity monad $\Id$ has both a
type of objects and a type of arrows (Equations \ref{eq:obj-defn} and
\ref{eq:arrow-defn}).  In the definition of $\infty$-groupoid above,
the invertibility of the arrows in the underlying opetopic type is a
consequence of the fact that the family of arrows is assumed to be
multiplicative.  Consequently, we obtain a reasonable notion of a
\emph{pre-$\infty$-category} by simply dropping this assumption, and
only requiring fibrancy after one application of the destructor
$\Rel$:
\[ \preooCat = \dsum{X : \OpType \Id} \isfibrant (\Rel X) \] The
prefix ``pre'' here refers to the fact that this definition is missing
a completeness axiom asserting that the invertible arrows coincide
with paths in the space of objects, that is, an axiom of
\emph{univalence} in the sense of \cite{ahrens2015univalent}.  Such an
axiom is easily worked out in the present setting, but as it would
distract us slightly from the main objective of the present work, we
will not pursue the matter here.
