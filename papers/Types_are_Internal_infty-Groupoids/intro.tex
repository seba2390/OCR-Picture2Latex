\section{Introduction}

Homotopy Type Theory has brought a new perspective to intensional
Martin-L\"{o}f type theory: the higher identity types of a type endow
it with the structure of an \emph{$\infty$-groupoid}, and ideas from
homotopy theory provide us with a means to predict and understand the
resulting tower of identifications.  While this perspective has been
enormously clarifying with respect to our understanding of the notion
of proof-relevant equality, leading, as it has, to a new class of
models as well as new computational principles, a number of
difficulties remain in order to complete the vision of type theory as
a foundation for a new, structural mathematics based on
homotopy-theoretic and higher categorical principles.

Foremost among these difficulties is the following: how does one
describe a well behaved theory of \emph{algebraic structures} on
arbitrary types?  The fundamental difficulty in setting up such a
theory is that, in a proof relevant setting, nearly all of the
familiar algebraic structures (monoids, groups, rings, and categories,
to take a few) become \emph{infinitary} in their presentation.
Indeed, the axioms of these theories, which take the form of a finite
list of mere \emph{properties} when the underlying types are sets,
constitute additional \emph{structure} when they are no longer assumed
to be so.  Consequently, in order to arrive at a well-behaved theory, the
axioms themselves must be subject to additional axioms, frequently
referred to generically as ``coherence conditions''. In short, in a
proof relevant setting, it no longer suffices to describe the
equations of an algebraic structure at the ``first level'' of
equality.  Rather, we must specify how the structure behaves
\emph{throughout the entire tower} of identity types, and this is an
infinite amount of data.  How do we organize and manipulate this data?

Similar problems have arisen in the mathematics of homotopy theory and
higher category theory, and many solutions and techniques are known.
Bafflingly, however, all attempts to import these ideas into plain
homotopy type theory have, so far, failed.  This appears to be a
result of a kind of circularity: all of the known classical techniques
at \emph{some} point rely on set-level algebraic structures themselves
(presheaves, operads, or something similar) as a means of presenting
or encoding higher structures.  Internally to type theory, however, we
do not have recourse to such techniques.  Indeed, without further
hypotheses, we do not even expect that the most basic objects of the
theory, types themselves, are presented by set-level structures. This
leaves us in a strange position: absent a theory of algebraic
structures, we have nothing to use to encode algebraic structures!

We suggest that a possible explanation for this phenomenon is the
following: contrary to our experience with set-level mathematics,
where an algebraic structure (i.e. a ``structured set'') can itself be
defined just in terms of sets: underlying sets, functions, sets of
relations and so on, when we pass to the world of homotopy theoretic
mathematics, the notion of \emph{type} and \emph{structured type} are
simply no longer independent of each other in the same way.
Consequently, some primitive notion of structured type must be defined
\emph{at the same time} as the notion of type itself.  The present
work is a first attempt at rendering this admittedly somewhat vague
idea precise.

We begin by imagining a type theory which, in addition to defining a
universe $\UU$ of \emph{types}, defines at the same time a universe
$\SC$ of \emph{structures}. Of course, we will need to be somewhat
more precise about what exactly we mean by \emph{structure}.  Category
theory suggests that one way of representing a structure is by the
monad on $\UU$ which it defines, so we might think of $\SC$ as a
universe of monads.  In practice, however, it will be useful to
restrict to a particularly well behaved class of monads, having
reasonable closure properties, and for which we have a good
understanding of their higher dimensional counterparts.  We submit
that a reasonable candidate for such a well-behaved collection is the
class of \emph{polynomial monads}~\cite{GK}.

We feel that this is an appropriate class of structures for a number
of reasons.  A first reason is that this class of monads arises quite
naturally in type theory already: indeed, a large literature exists on
the interpretation of inductive and coinductive types as initial
algebras and terminal coalgebras for polynomial monads, and we
consider our work as deepening and extending this connection.
Furthermore, this class of algebraic structures enjoys some pleasant
properties which make them particularly amenable to ``weakening''.
For example, the very general approach to weakening algebraic
structures developed by Baez and Dolan in \cite{BD98} can be smoothly
adapted to the polynomial case. While the cited work employs the
language of symmetric operads, connections with the theory of
polynomial functors were already described in \cite{KJBM}, and
moreover, recent work \cite{gepner2017infty} has shown that, in type
theory, we should expect symmetric operads to in fact be
\emph{subsumed} by the theory of polynomial monads.

The central intuition of Baez and Dolan's approach, is that each
polynomial monad $M$ determines its \emph{own} higher dimensional
collection of shapes (the $M$-opetopes) generated directly from the
syntactic structure of its terms.  They go on to introduce the notion
of an \emph{$M$-opetopic type} which is, roughly, a collection of well
formed decorations of these shapes, and the notion of weak $M$-algebra
is then defined as an $M$-opetopic type satisfying certain closure
properties.  In this sense, their approach differs from, say,
approaches based on simplices, cubes or spheres in that the geometry
is not fixed ahead of time, but adapted to the particular structure
under consideration.

With these considerations in mind, our plan in the present work is to
put the idea of a type theory with primitive structures to the test.
What might it look like, and what might it be able to prove?  In order
to answer these questions, we will build a prototype of the theory
\footnote{The Agda source is available here:
  \url{https://github.com/ericfinster/opetopic-types}} in the proof
assistant Agda, and exploit the recent addition of \emph{rewrite
  rules} \cite{cockx:hal-02901011} which permits us to extend
definitional equality by new well-typed reductions.  The use of such
rewrites is necessary to ensure that our primitive structures are not
subject to the same infinite regress of coherence conditions which has
so far obstructed more naive attempts and defining such objects.


Concretely, we will introduce a universe $\MM$,\footnote{We mean by
  this notation to distinguish the universe $\MM$ of this
  \emph{particular} implementation from the generic idea of a universe
  of structures $\mathcal{S}$, the properties of which we expect to be
  refined by further investigation.} whose elements we think of as
codes for polynomial monads and describe the structures they decode
to.  Because we think of the objects of the universe $\MM$ as
primitives of our theory, on the same level as types, we allow
ourselves the freedom to prescribe their computational behavior: in
particular, we will equip them with definitional associativity and
unit laws using the rewrite mechanism alluded to above.  We emphasize
that if structures are taken as defined in parallel with types, then
this kind of definitional behavior should be no more surprising than,
say, the definitional associativity of function composition.

We then show how the existence of our universe $\MM$ has some strong
consequences.  In particular, it allows us to implement the Baez and
Dolan definition of opetopic type alluded to above, and subsequently
to define a number of weak higher dimensional structures.  Among the
structures which we are able to define using this technique are
$\mathbb{A}_\infty$-monoids and groups, $(\infty,1)$-categories and
presheaves over them (in particular, our setup leads to a definition
of \emph{simplicial type}), and as a special case, $\infty$-groupoids
themselves.

There arises, then, the problem of justifying the correctness of our
definitions.  In order to do so, we will take up the example of
$\infty$-groupoids in some detail.  Indeed, since the homotopical
interpretation of type theory asserts that types should ``be''
$\infty$-groupoids, it seems natural to compare these two objects.
Our main result is the following:
\begin{thm*}
  There is an equivalence
  \[ \UU \simeq \infty\mhyphen\mathsf{Grp} \]
\end{thm*}
In other words, every type admits the structure of an
$\infty$-groupoid in our sense, and that structure is unique.  This
theorem, therefore, can be regarded as a (constructive)
internalization of the intuition provided by the various
meta-theoretic results to this effect~\cite{van2011types,
  lumsdaine2009weak}.

\subsection{Related Work}
\label{sec:related-work}

The so-called \emph{coherence problem}, which is the main motivation
for the present work, has been considered by a number of authors.  We
briefly compare our approach with two prominent other strains of
thought.

\subsubsection{Synthetic Structures}

One way to avoid some of the problems posed by the definition of
higher dimensional structures is to simply enlarge the collection of
basic objects to include them.  Such approaches may be described as
synthetic, in that they do not reduce higher structures to more
primitive objects in exactly the same way that homotopy type theory
itself does not define $\infty$-groupoids in terms of sets.  This
point of view is often adopted, for example, in the research into
\emph{directed type theories}, whether they be aimed at specific
structures like higher categories as in \cite{north2019towards}, or
allow for more general directed spaces as in \cite{riehl2017type}.

While our theory does indeed add some new primitive structures to type
theory, we collect these structures in a universe and decode them into
collections of ordinary types and maps between them.  Moreover, we go
on to use these additional strict structures to give analytic,
internal definitions of higher structures.

\subsubsection{Two-Level Type Theory}

Perhaps the closest related work to the current approach is that of
Two-Level Type Theory \cite{DBLP:journals/corr/AnnenkovCK17}.  There
it is advocated to add a second ``level'' to type theory with a
set-truncated equality type which one can then use to make
meta-theoretic statements about the inner level, whose objects are
typically taken to be the homotopically meaningful ones.  The
two-level approach provides, thus, a great deal of generality and
flexibility at the cost of restricting the applicability of
homotopical interpretation of types to the inner theory.  It is
likely, for example, that our theory could be developed inside a
two-level system and many of the rewrites we employ proven as theorems
in the outer level.  By constrast, our approach is, we feel, somewhat
more economical, extending the theory with a specific set of rewrite
rules, and pointing towards the possibility of a useful theory of
higher structures without the need to restrict homotopical principles
like univalence.

\subsection{Preliminaries}

The basis of our metatheory is the type theory implemented in
Agda~\cite{agda} which is an extension of the predicative part of
Martin-Löf type theory~\cite{ML75}. Among the particular types that
Agda implements, we shall use inductive types, records and coinductive
records.

As such, we adopt a style similar to Agda code, writing
$(x : A) \to B\, x$ for the dependent product (although we
occasionally also employ the $\prod_{(x : A)} B\, x$ notation if it
improves readability).  We also make use of the implicit counterpart
of the dependent products, written $\{x : A\} \to B\, x$.  This allows
us to hide arguments which can be inferred from the context and hence
clarify our notation.  Non-dependent functions are denoted $A \to B$,
as usual. Functions enjoy the usual $\eta$ conversion rule.

We shall make extensive use of coinductive record types, as well as
copatterns for producing elements of these types.  We write $\Unit$
for the empty record with a constructor $\ttt : \Unit$. We write
$\dsum{x : A} B\ x$ for the dependent sum as a record with constructor
$\_, \_$ and projections $\fst$ and $\snd$. Pairs which are not
dependent are denoted $A \times B$.

We write $\bot$ for the empty type, using absurd patterns where
appropriate, and writing $\botelim$ for the unique function for $\bot$
to any type.

The identity type $\_\equiv\_ : \{A : \UU\}\ (x\ y : A) \to \UU$ is
an inductive type with one constructor
$\refl : (x : A) \to x \equiv x$.

We shall make use of the notion of contractible type denoted
$\iscontr$ whose center of contraction will be referred to as
$\ctr$. Other notions defined in the HoTT book~\cite{hottbook} will be
employed including equivalence of types denoted $X \simeq Y$, function
extensionality denoted $\mathsf{funext}$, as well as the Univalence
Axiom.

We write $\UU$ for the universe of small types, and $\UU_1$ for the
next universe when necessary.

In order to unclutter and clarify the presentation, we
occasionally take liberties with the formal definitions, for example,
silently inserting applications to functional extensionality when
necessary, or reassociating $\Sigma$-types in order to avoid a
proliferation of commas. Our formal development in Agda allows no
such informalities to remain.

%%% Local Variables:
%%% mode: latex
%%% TeX-master: "types-are-grpds-ext.tex"
%%% End:
