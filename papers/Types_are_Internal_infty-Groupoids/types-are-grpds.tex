%%
%% types-are-grpds.tex - Types are Internal oo-Groupoids
%%

\documentclass[conference,10pt]{IEEEtran}

\usepackage{amsfonts}
\usepackage{amsmath}
\usepackage{amsthm}
\usepackage{amssymb}
\usepackage{mathtools}
\usepackage{upgreek}
\usepackage{xcolor}
\usepackage{tikz}
\usepackage{tikz-cd}
\usepackage{hyperref}

% correct bad hyphenation here
\hyphenation{op-tical net-works semi-conduc-tor}

\interdisplaylinepenalty=2500

\definecolor{eric}{HTML}{99c0ff}
\newcommand{\note}[1]{\fcolorbox{black}{eric}{#1}}
\newcommand{\bignote}[2][\columnwidth]{\noindent\fcolorbox{black}{eric}{\parbox{#1}{\color{black} \sf #2}}}
\definecolor{matthieu}{HTML}{fff000}
\newcommand{\bignotem}[2][\columnwidth]{\noindent\fcolorbox{black}{matthieu}{\parbox{#1}{\color{black} \sf #2}}}
\definecolor{antoine}{HTML}{9fdf9f}
\newcommand{\bignotea}[2][\columnwidth]{\noindent\fcolorbox{black}{antoine}{\parbox{#1}{\color{black} \sf #2}}}

\newtheorem*{thm*}{Theorem}
\newtheorem{thm}{Theorem}
\newtheorem{prop}{Proposition}
\newtheorem{defn}{Definition}
\newtheorem{lem}{Lemma}
\newtheorem{cor}{Corollary}

\newtheorem{thmapdx}{}
\newenvironment{thm-apdx}[1]
  {\renewcommand\thethmapdx{#1}\thmapdx}
  {\endthmapdx}

\newcommand\UU{\mathcal{U}}
\newcommand\SC{\mathcal{S}}
\newcommand\MM{\mathbb{M}}
\newcommand\MO{\mathbb{M}\mkern-4mu\downarrow}
\newcommand\Idx{\mathsf{Idx}\,}
\newcommand\Cns{\mathsf{Cns}\,}
\newcommand\Typ{\mathsf{Typ}\,}
\newcommand\Pos{\mathsf{Pos}\,}
\newcommand\Decor{\mathsf{Decor}\,}

\newcommand\Set{\mathcal{S}et}

\newcommand\Id{\mathsf{Id}}
\newcommand\Pb{\mathsf{Pb}\,}
\newcommand\Slice{\mathsf{Slice}\,}
\newcommand\Pd{\mathsf{Tree}\,}
\newcommand\Free{\mathsf{Free}\,}
\newcommand\Slc{\mathsf{Slc}\,}

\newcommand{\ooGrp}{\infty\mhyphen\mathsf{Grp}}
\newcommand{\preooCat}{\mathsf{pre}\mhyphen\infty\mhyphen\mathsf{Cat}}
\newcommand{\ovr}{\mkern-8mu\downarrow}
\newcommand{\smovr}{\mkern-4mu\downarrow}

\newcommand{\dsum}[1]{\textstyle{\sum_{(#1)}}\,}
\newcommand{\dprod}[1]{\textstyle{\prod_{(#1)}}\,}
\mathchardef\mhyphen="2D

\newcommand\refl{\mathsf{refl}}
\newcommand\ttt{\mathsf{tt}}
\newcommand\ctr{\mathsf{ctr}\,}
\newcommand\wit{\mhyphen\mathsf{wit}}
\newcommand\coh{\mhyphen\mathsf{coh}}
\newcommand\alg{\mhyphen\mathsf{alg}}

% Antoine's over commands

\newcommand{\da}{{\downarrow}}

\newcommand\MMd{\mathbb{M\da}}
\newcommand\Idxd{\mathsf{Idx}\da\,}
\newcommand\Cnsd{\mathsf{Cns}\da\,}
\newcommand\Typd{\mathsf{Typ}\da\,}
\newcommand\Posd{\mathsf{Pos}\da\,}

\newcommand\upetad{\upeta\da}
\newcommand\upmud{\upmu\da}

\newcommand\lfd{\operatorname{lf\da}}
\newcommand\ndd{\operatorname{nd\da}}

\newcommand\Idd{\mathsf{Id\da}\,}
\newcommand\Pbd{\mathsf{Pb\da}\,}
\newcommand\Sliced{\mathsf{Slice\da}\,}
% end 

\newcommand\Unit{\top}
\newcommand\Empty{\bot}
\newcommand\botelim{\bot\mhyphen\mathsf{elim}}
\newcommand\etapos{\upeta\mhyphen\mathsf{pos}\,}
\newcommand\etaposelim{\upeta\mhyphen\mathsf{pos}\mhyphen\mathsf{elim}\,}
\newcommand\etadec{\upeta\mhyphen\mathsf{dec}\,}
\newcommand\etadecd{\upeta\mhyphen\mathsf{dec}\da\,}
\newcommand\mupos{\upmu\mhyphen\mathsf{pos}\,}
\newcommand\muposfst{\upmu\mhyphen\mathsf{pos}\mhyphen\mathsf{fst}\,}
\newcommand\mupossnd{\upmu\mhyphen\mathsf{pos}\mhyphen\mathsf{snd}\,}
\newcommand\gammaposinl{\upgamma\mhyphen\mathsf{pos}\mhyphen\mathsf{inl}\,}
\newcommand\gammaposinr{\upgamma\mhyphen\mathsf{pos}\mhyphen\mathsf{inr}\,}
\newcommand\gammaposelim{\upgamma\mhyphen\mathsf{pos}\mhyphen\mathsf{elim}\,}

\newcommand\lf{\mathsf{lf}\,}
\newcommand\nd{\mathsf{nd}\,}
\newcommand\fst{\mathsf{fst}\,}
\newcommand\snd{\mathsf{snd}\,}
\newcommand\inl{\mathsf{inl}\,}
\newcommand\inr{\mathsf{inr}\,}

\newcommand\iscontr{\mathsf{is}\mhyphen\mathsf{contr}\,}
\newcommand\ismult{\mathsf{is}\mhyphen\mathsf{mult}\,}
\newcommand\isfibrant{\mathsf{is}\mhyphen\mathsf{fibrant}\,}
\newcommand\carmult{\mathsf{car}\mhyphen\mathsf{is}\mhyphen\mathsf{mult}\,}
\newcommand\relfib{\mathsf{rel}\mhyphen\mathsf{is}\mhyphen\mathsf{fibrant}\,}
\newcommand\isalgebraic{\mathsf{is}\mhyphen\mathsf{algebraic}\,}

\newcommand\OpType{\mathsf{OpetopicType}\,}
\newcommand\OvrOpType{\da\mathsf{OpType}\,}
\newcommand\Car{\mathcal{C}\,}
\newcommand\Rel{\mathcal{R}\,}

\newcommand\lflf{\operatorname{lf-lf}}
\newcommand\ndlf{\operatorname{nd-lf}}

\newcommand{\commentt}[1]{}


%%% Local Variables:
%%% mode: latex
%%% TeX-master: "lics-article"
%%% End:


%% generate labels from the appendix to be used in compact version
\newlabel{proof:muetarmumu-proof}{{3}{13}{Strong confluence for the \ref {rewrite:muetar}/\ref {rewrite:mumu} pair}{lem.3}{}}
\newlabel{eq:slc-idx}{{4}{14}{\appendixname }{equation.Appendix.A.4}{}}

\begin{document}

\setlength{\abovedisplayskip}{6pt}
\setlength{\belowdisplayskip}{6pt}

\title{Types Are Internal $\infty$-Groupoids}
\author{
  \IEEEauthorblockN{Eric Finster}
  \IEEEauthorblockA{Cambridge University\\
    Department of Computer Science \\
    ericfinster@gmail.com
  }
  \and
  \IEEEauthorblockN{Antoine Allioux}
  \IEEEauthorblockA{Inria \&
   IRIF, Université de Paris\\
   France\\
   antoine.allioux@irif.fr}
  \and
  \IEEEauthorblockN{Matthieu Sozeau}
  \IEEEauthorblockA{Inria \&
    LS2N, Université de Nantes\\
    France\\
    matthieu.sozeau@inria.fr}
}

\IEEEoverridecommandlockouts
\IEEEpubid{\makebox[\columnwidth]{978-1-6654-4895-6/21/\$31.00~
\copyright2021 IEEE \hfill} \hspace{\columnsep}\makebox[\columnwidth]{ }}
\maketitle

\begin{abstract}
  By extending type theory with a universe of definitionally
  associative and unital polynomial monads, we show how to arrive at a
  definition of \emph{opetopic type} which is able to encode a number
  of fully coherent algebraic structures.  In particular, our approach
  leads to a definition of $\infty$-groupoid internal to type theory
  and we prove that the type of such $\infty$-groupoids is equivalent
  to the universe of types.  That is, every type admits the structure
  of an $\infty$-groupoid internally, and this structure is unique.
\end{abstract}

% For peer review papers, you can put extra information on the cover
% page as needed:
% \ifCLASSOPTIONpeerreview
% \begin{center} \bfseries EDICS Category: 3-BBND \end{center}
% \fi
%
% For peerreview papers, this IEEEtran command inserts a page break and
% creates the second title. It will be ignored for other modes.
\IEEEpeerreviewmaketitle

% !TEX root = ../arxiv.tex

Unsupervised domain adaptation (UDA) is a variant of semi-supervised learning \cite{blum1998combining}, where the available unlabelled data comes from a different distribution than the annotated dataset \cite{Ben-DavidBCP06}.
A case in point is to exploit synthetic data, where annotation is more accessible compared to the costly labelling of real-world images \cite{RichterVRK16,RosSMVL16}.
Along with some success in addressing UDA for semantic segmentation \cite{TsaiHSS0C18,VuJBCP19,0001S20,ZouYKW18}, the developed methods are growing increasingly sophisticated and often combine style transfer networks, adversarial training or network ensembles \cite{KimB20a,LiYV19,TsaiSSC19,Yang_2020_ECCV}.
This increase in model complexity impedes reproducibility, potentially slowing further progress.

In this work, we propose a UDA framework reaching state-of-the-art segmentation accuracy (measured by the Intersection-over-Union, IoU) without incurring substantial training efforts.
Toward this goal, we adopt a simple semi-supervised approach, \emph{self-training} \cite{ChenWB11,lee2013pseudo,ZouYKW18}, used in recent works only in conjunction with adversarial training or network ensembles \cite{ChoiKK19,KimB20a,Mei_2020_ECCV,Wang_2020_ECCV,0001S20,Zheng_2020_IJCV,ZhengY20}.
By contrast, we use self-training \emph{standalone}.
Compared to previous self-training methods \cite{ChenLCCCZAS20,Li_2020_ECCV,subhani2020learning,ZouYKW18,ZouYLKW19}, our approach also sidesteps the inconvenience of multiple training rounds, as they often require expert intervention between consecutive rounds.
We train our model using co-evolving pseudo labels end-to-end without such need.

\begin{figure}[t]%
    \centering
    \def\svgwidth{\linewidth}
    \input{figures/preview/bars.pdf_tex}
    \caption{\textbf{Results preview.} Unlike much recent work that combines multiple training paradigms, such as adversarial training and style transfer, our approach retains the modest single-round training complexity of self-training, yet improves the state of the art for adapting semantic segmentation by a significant margin.}
    \label{fig:preview}
\end{figure}

Our method leverages the ubiquitous \emph{data augmentation} techniques from fully supervised learning \cite{deeplabv3plus2018,ZhaoSQWJ17}: photometric jitter, flipping and multi-scale cropping.
We enforce \emph{consistency} of the semantic maps produced by the model across these image perturbations.
The following assumption formalises the key premise:

\myparagraph{Assumption 1.}
Let $f: \mathcal{I} \rightarrow \mathcal{M}$ represent a pixelwise mapping from images $\mathcal{I}$ to semantic output $\mathcal{M}$.
Denote $\rho_{\bm{\epsilon}}: \mathcal{I} \rightarrow \mathcal{I}$ a photometric image transform and, similarly, $\tau_{\bm{\epsilon}'}: \mathcal{I} \rightarrow \mathcal{I}$ a spatial similarity transformation, where $\bm{\epsilon},\bm{\epsilon}'\sim p(\cdot)$ are control variables following some pre-defined density (\eg, $p \equiv \mathcal{N}(0, 1)$).
Then, for any image $I \in \mathcal{I}$, $f$ is \emph{invariant} under $\rho_{\bm{\epsilon}}$ and \emph{equivariant} under $\tau_{\bm{\epsilon}'}$, \ie~$f(\rho_{\bm{\epsilon}}(I)) = f(I)$ and $f(\tau_{\bm{\epsilon}'}(I)) = \tau_{\bm{\epsilon}'}(f(I))$.

\smallskip
\noindent Next, we introduce a training framework using a \emph{momentum network} -- a slowly advancing copy of the original model.
The momentum network provides stable, yet recent targets for model updates, as opposed to the fixed supervision in model distillation \cite{Chen0G18,Zheng_2020_IJCV,ZhengY20}.
We also re-visit the problem of long-tail recognition in the context of generating pseudo labels for self-supervision.
In particular, we maintain an \emph{exponentially moving class prior} used to discount the confidence thresholds for those classes with few samples and increase their relative contribution to the training loss.
Our framework is simple to train, adds moderate computational overhead compared to a fully supervised setup, yet sets a new state of the art on established benchmarks (\cf \cref{fig:preview}).

\section{A Universe of Polynomial Monads}

As we have explained in the introduction, type theory appears to lack
the ability to speak about infinitely coherent algebraic structures,
and our strategy for addressing this problem will be to distinguish a
collection of such structures which we consider as defined by the
theory itself.  We do so using a common technique in the type theory
literature: that of introducing a \emph{universe}.  We write
$\MM : \UU$ for our universe, and we think of its elements as
\emph{codes for polynomial monads}.  Just as a typical type theoretic
universe has some collection of base types and some collection of type
constructors, so we will populate our universe with a collection of
``base monads'' and ``monad constructors''.  In other words: we
will have a syntax of structures which parallels the syntax for types.

Typically, a universe of types $\mathbb{U}$ comes equipped with a
decoding function $El : \mathbb{U} \to \UU$.  In the case of our
universe of monads $\MM$, each element $M : \MM$ will decode not to a
single type, but to a collection of types and type families equipped
with some structure.  We will use rewrite rules to specify
the computational behavior of this structure.

\subsection{Polynomial Structure}

To begin, we first equip each $M : \MM$ with an underlying
\emph{polynomial} or \emph{indexed container} \cite{DBLP:journals/jfp/AltenkirchGHMM15}.
This is accomplished by postulating the following collection of decoding functions:
\begin{align*}
  \Idx &: \MM \to \UU\\
  \Cns &: (M : \MM) \to \Idx M \to \UU\\
  \Pos &: (M : \MM)\ \{i : \Idx M\} \to \Cns M\, i \to \UU \\
  \Typ &: (M : \MM)\ \{i : \Idx M\}\ (c : \Cns M i) \\
       &\to \Pos M\, c \to \Idx M 
\end{align*}
Polynomials of this sort appear in the computer science literature as
the ``data of a datatype declaration''.  They can equivalently be seen
as a way to describe the signature of an algebraic theory: concretely,
the elements of $\Idx M$, which we refer to as \emph{indices} serve as
the sorts of the theory, and for $i : \Idx M$, the type $\Cns M\, i$
is the collection of operation symbols whose ``output'' sort is $i$.
The type $\Pos M\, c$ then assigns to each operation a collection of
``input positions'' which are themselves assigned an index via the
function $\Typ$.

It follows that every monad $M$ induces a functor
$[\_] : (\Idx M \to \UU) \to (\Idx M \to \UU)$ called its
\emph{extension} given by
\begin{align*}
  [ M ]\, X\, i &= \sum_{(c : \Cns M\, i)} (p : \Pos M\, c) \to X \, (\Typ M\, c\ p)
\end{align*}
We may think of the value of this functor at a type family
$X : \Idx M \to \UU$ as the type of \emph{constructors of $M$ with
  inputs decorated by elements of $X$}.  Indeed, we will frequently
refer to a dependent function of the form
\[ (p : \Pos M\, c) \to X \, (\Typ M\, c\ p) \] where $X$ is as above,
as a \emph{decoration} of $c$ by elements of $X$.

\subsection{Monadic Structure}
\label{sec:mnd-struct}

Next, for each monad $M : \MM$, we are going to equip the
underlying polynomial of $M$ with an algebraic structure:
specifically, that structure required on the underlying
polynomial so that the associated extension $[ M ]$ becomes
a \emph{monad}.  In the case at hand, this takes the form
of a pair of functions
\begin{align*}
  \upeta &: (M : \MM)\ (i : \Idx M) \to \Cns M\, i\\
  \upmu &: (M : \MM)\ \{i : \Idx M\}\ (c : \Cns M\, i) \\
         &\to (\delta : (p : \Pos M\, c) \to \Cns M\, (\Typ M\, c\ p)) \\
         &\to \Cns M\, i
\end{align*}
which equip $M$ with a multiplication and unit operation.  We remark
that the second argument $\delta$ of the multiplication $\upmu$ is a
decoration of $c$ in the family $\Cns M$ of constructors, so that we
can think of the input to this function as a ``two-level'' tree.

Crucial for what follows will be that the monads we consider are
\emph{cartesian} in the sense of \cite{GK}.  Type theoretically, the
means we require each monad $M$ to come equipped with equivalences
\begin{align*}
  \Pos M\, (\upeta\ M\, i) &\simeq \Unit \\
  \Pos M\, (\upmu\ M\, c\, \delta) &\simeq \dsum{p : \Pos M\, c} \Pos M\, (\delta\, p)
\end{align*}
Since we are already modifying the definitional equality of our type
theory, it may be tempting to require these equivalences
definitionally by asserting that the type of positions reduces when
applied to constructors of the appropriate form. However, this will
not work: as we will see below, when we come to populate our universe
with concrete monads and monad constructors, the equivalences we find
are often in fact not definitional, even if they remain provable.  As
an alternative, we will equip each monad with introduction,
elimination and computation rules for its positions which will in
effect guarantee that we always have the required equivalence.  Each
monad definition will then be required to implement these rules in a
manner consistent with the various required typing laws.

In the case of $\upeta$, for example, we postulate introduction and
elimination rules of the form
\begin{align*}
  \etapos &: (M : \MM)\ (i : \Idx M) \to \Pos M\, (\upeta\ M\, i)\\
  \etaposelim &: (M : \MM)\ (i : \Idx M) \\
              &\to (X : (p : \Pos M\, (\eta\ M\, i)) \to \UU)\\ 
              &\to (u : X\ (\etapos M\, i)) \\
          &\to (p : \Pos M\, (\upeta\ M\, i)) \to X\ p
\end{align*}
with typing rule
\begin{equation}
  \label{rewrite:etaTyp}
  \tag{Typ-$\eta$}
  \begin{aligned}
    \Typ M\, (\upeta\ M\, i)\ p & \leadsto i 
  \end{aligned}
\end{equation}
and computation rule
\begin{align*}
  \etaposelim\ M\, i\ X\ u\ (\etapos M\, i) &\leadsto u \\
\end{align*}
Notice these are exactly the rules for an inductively defined indexed unit
type.\footnote{In principle, we would also like to have an
  $\eta$-rule for the unit $\upeta$, (that is, we would prefer the
  negative version as we have below for $\upmu$) but unfortunately
  this is not possible with the current implementation of rewriting in
  Agda.}  In particular, decorations of the constructor $\upeta\, M\, i$
in a type family $X : \Pos M\, (\upeta\, M\, i) \to \UU$ are completely determined
by a single element $x : X\, i$, a fact which we record in the following
definition to reduce clutter below:
\begin{align*}
  % \etadec &: (M : \MM)\, (X : \Idx M \to \UU)\\
  %         &\to \{i : \Idx M\}\, (x : X i)\, \\
  %         &\to \Decor (\upeta\, M\, i) X \\
  \etadec M\, X\, \{i\}\, x = \etaposelim M\, i\, (\lambda\, \_ \to X i)\, x 
\end{align*}

Next, for the multiplication $\upmu$, our rules simply mimic the
pairing and projections of the dependent sum.  That is, we
postulate an introduction rule
\begin{align*}
  \mupos &: (M : \MM)\ \{i : \Idx M\}\ \{c : \Cns M\, i\}\ \\
          &\to \{\delta : (p : \Pos M\, c) \to \Cns M\, (\Typ M\, c\ p)\}\\ 
          &\to (p : \Pos M\, c) \to (q : \Pos M\, (\delta\ p))\\
         &\to \Pos M\, (\upmu\ M\, c\ \delta)
\end{align*}
and elimination rules
\begin{align*}
  \muposfst &: (M : \MM)\ \{i : \Idx M\}\ \{c : \Cns M\, i\}\\
            &\to \{\delta : (p : \Pos M\, c) \to \Cns M\, (\Typ M\, c\ p)\}\\
            &\to \Pos M\ (\upmu\ M\, c\ \delta) \to \Pos M\, c\\
  \mupossnd &: (M : \MM)\ \{i : \Idx M\}\ \{c : \Cns M\, i\}\\
            &\to \{\delta : (p : \Pos M\, c) \to \Cns M\, (\Typ M\, c\ p)\}\\
            &\to (p : \Pos M\, (\upmu\ M\, c\ \delta))\\
            &\to \Pos M\, (\delta\ (\muposfst M\, p))
\end{align*}
with typing rule
\begin{align*}
  \label{rewrite:muTyp}
  \tag{Typ-$\mu$}
  \Typ M\, &(\upmu\ M\, c\ \delta)\ p \leadsto \\
           &\Typ M\, (\delta\ (\muposfst M\, p))\ \\
           &\hspace{1.5cm} (\mupossnd M\, p)
\end{align*}
and computation rules
\begin{align*}
  \label{rewrite:muposfst}
  \tag{$\mu$-pos-fst}
  \muposfst M\, (\mupos M\, p\ q) &\leadsto p\\
  \label{rewrite:mupossnd}
  \tag{$\mu$-pos-snd}
  \mupossnd M\, (\mupos M\, p\ q) &\leadsto q\\
  \label{rewrite:muposeta}
  \tag{$\mu$-pos-$\eta$}
  \mupos M\, (\muposfst M\, p)\, (\mupossnd M\, p) &\leadsto p \\
\end{align*}
With the handling of positions in place, we can now state the
unitality and associativity axioms for the monads in our universe.
These take the form of reductions:
\begin{align*}
  \tag{$\mu$-$\eta$-r}
  \label{rewrite:muetar}
  &\upmu\ M\, c\ (\lambda\ p \to \upeta\ M\, (\Typ M\, c\ p)) \leadsto c\\
  \tag{$\mu$-$\eta$-l}\label{rewrite:muetal}
  &\upmu\ M\, (\upeta\ M\, i)\ \delta \leadsto \delta\ (\etapos M\, i)\\
  \tag{$\mu$-$\mu$}\label{rewrite:mumu}
  &\upmu\ M\, (\upmu\ M\, c\ \delta)\ \epsilon \leadsto \\
  &\quad \upmu\ M\ c\ (\lambda\ p \to\ \upmu\ M\, (\delta\ p)\ (\lambda\ q \to \epsilon\ (\mupos M\, p\ q)))
\end{align*}
Additionally, we must posit laws which assert that the constructors
and eliminators for positions are compatible with these equations.  We
omit these for brevity, but the interested reader may consult the
development for details.

While we will not undertake an extensive investigation of the
meta-theoretic properties of our system in this article, we wish to
pause briefly to make at least of few observations to justify its
well-formedness.  For example, there are critical pairs in the rewrite
equations for the monad laws (between the first equation and the
others) so we need to ensure confluence and termination.
\begin{lem}[Strong confluence for $\eta$ and $\mu$]
  The rewrite rules are strongly confluent \cite{huet80}, hence globally confluent.
\end{lem}
\begin{proof}
  The rewrite system is strongly confluent using the rules 
  \eqref{rewrite:etaTyp} and \eqref{rewrite:muTyp} and the associated reduction rules
  for $\mupos$.
  We show the case for \ref{rewrite:muetar} and \ref{rewrite:muetal}.
  We omit $M$ which is fixed here.

  \begin{align*}
  \upmu\ (\upeta\ i)&\ (\lambda\ p \to \upeta\ (\Typ (\upeta\ i)\ p)) \leadsto_{\text{\ref{rewrite:muetar}}} \upeta\ i\\
  \upmu\ (\upeta\ i)&\ (\lambda\ p \to \upeta\ (\Typ (\upeta\ i)\ p)) \leadsto_{\text{\ref{rewrite:muetal}}} \\
  &(\lambda\ p \to \upeta\ (\Typ (\upeta\ i)\ p))\ (\etapos i) \leadsto_\beta \\ 
  &\upeta\ (\Typ (\upeta\ i)\ (\etapos i)) \leadsto_{\text{\ref{rewrite:etaTyp}}} \upeta\ i
  \end{align*}

  The resolution of the \ref{rewrite:muetar}/\ref{rewrite:mumu} pair
  can be found in the appendix of the extended version of this article
  \cite[Lemma \ref{proof:muetarmumu-proof}]{allioux2021types}.
\end{proof}

\vspace{1ex}

\begin{prop}[Termination of rewriting]
  All of the above rules form a terminating rewrite system.
\end{prop}
\begin{proof}
  The $\mupos$, $\etapos$ and \eqref{rewrite:etaTyp} rewrite rules are obviously terminating.
  For \eqref{rewrite:muTyp},  \eqref{rewrite:muetar},
  \eqref{rewrite:muetal} and \eqref{rewrite:mumu}, we need to 
  use a dependency-pairs path ordering as introduced by \cite{blanqui19}
  to verify termination. In particular for associativity, a 
  lexicographic lifting of the subterm relation is not 
  enough to verify \eqref{rewrite:mumu}'s termination as we are 
  going under binders and applying the $\delta$ and $\epsilon$ 
  functions to subterms. This is a variant of the ordinal type 
  eliminator proven to terminate in \cite[Example 14, p11]{blanqui19},
  which requires to ensure that the constructor types of our monads 
  are inductively generated. All the monads considered in this article satisfy this.
\end{proof}

The instances of the $\mu$ and $\eta$ operations for specific monads
will themselves be defined by structural recursion on inductive
datatypes and can be shown to respect the associativity and unitality
laws prositionally. Results such as can be found in~\cite[Lemma
6.8]{cockx:hal-02901011}, therefore, guarantee the consistency of the
system.  Furthermore, we conjecture that the rewrite system is
strongly normalizing in conjunction with the definitional equality of
Agda.

\subsection{Populating the Universe}

In the previous section, we described the generic structure
associated to every monad $M : \MM$.  We now proceed to implement this
structure in concrete cases, describing in each case the most salient
features and omitting unnecessary details where we feel it will improve
the presentation.  Complete definitions can be found in the Agda
formalization.

In the material which follows, we allow ourselves the freedom to use
standard techniques such as inductive definitions and pattern matching
during the definition of each monad.  In practice, this agrees with
the implementation: there, we first define all the necessary structure
using ordinary Agda definitions and subsequently install rewrites
which connect the decoding functions to their desired implementations.
So for example, in order to define the indices of the identity monad
(see below), we first make an ordinary Agda definition
\begin{align*}
  \mathsf{IdIdx} &: \UU \\
  \mathsf{IdIdx} &= \Unit
\end{align*}
and then postulate the rewrite
\[ \Idx \Id \leadsto \mathsf{IdIdx} \]
In the presentation which follows, we omit this auxiliary step
and just write ``$=$'' when defining the structure associated
to each monad.

\subsubsection{The identity monad}

We begin by adding a constant $\Id : \MM$ to the universe to represent
the \emph{identity monad} (so named since its extension induces the
identity monad on $\UU$ up to equivalence).  The polynomial part of
$\Id$ decodes as follows:
\begin{align*}
  &\Idx \Id\, &= \Unit\\
  &\Cns \Id\, \ttt &= \Unit\\
  &\Pos \Id\, \ttt &= \Unit\\
  &\Typ \Id\, \ttt\ \ttt &= \ttt
\end{align*}
Given the triviality of the associated polynomial, it is perhaps not
surprising that its unit and multiplication are equally trivial.  Indeed,
they are given by:
\begin{align*}
  &\upeta\, \Id\, i = \ttt \\
  &\upmu\, \Id\, \_\, \delta = \delta\, \ttt
\end{align*}
We omit the remaining structure, which has a similar flavor.

\subsubsection{The pullback monad}

Our first monad constructor starts from a monad $M : \MM$ and a family
$X : \Idx M \to \UU$ and refines the indices of $M$ by additionally
decorating the inputs and output of each constructor by
elements of $X$.  We refer to the resulting monad as the
\emph{pullback of $M$ along $X$} (cf. \cite[Section 2.4]{BD98}).  We
implement this construction by first postulating a function
\[ \Pb : (M : \MM)\, (X : \Idx M \to \UU) \to \MM \] which adds the
necessary code to our universe.  We next define the polynomial part of
$\Pb M\, X$ as follows:
\begin{align*}
  &\Idx (\Pb M\, X) &&= \dsum{i : \Idx M} X\ i\\
  &\Cns (\Pb M\, X)\ (i , x) &&= \\
  &&& \hspace{-2cm}\dsum{c : \Cns M\, i} \dprod{p : \Pos M\, c} X\, (\Typ M\, c\, p) \\
  &\Pos (\Pb M\, X)\ (c , \nu) &&= \Pos M\, c\\
  &\Typ (\Pb M\, X)\ (c , \nu)\ p &&= (\Typ M\, c\ p\, ,\, \nu\ p)
\end{align*}

The unit for the pullback monad simply calls the unit of the
underlying monad and decorates its input with the same value as its
output:
\[ \upeta\ (\Pb M\, X)\ (i \,, x) = (\upeta\ M\, i\, ,\, \etadec M\, X\, x) \]

As for the multiplication of the pullback monad, it again simply calls
the multiplication of the underlying monad, this time decorating the
result using the decorations of the second-level constructors,
and forgetting the intermediate decoration.  That is, we have
\[ \upmu\ (\Pb M\, X)\ (c \,, \nu)\ \delta = (\upmu\ M\, c\ \delta', \nu') \]
where
\begin{align*}
  \delta'\ p &= \fst\, (\delta\, p)\\
  \nu'\ p &= \snd\, (\delta\, (\muposfst p))\, (\delta\, (\mupossnd p))
\end{align*}
The remaining structure is easily worked out from these definitions,
and we omit the details.

\subsubsection{The Slice Monad}
\label{sec:slice-monad}

The Baez-Dolan slice construction is at the heart of the opetopic
approach: it is this construction which allows us to ``raise the
dimension'' of the coherences in our algebraic structures.  In our
setting, it will take the form of a monad constructor
$\Slice : \MM \to \MM$.  The basic intuition is that, for a monad
$M : \MM$, the monad $\Slice M$ may be described as the \emph{monad of
  relations in $M$}.  In order to realize this intuition, we have to
find a way to encode the relations of $M$ as some kind of data, just
as the identity type encodes the relations in an ordinary type as
data.  This data will then serve as the constructors for the slice
monad.

To begin, for a monad $M : \MM$, let us define
\[ \Idx (\Slice M) = \dsum{i : \Idx M} \Cns M\, i \] That is, the
indices of the monad $\Slice M$ are exactly the constructors of the
monad $M$.  Next, we are going to capture the notion of \emph{relation
  in $M$} with the help of a certain inductive family, defined as
follows:
\begin{align*}
  &\texttt{data}\hspace{1ex} \mathsf{Tree} : \Idx (\Slice M) \to \UU \hspace{1ex} \texttt{where} \\
  &\hspace{.2cm}\lf : (i : \Idx M) \to \mathsf{Tree}\ (i\, , \upeta\ M\, i) \\
  &\hspace{.2cm}\nd : \{i : \Idx M\}\ (c : \Cns M\, i)\\
  &\hspace{.5cm}\to (\delta : (p : \Pos M\, c) \to \Cns M\, (\Typ M\, c\ p))\\ 
  &\hspace{.5cm}\to (\epsilon : (p : \Pos M\, c) \to \mathsf{Tree}\ (\Typ M\, c\ p, \delta\ p))\\ 
  &\hspace{.5cm}\to \mathsf{Tree}\ (i\, , \upmu\ M\, c\ \delta)
\end{align*}
And we define $\Cns (\Slice M) = \mathsf{Tree}$.

The reader familiar with the theory of inductive types may recognize
this as a modified form of the \emph{indexed $W$-type} associated to a
polynomial or indexed container.  Here, as in that case, the elements
of this type are \emph{trees} generated by the constructors of the
polynomial in question (the underlying polynomial of $M$, in the case
at hand).  The difference in the present setup is that our polynomial
is equipped with a multiplication and unit, and we reflect this fact
by indexing our trees not just over the indices (as is typically the
case) but also over the constructors, applying the multiplication and
unit as appropriate.  The result is that we may view an element
$\sigma : \Cns (\Slice M)\, (i \,, c)$ as ``a tree generated by the
constructors of $M$ whose image under iterated multiplication is
$c$''.  It is in this sense that this definition captures the
\emph{relations} in the original monad $M$.

We now turn to the rest of the structure required to complete the
definition of $\Slice M$.  Intuitively speaking, the positions of
a tree $\sigma$ will be its \emph{internal nodes}.  This can be
accomplished by defining the positions by recursion on the
constructors as follows:
\begin{align*}
  &\Pos (\Slice M)\, (\lf\, i) &&= \Empty \\
  &\Pos (\Slice M)\, (\nd\, c\, \delta\, \epsilon)
                               &&= \\
  &&&\hspace{-1cm} \Unit \sqcup \sum_{(p : \Pos M\, c)} \Pos (\Slice M)\, (\epsilon\, p)
\end{align*}
In other words, if our tree is a leaf, it has no positions, and if it
is a node, its positions consist of either the unit type (to record
the current node) or else the choice of a position of the base
constructor and, recursively, a node of the tree attached to that
position.

Finally, the typing function $\Typ (\Slice M)\, \sigma\, p$ just
projects out the constructor of $M$ occurring at the node of $\sigma$
specified by position $p$:
\begin{align*}
  &\Typ (\Slice M)\, (\lf\, i)\, () \\
  &\Typ (\Slice M)\, (\nd\, \{i\}\, c\, \delta\, \epsilon)\, (\inl \ttt) &&= (i \,, c) \\
  &\Typ (\Slice M)\, (\nd\, \{i\}\, c\, \delta\, \epsilon)\, (\inl (p \,, q)) &&= \\
  &&&\hspace{-2cm} \Typ (\Slice M)\, (\epsilon\, p)\, q
\end{align*}

It remains to describe the unit and multiplication of the slice monad.
In accordance with the general laws for monads, the unit constructor
needs to have a unique position, and since the positions of a given
tree are given by occurrences of constructors, this implies that
the unit at a given constructor $c$ should be the \emph{corolla},
that is, a tree with one node consisting of $c$ itself.  Therefore
we set:
\begin{align*}
  &\upeta\, (\Slice M)\, (i \,, c) &&= \\
  &&& \hspace{-2cm} \nd c\, (\lambda\, p \to\, \upeta\, M\, (\Typ M\, c\, p)) \\
  &&& \hspace{-1.3cm} (\lambda\, p \to \lf\, (\Typ M\, c\, p))
\end{align*}
Note that this definition would not be type correct without the
assumption that $M$ is definitionally right unital.  A similar remark
applies to the rest of the definitions of the slice monad in this
section.  Indeed, it is exactly the problem of completing the
definition of the slice monad without any assumptions of truncation
which motives the introduction of our monadic universe in the first
place.

Let us now sketch the definition of the multiplication in the slice
monad.  As hypotheses, we are given a tree
$\sigma : \Cns (\Slice M) (i \,, c)$ for some $i : \Idx M$ and
$c : \Cns M\, i$, as well as a decoration
$$\phi : (p : \Pos (\Slice M)\, \sigma) \to \Cns (\Slice M)\, (\Typ
(\Slice M)\, \sigma\, p)$$  In view of the preceding discussion, this
means that $\phi$ assigns to every position of $\sigma$ a tree which
multiplies to the constructor which inhabits that position.  The
multiplication of $\Slice M$ may intuitively be described as
``substituting'' each of these trees into the node it decorates.

The definition of $\upmu (\Slice M)$ will require an auxillary
function $\upgamma$ with the following type:
\begin{align*}
  \upgamma &: (M : \MM)\, \{i : \Idx M\}\, (c : \Cns M\, i) \\
           &\to (\sigma : \Cns (\Slice M)\, (i \,, c)) \\
           &\to (\phi : (p : \Pos M\, c) \to \Cns M\, (\Typ M\, c\, p)) \\
           &\to (\psi : (p : \Pos M\, c) \to \Cns (\Slice M)\, (\Typ M\, c\, p \,, \phi\, p)) \\
           &\to \Cns (\Slice M) (i \,, \upmu\, M\, c\, \phi)
\end{align*}
The intuition of this function is that $\upgamma$ \emph{grafts} the
tree specified by $\psi$ onto the appropriate leaf of the tree
$\sigma$ (indeed, $\gamma$ may be seen as an incarnation of
multiplication in the \emph{free} monad generated by the underlying
polynomial of $M$).  This function simply operates by induction and
may be defined as follows:
\begin{align*}
  &\upgamma\, M\, (\lf\, i)\, \delta\, \epsilon\, &&= \epsilon\, (\etapos M\, i) \\
  &\upgamma\, M\, (\nd\, c\, \delta\, \epsilon)\, \phi\, \psi\, &&= \nd\, c\, \delta'\, \epsilon'
\end{align*}
where we define
\begin{align*}
  \phi'\, p\, q &= \phi (\mupos M\, c\, \delta\, p\, q) \\
  \psi'\, p\, q &= \psi (\mupos M\, c\, \delta\, p\, q) \\
  \delta'\, p &= \upmu\, M\, (\delta\, p)\, (\phi' p) \\
  \epsilon'\, p &= \upgamma\, M\, (\epsilon\, p)\, (\psi' p)
\end{align*}
With this function in hand, we may complete the definition
of the multiplication in the slice monad as
\begin{align*}
  &\upmu\, (\Slice M)\, (\lf\, i)\, \phi &&= \lf\, i \\
  &\upmu\, (\Slice M)\, (\nd\, c\, \delta\, \epsilon)\, \phi &&= \upgamma\, M\, w\, \delta\, \psi
\end{align*}
where we put
\begin{align*}
  w &= \phi\, (\inl \ttt) \\
  \phi'\, p \, q &= \phi\, (\inr (p \,, q)) \\
  \psi\, p &= \upmu\, (\Slice M)\, (\epsilon\, p)\, (\phi'\, p)
\end{align*}
This definition then says that substitution is trivial on leaves, and
when we are looking at a node, we first retrieve the tree living at
this position (called $w$ above), and then graft to it the result of
recursively substituting in the remaining branches.

We refer the reader to the formalization for details on the remaining
constructions handling positions in the slice monad.\\

%\input{slice-diagrams.tex}

\subsection{Dependent monads}

Since the notion of dependent type is one of the primitive aspects of
Martin-L\"{o}f type theory, it is perhaps not surprising that we
quickly find ourselves in need of a dependent version of our
polynomial monads.  We note there is a potential point of confusion
here: while a dependent type can be thought of as a family of types
dependent on a base type, a dependent monad in our sense is \emph{not}
a family of monads.  Rather, it is a monad structure on dependent
families of indices and constructors indexed over the indices and
constructors of the base monad $M$.  Put another way, under the
equivalence between dependent types with \emph{domain} $A$ and
functions with \emph{codomain} $A$, our dependent monads over a base
monad $M$ correspond to monads $M'$ equipped with a \emph{cartesian
  homomorphism} to $M$. \footnote{In fact, it is entirely possible to
  add a monadic form of dependent sum to the list of monad
  constructors of the universe $\MM$ so that this statement becomes
  literally true.  As we will not have need of this construction in
  the present article, however, we omit the details.}  The advantage
of working in a dependent style, however, is that we do not need to
axiomatize the notion of homomorphism using propositional equalities
as it is encoded directly in the typing of the multiplication
operator.

To begin, let us postulate, for each monad $M : \MM$, a universe
$\MMd\, M : \UU$ of \emph{monads over $M$}.  
\[
  \MM\da : \MM \rightarrow \UU
\]
That is, for $M : \MM$, the inhabitants of $\MMd M$ are codes for
monads equipped with a cartesian morphism to $M$.  For this reason,
when we are given a monad $M$ and a dependent monad $M\da : \MMd\, M$,
we often speak of the pair $(M \,, M\da)$ as a \emph{monad extension}.

The decoding functions for dependent monads follow the same setup as
the non-dependent case, simply repeating each of the definitions
fiberwise.  And since the dependent case resembles so closely the
non-dependent one, we have attempted to systematically name dependent
versions of the the monadic structure introduced above by
appending a ``$\da$'' to the previously given name.  For example,
$\Idxd$ for the dependent version of the family $\Idx$ of indices.

As a first step, a dependent monad will be equipped with a polynomial
lying over the base polynomial.  This corresponds to the following
three dependent families:
\begin{align*}
  \Idxd &: \{M : \MM\} \to \MMd\ M \to \Idx\ M \to \UU\\
  \Cnsd &: \{M : \MM\}\ (M\da : \MMd\ M)\ \{i : \Idx\ M\}\\ 
        &\to \Idxd\ M\da\ i \to \Cns\ M\ i \to \UU\\
  \Typd &: \{M : \MM\}\ (M\da : \MMd\ M)\\
        &\to \{i : \Idx\ M\}\ \{i\da : \Idxd\ M\da\ i\}\\
        &\to \{c : \Cns\ M\ i\}\ (c\da : \Cnsd\ M\da\ i\da\ c)\\
        &\to \Pos\ M\ c \to \Idxd\ M\da\ (\Typ\ M\ c\ p)
\end{align*}
The reader will notice, however, that there is no analog of dependent
positions.  This is because we are modelling \emph{cartesian}
morphisms of monads, and therefore positions of a dependent
constructor $c\da : \Cnsd\ M\da\ i\da\ c$ should be the same as those
of the underlying constructor $c$.  By working fiberwise, we can
reflect this requirement directly in the type signature.

The monadic structure of a dependent monad simply operates fiberwise,
following the multiplication in the base monad:
\begin{align*}
  \upetad &: \{M : \MM\}\ (M\da : \MMd\ M)\\ 
            &\to \{i : \Idx\ M\} \to \Idxd\ M\da\ i\\ 
            &\to \Cnsd\ M\da\ i\, (\upeta\, M\, i)\\
  \upmud &: \{M : \MM\}\ (M\da : \MMd\ M)\\ 
           &\to \{i : \Idx\ M\}\ \{c : \Cns\ M\ i\} \\
           &\to \{\delta : (p : \Pos\ M\ c) \to \Cns\ M\ (\Typ\ M\ c\ p)\}\\
           &\to (i\da : \Idxd\ M\da\ i)\ (c\da : \Cnsd\ M\da\ i\da\ c)\\
          &\to (\delta\da : (p : \Pos\ M\ c) \to \\
          &\hspace{1cm} \Cnsd\ M\da\ (\Typd\ M\da\ c\da\ p)\ (\delta\ p))\\
           &\to \Cnsd\ M\da\ i\da\ (\upmu\ M\ c\ \delta)
\end{align*}
The fact that we require the multiplication of a family of dependent
constructors to live over the multiplication of the base constructors
(and similarly for the unit) is what guarantees the homomorphism
property alluded to above.

Our dependent monads must also be equipped with equational laws making
them compatible with the corresponding laws of the monads they live
over.  For example, the typing functions for $\upetad$ and $\upmud$
respect the indices of parameters, just as in the base case:
\begin{align*}
  \Typd\ M\da\ &(\upetad\ M\da\ i\da)\ p \leadsto i\da\\
  \Typd\ M\da\ &(\upmud\ M\da\ c\da\ \delta\da)\ p \leadsto \\
               &\Typd\ M\ (\delta\da\ (\muposfst\da M\da\ p))\ \\
               &\hspace{1.5cm} (\mupossnd\da M\da\ p)
\end{align*}
There are similar laws asserting the definitional associativity and
unitality of the multiplication, but as these all follow exactly
the same pattern, we omit the details here and refer the curious
reader to the implementation.

We remark that, because their positions are the same, decorations of
the dependent constructor $\upeta\da$ are essentially constant just as
in the case of $\upeta$, and there is therefore an analogous function
$\etadecd$ generating such decorations from a single piece of data
with a similar definition.  This function occurs occasionally in the
code below.

\subsection{Populating the dependent universe}

We now quickly describe dependent counterparts of the base monads and
monad constructors of the previous section.  As most of the
definitions are routine and easily deduced from the absolute case, the
presentation here is brief.

\subsubsection{The identity monad}

The dependent identity monad is parametrized by a type $A : \UU$ and
indexed over the identity monad $\Id$.  That is, we have a dependent
monad constructor of the form
\[
  \Idd : \UU \to \MMd\ \Id 
\]
Its polynomial part is defined by
\begin{align*}
  &\Idxd (\Idd A)\, \ttt &= A\\
  &\Cnsd (\Idd A)\, x\, \ttt &= \Unit\\
  &\Posd (\Idd A)\, \ttt\, \ttt &= \Unit\\
  &\Typd (\Idd A)\, \{i\da = x\}\ \ttt\ \ttt &= x
\end{align*}
As in the base case, the multiplication and unit all take values in
the unit type, making the structure essentially trivial.

\subsubsection{The dependent pullback monad}

Just as we can refine the indices of a base monad, so the dependent
pullback monad allows us to refine the indices of a dependent monad.
Its constructor takes the form
\begin{align*}
  \Pbd &: \{M : \MM\}\ (M\da : \MMd\ M)\ \{X : \Idx\ M \to \UU\}\\
       &\to (X\da : \{i : \Idx\ M\} \to \Idxd\ M\da\ i \to X\ i \to \UU)\\
       &\to \MMd\ (\Pb\ M\ X)
\end{align*}
Note that the family $X\da$ may also depend on elements of the
refining family $X$ for the base monad.  The underlying polynomial of
the dependent pullback is then defined as follows:
\begingroup
\addtolength{\jot}{1em}
\begin{align*}
  &\Idxd\ (\Pbd M\da\ X\da)\ (i, x) = \sum_{(i\da : \Idxd\, M\da\, i)} X\da\ i\da\ x\\
  &\Cnsd\ (\Pbd M\da\ X\da)\ (i\da , x\da)\ (c , \nu) =\\
  &\hspace{.5cm} \sum_{(c\da : \Cnsd\, M\da\, i\da\, c)} \prod_{(p : \Pos M\, c)} X\da\, (\Typd M\da\, c\da\, p)\, (\nu\, p) \\
  &\Typd\ (\Pbd M\da\, X\da)\ (c\da , \nu\da)\ p = \Typd\ M\, c\da\, p, \nu\da\, p
\end{align*}
\endgroup with multiplicative structure following fiberwise the rules
for the base pullback $\Pb M\, X$.

\subsubsection{The dependent slice monad}

Finally, the dependent slice monad extends the Baez-Dolan
slice construction to the dependent case.  Its monad constructor
is typed as follows:
\[ \Sliced : \{M : \MM\}\, (M\da : \MMd\, M) \to \MMd\ (\Slice M) \]
As for the absolute case, the indices are given by the dependent
constructors.  That is, we set
\[ \Idxd (\Sliced M\da)\ (i , c) = \sum_{i\da : \Idxd M\da\, i} \Cnsd
  M\da\, i\da\, c \] Similarly, the type of constructors
$\Cnsd (\Sliced M\da)$ are trees lying over a tree in the base.  This
corresponds to the following (rather verbose) inductive type:
\begin{align*}
  &\texttt{data}\hspace{1ex} \mathsf{Tree\da} : \{i : \Idx (\Slice M)\} \to (i\da : \Idxd (\Sliced M\da) \\
  &\hspace{1cm} \to \mathsf{Tree}\, i \to \UU \hspace{1ex} \texttt{where} \\
  &\lfd : \{i : \Idx M\}\ (i\da : \Idxd M\da\, i)\\
  &\hspace{.2cm}\to \Cnsd (\Sliced M\da)\, (i\da, \upetad M\da\, i\da)\, (\lf i) \\
  &\ndd : \{i : \Idx M\}\ \{c : \Cns M\, i\}\\ 
  &\hspace{.2cm}\to \{\delta : (p : \Pos M\, c) \to \Cns M\, (\Typ M\, c\, p)\}\\
  &\hspace{.2cm}\to \{\epsilon : (p : \Pos M\, c) \\
  &\hspace{1.2cm}\to \Cns (\Slice M)\, (\Typ M\, c\, p, \delta\, p)\}\\ 
  &\hspace{.2cm}\to \{i\da : \Idxd M\da\, i\} \to (c\da : \Cnsd M\da\, i\da\, c)\\
  &\hspace{.2cm}\to (\delta\da : (p : \Pos M\, c) \to \Cnsd M\da\, (\Typd M\da\, c\da\, p))\\ 
  &\hspace{.2cm}\to (\epsilon\da : (p : \Pos M\, c) \\
  &\hspace{1.2cm}\to \Cnsd (\Sliced M\da)\, (\Typd M\da\, c\da\, p, \delta\da\, p))\\ 
  &\hspace{.2cm}\to \Cnsd (\Sliced M\da)\, (i\da \,, \upmud M\da\, c\da\, \delta\da)\, (\nd c\, \delta\, \epsilon)
\end{align*}
The rest of the description of the dependent slice follows exactly the
same pattern: duplicating the definitions and laws of the base case
routinely in each fiber.

%%% Local Variables:
%%% mode: latex
%%% TeX-master: "types-are-grpds-ext"
%%% End:

\section{Opetopic Types}
\label{sec:opetopic-types}

In this section, we show how to use the universes introduced above in
order to implement Baez and Dolan's definition of \emph{opetopic type}
\cite{BD98}.  We go on to explain how to use this definition to capture
the notion of \emph{weak $M$-algebra}, and finish with some examples.

\begin{defn}
  An \textbf{opetopic type over a monad $M$} is defined coninductively
  as follow:
  \begin{align*}
    &\texttt{record}\hspace{1ex} \OpType (M : \MM) : \UU_1 \hspace{1ex} \texttt{where} \\
    &\hspace{.5cm}\Car : \Idx M \to \UU \\
    &\hspace{.5cm}\Rel : \OpType (\Slice (\Pb M\, \Car))
  \end{align*}
\end{defn}
We see from the definition that an opetopic type consists of
an infinite sequence of dependent families
\[ \Car X \,, \Car (\Rel X) \,, \Car (\Rel (\Rel X)) \,, \dots \]
whose domain is the set of indices of a monad whose definition
incorporates all the previous families in the sequence.  Given an
opetopic type $X : \OpType M$, we will often denote this sequence of
dependent types more succinctly as just $X_0, X_1, X_2, \dots$ since
the destructor notation quickly becomes quite heavy.  We will use a
similar convention for the series of monads $M = M_0, M_1, M_2 \dots$
generated by the definition.  That is, we have:
\begin{equation}
  \label{eq:unfold}
  \begin{aligned}
    &M_0 = M && X_0 = \Car X : \Idx M \to \UU \\
    &M_1 = \Slice (\Pb M_0\, X_0) && X_1 = \Car (\Rel X) : \Idx M_1 \to \UU \\
    &M_2 = \Slice (\Pb M_1\, X_1) && X_2 = \Car (\Rel (\Rel X)) : \\
    &&&\hspace{2cm} \Idx M_2 \to \UU \\
    &\hspace{1cm} \vdots && \hspace{1cm} \vdots
  \end{aligned}
\end{equation}

Before describing the connection between opetopic types and weak
$M$-algebras, let us give some examples of how to think of the
resulting dependent families as ``fillers'' for a collection of
``shapes'' generated by the monad $M$.  For concreteness, we will fix
$M = \Id$ in our examples.  Given $X : \OpType \Id$, we can define the
type of \emph{objects} of $X$ as simply
\begin{equation}
  \label{eq:obj-defn}
  \begin{aligned}
    &\mathsf{Obj} : \UU \\
    &\mathsf{Obj} = \Car X\, \ttt
  \end{aligned}
\end{equation}
Next, after a single slice, $X$ provides us with a type of
\emph{arrows} between any two objects which can be defined as follows:
\begin{equation}
  \label{eq:arrow-defn}
  \begin{aligned}
    &\mathsf{Arrow} : (x\, y : \mathsf{Obj}) \to \UU \\
    &\mathsf{Arrow}\, x\, y = \Car (\Rel X)\, \\
    &\hspace{2cm} ((\ttt \,, y) \,, (\ttt , \etadec \Id\, (\Car X)\, x))
  \end{aligned}
\end{equation}
Furthermore, for a \emph{loop} $f$ in $X$, that is, an arrow with the
same domain and codomain, $X$ includes a family whose elements can be
thought of as ``null-homotopies of $f$'', and which is defined by
\begin{align*}
  &\mathsf{Null} : (x : \mathsf{Obj})\, (f : \mathsf{Arrow}\, x\, x) \to \UU \\
  &\mathsf{Null}\, x\, f = \Car (\Rel (\Rel X)) \\
  & \hspace{2cm} ((((\ttt \,, x) \,, (\ttt , \etadec \Id\, (\Car X)\, x)) \,, f) \,, \\
  & \hspace{2.5cm} \lf\, (\ttt \,, x) \,, \bot\mhyphen\mathsf{elim})
\end{align*}
More examples of shapes and filling families may be found in the
development.

\subsection{Weak Algebras and Fibrant Opetopic Types}
\label{sec:weak-alg}

We now wish to describe how an opetopic type $X : \OpType M$ encodes
the structure of a weak $M$-algebra.  Before we begin, it will be
convenient to adopt the following convention: recall that $X$ consists
of an infinite sequence of dependent types following the form of
Equation \ref{eq:unfold}.  In the discussion which follows, instead of
working with a fixed opetopic type $X$, we will rather just work with
abstract type families $X_0 , X_1 , \dots$ over monads
$M = M_0, M_1, \dots$ following the same pattern of dependencies. We
will then freely add new families of the form $X_i$ to our hypotheses
as they become necessary.  The advantage of working this way is that
our definitions are parameterized over just that portion of the
opetopic type which is necessary, as opposed to depending on the
entire opetopic type $X$ itself, and consequently, we will be able to
reuse our definitions and constructions starting at any point in the
infinite sequence generated by $X$.

We recall that for $M$ a polynomial monad, an \emph{$M$-algebra}
consists of a \emph{carrier family} $C : \Idx M \to \UU$ together with
a map
\[ \alpha : \{i : \Idx M\} \to [ M ]\, C\, i \to C\, i \] which
satisfies some equations expressing the compatibility of $\alpha$ with
the multiplication of $M$.  Indeed, it is the need for a complete
description of these equations in all dimensions which motivates the
present work.  Now, clearly the first dependent type
$X_0 : \Idx M \to \UU$ may serve as a carrier for an $M$-algebra
structure.  Let us now see what else this sequence of families
provides us with.

After one iteration, we obtain a type family $X_1 : \Idx M_1 \to \UU$,
and unfolding the definition of the indices of the slice and pullback
monads, we find that the domain of $X_1$ takes the form
\begin{align*}
  \sum_{(i : \Idx M)} \sum_{(x : X_0 i)}
  \sum_{(c : \Cns M\ i)} (p : \Pos M\, c) \to X_0\, (\Typ M\, c\, p)
\end{align*}
The elements of this type are 4-tuples $(i \,, x \,, c \,, \nu)$, and
we now observe that the three elements $i$, $c$ and $\nu$ are typed
such that they are exactly the arguments of the hypothetical algebra
map $\alpha$ introduced above.  We may regard the family $X_1$,
therefore, as a relation between triples $(i \,, c \,, \nu)$ and
elements $x : X_0\, i$, and in order to define a map $\alpha$, we only
need to impose that this relation is functional in the sense that
there is a \emph{unique} $x$ determined by any such triple.  When this
is the case, we will say that the family $X_1$ is
\emph{multiplicative}.  That is, we define:
\begin{align*}
  &\ismult : \{X_0 : \Idx M_0 \to \UU\}\, (X_1 : \Idx M_1 \to \UU) \to \UU \\ 
  &\ismult \{X_0\}\, X_1 = \{i : \Idx M\}\, (c : \Cns M\, i) \\
  &\hspace{1cm} \to (\nu : (p : \Pos M\, c) \to X_0\, (\Typ M\, c\, p))\\
  &\hspace{1cm} \to \iscontr (\sum_{x : X_0\, i} X_1\, (i \,, x \,, c \,, \nu))
\end{align*}
Supposing we are given a proof $m_1 : \ismult X_1$, we can define an
algebra map $\alpha$ as above by
\[\alpha\, (c \,, \nu) = \fst (\ctr (m_1\, c\, \nu)) \]
Furthermore, we will write
\[\alpha\wit\, (c \,, \nu) = \snd (\ctr (m_1\, c\, \nu)) \]
for the associated element of the relation
$X_1 (i \,, \alpha\, (c \,, \nu) \,, c \,, \nu)$ which witnesses
this multiplication.

Let us now suppose that our sequence extends one step further, that
is, that we are given a type family $ X_2 : \Idx M_1 \to \UU $ and a
proof $m_2 : \ismult X_2$.  We now show how to use this further
structure to derive some of the expected \emph{laws} for the algebra
map $\alpha$ we have just defined.  As a first example, we expect
$\alpha$ to satisfy a unit law: decorating a unit constructor with
some element $x$ and then applying $\alpha$ should return the element
$x$ itself.  In other words, we expect to be able to prove
\begin{align*}
  &\alpha^{\eta}\coh : \{i : \Idx M\} (x : X_0\, i) \\
  &\hspace{.3cm} \to \alpha\, (\upeta\, M\, i \,, \etadec M\, X_0\, x)\, \equiv x
\end{align*}
To prove this equality, let us define the following function:
\begin{align*}
  &\upeta\alg_{m_2} : \{i : \Idx M\} (x : X_0\, i) \\
  % &\upeta\alg_{m_2} : \{i : \Idx M\} (x : X_0\, i) \\
  &\hspace{.3cm} \to X_1\, ((i \,, x) \,, (\upeta\, M\, i , \etadec M\, X_0\, x)) \\
  &\upeta\alg_{m_2} = \fst (\ctr (m_2\, (\lf (i \,, x))\, \botelim)) 
\end{align*}
Now we simply notice that the pairs
\[ \ctr (m_1\, (\upeta\, M\, i)\, (\etadec M\, X_0\, x)) \equiv (x \,,
  \upeta\alg\, x) \] must be equal as indicated, since they inhabit a
contractible space.  Projecting on the first factor gives exactly the
desired equation.

We also expect our algebra map $\alpha$ to satisfy an equation
expressing its compatibility with multiplication of the following
form:
\begin{align*}
  &\alpha^{\upmu}\coh : \{i' : \Idx M\}\, (c' : \Cns M\, i)\\
  &\hspace{.2cm} \to (\delta' : (p : \Pos M\, c) \to \Cns M (\Typ M\, c\, p))\\
  &\hspace{.2cm} \to (\nu' : (p : \Pos M\, c') (q : \Pos M\, (\delta'\, p)) \\
  &\hspace{3cm} \to X_0\, (\Typ M\ (\delta'\, p)\, q)) \\
  &\hspace{.2cm} \to \alpha\, (\upmu\, M\, c' \, \delta')\, (\lambda p \to \nu'\, (\muposfst p)\, (\mupossnd p)) \equiv\\
  &\hspace{.7cm} \alpha\, c'\, (\lambda p \to \alpha\, (\delta'\, p)\, (\nu'\, p))
\end{align*}
We note that this equation is simply the type theoretic translation of
the familiar commutative diagram
\[
  \begin{tikzcd}
    {[ M ]\, [ M ]\, X_0} \ar[r,"\upmu_{X_0}"] \ar[d,"{[ M ]\, \alpha}"'] &
    {[ M ]\, X_0} \ar[d,"\alpha"] \\
    {[ M ]\, X_0} \ar[r,"\alpha"'] & X_0
  \end{tikzcd}
\]
To prove this axiom, we use $m_2$ to define the following
multiplication operation on elements of the family $X_1$:
\begin{align*}
  &\upmu\alg_{m_2} : \{i : \Idx M\}\, (c : \Cns M\, i)\\
  &\hspace{.3cm}\to (\nu : (p : \Pos M c) \to X_0\, (\Typ M\, c\, p)) \\
  % &\hspace{.3cm}\to (\delta : \Decor (c \,, \nu)\, (\Cns (\Pb M\, X_0))) \\
  &\hspace{.3cm}\to (\delta : (p : \Pos M c)\\
  &\hspace{2cm} \to  \Cns (\Pb M\, X_0) (\Typ (\Pb M\, X_0)\, (c, \nu)\, p)) \\
  &\hspace{.3cm}\to (x_0 : X_0\, i)\, (x_1 : X_1\, (i \,, x_0 \,, c \,, \nu)) \\
  &\hspace{.3cm}\to (\bar{x} : (p : \Pos M\, c) \to X_1 (\Typ (\Pb M\, X_0) (c \,, \nu) \,, \delta\, p)) \\
  &\hspace{.3cm}\to X_1 (i \,, x_0 \,, \upmu\, (\Pb M\, X_0)\, (c \,, \nu)\, \delta) \\
  &\upmu\alg_{m_2} = \fst (\ctr (m_2\, \sigma\, \theta))
\end{align*}
where
\begin{align*}
  &\sigma = \nd (c \,, \nu)\, \delta\, (\lambda p \to \upeta\, M_1\, ((\Typ M\, c\, p \,, \nu\, p) \,, \delta\, p)) \\
\end{align*}
is the two-level tree consisting of a base node $(c , \nu)$, as well
as a second level of constructors specified by the decoration
$\delta$, and $\theta$ is the decoration of the nodes
of $\sigma$ by elements of $X_1$ defined by:
\begin{align*}
  &\theta\, (\inl \ttt) = x_1 \\
  &\theta\, (\inr (p \,, \inl \ttt)) = \bar{x}\, p \\
\end{align*}
Now instantiating our function $\upmu\alg_{m_2}$ with arguments
\begin{align*}
  &c = c' &&x_1 = \alpha\mhyphen\mathsf{wit}\, (c \,, \nu) \\
  &\nu\, p = \alpha\, (\delta'\, p \,, \nu'\, p) &&\delta\, p = (\delta'\, p \,, \nu' \, p) \\
  &x_0 = \alpha\, (c \,, \nu) &&\bar{x}\, p = \alpha\mhyphen\mathsf{wit}\, (\delta'\, p \,, \nu'\, p) 
\end{align*}
we find that the pairs
\begin{align*}
  &\ctr (m_1\, (\upmu\, M\, c'\, \delta')\, (\lambda p \to \nu'\, (\muposfst p)\, (\mupossnd p))) \equiv \\
  &\hspace{.3cm} (\alpha\, (c \,, \nu) \,,  \upmu\alg_{m_2}\, c\, \nu\, x_0\, x_1\, \delta\, \bar{x})
\end{align*}
again inhabit a contractible space, whereby their first components are
equal, giving the desired equation.

We may think of the functions $\upeta\alg_{m_2}$ and $\upmu\alg_{m_2}$
defined above as the nullary and binary cases of a multiplicative
operation on the \emph{relations} of our algebra structure.  The key
insight, as we have seen, is that this multiplicative structure
encodes exactly the \emph{laws} for the algebra map $\alpha$ defined
one level lower.  Similarly, if we are able to extend our sequence on
\emph{further} step to a family $X_3$ which is itself multiplicative,
then we will be able to show that the operations $\upeta\alg_{m_2}$
and $\upmu\alg_{m_2}$ \emph{themselves satisfy unit and associativity
  laws}, and this in turn encodes the ``2-associativity'' and
``2-unitality'' of the algebra map $\alpha$.  This motivates
the following definition:

\begin{defn}
  An opetopic type $X$ over a monad $M$ is said to be \textbf{fibrant}
  if we are given an element of the following coinductively defined
  property:
  \begin{align*}
    &\texttt{record}\hspace{1ex} \isfibrant \{M : \MM\}\, (X : \OpType M) : \UU\\
    &\hspace{.3cm} \texttt{where} \\
    % &\hspace{.3cm} (X : \OpType M) : \UU \hspace{1ex} \texttt{where} \\
    &\hspace{.5cm} \carmult : \ismult M\, (\Car (\Rel X)) \\
    &\hspace{.5cm} \relfib : \isfibrant (\Rel X)
  \end{align*}
\end{defn}
Fibrant opetopic types, therefore, are our definition of infinitely
coherent $M$-algebras, with the multiplicativity of the relations
further in the sequence witnessing the higher dimensional laws
satisfied by the structure earlier in the sequence.

\subsection{Higher structures}
\label{sec:higher-structures}

We now use the preceding notions to define a number of coherent
algebraic structures.  A first example is that we obtain an internal
definition of the notion of $\infty$-groupoid as follows: 
\begin{defn}
  An \textbf{$\infty$-groupoid} is a fibrant opetopic type over
  the identity monad.  That is,
\[ \ooGrp = \dsum{X : \OpType \Id} \isfibrant X \] 
\end{defn}
\noindent We will attempt to justify the correctness of this
definition in the sections which follow.

Next, it happens that the monad $\Slice \Id$ is in fact the monad
whose algebras are monoids, and consequently, our setup leads
naturally to the definition of an $\mathbb{A}_\infty$-type, that is, a
type with a coherently associative binary operation.

\begin{defn}
  An \textbf{$\mathbb{A}_\infty$-type} is a fibrant opetopic type
  over the first slice of the identity monad.
  \[ \mathbb{A}_\infty\mhyphen\mathsf{type} = \dsum{X : \OpType
      (\Slice \Id)} \isfibrant X \]
\end{defn}
Furthermore, the notion of $\mathbb{A}_\infty$-group can now be
defined by imposing an invertibility axiom.  A classical theorem of
homotopy theory asserts that the type of $\mathbb{A}_\infty$-groups is
equivalent to the type of pointed, connected spaces via the loop-space
construction.  It would be interesting to see if the techniques of
this article lead to a proof of this fact in type theory.

The notion of $\infty$-category can also be defined using this setup.
Recall that an opetopic type over the identity monad $\Id$ has both a
type of objects and a type of arrows (Equations \ref{eq:obj-defn} and
\ref{eq:arrow-defn}).  In the definition of $\infty$-groupoid above,
the invertibility of the arrows in the underlying opetopic type is a
consequence of the fact that the family of arrows is assumed to be
multiplicative.  Consequently, we obtain a reasonable notion of a
\emph{pre-$\infty$-category} by simply dropping this assumption, and
only requiring fibrancy after one application of the destructor
$\Rel$:
\[ \preooCat = \dsum{X : \OpType \Id} \isfibrant (\Rel X) \] The
prefix ``pre'' here refers to the fact that this definition is missing
a completeness axiom asserting that the invertible arrows coincide
with paths in the space of objects, that is, an axiom of
\emph{univalence} in the sense of \cite{ahrens2015univalent}.  Such an
axiom is easily worked out in the present setting, but as it would
distract us slightly from the main objective of the present work, we
will not pursue the matter here.

\section{The $\infty$-groupoid associated to a type}
\label{sec:infty-group-assoc}

In this section, we use the machinery we have set up to produce an
$\infty$-groupoid associated to any type and eventually prove it is
unique.  As a first step, we will need a source of opetopic types.
Here is where the notion of dependent monad becomes important: we now
show that every dependent monad gives rise to an opetopic type.  The
reason for this phenomenon is simple: since our dependent monad
constructors mirror the monad constructors of the absolute case, any
monad extension $(M \,, M\da)$ in fact gives rise to a \emph{new}
monad extension as follows:
\begin{align*}
  M &\quad \mapsto \quad \Slice (\Pb M\, (\Idxd M\da)) \\
  M\da &\quad \mapsto \quad \Sliced (\Pbd M\da\, (\lambda\, j\, k \to j \equiv k))
\end{align*}
Notice how by pulling back along $\Idxd M\da$, the identity type gives
us a canonical family along which to apply the $\Pbd$ constructor.
Iterating this construction, then, we find that associated to every
monad extension $(M \,, M\da)$, is an infinite sequence
\[ (M \,, M\da) = (M_0 \,, M\da_0), (M_1 \,, M\da_1), (M_2 \,, M\da_2),
  \dots \] where $(M_{i+1} \,, M\da_{i+1})$ is obtained from
$(M_i \,, M\da_i)$ by the above transformation.

The above construction provides us with our desired source of opetopic
types.  Formally, we define (using copattern notation)
\begin{align*}
  &\OvrOpType M\, M\da : \OpType M  \\
  &\Car (\OvrOpType M\, M\da) = \Idxd\, M\da \\
  &\Rel (\OvrOpType M\, M\da) = \\
  &\hspace{.5cm}\OvrOpType (\Slice (\Pb (\Idxd\, M\da))) \\
  &\hspace{.8cm} (\Sliced (\Pbd\, M\da\, (\lambda\, j\, k \to j \equiv k)))
\end{align*}
Specializing to the case of the identity monad, we obtain the
following:

\begin{defn}
  For a type $A : \UU$, the \textbf{underlying opetopic type
    of $A$} is defined to be the opetopic type associated
  to the dependent identity monad determined by $A$.  That is,
  the opetopic type
  \[ \OvrOpType \Id\, (\Id\da\, A) \]
  in the notation of the previous paragraph.
\end{defn}
In order to show that every type $A$ determines an $\infty$-groupoid
in our sense, our next task is to show that this opetopic type is
in fact fibrant.

\subsection{Algebraic Extensions}
\label{sec:algebraic-extensions}

Let $M : \MM$ and $M\da : \MMd$.  We will say that the extension
$(M, M\da)$ is \emph{algebraic} if we have a proof
\begin{align*}
  &\isalgebraic : (M : \MM)\, (M\da : \MMd)\, \to \UU \\
  &\isalgebraic = \{i : \Idx M\}\, (c : \Cns M\, i) \\
  &\hspace{.3cm} \to (\nu : (p : \Pos M\, c) \to \Idxd M\da\, (\Typ M\, c\, p)) \\
  &\hspace{.3cm} \to \iscontr \left (\sum_{(i\da : \Idxd M\da)} \sum_{(c\da : \Cnsd M\da\, i\da)} \Typd M\da\, c\da \equiv \nu \right )
\end{align*}
An algebraic extension should be thought of as roughly analogous to a
generalized kind of opfibration: if we think of the constructors as
generalized arrows between their input indices and output, then the
hypothesis says we know a family of lifts over the source of our
constructor, and the conclusion is that there exists a unique
``pushforward'' consisting of a lift over the output as well as a
constructor connecting the two whose typing function agrees with the
provided input lifts.  Such a hypothesis is one way of encoding an
$M$-algebra, which motivates the name for this property.
See~\cite[Section 6.3]{leinster2004higher}.

The main use of the notion of algebraic extension is the following
lemma, whose proof is entirely straightforward:
\begin{lem}
  \label{lem:alg-to-fib}
  Suppose the pair $(M \,, M\da)$ is an algebraic extension.  Then
  the relation $\Idxd M\da_1$ is multiplicative.
\end{lem}
Consequently, just as dependent monads are a source of opetopic types,
algebraic extensions can be thought of as a source of multiplicative
relations.  Hence if we want to prove fibrancy of the opetopic type
associated to a monad extension, we will need to know which of the
extensions in the generated sequence are algebraic.  Our main theorem
is that after a single iteration of the slice construction,
\emph{every} monad extension becomes algebraic.  That is

\begin{thm}
  \label{thm:slice-alg}
  Let $(M \,, M\da)$ be a monad extension.  Then slice extension
  $(M_1 \,, M\da_1)$ is algebraic.
\end{thm}

A proof can be found in the extended version of this article
\cite{allioux2021types}.  The importance of the theorem is that it has
the following immediate corollaries:

\begin{cor}
  \label{cor:alg-is-fib}
  Let $(M \,, M\da)$ be an algebraic extension.  Then the opetopic
  type $\OvrOpType M\, M\da$ is fibrant.
\end{cor}

\begin{proof}
  The base case of the coinduction is Lemma \ref{lem:alg-to-fib} and
  the coinductive case is covered by Theorem
  \ref{thm:slice-alg}.
\end{proof}

\begin{cor}
  There is a map $\Gamma : \UU \to \ooGrp$.
\end{cor}

\begin{proof}
  Let $A : \UU$ be a type.  A short calculation shows that
  the monad extension $(\Id \,, \Id\da\, A)$ is algebraic.
  The result therefore follows from Corollary \ref{cor:alg-is-fib}.
\end{proof}

\subsection{Uniqueness of the Groupoid Structure}

We now turn to the task of showing the map $\Gamma : \UU \to \ooGrp$
is an equivalence.  Observe that there is a forgetful map
$\Upsilon : \ooGrp \to \UU$ which is given by extracting the type of
objects (Equation \ref{eq:obj-defn}) from the opetopic type underlying
a groupoid $G : \ooGrp$.  It is readily checked that the composite
$\Upsilon \circ \Gamma$ is definitionally the identity, and so what
remains to be shown is that any $G : \ooGrp$ is equivalent to $\Gamma$
applied to its type of objects.

Unwinding the definitions, we find that we are faced with the following
problem: suppose we are given a monad extension $(M \,, M\da)$ as well
as a opetopic type $X : \OpType M$.  Under what hypotheses can we
prove that $X \simeq_{o} \OvrOpType M\, M\da$ (where $\simeq_o$
denotes an appropriate notion of equivalence of opetopic types)?  A
first remark is that the opetopic type $\OvrOpType M\, M\da$ is
completely determined by the algebraic structure of the dependent
monad $M\da$.  Therefore, at a minimum, we must assume that the data
of the opetopic type $X$ is equivalent to the data provided by $M\da$
wherever they ``overlap''.

To see what this means concretely, let us begin at the base of the
sequence, writing $Z = \OvrOpType M\, M\da$ to reduce clutter.  Now,
the family $Z_0 : \Idx M \to \UU$ is, by definition, given by the
family of dependent indices $\Idxd M\da$ of the dependent monad
$M\da$.  On the other hand, without additional hypotheses, the
opetopic type $X$ only provides us with some abstract type family
$X_0 : \Idx M \to \UU$.  Clearly, then, we will need to assume an
equivalence $e_0 : (i : \Idx M) \to \Idxd M\da\, i \simeq X_0\, i$ in
order to have any chance to end up with the desired equivalence of
opetopic types.

Moving on to the next stage, here we find that $Z_1$ is given
by the dependent indices
\[ \Idxd M\da_1 : \Idx M_1 \to \UU \] of the first iteration of the
dependent slice-pullback construction.  Unfolding the definition,
these are of the form
\begin{align*}
  (\Idxd M\da_1)\, (i \,, j \,, c \,, v)
  &= \\
  & \hspace{-2cm} \sum_{(j : \Idxd M\da\, i)} \sum_{(r : j \equiv j')} 
    \sum_{(d : \Cnsd M\da\, c)} (\Typd M\da\, d \equiv \nu)
\end{align*}
With the 4-tuple $(i \,, j \,, c \,, v)$ as in Equation
\ref{eq:slc-idx}.  We notice that much of the data here is redundant:
by eliminating the equality $r$ and the equality relating $\nu$ to the
typing function of $d$, we find that the dependent indices are
essentially just dependent constructors of $M\da$, slightly reindexed.
In other words, a dependent equivalence
\[ e_1 : (i : \Idx M_1) \to \Idxd M\da_1 \simeq_{e_0} X_1 \] over the
previous equivalence $e_0$ amounts to saying that the relations of the
family $X_1$ ``are'' just the dependent constructors of $M\da$ (again,
reindexed according to the typing of their input and output
positions).  As this is again part of the data already provided by the
dependent monad $M\da$, we will additionally need to add such an
equivalence to our list of hypotheses.

To recap: assuming the equivalences $e_0$ and $e_1$ amounts to
requiring that the first two stages of the opetopic type $X$ are
equivalent to the indices and constructors of the dependent monad
$M\da$, respectively.  What structure remains? Well, the dependent
constructors of $M\da$ are equipped with the unit and multiplication
operators $\upetad$ and $\upmud$.  But now, recall from Section
\ref{sec:weak-alg} that if the family of relations $X_1$ extends
further in the sequence to a family $X_2$ and we have a proof
$m_2 : \ismult X_2$, then the family $X_1$ can be equipped with a
multiplicative structure given by the functions $\upeta\alg_{m_2}$ and
$\upmu\alg_{m_2}$ defined there.  This is the case in the current
situation, if we assume that the opetopic type $X$ is fibrant (in
fact, we only need assume that $\Rel X$ is fibrant to make this
statement hold).  Therefore, the last piece of information in order
that $X$ ``completely agrees'' with the dependent monad $M\da$ is that
the equivalence $e_1$ is additionally a \emph{homomorphism}, sending
$\upetad$ to $\upeta\alg_{m_2}$ and $\upmud$ to $\upmu\alg_{m_2}$. Our
theorem now is that this data suffices to prove an equivalence of
opetopic types:

\begin{thm}
  \label{thm:slice-unique}
  Suppose $(M , M\da)$ is a monad extension and $X : \OpType M$
  an opetopic type such that $\Rel X$ is fibrant.  Moreover,
  suppose we are given the data of
  \begin{itemize}
  \item An equivalence $e_0 : (i : \Idx M) \to \Idxd M\da\, i \simeq X_0\, i$
  \item An equivalence $e_1 : (i : \Idx M_1) \to \Idxd M\da_1 \simeq_{e_0} X_1$ over $e_0$
  \item Proofs that $s : \upetad M\da \equiv_{e_0 , e_1} \upeta\alg_{m_2}$ and $t : \upmud M\da \equiv_{e_0 , e_1} \upmu\alg_{m_2}$    
  \end{itemize}
  Then there is an equivalence of opetopic types
  \[ X \simeq_{o} \OvrOpType M\, M\da \]
\end{thm}

We have taken some liberties in the presentation of this theorem
(strictly speaking, we have not stated precisely in what sense the
second equivalence $e_1$ is ``over'' the equivalence $e_0$, nor
precisely what equality is implied by symbol the $\equiv_{e_0,e_1}$)
but these omissions can be made perfectly rigorous by standard
techniques, and we feel the statement above conveys the essential
ideas perhaps more clearly than a fully elaborated statement, which
would require a great deal more preparation, not to mention space.
See the appendix of the extended version of this article for a proof
\cite[{Theorem \ref{thm:slice-unique}}]{allioux2021types}.

We at last obtain our desired equivalence:

\begin{thm}
  \label{thm:types-are-oogrps}
  The map
  \[ \Gamma : \UU \to \ooGrp \]
  is an equivalence.
\end{thm}

\begin{proof}
  Given $G : \ooGrp$, we let $A : \UU$ be its type of objects.  We now
  apply Theorem~\ref{thm:slice-unique} with $M = \Id$ and
  $M\da = \Idd A$.  We may take $e_0$ to be the identity.  The
  equivalence $e_1$ is a consequence of \cite[Theorem 5.8.2]{hottbook}
  and the required equalities are a straightforward calculation.
\end{proof}

%%% Local Variables:
%%% mode: latex
%%% TeX-master: "types-are-grpds-ext"
%%% End:

\section{Conclusions and Perspectives}
\label{sec:conclpersp}

We introduce a new local learning algorithm named \landSVM.
It relies on a partitioning of the input space and on a projection of all points onto a set of landmarks.
Using the uniform stability framework, we show that \landSVM has theoretically generalization guarantees.
The empirical evaluation highlights that \landSVM is fast while being competitive with the state of the art.

While we introduced \landSVM with its ``default'' choices, the algorithm offers a lot of exciting perspectives.
First, we can refine many of the elements of \landSVM:
the partitioning using k-means can be replaced by other existing hard or soft clustering algorithms;
the random landmark selection procedure could be improved, for example using methods like DSELECT~\cite{kar2011similarity} and Stochastic Neighbor Compression~\cite{kusner2014stochastic}, or using density estimation~\cite{liu2016stein};
at a greater computational cost, a non-linear kernel can be used to have two levels of non-linearities (see Section~\ref{sec:where:cankerneltrick}).
% TODO: maybe, mention again the faster optimization method from Sec. 2.2
Even if the common landmarks act as a regularization of the local models, an overfitting is observed when the number of clusters becomes high.
The model could naturally accept explicit spatial regularization terms to increase the spatial smoothness of the models across clusters.
% TODO: autocite CVPR?
The speed and linearity of \landSVM also open the door to an auto-context approach (stacking): \landSVM would be reapplied on the data after projecting it on the previous level's support vectors.
Beyond stacking, we plan to explore a deep version of the algorithm, where the intermediate layers of projection are learned in a joint optimization problem.

% for us: full study and theory of the impact of the preprocessing
% for remi: link to deep half random (residual) models


% trigger a \newpage just before the given reference
% number - used to balance the columns on the last page
% adjust value as needed - may need to be readjusted if
% the document is modified later
%\IEEEtriggeratref{8}
% The "triggered" command can be changed if desired:
%\IEEEtriggercmd{\enlargethispage{-5in}}

\bibliographystyle{IEEEtran}
\bibliography{IEEEabrv,../refs}


\end{document}


