%%%% ijcai22.tex

\typeout{IJCAI--22 Instructions for Authors}

% These are the instructions for authors for IJCAI-22.

\documentclass{article}
\pdfpagewidth=8.5in
\pdfpageheight=11in
% The file ijcai22.sty is NOT the same as previous years'
\usepackage{ijcai22}

% Use the postscript times font!
\usepackage{times}
\usepackage{soul}
\usepackage{url}
\usepackage[hidelinks]{hyperref}
\usepackage[utf8]{inputenc}
\usepackage[small]{caption}
\usepackage{graphicx}
\usepackage{amsmath}
\usepackage{amsthm}
\usepackage{booktabs}
\usepackage{algorithm}
\usepackage{algorithmic}
\urlstyle{same}

\usepackage{subfigure}
\usepackage{multirow}
\usepackage{booktabs}
\usepackage{array}


% the following package is optional:
%\usepackage{latexsym}

% See https://www.overleaf.com/learn/latex/theorems_and_proofs
% for a nice explanation of how to define new theorems, but keep
% in mind that the amsthm package is already included in this
% template and that you must *not* alter the styling.
\newtheorem{example}{Example}
\newtheorem{theorem}{Theorem}
\usepackage{cleveref}
% Following comment is from ijcai97-submit.tex:
% The preparation of these files was supported by Schlumberger Palo Alto
% Research, AT\&T Bell Laboratories, and Morgan Kaufmann Publishers.
% Shirley Jowell, of Morgan Kaufmann Publishers, and Peter F.
% Patel-Schneider, of AT\&T Bell Laboratories collaborated on their
% preparation.

% These instructions can be modified and used in other conferences as long
% as credit to the authors and supporting agencies is retained, this notice
% is not changed, and further modification or reuse is not restricted.
% Neither Shirley Jowell nor Peter F. Patel-Schneider can be listed as
% contacts for providing assistance without their prior permission.

% To use for other conferences, change references to files and the
% conference appropriate and use other authors, contacts, publishers, and
% organizations.
% Also change the deadline and address for returning papers and the length and
% page charge instructions.
% Put where the files are available in the appropriate places.

% PDF Info Is REQUIRED.
% Please **do not** include Title and Author information
\pdfinfo{
/TemplateVersion (IJCAI.2022.0)
}


% Multiple author syntax (remove the single-author syntax above and the \iffalse ... \fi here)
% Check the ijcai22-multiauthor.tex file for detailed instructions

\begin{document}

\section{Supplementary Material}
\subsection{Effect of Different Parts of N-best Candidates}
We explore the effect of truncating the $n$-best list and using partial candidates to calculate MBR scores. The results are shown in Fig. \ref{f3}. Sample ratio that is defined as $r$ represents the proportion of candidates used to compute expected utility for each candidate. As $r$ increases, the BLEU scores of the 1-best candidates by both BLEU and COMET ranker go up and then down. The reason may be that partial candidates near the end of the list is extremely close to each other, but of poor quality. When $r$ increases, this part of candidates are more likely to be selected. When $r$ is around 0.7, BLEU scores of ranker BLEU and ranker COMET are close to the optimal.

\begin{figure}[htbp]
	\centering 
    \includegraphics[width=0.5\textwidth]{ratio.png}
	\caption{The quality of the 1-best candidates selected by MBR scores using different parts of candidates. y-axis is the BLEU score. x-axis is the proportion of candidates used to compute MBR scores.}
	\label{f3}
\end{figure}

\subsection{Effect of Different Utility Functions}
In Table \ref{table6}, we also report the results of using BLEU and BLEURT as the utility function to compute MBR scores. MBR\textsubscript{BLEU} ranker achieves a BLEU score of 34.54, which is still a gap from the highest BLEU score of 35.44. There is a gain in BLEU, COMET, and BLEURT scores for MBR\textsubscript{BLEU} $+$ LP or BLEURT $+$ LP ranker regularized by the quality regularizer, which is consistent with the results shown earlier.

\begin{table}[!htbp]
\centering
\begin{tabular}{cccc}
\toprule
Method          & COMET & BLEURT & BLEU  \\ \hline
Top-1 (beam=5)  & 34.22 & 15.99  & 34.17 \\
Top-1 (beam=30) & 34.79 & 16.16  & 34.28 \\
LP+BT           & \textbf{42.63} & 18.57  & 35.11 \\
LP+QE           & 38.84 & 19.53  & 35.37 \\
LP+LM           & 36.33 & 16.58  & 35.14 \\ \hline
MBR\textsubscript{BLEU}            & 34.39 & 16.39  & 34.54 \\
MBR\textsubscript{BLEU}+LP         & 34.75 & 16.64  & 34.56 \\
MBR\textsubscript{BLEU}+LP+BT      & 42.48 & 19.03  & 35.17 \\
MBR\textsubscript{BLEU}+LP+QE      & 38.68 & 19.75  & \textbf{35.44} \\
MBR\textsubscript{BLEU}+LP+LM      & 38.89 & 19.91  & 35.41 \\ \hline
MBR\textsubscript{BLEURT}          & 33.10 & \textbf{22.00}  & 33.01 \\
MBR\textsubscript{BLEURT}+LP       & 35.83 & 19.86  & 34.55 \\
MBR\textsubscript{BLEURT}+LP+BT    & 42.46 & 19.20  & 35.18 \\
MBR\textsubscript{BLEURT}+LP+QE    & 38.91 & 20.19  & 35.42 \\
MBR\textsubscript{BLEURT}+LP+LM    & 36.79 & 18.04  & 35.25 \\ \bottomrule
\end{tabular}
\caption{BLEU, COMET, and BLEURT score comparison.}
\label{table6}
\end{table}

\subsection{Results in Different Beam Sizes}
As shown in Table \ref{table7}.  There is a significant improvement for COMET scores and BLEURT scores, and a overall increasing trend for BLEU scores with increased beam sizes, which is reported in Fig. . The results suggest that our proposed re-ranking method can generate better translations as beam size increases.  There is a significant improvement for COMET scores and BLEURT scores, and a overall increasing trend for BLEU scores with increased beam sizes, which is reported in . The results suggest that our proposed re-ranking method can generate better translations as beam size increases. There is a significant improvement for COMET scores and BLEURT scores, and a overall increasing trend for BLEU scores with increased beam sizes, which is reported in Fig. . The results suggest that our proposed re-ranking method can generate better translations as beam size increases.

\begin{table}[!htbp]
\centering
\begin{tabular}{cccc}
\toprule
Method         & COMET & BLEURT & BLEU  \\ \hline
\multicolumn{4}{c}{beam=50}                \\ \hline
BT+LP          & 43.33 & 18.74  & 35.19 \\
QE+LP          & 39.12 & 19.64  & 35.35 \\
LM +LP         & 36.58 & 16.67  & 35.16 \\
MBR\textsubscript{COMET}          & 43.50 & 18.27  & 34.57 \\
MBR\textsubscript{COMET}+LP       & 42.35 & 18.11  & 34.94 \\
MBR\textsubscript{COMET}+LP+BT    & \textbf{44.42} & 18.97  & 35.31 \\
MBR\textsubscript{COMET}+LP+LM    & 42.87 & 18.96  & 35.58 \\
MBR\textsubscript{COMET}+LP+QE    & 42.74 & \textbf{20.26}  & \textbf{35.62} \\ \hline
\multicolumn{4}{c}{beam=5}                 \\ \hline
BT+LP          & 38.72 & 17.26  & 34.61 \\
QE+LP          & 36.85 & 18.26  & 34.77 \\
LM +LP         & 35.13 & 16.28  & 34.68 \\
MBR\textsubscript{COMET}          & 36.65 & 16.03  & 34.19 \\
MBR\textsubscript{COMET}+LP       & 36.44 & 16.47  & 34.40 \\
MBR\textsubscript{COMET}+LP+BT    & 38.99 & 17.38  & 34.69 \\
MBR\textsubscript{COMET}+LP+LM    & 36.73 & 16.81  & 34.78 \\
MBR\textsubscript{COMET}+LP+QE    & 38.09 & 18.20  & 34.86 \\ \hline
Top-1 (beam=5) & 34.22 & 15.99  & 34.17 \\  \bottomrule
\end{tabular}
\caption{BLEU, COMET, and BLEURT score comparison.}
\label{table7}
\end{table}
 
%% The file named.bst is a bibliography style file for BibTeX 0.9



\end{document}

