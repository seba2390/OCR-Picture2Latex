\documentclass[main_tikz.tex]{subfiles}

\begin{document}
\begin{tikzpicture}[thick, >=triangle 45, scale=0.8]

    \def\learncolor{black!7}
    \tikzset{%
        learnable/.style = {
            fill=\learncolor,
        },
        parameter/.style = {
            draw, rectangle,
            align=center,
           rounded corners,
           learnable
        },
        layer/.style = {
            parameter,
            % minimum height = 2.5em,
            % minimum width = 1.5cm,
        },
        mu/.style = {
            draw, rectangle,
            align=center,
           fill=white,
           rounded corners,
           learnable
        },
        sum/.style = {
            draw, circle,
            inner sep=0pt
        },
        % Define arrow style
        pil/.style={
               ->,
               thick,
              % shorten <=2pt,
               %shorten >=2pt,}
        }
    }

    \def\interlayer{.8em}
    \def\intermask{2em}

    % Define the nodes
    \node(X_obs){$x \odot \bar m$};
    \node[right=1em of X_obs, sum] (add0) {\Large$-$};
    \node[above=2.5em of add0, mu] (mu_obs){$\mu \odot \bar m$};
    \node[right=1.5em of add0, layer] (layer0) {$W^{(0)}$};
    \node[right=\intermask of layer0, sum] (add1) {\Large$+$};
    \node[right=\interlayer of add1, layer] (layer1) {$W^{(0)}$};
    \node[right=\intermask of layer1, sum] (add2) {\Large$+$};
    \node[right=\interlayer of add2, layer] (layer2) {$W^{(0)}$};
    \node[right=\intermask of layer2, sum] (add3) {\Large$+$};
    \node[right=\interlayer of add3, layer] (layer3) {MLP};
    \node[right=1.5em of layer3] (Y) {$Y$};
    
    % Draw the connections
    \draw[pil] (X_obs) -- (add0) -- (layer0) coordinate[midway] (rect1) 
                  -- (add1) coordinate[pos=.3] (mask0) -- (layer1)
                  -- (add2) coordinate[pos=.3] (mask1) -- (layer2)
                  -- (add3) coordinate[pos=.3] (mask2) -- (layer3)
                  -- (Y);
    \draw (mu_obs) -- (add0);
    
    
    % Mock nodes to draw residual connections
    \coordinate[right=1em of add0] (mock){};
    \coordinate[below=2em of mock] (start){};
    \draw (mock) |- (start);
    \foreach \i in {1, 2, 3}{
        \draw (start) -| (add\i) coordinate[midway] (start) {};
    
    }
    
    % then draw the masks
    \def\height{2em}
    \def\width{.5em}
    \def\nrect{10}
    \def\mask{0, 1, 2}
    \foreach \k in \mask{
    
        \def\blackidx{0, 2, 5, 6, 9}
        \def\masklabel{$\odot \bar m$}

        \draw (mask\k.west)
            -- ++(0, \height)
            -- ++(\width, 0) node[midway, above] {\masklabel}
            -- ++(0, -2*\height)
            -- ++(-\width, 0) node (anchor\k) {}
            -- cycle;
    
        \foreach \i in {1,...,\nrect}{
            \draw ($ (anchor\k.center) + (0, 2*\i*\height/\nrect)$)
                -- ++(\width, 0);
        }
        
        \foreach \i in \blackidx{
            \fill[black] ($ (anchor\k.center) + (0, 2*\i*\height /\nrect)$)
                -- ++(0, 2*\height/\nrect)
                -- ++(\width, 0)
                -- ++(0, -2*\height/\nrect)
                -- cycle;
        
        }
    }
    
    % Mock nodes to draw the box
    % \coordinate[right=.5em of mock.center] (rect1){};
    \coordinate[right=.2em of add3] (rect2){};
    \draw[dashed]
        (rect1) |-($(rect1) +(3em, 3.7em)$) -- ($(rect2) + (-3em, 3.7em)$)
        -| (rect2) |-($(rect2) -(4em, 4em)$) -- ($(rect1) - (-4em, 4em)$)
            node[below, pos=.8] (desc) {Neumann block}
        -| cycle;
        
    % Arrows non-linearity
    \node[right=12em of desc, orange] (labelNL) {Non-linearity};
    
    \foreach \k in \mask{
        \draw[orange,->] (labelNL.west) to[bend left=10] (anchor\k);
    }


\end{tikzpicture}
\end{document}