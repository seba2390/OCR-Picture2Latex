\documentclass[]{article}
\usepackage[english]{babel}
\usepackage[utf8]{inputenc}
\usepackage{amsmath}
\usepackage{graphicx}
\usepackage[colorinlistoftodos]{todonotes}
\usepackage{authblk}
\usepackage{hyperref}
\usepackage{ulem}
\usepackage[margin=1in]{geometry}
\newcommand{\gn}{\bf \color{blue!50}} %Comments and tracked changes by Gelu Nita
\newcommand{\SC}{\bf \color{red}}
\clubpenalty=10000
\widowpenalty=10000 % avoids having a single paragraph line on the next page (i.e. transfers min 2 lines)
\def\apj{The Astrophysical Journal}
\def\apjl{The Astrophysical Journal Letters}
\def\apjs{The Astrophysical Journal Supplement}
\def\aap{Astronomy and Astrophysics}
\def\aapr{Astronomy and Astrophysics Reviews}
\def\solphys{Solar Physics}
\def\mnras{Monthly Notices of the Royal Astronomical Society}
\def\ssr{Space Science Reviews}
\def\apss{Astrophysics and Space Science}
\def\nat{Nature}
\def\araa{Annual Review of Astron and Astrophys}
\def\prl{Physical Review Letters}

\usepackage{natbib}
\bibliographystyle{abbrvnat}
\setcitestyle{authoryear,open={(},close={)}}

\title{Challenges and Advances in Modeling of the Solar Atmosphere: \\A White Paper of Findings and Recommendations }

\author[1]{Serena Criscuoli}
\author[1,2]{Maria Kazachenko}
\author[3]{Irina Kitiashvili}
\author[3,4]{Alexander Kosovichev}
\author[1]{Valent\'in Mart\'inez Pillet}
\author[4]{Gelu Nita}
\author[5,6]{Viacheslav Sadykov}
\author[3]{Alan Wray}


%\affil[0]{TBD}
\affil[1]{National Solar Observatory}
\affil[2]{University of Colorado, Boulder}
\affil[3]{NASA Ames Research Center}
\affil[4]{New Jersey Institute of Technology}
\affil[5]{Bay Area Environmental Research Institute}
\affil[6]{Georgia State University}




%\date{\today}
\date{December 30, 2020}
\pagestyle{myheadings}
\markboth{Criscuoli et al.}{White Paper: Modeling of the Solar Atmosphere}
\begin{document}
\maketitle

%\todo[inline, color=blue!40, shadow]{@HDMIEC-CVMO WHITE PAPER SKELETON}

\begin{abstract}
The next decade will be an exciting period for solar astrophysics, as new ground- and space-based instrumentation will provide unprecedented observations of the solar atmosphere and heliosphere. The synergy between modeling effort and comprehensive analysis of observations is crucial for the understanding of the physical processes behind the observed phenomena. However, the unprecedented wealth of data on one hand, and the complexity of the physical phenomena on the other, require the development of new approaches in both data analysis and numerical modeling. In this white paper, we summarize recent numerical achievements to reproduce structure, dynamics, and observed phenomena from the photosphere to the low corona and outline challenges we expect to face for the interpretation of future observations. 

%However, as we approach smaller scales in models and observations, challenges emerge that require the development of new approaches in both data analysis and numerical modeling. 
\end{abstract}

\section{A forthcoming exciting Multi-Messenger era}
Investigations of physical processes occurring on the Sun, and their effects on the Earth, its atmosphere, and its space environment, exploit a large variety of messengers to provide the needed information. “Classical” photon-based astronomical observations create data such as disk-integrated spectra and full-disk and high spatial resolution images in wavelengths from radio to X-rays, as well as helioseismic imaging of the solar interior. The nearness of the Earth to the Sun allows measurements difficult or impossible for other stars,  such as in-situ measurements of solar wind properties (e.g., chemical composition, magnetic field, particle energy distributions, etc.), detection of neutrinos emitted in the solar core, and geomagnetic indices (which include direct measurements of properties of the magnetic field as well as indirect indices such as cosmogenic isotopes).  Each of these contributes to providing unique information about how energy is generated in the Sun, transported into the atmosphere and heliosphere, and  dissipated through a variety of phenomena on temporal scales of seconds to millennia.  

The next decade will be a new exciting period as new ground- and space-based instrumentation will provide unprecedented observations of the solar atmosphere and heliosphere. In particular, in 2021 the Daniel K. Inouye Solar Telescope {\citep[DKIST, ][]{rimmele2020}} will start to produce observations of the photosphere and chromosphere at extremely high spatial and temporal resolutions; these will be accompanied by spectral and spectropolarimetric observations of the corona at spectral ranges never or rarely observed before. The Parker Solar Probe (PSP, launched in August 2018) provides in-situ measurements of solar wind properties at unprecedented proximity to the Sun \citep{Fox2016}. The Solar Orbiter mission (launched in 2020) combines both in-situ and remote-sensing instrumentation and gives us the opportunity to expand our understanding about solar interior dynamics, the evolution and structure of polar magnetic fields, and the fast solar wind \citep{Muller2020}. Base knowledge from currently available observations from space missions (e.g., SDO, IRIS, HINODE, STEREO, THEMIS) and ground-based observatories (e.g., SOLIS, GST, CoMP, GONG, ALMA, EOVSA), accompanied by upcoming observations, will take us to a new level of understanding of solar dynamics and activity and their impact on the Heliospheric and Earth environments.

The synergy between an advanced modeling effort and a comprehensive analysis of observations has great potential to shed light on the physical processes behind the observed phenomena. However, as we approach smaller scales in models and observations, challenges emerge that require the development of new approaches in data analysis and numerical modeling. 
%In this white paper, we summarize recent numerical achievements to reproduce structure, dynamics, and observed phenomena from the photosphere to the low corona and outline challenges we expect to face for the interpretation of future observations.

\section{Modeling of Solar Atmosphere Dynamics}
% {\SC \sout{Dramatically growing computational capabilities, supported by substantial developments of theoretical and numerical approaches, and observing technologies, drive our movement from a simplistic 1D vision of the interior and atmosphere structure to consideration of the multidimensional nature of the observed phenomena and energy transport between different scales.}} 

Substantial development of theoretical and numerical approaches, supported by continually growing computational capabilities and novel observing technologies, are changing our view of solar phenomena. Thanks to such developments, we have been slowly moving away from a simplistic, one-dimensional, static interpretation of solar observations toward models that take into account the multidimensional nature of the observed phenomena as well as the energy transport mechanisms occurring between different spatial and temporal scales. However, the complexity of the multi-scale dynamics of turbulent magnetized plasma in highly stratified inhomogeneous conditions prevents us from creating a single universal model to reproduce the solar dynamics. Therefore, despite the joint goal, modeling approaches are often targeted to a specific class of problems. For instance, realistic-type simulations are focused on capturing a wide range of the physical phenomena in the solar plasma (e.g., turbulence, radiation, ionization, magnetic fields, abundances) to reproduce observed processes and phenomena.
On the other hand, data-driven models attempt to reproduce a particular event or events by forcing a model solution toward specific observations to create an accurate physical interpretation of the observed dynamics.
%In the next subsections, we briefly overview some achievements in both approaches and discuss challenges that necessary to address.

Magnetohydrodynamic (MHD) simulations in the Sun-in-a-box regime aim to solve the equations of conservation  of mass, momentum, energy, and magnetic flux in a highly stratified compressible medium, utilizing 3D multi-group radiative energy transfer between the fluid elements, a real-gas equation of state, ionization and excitation of all abundant species, and magnetic effects. Because of their computational cost, the resulting models cover only a small area on the Sun with a spatial resolution of tens of kilometers or higher. Nevertheless, this approach demonstrates its capability to reproduce solar observations. In particular, `ab-initio' modeling of the solar magnetoconvection and atmosphere pioneered by \cite{Nordlund1989} demonstrated the importance of the effects of 3D turbulent dynamics in reproducing spectroscopic observations \citep{Asplund2009}.  Since then, such models have allowed us to make progress in understanding the near surface structuring and dynamics of solar plasma and magnetic fields. In particular,  they have provided insight about granulation \citep[e.g.,][]{Stein2001}, supersonic horizontal flows \citep[e.g.,][]{Vitas2011}, fine structuring of sunspot umbrae \citep[e.g.,][]{Schuessler2006} and penumbrae \citep[e.g.,][]{Kitiashvili2009,Rempel2009,Rempel2009a,SiuTapia2018}, convective collapse \citep[e.g.,][]{Skartlien2000,Hewitt2014}, explosive events \citep[e.g.,][]{Kitiashvili2013,Danilovic2016,Chatterjee2016}, small-scale dynamos and self-formed magnetic structures \citep[e.g.,][]{Vogler2007,Kitiashvili2010,Kitiashvili2015,Stein2012,Rempel2014}, the formation of magnetic structures from predefined conditions such as twisted flux ropes or specific magnetic field topologies \citep[e.g.,][]{Rempel2009,Cheung2010,Fang2015} and others.

Thanks to intense cross-validation of the numerical models with observations, which has allowed achieving a high-degree realism of the simulated solar plasma, these 3D radiative models have become capable of providing critical support in the interpretation of spectroscopic and spectropolarimetric observations from the ground and in space. Figure~\ref{fig:stokesV} shows an example of continuum intensity synthesized from an MHD model \citep{Rempel2014} obtained with the MuRAM code \citep{Vogler2005} and a comparison of the Stokes V profiles obtained for the numerical resolution and resolutions corresponding to 4-m DKIST and 1.5-m GREGOR observations. Discrepancies in resulting line profiles illustrate how the spatial resolution can affect the accuracy of magnetic field measurements. Analysis of such discrepancies helps us to estimate uncertainties in the inferred physical and dynamical properties of the plasma. Indeed, the leveraging of cross-comparison between observations and simulations is crucial for several scientific issues discussed in the DKIST Critical Science Plan \citep{Rast2020}.

\begin{figure}[t] %[htbp]
    \centering
    \includegraphics[scale=0.4]{fig_stokesVa.jpg}
    \caption{Synthetic continuum image obtained from a MuRAM MHD model \citep{Rempel2014} shows the the granular structure and magnetized bright structuring in the intergranular lanes. A probe of Stokes V spectra (Fe I 630~nm) illustrates the dependence of the line profiles on the spatial resolution:  full numerical resolution(8~km, solid curve), corresponding resolutions of DKIST (40~km, dashed curve) and the GREGOR telescope (80~km, dashed-dotted curve).
}
    \label{fig:stokesV}
\end{figure}
%mkd add data-driven parts - 25 aug 2020
Coronal magnetic fields play a key role in determining when and where solar flares and coronal mass ejections (CMEs) occur \citep{Gibson2020}. Remote sensing of the coronal magnetic fields has not been possible in the past, thus only reconstructions based on magnetic field measurements at the photosphere/chromosphere level could be employed to model the 3D structure of the coronal magnetic field. However, newly developed observational techniques, such as visible/infrared (VIR) coronal seismology \citep{Yang2020a, Yang2020b} and microwave interferometry \citep{Fleishman2020, Chen2020}, have demonstrated the feasibility of off-limb and on-limb observations to be routinely performed in the near future using a new generation of already developed instruments, such as DKIST and The Expanded Owens Valley Solar Array \citep[EOVSA, ][]{EOVSA}, and proposed  instruments, such as the full-Sun Coronal Solar Magnetism Observatory \citep[COSMO, ][]{Tomczyk} and the Frequency Agile Solar Radiotelescope \citep[FASR, ][]{FASR}

Nevertheless, given the remote sensing nature of these newly available observations, they are generally limited by a line-of-sight ambiguity, which can only be resolved by combining them with sophisticated, yet feasible, multi-wavelength forward-fitting techniques and data-constrained modeling frameworks, such those offered by community-developed and maintained modeling tools like FORWARD \citep{Gibson2016} and GX Simulator \citep{Nita2015, Nita2018}.

Modeling of the magnetic field in the corona is essential for the study of the origin of  the solar active phenomena. There are primarily two groups of models to simulate evolving coronal magnetic fields in 3D: quasi-static (or time-independent) and dynamic (time-dependent). In the quasi-static group, potential, linear, and non-linear force-free field (NLFFF) extrapolations have been developed. These models apply the vacuum-limit assumption, which reasonably assumes that magnetic pressure dominates the gas pressure (low-beta regime). Although this modeling framework assumes that the corona is in a static force balance that is not affected by states at earlier times,  \citet{Fleishman2018} have shown that evolving sequences of such modeling snapshots may be used to infer the evolution of eruptive phenomena.

Another, more physics-based group of models are time-dependent models. In this group,  ``data-driven'' (DD) simulations, i.e., simulations where successive photospheric magnetic field observations are used to evolve the model's photospheric field, show great promise for investigating the 3D structure of the coronal magnetic field. Lately, two types of DD models have been developed: magnetofrictional \citep[e.g.][]{Cheung2012} and MHD models \citep[e.g.][]{Hayashi2018}. The magnetofrictional model assumes that the plasma velocity in the MHD induction equation is proportional to the local Lorentz force, leading to a relaxation of a magnetic configuration toward a force-free state. It is more computationally efficient than MHD and is suitable for the description of the slow quiescent evolution of active regions, but not for flares. The MHD models explicitly solve a full set of MHD equations including the plasma properties. The MHD approach is suitable for rapid evolution during flares but is too computationally expensive to model the long-term quiescent evolution of active regions. A hybrid framework, where the MF model is used to model quiescent periods of AR evolution and the MHD model is used to model flaring periods of AR evolution, has been recently developed within the Coronal Global Evolutionary model \citep{Hoeksema2020}.

  The major difference between data-driven and data-inspired models are the observations that the data-driven models make use of as lower boundary conditions. These are typically some combination of magnetic, velocity, and electric fields. Recently several approaches have been developed to derive electric fields using observations of the vector magnetic fields and Doppler velocities in the photosphere \citep[see e.g.][]{Fisher2020}.

%Comparison of dynamics and restructuring in the solar atmosphere {\SC is} often qualitative. Understanding why observed processes occur under certain conditions can be {\SC \sout{done} achieved} by forcing a model with observations. The data-driven approach allows to adjust {\SC \sout{of}} properties and structure of the atmosphere closer (in the topological sense) to {\SC the} observed plasma.
%Therefore, to better understand the dynamics of solar corona, one needs to have three-dimensional description of the magnetic fields in the corona. In general, however, measurements of coronal magnetic fields above the photosphere are limited, hence we have to rely on modeling approaches and measurements of the magnetic fields in the photosphere to determine the three-dimensional structure of the coronal magnetic field. 

The current effort to capture solar dynamics from the convection zone to the corona faces the challenge of numerically describing atmospheric layers with dramatic variations of plasma conditions. 
%Lack of understanding of dynamics and structure of the chromosphere and the transition zone is due to the complexity of {\SC multiscale interaction of the plasma with the magnetic fields on one hand,  and on the other to difficulties inherent to deriving observational constraints. 
%\sout{In particular, the chromospheric diagnostic is only possible to perform by observing and interpreting various lines (e.g., H-alpha, UV, IR Ca\,II\,lines, Mg\,II\,h\&k, He), for which a detailed synthesis in the NLTE approximation is crucial \citep[e.g.][]{Leenaarts2013a, Stepan2015}}
In addition, a Non-Local Thermodynamic Equilibrium (NLTE) approach is necessary on one hand to interpret chromospheric and transition region diagnostics 
 (e.g., H-alpha, UV, IR Ca\,II\,lines, Mg\,II\,h\&k, He\,I, Ly-alpha)  \citep[e.g.][]{Leenaarts2013a, Stepan2015} and therefore derive reliable observational constraints, and, on the other hand, to properly describe the interaction of radiation and matter in the higher layers of the atmosphere.
%{\SC The inclusion into the BIFROST code, even in a simplified manner,  of NLTE processes in the energy and particle balance} \citep{Carlsson2016,Hansteen2017} allows us to obtain one of the most advanced models of the solar chromosphere. 
 One of the most advanced codes in this respect is \textit {Bifrost} \citep{Gudiksen2011}, which,  although in a simplified manner,  includes NLTE processes in the energy and particle balance \citep{Carlsson2016,Hansteen2017}. Significant improvements have been further achieved by including additional effects, such as the Hall effect and ambipolar diffusion \citep[e.g.,][]{Martinez-Sykora2012, Khomenko2020}, and the development of approximations to describe chromospheric radiative cooling and heating \citep{Carlsson2012}. 


However, the computational cost of NLTE calculations poses significant limitations on the size of the computational domain and spatial resolution.
Therefore, modeling of the solar dynamics in broad regimes, from the interior to the corona, necessarily relies on the LTE approximation.  \citep[e.g.,][]{Rempel2017,Wray2018}. 
 In spite of this simplification, various codes have been used to simulate the domain from the photosphere to the corona, allowing the study of phenomena naturally driven by near-surface plasma dynamics, and addressing issues as coronal heating, the onset of flares, and others  \citep[e.g.,][]{Gudiksen2005,Carlsson2012,Rempel2017,Cheung2019}. These models provide theoretical support for the interpretation of observations. An example of synthetic SDO/AIA observations of the limb view of the modeled solar corona~\citep{Kitiashvili2020} is presented in Figure~\ref{fig:aiacorona}. %One can notice sporadic heating events in the solar transition region (visible as brightening in the SDO/AIA $304$\AA{} line at the bottom of the transition region) and a self-organized magnetic structure in the center of the region (evident as a darkening of SDO/AIA 171\,\AA emission).

\begin{figure}[t]
    \centering
    \includegraphics[width=0.75\linewidth]{WP_AIAsideview_t12072.png}
    \caption{Composite illustration of the Solar Dynamics Observatory Atmospheric Imaging Assembly (SDO/AIA) synthetic emission for a side view of a quiet Sun region, simulated with the StellarBox code \citep{Wray2018}. Red color corresponds to the synthetic SDO/AIA 171\,\AA{} emission (T$=7.9\times{}10^{5}$\,K), green corresponds to 131\,\AA{} (T$=5.6\times{}10^{5}$\,K), blue corresponds to 304\,\AA{} (T$=7.9\times{}10^{4}$\,K). The domain has a horizontal size of 12.8\,Mm and includes 10\,Mm of the solar atmosphere from the surface to corona.}
    \label{fig:aiacorona}
\end{figure}

 
 As alluded to above, the use of state-of-the-art atmosphere models to interpret solar observations is further complicated by difficulties inherent to the synthesis of specific observables.  %interpretation  The interpretation of observational data is a challenging task due to the non-linear processes occurring in highly dynamical magnetized plasma and is directly linked to knowledge gain of the nature of solar activity. 
%Several radiative transfer codes have been developed to estimate physical conditions from observations and obtain synthetic observables from model atmospheres. Each of the available codes has certain advantages and restrictions. %In particular, the Spinor code \citep{Frutiger2000} allows computing full Stokes profiles relatively fast. However, computations are restricted by Local Thermodynamic Equilibrium (LTE) approximation.
Several radiative transfer codes have been developed for this purpose, each with certain advantages and restrictions. For instance, the Spinor \citep{Frutiger2000} code is relatively fast, but computations are restricted by the Local Thermodynamic Equilibrium (LTE) approximation. The NICOLE code \citep{SocasNavarro2015} enables calculations in both LTE and NLTE, but the implementation of the NLTE approximation does not include partial frequency redistribution (PRD) effects, which are important for the modeling of chromospheric lines, such as the Ca II H \& K, and Mg II UV lines. The RH code \citep{Pereira2015} is also able to perform calculations in both approximations with a primary focus on the accurate implementation of NLTE effects in different types of geometries. The Multi3D code \citep{Leenaarts2009} is developed for massive calculations from 3D models of the solar atmosphere of a wide range of spectral lines using the NLTE approximation. However, the inclusion of detailed multi-dimensional radiative transfer effects comes at the cost of the increased computational time.  Moreover, the description of these physical effects is numerically challenging, especially in the chromosphere where large spatial and temporal gradients often make the codes unstable. Ultimately, the choice of the particular radiative transfer code depends on the task and the related approximations most suitable for the physical processes one wants to study. 


The interpretation of observational results often relies on solving the inversion problem, i.e., on modeling the physical characteristics of the atmosphere which produced the observed emission and spectra. 
Different codes are available to perform spectral ad spectropolarimetric inversions of observed spectra, but, similarly to the codes available for forward modeling, the vast majority are targeted at the interpretation of specific types of observations. For instance, codes developed for the interpretation of lines forming in the photosphere and chromosphere are often developed by integrating the radiative transfer codes described in the previous section with minimization algorithms that optimize the physical and magnetic stratification of the atmosphere (examples include the widely used NICOLE and SPINOR and the newly developed STiC and DeSIRE, based on the RH code). As such, these inversion codes suffer of the same limitations described above. Other codes have been or are under development to interpret specific observations, as for instance observations in the He I  1083.0 nm line (Hazel), limb observations in chromospheric lines (NICOLE), and flares (STiC). Interpretation of coronal observations is instead performed by using very different approaches, based on differential emission measures of the coronal plasma from the related EUV observations \citep[e.g.,][]{Cheung2015}. One of the major drawbacks of most of inversion techniques is their extremely high computational cost.  
One solution is to represent the highly-nonlinear mapping of the observed emission and line profiles to the structure of the atmosphere by a neural network trained on the set of physics-based inversions. Such an approach has been successfully applied for fast inversion of IRIS data \citep{SainzDalda2019} and SDO/AIA EUV emission \citep{Wright2019}. Although uncertainty quantification for such machine-learning-based models is challenging and still open, this is currently the only way to provide timely data products under one pipeline with the observational data. The development of such models and understanding and quantifying the related uncertainties should be encouraged.


\section{Access to computational resources and model data sharing}
 High spatial and temporal resolutions, augmented by the spectropolarimetric dimensions, dramatically increase the data-volume of both observational and synthetic data. Such massive data flows significantly limit the dissemination of these data among the community. 

On the one hand, sharing of the modeling and synthetic-observables data sets requires dramatic storage resources. On the other hand, taking these data and analyzing them on a local workstation is often impossible. As a result, the modeling data are highly truncated (e.g., include only selected layers, model parameters, and/or limited time-series). 
The problem is becoming more severe with time to the point that web links to the data are sometimes failing. Therefore, the development of a data-sharing platform and data-request system will benefit both potential customers and data providers. 

 Of particular importance is the collection of synthetic observables obtained with different radiative transfer codes and MHD models. 
% {\SC \sout{The database of available synthetic observables will support the interpretation effort of observations, instrumentation, and data analysis development, and will also improve understanding of the limitations of models and our understanding of solar activity.} 
 A database of synthetic observables, obtained by combining state-of-the-art model atmospheres and radiative transfer computations, will support the scientific community in several crucial ways, such as the interpretation of observations, the development and validation of new data analysis techniques, and the validation of new instrumentation and data reduction pipelines. Furthermore, a wider dissemination of synthetic data will improve our understanding of model limitations, thus fostering the development of new numerical techniques and a better description of the physical processes that drive the evolution of the solar plasma and magnetic fields.

\section{Recommendations}

 The development of numerical models and tools is essential for a robust physical interpretation of the observational data acquired by the DKIST and other ground and space missions that have recently come online. The analysis of spectral and spectropolarimetric observations acquired at high spatial and temporal resolution and at high spectropolarimetric sensitivity in different spectral ranges, requires the development of models capable of describing the non-linear connection between the local physical and dynamical properties of plasma in a radiating, turbulent, and magnetic environment. Despite significant advances in modeling solar dynamics, the cross-validation of models and observations is essential to detect any missed components of the observed phenomena.

Thus it is crucial to support the multi-level integration of modeling efforts and observations. In particular, we identify the following critical needs:
\begin{itemize}
    \item Improve the development of high-fidelity 3D MHD codes by 1) removing restrictions on code development commitments in proposal AOs, and 2) establishing new funding opportunities for the development of codes and data analysis approaches, including machine learning, data assimilation, and others.
    \item Encourage cross-comparison of observations and simulations and support the development of new tools for efficient analysis of big multi-dimensional data sets.
    \item Increase investment in high-end-computing (HEC) facilities in terms of both computing nodes and storage. The storage and computational requirements are amplified by including more physical effects into the models and for the post-processing of the resulting data. Improvement in HEC capabilities will increase workflow performance and will significantly speed up development of predictive capabilities for both global solar variability and short-term, high-energy release activity such as flares, coronal mass ejections, etc. Similarly, the management and analysis of large observational datasets (for instance through spectropolarimetric inversions) requires increasing computational and storage resources.
    \item Allow trial allocations to demonstrate the feasibility of proposed projects and to speed up the boarding process when the first full allocation is awarded. Trial allocations will provide quick access to computational resources when needed and will help foster a new generation of scientists and developers.
    \item Establish a NASA-NSF HEC collaboration to enable efficient workflow. Code performance can depend on supercomputer architecture, infrastructure, and available resources. Consequently, rigid funding linked to a specific supercomputer system can impede project success.
    \item Support the development of open access policies for models and synthetic data, in a manner similar to policies implemented for observational data acquired by NASA and NSF facilities.
    \item Support the development of platforms for requesting synthetic data targeted at the interpretation of specific observations. 
    \item Support the development of platforms for the distribution of models and synthetic data to leverage existing capabilities and efforts, to improve collaboration efficiency, and to stimulate interdisciplinary integration.

\end{itemize}

\section*{Acknowledgments}
This material is based upon work supported by the National Science Foundation under Grant AGS-1743321. Any opinions, findings, and conclusions or recommendations expressed in this material are those of the authors and do not necessarily reflect the views of the National Science Foundation.The National Solar Observatory is operated by the Association of Universities for Research in Astronomy, Inc. (AURA)
under cooperative agreement with the National Science Foundation. 
\bibliography{ref.bib}


\end{document}

\section{Manual References}
\todo[inline]{Kept this list here to be compared with the bibtex references}

Bastian, T. S.; Chintzoglou, G.; De Pontieu, B.; Shimojo, M.; Schmit, D.; Leenaarts, J.; Loukitcheva, M., 2017. “A First Comparison of Millimeter Continuum and Mg II Ultraviolet Line Emission from the Solar Chromosphere”, ApJ, 845, 19.

Bjørgen, J. P.; Leenaarts, J.; Rempel, M.; Cheung, M. C. M.; Danilovic, S.; de la Cruz Rodríguez, J.; Sukhorukov, A. V., 2019. “Three-dimensional modeling of chromospheric spectral lines in a simulated active region”. A\&A, 631, 33.

Buehler, D.;Lagg, A.; Solanki, S.K., 2013. “Quiet Sun magnetic fields observed by Hinode: Support for a local dynamo”. A\&A, 555A, 33B.

Carlsson, M., \& Stein, R. F. 2002. “Dynamic Hydrogen Ionization”, ApJ, 572, 626C

Cheung, M.C.M., Rempel, M., Chintzoglou, G. et al. 2019. “A comprehensive three-dimensional radiative magnetohydrodynamic simulation of a solar flare”, Nature Astronomy, 3, 160

Cheung, M.C.M., Boerner, P., Schrijver, C. et al. 2015. “Thermal Diagnostics with the Atmospheric Imaging Assembly on board the Solar Dynamics Observatory: A Validated Method for Differential Emission Measure Inversions”. ApJ, 807, 143C

de la Cruz Rodríguez, J., Leenaarts, J., Danolivic, S., Uitenbroek, H. 2019. “STiC: A multiatom non-LTE PRD inversion code for full-Stokes solar observations”, A\&A, 623, 74

Danilovic, S.; Rempel, M.; van Noort, M.; Cameron, R., 2016. “Observed and simulated power spectra of kinetic and magnetic energy retrieved with 2D inversions”. A\&A, 593A, 93D.

Gudiksen, B. V., Carlsson, M., Hansteen, V. H., et al. 2011. “The stellar atmosphere simulation code Bifrost. Code description and validation”, A\&A, 531, 154

Hudson, H. \& Svalgaard, Leif, 2019. “Commentary: multimessenger solar astrophysics”. Physics Today, vol. 72, issue 2, pp. 10-12.

Kitiashvili, I. N.; Kosovichev, A. G.; Mansour, N. N.; Wray, A. A, 2015. “Realistic Modeling of Local Dynamo Processes on the Sun”. ApJ, 809, 84.

Lamb, D. A.; Howard, T. A.; DeForest, C. E., 2014. “Spatial Nonlocality of the Small-scale Solar Dynamo.”. ApJ, 788, 7.

Leenaarts, J., Carlsson, M., Rouppe van der Voort, L. 2012. “The Formation of the H$\alpha$ Line in the Solar Chromosphere”, ApJ, 749, 136L
Leenaarts, J., Pereira, T. M. D., Carlsson, M., Uitenbroek, H., \& De Pontieu, B. 2013. “The Formation of IRIS Diagnostics. II. The Formation of the Mg II h\&k Lines in the Solar Atmosphere”, ApJ, 772, 90L
Leenaarts, J., Carlsson, M., Hansteen, V., Rouppe van der Voort, L. 2009. “Three-Dimensional Non-LTE Radiative Transfer Computation of the CA 8542 Infrared Line From a Radiation-MHD Simulation”. ApJL, 694, L128-131.

Leenaarts, J., Carlsson, M., Hansteen, V., \& Rutten, R. J. 2007. “Non-equilibrium hydrogen ionization in 2D simulations of the solar atmosphere”. A\&A, 473, 625

Martínez González, M. J.; Collados, M.; Ruiz Cobo, B., 2006. “On the validity of the 630 nm Fe I lines for magnetometry of the internetwork quiet Sun”. A\&A, 593, A93.

Munafò, A., Mansour, N., Panesi, M. 2017. “A Reduced-order NLTE Kinetic Model for Radiating Plasmas of Outer Envelopes of Stellar Atmospheres”, ApJ, 838, 126M.

Pereira, Tiago M. D., 2019.” The dynamic chromosphere: Pushing the boundaries of observations and models”, AdSpR, 63.1434P.

Pereira, T.M.D. and Uitenbroek, H. 2015. “RH 1.5D: a massively parallel code for multi-level radiative transfer with partial frequency redistribution and Zeeman polarisation”, A\&A, 574, 3.

Rathore, B., Carlsson, M., Leenaarts, J. and De Pontieu, B. 2015. “The Formation of IRIS Diagnostics. VI. The Diagnostic Potential of the C II Lines at 133.5 nm in the Solar Atmosphere”, ApJ, 811, 81R.

Rempel, M., 2018. “Small-scale Dynamo Simulations: Magnetic Field Amplification in Exploding Granules and the Role of Deep and Shallow Recirculation”. ApJ, 859, 161.

Sainz Dalda, A., de la Cruz Rodríguez, J., De Pontieu, B., Gošić, M. 2019. “Recovering Thermodynamics from Spectral Profiles observed by IRIS: A Machine and Deep Learning Approach”, ApJL, 875L, 18S.

Siu-Tapia, A. L.; Rempel, M.; Lagg, A., 2018. “Evershed and Counter-Evershed Flows in Sunspot MHD Simulations”. ApJ,852,66S.

Socas-Navarro, H., de la Cruz Rodríguez, J., Asensio Ramos, A. et al. 2015. “An open-source, massively parallel code for non-LTE synthesis and inversion of spectral lines and Zeeman-induced Stokes profiles”, A\&A, 577, 7.

Sollum, E. 1999. “Hydrogen Ionization in the Solar Atmosphere: Exact and Simplified Treat”. Master’s thesis, Institute of Theoretical Astrophysics, University of Oslo

Wright, P., Cheung, M.C.M., Thomas, R. et al. 2019. “DeepEM: A Deep Learning Approach for DEM Inversion”. In HelioML Ebook (http://helioml.org/)


