\section{Introduction}

In the pursuit of high performance, 
cryptographic software widely deployed throughout the internet 
is still often subject to dangerous 
%security-critical 
attacks~\cite{ws96,
  freestart, Heartbleed, tlstrunc, BhargavanDFPS14, ap13, POODLE,
  tlsattacker, BEAST, cryptoeprint:2016:475, sweet32, CRIME, polybug2,
  chachaoverflow,
  PincusB04, Szekeres2013,
  AfekS07,UseAfterFreeCWE,
  Dobrovitski03,DoubleFreeCWE,
  IntegerOverflowCWE}. 
Recognizing a clear need, %serious risk,
the programming language,
verification, and 
applied cryptography communities are 
devoting
significant efforts to develop implementations proven
secure by construction against 
broad classes of these attacks.

Focusing on low-level attacks caused by violations of
memory safety, several researchers have used high-level, type-safe
programming languages to implement standard protocols such as Transport Layer
Security (TLS).
% , which now handles most web traffic~\cite{morethan50}. 
For example, \citet{kaloper2015not} provide nqsbTLS, an implementation of TLS in
OCaml, which by virtue of its type and memory safety is impervious to attacks
(like \citet{Heartbleed}) that exploit buffer overflows.
\citet{bhargavan2014proving} program miTLS
in F\#, also enjoying type and memory safety, but go further using a refinement
type system to prove various higher-level
security properties of the protocol.
%
While this approach is attractive for its simplicity, to get
acceptable performance, both nqsbTLS and miTLS link with fast, unsafe
implementations of complex cryptographic algorithms, such as those provided by
\citet{nocrypto}, an implementation that mixes C and OCaml, and
libcrypto, a component of the widely used \citet{openssl}.
In the worst case, linking with vulnerable C code 
can void all the %security 
guarantees of the high-level code.

In this paper, we aim to bridge the gap between high-level,
safe-by-construction code, optimized for clarity and ease of
verification, and low-level code exerting fine control
over data representations and memory layout in order to achieve
better performance.
%
To this end, 
we introduce \lowstar, a domain-specific language
for verified, efficient, low-level programming embedded within 
\fstar~\cite{mumon}, an ML-like language with dependent
types designed for program verification.
%
We
% \ch{who's we? can we just make this passive (``one can use'')?
%   or use ``the [mythical] programmer'' as the subject? \lowstar
%   is not just for we to use, right? CF: we = the reader and us.}
use \fstar
% 's verification features 
to prove functional correctness
and security properties of high-level code. 
%
Where efficiency is paramount, we drop into its C-like \lowstar subset
while still relying on
%the host language's
the verification capabilities of~\fstar
to prove our code is memory safe, functionally correct, and secure.

We have applied \lowstar to program and verify a range of sequential low-level
programs, including libraries for multi-precision arithmetic and
buffers, and various cryptographic algorithms, constructions, and
protocols built on top of them.
%
Our experiments indicate that compiled \lowstar code yields performance
on par with existing C code. This code can be used on its own,
or used within existing software through the C ABI. In particular, our C code may
be linked to \fstar programs compiled to
OCaml, providing large speed-ups via its foreign-function interface (FFI)
without compromising safety or security.%
% \footnote{In this second case, as we switch to verified
%   \lowstar libraries, we still trust the OCaml runtime and its
%   foreign-function interface mechanism.}

% \cf{DONE. This wrongly suggested Ocaml is our main deployment scenario. We
%   want to replace C code, not just make Ocaml faster.}

% \ch{add ``sequential'' to make it clear that we don't deal with
%   concurrency in this paper}

% \ch{My other worry here is that we present ourselves as the only
%   potential users of this DSL and miTLS as the only use
%   case. Scratching your own itch is usually not enough to motivate a
%   general piece of work. It also plays badly with double-blind review,
%   since miTLS should be cited in 3rd person and in this paragraph
%   we=miTLS.}
% \nik{I don't see why you interpret ``we'' as ``miTLS''}

\subsection*{An Embedded DSL, Compiled Natively}

\lowstar programs are a subset of \fstar programs:
% \ch{Cedric dropped
%   this, but I think it's an important point. What do others think?}\cf{
%   This does not make sense grammatically.}\ch{Looks fine to me}
the programmer
writes \lowstar code using regular \fstar syntax, against a library we
provide that models a lower-level view of memory, akin to the
structured memory layout of a well-defined C program
(this is similar to the structured memory model
of CompCert~\cite{Leroy-Compcert-CACM, 2012-Leroy-Appel-Blazy-Stewart},
rather than the ``big array of bytes'' model systems programmers sometimes use).
%
\lowstar programs interoperate naturally with
other \fstar programs, and precise specifications of \lowstar and
\fstar code are intermingled when proving properties of their
combination. As usual in \fstar, programs are type-checked and compiled
to OCaml for execution, after erasing their computationally irrelevant parts,
such as proofs and specifications, using a process similar
to Coq's extraction mechanism~\cite{letouzey2002new}. 
In particular, our memory-model library compiles to a simple, heap-based OCaml implementation.
% \cf{Does it help? also adjusted the explanation below: manual memory
%   management is in low star, irrespective of the back-end.}

Importantly, \lowstar programs have a second, equivalent but more efficient
semantics via compilation to~C, with the predictable
performance that comes with a native implementation of their
lower-level memory model. 
% that reaps the benefits of programming against a lower-level memory model,
% with manual allocations and deallocations%
% including manual memory management%
%manually managed memory allocation and deallocation
This compilation is implemented by KreMLin, a new
compiler from the \lowstar subset of \fstar to~C. 
Figure~\ref{fig:bigpicture} illustrates the high-level design of
\lowstar and its compilation to native code.
%
Our main contributions are as follows:

{\hypersetup{hidelinks=true}
\begin{figure}
  \centering

  \definecolor{butter1}{HTML}{FCE94F}
  \definecolor{butter2}{HTML}{EDD400}
  \definecolor{orange1}{HTML}{FCAF3E}
  \definecolor{orange2}{HTML}{F57900}
  \definecolor{orange3}{HTML}{CE5C00}
  \definecolor{chocolate1}{HTML}{E9B96E}
  \definecolor{chocolate2}{HTML}{C17D11}
  \definecolor{chocolate3}{HTML}{8F5902}
  \definecolor{chameleon1}{HTML}{8AE234}
  \definecolor{chameleon2}{HTML}{73D216}
  \definecolor{chameleon3}{HTML}{4E9A06}
  \definecolor{skyblue1}{HTML}{729FCF}
  \definecolor{skyblue2}{HTML}{3465A4}
  \definecolor{skyblue3}{HTML}{204A87}
  \definecolor{plum1}{HTML}{AD7FA8}
  \definecolor{plum2}{HTML}{75507B}
  \definecolor{plum3}{HTML}{5C3566}
  \definecolor{red1}{HTML}{EF2929}
  \definecolor{red2}{HTML}{CC0000}
  \definecolor{red3}{HTML}{A40000}
  \definecolor{aluminium1}{HTML}{EEEEEC}
  \definecolor{aluminium2}{HTML}{D3D7CF}
  \definecolor{aluminium3}{HTML}{BABDB6}
  \definecolor{aluminium4}{HTML}{888A85}
  \definecolor{aluminium5}{HTML}{555753}
  \definecolor{aluminium6}{HTML}{2E3436}

  \definecolor{skyblue4}{HTML}{9CBBDD}
  \definecolor{plum4}{HTML}{C5A5C1}
  \definecolor{chameleon4}{HTML}{ADEA70}

  \begin{tikzpicture}[lang/.style={
    draw, thick, text width=4em, node distance=3cm, align=center,
    text height=.8em, text depth=.5ex
  },label/.style={
    text=\footnotesize
  },every node/.style={
    font=\sffamily\footnotesize
  }]

    % EMF* node + label
    \node [thick, fill=skyblue1, draw=skyblue3, minimum width=16em, minimum height=4em] (emf) {};
    \node [anchor=north west, draw=none] at ($(emf.north west)+(.2em,-.2em)$)
      { EM\fstar };
    % Low* node + label
    \node [anchor=south east, thick, fill=skyblue2, draw=skyblue3, minimum width=12em, minimum height=3em]
      at ($(emf.south east)+(-.4em,.4em)$)
      (lowstar) {};
    \node [anchor=north west, draw=none, text=aluminium1] at (lowstar.north west)
      (lowlabel)
      { \lowstar };
    % EMF*/first-order node + label
    \node (foemf)
      [lang,fill=skyblue3,text=aluminium1,anchor=south east,text width=5.8em]
      at ($(lowstar.south east)+(-.4em,.4em)$)
      { 1\textsuperscript{st}-order EM\fstar };

    % JP: previously this box was: \lowstar to \lamstar
    \node (lamstar) [lang, fill=plum2,text=aluminium1, anchor=east, node distance=4em]
      at ($(lowstar.east)+(0,-4.5em)$)
      { \lamstar };
    \node (cstar)   [lang, fill=plum2,text=aluminium1, left of=lamstar] { \cstar };
    \node (c)       [lang, fill=plum2,text=aluminium1, left of=cstar] { Clight };
    \node (cfile)   [lang, fill=chameleon3,text=aluminium1, anchor=west, node distance=4em]
      at ($(c.west)+(0,-4.5em)$)
      { \ttfamily{}.c };
    \node (exe)     [lang, fill=chameleon3,text=aluminium1, right of=cfile] { Exe };

    \draw [->, ultra thick]
      [postaction={
          decorate,
          decoration={
            raise=.4ex,
            text along path,
            text align=center,
            % reverse path,
            text={|\sffamily\scriptsize\color{aluminium1}|{$\approx$} erase} }
            }]
      [postaction={
          decorate,
          decoration={
            raise=-1.55ex,
            text along path,
            text align=center,
            % reverse path,
            text={|\sffamily\scriptsize\color{aluminium1}|{\S}3.0}
          }
      }]
    (lowlabel.south) to [out=-90,in=180] (foemf.west);
    \draw [ultra thick,->]
      (foemf.south)
      to [out=-90,in=90]
      node[left,yshift=-.5ex] { \scriptsize partial $\approx$ }
      node[right=.5ex,yshift=-.5ex] { \sref{lamstar} }
      (lamstar);
    \path [ultra thick,->] (cstar)
      edge [loop below]
      node [text=plum3,yshift=.3ex] (hoist) { \scriptsize hoist $\approx$ }
      (cstar);
    \draw [ultra thick,->] (lamstar) --
      node[above] { \normalsize$\approx$ }
      node[below] { \sref{lamstar-to-cstar} }
      ++ (cstar);
    \draw [ultra thick,->] (cstar) --
      node[above] { \normalsize$\approx$ }
      node[below] { \sref{to-clight} }
      ++ (c);
    \draw [ultra thick,->] (c) --
      node[midway,text=plum3] (print) { \hspace{2.5em}\scriptsize{}print }
      ++ (cfile);
    \draw [->, ultra thick] (cfile) --
      node[above] { \scriptsize{}compile }
      ++ (exe);

  \begin{scope}[on background layer]
    % \draw [draw=none,fill=skyblue4]
    %   ($(lowstar.north west)+(-.4em,.4em)$) rectangle ($(lowstar.south east)+(.4em,-.4em)$);
    \draw [thick, fill=chameleon1, draw=chameleon3]
      ($(cfile.north west)+(-.4em,.4em)$) rectangle ($(exe.south east)+(.4em,-.4em)$);
    \draw [thick, draw=plum3,fill=plum1]
      let \p1 = (c.west) in
      let \p2 = (lamstar.north) in
      ($(\x1, \y2)+(-.4em,.4em)$) rectangle ($(lamstar.south east)+(.4em,-.4em)$);
    % \draw [draw=none,fill=skyblue4]
    %   (erase.north west) rectangle (erase.south east);
    % \draw [draw=none,fill=plum4]
    %   (print.north west) rectangle (print.south east);
    % \draw [draw=none,fill=plum4]
    %   ($(hoist.north west)+(1em,1em)$) rectangle ($(hoist.north east)+(-1em,-1ex)$);

    \node [draw=none,anchor=north east] at ($(emf.north west)+(0,.2em)$)
      {\footnotesize\color{skyblue3}\fstar};
    \path
      let \p1 = (c.west) in
      let \p2 = (lamstar.north) in
      node [draw=none,anchor=south west]
      at ($(\x1, \y2)+(-.4em,.45em)$)
      {\footnotesize\color{plum3}Kremlin };
    \node [draw=none,anchor=south west] at ($(exe.south east)+(.4em,-.8em)$)
      {\footnotesize\color{chameleon3}GCC/Clang/CompCert };
  \end{scope}

  \end{tikzpicture}

  \caption{\lowstar embedded in \fstar, compiled to C, with soundness and
  security guarantees (details in \sref{formal})}
  \label{fig:bigpicture}
  \vspace{-2ex}
\end{figure}
}

% \ch{Can we try to make the F*/EMF*/Labels waste less vertical space
%     and use instead more of the unused horizontal space?}


\paragraph*{Libraries for low-level programming within \fstar~(\S\ref{sec:examples})}
At its core, \fstar is a purely functional language
to which effects like state are added programmatically using monads.
%
In this work, we instantiate the state monad of \fstar to use a CompCert-like
structured memory model that separates the stack and the heap,
supporting bulk %en masse 
allocation and deallocation on the stack,
and allocating and freeing individual blocks %reference cells 
on the heap.
%
Both the heap and the stack are further divided into disjoint logical
regions, which enables us to manage the separation properties
necessary for modular, stateful verification.
%
On top of this, we program a library of C-style arrays and structs passed by
reference, with support for pointer arithmetic and pointers to the interior of
an array or a struct.  
%
By virtue of \fstar typing, our libraries and all their well-typed
clients are guaranteed to be memory safe, e.g., they never access
out-of-bounds or deallocated memory.

\paragraph*{Designing \lowstar, a subset of \fstar easily compiled to C}
We intend to give \lowstar programmers precise control over the
performance profile of the generated C code. As much as possible, 
we aim for the programmer to control even the
syntactic structure of the C code, to
facilitate its review by security experts unfamiliar with \fstar. 
As such, to
a first approximation, \lowstar programs are \fstar programs
well-typed in the state monad described above, which, after all their
computationally irrelevant (ghost) parts have been erased, must
meet several restrictions, as follows: the code 
(1) must be first order, to prevent the need to allocate closures in C; 
(2) must make any heap allocation explicit; 
%\cf{how would I know? rewrite to: must explicitly allocate xxxx?}
(3) must not use any recursive datatype, since these would have to be compiled using additional
indirections to C structs; and 
(4) must be monomorphic, since C does not support polymorphism directly. 
%
% \cf{would be nice to mention inlining above. Trying:}%
% JP: rephrased the paragraph below
Importantly, \lowstar heavily leverages \fstar's capabilities for
partial evaluation, hence allowing the programmer to write high-level, reusable
code that is normalized via meta-programming into the \lowstar subset before
the restrictions are enforced.
%
We emphasize that these restrictions apply only to computationally
relevant code---proofs and specifications are free to use arbitrary
higher-order, dependently typed \fstar, and very often they do.


%% \aseem{Moved this text to beginning of s3.}
%% \paragraph{Formalizing \lowstar and its translation to Comp\-Cert
%%   Clight (\S\ref{sec:formal})}
%% We formalize different layers of our toolchain and prove several
%% properties about them.

%% For formalizing \fstar, we use EM\fstar~\cite{dm4free}, the
%% recently proposed core calculus of \fstar. To reason about the erasure
%% of computationally irrelevant code, we extend EM\fstar with a
%% \kw{Ghost} monad, and prove that the erasure of the \kw{Ghost} code
%% does not alter the semantics of an EM\fstar term.

%% We formalize the core of \lowstar as \lamstar, with a primitive
%% stack-based memory. The semantics of \lamstar contains a notion
%% of observable traces, where the traces contain the memory addresses
%% accessed by the \lamstar program, as well as its branching
%% behavior. We prove a non-interference property stating that the execution
%% trace of a \lamstar program that uses the secret values
%% \emph{abstractly}\ch{parametrically? at an abstract type?} is
%% secret-independent.  We have not yet formally
%% proved the simulation between the EM\fstar semantics (instantiated
%% with the stack-based memory model) and the \lamstar
%% semantics, but we conjecture that the proof would be on the similar
%% lines as~\cite{dm4free}.\ch{can we leave this conjecture for
%%   section 3? it seems too emphasized here and we could focus on
%%   what we proved, not on what we didn't prove}

%% We then formalize \cstar, a model of C and prove a
%% bisimulation between \lamstar and \cstar. To establish that \cstar is
%% a faithful model of the C-semantics, we formalize a translation
%% from \cstar to Comp\-Cert's Clight~\cite{Blazy-Leroy-Clight-09}, and show that this
%% translation preserves the notion of secret-independent traces. In
%% addition to ensuring that the functional behavior of a program is
%% preserved, trace equivalence also guarantees that the compiler does
%% not introduce unexpected side-channels due to timing or memory access
%% patterns, at least until it reaches Clight---a useful feature for
%% cryptographic code.
%% % \ch{This claim sounds too strong, Cedric thinks we should be very careful
%% %   about what we claim about side-channels.
%% % \nik{I added the ``until it reaches Clight''}}
%% % CH: replaced cache behavior with memory access patterns
%% %     (there are no caches at the Clight level)
%% %

%% Our formal results cover the translation of standalone \lowstar
%% programs to C, proving that execution in C preserves the original
%% \fstar semantics of the \lowstar program. More pragmatically, we have
%% built several cryptographic libraries in \lowstar, compiled them to C,
%% and integrated them within larger programs compiled to OCaml, relying
%% on OCaml's foreign function interface for interoperability and
%% trusting it for correctness.

\paragraph*{A formal translation from \lowstar to Comp\-Cert Clight (\sref{formal})}
Justifying its dual interpretation as a subset of \fstar and a subset of C,
we provide a formal model of \lowstar, called \lamstar,
give a translation from \lamstar to %Comp\-Cert's
Clight~\cite{Blazy-Leroy-Clight-09} and show that it
preserves %a notion of 
trace equivalence with respect to the original
\fstar semantics. In addition to ensuring that the
functional behavior of a program is preserved, our trace equivalence also
guarantees that the compiler does not accidentally introduce 
side-channels due to %timing or 
memory access patterns (as would be the case without the restrictions above)
at least until it reaches Clight, a useful sanity check for cryptographic code.
% \ch{This claim sounds too strong, Cedric thinks we should be very careful
%   about what we claim about side-channels.
% \nik{I added the ``until it reaches Clight''}}
% CH: replaced cache behavior with memory access patterns
%     (there are no caches at the Clight level)
%
Our theorems cover the translation of standalone \lamstar
programs to~C, proving that execution in C preserves the original
\fstar semantics of the \lamstar program. 

\paragraph*{%The implementation of 
KreMLin, a compiler from \lowstar to C (\S\ref{sec:impl})}
Our formal development guides the implementation of KreMLin, a new
tool that emits C code from \lowstar. KreMLin is designed to emit
well-formatted, idiomatic C code suitable for manual review.
% redundant:
% by security experts unfamiliar with \fstar. 
The resulting C programs can be
compiled with Comp\-Cert for greatest assurance, and with mainstream C
compilers, including GCC and Clang, for greatest performance.  
We have used KreMLin to extract to C the 20,000+ lines of 
\lowstar code we have written so far.
%
After compilation, our verified standalone C libraries can be
integrated within larger programs using standard means.
%% , e.g., they may
%% be used within other \fstar programs by relying on OCaml's foreign
%% function interface for interoperability and trusting it for
%% correctness.  
\cf{LATER: \nik{moved it here from the previous para}.Does KreMLin
  support the generation of FFI? Do we really claim that ocaml-interop
  is the only part of Kremlin not covered by our formal results? This
  is our usual ambiguity between \lowstar as our formal source
  language vs as the subset of \fstar currently supported by KreMLin.}

\paragraph*{An empirical evaluation (\S\ref{sec:moreexamples})}
We present a few developments of efficient, verified, interoperable
cryptographic libraries programmed in \lowstar.

(1) We provide \haclstar, a ``high-assurance crypto library''
implementing and proving (in $\sim$6,000 lines of \lowstar) several
cryptographic algorithms, including the Poly1305
MAC~\cite{bernstein2005poly1305}, the ChaCha20 cipher~\cite{chacha20},
and multiplication on the Curve25519 elliptic curve
~\cite{curve25519}.
%
We package these algorithms to provide the popular NaCl
API~\cite{bernstein2012security}, yielding the first performant
implementation of NaCl verified for memory safety and side-channel
resistance, along with functional correctness proofs for its core
components, including a verified bignum library customized for safe,
fast cryptographic use (\S\ref{sec:haclstar}).  
%
Using this API, we build new standalone applications such as
 \emph{PneuTube}, a new secure, asynchronous, file transfer 
application whose performance compares favorably with widely 
used, existing utilities like scp.
\cf{Should we cite the CSF paper? Not all of it is new}

(2) Emphasizing the applicability of \lowstar for high-level,
cryptographic security proofs on low-level code, we briefly describe
its use in programming and proving (in $\sim$14,000 lines of \lowstar)
the Authenticated Encryption with Associated Data (AEAD) construction
at the heart of the record layer of the new TLS 1.3 Internet Standard.
%
We prove memory safety, functional correctness, and cryptographic
security for its main ciphersuites, relying, where available, on
verified implementations of these ciphersuites provided by \haclstar.
%
The C code extracted from our verified implementation is easily
integrated within other applications, including, for example, an
implementation in \fstar of TLS separately verified and compiled to
OCaml (through OCaml's FFI).


%% (which still relies on the high-level memory model of \fstar).


%% We obtain an API with strong type-based guarantees.
%% \cf{not sure what to say on type abstraction or side-channel protections.}
%% %
%% We compile our \lowstar AEAD code both to OCaml and to C, linking it
%% in both cases with the \fstar implementation of miTLS compiled to
%% OCaml (which still relies on the high-level memory model of \fstar).
%% %
%% As expected by the design of \lowstar, relative to the full OCaml
%% version, miTLS linked with the C version of our libraries shows an
%% improvement in throughput of data transfer by at least two orders of
%% magnitude.  There is still room for improvement, however---one measure
%% shows our verified code integrated with miTLS is still be around
%% 5$\times$ slower than OpenSSL's libcrypto. (\S\ref{sec:aead})
%% \nik{TODO: revise this part after cleaning up the experimental story}
%
% \ch{All this sounds very impressive; isn't
%   this precisely the Oakland submission though? If so how do we still
%   mention this without claiming the same contribution in two different
%   papers, which I don't think is allowed.}
% %
% \nik{The Oakland submission is about the cryptographic model and the
%   proof of security. This paper is about how we program the code,
%   compile it and get good performance.}
\cf{LATER: cite the oakland paper.}


\paragraph*{Trusted computing base} To date,
we have focused on designing and evaluating our methodology of
programming and verifying low-level code shallowly embedded within a
high-level programming language and proof assistant.
%
We have yet to invest effort in minimizing the trusted computing base
of our work, an effort we plan to expend now that we have evidence
that our methodology is worthwhile.
%
Currently, the trusted computing base of our verified libraries
includes the implementation of the \fstar typechecker and the Z3 SMT
solver~\citep{MouraB08}.
%
Additionally, we trust the manual proofs of the metatheory relating
the semantics of \lamstar to CompCert Clight. The KreMLin tool is
informed by this metatheory, but is currently implemented in
unverified OCaml, and is also trusted.
%
Finally, we inherit the trust assumptions of the C compiler used to
compile the code extracted from \lowstar.

\paragraph*{Supplementary materials}
First, we provide, in the appendix, the hand proofs
of the theorems described in \sref{formal}.
%
The present paper is focused on the metatheory and tools; we also authored a
companion paper~\cite{record} that describes the cryptographic model we
used for the record layer of TLS 1.3.
%
Finally, we have an ongoing submission of a paper focused on our
\haclstar library~\cite{haclstar}, where we
describe in greater detail our
proof techniques for reusing the bignum formalization across different
algorithms and implementations, and provide a substantial performance
evaluation.

We also offer numerous software artifacts. Our tool \kremlin~\cite{kremlin} is
actively developed on GitHub, and so is \haclstar~\cite{haclstar-gh}. Most of
the \lowstar libraries live in the \fstar repository, also on GitHub. The
integration of \haclstar within miTLS is also available on GitHub. For
convenience, we offer a regularly-updated Docker image of Project
Everest~\cite{everest-website}, which bundles together
\fstar, miTLS, \haclstar, \kremlin. One may
fetch it via \texttt{docker pull projecteverest/everest}.  The Docker image
contains a \texttt{README.md} with an overview of the proofs and the code.
%

% A separate paper describing our cryptographic model of the TLS 1.3
% record layer will soon be published and is also available in the
% non-anonymous supplementary material.

%% Full formalization \cf{avoid?
%%   this somehow suggests a machine-checked development.} and hand
%% proofs of all the theorems in this paper are available in the long
%% version\ifanon{} included with this submission\fi.  We also provide
%% the KreMLin compiler and our verified \lowstar code for AEAD,
%% \haclstar, and PneuTube as supplementary materials. %% Our
%% %% experiments use \fstar and are partly based on miTLS, both of which
%% %% are publicly available in open source repositories.

%

\if0

An alternative execution path alternative execution

dual reading: for verification
purposes, it is an \fstar program with all the verification advantages
this entails; for execution purposes, it is a C program, that performs
predictable memory allocation and whose performance model is
clear. The \lowstar code may be low-level, but its proof is usually
not.




Like any \fstar program, a \lowstar program can be

C memory layout. Lik


, a language in
the tradition of ML equipped with dependent types, monadic effects,
refinement types and a weakest precondition calculus.

 If the program is well-typed and falls within the
\lowstar subset, then it abides by the constraints of the C memory model and can
be translated directly to well-defined C code.
A \lowstar program enjoys a dual reading: for verification purposes,
it is an \fstar program with all the verification advantages this
entails; for execution purposes, it is a C program, that
performs predictable memory allocation and whose performance model is
clear. The \lowstar code may be low-level, but its proof is usually not.




Security-critical components of the internet infrastructure have
received sustained interest from the verification community over the past few
years.

\subsection{High-level Verification for Low-Level Code}
Security-critical components of the internet infrastructure have
received sustained interest from the verification community over the past few
years. Indeed, a combination of design flaws and implementation mistakes has led
to a series of highly-publicized attacks, sometimes with devastating consequences,
and the realization we are still depending on substantial amounts of insecure code in
deployed systems.

One crucially important class of software is cryptographic
libraries,\ch{implementations/applications (broader), miTLS and
  HACL* is a crypto library, but PneuTube are not really libraries!?}
which pervade the entire internet stack.
Setting aside the logical or mathematical
attacks, which break the underlying mathematical robustness
assumptions~\cite{freestart} to the point that a brute-force attack may be
possible, cryptography is notoriously difficult to implement correctly. As one
cryptographer accurately puts it, ``the only people who hate [the GCM cipher]
are those who’ve had to implement it''~\cite{green-gcm}. Therefore,
most of the high-profile vulnerabilities in cryptographic
libraries\ch{implementations/applications} have targeted
implementation flaws~\cite{Heartbleed}.
\ch{other examples?}

Many strands of work\jp{ TODO: cite} have shown that re-building parts
of this infrastructure using verification and formal methods is a
promising way of removing such vulnerabilities.
%
Yet, verification often proves very challenging. One reason is that
cryptographic code works at a very low-level; the performance of cryptography is
often the limiting factor after network speed in many networked applications,
and the code is hand-optimized to squeeze the last bit of performance out of the
system. In practice, this means that any replacement must be at least
as good as the existing C libraries, and offer the same level of portability.


The contribution of this paper is \emph{the design, proof, implementation
and evaluation of \lowstar}, a domain-specific language for
verified low-level programming
that tries to combine the advantages of the approaches above.

\lowstar is a shallow embedding of C inside of \fstar, a language in
the tradition of ML equipped with dependent types, monadic effects,
refinement types and a weakest precondition calculus. The programmer writes
\lowstar code using regular \fstar syntax, against a library we provide that
models the C memory layout. If the program is well-typed and falls within the
\lowstar subset, then it abides by the constraints of the C memory model and can
be translated directly to well-defined C code.
A \lowstar program enjoys a dual reading: for verification purposes,
it is an \fstar program with all the verification advantages this
entails; for execution purposes, it is a C program, that
performs predictable memory allocation and whose performance model is
clear. The \lowstar code may be low-level, but its proof is usually not.

\subsection{An Overview of \lowstar}
% \lowstar is \emph{designed} as a subset of \fstar, along with a set of libraries that
% model the C stack and heap, and their mutable data structures (buffers). \lowstar exposes a
% \emph{curated} set of low-level primitives, meaning the programmer writes code in the
% \lowstar subset of \fstar, but may use any degree of sophistication in the proofs. In
% short, the code is low-level, but the verification is not. Once a \lowstar program
% type-checks, proofs are erased and it is extracted by a translation to C.
% By allowing what we deem a ``reasonable'' subset of C, the programmer is faced
% with less proof obligations than with a general C program, and can combine
% the performance benefits of low-level programming with the expressive power of
% the \fstar verification language.


\fref{bigpicture} depicts the various steps that convert a \lowstar
source program to executable machine code.\ch{How about calling the
  first box \lowstar?}
%
Black boxes denote intermediary languages;
colored background rectangles indicate which concrete tool implements a given
translation step in the diagram.

\lowstar is designed as a subset of \fstar. As such, we rely on the latest
formalization of \fstar~\cite{dm4free}, which introduces EM\fstar, a
dependently-typed core calculus with a user-extensible lattice of effects.
EM\fstar is pure, and user-defined effects are pure too; this allows the
verification system of \fstar to reason over pure effect combinators and
programs.\ch{Bringing up EMF* and formalization seems unjustified.
  Very much afraid we'll lose many readers by going in gory formal
  details and \fstar jargon for little gain. I would prefer explaining
  this in intuitive terms from more basic principles, like monads,
  than expecting the reader to know \cite{dm4free} already}

Programs are not generally executed against a pure implementation of the state;
rather, they rely on the execution environment's primitive notion of
state. The original \emfst~\cite{dm4free} semantics model the
built-in state of the OCaml extraction target of \fstar; the authors show that
\emfst is a simulation of EM\fstar, i.e. that a \emph{concrete} implementation
using native state preserves the pure semantics of the original, pure \li+ST+
effect that the programmer dealt with.\ch{Again, why not phrase this in
  terms of actual implementation, instead of theory? In terms of actual
  implementation \fstar erases types using an algorithm very similar to Coq;
  and uses primitive OCaml effects on top of that.}

\ch{How about just saying that effects in \fstar{} can be defined and
  customized by the user using monads. The state effect of \fstar{} is
  parameterized by the precise memory model.  Code using the state
  effect instantiated with HyperHeaps~\cite{mumon} can be translated
  to native OCaml state, with automatic memory management and garbage
  collection. Here we parametrize the state effect of \fstar{} with
  HyperStacks, a memory model corresponding to C stack allocation.}

In a similar fashion,\ch{the similar fashion part doesn't seem understandable
  to someone who didn't read \cite{dm4free}}
we design a ``HyperStack'' memory model that captures the
C stack discipline, and model stack-allocated, mutable, by-reference arrays with
static liveness and bounds checks.
Pointer arithmetic on buffers is checked statically;\ch{Do you really check the
  pointer arithmetic, not the actual memory accesses? So you can't get a off-by-one
  pointer at the end of a loop, as often happens in perfectly valid C?}
``ghost'' lengths that only exist in the type system allow verification to show that
well-typed programs are memory safe. We instantiate EM\fstar with these
``HyperStack'' semantics and obtain \emfhst, a language that enjoys a simulation
to EM\fstar, and whose semantics feature a primitive notion of state with a
stack of frames.  We further define an erasure transformation over \emfhst that
removes calls to lemmas and ghost code while preserving semantics.

\fstar effects are user-defined so this required no change to F* and could be
implemented as a library. This library does not
expose the full power of C; rather, it exposes a \emph{curated} set of
combinators. This strikes a balance between expressiveness, and ease of proof /
programmer productivity. Notably, programs may not take the address of an
arbitrary object, and may not perform any sort of memory trick.\ch{unclear
  what you mean by memory trick, maybe give concrete examples of the things
  you prohibit?}  This discipline
did not turn out to be a restriction for the several thousand lines of
cryptographic code we wrote in \lowstar.\ch{20,000? should maybe define
  a macro for that?}

This code can still be extracted to OCaml using the existing extraction facility
of \fstar; this would, however, be a poor idea: the presence of a garbage
collector exposes our program to side-channels, and the uniform, tagged memory
representation would make array subscripting operations linear instead of
constant-time in the C semantics.\ch{Still don't understand this one,
  seems very much like OCaml jargon}

At this stage, a program may not qualify as \lowstar; the programmer may, for
instance, use partial applications or closures. This is part of our design:
\lowstar is not a general-purpose compilation scheme to C for arbitrary \fstar
programs. Rather, we view it as a DSL that shines for writing cryptographic
systems. \ch{wasn't this already said above, all this seems just an
  unstructured collection of thoughts}

We write out the semantics of the \emfhst language as \lamstar, a core calculus
that thus models the \lowstar subset that the user programs in. This step
contains a degree of informality: we have not written out the output of
instantiating EM\fstar with HyperStack, and claim that \lamstar is the same as
``erased \emfhst'', modulo $\alpha$-renaming. \ch{gory details}

We wish to prove that our compilation scheme to C is sound. For that purpose, we
introduce \cstar, a statement language that we claim is an ``idealized''
subset of C. \cstar has an explicit notion of a stack and of block scope.
\lamstar programs require two simplification before they can be soundly
translated to \cstar; this is due to the function-level notion of scope in
\lamstar, along with the \li+StackInline+ effect our memory model offers.
The translated \cstar program refines the semantics of the
\lamstar program. \ch{gory details}

Finally, we wish to relate \cstar to an actual formalization of C. To that
effect, we define a translation from \cstar to Comp\-Cert's
Clight~\cite{Blazy-Leroy-Clight-09}. The final compilation step may be handled
by a certified compiler, such as
Comp\-Cert~\cite{Leroy-Compcert-CACM,compcert-url}, and is not covered here.
\ch{gory details}

In addition to semantics preservation, our simulations also prove that our
compilation scheme preserves side-channel security properties. Combined with our
proof that well-typed \lowstar programs are free of memory access-based and
branching-based side-channels, our toolchain thus provides a degree of
end-to-end side-channel protection.

\lowstar is \emph{implemented} via KreMLin, a new tool that extracts an \fstar
program into a set of \emph{readable}, \emph{manually-reviewable} C files that
map almost transparently to the original code. KreMLin implements a superset of
the transformations described in the present paper, and can already extract
the 20,000+ lines of cryptographic code we have written.

\lowstar is \emph{evaluted} using two flagship cryptographic developments.
The first one is \haclstar,
a \lowstar library that aims to provide a reference, gold-standard, verified set of
cryptographic routines.\ch{Could mention which ones?}
\haclstar implements the popular \li+box+/\li+unbox+ API
from \li+libsodium+; its performance is comparable to that of the original
\li+sodium+ library. The second is our AEAD\ch{what's AEAD? please explain}
development, that provides a
cryptographically-sound AEAD implementation for several combinations of stream
ciphers and MAC algorithms. We evaluate the interoperability claim via an
ongoing replacement of the cryptographic stack of the miTLS project with our
AEAD library; measured against OpenSSL, our development exhibits comparable
performance.

\ch{How about structuring the contributions better, e.g. separate subsection
  (or just used named paragraphs instead of subsections in the intro)}

\fi
% packaged as a set of C files. 

% so that it does not expose the full power of C; the programmer may not, for
% instance, take the adress of any expression. 

% At one end of the spectrum, one may start from an optimized, existing C
% implementation of cryptographic code, and show that memory accesses are safe;
% that certain invariants are preserved; and, eventually, that functional
% correctness is preserved. In short, one may build proof structure on top of
% optimized code. At the other end of the spectrum, one may start with a language
% designed for proofs; prove functional correctness; then gradually move the code
% towards a more low-level, efficient implementation while extending the proof to
% deal with the correctness of lower-level code.

% The present paper proposes a hybrid approach that mixes a low-level 

% \comment{JP: be more general and say that one can write code at the level of C
% directly (e.g. Frama-C, Vcc, Verifast) vs. in a high-level language}
% %
% There are several ways to tackle the issue. On one end of the spectrum, one may
% work within the Coq proof assistant on a deep embedding of the C language,
% possibly relying on Comp\-Cert's semantics. The deep embedding models the full
% semantics of C; if the Coq development is proven, then the code has the desired
% behavior~\cite{chlipala2013bedrock}. One may add cryptographic reasoning to the
% Coq development in the form of games. This approach has yielded successful
% results for self-contained core cryptographic building blocks, such as the HMAC
% algorithm~\cite{beringer2015verified}.

% At the other end of the spectrum, one may verify large-scale systems such as the
% TLS protocol~\cite{bhargavan2014proving}.\ch{Could take TLS as the
%   example from the end of the first paragraph already. Otherwise a bit
%   unclear for what HMAC is a building block.}
% Using a high-level language, one
% relies on module abstraction, higher-order proofs and ghost code to create an
% elaborate proof structure on top of a rather mundane implementation. The
% high-level nature of the host language makes it a prime tool for
% game-based cryptographic proofs; however, such a toolset is not suited to
% low-level code. For instance, the released version of the miTLS
% library~\cite{bhargavan2014proving} models the TLS Handshake and Record Layer,
% but relies on OpenSSL's cryptographic ciphers for lack of an F7 implementation.

% Our approach bridges the gap between high-level and low-level verification by
% offering a \emph{whole-stack verification result}.\ch{I thought that
%   this paper was about the methodology/tools, not yet a whole-stack
%   verification result. Or is AEAD a whole-stack result? Is this
%   paragraph about that?}
% Our low-level subset of \fstar,
% dubbed \lowstar, exposes the \emph{memory model of C with the verification power of
% \fstar}.\ch{Should reintroduce \fstar in the intro, when talking about old miTLS}
% The programmer may use any degree of sophistication in the proofs, as they
% are erased; thus, the code may be low-level, but the verification is not. A
% piece of code that is well-typed in \fstar \emph{and} fits within \lowstar can be
% compiled to the C language; a series of simulations show that our compilation
% scheme preserves semantics all the way to Clight
% \cite{Blazy-Leroy-Clight-09}, a subset of C formally specified by
% Comp\-Cert \cite{Leroy-Compcert-CACM,compcert-url}. The rest of the
% program may not fit in \lowstar; it may thus be compiled using the original
% OCaml extraction mechanism of \fstar.
% The whole program is verified; for critical parts,
% the programmer opts into the \lowstar model and enjoys the same performance as in
% the C, at the expense of an extra proof burden for memory safety.

% \comment{JP: say something along the lines of ``we provide spatial safety (no
% out-of-bounds access), temporal safety (no use-after-free) and resource safety
% (no memory leaks if the code is in \li+Stack+)''. Really should hammer that we
% offer a transparent, predictable allocation model. Also, we should list 1) the
% contributions and 2) the ingredients that we need for low-level crypto, then
% explain why we have these, e.g. we have well-engineered curves meaning we must
% have fixed-size stack-allocated arrays and pointer arithmetic, and not much
% more\ldots}\ch{Spelling out the contributions is really important.  At
%   this point, the intro is focused almost entirely on AEAD as the
%   contribution, so one might wonder whether there are any PL
%   contributions here too. You're sending this to PLDI after all, so
%   even AEAD should be explained from the POV of a PL audience.}

% To demonstrate the viability of our approach, we replace the cryptographic stack
% of miTLS with a verified \lowstar implementation, effectively removing the
% dependency on OpenSSL.\ch{Is all of this crypto replacement really already done?
%   Aren't you talking only about the transport layer for now?}
% We operate on the in-progress rewrite of miTLS in \fstar,
% which aims at modeling the interaction of TLS 1.2 and the upcoming TLS 1.3
% standard~\cite{record}. Within the \lowstar subset, we write memory-safe,
% functionally-correct low-level cryptographic modules, starting from big number
% arithmetic.  Building upon this, we implement finite fields $GF(128)$ and
% $GF(2^{130}-5)$. These fields are used respectively for MAC algorithms GHASH and
% Poly1305; combining the latter two with the AES and ChaCha20 ciphers yields
% various implementations of the AEAD construction in \lowstar. We leverage the
% expressiveness of the \fstar proof system to show that we have a
% cryptographically-strong AEAD implementation for any combination of cipher and
% MAC algorithm. Our implementation is made up of $10,500$ lines of \fstar, including
% whitespace and comments.

% The compilation pipeline is as follows. The AEAD library in \lowstar is extracted to
% human-readable C code using a new tool dubbed KreMLin~\cite{kremlin}. For the
% miTLS development, we write (unverified) \li+ocaml-ctypes+ bindings, which glue
% together the rest of the code extracted via the regular OCaml extraction backend
% of \fstar. For the demo purposes of this paper, we link the library against a sample
% program dubbed PneuTube, which performs public-key authenticated and encrypted
% file transfer.

% Our AEAD implementation offers, among other APIs, the \li+box/unbox+
% primitives as exposed by the popular \li+libsodium+ cryptographic library.
% Performance measurements for a large file transfer using the PneuTube demo
% application show our code to be 40\% faster than OpenSSL,\ch{Is the OpenSSL
%   code you test against in C or ASM?}
% head-to-head with
% unoptimized \li+libsodium+ and 40\% slower than hand-optimized \li+libsodium+
% with assembly.

% Our long-term plan is to extract the entire miTLS codebase to C by fitting it
% within the \lowstar subset of \fstar. Indeed, our ambition is to see our
% provably-secure TLS code execute in the ``real world''; that is, servers (such as
% Apache, Nginx or IIS) and browsers (such as Chrome, Firefox or Edge). In that
% context, extracting to OCaml is not an option. The first reason is performance:
% switching to the OCaml value representation at ABI boundaries, and GC pauses are
% a hard sell for performance-conscious browser vendors. Second the target
% audience will most likely not be familiar with OCaml. Thus, for social and
% technical reasons, our plan is to extract to C or C++.

