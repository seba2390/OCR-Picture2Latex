% Needed for proper arXiv processing, it seems
\pdfoutput=1

% JP: note: \newif takes a single argument, generates \foofalse, which is then
% called to set the initial boolean value.
\newif\ifjonathan\jonathanfalse
\newif\ifanon\anonfalse
\newif\ifdraft\draftfalse
\newif\iflong\longfalse
\newif\ifpagelimits\pagelimitsfalse
\newif\ifcamera\cameratrue

% override various ifs in your local.tex file
\IfFileExists{local.tex}
  {%!TEX root = hopfwright.tex

\subsection{Constructing a Newton-like operator}
\label{s:newtonlike}

In this section and in the appendices we often suppress the subscript in $F=F_\epsilon$.
We will find solutions to the equation $F(\alpha ,\omega , c)=0$ by the
constructing a Newton-like operator $T$ such that fixed points of $T$
corresponds precisely to zeros of $F$. In order to construct the map $T$ we
need an operator $A^{\dagger}$ which is an approximate inverse of 
$DF(\bx_\epsilon)$. 
% Since
% $\bx_\epsilon$ is an approximate zero of $F_\epsilon$ up to order
% $\cO(\epsilon^2)$ correction terms,
We will use an approximation $A$ of 
$DF( \bx_\epsilon )$ that is linear in~$\epsilon$ and correct up to $\cO(\epsilon^2)$.
% (recall that $F(\bx_\epsilon)=\cO(\eps^2)$). 
Likewise, we define $A^{\dagger}$ to be linear in $\epsilon$ (and again correct up to $\cO(\epsilon^2)$). 

It will be convenient to use the usual identification $i_\C : \R^2 \to \C$ given by $i_\C (x,y) = x+iy $. We also use $\omega_0 := \pi/2$.
 % order
% Since $x(\epsilon)$ is only correct up to order $\cO(\epsilon^2)$, then it only makes sense to compute our approximate derivative up to order $\cO( \epsilon^2)$.

% \marginpar{Jonathan: I tried to be careful about the spaces here, but it all seems a bit of a distraction since everything is explicit in coordinates}
\begin{definition}\label{def:A}
We introduce the linear maps $A:  \R^2 \times \ell^K_0 \to \ell^1$ and 
$ A^{\dagger}:  \ell^1 \to  \R^2 \times \ell^K_0 $ by
\begin{alignat*}{1}
A &:= A_0 + \epsilon A_1 \, , \\
A^{\dagger} &:= A_0^{-1} - \epsilon A_0^{-1} A_1 A_0^{-1} \,  ,
%\label{eq:ADagger}
\end{alignat*}
where the linear maps $ A_0 , A_1 : \R^2 \times \ell^K_0 \to \ell^1$  are defined below. Writing $x=(\alpha,\omega,c)$, we set
\begin{alignat*}{1}
A_0	x = A_0 (\alpha,\omega,c) & := i_\C A_{0,1} 
\!\left[\!\! \begin{array}{c} \alpha \\ \omega \end{array} \!\!\right]  \e_1
 + A_{0,*}  c , \\
A_1 x =	A_1 (\alpha,\omega,c) & := i_\C  A_{1,2}
\!\left[\!\! \begin{array}{c} \alpha \\ \omega \end{array} \!\!\right]  \e_2
 + A_{1,*}  c .
%\label{eq:ApproxDFdef}
\end{alignat*}
Here the matrices $A_{0,1}$ and $A_{1,2}$ are given by
\begin{equation}
A_{0,1} := 
\left[
\begin{matrix}
0 & - \pp \\
-1  & 1 
\end{matrix} 
\right]
\qquad\text{and}\qquad
A_{1,2} := \frac{1}{5}
\left[
\begin{matrix}
-2 & 2-\tfrac{3 \pi}{2} \\
-4  & 2(2+\pi) 
\end{matrix}  
\right]  ,
\label{eq:defA12}
\end{equation}
and the linear maps $A_{0,*} : \ell^K_0 \to \ell^1_0$ and
$A_{1,*} : \ell^K_0 \to \ell^1$
are given by
\begin{equation*}
% A_{0,*} :& \ell^1_0 \to \ell^1_0
% &
% A_{1,*} :& \ell^1_0 \to \ell^1  \\
% %%%%%%
% A_{0,1} :& \{ \alpha, \omega\} \to \{ Re \, F_1 , Im\, F_1 \}
% &
% A_{1,2} :& \{ \alpha, \omega\} \to \{ Re \,F_2 , Im \, F_2 \}
% \end{align*}
% and given by the equations below, taking $ \omega_0 = \pp$.
% \begin{align}
A_{0,*} 	 := \tfrac{\pi}{2} ( i K^{-1} + U_{\omega_0}) 
\qquad\text{and}\qquad
A_{1,*} 	:= \tfrac{\pi}{2} L_{\omega_0} .
\end{equation*}
%%%%%%%%%%%%%%%%%%%%
% A_{0,1} := &
% \left[
% \begin{matrix}
% 0 & - \pp \\
% -1  & 1
% \end{matrix}
% \right]
% &
% A_{1,2} :=& \frac{1}{5}
% \left[
% \begin{matrix}
% -2 & 2-\tfrac{3 \pi}{2} \\
% -4  & 2(2+\pi)
% \end{matrix}
% \right]
% \label{eq:defA12}
% \end{align}
\end{definition}

Since $K$ and $U_{\omega_0}$ both act as diagonal operators, the inverse 
$A_{0,*}^{-1} : \ell^1_0 \to \ell^K_0$ of $A_{0,*}$ is given by
\begin{equation*}
	  (A_{0,*}^{-1} a)_k = \frac{2}{\pi} \frac{a_k}{ik+e^{-ik\omega_0}} 
	  \qquad\text{for all } k \geq 2.
\end{equation*} 
An explicit computation, which we leave to the reader, shows that these approximations are indeed correct up to $\cO(\epsilon^2)$. 
In particular, $A^{\dagger} = \left[ DF( \bx_\epsilon ) \right]^{-1} + \cO(\epsilon^2)$.
In Appendix~\ref{sec:OperatorNorms} several additional properties of these operators are derived. The most important one is the following.
% \note[J]{I've tried to make this change for the new injectivity bound throughout.} \note[JB]{Seems fine, but wouldn't it be nicer to write $\tfrac{\sqrt{10}}{4}$ instead of $\tfrac{5}{2 \sqrt{10}}$?} \note[J]{Yes it would, made changes below. }
\begin{proposition}
	\label{prop:Injective}
	For 
%\change[J]{$0 \leq \epsilon < \tfrac{5}{2} ( 4 + \sqrt{10})^{-1} \approx 0.349$}
	$0 \leq \epsilon < \tfrac{\sqrt{10}}{4} \approx 0.790$
	 the operator $ A^{\dagger}$ is injective. 
\end{proposition}
\begin{proof}
	In order to show that $ A^{\dagger}$ is injective we show that 
	it has a left inverse. 
	Note that $ A A^{\dagger} = I - \epsilon^2 ( A_1 A_0^{-1})^2$. 
	By Proposition \ref{prop:A1A0} it follows that 
%	\change[J]{$ \| A_1 A_0^{-1} \| \leq \tfrac{2}{5} ( 4 + \sqrt{10})$}
	 $ \| A_1 A_0^{-1} \| \leq \tfrac{2 \sqrt{10}}{5} $.  
	By choosing 
%\change[J]{$ \epsilon < \tfrac{5}{2} ( 4 + \sqrt{10})^{-1}$}
$ \epsilon < \tfrac{\sqrt{10}}{4}$ 
we obtain 
	$\|  \epsilon^2 ( A_1 A_0^{-1})^2 \| < 1$, whereby $ A A^{\dagger}$ is 
	invertible, and so $ A^{\dagger}$ is injective. 
\end{proof}


\begin{definition}
We define the operator $ T: \R^2 \times \ell^K_0 \to \R^2 \times \ell^K_0 $ by
\begin{equation*}
	T(x) :=  x - A^{\dagger} F(x) ,
\end{equation*}
	where  $F$ is defined in Equation~\eqref{eq:FDefinition}  and $A^{\dagger}$ in Definition~\ref{def:A}.
	We note that $F$, $A^{\dagger}$ and $T$ depend on the parameter $\epsilon \geq 0$, although we suppress this in the notation.
\end{definition}

% \note[J]{When we got the better bound on $\|A_1 A_{0}^{-1}\|$, then $A^{\dagger}$ being injective ceased to be a bottleneck for doing a Hopf bifurcation. I don't think we'd lose much if we just delete this remark.  }
% \begin{remark}
% 	\label{r:Injective}
% \remove[J]{
% 	If $A^{\dagger}$ is injective, which is true for
% 	$0 \le \epsilon <  \tfrac{5}{2} ( 4 + \sqrt{10})^{-1}$ by Proposition 3.2, then the fixed points of $T$ correspond bijectively with the zeros of $F$.
% 	Since the periodic solution having $ \epsilon_0 = \tfrac{5}{2} ( 4 + \sqrt{10})^{-1}$ corresponds approximately to $\bar{\alpha}_{\epsilon_0} = \pp + 0.090$, above this value we cannot use the Newton-like operator $T$ to reliably study the SOPS to Wright's equation.
% 	Hence $ \alpha = \pp + 0.09$ represents an upper bound for doing an $\cO(\epsilon^2)$ Hopf bifurcation analysis.}
% \end{remark}
%


\subsection{Explicit contraction bounds}
\label{s:contraction}


The map $T$ is not continuous on all of $\R^2 \times \ell^K_0$,
since $ U_{\omega} c $ is not continuous in $\omega$.
While continuity is ``recovered'' for terms of the form $A^{\dagger} U_{\omega} c$,  this is not the case for the nonlinear part $ - \alpha \epsilon A^{\dagger} [ U_{\omega} c ] * c$.  
% The problem is that while in the $ U_{\omega} c$ term and that  $ \tfrac{\partial}{\partial \omega} U_{\omega} = - i K^{-1} U_{\omega}$.
% Since  the map $ A^{\dagger}$ is approximately $\tfrac{2 }{\pi i} K$, then the  $  A^{\dagger}  U_{\omega} c$ component of $ T$ is continuous in $\omega$.
%
%For any $ \omega_1, \omega_2\in \R$  then $ \| U_{\omega_1} - U_{\omega_2} \|  = 2$ when $ 2 \pi $ does not divide $ \omega_1 - \omega_2$.  
%This is not a problem for the $ A^{\dagger} ( i \omega K^{-1} + \alpha U_{\omega} ) c$ component of $T$; 
%the and then $ A^{\dagger } U_{\omega}$ is continuous in $\omega$.  
% However, since $ U_{\omega}$ is inside a convolution in the nonlinear part, this type of simplification cannot happen.
% 
We overcome this difficulty by fixing some $ \rho > 0$ and restricting the domain of $T$ to sets of the form 
%$\R^2 \times X_\rho$, where
\[
  \R^2 \times  \{ c \in \ell^K_0 : \|K^{-1} c\| \leq \rho \} = \R^2 \times \ell_\rho.
\]
Since we wish to center the domain of $T$ about the approximate solution~$\bx_\epsilon$, we introduce the following definition, which uses a triple of radii $r \in \R^3_+$, for which it will be convenient to use two different notations:
\[
  r = ( r_{\alpha } , r_{\omega} , r_c) = (r_1,r_2,r_3).
\]
\begin{definition}
	Fix   $ r \in \R^3_+$ and $ \rho > 0$ and let  $ \bx_\epsilon = ( \balpha_\epsilon , \bomega_\epsilon , \bc_\epsilon )$ be as defined in Definition~\ref{def:xepsilon}. 
    We define the $\rho$-ball $B_\epsilon(r,\rho) \subset \R^2 \times \ell^1_0$
    of radius $r$ centered at $\bx_\epsilon$ to be the set of points satisfying 
\begin{alignat*}{1}
	|  \alpha -\balpha_\epsilon | & \leq  r_\alpha  \\
	| \omega - \bomega_\epsilon  | & \leq  r_{\omega} \\
	\| c - \bc_\epsilon  \| & \leq r_c \\
	\|K^{-1} c\| & \leq  \rho .
\end{alignat*}
\end{definition}

We want to show that $T$ is a contraction map on some $\rho$-ball 
$B_\epsilon(r,\rho) \subset \R^2 \times \ell^1_0$ using a Newton-Kantorovich argument. 
This will require us to develop a bound on $DT$ using some norm on  $ X$.  
Unfortunately there is no natural choice of norm on the product space $ X$. 
Furthermore, it will not become apparent if one norm is better than another until after  significant calculation.  
For this reason, we use a notion of an ``upper bound'' which allows us to delay our choice of norm. 
We first introduce the operator $\zeta:  X  \to \R^3_{+}$
which consists of the norms of the three components:
\[
  \LL(x) :=   ( |\pi_\alpha x|, |\pi_\omega x|, \|\pi_c x\| )^T \in \R^3_{+}
  \qquad\text{for any } x \in X.
\]
% \note[JB]{Propose to revert to using single overlines for the upper bound; the double overlines I had introduced just look ridiculous to me now. There is a potential for confusion with $\bx_\epsilon$, but I can live with it.} \note[J]{I would also prefer using only one overline. }
\begin{definition}[upper bound]\label{def:upperbound}
We call $\upperbound{x} \in \R^3_+$ an upper bound on $x$ if $\LL(x) \leq \upperbound{x}$, where the inequality is interpreted componentwise in $\R^3$. 
Let $X'$ be a subspace of $X$ and let $X''$ be a subset of $X'$.   
An upper bound on a linear operator $A' : X' \to X $ over $X''$  is 
a $3 \times 3$ matrix $\upperbound{A'} \in \text{\textup{Mat}}(\R^3 , \R^3)$ such that
\[
   \LL(A' x ) \leq \upperbound{A'} \cdot \LL(x)  
     \qquad\text{for any }  x \in X'',
\]
where the inequality is again interpreted componentwise in $\R^3$. 
The notion of upper bound conveniently encapsulates bounds on the different components of the operator $A'$ on the product space $X$. Clearly the components of the matrix $\upperbound{A'}$ are nonnegative.


		% 	Let $ (\alpha , \omega , c) \in \R^2 \times \ell^1_0$ and $  \upperbound{x} = ( x_{ \alpha } , x_{\omega} , x_{c}) \in \R^3_+$.
		% 	Then $ \upperbound{x}$ is an \emph{upper bound} on $(\alpha , \omega , c)$ if $ | \alpha | \leq x_\alpha$ and $ | \omega | \leq x_{\omega}$ and $ \| c\|\leq x_{c}$.
		% Similarly, suppose that $ A' : X \to \R^2 \times \ell^1_0$ is a linear operator, defined on some domain $ X \subset \R^2 \times \ell^1_0$.
		% Then $ \upperbound{A'} \in Mat(\R^3 , \R^3)$ is an \emph{upper bound } on $ A'$  if $ \upperbound{A'} \cdot \upperbound{x} $ is an upper bound on $ A' x$ whenever $ \upperbound{x} \in \R^3_+$ is an upper bound on all $ x \in X$.
\end{definition}
		% The notion of upper bounds commutes with vector addition and matrix multiplication.
		% That is, if $\upperbound{x} , \upperbound{y} \in \R^3_+ $ are upper bounds on $ x,y \in \R^2 \times \ell^1_0$, then $ \upperbound{x} + \upperbound{y}$ is an upper bound on $x +y$.
		% Similarly, if we have two linear operators $A'$ and $A''$ with upper bounds
		% $\upperbound{A'}$ and $\upperbound{A''}$, respectively, then $ \upperbound{A'} \cdot \upperbound{A''}$ is an upper bound for $ A' \circ A''$.
		% Furthermore, if $\upperbound{A'} \in Mat(\R^3,\R^3)$ is an upper bound, then the entries of this matrix are necessarily non-negative.
For example, in Proposition \ref{prop:A0A1} we calculate an upper  bound on the map $A_0^{-1} A_1$.  
As for the domain of definition of $T$, in practice we use $X' = \R^2 \times  \ell^K_0  $ and  $X'' = \R^2 \times  \ell_\rho  $.
The subset $X''$ does not always affect the upper bound calculation (such as in Proposition \ref{prop:A0A1}). 
However, operators such as $U_{\omega} - U_{\omega_0}$ have upper bounds which contain $\rho$-terms (see for example Proposition \ref{prop:OmegaDerivatives}).

Using this terminology, we state a ``radii polynomial'' theorem, which allows us to check whether $T$ is a contraction map. This technique has been used frequently in a computer-assisted setting in the past decade. Early application include~\cite{daylessardmischaikow,lessardvandenberg}, while a previous implementation in the context of Wright's delay equation can be found in~\cite{lessard2010recent}. 
Although we use radii polynomials as well, our approach differs significantly from the computer-assisted setting mentioned above. 
While we do engage a computer (namely the Mathematica file~\cite{mathematicafile}) to optimize our quantitative results, the analysis is performed essentially in terms of pencil-and-paper mathematics (in particular, our operators do not involve any floating point numbers).
In our current setup we employ \emph{three} radii as a priori unknown variables,
which builds on an idea introduced in~\cite{vandenberg}.
We note that in most of the papers mentioned above the notation of $A$ and $A^\dagger$ is reversed compared to the current paper.

As preparation, the following lemma (of which the proof can be found in Appendix~\ref{sec:CompactDomain})  provides an explicit choice for $\rho$, as a function of $\epsilon$ and $r$, for which we have proper control on the image of $B_\epsilon(r,\rho)$ under $T$.
\begin{lemma}\label{lem:Crho}
For any $\epsilon \geq 0$ and $r \in\R^3_+$, let $C=C(\epsilon,r)$ be given  by Equation~\eqref{eq:RhoConstant}. 
If $C(\epsilon,r) >0$  then 
% Proposition~\ref{prop:DerivativeEndo} states that
\begin{equation}\label{e:Cepsr} 
  \| K^{-1} \pi_c  T(x) \| \leq \rho 
  \quad\text{whenever } x \in B_\epsilon(r,\rho) \text{ and } \rho \geq C(\epsilon,r).
\end{equation}
%%%%\marginpar{Jonathan: please fix appendix to reflect this (and define $C$ there)}
Moreover, $C(\epsilon,r)$ is nondecreasing in $\epsilon$ and $r$. 
\end{lemma}

\begin{proof}
See Proposition~\ref{prop:DerivativeEndo}.
\end{proof}


\begin{theorem}
	\label{thm:RadPoly}
	Let  
%\change[J]{$0 \leq \epsilon < \tfrac{5}{2} ( 4 + \sqrt{10})^{-1} $}
 $0 \leq \epsilon < \tfrac{\sqrt{10}}{4} $  
 and fix $r = (r_\alpha, r_\omega, r_c) \in \R^3_+$. Fix $\rho > 0$ such that $ \rho \geq C(\epsilon,r)$, as given by Lemma~\ref{lem:Crho}.
 % as in Proposition \ref{prop:DerivativeEndo} (REFORMULATE TO POINT TO THE PREPARATION ABOVE).
%
Suppose that $Y(\epsilon) $ is an upper bound on $ T(\bx_\epsilon) - \bx_\epsilon$ and $Z(\epsilon , r ,\rho) $ a (uniform) upper bound on $ DT(x) $ for all $ x \in B_\epsilon(r,\rho)$. 
Define the \emph{radii polynomials}
$P :\R^5_+ \to \R^3 $  by 
 \begin{equation}
 \label{eq:RadPolyDef}
  P(\epsilon,r,\rho) := Y(\epsilon) - \left[ I - Z( \epsilon,r,\rho) \right] \cdot r  \,  .
 \end{equation}
If each component of $P(\epsilon,r,\rho)$ is negative, then there is  a unique $\hat{x}_\epsilon \in B_\epsilon( r , \rho)$ such that $F(\hat{x}_\epsilon) =0$. 
\end{theorem}

\begin{proof}    
Let $r \in \R^3_+$ be a triple such that $P(\epsilon,r,\rho)<0$.
By Proposition \ref{prop:Injective}, if 
%\change[J]{$\epsilon < \tfrac{5}{2} ( 4 + \sqrt{10})^{-1} $}
$\epsilon <\tfrac{\sqrt{10}}{4} $
then $ A^{\dagger}$ is injective. 
Hence $ \hat{x}_{\epsilon} $ is a fixed point of $T$ if and only if $ F( \hat{x}_{\epsilon}) = 0$.  
In order to show  there is a unique fixed point $ \hat{x}_{\epsilon}$, we show that $T$ maps  $ B_{\epsilon}(r,\rho) $ into itself and that $ T $ is a contraction mapping. 

We first show that $T: B_\epsilon(r,\rho) \to B_\epsilon(r,\rho)$. 
Since $ \rho \geq C(\epsilon,r)$ then by Equation~\eqref{e:Cepsr} it follows that $ \| K^{-1} \pi_c T( x) \| \leq \rho$ for all $ x \in B_\epsilon(r,\rho)$.
In order to show that $T(x) \in B_\epsilon(r,\rho)$, it suffices to show that $ r=(r_\alpha , r_\omega, r_c)$ is an upper bound on $ T(x) - \bx_\epsilon$
for all $ x \in B_\epsilon(r,\rho)$.  
We decompose 
%by breaking $T(x) - \bx_\epsilon$ into two parts: 
\begin{equation}\label{e:Tsplit}
	T(x) - \bx_\epsilon = [T(\bx_\epsilon) -\bx_\epsilon] +
	[T(x) - T(\bx_\epsilon)],
\end{equation}
and estimate each part separately. Concerning the first term,
by assumption, $Y(\epsilon)$ is an upper bound on $T(\bx_\epsilon) - \bx_\epsilon$. 
%
Concerning the second term, we claim that $ Z(\epsilon,r,\rho) \cdot r$ is an upper bound on $T(x) - T(\bx_\epsilon)$.
Indeed, we have the following somewhat stronger bound: 
% if $ x,y\in B_\epsilon(r,\rho)$ and $\upperbound{\xi}$ is an upper bound on $y-x$, then
% $Z(\epsilon,r,\rho) \cdot \upperbound{\xi}$ is an upper bound on $T(y) - T(x)$,
% i.e.,
\begin{equation}\label{e:DTisboundedbyZ}
	\LL(T(y) - T(x)) \leq Z(\epsilon,r,\rho) \cdot \LL(y-x)
	\qquad\text{for all } x,y \in B_\epsilon(r,\rho) .
\end{equation}
The latter follows from the mean value theorem, since 
$T$ is continuously Fr\'echet differentiable on $B_\epsilon(r,\rho)$.
%
% MORE DETAILED ARGUMENT PROBABLY NOT NEEDED?
% \begin{equation}
% \label{eq:ZIntegrationBound}
% 	T( y) - T(x) \leq Z(\epsilon,r,\rho) \cdot \upperbound{\xi}
% \end{equation}
% Since $T$ is continuously Frechet differentiable on $B_\epsilon(r,\rho)$,
% then it follows that $T(y) - T(x) = \int_x^y DT( z) dz  $.
% %Since $ B_\epsilon(r,\rho)$ is convex, then $z= x+ s(y-x) \in B_\epsilon(r,\rho)$ for all $s \in [0,1]$.
% By assumption $Z(\epsilon,r,\rho)$ is an upper bound on $DT(z)$ for all $z \in B_\epsilon(r,\rho)$.
% If $ \upperbound{\xi} $ is an upper bound on $ y-x$, then we obtain the inequality $ T(y) - T(x) \leq \int_0^1 Z(\epsilon,r,\rho) \cdot \upperbound{\xi} \, ds$, from which Equation \ref{eq:ZIntegrationBound} follows.
%
Since $r$ is an upper bound on $x - \bx_\epsilon$ for all $ x \in B_\epsilon(r,\rho)$, we find, by using~\eqref{e:Tsplit}, that  
% have obtained that
% \begin{eqnarray}
% 	T(\bx_\epsilon) - \bx_\epsilon &=& \left[ T(\bx_\epsilon) - \bx_\epsilon \right] +
% 	\left[ T(x) - T(\bx_\epsilon) \right] \\
% 	&\leq& Y(\epsilon) + Z(\epsilon,r,\rho) \cdot r
% \end{eqnarray}
% By assumption each component of Equation \ref{eq:RadPolyDef} is negative, so
$Y(\epsilon) + Z(\epsilon,r,\rho) \cdot r \leq r$ (with the inequality, interpreted componentwise, following from $P(\epsilon,r,\rho)<0$) is an upper bound on $T(x) - \bx_\epsilon$
for all $ x \in B_\epsilon(r,\rho)$.  
%It follows that $r$ is an upper bound on $T(x) - \bx_\epsilon $. 
That is to  say, if all of the  radii polynomials are negative, 
then  $T$ maps $B_\epsilon(r,\rho) $ into itself.

To finish the proof we show that $T$ is a contraction mapping. 
We abbreviate $Z=Z(\epsilon,r,\rho)$ and  recall that $r=(r_\alpha,r_\omega,r_c)=(r_1,r_2,r_3) \in \R^3_+$
is such that $Z \cdot r < r$, hence for some $\kappa <1$ we have
\begin{equation}\label{e:defkappa}
  \frac{(Z \cdot r)_i}{r_i} \leq \kappa  \qquad\text{for } i=1,2,3.
\end{equation}

We now need to choose a norm on $X$. 
We define a norm $ \| \cdot \|_r$ on elements $x = (\alpha,\omega,c) \in X$
by
\[  
\| (\alpha, \omega, c) \|_r := \max 
\left\{  		  
	 \frac{|\alpha|}{r_\alpha},
	 \frac{|\omega|}{r_\omega},
	 \frac{\|c\|}{r_{c}} \right\} , 
\]
or
\[
  \|x\|_r = \max_{i=1,2,3} \frac{ \LL(x)_i}{r_i}
  \qquad \text{for all } x \in X.
\]
% We also introduce the compatible norm $ \| \cdot \|_{\tilde{r}}$ on $\R^3$ by
% $ \| (y_1, y_2, y_3) \|_{\tilde{r}} = \max_{i=1,2,3 }
% \{\frac{|y_i|}/{r_i} \}$, so that $\|x\|_r = \| \LL(x) \|_{\tilde{r}}$ for all $x \in X$.
%
By using the upper bound $Z$, we bound the Lipschitz constant of $T$ on $B_\epsilon(r, \rho)$ as follows:
\begin{alignat*}{1}
 \| T(y) - T(x) \|_r 
 % &= \|\LL(T(y) - T(x)) \|_{\tilde{r} } \\
    &= \max_{i=1,2,3} \frac{\LL(T(y) - T(x))_i} {r_i} \\
    &\leq  \max_{i=1,2,3}  \frac{(Z \cdot \LL(y-x))_i}{r_i} \\
    &\leq  \max_{i=1,2,3} \max_{j=1,2,3}\frac{\LL(y-x)_j}{r_j}  
				   \frac{(Z \cdot r )_i}{r_i} \\
    & = \| y-x \|_r \max_{i=1,2,3} \frac{(Z \cdot r )_i}{r_i} \\
    & \leq \kappa \| y-x \|_r,
\end{alignat*}
where we have used~\eqref{e:DTisboundedbyZ} and~\eqref{e:defkappa} with $\kappa<1$.
% \sup_{x,y \in B_\epsilon(r,\rho)} \frac{\|T(y) - T(x) \|_r}{\| y- x\|_r}
% \leq
% \sup_{ x,y \in B_\epsilon(r,\rho)}
% \frac{\left \|
% Z(\epsilon,r,\rho) \cdot \upperbound{\xi}
% \right\|_{\tilde{r}} }{  \| y -x\|_{r}} ,
% \]
% where $\upperbound{\xi}$ is any upper bound on $y-x$, as before.
% \marginpar{THIS IS STILL NOT ENTIRELY CLEAR!}
% If $u \in \R^3$ and $ \|u\|_{\tilde{r}} =1$, then $ \| Z \cdot u\|_{\tilde{r}}$ is maximized when $u=r$.
% Hence $ Lip(T) \leq \| Z(\epsilon,r,\rho) \cdot r \|_r$.
% Since all of the radii polynomials are negative, then $ Z \cdot r < r$component wise, thus proving that $ \|Z \cdot r\|_r <1$ and
Hence $T:B_{\epsilon}(r,\rho) \to B_{\epsilon}(r,\rho)$ is a contraction with respect to the $\| \cdot \|_r$ norm.

% We have thereby proved that  $T:B_{\epsilon}(r,\rho) \to B_{\epsilon}(r,\rho)$ is a contraction mapping.
Since $B_\epsilon(r,\rho)$ with this norm is a complete metric space, by the Banach fixed point theorem $T$~has a unique fixed point $ \hat{x}_\epsilon \in B_\epsilon(r,\rho)$. 
Since $A^\dagger$ is injective,  it follows that $ \hat{x}_\epsilon$   is the unique point in $B_\epsilon(r,\rho)$ for which $ F(\hat{x}_\epsilon) =0$. 
\end{proof}

\begin{remark}\label{r:boundDT}
Under the assumptions in Theorem~\ref{thm:RadPoly},
essentially the same calculation as in the proof above
leads to the estimate
\[
  \| DT(x) y \|_r \leq \kappa \|y\|_r 
  \qquad \text{for all } y \in \R^2 \times \ell^K_0 , 
  \, x \in B_\epsilon(r,\rho),
\]
where $\kappa := \max_{i=1,2,3} (Z\cdot r)_i / r_i$.
\end{remark}


In Appendix \ref{sec:YBoundingFunctions} and Appendix \ref{sec:BoundingFunctions} we construct explicit upper bounds 
$Y(\epsilon)$ and $ Z(\epsilon,r,\rho)$, respectively.  
These functions are constructed such that their components are (multivariate) polynomials in $\epsilon$, $r$ and $ \rho$ with nonnegative coefficients, hence they are increasing in these variables. 
This construction enables us to make use of the uniform contraction principle. 

\begin{corollary}\label{cor:eps0}
Let 
%\change[J]{$0 <\epsilon_0 < \tfrac{5}{2} ( 4 + \sqrt{10})^{-1} $}
 $0 < \epsilon_0 < \tfrac{\sqrt{10}}{4} $ 
and fix some $r = (r_\alpha, r_\omega, r_c) \in \R^3_+$.  
Fix $\rho > 0$ such that $ \rho \geq C(\epsilon_0,  r)$, as given by Lemma~\ref{lem:Crho}.
% as in Proposition \ref{prop:DerivativeEndo}. 
%
Let $Y(\epsilon)$ and $Z(\epsilon,r,\rho)$ be the upper bounds as given in  Propositions~\ref{prop:Ydef} and~\ref{prop:Zdef}. 
Let the radii polynomials $P$ be defined by Equation~\eqref{eq:RadPolyDef}.


If each component of  $P(\epsilon_0, r,\rho)$ is negative, 
then for all $ 0 \leq \epsilon \leq \epsilon_0$ there exists a unique $ \hat{x}_\epsilon \in B_\epsilon(  r , \rho)$ such that $ F(\hat{x}_\epsilon) =0$.  
The solution $\hat{x}_\epsilon$ depends smoothly on $\epsilon$.
\end{corollary}
\begin{proof} 
	Let $0 \leq  \epsilon \leq \epsilon_0$ be arbitrary.
	Because $\rho \geq C(\epsilon_0, r) \geq C(\epsilon, r)$ by Lemma~\ref{lem:Crho},
	Theorem~\ref{thm:RadPoly} implies that it suffices to show that $ P(\epsilon, r ,\rho) <0$. 	
Since  the bounds 
$Y(\epsilon)$ and $ Z(\epsilon,r,\rho)$ are monotonically increasing in their arguments, it follows that $ P(\epsilon,r,\rho) \leq P(\epsilon_0,r,\rho) <0$.  
Continuous and smooth dependence on $\epsilon$ of the fixed point follows from the uniform contraction principle (see for example~\cite{ChowHale}). 
\end{proof}


Given the upper bounds $ Y(\epsilon)$ and $ Z( \epsilon ,r , \rho)$, 
trying to apply Corollary~\ref{cor:eps0} amounts to finding values of $ \epsilon, r_\alpha, r_\omega, r_c,\rho$ for which the radii polynomials are negative.
Selecting a value for $ \rho$ is straightforward: all estimates improve with smaller values of $\rho$, and Proposition \ref{prop:DerivativeEndo} (see also Lemma~\ref{lem:Crho}) explicitly describes the smallest allowable choice of $\rho$ in terms of $ \epsilon,r_\alpha,r_\omega,r_c$. 

Beyond selecting a value for $ \rho$, it is difficult to pinpoint what constitutes an ``optimal'' choice of these variables. 
In general it is interesting to find such  viable radii (i.e.\ radii such that $P(r)<0$) which are both large and small.  
The smaller radius tells us how close the true solution is to our approximate solution. 
The larger radius tells us in how large a neighborhood our solution is unique.  With regard to $\epsilon$, larger values allow us to describe functions whose first Fourier mode is large. However this will ``grow'' the smallest viable radius and ``shrink'' the largest viable radius. 

Proposition \ref{prop:bigboxes} presents two selections of variables which satisfy the hypothesis of Corollary~\ref{cor:eps0}.  
We check the hypothesis is indeed satisfied by using interval arithmetic.
All details are provided in the Mathematica file~\cite{mathematicafile}. 
While the specific numbers used may appear to be somewhat arbitrary (see also the discussion in Remark~\ref{r:largeradii})  they have been chosen to be used later in Theorem 
\ref{thm:WrightConjecture} and Theorem \ref{thm:UniqunessNbd}.  


%%%
%%%BY DOING SOME CHOICES THAT HAVE NO MOTIVATION AT THIS POINT, BUT THAT WILL TURN OUT TO BE USEFUL IN SECTION~\ref{s:global} WE PROVE THE FOLLOWING USING MATHEMATICA  FILES.\marginpar{todo}

\begin{proposition}
		\label{prop:bigboxes}
Fix the constants $ \epsilon_0$, $(r_\alpha, r_\omega,r_c)$  and $\rho$ according to one of the following choices:
% \begin{enumerate}
% \item[\textup{(a)}]  $ \epsilon_0 = 0.029 $ and $ (r_\alpha , r_ \omega , r_c) = (  0.21, \, 0.16 , \, 0.09 ) $ and $\rho = 1.01$;
% \item[\textup{(b)}]  $ \epsilon_0 = 0.087 $ and $ (r_\alpha , r_ \omega , r_c) = (  0.1501, \, 0.0626 , \, 0.2092 ) $ and $\rho = 0.5672$.
% \end{enumerate}
% \note[J]{Version with new numbers below}
\begin{enumerate}
	\item[\textup{(a)}]  $ \epsilon_0 = 0.029 $ and $ (r_\alpha , r_ \omega , r_c) = (  0.13, \, 0.17 , \, 0.17 ) $ and $\rho = 1.78$; 
	\item[\textup{(b)}]  $ \epsilon_0 = 0.09 $ and $ (r_\alpha , r_ \omega , r_c) = (  0.1753, \, 0.0941 , \, 0.3829 ) $ and $\rho = 1.5940$. 
\end{enumerate}
For either of the choices (a) and (b) we have the following: 
for all $0 \leq \epsilon \leq \epsilon_0$ there exists a unique point 
$(\hat{\alpha}_\epsilon,\hat{\omega}_\epsilon,\hat{c}_\epsilon) \in B_{\epsilon}(r,\rho)$ 
satisfying $F_\epsilon(\hat{\alpha}_\epsilon,\hat{\omega}_\epsilon,\hat{c}_\epsilon) = 0$ and 
\[ 	
 | \hat{\alpha}_\epsilon - \balpha_\epsilon| \leq r_\alpha , 
 \quad
 |\hat{\omega}_\epsilon - \bomega_\epsilon| \leq  r_\omega  ,
 \quad
 \| \hat{c}_\epsilon - \bc_\epsilon\| \leq r_c     ,
 \quad
 \| K^{-1} \hat{c}_\epsilon \| \leq \rho  .
\]
\end{proposition}
\begin{proof}
In the Mathematica file~\cite{mathematicafile}  we check, using interval arithmetic, that  $\rho \geq C(\epsilon_0, r)$ and  the radii polynomials $P(\epsilon_0,r,\rho)$ are negative for the choices (a) and (b). The result then follows from Corollary~\ref{cor:eps0}.	
\end{proof}


\begin{remark}\label{r:largeradii}	
In Proposition~\ref{prop:bigboxes} we aimed for large balls on which the solution is unique.
Even for a fixed value of $ \epsilon$, it is not immediately obvious how to find a ``largest'' viable radius $r$, 
since $r$ has three components. In particular, there is a trade-off between the different components of $r$. On the other hand, as explained in Remark~\ref{r:smallradii}, no such difficulty arises when looking for a ``smallest'' viable radius.
\end{remark}




We will also need a rescaled version of the radii polynomials, which takes into account the asymptotic behavior of the bound $Y$ on the residue $T(\bar{x}_\epsilon) -\bar{x}_\epsilon = - A^\dagger F(\bx_\epsilon)$  as $\epsilon \to 0$, namely it is of the form
$Y(\epsilon)= \epsilon^2 \tilde{Y}(\epsilon)$,
see Proposition~\ref{prop:Ydef}.
The proofs of the following monotonicity properties can be found in 
Appendices~\ref{sec:YBoundingFunctions} and~\ref{sec:BoundingFunctions}. 
\begin{lemma}\label{lem:YZ}
Let $\epsilon \geq 0$, $\rho >0$ 
and $r \in\R^3_+$. 
Then there are upper bounds
$Y(\epsilon) =\epsilon^2 \tilde{Y}(\epsilon)$ on $ T(\bx_\epsilon) - \bx_\epsilon$ and a (uniform) upper bound 
$Z(\epsilon , r ,\rho) $  on $ DT(x) $ for all $ x \in B_\epsilon(r,\rho)$.
These bounds are given explicitly by Propositions~\ref{prop:Ydef} and~\ref{prop:Zdef}, respectively. Moreover, $\tilde{Y}(\epsilon)$ is nondecreasing in $\epsilon$,
while $Z(\epsilon , r ,\rho) $ is nondecreasing in $\epsilon$, $r$ and $\rho$.
\end{lemma}

This implies, roughly speaking, that if we are able to show that $T$ is a contraction map on 
$B_{\epsilon_0}( \epsilon_0^2 \rr,\rho)$ for a particular choice of $ \epsilon_0$, then it will be a contraction map on $B_\epsilon( \epsilon^2 \rr,\rho)$ for all $ 0 \leq \epsilon \leq \epsilon_0$. Here, and in what follows, we use the notation $r = \epsilon^2 \rr$ for the $\epsilon$-scaled version of the radii. 



\begin{corollary}
	\label{cor:RPUniformEpsilon}
	Let  
	 $0 < \epsilon_0 < \tfrac{\sqrt{10}}{4} $ 
	and fix some $\rr = (\rr_\alpha, \rr_\omega, \rr_c) \in \R^3_+$. 
	Fix $\rho > 0$ such that $ \rho \geq C(\epsilon_0, \epsilon_0^2 \rr)$, as given by Lemma~\ref{lem:Crho}. 
	Let $Y(\epsilon)$ and $Z(\epsilon,r,\rho)$ be the upper bounds as given by Lemma~\ref{lem:YZ}.  
Let the radii polynomials $P$ be defined by~\eqref{eq:RadPolyDef}. 

	If each component of  $P(\epsilon_0,\epsilon_0^2 \rr,\rho)$ is negative, 
	then for all $ 0 \leq \epsilon \leq \epsilon_0$ 
	there exists a unique $ \hat{x}_\epsilon \in B_\epsilon(\epsilon^2  \rr , \rho)$ 
	such that $ F(\hat{x}_\epsilon) =0$. 
	Furthermore, $\hat{x}_\epsilon$ depends smoothly on $\epsilon$.
\end{corollary}

\begin{proof}
	 Let $0 \leq  \epsilon < \epsilon_0$ be arbitrary.
	 Because $\rho \geq C(\epsilon_0,\epsilon_0^2 \rr) \geq C(\epsilon,\epsilon^2 \rr)$ by Lemma~\ref{lem:Crho},
	Theorem~\ref{thm:RadPoly} implies that it suffices to show that $ P(\epsilon,\epsilon^2 \rr ,\rho) <0$. 
	By using the monotonicity provided by Lemma~\ref{lem:YZ}, we obtain
	\begin{alignat*}{1}
		P(\epsilon,\epsilon^2 \rr ,\rho) &= Y(\epsilon) 
- \left[ I - Z(\epsilon,\epsilon^2 \rr,\rho)\right] \cdot \epsilon^2 \rr \\
		&=  (\epsilon / \epsilon_0)^{2} \left[ \epsilon_0^2   
		  \tilde{Y}(\epsilon) - \epsilon_0^2 \rr 
		+  Z(\epsilon,\epsilon^2 \rr,\rho) \cdot \epsilon_0^2 \rr  \right] \\
		&\leq  (\epsilon / \epsilon_0)^{2} \left[ \epsilon_0^2  
		  \tilde{Y}(\epsilon_0)  - \epsilon_0^2 \rr 
   +  Z(\epsilon_0,\epsilon_0^2 \rr,\rho) \cdot \epsilon_0^2 \rr  \right] \\
		&= (\epsilon / \epsilon_0)^{ 2} P(\epsilon_0 , \epsilon_0^2 \rr,\rho) \\
		& < 0,
	\end{alignat*}
where inequalities are interpreted componentwise in $\R^3$, as usual.
\end{proof}




%%%%%
%%%%%		THIS IS THE OLD VERSION OF THE UNIFORM \EPSILON^2 THEOREM
%%%%%
%%%%%\begin{corollary}
%%%%%	\label{prop:RPUniformEpsilon}
%%%%%	Let $ 0 < \epsilon_0 < \tfrac{5}{2} ( 4 + \sqrt{10})^{-1}$ and fix some $r = (r_\alpha, r_\omega, r_c) \in \R^3_+$ and 
%%%%%	fix $ k \in \{ 0,1,2\}$.  
%%%%%	Fix $\rho > 0$ such that $ \rho \geq C(\epsilon_0, (\epsilon_0)^2 r)$, as given by Lemma~\ref{lem:Crho}. 
%%%%%	Let $Y(\epsilon)$ and $Z(\epsilon,r,\rho)$ be the upper bounds as given by~\ref{lem:YZ}.  
%%%%%	Let the radii polynomials $P$ be defined by~\eqref{eq:RadPolyDef}.
%%%%%	If each component of  $P(\epsilon_0,{\epsilon_0}^k r,\rho)$ is negative, 
%%%%%	then for all $ 0 \leq \epsilon \leq \epsilon_0$ there exists a unique $ \hat{x}_\epsilon \in B_\epsilon(\epsilon^k  r , \rho)$ such that $ F(\hat{x}_\epsilon) =0$. Furthermore, $\hat{x}_\epsilon$ depends smoothly on $\epsilon$.
%%%%%\end{corollary}
%%%%%
%%%%%\begin{proof}
%%%%%	Let $0 \leq  \epsilon < \epsilon_0$ be arbitrary.
%%%%%	Because $\rho \geq C(\epsilon_0,\epsilon_0^k r) \geq C(\epsilon_0,\epsilon_0^k r)$ by Lemma~\ref{lem:Crho},
%%%%%	Theorem~\ref{thm:RadPoly} implies that it suffices to show that $ P(\epsilon,\epsilon^k r ,\rho) <0$. 
%%%%%	By using the monotonicity provided by Lemma~\ref{lem:YZ}, we obtain
%%%%%	\begin{alignat*}{1}
%%%%%	P(\epsilon,\epsilon^k r ,\rho) &= Y(\epsilon) - \left[ I - Z(\epsilon,\epsilon^k r,\rho)\right] \cdot \epsilon^k r \\
%%%%%	&=  (\epsilon / \epsilon_0)^{k} \left[ \epsilon_0^k  \epsilon^{2-k}  \tilde{Y}(\epsilon) - \epsilon_0^k r +  Z(\epsilon,\epsilon^k r,\rho) \cdot \epsilon_0^k r  \right] \\
%%%%%	&\leq  (\epsilon / \epsilon_0)^{k} \left[ \epsilon_0^k  \epsilon_0^{2-k}  \tilde{Y}(\epsilon_0)  - \epsilon_0^k r +  Z(\epsilon_0,\epsilon_0^k r,\rho) \cdot \epsilon_0^k r  \right] \\
%%%%%	&= (\epsilon / \epsilon_0)^{ k} P(\epsilon_0 , \epsilon_0^k r,\rho) \\
%%%%%	& < 0,
%%%%%	\end{alignat*}
%%%%%	where inequalities are interpreted componentwise in $\R^3$, as usual.
%%%%%\end{proof}
%%%%%




%%%%%%%%%%%%%%%%%%%%%%%%%%%%%%%%%%%%%%%%%%%%%%%%%%%%%%%%%%%%%%%%%%%%%%%%%%%%
%\subsection{Application of Radii Polynomials}

%\begin{remark}


These $\epsilon$-rescaled variables are used in
Proposition~\ref{prop:TightEstimate} below to derive \emph{tight} bounds on the
solution (in particular, tight enough to conclude that the bifurcation is
supercritical). The following remark explains that the monotonicity properties of
the bounds $Y$ and $Z$ imply that looking for small(est) radii which satisfy $P(r)<0$, is
a well-defined problem.


\begin{remark}\label{r:smallradii}
The set $R$ of radii for which the radii polynomials are negative is given by 
\[
  R := \{ r \in \R^3_+ : r_j > 0,  P_i(r) < 0 \text{ for } i,j=1,2,3 \} .
\] 
This set has the property that if
	$r,r' \in R$, then $r''\in R$, where $r''_j=\min\{ r_j,r'_j\}$.
Namely, the main observation is that we can write 
	$P_i(r)= \tilde{P}_i(r)-r_i$, where $\partial_{r_j} \tilde{P}_i \geq 0$ for all $i,j=1,2,3$.
Now fix any $i$; we want to show that $P_i(r'') < 0$.	
We have either $r''_i=r_i$ or $r''_i=r'_i$, hence assume $r''_i=r_i$ (otherwise
just exchange the roles of $r$ and $r'$). We infer that $P_i(r'') \leq P_i(r) <
0$, since $\partial_{r_j} P_i \geq 0$ for $j \neq i$.
We conclude that there are no trade-offs in looking for minimal/tight radii, as
opposed to looking for large radii, see Remark~\ref{r:largeradii}.
\end{remark}

%%%
%%%The optimization problem is simplified to a degree because the region $ P(\epsilon_0,r,\rho_0) <0$ is convex for fixed $\epsilon_0$ and $ \rho_0 $.  
%%%This is because the function $Z(\epsilon,r,\rho)$ is constructed out of polynomials with non-negative coefficients, whereby $\tfrac{\partial}{ \partial r_i} \tfrac{\partial }{\partial r_j} P_{r_k}(\epsilon_0,r,\rho_0) >0$ for all $ i,j,k \in \{ \alpha, \omega,c\}$. 
%%%\marginpar{I believe this is true, right? -JJ}


\begin{proposition}
		\label{prop:TightEstimate}
	Fix $ \epsilon_0 = 0.10$ and 
%\change[J]{$ (\rr_\alpha , \rr_ \omega , \rr_c) = (  0.1149, \, 0.0470 , \, 0.4711 ) $}
$ (\rr_\alpha , \rr_ \omega , \rr_c) = (  0.0594, \, 0.0260 , \, 0.4929 ) $ 
and 
%\change[J]{$\rho = 0.0279$}
$\rho = 0.3191$. 
	For all $0< \epsilon \leq \epsilon_0$ there exists a unique point $\hat{x}_\epsilon = (\hat{\alpha}_\epsilon,\hat{\omega}_\epsilon,\hat{c}_\epsilon)$ 
	satisfying $F(\hat{x}_\epsilon) = 0$ and 
	\begin{align}
	\label{eq:TightBound}
 | \hat{\alpha}_\epsilon - \balpha_\epsilon| <& \rr_\alpha \epsilon^2 , 
 %
 &|\hat{\omega}_\epsilon - \bomega_\epsilon| <&  \rr_\omega \epsilon^2 ,
 %
 &
 \| \hat{c}_\epsilon - \bc_\epsilon\| <& \rr_c  \epsilon^2   ,
  %
  &
  \| K^{-1} \hat{c}_\epsilon \| <& \rho  .
	\end{align}
Furthermore, $\hat{\alpha}_\epsilon > \pp$ for $ 0 < \epsilon < \epsilon_0$.
\end{proposition}

\begin{proof}
	In the Mathematica file~\cite{mathematicafile}  we check, using interval arithmetic, that  $\rho \geq C(\epsilon_0, \epsilon_0^2 \rr)$ and  the radii polynomials $P(\epsilon_0,\epsilon_0^2 \rr,\rho)$ are negative.  
	%I DO NOT UNDERSTAND THE NEXT SENTENCE
 The inequalities in Equation~\eqref{eq:TightBound} follow from Corollary~\ref{cor:RPUniformEpsilon}. 
 Since $\hat{\alpha}_\epsilon \geq \balpha_\epsilon - \rr_\alpha \epsilon^2
 = \pp +\frac{1}{5}(\frac{3\pi}{2}-1)\epsilon^2 - \rr_\alpha \epsilon^2$ and $ \rr_\alpha < \tfrac{1}{5} ( \tfrac{3 \pi}{2} -1) $, it follows that $ \hat{\alpha}_\epsilon > \pp $ for all $ 0 < \epsilon \leq \epsilon_0$. 
%%
%%STILL NEEDS AN EXPLANATION
%%\marginpar{Jonathan: I am not sure what the argument is \dots}
%% WHY IT IS UNIQUE IN THE BALL GIVEN BY~\eqref{eq:TightBound}. CLEARLY IT IS UNIQUE IN $B_\epsilon(r,\rho)$
%%WITH $\rho= C( \epsilon_0,\epsilon_0^2 r)$. WHY CAN THERE BE NO SOLUTIONS WITH
%%$\| K^{-1} c \| > \rho$ SATISFYING~\eqref{eq:TightBound} ? 
\end{proof}

\begin{remark}\label{r:nested}
% The pivotal result in Proposition~\ref{prop:TightEstimate} is that $\hat{\alpha}_\epsilon > \pp$, which implies that the bifurcation is subcritical.
Since $\epsilon_0^2\rr < r$ for the choices (a) and (b) in Proposition~\ref{prop:bigboxes},
and the choices of $\rho$ and $\epsilon_0$ are compatible as well, the solutions found in Proposition~\ref{prop:bigboxes} are the same as those described by Proposition~\ref{prop:TightEstimate}. While the former proposition provides large isolation/uniqueness neighborhoods for the solutions,
the latter provides tight bounds and confirms the  supercriticality of the bifurcation suggested in Definition \ref{def:xepsilon}.

% The bifurcation is supercritical (see eg.  \cite{faria2006normal} p 252
	
	
	
%		We note that for each (appropriate) $\epsilon$, the ball 
%	$ B_{\epsilon}(r,\rho)$ from Proposition \ref{prop:TightEstimate} is contained within the balls   
%	$ B_{\epsilon}(r_a,\rho_a)$  and 
%	$ B_{\epsilon}(r_b,\rho_b)$ from Proposition \ref{prop:bigboxes}. 
%	This means that the fixed points $  \hat{x}_{\epsilon} \in B_{\epsilon}(r,\rho)$ is the same fixed point $\hat{x}_{\epsilon} \in B_{\epsilon}(r_a,\rho_a)$ .

\end{remark}


%
%The method of radii polynomials is versatile. 
%With the goal of later proving Corollary \ref{prop:UniqunessNbd}, we added additional constraints to \emph{Mathematica}'s function \emph{NMaximize} to find the parameters for Proposition \ref{prop:WideEstimate}.
%
%When searching for the largest viable radius we add an additional constraint. 
%In Proposition \ref{prop:Cone}, we showed that for a given selection of $ \epsilon$, $r_\alpha$ and $ r_\omega$, then the unscaled variable $ \|c\|$ is either $\cO(1)$ or $\cO(\epsilon^2)$. 
%When we scale $c \to \epsilon c$, then we are only able to prove uniqueness of our solution in an $ \epsilon-$cone about the approximate solution. 
%We use this Proposition to select $r_c = ????$ in terms of $\epsilon$, $r_\alpha$ and $r_\omega$ so that any unscaled  solution $c$ is either $\cO(1)$ or $ c \in B_{\epsilon}(r,\rho)$.
%
%Even still, the larger we choose $ \epsilon$, the smaller we will need to take $ r_\omega$ in order to have a proof. 
%For the following theorem, we fixed $ \epsilon_0 =0.085$ and used \emph{NMaximize} to find a choice of variables $(\epsilon,r_\alpha,r_\omega,r_c)$  which maximized the objective function $r_w$ and for which all the radii polynomials were negative. 
%By slightly shrinking the estimate for the optimal radii, we obtain the following theorem.

}
  {}

% The following \documentclass options may be useful:
% preprint      Remove this option only once the paper is in final form.
% 10pt          To set in 10-point type instead of 9-point.
% 11pt          To set in 11-point type instead of 9-point.
% numbers       To obtain numeric citation style instead of author/year.
\ifanon
%% For double-blind review submission
\documentclass[acmsmall,review,anonymous]{acmart}\settopmatter{printfolios=true}
\else
%% For technical report and camera ready
\documentclass[format=acmsmall, review=false, screen=true]{acmart}\settopmatter{printfolios=true}
\fi

\setcopyright{rightsretained}
\acmJournal{PACMPL}
\acmYear{2017}
\acmVolume{1}
\acmNumber{ICFP}
\acmArticle{17}
\acmMonth{9}
\acmDOI{10.1145/3110261}
\acmPrice{}

\ifcamera\else
\makeatletter
\@ACM@printacmreffalse
\@ACM@printccsfalse
\def\@copyrightpermission{This work is licensed under a \href{https://creativecommons.org/licenses/by/4.0/}{Creative Commons Attribution 4.0 International License}}
\fancypagestyle{firstpagestyle}{%
  \fancyhf{}%
  \renewcommand{\headrulewidth}{\z@}%
  \renewcommand{\footrulewidth}{\z@}%
    \fancyhead[LE]{\ACM@linecountL\@headfootfont\thepage}%
    \fancyhead[RO]{\@headfootfont\thepage}%
    \fancyhead[RE]{\@headfootfont\@shortauthors}%
    \fancyhead[LO]{}%
    \fancyfoot[RO,LE]{}%
}
\fancypagestyle{standardpagestyle}{%
  \fancyhf{}%
  \renewcommand{\headrulewidth}{\z@}%
  \renewcommand{\footrulewidth}{\z@}%
    \fancyhead[LE]{\ACM@linecountL\@headfootfont\thepage}%
    \fancyhead[RO]{\@headfootfont\thepage}%
    \fancyhead[RE]{\@headfootfont\@shortauthors}%
    \fancyhead[LO]{\ACM@linecountL\@headfootfont\shorttitle}%
    \fancyfoot[RO,LE]{}%
}
\pagestyle{standardpagestyle}
\def\@mkbibcitation{}
% \ifcamera\else, manuscript\fi -- didn't set this for documentclass since that changes formatting
\@ACM@manuscripttrue % bit like \ifcamera\else, authorversion\fi

% arXiv screws up the title boldness, so need to convince it to do it right
\def\@titlefont{\sffamily\LARGE\bfseries}

\makeatother
\fi


%% Bibliography style
\bibliographystyle{ACM-Reference-Format}
%% Citation style
%% Note: author/year citations are required for papers published as an
%% issue of PACMPL.
\citestyle{acmauthoryear}   %% For author/year citations

% Because of the diacritic in JK's name
\usepackage[utf8]{inputenc}
\usepackage[T1]{fontenc}
\usepackage{amsmath}
\usepackage{amssymb}
\usepackage{stmaryrd}
\usepackage{amstext}
\usepackage{MnSymbol}
\usepackage{amsthm}
\usepackage{hyperref}
%% \usepackage{microtype}
\usepackage{balance}
\usepackage{upquote} %for a proper backtick `
% \usepackage{mathabx}
\usepackage{verbatim}

\usepackage{mathpartir}
\usepackage{tabularx}
  % A column which takes as much width as is available _and_ aligns right.
  \newcolumntype{R}{>{\raggedleft\arraybackslash}X}%
  \newcolumntype{Y}{>{\centering\arraybackslash}X}%

\usepackage{xparse}
\usepackage{enumerate}
\usepackage{microtype}

\hyphenation{as-surance}

\usepackage{tikz}
  \usetikzlibrary{calc}
  \usetikzlibrary{positioning}
  \usetikzlibrary{tikzmark}
  \usetikzlibrary{shapes.geometric}
  \usetikzlibrary{decorations.text}
  \usetikzlibrary{backgrounds}  % drawing the background after the foreground

% \pagelimitstrue\draftfalse  %% -- Turn this on to check your page limits
% PLDI CFP: Appendices should not be part of the paper,
%           but should be submitted as supplementary material.

\usepackage{listings}
  \newcommand\function{\sffamily\maybecolor{dkblue}function}

\makeatletter
\lstdefinelanguage{fstar}{%
  basicstyle=\lst@ifdisplaystyle\small\fi,
  morekeywords=[1]{type,and,val,fun,let,in,ref,try,if,then,else,match,with,open,as,module,rec,begin,end,assume,private,when,forall,ghost,assert,function,logic,array,pattern,effect,requires,ensures,decreases,modifies, abstract,total, reifiable, reflectable, new_effect, effect_actions, noeq, sub_effect},
  morekeywords=[2]{WP, Post, Pre, PURE, DIV, STATE, EXN, ALL, TotST, M, GHOST, Ghost, GTot, Lemma, Tot, Memo, READER, STATE_m, Effect, STEXN, RANDOM, ST, St, RAND, Rand},
  morekeywords=[3]{},
  morekeywords=[4]{},
  morestring=[b]",
  sensitive=true,%
  numbersep=4pt,
%  numbers=left,
  columns=[l]fullflexible,
  texcl=true,
  mathescape=true,
%  xleftmargin=10pt,
  identifierstyle={\sffamily},
  keywordstyle=[1]{\sffamily\maybecolor{dkblue}},
  keywordstyle=[2]{\sffamily\maybecolor{dkblue}},
  keywordstyle=[3]{\maybecolor{dkred}},
  keywordstyle=[4]{\rmfamily\itshape},
% Here is the range marker stuff
  rangeprefix=(*---\ ,
  includerangemarker=false,
  stringstyle=\ttfamily,
  showspaces=false,
  morecomment=[n]{(*}{*)},
  commentstyle={\itshape\maybecolor{dkred}},
  literate={->}{$\rightarrow\,$}{1}
           {True}{$\top$}{1}
           {False}{$\bot$}{1}
	   {delta}{$\delta$}{1}
	   {exists}{$\exists$}{1}
	   {forall}{$\forall$}{1}
           {fun}{$\lambda$}{1}
           {function}{{\sffamily\maybecolor{dkblue}function}}{1} % just putting function as a keyword is not enough because of fun
           {~}{$\neg$}{1}           
           {~>}{$\leadsto$}{1}
           {`}{\`{}}{1}
           {'a}{$\alpha$}{1}
           {'b}{$\beta$}{1}
           {'c}{$\gamma$}{1}
           {tau}{$\tau\,$}{1}
           {STATE0}{$\mathsf{ST}^\prime\,$}{1}
	   {/\\}{$\wedge\;$}{1}
	   {\\/}{$\vee\;$}{1}
           {>=}{$\geq\ $}{2}
           {<=}{$\leq\ $}{2}
	   {<==>}{$\Longleftrightarrow \ $}{3}
	   {==>}{$\Longrightarrow \ $}{3}
     {<>}{$\neq$}{1}
     {--}{-}{1}   % Added due to         - not displaying properly.  Weird. ... but may make arrows look bad  {...}{$\ldots$}{3}
     {<<}{$\prec\ $}{1}
           ,
  breaklines=false}
\makeatother

  \lstset{language=fstar,numbers=left,escapechar=!,basicstyle=\footnotesize}
  \let\lst\lstinline
  \let\li\lstinline

% \ifjonathan
%   \usepackage{minted}
%   \usemintedstyle{tango}
%   \newminted[ocaml]{ocaml}{mathescape,fontsize=\footnotesize,linenos,escapeinside=!!}
%   \newminted[cpp]{cpp}{mathescape,fontsize=\footnotesize,linenos,escapeinside=!!}
% \else
%   \usepackage{fancyvrb}
%   \DefineVerbatimEnvironment%
%     {cpp}{Verbatim}
%     {numbers=left,fontsize=\footnotesize}
%   \DefineVerbatimEnvironment%
%     {ocaml}{Verbatim}
%     {numbers=left,fontsize=\footnotesize}
% \fi

%% \newtheorem{lemma}{Lemma}
\newtheorem{remark}{Remark}
%% \newtheorem{definition}{Definition}
%% \newtheorem{theorem}{Theorem}
%% \newtheorem{conjecture}{Conjecture}
%% \newtheorem{corollary}{Corollary}
%TODO
\newcommand{\todo}[1]{{\color{red}{\bf [TODO]:~{#1}}}}

%THEOREMS
\newtheorem{theorem}{Theorem}
\newtheorem{corollary}{Corollary}
\newtheorem{lemma}{Lemma}
\newtheorem{proposition}{Proposition}
\newtheorem{problem}{Problem}
\newtheorem{definition}{Definition}
\newtheorem{remark}{Remark}
\newtheorem{example}{Example}
\newtheorem{assumption}{Assumption}

%HANS' CONVENIENCES
\newcommand{\define}[1]{\textit{#1}}
\newcommand{\join}{\vee}
\newcommand{\meet}{\wedge}
\newcommand{\bigjoin}{\bigvee}
\newcommand{\bigmeet}{\bigwedge}
\newcommand{\jointimes}{\boxplus}
\newcommand{\meettimes}{\boxplus'}
\newcommand{\bigjoinplus}{\bigjoin}
\newcommand{\bigmeetplus}{\bigmeet}
\newcommand{\joinplus}{\join}
\newcommand{\meetplus}{\meet}
\newcommand{\lattice}[1]{\mathbf{#1}}
\newcommand{\semimod}{\mathcal{S}}
\newcommand{\graph}{\mathcal{G}}
\newcommand{\nodes}{\mathcal{V}}
\newcommand{\agents}{\{1,2,\dots,N\}}
\newcommand{\edges}{\mathcal{E}}
\newcommand{\neighbors}{\mathcal{N}}
\newcommand{\Weights}{\mathcal{A}}
\renewcommand{\leq}{\leqslant}
\renewcommand{\geq}{\geqslant}
\renewcommand{\preceq}{\preccurlyeq}
\renewcommand{\succeq}{\succcurlyeq}
\newcommand{\Rmax}{\mathbb{R}_{\mathrm{max}}}
\newcommand{\Rmin}{\mathbb{R}_{\mathrm{min}}}
\newcommand{\Rext}{\overline{\mathbb{R}}}
\newcommand{\R}{\mathbb{R}}
\newcommand{\N}{\mathbb{N}}
\newcommand{\A}{\mathbf{A}}
\newcommand{\B}{\mathbf{B}}
\newcommand{\x}{\mathbf{x}}
\newcommand{\e}{\mathbf{e}}
\newcommand{\X}{\mathbf{X}}
\newcommand{\W}{\mathbf{W}}
\newcommand{\weights}{\mathcal{W}}
\newcommand{\alternatives}{\mathcal{X}}
\newcommand{\xsol}{\bar{\mathbf{x}}}
\newcommand{\y}{\mathbf{y}}
\newcommand{\Y}{\mathbf{Y}}
\newcommand{\z}{\mathbf{z}}
\newcommand{\Z}{\mathbf{Z}}
\renewcommand{\a}{\mathbf{a}}
\renewcommand{\b}{\mathbf{b}}
\newcommand{\I}{\mathbf{I}}
\DeclareMathOperator{\supp}{supp}
\newcommand{\Par}[2]{\mathcal{P}_{{#1} \to {#2}}}
\newcommand{\Laplacian}{\mathcal{L}}
\newcommand{\F}{\mathcal{F}}
\newcommand{\inv}[1]{{#1}^{\sharp}}
\newcommand{\energy}{Q}
\newcommand{\err}{\mathrm{err}}
\newcommand{\argmin}{\mathrm{argmin}}
\newcommand{\argmax}{\mathrm{argmax}}
\usepackage{float}
  \floatstyle{plaintop}
  \restylefloat{table}

% \geometry{head=23pt, headsep=6pt}

\begin{document}

% \acmBadgeR[https://github.com/project-everest/everest]{artifact_evaluated-reusable.png}
% \acmBadgeL[https://github.com/project-everest/everest]{artifact_available.png}

\special{papersize=8.5in,11in}
\setlength{\pdfpageheight}{\paperheight}
\setlength{\pdfpagewidth}{\paperwidth}

%% \conferenceinfo{CONF 'yy}{Month d--d, 20yy, City, ST, Country}
%% \copyrightyear{20yy}
%% \copyrightdata{978-1-nnnn-nnnn-n/yy/mm}
%% \copyrightdoi{nnnnnnn.nnnnnnn}

% Uncomment the publication rights you want to use.
%\publicationrights{transferred}
%\publicationrights{licensed}     % this is the default
%\publicationrights{author-pays}

%% \titlebanner{banner above paper title}        % These are ignored unless
%% \preprintfooter{short description of paper}   % 'preprint' option specified.

\title{Verified Low-Level Programming Embedded in \fstar
%\ifanon\vspace{-1.5cm}\fi
}
%% \subtitle{Subtitle Text, if any}

\ifanon
\author{}
\else
\author{Jonathan~Protzenko}
\affiliation{\institution{Microsoft Research}\ifcamera, USA\fi}
\author{Jean-Karim~Zinzindohoué}
\affiliation{\institution{INRIA Paris}\ifcamera, France\fi}
\author{Aseem~Rastogi}
\affiliation{\institution{Microsoft Research}\ifcamera, USA\fi}
\author{Tahina~Ramananandro}
\affiliation{\institution{Microsoft Research}\ifcamera, USA\fi}
\author{Peng~Wang}
\affiliation{\institution{MIT CSAIL}\ifcamera, USA\fi}
\author{Santiago~Zanella-Béguelin}
\affiliation{\institution{Microsoft Research}\ifcamera, USA\fi}
\author{Antoine~Delignat-Lavaud}
\affiliation{\institution{Microsoft Research}\ifcamera, USA\fi}
\author{C\u{a}t\u{a}lin~Hri\c{t}cu}
\affiliation{\institution{INRIA Paris}\ifcamera, France\fi}
\author{Karthikeyan Bhargavan}
\affiliation{\institution{INRIA Paris}\ifcamera, France\fi}
\author{Cédric~Fournet}
\affiliation{\institution{Microsoft Research}\ifcamera, USA\fi}
\author{Nikhil~Swamy}
\affiliation{\institution{Microsoft Research}\ifcamera, USA\fi}
\makeatletter
% \renewcommand{\@shortauthors}{J. Protzenko, J-K. Zinzindohoué, A. Rastogi, T.
% Ramananandro, P. Wang,\\ S. Zanella-Béguelin, A. Delignat-Lavaud, C. Hri\c{t}cu, K.
% Bhargavan, C. Fournet, and N. Swamy}
\renewcommand{\@shortauthors}{Protzenko \emph{et.al.}}
\makeatother
\fi

\begin{CCSXML}
<ccs2012>
<concept>
<concept_id>10003752.10003790.10011741</concept_id>
<concept_desc>Theory of computation~Hoare logic</concept_desc>
<concept_significance>500</concept_significance>
</concept>
<concept>
<concept_id>10003752.10003790.10011740</concept_id>
<concept_desc>Theory of computation~Type theory</concept_desc>
<concept_significance>300</concept_significance>
</concept>
<concept>
<concept_id>10011007.10010940.10010992.10010993</concept_id>
<concept_desc>Software and its engineering~Correctness</concept_desc>
<concept_significance>500</concept_significance>
</concept>
<concept>
<concept_id>10011007.10010940.10010992.10010998.10010999</concept_id>
<concept_desc>Software and its engineering~Software verification</concept_desc>
<concept_significance>500</concept_significance>
</concept>
<concept>
<concept_id>10011007.10011006.10011041.10011047</concept_id>
<concept_desc>Software and its engineering~Source code generation</concept_desc>
<concept_significance>500</concept_significance>
</concept>
<concept>
<concept_id>10011007.10011006.10011008.10011009.10011012</concept_id>
<concept_desc>Software and its engineering~Functional languages</concept_desc>
<concept_significance>300</concept_significance>
</concept>
<concept>
<concept_id>10011007.10011006.10011039.10011311</concept_id>
<concept_desc>Software and its engineering~Semantics</concept_desc>
<concept_significance>300</concept_significance>
</concept>
<concept>
<concept_id>10011007.10011006.10011041</concept_id>
<concept_desc>Software and its engineering~Compilers</concept_desc>
<concept_significance>300</concept_significance>
</concept>
</ccs2012>
\end{CCSXML}

\ccsdesc[500]{Theory of computation~Hoare logic}
\ccsdesc[300]{Theory of computation~Type theory}
\ccsdesc[500]{Software and its engineering~Correctness}
\ccsdesc[500]{Software and its engineering~Software verification}
\ccsdesc[500]{Software and its engineering~Source code generation}
\ccsdesc[300]{Software and its engineering~Functional languages}
\ccsdesc[300]{Software and its engineering~Semantics}
\ccsdesc[300]{Software and its engineering~Compilers}


%% \category{CR-number}{subcategory}{third-level}

% general terms are not compulsory anymore,
% you may leave them out
%% \terms
%% term1, term2

\keywords{verified compilation, low-level programming, verified cryptography}

% \ch{PLDI is double blind, previous work should be cited in 3rd person}

% \cf{I prefer F$^\star$, C$^\star$, etc}

% \ifdraft
% Title variations:
% \begin{itemize}
% \item Verified and Efficient Low-Level Programming in \fstar
%   % \ch{Cedric likes this title, and supports pitching \lowstar as
%   %   a language (maybe not domain specific) for low-level programming}
% \item A DSL for Verified, Efficient, Cryptographic Implementations in \fstar
% \item A DSL for Efficient, Verified Cryptographic Implementations in \fstar
% \item Programming, Verifying and Compiling Cryptographic Implementations in \fstar
% \item Programming, Verifying and Compiling Secure Cryptographic Implementations in \fstar
% \item The Lord of XORing
% \item An Embedded DSL for Verified and Efficient Low-Level Programming in \fstar
% \item Embedding ...
% \item Low-Level Verified Programming Embedded in \fstar
% \item Verified Low-Level Programming Embedded in \fstar
% \end{itemize}
% \fi

  In this paper, we explore the connection between secret key agreement and secure omniscience within the setting of the multiterminal source model with a wiretapper who has side information. While the secret key agreement problem considers the generation of a maximum-rate secret key through public discussion, the secure omniscience problem is concerned with communication protocols for omniscience that minimize the rate of information leakage to the wiretapper. The starting point of our work is a lower bound on the minimum leakage rate for omniscience, $\rl$, in terms of the wiretap secret key capacity, $\wskc$. Our interest is in identifying broad classes of sources for which this lower bound is met with equality, in which case we say that there is a duality between secure omniscience and secret key agreement. We show that this duality holds in the case of certain finite linear source (FLS) models, such as two-terminal FLS models and pairwise independent network models on trees with a linear wiretapper. Duality also holds for any FLS model in which $\wskc$ is achieved by a perfect linear secret key agreement scheme. We conjecture that the duality in fact holds unconditionally for any FLS model. On the negative side, we give an example of a (non-FLS) source model for which duality does not hold if we limit ourselves to communication-for-omniscience protocols with at most two (interactive) communications.  We also address the secure function computation problem and explore the connection between the minimum leakage rate for computing a function and the wiretap secret key capacity.
  
%   Finally, we demonstrate the usefulness of our lower bound on $\rl$ by using it to derive equivalent conditions for the positivity of $\wskc$ in the multiterminal model. This extends a recent result of Gohari, G\"{u}nl\"{u} and Kramer (2020) obtained for the two-user setting.
  
   
%   In this paper, we study the problem of secret key generation through an omniscience achieving communication that minimizes the 
%   leakage rate to a wiretapper who has side information in the setting of multiterminal source model.  We explore this problem by deriving a lower bound on the wiretap secret key capacity $\wskc$ in terms of the minimum leakage rate for omniscience, $\rl$. 
%   %The former quantity is defined to be the maximum secret key rate achievable, and the latter one is defined as the minimum possible leakage rate about the source through an omniscience scheme to a wiretapper. 
%   The main focus of our work is the characterization of the sources for which the lower bound holds with equality \textemdash it is referred to as a duality between secure omniscience and wiretap secret key agreement. For general source models, we show that duality need not hold if we limit to the communication protocols with at most two (interactive) communications. In the case when there is no restriction on the number of communications, whether the duality holds or not is still unknown. However, we resolve this question affirmatively for two-user finite linear sources (FLS) and pairwise independent networks (PIN) defined on trees, a subclass of FLS. Moreover, for these sources, we give a single-letter expression for $\wskc$. Furthermore, in the direction of proving the conjecture that duality holds for all FLS, we show that if $\wskc$ is achieved by a \emph{perfect} secret key agreement scheme for FLS then the duality must hold. All these results mount up the evidence in favor of the conjecture on FLS. Moreover, we demonstrate the usefulness of our lower bound on $\wskc$ in terms of $\rl$ by deriving some equivalent conditions on the positivity of secret key capacity for multiterminal source model. Our result indeed extends the work of Gohari, G\"{u}nl\"{u} and Kramer in two-user case.
\maketitle

% !TEX root = ../arxiv.tex

Unsupervised domain adaptation (UDA) is a variant of semi-supervised learning \cite{blum1998combining}, where the available unlabelled data comes from a different distribution than the annotated dataset \cite{Ben-DavidBCP06}.
A case in point is to exploit synthetic data, where annotation is more accessible compared to the costly labelling of real-world images \cite{RichterVRK16,RosSMVL16}.
Along with some success in addressing UDA for semantic segmentation \cite{TsaiHSS0C18,VuJBCP19,0001S20,ZouYKW18}, the developed methods are growing increasingly sophisticated and often combine style transfer networks, adversarial training or network ensembles \cite{KimB20a,LiYV19,TsaiSSC19,Yang_2020_ECCV}.
This increase in model complexity impedes reproducibility, potentially slowing further progress.

In this work, we propose a UDA framework reaching state-of-the-art segmentation accuracy (measured by the Intersection-over-Union, IoU) without incurring substantial training efforts.
Toward this goal, we adopt a simple semi-supervised approach, \emph{self-training} \cite{ChenWB11,lee2013pseudo,ZouYKW18}, used in recent works only in conjunction with adversarial training or network ensembles \cite{ChoiKK19,KimB20a,Mei_2020_ECCV,Wang_2020_ECCV,0001S20,Zheng_2020_IJCV,ZhengY20}.
By contrast, we use self-training \emph{standalone}.
Compared to previous self-training methods \cite{ChenLCCCZAS20,Li_2020_ECCV,subhani2020learning,ZouYKW18,ZouYLKW19}, our approach also sidesteps the inconvenience of multiple training rounds, as they often require expert intervention between consecutive rounds.
We train our model using co-evolving pseudo labels end-to-end without such need.

\begin{figure}[t]%
    \centering
    \def\svgwidth{\linewidth}
    \input{figures/preview/bars.pdf_tex}
    \caption{\textbf{Results preview.} Unlike much recent work that combines multiple training paradigms, such as adversarial training and style transfer, our approach retains the modest single-round training complexity of self-training, yet improves the state of the art for adapting semantic segmentation by a significant margin.}
    \label{fig:preview}
\end{figure}

Our method leverages the ubiquitous \emph{data augmentation} techniques from fully supervised learning \cite{deeplabv3plus2018,ZhaoSQWJ17}: photometric jitter, flipping and multi-scale cropping.
We enforce \emph{consistency} of the semantic maps produced by the model across these image perturbations.
The following assumption formalises the key premise:

\myparagraph{Assumption 1.}
Let $f: \mathcal{I} \rightarrow \mathcal{M}$ represent a pixelwise mapping from images $\mathcal{I}$ to semantic output $\mathcal{M}$.
Denote $\rho_{\bm{\epsilon}}: \mathcal{I} \rightarrow \mathcal{I}$ a photometric image transform and, similarly, $\tau_{\bm{\epsilon}'}: \mathcal{I} \rightarrow \mathcal{I}$ a spatial similarity transformation, where $\bm{\epsilon},\bm{\epsilon}'\sim p(\cdot)$ are control variables following some pre-defined density (\eg, $p \equiv \mathcal{N}(0, 1)$).
Then, for any image $I \in \mathcal{I}$, $f$ is \emph{invariant} under $\rho_{\bm{\epsilon}}$ and \emph{equivariant} under $\tau_{\bm{\epsilon}'}$, \ie~$f(\rho_{\bm{\epsilon}}(I)) = f(I)$ and $f(\tau_{\bm{\epsilon}'}(I)) = \tau_{\bm{\epsilon}'}(f(I))$.

\smallskip
\noindent Next, we introduce a training framework using a \emph{momentum network} -- a slowly advancing copy of the original model.
The momentum network provides stable, yet recent targets for model updates, as opposed to the fixed supervision in model distillation \cite{Chen0G18,Zheng_2020_IJCV,ZhengY20}.
We also re-visit the problem of long-tail recognition in the context of generating pseudo labels for self-supervision.
In particular, we maintain an \emph{exponentially moving class prior} used to discount the confidence thresholds for those classes with few samples and increase their relative contribution to the training loss.
Our framework is simple to train, adds moderate computational overhead compared to a fully supervised setup, yet sets a new state of the art on established benchmarks (\cf \cref{fig:preview}).

\ifpagelimits
\newpage
\fi
\section{Discussion on Approximation \textit{vs} Stability and Recovery}\label{sec:approx-stability}


In the world of approximation algorithms, for a maximization problem for which an algorithm outputs $S$ and the optimum is $S^*$, what we typically try to prove is that
$w(S)\ge w(S^*)/\alpha$, even in the worst case; this \textit{approximation inequality} means that the algorithm at hand is an $\alpha$-approximation, so it is a \textit{good} algorithm. Though one might be quick to say that recovery of $\alpha$-stable instances immediately follows from the approximation inequality, this is not true because of the intersection $S\cap S^*$; if we have no intersection, then recovery indeed follows. 

What the research on stability and exact recovery suggests, is that we should try to understand if some of our already known approximation algorithms have the stronger property $w(S\setminus S^*)\ge w(S^*\setminus S)/\alpha$ or at least if they have it on stable instances. We refer to the latter as the \textit{recovery inequality}. This would directly imply an exact recovery result for $\alpha$-stable instances because we could $\alpha$-perturb only the $S\setminus S^*$ part of the input and get: 
\[
\noindent w(S\setminus S^*)\ge w(S^*\setminus S)/\alpha \implies \alpha\cdot w(S\setminus S^*) +w(S\cap S^*) \ge w(S^*\setminus S) +w(S\cap S^*) = w(S^*)
\] thus violating the fact we were given an $\alpha$-stable instance, unless $S\setminus S^* = \emptyset$.

This would mean that the algorithm successfully retrieved $S^*$ and could potentially explain why many approximation algorithms behave far better in practice than in theory. Furthermore, from a theory perspective, it would mean that many results from the well-studied area of approximation algorithms could be translated in terms of stability and recovery.

As a concluding remark, we want to point out that even though one might think that an $\alpha$-approximation algorithm needs at least $\alpha$-stability for recovery, this is not true as the somewhat counterintuitive result from \cite{balcan2015k} tells us: asymmetric $k$-center cannot be approximated to any constant factor, but can be solved optimally on 2-stable instances. This was the
first problem that is hard to approximate to any constant factor in the worst case, yet can be optimally
solved in polynomial time for 2-stable instances. The other direction (having an $\alpha$-approximation algorithm that cannot recover arbitrarily stable instances) is also true. These findings suggest that there are interesting connections between stability, exact recovery and approximation.

\ifpagelimits
\newpage
\fi
\section{A formal translation from \lowstar to Clight}
\label{sec:formal}

\newcommand\emf{{\sc emf}$^\star$\xspace}
\newcommand\emfST{{\sc emf}$^\star_{\text {\sc st}}$\xspace} % consistent with 2 others

%% \tr{This title should be changed into something like: a formal model
%%   of our extraction from \fstar to C}

%% \jp{I feel like most of this has been said in the intro; what hasn't been said,
%% e.g. the precise specification of noninterference, perhaps is better suited to
%% the specific section.}

Figure~\ref{fig:bigpicture} on page~\pageref{fig:bigpicture} provides
an overview of our translation from \lowstar to CompCert Clight,
%
starting with \emf, a recently proposed model
of \fstar~\cite{dm4free}; then \lamstar, a formal core of \lowstar
after all erasure of ghost code and specifications; then \cstar, an
intermediate language that switches the calling convention closer to
C; and finally to Clight.
%
In the end, our theorems establish that: (a) the safety and functional
correctness properties verified at the \fstar level carry on to the
generated Clight code (via semantics preservation), and (b) \lowstar
programs that use the secrets parametrically enjoy the trace
equivalence property, at least until the Clight level, thereby
providing protection against side-channels.


%% , and show that
%% it preserves the functional behavior as well as the secret-independent
%% traces (\sref{to-clight}).  whose calling convention is close to that
%% of Clight

%% and show that erasing preserves verification. Then, following the
%% \emfST semantics of primitive state proposed
%% in~\cite{dm4free}, we present \lamstar, a formal core of \lowstar with
%% primitive stack-based allocation. We prove that \lamstar programs that
%% use secrets parametrically have secret-independent traces
%% (\sref{lamstar}). Next we present \cstar, an intermediate language
%% whose calling convention is close to that of Clight, but whose scoping
%% structure is closer to that of \lamstar. We translate \lamstar
%% to \cstar proving it a bisimulation
%% (\sref{lamstar-to-cstar}). Finally, we translate \cstar to CompCert
%% Clight, and show that it preserves the functional behavior as well as
%% the secret-independent traces (\sref{to-clight}).


\paragraph*{Prelude: Internal transformations in \emf}
We begin by briefly describing a few internal transformations on \emf,
focusing in the rest of this section on the pipeline from \lamstar to
Clight---the formal details are in the appendix. To express
computational irrelevance, we extend \emf with a primitive \kw{Ghost}
effect. An erasure transformation removes ghost subterms, and we prove
that this pass preserves semantics, via a logical relations
argument. Next, we rely on a prior result~\cite{dm4free} showing
that \emf programs in the \lst$ST$ monad can be safely reinterpreted
in \emfST, a calculus with primitive state. We obtain an instance
of \emfST suitable for \lowstar by instantiating its state type
with \lst$HyperStack.mem$. To facilitate the remainder of the
development, we transcribe \emfST to \lamstar, which is a restriction
of \emfST to first-order terms that only use stack memory, leaving the
heap out of \lamstar, since it is not a particularly interesting
aspect of the proof. This transcription step is essentially
straightforward, but is not backed by a specific proof. We plan to
fill this gap as we aim to mechanize our entire proof in the future.

%% and introduce
%% \lamstar, a formal core of \lowstar (\sref{lamstar}).
%% We have not yet formally proven
%% the simulation between \emf and \lamstar,


%% \paragraph{Primitive state} While the state effect in \emf is a
%% user-defined effect, parametric in the state, Ahman et
%% al.~\cite{dm4free} also present \emfST, a calculus with primitive
%% state, and prove a simulation between \emfST and
%% \emf. Relying on this result, we instantiate the
%% memory model of \emfST with stack-based memory,\ch{what's stack-based memory?
%%   you mean hyper-stacks? or you mean the hyper-stack part without the heap
%%   part ... this should be made explicit} and introduce
%% \lamstar, a formal core of \lowstar (\sref{lamstar}).
%% We have not yet formally proven
%% the simulation between \emf and \lamstar, but we conjecture that the
%% proof will be along similar lines as Ahman et al.
\nik{Mention non-termination here?}




%% The goal of our work is to provide users with a development and
%% verification environment consisting of a high-level language, \lowstar
%% (a subset of \fstar,) a compiler from \lowstar to C,
%% and \lowstar-level verification technology, such that their
%% combination establishes the following formal guarantees on the
%% extracted C code:
%% \begin{itemize}
%% \item \emph{safety:}
%% the extracted C program must not crash or reach any inconsistent state
%% leading to undefined behaviors. In particular, no out-of-bounds memory
%% accesses, or through dangling pointers, must ever occur (\emph{memory
%% safety}.)
%% \item \emph{functional correctness:}
%% the user should be able to specify, at the \lowstar level, whether the
%% extracted C program should diverge or terminate, and the return value
%% in the latter case, and the extracted C program must follow such
%% user-defined specification.
%% \item \emph{security:}
%% the extracted C program must be free of side channels. In this paper,
%% we focus on the following two sources of side channels:
%% \begin{itemize}
%% \item \emph{memory access noninterference:}
%% two executions of the same program starting from different secrets
%% perform the same sequence of memory accesses (although the actual
%% values stored in memory may differ), so that an attacker cannot
%% distinguish two such executions through memory cache-based side
%% channels.
%% \item \emph{control flow noninterference:}
%% two executions of the same program starting from different secrets
%% follow the same control flow. In the \emph{program counter model},
%% this property implies that two such executions will actually execute
%% the same sequence of instructions, so that an attacker cannot
%% distinguish them through time or energy-based side channels.
%% \end{itemize}
%% \end{itemize}

%% Our approach is to leverage \fstar's strong type system with dependent
%% types, so that any well-typed \lowstar program enjoys all of the above
%% safety, functional correctness and security properties. Then, we prove
%% that extraction from \lowstar to C preserves all those guarantees down
%% to the extracted C code.

%% \paragraph{Summary of passes}
%% \jp{This high-level overview has been mentioned in the intro\ldots what about
%% condensing it to: ``The following sections present \lowstar and erasure, then
%% \lamstar and the two translation steps to C and Clight''?}
%% In Section~\ref{sec:lowstar}, we define the \lowstar source
%% language, as a subset of \fstar, allowing the
%% user to augment their programs with specifications as \emph{ghost
%% code}.  Then, we describe how those specifications are erased in a
%% semantics-preserving and security-preserving way.

%% Then, in Section~\ref{sec:lamstar}, we formally define the \lamstar
%% language, and we describe how \lowstar programs obtained after
%% erasure, are reinterpreted in \lamstar. Then, we prove that
%% well-typed \lamstar programs make our security properties hold.

%% Then, in Section~\ref{sec:low-to-c}, we describe the \cstar language,
%% which we design as closely as possible to a subset of C, and we
%% present the transformation from \lamstar to \cstar and its proof.

%% Finally, in Section~\ref{sec:to-clight}, we transform \cstar programs
%% into C programs, with our correctness proof, where the target language
%% is actually the Clight \cite{} subset of C formally specified in CompCert \cite{}.

  
%% \subsection{\emf and erasure}
%% \label{sec:lowstar}

%% \emf~\cite{dm4free}, the formal model of \fstar, is a
dependently typed calculus for pure programs, that can be extended
with a partially ordered user-defined effects (such as state,
exceptions, etc.) represented in terms of the primitive
\kw{Pure} effect. %% The dynamic semantics of \emf reduces the \kw{Pure}
%% effect primitively, while the user-defined effects are reduced using
%% the underlying \kw{Pure} definitions of their monadic \kw{return} and
%% \kw{bind} functions.
%% \paragraph{Erasure}
To express computational irrelevance, we extend \emf with a
primitive \kw{Ghost} effect whose dynamic semantics is identical
to \kw{Pure}.
%
As such, we provide a subsumption rule to treat
proof-irrelevant \kw{Ghost} computations (e.g., those returning
non-informative types like \lst$unit$) as \kw{Pure} computations, and,
dually,
%
allow all \kw{Pure} computations to be subsumed
to \kw{Ghost}. Otherwise, we forbid \kw{Ghost} code from being
composed with any other effect, preventing computationally relevant
computations from relying on \kw{Ghost} code.
%
We define an erasure relation, where all unit-returning \kw{Ghost}
and \kw{Pure} sub-terms of a well-typed term are erased
to \lst$()$. We prove that erasure preserves typing, meaning that all
properties proven of the source program, remain valid after erasure.
%
The appendix contains a formal statement and proof, omitted here for
lack of space, since we focus mainly on the translation to Clight,
rather the transformations internal to \fstar.

\paragraph{Primitive state} While the state effect in \emf is a
user-defined effect, parametric in the state, Ahman et
al.~\cite{dm4free} also present \emfST, a calculus with primitive
state, and prove a simulation between \emfST and
\emf. Relying on this result, we instantiate the
memory model of \emfST with stack-based memory,\ch{what's stack-based memory?
  you mean hyper-stacks? or you mean the hyper-stack part without the heap
  part ... this should be made explicit} and introduce
\lamstar, a formal core of \lowstar (\sref{lamstar}).
We have not yet formally proven
the simulation between \emf and \lamstar, but we conjecture that the
proof will be along similar lines as Ahman et al.
\nik{Mention non-termination here.}
\tr{It follows from the bisimulation that since the original program is memory
safe, then the resulting \lamstar program is memory safe.}


%% \tr{As far as I understood, we define \lowstar as the subset of
%%   EM\fstar where the non-ghost code part is first-order. Then we
%%   describe erasure.}

%% \tr{
%%   Aseem 1/2
%% }

\subsection{\lamstar: A Formal Core of \lowstar Post-Erasure}
\label{sec:lamstar}

\begin{figure}[t]
\vspace{-1em}
\begin{small}
\[\!\!\!\!\!
  \begin{array}{r@{~}c@{~}l@{}}
    \tau  & ::= & \kw{int} \mid \kw{unit} \mid \{\ls{\fd=\tau}\} \mid \kw{buf}~\tau \mid \alpha\\
    \lv   & ::= & x \mid n \mid () \mid \{\ls{\fd=\lv}\} \mid (b, n, \ls{\fd}) \\
    \lexp & ::= & \elet{x:\tau}{\ereadbuf {\lexp_1}{\lexp_2}}{\lexp} \mid \elet{\_}{\ewritebuf{\lexp_1}{\lexp_2}{\lexp_3}}{\lexp} \\
          & \mid &\elet{x}{\enewbuf{n}{(\lexp_1:\tau)}}{\lexp_2} \mid \esubbuf{\lexp_1}{\lexp_2}  \\
          & \mid & \elet{x:\tau}{\ereadstruct {\lexp_1}}{\lexp} \mid \elet{\_}{\ewritestruct{\lexp_1}{\lexp_2}}{\lexp} \\
          & \mid &\elet{x}{\enewstruct{(\lexp_1:\tau)}}{\lexp_2} \mid \estructfield{\lexp_1}{\fd}  \\
          & \mid &\withframe\;\lexp \mid \epop\;\lexp \mid \eif {\lexp_1}{\lexp_2}{\lexp_3} \\
          & \mid &\elet{x:\tau}{d\;\lexp_1}{\lexp_2} \mid \elet{x:\tau}{\lexp_1}{\lexp_2} \mid \{\ls{\fd=\lexp}\} \mid \lexp.\fd  \mid \lv \\
    \lp   & ::= &\cdot \mid \etlet{d}{\lambda y:\tau_1. \; \lexp : \tau_2}, \lp \\
\end{array} 
\]
\end{small}
\caption{\lamstar syntax}
\label{fig:lamstar-syntax}
\end{figure}

The meat of our formalization of \lowstar begins with \lamstar, a
first-order, stateful language, whose state is structured as a stack
of memory regions. It has a simple calling convention using a
traditional, substitutive $\beta$-reduction rule. Its small-step
operational semantics is instrumented to produce traces that record
branching and the accessed memory addresses. As such, our traces
account for side-channel vulnerabilities in programs based on the
program counter model~\cite{molnar05pcmodel} augmented to track
potential leaks through cache behavior~\cite{barthe-ccs2014}. We
define a simple type system for \lamstar and prove that programs
well-typed with respect to some values at an abstract type produce
traces independent of those values, e.g., our bigint library when
translated to \lamstar is well-typed with respect to an abstract type
of \lst$limb$s and leaks no information about them via their traces.

\paragraph{Syntax} Figure~\ref{fig:lamstar-syntax} shows the syntax
of \lamstar. A program $P$ is a sequence of top-level function
definitions, $d$. We omit loops but allow recursive function definitions.
%
Values $v$ include constants, immutable records, and buffers $(b, n, [])$ and mutable structures $(b, n, \ls{\fd})$
passed by reference, where $b$ is the address of the buffer or structure, $n$ is the
offset in the buffer, and $\ls{\fd}$ designates the path to the structure field to take a reference of (this path, as a list, can be longer than 1 in the case of nested mutable structures.)
%
Stack allocated buffers (\kw{readbuf}, \kw{writebuf}, \kw{newbuf}, and
\kw{subbuf}), and their mutable structure counterparts (\kw{readstruct}, \kw{writestruct}, \kw{newstruct}, $\structfield$), are the main feature of the expression language, along
with $\withframe\;\lexp$, which pushes a new frame on the stack for
the evaluation of $\lexp$, after which it is popped (using
$\epop\;\lexp$, an administrative form internal to the calculus).
%
Once a frame is popped, all its local buffers and mutable structures become inaccessible.

Mutable structures can be nested, and stored into buffers, in both cases without extra indirection. However, the converse is not true, as \lamstar currently does not allow arbitrary nesting of arrays within mutable structures without explicit indirection via separately allocated buffers. We leave such generalization as future work.

%% We expect function applications $d e$ to
%% be explicitly let-bound.


%% and immutable records
%% ($\{\ls{\fd=\lexp}\}$ and $\lexp.\fd$).

%% $\withframe\;\lexp$ models a C
%% block -- it pushes a new frame on the stack, and once $\lexp$ reduces
%% to a value, the frame is popped. More importantly, once a frame is
%% popped, all its local buffers also become inaccessible.


%% The \lamstar expression form $e$ is closer to the C
%% statement form with explicit \kw{let}-bindings.\ch{C has no explicit
%%   let bindings, but that's probably not what you mean}

%% We model stack
%% allocated buffers (\kw{readbuf}, \kw{writebuf}, \kw{newbuf}, and
%% \kw{subbuf}) and immutable records ($\{\ls{\fd=\lexp}\}$ and
%% $\lexp.\fd$). $\withframe\;\lexp$ models a C block -- it pushes a new
%% frame on the stack, and once $\lexp$ reduces to a value, the frame is
%% popped. More importantly, once a frame is popped, all its local
%% buffers also become inaccessible.\ch{Even more importantly should
%%   relate this to push\_frame and pop\_frame from the previous
%%   section. I guess that the Stack effect enforces that push\_frame and
%%   pop\_frame are balanced, thus you can replace them with
%%   with\_frame?}  


\paragraph{Type system} \lamstar types include the base
types \kw{int} and \kw{unit}, record types $\{\ls{\fd=\tau}\}$, buffer
types $\kw{buf}\;\tau$, mutable structure types $\kw{struct}\;\tau$, and abstract types $\alpha$.  The typing
judgment has the form, $\Gamma_P; \Sigma; \Gamma \vdash e : \tau$,
where $\Gamma_P$ includes the function signatures; $\Sigma$ is the
store typing; and $\Gamma$ is the usual context of variables. We elide
the rules, as it is a standard, simply-typed type system. The type
system guarantees preservation, but not progress, since it does not
attempt to account for bounds checks or buffer/mutable structure lifetime. However,
memory safety (and progress) is a consequence of \lowstar typing
and its semantics-preserving erasure to \lamstar.

\paragraph{Semantics} We define evaluation contexts $E$
for standard call-by-value, left-to-right evaluation order. The memory
$H$ is a stack of frames, where each frame maps addresses $b$ to a
sequence of values $\ls{v}$. The \lamstar small-step semantics
judgment has the form $P \vdash (H, \lexp) \rightarrow_{\trace}
(H', \lexp')$, meaning that under the program $P$, configuration $(H,
e)$ steps to $(H', e')$ emitting a trace $\trace$, including
reads and writes to buffer references or mutable structure references, and branching behavior, as
shown below.

\vspace{-0.5cm}
\[
\small
\begin{array}{rl}
\trace ::= \cdot \mid \kw{read}(b, n, \ls{\fd}) \mid \kw{write}(b,
n, \ls{\fd}) \mid \kw{brT} \mid \kw{brF} \mid \trace_1, \trace_2
\end{array}
\]

\begin{figure*}
  \small
  \begin{mathpar}
  \inferrule* [Right=WF]
  {
    %% H_1 = H; \{\} \quad e_1 = \epop\;\lexp
  }
  {
    \lp \vdash (H, \withframe\;\lexp) \rightarrow_{\cdot} (H;\{\}, \epop\;\lexp)
  }

\inferrule* [Right=Pop]
{
}
{
  \lp \vdash (H;\_, \epop\;\lv) \rightarrow_{\cdot} (H, \lv)
}

\inferrule* [Right=LIfF]
{
\;
}
{
  \lp \vdash (H, \eif{0}{\lexp_1}{\lexp_2}) \rightarrow_\brf (H, \lexp_2)
}

\inferrule* [Right=App]
{
  \lp(f)=\lambda y:\tau_1.\;\lexp_1:\tau_2
}
{
  \lp \vdash (H, \elet{x:\tau}{f\;v}{\lexp}) \rightarrow (H, \elet{x:\tau}{\lexp_1[v/y]}{e})
}

\inferrule* [Right=LRd]
{
  H(b, n+n_1, []) = \lv \\
  \trace = \kw{read}(b, n + n_1, [])
}
{
  \lp \vdash (H, \elet{x}{\ereadbuf{(b,n,[])}{n_1}}{\lexp}) \rightarrow_{\trace} (H, \lexp[\lv/x])
}

\inferrule* [Right=New]
{
  b \notin \kw{dom}(H;h) \quad   h_1 = h[b\mapsto \lv^n] \quad \lexp_1 = \lexp[(b, 0)/x] \\
 \trace = \kw{write}(b, 0), \dots, \kw{write}(b, n - 1)
}
{
  \lp \vdash (H;h, \elet{x}{\enewbuf{n}{(\lv:\tau)}}{\lexp}) \rightarrow_{\trace} (H;h_1, \lexp_1)
}

\end{mathpar}

\caption{Selected semantic rules from \lamstar}
\label{fig:lamstar-sem}
\end{figure*}

Figure~\ref{fig:lamstar-sem} shows selected reduction rules from
\lamstar.
%
Rule {\sc{WF}} pushes an empty frame on the stack, and rule {\sc{Pop}}
pops the topmost frame once the expression has been evaluated.
%
Rule {\sc{LIfF}} is standard, except for the trace $\kw{brF}$ recorded
on the transition.
%
Rule {\sc{App}} is a standard, substitutive $\beta$-reduction.
%
Rule {\sc{LRd}} returns the value at the $(n + n_1)$ offset in the
buffer at address $b$, and emits a $\kw{read}(b, n + n_1, [])$ event.
%
Rule {\sc{New}} initializes the new buffer, and emits write events
corresponding to each offset in the buffer.

\paragraph{Secret independence}
A \lamstar program can be written against an interface providing
secrets at an abstract type.
%
For example, for the abstract type \lst$limb$, one might augment the
function signatures $\Gamma_P$ of a program with an interface for the
abstract type $\Gamma_{\kw{limb}} =$
%
\lst@eq_mask : limb$^2$ -> limb@, and typecheck a source program
with free \lst$limb$ variables ($\Gamma =$ \lst$secret:limb$),
and empty store typing, using the judgment
%
$\Gamma_{\kw{limb}}, \Gamma_p; \cdot; \Gamma \vdash \lexp : \tau$.
%
Given any representation $\tau$ for \lst$limb$, an implementation for
\lst$eq_mask$ whose trace is input independent, and any pair of values
$v_0:\tau, v_1:\tau$, we prove that running $e[v_0/\text{\lst{secret}}]$ and $e[v_1/\text{\lst{secret}}]$
produces identical traces, i.e., the traces reveal no information
about the secret $v_i$. We sketch the formal development next, leaving
details to the appendix.

Given a derivation
%
$\Gamma_s, \Gamma_P; \Sigma; \Gamma \vdash e : \tau$, let $\Delta$ map
type variables in the interface $\Gamma_s$ to concrete types and let $P_s$ contain
the implementations of the functions (from $\Gamma_s$) that operate on
secrets.
%
To capture the secret independence of $P_s$, we define a notion of an
\emph{equivalence modulo secrets}, a type-indexed relation for values
($v_1 \equiv_{\tau} v_2$) and memories ($\Sigma \vdash H_1 \equiv
H_2$). Intuitively two values (resp. memories) are equivalent modulo
secrets if they only differ in subterms that have abstract types in
the domain of the $\Delta$ map---we abbreviate ``equivalent modulo
secrets'' as ``related'' below.
%
We then require that each function $f \in P_s$, when applied in
related stores to related values, always returns related results, while
producing \emph{identical} traces.
%
Practically, $P_s$ is a (small) library written carefully to ensure
secret independence.

Our secret-independence theorem is then as follows:

\begin{theorem}[Secret independence]
  Given
  \begin{enumerate}
  \item a program well-typed against a secret interface, $\Gamma_s$,
    i.e, $\Gamma_s, \Gamma_P; \Sigma; \Gamma \vdash (H, e) : \tau$,
    
  \item a well-typed implementation of the $\Gamma_s$ interface,
    $\Gamma_s; \Sigma; \cdot \vdash_{\Delta} P_s$, such that $P_s$ is
    equivalent modulo secrets,

  \item a pair $(\rho_1, \rho_2)$ of well-typed substitutions for $\Gamma$,
  \end{enumerate}
  
  then either:
  \begin{enumerate}
    \item 
      both programs cannot reduce further, i.e.
      $P_s, P \vdash (H, e)[\rho_1] \nrightarrow$
      and
      $P_s, P \vdash (H, e)[\rho_2] \nrightarrow$, or
    \item 
      both programs make progress with the same trace, i.e.
      there exists $\Sigma' \supseteq \Sigma, \Gamma' \supseteq \Gamma,
      H', e'$, a pair $(\rho_1', \rho_2')$ of well-typed substitutions
      for $\Gamma'$, and a trace $\trace$ such that

  \begin{enumerate}[i)]
  \item
         $P_s, P \vdash (H, e)[\rho_1] \rightarrow^{+}_{\trace} (H', e')[\rho'_1]$
         and
         $P_s, P \vdash (H, e)[\rho_2] \rightarrow^{+}_{\trace} (H', e')[\rho'_2]$, and
  \item  $\Gamma_s, \Gamma_P; \Sigma'; \Gamma' \vdash (H', e') : \tau$
  \end{enumerate}
  \end{enumerate}
\end{theorem}  
         
%% \Gamma'_f \vdash_{\Delta} \rho'_i$.
   

%%     , $i \in \{1, 2\}$. (2)
%% $\Sigma; \Gamma_s, \Gamma_f \vdash_{\Delta}
%% \rho_i$, (3) $P; \Sigma; G_s \vdash_{\Delta} P_s$, then 
%% \end{theorem}

%% It states that for any two substitutions of secret variables, $\rho_1$
%% and $\rho_2$, the output trace $\trace$ is the same, thereby
%% establishing the absence of secret-dependent side channels.


%% functions
%% who can ensure the \emph{fixed trace} property by manual audit.


%% We define a typed notion of \emph{secret independence} for $P_s$
%% relation means that the two values
%% (resp. memories) only differ in secrets.

%% We
%% provide the complete defintions in the appendix, but intuitively the
%% relation means that the two values (resp. memories) only differ in
%% secrets.

%% Then, we require that, $\forall d_s \in P_s$ s.t. $d_s:\tau_1
%% \rightarrow \tau_2 \in \Gamma_s$, if $(H_1, d_s~v_1)$ and $(H_2,
%% d_s~v_2)$ are well-typed, and $v_1 \equiv_{\tau_1} v_2$, $\Sigma
%% \vdash H_1 \equiv H_2$, then $\exists m, n, \trace, H'_1, v'_1, H'_2,
%% v'_2, \Sigma'$ s.t. $P, P_s \vdash (H_1, d_s~v_1)
%% \rightarrow^{m}_{\trace} (H'_1, v_1')$, $P, P_s \vdash (H_2, d_s~v_2)
%% \rightarrow^{n}_{\trace} (H'_2, v_2')$, $v'_1 \equiv_{\tau_2} v'_2$
%% and $\Sigma' \vdash H'_1 \equiv H'_2$.



%% In addition to the standard well-formedness conditions on each of
%% these, our theorem for secret-independent traces relies on a
%% \emph{fixed trace} property of the functions in $P_s$. We first define
%% an \emph{equivalent-modulo-secrets} relation for values ($v_1
%% \equiv_{\tau} v_2$) and memories ($\Sigma \vdash H_1 \equiv H_2$). We
%% provide the complete defintions in the appendix, but intuitively the
%% relation means that the two values (resp. memories) only differ in
%% secrets. Then, we require that, $\forall d_s \in P_s$ s.t. $d_s:\tau_1
%% \rightarrow \tau_2 \in \Gamma_s$, if $(H_1, d_s~v_1)$ and $(H_2,
%% d_s~v_2)$ are well-typed, and $v_1 \equiv_{\tau_1} v_2$, $\Sigma
%% \vdash H_1 \equiv H_2$, then $\exists m, n, \trace, H'_1, v'_1, H'_2,
%% v'_2, \Sigma'$ s.t. $P, P_s \vdash (H_1, d_s~v_1)
%% \rightarrow^{m}_{\trace} (H'_1, v_1')$, $P, P_s \vdash (H_2, d_s~v_2)
%% \rightarrow^{n}_{\trace} (H'_2, v_2')$, $v'_1 \equiv_{\tau_2} v'_2$
%% and $\Sigma' \vdash H'_1 \equiv H'_2$. In other words, when the
%% functions in $P_s$ are applied to equivalent-modulo-secret inputs,
%% then they terminate and produce a \emph{fixed} trace $\trace$
%% alongwith equivalent-modulo-secrets outputs. Practically, $P_s$ is a
%% (small) library written carefully by the experts who can ensure the
%% \emph{fixed trace} property by manual audit.

%% ; and let $\rho$ be a well-typed substitution for the variables
%% in $\Gamma$.

%% Our secret-independent traces theorem is then as follows:


%% The interface exposes (a)
%% type variables, as uninterpreted types for the secret variables, (b)
%% signatures of functions that operate on secrets, and (c) free
%% variables for secret values:

%% \vspace{-0.5cm}
%% \[
%% \begin{small}
%% \begin{array}{rl}
%% \Gamma_s ::= \cdot \mid \alpha \mid
%% d_s:\tau_1 \rightarrow \tau_2 \mid \Gamma_s, \Gamma'_s\;\;\;\;\;\;\;\;\Gamma_f ::= \cdot \mid x:\alpha, \Gamma_f
%% \end{array}
%% \end{small}
%% \]

%% The judgment has the standard form,
%% $P; \Sigma; \Gamma_s, \Gamma_f, \Gamma \vdash e : \tau$, where $\Sigma$
%% is the stack typing, and $\Gamma$ is the typing for free variables in
%% $e$ that are not secrets.\ch{What's $\Gamma_f$?}
%% We elide the rules, as it is a standard, simply-typed type system.

%% Let $\Delta$ map type variables in $\Gamma_s$ to a
%% concrete types, $\rho$ map secret variables in $\Gamma_f$ to
%% concrete values, and $P_s$ contain the implementations of the
%% functions (from $\Gamma_s$) that operate on secrets.

%% In addition to the standard well-formedness conditions on each of
%% these, our theorem for secret-independent traces relies on a
%% \emph{fixed trace} property of the functions in $P_s$. We first define
%% an \emph{equivalent-modulo-secrets} relation for values ($v_1
%% \equiv_{\tau} v_2$) and memories ($\Sigma \vdash H_1 \equiv H_2$). We
%% provide the complete defintions in the appendix, but intuitively the
%% relation means that the two values (resp. memories) only differ in
%% secrets. Then, we require that, $\forall d_s \in P_s$ s.t. $d_s:\tau_1
%% \rightarrow \tau_2 \in \Gamma_s$, if $(H_1, d_s~v_1)$ and $(H_2,
%% d_s~v_2)$ are well-typed, and $v_1 \equiv_{\tau_1} v_2$, $\Sigma
%% \vdash H_1 \equiv H_2$, then $\exists m, n, \trace, H'_1, v'_1, H'_2,
%% v'_2, \Sigma'$ s.t. $P, P_s \vdash (H_1, d_s~v_1)
%% \rightarrow^{m}_{\trace} (H'_1, v_1')$, $P, P_s \vdash (H_2, d_s~v_2)
%% \rightarrow^{n}_{\trace} (H'_2, v_2')$, $v'_1 \equiv_{\tau_2} v'_2$
%% and $\Sigma' \vdash H'_1 \equiv H'_2$. In other words, when the
%% functions in $P_s$ are applied to equivalent-modulo-secret inputs,
%% then they terminate and produce a \emph{fixed} trace $\trace$
%% alongwith equivalent-modulo-secrets outputs. Practically, $P_s$ is a
%% (small) library written carefully by the experts who can ensure the
%% \emph{fixed trace} property by manual audit.


%% \begin{figure}
%% \begin{footnotesize}
%% \[
%% \begin{array}{rl}
%% ev & ::= \symread\;(loc, \tau) \mid \symwrite\;(loc, \tau) \mid \brt \mid \brf \\
%% t & ::= ev ; \dots ; ev   \\
%% T & ::= t \mid ev ; \dots \\
%% beh & ::= \mathsf{Terminates} (t, n) \mid \mathsf{Diverges}(T) \mid \mathsf{GoesWrong}(t)
%% \end{array}
%% \]
%% \end{footnotesize}
%% \caption{Observable behaviors}
%% \label{fig:behaviors}
%% \end{figure}


%%   \lowstar is a first-order subset of \fstar that uses explicit memory operations and
%%   let-bindings to mimic C statements.
%% \tr{TODO: strengthen the link with EM\fstar, etc.}

%% The syntax of \lowstar is presented in
%% \fref{lamstar-syntax}. A \lowstar program is a sequence of
%% top-level function declarations. A \lowstar expression may be an
%% integer constant, the unit value, a variable, a local structure, a
%% structure field projection, a local binding, a conditional expression,
%% a (first-order) function application, a buffer read, a buffer write,
%% a buffer allocation, or the $\withframe$ special construct.

%% $\withframe$ is designed to mimic a C block: local buffers allocated
%% within a $\withframe$ (or block) go out of scope and out of life at
%% the end of the current $\withframe$, meaning the \emph{lifetime} of a
%% local buffer may extend beyond its lexical scope. $\withframe$ can be thought as
%% a pair of $\epush$ and $\epop$. $\epop$ arises during evaluation, but not in the
%% surface program.

%% \paragraph{Events and observable behaviors}

%% \jp{First sentence un-necessary?. Also the two bullet points seem redundant with
%% 3.0 above.}
%% We design the semantics of \lamstar with safety, functional
%% correctness and security in mind, through the notion
%% of \emph{observable behaviors}. In addition to termination or
%% divergence, the semantics collects a sequence of \emph{observable
%% events} all along an execution to model security guarantees :
%% \begin{itemize}
%% \item
%% For memory access noninterference, we need to account for the sequence
%% of all memory accesses produced by an execution, so we record each
%% memory access as an observable event bearing the nature of the memory
%% access (read or write), the memory location, and the number of bytes
%% (or the type of the value) read or written.
%% \item
%% For control flow noninterference, we need to account for the sequences
%% of branchings, i.e. which branch the program takes at each conditional
%% statement. So we record each such decision as an observable event
%% bearing the boolean result of the corresponding conditional test.
%% \end{itemize}

%% Thus, in the small-step semantics of \lamstar and \cstar, each
%% transition step produces a finite event trace, empty except at memory
%% accesses and branchings. Then, following CompCert \cite[\S 3.5]{xav},
%% we big-step the small-step semantics to obtain an observable behavior,
%% as summarized in Figure~\ref{fig:behaviors}, collecting all events
%% generated by an execution.

%% \begin{figure}
%% \begin{footnotesize}
%% \[
%% \begin{array}{rl}
%% ev & ::= \symread\;(loc, \tau) \mid \symwrite\;(loc, \tau) \mid \brt \mid \brf \\
%% t & ::= ev ; \dots ; ev   \\
%% T & ::= t \mid ev ; \dots \\
%% beh & ::= \mathsf{Terminates} (t, n) \mid \mathsf{Diverges}(T) \mid \mathsf{GoesWrong}(t)
%% \end{array}
%% \]
%% \end{footnotesize}
%% \caption{Observable behaviors}
%% \label{fig:behaviors}
%% \end{figure}

%% Thanks to those event traces, we easily characterize both security
%% properties at once:
%% \begin{definition}[Security]
%% A program is secure if, and only if, two executions of the same program
%% starting with different secrets produce the same observable behavior.
%% \end{definition}

%% Then, we need to prove that each pass of our extraction from \lamstar
%% to C preserves security. In the special case where a pass does not
%% change the memory layout and pointer values between the source and the
%% compiled code, it is enough to prove that observable behaviors are
%% exactly preserved. This is the case for some but not all of our
%% passes.

%% \paragraph{Semantics}
%% \jp{The $le'$ variant of expressions comes out of nowhere and jumps at the
%% reader when looking at the figure\ldots I find it confusing. Also, what was
%% wrong with Perry's style of having just one $le$ and saying that $lv$ and $pop$
%% are used only when reducing?}

%% Selected rules of the \lamstar semantics are shown in
%% Figure~\ref{fig:lamstar-sem}. The small-step semantics of \lamstar is
%% based on a reduction context to ensure a strict call-by-value
%% semantics with left-to-right evaluation of construct arguments.

%% A value is either a numerical constant, the unit value, or a pointer
%% value $(b, n, \ls{fd})$ where $b$ is a memory block containing an
%% array of local structures, $n$ is the array cell index, and $\ls{fd}$
%% is the path of structure fields to follow. The configuration is of the
%% form $(H, le')$ where $H$ is a stack of memory frames, each memory
%% frames containing memory blocks.  Reading (resp. writing) a value to
%% the stack produces a $\symread$ (resp. $\symwrite$) event. Similarly,
%% a conditional produces a $\brt$ (resp. $\brf$) event if the
%% conditional is true (resp. false). $\enewbuf{}{}{}$ allocates a new
%% memory block within the current (top-most) memory frame, and
%% initializes each array cell, hence producing a $\symwrite$ event per
%% cell. $\withframe$ creates a new memory frame, while $\epop$ discards
%% the top-most frame, making any pointers to this frame dangling.

%% \paragraph{Abstraction}

%% \tr{
%%   Aseem 2/2: noninterference proof.
%% }


\subsection{\cstar: An Intermediate Language}
\label{sec:lamstar-to-cstar}

\label{sec:low-to-c}
We move from \lamstar to Clight in two steps. The \cstar intermediate
language retains \lamstar's explicit scoping structure, but switches
the calling convention to maintain an explicit call-stack of
continuations (separate from the stack memory regions). \cstar also
switches to a more C-like syntax, separates side effect-free
expressions from effectful statements.
    
%% \paragraph{The \cstar language}
%%     \cstar is a reinterpretation of \lamstar which separates side
%%     effect-free expressions from effectful statements, reflecting this
%%     split as a C-style syntax and a small-step semantics for
%%     statements as opposed to a big-step semantics for expression
%%     evaluation. \cstar introduces a block-scoped notion of local
%%     arrays; however, contrary to C and similarly to \lamstar, block
%%     scopes are not required for branches of a conditional statement,
%%     so that any local variable or local array declared in a
%%     conditional branch, if not enclosed by a further block, is still
%%     live after the conditional statement.
% \begin{figure}[h]
\vspace{-.7em}
\[
  \begin{array}{rl}
    \cp & ::= \ls{\ecfun fx{\tau}{\tau}{\ls{\cstmt}}} \\
    \cexp & ::= n \mid () \mid x \mid \cexp+\cexp \mid \{\ls{\fd=\cexp}\} \mid \cexp.\fd \mid \eptrfd{\cexp}{\fd} \\
    \cstmt & ::= \evardecl {\tau}x{\cexp} \mid \evardecl{\tau}{x}{\eapply f{\cexp}} \mid \eif{\cexp}{\ls{\cstmt}}{\ls{\cstmt}} \mid \ereturn \cexp \\
    & \mid \{\ls{\cstmt}\} \mid \earray {\tau}xn \mid \evardecl{\tau}{x}{\eread {\cexp}} \mid \ewrite {\cexp}{\cexp} \mid \memset{\cexp}{n}{\cexp} \\
  \end{array}
\]

\noindent
The syntax is unsurprising, with two notable exceptions.
%
First, despite the closeness to C syntax, contrary to C and similarly
to \lamstar, block scopes are not required for branches of a
conditional statement, so that any local variable or local array
declared in a conditional branch, if not enclosed by a further block,
is still live after the conditional statement.
%
Second, non-array local variables are immutable after
initialization.

%%%%%%%%%%%%%%%%%%%%%%%%%%%%%%%%%%%%%%%%%%%%%%%%%%%%%%%%%%%%%%%%%%%%%%%%%%%%%%%%

\begin{figure*}
  \small
  \begin{mathpar}
  \inferrule* [Right=Block]
  {
    %%\;
  }
  {
    \cp \vdash (S, V, \{\ls{\cstmt_1}\};\ls{\cstmt_2}) \step (S;(\{\},V,\symhole;\ls{\cstmt_2}), V, \ls{\cstmt_1})
  }

  \inferrule* [Right=Empty]
{
}
{
  \cp \vdash (S; (M, V', E), V, []) \step (S, V', \fplug{E}{()})
}

\inferrule* [Right=CIfF]
{
  \eval{\cexp}{(V)} = 0
}
{
  \cp \vdash (S, V, \eif{\cexp}{\ls{\cstmt_1}}{\ls{\cstmt_2}};\ls{\cstmt}) \step_\brf (S, V, \ls{\cstmt_2};\ls{\cstmt})
}

\inferrule* [Right=Call]
{
  \cp(f)=\ecfuntwo{y}{\tau_1}{\tau_2}{\ls{\cstmt_1}} \\
  \eval{\cexp}{(V)}=v
}
{
  \cp \vdash (S, V, \tau\;x=f\;\cexp; \ls{\cstmt}) \step (S;(\None, V, \tau\;x=\symhole;\ls{\cstmt}), \{\}[y\mapsto v], \ls{\cstmt_1})
}

\inferrule* [Right=CRead]
{
  \eval{\cexp}{(V)} = (b, n, \ls{\fd}) \quad
  \symget(S, (b, n, \ls{\fd})) = v \quad
  \trace = \symread\;(b,n,\ls{\fd})
}
{
  \cp \vdash (S, V, \evardecl{\tau}{x}{\eread \cexp}; \ls{\cstmt}) \step_{\trace} (S, V[x \mapsto v], \ls{\cstmt})
}

\inferrule* [Right=ArrDecl]
{
  \quad \\\\
  S = S'; (M, V, E) \\
  b\not\in S \\
  V' = V[x\mapsto (b, 0, [])]
}
{
  \cp \vdash (S, V, \tau\;x[n]; \ls{\cstmt}) \step (S';(M[b\mapsto \None^n], V, E), V', \ls{\cstmt})
}

\end{mathpar}
\caption{Selected semantic rules from \cstar}
\label{fig:cstar-sem}
\end{figure*}

\paragraph{Operational semantics, in contrast to \lamstar}
A \cstar evaluation configuration $C$ consists of a stack $S$, a
variable assignment $V$ and a statement list $\ls{\cstmt}$ to be
reduced. A stack is a list of frames. A frame $F$ includes frame
memory $M$, local variable assignment $V$ to be restored upon function
exit, and continuation $E$ to be restored upon function exit.
%
Frame memory $M$ is optional: when it is $\bot$, the frame is called a
``call frame''; otherwise a ``block frame'', allocated whenever
entering a statement block and deallocated upon exiting such block. A
frame memory is just a partial map from block identifiers to value
lists. Each \cstar statement performs at most one function call, or
otherwise, at most one side effect. Thus, \cstar is deterministic.

The semantics of \cstar is shown to the right in
Figure~\ref{fig:cstar-sem}, also illustrating the translation
from \lamstar to \cstar. There are three main differences.
First, \cstar's calling convention (rule {\sc
Call}) shows an explicit call frame being pushed on the stack, unlike
\lamstar's $\beta$ reduction.
%
Additionally, \cstar expressions do not have side effects and do not
access memory; thus, their evaluation order does not matter and their
evaluation can be formalized as a big-step semantics; by themselves,
expressions do not produce events. This is apparent in rules like {\sc
CIfF} and {\sc CRead}, where the expressions are evaluated atomically
in the premises.
%
Finally, \kw{newbuf} in \lamstar is translated to an array declaration
followed by a separate initialization. In \cstar, declaring an array
allocates a fresh memory block in the current memory frame, and makes
its memory locations available but uninitialized.  Memory write
(resp. read) produces a $\symwrite$ (resp. $\symread$)
event. $\memset{\cexp_1}{m}{\cexp_2}$ produces $m$ $\symwrite$ events,
and can be used only for arrays.

%% \paragraph{\lamstar to \cstar translation}

%% \tr{Figure disappeared, so the following paragraph can be shrunk.}

%% The compilation procedure is defined in Figure \ref{fig:lamtoc} as
%% inference rules, which should be read as functions defined by
%% pattern-matching. The compilation is a partial function, encoding
%% syntactic constraints on \lamstar programs that can be compiled. For
%% example, compilable \lamstar top-level functions must be wrapped in a
%% $\withframe$ construct.  Also, arguments to $\kw{newbuf}$, etc., or to
%% functions, and local definitions, must be pure (i.e. may contain
%% variables, but must not contain $\kw{if}$, $\kw{withframe}$ or any
%% kind of $\kw{let}$). Those syntactic restrictions are enforced\ch{met?} by
%% \lowstar-to-\lowstar transformations whose correctness proofs are left
%% as future work.

%% Given those syntactic restrictions, most of the translation rules are
%% straightforward. $\downarrow$ translates \lamstar expressions into
%% \cstar statement lists, whereas $\downlsquigarrow$ translates
%% variables and pure \lamstar expressions into \cstar
%% expressions. $\withframe$ is translated into a \cstar
%% block. $\kw{newbuf}$ is translated into an array declaration followed
%% by a $\kw{memset}$. The translation of functions $\downdownarrows$
%% requires that the translated function body be a list of \cstar
%% statements followed by a \cstar expression, so that the expression
%% serves as the returned result.

\paragraph{Correctness of the \lamstar-to-\cstar transformation}

We proved that execution traces are exactly preserved from
\lamstar to \cstar:

\begin{lemma}[\lamstar to \cstar] \label{lem:lamstar-to-cstar}
 Let $\lp$ be a \lamstar program and $\lexp$ be a \lamstar entry point
 expression, and assume that they compile: $\lowtocd(\lp) = \cp$ for
 some \cstar program $\cp$ and $\lowtoc(\lexp) = \ls{\cstmt}; \cexp$
 for some \cstar list of statements $\ls{\cstmt}$ and expression
 $\cexp$.
 
 Let $V$ be a mapping of local variables containing the initial values
 of secrets. Then, the \cstar program $\cp$ terminates with trace
 $\trace$ and return value $\lv$, i.e., $\cp \vdash ([], V,
 \ls{\cstmt}; \kw{return} ~ \cexp) \stackrel{\trace,\ast}{\rightarrow}
 ([], V', \kw{return} ~ \lv)$ if, and only if, so does the \lamstar
 program: $\lp \vdash (\{\}, \lexp[V])
 \stackrel{\trace,\ast}{\rightarrow} (H', \lv)$; and similarly for
 divergence.
\end{lemma}

In particular, if the source \lamstar program is safe, then so is the
target \cstar program. It also follows that the trace equality
security property is preserved from \lamstar to \cstar.
%
We prove this theorem by bisimulation. In fact, it is enough to prove
that any \lamstar behavior is a \cstar behavior, and flip the diagram
since \cstar is deterministic. That \cstar semantics use
big-step semantics for \cstar expressions complicates the bisimulation
proof a bit because \lamstar and \cstar steps may go out-of-sync at
times. Within the proof we used a relaxed notion of simulation
(``quasi-refinement'') that allows this temporary discrepancy by some
stuttering, but still implies bisimulation.


\subsection{From \cstar to CompCert Clight and Beyond}
\label{sec:to-clight}

CompCert Clight is a deterministic (up to system I/O) subset of C with
no side effects in expressions, and actual byte-level representation
of values. Clight has a realistic formal
semantics \cite{Blazy-Leroy-Clight-09,compcert-url} and tractable
enough to carry out the correctness proofs of our transformations
from \lamstar to C.
More importantly, Clight is the source language of the CompCert compiler
backend, which we can
thus leverage to preserve at least safety and functional correctness properties
of \lowstar programs down to assembly.\footnote{As a subset of C,
Clight can be compiled by any C compiler, but only CompCert provides
formal guarantees.}

Recall that we need to produce an event in the trace whenever a memory
location is read or written, and whenever a conditional branch is
taken, to account for memory accesses and statements in the semantics
of the generated Clight code for the purpose of our noninterference
security guarantees. However, the semantics of CompCert
Clight \emph{per se} produces no events on memory accesses; instead,
CompCert provides a syntactic program annotation mechanism using
no-op \emph{built-in calls}, whose only purpose is to add extra events
in the trace. Thus, we leverage this mechanism by prepending each
memory access and conditional statement in the Clight generated code
with one such built-in call producing the corresponding events.

The main two
differences between \cstar and Clight, which our translation deals
with as described below, are immutable local structures, and scope management
for local variables.

\paragraph{Immutable local structures}
\cstar handles immutable local structures as
first-class values, whereas Clight only supports non-compound
data (integers, floating-points or pointers) as values.

If we naively translate immutable local \cstar structures to C structures in
Clight, then CompCert will allocate them in memory.
This increases the number of memory accesses, which not only
introduces discrepancies in the security preservation proof from \cstar
to Clight, but also introduces significant performance
overhead compared to GCC, which optimizes away structures whose
addresses are never taken.

Instead, we split an immutable structure into the sequence of all its
non-compound fields, each of which is to be taken as a potentially
non-stack-allocated local variable,\footnote{Our benchmark without
this structure erasure runs 20\% slower than with structure erasure,
both with CompCert 2.7.
% on a 4-core Intel Core i7 1.70 GHz laptop.
% with 8 Gb RAM running Ubuntu 14.04.
Without structure erasure, code
generated with CompCert is 60\% slower than with
{\tt gcc -O1}. CompCert-generated code without structure erasure may even
segfault, due to stack overflow, which structure erasure successfully
overcomes.} except for functions that return structures, where, as
usual, we add, as an extra argument to the callee, a pointer to the
memory location written to by the callee and read by the caller.

\paragraph{Local variable hoisting}
Scoping rules for \cstar local arrays are not exactly the same as in
C, in particular for branches of conditional statements. So, it is
necessary to hoist all local variables to function-scope.  CompCert
2.7.1 does support such hoisting but as an unproven elaboration
step. While existing formal proofs (e.g., Dockins'
\cite[\S 9.3]{dockins-phd}) only prove functional
correctness, we also prove preservation of security guarantees, as
shown below.

\paragraph{Proof techniques}
Contrary to the \lamstar-to-\cstar transformation, our subsequent
passes modify the memory layout leading to differences in traces
between \cstar to Clight, due to pointer values. Thus, we need to
address security preservation separately from functional correctness.

   For each pass changing the memory layout, we split it into three
    passes. First, we \emph{reinterpret} the program by replacing each
    pointer value in event traces with the function name and recursion
    depth of its function call, the name of the corresponding local
    variable, and the array index and structure field name within this
    local variable. Then, we perform the actual transformation and
    prove that it exactly preserves traces in this new ``abstract''
    trace model. Finally, we reinterpret the generated code back to
    concrete pointer values.  We successfully used this technique to
    prove functional correctness and security preservation for
    variable hoisting.
    
  For each pass that adds new memory accesses, we split it into
    two passes. First, a reinterpretation pass produces new events
    corresponding to the provisional memory accesses (without actually
    performing those memory accesses). Then, this pass is followed by
    the actual trace-preserving transformation that goes back to the
    non-reinterpreted language but adds the actual memory accesses
    into the program.
    We successfully used this technique to prove functional
    correctness and security preservation for structure return, where
    we add new events and memory accesses whenever a \cstar function
    returns a structure value.

In both cases, we mean \emph{reinterpretation} as defining a new
language with the same syntax and small-step semantic rules except
that the produced traces are different, and relating executions of
the same program in the two languages. There, it is easy to prove
functional correctness, but for security preservation, we need to
prove an invariant on two small-step executions of the same program
with different secrets, to show that two equal pointer values in event
traces coming from those two different executions will actually turn
into two equal abstract pointer values in the reinterpreted language.

%% For reinterpretations, it is easy to prove functional correctness.
%% For security preservation, we prove an invariant on two small-step
%% executions of the same program with different secrets, so that they
%% produce the same trace in each of the two trace models before and
%% after reinterpretation.
%% For hoisting, reinterpretation replaces each pointer value in event
%% traces with the function name and recursion depth of its function
%% call, the name of the corresponding local variable, and the array
%% index and structure field name within this local variable.  For
%% structure return, reinterpretation produces new events corresponding
%% to the memory accesses that will be added by the actual
%% transformation, but without actually performing them at first in the
%% reinterpreted semantics.
Our detailed functional correctness and security preservation proofs
from \lamstar to Clight can be found in the appendix.

\paragraph{Towards verified assembly code}
We conjecture that our reinterpretation techniques can be generalized to most passes
of CompCert down to assembly. %% , since we believe that accesses to stack-allocated
%% variables are in most cases exactly preserved by compilation, and any
%% reordering, removal or addition of memory accesses (in the case of
%% register allocation and spilling, to constant offsets within the
%% stack) is made without introducing spurious tests, and thus in a
%% secret-independent way}
While we leave such generalization as future work, some guarantees
from C to assembly can be derived by instrumenting CompCert \cite{barthe-ccs2014}
and LLVM \cite{DBLP:conf/popl/ZhaoNMZ12,DBLP:conf/pldi/ZhaoNMZ13,almeida-usenix2016} 
and turning them into \emph{certifying} (rather than certified) compilers where
security guarantees are statically rechecked on the compiled code
through translation validation, thus re-establishing them
independently of source-level security proofs. In this case, rather
than being fully preserved down to the compiled code,
\lowstar-level proofs are still useful to \emph{practically} reduce the
risk of failures in translation validation.
%% By contrast, applying our
%% proof-preservation techniques to CompCert aims to avoid this further
%% compile-time check, eliminating the risk of compilation
%% failures.
\ch{Can we make this much shorter? it talks about applying
  something we didn't show at all to other layers.}


\ifpagelimits
\newpage
\fi

\section{KreMLin: a Compiler from \lowstar to C}
\label{sec:impl}

\subsection{From \lowstar to Efficient, Elegant C}

As explained previously, \lamstar is the core of \lowstar, post
erasure. Transforming \lowstar into \lamstar proceeds in several
stages. First, we rely on \fstar's existing normalizer and erasure and
extraction facility (similar to features in Coq~\citep{Letouzey08}),
to obtain an ML-like AST for \lowstar terms. Then, we use our new tool
KreMLin that transforms this AST further until it falls within the
\lamstar subset formalized above. KreMLin then performs the \lamstar
to \cstar transformation, followed by the \cstar to C transformation
and pretty-printing to a set of C files. KreMLin generates C11 code
that may be compiled by GCC; Clang; Microsoft's C compiler or
CompCert. We describe the main transformations performed by KreMLin,
beyond those formalized in \sref{formal}, next.

\paragraph{Structures by value}
We described earlier (\sref{structs}) our \lowstar struct library that grants the
programmer fine-grained control over the memory layout, as well as
mutability of interior fields. As an alternative, \kremlin supports immutable,
by-value structs. Such structures, being pure, come with no liveness proof obligations.
The performance of the generated C code, however, is less
predictable: in many cases, the C compiler will optimize and pass such structs
by reference, but on some ABIs (x86), the worst-case scenario may be costly.

Concretely, the \fstar programmer uses tuples and inductive
types. Tuples are monomorphized into specialized inductive
types. Then, inductive types are translated into idiomatic C code:
single-branch inductive types (e.g., records) become actual C structs,
inductives with only constant constructors become C enums, and other
inductives becomes C tagged unions, leveraging C11 anonymous unions
for syntactic elegance. Pattern matches become, respectively,
switches, let-bindings, or a series of cascading if-then-elses.

\paragraph{Whole-program transformations}
\kremlin perform a series of whole-program transformations. First, the
programmer is free to use parameterized type abbreviations. \kremlin substitutes
an application of a type abbreviation with its definition, since C's \li+typedef+ does
not support parameters. (C++11 alias templates would support this use-case.)
%
Second, \kremlin recursively inlines all \li+StackInline+ functions, as required for
soundness (cf. \sref{crypto}).
%
Third, \kremlin performs a reachability analysis. Any function that is not
reachable from the \li+main+ function or, in the case of a library, from a
distinguished API module, is dropped. This is essential for generating
palatable C code that does not contain unused helper functions used only for
verification.
%
Fourth, \kremlin supports a concept of ``bundle'', meaning that several \fstar
modules may be grouped together into a single C translation unit, marking all of
the functions as \li+static+, except for those reachable via the distinguished API
module. This not only makes the code much more idiomatic, but also triggers a
cascade of optimizations that the C compiler is unable to perform across
translation units.

\paragraph{Going to an expression language}
\fstar is, just like ML, an expression language. Two transformations are
required to go to a statement language: \emph{stratification} and
\emph{hoisting}. Stratification places buffer
allocations, assignments and conditionals in statement
position before going to \cstar. Hoisting, as discussed in
\sref{to-clight}, deals with the discrepancy between C99 block scope
and \lowstar \li[language={}]{with_frame}; a buffer allocated under a \li+then+
branch must be hoisted to the nearest enclosing \li+push_frame+,
otherwise its lifetime would be shortened by the resulting C99
block after translation.

\paragraph{Readability}
KreMLin puts a strong emphasis on generating readable C, in the hope that
security experts not familiar with \fstar can review the generated C code.
Names are preserved; we use \li+enum+ and \li+switch+ whenever possible;
functions that take \li+unit+ are compiled into functions with no parameters;
functions that return \li+unit+ are compiled into \li+void+-returning functions.
The internal architecture relies on an abstract C AST and what we believe is a
correct C pretty-printer.

\paragraph{Implementation}
KreMLin represents about 10,000 lines of OCaml, along with a minimal set of
primitives implemented in a few hundred lines of C. After \fstar has extracted and
erased the AEAD development, KreMLin takes less than a second to generate the
entire set of C files. The implementation of KreMLin is optimized for
readability and modularity; there was no specific performance concern in this
first prototype version. KreMLin was designed to support multiple backends; we
are currently implementing a WebAssembly backend to provide verified, efficient
cryptographic libraries for the web.

\subsection{Integrating \kremlin's Output}

\kremlin generates a set of C files that have
no dependencies, beyond a single \li+.h+ file and C11 standard headers, meaning
\kremlin's output can be readily integrated into an existing source tree.

To allow code sharing and re-use, programmers may want to generate a shared
library, that is, a \li+.dll+ or \li+.so+ file that can be distributed along
with a public header (\li+.h+) file. The programmer can achieve this by writing
a distinguished API module in \fstar, exposing only carefully-crafted function
signatures. As exemplified earlier (\fref{chacha20-both}), the translation is
predictable, meaning the programmer can precisely control, in \fstar, what
becomes, in C, the library's public header. The
bundle feature of \kremlin then generates a single C file for the library;
upon compiling it into a shared object, the only visible symbols are those
exposed by the programmer in the header file.

We used this approach for our \haclstar library. Our public header file exposes
functions that have the exact same signature as their counterpart in the NaCL
library. If an existing binary was compiled against NaCL's public header file,
then one can configure the dynamic linker to use our \haclstar library instead,
without recompiling the original program (using the infamous ``LD preload trick'').

The functions exposed by the library comply with the C ABI for the chosen toolchain.
This means that one may use the library from a variety of programming languages,
relying on foreign-function interfaces to interoperate. One popular approach is
to generate bindings for the C library \emph{at run-time} using the ctypes
and the \li+libffi+~\cite{libffi} libraries. This is an approach leveraged by
languages such as JavaScript, Python or OCaml, and requires no recompilation.

An alternative is to write bindings by hand, which allows for better performance
and control over how data is transformed at the boundary, but requires writing
and recompiling potentially error-prone C code. This is the historical way of
writing bindings for many languages, including OCaml. We plan to have \kremlin
generate these bindings automatically. We used this approach in miTLS,
effectively making it a mixed C/OCaml project. We intend to eventually lower all
of miTLS into \lowstar.

\section{Building Verified \lowstar Libraries and Applications}
\label{sec:moreexamples}
\newcommand\fixme[1]{{\color{red}{#1}}}
\begin{table}[ht] \centering
\begin{tabular}{|l|r|r|r|r|}
  \hline
  \multicolumn{1}{|c|}{Codebase} & \multicolumn{1}{c|}{LoC} & \multicolumn{1}{c|}{C LoC} & \multicolumn{1}{c|}{\%annot} & \multicolumn{1}{c|}{Verif. time} \\
  \hline
  \lowstar standard library& 8,936  &  N/A   &  N/A   & 8m \\\hline
  \haclstar                & 6,050  & 11,220 &  28\%  & 12h \\
  miTLS AEAD               & 13,743 & 14,292 & 56.5\% & 1h 10m \\
  \hline
\end{tabular}
\caption{Evaluation of verified \lowstar libraries and applications (time reported on an Intel Core E5 1620v3 CPU)}
\label{tab:applications}
\end{table}
%% (+ 8936 6050 13743) = 28,729

%% wc -l FStar.FunctionalExtensionality.fst FStar.PropositionalExtensionality.fst FStar.PredicateExtensionality.fst FStar.TSet.fst FStar.Heap.fst FStar.Set.fst FStar.Map.fst FStar.Squash.fst FStar.Classical.fst FStar.List.Tot.Base.fst FStar.List.Tot.Properties.fst FStar.List.Tot.fst FStar.HyperHeap.fst FStar.HyperStack.fst FStar.Ghost.fst FStar.ST.fst FStar.All.fst FStar.StrongExcludedMiddle.fst FStar.Seq.Base.fst FStar.Seq.Properties.fst FStar.Seq.fst FStar.Mul.fst FStar.BitVector.fst FStar.Math.Lib.fst FStar.Math.Lemmas.fst FStar.UInt.fst FStar.UInt32.fst FStar.Buffer.fst FStar.DependentMap.fst FStar.Int.fst FStar.Int8.fst FStar.Struct.fst FStar.UInt8.fst
%%     35 FStar.FunctionalExtensionality.fst
%%      2 FStar.PropositionalExtensionality.fst
%%     14 FStar.PredicateExtensionality.fst
%%    108 FStar.TSet.fst
%%     80 FStar.Heap.fst
%%    129 FStar.Set.fst
%%    140 FStar.Map.fst
%%     26 FStar.Squash.fst
%%    127 FStar.Classical.fst
%%    454 FStar.List.Tot.Base.fst
%%    854 FStar.List.Tot.Properties.fst
%%      3 FStar.List.Tot.fst
%%    293 FStar.HyperHeap.fst
%%    322 FStar.HyperStack.fst
%%     87 FStar.Ghost.fst
%%     56 FStar.ST.fst
%%     42 FStar.All.fst
%%      4 FStar.StrongExcludedMiddle.fst
%%    241 FStar.Seq.Base.fst
%%    803 FStar.Seq.Properties.fst
%%      3 FStar.Seq.fst
%%      4 FStar.Mul.fst
%%    112 FStar.BitVector.fst
%%    122 FStar.Math.Lib.fst
%%    627 FStar.Math.Lemmas.fst
%%    817 FStar.UInt.fst
%%    165 FStar.UInt32.fst
%%   1236 FStar.Buffer.fst
%%    211 FStar.DependentMap.fst
%%    158 FStar.Int.fst
%%    157 FStar.Int8.fst
%%   1338 FStar.Struct.fst
%%    166 FStar.UInt8.fst
%%   8936 total

%% Verified module: FStar.FunctionalExtensionality (191 milliseconds)
%% Verified module: FStar.PropositionalExtensionality (21 milliseconds)
%% Verified module: FStar.PredicateExtensionality (74 milliseconds)
%% Verified module: FStar.TSet (1156 milliseconds)
%% Verified module: FStar.Heap (568 milliseconds)
%% Verified module: FStar.Set (2224 milliseconds)
%% Verified module: FStar.Map (4292 milliseconds)
%% Verified i'face (or impl+i'face): FStar.Squash (36 milliseconds)
%% Verified module: FStar.Classical (2305 milliseconds)
%% Verified module: FStar.List.Tot.Base (4500 milliseconds)
%% Verified module: FStar.List.Tot.Properties (13010 milliseconds)
%% Verified module: FStar.List.Tot (15 milliseconds)
%% Verified module: FStar.HyperHeap (15689 milliseconds)
%% Verified module: FStar.HyperStack (13793 milliseconds)
%% Verified module: FStar.Ghost (1101 milliseconds)
%% Verified module: FStar.ST (10937 milliseconds)
%% Verified module: FStar.All (31 milliseconds)
%% Verified module: FStar.StrongExcludedMiddle (123 milliseconds)
%% Verified module: FStar.Seq.Base (12797 milliseconds)
%% Verified module: FStar.Seq.Properties (35056 milliseconds)
%% Verified module: FStar.Seq (19 milliseconds)
%% Verified module: FStar.Mul (27 milliseconds)
%% Verified module: FStar.BitVector (8754 milliseconds)
%% Verified module: FStar.Math.Lib (6568 milliseconds)
%% Verified module: FStar.Math.Lemmas (53978 milliseconds)
%% Verified module: FStar.UInt (72911 milliseconds)
%% Verified module: FStar.UInt32 (9854 milliseconds)
%% Verified module: FStar.Buffer (130905 milliseconds)
%% Verified module: FStar.DependentMap (3470 milliseconds)
%% Verified module: FStar.Int (6559 milliseconds)
%% Verified module: FStar.Int8 (10474 milliseconds)
%% Verified module: FStar.Struct (47486 milliseconds)
%% Verified module: FStar.UInt8 (11871 milliseconds)
%% All verification conditions discharged successfully

%% real    8m5.803s
%% user    0m0.000s
%% sys     0m0.045s
  
In this section, we describe two examples (summarized in
Table~\ref{tab:applications}) that show how \lowstar can be used to
build applications that balance complex verification goals with high
performance.
%
First, we describe \haclstar{}, an efficient library of cryptographic
primitives that are verified to be memory safe, side-channel
resistant, and, where there exists a simple mathematical
specification, functionally correct.
%
%Integrating \haclstar into a larger cryptographic application, we
%describe \emph{PneuTube}, an efficient, verified, secure file transfer
%application implemented in \lowstar.
%
Then, we show how to use \lowstar for type-based cryptographic
security verification by implementing and verifying the AEAD
construction in the Transport Layer Security (TLS) protocol.
%
We show how this \lowstar library can be integrated within miTLS, 
an \fstar implementation of TLS that is compiled to OCaml.

\subsection{\haclstar: A Fast and Safe Cryptographic Library}
\label{sec:haclstar}

%% In the wake of the Heartbleed vulnerability in OpenSSL, many concerns
%% have been raised about the quality of cryptographic implementations,
%% and indeed, numerous other memory safety (see e.g. CVE-2016-7054), functional
%% correctness~\cite{brumley2012practical}, and side-channel
%% issues~\cite{albrecht2016lucky} continue to be in found popular
%% cryptographic libraries.
%%NS: Already cited a ton of them in the intro

In the wake of numerous security vulnerabilities,
\citet{bernstein2012security} argue that libraries like OpenSSL are
inherently vulnerable to attacks because they are too large, offer too many obsolete options, and
expose a complex API that programmers find hard to use securely.
%
Instead they propose a new cryptographic API called NaCl that uses a
small set of modern cryptographic primitives, such as
Curve25519~\cite{curve25519} for key exchange, the Salsa family of
symmetric encryption algorithms~\cite{bernstein2008salsa20}, which
includes Salsa20 and ChaCha20, and Poly1305 for message
authentication~\cite{bernstein2005poly1305}.
%
These primitives were all designed to be fast and easy to implement
in a side-channel resistant coding style.
%
Furthermore, the NaCl API does not directly expose these low-level
primitives to the programmer. Instead it recommends the use of simple
composite functions for symmetric key authenticated encryption
(\texttt{secretbox}/\texttt{secretbox\_open}) 
and for public key authenticated encryption (\texttt{box}/\texttt{box\_open}).

The simplicity, speed, and robustness of the NaCl API has proved popular among 
developers. Its most popular implementation is Sodium~\cite{libsodium}, which has bindings
for dozens of programming languages and is written mostly in C, with a few components in assembly.
%
An alternative implementation called TweetNaCl~\cite{bernstein2014tweetnacl} seeks
to provide a concise implementation that is both readable and \emph{auditable} for 
memory safety bugs, a useful point of comparison for our work.
%
With \lowstar, we show how we can take this approach even further by
placing it on formal, machine-checked ground, without compromising
performance.

\paragraph*{A Verified NaCl Library}
We implement the NaCl API, including all its component algorithms, 
in a \lowstar library called \haclstar, mechanically verifying that 
all our code is memory safe, functionally correct,  and side-channel resistant.
%
The C code generated from \haclstar is ABI-compatible and can be used as a drop-in replacement for
Sodium or TweetNaCl in any application, in C or any other language, that relies on these libraries.
%
Our code is written and optimized for 64-bit platforms; on 32-bit
machines, we rely on a stub library for performing 64x64-bit 
multiplications and other 128-bit operations.

We implement and verify four cryptographic primitives:
ChaCha20, Salsa20, Poly1305, and Curve25519, and then use them to
build three cryptographic constructions: AEAD, \texttt{secretbox} 
and \texttt{box}. 
%
For all our primitives, we prove that our stateful optimized code
matches a high-level functional specification written in \fstar.
%
These are new verified implementations. Previously, \citet{saw-cryptol} used SAW
and Cryptol to verify C and Java implementations of Chacha20,
Salsa20, Poly1305, AES, and ECDSA. Using a different methodology, \citet{vale}
verifies an assembly version of Poly1305.
%
Curve25519 has been verified before:
%
\citet{chen2014verifying} verified an optimized 
low-level assembly implementation using an SMT solver;
%
\citet{ZBB16} wrote and verified a high-level
library of three curves, including Curve25519, in F* and generated an
OCaml implementation from it.
%
Our verified Curve25519 code explores a third direction by targeting 
reference C code that is both fast and readable.

A companion paper currently under review~\cite{haclstar} is entirely devoted to the \haclstar
library, and contains an in-depth evaluation of the proof methodology, several
new algorithms that were verified since the present paper was written, along
with a more comprehensive performance analysis.


\begin{table}[h]
  \footnotesize
%% \resizebox{\columnwidth}{!}{
\begin{tabular}{|l|r|r|r|r||r|}
  \hline
  \multicolumn{1}{|c|}{Algorithm} & \multicolumn{1}{c|}{\haclstar} & \multicolumn{1}{c|}{Sodium} & \multicolumn{1}{c|}{TweetNaCl} &\multicolumn{1}{c||}{OpenSSL} & \multicolumn{1}{c|}{eBACS Fastest}   \\
  \hline
  ChaCha20      & 6.17 cy/B  & 6.97 cy/B  & - & 8.04 cy/B & 1.23 cy/B\\
  Salsa20       & 6.34 cy/B  & 8.44 cy/B  & 15.14 cy/B & - & 1.39 cy/B\\
  Poly1305      & 2.07 cy/B  & 2.48 cy/B  & 32.32 cy/B & 2.16 cy/B & 0.68 cy/B \\
  Curve25519    & 157k cy/mul & 162k cy/mul  & 1663k cy/mul & 359k cy/mul & 145k cy/mul\\
  \hline
  AEAD-ChaCha20-Poly1305          & 8.37 cy/B & 9.60 cy/B  & - & 8.53 cy/B  & \\
  SecretBox     & 8.43 cy/B & 11.03 cy/B  & 50.56 cy/B & -  &  \\
  Box           & 18.10 cy/B & 20.97 cy/B  & 149.22 cy/B & -  &  \\
  \hline
\end{tabular}
%% }
\caption{Performance in CPU cycles: 64-bit \haclstar,  64-bit Sodium (pure C, no assembly), 
32-bit TweetNaCl, 64-bit OpenSSL (pure C, no assembly), and the fastest assembly implementation
included in eBACS SUPERCOP. All code was compiled using \texttt{gcc -O3} optimized and
run on a 64-bit Intel Xeon CPU E5-1630. Results are averaged over 1000 measurements, each
processing a random block of $2^{14}$ bytes; Curve25519 was averaged over 1000 random key-pairs.}
\label{tab:haclperf}
\end{table}

\begin{table}[h]
  \footnotesize
  \begin{tabular}{|l|r|r|r|}
    \hline
    \multicolumn{1}{|c|}{Algorithm} & \multicolumn{1}{c|}{\haclstar} & \multicolumn{1}{c|}{OpenSSL} & \multicolumn{1}{c|}{CNG} \\
    \hline
    Curve25519 \hspace{2em} & 17700 mul/s ($\sigma=246$) & 8033 mul/s ($\sigma=120$) & 7490 mul/s ($\sigma=114$) \\
    \hline
  \end{tabular}
  \caption{Performance in operations per second: 64-bit \haclstar, 64-bit
  OpenSSL (assembly disabled) and Microsoft's ``Crypto New Generation'' (CNG)
  library on a 64-bit Windows 10 machine. These results were obtained by writing
  an OpenSSL engine that calls back to either \haclstar, CNG, or OpenSSL
  itself (so as to include the overhead of going through a pluggable engine).
  The \li+speed ecdhx25519+ command runs multiplications for 10s, then counts
  the number of multiplications performed. We show the average over 10 runs of
  this command. The machine is a desktop machine with a 64-bit Intel Xeon CPU
  E5-1620 v2 nominally clocked at 3.70Ghz.}
\end{table}

\paragraph*{Performance}
%
Table~\ref{tab:haclperf} compares the performance of \haclstar to Sodium, TweetNaCl,
and OpenSSL by running each primitive on a 16KB input; we chose this size since it
corresponds to the maximum record size in TLS and represents a good balance
between small network messages and large files. 
%
We report averages over 1000 iterations expressed in cycles/byte.
%
For Curve25519, we measure the time taken for one call to scalar multiplication.
%
For comparison with state-of-the-art assembly implementations, for each primitive,
we also include the best performance for any implementation (assembly or C)
included in the eBACS SUPERCOP benchmarking framework.\footnote{\url{https://bench.cr.yp.to/supercop.html}} 
%
These fastest implementation are typically in architecture-specific assembly.

We performed these tests on a variety of 64-bit Intel CPUs (the most
popular desktop configuration) and these performance numbers were
similar across machines. To confirm these measurements, we also ran
the full eBACS SUPERCOP benchmarks on our code, as well as the OpenSSL
\texttt{speed} benchmarks, and the results closely mirrored
Table~\ref{tab:haclperf}. However, we warn the performance numbers 
could be quite different on (say) 32-bit ARM platforms.

We observe that for ChaCha20, Salsa20, and Poly1305, \haclstar achieves comparable performance 
to the optimized C code in OpenSSL and Sodium and significantly better performance than 
TweetNaCl's concise C implementation.
%
Assembly implementations of these primitives are about 3-4 times faster; they typically
rely on CPU-specific vector instructions and careful hand-optimizations.

Our Curve25519 implementation is about the same speed as Sodium's
64-bit C implementation (\texttt{donna\_c64}) and an order of magnitude
faster than TweetNaCl's 32-bit code.  It is also significantly faster
than OpenSSL because even 64-bit OpenSSL uses a Curve25519 implementation 
that was optimized for 32-bit integers, whereas the implementations in 
Sodium and \haclstar take advantage of the 64x64-bit multiplier available 
on Intel's 64-bit platforms.
%
The previous \fstar implementation of Curve25519~\cite{ZBB16} 
running in OCaml was not optimized for performance; it
is more than 100x slower than \haclstar.
%
The fastest assembly code for Curve25519 on eBACS is the one verified
in ~\cite{chen2014verifying}. This implementation is only 1.08x faster
than our C code, at least on the platform on which we tested, which
supported vector instructions up to 256 bits.  We anticipate that the
assembly code may be significantly faster on platforms that support
larger 512-bit vector instructions.
%


AEAD and \texttt{secretbox} essentially amount to a ChaCha20/Salsa20 cipher sequentially
followed by Poly1305, and their performance reflects the sum of the two primitives.
%
Box uses Curve25519 to compute a symmetric key, which it then uses to encrypt 
a 16KB input. Here, the cost of symmetric encryption dominates over Curve25519.

In summary, our measurements show that \haclstar is as fast as (or faster than)
state-of-the-art C crypto libraries and within a small factor of hand-optimized assembly code.
%
This finding is not entirely unexpected, since we wrote our \lowstar code 
by effectively porting the fastest C implementations to \fstar, and any
algorithmic optimization that is implemented in C can, in principle,
be written (and verified) in \lowstar.
%
What is perhaps surprising is that we get good performance even though
our \lowstar code, and consequently the generated C, heavily relies on 
functional programming patterns such as tail-recursion, and even though
we try to write generic compact code wherever possible, rather than trying 
to mimic the verbose inlined style of assembly code.
%
We find that modern compilers like GCC and CLANG are able to optimize
our code quite well, and we are able to benefit from their advancements,
without having to change our coding style.
%
Where needed, KreMLin helps the C compiler by inserting attributes like
\texttt{const}, \texttt{static} and \texttt{inline} that act as optimization hints.

\paragraph*{Balancing Trust and Performance}
%
All the above performance numbers were obtained with GCC-6 with most
architecture-specific optimizations turned on
(\texttt{-march=native}).  Consequently, any bug in GCC or its plugins
could break the correctness and security guarantees we proved in
\fstar for our source code.  For example, GCC has an auto-vectorizer
that significantly improves the performance of our ChaCha20 and
Salsa20 code in certain use cases, but does so by substantially
changing its structure to take advantage of the parallelism provided
by SIMD vector instructions. To avoid trusting this powerful 
but unverified mechanism, and for more consistent results across platforms,
we turned off auto-vectorization (\texttt{-fno-tree-vectorize}) 
for the numbers in Table~\ref{tab:haclperf}. For similar reasons, 
we turned off link-time optimization (\texttt{-fno-lto}) since 
it relies on an external linker plugin, and can change the 
semantics of our library every time it is linked with a new application.

Ideally, we would completely remove the burden of trust on the C
compiler by moving to CompCert, but at significant performance
cost. Our Salsa20 and ChaCha20 code incurs a relatively modest 3x
slowdown when compiled with CompCert 3.0 (with \texttt{-O3}). 
However, our Poly1305 and Curve25519 code incurs a 30-60x slowdown,
which makes the use of CompCert impractical for our library.
We anticipate that this penalty will reduce as CompCert
improves, and as we learn how to generate C code that would be
easier for CompCert to optimize. For now, we continue to use GCC and
CLANG and comprehensively test the generated code using third-party
tools. For example, we test our code against other implementations,
and run all the tests packaged with OpenSSL. We also 
test our compiled code for side-channel leaks using tools like 
DUDECT.\footnote{\url{https://github.com/oreparaz/dudect}}


%% \begin{figure}
%% \centerline{\includegraphics[width=0.8\columnwidth]{pneutube}}
%% \caption{PneuTube: Asynchronous file transfer with end-to-end confidentiality and 
%%   authentication for file contents and metadata (file name, size, etc.).}
%% \label{fig:pneutube}
%% \end{figure}
\paragraph*{PneuTube: Fast encrypted file transfer}
Using \haclstar, we can build a variety of high-assurance security
applications directly in \lowstar.
%
PneuTube is a \lowstar program that securely transfers files from a host $A$ to a
host $B$ across an untrusted network.
%
Unlike classic secure channel protocols like TLS and SSH, PneuTube is
\emph{asynchronous}, meaning that if $B$ is offline, the file may be
cached at some untrusted cloud storage provider and retrieved later. 

PneuTube breaks the file into \emph{blocks} and encrypts each block
using the \texttt{box} API in \haclstar (with an optimization that
caches the result of Curve25519).
%
It also protects file metadata, including the file name and
modification time, and it hides the file size by padding the file
before encryption to a user-defined size.
%
We verify that our code is memory-safe, side-channel resistant, and that
it uses the I/O libraries correctly (e.g., it only reads or writes
a file or a socket between calling open and close).

PneuTube's performance is determined by a combination of the crypto
library, disk access (to read and write the file at each end) and  network I/O. 
%
Its aynchronous design is particularly rewarding on high-latency
network connections, but even when transferring a 1GB file from one
TCP port to another on the same machine, PneuTube takes just 6s.
%
In comparison, SCP (using SSH with ChaCha20-Poly1305) takes 8 seconds. 



\iffalse
%ADL what do you want to do with this?
\begin{lstlisting}
val crypto_box_easy:
  c:uint8_p -> (* To store the ciphertext *)
  m:uint8_p -> (* Plaintext *)
  mlen:u64{...}  -> (* Plaintext length *)
  n:uint8_p{...} -> (* Nonce *)
  pk:uint8_p{...} -> (* Public Key of the other party *)
  sk:uint8_p{...} -> (* Our private key *)
  Stack u32
    (requires (fun h -> crypto_box_pre h c m mlen c pk sk))
    (ensures  (fun h0 z h1 -> crypto_box_post h0 z h1 c m
                                          mlen c pk sk))
\end{lstlisting}
\fi

\subsection{Cryptographically Secure AEAD for miTLS}
\label{sec:aead}

We use our cryptographically secure AEAD library (\S\ref{sec:crypto})
within miTLS~\cite{mitls}, an existing implementation of TLS in \fstar.
%
In a previous verification effort, AEAD encryption was idealized as a
cryptographic assumption (concretely realized using bindings to
OpenSSL) to show that miTLS implements a secure authenticated channel.
%
However, given vulnerabilities such as CVE-2016-7054, this AEAD
idealization is a leap of faith that can undermine security when the
real implementation diverges from its ideal behavior.

%% Recall from \S\ref{sec:crypto} that \fstar can be used to prove cryptographic
%% security by encoding cryptographic assumptions as idealized interfaces, and
%% by reasoning on the properties of the ideal functionality.
%% %
%% This approach has been successfully applied to large-scale projects such as
%% miTLS~\cite{mitls}. In miTLS, AEAD encryption
%% is idealized as a cryptographic assumption (concretely realized using bindings
%% to OpenSSL) to show that TLS implements a secure authenticated channel.
%% %
%% As shown by vulnerabilities such as CVE-2016-7054, this AEAD idealization is
%% a major leap of faith that can undermine the guarantees of miTLS when the
%% real implementation diverges from its ideal behavior.
%% %
%% We replace the AEAD idealization with a provably secure AEAD construction
%% that idealizes the pseudo-random function (which uses our verified ChaCha20
%% implementation concretely) and the authentication tag computation
%% (which uses our verified Poly1305 implementation concretely).
%% %
%% Although the details of this security proof are intricate and are detailed in a
%% separate submission (see \S\ref{sec:related}), we emphasize that \lowstar
%% lets one reason about the ideal
%% behavior of concrete low-level implementations that manage various kinds
%% of state (ghost state for the ideal functionality, heap state of the PRF,
%% stack state of input and output buffers). 

We integrated our verified AEAD construction within miTLS at two
levels~\cite{record}.
%
First, we replace the previous AEAD idealization with a module that
implements a similar ideal interface but translates the state and
buffers to \lowstar representations.
%
This reduces the security of TLS to the PRF and MAC idealizations in AEAD.
%
We integrate AEAD at the C level by substituting
the OpenSSL bindings with bindings to the C-extracted version of AEAD.
%
This introduces a slight security gap, as a small adapter that
translates miTLS bytes to \lowstar buffers and calls into AEAD in C is
not verified.
%
We confirm that miTLS with our verified AEAD interoperates with
mainstream implementations of TLS 1.2 and TLS 1.3 on ChaCha20-Poly1305
ciphersuites.

%% %
%% The OCaml-extracted version only achieves a 74 KB/s throughput on a
%% 1GB file transfer. Thus, while it is useful as a reference for
%% verification, it is not usable in practice.
%% %
%% In C, we achieve a 69 MB/s throughput, which is about 20\% of the 354
%% MB/s obtained with OpenSSL.
%% %
%% There is a noticeable cost stemming from the AEAD proof
%% infrastructure, such as indirections needed for algorithmic agility.
%% %
%% Better specialization support in KreMLin may help reduce this
%% overhead.

\iffalse
%ADL Inlining this table for now as we don't report AES
\begin{table}[ht]
\resizebox{\columnwidth}{!}{
\begin{tabular}{|l|c|c|c|}
  \hline
   & AEAD (ML) & AEAD (C) & OpenSSL \\
  \hline
  ChaCha20-Poly1305 & 74 KB/s & 69 MB/s & 354 MB/s \\
  \hline
\end{tabular}
}
  \caption{miTLS Throughput Comparison}
\label{tab:tls}
\vspace{-5mm}
\end{table}
\fi

\iffalse

\subsection{PneuTube}
Our last application, PneuTube, combines cryptographic security verification
and native levels of performance.
%


Our first verification goal is file integrity and sender authentication:
if $B$ receives a file (ostensibly) from $A$, with metadata $m$ and
contents $c$, then $A$ must have sent this file to $B$.
%
In other words, an attacker who controls the network and the
cloud storage provider cannot tamper with files en route.
%
Our second goal is end-to-end confidentiality: the attacker cannot distinguish
between the file contents and a random stream of data of the same length.
%
PneuTube also protects the metadata: it encrypts the file name and
modification time and it hides the file size by padding the file before
encryption.
%
Hence, PneuTube provides better traffic analysis protection than even SSH,
which leaks file length.

\begin{figure}
\centerline{\includegraphics[width=0.8\columnwidth]{pneutube-arch}}
\caption{PneuTube Code Structure}
\label{fig:pneutube-arch}
\end{figure}

The modular structure of PneuTube is depicted in Figure~\ref{fig:pneutube-arch}.
%
The \li$PneuTube$ module uses trusted C libraries (\li$FileIO$, \li$SocketIO$)
for accessing the file system and TCP sockets, and it uses the \haclstar library 
for boxing and unboxing blocks of data.
%
We verify our security goals by type-checking \li$PneuTube$ against the typed
interfaces for these modules, and by relying on the cryptographic assumption
that \haclstar implements public key authenticated encryption (PKAE).
%
Unlike \S\ref{sec:aead}, we do not prove that the box construction of
\haclstar is PKAE-secure based on lower-level assumptions.

We evaluate PneuTube by compiling it to C and by downloading a 1GB
file over the local network. This takes about 5s with PneuTube, while
transferring the same file over SSH (using scp) takes about 8s. 
\fi

\iffalse
% ADL leftover text?
Leveraging on our previous PKAE example, we build a secure file transfer application that
consumes our \textit{box} API. We model typestate by
tying the state of a TCP socket to the current (monadic) heap.

\begin{lstlisting}
module Tcp
val close: s:socket -> Stack bool
  (requires (fun h0 -> current_state h0 s = Open))
  (ensures  (fun _ r h1 -> r ==> current_state h1 s = Closed))
\end{lstlisting}
\fi

\section{Related Work}
\label{sec:related}

\ch{Do we need to relate to Fiat Crypto.
  Independent parallel work.
  \url{https://github.com/mit-plv/fiat-crypto}
  \url{https://people.csail.mit.edu/jgross/personal-website/papers/2017-fiat-crypto-pldi-draft.pdf}
}

\ch{Idris also compiles to C. Want to relate?
  \url{http://docs.idris-lang.org/en/latest/reference/codegen.html}}

Many approaches have been proposed for verifying the functional
correctness and security of efficient low-level code.
%
A first approach is to build verification frameworks for C using
verification condition generators and SMT solvers~\cite{Kirchner:2015,
  cohen2009vcc, verifast}.
%
While this approach has the advantage of being able to verify existing
C code, this is very challenging: one needs to deal with the
complexity of C and with any possible optimization trick in the book.
%
Moreover, one needs an expressive specification language and escape
hatches for doing manual proofs in case SMT automation fails.
%
So others have deeply embedded C, or C-like languages, into proof
assistants such as Coq~\cite{beringer2015verified,
  Appel15, ChenWSLG16} and Isabelle~\cite{WinwoodKSACN09,
  Schirmer2006} and built program logics and verification
infrastructure starting from that.
%
This has the advantage of using the full expressive power of the proof
assistant for specifying and verifying properties of low-level programs.
%
This remains a very labor-intensive task though, because C programs
are very low-level and working with a deep embedding is often cumbersome.
%
Acknowledging that uninteresting low-level reasoning was a determining
factor in the size of the seL4 verification effort~\cite{Klein09sel4:formal},
\citet{GreenawayLAK14, GreenawayAK12} have recently
proposed sophisticated tools for automatically abstracting the
low-level C semantics into higher-level monadic specifications to ease reasoning.
%
We take a different approach: we give up on verifying existing C code
and embrace the idea of writing low-level code in a subset of C
shallowly embedded in \fstar{}.
%
This shallow embedding has significant advantages in terms of reducing
verification effort and thus scaling up verification to larger programs.
%
This also allows us to port to C only the parts of an \fstar program
that are a performance bottleneck, and still be able to verify the
complete program.

\ch{Can we use the \%annot column to quantitatively support the
  ``reducing verification effort'' claim?}

% \ch{Don't think that the latest Isabelle
% work~\cite{GreenawayLAK14, GreenawayAK12} classifies as a deep embedding. They
% present a tool that automatically abstracts low-level C semantics
% into higher level specifications = a shallow monadic embedding. The
% aim is to generate a verified, human-readable specification,
% convenient for further reasoning.'' So what they do goes in
% the opposite direction from what we do. They perform control-flow
% abstraction, heap abstraction, and word to integers abstraction.}

% \ch{Now intro starts with this}
% Finally, one last strand of work uses a radically different approach,
% and relies on a high-level, memory-safe languages such as OCaml for
% writing the cryptographic protocol implementations, and OCaml's
% foreign function interface (FFI) for integrating within existing
% codebases~\cite{kaloper2015not, bhargavan2014proving, LochbihlerZ14}.
% A typical concern is the presence of side-channels due to the use of a
% garbage-collector. Furthermore, since overall performance depends so
% much on the underlying cryptographic primitives, ironically these
% high-level implementations link to insecure C code such as OpenSSL
% libcrypto to perform the actual cryptographic operations.

Verifying the correctness of low-level cryptographic code is
receiving increasing attention~\cite{Appel15, cryptol-s2n,
  beringer2015verified}.
%
The verified cryptographic applications we have written in \lowstar
and use for evaluation in this paper are an order of magnitude larger
than most previous work.
%
Moreover, for AEAD we target not only functional correctness, but also
cryptographic security.

% \ch{Could relate this to the Appel et al proof}
% %
% Galois develops the Cryptol DSL~\cite{cryptol} for specifying crypto
% algorithms and the SAW tool for verifying them against C and Java
% implementations~\cite{saw}.
% % (and at least originally, also compiling such specifications to VHDL).
% SAW uses symbolic execution to recover a formal model of the program
% and then verification condition generation to prove equivalence of models.
% %
% Cryptol and SAW were recently used~\cite{cryptol-s2n} to prove the
% correctness of the C implementation of HMAC in Amazon's s2n TLS
% library~\cite{s2n}.\ch{That's only 100 lines of code though!}

In order to prevent the most devastating low-level attacks, several researchers
have advocated dialects of C equipped with type systems for memory safety
\cite{condit2007dependent, jim2002cyclone, Tarditi16}.
% sound\ch{Added Checked C to the list,
%   although that's not sound. Do we really want to focus on the sound part;
%   all these systems have escape hatches that break soundness, no?}
Others have designed new languages with type
systems aimed at low-level programming, including for instance linear
types as a way to deal with memory
management~\cite{amani2016cogent, matsakis2014rust}. One drawback is
the expressiveness limitations of such type systems:
%the ``complexity wall'' of the type system
once memory safety relies on more complex invariants than these type
systems can express, compromises need to be made, in terms of
verification or efficiency.
%
\lowstar can perform arbitrarily sophisticated reasoning to establish
memory safety, but does not enjoy the benefits of efficient decision
procedures~\cite{rust} and currently cannot deal with concurrency.

% \ch{Merged with previous para}
% Rust~\cite{rust} is an increasingly popular systems programming language. Its type
% system aims to guarantee memory and thread safety, while imposing no
% performance penalty. Programs that fit within Rust's {\em lifetime}
% type discipline are memory safe. We can perform arbitrarily
% sophisticated reasoning to establish memory safety, but do not enjoy
% the benefits of the lifetime automated decision procedure. We also do
% not deal with concurrency in \lowstar.
We are not the first to propose writing efficient and verified C code
in a high-level language. LMS-Verify~\cite{lms-verify} recently
extended the LMS meta-programming framework for Scala with
support for lightweight verification. Verification
happens at the generated C level, which has the advantage of taking
the code generation machinery out of the TCB, but has the disadvantage
of being far away from the original source code.
% This has serious scalability consequences:
% our verification experiments are orders of magnitude larger than the
% HTTP parser verified so far with LMS-Verify, although we deal with functional
% correctness and security, not just memory safety.\ch{Need to be more
%   careful with this since Nada is on the ERC!?}

Bedrock~\cite{chlipala2013bedrock} is a generative meta-programming
tool for verified low-level programming in Coq.
%
The idea is to start from assembly and build up structured code
generators that are associated verification condition generators.
%
The main advantage of this ``macro assembly language'' view of
low-level verification is that no performance is sacrificed while
obtaining some amount of abstraction.
%
One disadvantage is that the verified code is not portable.

\ch{Could also compare to Ironclad / Ironfleet, Compiler from Dafny to
  x86 -- translation validation ... their high-level specs are at
  least checked before translating; they do some crypto too}
% Spartan. -- parallel submission, won't relate to it

Our companion paper ``Implementing and Proving
the TLS 1.3 Record Layer''~\cite{record} is available online.
It describes a cryptographic model and
proof of security for AEAD using a combination of \fstar verification
and meta-level cryptographic idealization arguments. To make the point that
verified code need not be slow, the paper mentions that the
AEAD implementation can be ``extracted to C using an experimental backend
for \fstar'', but makes no further claims about this backend. The
current work introduces the design, formalization,
implementation, and experimental evaluation of this C backend for \fstar.
%
% > Thanks for letting me know of your concurrent submission. The
% papers are clearly related at some level. In evaluating the paper,
% we will take that into account. I see that the record.pdf submission
% is blinded (i.e. author names are removed), thus, reviewers can
% potentially take a look.
% > One thing you can do to clarify the
% relationship you described in your mail is to take the record.pdf
% and make it available as a supplementary material, and perhaps
% describe the differences, as you did in the mail, directly in the
% related work of the paper. This will make it easier for us.

% {\scriptsize
% \begin{verbatim}
% Related work, compare to:
% 1. deep embedding in proof assistants
%    and extrinsic proofs of deep properties
% 2. Hoare-style verifiers for C programs:
%    VCC, Verifast
%    + they verify *existing* C programs
%    + embrace as much as C as possible
%    + try to do all their proofs by VC generation
% 3. Dependently typed: Deputy, Cyclone etc.
%    + C with a type system for simple invariants;
%      focus is on memory safety
%    + the reason this failed is that you hit a wall of complexity; for
%      which simple invariants are not enough; no such problem with F*
% 2. fiat crypto, Gallois, DeepSpec
%   + verify low-level directly
%   + struggling with the low-level
% \end{verbatim}
% }

\comment{
old ideas of compiling ML to Rust, compare to optimizing compilers
(e.g. OCaml), etc. esp. how we compare to OCaml (extra bit, no pointer
arithmetic).
}\ch{Not sure what this is about}


% Could also compare to WysStar, which is another security-verification
% oriented DSL on top of \fstar
% \url{https://research.microsoft.com/en-us/um/people/nswamy/papers/wys.pdf}

\section{Conclusion}
% \vspace{-0.5em}
\section{Conclusion}
% \vspace{-0.5em}
Recent advances in multimodal single-cell technology have enabled the simultaneous profiling of the transcriptome alongside other cellular modalities, leading to an increase in the availability of multimodal single-cell data. In this paper, we present \method{}, a multimodal transformer model for single-cell surface protein abundance from gene expression measurements. We combined the data with prior biological interaction knowledge from the STRING database into a richly connected heterogeneous graph and leveraged the transformer architectures to learn an accurate mapping between gene expression and surface protein abundance. Remarkably, \method{} achieves superior and more stable performance than other baselines on both 2021 and 2022 NeurIPS single-cell datasets.

\noindent\textbf{Future Work.}
% Our work is an extension of the model we implemented in the NeurIPS 2022 competition. 
Our framework of multimodal transformers with the cross-modality heterogeneous graph goes far beyond the specific downstream task of modality prediction, and there are lots of potentials to be further explored. Our graph contains three types of nodes. While the cell embeddings are used for predictions, the remaining protein embeddings and gene embeddings may be further interpreted for other tasks. The similarities between proteins may show data-specific protein-protein relationships, while the attention matrix of the gene transformer may help to identify marker genes of each cell type. Additionally, we may achieve gene interaction prediction using the attention mechanism.
% under adequate regulations. 
% We expect \method{} to be capable of much more than just modality prediction. Note that currently, we fuse information from different transformers with message-passing GNNs. 
To extend more on transformers, a potential next step is implementing cross-attention cross-modalities. Ideally, all three types of nodes, namely genes, proteins, and cells, would be jointly modeled using a large transformer that includes specific regulations for each modality. 

% insight of protein and gene embedding (diff task)

% all in one transformer

% \noindent\textbf{Limitations and future work}
% Despite the noticeable performance improvement by utilizing transformers with the cross-modality heterogeneous graph, there are still bottlenecks in the current settings. To begin with, we noticed that the performance variations of all methods are consistently higher in the ``CITE'' dataset compared to the ``GEX2ADT'' dataset. We hypothesized that the increased variability in ``CITE'' was due to both less number of training samples (43k vs. 66k cells) and a significantly more number of testing samples used (28k vs. 1k cells). One straightforward solution to alleviate the high variation issue is to include more training samples, which is not always possible given the training data availability. Nevertheless, publicly available single-cell datasets have been accumulated over the past decades and are still being collected on an ever-increasing scale. Taking advantage of these large-scale atlases is the key to a more stable and well-performing model, as some of the intra-cell variations could be common across different datasets. For example, reference-based methods are commonly used to identify the cell identity of a single cell, or cell-type compositions of a mixture of cells. (other examples for pretrained, e.g., scbert)


%\noindent\textbf{Future work.}
% Our work is an extension of the model we implemented in the NeurIPS 2022 competition. Now our framework of multimodal transformers with the cross-modality heterogeneous graph goes far beyond the specific downstream task of modality prediction, and there are lots of potentials to be further explored. Our graph contains three types of nodes. while the cell embeddings are used for predictions, the remaining protein embeddings and gene embeddings may be further interpreted for other tasks. The similarities between proteins may show data-specific protein-protein relationships, while the attention matrix of the gene transformer may help to identify marker genes of each cell type. Additionally, we may achieve gene interaction prediction using the attention mechanism under adequate regulations. We expect \method{} to be capable of much more than just modality prediction. Note that currently, we fuse information from different transformers with message-passing GNNs. To extend more on transformers, a potential next step is implementing cross-attention cross-modalities. Ideally, all three types of nodes, namely genes, proteins, and cells, would be jointly modeled using a large transformer that includes specific regulations for each modality. The self-attention within each modality would reconstruct the prior interaction network, while the cross-attention between modalities would be supervised by the data observations. Then, The attention matrix will provide insights into all the internal interactions and cross-relationships. With the linearized transformer, this idea would be both practical and versatile.

% \begin{acks}
% This research is supported by the National Science Foundation (NSF) and Johnson \& Johnson.
% \end{acks}

\section*{Acknowledgments}
We thank the anonymous reviewers for their excellent reviews. We also thank
Abhishek Anand and Mike Hicks, for useful feedback and discussion which helped
shape the work presented here, as well as Armaël Guéneau, for his work on a
mechanized proof.
%
J-K. Zinzindohou\'e and K. Bhargavan received funding from the
European Research Council (ERC) under the European Union’s Horizon
2020 research and innovation programme (grant agreement no. 683032 - CIRCUS).
%
C. Hri\c{t}cu was in part supported by the European Research Council
under ERC Starting Grant SECOMP (715753).

%%%%%%%%%%%%%%%%%%%%%%%%%%%%%%%%%%%%%%%%%%%%%%%%%%%%%%%%%%%%%%%%%%%%%%%%%%%%%%%%

\iflong
\newpage
\appendix

%% taramana 2016-11-14: from \lamstar to \cstar
\section{Big-stepping a small-step semantics}

Actual observable behaviors will not account for the detailed
sequence of small-step transitions taken. Given an execution first
represented as the sequence of its transition steps from the initial
state, we follow CompCert to derive an observable behavior by only
characterizing termination or divergence and collecting the event
traces, thus erasing all remaining information about the execution
(number of transition steps, sequence of configurations, etc.)  by
\emph{big-stepping} the small-step semantics as shown in
Figure~\ref{fig:bigstep}.

\begin{figure}[!htbp]
  \begin{mathpar}
  \arraycolsep=1.4pt\def\arraystretch{3.2}

\inferrule*
{
  s_0 \stackrel{t_0}{\rightarrow} s_1 \stackrel{t_1}{\rightarrow} \dots \stackrel{t_{n-2}}{\rightarrow} s_{n-1}\stackrel{t_{n-1}}{\rightarrow} s_n \\
  s_0 ~ \text{initial} \\
  s_n ~ \text{final with return value} ~ r \\
  t = t_0 ; t_1 ; \dots ; t_{n-2} ; t_{n-1}
}
{
  \kw{Terminates}(t, r)
}

\\

\inferrule*
{
  s_0 \stackrel{t_0}{\rightarrow} s_1 \stackrel{t_1}{\rightarrow} \dots \\
  s_0 ~ \text{initial} \\
  T = t_0 ; t_1 ; \dots
}
{
  \kw{Diverges}(T)
}

\\

\inferrule*
{
  s_0 \stackrel{t_0}{\rightarrow} s_1 \stackrel{t_1}{\rightarrow} \dots \stackrel{t_{n-2}}{\rightarrow} s_{n-1}\stackrel{t_{n-1}}{\rightarrow} s_n \\
  s_0 ~ \text{initial} \\
  s_n ~ \text{not final} \\
  t = t_0 ; t_1 ; \dots ; t_{n-2} ; t_{n-1}
}
{
  \kw{GoesWrong}(t)
}
\end{mathpar}
\caption{Big-stepping a small-step semantics}
\label{fig:bigstep}
\end{figure}

\section{\cstar and \lamstar Definition}

Notations used in the document are summarized in Figure \ref{fig:notations}.
Function name $f$ and variable name $x$ are of different syntax classes. A term
is closed if it does not contain unbound variables (but can contain function
names). The grammar of \cstar and \lamstar are listed in Figure \ref{fig:cstar-syntax}
and \ref{fig:lowstar-syntax} respectively. \cstar syntax is defined in such a way
that \cstar expressions do not have side effects (but can fail to evaluate because
of e.g. referring to a nonexistent variable). Locations, which only appear during reduction, consist
of a block id, an offset and a list of field names (a ``field path''). The
``getting field address'' syntax $\eptrfd{e}{fd}$ is for constructing a pointer
to a field of a struct pointed to by pointer $e$.

In \lamstar syntax, buffer allocation, buffer write and function application (as well as mutable struct allocation, mutable struct write) are
distinctive syntax constructs (not special cases of let-binding). In this way we
force effectful operations to be in let-normal-form, to be aligned with \cstar (\cstar
does not allow effectfull expressions because of C's nondeterministic expression
evaluation order). Let-binding and anonymous let-binding are also distinctive
syntax constructs, because they need to be translated into different \cstar
constructs. Locations and $\epop\;le$ only appear during reduction.

The operational semantics of \cstar is listed in Figure \ref{fig:cstar-expr-eval}
and \ref{fig:cstar-stmts-reduction}. Because C expressions do not have a
deterministic evaluation order, in \cstar we use a mixed big-step/small-step
operational semantics, where \cstar expressions are evaluated with big-step
semantics defined by the evaluation function (interpreter) $\eval{e}{(p,V)}$,
while \cstar statements are evaluated with small-step semantics. Definitions used in
\cstar semantics are summarized in Figure \ref{fig:cstar-semantics-defs}. A \cstar
evaluation configuration $C$ consist of a stack $S$, a variable assignment $V$
and a statement list $ss$ to be reduced. A stack is a list of frames. A frame
$F$ includes frame memory $M$, variable assignment $V$ to be restored upon
function exit, and continuation $E$ to be restored upon function exit. Frame
memory $M$ is optional: when it is none, the frame is called a ``call frame'';
otherwise a ``block frame''. A frame memory is just a partial map from block ids
to value lists.

Both \cstar and \lamstar reductions generate traces that include memory read/write with
the address, and branching to true/false. Reduction steps that don't have these
effects are silent.

\begin{figure}[!htbp]
\begin{center}
  \begin{tabularx}{\columnwidth}{lRlR}
    $\ls{a}$ & list &
    $\option{a}$ & option $a$ \\
    $\None$ & None &
    $\Some{a}$ & Some a \\
    $n$ & integer &
    $x$ & variable name \\
    $f$ & function name &
    $fd$ & field name \\
    $a\sympartial b$ & partial map & $\{\}$ & empty map \\
    $\{x\mapsto a\}$ & singleton map & $m[x\mapsto a]$ & map update \\
    $[]$ & empty list & $a;b$ & list concat or cons \\
    $\subst{x}{a}{b}$ & substitute $a$ for $x$ in $b$ & & \\
  \end{tabularx}
\end{center}
\caption{Notations}
\label{fig:notations}
\end{figure}

\subsection{\cstar Definition}

The following are the definitions of the syntax and operational semantics of \cstar.

\begin{figure}[!htbp]
\begin{center}
  \begin{tabularx}{\columnwidth}{rlR}
    $p ::= $ & & program \\
      & $\ls d$                      & series of declarations \\[1.2mm]

    $d ::= $ & & declaration \\
      & $\ecfun fxtt{ss}$                & top-level function \\
    & $\evardecl txv$               & top-level value \\
    [1.2mm]

    $ss ::= $ & & statement lists \\
    & $\ls{s}$                  &  \\
    [1.2mm]

    $s ::= $ & & statements \\
    & $\evardecl txe$                  & immutable variable declaration \\
    & $\earray txn$                & array declaration \\
    & $\memset{e}{n}{e}$                & memory set \\
    & $\evardecl{t}{x}{\eapply fe}$    & application \\
    & $\evardecl{t}{x}{\eread e}$      & read \\
    & $\ewrite ee$               & write \\
    & $\eif{e}{ss}{ss}$ & conditional \\
    & \{\stmts\} &  block \\
    & e & expression \\
    & \ereturn e & return \\
    [1.2mm]

    $e ::= $ & & expressions \\
    & $n$    & integer constant \\
    & $()$ & unit value \\
      %% & $\ecifthenelse eee$           & conditional \\
      %% & $\esequence {\vec s}e$        & sequence \\
      %% & $\eapply fe$                  & application \\
      & $x$                           & variable \\
      & $e_1+e_2$                        & pointer add ($e_1$ is a pointer and $e_2$ is an $\kw{int}$) \\
    & $\{\ls{fd=e}\}$ & struct \\
    & $e.fd$ & struct field projection \\
    & $\eptrfd{e}{fd}$ & struct field address ($e$ is a pointer) \\
    & $loc$                        & location \\
    [1.2mm]

    $loc ::= $ & & locations \\
    & $(b, n, \ls{fd})$ &  \\

  \end{tabularx}
\end{center}
\caption{\cstar Syntax}
\label{fig:cstar-syntax}
\end{figure}


\begin{figure}[!htbp]
\begin{small}
\begin{center}
  \begin{tabularx}{\columnwidth}{rlR}

    $v ::= $ & & values \\
    & $n$    & constant \\
    & $()$ & unit value \\
    & $\{\ls{fd=v}\}$ & constant struct \\
    & $loc$                        & location \\
    [1.2mm]

    $E ::=$ & & evaluation ctx (plug expr to get stmts) \\
    & $\symhole; ss$ & discard returned value \\
    & $t\;x=\symhole; ss$ & receive returned value \\
    [1.2mm]

    $F ::=$ & & frames \\
    & $(\None, V, E)$ & call frame \\
    & $(\Some{M}, V, E)$ & block frame \\
    [1.2mm]

    $M ::=$ & & memory \\
    & $b\sympartial \ls{\option{v}}$ & map from block id to list of optional values \\
    [1.2mm]

    $V ::=$ & & variable assignments \\
    & $x\sympartial v$ & map from variable to value \\
    [1.2mm]

    $S ::=$ & & stack \\
    & $\ls{F}$ & list of frames \\
    [1.2mm]

    $C ::=$ & & configuration \\
    & $(S, V, ss)$ &  \\
    [1.2mm]

    $l ::=$ & & label \\
    & $\symread\;loc$ & read \\
    & $\symwrite\;loc$ & write \\
    & $\brt$ & branch true \\
    & $\brf$ & branch false \\

  \end{tabularx}
\end{center}
\end{small}
\caption{\cstar Semantics Definitions}
\label{fig:cstar-semantics-defs}
\end{figure}

\begin{figure*}[!htbp]
\begin{small}
\begin{flushleft}
\fbox{$\eval{e}{(p, V)}=v$}
\end{flushleft}
\begin{mathpar}

\inferrule* [Right=Var]
{
  V(x) = v
}
{
  \eval{x}{(p,V)}=v
}

\quad
\quad
\quad

\inferrule* [Right=PtrAdd]
{
  \eval{e_1}{(p,V)} = (b, n, []) \\
  \eval{e_2}{(p,V)} = n' \\
}
{
  \eval{e_1+e_2}{(p,V)}=(b, n+n', [])
}

\\

\inferrule* [Right=PtrFd]
{
  \eval{e}{(p,V)} = (b, n, \ls{fd}) \\
}
{
  \eval{\eptrfd{e}{fd}}{(p,V)}=(b, n, \ls{fd};fd)
}

\quad
\quad
\quad

\inferrule* [Right=GVar]
{
  x\not\in V \\
  p(x) = v
}
{
  \eval{x}{(p,V)}=v
}

\quad
\quad
\quad
\quad

\inferrule* [Right=NonMutStruct]
{
  \;
}
{
  \eval{\{\ls{fd=e}\}}{(p,V)}=\{\ls{fd=\eval{e}{(p,V)}}\}
}

\\

\inferrule* [Right=Proj]
{
  \eval{e}{(p,V)}=\{\ls{fd=v}\} \\
  \{\ls{fd=v}\}(fd')=v'
}
{
  \eval{e.fd'}{(p,V)}=v'
}

\quad
\quad
\quad
\quad

\inferrule*[Right=Val]
{
}{
  \eval{v}{(p,V)} = v
}
\end{mathpar}
\end{small}
\caption{\cstar Expression Evaluation}
\label{fig:cstar-expr-eval}
\end{figure*}


\begin{figure*}[!htbp]
\begin{small}
\begin{flushleft}
\fbox{$p \vdash C\step_l C'$}
\end{flushleft}
\begin{mathpar}
\inferrule* [Right=VarDecl]
{
  \eval{e}{(p,V)}=v
}
{
  p \vdash (S, V, t\;x=e; ss) \step (S, V[x\mapsto v], ss)
}

\quad
\quad
\quad
\quad
\quad
\quad

\inferrule* [Right=ArrDecl]
{
  S = S'; (M, V, E) \\
  b\not\in S
}
{
  p \vdash (S, V, t\;x[n]; ss) \step (S';(M[b\mapsto \None^n], V, E), V[x\mapsto (b, 0, [])], ss)
}

\\

\inferrule* [Right=Memset]
{
  \eval{e_1}{(p, V)} = (b, n, []) \\
  \eval{e_2}{(p, V)} = v \\
  \symset(S, (b, n, []), v^m) = S'
}
{
  p \vdash (S, V, \memset{e_1}{m}{e_2}; ss) \step_{\symwrite\;(b,n,[]), \dots, \symwrite\;(b,n+m-1,[])} (S', V, ss)
}

\\

\inferrule* [Right=Read]
{
  \eval{e}{(p, V)} = (b, n, \ls{fd}) \\
  \symget(S, (b, n, \ls{fd})) = v
}
{
  p \vdash (S, V, t\;x=*[e]; ss) \step_{\symread\;(b,n,\ls{fd})} (S, V[x\mapsto v], ss)
}

\quad
\quad
\quad
\quad

\\

\inferrule* [Right=Write]
{
  \eval{e_1}{(p, V)} = (b, n, \ls{fd}) \\
  \eval{e_2}{(p, V)} = v \\
  \symset(S, (b, n, \ls{fd}), v) = S'
}
{
  p \vdash (S, V, *e_1=e_2; ss) \step_{\symwrite\;(b,n,\ls{fd})} (S', V, ss)
}

\quad
\quad
\quad
\quad
\quad

\inferrule* [Right=Ret]
{
  \eval{e}{(p,V)}=v
}
{
  p \vdash (S;(\None, V',E), V, \ereturn\;e; ss) \step (S, V', \fplug{E}{v})
}

\\

\inferrule* [Right=Call]
{
  p(f)=\ecfuntwo{y}{t_1}{t_2}{ss_1} \\
  \eval{e}{(p,V)}=v
}
{
  p \vdash (S, V, t\;x=f\;e; ss) \step (S;(\None, V, t\;x=\symhole;ss), \{\}[y\mapsto v], ss_1)
}

\quad
\quad
\quad
\quad

\inferrule* [Right=RetBlk]
{
  \eval{e}{(p,V)}=v
}
{
  p \vdash (S;(M, V',E), V, \ereturn\;e; ss) \step (S, \{\}, \ereturn\;v)
}

\\

\inferrule* [Right=Expr]
{
  \eval{e}{(p,V)} = v
}
{
  p \vdash (S, V, e; ss) \step (S, V, ss)
}

\quad
\quad
\quad
\quad

\inferrule* [Right=Empty]
{
  \;
}
{
  p \vdash (S; (M, V', E), V, []) \step (S, V', \fplug{E}{()})
}

\\

\inferrule* [Right=Block]
{
  \;
}
{
  p \vdash (S, V, \{ss_1\};ss_2) \step (S;(\{\},V,\symhole;ss_2), V, ss_1)
}

\\

\inferrule* [Right=IfT]
{
  \eval{e}{(p,V)} = n \\
  n\not=0
}
{
  p \vdash (S, V, \eif{e}{ss_1}{ss_2};ss) \step_\brt (S, V, ss_1;ss)
}

\quad
\quad
\quad

\inferrule* [Right=IfF]
{
  \eval{e}{(p,V)} = n \\
  n=0
}
{
  lp \vdash (S, V, \eif{e}{ss_1}{ss_2};ss) \step_\brf (S, V, ss_2;ss)
}

\end{mathpar}
\end{small}
\caption{\cstar Configuration Reduction}
\label{fig:cstar-stmts-reduction}
\end{figure*}

\clearpage

\subsection{\lamstar Definition}

The following are the definitions of the syntax and operational semantics of \lamstar.

\begin{figure}[!htbp]
%% \begin{figure}[H]
\begin{center}
  \begin{tabularx}{\columnwidth}{rlR}
    $lp ::= $ & & program \\
      & $\ls{ld}$                      & series of declarations \\[1.2mm]

    $ld ::= $ & & declaration \\
      & $\etlet{x}{\lambda y:t.\;le:t}$                & top-level function \\
    & $\etlet{x:t}{v}$               & top-level value \\
    [1.2mm]

    $le ::= $ & & expressions \\
    & $n$    & constant \\
    & $()$ & unit value \\
    & $x$                           & variable \\
    & $\{\ls{fd=le}\}$ & struct as value \\
    & $le.fd$ & immutable struct field projection \\
    & $\estructfield{le}{fd}$           & sub-structure: mutable structure field projection \\
    & $\esubbuf{le}{le}$                & sub-buffer \\
    & $\eif {le}{le}{le}$            & conditional \\
    & $\elet{x:t}{le}{le}$                   & let-binding \\
    & $\elet{\_}{le}{le}$                   & anonymous let-binding \\
    & $\elet{x:t}{f\;le}{le}$                   & application \\
    & $\elet{x}{\enewbuf{n}{(le:t)}}{le}$                 & new buffer \\
    & $\elet{x:t}{\ereadbuf {le}{le}}{le}$  & read from buffer \\
    & $\elet{\_}{\ewritebuf{le}{le}{le}}{le}$              & write to buffer \\
    & $\elet{x}{\enewstruct{(le:t)}}{le}$                 & new mutable structure \\
    & $\elet{x:t}{\ereadstruct{le}}{le}$                & read from mutable structure \\
    & $\elet{\_}{\ewritestruct{le}}{le}$               & write to mutable structure \\
    & $\withframe\;le$ & with-frame \\
    & $\epop\;le$ & pop frame \\
    & $loc$                        & location \\
    [1.2mm]

  \end{tabularx}
\end{center}
\caption{\lamstar Syntax}
\label{fig:lowstar-syntax}
\end{figure}

\clearpage

\begin{figure}[!htbp]
  $$\begin{array}{rrcl}

    \multicolumn{4}{l}{\textrm{values}} \\
    & lv & ::= & n \mid () \mid \{\ls{fd=lv}\} \mid loc \\
    \\
    \multicolumn{4}{l}{\textrm{evaluation contexts}} \\
    & LE & ::= & \symhole \mid LE.fd \mid \estructfield{LE}{fd} \mid \{\ls{fd=lv};fd=LE;\ls{fd=le}\} \\
    & & \mid & \esubbuf{LE}{le} \mid \esubbuf{lv}{LE} \\
    & & \mid & \eif {LE}{le}{le} \\
    & & \mid & \elet{x:t}{LE}{le} \mid \elet{\_}{LE}{le} \\
    & & \mid & \elet{x:t}{f\;LE}{le} \\
    & & \mid & \elet{x}{\enewbuf{n}{(LE:t)}}{le} \\
    & & \mid & \elet{x:t}{\ereadbuf{LE}{le}}{le} \\
    & & \mid & \elet{x:t}{\ereadbuf{lv}{LE}}{le} \\
    & & \mid & \elet{\_}{\ewritebuf{LE}{le}{le}}{le} \\
    & & \mid & \elet{\_}{\ewritebuf{lv}{LE}{le}}{le} \\
    & & \mid & \elet{\_}{\ewritebuf{lv}{lv}{LE}}{le} \\
    & & \mid & \elet{x}{\enewstruct{(LE:t)}}{le} \\
    & & \mid & \elet{x:t}{\ereadstruct{LE}}{le} \\
    & & \mid & \elet{\_}{\ewritestruct{LE}{le}}{le} \\
    & & \mid & \elet{\_}{\ewritestruct{lv}{LE}}{le} \\
    & & \mid & \epop\;LE \\
    \\
    \multicolumn{4}{l}{\textrm{stack}} \\
    & H & ::= & \ls{h} \\
    \\
    \multicolumn{4}{l}{\textrm{stack frame}} \\
    & h & ::= & b\sympartial \ls{v} \\

\end{array}$$
\caption{\lamstar Semantics Definitions}
\label{fig:lowstar-semantics-defs}
\end{figure}

\clearpage

\begin{figure*}[!htbp]
\begin{scriptsize}
\begin{flushleft}
\fbox{$lp \vdash (H, le) \astep_l (H', le')$}\quad\text{and}\quad
\fbox{$lp \vdash (H, le) \step_l (H', le')$}
\end{flushleft}
\begin{mathpar}

\inferrule* [Right=ReadBuf]
{
  H(b, n+n', []) = lv
}
{
  lp \vdash (H, \elet{x}{\ereadbuf{(b,n,[])}{n'}}{le}) \astep_{\symread\;(b,n+n',[])} (H, \subst{x}{lv}{le})
}

\\


\inferrule* [Right=ReadStruct]
{
  H(b, n, \ls{fd}) = lv
}
{
  lp \vdash (H, \elet{x}{\ereadstruct{(b,n,\ls{fd})}}{le}) \astep_{\symread\;(b,n,\ls{fd})} (H, \subst{x}{lv}{le})
}

\\

\inferrule* [Right=App]
{
  lp(f)=\lambda y:t_1.\;le_1:t_2
}
{
  lp \vdash (H, \elet{x:t}{f\;v}{le}) \astep (H, \elet{x:t}{\subst{y}{v}{le_1}}{le})
}

\\

\inferrule* [Right=WriteBuf]
{
  (b,n+n',[])\in H
}
{
  lp \vdash (H, \elet{\_}{\ewritebuf{(b,n,[])}{n'}{lv}}{le}) \astep_{\symwrite\;(b,n+n',[])} (H[(b, n+n',[])\mapsto lv], le)
}

\\

\inferrule* [Right=WriteStruct]
{
  (b,n,\ls{fd})\in H
}
{
  lp \vdash (H, \elet{\_}{\ewritestruct{(b,n,\ls{fd})}{lv}}{le}) \astep_{\symwrite\;(b,n,\ls{fd})} (H[(b, n,\ls{fd})\mapsto lv], le)
}

\\

\inferrule* [Right=Subbuf]
{
  \;
}
{
  lp \vdash (H, \esubbuf{(b,n,[])}{n'}) \astep (H, (b, n+n',[]))
}


\\

\inferrule* [Right=StructField]
{
  \;
}
{
  lp \vdash (H, \estructfield{(b,n,\ls{fd})}{fd'}) \astep (H, (b, n, (\ls{fd}; fd')))
}

\\

\inferrule* [Right=Let]
{
  \;
}
{
  lp \vdash (H, \elet{x:t}{v}{le}) \astep (H, \subst{x}{v}{le})
}

\quad
\quad
\quad

\inferrule* [Right=ALet]
{
  \;
}
{
  lp \vdash (H, \elet{\_}{v}{le}) \astep (H, le)
}

\quad
\quad
\quad
\quad

\inferrule* [Right=Proj]
{
  \{\ls{fd=lv}\}(fd') = lv'
}
{
  lp \vdash (H, \{\ls{fd=lv}\}.fd') \astep (H, lv')
}

\\

\inferrule* [Right=IfT]
{
  n\not=0
}
{
  lp \vdash (H, \eif{n}{le_1}{le_2}) \astep_\brt (H, le_1)
}

\quad
\quad
\quad

\inferrule* [Right=IfF]
{
  n=0
}
{
  lp \vdash (H, \eif{n}{le_1}{le_2}) \astep_\brf (H, le_2)
}

\\

\inferrule* [Right=NewBuf]
{
  b\not\in H
}
{
  lp \vdash (H, \elet{x}{\enewbuf{n}{(lv:t)}}{le}) \astep_{\symwrite\;(b,0,[]),\dots,\symwrite\;(b,n-1,[])} (H[b\mapsto lv^n], \subst{x}{(b, 0,[])}{le})
}

\\

\inferrule* [Right=NewStruct]
{
  b\not\in H
}
{
  lp \vdash (H, \elet{x}{\enewstruct{(lv:t)}}{le}) \astep_{\symwrite\;(b,0,[])} (H[b\mapsto lv], \subst{x}{(b, 0,[])}{le})
}

\\

\inferrule* [Right=WF]
{
  \;
}
{
  lp \vdash (H, \withframe\;le) \astep (H;\{\}, \epop\;le)
}

\quad
\quad
\quad

\inferrule* [Right=Pop]
{
  \;
}
{
  lp \vdash (H;h, \epop\;lv) \astep (H, lv)
}


\\

\inferrule* [Right=Step]
{
  lp \vdash (H, le) \astep_l (H', le')
}
{
  lp \vdash (H, \fplug{LE}{le}) \step_l (H', \fplug{LE}{le'})
}

\end{mathpar}
\end{scriptsize}
\caption{\lamstar Atomic Reduction and Reduction}
\label{fig:lowstar-reduction}
\end{figure*}

\newpage

\section{\lamstar to \cstar Compilation}

The compilation procedure is defined in Figure \ref{fig:lowtoc} as inference rules, which should be read as a function defined by pattern-matching, with earlier rules shadowing later rules. The compilation is a partial function, encoding syntactic constraints on \lamstar programs that can be compiled. For example, compilable \lamstar top-level functions must be wrapped in a $\withframe$ construct.


\begin{figure*}[!htbp]
\begin{scriptsize}
\begin{flushleft}
\fbox{$\lowtoce{le}=e$}\quad\text{and}\quad
\fbox{$\lowtoc{le}=ss$}\quad\text{and}\quad
\fbox{$\lowtocd{ld}=d$}
\end{flushleft}
\begin{mathpar}
\inferrule*
{
  \;
}
{
  \lowtoce{n} = n
}

\quad
\quad

\inferrule*
{
  \;
}
{
  \lowtoce{(b,n,[])} = (b,n,[])
}

\quad
\quad

\inferrule*
{
  \;
}
{
  \lowtoce{\{\ls{fd=le}\}} = \{\ls{fd=\lowtoce{le}}\}
}

\quad
\quad

\inferrule*
{
  \;
}
{
  \lowtoce{x} = x
}

\quad
\quad

\inferrule*
{
  \;
}
{
  \lowtoce{\esubbuf{le_1}{le_2}} = \lowtoce{le_1} + \lowtoce{le_2}
}

\\

\inferrule*
{
  \;
}
{
  \lowtoce{\estructfield{le}{fd}} = \eptrfd{\lowtoce{le}}{fd}
}


\\

\inferrule*
{
  \lowtoce{le}=e \\
  \lowtoc{le_1}=ss
}
{
  \lowtoc{(\elet{x:t}{f\;le}{le_1})} = (t\;x=f(e); ss)
}

\quad
\quad

\inferrule*
{
  \lowtoce{le_i}=e_i \;(i=1,2) \\
  \lowtoc{le}=ss
}
{
  \lowtoc{\elet{x:t}{\ereadbuf{le_1}{le_2}}{le}} = (t\;x=\eread{e_1+e_2};ss)
}

\\

\inferrule*
{
  \lowtoce{le_i}=e_i \;(i=1) \\
  \lowtoc{le}=ss
}
{
  \lowtoc{\elet{x:t}{\ereadstruct{le_1}}{le}} = (t\;x=\eread{e_1};ss)
}

\\

\inferrule*
{
  \lowtoce{le_i}=e_i \;(i=1,2,3) \\
  \lowtoc{le}=ss
}
{
  \lowtoc{(\elet{\_}{\ewritebuf{le_1}{le_2}{le_3}}{le})} = (\ewrite{e_1+e_2}{e_3}; ss)
}

\\


\inferrule*
{
  \lowtoce{le_i}=e_i \;(i=1,2) \\
  \lowtoc{le}=ss
}
{
  \lowtoc{(\elet{\_}{\ewritestruct{le_1}{le_2}}{le})} = (\ewrite{e_1}{e_2}; ss)
}

\\

\inferrule*
{
  \lowtoce{le}=e \\
  \lowtoc{le_1}=ss
}
{
  \lowtoc{(\elet{x}{\enewbuf{n}{(le:t)}}{le_1})} = (\earray txn; \memset{x}{n}{e}; ss)
}

\\

\inferrule*
{
  \lowtoce{le}=e \\
  \lowtoc{le_1}=ss
}
{
  \lowtoc{(\elet{x}{\enewstruct{(le:t)}}{le_1})} = (\earray tx1; \memset{x}{1}{e}; ss)
}

\\

\inferrule*
{
  \;
}
{
  \lowtoc{(\withframe\;le)} = \{\lowtoc{le}\}
}

\quad
\quad

\inferrule*
{
  \lowtoce{le}=e \\
  \lowtoc{le_i}=ss_i \;(i=1,2)
}
{
  \lowtoc{(\eif{le}{le_1}{le_2})} = (\eif{e}{ss_1}{ss_2})
}

\\

\inferrule*
{
  \lowtoce{le}=e \\
  \lowtoc{le_1}=ss
}
{
  \lowtoc{(\elet{x:t}{le}{le_1})} = (t\;x=e; ss)
}

\quad
\quad

\inferrule*
{
  \lowtoce{le}=e \\
  \lowtoc{le_1}=ss
}
{
  \lowtoc{(\elet{\_}{le}{le_1})} = (e; ss)
}

\quad
\quad

\inferrule*
{
  \;
}
{
  \lowtoc{le} = \lowtoce{le}
}

\\

\inferrule*
{
  \;
}
{
  \lowtocd{(\etlet{x:t}{lv})} = (t\;x=\lowtoce{lv})
}

\quad
\quad

\inferrule*
{
  \lowtoc{le}=ss;e
}
{
  \lowtocd{(\etlet{f}{\lambda x:t_1.\;\withframe\;le:t_2})} = \ecfun{f}{x}{t_1}{t_2}{ss; \ereturn{e}}
}

\end{mathpar}
\end{scriptsize}
\caption{\lamstar to \cstar compilation}
\label{fig:lowtoc}
\end{figure*}

\newpage

\section{Bisimulation Proof} \label{section-bisim}

The main results are Theorem \ref{thm-safety} and \ref{thm-bisim}, in terms of some notions defined before them in this section. The two theorems are proved by using the crucial Lemma \ref{lemma-reverse} to ``flip the diagram'', i.e., proving \cstar refines \lamstar by proving \lamstar refines \cstar. The flipping relies on the fact that \cstar is deterministic modulo renaming of block identifiers. An alternative way of determinization to renaming of block identifiers is to have the stream of random coins for choosing block identifiers as part of the configuration (state).

That \cstar semantics use
big-step semantics for \cstar expressions complicates the bisimulation
proof a bit because \lamstar and \cstar steps may go out-of-sync at
times. Within the proof we used a relaxed notion of simulation
(``quasi-refinement'') that allows this temporary discrepancy by some
stuttering, but still implies bisimulation.

The specific relation between a \lamstar configuration and its \cstar counterpart is defined in Definition \ref{def-R} as Relation $R$.  It is defined in terms of a \cstar-to-\lamstar back-translation, listed in Fig \ref{fig:ctolow}. We need a back-translation here instead of the \lamstar-to-\cstar forward-translation as defined before, because a \cstar configuration has a clear call-stack with each frame containing its own variable environment and continuation, while a \lamstar configuration contains one giant \lamstar expression which makes it impossible to recover the call-stack. Hence for relating two configurations in the middle of reduction, only the \cstar-to-\lamstar direction is possible.

\begin{definition}[(Labelled) Transition System] \label{def-trsys}
  A transition system is a 5-tuple $(\Sigma, L, \step, s_0, F)$, where $\Sigma$ is a set of states, L is a monoid which is the set of labels, $\step\;\subseteq\Sigma\times \kw{option}\;L\times \Sigma$ is the step relation, $s_0$ is the initial state, and $F$ is a set of designated final state.
\end{definition}

In the following text, we use $\epsilon$ to denote an empty label (the unit of the monoid $L$), $l$ to range over non-empty (non-$\epsilon$) labels and $\olabel$ to range over possibly empty labels. When the label is empty, we can omit it. The label of multiple steps is the combined label of each steps, using the addition operator of monoid $L$. We define $a\Downarrow_\olabel a'\defeq a\step^*_\olabel a'$ and $a'\in F$.

\begin{definition}[Safety] \label{def-safety}
  A transition system $A$ is safe iff for all $s$ so that $s_0\step^* s$, $s$ is unstuck, where $\unstuck(s)$ is defined as either $s\in F$ or there exists $s'$ so that $s\step s'$.
\end{definition}

\begin{definition}[Refinement] \label{def-refine}
  A transition system $A$ refines a transition system $B$ (with the same label set) by $R$ (or $R$ is a refinement for transition system $A$ of transition system $B$) iff
  \begin{enumerate}
  \item there exists a well-founded measure $|-|_b$ (indexed by a $B$-state $b$) defined on the set of $A$-states $\{a:\Sigma_A\;|\;a\;R\;b\}$;
  \item $a_0\;R\;b_0$, that is, the two initial states are in  relation $R$;
  \item for all $a:\Sigma_A$ and $b:\Sigma_B$ such that $a\;R\;b$,
    \begin{enumerate}
    \item if $a\step_\olabel a'$ for some $a':\Sigma_A$, then there exists $b':\Sigma_B$ and $n$ such that $b\step^n_\olabel b'$ and $a'\;R\;b'$ and that $n=0$ implies that $|a|_b>|a'|_b$;
    \item if $a\in F_A$, then there exists $b':\Sigma_B$ such that $b\Downarrow_\epsilon b'$ and $a\;R\;b'$.
    \end{enumerate}
  \end{enumerate}
  $A$ refines $B$ iff there exists $R$ so that $A$ refines $B$ by $R$.
\end{definition}

\begin{definition}[Bisimulation] \label{def-bisim}
  A transition system $A$ bisimulates a transition system $B$ iff $A$ refines $B$ and $B$ refines $A$.
\end{definition}

\begin{definition}[Transition System of \cstar and \lamstar] \label{def-trsys-cstar}
  %% \begin{itemize}
  %% \item
  $$\begin{array}{rcl}
    \sys_{\cstar}(p,V,ss)&\defeq&(C, \{\ls{l}\}, p\vdash\step, ([], V, ss), \{([], V', \ereturn{e})\}) \\
    \sys_{\lamstar}(lp,le)&\defeq&(\{(H,le)\}, \{\ls{l}\}, lp\vdash\step, ([], le), \{([], lv)\})
  \end{array}$$
  %% \end{itemize}
\end{definition}

In the following text, we treat label $\symread/\symwrite\;(b,n)$ and $\symread/\symwrite\;(b,n,[])$ as equal, and if we have a \lamstar value or substitution, we freely use it as a \cstar one because coercion from \lamstar value to \cstar value is straight-forward.

\begin{theorem}[Safety] \label{thm-safety}
  For all \lamstar program $lp$, closed expression $le$ and closing substitution $V$, if $\lowtocd{lp}=p$, $\lowtoc{le}=ss$ and $\sys_{\lamstar}(lp,V(le))$ is safe, then $\sys_{\cstar}(p, V, ss)$ is safe.
\end{theorem}
\begin{proof}
Appeal to Lemma \ref{lemma-reverse}, Lemma \ref{lemma-cstar-deter} and Lemma \ref{lemma-back-refine}.
\end{proof}

\begin{theorem}[Bisimulation] \label{thm-bisim}
  For all \lamstar program $lp$, closed expression $le$ and closing substitution $V$, if $\lowtocd{lp}=p$ and $\lowtoc{le}=ss$, then $\sys_{\cstar}(p, V, ss)$ bisimulates $\sys_{\lamstar}(lp,V(le))$.
\end{theorem}
\begin{proof}
Appeal to Corollary \ref{coro-reverse}, Lemma \ref{lemma-cstar-deter} and Lemma \ref{lemma-back-refine}.
\end{proof}

\begin{definition}[Determinism] \label{def-deter}
  A transition system $A$ is deterministic iff for all $s$ so that $s_0\step^* s$, $s\in F$ implies that $s$ cannot take any step, and $s\step_{o_1} s_1$ and $s\step_{o_2} s_2$ implies that $o_1 = o_2$ and $s_1 = s_2$.
\end{definition}

\begin{definition}[Quasi-Refinement] \label{def-quasi-refine}
  A transition system $A$ quasi-refines a transition system $B$ (with the same label set) by $R$ (or $R$ is a quasi-refinement for transition system $A$ of transition system $B$) iff
  \begin{enumerate}
  \item there exists a well-founded measure $|-|_b$ (indexed by a $B$-state $b$) defined on the set of $A$-states $\{a:\Sigma_A\;|\;a\;R\;b\}$;
  \item $a_0\;R\;b_0$, that is, the two initial states are in  relation $R$;
  \item for all $a:\Sigma_A$ and $b:\Sigma_B$ such that $a\;R\;b$,
    \begin{enumerate}
    \item if $a\step_\olabel a'$ for some $a':\Sigma_A$, then there exists $a'':\Sigma_A$, $b':\Sigma_B$ and $n$ such that $a'\step^*_\epsilon a''$ and $b\step^n_\olabel b'$ and $a''\;R\;b'$ and that $n=0$ implies that $|a|_b>|a'|_b$;
    \item if $a\in F_A$, then there exists $b':\Sigma_B$ such that $b\Downarrow_\epsilon b'$ and $a\;R\;b'$.
    \end{enumerate}
  \end{enumerate}
  $A$ quasi-refines $B$ iff there exists $R$ so that $A$ quasi-refines $B$ by $R$.
\end{definition}

\begin{lemma}[Quasi-refine-Refine] \label{lemma-quasi}
  If transition system $A$ is deterministic, then $A$ quasi-refines transition system $B$ implies that $A$ refines $B$.
\end{lemma}
\begin{proof}
  Let $R$ be the quasi-refinement for $A$ of $B$. \\
  Define $R'$ to be: $a\;R'\;b$ iff $\exists n.\;(\exists a'.\;a\step^n_\epsilon a' \eand a'\;R\;b).$ \\
  We are to show that $A$ refines $B$ by $R'$. Unfold Definition \ref{def-refine}. \\
  For Condition 1, define $|a|_b$ to be the minimal of the number $n$ in the definition of $R'$, which uniquely exists. \\
  For Condition 2, we are to show $a_0\;R'\;b_0$. We know that $a_0\;R\;b_0$, so it's obviously true. \\
  For Condition 3(a), we have $a\;R'\;b$ and $a\step_\olabel a'$. \\
  We are to exhibit $b'$ and $n$ so that $b\step^n_\olabel b'$ and $a'\;R'\;b'$ and that $n=0$ implies $|a|_b>|a'|_b$. \\
  From $a\;R'\;b$, we have $a\step^m_\epsilon a''$ and $a''\;R\;b$. \\
  If $m=0$, we know $a\;R\;b$. Because $A$ quasi-refines $B$ by $R$, we have $a'\step^*_\epsilon a_2$ and $b\step^n_\olabel b'$ and $a_2\;R\;b'$ and that $n=0$ implies $|a|_b>|a'|_b$. \\
  Pick $b'$ to be $b'$ and $n$ to be $n$. It suffices to show $a'\;R'\;b'$, which is true because $a'\step^*_\epsilon a_2$ and $a_2\;R\;b'$. \\
  If $m>0$, pick $b'$ to be $b$ and $n$ to be 0. Because $A$ is deterministic, we know $a'\;R'\;b$ with $m-1$ and $|a|_b=m$ and $|a'|_b=m-1$. \\
  For Condition 3(b), we have $a\;R'\;b$ and $a\in F_A$. \\
  We are to exhibit $b'$ such that $b\Downarrow_\epsilon b'$ and $a\;R'\;b'$. \\
  Because $A$ is deterministic and $a\in F_A$, we have $m=0$ and $a\;R\;b$. \\
  Because $A$ quasi-refines $B$ with $R$, we have $b\Downarrow_\epsilon b'$ and $a\;R\;b'$. \\
  Pick $b'$ to be $b'$. It suffices to show $a\;R'\;b'$, which is trivially true. \\
\end{proof}

\begin{lemma}[Refine-Safety] \label{lemma-refine-safety}
  If transition system $A$ refines transition system $B$ by $R$ and for any $a$ and $b$ we have $a\;R\;b$ implies $\unstuck(a)$, then $A$ is safe.
\end{lemma}
\begin{proof}
  From Definition \ref{def-refine} we know $(\exists b.\;a\;R\;b)$ is an invariant of $A$. Hence $\unstuck(a)$ is also an invariant of $A$.
\end{proof}

\begin{lemma}[Deterministic Reverse] \label{lemma-reverse}
  If transition system $A$ is deterministic and transition system $B$ is safe, then $B$ refines $A$ implies that $A$ refines $B$ and $A$ is safe.
\end{lemma}
\begin{proof}
  Appealing to Lemma \ref{lemma-refine-safety}, we will exhibit the refinement for $A$ of $B$ and show that it implies unstuckness. \\
  Let $R$ be the refinement for $B$ of $A$. Define $R'$ to be: \\
  $a\;R'\;b$ iff \\
  $b_0\step^*b\eand((\exists \olabel\;b'.\;b\step_\olabel b'\eand\exists n_1\;a_2\;a_3\;a_4.\;(a\step^*_\epsilon a_2\eor a_2\step^*_\epsilon a)\eand b\;R\;a_2\eand a_2\step^*_\olabel a_3\eand a\step^{n_1}_\olabel a_3\eand a_3\step^*_\epsilon a_4\eand b'\;R\;a_4)\eor(b\in F_B\eand \exists n_2\;a_2\;a_3.\;(a\step^*_\epsilon a_2\eor a_2\step^*_\epsilon a)\eand b\;R\;a_2\eand a_2\step^*_\epsilon a_3\eand a\step^{n_2}_\epsilon a_3\eand a_3\in F_A\eand b\;R\;a_3))$. \\
  Let's first prove the fact (Fact 1) that if $b_0\step^* b$ and $a\step^*_\epsilon a'$ and $b\;R\;a$, then $a\;R'\;b$. \\
  Because $B$ is safe, we know that either $b\step_\olabel b'$ or $b\in F_B$. \\
  In the first case, because $B$ refines $A$ by $R$, we have $a'\step^n_\olabel a''$ and $b\;R\;a''$. \\
  It's easy to show $a\;R'\;b$ by choosing the first disjunct and picking $\olabel,b',a_2, a_4$ to be $\olabel,b',a',a''$. $n_1$ and $a_3$ exist in this case. \\
  In the second case, because $B$ refines $A$ by $R$, we have $a'\Downarrow_\epsilon a''$ and $b\;R\;a''$. \\
  It's easy to show $a\;R'\;b$ by choosing the second disjunt and picking $a_2, a_3$ to be $a', a''$. $n_2$ obviously exists. \\
  \\
  Now we are to show $A$ refines $B$ by $R'$. Unfold Definition \ref{def-refine}. \\
  For Condition 1, define $|a|_b$ to be lexicographic order of two numbers. \\
  The first number is the minimal of the number $n_2$ in the definition of $R'$ if $b$ is a value, which uniquely exists; or 0 otherwise. \\
  The second number is the minimal of the number $n_1$ in the definition of $R'$ if $b$ can take a step, which uniquely exists; or 0 otherwise. \\
  For Condition 2, we are to show $a_0\;R'\;b_0$, which is true because of $b_0\;R\;a_0$ and Fact 1. \\
  For Condition 3(a), we have $a\;R'\;b$ and $a\step_\olabel a'$. \\
  We are to exhibit $b'$ and $n$ such that $b\step^n_\olabel b'$ and $a'\;R\;b'$ and that $n=0$ implies $|a|_b>|a'|_b$. \\
  Unfold $a\;R'\;b$, we have the two disjuncts. \\
  In case $\olabel=l$, only the first disjunt is possible, and we have $b\step_l b'$ and $a'\step^*_\epsilon a_4$ and $b'\;R\;a_4$. \\
  Pick $b',n$ to be $b',1$. From Fact 1, we know $a'\;R'\;b'$. \\
  In case $\olabel=\epsilon$, both disjuncts of $a\;R'\;b$ are possible. \\
  If $a\;R'\;b$ because of the first conjunct, we have $b\step_{\olabel'} b'\eand (a\step^*_\epsilon a_2\eor a_2\step^*_\epsilon a)\eand b\;R\;a_2\eand a_2\step^*_{\olabel'} a_3\eand a\step^{n_1}_{\olabel'} a_3\eand a_3\step^*_\epsilon a_4\eand b'\;R\;a_4$. \\
  %% Because $B$ is safe and $B$ refines $A$ by $R$, we can step on the $B$ side for finite steps to reach $b_2$ such that $b'\step^*_{\epsilon b_2$ and $b_2\;R\;a$ and either $b_2\in F_B$ or  \\
  We case-analyse on whether $\olabel'$ is $\epsilon$. \\
  \\
  If $\olabel'=l$, let the second component of $|a|_b$ (denoted by $|a|_b.2$) be $m$. We case-analyse on whether $m>1$. \\
  If $m>1$, pick $b',n$ to be $b, 0$ (i.e. do not move on the $B$ side). We need to show $a'\;R'\;b$ and $|a|_b>|a'|_b$. \\
  $a'\;R'\;b$ because according to $m>1$ and $a\step_\epsilon a'$, we know that $a'$ is still before the $l$-label step. \\
  $|a|_b>|a'|_b$ is true because according to $m>1$, it must be the case that $|a'|_b.2=|a|_b.2-1$; and as for $|a'|_b.1$, which represents the minimal number of steps to terminate (or 0 otherwise), taking one step will not increase it. \\
  If $m\leq 1$, we know that $a\step_l a''$ for some $a''$. But we also have $a\step_\epsilon a'$, so this case is impossible because of $A$'s determinism. \\
  \\
  If $\olabel'=\epsilon$, because $B$ is safe and $B$ refines $A$ by $R$, we can step on the $B$ side for finite steps to reach $b_2$ such that $b'\step^*_\epsilon b_2$ and $b_2\;R\;a$ and either $b_2\step_{\olabel''} b_3\eand a\step^+_{\olabel''}a''\eand b_3\;R\;a''$ or $b_2\in F_B\eand a\step^*_\epsilon a''\eand a''\in F_A\eand b_2\;R\;a''$. \\
  In the first case, because $A$ is deterministic, we have $a\step_\epsilon a'\step^*_{\olabel''}a''$. \\
  If $\olabel''=\epsilon$, pick $b'$ to be $b_3$. Because of $a'\step^*_\epsilon a''$ and $b_3\;R\;a''$ and Fact 1, we get $a'\;R'\;b_3$. \\
  If $\olabel''=l$, pick $b'$ to be $b_2$. Since $b\step_\epsilon b'\step^*_\epsilon b_2$, we just need to show that $a'\;R'\;b_2$, which is easy to show by choosing the first disjunct for $R'$ and picking $b',a_2,a_4$ to be $b_3, a, a''$. \\
  In the second case ($b_2\in F_B$), it must be case that $a\step_\epsilon a'\step^*_\epsilon a''\in F_A$. Pick $b'$ to be $b_2$, we need to show $a'\;R'\;b_2$, which is true because $a'\step^*_\epsilon a''$ and $a''\;R'\;b_2$. \\
  \\
  If $a\;R'\;b$ because of the second conjunct, we have $b\in F_B\eand (a\step^*_\epsilon a_2\eor a_2\step^*_\epsilon a)\eand b\;R\;a_2\eand a_2\step^*_\epsilon a_3\eand a\step^{n_2}_\epsilon a_3\eand a_3\in F_A\eand b\;R\;a_3$. \\
  Because $A$ is deterministic, it must be the case that $a\step_\epsilon a'\step^*_\epsilon a_3$. \\
  Pick $b',n$ to be $b,0$. We need to show $a'\;R'\;b$ and $|a|_b>|a'|_b$. $a'\;R'\;b$ because $a'\step^*_\epsilon a_3$ and $a_3\;R'\;b$. $|a|_b>|a'|_b$ because $|a|_b.1>|a'|_b.1$, which is true because $a'$ is one step closer to terminate. \\
  \\
  For Condition 3(b), we have $a\;R'\;b$ and $a\in F_A$. \\
  We are to exhibit $b'$ such that $b\Downarrow_\epsilon b'$ and $a\;R'\;b'$. \\
  If $a\;R'\;b$ because of the second disjunct, we have $b\in F_B\eand a\step^*_\epsilon a_3\eand b\;R\;a_3$. Because $a\in F_A$ and $A$ is deterministic, we know that $a_3=a$\\
  Pick $b'$ to be $b$. $a\;R'\;b$ is true because $b\;R\;a$ and Fact 1. \\
  If $a\;R'\;b$ because of the first disjunct, we have $b\step_\olabel b_2\eand a\step^*_\olabel a_4\eand b_2\;R\;a_4$. \\
  Because $a\in F_A$ and $A$ is deterministic, we know that $a_4=a$ and $\olabel=\epsilon$. \\
  If $b_2\in F_B$, pick $b'$ to be $b_2$. $a\;R'\;b_2$ is true with the same reasoning as before. \\
  Otherwise, because $B$ is safe and $B$ refines $A$ by $R$, we can step $b_2$ for finite steps (because $a$ cannot step and $|b|_a>|b_2|_a$) to have $b_2\step^*_\epsilon b_3\eand b_3\;R\;a$. \\
  Pick $b'$ to be $b_3$. $a\;R'\;b_3$ is true with the same reasoning as before. \\
  \\
  Now we prove that $a\;R'\;b$ implies $\unstuck(a)$ for any $a$ and $b$. \\
  Unfolding $a\;R'\;b$, in both disjuncts we have $a\step^na'$ and $b'\;R\;a'$ for some $b'$. \\
  If $n>0$, $\unstuck(a)$ is obviously true. \\
  If $n=0$, we have $b'\;R\;a$. Because $B$ is safe, we know that either $b'\step b''$ or $b'\in F_B$. \\
  In case $b'\in F_B$, we know $a\Downarrow_\epsilon a_2$. Because $A$ is deterministic, $\unstuck(a)$ is true. \\
  In case $b'\step b''$, because $B$ refines $A$ by $R$, we know $a\step^ka_2$ and $b''\;R\;a_2$ and that $k=0$ implies $|b'|_a>|b''|_a$. \\
  Thus $b'$ can step finite number of $k=0$ steps before hitting the $b'\in F_B$ case or the $k>0$ case, in both of which we have $\unstuck(a)$.
\end{proof}

\begin{corollary}[Deterministic Reverse] \label{coro-reverse}
  If transition system $A$ is deterministic and transition system $B$ is safe, then $B$ refines $A$ implies that $A$ bisimulates $B$ and $A$ is safe.
\end{corollary}

\begin{definition}[Relation $R$] \label{def-R}
  For any $p$ and $lp$, define relation $R_{p,lp}$ as: $(H, le)\;R_{p,lp}\;(S, V, ss)$ iff there exists a minimal $n$ such that $(H, le)\step_{lp}^n(H, le')$ and $(H, le')=\ctolowc{(S, V, ss)}$, where $\ctolowc{(S, V, ss)}\defeq(\mem(S), \unravel(S, V(\ctolow{(\normal{ss}{(p,V)})})))$ and $\mem(S)$ is all the memory parts of $S$ collected together (and requiring that there is no $\None$ in $S$'s memory parts). \\
  $\unravel(S, le)\defeq\foldl\;\unravelframe\;le\;S$ \\
  $\unravelframe((M, V, E), le)\defeq \\ \begin{cases}
    V(\fplug{(\ctolowE{E})}{le}) & \text{if }M=\None\\
    V(\fplug{(\ctolowE{E})}{\epop\;le}) & \text{if }M=\Some{\_}
  \end{cases}$.
\end{definition}

\begin{figure*}[!htbp]
\begin{small}
\begin{flushleft}
\fbox{$\normal{ss}{(p,V)}=ss$}
\end{flushleft}
\begin{mathpar}
\inferrule*
{
  \eval{e}{(p,V)}=v
}
{
  \normal{(t\;x=e;ss)}{(p,V)} = (t\;x=v;ss)
}

\quad
\quad

\inferrule*
{
  \eval{e}{(p,V)}=v
}
{
  \normal{(t\;x=f(e);ss)}{(p,V)} = (t\;x=f(v);ss)
}

\quad
\quad

\inferrule*
{
}
{
  \normal{(\earray{t}{x}{n};ss)}{(p,V)} = (\earray{t}{x}{n};ss)
}

\\

\inferrule*
{
  \eval{e}{(p,V)}=v
}
{
  \normal{(\ereturn{e};ss)}{(p,V)} = (\ereturn{v};ss)
}

\quad
\quad

\inferrule*
{
  \eval{e_i}{(p,V)}=v_i \;(i=1,2)
}
{
  \normal{(t\;x=\eread{e_1}{e_2};ss)}{(p,V)} = (t\;x=\eread{v_1}{v_2};ss)
}

\quad
\quad

\inferrule*
{
  \eval{e_i}{(p,V)}=v_i \;(i=1,2,3)
}
{
  \normal{(\ewrite{e_1+e_2}{e_3};ss)}{(p,V)} = (\ewrite{v_1+v_2}{v_3};ss)
}

\\

\inferrule*
{
  \eval{e}{(p,V)}=v
}
{
  \normal{(e;ss)}{(p,V)} = (v;ss)
}

\quad
\quad

\inferrule*
{
  \eval{e_i}{(p,V)}=v_i \;(i=1,2)
}
{
  \normal{(\memset{e_1}{n}{e_2};ss)}{(p,V)} = (\memset{v_1}{n}{v_2};ss)
}
\end{mathpar}
\end{small}
\caption{Normalize \cstar head expression}
\label{fig:normalize}
\end{figure*}

\clearpage

\begin{figure*}[!htbp]
\begin{scriptsize}
\begin{flushleft}
\fbox{$\ctolowe{e}=le$}\quad\text{and}\quad
\fbox{$\ctolow{ss}=le$}\quad\text{and}\quad
\fbox{$\ctolowd{d}=ld$}\quad\text{and}\quad
\fbox{$\ctolowE{E}=LE$}
\end{flushleft}
\begin{mathpar}
\inferrule*
{
  \;
}
{
  \ctolowe{n} = n
}

\quad
\quad

\inferrule*
{
  \;
}
{
  \ctolowe{(b,n,[])} = (b,n,[])
}

\quad
\quad

\inferrule*
{
  \;
}
{
  \ctolowe{\{\ls{fd=e}\}} = \{\ls{fd=\ctolowe{e}}\}
}

\quad
\quad

\inferrule*
{
}{
  \ctolowe{()} = ()
}

\\

\inferrule*
{
}{
  \ctolowe{x} = x
}

\quad
\quad

\inferrule*
{
  \ctolowe{le_i}=e_i \;(i=1,2)
}
{
  \ctolowe{(e_1+e_2)} = \esubbuf{le_1}{le_2}
}

\quad
\quad

\inferrule*
{
  \ctolowe{le_i}=e_i \;(i=1)
}
{
  \ctolowe{\eptrfd{e_1}{fd}} = \estructfield{le_1}{fd}
}

\\

\inferrule*
{
  \ctolowe{e}=le_1 \\
  \ctolow{ss} = le
}
{
  \ctolow{(t\;x=f(e);ss)} = (\elet{x:t}{f\;le_1}{le})
}

\quad
\quad

\inferrule*
{
  \ctolowe{e}=le_1 \\
  \ctolow{ss} = le
}
{
  \ctolow{(\earray{t}{x}{n};\memset{x}{n}{e};ss)} = (\elet{x}{\enewbuf{n}{(le_1:t)}}{le})
}

\\


\inferrule*
{
  \ctolowe{e}=le_1 \\
  \ctolow{ss} = le \\
  t ~ \text{is a struct type}
}
{
  \ctolow{(\earray{t}{x}{1};\memset{x}{1}{e};ss)} = (\elet{x}{\enewstruct{(le_1:t)}}{le})
}

\quad
\quad

\inferrule*
{
  \ctolowe{e}=le_1 \\
  \ctolow{ss} = le
}
{
  \ctolow{(t\;x=e;ss)} = (\elet{x:t}{le_1}{le})
}

\\

\inferrule*
{
  \ctolowe{e_i}=le_i \; (i=1,2) \\
  \ctolow{ss} = le
}
{
  \ctolow{(t\;x=\eread{e_1+e_2};ss)} = (\elet{\_}{\ereadbuf{le_1}{le_2}}{le})
}

\quad
\quad

\inferrule*
{
  \ctolowe{e_i}=le_i \; (i=1) \\
  \ctolow{ss} = le
}
{
  \ctolow{(t\;x=\eread{e_1};ss)} = (\elet{\_}{\ereadstruct{le_1}}{le})
}

\\

\inferrule*
{
  \ctolowe{e_i}=le_i \; (i=1,2,3) \\
  \ctolow{ss} = le
}
{
  \ctolow{(\ewrite{e_1+e_2}{e_3};ss)} = (\elet{\_}{\ewritebuf{le_1}{le_2}{le_3}}{le})
}

\quad
\quad

\inferrule*
{
  \ctolowe{e_i}=le_i \; (i=1,2) \\
  \ctolow{ss} = le
}
{
  \ctolow{(\ewrite{e_1}{e_2};ss)} = (\elet{\_}{\ewritestruct{le_1}{le_2}}{le})
}

\\

\inferrule*
{
  \ctolow{ss_1}=le_1 \\
  \ctolow{ss} = le
}
{
  \ctolow{(\{ss_1\};ss)} = (\elet{\_}{\withframe\;le_1}{le})
}

\quad
\quad

\inferrule*
{
  \ctolowe{e}=le_1 \\
  \ctolow{ss} = le
}
{
  \ctolow{(e;ss)} = (\elet{\_}{le_1}{le})
}

\\

\inferrule*
{
  \ctolowe{e}=le \\
  \ctolow{ss_i}=le_i \;(i=1,2,3)
}
{
  \ctolow{(\eif{e}{ss_1}{ss_2};ss_3)} = (\elet{\_}{\eif{le}{le_1}{le_2}}{le_3})
}

\quad
\quad

\inferrule*
{
  \ctolowe{e}=le
}
{
  \ctolow{[e]} = le
}

\quad
\quad

\inferrule*
{
  \;
}
{
  \ctolow{[]} = ()
}

\\

\inferrule*
{
  \;
}
{
  \ctolowd{(t\;x=v)} = (\etlet{x:t}{\ctolowe{v}})
}

\quad
\quad

\inferrule*
{
  \ctolow{(ss;e)}=le
}
{
  \ctolowd{(\ecfun{f}{x}{t_1}{t_2}{ss; \ereturn{e}})} = (\etlet{f}{\lambda x:t_1.\;\withframe\;le:t_2})
}

\\

\inferrule*
{
  \ctolow{ss} = le
}
{
  \ctolowE{(\symhole;ss)} = (\elet{\_}{\symhole}{le})
}

\quad
\quad

\inferrule*
{
  \ctolow{ss} = le
}
{
  \ctolowE{(t\;x=\symhole;ss)} = (\elet{x:t}{\symhole}{le})
}
\end{mathpar}
\end{scriptsize}
\caption{\cstar to \lamstar back-translation}
\label{fig:ctolow}
\end{figure*}

\clearpage

\begin{lemma}[\lamstar Refines \cstar] \label{lemma-back-refine}
  For all \lamstar program $lp$, closed expression $le$ and closing substitution $V$, if $\lowtocd{lp}=p$ and $\lowtoc{le}=ss$, then $\sys_{\lamstar}(lp,V(le))$ refines $\sys_{\cstar}(p, V, ss)$.
\end{lemma}
\begin{proof}
  We apply Lemma \ref{lemma-quasi} and \ref{lemma-lowstar-deter}, and prove that $\sys_{\lamstar}(lp,le)$ quasi-refines $\sys_{\cstar}(p, ss)$.
  We pick the relation $R_{p,lp}$ in Definition \ref{def-R} to be the simulation relation and prove $R_{p,lp}$ is a quasi-refinement for $\sys_{\lamstar}(lp,le)$ of $\sys_{\cstar}(p, ss)$. \\
  Unfold Definition \ref{def-quasi-refine}. \\
  For Condition 1, define the well-founded measure $|(H,le)|_{(S, V, ss)}$ (where $(H,le)\;R\;(S,V,ss)$) to be the minimal of the number $n$ in $R$'s definition. \\
  For condition 2, appeal to Lemma \ref{lemma-init}. \\
  Now prove Condition 3(a). Let $(H,le)$ be the \lamstar configuration and $C=(S,V,ss)$ be the \cstar configuration. \\
  We are to exhibit $(H'', le'')$ and $C'$ and $n$ such that $(H',le')\step^*(H'',le'')$ and $C\step^n C'$ and $(H'',le'')\;R\;C'$ and that $n=0$ implies $|(H,le)|_C > |(H',le')|_C$. \\
  For all the cases except \emph{Case Pop}, we pick $(H'', le'')$ to be $(H',le')$ (i.e. do not use the extra flexibility offered by Quasi-Refinement). \\
  Induction on $(H, le)\step(H', le')$. \\
  \\
  \emph{Case Let}: on case $(H, \fplug{LE}{\elet{x:t}{lv}{le}})\step(H, \fplug{LE}{\subst{x}{lv}{le}})$. \\
  We are to exhibit $C'$ such that $(S, V, ss)\step^+C'$ and \\
  $(H, \fplug{LE}{\subst{x}{lv}{le}})\;R\;C'$.\\
  Apply Lemma \ref{lemma-invert-let}. \\
  In the first case, we have $le=V(\ctolow{ss'})$ and $lv=\ctolowe{v}$ and $LE=\unravel(S, \symhole)$, where $v\defeq\eval{e}{(p,V)}$. \\
  The \cstar side runs with nonzero steps to $(S, V[x\mapsto v], ss')$. \\
  Pick $C'$ to be this configuration. \\
  It suffices to show that $(H, \fplug{LE}{\subst{x}{lv}{le}})\;R\;(S, V[x\mapsto v], ss')$. \\
  Appealing to Lemma \ref{lemma-eq-normal}, it suffices to show that $\fplug{LE}{\subst{x}{lv}{le}}=\unravel(S, V[x\mapsto v](\ctolow{ss'}))$, which is true. \\
  In the second case, the \cstar side runs with nonzero steps to $(S', V'[x\mapsto v], ss')$. The proof is the same as the first case. \\
  End of case. \\
  %% From now on, we omit the second case from each inversion lemma, since they just add a ``return'' step on top of the first case. \\
  \\
  \emph{Case ALet}: on case $(H, \fplug{LE}{\elet{\_}{lv}{le}})\step(H, \fplug{LE}{le})$. \\
  Appealing to Lemma \ref{lemma-invert-alet}, the proof is similar to the previous case. \\
  End of case. \\
  \\
  \emph{Case App}: on case $(H, \fplug{LE}{\elet{x:t}{f\;lv}{le}})\step(H, \fplug{LE}{\elet{x:t}{\subst{y}{lv}{le_1}}}{le})$ and $lp(f)=\lambda y:t_1.\;le_1:t_2$. \\
  Appealing to Lemma \ref{lemma-invert-app}, we have $ss=(t\;x=f(v);ss')$ and $le=V(\ctolow{ss'})$ and $lv=\ctolowe{v}$ and $LE=\unravel(S, \symhole)$. \\
  Because $\lowtocd{lp}=p$, we know $le_1=\withframe\;le_2$ and $\lowtoc{le_2}=ss_2;e$ and $p(f)=\ecfuntwo{y}{t_1}{t_2}{ss_1}$ and $ss_1=(ss_2;\ereturn\;e)$. \\
  Appealing to Lemma \ref{lemma-lowtoc-ctolow}, we know $\ctolow{ss_1}=le_2$ hence $\ctolow{\{ss_1\}}=le_1$. \\
  Pick $C'$ to be $(S;(\None, V, t\;x=\symhole;ss'), \{y\mapsto v\}, \{ss_1\})$. \\
  It suffices to show that \\
  $(H, \fplug{LE}{\elet{x:t}{\subst{v}{lv}{le_1}}{le}})\;R\;(S;(\None, V, t\;x=\symhole;ss'), \{y\mapsto v\}, \{ss_1\})$, which is true. \\
  End of case. \\
  \\
  \emph{Case Withframe}: \\
  on case $(H, \fplug{LE}{\withframe\;le})\step(H;\{\}, \fplug{LE}{\epop\;le})$. \\
  Appealing to Lemma \ref{lemma-invert-withframe}, we have $ss=\{ss_1\};ss_2$ and $le=V(\ctolow{ss_1})$ and $LE=\unravel(S, V(\ctolowE{(\symhole;ss_2)}))$. \\
  Pack $C'$ to be $(S;(\{\}, V, \symhole;ss_2), V, ss_1)$. \\
  It suffices to show that \\
  $(H;\{\}, \fplug{LE}{\epop\;le})\;R\;(S;(\{\}, V, \symhole;ss_2), V, ss_1)$, which is true. \\
  End of case. \\
  \\
  \emph{Case Newbuf}: on case $(H;h, \fplug{LE}{\elet{x}{\enewbuf{n}{(lv:t)}}{le}})\step_{\symwrite\;(b,0,[]),\dots,\symwrite\;(b,n-1,[])}(H;h[b\mapsto lv^n], \fplug{LE}{\subst{x}{(b,0,[])}{le}})$ and $b\not\in H;h$. \\
  We have $(H;h, \fplug{LE}{\elet{x}{\enewbuf{n}{(lv:t)}}{le}})\;R\;(S, V, ss)$. \\
  We are to exhibit $C'$ so that \\
  $(H;h[b\mapsto lv^n], \subst{x}{(b,0,[])}{le})\;R\;C'$ and $(S, V, ss)\step^+C'$. \\
  Appealing to Lemma \ref{lemma-invert-newbuf}, we have $ss=(\earray{t}{x}{n};\memset{x}{n}{v};ss')$ and $le=V(\ctolow{ss'})$ and $lv=\ctolowe{v}$ and $LE=\unravel(S, \symhole)$ and $S=S';(M, V', E)$. \\
  Pick $C'$ to be $(S';(M[b\mapsto v^n], V', E), V[x\mapsto(b,0, [])], ss')$. \\
  It suffices to show that $(H;h[b\mapsto lv^n], \fplug{LE}{\subst{x}{(b,0,[])}{le}})\;R\;(S';(M[b\mapsto v^n], V', E), V[x\mapsto(b,0,[])], ss')$, which is true. \\
  End of case. \\
  \\
  \emph{Case Newstruct}: on case $(H;h,\fplug{LE}{\elet{x}{\enewstruct{(lv:t)}}{le}})\step_{\symwrite\;(b,0,[])}(H;h[b\mapsto lv], \subst{x}{(b,0,[])}{le})$ and $b\not\in H;h$. \\
  We have $(H;h, \elet{x}{\enewstruct{(lv:t)}}{le})\;R\;(S, V, ss)$. \\
  We are to exhibit $C'$ so that \\
  $(H;h[b\mapsto lv], \fplug{LE}{\subst{x}{(b,0,[])}{le}})\;R\;C'$ and $(S, V, ss)\step^+C'$. \\
  Appealing to Lemma \ref{lemma-invert-newstruct}, we have $ss=(\earray{t}{x}{1};\memset{x}{1}{v};ss')$ and $le=V(\ctolow{ss'})$ and $lv=\ctolowe{v}$ and $LE=\unravel(S, \symhole)$ and $S=S';(M, V', E)$. \\
  Pick $C'$ to be $(S';(M[b\mapsto v], V', E), V[x\mapsto(b,0, [])], ss')$. \\
  It suffices to show that $(H;h[b\mapsto lv], \fplug{LE}{\subst{x}{(b,0,[])}{le}})\;R\;(S';(M[b\mapsto v], V', E), V[x\mapsto(b,0,[])], ss')$, which is true. \\
  End of case. \\
  \\
  \emph{Case Readbuf}: on case \\
  $(H, \fplug{LE}{\elet{x:t}{\ereadbuf{(b,n,[])}{n'}}{le}})\step_{\symread\;(b,n+n',[])}(H, \fplug{LE}{\subst{x}{lv}{le}})$ and $H(b,n+n',[])=lv$. \\
  Appealing to Lemma \ref{lemma-invert-readbuf}, we have $ss=(t\;x=(b,n,[])[n'];ss')$ and $le=V(\ctolow{ss'})$ and $LE=\unravel(S, \symhole)$. \\
  Pick $C'$ to be $(S, V[x\mapsto v], ss')$ where $v=\lowtoce{lv}$. \\
  We know $C\step^+_{\symread\;(b,n+n',[])}C'.$ \\
  It suffices to show that $(H, \fplug{LE}{\subst{x}{lv}{le}})\;R\;(S, V[x\mapsto v], ss')$, which is true because $\subst{x}{lv}{le}=V[x\mapsto v](\ctolow{ss'})$. \\
  End of case. \\
  \\
  \emph{Case Readstruct}: on case \\
  $(H, \fplug{LE}{\elet{x:t}{\ereadstruct{(b,n,\ls{fd})}}{le}})\step_{\symread\;(b,n,\ls{fd})}(H, \fplug{LE}{\subst{x}{lv}{le}})$ and $H(b,n,\ls{fd})=lv$. \\
  Appealing to Lemma \ref{lemma-invert-readstruct}, we have $ss=(t\;x=\eread{(b,n,\ls{fd})};ss')$ and $le=V(\ctolow{ss'})$ and $LE=\unravel(S, \symhole)$. \\
  Pick $C'$ to be $(S, V[x\mapsto v], ss')$ where $v=\lowtoce{lv}$. \\
  We know $C\step^+_{\symread\;(b,n,\ls{fd})}C'.$ \\
  It suffices to show that $(H, \fplug{LE}{\subst{x}{lv}{le}})\;R\;(S, V[x\mapsto v], ss')$, which is true because $\subst{x}{lv}{le}=V[x\mapsto v](\ctolow{ss'})$. \\
  End of case. \\
  \\
  \emph{Case Writebuf}: on case \\
  $(H, \fplug{LE}{\elet{\_}{\ewritebuf{(b,n,[])}{n'}{lv}}{le}})\step_{\symwrite\;(b,n+n',[])}(H[(b,n+n',[])\mapsto lv], \fplug{LE}{le})$ and $(b,n+n',[])\in H$. \\
  Appealing to Lemma \ref{lemma-invert-writebuf}, we have $ss=((b,n,[])[n']=v;ss')$ and $le=V(\ctolow{ss'})$ and $lv=\ctolowe{v}$ and $LE=\unravel(S, \symhole)$. \\
  Pick $C'$ to be $(S', V, ss')$ where $\symset(S, (b,n,[]), v) = S'$. \\
  We know $C\step^+_{\symwrite\;(b,n+n',[])}C'.$ \\
  It suffices to show that $(H[(b,n+n',[])\mapsto lv], \fplug{LE}{le})\;R\;(S', V, ss')$, which is true. \\
  End of case. \\
  \\
  \emph{Case Writestruct}: on case \\
  $(H, \fplug{LE}{\elet{\_}{\ewritestruct{(b,n,\ls{fd})}{lv}}{le}})\step_{\symwrite\;(b,n,\ls{fd})}(H[(b,n,\ls{fd})\mapsto lv], \fplug{LE}{le})$ and $(b,n,\ls{fd})\in H$. \\
  Appealing to Lemma \ref{lemma-invert-writestruct}, we have $ss=(\ewrite{(b,n,\ls{fd})}{v};ss')$ and $le=V(\ctolow{ss'})$ and $lv=\ctolowe{v}$ and $LE=\unravel(S, \symhole)$. \\
  Pick $C'$ to be $(S', V, ss')$ where $\symset(S, (b,n,\ls{fd}), v) = S'$. \\
  We know $C\step^+_{\symwrite\;(b,n,\ls{fd})}C'.$ \\
  It suffices to show that $(H[(b,n,\ls{fd})\mapsto lv], \fplug{LE}{le})\;R\;(S', V, ss')$, which is true. \\
  End of case. \\
  \\
  \emph{Case Subbuf}: on case $(H, \fplug{LE}{\esubbuf{(b,n,[])}{n'}})\step(H, \fplug{LE}{(b,n+n',[])})$. \\
  Pick $C'$ to be $(S, V, ss)$. \\
  Because $(H, \fplug{LE}{\esubbuf{(b,n,[])}{n'}})\;R\;C'$ with some $m$, it must be the case that $m\geq 1$ and $(H, \fplug{LE}{(b,n+n',[])})\;R\;C'$ with $m-1$. \\
  End of case. \\
  \\
  \emph{Case Structfield}: on case $(H, \fplug{LE}{\estructfield{(b,n,\ls{fd})}{fd'}})\step(H, \fplug{LE}{(b,n,(\ls{fd}; fd'))})$. \\
  Pick $C'$ to be $(S, V, ss)$. \\
  Because $(H, \fplug{LE}{\estructfield{(b,n,\ls{fd})}{fd'}})\;R\;C'$ with some $m$, it must be the case that $m\geq 1$ and $(H, \fplug{LE}{(b,n,(\ls{fd};fd'))})\;R\;C'$ with $m-1$. \\
  End of case. \\
  \\
  \emph{Case IfTrue}: on case $(H, \fplug{LE}{\eif{n}{le_1}{le_2}})\step_\brt (H, \fplug{LE}{le_1})$ and $n\not=0$. \\
  Appealing to Lemma \ref{lemma-invert-if}, we have $ss=\eif{e}{ss_1}{ss_2};ss'$ and $\eval{e}{(p,V)}=n$ and $le_i=V(\ctolow{ss_i})$ ($i=1,2$) and $LE=\unravel(S, V(\ctolowE{(\symhole;ss')}))$. \\
  Pick $C'$ to be $(S, V, ss_1;ss')$. \\
  We know $C\step^+_\brt C'.$ \\
  It suffices to show that $(H, \fplug{LE}{le_1})\;R\;(S, V, ss_1;ss')$, which is true. \\
  End of case. \\
  \\
  \emph{Case IfFalse}: on case $(H, \fplug{LE}{\eif{n}{le_1}{le_2}})\step_\brf (H, \fplug{LE}{le_2})$ and $n=0$. \\
  Similar to previous case. \\
  End of case. \\
  \\
  \emph{Case Proj}: on case $(H, \fplug{LE}{\{\ls{fd=lv}\}.fd'})\step(H, \fplug{LE}{lv'})$ and $\{\ls{fd=lv}\}(fd')=lv'$. \\
  Pick $C'$ to be $(S, V, ss)$. \\
  Because $(H, \fplug{LE}{\{\ls{fd=lv}\}.fd'})\;R\;C'$ with some $m$, it must be the case that $m\geq 1$ and $(H, \fplug{LE}{lv'})\;R\;C'$ with $m-1$. \\
  End of case. \\
  \\
  \emph{Case Pop}: on case $(H;h, \fplug{LE}{\epop\;lv})\step(H, \fplug{LE}{lv})$. \\
  Apply Lemma \ref{lemma-invert-pop}. \\
  In the first case, from $LE=\unravel(S', V'(\ctolowE{\symhole;ss'}))$ we know $LE=(\elet{\_}{\symhole}{le})$ and $le=\unravel(S',V'(ss'))$. \\
  pick $C'$ to be $(S',V',ss')$ and $(H'', le'')$ to be $(H, le)$. \\
  Obviously $(H, \fplug{LE}{lv})\step^*(H, le)$. It suffices to show that $(H,le)\;R\;(S',V',ss')$, which is true. \\
  In the second case, pick $C'$ to be $(S',V',\fplug{E}{v})$ and $(H'', le'')$ to be $(H, \fplug{LE}{lv})$. \\
  To suffices to show $(H, \fplug{LE}{lv})\;R\;(S', V', \fplug{E}{v})$, which follows from $LE=\unravel(S',V'(\ctolowE{E}))$. \\
  \\
  For Condition 3(b), because \lamstar and \cstar's values are almost the same (except that \cstar locations have a field-path component), every \lamstar value has an obvious corresponding \cstar value, so condition 3(b) is trivially true.
\end{proof}

\begin{lemma}[\cstar Deterministic] \label{lemma-cstar-deter}
  For all $p$ and $ss$, transition system $\sys_{\cstar}(p, ss)$ is deterministic, modulo renaming of block identifiers.
\end{lemma}

\begin{lemma}[\lamstar Deterministic] \label{lemma-lowstar-deter}
  For all $lp$ and $le$, transition system $\sys_{\lamstar}(lp, le)$ is deterministic, modulo renaming of block identifiers.
\end{lemma}

\begin{lemma}[Init] \label{lemma-init}
  For all \lamstar program $lp$, closed expression $le$ and closing substitution $V$, if $\lowtocd{lp}=p$ and $\lowtoc{le}=ss$, then $([],V(le))\;R_{p,lp}\;([],V,ss)$.
\end{lemma}
\begin{proof}
  Unfold $R$'s definition, it suffices to show: \\
  $([], V(le))\step^*([], V(\ctolow{(\normal{\lowtoc{le}}{(p, \{\})})}))$.
\end{proof}

\begin{lemma}[Equal-Normalize] \label{lemma-eq-normal}
  If $H=\mem(S)$ and $le=\unravel(S, V(\ctolow{ss}))$ and $\normal{ss}{(p,V)}=ss'$, then $(H, le)\step^*(H, \unravel(S, V(\ctolow{ss'})))$.
\end{lemma}


\begin{lemma}[Invert Let] \label{lemma-invert-let}
  If $(H, \fplug{LE}{\elet{x:t}{lv}{le}})\;R\;(S, V, ss)$, then either $ss=(t\;x=e;ss')$ and $le=V(\ctolow{ss'})$ and $lv=\ctolowe{v}$ and $LE=\unravel(S, \symhole)$, where $v\defeq\eval{e}{(p,V)}$ or $S=S';(\None, V', t\;x=\symhole;ss')$ and $ss=\ereturn{v}$ and $le=V'(\ctolow{ss'})$ and $lv=\ctolowe{v}$ and $LE=\unravel(S', \symhole)$.
\end{lemma}

\begin{lemma}[Invert ALet] \label{lemma-invert-alet}
  If $(H, \fplug{LE}{\elet{\_}{lv}{le}})\;R\;(S, V, ss)$, then either $ss=(e;ss')$ and $le=V(\ctolow{ss'})$ and $lv=\ctolowe{v}$ and $LE=\unravel(S, \symhole)$, where $v\defeq\eval{e}{(p,V)}$ or $S=S';(\None, V', \symhole;ss')$ and $ss=\ereturn{v}$ and $le=V'(\ctolow{ss'})$ and $lv=\ctolowe{v}$ and $LE=\unravel(S', \symhole)$.
\end{lemma}

\begin{lemma}[Invert Newbuf] \label{lemma-invert-newbuf}
  If $(H;h, \fplug{LE}{\elet{x}{\enewbuf{n}{(lv:t)}}{le}})\;R\;(S, V, ss)$, then $ss=(\earray{t}{x}{n};\memset{x}{n}{v};ss')$ and $le=V(\ctolow{ss'})$ and $lv=\ctolowe{v}$ and $LE=\unravel(S, \symhole)$ and $S=S';(M, V', E)$.
\end{lemma}

\begin{lemma}[Invert Newstruct] \label{lemma-invert-newstruct}
  If $(H;h, \fplug{LE}{\elet{x}{\enewstruct{(lv:t)}}{le}})\;R\;(S, V, ss)$, then $ss=(\earray{t}{x}{1};\memset{x}{1}{v};ss')$ and $le=V(\ctolow{ss'})$ and $lv=\ctolowe{v}$ and $LE=\unravel(S, \symhole)$ and $S=S';(M, V', E)$.
\end{lemma}

\begin{lemma}[Invert Readbuf] \label{lemma-invert-readbuf}
  If \\
  $(H;h, \fplug{LE}{\elet{x:t}{\ereadbuf{(b,n,[])}{n'}}{le}})\;R\;(S, V, ss)$, then $ss=(t\;x=(b,n,[])[n'];ss')$ and $le=V(\ctolow{ss'})$ and $LE=\unravel(S, \symhole)$.
\end{lemma}

\begin{lemma}[Invert Readstruct] \label{lemma-invert-readstruct}
  If \\
  $(H;h, \fplug{LE}{\elet{x:t}{\ereadstruct{(b,n,\ls{fd})}}{le}})\;R\;(S, V, ss)$, then $ss=(t\;x=\eread{(b,n,\ls{fd})};ss')$ and $le=V(\ctolow{ss'})$ and $LE=\unravel(S, \symhole)$.
\end{lemma}

\begin{lemma}[Invert Writebuf] \label{lemma-invert-writebuf}
  If \\
  $(H;h, \fplug{LE}{\elet{\_}{\ewritebuf{(b,n,[])}{n'}{lv}}{le}})\;R\;(S, V, ss)$, then $ss=((b,n,[])[n']=v;ss')$ and $le=V(\ctolow{ss'})$ and $lv=\ctolowe{v}$ and $LE=\unravel(S, \symhole)$.
\end{lemma}

\begin{lemma}[Invert Writestruct] \label{lemma-invert-writestruct}
  If \\
  $(H;h, \fplug{LE}{\elet{\_}{\ewritestruct{(b,n,\ls{fd})}{lv}}{le}})\;R\;(S, V, ss)$, then $ss=(\ewrite{(b,n,\ls{fd})}{v};ss')$ and $le=V(\ctolow{ss'})$ and $lv=\ctolowe{v}$ and $LE=\unravel(S, \symhole)$.
\end{lemma}

\begin{lemma}[Invert App] \label{lemma-invert-app}
  If $(H, \fplug{LE}{\elet{x:t}{f\;lv}{le}})\;R\;(S, V, ss)$, then $ss=(t\;x=f(v);ss')$ and $le=V(\ctolow{ss'})$ and $lv=\ctolowe{v}$ and $LE=\unravel(S, \symhole)$.
\end{lemma}

\begin{lemma}[Invert Withframe] \label{lemma-invert-withframe}
  If $(H, \fplug{LE}{\withframe\;le})\;R\;(S, V, ss)$, then $ss=\{ss_1\};ss_2$ and $le=V(\ctolow{ss_1})$ and $LE=\unravel(S, V(\ctolowE{(\symhole;ss_2)}))$.
\end{lemma}

\begin{lemma}[Invert If] \label{lemma-invert-if}
  If $(H, \fplug{LE}{\eif{n}{le_1}{le_2}})\;R\;(S, V, ss)$, then $ss=\eif{e}{ss_1}{ss_2};ss'$ and $\eval{e}{(p,V)}=n$ and $le_i=V(\ctolow{ss_i})$ ($i=1,2$) and $LE=\unravel(S, V(\ctolowE{(\symhole;ss')}))$.
\end{lemma}

\begin{lemma}[Invert Pop] \label{lemma-invert-pop}
  If $(H;h, \fplug{LE}{\epop\;lv})\;R\;(S, V, ss)$, then either $ss=e$ and $\eval{e}{(p,V)}=v$ and $\ctolowe{v}=lv$ and $S=S';(M, V', \symhole;ss')$ and $LE=\unravel(S', V'(\ctolowE{\symhole;ss'}))$, or $ss=\ereturn{e}$ and $\eval{e}{(p,V)}=v$ and $\ctolowe{v}=lv$ and $S=S';(M,V',E)$ and $LE=\unravel(S',V'(\ctolowE{E}))$.
\end{lemma}

\begin{lemma}[Low2C-C2Low] \label{lemma-lowtoc-ctolow}
  If $\lowtoc{le}=ss;e$, then $\ctolow{(ss;\ereturn\;e)}=le$.
\end{lemma}




%% taramana 2016-10-06: from \cstar to CompCert C and beyond
\section{From \cstar to CompCert C and beyond}

To further back our claim that KreMLin offers a practical yet
trustworthy way to preserve security properties of \fstar programs down to
the executable code, we have to demonstrate that security guarantees
can be propagated from \cstar down to assembly.

Our idea here is to use the CompCert verified C compiler
\cite{Leroy-Compcert-CACM,compcert-url}. CompCert formally proves the
preservation of functional correctness guarantees from C down to
assembly code (for x86, PowerPC and ARM platforms.)

\ignore{
\subsection{Side channels and verified compilation}

Regarding side-channel security guarantees, one has to be careful
because the compiler may very well introduce additional memory
accesses due to register allocation. Thus, although we prove the
preservation of the sequences of memory accesses from \lowstar to \cstar, as
well as the preservation of branchings in the program counter model,
it may very well be the case that compiled code accesses many more
\emph{spilling locations} in one secret-controlled branch (PW: but
there is no secret-controlled branch (TR: you are right in the source,
but experiments should be performed to show that the compiler does not
introduce any)) than in another, yielding potentially different cache
behaviors, which an attacker could exploit to deduce some secrets.

To mitigate this issue, there are several solutions. One by Barthe et
al. \cite{barthe-ccs2014}, consists in performing a static analysis on
the Mach code (after most CompCert passes, but just before assembly
generation), with alias analysis to determine the sequence of memory
accesses and prove that they do not change across different
secret-controlled branches. This way, the security property is
guaranteed, actually turning CompCert into a certifying compiler to
this respect. That work has been extended to a full-scale LLVM static
analysis in \cite{almeida-usenix2016}.

Another solution (not explored yet, due to lack of time) would be to
directly make all intermediate passes annotate the generated code with
hints about memory accesses they are adding, and use those annotations
to introduce a pass (just before assembly generation) adding unused
spilling memory accesses to guarantee that those locations are always
accessed across all secret-controlled branches. This way, we could
actually avoid performing a full-scale static analysis on the code, by
distinguishing between source-level memory accesses to stack-allocated
variables (which are exactly preserved) and additional spilling memory
accesses. In practice, this approach could avoid the risk of static
analysis failures.

We claim that the latter approach (although not modular, since it
would require pervasive modifications across all passes of CompCert)
is feasible because all additional memory accesses introduced by
CompCert are at constant offsets within the stack block of the
function; and we claim that those memory accesses are not leaked
through function calls, neither directly nor by pointer arithmetics
(if they were, then they would correspond to no valid operation in the
source program.)

However, generating Clight code (as we describe below) and then using
Barthe et al.'s instrumented CompCert \cite{barthe-ccs2014} or Almeida
et al.'s instrumented LLVM \cite{almeida-usenix2016} is already
practical enough and, at least at the metatheoretical level, already
gives enough guarantees about the preservation of side-channel
security properties.
}

\subsection{Reminder: CompCert Clight}

CompCert Clight \cite{Blazy-Leroy-Clight-09} is a subset of C with no
side effects in expressions, and actual byte-level representation of
values. Syntax in Figure~\ref{fig:clight-syntax}. Semantics
definitions in Figure~\ref{fig:clight-semantics-defs}. Evaluation of
expressions in Figure~\ref{fig:clight-expr}. Small-step semantics in
Figure~\ref{fig:clight-stmts-reduction}.

The semantics of a Clight program is given by the return value of its
$\kw{main}$ function called with no arguments.\footnote{CompCert does
  not support semantics preservation with system arguments.}  Thus,
given a Clight program $p$, the initial configuration of a CompCert
Clight transition from $p$ is $(\{ \}, [], [], [], \kw{int} ~ r =
\kw{main}())$, and a configuration is final with return value $i$ if,
and only if, it is of the form $(\{ \}, \_, \_V, \_, [])$ with $\_V(r)
= i$.

% TODO: move definition to main
\newcommand{\claddr}[1]{\kw{\&}{#1}}
\newcommand{\clderef}[1]{\kw{*}{#1}}
\newcommand{\clread}[3]{{#1} =_{#2} \left\lbrack{#3} \right\rbrack}
\newcommand{\clwrite}[3]{{#2} =_{#1} {#3}}
\newcommand{\clvalloc}{\kw{valloc}}
\newcommand{\clalloc}{\kw{Alloc}}
\newcommand{\clunkn}{\kw{unkn}}
\newcommand{\clannot}{\kw{annot}}

\begin{figure}[h]
\begin{center}
  \begin{tabularx}{\columnwidth}{rlR}
    $p ::= $ & & program \\
      & $\ls d$                      & series of declarations \\[1.2mm]

    $d ::= $ & & declaration \\
    & $\ecfun fxtt{\ls{ad},ss}$                & top-level function \\
    & & with stack-allocated local variables $\ls{ad}$ \\
    & ${ad}$ & top-level value \\ [1.2mm]
    
    ${ad} ::= $ & & array declaration \\
    & $\earray{t}{x}{n}{}$               & uninitialized global variable \\
    [1.2mm]

    $ss ::= $ & & statement lists \\
    & $\ls{s}$                  &  \\
    [1.2mm]
    
    $s ::= $ & & statements \\
    & $\_x = e$ & assign rvalue to a non-stack-allocated local variable \\
    & $\_x = {\eapply fe}$                  & application \\
    & $\clread{\_x}{t}{e}$               & memory read from lvalue \\
    & $\clwrite tee$               & memory write rvalue to lvalue \\
    & $\clannot(\symread,t,e)$ & annotation to produce read event \\
    & $\clannot(\symwrite,t,e)$ & annotation to produce write event \\
    & $\eif{e}{ss}{ss}$ & conditional \\
    & $\{\stmts\}$ &  block \\
    & \ereturn e & return \\
    [1.2mm]
    
    $e ::= $ & & expressions \\
    & $n$    & integer constant (rvalue) \\
      & $x$                           & stack-allocated variable (lvalue) \\
      & $\_x$                           & non-stack-allocated variable (rvalue) \\ 
     & $e_1+_te_2$                        & pointer add (rvalue, $e_1$ is a rvalue pointer to a value of type $t$ and $e_2$ is a rvalue $\kw{int}$) \\
    & $e._tfd$ & struct field projection (lvalue, $e$ lvalue) \\
    & $\claddr{e}$ & address of a lvalue (rvalue, $e$ lvalue) \\
    & $\clderef{e}$ & pointer dereference (lvalue, $e$ rvalue)
  \end{tabularx}
\end{center}
\caption{Clight Syntax}
\label{fig:clight-syntax}
\end{figure}

Just like C, there are two ways to evaluate Clight expressions: in
lvalue position or in rvalue position. Roughly speaking, in an
expression assignment $\clwrite t{e_l}{e_r}$, expression $e_l$ is said to be at
lvalue position and thus must evaluate into a memory location, whereas
$e_r$ is said to be at rvalue position and evaluates into a value
(integer or pointer to memory location). The operation $\claddr{e}$
takes an lvalue $e$ and transforms it into a rvalue, namely the
pointer to the memory location $e$ designates as an
lvalue. Conversely, $\clderef{e}$ takes an rvalue $e$, which must
evaluate to a pointer, and turns it into the corresponding memory
location as an lvalue.

\paragraph{Memory accesses in the trace}
To account for memory accesses in the trace, we make each statement
perform at most one memory access in our generated Clight code. Then,
we prepend each such memory access statement with a \emph{built-in
  call}, a no-op annotation $\clannot(\mathit{ev},t,e)$ whose
semantics is merely to produce the corresponding memory access event
$\mathit{ev}(b, n)$ in the trace, where $e$ evaluates to the pointer
to offset $n$ within block $e$.

\begin{figure}[h]
\begin{small}
\begin{center}
  \begin{tabularx}{\columnwidth}{rlR}
    $v ::= $ & & values \\
    & $n$    & integer constant \\
    & $(b, n)$                        & memory location \\
    & $\clunkn{}$                        & defined but unknown value \\
    [1.2mm]

    ${vf} ::= $ & & value fragments \\
    & $n$    & byte constant \\
    & $((b,n), n')$ & $n'$-th byte of pointer value $(b,n)$ \\
    & $\clunkn{}$                        & defined but unknown value fragment \\
    [1.2mm]
    
    $V ::=$ & & stack-allocated variable assignments \\
    & $x\sympartial b$ & map from variable to memory block identifier \\
    [1.2mm]

    $\_V ::=$ & & non-stack-allocated variable assignments \\
    & $\_x\sympartial v$ & map from variable to value \\
    [1.2mm]

    $E ::=$ & & evaluation ctx (plug expr to get stmts) \\
    & $\symhole; ss$ & discard returned value \\
    & $\_x=\symhole; ss$ & receive returned value \\
    [1.2mm]

    $F ::=$ & & frames \\
    & $(V, \_V, E)$ & stack frame \\
    [1.2mm]
    
    $S ::=$ & & stack \\
    & $\ls{F}$ & list of frames \\
    [1.2mm]

    $M ::=$ & & memory \\
    & $(b, n) \sympartial vf$ & map from block id and offset to value fragment \\
    [1.2mm]

    $C ::=$ & & configuration \\
    & $(S, V, \_V, M, ss)$ &  \\
    [1.2mm]
  \end{tabularx}
\end{center}
\end{small}
\caption{Clight Semantics Definitions}
\label{fig:clight-semantics-defs}
\end{figure}

% TODO: move to main.tex
\newcommand{\cllv}[2]{\kw{lv}({#1}, {#2})}
\newcommand{\clrv}[2]{\kw{rv}({#1}, {#2})}

\begin{figure*}[h]
\begin{scriptsize}
\begin{flushleft}
  \fbox{$\cllv{e}{(p, V, \_V)}=(b,n)$} \quad \text{and} \quad
  \fbox{$\clrv{e}{(p, V, \_V)}=v$}
\end{flushleft}
\begin{mathpar}
\inferrule* [Right=Var]
{
  V(x) = b
}
{
  \cllv{x}{(p,V, \_V)}=(b, 0)
} 

\quad
\quad
\quad

\inferrule* [Right=GVar]
{
  x \not\in V \\
  p(x) = b
}
{
  \cllv{x}{(p,V, \_V)}=(b, 0)
} 

\\

\inferrule* [Right=PtrFd]
{
  \cllv{e}{(p,V, \_V)} = (b, n) \\
}
{
  \cllv{e._t{fd}}{(p,V, \_V)}=(b, n + \textsf{offsetof}(t, fd))
}

\quad
\quad
\quad
\quad
\quad

\inferrule* [Right=PtrDeref]
{
  \clrv{e}{(p,V,\_V)} = (b,n)
}
{
  \cllv{\clderef{e}}{(p,V,\_V)} = (b,n)
}

\\

\inferrule* [Right=RVar]
{
  \_V(\_x) = v
}
{
  \clrv{\_x}{(p,V,\_V)}=v
} 

\quad
\quad
\quad

\inferrule* [Right=PtrAdd]
{
  \clrv{e_1}{(p,V,\_V)} = (b, n) \\
  \clrv{e_2}{(p,V,\_V)} = n' \\
}
{
  \clrv{e_1+e_2}{(p,V)}=(b, n+n')
} 

\quad
\quad
\quad
\quad

\inferrule* [Right=AddrOf]
{
  \cllv{e}{(p,V,\_V)}= (b, n)
}
{
  \clrv{\claddr{e}}{(p,V,\_V)}= (b,n)
}
\end{mathpar}
\end{scriptsize}
\caption{\label{fig:clight-expr}Clight Expression Evaluation}
\end{figure*}


\begin{figure*}[h]
\begin{scriptsize}
\begin{flushleft}
  \fbox{$p \vdash C\step_\olabel C'$}
\end{flushleft}
\begin{mathpar}
\inferrule* [Right=Read]
{
  \cllv{e}{(p, V, \_V)} = (b, n) \\
  \symget(M, (b, n), \kw{sizeof}(t)) = v
}
{
  p \vdash (S, V, \_V, M, \clread{\_x}{t}{e}; ss) \step (S, V, \_V[\_x\mapsto v], M, ss)
} 

\quad
\quad
\quad
\quad

\\

\inferrule* [Right=Write]
{
  \cllv{e_1}{(p, V, \_V)} = (b, n) \\
  \clrv{e_2}{(p, V, \_V)} = v \\
  \symset(M, (b, n), \kw{sizeof}(t), v) = S'
}
{
  p \vdash (S, V, \_V, M, \clwrite{t}{e_1}{e_2}; ss) \step (S', V, \_V, M, ss)
} 

\\

\inferrule* [Right=Annot]
{
  \cllv{e}{(p, V, \_V)} = (b, n) \\
}
{
  p \vdash (S, V, \_V, M, \clannot{\mathit{ev}, e}; ss) \step_{\mathit{ev}\;(b,n)}(S, V, \_V, M, SS)
}

\\

\inferrule* [Right=Ret]
{
  \clrv{e}{(p,V,\_V)}=v
}
{
  p \vdash (S;(V', \_V',E), V, M, \ereturn\;e; ss) \step (S, V', \_V', M, \fplug{E}{v})
}

\\

\inferrule* [Right=Call]
{
  p(f)=\ecfuntwo{\_y}{t_1}{t_2}{ads; ss_1} \\
  \clrv{e}{(p,V,\_V)}=v \\
  \clvalloc{}(ads, \None, M) = (V', M')
}
{
  p \vdash (S, V, \_V, M, \_x=f\;e; ss) \step (S;(V, \_V, \_x=\symhole;ss), V', \{\}[\_y\mapsto v], M', ss_1)
} 

\\

\inferrule* [Right=AllocNil]
{
}
{
  \clvalloc{}([], V, M) = (V, M)
}

\quad
\quad
\quad
\quad
\quad

\inferrule* [Right=AllocCons]
{
  \clalloc(M,n \times \kw{sizeof}(t)) = (b,M_1) \\
  \clvalloc{}(ads, V[x \mapsto (b,0)], M_1) = (V', M')
}
{
  \clvalloc{}((\earray{t}{x}{n}{}; ads), V, M) = (V', M')
}
  
\\

\inferrule* [Right=Expr]
{
  \clrv{e}{(p,V,\_V)} = v
}
{
  p \vdash (S, V, \_V, M, e; ss) \step (S, V, \_V, M, ss)
} 

\quad
\quad
\quad

\inferrule* [Right=Empty]
{
  \;
}
{
  p \vdash (S, V, \_V, M, []) \step (S, V, \_V, M, \ereturn\;\clunkn{})
} 

\\

\inferrule* [Right=Block]
{
  \;
}
{
  p \vdash (S, V, \_V, M, \{ss_1\};ss_2) \step (S; V, \_V, M, ss_1; ss_2)
} 

\\

\inferrule* [Right=IfT]
{
  \clrv{e}{(p, V, \_V)} = v \\
  v \not= 0 \\
  v \not= \clunkn{}
}
{
  p \vdash (S, V, \_V, M, \eif{e}{ss_1}{ss_2};ss) \step_\brt (S, V, ss_1;ss)
} 

\quad
\quad
\quad

\inferrule* [Right=IfF]
{
  \clrv{e}{(p, V, \_V)} = 0
}
{
  p \vdash (S, V, \_V, M, \eif{e}{ss_1}{ss_2};ss) \step_\brf (S, V, \_V, M, ss_2;ss)
} 
\end{mathpar}
\end{scriptsize}
\caption{\label{fig:clight-stmts-reduction}Clight Configuration Reduction}
\end{figure*}

\paragraph{The CompCert memory model}
The semantics of CompCert Clight statements depends on the CompCert
memory model \cite{Leroy-Blazy-memory-model}. Here we need three
operations: $\symget$, $\symset$ and $\clalloc$, whose description
follows.\footnote{In the current version of CompCert and its memory
  model \cite{2012-Leroy-Appel-Blazy-Stewart}, each memory location is
  equipped with a \emph{permission} to more faithfully model the fact
  that the memory locations of a local variable cannot be read or
  coincidentally reused in a Clight program after exiting the scope of
  the variable. To this end, thanks to this permission model, the
  CompCert memory model also defines a $\kw{free}$ operation which
  invalidates memory accesses while preventing from reusing the memory
  block for further allocations; although we do not describe it here,
  CompCert Clight actually uses this operator to free all local
  variables upon function exit. Thus, it is also necessary to amend
  the semantics of \cstar in a similar way.}

$\symget(M, (b, n), n')$ reads $n'$ bytes from memory $M$ at offset
$n$ within block $b$, and decodes the obtained byte fragments into a
value. It fails if not all locations from $(b, n)$ to $(b, n'-1)$ are
defined. It returns $\clunkn$ if all locations are defined but the
decoding fails.

$\symset(M, (b, n), n', v)$ writes $n'$ bytes from memory $M$ at
offset $n$ within block $b$ corresponding to encoding value $v$ into
$n'$ value fragments. It fails if not all locations from $(b, n)$ to
$(b, n'-1)$ are defined.

$\clalloc{}(M, n)$ returns a pair $(b, M')$ where $b \not\in M$ is a
fresh block identifier and all locations from $(b,0)$ to $(b,n-1)$ in
$M'$ contain $\clunkn$.

\subsection{Issues}

\paragraph{Local structures}

The main difference between \cstar and Clight is that \cstar allows structures
as values. Although converting a \cstar value into a Clight value is no
problem in terms of memory representation (since the layout of Clight
structures\footnote{which can be chosen when configuring CompCert with
  a suitable platform} is already formalized in CompCert with basic
proofs such as the fact that two distinct fields of a structure
designate disjoint sets of memory locations), local structures cause
issues in terms of managing memory accesses (due to our desire for
noninterference in terms of memory accesses), as we describe in
Section~\ref{sec:local-struct}.

\paragraph{Stack-allocated local variables}

Another difference between \cstar and Clight is that, whereas \cstar allows
the user to stack-allocate local variables on the fly, Clight mandates
all local variables of a function to be hoisted to the beginning of
the function (in fact, the list of all stack-allocated variables of a
function is actually part of the function definition), and so they are
allocated all at once when entering the function.

Hoisting local variables is not supported in the verified part of
CompCert.
\ignore{
In practice, when compiling a piece of C code with CompCert,
an unverified elaboration pass performs this operation.\footnote{See
  \cite{compcert-url}, section ``Architecture of the compiler''. By
  reading the source code of the unverified elaborator in
  \texttt{cparser/Unblock.ml}, we figured out the hoisting strategy
  adopted by CompCert 2.7.1, as we describe it here.} This unverified
phase of CompCert hoists all local variables of a function, without
trying to merge two local variables belonging to disjoint C code
blocks, contrary to GNU GCC
\footnote{If we were to follow the GNU GCC hoisting strategy, then we
  would have to think about how to merge two variables of different
  types. However, type annotations are present in CompCert until
  Clight, and CompCert then erases those type annotations in a pass
  from Clight to C$\sharp$minor, which is very similar to Clight
  except that it no longer has any C-style types (structs, unions,
  etc.) so that a stack-allocated variable only designates a memory
  block of some constant size in bytes. So, in that case, I claim that
  hoisting could and should be done at the level of C$\sharp$minor,
  rather than in Clight or KreMLin. However, this would require
  modifying CompCert Clight so that local variable allocation sites
  would be attached to code blocks rather than functions.}  Regardless
of the hoisting strategy, verifying this pass may be an interesting
problem within CompCert itself.

Instead of trying to prove hoisting on Clight or any language of
CompCert, we could perform this phase at the \cstar level. We propose a
solution in Section~\ref{sec:hoisting}.

However,
}

Consider the following \cstar example, for a given conditional expression
$e$:
  \[
  \eifthenelse
      {e}
      {\earray{\kint}{x}{1}{18}; }
      {\{ \earray{\kword}{x}{1}{42}; \ewrite{x+0}{1729} \} }
   \]
After hoisting, following the same strategy as the corresponding
unverified pass of CompCert, \cstar code will look like this:
    \[
  \earray{\kint}{x_1}{1}{};
  \earray{\kword}{x_2}{1}{};
  \]
  \[
  \eifthenelse
      {e}
      {\ewrite{x_1+0}{18}; }
      {\{ \ewrite{x_2+0}{42}; \ewrite{x_2+0}{1729} \} }
  \]
This example shows the following issue:
when producing the trace, the
  memory blocks corresponding to the accessed memory location will
  differ between the non-hoisted and the hoisted \cstar code. Indeed, in
  the non-hoisted \cstar code, only one variable $x$ is allocated in the
  stack, whereas in the hoisted \cstar code, two variables $x_1$ and $x_2$
  corresponding to both branches will be allocated anyway, regardless
  of the fact that only one branch is executed. Thus, in the \cstar code
  before hoisting, allocating $x$ will create, say, block 1, whereas
  after hoisting, two variables will be allocated, $x_1$ at block 1,
  and $x_2$ at block 2. Thus, the statement $\ewrite{x+0}{1729}$ in
  the \cstar code before hoisting will produce $\symread\;(1,0,[])$ on the
  trace, whereas its corresponding translation after hoisting,
  $\ewrite{x_2+0}{1729}$, will produce
  $\symread\;(2,0,[])$.

\ignore{
Anyway, a similar problem of discrepancy of traces of memory accesses
will also exist in CompCert passes after Clight, this time because the
structure of the stack will change across intermediate languages of
CompCert (in particular, at some point, all stack-allocated variables
of one function call will be merged into one single stack block for
the entire stack frame.) Thus, so far, CompCert will very well
preserve functional correctness down to the assembly, but will require
more work to \emph{preserve} security guarantees beyond Clight, unless
we instrument it as described before.
}

One quick solution to ensure that those event traces are exactly
preserved, is to replace the actual pointer $(b, n,
\mathit{fd})$ on $\symread$ and $\symwrite$ events with $(f, i, x, n,
\mathit{fd})$ where $f$ is the name of the function, $i$ is the
recursion depth of the function (which would be maintained as a global
variable, increased whenever entering the function, and decreased
whenever exiting) and $x$ is the local variable being accessed. (In
concurrent contexts, one could add a parameter $\theta$ recording the
identifier of the current thread within which $f$ is run, and so the
global variable maintaining recursion depth would become an array
indexed by thread identifiers.)

\ignore{
[Catalin: Isn't this much weakening the attacker model though?
          And how does this new model apply to machine code?]

[Tahina: This abstract trace model is only for the purpose of proofs
  of memory transformations, beginning with the \cstar to Clight
  proof. This model also applies to C$\sharp$minor (exact
  preservation, the only difference being the erasure of C-style type
  information), and it can also be applied to Cminor (where all local
  variables are merged into one single stack frame) when translating
  from C$\sharp$minor. Then, from Cminor on, it becomes useless. In
  fact, from Cminor on, the local variable region becomes contiguous
  for a given function call, and (with the notable exception of in the
  function inlining pass, which may very well merge several stack
  frames of inlined callees into one single one. Anyway, function
  inlining requires special treatment.) there are no pointer
  transformations until the time spilling locations are introduced in
  memory when generating assembly code.

  Once again, the ultimate trace model that we want for our final
  side-effect security correctness statement will definitely be the
  one with concrete pointer values, and abstract pointer values are
  only for the purpose of compiler verification.]
}

Finally, we adopted this solution as we describe in
Section~\ref{sec:norm-traces}, and we heavily use it to prove the
correctness of hoisting within \cstar in Section~\ref{sec:hoisting}.

\subsection{Summary: from \cstar to Clight}

Given a \cstar program $p$ and an entrypoint $ss$, we are going to
transform it into a CompCert Clight program in such a way that both
functional correctness and noninterference are preserved.

This will not necessarily mean that traces are exactly preserved
between \cstar and Clight, due to the memory representation discrepancy
described before. Instead, by functional correctness, we mean that a
safe \cstar program is turned into a safe Clight program, and for such
safe programs, termination, I/O events and return value are preserved;
and by noninterference, we mean that if two executions with different
secrets produce identical (whole) traces in \cstar, then they will also
produce identical traces in CompCert Clight (although the trace may
have changed between \cstar and Clight.)

\begin{enumerate}
\item In section \ref{sec:unambiguous-variables}, we transform a \cstar
  programs into a program with unambiguous variable names, and we take
  advantage of such a syntactic property by enriching the
  configuration of \cstar with more information regarding variable names,
  thus yielding the \cstar2 language.
\item In section \ref{sec:norm-traces-detail}, we execute the obtained \cstar2
  program with a different, more abstract, trace model, as proposed
  above. This is not a program transformation, but only a
  reinterpretation of the same \cstar program with a different operational
  semantics, which we call \cstar3.
\item In section \ref{sec:hoisting}, we transform the \cstar3 program into
  a \cstar3 program where all local arrays are hoisted from block-scope to
  function-scope. The abstract trace model critically helps in the
  success of this proof where memory state representations need to
  change.
\item In section \ref{sec:struct-return}, we transform the obtained
  \cstar3 program into a \cstar3 program where functions returning structures
  are replaced with functions taking a pointer to the return location
  as additional argument. Thus, we need to account for an additional
  memory access, which we do through the \cstar4 intermediate semantics,
  another reinterpretation of the source \cstar3 program producing new
  events at functions returning structures. Then, our reinterpreted
  \cstar4 program is translated back to \cstar3 with those additional memory
  accesses made explicit.
\item In section \ref{sec:struct-events}, we reinterpret our obtained
  \cstar3 program with a different event model where memory access events
  of structure type are replaced by the sequences of memory access
  events of all their non-structure fields. We call this new language
  \cstar5.
\item In section \ref{sec:local-struct-detail}, we transform our \cstar5
  program back into \cstar3 by erasing all local structures that are not
  local arrays, replacing them with their individual non-structure
  fields. Thus, the more ``elementary'' memory accesses introduced in
  the \cstar3 to \cstar5 reinterpretation are made concrete.
\item We then reinterpret the obtained \cstar3 program back into \cstar2 as
  described in \ref{sec:norm-traces-detail}, reverting to the traces
  with concrete memory locations at events, thus accounting for all
  memory accesses.
\item Finally, in section \ref{sec:clight-gen}, we compile the
  obtained \cstar2 program, now in the desired form, into CompCert Clight.
\end{enumerate}

\subsection{Normalized event traces in \cstar} \label{sec:norm-traces}

As described above, traces where memory locations explicitly appear
are notoriously hard to reason about in terms of semantics
preservation for verified compilation. Thus, it becomes desirable to
find a common representation of traces that can be preserved between
different memory layouts across different intermediate languages.

In particular here, we would like to replace concrete pointers into
abstract pointers representing the local variable being modified in a
given nested function call.

\subsubsection{Disambiguation of variable names} \label{sec:unambiguous-variables}

To this end, we first need to disambiguate the names of the local
variables of a \cstar function:

\begin{definition}[Unambiguous local variables]
  We say that a list of \cstar instructions has \emph{unambiguous local
    variables} if, and only if, it contains no two distinct array
  declarations with the same variable name, and does not contain both
  an array declaration and a non-array declaration with the same
  variable name.
  
  We say that a \cstar program has unambiguous local variables if, and
  only if, for each of its functions, its body has unambiguous local
  variables.

  We say that a \cstar transition system $\sys(p, V, ss)$ has unambiguous
  local variables if, and only if, $p$ has unambiguous local
  variables, $ss$ has unambiguous local variables, and $V$ does not
  define any variable with the same name as an array declared in $ss$.
\end{definition}

\begin{lemma}[\label{lemma-cstar-alpha} Disambiguation]
  There exists a transformation $T$ on lists of instructions (extended
  to programs by morphism) such that, for any \cstar program $p$, and for
  any list of \cstar instructions $ss$, %
  %% TODO taramana 2016-10-11: clarify any additional properties
  %% required for noninterference
  %%
  %% for any set of variables $D'$, there
  %% is a set of variables $D$ such that
  for any variable mapping $V'$ %with domain $D'$
  such that $\sys(T(p),V',T(ss))$ has unambiguous local variables,
  there exists a variable mapping $V$ % with domain $D$  
  such that $\sys(p,V,ss)$ and $\sys(T(p),V',T(ss))$ have the same
  execution traces.
\end{lemma}
\begin{proof} $\alpha$-renaming. \end{proof}

The noninterference property can be proven to be stable by such
$\alpha$-renaming:

%% \begin{lemma}
%%   For any \cstar program $p$ and list of instructions $ss$, if $\dom V
%%   \subseteq \dom V'$ and if $\sys(p,V,ss)$ is safe, then
%%   $\sys(p,V,ss)$ and $\sys(p,V',ss)$ have the same execution traces.
%% \end{lemma}
%% \begin{proof}
%%   Considering $n$ steps of execution of each of the two systems, by
%%   induction over $n$ (lock-step bisimulation.)
%% \end{proof}

\begin{lemma}
  Let $p$ be a \cstar program and $ss$ be a list of instructions. Assume
  that for any $V_1, V_2$ such that $\sys(p, V_1, ss)$ and $\sys(p,
  V_2, ss)$ are both safe, then they have the same execution
  traces.

  Then, for any $V_1', V_2'$ such that $\sys(T(p), V_1', T(ss))$ and
  $\sys(T(p), V_2', T(ss))$ are both safe, they have the same
  execution traces.
\end{lemma}
\begin{proof}
  By Lemma~\ref{lemma-cstar-alpha}, there is some $V_1$ such that
  $\sys(p, V_1, ss)$ and $\sys(T(p), V_1', T(ss))$ have the same
  execution traces, thus in particular, $\sys(p, V_1, ss)$ is
  safe. Same for some $V_2$. By hypothesis, $\sys(p, V_1, ss)$ and
  $\sys(p, V_2, ss)$ have the same execution traces, thus the result
  follows by transitivity of equality.
\end{proof}

Thus, we can now restrict our study to \cstar programs whose functions
have no two distinct array declarations with the same variable names.

Let us first enrich the configuration $(S, V, ss)$ of \cstar small-step
semantics with additional information recording the current function
$f$ being executed (or maybe $\None$) and the set $A$ of the variable
names of local arrays currently declared in the scope. Thus, a \cstar
stack frame $(\None, V, E)$ becomes $(\None, V, E, f, A)$ where $f$ is
the caller, a block frame $(M, V, E)$ becomes $(M, V, E, f, A)$ where
$f$ is the enclosing function of the block, and the configuration $(S,
V, ss)$ becomes $(S, V, ss, f, A)$ where $f$ is the current function
(with the frames of $S$ changed accordingly.) Let us then change some
rules accordingly as described in Fig.~\ref{fig:cstar-2}, leaving
other rules unchanged except with the corresponding $f$ and $A$
components preserved.

\begin{figure*}
\begin{small}
  \begin{mathpar}
\inferrule* [Right=$\text{ArrDecl}_2$]
{
  \eval{e}{(p,V)}=v \\
  S = S'; (M, V, E, f', A') \\
  b\not\in S
}
{
  p \vdash (S, V, t\;x[n]=e; ss, f, A) \step (S';(M[b\mapsto v^n], V, E, f', A'), V[x\mapsto (b, 0, [])], ss, f, A\cup\{x\})
}

\\

\inferrule* [Right=$\text{Ret}_2$]
{
  \eval{e}{(p,V)}=v
}
{
  p \vdash (S;(\None, V',E, f', A'), V, \ereturn\;e; ss, f, A) \step (S, V', \fplug{E}{v}, f', A')
}

\\
    
\inferrule* [Right=$\text{Call}_2$]
{
  p(f)=\ecfuntwo{y}{t_1}{t_2}{ss_1} \\
  \eval{e}{(p,V)}=v
}
{
  p \vdash (S, V, t\;x=f\;e; ss, f', A') \step (S;(\None, V, t\;x=\symhole;ss, f', A'), \{\}[y\mapsto v], ss_1, f, \{\})
} 

  \end{mathpar}
\end{small}
\caption{\cstar2 Amended Configuration Reduction} \label{fig:cstar-2}
\end{figure*}

Let us call \cstar2 the obtained language where the initial state of the
transition system $\sys(p, V, ss)$ shall now be $([], V, ss, \None,
[])$.

Then, it is easy to prove the following:
\begin{lemma}[\label{lemma-cstar-to-cstar-2]}\cstar to \cstar2]
    If $\sys(p, V, ss)$ has unambiguous local variables and is safe in
    \cstar, then it has the same execution traces in \cstar as in \cstar2 (and in
    particular, it is also safe in \cstar2.)

    Thus, both functional correctness and noninterference are
    preserved from \cstar to \cstar2.
\end{lemma}
\begin{proof}
  Lock-step bisimulation where the common parts of the configurations
  (besides the \cstar2-specific $f, A$ parts) are equal between \cstar and
  \cstar2.
\end{proof}

Then, we can prove an invariant over the small-step execution of a \cstar2
program:
\begin{lemma}[\label{lemma-cstar-2-invar}\cstar2 invariant]
Let $p$ be a \cstar2 program and $V$ a variable environment such that
$\sys(p,V,ss)$ has unambiguous local variables.

Let $n \in \mathbb N$. Then, for any \cstar2 configuration $(S, V', ss',
f, A)$ obtained after $n$ \cstar2 steps from $(\{\}, V, ss, \None, \{\})$,
the following invariants hold:
\begin{enumerate}
\item for any variable or array declaration $x$ in $ss'$, it does not
  appear in $A_1$
\item any variable name in $A$ or in an array declaration of $ss'$ is
  in an array declaration of $ss$ (if $f = \None$) or the body of $f$
  (otherwise.)
\item for each frame of $S$ of the form $(\_, \_, E, f'', A'')$, then
  any variable name or array declaration in $E$ does not appear in
  $A''$, and any variable name in $A''$ or in an array declaration of
  $E$ is in an array declaration in $ss$ (if $f'' = \None$) or the
  body of $f''$ (otherwise.)
\item \label{cstar-2-invar-memory-block-nil} if $S = S'; (M, \_, \_,
  f', A')$ with $M \not= \None$, then $f' = f$, $A' \subseteq A$ and
  for all block identifiers $b$ defined in $M$, there exists a unique
  variable $x \in A$ such that $V'(x) = b$
\item \label{cstar-2-invar-memory-block-cons} for any two consecutive
  frames $(M, \_, \_, f_1, A_1)$ just below $(\_, V_2, \_, f_2, A_2)$
  with $M \not= \None$, then $f_1 = f_2$ and $A_1 \subseteq A_2$ and
  $V_1(x) = V_2(x)$ for all variables $x \in A_1$, and for all block
  identifiers $b$ defined in $M$, there exists a unique variable $x
  \in A_2$ such that $V_2(x) = b$
\item \label{cstar-2-invar-memory-disjoint} for any two different
  frames of $S$ of the form $(M_1, \_, \_, \_, \_)$ and $(M_2, \_, \_,
  \_, \_)$ with $M_1 \not= \None$ and $M_2 \not= \None$, the sets of
  block identifiers of $M_1$ and $M_2$ are disjoint
\end{enumerate}
\end{lemma}
\begin{proof}
  By induction on $n$ and case analysis on $\step$.
\end{proof}

Then, we can prove a strong invariant between two executions of
the same \cstar2 program with different secrets. This strong invariant
will serve as a preparation towards changing the event traces of \cstar2.

\begin{lemma}[\label{lemma-cstar-2-noninterference-invar} \cstar2 noninterference invariant]
Let $p$ be a \cstar2 program, and $V_1, V_2$ be two variable environments
such that $\sys(p, V_1, ss)$ and $\sys(p, V_2, ss)$ have unambiguous
local variables, are both safe and produce the same traces in \cstar2.

Let $n \in \mathbb N$. Then, for any two \cstar2 configurations $(S_1,
V_1', ss_1, f_1, A_1)$ and $(S_2, V_2', ss_2, f_2, A_2)$ obtained
after $n$ \cstar2 execution steps, the following invariants hold:
\begin{itemize}
\item $ss_1 = ss_2$
\item $S_1, S_2$ have the same length
\item $f_1 = f_2$
\item $A_1 = A_2$
\item $V_1(x) = V_2(x)$ for each $x \in A_1$
\item for each $i$, if the $i$-th frames of $S_1, S_2$ are $(M_1,
  V_1'', E_1, f_1'', A_1'')$ and $(M_2, V_2'', E_2, f_2'', A_2'')$,
  then $E_1 = E_2$, $f_1'' = f_2''$ and $A_1'' = A_2''$ and $V_1''(x)
  = V_2''(x)$ for any $x \in A_1''$. Moreover, $M_1 = \None$ if and
  only if $M_2 = \None$, and if $M_1 \not= \None$, then $M_1$ and
  $M_2$ have the same block domain.
\end{itemize}
Thus, the $n+1$-th step in both executions applies the same \cstar2 rule.
\end{lemma}
\begin{proof}
  By induction on $n$ and case analysis on $\step$, also using the
  invariant of Lemma~\ref{lemma-cstar-2-invar}. In particular the
  equality of codes is a consequence of the fact that there are no
  function pointers in \cstar.\footnote{If we were to allow function
    pointers in \cstar, then we would have to add function call/return
    events into the \cstar trace beforehand, and assume that traces with
    those events are equal before renaming = prove that they are equal
    on the \lowstar program as well. This might have consequences in the
    proof of function inlining in the \fstar-to-\cstar translation.} Then,
  both executions apply the same \cstar2 rules since \cstar2 small-step rules
  are actually syntax-directed.
\end{proof}

\subsubsection{Normalized traces} \label{sec:norm-traces-detail}

Now, consider an execution of \cstar2 from some initial state. In fact,
for any block identifier $b$ defined in $S$, it is easy to prove that
it actually corresponds to some variable defined in the scope. The
corresponding \textsc{VarOfBlock} algorithm is shown in
Figure~\ref{fig:cstar-2-block-to-variable}. \ignore{ \textbf{FIXME: global
  variables}\footnote{Global variables (top-level values) are not
  allocated anywhere in the memory of a \cstar program, so \cstar cannot
  access them. \cstar can only access the local variables of the
  entrypoint list of statements, which is not part of the program.}}

\begin{figure}
  \textbf{Algorithm:} \textsc{VarOfBlock}
  
  \textbf{Inputs:}
  \begin{itemize}
    \item \cstar2 configuration $(S, V, \_, \_, A)$ such that the invariants
      of Lemma~\ref{lemma-cstar-2-invar} hold
    \item Memory block $b$ defined in $S$
  \end{itemize}
  \textbf{Output:} function, recursion depth and local variable
  corresponding to the memory block

  Let $S = S_1 ; (M, \_, \_, f, \_) ; S_2$ such that $b$ defined in
  $M$. (Such a decomposition exists and is unique because of
  Invariant~\ref{cstar-2-invar-memory-disjoint}. $f$ may be $\None$.)

  Let $n$ be the number of frames in $S_1$ of the form $(\None, \_,
  \_, f', \_)$ with $f' = f$.

  Let $V'$ and $A'$ such that $S_2 = (\_, V', \_, \_, A) ; \_$, or $V'
  = V$ and $A' = A$ if $S_2 = \{\}$.

  Let $x$ such that $V'(x) = (b, 0)$ (exists and is unique because of
  Invariants~\ref{cstar-2-invar-memory-block-nil}
  and~\ref{cstar-2-invar-memory-block-cons}.)

  \textbf{Result:} $(f, n, x)$
  
  \caption{\cstar2: retrieving the local variable corresponding to a
    memory block}
  \label{fig:cstar-2-block-to-variable}
\end{figure}

Then, let \cstar3 be the \cstar2 language where the \textsc{Read} and
\textsc{Write} rules are changed according to
Figure~\ref{fig:cstar-3}, with event traces where the actual pointer
is replaced into an abstract pointer obtained using the
\textsc{VarOfBlock} algorithm above.


\begin{figure*}
\begin{small}
  \begin{mathpar}
\inferrule* [Right=$\text{Read}_3$]
{
  C = (S, V, t\;x=*[e]; ss, f, A) \\
  \eval{e}{(p, V)} = (b, n, \ls{fd}) \\
  \symget(S, (b, n, \ls{fd})) = v \\
  \ell = \textsc{VarOfBlock}(C, b)
}
{
  p \vdash C \step_{\symread\;(\ell,n,\ls{fd})} (S, V[x\mapsto v], ss, f, A)
} 

\\

\inferrule* [Right=$\text{Write}_3$]
{
  C = (S, V, *e_1=e_2; ss, f, A) \\
  \eval{e_1}{(p, V)} = (b, n, \ls{fd}) \\
  \eval{e_2}{(p, V)} = v \\\\
  \symset(S, (b, n, \ls{fd}), v) = S' \\
  \ell = \textsc{VarOfBlock}(C, b)
}
{
  p \vdash C \step_{\symwrite\;(\ell,n,\ls{fd})} (S', V, ss, f, A)
} 
  \end{mathpar}
\end{small}
\caption{\cstar3 Amended Configuration Reduction} \label{fig:cstar-3}
\end{figure*}

\begin{lemma}[\label{lemma-cstar-2-to-cstar-3-correct}\cstar2 to \cstar3 functional correctness]
  If $\sys(p, V, ss)$ has no unambiguous variables, then $\sys(p, V,
  ss)$, has the same behaviors in \cstar2 with event traces with
  $\symread, \symwrite$ removed, as in \cstar3 with event traces with
  $\symread, \symwrite$ removed.
\end{lemma}
\begin{proof}
  Lock-step bisimulation with equal configurations. Steps
  \textsc{Read} and \textsc{Write} need the invariant of
  Lemma~\ref{lemma-cstar-2-invar} on \cstar2 to prove that \cstar3 does not
  get stuck (ability to apply \textsc{VarOfBlock}.)
\end{proof}

\begin{lemma}[\label{lemma-cstar-2-varofblock-invert}\textsc{VarOfBlock} inversion]
  Let $C_1, C_2$ two \cstar2 configurations such that invariants of
  Lemma~\ref{lemma-cstar-2-invar} and
  Lemma~\ref{lemma-cstar-2-noninterference-invar} hold. Then, for any
  block identifiers $b_1, b_2$ such that $\textsc{VarOfBlock}(C_1,
  b_1)$ and $\textsc{VarOfBlock}(C_2, b_2)$ are both defined and
  equal, then $b_1 = b_2$.
\end{lemma}
\begin{proof}
   Assume $\textsc{VarOfBlock}(C_1, b_1) = \textsc{VarOfBlock}(C_2,
   b_2) = (f, n, x)$. When applying $\textsc{VarOfBlock}$, consider
   the frames $F_1, F_2$ holding the memory states defining $b_1,
   b_2$. Consider the variable mapping $V'_2$ in the frame just above
   $F_2$ (or in $C_2$ if such frame is missing.) Then, it is such that
   $V'_2(x) = b_2$.
  \begin{itemize}
  \item If $F_1$ and $F_2$ are at the same level in their respective
    stacks, then the variable mapping $V'_1$ in the frame directly
    above $F_1$ (or in $C_1$ if such frame is missing) is such that
    $V'_1(x) = b_1$, and also $V'_1(x) = V'_2(x)$ by the invariant, so
    $b_1 = b_2$.
  \item Otherwise, without loss of generality (by symmetry), assume
    that $F_2$ is strictly above $F_1$ (i.e. $F_2$ is strictly closer
    to the top of its own stack than $F_1$ is in its own stack.) Thus,
    in the stack of $C_1$, all frames in between $F_1$ and the frame
    $F_1''$ corresponding to $F_2$ are of the form $(M', V', \_, f',
    \_)$ with $f' = f$ and $M' \not= \None$ (otherwise the functions
    and/or recursion depths would be different.) By
    invariant~\ref{cstar-2-invar-memory-block-cons} of
    Lemma~\ref{lemma-cstar-2-invar}, it is easy to prove that $V'(x) =
    b_1$, and thus also for the variable mapping $V'_1$ in the frame
    just above $F_1''$ (or in $C_1$ directly if there is no such
    frame.) By invariants of
    Lemma~\ref{lemma-cstar-2-noninterference-invar}, we have $V'_1(x)
    = V'_2(x)$, thus $b_1 = b_2$.
  \end{itemize}
\end{proof}
  
\begin{lemma}[\label{lemma-cstar-2-to-cstar-3-noninterference}\cstar2 to \cstar3 noninterference]
  Assume that $\sys(p, V_1, ss)$ and $\sys(p, V_2, ss)$ have no
  unambiguous variables. Then, they are both safe in \cstar2 and produce
  the same traces in \cstar2, if and only if they are both safe in \cstar3 and
  produce the same traces in \cstar3.
\end{lemma}
\begin{proof}
  Use the invariants of Lemma~\ref{lemma-cstar-2-invar} and
  Lemma~\ref{lemma-cstar-2-noninterference-invar}. In fact, the
  configurations and steps are the same in \cstar2 as in \cstar3, only the
  traces differ between \cstar2 and \cstar3. \textsc{Read} and \textsc{Write}
  steps match between \cstar2 and \cstar3 thanks to
  Lemma~\ref{lemma-cstar-2-varofblock-invert}.
\end{proof}

\begin{lemma}[\cstar3 invariants]
  The invariants of Lemma~\ref{lemma-cstar-2-invar} and
  Lemma~\ref{lemma-cstar-2-noninterference-invar} also hold in \cstar3.
\end{lemma}

\subsection{Local variable hoisting} \label{sec:hoisting}

On \cstar3, hoisting can be performed, which will modify the structure of
the memory (namely the number of memory blocks allocated), which is
fine thanks to the fact that event traces carry abstract pointer
representations instead of concrete pointer values.

\paragraph{Memory allocator and dangling pointers}
However, we have to cope with dangling pointers whose address should
not be reused. Consider the following \cstar code:
\[
\begin{array}{l}
  \earray{\kint}{x}{1}{18}; \\
  \earray{\kint *}{p}{1}{x}; \\
  \{ \\
  ~ \earray{\kint}{y}{1}{42}; \\
  ~ \ewrite{p}{y}; \\
  \} \\
  \{ \\
  ~ \earray{\kint}{z}{1}{1729}; \\
  ~ \evardecl{\kint *}{q}{\eread{p}}; \\
  ~ f(q) \\
  \} \\
\end{array}
\]
With a careless memory allocator which would reuse the space of $y$
for $z$, the above program would call $f$ not with a dangling pointer
to $y$, but instead with a valid pointer to $z$, which might not be
expected by the programmer. Then, if $f$ uses its argument to access
memory, what should the \textsc{VarOfBlock} algorithm compute? I claim
that such a \cstar3 program generated from a safe \fstar program should never
try to access memory through dangling pointers.

As far as I understood, a \lowstar program obtained from a well-typed \fstar
program should be safe \emph{with any memory allocator}, including
with a memory allocator which never reuses previously allocated block
identifiers, as in CompCert.\footnote{Formally, a \lowstar (or \cstar)
  configuration should be augmented with a state $\Sigma$ so that the
  \lowstar \textsc{NewBuf} rule (or the \cstar \textsc{ArrDecl} rule), instead
  of picking a block identifier $b$ not in the domain of the memory,
  call an allocator $\kw{alloc}$ with two parameters, the domain $D$
  of the memory and the state $\Sigma$, and returning the fresh block
  $b \not\in D$ and a new state $\Sigma'$ for future
  allocations. Then, a CompCert-style allocator would, for instance,
  use $\mathbb N$ as the type of block identifiers, as well as for the
  type of $\Sigma$, so that if $\kw{alloc}(D, \Sigma) = (b, \Sigma')$,
  then it is ensured that $b \not\in D$, $\Sigma \leq b$ and $b <
  \Sigma'$. In that case, the domain of the memory being always within
  $\Sigma$, could then be easily proven as an invariant of \lowstar (or
  \cstar).} In particular, a \lowstar program safe with such a CompCert-style
allocator will actually never try to access memory through a dangling
pointer to a local variable no longer in scope.

Then, traces with concrete pointer values are preserved from \lowstar to
\cstar2 \emph{with the allocator fixed} in advance in all of \lowstar, \cstar and
\cstar2; and functional correctness and noninterference are also
propagated down to \cstar3 using the same memory allocator.

There should be a way to prove the following:
\begin{lemma}
  If a \cstar3 program is safe with a CompCert-style memory allocator,
  then it is safe with any memory allocator and the traces (with
  abstract pointer representations) are preserved by change of memory
  allocator.
\end{lemma}
\begin{proof}
  Lock-step simulation where the configurations have the same
  structure but a (functional but not necessarily injective) renaming
  of block identifiers from a CompCert-style allocator to any
  allocator is maintained and augmented throughout the execution. In
  particular, we have to prove that \textsc{VarOfBlock} is stable
  under such renaming.
\end{proof}

If so, then for the remainder of this paper, we can consider a
CompCert-style allocator.

\paragraph{Hoisting}

\begin{definition}[Hoisting]
  For any list of statements $\mathit{ss}$ with unambiguous local
  variables, the \emph{hoisting} operation $\kw{hoist}(\mathit{ss}) =
  (\mathit{ads}, \mathit{ss}')$ is so that $\mathit{ads}$ is the list
  of all array declarations in $\mathit{ss}$ (regardless of their
  enclosing code blocks) and $\mathit{ss}'$ is the list of statements
  $\mathit{ss}$ with all array declarations replaced with $()$.

  Then, hoisting the local variables in the body $ss$ of a function is
  defined as replacing $ss$ with the code block $\{ \mathit{ads};
  \mathit{ss}' \} $ where $\kw{hoist}(\mathit{ss}) = (\mathit{ads},
  \mathit{ss}')$; and then, hoisting the local variables in a program
  $p$, $\kw{hoist}(p)$, is defined as hoisting the local variables in
  each of its functions.
\end{definition}

\begin{definition}[Renaming of block identifiers] \label{def:cstar-3-rename-blocks}
  Let $C_1, C_2$ be two \cstar3 configurations. Block identifier $b_1$ is
  said to \emph{correspond} to block identifier $b_2$ from $C_1$ to
  $C_2$ if, and only if, either $\textsc{VarOfBlock}(C_1, b_1)$ is
  undefined, or $\textsc{VarOfBlock}(C_1, b_1)$ is defined and equal
  to $\textsc{VarOfBlock}(C_2, b_2)$.

  Then, value $v_1$ corresponds to $v_2$ from $C_1$ to $C_2$ if, and
  only if, either they are equal integers, or they are pointers $(b_1,
  n_1, fd_1)$, $(b_2, n_2, fd_2)$ such that $fd_1 = fd_2$, $n_1 = n_2$
  and $b_1$ corresponds to $b_2$ from $C_1$ to $C_2$, or they are
  structures with the same field identifiers and, for each field $f$,
  the value of the field $f$ in $v_1$ corresponds to the value of the
  field $f$ in $v_2$ from $C$ to $C'$.
\end{definition}

\begin{theorem}[Correctness of hoisting]
  If $\sys(p, V, ss)$ is safe in \cstar3 with a CompCert-style allocator,
  then $\sys(p, V, ss)$ and $\sys(\kw{hoist}(p), V, \kw{hoist}(ss))$
  have the same execution traces (and in particular, the latter is
  also safe) in \cstar3 using the same CompCert-style allocator.
\end{theorem}
\begin{proof}
  Forward downward simulation from \cstar3 before to \cstar3 after hoisting,
  where one step before corresponds to one step after, except at
  function entry where at least two steps are required in the compiled
  program (function entry, followed by entering the enclosing block
  that was added at function translation, then allocating all local
  variables if any), and at function exit, where two steps are
  required in the compiled program (exiting the added block before
  exiting the function.)

  Then, since \cstar3 is deterministic, the forward downward simulation is
  flipped into an upward simulation in the flavor of CompCert; thus
  preservation of traces.

  For the simulation diagram, we combine the invariants of
  Lemma~\ref{lemma-cstar-2-invar} with the following invariant between
  configurations $C = (S, V, ss, f, A)$ before hoisting and $C' = (S',
  V', ss', f', A')$ after hoisting:
  \begin{itemize}
  \item for all variables $x$ defined in $V$, $V(x)$, if defined,
    corresponds to $V'(x)$ from $C$ to $C'$
  \item $ss'$ is obtained from $ss$ by replacing all array
    declarations with $()$
  \item $f' = f$
  \item $A \subseteq A'$
  \item the set of variables declared in $ss$ is included in $A'$
  \item if a block identifier $b$ corresponds to $b'$ from $C$ to
    $C'$, then the value $\symget(S, b, n, fds)$, if defined, corresponds
    to $\symget(S', b', n, fd)$ from $C$ to $C'$
  \end{itemize}
  Each frame of the form $(\None, V_1, E, f_1, A_1)$ in $S$ is
  replaced with two frames in $S'$, namely $(\None, V_1', E, f_1,
  A_1') ; (M', V_2', \symhole, f_2, A_2')$ where:
  \begin{itemize}
  \item for all variables $x$ defined in $V_1$, $V_1(x)$ corresponds to
    $V'_1(x)$ from $C$ to $C'$
  \item all array declarations of $E$ are present in $A_1'$
  \item $E'$ is obtained from $E$ by replacing all array
    declarations with $()$
  \item $f_1' = f_1$
  \item $A_1 \subseteq A_1'$
  \item $A_2'$ contains all variable names of arrays declared in
    $f_2$, and is the block domain of $M'$
  \item $V_2'$ is defined for all variable names in $A_2'$ as a block
    identifier valid in $M'$
  \item all memory locations of arrays declared in $f_2$ are valid in
    $M'$
  \end{itemize}
  Each frame of the form $(M, V_1, E, f_1, A_1)$ in $S$ with $M \not=
  \None$ is replaced with one frame in $S'$, namely $(\{ \}, V_1', E',
  f_1', A_1')$ where:
  \begin{itemize}
  \item all blocks of $M$ are defined in $A_1'$
  \item for all variables $x$ defined in $V_1$, $V_1(x)$, if defined,
    corresponds to $V_1'(x)$ from $C$ to $C'$
  \item all array declarations of $E$ are present in $A_1'$
  \item $E'$ is obtained from $E$ by replacing all array declarations
    with $()$
  \item $f_1' = f_1$
  \item $A_1 \subseteq A_1'$
  \end{itemize}

  The fact that we are using a CompCert-style memory allocator is
  crucial here to ensure that, once a source block identifier $b$
  starts corresponding to a target one, it remains so forever, in
  particular after its block has been freed (i.e. after its
  corresponding variable has fallen out of scope), since in the latter
  case, it corresponds to any block identifier and nothing has to be
  proven then (since accessing memory through it will fail in the
  source, per the fact that the CompCert-style memory allocator will
  never reuse $b$.)
\end{proof}

\subsection{Local structures} \label{sec:local-struct}

\cstar has structures as values, unlike CompCert C and Clight, which both
need all structures to be allocated in memory. With a naive \cstar-to-C
compilation phase, where \cstar structures are compiled as C structures
and passed by value to functions, we experienced more than 60\%
slowdown with CompCert compared to GCC -O1, using the \lamstar
benchmark in Figure~\ref{fig:struct-erase-benchmark}, extracted to C as Figure~\ref{fig:struct-erase-before}. This
is because, unlike GCC, CompCert cannot detect that a structure is
never taken its address, which is mostly the case for local structures
in code generated from \cstar. This is due to the fact that, even at the
level of the semantics of C structures in CompCert, a field access is
tantamount to reading in memory through a constant offset. In other
words, CompCert has no view of C structures other than as memory
regions. To solve this issue, we replace local structures with their
individual non-compound fields, dubbed as \emph{structure erasure}. Our
benchmark after structure erasure is shown in Figure~\ref{fig:struct-erase-after}.

\begin{figure}
\begin{lstlisting}[language=fstar]
module StructErase
open FStar.Int32
open FStar.ST

type u = { left: Int32.t; right: Int32.t }

let rec f (r: u) (n: Int32.t): Stack unit (fun _ -> true) (fun _ _ _ -> true)  =
 push_frame();
 (
  if lt n 1l
  then ()
  else
   let r' : u = { left = sub r.right 1l ; right = add r.left 1l } in
   f r' (sub n 1l)
 );
 pop_frame()

let test () = 
 let r : u = { left = 18l ; right = 42l } in
 let z2 = mul 2l 2l in
 let z4 = mul z2 z2 in
 let z8 = mul z4 z4 in
 let z16 = mul z8 z8 in
 let z24 = mul z8 z16 in
 let z = mul z24 2l in
 f r z  (* without structure erasure, CompCert segfaults
           if replaced with 2*z *)
\end{lstlisting}
\caption{\lowstar benchmarking for structure erasure}
\label{fig:struct-erase-benchmark}
\end{figure}

\begin{figure}
\end{figure}

\begin{figure}
\begin{lstlisting}
typedef struct {
  int32_t left;
  int32_t right;
} StructErase_u;

void StructErase_f(StructErase_u r, int32_t n) {
  if (n < (int32_t )1) { } else {
    StructErase_u r_ = {
      .left = r.right - (int32_t )1,
      .right = r.left + (int32_t )1
    };
    StructErase_f(r_, n - (int32_t )1);
  }
}

void StructErase_test() {
  StructErase_u r = {
    .left = (int32_t )18,
    .right = (int32_t )42
  };
  int32_t z2 = (int32_t )4;
  int32_t z4 = z2 * z2;
  int32_t z8 = z4 * z4;
  int32_t z16 = z8 * z8;
  int32_t z24 = z8 * z16;
  int32_t z = z24 * z2;
  StructErase_f(r, z);
  return;
}
\end{lstlisting}
\caption{Extracted C code, before structure erasure}
\label{fig:struct-erase-before}
\end{figure}

\begin{figure}
\begin{lstlisting}
void StructErase_f(int32_t r_left, int32_t r_right, int32_t n) {
  if (n < (int32_t )1) { } else {
    int32_t r__left = r_right - (int32_t )1
    int32_t r__right = r_left + (int32_t )1;
    StructErase_f(r__left, r__right, n - (int32_t )1);
  }
}

void StructErase_test() {
  int32_t r_left = (int32_t )18;
  int32_t r_right = (int32_t )42;
  int32_t z2 = (int32_t )4;
  int32_t z4 = z2 * z2;
  int32_t z8 = z4 * z4;
  int32_t z16 = z8 * z8;
  int32_t z24 = z8 * z16;
  int32_t z = z24 * z2;
  StructErase_f(r_left, r_right, z);
  return;
}
\end{lstlisting}
\caption{Extracted C code, after structure erasure}
\label{fig:struct-erase-after}
\end{figure}

In our noninterference proofs where we prove that memory accesses are
the same between two runs with different secrets, treating all local
structures as memory accesses would become a problem, especially
whenever a field of a local structure is read as an expression (in
addition to the performance decrease using CompCert.)  This is another
reason why, in this paper (although a departure from our current
KreMLin implementation), we propose an easier proof based on the fact
that \cstar local structures should not be considered as memory regions in
the generated C code.

In addition to buffers (stack-allocated arrays), \cstar uses local
structures in three ways: as local expressions, passed as an argument
to a function by value, and returned by a function. Here we claim that
it is always possible to not take them as memory accesses, except for
structures returned by a function: in the latter case, it is necessary
for the caller to allocate some space on its own stack and pass a
pointer to it to the callee, which will use this pointer to store its
result; then, the caller will read the result back from this memory
area. Thus, we claim that, at the level of CompCert Clight, the only
additional memory accesses due to local structures are structures
returned by value.

So we extend \cstar3 with the ability for functions to have several
arguments, all of which shall be passed at each call site (there shall
be no partial applications.)

\subsubsection{Structure return} \label{sec:struct-return}
To handle structure return, we also have to account for their memory
accesses by adding corresponding events in the trace. Instead of
directly adding the memory accesses and trying to prove both program
transformation and trace transformation at the same time, we will
first add new $\symread$ and $\symwrite$ events at function return,
without those events corresponding to actual memory accesses yet;
then, in a second pass, we will actually introduce the corresponding
new stack-allocated variables.

We assume given a function $\kw{FunResVar}$ such that for any list of
statements $ss$ and any variable $x$, $\kw{FunResVar}(ss, x)$ is a
local variable that does not appear in $ss$ and is distinct from
$\kw{FunResVar}(ss, x')$ for any $x' \not= x$.

Let $p$ be a program $p$ and $ss$ be an entrypoint list of statements,
so we define $\kw{FunResVar}(f, x) = \kw{FunResVar(ss', x)}$ if $f
(\_) \{ ss' \}$ is a function defined in $p$, and
$\kw{FunResVar}(\None, x) = \kw{FunResVar}(ss,x)$.

Then, we define \cstar4 as the language \cstar3 where the $\textsf{Ret}_2$
function return rule is replaced with two rules following
Figure~\ref{fig:cstar-4}, adding the fake $\symread$ and
$\symwrite$. We do not produce any such memory access event if the
result is discarded by the caller; thus, we also need to check in the
callee whether the caller actually needs the result. To prepare for
the second pass where this check will be done by testing whether the
return value pointer argument is null, we need to account for this
test in the event trace in \cstar4 as well.

\begin{figure*}
\begin{small}
  \begin{mathpar}
\inferrule* [Right=$\text{Ret}_4\text{Some}$]
{
  \eval{e}{(p,V)}=v \\
  \kw{FunResVar}(f', x) = x' \\
  \theta = \brt ; \symwrite\;(x', 0, []);\symread\;(x',0,[])
}
{
  p \vdash (S;(\None, V', t ~ x = \symhole; ss', f', A'), V, \ereturn\;e; ss, f, A) \step_{\theta}(S, V', t ~ x = v ; ss', f', A')
}

\\

\inferrule* [Right=$\text{Ret}_4\text{None}$]
{
  \eval{e}{(p,V)}=v
}
{
  p \vdash (S;(\None, V', \symhole; ss', f', A'), V, \ereturn\;e; ss, f, A) \step_{\brf}(S, V', ss', f', A')
}
  \end{mathpar}
\end{small}
\caption{\cstar4 Amended Configuration Reduction} \label{fig:cstar-4}
\end{figure*}

\begin{theorem}[\cstar3 to \cstar4 functional correctness]
  If $\sys(p, V, ss)$ is safe in \cstar3 and has unambiguous local
  variables, then it has the same behavior and trace as in \cstar4 with
  $\brt$, $\brf$, $\symread$ and $\symwrite$ events removed.
\end{theorem}
\begin{proof}
  With all such events removed, \cstar3 and \cstar4 are actually the same
  language.
\end{proof}

\begin{lemma}[\cstar4 invariants]
  The \cstar3 invariants of Lemma~\ref{lemma-cstar-2-invar}
  and~\ref{lemma-cstar-2-noninterference-invar} also hold on \cstar4.
\end{lemma}
\begin{proof}
  This is true because the invariants of
  Lemma~\ref{lemma-cstar-2-invar} actually do not depend on the traces
  produced; and it is obvious to prove that, if two executions have
  the same traces in \cstar4, then they have the same traces in \cstar3
  (because in \cstar3, some events are just removed.)
\end{proof}

\begin{theorem}[\cstar3 to \cstar4 noninterference]
  If $\sys(p, V_1, ss)$ and $\sys(p, V_2, ss)$ are safe in \cstar3, have
  unambiguous local variables, and produce the same traces in \cstar3, then
  they also produce the same traces in \cstar4.
\end{theorem}
\begin{proof}
  Two such executions actually make the same \cstar4 steps.
\end{proof}

Then, we define the $\kw{StructRet}$ structure return transformation
from \cstar4 to \cstar3 in Figure~\ref{fig:cstar-3-struct-ret}, thus removing
all structure returns from \cstar3 programs.

\begin{figure*}
\begin{small}
\[
\kw{StructRet}(\_, \kw{return}\; e, x) = \left\{
\begin{array}{ll}
  \eif{x}{ *[x] = e}{()} ; \kw{return}\;() & \text{if} ~ x \not= \None \\
  \kw{return}\; e & \text{otherwise}
\end{array}
\right.
\]

\[
\kw{StructRet}(f', t ~ x = f(e), \_) = \left\{
\begin{array}{ll}
  t ~ x'[1]; f(x', e); t ~ x = *[x'] & \text{if} ~ t ~ \text{is a} ~ \kw{struct} \\
  & \text{and} ~ x' = \kw{FunRetVar}(f', x) \\
  t ~ x = f(e) & \text{otherwise}
\end{array}
\right.
\]

\[
\kw{StructRet}(\_, f(e), \_) = \left\{
\begin{array}{ll}
  f(0, e) & \text{if the return type of } ~ f ~ \text{is a} ~ \kw{struct} \\
  f(e) & \text{otherwise}
\end{array}
\right.
\]

\newcommand{\ecfuntwoargs}[7]                {%
  \ensuremath{\kw{fun}\;#1\,(#2:#3,#4:#5):#6\,\{\;#7\}}%
}

\[
\kw{StructRet}(\ecfun{f}{x}{t}{t'}{ss}) = \left\{
\begin{array}{ll}
  \ecfuntwoargs{f}{r}{t'*}{x}{t}{\kw{unit}}{\{ \kw{StructRet}(f, ss, r) \} } &
  \text{if} ~ t' ~ \text{is a} ~ \kw{struct} \\
  & (r ~ \text{fresh}) \\
  \ecfun{f}{x}{t}{t'}{\{ \kw{StructRet}(f, ss, \None) \} } & \text{otherwise}
\end{array}
\right.
\]
\end{small}
\caption{\cstar4 to \cstar3 structure return
  transformation} \label{fig:cstar-3-struct-ret}
\end{figure*}

Then, the transformation back to \cstar3 exactly preserves the traces of
\cstar4 programs, so that we obtain both functional correctness and
noninterference at once:

\begin{theorem}[$\kw{StructRet}$ correctness]
  If $\sys(p, V, ss)$ is safe in \cstar4 and has unambiguous local
  variables, then it has the same behavior and trace as
  $\sys(\kw{StructRet}(p), V, \kw{StructRet}(ss, \None))$.
\end{theorem}
\begin{proof}
  Forward downward simulation, where the compilation invariant also
  involves block identifier renaming from
  Definition~\ref{def:cstar-3-rename-blocks} due to the new local
  arrays introduced by the transformed program.

  Each \cstar4 step is actually matched by the same \cstar3 step, except for
  function return and return from block: for the $\textsc{RetBlock}$
  rule, the simulation diagram has to stutter as many times as the
  level of block nesting in the source program before the actual
  application of a $\textsc{Ret}_4$ rule. Then, when the
  $\textsc{RetSome}_4$ rule applies, the trace events are produced by
  the transformed program, the $\brt$ and the $\symwrite$ events from
  within the callee, then the callee blocks are exited, and finally
  the $\symread$ event is produced from within the caller.

  Then, the diagram is turned into bisimulation since \cstar3 is
  deterministic.
\end{proof}

After a further hoisting pass, we can now restrict our study to those
\cstar3 programs with unambiguous local variables, functions with multiple
arguments, function-scoped local arrays, and no functions returning
structures.

\subsubsection{Events for accessing structure buffers} \label{sec:struct-events}

Now, we transform an access to one structure into the
sequence of accesses to all of its individual atomic (non-structure)
fields.

\ignore{
\textbf{NOTE:} we need to have some typing information added to the
$\symread$, $\symwrite$ events of \cstar and each of \cstar$n$: whenever we
read or write from a buffer, we need to mark the type of the data
read/written. This makes sense in the context of observing memory
accesses, since without this data, two programs accessing the same
memory location but reading data of different types, thus of
potentially different lengths, would still be considered as having the
same trace, which we would like to rule out.  So, in the remaining
parts of this document, we assume that $\symread$ and $\symwrite$
events carry the type of the data read or written.
}

Consider the following transformation for \cstar3 $\symread$ (and
similarly for $\symwrite$) events:

\[
\begin{array}{ll}
  \multicolumn{2}{l}{\llbracket \symread\;(f, i, x, j, \ls{fd}, t) \rrbracket} \\
  = & \llbracket \symread\; (f, i, x, j, \ls{fd};fd_1, t_1) \rrbracket \\
  ; & \dots \\
  ; & \llbracket \symread\; (f, i, x, j, \ls{fd};fd_n, t_n) \rrbracket \\
  \text{if} & t = \kw{struct} \{ fd_1: t_1, \dots, fd_n: t_n \} \\
\\
  \multicolumn{2}{l}{\llbracket \symread\;(f, i, x, j, \ls{fd}, t) \rrbracket} \\
  = & \symread\; (f, i, x, j, \ls{fd}, t) \\
  \multicolumn{2}{l}{\text{otherwise}}
\end{array}
\]

Then, let \cstar5 be the \cstar3 language obtained by replacing each
$\symread$, $\symwrite$ event with its translation. Then, it is easy
to show the following:
\begin{lemma}[\cstar3 to \cstar5 correctness]
  Let $p$ be a \cstar3 program. Then, $p$ has a trace $t$ in \cstar5 if, and
  only if, there exists a trace $t'$ such that $p$ has trace $t'$ in
  \cstar3 and $\llbracket t' \rrbracket = t$.
\end{lemma}

Thus, this trace transformation preserves functional correctness; and,
although this trace transformation is not necessarily injective (since
it is not possible to disambiguate between an access to a 1-field
structure and an access to its unique field), noninterference is also
preserved.

After such transformation, all $\symread$ and $\symwrite$ events now
are restricted to atomic (non-structure) types.

\subsubsection{Local structures} \label{sec:local-struct-detail}

Now, we are removing all local structures, in such a way that the only
remaining structures are those of local arrays, and all structures are
accessed only through their atomic fields. In particular, we are
replacing every local (non-array) variable $x$ of type struct with the
sequence of variable names $x\_\ls{fds}$ for all field name sequences
$\ls{fds}$ valid from $x$ such that $x.\ls{fds}$ is of
non-$\kw{struct}$ type. (We omit the details as to how to construct
names of the form $x\_\ls{fds}$ so that they do not clash with other
variables; at worst, we could also rename other variables to avoid
clashes as needed.)

Using our benchmark in Figure~\ref{fig:struct-erase-benchmark}, with C
code after structure erasure in Figure~\ref{fig:struct-erase-after},
on a 4-core Intel Core i7 1.7 GHz laptop with 8 Gb RAM, structure
erasure saves 20\% time with CompCert 2.7.

If we assume that we know about the type of a \cstar expression, then it
can be first statically reduced to a normal form as in
Figure~\ref{fig:cstar-expr-struct-erase}.

\begin{figure*}
\begin{scriptsize}
  \begin{mathpar}
\inferrule*
    [Right=Int]
    {~}
    {
      \Gamma \vdash n \downarrow^{\kw{Int}} n
    }

\inferrule*
    [Right=Var]
    {
      (x : t) \in \Gamma
    }
    {
      \Gamma \vdash x \downarrow^t x
    }

\inferrule*
    [Right=PtrAdd]
    {
      \Gamma \vdash e_1 \downarrow^{t*} e_1' \\
      \Gamma \vdash e_2 \downarrow^{\kw{int}} e_2'
    }
    {
      \Gamma \vdash e_1 + e_2 \downarrow^{t*} e_1' + e_2'
    }

\\

\inferrule*
    [Right=PtrFd]
    {
      \Gamma \vdash e \downarrow^{\kw{struct} \{ fd : t; \dots \} *} e' \\
    }
    {
      \Gamma \vdash \eptrfd{e}{fd} \downarrow^{t*} \eptrfd{e'}{fd}
    }
    
\inferrule*
    [Right=StructFieldName]
    {
      \Gamma \vdash e \downarrow^{\kw{struct} \{ fd : t ; \dots \}} x.\ls{fds} \\
      t ~ \text{is a} ~ \kw{struct}
    }
    {
      \Gamma \vdash e.fd \downarrow^t x.\ls{fds}.fd
    }

\\

\inferrule*
    [Right=ScalarFieldName]
    {
      \Gamma \vdash e \downarrow^{\kw{struct} \{ fd : t ; \dots \}} x.\ls{fds} \\
      t ~ \text{is not a} ~ \kw{struct}
    }
    {
      \Gamma \vdash e.fd \downarrow^t x\_\ls{fds}\_fd
    }

    \\

\inferrule*
    [Right=FieldProj]
    {
      \Gamma \vdash e \downarrow^{\kw{struct} \{ fd : t ; \dots \}} \{ f = e' ; \dots \}
    }
    {
      \Gamma \vdash e.fd \downarrow^t e'
    }

\inferrule*
    [Right=Struct]
    {
      \ls{\Gamma \vdash e_i \downarrow^{t_i} e'_i}
    }
    {
      \Gamma \vdash \{ \ls{fd_i = e_i} \} \downarrow^{\kw{struct} \{ \ls{fd_i : t_i} \} } \{ \ls{fd_i = e'_i} \}
    }
\end{mathpar}
\end{scriptsize}
  \caption{\cstar Structure Erasure: Expressions}
  \label{fig:cstar-expr-struct-erase}
\end{figure*}

\begin{lemma}[\label{lemma-cstar-expr-struct-erase-correct} \cstar structure erasure in expressions: correctness]
  For any value $v$ of type $t$, if $\llbracket e \rrbracket_{(p,V)} =
  v$ and $\Gamma \vdash e \downarrow^t e'$ and $V'$ is such that:
  \begin{itemize}
  \item for any $(x': t') \in \Gamma$,  $V(x')$ exists, is of type $t'$ and is equal to $V(x)$
  \item for any $(x': t') \in \Gamma$ that is a struct and for any $\ls{fds}$ such that $x.\ls{fds}$ is not a struct, then $V'(x\_\ls{fds}) = V(x)(\ls{fds})$
  \end{itemize}
  Then, $\llbracket e' \rrbracket_{(p,V')} = v$
\end{lemma}
\begin{proof}
  By structural induction on $\downarrow$.
\end{proof}

\begin{definition}[\cstar expression without structures]
  A \cstar expression $e$ is said to be of type $t$ without structures if, and only
  if, one of the following is true:
  \begin{itemize}
  \item $e$ contains neither a structure field projection nor a structure expression
  \item $e$ is of the form $x.\ls{fds}$ where $x$ is a variable such
    that $x.\ls{fds}$ is of $\kw{struct}$ type
  \item t is of the form $\{ \ls{fd_i : t_i} \}$ and $e$ is of the form $\{ \ls{fd_i = e_i} \}$ where for each $i$, $e_i$ is of type $t_i$ without structures
  \end{itemize}
\end{definition}

\begin{lemma}[\label{lemma-cstar-expr-struct-erase-shape}\cstar expression reduction: shape]
  If $\Gamma \vdash e \downarrow^t e'$, then $e'$ is of type $t$
  without structures.
\end{lemma}
\begin{proof}
  By structural induction on $\downarrow$.
\end{proof}

Then, once structure expressions are reduced within an expression
computing a non-structure value, we can show that evaluating such a
reduced expression no longer depends on any local structures:

\begin{lemma} \label{lemma-cstar-expr-struct-erase-non-struct}
  If $\llbracket e \rrbracket_{(p,V)} = v$ for some value $v$, and $e$
  is of type $t$ without structures, and $t$ is not a $\kw{struct}$
  type, then, for any variable mapping $V'$ such that $V'(x) = V(x)$
  for all variables $x$ of non-$\kw{struct}$ types, $\llbracket e
  \rrbracket_{(p,V')} = v$.
\end{lemma}
\begin{proof}
  By structural induction on $\llbracket e \rrbracket_{(p,V)}$.
\end{proof}

Now, we take advantage of this transformation to transform \cstar5
statements into \cstar3 statements without structure assignments.  This
$\downarrow$ translation is detailed in
Figure~\ref{fig:cstar-5-to-cstar-3}

In particular, each function parameter of structure type passed by
value is replaced with its recursive list of all non-structure
fields.\footnote{Our solution, although semantics-preserving as we
  show further down, yet causes ABI compliance issues. Indeed, in the
  System V x86 ABI, structures passed by value must be replaced not
  with their fields, but with their sequence of bytes, some of which
  may correspond to padding related to no field of the original
  structure. CompCert does support this feature but as an
  \textbf{unverified} elaboration pass over source C code. So, we
  should investigate whether we really need to expose functions taking
  structures passed by value at the interface level.}

\begin{figure*}
\begin{footnotesize}
  \begin{mathpar}
    \inferrule*
    [Right=ReadScalar]
    {
      t ~ \text{not a} ~ \kw{struct} \\
      \Gamma \vdash e \downarrow^{t*} e'
    }
    {
      p,\Gamma \vdash t ~ x = \eread{e} \downarrow t ~ x = \eread{e'} \\\\
      \Gamma[x] \leftarrow t
    }

    \quad
    \quad
    
    \inferrule*
    [Right=WriteScalar]
    {
      t ~ \text{not a} ~ \kw{struct} \\\\
      \Gamma \vdash e_1 \downarrow^{t*} e'_1 \\
      \Gamma \vdash e_2 \downarrow^{t} e'_2 \\
    }
    {
      p,\Gamma \vdash \ewrite{e_1}{e_2} \downarrow \ewrite{e'_1}{e'_2}
    }

    \\    
    
\inferrule*
    [Right=ReadStruct]
    {
      t = \kw{struct} \{ \ls{\mathit{fd_i} : t_i} \} \\
      \Gamma \vdash e \downarrow^{t*} e' \\
      (x, \_) \not\in \Gamma \\
      (x', \_) \not\in \Gamma \\\\
      \ls{p,\Gamma \cup (x' : t*) \cup (x\_{\mathit{fd}_i} : t_i) \vdash x\_{\mathit{fd}_i} = \eread{\eptrfd{x'}{\mathit{fd}_i}} \downarrow \mathit{ss}_i}
    }
    {
      p,\Gamma \vdash t ~ x = \eread{e} \downarrow t* ~ x' = e'; \ls{\mathit{ss}_i} \\
      \Gamma[x] \leftarrow t
    }

    \\    
    
\inferrule*
    [Right=WriteStruct]
    {
      t = \kw{struct} \{ \ls{\mathit{fd_i} : t_i} \} \\
      \Gamma \vdash e_1 \downarrow^{t*} e'_1 \\
      \Gamma \vdash e_2 \downarrow^{t} e'_2 \\
      (x', \_) \not\in \Gamma \\\\
      \ls{p,\Gamma \cup (x' : t*) \vdash \ewrite{\eptrfd{x'}\mathit{fd}_i}{e'_2.\mathit{fd}_i} \downarrow \mathit{ss}_i}
    }
    {
      p,\Gamma \vdash \ewrite{e_1}{e_2} \downarrow t* ~ x' = e'_1 ; \ls{\mathit{ss}_i}
    }

    \\

    \inferrule*
    [Right=Ret]
    {
      t ~ \text{not a} ~ \kw{struct} \\
      \Gamma \vdash e \downarrow^{t} e'
    }
    {
      p,\Gamma \vdash \kw{return} ~ e \downarrow \kw{return} ~ e'
    }

\inferrule* [Right=Block]
{
  p,\Gamma \vdash ss \downarrow ss'
}
{
  p,\Gamma \vdash \{ ss \} \downarrow \{ ss' \}
}

    \inferrule* [Right=Call]
    {
      p(f)= \kw{fun} (\ls{\_ : t_i}) : t \{ \_ \} \\
      t ~ \text{not a} ~ \kw{struct} \\
      p,\Gamma \vdash (\ls{t_i}, \mathit{el}) \downarrow \mathit{el}'
    }
    {
      p,\Gamma \vdash t\;x=f\;(\mathit{el}) \downarrow t\;x = f(\mathit{el}') \\
      \Gamma[x] \leftarrow t
    } 

    \\
    
    \inferrule*
    [Right=ArgNil]
    {~}
    {
      p,\Gamma \vdash ([],[]) \downarrow []
    }

    \inferrule*
        [Right=ArgCons]
        {
          p,\Gamma \vdash (t, e) \downarrow \mathit{el}_1 \\
          p,\Gamma \vdash (\mathit{tl}, \mathit{el}) \downarrow \mathit{el}_2
        }
        {
          p,\Gamma \vdash (t; \mathit{tl}, e; \mathit{el}) \downarrow \mathit{el}_1; \mathit{el}_2
        }

        \\
        
    \inferrule*
        [Right=ScalarArg]
        {
          t ~ \text{not a} ~ \kw{struct} \\
          \Gamma \vdash e \downarrow^t e'
        }
        {
          p, \Gamma \vdash (t, e) \downarrow [e']
        }

    \inferrule*
        [Right=StructArg]
        {
          t = \kw{struct} \{ \ls{\mathit{fd}_i : t_i} \} \\
          \Gamma \vdash e \downarrow^t e' \\\\
          \ls{p,\Gamma \vdash (t_i, e'.\mathit{fd}_i) \downarrow \mathit{el}_i}
        }
        {
          p, \Gamma \vdash (t, e) \downarrow \ls{\mathit{el}_i}
        }

    \\


\inferrule* [Right=If]
{
  t ~ \text{not a} ~ \kw{struct} \\
  \Gamma \vdash e \downarrow^t e' \\
  \ls{p,\Gamma \vdash ss_i \downarrow ss'_i}
}
{
  p,\Gamma \vdash \eif{e}{ss_1}{ss_2} \downarrow \eif{e'}{ss_1'}{ss_2'}
} 

\\    
        
    \inferrule*
    [Right=ScalarParam]
    {
      t ~ \text{not a} ~ \kw{struct} \\
    }
    {
      (x : t) \downarrow (x : t)
    }

    \inferrule*
    [Right=StructParam]
    {
      t = \kw{struct} \{ \ls{\mathit{fd}_i : t_i} \} \\\\
      \ls{ (x\_\mathit{fd}_i : t_i) \downarrow \mathit{vt}_i }
    }
    {
      (x : t) \downarrow \ls{\mathit{vt}_i}
    }
    \\

    
    \inferrule* [Right=Fun]
    {
      \mathit{vt} \downarrow \mathit{vt}' \\
      p, (\mathit{vt} \cup \ls{x_i : t_i*}) \vdash \mathit{ss} \downarrow \mathit{ss}'
    }
    {
      \kw{fun} (\mathit{vt}) : t \{ \ls{t_i ~ x_i[\_] = \{ \} } ; \mathit{ss} \}
      \downarrow
      \kw{fun} (\mathit{vt}') : t \{ \ls{t_i ~ x_i[\_] = \{ \} } ; \mathit{ss}' \}
    }
\end{mathpar}
\end{footnotesize}
  \caption{\cstar5 to \cstar3 Structure Erasure: Statements}
  \label{fig:cstar-5-to-cstar-3}
\end{figure*}

\begin{theorem}[\cstar5 to \cstar3 structure erasure: shape]
  If $p \downarrow p'$ following Figure~\ref{fig:cstar-5-to-cstar-3},
  then $p'$ no longer has any variables of local structure type, and no
  longer has any structure or field projection expressions.
\end{theorem}
\begin{proof}
  By structural induction over $\downarrow$, also using
  Lemma~\ref{lemma-cstar-expr-struct-erase-shape}.
\end{proof}

\begin{theorem}[\cstar5 to \cstar3 structure erasure: correctness]
  If $p$ is a \cstar5 program (that is, syntactically, a \cstar program with
  unambiguous local variables, no block-scoped local arrays other than
  function-scoped, and no functions returning structures) such that
  $p$ is safe in \cstar5 and $p \downarrow p'$ following
  Figure~\ref{fig:cstar-5-to-cstar-3}, then $p$ and $p'$ have the same
  execution traces.
\end{theorem}
\begin{proof}
  Forward downward simulation where one \cstar5 step triggers one or
  several \cstar3 steps. Then, determinism of \cstar3 turns this forward
  downward simulation into bisimulation.

  The compilation invariant is as follows: the code fragments are
  translated using $\downarrow$, and variable maps $V$ in source \cstar5
  vs. their compiled \cstar3 counterparts $V'$ follow the conditions of
  Lemma~\ref{lemma-cstar-expr-struct-erase-correct}, also using
  Lemma~\ref{lemma-cstar-expr-struct-erase-non-struct}. Memory states
  $M$ are exactly preserved, as well as the structure of the stack.
\end{proof}

Then, after a further $\alpha$-renaming pass, we obtain
a \cstar3 program that no longer has any local (non-stack-allocated)
structures at all, and where all memory accesses are of non-structure
type. The shape of this program is now suitable for translation to a
CompCert Clight program in a straightforward way, which we describe
in the next subsection.

\subsection{Generation of CompCert Clight code} \label{sec:clight-gen}

Recall that going from \cstar3 (with abstract pointer events) back to \cstar2
(with concrete pointer events) is possible thanks to the fact that
Lemma~\ref{lemma-cstar-2-to-cstar-3-correct} and
Lemma~\ref{lemma-cstar-2-to-cstar-3-noninterference} are actually
equivalences.

Recall that a \cstar$n$ transition system is of the form $\sys(p, V,
\mathit{ss})$ where $p$ is a list of functions\footnote{and global
  variables, although the semantics of \cstar says nothing about how to
  actually initially allocate them in memory}, $\mathit{ss}$ is a list
of \cstar statements with undeclared local variables, the values of which
shall be taken from the map $V$. $\mathit{ss}$ is actually taken as
the entrypoint of the program, and $V$ is deemed to store the initial
values of secrets, ensuring that $p$ and $\mathit{ss}$ are
syntactically secret-independent.

In CompCert Clight, it is not nominally possible to start with a set
of undeclared variables and a map to define them. So, when translating
the \cstar entrypoint into Clight, we have to introduce a
\emph{secret-independent} way of representing $V$ and how they are
read in the entrypoint. Fortunately, CompCert introduces the notion of
\emph{built-in functions}, which are special constructs whose
semantics can be customized and that are guaranteed to be preserved by
compilation down to the assembly.

Thus, we can populate the values of local non-stack-allocated
variables of a \cstar entrypoint by uniformly calling builtins in Clight,
and only the semantics of those builtins will depend on secrets, so
that the actually generated Clight code will be syntactically
secret-independent.

Translating \cstar2 expressions with no structures or structure field
projections into CompCert Clight is straightforward, as shown in
Figure~\ref{fig:cstar-2-to-clight-expr}. For any \cstar2 expression $e$ of
type $t$, assuming that $A$ is a set of local variables to be
considered as local arrays, we define $\mathbb C^t_A(e)$ to be the
compiled Clight expression corresponding to $e$.

\begin{figure}
  \[
  \begin{array}{lcl}
    \mathbb C^{\kw{int}}_A (n) & = & n \\
    \mathbb C^{\kw{unit}}_A (()) & = & 0 \\
    \mathbb C^t_A (x) & = & \&x \\
     & \text{if} & x \in A \\
    \mathbb C^t_A (x) & = & \_x \\
     & \text{if} & x \not\in A \\
    \mathbb C^{t*}_A (e_1 + e_2) & = & \mathbb C^{t*}_A (e_1) +_t \mathbb C^{\kw{int}}_A (e_2) \\
    \mathbb C^{t*}_A (\eptrfd{e}{fd}) & = & \&(*\mathbb C^{t'*}_A (e) ._{t'}\mathit{fd}) \\
    & \text{if} & t' = \kw{struct}\{ \mathit{fd} : t, \dots \}
  \end{array}
  \]
  \caption{\cstar2 to Clight: Expressions}
  \label{fig:cstar-2-to-clight-expr}
\end{figure}

\begin{lemma}
  Let $V$ be a \cstar2 local variable map, and $A$ be a set of local
  variables to be considered as local arrays. Assume that, for all $x
  \in A$, there exists a block identifier $b$ such that $V(x) = (b,
  0)$, and define $V'(x)$ be such block identifier $b$. Then, define
  $\_V'(\_x) = V(x)$ for all $x \not\in A$.

  Then, for any expression $e$ with no structures or structure
  projections, $\kw{rv}(\mathbb C^t_A (e), (p, V', \_V')) = \llbracket
  e \rrbracket_{(p, V)}$.
\end{lemma}
\begin{proof}
  By structural induction on $e$. \ignore{\textbf{TODO:} global variables.}
\end{proof}

Translating \cstar2 statements with no local array declarations, no read
or write of structure type and no functions returning structures into
Clight is straightforward as well, as shown in
Figure~\ref{fig:cstar-2-to-clight-stmt}.

\begin{figure}
  \[
  \arraycolsep=1pt
  \begin{array}{lcl}
    \mathbb C_A (t ~ x = e) & = & t ~ x = C^t_A(e) \\
    \mathbb C_A (t ~ x = f(\ls{e_i})) & = & t ~ x = f(\ls{C^{t_i}_A(e_i)}) \\
    & & \text{if} ~ f ~ \text{is} ~ \kw{fun}(\ls{(\_ : t_i)}) : \_ \{ \_ \} \\
    \mathbb C_A (t ~ x = \eread{e}) & = & \clannot(\symread,t,e) ; t ~ x = [*\mathbb C^{t*}_A (e) ] \\
    \mathbb C_A (\ewrite{e_1}{e_2}) & = & \clannot(\symwrite,t,e) ; *\mathbb C^{t*}_A(e_1) = \mathbb C^{t}_A(e_2) \\
    \mathbb C_A (\eifthenelse{e}{ss_1}{ss_2}) & = & \eifthenelse{\mathbb C^t_A (e) }{\mathbb C_A(ss_1)}{\mathbb C_A(ss_2)} \\
    & & \text{for some} ~ t ~ \text{not} ~ \kw{struct} \\
    \mathbb C_A ( \{ ss \} ) & = & \{ \mathbb C_A (ss) \} \\
    \mathbb C_A ( \kw{return} ~ e ) & = & \kw{return} ~ \mathbb C^t_A(e) \\
  \end{array}
  \]
  \caption{\cstar2 to Clight: Statements}
  \label{fig:cstar-2-to-clight-stmt}
\end{figure}

Let $p$ be a \cstar2 program, and $ss$ be a \cstar2 entrypoint sequence of
statements. Assume that $p$ and $ss$ have unambiguous local variables,
no functions returning structures, no local arrays other than
function-scoped arrays, and no local structures other than local
arrays. Further assume that $p$ has no function called $\kw{main}$,
and no function with the same name as a built-in function. Then, we
can define the compiled CompCert Clight program $\mathbb C(p, ss)$ as
in Figure~\ref{fig:cstar-2-to-clight-prog}.
\begin{figure}
\[
\begin{array}{lcl}
  \mathbb C(p, ss)(f) & = & \kw{fun} ~ (\ls{x : t}) : t' \{ \ls{t_i ~ x_i[n_i]} ; \mathbb C_{\ls{x_i}}(ss') \} \\
  & \text{if} & p(f) = \kw{fun} ~ (\ls{x : t}) : t' \{ \ls{t_i ~ x_i[n_i]} ; ss' \} \\
  \mathbb C(p, ss)(\kw{main}) & = & \kw{fun} () : \kw{int} \{ \\
  & & ~ \ls{t_i ~ x_i[n_i]} ; \\
  & & ~ \ls{\_y_i = \kw{get}\_y_i()} ; \\
  & & ~ \mathbb C_{\ls{x_i}}(ss') \\
  & & \} \\
  & \text{if} & \mathit{ss} = \{ \ls{t_i ~ x_i[n_i]} ; ss' \} \\
  & & \text{with free variables} ~ \ls{y_i}
\end{array}
\]
\caption{\cstar2 to Clight: Program and Entrypoint}
\label{fig:cstar-2-to-clight-prog}
\end{figure}

\begin{theorem}[\cstar2 to Clight: correctness]
  If $\sys(p, V, ss)$ is safe in \cstar2, then it has the same execution
  trace as $\mathbb C(p, ss)$ in Clight, when the semantics of the
  built-in functions $\kw{get}\_x$ are given by $V$.
\end{theorem}
\begin{proof}
  Forward downward simulation where one \cstar2 step corresponds to one or
  several Clight steps. Then, since Clight is deterministic, the
  forward downward simulation diagram is turned into a bisimulation.

  The structure of the Clight stack is the same as in the \cstar2 stack,
  and the values of variables are the same, as well as the memory
  block identifiers. The only change is in the Clight representation
  of \cstar2 structure values . A \cstar2 memory state $M$ is said to
  correspond to a Clight memory state $M'$ if, for any block
  identifier $b$, for any array index $i$, and for any field sequence
  $\mathit{fds}$ leading to a non-structure value, $M'(b, n +
  \kw{offsetof}(\mathit{fds})) = M(b, n)(\mathit{fds})$.
\end{proof}

Thus, since both \cstar2 and Clight records all memory accesses in their
traces, this theorem entails both functional correctness and
preservation of noninterference.


\newpage
\section{Proof of the secret independence theorem}

\begin{small}
\verbatiminput{abstraction.txt}
\end{small}

%% \appendix
%% \section{Appendix Title}

%% This is the text of the appendix, if you need one.

%% \acks

%% Acknowledgments, if needed.

% We recommend abbrvnat bibliography style.

%% \begin{thebibliography}{}
%% \softraggedright

%% \bibitem[Smith et~al.(2009)Smith, Jones]{smith02}
%% P. Q. Smith, and X. Y. Jones. ...reference text...

%% \end{thebibliography}

\newpage
\fi%long

%%%%%%%%%%%%%%%%%%%%%%%%%%%%%%%%%%%%%%%%%%%%%%%%%%%%%%%%%%%%%%%%%%%%%%%%%%%%%%%%

% The bibliography should be embedded for final submission.
% taramana 2016-10-07: Until then, let us use the following:
\bibliography{biblio,fstar,tls}

\iflong
\else
\balance
\fi

\end{document}
