% Colors
\newcommand\maybecolor[1]{\color{#1}}
\definecolor{dkblue}{rgb}{0,0.1,0.5}
\definecolor{dkgreen}{rgb}{0,0.4,0}
\definecolor{dkred}{rgb}{0.6,0,0}
\definecolor{dkpurple}{rgb}{0.7,0,1.0}
\definecolor{olive}{rgb}{0.4, 0.4, 0.0}
\definecolor{teal}{rgb}{0.0,0.4,0.4}
\definecolor{azure}{rgb}{0.0, 0.5, 1.0}
\definecolor{butter3}{HTML}{C4A000}
\definecolor{lightblue}{rgb}{0.2,0.2,1.0}
\definecolor{lightgrey}{rgb}{0.8,0.8,0.8}
\definecolor{linkColor}{rgb}{0,0,0.5}
\definecolor{lightgray}{rgb}{.9,.9,.9}
\definecolor{darkgray}{rgb}{.4,.4,.4}
\definecolor{purple}{rgb}{0.65, 0.12, 0.82}

% Comments

\renewcommand{\comment}[1]{{\ifdraft\color{red}#1\fi}}
\newcommand{\comm}[3]{\ifdraft{\color{#1}[#2: #3]}\fi}
\newcommand{\cf}[1]{\comm{olive}{Cédric}{#1}}
\newcommand{\nik}[1]{\comm{dkpurple}{Nik}{#1}}
\newcommand{\ch}[1]{\comm{teal}{CH}{#1}}
\newcommand{\aseem}[1]{\comm{magenta}{Aseem}{#1}}
\newcommand{\jp}[1]{\comm{butter3}{JP}{#1}}
\newcommand{\kb}[1]{\comm{blue}{KB}{#1}}
\newcommand{\tr}[1]{\comm{cyan}{TR}{#1}}
\newcommand{\ignore}[1]{}

% Languages

\newcommand{\fstar}{\relax\ifmmode{F^\ast}\else{F$^\ast$}\fi\xspace}
\newcommand{\lowstar}{\relax\ifmmode{Low^\ast}\else{Low$^\ast\!$}\fi\xspace}
\newcommand{\lamstar}{\texorpdfstring{\ensuremath{\lambda}ow\ensuremath{^\ast}\xspace}{low*}}
\newcommand{\cstar}{\relax\ifmmode{C^\ast}\else{C$^\ast$}\fi\xspace}
\newcommand{\haclstar}{HACL$^\ast$\xspace}
\newcommand{\kremlin}{KreMLin\xspace}
\newcommand{\emfhst}{EMF$^\ast_{\text{HST}}$\xspace}
\newcommand{\emfst}{EMF$^\ast_{\text{ST}}$\xspace}

\newcommand{\cL}{{\cal L}}
\newcommand{\fref}[1]{Figure~\ref{fig:#1}}
\newcommand{\sref}[1]{§\ref{sec:#1}}
\newcommand{\lref}[1]{line~\ref{line:#1}}
\newcommand{\sep}{\ensuremath{\ |\ }}

\newcommand{\ufstar}{$\mu$\fstar}
\newcommand{\kw}[1]               {\text{\textsf{#1}}}
\newcommand{\kword}{\kw{word}}
\newcommand{\klayout}{\kw{layout}}
\newcommand{\kint}{\kw{int}}
\newcommand{\kskip}{\kw{skip}}
\newcommand{\ksizeof}{\kw{sizeOf}}
\newcommand{\etlet}[2]            {\ensuremath{\kw{let}\ #1=#2}}
\newcommand{\elet}[3]             {\ensuremath{\kw{let}\ #1=#2\ \kw{in}\ #3}}
\newcommand{\efun}[3]             {\ensuremath{\kw{fun}\ (#1:#2) \to #3}}
\newcommand{\eifthenelse}[3]      {\ensuremath{\kw{if}\ #1\ \kw{then}\ #2\ \kw{else}\ #3}}
\newcommand{\eif}[3] {\kw{if}\;#1\;\kw{then}\;#2\;\kw{else}\;#3}
\newcommand{\esequence}[2]        {\ensuremath{#1;\ #2}}
\newcommand{\eassign}[2]          {\ensuremath{#1 \leftarrow #2}}
\newcommand{\eapply}[2]           {\ensuremath{#1(#2)}}
\newcommand{\esizeof}[1]          {\eapply\ksizeof{#1}}
\newcommand{\efst}[1]             {\eapply{\kw{fst}}{#1}}
\newcommand{\esnd}[1]             {\eapply{\kw{snd}}{#1}}
\newcommand{\ewhile}[2]           {\ensuremath{\kw{while }(#1)\ \kw{do}\ #2}}
\newcommand{\enewbuf}[2]          {\ensuremath{\kw{newbuf}\ #1\ #2}}
\newcommand{\ereadbuf}[2]         {\ensuremath{\kw{readbuf}\ #1\ #2}}
\newcommand{\ewritebuf}[3]        {\ensuremath{\kw{writebuf}\ #1\ #2\ #3}}
\newcommand{\esubbuf}[3]          {\ensuremath{\kw{subbuf}\ #1\ #2\ #3}}
\newcommand{\enewstruct}[1]       {\ensuremath{\kw{newstruct}\ #1}}
\newcommand{\ereadstruct}[1]         {\ensuremath{\kw{readstruct}\ #1}}
\newcommand{\ewritestruct}[2]        {\ensuremath{\kw{writestruct}\ #1\ #2}}
\newcommand{\structfield}[0]          {\triangleright}
\newcommand{\estructfield}[2]          {\ensuremath{{#1} \structfield {#2}}}

\newcommand{\ecfun}[5]                {%
  \ensuremath{\kw{fun}\;#1\,(#2:#3):#4\,\{\;#5\;\}}%
}
\newcommand{\ecfuntwo}[4]                {%
  \ensuremath{\kw{fun}\;(#1:#2):#3\,\{\;#4\;\}}%
}
\newcommand{\ecifthenelse}[3]     {\ensuremath{#1\ \kw{?}\ #2\ \kw{:}\ #3}}
\newcommand{\evardecl}[3]            {\ensuremath{#1\ #2 = #3}}
\newcommand{\swhile}[4]           {\ensuremath{%
  \kw{while}\ (#1)\ \kw{inv}\ #2\,\{_{#3}\ #4\ \}}%
}
\newcommand{\earray}[3]           {\ensuremath{#1\ #2[#3]}}
\newcommand{\memset}[3]           {\kw{memset}\;#1\;#2\;#3}
\newcommand{\earrayget}[2]           {\ensuremath{#1[#2]}}
\newcommand{\earrayset}[3]         {\ensuremath{#1[#2] = #3}}
\newcommand{\eread}[1]         {\ensuremath{*[#1]}}
\newcommand{\ewrite}[2]         {\ensuremath{*[#1] = #2}}
\newcommand{\ereturn}[1]         {\ensuremath{\kw{return}\ #1}}
\newcommand{\symread}{\kw{read}}
\newcommand{\symwrite}{\kw{write}}

\newcommand{\stmts}{ss}
\newcommand{\ls}[1]{\overrightharpoon{#1}}
\newcommand{\eptrfd}[2]{\&#1 \rightarrow #2}
\newcommand{\symhole}{\Box}
\newcommand{\None}{\bot}
\newcommand{\Some}[1]{\lfloor #1 \rfloor}
\newcommand{\sympartial}{\rightharpoonup}
\newcommand{\eval}[2]{{\left \lsem #1 \right \rsem}_{#2}}
\newcommand{\symget}{\kw{Get}}
\newcommand{\symset}{\kw{Set}}
\newcommand{\step}{\leadsto}
%% \newcommand{\astep}{\leadsto_0}
\newcommand{\astep}{\rightarrowtriangle}
\newcommand{\fplug}[2]{#1\;[#2]}
\newcommand{\option}[1]{\widetilde{#1}}
\newcommand{\withframe}{\kw{withframe}}
\newcommand{\epush}{\kw{push}}
\newcommand{\epop}{\kw{pop}}
\newcommand{\subst}[3]{ [#2/ #1] #3 }
\newcommand{\lowtoc}[1]{\downarrow #1}
\newcommand{\lowtoce}[1]{\downlsquigarrow #1}
\newcommand{\lowtocd}[1]{\downdownarrows #1}
\newcommand{\ctolow}[1]{\uparrow #1}
\newcommand{\ctolowe}[1]{\uprsquigarrow #1}
\newcommand{\ctolowd}[1]{\upuparrows #1}
%% \newcommand{\ctolowc}[1]{\Uparrow #1}
%% \newcommand{\ctolowc}[1]{\uparrowtail #1}
\newcommand{\ctolowc}[1]{\rhookuparrow #1}
\newcommand{\ctolowE}[1]{\twoheaduparrow #1}
\newcommand\defeq{\mathrel{\overset{\makebox[0pt]{\mbox{\normalfont\tiny\sffamily def}}}{=}}}
\newcommand{\sys}{\kw{sys}}
\newcommand{\unstuck}{\kw{unstuck}}
\newcommand{\normal}[2]{\Downarrow_{#2} #1}
\newcommand{\unravel}{\kw{unravel}}
\newcommand{\foldl}{\kw{foldl}}
\newcommand{\unravelframe}{\kw{unravel\_frame}}
\newcommand{\mem}{\kw{mem}}
\newcommand{\symmin}{\kw{min}}
\newcommand{\eand}{\wedge}
\newcommand{\eor}{\vee}
%% \newcommand{\symImply}{\Rightarrow}
%% \newcommand{\symIff}{\Leftrightarrow}
\newcommand{\symImply}{\rightarrow}
\newcommand{\symIff}{\leftrightarrow}
\newcommand{\olabel}{o}
%% \newcommand{\lolabel}{{lo}}
\newcommand{\brt}{\kw{brT}}
\newcommand{\brf}{\kw{brF}}

% LamStar
\newcommand \lp{P}
\newcommand \lexp{e}
\newcommand \fd{f}
\newcommand \lv{v}
\newcommand \trace{\ell}

% CStar
\newcommand \cp{\hat{P}}
\newcommand \cexp{\hat{e}}
\newcommand \cstmt{s}


\usepackage{xspace}

\newcommand*{\EG}{e.g.,\xspace}
\newcommand*{\IE}{i.e.,\xspace}
\newcommand*{\ETAL}{et al.\xspace}
\newcommand*{\ETC}{etc.\xspace}

\newcommand*{\ii}[1]{\ensuremath{\mathit#1}}
