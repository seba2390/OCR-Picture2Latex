\begin{abstract}
We present \lowstar, a language for low-level programming and
verification, and its application to high-assurance optimized cryptographic libraries.
%
\lowstar is a shallow embedding of a small, sequential, well-behaved
subset of C in \fstar, a dependently-typed variant of ML aimed at
program verification.
%
Departing from ML, \lowstar does not involve any garbage collection or
implicit heap allocation; instead, it has a structured memory model
\`a la CompCert, and it provides the control required for writing
efficient low-level security-critical code.

By virtue of typing, any \lowstar program is memory safe.
%
In addition, the programmer can make full use of the verification
power of \fstar to write high-level specifications and verify the
functional correctness of \lowstar code using a combination of SMT
automation and sophisticated manual proofs.
%
At extraction time, specifications and proofs are erased, and the
remaining code enjoys a predictable translation to C.
%
We prove that this translation preserves semantics and
side-channel resistance.

We provide a new compiler back-end from \lowstar to C and, 
to evaluate our approach, we implement and verify various cryptographic algorithms,
constructions, and tools for a total of about 28,000
lines of code, specification and proof.
%
We show that our \lowstar code delivers performance competitive with
existing (unverified) C cryptographic libraries, suggesting our
approach may be applicable to larger-scale low-level
software.
\cf{This suggestion is lukewarm; could also talk about usability and
  the risks of crypto libraries. Noop.}
\end{abstract}
