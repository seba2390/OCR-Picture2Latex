\emf~\cite{dm4free}, the formal model of \fstar, is a
dependently typed calculus for pure programs, that can be extended
with a partially ordered user-defined effects (such as state,
exceptions, etc.) represented in terms of the primitive
\kw{Pure} effect. %% The dynamic semantics of \emf reduces the \kw{Pure}
%% effect primitively, while the user-defined effects are reduced using
%% the underlying \kw{Pure} definitions of their monadic \kw{return} and
%% \kw{bind} functions.
%% \paragraph{Erasure}
To express computational irrelevance, we extend \emf with a
primitive \kw{Ghost} effect whose dynamic semantics is identical
to \kw{Pure}.
%
As such, we provide a subsumption rule to treat
proof-irrelevant \kw{Ghost} computations (e.g., those returning
non-informative types like \lst$unit$) as \kw{Pure} computations, and,
dually,
%
allow all \kw{Pure} computations to be subsumed
to \kw{Ghost}. Otherwise, we forbid \kw{Ghost} code from being
composed with any other effect, preventing computationally relevant
computations from relying on \kw{Ghost} code.
%
We define an erasure relation, where all unit-returning \kw{Ghost}
and \kw{Pure} sub-terms of a well-typed term are erased
to \lst$()$. We prove that erasure preserves typing, meaning that all
properties proven of the source program, remain valid after erasure.
%
The appendix contains a formal statement and proof, omitted here for
lack of space, since we focus mainly on the translation to Clight,
rather the transformations internal to \fstar.

\paragraph{Primitive state} While the state effect in \emf is a
user-defined effect, parametric in the state, Ahman et
al.~\cite{dm4free} also present \emfST, a calculus with primitive
state, and prove a simulation between \emfST and
\emf. Relying on this result, we instantiate the
memory model of \emfST with stack-based memory,\ch{what's stack-based memory?
  you mean hyper-stacks? or you mean the hyper-stack part without the heap
  part ... this should be made explicit} and introduce
\lamstar, a formal core of \lowstar (\sref{lamstar}).
We have not yet formally proven
the simulation between \emf and \lamstar, but we conjecture that the
proof will be along similar lines as Ahman et al.
\nik{Mention non-termination here.}
\tr{It follows from the bisimulation that since the original program is memory
safe, then the resulting \lamstar program is memory safe.}
