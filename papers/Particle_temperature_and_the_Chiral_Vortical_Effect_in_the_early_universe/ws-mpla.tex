%%%%%%%%%%%%%%%%%%%%%%%%%%%%%%%%%%%%%%%%%%%%%%%%%%%%%%%%%%%%%%%%%%%%%%%%%%%%
%% Trim Size: 9.75in x 6.5in
%% Text Area: 8in (include Runningheads) x 5in
%% ws-mpla.tex   :   29-9-2008
%% TeX file to use with ws-mpla.cls written in Latex2E.
%% The content, structure, format and layout of this style file is the
%% property of World Scientific Publishing Co. Pte. Ltd.
%% Copyright 1995, 2002 by World Scientific Publishing Co.
%% All rights are reserved.
%%%%%%%%%%%%%%%%%%%%%%%%%%%%%%%%%%%%%%%%%%%%%%%%%%%%%%%%%%%%%%%%%%%%%%%%%%%%
%%

\documentclass{ws-mpla}
\usepackage[super]{cite}
\usepackage{graphicx}
\begin{document}

\markboth{Authors' Names}
{Instructions for Typing Manuscripts (Paper's Title)}

%%%%%%%%%%%%%%%%%%%%% Publisher's Area please ignore %%%%%%%%%%%%%%
\catchline{}{}{}{}{}
%%%%%%%%%%%%%%%%%%%%%%%%%%%%%%%%%%%%%%%%%%%%%%%%%%%%%%%%%%%%%%%%%%%

\title{INSTRUCTIONS FOR TYPESETTING MANUSCRIPTS\\
USING \TeX\ OR \LaTeX\footnote{For the title, try not to use more than
three lines. Typeset the title in 10 pt Times Roman, uppercase and
boldface.}
}

\author{\footnotesize FIRST AUTHOR\footnote{
Typeset names in 8 pt Times Roman, uppercase. Use the footnote to
indicate the present or permanent address of the author.}}

\address{University Department, University Name, Address\\
City, State ZIP/Zone,
Country\footnote{State completely without abbreviations, the
affiliation and mailing address, including country and e-mail address.
Typeset in 8 pt Times Italic.}\\
author@emailaddress}

\author{SECOND AUTHOR}

\address{Group, Laboratory, Address\\
City, State ZIP/Zone, Country
}

\maketitle

\pub{Received (Day Month Year)}{Revised (Day Month Year)}

\begin{abstract}
The abstract should summarize the context, content and conclusions of
the paper in less than 200 words. It should not contain any references
or displayed equations. Typeset the abstract in 8 pt Times Roman with
baselineskip of 10 pt, making an indentation of 1.5 pica on the left
and right margins.

\keywords{Keyword1; keyword2; keyword3.}
\end{abstract}

\ccode{PACS Nos.: include PACS Nos.}

\section{General Appearance}	

Contributions should be in English. Authors are encouraged to have
their contribution checked for grammar.  American spelling should be
used. Abbreviations are allowed but should be spelt out in full when
first used. Integers ten and below are to be spelt out. Italicize
foreign language phrases (e.g.~Latin, French).

The text should be in 10 pt Times Roman, single spaced with
baselineskip of 13~pt. Text area (including copyright block)
is 8 inches high and 5 inches wide for the first page.
Text area (excluding running title) is 7.7 inches high and
5 inches wide for subsequent pages.  Final pagination and
insertion of running titles will be done by the publisher.

\section{Running Heads}

Please provide a shortened runninghead (not more than eight words) for
the title of your paper. This will appear on the top right-hand side
of your paper.

\section{Major Headings}

Major headings should be typeset in boldface with the first
letter of important words capitalized.

\subsection{Sub-headings}

Sub-headings should be typeset in boldface italic and capitalize
the first letter of the first word only. Section number to be in
boldface Roman.

\subsubsection{Sub-subheadings}

Typeset sub-subheadings in medium face italic and capitalize the
first letter of the first word only. Section numbers to be in Roman.

\subsection{Numbering and spacing}

Sections, sub-sections and sub-subsections are numbered in
Arabic.  Use double spacing before all section headings, and
single spacing after section headings. Flush left all paragraphs
that follow after section headings.

\subsection{Lists of items}

Lists may be laid out with each item marked by a dot:
\begin{itemlist}
 \item item one,
 \item item two.
\end{itemlist}
Items may also be numbered in lowercase Roman numerals:
\begin{romanlist}[(ii)]
\item item one
\item item two
	\begin{romanlist}[(b)]
	\item Lists within lists can be numbered with lowercase
              Roman letters,
	\item second item.
	\end{romanlist}
\end{romanlist}

\section{Equations}

Displayed equations should be numbered consecutively in each
section, with the number set flush right and enclosed in
parentheses
\begin{equation}
\mu(n, t) = \frac{\sum^\infty_{i=1} 1(d_i < t,
N(d_i) = n)}{\int^t_{\sigma=0} 1(N(\sigma) = n)d\sigma}\,.
\label{diseqn}
\end{equation}

Equations should be referred to in abbreviated form,
e.g.~``Eq.~(\ref{diseqn})'' or ``(2)''. In multiple-line
equations, the number should be given on the last line.

Displayed equations are to be centered on the page width.
Standard English letters like x are to appear as $x$
(italicized) in the text if they are used as mathematical
symbols. Punctuation marks are used at the end of equations as
if they appeared directly in the text.

\section{Theorem Environments}

\begin{theorem}
Theorems, lemmas, etc. are to be numbered
consecutively in the paper. Use double spacing before and after
theorems, lemmas, etc.
\end{theorem}

\begin{proof}
Proofs should end with $\square$.
\end{proof}

\section{Illustrations and Photographs}
Figures are to be inserted in the text nearest their first
reference. eps files or postscript files are preferred. If
photographs are to be used, only black and white ones are acceptable.

\begin{figure}[ph]
\centerline{\includegraphics[width=2.0in]{mplaf1}}
\vspace*{8pt}
\caption{A schematic illustration of dissociative recombination. The
direct mechanism, 4$m^2_\pi$ is initiated when the
molecular ion $S_{L}$ captures an electron with kinetic
energy.\protect\label{fig1}}
\end{figure}

%%%Type this at body text - Fig.~\ref{fig1}
Figures are to be placed either top or bottom and sequentially
numbered in Arabic numerals. The caption must be placed below the figure.
Typeset in 8 pt Times Roman with baselineskip of 10 pt. Use double
spacing between a caption and the text that follows immediately.

Previously published material must be accompanied by written
permission from the author and publisher.

\section{Tables}

Tables should be inserted in the text as close to the point of
reference as possible. Some space should be left above and below
the table.

Tables should be numbered sequentially in the text in Arabic
numerals. Captions are to be centralized above the tables.
Typeset tables and captions in 8 pt Times Roman with baselineskip
of 10 pt.

\begin{table}[h]
\tbl{Comparison of acoustic for frequencies for piston-cylinder problem.}
{\begin{tabular}{@{}cccc@{}} \toprule
Piston mass & Analytical frequency & TRIA6-$S_1$ model &
\% Error \\
& (Rad/s) & (Rad/s) \\
\colrule
1.0\hphantom{00} & \hphantom{0}281.0 & \hphantom{0}280.81 & 0.07 \\
0.1\hphantom{00} & \hphantom{0}876.0 & \hphantom{0}875.74 & 0.03 \\
0.01\hphantom{0} & 2441.0 & 2441.0\hphantom{0} & 0.0\hphantom{0} \\
0.001 & 4130.0 & 4129.3\hphantom{0} & 0.16\\ \botrule
\end{tabular}\label{ta1} }
\end{table}

If tables need to extend over to a second page, the continuation
of the table should be preceded by a caption,
e.g.~``{\it Table \ref{ta1}.} $(${\it Continued}$)$''

\section{Footnotes}

Footnotes should be numbered sequentially in superscript
lowercase Roman letters.\footnote{Footnotes should be
typeset in 8 pt Times Roman at the bottom of the page.}

\appendix

\section{Appendices}

Appendices should be used only when absolutely necessary. They
should come before Acknowledgments. If there is more than one
appendix, number them alphabetically. Number displayed equations
occurring in the Appendix in this way, e.g.~(\ref{appeqn}), (A.2),
etc.
\begin{equation}
\mu(n, t) = \frac{\sum^\infty_{i=1} 1(d_i < t, N(d_i)
= n)}{\int^t_{\sigma=0} 1(N(\sigma) = n)d\sigma}\,.
\label{appeqn}
\end{equation}

\section*{Acknowledgments}

This section should come before the References. Dedications and funding
information may also be included here.

\section*{References}

They are to be cited in the text in superscript
after the punctuation marks e.g.~word,\cite{Marnelius} and word:\cite{Bjorken}
If it is mentioned in the text as part of a sentence, it should be of normal size,
e.g.~see Ref.~\refcite{Bohr}. Please list using the style shown in the following examples.
For journal names, use the standard abbreviations. Typeset references in 9 pt Times Roman.
Each reference number should consist of one reference only.

\begin{thebibliography}{0}
\bibitem{Marnelius} R. Marnelius, {\it Acta Phys. Pol. B} {\bf 13},
669 (1982).

\bibitem{Bjorken} J. D. Bjorken, in {\it Lecture Notes on Current-Induced
Reactions}, eds.~J. Komer {\it et al.} (Springer, 1975).

\bibitem{Bohr} A. Bohr and B. R. Mottelson, {\it Nuclear Structure}
(Benjamin, 1969), Vol.~1, pp.~100--102.

\bibitem{Webb} R. C. Webb, Ph.D. thesis, Princeton University, 1972.

\bibitem{Toimela} T. Toimela, Helsinki Research Institute for
Theoretical Physics, Report No. HU-TFT-82-37, 1982 (unpublished).
\end{thebibliography}
\end{document}
