\section{Discussion and Conclusions}
\label{sec:conclusion}
In this paper, we present a novel methodology to predict the spatial forces generated by a class of fiber-reinforced fluid-driven actuators (FREEs) that are wound with only one family of fibers.
Our approach is based on the notion of a \emph{fluid Jacobian}, which we propose as the multidimensional and soft equivalent of the cross section $A$ of a traditional pneumatic or hydraulic cylinder.
We use this new modeling approach to predict and visualize the force generation capabilities of parallel combinations of such actuators to yield controllable multidimensional end effector forces.
%The model, derived from energy conservation of a fluid volume constrained by inextensible elements and the superposition of forces acting on a common end effector, predicts the forces generated by a parallel combination of FREEs based on the internal pressures of the actuators and the position of the end effector. 
Our approach is verified experimentally, predicting the forces and moments generated by a parallel combination of three specificially configured FREEs within \unit[3]{N} and \unit[$19 \times 10^{-3}$]{Nm}, respectively.

We envision that the force zonotope presented becomes an instrumental tool in the design of soft robotic systems.
\revcomment{3.5}{This approach could be used to choose a suitable arrangement of FREEs to actuate a compliant manipulator arm, soft orthotic device, or any other application where a custom force profile is desired.}
Here, we used it to systematically arrange the three FREEs in the experimental validation to ensure that the forces and torques created by the individual actuators covered all desired dimensions.
Furthermore, studying the zonotope as a function of state (Fig.~\ref{fig:zntp_vs_x}) allows us to make predictions about the achievable workspace of a soft robotic system.
For example, for the experimental system presented in this work, the zonotope will collapse for compressions of more than \unit[-10]{mm}.
At this point, it will become impossible to create contracting axial forces.

The force zonotope-based model captures the physical behavior of FREEs well enough to be useful in the design of robotic applications.
However, several modeling assumptions limit its accuracy and precision as they might be needed for high-fidelity control. 
Most notably, FREEs are assumed to be cylindrical. 
This assumption introduces inaccuracy when a FREE is bent, buckled, or kinked, \revcomment{2.10}{requiring the development of further models to predict the conditions under which these undesirable behaviors will occur. However, while the model presented does not account for bending or buckling,} 
the fluid Jacobian approach is not inherently limited to cylindrical models and could be extended to more complicated geometrical descriptions of a FREE.
A more fundamental shortcoming is that the model does not explicitly describe the elastomer contributions to force.
That is, our approach must be supplemented with a suitable elastomer model to fully capture the force characteristics of of FREEs.
While this was not the focus of the current work, such a model could be linear based on empirical data \cite{bruder2017model}, derived as a continuum model from first principles \cite{sedal2017constitutive}, or computed via finite element analysis \cite{connolly2015mechanical}.


For soft robots to leverage the advantages in maneuverability, safety, and robustness of non-rigid structures, they require actuators that do not inhibit their compliance. 
A soft actuator such as a FREE meets this criterion because it generates a spacial force without constraining motion to occur in the direction of that force. 
A FREE also has the added benefit of a customizable force direction based on fiber angle; a property that was explicitly exploited in this work.
Parallel combinations of FREEs can be constructed to generate arbitrary spacial forces while still retaining compliance. 
By characterizing the forces generated by parallel combinations of FREEs, this work lays the foundation for future applications of complex soft robotic manipulators.