\vspace{-1em}
\section{Related Work}
\vspace{-0.6em}
\label{SEC:REL}

Here we present the previous studies about failures of the flash-based memories under different conditions. The previous studies can be investigated in two groups where the studies in the first group mainly focus on different types of failures due to internal structure of flash cells  such as endurance, read disturbance, and write disturbance. In the second group, the impact of external failures on flash-based memories such as power outage is analyzed. In the following we provide the previous studies in more details.

The studies on flash-based memory systems reliability such as \cite{meza2015large, Boboila2010a, narayanan2016ssd, grupp2009characterizing,schroeder2016flash} have investigated the failures such as read disturbance, write disturbance, and write endurance which commonly occur on flash-based memories in chip level and device level designs.
Meza et al. have studied the failures on the SSDs in Facebook datacenters during four years of operation \cite{meza2015large}. They observed that the special failure trend in SSDs is same as \say{bathtube curve} which consists of early detection, early failure, usable lifetime, and wearout phases  which the wear-out phase does not experience a monotonic failure rate.
Other similar studies such as \cite {narayanan2016ssd} have investigated the SSD failures in the larger scale of production environment presenting more realistic results.
\cite {Boboila2010a, grupp2009characterizing} have measured performance, power consumption, and reliability of flash memories in order to provide the best trade-off for storage system configuration. They have observed that there is considerable difference between experimental results and provided datasheets  by manufacturers. 
%Although these works presented applicable information about some failures in flash-based memory device vendors they don't consider power outage-caused failures which is a normal and frequent accident in data centers. 

%Manufacturers always ignore the impact of power outage on flash-based memories and provide little information about storage device characteristics corresponded to the power faults in devices' datasheets. Despite of that some high-end devices use batteries and supercapacitors to be able to retain their data during power outage, low-end devices does not support these costly recovery techniques. In addition, these techniques try to ensure the completeness of sent requests during power loss while power outage leads to some unpredictable failures which are not related to just the under writing data. Hence, investigation the effects of power failures on the  entire flash-based device can assist the manufacturers to design more reliable devices. 

In order to investigate the effects of power failure on embedded systems, a software-based test platform  is presented in \cite {kim2007virtual}.
 This  platform simulates the power failures in \emph{Flash Translation Layer} (FTL) of SSDs and file systems in \emph{Operating System} (OS) layer. Such software test platform is able to detect only a limited number of expected (i.e., previously defined) failures which is not capable in modeling real faults and detecting the corresponding failures. 
Limited number of recent studies such as \cite{tseng2011understanding, zheng2013understanding} have examined real experiments in order to investigate the impact of real power failures on flash-based memory systems. Tseng et al. have proposed an FPGA-based test framework which cuts off the power of flash chip by employing high-speed power transistors controlled by FPGA \cite{tseng2011understanding}. They have observed several failures caused by power outage in the chip level design of flash-based memory systems.
However, due to the applied recovery mechanisms in the device level design of such systems (e.g., SSDs), most of chip level failures are eliminated in device level products that would not result in data failures in such devices. Therefore, later studies have evaluated the reliability of flash-based memory systems in device level designs such as SSDs in order to reveal the the behavior of them under power failures.
%testing in device level could assess pure uncorroctable failures. 
To this end, Zheng et al. have proposed a test framework to evalute SSD failures under power faults. Fifteen SSDs from five enterprise vendors have been examined and the results reveal that thirteen out of fifteen SSDs have experienced several failures due to power outage \cite{zheng2013understanding}.
This study only measures failures by submitting I/O requests of one constant simple workload (random and sequential write) while the 
impact of  several important workload based parameters such as 
1) workloads WSS, 2) requests size, 3) requests type (read/write), 4) sequence of the accesses such as \emph{Read After Read} (RAR), \emph{Read After Write} (RAW), \emph{Write After Read} (WAR), and \emph{Write After Write} (WAW), 5) type of application level operations, and 6) requests access pattern (random/sequential)
 are neglected during power failure analysis. Moreover, the hardware fault injection mechanism which is employed in \cite{tseng2011understanding, zheng2013understanding} involves high-speed power transistors to cut off the power of SSDs. Such power failures would cut off the power without considering the impact of large size capacitors employed in PSU on the rise/fall delay of the SSDs voltage.
 
%
%\section{Motivation}
%\label{SEC:MOT}
%
%Power outage can cause some unpredictable failures which is not completely investigated in the field. Unlike the expectations about the data durability in flash memory cells, they manifest partial volatile behavior under power fault. Flash memory manufacturers provide limited information about the power outage failures while the data durability in dependability-critical area is significantly important. 
%
%The partial volatile behavior under power ouatge originate from the complex program and erase operations in flash memories. For a single page programming, the flash controller iterates several program-read-verify procedure to determine whether the page cells have reached to the desired state or not. However, the iterations may be corrupted by the power outage leaving the page cells data being corrupted. Furthermore, erasing in flash memories needs more time to be completed which makes them more susceptible against power outage. This more time is due to the block operation in erasing versus page operation in programming, in which each cell of the block should be throughly erased and also prepared for the subsequent programming operations. 
%
%MLC, TLC and other kind of flash memories sustained two or more bits per cell exhibit more susceptibility against power loss comparing to SLCs which stores only one bit per cell. In fact, MLC flash memories is more complex, less reliable and slower than SLCs. In order to improve performance of MLC flash memories, one solution is to consider two bits from each physical cell to be mapped for distinct logical pages. This approach is universally used by flash memory manufacturers. However, since programming and erasing to a flash cell is a complex and iterative procedure, writing to the second logical page could corrupt the previously written page. In addition, the power outage can interrupt this multi level programming/erasing procedure which increases the corruption probability of the previously successful written page. This effect worsens when the number of bits (logical pages) per physical cell increases. In essence, TLC flash memories stored three bits per cell is more vulnerable to power outage than MLCs, and so on.
%
%As flash memories are subject to several failures such as write endurance, read/write disturbance and cell-to-cell program intereference (discussed in Section~\ref{SEC:BAK}), they are likely to have several failures due to the power loss as well. Unexpectedly, power outage not only do disturb the under writing cell, but it also can corrupt the cells completely written before. Our investigations show that single power outage corrupts the under writing request and also several previous requests which are completely written. It reveals that non-volatile characteristics of flash memories exists just when the predefined amount of time will be elapsed after the request completeness. During this time, power outage leads to the corruption of the completely written data.This unexpected behavior of flash memories is not investigated and also the manufacturers have not emphasised of that, while this issue in flash memories is considerably threatening in dependability-critical area. We aim in this paper to throughly investigate the effects of power outage under different workloads to address the recovery of failures.   


