\section{Background}
\label{SEC:BAK}
\subsection{Physics of FLASH Memories}
\label{sec:phisics}


Flash memories are widely employed in the memory portion of solid state drives. Each flash memory cell is a special kind of transistor that can store data in addition to the switching operation named ``floating gate transistor". Fig.~\ref{FLASH_CELL}(a) shows the cross section of floating gate transistor consisting of floating gate (FG), control gate (CG), and two dielectic material parts during program and erase operation. In program operation, when an extera high electric voltage (about +10V) is applied to the CG, electrons are injected from source to the FG via F-N tunneling procedure. The trapped electrons in the FG can stay for long time (about minimum of 10 years) even when the power supply cuts off. Therefore, floating gate transistors are one of non-voltaile memories (NVM),  which can retain the data during power loss. However, in erase operation the trapped electrons in the FG are released to the source by applying extra negative voltage in the CG. 

The amount of charge stored in the FG determines the $V_{th}$ of the transistor. The injected charge in program operation results in the increase of transistor's $V_{th}$, and removing of the charge in erase operation reduces the $V_{th}$. The two different $V_{th}$ can specify one bit of data, and is accomplished in single-level cell (SLC) flash memories. There are several types of flash memories that can store more than one bit per cell. As an example, in multi-level cell (MLC) flash memories, there exists 4 different amount of charges (and consequently 4 different $V_{th}$) that can specify 2 bits of data. MLC flash memories are employed for higher density, whereas SLC flash memories are used for more reliable and fast operations.

\subsection{FLASH Memory Vulnerability to Several Failures}
\label{sec:Vulnerabilities}

Fig.~\ref{fig:FLASH_CELL}(b) demonstrates the program and read voltages to the cells in NAND flash memories where floating gate transistors have a NAND like sturcture. Program and read operation are performed for one raw of cells named ``page", while erasing can be performed for multi pages named ``block". The value of a cell is distinguished by the $V_{th}$ it has. As shown in Fig.~\ref{fig:FLASH_CELL}(c), corresponding $V_{th}$ of two cell states is given. $V_{TH,1}$/$V_{TH,0}$ is the threshold voltage when cell value is ``1/0" respectively. According to Fig.~\ref{fig:FLASH_CELL}(b), in order to read the cell value, $V_{INT}$ is applied to the desired cell's word line, and to propagate the resulting current to the sense amplifiers, the other cells' word lines will be set to pass-through voltage ($V_{PASS}$) in the same block. $V_{PASS}$ is the voltage in which floating gate transistors conduct regardless of the values they store. The sense amplifiers measure the amount of bit line current, and existance/nonexistance of the currents in the bit lines determines the cell value of ``1/0" respectively. 

\begin{figure}[!t]
	\centering
	\subfloat[]{\includegraphics[width=.16\textwidth]{FLASH_CELL}%
		\label{fig:flash_cell}}
	\hfil
	\hspace{-.8pt}
	\subfloat[]{\includegraphics[width=.16\textwidth]{NAND_flash}%
		\label{fig:nand_flash}}
	\hfil
	\hspace{-.8pt}
	\subfloat[]{\includegraphics[width=.16\textwidth]{vgs-id}%
		\label{fig:vgs_id}}
	
	\vspace{-0.5em}
	\caption{(a) F-N tunneling from channel to the FG and vice-versa during program and erase operation in flash memories, (b) Read and program operation can cause $V_{th}$ increase of the cells within the same block due to applying $V_{PASS}$ on the unselected cells , and (c) the amount of $V_{PASS}$, $V_{INT}$ and $V_{th}$ of two cell states.}
	\vspace{-1.1em}
	\label{fig:FLASH_CELL}
\end{figure}


The voltages applied to target or even neighboring cells in flash memories are reletively high and can cause some of failures. Therefore, Reliability of flash memories are faced to several threats such as write enduarnce, read/write disturbance, and cell-to-cell program interference which are important issues regarded to flash memories. Since trapped electrons in the erase operation is not completely ejected to the substrate, it can degrade the transistor's dielectric in each write operation and eventually causes a permanent error. The reported write endurance in SLC/MLC flash memories is about 100,000/10,000 of program/erase (P/E) cycles respectively.

Read and write disturbance are likely to happen in flash memories. When a read operation is performed to one page of cells, the $V_{PASS}$ is applied to the cells not involved in the read imposing the ``weak programming voltage''. Although this voltage (about +6V) is less than the programming voltage (+10V), it is still high enough to shift the $V_{th}$ of the cells within the same block. The accumulated effect of weak programming voltages is read disturb error which can change the cell state. As the same way, write disturb occurs during program operation due to applying $V_{PASS}$ to cells not selected for programming. On the other hand, increase of the transistor's $V_{th}$ of the target cells during program operation can increase the $V_{th}$ of the neighboring cells as well. This is the cell-to-cell program interference which is a consequence of capacitve coupling between the neighboring flash memory cells. However, except write endurance, all of the issues cause transient errors and can be corrected by error correcting codes (ECC).
