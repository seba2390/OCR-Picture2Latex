\vspace{-1em}
\section{Introduction}
\vspace{-0.6em}
\emph{Solid-State Drives} (SSDs) are known as high-performance non-volatile drives that are widely employed in modern storage systems. Unlike \emph{Hard Disk Drives} (HDDs), SSDs consume less power and provide higher performance because of their non-mechanical structure. However, SSDs cost about 10X higher than HDDs and support only a limited number of writes \cite{micheloni2012inside,salkhordeh2015operating,reca}. SSDs are composed of high performance flash memories where each flash cell 
%includes floating gate transistors namely \emph{Floating Gate MOSFET} (FGMOS) which 
can retain data for considerable amount of time (minimum of 10 years) by trapping the electrons in the floating gate \cite{cai2015data, bez2003introduction_flash}. 

%By storing multi levels of charges in each flash cell, SSDs offer different types of flash memories named \emph{Single-Level Cell} (SLC), \emph{Multi-Level Cell} (MLC), and \emph{Triple-Level Cell} (TLC) which store one, two, and three bits per cell, respectively. 
%SLCs provide higher performance and reliability but they are costly compared to MLCs and TLCs \cite{grupp2009characterizing, grupp2012bleak_flash}. 
During erase operation in flash cell, electrons \say{stick} in the floating gate which results in the transistor dielectric degradation named \emph{write endurance}. This makes flash memories to provide limited number of program/erase cycles \cite{soundararajan2010extending, Boboila2010a}.
In addition, they suffer from other different reliability problems such as read/write disturbance and cell-to-cell program interference \cite{cai2017vulnerabilities}.
Furthermore, flash memories suffer from lack of in-place-update where an additional erase is required for each write operation. To overcome this limitation, \emph{Flash Translation Layer} (FTL) is employed in SSDs which implements various functionalities  including a) address mapping algorithm to hide SSD limitations from the upper levels, b) garbage collection, and c) wear leveling algorithms in order to alleviate the negative impact of write operations on the endurance of the SSDs \cite{chen2011caftl, gal2005algorithms_flash}.

In addition to aforementioned shortcomings, flash-based SSDs suffer from variant failures due to power failures. 
%Power outage can cause some unpredictable failures which is not completely investigated in the field. 
Unlike data durability expectation from flash cells, they manifest partial volatile behavior under power fault which results in data failures in SSDs. Such partial volatile behavior is originated from complex program and erase operations in flash memories. For a single page programming, the flash controller iterates several program-read-verify procedure to determine whether the page cells have reached to the desired state or not. Hence, such long intervals of iterations may be disrupted by sudden power failure which results in data corruption of the page cells. Furthermore, erase operation in flash memories takes long delay to be completed which makes them more susceptible against power failures. 
In addition, such power failures may disrupt the operation of FTL. 

In order to provide high performance write operations, SSDs keep write pending requests in a volatile write-back DRAM cache. Such caching scheme within SSDs is susceptible to data loss due to power failures \cite{zheng2016reliability, zheng2013understanding}. 
To alleviate this problem, some high-end devices employ batteries and supercapacitors while low-end devices do not support such costly recovery schemes. In addition, such schemes only provide the condition to move the write pending data in DRAM cache to the flash memories while power outage leads to some unpredictable failures which may not be recovered in such manner.
Manufacturers always ignore the impact of power failure on the reliability of SSDs while such failure is one of the frequently occurring failures in data centers \cite{mcmillan2012amazon, claburnamazon, miller2012human, leach2012level}.
 Hence, investigating the impact of power failures on the entire flash-based devices such as SSDs assists a) manufacturers to design more reliable devices and b) designers to carefully architect storage systems. 

In this paper, we investigate the impact of various parameters such as workload \emph{Working Set Size} (WSS), request size, request type, access pattern, sequence of accesses, and the cache residing in the SSD on the ratio of data failure in presence of  power faults. To this end, we implement a power failure test platform in order to investigate the impact of power outages on SSDs. This platform includes software and hardware parts in which the hardware part is responsible for injecting real physical faults to SSDs while the software part schedules the faults and issues IO request to SSDs in two distinct threads. Failure detection process is performed by comparing the checksum of the written data and the original data. Experimental results reveal that single power outage not only disturbs the under writing cell, it also may corrupt the cells that are previously written to the SSD. In addition, the results show that data failure occurs in SSDs in a period of time (which cannot be determined clearly) after completion of the request.

Compared to previous studies such as \cite{kim2007virtual} which emulates the power fault effect on flash based memories, our proposed platform examines the realistic power faults on SSDs. Moreover, in our proposed test platform, the under test SSDs experience the real power outage by considering the exact delay of discharging (i.e., delay of discharging large size capacitors employed in \emph{Power Supply Unit} (PSU)) while such delay has not been included during power failures of the platforms presented by state-of-the-art studies such as \cite{tseng2011understanding, zheng2013understanding}.

In summary, the main contributions of this work are as follows.
\begin{itemize}
\item To our knowledge, this paper is the first to inject realistic power faults to the under test SSDs where the SSDs experience the real delay of PSU discharging phase during power failure.

\item By conducting extensive workload analysis, we have investigated the impact of several workload dependent parameters such as workload WSS, request size, request type, access pattern, and sequence of accesses on data failures and observed significant impact of such parameters on failure rate.

\item We propose a fault injection and failure detection platform which includes hardware-software co-design in order to evaluate the behavior of SSDs under power failures. The proposed platform detects three types of \emph{IO failures} that may occur for a request during power faults on SSDs namely: 1) \emph{data failure}, 2) \emph{False Write-Acknowledge (FWA)}, and 3) \emph{IO error}. These three types of IO failures have not been addressed in the previous work.

\item We have examined more than five SSDs from different vendors and investigated the impact of power failures on their reliability.



%\item We introduce three types of \emph{IO failures} that may occur for a request during power faults on the SSDs namely: 1) \emph{Data Failure}, 2) \emph{Not Inserted IO}, and 3) \emph{IO Error}.
%\item We have
\end{itemize} 
The rest of paper is organized as follows. Section \ref{SEC:REL} discusses related work. In Section \ref{sec:proposed}, we present our proposed test platform. Section \ref{SEC:EXPR} provides experimental setup and results. Finally, Section \ref{SEC:CONC} concludes the paper.


