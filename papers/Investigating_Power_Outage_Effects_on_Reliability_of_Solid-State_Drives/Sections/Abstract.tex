\emph{Solid-State Drives} (SSDs) are recently employed in enterprise servers and high-end storage systems in order to enhance performance of storage subsystem. Although employing high speed SSDs in the storage subsystems can significantly improve system performance, it comes with significant reliability threat for write operations upon power failures. 
In this paper, we present a comprehensive analysis investigating the impact of workload dependent parameters
on the reliability of SSDs under power failure for variety of SSDs (from top manufacturers). To this end, we first develop a platform to perform two important features required for study: a) a realistic fault injection into the SSD in the computing systems and b) data loss detection mechanism on the SSD upon power failure. In the proposed physical fault injection platform, SSDs experience a real discharge phase of \emph{Power Supply Unit} (PSU) that occurs during power failure in data centers which was neglected in previous studies.
The impact of workload dependent parameters such as workload \emph{Working Set Size} (WSS), request size, request type, access pattern, and sequence of accesses on the failure of SSDs is carefully studied in the presence of realistic power failures. Experimental results over thousands number of fault injections show that data loss occurs even after completion of the request (up to 700ms) where the failure rate is influenced by the type, size, access pattern, and sequence of IO accesses while other parameters such as workload WSS has no impact on the failure of SSDs.
