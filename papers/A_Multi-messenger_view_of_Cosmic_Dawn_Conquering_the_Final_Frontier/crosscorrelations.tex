


\begin{figure}
\centering
\includegraphics[width = 0.5\textwidth]{clexample.pdf}\includegraphics[width = 0.5\textwidth]{contour_forreview.pdf}
\caption{\textit{Left panel:} Angular cross-correlation power spectrum computed from \eq{angcross} with a CHIME-DESI-like survey combination at redshifts 0.8, 1.2, and 1.6. The error bars denote the corresponding $\Delta C_{\ell}$ at the lowest redshift bin. Figure from \cite{hparaa2019}. \textit{Right panel:} Contours of cosmological parameters $\sigma_8$ and $\Omega_m$ with \textit{all} other parameters fixed in the cross-correlation.}
\label{fig:contour1}
\end{figure} 





\section{The whole is greater than the sum of its parts}
\label{sec:crosscorrelations}

In this section, we describe how bringing together two (or more) different surveys leads to several advantages in the measurement of cosmological and and astrophysical parameters from baryonic tracers. Furthermore, it offers a direct route to extend the modelling frameworks towards the reionization regime, connecting up with future gravitational wave measurements to provide a holistic picture of cosmic dawn.   

\subsection{Gas-galaxy cross-correlations}

If we cross-correlate maps of gas emission (like HI) and galaxy surveys in the optical band, we can significantly improve our accuracy in the prediction of astrophysical parameters. This happens due to the foregrounds and systematics in the two surveys being mitigated in the cross-correlation between the two maps, thereby increasing the significance of the detection. The first detections of HI in intensity mapping took place using this approach \cite{Chang:2010jp, switzer13, masui13} with the HI observations from the Green Bank Telescope cross-correlated with galaxy data from the WiggleZ Dark Energy Survey probing $z \sim 1$, and more recently with galaxies from the eBOSS survey \cite{wolz2021}. Also, the cross-correlation of \HI\ intensity maps obtained from the Parkes telescope with 2dF galaxy Redshift Survey has recently been conducted at a lower redshift, $z \sim 0.08$ \cite{anderson2018}. 

For a cross-correlation survey of \HI\ and galaxies, the observable angular power spectrum is modified from \eq{cllimber} to read (e.g., \cite{hpcrosscorr2020}):
\begin{equation}
C_{\ell, \times} = \frac{1}{c} \int dz \frac{{W_{\rm HI}(z) W_{\rm gal}(z)} 
H(z)}{R(z)^2} 
(P_{\rm HI} P_{\rm gal})^{1/2}
\label{angcross}
\end{equation}
which is illustrated in the left panel of Fig. \ref{fig:contour1} for a Canadian Hydrogen Intensity Mapping Experiment (CHIME)-like survey covering the redshift range 0.8-2.5, cross-correlated with a Dark Energy Spectroscopic Instrument (DESI, \cite{desi2016})-like Emission Line Galaxy (ELG) galaxy survey over the range $z \sim 0.6-1.8$. The angular power spectrum above contains both the \HI\ and galaxy linear power spectra, as well as the corresponding window functions  ($W_{\rm gal}$ can in general be different from $W_{\rm HI}$, and depends on the details of the selection function of the galaxy survey, e.g.,\cite{smail1995}). Such a cross-correlation promises much more stringent constraints (shown in the right panel of Fig. \ref{fig:contour1}) on the astrophysical and cosmological parameters than auto-correlation (correlating galaxies with galaxies), as already shown in several recent analyses,  e.g., cross-correlating a CHIME-like and DESI-like survey leads to an improvement by  factors of a few in the cosmological and astrophysical constraints when compared to those from the CHIME-like survey alone (Ref.\cite{hpcrosscorr2020}, shown in Fig.\ref{fig:contour2}). Notably, this improvement occurs in spite of the fact that the redshift coverage of the cross-correlation is only about half that of the CHIME-like autocorrelation survey, so this illustrates the extent to which adding the galaxy survey information helps to improve the cosmological constraints. 
 It was shown \cite{shi2020} using the halo model framework  described in Sec. \ref{sec:analytical} that the broadband BAO feature measured from the angular cross-correlation power spectra between a DESI  Emission Line Galaxy (ELG) survey and the 21 cm intensity maps measured from the TianLai survey\footnote{http://tianlai.bao.ac.cn/}, can be used to forecast a constraint on the angular diameter distance with a precision of 2.7\% over $0.775 < z < 1.03$, which is complementary to the BAO cosmic distance measured by galaxy-galaxy auto-correlation.
 
 \begin{figure}
\centering
\includegraphics[width = \textwidth]{barchart_evolution_chime_desi_nomarg.pdf}
\caption{Relative accuracy on forecasted cosmological and astrophysical parameters with a CHIME-DESI-like survey cross-correlation covering redshifts $0.8 < z < 1.6$. \textit{Left panel:} Constraints on the cosmological parameters $\{h, \Omega_m, n_s, \Omega_b, \sigma_8\}$ in a flat $\Lambda$CDM framework, for (i) fixed values of astrophysical parameters (red) and (ii) marginalizing over astrophysical parameters (yellow). \textit{Right panel:} Constraints on the HI-halo mass parameters ($v_{c,0}$ and $\beta$ from Table \ref{table:constraints}, and the $Q$ parameter which denotes the large scale galaxy bias \cite{cole2005}) for (i) fixed values of cosmological parameters (green), (ii) marginalizing over the cosmological parameters but adding an astrophysical prior (cyan). The extent of the astrophysical prior from current data is plotted as the violet band in each case.
 Figure from \cite{hparaa2019}.}
\label{fig:contour2}
\end{figure}


In the sub-millimetre regime, the cross-correlation prospects for the COMAP (CO Mapping Array Project) survey and the photometric COSMOS and spectroscopic HETDEX Lyman-alpha fields \cite{chung2019} showed that about 0.3\% accuracy in redshifts with greater than 0.0001 sources per cubic Mpc, with spectroscopic redshift determination should enable a CO-galaxy cross spectrum detection significance at least twice that of the CO auto spectrum (even if the CO survey covers only a few square degrees). This illustrates that cross-correlations with galaxy surveys can make significant improvements to the goals of future intensity mapping experiments in the submillimetre regime. 


