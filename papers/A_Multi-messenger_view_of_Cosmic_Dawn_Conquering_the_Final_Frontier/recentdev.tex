
\subsection{Extensions to the halo model framework}
\label{sec:future}

\begin{figure}
\begin{center}
\includegraphics[width = 0.6\textwidth]{HIparkes_withrsd.pdf}
\caption{Power spectrum in redshift space computed using the halo model framework for HI corrected for the effects of redshift space distortions \eq{rs}. The points with error bars show the measurements of HI-galaxy cross power at $z \sim 0.082$, derived from the cross-correlation of Parkes and 2dF galaxy maps \cite{anderson2018}. The large scale HI bias is  assumed to be $b = 0.85$ with respect to the dark matter, consistently with the findings of \cite{martin12}.}
\end{center}
\label{fig:hiparkeswithrsd}
\end{figure}


Throughout the analysis so far, we have considered the halo model for clustering of dark matter and baryons in the absence of additional non-linear effects like redshift space distortions (RSDs), which originate from the peculiar motions of the gas. It is relatively straightforward to include these effects into the framework when one is interested in analysing real data. Redshift space distortions lead to two main effects on the halo model power spectrum: (i) a boost of power on large scales, due to streaming of matter into overdense regions, and (ii) a suppression of power on small scales due to virial motions within collapsed objects \cite{Kaiser:1987qv, peacock1994}. Both these effects serve to modify the power spectrum in \eq{onehalo} and \eq{twohalo}. By paralleling the approach developed in \cite{seljak2001, white2001} for the galaxy-halo connection, the resulting modifications for the case of gas (using \HI\ as an example) can be shown to have the form :


\begin{eqnarray}
& &P_{\rm HI}(k) = \left( F_{\rm HI}^2+{2\over 3}F_{\rm m}F_{\rm HI} + {1\over 5}F_{\rm m}^2\right)P_{\rm lin}(k)
   \\
& & + 
{1 \over  \bar{\rho_{\rm HI}}^2}
\int n(M) dM {M_{\rm HI}^2}
{\cal R}_2(k\sigma) |u_{\rm HI}(k,M)|^2,
\nonumber
\label{rs}
\end{eqnarray}
where 
\begin{eqnarray}
F_{\rm m}&=&f\int n(M) dM \ b(M) {\cal R}_1(k\sigma) u(k, M) 
\nonumber \\
F_{\rm HI}&=&
{{1}\over \bar{\rho_{\rm HI}}} \int
n(M) d M  M_{\rm HI} (M) b(M) {\cal R}_1(k\sigma)u_{\rm HI} (k,M)
\label{p0}
\end{eqnarray}
where $u(k,M)$ and $u_{\rm HI}(k,M)$ are the normalized Hankel transforms of the dark matter and \HI\ profiles, and
$f = d \log \delta/ d \log a \approx \Omega^{0.6}$ captures the effect of redshift space distortions.
The $\mathcal{R}$ terms are given by [with $\sigma = [GM/2R_v(M)]^{1/2}$ being the rms matter density fluctuation smoothed on the scale of the halo]:
\begin{equation}
  {\cal R}_1(k\sigma) = \sqrt{\pi\over 2} { {\rm erf}(k \sigma/\sqrt{2})\over k \sigma},
\label{r2}
\end{equation}
and
\begin{eqnarray}
  {\cal R}_2(y=k\sigma) &=& {\sqrt{\pi}\over 8} { {\rm erf}(y)\over y^5}
    \left[ 3f^2+4fy^2+4y^4 \right] \nonumber \nonumber \\
  &-& {e^{-y^2}\over 4y^4}\left[ f^2(3+2y^2)+4fy^2 \right]
\end{eqnarray}
\label{r2}
The  above redshift-space modifications can be applied to the \HI\ power spectrum to compare with the recent cross-correlation observations of \cite{anderson2018}, which finds some evidence for a low-amplitude clustering of HI relative to dark matter in the cross-correlation of HI maps from the Parkes survey cross-correlated with galaxies from the 2dF (Two degree Field) survey. The results are shown in Fig. \ref{fig:hiparkeswithrsd}. The observations are well matched to the model within the expected scatter, though there may be some evidence for more significant suppression on scales $k \sim 1$ h Mpc$^{-1}$.


In angular space, the nonlinear modelling of redshift space distortions is more complex and depends on the redshift binning under consideration \cite{jalilvand2020}.
    The treatment of RSDs in the Limber approximation on linear scales has recently been developed \cite{2019MNRAS.489.3385T}, while N-body and hydrodynamical simulations \cite{seehars2016, villaescusa2018} have also been used to model the effect of redshift space distortions on HI intensity maps. Just as in the case of galaxy RSDs, the HI RSDs can potentially constrain several interesting effects and break degeneracies. For example, \cite{hall2017} illustrate how peculiar velocity effects can be constrained 
using the dipole of the redshift space cross-correlation between 21 cm and 
optical surveys.  It is potentially interesting to analyse how redshift space distortions can affect the intensity mapping of submillimetre lines  \cite{chung2019pi} as well, though typically, the large scale effects need a factor $\sim 10$ times more area than available in current configurations, and the small scale effects are often suppressed by the finiteness of the beam in these experiments that wipe out scales of the order of $k \sim 1 - 10 h$ Mpc $^{-1}$ (e.g., Ref. \cite{schaan2021}, as can also be seen from \fig{fig:oiii}). This also makes it difficult to distinguish the 1-halo term from shot noise in these experiments. In the case of cosmological forecasts, these are not expected to make a significant difference to the scales of present interest within the scatter of the results.  



    


