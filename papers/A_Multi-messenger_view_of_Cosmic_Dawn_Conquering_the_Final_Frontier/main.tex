 
\documentclass{article}

\usepackage{graphicx}  % standard LaTeX graphics tool;
\usepackage{amsmath}   % For getting proper math eqns
\usepackage{amssymb}   % Used for getting various symbols eg. gtrsim, lesssim etc.
\usepackage{bm}         % Math bold - gives bold symbols for l.c. greeks as well


 
\usepackage[compress]{cite}
\usepackage[T1]{fontenc}                
\usepackage{lipsum}
\usepackage{hyperref}
\usepackage{mathrsfs}    


\usepackage{lscape}
\usepackage{longtable, supertabular}
\usepackage{times}


\usepackage{framed}
 

\addtolength{\textwidth}{1.6 in}
\addtolength{\oddsidemargin}{-.8in}
\addtolength{\evensidemargin}{-.8in}


\def\eq#1{{Eq.~(\ref{#1})}}
\def\eqs#1{{Eqs.~(\ref{#1})}}
\def\fig#1{{Fig.~\ref{#1}}}
\def\figs#1{{Figs.~\ref{#1}}}


\def\HI{{{\textrm{H}}I}}
\def\HII{{{\textrm{H}}~$\rm \scriptstyle II$}} 
\def\HeI{{{\textrm{He}}~$\rm \scriptstyle I$}}
\def\HeII{{{\textrm{He}}~$\rm \scriptstyle II$}}
\def\HeIII{{{\textrm{He}}~$\rm \scriptstyle III$}} 

\def\MgII{{{\textrm{Mg}}~$\rm \scriptstyle II$}}
\def\OVI{{{\textrm{O}}~$\rm \scriptstyle VI$}}

\def\SiIV{{{\textrm{Si}}~$\rm \scriptstyle IV$}}

\def\CIV{{{\textrm{C}}~$\rm \scriptstyle IV$}}

\def\ors{{\Omega_R(t_*)}}
\def\oms{{\Omega_m(t_*)}}
\def\ohs{{H(t_*)}}


\def\cn{{CosMIn}}

\def\cv{{CosMVel}}

\def\cc{{cosmological\ constant}}

\def\g{{\sqrt{-g}}}


\def\om{{\Omega_m}}

\def\oh{{H_0}}

\def\orr{{\Omega_R}}



\def\frab#1#2{\left(\frac{#1}{#2}\right)}      

\def\ket#1{|#1\rangle}                    %%%%   ket
\def\bra#1{\langle #1|}                   %%%%   bra
\def\bk#1#2#3{{\langle #1|#2|#3\rangle}}  %%%%   bracket
\def\amp#1#2{\langle #1 | #2\rangle}      %%%%   amplitude


  \title{\bf A Multi-messenger view of Cosmic Dawn:\\
\textit{Conquering the Final Frontier}}
\smallskip
\author{Hamsa Padmanabhan\footnote{\textit{Email:} hamsa.padmanabhan@unige.ch}\\
\\
Universit\'e de Gen\`eve, D\'epartement de Physique Th\'eorique,\\
24 quai Ernest-Ansermet, CH-1211 Gen\`eve 4, Switzerland\\
}

\date{}
  
 
 
 
 
  \begin{document}
 
  \maketitle
  
  \hrule
 
  \begin{abstract}
The epoch of Cosmic Dawn, when the first stars and galaxies were born, is widely considered the final frontier of observational cosmology today. Mapping the period between Cosmic Dawn and the present-day provides access to more than 90\% of the baryonic (normal) matter in the Universe, and unlocks several thousand times more Fourier modes of information than available in today's cosmological surveys. We review the progress in modelling baryonic gas observations as tracers of the cosmological large-scale structure from Cosmic Dawn to the present day. We illustrate how the description of dark matter haloes can be extended to describe baryonic gas abundances and clustering. This innovative approach allows us to fully utilize our current knowledge of astrophysics to constrain cosmological parameters from future observations. Combined with the information content of multi-messenger probes, this will also elucidate the properties of the first supermassive black holes at Cosmic Dawn. We present a host of fascinating implications for constraining physics beyond the $\Lambda$CDM model, including tests of the theories of inflation and the cosmological principle, the effects of non-standard dark matter, and possible deviations from Einstein's general relativity on the largest scales.
 \end{abstract}
 \bigskip

Keywords: intensity mapping  -- structure formation in the universe -- fundamental physics from cosmology


\bigskip

\hrule


\newpage
\tableofcontents
\newpage




% \leavevmode
% \\
% \\
% \\
% \\
% \\
\section{Introduction}
\label{introduction}

AutoML is the process by which machine learning models are built automatically for a new dataset. Given a dataset, AutoML systems perform a search over valid data transformations and learners, along with hyper-parameter optimization for each learner~\cite{VolcanoML}. Choosing the transformations and learners over which to search is our focus.
A significant number of systems mine from prior runs of pipelines over a set of datasets to choose transformers and learners that are effective with different types of datasets (e.g. \cite{NEURIPS2018_b59a51a3}, \cite{10.14778/3415478.3415542}, \cite{autosklearn}). Thus, they build a database by actually running different pipelines with a diverse set of datasets to estimate the accuracy of potential pipelines. Hence, they can be used to effectively reduce the search space. A new dataset, based on a set of features (meta-features) is then matched to this database to find the most plausible candidates for both learner selection and hyper-parameter tuning. This process of choosing starting points in the search space is called meta-learning for the cold start problem.  

Other meta-learning approaches include mining existing data science code and their associated datasets to learn from human expertise. The AL~\cite{al} system mined existing Kaggle notebooks using dynamic analysis, i.e., actually running the scripts, and showed that such a system has promise.  However, this meta-learning approach does not scale because it is onerous to execute a large number of pipeline scripts on datasets, preprocessing datasets is never trivial, and older scripts cease to run at all as software evolves. It is not surprising that AL therefore performed dynamic analysis on just nine datasets.

Our system, {\sysname}, provides a scalable meta-learning approach to leverage human expertise, using static analysis to mine pipelines from large repositories of scripts. Static analysis has the advantage of scaling to thousands or millions of scripts \cite{graph4code} easily, but lacks the performance data gathered by dynamic analysis. The {\sysname} meta-learning approach guides the learning process by a scalable dataset similarity search, based on dataset embeddings, to find the most similar datasets and the semantics of ML pipelines applied on them.  Many existing systems, such as Auto-Sklearn \cite{autosklearn} and AL \cite{al}, compute a set of meta-features for each dataset. We developed a deep neural network model to generate embeddings at the granularity of a dataset, e.g., a table or CSV file, to capture similarity at the level of an entire dataset rather than relying on a set of meta-features.
 
Because we use static analysis to capture the semantics of the meta-learning process, we have no mechanism to choose the \textbf{best} pipeline from many seen pipelines, unlike the dynamic execution case where one can rely on runtime to choose the best performing pipeline.  Observing that pipelines are basically workflow graphs, we use graph generator neural models to succinctly capture the statically-observed pipelines for a single dataset. In {\sysname}, we formulate learner selection as a graph generation problem to predict optimized pipelines based on pipelines seen in actual notebooks.

%. This formulation enables {\sysname} for effective pruning of the AutoML search space to predict optimized pipelines based on pipelines seen in actual notebooks.}
%We note that increasingly, state-of-the-art performance in AutoML systems is being generated by more complex pipelines such as Directed Acyclic Graphs (DAGs) \cite{piper} rather than the linear pipelines used in earlier systems.  
 
{\sysname} does learner and transformation selection, and hence is a component of an AutoML systems. To evaluate this component, we integrated it into two existing AutoML systems, FLAML \cite{flaml} and Auto-Sklearn \cite{autosklearn}.  
% We evaluate each system with and without {\sysname}.  
We chose FLAML because it does not yet have any meta-learning component for the cold start problem and instead allows user selection of learners and transformers. The authors of FLAML explicitly pointed to the fact that FLAML might benefit from a meta-learning component and pointed to it as a possibility for future work. For FLAML, if mining historical pipelines provides an advantage, we should improve its performance. We also picked Auto-Sklearn as it does have a learner selection component based on meta-features, as described earlier~\cite{autosklearn2}. For Auto-Sklearn, we should at least match performance if our static mining of pipelines can match their extensive database. For context, we also compared {\sysname} with the recent VolcanoML~\cite{VolcanoML}, which provides an efficient decomposition and execution strategy for the AutoML search space. In contrast, {\sysname} prunes the search space using our meta-learning model to perform hyperparameter optimization only for the most promising candidates. 

The contributions of this paper are the following:
\begin{itemize}
    \item Section ~\ref{sec:mining} defines a scalable meta-learning approach based on representation learning of mined ML pipeline semantics and datasets for over 100 datasets and ~11K Python scripts.  
    \newline
    \item Sections~\ref{sec:kgpipGen} formulates AutoML pipeline generation as a graph generation problem. {\sysname} predicts efficiently an optimized ML pipeline for an unseen dataset based on our meta-learning model.  To the best of our knowledge, {\sysname} is the first approach to formulate  AutoML pipeline generation in such a way.
    \newline
    \item Section~\ref{sec:eval} presents a comprehensive evaluation using a large collection of 121 datasets from major AutoML benchmarks and Kaggle. Our experimental results show that {\sysname} outperforms all existing AutoML systems and achieves state-of-the-art results on the majority of these datasets. {\sysname} significantly improves the performance of both FLAML and Auto-Sklearn in classification and regression tasks. We also outperformed AL in 75 out of 77 datasets and VolcanoML in 75  out of 121 datasets, including 44 datasets used only by VolcanoML~\cite{VolcanoML}.  On average, {\sysname} achieves scores that are statistically better than the means of all other systems. 
\end{itemize}


%This approach does not need to apply cleaning or transformation methods to handle different variances among datasets. Moreover, we do not need to deal with complex analysis, such as dynamic code analysis. Thus, our approach proved to be scalable, as discussed in Sections~\ref{sec:mining}.
\section{The halo model framework: from dark matter to baryons}\label{s:HIhalomodel}

As we have gathered, mapping the evolution of baryonic material across cosmic time using their line transitions promises deep insights into galaxy evolution, as well as theories of gravity and fundamental physics. The 21-cm line of neutral hydrogen, which is currently a useful probe of the \HI\ content of galaxies, is ideally suited for the bulk of the material, since hydrogen is the most abundant element in the Universe. 

Traditionally, galaxy surveys have been used as probes of the neutral hydrogen distribution \cite{zwaan05, zwaan2005a, martin12} at late times. The limits of current radio facilities, however, hamper the detection of 21-cm in emission from normal galaxies at early times. At these epochs, the HI distribution has been studied via the identification of Damped Lyman Alpha systems (DLAs, e.g., \cite{noterdaeme12, zafar2013, crighton2015}) — which are known to be the primary reservoirs of neutral hydrogen. 

A relatively new technique used to study HI evolution is known as intensity mapping. In this technique, the large-scale distribution of a tracer (like HI) can be mapped without individually resolving the galaxies which host the tracer. Being thus faster and less expensive than traditional galaxy surveys, intensity mapping of the 21 cm line had already been shown over the last two decades, e.g., \cite{chang10, wyithe2008a, anderson2018} to have the potential to provide constraints on cosmological parameters which are competitive with those from next generation experiments. 

Given the deluge of data expected from forthcoming radio observations, especially with the intensity mapping technique, it is therefore
timely and important to unify the data into a self-consistent framework that addresses both high- and low-redshift observations. Such a framework, which encapsulates the astrophysical information contained in HI (and, in general, baryonic gas tracers) into a set of key physical parameters, is crucial to properly take into account astrophysical uncertainties in order to place the most realistic constraints on Fundamental Physics and cosmology. Furthermore, it enables the most effective comparison to physical models of galaxy astrophysics, allowing us to unravel several outstanding questions in their evolution.

In a map of unresolved line emission over a large area in the sky, the main quantity of interest is the three-dimensional power spectrum, introduced for \HI\ in Sec. \ref{sec:21cmintro} as $P_{\rm HI} (k,z)$. This power spectrum is a measure of the strength of the intensity fluctuations as a function of wave number ($k$) at redshift $z$. To compute the power spectrum, one needs to establish a framework connecting dark matter and baryons. The most natural method to achieve this is via a data-driven halo model framework \cite{hpar2017}  for the evolution of baryons as tracers of large scale structure. This can be achieved by combining the different datasets available (e.g. \cite{hptrcar2015}), and bringing together the models available in the literature \cite{hptrcar2016} into a comprehensive picture. In so doing, it is found that there are two main ingredients required to connect the baryonic quantities (mass, luminosity) to the underlying dark matter halo model (described in Sec.\ref{sec:introcosmo}), namely:
(i) the average mass (or luminosity) associated with a dark matter halo of mass $M$ at redshift $z$, and (ii) if relevant, the distribution of the mass as a function of scale, $r$ from the centre of the dark matter halo. The latter quantity is relevant when the data constrains the detailed small-scale structure of the baryons, as for the case of \HI. We will now  describe this framework, closely paralleling the development of the dark matter halo model framework in Sec. \ref{sec:introcosmo}.

Given a prescription for populating the halos with baryons, e.g., $M_{\rm HI}(M)$ (for the case of \HI), defined as the average mass of \HI\ contained in a halo of mass $M$, we begin by computing the neutral hydrogen density, $\bar{\rho}_{\rm HI}(z)$ in terms of the dark matter mass function $n(M,z)$ as:
\begin{equation}
\bar{\rho}_{\rm HI}(z)= \int_{M_{\rm min}}^{\infty} dM n(M,z) M_{\rm HI}(M) \,
\end{equation}
in which $M_{\rm min}$ is an astrophysical parameter that quantifies the minimum mass of the dark matter halo that is able to host \HI. In practice, this minimum mass requirement is usually built-in to the $M_{\rm HI}(M)$ function itself, by using a suitable cutoff scale.
For quantifying large-scale clustering, we define the large-scale bias parameter $b_{\rm HI} (z)$, that describes how strongly \HI\ is clustered relative to the dark matter,  as:
\begin{equation}
b_{\rm HI} (z) = \frac{1}{\bar{\rho}_{\rm HI}(z)}\int_{M_{\rm min}}^{\infty} dM
n(M,z) b(M) M_{\rm HI}(M).
\label{biasHI}
\end{equation}
To describe small-scale clustering, we define the normalized Hankel transform of the density profile $\rho_{\rm HI} (r)$:
 \begin{equation}
 u_{\rm HI}(k|M) = \frac{4 \pi}{M_{\rm HI} (M)} \int_0^{R_v} \rho_{\rm HI}(r) \frac{\sin kr}{kr} r^2 \ dr
\end{equation}
analogously to \eq{udm} for dark matter. The 
normalization of $u(k|M)$ above is to the total \HI{} mass, i.e. $M_{\rm HI} (M)$. In most of the standard analyses, the baryonic profile is assumed truncated at the halo virial radius, which is taken to be a standard `scale' cutting off the halo. The corresponding scale radius for the \HI\ is defined through $r_s = R_v(M)/c_{\rm HI}$ where $c_{\rm HI}$ is known as the  concentration parameter, analogous to the corresponding one for dark matter defined in \eq{concparamdm}. \footnote{Profiles like the NFW diverge as $\rho(r) \sim r^{-3}$ as $r \to \infty$, causing a logarithmic divergence of the mass contained within them. Thus, it is helpful to truncate the profile at the virial radius which defines the `boundary' enclosing the total mass, just as done for the dark matter case. Recent studies advocate the use of the \textit{splashback} radius, at which the accreted material reaches its first apocenter after turnaround \cite{diemer2018} and is separated from the virial radius by factors of a few, as a more precise definition of the halo boundary.}


The dark matter framework is then followed to compute the one- and two halo terms of the \HI{} power spectrum, given by:
\begin{equation}
P_{\rm 1h, HI} =  \frac{1}{\bar{\rho}_{\rm HI}^2} \int dM \  n(M) \ M_{\rm HI}^2 \ |u_{\rm HI} (k|M)|^2
\label{onehalo}
\end{equation}
and
\begin{equation}
P_{\rm 2h, HI} =  P_{\rm lin} (k) \left[\frac{1}{\bar{\rho}_{\rm HI}} \int dM \  n(M) \ M_{\rm HI} (M) \ b (M) \ |u_{\rm HI} (k|M)| \right]^2
\label{twohalo}
\end{equation}
which are analogous to \eq{powerspecdm}.
Some studies in the literature \cite{wolz2019, villaescusa2018} additionally model the shot noise of the halo power spectrum, $P_{\rm SN}$, which may be computed as:

\begin{equation}
   P_{\rm SN} =  \frac{1}{\bar{\rho}_{\rm HI}^2} \int dM \  n(M) \ M_{\rm HI}^2 \ 
   \label{shotnoiseHI}
\end{equation}
and corresponds to the $k \to 0$ limit of the one-halo power spectrum. This term provides a rough measure of the  Poissonian noise due to the finite number of discrete sources (such as HI-bearing galaxies).
In practice, however, in the redshift range considered in the context of \HI\ intensity
mapping, the shot noise contribution is expected to be negligible, e.g., \cite{seo2010, seehars2016} and relatively unconstrained by observations, so it is not relevant in a data-driven approach. It will, however, become important in the context of submillimetre observations described in the next section.

In all of the above, the dark matter mass function $n(M, z)$ can be computed from analytical arguments or simulations. In most of our analyses in the following sections, we will assume the Sheth - Tormen form defined in Sec. \ref{sec:astrocosmo}, though various other forms are possible \cite{tinker2008, shirasaki2021}.
 
Finally, the neutral hydrogen fraction, required for calculating the brightness temperature as described in Sec. \ref{sec:21cmintro}, is computed as:
\begin{equation}
 \Omega_{\rm HI} (z) = \frac{\bar{\rho}_{\rm HI}(z)}{\rho_{c,0}}
 \label{omegaHIanalyt}
\end{equation} 
where $\rho_{c,0} \equiv 3 H_0^2/8 \pi G$ is the critical density of the Universe at redshift 0.



In several recent observations, the real-space correlation function is measured, which is defined as the Fourier transform of the power spectrum ($P_{\rm HI, 1h} + P_{\rm HI, 2h})$. It is computed as:
\begin{equation}
\xi_{\rm HI} (r) = \frac{1}{2 \pi ^2} \int k^3 (P_{\rm 1h, HI} + P_{\rm 2h, HI}) \frac{\sin kr}{kr} \frac{dk}{k}
\end{equation}

The above discussion completes the generic treatment for analysing observations of baryonic gas in a halo model framework, given the tracer-to-halo mass relation and its small-scale profile. Note that the above framework is most naturally suited to describe intensity mapping observations, since it is a mass- (or luminosity) weighted measurement analogous to the approach for dark matter, similar to that employed in studies of the cosmic infrared background, e.g., \cite{fernandez2006}. This is different from, e.g., a number-counts weighted approach which is commonly used in surveys of discrete tracers like galaxies, e.g., \cite{cooraysheth2002} with a halo occupation distribution (HOD). While number-count- and mass-count-weighted frameworks are completely equivalent in their capture of small-scale effects (via the `satellite' galaxies distribution and the 1-halo term respectively), it is more natural to describe intensity mapping observations using a mass-weighted approach, since we are dealing here with diffuse gas not all of which may be located inside galaxies and as such does not form a discrete set. This also facilitates the extension of this parameterization to the reionization regime and beyond. Quantifying the gas content in discrete systems (such as galaxies and Damped Lyman-Alpha systems) is made possible due to the flexibility in mass and cutoff scales in the framework, as we show below by describing how specific expressions that arise in the context of \HI\ observations from galaxy surveys and DLAs can be modelled with the above ingredients.



\subsection{21 cm from galaxy surveys}
\label{sec:21cmgal}

The main observable in a survey of galaxies detected in 21 cm is their \textit{mass function}, which measures the volume density of \HI-selected galaxies as a function of the \HI\ mass in different mass bins. Denoted by $\phi(M_{\rm HI})$, it is typically found to follow a functional form of the type \cite{martin10, zwaan05}:
\begin{equation}
 \phi(M_{\rm HI}) \equiv \frac{dn}{d \log_{10} M_{\rm HI}} =  \frac{\phi_*}{M_0} \ \left(\frac{M_{\rm HI}}{M_0} \right)^{-a} e^{-M_{\rm HI}/M_0}
\end{equation} 
where $\phi_*, a$ and $M_0$ are free parameters.
Given the HI mass - halo mass relation $M_{\rm HI}(M,z)$, the above mass function can be modelled by:
\begin{equation}
    \phi(M_{\rm HI}) = \frac{dn}{d \log_{10} M} \left|\frac{d \log_{10} M}{d \log_{10} M_{\rm HI}}\right|
    \label{phifrommhi}
\end{equation}
with the first term calculated from $n(M,z)$.



The \HI\ mass function is then used to calculate the mass density of \HI\ in galaxies:
\begin{equation}
 \rho_{\rm HI} = \int M_{\rm HI} \phi(M_{\rm HI}) d M_{\rm HI} = M_0 \phi_* \Gamma(2 - a) 
\end{equation}
from which the density parameter of \HI\ in galaxies, $\Omega_{\rm HI, gal} = $ can be calculated:
\begin{equation}
  \Omega_{\rm HI, gal} =  \rho_{\rm HI}/\rho_{c,0}
\end{equation}



The clustering of HI-selected galaxies can generally be computed following \eq{biasHI}, with the modification that the lower limits in the integrals in \eq{biasHI} are fixed to the halo mass $M_{\rm min}$ corresponding to the minimum \HI\ mass observable by the survey. Thus, the \HI\ bias measured from galaxy surveys is modelled by:
\begin{equation}
b_{\rm HI,gal} (z) = \frac{\int_{M_{\rm min}}^{\infty} dM \ n(M,z)\ b (M,z) \ M_{\rm HI} (M,z)}{\int_{M_{\rm min}}^{\infty} dM \  n(M,z) \ M_{\rm HI} (M,z)}
\label{biasHIgal}
\end{equation}

We note in passing that, given the \HI\ mass function and reversing the procedure described in \eq{phifrommhi}, commonly known as the \textit{abundance matching} technique in the literature, we can directly estimate $M_{\rm HI}(M)$ from observational data. This makes the assumption that $M_{\rm HI}(M)$ is a monotonic function of $M$, and is given by solving the equation \cite{vale2004}:
\begin{equation}
    \int_{M(M_{\rm HI})}^{\infty} \frac{dn}{d \log_{10} M'} d \log_{10} M' = \int_{M_{\rm HI}}^{\infty} \phi(M_{\rm HI'}) d \log_{10} M_{\rm HI}'
    \label{abmatchhi}
\end{equation}
and an appropriate functional form is fitted to the resulting datapoints $\{M_{\rm HI}, M\}$. Such an approach, for the case of \HI\ and using the mass functions from recent literature \cite{martin10, zwaan05} is found to lead to results which are consistent \cite{hpgk2017} with the direct fitting to observations described below.



\subsection{Neutral hydrogen  from Damped Lyman Alpha systems}
\label{sec:dlahimodels}

As we have seen in Sec. \ref{sec:astrocosmo}, Damped Lyman Alpha systems (DLAs) represent the highest column density \HI-bearing systems in the IGM.
In a survey measuring the neutral hydrogen fraction using DLAs, the primary observable is called the \textit{column density distribution function} defined in \eq{coldens} denoted by $f_{\rm HI} (N_{\rm HI})$, where $N_{\rm HI}$ is the column density of the DLAs:
\begin{equation}
 d^2 \mathcal{N} = f_{\rm HI} (N_{\rm HI}, X) dN dX
\end{equation} 
with $\mathcal{N}$ being the observed incidence rate of DLAs in the absorption interval $dX$ and the column density range $dN_{\rm HI}$. 
To model the column density distribution using the framework described in the previous section, the hydrogen density profile $\rho_{\rm HI}(r)$ as a function of $r$ is first used to calculate the column density of DLAs using the relation:
\begin{equation}
 N_{\rm HI}(s) = \frac{2}{m_H} \int_0^{\sqrt{R_v(M)^2 - s^2}} \rho_{\rm HI} (r = \sqrt{s^2 + l^2}) \ dl 
 \label{coldenss}
\end{equation} 
where $s$ is the impact parameter of a line-of-sight through the DLA. 

The cross-section for a system to be identified as a DLA, $\sigma_{\rm DLA}$ can then be computed by 
\begin{equation}
    \sigma_{\rm DLA} = \pi s_*^2
\end{equation}
where $s_*$ is the root\footnote{If no positive root $s^*$ exists, it physically means that the column density in the line-of-sight does not reach $N_{\rm HI} = 10^{20.3}$ cm$^{-2}$ even at zero impact parameter, so the cross-section is zero for such systems. Hence, in such cases, $s ^{*}$ is set to zero.}  of the equation $N_{\rm HI}(s_*) = 10^{20.3}$ cm$^{-2}$ (which is the column density threshold for the appearance of DLAs). 

Given the DLA cross-section, 
the column density distribution $f_{\rm HI}(N_{\rm HI}, z)$ is modelled by:
\begin{equation}
 f(N_{\rm HI}, z) = \frac{c}{H_0} \int_0^{\infty} n(M,z) \left|\frac{d \sigma}{d N_{\rm HI}} (M,z) \right| \ dM 
 \label{coldensdef}
\end{equation} 
where the $d \sigma/d N_{\rm HI} =  2 \pi \ s \ ds/d N_{\rm HI}$, with $N_{\rm HI} (s)$ defined as in \eq{coldenss}.

The clustering of DLAs is captured by the DLA bias, $b_{\rm DLA}$  defined by:
\begin{equation}
 b_{\rm DLA} (z) =  \frac{\int_{0}^{\infty} dM n (M,z) b(M,z) \sigma_{\rm DLA} (M,z)}{\int_{0}^{\infty} dM n (M,z) \sigma_{\rm DLA} (M,z)}.
 \label{bdla}
\end{equation} 
Note that the above expression is almost identical to \eq{biasHI}, with the only difference being the weighting of the bias by the cross-section of DLA absorbers.

Another observable is the incidence of the DLAs, denoted by $dN/dX$ which quantifies the number of systems per absorption path length, and is calculated as:
\begin{equation}
 \frac{dN}{dX} = \frac{c}{H_0} \int_0^{\infty} n(M,z) \sigma_{\rm DLA}(M,z) \ dM
 \label{dndxdef}
\end{equation} 
Finally, from the column density distribution, the density parameter of hydrogen in DLAs can be calculated as:
\begin{equation}
 \Omega_{\rm HI}^{\rm DLA} = \frac{m_H H_0}{c \rho_{c,0}} \int_{N_{\rm HI, min}}^{\infty} N_{\rm HI} f_{\rm HI} (N_{\rm HI}, X) dN_{\rm HI} dX \, ,
\end{equation} 
in which the lower limit\footnote{The lower limit changes to $10^{19}$ cm$^{-2}$, \cite{zafar2013} in case the sub-DLAs too are accounted for while calculating the gas density parameter. Lower column-density systems, such as the Lyman-$\alpha$ forest,  make negligible contributions to the total gas density.} of the integral is set by the column density threshold for DLAs, i.e. $N_{\rm HI, min} = 10^{20.3}$ cm$^{-2}$. 



In alternate approaches to modelling DLAs \cite{villaescusa2018}, the cross section $\sigma_{\rm DLA} (M)$ itself may directly be modelled using a functional form, and the DLA quantities calculated from $\sigma_{\rm DLA}$. Various simulation-based approaches have also been used to quantify the neutral hydrogen at different redshifts from DLAs \cite{pontzen2008, bird2014}.


\subsection{HI-halo mass relations and density profiles}
\label{sec:analytical}
We have seen in the above discussion that the two key inputs needed in the halo model framework for hydrogen are $M_{\rm HI}(M)$, the prescription for
assigning \HI\ to the dark matter haloes, and $\rho_{\rm HI}(r)$, which describes how this mass is distributed as a function of scale.  

Several forms had been used to model the $M_{\rm HI}(M)$ function in the past literature. At redshifts probed chiefly by DLA observations, $z \sim 2-5$, a relation between HI mass and halo mass of the form 
$M_{\rm HI} = \alpha M \exp(-M/M_0)$
where $\alpha$ is a constant of normalization, and $M_0$ is a lower mass cutoff was commonly used\cite{barnes2014}. 
At lower redshifts, various forms had been proposed, among which are $M_{\rm HI} = f M$ which is a constant fraction of the halo mass between a fixed lower limit in halo mass, $M_{\rm min}$ and upper limit $M_{\rm max}$ \cite{bagla2010}. 

Reconciling the high- and low-redshift approaches \cite{hptrcar2016} leads to some fascinating insights about the occupation of \HI\ in dark matter haloes. Specifically, it is found that in order to match all the current observational data (DLAs, 21 cm galaxy surveys and intensity mapping observations) over $z \sim 0-5$, the connection between HI mass and halo mass needs to be modelled using a function of the form:
\begin{eqnarray}
M_{\rm HI} (M) &=& \alpha f_{\rm H,c} M \left(\frac{M}{10^{11} h^{-1} M_{\odot}}\right)^{\beta} \exp\left[-\left(\frac{v_{c0}}{v_c(M)}\right)^3\right] \nonumber \\
\end{eqnarray}
We now describe the form of this expression in some detail. 
It involves three free parameters, $\alpha$, $\beta$ and $v_{c,0}$:

(i) $\alpha$ is the  overall normalization of the HI-halo mass relation. Physically, it represents the fraction of HI, relative to the cosmic fraction ($f_{\rm H,c}$) that resides in a halo of mass $M$ at redshift $z$. The cosmic fraction is the primordial hydrogen fraction by mass, defined as:
\begin{equation}
f_{\rm H,c} = (1 - Y_{\rm He}) \Omega_b/\Omega_m \, ,
\label{fhc}
\end{equation}
in which $Y_{\rm He} = 0.24$ is the primordial helium fraction.

(ii) $\beta$ is the logarithmic slope of \HI\ mass to halo mass. Any nonzero value of $\beta$ describes a  departure from proportionality of the HI mass and halo mass. The $\beta$ is physically connected to physical processes that deplete \HI\ in galaxies (such as quenching and feedback from the intergalactic medium).

(iii) The parameter $v_{\rm c, 0} (M)$  represents a lower cutoff to the HI-halo mass relation. It describes the minimum mass (or equivalent circular velocity) of a halo able to host neutral hydrogen. In the above equation, the circular velocity is calculated through:
\begin{equation}
    v_{c} = \sqrt\frac{GM}{R_v(M)}
    \label{vcRv}
\end{equation}
where $R_v(M)$ is the virial radius defined in \eq{virialradius}. 
\footnote{An equivalent representation of the lower cutoff is to use a minimum halo mass ($M_{\rm min} (z)$) in place of the circular velocity. Since we see from \eq{vcRv} and \eq{virialradius} that $v_c \propto M^{1/3}(1+z)^{1/2}$, the exponential term  can be written as  $\exp(-M/M_{\rm min} (z))$ if we are using a mass cutoff. However, using the circular velocity -- instead of halo mass  -- to denote the cutoff makes the physical connection to the intergalactic ionizing background more natural, as we will see in the next section.}

Just like the HI-halo mass relation, it is also important to parametrize the profile of the neutral hydrogen in the dark matter halo, $\rho_{\rm HI}(r)$. In the literature, this function had been modelled by an altered version of the NFW profile introduced in Sec. \ref{sec:astrocosmo} \cite{maller2004, barnes2014, hpar2017}:
\begin{equation}
\rho_{\rm HI} (r) = \frac{\rho_0 r_{\rm s, HI}^3}{(r + 0.75 r_{\rm s, HI}) (r+r_{\rm s, HI})^2}
\label{rhodefnfw}
\end{equation}
The quantity $r_{\rm s,HI}$ is the scale radius of \HI, which is defined as $r_{\rm s, HI} \equiv R_v(M)/c_{\rm HI}$, introducing a concentration parameter $c_{\rm HI}$ analogous to the one for dark matter in \eq{concparamdm}:
\begin{equation}
    c_{\rm HI}(M,z) =  c_{\rm HI, 0} \left(\frac{M}{10^{11} M_{\odot}} \right)^{-0.109} \frac{4}{(1+z)^{\gamma}}
    \label{concparamhi}
\end{equation}
We thus see that the profile function brings two additional parameters to the model: $c_{\rm HI, 0}$ and $\gamma$. Of these, $c_{\rm HI}$ is the overall normalization of the concentration, and $\gamma$ describes how it evolves with redshift. \footnote{Note that the profile has a negligible halo mass dependence (the power of -0.109). This is inherited from the dark matter framework\cite{duffy}  and does not have any significant bearing on the modelling.}
Recent data, especially from HI-rich disk galaxies \cite{bigiel2012} at low redshifts, favours the profile for \HI\ in haloes being of the exponential form (rather than the modified NFW relation above):
\begin{equation}
    \rho(r,M) = \rho_0 \exp(-r/r_{\rm s, HI})
\label{rhodefexp}
\end{equation}
In both forms of the profile, the parameter $\rho_0$ is fixed by normalization to the total HI mass, $M_{\rm HI}(M)$, just as done for  the case of the dark matter halo model.
The profile and the HI-halo mass relation can now be used to compute the 1- and 2-halo terms of the power spectrum defined in \eq{onehalo} and \eq{twohalo} respectively. For this, it is necessary to compute the normalized Hankel transform of the profile function, which can be expressed analytically for both the profile choices above \cite{hparaa2017}.
As an explicit example, for  the \HI\ profile in the exponential form, the  expression for the normalized Hankel transform is given by:
\begin{equation}
u_{\rm HI}(k|M) = \frac{4 \pi \rho_0 r_{\rm s, HI}^3 u_1(k|M)}{M_{\rm HI} (M)}
\end{equation}
where
\begin{equation}
u_1(k|M) = \frac{2}{(1 + k^2 r_{\rm s, HI}^2)^2},
\end{equation}
which can then be used to compute the full power spectrum.

The five parameters  $\{c_{\rm HI, 0}, \alpha, \beta, v_{c, 0}, \gamma\}$ can now be fixed by fitting the framework above to the current astrophysical data describing \HI.  This is achieved using a Markov Chain Monte Carlo (MCMC) approach \cite{hparaa2017}  and the resulting best-fitting parameters and their uncertainties are summarized in the first column of Table \ref{table:constraints}. There are a few salient points that emerge from the analysis:
\begin{enumerate}
    \item The best fitting value of $\alpha$ is $\alpha = 0.09$. This denotes that about $10\%$ of the hydrogen is in atomic form over the post-reionization Universe, and is in line with recent simulations and observations that predict the fraction of cold gas (i.e. the sum of the atomic and molecular components) to be in the range of $\sim 10-20$\% in low-redshift galaxies\cite{stern2016}. Interestingly, the observations do not favour an evolution in $\alpha$ with redshift. This in in line with the findings from DLA studies \cite{prochaska09}, which indicate evidence for non-evolution of hydrogen in galaxies over $z \sim 0-5$, and reiterates the role of HI as an `intermediary' in the baryon cycle \cite{wang2020, bouche2010, lilly2013, hploebsfr2020}: the HI replenishment from the IGM is compensated by its conversion to molecular hydrogen, $H_2$ which is used up by star formation.
    
    \item The slope, $\beta$ is found to have the (negative) value $\beta = -0.58$. It is found that the slope is dominantly influenced by the form of the HI mass function observed at low redshifts for which exquisite constraints are available \cite{zwaan05, martin10}. The suppression of the resultant HI-halo mass slope from unity is in line with evidence for quenching, or the suppression of star formation in massive haloes due to feedback from the IGM \cite{birnboim2007, finlator2017}.
    
    \item The cutoff $v_{\rm c,0}$ has the value 36.3 km/s. This can be directly related to the circular speed of suppression of dwarf galaxies by the ambient ionizing background in the IGM, which was worked out in the mid-80s and 90s from analytical arguments balancing ionization and recombination \cite{rees1986, efstathiou1992, quinn1996}. Fascinatingly, such an analysis favours a value of $v_c \sim 37$ km/s, almost perfectly matched to the value obtained by combining the latest ($\sim$ 2017)  \HI\ observations today! This greatly strengthens the case for using circular velocity as a lower cutoff for \HI\ in haloes, and sheds light on the physics associated with this parameter. 
    

\end{enumerate}
    




\subsection{The sub-millimetre regime}
\label{s:submmhalomodel}

Although hydrogen is the most abundant element in the Universe, there are several exciting prospects for making intensity maps of other salient lines, an important example being molecular lines, like the carbon monoxide (CO) lines introduced in Sec. \ref{sec:submmintro}. CO is the second most abundant molecule in the universe (after molecular hydrogen) and much easier to detect from ground-based experiments. CO behaves as a tracer of molecular hydrogen, which has no permanent dipole moment due to symmetry and thus no rotational transitions of its own.  CO lines are thus the primary way to trace molecular gas within and outside our galaxy, and very sensitive to the spatial distribution of star formation \cite{hploebsfr2020}. The CO line corresponding to the transition between rotational quantum numbers $J$ and $J - 1$ has a rest frequency of approximately $J \times$115.27 GHz, making it an ideal target in the sub millimetre regime. It is easy to separate the signal from contaminants due to its multiple emission lines (with different values of the angular momentum quantum number $J$) which have a  well defined frequency relationship, a feature that is not available to other tracers. This also enables effective cross-correlation between observations at lower and higher frequency bands which are integral multiples of each other (e.g., a  frequency band covering 26-34 GHz will be sensitive to the CO 2-1 line at $z \sim$ 6-8, while also capturing the CO 1-0 transition at $z \sim$ 2-3.) 



The CO Mapping Array Project (COMAP) aims to detect the CO molecule in emission during the epoch of Galaxy Assembly, about two billion years after the Big Bang. Science observations with the COMAP Pathfinder began in 2019 using a 19-pixel 26-34 GHz receiver mounted on a 10.4 metre dish at the  Owens Valley Radio Observatory.  It was shown \cite{li2015} using simulations, that the COMAP experiment is capable of providing close to 8$\sigma$ constraints on the CO intensity power spectrum at large scales. In November 2021, the COMAP Pathfinder released the first direct 3D measurement of the CO power spectrum on large scales \cite{cleary2021}, nearly an
              order of magnitude improvement compared to the previous
              best measurement \cite{keating2020}. 

Observations made using the Karl G. Jansky Very Large Array (JVLA) and the Atacama Large Millimeter Array (ALMA) have recently shown that line emission from the CO transitions will be bright even at high redshift, $z > 6$.  The levels of foreground contamination in a CO survey are also much lower than for many other types of line intensity mapping, making it a promising target for ground-based observations. Recently, the CO Power Spectrum Survey [COPSS; Ref. \cite{keating2016}] detected, for the first time, the aggregate CO intensity in emission from galaxies at the peak of the star formation history of the universe. The mmIME experiment recently \cite{keating2020} announced the detection of unresolved intensity from the aggregate CO (3-2) emission in galaxies at $z \sim 2.5$ in the shot noise regime. In \cite{pullen2018}, confirmed by \cite{yang2019}, there was a tentative detection of the 158 micron line of [CII], an excellent tracer of star formation, reported, for the first time, by combining Planck CMB maps with  quasars from the BOSS and CMASS galaxy surveys.




By developing a framework that can incorporate current observational constraints on the abundances and clustering of the tracers, we can readily use the wealth of upcoming submillimetre observations to constrain galaxy evolution. Similar to the case of \HI, the main observable in the case of submillimetre intensity mapping observations is the power spectrum. However, in contrast to the \HI\ case, the luminosity of the tracer is normally used instead of the mass, so the prescription to be modelled is, e.g.,  $L_{\rm CO} (M,z)$ in the case of CO. Another difference is that the modelling of sub-millimetre intensity mapping usually focuses on the linear regime alone (since the data typically does not constrain the behaviour of the profile of the tracer). Instead of modelling the  full one-halo term, what is typically modelled is the contribution from the \textit{shot noise} alone, which is analogous to the term introduced briefly in the previous section in \eq{shotnoiseHI}.



To compute the power spectrum, we use the specific intensity of a submillimetre line  observed at a frequency, $\nu_{\rm obs}$, given by:
\begin{equation}
 I(\nu_{\rm obs}) = \frac{c}{4 \pi} \int_0^{\infty} dz' \frac{\epsilon[\nu_{\rm obs} (1 + z')]}{H(z') (1 + z')^4}
\end{equation} 
in which $\epsilon[\nu_{\rm obs} (1 + z')]$ is known as the volume emissivity of the emitted line.
It is usually assumed that the profile of each line is a delta function \footnote{The effects of line broadening on the intensity and power spectra are analytically treated in Ref. \cite{chung2021lb}.} at the rest frequency $\nu_{\rm em}$. This implies that the emissivity can be expressed as an integral of the host halo mass $M$:
\begin{equation}
 \epsilon(\nu, z) = \delta_D(\nu - \nu_{\rm em}) (1 + z)^3 f_{\rm duty} \int_{M_{\rm min}}^{\infty} dM \frac{dn}{dM} L(M,z)
 \label{emissivity}
\end{equation} 
where $L (M,z)$ is the luminosity of the line under consideration, and it is assumed that a fraction $f_{\rm duty}$ (usually called the `duty-cycle' factor) of all haloes above a mass $M_{\rm min}$ contribute to the observed emission  in the line of interest, e.g., \cite{lidz2011}. This parameter can be approximated by $f_{\rm duty} = t_s/t_H$, where $t_s$ is the star formation timescale and $t_H$ is the Hubble time at the redshift under consideration. 
It is to be noted that both of these parameters are fairly poorly constrained by the data;  $f_{\rm duty}$ in particular is found to differ by more than an order of magnitude between different models \cite{pullen2013, keating2016}. For this reason, $f_{\rm duty}$ is alternatively taken into account by introducing intrinsic scatter parameters \cite{li2015} to account for the differences in the star formation activity of haloes.

Using \eq{emissivity}, the specific intensity can be rewritten as:
\begin{equation}
I(\nu_{\rm obs}) = \frac{c}{4 \pi} \frac{1}{\nu_{\rm em} H(z_{\rm em})}  f_{\rm duty} \int_{M_{\rm min}}^{\infty} dM \frac{dn}{dM} L(M,z)
\label{COspint}
\end{equation} 
where $z_{\rm em}$ is the redshift of the emitting source.
Just as in the case of \HI,
the brightness temperature, $T$  can be derived from the specific intensity through the relation $I(\nu_{\rm obs}) = 2 k_B \nu_{\rm obs}^2 T /c^2$. 
From this, the expression for the brightness temperature becomes:
\begin{equation}
T(z) = \frac{c^3}{8 \pi}\frac{(1 + z_{\rm em})^2}{k_B \nu_J^3 H(z_{\rm em})} f_{\rm duty} \int_{M_{\rm min}}^{\infty} dM \frac{dn}{dM} L(M,z)
\label{tco}
\end{equation} 
The clustering of the submillimetre sources  can be modelled in analogy with the \HI\ case by weighting the dark matter halo bias by the tracer luminosity-halo mass relation.
The expression for the clustering is therefore given by:
\begin{equation}
 b_{\rm submm}(z) = \frac{\int_{M_{\rm min}}^{\infty} dM (dn/dM) L (M,z) b(M,z)}{\int_{M_{\rm min}}^{\infty} dM (dn/dM) L (M,z)}
\end{equation} 
in which we see that $L(M,z)$ has now taken the place of $M_{\rm HI}(M,z)$ in \eq{biasHI}.

The shot noise contribution to the power is expressed as:
\begin{equation}
 P_{\rm shot}(z) = \frac{1}{f_{\rm duty}}\frac{\int_{M_{\rm min}}^{\infty} dM (dn/dM) L (M,z)^2}{\left(\int_{M_{\rm min}}^{\infty} dM (dn/dM) L (M,z)\right)^2}
\end{equation} 
The total power spectrum of the intensity fluctuations is the sum of the clustering (two-halo) and shot-noise components:
\begin{equation}
 P_{\rm submm}(k,z) =  T (z)^2 [b_{\rm submm}(z)^2 P_{\rm lin}(k,z) + P_{\rm shot}(z)]
 \label{submmpower}
\end{equation} 
in which $P_{\rm lin}(z)$ denotes the dark matter power spectrum calculated in linear theory. \footnote{\eq{submmpower} assumes that the power is measured in units of ${\rm K}^2$, alternatively, it can directly be measured in units of Jy/sr$^2$ (as commonly done for the cases of [CII] and [OIII]) in which case the intensity in \eq{COspint} is  used: $P_{\rm submm}(k,z) = I(\nu_{\rm obs})^2 b_{\rm submm}^2(z) P_{\rm lin}(k,z) + P_{\rm shot}(z)$.}
Several times,  we need to plot the power spectrum in logarithmic $k$-bins, which is given by:
\begin{equation}
 \Delta_{k}^2(z) = \frac{k^3  P_{\rm HI/submm}(k,z)^2}{2 \pi^2}
 \label{COpowspeclog}
\end{equation} 
 which is equally valid for both the HI power spectrum in \eq{onehalo} and \eq{twohalo}, and the submillimetre one in \eq{submmpower}.
 
To model the main observable for submillimetre intensity mapping, the abundance matching technique (introduced at the end of Sec. \ref{sec:21cmgal}) is found to be suitable for both CO and [CII] at $z \sim 0$, due to the availability of the luminosity functions at low redshifts for both these tracers \cite{hemmati2017, keres2003}, which is denoted by $\phi(L)$ and measures the number density of CO- or CII-luminous galaxies in logarithmic luminosity bins.\footnote{The data do not constrain a non-monotonic behaviour of the luminosity-halo mass relations, so this is a reasonable approach.} Specifically, \eq{abmatchhi} gets modified to:
\begin{equation}
    \int_{M(L)}^{\infty} \frac{dn}{d \log_{10} M'} d \log_{10} M' = \int_{L}^{\infty} \phi(L') d \log_{10} L'
    \label{abmatchsubmm}
\end{equation}
to find $L(M,z = 0)$, which is then inserted into the framework above and propagated to high redshifts using an appropriate functional form, whose parameters are matched to the observations. 

The CO luminosity from galaxy surveys is usually measured in units of K km/s pc$^2$. This quantity is related to  the observed flux density of CO, and its linewidth, by the relation \cite{solomon2006}:
\begin{equation}
 L_{\rm CO} = 3.3 \times 10^{13} S \Delta v (1 + z)^{-3} \nu_{\rm obs}^{-2} D_L^2 \ \rm{K \  km/s \  pc}^2
\end{equation} 
with $D_L$ being the luminosity distance to the source in Gpc, $\nu_{\rm obs}$ the observed frequency in GHz, $S$ the flux density  in Jy, and $\Delta v$ the velocity width  in km/s. For the case of CO, a double power law form relating $L_{\rm CO}$ to halo mass is found to be a good fit to the data at low and high redshifts:
\begin{equation}
    L_{\rm CO} (M, z) = 2N(z) M [(M/M_1(z))^{-b(z)} + (M/M_1(z))^{y(z)}]^{-1}
\end{equation}
with the best fitting parameters and their uncertainties given in Table \ref{table:constraints}. This form is identical to the corresponding form of the abundance matched stellar mass to halo mass relation, found in data-driven approaches \cite{behroozi2010, behroozi2019, moster2010} and summarized in Table \ref{table:constraints}. 

For the case of [CII], a power law with an exponential cutoff matches the low-redshift data well, and the evolution to high redshifts is assumed to follow that of the star formation rate (since [CII] is found to be highly correlated with the star-formation rate, e.g., \cite{knudsen2016}) which is fitted utilizing a data driven procedure in \cite{behroozi2013, behroozi2019}. The relation is thus expressed as
\begin{equation}
 L_{\rm CII}(M,z) = \displaystyle{\left(\frac{M}{M_1}\right)}^{\beta} \exp(-N_1/M) \displaystyle{\left(\frac{(1+z)^{2.7}}{1 + [(1+z)/2.9)]^{5.6}} \right)^{\alpha}}   
\end{equation}
with the best fitting parameters  summarized in Table \ref{table:constraints}.
For [OIII], a fit to available observations of individual galaxies at high redshifts \cite{harikane2020} leads to a parametrization of the luminosity directly in terms of the star formation rate (Table \ref{table:constraints}).


 The advantages of using the above data-driven relations are manifold. A special feature of baryonic line emission power spectra (which can be seen from expressions like \eq{onehalo} and \eq{twohalo}) is that they
depend on the underlying cosmology as well as on the astrophysics of the  systems, and hence can offer constraints on both these aspects. The astrophysics acts as an effective ‘systematic’ uncertainty when making cosmological predictions from such surveys. At the same time, the astrophysical parameters themselves contain valuable information about the role of gas in galaxy formation and evolution. Using the latest available data to constrain the parameters of this framework therefore allows us to  precisely separate both these aspects. Furthermore, the models' computational simplicity allows us to include the effects of several additional parameters, easily vary the physics and cosmology to explore different scenarios, and to impose the most realistic priors (by definition, based on all the data available today) on the astrophysics conveniently within a Fisher matrix analysis. This has advantages for both cosmological and theoretical physics avenues, as it easily accounts for extensions beyond $\Lambda$CDM and is thus uniquely suited to extract cosmological constraints from baryonic data  and make predictions for future surveys \cite{hparaa2019}. We describe how this is done in the forthcoming sections.


\begin{table}[h]
\centering
\begin{tabular}{l|ll}
$k \setminus j$ & $2$ & $4$ \\
\hline
$2$ & $q^{-2}$ &  \\
$-2$ & $q^{2}$ & $1$ \\
\end{tabular}
\caption*{$3_1$}
\end{table}



\section{Learning cosmology from the baryons}\label{s:forecasts}
Over the past decade, the standard model of cosmology (i.e. $\Lambda$CDM, described in Sec. \ref{sec:astrocosmo}), has become fairly well established, with the latest constraints approaching sub-percent levels of accuracy in the measurement of cosmological parameters. However, from a theoretical point of view, several outstanding questions remain to be answered in the context of this model, of which the chief ones are: i) the nature of the late-time accelerated expansion of the Universe,  ii) the mechanism responsible for generating the primordial perturbations that led to structure formation, and (iii) the nature of dark matter. The first phenomenon is often attributed to dark energy, which behaves like a fluid component having negative pressure. Most observations are consistent with dark energy being a cosmological constant as defined in Sec. \ref{sec:astrocosmo}, however, the breakdown of the general theory of relativity on cosmological scales (also known as modifications of gravity) may also explain the observed cosmic acceleration. Modified gravity theories thus have a strong significance in understanding the nature of dark energy.  

The generation of primordial perturbations is believed to be related to an early inflationary epoch of the Universe. The detection of primordial non-Gaussianity, which is characterized by a nonzero value of the parameter $f_{\rm NL}$, places stringent constraints on theories of inflation, since standard single-field slow-roll inflationary models predict negligible non-Gaussianity (e.g., \cite{Maldacena:2002vr}), while non-standard scenarios allow for larger amounts. So far, searches for primordial non-Gaussianity have been through anisotropies in the Cosmic Microwave Background (CMB), or from the clustering of galaxies (e.g. \cite{Giannantonio:2013uqa, castorina2019}).

We are well-placed to study the impact of astrophysics on cosmological forecasts using the halo model treatments developed in the preceding sections. To do so, it is most convenient to \cite{hparaa2019} use a Fisher matrix technique, which enables realistic priors on the astrophysics imposed from the halo model framework. For large area sky surveys (such as those with \HI), the power spectra defined in Sec. \ref{s:HIhalomodel} are first converted into their projected \textit{angular} forms, denoted by $C_{\ell}(z)$ which represent
the observable, on-sky quantities in cosmological mapping surveys. The angular power spectrum can be used to perform a
tomographic analysis of clustering in multiple redshift bins without the 
assumption of an underlying cosmological model, e.g., \cite{seehars2016}. The expression for $C_{\ell}(z)$  can be related to the power spectra defined in Sec. \ref{s:HIhalomodel} and Sec. \ref{s:submmhalomodel} by using the  Limber approximation (accurate to within 1\% for scales above $\ell \sim 10$; e.g. Ref.\cite{limber1953}) that makes use of the angular window function, $W_{\rm HI}(z)$ of the survey:
\begin{equation}
C_{\ell} (z) = \frac{1}{c} \int dz  \frac{{W_{\rm HI}}(z)^2 
H(z)}{R(z)^2} 
P_{\rm HI} [\ell/R(z), z]
\label{cllimber}
\end{equation}
where $H(z)$ is the Hubble parameter at redshift $z$, and $R(z)$ is the comoving distance to redshift $z$. The angular window function is usually assumed to have a top-hat form in redshift space, though other choices are possible.
From the above angular power spectrum, and given an experimental configuration, 
a Fisher forecasting formalism can be used to place constraints on the 
 cosmological [e.g., $\{h, \Omega_m, n_s, \Omega_b, \sigma_8\}$]  and 
astrophysical [e.g., $\{c_{\rm HI}, \alpha, \beta, \gamma, v_{\rm c,0}\}$] 
parameters, which are generically denoted by $A$. The Fisher matrix element 
corresponding to the parameters $\{A,B\}$ and at a redshift bin centred at $z_i$ is then defined as:
\begin{equation}
F_{AB} (z_i) = \sum_{\ell < \ell_{\rm max}} \frac{\partial_A C_\ell (z_i)
\partial_B C_\ell(z_i)}{\left[\Delta C_\ell(z_i)\right]^2},
\end{equation}
where $\partial_A$ is the partial derivative of $C_{\ell}$ with respect to 
$A$. In the above expression, the standard deviation, $\Delta C_{\ell} (z_i)$ is defined 
in terms of the noise of the experiment, $N_{\ell}$ and the sky 
coverage of the survey, $f_{\rm sky}$:
\begin{equation}
\Delta C_\ell = \sqrt{\frac{2}{(2 \ell + 1)f_{\rm sky}}} 
\left(C_\ell + N_\ell\right),\label{eq:Delta_Cl}
\end{equation}
 The full Fisher matrix for an experiment is constructed by summing the individual Fisher matrices in each of the $z$-bins
in the redshift range covered by the survey: 
\begin{equation}
 \mathbf{F}_{AB} = \sum_{z_i} F_{AB}(z_i)
 \label{fisher}
\end{equation}


The above treatment for cosmological forecasting  exactly parallels the corresponding one for the CMB, e.g., \cite{bond1997}, with the additional appearance of the baryonic gas parameters. Note that the Fisher matrix approach assumes that the individual parameter likelihoods are approximately Gaussian, which may not always be the case if the parameters are strongly degenerate. In the present case, however, it is found \cite{hparaa2019} that a full Markov Chain Monte Carlo treatment leads to negligible differences from that obtained by the Fisher matrix technique.

For sub-millimetre surveys that typically cover only a few square degrees of the sky, the three-dimensional power spectrum $P_{\rm submm}(k)$ defined in \eq{submmpower} is directly used to compute the Fisher matrix, which is written as:
\begin{equation}
F_{AB} (z_i) = \sum_{k < k_{\rm max}} \frac{\partial_A P_{\rm submm}(k, z_i)
\partial_B P_{\rm submm}(k,z_i)}{\left[\sigma_P(k,z_i)\right]^2},
\end{equation}
The variance of the power spectrum is denoted by $\sigma_P^2(k, z_i)$ and is computed as:
\begin{equation}
    \sigma_P^2 = \frac{(P_{\rm submm}(k, z_i) + P_{\rm N})^2}{{N_{\rm modes}(k)}}
    \label{varianceauto}
\end{equation}
where $P_{\rm N}$ is the noise power spectrum, analogous to $N_{\ell}$ in \eq{eq:Delta_Cl} above, and $N_{\rm modes}$ is the number of Fourier modes probed by the survey, defined as:
\begin{equation}
    N_{\rm modes} = 2 \pi k^2 \Delta k \frac{V_{\rm surv}}{(2 \pi)^3} 
\end{equation}
in which $V_{\rm surv}$ is the volume of the survey and $\Delta k$ denotes the spacing of the $k$-bins.

To complete the treatment, we need to consider the experimental noise in various configurations which defines  $N_{\ell}$ and $P_{\rm N}$ above.  The way the noise is calculated differs according to the configuration in which the facility is constructed and used. For HI, the experimental configurations can be roughly divided into three categories: (i) single dish telescopes, such as the Five Hundred Metre Aperture Spherical Telescope (FAST, \cite{smoot2017}), BAO in Neutral Gas Observations (BINGO, \cite{battye2012}), and the Green Bank Telescope (GBT) and  (ii) dish interferometers, such as the TianLai \cite{chen2012}, the Square Kilometre Array (SKA) and its pathfinders, like the Meer-Karoo Radio Telescope (MeerKAT, \cite{santos2017}) and the Australian SKA Pathfinder (ASKAP, \cite{johnston2008}), the Hydrogen Intensity and Real-time Analysis eXperiment (HIRAX, \cite{newburgh2016}) and cylindrical configurations, like the Canadian Hydrogen Intensity Mapping Experiment (CHIME, \cite{bandura2014}) and the planned CHORD (Canadian Hydrogen Observatory and Radio transient Detector) \cite{liu2019} facilities. For submillimetre lines, planned or already-fielded instruments include interferometer arrays, such as the Sunyaev-Zeldovich Array (SZA) \cite{keating2016}, as well as single dish facilities such as COMAP\footnote{http://comap.caltech.edu} and CCAT-p \cite{terry2019, parshley2018}, in addition to balloon-based experiments such as EXCLAIM \cite{exclaimpaper2020}. A summary of the noise expressions for the various configurations relevant to these experiments is provided in Table \ref{table:noise}. 



\begin{longtable}{l|l|l}
\caption{Expressions for the noise terms $N_{\ell}$ and $P_{\rm N}$ used in the Fisher information matrix to constrain cosmological and astrophysical parameters with the intensity mapping technique at various redshifts and using various tracers. Key to symbols: $\Delta \nu$ : observed frequency interval;  $T_{\rm sys}$: system temperature, $f_{\rm sky}$ : sky fraction covered, $\lambda_{\rm obs}$ : observed wavelength, $\theta_{\rm beam}$ : telescope beam size, $\bar{T}$ : mean temperature of hydrogen, $D_{\rm dish}$ : the dish size, $N_{\rm dish}$ : the number of dishes, $\nu$ : the observed frequency, $W_{\rm cyl}$ : width of the cylinder, the FoV : Field of View, $A_{\rm eff}$ : effective area,
$t_{\rm pix}$ : time observing per pixel, $n_{\rm base}$ : number of baselines, $n_{\rm pol}$ : number of polarization directions, $N_{\rm beam}$ : number of beams, $D_{\rm max/,min}$: maximum and minimum baselines of the interferometer configuration,
   $V_{\rm vox}$: volume of the `voxel' (volume pixel) in submillimetre surveys, $N_{\rm det}$: number of detectors, $\sigma_{\rm N}$: detector noise per voxel,  $t_{\rm vox}$: time spent observing the voxel, and $t_{\rm tot}$ : total observational time.}
\\
\hline
&&\\
\tablehead
{\hline} Experiment   & Noise expression & Reference \\
&&\\
\hline\hline
 & & \\
 {\large \bf  (i) HI single dish}         &   $N_\ell^{\rm HI,dish} = {W _{\rm beam}^2(\ell)}/{2N_{\rm dish}  t_{\rm pix} \Delta \nu} \left({T_{\rm sys}}/{\bar{T}}\right)^2\left({\lambda_{\rm obs}}/{D_{\rm dish}}\right)^2$ & \\
&  & Ref. \cite{camera2020} \\
& $W _{\rm beam}^2(\ell)=\exp\left[{\ell(\ell+1)\theta_{\rm beam}^2}/{8\ln2}\right]$  & \\
   {\large \bf (ii) HI interferometer} & & \\
  {\large \bf in single dish mode}  & $N_\ell^{\rm HI,intSD} = 4 \pi f_{\rm sky} T_{\rm sys}^2/(2 N_{\rm dish} t_{\rm tot} \Delta \nu)$ & \\
  & & \\
 &   $T_{\rm sys} = 25 + 60 \left({300 \ {\rm MHz}}/{\nu}\right)^{2.5}$ & Refs. \cite{knox1995,bull2014,ballardini2019,bauer2021}  \\
{\large \bf (iii) HI cylindrical} & $N_\ell^{\rm HI,cyl} = \displaystyle{\frac{4 \pi f _{\rm sky}}{{\rm FoV} n_{\rm base}(u) n_{\rm pol} N_{\rm beam}  t_{\rm tot} \Delta \nu}} \displaystyle{\left(\frac{\lambda_{\rm obs}^2}{A_{\rm eff}}\right)^2}\displaystyle{\left(\frac{T_{\rm sys}}{\bar{T}}\right)^2};$ &   \\
   &  &  \\
   & & \\
 & ${\rm FoV} \approx \pi/2 \lambda_{\rm obs}/W_{\rm cyl}$; & \\
 & & \\
 & $\Delta \nu = \nu_{\rm HI} \Delta z/(1+z)^2$ & Refs. \cite{camera2020, newburgh2014, jalilvand2019, obuljen2018}  \\
 & & \\
 {\large \bf (iv)  HI interferometer} & $N_{\ell}^{\rm HI,  int} = T_{\rm sys}^2 {\rm FoV}^2/(T_b^2 n_{\rm pol} n(u = \ell/2\pi)  t_{\rm tot} \Delta \nu)$ & \\
 & & \\
 &  $n(u) = N_{\rm dish} (N_{\rm dish} - 1)/(2 \pi (u_{\rm max}^2 - u_{\rm min}^2))$ & \\
 & & \\
 & FoV $= \lambda^2/D_{\rm dish}^2$; \ $u_{\rm max/min} = D_{\rm max,min}/\lambda$ & Refs. \cite{bauer2021, pourtsidou2016, bull2014}  \\
 & & \\
{\large \bf (v) CO} & $P_{\rm N} = V_{\rm vox} \sigma^2_{\rm vox}$ &  \\
& & \\
 & $\sigma_{\rm vox} = T_{\rm sys}/(N_{\rm det} \Delta \nu t_{\rm vox})^{1/2}$ & Ref. \cite{liu2021} \\
 & & \\
{\large \bf   (vi) CII/OIII IM} &  $P_{\rm N} = V_{\rm vox} \sigma_{\rm N}^2/t_{\rm vox}$ & Ref. \cite{hpcii2019} \\
& &
 \label{table:noise}
 \\
\hline
\end{longtable}
Given the Fisher matrix in \eq{fisher}, we can then compute the standard deviation in the measurement of $A$ for the two cases of fixing and marginalizing over the remaining parameters, as:
\begin{equation}
   \sigma^2_{A, \rm fixed} = (\mathbf{F}_{AA})^{-1}; \ \sigma^2_{A, \rm marg} = (\mathbf{F}^{-1})_{AA};
\end{equation}
The Fisher matrix treatment thus captures the effects of `astrophysical degradation' in the precision of cosmological parameter forecasts, and their evolution with redshifts for different experiment combinations. It was found that, in the case of \HI, the astrophysical degradation was largely mitigated by our prior information coming from the present knowledge of the astrophysics \cite{hparaa2019}. It also revealed an important robustness feature of the halo model framework in Sec. \ref{s:HIhalomodel}: that physically motivated extensions to the halo model did not cause significant changes in the forecasts, which was important to consolidate the utility of the halo model framework for constraining theories of fundamental physics.

It is also important to assess the influence of astrophysical uncertainties on the \textit{accuracy} of cosmological parameter forecasts \cite{camera2020}. This can be addressed  by  employing the nested likelihoods framework \cite{Heavens:2007ka}, which is a measurement of how much the uncertainty in our astrophysical knowledge causes a bias in the forecasted cosmological parameters. Specifically, given the Fisher matrix in \eq{fisher}, we
split the space of parameters into two 
subsets: one containing the parameters of interest and the other containing the parameters
deemed `nuisance' or systematic for the analysis under consideration. When constraining cosmology with baryons,
these two sets could represent `cosmological' and `astrophysical' 
parameters respectively. The bias on a given cosmological parameter $A$, 
denoted by $b_{A}$, is then computed as:
\begin{equation}
b_{A} = \delta \mu F_{B\mu}\left(\mathbf 
F^{-1}\right)_{AB}.\label{eq:bias}
\end{equation}
Here, $\mathbf F^{-1}$ is identical to \eq{fisher} and represents the full Fisher matrix of astrophysical and 
cosmological parameters, and $F_{B\mu}$ stands for the particular sub-matrix mixing 
cosmological and astrophysical parameters. The term $\delta \mu$ denotes 
the vector of the shifts between the fiducial and true values of the astrophysical parameters $\mu$:
\begin{equation}
\delta \mu=\mu^{\rm fid}-\mu^{\rm true}.\label{eq:shift}
\end{equation}





This approach thus allows us to quantify the accuracy of cosmological forecasts obtainable from baryonic surveys. The relative bias on various cosmological parameters from a SKAI-MID like survey, obtained by shifting the parameters $v_{c,0}$ and $\beta$ from their fiducial values in Table \ref{table:constraints}, is plotted in Fig. \ref{fig:relbias1}.
As a bonus, it was found that \cite{camera2020} this technique leads to a powerful way of incorporating effects beyond the standard $\Lambda$CDM framework into the halo model. Specifically, we can use this method to investigate a non-zero primordial non-Gaussianity effect imprinted on the power spectrum, and also incorporate modified gravity scenarios which are an important test of Einstein’s general relativity. Encouragingly, it was found that the primordial non-Gaussianity is negligibly affected by astrophysical uncertainties as can be seen from Fig. \ref{fig:relbias1}, which promises an optimistic outlook for one of the strongest science cases for future intensity mapping experiments.

Dark matter (DM) is also a component of the $\Lambda$CDM cosmological model, however, its nature continues to be a mystery. The only dark matter candidate in the standard model of particle physics, the neutrino, is known to make up less than 1\% of the total DM abundance because its relativistic velocity makes it too “hot” to account for the observed structure formation (e.g., \cite{Alam:2016hwk}). Observations  favour the majority of the remaining DM being composed of a single species of cold, collisionless DM (CDM). One candidate for a significant fraction of DM are axions, which occur in  many extensions of the standard model \cite{PecceiQuinn1977, Weinberg1978}. Axions with masses $m_a \sim 10^{-22}$ eV, known as ``fuzzy DM" \cite{Hu:2000ax}, can make up a significant fraction of the DM, and furthermore have a host of interesting phenomenological consequences on galaxy formation (e.g. \cite{Arvanitaki_2010, Marsh_review2016, Niemeyer:2019aqm}). 

Ref. \cite{bauer2021} used the halo model for HI described in Sec. \ref{sec:analytical} to explore the effects of an axion subspecies of DM in the mass range $10^{-32}$ eV $\leq m_a \leq$ $10^{-22}$ eV on the HI power spectrum at $z \leq 6$. It was found that lighter axions introduce a scale-dependent feature even on linear scales due to the suppression of the matter power spectrum near the Jeans scale. For the first time, it was possible to forecast the bounds on the axion fraction of DM in the presence of astrophysical and model uncertainties, achievable with upcoming facilities such as the HIRAX and SKA. A compilation of the latest forecasted constraints on the above beyond-$\Lambda$ CDM parameters with future surveys mapping baryonic gas is provided in Table \ref{table:beyondlcdm}.
\begin{figure}
\centering
\includegraphics[width = 0.9\textwidth]{relbias1.pdf}
\caption{Relative bias $b/\sigma$ on cosmological parameters, including those beyond the standard $\Lambda$CDM 
framework,  with a SKA I MID-like experimental configuration, obtained on shifting either 
astrophysical parameter, $\beta$ (left panels) or $\log v_{\rm c,0}$ (right panels), 
by $1 \sigma$ from its mean value given in Table \ref{table:constraints}. Top panels show the biases in
$\Lambda$CDM+$\gamma$ where $\gamma$ is a measure of modified gravity \cite{camera2020}, and lower panels show biases for those in $\Lambda$CDM+$f_{\rm NL}$. The empty 
(filled) circles indicate negative (positive) values of  biases. It can be seen that the bias values are well within 1$\sigma$ for all the parameters considered, including the $f_{\rm NL}$. Shaded areas (which are not reached by the results) indicate the range $b/\sigma > 3$, for which it is possible for the Gaussian approximation to the likelihood to become inaccurate. Figure taken from \cite{camera2020}. }
\label{fig:relbias1}
\end{figure}

\begin{landscape}
\begin{longtable}{l|l|l}
\caption[]{Latest forecasted constraints on non-standard dark matter (e.g., \cite{khlopov2013}), tests of inflation and modified gravity with the future experiments tracing baryonic gas, primarily in the radio and sub-millimetre regimes.} 
\tablehead
{\hline Type of constraint  & Limit & Experiment/Reference \\ \hline\hline} 
\tabletail
{\hline \multicolumn{4}{r}{\textit{Continued on next page}}\\}
\tablelasttail{\hline} \\
\hline
{\bf Constraints on parameters describing non-cold dark matter} & &  \\
\hline
Warm dark matter particle mass & $m_{\rm WDM} = 4$ keV ruled out at $> 2 \sigma$ at & \\
&  $z \sim 5$ & SKAI-LOW \cite{carucci2015} \\
Effective parameter for & & \\
dark matter decays & $\Theta_{\chi} \approx 10^{-40}$ at $10^{-5}$ eV & HERA/HIRAX/CHIME \\
& & \cite{bernal2021} \\
Axion mass & $m_a \approx 10^{-22}$ eV, at 1\% & SKAI-MID + CMB \cite{bauer2021} \\
Particle to two-photons coupling of axion-like particles (ALPs) & $g_{a \gamma \gamma} \leq 10^{-11}$/GeV & SPHEREx/LSST \cite{shirasaki2021} \\
\hline
{\bf Primordial non-Gaussianity constraints}  & & \\
  \hline
Standard deviation of  & $\sigma(f_{\rm NL}) \sim 4.07$ & SKA \cite{gomes2020} \\
 non-Gaussianity parameter & $\sigma(f_{\rm NL}) < 1$ & PUMA \cite{karagiannis2020} \\
& $\sigma(f_{\rm NL}) \sim 10$ & COMAP \cite{liu2021} \\
& $\sigma(f_{\rm NL}) \sim 2-3$ & ngVLA  \\
& $\sigma(f_{\rm NL}) \sim 2-3$ & PIXIE/OST MRSS \\
& & \cite{dizgah2019}\\
& $\sigma(f_{\rm NL}) < 1$ & SKA1+Euclid-like+CMB Stage 4 \cite{ballardini2019a} \\
\hline
{\bf Constraints on modified gravity parameters}  & &  \\
 \hline
Effective gravitational strength; & &  \\
initial condition parameter of matter perturbations & $Y (G_{\rm eff}), \alpha < 1\%$ at $z \sim 6-11$ & SKA \cite{heneka2018} \\
Parameter modification of $f(R)$ gravity & $B_0 < 7 \times 10^{-5}$ & CMB + 21-cm \cite{hall2013} \\
Value of $f(R)$ field in background today & $|f_{R0}| < 9 \times 10^{-6}$ & 21 cm \cite{masui2010} \\
\hline
\label{table:beyondlcdm}
\end{longtable}
\end{landscape}





\begin{figure}
\centering
\includegraphics[width = 0.5\textwidth]{clexample.pdf}\includegraphics[width = 0.5\textwidth]{contour_forreview.pdf}
\caption{\textit{Left panel:} Angular cross-correlation power spectrum computed from \eq{angcross} with a CHIME-DESI-like survey combination at redshifts 0.8, 1.2, and 1.6. The error bars denote the corresponding $\Delta C_{\ell}$ at the lowest redshift bin. Figure from \cite{hparaa2019}. \textit{Right panel:} Contours of cosmological parameters $\sigma_8$ and $\Omega_m$ with \textit{all} other parameters fixed in the cross-correlation.}
\label{fig:contour1}
\end{figure} 





\section{The whole is greater than the sum of its parts}
\label{sec:crosscorrelations}

In this section, we describe how bringing together two (or more) different surveys leads to several advantages in the measurement of cosmological and and astrophysical parameters from baryonic tracers. Furthermore, it offers a direct route to extend the modelling frameworks towards the reionization regime, connecting up with future gravitational wave measurements to provide a holistic picture of cosmic dawn.   

\subsection{Gas-galaxy cross-correlations}

If we cross-correlate maps of gas emission (like HI) and galaxy surveys in the optical band, we can significantly improve our accuracy in the prediction of astrophysical parameters. This happens due to the foregrounds and systematics in the two surveys being mitigated in the cross-correlation between the two maps, thereby increasing the significance of the detection. The first detections of HI in intensity mapping took place using this approach \cite{Chang:2010jp, switzer13, masui13} with the HI observations from the Green Bank Telescope cross-correlated with galaxy data from the WiggleZ Dark Energy Survey probing $z \sim 1$, and more recently with galaxies from the eBOSS survey \cite{wolz2021}. Also, the cross-correlation of \HI\ intensity maps obtained from the Parkes telescope with 2dF galaxy Redshift Survey has recently been conducted at a lower redshift, $z \sim 0.08$ \cite{anderson2018}. 

For a cross-correlation survey of \HI\ and galaxies, the observable angular power spectrum is modified from \eq{cllimber} to read (e.g., \cite{hpcrosscorr2020}):
\begin{equation}
C_{\ell, \times} = \frac{1}{c} \int dz \frac{{W_{\rm HI}(z) W_{\rm gal}(z)} 
H(z)}{R(z)^2} 
(P_{\rm HI} P_{\rm gal})^{1/2}
\label{angcross}
\end{equation}
which is illustrated in the left panel of Fig. \ref{fig:contour1} for a Canadian Hydrogen Intensity Mapping Experiment (CHIME)-like survey covering the redshift range 0.8-2.5, cross-correlated with a Dark Energy Spectroscopic Instrument (DESI, \cite{desi2016})-like Emission Line Galaxy (ELG) galaxy survey over the range $z \sim 0.6-1.8$. The angular power spectrum above contains both the \HI\ and galaxy linear power spectra, as well as the corresponding window functions  ($W_{\rm gal}$ can in general be different from $W_{\rm HI}$, and depends on the details of the selection function of the galaxy survey, e.g.,\cite{smail1995}). Such a cross-correlation promises much more stringent constraints (shown in the right panel of Fig. \ref{fig:contour1}) on the astrophysical and cosmological parameters than auto-correlation (correlating galaxies with galaxies), as already shown in several recent analyses,  e.g., cross-correlating a CHIME-like and DESI-like survey leads to an improvement by  factors of a few in the cosmological and astrophysical constraints when compared to those from the CHIME-like survey alone (Ref.\cite{hpcrosscorr2020}, shown in Fig.\ref{fig:contour2}). Notably, this improvement occurs in spite of the fact that the redshift coverage of the cross-correlation is only about half that of the CHIME-like autocorrelation survey, so this illustrates the extent to which adding the galaxy survey information helps to improve the cosmological constraints. 
 It was shown \cite{shi2020} using the halo model framework  described in Sec. \ref{sec:analytical} that the broadband BAO feature measured from the angular cross-correlation power spectra between a DESI  Emission Line Galaxy (ELG) survey and the 21 cm intensity maps measured from the TianLai survey\footnote{http://tianlai.bao.ac.cn/}, can be used to forecast a constraint on the angular diameter distance with a precision of 2.7\% over $0.775 < z < 1.03$, which is complementary to the BAO cosmic distance measured by galaxy-galaxy auto-correlation.
 
 \begin{figure}
\centering
\includegraphics[width = \textwidth]{barchart_evolution_chime_desi_nomarg.pdf}
\caption{Relative accuracy on forecasted cosmological and astrophysical parameters with a CHIME-DESI-like survey cross-correlation covering redshifts $0.8 < z < 1.6$. \textit{Left panel:} Constraints on the cosmological parameters $\{h, \Omega_m, n_s, \Omega_b, \sigma_8\}$ in a flat $\Lambda$CDM framework, for (i) fixed values of astrophysical parameters (red) and (ii) marginalizing over astrophysical parameters (yellow). \textit{Right panel:} Constraints on the HI-halo mass parameters ($v_{c,0}$ and $\beta$ from Table \ref{table:constraints}, and the $Q$ parameter which denotes the large scale galaxy bias \cite{cole2005}) for (i) fixed values of cosmological parameters (green), (ii) marginalizing over the cosmological parameters but adding an astrophysical prior (cyan). The extent of the astrophysical prior from current data is plotted as the violet band in each case.
 Figure from \cite{hparaa2019}.}
\label{fig:contour2}
\end{figure}


In the sub-millimetre regime, the cross-correlation prospects for the COMAP (CO Mapping Array Project) survey and the photometric COSMOS and spectroscopic HETDEX Lyman-alpha fields \cite{chung2019} showed that about 0.3\% accuracy in redshifts with greater than 0.0001 sources per cubic Mpc, with spectroscopic redshift determination should enable a CO-galaxy cross spectrum detection significance at least twice that of the CO auto spectrum (even if the CO survey covers only a few square degrees). This illustrates that cross-correlations with galaxy surveys can make significant improvements to the goals of future intensity mapping experiments in the submillimetre regime. 






\begin{figure}
\begin{center}
\includegraphics[width = 0.45\textwidth]{autopowerspecdesign_7noise_forrev.pdf} \includegraphics[width=0.45\textwidth]{crosspower_7design.pdf}
\caption{\textit{Left panel}: Thick lines show the forecasts for the autocorrelation power spectra (computed using \eq{COpowspeclog} with a correction for the finite size of the telescope beam) of the [OIII]  88 $\mu$m transition (green) and the [CII] 158 $\mu$m transitions (blue) at $z \sim 7$. Overplotted in thin steps are the noise estimates (using \eq{varianceauto}) for a designed survey exploiting the potential of current architecture. \textit{Right panel:} Cross correlation of [OIII] 88 $\mu$m with [CII] 158 $\mu$m with the design configuration (red line), and the associated noise (thin steps). The total signal-to-noise is indicated in all cases. Figure adapted from \cite{hpoiii}.}
\label{fig:oiii}
\end{center}
\end{figure}


\subsection{Towards the reionization regime}
\label{s:estimator}

Thus far, we have considered the evolution of baryonic gas in the late-time universe, notably hydrogen and carbon inside galaxies and discussed the cosmological and physics constraints that can be extracted from their surveys. As we saw from Sec. \ref{sec:astrocosmo}, up to redshifts of about $z \leq 6$ or so, the neutral gas is primarily inside galaxies, with most of the intervening material being taken up by the ionized Lyman-$\alpha$ forest.

At higher redshifts ($z > 6-10$ or so), the situation is expected to be very different. The intergalactic hydrogen is predominantly neutral since the universe has not yet been reionized. Moreover, before virialized structures (stars, galaxies) form in large numbers, the baryons act as excellent tracers of the underlying dark matter.  The neutral hydrogen (the predominant baryon at high redshifts) is therefore a direct probe of the \textit{cosmic web}, and its mapping is thus expected to shed light on the process of structure formation itself. 


The epoch of Cosmic Dawn, where the first stars and galaxies were born, signals the start of the second major phase transition of nearly all the normal matter in the universe, i.e., Cosmic Reionization which was introduced in Sec. \ref{sec:reion}. 
Reionization is characterized by the development of ionized regions (called ‘bubbles’) around the first luminous sources, and the end of reionization is marked by the complete overlap of these bubbles. The distribution of the bubble sizes leads to fluctuations in the neutral hydrogen density field, and thus in the signal observed with future 21 cm experiments. Hence, mapping the distribution of the ionized bubbles is crucial for modelling the observable signal. The bubble size distribution has been investigated from both analytical \cite{Furlanetto:2004ha, hpaseem} and seminumerical/simulation-based \cite{choudhury2021, molaro2019, mesinger2011} approaches.
The SKA will be able to image these ionized bubbles at Cosmic Dawn, and its pathfinder, the Murchinson Widefield Array (MWA; e.g., Ref. \cite{lonsdale2009}), will aim to map the evolution of reionization using the redshifted 21 cm line of neutral hydrogen. The future James Webb Space Telescope (JWST) and European Extremely Large Telescope (ELT) will provide enhanced constraints on the properties of the galaxies responsible for the ionization \cite{park2020, zackrisson2020}. Recently, the Mayall telescope imaged the EGS77 group \cite{tilvi2020} and led to the first observations of ionized bubbles at the highest redshift of $z \sim 7.7$. 




There are excellent prospects for investigating reionization using molecular lines. The CO molecular spectrum has a `ladder' of quantum states separated by integer valued quantum numbers, and hence the CO 1-0 line  observations from the epoch of peak star formation, $z \sim 2 - 3$ traced, e.g., by COMAP, also contain a contribution from the CO 2-1 line from the mid to late stages of reionization, $z \sim 6-8$. The latest forecasts predict a detection \cite{breysse2021} of the Reionization signal at high significance
for the next stage COMAP-EoR survey over $z \sim 6 - 8$. It allows us to place very tight constraints on the cosmic molecular
gas density where there is a significant contribution from faint galaxies {\it that would otherwise be missed by current
and future galaxy surveys}, re-iterating the unique ability of line intensity mapping to constrain the properties of the
earliest galaxies. As we saw in Sec. \ref{sec:submmintro}, the redshifted 158 micron line of the singly ionized carbon ion, [CII] and the doubly ionized oxygen ion, [OIII], are also salient probes of reionization and high-redshift galaxies. It can be shown \cite{hpoiii} for a future survey targeting the [OIII] and [CII] species in the sub-millimetre regime at $z \sim 7$ during the period of reionization, which is designed to jointly exploit the potential of the currently proposed EXCLAIM (Experiment for Cryogenic Large-Aperture Intensity Mapping, a balloon based facility), Ref. \cite{cataldo2021} and the FYST (Fred Young Submillimetre Telesope, a ground-based facility \cite{terry2019}) experiments lead to several tens of sigma detection in auto- and cross-correlation modes. Examples of the auto- and cross-correlation power spectra from such a survey at $z \sim 7$ are shown in Fig. \ref{fig:oiii}.


\subsection{Multi-messenger cosmology: the gravitational wave regime}



An outstanding question in cosmology is the formation and fuelling of the earliest black holes in the universe, as we described in Sec. \ref{sec:firstbh}. As we have seen, the bulge mass of the galaxy is connected to that of its central black hole. The results of \cite{behroozi2019} provide a data-driven approach towards constraining the evolution of the stellar mass - halo mass relation as pointed out in Sec. \ref{sec:analytical}. These results indicate that the stellar to halo mass relation evolves only by a factor of $\sim 1.6$ over $z \sim 0-6$, in contrast to the black hole mass - halo mass relation that evolves as a steep function of the halo virial velocity, $M_{\rm BH} \propto v_c^{\gamma}$ where $\gamma \sim 5$.

The above result leads to a fascinating conclusion. Specifically, since the stellar mass evolves much more modestly than the black hole mass (also found in recent observations, e.g., \cite{venemans2016, decarli2018}), the dominance of the black hole can be felt at high redshifts to a much greater distance from the centre of the system. This leads to a 
 direct, observable signature \cite{hploebbh2020} of the prevalence of massive black holes in the centres of the first galaxies during the period of reionization. It is found that the influence of the central black hole can dominate the kinematics  up to a distance of $\gtrsim 0.5$ kpc from the centre of the dark matter halo at redshifts $z \gtrsim 6$.


Constraining the properties of these earliest sources of reionization, is possible through their gravitational wave signatures detectable by the next-generation Laser Interferometer Space Antenna (LISA) instrument, e.g., \cite{gair2011}. With the above evolution of the black-hole mass to halo mass relation combined with the framework describing merger rates of dark matter haloes, e.g., \cite{fakhouri2010}, it becomes possible to use a future detection rate from LISA to place constraints on the astrophysical parameters (denoted by $f_{\rm bh}$, the occupation fraction, $\epsilon_0$, the normalization and $\gamma$, the slope  described in Sec. \ref{sec:firstbh} and Table \ref{table:constraints}) governing the occupation of black holes in high redshift haloes, similarly to the constraints on astrophysical parameters from baryonic gas occupation of haloes. It can be shown \cite{hploeblisa2020c} that for three different confidence scenarios, each assumed to have 100, 200 or 400 events per year for a survey of a 5 year duration, the above parameters can be constrained to percent or sub-percent accuracy over $z \sim 1-5$ (and out to $z \sim 8$, Fig. \ref{fig:multimessenger}). Even before the advent of LISA, gravitational waves from Pulsar Timing Arrays (PTAs) such as those with the Parkes and the European Pulsar Timing Array (EPTA, \cite{babak2016}) have the potential to provide us with exquisite constraints on the nature of the earliest supermassive black holes. It is possible to identify the potential host galaxies of these systems using LSST on the Vera Rubin Observatory, which is expected to detect $\sim 100-200$ galaxies with black hole masses above $10^{6.5} M_{\odot}$. Considering the gas accretion emission from the merger leads to similar conclusions, reaching about 1-1000 electromagnetic counterparts \cite{kocsis2006}. The above results thus provide a holistic, multi-messenger view into the first galaxies. 



We are thus on the brink of significantly advancing our understanding of baryonic cosmology over a nearly 13 billion year timescale, and developing novel techniques that can be extended to other investigations of the early universe. It enables us to combine the largest available datasets of gravitational-wave, radio, millimeter and optical observations to create the most detailed models and simulations of baryonic gas and galaxies, both for the epoch of reionization and the post-reionization universe.


\begin{figure}
\begin{center}
\includegraphics[width =0.7\textwidth]{contour_array_fixedz8_forreview.pdf}
\end{center}
\label{fig:multimessenger}
\caption{Constraints on the astrophysical parameters in the black hole mass - halo mass relation (the occupation fraction $f_{\rm bh}$, and $\epsilon_0$ and $\gamma$ defined in Table \ref{table:constraints}) from observations of LISA events at $z \sim 8$. Contours indicate 1- and 2-$\sigma$ confidence levels assuming a fiducial 5-year LISA survey having 200 events per year. Figure adapted from \cite{hploeblisa2020c}.}
\end{figure}

\section{Outlook for the future}
\label{sec:outlook}
In this concluding section, we will summarize some open challenges and recent developments in the areas we have described above, and indicate the theoretical and observational outlook for the future.

\subsection{Unravelling the nature of DLAs}


As we have seen in Sec. \ref{sec:igm}, the intergalactic medium is primarily composed of regions having a column density (in \HI) of $10^{12} - 10^{17}$ cm$^{-2}$.  Above this limit, the hydrogen gas becomes self-shielded to the ionizing radiation, and  
regions that have hydrogen column densities higher than $10^{20.3} \ {\rm cm}^{-2}$ form systems called Damped Lyman Alpha systems (DLAs). These are known to be the highest reservoirs of atomic (hydrogen) gas at intermediate redshifts \cite{wolfe1986, lanzetta1991, storrielombardi2000, gardner1997, prochaska2005} between $z \sim 2$ and $\sim 5$, and our current understanding of the distribution of \HI\ comes from their  observations (Sec. \ref{sec:analytical}) in the spectra of high-redshift quasars \cite{noterdaeme09, noterdaeme12, prochaska09, prochaska2005, zafar2013}.
As the primary reservoirs of neutral gas fuel for the formation of stars at lower redshifts, they are thus the progenitors of today's  galaxies.

Nearly fifty years since they were first discovered \cite{lowrance1972, beaver1972},  a precise understanding of the nature of DLAs still remains elusive. Identifying the host galaxies associated with the absorbers lends clues to their host halo masses and other properties. The presence of the bright background quasar may make the direct imaging of DLAs difficult (though this may also be a function of the impact parameter of observation, e.g., \cite{mackenzie2019}. Imaging surveys for DLAs \cite{fynbo2010, fynbo2011, fynbo2013, bouche2013, rafelski2014, fumagalli2014} thus far point to evidence for DLAs arising in the vicinity of faint, low star-forming galaxies due to the absence of high star-formation rates or luminosities in the samples. There is also an observed lack of high-luminosity galaxies in the vicinity of DLAs, which also points to evidence for DLAs being associated with dwarf galaxies at high redshifts \cite{cooke2015}. The results of simulations find DLAs to arise in host haloes of masses $10^9 - 10^{11} M_{\odot}$ at redshift $z \sim 3$ \cite{pontzen2008, tescari2009, fumagalli2011, cen2012, voort2012, bird2013, rahmati2014}. As far as their structure is concerned, DLAs have been modelled to arise as rotating disks \cite{prochaska2010}, though protogalactic clumps \cite{haehnelt1998} are also consistent with their observed properties. Interestingly, in the redshift regime of relevance to DLAs ($z \sim 2-5$), the statistical halo model framework described in Sec. \ref{sec:dlahimodels} matched to the data assigns an equal likelihood to both these possibilities \cite{hparaa2017}.
 
The cross-correlation of DLAs and the Lyman-$\alpha$ forest from the  twelfth Data Release (DR12) of the Baryon Oscillations Spectroscopic Survey
(BOSS) from the Sloan Digital Sky Survey III (SDSS-III) led to the interesting result \cite{fontribera2012}, that the bias parameter of DLAs -- which is a measure of how strongly the DLAs are clustered  -- is somewhat larger ($b_{\rm DLA} \sim 1.9$) than predicted by the standard evolution expected from lower redshifts, $b_{\rm DLA} \sim 1.5-1.8$ \cite{hptrcar2016}, using the results in \eq{bdla}. Follow-up analyses \cite{perez2018} have measured $b_{\rm DLA} \sim 1.5-2.5$, also finding an increase in the bias with metallicity. 

 The observed high bias $b_{\rm DLA}$ may be consistent with the imaging results if the DLAs arise from dwarf galaxies which are satellites of massive galaxies \cite{fontribera2012}. Theoretically, such a value could arise in models in which the neutral hydrogen in shallow potential wells is depleted, leading to the possibility of very efficient stellar feedback \cite{barnes2014}. However, it is difficult to reconcile models having very strong feedback with the low-redshift observations of \HI\ bias and abundance. 
 The bias value is a crucial pointer to the mass of dark matter haloes hosting the high-$z$ DLA systems, and helps shed light on the (as-yet unsolved) question of the nature of DLAs at high redshifts. It is thus interesting to investigate whether there is scope for placing constraints on the scale- and/or metallicity-dependences \cite{digioia2020} of the clustering using future data. 
 Radio surveys targeting the 21-cm line in \textit{absorption} against bright background quasars (for which the physics follows a description similar to that outlined in Sec. \ref{sec:igm} but does not lead to line saturation), may preferentially pick up larger spiral galaxies and thus be sensitive to more luminous galaxies. The First Large Absorption Survey of Hydrogen (FLASH; Ref. \cite{allison2020}) using the ASKAP telescope, and future surveys with the Canadian Hydrogen Intensity Mapping Experiment \cite{yu2014} aim to look for and localize 21 cm analogs of DLA systems, thus providing a complementary view on the origin of DLAs.


\subsection{Extensions to the halo model framework}
\label{sec:future}

\begin{figure}
\begin{center}
\includegraphics[width = 0.6\textwidth]{HIparkes_withrsd.pdf}
\caption{Power spectrum in redshift space computed using the halo model framework for HI corrected for the effects of redshift space distortions \eq{rs}. The points with error bars show the measurements of HI-galaxy cross power at $z \sim 0.082$, derived from the cross-correlation of Parkes and 2dF galaxy maps \cite{anderson2018}. The large scale HI bias is  assumed to be $b = 0.85$ with respect to the dark matter, consistently with the findings of \cite{martin12}.}
\end{center}
\label{fig:hiparkeswithrsd}
\end{figure}


Throughout the analysis so far, we have considered the halo model for clustering of dark matter and baryons in the absence of additional non-linear effects like redshift space distortions (RSDs), which originate from the peculiar motions of the gas. It is relatively straightforward to include these effects into the framework when one is interested in analysing real data. Redshift space distortions lead to two main effects on the halo model power spectrum: (i) a boost of power on large scales, due to streaming of matter into overdense regions, and (ii) a suppression of power on small scales due to virial motions within collapsed objects \cite{Kaiser:1987qv, peacock1994}. Both these effects serve to modify the power spectrum in \eq{onehalo} and \eq{twohalo}. By paralleling the approach developed in \cite{seljak2001, white2001} for the galaxy-halo connection, the resulting modifications for the case of gas (using \HI\ as an example) can be shown to have the form :


\begin{eqnarray}
& &P_{\rm HI}(k) = \left( F_{\rm HI}^2+{2\over 3}F_{\rm m}F_{\rm HI} + {1\over 5}F_{\rm m}^2\right)P_{\rm lin}(k)
   \\
& & + 
{1 \over  \bar{\rho_{\rm HI}}^2}
\int n(M) dM {M_{\rm HI}^2}
{\cal R}_2(k\sigma) |u_{\rm HI}(k,M)|^2,
\nonumber
\label{rs}
\end{eqnarray}
where 
\begin{eqnarray}
F_{\rm m}&=&f\int n(M) dM \ b(M) {\cal R}_1(k\sigma) u(k, M) 
\nonumber \\
F_{\rm HI}&=&
{{1}\over \bar{\rho_{\rm HI}}} \int
n(M) d M  M_{\rm HI} (M) b(M) {\cal R}_1(k\sigma)u_{\rm HI} (k,M)
\label{p0}
\end{eqnarray}
where $u(k,M)$ and $u_{\rm HI}(k,M)$ are the normalized Hankel transforms of the dark matter and \HI\ profiles, and
$f = d \log \delta/ d \log a \approx \Omega^{0.6}$ captures the effect of redshift space distortions.
The $\mathcal{R}$ terms are given by [with $\sigma = [GM/2R_v(M)]^{1/2}$ being the rms matter density fluctuation smoothed on the scale of the halo]:
\begin{equation}
  {\cal R}_1(k\sigma) = \sqrt{\pi\over 2} { {\rm erf}(k \sigma/\sqrt{2})\over k \sigma},
\label{r2}
\end{equation}
and
\begin{eqnarray}
  {\cal R}_2(y=k\sigma) &=& {\sqrt{\pi}\over 8} { {\rm erf}(y)\over y^5}
    \left[ 3f^2+4fy^2+4y^4 \right] \nonumber \nonumber \\
  &-& {e^{-y^2}\over 4y^4}\left[ f^2(3+2y^2)+4fy^2 \right]
\end{eqnarray}
\label{r2}
The  above redshift-space modifications can be applied to the \HI\ power spectrum to compare with the recent cross-correlation observations of \cite{anderson2018}, which finds some evidence for a low-amplitude clustering of HI relative to dark matter in the cross-correlation of HI maps from the Parkes survey cross-correlated with galaxies from the 2dF (Two degree Field) survey. The results are shown in Fig. \ref{fig:hiparkeswithrsd}. The observations are well matched to the model within the expected scatter, though there may be some evidence for more significant suppression on scales $k \sim 1$ h Mpc$^{-1}$.


In angular space, the nonlinear modelling of redshift space distortions is more complex and depends on the redshift binning under consideration \cite{jalilvand2020}.
    The treatment of RSDs in the Limber approximation on linear scales has recently been developed \cite{2019MNRAS.489.3385T}, while N-body and hydrodynamical simulations \cite{seehars2016, villaescusa2018} have also been used to model the effect of redshift space distortions on HI intensity maps. Just as in the case of galaxy RSDs, the HI RSDs can potentially constrain several interesting effects and break degeneracies. For example, \cite{hall2017} illustrate how peculiar velocity effects can be constrained 
using the dipole of the redshift space cross-correlation between 21 cm and 
optical surveys.  It is potentially interesting to analyse how redshift space distortions can affect the intensity mapping of submillimetre lines  \cite{chung2019pi} as well, though typically, the large scale effects need a factor $\sim 10$ times more area than available in current configurations, and the small scale effects are often suppressed by the finiteness of the beam in these experiments that wipe out scales of the order of $k \sim 1 - 10 h$ Mpc $^{-1}$ (e.g., Ref. \cite{schaan2021}, as can also be seen from \fig{fig:oiii}). This also makes it difficult to distinguish the 1-halo term from shot noise in these experiments. In the case of cosmological forecasts, these are not expected to make a significant difference to the scales of present interest within the scatter of the results.  



    




A new family of Langevin samplers was introduced in this paper. These new SDE samplers consist of perturbations of the underdamped Langevin dynamics (that is known to be ergodic with respect to the canonical measure), where auxiliary drift terms in the equations for both the position and the momentum are added, in a way that the perturbed family of dynamics is ergodic with respect to the same (canonical) distribution. These new Langevin samplers were studied in detail for Gaussian target distributions where it was shown, using tools from spectral theory for differential operators, that an appropriate choice of the perturbations in the equations for the position and momentum can improve the performance of the Langvin sampler, at least in terms of reducing the asymptotic variance. The performance of the perturbed Langevin sampler to non-Gaussian target densities was tested numerically on the problem of diffusion bridge sampling.

The work presented in this paper can be improved and extended in several directions. First, a rigorous analysis of the new family of Langevin samplers for non-Gaussian target densities is needed. The analytical tools developed in~\cite{duncan2016variance} can be used as a starting point. Furthermore, the study of the actual computational cost and its minimization by an appropriate choice of the numerical scheme and of the perturbations in position and momentum would be of interest to practitioners. In addition, the analysis of our proposed samplers can be facilitated by using tools from symplectic and differential geometry. Finally, combining the new Langevin samplers with existing variance reduction techniques such as zero variance MCMC, preconditioning/Riemannian manifold MCMC can lead to sampling schemes that can be of interest to practitioners, in particular in molecular dynamics simulations. All these topics are currently under investigation.


\begin{comment}
\notate{There needs to be a conclusion to the paper}
\subsection{'Symplectic manifold MCMC'}

The generator of the unperturbed Langevin dynamics (\ref{eq:langevin})
given by 
\[
\mathcal{L}_{0}=M^{-1}p\cdot\nabla_{q}-\nabla V(q)\cdot\nabla_{p}-\Gamma M^{-1}p\cdot\nabla_{p}+\nabla\Gamma\nabla
\]
is often written as 
\[
\mathcal{L}_{0}=\{H,\cdot\}-\Gamma M^{-1}p\cdot\nabla_{p}+\nabla\Gamma\nabla,
\]
where the Hamiltonian is expressed in terms of the \emph{Hamiltonian
	\[
	H(q,p)=V(q)+\frac{1}{2}p^{T}M^{-1}p
	\]
}and the \emph{Poisson bracket
	\[
	\{A,B\}=\big((\nabla_{q}A)^{T}(\nabla_{p}A)^{T}\Pi_{0}\left(\begin{array}{c}
	\nabla_{q}B\\
	\nabla_{p}B
	\end{array}\right).
	\]
}Here $\Pi_{0}$ denotes the $2d\times2d$-matrix 
\[
\Pi_{0}=\left(\begin{array}{cc}
\boldsymbol{0} & -I\\
I & \boldsymbol{0}
\end{array}\right)
\]
and $A,B:\mathbb{R}^{2d}\rightarrow\mathbb{R}^{2d}$ are sufficiently
regular functions. Now observe that the generator (\ref{eq:generator})
can be expressed in the same way if $\Pi_{0}$ is replaced by 
\[
\tilde{\Pi}=\left(\begin{array}{cc}
\mu J_{1} & -I\\
I & \nu J_{2}
\end{array}\right).
\]
Therefore, the perturbations under investigation in this paper can
be interpreted more abstractly as a change of Poisson structure. In
this framework, the unperturbed Langevin dynamics (\ref{eq:langevin})
should be thought of as evolving in $\mathbb{R}^{2d}$ equipped with
the canonical symplectic structure associated to $\Pi_{0}$. The perturbed
dynamics (\ref{eq:perturbed_underdamped}) then represent
a process in $\mathbb{R}^{2d}$ equipped with the symplectic structure
given rise to $\tilde{\Pi}$. This alternative viewpoint has multiple
advantages. For instance, the underlying symplectic structure suggests
efficient numerical integrators for the perturbed dynamics. Moreover,
this formulation naturally allows for the possibility of introducing
point-dependent Poisson structures connented to perturbations $J_{1}(q,p)$
and $J_{2}(q,p)$ that depend on $q$ and $p$. In this way, it might
be possible to extend the results of this paper to target measures
with locally different covariance structures or targets that are far
away from Gaussian. Lastly, this formulation interacts nicely with
\emph{Riemannian manifold Monte Carlo }approaches (see \cite{GirolamiCalderhead2011}),
where the metric structure of the underlying manifold instead of the
symplectic structure is changed. All of those connections will be
explored further in a subsequent publication.


\end{comment}

In this work, we study the effect of competition in prophet settings.
%
We show that under both random and ranked tie-breaking rules, agents have simple strategies that grant them high guarantees, ones that are tight even with respect to equilibrium profiles under some distributions.

Under the ranked tie-breaking rule, we show an interesting correspondence between the equilibrium strategies of the $k$ competing agents and the optimal strategy of a single decision maker that can select up to $k$ rewards. 
It would be interesting to study whether this phenomenon applies more generally, and what are the conditions under which it holds.
 



Below we list some future directions that we find particularly natural. 
\begin{itemize}
	\item Study competition in additional problems related to optimal stopping theory, such as Pandora's box \cite{weitzman1979optimal}.% and game of Googol \cite{gnedin1994solution} settings.
	\item Study competition in prophet (and secretary) settings under additional tie-breaking rules, such as random tie breaking with non-uniform distribution, and tie-breaking rules that allow to split rewards among agents.
	\item Study competition in  scenarios where agents can choose multiple rewards, under some feasibility constraints (such as matroid or downward-closed feasibility constraints). 
	\item Consider prophet settings with the objective of outperforming the other agents, as in \cite{immorlica2011dueling}, or different agents' objectives.
	\item Consider competition settings with non-immediate decision making, as in \cite{ezra2020competitive}.
\end{itemize}






\section{Acknowledgments}
I acknowledge support from the Swiss National Science Foundation (SNSF) via Ambizione grant PZ00P2\_179934.

 


\bibliographystyle{unsrt}

\def\aj{AJ}                   
\def\araa{ARA\&A}             
\def\apj{ApJ}                 
\def\apjl{ApJ}                
\def\apjs{ApJS}               
\def\ao{Appl.Optics}          
\def\apss{Ap\&SS}             
\def\aap{A\&A}                
\def\aapr{A\&A~Rev.}          
\def\aaps{A\&AS}              
\def\azh{AZh}                 
\def\baas{BAAS}
\def\jcap{JCAP}
\def\jrasc{JRASC}             
\def\memras{MmRAS}
\def\na{New Astronomy}
\def\nat{Nature}
\def\mnras{MNRAS}             
\def\pra{Phys.Rev.A}          
\def\prb{Phys.Rev.B}          
\def\prc{Phys.Rev.C}          
\def\prd{Phys.Rev.D}          
\def\prl{Phys.Rev.Lett}       
\def\pasp{PASP}    
\def\pasa{PASA}           
\def\pasj{PASJ}
\def\physrep{Phys. Repts.}
\def\qjras{QJRAS}             
\def\skytel{S\&T}             
\def\solphys{Solar~Phys.}     
\def\sovast{Soviet~Ast.}      
\def\ssr{Space~Sci.Rev.}      
\def\zap{ZAp}                 
\let\astap=\aap
\let\apjlett=\apjl
\let\apjsupp=\apjs

\bibliography{mybib, refs, mybib1, mybib1a, mybib2, mybib3, mybib4, mybib5, mybib6, main_bib}

 \end{document}


