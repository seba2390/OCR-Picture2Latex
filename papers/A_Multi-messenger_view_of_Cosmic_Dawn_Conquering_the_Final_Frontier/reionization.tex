


\begin{figure}
\begin{center}
\includegraphics[width = 0.45\textwidth]{autopowerspecdesign_7noise_forrev.pdf} \includegraphics[width=0.45\textwidth]{crosspower_7design.pdf}
\caption{\textit{Left panel}: Thick lines show the forecasts for the autocorrelation power spectra (computed using \eq{COpowspeclog} with a correction for the finite size of the telescope beam) of the [OIII]  88 $\mu$m transition (green) and the [CII] 158 $\mu$m transitions (blue) at $z \sim 7$. Overplotted in thin steps are the noise estimates (using \eq{varianceauto}) for a designed survey exploiting the potential of current architecture. \textit{Right panel:} Cross correlation of [OIII] 88 $\mu$m with [CII] 158 $\mu$m with the design configuration (red line), and the associated noise (thin steps). The total signal-to-noise is indicated in all cases. Figure adapted from \cite{hpoiii}.}
\label{fig:oiii}
\end{center}
\end{figure}


\subsection{Towards the reionization regime}
\label{s:estimator}

Thus far, we have considered the evolution of baryonic gas in the late-time universe, notably hydrogen and carbon inside galaxies and discussed the cosmological and physics constraints that can be extracted from their surveys. As we saw from Sec. \ref{sec:astrocosmo}, up to redshifts of about $z \leq 6$ or so, the neutral gas is primarily inside galaxies, with most of the intervening material being taken up by the ionized Lyman-$\alpha$ forest.

At higher redshifts ($z > 6-10$ or so), the situation is expected to be very different. The intergalactic hydrogen is predominantly neutral since the universe has not yet been reionized. Moreover, before virialized structures (stars, galaxies) form in large numbers, the baryons act as excellent tracers of the underlying dark matter.  The neutral hydrogen (the predominant baryon at high redshifts) is therefore a direct probe of the \textit{cosmic web}, and its mapping is thus expected to shed light on the process of structure formation itself. 


The epoch of Cosmic Dawn, where the first stars and galaxies were born, signals the start of the second major phase transition of nearly all the normal matter in the universe, i.e., Cosmic Reionization which was introduced in Sec. \ref{sec:reion}. 
Reionization is characterized by the development of ionized regions (called ‘bubbles’) around the first luminous sources, and the end of reionization is marked by the complete overlap of these bubbles. The distribution of the bubble sizes leads to fluctuations in the neutral hydrogen density field, and thus in the signal observed with future 21 cm experiments. Hence, mapping the distribution of the ionized bubbles is crucial for modelling the observable signal. The bubble size distribution has been investigated from both analytical \cite{Furlanetto:2004ha, hpaseem} and seminumerical/simulation-based \cite{choudhury2021, molaro2019, mesinger2011} approaches.
The SKA will be able to image these ionized bubbles at Cosmic Dawn, and its pathfinder, the Murchinson Widefield Array (MWA; e.g., Ref. \cite{lonsdale2009}), will aim to map the evolution of reionization using the redshifted 21 cm line of neutral hydrogen. The future James Webb Space Telescope (JWST) and European Extremely Large Telescope (ELT) will provide enhanced constraints on the properties of the galaxies responsible for the ionization \cite{park2020, zackrisson2020}. Recently, the Mayall telescope imaged the EGS77 group \cite{tilvi2020} and led to the first observations of ionized bubbles at the highest redshift of $z \sim 7.7$. 




There are excellent prospects for investigating reionization using molecular lines. The CO molecular spectrum has a `ladder' of quantum states separated by integer valued quantum numbers, and hence the CO 1-0 line  observations from the epoch of peak star formation, $z \sim 2 - 3$ traced, e.g., by COMAP, also contain a contribution from the CO 2-1 line from the mid to late stages of reionization, $z \sim 6-8$. The latest forecasts predict a detection \cite{breysse2021} of the Reionization signal at high significance
for the next stage COMAP-EoR survey over $z \sim 6 - 8$. It allows us to place very tight constraints on the cosmic molecular
gas density where there is a significant contribution from faint galaxies {\it that would otherwise be missed by current
and future galaxy surveys}, re-iterating the unique ability of line intensity mapping to constrain the properties of the
earliest galaxies. As we saw in Sec. \ref{sec:submmintro}, the redshifted 158 micron line of the singly ionized carbon ion, [CII] and the doubly ionized oxygen ion, [OIII], are also salient probes of reionization and high-redshift galaxies. It can be shown \cite{hpoiii} for a future survey targeting the [OIII] and [CII] species in the sub-millimetre regime at $z \sim 7$ during the period of reionization, which is designed to jointly exploit the potential of the currently proposed EXCLAIM (Experiment for Cryogenic Large-Aperture Intensity Mapping, a balloon based facility), Ref. \cite{cataldo2021} and the FYST (Fred Young Submillimetre Telesope, a ground-based facility \cite{terry2019}) experiments lead to several tens of sigma detection in auto- and cross-correlation modes. Examples of the auto- and cross-correlation power spectra from such a survey at $z \sim 7$ are shown in Fig. \ref{fig:oiii}.


\subsection{Multi-messenger cosmology: the gravitational wave regime}



An outstanding question in cosmology is the formation and fuelling of the earliest black holes in the universe, as we described in Sec. \ref{sec:firstbh}. As we have seen, the bulge mass of the galaxy is connected to that of its central black hole. The results of \cite{behroozi2019} provide a data-driven approach towards constraining the evolution of the stellar mass - halo mass relation as pointed out in Sec. \ref{sec:analytical}. These results indicate that the stellar to halo mass relation evolves only by a factor of $\sim 1.6$ over $z \sim 0-6$, in contrast to the black hole mass - halo mass relation that evolves as a steep function of the halo virial velocity, $M_{\rm BH} \propto v_c^{\gamma}$ where $\gamma \sim 5$.

The above result leads to a fascinating conclusion. Specifically, since the stellar mass evolves much more modestly than the black hole mass (also found in recent observations, e.g., \cite{venemans2016, decarli2018}), the dominance of the black hole can be felt at high redshifts to a much greater distance from the centre of the system. This leads to a 
 direct, observable signature \cite{hploebbh2020} of the prevalence of massive black holes in the centres of the first galaxies during the period of reionization. It is found that the influence of the central black hole can dominate the kinematics  up to a distance of $\gtrsim 0.5$ kpc from the centre of the dark matter halo at redshifts $z \gtrsim 6$.


Constraining the properties of these earliest sources of reionization, is possible through their gravitational wave signatures detectable by the next-generation Laser Interferometer Space Antenna (LISA) instrument, e.g., \cite{gair2011}. With the above evolution of the black-hole mass to halo mass relation combined with the framework describing merger rates of dark matter haloes, e.g., \cite{fakhouri2010}, it becomes possible to use a future detection rate from LISA to place constraints on the astrophysical parameters (denoted by $f_{\rm bh}$, the occupation fraction, $\epsilon_0$, the normalization and $\gamma$, the slope  described in Sec. \ref{sec:firstbh} and Table \ref{table:constraints}) governing the occupation of black holes in high redshift haloes, similarly to the constraints on astrophysical parameters from baryonic gas occupation of haloes. It can be shown \cite{hploeblisa2020c} that for three different confidence scenarios, each assumed to have 100, 200 or 400 events per year for a survey of a 5 year duration, the above parameters can be constrained to percent or sub-percent accuracy over $z \sim 1-5$ (and out to $z \sim 8$, Fig. \ref{fig:multimessenger}). Even before the advent of LISA, gravitational waves from Pulsar Timing Arrays (PTAs) such as those with the Parkes and the European Pulsar Timing Array (EPTA, \cite{babak2016}) have the potential to provide us with exquisite constraints on the nature of the earliest supermassive black holes. It is possible to identify the potential host galaxies of these systems using LSST on the Vera Rubin Observatory, which is expected to detect $\sim 100-200$ galaxies with black hole masses above $10^{6.5} M_{\odot}$. Considering the gas accretion emission from the merger leads to similar conclusions, reaching about 1-1000 electromagnetic counterparts \cite{kocsis2006}. The above results thus provide a holistic, multi-messenger view into the first galaxies. 



We are thus on the brink of significantly advancing our understanding of baryonic cosmology over a nearly 13 billion year timescale, and developing novel techniques that can be extended to other investigations of the early universe. It enables us to combine the largest available datasets of gravitational-wave, radio, millimeter and optical observations to create the most detailed models and simulations of baryonic gas and galaxies, both for the epoch of reionization and the post-reionization universe.


\begin{figure}
\begin{center}
\includegraphics[width =0.7\textwidth]{contour_array_fixedz8_forreview.pdf}
\end{center}
\label{fig:multimessenger}
\caption{Constraints on the astrophysical parameters in the black hole mass - halo mass relation (the occupation fraction $f_{\rm bh}$, and $\epsilon_0$ and $\gamma$ defined in Table \ref{table:constraints}) from observations of LISA events at $z \sim 8$. Contours indicate 1- and 2-$\sigma$ confidence levels assuming a fiducial 5-year LISA survey having 200 events per year. Figure adapted from \cite{hploeblisa2020c}.}
\end{figure}
