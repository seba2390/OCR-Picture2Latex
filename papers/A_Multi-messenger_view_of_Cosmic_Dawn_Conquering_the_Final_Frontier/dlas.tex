\section{Outlook for the future}
\label{sec:outlook}
In this concluding section, we will summarize some open challenges and recent developments in the areas we have described above, and indicate the theoretical and observational outlook for the future.

\subsection{Unravelling the nature of DLAs}


As we have seen in Sec. \ref{sec:igm}, the intergalactic medium is primarily composed of regions having a column density (in \HI) of $10^{12} - 10^{17}$ cm$^{-2}$.  Above this limit, the hydrogen gas becomes self-shielded to the ionizing radiation, and  
regions that have hydrogen column densities higher than $10^{20.3} \ {\rm cm}^{-2}$ form systems called Damped Lyman Alpha systems (DLAs). These are known to be the highest reservoirs of atomic (hydrogen) gas at intermediate redshifts \cite{wolfe1986, lanzetta1991, storrielombardi2000, gardner1997, prochaska2005} between $z \sim 2$ and $\sim 5$, and our current understanding of the distribution of \HI\ comes from their  observations (Sec. \ref{sec:analytical}) in the spectra of high-redshift quasars \cite{noterdaeme09, noterdaeme12, prochaska09, prochaska2005, zafar2013}.
As the primary reservoirs of neutral gas fuel for the formation of stars at lower redshifts, they are thus the progenitors of today's  galaxies.

Nearly fifty years since they were first discovered \cite{lowrance1972, beaver1972},  a precise understanding of the nature of DLAs still remains elusive. Identifying the host galaxies associated with the absorbers lends clues to their host halo masses and other properties. The presence of the bright background quasar may make the direct imaging of DLAs difficult (though this may also be a function of the impact parameter of observation, e.g., \cite{mackenzie2019}. Imaging surveys for DLAs \cite{fynbo2010, fynbo2011, fynbo2013, bouche2013, rafelski2014, fumagalli2014} thus far point to evidence for DLAs arising in the vicinity of faint, low star-forming galaxies due to the absence of high star-formation rates or luminosities in the samples. There is also an observed lack of high-luminosity galaxies in the vicinity of DLAs, which also points to evidence for DLAs being associated with dwarf galaxies at high redshifts \cite{cooke2015}. The results of simulations find DLAs to arise in host haloes of masses $10^9 - 10^{11} M_{\odot}$ at redshift $z \sim 3$ \cite{pontzen2008, tescari2009, fumagalli2011, cen2012, voort2012, bird2013, rahmati2014}. As far as their structure is concerned, DLAs have been modelled to arise as rotating disks \cite{prochaska2010}, though protogalactic clumps \cite{haehnelt1998} are also consistent with their observed properties. Interestingly, in the redshift regime of relevance to DLAs ($z \sim 2-5$), the statistical halo model framework described in Sec. \ref{sec:dlahimodels} matched to the data assigns an equal likelihood to both these possibilities \cite{hparaa2017}.
 
The cross-correlation of DLAs and the Lyman-$\alpha$ forest from the  twelfth Data Release (DR12) of the Baryon Oscillations Spectroscopic Survey
(BOSS) from the Sloan Digital Sky Survey III (SDSS-III) led to the interesting result \cite{fontribera2012}, that the bias parameter of DLAs -- which is a measure of how strongly the DLAs are clustered  -- is somewhat larger ($b_{\rm DLA} \sim 1.9$) than predicted by the standard evolution expected from lower redshifts, $b_{\rm DLA} \sim 1.5-1.8$ \cite{hptrcar2016}, using the results in \eq{bdla}. Follow-up analyses \cite{perez2018} have measured $b_{\rm DLA} \sim 1.5-2.5$, also finding an increase in the bias with metallicity. 

 The observed high bias $b_{\rm DLA}$ may be consistent with the imaging results if the DLAs arise from dwarf galaxies which are satellites of massive galaxies \cite{fontribera2012}. Theoretically, such a value could arise in models in which the neutral hydrogen in shallow potential wells is depleted, leading to the possibility of very efficient stellar feedback \cite{barnes2014}. However, it is difficult to reconcile models having very strong feedback with the low-redshift observations of \HI\ bias and abundance. 
 The bias value is a crucial pointer to the mass of dark matter haloes hosting the high-$z$ DLA systems, and helps shed light on the (as-yet unsolved) question of the nature of DLAs at high redshifts. It is thus interesting to investigate whether there is scope for placing constraints on the scale- and/or metallicity-dependences \cite{digioia2020} of the clustering using future data. 
 Radio surveys targeting the 21-cm line in \textit{absorption} against bright background quasars (for which the physics follows a description similar to that outlined in Sec. \ref{sec:igm} but does not lead to line saturation), may preferentially pick up larger spiral galaxies and thus be sensitive to more luminous galaxies. The First Large Absorption Survey of Hydrogen (FLASH; Ref. \cite{allison2020}) using the ASKAP telescope, and future surveys with the Canadian Hydrogen Intensity Mapping Experiment \cite{yu2014} aim to look for and localize 21 cm analogs of DLA systems, thus providing a complementary view on the origin of DLAs.
