\subsection{Fundamental physics from the Cosmic Dawn}\label{s:applications}

Including redshifts up to the Cosmic Dawn in a fully data-driven formalism of baryonic gas and galaxies opens up the exciting possibility of constraining Fundamental Physics from the best combination of the data we have. Most importantly, it facilitates realistically addressing these questions, for the first time, from a thorough understanding of the astrophysics. This is crucial since an inadequate modelling of the astrophysics in surveys can indicate false inconsistencies in different data sets and lead us to incorrect conclusions, like e.g., new physics. Executing this objective would thus be the most ambitious and groundbreaking theme of exploration in the coming years.  \cite{weltman2020} explores ways in which the SKA will deliver unprecedented constraints that can transform our understanding of fundamental physics.  

At low redshifts, intensity mapping has already been shown to be a powerful probe of dark energy \cite{chang10}; the acoustic oscillations in these maps may constrain dark energy out to high redshifts \cite{wyithe2008a}. 21 cm cosmology allows a unique test of the (possible) variation of fundamental constants (such as the fine structure constant and electron mass) across space and time, due to its ability to probe a huge range of redshifts up to Cosmic Dawn ($z \sim 30$). In particular, it was shown that such variations are constrainable at the level of 1 part in 1000 with the upcoming SKA \cite{lopez2020}. The recent EDGES signal of 21-cm at Cosmic Dawn has been interpreted as an indication of dark matter being possibly composed of axions \cite{sikivie2019}; in general, the imprint of axion dark matter on the epoch of reionization has been explored in \cite{carucci2019}. Another interesting prospect for fundamental physics from the Cosmic Dawn is to constrain the running of the spectral index of the matter power spectrum with an interferometer (of size 300 km) probing up to the dark ages\cite{weltman2020}.

There are signs to indicate that the primordial gravitational wave background can place constraints on non-Gaussianity at early times using future interferometers such as LISA and the SKA \cite{unal2019}. In \cite{cai2019}, it was found that if dark matter consists of primordial black holes with asteroid masses, the accompanying gravitational waves occur at mHz frequencies and are detectable by LISA irrespective of the local non-Gaussianity. It is also possible to place limits on the fraction of dark matter in different sub-species, such as axions \cite{unal2020}, by using the spin of black holes in Active Galactic Nuclei which emit radiation in the [OIII] forbidden line transition. Similarly, inflationary gravitational waves can be probed through the circular polarisation of the 21-cm line (e.g., \cite{mishra2018}). An important use of the gravitational wave measurements from standard sirens is to constrain the value of the Hubble parameter and its evolution with cosmic time, shedding light into the presently unresolved problem of the discrepancy between the local and CMB measurements of the Hubble parameter (e.g., \cite{Riess_2019}).

The intrinsic dipole of the large-scale structure distribution, as traced by galaxies, and its potential deviation from that of the CMB is an important test of the isotropy of the universe, one of the cornerstones of the standard cosmological model. Recent work \cite{nadolny2021} shows that it may be possible to measure this dipole independently of the dipole associated with the speed and direction of our motion,  using a large catalog of radio sources such as measured with a SKA-like radio survey in the coming years. This is an important probe of fundamental physics which can be refined with future large datasets.

The signatures of the first black holes in the gravitational wave, millimetre and radio wavebands can be used to combine this with an understanding of the co-evolution of the first black holes and galaxies at the epoch of Cosmic Dawn. This will also provide insights into a hitherto unresolved problem in cosmology: how did the first black holes form, and what was their contribution to reionization?

Finally, the baryonic gas and galaxy parametrizations, together with the latest theoretical insights into the formation of the first structures at Cosmic Dawn, can be used to investigate the physics constraints on the nature of dark matter, dark energy, theories of gravity and the Universe’s earliest moments, that one can glean from the largest fraction of the Universe observed till date. There is already tremendous promise in the fact that wider surveys (such as the CHIME and the SKA) are able to achieve much tighter constraints on cosmological parameters, even fully alleviating the degradation caused by the astrophysics. With the advances in theoretical understanding of physics at its most extreme — such as close to the first supermassive black holes — we can be expected to gain insights into constraining a quantum theory of gravity.




