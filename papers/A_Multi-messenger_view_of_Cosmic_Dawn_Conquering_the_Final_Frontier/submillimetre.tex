\subsection{The sub-millimetre regime}
\label{s:submmhalomodel}

Although hydrogen is the most abundant element in the Universe, there are several exciting prospects for making intensity maps of other salient lines, an important example being molecular lines, like the carbon monoxide (CO) lines introduced in Sec. \ref{sec:submmintro}. CO is the second most abundant molecule in the universe (after molecular hydrogen) and much easier to detect from ground-based experiments. CO behaves as a tracer of molecular hydrogen, which has no permanent dipole moment due to symmetry and thus no rotational transitions of its own.  CO lines are thus the primary way to trace molecular gas within and outside our galaxy, and very sensitive to the spatial distribution of star formation \cite{hploebsfr2020}. The CO line corresponding to the transition between rotational quantum numbers $J$ and $J - 1$ has a rest frequency of approximately $J \times$115.27 GHz, making it an ideal target in the sub millimetre regime. It is easy to separate the signal from contaminants due to its multiple emission lines (with different values of the angular momentum quantum number $J$) which have a  well defined frequency relationship, a feature that is not available to other tracers. This also enables effective cross-correlation between observations at lower and higher frequency bands which are integral multiples of each other (e.g., a  frequency band covering 26-34 GHz will be sensitive to the CO 2-1 line at $z \sim$ 6-8, while also capturing the CO 1-0 transition at $z \sim$ 2-3.) 



The CO Mapping Array Project (COMAP) aims to detect the CO molecule in emission during the epoch of Galaxy Assembly, about two billion years after the Big Bang. Science observations with the COMAP Pathfinder began in 2019 using a 19-pixel 26-34 GHz receiver mounted on a 10.4 metre dish at the  Owens Valley Radio Observatory.  It was shown \cite{li2015} using simulations, that the COMAP experiment is capable of providing close to 8$\sigma$ constraints on the CO intensity power spectrum at large scales. In November 2021, the COMAP Pathfinder released the first direct 3D measurement of the CO power spectrum on large scales \cite{cleary2021}, nearly an
              order of magnitude improvement compared to the previous
              best measurement \cite{keating2020}. 

Observations made using the Karl G. Jansky Very Large Array (JVLA) and the Atacama Large Millimeter Array (ALMA) have recently shown that line emission from the CO transitions will be bright even at high redshift, $z > 6$.  The levels of foreground contamination in a CO survey are also much lower than for many other types of line intensity mapping, making it a promising target for ground-based observations. Recently, the CO Power Spectrum Survey [COPSS; Ref. \cite{keating2016}] detected, for the first time, the aggregate CO intensity in emission from galaxies at the peak of the star formation history of the universe. The mmIME experiment recently \cite{keating2020} announced the detection of unresolved intensity from the aggregate CO (3-2) emission in galaxies at $z \sim 2.5$ in the shot noise regime. In \cite{pullen2018}, confirmed by \cite{yang2019}, there was a tentative detection of the 158 micron line of [CII], an excellent tracer of star formation, reported, for the first time, by combining Planck CMB maps with  quasars from the BOSS and CMASS galaxy surveys.




By developing a framework that can incorporate current observational constraints on the abundances and clustering of the tracers, we can readily use the wealth of upcoming submillimetre observations to constrain galaxy evolution. Similar to the case of \HI, the main observable in the case of submillimetre intensity mapping observations is the power spectrum. However, in contrast to the \HI\ case, the luminosity of the tracer is normally used instead of the mass, so the prescription to be modelled is, e.g.,  $L_{\rm CO} (M,z)$ in the case of CO. Another difference is that the modelling of sub-millimetre intensity mapping usually focuses on the linear regime alone (since the data typically does not constrain the behaviour of the profile of the tracer). Instead of modelling the  full one-halo term, what is typically modelled is the contribution from the \textit{shot noise} alone, which is analogous to the term introduced briefly in the previous section in \eq{shotnoiseHI}.



To compute the power spectrum, we use the specific intensity of a submillimetre line  observed at a frequency, $\nu_{\rm obs}$, given by:
\begin{equation}
 I(\nu_{\rm obs}) = \frac{c}{4 \pi} \int_0^{\infty} dz' \frac{\epsilon[\nu_{\rm obs} (1 + z')]}{H(z') (1 + z')^4}
\end{equation} 
in which $\epsilon[\nu_{\rm obs} (1 + z')]$ is known as the volume emissivity of the emitted line.
It is usually assumed that the profile of each line is a delta function \footnote{The effects of line broadening on the intensity and power spectra are analytically treated in Ref. \cite{chung2021lb}.} at the rest frequency $\nu_{\rm em}$. This implies that the emissivity can be expressed as an integral of the host halo mass $M$:
\begin{equation}
 \epsilon(\nu, z) = \delta_D(\nu - \nu_{\rm em}) (1 + z)^3 f_{\rm duty} \int_{M_{\rm min}}^{\infty} dM \frac{dn}{dM} L(M,z)
 \label{emissivity}
\end{equation} 
where $L (M,z)$ is the luminosity of the line under consideration, and it is assumed that a fraction $f_{\rm duty}$ (usually called the `duty-cycle' factor) of all haloes above a mass $M_{\rm min}$ contribute to the observed emission  in the line of interest, e.g., \cite{lidz2011}. This parameter can be approximated by $f_{\rm duty} = t_s/t_H$, where $t_s$ is the star formation timescale and $t_H$ is the Hubble time at the redshift under consideration. 
It is to be noted that both of these parameters are fairly poorly constrained by the data;  $f_{\rm duty}$ in particular is found to differ by more than an order of magnitude between different models \cite{pullen2013, keating2016}. For this reason, $f_{\rm duty}$ is alternatively taken into account by introducing intrinsic scatter parameters \cite{li2015} to account for the differences in the star formation activity of haloes.

Using \eq{emissivity}, the specific intensity can be rewritten as:
\begin{equation}
I(\nu_{\rm obs}) = \frac{c}{4 \pi} \frac{1}{\nu_{\rm em} H(z_{\rm em})}  f_{\rm duty} \int_{M_{\rm min}}^{\infty} dM \frac{dn}{dM} L(M,z)
\label{COspint}
\end{equation} 
where $z_{\rm em}$ is the redshift of the emitting source.
Just as in the case of \HI,
the brightness temperature, $T$  can be derived from the specific intensity through the relation $I(\nu_{\rm obs}) = 2 k_B \nu_{\rm obs}^2 T /c^2$. 
From this, the expression for the brightness temperature becomes:
\begin{equation}
T(z) = \frac{c^3}{8 \pi}\frac{(1 + z_{\rm em})^2}{k_B \nu_J^3 H(z_{\rm em})} f_{\rm duty} \int_{M_{\rm min}}^{\infty} dM \frac{dn}{dM} L(M,z)
\label{tco}
\end{equation} 
The clustering of the submillimetre sources  can be modelled in analogy with the \HI\ case by weighting the dark matter halo bias by the tracer luminosity-halo mass relation.
The expression for the clustering is therefore given by:
\begin{equation}
 b_{\rm submm}(z) = \frac{\int_{M_{\rm min}}^{\infty} dM (dn/dM) L (M,z) b(M,z)}{\int_{M_{\rm min}}^{\infty} dM (dn/dM) L (M,z)}
\end{equation} 
in which we see that $L(M,z)$ has now taken the place of $M_{\rm HI}(M,z)$ in \eq{biasHI}.

The shot noise contribution to the power is expressed as:
\begin{equation}
 P_{\rm shot}(z) = \frac{1}{f_{\rm duty}}\frac{\int_{M_{\rm min}}^{\infty} dM (dn/dM) L (M,z)^2}{\left(\int_{M_{\rm min}}^{\infty} dM (dn/dM) L (M,z)\right)^2}
\end{equation} 
The total power spectrum of the intensity fluctuations is the sum of the clustering (two-halo) and shot-noise components:
\begin{equation}
 P_{\rm submm}(k,z) =  T (z)^2 [b_{\rm submm}(z)^2 P_{\rm lin}(k,z) + P_{\rm shot}(z)]
 \label{submmpower}
\end{equation} 
in which $P_{\rm lin}(z)$ denotes the dark matter power spectrum calculated in linear theory. \footnote{\eq{submmpower} assumes that the power is measured in units of ${\rm K}^2$, alternatively, it can directly be measured in units of Jy/sr$^2$ (as commonly done for the cases of [CII] and [OIII]) in which case the intensity in \eq{COspint} is  used: $P_{\rm submm}(k,z) = I(\nu_{\rm obs})^2 b_{\rm submm}^2(z) P_{\rm lin}(k,z) + P_{\rm shot}(z)$.}
Several times,  we need to plot the power spectrum in logarithmic $k$-bins, which is given by:
\begin{equation}
 \Delta_{k}^2(z) = \frac{k^3  P_{\rm HI/submm}(k,z)^2}{2 \pi^2}
 \label{COpowspeclog}
\end{equation} 
 which is equally valid for both the HI power spectrum in \eq{onehalo} and \eq{twohalo}, and the submillimetre one in \eq{submmpower}.
 
To model the main observable for submillimetre intensity mapping, the abundance matching technique (introduced at the end of Sec. \ref{sec:21cmgal}) is found to be suitable for both CO and [CII] at $z \sim 0$, due to the availability of the luminosity functions at low redshifts for both these tracers \cite{hemmati2017, keres2003}, which is denoted by $\phi(L)$ and measures the number density of CO- or CII-luminous galaxies in logarithmic luminosity bins.\footnote{The data do not constrain a non-monotonic behaviour of the luminosity-halo mass relations, so this is a reasonable approach.} Specifically, \eq{abmatchhi} gets modified to:
\begin{equation}
    \int_{M(L)}^{\infty} \frac{dn}{d \log_{10} M'} d \log_{10} M' = \int_{L}^{\infty} \phi(L') d \log_{10} L'
    \label{abmatchsubmm}
\end{equation}
to find $L(M,z = 0)$, which is then inserted into the framework above and propagated to high redshifts using an appropriate functional form, whose parameters are matched to the observations. 

The CO luminosity from galaxy surveys is usually measured in units of K km/s pc$^2$. This quantity is related to  the observed flux density of CO, and its linewidth, by the relation \cite{solomon2006}:
\begin{equation}
 L_{\rm CO} = 3.3 \times 10^{13} S \Delta v (1 + z)^{-3} \nu_{\rm obs}^{-2} D_L^2 \ \rm{K \  km/s \  pc}^2
\end{equation} 
with $D_L$ being the luminosity distance to the source in Gpc, $\nu_{\rm obs}$ the observed frequency in GHz, $S$ the flux density  in Jy, and $\Delta v$ the velocity width  in km/s. For the case of CO, a double power law form relating $L_{\rm CO}$ to halo mass is found to be a good fit to the data at low and high redshifts:
\begin{equation}
    L_{\rm CO} (M, z) = 2N(z) M [(M/M_1(z))^{-b(z)} + (M/M_1(z))^{y(z)}]^{-1}
\end{equation}
with the best fitting parameters and their uncertainties given in Table \ref{table:constraints}. This form is identical to the corresponding form of the abundance matched stellar mass to halo mass relation, found in data-driven approaches \cite{behroozi2010, behroozi2019, moster2010} and summarized in Table \ref{table:constraints}. 

For the case of [CII], a power law with an exponential cutoff matches the low-redshift data well, and the evolution to high redshifts is assumed to follow that of the star formation rate (since [CII] is found to be highly correlated with the star-formation rate, e.g., \cite{knudsen2016}) which is fitted utilizing a data driven procedure in \cite{behroozi2013, behroozi2019}. The relation is thus expressed as
\begin{equation}
 L_{\rm CII}(M,z) = \displaystyle{\left(\frac{M}{M_1}\right)}^{\beta} \exp(-N_1/M) \displaystyle{\left(\frac{(1+z)^{2.7}}{1 + [(1+z)/2.9)]^{5.6}} \right)^{\alpha}}   
\end{equation}
with the best fitting parameters  summarized in Table \ref{table:constraints}.
For [OIII], a fit to available observations of individual galaxies at high redshifts \cite{harikane2020} leads to a parametrization of the luminosity directly in terms of the star formation rate (Table \ref{table:constraints}).


 The advantages of using the above data-driven relations are manifold. A special feature of baryonic line emission power spectra (which can be seen from expressions like \eq{onehalo} and \eq{twohalo}) is that they
depend on the underlying cosmology as well as on the astrophysics of the  systems, and hence can offer constraints on both these aspects. The astrophysics acts as an effective ‘systematic’ uncertainty when making cosmological predictions from such surveys. At the same time, the astrophysical parameters themselves contain valuable information about the role of gas in galaxy formation and evolution. Using the latest available data to constrain the parameters of this framework therefore allows us to  precisely separate both these aspects. Furthermore, the models' computational simplicity allows us to include the effects of several additional parameters, easily vary the physics and cosmology to explore different scenarios, and to impose the most realistic priors (by definition, based on all the data available today) on the astrophysics conveniently within a Fisher matrix analysis. This has advantages for both cosmological and theoretical physics avenues, as it easily accounts for extensions beyond $\Lambda$CDM and is thus uniquely suited to extract cosmological constraints from baryonic data  and make predictions for future surveys \cite{hparaa2019}. We describe how this is done in the forthcoming sections.

