
\subsection{Observational outlook}
\label{s:results}

One of the most significant challenges for the  detection of a 21 cm signal from the Cosmic Dawn is that of the overwhelming foregrounds. The foreground problem is typically broken into three independent components – Galactic synchrotron, that contributes around 70\% of the total foreground emission \cite{shaver1999}; extragalactic sources which contribute about 27\% \cite{mellema2013}; and Galactic free-free emission which comprises the remaining $\sim$ 1\%. Altogether, these foregrounds are expected to dominate the 21 cm signal brightness temperature by up to 5 orders of magnitude, though this figure reduces to 2–3 when considering the angular brightness fluctuations \cite{bernardi2009}. Furthermore, each source is expected to occupy a different region of angular-spectrum space, e.g., \cite{chapman2016}. A review of recent developments and challenges on the 21 cm data analysis and software fronts is provided in Ref. \cite{liu2020a}.

Current research indicates that foreground removal techniques show a great deal of promise in extracting the 21 cm signal without biases but may introduce larger uncertainties in the measurement of cosmological parameters. This is especially the case for the primordial non-Gaussianity forecasts, since these are degenerate with the large-scale modes which are preferentially washed out by the foregrounds. Recently, it was shown that a multi-parameter fit to the data and foregrounds, taken together, recovers the cosmological parameters extremely well and does not significantly bias the results \cite{fonseca2020}, which thus brings in a very promising outlook for studies of primordial non-Gaussianity. Also, very recently, machine learning tools have been employed to demonstrate the possibility of removing the foregrounds in a satisfactory manner to recover the shapes and sizes of ionised bubbles at the epoch of reionization from SKA and the Hydrogen Epoch of Reionization Array (HERA) images, and found to be robust to instrumental effects \cite{gagnon2021}. It is important to note that foregrounds are not a major problem for line-intensity mapping with other lines -- such as the CO line, which is primarily dominated by point source foregrounds \cite{keating2016} which are about three orders of magnitude larger than the signal, but can be effectively mitigated due to their spectral flatness.

The above main challenge is, however, more than balanced by one of the richest rewards in the domain of cosmology and astrophysics: to be able to capture, for the first time, the evolution of the Universe almost up to the beginning of the dark ages, closing the gap between the local universe and the CMB, approaching an enormous amount of information contained in nearly $10^{15}$ independent Fourier modes, representing the largest cosmological dataset.
 On a broader scale, the data science and machine learning tools needed for achieving the goals are also at the forefront of several research directions today. 

