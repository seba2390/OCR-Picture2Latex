
\section{Introduction}\label{s:intro}

The last few decades have witnessed several rapid advances in the field of observational cosmology, with  probes of the universe over a wide range of wavelengths from the radio to the X-ray bands, as well as — recently —in the gravitational wave regime. The wealth of observational data has led to the development of precision cosmology, whereby the cosmological parameters are measured to unprecedented accuracy with the available data. The fact that all the independent observational constraints are consistent with the `standard model' of theoretical cosmology, known as $\Lambda$CDM, is one of the most impressive successes of the theory. 

However, the standard model of cosmology also leaves us with several challenging problems. About 70\% of the present-day energy density of the universe is in the form of ‘dark energy’, which behaves like a smooth fluid having negative pressure, and leads to the accelerated expansion of the universe. This component is consistent with a cosmological constant term introduced in Einstein’s equations. Another 25\% is in the form of ‘dark matter’ — which does not interact with radiation, but participates in gravitational clustering. We do not have a physical understanding of (or laboratory evidence for) either of these two components, which together comprise about 95\% of the present-day universe. Only the remaining approximately 5\% of the energy density is in the form of ordinary matter (with a tiny amount in the form of radiation), whose physical properties are familiar to us. 

At the earliest epochs, radiation decoupled from the neutral baryonic matter in the first major phase transition of the observable universe known as the epoch of recombination, which occurred about 300,000 years after the Big Bang, and is observable today as the cosmic microwave background (CMB). Most of the baryonic material in the universe was (and is) in the diffuse plasma and gas between galaxies, known as the intergalactic medium (IGM) -- with a very small amount ($\sim 10\%$) in stars, as has been roughly the case throughout the timescale of the observable Universe \cite{shull2012, famaey2012}.\footnote{If it is sobering to note that we are not made up of the most abundant component of the Universe, namely dark energy, it is perhaps even more humbling to find that stars and stellar systems -- i.e., the baryonic material we are most familiar with -- did not even represent the majority of the cosmological \textit{baryons} over all epochs!}  The baryonic matter in the universe was almost fully hydrogen [with small ($\sim$ 10\%) amounts of neutral helium], in its electrically neutral, gaseous form (known as neutral hydrogen, hereafter referred to as HI, as commonly done in the literature) at the end of the epoch of recombination, and remained so (during a period known as the dark ages of the universe) until the first stars and galaxies formed about a few hundred million years later. These luminous sources contributed ionizing photons to complete the second major phase transition in the observable universe known as cosmic reionization. In this process, the radiation from starlight was primarily responsible for ionizing the hydrogen in the universe – and this period, which lasted for about a few hundred million years, that immediately followed the dark ages is referred to as the Cosmic Dawn. Mapping the period between Cosmic Dawn and the present-day provides access to more than 90\% of the Universe's baryonic (normal) matter, and unlocks almost all the information in cosmological baryons.

It is therefore obvious that our most accurate understanding of cosmology and fundamental physics in the future will come only through the study of the baryonic gas tracers of the Universe, the majority of which lie between the local universe probed by galaxy surveys, and the CMB surface of last scattering. This allows access to almost three orders of magnitude more independent Fourier modes of information than currently available from the combination of galaxy surveys (which extend only to the edge of the local universe), and the CMB.  In the last several years, a tremendous effort to revolutionise our understanding of cosmology from baryonic gas is bearing fruit. Studies of neutral hydrogen in the intergalactic medium at the epoch of Reionization and Cosmic Dawn are chiefly probed by the 21 cm spin-flip transition in the radio band. After reionization, neutral hydrogen exists in the form of dense clumps in galaxies, and extremely dense systems known as Lyman-limit systems and Damped Lyman-Alpha systems (DLAs). The carbon monoxide (CO) molecular abundance also offers exciting prospects for placing constraints on the global star-formation rate, which is observed to peak around 2 billion years after the Big Bang \cite{lilly1996, madau1998}. The technique of intensity mapping (IM) has emerged as the powerful tool to explore this phase of the Universe by measuring the integrated emission from sources over a broad range of frequencies, providing a tomographic, or three-dimensional picture of the Universe. Notably, this is being complemented by large scale efforts to probe the first galaxies and black holes by using the combined power of the electromagnetic and gravitational wave bands. We are thus on the threshold of the richest available cosmological dataset in the coming years, facilitating the most precise constraints on theories of Fundamental Physics. 

This review will address the major milestones in the field of cosmology and astrophysics with baryonic tracers. We begin (in Sec. \ref{sec:astrocosmo}) with an overview of the astrophysical and cosmological background needed for the results (readers familiar with cosmology and structure formation could skip Sec. \ref{sec:introcosmo}). In the next section, we provide a detailed overview of state-of-the-art techniques for modelling neutral gas occupation in dark matter haloes, with emphasis on hydrogen [both in galaxies, Sec. \ref{sec:21cmgal}, and in Damped Lyman Alpha systems (DLAs), Sec. \ref{sec:dlahimodels}]. We then (Sec. \ref{s:submmhalomodel}) illustrate how this framework can be carried over to the sub-millimetre regime, exploring the carbon monoxide (CO) and singly ionized carbon [CII] line transitions. We then describe how these models can be combined with the latest available data to constrain the best-fitting values of the parameters (Sec. \ref{s:constraints}) and subsequently can be used to (Sec. \ref{s:forecasts})  forecast constraints on cosmology and astrophysics, particularly on physics beyond the standard $\Lambda$CDM cosmological model. Finally, we describe (Sec. \ref{sec:crosscorrelations}) how the different probes of baryonic tracers can be effectively combined to produce cross-correlation forecasts, significantly increasing the sensitivity of the detections and offering a holistic view into the epoch of Cosmic Dawn.  We summarize the theoretical and observational outlook, and discuss open problems and future challenges in a concluding section (Sec. \ref{sec:outlook}). 

\section{Overview of cosmology and structure formation}
\label{sec:astrocosmo}
In this section, we provide a review of cosmology, structure formation and the key milestones in the evolution of baryonic material. We also discuss the major observational probes of gas within and around galaxies.
\subsection{The cosmological background}
\label{sec:introcosmo}

Current observations indicate that our universe is homogeneous and isotropic on the largest scales, to the level of better than 1 part in $10^5$. 
According to the standard model of cosmology (the hot Big Bang model), the homogeneous and isotropic universe is described by the Friedmann-Robertson-Walker (FRW) metric of the form:
\begin{equation}
ds^2 = -c^2 dt^2 + a(t)^2 \left[\frac{dr^2}{1 - Kr^2} + r^2(d \theta^2 + \sin^2 \theta d \phi^2)\right]                                                                                  \end{equation} 
where $K$ is a constant which determines the curvature and can take the values $0, 1, -1$, $a(t)$ is called the cosmic scale factor, and $r, \theta, \phi$ are comoving spherical coordinates centered on the observer. The Einstein field equation relates the geometry of this spacetime to the underlying matter-energy content (including the vacuum energy, taken to be a cosmological constant term $\Lambda$ throughout this review):
\begin{equation}
 R_{ij} - \frac{1}{2} g_{ij} R + g_{ij} \Lambda = \frac{8 \pi G}{c^4} T_{ij}
\end{equation} 
In the above equation, $R_{ij}$ is the Ricci curvature tensor of the spacetime, $g_{ij}$ is the metric, $R$ is the Ricci scalar,  $T_{ij}$ is the energy-momentum tensor of the matter content of the universe, which has the form $T^i_j = {\rm diag}(-\rho c^2, p, p, p)$ with $\rho c^2$ being the energy density and $p$ being the pressure.  With this, 
 the Einstein equation for the FRW metric can be shown to have the form:
 \begin{equation}
 \left( \frac{\dot{a}}{a} \right)^2 = \frac{8 \pi G}{3} \rho - \frac{K c^2}{a^2} + \frac{\Lambda c^2}{3}
 \end{equation} 
 The above equation is known as the Friedmann equation. 
 Throughout this review,  we will work with the flat universe in which the curvature component $K = 0$.
 A source located at the physical distance $l = a(t) r$ from an observer moves at a velocity  $v = dl/dt = \dot{a} l = \dot{a} r/a(t)$ (due to the expansion of the universe, assuming that the source does not have a peculiar velocity of its own). If we define 
 the Hubble parameter $H(t) = \dot{a} (t)/ a(t)$ at the cosmic time $t$, we  obtain the relation $v = H(t) r$, which is known as Hubble's law. Radiation emitted by a source at time $t$ is observed today (at time $t_0$) with a redshift $z = a(t_0)/a(t) - 1$.
 The radiation and matter content of the universe are usually parametrized in terms of the dimensionless numbers, $\orr$ and $\om$, which are defined as
\begin{equation}
 \orr \equiv \frac{8\pi G \rho_R(t_0)}{3 \oh^2}; \quad \om \equiv \frac{8\pi G \rho_m(t_0)}{3 \oh^2}
\end{equation} 
Here $t_0$ is the current age of the universe, $H_0\equiv(\dot a/a)_{t_0}$ is the present value of the Hubble parameter, and $\rho_R(t_0), \rho_m(t_0)$ are the energy densities of radiation and matter in the universe at $t= t_0$. 
In terms of these components of the universe, the Friedmann equation can be expressed as:
\begin{equation}
 \frac{\dot a^2}{a^2} = H_0^2 \left[ ( 1 - \Omega_R - \Omega_m)+ \frac{\Omega_R a_0^4}{a^4} + \frac{\Omega_m a_0^3}{a^3}\right]
\label{a1}
\end{equation} 
where the symbols $\orr$ and $\om$ refer to density parameters at the present epoch. The present value of the Hubble parameter, $H_0$, is usually specified in terms of the dimensionless number $h$, as $H_0 = 100 h$ km s$^{-1} {\rm Mpc}^{-1}$. 

The earliest epoch of the universe consisted of an inflationary phase characterized by rapid exponential expansion; subsequently, the universe entered the radiation dominated phase. The quantum-mechanical vacuum fluctuations during inflation generated small inhomogeneities that manifested as perturbations in the matter-radiation components of the universe.  The formation of light nuclei occurred when the universe had cooled to a temperature $T_{\rm nuc} \sim 0.1$ MeV. The contributions to the baryonic mass were roughly $\sim 75\%$ hydrogen and $25\%$ helium nuclei, with very small amounts of deuterium, lithium and other light elements.  The transition from the radiation to matter (consisting mostly of dark matter) dominated epoch occurred around redshift $z \sim 3300$. Around $z \sim 1100$, the electrons and protons recombined to form neutral atoms, when the temperature was around $T_{\rm rec} \sim 0.3$ eV. This was also the epoch at which the ambient radiation decoupled from the baryonic matter.
The primordial density perturbations, which had the fractional amplitude of $\sim 10^{-5}$ grew over time in the matter dominated epoch of the universe. The perturbations collapsed to form gravitational bound objects known as haloes, which were initially formed on small scales. 
In the simplest model of spherical collapse, an overdense region expands more slowly as compared to the background, reaches a maximum radius, turns around and contracts and virializes to form the bound halo.
In a flat matter-dominated universe,
this model leads to the result that the density of the collapsed halo is about a factor $18 \pi^2 \approx 178$ times the background density at the time of collapse. 
 In a universe with $\om + \Omega_{\Lambda} = 1$, the expression for the overdensity of the virialized structure is well-described by the fitting formula:
 \begin{equation}
  \Delta_v \approx 18 \pi^2 + 82 d - 39 d^2
 \end{equation} 
 where $d = \Omega_m(z) - 1$, with \cite{bryan1998}:
 \begin{equation}
  \Omega_m(z) - 1 = \Omega_m (1+z)^3/(\Omega_m (1+z)^3 + \Omega_{\Lambda}) - 1
 \end{equation} 
 By solving the equation of motion of collapse and using the virial theorem,  we can obtain the characteristic radius and virial velocity of the collapsed structures, and relate them to the mass and the redshift of collapse.
 Halos of mass $M$ collapsing at redshift $z (\gg 1)$ are found to have the virial radius:
   \begin{equation}
 R_v (M) = 46.1 \ {\rm{kpc}} \  \left(\frac{\Delta_v \Omega_m h^2}{24.4} \right)^{-1/3} \left(\frac{1+z}{3.3} \right)^{-1} \left(\frac{M}{10^{11} M_{\odot}} \right)^{1/3}
 \label{virialradius}
 \end{equation} 
 and the corresponding virial (or circular) velocity of the halo:
 \begin{equation}
  v_c = \left(\frac{G M}{R_v(M)}\right)^{1/2}
  \label{virvel}
 \end{equation} 
 Even though the collapse of an overdense region is characterized by nonlinear gravitational growth, it is also useful in several contexts to define the quantity \textit{linearly extrapolated overdensity} $\delta_c(z)$ at every redshift. This corresponds to the overdensity predicted by linear theory and extrapolated into the nonlinear epoch. This linearized overdensity, extrapolated to the present, of an collapsed object at redshift $z$ (in the matter-dominated era) can be shown to be given by $\delta_c(z) \approx 1.686(1 + z)$.
 The theories of structure formation also predict the abundance of dark matter haloes i.e. the number density of haloes as a function of mass, at any redshift. The Press-Schechter model, based on the idea of spherical collapse, predicts the comoving number density of dark matter haloes to be given by:
 \begin{equation}
  n(M) = \frac{\rho_m}{M} \frac{d \ln \sigma^{-1}}{d M} f(\sigma)
 \end{equation} 
 where $\sigma(M)$ is the variance of the initial density field (assumed to be a Gaussian random field) on mass scale $M$, and 
 \begin{equation}
f(\sigma) = \sqrt{\frac{2}{\pi}} \frac{\delta_c(z)}{\sigma} \exp\left(-\frac{\delta_c^2}{2 \sigma^2} \right)
\end{equation}  
 where $\delta_c(z)$ is the linearly extrapolated overdensity at the collapse of the halo. 
 Detailed numerical calculations indicate that ellipsoidal, instead of spherical collapse, may be a better description to the abundance of haloes and has led to the Sheth-Tormen (ST) mass function \cite{sheth2002}:
 \begin{equation}
 f(\sigma) = A \sqrt{\frac{2 a}{\pi}} \frac{\delta_c(z)}{\sigma} \exp\left(-\frac{a \delta_c^2}{2 \sigma^2} \right) \left[1 + \left(\frac{\sigma^2}{a \delta_c^2}\right)^p \right]
 \end{equation} 
 where $a = 0.707, p = 0.3$, and $A = 0.322$. 
 
 The clustering of the dark matter haloes 
 reflects the effects of gravitational growth and
 describes the deviation from the smooth universe. This clustering is well described by the bias function $b(M,z)$ which specifies the clustering of the dark matter haloes as a function of halo mass.
 In the Press-Schechter model,  the bias function takes the form:
 \begin{equation}
b_{\rm PS}(M)  = 1 + \frac{\nu_c^2 - 1}{\delta_{c} (z = 0)}   
\end{equation} 
where $\nu_c = \delta_c(z)/\sigma(M)$. In the Sheth-Tormen formalism, the corresponding expression is given by \cite{scoccimarro2001}:
\begin{equation}
 b_{\rm ST}(M)  = 1 + \frac{a \nu_c^2 - 1}{\delta_{c} (z = 0)} + \frac{2 q/\delta_{c} (z = 0)}{1 + (a \nu_c^2)^q}
\end{equation} 
It is to be noted that the very small haloes can become ``antibiased'' $(b_{\rm ST} < 1)$ at late times, because these tend to form in underdense regions (the overdense regions being already taken up by more massive haloes).

To a good approximation, the dark matter haloes may be modelled as spherical objects. Numerical simulations indicate a near-uniform radial density profile $\rho(r)$ for the radial structure of the dark matter haloes. One of the widely used forms is known as the Navarro-Frenk-White (NFW) profile. In this model, the dark matter haloes are described as spheres with the density distribution:
\begin{equation}
 \rho(r) = \frac{\rho_0}{(r/r_s)(1 + (r/r_s))^2} 
\end{equation} 
where $r_s$ is known as the scale radius, and $\rho_0$ is a characteristic density. This profile is found to be a good model of dark matter haloes over a range of masses in cold dark matter cosmologies. The scale radius is related to the virial radius defined in \eq{virialradius}, by $r_s = R_v(M)/c(M,z)$ where the concentration parameter, $c(M,z)$ is given by:
\begin{equation}
    c(M,z)  = \frac{9}{1+z} \left(\frac{M}{M_*(z)}\right)^{-0.13}
    \label{concparamdm}
\end{equation}
and $M_*$ is defined as the halo mass at which $\nu_c \equiv \delta_c(z)/\sigma(M) = 1$.

The \textit{halo model} of dark matter clustering unifies the ingredients of the halo distribution above (the mass function, clustering, and density profiles) into a framework for computing the statistical properties of the collapsed objects even into the nonlinear regime. The model is found to work extremely well when compared to the results of numerical simulations. In the halo model, the parameter $\rho_0$ in the profile is fixed by normalizing the total mass of dark matter in the halo:
\begin{equation}
    M = 4 \pi \int_0^{R_{v}} \rho(r) dr 
\end{equation}
where it is assumed that the halo is truncated at its virial radius. Given this parametrization, the clustering of the haloes is computed by defining the Fourier transform of the profile (normalized to unity on large scales):
\begin{equation}
    u(k|M) = \frac{4 \pi}{M} \int_0^{R_v} dr \ r^2 \frac{\sin kr}{kr}{\rho(r)}
    \label{udm}
\end{equation}
in terms of which the \textit{power spectrum} of density fluctuations can be defined as:
\begin{equation}
    P(k,z) = P_{1h} (k,z) + P_{2h} (k,z)
\end{equation}
having two terms, the \textit{two-halo} term, $P_{2h} (k,z)$ describing correlations between particles in different haloes, and the \textit{one-halo} term, $P_{\rm 1h}(k,z)$ describing the structures within each halo:
\begin{eqnarray}
    P_{1h}(k, z) &=& \int dM \  n(M, z) \left(\frac{M}{\bar{\rho}}\right)^2 |u(k|M)|^2 \nonumber \\
     P_{2h}(k, z) &=& P_{\rm lin} (k) \int dM \ \left[ \left(\frac{M}{\bar{\rho}}\right) b(M,z)  n(M, z)  |u(k|M)|\right]^2
     \label{powerspecdm}
\end{eqnarray}
where $P_{\rm lin}$ is the power spectrum computed  in the linear theory, and $\bar{\rho}$ is the mean dark matter density, defined as:
\begin{equation}
    \bar{\rho} = \int dM n(M,z) M 
\end{equation}



\subsection{Evolution of baryons}
The baryons in the very early universe were in the form of a hot, ionized plasma which was tightly coupled to the ambient radiation. At a redshift of $z_{\rm rec} \sim 1100$,  when the universe had expanded and cooled enough, the protons and electrons in the plasma "re"-combined to form neutral hydrogen atoms. At this stage, the ambient radiation decoupled from the newly formed neutral atoms.
After the recombination epoch, the gas in the universe was predominantly neutral (to better than 1 part in $10^4$) and the Universe entered the ``dark ages'' which were characterized by the absence of radiation (other than the relic CMB radiation). For the first hundreds of millions of years after the recombination, the baryons in the Universe were mostly in the form of neutral hydrogen ($\sim 93\%$ by number) and small amounts of neutral helium ($\sim 7\%$ by number), with negligible amounts of heavier elements.

As the dark matter haloes formed by gravitational clustering, the baryonic material (hydrogen, with small amounts of helium) was pulled into the potential wells created by the dark matter haloes. The speed of infalling gas is of the order of the virial velocity $v_c$ of the halo, \eq{virvel}. As the gas cools and becomes Jeans unstable, i.e. the total gas mass exceeds the Jeans mass, it is able to form stars.  The first stars thus formed in the universe were massive, luminous and metal-free,  and termed as Population III stars. In the $\Lambda$ CDM models, these stars had masses $\sim 10^5 M_{\odot}$ and formed at $z \gtrsim 30$. However, the majority of the matter today -- and at the early epochs of first stars -- was outside these structures, and resided in the intergalactic medium (IGM) -- which is the term used to describe the baryonic material lying between galaxies and not directly associated with dark matter haloes.  At sufficiently early epochs, of course, all the baryons in the universe resided in the IGM. Today, only about 10\% of the baryons are locked up in luminous structures and the remaining are believed to reside in the IGM (with a small fraction in the intra-cluster medium of galaxies and the circumgalactic medium surrounding galaxies). The IGM provides the fuel for star and galaxy formation, interacts with the radiation emitted by the galaxies, and is enriched by the gas and metal outflows from galaxies. Being fairly less affected by the nonlinear astrophysical processes associated with galaxy formation, it offers a much clearer view of the underlying cosmology, the cosmological changes since recombination, and the physics of structure formation and clustering. 

As soon as the first luminous sources formed, they started to emit UV radiation and ionize the hydrogen gas in the IGM back into free electrons and protons -- this process is referred to as cosmological reionization. Both the temperature and the ionization state of the gas were affected by the reionization process. The epoch of reionization, which was the epoch at which the majority of hydrogen got ionized, thus represents the second major phase transition (the first being recombination) of the baryonic material in the universe. 




\subsection{Cosmic reionization}
\label{sec:reion}

The epoch of reionization is associated with the end of the ``dark ages'' of the universe. The majority of the baryonic material in the universe is hydrogen, and hence ``reionization'' usually refers to the process of hydrogen reionization.  Various probes are used to determine the epoch of hydrogen reionization in the universe (for reviews on reionization, we refer to Refs. \cite{loeb2001, barkana, lidz2016rev, fan2006, choudhury2006, choudhury2009, zaroubi2013, natarajan2014, ferrara2014}):

(a) The Thomson scattering optical depth of the cosmic microwave background offers an integrated probe of the free electron density between the present epoch and the epoch of reionization. If reionization is assumed to be a sharp transition from fully neutral to fully ionized IGM, occurring at redshift $z_{\rm ri}$, the expression for the Thomson optical depth along a line-of-sight is:
\begin{equation}
 \tau_{\rm Th} = \int_0^{z_{\rm ri}} n_e(z) \ \sigma_e \  dz \left|\frac{dl(z)}{dz}\right|
\end{equation} 
For the standard $\Lambda$CDM cosmology, the above integral can be done analytically and leads to an expression in terms of the standard cosmological parameters and the primordial helium abundance \cite{venkatesan2008}. In the high-redshift limit of $\Omega_m (1+z)^3 \gg \Omega_{\Lambda}$, the expression simplifies to \cite{moandwhite}:
\begin{equation}
 \tau_{\rm Th} = 0.07 \left(\frac{h}{0.7}\right) \left(\frac{\Omega_{b}}{0.04}\right) \left(\frac{\Omega_{m}}{0.3}\right)^{-1/2} \left(\frac{1 + z_{\rm ri}}{10}\right)^{3/2}
\end{equation} 
where $\sigma_e$ is the Thomson scattering cross-section and $n_e(z)$ is the comoving number density of electrons at redshift $z$ along the sightline. 
 The WMAP temperature and polarization measurements of optical depth ($\tau_e = 0.087 \pm 0.017$) pointed to the (instantaneous) redshift of reionization being about $z_{\rm ri} \sim 11.0 \pm 1.4$ \cite{dunkley2009}.  The recent Planck measurements \cite{planck2015}, on the other hand, favour a later epoch of completion of reionization ($z \sim 6-10$, Ref. \cite{mitra2019, robertson2015}) due to the smaller measured value of the Thomson optical depth ($\tau_e = 0.066 \pm  0.016$). In reality, reionization is not a sharp transition and hence the optical depth measurement only provides an estimate of the average reionization epoch.
 
(b) A second probe of the epoch of reionization is the Gunn-Peterson optical depth of redshifted Lyman$-\alpha$ photons in the spectra of high-redshift quasars.\footnote{ {Strictly speaking, the term `quasar' is commonly used  for describing radio-loud quasi-stellar objects. However, as frequently done in the literature,  we will use the terms `quasars' or QSOs interchangeably to indicate quasi-stellar objects throughout this review, irrespective of their radio properties.}}
 
 The hydrogen Lyman-$\alpha$ line, which denotes the transition from the $n = 1$ to $n = 2$ electronic states, has been a powerful probe of hydrogen in cosmology, in the spectra of astrophysical objects such as quasars, galaxies and gamma-ray bursts (GRBs). The rest wavelength of the line is $\lambda_{\alpha} = 1216$ \AA. Photons from a cosmological source, whose wavelength is less than 1216 \AA\ are redshifted into the Lyman-$\alpha$ wavelength as they propagate through the IGM. These can be absorbed by a hydrogen atom and re-emitted in a different direction. The optical depth of the IGM can thus be probed by the flux decrement in the spectra of high-redshift luminous sources. 

The optical depth from absorption of redshifted Lyman-$\alpha$ photons (with $\Lambda_{\alpha}$ being the decay rate) from a uniform neutral hydrogen distribution is given by \cite{gunnpeterson}: 
\begin{eqnarray}
 \tau(z) &=& \frac{3 \Lambda_{\alpha} \lambda_{\alpha}^3 x_{\rm HI} n_{\rm H} (z)}{8 \pi H(z)} \nonumber \\
 &\sim & 1.6 \times 10^5 x_{\rm HI} (1 + \delta) \left(\frac{1+z}{4}\right)^{3/2}
\end{eqnarray} 
assuming standard values for the cosmological parameters, the Hubble parameter $H(z)$ evaluated during the matter-dominated era, and $n_{\rm H}(z) = \bar{n}_{\rm H}(z) (1 + \delta)$ where $\bar{n}_{\rm H}$ is the mean cosmic hydrogen density. 
Here, $x_{\rm HI}$ is the neutral fraction of hydrogen. We thus see that a neutral fraction of even about 1 part in $10^4$ (i.e. $x_{\rm HI} \sim 10^{-4}$) would give rise to complete Lyman-$\alpha$ absorption. Any transmission, therefore, implies a highly ionized (better than 1 part in $10^4$) IGM. At moderate redshifts ($z < 5.5$), \textit{no  complete absorption signatures} (known as the Gunn-Peterson absorption troughs) are seen in the spectra of quasars. Hence, it is concluded that the IGM must be very highly (re)ionized upto these redshifts. 

Recent results based on the spectra of quasars above $z \sim 6$ show the first hints of Gunn-Peterson absorption troughs indicating the possible transition to the neutral IGM. However, because of the small neutral fraction required to saturate the Gunn-Peterson trough, it is difficult to place constraints on the level of ionization of the IGM at these redshifts. Hence, this data alone does not constrain the history of reionization, i.e. whether reionization was a sharp transition occurring around redshift 6 or a fairly gradual transition beginning from a much higher redshift.  The latest studies combining the CMB measurement with the spatial fluctuations in the IGM optical depth \cite{kulkarni2019, nasir2020, raste2021, davies2016} may indicate a late, rapid end to reionization occurring around $z \sim 5.3$, however, these fluctuations could also arise due to variations in mean free paths of hydrogen atoms in an  ultraviolet background dominated by galaxies \cite{davies2016}, or due temperature fluctuations in the IGM \cite{dalonsio2015} in more extended reionization scenarios ending at $z \sim 6$.

(c) The hydrogen 21-cm spin-flip line is a third important probe of the cosmic reionization. This line is inherently weak, and hence does not get completely saturated even over the early-to-mid stages of reionization. Being a line transition, it also provides a three-dimensional probe of the universe (two dimensional surface information at every frequency). Since the sensitivity extends to the Jeans' scale of the baryonic material, it is unaffected by Silk damping and hence gives access to many more modes than the CMB. The 21-cm transition is described in further detail in Sec. \ref{sec:21cmintro}.

Apart from the probes described above, studies of Lyman-$\alpha$ emitters (LAEs) (for a review, see Ref. \cite{ouchi2019}) and $\gamma$-ray bursts (GRBs), e.g., \cite{totani2006, kistler2009, ishida2011,robertsonellis2012, lidz2021a} and Lyman-Break Galaxies (LBGs, Ref. \cite{kashino2020}) are also useful probes of reionization. Studies of the high-redshift LAE and LBG data may also favour a late reionization scenario, e.g., \cite{choudhury2015, mesinger2015, kashino2020}. 


\subsection{The first black holes}
\label{sec:firstbh}

The period of Cosmic Dawn also overlapped with the appearance of the first black holes, whose formation mechanism is considered one of the most outstanding challenges in cosmology today. Hence, gaining insights into their properties by means of their signatures on the environment is a complementary tool to the study of the intergalactic medium. The black hole masses are related to host halo mass through the black hole mass - bulge mass relation; (for a review, see, e.g.,  \cite{kormendy2013}).
The circular velocity of the dark matter halo, $v_{c}$, is found to tightly correlate with the velocity dispersion of the galaxy, which can then be used to relate the mass of the black hole to that of the halo, through:
\begin{equation}
    M_{\rm BH} \propto v_c^{\gamma} 
\end{equation}
with $\gamma \sim 4-5$, e.g., \cite{wyithe2002}. For the host galaxies, data-driven formalisms, e.g., \cite{behroozi2010} find the stellar mass to evolve as a broken power law with respect to the host halo mass:  $M_* \propto M^{2.3}$
when $M \leq 10^{12} M_{\odot}$, and 
$M_* \propto M^{0.29}$
for $M
\geq 10^{12} M_{\odot}$. These treatments can thus be used to relate the black hole mass to the host halo mass by substituting for the circular velocity from \eq{virvel} and introducing an overall normalization factor, $\epsilon_0$ \cite{wyithe2002}:
\begin{equation}
    M_{\rm BH} (M) = M \epsilon_0 \displaystyle{\left(\frac{M}{10^{12} M_{\odot}}\right)^{\gamma/3 - 
1}} \left(\frac{\Delta_v \Omega_m h^2}{18 \pi^2}\right)^{\gamma/6} 
(1+z)^{\gamma/2} 
\end{equation}

Massive black holes in binary systems during the epoch of reionization may be detected via the gravitational waves that they emit in the milliHertz to nanoHertz regime, with upcoming facilities such as the Pulsar Timing Arrays (PTAs) and the Laser Interferometer Space Antenna (LISA).  This allows the inference of several properties of the black holes, such as the masses of the binary members, the rough sky location and the source luminosity distance (which can be converted into an equivalent redshift for an assumed cosmology). This information complements the effect of the black hole and its host galaxy (or quasar) on the intergalactic medium, such as the ionization of the gas, and hence can be used as a powerful complementary probe of the evolution of reionization.

\subsection{The intergalactic medium : quasar absorption lines}
\label{sec:igm}

As we have seen, the optical depth of the IGM to photons from quasars provide useful clues to the timing and average redshift of reionization. The absorption lines in the spectra of quasars are also effective probes of the physical state of the IGM. The evolution of the IGM over a large timescale can be studied with the quasar absorption lines, since the quasars are observed out to fairly high redshifts, $0 < z < 5$.
 Neutral hydrogen clouds, that occur at different positions along the line-of-sight from the observer to the distant quasar, absorb Lyman-$\alpha$ photons at frequencies corresponding to the redshifts of the individual clouds. 

The absorption line profiles are expressed using the Voigt function which accounts for both the Doppler (thermal, with some turbulent) broadening which is dominant in the central region of the line, and the natural broadening which is dominant in the wings (for a review, see \cite{djikstra2014}). The Voigt profile is a convolution of these two mechanisms:
\begin{equation}
 \phi(\nu) = \int_{-\infty}^{\infty} P(v) L[\nu(1 - v/c)] \ d v
\end{equation} 
where $P(v)$ is the Maxwellian distribution of the velocities of the absorbing atoms:
\begin{equation}
P(v) \ dv = \frac{1}{\sqrt{\pi b^2}} \exp\left(-\frac{v^2}{b^2}\right) \ dv
\end{equation} 
and $b$ is the Doppler parameter of the gas and includes contributions of both thermal and turbulent motions:
\begin{equation}
 b^2 = \frac{2 k_B T}{m} + b_{\rm turb}^2
\end{equation} 
The line profile, $L$, occurs due to the natural line broadening in accordance with the Heisenberg uncertainty principle and the finite lifetime of the excited state. It has the form of a Lorentzian centered around the frequency of the transition $\nu_{\alpha}$:
\begin{equation}
 L(\nu) = \frac{1}{\pi} \frac{\gamma}{(\nu - \nu_{\alpha})^2 + \gamma^2}
\end{equation} 
where $\gamma = A_{21}/4\pi$, $A_{21}$ being the spontaneous transition coefficient. 

The Voigt profile can thus be expressed as:
\begin{equation}
 \phi(\nu) = \frac{1}{\sqrt{\pi}} \frac{c}{b} \frac{V(\nu)}{\nu}
\end{equation}
where
\begin{equation}
V(\nu)  = \frac{A}{\pi}  \int_{-\infty}^{\infty} dy \ \frac{e^{-y^2}}{(B - y)^2 + A^2} 
\end{equation}
with $A = c\gamma/b \nu$; $B = c(\nu - \nu_{\alpha})/b \nu$.
The Voigt profile can be approximated to lowest order as:
\begin{equation}
 V(\nu) \approx \exp(-B^2) + \frac{1}{\sqrt{\pi}} \frac{A}{A^2 + B^2}
\end{equation} 
The above expression clearly separates the thermal broadening (near the center of the profile) and natural broadening (near the wings of the profile). 

The above discussion was for the case of a single absorption line profile. The numerous neutral hydrogen clouds along the line-of-sight to a luminous source (such as a quasar) typical produce their own line profiles.
The collective set of absorption lines produced by the neutral hydrogen clouds from the low-to moderate density IGM (with typical overdensities $\Delta \lesssim 10$) along the line-of-sight to a luminous source such as a quasar,  is known as the Lyman$-\alpha$ forest. 
The forest is believed to arise from the baryonic fluctuations during hierarchical clustering \cite{cen, zhang, miralda, hernquist}, and hence is a probe of the sheets, filaments and voids in the cosmic web. After reionization, an ionizing  background radiation is established, and the gas in the Lyman$-\alpha$ forest is in photoionization equilibrium with the background. 

The column densities (i.e., number densities measured along the column of the line-of-sight) of the Lyman-$\alpha$ forest lines are in the range $10^{12} - 10^{17}$ cm$^{-2}$. 
Clouds having column densities higher than this are known as Lyman-limit systems (LLSs), which are optically thick to the ionizing radiation.  If the column densities are above $10^{20.3}$ cm$^{-2}$, the systems have prominent damping wings from the natural line broadening, and are therefore classified as Damped Lyman Alpha systems (DLAs). Absorption systems having column densities between $10^{19}$ to $10^{20.3}$ cm$^{-2}$ are termed as sub-Damped Lyman Alpha systems (sub-DLAs).

The number density of IGM absorbers in a column density interval $(N_{\rm HI}, N_{\rm HI} + d N_{\rm HI})$ and in a redshift interval $(z, z+dz)$ is given by  ${d^2 \mathcal{N}/d N_{\rm HI} dz}$, where $\mathcal{N}$ is the observed number of absorbers. A quantity known as the column density distribution is frequently used in observational studies, denoted as:
\begin{equation}
 f_{\rm HI} (N_{\rm HI}, X) = \frac{d^2 \mathcal{N}}{d N_{\rm HI} dX} 
 \label{coldens}
\end{equation} 
measured with respect to the interval $X$, defined through $dX/dz = H_0 (1+z)^2/H(z)$. 



Once the distribution function is known, the comoving density of neutral hydrogen may be computed as:
\begin{equation}
 \rho_{\rm HI} (z) = \frac{m_H H_0}{c} \int f_{\rm HI}(N_{\rm HI}, z) N_{\rm HI} d N_{\rm HI}
\end{equation} 
where $m_H$ is the hydrogen mass,
and this is typically expressed in terms of the critical density at $z = 0$:
\begin{equation}
 \Omega_{\rm HI} (z) = \frac{\rho_{\rm HI} (z)}{\rho_{c,0}}
\end{equation} 



\subsection{The 21-cm transition}
\label{sec:21cmintro}

As we have seen above, absorption of the Lyman-$\alpha$ radiation against a luminous source (such as a bright quasar or a bright GRB afterglow at high redshifts) allows us to probe the number density of neutral hydrogen atoms. The Lyman$-\alpha$ transition is therefore an excellent probe of the low column density, partially ionized gas in the post-reionization universe. However, as discussed in Sec. \ref{sec:reion}, due to the large optical depth of the Lyman-$\alpha$ absorption (the Einstein A-coefficient is very large, $A_{{\rm{Ly}} \alpha} \sim 10^{7}$ s$^{-1}$) the IGM is opaque to Lyman-$\alpha$ absorption even if its neutral fraction is as small as $f_{\rm HI} \sim 10^{-4}$, which happens above $z \sim 6$. 

Using the 21-cm transition of neutral hydrogen provides complementary information about the evolution of the baryonic material. This transition takes place between the two hyperfine states corresponding to parallel and antiparallel spins of the proton and electron in the hydrogen atom. 
In contrast to the Lyman-$\alpha$ transition, the 21-cm line is strongly forbidden (the Einstein A-coefficient is $A_{10} = 2.85 \times 10^{-15} s^{-1}$ which corresponds to a lifetime $\sim 10^8$ years). Hence, the line is inherently weak, and difficult to observe in the laboratory. However, it is used extensively in  astrophysical contexts to study the hydrogen gas in and around the Milky Way and other galaxies. The inherent weakness of the line transition also prevents the saturation of the line, thus enabling it to serve as a direct probe of the neutral gas content of the intergalactic medium during the dark 
ages and cosmic dawn prior to hydrogen reionization. At later epochs, over the last 12 billion years (redshifts 0 to 5), the 21-cm line emission of HI is expected to trace the underlying dark matter distribution, due to the absence of the complicated reionization astrophysics \cite{bharadwaj2001a, bharadwaj2001, bharadwaj2004, wyithe2008, wyithe2009, bharadwaj2009, wyithe2010} and since most of the neutral hydrogen at these redshifts ($z \sim 2-5$) resides in the high column-density systems which are preferentially probed by this line. 

 The 21-cm emission line also allows for the measurement of the intensity of fluctuations across frequency ranges or equivalently across cosmic time, thus making it a three-dimensional, or tomographic, probe of the universe. This is because at every frequency interval, one has access to a two-dimensional surface. This 2D surface, along with the frequency (or equivalently, redshift) being the third ``axis'', gives us the three-dimensional information. The combination of angular and frequency structure allows us to use the 21-cm line for  tomography, or mapping almost 90\% of the baryonic material from Cosmic Dawn to the present day (e.g.,\cite{loeb2013}). This is a much larger comoving volume than galaxy surveys in the visible band, and consequently promises a much higher precision in the measurement of the matter power spectrum and cosmological parameters. Since the power spectrum can be probed to the Jeans' length of the baryonic material, it allows a sensitivity to much smaller scales than those allowed by the CMB.




The singlet state (denoted by subscript 0, and characterized by antiparallel spins of the proton and electron) and the triplet state (denoted by subscript 1, and characterized by parallel spins of the proton and electron), are separated by an energy difference of $5.9 \times 10^{-6}$ eV, corresponding to the temperature $T_{10} \approx 0.068$ K (and a wavelength of 21 cm). In equilibrium, the relative populations of these two states follow:

\begin{equation}
 \frac{n_1}{n_0} = \left(\frac{g_1}{g_0}\right) \exp\left[-\frac{T_{10}}{T_s}\right]
\end{equation} 
where the $g_1/g_0 = 3$ and $T_s$ is known as the spin temperature and characterizes the thermal equilibrium between the two states. 

 When the spin temperature is far greater than $T_{10} = 68$ mK (which corresponds to the energy difference between the two levels), we can approximate the previous equation as:
 \begin{equation}
n_1 \approx 3 n_0 \approx \frac{3 n_{\rm HI}}{4}        
\end{equation} 
where $n_{\rm HI}$ is the total number density of neutral hydrogen.

The intergalactic 21-cm line is typically observed in emission or absorption against the CMB, which can excite the hydrogen atoms from the singlet to the triplet state. The spin temperature of the neutral gas also changes due to collisional excitation and de-excitation processes and/or  Lyman-$\alpha$ coupling (the Wouthuysen-Field process, \cite{wouthuysen1952, field1958}). The expression for the spin temperature is given by:
\begin{equation}
 T_s^{-1} = \frac{T_{\rm CMB}^{-1} + x_c T_{K}^{-1} + x_{\alpha} T_{\alpha}^{-1}}{1 + x_c + x_{\alpha}}
\end{equation} 
where $T_K$ is the kinetic temperature of the gas, $T_{\rm CMB}$ is the CMB temperature, and $T_{\alpha}$ is known as the effective  Lyman-$\alpha$ color temperature, defined through the relation:
\begin{equation}
 \frac{P_{01}}{P_{10}} = 3 \left(1 - \frac{T_{01}}{T_c}\right)
\end{equation} 
where $P_{01}$ and $P_{10}$ are the spin excitation and de-excitation rates, respectively, from Lyman-$\alpha$ absorptions. In the high redshift IGM,  $T_c \to T_K$ due to the large number of Lyman-$\alpha$ scatterings. The coefficients $x_c$ and $x_{\alpha}$ are the coupling factors of the collisional and Wouthuysen-Field processes, respectively.


At very early epochs, the spin temperature tends to be in equilibrium with the ambient CMB temperature. When the gas density is high, collisions couple the spin temperature to the kinetic temperature. During the dark ages and reionization, the interplay of different processes such as heating of the gas, coupling of the gas temperature and the spin temperature, and the evolving ionization of the gas change the frequency structure of the signal (for a detailed review of the various processes and their impact on the signal, see Ref. \cite{furlanettorev}). This makes the spin temperature a powerful probe of the onset and timing of astrophysical processes during reionization.  
 
 The brightness temperature $T_b(\nu)$ of a radio source (assumed to be a neutral hydrogen cloud located at redshift $z$) in the Rayleigh-Jeans limit (an excellent approximation at these frequencies) can be expressed in terms of its intensity $I_{\nu}$ as:
 \begin{equation}
  T_b' = \frac{I_{\nu} c^2}{2 k_B \nu^2}
  \label{intensitytemperature}
 \end{equation} 
with $k_B$ being Boltzmann's constant. Using the equation of radiative transfer through a neutral hydrogen cloud having the spin temperature $T_s$, the brightness temperature of the cloud
is given by:
\begin{equation}
 T_b'(z) = T_{\rm CMB}(z) \exp(-\tau_{10}) + T_s (1 - \exp(-\tau_{10}))
\end{equation} 
 where $\tau_{10}$ is the optical depth for absorption of CMB photons and subsequent excitation from the singlet to the triplet level. It can be shown \cite{loeb2013, furlanettorev} that the optical depth is related to the number density of neutral hydrogen atoms,  $n_{\rm HI}$ by the expression:
 \begin{equation}
 \tau_{10} = \frac{3}{32 \pi} \frac{h c^2 A_{10}}{k_B T_s (z) \nu_{10}^2}\frac{n_{\rm HI} (z)}{(1+z) (dv_{\parallel}/dr_{\parallel})}
\end{equation} 
where $\nu_{10}$ is the frequency of the transition (1420 MHz) and $h$ is Planck's constant, and $dv_{\parallel}/dr_{\parallel}$ is the gradient of the proper velocity along the line of sight. 
From this, it can be shown \cite{loeb2013, furlanettorev} that the observed differential brightness temperature from a cloud at redshift $z$  against the CMB, given by $T_b(\nu) \equiv T_b'(\nu(1+z))/(1+z)$ (in the limit $\tau_{10} \ll 1$, $z \gg 1$) is given by:
\begin{equation}
 T_b(z)  = \frac{T_s - T_{\rm CMB}}{(1+z)} \tau_{10} \approx 28 \ {\rm{mK}} \left(\frac{\Omega_b h}{0.03}\right) \left(\frac{\Omega_m}{0.3}\right)^{-1/2} \left(\frac{1+z}{10}\right)^{1/2} x_{\rm HI} \left(1 - \frac{T_{\rm CMB}}{T_s}\right)
\end{equation} 
where $x_{\rm HI}$ is the neutral fraction of hydrogen, and the effects of peculiar velocities are neglected.

The above equation is suitable to describe the spin temperature evolution in the Cosmic Dawn and pre-reionization IGM ($z > 10$). From the above expression, we can see that that there is no signal if the spin temperature equals the CMB temperature. If $T_{\rm s} < T_{\rm CMB}$, the signal appears in absorption, and it appears in emission if $T_{\rm s} > T_{\rm CMB}$. A measurement of the brightness temperature was recently reported by the EDGES experiment \cite{bowman2018}) at $z \sim 17$. The Hydrogen Epoch of Reionization Array (HERA) recently \cite{hera2021} constrained the spin temperature of the $z \sim 8$ neutral IGM, placing limits on several physical models of reionization. It was also \cite{garsden2021} shown that the  Owens Valley Long Wavelength Array (OVRO-LWA) has sufficient sensitivity for a 21-cm detection at almost the edge of the Cosmic Dawn window, $z \sim 30$. 

In the post-reionization universe, the absorption is usually neglected as $T_s \gg T_{\rm CMB}$, and the neutral hydrogen fraction, $x_{\rm HI} \ll 1$ is usually  expressed as $x_{\rm HI} = \Omega_{\rm HI} (1+ \delta_{\rm HI})$ 
in terms of the comoving density parameter relative to the present-day critical density, $\Omega_{\rm HI} (z)$. This requires the substitution $x_{\rm HI}(z)  = \Omega_{\rm HI} (z) (1 + \delta_{\rm HI})$ where  $\delta_{\rm HI}$ is the overdensity of \HI. This gives \cite{bull2014, battye2012}:
 \begin{equation}
  T_b(z) = \frac{3 h c^3 A_{10} (1 + z)^2}{32 \pi k_B H(z) \nu_{10}^2}\frac{\Omega_{\rm HI} (z) \rho_{c,0} (1 + \delta_{\rm HI})}{m_H}
  \label{brtemp}
 \end{equation} 
 where where $\rho_{c,0} \equiv 3 H_0^2/8 \pi G$ is the present-day critical density.
\eq{brtemp} allows for the separation of the brightness temperature into a mean and a fluctuating component:
 \begin{equation}
   T_b(z) = \bar{T_b}(z) (1 + \delta_{\rm HI})
 \end{equation} 
and we have:
 \begin{equation}
  \bar{T_b}(z) = 44 \ \mu {\rm K} \left(\frac{\Omega_{\rm HI}(z) h}{2.45 \times 10^{-4}}\right) \frac{(1+z)^2}{E(z)}
  \label{tbar}
 \end{equation} 
 where $E(z) = H(z)/H_0$.
The difference in units between \eq{tbar} and the previous \eq{brtemp} ($\mu$K and mK) effectively reflects the difference in ambient neutral hydrogen densities prior to ($z >10$) and after ($z < 6$) reionization. In the post-reionization regime, it is relevant to study 
 the intensity \textit{fluctuations} of the brightness temperature, given by $\delta T_{\rm HI} = T_b(z) -  \bar{T_b}(z)$.
 In this regime, similar to the dark matter power spectrum, we can thus define the three-dimensional power spectrum of this intensity fluctuation, $P_{\rm HI} \equiv [\delta T_{\rm HI}(k,z)]^2$. We discuss the detailed method to evaluate this starting from the baryon population in haloes in the next section.
 
 \begin{figure}
\vskip-0.2in
 \includegraphics[width = \textwidth]{cross-correlations.pdf}
\caption{Summary of the redshift ranges constrained by the different types of observations.}
\label{fig:summary}
\end{figure}
 
 
\subsection{Molecular gas and the sub-millimetre transitions}
\label{sec:submmintro}
In addition to the 21 cm transition of hydrogen considered above, there are several probes of the reionization and post-reionization epochs using molecular lines, such as those from the carbon monoxide (CO) molecule. CO, being an indicator of molecular hydrogen ($H_2$)  primarily traces star-forming regions \cite{breysse2014, mashian2015}, has a ladder of states covering rest frequencies of approximately $115 (J+1)$ GHz corresponding to its rotational transitions, where $J = 0, 1, ...$ is the quantum number of the transition. CO has been studied extensively from observations of local galaxies and recently also at intermediate redshifts, $z \sim 1-3$ \cite{keating2016, walter2016} and quasars out to as high as $z \sim 7$ \cite{venemans2017}. Accessing CO at the epoch of reionization is expected to offer valuable clues to the star formation and baryon cycle in the earliest galaxies. 

Two other tracers of interest in this regime are the  fine-structure line of singly ionized carbon (denoted by [CII]) with the rest wavelength 158$\mu$m (corresponding to 1.9 THz), which originates from a hyperfine transition between the $^{2}P_{3/2}$ and $^{2}P_{1/2}$
states of singly ionized carbon, and [OIII] line at $88 \mu$m, from the transition between the $^{3}P_{1}$ and $^{3}P_{0}$ states of doubly ionized oxygen.  The CII ion is a dominant coolant for the
interstellar medium, and the [CII] 158 $\mu$m line is the brightest far-infrared line and relatively easy to observe from the ground. The [OIII] line is commonly found in the vicinity of hot O-type stars,  and is senstitive to several physical properties of the ionized medium.  Both [CII] and [OIII] have been detected in galaxies out to $z \sim 9$ using the ALMA telescope \cite{pentericci2016, smit2018, laporte2017, harikane2020, tamura2019, hashimoto2018c, laporte2019, carniani2017}, and efforts are underway to observe them at these epochs without resolving individual galaxies \cite{terry2019, lagache2018, exclaimpaper2020}, the framework for which we describe in Sec. \ref{s:submmhalomodel}.


A schematic summary of all the tracers we consider, and their probes over various redshift ranges (which we discuss in detail in the forthcoming sections) is provided in Fig. \ref{fig:summary}.

