


\section{Learning cosmology from the baryons}\label{s:forecasts}
Over the past decade, the standard model of cosmology (i.e. $\Lambda$CDM, described in Sec. \ref{sec:astrocosmo}), has become fairly well established, with the latest constraints approaching sub-percent levels of accuracy in the measurement of cosmological parameters. However, from a theoretical point of view, several outstanding questions remain to be answered in the context of this model, of which the chief ones are: i) the nature of the late-time accelerated expansion of the Universe,  ii) the mechanism responsible for generating the primordial perturbations that led to structure formation, and (iii) the nature of dark matter. The first phenomenon is often attributed to dark energy, which behaves like a fluid component having negative pressure. Most observations are consistent with dark energy being a cosmological constant as defined in Sec. \ref{sec:astrocosmo}, however, the breakdown of the general theory of relativity on cosmological scales (also known as modifications of gravity) may also explain the observed cosmic acceleration. Modified gravity theories thus have a strong significance in understanding the nature of dark energy.  

The generation of primordial perturbations is believed to be related to an early inflationary epoch of the Universe. The detection of primordial non-Gaussianity, which is characterized by a nonzero value of the parameter $f_{\rm NL}$, places stringent constraints on theories of inflation, since standard single-field slow-roll inflationary models predict negligible non-Gaussianity (e.g., \cite{Maldacena:2002vr}), while non-standard scenarios allow for larger amounts. So far, searches for primordial non-Gaussianity have been through anisotropies in the Cosmic Microwave Background (CMB), or from the clustering of galaxies (e.g. \cite{Giannantonio:2013uqa, castorina2019}).

We are well-placed to study the impact of astrophysics on cosmological forecasts using the halo model treatments developed in the preceding sections. To do so, it is most convenient to \cite{hparaa2019} use a Fisher matrix technique, which enables realistic priors on the astrophysics imposed from the halo model framework. For large area sky surveys (such as those with \HI), the power spectra defined in Sec. \ref{s:HIhalomodel} are first converted into their projected \textit{angular} forms, denoted by $C_{\ell}(z)$ which represent
the observable, on-sky quantities in cosmological mapping surveys. The angular power spectrum can be used to perform a
tomographic analysis of clustering in multiple redshift bins without the 
assumption of an underlying cosmological model, e.g., \cite{seehars2016}. The expression for $C_{\ell}(z)$  can be related to the power spectra defined in Sec. \ref{s:HIhalomodel} and Sec. \ref{s:submmhalomodel} by using the  Limber approximation (accurate to within 1\% for scales above $\ell \sim 10$; e.g. Ref.\cite{limber1953}) that makes use of the angular window function, $W_{\rm HI}(z)$ of the survey:
\begin{equation}
C_{\ell} (z) = \frac{1}{c} \int dz  \frac{{W_{\rm HI}}(z)^2 
H(z)}{R(z)^2} 
P_{\rm HI} [\ell/R(z), z]
\label{cllimber}
\end{equation}
where $H(z)$ is the Hubble parameter at redshift $z$, and $R(z)$ is the comoving distance to redshift $z$. The angular window function is usually assumed to have a top-hat form in redshift space, though other choices are possible.
From the above angular power spectrum, and given an experimental configuration, 
a Fisher forecasting formalism can be used to place constraints on the 
 cosmological [e.g., $\{h, \Omega_m, n_s, \Omega_b, \sigma_8\}$]  and 
astrophysical [e.g., $\{c_{\rm HI}, \alpha, \beta, \gamma, v_{\rm c,0}\}$] 
parameters, which are generically denoted by $A$. The Fisher matrix element 
corresponding to the parameters $\{A,B\}$ and at a redshift bin centred at $z_i$ is then defined as:
\begin{equation}
F_{AB} (z_i) = \sum_{\ell < \ell_{\rm max}} \frac{\partial_A C_\ell (z_i)
\partial_B C_\ell(z_i)}{\left[\Delta C_\ell(z_i)\right]^2},
\end{equation}
where $\partial_A$ is the partial derivative of $C_{\ell}$ with respect to 
$A$. In the above expression, the standard deviation, $\Delta C_{\ell} (z_i)$ is defined 
in terms of the noise of the experiment, $N_{\ell}$ and the sky 
coverage of the survey, $f_{\rm sky}$:
\begin{equation}
\Delta C_\ell = \sqrt{\frac{2}{(2 \ell + 1)f_{\rm sky}}} 
\left(C_\ell + N_\ell\right),\label{eq:Delta_Cl}
\end{equation}
 The full Fisher matrix for an experiment is constructed by summing the individual Fisher matrices in each of the $z$-bins
in the redshift range covered by the survey: 
\begin{equation}
 \mathbf{F}_{AB} = \sum_{z_i} F_{AB}(z_i)
 \label{fisher}
\end{equation}


The above treatment for cosmological forecasting  exactly parallels the corresponding one for the CMB, e.g., \cite{bond1997}, with the additional appearance of the baryonic gas parameters. Note that the Fisher matrix approach assumes that the individual parameter likelihoods are approximately Gaussian, which may not always be the case if the parameters are strongly degenerate. In the present case, however, it is found \cite{hparaa2019} that a full Markov Chain Monte Carlo treatment leads to negligible differences from that obtained by the Fisher matrix technique.

For sub-millimetre surveys that typically cover only a few square degrees of the sky, the three-dimensional power spectrum $P_{\rm submm}(k)$ defined in \eq{submmpower} is directly used to compute the Fisher matrix, which is written as:
\begin{equation}
F_{AB} (z_i) = \sum_{k < k_{\rm max}} \frac{\partial_A P_{\rm submm}(k, z_i)
\partial_B P_{\rm submm}(k,z_i)}{\left[\sigma_P(k,z_i)\right]^2},
\end{equation}
The variance of the power spectrum is denoted by $\sigma_P^2(k, z_i)$ and is computed as:
\begin{equation}
    \sigma_P^2 = \frac{(P_{\rm submm}(k, z_i) + P_{\rm N})^2}{{N_{\rm modes}(k)}}
    \label{varianceauto}
\end{equation}
where $P_{\rm N}$ is the noise power spectrum, analogous to $N_{\ell}$ in \eq{eq:Delta_Cl} above, and $N_{\rm modes}$ is the number of Fourier modes probed by the survey, defined as:
\begin{equation}
    N_{\rm modes} = 2 \pi k^2 \Delta k \frac{V_{\rm surv}}{(2 \pi)^3} 
\end{equation}
in which $V_{\rm surv}$ is the volume of the survey and $\Delta k$ denotes the spacing of the $k$-bins.

To complete the treatment, we need to consider the experimental noise in various configurations which defines  $N_{\ell}$ and $P_{\rm N}$ above.  The way the noise is calculated differs according to the configuration in which the facility is constructed and used. For HI, the experimental configurations can be roughly divided into three categories: (i) single dish telescopes, such as the Five Hundred Metre Aperture Spherical Telescope (FAST, \cite{smoot2017}), BAO in Neutral Gas Observations (BINGO, \cite{battye2012}), and the Green Bank Telescope (GBT) and  (ii) dish interferometers, such as the TianLai \cite{chen2012}, the Square Kilometre Array (SKA) and its pathfinders, like the Meer-Karoo Radio Telescope (MeerKAT, \cite{santos2017}) and the Australian SKA Pathfinder (ASKAP, \cite{johnston2008}), the Hydrogen Intensity and Real-time Analysis eXperiment (HIRAX, \cite{newburgh2016}) and cylindrical configurations, like the Canadian Hydrogen Intensity Mapping Experiment (CHIME, \cite{bandura2014}) and the planned CHORD (Canadian Hydrogen Observatory and Radio transient Detector) \cite{liu2019} facilities. For submillimetre lines, planned or already-fielded instruments include interferometer arrays, such as the Sunyaev-Zeldovich Array (SZA) \cite{keating2016}, as well as single dish facilities such as COMAP\footnote{http://comap.caltech.edu} and CCAT-p \cite{terry2019, parshley2018}, in addition to balloon-based experiments such as EXCLAIM \cite{exclaimpaper2020}. A summary of the noise expressions for the various configurations relevant to these experiments is provided in Table \ref{table:noise}. 



\begin{longtable}{l|l|l}
\caption{Expressions for the noise terms $N_{\ell}$ and $P_{\rm N}$ used in the Fisher information matrix to constrain cosmological and astrophysical parameters with the intensity mapping technique at various redshifts and using various tracers. Key to symbols: $\Delta \nu$ : observed frequency interval;  $T_{\rm sys}$: system temperature, $f_{\rm sky}$ : sky fraction covered, $\lambda_{\rm obs}$ : observed wavelength, $\theta_{\rm beam}$ : telescope beam size, $\bar{T}$ : mean temperature of hydrogen, $D_{\rm dish}$ : the dish size, $N_{\rm dish}$ : the number of dishes, $\nu$ : the observed frequency, $W_{\rm cyl}$ : width of the cylinder, the FoV : Field of View, $A_{\rm eff}$ : effective area,
$t_{\rm pix}$ : time observing per pixel, $n_{\rm base}$ : number of baselines, $n_{\rm pol}$ : number of polarization directions, $N_{\rm beam}$ : number of beams, $D_{\rm max/,min}$: maximum and minimum baselines of the interferometer configuration,
   $V_{\rm vox}$: volume of the `voxel' (volume pixel) in submillimetre surveys, $N_{\rm det}$: number of detectors, $\sigma_{\rm N}$: detector noise per voxel,  $t_{\rm vox}$: time spent observing the voxel, and $t_{\rm tot}$ : total observational time.}
\\
\hline
&&\\
\tablehead
{\hline} Experiment   & Noise expression & Reference \\
&&\\
\hline\hline
 & & \\
 {\large \bf  (i) HI single dish}         &   $N_\ell^{\rm HI,dish} = {W _{\rm beam}^2(\ell)}/{2N_{\rm dish}  t_{\rm pix} \Delta \nu} \left({T_{\rm sys}}/{\bar{T}}\right)^2\left({\lambda_{\rm obs}}/{D_{\rm dish}}\right)^2$ & \\
&  & Ref. \cite{camera2020} \\
& $W _{\rm beam}^2(\ell)=\exp\left[{\ell(\ell+1)\theta_{\rm beam}^2}/{8\ln2}\right]$  & \\
   {\large \bf (ii) HI interferometer} & & \\
  {\large \bf in single dish mode}  & $N_\ell^{\rm HI,intSD} = 4 \pi f_{\rm sky} T_{\rm sys}^2/(2 N_{\rm dish} t_{\rm tot} \Delta \nu)$ & \\
  & & \\
 &   $T_{\rm sys} = 25 + 60 \left({300 \ {\rm MHz}}/{\nu}\right)^{2.5}$ & Refs. \cite{knox1995,bull2014,ballardini2019,bauer2021}  \\
{\large \bf (iii) HI cylindrical} & $N_\ell^{\rm HI,cyl} = \displaystyle{\frac{4 \pi f _{\rm sky}}{{\rm FoV} n_{\rm base}(u) n_{\rm pol} N_{\rm beam}  t_{\rm tot} \Delta \nu}} \displaystyle{\left(\frac{\lambda_{\rm obs}^2}{A_{\rm eff}}\right)^2}\displaystyle{\left(\frac{T_{\rm sys}}{\bar{T}}\right)^2};$ &   \\
   &  &  \\
   & & \\
 & ${\rm FoV} \approx \pi/2 \lambda_{\rm obs}/W_{\rm cyl}$; & \\
 & & \\
 & $\Delta \nu = \nu_{\rm HI} \Delta z/(1+z)^2$ & Refs. \cite{camera2020, newburgh2014, jalilvand2019, obuljen2018}  \\
 & & \\
 {\large \bf (iv)  HI interferometer} & $N_{\ell}^{\rm HI,  int} = T_{\rm sys}^2 {\rm FoV}^2/(T_b^2 n_{\rm pol} n(u = \ell/2\pi)  t_{\rm tot} \Delta \nu)$ & \\
 & & \\
 &  $n(u) = N_{\rm dish} (N_{\rm dish} - 1)/(2 \pi (u_{\rm max}^2 - u_{\rm min}^2))$ & \\
 & & \\
 & FoV $= \lambda^2/D_{\rm dish}^2$; \ $u_{\rm max/min} = D_{\rm max,min}/\lambda$ & Refs. \cite{bauer2021, pourtsidou2016, bull2014}  \\
 & & \\
{\large \bf (v) CO} & $P_{\rm N} = V_{\rm vox} \sigma^2_{\rm vox}$ &  \\
& & \\
 & $\sigma_{\rm vox} = T_{\rm sys}/(N_{\rm det} \Delta \nu t_{\rm vox})^{1/2}$ & Ref. \cite{liu2021} \\
 & & \\
{\large \bf   (vi) CII/OIII IM} &  $P_{\rm N} = V_{\rm vox} \sigma_{\rm N}^2/t_{\rm vox}$ & Ref. \cite{hpcii2019} \\
& &
 \label{table:noise}
 \\
\hline
\end{longtable}
Given the Fisher matrix in \eq{fisher}, we can then compute the standard deviation in the measurement of $A$ for the two cases of fixing and marginalizing over the remaining parameters, as:
\begin{equation}
   \sigma^2_{A, \rm fixed} = (\mathbf{F}_{AA})^{-1}; \ \sigma^2_{A, \rm marg} = (\mathbf{F}^{-1})_{AA};
\end{equation}
The Fisher matrix treatment thus captures the effects of `astrophysical degradation' in the precision of cosmological parameter forecasts, and their evolution with redshifts for different experiment combinations. It was found that, in the case of \HI, the astrophysical degradation was largely mitigated by our prior information coming from the present knowledge of the astrophysics \cite{hparaa2019}. It also revealed an important robustness feature of the halo model framework in Sec. \ref{s:HIhalomodel}: that physically motivated extensions to the halo model did not cause significant changes in the forecasts, which was important to consolidate the utility of the halo model framework for constraining theories of fundamental physics.

It is also important to assess the influence of astrophysical uncertainties on the \textit{accuracy} of cosmological parameter forecasts \cite{camera2020}. This can be addressed  by  employing the nested likelihoods framework \cite{Heavens:2007ka}, which is a measurement of how much the uncertainty in our astrophysical knowledge causes a bias in the forecasted cosmological parameters. Specifically, given the Fisher matrix in \eq{fisher}, we
split the space of parameters into two 
subsets: one containing the parameters of interest and the other containing the parameters
deemed `nuisance' or systematic for the analysis under consideration. When constraining cosmology with baryons,
these two sets could represent `cosmological' and `astrophysical' 
parameters respectively. The bias on a given cosmological parameter $A$, 
denoted by $b_{A}$, is then computed as:
\begin{equation}
b_{A} = \delta \mu F_{B\mu}\left(\mathbf 
F^{-1}\right)_{AB}.\label{eq:bias}
\end{equation}
Here, $\mathbf F^{-1}$ is identical to \eq{fisher} and represents the full Fisher matrix of astrophysical and 
cosmological parameters, and $F_{B\mu}$ stands for the particular sub-matrix mixing 
cosmological and astrophysical parameters. The term $\delta \mu$ denotes 
the vector of the shifts between the fiducial and true values of the astrophysical parameters $\mu$:
\begin{equation}
\delta \mu=\mu^{\rm fid}-\mu^{\rm true}.\label{eq:shift}
\end{equation}





This approach thus allows us to quantify the accuracy of cosmological forecasts obtainable from baryonic surveys. The relative bias on various cosmological parameters from a SKAI-MID like survey, obtained by shifting the parameters $v_{c,0}$ and $\beta$ from their fiducial values in Table \ref{table:constraints}, is plotted in Fig. \ref{fig:relbias1}.
As a bonus, it was found that \cite{camera2020} this technique leads to a powerful way of incorporating effects beyond the standard $\Lambda$CDM framework into the halo model. Specifically, we can use this method to investigate a non-zero primordial non-Gaussianity effect imprinted on the power spectrum, and also incorporate modified gravity scenarios which are an important test of Einstein’s general relativity. Encouragingly, it was found that the primordial non-Gaussianity is negligibly affected by astrophysical uncertainties as can be seen from Fig. \ref{fig:relbias1}, which promises an optimistic outlook for one of the strongest science cases for future intensity mapping experiments.

Dark matter (DM) is also a component of the $\Lambda$CDM cosmological model, however, its nature continues to be a mystery. The only dark matter candidate in the standard model of particle physics, the neutrino, is known to make up less than 1\% of the total DM abundance because its relativistic velocity makes it too “hot” to account for the observed structure formation (e.g., \cite{Alam:2016hwk}). Observations  favour the majority of the remaining DM being composed of a single species of cold, collisionless DM (CDM). One candidate for a significant fraction of DM are axions, which occur in  many extensions of the standard model \cite{PecceiQuinn1977, Weinberg1978}. Axions with masses $m_a \sim 10^{-22}$ eV, known as ``fuzzy DM" \cite{Hu:2000ax}, can make up a significant fraction of the DM, and furthermore have a host of interesting phenomenological consequences on galaxy formation (e.g. \cite{Arvanitaki_2010, Marsh_review2016, Niemeyer:2019aqm}). 

Ref. \cite{bauer2021} used the halo model for HI described in Sec. \ref{sec:analytical} to explore the effects of an axion subspecies of DM in the mass range $10^{-32}$ eV $\leq m_a \leq$ $10^{-22}$ eV on the HI power spectrum at $z \leq 6$. It was found that lighter axions introduce a scale-dependent feature even on linear scales due to the suppression of the matter power spectrum near the Jeans scale. For the first time, it was possible to forecast the bounds on the axion fraction of DM in the presence of astrophysical and model uncertainties, achievable with upcoming facilities such as the HIRAX and SKA. A compilation of the latest forecasted constraints on the above beyond-$\Lambda$ CDM parameters with future surveys mapping baryonic gas is provided in Table \ref{table:beyondlcdm}.
\begin{figure}
\centering
\includegraphics[width = 0.9\textwidth]{relbias1.pdf}
\caption{Relative bias $b/\sigma$ on cosmological parameters, including those beyond the standard $\Lambda$CDM 
framework,  with a SKA I MID-like experimental configuration, obtained on shifting either 
astrophysical parameter, $\beta$ (left panels) or $\log v_{\rm c,0}$ (right panels), 
by $1 \sigma$ from its mean value given in Table \ref{table:constraints}. Top panels show the biases in
$\Lambda$CDM+$\gamma$ where $\gamma$ is a measure of modified gravity \cite{camera2020}, and lower panels show biases for those in $\Lambda$CDM+$f_{\rm NL}$. The empty 
(filled) circles indicate negative (positive) values of  biases. It can be seen that the bias values are well within 1$\sigma$ for all the parameters considered, including the $f_{\rm NL}$. Shaded areas (which are not reached by the results) indicate the range $b/\sigma > 3$, for which it is possible for the Gaussian approximation to the likelihood to become inaccurate. Figure taken from \cite{camera2020}. }
\label{fig:relbias1}
\end{figure}

\begin{landscape}
\begin{longtable}{l|l|l}
\caption[]{Latest forecasted constraints on non-standard dark matter (e.g., \cite{khlopov2013}), tests of inflation and modified gravity with the future experiments tracing baryonic gas, primarily in the radio and sub-millimetre regimes.} 
\tablehead
{\hline Type of constraint  & Limit & Experiment/Reference \\ \hline\hline} 
\tabletail
{\hline \multicolumn{4}{r}{\textit{Continued on next page}}\\}
\tablelasttail{\hline} \\
\hline
{\bf Constraints on parameters describing non-cold dark matter} & &  \\
\hline
Warm dark matter particle mass & $m_{\rm WDM} = 4$ keV ruled out at $> 2 \sigma$ at & \\
&  $z \sim 5$ & SKAI-LOW \cite{carucci2015} \\
Effective parameter for & & \\
dark matter decays & $\Theta_{\chi} \approx 10^{-40}$ at $10^{-5}$ eV & HERA/HIRAX/CHIME \\
& & \cite{bernal2021} \\
Axion mass & $m_a \approx 10^{-22}$ eV, at 1\% & SKAI-MID + CMB \cite{bauer2021} \\
Particle to two-photons coupling of axion-like particles (ALPs) & $g_{a \gamma \gamma} \leq 10^{-11}$/GeV & SPHEREx/LSST \cite{shirasaki2021} \\
\hline
{\bf Primordial non-Gaussianity constraints}  & & \\
  \hline
Standard deviation of  & $\sigma(f_{\rm NL}) \sim 4.07$ & SKA \cite{gomes2020} \\
 non-Gaussianity parameter & $\sigma(f_{\rm NL}) < 1$ & PUMA \cite{karagiannis2020} \\
& $\sigma(f_{\rm NL}) \sim 10$ & COMAP \cite{liu2021} \\
& $\sigma(f_{\rm NL}) \sim 2-3$ & ngVLA  \\
& $\sigma(f_{\rm NL}) \sim 2-3$ & PIXIE/OST MRSS \\
& & \cite{dizgah2019}\\
& $\sigma(f_{\rm NL}) < 1$ & SKA1+Euclid-like+CMB Stage 4 \cite{ballardini2019a} \\
\hline
{\bf Constraints on modified gravity parameters}  & &  \\
 \hline
Effective gravitational strength; & &  \\
initial condition parameter of matter perturbations & $Y (G_{\rm eff}), \alpha < 1\%$ at $z \sim 6-11$ & SKA \cite{heneka2018} \\
Parameter modification of $f(R)$ gravity & $B_0 < 7 \times 10^{-5}$ & CMB + 21-cm \cite{hall2013} \\
Value of $f(R)$ field in background today & $|f_{R0}| < 9 \times 10^{-6}$ & 21 cm \cite{masui2010} \\
\hline
\label{table:beyondlcdm}
\end{longtable}
\end{landscape}

