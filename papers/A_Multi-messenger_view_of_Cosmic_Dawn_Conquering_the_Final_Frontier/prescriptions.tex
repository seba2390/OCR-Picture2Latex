\section{The halo model framework: from dark matter to baryons}\label{s:HIhalomodel}

As we have gathered, mapping the evolution of baryonic material across cosmic time using their line transitions promises deep insights into galaxy evolution, as well as theories of gravity and fundamental physics. The 21-cm line of neutral hydrogen, which is currently a useful probe of the \HI\ content of galaxies, is ideally suited for the bulk of the material, since hydrogen is the most abundant element in the Universe. 

Traditionally, galaxy surveys have been used as probes of the neutral hydrogen distribution \cite{zwaan05, zwaan2005a, martin12} at late times. The limits of current radio facilities, however, hamper the detection of 21-cm in emission from normal galaxies at early times. At these epochs, the HI distribution has been studied via the identification of Damped Lyman Alpha systems (DLAs, e.g., \cite{noterdaeme12, zafar2013, crighton2015}) — which are known to be the primary reservoirs of neutral hydrogen. 

A relatively new technique used to study HI evolution is known as intensity mapping. In this technique, the large-scale distribution of a tracer (like HI) can be mapped without individually resolving the galaxies which host the tracer. Being thus faster and less expensive than traditional galaxy surveys, intensity mapping of the 21 cm line had already been shown over the last two decades, e.g., \cite{chang10, wyithe2008a, anderson2018} to have the potential to provide constraints on cosmological parameters which are competitive with those from next generation experiments. 

Given the deluge of data expected from forthcoming radio observations, especially with the intensity mapping technique, it is therefore
timely and important to unify the data into a self-consistent framework that addresses both high- and low-redshift observations. Such a framework, which encapsulates the astrophysical information contained in HI (and, in general, baryonic gas tracers) into a set of key physical parameters, is crucial to properly take into account astrophysical uncertainties in order to place the most realistic constraints on Fundamental Physics and cosmology. Furthermore, it enables the most effective comparison to physical models of galaxy astrophysics, allowing us to unravel several outstanding questions in their evolution.

In a map of unresolved line emission over a large area in the sky, the main quantity of interest is the three-dimensional power spectrum, introduced for \HI\ in Sec. \ref{sec:21cmintro} as $P_{\rm HI} (k,z)$. This power spectrum is a measure of the strength of the intensity fluctuations as a function of wave number ($k$) at redshift $z$. To compute the power spectrum, one needs to establish a framework connecting dark matter and baryons. The most natural method to achieve this is via a data-driven halo model framework \cite{hpar2017}  for the evolution of baryons as tracers of large scale structure. This can be achieved by combining the different datasets available (e.g. \cite{hptrcar2015}), and bringing together the models available in the literature \cite{hptrcar2016} into a comprehensive picture. In so doing, it is found that there are two main ingredients required to connect the baryonic quantities (mass, luminosity) to the underlying dark matter halo model (described in Sec.\ref{sec:introcosmo}), namely:
(i) the average mass (or luminosity) associated with a dark matter halo of mass $M$ at redshift $z$, and (ii) if relevant, the distribution of the mass as a function of scale, $r$ from the centre of the dark matter halo. The latter quantity is relevant when the data constrains the detailed small-scale structure of the baryons, as for the case of \HI. We will now  describe this framework, closely paralleling the development of the dark matter halo model framework in Sec. \ref{sec:introcosmo}.

Given a prescription for populating the halos with baryons, e.g., $M_{\rm HI}(M)$ (for the case of \HI), defined as the average mass of \HI\ contained in a halo of mass $M$, we begin by computing the neutral hydrogen density, $\bar{\rho}_{\rm HI}(z)$ in terms of the dark matter mass function $n(M,z)$ as:
\begin{equation}
\bar{\rho}_{\rm HI}(z)= \int_{M_{\rm min}}^{\infty} dM n(M,z) M_{\rm HI}(M) \,
\end{equation}
in which $M_{\rm min}$ is an astrophysical parameter that quantifies the minimum mass of the dark matter halo that is able to host \HI. In practice, this minimum mass requirement is usually built-in to the $M_{\rm HI}(M)$ function itself, by using a suitable cutoff scale.
For quantifying large-scale clustering, we define the large-scale bias parameter $b_{\rm HI} (z)$, that describes how strongly \HI\ is clustered relative to the dark matter,  as:
\begin{equation}
b_{\rm HI} (z) = \frac{1}{\bar{\rho}_{\rm HI}(z)}\int_{M_{\rm min}}^{\infty} dM
n(M,z) b(M) M_{\rm HI}(M).
\label{biasHI}
\end{equation}
To describe small-scale clustering, we define the normalized Hankel transform of the density profile $\rho_{\rm HI} (r)$:
 \begin{equation}
 u_{\rm HI}(k|M) = \frac{4 \pi}{M_{\rm HI} (M)} \int_0^{R_v} \rho_{\rm HI}(r) \frac{\sin kr}{kr} r^2 \ dr
\end{equation}
analogously to \eq{udm} for dark matter. The 
normalization of $u(k|M)$ above is to the total \HI{} mass, i.e. $M_{\rm HI} (M)$. In most of the standard analyses, the baryonic profile is assumed truncated at the halo virial radius, which is taken to be a standard `scale' cutting off the halo. The corresponding scale radius for the \HI\ is defined through $r_s = R_v(M)/c_{\rm HI}$ where $c_{\rm HI}$ is known as the  concentration parameter, analogous to the corresponding one for dark matter defined in \eq{concparamdm}. \footnote{Profiles like the NFW diverge as $\rho(r) \sim r^{-3}$ as $r \to \infty$, causing a logarithmic divergence of the mass contained within them. Thus, it is helpful to truncate the profile at the virial radius which defines the `boundary' enclosing the total mass, just as done for the dark matter case. Recent studies advocate the use of the \textit{splashback} radius, at which the accreted material reaches its first apocenter after turnaround \cite{diemer2018} and is separated from the virial radius by factors of a few, as a more precise definition of the halo boundary.}


The dark matter framework is then followed to compute the one- and two halo terms of the \HI{} power spectrum, given by:
\begin{equation}
P_{\rm 1h, HI} =  \frac{1}{\bar{\rho}_{\rm HI}^2} \int dM \  n(M) \ M_{\rm HI}^2 \ |u_{\rm HI} (k|M)|^2
\label{onehalo}
\end{equation}
and
\begin{equation}
P_{\rm 2h, HI} =  P_{\rm lin} (k) \left[\frac{1}{\bar{\rho}_{\rm HI}} \int dM \  n(M) \ M_{\rm HI} (M) \ b (M) \ |u_{\rm HI} (k|M)| \right]^2
\label{twohalo}
\end{equation}
which are analogous to \eq{powerspecdm}.
Some studies in the literature \cite{wolz2019, villaescusa2018} additionally model the shot noise of the halo power spectrum, $P_{\rm SN}$, which may be computed as:

\begin{equation}
   P_{\rm SN} =  \frac{1}{\bar{\rho}_{\rm HI}^2} \int dM \  n(M) \ M_{\rm HI}^2 \ 
   \label{shotnoiseHI}
\end{equation}
and corresponds to the $k \to 0$ limit of the one-halo power spectrum. This term provides a rough measure of the  Poissonian noise due to the finite number of discrete sources (such as HI-bearing galaxies).
In practice, however, in the redshift range considered in the context of \HI\ intensity
mapping, the shot noise contribution is expected to be negligible, e.g., \cite{seo2010, seehars2016} and relatively unconstrained by observations, so it is not relevant in a data-driven approach. It will, however, become important in the context of submillimetre observations described in the next section.

In all of the above, the dark matter mass function $n(M, z)$ can be computed from analytical arguments or simulations. In most of our analyses in the following sections, we will assume the Sheth - Tormen form defined in Sec. \ref{sec:astrocosmo}, though various other forms are possible \cite{tinker2008, shirasaki2021}.
 
Finally, the neutral hydrogen fraction, required for calculating the brightness temperature as described in Sec. \ref{sec:21cmintro}, is computed as:
\begin{equation}
 \Omega_{\rm HI} (z) = \frac{\bar{\rho}_{\rm HI}(z)}{\rho_{c,0}}
 \label{omegaHIanalyt}
\end{equation} 
where $\rho_{c,0} \equiv 3 H_0^2/8 \pi G$ is the critical density of the Universe at redshift 0.



In several recent observations, the real-space correlation function is measured, which is defined as the Fourier transform of the power spectrum ($P_{\rm HI, 1h} + P_{\rm HI, 2h})$. It is computed as:
\begin{equation}
\xi_{\rm HI} (r) = \frac{1}{2 \pi ^2} \int k^3 (P_{\rm 1h, HI} + P_{\rm 2h, HI}) \frac{\sin kr}{kr} \frac{dk}{k}
\end{equation}

The above discussion completes the generic treatment for analysing observations of baryonic gas in a halo model framework, given the tracer-to-halo mass relation and its small-scale profile. Note that the above framework is most naturally suited to describe intensity mapping observations, since it is a mass- (or luminosity) weighted measurement analogous to the approach for dark matter, similar to that employed in studies of the cosmic infrared background, e.g., \cite{fernandez2006}. This is different from, e.g., a number-counts weighted approach which is commonly used in surveys of discrete tracers like galaxies, e.g., \cite{cooraysheth2002} with a halo occupation distribution (HOD). While number-count- and mass-count-weighted frameworks are completely equivalent in their capture of small-scale effects (via the `satellite' galaxies distribution and the 1-halo term respectively), it is more natural to describe intensity mapping observations using a mass-weighted approach, since we are dealing here with diffuse gas not all of which may be located inside galaxies and as such does not form a discrete set. This also facilitates the extension of this parameterization to the reionization regime and beyond. Quantifying the gas content in discrete systems (such as galaxies and Damped Lyman-Alpha systems) is made possible due to the flexibility in mass and cutoff scales in the framework, as we show below by describing how specific expressions that arise in the context of \HI\ observations from galaxy surveys and DLAs can be modelled with the above ingredients.



\subsection{21 cm from galaxy surveys}
\label{sec:21cmgal}

The main observable in a survey of galaxies detected in 21 cm is their \textit{mass function}, which measures the volume density of \HI-selected galaxies as a function of the \HI\ mass in different mass bins. Denoted by $\phi(M_{\rm HI})$, it is typically found to follow a functional form of the type \cite{martin10, zwaan05}:
\begin{equation}
 \phi(M_{\rm HI}) \equiv \frac{dn}{d \log_{10} M_{\rm HI}} =  \frac{\phi_*}{M_0} \ \left(\frac{M_{\rm HI}}{M_0} \right)^{-a} e^{-M_{\rm HI}/M_0}
\end{equation} 
where $\phi_*, a$ and $M_0$ are free parameters.
Given the HI mass - halo mass relation $M_{\rm HI}(M,z)$, the above mass function can be modelled by:
\begin{equation}
    \phi(M_{\rm HI}) = \frac{dn}{d \log_{10} M} \left|\frac{d \log_{10} M}{d \log_{10} M_{\rm HI}}\right|
    \label{phifrommhi}
\end{equation}
with the first term calculated from $n(M,z)$.



The \HI\ mass function is then used to calculate the mass density of \HI\ in galaxies:
\begin{equation}
 \rho_{\rm HI} = \int M_{\rm HI} \phi(M_{\rm HI}) d M_{\rm HI} = M_0 \phi_* \Gamma(2 - a) 
\end{equation}
from which the density parameter of \HI\ in galaxies, $\Omega_{\rm HI, gal} = $ can be calculated:
\begin{equation}
  \Omega_{\rm HI, gal} =  \rho_{\rm HI}/\rho_{c,0}
\end{equation}



The clustering of HI-selected galaxies can generally be computed following \eq{biasHI}, with the modification that the lower limits in the integrals in \eq{biasHI} are fixed to the halo mass $M_{\rm min}$ corresponding to the minimum \HI\ mass observable by the survey. Thus, the \HI\ bias measured from galaxy surveys is modelled by:
\begin{equation}
b_{\rm HI,gal} (z) = \frac{\int_{M_{\rm min}}^{\infty} dM \ n(M,z)\ b (M,z) \ M_{\rm HI} (M,z)}{\int_{M_{\rm min}}^{\infty} dM \  n(M,z) \ M_{\rm HI} (M,z)}
\label{biasHIgal}
\end{equation}

We note in passing that, given the \HI\ mass function and reversing the procedure described in \eq{phifrommhi}, commonly known as the \textit{abundance matching} technique in the literature, we can directly estimate $M_{\rm HI}(M)$ from observational data. This makes the assumption that $M_{\rm HI}(M)$ is a monotonic function of $M$, and is given by solving the equation \cite{vale2004}:
\begin{equation}
    \int_{M(M_{\rm HI})}^{\infty} \frac{dn}{d \log_{10} M'} d \log_{10} M' = \int_{M_{\rm HI}}^{\infty} \phi(M_{\rm HI'}) d \log_{10} M_{\rm HI}'
    \label{abmatchhi}
\end{equation}
and an appropriate functional form is fitted to the resulting datapoints $\{M_{\rm HI}, M\}$. Such an approach, for the case of \HI\ and using the mass functions from recent literature \cite{martin10, zwaan05} is found to lead to results which are consistent \cite{hpgk2017} with the direct fitting to observations described below.



\subsection{Neutral hydrogen  from Damped Lyman Alpha systems}
\label{sec:dlahimodels}

As we have seen in Sec. \ref{sec:astrocosmo}, Damped Lyman Alpha systems (DLAs) represent the highest column density \HI-bearing systems in the IGM.
In a survey measuring the neutral hydrogen fraction using DLAs, the primary observable is called the \textit{column density distribution function} defined in \eq{coldens} denoted by $f_{\rm HI} (N_{\rm HI})$, where $N_{\rm HI}$ is the column density of the DLAs:
\begin{equation}
 d^2 \mathcal{N} = f_{\rm HI} (N_{\rm HI}, X) dN dX
\end{equation} 
with $\mathcal{N}$ being the observed incidence rate of DLAs in the absorption interval $dX$ and the column density range $dN_{\rm HI}$. 
To model the column density distribution using the framework described in the previous section, the hydrogen density profile $\rho_{\rm HI}(r)$ as a function of $r$ is first used to calculate the column density of DLAs using the relation:
\begin{equation}
 N_{\rm HI}(s) = \frac{2}{m_H} \int_0^{\sqrt{R_v(M)^2 - s^2}} \rho_{\rm HI} (r = \sqrt{s^2 + l^2}) \ dl 
 \label{coldenss}
\end{equation} 
where $s$ is the impact parameter of a line-of-sight through the DLA. 

The cross-section for a system to be identified as a DLA, $\sigma_{\rm DLA}$ can then be computed by 
\begin{equation}
    \sigma_{\rm DLA} = \pi s_*^2
\end{equation}
where $s_*$ is the root\footnote{If no positive root $s^*$ exists, it physically means that the column density in the line-of-sight does not reach $N_{\rm HI} = 10^{20.3}$ cm$^{-2}$ even at zero impact parameter, so the cross-section is zero for such systems. Hence, in such cases, $s ^{*}$ is set to zero.}  of the equation $N_{\rm HI}(s_*) = 10^{20.3}$ cm$^{-2}$ (which is the column density threshold for the appearance of DLAs). 

Given the DLA cross-section, 
the column density distribution $f_{\rm HI}(N_{\rm HI}, z)$ is modelled by:
\begin{equation}
 f(N_{\rm HI}, z) = \frac{c}{H_0} \int_0^{\infty} n(M,z) \left|\frac{d \sigma}{d N_{\rm HI}} (M,z) \right| \ dM 
 \label{coldensdef}
\end{equation} 
where the $d \sigma/d N_{\rm HI} =  2 \pi \ s \ ds/d N_{\rm HI}$, with $N_{\rm HI} (s)$ defined as in \eq{coldenss}.

The clustering of DLAs is captured by the DLA bias, $b_{\rm DLA}$  defined by:
\begin{equation}
 b_{\rm DLA} (z) =  \frac{\int_{0}^{\infty} dM n (M,z) b(M,z) \sigma_{\rm DLA} (M,z)}{\int_{0}^{\infty} dM n (M,z) \sigma_{\rm DLA} (M,z)}.
 \label{bdla}
\end{equation} 
Note that the above expression is almost identical to \eq{biasHI}, with the only difference being the weighting of the bias by the cross-section of DLA absorbers.

Another observable is the incidence of the DLAs, denoted by $dN/dX$ which quantifies the number of systems per absorption path length, and is calculated as:
\begin{equation}
 \frac{dN}{dX} = \frac{c}{H_0} \int_0^{\infty} n(M,z) \sigma_{\rm DLA}(M,z) \ dM
 \label{dndxdef}
\end{equation} 
Finally, from the column density distribution, the density parameter of hydrogen in DLAs can be calculated as:
\begin{equation}
 \Omega_{\rm HI}^{\rm DLA} = \frac{m_H H_0}{c \rho_{c,0}} \int_{N_{\rm HI, min}}^{\infty} N_{\rm HI} f_{\rm HI} (N_{\rm HI}, X) dN_{\rm HI} dX \, ,
\end{equation} 
in which the lower limit\footnote{The lower limit changes to $10^{19}$ cm$^{-2}$, \cite{zafar2013} in case the sub-DLAs too are accounted for while calculating the gas density parameter. Lower column-density systems, such as the Lyman-$\alpha$ forest,  make negligible contributions to the total gas density.} of the integral is set by the column density threshold for DLAs, i.e. $N_{\rm HI, min} = 10^{20.3}$ cm$^{-2}$. 



In alternate approaches to modelling DLAs \cite{villaescusa2018}, the cross section $\sigma_{\rm DLA} (M)$ itself may directly be modelled using a functional form, and the DLA quantities calculated from $\sigma_{\rm DLA}$. Various simulation-based approaches have also been used to quantify the neutral hydrogen at different redshifts from DLAs \cite{pontzen2008, bird2014}.


\subsection{HI-halo mass relations and density profiles}
\label{sec:analytical}
We have seen in the above discussion that the two key inputs needed in the halo model framework for hydrogen are $M_{\rm HI}(M)$, the prescription for
assigning \HI\ to the dark matter haloes, and $\rho_{\rm HI}(r)$, which describes how this mass is distributed as a function of scale.  

Several forms had been used to model the $M_{\rm HI}(M)$ function in the past literature. At redshifts probed chiefly by DLA observations, $z \sim 2-5$, a relation between HI mass and halo mass of the form 
$M_{\rm HI} = \alpha M \exp(-M/M_0)$
where $\alpha$ is a constant of normalization, and $M_0$ is a lower mass cutoff was commonly used\cite{barnes2014}. 
At lower redshifts, various forms had been proposed, among which are $M_{\rm HI} = f M$ which is a constant fraction of the halo mass between a fixed lower limit in halo mass, $M_{\rm min}$ and upper limit $M_{\rm max}$ \cite{bagla2010}. 

Reconciling the high- and low-redshift approaches \cite{hptrcar2016} leads to some fascinating insights about the occupation of \HI\ in dark matter haloes. Specifically, it is found that in order to match all the current observational data (DLAs, 21 cm galaxy surveys and intensity mapping observations) over $z \sim 0-5$, the connection between HI mass and halo mass needs to be modelled using a function of the form:
\begin{eqnarray}
M_{\rm HI} (M) &=& \alpha f_{\rm H,c} M \left(\frac{M}{10^{11} h^{-1} M_{\odot}}\right)^{\beta} \exp\left[-\left(\frac{v_{c0}}{v_c(M)}\right)^3\right] \nonumber \\
\end{eqnarray}
We now describe the form of this expression in some detail. 
It involves three free parameters, $\alpha$, $\beta$ and $v_{c,0}$:

(i) $\alpha$ is the  overall normalization of the HI-halo mass relation. Physically, it represents the fraction of HI, relative to the cosmic fraction ($f_{\rm H,c}$) that resides in a halo of mass $M$ at redshift $z$. The cosmic fraction is the primordial hydrogen fraction by mass, defined as:
\begin{equation}
f_{\rm H,c} = (1 - Y_{\rm He}) \Omega_b/\Omega_m \, ,
\label{fhc}
\end{equation}
in which $Y_{\rm He} = 0.24$ is the primordial helium fraction.

(ii) $\beta$ is the logarithmic slope of \HI\ mass to halo mass. Any nonzero value of $\beta$ describes a  departure from proportionality of the HI mass and halo mass. The $\beta$ is physically connected to physical processes that deplete \HI\ in galaxies (such as quenching and feedback from the intergalactic medium).

(iii) The parameter $v_{\rm c, 0} (M)$  represents a lower cutoff to the HI-halo mass relation. It describes the minimum mass (or equivalent circular velocity) of a halo able to host neutral hydrogen. In the above equation, the circular velocity is calculated through:
\begin{equation}
    v_{c} = \sqrt\frac{GM}{R_v(M)}
    \label{vcRv}
\end{equation}
where $R_v(M)$ is the virial radius defined in \eq{virialradius}. 
\footnote{An equivalent representation of the lower cutoff is to use a minimum halo mass ($M_{\rm min} (z)$) in place of the circular velocity. Since we see from \eq{vcRv} and \eq{virialradius} that $v_c \propto M^{1/3}(1+z)^{1/2}$, the exponential term  can be written as  $\exp(-M/M_{\rm min} (z))$ if we are using a mass cutoff. However, using the circular velocity -- instead of halo mass  -- to denote the cutoff makes the physical connection to the intergalactic ionizing background more natural, as we will see in the next section.}

Just like the HI-halo mass relation, it is also important to parametrize the profile of the neutral hydrogen in the dark matter halo, $\rho_{\rm HI}(r)$. In the literature, this function had been modelled by an altered version of the NFW profile introduced in Sec. \ref{sec:astrocosmo} \cite{maller2004, barnes2014, hpar2017}:
\begin{equation}
\rho_{\rm HI} (r) = \frac{\rho_0 r_{\rm s, HI}^3}{(r + 0.75 r_{\rm s, HI}) (r+r_{\rm s, HI})^2}
\label{rhodefnfw}
\end{equation}
The quantity $r_{\rm s,HI}$ is the scale radius of \HI, which is defined as $r_{\rm s, HI} \equiv R_v(M)/c_{\rm HI}$, introducing a concentration parameter $c_{\rm HI}$ analogous to the one for dark matter in \eq{concparamdm}:
\begin{equation}
    c_{\rm HI}(M,z) =  c_{\rm HI, 0} \left(\frac{M}{10^{11} M_{\odot}} \right)^{-0.109} \frac{4}{(1+z)^{\gamma}}
    \label{concparamhi}
\end{equation}
We thus see that the profile function brings two additional parameters to the model: $c_{\rm HI, 0}$ and $\gamma$. Of these, $c_{\rm HI}$ is the overall normalization of the concentration, and $\gamma$ describes how it evolves with redshift. \footnote{Note that the profile has a negligible halo mass dependence (the power of -0.109). This is inherited from the dark matter framework\cite{duffy}  and does not have any significant bearing on the modelling.}
Recent data, especially from HI-rich disk galaxies \cite{bigiel2012} at low redshifts, favours the profile for \HI\ in haloes being of the exponential form (rather than the modified NFW relation above):
\begin{equation}
    \rho(r,M) = \rho_0 \exp(-r/r_{\rm s, HI})
\label{rhodefexp}
\end{equation}
In both forms of the profile, the parameter $\rho_0$ is fixed by normalization to the total HI mass, $M_{\rm HI}(M)$, just as done for  the case of the dark matter halo model.
The profile and the HI-halo mass relation can now be used to compute the 1- and 2-halo terms of the power spectrum defined in \eq{onehalo} and \eq{twohalo} respectively. For this, it is necessary to compute the normalized Hankel transform of the profile function, which can be expressed analytically for both the profile choices above \cite{hparaa2017}.
As an explicit example, for  the \HI\ profile in the exponential form, the  expression for the normalized Hankel transform is given by:
\begin{equation}
u_{\rm HI}(k|M) = \frac{4 \pi \rho_0 r_{\rm s, HI}^3 u_1(k|M)}{M_{\rm HI} (M)}
\end{equation}
where
\begin{equation}
u_1(k|M) = \frac{2}{(1 + k^2 r_{\rm s, HI}^2)^2},
\end{equation}
which can then be used to compute the full power spectrum.

The five parameters  $\{c_{\rm HI, 0}, \alpha, \beta, v_{c, 0}, \gamma\}$ can now be fixed by fitting the framework above to the current astrophysical data describing \HI.  This is achieved using a Markov Chain Monte Carlo (MCMC) approach \cite{hparaa2017}  and the resulting best-fitting parameters and their uncertainties are summarized in the first column of Table \ref{table:constraints}. There are a few salient points that emerge from the analysis:
\begin{enumerate}
    \item The best fitting value of $\alpha$ is $\alpha = 0.09$. This denotes that about $10\%$ of the hydrogen is in atomic form over the post-reionization Universe, and is in line with recent simulations and observations that predict the fraction of cold gas (i.e. the sum of the atomic and molecular components) to be in the range of $\sim 10-20$\% in low-redshift galaxies\cite{stern2016}. Interestingly, the observations do not favour an evolution in $\alpha$ with redshift. This in in line with the findings from DLA studies \cite{prochaska09}, which indicate evidence for non-evolution of hydrogen in galaxies over $z \sim 0-5$, and reiterates the role of HI as an `intermediary' in the baryon cycle \cite{wang2020, bouche2010, lilly2013, hploebsfr2020}: the HI replenishment from the IGM is compensated by its conversion to molecular hydrogen, $H_2$ which is used up by star formation.
    
    \item The slope, $\beta$ is found to have the (negative) value $\beta = -0.58$. It is found that the slope is dominantly influenced by the form of the HI mass function observed at low redshifts for which exquisite constraints are available \cite{zwaan05, martin10}. The suppression of the resultant HI-halo mass slope from unity is in line with evidence for quenching, or the suppression of star formation in massive haloes due to feedback from the IGM \cite{birnboim2007, finlator2017}.
    
    \item The cutoff $v_{\rm c,0}$ has the value 36.3 km/s. This can be directly related to the circular speed of suppression of dwarf galaxies by the ambient ionizing background in the IGM, which was worked out in the mid-80s and 90s from analytical arguments balancing ionization and recombination \cite{rees1986, efstathiou1992, quinn1996}. Fascinatingly, such an analysis favours a value of $v_c \sim 37$ km/s, almost perfectly matched to the value obtained by combining the latest ($\sim$ 2017)  \HI\ observations today! This greatly strengthens the case for using circular velocity as a lower cutoff for \HI\ in haloes, and sheds light on the physics associated with this parameter. 
    

\end{enumerate}
    



