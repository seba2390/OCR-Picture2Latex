\documentclass[12pt,onecolumn,draftcls]{IEEEtran}
\usepackage{amsfonts}
\usepackage{graphicx}
\usepackage{color}
\usepackage{amsmath,amsfonts,amssymb,amsthm,epsfig,epstopdf,url,array}
\usepackage{url,textcomp}
\usepackage{authblk}
\usepackage{cite}
\makeatletter
\newcommand*{\rom}[1]{\expandafter\@slowromancap\romannumeral #1@}
\makeatother
\newcommand{\bs}{\boldsymbol}
\newtheorem{theorem}{Theorem}
\newtheorem{lemma}{Lemma}
\newtheorem{corollary}{Corollary}
\newtheorem{mydef}{Definition}
\newtheorem{remark}{Remark}
\newtheorem{property}{Property}
\newtheorem{proposition}{Proposition}
\begin{document}
\title{Asymptotic Outage Analysis of HARQ-IR over Time-Correlated Nakagami-$m$ Fading Channels}
\author{Zheng~Shi,
%        Haichuan~Ding,
        Shaodan~Ma, Guanghua~Yang, Kam-Weng~Tam, and Minghua Xia
%\thanks{Manuscript received January 14, 2015; revised May 29, 2015; accepted July 19, 2015. The associate editor coordinating the review of this paper and approving it for publication was M. Elkashlan.}
\thanks{Zheng Shi, Shaodan Ma and Kam-Weng Tam are with the Department of Electrical and Computer Engineering, University of Macau, Macao (e-mail:shizheng0124@gmail.com, shaodanma@umac.mo, kentam@umac.mo).}
%\thanks{Haichuan Ding was with University of Macau, and is now with the Department of Electrical and Computer Engineering, University of Florida, U.S.A. (email: dhcbit@gmail.com).}
\thanks{Guanghua Yang is with the Institute of Physical Internet, Jinan University, China (e-mail: ghyang@jnu.edu.cn).}
\thanks{Minghua Xia is with the Department of Electronics and Communication Engineering, Sun Yat-Sen University, Guangzhou, China (email: xiamingh@mail.sysu.edu.cn).}
%\thanks{The corresponding author is Shaodan Ma.}
%\thanks{This work was supported by the Research Committee of University of Macau under grants: MYRG078(Y1-L2)-FST12-MSD and MYRG101(Y1-L3)-FST13-MSD.}
}
\maketitle
\begin{abstract}
In this paper, outage performance of hybrid automatic repeat request with incremental redundancy (HARQ-IR) is analyzed. Unlike prior analyses, time-correlated Nakagami-$m$ fading channel is considered. The outage analysis thus involves the probability distribution analysis of a product of multiple correlated shifted Gamma random variables and is more challenging than prior analyses. Based on the finding of the conditional independence of the received signal-to-noise ratios (SNRs), the outage probability is exactly derived by using conditional Mellin transform. Specifically, the outage probability of HARQ-IR under time-correlated Nakagami-$m$ fading channels can be written as a weighted sum of outage probabilities of HARQ-IR over independent Nakagami fading channels, where the weightings are determined by a negative multinomial distribution. This result enables not only an efficient truncation approximation of the outage probability with uniform convergence but also asymptotic outage analysis to further extract clear insights which have never been discovered for HARQ-IR even under fast fading channels. The asymptotic outage probability is then derived in a simple form which clearly quantifies the impacts of transmit powers, channel time correlation and information transmission rate. It is proved that the asymptotic outage probability is an inverse power function of the product of transmission powers in all HARQ rounds, an increasing function of the channel time correlation coefficients, and a monotonically increasing and convex function of information transmission rate. The simple expression of the asymptotic result enables optimal power allocation and optimal rate selection of HARQ-IR with low complexity. Finally, numerical results are provided to verify our analytical results and justify the application of the asymptotic result for optimal system design.


%is theoretically analyzed that full diversity can be achieved by HARQ-IR even under time-correlated fading channels while time correlation has a detrimental impact on the outage performance. It is also proven that the asymptotic outage probability is a function of the product of transmit powers in all HARQ rounds and is a convex function of information transmission rate, which enable the optimal power allocation and optimal rate selection of HARQ-IR, respectively. In addition, HARQ-IR under quasi-static fading channels is particularly discussed as a special case of time-correlated fading channels, and it is revealed that no time diversity can be achieved from retransmissions of HARQ. Finally, numerical results are presented for verifications and discussions.
\end{abstract}
% Note that keywords are not normally used for peerreview papers.
\begin{IEEEkeywords}
Hybrid automatic repeat request with incremental redundancy, time correlation, Nakagami-$m$ fading, asymptotic outage analysis, product of multiple random variables.
\end{IEEEkeywords}
\IEEEpeerreviewmaketitle

\section{Introduction}\label{sec:int}
Recently, hybrid automatic repeat request (HARQ) technique has found wide applications in wireless communications due to its high potential for reliable transmissions. It has been proved from an information-theoretic view in \cite{caire2001throughput} that HARQ with incremental redundancy (HARQ-IR) can achieve the ergodic capacity in Gaussian collision channels. It also provides superior performance over other types of HARQ since extra coding gain is obtained through code combining. Thus this paper focuses on the analysis of HARQ-IR. As shown in \cite{makki2014performance}, the most fundamental metric to evaluate the performance of HARQ-IR is outage probability and its analysis essentially turns to determine the cumulative distribution function (CDF) of accumulated mutual information.
%It is proved that the derivation of the CDF of accumulated mutual information is equivalent to determining the CDF of a product of multiple shifted random variables (RVs) \cite{shen2009average,makki2013green,to2015power,kim2014optimal,sesia2004incremental,jinho2013energy,stanojev2009energy,
%aghajanzadeh2011outage,yilmaz2010outage,chelli2013performance,larsson2014throughput,
%larsson2016throughput}. Specifically, for Nakagami-$m$ fading channels, these RVs follow Gamma distribution.
%Specifically, these RVs comply with Gamma distribution for Nakagami-$m$ fading channels, which involve Rayleigh fading as its special case with fading order $m=1$.

In prior literature, various methods have been proposed to derive the CDF of accumulated mutual information for HARQ-IR under either quasi-static \cite{shen2009average,makki2013green} or fast fading channels \cite{to2015power,kim2014optimal,sesia2004incremental,jinho2013energy,stanojev2009energy,
aghajanzadeh2011outage,yilmaz2010outage,chelli2013performance,
larsson2014throughput,larsson2016throughput}. To be more specific, for quasi-static fading channels where the same channel realization is experienced by the transmitted signal in each HARQ round, the CDF of accumulated mutual information is easy to be derived because of the simplicity of handling only one single random variable (RV). Hereby, \cite{shen2009average} and \cite{makki2013green} have conducted average rate analysis and power optimization for HARQ-IR, respectively. Unfortunately, the analytical results under quasi-static fading channels are only applicable to low mobility environment. In high mobility environment, the transmitted signals among all HARQ rounds would experience fast fading channels, where the channel responses vary independently from one transmission to another. Under fast fading channels, the derivation of the CDF of accumulated mutual information essentially turns to determine the distribution of the product of multiple independent shifted RVs. There are several approaches proposed to tackle this problem in the literature. For example, in \cite{to2015power,kim2014optimal,sesia2004incremental}, Log-normal approximation is proposed based on central limit theorem (CLT). In \cite{jinho2013energy,stanojev2009energy,aghajanzadeh2011outage}, a lower bound and an upper bound of the CDF are derived based on Jensen's inequality and Minkowski inequality, respectively. To calculate the exact CDF, Mellin transform and multi-fold convolution have been applied in \cite{yilmaz2010outage,chelli2013performance} and \cite{larsson2014throughput} respectively. Specifically, based on Mellin transform,
\cite{yilmaz2010outage} derives the exact CDF in terms of generalized Fox's H function. The analytical results are then applied for outage analysis in \cite{chelli2013performance}. In \cite{larsson2014throughput}, noticing that the accumulated mutual information is a sum of independent RVs under fast fading channels, the exact CDF is thus derived by using multi-fold convolution. Moreover, by associating the analytical results in \cite{larsson2014throughput} with matrix exponential distribution through Laplace transform, \cite{larsson2016throughput} approximates the CDF of accumulated mutual information to a matrix exponential distribution. Unfortunately, the exact results in \cite{yilmaz2010outage,chelli2013performance,larsson2014throughput}
are too complicated to provide meaningful insights due to difficulties in handling generalized Fox's H function and multi-fold integrals, while the approximated results in \cite{to2015power,kim2014optimal,sesia2004incremental} and \cite{larsson2016throughput} involve certain numerical calculations which also hinder the extraction of clear insights.

%this approximation becomes instable for higher signal-to-noise ratio (SNR). However, it is difficult to extract clear insights to ease the system design through this approximation. However, the analytical results obtained in \cite{yilmaz2010outage,chelli2013performance,larsson2014throughput} are too complicated to provide meaningful insights due to difficulties in handling special functions and multi-fold integrals.
%few insights have been extracted from the closed-form expressions because of difficulties in handling special functions and multi-fold integrals.

The analysis of HARQ-IR becomes more challenging while considering time-correlated fading channels, which usually occur in low-to-medium mobility environment, because a product of multiple shifted correlated RVs is involved in the CDF of accumulated mutual information. To the best of our knowledge, there are only few approximation approaches available to analyze the performance of HARQ-IR over time-correlated fading channels, that is, Log-normal approximation \cite{yang2014performance}, polynomial fitting technique \cite{shi2015analysis} and inverse moment matching method \cite{shi2016inverse}. Unfortunately, the Log-normal approximation in \cite{yang2014performance} is inaccurate when the fading channels have medium-to-high time correlation. Although the results in \cite{shi2015analysis} and \cite{shi2016inverse} usually can achieve a good approximation of the CDF of accumulated mutual information, their tightness is proved in mean square error (MSE) sense. It does not necessarily imply that the approximation error always approaches to zero and the occurrence of inaccuracy is found especially under low outage probability/hign signal-to-noise ratio (SNR). In addition, the approximated results in \cite{shi2015analysis,shi2016inverse} are still complicated with little insights, thus limiting their applications in practical system design.

In this paper, asymptotic outage analysis is conducted to thoroughly investigate the performance of HARQ-IR over time-correlated Nakagami-$m$ fading channels with meaningful insights. Based on our finding that the received SNRs in multiple HARQ transmissions are conditionally independent given a certain RV, the CDF of accumulated mutual information can be exactly derived by using conditional Mellin transform. With the result, the outage probability can be derived as a weighted sum of outage probabilities of HARQ-IR over independent Nakagami fading channels, where the weightings are determined by a negative multinomial distribution. A truncation approximation with uniform convergence is then proposed to ease the computation of the exact outage probability. Notice that the uniform convergence is stronger than the MSE convergence in \cite{shi2015analysis,shi2016inverse} and thus offers higher accuracy in the approximation. The analytical result in this paper also enables the asymptotic outage analysis, which has never been conducted for HARQ-IR even under fast fading channels. More specifically, the asymptotic outage probability is derived in a simple form which clearly quantifies the impacts of transmission powers in multiple transmissions, channel time correlation and information transmission rate. It is also proved that the asymptotic outage probability is an inverse power function of the product of transmission powers in all HARQ rounds, an increasing function of the channel time correlation coefficients, and a monotonically increasing and convex function of information transmission rate. The simple expression of the asymptotic result enables optimal power allocation and optimal rate selection of HARQ-IR through plenty of readily available optimization techniques with low complexity. Finally, numerical results are provided to verify our analytical results and justify the application of the asymptotic result for optimal system design.


% The theoretical analysis reveals that full diversity can be achieved by HARQ-IR even under time-correlated fading channels except for quasi-static fading channels while the time correlation of fading channels has a detrimental impact on system performance. The asymptotic outage probability is proven to be a function of the product of transmit powers in all HARQ rounds and a convex function of coding rate, which will enable optimal power allocation and optimal rate selection of HARQ-IR through plenty of readily available optimization methods. Furthermore, the analysis of HARQ-IR under quasi-static fading channels is separately discussed because of its particularity among time-correlated fading channels, and it is found that no time diversity can be achieved from retransmissions of HARQ. Finally, the numerical results are shown to verify our analytical results together with some discussions.


%To this end, it is necessary to derive the exact outage probability of HARQ-IR, which becomes more challenging than prior analyses when both channel time correlation and Nakagami-$m$ fading channels are considered. Fortunately,


%Specifically, since the asymptotic outage probability is found to be a function of power-product term, optimal power allocation can be used to minimize the outage probability given a constraint on average total power. The monotonicity and the convexity of outage probability with respect to coding rate facilitates the optimal rate selection for HARQ-IR system.

The remainder of this paper is organized as follows. Section \ref{sec:sys_mod} introduces the system model and formulates the outage probability. The probability distribution of the product of multiple correlated shifted RVs which is necessary for outage analysis is derived and the exact outage probability is given in Section \ref{sec:exa}. Asymptotic outage probability is then derived in a simple form and meaningful insights are shown in Section \ref{sec:asy}. In Section \ref{sec:num_res}, numerical results are presented for validation and discussion. Finally, Section \ref{sec:con} concludes this paper.
%The representation of correlated Gamma RVs and its application is introduced in Section \ref{sec:corr}. Sections \ref{sec:dis_prod_gam} and \ref{sec:dis_prod_shift_cor} study the statistics of the product of correlated and correlated shifted-Gamma RVs, respectively. The analytical results are validated in Section \ref{sec:num_res}. Finally, Section \ref{sec:con} concludes this paper.

\section{System Model and Outage Formulation}\label{sec:sys_mod}
A point-to-point HARQ-IR enabled system operating over time-correlated Nakagami-$m$ fading channels is considered in this paper. The details of the system are introduced as follows.

%To facilitate outage formulation, HARQ-IR protocol and time-correlated Nakagami-$m$ fading channels are first introduced.

\subsection{HARQ-IR Protocol}\label{sec:harq_ir}
Following HARQ-IR protocol, the source first encodes every $b$-bits information message into $K$ codewords each with length of $L$, where $K$ denotes the maximum allowable number of transmissions for each message. As a result, the initial coding rate is $c = b/L$ \cite{wu2010performance}. The $K$ codewords will be sequentially transmitted to the destination until the message is successfully decoded. In each transmission, the previously received codewords are combined with the most recently received codeword for joint decoding. If successful, an acknowledgement (ACK) message is fed back from the destination to the source and the source then moves to the transmission of the next information message. Otherwise, a negative acknowledgement (NACK) message is fed back from the destination to the source and the source transmits the next codeword until the maximum number of transmissions $K$ is reached. Similarly to the analyses in the literature \cite{chelli2013performance,larsson2014throughput,larsson2016throughput}, error-free feedback channels are assumed here, that is, all feedback messages can be successfully decoded.
%
%
%The success or failure of the decoding operation will be declared by feeding back a single bit of acknowledgement (ACK) or negative acknowledgement (NACK) message to the source, respectively. Specifically, if a NACK is received and the maximum number of transmissions $K$ is not reached, the next packet will be transmitted. However, once either a ACK is received or the maximum transmission number $K$ is reached, the source will start delivery of a new message following the same procedure. In addition, we assume an error-free feedback channel, that is, all feedback signals can be correctly decoded.
\subsection{Time-Correlated Nakagami-$m$ Fading Channels}\label{sec:corr}
Denote ${\bf x}_k$ as the $k$th codeword with length of $L$. It is transmitted over a block fading channel, i.e., each symbol of ${\bf x}_k$ experiences an identical channel realization during the $k$th transmission. Therefore, the signal received in the $k$th transmission is given by
\begin{equation}\label{eqn:sign_mod}
  {\bf y}_k = h_k {\bf x}_k + {\bf n}_k,
\end{equation}
where ${\bf n}_k$ denotes a complex additive white Gaussian noise (AWGN) vector with zero mean vector and covariance matrix $\mathcal N_0{\bf I}_{L}$, i.e., ${\bf n}_k \sim {\cal CN}(0,\mathcal N_0{{\bf I}_{L}})$, ${\bf I}_{L}$ represents an $L \times L$ identity matrix, and $h_k$ denotes the block fading channel coefficient in the $k$th transmission.

%\textcolor[rgb]{1.00,0.00,0.00}{Unlike prior works, time correlation of fading channels is considered for the analysis of HARQ-IR. [Deleted] zhu2016Performance}

%such that $|h_k| \sim {\rm Nakagami}(m,{\sigma_k}^2)$,

Notice that Nakagami-$m$ fading is a general channel model whose parameters can be adjusted to fit a variety of empirical measurements \cite{goldsmith2005wireless}, e.g., it covers one-sided Gaussian channel and Rayleigh fading as special cases by setting $m=\frac{1}{2}$ and $m=1$ respectively. It is thus considered here. Unlike most of the prior analyses, time-correlated fading channels are considered, that is, the channel coefficients among the $K$ transmissions are correlated. The time correlation usually occurs in low-to-medium mobility environment \cite{kim2011optimal}. Herein, a widely used Nakagami-$m$ fading channel model with generalized correlation is adopted and the channel magnitude $|h_k|$ is written as \cite{beaulieu2011novel,lopez2013bivariate,shi2012correlated}
\begin{equation}\label{eqn:R_k_def}
|{h_k}| = \sqrt {\frac{{{\sigma _k}^2}}{m}} \left\| {\sqrt {1 - {\lambda _k}^2} \left( {\begin{array}{*{20}{c}}
{{\vartheta _{k,1}}}\\
%{{\vartheta _{k,2}}}\\
 \vdots \\
{{\vartheta _{k,m}}}
\end{array}} \right) + {\lambda _k}\left( {\begin{array}{*{20}{c}}
{{\vartheta _{0,1}}}\\
%{{\vartheta _{0,2}}}\\
 \vdots \\
{{\vartheta _{0,m}}}
\end{array}} \right)} \right\| ,
\end{equation}
where $\left\| \cdot \right\|$ represents ${\ell }^2$ norm, ${\vartheta_{k,l}} $ and ${\vartheta_{0,l}}$ are independent circularly-symmetric complex Gaussian random variables (RVs) with zero mean and unit variance, i.e., ${\vartheta_{0,l}}, {\vartheta_{k,l}} \sim \mathcal{CN}\left( {0,1} \right)$, $m$ represents the fading order that indicates the severity of fading channels, ${\sigma_k}^2$ denotes Nakagami spread and is equal to the expectation of the squared channel magnitude, i.e., ${\rm  E}\{|h_k|^2\}={\sigma_k}^2$, and ${\boldsymbol{\lambda }} = \left( {{\lambda _1},{\lambda _2}, \cdots ,{\lambda _K}} \right)$ specifies the generalized time correlation among fading channels of all HARQ transmissions \cite{beaulieu2011novel}. Under this model, the magnitude of channel coefficient ${\left| {{h_{k}}} \right|}$ follows a Nakagami-$m$ distribution, i.e., $\left| {{h_k}} \right| \sim {\rm Nakagami}(m,{\sigma_k}^2)$, and the channel magnitudes ${\left| {{h_{l}}} \right|}$ and ${\left| {{h_{k}}} \right|}$ are correlated. Moreover, the cross correlation coefficient between the squared channel magnitudes ${\left| {{h_{l}}} \right|^2}$ and ${\left| {{h_{k}}} \right|^2}$ is determined by the time correlation coefficients $\bs \lambda$ as \cite{beaulieu2011novel}
\begin{align}
\label{eqn_cor_fa}
 \frac{{{\rm{E}}\left( {{{\left| {{h_{l}}} \right|}^2}{{\left| {{h_{k}}} \right|}^2}} \right) - {\rm{E}}\left( {{{\left| {{h_{l}}} \right|}^2}} \right){\rm{E}}\left( {{{\left| {{h_{k}}} \right|}^2}} \right)}}{{\sqrt {{\rm{Var}}\left( {{{\left| {{h_{l}}} \right|}^2}} \right){\rm{Var}}\left( {{{\left| {{h_{k}}} \right|}^2}} \right)} }}
= {\lambda _{l}}^2{\lambda _{k}}^2,\, 1 \le l \ne k \le K,
\end{align}
where ${\rm Var}(\cdot)$ denotes the variance of the random variable in the brackets. Without loss of generality, the time correlation coefficients are assumed to be non-negative, i.e., $0 \le \lambda _1,\cdots,\lambda _K \le 1$. Clearly from (\ref{eqn:R_k_def}), when ${\boldsymbol{\lambda }} = {\bf 1}_K$, the time-correlated fading channel reduces to quasi-static fading channel with $h_1=h_2=\cdots=h_K$, while when ${\boldsymbol{\lambda }} = {\bf 0}_K$, the time-correlated fading channel reduces to a fast fading channel where the channel coefficients $h_1$, $h_2$, $\cdots$, $h_K$ are mutually independent. In other words, the Nakagami-$m$ fading channel with generalized time correlation in (\ref{eqn:R_k_def}) includes quasi-static fading channel and fast fading channel as its special cases. Here ${\bf 1}_K$ and ${\bf 0}_K$ denote an all-ones vector and a null vector, each with length $K$, respectively. Unless otherwise indicated, subscript $K$ is omitted in the sequel.


According to (\ref{eqn:sign_mod}), the received signal-to-noise ratio (SNR) in the $k$th transmission is given by
\begin{equation}\label{eqn:SNR}
{\gamma _k} = \frac{{P_k}{\left| {{h_k}} \right|^2}}{\mathcal N_0} ,
\end{equation}
where $P_k$ denotes the transmitted signal power in the $k$th transmission. Since $\left| {{h_k}} \right|$ is Nakagami-$m$ distributed, i.e., $\left| {{h_k}} \right| \sim {\rm Nakagami}(m,{\sigma_k}^2)$, it is readily proved that ${\gamma _k} $ complies with \emph{Gamma distribution}, i.e., ${\gamma _k} \sim \mathcal G\left( {m,\frac{{P_k}{\sigma _k}^2}{m\mathcal N_0}} \right)$. Due to the time correlation among the channel coefficients as given in (\ref{eqn_cor_fa}), the SNRs $\bs \gamma = (\gamma _1,\gamma _2,\cdots,\gamma _K)$ are correlated Gamma RVs.% To proceed with our analysis, the joint probability density function (PDF) of correlated SNRs $\bs \gamma$ will be first derived.


%\section{Problem Formulation}
\subsection{Outage Formulation}
Outage probability has been proved as the most fundamental performance metric of HARQ schemes \cite{caire2001throughput}. For HARQ-IR, the outage probability is directly determined by the CDF of accumulated mutual information. Specifically, assuming that information-theoretic capacity achieving channel coding is adopted for HARQ-IR, an outage event happens when the information cannot be successfully decoded after $K$ transmissions, i.e.,  the accumulated mutual information $I_K$ is below the information transmission rate $\mathcal R$. Notice that the information transmission rate $\mathcal R$ depends on the coding rate $c$, modulation scheme and symbol transmission rate. The outage probability after $K$ transmissions is thus written as
\begin{equation}\label{eqn:out_prob_def}
{p_{out,K}} = \Pr \left( {{I_K}  < \mathcal R} \right) = F_{{I_K}}(\mathcal R).
\end{equation}
where $F_{{I_K}}(\cdot)$ denotes the CDF of ${I_K}$. In HARQ-IR protocol, decoding at each transmission is performed based on the combined codewords received in the previously transmissions and the current transmission and the accumulated mutual information after $K$ transmissions is given by
\begin{equation}\label{eqn:def:accum_inf}
{I_K} = \sum\nolimits_{k = 1}^K {{{\log }_2}\left( {1 + {\gamma _k}} \right)}.
\end{equation}
Accordingly, the outage probability ${p_{out,K}}$ becomes
\begin{equation}\label{eqn:out_prob_rew}
{p_{out,K}} = \Pr \left( {G_K \triangleq \prod\nolimits_{k = 1}^K {\left( {1 + {\gamma _k}} \right)}  < {2^{\mathcal R}}} \right) = {F_{G_K}}\left( {{2^{\mathcal R}}} \right),
\end{equation}
where $F_{G_K}(\cdot)$ denotes the CDF of $G_K$. Noticing that $(\gamma _1,\gamma _2,\cdots,\gamma _K)$ are correlated Gamma RVs, the derivation of outage probability ${p_{out,K}}$ essentially turns to determining the CDF of the product of multiple correlated shifted-Gamma RVs, i.e., ${F_{G_K}}\left( {{x}} \right)$. Due to the presence of time correlation and Gamma distribution, the outage analysis  is more challenging than those for quasi-static \cite{shen2009average,makki2013green} or fast fading channels \cite{to2015power,kim2014optimal,sesia2004incremental,jinho2013energy,stanojev2009energy,aghajanzadeh2011outage,
yilmaz2009productshifted,yilmaz2010outage,chelli2013performance,larsson2014throughput,larsson2016throughput}.  %Due to the involvement of multiple correlated RVs, it is more challenging to derive a closed-form expression for ${p_{out,K}}$ compared with the cases under quasi-static fading channels ($\bs \lambda = \bf 1$) and fast fading channels ($\bs \lambda = \bf 0$).

It is worth noting that the CDF of the product of multiple shifted and correlated RVs has numerous applications in wireless communications and the outage formulation in (\ref{eqn:out_prob_def}) is also applicable to parallel transmission systems and orthogonal frequency-division multiplexing (OFDM) systems \cite{bai2013outage,luo2005service}. Specifically, in OFDM systems, coded signals are transmitted through multiple subcarriers. Due to close frequency spacing of the subcarriers and limited Doppler spread, the fading channels on multiple subcarriers are usually correlated \cite{lu2002ldpc}. The outage analysis in OFDM systems is thus also reduced to the analysis of the CDF of the product of multiple shifted and correlated RVs in (\ref{eqn:out_prob_rew}).


%the CDF of the product of multiple shifted and correlated RVs has numerous applications in wireless communications. For example, (\ref{eqn:out_prob_def}) can be regarded as the outage probability of the signal transmission over parallel fading channels in multi-carrier transmission systems or orthogonal frequency-division multiplexing (OFDM) systems \cite{bai2013outage,luo2005service}. Due to close frequency spacing of the subcarriers and limited Doppler spread, the fading channels on multiple subcarriers in OFDM systems  are usually correlated \cite{lu2002ldpc}. Our analysis thus has a wide range of applications, which justifies the significance of our work.

%, where parallel fading channel model is frequently adopted to model orthogonal frequency-division multiplexing (OFDM) system \cite{bai2013outage,luo2005service}.

\section{Probability Distribution of the Product of Multiple Correlated Shifted-Gamma RVs}\label{sec:exa}
%In order to conduct the asymptotic outage analysis of HARQ-IR, it is necessary to exactly derive the outage probability. Unlike the outage analysis under quasi-static \cite{shen2009average,makki2013green} or fast fading channels \cite{to2015power,kim2014optimal,sesia2004incremental,jinho2013energy,stanojev2009energy,aghajanzadeh2011outage,
%yilmaz2009productshifted,yilmaz2010outage,chelli2013performance,larsson2014throughput,larsson2016throughput}, it is more challenging to derive an exact expression for outage probability ${p_{out,K}}$ when fading channels are time-correlated and Nakagami distributed.
As aforementioned, the derivation of outage probability ${p_{out,K}}$ essentially turns to determining the CDF of the product of multiple shifted and correlated Gamma RVs. It is mathematically difficult due to the presence of time correlation. In the literature, only a few \emph{approximated} results are available in \cite{yang2014performance,shi2015analysis,shi2016inverse}. They are complicated without clear insights and thus are not favorable for system design. Moreover, their accuracy can not be guaranteed under certain scenarios. In this paper, we will derive the \emph{exact} CDF of the product of multiple shifted and correlated Gamma RVs, based on which a truncation approximation with uniform convergence can be proposed and asymptotic outage probability can be derived in a simple form with clear insights.

\subsection{Exact CDF}
From (\ref{eqn:R_k_def}) and (\ref{eqn:SNR}), we can find that the SNRs $\{\gamma _1,\gamma _2,\cdots,\gamma _K\}$ are conditionally independent, given the complex Gaussian RVs ${\vartheta_{0,1}},\cdots,{\vartheta_{0,m}}$. It has been proved in \cite[Theorem 1.3.4]{muirhead2009aspects} that $\{\gamma _1,\gamma _2,\cdots,\gamma _K\}$ follow \emph{independent noncentral chi-squared distributions} with $2m$ degrees-of-freedom when conditioned on $T \triangleq \sum\nolimits_{l = 1}^m {{{\left| {\vartheta_{0,l}} \right|}^2}} $. The conditional probability density function (PDF) is given in the following lemma.
\begin{lemma}\cite[Theorem 1.3.4]{muirhead2009aspects}
The conditional PDF of the SNR $\gamma_k$ given $T =t$ is written as
\begin{align}\label{eqn:cond_pdf_r_k_rew}
{{f_{\left. {{\gamma _k}} \right|T}}\left( {\left. {{x_k}} \right|t} \right)} &{ = {{\left( {\frac{1}{{{\Omega _k}}}} \right)}^m}\frac{{{x_k}^{m - 1}}}{{\Gamma \left( m \right)}}{e^{ - \frac{{{u_k}{\lambda _k}^2t + {x_k}}}{{{\Omega _k}}}}}_0{F_1}\left( {;m;{{\left( {\frac{{\sqrt {{u_k}{\lambda _k}^2{x_k}t} }}{{{\Omega _k}}}} \right)}^2}} \right),}\,  {{\lambda _k}} \ne 1,
\end{align}
where ${u_k} = \frac{{P_k}{\sigma _k}^2}{m\mathcal N_0}$, $\Omega _k = \frac{{P_k}{\sigma _k}^2\left( {1 - {\lambda _k}^2} \right)}{m\mathcal N_0}$,  $\Gamma(\cdot)$ and ${}_0F_1(\cdot)$ denote Gamma function and the confluent hypergeometric limit function \cite[Eq.16.2.1]{olver2010nist}, respectively.
%\footnote{Although $m$ is assumed to be an integer in (\ref{eqn:R_k_def}), the conditional PDF (\ref{eqn:cond_pdf_r_k_rew}) can factually be extended to the cases with arbitrary $m \ge \frac{1}{2}$ \cite{beaulieu2011novel}.}.
\end{lemma}
It should be mentioned that (\ref{eqn:cond_pdf_r_k_rew}) is not applicable to quasi-static fading channels, i.e., $\bs \lambda = \bf 1$, because of the occurrence of singularity. Due to its speciality, the outage analysis of HARQ-IR over quasi-static fading channels will be discussed separately. %The performance of HARQ-IR over quasi-static fading channels will be discussed individually.

Noticing that Mellin transform\footnote{
The Mellin transform with respect to a function $f(x)$ is defined as $\left\{ {\mathcal M f} \right\}\left( s \right) = \int\nolimits_0^\infty  {{x^{s - 1}}f\left( x \right)dx} \triangleq \tilde f\left( s \right)$, and the associated inverse Mellin transform is given by $f\left( x \right) = \frac{1}{{2\pi {\rm{i}}}}\int\nolimits_{c - {\rm{i}}\infty }^{c + {\rm{i}}\infty } {{x^{ - s}}\tilde f\left( s \right)dx}$ \cite{debnath2010integral}.} is a powerful mathematical tool to obtain the probability distribution of a product of multiple independent RVs \cite{yilmaz2009productshifted,yilmaz2010outage,chelli2013performance}, the conditional independence of SNRs $\bs \gamma$ given $T$ inspires us to derive the conditional PDF of $G_K$ in closed-form by using conditional Mellin transform. Specifically, with the conditional independence, the Mellin transform of the conditional PDF of $G_K$ can be written as a product of the Mellin transforms corresponding to the conditional PDFs of the shifted SNRs. With this special property, as proved in Appendix \ref{app:phi_derivation}, the conditional PDF of $G_K$ given $T=t$, ${f_{\left. G_K \right|T}}(x|t)$, can be derived as
\begin{align}\label{eqn:g_hat_cond_t}
 {f_{\left. { G_K} \right|T}}\left( {\left. x \right|t} \right)
% &= \left\{ {{M^{ - 1}} \phi } \right\}\left( s \right) = \frac{1}{{2\pi {\rm{i}}}}\int\nolimits_{c - {\rm{i}}\infty }^{c + {\rm{i}}\infty } {{x^{ - s}}}  \phi \left( {\left. s \right|t} \right)ds \notag \\
% &= \frac{1}{{2\pi {\rm{i}}}}\int\nolimits_{c - {\rm{i}}\infty }^{c + {\rm{i}}\infty } {{x^{ - s}}} \sum\limits_{{\ell_1}, \cdots ,{\ell_K} = 0}^\infty  {{t^{\sum\limits_{k = 1}^K {{\ell_k}} }}{e^{ - t\sum\limits_{k = 1}^K {\frac{{{\lambda _k}^2}}{{1 - {\lambda _k}^2}}} }}\prod\limits_{k = 1}^K {\frac{{{{\left( {\frac{{{\lambda _k}^2}}{{1 - {\lambda _k}^2}}} \right)}^{{\ell_k}}}}}{{{\ell_k}!{{\left( {{\Omega _k}} \right)}^{m + {\ell_k}}}}}\Psi \left( {m + {\ell_k},m + {\ell_k} + s;\frac{1}{{{\Omega _k}}}} \right)} } ds \notag \\
& = \sum\limits_{{\ell_1}, \cdots ,{\ell_K} = 0}^\infty  {{t^{\sum\nolimits_{k = 1}^K {{\ell_k}} }}{e^{ - t\sum\nolimits_{k = 1}^K {\frac{{{\lambda _k}^2}}{{1 - {\lambda _k}^2}}} }}\prod\limits_{k = 1}^K {\frac{1}{{{\ell_k}!}}{{{{\left( {\frac{{{\lambda _k}^2}}{{1 - {\lambda _k}^2}}} \right)}^{{\ell_k}}}}}} } {f_{{{ {\mathcal A}}_{\bs{\ell}}}}}(x),
\end{align}
where ${ \bs{\ell}}$ is a vector with $K$ elements as ${ \bs{\ell}} = (\ell_1,\cdots,\ell_K)$ and ${{f_{{{ {\cal A}}_{\bs{\ell}}}}}(x)}$ is given by
\begin{align}\label{eqn:pdf_produc_gamma_ind_sec}
{f_{{{ {\mathcal A}}_{\bs{\ell}}}}}(x) &= \frac{1}{{2\pi {\rm{i}}}}\int\nolimits_{c - {\rm{i}}\infty }^{c + {\rm{i}}\infty } {\prod\limits_{k = 1}^K {\frac{{\Psi \left( {m + {\ell_k},s + m + {\ell_k};\frac{1}{{{\Omega _k}}}} \right)}}{{{{\left( {{\Omega _k}} \right)}^{m + {\ell_k}}}}}} {x^{ - s}}ds}\notag\\
&= \frac{1}{{\prod\nolimits_{k = 1}^K {{\Omega _k}} }}Y_{0,K}^{K,0}\left[ {\left. {\begin{array}{*{20}{c}}
 - \\
{\left( {0,1,\frac{1}{{{\Omega _1}}},m + {\ell_1}} \right), \cdots ,\left( {0,1,\frac{1}{{{\Omega _K}}},m + {\ell_K}} \right)}
\end{array}} \right|\frac{x}{{\prod\nolimits_{k = 1}^K {{\Omega _k}} }}} \right],
\end{align}
with ${\rm i} = \sqrt{-1}$, $\Psi \left( {\alpha ,\gamma ;z} \right) = \frac{1}{{\Gamma \left( \alpha  \right)}}\int\nolimits_0^\infty  {{e^{ - zt}}{t^{\alpha  - 1}}{{\left( {1 + t} \right)}^{\gamma  - \alpha  - 1}}dt}$ denoting Tricomi's confluent hypergeometric function \cite[Eq.9.211.4]{gradshteyn1965table}, and $Y_{p,q}^{m,n}[\cdot]$ denoting the generalized Fox's H function defined by Mellin-Barnes integral as \cite{yilmaz2010outage,chelli2013performance}
%The definition of Fox's H function is given in Appendix \ref{app:fox_H_fun},
%For the generalized Fox H's function, it is explicitly defined in [12,13], as copied below.
%\begin{mydef}%\cite{yilmaz2010outage,chelli2013performance},
% frequently adopted to formulate the distribution of the product of independent shifted-Gamma RVs in a generalized form, and
\begin{equation}\label{eqn_fox_h_def}
Y_{p,q}^{m,n}\left[ {\left. {\begin{array}{*{20}{c}}
{\left( {{a_1},{\alpha _1},{ A_1},{\varphi _1}} \right), \cdots ,\left( {{a_p},{\alpha _p},{ A_p},{\varphi _p}} \right)}\\
{\left( {{b_1},{\beta _1},{B_1},{\phi _1}} \right), \cdots ,\left( {{b_q},{\beta _q},{B_q},{\phi _q}} \right)}
\end{array}} \right|x} \right] = \frac{1}{{2\pi {\rm i}}}\int_{\mathcal L} {M_{p,q}^{m,n}\left[ s \right]{x^{ - s}}ds},
\end{equation}
where $\mathcal{L}$ is a Mellin-Barnes contour in the complex $s$-plane running from $c-{\rm i}\infty $ to $c+{\rm i}\infty$, $c \in \mathbb{R}$ and $M_{p,q}^{m,n}\left[ s \right]$ is written as
\begin{multline}\label{eqn:def_M}
M_{p,q}^{m,n}\left[ s \right] = \frac{{\prod\nolimits_{j = 1}^m {{B_j}^{{\phi _j} + {b_j} + {\beta _j}s - 1}\Psi \left( {{\phi _j},{\phi _j} + {b_j} + {\beta _j}s;{B_j}} \right)} }}{{\prod\nolimits_{i = n + 1}^p {{ A_i}^{{\varphi _i} + {a_i} + {\alpha _i}s - 1}\Psi \left( {{\varphi _i},{\varphi _i} + {a_i} + {\alpha _i}s;{ A_i}} \right)} }}\times \\
\frac{{\prod\nolimits_{i = 1}^n {{ A_i}^{{\varphi _i} - {a_i} - {\alpha _i}s}\Psi \left( {{\varphi _i},{\varphi _i} + 1 - {a_i} - {\alpha _i}s;{ A_i}} \right)} }}{{\prod\nolimits_{j = m + 1}^q {{B_j}^{{\phi _j} - {b_j} - {\beta _j}s}\Psi \left( {{\phi _j},{\phi _j} + 1 - {b_j} - {\beta _j}s;{B_j}} \right)} }}.
\end{multline}
%and the contour $\mathcal{L}$ is specially choosen such that it is a Mellin-Barnes contour in the complex $s$-plane running from $c-\infty $ to $c+\infty$, $c \in \mathbb{R}$ and the points $s=(-b_j-k)/\beta_j$ for $1 \le j\le m$ and $k=0,1,2,\cdots$ and the points $s=(1-a_i+k)/\alpha_i$ for $1 \le i \le n$ and $k=0,1,2,\cdots$ lie to the left and right of the chosen contour, respectively.\\
%\end{mydef}
Notice that the generalized Fox's H function does have two special properties as shown in Appendix \ref{app:fox_H_fun} which can simplify our mathematical derivations, and an efficient MATHEMATICA{\textregistered} implementation of the generalized Fox's H function in (\ref{eqn_fox_h_def}) can be found in \cite{yilmaz2010outage}. Interestingly from (\ref{eqn:pdf_produc_gamma_ind_sec}), $f_{{\mathcal A}_{\bs{\ell}}}(x)$ is the inverse Mellin transform of a product function $\prod\nolimits_{k = 1}^K {\frac{{\Psi \left( {m + {\ell_k},s + m + {\ell_k};\frac{1}{{{\Omega _k}}}} \right)}}{{{{\left( {{\Omega _k}} \right)}^{m + {\ell_k}}}}}}$.
%  the Mellin transform of $f_{{\mathcal A}_{\bs{\ell}}}(x)$ is
%\begin{equation}\label{eqn:meillin_trans_ind_gam}
%\left\{ {\mathcal M{f_{{{ {\mathcal A}}_{\bs{\ell}}}}}} \right\}\left( s \right) = \prod\limits_{k = 1}^K {\int\nolimits_0^\infty  {\frac{{{{\left( {1 + x} \right)}^{s - 1}}{x^{m + {\ell_k} - 1}}{e^{ - \frac{x}{{{\Omega _k}}}}}}}{{{{\left( {{\Omega _k}} \right)}^{m + {\ell_k}}}\Gamma \left( {m + {\ell_k}} \right)}}dx} }  = \prod\limits_{k = 1}^K {\frac{{\Psi \left( {m + {\ell_k},s + m + {\ell_k};\frac{1}{{{\Omega _k}}}} \right)}}{{{{\left( {{\Omega _k}} \right)}^{m + {\ell_k}}}}}}.
%\end{equation}
%\begin{equation}\label{eqn:meillin_trans_ind_gam}
%\left\{ {\mathcal M{f_{{{ {\mathcal A}}_{\bs{\ell}}}}}} \right\}\left( s \right) = \prod\limits_{k = 1}^K {\frac{{\Psi \left( {m + {\ell_k},s + m + {\ell_k};\frac{1}{{{\Omega _k}}}} \right)}}{{{{\left( {{\Omega _k}} \right)}^{m + {\ell_k}}}}}}.
%\end{equation}
Here the multiplier ${\frac{{\Psi \left( {m + {\ell_k},s + m + {\ell_k};\frac{1}{{{\Omega _k}}}} \right)}}{{{{\left( {{\Omega _k}} \right)}^{m + {\ell_k}}}}}}$ in fact is the Mellin transform of the PDF of a shifted-Gamma RV $(1+R_{{\bs{\ell}},k})$, where $R_{{\bs{\ell}},k} \sim \mathcal {G}(m+\ell_k,{\Omega _k})$. Therefore, ${f_{{{ {\mathcal A}}_{\bs{\ell}}}}}(x)$ can be regarded as the PDF of the product of $K$ independent shifted-Gamma RVs i.e., $\mathcal A_{\bs{\ell}} = \prod\nolimits_{k = 1}^K (1+R_{{\bs{\ell}},k})$.
%
%, where $R_{{\bs{\ell}},k} \sim \mathcal {G}(m+\ell_k,{\Omega _k})$ \footnote{
%This is because that the Mellin transform of $f_{{\mathcal A}_{\bs{\ell}}}(x)$ is equivalent to (\ref{eqn:pdf_produc_gamma_ind_sec}), such that
%\begin{equation*}\label{eqn:meillin_trans_ind_gam}
%\left\{ {\mathcal M{f_{{{ {\mathcal A}}_{\bs{\ell}}}}}} \right\}\left( s \right) = \prod\limits_{k = 1}^K {\int\nolimits_0^\infty  {\frac{{{{\left( {1 + x} \right)}^{s - 1}}{x^{m + {\ell_k} - 1}}{e^{ - \frac{x}{{{\Omega _k}}}}}}}{{{{\left( {{\Omega _k}} \right)}^{m + {\ell_k}}}\Gamma \left( {m + {\ell_k}} \right)}}dx} }  = \prod\limits_{k = 1}^K {\frac{{\Psi \left( {m + {\ell_k},s + m + {\ell_k};\frac{1}{{{\Omega _k}}}} \right)}}{{{{\left( {{\Omega _k}} \right)}^{m + {\ell_k}}}}}}.
%\end{equation*}
%}.

With (\ref{eqn:g_hat_cond_t}), the PDF of $G_K$ can be eventually obtained by integrating the conditional probability ${f_{\left. G_K \right|T}}\left( {\left. x \right|t} \right)$ over the distribution of $T$, such that
\begin{equation}\label{eqn_mar_G}
{{f_{G_K}}\left( x \right) = {\rm E}_{T}\left\{{f_{\left. G_K \right|T}}\left( x|t \right)\right\} = \int\nolimits_0^\infty  {{f_{\left. G_K \right|T}}\left( {\left. x \right|t} \right){f_T}\left( t \right)dt} },
\end{equation}
where ${f_T}\left( t \right)$ denotes the PDF of $T$.
With respect to the distribution of $T \triangleq \sum\nolimits_{l = 1}^m {{{\left| {\vartheta_{0,l}} \right|}^2}} $, since $\vartheta_{0,1},\cdots,\vartheta_{0,m}$ are independent and identically distributed (i.i.d.) complex Gaussian RVs with zero mean and unit variance, it can be proved that $\left|\vartheta_{0,1}\right|^2,\cdots,\left|\vartheta_{0,m}\right|^2$ are i.i.d. Gamma RVs, i.e., ${{\left| {\vartheta_{0,l}} \right|}^2} \sim \mathcal{G}\left({1,1}\right)$. As a sum of $m$ independent Gamma RVs with identical scale parameter, the random variable $T$ then follows a Gamma distribution with the PDF of \cite[p289]{degroot1986probability}
\begin{equation}\label{eqn_pdf_T_nak}
{f_T}\left( t \right) = \frac{{{t^{m - 1}}}}{{\Gamma \left( m \right)}}{e^{ - t}},\, t \ge 0.
\end{equation}

Plugging (\ref{eqn:g_hat_cond_t}) and (\ref{eqn_pdf_T_nak}) into (\ref{eqn_mar_G}), it follows by using \cite[Eq.3.381.4]{gradshteyn1965table} that
\begin{align}\label{eqn_pdf_f_g_1_p}
{f_{ G_K}}\left( x \right)
%&= \int\nolimits_0^\infty  {{f_{\left. { G} \right|T}}\left( {\left. x \right|t} \right){f_T}\left( t \right)dt} \notag \\
%& = \frac{1}{{\Gamma \left( m \right)}}\int\nolimits_0^\infty  {{f_{\left. { G} \right|T}}\left( {\left. x \right|t} \right){t^{m - 1}}{e^{ - t}}dt} \notag\\
 &= \frac{1}{{\Gamma \left( m \right)}}\sum\limits_{{\ell_1}, \cdots ,{\ell_K} = 0}^\infty  {\int\nolimits_0^\infty  {{t^{\sum\limits_{k = 1}^K {{\ell_k}}  + m - 1}}{e^{ - t\left( {1 + \sum\limits_{k = 1}^K {\frac{{{\lambda _k}^2}}{{1 - {\lambda _k}^2}}} } \right)}}dt} \prod\limits_{k = 1}^K {\frac{1}{{{\ell_k}!}}{{{{\left( {\frac{{{\lambda _k}^2}}{{1 - {\lambda _k}^2}}} \right)}^{{\ell_k}}}}}{f_{{{ {\mathcal A}}_{\bs{\ell}}}}}(x)} } \notag\\
 %= \frac{1}{{\Gamma \left( m \right)}}\sum\limits_{{\ell _1}, \cdots ,{\ell _K} = 0}^\infty  {\frac{{\Gamma \left( {m{\rm{ + }}\sum\limits_{k = 1}^K {{\ell _k}} } \right)}}{{{{\left( {1 + \sum\limits_{k = 1}^K {\frac{{{\lambda _k}^2}}{{1 - {\lambda _k}^2}}} } \right)}^{m{\rm{ + }}\sum\limits_{k = 1}^K {{\ell _k}} }}}}\prod\limits_{k = 1}^K {\frac{{{{\left( {\frac{{{\lambda _k}^2}}{{1 - {\lambda _k}^2}}} \right)}^{{\ell _k}}}}}{{{\ell _k}!}}{f_{{{\cal A}_\ell }}}(x)} }
 &= \sum\nolimits_{{\ell_1}, \cdots ,{\ell_K} = 0}^\infty  {{W_{\bs{\ell}}}{f_{{{ {\mathcal A}}_{\bs{\ell}}}}}(x)},
 %&= \sum\limits_{{\ell_1}, \cdots ,{\ell_K} = 0}^\infty  {\underbrace {\frac{{\Gamma \left( {m + \sum\limits_{k = 1}^K {{\ell_k}} } \right)}}{{\Gamma \left( m \right)}}{{\left( {1 + \sum\limits_{k = 1}^K {\frac{{{\lambda _k}^2}}{{1 - {\lambda _k}^2}}} } \right)}^{ - m}}\prod\limits_{k = 1}^K {\frac{{{w_k}^{{\ell_k}}}}{{{\ell_k}!}}} }_{{W_{\bs{\ell}}}}{f_{{{\cal A}_{\bs{\ell}}}}}(x)},\\
 %&=\sum\limits_{{\ell_1}, \cdots ,{\ell_K} = 0}^\infty  {\underbrace {{{\left( {1 + \sum\limits_{k = 1}^K {\frac{{{\lambda _k}^2}}{{1 - {\lambda _k}^2}}} } \right)}^{ - m}}{{\left( m \right)}_{\sum\limits_{k = 1}^K {{\ell_k}} }}\prod\limits_{k = 1}^K {\frac{{{w_k}^{{\ell_k}}}}{{{\ell_k}!}}} }_{{W_{\bs{\ell}}}}{f_{{{\cal A}_{\bs{\ell}}}}}(x)} ,
\end{align}
where ${{\bs{\ell}}}$ is a vector with $K$ elements as ${{\bs{\ell}}} = (\ell_1,\cdots,\ell_K)$ and the weighting ${W_{\bs{\ell}}}$ is given as
\begin{equation}\label{eqn:W_l_def_sec}
%{W_{\bs{\ell}}} \triangleq {\left( {1 + \sum\limits_{k = 1}^K {\frac{{{\lambda _k}^2}}{{1 - {\lambda _k}^2}}} } \right)^{ - m}}{\left( m \right)_{\sum\limits_{k = 1}^K {{\ell_k}} }}\prod\limits_{k = 1}^K {\frac{{{w_k}^{{\ell_k}}}}{{{\ell_k}!}}} ,
{W_{\bs{\ell}}} \triangleq \frac{{\Gamma \left( {m + \sum\nolimits_{k = 1}^K {{\ell_k}} } \right)}}{{\Gamma \left( m \right)}}{\left( {1 + \sum\limits_{k = 1}^K {\frac{{{\lambda _k}^2}}{{1 - {\lambda _k}^2}}} } \right)^{ - m}}\prod\limits_{k = 1}^K {\frac{{{w_k}^{{\ell_k}}}}{{{\ell_k}!}}},\, {{\bs{\ell}=(\ell_1,\cdots,\ell_K)} \in {{\mathbb N}_0}^K},
\end{equation}
${w_k} = \frac{{{\lambda _k}^2}}{{1 - {\lambda _k}^2}}{\left( {1 + \sum\nolimits_{l = 1}^K {\frac{{{\lambda _l}^2}}{{1 - {\lambda _l}^2}}} } \right)^{ - 1}}$ and ${{\mathbb N}_0}^K$ denotes $K$-ary Cartesian power of natural number set. %and $(q)_n$ is the Pochhammer symbol. probability mass function (pmf) of negative multinomial distribution
By comparing (\ref{eqn:W_l_def_sec}) with \cite[Eq.13.8-1]{bishop2007discrete}, it is found that $\left\{{W_{\bs{\ell}}},{{{\bs{\ell}}} \in {{\mathbb N}_0}^K}\right\}$  are probabilities of a negative multinomial distributed vector RV ${\bs{\ell}}$, i.e., ${\bs{\ell}} \sim {\rm NM}(m,\bf w)$, where ${\bf w}=(w_1,\cdots,w_K)$. Clearly, we have $\sum\nolimits_{{\bs{\ell}} \in {{\mathbb N}_0}^K}  {{W_{\bs{\ell}}}}  = 1$. As such, ${f_{ G_K}}\left( x \right)$ is expressed as a weighted sum of the PDFs corresponding to $\left\{\mathcal A_{\bs{\ell}},{{{\bs{\ell}}} \in {{\mathbb N}_0}^K}\right\}$.%As a consequence, the PDF of $G$ shown in (\ref{eqn_pdf_f_g_1_p}) can be rephrased as a weighted sum of PDFs of products of independent shifted-Gamma RVs $\left\{\mathcal A_{\bs{\ell}},{{{\bs{\ell}}} \in {{\mathbb N}_0}^K}\right\}$.
%, such that
%\begin{equation}\label{eqn:f_G_def_sec}
%{f_G}\left( x \right) = {{\rm{E}}_{\bs{\ell}}}\left\{ {{f_{{\mathcal A_{\bs{\ell}}}}}(x)}\right\} = \sum\limits_{{{\bs{\ell}}} \in {{\mathbb N}_0}^K}  {{W_{\bs{\ell}}}{f_{{\mathcal A_{\bs{\ell}}}}}(x)}.
%\end{equation}

Based on (\ref{eqn_pdf_f_g_1_p}), the CDF of the product of multiple correlated shifted-Gamma RVs ${F_{ G_K}}\left( x \right)$ can be derived as shown in the following theorem.
\begin{theorem}\label{the:cdf_pdf_corr_gam_shif}
The CDF of $G_K = \prod\nolimits_{k = 1}^K {\left( {1 + {\gamma _k}} \right)}$ is given by
\begin{equation}\label{eqn:CDF_G_def_sec}
{F_{ G_K}}\left( x \right) = \int\nolimits_0^x {{f_{ G_K}}\left( t \right)dt}  %= {{\rm{E}}_{\bs{\ell}}}\left\{ {\int\nolimits_0^x {{f_{{{ {\mathcal A}}_{\bs{\ell}}}}}(t)dt}}\right\}= {{\rm{E}}_{\bs{\ell}}}\left\{ {{F_{{\mathcal A_{\bs{\ell}}}}}(x)}\right\} = \sum\limits_{{{\bs{\ell}}} \in {{\mathbb N}_0}^K}  {{W_{\bs{\ell}}}{F_{{{ {\cal A}}_{\bs{\ell}}}}}\left( x \right)}
=\sum\nolimits_{{\ell_1}, \cdots ,{\ell_K} = 0}^\infty  {{W_{\bs{\ell}}}\int\nolimits_0^x {{f_{{{ {\mathcal A}}_{\bs{\ell}}}}}(t)dt}}
= \sum\nolimits_{{\ell_1}, \cdots ,{\ell_K} = 0}^\infty  {{W_{\bs{\ell}}}{F_{{{ {\mathcal A}}_{\bs{\ell}}}}}(x)},
\end{equation}
where ${{F_{{{ {\mathcal A}}_{\bs{\ell}}}}}(x)}$ denotes the CDF of the product of independent shifted-Gamma RVs $\mathcal A_{\bs{\ell}} = \prod\nolimits_{k = 1}^K (1+R_{{\bs{\ell}},k})$ with $R_{{\bs{\ell}},k} \sim \mathcal {G}(m+\ell_k,{\Omega _k})$ and ${{\bs{\ell}}} = (\ell_1,\cdots,\ell_K)$. Specifically, ${{F_{{{ {\mathcal A}}_{\bs{\ell}}}}}(x)}$ can be expressed in terms of the generalized Fox's H function as
%$\mathcal {G}(m+\ell_k,{\Omega _k})$
\begin{multline}\label{eqn:CDF_F_A_def_sec}
{F_{{{ {\mathcal A}}_{\bs{\ell}}}}}(x) = Y_{1,K + 1}^{K,1}\left[ {\left. {\begin{array}{*{20}{c}}
{\left( {1,1,0,1} \right)}  \\
{\left( {1,1,\frac{1}{{{\Omega _1}}},m + {\ell_1}} \right), \cdots ,\left( {1,1,\frac{1}{{{\Omega _K}}},m + {\ell_K}} \right),\left( {0,1,0,1} \right)}
\end{array}} \right|\frac{x}{{\prod\nolimits_{k = 1}^K {{\Omega _k}} }}} \right].
\end{multline}
\end{theorem}
\begin{proof}
Please see Appendix \ref{app:fox_H_fun}.

% holds by using the property 3 regarding generalized Fox's H Function in Appendix \ref{app:fox_H_fun}.
\end{proof}

%Therefore, it is found from Theorem \ref{the:cdf_pdf_corr_gam_shif} that the CDF of $G$ is a weighted sum of CDFs of products of independent shifted-Gamma RVs $\left\{\mathcal A_{\bs{\ell}},{{{\bs{\ell}}} \in {{\mathbb N}_0}^K}\right\}$, where the weightings are probabilities of negative multinomial distribution ${\rm NM}(m,\bf w)$.
%Accordingly, the outage probability ${p_{out,K}}$ can be obtained on the basis of Theorem \ref{the:cdf_pdf_corr_gam_shif}.
%\subsection{Outage Probability ${p_{out,K}}$ and Its Implementation}\label{sec:out_phy}
Therefore, the outage probability ${p_{out,K}}$ can be obtained by substituting (\ref{eqn:CDF_G_def_sec}) into (\ref{eqn:out_prob_rew}), such that
\begin{equation}\label{eqn:out_prob_def_hat}
{p_{out,K}} = {F_{ G}}\left( {{2^{\mathcal R}} } \right) = \sum\nolimits_{{\ell_1}, \cdots ,{\ell_K} = 0}^\infty  {{W_{\bs{\ell}}}{F_{{{ {\mathcal A}}_{\bs{\ell}}}}}(2^{\mathcal R})}.
\end{equation}
%Due to the representation of infinite series of (\ref{eqn:out_prob_def_hat}), it is impossible to accurately compute (\ref{eqn:out_prob_def_hat}).
%To enable the evaluation of (\ref{eqn:out_prob_def_hat}), ${p_{out,K}}$ can be approximately computed by truncating (\ref{eqn:out_prob_def_hat}) to a sum of finite terms, and its computational accuracy will be verified in Section \ref{sec:num_res}. Furthermore, meaningful insights can be extracted from (\ref{eqn:out_prob_def_hat}), as given in the following remark.
%\begin{remark}\label{rem:chann}
Notice that $\mathcal A_{\bs{\ell}} = \prod\nolimits_{k = 1}^K (1+R_{{\bs{\ell}},k})$, $R_{{\bs{\ell}},k}$ follows Gamma distribution as $R_{{\bs{\ell}},k} \sim \mathcal {G}(m+\ell_k,{\Omega _k})$ and $\Omega _k = \frac{{P_k}{\sigma _k}^2\left( {1 - {\lambda _k}^2} \right)}{m\mathcal N_0}$. Clearly, the random variable $R_{{\bs{\ell}},k}$ can be factorized as $R_{{\bs{\ell}},k} = \frac{P_k\left| {{h_{{{\mathcal A}_{\bs{\ell}}},k}}} \right|^2}{\mathcal N_0}$ where $\left| {{h_{{{\mathcal A}_{\bs{\ell}}},k}}} \right|$ follows Nakagami distribution as
\begin{equation}\label{eqn:channel_coeff_dec}
\left| {{h_{{{\mathcal A}_{\bs{\ell}}},k}}} \right| \sim {\rm Nakagami}\left( {m + {\ell_k},\frac{\left( {m + {\ell_k}} \right)(1-{\lambda_k}^2){\sigma_k}^2}{m}} \right),\, 1 \le k \le K.
\end{equation}
Therefore, ${F_{{{ {\cal A}}_{\bs{\ell}}}}}\left( 2^{\mathcal R} \right)$ can be regarded as the outage probability of HARQ-IR after $K$ transmissions over independent Nakagami-$m$ fading channels, where $R_{{\bs{\ell}},k}$ denotes the received SNR in the $k$th HARQ transmission and ${{h_{{{\mathcal A}_{\bs{\ell}}},k}}}$ denotes the Nakagami-$m$ fading channel coefficient in the $k$th transmission. Consequently, the outage probability ${p_{out,K}}$ can be rephrased as a weighted sum of outage probabilities of HARQ-IR over independent Nakagami fading channels where the weightings ${W_{\bs{\ell}}}$ are probabilities of the negative multinomial distribution ${\rm NM}(m,\bf w)$.
%\end{remark}

%\begin{remark}\label{fastfading}
It is worth mentioning that the outage result in (\ref{eqn:out_prob_def_hat}) is applicable to fast fading channels. Under fast fading channels, the time correlation coefficients are equal to zero,  i.e., $\bs \lambda =\bf 0$. Putting it into (\ref{eqn:W_l_def_sec}), we have $W_{\bf{0}} = 1$ and $W_{\bs{\ell}} = 0$ for all ${\bs{\ell}} \ne {\bf 0}$. Then the outage probability ${p_{out,K}}$ reduces to ${F_{{{\mathcal A}_{\bf 0}}}}({2^R})$.
%\end{remark}

\subsection{Truncation Approximation}
Considering that $p_{out,K}$ is represented by an infinite series in (\ref{eqn:out_prob_def_hat}), it is impossible to compute the exact value by adding an infinite number of terms up. To enable its computation, it is natural to truncate $p_{out,K}$ into a finite series. Towards this end, an effective truncation approach is proposed. Specifically, $p_{out,K}$ in (\ref{eqn:out_prob_def_hat}) is approximated with a truncation order $N$ as
\begin{equation}\label{eqn:F_G_x_trunc_ser}
 p_{out,K} \approx {\sum\nolimits_{{\sum\nolimits_{k=1}^K  \ell_k \le N}} {{W_{\bs{\ell}}}{F_{{ {\cal A}_{\bs{\ell}}}}}\left( 2^{\cal R} \right)} }  \triangleq \tilde p_{out,K}^N.
\end{equation}
%%Noticing the importance of the truncation order to governing the approximation accuracy, the convergence analysis with regard to $\tilde p_{out,K}^N$ is carried out next.
%
%%\subsection{Convergence Analysis of $\tilde p_{out,K}^N$}\label{sec:conv_ana}
%
The truncation error $\nabla(N)$ is characterized by the difference between $p_{out,K}$ and $\tilde p_{out,K}^N$, such that
 %The difference between $p_{out,K}$ and $\tilde p_{out,K}^N$, i.e., the truncation error $\nabla(N)$, can be expressed as
\begin{equation}\label{eqn:trun_err_prod_gam}
 \nabla(N)= p_{out,K} - \tilde p_{out,K}^N = {\sum\nolimits_{{\sum\nolimits_{k=1}^K  \ell_k \ge N+1}}  {{W_{\bs{\ell}}}{F_{{ {\cal A}_{\bs{\ell}}}}}\left( 2^{\cal R} \right)} }  \ge 0.
\end{equation}
Clearly from (\ref{eqn:trun_err_prod_gam}), $\nabla(N)$ is a monotonically decreasing function of $N$ and satisfies $\lim_{N \rightarrow \infty} \nabla(N) =0$. Moreover, since ${{F_{{ {\cal A}_{\bs{\ell}}}}}\left( 2^{\cal R} \right)} \le 1$ and ${{W_{\bs{\ell}}}} \le 1$, the truncation error $\nabla(N)$ is in fact uniformly upper bounded by ${\sum\nolimits_{{\sum\nolimits_{k=1}^K  \ell_k \ge N+1}}  {{W_{\bs{\ell}}}} }$ which is irrespective of the rate $\cal R$ and converges to zero when $N$ increases. Thus strictly speaking, the truncation approximation in (\ref{eqn:F_G_x_trunc_ser}) actually admits a uniform convergence\cite[p147]{rudin1964principles}\footnote{We say that $f_n(x)$ converges uniformly to $f(x)$ on its domain $\cal D$ if, given $\varepsilon $, there exists an integer $N(\varepsilon )$ independent of $x$, such that $|f_n(x)-f(x)|\le \varepsilon $ for all $x \in \cal D$ whenever $n \ge N(\varepsilon )$\cite[p147]{rudin1964principles}.}. This uniform convergence guarantees the truncation approximation of outage probability with high accuracy for any $\cal R$. It is different from the convergence in MSE in \cite{shi2015analysis,shi2016inverse} which does not necessarily imply that the approximation error approaches to zero for arbitrary $\cal R$ \cite[p86]{adams2013continuous}. Therefore our truncation approximation is expected to perform better than those in \cite{shi2015analysis,shi2016inverse}, which will be further demonstrated in Section \ref{sec:num_res}.

As proved in Appendix \ref{app:uperb_err}, we further notice that the truncation error $\nabla(N)$ is bounded as % given $\varepsilon > 0$ and $\cal R$, such that $\nabla \le \varepsilon$
\begin{align}\label{eqn:trun_err_fina_hyp}
\nabla(N)  %&< {{F_{{ {\cal A}_{\bs{\ell}}},N}^{\rm max}}\left( 2^{\cal R} \right)} {W_{\bf{0}}}\sum\limits_{n = 0}^\infty  {\frac{{{{\left( m \right)}_{n + N + 1}}}}{{\left( {n + N + 1} \right)!}}{{\left( {\sum\limits_{k = 1}^K {{w_k}} } \right)}^{n + N + 1}}} \notag \\
 %&= {{F_{{ {\cal A}_{\bs{\ell}}},N}^{\rm max}}\left( 2^{\cal R} \right)} {W_{\bf{0}}}{\left( {\sum\limits_{k = 1}^K {{w_k}} } \right)^{N + 1}}\frac{{{{\left( m \right)}_{N + 1}}}}{{\left( {N + 1} \right)!}}\sum\limits_{n = 0}^\infty  {\frac{{{{\left( {m + N + 1} \right)}_n}{{\left( 1 \right)}_n}}}{{{{\left( {N + 2} \right)}_n}}}\frac{{{{\left( {\sum\limits_{k = 1}^K {{w_k}} } \right)}^n}}}{{n!}}}. %\notag \\
 &\le  {W_{\bf{0}}}{{F_{{ {\cal A}_{\bs{\ell}}},N}^{\rm max}}\left( 2^{\cal R} \right)} \xi \left( N \right) \triangleq \mathcal B_u, %\underbrace {{W_{\bf{0}}}{{\left( {\sum\limits_{k = 1}^K {{w_k}} } \right)}^{N + 1}}\frac{{{{\left( m \right)}_{N + 1}}}}{{\left( {N + 1} \right)!}}{}_2{F_1}\left( {m + N + 1,1;N + 2;\sum\limits_{k = 1}^K {{w_k}} } \right)}_{\xi \left( N \right)}.
% \label{eqn:trun_err_fina_hyp1}
% &\le {W_{\bf{0}}}{\left( {\sum\limits_{k = 1}^K {{w_k}} } \right)^{N + 1}}\frac{{{{\left( m \right)}_{N + 1}}}}{{\left( {N + 1} \right)!}}{}_2{F_1}\left( {m + N + 1,1;N + 2;\sum\limits_{k = 1}^K {{w_k}} } \right) \le 1.
\end{align}
where ${{F_{{ {\cal A}_{\bs{\ell}}},N}^{\rm max}}\left( 2^{\cal R} \right)} = \mathop {\max }\nolimits_{\sum\nolimits_{k=1}^K  \ell_k = N + 1} \left( {{F_{{ {\cal A}_{\bs{\ell}}}}}\left( 2^{\cal R} \right)} \right)$, ${\xi \left( N \right)}$ is a decreasing function of $N$ as $\xi \left( N \right) ={\left( {\sum\nolimits_{k = 1}^K {{w_k}} } \right)^{N + 1}}\frac{{{{\left( m \right)}_{N + 1}}}}{{\left( {N + 1} \right)!}} {}_2{F_1}\left( {m + N + 1,1;N + 2;\sum\nolimits_{k = 1}^K {{w_k}} } \right)$, the notation $(\cdot)_n$ stands for Pochhammer symbol, and ${}_2F_1(a,b;c;z)=\sum\nolimits_{s = 0}^\infty  {\frac{{{{\left( a \right)}_s}{{\left( b \right)}_s}}}{{\Gamma \left( {c + s} \right)s!}}{z^s}} $ denotes hypergeometric function \cite[Eq.15.1.1]{olver2010nist}. For Rayleigh fading channels, i.e., $m=1$, the term $\xi \left( N \right)$ reduces to $\xi \left( N \right) = {W_{\bf{0}}}^{ - 1}{\left( {\sum\nolimits_{k = 1}^K {{w_k}} } \right)^{N + 1}}$ by using ${}_2{F_1}\left( {N + 2,1;N + 2;\sum\nolimits_{k = 1}^K {{w_k}} } \right) = {W_{\bf{0}}}^{ - 1}$ in \cite[Eq.1.40]{mathai2009h}. Together with ${{F_{{ {\cal A}_{\bs{\ell}}},N}^{\rm max}}\left( 2^{\cal R} \right)} \le 1$, the truncation error under Rayleigh fading channels is thus further bounded as $\nabla(N) \le \mathcal B_u \le {\left( {\sum\nolimits_{k = 1}^K {{w_k}} } \right)^{N + 1}}$. Since ${\sum\nolimits_{k = 1}^K {{w_k}} } < 1$, the truncation error under Rayleigh fading channels exponentially decays with $N$. It implies that the convergence speed of our truncation approach is fast and also justifies the effectiveness of our truncation approach.

In practice, the truncation order plays an important role in improving the approximation accuracy and an efficient selection of the truncation order is necessary. Specifically, for a given accuracy requirement $\varepsilon $, the minimal truncation order $N(\varepsilon)$ can be determined by setting the upper bound $\mathcal B_u$ in (\ref{eqn:trun_err_fina_hyp}) less than the maximum tolerable approximation error $\varepsilon $, such that $N(\varepsilon) = \mathop {\min } \left\{\left. N \ge 0 \right| \mathcal B_u \le \varepsilon   \right\}$.

The impact of truncation order on the approximation accuracy is then examined and the results are shown in Fig. \ref{fig:N_K}. Clearly, the truncated outage probability $\tilde p_{out,K}^N$ converges quite fast to the exact outage probability $p_{out,K}$ and the truncation order of $N=3$ is sufficient for a very good approximation of the outage probability. Plugging $N=3$ together with ${\cal R}=1$bps/Hz into (\ref{eqn:trun_err_fina_hyp}), the upper bounds $\mathcal B_u$ of four cases from I to IV can be calculated as  $7.0*10^{-5}$, $7.4*10^{-6}$, $1.4*10^{-4}$ and $3.7*10^{-8}$, respectively. They are very small and thus verify the accuracy of our truncation approach. By comparing the upper bound $\mathcal B_u$  of Case I with that of Case II, it can be observed that the increase of $K$ decreases $\mathcal B_u$ from $7.0*10^{-5}$ to $7.4*10^{-6}$, which means that the increase of $K$ would be beneficial to the reduction of the truncation error and thus improve the accuracy of our truncation approach. This can be roughly explained as follows. From the upper bound $\mathcal B_u$ in (\ref{eqn:trun_err_fina_hyp}), we can see that increasing $K$ has twofold impact on the upper bound $\mathcal B_u$. Specifically, the increase of $K$ reduces the first component ${W_{\bf{0}}}{{F_{{ {\cal A}_{\bs{\ell}}},N}^{\rm max}}\left( 2^{\cal R} \right)}$ of $\mathcal B_u$ on one hand, while increases the second component $\xi \left( N \right)$ on the other hand. Since the first component dominates particularly in a low-to-medium outage region which is the most concerned region for practical applications, the upper bound $\mathcal B_u$ can be viewed as a decreasing function of $K$.

%The influence of the maximal number of transmissions $K$ on the tightness of our truncation approach is difficult to be found directly from the truncation error $\nabla(N)$ in (\ref{eqn:trun_err_prod_gam}). However, it can be investigated from the analysis of the upper bound $\mathcal B_u$ in (\ref{eqn:trun_err_fina_hyp}). It can be found that increasing $K$ has twofold impact on the upper bound $\mathcal B_u$. Specifically, the increase of $K$ reduces the first component ${W_{\bf{0}}}{{F_{{ {\cal A}_{\bs{\ell}}},N}^{\rm max}}\left( 2^{\cal R} \right)}$ of $\mathcal B_u$ on one hand, while increases the second component $\xi \left( N \right)$ on the other hand. Since the first component dominates particularly in a low-to-medium outage region which is the most concerned region for practical applications, the upper bound $\mathcal B_u$ can be viewed as a decreasing function of $K$. It means that the increase of $K$ would be beneficial to the reduction of the truncation error and thus improve the accuracy of our truncation approach.

%For fixed values of $\varepsilon > 0$ and $\cal R$, the minimal truncation order $N(\varepsilon)$ can be determined by setting the upper bound $\mathcal B_u$ from (\ref{eqn:trun_err_fina_hyp}) less than the maximum tolerable approximation error, such that $N(\varepsilon) = \mathop {\min } \left\{\left. N \ge 0 \right| \mathcal B_u \le \varepsilon   \right\}$. Herein, it is worthwhile to note that the influence of the maximal number of transmissions $K$ on the selection of $N(\varepsilon)$ is very complicated due to the twofold impact on $\mathcal B_u$ from increasing $K$. Specifically, the increase of $K$ reduces the first component ${W_{\bf{0}}}{{F_{{ {\cal A}_{\bs{\ell}}},N}^{\rm max}}\left( 2^{\cal R} \right)}$ of $\mathcal B_u$ on one hand, while increases the second component $\xi \left( N \right)$ on the other hand. Nevertheless, the first component dominates particularly in a low-to-medium outage. In this sense, the increase of $K$ is beneficial to the reduction of $N(\varepsilon)$.

% due to the physical interpretation of ${{F_{{ {\cal A}_{\bs{\ell}}}}}\left( 2^{\cal R} \right)}$


%coming from the upper bound

 %(\ref{eqn:comp_comp}) does not imply that $\chi_{\varepsilon}$ increases with . (\ref{eqn:trun_err_fina_hyp})

%From (\ref{eqn:trun_err_fina_hyp}),  that the increase of the maximal number of transmissions $K$ may not lead to the increase of the truncation error $\varepsilon$


%Interestingly from (\ref{eqn:trun_err_fina_hyp}), insightful results can be extracted, that is, the truncation order is essentially determined by the relative accuracy $\upsilon = {\frac{\varepsilon}{p_{out,K}}}$ rather than the absolute accuracy $\varepsilon$. This is due to the fact that ${F_{{ {\cal A}_{\bs{\ell}}},N}^{\rm max}}\left( 2^{\cal R} \right) \ll {{F_{{ {\cal A}_{\bf{0}}}} \left( 2^{\cal R} \right)}}$ and $W_{\bf 0}{{F_{{ {\cal A}_{\bf{0}}}} \left( 2^{\cal R} \right)}} \le p_{out,K}$, which eventually yields $\varepsilon\ll p_{out,K} \xi \left( N \right)$ from (\ref{eqn:trun_err_fina_hyp}). Herein, ${F_{{ {\cal A}_{\bs{\ell}}},N}^{\rm max}}\left( 2^{\cal R} \right) \ll {{F_{{ {\cal A}_{\bf{0}}}} \left( 2^{\cal R} \right)}}$ holds because of the physical interpretation of ${{F_{{ {\cal A}_{\bs{\ell}}}} \left( 2^{\cal R} \right)}}$ in Section \ref{sec:out_phy}, which indicates that the fading order in each transmission associated with ${{F_{{ {\cal A}_{\bs{\ell}}}} \left( 2^{\cal R} \right)}}$ is no less than the fading order associated with ${{F_{{ {\cal A}_{\bf{0}}}} \left( 2^{\cal R} \right)}}$, i.e., $m$, when $\bs \ell \ne \bf 0$, and the increase of fading order would dramatically reduce the outage probability of HARQ-IR. In addition, $W_{\bf 0}{{F_{{ {\cal A}_{\bf{0}}}} \left( 2^{\cal R} \right)}} \le p_{out,K}$ follows from (\ref{eqn:out_prob_def_hat}). Hereby, noticing the decreasing monotonicity of $\xi(N)$, we have $N \ll \left\lceil {{\xi ^{ - 1}}\left( {\upsilon} \right)} \right\rceil$, where $\left\lceil \cdot \right\rceil$ represents the ceiling function and $\xi ^{ - 1}$ refers to the inverse function of $\xi(N)$.

%As shown in Fig. \ref{fig:N_K}, the impact of truncation order is examined to verify the foregoing convergence analysis. Clearly from Fig. \ref{fig:N_K}, the truncation order $N=3$ is enough to very well approximate the outage probability. Plugging $N=3$ together with ${\cal R}=1$bps/Hz into (\ref{eqn:trun_err_fina_hyp}), the numerical analysis shows that the upper bounds $\mathcal B_u$ of four cases from I to IV are $7.0*10^{-5}$, $7.4*10^{-6}$, $1.4*10^{-4}$ and $3.7*10^{-8}$, respectively, which verify the computational accuracy of the truncation approach. By comparing the upper bound $\mathcal B_u$  of Case I with that of Case II, it can be observed that the increase of $K$ decreases $\mathcal B_u$ from $7.0*10^{-5}$ to $7.4*10^{-6}$, which justifies the benefit of $K$ to the reduction of $N(\varepsilon)$. Besides, it can be seen in Fig. \ref{fig:N_K} that $\tilde p_{out,K}^N$ converges quite fast to the exact value $p_{out,K}$. Such a rapid convergence speed can be illustrated by taking Rayleigh fading channels, i.e., $m=1$, as an example. Thereupon $\xi \left( N \right)$ reduces to $\xi \left( N \right) = {W_{\bf{0}}}^{ - 1}{\left( {\sum\nolimits_{k = 1}^K {{w_k}} } \right)^{N + 1}}$ by using \cite[Eq.1.40]{mathai2009h}, i.e., ${}_2{F_1}\left( {N + 2,1;N + 2;\sum\nolimits_{k = 1}^K {{w_k}} } \right) = {W_{\bf{0}}}^{ - 1}$. From (\ref{eqn:trun_err_fina_hyp}) together with ${{F_{{ {\cal A}_{\bs{\ell}}},N}^{\rm max}}\left( 2^{\cal R} \right)} \le 1$, the truncation error is thus bounded as $\nabla \le  {\left( {\sum\nolimits_{k = 1}^K {{w_k}} } \right)^{N + 1}}$. Since ${\sum\nolimits_{k = 1}^K {{w_k}} } < 1$, the truncation error exponentially decays with $N$, which justifies the effectiveness of the truncation approach.
\begin{figure}
  \centering
  % Requires \usepackage{graphicx}
  \includegraphics[width=2.5in]{./conver_ana2.eps}\\
  \caption{The impact of truncation order by setting $\lambda_1=\cdots=\lambda_K=0.8$, $\frac{{\sigma_1}^2}{\mathcal N_0}=\cdots=\frac{{\sigma_K}^2}{\mathcal N_0}=1$, $P_1=\cdots=P_K=P$ and $\mathcal R=1$bps/Hz.}\label{fig:N_K}%, $\upsilon_1=0.5$ and $\upsilon_2=0.1$
\end{figure}
%\begin{equation}\label{eqn:N_var_R}
%N(\varepsilon) = \left\lceil {{\xi ^{ - 1}}\left( {\frac{\varepsilon}{{{F_{{ {\cal A}_{\bs{\ell}}},N}^{\rm max}}\left( 2^{\cal R} \right)}}} \right)} \right\rceil,
%\end{equation}
%
 %The computation complexity $\chi_\varepsilon$ of the truncation approach can be characterized by the number of summation terms in (\ref{eqn:F_G_x_trunc_ser}), namely  $\chi_\varepsilon = \frac{K^{N(\varepsilon)}-1}{K-1}$.

 %In contrast with using the simulation approach to obtain the outage probability, the computational complexity generally depends on the absolute accuracy $\varepsilon$.

%It thus justifies the benefit of using this truncation approach to obtain the outage probability, which is clearly better than using the simulation approach.  %It thus justifies that the usage of this truncation approximation would perform better than the simulation approach.
%\subsection{Computational Complexity}
%Since the computation of $\tilde p_{out,K}^N$ involves the evaluation of the generalized Fox's H function, whose time complexity is usually fixed and extremely low due to the single-fold integral. Thus (\ref{eqn:F_G_x_trunc_ser}) implies that the computational complexity of evaluating $\tilde p_{out,K}^N$ grows linearly proportional to the number of summation terms. Given the truncation order ${N(\varepsilon)}$, the number of summation terms is $\sum\nolimits_{{n_1} = 0}^{N(\varepsilon )} {\sum\nolimits_{{n_2} = 0}^{{n_{1}}} { \cdots \sum\nolimits_{{n_K} = 0}^{{n_{K - 1}}} {C_{{n_{K }}}^1} }} = C_{N(\varepsilon) + K}^K$, where $C_n^k$ stands for binomial coefficient. The computational complexity can eventually be expressed as $\mathcal O\left(C_{N(\varepsilon) + K}^K\right)$, where $\mathcal O(\cdot)$ denotes the big-O notation.
%
%In contrast with approximating $p_{out,K}$ by conducting Monte Carlo simulation, the computational complexity solely depends on the absolute accuracy $\varepsilon$, and it typically consists of at least ${\varepsilon}^{-1}$ simulation runs that leads to a computational complexity of $\mathcal O \left({\varepsilon}^{-1}\right)$. In general, the truncation approach outperforms the simulation approach. Taking Case II in Fig. \ref{fig:N_K} as an example, if $\varepsilon=10^{-3}$, the computational complexity of the truncation approach is $\mathcal O\left(C_{N(\varepsilon) + K}^K\right)=\mathcal O\left(4\right)$ with $N(\varepsilon)=1$, while the computational complexity of the simulation approach is $\mathcal O\left(10^3\right)$. If $\varepsilon$ decreases to $10^{-5}$, the computational complexities of the truncation approach and the simulation approach become $\mathcal O\left(C_{N(\varepsilon) + K}^K\right)=\mathcal O\left(35\right)$ with $N(\varepsilon)=3$ and $\mathcal O\left(10^5\right)$, respectively. The computational complexities of these two approaches differ substantially in order of magnitude, which obviously demonstrates the effectiveness of the truncation approach.

%\begin{equation}\label{eqn:comp_comp}
%\chi_{\varepsilon} \propto \frac{K^{N(\varepsilon)+1}-1}{K-1},
%\end{equation}
%As verified in Section \ref{sec:conv_ana}, $\varepsilon$ dramatically decreases with $N$, and the truncation order $N(\varepsilon)=3$ is sufficient to achieve a very good approximation. The computational complexity of this evaluation can therefore be viewed as a constant.



% In contrast with approximating $p_{out,K}$ by conducting Monte Carlo simulation, its computational complexity solely depends on $\varepsilon$. More specifically, it requires that the Monte Carlo simulation should consist of at least $\frac{1}{\varepsilon}$ runs.



%Clearly, the computational complexity of adopting simulation dramatically increases with the requirement of the absolute computational accuracy. For a small $\varepsilon$, the computational complexity of the simulation method is definitely much higher than that of using (R\ref{eqn:pout_trun}). Meanwhile, another benefit of using  (R\ref{eqn:pout_trun}) is that its computational complexity almost does not change with the absolute computational accuracy $\varepsilon$, which underlines its practical significance.






%where $\upsilon$ denotes the average computational complexity of the associated operations on each term, including integration, addition, multiplication, etc.


%With respect to the computational complexity of evaluating $\tilde p_{out,K}^N$






%\begin{equation}\label{eqn:trun_err_m_1}
%\xi \left( N \right) %= {W_{\bf{0}}}{\left( {\sum\limits_{k = 1}^K {{w_k}} } \right)^{N + 1}}{}_2{F_1}\left( {m + N + 1,1;N + 2;\sum\limits_{k = 1}^K {{w_k}} } \right)
%= {\left( {\sum\limits_{k = 1}^K {{w_k}} } \right)^{N + 1}},
%\end{equation}

%\begin{equation}\label{eqn:hyper_fun_2_1_red}
%{}_2{F_1}\left( {N + 2,1;N + 2;\sum\limits_{k = 1}^K {{w_k}} } \right) = {\left( {1 - \sum\limits_{k = 1}^K {{w_k}} } \right)^{ - 1}} = {W_{\bf{0}}}^{ - 1}.
%\end{equation}



%However, due to the complicated form of the function $\xi$, it is still intractable to examine the convergence rate with respect to $N$.


% to find a simple expression to . However, when fading channels experience Rayleigh fading, i.e., $m=1$, the following remark proves that the upper bound of $\varepsilon$ is an exponential function of $N$.

%The following proposition is proposed to simplify the selection of the truncation order $N$.
%\begin{proposition}
%For any truncation order $N$, noticing that ${{F_{{ {\cal A}_{\bs{\ell}}}}}\left( 2^{\cal R} \right)} \le 1$ together with (\ref{eqn:trun_err_fina_hyp}), the truncation error $\varepsilon$ is upper bounded as
%\begin{align}\label{eqn:trun_err_fina_hyp1}
%\varepsilon &< {W_{\bf{0}}}{\left( {\sum\limits_{k = 1}^K {{w_k}} } \right)^{N + 1}}\frac{{{{\left( m \right)}_{N + 1}}}}{{\left( {N + 1} \right)!}}{}_2F_1\left( {m + N + 1,1;N + 2;\sum\limits_{k = 1}^K {{w_k}} } \right).
%\end{align}
%%Clearly, the upper bound of $\varepsilon$ is less than $1$ because of $\sum\limits_{{\bf l} \in {{\mathbb N}_0}^K}  {{W_{\bs{\ell}}}}  = 1$.
%\end{proposition}





%Given a certain threshold of truncation error $\varepsilon$, truncation order $N$ can be properly chosen by using (\ref{eqn:trun_err_fina_hyp1}). Due to the complicated form of hypergeometric function, it is not easy to find the decreasing trend of the truncation error $\varepsilon$ with the increase of the truncation order $N$ for arbitrary $m$. However, when fading channels experience Rayleigh fading, i.e., $m=1$, the following remark proves that the upper bound of $\varepsilon$ is an exponential function of $N$.


%In addition, from (\ref{eqn:trun_err_fina_hyp}), it will be proved in asymptotic analysis that the truncation error $\varepsilon$ significantly decreases as the SNRs $\{\gamma_k\}_{k=1}^K$ increase. According to our numerical results, it is generally enough to achieve a good approximation by choosing $N=3$.

%This remark, of independent interest, is very useful in the following asymptotic outage analysis. Moreover, it is easily found that the outage probability ${p_{out,K}}$ under fast fading channels reduces to ${F_{{{\mathcal A}_{\bf 0}}}}({2^R})$ , i.e., $\bs \lambda =\bf 0$, because $W_{\bs{\ell}} = 0$ for all ${\bs{\ell}} \ne {\bf 0}$ and $W_{\bf{0}} = 1$. Therefore, the following asymptotic outage analysis of HARQ-IR can cover fast fading channels as a special case, whose asymptotic outage analysis has never been discussed in prior literature. Moreover, it should be mentioned that the preceding analytical results can not be applied to quasi-static fading channels because of $\bs \lambda \ne \bf 1$ in (\ref{eqn:cond_pdf_r_k_rew}).





\section{Asymptotic Outage Analysis}\label{sec:asy}
\subsection{Asymptotic Outage Probability}
Although the outage probability of HARQ-IR over time-correlated Nakagami fading channels can be exactly derived in closed-form as shown in (\ref{eqn:out_prob_def_hat}) and can be approximated as (\ref{eqn:F_G_x_trunc_ser}), they are still complex and hinder the extraction of meaningful insights. Fortunately, after analyzing the CDF ${F_{{{ {\cal A}}_{\bs{\ell}}}}}\left( x \right)$, we can find some special properties of ${F_{{{ {\cal A}}_{\bs{\ell}}}}}\left( x \right)$, which enable us to further conduct asymptotic outage analysis so that the expression of outage probability can be simplified with clear insights. To proceed with the analysis, we rewrite the transmission SNRs as
\begin{equation}\label{eqn:gamma_snr_rel}
\left( {\frac{{{P_1}}}{{{{\cal N}_0}}},\frac{{{P_2}}}{{{{\cal N}_0}}}, \cdots ,\frac{{{P_K}}}{{{{\cal N}_0}}}} \right) = \frac{{{P_{total}}}}{{{{\cal N}_0}}}\left( {\frac{{{P_1}}}{{{P_{total}}}},\frac{{{P_2}}}{{{P_{total}}}}, \cdots ,\frac{{{P_K}}}{{{P_{total}}}}} \right) = \gamma \bs \theta,
\end{equation}
where ${P_{total}} = {P_1} +  \cdots  + {P_K}$, $\gamma \triangleq \frac{{{P_{total}}}}{{{{\cal N}_0}}}$ denotes the total transmission SNR and $\bs \theta = ({\theta_1},\cdots,{\theta_K})\triangleq\left( {\frac{{{P_1}}}{{{P_{total}}}}, \cdots ,\frac{{{P_K}}}{{{P_{total}}}}} \right)$ represents the power allocation vector. With this definition of the transmission SNR, the CDF ${F_{{{ {\cal A}}_{\bs{\ell}}}}}\left( x \right)$ in (\ref{eqn:CDF_F_A_def_sec}) can be rewritten as shown in the following theorem.
\begin{theorem} \label{lemma:lemma_gener_fox_H}
The CDF ${F_{{{ {\mathcal A}}_{\bs{\ell}}}}}(x)$ can be rewritten as
\begin{multline}\label{eqn:der_A_1_CDF}
%{F_{{{\cal A}_{\bs{\ell}}}}}(x) = {\gamma ^{ - mK - \sum\limits_{k = 1}^K {{\ell_k}} }}\prod\limits_{k = 1}^K {\frac{{{{\left( {\frac{m}{{{\theta _k}{\sigma _k}^2\left( {1 - {\lambda _k}^2} \right)}}} \right)}^{m + {\ell_k}}}}}{{\Gamma \left( {m + {\ell_k}} \right)}}} \sum\limits_{{n_1}, \cdots ,{n_K} = 0}^\infty  {\prod\limits_{k = 1}^K {\frac{{{{\left( { - \frac{m}{{\gamma {\theta _k}{\sigma _k}^2\left( {1 - {\lambda _k}^2} \right)}}} \right)}^{{n_k}}}}}{{{n_k}!}}} } {g_{{\bf{n}} + {\bs{\ell}}}}\left( x \right),
{F_{{{\cal A}_{\bs{\ell}}}}}(x) = {\gamma ^{-d_{{ {\cal A}_{\bs{\ell}}}}}}\prod\nolimits_{k = 1}^K {\frac{1}{{\Gamma \left( {m + {\ell_k}} \right)}}{{\left( {\frac{m}{{{\theta _k}{\sigma _k}^2\left( {1 - {\lambda _k}^2} \right)}}} \right)}^{m + {\ell_k}}}} \\
\times \sum\nolimits_{{n_1}, \cdots ,{n_K} = 0}^\infty  {\prod\nolimits_{k = 1}^K {\frac{1}{{{n_k}!}}{{\left( { - \frac{m}{{\gamma {\theta _k}{\sigma _k}^2\left( {1 - {\lambda _k}^2} \right)}}} \right)}^{{n_k}}}} } {g_{{\bf{n}} + {\bs{\ell}}}}\left( x \right),
\end{multline}
where $d_{{ {\cal A}_{\bs{\ell}}}} = mK + \sum\nolimits_{k = 1}^K { { {\ell_k}} }$, ${\bf n} = (n_1,\cdots,n_K)$, ${{\bs{\ell}}} = (\ell_1,\cdots,\ell_K)$ and ${g_{{\bs{\ell}}}}\left( x \right)$ is given by
\begin{align}\label{eqn:sec_def_g_fun}
%{g_{{\bf{n}},{\bs{\ell}}}}\left( x \right) = \prod\limits_{k = 1}^K {\frac{1}{{\Gamma \left( {m + {\ell_k}} \right){n_k}!}}} \int\limits_0^{x - 1} { \cdots \int\limits_0^{x\prod\limits_{k = 1}^{K - 2} {{{\left( {1 + {t_k}} \right)}^{ - 1}}}  - 1} {\int\limits_0^{x\prod\limits_{k = 1}^{K - 1} {{{\left( {1 + {t_k}} \right)}^{ - 1}}}  - 1} {\prod\limits_{k = 1}^K {{t_k}^{m + {\ell_k} + {n_k} - 1}} d{t_1} \cdots d{t_{K - 1}}d{t_K}} } }.
{g_{\bs{\ell}}}\left( x \right) &= \int\nolimits_{\prod\nolimits_{k = 1}^K {\left( {1 + {t_k}} \right)}  \le x} {\prod\limits_{k = 1}^K {{t_k}^{m + {\ell_k} - 1}} d{t_1} \cdots d{t_{K - 1}}d{t_K}}\\
\label{eqn:g_0_0_der_meij_rem}
&= \prod\limits_{k = 1}^K {\Gamma \left( {m + {\ell_k}} \right)} G_{K + 1,K + 1}^{0,K + 1}\left( {\left. {\begin{array}{*{20}{c}}
{1,1 + {\ell_1} + m, \cdots ,1 + {\ell_K} + m}\\
{1, \cdots ,1,0}
\end{array}} \right|x} \right),
\end{align}
%\begin{equation}\label{eqn:g_0_0_der_meij_rem}
%{g_{\bs{\ell}}}\left( x \right) = \prod\limits_{k = 1}^K {\Gamma \left( {m + {\ell_k}} \right)} G_{K + 1,K + 1}^{0,K + 1}\left( {\left. {\begin{array}{*{20}{c}}
%{1,1 + {\ell_1} + m, \cdots ,1 + {\ell_K} + m}\\
%{1, \cdots ,1,0}
%\end{array}} \right|x} \right),%\,
% %{{\rm min}\{\bs{\ell}\}}+m > 0.
%\end{equation}
with $G_{p,q}^{m,n}\left( {\cdot|x} \right)$ denoting Meijer-G function \cite[Eq.9.301]{gradshteyn1965table}. % and the proof of (\ref{eqn:g_0_0_der_meij_rem}) is given in Appendix \ref{app:proof_g_n_l}.
%Therefore, ${g_{{\bf{0}}}}\left( 2^{\mathcal R} \right)$ can be easily obtained with (\ref{eqn:g_0_0_der_meij_rem}). Specifically, when $m$ is an integer, it follows by applying Residue theorem and general Leibniz rule into (\ref{eqn:g_0_0_der_meij_rem}) that
%When $\bs \ell = \bf 0$ and $m$ is an integer, ${g_{{\bs{\ell}}}}\left( x \right)$ reduces to
%\begin{multline}\label{eqn:G_meij_factor_poly}
%{g_{{\bf{0}}}}\left( x \right)
%%&= \frac{1}{2\pi \rm i} \int\limits_{c_1 - {\rm{i}}\infty }^{c_1 + {\rm{i}}\infty } {\frac{{\Gamma \left( s \right)}}{{\Gamma \left( {s + 1} \right)}}\prod\limits_{k = 1}^K {\frac{{\Gamma \left( { - m + s} \right)}}{{\Gamma \left( s \right)}}} {x^s}ds}  \notag \\
%%= {\left( {\Gamma \left( m \right)} \right)^K}\frac{1}{{2\pi {\rm{i}}}}\int\limits_{{c_1} - {\rm{i}}\infty }^{{c_1} + {\rm{i}}\infty } {\frac{{{x^s}}}{{s\prod\limits_{t = 1}^m {{{\left( {s - t} \right)}^K}} }}ds}  = {\left( {\Gamma \left( m \right)} \right)^K}\sum\limits_{a = 0}^m {{\rm{Res}}\left\{ {\frac{{{x^s}}}{{s\prod\limits_{t = 1}^m {{{\left( {s - t} \right)}^K}} }},s = a} \right\}}  \\
%%& = {\left[ {\frac{{{x^s}}}{{\prod\limits_{t = 1}^m {{{\left( {s - t} \right)}^K}} }}} \right]_{s = 0}} + \sum\limits_{t = 1}^m {\frac{1}{{\left( {K - 1} \right)!}}\frac{{{\partial ^{K - 1}}}}{{\partial {s^{K - 1}}}}{{\left[ {\frac{{{x^s}}}{{s\prod\limits_{u = 1,u \ne t}^m {{{\left( {s - u} \right)}^K}} }}} \right]}_{s = t}}}. \notag\\
% =\frac{{{{\left( { - 1} \right)}^{mK}}}}{{{m^K}}} + \frac{{{{\left( {\Gamma \left( m \right)} \right)}^K}{{\left( { - 1} \right)}^{K - 1}}}}{{{{\left( {\left( {K - 1} \right)!} \right)}^{m - 1}}}}\sum\nolimits_{a = 1}^m {\sum\nolimits_{{n_1} +  \cdots  + {n_{m + 1}} = K - 1} {\frac{{{{\left( { - 1} \right)}^{{n_m}}}{{\left( {\ln x} \right)}^{{n_m}}}{x^a}}}{{{n_1}! \cdots {n_m}!{a^{1 + {n_{m + 1}}}}}}} }   \\
%\times \prod\nolimits_{t = 1}^{a - 1} {\frac{{\Gamma \left( {K + {n_t}} \right)}}{{{{\left( {a - t} \right)}^{1 + K + {n_t}}}}}} \prod\nolimits_{t = a + 1}^m {\frac{{\Gamma \left( {K + {n_{t - 1}}} \right)}}{{{{\left( {a - t} \right)}^{1 + K + {n_{t - 1}}}}}}},\, m=1,2,\cdots
%% and general Leibniz rule
%\end{multline}
%where ${\rm Res}(f,a_k)$ denotes the residue of $f$ at $a_k$ and $c_1 > m$.
\end{theorem}
\begin{proof}
Please see Appendix \ref{app:lemma_gener_fox_H}.
\end{proof}








%Herein, it is worth noting that (\ref{eqn:sec_def_g_fun}) is rewritten as  (\ref{eqn:g_0_0_der_meij_rem}) in terms of Meijer-G function because of the low computational complexity of evaluating ${g_{{\bs{l}}}}\left( x \right)$ and the simplicity of handling its special properties. %the representation of ${g_{{\bs{l}}}}\left( x \right)$ given in (\ref{eqn:g_0_0_der_meij_rem}) can ease its computation due to the difficulty of tackling multiple-fold integral of (\ref{eqn:sec_def_g_fun}).
%%However, it is very difficult to evaluate ${g_{{\bf{0}}}}\left( x \right)$ using (\ref{eqn:sec_def_g_fun}) due to its high computational complexity of tackling multiple-fold integral.
%%Moreover, it is also very difficult to disclose the effect of $\mathcal R$ on  ${g_{{\bf{0}}}}\left( 2^{\mathcal R} \right)$ from (\ref{eqn:sec_def_g_fun}). To ease the computation of ${g_{{\bf{0}}}}\left( 2^{\mathcal R} \right)$ and enable later discussion, an alternative simple representation of ${g_{{\bs{\ell}}}(x)}$ is written as
%%\begin{theorem}\label{the:g_n_l_rep_ant}
%%As proved in Appendix \ref{app:proof_g_n_l}, ${g_{{\bf{0}}}}\left( x \right)$ can be rewritten as
%Moreover, through combining (\ref{eqn:sec_def_g_fun}) with (\ref{eqn:g_0_0_der_meij_rem}), the same analytical method can be applied into deriving the asymptotic outage probability of HARQ-IR over a variety of independent fading channels, including Nakagami-q fading, Racian fading, Weibull fading, double Rayleigh fading, etc. \cite{simon2005digital,chelli2014performance}.

From Theorem \ref{lemma:lemma_gener_fox_H}, when the transmission SNR is high, i.e., $\gamma \to \infty$, the CDF ${F_{{{ {\mathcal A}}_{\bs{\ell}}}}}(x)$ can be further expressed as
\begin{equation}\label{eqn:der_A_1_asyCDF}
{F_{{{\cal A}_{\bs{\ell}}}}}(x) = {\gamma ^{ - {d_{{\mathcal A_{\bs{\ell}}}}}}}\prod\limits_{k = 1}^K {\frac{1}{{\Gamma \left( {m + {\ell_k}} \right)}}{{\left( {\frac{m}{{{\theta _k}{\sigma _k}^2\left( {1 - {\lambda _k}^2} \right)}}} \right)}^{m + {\ell_k}}}} {g_{\bs{\ell}}}\left( x \right) + o\left( {{\gamma ^{ - {d_{{\mathcal A_{\bs{\ell}}}}}}}} \right),
\end{equation}
where $o(\cdot)$ refers to the little-O notation, and $f(\gamma) \in o(\phi(\gamma))$ provided that $\lim \limits_{\gamma \to \infty}f(\gamma)/\phi(\gamma) = 0$. Based on (\ref{eqn:der_A_1_asyCDF}), the following property of the CDF ${F_{{{ {\mathcal A}}_{\bs{\ell}}}}}(x)$ holds since $d_{{ {\cal A}_{\bs{\ell}}}} = mK + \sum\nolimits_{k = 1}^K { { {\ell_k}} } > d_{{ {\cal A}_{\bf{0}}}} = mK$ for ${\bs{\ell} \ne \bf{0}}$.
\begin{lemma}\label{the:cdf_pdf_relatoin_hat}
As $\gamma \to \infty$, the ratio of ${{F_{{ {\mathcal A}_{\bs{\ell}}}}}\left( x \right)}$ to ${{F_{{ {\mathcal A}_{\bf{0}}}}}\left( x \right)}$ satisfies
\begin{equation}\label{eqn:F_AI_F_A0_rel_hat}
\frac{{{F_{{ {\cal A}_{\bs{\ell}}}}}\left( x \right)}}{{{F_{{ {\cal A}_{\bf{0}}}}}\left( x \right)}} = o\left( 1 \right),\, {{{\bs{\ell}}} \in {{\mathbb N}_0}^K} \,{\rm and }\, {\bs{\ell} \ne \bf{0}}.
\end{equation}
%\begin{equation}\label{eqn:F_AI_F_A0_rel_hat}
%\frac{{{F_{{ {\cal A}_{{\bs{\ell}}^{\left(1\right)}}}}}\left( x \right)}}{{{F_{{ {\cal A}_{{\bs{\ell}}^{\left(2\right)}}}}}\left( x \right)}} = o\left( 1 \right),\quad \sum\limits_{k = 1}^K {\ell_k^{\left( 1 \right)}}  < \sum\limits_{k = 1}^K {\ell_k^{\left( 2 \right)}} .
%\end{equation}
%where ${{\bs{\ell}}^{\left( 1 \right)}} = \left( {\ell_1^{\left( 1 \right)},\cdots,\ell_K^{\left( 1 \right)}} \right)$ and ${{\bs{\ell}}^{\left( 2 \right)}} = \left( {\ell_1^{\left( 2 \right)},\cdots,\ell_K^{\left( 2 \right)}} \right)$. Hence, for a sufficiently large SNR, i.e., $\gamma \gg 1$, we have
%\begin{equation}\label{eqn:bound_F_A_l}
%\mathop {\max }\limits_{{\ell_1} + \cdots  + {\ell_K} \ge N+1} \left( {{F_{{ {\mathcal A}_{\bs{\ell}}}}}\left( x \right)} \right) = \mathop {\max }\limits_{{\ell_1} +  \cdots  + {\ell_K} = N+1} \left( {{F_{{ {\mathcal A}_{\bs{\ell}}}}}\left( x \right)} \right).
%\end{equation}
\end{lemma}
\begin{proof}
From (\ref{eqn:der_A_1_asyCDF}), we have
  \begin{align}\label{eqn:lemma2_proof}
\mathop {\lim }\limits_{\gamma  \to \infty } \frac{{{F_{{\mathcal A_{\bs \ell} }}}\left( x \right)}}{{{F_{{\mathcal A_{\bf{0}}}}}\left( x \right)}} %&= \mathop {\lim }\limits_{\gamma  \to \infty } \frac{{{\gamma ^{ - {d_{{{\cal A}_{\bs \ell} }}}}}\prod\limits_{k = 1}^K {\frac{1}{{\Gamma \left( {m + {\ell _k}} \right)}}{{\left( {\frac{m}{{{\theta _k}{\sigma _k}^2\left( {1 - {\lambda _k}^2} \right)}}} \right)}^{m + {\ell _k}}}} {g_{\bs \ell} }\left( x \right) + o\left( {{\gamma ^{ - {d_{{{\cal A}_{\bs \ell} }}}}}} \right)}}{{{\gamma ^{ - d_{{{\cal A}_{\bf 0} }}}}\prod\limits_{k = 1}^K {\frac{1}{{\Gamma \left( m \right)}}{{\left( {\frac{m}{{{\theta _k}{\sigma _k}^2\left( {1 - {\lambda _k}^2} \right)}}} \right)}^m}} {g_{\bf{0}}}\left( x \right) + o\left( {{\gamma ^{ - d_{{{\cal A}_{\bf 0} }}}}} \right)}} \notag\\
 &= \prod\limits_{k = 1}^K {\frac{{\Gamma \left( m \right)}}{{\Gamma \left( {m + {\ell _k}} \right)}}{{\left( {\frac{m}{{{\theta _k}{\sigma _k}^2\left( {1 - {\lambda _k}^2} \right)}}} \right)}^{{\ell _k}}}} \frac{{{g_{\bs \ell} }\left( x \right)}}{{{g_{\bf{0}}}\left( x \right)}}\mathop {\lim }\limits_{\gamma  \to \infty } {\gamma ^{d_{{{\cal A}_{\bf 0} }} - {d_{{{\cal A}_{\bs \ell} }}}}} = 0,
\end{align}
where the last equality holds because of ${d_{{{\cal A}_{\bf 0} }} - {d_{{{\cal A}_{\bs \ell} }}}} < 0$ when $\bs \ell \ne \bf 0$. Then by using the little-O notation, (\ref{eqn:F_AI_F_A0_rel_hat}) directly follows.%, we have $\frac{{{F_{{\mathcal A_{\bs \ell} }}}\left( x \right)}}{{{F_{{\mathcal A_{\bf{0}}}}}\left( x \right)}} = o(1)$.
\end{proof}
By applying Lemma \ref{the:cdf_pdf_relatoin_hat} into (\ref{eqn:out_prob_def_hat}), as $\gamma \to \infty$, the outage probability ${p_{out,K}}$ can be rewritten as \begin{align}\label{eqn:cdf_F_shif_gam_pro_asy}
{p_{out,K}} &= {W_{\bf{0}}}{F_{{{ {\mathcal A}}_{\bf{0}}}}}\left( 2^{\mathcal R} \right) + \sum\nolimits_{{\ell_1} +  \cdots  + {\ell_K} > 0}  {{W_{\bs{\ell}}}{F_{{{ {\mathcal A}}_{\bs{\ell}}}}}\left( 2^{\mathcal R} \right)} \notag \\
&= {W_{\bf{0}}}{F_{{{ {\mathcal A}}_{\bf{0}}}}}\left( 2^{\mathcal R} \right)\left( {1 + \frac{1}{{{W_{\bf{0}}}}}\sum\nolimits_{{\ell_1} + \cdots  + {\ell_K} > 0}  {{W_{\bs{\ell}}}\frac{{{F_{{{ {\mathcal A}}_{\bs{\ell}}}}}\left( 2^{\mathcal R} \right)}}{{{F_{{{ {\mathcal A}}_{\bf{0}}}}}\left( 2^{\mathcal R} \right)}}} } \right)\notag\\
& = {W_{\bf{0}}}{F_{{{ {\mathcal A}}_{\bf{0}}}}}\left( 2^{\mathcal R} \right)\left( {1 + \frac{1}{{{W_{\bf{0}}}}}\sum\nolimits_{{\ell_1} +  \cdots  + {\ell_K} > 0}  {{W_{\bs{\ell}}}o\left( 1  \right)} } \right) = {W_{\bf{0}}}{F_{{{ {\mathcal A}}_{\bf{0}}}}}\left( 2^{\mathcal R} \right)\left( {1 + o\left( 1 \right)} \right),
\end{align}
where the last equality holds since $W_{\bs{\ell}}$ is irrelevant to the transmission SNR $\gamma$ and $\sum\nolimits_{{\ell_1} + \cdots  + {\ell_K} > 0}  {{W_{\bs{\ell}}}} = 1-W_{\bf 0} < 1$. Now putting (\ref{eqn:der_A_1_asyCDF}) into (\ref{eqn:cdf_F_shif_gam_pro_asy}) and neglecting the high order terms $o\left( 1 \right)$ and $o\left( {{\gamma ^{ - {d_{{\mathcal A_{\bs{\ell}}}}}}}} \right)$ when $\gamma \to \infty$, the outage probability can be asymptotically approximated as
%From (\ref{eqn:cdf_F_shif_gam_pro_asy}), as $\gamma \to \infty$, the outage probability ${ {p_{out,K}}}$ can be asymptotically approximated by neglecting the higher order term $o\left( 1 \right)$ as
%\begin{equation}\label{eqn:asym_out}
%{ {p_{out,K}}} \approx {W_{\bf{0}}}{F_{{ {\mathcal A}_{\bf{0}}}}}\left( {{2^{\cal R}} } \right),
%\end{equation}
%which means that the outage probability under time-correlated Nakagami-$m$ fading channels can be asymptotically expressed as a weighted outage probability under independent Nakagami-$m$ fading channels, i.e., $\left| {{h_{{{\mathcal A}_{\bf 0}},k}}} \right| \sim {\rm Nakagami}\left( {m,(1-{\lambda_k}^2){\sigma_k}^2} \right), 1 \le k \le K$, and the weighting is ${W_{\bf{0}}}={\left( {1 + \sum\limits_{k = 1}^K {\frac{{{\lambda _k}^2}}{{1 - {\lambda _k}^2}}} } \right)^{ - m}}$.
%
%With (\ref{eqn:der_A_1_asyCDF}), (\ref{eqn:asym_out}) can be further simplified through neglecting higher order term $o\left( {{\gamma ^{ - {d_{{\mathcal A_{\bs{\ell}}}}}}}} \right)$ as
\begin{align}\label{eqn:der_hat_G_CDF_asy_fur}
{p_{out,K}} %&= {W_{\bf{0}}}{\gamma ^{ - mK}}\left( {{g_{\bf{0}}}\left( 2^{\cal R} \right)\prod\limits_{k = 1}^K {\frac{1}{{\Gamma \left( m \right)}}{{\left( {\frac{m}{{{\theta _k}{\sigma _k}^2\left( {1 - {\lambda _k}^2} \right)}}} \right)}^m}}  + o\left( \gamma^{-1/2} \right)} \right)\left( {1 + o\left( \gamma^{-1/2} \right)} \right) \notag\\
% &= {W_{\bf{0}}}{\gamma ^{ - mK}}{g_{\bf{0}}}\left(  2^{\cal R} \right)\prod\limits_{k = 1}^K {\frac{1}{{\Gamma \left( m \right)}}{{\left( {\frac{m}{{{\theta _k}{\sigma _k}^2\left( {1 - {\lambda _k}^2} \right)}}} \right)}^m}} \left( {1 + o\left( \gamma^{-1/2} \right)} \right).
 &\simeq  {W_{\bf{0}}}{\gamma ^{ - mK}}{g_{\bf{0}}}\left(  2^{\cal R} \right)\prod\limits_{k = 1}^K {\frac{1}{{\Gamma \left( m \right)}}{{\left( {\frac{m}{{{\theta _k}{\sigma _k}^2\left( {1 - {\lambda _k}^2} \right)}}} \right)}^m}} \triangleq {p_{out\_{asy},K}}. %\notag\\
 %&= \underbrace {\prod\limits_{k = 1}^K {\frac{1}{{\Gamma \left( m \right)}}{{\left( {\frac{m}{{{\theta _k}{\sigma _k}^2}}} \right)}^m}} }_\zeta \underbrace {{{\left( {\ell \left( {\bs \lambda ,K} \right)} \right)}^{ - m}}}_\varrho {\left( {\underbrace {{{\left( {{g_{\bf{0}}}\left( x \right)} \right)}^{ - \frac{1}{d}}}}_{\rm{C}}\gamma } \right)^{ - d}} + o\left( {{\gamma ^{ - d}}} \right),
\end{align}

Substituting (\ref{eqn:W_l_def_sec}) into (\ref{eqn:der_hat_G_CDF_asy_fur}), under high SNR regime, i.e., as $\gamma \to \infty$, the asymptotic outage probability ${p_{out\_{asy},K}}$ can be factorized as
\begin{align}\label{eqn:der_hat_G_CDF_asy_furre}
{p_{out\_{asy},K}} &= \underbrace {\prod\limits_{k = 1}^K {\frac{1}{{\Gamma \left( m \right)}}{{\left( {\frac{m}{{{\theta _k}{\sigma _k}^2}}} \right)}^m}} }_{\triangleq \zeta(\bs \theta)} \underbrace {{{\left( {\ell \left( {\bs \lambda ,K} \right)} \right)}^{ - m}}}_{\triangleq \varrho(\bs \lambda)} {\left( {\underbrace {{{\left( {{g_{\bf{0}}}\left(  2^{\cal R} \right)} \right)}^{ - \frac{1}{mK}}}}_{\triangleq {C}(\mathcal R)}\gamma } \right)^{ - mK}},
\end{align}
where $\ell \left( {{\bs{\lambda }},K} \right)$ is explicitly given as
 \begin{equation}\label{eqn:ell_def}
\ell \left( {{\bs{\lambda }},K} \right) = \left( {1 + \sum\nolimits_{k = 1}^K {\frac{{{\lambda _k}^2}}{{1 - {\lambda _k}^2}}} } \right)\prod\nolimits_{k = 1}^K {\left( {1 - {\lambda _k}^2} \right)},
\end{equation}
$\zeta (\bs \theta)$ and $\varrho(\bs \lambda)$ quantify the impacts of transmission power allocation and channel time correlation and thus are regarded as power allocation impact factor and time correlation impact factor, respectively, and $C(\mathcal R)$ is termed as coding and modulation gain since it depends on the information transmission rate $\mathcal R$ which eventually is determined by the coding rate, modulation scheme and symbol transmission rate.

%\begin{remark}\label{rem:out_asy_fin_hat}
%$\zeta$, $C$ and $d$ are given respectively as
%\begin{equation}\label{eqn:alpha}
%\zeta = {\left( {\ell \left( {\bs \lambda ,K} \right)\prod\limits_{k = 1}^K {\frac{{{\theta _k}{\sigma _k}^2}}{m}} } \right)^{ - m}},
%C = {\left( {{g_{{\bf{0}},{\bf{0}}}}({2^{\mathcal R}})} \right)^{-\frac{1}{d}}},
%d = mK.
%\end{equation}
%Therefore, it can be seen that the scaling constant $\zeta$ constitutes two parts, including time correlation of fading channels $|\bs \lambda |$ and power allocation factors ${\theta _1,\cdots,\theta _K}$. The coding gain $C$ is determined by coding rate $\mathcal R$. Additionally, for diversity order $d$, since a Nakagami-$m$ fading channel can be viewed as a set of $m$ parallel independent Rayleigh fading channels, it is readily found that full diversity can be achieved by HARQ-IR using a maximum of $K$ transmissions under time-correlated fading channels.
%%Whereas, the coding gain determines the relative horizontal shift of this curve.
%% which depends on the modulation employed and the coding rate \cite{jin2011optimal}
%\end{remark}
%From (\ref{eqn:der_hat_G_CDF_asy_furre}), it is readily found that the outage performance is influenced by time correlation, transmit powers and coding rate.
\subsection{Discussions}
Clearly from (\ref{eqn:der_hat_G_CDF_asy_furre}), the outage performance is determined by the number of transmissions, transmission power allocation, channel time correlation and information transmission rate. Their impacts will be thoroughly investigated through the analysis of the terms $\zeta(\bs \theta) $, $\varrho(\bs \lambda)$ and $C(\mathcal R)$. In addition, outage probability of HARQ-IR under quasi-static fading channels is particularly discussed as a special case of time-correlated fading channels.
\subsubsection{Diversity Order}
Diversity order indicates the number of degrees of freedom in communication systems. Roughly speaking, it is equivalent to the number of independently faded paths that a transmitted signal experiences. Specifically, it is defined as the slope of the outage probability against the transmission SNR on a log-log scale as \cite{chelli2014performance}
\begin{equation}\label{eqn:diver_order_def}
d = -\mathop {\lim }\limits_{\gamma  \to \infty } \frac{{\log \left( {{ p_{out,K}}} \right)}}{{\log \left( \gamma  \right)}}.
\end{equation}
Putting (\ref{eqn:der_hat_G_CDF_asy_furre}) into (\ref{eqn:diver_order_def}), the diversity order of HARQ-IR over time-correlated Nakagami-$m$ fading channels directly follows as $d=mK$. Noticing that a Nakagami-$m$ fading channel can be viewed as a set of $m$ parallel independent Rayleigh fading channels, the maximum number of independent fading channels in a HARQ-IR system with a maximum number of $K$ transmissions is $mK$. In other words, channel time correlation would not degrade the diversity order and full diversity can be achieved by HARQ-IR even under time-correlated fading channels. This result is consistent with those in \cite{shi2015analysis,shi2016inverse}, and demonstrates the correctness of our asymptotic outage analysis.

\subsubsection{Power Allocation Impact Factor $\zeta (\bs \theta)$}
It is clear from (\ref{eqn:der_hat_G_CDF_asy_furre}) that power allocation impact factor $\zeta(\bs \theta)$ characterizes the impact of power allocation on outage probability. Specifically, $\zeta(\bs \theta)$ is an inverse power function of the the product of power allocation factors ${\prod\nolimits_{k = 1}^K {{\theta_k}} }$. It would decrease as the product of power allocation factors $\bs \theta$ increases, which eventually results in the improvement of outage performance. With the definition of $\theta_k=\frac{{{P_k}}}{{{P_{total}}}}$, we also can conclude that the asymptotic outage probability is an inverse power function of the product of the transmission powers in all HARQ rounds. Notice that this clear quantitative relationship between the power allocation factors and the outage probability hasn't been discovered even under independent fading channels. With this quantitative relationship, optimal power allocation to achieve various objectives is enabled. Taking the energy-limited outage minimization as an example, the power allocation problem can be formulated as
\begin{equation}\label{eqn:opt_prob_simp}
\begin{array}{*{20}{cl}}
{\mathop {\rm min}\limits_{{P_1},\cdots,{P_K}} }&{p_{out,K}}\\
{{\rm{s.}}\,{\rm{t.}}}&{{\sum\nolimits_{k = 1}^K {{P_k}{p_{out,k - 1}}}} \le P_{\rm given} }\\
{}&{{P_k} \ge 0,\quad{\rm for}\quad 0 \le k \le K},\\
\end{array}
\end{equation}
where ${\sum\nolimits_{k = 1}^K {{P_k}{p_{out,k - 1}}}}$ refers to the average transmission energy normalized to the codeword length \cite{makki2013green},  $p_{out,0}=1$ and $P_{\rm given}$ is the average energy constraint. By substituting $P_k= {\theta_k} P_{total}$ and using the asymptotic outage probability ${p_{out\_{asy},k}}$ as the approximation of $p_{out,k}$, the optimization problem is reduced to the maximization of the product of power allocation factors ${\prod\nolimits_{k = 1}^K {{\theta_k}} }$ subject to certain constraints. Accordingly, (\ref{eqn:opt_prob_simp}) can be easily converted into the generalized power optimization problem in \cite[Eq.6]{chaitanya2016energy}. Karush-Khun-Tucker conditions can then be applied to derive the optimal solution in closed-form as shown in \cite[Eq.15]{chaitanya2016energy}.

%It is \ref{eqn:opt_prob_simp}
%On the basis of, the optimal power allocation solution can be found in closed-form by using the optimization technique developed .
%
%
% In order to guarantee the power efficiency of HARQ schemes, power allocation of HARQ-IR under time-correlated fading channels should be optimally designed. To proceed with power allocation, the asymptotic outage probability in (\ref{eqn:der_hat_G_CDF_asy_furre}) can be rewritten by using (\ref{eqn:gamma_snr_rel}) as
%\begin{equation}\label{eqn:Pout_def_asy}
%{p_{out\_asy,K}} =  {{{\psi _K}}}{{{{P_{\Pi,K} }^{-m}}}},
%\end{equation}
%where ${P_{\Pi,K} }={\prod\nolimits_{k = 1}^K {{P_k}} }$, ${\psi _K} = \frac{g_{{\bf{0}}}({2^{\mathcal R}})}{\left(\Gamma(m)\right)^{{K}}}{\left( {\ell \left( {|\bs \lambda |,K} \right)}{\prod\nolimits_{k = 1}^K {\frac{{{\sigma _k}^2}}{m\mathcal N_0}} } \right)^{ - m}}$ and ${\psi _0}=1$. Clearly from (\ref{eqn:Pout_def_asy}), when power allocation schemes are considered to minimize outage probability, it is equivalent to maximizing the power-product term ${P_{\Pi,K} }$. It is noteworthy that the asymptotic outage probability of HARQ-IR has never been accurately derived even under independent fading channels, which further justifies the great significance of our work. It is expected that (\ref{eqn:Pout_def_asy}) would have wide applications in optimal system design due to its simple form. Taking optimal power allocation as an example,
%the outage probability is minimized given a constraint on average transmit power, such that
%\begin{equation}\label{eqn:opt_prob_simp}
%\begin{array}{*{20}{cl}}
%{\mathop {\rm minimize}\limits_{{P_1},\cdots,{P_K}} }&{p_{out,K}}\\
%{{\rm{subject}}\,{\rm{to}}}&{{P_k} \ge 0,\quad{\rm for}\quad 0 \le k \le K}\\
%{}&{{\sum\limits_{k = 1}^K {{P_k}{p_{out,k - 1}}}} \le P_{\rm given} }.\\
%\end{array},
%\end{equation}
%where $p_{out,0}=1$ and $P_{\rm given}$ is the average power constraint. %However, independent Rayleigh fading channels are considered in \cite{chaitanya2016energy}, whose results are certainly inapplicable to our cases. On the other hand, in \cite{chaitanya2016energy}, the asymptotic outage probability of HARQ-IR is actually approximated with a lower bound based on Jensen's inequality, which leads to inaccurate solutions in real applications.
%Since the asymptotic outage probability ${p_{out\_asy,k}}$ in (\ref{eqn:Pout_def_asy}) is found to be a function of power-product term ${P_{\Pi,k} }$ for $1 \le k \le K$, the optimization method developed in \cite{chaitanya2016energy} can be adopted to solve problem (\ref{eqn:opt_prob_simp}) in closed-form. For more details, please consult with \cite{chaitanya2016energy}.
%
%
%%With the help of , the exact and closed-form solution to problem (\ref{eqn:opt_prob_simp}) can be easily obtained for HARQ-IR over time-correlated Nakagami-$m$ fading channels. %, such that
%% \begin{equation}\label{eqn:power_allo_p1}
%%{P_1} = \frac{{m{{\left( {m + 1} \right)}^{K - 1}}}}{{{{\left( {m + 1} \right)}^K} - 1}}{P_{\rm given}},
%%\end{equation}
%%\begin{equation}\label{eqn:power_allo_pk}
%%{P_k} = \prod\limits_{\iota = 1}^{k - 1} {{{\left( {\frac{{{\psi _{\iota - 1}}}}{{\left( {m + 1} \right){\psi _\iota}}}} \right)}^{{{\left( {m + 1} \right)}^{k - \iota - 1}}}}} {P_1}^{{{\left( {m + 1} \right)}^{k - 1}}},\quad{\rm for}\quad 2 \le k \le K.
%%\end{equation}
%%

\subsubsection{Time Correlation Impact Factor $\varrho(\bs \lambda)$}
The time correlation impact factor $\varrho(\bs \lambda)$ quantifies the impact of channel time correlation on the outage probability. Notice that when only one transmission is allowed, i.e., $K=1$, no time correlation is involved and we always have $\ell \left( {{\bs{\lambda }},K} \right)=1$ and $\varrho(\bs \lambda)=1$. When $K>1$, the time correlation impact factor $\varrho(\bs \lambda)$ has one property as shown in the following lemma. For notational convenience, we define a partial ordering for two vectors ${\bf x}= (x_1, x_2, \cdots, x_K), {\bf y}= (y_1, y_2, \cdots, y_K) \in \mathbb{R}^K$ as ${\bf x} \preceq \bf y$ if $x_i \le y_i, i = 1,2,\cdots,K$.
%
%
%In (\ref{eqn:der_hat_G_CDF_asy_furre}), the term $\ell \left( {{\bs{\lambda }},K} \right)$ exhibits the impact of time correlation on $\varrho(\bs \lambda)$, which consequently influences outage probability. To thoroughly examine the effect of correlation coefficients, the following remark is proved. For notational convenience, we define the partial ordering as ${\bf x} \prec _{\mathbb{R}_+^K} \bf y$ if $x_i < y_i, i = 1,2,\cdots,K$ and ${\bf x},{\bf y} \in \mathbb{R}^K$.
\begin{lemma}\label{cor:time_corr}
Given $K>1$ and two time correlation vectors $\bs{\lambda}_1 \preceq \bs{\lambda}_2$, we have
\begin{equation}\label{eqn:lam_ell_con}
1 \ge \ell \left( {{\bs{\lambda }_1},K} \right) \ge \ell \left( {{\bs{\lambda }_2},K} \right),
\end{equation}
\begin{equation} \label{eqn:varrho}
1 \le \varrho(\bs \lambda_1) \le \varrho(\bs \lambda_2).
\end{equation}
The left equalities in (\ref{eqn:lam_ell_con})-(\ref{eqn:varrho}) hold if and only if $\bs \lambda_1 = \bf 0$, while the right equalities in (\ref{eqn:lam_ell_con})-(\ref{eqn:varrho}) hold if and only if $\bs \lambda_1 = \bs \lambda_2$ .
\end{lemma}
\begin{proof}
Please see Appendix \ref{app:proo_cor}.
\end{proof}

It can be concluded from (\ref{eqn:der_hat_G_CDF_asy_furre}) and Lemma \ref{cor:time_corr} that although the time correlation does not affect the diversity order, the increase of time correlation coefficients would cause the increases of $\varrho(\bs \lambda)$ and the outage probability, thus resulting in the degradation of outage performance.
%
%According to the preceding analysis, even though the time correlation of fading channels does not influence the diversity order (except for quasi-static fading channels), it affects the time correlation scaling constant $\varrho(\bs \lambda)$ from (\ref{eqn:der_hat_G_CDF_asy_furre}), eventually deteriorates the outage performance.

\subsubsection{Coding and Modulation Gain $C(\mathcal R)$}
As defined in (\ref{eqn:der_hat_G_CDF_asy_furre}), $C(\mathcal R)=\left({g_{{\bf{0}}}}\left( 2^{\cal R} \right)\right)^{-\frac{1}{d}}$. For a given outage probability, the increase of $C(\mathcal R)$ can result in the reduction of the SNR $\gamma $. In other words, $C(\mathcal R)$ can quantify the amount of SNR reduction for a given outage probability under certain coding and modulation scheme. It is thus termed as coding and modulation gain \cite{goldsmith2005wireless}. After analyzing the function ${g_{{\bf{0}}}}\left( 2^{\mathcal R} \right)$, we have the following property of the coding and modulation gain.
\begin{lemma}\label{the:coding_modulation_gain}
The function ${g_{{\bf{0}}}}\left( 2^{\mathcal R} \right)$ is a monotonically increasing function of the information transmission rate $\mathcal R$, and is convex with respect to $\mathcal R$ for any fading order $m \ge 1$, and thus the coding and modulation gain $C(\mathcal R)$ is a monotonically decreasing function of the information transmission rate $\mathcal R$.
\end{lemma}
\begin{proof}
Please see Appendix \ref{app:coding_gain}.
\end{proof}

It follows from (\ref{eqn:der_hat_G_CDF_asy_furre}) and Lemma \ref{the:coding_modulation_gain} that the asymptotic outage probability is a monotonically increasing and convex function of the information transmission rate $\mathcal R$ when $m \ge 1$. In order to achieve a desired performance, the information transmission rate should be properly chosen. Owing to the simple analytical expression in (\ref{eqn:der_hat_G_CDF_asy_furre}), the optimal rate can be easily found. Taking the maximization of the long term average throughput (LTAT) given an allowable outage constraint $\epsilon $ as an example, the rate selection problem can be formulated as
\begin{align}\label{eqn_op}
{\mathop {{\rm{max}}}\limits_{\mathcal R \in \left\{\mathcal R \in \mathbb R_+: {p_{out,K}} \le \epsilon \right\}} }&{\quad \bar {\mathcal T}  = \frac{{\mathcal R\left( {1 - {p_{out,K}}} \right)}}{{\sum\nolimits_{k = 0}^{K - 1} {{p_{out,k}}} }}},
\end{align}
where $\bar {\mathcal T}$ denotes the LTAT. Clearly, the numerator ${\mathcal R\left( {1 - {p_{out,K}}} \right)}$ and the denominator ${ \sum\nolimits_{k = 0}^{K - 1} {{p_{out,k}}} }$ in (\ref{eqn_op}) are concave and convex with respect to $\mathcal R$, respectively, while the feasible region $\left\{\mathcal R \in \mathbb R_+: {p_{out,K}} \le \epsilon \right\}$ is a convex set, when $m \ge 1$. As shown in \cite{dinkelbach1967nonlinear}, the optimization problem in (\ref{eqn_op}) is a concave fractional programming problem and the globally optimal solution can be easily found using the techniques proposed in \cite{dinkelbach1967nonlinear}.

%Therefore, Dinkelbach's algorithm proposed in \cite{dinkelbach1967nonlinear} can be adopted to solve the nonlinear fractional programming problem (\ref{eqn_op}).

\subsubsection{Quasi-Static Fading channels}
%Considering the independence of $\bs \gamma$ conditioned on $T$, the conditional joint PDF of SNRs $\bs \gamma$ can be written as
%\begin{equation}\label{eqn:joint_pdf_r_cond}
%{f_{\left. \bs \gamma  \right|T}}\left( {\left. {{x_1}, \cdots ,{x_K}} \right|T} \right) = \prod\limits_{k = 1}^K {{f_{\left. {{\gamma _k}} \right|T}}\left( {\left. {{x_k}} \right|T} \right)}.
%\end{equation}
%
%
%
%With the conditional joint PDF of SNRs $\bs \gamma$ in (\ref{eqn:joint_pdf_r_cond}), the joint PDF of ${\bs {\gamma}}$ can be derived by taking its expectation with respect to $T$, such that
%\begin{multline}\label{eqn:joint_pdf_gamma}
%{f_{\bs \gamma} }\left( {{x_1}, \cdots ,{x_K}} \right) = \int\nolimits_0^\infty  {{f_T}\left( t \right){f_{\left. {\bs \gamma}  \right|T}}\left( {\left. {{x_1}, \cdots ,{x_K}} \right|T = t} \right)dt} \\
% = \int\nolimits_0^\infty  {\frac{{{t^{m - 1}}}}{{\Gamma \left( m \right)}}{e^{ - t}}\prod\limits_{k = 1}^K {{{\left( {\frac{1}{{{\Omega _k}}}} \right)}^m}\frac{{{x_k}^{m - 1}}}{{\Gamma \left( m \right)}}{e^{ - \frac{{{u_k}{\lambda _k}^2t + {x_k}}}{{{\Omega _k}}}}}{}_0{F_1}\left( {;m;{{\left( {\frac{{\sqrt {{u_k}{\lambda _k}^2{x_k}t} }}{{{\Omega _k}}}} \right)}^2}} \right)} dt} .
%\end{multline}
%
%As pointed out in (\ref{eqn:cond_pdf_r_k_rew}), it is worthwhile noting that (\ref{eqn:joint_pdf_gamma}) is not applicable to quasi-static fading channels, i.e., $|{\boldsymbol{\lambda }}| = {\bf 1}_K$. However, taking the limit of (\ref{eqn:joint_pdf_gamma}) as $|{\boldsymbol{\lambda }}| \to {\bf 1}_K$ yields
%\begin{align}\label{eqn:joint_gamma_rho_1}
%\mathop {\lim }  \limits_{|{\boldsymbol{\lambda }}| \to {\bf 1}_K}  {f_{\bs{\gamma }}}\left( {{x_1}, \cdots ,{x_K}} \right)  &= \frac{{{x_1}^{m - 1}}}{{\Gamma \left( m \right){{\left( {{u_1}^2} \right)}^m}}}{e^{ - \frac{{{x_1}}}{{{u_1}^2}}}}\prod\limits_{k = 2}^K {\delta \left( {{x_k} - \frac{{{u_k}{x_1}}}{{{u_1}^2}}} \right)},
%\end{align}
%where $\delta(\cdot)$ represents Dirac delta function \cite[1.17.5]{olver2010nist}. Apparently, (\ref{eqn:joint_gamma_rho_1}) is the joint PDF of the received SNRs ${\bs{\gamma }}$ under quasi-static fading channels. Nevertheless, it does not mean that the analytical results for correlated fading channels with $|\lambda_1|,\cdots, |\lambda_K| \ne 1$ can be applied to quasi-static fading channels. Due to its particularity, the asymptotic outage analysis of HARQ-IR over quasi-static fading channels will be discussed separately.

As aforementioned, the probability distribution in Section \ref{sec:exa} and the asymptotic outage probability in (\ref{eqn:der_hat_G_CDF_asy_furre}) are not applicable to the case of quasi-static fading channels with ${\boldsymbol{\lambda }} = {\bf 1}$. Particularly, under quasi-static fading channels, the channel coefficients are constant among multiple transmissions, i.e., $|h_1|=\cdots=|h_K| \triangleq |h| \sim {\rm Nakagami}(m,{\sigma}^2)$. Assuming constant transmission powers $P_1 = \cdots= P_K \triangleq P$, that is, $\theta_1 = \cdots= \theta_K \triangleq \theta$, the outage probability ${p_{out,K}}$ under quasi-static fading channels is readily obtained as
\begin{equation}\label{eqn:out_prob_quasi_static}
{p_{out,K}} = \Pr \left( {{{\log }_2}{{\left( {1 + \frac{{P}{{\left| {{h}} \right|}^2}}{\mathcal N_0}} \right)}^K} < \mathcal R} \right) = \frac{1}{{\Gamma \left( m \right)}}\Upsilon \left( {m,\frac{{m\mathcal N_0\left( {{2^{\mathcal R/K}} - 1} \right)}}{{{P}{\sigma}^2}}} \right).
\end{equation}
%Even though (\ref{eqn:joint_gamma_rho_1}) establishes the limit relationship between joint PDFs of the received SNRs under time-correlated fading channels with $|{\boldsymbol{\lambda }}| \ne {\bf 1}_K$ and under quasi-static fading channels, it is clear that the same relationship as (\ref{eqn:joint_gamma_rho_1}) does not hold for (\ref{eqn:out_prob_def_hat}) and (\ref{eqn:out_prob_quasi_static}), because the operations of limit and Mellin transform are not exchangeable.
%Undoubtedly, time correlation of fading channels, power allocation and coding rate affect asymptotic outage probability, whose impacts will be elaborated later. However, it is worth noting that the same results in Theorem \ref{rem:out_asy_fin_hat}  does not hold for quasi-static fading channels.
By applying \cite[Eq.8.354.1]{gradshteyn1965table} into (\ref{eqn:out_prob_quasi_static}), the outage probability can be rewritten as
\begin{align}\label{eqn:out_prob_quasi_asy}
{p_{out,K}}&= \frac{1}{{\Gamma \left( m \right)}}\sum\limits_{n = 0}^\infty  {\frac{{{{\left( { - 1} \right)}^n}}}{{n!\left( {m + n} \right)}}{{\left( {\frac{{m\left( {{2^{{\cal R}/K}} - 1} \right)}}{{\gamma {\theta{\sigma}^2}}}} \right)}^{m + n}}} \notag \\
&= {\left( {\frac{m}{{{\theta}{\sigma}^2}}} \right)^m}{\left( {\frac{{{{\left( {\Gamma \left( {m + 1} \right)} \right)}^{1/m}}}}{{{2^{{\cal R}/K}} - 1}}\gamma } \right)^{ - m}} + o\left( {{\gamma ^{ - m}}} \right).
\end{align}
Then the asymptotic outage probability under this quasi-static fading channels can be written as ${p_{out\_{asy},K}} = \zeta(\bs \theta) \left( C(\mathcal R \right ) \gamma)^{-m}$, where the power allocation impact factor is given as $\zeta(\bs \theta) = {\left( {\frac{m}{{{\theta{\sigma}^2}}}} \right)^{ m}}$, while the coding and modulation gain becomes $C(\mathcal R) = \frac{{{{\left( {\Gamma \left( {m + 1} \right)} \right)}^{1/m}}}}{{{2^{{\mathcal R}/K}} - 1}}$ and the diversity order reduces to $d=m$ due to the full correlation of fading channels. It means that no time diversity can be achieved from multiple transmissions under quasi-static fading channels.

\section{Numerical Results And Optimal Design} \label{sec:num_res}
The analytical results are now verified and optimal system design is discussed in this section. For illustration, we take systems with equicorrelated channels (i.e., $\bs \lambda_{\rm eq}=(\rho,\cdots,\rho)$) \cite{alouini2001sum,aalo1995performance,chen2004distribution} and unit Nakagami spread ${\sigma_1}^2=\cdots={\sigma_K}^2=1$ as examples, unless otherwise indicated. %Notice that the exact outage probability in (\ref{eqn:out_prob_def_hat}) is calculated approximately by truncating the infinite series into a finite series with $\sum\limits_{k = 1}^K {{\ell _k}} \le N$.
In the following numerical analysis, the exact outage probability is approximated by (\ref{eqn:F_G_x_trunc_ser}) with the truncation order set as $N=3$ and the involved generalized Fox's H function is efficiently calculated by the MATHEMATICA{\textregistered} program in \cite{yilmaz2010outage} with a properly chosen Mellin-Barnes contour.


%For comparison, the exponential correlation channels with correlation coefficients as $\bs \lambda_{\rm exp}=(\rho,\rho^2,\cdots,\rho^K)$ \cite{aalo1995performance,kim2011optimal} and equal power allocation (EPA) scheme with $\theta_1=\cdots=\theta_K=1/K$ are also considered.

%\subsection{Verification of Analytical Results}
\subsection{Performance Evaluation}
In Fig. \ref{fig:out_s}, the outage probability $p_{out,K}$ is plotted against the transmission SNR $\gamma$ by setting $K=4$ and $\mathcal R=4$bps/Hz. The approximated results based on  polynomial fitting technique \cite{shi2015analysis} and inverse moment matching method \cite{shi2016inverse} are also presented for comparison. Notice that polynomial fitting technique in \cite{shi2015analysis} is proposed for HARQ-IR under Rayleigh fading channels and thus its result only for Rayleigh fading (i.e., $m=1$) is shown in Fig. \ref{fig:out_s}. It is clear that the approximated results (\ref{eqn:F_G_x_trunc_ser}) coincide with the simulation results, while the asymptotic results (\ref{eqn:der_hat_G_CDF_asy_furre}) approach to the approximated/simulation results under high SNR regime. However, under the considered scenarios, neither \cite{shi2015analysis} nor \cite{shi2016inverse} can provide an accurate approximation under high SNR regime due to the fact that those proposed methods can only guarantee their convergence in MSE. In addition, the outage probabilities decrease with the increase of the transmission SNR $\gamma$ and the decreasing rate becomes larger when the fading order $m$ increases, because the diversity order is $mK$ and $p_{out\_asy,K}$ is directly proportional to $\gamma^{-mK}$. As expected, channel time correlation has a detrimental impact on outage probability. For a given fading order $m$, the outage probability curves corresponding to different correlations become parallel as $\gamma$ increases due to the same diversity order which is irrelevant to the correlation coefficient. These numerical results thus validate the results in Section \ref{sec:asy}.

%Thus this comparison further demonstrates the significance of our asymptotic outage analysis. More interestingly, it is worth emphasizing that the usage of (\ref{eqn:der_hat_G_CDF_asy_furre}) to evaluate outage probability definitely has a much lower computational complexity than those of simulation/numerical methods due to the simple form of (\ref{eqn:der_hat_G_CDF_asy_furre}). Particularly for an integer $m$, ${{g_{\bf{0}}}\left(  2^{\cal R} \right)} $ can be further simplified by using (\ref{eqn:G_meij_factor_poly}), then the evaluation of asymptotic outage probability in (\ref{eqn:der_hat_G_CDF_asy_furre}) only involves a number of basic arithmetical operations.

\begin{figure}
  \centering
  % Requires \usepackage{graphicx}
  \includegraphics[width=2.5in]{./out_validation9.eps}\\
  \caption{Outage probability $p_{out,K}$ versus transmission SNR $\gamma$.}\label{fig:out_s}
\end{figure}

To further investigate the effect of time correlation, Fig. \ref{fig:corr} depicts time correlation impact factors $\varrho(\bs \lambda)$ under two different channel correlation models, i.e., equal correlation and exponential correlation. For exponential correlation model, the correlation coefficients follow as $\bs \lambda_{\rm exp}=(\rho,\rho^2,\cdots,\rho^K)$ \cite{aalo1995performance,kim2011optimal}. It can be seen that the time correlation impact factor $\varrho(\bs \lambda)$ increases as $\rho$ increases, thus causing the degradation of the outage performance. The curves become steeper with the increase of $\rho$, which means the outage performance degradation would become more significant when time correlation is high. By comparing the time correlation impact factors $\varrho(\bs \lambda_{\rm eq})$ and $\varrho(\bs \lambda_{\rm exp})$, we can find that $\bs\lambda_{\rm exp} \preceq \bs\lambda_{\rm eq} $ and  $\varrho(\bs \lambda_{\rm exp}) \le \varrho(\bs \lambda_{\rm eq})$, which is consistent with Lemma \ref{cor:time_corr}. In addition, it is found that the increase of the maximum number of transmissions $K$ will lead to the increase of the time correlation impact factor $\varrho(\bs \lambda)$ no matter under equal correlation model or under exponential correlation model. This is because that $\varrho(\bs \lambda_{K}) = \varrho((\bs \lambda_K,0))$ and $(\bs \lambda_K,0) \preceq (\bs \lambda_K,\lambda_{K+1})$. It follows from Lemma \ref{cor:time_corr} that $\varrho(\bs \lambda_K) \le \varrho((\bs \lambda_K,\lambda_{K+1}))$ and the equality holds if and only if $\lambda_{K+1}=0$. However, when $K$ increases, the increase of the time correlation impact factor will be offset by the decrease of the term $\left( C(\mathcal R \right ) \gamma)^{-mK}$ and thus would not cause the degradation of the outage performance.

% Moreover, it is also found from Fig. \ref{fig:corr} that the increase of $m$ will lead to the increase of $\varrho(\bs \lambda)$, but it does not cause the increase of outage probability because the diversity order dominates $p_{out\_asy,K}$.
\begin{figure}
  \centering
  % Requires \usepackage{graphicx}
  \includegraphics[width=2.5in]{./corr_effect4.eps}
  \caption{Time correlation impact factor with $m=2$.}\label{fig:corr}
\end{figure}

Finally, Fig. \ref{fig:coding_gain} illustrates the impacts of the information transmission rate $\cal R$ and the fading order $m$ on the coding and modulation gain $C(\cal R)$ under both time-correlated fading channels and quasi-static fading channels. Clearly, the coding and modulation gain $C(\cal R)$ decreases with the increase of the rate $\mathcal R$, which has already been proved in Lemma \ref{the:coding_modulation_gain}. Additionally, the coding and modulation gain over quasi-static fading channels is superior to the gain over time-correlated fading channels. The reason behind this is that multiple HARQ transmissions under quasi-static fading channels can be viewed as one single transmission with lower coding rate due to the same channel realization experienced, thus leading to the improvement of coding and modulation gain. However, such coding and modulation gain improvement is achieved in sacrifice of time diversity, that is, the diversity order of quasi-static fading channels reduces to $m$. Moreover, Fig. \ref{fig:coding_gain} shows that the coding and modulation gain $C(\mathcal R)$ could benefit from the increase of fading order $m$. %Particularly for quasi-static fading channels, this can be rigorously proved by applying Lyapunov's inequality and the detailed proof is skipped due to space limitation. %which can be rigorously proved by applying Lyapunov inequality to the definition of $C(\mathcal R)$. %A rigorous proof of this result is given in Appendix \ref{app:relation_c_m}.
\begin{figure}
  \centering
  % Requires \usepackage{graphicx}
  \includegraphics[width=2.5in]{./coding_gain4.eps}\\
  \caption{Coding and modulation gain $C(\mathcal R)$ versus information transmission rate $\mathcal R$ for systems with $K=4$.}\label{fig:coding_gain}
\end{figure}
\subsection{Optimal System Design}
With the simple expression, the asymptotic result in (\ref{eqn:der_hat_G_CDF_asy_furre}) would facilitate optimal system design for HARQ-IR with low complexity\footnote{Specifically, the asymptotic outage probability (\ref{eqn:der_hat_G_CDF_asy_furre}) can be easily computed by a few arithmetical operations. Its computational complexity is much lower than the simulation results which require the calculations on a large number of channel realizations. For example, in order to obtain the outage probability of $10^{-5}$, more than $10^6$ channel realizations are required to be simulated, which definitely causes very high computational complexity.}. Here optimal power allocation in (\ref{eqn:opt_prob_simp}) and optimal rate selection in (\ref{eqn_op}) are particularly investigated as examples.

The optimal power allocation (OPA) schemes in (\ref{eqn:opt_prob_simp}) designed based on approximated outage probability (\ref{eqn:F_G_x_trunc_ser}) and asymptotic outage probability (\ref{eqn:der_hat_G_CDF_asy_furre}) are first compared with an optimal equal power allocation (OEPA) scheme where the outage probability is minimized as (\ref{eqn:opt_prob_simp}) with additional equal power allocation constraint (i.e., $P_1=\cdots=P_K$). The achieved optimal outage probabilities $p_{out,K}^*$ for a system with $K=2$, $\mathcal R=2$bps/Hz and $\rho=0.5$ are shown in Fig. \ref{fig:ppa_epa_comp}. It is clear that the OPA solutions found based on approximated outage probability and asymptotic outage probability agree well and can lead to similar outage probability, which justifies the adoption of the asymptotic outage probability for system design. Notice that the OPA solution based on asymptotic outage probability can be easily found in closed-form as shown in \cite{chaitanya2016energy} and its computational complexity is significantly reduced compared to the optimization based on approximated outage probability which requires exhaustive search with a large number of numerical computations. Moreover, OPA scheme performs better than OEPA since the transmission powers in all HARQ rounds are optimized.

\begin{figure}
  \centering
  % Requires \usepackage{graphicx}
  \includegraphics[width=2.5in]{./ppa_epa6.eps}\\
  \caption{Comparison between OPA and OEPA.}\label{fig:ppa_epa_comp}
\end{figure}

Fig. \ref{fig:opt_rate} then illustrates the maximal LTAT $\bar {\mathcal T}$ achieved through optimal rate selection in (\ref{eqn_op}) by setting $m=1$ and $K=4$. Similarly, the results based on the approximated outage probability and the asymptotic outage probability match well, further validating our asymptotic results. It is also observed that the maximal LTAT increases when the outage constraint $\epsilon $ increases and/or the channel time correlation reduces.
\begin{figure}
  \centering
  % Requires \usepackage{graphicx}
  \includegraphics[width=2.5in]{./max_ltat_3.eps}\\
  \caption{Maximal LTAT $\bar {\mathcal T}$ versus transmission SNR $\gamma$.}\label{fig:opt_rate}
\end{figure}
\section{Conclusions} \label{sec:con}
In this paper, asymptotic outage analysis has been conducted to thoroughly investigate the impacts of channel time correlation, transmit powers and information transmission rate on the performance of HARQ-IR over time correlated Nakagami fading channels. Clear insights have been discovered. Particularly, it has been revealed that the power allocation impact factor is an inverse power function of the product of power allocation factors, the time correlation impact factor is an increasing function of the channel time correlation coefficients, while the coding and modulation gain is a monotonically decreasing function of the information transmission rate. Therefore, high product of power allocation factors, low channel time correlation and low information transmission rate are favorable for improving outage performance. The simple form and special properties of asymptotic outage probability would effectively facilitate the optimal system design to achieve various objectives, e.g., optimal power allocation to minimize outage probability and optimal rate selection to maximize the LTAT.

\appendices
\section{The derivation of ${f_{\left. G_K \right|T}}(x|t)$}\label{app:phi_derivation}
The Mellin transform with respect to the conditional PDF of $G_K$ given $T=t$, ${f_{\left. G_K \right|T}}(x|t)$, can be written as
\begin{equation}\label{eqn_mellin_Rk1_def}
\left\{ {\mathcal M{f_{\left. G_K \right|T}}} \right\}\left( s \right) = {\rm E}\left\{ {\left. {{G_K^{s - 1}}} \right|T=t} \right\} \triangleq \phi \left( {\left. s \right|t} \right).
\end{equation}
Due to the independence of SNRs $\bs \gamma $ given $T=t$ and with the definition of $G_K \triangleq \prod\nolimits_{k = 1}^K {\left( {1 + {\gamma _k}} \right)}$, $\phi \left( {\left. s \right|t} \right)$ can be rewritten as
\begin{equation}\label{eqn_mellin_Rk}
\phi \left( {\left. s \right|t} \right) = \prod\nolimits_{k = 1}^K {{\rm E}\left\{ {\left. {{{{{\left(1 + \gamma_k\right)}} }^{s - 1}}} \right|t} \right\}}.
\end{equation}
Herein, with the conditional PDF ${{f_{\left. {{\gamma _k}} \right|T}}\left( {\left. {{x_k}} \right|t} \right)}$ in (\ref{eqn:cond_pdf_r_k_rew}), ${\rm E}\left\{ {\left. {{{\left( {1 + {\gamma_k}} \right)}^{s - 1}}} \right|t} \right\}$ can be derived as
\begin{equation}\label{eqn_mellin_r_1_p}
{\rm{E}}\left\{ {\left. {{{\left( {1 + {\gamma_k}} \right)}^{s - 1}}} \right|t} \right\} %= \int\nolimits_0^\infty  {{{\left( {1 + {x_k}} \right)}^{s - 1}}{f_{\left. {{\gamma_k}} \right|T}}\left( {\left. {{x_k}} \right|t} \right)d{x_k}} \\
= \frac{{{e^{ - \frac{{{u_k}{\lambda _k}^2t}}{{{\Omega _k}}}}}}}{{\Gamma \left( m \right){{\left( {{\Omega _k}} \right)}^m}}}\int\nolimits_0^\infty  {{{\left( {1 + {x_k}} \right)}^{s - 1}}{x_k}^{m - 1}{e^{ - \frac{{{x_k}}}{{{\Omega _k}}}}}{}_0{F_1}\left( {;m;{{\left( {\frac{{\sqrt {{u_k}{\lambda _k}^2t} }}{{{\Omega _k}}}} \right)}^2}{x_k}} \right)d{x_k}}.
\end{equation}
By using the series expansion of the confluent hypergeometric limit function \cite[Eq.16.2.1]{olver2010nist}, (\ref{eqn_mellin_r_1_p}) is further derived as
\begin{align}\label{eqn_expre_mellin_r_expansion}
{\rm{E}}\left\{ {\left. {{{\left( {1 + \gamma_k} \right)}^{s - 1}}} \right|t} \right\} %&= \frac{{{e^{ - \frac{{{u_k}{\lambda _k}^2t}}{{{\Omega _k}}}}}}}{{\Gamma \left( m \right){{\left( {{\Omega _k}} \right)}^m}}} \times \notag \\
%&  \int\nolimits_0^\infty  {{{\left( {1 + {x_k}} \right)}^{s - 1}}{x_k}^{m - 1}{e^{ - \frac{{{x_k}}}{{{\Omega _k}}}}} \sum\limits_{{\ell_k} = 0}^\infty  {\frac{1}{{{\ell_k}!{{\left( m \right)}_{{\ell_k}}}}}{{\left( {{{\left( {\frac{{\sqrt {{u_k}{\lambda _k}^2t} }}{{{\Omega _k}}}} \right)}^2}{x_k}} \right)}^{{\ell_k}}}} d{x_k}}\notag\\
 &= \frac{{{e^{ - \frac{{{u_k}{\lambda _k}^2t}}{{{\Omega _k}}}}}}}{{{{\left( {{\Omega _k}} \right)}^m}}}\sum\limits_{{\ell_k} = 0}^\infty  {\frac{1}{{{\ell_k}!\Gamma \left( {m + {\ell_k}} \right)}}{{{{\left( {\frac{{\sqrt {{u_k}{\lambda _k}^2t} }}{{{\Omega _k}}}} \right)}^{2{\ell_k}}}}}} \int\nolimits_0^\infty  {{{\left( {1 + {x_k}} \right)}^{s - 1}}{x_k}^{m + {\ell_k} - 1}{e^{ - \frac{{{x_k}}}{{{\Omega _k}}}}}d{x_k}} \notag \\
 &= \frac{{{e^{ - \frac{{{u_k}{\lambda _k}^2t}}{{{\Omega _k}}}}}}}{{{{\left( {{\Omega _k}} \right)}^m}}}\sum\limits_{{\ell_k} = 0}^\infty  {\frac{1}{{{\ell_k}!}}{{{{\left( {\frac{{\sqrt {{u_k}{\lambda _k}^2t} }}{{{\Omega _k}}}} \right)}^{2{\ell_k}}}}}} \Psi \left( {m + {\ell_k}, m + {\ell_k} + s;\frac{1}{{{\Omega _k}}}} \right),
\end{align}
where $\Psi \left( \cdot \right)$ denotes Tricomi's confluent hypergeometric function \cite[Eq.9.211.4]{gradshteyn1965table}. Plugging (\ref{eqn_expre_mellin_r_expansion}) into (\ref{eqn_mellin_Rk}) together with some algebraic manipulations yields
%\begin{align}\label{eqn_psi_1_p_app}
%\phi \left( {\left. s \right|t} \right) &= \prod\limits_{k = 1}^K {\frac{{{e^{ - \frac{{{u_k}{\lambda _k}^2t}}{{{\Omega _k}}}}}}}{{{{\left( {{\Omega _k}} \right)}^m}}}\sum\limits_{{\ell_k} = 0}^\infty  {\frac{{{{\left( {\frac{{\sqrt {{u_k}{\lambda _k}^2t} }}{{{\Omega _k}}}} \right)}^{2{\ell_k}}}}}{{{\ell_k}!}}} \Psi \left( {m + {\ell_k},m + {\ell_k} + s;\frac{1}{{{\Omega _k}}}} \right)}.
%\end{align}
%After some algebraic manipulations, $\phi \left( {\left. s \right|t} \right)$ can be finally derived as .
\begin{equation}\label{eqn_mellin_Rk1}
\phi \left( {\left. s \right|t} \right)
= \sum\limits_{{\ell_1}, \cdots ,{\ell_K} = 0}^\infty  {{t^{\sum\limits_{k = 1}^K {{\ell_k}} }}{e^{ - t\sum\limits_{k = 1}^K {\frac{{{\lambda _k}^2}}{{1 - {\lambda _k}^2}}} }}\prod\limits_{k = 1}^K {\frac{{{{\left( {\frac{{{\lambda _k}^2}}{{1 - {\lambda _k}^2}}} \right)}^{{\ell_k}}}}}{{{\ell_k}!{{\left( {{\Omega _k}} \right)}^{m + {\ell_k}}}}}\Psi \left( {m + {\ell_k},m + {\ell_k} + s;\frac{1}{{{\Omega _k}}}} \right)} }.
\end{equation}
Meanwhile, by using inverse Mellin transform, the conditional PDF $f_{\left. G_K \right|T}(x|t)$ can be written as \cite{debnath2010integral}
\begin{equation}\label{eqn_pdf_G_on_T}
{f_{\left. G_K \right|T}}\left( {\left. x \right|t} \right) = \left\{ {{\mathcal M^{ - 1}}\phi } \right\}\left( \left. x \right| t\right) = \frac{1}{{2\pi \rm i}}\int\nolimits_{c - {\rm i}\infty }^{c + {\rm i}\infty } {{x^{ - s}}} \phi \left( {\left. s \right|t} \right)ds,
\end{equation}
where ${\rm i} = \sqrt{-1}$. Putting (\ref{eqn_mellin_Rk1}) into (\ref{eqn_pdf_G_on_T}) finally leads to (\ref{eqn:g_hat_cond_t}).


%By putting (\ref{eqn:cond_pdf_r_k_rew}) into (\ref{eqn_mellin_Rk}), it yields


%Due to the independence of the SNRs $\bs \gamma$ together with (\ref{eqn_mellin_Rk1}), the Mellin transform with respect to ${f_{\left. { G} \right|T}}(x|t)$ is thus derived as
%\begin{equation}\label{eqn_mellin_Rk}
%\phi \left( {\left. s \right|t} \right) = \prod\limits_{k = 1}^K {{\rm E}\left\{ {\left. {{{{{\left(1 + \gamma_k\right)}} }^{s - 1}}} \right|t} \right\}},
%\end{equation}


\section{Proof of Theorem \ref{the:cdf_pdf_corr_gam_shif}} \label{app:fox_H_fun}
%\subsection{Definition of Generalized Fox's H Function}
%\begin{mydef}As defined in \cite{yilmaz2010outage,chelli2013performance}, generalized Fox's H function is defined by Mellin-Barnes integral as follows
%\begin{equation}\label{eqn_fox_h_def}
%Y_{p,q}^{m,n}\left[ {\left. {\begin{array}{*{20}{c}}
%{\left( {{a_1},{\alpha _1},{ A_1},{\varphi _1}} \right), \cdots ,\left( {{a_p},{\alpha _p},{ A_p},{\varphi _p}} \right)}\\
%{\left( {{b_1},{\beta _1},{B_1},{\phi _1}} \right), \cdots ,\left( {{b_q},{\beta _q},{B_q},{\phi _q}} \right)}
%\end{array}} \right|x} \right] = \frac{1}{{2\pi i}}\int_{\mathcal L} {M_{p,q}^{m,n}\left[ s \right]{x^{ - s}}ds},
%\end{equation}
%where $M_{p,q}^{m,n}\left[ s \right]$ is written as
%\begin{multline}\label{eqn:def_M}
%M_{p,q}^{m,n}\left[ s \right] = \frac{{\prod\limits_{j = 1}^m {{B_j}^{{\phi _j} + {b_j} + {\beta _j}s - 1}\Psi \left( {{\phi _j},{\phi _j} + {b_j} + {\beta _j}s;{B_j}} \right)} }}{{\prod\limits_{i = n + 1}^p {{ A_i}^{{\varphi _i} + {a_i} + {\alpha _i}s - 1}\Psi \left( {{\varphi _i},{\varphi _i} + {a_i} + {\alpha _i}s;{ A_i}} \right)} }}\times \\
%\frac{{\prod\limits_{i = 1}^n {{ A_i}^{{\varphi _i} - {a_i} - {\alpha _i}s}\Psi \left( {{\varphi _i},{\varphi _i} + 1 - {a_i} - {\alpha _i}s;{ A_i}} \right)} }}{{\prod\limits_{j = m + 1}^q {{B_j}^{{\phi _j} - {b_j} - {\beta _j}s}\Psi \left( {{\phi _j},{\phi _j} + 1 - {b_j} - {\beta _j}s;{B_j}} \right)} }},
%\end{multline}
%and the contour $\mathcal{L}$ is specially choosen such that it is a Mellin-Barnes contour in the complex $s$-plane running from $c-\infty $ to $c+\infty$, $c \in \mathbb{R}$ and the points $s=(-b_j-k)/\beta_j$ for $1 \le j\le m$ and $k=0,1,2,\cdots$ and the points $s=(1-a_i+k)/\alpha_i$ for $1 \le i \le n$ and $k=0,1,2,\cdots$ lie to the left and right of the chosen contour, respectively.
%\end{mydef}

%\subsection{Properties of Generalized Fox's H function}
Based on (\ref{eqn_pdf_f_g_1_p}), the CDF of the product of multiple correlated shifted-Gamma RVs ${F_{ G_K}}\left( x \right)$ directly follows as (\ref{eqn:CDF_G_def_sec}) where the CDF of $\mathcal A_{\bs \ell}$ can be written based on (\ref{eqn:pdf_produc_gamma_ind_sec}) as
\begin{equation}\label{eqn:CDF_A_ell_def}
{F_{{{\cal A}_{\bs \ell} }}}(x) %&= \int_0^x {{f_{{{\cal A}_{\bs \ell} }}}(t)dt} \notag \\
 = \frac{1}{{\prod\nolimits_{k = 1}^K {{\Omega _k}} }}\int_0^x {Y_{0,K}^{K,0}\left[ {\left. {\begin{array}{*{20}{c}}
 - \\
{\left( {0,1,\frac{1}{{{\Omega _1}}},m + {\ell _1}} \right), \cdots ,\left( {0,1,\frac{1}{{{\Omega _K}}},m + {\ell _K}} \right)}
\end{array}} \right|\frac{t}{{\prod\nolimits_{k = 1}^K {{\Omega _k}} }}} \right]dt} .
\end{equation}
Regarding to the generalized Fox's H function in (\ref{eqn:CDF_A_ell_def}), we have the following properties.

%\subsubsection{property 1}
\begin{property}\label{prop:1}
\begin{multline}
{t^\rho }Y_{p,q}^{m,n}\left[ {\left. {\begin{array}{*{20}{c}}
{\left( {{a_1},{\alpha _1},{A_1},{\varphi _1}} \right), \cdots ,\left( {{a_p},{\alpha _p},{A_p},{\varphi _p}} \right)}\\
{\left( {{b_1},{\beta _1},{B_1},{\phi _1}} \right), \cdots ,\left( {{b_q},{\beta _q},{B_q},{\phi _q}} \right)}
\end{array}} \right|t} \right]\\
% = \frac{1}{{2\pi i}}\int_{\mathcal L} {\frac{{\prod\limits_{j = 1}^m {{B_j}^{{\phi _j} + {b_j} + {\beta _j}s - 1}\Psi \left( {{\phi _j},{\phi _j} + {b_j} + {\beta _j}s;{B_j}} \right)} }}{{\prod\limits_{i = n + 1}^p {{A_i}^{{\varphi _i} + {a_i} + {\alpha _i}s - 1}\Psi \left( {{\varphi _i},{\varphi _i} + {a_i} + {\alpha _i}s;{A_i}} \right)} }}\frac{{\prod\limits_{i = 1}^n {{A_i}^{{\varphi _i} - {a_i} - {\alpha _i}s}\Psi \left( {{\varphi _i},{\varphi _i} + 1 - {a_i} - {\alpha _i}s;{A_i}} \right)} }}{{\prod\limits_{j = m + 1}^q {{B_j}^{{\phi _j} - {b_j} - {\beta _j}s}\Psi \left( {{\phi _j},{\phi _j} + 1 - {b_j} - {\beta _j}s;{B_j}} \right)} }}{x^{ - s + \rho }}ds} \\
% = \frac{1}{{2\pi i}}\int_{\mathcal L} {\frac{{\prod\limits_{j = 1}^m {{B_j}^{{\phi _j} + {b_j} + \rho {\beta _j} + {\beta _j}s - 1}\Psi \left( {{\phi _j},{\phi _j} + {b_j} + \rho {\beta _j} + {\beta _j}s;{B_j}} \right)} }}{{\prod\limits_{i = n + 1}^p {{A_i}^{{\varphi _i} + {a_i} + \rho {\alpha _i} + {\alpha _i}s - 1}\Psi \left( {{\varphi _i},{\varphi _i} + {a_i} + \rho {\alpha _i} + {\alpha _i}s;{A_i}} \right)} }}\frac{{\prod\limits_{i = 1}^n {{A_i}^{{\varphi _i} - {a_i} - \rho {\alpha _i} - {\alpha _i}s}\Psi \left( {{\varphi _i},{\varphi _i} + 1 - {a_i} - \rho {\alpha _i} - {\alpha _i}s;{A_i}} \right)} }}{{\prod\limits_{j = m + 1}^q {{B_j}^{{\phi _j} - {b_j} - \rho {\beta _j} - {\beta _j}s}\Psi \left( {{\phi _j},{\phi _j} + 1 - {b_j} - \rho {\beta _j} - {\beta _j}s;{B_j}} \right)} }}{x^{ - s}}ds} \\
 = Y_{p,q}^{m,n}\left[ {\left. {\begin{array}{*{20}{c}}
{\left( {{a_1} + \rho {\alpha _1},{\alpha _1},{A_1},{\varphi _1}} \right), \cdots ,\left( {{a_p} + \rho {\alpha _q},{\alpha _p},{A_p},{\varphi _p}} \right)}\\
{\left( {{b_1} + \rho {\beta _1},{\beta _1},{B_1},{\phi _1}} \right), \cdots ,\left( {{b_q} + \rho {\beta _q},{\beta _q},{B_q},{\phi _q}} \right)}
\end{array}} \right|t} \right].
\end{multline}
\end{property}


%\subsubsection{property 2}
\begin{property}\label{prop:2}
\begin{multline}\label{eqn:x_1_fox_H}
Y_{p,q}^{m,n}\left[ {\left. {\begin{array}{*{20}{c}}
{\left( {{a_1},{\alpha _1},{A_1},{\varphi _1}} \right), \cdots ,\left( {{a_p},{\alpha _p},{A_p},{\varphi _p}} \right)}\\
{\left( {{b_1},{\beta _1},{B_1},{\phi _1}} \right), \cdots ,\left( {{b_q},{\beta _q},{B_q},{\phi _q}} \right)}
\end{array}} \right|t} \right]\\
% = \frac{1}{{2\pi i}}\int_{\mathcal L} {M_{p,q}^{m,n}\left[ { - s} \right]{x^s}ds}
%& = \frac{1}{{2\pi i}}\int_{\mathcal L} {\frac{{\prod\limits_{i = 1}^n {{A_i}^{{\varphi _i} - {a_i} + {\alpha _i}s}\Psi \left( {{\varphi _i},{\varphi _i} + 1 - {a_i} + {\alpha _i}s;{A_i}} \right)} }}{{\prod\limits_{j = m + 1}^q {{B_j}^{{\phi _j} - {b_j} + {\beta _j}s}\Psi \left( {{\phi _j},{\phi _j} + 1 - {b_j} + {\beta _j}s;{B_j}} \right)} }}\frac{{\prod\limits_{j = 1}^m {{B_j}^{{\phi _j} + {b_j} - {\beta _j}s - 1}\Psi \left( {{\phi _j},{\phi _j} + {b_j} - {\beta _j}s;{B_j}} \right)} }}{{\prod\limits_{i = n + 1}^p {{A_i}^{{\varphi _i} + {a_i} - {\alpha _i}s - 1}\Psi \left( {{\varphi _i},{\varphi _i} + {a_i} - {\alpha _i}s;{A_i}} \right)} }}{x^s}ds} \notag\\
 = Y_{q,p}^{n,m}\left[ {\left. {\begin{array}{*{20}{c}}
{\left( { 1 - {b_1},{\beta _1},{B_1},{\phi _1}} \right), \cdots ,\left( { 1 - {b_q},{\beta _q},{B_q},{\phi _q}} \right)}\\
{\left( { 1 - {a_1},{\alpha _1},{A_1},{\varphi _1}} \right), \cdots ,\left( { 1 - {a_p},{\alpha _p},{A_p},{\varphi _p}} \right)}
\end{array}} \right|t^{-1}} \right].
\end{multline}
\end{property}

Notice that Property \ref{prop:1} is similar to the property of Meijer-G function in \cite[Eq.9.31.5]{gradshteyn1965table} and the property of generalized upper incomplete Fox's H functions in \cite[Eq.A.10]{yilmaz2009productshifted}, while Property \ref{prop:2} is similar to the property of Meijer-G function in \cite[Eq.9.31.2]{gradshteyn1965table} and the property of generalized upper incomplete Fox's H functions in \cite[Eq.A.9]{yilmaz2009productshifted}. They can thus be proved using similar approaches in \cite{yilmaz2009productshifted} and \cite{gradshteyn1965table}.


By using Property \ref{prop:1}, (\ref{eqn:CDF_A_ell_def}) can be derived as
\begin{equation}\label{eqn:F_a_ell_step1}
{F_{{{\cal A}_{\bs \ell} }}}(x) = \int_0^x {{t^{ - 1}}Y_{0,K}^{K,0}\left[ {\left. {\begin{array}{*{20}{c}}
 - \\
{\left( {1,1,\frac{1}{{{\Omega _1}}},m + {\ell _1}} \right), \cdots ,\left( {1,1,\frac{1}{{{\Omega _K}}},m + {\ell _K}} \right)}
\end{array}} \right|\frac{t}{{\prod\nolimits_{k = 1}^K {{\Omega _k}} }}} \right]dt}.
\end{equation}

Then applying Property \ref{prop:2} into (\ref{eqn:F_a_ell_step1}) yields
\begin{equation}\label{eqn:F_a_ell_step2}
{F_{{{\cal A}_{\bs \ell} }}}(x) = \int_0^x {{t^{ - 1}}Y_{K,0}^{0,K}\left[ {\left. {\begin{array}{*{20}{c}}
{\left( {0,1,\frac{1}{{{\Omega _1}}},m + {\ell _1}} \right), \cdots ,\left( {0,1,\frac{1}{{{\Omega _K}}},m + {\ell _K}} \right)}\\
 -
\end{array}} \right|\frac{{\prod\nolimits_{k = 1}^K {{\Omega _k}} }}{t}} \right]dt}.
\end{equation}

With the definition of generalized Fox's H function \cite{yilmaz2010outage,chelli2013performance}, (\ref{eqn:F_a_ell_step2}) can be further derived as
\begin{align}\label{eqn:F_a_ell_step3}
{F_{{{\cal A}_{\bs \ell} }}}(x) &= \frac{1}{{\prod\nolimits_{k = 1}^K {{\Omega _k}} }}\frac{1}{{2\pi {\rm i}}}\int_{c_1-{\rm i}\infty}^{c_1-{\rm i}\infty} {\int_0^x {{{\left( {\frac{t}{{\prod\nolimits_{k = 1}^K {{\Omega _k}} }}} \right)}^{s - 1}}M_{K,0}^{0,K}\left( s \right)dt} ds} \notag\\
 %&= \frac{1}{{2\pi {\rm i}}}\int_{c_1-{\rm i}\infty}^{c_1-{\rm i}\infty} {\frac{1}{s}{{\left( {\frac{x}{{\prod\limits_{k = 1}^K {{\Omega _k}} }}} \right)}^s}M_{K,0}^{0,K}\left( s \right)ds} \notag\\
 &= \frac{1}{{2\pi {\rm i}}}\int_{c_1-{\rm i}\infty}^{c_1-{\rm i}\infty} {{{\left( {\frac{x}{{\prod\nolimits_{k = 1}^K {{\Omega _k}} }}} \right)}^s}\frac{{\Gamma \left( s \right)}}{{\Gamma \left( {s + 1} \right)}}M_{K,0}^{0,K}\left( s \right)ds},
\end{align}
where $M_{K,0}^{0,K}\left( s \right) = \frac{1}{{\prod\nolimits_{j = 1}^K {\Xi \left( {0,1,\frac{1}{{{\Omega _j}}},m + {\ell _j}} \right)} }}$ %$M_{K,0}^{0,K}\left( s \right)$ is given as
%\begin{equation}\label{eqn:M_def}
%M_{K,0}^{0,K}\left( s \right) = \frac{1}{{\prod\limits_{j = 1}^K {\Xi \left( {0,1,\frac{1}{{{\Omega _j}}},m + {\ell _j}} \right)} }}.
%\end{equation}
%and $\Xi (a,\alpha,A,\varphi)$ is defined as
and $\Xi \left( {a,\alpha ,A,\varphi } \right) = {A^{\varphi  + a + \alpha s - 1}}\Psi \left( {\varphi ,\varphi  + a + \alpha s;A} \right)$.
%\begin{equation}\label{eqn:Xi_def}
%\Xi \left( {a,\alpha ,A,\varphi } \right) = {A^{\varphi  + a + \alpha s - 1}}\Psi \left( {\varphi ,\varphi  + a + \alpha s;A} \right).
%\end{equation}
The function $\Xi \left( {a,\alpha ,A,\varphi } \right)$ can be rewritten by using one important property of Tricomi's confluent hypergeometric function $\Psi \left( {\alpha ,\gamma ;z} \right)$ in \cite[Eq.9.210.2]{gradshteyn1965table} as
\begin{multline}\label{eqn:Xi_rew}
\Xi \left( {a,\alpha ,A,\varphi } \right) = {A^{\varphi  + a + \alpha s - 1}}\frac{{\Gamma \left( {1 - \left( {\varphi  + a + \alpha s} \right)} \right)}}{{\Gamma \left( {1 - a - \alpha s} \right)}}{}_1{F_1}\left( {\varphi ,\varphi  + a + \alpha s;A} \right)\\
 + \frac{{\Gamma \left( {\varphi  + a + \alpha s - 1} \right)}}{{\Gamma \left( \varphi  \right)}}{}_1{F_1}\left( {1 - a - \alpha s,2 - \left( {\varphi  + a + \alpha s} \right);A} \right),
\end{multline}
where ${}_{1}F_{1}(\cdot)$ represents the confluent hypergeometric function \cite[Eq.9.210.1]{gradshteyn1965table}. Thus it follows that
\begin{equation}\label{eqn:gamma__plus_s}
\Xi \left( {a,1,0,1} \right)= \Gamma \left( {a + s} \right) = \mathop {\lim }\limits_{u \to 0} {u^{a + s}}\Psi \left( {1,1 + a + s;u} \right), \, a+s > 0.
\end{equation}

By using (\ref{eqn:gamma__plus_s}) together with the definition of generalized Fox's H function, (\ref{eqn:F_a_ell_step3}) can be further rewritten as
\begin{multline}\label{eqn:use_integral_inv_fox_H_1}
{F_{{{\cal A}_{\bs \ell} }}}(x) = \mathop {\lim }\limits_{u \to 0} \frac{1}{{2\pi {\rm i}}}\int_{c_1-{\rm i}\infty}^{c_1+{\rm i}\infty} {{{\left( {\frac{x}{{\prod\nolimits_{k = 1}^K {{\Omega _k}} }}} \right)}^s}\frac{{{u^s}\Psi \left( {1,1 + s;u} \right)}}{{{u^{1 + s}}\Psi \left( {1,2 + s;u} \right)\prod\nolimits_{j = 1}^K {\Xi \left( {0,1,\frac{1}{{{\Omega _j}}},m + {\ell _j}} \right)} }}ds} \\
 = Y_{K + 1,1}^{1,K}\left[ {\left. {\begin{array}{*{20}{c}}
{\left( {0,1,\frac{1}{{{\Omega _1}}},m + {\ell _1}} \right), \cdots ,\left( {0,1,\frac{1}{{{\Omega _K}}},m + {\ell _K}} \right),\left( {1,1,0,1} \right)}\\
{\left( {0,1,0,1} \right)}
\end{array}} \right|\frac{{\prod\nolimits_{k = 1}^K {{\Omega _k}} }}{x}} \right].
\end{multline}
Finally, applying Property \ref{prop:2} to (\ref{eqn:use_integral_inv_fox_H_1}) yields (\ref{eqn:CDF_F_A_def_sec}).

%and
%step (a) holds by using the property 1, steps (b) and (d) hold by using the property 2, and step (c) holds by using , such that
%\begin{align}\label{eqn:gamma_def_M}
%\mathop {\lim }\limits_{u \to 0} {u^{a  + s}}\Psi \left( {1 ,1  + a + s;u} \right)
% &= \mathop {\lim }\limits_{u \to 0} {u^{a + \alpha  + s - 1}} \left( \begin{array}{l}
%\frac{{\Gamma \left( {1 - \alpha  - a - s} \right)}}{{\Gamma \left( {1 - a - s} \right)}}{}_1{F_1}\left( {\alpha ,\alpha  + a + s;u} \right) \notag\\
% + \frac{{\Gamma \left( {\alpha  + a + s - 1} \right)}}{{{u^{1 - \alpha  - a - s}}\Gamma \left( \alpha  \right)}}{}_1{F_1}\left( {1 - a - s,2 - \alpha  - a - s;u} \right)
%\end{array} \right) \notag\\
% &= \mathop {\lim }\limits_{u \to 0} \left( {\frac{{\Gamma \left( {1 - \alpha  - a - s} \right)}}{{\Gamma \left( {1 - a - s} \right)}}{u^{a + \alpha  + s - 1}} + \frac{{\Gamma \left( {\alpha  + a + s - 1} \right)}}{{\Gamma \left( \alpha  \right)}}} \right) \notag\\
% &= {\Gamma \left( {a + s} \right)};\, a  + s > 0.
%\end{align}
%%Specifically, if $\alpha = 1, a = 1$, then we have
%\begin{equation}
%  \mathop {\lim }\limits_{u \to 0} {u^{a + \alpha  + s - 1}}\Psi \left( {\alpha ,\alpha  + a + s;u} \right) = \Gamma \left( {s + 1} \right),
%\end{equation}
%and if $\alpha = 1, a=0$, then it follows that
%\begin{equation}
%  \mathop {\lim }\limits_{u \to 0} {u^{a + \alpha  + s - 1}} \Psi \left( {\alpha ,\alpha  + a + s;u} \right) = \Gamma \left( s \right).
%\end{equation}


%\subsubsection{property 1}
%Similar to the property of Meijer-G function \cite[Eq. 9.31.5]{gradshteyn1965table}, it follows that
%\begin{equation}
%\begin{array}{l}
%{x^\rho }Y_{p,q}^{m,n}\left[ {\left. {\begin{array}{*{20}{c}}
%{\left( {{a_1},{\alpha _1},{A_1},{\varphi _1}} \right), \cdots ,\left( {{a_p},{\alpha _p},{A_p},{\varphi _p}} \right)}\\
%{\left( {{b_1},{\beta _1},{B_1},{\phi _1}} \right), \cdots ,\left( {{b_q},{\beta _q},{B_q},{\phi _q}} \right)}
%\end{array}} \right|x} \right]\\
%% = \frac{1}{{2\pi i}}\int_{\mathcal L} {\frac{{\prod\limits_{j = 1}^m {{B_j}^{{\phi _j} + {b_j} + {\beta _j}s - 1}\Psi \left( {{\phi _j},{\phi _j} + {b_j} + {\beta _j}s;{B_j}} \right)} }}{{\prod\limits_{i = n + 1}^p {{A_i}^{{\varphi _i} + {a_i} + {\alpha _i}s - 1}\Psi \left( {{\varphi _i},{\varphi _i} + {a_i} + {\alpha _i}s;{A_i}} \right)} }}\frac{{\prod\limits_{i = 1}^n {{A_i}^{{\varphi _i} - {a_i} - {\alpha _i}s}\Psi \left( {{\varphi _i},{\varphi _i} + 1 - {a_i} - {\alpha _i}s;{A_i}} \right)} }}{{\prod\limits_{j = m + 1}^q {{B_j}^{{\phi _j} - {b_j} - {\beta _j}s}\Psi \left( {{\phi _j},{\phi _j} + 1 - {b_j} - {\beta _j}s;{B_j}} \right)} }}{x^{ - s + \rho }}ds} \\
%% = \frac{1}{{2\pi i}}\int_{\mathcal L} {\frac{{\prod\limits_{j = 1}^m {{B_j}^{{\phi _j} + {b_j} + \rho {\beta _j} + {\beta _j}s - 1}\Psi \left( {{\phi _j},{\phi _j} + {b_j} + \rho {\beta _j} + {\beta _j}s;{B_j}} \right)} }}{{\prod\limits_{i = n + 1}^p {{A_i}^{{\varphi _i} + {a_i} + \rho {\alpha _i} + {\alpha _i}s - 1}\Psi \left( {{\varphi _i},{\varphi _i} + {a_i} + \rho {\alpha _i} + {\alpha _i}s;{A_i}} \right)} }}\frac{{\prod\limits_{i = 1}^n {{A_i}^{{\varphi _i} - {a_i} - \rho {\alpha _i} - {\alpha _i}s}\Psi \left( {{\varphi _i},{\varphi _i} + 1 - {a_i} - \rho {\alpha _i} - {\alpha _i}s;{A_i}} \right)} }}{{\prod\limits_{j = m + 1}^q {{B_j}^{{\phi _j} - {b_j} - \rho {\beta _j} - {\beta _j}s}\Psi \left( {{\phi _j},{\phi _j} + 1 - {b_j} - \rho {\beta _j} - {\beta _j}s;{B_j}} \right)} }}{x^{ - s}}ds} \\
% = Y_{p,q}^{m,n}\left[ {\left. {\begin{array}{*{20}{c}}
%{\left( {{a_1} + \rho {\alpha _1},{\alpha _1},{A_1},{\varphi _1}} \right), \cdots ,\left( {{a_p} + \rho {\alpha _q},{\alpha _p},{A_p},{\varphi _p}} \right)}\\
%{\left( {{b_1} + \rho {\beta _1},{\beta _1},{B_1},{\phi _1}} \right), \cdots ,\left( {{b_q} + \rho {\beta _q},{\beta _q},{B_q},{\phi _q}} \right)}
%\end{array}} \right|x} \right]
%\end{array}
%\end{equation}
%
%\subsubsection{property 2}
%Similar to the property of Meijer-G function \cite[Eq. 9.31.2]{gradshteyn1965table}, we have
%\begin{equation}\label{eqn:x_1_fox_H}
%\begin{array}{l}
%Y_{p,q}^{m,n}\left[ {\left. {\begin{array}{*{20}{c}}
%{\left( {{a_1},{\alpha _1},{A_1},{\varphi _1}} \right), \cdots ,\left( {{a_p},{\alpha _p},{A_p},{\varphi _p}} \right)}\\
%{\left( {{b_1},{\beta _1},{B_1},{\phi _1}} \right), \cdots ,\left( {{b_q},{\beta _q},{B_q},{\phi _q}} \right)}
%\end{array}} \right|x} \right] %= \frac{1}{{2\pi i}}\int_{\mathcal L} {M_{p,q}^{m,n}\left[ { - s} \right]{x^s}ds}
%\\
%% = \frac{1}{{2\pi i}}\int_{\mathcal L} {\frac{{\prod\limits_{i = 1}^n {{A_i}^{{\varphi _i} - {a_i} + {\alpha _i}s}\Psi \left( {{\varphi _i},{\varphi _i} + 1 - {a_i} + {\alpha _i}s;{A_i}} \right)} }}{{\prod\limits_{j = m + 1}^q {{B_j}^{{\phi _j} - {b_j} + {\beta _j}s}\Psi \left( {{\phi _j},{\phi _j} + 1 - {b_j} + {\beta _j}s;{B_j}} \right)} }}\frac{{\prod\limits_{j = 1}^m {{B_j}^{{\phi _j} + {b_j} - {\beta _j}s - 1}\Psi \left( {{\phi _j},{\phi _j} + {b_j} - {\beta _j}s;{B_j}} \right)} }}{{\prod\limits_{i = n + 1}^p {{A_i}^{{\varphi _i} + {a_i} - {\alpha _i}s - 1}\Psi \left( {{\varphi _i},{\varphi _i} + {a_i} - {\alpha _i}s;{A_i}} \right)} }}{x^s}ds} \\
% = Y_{q,p}^{n,m}\left[ {\left. {\begin{array}{*{20}{c}}
%{\left( { 1 - {b_1},{\beta _1},{B_1},{\phi _1}} \right), \cdots ,\left( { 1 - {b_q},{\beta _q},{B_q},{\phi _q}} \right)}\\
%{\left( { 1 - {a_1},{\alpha _1},{A_1},{\varphi _1}} \right), \cdots ,\left( { 1 - {a_p},{\alpha _p},{A_p},{\varphi _p}} \right)}
%\end{array}} \right|\frac{1}{x}} \right]
%\end{array}
%\end{equation}
\section{Proof of (\ref{eqn:trun_err_fina_hyp})}\label{app:uperb_err}
Following from (\ref{eqn:trun_err_prod_gam}), the truncation error $\nabla(N)$ is upper bounded as
\begin{align}\label{eqn:trun_err_upp}
\nabla(N) &= \sum\nolimits_{n = N + 1}^\infty  {\sum\nolimits_{{\sum\nolimits_{k=1}^K  \ell_k = n}}  {{W_{\bs{\ell}}}{F_{{ {\cal A}_{\bs{\ell}}}}}\left( 2^{\cal R} \right)} }\notag\\
 &\le {{F_{{ {\cal A}_{\bs{\ell}}},N}^{\rm max}}\left( 2^{\cal R} \right)} \sum\nolimits_{n = N + 1}^\infty  {\sum\nolimits_{{{\sum\nolimits_{k=1}^K  \ell_k = n}}} {{W_{\bs{\ell}}}} }\notag\\
& %\notag \\
 =% {{F_{{ {\cal A}_{\bs{\ell}}},N}^{\rm max}}\left( 2^{\cal R} \right)} {W_{\bf{0}}}\sum\limits_{n = N + 1}^\infty  {{{\left( m \right)}_n}\sum\limits_{{{\sum\nolimits_{k=1}^K  \ell_k = n}}} {\prod\limits_{k = 1}^K {\frac{{{{\left( {{w_k}} \right)}^{{\ell_k}}}}}{{{\ell_k}!}}} } },
 {{F_{{ {\cal A}_{\bs{\ell}}},N}^{\rm max}}\left( 2^{\cal R} \right)} {W_{\bf{0}}}\sum\nolimits_{n = N + 1}^\infty  {\frac{{{{\left( m \right)}_n}}}{{n!}}\sum\nolimits_{{\sum\nolimits_{k=1}^K  \ell_k = n}} {n!\prod\nolimits_{k = 1}^K {\frac{{{{\left( {{w_k}} \right)}^{{\ell_k}}}}}{{{\ell_k}!}}} } },
\end{align}
where ${{F_{{ {\cal A}_{\bs{\ell}}},N}^{\rm max}}\left( 2^{\cal R} \right)} = \mathop {\max }\nolimits_{\sum\nolimits_{k=1}^K  \ell_k \ge N + 1} \left( {{F_{{ {\cal A}_{\bs{\ell}}}}}\left( 2^{\cal R} \right)} \right)$ and $(\cdot)_n$ denotes Pochhammer symbol. Herein, it is readily found that $\mathop {\max }\nolimits_{\sum\nolimits_{k=1}^K  \ell_k = N_1} \left( {{F_{{ {\cal A}_{\bs{\ell}}}}}\left( 2^{\cal R} \right)} \right) \ge \mathop {\max }\nolimits_{\sum\nolimits_{k=1}^K  \ell_k = N_2} \left( {{F_{{ {\cal A}_{\bs{\ell}}}}}\left( 2^{\cal R} \right)} \right)$ if $N_1 < N_2$ by the physical interpretation of ${{F_{{ {\cal A}_{\bs{\ell}}}} \left( 2^{\cal R} \right)}}$ in Section \ref{sec:exa}. Thus we have ${{F_{{ {\cal A}_{\bs{\ell}}},N}^{\rm max}}\left( 2^{\cal R} \right)} = \mathop {\max }\nolimits_{\sum\nolimits_{k=1}^K  \ell_k = N + 1} \left( {{F_{{ {\cal A}_{\bs{\ell}}}}}\left( 2^{\cal R} \right)} \right)$. By means of multinomial theorem \cite[Eq.26.4.9]{NIST:DLMF}, it yields
\begin{align}\label{eqn:trun_erro_mul_exp}
\nabla(N)  %&\le {{F_{{ {\cal A}_{\bs{\ell}}},N}^{\rm max}}\left( 2^{\cal R} \right)} {W_{\bf{0}}}\sum\limits_{k = N + 1}^\infty  {\frac{{{{\left( m \right)}_n}}}{{n!}}\sum\limits_{{\sum\nolimits_{k=1}^K  \ell_k = n}} {n!\prod\limits_{k = 1}^K {\frac{{{{\left( {{w_k}} \right)}^{{\ell_k}}}}}{{{\ell_k}!}}} } } \notag \\
 &\le {W_{\bf{0}}}{{F_{{ {\cal A}_{\bs{\ell}}},N}^{\rm max}}\left( 2^{\cal R} \right)} \sum\nolimits_{n = N + 1}^\infty  {\frac{{{{\left( m \right)}_n}}}{{n!}}{{\left( {\sum\nolimits_{k = 1}^K {{w_k}} } \right)}^n}}.
\end{align}
Reformulating the summation term in the right hand side of (\ref{eqn:trun_erro_mul_exp}) in terms of hypergeometric function ${}_2F_1(\cdot)$ [29,Eq.15.1.1], the truncation error $\nabla(N)$ is consequently upper bounded as (\ref{eqn:trun_err_fina_hyp}).
%\begin{align}\label{eqn:trun_err_fina_hypapp}
%\varepsilon  &\le {{F_{{ {\cal A}_{\bs{\ell}}},N}^{\rm max}}\left( 2^{\cal R} \right)} {W_{\bf{0}}}\sum\limits_{n = 0}^\infty  {\frac{{{{\left( m \right)}_{n + N + 1}}}}{{\left( {n + N + 1} \right)!}}{{\left( {\sum\limits_{k = 1}^K {{w_k}} } \right)}^{n + N + 1}}} \notag \\
% &= {{F_{{ {\cal A}_{\bs{\ell}}},N}^{\rm max}}\left( 2^{\cal R} \right)} {W_{\bf{0}}}{\left( {\sum\limits_{k = 1}^K {{w_k}} } \right)^{N + 1}}\frac{{{{\left( m \right)}_{N + 1}}}}{{\left( {N + 1} \right)!}}\sum\limits_{n = 0}^\infty  {\frac{{{{\left( {m + N + 1} \right)}_n}{{\left( 1 \right)}_n}}}{{{{\left( {N + 2} \right)}_n}}}\frac{{{{\left( {\sum\limits_{k = 1}^K {{w_k}} } \right)}^n}}}{{n!}}}. %\notag \\
% %&= {{F_{{ {\cal A}_{\bs{\ell}}},N}^{\rm max}}\left( 2^{\cal R} \right)} {W_{\bf{0}}}{\left( {\sum\limits_{k = 1}^K {{w_k}} } \right)^{N + 1}}\frac{{{{\left( m \right)}_{N + 1}}}}{{\left( {N + 1} \right)!}}{}_2{F_1}\left( {m + N + 1,1;N + 2;\sum\limits_{k = 1}^K {{w_k}} } \right).
%% \label{eqn:trun_err_fina_hyp1}
%% &\le {W_{\bf{0}}}{\left( {\sum\limits_{k = 1}^K {{w_k}} } \right)^{N + 1}}\frac{{{{\left( m \right)}_{N + 1}}}}{{\left( {N + 1} \right)!}}{}_2{F_1}\left( {m + N + 1,1;N + 2;\sum\limits_{k = 1}^K {{w_k}} } \right) \le 1.
%\end{align}
\section{Proof of Theorem \ref{lemma:lemma_gener_fox_H}}\label{app:lemma_gener_fox_H}
%\subsection{Proof of (\ref{eqn:der_A_1_CDF}) and (\ref{eqn:sec_def_g_fun})}
Recalling from Theorem \ref{the:cdf_pdf_corr_gam_shif} that ${F_{{{ {\mathcal A}}_{\bs{\ell}}}}}(x)$ is the CDF of the product of multiple independent shifted-Gamma RVs, i.e., $\mathcal A_{\bs{\ell}} = \prod\nolimits_{k = 1}^K (1+R_{{\bs{\ell}},k})$ and $R_{{\bs{\ell}},k} \sim \mathcal {G}(m+\ell_k,{\Omega _k})$, it can be expressed as
\begin{equation}\label{eqn:F_hat_A_1}
{F_{{{ {\mathcal A}}_{\bs{\ell}}}}}(x) = \Pr \left( {\prod\nolimits_{k = 1}^K {\left( {1 + {R_{{\bs{\ell}},k}}} \right)}  \le x} \right) %\notag\\
% &= \int\nolimits_0^{x - 1} { \cdots \int\nolimits_0^{x\prod\limits_{k = 1}^{K - 2} {{{\left( {1 + {t_k}} \right)}^{ - 1}}}  - 1} {\int\nolimits_0^{x\prod\limits_{k = 1}^{K - 1} {{{\left( {1 + {t_k}} \right)}^{ - 1}}}  - 1} {{f_{\bf{R}}}\left( {{r_1},{r_2}, \cdots ,{r_K}} \right)d{t_1} \cdots d{t_{K - 1}}d{t_K}} } } \notag\\
% &= \int\nolimits_0^{x - 1} { \cdots \int\limits_0^{x\prod\limits_{k = 1}^{K - 2} {{{\left( {1 + {t_k}} \right)}^{ - 1}}}  - 1} {\int\limits_0^{x\prod\limits_{k = 1}^{K - 1} {{{\left( {1 + {t_k}} \right)}^{ - 1}}}  - 1} {\prod\limits_{k = 1}^K {\frac{{{t_k}^{m + {\ell_k} - 1}{e^{ - \frac{{{t_k}}}{{{\Omega _k}}}}}}}{{{{\left( {{\Omega _k}} \right)}^{m + {\ell_k}}}\Gamma \left( {m + {\ell_k}} \right)}}} d{t_1} \cdots d{t_{K - 1}}d{t_K}} } }.
=\int\nolimits_{\prod\nolimits_{k = 1}^K {\left( {1 + {t_k}} \right)}  \le x} {\prod\nolimits_{k = 1}^K {\frac{{{t_k}^{m + {\ell_k} - 1}{e^{ - \frac{{{t_k}}}{{{\Omega _k}}}}}}}{{{{\left( {{\Omega _k}} \right)}^{m + {\ell_k}}}\Gamma \left( {m + {\ell_k}} \right)}}} d{t_1} \cdots d{t_{K - 1}}d{t_K}}.
\end{equation}
By using Maclaurin series of exponential function, ${F_{{{ {\mathcal A}}_{\bs{\ell}}}}}(x)$ in (\ref{eqn:F_hat_A_1}) can then be derived as
\begin{align}\label{eqn:F_hat_A_1_fur}
{F_{{{ {\mathcal A}}_{\bs{\ell}}}}}(x)
&= \prod\nolimits_{k = 1}^K {\frac{{\frac{1}{{{{\left( {{\Omega _k}} \right)}^{m + {\ell_k}}}}}}}{{\Gamma \left( {m + {\ell_k}} \right)}}}
%&\times \int\limits_0^{x - 1} { \cdots \int\limits_0^{x\prod\limits_{k = 1}^{K - 2} {{{\left( {1 + {t_k}} \right)}^{ - 1}}}  - 1} {\int\limits_0^{x\prod\limits_{k = 1}^{K - 1} {{{\left( {1 + {t_k}} \right)}^{ - 1}}}  - 1} {\prod\limits_{k = 1}^K {{t_k}^{m + {\ell_k} - 1}\sum\limits_{{n_1}, \cdots ,{n_K} = 0}^\infty  {\frac{{{{\left( { - \frac{{{t_k}}}{{{\Omega _k}}}} \right)}^{{n_k}}}}}{{{n_k}!}}} } d{t_1} \cdots d{t_{K - 1}}d{t_K}} } } \notag\\
 \int\nolimits_{\prod\nolimits_{k = 1}^K {\left( {1 + {t_k}} \right)}  \le x} {\prod\nolimits_{k = 1}^K {{t_k}^{m + {\ell_k} - 1}\sum\nolimits_{{n_1}, \cdots ,{n_K} = 0}^\infty  {\frac{{{{\left( { - \frac{{{t_k}}}{{{\Omega _k}}}} \right)}^{{n_k}}}}}{{{n_k}!}}} } d{t_1} \cdots d{t_{K - 1}}d{t_K}}  \notag\\
 &= \prod\nolimits_{k = 1}^K {\frac{1}{{{{\left( {{\Omega _k}} \right)}^{m + {\ell_k}}}}}} \sum\nolimits_{{n_1}, \cdots ,{n_K} = 0}^\infty  {\prod\nolimits_{k = 1}^K {\frac{1}{{\Gamma \left( {m + {\ell_k}} \right){n_k}!}}{{\left( { - \frac{1}{{{\Omega _k}}}} \right)}^{{n_k}}}} } {g_{{\bf{n}} + {\bs{\ell}}}}\left( x \right),
\end{align}
where ${\bf n} = (n_1,\cdots,n_K)$ and ${g_{{\bs{\ell}}}}\left( x \right)$ is given by (\ref{eqn:sec_def_g_fun}). After putting $\Omega _k = \frac{{P_k}{\sigma _k}^2\left( {1 - {\lambda _k}^2} \right)}{m\mathcal N_0}$ and (\ref{eqn:gamma_snr_rel}) into (\ref{eqn:F_hat_A_1_fur}), the CDF expression in (\ref{eqn:der_A_1_CDF}) directly follows.
%\begin{align}\label{eqn:inf_F_A_hat_ser}
%{F_{{{\mathcal A}_{\bs{\ell}}}}}(x) &= \prod\limits_{k = 1}^K {{{\left( {\frac{m}{{\gamma {\theta _k}{\sigma _k}^2\left( {1 - {\lambda _k}^2} \right)}}} \right)}^{m + {\ell_k}}}} \sum\limits_{{n_1}, \cdots ,{n_K} = 0}^\infty  {\prod\limits_{k = 1}^K {{{\left( { - \frac{m}{{\gamma {\theta _k}{\sigma _k}^2\left( {1 - {\lambda _k}^2} \right)}}} \right)}^{{n_k}}}} } {g_{{\bf{n}},{\bs{\ell}}}}\left( x \right)\notag\\
% &=  c\left( {x,\gamma ,{\boldsymbol \lambda},{\bs{\ell}},m,K,{\bs{\theta }}} \right){\gamma ^{ - {d_{{{ \mathcal A}_{\bs{\ell}}}}}}},
%\end{align}
%where ${d_{{{\mathcal A}_{\bs{\ell}}}}} = \sum\limits_{k = 1}^K {\left( {m + {\ell_k}} \right)} = mK + \sum\limits_{k = 1}^K {{{\ell_k}} } $ and $ c\left( {x,\gamma ,\bs{\lambda },{\bs{\ell}},m,K,{\bf{\theta }}} \right)$ is given by (\ref{eqn:hat_c_def}).
%
%
%$ c\left( {x,\gamma ,{\boldsymbol \lambda},{\bs{\ell}},m,K,{\bs{\theta }}} \right)$ is given as
%\begin{multline}\label{eqn:hat_c_def}
% c\left( {x,\gamma ,{\boldsymbol \lambda},{\bs{\ell}},m,K,{\bs{\theta }}} \right) \\
% = \prod\limits_{k = 1}^K {{{\left( {\frac{m}{{{\theta _k}{\sigma _k}^2\left( {1 - {\lambda _k}^2} \right)}}} \right)}^{m + {\ell_k}}}} \left( {{g_{{\bf{0}},{\bs{\ell}}}}\left( x \right) + \sum\limits_{\sum\nolimits_{k = 1}^K {{n_k}}  > 0}^\infty  {\prod\limits_{k = 1}^K {{{\left( { - \frac{m}{{\gamma {\theta _k}{\sigma _k}^2\left( {1 - {\lambda _k}^2} \right)}}} \right)}^{{n_k}}}} } {g_{{\bf{n}},{\bs{\ell}}}}\left( x \right)} \right) \\
% = \prod\limits_{k = 1}^K {{{\left( {\frac{m}{{{\theta _k}{\sigma _k}^2\left( {1 - {\lambda _k}^2} \right)}}} \right)}^{m + {\ell_k}}}} {g_{{\bf{0}},{\bs{\ell}}}}\left( x \right) + o\left( 1 \right),
%\end{multline}
%\begin{align}\label{eqn:def_c_hat_app}
% c\left( {x,\gamma ,\left|{\boldsymbol \lambda}\right|,{\bs{\ell}},m,K,{\bs{\theta }}} \right) &= \prod\limits_{k = 1}^K {{{\left( {\frac{m}{{{\theta _k}{\sigma _k}^2\left( {1 - {\lambda _k}^2} \right)}}} \right)}^{m + {\ell_k}}}} \left( {{g_{{\bf{0}},{\bs{\ell}}}}\left( x \right) + \sum\limits_{\sum\limits_{k = 1}^K {{n_k}}  > 0}^\infty  {\prod\limits_{k = 1}^K {{{\left( { - \frac{m}{{\gamma {\theta _k}{\sigma _k}^2\left( {1 - {\lambda _k}^2} \right)}}} \right)}^{{n_k}}}} } {g_{{\bf{n}},{\bs{\ell}}}}\left( x \right)} \right) \notag\\
% &= \prod\limits_{k = 1}^K {{{\left( {\frac{m}{{{\theta _k}{\sigma _k}^2\left( {1 - {\lambda _k}^2} \right)}}} \right)}^{m + {\ell_k}}}} {g_{{\bf{0}},{\bs{\ell}}}}\left( x \right) + o\left( {{\gamma ^{ - \kappa }}} \right), \,\kappa  < 1
%\end{align}



%\begin{align}\label{eqn:def_g}
%{g_{{\bf{n}},{\bs{\ell}}}}\left( x \right) &= \prod\limits_{k = 1}^K {\frac{1}{{\Gamma \left( {m + {\ell_k}} \right){n_k}!}}} \int\limits_0^{x - 1} { \cdots \int\limits_0^{x\prod\limits_{k = 1}^{K - 2} {{{\left( {1 + {t_k}} \right)}^{ - 1}}}  - 1} {\int\limits_0^{x\prod\limits_{k = 1}^{K - 1} {{{\left( {1 + {t_k}} \right)}^{ - 1}}}  - 1} {\prod\limits_{k = 1}^K {{t_k}^{m + {\ell_k} + {n_k} - 1}} d{t_1} \cdots d{t_{K - 1}}d{t_K}} } } \notag \\
% &\le \prod\limits_{k = 1}^K {\frac{1}{{\Gamma \left( {m + {\ell_k}} \right){n_k}!}}} \int\limits_0^{x - 1} { \cdots \int\limits_0^{x - 1} {\int\limits_0^{x - 1} {\prod\limits_{k = 1}^K {{t_k}^{m + {\ell_k} + {n_k} - 1}} d{t_1} \cdots d{t_{K - 1}}d{t_K}} } } \notag\\
% &= \prod\limits_{k = 1}^K {\frac{1}{{\Gamma \left( {m + {\ell_k}} \right){n_k}!}}\frac{{{{\left( {x - 1} \right)}^{m + {\ell_k} + {n_k}}}}}{{m + {\ell_k} + {n_k}}}} \notag \\
% &\le \prod\limits_{k = 1}^K {\frac{{{{\left( {x - 1} \right)}^{m + {\ell_k} + {n_k}}}}}{{\Gamma \left( {m + {\ell_k}} \right){n_k}!}}}
%\end{align}
%
%wherein the last step holds since
%\begin{align}\label{eqn:infi_equ_F_hat}
% \sum\limits_{\sum\limits_{k = 1}^K {{n_k}}  > 0}^\infty  {\prod\limits_{k = 1}^K {{{\left( { - \frac{m}{{\gamma {\theta _k}{\sigma _k}^2\left( {1 - {\lambda _k}^2} \right)}}} \right)}^{{n_k}}}} } {g_{{\bf{n}},{\bs{\ell}}}}\left( x \right)&= \sum\limits_{S = 1}^\infty  {{\gamma ^{ - S}}\sum\limits_{\sum\limits_{k = 1}^K {{n_k}}  = S} {\prod\limits_{k = 1}^K {{{\left( { - \frac{m}{{{\theta _k}{\sigma _k}^2\left( {1 - {\lambda _k}^2} \right)}}} \right)}^{{n_k}}}} {g_{{\bf{n}},{\bs{\ell}}}}\left( x \right)} } \notag \\
% &\le \sum\limits_{S = 1}^\infty  {{\gamma ^{ - S}}\sum\limits_{\sum\limits_{k = 1}^K {{n_k}}  = S} {\prod\limits_{k = 1}^K {{{\left( {\frac{m}{{{\theta _k}{\sigma _k}^2\left( {1 - {\lambda _k}^2} \right)}}} \right)}^{{n_k}}}} {g_{{\bf{n}},{\bs{\ell}}}}\left( x \right)} } \notag \\
% &\mathop  \le \limits^{\left( 1 \right)}  \frac{{{{\left( {x - 1} \right)}^{\sum\limits_{k = 1}^K {\left( {m + {\ell_k}} \right)} }}}}{{\prod\limits_{k = 1}^K {\Gamma \left( {m + {\ell_k}} \right)} }}\sum\limits_{S = 1}^\infty  {{\gamma ^{ - S}}{{\left( {x - 1} \right)}^S}\sum\limits_{\sum\limits_{k = 1}^K {{n_k}}  = S} {\prod\limits_{k = 1}^K {\frac{{{{\left( {\frac{m}{{{\theta _k}{\sigma _k}^2\left( {1 - {\lambda _k}^2} \right)}}} \right)}^{{n_k}}}}}{{{n_k}!}}} } } \notag \\
% &\mathop  \le \limits^{\left( 2 \right)} \frac{{{{\left( {x - 1} \right)}^{\sum\limits_{k = 1}^K {\left( {m + {\ell_k}} \right)} }}}}{{\prod\limits_{k = 1}^K {\Gamma \left( {m + {\ell_k}} \right)} }}\sum\limits_{S = 1}^\infty  {\frac{{{{\left( {\left( {x - 1} \right)\sum\limits_{k = 1}^K {\frac{m}{{{\theta _k}{\sigma _k}^2\left( {1 - {\lambda _k}^2} \right)}}} } \right)}^S}}}{{S!}}{\gamma ^{ - S}}} \notag \\
% & = o\left( {{\gamma ^{ - \kappa }}} \right), \quad \kappa < 1.
%\end{align}
%and the step (1) holds by using (\ref{eqn:def_g}), and the step (2) holds by using multinomial expansion. Similar to (\ref{eqn:c_app_der_asym}), it follows that
%\begin{align}\label{eqn:c_ratio_ln_gamma}
%\mathop {\lim }\limits_{\gamma  \to \infty } \frac{{\log \left( { c\left( {x,\gamma ,\left|{\boldsymbol \lambda}\right|,{\bs{\ell}},m,K,{\bs{\theta }}} \right)} \right)}}{{\log \left( \gamma  \right)}} = \mathop {\lim }\limits_{\gamma  \to \infty } \frac{{\log \left( {\prod\limits_{k = 1}^K {{{\left( {\frac{m}{{{\theta _k}{\sigma _k}^2\left( {1 - {\lambda _k}^2} \right)}}} \right)}^{m + {\ell_k}}}} {g_{{\bf{0}},{\bs{\ell}}}}\left( x \right) + o\left( {{\gamma ^{ - \kappa }}} \right)} \right)}}{{\log \left( \gamma  \right)}} \notag\\
% = \mathop {\lim }\limits_{\gamma  \to \infty } \frac{{\log \left( {\prod\limits_{k = 1}^K {{{\left( {\frac{m}{{{\theta _k}{\sigma _k}^2\left( {1 - {\lambda _k}^2} \right)}}} \right)}^{m + {\ell_k}}}} {g_{{\bf{0}},{\bs{\ell}}}}\left( x \right)} \right)}}{{\log \left( \gamma  \right)}} + \mathop {\lim }\limits_{\gamma  \to \infty } \frac{{\log \left( {1 + o\left( {{\gamma ^{ - \kappa }}} \right)} \right)}}{{\log \left( \gamma  \right)}} = 0
%\end{align}
%

%\subsection{Proof of (\ref{eqn:g_0_0_der_meij_rem})} \label{app:proof_g_n_l}
%Recalling from (\ref{eqn:use_integral_inv_fox_H_1}), ${F_{{{ {\mathcal A}}_{\bs{\ell}}}}}(x)$ can also be rewritten as
%\begin{equation}\label{eqn:F_A0_another_form}
%{F_{{{ {\mathcal A}}_{\bs{\ell}}}}}(x) = Y_{K + 1,1}^{1,K}\left[ {\left. {\begin{array}{*{20}{c}}
%{\left( {0,1,\frac{1}{{{\Omega _1}}}, \ell_1 + m} \right), \cdots ,\left( {0,1,\frac{1}{{2{\Omega _K}^2}}, \ell_K + m} \right),\left( {1,1,0,1} \right)}\\
%{\left( {0,1,0,1} \right)}
%\end{array}} \right|\frac{{\prod\limits_{k = 1}^K {{\Omega _k}} }}{x}} \right].
%\end{equation}
On the other hand, ${F_{{{ {\cal A}}_{\bs{\ell}}}}}(x)$ can also be expressed in the form of Mellin-Barnes integral from (\ref{eqn:F_a_ell_step3}) as
%By expressing (\ref{eqn:use_integral_inv_fox_H_1}) in the form of Mellin-Barnes integral, it follows from (\ref{eqn:F_a_ell_step3}) that
\begin{align}\label{eqn:F_A0_brane_inte}
{F_{{{ {\cal A}}_{\bs{\ell}}}}}(x) &= \frac{1}{{2\pi {\rm{i}}}}\int\nolimits_{c_1 - {\rm{i}}\infty }^{c_1 + {\rm{i}}\infty } {\frac{{\Gamma \left( s \right)}}{{\Gamma \left( {s + 1} \right)}}\prod\nolimits_{k = 1}^K {\frac{{\Psi \left( {m ,m + \ell_k + 1 - s;\frac{1}{{{\Omega _k}}}} \right)}}{{{{\left( {{\Omega _k}} \right)}^{m+\ell_k}}}}} {x^s}ds} \notag \\
%&=\frac{1}{{2\pi {\rm{i}}}}\int_{{c_2} - {\rm{i}}\infty }^{{c_2} + {\rm{i}}\infty } {\left\{ {{\cal M}{F_{{\mathcal A_{\bs \ell} }}}} \right\}\left( s \right){x^{ - s}}ds}.
&=\frac{1}{{2\pi {\rm{i}}}}\int_{{c_2} - {\rm{i}}\infty }^{{c_2} + {\rm{i}}\infty } {\underbrace {\frac{{\Gamma \left( { - s} \right)}}{{\Gamma \left( { - s + 1} \right)}}\prod\nolimits_{k = 1}^K {\frac{{\Psi \left( {m,m + {\ell _k} + 1 + s;\frac{1}{{{\Omega _k}}}} \right)}}{{{{\left( {{\Omega _k}} \right)}^{m + {\ell _k}}}}}} }_{\left\{ {\mathcal M{F_{{{\cal A}_{\bs \ell} }}}} \right\}\left( s \right)}{x^{ - s}}ds}.
%&= \frac{1}{{2\pi {\rm{i}}}}\int\limits_{c_1 - {\rm{i}}\infty }^{c_1 + {\rm{i}}\infty } {\frac{{\Gamma \left( s \right)}}{{\Gamma \left( {s + 1} \right)}}\prod\limits_{k = 1}^K {{{\left( {\frac{1}{{{\Omega _k}}}} \right)}^{m - s}}\Psi \left( {m ,m + 1 - s;\frac{1}{{{\Omega _k}}}} \right)} {{\left( {\frac{x}{{\prod\limits_{k = 1}^K {{\Omega _k}} }}} \right)}^s}ds} \notag \\
\end{align}
where
$c_2 = -c_1$ and the fundamental strip\footnote{$ \left\{ {\mathcal M{{F_{{{ {\cal A}}_{\bs{\ell}}}}}}} \right\}\left( s \right)$ exists for any complex number $s$ in the fundamental strip.} of $ \left\{ {\mathcal M{{F_{{{ {\cal A}}_{\bs{\ell}}}}}}} \right\}\left( s \right)$ implies $c_2 \in (-\infty,0)$ because $ {F_{{\mathcal A_{\bs \ell} }}}\left( x \right)$ admits\cite[p400]{szpankowski2010average}
 \begin{equation}\label{eqn:fundamental_steip_F_al}
 {F_{{\mathcal A_{\bs \ell} }}}\left( x \right) = \left\{ {\begin{array}{*{20}{c}}
{\mathcal O\left( {{x^\infty }} \right)}&{x \to 0}\\
{\mathcal O\left( {{x^0}} \right)}&{x \to \infty }
\end{array}} \right.
 \end{equation}
 where $\mathcal O(\cdot)$ denotes the big O notation and the first equation of (\ref{eqn:fundamental_steip_F_al}) holds because of ${F_{{\mathcal A_{\bs \ell} }}}\left( x \right) = 0$ for $x \le 1$. Thus $c_1 \in (0,\infty)$.


By adopting \cite[Eq.9.210.2]{gradshteyn1965table} into (\ref{eqn:F_A0_brane_inte}), it yields
\begin{multline}\label{eqn:F_hat_A_0_der_fur_ex}
{F_{{{ {\cal A}}_{\bs{\ell}}}}}(x) = \frac{1}{{2\pi {\rm{i}}}}\int\nolimits_{c_1 - {\rm{i}}\infty }^{c_1 + {\rm{i}}\infty } {\frac{{\Gamma \left( s \right)}}{{\Gamma \left( {s + 1} \right)}} } \\
\times \prod\limits_{k = 1}^K {\frac{1}{{{{\left( {{\Omega _k}} \right)}^{m+\ell_k }}}}}\left( {\begin{array}{*{20}{l}}
{\frac{{\Gamma \left( { - m -\ell_k+ s} \right)}}{{\Gamma \left( s \right)}}{}_1{F_1}\left( {m+\ell_k ,m+\ell_k + 1 - s;\frac{1}{{{\Omega _k}}}} \right) + }\\
{\frac{{\Gamma \left( {m+\ell_k - s} \right)}}{{\Gamma \left( {m+\ell_k } \right)}}{{\left( {\frac{1}{{{\Omega _k}}}} \right)}^{s - m-\ell_k }}{}_1{F_1}\left( {s,1 - m-\ell_k + s;\frac{1}{{{\Omega _k}}}} \right)}
\end{array}} \right){x^s}ds.
\end{multline}
where ${}_1{F_1}\left( {\alpha ,\beta ;\frac{1}{{{\Omega _k}}}} \right)$ can be expanded as
\begin{equation}\label{eqn:F_11_series_exp}
{}_1{F_1}\left( {\alpha ,\beta ;\frac{1}{{{\Omega _k}}}} \right) = 1 + \frac{\beta }{\alpha }\frac{1}{{{\Omega _k}}} + o\left( {\frac{1}{{{\Omega _k}}}} \right).
\end{equation}
Clearly, as $\gamma$ approaches to infinity, the dominant term in (\ref{eqn:F_11_series_exp}) is $1$.
 %\begin{align}\label{eqn:F_Al_as_0}
%{F_{{A_{\bs \ell} }}}\left( x \right) &= \Pr \left( {\prod\limits_{k = 1}^K {(1 + {R_{{\bs \ell} ,k}})}  \le x} \right) \notag\\
% &\le \Pr \left( {{R_{{\bs \ell},1}} \le x - 1} \right) = \frac{{\Upsilon \left( {m + {l_1},{\Omega _1}x} \right)}}{{\Gamma \left( {m + {l_1}} \right)}} = O\left( {{x^\infty }} \right)
% \end{align}

Herein, we assume $c_1>m+{\rm max}\{\bs{\ell}\}$ because $c_1 $ could be any point in $(0,\infty)$. Then by substituting (\ref{eqn:F_11_series_exp}) into (\ref{eqn:F_hat_A_0_der_fur_ex}), we have
\begin{align}\label{eqn:F_A_0_asym_F_hy_exp}
&{{F_{{{ {\cal A}}_{\bs{\ell}}}}}(x)}{ = \prod\limits_{k = 1}^K {\frac{1}{{{{\left( {{\Omega _k}} \right)}^{m+\ell_k}}}}} \frac{1}{{2\pi {\rm{i}}}}\int\nolimits_{{c_1} - {\rm{i}}\infty }^{{c_1} + {\rm{i}}\infty } {\frac{{\Gamma \left( s \right)}}{{\Gamma \left( {s + 1} \right)}}\prod\limits_{k = 1}^K {\frac{{\Gamma \left( { - m-\ell_k + s} \right)}}{{\Gamma \left( s \right)}}} {x^s}ds}  + o\left( {\prod\limits_{k = 1}^K {\frac{1}{{{{\left( {{\Omega _k}} \right)}^{m+\ell_k}}}}} } \right)} \notag\\
{}&{ = \prod\limits_{k = 1}^K {\frac{1}{{{{\left( {{\Omega _k}} \right)}^{m +\ell_k}}}}} G_{K + 1,K + 1}^{0,K + 1}\left( {\left. {\begin{array}{*{20}{c}}
{1,1 +\ell_1+ m , \cdots ,1 +\ell_K+ m}\\
{1, \cdots ,1,0}
\end{array}} \right|x} \right) + o\left( {\prod\limits_{k = 1}^K {\frac{1}{{{{\left( {{\Omega _k}} \right)}^{m+\ell_k }}}}} } \right)}.
\end{align}
Putting (\ref{eqn:gamma_snr_rel}) into (\ref{eqn:F_A_0_asym_F_hy_exp}), ${{F_{{{ {\cal A}}_{\bs{\ell}}}}}(x)}$ can be finally written as
\begin{align}\label{eqn:F_A_0_asym_F_hy_exp1}
{{F_{{{ {\cal A}}_{\bs{\ell}}}}}(x)}&{ = \prod\limits_{k = 1}^K {{{\left( {\frac{m}{{{\theta _k}{\sigma _k}^2\left( {1 - {\lambda _k}^2} \right)}}} \right)}^{m +\ell_k }}} G_{K + 1,K + 1}^{0,K + 1}\left( {\left. {\begin{array}{*{20}{c}}
{1,1 +\ell_1+ m, \cdots ,1 +\ell_K+m}\\
{1, \cdots ,1,0}
\end{array}} \right|x} \right){\gamma ^{-d_{{ {\cal A}_{\bs{\ell}}}}}} }\notag \\
&+ o\left( {{\gamma ^{-d_{{ {\cal A}_{\bs{\ell}}}}}}} \right).
\end{align}
Comparing (\ref{eqn:F_A_0_asym_F_hy_exp1}) with (\ref{eqn:der_A_1_CDF}) and setting the coefficients corresponding to ${\gamma ^{-d_{{ {\cal A}_{\bs{\ell}}}}}}$ equal, we finally have (\ref{eqn:g_0_0_der_meij_rem}).
%Moreover, when $\bs \ell=\bf 0$ and $m$ is an integer, it follows by applying Residue theorem into (\ref{eqn:g_0_0_der_meij_rem}) that
%\begin{equation}\label{eqn:G_meij_factor_polyapp}
%{g_{{\bf{0}}}}\left( x \right)
%%&= \frac{1}{2\pi \rm i} \int\limits_{c_1 - {\rm{i}}\infty }^{c_1 + {\rm{i}}\infty } {\frac{{\Gamma \left( s \right)}}{{\Gamma \left( {s + 1} \right)}}\prod\limits_{k = 1}^K {\frac{{\Gamma \left( { - m + s} \right)}}{{\Gamma \left( s \right)}}} {x^s}ds}  \notag \\
%= \frac{{\left( {\Gamma \left( m \right)} \right)^K}}{{2\pi {\rm{i}}}}\int\nolimits_{{c_1} - {\rm{i}}\infty }^{{c_1} + {\rm{i}}\infty } {\frac{{{x^s}}}{{s\prod\nolimits_{t = 1}^m {{{\left( {s - t} \right)}^K}} }}ds}
% = {\left( {\Gamma \left( m \right)} \right)^K}\sum\nolimits_{a = 0}^m {{\rm{Res}}\left\{ {\frac{{{x^s}}}{{s\prod\nolimits_{t = 1}^m {{{\left( {s - t} \right)}^K}} }},s = a} \right\}}.  %\\
%%& = {\left[ {\frac{{{x^s}}}{{\prod\limits_{t = 1}^m {{{\left( {s - t} \right)}^K}} }}} \right]_{s = 0}} + \sum\limits_{t = 1}^m {\frac{1}{{\left( {K - 1} \right)!}}\frac{{{\partial ^{K - 1}}}}{{\partial {s^{K - 1}}}}{{\left[ {\frac{{{x^s}}}{{s\prod\limits_{u = 1,u \ne t}^m {{{\left( {s - u} \right)}^K}} }}} \right]}_{s = t}}}. \notag\\
% %=\frac{{{{\left( { - 1} \right)}^{mK}}}}{{{m^K}}} + \frac{{{{\left( {\Gamma \left( m \right)} \right)}^K}{{\left( { - 1} \right)}^{K - 1}}}}{{{{\left( {\left( {K - 1} \right)!} \right)}^{m - 1}}}}\sum\limits_{a = 1}^m {\sum\limits_{{n_1} +  \cdots  + {n_{m + 1}} = K - 1} {\frac{{{{\left( { - 1} \right)}^{{n_m}}}{{\left( {\ln x} \right)}^{{n_m}}}{x^a}}}{{{n_1}! \cdots {n_m}!{a^{1 + {n_{m + 1}}}}}}} }   \\
%%\times \prod\limits_{t = 1}^{a - 1} {\frac{{\Gamma \left( {K + {n_t}} \right)}}{{{{\left( {a - t} \right)}^{1 + K + {n_t}}}}}} \prod\limits_{t = a + 1}^m {\frac{{\Gamma \left( {K + {n_{t - 1}}} \right)}}{{{{\left( {a - t} \right)}^{1 + K + {n_{t - 1}}}}}}},
%% and general Leibniz rule
%\end{equation}
%where ${\rm Res}(f,a_k)$ denotes the residue of $f$ at $a_k$ and $c_1 > m$. Applying general Leibniz rule \cite[Eq.1.4.12]{NIST:DLMF} finally leads to (\ref{eqn:G_meij_factor_poly}).

%It is worth noting that (\ref{eqn:g_0_0_der_meij_rem}) holds for ${{\rm min}\{\bs{\ell}\}}+m > 0$  considering from Remark \ref{rem:chann}
%that $R_{{\bs{\ell}},k} \sim \mathcal {G}(m+\ell_k,{\Omega _k})$, where $\ell_k + m > 0$.
% By using the relationship between (\ref{eqn:sec_def_g_fun}) and (\ref{eqn:g_0_0_der_meij_rem}), we can prove ${g_{\bs{\ell}}}\left( x \right)$ is an increasing and convex function of $x$.

\section{Proof of Lemma \ref{cor:time_corr}} \label{app:proo_cor}
%Without loss of generality, the coefficients of correlation factor are assumed to be non-negative, i.e., $ {\bs{\lambda }} \succ _{\mathbb{R}_+^K} \bf 0$. Thus $|{\bs{\lambda }}| = {\bs{\lambda }} $.
Defining ${\bs{\lambda }_1} = \left( {{\lambda _1}, \cdots,{\lambda _K}} \right)$ and given $K>1$, we first consider the simplest case where $\bs{\lambda}_1 \preceq \bs{\lambda}_2$ and ${\bs{\lambda }_2} = \left( {{\lambda _1}+ {\Delta _1}, \cdots ,{\lambda _K}} \right)$ with ${\Delta _1} \ge 0$. With (\ref{eqn:ell_def}), $\ell \left( {{\bs \lambda _2},K} \right)$ can be written as
\begin{equation}\label{eqn:ell_lambda_2_def}
\ell \left( {{\bs \lambda _2},K} \right) = \left( {1 + \sum\limits_{k = 2}^K {\frac{{{\lambda _k}^2}}{{1 - {\lambda _k}^2}}}  + \frac{{{{\left( {{\lambda _1} + {\Delta _1}} \right)}^2}}}{{1 - {{\left( {{\lambda _1} + {\Delta _1}} \right)}^2}}}} \right)\left( {1 - {{\left( {{\lambda _1} + {\Delta _1}} \right)}^2}} \right)\prod\limits_{k = 2}^K {\left( {1 - {\lambda _k}^2} \right)}.
\end{equation}
Since ${\Delta _1} \ge 0$, we have
\begin{align}\label{eqn:lambda_2_writ_re}
\ell \left( {{\bs \lambda _2},K} \right) &= \left( {1 + \sum\limits_{k = 2}^K {\frac{{{\lambda _k}^2\left( {1 - {{\left( {{\lambda _1} + {\Delta _1}} \right)}^2}} \right)}}{{1 - {\lambda _k}^2}}} } \right)\prod\limits_{k = 2}^K {\left( {1 - {\lambda _k}^2} \right)} \notag\\
 &\le \left( {1 + \sum\limits_{k = 2}^K {\frac{{{\lambda _k}^2\left( {1 - {\lambda _1}^2} \right)}}{{1 - {\lambda _k}^2}}} } \right)\prod\limits_{k = 2}^K {\left( {1 - {\lambda _k}^2} \right)}
 %&= \left( {\frac{1}{{1 - {\lambda _1}^2}} + \sum\limits_{k = 2}^K {\frac{{{\lambda _k}^2}}{{1 - {\lambda _k}^2}}} } \right)\left( {1 - {\lambda _1}^2} \right)\prod\limits_{k = 2}^K {\left( {1 - {\lambda _k}^2} \right)}
= \ell \left( {{\bs \lambda _1},K} \right),
\end{align}
where the equality holds if and only if ${\Delta _1} = 0$. Now gradually adding non-negative increments $\Delta_i$ to the other elements $\lambda_i$, $i=2, \cdots, K$, in ${\bs \lambda _2}$ and using the similar approach as (\ref{eqn:lambda_2_writ_re}), we can finally have $\ell \left( {{\bs \lambda _2},K} \right) \le \ell \left( {{\bs \lambda _1},K} \right)$ for general ${\bs{\lambda }_2} = \left( {{\lambda _1}+ {\Delta _1}, \cdots, {\lambda _i}+{\Delta _i}, \cdots,{\lambda _K}+{\Delta _K}} \right)$ with ${\Delta _i} \ge 0$ where the equality holds if and only if ${\Delta _i} =0$. Clearly, when $K > 1$, we have $\ell \left( {{\bs \lambda _1},K} \right) \le \ell \left( {{\bf 0},K} \right)=1$, where the equality holds if and only if ${\bs \lambda _1} = {\bf 0}$. Then the proof completes.



\section{Proof of Lemma \ref{the:coding_modulation_gain}}\label{app:coding_gain}

%
%
%As pointed out in \cite{goldsmith2005wireless}, the coding gain refers to the amount that the SNR can be reduced under the coding technique for a given $p_{out,K}$. From (\ref{eqn:der_hat_G_CDF_asy_furre}), it is shown that the coding gain $C(\mathcal R)=\left({g_{{\bf{0}}}}\left( 2^{\cal R} \right)\right)^{-\frac{1}{d}}$ is affected by the coding rate $\mathcal R$. In order to examine the impact of coding rate $\cal R$, some special properties of ${g_{{\bf{0}}}}\left( 2^{\mathcal R} \right)$ can be found through combining (\ref{eqn:sec_def_g_fun}) with (\ref{eqn:g_0_0_der_meij_rem}). With (\ref{eqn:g_0_0_der_meij_rem}), ${g_{{\bf{0}}}}\left( 2^{\mathcal R} \right)$ can be easily calculated. Specifically, when $m$ is an integer, it follows by applying Residue theorem and general Leibniz rule into (\ref{eqn:g_0_0_der_meij_rem}) that
%\begin{align}\label{eqn:G_meij_factor_poly}
%{g_{{\bf{0}}}}\left( 2^{\cal R} \right)
%%&= \frac{1}{2\pi \rm i} \int\limits_{c_1 - {\rm{i}}\infty }^{c_1 + {\rm{i}}\infty } {\frac{{\Gamma \left( s \right)}}{{\Gamma \left( {s + 1} \right)}}\prod\limits_{k = 1}^K {\frac{{\Gamma \left( { - m + s} \right)}}{{\Gamma \left( s \right)}}} {x^s}ds}  \notag \\
%&= {\left( {\Gamma \left( m \right)} \right)^K}\frac{1}{{2\pi {\rm{i}}}}\int\nolimits_{{c_1} - {\rm{i}}\infty }^{{c_1} + {\rm{i}}\infty } {\frac{{{2^{\mathcal Rs}}}}{{s\prod\limits_{t = 1}^m {{{\left( {s - t} \right)}^K}} }}ds} \notag\\
% &= {\left( {\Gamma \left( m \right)} \right)^K}\sum\limits_{a = 0}^m {{\rm{Res}}\left\{ {\frac{{{2^{\mathcal Rs}}}}{{s\prod\limits_{t = 1}^m {{{\left( {s - t} \right)}^K}} }},s = a} \right\}} \notag \\
%%& = {\left[ {\frac{{{x^s}}}{{\prod\limits_{t = 1}^m {{{\left( {s - t} \right)}^K}} }}} \right]_{s = 0}} + \sum\limits_{t = 1}^m {\frac{1}{{\left( {K - 1} \right)!}}\frac{{{\partial ^{K - 1}}}}{{\partial {s^{K - 1}}}}{{\left[ {\frac{{{x^s}}}{{s\prod\limits_{u = 1,u \ne t}^m {{{\left( {s - u} \right)}^K}} }}} \right]}_{s = t}}}. \notag\\
% &=\frac{{{{\left( { - 1} \right)}^{mK}}}}{{{m^K}}} + \frac{{{{\left( {\Gamma \left( m \right)} \right)}^K}{{\left( { - 1} \right)}^{K - 1}}}}{{{{\left( {\left( {K - 1} \right)!} \right)}^{m - 1}}}}\sum\limits_{a = 1}^m {\sum\limits_{{n_1} +  \cdots  + {n_{m + 1}} = K - 1} {\frac{{{{\left( { - 1} \right)}^{{n_m}}}{{\left( {\mathcal R \ln 2} \right)}^{{n_m}}}{2^{a\mathcal R}}}}{{{n_1}! \cdots {n_m}!{a^{1 + {n_{m + 1}}}}}}} }   \notag\\
%&\quad\times \prod\limits_{t = 1}^{a - 1} {\frac{{\Gamma \left( {K + {n_t}} \right)}}{{{{\left( {a - t} \right)}^{1 + K + {n_t}}}}}} \prod\limits_{t = a + 1}^m {\frac{{\Gamma \left( {K + {n_{t - 1}}} \right)}}{{{{\left( {a - t} \right)}^{1 + K + {n_{t - 1}}}}}}},
%% and general Leibniz rule
%\end{align}
%where ${\rm Res}(f,a_k)$ denotes the residue of $f$ at $a_k$ and $c_1 > m$.

%Moreover, clearly from (\ref{eqn:sec_def_g_fun}), it is not hard to prove that ${g_{{\bf{0}}}}\left( 2^{\cal R} \right)$ is a monotonically decreasing function of $\mathcal R$, and by using (\ref{eqn:g_0_0_der_meij_rem}) together with (\ref{eqn:sec_def_g_fun}), the conclusion of the convexity of ${{g_{\bf{0}}}\left(  2^{\cal R} \right)}$ with respect to $\mathcal R$ is drawn in the following remark.
%\begin{remark}\label{the:coding_gain}
%${{g_{\bs{\ell}}}\left(  2^{\cal R} \right)}$ is a convex function of $\mathcal R$ if ${\rm max}{\{\bs{\ell}\}}+m \ge 1$. Therefore, ${{g_{\bf{0}}}\left(  2^{\cal R} \right)}$ is a convex function of $\mathcal R$ for any $m \ge 1$.
%\end{remark}
%\begin{proof}
%Please see Appendix \ref{app:coding_gain}.
%\end{proof}

%Accordingly, coding gain $C(\mathcal R)$ is a decreasing function of $\mathcal R$. In other words, the increase of $\mathcal R$ will degrade the outage performance.
%\end{theorem}
%\begin{proof}
%Please see Appendix \ref{app:proof_g_n_l}.
%\end{proof}
%In addition, ${p_{out\_asy,K}}$ is an increasing function of $\mathcal R$.
%monotonically increasing and
%\begin{remark}\label{the:coding_gain}
%${{g_{\bs{\ell}}}\left(  2^{\cal R} \right)}$ is a monotone increasing and convex function of $\mathcal R$ if ${\rm max}{\{\bs{\ell}\}}+m \ge 1$. Accordingly, from (\ref{eqn:Pout_def_asy}), the asymptotic outage probability ${p_{out\_asy,K}}$ is an increasing and convex function of coding rate $\mathcal R$ when $m \ge 1$.
%\end{remark}


It is readily found from (\ref{eqn:sec_def_g_fun}) that ${g_{\bs{\ell}}}\left( 2^{\mathcal R} \right)$ is a monotonically increasing function of $\mathcal R$. Therefore the first derivative of ${g_{\bs{\ell}}}\left( 2^{\mathcal R} \right)$ with respect to $\mathcal R$ is greater than $0$, i.e., ${g_{\bs{\ell}}}^\prime \left( 2^{\mathcal R} \right) > 0$. By applying the property of derivatives of Laplace transform into (\ref{eqn:g_0_0_der_meij_rem}) \cite{debnath2010integral}, it follows that
\begin{align}\label{eqn:first_der_g}
{g_{\bs{\ell}}}^\prime \left( {{2^{\cal R}}} \right) &= \Theta \prod\limits_{k = 1}^K {\Gamma \left( {m + {\ell_k}} \right)} \frac{1}{{2\pi {\rm{i}}}}\int_{{c_1} - {\rm{i}}\infty }^{{c_1} + {\rm{i}}\infty } {\prod\limits_{k = 1}^K {\frac{{\Gamma \left( { - m - {\ell_k} + s} \right)}}{{\Gamma \left( s \right)}}} {2^{{\cal R}s}}ds} > 0. %\notag\\
% &= \ln 2\prod\limits_{k = 1}^K {\Gamma \left( {m + {\ell_k}} \right)} G_{K,K}^{0,K}\left( {\left. {\begin{array}{*{20}{c}}
%{1 + {\ell_1} + m, \cdots ,1 + {\ell_K} + m}\\
%{1, \cdots ,1}
%\end{array}} \right|{2^{\cal R}}} \right) > 0.
\end{align}
where $\Theta = \ln 2$. Similarly, the second derivative of ${g_{\bs{\ell}}}\left( 2^{\mathcal R} \right)$ with respect to $\mathcal R$ is given by
\begin{equation}\label{eqn:sec_der_g}
{g_{\bs{\ell}}}^{\prime \prime }\left( 2^{\mathcal R} \right) = {\Theta^2}\prod\limits_{k = 1}^K {\Gamma \left( {m + {\ell_k}} \right)} \frac{1}{{2\pi {\rm{i}}}}\int_{{c_1} - {\rm{i}}\infty }^{{c_1} + {\rm{i}}\infty } {s\prod\limits_{k = 1}^K {\frac{{\Gamma \left( { - m - {\ell_k} + s} \right)}}{{\Gamma \left( s \right)}}} {2^{{\cal R}s}}ds}.
\end{equation}
Without loss of generality, we assume that $\ell_1={\rm max}\{\bs{\ell}\}$. By rewriting $s=(s - m - \ell_1) + (m + \ell_1)$, the integral in (\ref{eqn:sec_der_g}) can be derived as
\begin{multline}\label{eqn:mellin_g_0_secd}
\frac{1}{{2\pi {\rm{i}}}}\int_{{c_1} - {\rm{i}}\infty }^{{c_1} + {\rm{i}}\infty } {s\prod\limits_{k = 1}^K {\frac{{\Gamma \left( { - m - {\ell_k} + s} \right)}}{{\Gamma \left( s \right)}}} {2^{\mathcal Rs}}ds}
=  \left( {m + {\ell_1}} \right)\frac{1}{{2\pi {\rm{i}}}}\int_{{c_1} - {\rm{i}}\infty }^{{c_1} + {\rm{i}}\infty } {\prod\limits_{k = 1}^K {\frac{{\Gamma \left( { - m - {\ell_k} + s} \right)}}{{\Gamma \left( s \right)}}} {2^{\mathcal Rs}}ds}\\
 + \frac{1}{{2\pi {\rm{i}}}}\int_{{c_1} - {\rm{i}}\infty }^{{c_1} + {\rm{i}}\infty } {\frac{{\Gamma \left( {1 - m - {\ell_1} + s} \right)}}{{\Gamma \left( s \right)}}\prod\limits_{k = 2}^K {\frac{{\Gamma \left( { - m - {\ell_k} + s} \right)}}{{\Gamma \left( s \right)}}} {2^{\mathcal Rs}}ds}.
\end{multline}
Putting (\ref{eqn:mellin_g_0_secd}) into (\ref{eqn:sec_der_g}) along with (\ref{eqn:first_der_g}), it follows that
\begin{equation}\label{eqn:sec_der_g_fin}
{g_{\bs{\ell}}}^{\prime \prime }\left( x \right){\rm{ = }}\left\{ {\begin{array}{*{20}{c}}
{\Theta {g_{{{\bs{\ell}}_{/1}}}}^\prime \left( {{2^{\cal R}}} \right) + \Theta {g_{\bs{\ell}}}^\prime \left( {{2^{\cal R}}} \right)>0,}&{{\ell_1} + m = 1{\rm{;}}}\\
{\Theta \frac{{\Gamma \left( {m + {\ell_1}} \right)}}{{\Gamma \left( {m + {\ell_1} - 1} \right)}}{g_{\bar {{\bs{\ell}}}}}^\prime \left( {{2^{\cal R}}} \right) + \Theta \left( {m + {\ell_1}} \right){g_{\bs{\ell}}}^\prime \left( {{2^{\cal R}}} \right)>0,}&{{\ell_1} + m > 1.}
\end{array}} \right.
\end{equation}
where ${{\bs{\ell}}_{/1}} = \left( {{\ell_2},{\ell_3}, \cdots ,{\ell_K}} \right)$ and $\bar {{\bs{\ell}}} = \left( {{\ell_1} - 1,{\ell_2},{\ell_3}, \cdots ,{\ell_K}} \right)$. Thus it proves that ${{g_{\bs{\ell}}}\left(  2^{\cal R} \right)}$ is a convex function of $\mathcal R$ if ${\rm max}{\{\bs{\ell}\}}+m \ge 1$.  Then the proof directly follows.%Therefore, ${{g_{\bf{0}}}\left(  2^{\cal R} \right)}$ is a convex function of $\mathcal R$ for any $m \ge 1$.



%\begin{equation}\label{eqn:g_0_l}
%{g_{{\bf{0}},{\bs{\ell}}}}\left( x \right) = G_{K + 1,K + 1}^{0,K + 1}\left( {\left. {\begin{array}{*{20}{c}}
%{1,1 + {\ell_1} + m, \cdots ,1 + {\ell_K} + m}\\
%{1, \cdots ,1,0}
%\end{array}} \right|x} \right)
%\end{equation}
%\section{Relationship between Coding Gain and Fading Order}\label{app:relation_c_m}
%Plugging (\ref{eqn:sec_def_g_fun}) into $C(\mathcal R)$ leads to
%\begin{equation}\label{eqn:C_R_rew}
%C\left( {\cal R} \right) = {\left( {{{\left( {\underbrace {\int_{\prod\nolimits_{k = 1}^K {\left( {1 + {t_k}} \right)}  \le x} {\prod\limits_{k = 1}^K {{t_k}^{m - 1}} d{t_1} \cdots d{t_{K - 1}}d{t_K}} }_{{\cal K}\left( m \right)}} \right)}^{\frac{1}{m}}}} \right)^{ - \frac{1}{K}}}.
%\end{equation}
%We define a function as
%\begin{multline}\label{eqn:func_aux}
%\int_{\prod\nolimits_{k = 1}^K {\left( {1 + {t_k}} \right)}  \le x} {{{\left( {u{{\left( {\prod\limits_{k = 1}^K {{t_k}} } \right)}^{\frac{{m}}{2}}} + v{{\left( {\prod\limits_{k = 1}^K {{t_k}} } \right)}^{\frac{{m - 2}}{2}}}} \right)}^2}d{t_1} \cdots d{t_{K - 1}}d{t_K}} \\
% = {u^2}\mathcal K\left( m+1 \right) + {v^2}\mathcal K\left( {m - 1} \right) + 2uv \mathcal K\left( {m} \right),
%\end{multline}
%which is evidently non-negative. Thus we have ${\left( {\mathcal K\left( {m} \right)} \right)^2} \le \mathcal K\left( m+1 \right)\mathcal K\left( {m-1} \right)$, or finally
%\begin{equation}\label{eqn:kappa_relation}
%{\left( {\mathcal K\left( {m} \right)} \right)^{2 {m}}} \le {\left( {\mathcal K\left( m+1 \right)} \right)^{m}}{\left( {\mathcal K\left( {m-1} \right)} \right)^{m}}.
%\end{equation}
%Replacing here $m$ successively by $2,3,\cdots,m-1$ and multiplying all the inequalities then yields ${\left( {\mathcal K\left( {m} \right)} \right)^{\frac{1}{{m}}}} \le {\left( {\mathcal K\left( {m + 1} \right)} \right)^{\frac{1}{{m + 1}}}}$. Hereby, with this result along with (\ref{eqn:C_R_rew}), we conclude that the increase of fading order will lead to the increase of coding gain.


\bibliographystyle{ieeetran}
\bibliography{Asy_ana}

\end{document}


