Semantic segmentation, one of the most widely researched topics in computer vision, has achieved remarkable success \cite{chen2017deeplab,pspnet,huang2019ccnet,huang2021alignseg} with the development of deep learning techniques \cite{resnet}. However, traditional segmentation models are only capable of segmenting a few predefined categories within a closed vocabulary \cite{pascal, coco, miao2021vspw, miao2022large}, which is much smaller than the number of categories used by humans to describe the real world. Therefore, zero-shot segmentation \cite{spnet, zs5, cagnet, han2023global} is introduced to segment objects using arbitrary categories described by texts.

Recently, large-scale visual-language pre-training models (\textit{e.g.} CLIP \cite{radford2021learning} and ALIGN \cite{jia2021scaling}) have shown impressive transferability in recognizing novel categories, leading to their increased adoption for tackling the challenging zero-shot segmentation task \cite{zegformer,zsseg,ovseg,freeseg}. A mainstream solution follows the "frozen CLIP" paradigm, which executes the zero-shot segmentation with two steps: 1) first employing a Proposal Generator to produce class-agnostic mask proposals and 2) then leveraging a frozen pre-trained CLIP to classify each mask proposal via similarity matching in the aligned image-text feature space. While acceptable results are obtained, we reveal that these approaches overlook a crucial issue, \textit{i.e.} the frozen CLIP is insensitive to different mask proposals and tends to produce similar predictions for various proposals of the same image. 

To better illustrate the above-mentioned issue, we show several examples in Fig. \ref{fig:intro}. We use MaskFormer \cite{cheng2021maskformer} to generate a series of mask proposals and select three typical ones. When using frozen CLIP for classification, we observe that it correctly classifies the high-quality \textit{swan} proposal $p_1$. However, for the other two proposals $p_2$ and $p_3$, which respectively contain only shape information of \textit{swan} and both regions of \textit{swan} and \textit{river}, the frozen CLIP produces similar predictions compared to $p_1$. 
This is reasonable since CLIP is trained by image-text pairs, making it insensitive to pixel-level information (\textit{e.g.} background noise), and resulting in numerous false positives.
Based on the above observations, we consider that an expected CLIP for zero-shot segmentation task should \textbf{1) be sensitive to different mask proposals, 2) not compromise its original transferability on novel classes.}
 
To this end, we introduce a Mask-aware CLIP Fine-tuning method (dubbed MAFT). To make CLIP sensitive to different mask proposals, we devise an Image-Proposals CLIP Encoder (IP-CLIP Encoder), which utilizes mask proposals to perform masked Multihead Attention \cite{cheng2021maskformer, cheng2021mask2former}. This design enables the model to handle arbitrary numbers of images and proposals simultaneously. The \textit{mask-aware loss} is proposed to minimise the distance between the IoU score of mask proposals and the classification score of IP-CLIP Encoder, prompting IP-CLIP Encoder to differentiate various proposals. 
Besides, to preserve CLIP's zero-shot transferability, we utilize a frozen CLIP as a teacher network to facilitate fine-tuning. This is achieved by aligning the outputs of the frozen CLIP and IP-CLIP Encoder through \textit{self-distillation loss}.
By performing MAFT, several advantages are provided: 1) Fine-tuning is efficient since only a few mask proposals need to be classified. 2) Compared to pixel-level fine-tuning, mask-aware fine-tuning hardly alters the structure of CLIP itself, preserving its maximum transferability. 3) Mask-aware fine-tuning of CLIP is released from the segmentation module, making it plug-and-play and applicable to any "frozen CLIP" approaches. As shown in Fig. \ref{fig:intro}, the mask-aware CLIP can well distinguish different proposals and provide proper classification scores for both seen (\textit{river}) and unseen (\textit{swan}) classes.


% We evaluate our MAFT on three commonly used zero-shot segmentation benchmarks: COCO-Stuff, Pascal-VOC, and ADE20K. Extensive experiments show that MAFT works well with various zero-shot segmentation methods. It is significantly better than the freezing CLIP counterpart, leading to new state-of-the-art results on all three datasets.
We evaluate our MAFT on three commonly used zero-shot segmentation benchmarks: COCO-Stuff \cite{coco}, Pascal-VOC \cite{pascal}, and ADE20K \cite{ade20k}. Extensive experiments show that MAFT works well with various zero-shot segmentation methods. In particular, by plugging MAFT, the state-of-the-art approach FreeSeg \cite{freeseg} achieves superior performance on COCO-Stuff (42.2\% $\rightarrow$ 50.4\%), Pascal-VOC (78.6\% $\rightarrow$ 81.8\%) and ADE20K (4.4\% $\rightarrow$ 8.7\%) in terms of mIoU of unseen classes. Furthermore, we conduct experiments in a \textit{open-vocabulary} setting, where MAFT enhances the performance of A-847 \cite{ade20k}, A-150 \cite{ade20k}, PC-459 \cite{pc}, PC-59 \cite{pc} and PAS-20 \cite{pascal} datasets by +3.0\%, +11.2\%, +6.4\%, +19.1\% and +4.4\%, respectively.
Notably, our approach outperforms the freezing CLIP counterpart and establishes new state-of-the-art results on all datasets.


\begin{figure}[t]
\centering
\begin{subfigure}{0.46\textwidth}
\includegraphics[width=\textwidth]{figures/intro_crop.pdf}
\caption{Scene bias in random crop}\label{fig:intro-crop}
\end{subfigure}~
\begin{subfigure}{0.33\textwidth}
\includegraphics[width=\textwidth]{figures/intro_multi.pdf}
\caption{Performance drop}\label{fig:intro-multi}
\end{subfigure}~
\begin{subfigure}{0.19\textwidth}
\includegraphics[width=\textwidth]{figures/intro_bg.pdf}
\caption{Biased prediction}\label{fig:intro-bg}
\end{subfigure}
\caption{
Scene bias issue (a) and its negative effects on contrastive learning. (b) Linear evaluation \citep{kolesnikov2019revisiting} of the original MoCov2 \citep{he2020momentum} and our debiased method, trained and evaluated on the COCO \citep{lin2014microsoft} and Flowers \citep{nilsback2006visual} datasets, respectively, using the ResNet-50 architecture \cite{he2016deep}. The vanilla MoCov2 often loses its discriminative power as training goes as it entangles different objects, while the debiased model stably improves the classification performance. (c) Prediction of MoCov2 on an image from the Background Challenge \citep{xiao2021noise}. The vanilla MoCov2 makes decisions from the background instead of the object, leading to biased prediction on background-shifted images.
}\label{fig:intro}
\end{figure}


