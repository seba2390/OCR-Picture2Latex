\documentclass{article}
\pdfoutput=1

% if you need to pass options to natbib, use, e.g.:
%     \PassOptionsToPackage{numbers, compress}{natbib}
% before loading neurips_2023


% ready for submission
% \usepackage{neurips_2023}


% to compile a preprint version, e.g., for submission to arXiv, add add the
% [preprint] option:
\usepackage[preprint]{neurips_2023}
\usepackage{natbib}
\setcitestyle{numbers,square}

% to compile a camera-ready version, add the [final] option, e.g.:
% \usepackage[final]{neurips_2023}


% to avoid loading the natbib package, add option nonatbib:
%    \usepackage[nonatbib]{neurips_2023}


\usepackage[utf8]{inputenc} % allow utf-8 input
\usepackage[T1]{fontenc}    % use 8-bit T1 fonts
\usepackage[colorlinks=true,linkcolor=red,citecolor=green,urlcolor=red,]{hyperref}
\usepackage{hyperref}       % hyperlinks

\usepackage{url}            % simple URL typesetting
\usepackage{booktabs}       % professional-quality tables
\usepackage{amsfonts}       % blackboard math symbols
\usepackage{nicefrac}       % compact symbols for 1/2, etc.
\usepackage{microtype}      % microtypography
\usepackage{xcolor}         % colors

% my package
\usepackage{graphicx}
\usepackage{enumitem}
\usepackage{multirow}
% \usepackage[dvipsnames,table]{xcolor} 
\usepackage{colortbl}
\usepackage{arydshln}       % 负责画虚线的包

\usepackage{float}
\usepackage{subfig}
\usepackage{amsmath}
\usepackage{makecell}


\usepackage{appendix}
% \usepackage[table]{xcolor}

%\title{Mask-aware CLIP Fine-tuning for \\ Zero-Shot Segmentation}
\title{Learning Mask-aware CLIP Representations for \\ Zero-Shot Segmentation}

\author{%
  Siyu Jiao$^{1, 2, 3}$\thanks{Work done during an internship at Picsart AI Research (PAIR).},\quad Yunchao Wei$^{1, 2, 3}$, \quad Yaowei Wang$^{3}$, \quad Yao Zhao$^{1, 2, 3}$, \quad \textbf{Humphrey Shi} $^{4}$\\
  \\
  $^1$~Institute of Information Science, Beijing Jiaotong University \\
  $^{2}$~Beijing Key Laboratory of Advanced Information Science and Network \\
  $^3$~Peng Cheng Laboratory \quad  $^4$~Picsart AI Research (PAIR) \\
  \texttt{jiaosiyu99@bjtu.edu.cn} \\
}


\begin{document}


\maketitle


\begin{abstract}

Recently, pre-trained vision-language models have been increasingly used to tackle the challenging zero-shot segmentation task. Typical solutions follow the paradigm of first generating mask proposals and then adopting CLIP to classify them. To maintain the CLIP's zero-shot transferability, previous practices favour to freeze CLIP during training. However, in the paper, we reveal that CLIP is insensitive to different mask proposals and tends to produce similar predictions for various mask proposals of the same image. This insensitivity results in numerous false positives when classifying mask proposals. This issue mainly relates to the fact that CLIP is trained with image-level supervision.
To alleviate this issue, we propose a simple yet effective method, named Mask-aware Fine-tuning (MAFT). Specifically,  Image-Proposals CLIP Encoder (IP-CLIP Encoder) is proposed to handle arbitrary numbers of image and mask proposals simultaneously. Then, \textit{mask-aware loss} and \textit{self-distillation loss} are designed to fine-tune IP-CLIP Encoder, ensuring CLIP is responsive to different mask proposals while not sacrificing transferability.
In this way, mask-aware representations can be easily learned to make the true positives stand out. Notably, our solution can seamlessly plug into most existing methods without introducing any new parameters during the fine-tuning process. 
We conduct extensive experiments on the popular zero-shot benchmarks. With MAFT, the performance of the state-of-the-art methods is promoted by a large margin: 50.4\% (+ 8.2\%) on COCO, 81.8\% (+ 3.2\%) on Pascal-VOC, and 8.7\% (+4.3\%) on ADE20K in terms of mIoU for unseen classes. 
Code is available at 
\href{https://github.com/jiaosiyu1999/MAFT.git}{github.com/jiaosiyu1999/MAFT.git}.

\end{abstract}



\section{Introduction}
\label{sec:intro}

Neural networks are powerful models that excel at a wide range of tasks.
However, they are notoriously difficult to interpret and extracting explanations 
    for their predictions is an open research problem. Linear models, in contrast, are generally considered interpretable, because
    the \emph{contribution} 
    (`the weighted input') of every dimension to the output is explicitly given.
Interestingly, many modern neural networks implicitly model the output as a linear transformation of the input;
    a ReLU-based~\cite{nair2010rectified} neural network, e.g.,
    is piece-wise linear and the output thus a linear transformation of the input, cf.~\cite{montufar2014number}.
    However, due to the highly non-linear manner in which these linear transformations are `chosen', the corresponding contributions per input dimension do not seem to represent the learnt model parameters well, cf.~\cite{adebayo2018sanity}, and a lot of research is being conducted to find better explanations for the decisions of such neural networks, cf.~\cite{simonyan2013deep,springenberg2014striving,zhou2016CAM,selvaraju2017grad,shrikumar2017deeplift,sundararajan2017axiomatic,srinivas2019full,bach2015pixel}.
    
In this work, we introduce a novel network architecture, the \textbf{Convolutional Dynamic Alignment Networks (CoDA-Nets)}, {for which the model-inherent contribution maps are faithful projections of the internal computations and thus good `explanations' of the model prediction.} 
There are two main components to the interpretability of the CoDA-Nets. 
    First, the CoDA-Nets are \textbf{dynamic linear}, i.e., they compute their outputs through a series of input-dependent linear transforms, which are based on our novel \mbox{\textbf{Dynamic Alignment Units (DAUs)}}. 
        As in linear models, the output can thus be decomposed into individual input contributions, see Fig.~\ref{fig:teaser}.
    Second, the DAUs are structurally biased to compute weight vectors that \textbf{align with \mbox{relevant} patterns} in their inputs. 
In combination, the CoDA-Nets thus inherently  
produce contribution maps that are `optimised for interpretability': 
since each linear transformation matrix and thus their combination is optimised to align with discriminative features, the contribution maps reflect the most discriminative features \emph{as used by the model}.

With this work, we present a new direction for building inherently more interpretable neural network architectures with high modelling capacity.
In detail, we would like to highlight the following contributions:
\begin{enumerate}[wide, label={\textbf{(\arabic*)}}, itemsep=-.5em, topsep=0em, labelwidth=0em, labelindent=0pt]
    \item We introduce the Dynamic Alignment Units (DAUs), which 
    improve the interpretability of neural networks and have two key properties:
    they are 
    \emph{dynamic linear} 
    and align their weights with discriminative input patterns.
    \item Further, we show that networks of DAUs \emph{inherit} these two properties. In particular, we introduce Convolutional Dynamic Alignment Networks (CoDA-Nets), which are built out of multiple layers of DAUs. As a result, the \emph{model-inherent contribution maps} of CoDA-Nets highlight discriminative patterns in the input.
    \item We further show that the alignment of the DAUs can be promoted 
    by applying a `temperature scaling' to the final output of the CoDA-Nets. 
    \item We show that the resulting contribution maps 
    perform well under commonly employed \emph{quantitative} criteria for attribution methods. Moreover, under \emph{qualitative} inspection, we note that they exhibit a high degree of detail.
    \item Beyond interpretability, 
    CoDA-Nets are performant classifiers and yield competitive classification accuracies on the CIFAR-10 and TinyImagenet datasets.
\end{enumerate}

\section{Related Work}
\label{sec:related}
\section{Related Work}
\label{sec:related_work}


\subsection{User Behavior Modeling}
\label{subsec:rw:user}

User behavior modeling is an important topic in industrial ads, search, and recommendation system. A notable pioneering work that leverages the power of neural network is provided by Youtube Recommendation \cite{covington2016deep}. User historical interactions with the system are embedded first, and sum-pooled into fixed width input for downstream multi-layer perception. 

Follow-up work starts exploring the sequential nature of these interactions. Among these, earlier work exploits sequence models such as RNN \cite{hidasi2015session}, while later work starting with \cite{li2017neural} mostly adopts attention between the target example and user historical behavior sequence, notably DIN \cite{zhou2018deep} and KFAtt\cite{liu2020kalman}. 

More recently, self-attention \cite{kang2018self} and graph neural net \cite{wu2019session,pang2021heterogeneous} have been successfully applied in the sequential recommendation domain. 




% User Behavior Modeling Methods contains kinds of Methods:Pooling-based, Attention-based和Sequence-based。Pooling-based\cite{covington2016deep}和Attention-based\cite{zhou2018deep}, \cite{vaswani2017attention} Methods会将用户行为看成无序的集合,其中Pooling-based将所有用户行为看做是等价的。Sequence-based\cite{hidasi2015session}则可以根据用户行为序列提取信息,具有时间属性。

% ps:
% \cite{covington2016deep}: Deep Neural Networks for YouTube Recommendations.
% \cite{zhou2018deep}: Deep interest network for click-through rate prediction.
% \cite{vaswani2017attention}: Attention is all you need.
% \cite{hidasi2015session}: Session-based recommendations with recurrent neural networks.
% \label{subsec:rw:behavior}

\subsection{RNN in search and recommendation}
\label{subsec:rw:rnn}

While attention excels in training efficiency, RNN still plays a useful role in settings like incremental model training and updates. 
Compared to DNN, RNN is capable of taking the entire history of a user into account, effectively augmenting the input feature space. Furthermore, it harnesses the sequential nature of the input data efficiently, by constructing a training example at every event in the sequence, rather than only at the last event \cite{zhang2014sequential}. Since the introduction of Attention in \cite{vaswani2017attention}, however, RNN starts to lose its dominance in the sequential modeling field, mainly because of its high serving latency. 

We argue however RNN saves computation in online serving, since it propagates the user hidden state in a forward only manner, which is friendly to incremental update. In the case when user history can be as long as thousands of sessions, real time attention computation can be highly impractical, unless mitigated by some approximation strategies \cite{drachsler2008personal}. The latter introduces additional complexity and can easily lose accuracy. 

Most open-source implementations of reinforcement learning framework for search and recommendation system implicitly assume an underlying RNN backbone \cite{chen2019top}. The implementation however typically simplifies the design by only feeding a limited number of ID sequences into the RNN network \cite{zhao2019deep}. 

\cite{zhang2019deep} contains a good overview of existing RNN systems in Search / Recommendations. In particular, they are further divided into those with user identifier and those without. In the latter case, the largest unit of training example is a single session from which the user makes one or more related requests. While in the former category, a single user could come and go multiple times over a long period of time, thus providing much richer contexts to the ranker. It is the latter scenario that we focus on in this paper. To the best of our knowledge, such settings are virtually unexplored in the search ranking setting.




% RNN具备记忆能力,当涉及到连续的、与上下文相关的任务时,它比其他神经网络具有更大的优势。论文1\cite{zhang2019deep}Deep learning based recommender system: A survey and new perspectives里面将RNN在推荐系统中的应用分为三种:Session-based Recommendation without User Identifier, Sequential Recommendation with User Identifier和Feature Representation Learning with RNNs。其中不要求用户注册和登录的应用或网站没有用户标识,这些系统通常使用第一种方式通过the session或cookie获取用户的短期偏好。可以获取到用户标识的系统则使用第二种方式建模序列推荐任务,同时RNN也可作为一种特征表示的学习方式。RNN在搜索中的应用与在推荐场景中类似\cite{zhang2014sequential}。

\subsection{Deep Reinforcement Learning}
\label{subsec:rw:rl}

While the original reinforcement learning idea was proposed more than 3 decades ago, there has been a strong resurgence of interest in the past few years, thanks in part to its successful application in playing Atari games \cite{mnih2013playing}, DeepMind's AlphaGo \cite{silver2017mastering} and in text generation domains \cite{chen2019reinforcement,gong2019reinforcement}. Both lines of work achieve either super-human level or current state-of-the-art performance on a wide range of indisputable metrics.  

Several important technical milestones include Double DQN \cite{van2016deep} to mitigate over-estimation of Q value, and \cite{schaul2015prioritized}, which introduces experience replay. However, most of the work focuses on settings like gaming and robotics. We did not adopt experience replay in our work because of its large memory requirement, given the billion example scale at which we operate. 

The application in personalized search and recommendation has been more recent. Majority of the work in this area focuses on sequential recommendation such as \cite{zhao2018deep} as well as ads placement within search and recommendation results \cite{zhao2021dear}.

An interesting large scale off-policy recommendation work is presented in \cite{chen2019top} for youtube recommendation. They make heuristic correction of the policy gradient calculation to bridge the gap between on-policy and off-policy trajectory distributions. We tried it in our problem with moderate offline success, though online performance was weaker, likely because our changing user queries make the gradient adjustment less accurate.

Several notable works in search ranking include \cite{hu2018reinforcement} which takes an on-policy approach and \cite{xu2020reinforcement} which uses pairwise training examples similar to ours. However both works consider only a single query session, which is similar to the sequential recommendation setting, since the query being fixed can be treated as part of the user profile. In contrast, our work considers the user interactions on a search platform over an extended period of time, which typically consist of hundreds of different query sessions. 




% Several notable works include xyz (dawei yin, . Despite the dissimilarity between search ranking and recommendation tasks, namely the existence of a query, the problem setup is often quite similar. In particular, for learning to rank problems, typically only a single query session is considered. Many of the ranking related RL papers also focus on relevance learning, with a notable exception of xyz (Hu from alibaba). The latter however also considers a single query as the entire user history. Thus to the best of our knowledge, multi-query cross session reinforcement learniing has not been fully explored under RL.

% We also mention the work in the personalized Ads targeting domain, from ByteDance. Since it deals with the mixture of Ads and recommendation results, it does not directly address the homogeneous ranking problem that we face.


% \begin{itemize}
% \item Value-Based RL —DQN及其改进算法
% \begin{itemize}
% \item 论文1\cite{mnih2013playing}: Playing Atari with Deep Reinforcement Learning【深度强化学习开山之作】第一次提出了DQN算法,是深度强化学习真正意义的开山之作,算法用卷积神经网络构造Q网络,网络输入经过处理后的最近4帧游戏画面,输出在这种状态下执行各种动作的Q函数值。在绝大部分游戏上,DQN超过了之前最好的算法,在部分游戏上,甚至超过了人类玩家的水平;
% \item 论文2\cite{mnih2015human}: Human-level control through deep reinforcement learning 对论文1进行改进,构建目标Q网络,目标Q网络和Q网络之间周期性同步参数,提升了算法的收敛性;
% \item 论文3\cite{van2016deep}:  Deep reinforcement learning with double q-learning【Duoble DQN】 使用当前值网络的参数θ选择最优动作,用目标值网络的参数θ-评估该最优动作,将动作选择和策略评估分离,降低了过高估计Q值的风险;
% \item 论文4\cite{schaul2015prioritized}: Prioritized experience replay【Prioritized replay 样本采样方式优化】基于优先级采样的DQN,是对经验回放机制的改进,为经验池中的每个样本计算优先级,增大有价值的训练样本在采样时的概率。加快收敛速度和提升效果;
% \item 论文5\cite{wang2016dueling}: Dueling network architectures for deep reinforcement learning【Dueling networks】将CNN卷积层之后的全连接层替换为两个分支,其中一个分支拟合状态价值(state values)函数V(s),另外一个分支拟合动作优势(action advantages)函数A(s,a)。最后将两个分支的输出值相加,形成Q函数值。这种改进能够更准确的估计Q值。
% \item 论文6\cite{hausknecht2015deep}:Deep recurrent q-learning for partially observable MDPs. 在CNN的卷积层之后加入LSTM单元,记住之前的信息。
% \item 论文7\cite{hessel2018rainbow}: Rainbow: Combining Improvements in Deep Reinforcement Learning 整合了DQN的诸多优化。包括Double Q-learning,Prioritized replay,Dueling networks,Multi-step learning,Distributional RL 和 Noisy Nets.
% \item DQN适用范围:DQN是求每个action的\(max_aQ(s,a)\),适用于低维,离线的动作空间,在连续空间不适用。
% \end{itemize}
% \item Policy-Based RL算法
% \begin{itemize}
% \item  policy based RL 直接对策略建模,通过reward来直接对策略进行更新,使得累计回报最大。适用于连续的动作空间,但是无法衡量策略究竟是不是最优,策略评估高方差。
% \item  最开始的reinforcement算法\cite{williams1992simple} Simple Statistical Gradient-Following Algorithms for Connectionist Reinforcement Learning[1992, Williams] 
% \item  策略梯度算法\cite{sutton1999policy} Policy Gradient Methods for Reinforcement Learning with Function Approximation[2000, Sutton]
% \end{itemize}

% \item Actor-Critic算法
% \begin{itemize}
% \item 结合Policy-Based RL和Value-Based RL方法,基本上解决了高维状态与动作空间的问题,并使性能有明显的提升。但原始的Actor-Critic方法对于复杂问题可能会不稳定。
% \item DPG算法\cite{silver2014deterministic}:Deterministic Policy Gradient Algorithms【DPG】
% \item DDPG算法\cite{lillicrap2015continuous}:Continuous Control With Deep Reinforcement Learning【DDPG】
% \item TD3算法\cite{fujimoto2018addressing}:Addressing Function Approximation Error in Actor-Critic Methods【Twin Delayed Deep Deterministic policy gradient TD3】
% \end{itemize}

% \end{itemize}

% \label{subsec:rw:rl}

% \subsection{Reinforcement Learning in search and recommendation、}
% \begin{itemize}
% \item 传统的搜索/推荐算法
% \begin{itemize}
% \item 认为搜索/推荐是一个静态的过程;
% \item 建模即时reward,仅考虑当前的商品是否被点击/购买,忽略长期价值;
% \end{itemize}
% \item 强化学习在推荐中的应用
% \begin{itemize}
% \item  MDP推荐系统\cite{shani2005mdp}:An MDP-Based Recommender System
% \item  RL推荐系统(最大化长期价值)\cite{theocharous2015ad}:Personalized ad recommendation systems for life-time value optimization with guarantees
% \begin{itemize}
% \item 把个性化⼴告推荐系统定义为强化学习问题,最⼤化⽣命周期值(life-time value)
% \end{itemize}
% \item  DRN新闻推荐\cite{zheng2018drn}:DRN:A deep reinforcement learning framework for news recommendation
% \begin{itemize}
% \item 基于【DQN】的推荐算法
% \item 主要贡献:1.提出基于deep Q-learning的推荐框架,该框架可以明确建模长期reward(MAB-based方法不更清晰地给出future reward,MDP-based方法不适用于大规模数据);2.引入用户活跃度(用户返回APP的情况),作为点击/不点击标签的补充,从而获取更多的的用户反馈信息;3.加入了探索策略(采用Dueling Bandit Gradient Descent方法挑选当前推荐环境下候选items),为用户寻找新的有吸引力的新闻;
% \end{itemize}
% \item Listwise推荐\cite{zhao2017deep}:Deep reinforcement learning for list-wise recommendations
% \begin{itemize}
% \item 基于【DDPG】的推荐算法
% \item 主要缺点:使用全联接网络表示用户状态,不能很好的建模用户和商品之间的关系
% \end{itemize}
% \item Pagewise推荐\cite{zhao2018deep}:Deep Reinforcement Learning for Page-wise Recommendations
% \begin{itemize}
% \item 基于【DDPG】的推荐算法
% \item 主要贡献:生成每个推荐物品的同时,也决定每个推荐维度在二维屏幕上的位置
% \end{itemize}
% \item 考虑正负反馈的推荐\cite{zhao2018recommendations}:Recommendations with Negative Feedback via
% Pairwise Deep Reinforcement Learning
% \begin{itemize}
% \item 基于【DQN】的推荐算法
% \item 主要贡献:考虑负反馈以及商品的偏序关系,并将这种偏序关系建模到DQN的loss函数中。用户跳过或者是没有任何行为的商品,不仅能够影响用户的行为,还可以让我们更好的了解用户的偏好。
% \end{itemize}
% \item User-Item Interactions Modeling\cite{liu2018deep} : Deep Reinforcement Learning based Recommendation with Explicit User-Item Interactions Modeling
% \begin{itemize}
% \item 基于【DDPG】的推荐算法
% \item 主要贡献:状态表征模块设计,文中强调了状态表征的重要性,并设计了三种状态表征模块;DRR-p:商品之间组pair,pair内item embedding相乘,和原始商品信息concat;DRR-u:用户embedding和item embedding相乘,同时concat商品pair相乘结果;DRR-ave:DRR-u基础上item处考虑position weight,并经过average pooling;
% \item 不足之处:虽然设计了状态表征模块,但是仅使用了用户最近的N个正反馈商品
% \end{itemize}
% \end{itemize}

% \item 强化学习在搜索中的应用
% \begin{itemize}
% \item 阿里RL搜索排序\cite{hu2018reinforcement}:Reinforcement learning to rank in e-commerce search engine: Formalization, analysis, and application
% \end{itemize}

% \end{itemize}

\label{subsec:rw:rl_search}

% mainly low rank approximation and pq based.


\section{Preliminary}
\label{sec:prelimiary}

% \subsection{Problem Setting}
\noindent \textbf{Problem Setting.}
Zero-shot segmentation aims at training a segmentation model capable of segmenting novel objects using text descriptions. Given two category sets $C_{seen}$ and $C_{unseen}$ respectively, where $C_{seen}$ and $C_{unseen}$ are disjoint in terms of object categories ($C_{seen} \cap C_{unseen} = \emptyset$). The model is trained on $C_{seen}$ and directly tested on both $C_{seen}$ and $C_{unseen}$. Typically, $C_{seen}$ and $C_{unseen}$ are described with semantic words (\textit{e.g.} sheep, grass).

% \subsection{Revisiting the "frozen CLIP" paradigm}
\noindent \textbf{Revisiting the "frozen CLIP" paradigm.}
\label{sec:Revisiting}
The "frozen CLIP" approaches \cite{zegformer, zsseg, freeseg, ovseg} execute zero-shot segmentation in two steps: mask proposals generation and mask proposals classification. 
In the first step, these approaches train a Proposal Generator to generate $N$ class-agnostic mask proposals (denoting as $M$, $M \in \mathbb{R}^{N \times H \times W}$) and their corresponding classification scores (denoting as $A^{p}$, $A^{p} \in \mathbb{R}^{N \times |C_{seen}|}$). MaskFormer \cite{cheng2021maskformer} and Mask2Former \cite{cheng2021mask2former} are generally used as the Proposal Generator since the Hungarian matching \cite{kuhn1955hungarian} in the training process makes the mask proposals strongly generalizable.
In the second step, $N$ suitable sub-images ($I_{sub}$) are obtained by \textit{merging} $N$ mask proposals and the input image. $I_{sub}$ is then fed into the CLIP Image Encoder to obtain the image embedding ($E^I$). Meanwhile, text embedding ($E^T$) is generated by a CLIP Text Encoder. The classification score ($A^{c}, A^{c}  \in \mathbb{R}^{N \times C}$) predicted by CLIP is calculated as:
\begin{equation}
\label{eq:prob} 
A^{c}_i = \mathrm{Softmax}(\frac{\exp(\frac{1}{\tau}s_{c} (E^T_{i}, E^I))}{\sum_{i=0}^{C}\exp(\frac{1}{\tau}s_{c}(E^T_{i}, E^I))}), i = [1,2,...C]
\end{equation}
where  $\tau$ is the temperature hyper-parameter. $s_{c}(E^T_{i}, E^I)=\frac{E^T_{i} \cdot E^I }{|E^T_{i}| |E^I|}$ represents the cosine similarity between $E^T_{i}$ and $E^I$. $C$ is the number of classes, with $C = |C_{seen}|$ during training and $C = |C_{seen}\cup C_{unseen}|$ during inference. Noting that CLIP is frozen when training to avoid overfitting.

To further enhance the reliability of $A^{c}$, the classification score of the Proposal Generator ($A^{p}$) is ensembled with $A^{c}$ since $A^{p}$ is more reliable on seen classes. This \textit{ensemble} operation is wildly used in "frozen CLIP" approaches.  The pipeline of "frozen CLIP", as well as the \textit{merge} and \textit{ensemble} operations, are described in detail in the Appendix. 

Although  "frozen CLIP" approaches have achieved promising results, it is clear that directly adopting an image-level pre-trained CLIP for proposal classification can be suboptimal. A frozen CLIP usually produces numerous false positives, and the \textit{merge} operation may destroy the context information of an input image. In view of this, we rethink the paradigm of the frozen CLIP and explore a new solution for proposal classification.

\section{Methodology}
\label{sec:method}

\begin{figure}
% \vspace{-15mm}
\begin{center}
   \includegraphics[width=0.99\linewidth]{figs/pdf/fintune.pdf}
\end{center}
% \vspace{-2mm}
   \caption{
    Overview of the Mask-Aware Fine-tuning (MAFT). 
    In IP-CLIP Encoder, we modify the CLIP Image Encoder, and apply the mask proposals as attention bias in Multihead Attention from the $L^{th}$ layer. The final projection unit is an MLP module used for reshaping the channels of $F_{cls}$. \textit{w.o.} $M$ denotes IP-CLIP Encoder processes image without utilizing mask proposals ($M$). \textit{Mask-aware} Loss is designed to train CLIP to be mask-aware, while \textit{Self-distillation} Loss is designed to maintain the transferability. Only the IP-CLIP Encoder is trained (\textcolor{orange}{orange} part), the Proposal Generator and the CLIP Text Encoder are frozen (\textcolor{cyan}{blue} part).
   }
\label{fig:finetune}
% \vspace{-5mm}
\end{figure}


We introduce Mask-Aware Fine-tuning (MAFT), a method for learning mask-aware CLIP representations. 
Within MAFT, we first propose the Image-Proposal CLIP Encoder (IP-CLIP Encoder) to handle images with any number of mask proposals simultaneously (Sec. \ref{sec:IP-CLIP}). Then, \textit{mask-aware loss}  and \textit{self-distillation loss}  are introduced to fine-tune the IP-CLIP Encoder and make it distinguishable for different mask proposals while maintaining transferability (Sec. \ref{sec:Mask-aware tuning}).
The complete diagram of the MAFT is shown in Fig.~\ref{fig:finetune}, we use the ViT-B/16 CLIP model for illustration.


\subsection{Image-Proposal CLIP Encoder (IP-CLIP Encoder)}
\label{sec:IP-CLIP}
IP-CLIP Encoder aims to process arbitrary numbers of images and mask proposals simultaneously. We draw inspiration from MaskFormer \cite{cheng2021mask2former, cheng2021maskformer}, which uses attention-masks in Multihead Attention and provides the flexibility for accepting any number of queries and features of different masked regions. Accordingly, we apply mask proposals as attention-masks in Multihead Attention and designate independent classification queries for each mask proposal.
 % We draw inspiration from MaskFormer \cite{cheng2021mask2former, cheng2021maskformer}, which uses attention-masks to calculate Multihead Attention between queries and features.  It provides the flexibility for Multihead Attention to accept any number of queries and features of different masked regions.
 
In the IP-CLIP Encoder shown in Fig. \ref{fig:finetune}, we denote the features propagate between Transformer layers as $F^i$, where $i = [1,2...12]$. We can express $F^i$ as $F^i = [F^i_{cls};~ F^i_{feat}], \in \mathbb{R}^{(1 + hw) \times d}$, here $1$ represents a class-embedding vector ($F^i_{cls}$), $hw$ represents the number of the flattened image features ($F^i_{feat}$). 
% The output class-embedding vector $F^{12}_{cls}$ is utilized for classification (equals to $E^I$ in Sec. \ref{sec:Revisiting}). 
To obtain the classifications of all mask proposals simultaneously, we repeat $F^i_{cls}$ at layer $L$ $N$ times, where $N$ is the number of mask proposals, denoting the repeated class-embedding vectors as $F^{i*}_{cls}$. We can express the modified features ($F^{i*}$) as $F^{i*} = [F^{i*}_{cls};~ F^i_{feat}], \in \mathbb{R}^{(N + hw) \times d}$.

% Therefore, in the first $L$ Transformer layers, the propagation of $F^{i}$ keeps same with standard CLIP,
% \begin{equation}
%    F^{i+1} =\mathrm{TLayer}^i(F^i)
% \end{equation}
% $\mathrm{TLayer}^i$ denote i$^{th}$ Transformer layer. We simplify the representation of
% Transformer layer, whereas the start $L$ Transformer layers conduct the same structure with CLIP.

% Thus a standard Multihead Attention process in Transformer layers can be formulated as follows:
% \begin{equation}
%    \mathrm{MHAtten}(F^i) =\mathrm{Softmax}(\frac{Que(F^i)Key(F^i)^T}{\sqrt{d}})Val(F^i)
% \end{equation}
% where $Que(\cdot)$, $Key(\cdot)$, and $Val(\cdot)$ denote linear projections, $d$ is the hidden dimension of $F^i$. 

%  \begin{equation}
%     F^{i*} =[F^{i*}_{cls};~ F^i_{feat}], F^{i*} \in \mathbb{R}^{C \times (N + hw)}
% \end{equation}
\noindent \textbf{Propagation of $F^{i}$, where $i = [1, 2, ...L]$.}
We consider that CLIP's classification significantly relies on context information. In the first $L$ Transformer layers, the propagation of $F^{i}$ is the same as in standard CLIP. Specifically, $F^{i}_{cls}$ utilizes cross-attention with all pixels within $F^{i}_{feat}$, effectively retaining the context information. 

In the subsequent $12-L$ Transformer layers, the propagation of $F^{i*}$ can be partitioned into two parts: the propagation of $F^{i*}_{cls}$ and the propagation of $F^{i}_{feat}$.

\noindent \textbf{Propagation of $F^{i*}_{cls}$.}
We use $F^{i*}_{cls}$[$n$] and $M$[$n$] to represent the position $n$ in $F^{i*}_{cls}$ and $M$, where $n=[1,2...N]$. It is expected $F^{i*}_{cls}$[$n$] computes Multihead Attention for the positions where $M$[$n$]$=1$ and itself. To achieve this, we construct an attention bias $B \in \mathbb{R}^{N \times (N+hw)}$ as follows:
\begin{equation}
B_{(i,j)}=\left\{
\begin{aligned}
0  &, \mathrm{if} ~ {\hat{M}}_{(i,j)} = 1\\
-\infty  &, \mathrm{if} ~ {\hat{M}}_{(i,j)} = 0\\
\end{aligned}
\right.
,~~~ \hat{M} = [\mathrm{I}(N,N);~ \mathrm{Flat}(M)]
\end{equation}
here $\mathrm{I}(N,N)$ denotes $N^{th}$ order identity matrix, $\mathrm{Flat}$($\cdot$) denotes the \textit{flatten} operation. $\hat{M}$ is an intermediate variable for better representation. Therefore, a masked Multihead Attention is used for propagating $F^{i*}_{cls}$ 
% \footnote{We omit the MLP Layer and some Layer Normalizations in Transformer layers to  simplify the representation.}
:
\begin{equation}
   F^{(i+1)*}_{cls} =\mathrm{Softmax}(\frac{\mathrm{Que}(F^{i*}_{cls})\mathrm{Key}(F^{i*})^T}{\sqrt{d}} + B)\mathrm{Val}(F^{i*})
   \label{con:modified tlayers1}
\end{equation}
where $\mathrm{Que}(\cdot)$, $\mathrm{Key}(\cdot)$, and $\mathrm{Val}(\cdot)$ denote linear projections, $d$ is the hidden dimension of $F^{i*}$. Notably, We omit the MLP Layer and Layer Normalizations in Transformer layers to simplify the representation in Eq. \ref{con:modified tlayers1} and Eq. \ref{con:modified tlayers2}.

\noindent \textbf{Propagation of $F^{i}_{feat}$.}
A standard Multihead Attention is used for propagating $F^{i}_{feat}$ 
% \footnote{Similar to Eq. \ref{con:modified tlayers}, MLP Layer and Layer Normalizations are simplified.}
: 
\begin{equation}
   F^{i+1}_{feat} =\mathrm{Softmax}(\frac{\mathrm{Que}(F^{i}_{feat})\mathrm{Key}(F^{i}_{feat})^T}{\sqrt{d}})\mathrm{Val}(F^{i}_{feat})
   \label{con:modified tlayers2}
\end{equation}
Therefore, for any given mask proposal $M$[$n$], the corresponding class-embedding $F^{i*}_{cls}$[$n$] only performs Multihead Attention with $F^{i}_{feat}$ where $M$[$n$]$=1$ and $F^{i*}_{cls}$[$n$]. The propagation of $F^{i}_{feat}$ remains undisturbed by attention-masks. Compared with the frozen CLIP,  IP-CLIP Encoder leverages context information effectively and reduces computational costs.



\subsection{Objective}
\label{sec:Mask-aware tuning}
IP-CLIP Encoder with CLIP pre-trained parameters remains challenging in distinguishing different mask proposals, \textit{e.g.}, when the proposals contain more background regions than foreground objects, IP-CLIP may tend to classify them into the foreground categories. To overcome this limitation, we introduce \textit{mask-aware loss} and \textit{self-distillation loss} to fine-tune the IP-CLIP Encoder to be mask-aware without sacrificing transferability. 

We conduct the \textit{mask-aware} loss function ($\mathcal{L}_{ma}$) on $A^c$.  The goal is to assign high scores to high-quality proposals and low scores to low-quality proposals in $A^c$. Concretely, we use the Intersection over Union (IoU) score obtained from ground-truth and align it with the $A^c$ to prompt CLIP to become mask-aware. Assuming there are $k$ classes in ground-truth, we can generate $k$ binary maps of ground-truth and calculate the IOU score ($S_{IoU}$) with $N$ mask proposals. We identify a discrepancy between the maximum values of $A^c$ and $S_{IoU}$. The maximum value of $A^c$ tends to approach 1, whereas the maximum value of $S_{IoU}$ ranges from 0.75 to 0.99. This inconsistency can hinder the alignment between these two metrics. Therefore, we introduced a min-max normalization technique for $S_{IoU}$ as follows:
\begin{equation}
S_{IoU}^{norm} = \frac{S_{IoU} - min(S_{IoU})}{max(S_{IoU}) - min(S_{IoU})},  S_{IoU}\in \mathbb{R}^{K \times N}
\end{equation}
Meanwhile, we select $k$ pre-existing classes in $A^c$ ($A^c_{select}, A^c_{select}\in \mathbb{R}^{K \times N}$), and employ $SmoothL1$ Loss to align it with $S_{IoU}^{norm}$. Therefore, $\mathcal{L}_{ma}$ can be formulated as follows:
\begin{equation}
\mathcal{L}_{ma}(A^c_{select}, S_{IoU}^{norm}) = \mathrm{SmoothL1} (A^c_{select}, S_{IoU}^{norm})
\end{equation}
\begin{equation}
\mathrm{SmoothL1}(x, y) = \left\{
\begin{aligned}
 0.5\cdot (x - y)^2  &, ~~~ \mathrm{if} ~ |x - y| < 1\\
|x - y| - 0.5  &, ~~~ \mathrm{otherwise} ~ \\
\end{aligned}
\right.
\end{equation}

In addition to $\mathcal{L}_{ma}$, we also introduce a \textit{self-distillation} loss $\mathcal{L}_{dis}$ to maintain CLIP's transferability and alleviate overfitting on $C_{seen}$. 
Within $\mathcal{L}_{dis}$, we use a frozen CLIP as the \textit{teacher} net, the  IP-CLIP as the \textit{student} net for self-distillation.
The predictions of the frozen CLIP and IP-CLIP are expected to be the same when no mask is included. Denoting the output of the frozen CLIP as $A_{T}$, and the output of the fine-tuned IP-CLIP without masks as $A_{S}$. We use $SmoothL1$ Loss to minimize the difference as follows:
\begin{equation}
\mathcal{L}_{dis}(A_{S}, A_{T}) = \mathrm{SmoothL1} (A_{S}, A_{T})
\end{equation}
It is important to note that when processing an image through IP-CLIP without mask proposals, the resulting $A_{S}$ is a matrix with dimensions $\mathbb{R}^{C \times 1}$.
Therefore, the final loss function can be formulated as: $\mathcal{L} = \mathcal{L}_{ma} + {\lambda} {\mathcal{L}_{dis}}$, where we set the constant $\lambda$ to 1 in our experiments. The mask-aware fine-tuning process is efficient as we only perform a few iterations (less than 1 epoch).

\section{Experiments}
\label{sec:exp}
% \section{Bib\TeX{} Files}
% \label{sec:bibtex}

% Unicode cannot be used in Bib\TeX{} entries, and some ways of typing special characters can disrupt Bib\TeX's alphabetization. The recommended way of typing special characters is shown in Table~\ref{tab:accents}.

% Please ensure that Bib\TeX{} records contain DOIs or URLs when possible, and for all the ACL materials that you reference.
% Use the \verb|doi| field for DOIs and the \verb|url| field for URLs.
% If a Bib\TeX{} entry has a URL or DOI field, the paper title in the references section will appear as a hyperlink to the paper, using the hyperref \LaTeX{} package.
\respace
\respace
\section{Experiments}
\respace
\label{sec:experiments}

Experiments are conducted on two common NLP tasks: question answering (QA) and sentiment analysis (SA), each with several available domains.

\respace
\paragraph{Question Answering} 
We employ 6 diverse QA datasets from the MRQA 2019 workshop \citep{fisch2019mrqa}, shown in Table~\ref{datasets}.\footnote{The workshop pre-processed all datasets into a similar format, for fully answerable, span-extraction QA: \url{https://github.com/mrqa/MRQA-Shared-Task-2019}.}
We sample 60k examples from each dataset for training, 5k for validation, and 5k for testing. 
Questions and contexts are collected with varying procedures and sources, representing a wide diversity of datasets.

\section{Low-Voltage Load Forecasting Datasets} 
\label{secdatasets}

A number of interesting features were discovered about the data in the reviewed 221 papers. Firstly, only 52 use at least one openly available datasets to illustrate the results, i.e. less than 24\% of the journals presented results that could be potentially replicated by the wider research community. Of these 52 papers using open data, 22 (or $42\%$)  of them used the Irish CER Smart Metering Project data~\cite{Commission2012csm}, four used data from UK Low Carbon London project~\cite{UK2014ulc}, four from Ausgrid\footnote{\url{https://www.ausgrid.com.au/Industry/Our-Research/Data-to-share/Solar-home-electricity-data}} and three used the UMass dataset. In other words, out of the papers using open data, $56\%$, presented results that used data from only four open data sets. 

The overuse of a particular dataset can result in biases (both conscious and unconscious) where methods are developed and tested but the features of the data may be well known or familiar from overuse. In these cases a scientifically rigorous experiment is impossible. Further, reliance on a single dataset (especially those which are no more than 2 years in length like the Irish CER dataset~\cite{Commission2012csm}) risks the development of models which may be based on spurious features and patterns and may not be representative of the wider energy system. This can be alleviated somewhat by including multiple open data sets, as it was done in some of the papers reviewed (\cite{abera2020mla, Laurinec2019due, Wang2018aef}). To support the LV forecast research community the authors are going to share a modifiable list of open data sets at the LV level. A list and some of the properties of the data are shown in Table \ref{tab:datasets}.  

With the rapid change in low carbon technologies being connected to the grid and home, new energy efficiency interventions, and adjustments in demand usage behaviours, demand data can quickly become irrelevant or unrepresentative. Further, such data sets are based on trials where participants are subject to incentives or other interventions. For example, different tariffs were considered for some households in the Irish CER dataset~\cite{Commission2012csm}. This means that their demand may not represent `normal', every-day behaviour.  

There are now some initiatives that are attempting to solve some of the issues of sparse and intermittent open data produced by limited innovation projects. An example of this is the Smart Energy Research Lab from UCL in the  UK\footnote{\url{https://www.ucl.ac.uk/bartlett/energy/research/energy-and-buildings/smart-energy-research-group-serg}} which is attempting to make smart meter data (as well as other useful data sets associated with the same homes) available on an ongoing basis for research from a wider range of participants as well as a large control group of households which will not participate in any initiatives or trials. 

\subsection{Data resolution}
Resolution is another important aspect of the data. Half-hourly data is the standard resolution of smart meter data although it can be as low as 10 or 15 minutes. The smallest stated resolution of the data within the papers had 48 with hourly data, 51 half-hourly, 23 as 15 minutes and 8 as 10 minutes. There was also a large number of papers where the resolution was not clear or no real data was presented (62 papers). Data with resolutions of between ten minutes and an hour are probably sufficient for demand control applications and are likely representative of what data is available in practice. However, this isn't sufficient for more high-resolution applications such as voltage control. In fact, only 11 papers considered data of resolution of 1 minute or less. This could pose a difficulty for validating the common voltage and Var control application in this review (see Section~\ref{sec:LVLF-applications}). A large number of papers where the resolution is not clear should also be a concern, as this prevents recreation of the results. 

\subsection{Forecast horizon}
Another crucial aspect of this review is the forecast horizon. Different horizons are useful for different applications. Short term (day to the week ahead) are typical for operational time scales, whereas long-term forecasts (over a year) are more useful for planning. The majority of the papers reviewed were at a short term time scale with 80 of the papers considering day ahead forecasts, another common horizon was an hour ahead. Very few papers went at shorter horizons than an hour (twelve). There were slightly more papers that forecast beyond a day (16 were between 2 days and a week ahead) and only 13 papers were at horizons of a month or more. Once again there was a large number of papers, 80 in total, where the horizon length was not identifiable. 

\subsection{Overview of LV datasets}
As we have already mentioned, the current choice of open datasets that can be used for a benchmark is very limited, relying mostly on the CER Irish smart meter data, which is now a decade old and has some selection bias limitations (most of the houses have 3-4 bedrooms etc.). In order to continually expand research in this area, a strategy is required to regularly open more diverse datasets, converging towards common formats and standards and clarifying licences and terms of usage. Clear license information is especially relevant for industry-based research. We have established a list of open datasets (see Table~\ref{tab:datasets} and \url{https://low-voltage-loadforecasting.github.io/}) with the hope that it will continue to grow, and that new methods for privacy and safety protection (anonymisation, aggregation, synthetic 'look alike' datasets etc.) will enable more availability of datasets in the future. Finally it is vital to provide proper and thorough documentation with the data sets. The quality of the datasets is not clear and in many cases any preprocessing or data-cleaning techniques are not provided with the data.

Most datasets are at the residential level collected from smart meters. More diverse datasets of non-residential customers and different grid levels (substations and transformers) are needed for better LV forecasting research. 

Some datasets that have been cited in the literature like PLAID~\cite{Medico2020ava} OCTES, BLUED~\cite{Anderson2012baf},  and DRED~\cite{UttamaNambi2015lle} were offline at the time of writing. The Pecanstreet Dataport~\cite{Pecan2018d} database was once publicly available for research but then closed, and now only a subset is still accessible. Other datasets are available for download but are hard to trace, as identifiers like a DOI or a paper to cite are not available. All of these obstruct the reproducibility of the research. Therefore, new datasets should be published on archiving platforms like IEEE data port~\footnote{\url{https://ieee-dataport.org/}}, Zenodo~\footnote{\url{https://zenodo.org/}}, Figshare~\footnote{\url{https://figshare.com/}}, or  arXiv~\footnote{\url{https://arxiv.org/help/submit\#datasets}}. A recent contribution to more reproducible time series research is the Monash Time Series Forecasting Archive~\cite{godahewa2021mts}. It contains the dataset of the UK Low Carbon London trial~\cite{UK2014ulc} and the UCI datasets~\cite{candanedo2017ddp, Hebrail2012ihe}.


\begin{sidewaystable*}
	\tiny
	\caption{Overview of Low-voltage Load Datasets (see online version for embedded hyperlink to the data set by clicking on the name).} \label{tab:datasets}
	\resizebox{!}{0.9\height}{
		\begin{tabular}{p{0.16\linewidth}p{0.05\linewidth}p{0.04\linewidth}p{0.05\linewidth}p{0.04\linewidth}p{0.04\linewidth}p{0.02\linewidth}p{0.02\linewidth}p{0.12\linewidth}p{0.14\linewidth}p{0.14\linewidth}}
			\\
			\toprule
			Name & Type  & No. Customers & Resolution & Duration & Intervention & Sub-metering & Weather avail. & Location & Other data provided & Access/Licence \\
			\midrule 
			\href{https://www.ea.tuwien.ac.at//projects/adres_concept/EN/}{ADRES}~\cite{einfalt2011kfa}
			& Households & 30    & 1 s   & 2 weeks & None  & No    & No    & Austria (Upper Austria) & Voltage & Free for Research (E-Mail) \\
			\href{https://www.ausgrid.com.au/Industry/Our-Research/Data-to-share/Solar-home-electricity-data}{Ausgrid Solar Home} &  Households &  300 & 30 min &   3 years & None &  No &  No &  Australia (NSW) &  &   No Licence \\
			\href{https://www.ausgrid.com.au/Industry/Our-Research/Data-to-share/Distribution-zone-substation-data}{Ausgrid substation data} &   Substation & 225 & 15 min &  20 years & None &  No &  No &  Australia (NSW) &  & No Licence \\
			\href{https://sourceforge.net/projects/greend/}{GREEND Electrical Energy Dataset (GREEND)}~\cite{monacchi2014gae} & Households & 8     & 1 s   & 3-6 months & None  & Yes   & No    & Austria, Italy & Occupancy, Building type & Free (Access Form) \\
			\href{https://archive.ics.uci.edu/ml/datasets/Appliances+energy+prediction}{UCI Appliances}~\cite{candanedo2017ddp} & Households & 1     & 10 min & 4.5 months & None  & No    & Yes   & Belgium (Mons) & Lights, Building information & Free (No Licence) \\
			\href{https://ieee-dataport.org/open-access/industrial-machines-dataset-electrical-load-disaggregation}{INDUSTRIAL MACHINES}\cite{Bandeira2018imd}
			& Industrial & 1 & 1 Hz  & 1 month & None  & Yes   & No    & Brasil (Minas Gerais) &       & CC BY \\
			\href{https://dataverse.harvard.edu/dataset.xhtml?persistentId=doi:10.7910/DVN/ZJW4LC}{Rainforest Automation Energy} \cite{Makonin2018rtr} & Households & 2     & 1 Hz  & 2 months & None  & Yes   & Yes   & Canada & Environmental, Heat Pump,  & CC BY \\
			\href{https://dataverse.harvard.edu/dataset.xhtml?persistentId=doi\%3A10.7910/DVN/FIE0S4\%20}{AMPds2} \cite{Makonin2016ata, Makonin2016ewa}  & Households & 1     & 1min  & 2 years & None  & Yes   & Yes   & Canada (Alberta) & Gas, Water, Building Type and Plan & CC BY \\
			\href{https://carleton.ca/sbes/publications/electric-demand-profiles-downloadable/}{Sustainable Building Energy Systems 2017} \cite{Johnson2017edf} & Households & 23    & 1 min & 1 year & None  & Yes   & No    & Canada (Ottawa) & Sociodemographic (Occupants, Age, Size) & Free (Attribution, E-Mail) \\
			\href{https://carleton.ca/sbes/publications/electric-demand-profiles-downloadable/}{Sustainable Building Energy Systems 2013} \cite{Saldanha2012mee} & Households & 12    & 1 min & 1 year & None  & Yes   & No    & Canada (Ottawa) & Sociodemographic (Occupants, Age, Size) & Free (Attribution, E-Mail) \\
			\href{https://data.lab.fiware.org/organization/9569f9bd-42bd-414f-b8d9-112553ea9dfb?tags=FINESCE}{FINESCE Horsens}
			& Households & 20    & 1 h   & several days & None  & Yes   & Yes   & Denmark (Horsens) & EV, PV, Heat Pump, Heating, Smart Home,  & CC BY-SA \\
			
			\href{https://archive.ics.uci.edu/ml/datasets/Individual+household+electric+power+consumption}{UCI Individual household electric power cons.}~\cite{Hebrail2012ihe} & Households & 1     & 1min  & 4 years & None  & Yes   & No    & France (Sceaux) & Reactive Power, Voltage & CC BY \\
			\href{https://mediatum.ub.tum.de/1375836}{BLOND-50}~\cite{Kriechbaumer2017bbo} & Commerical & 1     & 50 kHz & 213 days & None  & Yes   & No    & Germany &       & CC BY \\
			\href{https://mediatum.ub.tum.de/1375836}{BLOND-250}~\cite{Kriechbaumer2017bbo} & Commerical & 1     & 250 kHz & 50 days & None  & Yes   & No    & Germany &       & CC BY \\
			\href{https://zenodo.org/record/3855575\#.YKQgGKgzaUk}{Fresh Energy}~\cite{Beyertt2020fzb} & Households & 200   & 15 min & 1 year & Behaviorial & Yes   & No    & Germany & Agegroup, Gender of main customer & CC BY \\
			\href{https://data.lab.fiware.org/organization/9569f9bd-42bd-414f-b8d9-112553ea9dfb?tags=FINESCE}{FINESCE Factory} 
			& Industrial & 1     & 1 min & 2 days & None  & Yes   & No    & Germany (Aachen) & Machines & CC BY-SA \\
			\href{https://pvspeicher.htw-berlin.de/wp-content/uploads/MFH-Lastprofil_2014_17274_kWh.csv}{HTW Lichte Weiten}~\cite{htw2019ldb}
			& Households & 1 building & 15 minute & 1 year & None  & No    & No    & Germany (Berlin) &       & Free (No Licence) \\
			\href{https://pvspeicher.htw-berlin.de/veroeffentlichungen/daten/lastprofile/}{HTW Synthetic}~\cite{Tjaden2015rel} & Households & 74    & 1 s   & 1 year & None  & No    & No    & Germany (Representative) & Synthetic dataset merging & CC BY-NC \\
			\href{https://data.open-power-system-data.org/household_data/}{CoSSMic} \cite{Open2020dph} & Households, SME & 11    & 1min, 15min, 1H & 1-3 years & None  & Yes   & No    & Germany (South) & PV, EV, Type (Residential/SME) & CC BY \\
			\href{https://im.iism.kit.edu/sciber.php}{SciBER}~\cite{Staudt2018san} & Municipal & 107   & 15min & 3 years & None  & No    & No    & Germany (South) & Type (Office, Gym, ...) & CC BY \\
			\href{https://iawe.github.io/}{iAWE}~\cite{batra2013idi}
			& Households & 1     & 1 Hz  & 2 months & None  & Yes   & No    & India (New Delhi) & Water & Free (No Licence) \\
			\href{https://combed.github.io/}{COMBED}~\cite{Batra2014aco} & Commerical & 1     & 30 s  & 1 month & None  & Yes   & No    & India (New Delhi) &       & Free (No Licence) \\
			\href{http://www.ucd.ie/issda/data/commissionforenergyregulationcer/}{Irish CER Smart Metering Project data}~\cite{Commission2012csm} & Households, SME, Other & 3835  & 30min & 1.5 years & Tariff & No    & No    & Ireland & Type (Residential/SME/Other) & Free (Signed Access Form) \\
			
			\href{https://github.com/Nikasa1889/ShortTermLoadForecasting}{Hvaler Substation Level data}~\cite{DangHa2017lst} & Substation & 20    & 1 h   & 2 years & None  & No    & No    & Norway (Hvaler) &       & Free (No Licence) \\
			\href{http://web.lums.edu.pk/~eig/CXyzsMgyXGpW1sBo}{Energy Informatics Group Pakistan}~\cite{Pereira2014sap} & Households & 42    & 1 min & 1 year & None  & Yes   & No    & Pakistan & Sociodemographic (building properties, no of people, devices) & Free (No Licence) \\
			\href{https://archive.ics.uci.edu/ml/datasets/ElectricityLoadDiagrams20112014}{UCI Electricity Load Diagrams}~\cite{Godahewa2021ehd}
			& Different & 370   & 15 min & 2 years & None  & No    & No    & Portugal &       & Free (No Licence) \\
			
			\href{http://www.vs.inf.ethz.ch/res/show.html?what=eco-data}{Electricity Consumption and Occupancy (ECO)}~\cite{Christian2014ted, Wilhelm2015hom} & Households & 6     & 1 Hz  & 8 months & None  & Yes   & No    & Switzerland & Occupancy & CC BY \\
			\href{https://www.gov.uk/government/publications/household-electricity-survey--2}{Household Electricity Survey (HES)}~\cite{Zimmermann2012hes} & Households & ~{}250 & 2 min & 1 month (255) to 1 year (26) & None  & Yes   & No    & UK    & Consumer Archetype & Request \\
			\href{https://beta.ukdataservice.ac.uk/datacatalogue/studies/study?id=8634}{METER}~\cite{Grunewald2019muh} & Households & 529   & 1 min & 28 hours & None  & No    & No    & UK    & Activity data, Sociodemographic & Free for Research (Access Form) \\
			\href{https://datashare.ed.ac.uk/handle/10283/3647}{IDEAL Household Energy Dataset}~\cite{goddard2020ihe}
			& Households & 255   & 1 s     & 3 years & None  & Yes   & No    & UK    & Smart Home, Sociodemographic, energy awareness survey, room temperature and humidity, building characteristics & CC BY \\
			\href{http://www.networkrevolution.co.uk/resources/project-data/}{Customer-Led Network Revolution project data}~\cite{sidebotham2015cln} & Households, SMEs & ~{}12000 & 30 min & > 1 year & Time of Use & No    & No    & UK    & EV, PV, Heatpump, Tariff,  & CC BY-SA \\
			\href{https://jack-kelly.com/data/}{UK Domestic Appliance-Level Electricity (UK-DALE)}~\cite{Jack2015tud} & Households & 5     & 16 kHz, 1s & months, one house > 4 years & None  & Yes   & No    & UK (London area) &       & CC BY \\
			\href{https://data.london.gov.uk/dataset/smartmeter-energy-use-data-in-london-households}{UK Low Carbon London}~\cite{UK2014ulc,Godahewa2021lsm} & Households & 5567  & 30min & 2 years & Time of Use & No    &  No   & UK (London) & CACI Acorn group & Free (No Licence) \\
			\href{https://www.refitsmarthomes.org/datasets/}{REFIT}~\cite{Murray2016rel, Murray2017ael} & Households & 20    & 8 s   & 2 years & None  & Yes   & Yes   & UK (Loughborough) & PV, Gas, Water, Sociodemographic (Occupancy, Dwelling Age, Dwelling Type, No. Bedrooms) & CC BY \\
			\href{https://www.spenergynetworks.co.uk/pages/flexible_network_data_share.aspx}{Flexible Networks for a Low Carbon Future} & Substations & Several Secondary  & 30 min & 1 year & None  & No    & No    & UK (St Andrews, Whitchurch, Ruabon) &       & Free (Access Form) \\
			\href{https://ukerc.rl.ac.uk/DC/cgi-bin/edc_search.pl?GoButton=Detail\&WantComp=146\&\&RELATED=1}{NTVV Substations} & Substation & 316   & 5 s   & > 4 years & None  & No    & No    & UK (Thames Valley) &       & Open Access (Any purpose) \\
			\href{https://ukerc.rl.ac.uk/DC/cgi-bin/edc_search.pl?GoButton=Detail\&WantComp=147\&\&RELATED=1}{NTVV Smart Meter} & Buildings & 316   & 30 min & > 4 years & None  & No    & No    & UK (Thames Valley) &       & Open Access (Any purpose) \\
			
			\href{https://site.ieee.org/pes-iss/data-sets/}{IEEE PES Open Data Sets} 
			& Households, Commercial & 15    & 1 min, 5 min, 15 min & 2 weeks & None  & No    & No    & USA   & Connection limit & Free (No Licence) \\
			\href{http://redd.csail.mit.edu/}{Reference Energy Disaggregation Data Set (REDD)}~\cite{Kolter2011rap} & Households &  ~{}10 & 1 kHz & 3-19 days & None  & Yes   & No    & USA (Boston) & Voltage & Free (Attribution, E-Mail) \\
			\href{http://wzy.ece.iastate.edu/Testsystem.html}{Iowa Distribution Test Systems}~\cite{Bu2019atd} & Substation & 240 nodes & 1 H   & 1 year & None  & Yes   & No    & USA (Iowa) & Grid data & Free (Attribution) \\
			\href{https://www.pecanstreet.org/dataport/}{Pecanstreet Dataport (Academic)}~\cite{Pecan2018d} & Households & 30    & 1min, 15min, 1H & 2-3 years & None  & Yes   & Yes   & USA (mostly Austin and Boulder) & PV, EV, Water, Gas, Sociodemographic & Free for Research (Access Form) \\
			
			\href{https://neea.org/resources/rbsa-ii-combined-database}{Residential Building Stock Assessment}~\cite{Larson2014jua} & Households & 101   & 15 min & 27 months & None  & Yes   & No    & USA (North West Region) & Building Type (Single Family, Manufactured, Multifamily) & Free (Access Form) \\
			
			\href{http://lass.cs.umass.edu/projects/smart/}{SMART* Home 2017}~\cite{Barker2012sao} & Households & 7     & 1 Hz  & > 2 years & None  & Yes   & Yes   & USA (Western Massachussets) &       & Free (No Licence) \\
			\href{http://lass.cs.umass.edu/projects/smart/}{SMART* Apartment}~\cite{Barker2012sao} & Households & 114   & 1 min & 2 years & None  & No    & Yes   & USA (Western Massachussets) &       & Free (No Licence) \\
			\href{http://lass.cs.umass.edu/projects/smart/}{SMART* Occupancy}~\cite{Barker2012sao} & Households & 2     & 1 min & 3 weeks & None  & No    & No    & USA (Western Massachussets) & Occupancy & Free (No Licence) \\
			\href{http://lass.cs.umass.edu/projects/smart/}{SMART* Microgrid}~\cite{Barker2012sao} & Households & 443   & 1 min & 1 day & None  & No    & No    & USA (Western Massachussets) &       & Free (No Licence) \\
			\href{http://lass.cs.umass.edu/projects/smart/}{SMART* Home 2013}~\cite{Barker2012sao} & Households & 3     & 1 Hz  & 3 months & None  & Yes   & No    & USA (Western Massachussets) & Solar, Wind, Environmental, Smart Home, Voltage,  & Free (No Licence) \\
			\bottomrule
		\end{tabular}
	}
\end{sidewaystable*}


\paragraph{Sentiment Analysis}
For the sentiment analysis classification task, we follow \citep{blitzer-etal-2007-biographies} and \citep{ruder2018strong} by randomly selecting 6 Amazon multi-domain review datasets, as well as Yelp reviews \citep{asghar2016yelp} and IMDB movie reviews datasets \citep{maas-EtAl:2011:ACL-HLT2011}.~\footnote{\url{https://jmcauley.ucsd.edu/data/amazon/}, \url{https://www.yelp.com/dataset}, \url{https://ai.stanford.edu/~amaas/data/sentiment/}.}
Altogether, these datasets exhibit wide diversity based on review length and topic (see Table~\ref{datasets}).
We normalize all datasets to have 5 sentiment classes: very negative, negative, neutral, positive, and very positive. 
We sample 50k examples for training, 5k for validation, and 5k for testing.

\respace
\paragraph{Experimental Setup}
\label{sec:ex-setup}
To evaluate methods for the multi-domain active learning task, we conduct the experiment described in Section~\ref{sec:task} for each acquisition method, rotating each domain as the target set.
Model $M$, a BERT-Base model ~\citep{devlin2019bert}, is chosen via hyperparameter grid search over learning rate, number of epochs, and gradient accumulation.
The large volume of experiments entailed by this search space limits our capacity to benchmark performance variability due to isolated factors (the acquisition method, the target domain, or fine-tuning final models).
However, our hyper-parameter search closely mimics the process of an ML practitioner looking to select a best method and model, so we believe our experiment design captures a fair comparison among methods.
See Algorithm~\ref{alg:experiment} in Appendix Section \ref{sec:appendix-expdesign} for full details.






\section{Conclusion}
In this paper, we rethink the "frozen CLIP" paradigm in zero-shot segmentation and propose Mask-Aware Fine-Tune (MAFT) for fine-tuning CLIP. 
Firstly, IP-CLIP Encoder is proposed to handle images with any number of mask proposals. Then, $\mathcal{L}_{ma}$ and $\mathcal{L}_{dis}$ are designed for fine-tuning CLIP to be mask-aware without sacrificing its transferability. MAFT is plug-and-play and can be applied to any "frozen CLIP" approach. Extensive experiments well demonstrate the performance of various zero-shot segmentation methods is improved by plugging MAFT.

\textbf{Limitations.}
Our MAFT introduces a CLIP fine-tining framework to the research of zero-shot segmentation. However, the classification ability for novel classes is still limited by pre-trained vision-language models. How to further narrow this limitation is our future research focus.
\begin{figure}
% \vspace{-15mm}
\begin{center}
   \includegraphics[width=0.99\linewidth]{figs/pdf/vis-proposalv2-maskimage.pdf}
\end{center}
\vspace{-2mm}
   \caption{
   Visualizations of typical proposals \& top 5 $A^c$ by \textcolor{cyan}{frozen CLIP} and \textcolor{orange}{mask-aware CLIP}.
   }
\label{fig:vis-proposal}
% \vspace{-2mm}
\end{figure}

\begin{figure}
% \vspace{-3mm}
\begin{center}
   \includegraphics[width=0.99\linewidth]{figs/pdf/vis-finalv2.pdf}
\end{center}
\vspace{-3mm}
   \caption{
   Qualitative results. The models are trained with COCO-Stuff and directly tested on VOC2012, COCO, and ADE20K.
   }
\label{fig:vis-final}
% \vspace{-2mm}
\end{figure}



\newpage

\appendix
\section*{Appendix}
\newpage
\appendix
\section{Pricing equations}
\subsection{Credit default swap}
\label{CDS_pricing}
A credit default swap (CDS) is a contract designed to exchange credit risk of a Reference Name (RN) between a Protection Buyer (PB) and a Protection Seller (PS). PB makes periodic coupon payments to PS conditional on no default of RN, up to the nearest payment date, in the exchange for receiving from PS the loss given RN's default.

Consider a CDS contract written on the first bank (RN), denote its price $C_1(t, x)$.\footnote{For the CDS contracts written on the second bank, the similar expression could be provided by analogy.} We assume that the coupon is paid continuously and equals to $c$. Then, the value of a standard CDS contract can be given (\cite{BieleckiRutkowski}) by the solution of  (\ref{kolm_1})--(\ref{kolm_2})  with $\chi(t, x) = c$ and terminal condition
\begin{equation*}
	\psi(x) = 
	\begin{cases}
		1 - \min(R_1, \tilde{R}_1(1)), \quad (x_1, x_2) \in D_2, \\
		1 - \min(R_1, \tilde{R}_1(\omega_2)), \quad (x_1, x_2) \in D_{12}, \\		
	\end{cases}
\end{equation*}
where $\omega_2 = \omega_2(x)$ is defined in (\ref{term_cond}) and 
\begin{equation*}
	\tilde{R}_1(\omega_2) = \min \left[1, \frac{A_1(T) +  \omega_2 L_{2 1}(T)}{L_1(T) + \omega_2 L_{12}(T)}\right].
\end{equation*}
Thus, the pricing problem for CDS contract on the first bank is
\begin{equation}
\begin{aligned}
		& \frac{\partial}{\partial t} C_1(t, x) + \mathcal{L} C_1(t, x) = c, \\
		& C_1(t, 0, x_2) = 1 - R_1, \quad C_1(t, \infty, x_2) = -c(T-t), \\
		& C_1(t, x_1, 0) = \Xi(t, x_1) = 
		\begin{cases}
			c_{1,0}(t, x_1), & x_1 \ge \tilde{\mu}_1, \\
			1-R_1, & x_1 < \tilde{\mu}_i,
		\end{cases} \quad C_1(t, x_1, \infty) = c_{1,\infty}(t, x_1),\\
		& C_1(T, x) = \psi(x) = 
	\begin{cases}
		1 - \min(R_1, \tilde{R}_1(1)), \quad (x_1, x_2) \in D_2, \\
		1 - \min(R_1, \tilde{R}_1(\omega_2)), \quad (x_1, x_2) \in D_{12}, \\		
	\end{cases}
\end{aligned}
\end{equation}
where $c_{1,0}(t, x_1)$ is the solution of the following boundary value problem:
\begin{equation}
\begin{aligned}
		& \frac{\partial}{\partial t} c_{1, 0}(t, x_1) + \mathcal{L}_1 c_{1, 0}(t, x_1) = c, \\
		& c_{1, 0}(t, \tilde{\mu}_1^{<}) = 1 - R_1, \quad c_{1, 0}(t, \infty) = -c(T-t), \\
		& c_{1, 0}(T, x_1) = (1 - R_1) \mathbbm{1}_{\{\tilde{\mu}_1^{<} \le x_1 \le \tilde{\mu}_1^{=}\}}, 
\end{aligned}
\end{equation}
and $c_{1,\infty}(t, x_1)$ is the solution of the following boundary value problem
\begin{equation}
\begin{aligned}
		& \frac{\partial}{\partial t} c_{1, \infty}(t, x_1) + \mathcal{L}_1 c_{1, \infty}(t, x_1) = c, \\
		& c_{1, \infty}(t, 0) = 1 - R_1, \quad c_{1, \infty}(t, \infty) = -c(T-t), \\
		& c_{1, \infty}(T, x_1) = (1 - R_1) \mathbbm{1}_{\{x_1 \le \mu_1^{=}\}}.
\end{aligned}
\end{equation}

\subsection{First-to-default swap}
An FTD contract refers to a basket of reference names (RN). Similar to a regular CDS, the Protection Buyer (PB) pays a regular coupon payment $c$ to the Protection Seller (PS) up to the first default of any of the RN in the basket or maturity time $T$. In return, PS compensates PB the loss caused by the first default.

Consider the FTD contract referenced on $2$ banks, and denote its price $F(t, x)$. We assume that the coupon is paid continuously and equals to $c$. Then, the value of FTD contract can be given (\cite{LiptonItkin2015}) by the solution of  (\ref{kolm_1})--(\ref{kolm_2})  with $\chi(t, x) = c$ and terminal condition
\begin{equation*}
	\psi(x) = \beta_0  \mathbbm{1}_{\{x \in D_{12}\}} + \beta_1 \mathbbm{1}_{\{x \in D_{1}\}} + \beta_2 \mathbbm{1}_{\{x \in D_{2}\}},
\end{equation*}
where
\begin{equation*}
	\begin{aligned}
		\beta_0 = 1 - \min[\min(R_1, \tilde{R}_1(\omega_2), \min(R_2, \tilde{R}_2(\omega_1)], \\
		\beta_1 = 1 - \min(R_2, \tilde{R}_2(1)), \quad \beta_2 = 1 - \min(R_1, \tilde{R}_1(1)),
	\end{aligned}
\end{equation*}
and
\begin{equation*}
	\tilde{R}_1(\omega_2) = \min \left[1, \frac{A_1(T) +  \omega_2 L_{2 1}(T)}{L_1(T) + \omega_2 L_{12}(T)}\right], \quad \tilde{R}_2(\omega_1) = \min \left[1, \frac{A_2(T) +  \omega_1 L_{1 2}(T)}{L_2(T) + \omega_1 L_{21}(T)}\right].
\end{equation*}
with $\omega_1 = \omega_1(x)$ and $\omega_2 = \omega_2(x)$ defined in (\ref{term_cond}).

Thus, the pricing problem for a FTD contract is
\begin{equation}
\begin{aligned}
		& \frac{\partial}{\partial t} F(t, x) + \mathcal{L} F(t, x) = c, \\
		& F(t, x_1, 0) = 1 - R_2,  \quad F(t, 0, x_2) = 1 - R_1, \\
		& F(t, x_1, \infty) = f_{2,\infty}(t, x_1), \quad F(t, \infty, x_2) = f_{1,\infty}(t, x_2), \\
		& F(T, x) = \beta_0  \mathbbm{1}_{\{x \in D_{12}\}} + \beta_1 \mathbbm{1}_{\{x \in D_{1}\}} + \beta_2 \mathbbm{1}_{\{x \in D_{2}\}},
\end{aligned}
\end{equation}
where $f_{1,\infty}(t, x_1)$ and $f_{2,\infty}(t, x_2)$ are the solutions of the following boundary value problems
\begin{equation}
\begin{aligned}
		& \frac{\partial}{\partial t} f_{i, \infty}(t, x_i) + \mathcal{L}_i f_{i, \infty}(t, x_i) = c, \\
		& f_{i, \infty}(t, 0) = 1 - R_i, \quad f_{i, \infty}(t, \infty) = -c(T-t), \\
		& f_{1, \infty}(T, x_i) = (1 - R_i) \mathbbm{1}_{\{x_i \le \mu_i^{=}\}}.
\end{aligned}
\end{equation}

\subsection{Credit and Debt Value Adjustments for CDS}

Credit Value Adjustment and Debt Value Adjustment can be considered either unilateral or bilateral. For unilateral counterparty risk, we need to consider only two banks (RN, and PS for CVA and PB for DVA), and a two-dimensional problem can be formulated, while bilateral counterparty risk requires a three-dimensional problem, where Reference Name, Protection Buyer, and Protection Seller are all taken into account. We follow \cite{LiptonSav} for the pricing problem formulation but include jumps and mutual liabilities, which affects the boundary conditions.

\paragraph{Unilateral CVA and DVA}
The Credit Value Adjustment represents the additional price associated with the possibility of a counterparty's default. Then, CVA can be defined as
\begin{equation}
	V^{CVA} = (1- R_{PS}) \mathbb{E}[\mathbbm{1}_{\{\tau^{PS} < \min(T, \tau^{RN}) \}} (V_{\tau^{PS}}^{CDS})^{+} \, | \mathcal{F}_t],
\end{equation}
where $R_{PS}$ is the recovery rate of PS, $\tau^{PS}$ and $\tau^{RN}$ are the default times of PS and RN, and $V_t^{CDS}$ is the price of a CDS without counterparty credit risk.

We associate $x_1$ with the Protection Seller and $x_2$ with the Reference Name, then CVA can be given by the solution of  (\ref{kolm_1})--(\ref{kolm_2})  with $\chi(t, x) = 0$ and $\psi(x) = 0$. Thus,
\begin{equation}
\begin{aligned}
		& \frac{\partial}{\partial t} V^{CVA}+ \mathcal{L} V^{CVA} = 0, \\
		& V^{CVA}(t, 0, x_2) = (1 - R_{PS}) V^{CDS}(t, x_2)^{+}, \quad V^{CVA}(t, x_1, 0) = 0, \\
		& V^{CVA}(T, x_1, x_2) = 0.
\end{aligned}
\end{equation}

Similar, Debt Value Adjustment represents the additional price associated with the default and defined as
\begin{equation}
	V^{DVA} = (1- R_{PB}) \mathbb{E}[\mathbbm{1}_{\{\tau^{PB} < \min(T, \tau^{RN}) \}} (V_{\tau^{PB}}^{CDS})^{-} \, | \mathcal{F}_t],
\end{equation}
where $R_{PB}$ and $\tau^{PB}$ are the recovery rate and default time of the protection buyer.

Here, we associate $x_1$ with the Protection Buyer and $x_2$ with the Reference Name, then, similar to CVA,  DVA can be given by the solution of  (\ref{kolm_1})--(\ref{kolm_2}),
\begin{equation}
\begin{aligned}
		& \frac{\partial}{\partial t} V^{DVA}+ \mathcal{L} V^{DVA} = 0, \\
		& V^{DVA}(t, 0, x_2) = (1 - R_{PB}) V^{CDS}(t, x_2)^{-}, \quad V^{DVA}(t, x_1, 0) = 0, \\
		& V^{DVA}(T, x_1, x_2) = 0.
\end{aligned}
\end{equation}

\paragraph{Bilateral CVA and DVA}

When we defined unilateral CVA and DVA, we assumed that either protection  buyer, or protection seller are risk-free. Here we assume that they are both risky. Then, 
The Credit Value Adjustment represents the additional price associated with the possibility of counterparty's default and defined as
\begin{equation}
	V^{CVA} = (1 - R_{PS}) \mathbb{E}[\mathbbm{1}_{\{\tau^{PS} < \min(\tau^{PB}, \tau^{RN}, T)\}} (V^{CDS}_{\tau^{PS}})^{+} \, | \mathcal{F}_t],
\end{equation} 

Similar, for DVA
\begin{equation}
	V^{DVA} = (1 - R_{PB}) \mathbb{E}[\mathbbm{1}_{\{\tau^{PB} < \min(\tau^{PS}, \tau^{RN}, T)\}} (V^{CDS}_{\tau^{PB}})^{-} \, | \mathcal{F}_t],
\end{equation} 


We associate $x_1$ with protection seller, $x_2$ with protection buyer, and $x_3$ with reference name. Here, we have a three-dimensional process. Applying three-dimensional version of (\ref{kolm_1})--(\ref{kolm_2}) with $\psi(x) = 0, \chi(t, x) = 0$, we get
\begin{equation}
	\label{CVA_pde}
\begin{aligned}
		& \frac{\partial}{\partial t} V^{CVA} + \mathcal{L}_3 V^{CVA} = 0, \\
		& V^{CVA}(t, 0, x_2, x_3) = (1 - R_{PS}) V^{CDS}(t, x_3)^{+}, \\
		& V^{CVA}(t, x_1, 0, x_3 ) = 0, \quad V^{CVA}(t, x_1, x_2, 0)  = 0, \\
		& V^{CVA}(T, x_1, x_2, x_3) = 0,
\end{aligned}
\end{equation}
and
\begin{equation}
\label{DVA_pde}
\begin{aligned}
		& \frac{\partial}{\partial t} V^{DVA} + \mathcal{L}_3 V^{DVA} = 0, \\
		& V^{DVA}(t, 0, x_2, x_3) = (1 - R_{PB}) V^{CDS}(t, x_3)^{-}, \\
		& V^{DVA}(t, x_1, 0, x_3 ) = 0, \quad V^{DVA}(t, x_1, x_2, 0)  = 0, \\
		& V^{DVA}(T, x_1, x_2, x_3) = 0,
\end{aligned}
\end{equation}
where $\mathcal{L}_3 f$ is the three-dimensional infinitesimal generator.




\newpage

% {\small
\bibliographystyle{plain}
\bibliography{egbib}
% }



\end{document}