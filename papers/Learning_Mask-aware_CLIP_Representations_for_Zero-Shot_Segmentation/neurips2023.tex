\documentclass{article}
\pdfoutput=1

% if you need to pass options to natbib, use, e.g.:
%     \PassOptionsToPackage{numbers, compress}{natbib}
% before loading neurips_2023


% ready for submission
% \usepackage{neurips_2023}


% to compile a preprint version, e.g., for submission to arXiv, add add the
% [preprint] option:
\usepackage[preprint]{neurips_2023}
\usepackage{natbib}
\setcitestyle{numbers,square}

% to compile a camera-ready version, add the [final] option, e.g.:
% \usepackage[final]{neurips_2023}


% to avoid loading the natbib package, add option nonatbib:
%    \usepackage[nonatbib]{neurips_2023}


\usepackage[utf8]{inputenc} % allow utf-8 input
\usepackage[T1]{fontenc}    % use 8-bit T1 fonts
\usepackage[colorlinks=true,linkcolor=red,citecolor=green,urlcolor=red,]{hyperref}
\usepackage{hyperref}       % hyperlinks

\usepackage{url}            % simple URL typesetting
\usepackage{booktabs}       % professional-quality tables
\usepackage{amsfonts}       % blackboard math symbols
\usepackage{nicefrac}       % compact symbols for 1/2, etc.
\usepackage{microtype}      % microtypography
\usepackage{xcolor}         % colors

% my package
\usepackage{graphicx}
\usepackage{enumitem}
\usepackage{multirow}
% \usepackage[dvipsnames,table]{xcolor} 
\usepackage{colortbl}
\usepackage{arydshln}       % 负责画虚线的包

\usepackage{float}
\usepackage{subfig}
\usepackage{amsmath}
\usepackage{makecell}


\usepackage{appendix}
% \usepackage[table]{xcolor}

%\title{Mask-aware CLIP Fine-tuning for \\ Zero-Shot Segmentation}
\title{Learning Mask-aware CLIP Representations for \\ Zero-Shot Segmentation}

\author{%
  Siyu Jiao$^{1, 2, 3}$\thanks{Work done during an internship at Picsart AI Research (PAIR).},\quad Yunchao Wei$^{1, 2, 3}$, \quad Yaowei Wang$^{3}$, \quad Yao Zhao$^{1, 2, 3}$, \quad \textbf{Humphrey Shi} $^{4}$\\
  \\
  $^1$~Institute of Information Science, Beijing Jiaotong University \\
  $^{2}$~Beijing Key Laboratory of Advanced Information Science and Network \\
  $^3$~Peng Cheng Laboratory \quad  $^4$~Picsart AI Research (PAIR) \\
  \texttt{jiaosiyu99@bjtu.edu.cn} \\
}


\begin{document}


\maketitle


\begin{abstract}

Recently, pre-trained vision-language models have been increasingly used to tackle the challenging zero-shot segmentation task. Typical solutions follow the paradigm of first generating mask proposals and then adopting CLIP to classify them. To maintain the CLIP's zero-shot transferability, previous practices favour to freeze CLIP during training. However, in the paper, we reveal that CLIP is insensitive to different mask proposals and tends to produce similar predictions for various mask proposals of the same image. This insensitivity results in numerous false positives when classifying mask proposals. This issue mainly relates to the fact that CLIP is trained with image-level supervision.
To alleviate this issue, we propose a simple yet effective method, named Mask-aware Fine-tuning (MAFT). Specifically,  Image-Proposals CLIP Encoder (IP-CLIP Encoder) is proposed to handle arbitrary numbers of image and mask proposals simultaneously. Then, \textit{mask-aware loss} and \textit{self-distillation loss} are designed to fine-tune IP-CLIP Encoder, ensuring CLIP is responsive to different mask proposals while not sacrificing transferability.
In this way, mask-aware representations can be easily learned to make the true positives stand out. Notably, our solution can seamlessly plug into most existing methods without introducing any new parameters during the fine-tuning process. 
We conduct extensive experiments on the popular zero-shot benchmarks. With MAFT, the performance of the state-of-the-art methods is promoted by a large margin: 50.4\% (+ 8.2\%) on COCO, 81.8\% (+ 3.2\%) on Pascal-VOC, and 8.7\% (+4.3\%) on ADE20K in terms of mIoU for unseen classes. 
Code is available at 
\href{https://github.com/jiaosiyu1999/MAFT.git}{github.com/jiaosiyu1999/MAFT.git}.

\end{abstract}



\section{Introduction}
\label{sec:intro}
% !TEX root = 0-20SPAWC_SAND.tex
% DO NOT REMOVE THE ABOVE COMMENT!
\section{Introduction}
%
Millimeter-wave (mmWave) and massive multi-user (MU) multiple-input multiple-output (MIMO) will be  core technologies for future wireless systems~\cite{larsson14a, rappaport15a}.
%
The combination of these technologies enables simultaneous communication to multiple user equipments (UEs) at unprecedentedly high data rates. 
%
These advantages come at the cost of significantly increased power consumption, implementation complexity, and system costs. A viable solution to address these challenges is the use of low-resolution data converters combined with sophisticated but efficient baseband processing algorithms in all-digital basestations (BS) architectures~\cite{dutta2019case,jacobsson17b,li17b,mo16b,panagiotis20}. 

\subsection{Channel Estimation with Low-Resolution Data Converters}
%
Coarse quantization of the received baseband samples, due to the use of low-resolution analog-to-digital converters (ADCs) at the BS, together with the high path loss at mmWave or terahertz (THz) frequencies~\cite{rappaport15b, gao16}, renders the acquisition of accurate channel estimates a challenging task.
%
Fortunately,  wave propagation at mmWave or THz frequencies is directional~\cite{akdeniz14a} and channels typically consist only of a few dominant propagation paths~\cite{rappaport13a,rappaport15a}. Both of these properties cause the channel vectors to be sparse in the beamspace domain, which can be exploited to perform denoising that improves reliability of data transmission~\cite{alkhateeb14a,mo14b,tang13,brady13,ghods19a}. 

Practical sparsity-exploiting channel denoising methods for mmWave massive MU-MIMO systems must exhibit low computational complexity due to the large number of BS antenna elements and the potentially large number of UEs that commmunicate simultaneously.
%
A low-complexity mmWave channel denoising algorithm called BEACHES (short for beamspace channel estimation) has been proposed recently in~\cite{ghods19a}. This method has orders-of-magnitude lower complexity than state-of-the-art denoising methods, such as atomic norm minimization (ANM)~\cite{bhaskar13} and Newtonized orthogonal matching pursuit (NOMP) \cite{mamandipoor16}. 
%
However, all of these existing denoising methods perform poorly when denoising channel vectors that were acquired through low-resolution data converters. 
%
Channel estimation with 1-bit ADCs has been analyzed in~\cite{li17b,li16a,jacobsson17b,mollen16c,studer16a}. 
%
Beamspace sparsity of mmWave channels has been exploited to denoise channel vectors from 1-bit measurements in \cite{mo16b, huang19,kaushik18}.
%
However, all of these denoising methods exhibit high complexity, ignore beamspace sparsity, and/or require a number of parameters that must be adapted to the instantaneous propagation conditions, such as the number of dominant propagation paths.  


\subsection{Contributions}
%
We propose low-complexity channel estimation algorithms for mmWave massive MU-MIMO systems that operate with 1-bit data converters.
%
By using a Bussgang-like decomposition~\cite{bussgang52a} of the 1-bit measurement process, our methods adapt the optimal denoising parameters to the channel's instantaneous sparsity via Stein's unbiased risk estimate (SURE).
%
We propose two methods that build upon BEACHES put forward in~\cite{ghods19a} and a novel method, referred to as Sparsity-Adaptive oNe-bit Denoiser (SAND), which automatically tunes two algorithm parameters to minimize the channel estimation mean-square error (MSE). 
%
To demonstrate the efficacy of our channel estimation algorithms, we perform  MSE and bit error rate (BER) simulations with  line-of-sight (LoS) and non-LoS mmWave channels in a massive MU-MIMO system. 

\subsection{Notation}
%
Lowercase and uppercase boldface letters denote column vectors and matrices, respectively. 
%
The $k$th entry of the vector~$\bma$ is~$a_k$; 
the real and imaginary parts are $\realindex{\bma} = \realpart{\bma}$ and $\imagindex{\bma} = \imagpart{\bma}$, respectively. 
For a matrix $\bA$, 
its transpose and Hermitian transpose are $\bA^\Tran$ and $\bA^\Herm$, respectively. 
A complex Gaussian vector $\bma$ with mean $\bmm$ and covariance $\bK$ is written as $\bma\sim\CN(\bmm, \bK)$.
%
Expectation is denoted by $\smolE{\cdot}$. 
%

\section{Related Work}
\label{sec:related}

\myparagraph[0]{Interpretability.}
In order to make machine learning models more interpretable, a variety of techniques has been developed.
On the one hand, 
    research has been undertaken to develop model-agnostic explanation methods for which the model behaviour
    under different inputs is analysed; this includes among others \cite{lundberg2017unified,petsiuk2018rise,lime}.
    While their generality and the applicability to any model are advantageous,
    these methods typically require evaluating the respective model several times and are therefore costly
    approximations of model behaviour.
    %
On the other hand,
    many techniques that explicitly take advantage of the internal computations have been proposed for explaining
    the model predictions, including, for example, \cite{simonyan2013deep,springenberg2014striving,zhou2016CAM,selvaraju2017grad,shrikumar2017deeplift,sundararajan2017axiomatic,srinivas2019full,bach2015pixel}.\\
    %
In contrast to techniques that aim to explain models \emph{post-hoc},
some recent work has focused on designing new types of network architectures, which are \emph{inherently} more interpretable.
Examples of this are the prototype-based neural networks~\cite{chen2019looks}, the BagNet~\cite{brendel2018approximating}
and the self-explaining neural networks (SENNs)~\cite{melis2018towards}.
Similarly to our proposed architectures,
    the SENNs and the BagNets derive their explanations 
    from a linear decomposition of the output into contributions from the input (features).
This \emph{dynamic linearity}, i.e., the property that the output is computed via some form of an input-dependent linear mapping, is additionally shared by the entire model family of piece-wise linear networks (e.g., ReLU-based networks). In fact, the contribution maps of the CoDA-Nets are conceptually similar to  evaluating the `Input$\times$Gradient' (IxG), cf.~\cite{adebayo2018sanity}, on piece-wise linear models, which also yields a linear decomposition in form of a contribution map.
However, in contrast to the piece-wise linear functions, we combine this \emph{dynamic linearity} with a structural bias towards an alignment between the contribution maps and the discriminative patterns in the input. This results in explanations of much higher quality, whereas IxG on piece-wise linear models has been found to yield unsatisfactory explanations of model behaviour~\cite{adebayo2018sanity}.

\myparagraph{Architectural similarities.} In our CoDA-Nets, the convolutional kernels are dependent on the specific patch that they are applied on; i.e., a CoDA-Layer might apply different filters at every position in the input. As such, these layers can be regarded as an instance of dynamic local filtering layers as introduced in~\cite{jia2016dynamic}.
Further, our dynamic alignment units (DAUs) share some high-level similarities to attention networks, cf.~\cite{xu2015show}, in the sense that each DAU has a limited budget to distribute over its dynamic weight vectors (bounded norm), which is then used to compute a weighted sum. However, whereas in attention networks the weighted sum is typically computed over vectors, which might even differ from the input to the attention module, a DAU outputs a \emph{scalar} which is a weighted sum of all scalar entries in the input. Moreover, we note that at their optimum (maximal average output over a set of inputs), the DAUs solve a constrained low-rank matrix approximation problem~\cite{eckart1936approximation}. While low-rank approximations have been used for increasing parameter efficiency in neural networks, cf.~\cite{yu2017compressing}, this concept has to the best of our knowledge not been used in order to endow neural networks with a structural bias towards finding low-rank approximations of the input for increased interpretability in classification tasks. Lastly, the CoDA-Nets 
are related to capsule networks. However, whereas in classical capsule networks the activation vectors of the capsules directly serve as input to the next layer, in CoDA-Nets the corresponding vectors are used as convolutional filters. 
We include a detailed comparison in the supplement. 

\section{Preliminary}
\label{sec:prelimiary}

% \subsection{Problem Setting}
\noindent \textbf{Problem Setting.}
Zero-shot segmentation aims at training a segmentation model capable of segmenting novel objects using text descriptions. Given two category sets $C_{seen}$ and $C_{unseen}$ respectively, where $C_{seen}$ and $C_{unseen}$ are disjoint in terms of object categories ($C_{seen} \cap C_{unseen} = \emptyset$). The model is trained on $C_{seen}$ and directly tested on both $C_{seen}$ and $C_{unseen}$. Typically, $C_{seen}$ and $C_{unseen}$ are described with semantic words (\textit{e.g.} sheep, grass).

% \subsection{Revisiting the "frozen CLIP" paradigm}
\noindent \textbf{Revisiting the "frozen CLIP" paradigm.}
\label{sec:Revisiting}
The "frozen CLIP" approaches \cite{zegformer, zsseg, freeseg, ovseg} execute zero-shot segmentation in two steps: mask proposals generation and mask proposals classification. 
In the first step, these approaches train a Proposal Generator to generate $N$ class-agnostic mask proposals (denoting as $M$, $M \in \mathbb{R}^{N \times H \times W}$) and their corresponding classification scores (denoting as $A^{p}$, $A^{p} \in \mathbb{R}^{N \times |C_{seen}|}$). MaskFormer \cite{cheng2021maskformer} and Mask2Former \cite{cheng2021mask2former} are generally used as the Proposal Generator since the Hungarian matching \cite{kuhn1955hungarian} in the training process makes the mask proposals strongly generalizable.
In the second step, $N$ suitable sub-images ($I_{sub}$) are obtained by \textit{merging} $N$ mask proposals and the input image. $I_{sub}$ is then fed into the CLIP Image Encoder to obtain the image embedding ($E^I$). Meanwhile, text embedding ($E^T$) is generated by a CLIP Text Encoder. The classification score ($A^{c}, A^{c}  \in \mathbb{R}^{N \times C}$) predicted by CLIP is calculated as:
\begin{equation}
\label{eq:prob} 
A^{c}_i = \mathrm{Softmax}(\frac{\exp(\frac{1}{\tau}s_{c} (E^T_{i}, E^I))}{\sum_{i=0}^{C}\exp(\frac{1}{\tau}s_{c}(E^T_{i}, E^I))}), i = [1,2,...C]
\end{equation}
where  $\tau$ is the temperature hyper-parameter. $s_{c}(E^T_{i}, E^I)=\frac{E^T_{i} \cdot E^I }{|E^T_{i}| |E^I|}$ represents the cosine similarity between $E^T_{i}$ and $E^I$. $C$ is the number of classes, with $C = |C_{seen}|$ during training and $C = |C_{seen}\cup C_{unseen}|$ during inference. Noting that CLIP is frozen when training to avoid overfitting.

To further enhance the reliability of $A^{c}$, the classification score of the Proposal Generator ($A^{p}$) is ensembled with $A^{c}$ since $A^{p}$ is more reliable on seen classes. This \textit{ensemble} operation is wildly used in "frozen CLIP" approaches.  The pipeline of "frozen CLIP", as well as the \textit{merge} and \textit{ensemble} operations, are described in detail in the Appendix. 

Although  "frozen CLIP" approaches have achieved promising results, it is clear that directly adopting an image-level pre-trained CLIP for proposal classification can be suboptimal. A frozen CLIP usually produces numerous false positives, and the \textit{merge} operation may destroy the context information of an input image. In view of this, we rethink the paradigm of the frozen CLIP and explore a new solution for proposal classification.

\section{Methodology}
\label{sec:method}

\begin{figure}
% \vspace{-15mm}
\begin{center}
   \includegraphics[width=0.99\linewidth]{figs/pdf/fintune.pdf}
\end{center}
% \vspace{-2mm}
   \caption{
    Overview of the Mask-Aware Fine-tuning (MAFT). 
    In IP-CLIP Encoder, we modify the CLIP Image Encoder, and apply the mask proposals as attention bias in Multihead Attention from the $L^{th}$ layer. The final projection unit is an MLP module used for reshaping the channels of $F_{cls}$. \textit{w.o.} $M$ denotes IP-CLIP Encoder processes image without utilizing mask proposals ($M$). \textit{Mask-aware} Loss is designed to train CLIP to be mask-aware, while \textit{Self-distillation} Loss is designed to maintain the transferability. Only the IP-CLIP Encoder is trained (\textcolor{orange}{orange} part), the Proposal Generator and the CLIP Text Encoder are frozen (\textcolor{cyan}{blue} part).
   }
\label{fig:finetune}
% \vspace{-5mm}
\end{figure}


We introduce Mask-Aware Fine-tuning (MAFT), a method for learning mask-aware CLIP representations. 
Within MAFT, we first propose the Image-Proposal CLIP Encoder (IP-CLIP Encoder) to handle images with any number of mask proposals simultaneously (Sec. \ref{sec:IP-CLIP}). Then, \textit{mask-aware loss}  and \textit{self-distillation loss}  are introduced to fine-tune the IP-CLIP Encoder and make it distinguishable for different mask proposals while maintaining transferability (Sec. \ref{sec:Mask-aware tuning}).
The complete diagram of the MAFT is shown in Fig.~\ref{fig:finetune}, we use the ViT-B/16 CLIP model for illustration.


\subsection{Image-Proposal CLIP Encoder (IP-CLIP Encoder)}
\label{sec:IP-CLIP}
IP-CLIP Encoder aims to process arbitrary numbers of images and mask proposals simultaneously. We draw inspiration from MaskFormer \cite{cheng2021mask2former, cheng2021maskformer}, which uses attention-masks in Multihead Attention and provides the flexibility for accepting any number of queries and features of different masked regions. Accordingly, we apply mask proposals as attention-masks in Multihead Attention and designate independent classification queries for each mask proposal.
 % We draw inspiration from MaskFormer \cite{cheng2021mask2former, cheng2021maskformer}, which uses attention-masks to calculate Multihead Attention between queries and features.  It provides the flexibility for Multihead Attention to accept any number of queries and features of different masked regions.
 
In the IP-CLIP Encoder shown in Fig. \ref{fig:finetune}, we denote the features propagate between Transformer layers as $F^i$, where $i = [1,2...12]$. We can express $F^i$ as $F^i = [F^i_{cls};~ F^i_{feat}], \in \mathbb{R}^{(1 + hw) \times d}$, here $1$ represents a class-embedding vector ($F^i_{cls}$), $hw$ represents the number of the flattened image features ($F^i_{feat}$). 
% The output class-embedding vector $F^{12}_{cls}$ is utilized for classification (equals to $E^I$ in Sec. \ref{sec:Revisiting}). 
To obtain the classifications of all mask proposals simultaneously, we repeat $F^i_{cls}$ at layer $L$ $N$ times, where $N$ is the number of mask proposals, denoting the repeated class-embedding vectors as $F^{i*}_{cls}$. We can express the modified features ($F^{i*}$) as $F^{i*} = [F^{i*}_{cls};~ F^i_{feat}], \in \mathbb{R}^{(N + hw) \times d}$.

% Therefore, in the first $L$ Transformer layers, the propagation of $F^{i}$ keeps same with standard CLIP,
% \begin{equation}
%    F^{i+1} =\mathrm{TLayer}^i(F^i)
% \end{equation}
% $\mathrm{TLayer}^i$ denote i$^{th}$ Transformer layer. We simplify the representation of
% Transformer layer, whereas the start $L$ Transformer layers conduct the same structure with CLIP.

% Thus a standard Multihead Attention process in Transformer layers can be formulated as follows:
% \begin{equation}
%    \mathrm{MHAtten}(F^i) =\mathrm{Softmax}(\frac{Que(F^i)Key(F^i)^T}{\sqrt{d}})Val(F^i)
% \end{equation}
% where $Que(\cdot)$, $Key(\cdot)$, and $Val(\cdot)$ denote linear projections, $d$ is the hidden dimension of $F^i$. 

%  \begin{equation}
%     F^{i*} =[F^{i*}_{cls};~ F^i_{feat}], F^{i*} \in \mathbb{R}^{C \times (N + hw)}
% \end{equation}
\noindent \textbf{Propagation of $F^{i}$, where $i = [1, 2, ...L]$.}
We consider that CLIP's classification significantly relies on context information. In the first $L$ Transformer layers, the propagation of $F^{i}$ is the same as in standard CLIP. Specifically, $F^{i}_{cls}$ utilizes cross-attention with all pixels within $F^{i}_{feat}$, effectively retaining the context information. 

In the subsequent $12-L$ Transformer layers, the propagation of $F^{i*}$ can be partitioned into two parts: the propagation of $F^{i*}_{cls}$ and the propagation of $F^{i}_{feat}$.

\noindent \textbf{Propagation of $F^{i*}_{cls}$.}
We use $F^{i*}_{cls}$[$n$] and $M$[$n$] to represent the position $n$ in $F^{i*}_{cls}$ and $M$, where $n=[1,2...N]$. It is expected $F^{i*}_{cls}$[$n$] computes Multihead Attention for the positions where $M$[$n$]$=1$ and itself. To achieve this, we construct an attention bias $B \in \mathbb{R}^{N \times (N+hw)}$ as follows:
\begin{equation}
B_{(i,j)}=\left\{
\begin{aligned}
0  &, \mathrm{if} ~ {\hat{M}}_{(i,j)} = 1\\
-\infty  &, \mathrm{if} ~ {\hat{M}}_{(i,j)} = 0\\
\end{aligned}
\right.
,~~~ \hat{M} = [\mathrm{I}(N,N);~ \mathrm{Flat}(M)]
\end{equation}
here $\mathrm{I}(N,N)$ denotes $N^{th}$ order identity matrix, $\mathrm{Flat}$($\cdot$) denotes the \textit{flatten} operation. $\hat{M}$ is an intermediate variable for better representation. Therefore, a masked Multihead Attention is used for propagating $F^{i*}_{cls}$ 
% \footnote{We omit the MLP Layer and some Layer Normalizations in Transformer layers to  simplify the representation.}
:
\begin{equation}
   F^{(i+1)*}_{cls} =\mathrm{Softmax}(\frac{\mathrm{Que}(F^{i*}_{cls})\mathrm{Key}(F^{i*})^T}{\sqrt{d}} + B)\mathrm{Val}(F^{i*})
   \label{con:modified tlayers1}
\end{equation}
where $\mathrm{Que}(\cdot)$, $\mathrm{Key}(\cdot)$, and $\mathrm{Val}(\cdot)$ denote linear projections, $d$ is the hidden dimension of $F^{i*}$. Notably, We omit the MLP Layer and Layer Normalizations in Transformer layers to simplify the representation in Eq. \ref{con:modified tlayers1} and Eq. \ref{con:modified tlayers2}.

\noindent \textbf{Propagation of $F^{i}_{feat}$.}
A standard Multihead Attention is used for propagating $F^{i}_{feat}$ 
% \footnote{Similar to Eq. \ref{con:modified tlayers}, MLP Layer and Layer Normalizations are simplified.}
: 
\begin{equation}
   F^{i+1}_{feat} =\mathrm{Softmax}(\frac{\mathrm{Que}(F^{i}_{feat})\mathrm{Key}(F^{i}_{feat})^T}{\sqrt{d}})\mathrm{Val}(F^{i}_{feat})
   \label{con:modified tlayers2}
\end{equation}
Therefore, for any given mask proposal $M$[$n$], the corresponding class-embedding $F^{i*}_{cls}$[$n$] only performs Multihead Attention with $F^{i}_{feat}$ where $M$[$n$]$=1$ and $F^{i*}_{cls}$[$n$]. The propagation of $F^{i}_{feat}$ remains undisturbed by attention-masks. Compared with the frozen CLIP,  IP-CLIP Encoder leverages context information effectively and reduces computational costs.



\subsection{Objective}
\label{sec:Mask-aware tuning}
IP-CLIP Encoder with CLIP pre-trained parameters remains challenging in distinguishing different mask proposals, \textit{e.g.}, when the proposals contain more background regions than foreground objects, IP-CLIP may tend to classify them into the foreground categories. To overcome this limitation, we introduce \textit{mask-aware loss} and \textit{self-distillation loss} to fine-tune the IP-CLIP Encoder to be mask-aware without sacrificing transferability. 

We conduct the \textit{mask-aware} loss function ($\mathcal{L}_{ma}$) on $A^c$.  The goal is to assign high scores to high-quality proposals and low scores to low-quality proposals in $A^c$. Concretely, we use the Intersection over Union (IoU) score obtained from ground-truth and align it with the $A^c$ to prompt CLIP to become mask-aware. Assuming there are $k$ classes in ground-truth, we can generate $k$ binary maps of ground-truth and calculate the IOU score ($S_{IoU}$) with $N$ mask proposals. We identify a discrepancy between the maximum values of $A^c$ and $S_{IoU}$. The maximum value of $A^c$ tends to approach 1, whereas the maximum value of $S_{IoU}$ ranges from 0.75 to 0.99. This inconsistency can hinder the alignment between these two metrics. Therefore, we introduced a min-max normalization technique for $S_{IoU}$ as follows:
\begin{equation}
S_{IoU}^{norm} = \frac{S_{IoU} - min(S_{IoU})}{max(S_{IoU}) - min(S_{IoU})},  S_{IoU}\in \mathbb{R}^{K \times N}
\end{equation}
Meanwhile, we select $k$ pre-existing classes in $A^c$ ($A^c_{select}, A^c_{select}\in \mathbb{R}^{K \times N}$), and employ $SmoothL1$ Loss to align it with $S_{IoU}^{norm}$. Therefore, $\mathcal{L}_{ma}$ can be formulated as follows:
\begin{equation}
\mathcal{L}_{ma}(A^c_{select}, S_{IoU}^{norm}) = \mathrm{SmoothL1} (A^c_{select}, S_{IoU}^{norm})
\end{equation}
\begin{equation}
\mathrm{SmoothL1}(x, y) = \left\{
\begin{aligned}
 0.5\cdot (x - y)^2  &, ~~~ \mathrm{if} ~ |x - y| < 1\\
|x - y| - 0.5  &, ~~~ \mathrm{otherwise} ~ \\
\end{aligned}
\right.
\end{equation}

In addition to $\mathcal{L}_{ma}$, we also introduce a \textit{self-distillation} loss $\mathcal{L}_{dis}$ to maintain CLIP's transferability and alleviate overfitting on $C_{seen}$. 
Within $\mathcal{L}_{dis}$, we use a frozen CLIP as the \textit{teacher} net, the  IP-CLIP as the \textit{student} net for self-distillation.
The predictions of the frozen CLIP and IP-CLIP are expected to be the same when no mask is included. Denoting the output of the frozen CLIP as $A_{T}$, and the output of the fine-tuned IP-CLIP without masks as $A_{S}$. We use $SmoothL1$ Loss to minimize the difference as follows:
\begin{equation}
\mathcal{L}_{dis}(A_{S}, A_{T}) = \mathrm{SmoothL1} (A_{S}, A_{T})
\end{equation}
It is important to note that when processing an image through IP-CLIP without mask proposals, the resulting $A_{S}$ is a matrix with dimensions $\mathbb{R}^{C \times 1}$.
Therefore, the final loss function can be formulated as: $\mathcal{L} = \mathcal{L}_{ma} + {\lambda} {\mathcal{L}_{dis}}$, where we set the constant $\lambda$ to 1 in our experiments. The mask-aware fine-tuning process is efficient as we only perform a few iterations (less than 1 epoch).

\section{Experiments}
\label{sec:exp}

\begin{table}[t]
  \centering
  \footnotesize
\begin{minipage}[t]{\textwidth}
  \caption{Comparison with state-of-the-art methods in zero-shot segmentation. mIoU$^s$ and mIoU$^u$ denote the mIoU(\%) of seen classes and unseen classes. }
\resizebox{1.0\textwidth}{!}{

    \renewcommand\arraystretch{1.2} % 1.95
    
    % \begin{tabular}{l|ccc|ccc|ccc}
    \begin{tabular}{l|lll|lll|lll}
      \Xhline{0.8pt}
      % \toprule
      \multirow{2}{*}{\textbf{Method}} &\multicolumn{3}{c|}{COCO-Stuff} & \multicolumn{3}{c|}{Pascal-VOC} & \multicolumn{3}{c}{ADE20K}\\ 
      & \textbf{mIoU$^s$} & \textbf{mIoU$^u$} & \textbf{hIoU} & \textbf{mIoU$^s$} & \textbf{mIoU$^u$} & \textbf{hIoU} & \textbf{mIoU$^s$} & \textbf{mIoU$^u$} & \textbf{hIoU} \\ 
      \hline
      SPNet\cite{spnet} & 34.6 & 26.9 & 30.3 & 77.8 & 25.8 & 38.8 & - & - & -\\
      ZS5\cite{zs5} & 34.9 & 10.6 & 16.2 & 78.0 & 21.2 & 33.3 & - & - & - \\
      CaGNet\cite{cagnet} & 35.6 & 13.4 & 19.5 & 78.6 & 30.3 & 43.7 & - & - & -\\
      STRICT\cite{STRICT} & 35.3 & 30.3 & 32.6 & 82.7 & 35.6 & 73.3 & - & - & -\\
      \cdashline{1-10}[0.8pt/2pt]
      ZegFormer\cite{zegformer} & 36.7 & 36.2 & 36.4 & 90.1 & 70.6 & 79.2 & 17.4 & 5.1 & 7.9\\
      ZegFormer +MAFT & 36.4 $_{\textcolor{blue}{-0.3}}$ & 40.1 $_{\textcolor{red}{+3.9}}$ & 38.1 $_{\textcolor{red}{+1.7}}$ & 91.5 $_{\textcolor{red}{+1.4}}$ & 80.7 $_{\textcolor{red}{+10.1}}$ & 85.7 $_{\textcolor{red}{+6.5}}$ & 16.6 $_{\textcolor{blue}{-0.8}}$ & 7.0 $_{\textcolor{red}{+1.9}}$ & 9.8 $_{\textcolor{red}{+1.9}}$\\
      \cdashline{1-10}[0.8pt/2pt]
      ZSSeg\cite{zsseg} & 40.4 & 36.5 & 38.3 & 86.6 & 59.7 & 69.4 & 18.0 & 4.5 & 7.2 \\
      ZSSeg +MAFT & 40.6 $_{\textcolor{red}{+0.2}}$ & 40.1 $_{\textcolor{red}{+3.6}}$ & 40.3 $_{\textcolor{red}{+2.0}}$ & 88.4 $_{\textcolor{red}{+1.8}}$ & 66.2 $_{\textcolor{red}{+6.5}}$ & 75.7 $_{\textcolor{red}{+6.3}}$ & 18.9 $_{\textcolor{red}{+0.9}}$ & 6.7 $_{\textcolor{red}{+2.2}}$ & 9.9 $_{\textcolor{red}{+2.7}}$ \\
      \cdashline{1-10}[0.8pt/2pt]
      FreeSeg\cite{freeseg} & 42.4 & 42.2 & 42.3 & 91.9 & 78.6 & 84.7 & 22.3 & 4.4 & 7.3 \\
      FreeSeg +MAFT & 43.3 $_{\textcolor{red}{+0.9}}$ & 50.4 $_{\textcolor{red}{+8.2}}$ & 46.5 $_{\textcolor{red}{+4.2}}$ & 91.4 $_{\textcolor{blue}{-0.5}}$ & 81.8 $_{\textcolor{red}{+3.2}}$ & 86.3 $_{\textcolor{red}{+1.6}}$ & 21.4 $_{\textcolor{blue}{-0.9}}$ & 8.7 $_{\textcolor{red}{+4.3}}$  & 12.4 $_{\textcolor{red}{+5.1}}$ \\
      % \rowcolor{gray!10}{} Ours & \textbf{42.2} & \textbf{49.1} & \textbf{45.3} & \textbf{91.8} & \textbf{82.6} & \textbf{86.9} & \textbf{44.2} & \textbf{28.6} & \textbf{39.8} \\
      \Xhline{0.8pt}
      % \bottomrule
      \end{tabular}
      }
      % \end{threeparttable}
      \label{tab:zss}
\end{minipage}

\begin{minipage}[t]{\textwidth}
\caption{Results on representative methods \cite{zegformer, zsseg, freeseg} with/without MAFT. Here we remove the \textit{ensemble} operation, and only maintain CLIP classifier results.}
 \resizebox{1.0\textwidth}{!}{

    \renewcommand\arraystretch{1.2} % 1.95
    
    % \begin{tabular}{l|ccc|ccc|ccc}
    \begin{tabular}{l|lll|lll|lll}
      \Xhline{0.8pt}
      % \toprule
      \multirow{2}{*}{\textbf{Method}} &\multicolumn{3}{c|}{COCO-Stuff} & \multicolumn{3}{c|}{Pascal-VOC} & \multicolumn{3}{c}{ADE20K}\\ 
      %\cline{2-10}
      & \textbf{mIoU$^s$} & \textbf{mIoU$^u$} & \textbf{hIoU} & \textbf{mIoU$^s$} & \textbf{mIoU$^u$} & \textbf{hIoU} & \textbf{mIoU$^s$} & \textbf{mIoU$^u$} & \textbf{hIoU} \\ 
      \hline
      ZegFormer\cite{zegformer} & 18.5 & 23.0 & 20.5 & 81.4 & 76.8 & 79.0 & 5.1 & 2.6 & 3.5\\
      ZegFormer +MAFT & 35.1 $_{\textcolor{red}{+16.6}}$ & 31.6 $_{\textcolor{red}{+7.6}}$ & 33.3 $_{\textcolor{red}{+12.7}}$ & 87.6 $_{\textcolor{red}{+6.2}}$ & 79.9 $_{\textcolor{red}{+3.1}}$ & 83.5 $_{\textcolor{red}{+4.5}}$ & 15.8 $_{\textcolor{red}{+10.8}}$ & 7.0 $_{\textcolor{red}{+4.4}}$ & 9.8 $_{\textcolor{red}{+6.3}}$\\
      \cdashline{1-10}[0.8pt/2pt]
      ZSSeg\cite{zsseg} & 20.6 & 27.4 & 23.6 & 82.0 & 71.2 & 76.2 & 5.9 & 2.8 & 3.9 \\
      ZSSeg +MAFT & 36.1 $_{\textcolor{red}{+15.5}}$ & 35.9 $_{\textcolor{red}{+8.3}}$ & 36.0 $_{\textcolor{red}{+12.4}}$ & 87.1 $_{\textcolor{red}{+5.1}}$ & 76.1 $_{\textcolor{red}{+4.9}}$ & 81.2 $_{\textcolor{red}{+5.0}}$ & 17.2 $_{\textcolor{red}{+11.3}}$ & 7.2 $_{\textcolor{red}{+4.4}}$ & 10.2 $_{\textcolor{red}{+6.3}}$ \\
      \cdashline{1-10}[0.8pt/2pt]
      FreeSeg\cite{freeseg} & 22.3 & 29.3 & 25.3 & 87.4 & 74.7 & 80.5 & 6.5 & 2.8 & 3.9 \\
      FreeSeg +MAFT & 40.1 $_{\textcolor{red}{+17.8}}$ & 49.7 $_{\textcolor{red}{+20.4}}$ & 44.4 $_{\textcolor{red}{+19.1}}$ & 90.4 $_{\textcolor{red}{+3.0}}$ & 84.7 $_{\textcolor{red}{+10.0}}$ & 87.5 $_{\textcolor{red}{+7.0}}$ & 21.3 $_{\textcolor{red}{+14.8}}$ & 8.7 $_{\textcolor{red}{+5.9}}$  & 12.2 $_{\textcolor{red}{+8.3}}$ \\
      % \rowcolor{gray!10}{} Ours & \textbf{42.2} & \textbf{49.1} & \textbf{45.3} & \textbf{91.8} & \textbf{82.6} & \textbf{86.9} & \textbf{44.2} & \textbf{28.6} & \textbf{39.8} \\
      \Xhline{0.8pt}
      % \bottomrule
      \end{tabular}
      }
      % \end{threeparttable}
      \label{tab:zss-woensem}
\end{minipage}

  \vspace{-2mm}
 \end{table}


% \begin{table}[h]
%   \centering
%   \footnotesize
% \begin{minipage}[b]{80mm}



    
%     \begin{tabular}{l|ccc|ccc|ccc}

%       \end{tabular}
      
%       % \end{threeparttable}
%       \label{tab:zss}
% \end{minipage}

% \begin{minipage}[b]{80mm}


    
%     \begin{tabular}{l|ccc|ccc|ccc}

%       \end{tabular}


% \end{minipage}

%  \end{table}
\subsection{Setting}
\noindent \textbf{Dataset.}
We first follow \cite{zs5, gu2020context, pastore2021closer, zegformer, zsseg} to conduct experiments on three popular zero-shot segmentation benchmarks, Pascal-VOC, COCO-Stuff and ADE20K, to evaluate our method. Then, we evaluate MAFT on the \textit{open-vocabulary} setting \cite{ovseg, zsseg}, \textit{i.e.}, training on COCO-Stuff and testing on ADE20K (A-847, A-150), Pascal-Context (PC-459, PC-59), and Pascal-VOC (PAS-20). More details of the dataset settings are provided in the Appendix.

\noindent \textbf{Evaluation Metrics.}
To quantitatively evaluate the performance, we follow standard practice \cite{zs5, spnet, cagnet, STRICT, zegformer, zsseg, freeseg}, adopt mean Intersection over Union (mIoU) to respectively evaluate the performance for seen classes (IoU$^s$) and unseen classes (IoU$^u$). We also employ the harmonic mean IoU (hIoU) among the seen and unseen classes to measure comprehensive performance.

\noindent \textbf{Methods.}
Three representative methods are used to verify the generality of MAFT. We unify the three methods into the same framework, with all methods using ResNet101 as the backbone of Proposal Generator and ViT-B/16 CLIP model for a fair comparison.

\begin{itemize}[itemsep=2pt,topsep=0pt,parsep=0pt]
\item \textbf{ZegFormer} (CVPR 2022) \cite{zegformer} is an early adopter of the "frozen CLIP" paradigm. It uses MaskFormer as Proposal Generator and employs an \textit{ensemble} operation to improve the confidence of the results.
\item \textbf{ZSSeg} (ECCV 2022) \cite{zsseg} uses MaskFormer as Proposal Generator and introduces learnable prompts to improve classification accuracy, which significantly affects the subsequent methods. ZSSeg also adopts a self-training strategy, this strategy is excluded from all methods for a fair comparison.
\item \textbf{FreeSeg} (CVPR 2023) \cite{freeseg} represents the state-of-the-art method, unifies semantic, instance, and panoptic segmentation tasks and uses annotations from all three tasks for fusion training. We retrain FreeSeg with only the semantic annotations to ensure fairness.
\end{itemize}

\noindent \textbf{Implementation details.}
We employ ResNet101 as backbone of the Proposal Generator and ViT-B/16 CLIP model. The training process  consists of two stages.
For the \textbf{first} stage, we follow the official code of ZegFormer, ZSSeg and FreeSeg for model training. 
For the \textbf{second} stage, we fine-tune IP-CLIP Encoder with MAFT. We take the batch size
of 16 and set CLIP input image size to 480$\times$480. The optimizer is AdamW with a learning rate of 0.00001 and weight decay of 0.00001. The number of training iterations is set to 100 for Pascal-VOC, 1000 for COCO-Stuff and 5000 for ADE20K.

\subsection{Comparisons with State-of-the-art Methods}
\begin{table}
  \centering
  \footnotesize
  % \vspace{-10pt}
  \caption{Comparison with state-of-the-art methods on the \textit{open-vocabulary} setting. mIoU is used to evaluate the performance. * denotes additional training data is used.}
   % \vspace{-10pt}
  % \begin{threeparttable}
  % \resizebox{0.9\textwidth}{
    \renewcommand\arraystretch{1.05} % 1.95
    % \begin{tabular}{l|ccccc}
    \begin{tabular}{l|lllll}
      % \toprule
      \Xhline{0.7pt}

      & \textbf{A-847} & \textbf{A-150} & \textbf{PC-459} & \textbf{PC-59} & \textbf{PAS-20}\\ 
      \hline
      SPNet\cite{spnet} & ~~~- & ~~~- & ~~~- & ~~~24.3 & ~~~18.3\\
      ZSSeg\cite{zs5}  & ~~~- & ~~~- & ~~~- & ~~~19.4 & ~~~38.3\\
      LSeg+\cite{ghiasi2021open} & ~~~2.5 & ~~~13.0 & ~~~5.2 & ~~~36.0 & ~~~59.0\\
      OVSeg\cite{ovseg} & ~~~7.1 & ~~~24.8 & ~~~11.0 & ~~~53.3 & ~~~92.6\\
      OpenSeg* \cite{ghiasi2022scaling} &  ~~~8.8 & ~~~28.6 &  ~~~12.2  &  ~~~48.2 & ~~~72.2 \\     
      \cdashline{1-6}[0.8pt/2pt]
      FreeSeg\cite{freeseg} & ~~~7.1 & ~~~17.9 & ~~~6.4 & ~~~34.4 & ~~~85.6 \\
      FreeSeg +MAFT & ~~~10.1 $_{\textcolor{red}{+3.0}}$ & ~~~29.1 $_{\textcolor{red}{+11.2}}$ & ~~~12.8 $_{\textcolor{red}{+6.4}}$ & ~~~53.5 $_{\textcolor{red}{+19.1}}$ & ~~~90.0 $_{\textcolor{red}{+4.4}}$ \\
      \Xhline{0.7pt}
      % \bottomrule
      \end{tabular}
      % }
      % \end{threeparttable}
      \label{tab:ovs}
  \vspace{-2mm}
  \end{table}  




In this section, three representative methods are used  \cite{zegformer, zsseg, freeseg} to evaluate the effectiveness of MAFT. We compare three representative methods with MAFT and frozen CLIP. Additionally, we compare the results with previous state-of-the-art methods  \cite{spnet, zs5, cagnet, STRICT}.

\noindent \textbf{Comparisons in the \textit{zero-shot} setting.}
In Tab. \ref{tab:zss}, MAFT remarkably improves the performance. MAFT promotes the state-of-the-art performance by + 8.2\% on COCO, + 3.2\% on Pascal, and +4.3\% on ADE20K in terms of mIoU for unseen classes. It is important to note that the results for seen classes are mainly based on $A^p$ rather than $A^c$ due to the \textit{ensemble} operation in \cite{zegformer, zsseg, freeseg} (Details in Sec. \ref{sec:prelimiary}). Therefore, the effect of MAFT on the seen classes is relatively insignificant. 

\noindent \textbf{Comparisons without ensemble strategy.}
To better showcase the performance gains from MAFT, we removed the \textit{ensemble} operation in \cite{zegformer, zsseg, freeseg} and presented the results in Tab. \ref{tab:zss-woensem}.  It can be seen that the performance of different methods is significantly improved after applying MAFT. In particular, the state-of-the-art method FreeSeg achieves hIoU improvements of 19.1\%, 7.0\%, and 8.3\% on COCO, VOC2012 and ADE20K datasets. 

\noindent \textbf{Comparisons in the \textit{open-vocabulary} setting.}
We further evaluated the transferability of MAFT in the \textit{open-vocabulary} setting \cite{ovseg, zsseg}, using FreeSeg as a baseline for comparison. Results are shown in Tab. \ref{tab:ovs}.
Compared with OVSeg \cite{ovseg} and OpenSeg \cite{ghiasi2022scaling}, FreeSeg achieves suboptimal performance. However, the proposed MAFT enhances the performance of A-847, A-150, PC-459, PC-59 and PAS-20 by 3.0\%,11.2\%, 6.4\%, 19.1\% and 4.4\%, and outperforms OpenSeg on all five datasets.

\subsection{Ablation Study}
\begin{table*}
\vspace{-2mm}
\caption{\textbf{Ablations on COCO dataset.} GFLOPs in (a) is used to measure the computation of CLIP Image Encoder. The best results are highlighted with \textcolor{red}{red}, and the default settings are highlighted with \textcolor{gray}{gray} background.}
\label{tab:ablations}
% \vspace{-1mm}
\resizebox{1.0\textwidth}{!}{
\centering
% a - Ablation on Statics, Hand-Craft and Learnable
    \subfloat[Ablation on components of \textbf{MAFT}. $ft$ denotes the mask-aware fine-tining
    \label{tab:ab_1}]
    { 
    % \!\!\!\!\!
    % \renewcommand\tabcolsep{11.5pt}
    \renewcommand\arraystretch{1.1} % 1.95
    % \footnotesize
    \small
    \begin{tabular}
    {l|lll|c}
    % {p{27.8mm}|p{15mm}p{15mm}}
    \Xhline{0.7px}
    % \Xhline{0.7px}
    \centering
     & \textbf{mIoU$^s$} & \textbf{mIoU$^u$} & \textbf{hIoU} & GFLOPs \\
    \hline
    \multicolumn{1}{c|}{FreeSeg} & 22.3 &  29.3 &  25.3 &  1127.0\\
    \multicolumn{1}{c|}{+ IP-CLIP} & 29.4 $_{+7.1}$ &  36.2 $_{+6.9}$ &  32.4 $_{+7.1}$ &  53.4\\
    \multicolumn{1}{c|}{+ $ft$ ($\mathcal{L}_{ma}$)} & 39.9 $_{+17.6}$ &  47.1 $_{+17.8}$ &  43.1 $_{+17.8}$ &  53.4\\
    \rowcolor{gray!10}\multicolumn{1}{c|}{+ $ft$ ($\mathcal{L}_{ma}$ + $\mathcal{L}_{dis}$)} & 
    40.1$_{\textcolor{red}{+17.8}}$ &  
    49.7$_{\textcolor{red}{+20.4}}$ &  
    44.4$_{\textcolor{red}{+19.0}}$ &  
    53.4\\
    \Xhline{0.7px}
    % \Xhline{0.7px}
    \end{tabular}
    }
    \hspace{2mm}

% b - Ablation on SI module
    \subfloat[Ablation on \textbf{mask-aware loss $\mathcal{L}_{ma}$}. $\mathcal{L}_{dis}$ is removed.
    \label{tab:ab_2}]
    { 
    % \!\!\!\!\!
    % \renewcommand\tabcolsep{11.5pt}
    \renewcommand\arraystretch{1.1} % 1.95
    % \footnotesize
    \small
    \begin{tabular}
    {l|ccc}
    % {p{27.8mm}|p{15mm}p{15mm}}
    \Xhline{0.7px}
    \centering
     & \textbf{mIoU$^s$} & \textbf{mIoU$^u$} & \textbf{hIoU} \\
    \hline 
    % \multicolumn{1}{c|}{FreeSeg} & 22.3 &  29.3 &  25.3\\
    \multicolumn{1}{c|}{$L_1$} & 38.6 $_{+16.3}$ &  45.8 $_{+16.5}$ &  41.8 $_{+16.5}$ \\
    \multicolumn{1}{c|}{$L_2$} & 40.0 $_{+17.7}$ &  45.8 $_{+16.5}$ &  42.7 $_{+17.4}$\\
    \rowcolor{gray!10}\multicolumn{1}{c|}{$SmoothL_1$} 
    & 39.9 $_{+17.6}$
    & 47.1 $_{\textcolor{red}{+17.8}}$ 
    & 43.1 $_{\textcolor{red}{+17.8}}$\\
    \multicolumn{1}{c|}{$KL$} & 40.9 $_{\textcolor{red}{+18.6}}$ &  41.8 $_{+12.5}$ &  41.3 $_{+16.0}$\\
    \Xhline{0.7px}
    \end{tabular}
    }
}


\resizebox{1.0\textwidth}{!}{

\centering

    % \hspace{4mm}
    \subfloat[Ablation of the \textbf{training iterations}
    \label{tab:ab-iter}]
    { 
    \centering%
    \footnotesize %scriptsize
    \renewcommand\tabcolsep{10pt}
    \renewcommand\arraystretch{1.2} % 1.95

        \begin{tabular}{l|cc}
        \Xhline{0.7px}
        \centering
         & \textbf{mIoU$^s$} & \textbf{mIoU$^u$}\\
        \hline
         500 iters & 37.7 &  47.0 \\
      \rowcolor{gray!10}  1k iters & 40.0 &  \textcolor{red}{49.7} \\ 
        2k iters & 41.1 &  47.6  \\
        3k iters & 41.4 &  46.5 \\
        4k iters & 41.5 &  46.1 \\
        5k iters & \textcolor{red}{42.0} &  45.7\\
        
        \Xhline{0.7px}
        \end{tabular}
    }
    \hspace{4mm}
    
% d - Ablation on SI module
    \subfloat[Ablation of the \textbf{frozen units in CLIP}
    \label{tab:ab-units}]
    { 
    \centering%
    \footnotesize
    \renewcommand\tabcolsep{6pt}
    \renewcommand\arraystretch{1.2} % 1.95

        \begin{tabular}{l|ccc}
        \Xhline{0.7px}
        \centering
         & \textbf{mIoU$^s$} & \textbf{mIoU$^u$} & \textbf{hIoU} \\
        \hline
         None & 40.6 &  44.7 &  42.5 \\
        + $cls.$ & \textcolor{red}{40.7} &  44.7 &  42.7 \\ 
        + $pos.$ & 40.6 &  44.9 &  42.8 \\
        + $mlp$ & 40.3 &  48.7 &  44.1 \\
 \rowcolor{gray!10}  + $conv.$ & 40.0 & \textcolor{red}{49.7} &  \textcolor{red}{44.3} \\
        + $proj.$ & 40.2 &  49.1 &  44.2 \\
        
        \Xhline{0.7px}
        \end{tabular}
    }
    \hspace{4mm}
    
% e - Ablation on SI module
    \subfloat[Ablation of the \textbf{start mask attention layer $L$}
    \label{tab:ab-layer}]
    { 
    \centering%
    \footnotesize
    \renewcommand\tabcolsep{11pt}
    \renewcommand\arraystretch{1.2} % 1.95

        \begin{tabular}{l|ccc}
        \Xhline{0.7px}
        \centering
         & \textbf{mIoU$^s$} & \textbf{mIoU$^u$} & \textbf{hIoU} \\
        \hline
         0 & 39.3 &  46.4 &  42.6 \\
        2 & 39.2 &  46.4 &  42.5 \\ 
        4 & 39.5 &  46.6 &  42.6 \\
        6 & \textcolor{red}{40.0} &  47.8 &  43.6 \\
      \rowcolor{gray!10}  8 & \textcolor{red}{40.0} &  \textcolor{red}{49.7} &  \textcolor{red}{44.3} \\
        10 & 39.9 &  45.7 &  42.6 \\
        
        \Xhline{0.7px}
        \end{tabular}
    }
}

% main caption
\vspace{-3mm}
\end{table*}




We conduct ablation studies on various choices of designs of our MAFT to show their contribution to the final results in Tab. \ref{tab:ablations}. FreeSeg is used as the baseline model and \textit{ensemble} operation is removed.

\noindent \textbf{Component-wise ablations.} To understand the effect of each component in the MAFT, including the IP-CLIP Encoder and the fine-tuning strategy ($\mathcal{L}_{ma}$, $\mathcal{L}_{dis}$), we start with standard FreeSeg and progressively add each design. (Tab. \ref{tab:ab_1}). 
FreeSeg uses frozen CLIP and yields inferior performance due to CLIP's mask-unaware property ($1^{st}$ row). Then, IP-CLIP Encoder obtains rich context information and greatly reduces the omputational costs, resulting in an improvement of 7.1\% on seen classes and 6.9\% on unseen classes. However, mask-aware is not accomplished at this point.
Using only $\mathcal{L}_{ma}$ for fine-tuning CLIP produces decent performance  (the $3^{rd}$ result). The introduction of $\mathcal{L}_{dis}$ (the $4^{th}$ result) maintains transferability while learning mask-aware representations, which further enhances the performance on unseen classes by 2.6\%.

\noindent \textbf{Effect of different $\mathcal{L}_{ma}$.} 
\textit{Mask-aware} Loss $\mathcal{L}_{ma}$ is an essential component of MAFT. In Tab. \ref{tab:ab_2}, we investigate how different loss functions ($L1$, $L2$, $SmoothL1$ and $KL$ Loss) impact performance, here we remove $\mathcal{L}_{dis}$ for analysis. Results show $SmoothL1$ Loss boosts performance on $C_{unseen}$ to 47.1\% (+17.8\%), $KL$ Loss provides +12.5\% improvement on $C_{seen}$, but only +11.8\% on $C_{unseen}$, manifesting $KL$ Loss compromises the model of transferability comparing with $SmoothL1$ Loss.

\noindent \textbf{Training iterations.} 
Tab. \ref{tab:ab-iter} examines the impact of training iterations. Increasing the number of iterations leads to gradual improvement of IoU$^s$, but it also results in significant overfitting on unseen classes. Therefore, we choose to fine-tune 1k iterations to maximize the zero-shot ability.

\noindent \textbf{Frozen units in CLIP.} 
We also explore the impact of fine-tuning units within IP-CLIP Encoder. As illustrated in Fig. \ref{fig:finetune}, IP-CLIP Encoder comprises convolution layers (dubbed as $conv.$), class embedding ($cls.$), Transformer layers, final projection ($proj.$) and positional embedding ($pos.$, not shown in Fig. \ref{fig:finetune}). We start with fine-tuning the entire IP-CLIP Encoder, and then freezing each unit sequentially, as specified in Tab. \ref{tab:ab-units}. We only freeze $MLP$ in the Transformer layers (dubbed as $mlp$). Compared with fine-tuning the entire IP-CLIP Encoder, the performance of mIoU$^u$ is improved by 5.0\% when freezing $conv.$, $cls.$, $pos.$ and $mlp$.

\noindent \textbf{Start mask attention layer}.
Tab. \ref{tab:ab-layer} presents the results of the start mask attention layer ($L$). 
We observe a significant improvement in the performance of unseen classes by +3.4\% when the value of $L$ increases from 0 to 8. This could be attributed to the fact that starting masked Multihead Attention later enables $F^{i*}_{cls}$ to gain more context information. However, the performance significantly drops when $L=10$ (from 49.7\% to 45.7\%), which may be due to the loss of mask-aware property.

\subsection{Extending MAFT with SAM}
% \begin{table*}
% \vspace{-2mm}
% \caption{Comparison with SAM.}
% \label{tab:sam}
% % \vspace{-1mm}
% \resizebox{1.0\textwidth}{!}{
% \centering
% % a - Ablation on Statics, Hand-Craft and Learnable
%     \subfloat[Results in zero-shot segmentation. Pascal-VOC.
%     \label{tab:sam-1}]
%     { 
%     % \!\!\!\!\!
%     % \renewcommand\tabcolsep{11.5pt}
%     \renewcommand\arraystretch{1.1} % 1.95
%     % \footnotesize
%     \small
%     \begin{tabular}
%     {l|llll}
%     \Xhline{0.7px}
%     \centering
%      & \textbf{mIoU$^s$} & \textbf{mIoU$^u$} & \textbf{hIoU} & \textbf{mIoU} \\
%     \hline
%     \multicolumn{1}{c|}{SAM} & 85.1 &  86.7 &  85.9 &  85.5\\
%     \multicolumn{1}{c|}{SAM + MAFT} & 91.0 _\textcolor{red}{+5.9} &  88.6 _\textcolor{red}{+1.9} &  89.8 _\textcolor{red}{+3.9} &  90.4 _\textcolor{red}{+4.9} \\
%     \Xhline{0.7px}
%     \end{tabular}
%     }
%     % \hspace{2mm}

% % b - Ablation on SI module
%     \subfloat[Results in zero-shot segmentation. COCO-Stuff.
%     \label{tab:sam-2}]
%     { 
%     % \!\!\!\!\!
%     % \renewcommand\tabcolsep{11.5pt}
%     \renewcommand\arraystretch{1.1} % 1.95
%     % \footnotesize
%     \small
%     \begin{tabular}
%     {l|llll}
%     % {p{27.8mm}|p{15mm}p{15mm}}
%     \Xhline{0.7px}
%     \centering
%      & \textbf{mIoU$^s$} & \textbf{mIoU$^u$} & \textbf{hIoU} & \textbf{mIoU} \\
%     \hline 
%     \multicolumn{1}{c|}{SAM} & 43.1 &  43.3 &  43.2 &  42.1\\
%     \multicolumn{1}{c|}{SAM + MAFT} & 43.4 _\textcolor{red}{+0.3} &  51.5 _\textcolor{red}{+8.2} &  47.1 _\textcolor{red}{+3.9} & 44.1 _\textcolor{red}{+2.0}\\
%     \Xhline{0.7px}
%     \end{tabular}
%     }
% }
% \resizebox{0.9\textwidth}{!}{
% \centering
%     \hspace{8mm}
%     \subfloat[Results in open-vocabulary segmentation.
%     \label{tab:sam-3}]
%     { 
%     \centering%
%     \footnotesize %scriptsize
%     \renewcommand\tabcolsep{10pt}
%     % \renewcommand\arraystretch{1.2} % 1.95

%     \begin{tabular}{c|lllll}
%       % \toprule
%       \Xhline{0.7pt}

%       & \textbf{A-847} & \textbf{A-150} & \textbf{PC-459} & \textbf{PC-59} & \textbf{PAS-20}\\ 
%       \hline
%        SAM & ~~~7.1 & ~~~17.9 & ~~~6.4 & ~~~34.4 & ~~~85.6 \\
%       SAM + MAFT & ~~~10.1 $_{\textcolor{red}{+3.0}}$ & ~~~29.1 $_{\textcolor{red}{+11.2}}$ & ~~~12.8 $_{\textcolor{red}{+6.4}}$ & ~~~53.5 $_{\textcolor{red}{+19.1}}$ & ~~~90.0 $_{\textcolor{red}{+4.4}}$ \\
%       \Xhline{0.7pt}
%       % \bottomrule
%       \end{tabular}
%     }    
% }
%   \vspace{-5mm}
% \end{table*}



\begin{table*}
\vspace{-2mm}
\caption{Comparison with SAM. We use SAM-H as the proposal generator.}
\label{tab:sam}
% \vspace{-1mm}
\resizebox{1.0\textwidth}{!}{
\centering
% a - Ablation on Statics, Hand-Craft and Learnable
    % \hspace{8mm}
    \subfloat[Results in zero-shot segmentation.
    \label{tab:sam-1}]
    { 
    % \!\!\!\!\!
    % \renewcommand\tabcolsep{11.5pt}
    \renewcommand\arraystretch{1.1} % 1.95
    \footnotesize
    % \small
    \begin{tabular}
    {l|llll|llll}
    \Xhline{0.7px}
    \centering
    & \multicolumn{4}{c|}{Pascal-VOC} & \multicolumn{4}{c}{COCO-Stuff} \\
     & \textbf{mIoU$^s$} & \textbf{mIoU$^u$} & \textbf{hIoU} & \textbf{mIoU} & \textbf{mIoU$^s$} & \textbf{mIoU$^u$} & \textbf{hIoU} & \textbf{mIoU} \\
    \hline
    \multicolumn{1}{l|}{SAM} & 85.1 &  86.7 &  85.9 &  85.5 & 43.1 &  43.3 &  43.2 &  42.1\\
    \multicolumn{1}{l|}{SAM + MAFT} & 91.0 $_{\textcolor{red}{+5.9}}$ &  88.6 $_{\textcolor{red}{+1.9}}$ &  89.8 $_{\textcolor{red}{+3.9}}$ &  90.4 $_{\textcolor{red}{+4.9}}$ & 43.4 $_{\textcolor{red}{+0.3}}$ &  51.5 $_{\textcolor{red}{+8.2}}$ &  47.1 $_{\textcolor{red}{+3.9}}$ & 44.1 $_{\textcolor{red}{+2.0}}$\\
    \Xhline{0.7px}
    \end{tabular}
    }
    % \hspace{2mm}
}
\resizebox{1.0\textwidth}{!}{
\centering
    % \hspace{8mm}
    \subfloat[Results in open-vocabulary segmentation.
    \label{tab:sam-3}]
    { 
    \centering%
    \footnotesize %scriptsize
    \renewcommand\tabcolsep{10pt}
    \renewcommand\arraystretch{1.1} % 1.95

    \begin{tabular}{l|lllll}
      % \toprule
      \Xhline{0.7pt}

      & \textbf{A-847} & \textbf{A-150} & \textbf{PC-459} & \textbf{PC-59} & \textbf{PAS-20}\\ 
      \hline
       SAM & ~~~7.1 & ~~~17.9 & ~~~6.4 & ~~~34.4 & ~~~85.6 \\
      SAM + MAFT & ~~~10.1 $_{\textcolor{red}{+3.0}}$ & ~~~29.1 $_{\textcolor{red}{+11.2}}$ & ~~~12.8 $_{\textcolor{red}{+6.4}}$ & ~~~53.5 $_{\textcolor{red}{+19.1}}$ & ~~~90.0 $_{\textcolor{red}{+4.4}}$ \\
      \Xhline{0.7pt}
      % \bottomrule
      \end{tabular}
    }    
}
  \vspace{-5mm}
\end{table*}

We explore using the Segment Anything Model \cite{kirillov2023segment} (SAM) as the proposal generator. We evaluate the performance with SAM-H using an original CLIP (dubbed $\mathrm{SAM}$) or a mask-aware fine-tuned CLIP (dubbed $\mathrm{SAM+MAFT}$). In fact, SAM can be seamlessly integrated into our framework as the proposal generator. The results are shown in Tab. \ref{tab:sam}. Experiments are conducted under both \textit{zero-shot} setting and \textit{open-vocabulary} setting.

It can be observed that $\mathrm{SAM+MAFT}$ obtains significant improvement over $\mathrm{SAM}$ under both settings. Besides, $\mathrm{SAM+MAFT}$ also surpasses $\mathrm{FreeSeg+MAFT}$ on all benchmarks. Particularly, in the zero-shot setting (Pascal-VOC), $\mathrm{SAM+MAFT}$ outperforms $\mathrm{FreeSeg+MAFT}$ by 6.8\% in terms of mIoU$^u$. This enhancement can be attributed to the stronger generalization capabilities of SAM for unseen classes. 

\subsection{Extending MAFT with more Vision-Language Models}
\begin{table}[h]
  \centering
  \footnotesize
 \vspace{-6mm}
  \caption{Comparison with more Vision-Language Models.}
\resizebox{1.0\textwidth}{!}{

    \renewcommand\arraystretch{1.1} % 1.95
    
    \begin{tabular}{l|c|lllll}
      \Xhline{0.7pt}

      & \textbf{backbone} & \textbf{~A-847} & \textbf{~A-150} & \textbf{PC-459} & \textbf{PC-59} & \textbf{PAS-20}\\ 
      \hline
       OVSeg \cite{ovseg} & \multirow{3}{*}{{ViT-L}} & ~~~9.0 & ~~~29.6 & ~~~12.4 & ~~~55.7 & ~~~94.5 \\
       FreeSeg \cite{freeseg} & & ~~~8.5 & ~~~21.0 & ~~~7.6 & ~~~33.8 & ~~~86.4 \\
      FreeSeg + MAFT & & ~~~12.1 $_{\textcolor{red}{+3.6}}$ & ~~~32.0 $_{\textcolor{red}{+11.0}}$ & ~~~15.7 $_{\textcolor{red}{+8.1}}$ & ~~~58.5 $_{\textcolor{red}{+24.7}}$ & ~~~92.1 $_{\textcolor{red}{+5.7}}$ \\
      \cdashline{1-7}[0.8pt/2pt]
      FreeSeg \cite{freeseg} &\multirow{2}{*}{{Res50}} &  ~~~5.3 & ~~~15.5 & ~~~5.4 & ~~~28.2 & ~~~87.1 \\
      FreeSeg + MAFT & & ~~~8.4 $_{\textcolor{red}{+3.1}}$ & ~~~27.0 $_{\textcolor{red}{+11.5}}$ & ~~~9.9 $_{\textcolor{red}{+4.5}}$ & ~~~50.8 $_{\textcolor{red}{+22.6}}$ & ~~~89.0 $_{\textcolor{red}{+1.9}}$ \\
      \Xhline{0.7pt}
      \end{tabular}
      }
      \label{tab:backbone}

 \end{table}

In order to demonstrate the efficacy and robustness of MAFT, we conduct experiments using stronger (CLIP-ViT-L) and ResNet-based (CLIP-Res50) Vision-Language Models. The open-vocabulary results are shown in Tab. \ref{tab:backbone}, we also include the results of OVSeg with CLIP-ViT-L for comparison.

\noindent \textbf{CLIP-ViT-L.}
According to Tab. \ref{tab:backbone}, FreeSeg with a standard CLIP-ViT-L model (dubbed $\mathrm{FreeSeg}$) still can not achieve satisfactory results. However, by integrating our MAFT (dubbed $\mathrm{FreeSeg+MAFT}$), the segmentation results are remarkably enhanced, thus establishing new state-of-the-art benchmarks.

\noindent \textbf{CLIP-Res50.}
Our MAFT can easily adapted into ResNet-based models. Specifically, we modified the $\mathrm{AttentionPool2d}$ unit within CLIP-R50 Image Encoder. The mask proposals are introduced as attention bias ($B$) in Multihead Attention, with $F_{cls}$ being repeated N times. Notably in CLIP-R50, $F_{cls}$ is obtained via $\mathrm{GlobalAveragePooling}$ performing on $F_{feat}$. The results are presented in Tab. \ref{tab:backbone}. The performance on all 5 datasets is improved by a large margin. $\mathrm{FreeSeg+MAFT}$ with CLIP-R50 achieves competitive results with some CLIP-ViT-B-based methods according to Tab. \ref{tab:ovs}.

\subsection{Qualitative Study}

\noindent \textbf{Visualizations of typical proposals.}
Fig. \ref{fig:vis-proposal} shows frozen CLIP and mask-aware CLIP classifications of typical proposals, 
including high-quality proposals of foreground ($p_1$, $p_4$), high-quality proposals of background ($p_3$, $p_6$), a proposal with background noise ($p_2$), and a proposal containing part of the foreground ($p_5$). The proposal regions are highlighted in green or yellow. \\
Several observations can be obtained: (1) The frozen CLIP provides good predictions for $p_1$ and $p_4$. (2) The frozen CLIP assigns $p_2$ as $cat$ and $p_5$ as $horse$, with scores even higher than $p_1$, $p_4$, indicating the frozen CLIP cannot distinguish proposals containing information on the same objects. (3) The frozen CLIP fails to give correct predictions for $p_3$ and $p_6$, which may be due to the lack of context information. (4) Our mask-aware CLIP gives good predictions for high-quality proposals ($p_1$, $p_3$, $p_4$, $p_6$) and provides suitable predictions for $p_2$ and $p_5$.


\noindent \textbf{Qualitative analysis.}
We show some visual examples in Fig. \ref{fig:vis-final}. Some segmentation results of FreeSeg contain background noise (\textit{e.g.} the $1^{st}$ \& $2^{nd}$ row, $3^{rd}$ column) or contain only part of the objects ($3^{rd}$ row, $3^{rd}$ column). In ADE20K-847 dataset, too many classes may lead to the anticipated results (last row, $3^{rd}$ column) with the frozen CLIP.
Using a mask-aware CLIP to learn mask-aware representations can significantly improve these segmentation results, as evident from the last column.

More visual samples are shown in the Appendix.


\section{Conclusion}
In this paper, we rethink the "frozen CLIP" paradigm in zero-shot segmentation and propose Mask-Aware Fine-Tune (MAFT) for fine-tuning CLIP. 
Firstly, IP-CLIP Encoder is proposed to handle images with any number of mask proposals. Then, $\mathcal{L}_{ma}$ and $\mathcal{L}_{dis}$ are designed for fine-tuning CLIP to be mask-aware without sacrificing its transferability. MAFT is plug-and-play and can be applied to any "frozen CLIP" approach. Extensive experiments well demonstrate the performance of various zero-shot segmentation methods is improved by plugging MAFT.

\textbf{Limitations.}
Our MAFT introduces a CLIP fine-tining framework to the research of zero-shot segmentation. However, the classification ability for novel classes is still limited by pre-trained vision-language models. How to further narrow this limitation is our future research focus.
\begin{figure}
% \vspace{-15mm}
\begin{center}
   \includegraphics[width=0.99\linewidth]{figs/pdf/vis-proposalv2-maskimage.pdf}
\end{center}
\vspace{-2mm}
   \caption{
   Visualizations of typical proposals \& top 5 $A^c$ by \textcolor{cyan}{frozen CLIP} and \textcolor{orange}{mask-aware CLIP}.
   }
\label{fig:vis-proposal}
% \vspace{-2mm}
\end{figure}

\begin{figure}
% \vspace{-3mm}
\begin{center}
   \includegraphics[width=0.99\linewidth]{figs/pdf/vis-finalv2.pdf}
\end{center}
\vspace{-3mm}
   \caption{
   Qualitative results. The models are trained with COCO-Stuff and directly tested on VOC2012, COCO, and ADE20K.
   }
\label{fig:vis-final}
% \vspace{-2mm}
\end{figure}



\newpage

\appendix
\section*{Appendix}
\newpage
\section{Dataset Visualizations}
\label{sec:app_dataset_visuals}

%%%%%%
%%
%%
\subsection{Examples of each view class}
\newcommand{\BC}{0.33}
\setlength{\tabcolsep}{0.1cm}
\begin{figure}[!h]
\begin{tabular}{c c c c}
    PLAX  & PSAX & OTHER 
    \\
    \includegraphics[width=\BC\textwidth]{figures/small_appendix/Appendix_PLAX1.jpg}
    &
    \includegraphics[width=\BC\textwidth]{figures/small_appendix/Appendix_PSAX1.jpg}
    &
    \includegraphics[width=\BC\textwidth]{figures/small_appendix/Appendix_Other1.jpg}
    &
   
    \\
    
    \includegraphics[width=\BC\textwidth]{figures/small_appendix/Appendix_PLAX2.jpg}
    &
    \includegraphics[width=\BC\textwidth]{figures/small_appendix/Appendix_PSAX2.jpg}
    &
    \includegraphics[width=\BC\textwidth]{figures/small_appendix/Appendix_Other2.jpg}
    &
   
     \\
     
     \includegraphics[width=\BC\textwidth]{figures/small_appendix/Appendix_PLAX3.jpg}
    &
    \includegraphics[width=\BC\textwidth]{figures/small_appendix/Appendix_PSAX3.jpg}
    &
    \includegraphics[width=\BC\textwidth]{figures/small_appendix/Appendix_Other3.jpg}
    &
   
     \\
     
     \includegraphics[width=\BC\textwidth]{figures/small_appendix/Appendix_PLAX4.jpg}
    &
    \includegraphics[width=\BC\textwidth]{figures/small_appendix/Appendix_PSAX4.jpg}
    &
    \includegraphics[width=\BC\textwidth]{figures/small_appendix/Appendix_Other4.jpg}
    &
   
    \end{tabular}	
    \caption{Examples of images for each possible view label in our dataset. \emph{From left to right:} Four examples of peristernal long axis (PLAX) view, four examples of peristernal short axis (PSAX) view, and four examples of other kinds of view in our ``Other'' class. }
    \label{fig:VIEW_SAMPLES_APPENDIX}
\end{figure}

%%%%%%
%%
%%
\newpage
\subsection{Examples of each view for a Severe AS patient}
\newcommand{\BA}{0.33}
\setlength{\tabcolsep}{0.1cm}
\begin{figure}[!h]
\begin{tabular}{c c c c}
    PLAX  & PSAX & OTHER 
    \\
    \includegraphics[width=\BA\textwidth]{figures/small_appendix/SevereAS_11112007_PLAX1.jpg}
    &
    \includegraphics[width=\BA\textwidth]{figures/small_appendix/SevereAS_11112007_PSAX1.jpg}
    &
    \includegraphics[width=\BA\textwidth]{figures/small_appendix/SevereAS_11112007_Other1.jpg}
    &
    
    \\
    
    \includegraphics[width=\BA\textwidth]{figures/small_appendix/SevereAS_11112007_PLAX2.jpg}
    &
    \includegraphics[width=\BA\textwidth]{figures/small_appendix/SevereAS_11112007_PSAX2.jpg}
    &
    \includegraphics[width=\BA\textwidth]{figures/small_appendix/SevereAS_11112007_Other2.jpg}
    &
   
     \\
     
     \includegraphics[width=\BA\textwidth]{figures/small_appendix/SevereAS_11112007_PLAX3.jpg}
    &
    \includegraphics[width=\BA\textwidth]{figures/small_appendix/SevereAS_11112007_PSAX3.jpg}
    &
    \includegraphics[width=\BA\textwidth]{figures/small_appendix/SevereAS_11112007_Other3.jpg}
    &
  
    \end{tabular}	
    \caption{Examples of images from a patient with Severe AS in our dataset. \emph{From left to right:} Three examples of parasternal long axis (PLAX) view, three examples of parasternal short axis (PSAX) view, and three examples of other kinds of view in our ``Other'' class. }
    \label{fig:PatientSevereAS}
\end{figure}


%%%%%%
%%
%%
\newpage
\subsection{Examples of each view for a No AS patient}
\newcommand{\BB}{0.33}
\setlength{\tabcolsep}{0.1cm}
\begin{figure}[!h]
\begin{tabular}{c c c c}
    PLAX  & PSAX & OTHER 
    \\
    \includegraphics[width=\BB\textwidth]{figures/small_appendix/NoAS_1996889_PLAX1.jpg}
    &
    \includegraphics[width=\BB\textwidth]{figures/small_appendix/NoAS_1996889_PSAX1.jpg}
    &
    \includegraphics[width=\BB\textwidth]{figures/small_appendix/NoAS_1996889_Other1.jpg}
    &
    
    \\
    
    \includegraphics[width=\BB\textwidth]{figures/small_appendix/NoAS_1996889_PLAX2.jpg}
    &
    \includegraphics[width=\BB\textwidth]{figures/small_appendix/NoAS_1996889_PSAX2.jpg}
    &
    \includegraphics[width=\BB\textwidth]{figures/small_appendix/NoAS_1996889_Other2.jpg}
    &
   
     \\
     
     \includegraphics[width=\BB\textwidth]{figures/small_appendix/NoAS_1996889_PLAX3.jpg}
    &
    \includegraphics[width=\BB\textwidth]{figures/small_appendix/NoAS_1996889_PSAX3.jpg}
    &
    \includegraphics[width=\BB\textwidth]{figures/small_appendix/NoAS_1996889_Other3.jpg}
    &
  
    \end{tabular}	
    \caption{Examples of images from a patient with No AS in our dataset. \emph{From left to right:} Three examples of parasternal long axis (PLAX) view, three examples of parasternal short axis (PSAX) view, and three examples of other kinds of view in our ``Other'' class. }
    \label{fig:PatientNoAS}
\end{figure}



\newpage 
\section{Further Results}

\subsection{Assessment of ensembling}

Table~\ref{tab:best_single_checkpoint_VS_ensemble_FS_echo260} compares using a single checkpoint (one point estimate of neural network weight vector $\theta$) to using an ensemble of parameters aggregated from the last 25 checkpoints (one per epoch).

\begin{table}[!h]
    \centering
    \begin{tabular}{c|cccc|c}
    \textit{Diagnosis classification} & Split 1  & Split 2 & Split 3 & Split 4 & Average\\
    \hline
    Best single checkpoint  & 61.81 & 59.79 & 56.05 & 64.21 & 60.46\\
    Ensemble  & 62.95 & 61.03 & 56.58 & 63.84 & \textbf{61.13}
	\\ \hline
    \textit{View classification}  &   &  &  &  & 
    \\ \hline
    Best single checkpoint  & 93.03 & 93.24 & 92.39 & 93.79 & 93.11\\
    Ensemble  & 92.37 & 93.24 & 93.72 & 93.87 & \textbf{93.30}\\
    \end{tabular}
    \caption{Comparing best single checkpoint performance with ensemble performance on \textbf{Full-size \datasetName-156-52}}
    \label{tab:best_single_checkpoint_VS_ensemble_FS_echo260}
\end{table}


%%%%%%
%%
%%
\subsection{Patient-level diagnosis performance on bonus heldout set}

Table~\ref{tab:diagnosis classification patient unlabeled_heldout_174} examines the performance of the best labeled-set-only methods and MixMatch methods on the 174 patient studies that have diagnosis but no view labels.
 While the images used here were originally included in the unlabeled training set (which was used to train SSL methods like MixMatch), the diagnosis labels were not provided at all during training time. 
 We thus still believe this is an authentic test of generalization given the scarcity of labeled data available for our task.
 Of course, additional independent evaluation (especially from another institution) is needed.

\begin{table}[!h]
    \centering
    \begin{tabular}{l l l|rrrr|c}
    Pretrain & Method & Voting
    & Split 1  & Split 2 & Split 3 & Split 4 & average\\
    \hline
    & Basic WRN & Simple average & 76.73 & 75.25 & 76.87 & 81.88 & 77.68\\
    & Basic WRN & View-prioritized & 73.63 & 83.21 & 79.70 & 80.08 & 79.18\\
    %SSL & FS & MixMatch & Priority view + confidence & 94.58 & 84.17 & 77.50 & 92.5 & 87.19\\
    \hline
    & MixMatch & Simple average & 85.32 & 76.29 & 74.14 & 79.95 & 78.93\\
    view & MixMatch & Simple average & 83.36 & 77.96 & 75.61 & 81.37 & 79.58\\
    & MixMatch & View-prioritized & 83.27 & 83.76 & 82.34 & 82.83 & \textbf{83.05}\\
    view & MixMatch & View-prioritized & 82.53 & 86.15 & 79.62 & 83.27 & 82.89\\
    %view & MixMatch & LR with view-priority & 80.42 & 84.24 & 76.58 & 80.67 & 80.48\\
    %(MixMatch transfered) + MysteryMethod & NA & NA & NA\\ 
    \end{tabular}
    \caption{Patient-level AS Severity Diagnosis Classification on the \textbf{bonus heldout set} of 174 patients for whom we have diagnosis labels only (no view labels). We show balanced accuracy on models trained on each of the four folds on four \textbf{full-size \datasetName-156-52} dataset.
    }%endcaption
    \label{tab:diagnosis classification patient unlabeled_heldout_174}
\end{table}


%%%%%%
%%
%%
\subsection{Assessment of MixMatch hyperparameter sensitivity}

In Table~\ref{tab:MixMatch hyperparameters ablation study}, we consider four possible strategies for setting the hyperparameters of MixMatch, varying two  key settings for the weight on unlabeled loss $\lambda$. First, we vary whether the final value of $\lambda$ is set to its \emph{best} value among a grid of candidates (based on validation set performance), or \emph{fixed} to a constant.
Second, we vary whether $\lambda$ remains fixed over iterations throughout a training run, or is updated over iterations on a linear ramp schedule from 0 to its final target value. 

From this comparison, we see we consistent gains across splits (average gain across splits of over 1.6\% balanced accuracy) for using a delayed ramp up schedule with target value selected via grid search.

\begin{table}[!h]
    \centering
    \begin{tabular}{l l| rrrr | r}
    Final $\lambda$ value & $\lambda$ update schedule & Split 1  & Split 2 & Split 3 & Split 4 & Average\\
    \hline
    best on val & Delayed ramp-up  & 65.57 & 62.69 & 60.87 & 66.29 & 63.86\\
    best on val & Immediate ramp-up & 65.07 & 61.87 & 60.82 & 65.37 & 63.28\\
    best on val & Constant  & 65.03 & 61.52 & 58.87 & 65.22 & 62.66\\
    100 (fixed) & Constant & 63.94 & 61.79 & 58.87 & 64.35 & 62.24\\
    \end{tabular}
    \caption{Ablation study of different settings of the unlabeled loss weight $\lambda$ for MixMatch. AS severity diagnosis classification for individual images on the \textbf{full-size \datasetName-156-52} dataset. showing balanced accuracy averaged over the test sets from multiple folds (each fold’s test set contains all images from 52 patients). }%endcaption
    \label{tab:MixMatch hyperparameters ablation study}
\end{table}



%%%%%%
%%
%%
\subsection{Assessment of alternative view prioritization strategy using thresholding}


An anonymous reviewer suggested an alternative strategy for prioritizing images of relevant view.
The alternative strategy works as follows: for each image, we compute the predicted probability that the image is a ``relevant view'' (either PLAX and PSAX) by summing the probabilities of each view type.
However, instead of using this raw probability as a weight (as our chosen method does), we use a \emph{cutoff threshold} and simply average the diagnosis predictions of images whose relevant view probability is above the cutoff.
For each patient, we use the majority vote prediction of the diagnosis from the images of relevant views.
The value of the cutoff threshold is selected using the validation set to maximize balanced accuracy.

Table~\ref{tab:Suggested_Aggregation_Ablation} shows the performance of this strategy (``threshold-then-average'') on the full-size dataset.
Using this alternative prioritization strategy together with our suggested methodology for patient-level diagnosis (using MixMatch, pretraining on view), we find the average test set balanced accuracy is around 85.8\%, while the weighted average strategy in the main paper achieves over 90\% balanced accuracy. We take this as reasonably decisive evidence that a weighted average (rather than a simple cutoff) should be preferred.

\begin{table}[!h]
    \centering
    \begin{tabular}{l l l|rrrr|c}
    Pretrain & Method & Aggregation across images
    & Split 1  & Split 2 & Split 3 & Split 4 & average\\
    \hline
    & Basic WRN & Threshold-then-Average & 85.42 & 86.25 & 79.17 & 92.50 & 85.84 \\
    %SSL & FS & MixMatch & Priority view + confidence & 94.58 & 84.17 & 77.50 & 92.5 & 87.19\\
    & MixMatch & Threshold-then-Average & 83.33 & 84.17 & 77.50 & 94.58 & 84.90 \\
    view & MixMatch & Threshold-then-Averagen & 86.67 & 80.00 & 82.50 & 94.17 & 85.84\\
    %view & MixMatch & LR with view-priority & 87.08 & 82.08 & 85.00 & 88.75 & 85.73\\
    %(MixMatch transfered) + MysteryMethod & NA & NA & NA\\ 
    \end{tabular}
    \caption{Alternative view-prioritizing strategy for patient-level AS severity diagnosis classification on the \textbf{full-size \datasetName-156-52} dataset, showing balanced accuracy on the test set across multiple folds (each fold’s test set contains 52 patients).}
    %endcaption
    \label{tab:Suggested_Aggregation_Ablation}
\end{table}



%%%%%%
%%
%%
\subsection{ROC Curve of patient-level diagnosis: no AS vs. mild/moderate/severe AS}

Fig.~\ref{fig: No AS vs Some AS} shows receiver operating curves for several methods for the task of distinguishing no AS vs Some AS (which aggregates both the mild/moderate and severe levels in the 3-level diagnosis task of the main paper).

\begin{figure}[!h]
\begin{tabular}{c c}
	\includegraphics[width=0.43\textwidth]{figures/fold0_multitask_PatientLevel_NoVSSome_NormalizedPriorityStrategyClassProbabilityScore.pdf}
	&
    \includegraphics[width=0.43\textwidth]{figures/fold1_multitask_PatientLevel_NoVSSome_NormalizedPriorityStrategyClassProbabilityScore.pdf}
	\\
	(a) Split 1 & (b) Split 2
	\\
	\includegraphics[width=0.43\textwidth]{figures/fold2_multitask_PatientLevel_NoVSSome_NormalizedPriorityStrategyClassProbabilityScore.pdf}
	&
    \includegraphics[width=0.43\textwidth]{figures/fold3_multitask_PatientLevel_NoVSSome_NormalizedPriorityStrategyClassProbabilityScore.pdf}
	\\
	(c) Split 3 & (d) Split 4
\end{tabular}
    
\caption{ROC curves for binary diagnosis task (no AS vs ``mild/moderate/severe AS'') on \textbf{full-size \datasetName-156-52}.
    }%endcaption
    \label{fig: No AS vs Some AS}
\end{figure}

\section{Methodological Details}

\subsection{Image processing details}
\label{sec:removing_doppler}

\paragraph{Removing doppler images.}
In the raw data of all imagery available for an echocardiogram study, 
we obtained TIFF files that represent both cineloops and Doppler images.

We verified in our labeled set that all Doppler images have one of the following landscape aspect ratio $(831, 323)$, $(901, 384)$, $(901, 390)$, $(704, 305)$, $(831, 421)$, $(901, 469)$ or $(563, 294)$. Only the Dopplers have these aspect ratios. We thus filtered out Doppler completely via these aspect ratios. 

\paragraph{Downsizing}
The original images are provided as high-resolution TIFF format images (hundreds of pixels per side) of varying aspect ratios. Generally, we can expect that both view and diagnosis classifiers would perform better given higher-resolution input (and holding other factors the same). The main trade-off of processing higher-resolution images is increased runtime and memory requirements. In our preliminary experiments, we compared downsizing all images to a standard square aspect ratio at 3 possible sizes: 32x32, 64x64 and 128x128. We found that 64x64 achieves a good balance between model performance and computation cost. 
A prior study by \citet{madaniDeepEchocardiographyDataefficient2018} provides a more extensive study of optimal resolution size. The interested reader can refer to their work for more details. 


\subsection{Architecture Settings and Hyperparameters}
\label{sec:arch_and_hyperparameters}

\paragraph{Weighted cross-entropy for labeled loss}
To counteract the effect of class imbalance in the dataset, we use weighted cross-entropy for the labeled loss. For an input image $x$ whose true label $y$ indicates it belongs to class $c$, the weighted cross-entropy assumes the following form:
\begin{align}
\mathcal{L}^L(\theta, x) = - w_{c} \log \hat{p}_{c}(\theta, x),
\end{align}
where $\hat{p}_{c}$ is the predicted probability of class $c$. The weight $w_{c}$ is calculated using the training set statistics as follow:
\begin{align}
w_{c} = \frac{\prod_{k\neq c}{N_{k}}}{\sum_{j}\prod_{k \neq j}{N_{k}}}
\end{align}
where $N_{k}$ is the number of images of class $k$ in the training set.

\paragraph{Common architecture.}
Following~\citet{oliverRealisticEvaluationDeep2018}, for all considered methods, we use the \emph{same} backbone neural network architecture: a wide residual network~\citep{zagoruykoWideResidualNetworks2017} with 28 layers (WRN-28), which has total of 5,931,683 parameters.
This same network architecture is used in the original MixMatch evaluation~\citep{berthelotMixmatchHolisticApproach2019} with promising results.

\paragraph{Common training protocol.}
All SSL methods we consider follow the loss minimization framework with two primary losses (one for ``labeled'' data and one for ``unlabeled'' data) in Eq.~\eqref{eq:standard-SSL-loss-template}.
We allow every method to train for 32 epochs (where each epoch processes $2^{16}$ images, as in \citet{berthelotMixmatchHolisticApproach2019}).
Our preliminary experiments suggest that after 30 epochs all methods effectively converge in terms of validation balanced accuracy. 

\paragraph{Common regularization.}
For all methods, we expect performance will be vulnerable to overfitting, so we impose an L2-norm penalty on the weights $\theta$, also known as weight decay. Each method selects its preferred value of this penalty strength hyperparameter. We searched values in [0.0002, 0.002, 0.02].

\paragraph{Common optimization.}
We use ADAM \citep{kingma2014adam} to optimize each model.
Each method selects the value of the step size (learning rate) as a hyperparameter. We experimented with 0.002 and 0.0007
%HZ: 'performance being sensitive to learning rate' is very reasonable. But we don't have an ablation to back it. 
%We find performance is sensitive to the step size (learning rate) hyperparameter, so we perform a grid search and select the value that maximizes balanced accuracy on the validation set.

\paragraph{Hyperparameters for Pseudo-Label.}
Beyond the usual hyperparameters for our loss-minimization SSL framework, another important hyperparameter for pseudo-label is the threshold $\tau$. We find that performance is not very sensitive to the chosen $\tau$ value as long as it is within a certain range. We set $\tau$ to 0.95, as done in past literature that evaluates Pseudo-Label as an SSL method ~\citep{oliverRealisticEvaluationDeep2018,berthelotMixmatchHolisticApproach2019, berthelotRemixmatchSemisupervisedLearning2019, sohnFixmatchSimplifyingSemisupervised2020}.


\paragraph{Hyperparameters for VAT.}
Beyond the usual hyperparameters for our SSL framework, for VAT we need to select a value for $\epsilon$.
In \citet{miyatoVirtualAdversarialTraining2019}, the authors claimed that they can achieve superior performance by tuning only $\epsilon$ and fixing $\lambda$ to 1. In our experiment, we used the default $\lambda$ as in \cite{berthelotMixmatchHolisticApproach2019} and searched the value of $\epsilon$ in [2, 6, 18], together with learning rate and weight decay. We select the best hyperparameters using validation set performance. 


\paragraph{Hyperparameters for MixMatch.}
Beyond the usual hyperparameters for our SSL framework, the key hyperparameters for MixMatch include the number of augmentations $K$, the temperature $T>0$ used for sharpening, interpolation hyperparameter $\alpha$ and unlabeled loss coefficient $\lambda$. We set $K=2$, $T=0.5$, and $\alpha=0.75$ as done in \citet{berthelotMixmatchHolisticApproach2019}, and search for $\lambda$ in the range [10, 30, 75, 100, 130] using validation set. 

\paragraph{Hyperparameters for Multitask training.}
We searched $\gamma$, the hyperparameter that control the strength of the auxilliary view loss in Eq.~\eqref{eq:multitask}, in the range [10, 3, 1, 0.3, 0.1]. The best $\alpha$ is selected together with other hyperparameters on validation set. 


\newpage

% {\small
\bibliographystyle{plain}
\bibliography{egbib}
% }



\end{document}