\def\TheAcknowledgments{I would like to thank Jan Burse for discussions on the SWI Prolog
mailing list about using Prolog for differentiation and for prompting me to
investigate the problem of computing multiple derivatives (for the purpose of finding
Taylor series coefficients), which led to the second, improved implementation presented here).}

\def\TheAuthors{Samer Abdallah (\texttt{samer@jukedeck.com})}
\def\TheInstitution{Jukedeck Ltd.}

\def\TheTitle{Automatic Differentiation using Constraint Handling Rules in Prolog}

\def\TheAbstract{Automatic differentiation is a technique which allows a programmer
to define a numerical computation via compositions of a broad range of
numeric and computational primitives and have the underlying system support
the computation of partial derivatives of the result with respect
to any of its inputs, without making any finite difference approximations, and without
manipulating large symbolic expressions representing the computation. This note describes 
a novel approach to reverse mode automatic differentiation using constraint logic programmming,
specifically, the constraint handling rules (CHR) library of SWI Prolog,
resulting in a very small (50 lines of code) implementation. 
When applied to a differentiation-based implementation of the inside-outside
algorithm for parameter learning in probabilistic grammars, the CHR based implementations
outperformed two well-known frameworks for optimising differentiable functions, Theano and TensorFlow, 
by a large margin.}
