% CVPR 2023 Paper Template
% based on the CVPR template provided by Ming-Ming Cheng (https://github.com/MCG-NKU/CVPR_Template)
% modified and extended by Stefan Roth (stefan.roth@NOSPAMtu-darmstadt.de)

\documentclass[10pt,twocolumn,letterpaper]{article}

%%%%%%%%% PAPER TYPE  - PLEASE UPDATE FOR FINAL VERSION
% \usepackage[review]{cvpr}      % To produce the REVIEW version
\usepackage{cvpr}              % To produce the CAMERA-READY version
%\usepackage[pagenumbers]{cvpr} % To force page numbers, e.g. for an arXiv version

% Include other packages here, before hyperref.
\usepackage{graphicx}
\usepackage{amsmath}
\usepackage{amssymb}
\usepackage{booktabs}
\usepackage{bm}
\usepackage{multicol,multirow}
\usepackage[accsupp]{axessibility}
% \usepackage{subfigure}

% It is strongly recommended to use hyperref, especially for the review version.
% hyperref with option pagebackref eases the reviewers' job.
% Please disable hyperref *only* if you encounter grave issues, e.g. with the
% file validation for the camera-ready version.
%
% If you comment hyperref and then uncomment it, you should delete
% ReviewTempalte.aux before re-running LaTeX.
% (Or just hit 'q' on the first LaTeX run, let it finish, and you
%  should be clear).
\usepackage[pagebackref,breaklinks,colorlinks,urlcolor=black]{hyperref}


% Support for easy cross-referencing
\usepackage[capitalize]{cleveref}
\crefname{section}{Sec.}{Secs.}
\Crefname{section}{Section}{Sections}
\Crefname{table}{Table}{Tables}
\crefname{table}{Tab.}{Tabs.}

\newcommand{\cpy}[1]{\textcolor{red}{#1}}

%%%%%%%%% PAPER ID  - PLEASE UPDATE
\def\cvprPaperID{1697} % *** Enter the CVPR Paper ID here
\def\confName{CVPR}
\def\confYear{2023}


\begin{document}

%%%%%%%%% TITLE - PLEASE UPDATE
\title{Q-DETR: An Efficient Low-Bit Quantized Detection Transformer}

\author{Sheng~Xu\textsuperscript{1}$^{\dagger}$, Yanjing~Li\textsuperscript{1}$^{\dagger}$, Mingbao~Lin\textsuperscript{3}, 
Peng~Gao\textsuperscript{4},
Guodong~Guo\textsuperscript{5},
Jinhu~L\"u\textsuperscript{1,2},
Baochang~Zhang\textsuperscript{1,2}$^{*}$\\
\textsuperscript{1} Beihang University \quad
\textsuperscript{2} Zhongguancun Laboratory\quad
\textsuperscript{3} Tencent\\
\textsuperscript{4} Shanghai AI Laboratory \quad 
\textsuperscript{5} UNIUBI Research, Universal Ubiquitous Co. \\
}
\maketitle
\newcommand\blfootnote[1]{% 
\begingroup 
\renewcommand\thefootnote{}\footnotetext{#1}% 
\addtocounter{footnote}{0}% 
\endgroup 
}
\blfootnote{$^{\dagger}$ Equal contribution.}
\blfootnote{$^*$ Corresponding author: bczhang@buaa.edu.cn}
\blfootnote{$^1$ Code: \url{https://github.com/SteveTsui/Q-DETR}}
%%%%%%%%% ABSTRACT
\begin{abstract}
%
%
%
The recent detection transformer~(DETR) has advanced object detection, but its application on resource-constrained devices requires massive computation and memory resources. Quantization stands out as a solution by representing the network in low-bit parameters and operations. However, there is a significant performance drop when performing low-bit quantized DETR (Q-DETR) with existing quantization methods. We find that the bottlenecks of Q-DETR come from the query information distortion through our empirical analyses. This paper addresses this problem based on a distribution rectification distillation (DRD). We  formulate our DRD as a bi-level optimization problem, which can be derived by generalizing the information bottleneck (IB) principle to the learning of Q-DETR. At the inner level, we conduct a distribution alignment for the queries to maximize the self-information entropy. At the upper level, we  introduce a new foreground-aware query matching scheme to effectively transfer the teacher information to distillation-desired features to minimize the conditional information entropy. Extensive experimental results show that our method performs much better than  prior arts. For example, the 4-bit Q-DETR can theoretically accelerate DETR with ResNet-50 backbone by 6.6$\times$ and achieve 39.4\% AP, with only 2.6\% performance gaps than its real-valued counterpart on the COCO dataset\,$^1$.
\end{abstract}

%%%%%%%%% BODY TEXT
\section{Introduction}
\label{sec:intro}

Inspired by the success of natural language processing (NLP), object detection with transformers (DETR) has been introduced to train an end-to-end detector via a transformer encoder-decoder~\cite{carion2020end}. Unlike early works~\cite{ren2016faster,liu2016ssd} that often employ convolutional neural networks (CNNs) and require post-processing procedures, {\em e.g.}, non-maximum suppression (NMS), and hand-designed sample selection, DETR treats object detection as a direct set prediction problem.% {\color{blue} DETR constructs the spatial dependencies between $N$ object queries and encoded features. Then the co-attended objected queries are fed into box coordinates and class labels, resulting $N$ final predictions.}
%
%

Despite this attractiveness, DETR usually has a tremendous number of parameters and float-pointing operations (FLOPs). 
For instance, there are 39.8M parameters taking up 159MB memory usage and 86G FLOPs in the DETR model with ResNet-50 backbone~\cite{he2016deep} (DETR-R50). 
This leads to an unacceptable memory and computation consumption during inference, and challenges deployments on devices with limited supplies of resources.

%This limits these models for the deployment on resource-limited platforms. Therefore, compressed DETR methods are urgently needed for real applications. 

\begin{figure}
    \centering
    \includegraphics[scale=.36]{motivation1.pdf} %attention_distance.pdf
    \caption{The histogram of query values ${\bf q}$ (blue shadow) and corresponding PDF curves (red curve) of Gaussian distribution~\cite{li2022q}, {\em w.r.t} the cross attention of different decoder layers in (a) real-valued DETR-R50, and (b) 4-bit quantized DETR-R50 (baseline). 
    %
    Gaussian distribution is generated from the statistical mean and variance of the query values.
    %
    The query in quantized DETR-R50 bears information distortion compared with the real-valued one. Experiments are performed on the VOC dataset~\cite{voc2007}.
    }
    \label{fig:motivation1}
\end{figure}

%
Therefore, substantial efforts on network compression have been made towards efficient online inference~\cite{denil2013predicting,xu2021layer,xu2022ida,romero2014fitnets}. Quantization is particularly popular for deploying on AI chips by representing a network in low-bit formats.
%
Yet prior post-training quantization (PTQ) for DETR~\cite{liu2021post} derives quantized parameters from pre-trained real-valued models, which often restricts the model performance in a sub-optimized state due to the lack of fine-tuning on the training data. In particular, the performance drastically drops when quantized to ultra-low bits (4-bits or less).
%
Alternatively, quantization-aware training (QAT)~\cite{liu2020reactnet,xu2022recurrent} performs quantization and fine-tuning on the training dataset simultaneously, leading to trivial performance degradation even with significantly lower bits. Though QAT methods have been proven to be very effective in compressing CNNs~\cite{liu2018bi,esser2019learned} for computer vision tasks, an exploration of low-bit DETR remains untouched. 




\begin{figure}
    \centering
    \includegraphics[scale=.25]{motivation2.pdf} 
    \caption{Spatial attention weight maps in %the cross-attention layer of 
    the last decoder of (a) real-valued DETR-R50, and (b) 4-bit quantized DETR-R50.
    % 
    The green rectangle denotes the ground-truth bounding box.
    Following~\cite{meng2021conditional}, the highlighted area denotes the large attention weights in the selected four heads in compliance with bound prediction. Compared to its real-valued counterpart that focuses on the ground-truth bounds, quantized DETR-R50 deviates significantly.
    %
%    In contrary to the real-valued DETR-R50, the spatial attention weight maps responsible for the top, left, and right edges (the 2-nd, 3-rd and 4-th images in the second row) from 4-bit DETR-R50 cannot highlight the extremities satisfactorily. The green box is the ground-truth box. The highlighted area denotes the large attention weights, in which each head relates to an edge of predicted bounding boxes. 
    }
    \label{fig:motivation2}
\end{figure}

In this paper, we first build a low-bit DETR baseline, a straightforward solution based on common QAT techniques~\cite{bhalgat2020lsq}. Through an empirical study of this baseline, we observe significant performance drops on the VOC~\cite{voc2007} dataset. For example, a 4-bit quantized DETR-R50 using LSQ~\cite{esser2019learned} only achieves 76.9\% AP$_{50}$, leaving a 6.4\% performance gaps compared with the real-valued DETR-R50. 
%
%
We find that the incompatibility of existing QAT methods mainly stems from the unique attention mechanism in DETR, where the spatial dependencies are first constructed between the object queries and encoded features. Then the co-attended object queries are fed into box coordinates and class labels by a feed-forward network. A simple application of existing QAT methods on DETR leads to query information distortion, and therefore the performance severely degrades.
%
%
Fig.\,\ref{fig:motivation1} exhibits an example of information distortion in query features of 4-bit DETR-R50, where we can see significant distribution variation of the query modules in quantized DETR and real-valued version.
% 
% More statistics on the VOC dataset~\cite{voc2007} are provided in the supplementary material, where similar phenomena can also be observed.
%
%
The query information distortion causes the inaccurate focus of spatial attention, which can be verified by following~\cite{meng2021conditional} to visualize the spatial attention weight maps in 4-bit and real-valued DETR-R50 in Fig.\,\ref{fig:motivation2}. We can see that the quantized DETR-R50 bear's inaccurate object localization. Therefore, a more generic method for DETR quantization is necessary.




%As shown in the second row, the spatial attention weight maps from the cross-attention in 4-bit DETR-R50 do not correctly highlight the left, right and top bounds for the corresponding extremities. Consequently, the quantized DETR-R50 fails to precisely localize the foreground and determine the spatial range. This phenomenon indicates that the co-attended?? co-attention??? features centering around the ground truth in quantized DETR-R50 are rectification-desired.  
%\textbf{The logic of this paragraph is not clear to me.....!!!}
%??????.

To tackle the issue above, we propose an efficient low-bit quantized DETR (Q-DETR) by rectifying the query information of the quantized DETR as that of the real-valued counterpart. Fig.\,\ref{fig:framework} provides an overview of our Q-DETR, which is mainly accomplished by a distribution rectification knowledge distillation method (DRD). We find ineffective knowledge transferring from the real-valued teacher to the quantized student primarily because of the information gap and distortion. 
%
Therefore, we formulate our DRD as a bi-level optimization framework established on the information bottleneck principle (IB). Generally, it includes an inner-level optimization to maximize the self-information entropy of student queries and an upper-level optimization to minimize the conditional information entropy between student and teacher queries.
%
At the inner level, we conduct a distribution alignment for the query guided by its Gaussian-alike distribution, as shown in Fig.\,\ref{fig:motivation1}, leading to an explicit state in compliance with its maximum information entropy in the forward propagation. At the upper level, we introduce a new foreground-aware query matching that filters out low-qualified student queries for exact one-to-one query matching between student and teacher, providing valuable knowledge gradients to push minimum conditional information entropy in the backward propagation.


This paper attempts to introduce a generic method for DETR quantization. The significant contributions in this paper are outlined as follows: 
%
(1) We develop the first QAT quantization framework for DETR, dubbed Q-DETR.
(2) We use a bi-level optimization distillation framework, abbreviated as DRD.
(3) We observe a significant performance increase compared to existing quantized baselines.



% 
% hard to be optimized directly due to 

% which is however ineffective 

% which is  formulated as  a bi-level optimization  based on the information bottleneck principle for the Q-DETR learning.   
% 
%At the inner-level optimization, we conduct a distribution alignment (DA) method to maximize the self information entropy of queries, { which effectively bridges the information gap from the distribution perspective.}
%At the upper-level optimization, we develop a foreground-aware query matching (FQM) to effectively localize the distillation-desired student queries, inspired by Fig.\,\ref{fig:motivation2}.  
%After that, we achieve a powerful distribution rectification distillation (DRD) method for quantized DETR methods (Q-DETR). Our contributions are summarized as: 

%\begin{itemize}
%	\item To the best of our knowledge, we propose the first quantized DETR framework, dubbed as Q-DETR. We  lead  an effective distribution rectification distillation (DRD) method with   a bi-level formulation.

%	\item In our DRD framework, we first solve the inner-level optimization by maximizing the self information entropy. We further develop a foreground-aware query matching scheme to effectively localize the distillation-desired student queries, by minimizing the conditional entropy in the upper level.

%	\item We evaluate our Q-DETR on the VOC and the large-scale COCO datasets compared to state-of-the-art quantized baseline and KD methods. The results show that our methods outperform others by a sizable margin. For example, the 4-bit Q-DETR-R50 achieves 39.6\% mAP with about 6.6$\times$ acceleration on COCO, leaving small gaps compared with the real-valued counterparts.
% \end{itemize}


\begin{figure*}
\centering
\includegraphics[scale=.28]{framework.pdf} %attention_distance.pdf
\caption{Overview of the proposed Q-DETR framework. We introduce the distribution rectification distillation method (DRD) to refine the performance of Q-DETR. From left to right, we respectively show the detailed decoder architecture of Q-DETR and the learning framework of Q-DETR. The Q-Backbone, Q-Encoder, and Q-Decoder denote quantized architectures, respectively.}
\label{fig:framework}
\end{figure*}




\section{Related Work}

\textbf{Quantization}. Quantized neural networks often possess low-bit (1$\sim$4-bit) weights and activations to accelerate the model inference and save memory. 
%
For example, DoReFa-Net~\cite{zhou2016dorefa} exploits convolution kernels with low bit-width parameters and gradients to accelerate training and inference. 
%
TTQ~\cite{zhu2016trained} uses two real-valued scaling coefficients to quantize the weights to ternary values.
%
Zhuang~\textit{et al.}~\cite{Zhuang_2018_CVPR} present a $2\!\sim\!4$-bit quantization scheme using a two-stage approach to alternately quantize the weights and activations, providing an optimal tradeoff among memory, efficiency, and performance. 
%
In~\cite{jung2019learning}, the quantization intervals are parameterized, and optimal values are obtained by directly minimizing the task loss of the network. %and also the accuracy degeneration with further bit-width reduction. 
%
ZeroQ~\cite{cai2020zeroq} supports uniform and mixed-precision quantization by optimizing for a distilled dataset which is engineered to match the statistics of the batch normalization across different network layers. 
%
Xie~\textit{et al.}~\cite{xie2020deep} introduced transfer learning into network quantization to obtain an accurate low-precision model by utilizing Kullback-Leibler (KL) divergence.
%
Fang~\textit{et al.}~\cite{fang2020post} enabled accurate approximation for tensor values that have bell-shaped distributions with long tails and found the entire range by minimizing the quantization error. 
Li~\textit{et al.}~\cite{li2022q} proposed an information rectification module and distribution-guided distillation to push the bit-width in a quantized vision transformer. At the same time, we address the quantization in DETR from the IB principle.  
%
The architectural design has also drawn increasing attention using extra shortcut~\cite{liu2018bi}, and parallel parameter-free shortcuts~\cite{liu2020reactnet} for example.

\textbf{Detection Transformer}.
%
Driven by the success of transformers~\cite{vaswani2017attention}, several researchers have also explored transformer frameworks for vision tasks. The first DETR~\cite{carion2020end} work introduces the Transformer structure based on the attention mechanism for object detection. But the main drawback of DETR lies in the  highly inefficient training process. %, which becomes the main motivation for many following works. 
The approachh of another work modifies the multi-head attention mechanism (MHA). Deformable-DETR~\cite{zhu2020deformable} constructs a sparse and point-to-point MHA mechanism using a static point-wise query sampling method around the reference points. SMCA-DETR~\cite{gao2021fast} introduces a Gaussian-distributed spatial function  before formulating a spatially modulated co-attention. DAB-DETR~\cite{liu2022dab} re-defines the query of DETR as dynamic anchor boxes and performs soft ROI pooling layer-by-layer in a cascade manner. DN-DETR~\cite{li2022dn} introduces query denoising into query generation, reducing the bipartite graph matching difficulty and leading to faster convergence. Another set of arts improves DETR methods using additional learning constraints. For example, UP-DETR~\cite{dai2021up} proposes a novel self-supervised loss to enhance the convergence speed and the performance of DETR. 


%
%
However, prior arts mainly focus on the training efficiency of DETR, few of which have discussed the quantization of DETR. 
{To this end, we first build a  quantized DETR baseline and then address the query information distortion problem   based on the IB principle. Finally, a new KD method based on a foreground-aware query matching scheme is achieved  to solve Q-DETR effectively.
}
% Another drawback of some prior arts is that they are unfriendly to quantization due to the 

\section{The Challenge of Quantizing DETR}

\subsection{Quantized DETR baseline}
%
%
%
We first construct a baseline to study the low-bit DETR since no relevant work has been previously proposed.
%
To this end, we follow LSQ+~\cite{bhalgat2020lsq} to introduce a general framework of asymmetric activation quantization and symmetric weight quantization:
%
%
\begin{equation}\label{eq:quantization}
\small
    \begin{aligned}
    \bm{x}_q = &\lfloor \operatorname{clip}\{\frac{(\bm{x} - z)}{\alpha_x}, Q_n^x, Q_p^x\} \rceil, {\bf w}_{q} = \lfloor \operatorname{clip}\{\frac{{\bf w}}{\alpha_{\bf w}}, Q_n^{\bf w}, Q_p^{\bf w}\} \rceil,
    \\&
    Q_a(x) = \alpha_x \circ \bm{x}_q + z, \;\;\;\;\;\;\;\;\;\;\;\;\, Q_w(x) = \alpha_{\bf w}\circ {\bf w}_{q},
    \end{aligned}
\end{equation}
%
%
where $\operatorname{clip}\{y, r_1, r_2\}$ clips the input $y$ with value bounds $r_1$ and $r_2$; the $\lfloor y \rceil$ rounds $y$ to its nearest integer; the $\circ$ denotes the channel-wise multiplication. %, indicating that we utilize channel-wise quantizer in this paper.
%
And $Q_n^x = - 2^{a-1}, Q_p^x = 2^{a-1}-1$, $Q_n^{\bf w} = - 2^{b-1}, Q_p^{\bf w} = 2^{b-1}-1$ are the discrete bounds for $a$-bit activations and $b$-bit weights. $x$ generally denotes the activation in this paper, including the input feature map of convolution and fully-connected layers and input of multi-head attention modules. Based on this, we first give the quantized fully-connected layer as: 
\begin{equation}
\small
     \operatorname{Q-FC}(\bm{x}) = Q_a({\bm{x}}) \cdot Q_w({\bf w}) = \alpha_x\alpha_{\bf w} \circ (\bm{x}_q \odot {\bf w}_q + z / \alpha_x \circ {\bf w}_q), 
     \label{q-fc}
\end{equation}
% 
where $\cdot$ denotes the matrix multiplication and $\odot$ denotes the matrix multiplication with efficient bit-wise operations. The straight-through estimator (STE)~\cite{bengio2013estimating} is used to retain the derivation of the gradient in backward propagation. 


%
In DETR~\cite{carion2020end}, the visual features generated by the backbone are augmented with position embedding and fed into the transformer encoder.
% 
% Multi-head self-attention would be applied to generate the key, query, and value features ${\bf q}, {\bf k}, {\bf v}$ to exchange information between features at all spatial positions. The quantized multi-head dot-product attention is formulated as
% \begin{equation}\label{eq:attention}
%     \begin{aligned}
%     {\bf A}_i &= \operatorname{softmax}(Q_a({\bf q})_i \cdot Q_a({\bf k})_i^{\top}/\sqrt{d}), \\
%     {\bf E}_i &= Q_a({\bf A})_i\cdot Q_a({\bf v})_i,
%     \end{aligned}
% \end{equation}
% where ${\bf E}$ is the output of the multi-head self-attention module and $i$ denotes the head index. The matrix multiplication $\cdot$ can be expanded as the efficient bit-wise operations, similar to Eq.~\ref{q-fc}. Then ${\bf E}$ is further transformed as the input for the decoder of the Transformer. 
% 
Given an encoder output ${\bf E}$, DETR performs co-attention between object queries ${\bf O}$ and the visual features ${\bf E}$, which are formulated as:
%
\begin{equation}
    \begin{aligned}
    {\bf q} &= \operatorname{Q-FC}({\bf O}),\;\;{\bf k},{\bf v} = \operatorname{Q-FC}({\bf E})\\
    {\bf A}_i &= \operatorname{softmax}(Q_a({\bf q})_i \cdot Q_a({\bf k})_i^{\top}/\sqrt{d}), \\
    {\bf D}_i &= Q_a({\bf A})_i\cdot Q_a({\bf v})_i,
    \end{aligned}
    \label{decoder}
\end{equation}
%
where ${\bf D}$ is the multi-head co-attention module, {\em i.e.}, the co-attended feature for the object query. The $d$ denotes the feature dimension in each head.
%
%the channel dimension of each head.
More FC layers transform the decoder's output features of each object query for the final output. Given box and class predictions, the Hungarian algorithm~\cite{carion2020end} is applied between predictions and ground-truth box annotations to identify the learning targets of each object query. 

\begin{figure}[t]
\centering
\includegraphics[scale=.26]{performance.pdf}
%\caption{We analyze the module sensitivity with quantization. We report the accuracy of 3/4-bit quantized DETR-R50 on VOC about quantizing relevant parts.}
%
%\vspace{-1.0em}
\caption{Performance of 3/4-bit quantized DETR-R50 on VOC with different quantized modules.}
\label{fig:quantization}
%\vspace{-1.3em}x
\end{figure}

\subsection{Challenge Analysis}
%
Intuitively, the performance of the quantized DETR baseline largely depends on the information representation capability mainly reflected by the information in the multi-head attention module. 
%
Unfortunately, such information is severely degraded by the quantized weights and inputs in the forward pass. Also, the rounded and discrete quantization significantly affect the optimization during backpropagation. 


We conduct the quantitively ablative experiments by progressively replacing each module of the real-valued DETR baseline with a quantized one and compare the average precision (AP) drop on the VOC dataset~\cite{voc2007} as shown in Fig.\,\ref{fig:quantization}. 
We find that quantizing the MHA decoder module to low bits, {\em i.e.}, (1)+(2)+(3), brings the most significant accuracy drops of accuracy among all parts of the DETR methods, up to 2.1\% in the 3-bit DETR-R50. At the same time, other parts of DETR show comparative robustness to the quantization function. 
Consequently, the critical problem of improving the quantized DETR methods is restoring the information in MHA modules after quantization. Other qualitative results in Fig.\,\ref{fig:motivation1} and Fig.\,\ref{fig:motivation2} also indicate that the degraded information representation is the main obstacle to a better quantized DETR.

\section{The Proposed Q-DETR}
%Based on the above observation and analysis, we aim to bridge the large performance gap between quantized DETR and real-valued counterparts in this section, with the proposed distribution rectification distillation in this section.

\subsection{Information Bottleneck of Q-DETR}

% Usually, knowledge distillation involves a to-be-trained student detector and a pre-trained teacher detector, and we distinguish them with scripts $\mathcal{S}$ and $\mathcal{T}$, respectively.
To address the information distortion of the quantized DETR, we aim to improve the representation capacity of the quantized networks in a knowledge distillation framework. Generally, we utilize a real-valued DETR as a teacher and a quantized DETR as a student, which are distinguished with superscripts $\mathcal{T}$ and $\mathcal{S}$, respectively.
%



Our Q-DETR  pursues the best tradeoff between performance and compression, which is precisely the goal of the information bottleneck (IB) method through quantifying the mutual information that the intermediate layer contains about the input (less is better) and the desired output (more is better)~\cite{shwartz2017opening,tishby2000information}. 
%
In our case, the intermediate layer comes from the student, while the desired output includes the ground-truth labels as well as the queries of the teacher for distillation. 
%
Thus, the objective target of our Q-DETR is:
%
%
%
\begin{equation}
    \begin{aligned}
    \mathop{\min}_{\theta^{\mathcal{S}}} I(X; {\bf E}^{\mathcal{S}}) - \beta I( {\bf E}^{\mathcal{S}}, {\bf q}^{\mathcal{S}}; \bm{y}^{GT}) - \gamma I({\bf q}^{\mathcal{S}}; {\bf q}^{\mathcal{T}}), 
    \end{aligned}
    \label{eq:distill_IB}
\end{equation}
%
where ${\bf q}^{\mathcal{T}}$ and ${\bf q}^{\mathcal{S}}$ represent the queries in the teacher and student DETR methods as predefined in Eq.\,(\ref{decoder}); $\beta$ and $\gamma$ are the Lagrange multipliers~\cite{shwartz2017opening}; $\theta^{\mathcal{S}}$ is the parameters of the student; and $I(\cdot)$ returns the mutual information of two input variables. 
%
%
The first item $I(X; {\bf E}^{\mathcal{S}})$ minimizes information between input and visual features ${\bf E}^{\mathcal{S}}$ to extract task-oriented hints~\cite{wang2020bidet}. The second item $I( {\bf E}^{\mathcal{S}}, {\bf q}^{\mathcal{S}}; \bm{y}^{GT})$  maximizes information between extracted visual features and ground-truth labels for better object detection. These two items can be easily accomplished by common network training and detection loss constraints, such as proposal classification and coordinate regression.


The core issue of this paper is to solve the third item $I({\bf q}^{\mathcal{S}}; {\bf q}^{\mathcal{T}})$, which attempts to address the information distortion in student query via introducing teacher query as a priori knowledge. To accomplish our goal, we first expand the third item and reformulate it as:
%
%
%
\begin{equation}
\begin{aligned}
I({\bf q}^{\mathcal{S}}; {\bf q}^{\mathcal{T}}) = H({\bf q}^{\mathcal{S}}) - H({\bf q}^{\mathcal{S}}|{\bf q}^{\mathcal{T}}), 
\end{aligned}
\label{eq:mutual}
\end{equation}
%
%
where $H({\bf q}^{\mathcal{S}})$ returns the self information entropy expected to be maximized while $H({\bf q}^{\mathcal{S}}|{\bf q}^{\mathcal{T}})$ is the conditional entropy expected to be minimized. 
%
%
It is challenging to optimize the above maximum \& minimum items simultaneously. Instead, we make a compromise to reformulate Eq.\,(\ref{eq:mutual}) as a bi-level issue~\cite{liu2021investigating,colson2007overview} that alternately optimizes the two items, which is explicitly defined as:
%
%
\begin{equation}
\begin{aligned}
& \mathop{\min}_{\theta}  H({\bf q}^{\mathcal{S}^*}|{\bf q}^{\mathcal{T}}), \\
\operatorname{s.t.} \;\;\;\; &{\bf q}^{\mathcal{S}^*} = \mathop{\arg \max}_{{\bf q}^{\mathcal{S}}} H({\bf q}^{\mathcal{S}}). 
\end{aligned}
\label{eq:bi-level}
\end{equation}


Such an objective involves two sub-problems, including an inner-level optimization to derive the current optimal query ${\bf q}^{\mathcal{S}^*}$ and an upper-level optimization to conduct knowledge transfer from the teacher to the student. Below, we show that the two sub-problems can be solved in the forward \& backward network propagation's.



% Within each training iteration, maximizing self information entropy $H({\bf q}^{\mathcal{S}})$ is solved in the forward, while minimizing conditional entropy $H({\bf q}^{\mathcal{S}}|{\bf q}^{\mathcal{T}})$ is realized in the backward process.

% 
% Firstly, we regard the distillation for Q-DETR in the manner of information bottleneck (IB) principle~\cite{shwartz2017opening, wang2020bidet,wang2022efficient}, which is relevant to compression with the best hypothesis that the distribution misfit and the model representation capacity should simultaneously be minimized. 
% 
% As a result, the distorted information of quantized student network is rectified, which fully utilizes the capacity of the student model. 
% 

% 
% 

\subsection{Distribution Rectification Distillation}
\label{sec:DRD}

\textbf{Inner-level optimization}. 
%
%
We first detail the maximization of self-information entropy.
%
According to the definition of self information entropy, $H({\bf q}^{\mathcal{S}})$ can be implicitly expanded as:
%
\begin{equation}
    H({\bf q}^{\mathcal{S}}) = -\int_{{\bf q}^{\mathcal{S}}_i \in {\bf q}^{\mathcal{S}}} p({\bf q}^{\mathcal{S}}_i){\operatorname{log} p({\bf q}^{\mathcal{S}}_i)}.
\end{equation}
% 
% 
 
However, an explicit form of $H({\bf q}^{\mathcal{S}})$ can only be parameterized with a regular distribution $p({\bf q}^{\mathcal{S}}_i)$.
%
Luckily, the statistical results in Fig.\,\ref{fig:motivation1} shows that the query distribution tends to follow a Gaussian distribution, which is also observed in~\cite{li2022q}. This enables us to solve the inner-level optimization in a distribution alignment fashion. To this end, we first calculate the mean $\mu({\bf q}^{\mathcal{S}})$ and variance $\sigma({\bf q}^{\mathcal{S}})$ of query ${\bf q}^{\mathcal{S}}$ whose distribution is then modeled as ${\bf q}^{\mathcal{S}} \sim \mathcal{N}(\mu({\bf q}^{\mathcal{S}}), \sigma({\bf q}^{\mathcal{S}}))$. Then, the self-information entropy of the student query can be proceeded as:%
%
%
%
\begin{equation}
    \begin{aligned}
    H({\bf q}^{\mathcal{S}}) &= -\mathbb{E}[\operatorname{log} \mathcal{N}(\mu({\bf q}^{\mathcal{S}}), \sigma({\bf q}^{\mathcal{S}}))]\\
    &= -\mathbb{E}[\operatorname{log} [{(2\pi{\sigma({{\bf q}^{\mathcal{S}}})}^2)}^{\frac{1}{2}}\operatorname{exp}(-\frac{{({\bf q}^{\mathcal{S}}_i - \mu({\bf q}^{\mathcal{S}}))}^2}{2 {\sigma({{\bf q}^{\mathcal{S}}})^2}})]] \\
    &= \frac{1}{2}\operatorname{log}2\pi{\sigma({{\bf q}^{\mathcal{S}}})}^2.
    \end{aligned}
\end{equation}

%
The above objective reaches its maximum of $H({\bf q}^{\mathcal{S}^*}) = (1/2) \log 2 \pi e [\sigma({{\bf q}^{\mathcal{S}})}^2 + \epsilon_{{\bf q}^{\mathcal{S}}}]$ when ${\bf q}^{\mathcal{S}^*} = [{\bf q}^{\mathcal{S}} - \mu({\bf q}^{\mathcal{S}})] / [{\sqrt{\sigma{({\bf q}^{\mathcal{S}})}^2 + \epsilon_{{\bf q}^{\mathcal{S}}}}}]$ where $\epsilon_{{\bf q}^{\mathcal{S}}} = 1e^{-5}$ is a small constant added to prevent a zero denominator.
%
%
%
In practice, the mean and variance might be inaccurate due to query data bias. To solve this we use the concepts in batch normalization (BN)~\cite{santurkar2018does,ioffe2015batch} where a learnable shifting parameter $\beta_{{{\bf q}}^{\mathcal{S}}}$ is added to move the mean value. A learnable scaling parameter $\gamma_{{\bf q}^{\mathcal{S}}}$ is multiplied to move the query to the adaptive position. In this situation, we rectify the information entropy of the query in the student as follows:
%
    \begin{equation}
    % \small
        \begin{aligned}
        {\bf q}^{\mathcal{S}^*} &= \frac{{\bf q}^{\mathcal{S}} - \mu({\bf q}^{\mathcal{S}})} {{ \sqrt{\sigma{({\bf q}^{\mathcal{S}})}^2 + \epsilon_{{\bf q}^{\mathcal{S}}}}}}\gamma_{{\bf q}^{\mathcal{S}}} + \beta_{{\bf q}^{\mathcal{S}}},
        \end{aligned}
    \label{eq:rectification}
    \end{equation}
    %
    in which case the maximum self-information entropy of student query becomes $H({\bf q}^{\mathcal{S}^*}) = (1/2)\log 2 \pi e [(\sigma^2_{{\bf q}^{\mathcal{S}}} + \epsilon_{{\bf q}^{\mathcal{S}}})/\gamma^2_{{\bf q}^{\mathcal{S}}}]$. 
 %   Note that, alike to BN, $\gamma_{{\bf q}^{\mathcal{S}}}$ and $\beta_{{\bf q}^{\mathcal{S}}}$ are two learnable parameters.
    %
    Therefore, in the forward propagation, we can obtain the current optimal query ${\bf q}^{\mathcal{S}^*}$ via Eq.\,(\ref{eq:rectification}), after which, the upper-level optimization is further executed as detailed in the following contents.
    
    % Based on the derivation above, the diltillation supervision, {\em i.e.}, third item of Eq.~\ref{eq:distill_IB}, for Q-DETR can be temporally represented as 
    % \begin{equation}
    %     \begin{aligned}
    %      \mathop{\min}_{\theta} H({\bf q}^{\mathcal{S}}|{\bf q}^{\mathcal{T}}).
    %     \end{aligned}
    %     \label{eq:L_select}
    % \end{equation}
    


\textbf{Upper-level optimization}.
%
%
We continue minimizing the conditional information entropy between the student and the teacher. %Before a comprehensive description, we introduce some basic notations.
%
%
Following DETR~\cite{carion2020end}, we denote the ground-truth labels by $\bm{y}^{GT}=\{c^{GT}_i, b^{GT}_i\}_{i=1}^{N_{gt}}$ as a set of ground-truth objects where $N_{gt}$ is the number of foregrounds, $c_i^{GT}$ and $b_i^{GT}$ respectively represent the class and coordinate (bounding box) for the $i$-th object.
%
In DETR, each query is associated with an object. Therefore, we can obtain $N$ objects for teacher and student as well, denoted as $\bm{y}^{\mathcal{S}} = \{c^{\mathcal{S}}_j, b^{\mathcal{S}}_j\}_{j=1}^N$ and $\bm{y}^{\mathcal{T}} = \{c^{\mathcal{T}}_j, b^{\mathcal{T}}_j\}_{j=1}^N$.


%
%
The minimization of the conditional information entropy requires the student and teacher objects to be in a one-to-one matching. However, it is problematic for DETR due primarily to the sparsity of prediction results and the instability of the query’s predictions~\cite{li2022dn}. We propose a foreground-aware query matching to rectify ``well-matched'' queries to solve this.  Concretely, we match the ground-truth bounding boxes with this student to find the maximum coincidence as:
%
%首先,因为detr中的query没有位置偏置,所以需要先把student query与GT进行配对,来根据giou找到最匹配某个GT 的student query,并且保留这个最大的giou.
\begin{equation}
    \begin{aligned}
       G_i = \mathop{\max}_{1\leq j \leq N} \operatorname{GIoU}(b^{GT}_{i}, b^{\mathcal{S}}_{j}),
    \end{aligned}
    \label{eq:sigma}
\end{equation}
%
where $\operatorname{GIoU}(\cdot)$ is the generalized intersection over union function~\cite{rezatofighi2019generalized}. Each $G_i$ reflects the ``closeness'' of student proposals to the $i$-th ground-truth object.
%
Then, we retain highly qualified student proposals around at least one ground truth to benefit object recognition~\cite{wang2019distilling} as:
\begin{align}
\begin{split}
\small
b_j^{\mathcal{S}} = \left \{
\begin{array}{ll}
    {b}_j^{\mathcal{S}},         \!\!\!\! & {\operatorname{GIoU}}(b^{GT}_{i}, b^{\mathcal{S}}_{j}) > \tau G_i, \,\,\forall\; i  \\
    \varnothing,                                 & \text{otherwise},
\end{array}
\right.
\end{split}
\end{align}
%
where $\tau$ is a threshold controlling the proportion of distilled queries. After removing object-empty ($\varnothing$) queries in $\tilde{\bm q}^{\mathcal{S}}$, we form a distillation-desired query set of students denoted as $\tilde{{\bm q}}^{\mathcal{S}}$ associated with its object set $\tilde{{\bm y}}^{\mathcal{S}} =  \{\tilde{c}^{\mathcal{S}}_j, \tilde{b}^{\mathcal{S}}_j\}_{j=1}^{\tilde{N}}$. Correspondingly, we can obtain a teacher query set $\tilde{{\bm y}}^{\mathcal{T}} =  \{\tilde{c}^{\mathcal{T}}_j, \tilde{b}^{\mathcal{T}}_j\}_{j=1}^{\tilde{N}}$. For the $j$-th student query, its corresponding teacher query is matched as:
%
\begin{equation}
\tilde{c}^{\mathcal{T}}_j, \tilde{b}^{\mathcal{T}}_j = \mathop{\arg \max}_{\tilde{c}^{\mathcal{T}}_k, \tilde{b}^{\mathcal{T}}_k}\sum^{N}_{k=1} \mu_1 \operatorname{GIoU}(\tilde{b}^{\mathcal{S}}_{j}, b^{\mathcal{T}}_{k})-\mu_2\|\tilde{b}^{\mathcal{S}}_{j} - b^{\mathcal{T}}_{k}\|_1, 
\end{equation}
%
where $\mu_1=2$ and $\mu_2=5$ control the matching function, values of which is to follow~\cite{carion2020end}.

% 找到每个gt对应的最大giou后,我们生成了一个需要被蒸馏的student query集合。(因为我们认为在GT周围的student query是对于预测bbox最重要的。)
%
%As mentioned above, the orderly corresponding teacher’s results to the student’s predictions is necessarily needed for distilling DETR-based detectors~\cite{carion2020end}. Inspired by FGFI~\cite{wang2019distilling}, we find that the queries centering around the ground-truth bounding boxes play the essential role in object localization and classification . %As shown in Fig.\,\ref{fig:motivation2}, the degradation of quantized cross-attention representation is a significant performance bottleneck of the quantized DETR. Specifically, the degraded attention representation lead to implicit localization capacity, which deteriorate the performance of quantized DETR methods. %To address this issue, we introduce knowledge distillation (KD) to rectify such degradation. 
%
%Further, we match student queries with these in teacher as:
%
%\begin{equation}
%\hat{\sigma} = \mathop{\arg \max}_{{\sigma}}\sum^{N}_{j=1} \mu_1 \operatorname{GIoU}(\hat{b}^{\mathcal{S}}_{j}, b^{\mathcal{T}}_{\sigma(j)})-\mu_2\|\hat{b}^{\mathcal{S}}_{j}- b^{\mathcal{T}}_{\sigma(j)}\|_1, 
%\end{equation}
%
%where $\mu_1=2$ and $\mu_2=5$ controls the matching function, values of which is to follow~\cite{carion2020end}.
%其中mu_1跟mu_2是超参数,我们follow DETR去设定2 跟 5两个值。
%
%
%由于DETR中,对于teacher跟student网络,N这一维度不是一一对应的,即,直接去对q^t跟q^s做监督是无意义的。我们先需要在N这一维度上提出一种匹配算法来对teacher跟student进行匹配,才能进行知识蒸馏。受到图2启发,student里面最需要收到监督的query对应的localization预测(b^s)应该在空间上围绕在GT的bounding box周围。
%
%
%As mentioned above, the orderly corresponding teacher’s results to the student’s predictions is necessarily needed for distilling DETR-based detectors~\cite{carion2020end}. Inspired by FGFI~\cite{wang2019distilling}, we find that the queries centering around the ground-truth bounding boxes play the essential role in object localization and classification . As shown in Fig.\,\ref{fig:motivation2}, the degradation of quantized cross-attention representation is a significant performance bottleneck of the quantized DETR. Specifically, the degraded attention representation lead to implicit localization capacity, which deteriorate the performance of quantized DETR methods. To address this issue, we introduce knowledge distillation (KD) to rectify such degradation. 
%
% 受到FGFI启发,我们发现student中在GT周围的query对于目标检测性能更加重要。如图Fig.\,\ref{fig:motivation2}所示,quantized cross-attention representation的退化是 quantized DETR性能的主要瓶颈。具体地,退化的attention representation导致物体定位能力不足,影响了quantized DETR methods的性能。为了解决这个问题,我们引入KD来矫正这个退化。
%
%Firstly, following the DETR, we denote by $\bm{y}^{GT}=\{c^{GT}_i, b^{GT}_i\}_{i=1}^{N^{GT}}$ as the ground-truth set of objects, where $N^{GT}$ is the number of foregrounds.
% 
%$c^{GT}_i$ is the target class label and $b^{GT}_i \in [0, 1]^4$ is a vector containing the target box center coordinates and its height and width relative to the image size. 
% 
%For the teacher and student DETR model, we define $\bm{y}^{\mathcal{S}} = \{c^{\mathcal{S}}_j, b^{\mathcal{S}}_j\}_{j=1}^N$ and $\bm{y}^{\mathcal{T}} = \{c^{\mathcal{T}}_j, b^{\mathcal{T}}_j\}_{j=1}^N$, respectively. 
% 
%We match the ground-truth bounding boxes and student query priors to find the maximum coincide as
%首先,因为detr中的query没有位置偏置,所以需要先把student query与GT进行配对,来根据giou找到最匹配某个GT 的student query,并且保留这个最大的giou.
%\begin{equation}
%    \begin{aligned}
%       G_i = \mathop{\max}_{1\leq j \leq N} \operatorname{GIoU}(b^{GT}_{i}, b^{\mathcal{S}}_{j}),
%    \end{aligned}
%    \label{eq:sigma}
%\end{equation}
%where $\operatorname{GIoU}(\cdot)$~\cite{rezatofighi2019generalized} is the generalized intersection over union function. 
% Here, $i\in [1, \cdots, N^{GT}]$ denotes the ground truth foreground index and $N^{GT}$ is the number of ground truth foregrounds.
%After selecting the queries with maximum coincidence, we form a distillation-desired query set of student (quantized) DETR as: 
%\begin{align}
%\begin{split}
%\small
%\hat{b}_j^{\mathcal{S}} = \left \{
%\begin{array}{ll}
%    {b}_j^{\mathcal{S}},         \!\!\!\! & {\operatorname{GIoU}}(b^{GT}_{i}, b^{\mathcal{S}}_{j}) > \tau G_i, \;\forall\; i \in [1, \cdots, N^{GT}] \\
%    \varnothing,                                 & otherwise
%\end{array}
%\right.
%\end{split}
%\end{align}
%where $j\in [1, \cdots, N]$ and $\tau$ is the threshold hyper-parameter controlling the proportion of distilled queries. 
% 找到每个gt对应的最大giou后,我们生成了一个需要被蒸馏的student query集合。(因为我们认为在GT周围的student query是对于预测bbox最重要的。)
%
%We then match these student queries with the teacher queries as
%\begin{equation}
%\hat{\sigma} = \mathop{\arg \max}_{{\sigma}}\sum^{N}_{j=1} \mu_1 \operatorname{GIoU}(\hat{b}^{\mathcal{S}}_{j}, b^{\mathcal{T}}_{\sigma(j)})-\mu_2\|\hat{b}^{\mathcal{S}}_{j}- b^{\mathcal{T}}_{\sigma(j)}\|_1, \\
%\end{equation}
%where $\mu_1$ and $\mu_2$ are hyper-parameter controlling the matching function. Following~\cite{carion2020end}, we set $\mu_1=2$ and $\mu_2=5$, respectively. 
%其中mu_1跟mu_2是超参数,我们follow DETR去设定2 跟 5两个值。
%
%
Finally, the upper-level optimization after rectification in Eq.\,(\ref{eq:bi-level}) becomes: 
%在进行完匹配之后,我们将upper-level的优化问题改写为
\begin{equation}
\begin{aligned}
 \mathop{\min}_{\theta} H(\tilde{{\bf q}}^{\mathcal{S}^*}|\tilde{{\bf q}}^{\mathcal{T}}). 
\end{aligned}
\label{eq:L_select2}
\end{equation}
% 


Optimizing Eq.\,(\ref{eq:L_select2}) is challenging. Alternatively, we minimize the norm distance between $\tilde{\bf q}^{\mathcal{S}^*}$ and $\tilde{{\bf q}}^{\mathcal{T}}$, optima of which, \emph{i.e.}, $\tilde{\bf q}^{\mathcal{S}^*} = \tilde{\bf q}^{\mathcal{T}}$, is exactly the same with that in Eq.\,(\ref{eq:L_select2}).
%
%
Thus, the final loss for our distribution rectification distillation loss becomes: 
\begin{equation}
\mathcal{L}_{DRD}(\tilde{{\bf q}}^{\mathcal{S}^*}, \tilde{{\bf q}}^{\mathcal{T}}) = \mathbb{E}[\|\tilde{\bf D}^{\mathcal{S}^*} - \tilde{\bf D}^{\mathcal{T}}\|_2],
\label{drd}
\end{equation}
%
where we use the Euclidean distance of co-attented feature $\tilde{\bf D}$ (see Eq. \ref{decoder}) containing the information query $\tilde{\bf q}$ for optimization.

In backward propagation, the gradient updating drives the student queries toward their teacher hints. Therefore we accomplish our distillation.
%
The overall training losses for our Q-DETR model are:
%
\begin{equation}
\small
\mathcal{L} = \mathcal{L}_{GT}(\bm{y}^{GT}, \bm{y}^{\mathcal{S}}) + \lambda \mathcal{L}_{DRD}(\tilde{{\bf q}}^{\mathcal{S}^*}, \tilde{{\bf q}}^{\mathcal{T}}),
\label{final}
\end{equation}
%
%
where $L_{GT}$ is the common detection loss for missions such as proposal classification and coordinate regression~\cite{carion2020end}, and $\lambda$ is a tradeoff hyper-parameter.





\section{Experiments}
In this section, we evaluate the performance of the proposed Q-DETR mode using popular DETR~\cite{carion2020end} and SMCA-DETR~\cite{gao2021fast} models. To the best of our knowledge, there is no publicly available source code  on quantization-aware training of DETR methods at this point, so we implement the baseline and LSQ~\cite{esser2019learned} methods ourselves.

\subsection{Datasets and Implementation Details}
\label{sec:setup}
\textbf{Datasets}.
We first conduct the ablative study and hyper-parameter selection on the PASCAL VOC dataset \cite{voc2007}, which contains natural images from 20 different classes. We use the VOC {\tt trainval2012}, and VOC {\tt trainval2007} sets to train our model, which contains approximately 16k images, and the VOC {\tt test2007} set to evaluate our Q-DETR, which contains 4952 images. We report COCO-style metrics for the VOC dataset: AP, AP$_{50}$ (default VOC metric), and AP$_{75}$.
We further conduct the experiments on the COCO {\tt 2017} \cite{coco2014} object detection tracking. Specifically, we train the models on COCO {\tt train2017} and evaluate the models on COCO {\tt val2017}.
We list the average precision (AP) for IoUs$\in [0.5:0.05:0.95]$, designated as AP, using COCO's standard evaluation metric. For further analyzing our method, we also list AP$_{50}$, AP$_{75}$, AP$_s$, AP$_m$, and AP$_l$.

\begin{figure}
        \centering
        \begin{subfigure}{0.495\linewidth}
    	\centering
    	\includegraphics[width=1.0\linewidth]{tau_lambda.pdf}
    	\caption{Effect of $\tau$ and $\lambda$.}
    	\label{motivation:teacher1}
        \end{subfigure}
        \begin{subfigure}{0.495\linewidth}
    	\centering
    	\includegraphics[width=1.0\linewidth]{information.pdf}
 		\caption{Mutual information curves.}
    	\label{motivation:teacher_false1}
        \end{subfigure}
        \caption{(a) We select $\tau$ and $\lambda$ using 4-bit Q-DETR-R50 on VOC. (b) The mutual information curves of $I(X; {\bf E})$ and $I(\bm{y}^{GT}; {\bf E}, {\bf q})$ (Eq.\;\ref{eq:distill_IB}) on the information plane. The red curves represent the teacher model (DETR-R101). The orange, green, red, and purple lines represent the 4-bit baseline, 4-bit baseline + DA, 4-bit baseline + FQM, and 4-bit baseline + DA + FQM (4-bit Q-DETR).}
        \label{hyper-parameter}
    \end{figure}

\textbf{Implementation Details}. 
Our Q-DETR is trained with the DETR~\cite{carion2020end} and SMCA-DETR~\cite{gao2021fast} framework. We select the ResNet-50 \cite{he2016deep} and modify it with Pre-Activation structures and RPReLU \cite{liu2020reactnet} function following \cite{liu2022nonuniform}. PyTorch \cite{paszke2017automatic} is used for implementing Q-DETR. We run the experiments on 8 NVIDIA Tesla A100 GPUs with $80$ GB memory. We use ImageNet ILSVRC12 \cite{imagenet12} to pre-train the backbone of a quantized student. The training protocol is the same as the employed frameworks \cite{carion2020end, gao2021fast}. 
Specifically, we use a batch size of 16. AdamW \cite{loshchilov2017decoupled} is used to optimize the Q-DETR, with the initial learning rate of $1e^{-4}$.
We train for 300/500 epochs for the Q-DETR on VOC/COCO dataset, and the learning rate is multiplied by 0.1 at the 200/400-th epoch, respectively. Following the SMCA-DETR, we train the Q-SMCA-DETR for 50 epochs, and the learning rate is multiplied by 0.1 at the 40-th epoch on both the VOC and COCO datasets. We utilize a multi-distillation strategy, where we save the encoder and decoder network as real-valued at the first stage. Then we train the fully quantized DETR at the second stage, where we load the weight from the checkpoint of first stage. 
%
We select real-valued DETR-R101 (84.5\% AP$_{50}$ on VOC and 43.5\% AP on COCO) and SMCA-DETR-R101 (85.3\% AP$_{50}$ on VOC and 44.4\% AP on COCO) as teacher network. 


\subsection{Ablation Study}
% For ablation experiments, we train the Q-DETR-R50 model on VOC object detection following the setup in Sec.~\ref{sec:setup} with 300 epochs. 

\textbf{Hyper-parameter selection}.  
As mentioned above, we select hyper-parameters $\tau$ and $\lambda$ in this part using the 4-bit Q-DETR model. We show the model performance (AP$_{50}$) with different setups of hyper-parameters $\{\tau,\lambda\}$ in Fig.\,\ref{hyper-parameter} (a), where we conduct ablative experiments on the baseline + DA (AP$_{50}$=78.8\%).
As can be seen, the performances increase first and then decrease with the increase of $\tau$ from left to right. Since $\tau$ controls the proportion of selected distillation-desired queries, we show that the full-imitation ($\tau=0$) performs worse than the vanilla baseline with no distillation ($\tau=1$), showing query selection is necessary. The figure also shows that the performances increase first and then decrease with the increase of $\tau$ from left to right. The Q-DETR obtains better performances with $\tau$ set as 0.5 and 0.6. 
With the varying value of $\lambda$, we find $\{\lambda,\tau\}$ = \{2.5, 0.6\} boost the performance of Q-DETR most, achieving 82.7\% AP on VOC {\tt test2007}. Based on the ablative study above, we set hyper-parameters $\tau$ and $\lambda$ as 0.6 and 2.5 for the experiments in this paper.


\textbf{Effectiveness of components}. We show quantitative improvements of components in Q-DETR in Tab.\,\ref{tab:ablation}.
As shown in Tab.\,\ref{tab:ablation}, the quantized DETR baseline suffers a severe performance drop on AP$_{50}$ (13.6\%, 6.5\%, and 5.3\% with 2/3/4-bit, respectively). DA and FQM improve the performance when used alone, and the two techniques further boost the performance considerably when combined. For example, the DA improves the 2-bit baseline by 1.9\%, and the FQM achieves a 5.2\% performance improvement. While combining the DA and FQM, the performance improvement achieves 6.7\%. 

\textbf{Information analysis}. We further show the information plane following \cite{wang2021revisiting} in Fig.\;\ref{hyper-parameter}(b). We adopt the test AP$_{50}$ to quantify $I(\bm{y}^{GT}; {\bf E}, {\bf q})$. We employ a reconstruction decoder to decode the encoded feature ${\bf E}$ to reconstruct the input and quantify $I(X;{\bf E})$ using the $\ell_1$ loss. As shown in  Fig.\;\ref{hyper-parameter}(b), the curve of the larger teacher DETR-R101 is usually on the right of the curve of small student models, which indicates a greater ability of information representation. Likewise, the purple line (Q-DETR-R50) is usually on the right of the three left curves, showing the information representation improvements with the proposed methods. 


% \noindent\textbf{Function of distillation loss}. In Eq. \ref{}




\begin{table}[]
\centering
\caption{Evaluating the components of Q-DETR-R50 on the VOC dataset. \#Bits (W-A-Attention) denotes the bit-width of weights, activations, and attention activations. DA denotes the distribution alignment module. FQM denotes foreground-aware query matching.}
\setlength{\tabcolsep}{1.1mm}{
\begin{tabular}{ccccccc}
\hline
Method                                                              & \#Bits   & AP$_{50}$     & \#Bits & AP$_{50}$     & \#Bits & AP$_{50}$     \\ \hline
Real-valued                                                         & 32-32-32 & 83.3          & -      & -             & -      & -             \\ \hline
Baseline                                                            & 4-4-8    & 78.0          & 3-3-8  & 76.8          & 2-2-8  & 69.7          \\
+DA                                                                 & 4-4-8    & 78.8          & 3-3-8  & 78.0          & 2-2-8  & 71.6          \\
+FQM                                                                & 4-4-8    & 81.5          & 3-3-8  & 80.9          & 2-2-8  & 74.9          \\
\textbf{\begin{tabular}[c]{@{}c@{}}+DA+FQM\\ (Q-DETR)\end{tabular}} & 4-4-8    & \textbf{82.7} & 3-3-8  & \textbf{82.1} & 2-2-8  & \textbf{76.4} \\ \hline
\end{tabular}}
\label{tab:ablation}
\end{table}


\begin{table}[]
\centering
\caption{We report AP, AP$_{50}$, and AP$_{75}$ ($\%$) with state-of-the-art quantization methods on DETR and SMCA-DETR using VOC {\tt test2007}. \#Bits (W-A-Attention) denotes the bit-width of weights, activations, and attention activations.}
\setlength{\tabcolsep}{1.0mm}{
\begin{tabular}{cccccc}
\hline
Model                                                                      & Method          & \#Bits                 & AP            & AP$_{50}$     & AP$_{75}$     \\ \hline
\multirow{12}{*}{DETR-R50}                                                 & Real-valued     & 32-32-32               & 59.5          & 83.3          & 64.7          \\ \cline{2-6} 
& Percentile      & \multirow{2}{*}{8-8-8} & 54.7          & 79.2          & 60.1          \\
& VT-PTQ          &                        & 57.6          & 82.3          & 63.1          \\ \cline{2-6} 
& LSQ             & \multirow{3}{*}{4-4-8} & 49.7          & 76.9          & 53.0          \\
& Baseline        &                        & 51.3          & 78.0          & 54.1          \\
& \textbf{Q-DETR} &                        & \textbf{57.1} & \textbf{82.7} & \textbf{61.5} \\ \cline{2-6} 
& LSQ             & \multirow{3}{*}{3-3-8} & 47.0          & 75.3          & 49.1          \\
& Baseline        &                        & 49.2          & 76.8          & 51.8          \\
& \textbf{Q-DETR} &                        & \textbf{56.8} & \textbf{82.1} & \textbf{61.2} \\ \cline{2-6} 
& LSQ             & \multirow{3}{*}{2-2-8} & 42.6          & 68.2          & 44.8          \\
& Baseline        &                        & 44.0          & 69.7          & 45.8          \\
& \textbf{Q-DETR} &                        & \textbf{50.7} & \textbf{76.4} & \textbf{54.1} \\ \hline
\multirow{12}{*}{\begin{tabular}[c]{@{}c@{}}SMCA-DETR\\ -R50\end{tabular}} & Real-valued     & 32-32-32               & 56.7          & 83.7          & 62.0          \\ \cline{2-6} 
& Percentile      & \multirow{2}{*}{8-8-8} & 54.7          & 79.2          & 60.1          \\
& VT-PTQ          &                        & 55.9          & 83.0          & 61.3          \\ \cline{2-6} 
& LSQ             & \multirow{3}{*}{4-4-8} & 49.6          & 78.6          & 53.4          \\
& Baseline        &                        & 50.7          & 79.5          & 55.4          \\
& \textbf{Q-DETR}          &                        & \textbf{56.2} & \textbf{83.3} & \textbf{61.6} \\ \cline{2-6} 
& LSQ             & \multirow{3}{*}{3-3-8} &      47.7         &    76.5           &       51.7        \\
& Baseline        &                        &     49.9          &      77.5         &      53.6         \\
& \textbf{Q-DETR}          &                        &       \textbf{54.3}       & \textbf{82.6}     & \textbf{59.5}     \\ \cline{2-6} 
& LSQ             & \multirow{3}{*}{2-2-8} &        42.3       &      69.7         &      44.8         \\
& Baseline        &                        &     43.9          &    70.4           &      46.1         \\
& \textbf{Q-DETR}          &                        &        \textbf{50.2}       &        \textbf{76.7}       &        \textbf{52.6}       \\ \hline
\end{tabular}}
\label{voc}
\end{table}

\subsection{Results on PASCAL VOC}

We first compare our method with the 2/3/4-bit baseline and  LSQ~\cite{esser2019learned} based on the same frameworks for object detection task with the VOC dataset. We also report the detection performance of the 8-bit post-training quantization networks, such as percentile \cite{lin2021fq}, VT-PTQ~\cite{liu2021post}. We use the input resolution following \cite{carion2020end}, {\em i.e.} 1333$\times$800. We mainly discuss the AP$_{50}$ (default VOC metric) in the following. 

We evaluate the proposed Q-DETR on DETR-R50 models in Tab.\,\ref{voc}. For the DETR-R50 model, compared with the 8-bit PTQ method, our 4-bit Q-DETR achieves a much larger compression ratio than 8-bit VT-PTQ, but with a bit of performance improvement (82.7\% {\em vs.} 82.3\%). Also, the proposed method boosts the performance of 2/3/4-bit baseline by 6.7\%, 5.3\%, and 4.7\% with the same architecture and bit-width, which significantly validates the effectiveness of our method. 






\begin{table*}[!t]
\centering
\caption{Comparison with state-of-the-art quantization methods using DETR and SMCA-DETR on COCO {\tt val2017}. \#Bits (W-A-Attention) denotes bit-width of weights, activations, and attention activations.}
\begin{tabular}{ccccccccccc}
\hline
Model                           & Method          & \#Bits                 & Size$_{\rm (MB)}$        & OPs$_{\rm (G)}$          & AP & AP$_{50}$     & AP$_{75}$     & AP$_{s}$      & AP$_{m}$      & AP$_{l}$      \\ \hline
\multirow{12}{*}{DETR-R50}      & Real-valued     & 32-32-32               & 159.32                 & 85.51                  & 42.0                                                       & 62.4          & 44.2          & 20.5          & 45.8          & 61.1          \\ \cline{2-11} 
& Percentile      & \multirow{2}{*}{8-8-8} & \multirow{2}{*}{39.83} & \multirow{2}{*}{23.01} & 38.6                                                       & -             & -             & -             & -             & -             \\
& VT-PTQ          &                        &                        &                        & 41.2                                                       & -             & -             & -             & -             & -             \\ \cline{2-11} 
& LSQ             & \multirow{3}{*}{4-4-8} & \multirow{3}{*}{19.92} & \multirow{3}{*}{13.02} & 33.3                                                       & 53.7          & 33.9          & 12.8          & 37.0          & 51.6          \\
& Baseline        &                        &                        &                        & 34.1                                                       & 55.3          & 35.4          & 14.3          & 38.0          & 53.8          \\
& \textbf{Q-DETR} &                        &                        &                        & \textbf{39.4}                                              & \textbf{60.2} & \textbf{41.4} & \textbf{17.7} & \textbf{43.4} & \textbf{59.9} \\ \cline{2-11} 
& LSQ             & \multirow{3}{*}{3-3-8} & \multirow{3}{*}{15.03}  & \multirow{3}{*}{7.61}  & 31.0                                                       & 52.3          & 32.1          & 11.3          & 33.9          & 48.5          \\
& Baseline        &                        &                        &                        & 32.3                                                       & 52.2          & 32.9          & 12.3          & 35.4          & 50.3          \\
& \textbf{Q-DETR} &                        &                        &                        & \textbf{36.1}                                              & \textbf{55.9} & \textbf{37.5} & \textbf{14.6} & \textbf{39.4} & \textbf{55.2} \\ \cline{2-11} 
& LSQ             & \multirow{3}{*}{2-2-8} & \multirow{3}{*}{10.03}  & \multirow{3}{*}{5.32}  & 24.7                                                       & 44.6          & 26.5          & 6.3           & 25.3          & 42.7          \\
& Baseline        &                        &                        &                        & 26.6                                                       & 46.6          & 26.5          & 8.4           & 28.2          & 44.4          \\
& \textbf{Q-DETR} &                        &                        &                        & \textbf{31.4}                                              & \textbf{51.3} & \textbf{31.6} & \textbf{11.6} & \textbf{34.3} & \textbf{49.6} \\ \hline
\multirow{12}{*}{SMCA-DETR-R50} & Real-valued     & 32-32-32               & 164.75                 & 86.65                  & 41.0                                                       & 62.2          & 43.6          & 21.9          & 44.3          & 59.1          \\ \cline{2-11} 
& Percentile      & \multirow{2}{*}{8-8-8} & \multirow{2}{*}{41.19} & \multirow{2}{*}{23.66} & 37.5                                                       & 58.5          & 40.1          & 17.6          & 39.1          & 55.9          \\
& VT-PTQ          &                        &                        &                        & 40.2                                                       & 61.0          & 42.6          & 20.3          & 42.9          & 57.7          \\ \cline{2-11} 
& LSQ             & \multirow{3}{*}{4-4-8} & \multirow{3}{*}{20.59} & \multirow{3}{*}{13.48} & 33.9                                                       & 55.0          & 35.0          & 13.2          & 37.2          & 51.4          \\
& Baseline        &                        &                        &                        & 35.0                                                       & 56.4          & 36.4          & 15.6          & 38.3          & 52.5          \\
& \textbf{Q-DETR} &                        &                        &                        & \textbf{38.3}                                              & \textbf{59.7} & \textbf{39.8} & \textbf{17.7} & \textbf{41.7} & \textbf{56.8} \\ \cline{2-11} 
& LSQ             & \multirow{3}{*}{3-3-8} & \multirow{3}{*}{15.68} & \multirow{3}{*}{8.05}  & 30.1                                                       & 52.6          & 31.4          & 11.9          & 33.4          & 46.6          \\
& Baseline        &                        &                        &                        & 31.8                                                       & 53.7          & 32.6          & 12.6          & 35.2          & 49.8          \\
& \textbf{Q-DETR} &                        &                        &                        & \textbf{35.0}                                              & \textbf{56.3}          & \textbf{36.9}          & \textbf{15.0} & \textbf{39.0} & \textbf{53.1} \\ \cline{2-11} 
& LSQ             & \multirow{3}{*}{2-2-8} & \multirow{3}{*}{10.84}  & \multirow{3}{*}{4.54}  & 23.9                                                       & 42.2          & 24.2          & 9.4           & 26.2          & 37.5          \\
& Baseline        &                        &                        &                        & 25.4                                                       & 44.3          & 25.2          & 8.4           & 27.2          & 40.3          \\
& \textbf{Q-DETR} &                        &                        &                        & \textbf{30.5}                                              & \textbf{51.8} & \textbf{31.8} & \textbf{12.0} & \textbf{33.2} & \textbf{48.0} \\ \hline
\end{tabular}
\label{COCO}
\end{table*}



Besides, our method generates convincing results on SMCA-DETR. As shown in Tab.\,\ref{voc}, the performance of the proposed Q-DETR with SMCA-DETR-R50 outperforms the 2/3/4-bit Baseline method by 6.3\% , 5.1\% and 3.8\% on AP$_{50}$, a large margin. Compared with 8-bit post-training quantization methods, our method achieves a significantly higher compression rate and comparable performance. 

\subsection{Results on COCO}

% Because of its diversity and scale, the COCO dataset presents a more significant challenge in the object detection task compared with PASCAL VOC. 
We further show comparison on the large-scale COCO \cite{coco2014} dataset. We compare our method with the 2/3/4-bit baseline and  LSQ~\cite{esser2019learned} based on the same frameworks. We also report the detection performance of the 8-bit post-training quantization networks, such as percentile \cite{lin2021fq} , VT-PTQ~\cite{liu2021post}. The AP with different IoU thresholds, and AP of objects with varying scales are all reported in Tab.\,\ref{COCO}. 

Tab.\,\ref{COCO} lists the comparison of several quantization approaches and detection frameworks in computing complexity, storage cost. Our Q-DETR significantly accelerates computation and reduces storage requirements for various detectors. 
We follow~\cite{wang2020bidet} to calculate memory usage, by adding 32$\times$ the number of real-valued weights and $a\times$ the number of quantized weights in the $a$-bit networks. 
The number of operations (OPs) is calculated in the same way as~\cite{wang2020bidet}. Current CPUs can handle both bit-wise XNOR and bit-count operations in parallel. The respective number of FLOPs adds $\{\frac{1}{32},\frac{1}{16},\frac{1}{8}\}$ of the number of $\{$2,3,4$\}$-bit multiplications equals the OPs following \cite{liu2020bi}. 

We summarize the experimental results on COCO {\tt val2017} of Q-DETR-R50 from lines 2 to 17 in Tab.\,\ref{COCO}. For the DETR-R50 model, compared with the 8-bit PTQ method, our 4-bit Q-DETR achieves a much larger acceleration than the 8-bit VT-PTQ but with an acceptable performance gap. Also, the proposed method boosts the performance of 2/3/4-bit baseline by 4.8\%, 3.8\% and 5.1\% AP with the same architecture and bit-width, which is significant on the large-scale COCO dataset. Compared with the real-valued counterparts, the proposed 2/3/4-bit Q-DETR achieves computation acceleration and storage savings by 16.07$\times$/11.23$\times$/6.57$\times$ and 15.88$\times$/10.60$\times$/7.99$\times$. The above results are of great significance in the real-time inference of object detection. All of the improvements have impacts on object detection.


For the SMCA-DETR-R50 model, we observe similar performance improvements and compression ratios. For example, the 4-bit Q-SMCA-DETR-R50 theoretically accelerates 6.42$\times$ with only a 2.7\% performance gap compared with the real-valued counterpart, which is significant for real-time DETR methods.

\section{Conclusion}
This paper introduces a novel method for training quantized DETR (Q-DETR) with knowledge distillation to rectify the query distribution. Q-DETR generalizes the information bottleneck (IB) principle and leads a bi-level distribution rectification distillation. We effectively employ a distribution alignment module to solve inner-level and a foreground-aware query matching scheme to solve upper level. As a result, Q-DETR significantly boosts performance of low-bit DETR. Extensive experiments show that Q-DETR surpasses state-of-the-arts in DETR quantization.

\section{Acknowledgements}
This work was supported by National Natural Science Foundation of China under Grant 62141604, 62076016, Beijing Natural Science Foundation L223024. 


% \usepackage{multir

% we propose an efficient information restoration transformation (IRT) method based on information theory, which statistically maximizes the entropy of representation and revives the attention mechanism in the Q-DETR. 
% The statistical results show that the distribution of queries in DETR-based architectures intending to follow Gaussian distributions under the distilling supervision, whose histograms are in bell-shape~\cite{qin2022bibert}. 
% For example, in Fig.\,\ref{fig:motivation1}, we have shown the query and key distributions and their corresponding Probability Density Function (PDF) using the calculated mean and standard deviation for each cross-attention layer in the decoders of DETR. 

% Therefore, the distributions of query in the decoders of real-valued counterparts are formulated as
% \begin{equation}\label{eq:qk_distribution}
%     \begin{aligned}
%     {\bf q} \sim \mathcal{N}(\mu({\bf q}), \sigma({\bf q})). 
%     \end{aligned}
% \end{equation}
% Since the weight and the activation  with extremely compressed bit-width in fully quantized ViT have limited capabilities, the ideal quantization process should preserve the corresponding real-valued counterparts as much as possible, thus the mutual information between quantized and real-valued representations should be maximized~\cite{qin2022bibert}.
% When the deterministic quantization function is applied to quantized DETR, such objective is equivalent to maximizing the information entropy $\mathcal{H}(Q_{\bf q})$ of quantized representation $Q_{\bf q}$~\cite{messerschmitt1971quantizing} in Eq.\,(\ref{eq:attention}), which is defined as 
% \begin{equation}\label{eq:max_entropy}
%     \begin{aligned}
%     &\mathcal{H}(Q_a({\bf q})) = - \sum_{{\rm q} \in Q_a({\bf q})} p({\rm q}) \log p({\rm q}) = \frac{1}{2}\log 2 \pi e \sigma^2_{\bf q}, \\
%     &\mathop{\max} \mathcal{H}(Q_a({\bf q})) = \frac{n \ln 2}{2^n}, \;\; \operatorname{when} \; p({\rm q}) = \frac{1}{2^n}, 
%     \end{aligned}
% \end{equation}
% where ${\rm q}$ is the random quantized variables in $Q_a({\bf q})$ with probability mass function $p(\cdot)$. For better retaining the information contained in the MHA modules from the real-valued counterparts, the information entropy in the quantization process should be maximized. 

% However, direct application of quantization function converting the values into finite fixed points brings irreversible disturbance to the distributions and the information entropy $\mathcal{H}(Q_a({\bf q}))$ degenerates to a much lower level than the real-valued counterparts. To mitigate the information degradation from the quantization process in the attention mechanism, a information restoration transformation (IRT) is proposed for effectively maximizing the information entropy of quantized queries 
% \begin{equation}\label{eq:IRM}
%     \begin{aligned}
%     Q_a(\tilde{\bf q}) = Q_a({\bf q} * \beta + \gamma), 
%     \end{aligned}
% \end{equation}
% where $\beta$ and $\gamma$ are learnable parameters for modifying the distribution of $\tilde{\bf q}$. Thus after IRT, the information entropy $\mathcal{H}(Q_a({\tilde{\bf q}}))$ are formulated as
% \begin{equation}\label{eq:IRM_entropy}
%     \begin{aligned}
%     \mathcal{H}(Q({\tilde{\bf q}})) = \frac{1}{2}\log 2 \pi e (\beta^2 \cdot \sigma^2_{\bf q}). 
%     \end{aligned}
% \end{equation}
% Then to revive the attention mechanism to capture critic elements  by information entropy maximization, the learnable parameters $\beta$ and $\gamma$ reshape the distributions of the queries to achieve the state of information maximization. 

% {\color{blue} Following~\cite{messerschmitt1971quantizing}, given a quantizer with $a$ bits by quantizing real-valued values ${\bf x}$ into a set $\mathcal{Q} = \{[Q_1 = -2^{a-1}, Q_2, \cdots, Q_{N - 1}, Q_N = 2^{a-1}-1]\}, N = 2^a$, where the quantized value $Q({\bf x})$ is in the set $\mathcal{Q}$. The average mutual information $I({\bf x}; Q({\bf x}))$ is
%     \begin{equation}
%         \begin{aligned}
%         I({\bf x}; Q({\bf x})) = H(Q({\bf x})) - H(Q({\bf x}) | {\bf x}) = H(Q({\bf x})). 
%         \end{aligned}
%     \end{equation}
%     For fixed bit width $a$, $I({\bf x}; Q({\bf x}))$ is maximized by choosing
%     \begin{equation}
%         \begin{aligned}
%         p_k = Pr\{Q({\bf x} = Q_k)\} = \frac{1}{2^a}, k \in {1, \cdots, N}, 
%         \end{aligned}
%     \end{equation}
%     where $Pr\{\cdot\}$ denotes the probability. 
%     The process of minimizing the average error (MAE) between real-valued values and quantized values is written as 
%     \begin{equation}
%         \begin{aligned}
%         E_{\theta} = \sum_{k=1}^N \int_{Q_k}^{Q_{k+1}} p_k \cdot |{\bf x} - Q({\bf X}) |^{\theta} d{\bf x}. 
%         \end{aligned}
%     \end{equation}
%     And as mentioned in~\cite{smith1957instantaneous,messerschmitt1971quantizing}, an approximate relationship for the MAE objective with Gaussian distribution is
%     \begin{equation}
%         \begin{aligned}
%         \int_{Q_k}^{Q_{k+1}} p_k^* d{\bf x} \cong \frac{1}{N}, \\
%         p_k^* = A p_k^{\frac{1}{1 + \theta}}, 
%         \end{aligned}
%     \end{equation}
%     where $A$ is a constant. Thus the quantization process of minimizing the quantization error is approximately the same as maximizing the information entropy. We propose IRM for modifying the information entropy and distributions of the quantized representations in the forward process, and DGD for minimizing the information gap between real-valued representations and quantized counterparts in the backward process.
%     In this case, we prove that our method improves the performance of quantized ViT through maximizing the information entropy of the quantized representations.}



    
    %which is defined as
    % \begin{equation}
    %     \mathcal{L}_{match} = \mathcal{L}_{\operatorname{cls}}(c^{\mathcal{T}}_\sigma(n), c^{\mathcal{S}}_n) + \mathcal{L}_{\operatorname{bbox}}(b^{\mathcal{T}}_\sigma(n), b^{\mathcal{S}}_k), 
    % \end{equation}
    


%the sparsity of prediction results and the instability of query’s predictions~\cite{} make it difficult for DETR methods to orderly correspond the teacher’s results to the student’s predictions



% Since the representations with extremely compressed bit-width in fully quantized ViT have limited capabilities, the ideal quantized representation should preserve the given real-valued counterparts as much as possible, which means the mutual information between quantized and real-valued representations should be maximized as mentioned in~\cite{qin2022bibert}. 



%-------------------------------------------------------------------------



%%%%%%%%% REFERENCES
{\small
\bibliographystyle{ieee_fullname}
\bibliography{egbib}
}

\end{document}
