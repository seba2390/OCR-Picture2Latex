\documentclass[nonblindrev]{informs3}
\linespread{1.0}
\setlength{\parindent}{0pt}
%%\DoubleSpacedXI % Made default 4/4/2014 at request
\OneAndAHalfSpacedXI % current default line spacing
%%\OneAndAHalfSpacedXII
%%\DoubleSpacedXII
% Recommended, but optional, packages for figures and better typesetting:
\usepackage{microtype}
\usepackage{graphicx}
\usepackage{algorithm}
\usepackage{algorithmic}
\usepackage{subfigure}
\usepackage{booktabs} % for professional tables
\usepackage{multirow}
\usepackage{changepage}


\usepackage{hyperref}
\usepackage{natbib}
 \bibpunct[, ]{(}{)}{,}{a}{}{,}%
 \def\bibfont{\small}%
 \def\bibsep{\smallskipamount}%
 \def\bibhang{24pt}%
 \def\newblock{\ }%
 \def\BIBand{and}%

\usepackage{comment}

% Attempt to make hyperref and algorithmic work together better:
\newcommand{\theHalgorithm}{\arabic{algorithm}}

\OneAndAHalfSpacedXI

%% Setup of theorem styles. Outcomment only one.
%% Preferred default is the first option.
\TheoremsNumberedThrough     % Preferred (Theorem 1, Lemma 1, Theorem 2)
%\TheoremsNumberedByChapter  % (Theorem 1.1, Lema 1.1, Theorem 1.2)
\ECRepeatTheorems

% For theorems and such
\usepackage{amsmath}
\usepackage{amssymb}
\usepackage{mathtools}
%\usepackage{amsthm}

\usepackage[capitalize,noabbrev]{cleveref}
\crefname{assumption}{Assumption}{assumptions}
%%%%%%%%%%%%%%%%%%%%%%%%%%%%%%%%
% THEOREMS
%%%%%%%%%%%%%%%%%%%%%%%%%%%%%%%%
\theoremstyle{plain}
% \newtheorem{theorem}{Theorem}[section]
% \newtheorem{proposition}[theorem]{Proposition}
% \newtheorem{lemma}[theorem]{Lemma}
% \newtheorem{corollary}[theorem]{Corollary}
% %\theoremstyle{definition}
% \newtheorem{definition}[theorem]{Definition}
 % \newtheorem{assumption}[theorem]{Assumption}
% \newtheorem{remark}[theorem]{Remark}

\newcommand{\out}{\text{out}}
\newcommand{\inn}{\text{in}}

\newcommand{\ju}[1]{{\color{magenta} [JU: #1]}}
\newcommand{\zq}[1]{{\color{blue} [ZQ: #1]}}

\begin{document}

 \RUNAUTHOR{Qu, Galichon, Ugander}

 \RUNTITLE{Sinkhorn's Algorithm and Choice Modeling}

% Full title. Sample:
% \TITLE{Bundling Information Goods of Decreasing Value}
% Enter the full title:
\TITLE{On Sinkhorn's Algorithm and Choice Modeling}

% It is OKAY to include author information, even for blind
% submissions: the style file will automatically remove it for you
% unless you've provided the [accepted] option to the icml2023
% package.

% List of affiliations: The first argument should be a (short)
% identifier you will use later to specify author affiliations
% Academic affiliations should list Department, University, City, Region, Country
% Industry affiliations should list Company, City, Region, Country

% You can specify symbols, otherwise they are numbered in order.
% Ideally, you should not use this facility. Affiliations will be numbered
% in order of appearance and this is the preferred way.
\ARTICLEAUTHORS{%
\AUTHOR{Zhaonan Qu}
\AFF{Department of Economics, Stanford University, California, USA \EMAIL{zhaonanq@stanford.edu}} 
\AUTHOR{Alfred Galichon}
\AFF{Department of Mathematics and Department of Economics, New York University, New York, USA\\ Department of Economics, Sciences Po, Paris, France \EMAIL{alfred.galichon@nyu.edu}}
\AUTHOR{Johan Ugander}
\AFF{Department of Management Science \& Engineering, Stanford University, California, USA \EMAIL{jugander@stanford.edu}}
% Enter all authors
} % end of the block

\ABSTRACT{For a broad class of choice and ranking models based on Luce's choice axiom, 
including the Bradley--Terry--Luce and Plackett--Luce models, we show that the associated maximum likelihood estimation problems are equivalent to a classic matrix balancing problem with target row and column sums. This perspective opens doors between two seemingly unrelated research areas, and allows us to unify existing algorithms in the choice modeling literature as special instances or analogs of Sinkhorn's celebrated algorithm for matrix balancing. 
We draw inspirations from these connections and resolve important open problems on the study of Sinkhorn's algorithm. We first prove the global linear convergence of Sinkhorn's algorithm for non-negative matrices whenever finite solutions to the matrix balancing problem exist. We characterize this global rate of convergence in terms of the algebraic connectivity of the bipartite graph constructed from data. Next, we also derive the sharp asymptotic rate of linear convergence, which generalizes a classic result of Knight (2008), but with a more explicit analysis that exploits an intrinsic orthogonality structure. To our knowledge, these are the first quantitative linear convergence results for Sinkhorn's algorithm for general non-negative matrices and positive marginals. 
The connections we establish in this paper between matrix balancing and choice modeling could help motivate further transmission of ideas and interesting results in both directions.
}
\KEYWORDS{Choice Modeling, Matrix Balancing, Sinkhorn's Algorithm, Algebraic Connectivity}
\maketitle

Reinforcement learning has achieved great success in areas such as Game-playing \citep{silver2018general,vinyals2019grandmaster}, robotics \cite{kober2013reinforcement}, large language models \citep{ouyang2022training}, etc.
However, due to safety concerns or physical limitations, in some real-world reinforcement learning problems, we must consider additional constraints that may influence the optimal policy and the learning process \citep{garcia2015comprehensive}.
% For example, a robotic arm must not take actions that may cause harm to itself or the environments.
A standard framework to handle such cases is the constrained Markov Decision Process (CMDP) \citep{altman1999constrained}.
Within the CMDP framework, the agent has to maximize
the expected cumulative reward while
obeying a finite number of constraints, which are usually in the form of expected cumulative cost criteria.

However, we are sometimes concerned with the problem with a continuum of constraints.
For example,
the constraints we meet might be time-evolving or subject to uncertain parameters, which
cannot be formulated as an ordinary CMDP
(see Examples \ref{Example_Time_Evolving} and  \ref{Example_Uncertain}).
In this paper we would study a generalized CMDP  
to address the above problem.  Because the constraints are not only infinite-number but also lie
in a continuous set,
the generalization is not trivial. Fortunately, we find that we can borrow the idea behind semi-infinite programming (SIP) \citep{remez1934determination, hettich1993semi} to deal with the semi-infinite constraints.
Accordingly, we propose \emph{semi-infinitely constrained Markov decision processes} (SICMDPs)
as a novel complement to the ordinary CMDP framework.
%More specifically,  an SICMDP model %, we consider 
%contains a continuum of constraints whereas an ordinary CMDP contains a finite number of constraints. 

%This generalization is natural but not trivial. However, we can brows the idea  
%The idea is quite natural and can be backtracked
%to the practice of extending linear programming to linear semi-infinite programming (LSIP) %\cite{remez1934determination, GobernaLSIO1998}.
%In addition, 
%As a complementary approach to the ordinary CMDP framework, 
%SICMDP can be used to model these problems  which cannot be described by a finite number of constraints
%that are not covered by .
%For example,
%the restrictions we consider can be time-evolving or subject to uncertain parameters
%, thus
%cannot be described by a finite number of constraints but a continuum of constraints 
%(see Examples \ref{Example_Time_Evolving} and  \ref{Example_Uncertain}).

We also present two reinforcement learning algorithms to solve SICMDPs called SI-CRL and SI-CPO, respectively.
SI-CRL is a model-based reinforcement learning algorithm designed for tabular cases, and SI-CPO is a policy optimization algorithm for non-tabular cases.
% and analyze its performance both theoretically and empirically.
The main challenge is that we need to deal with a continuum of constraints, thus reinforcement learning algorithms for ordinary CMDPs do not work anymore.
In SI-CRL, we tackle this difficulty by first transforming the reinforcement learning problem to an equivalent LSIP problem, which can then be solved using methods in the LSIP literature like the dual exchange methods \citep{Hu1990,reemtsen1998numerical}.
In SI-CPO, we resort to the idea of cooperative stochastic approximation developed in \cite{lan2020algorithms, wei2020comirror}.
As far as we know, we are the first to introduce tools from semi-infinitely programming (SIP) into the reinforcement learning community for solving constrained reinforcement learning problems.

% To the best of our knowledge, we are the first to apply tools from semi-infinitely programming (SIP) to solve reinforcement learning problems.
Furthermore, we give theoretical analysis for both SI-CRL and SI-CPO.
We decompose the error of SI-CRL into two parts: the statistical error from approximating the true SICMDP with an offline dataset and the optimization error due to the fact that the solution of the LSIP problem obtained by the dual exchange method is inexact.
On the optimization side, we show that the iteration complexity of SI-CRL is $O\left(\left\{\mathrm{diam}(Y)L\sqrt{|\gS|^2|\gA|m}/\left[(1-\gamma)\epsilon\right]\right\}^m\right)$.
On the statistical side, we show that the sample complexity of SI-CRL is $\widetilde O\left(\frac{|S|^2|A|^2}{\epsilon^2(1-\gamma)^3}\right)$ if the offline dataset is generated by a generative model, and $\widetilde O\left(\frac{|S||A|}{\nu_{\min} \epsilon^2(1-\gamma)^3}\right)$ if the dataset is generated by a probability measure $\nu$ as considered in \cite{chen2019information}.
Here $\widetilde O$ means that all logarithm terms are discarded.
For SI-CPO, things become a little more complicated because other than the statistical error and the optimization error, we also need to consider the function approximation error, which comes from imperfect policy parametrizations.
It is shown if the function approximation error can be controlled to $O(\epsilon)$ order, the iteration complexity of SI-CPO is $\widetilde{O}\left(\frac{1}{\epsilon^2(1-\gamma)^6}\right)$ and the sample complexity of SI-CPO is $\widetilde{O}(\frac{1}{\epsilon^4(1-\gamma)^{10}})$.
Here our iteration complexity bound is equivalent to a typical $\widetilde O(1/\sqrt{T})$ global convergence rate.

We perform a set of numerical experiments to illustrate the SICMDP model and validate our proposed algorithms.
Specifically, we examine two numerical examples, namely the discharge of sewage and ship route planning.
Through the discharge of sewage example, we show the advantage of the SICMDP framework over the CMDP baseline obtained by naive discretization in modeling realistic sequential decision-making problems.
Moreover, we demonstrate the effectiveness of the SI-CRL and SI-CPO algorithms in such tabular environments. 
In the ship route planning example, we illustrate the benefits of the SICMDP framework and the ability of the SI-CPO algorithm to address complex continuous control tasks involving continuous state spaces with modern deep reinforcement learning techniques.

% In summary, our contributions are listed as follows.
% First, we present the SICMDP model, which can be viewed as a generalization of the ordinary CMDP model.
% Second, we propose an algorithm to perform reinforcement learning for SICMDPs, which is called SI-CRL, and we believe that we are the first to apply tools from SIP
% to solve reinforcement learning problems.
% Third, we give a theoretical analysis of SI-CRL and identify both its sample complexity and iteration complexity.
% In addition, we perform numerical experiments to illustrate the SICMDP model and validate the SI-CRL algorithm.
% \{This paragraph can be removed!!! \}






\paragraph{Modelling the other:} Modelling other agents is a key component of multi-agent reinforcement learning (MARL). There exists a vast body of work ranging from inferring the other's policy \cite{foerster:aamas18,wen2019probabilistic,hu2020other,shu2018m}, goals and beliefs \cite{Raileanu2018ModelingOU,moreno2021neural} and value functions \cite{zhao2022mcmarl,pmlr-v48-he16}. These works predominantly assume that all agents are being trained concurrently which differs from our intended setting where only a single agent is trained. Modelling agents who behave according to a fixed pre-trained policy can be tackled with works in Theory of Mind (ToM) \cite{pmlr-v80-rabinowitz18a} and Inverse Reinforcement Learning (IRL) \cite{ng2000algorithms} literature. A small subset of MARL combines these two problems to create environments in which a learning agent (to be trained) coexists with and tries to model a pre-trained (independent) agent \cite{papoudakis2021agent,SympathyPaper}, in line with the problem setting of our work. However, such works generally do not produce interpretable models, and produce arbitrarily scaled action-value estimates, which impedes the accurate inference of agent behaviours.% \manisha{added in previous line}
%\thommen{Can we say something like `However, such works generally do not produce interpretable models, and produce arbitrarily scaled action-value estimates, which impedes the accurate inference of agent behaviours.'} %Our work targets this subset.
%One such paper is that by  in which the learning agent is assumed to only have access to the other agent's trajectories during training. As such, to incorporate the other's behaviour into the learning agent's policy, a latent representation is extracted and trained on. \thommen{What is this paper's relation to the current work?}

\paragraph{Modelling based on oneself:} A selection of works model the other agent based on their own model. Using ToM, \cite{Raileanu2018ModelingOU} trains the learning agent on all possible goals during training and uses this information to infer the hidden goal of the other agent based on its behaviour. This method differs from our setting as it is constrained to games that have set goals which can be experienced by the learner. Inspired by empathy as well, \cite{TowardsEmpathicDQN} proposes 
%has a learning agent that shares an environment with a single independent agent. In order 
imposing the learning agent's own value function directly on the independent agent, using this to infer the other's intent. A limitation is that imposing the same value function on the independent agent assumes this agent has the same values as the learner. Our work eschews this assumption, allowing for different and even opposing intentions. 

\paragraph{Composite value and reward functions:} In multi-agent scenarios with composite reward or value functions (e.g. summation of two or more reward or value estimates), it is important they are scaled appropriately to ensure stable behaviours. %\st{ensuring comparability}\thommen{it might be unclear what we mean by comparability. How about `When summating two or more reward or value estimates, to achieve stable behaviours, it is important to ensure they are scaled appropriately.'} \st{is important to produce stable behaviours.} 
 Approaches such as VDN \cite{VDN} and QMIX \cite{rashid2018qmix} combine separate agent value functions to conduct centralised training, thus obviating the need for such scaling. However, this does not allow for independent agents with pre-trained policies. %When all agents are being trained, these value functions do not need any further scaling or adjustment prior to combining them together. 
 More closely related, \cite{alamdari2021considerate} builds a joint function of learning agent and independent agent rewards (whose rewards are already known). A similar joint reward function is built by \cite{SympathyPaper} however they use IRL to to infer the rewards of the independent agents.
%\st{More closely related algorithms include that by \cite{alamdari2021considerate}, where a joint function of the learning agent's reward together with the reward of all independent agents is constructed to train the learning agent. No scaling was applied, as it was assumed that a distribution over reward functions for the independent agent was known. \cite{SympathyPaper} also trains a learning agent on a similar joint reward function, with the independent agent's reward function being inferred via IRL.}
As a result of the space of potential rewards that can emerge through IRL \cite{PolicyInvariance}, the paper mitigates the issues of a misalignment by scaling the independent agent's functions by a constant (the ratio of the $l1$ norms of the learning agent's reward vector and the IRL inferred rewards of the independent agent). This method was also applied by \cite{Noothigattu2019}. %\st{, where an ethical agent was trained using a multi-arm bandit reward function (using a combination of rewards).} 
This simple $l1$ norm based normalisation may fail in many complex scenarios, and additionally, is constrained to problems that only sum two reward or action-value functions, motivating need for alternatives. %Our proposed work aims to address this issue, along with other  \manisha{one of which our work proposes.}
%\thommen{Can add a concluding statement for the related literature}%\st{In the case of \textcolor{red}{Sympathy Paper}, this restricts the approach to handling no more than one independent agent. Scaling by a constant does not guarantee comparable reward or value functions between agents, which motivates the need for better alternatives. As such, we propose the use of the learning agent's own action value estimates to infer that of the independent agent, addressing the aforementioned issues of comparability.}

%Interest in designing agents with considerate and ethical behaviours has been growing over time \cite{Abel2016ReinforcementLA}, \cite{SachiyoArai2014}, \cite{EmpDC}, \cite{alamdari2021considerate}. Much like our work, \cite{Raileanu2018ModelingOU} used an empathy-like approach to model other agents based on the learning agent's own model. In each episode of training, each agent is randomly assigned a task. The rewards obtained are dependent on the goals of both agents, incentivizing the need to model and understand the goals of the other. 
%\thommen{From here till ...player is?" can be skipped I think.}
%The goal of each agent in an environment is binary (e.g. collect a specified colour ball), with each agent being randomly allocated one of the possible goals at the start of each episode. Due to the change of goal in each episode, an agent is able to experience each possible goal and incorporate this information into their model. As a result, inferring the goal of the other agents becomes easier as they are able to reference their own experiences to infer, based on observations of the others actions, what the other's goal may be by asking the question "What would my goal have been had I acted in the way the other player is?". 
%The drawback of such a method is that it requires all agent-goal combinations to be sampled. Because each agent experiences all possible tasks, learning to infer what the goals of the other becomes easier as the agent can draw back to direct experiences they have had in the past with that same task. Another work by \cite{TowardsEmpathicDQN} involves a learning agent sharing an environment with a pre-trained independent agent, the authors evoke empathy by imposing the value function of the learning agent over the independent agent to determine which action to take by asking the questions, "If I was in the other's position, what action would I want the other to take?". The limitation of this work is the assumption that both agents have the same objective. In contrast to both of these works, we focus on a more general setting, where the reward functions or goals of both agents are not necessarily identical.

%Closely related to our work, \cite{SympathyPaper} developed a Sympathy Framework which allowed both agents to have differing objectives. To model the independent agent, the authors utilised the Cascaded Supervised Learning method \cite{cascadedSuperIRL2013} which is able to infer a value function and reward function of the learning agent. Using these, the authors construct a sympathetic reward as a convex combination of the learning agent's reward and the inferred reward of the independent agent on which a sympathetic policy in trained for the learning agent. The weighting is determined by a degree of sympathy term which adaptively adjusts itself to appropriate values depending on the situation faced. 
%In IRL it is commonly known that there is a space of possible reward functions that can express the behaviour observed in an agent \cite{PolicyInvariance}. 
%Although the authors were able to produce considerate behaviours in the learning agent, the IRL method is susceptible to inferring an incorrect reward function due to the vast space of possible reward functions that could corresponding to an agent's observed behaviour \cite{PolicyInvariance}. As the sympathetic reward is a combination of both agent's rewards, it is particularly important that the rewards of both agents are in a consistent and comparable range. To address this, the authors scaled the inferred reward function of the learning agent to have the same $l1$ norm as the independent agent. This method may not always be appropriate or work. To address this limitation, we propose utilising the learning agent's own value function as a reference for the independent agent's value function to ensure the resulting value and reward functions will be consistent between both agents.
\section{Preliminaries}\label{chpt:preliminiaries}
In this chapter we will introduce some of the mathematical background and notation needed for this thesis. In particular, we will shortly introduce the differential geometric description of spacetime in Section \ref{sec:spacetime_geometry} and give an introduction to the notion of global hyperbolicity and its connection to Green- and normally-hyperbolic operators in Section \ref{sec:global_hyperbolicity}. In a bit more detail, we will introduce the notion of differential forms and give explicit definitions, also in terms of an index based notation, in Section \ref{sec:differential_forms}. For completeness, in Section \ref{sec:cat-theory}, we present basic definitions of category theory. The reader familiar with these topics can safely skip this chapter and refer to it when interested in the chosen conventions.
%
%
%
%
%%%%%%
%%SPACTIME GEOMETRY
%%%%%
%
%
%
\subsection{Spacetime geometry}\label{sec:spacetime_geometry}
In GR, the universe is mathematically described as a four dimensional \emph{spacetime}, consisting of a smooth, four dimensional manifold \gls{M} (assumed to be Hausdorff, connected, oriented, time-oriented and para-compact) and a Lorentzian metric $g$. We will assume the signature of the Lorentzian metric $g$ to be $(-,+,+,+)$. The Levi-Civita connection on $(\M,g)$ is as usual denoted by \gls{nabla}.
Throughout this thesis, we treat spacetime as fixed, implementing a gravitational background determined classically by Einstein's field equations. Hence, we neglect any back-reaction of the fields on the metric, both in the quantum and the classical case. In that sense, we treat the fields as \emph{test fields}.\par
For the basic mathematical theory regarding Lorentzian manifolds, we refer to the literature: An introduction to the topic with an emphasis on the physical application in GR is for example given in \cite{wald_GR} and \cite{carroll_spacetime-and-gr}.
Here, we will shortly recap the notion of a tangent space and tangent bundle and generalize to the notion of a vector bundle which we will use in the general description of normally hyperbolic operators and differential forms.
In the following, we generalize the setting to an arbitrary smooth manifold $\N$ of dimension $N$ with either Lorentzian or Riemannian metric $k$.\par
%
%
A \emph{tangent vector} $v_x$ at point $x \in \N$ is a linear map $v_x : C^\infty(\N , \IR) \to \IR$ that obeys the Leibniz rule, that is, for $f,g \in C^\infty (\N,\IR)$ it holds $v_x(fg) = f(x)v_x(g) + v_x(f)g(x)$.
We define the \emph{tangent space} \gls{TxN} of $\N$ at $x$ as the real $N$-dimensional vector space of all tangent vectors at point $x$.
The disjoint union of all tangent spaces is called the \emph{tangent bundle} \gls{TN} of $\N$ and is itself a manifold of dimension $2N$. A \emph{vector field} is a map $v: \N \to T\N$ such that $v(x) \in T_x\N$.
The respective dual spaces, that is the space of all linear functionals, the \emph{co-tangent space} and the \emph{co-tangent bundle}, are denoted by \gls{TsxN} and \gls{TsN} respectively.\par
%
For Lorentzian manifolds, we call a tangent vector $v$ at $x \in \N$ \emph{timelike} if $k_{\mu \nu} v^\mu v^\nu < 0$, \emph{spacelike} if $k_{\mu \nu} v^\mu v^\nu > 0$ and \emph{null} (or lightlike) if $k_{\mu \nu} v^\mu v^\nu = 0$. At every point $x \in \N$, we define the set of all \emph{causal}, that is, either timelike or null, tangent vectors in the tangent space at $x$. This set is called the \emph{light cone} at $x$ and it is split up into two distinct parts, one that we call the future light cone, and one that we call the past light cone at $x$. Since we assume the manifold to be time orientable, there exists a smooth vector field $t$ that is timelike at every $x \in \N$. Given this time orientation, we identify the future (past) light cone with the set of tangent vectors $v \in T_x\N$ such that $k_{\mu\nu} v^\mu t^\nu < 0$ (respectively $> 0$). Therefore, a tangent vector $v$ at $x$ is called \emph{future directed} (past directed) if it lies in the future (past) light cone at $x$.\\
Accordingly, a curve $\gamma : I \to \N$ is called timelike (spacelike, null, causal, future or past directed) if its tangent vector $\dot{\gamma}$ is timelike (spacelike, null, causal, future or past directed) at every $x \in \N$.  For every point $x \in \N$ we define the \emph{causal future/past} \gls{causalfuturepast} of $x$ as the set of all points $q \in \N$ that can be reached by a future directed causal curve originating in $x$. For any subset $S \in \N$ we define $J^\pm (S) = \bigcup_{x \in S} J^\pm(x)$ and $J(S) = J^+(S) \cup J^- (S)$. Finally, the future/past domain of dependence $\gls{futurepastdomainofdependence}$ of a set $S \subset \N$ is the set of all points $x \in \N$ such that every inextendible causal curve through $x$ intersects $S$. The \emph{domain of dependence} \gls{domainofdependence} of $S$ is the union of the future and past domain of dependence of the set $S$.
For more details on the causal structure of spacetime we refer to for example \cite[Chapter 8]{wald_GR}.\par
%
%
%
The notion of tangent bundles can be generalized to the notion of a vector bundle. Instead of ``attaching'' the vector spaces $T_x \N$ to every point $x$ of the manifold, we allow for the occurrence of arbitrary vector spaces, called the fibres of the vector bundle. A vector bundle then consists of the base manifold, in our case $\N$, the total space and a map $\pi$ from the total space to the base manifold, that can be locally trivialized. At each point of the base manifold, the pre-image of $\pi$ is the fibre of the vector bundle. To be precise we define, following \cite{rudolph_schmidt}:
\begin{definition}[Vector bundle]
	A smooth \emph{vector bundle} over $\N$ is a tuple $\gls{vectorbundle} = (E,\N, \pi)$, where $E$ is a smooth manifold and $\pi : E \to \N$ is a smooth surjective map satisfying:
	\begin{enumerate}
		\item For every $x \in \N$, $\pi^{-1}(x)$ is a vector space, called the fibre of the bundle at point $x$.
		\item There exists a finite dimensional vector space $F$, an open covering $\left\{ U_\alpha\right\}_\alpha$ of $\N$ and a family of diffeomorphisms $\chi_\alpha : \pi^{-1}(U_\alpha) \to U_\alpha \times F$ such that for all $\alpha$ it holds $\chi_\alpha \comp \text{pr}_1 =  \restr{\pi}{\pi^{-1}(U_\alpha)}$ and for every $x \in \N$ the map $\text{pr}_2 \comp \restr{\chi_\alpha}{\pi^{-1}(x)} : \pi^{-1}(x) \to F$ is linear.
	\end{enumerate}
\end{definition}
Here, the maps $\text{pr}_1$ and $\text{pr}_2$ denote the projection onto the first respectively second component of an element in $U_\alpha \times F$. The properties graphically mean that \emph{locally}, the vector bundle ``looks like" the product of the base manifold with the fibre. The tuples $(U_\alpha, \chi_\alpha)$ are called \emph{local trivializations} of the vector bundle. Like for vector spaces, we can define the sum and product of vector bundles, by using the according vector space definitions on the fibres of the bundle.\par
Let $\mathfrak{X}, \mathfrak{Y}$ be vector bundles over $\N$ with fibres $X_x$ and $Y_x$ at $x \in \N$. We denote by \gls{whitneysum} the \emph{Whitney sum} of the two vector bundles - the vector bundle over $\N$ whose fibres are given by the direct sum $X_x \oplus Y_x$. Similarly, one obtains the local trivializations of the Whitney sum from the trivializations of $\mathfrak{X}, \mathfrak{Y}$ and direct sums.\par
Accordingly, let $\mathfrak{X}, \mathfrak{Y}$ be vector bundles over $\N$ and $\widetilde{\N}$, with fibres $X_x$ and $Y_{\tilde{x}}$ at $x \in \N$, $\tilde{x} \in \widetilde{\N}$ respectively. We denote by \gls{outerproductbundle} the \emph{outer product} of the two vector bundles - the vector bundle over $\N \times \widetilde{\N}$ whose fibres are given by the tensor products $X_x \otimes Y_x$. Similarly, one obtains the local trivializations of the outer product from the trivializations of $\mathfrak{X}, \mathfrak{Y}$ and tensor products. \par
%
Finally, we generalize the notion of vector fields:
\begin{definition}[Sections of vector bundles]
Let $\mathfrak{X}=(E,\N,\pi)$ be a vector bundle with fibres $X_x=\pi^{-1}(x)$ at $x \in \N$. A \emph{smooth section} of the vector bundle is a smooth map $\gamma : \N \to E$ such that $\gamma(x) \in X_x$ for all $x \in \N$. The \emph{vector space of smooth sections} of $\mathfrak{X}$ is denoted by \gls{gammax}, the one with compactly supported sections is as usual denoted by \gls{gammaxzero}.
\end{definition}
In this language, a vector field $v$ is just a smooth section of the tangent bundle of a manifold, $v \in \Gamma(T\N)$. One may therefore identify the physical notion of fields with smooth sections of vector bundles. This point of view will be used to define the notion of differential forms in Section \ref{sec:differential_forms}.\par
In this thesis, we usually are interested in complex valued functions (or sections in general). Therefore, we view all occurring vector bundles as complex, in the sense that we take two distinct copies of the vector bundle, one representing the real, one the imaginary part of the bundle. A section of that complex vector bundle is just a pair of two sections of the real vector bundle under consideration. From now, if not specified explicitly, we will view all vector bundles, including the tangent bundle $T\N$, as complex vector bundles. Accordingly, smooth sections of those bundles will in general be complex valued.
%
%
%
%
%
%
%
%
%%%%%%%
%%PARTIAL DIFFERENTIAL OPERATORS AND GLOBAL HYPERBOLICITY
%%%%%%%
%
%
%
\subsection{Partial differential operators and global hyperbolicity}\label{sec:global_hyperbolicity}
When dealing with field theories, whether classical or quantum, one is, of course, interested in the dynamics of the fields. These are usually described by some partial differential equation, often of second order. In the following, we give a short introduction to the theory of certain partial differential operators acting on smooth sections of a vector bundle over the spacetime $(\M,g)$.\par
%
As we have seen, these smooth sections are generalizations of the notion of a field.  In the following, let $\mathfrak{X}$ denote a vector bundle over the manifold $\M$ and let $P: \Gamma(\mathfrak{X}) \to \Gamma(\mathfrak{X})$ be a partial differential operator acting on smooth sections of the bundle. As in the case of flat spacetime, we are interested in basic questions regarding the differential equation $Pf = j$, for example: Can we formulate a (globally) well posed initial value problem? Does the differential equation possess (unique) solutions? To answer these questions, we will now restrict to the case where $P$ is linear and of second order, as it is often the case in physical applications. One can show that for a certain class of such operators, namely normally hyperbolic partial differential operators of second order, we can rigorously treat these questions.\par
Choosing local coordinates $x=(x_\mu)$ on $\M$ and a local trivialization of $\mathfrak{X}$, a linear partial differential operator of second order is called \emph{normally hyperbolic} if it takes the form
\begin{align}
	P = - \sum_{\mu,\nu} g^{\mu \nu} \partial_\mu \partial_\nu + \sum_{\alpha} A_\alpha (x) \partial_\alpha + B(x) \formspace,
\end{align}
where $A_\alpha$ and $B$ are matrix-valued coefficients depending smoothly on the coordinate $x$ (see. \cite[Chapter 1.5]{baer_ginoux_pfaeffle}). One can also formulate a coordinate independent definition in terms of the principal symbol, which we will not present here (see for example \cite[Section 1.5]{baer_ginoux_pfaeffle} ). \par
%
Normally hyperbolic operators possess unique fundamental solutions (see for example the fundamental solutions to the wave operator as noted in Lemma \ref{lem:fundamental_solution_wave_operator}). These fundamental solutions fulfill certain physically important properties, such as a finite propagation speed smaller than the speed of light. Furthermore, specifying the initial data on some space-like hypersurface $X \in  \M$ specifies a unique solution on the domain of dependence $D(X)$ of $X$. Due to these properties, one often calls normally hyperbolic operators just \emph{wave operators}. But to state a \emph{globally} well posed initial value problem for a wave equation, we need to restrict the class of spacetimes $\M$ under consideration to those that possess space-like hypersurfaces $X$ whose domain of dependence is all of the spacetime, $D(X) = \M$. This leads to the notion of \emph{globally hyperbolic} spacetimes:
\begin{definition}[Global Hyperbolicity]
	A spacetime $\M$ is called \emph{globally hyperbolic} if there exists a Cauchy surface $\gls{sigma}$ in $\M$.
\end{definition}
\noindent Here, a Cauchy surface is a space-like hypersurface $\Sigma \subset \M$ such that every inextendible causal curve $\gamma$ intersects $\Sigma$ exactly once. One can show that Cauchy surfaces fulfill the desired property mentioned above, that is,  $D(\Sigma) = \M$. Furthermore, one can show that any globally hyperbolic spacetime $\M$ is foliated by a one-parameter family $\left\{ \Sigma_t \right\}_t$ of Cauchy surfaces (see for example \cite[Theorem 8.3.14]{wald_GR}). \par
In physical applications, one often finds the dynamics of a theory to be described by wave operators. Most prominently, the Klein-Gordon operator $(\square + m^2)$ acting on scalar fields, or its generalization, the wave operator acting on differential forms introduced in Section \ref{sec:differential_forms}, is normally hyperbolic. But there are also important physical field theories that are not described by wave operators, such as the Proca field treated in this thesis. It turns out that the Proca operator (see Definition \ref{def:proca_operator}) is a so called \emph{Green-hyperbolic} operator. These are again partial differential operators $P$ of second order acting on smooth sections of some vector bundle, such that $P$ (and its dual $P'$) posses fundamental solutions. Obviously, normally hyperbolic operators are Green-hyperbolic, but the opposite is not true. One can generalize some results obtained by studying normally hyperbolic operators to Green-hyperbolic operators. An introduction to this topic is given in \cite{baer_green-hyperbolic}, where it is also shown that the Proca operator is Green-hyperbolic but not normally hyperbolic.\par
For our application, the notion of Green-hyperbolicity is not of vast importance, but it is worth mentioning that there exists a more detailed mathematical background on the treatment of such operators.
A very detailed description of normally hyperbolic operators on Lorentzian manifolds, including proofs of the above statements regarding the initial value problem and the existence of fundamental solutions, is given in \cite{baer_ginoux_pfaeffle}, also with an overview of quantization. A shorter introduction to the topic is for example treated in \cite{baer-ginoux_classical-and-quantum-fields}, also with a description of quantization.
%
%
%
%
%
%
%%%
%
%
%
%%
%%%%%%%%%
%%%DIFFERENTIAL FORMS
%%%%%%%%
%
%
%
\subsection{Differential forms}\label{sec:differential_forms}
%
%
Differential forms provide an elegant, coordinate independent description of calculus on smooth manifolds. In particular, they generalize the notion of line- and volume-integrals that are known from analysis. Differential forms play a remarkable role in physics, as one can argue that they indeed describe fundamental physical entities. As an example, instead of viewing a classical force as a vector, one can think of it, more closely related to experiments, as a differential one-form that assigns a scalar to a tangent vector of a curve. This scalar is the (infinitesimal) work associated with the force along the curve. Also, differential forms allow for an elegant geometric description of field theories, for example the Maxwell and Proca field theories that we encounter in this thesis. In Maxwell's classical theory of electromagnetism, instead of viewing the electric and magnetic field (which are conceptually just forces) as the fundamental physical entities, one introduces the \emph{vector potential}, a one-form, consisting of the scalar electric potential and the vector potential associated with the magnet field. Experiments like the Aharonov-Bohm experiment allow for an interpretation of the vector potential as the fundamental physical object, rather than the associated electromagnetic field. \\
Even more fundamentally, the two main theories of physics, General Relativity and the Standard Model of particle physics, are field theories. They are deeply connected to a geometric interpretation and can be elegantly described using differential forms. \par
%
%
Despite of all this, differential forms are usually not part of the standard curriculum of physicists. We shall therefore introduce the basic aspects and definitions regarding differential forms that are used in this thesis. For a more detailed introduction we refer to the literature: For example \cite[Chapter 2 and 4]{rudolph_schmidt} or \cite[Appendix B]{wald_GR} provide introductions to the topic.\par
%
%
In the following, let $\N$ denote a smooth $N$-dimensional manifold, assumed to be Hausdorff, connected, oriented and para-compact, with either Lorentzian or Riemannian metric $k$ and Levi-Civita connection $\nabla$. For a Lorentzian manifold we use the sign convention $(-,+,\dots,+)$ of the metric $k$. The number of negative eigenvalues of $k$ is denoted by $s$, so $s=0$ for a Riemannian manifold and, in our convention, $s=1$ for a Lorentzian manifold.
Later, we will specify to a four dimensional (globally hyperbolic) spacetime consisting of a four dimensional manifold $\M$ with Lorentzian metric $g$ and Cauchy surface $\Sigma$ with induced Riemannian metric $h$.
%
We define:
\begin{definition}[Differential form]
	Let $p\in \{0,1,\dots,N\}$. A \emph{differential form} $\omega$ of degree $p$, or $p$-form for short, on the manifold $\N$ is an anti-symmetric tensor field of rank $(0,p)$. That is, at every point $x \in \N$, $\omega_x$ is an anti-symmetric multi-linear map
	\begin{align}
	\omega_x : \underbrace{T_x \N \times T_x \N \times \cdots \times T_x \N}_{p\text{-times}} \to \IR \formspace.
	\end{align}
	We denote the vector space\footnote{Naturally, addition and scalar multiplication are defined point-wise.} of $p$-forms on $\N$ by $\gls{omegap}$, the space with compactly supported ones by \gls{omegapz}.
\end{definition}
As an example, a zero-form $f \in \Omega^0(\N)$ is just a $C^\infty$-function from $\N$ to $\IR$, hence we can identify $\Omega^0(\N) = C^\infty (\N, \IR)$. A one-form $A \in \Omega^1(\N)$ is nothing more than a co-vector field and in a physical context usually denoted in local coordinates by $A_\mu$. Note, that alternatively one can directly define a $p$-form as a smooth section of the $p$-th exterior product of the co-tangent bundle and hence identify $\Omega^p(\N) = \Gamma \big( \largewedge^k T^*\N\big)$. As mentioned in Section \ref{sec:spacetime_geometry}, we view the tangent bundle as a complex bundle. Therefore, the sections of that bundle will be complex valued functionals. In that fashion, we will usually view the spaces $\Omega^p(\N)$ as complex valued differential forms.\par
%
Next we define the basic operations, besides addition and scalar multiplication, that one can perform on differential forms.
%
\begin{definition}[Exterior product]
	Let $A \in \Omega^p(\N)$ be a $p$-form and  $B\in \Omega^q(\N)$ a $q$-form on $\N$. \\
	The \emph{exterior product} $\gls{wedge}:\Omega^p(\N) \times \Omega^q(\N) \to \Omega^{p+q} (\N)$ is defined by
	\begin{align}
	(A \wedge B)_{\mu_1\dots\mu_p \nu_1\dots\nu_q} = \frac{(p+q)!}{p!q!}\, A_{[\mu_1 \dots \mu_p} B_{\nu_1\dots\nu_q]} \formspace,
	\end{align}
	where the anti-symmetrization of a tensor $T$ is given through
	\begin{align}
	T_{[\mu_1\dots\mu_p]} = \frac{1}{p!} \sum\limits_{\sigma\in S_N }\textrm{sgn}(\sigma) T_{\sigma(\mu_1)\dots\sigma(\mu_p)} \formspace.
	\end{align}
\end{definition}
Here, $S_N$ denotes the symmetric group\footnote{Usually the symmetric group is defined as the set of permutations of $\{1,2,\dots,N\}$ but we chose the index to run over $\{0,1,\dots,N-1\}$, identifying the time component with zero rather then one.} of degree $N$, consisting of permutations of the set $\{0,1,\dots,N-1\}$.
With this notion of multiplication, point-wise addition and scalar multiplication, the space $\gls{omega} \coloneqq \bigoplus_{p = 0}^\infty \Omega^p(\N) = \bigoplus_{p = 0}^N \Omega^p(\N)$ becomes an algebra, usually called the Grassmann- or \emph{exterior algebra} of differential forms on $\N$. We have used that obviously $\Omega^k(\N) =0$ for $k >N$ due to the anti-symmetrization.\par
Furthermore, we find a notion of how to \emph{pullback} differential forms on manifolds to another manifold, for example the pullback of a differential form on the spacetime $\M$ to differential forms on its Cauchy surface $\Sigma$. Given a $C^\infty$-map $\psi: \widetilde{\N} \to \N$, where $\N, \widetilde{\N}$ are manifolds, we can naturally define the pullback of a function $f \in \Omega^0(\N)$ to a function $(\psi^* f) \in \Omega^0(\widetilde{\N})$ by composing $f$ with $\psi$:
\begin{align}
\psi^* f \coloneqq f \comp \psi \formspace.
\end{align}
\newpage
With the pullback of functions defined, we can define how to \emph{push forward}, or carry along, vector fields on $\widetilde{\N}$ to vector fields on $\N$: Let $f\in \Omega^0(\N)$ and $\tilde{v} \in \Gamma(T\widetilde{\N})$ and $\tilde{x} \in \widetilde{\N}$. Then
\begin{align}
(\psi_* \tilde{v})_{\psi(\tilde{x})} (f) \coloneqq \tilde{v}_{\tilde{x}}(\psi^* f)
\end{align}
defines the vector field $(\psi_* v) \in \Gamma(T\N)$. With these basic operations at hand, we can generalize to define the pullback of differential forms:
\begin{definition}[Pullback]\label{def:pullback}
	Let $\N, \widetilde{\N}$ be manifolds of dimension $N,\widetilde{N}$ respectively, and let $\psi: \widetilde{\N} \to \N$ be a smooth map. Then, $\psi$ defines an algebra homomorphism $\psi^* : \Omega(\N) \to  \Omega(\widetilde{\N})$,
	called the \emph{pullback} of differential forms. For $\omega \in \Omega^p(\N)$, $\tilde{x} \in \widetilde{\N}$ and $\tilde{v}_i \in T_x \widetilde{\N}$, $i=1,2,\dots,p$, it is defined by
	\begin{align}
	\left( \psi^* \omega \right)_{\tilde{x}}  (\tilde{v}_1,\tilde{v}_2,\dots,\tilde{v}_p) \coloneqq \omega_{\psi(\tilde{x})} (\psi_* \tilde{v}_1, \dots , \psi_* \tilde{v}_p) \formspace.
	\end{align}
\end{definition}
%
%
%
%
On the exterior algebra we find a duality, provided by the Hodge operator:
\begin{definition}[Hodge dual]
	The hodge star operator $\gls{hodge}: \Omega^p(\N) \to \Omega^{N-p}(\N)$ is defined through
	\begin{align}
	B \wedge *A = \frac{1}{p!} B^{\mu_1\dots\mu_p}A_{\mu_1\dots\mu_p} \dvolk \formspace,
	\end{align}
	which yields the coordinate representation
	\begin{align}
	(*A)_{\mu_{p+1}\dots\mu_N} = \frac{\detk}{p!} \, \epsilon_{\mu_1\dots\mu_N} A^{\mu_1\dots\mu_p} \formspace.
	\end{align}
\end{definition}
Here, \gls{levicivita} denotes the fully antisymmetric tensor of rank $N$ (Levi-Civita symbol) satisfying $\epsilon_{12,\dots,N} =1$ and the \emph{volume element} \gls{dvolk} is defined by
\begin{align}
\left( \gls{dvolk} \right)_{\alpha_1\dots\alpha_N} = \detk \, \epsilon_{\alpha_1\dots\alpha_N} \formspace.
\end{align}
In a sense, the volume element describes how the curvature of the manifold deforms a unit volume.
The duality follows from the important property of the Hodge operator as stated in the following lemma:
\begin{lemma}
	Let $*$ denote the Hodge star operator on the exterior algebra $\Omega(\N) $. It holds that
	\begin{align}
	** = (-1)^{s+p(N-p)} \, \mathbbm{1} \formspace,
	\end{align}
	which is trivially equivalent to $*^{-1} = (-1)^{s+p(N-p)} \, *$.
\end{lemma}
\begin{proof}
	Let $A \in \Omega^p(\N)$ be a $p$-form on $\N$. Then:
	\begin{align}
	(*{*A})_{\mu_1 \dots \mu_p}
	&= \frac{\detk \, \detk}{p! \, (N-p)!} \; \epsilon_{\alpha_{p+1}\dots\alpha_N \mu_1 \dots \mu_p}\;\epsilon^{\alpha_{1}\dots\alpha_N}\;A_{\alpha_1\dots\alpha_p} \notag\\
	&= (-1)^{p(N-p)} \frac{\detk \, \detk}{p! \, (N-p)!} \; \epsilon_{\alpha_{p+1}\dots\alpha_N \mu_1 \dots \mu_p}\;\epsilon^{\alpha_{p+1}\dots\alpha_{N}\alpha_1\dots\alpha_p}\;A_{\alpha_1\dots\alpha_p}  \notag\\
	&= (-1)^{s+p(N-p)} \delta\indices{^{[\alpha_{1}}_{\mu_{1}}}\, \dots \, \delta\indices{^{\alpha_p ] }_{\mu_p}} \;A_{\alpha_1\dots\alpha_p} \notag\\
	&=  (-1)^{s+p(N-p)}\;A_{\mu_1\dots\mu_p} \formspace
	\end{align}
	We have used Lemma \ref{lem:epsilon_contraction} and, in the last step, that the anti-symmetrization is absorbed by contraction because $A$ is antisymmetric.
\end{proof}
%
%
%
%
%
Furthermore, we can equip the exterior algebra with a differentiable structure, introducing the notion of the exterior derivative.
\begin{definition}[Exterior derivative]
	The \emph{exterior derivative} $\gls{d}:\Omega^p(\N) \to \Omega^{p+1} (\N)$ is defined by the following properties:
	\begin{enumerate}
		\item $d$ is linear
		\item $d$ obeys a graded Leibniz rule: Let $A \in \Omega^p(\N)$ and  $B\in \Omega^q(\N)$, then
		\begin{align}
		d(A \wedge B) = dA \wedge B + (-1)^p \, A \wedge dB
		\end{align}
		\item $d$ is nilpotent, that is,  $d^2 = 0$.
	\end{enumerate}
	In local coordinates, this is equivalent to the representation
	\begin{align}
	(dA)_{\mu \alpha_1\dots\alpha_p} = (p+1)\, \nabla_{[\mu}A_{\alpha_1\dots\alpha_p]} \formspace.
	\end{align}
\end{definition}
An important property of the exterior derivative is that it commutes (or rather intertwines its action) with pullbacks (see \cite[Proposition 4.1.7]{rudolph_schmidt}).
A $p$-form $\omega \in \Omega^p(\N)$ is called \emph{exact} if there is a $(p-1)$-form $\alpha \in \Omega^{p-1}(\N)$ such that $\omega = d\alpha$. We call $\omega$ \emph{closed} if $d \omega =0$. Accordingly, the space of closed $p$-forms is denoted by \gls{omegapd}, the space of exact ones by \gls{domegap}. As usual, the ones with compact support are denoted by a subscript zero. Note, that every exact form is closed, using that $d$ is by definition nilpotent, but the reverse is in general not true. It does hold, however, on certain manifolds with trivial topology, such as Minkowski spacetime. This is expressed in the so called Poincar\'e-Lemma (see for example \cite[Chapter 4]{bott_tu}) based on the study of de Rham cohomology.\par
%
Moreover, $N$-forms can naturally be integrated. Using local coordinates and a partition of unity, we define the integral of $N$-forms via the well known integration on $\IR^N$:
\begin{definition}[Integration on manifolds]
	Let $\left\{U_\alpha, \psi_\alpha\right\}_\alpha$ be an atlas of the manifold $\N$ and $\left\{\chi_\alpha\right\}_\alpha$ a partition of unity subordinate to the locally finite open cover $\left\{U_\alpha\right\}_\alpha$. Let $x^\mu_{(\alpha)}$ be a coordinate basis of $\psi$ on $U_\alpha$. For any $N$-form $\omega \in \Omega^N_0(\M)$ we define the integral
	\begin{align}
	\int\limits_{\N} \omega &\coloneqq \sum_{\alpha} \int\limits_{\psi_\alpha (U_\alpha)} w(x_{(\alpha)}^0,\dots,x_{(\alpha)}^1)\; dx_{(\alpha)}^0 \cdots dx_{(\alpha)}^{N-1} \formspace,
	\end{align}
	where $w$ are the components of $\omega$ in the coordinates $x_{(\alpha)}^\mu$, that is $\omega = w dx_{(\alpha)}^0 \wedge \cdots \wedge dx_{(\alpha)}^{N-1}$.
	This definition is independent of the choice of the atlas and the partition of unity (see \cite[Proposition 3.3]{bott_tu}).
\end{definition}
With integration at our disposal, we present an important theorem regarding the integration of exact differential forms:
\begin{theorem}[Stoke's Theorem]\label{thm:stokes}
	Let $\N$ be an oriented manifold of dimension $N$ and let its boundary $\partial \N$ be endowed with the induced orientation. Let $\gls{inclusionmap} : \partial \N \hookrightarrow \N$ be the inclusion operator.
	Let $\omega \in \Omega^{N-1}_0(\N)$ be a compactly supported $(N-1)$-form on $\N$. Then it holds
	\begin{align}
	\int\limits_\N d\omega = \int\limits_{\partial \N} i^*\omega \formspace.
	\end{align}
\end{theorem}
\begin{proof}
	A proof is given in most of the introductory literature on differential geometry (see for example \cite[Chapter 17, Theorem 2.1]{lang}).
	Note that one can equivalently formulate Stoke's theorem on a \emph{compact} manifold but for {arbitrary} (that is, in general not compactly supported) $(N-1)$-forms on the manifold (see for example \cite[Theorem 4.2.14]{rudolph_schmidt}). This will be of importance in later calculations.
\end{proof}
%
Furthermore, we can define a bilinear map on $\Omega^p(\N)$ using the integration of $N$-forms:
\begin{definition}
	Let $A,B \in \Omega^p(\N)$ such that their supports have a compact intersection. Define the bilinear map $\gls{innerprod} : \Omega^p(\N) \times \Omega^p(\N) \to \IC$ by
	\begin{align}
	\langle A, B \rangle_\N \coloneqq  \int_{\N } A \wedge * B = \int_{\N } A_{\mu_1 \dots \mu_p}B^{\mu_1 \dots \mu_p}\,\dvolk \formspace.
	\end{align}
\end{definition}
Since by definition $A \wedge * B$ is a compactly supported $N$-form, this is well defined. We may sometimes refer to $\langle \cdot , \cdot \rangle_\N$ as an inner product for simplicity, even though it is not positive definite.
%
%
%
%
%
Using the exterior derivative, we define the interior or co-derivative:
\begin{definition}[Interior derivative]
	The \emph{interior derivative} $\gls{delta} : \Omega^p(\N) \to \Omega^{p-1}(\N)$ is defined by
	\begin{align}
	\delta \coloneqq (-1)^{s+1+N(p-1)}\, {*{d*}} \formspace.
	\end{align}
	From the defining properties of $d$ and $*$ it follows $\delta^2 =0$.
\end{definition}
Here, $s$ again denotes the number of negative eigenvalues of the metric $k$ of $\N$. In accordance with our nomenclature, we call a $p$-form $\omega$ co-exact if there exists a $\alpha \in \Omega^{p+1}(\N)$ such that $\omega = \delta \alpha$ and co-closed if $\delta \omega = 0$. Accordingly, the spaces of co-closed and co-exact $p$-forms are denoted by \gls{omegapdelta} and \gls{deltaomegap} respectively.\par
Using the exterior and interior derivative we define the partial differential operator:
\begin{definition}[D'Alembert Operator]
	The d'Alembert (or Laplace - de Rham) operator $\gls{dalembert}: \Omega^p(\N) \to \Omega^{p}(\N)$ is defined by
	\begin{align}
	\square \coloneqq \delta d +d \delta \formspace.
	\end{align}
\end{definition}
By definition of the exterior and interior derivative, it is easy to show that $\square$ commutes with both $d$ and $\delta$:
\begin{align}
\square d &= (\delta d + d \delta )d \notag \\
&= d \delta d \notag \\
&= d (\delta d + d \delta) \formspace,
\end{align}
and analogously for $\delta$.
The d'Alembert operator, and its generalization to $(\square + m^2)$ for some constant $m > 0$, are important examples for a normally hyperbolic differential operators (see Section \ref{sec:global_hyperbolicity}) and we may therefore sometimes just refer to them as \emph{wave operators}.\par
The sign convention in the definition of the exterior derivative is chosen such that on any Lorentzian or Riemannian manifold the interior derivative is formally adjoint to the exterior derivative, that is,  for $A \in \Omega^{p}(\N)$ and $B \in \Omega^{p+1}(\N)$ it holds that
\begin{align}
\langle dA , B \rangle_{\N} = \langle A , \delta B \rangle_\N \formspace,
\end{align}
which leads to a representation in local coordinates of the Manifold given by:
\begin{align}
(\delta A)_{\mu_2\dots\mu_p} = - \nabla^{\mu_1}A_{\mu_1\dots\mu_p} \formspace.
\end{align}
To see that this is consistent, let $A \in \Omega^{p-1}(\N)$ and $B \in \Omega^{p}(\N)$ such that their supports have compact intersection.
We obtain, using Stoke's Theorem \ref{thm:stokes}:
\begin{align}
0 &= \int \limits_{\partial \N} i^* (A \wedge *B) \notag\\
&= \int \limits_{\N} d(A \wedge *B)  \notag\\
&= \int \limits_{\N} dA \wedge *B + (-1)^{p-1} A \wedge d{*B} \notag\\
&= \int \limits_{\N} dA \wedge *B + (-1)^{p-1} A \wedge *{*^{-1}}\underbrace{d{*B}}_{\textrm{is a } (N-p+1) \textrm{ form.}} \notag\\
&= \int \limits_{\N} dA \wedge *B + (-1)^{p-1}(-1)^{s+(N-p+1)(N-N+p-1)} A \wedge *{*d{*B}} \notag\\
&= \int \limits_{\N} dA \wedge *B + (-1)^{p+(1-p)(p-1)} A \wedge *\delta B \formspace.
\end{align}
It can easily be proven by induction that $\big(p+(1-p)(p-1)\big)$ is odd for any $p \in \IN$, which yields the result
\begin{align}
\langle dA , B \rangle_{\N} = \langle A , \delta B \rangle_\N \formspace.
\end{align}
The definitions stated above thus fulfill the requirement of formal adjointness of the exterior and interior derivate on an arbitrary Lorentzian or Riemannian manifold $\N$.
In local coordinates we use a partial integration to obtain
\begin{align}
\langle dA , B \rangle_\N &= \int \limits_{\N} dA \wedge * B \notag\\
%&= \int \limits_{\N} \frac{1}{p!} (dA)^{\alpha_1\dots\alpha_p}\,B_{\alpha_1 \dots \alpha_p} \, \dvolk \notag\\
&= \int \limits_{\N}  \frac{p}{p!} \nabla^{[\alpha_1}A^{\alpha_2\dots\alpha_p]}\,B_{\alpha_1 \dots \alpha_p} \, \dvolk \notag\\
&= \int \limits_{\N}  \frac{1}{(p-1)!} \nabla^{\alpha_1}A^{\alpha_2\dots\alpha_p}\,B_{\alpha_1 \dots \alpha_p} \, \dvolk \notag\\
&= - \int \limits_{\N}  \frac{1}{(p-1)!} A^{\alpha_2\dots\alpha_p}\, \nabla^{\alpha_1}B_{\alpha_1 \dots \alpha_p} \, \dvolk \notag\\
&= \langle A, \delta B \rangle_\N \formspace,
\end{align}
which yields
\begin{align}
-\nabla^{\alpha_1}B_{\alpha_1 \dots \alpha p} = (\delta B)_{\alpha_2 \dots \alpha_p}\formspace.
\end{align}
On the four dimensional spacetime $(\M,g)$ the definitions of the Hodge star operator and the interior derivative simplify, such that
\begin{align}
*_{(\M)}*_{(\M)} &= (-1)^{p+1} \mathbbm{1} \\
\delta_{(\M)} &= *_{(\M)}{d_{(\M)}*_{(\M)}} \formspace ,
\end{align}
holds on the spacetime $(\M,g)$ and
\begin{align}
*_{(\Sigma)}*_{(\Sigma)} &= \mathbbm{1} \\
\delta_{(\Sigma)} &= (-1)^p *_{(\Sigma)}{d_{(\Sigma)}*_{(\Sigma)}}
\end{align}
holds on  $(\Sigma,h)$. In the following we will drop the subscript ${(\M)}$, since we will perform all the calculations on a four dimensional spacetime, except when explicitly noted (for example with a subscript $(\Sigma)$).
%
%
%
%
%
%
%
%
%%%%%%
%%CATEGORY THEORY
%%%%%%
\subsection{Category theory}\label{sec:cat-theory}
The description of Quantum Field Theory on Curved Spacetimes (QFTCS) in the framework of \name{Brunetti}, \name{Fredenhagen} and \name{Verch} \cite{Brunetti_Fredenhagen_Verch} is based on category theory. In this thesis, we will not go into detail on those categorical aspects, however we will need some basic definitions to formulate the theory rigorously, that is namely the notion of a category and that of covariant functors, since, in the used framework, the generally covariant QFTCS is a functor.\par
Here, we present definitions given in \cite[Appendix A.1]{baer_ginoux_pfaeffle} and refer to the appropriate literature for details. We define:
\begin{definition}[Category]
	A \emph{category} $\mathsf{Cat}$ consists of the following:
	\begin{enumerate}
		\item a class $\mathsf{Obj}_\mathsf{Cat}$ whose members are called \emph{objects},
		\item a set $\mathsf{Mor}_\mathsf{Cat}(A,B)$, for any two objects $A,B \in \mathsf{Obj}_\mathsf{Cat}$, whose elements are called \emph{morphisms},
		\item for any three objects $A,B,C \in \mathsf{Obj}_\mathsf{Cat}$ there is a map
		\begin{align}
\mathsf{Mor}_\mathsf{Cat}(B,C) \times \mathsf{Mor}_\mathsf{Cat}(A,B) &\to \mathsf{Mor}_\mathsf{Cat}(A,C) \notag\\
(\psi,\phi) &\mapsto \psi \comp \phi
		\end{align}
		called the composition of morphisms subject to the relations:\vspace{4mm}
		\begin{enumerate}[label=(\arabic*)]
			\item for non equal pairs $(A,B)$, $(A',B')$ of objects, the sets $\mathsf{Mor}_\mathsf{Cat}(A,B)$ and $\mathsf{Mor}_\mathsf{Cat}(A',B')$ are disjoint,
			\item for every object $A$ there exists a morphism $\text{id}_A \in \mathsf{Mor}_\mathsf{Cat}(A,A)$ such that it holds for all objects $B$, morphisms $\psi \in \mathsf{Mor}_\mathsf{Cat}(B,A)$ and $\phi \in \mathsf{Mor}_\mathsf{Cat}(A,B)$
			\begin{align}
				\text{id}_A \comp \psi &= \psi \quad \text{and}\\
				\phi \comp \text{id}_A &= \phi \quad,
			\end{align}
			\item the composition law is associative, that is for an objects $A,B,C,D$ and any morphisms $\psi \in \mathsf{Mor}_\mathsf{Cat}(A,B)$, $\phi \in \mathsf{Mor}_\mathsf{Cat}(B,C)$ and $\chi \in \mathsf{Mor}_\mathsf{Cat}(C,D)$ it holds
			\begin{align}
				(\chi \comp \phi) \comp \psi = \chi \comp (\phi \comp \psi) \formspace.
			\end{align}
		\end{enumerate}
	\end{enumerate}
\end{definition}
%
%
%
\begin{definition}[Functor]
	Let $\mathsf{Cat1}$ and $\mathsf{Cat2}$ be categories. A \emph{covariant functor} $\mathscr{A}: \mathsf{Cat1} \to \mathsf{Cat2}$ consists of the map $\mathscr{A} : \mathsf{Obj}_\mathsf{Cat1} \to \mathsf{Obj}_\mathsf{Cat2}$ and maps $\mathscr{A}: \mathsf{Mor}_\mathsf{Cat1}(A,B) \to \mathsf{Mor}_\mathsf{Cat2}\big(\mathscr{A}(A),\mathscr{A}(B)\big)$ for any two objects $A,B \in \mathsf{Obj}_\mathsf{Cat1}$ such that
	\begin{enumerate}
		\item {the composition is preserved, that is for all objects $A,B,C \in \mathsf{Obj}_\mathsf{Cat1}$ and for any morphisms $\psi \in \mathsf{Mor}_\mathsf{Cat1}(A,B)$ and $\phi \in \mathsf{Mor}_\mathsf{Cat1}(B,C)$ it holds
		\begin{align}
			\mathscr{A}(\phi \comp \psi) = \mathscr{A}(\phi) \comp \mathscr{A}(\psi) \formspace,
		\end{align}}
		\item{
			$\mathscr{A}$ maps identities to identities, that is for any object $A \in \mathsf{Obj}_\mathsf{Cat1}$ it holds
			\begin{align}
				\mathscr{A}(\text{id}_\mathsf{A}) = \text{id}_{\mathscr{A}(A)} \formspace.
			\end{align}
			}
	\end{enumerate}
\end{definition}
%
%
%
%
%
%
%
%
%
%
%
%
%%%%%%
%%SIGN CONVENTIONS
%%%%%%
%
%
\subsection{Sign conventions}\label{sec:sign_conventions}
At certain points throughout this chapter we have had a freedom of choice regarding the signs of some entities, in particular the sign of the signature of the Lorentzian metric $g$ and that of the interior derivative $\delta$. Though at this stage the choice can be made arbitrarily, we want to make it in a way that in the end allows us to make certain physical interpretations on some parameters. More precisely, we want to interpret the parameter $m$ of the Klein-Gordon equation\footnote{or its generalization on $p$-forms} $(\square + m^2) f = 0$ for a zero-form $f \in \Omega^0(\M)$ as a mass in the physical sense. With the chosen sign convention for $\delta$ we find, using ${\delta}f = 0$:
\begin{align}
	\square f
	&= (\delta d + d \delta) f \notag\\
	&= \delta d f \notag\\
	&= - \nabla^\mu \nabla_\mu f \formspace.
\end{align}
In the following heuristic (local) argument we see
\begin{align}
	\square + m^2
	&= -\nabla^\mu \nabla_\mu + m^2 \notag\\
	&\sim \partial_t^2 + \sum_i \partial_i^2 + m^2\notag\\
	&\sim -E^2 + \abs{\vector{p}}^2 + m^2
\end{align}
which yields the correct relativistic relation of energy, momentum and mass according to $E^2 = \abs{\vector{p}}^2 + m^2$.
A similar calculation holds for the Klein-Gordon operator generalized to act on one-forms. If we had found a ``wrong'' relation between energy, momentum and mass, we would have had to adapt the chosen signs. Usually one chooses the sign of the metric and the interior derivative such that they are in some sense mathematically convenient (although one might disagree with another one's choice). We have made the choice of the metric, such that the Cauchy surfaces become Riemannian rather that ``anti-Riemannian'' (with an all minus signature), which seems more natural to some. Also, a lot of the used references on spacetime geometry (in particular the book by \name{Wald} \cite{wald_GR}) use this sign convention, which makes the application of certain formulas easier. As mentioned, the sign of the interior derivative was chosen such that it is formally adjoint to the exterior derivative (with respect the specified inner product) on all Lorentzian and Riemannian manifolds. It seemed convenient for the actual calculations to fix the sign regardless of the signature of the metric of the underlying manifold. One could equivalently have fixed the opposite sign, yielding the two derivatives to be skew-adjoint, which is also done in the literature. However, in the end, one has one freedom left to make the energy-momentum-mass relation work: that is the sign in front of the mass in the Klein-Gordon equation and all other wave equations accordingly. Hence, one regularly also finds the Klein-Gordon equation to be defined with a flipped sign of the mass term. But for our case, we want the mass $m$ in any wave equation to appear with a positive sign.
%
%

\section{Connecting Choice Modeling and Matrix Balancing}
\label{sec:equivalence}

In this section, we formally establish the connection between choice modeling and matrix balancing. We show that
maximizing the log-likelihood \eqref{eq:log-likelihood} is equivalent to solving a canonical matrix balancing problem. We also precisely describe the correspondence between the relevant conditions in the two problems. In view of this equivalence, we show that Sinkhorn's algorithm, when applied to estimate Luce choice models, is in fact a \emph{parallelized} generalization of the classic iterative algorithm for choice models, dating back to \citet{zermelo1929berechnung,dykstra1956note,ford1957solution}, and studied extensively also by \citet{hunter2004mm,vojnovic2020convergence}.

\subsection{Maximum Likelihood Estimation of Luce Choice Model as Matrix Balancing}
\label{subsec:reformulation}
The optimality conditions for maximizing the log-likelihood \eqref{eq:log-likelihood} for each $s_j$ are given by
\begin{align*}
\partial_{s_{j}}\ell(s)=\sum_{j\mid (j,S_i)}\frac{1}{s_{j}}-\sum_{i\mid j\in S_{i}}\frac{1}{\sum_{k\in S_{i}}s_{k}} & =0.
\end{align*}
 Multiplying by $s_{j}$ and dividing by $1/n$, we have 
\begin{align}
\label{eq:optimality-original}
\frac{W_j}{n} & = \frac{1}{n} \sum_{i\mid j\in S_{i}}\frac{s_{j}}{\sum_{k\in S_{i}}s_{k}},
\end{align}
where $W_j:=|\{i\mid (j,S_i)\}|$ is the number of observations where $j$ is selected.

Note that in the special case where $S_i\equiv [n]$, i.e., every choice set contains \emph{all} items, the MLE simply reduces to the familiar empirical frequencies $\hat s_j = {W_j}/{n}$. However, when the choice sets $S_i$ vary, no closed form solution to \eqref{eq:optimality-original} exists, which is the primary motivation behind the long line of works on the algorithmic problem of solving \eqref{eq:optimality-original}. 

With varying $S_i$, we can interpret the optimality condition as requiring the \emph{observed} frequency of $j$ being chosen (left hand side) be equal to the conditional \emph{expected} probability of $j$ being chosen among all observations $i$ where it is part of the choice set $S_i$ (right hand side). In addition, note that since the optimality condition in \eqref{eq:optimality-original} only involves the \emph{frequency} of selection, distinct datasets could yield the same optimality condition and hence the same MLEs. For example, suppose that two alternatives $j$ and $k$ both appear in choice sets
$S_{i}$ and $S_{i'}$, with $j$ selected in $S_{i}$ and  $k$ selected in $S_{i'}$. Then switching
the choices in $S_{i}$ and $S_{i'}$ does not alter the likelihood and optimality conditions. This feature holds more generally with longer cycles of items and choice sets, and can be viewed as a consequence of the context-independent nature of Luce's choice axiom. In some sense, it is also the underpinning of many works in economics that estimate choice models based on \emph{marginal} sufficient statistics. A famous example is  \citet{berry1995automobile}, which estimates consumer preferences using only \emph{aggregate} market shares of products.

In practice, the choice sets $S_i$ of many observations may be identical to each other. Because \eqref{eq:optimality-original} only depends on the total winning counts of items, we may aggregate over observations with the same $S_i$:
\begin{align*}
\sum_{i\mid j\in S_{i}}\frac{s_{j}}{\sum_{k\in S_{i}}s_{k}} & = \sum_{i'\mid j\in S^\ast_{i'}} R_{i'} \cdot \frac{s_{j}}{\sum_{k\in S^\ast_{i'}}s_{k}},
\end{align*}
where each $S_{i'}^\ast$ is a unique choice set that appears in $R_{i'}\geq 1$ observations, for $i'=1,\dots,n^\ast \leq n$. By construction, $\sum_{i'=1}^{n^\ast}R_{i'} =n$. Note, however, that the selected item could vary across different appearances of $S_i^\ast$, but the optimality conditions only involve each item's winning count $W_j$. From now on, we assume this reduction and drop the $^\ast$ superscript. In other words, we assume that we observe $n$ unique choice sets, and choice set $S_i$ has \emph{multiplicity} $R_i$. The resulting problem has optimality conditions
\begin{align}
\label{eq:optimality}
{W_j} & = \sum_{i\mid j\in S_{i}}{R_i} \cdot \frac{s_{j}}{\sum_{k\in S_{i}}s_{k}}.
\end{align}
We are now ready to reformulate \eqref{eq:optimality} as a canonical matrix balancing problem. Define $p\in\mathbb{R}^{n}$ as $p_i=R_i$, i.e., the number of times choice set $S_i$ appears in the data. Define
$q\in\mathbb{R}^{m}$ as $q_j={W_j}$,
i.e., the number of times item $j$ was \emph{selected} in the data. By construction we have $\sum_i p_i=\sum_j q_j$, and $p_i,q_j>0$ whenever \cref{ass:strong-connected} holds.

Now define the $n\times m$ binary matrix $A$ by
$A_{ij}=1\{j\in S_{i}\}$, so the $i$-th row of $A$
is the indicator of which items appear in the (unique) choice set $S_i$, and the $j$-th column of $A$ is the indicator of which choice sets
item $j$ appears in. We refer to this $A$ constructed from a choice dataset as the \emph{participation matrix}. By construction, $A$ has distinct rows, but may still have identical columns. If necessary, we can also remove repeated columns by ``merging'' items and their win counts. Their estimated scores can be computed from the score of the merged item proportional to their respective win counts.

Let $D^{0}\in\mathbb{R}^{m\times m}$ be the diagonal matrix with
$D_{j}^{0}=s_{j}$ and $D^{1}\in\mathbb{R}^{n\times n}$ be the
diagonal matrix with $D_{i}^{1}={R_i}/{\sum_{k\in S_{i}}s_{k}}$,
and define the scaled matrix
\begin{align}
\label{eq:scaled-matrix}
\hat{A} & :=D^{1}AD^{0}.
\end{align}
The matrices $D^{1}$ and $D^{0}$ are scalings of rows and columns
of $A$, respectively, and
\begin{align*}
    \hat{A}_{ij} = \frac{R_i}{\sum_{k\in S_{i}}s_{k}}\cdot1\{j\in S_{i}\}\cdot s_{j}.
\end{align*}
The key observation is that the optimality condition \eqref{eq:optimality} can be rewritten as
\begin{align}
\label{eq:bridge}
\hat{A}^T \mathbf{1}_n & = q.
\end{align}
Moreover, by construction $\hat{A}$ also satisfies
\begin{align}
\label{eq:marginal}
\hat{A}\mathbf{1}_m & = p.
\end{align}
 Therefore, if $s_j$'s satisfy the optimality conditions for maximizing \eqref{eq:log-likelihood}, then $D^0,D^1$  defined above solve the matrix balancing problem in \cref{eq:scaled-matrix,eq:bridge,eq:marginal}. Moreover, the converse is also true, and we thus establish the equivalence between choice maximum likelihood estimation and matrix balancing. All omitted proofs appear in \Cref{app:proofs}.
\begin{theorem}
\label{prop:mle-scaling}
Let $p\in\mathbb{R}^{n}$ with $p_{i}=R_i$, $q\in\mathbb{R}^{m}$ with $q_{j}=W_j$, and 
$A\in\mathbb{R}^{n\times m}$
with $A_{ij}=1\{j\in S_{i}\}$ be constructed from the choice dataset. Then  $D^{0},D^{1}>0$ with $\sum_j D_j^0=1$ solves the matrix balancing problem
\begin{align}
\label{eq:equation-system}
\begin{split}
\hat{A} & =D^{1}AD^{0}\\
\hat{A}\mathbf{1}_m & =p\\
\hat{A}^{T} \mathbf{1}_n & =q
\end{split}
\end{align}
if and only if $s \in \Delta_m$ with $s_j=D^0_j$ satisfies the optimality condition \eqref{eq:optimality} of the ML estimation problem.
\end{theorem}
In particular, \eqref{eq:log-likelihood} has a unique maximizer $s$ in the interior of the probability simplex if and only if \eqref{eq:equation-system} has a unique normalized solution $D^0$ as well. The next question, naturally, is then how \cref{ass:strong-connected} and \cref{ass:weak-connected} for choice modeling are connected to \cref{ass:matrix-existence} and \cref{ass:matrix-uniqueness} for matrix balancing.
\begin{theorem}
\label{thm:necessary-and-sufficient}
Let (A,p,q) be constructed from the choice dataset as in \cref{prop:mle-scaling}, with $p,q>0$. \cref{ass:weak-connected} is equivalent to \cref{ass:matrix-uniqueness}. Furthermore, \cref{ass:strong-connected} holds if and only if $(A,p,q)$ satisfy \cref{ass:matrix-existence} and $A$ satisfies \cref{ass:matrix-uniqueness}.
\end{theorem} 
Thus when the ML estimation problem is cast as a matrix balancing problem, \cref{ass:matrix-existence} exactly characterizes the \emph{gap} between \cref{ass:weak-connected} and \cref{ass:strong-connected}. 
 We provide some intuition for \cref{thm:necessary-and-sufficient}. When we construct a triplet $(A,p,q)$ from a choice dataset, with $p$ the numbers of appearances of unique choice sets and $q$ the winning counts, \cref{ass:matrix-uniqueness} precludes the possibility of partitioning the items into two subsets that never get compared with each other, i.e., \cref{ass:weak-connected}. Then \cref{ass:matrix-existence} requires that whenever a strict subset $M\subsetneq [m]$ of objects only appear in a strict subset $N\subsetneq [n]$ of the observations, their total winning counts are \emph{strictly} smaller than the total number of these observations, i.e., there is some object $j\notin M$ that is selected in $S_i$ for some $i\in N$, which is required by \cref{ass:strong-connected}.

Interestingly, while \cref{ass:strong-connected} requires the directed comparison graph, defined by the $m\times m$ matrix of counts of item $j$ being chosen over item $k$, to be strongly connected, the corresponding conditions for the equivalent matrix balancing problem concern the $n\times m$ participation matrix $A$ and positive vectors $p,q$, which do not explicitly encode the specific \emph{choice} of each observation. This apparent discrepancy is due to the fact that $(A,p,q)$ form the \emph{sufficient statistics} of the Luce choice model. In other words, there can be more than one choice dataset with the same optimality condition \eqref{eq:optimality} and $(A,p,q)$ defining the equivalent matrix balancing problem.

\textbf{Remark.} This feature where ``marginal'' quantities constitute the sufficient statistics of a parametric model is an important one that underlies many works in economics and statistics \citep{kullback1997information,stone1962multiple,good1963maximum,birch1963maximum,theil1967economics,fienberg1970iterative,berry1995automobile,fofana2002balancing,maystre2017choicerank}. It makes the task of estimating a \emph{joint} model from marginal quantities feasible. This feature is useful because in many applications, only marginal data is available due to high sampling cost or privacy reasons. 

Having formulated a particular matrix balancing problem from the estimation problem given choice data, we may ask how one can go in the other direction. In other words, when/how can we construct a ``choice dataset'' whose sufficient statistics is a given triplet $(A,p,q)$? First off, for $(A,p,q)$ to be valid sufficient statistics of a Luce choice model, $p,q$ need to be positive integers. Moreover, $A$ has to be a binary matrix, with each row containing at least two non-zero elements (valid choice sets have at least two items). Given such a $(A,p,q)$ satisfying Assumptions \ref{ass:matrix-existence} and \ref{ass:matrix-uniqueness}, a choice dataset can be constructed efficiently. Such a procedure is described, for example, in \citet{kumar2015inverting}, where $A$ is motivated
by random walks on a graph instead of matrix balancing (\cref{sec:connections}). Their construction relies on finding the max flow on the bipartite graph $G_b$. For rational $p,q$, this maximum flow can be found efficiently in polynomial time \citep{balakrishnan2004polynomial,idel2016review}. Moreover, the maximum flow implies a matrix $A'$ satisfying \cref{ass:matrix-existence}(a), thus providing a feasibility certificate for the matrix balancing problem as well. 

We have thus closed the loop and fully established the equivalence of the maximum likelihood estimation of Luce choice models and the canonical matrix balancing problem.
\begin{corollary}
    There is a one-to-one correspondence between classes of maximum likelihood estimation problems with the same optimality conditions \eqref{eq:optimality} and canonical matrix balancing problems with $(A,p,q)$, where $A$ is a valid participation matrix and $p,q>0$ have integer entries. 
\end{corollary}
We next turn our attention to the algorithmic connections between choice modeling and matrix balancing. 
\subsection{Algorithmic Connections between Matrix Balancing and Choice Modeling}
\label{subsec:IPF}
 Given the equivalence between matrix balancing and choice modeling, we can naturally consider applying Sinkhorn's algorithm to maximize \eqref{eq:log-likelihood}. In this case, one can verify that the updates in each full iteration of \cref{alg:scaling} reduce algebraically to
\begin{align}
\label{eq:scaling-iteration}
s_{j}^{(t+1)} & =W_{j}/\sum_{i\mid j\in S_{i}}\frac{R_i}{\sum_{k\in S_{i}}s_{k}^{(t)}}
\end{align}
 in the $t$-th iteration. 
 Comparing \eqref{eq:scaling-iteration} to the optimality condition in \eqref{eq:optimality}, which we recall is given by 
\begin{align*}
 W_j & =\sum_{i\mid j\in S_{i}} R_i\frac{s_{j}}{\sum_{k\in S_{i}}s_{k}}= s_{j} \cdot \sum_{i\mid j\in S_{i}}\frac{R_i}{\sum_{k\in S_{i}}s_{k}},
 \end{align*}
we can therefore interpret the iterations as simply dividing the winning count $W_j$ by the coefficient of $s_j$ on the right repeatedly, in the hope of converging to a \emph{fixed point}. A similar intuition was given by \citet{ford1957solution} in the special case of pairwise comparisons. Indeed, the algorithm proposed by \citet{ford1957solution} is a cyclic variant of \eqref{eq:scaling-iteration} applied to pairwise comparisons. However, this connection is mainly algebraic, as the optimality condition in \citet{ford1957solution} does not admit a reformulation as the matrix balancing problem in \eqref{eq:equation-system}.

In \cref{sec:connections}, we provide further discussions on the connections of Sinkhorn's algorithm to existing frameworks and algorithms in the choice modeling literature, and connect it to distributed optimization as well. We demonstrate that many existing algorithms for Luce choice model estimation are in fact special cases or analogs of Sinkhorn's algorithm. These connections also illustrate the many interpretations of Sinkhorn's algorithm, e.g., as a distributed optimization algorithm as well as a minorization-majorization (MM) algorithm. However, compared to most algorithms for choice modeling discussed in this work, Sinkhorn's algorithm is more general as it applies to non-binary $A$ and non-integer $p,q$, and has the critical advantage of being paralellized and distributed, hence more efficient in practice.

The mathematical and algorithmic connections between matrix balancing and choice modeling we establish in this paper allow the transfusion of ideas in both directions. For example, inspired by regularized maximum likelihood estimation \citep{maystre2017choicerank}, we propose a regularized version of Sinkhorn's algorithm in \cref{subsec:regularization}, which is guaranteed to converge even when the original Sinkhorn's algorithm does not converge. Moreover, the importance of algebraic connectivity in quantifying estimation and computation efficiency in choice modeling motivates us to solve some important open problems on the convergence of Sinkhorn's algorithm. We turn to this topic next. 
\section{Linear Convergence of Sinkhorn's Algorithm for Non-negative Matrices}
\label{sec:linear-convergence}
In this section, we turn our focus to matrix balancing and study the global and asymptotic linear convergence rates of Sinkhorn's algorithm for general non-negative matrices and positive marginals. We first discuss relevant quantities and important concepts before presenting the convergence results in \cref{subsec:global-linear-convergence} and \cref{subsec:sharp-rate}.
Throughout, we use superscript $(t)$ to denote quantities after $t$ iterations of Sinkhorn's algorithm.

\subsection{Preliminaries}
\begin{table*}
\caption{Summary of some convergence results on Sinkhorn's algorithm. In \citet{franklin1989scaling}, $\kappa(A)=\frac{\theta(A)^{1/2}-1}{\theta(A)^{1/2}+1}$,
where $\theta(A)$ is the diameter of $A$ in the Hilbert metric.
The norm in \citet{knight2008sinkhorn} is not explicitly specified, and $\sigma_{2}(\hat{A})$
denotes the second largest singular value of the scaled
doubly stochastic matrix $\hat{A}$. The bound in \citet{altschuler2017near} was
originally stated as $\|r^{(t)}-p\|_{1}\protect\leq\epsilon'$
in $t=O(\epsilon'^{-2}\log(\frac{\sum_{ij}A_{ij}}{\min_{ij}A_{ij}}))$
iterations. The result in \citet{leger2021gradient} applies more generally to couplings
of probability distributions. In view of Pinsker's inequality, it implies the bound in \citet{altschuler2017near} but with a constant that is finite even when $A$ has zero entries. In the bound in \citet{knight2008sinkhorn} and our
asymptotic result, the $\lambda+\epsilon$ denotes an asymptotic rate, with the
bound valid for any $\epsilon>0$ and all $t$ sufficiently large. In our global bound, the linear rate $\lambda_{-2}(\mathcal{L})$ is the second \emph{smallest} eigenvalue of the Laplacian of the bipartite graph defined by $A$ (see \cref{subsec:graph-laplacian}), $ l=\min \{\max_j (A^T\mathbf{1}_n)_j, \max_i (A\mathbf{1}_m)_i\}$,
$c_B=\exp(-4B)$, and $B$ is a bound on the initial sub-level set, which is finite if and only if \cref{ass:matrix-existence} holds.
}
 \begin{adjustwidth}{-1.5cm}{}
\begin{centering}
\begin{tabular}{c|c|c|c|c}
 & convergence statement & $\lambda$ & $A$ & $p,q$\tabularnewline
\hline 
\citet{franklin1989scaling} & $d_{\text{Hilbert}}(r^{(t)},p)\leq\lambda^t d_{\text{Hilbert}}(r^{(0)},p)$ & $\kappa^{2}(A)$ & $A>0$, rectangular & uniform\tabularnewline
\hline 
\citet{luo1992convergence} & $g(u^{(t)},v^{(t)})-g^\ast\leq\lambda^t (g(u^{(0)},v^{(0)})-g^\ast)$ & \text{unknown} & $A\geq0$, rectangular & general\tabularnewline
\hline 
\citet{knight2008sinkhorn} & $\|D_{t+1}^{0}-D^{0}\|_{\ast}\leq(\lambda+\epsilon)\|D_{t}^{0}-D^{0}\|_{\ast}$ & $\sigma_{2}^{2}(\hat{A})$ & $A\geq0$, square  & uniform\tabularnewline
\hline 
\citet{pukelsheim2009iterative} & $\|r^{(t)}-p\|_{1}\rightarrow0$ & no rate & $A\geq0$, rectangular & general\tabularnewline
\hline 
\citet{altschuler2017near} & $\|r^{(t)}-p\|_{1}\leq c \sqrt{\frac{\lambda}{t}}$ & $\log(\frac{\sum_{ij}A_{ij}}{\min_{ij}A_{ij}})$ & $A>0$, rectangular & general\tabularnewline
\hline 
\citet{leger2021gradient} & $D_{\text{KL}}(r^{(t)}\| p) \leq\frac{\lambda}{t}$ & $D_{\text{KL}}(\hat{A}\| A)$ & $A\geq0$, continuous & general\tabularnewline
\hline 
current work, asymptotic & $\|\frac{r^{(t+1)}}{\sqrt{p}}-\sqrt{p}\|_{2}\leq(\lambda+\epsilon)\|\frac{r^{(t)}}{\sqrt{p}}-\sqrt{p}\|_{2}$ & $\lambda_{2}(\tilde{A}^T\tilde{A})$ & $A\geq0$, rectangular & general\tabularnewline
\hline 
current work, global & $g(u^{(t)},v^{(t)})-g^\ast\leq\lambda^t (g(u^{(0)},v^{(0)})-g^\ast)$ & $1-c_B\lambda_{-2}(\mathcal{L})/l$ & $A\geq0$, rectangular & general\tabularnewline
\end{tabular}
\par\end{centering}
\label{tab:convergence-summary}
\end{adjustwidth}
\end{table*}
We start with the optimization principles associated with matrix balancing and Sinkhorn's algorithm. Consider the following KL divergence (relative entropy) minimization problem 
     \begin{align}
\label{eq:relative-entropy-minimization}
\begin{split}
  \min_{\hat A\in \mathbb{R}^{n\times m}_+} & D_{\text{KL}}(\hat{A}\| A)\\
\hat{A}\mathbf{1}_m & =p\\
\hat{A}^{T}\mathbf{1}_n & =q.
\end{split}
\end{align}
It is well-known that solutions $\hat{A}=D^1AD^0$ to the matrix balancing problem with $(A,p,q)$ are minimizers of \eqref{eq:relative-entropy-minimization} \citep{ireland1968contingency,bregman1967proof}. Moreover, Sinkhorn's algorithm can be interpreted as a block coordinate descent type algorithm applied to minimize the following dual problem of \eqref{eq:relative-entropy-minimization}:
 \begin{align}
 \label{eq:log-barrier}
     g(d^0,d^1)	:=(d^1)^{T}Ad^0-\sum_{i=1}^{n}p_{i}\log d^1_{i}-\sum_{j=1}^{m}q_{j}\log d^0_{j},
\end{align} 
\citet{luo1992convergence} study the linear convergence of block coordinate descent algorithms. Their result implies that the
convergence of Sinkhorn's algorithm, measured in terms of the optimality gap of $g$, is linear with some implicit rate $\lambda>0$,
 as long as finite positive scalings $D^0,D^1$ exist for the matrix balancing problem. Minimizers $d^0,d^1$ of \eqref{eq:log-barrier} precisely give the diagonals of $D^0,D^1$. The function $g$, known to be a \emph{potential function} of Sinkhorn's algorithm, also turns out to be crucial in quantifying the global linear convergence rate in the present work. 

\textbf{Remark.} Interestingly, minimizing \eqref{eq:log-barrier} is in fact equivalent to maximizing the log-likelihood function $\ell(s)$ in \eqref{eq:log-likelihood} for valid $(A,p,q)$, because $\min_{d^1}g(d^0,d^1)=-\ell(d^0)+c$ for some $c>0$. Moreover, the optimality condition of minimizing $g$ with respect to $d^0$ reduces to the optimality condition \eqref{eq:optimality}. A detailed discussion can be found in \cref{sec:sinkhorn-MM}. This connection relates choice modeling and matrix balancing from an optimization perspective.
     
      Although convergence results on Sinkhorn's algorithm are abundant, the recent work of \citet{leger2021gradient} stands out as the first \emph{explicit} global convergence result applicable to general non-negative matrices, with a sub-linear $\mathcal O(1/t)$ bound on the KL divergence with respect to target marginals. It implies the bounds in \citet{chakrabarty2021better,altschuler2017near} but with a constant that is finite even when $A$ has zero entries. The result in \citet{leger2021gradient} applies more generally to couplings of continuous probability distributions, but when restricted to the discrete matrix balancing problem, it 
     holds under the following equivalent conditions that are weaker than \cref{ass:matrix-existence}.
\begin{assumption}[\textbf{Weak Existence}]
    \label{ass:matrix-weak-existence}
    
    \textbf{(a)} There exists a non-negative matrix $A'\in \mathbb{R}_+^{n\times m}$ that inherits all zeros of $A$ and has row and column sums $p$ and $q$. Or, equivalently,
    
\textbf{(b)} For every pair of sets of indices $N \subsetneq [n]$ and $M \subsetneq [m]$ such that $A_{ij}=0$ for $i\notin N$ and $j\in M$, $\sum_{i\in N}p_i \geq \sum_{j\in M}q_j$.
\end{assumption}

The equivalence of these conditions follows from Theorem 4 in \citet{pukelsheim2009iterative}, which also shows that they are the minimal requirements for the convergence of Sinkhorn's algorithm. \cref{ass:matrix-weak-existence}(a) precisely guarantees 
that the optimization problem \eqref{eq:relative-entropy-minimization} is feasible and bounded. It relaxes \cref{ass:matrix-existence}(a) by allowing additional zeros in the matrix $A'$. Similarly, \cref{ass:matrix-weak-existence}(b) relaxes \cref{ass:matrix-existence}(b) by allowing equality between $\sum_{i\in N}p_i$ and $\sum_{j\in M}q_j$ even when $M,N$ do not correspond to a block-diagonal structure. 

The distinction between \cref{ass:matrix-existence} and \cref{ass:matrix-weak-existence} is crucial for the matrix balancing problem and Sinkhorn's algorithm. Recall that \cref{ass:matrix-existence} guarantees the matrix balancing problem has a solution $(D^0,D^1)$, and $D^1AD^0$ is always a solution to \eqref{eq:relative-entropy-minimization}. On the other hand, the weaker condition \cref{ass:matrix-weak-existence} guarantees that \eqref{eq:relative-entropy-minimization} has a solution $\hat A$.
If indeed $\hat A$ has additional zeros relative to $A$, then no direct (finite and positive) scaling $(D^{0},D^{1})$ exists such that $\hat A=D^1AD^0$. However, the sequence of scaled matrices $\hat A^{(t)}$ from Sinkhorn's algorithm still converges to $\hat A$. In this case, the matrix balancing problem is said to have a \emph{limit} scaling, where some entries of $d^{0},d^{1}$ in Sinkhorn iterations approach 0 or $\infty$, resulting in additional zeros in $\hat A$. Below we give an example adapted from \citet{pukelsheim2009iterative}, where $p,q=(3,3)$ and the scaled matrices $\hat A^{(t)}$ converge but no direct scaling exists:
\begin{align*}
{D^{1}}^{(t)}\begin{bmatrix}3&1\\
0 & 2
\end{bmatrix}{D^{0}}^{(t)} = \begin{bmatrix}1&0\\
0 & \frac{3t}{2}
\end{bmatrix}\begin{bmatrix}3&1\\
0 & 2
\end{bmatrix}\begin{bmatrix}1&0\\
0 & \frac{1}{t}
\end{bmatrix}	\rightarrow \begin{bmatrix}3&0\\
0 & 3
\end{bmatrix}.
\end{align*}

Given these discussions, it is therefore important to clarify the convergence behaviors of Sinkhorn's algorithm in different regimes. In particular, it remains to reconcile the gap between the implicit linear convergence result of \citet{luo1992convergence} under strong existence, and the quantitative sub-linear bound of \citet{leger2021gradient} under weak existence. Furthermore, it remains to provide explicit characterizations of both the global and asymptotic (local) rates when Sinkhorn's algorithm does converge linearly. 

Our results in this section provide answers to these questions. We show that the $\mathcal O(1/t)$ rate can be sharpened to a global $\mathcal O(\lambda^t)$ bound if and only if the weak existence condition (\cref{ass:matrix-weak-existence}) is replaced by the strong existence condition (\cref{ass:matrix-existence}). Moreover, we provide an explicit global linear convergence rate $\lambda$ in terms of the \emph{algebraic connectivity}, revealing the structure-dependent nature of Sinkhorn's algorithm for problems with non-negative matrices. This generalizes the implicit result of \citet{luo1992convergence} and sheds light on how different assumptions impact Sinkhorn's convergence, which is explicitly reflected in the constants of the bound. Going further, we characterize the sharp asymptotic rate of linear convergence in terms of the second largest singular value of $\mathcal{D}(1/\sqrt{p})\cdot\hat{A}\cdot\mathcal{D}(1/\sqrt{q})$, where $\mathcal{D}$ denotes the diagonalization of a vector. This asymptotic rate reduces to that given by \citet{knight2008sinkhorn} for $m=n$ and uniform $p,q$.

The choice of convergence measure is important, and previous works have used different convergence measures. First note that after each iteration in \cref{alg:scaling}, the column constraint is always satisfied: ${A}^{(t)}\mathbf{1}_n=q$, where ${A}^{(t)}$ is the scaled matrix after $t$ iterations. 
\citet{leger2021gradient} uses the KL divergence $D_{\text{KL}}(r^{(t)}\| p)$ between the row sum $r^{(t)}={A}^{(t)}\mathbf{1}_m$ and the target row sum $p$ to measure convergence. \citet{franklin1989scaling} use the Hilbert projective metric between $r^{(t)}$ and $p$. \citet{pukelsheim2009iterative} and \citet{altschuler2017near} use the $\ell^1$ distance, which is upper bounded by the KL divergence via Pinsker's inequality. \citet{knight2008sinkhorn} focuses on the convergence of the scaling diagonal matrix $D^0=\mathcal{D}(d^0)$ to the optimal solution \emph{line}, but does not explicitly specify the norm.
Some bounds are \emph{a priori} and hold globally for all iterations, while others hold locally in a neighborhood of the optimum.
We summarize the relevant convergence results in \cref{tab:convergence-summary}. Here $\lambda_{-2}(S)$ denotes the second smallest eigenvalue of a real symmetric matrix $S$, and $\lambda_{2}(S)$ the second largest eigenvalue. In our work, we characterize the global linear convergence through the optimality gap of \eqref{eq:log-barrier}, which naturally leads to a bound on $\|r^{(t)}-p\|_1$. For the sharp asymptotic rate, we choose to use the $\ell^2$ distance $\|r^{(t)}/\sqrt{p}-\sqrt{p}\|_2$ in order to exploit an intrinsic orthogonality structure afforded by Sinkhorn's algorithm. This approach results in a novel analysis compared to \citet{knight2008sinkhorn} that most explicitly reveals the importance of spectral properties in  the rate of convergence.

\subsection{Global Linear Convergence}
\label{subsec:global-linear-convergence}
We first present the global linear convergence results. Our analysis starts with the following change of variables to transform the potential function \eqref{eq:log-barrier}:
\begin{align}
\label{eq:change-of-variables}
    u:=\log d^0,\quad v:=-\log d^1.
\end{align}
This results in the potential function $g(u,v)$ defined as
\begin{align}
\label{eq:transformed-potential}
  g(u,v)	:=\sum_{ij}A_{ij}e^{-v_{i}+u_{j}}+\sum_{i=1}^{n}p_{i}v_{i}-\sum_{j=1}^{m}q_{j}u_{j},
\end{align}
and we can verify that Sinkhorn's algorithm is equivalent to the alternating minimization algorithm \citep{bertsekas1997nonlinear,beck2013convergence} for \eqref{eq:transformed-potential}, which alternates between minimizing with respect to $u$ and $v$, holding the other block fixed:
\begin{align}
   \label{eq:alternating-minimization} u_j^{(t)}\leftarrow  \log \frac{q_j}{\sum_i A_{ij}e^{-v^{(t-1)}_i}},\quad v_i^{(t)}\leftarrow  \log \frac{p_i}{\sum_j A_{ij}e^{u^{(t)}_j}}.
\end{align}
The Hessian $\nabla^2g(u,v)$ always has $\mathbf{1}_{m+n}$ in its null space. On the surface, standard linear convergence results for first-order methods, which require strong convexity (or related properties like the Polyak--Lojasiewicz condition) of the objective function, do not apply to $g(u,v)$. However, we show that under strong existence and uniqueness conditions for the matrix balancing problem, $g(u,v)$ is in fact strongly convex when \emph{restricted} to the subspace 
\begin{align*}
   \mathbf{1}_{m+n}^\perp:= \{u\in \mathbb{R}^m,v\in \mathbb{R}^n:(u,v)^T\mathbf{1}_{m+n}=0\}.
\end{align*}
As a result, Sinkhorn's algorithm converges linearly with a rate that depends on the (restricted) condition number of its Hessian.

Before proceeding, we introduce a slew of useful definitions. Let Sinkhorn's algorithm initialize with $u^{(0)}$, and $v^{(0)}$ given by \eqref{eq:alternating-minimization}. 
Define the constant $B$ as 
\begin{align*}
   B:&= \sup_{(u,v)} \|(u,v)\|_\infty\\  \text{subject to } (u,v)^{T}&\mathbf{1}_{m+n}=0,\\
   g(u,v)&\leq g(u^{(0)},v^{(0)}).
\end{align*}
In other words, $B$ is the \emph{diameter} of the initial normalized sub-level set. We will show that $B$ is finite and that it bounds normalized Sinkhorn iterates by the \emph{coercivity} of $g(u,v)$ under \cref{ass:matrix-existence}.  
We similarly define the normalized optimal solution pair 
\begin{align}
\label{eq:normalized-optimum}
    (u^\ast,v^\ast):=\arg \min_{(u,v)\in \mathbf{1}_{m+n}^\perp} g(u,v),
\end{align}
and $g^\ast:=g(u^\ast,v^\ast)$.
Finally, define 
\begin{align*}
    l_0:= \max_j (A^T\mathbf{1}_n)_j, \quad l_1:= \max_i (A\mathbf{1}_m)_i,
\end{align*}
which are the Lipschitz constants of the two sub-blocks of $g(u,v)$. Next, define the \emph{Laplcian} matrix $\mathcal{L}$ of the bipartite graph $G_b$ as 
\begin{align*}
\mathcal{L}:&=\begin{bmatrix}\mathcal{D}({A}\mathbf{1}_{m}) & -{A}\\
-{A}^{T} & \mathcal{D}({A}^{T}\mathbf{1}_{n})
\end{bmatrix}, 
\end{align*}
and refer to the second smallest eigenvalue $\lambda_{-2}(\mathcal{L})$ as the Fiedler eigenvalue. For details on the graph Laplacian and the Fielder eigenvalue, see \cref{subsec:graph-laplacian}.

Using the above notation, we can now state one of our main contributions to the study of Sinkhorn's algorithm. 
\begin{theorem}[\textbf{Global Linear Convergence}]
\label{thm:global-convergence}
Suppose \cref{ass:matrix-existence} and \cref{ass:matrix-uniqueness} hold. For all $t>0$,
\begin{align}
\label{eq:potential convergence}
g(u^{(t+1)},v^{(t+1)}) - g^\ast \leq (1-c_B \frac{\lambda_{-2}(\mathcal{L})} {l})\left( g(u^{(t)},v^{(t)})-g^\ast \right),
    \end{align}
    where $c_B=e^{-4B}$ and $l=\min\{l_0,l_1\}$.
      
    As a consequence, we have the following bound:
    \begin{align*}
       \|r^{(t)}-p\|_{1} &\leq c'_Be^{-c_{B}\frac{\lambda_{-2}(\mathcal{L})}{\min\{l_{0},l_{1}\}}\cdot t},
    \end{align*} 
    where $(c'_B)^2=(8B\sum_{i}p_{i})$.
\end{theorem}
\cref{thm:global-convergence} immediately implies the following iteration complexity bound.
\begin{corollary}[\textbf{Iteration Complexity}]
Under \cref{ass:matrix-existence} and \cref{ass:matrix-uniqueness}, $\|r^{(t)}-p\|_1\leq \epsilon$ after
      \begin{align*}
          \mathcal{O} \left (\frac{\min\{l_{0},l_{1}\}}{\lambda_{-2}(\mathcal{L})} \cdot \log (1/\epsilon) \right )
      \end{align*}
       iterations of Sinkhorn's algorithm.
\end{corollary}

\textbf{Remark.} The ability of Sinkhorn's algorithm to exploit the strong convexity of $g(u,v)$ on $\mathbf{1}_{m+n}^\perp$ relies critically on the invariance of $g(u,v)$ under \emph{normalization}, which is an intrinsic feature of the problem that has been largely set aside in the convergence analysis so far. Recall that $u=\log d^0$ and $v=-\log d^1$, where $d^0,d^1$ are the diagonals of the scaling $(D^0,D^1)$. Scalings are only determined up to multiplication by $(1/c,c)$ for $c>0$, and the translation $(u,v)\rightarrow(u-\log c,v-\log c)$ does not alter the objective value in \eqref{eq:transformed-potential}. We may therefore impose an \emph{auxiliary} normalization $(u,v)^T\mathbf{1}_{m+n}=0$, or equivalently $\prod_j d^0_j = \prod_i d^1_i$. This normalization is easily achieved by requiring that after every update of Sinkhorn's algorithm, a normalization $(d^0/c,c d^1)$ is performed using the normalizing constant 
\begin{align}
\label{eq:normalization}
   c=\sqrt{\prod_j d^0_j /\prod_i d^1_i}.
\end{align}
See \cref{alg:scaling-normalized}. Note, however, that this normalization is only a supplementary construction in our analysis. The final convergence result applies to the original Sinkhorn's algorithm without normalization, since it does not alter the objective value. Normalization of Sinkhorn's algorithm is discussed in \citet{carlier2023sista}, although they use the asymmetric condition $u_0=0$, which does not guarantee that normalized Sinkhorn iterates stay in $\mathbf{1}_{m+n}^\perp$. 
\begin{algorithm}[tb]
\caption{Normalized Sinkhorn's Algorithm}
   \label{alg:scaling-normalized}
\begin{algorithmic}
   \STATE {\bfseries Input:}  $A, p, q,\epsilon_{\text{tol}}$.
   \STATE {\bfseries initialize} $d^{0}\in\mathbb{R}_{++}^{m}$
   \REPEAT
   \STATE $d^{1} \leftarrow  p/( A d^0)$ 
   \STATE 
  normalization  $(d^0,d^1) \leftarrow (d^0/c,c d^1),c>0$
   \STATE $d^{0}\leftarrow  q/({A}^{T} d^{1})$
   \STATE 
  normalization  $(d^0,d^1) \leftarrow (d^0/c,c d^1),c>0$ 
   \STATE 
$\epsilon\leftarrow$  update of $(d^{0},d^1)$
\UNTIL{$\epsilon<\epsilon_{\text{tol}}$}
\end{algorithmic}
\end{algorithm}

The proof of \cref{thm:global-convergence} then relies crucially on the observation that the Hessian of $g(u,v)$ at $(0,0)$ is precisely the \emph{Laplacian} $\mathcal{L}$ of the bipartite graph $G_b$. Therefore, as $(u,v)$ are \emph{bounded} throughout the iterations thanks to the coercivity of $g$, the Fiedler eigenvalue of $\mathcal{L}$ quantifies the strong convexity on $\mathbf{1}_{m+n}^\perp$. The linear convergence then follows from standard results on block coordinate descent and alternating minimization methods for strongly convex and smooth functions \citep{beck2013convergence}. Typically, the leading eigenvalue of the Hessian quantifies the smoothness \citep{luenberger1984linear}. This is given by $2\max\{l_0,l_1\}$ for $\mathcal{L}$. However, for alternating minimization methods, the better smoothness constant $\min\{l_0,l_1\}$ is available. Thus the quantity $\min\{l_{0},l_{1}\}/\lambda_{-2}(\mathcal{L})$ can be interpreted as a type of  ``condition number'' of the graph Laplacian $\mathcal{L}$. When $A$ is positive (not just non-negative), then the strong existence and uniqueness conditions are trivially satisfied, and our results continue to hold with the rate quantified by $\min\{l_{0},l_{1}\}/\lambda_{-2}(\mathcal{L})$.

\textbf{Remark.} The importance of Assumptions \ref{ass:matrix-existence} and \ref{ass:matrix-uniqueness} are clearly reflected in the bound \eqref{eq:potential convergence}.
First, note that the Fiedler eigenvalue $\lambda_{-2}(\mathcal{L})>0$ iff \cref{ass:matrix-uniqueness} holds (see \cref{subsec:graph-laplacian}). On the other hand, \cref{ass:matrix-existence} guarantees the \emph{coercivity} of $g$ on $\mathbf{1}_{m+n}^\perp$. This property ensures that $B<\infty$, and consequently, that normalized iterates stay bounded by $B$. That \cref{ass:matrix-existence} guarantees  $g(u,v)$ is coercive should be compared to the observation by \citet{hunter2004mm} that \cref{ass:strong-connected} guarantees the upper compactness (a closely related concept) of the log-likelihood function \eqref{eq:log-likelihood}.
In contrast, when only the weak existence condition (\cref{ass:matrix-weak-existence}) holds, finite minimizer of $g(u,v)$ may not exist, in which case the diameter $B$ of the initial sub-level set may become infinite.

\cref{ass:matrix-weak-existence} corresponds to the ``limit scaling'' regime of Sinkhorn's algorithm, where the scaled matrices $A^{(t)}$ are guaranteed to converge to a finite matrix $\hat A$ with the desired marginal distributions that solves \eqref{eq:relative-entropy-minimization}, and may have additional zeros compared to $A$. Under \cref{ass:matrix-weak-existence}, \citet{leger2021gradient} shows the slower $\mathcal O(1/t)$ convergence in KL divergence $D_{\text{KL}}(r^{(t)}\| p)$. We now show that this rate is tight, which fully characterizes the following convergence behavior of Sinkhorn's algorithm: whenever a direct scaling exists for the matrix balancing problem, Sinkhorn's algorithm converges linearly. If only a limit scaling exists, then convergence deteriorates to 
$\mathcal O(1/t)$.
\begin{theorem}
\label{thm:lower-bound}
       For general non-negative matrices, Sinkhorn's algorithm converges linearly
iff $(A,p,q)$ satisfy \cref{ass:matrix-existence} and \cref{ass:matrix-uniqueness}. The convergence deteriorates to sub-linear iff the weak existence condition \cref{ass:matrix-weak-existence} holds but \cref{ass:matrix-existence} fails.
\end{theorem}

The regime of sub-linear convergence also has an interpretation in the choice modeling framework. The weak existence condition \cref{ass:matrix-weak-existence}, when applied to $(A,p,q)$ constructed from a choice dataset, allows the case where some subset $S$ of items is always preferred over $S^C$, which implies, as observed already by the early work of \citet{ford1957solution}, that the log-likelihood function \eqref{eq:log-likelihood} is only maximized at the \emph{boundary} of the probability simplex, by shrinking  $s_j$ for $j\in S^C$ towards 0, i.e., $D^0_j \rightarrow 0$. Incidentally, \citet{bacharach1965estimating} also refers to the corresponding regime in matrix balancing as ``boundary solutions''. 

\subsection{Sharp Asymptotic Rate}
\label{subsec:sharp-rate}
Having established the global convergence of Sinkhorn's algorithm when finite scalings exist, we now turn to the open problem of characterizing its asymptotic linear convergence rate for non-uniform marginals. Our analysis relies on an \emph{intrinsic} orthogonality structure of Sinkhorn's algorithm instead of the auxiliary  normalization used to prove the global linear convergence above. Note that unlike the global rate, which depends on $A$, the asymptotic rate now depends on the associated solution $\hat A$ (and $p,q$), as expected.  
\begin{theorem}[\textbf{Sharp Asymptotic Rate}]
\label{thm:convergence}
Suppose $(A,p,q)$ satisfy \cref{ass:matrix-existence} and \cref{ass:matrix-uniqueness}. Let $\hat{A}$
be the unique scaled matrix with marginals $p,q$. Then
\begin{align*}
\lim_{t\rightarrow \infty} \frac{\|r^{(t+1)}/\sqrt{p}-\sqrt{p}\|_{2} }{\|r^{(t)}/\sqrt{p}-\sqrt{p}\|_{2}} = \lambda_\infty,
\end{align*}
where the asymptotic linear rate of convergence $\lambda_\infty$ is
\begin{align*}
\lambda_\infty & :=\lambda_{2}(\tilde{A}\tilde{A}^{T})=\lambda_{2}(\tilde{A}^{T}\tilde{A})\\
\tilde{A} & :=\mathcal{D}(1/\sqrt{p})\cdot\hat{A}\cdot\mathcal{D}(1/\sqrt{q}),
\end{align*}
where $\lambda_{2}(\cdot)$ denotes the second largest eigenvalue. 
\end{theorem}

Intuitively, the dependence of the linear rate of convergence on the second largest eigenvalue of $\tilde{A}^T\tilde{A}$ (and $\tilde{A}\tilde{A}^T$) is due to the fact that near the optimum $\sqrt{p}$, $\tilde{A}\tilde{A}^T$ (which is the Jacobian at $\sqrt{p}$) approximates the first order change in $r^{(t)}/\sqrt{p}$. Normally, the \emph{leading} eigenvalue quantifies this change. The unique leading eigenvalue of $\tilde{A}\tilde{A}^T$ is equal to 1 with eigenvector $\sqrt{p}$, which does not imply contraction. Fortunately, using the quantity  $r^{(t)}/\sqrt{p}$
allows us to exploit the following orthogonality structure:
\begin{align*}
(r^{(t)}/\sqrt{p}-\sqrt{p})^{T}\sqrt{p} & =\sum_{i}(r_{i}^{(t)}-p_{i})=0
\end{align*}
by virtue of Sinkhorn's algorithm preserving the quantities $r^{(t)T}\mathbf{1}_{n}$
for all $t$. Thus, the residual $r^{(t)}/\sqrt{p}-\sqrt{p}$ is always \emph{orthogonal} to
$\sqrt{p}$, which is both the leading eigenvector and the fixed point of the iteration. The convergence is then controlled by the \emph{second} largest eigenvalue of $\tilde{A}\tilde{A}^T$. This proof approach echoes that of the global linear convergence result in \cref{thm:global-convergence}, where we also exploit an orthogonality condition to obtain a meaningful bound. In \cref{thm:global-convergence} the bound depends on the second smallest eigenvalue of the Hessian, while in \cref{thm:convergence} the bound depends on the second largest eigenvalue of the Jacobian. 

In the special case of $m=n$ and $p=q=\mathbf{1}$, the asymptotic rate in \cref{thm:convergence} reduces to that in \citet{knight2008sinkhorn}. Note, however, that the convergence metric is different: we use the $\ell^2$ norm $\|r^{(t)}/\sqrt{p}-\sqrt{p}\|_2$ while \citet{knight2008sinkhorn} uses an \emph{implicit} norm that measures the convergence of $D^0$ to the solution \emph{line} due to scale invariance. Our analysis exploits the orthogonality structure of Sinkhorn's algorithm and more explicitly reveals the dependence of the convergence rate on the spectral structure of the data. 

Our results in this section are relevant in several respects. First, we clarify the gap between the $\mathcal O(1/t)$ and $\mathcal O(\lambda^t)$ convergence of Sinkhorn's algorithm: the slowdown happens if and only if Sinkhorn's algorithm converges but the canonical matrix balancing problem does not have a \emph{finite} scaling $(D^0,D^1)$. This slowdown has been observed in the literature but not systematically studied. Second, 
 we settle open problems and establish the first quantitative global linear convergence result for Sinkhorn's algorithm applied to general non-negative matrices. We also characterize the asymptotic linear rate of convergence, generalizing the result of \citet{knight2008sinkhorn} but with a novel analysis. 
  Third, our analysis reveals the importance of algebraic connectivity for the convergence of Sinkhorn's algorithm. Although an important quantity in the choice modeling literature, algebraic connectivity has not been previously used to in the analysis of Sinkhorn's algorithm. The importance of algebraic connectivity for Sinkhorn's algorithm becomes less surprising once we connect it to the distributed optimization literature in \cref{sec:connections}, where it is well-known that the spectral gap of the \emph{gossip} matrix, which defines the decentralized communication network, governs the rates of convergence.

\begin{comment}
\begin{figure}
\includegraphics[width=\linewidth]{figs/beyond_tss_lesion.pdf}
\caption[]{End-to-End runtime lesion study of the entire MNIST dataset and the FMA featurized music dataset. Each of DROP's contributions provides a runtime improvement.}
\label{fig:beyond_lesion}
\end{figure}
\end{comment}



\section{Conclusion}
\label{sec:conclusion}

Advanced data analytics techniques must scale to rising data volumes. 
DR techniques offer a powerful toolkit when processing these datasets, with PCA frequently outperforming popular techniques in exchange for high computational cost. 
In response, we propose DROP, a new dimensionality reduction optimizer. 
DROP combines progressive sampling, progress estimation, and online aggregation to identify high quality low dimensional bases via PCA without processing the entire dataset by balancing the runtime of downstream tasks and achieved dimensionality. 
Thus, DROP provides a first step in bridging the gap between quality and efficiency in end-to-end DR for downstream \red{analytics}. 

%We revisit canonical operators for time series dimensionality reduction and the measurement study of~\cite{keogh-study}, and show that PCA is more effective than popular alternatives in the data mining literature often by a margin of over $2\times$ on average on gold-standard time series benchmark data sets with respect to output data dimension. More surprisingly, we empirically demonstrate that a small number of samples are sufficient to accurately characterize directions of maximum variance and obtain a high-quality low-dimensional transformation.




\bibliography{sinkhorn-choice}
\bibliographystyle{informs2014}

%%%%%%%%%%%%%%%%%%%%%%%%%%%%%%%%%%%%%%%%%%%%%%%%%%%%%%%%%%%%%%%%%%%%%%%%%%%%%%%
%%%%%%%%%%%%%%%%%%%%%%%%%%%%%%%%%%%%%%%%%%%%%%%%%%%%%%%%%%%%%%%%%%%%%%%%%%%%%%%
% APPENDIX
%%%%%%%%%%%%%%%%%%%%%%%%%%%%%%%%%%%%%%%%%%%%%%%%%%%%%%%%%%%%%%%%%%%%%%%%%%%%%%%
%%%%%%%%%%%%%%%%%%%%%%%%%%%%%%%%%%%%%%%%%%%%%%%%%%%%%%%%%%%%%%%%%%%%%%%%%%%%%%%
\newpage
\begin{APPENDICES}
\crefalias{section}{appendix}
\section{Graph Laplacians and Algebraic Connectivity}
\label{subsec:graph-laplacian}
In this section, we introduce the quantities central to our global linear convergence analysis, especially the \emph{graph Laplacian} matrices associated with the graphs defined by a non-negative matrix $A$ and the Fielder eigenvalues.

Given a non-negative matrix $A\in \mathbb{R}_+^{n\times m}$, we define the associated (weighted) bipartite graph $G_b$ on $V\cup U$ by the adjacency matrix $A^b \in \mathbb{R}^{(m+n)\times (m+n)}$ defined as
\begin{align*}
    A^b := \begin{bmatrix}\mathbf{0} & {A}\\
{A}^{T} & \mathbf{0}
\end{bmatrix}.	
\end{align*}
The rows of $A$ correspond to vertices in $V$
with $|V|=n$, while the columns of $A$ correspond to vertices in
$U$ with $|U|=m$, and $V\cap U=\emptyset$. The matrix $A$ here is sometimes called the \emph{biadjacency matrix} of the bipartite graph.

The matrix $A$ also defines
an \emph{undirected} ``comparison'' graph $G_c$ on $m$ items. This is most easily understood when $A$ is binary and we can associate it with the participation matrix of a choice dataset, but the definition below is more general. Define the adjacency matrix $A^c \in \mathbb{R}^{m\times m}$ by
\begin{align*}
{A^c}_{jj'} & =\begin{cases}
0 & j=j'\\
(A^{T}A)_{jj'} & j\neq j',
\end{cases}
\end{align*}
If $A$ is a binary participation matrix associated with a choice dataset, then there is a (weighted) edge in $G_{c}$ between items $j$ and $j'$
if and only the two appear in some choice set together, with the edge
weight equal to the number of times of their co-occurrence. This undirected comparison graph $G_c$ is not the same as the directed comparison graph in \cref{ass:strong-connected}, since it does not encode the \emph{choice} of each observation. However, it is also an important object in choice modeling. For example, the uniqueness condition in \cref{ass:weak-connected} for choice maximum likelihood estimation has a concise graph-theoretic interpretation as it is a requirement that $G_c$ be connected.

For a (generic) undirected graph $G$ with adjacency matrix $M$, the graph Laplacian matrix (or simply the Laplacian) is defined as $L(M):=\mathcal{D}(M\mathbf{1})-M$, where recall $\mathcal{D}$ is the diagonalization of a vector. The graph Laplacian $L(M)$ is always positive semidefinite as a result of the Gershgorin circle theorem, since $L(M)$ is diagonally dominant  with positive diagonal and non-positive off-diagonals. Moreover, the Laplacian always has $\mathbf{1}$ in its null space.

For the graphs $G_b,G_c$, their Laplacians are  given respectively by 
\begin{align}
    \label{eq:Laplacians}
\mathcal{L}:&=L(A^b)=\begin{bmatrix}\mathcal{D}({A}\mathbf{1}_{m}) & -{A}\\
-{A}^{T} & \mathcal{D}({A}^{T}\mathbf{1}_{n})
\end{bmatrix} \\L:&=L(A^c)=A^TA\mathbf{1}_m-A^TA,
\end{align}
where we can
verify that for the comparison graph $G_c$, its Laplacian $L$ satisfies
\begin{align*}
L=A^c\mathbf{1}_{m}-A^c & =A^{T}A\mathbf{1}_{m}-A^{T}A.
\end{align*}
The graph Laplacian $\mathcal{L}$ based on $A^b$ and $L$ based on $A^c$ are closely connected through the identity
\begin{align*} (A^{b})^{2}&=\begin{bmatrix}AA^{T} & 0\\
0 & A^{T}A
\end{bmatrix},
\end{align*}
which implies that $L$ is the lower right block of the graph Laplacian $\mathcal{D}((A^{b})^{2}\mathbf{1}_{m+n})-(A^{b})^{2}$. Moreover, $L$ plays a central role in works on the statistical and computational efficiency in choice modeling \citep{shah2015estimation,seshadri2020learning,vojnovic2020convergence}.

An important concept in spectral graph theory is the \emph{algebraic connectivity} of a graph, quantified by the second smallest eigenvalue $\lambda_{-2}$ of the graph Laplacian matrix, also called the Fiedler eigenvalue \citep{fiedler1973algebraic,spielman2007spectral}.
% The Fiedler eigenvalue $\lambda_{-2}$ features prominently in Cheeger's inequality,
% \begin{align}
% \label{eq:Cheeger}
%   \frac{h^2(G)}{2\max_j(M\mathbf{1})_j}   \leq \lambda_{-2}(\mathcal{D}(M\mathbf{1})-M) \leq  2h(G),
% \end{align}
% where $h(G)\geq0$ is the Cheeger constant, positive iff $G$ is connected.
Intuitively, Fiedler eigenvalue quantifies how well-connected a graph is in terms of how many edges need to be removed for the graph to become disconnected. It is well-known that the multiplicity of the smallest eigenvalue of the graph Laplacian, which is 0, describes the number of connected components of a graph. The uniqueness condition for matrix balancing in \cref{ass:matrix-uniqueness} therefore guarantees that the Fiedler eigenvalue of $G_b$ is positive: $\lambda_{-2}(\mathcal{L}) >0$. This property is important for our results, since $\lambda_{-2}(\mathcal{L})$ quantifies the \emph{strong} convexity of the potential function and hence the linear convergence rate of Sinkhorn's algorithm.

\section{Further Connections to Choice Modeling and Optimization}
\label{sec:connections}
In this section, we demonstrate that our matrix balancing formulation \eqref{eq:equation-system} of the maximum likelihood problem \eqref{eq:log-likelihood} provides a unifying perspective on many existing works on choice modeling, and establishes interesting connections to distributed optimization as well. Throughout, Sinkhorn's algorithm will serve as the connecting thread. In particular, it reduces algebraically to the algorithms in \citet{zermelo1929berechnung,dykstra1956note,ford1957solution,hunter2004mm,maystre2017choicerank} in their respective choice model settings. This motivates us to provide an interpretation of Sinkhorn's algorithm as a ``minorization-maximization'' (MM) algorithm \citep{lange2000optimization}. Moreover, Sinkhorn's Algorithm is also related to the ASR algorithm of \citet{agarwal2018accelerated} for choice modeling, as they can both be viewed as message passing algorithms in distributed optimization \citep{balakrishnan2004polynomial}. Last but not least, we establish a connection between Sinkhorn's algorithm and the well-known BLP algorithm of \citet{berry1995automobile}, widely used in economics to estimate consumer preferences from data on market shares. 

\subsection{Pairwise Comparisons}
The same algorithmic idea in many works on pairwise comparisons appeared as early as \citet{zermelo1929berechnung}. For example,
\citet{dykstra1956note} gives the following update formula: 
\begin{align}
\label{eq:pairwise}
s_{j}^{(t+1)} & =W_{j}/\sum_{j\neq k}\frac{C_{jk}}{s_{j}^{(t)}+s_{k}^{(t)}},
% s_{j}^{(t+1)} & =W_{j}\cdot\left[\sum_{j\neq k}\frac{C_{jk}}{s_{j}^{(t)}+s_{k}^{(t)}}\right]^{-1},
\end{align}
 where again $W_{j}=|\{i\mid (j,S_i)\}|$ is the number of times
item $j$ is chosen (or ``wins''), and $C_{jk}$ is the number of comparisons between $j$
and $k$. \cref{ass:strong-connected} guarantees $C_{jk}>0$ for any $j,k$.  \citet{zermelo1929berechnung}
proved that under this assumption
$s^{(1)},s^{(2)},\dots$ converge to the unique maximum likelihood estimator,
and the sequence of log-likelihoods $\ell(s^{(1)}),\ell(s^{(2)}),\dots$
is monotone increasing. A cyclic version of \eqref{eq:pairwise} appeared in \citet{ford1957solution} with an independent proof of convergence. One can verify that by aggregating choice sets $S_i$ in \eqref{eq:scaling-iteration} over pairs of objects, it reduces to \eqref{eq:pairwise}. However, \eqref{eq:pairwise} as is written does not admit a matrix balancing formulation. A generalization of the algorithm of \citet{zermelo1929berechnung,ford1957solution,dykstra1956note} for pairwise comparison to ranking data was not achieved until the influential works of \citet{lange2000optimization} and \citet{hunter2004mm}. 
\subsection{MM Algorithm of \citet{hunter2004mm} for Ranking
Data} 
Motivated by the observation in \citet{lange2000optimization} that \eqref{eq:pairwise} is an instance of an minorization-maximization (MM) algorithm, the seminal work of \citet{hunter2004mm} proposed the general approach of solving ML estimation of choice models via MM algorithms, which relies on the inequality 
\begin{align*}
-\log x & \geq1-\log y-(x/y)
\end{align*}
to construct a lower bound (minorization) on the log-likelihood that has an explicit maximizer (maximization), and iterates between the two steps. \citet{hunter2004mm} develops such an algorithm for the Plackett--Luce
model for ranking data and proves its monotonicity and convergence.

Given $n$ partial rankings, where the $i$-th partial ranking on $l_{i}$ objects is indexed by $a(i,1)\rightarrow a(i,2)\rightarrow \cdots \rightarrow a(i,l_{i})$, the MM algorithm of \citet{hunter2004mm} takes the form 
\begin{align}
\label{eq:mm}
s_{k}^{(t+1)} & =\frac{w_{k}}{\sum_{i=1}^{n}\sum_{j=1}^{l_{i}-1}\delta_{ijk}[\sum_{j'=j}^{l_{i}}s_{a(i,j')}^{(t)}]^{-1}},
\end{align}
 where $\delta_{ijk}$ is the indicator that item $k$ ranks no better
than the $j$-th ranked item in the $i$-th ranking, and $w_{k}$ is the number of
rankings in which $k$ appears but is not ranked last. 
\begin{proposition}
\label{lem:mm}
Sinkhorn's algorithm applied to the ML estimation of the Plackett--Luce model is algebraically equivalent to \eqref{eq:mm}. 
\end{proposition}
Therefore, Sinkhorn's algorithm applied to the ML estimation of the Plackett--Luce model reduces algebraically to the MM algorithm of \citet{hunter2004mm}. However, \cref{alg:scaling} applies to more general choice models with minimal or no change, while the approach in \citet{hunter2004mm} requires deriving the minorization-majorization step for every new optimization objective. This was carried out, for example, in \citet{maystre2017choicerank} for a network choice model. We show in \cref{prop:choicerank} that their ChoiceRank algorithm is also a special case of (regularized) Sinkhorn's algorithm. From a computational perspective, even when algorithms are equivalent algebraically, their empirical performance can vary drastically depending
on the particular implementation. Another advantage of Sinkhorn's algorithm is that it computes \emph{all}
entries simultaneously through vector and matrix operations, while
the analytical formula in \eqref{eq:mm} is hard to parallelize. This distinction is likely behind
the discrepancy in \cref{sec:empirics} between our experiments and those in \citet{maystre2015fast},
who conclude that the MM version \eqref{eq:mm} is slower in terms of wall clock time than their Iterative Luce Spectral Ranking (I-LSR) algorithm for the Plackett--Luce model on $k$-way partial ranking data. 

\subsection{Markov Steady State Inversion and Network Choice}
\label{subsec:steady-state}
Our work is related to the works of \citet{kumar2015inverting,maystre2017choicerank} on Markov chains on graphs, where transition matrices are parameterized by node-dependent scores prescribed by Luce's choice axiom. More precisely, given a directed graph $G=(V,E)$ and $N^\out_j, N^\inn_j \subseteq V$ the neighbors with edges going out from and into $j\in V$, and a target stationary distribution $\pi$, the (unweighted) steady state inversion problem of \citet{kumar2015inverting} seeks scores $s_j$ such that the transition matrix $T_{j,k}=\frac{s_k}{\sum_{k'\in N^\out_j}s_{k'}}$ has the desired stationary distribution $\pi$. Their Theorem 13 shows that a bipartite version of this problem is equivalent to solving the ML estimation conditions \eqref{eq:optimality} of the choice model. Furthermore, one can verify that their bipartite inversion problem has the same form as \eqref{eq:bridge} in our paper, with the bipartite graph defined using $A$. Their existence condition (termed ``consistency'') is equivalent to \cref{ass:matrix-weak-existence}(a) \citep{menon1968matrix} for the matrix balancing problem. Despite these connections, the key difference in our work is the reformulation of \eqref{eq:optimality} as one involving diagonal scalings of rows and columns of $A$, which was absent in \citet{kumar2015inverting}. Consequently, they proposed a different algorithm instead of applying Sinkhorn's algorithm. 

Building on \citet{kumar2015inverting}, \citet{maystre2017choicerank} consider a similar Markov chain on $(V,E)$, where now for each edge $(j,k) \in E$ one observes a finite \emph{number} $c_{jk}$ of transitions along it, and take a maximum likelihood approach to estimate the scores $s_j$.  They show that, as one might expect, the steady state inversion problem of \citet{kumar2015inverting} is the asymptotic version of the ML estimation problem in their network choice model. 

An additional contribution of \citet{maystre2017choicerank} is the regularization of the inference problem via a Gamma prior on $s_j$'s, which eliminates the necessity of any assumptions on the choice dataset such as \cref{ass:strong-connected}. They then follow the proposal of \citet{hunter2004mm} and develop an MM algorithm for maximum likelihood estimation called ChoiceRank, the unregularized version of which can be written as follows:
\begin{align}
    \label{eq:choicrank}
    s_j^{(t+1)} = \frac{c_j^\inn}{\sum_{k\in N_j^\inn} \gamma_k^{(t)}}, \gamma_j^{(t)}=\frac{c_j^\out}{\sum_{k\in N_j^\out}s_k^{(t)}},
\end{align}
where $c_j^\inn=\sum_{k\in N_j^\inn}c_{kj}$ and $c_j^\out=\sum_{k\in N_j^\out}c_{jk}$ are the total number of observed transitions into and out of $j\in V$. 
\begin{proposition}
    \label{prop:choicerank}
   The network choice model of \citet{maystre2017choicerank} is a special case of the choice model \eqref{eq:model}, and  Sinkhorn's algorithm applied to this case reduces to an iteration algebraically equivalent to \eqref{eq:choicrank}.
\end{proposition}
We also explore the regularization approach of \citet{maystre2017choicerank} in \cref{subsec:regularization} and demonstrate that Sinkhorn's algorithm can easily accommodate this extension, resulting in a regularized version of Sinkhorn's algorithm for matrix balancing that \emph{always} converges. This is given in \cref{alg:regularized}. Once again, insights from choice modeling yield useful improvements in matrix balancing.

\subsection{Sinkhorn's Algorithm as an MM Algorithm}
\label{sec:sinkhorn-MM}
That Sinkhorn's algorithm reduces to MM algorithms when applied to various choice models is not a coincidence. 
In this section, we establish the connection between choice modeling
and matrix balancing through an optimization perspective. This connection
provides an interesting interpretation of Sinkhorn's algorithm
as optimizing a dominating function, i.e., an MM algorithm. See \citet{lange2000optimization,lange2016mm} for a discussion of the general correspondence between block
coordinate descent algorithms and MM algorithms.

First, we discuss the connection between the log-likelihood function
\eqref{eq:log-likelihood} and the dual potential function \eqref{eq:log-barrier} when $(A,p,q)$ corresponds to a choice dataset. Consider maximizing the negative
dual potential function 
\begin{align*}
h(d^{0},d^{1}):=-g(d^{0},d^{1}) & =\sum_{j=1}^{m}q_{j}\log d_{j}^{0}+\sum_{i=1}^{n}p_{i}\log d_{i}^{1}-(d^{1})^{T}Ad^{0}.
\end{align*}
For each fixed $d^{0}$, the function is concave in $d^{1}$, and
maximization with respect to $d^{1}$ yields first order conditions
\begin{align*}
d^{1} & =p/(Ad^{0}).
\end{align*}
Substuting this back into $h(d^{0},d^{1})$, we obtain 
\begin{align*}
f(d^{0}):=h(d^{0},p/(Ad^{0})) & =\sum_{j=1}^{m}q_{j}\log d_{j}^{0}+\sum_{i=1}^{n}p_{i}\log(\frac{p_{i}}{(Ad^{0})_{i}})-\sum_{i}p_{i}\\
 & =\sum_{j=1}^{m}q_{j}\log d_{j}^{0}-\sum_{i=1}^{n}p_{i}\log(Ad^{0})_{i}+\sum_{i}p_{i}\log p_{i}-\sum_{i}p_{i}.
\end{align*}
If $A$ is a valid participation matrix for a choice dataset and $p,q$
are integers, we can identify $(A,p,q)$ with a choice dataset. Each
row of the participation matrix $A$ is the indicator vector of choice
set $S_{i}$, and $d_{j}^{0}$ is the quality score. In this
case $(Ad^{0})_{i}=\sum_{k\in S_{i}}d_{k}^{0}$, so that 
\begin{align*}
\sum_{j=1}^{m}q_{j}\log d_{j}^{0}-\sum_{i=1}^{n}p_{i}\log(Ad^{0})_{i} & =\sum_{j=1}^{m}q_{j}\log d_{j}^{0}-\sum_{i=1}^{n}p_{i}\sum_{k\in S_{i}}d_{k}^{0}=\ell(d^{0}).
\end{align*}

It then follows that
\begin{align*}
\min_{d^{0},d^{1}}g(d^{0},d^{1})\Leftrightarrow\max_{d^{0},d^{1}}h(d^{0},d^{1})\Leftrightarrow\max_{d^{0}}\max_{d^{1}}h(d^{0},d^{1})\Leftrightarrow\max_{d^{0}}f(d^{0})\Leftrightarrow\max_{d^{0}}\ell(d^{0}),
\end{align*}
so that minimizing the potential function $g$ is equivalent to maximizing
the log-likelihood function $\ell$. Moreover, the first order condition of maximizing $h$ with respect to $d^1$ is 
$d^0=q/(A^Td^1)$, which when $(A,p,q)$ is identified with a choice dataset reduces to 
\begin{align*}
    q_j =\sum_{i\mid j\in S_i}p_i \frac{d^0_j}{\sum_{k\in S_i}d^0_k},
\end{align*}
which is the optimality condition \eqref{eq:optimality} of the choice model.

Next, given $d^{0(t)}$ the estimate of $d^{0}$ after the
$t$-th iteration, define the function 
\begin{align*}
f(d^{0}\mid d^{0(t)}):=h(d^{0},p/(Ad^{0(t)})) & =\sum_{j=1}^{m}q_{j}\log d_{j}^{0}+\sum_{i=1}^{n}p_{i}\log\frac{p}{Ad^{0(t)}}-\frac{p_{i}}{(Ad^{0(t)})_{i}}(Ad^{0})_{i}.
\end{align*}
We can verify that 
\begin{align*}
f(d^{0(t)}\mid d^{0(t)}) & =f(d^{0(t)})\\
f(d^{0}\mid d^{0(t)}) & \leq f(d^{0}),
\end{align*}
 so that $f(d^{0}\mid d^{0(t)})$ is a valid minorizing function of
$f(d^{0})$ \citep{lange2000optimization} that guarantees the ascent property $f(d^{0(t+1)})\geq f(d^{0(t+1)}\mid d^{0(t)})=\max_{d^{0}}f(d^{0}\mid d^{0(t)})\geq f(d^{0(t)}\mid d^{0(t)})=f(d^{0(t)})$.
The update in the maximization step
\begin{align*}
d^{0(t+1)}=\arg\max_{d^{0}}f(d^{0}\mid d^{0(t)}) & =q/A^{T}\frac{p}{Ad^{0(t)}}
\end{align*}
 is precisely one full iteration of Sinkhorn's algorithm. Note that this interpretation of Sinkhorn's algorithm does not require $A$ to be binary, and $p,q$ to be integers.

On the other hand, using the property $-\ln x\geq1-\ln y-(x/y)$,
we can \emph{directly} construct a minorizing function of $\ell$
by 
\begin{align*}
\ell(d^{0})=\sum_{j=1}^{m}q_{j}\log d_{j}^{0}-\sum_{i=1}^{n}p_{i}\log(Ad^{0})_{i} & \geq\sum_{j=1}^{m}q_{j}\log d_{j}^{0}+\sum_{i=1}^{n}p_{i}(-\frac{(Ad^{0})_{i}}{(Ad^{0(t)})_{i}}-\log(Ad^{0(t)})_{i}+1)=\ell(d^{0}\mid d^{0(t)}),
\end{align*}
 where $\ell(d^{0}\mid d^{0(t)})$ is a valid minorizing function
of $\ell$. Maximizing $\ell(d^{0}\mid d^{0(t)})$ with respect to
$d^{0}$, the update in the maximization step is
\begin{align*}
d_{j}^{0(t+1)} & =q_{j}/\sum_{i\mid j\in S_i}\frac{p_{i}}{(Ad^{0(t)})_{i}}
\end{align*}
 which again is one full iteration of Sinkhorn's algorithm applied to the Luce choice model. Moreover,
\begin{align*}
\ell(d^{0}\mid d^{0(t)})+\sum_{i}p_{i}\log p_{i}-\sum_{i}p_{i} & =\sum_{j=1}^{m}q_{j}\log d_{j}^{0}+\sum_{i=1}^{n}p_{i}(-\frac{(Ad^{0})_{i}}{(Ad^{0(t)})_{i}}-\log(Ad^{0(t)})_{i}+1)+\sum_{i}p_{i}\log p_{i}-\sum_{i}p_{i}\\
 & =\sum_{j=1}^{m}q_{j}\log d_{j}^{0}+\sum_{i=1}^{n}p_{i}(-\frac{(Ad^{0})_{i}}{(Ad^{0(t)})_{i}}+\log\frac{p_{i}}{(Ad^{0(t)})_{i}})\\
 & =f(d^{0}\mid d^{0(t)}).
\end{align*}
 Therefore, the minorizing function $\ell(d^{0}\mid d^{0(t)})$ constructed
using $-\ln x\geq1-\ln y-(x/y)$ for the log-likelihood and the minorizing
function $f(d^{0}\mid d^{0(t)})$ constructed for $\max_{d^{1}}h(d^{0},d^{1})$
are identical modulo a constant $\sum_{i}p_{i}\log p_{i}-\sum_{i}p_{i}$.
Sinkhorn's algorithm is in fact the MM algorithm corresponding to both minorizations. However, the perspective using $f(d^{0}\mid d^{0(t)})$ is more general since it applies to general $(A,p,q)$ as long as they satisfy Assumptions \cref{ass:matrix-existence} and \cref{ass:matrix-uniqueness}, whereas the MM algorithm based on $\ell(d^{0}\mid d^{0(t)})$ is designed for choice dataset, so requires $A$ to be binary.

\subsection{ Sinkhorn's Algorithm and Distributed Optimization} 
We now shift our focus to algorithms in distributed optimization, where Sinkhorn's algorithm can be interpreted as a message passing/belief propagation algorithm \citep{balakrishnan2004polynomial}. We start by observing a connection to the ASR algorithm for estimating Luce choice models \citep{agarwal2018accelerated}, which returns the same approximate ML estimators as the RC \citep{negahban2012iterative} and LSR \citep{maystre2015fast} algorithms, but has provably faster convergence. 

Consider the bipartite graph $G_b$ defined by $A$ in \cref{subsec:graph-laplacian}, which consists of choice set nodes $V$ on one hand and item nodes $U$ on the other, where there is an edge between $i\in V$ and $j\in U$ if and only if $j\in S_i$. \citet{agarwal2018accelerated} provide the following message passing interpretation of ASR on the bipartite graph: at every iteration, the item nodes send a ``message'' to their
neighboring choice set nodes consisting of each item node's current estimate of their own $s_j$; the choice set nodes then aggregate the messages they receive by summing up these estimates, and then sending back the sums to their neighboring item nodes. The item nodes use these sums to update estimates of their own $s_j$. \citet{agarwal2018accelerated} show that since the ASR algorithm is an instance of the message passing algorithm, it can be implemented in a distributed manner.% similar to the parallelization capability of Sinkhorn's algorithm. 

We now explain how Sinkhorn's algorithm is another instance of the message passing algorithm described above. Recall that $d^0$ is identified with the $s_j$'s in the Luce choice model, so that $d^{1} \leftarrow p/(Ad^{0})$ precisely corresponds to item nodes ``passing'' their current estimates to set nodes, which then sum up the received estimates and then take the weighted \emph{inverse} of this sum. Similarly, $d^0 \leftarrow q/(A^{T} d^{1})$ corresponds to choice set nodes passing their current estimates of $d^1$ back to item nodes, which then sum up the received messages and take the weighted inverse as their updated estimates of $s_j$. The main difference with ASR lies in how each item node $j$ updates its estimate of $s_j$ based on the messages it receives from neighboring set nodes. In Sinkhorn's algorithm, the update to $s_j$ is achieved by dividing $p$ by a weighted average of the \emph{inverse} of summed messages $1/\sum_{k\in S_{i}}s_{k}^{(t)}$:
\begin{align*}
  s_j^{(t+1)} \leftarrow  q_j/(A^{T} d^{1})_j=W_j/\sum_{i\mid j\in S_{i}}\frac{R_i}{\sum_{k\in S_{i}}s_{k}^{(t)}},
\end{align*}
whereas in ASR, the update is an average of the summed messages $\sum_{k\in S_{i}}s_{k}^{(t)}$ without taking their inverses first: 
\begin{align*}
    s^{(t+1)}_j \leftarrow  \frac{1}{\sum_{i\mid j\in S_{i}} R_i} \sum_{i\mid j \in S_i}  W_{ji} \sum_{k\in S_{i}}s_{k}^{(t)},
\end{align*}
where $W_{ji}$ is the number of times item $j$ is selected from all observations having choice set $S_i$, with $\sum_i W_{ji} = W_j$.

From another perspective, the two algorithms arise from different \emph{moment} conditions. While Sinkhorn's algorithm is based on the optimality condition \eqref{eq:optimality}, ASR is based on the condition
\begin{align*}
    \sum_{i\mid j\in S_{i}} R_i = \sum_{i\mid j \in S_i}  W_{ji}/\frac{s_j}{\sum_{k\in S_{i}}s_{k}},
\end{align*}
which results in an approximate instead of exact MLE. 

The message passing interpretation also provides further insights on the importance of algebraic connectivity to the convergence rate of Sinkhorn's algorithm. Graph theoretic conditions like \cref{ass:strong-connected} are related to network flow and belief propagation, and characterize how fast information can be distributed across the bipartite network with the target distributions $p,q$. It is well-known that convergence of distributed algorithms on networks depends critically on the network topology through the spectral gap of the associated averaging matrix. We can understand \cref{thm:convergence} on the asymptotic convergence rate of Sinkhorn's algorithm as a result of this flavor, although a precise equivalence is left for future works. 

\subsection{The Berry--Levinsohn--Pakes Algorithm}
Last but not least, our work is also closely related to the economics literature that studies consumer behavior based on discrete choices \citep{mcfadden1973conditional,mcfadden1978modelling,mcfadden1981econometric,berry1995automobile}. Here we discuss the particular connection with the work of Berry,
Levinsohn, and Pakes \cite{berry1995automobile}, often referred to as BLP. To estimate consumer preferences over automobiles across different markets (e.g., geographical), they propose a random utility (RUM) model indexed by individual $i$, product
$j$, and market $t$:
\begin{align*}
U_{ijt} & =\beta_{i}^{T}X_{jt}+\theta_{jt}+\epsilon_{ijt},
\end{align*}
 where $\theta_{jt}$ is an unobserved product characteristic, such as the overall popularity of certain types of cars in different regions, and $\epsilon_{ijt}$
are \emph{i.i.d.} double exponential random variables.
% \begin{align*}
% f(\epsilon) & =\exp(-\epsilon-\exp(-\epsilon))
% \end{align*}
The individual-specific coefficient $\beta_{i}$ is random with 
\begin{align*}
\beta_{i} & =Z_{i}^T\Gamma+\eta_{i}\\
\eta_{i}\mid Z_{i} & \sim\mathcal{N}(\beta,\Sigma),
\end{align*}
and the observations consist of \emph{market shares} $\hat{p}_{jt}$ of each
product $j$ in market $t$ and observable population characteristics $Z_{i}$ in each
market. Given a model with fixed $\beta,\Gamma,\Sigma$ and observations, the task is to estimate $\theta_{jt}$. 

For every value of $\theta_{jt}$, we can compute, or simulate if necessary, the \emph{expected} market shares $p_{jt}$, which is the likelihood of product $j$ being chosen in market $t$. For example, in the special case that $\beta,\Gamma,\Sigma \equiv 0$ and $\exp(\theta_{jt})\equiv s_{j}$ for all
$j,t$, i.e., (perceived) product characteristic does not vary across
markets, the expected market share reduces to the familiar formula
\begin{align*}
p_{jt} & =\frac{\exp(\theta_{jt})}{\sum_{k}\exp(\theta_{kt})}=\frac{s_{j}}{\sum_{k}s_{k}}.
\end{align*}
The generalized method of moments (GMM) approach of \citet{berry1995automobile} is to find $\theta_{jt}$ such that $p_{jt}=\hat{p}_{jt}$, i.e., the implied expected market share equals the observed share. Recall the similarity to the optimality condition \eqref{eq:optimality} of the Luce choice model. BLP propose the iteration 
\begin{align}
\label{eq:blp}
\theta_{jt}^{(m+1)} & =\theta_{jt}^{(m)}+\log\hat{p}_{jt}-\log p_{jt}(\theta^{(m)},\beta,\Gamma,\Sigma),
\end{align}
 and show that it is a \emph{contraction mapping}, whose fixed point is the desired estimates of $\theta_{jt}$. 
\begin{proposition}
    \label{prop:BLP}
    When $\beta,\Gamma,\Sigma \equiv 0$ and $\exp(\theta_{jt})\equiv s_{j}$, the GMM condition of BLP on market shares is equivalent to the optimality condition \eqref{eq:optimality} for a Luce choice model where all alternatives are available in every observation. Furthermore, the BLP algorithm is equivalent to Sinkhorn's algorithm in this model.
\end{proposition}
For a more detailed correspondence, see \citet{bonnet2022yogurts}. Importantly, \cref{prop:BLP} does not imply that the Luce choice model and Sinkhorn's algorithm is a strict special case of the BLP framework. The key difference is that BLP, and most discrete choice models in econometrics, implicitly assumes that the entire set of alternatives is always available in each observation. This assumption translates to a participation matrix $A$ in \eqref{eq:equation-system} that has 1's in all entries. In this setting, the MLE of $s_j$ is simply the empirical winning frequencies. %Note that \cref{ass:strong-connected} may still be violated in this case, highlighting again the gap between what is required for the ML estimation problem and by the matrix scaling problem.
On the other hand, while the Luce choice model allows different choice sets $S_i$ across observations, they do not include covariate information on the alternatives or decision makers, which is important in discrete choice modeling. One can reconcile this difference by relabeling alternatives with different covariates as \emph{distinct} items, and we leave investigations on further connections in this direction to future works.
\section{Regularization of Luce Choice Models and Matrix Balancing Problems}
\label{subsec:regularization}
In practice, many choice and ranking datasets may not satisfy \cref{ass:strong-connected}, which is required for the maximum likelihood estimation to be well-posed. Equivalently, for the matrix balancing problem, when a triplet $(A,p,q)$ does not satisfy \cref{ass:matrix-existence} and \cref{ass:matrix-uniqueness}, no finite scalings exist and Sinkhorn's algorithm may diverge. In this section, we discuss some regularization techniques to address these problems. They are easy to implement and require minimal modifications to Sinkhorn's algorithm. Nevertheless, they can be very useful in practice to regularize ill-posed problems. Given the equivalence between the problem of computing the MLE of Luce choice models and the problem of matrix balancing, our proposed regularization methods apply to both. 
\subsection{Regularization via Gamma Prior}
 As discussed in \cref{sec:formulations},  for a choice dataset to have a well-defined maximum likelihood estimator, it needs to satisfy \cref{ass:strong-connected}, which requires the directed comparison graph to be strongly connected. Although this condition is easy to verify, the question remains what one can do in case it does \emph{not} hold. As one possibility, we may introduce a prior on the parameters $s_j$, which serves as a regularization of the log-likelihood that results in a unique maximizer. Many priors are possible. For example, \citet{maystre2017choicerank}, following \citet{caron2012efficient}, use independent Gamma priors on $s_j$. In view of the fact that the unregularized problem and algorithm in \citet{maystre2017choicerank} is a special case of the Luce choice model and Sinkhorn's algorithm, we can also incorporate the Gamma prior to the Luce choice model \eqref{eq:log-likelihood} to address identification problems. 

More precisely, suppose now that each $s_j$ in the Luce choice model are i.i.d. Gamma$(\alpha,\beta)\propto s_j^{\alpha-1}e^{-\beta s_j}$. This leads to the following regularized log-likelihood:
\begin{align}
\label{eq:regularized-log-likelihood}
\ell^R(s):=\sum_{i=1}^{n}\log s_{j_i}-\log\sum_{k\in S_{i}}s_{k} +(\alpha-1)\sum_{j=1}^{m}\log s_j - \beta \sum_{j=1}^{m} s_j.
\end{align}
The corresponding first order condition is given by
\begin{align}
\label{eq:optimality-regularized}
\frac{W_j+\alpha-1}{n} & = \frac{1}{n} \left(\sum_{i\mid j\in S_{i}}R_i \frac{s_{j}}{\sum_{k\in S_{i}}s_{k}} +\beta s_j\right),
\end{align}
which leads to the following modified Sinkhorn's algorithm, which generalizes the ChoiceRank algorithm of \citet{maystre2017choicerank}: 
\begin{align}
\label{eq:regularized-sinkhorn}
 d^0 \leftarrow (q+\alpha -1)/ (A^Td^1+\beta),  \quad d^1 \leftarrow p/A d^0.
\end{align}
 The choice of $\beta$ determines the normalization of $s_j$. With $u_j=\log s_j$, we can show in a similar way as Theorem 2 of \citet{maystre2017choicerank} that \eqref{eq:regularized-log-likelihood} always has a unique maximizer whenever $\alpha>1$. Regarding the convergence, \citet{maystre2017choicerank} remarked that since their ChoiceRank algorithm can be viewed as an MM algorithm, it inherits the local linear convergence of MM algorithms \citep{lange2000optimization}, but ``a detailed investigation of convergence behavior is left for future works''. With the insights we develop in this paper, we can in fact provide an explanation for the validity of the Gamma priors from an optimization perspective. This perspective allows us to conclude directly that \eqref{eq:regularized-log-likelihood} always has a unique solution in the interior of the probability simplex, and that furthermore the iteration in \eqref{eq:regularized-sinkhorn} has global linear convergence. 
 Consider now the following regularized potential function
 \begin{align}
     g^R(d^0,d^1)	:= ((d^1)^{T}A+\beta (\mathbf{1}_m)^T)d^0-\sum_{i=1}^{n}p_{i}\log d^1_{i}-\sum_{j=1}^{m}(q_{j}+\alpha-1)\log d^0_{j}.
\end{align}
We can verify that by substituting the optimality condition of $d^1$ into $-g^R$, it reduces to the log posterior \eqref{eq:regularized-log-likelihood}. Moreover, the iteration \eqref{eq:regularized-sinkhorn} is precisely the alternating minimization algorithm for $g^R$. When $\alpha-1,\beta>0$, the reparameterized potential function
\begin{align*}
    \sum_{ij}(e^{-v_{i}}A_{ij}e^{u_{j}})+\beta\sum_{j}e^{u_{j}}+\sum_{i}p_{i}v_{i}-\sum_{j}(q_{j}+\alpha-1)u_{j}
\end{align*} 
is always coercive regardless of whether \cref{ass:matrix-existence} holds. Therefore, during the iterations \eqref{eq:regularized-sinkhorn}, $(u,v)$ stay \emph{bounded}. Moreover, the Hessian is 
\begin{align*} \begin{bmatrix}\sum_{j}e^{-v_{i}}A_{ij}e^{u_{j}} & -e^{-v_{i}}A_{ij}e^{u_{j}}\\
-e^{-v_{i}}A_{ij}e^{u_{j}} & \sum_{i}e^{-v_{i}}A_{ij}e^{u_{j}}+\beta e^{u_{j}}
\end{bmatrix}	\succ0,
\end{align*}
 which is now positive definite.  As a result, $g^{R}(u,v)$ is strongly convex and smooth, so that  \eqref{eq:regularized-sinkhorn} converges linearly. From the perspective of the matrix balancing problem, we have thus obtained a regularized version of Sinkhorn's algorithm, summarized in \cref{alg:regularized}, which is guaranteed to converge linearly to a finite solution $(D^1,D^0)$, even when \cref{ass:matrix-existence} does not hold for the triplet $(A,p,q)$. Moreover, the regularization also improves the convergence of Sinkhorn's algorithm even when it converges, as the regularized Hessian becomes more-behaved. This regularized algorithm could be very useful in practice to deal with real datasets that result in slow, divergent, or oscillating Sinkhorn iterations.
 \begin{algorithm}[tb]
\caption{Regularized Sinkhorn's Algorithm}
   \label{alg:regularized}
\begin{algorithmic}
   \STATE {\bfseries Input:}  $A, p, q,\alpha>1,\beta>0,\epsilon_{\text{tol}}$.
   \STATE {\bfseries initialize} $d^{0}\in\mathbb{R}_{++}^{m}$
   \REPEAT
   \STATE $d^{1} \leftarrow  p/( A d^0)$ 

   \STATE $d^{0}\leftarrow  (q+\alpha-1)/({A}^{T} d^{1}+\beta)$

   \STATE 
$\epsilon\leftarrow$  update of $(d^{0},d^1)$
\UNTIL{$\epsilon<\epsilon_{\text{tol}}$}
\end{algorithmic}
\end{algorithm}

\subsection{Regularization via Data Augmentation}
The connection between Bayesian methods and \emph{data augmentation} motivates us to also consider direct data augmentation methods. This is best illustrated in the choice modeling setting. Suppose for a choice dataset we construct participation matrix $A$, $p$ the counts of distinct choice sets, and $q$ the counts of each item being selected. We know that $(A,p,q)$ has a finite scaling solution if and only if  \cref{ass:strong-connected} holds, i.e., the directed comparison graph is strongly connected. We now propose the following modification of $(A,p,q)$ such that the resulting problem is always valid. 

First, if $A$ does not already contain a row equal to $\mathbf{1}_m^T$, i.e., containing all 1's, add this additional row to $A$. Call the resulting matrix $A'$. Then, expanding the dimension of $p$ if necessary, add $m\epsilon$ to the entry corresponding to $\mathbf{1}_m$, where we can assume for now that $\epsilon\geq 1$ is an integer.
This procedure effectively adds $m \epsilon$ ``observations'' that contain all $m$ items. For these additional observations, we let each item be selected exactly $\epsilon$ times. Luce's choice axiom guarantees that the exact choice of each artificial observation is irrelevant, and we just need to add $\epsilon \mathbf{1}_m$ to $q$. This represents augmenting each item with an additional $\epsilon$ ``wins'', resulting in the triplet $(A',p+(m\epsilon)\mathbf{e},q+\epsilon \mathbf{1}_m)$, where $\mathbf{e}$ is the one-hot indicator of the row $\mathbf{1}_m^T$ in $A'$. Now by construction, in any partition of $[m]$ into two non-empty subsets, any item from one subset is selected at least $\epsilon$ times over any item from the other subset. Therefore, \cref{ass:strong-connected} holds, and the maximum likelihood estimation problem, and equivalently the matrix balancing problem with $(A',p+(m\epsilon)\mathbf{e},q+\epsilon \mathbf{1}_m)$, is well-defined. This regularization method applies more generally to any non-negative $A$, even if it is not a participation matrix, i.e., binary. Although in the above construction based on choice dataset, $\epsilon$ is taken to be an integer, for the regularized matrix balancing problem with $(A',p+(m\epsilon)\mathbf{e},q+\epsilon \mathbf{1}_m)$, we can let $\epsilon \rightarrow 0$. %An interesting question is whether the sequence of solutions to the matrix balancing problems indexed by $\epsilon$ converges, and what point in the probability simplex it converges to, when the original triplet $(A,p,q)$ does not define a valid matrix balancing problem. 
%\zq{Future directions. Optimization: Can design regularization procedures that improves the algebraic connectivity for fixed data. Sampling: Can design sampling procedures that improve the algebraic connectivity for streaming data.}
\section{Empirical Study}
\label{sec:3}
To facilitate automated detection of configuration compatibility issues, we conducted an empirical study on the characteristics and symptoms of such issues in real-world Android apps.
The study aims at answering the following two research questions:
\begin{itemize}
	\item \textbf{RQ1 (Issue types and root causes):} What are the common types and the corresponding root causes of configuration compatibility issues?
	\item \textbf{RQ2 (Issue symptoms):} What are the common symptoms of configuration compatibility issues?
\end{itemize}

\subsection{Dataset Collection}
We collected bug-related code revisions from well-maintained open-source
Android apps as the empirical dataset.
To this end, we searched for suitable subjects on F-Droid
~\cite{fdroid}, which is a famous repository containing high-quality open-source Android apps.
Specifically, we selected subjects that meet the following criteria: (1)
maintaining a public issue tracking system, (2) receiving more than 500 stars on
GitHub~\cite{github} (popularity), and (3) pushing the latest git commit within 
the most recent three months (well-maintenance).
We chose these three criteria because the configuration compatibility issues
located in these selected subjects are likely to affect many users due to the popularity of the apps. 
As a result, 43 apps were returned. 

In order to locate the configuration compatibility issues affecting the 43
selected apps, we used the following two types of keywords to search for
related code revisions:
\sethlcolor{orange}
\begin{itemize}
	\item Keywords related to Android framework versions. In practice, developers often indicate the specific versions of the Android framework in which compatibility issues occur in the changelog.
	Specifically, we used two keywords, \texttt{API} (API level for short), and \texttt{Android [i]} where \texttt{[i]} stands for an integer, to search for Android system versions in changelogs.
	Besides, we also looked for code revisions that contain version-specific XML files, which are stored in the path that contains a version qualifier \texttt{v[L]}, where \texttt{[L]} represents the minimum API level applicable to the files.
	\item Keywords related to XML configuration files in Android apps.
	Specifically, we chose the following two keywords: \texttt{resource}, and \texttt{AndroidManifest}, so that they can effectively cover all types of XML configuration files supported in the Android framework.
\end{itemize}
In total, 2,376 unique code revisions were identified from the 43 apps after removing duplicates from the searching results.

{Next, we conducted manual analysis on the 2,376 code revisions to refine configuration compatibility issues. Specifically, we collected the code revisions in three steps. First, we screened out the code revisions unrelated to valid configuration compatibility issues because some irrelevant code revisions (e.g., introducing new app features) can be accidentally returned by our keyword-based search. Second, we collected the incompatibility-inducing attributes and XML elements from the revision-related commit logs, bug reports, or code diffs. 
To answer RQ1, the code changes related to the incompatibility-inducing attributes and XML elements should also be identified in the update history of the Android framework to investigate how these changes can cause issues. Third, to answer RQ2, we referred to the information of code revisions and online discussions of similar issues for the consequences when developers did not handle problematic XML elements or attributes well. Eventually, we collected 196 configuration compatibility issues from code revisions as the empirical dataset.}

\subsection{RQ1: Issue Root Causes}
\label{sec:RQ1}
\begin{table}[t]
	\caption{Common root causes of configuration compatibility issues}
	\begin{tabular}{lp{5cm}r}
		\toprule
		&\multicolumn{1}{c}{\textbf{Root Causes}}         & \textbf{Issue \#} \\ \hline
		Type 1&Unavailable configuration APIs & 116 (59.2\%)       \\
		Type 2&Inconsistent configuration APIs & 42 (21.4\%)       \\
		Type 3&Inconsistent Android internal XML configuration files & 19 (9.7\%)\\
		Type 4&Inconsistent attribute dependencies    & 9 (4.6\%)        \\
		Type 5&Inconsistent attribute usages             & 9 (4.6\%)    \\
		Type 6&Inconsistent attribute default values         & 1 (0.5\%)    \\
		\bottomrule
		\label{tab:issuecategorization}
	\end{tabular}
\end{table}

We elaborated on the six common types (or causes) identified from the 196 configuration compatibility issues as shown in Table~\ref{tab:issuecategorization}.

\begin{figure}[t]
	\centering
	\includegraphics[width=0.5\textwidth]{./img/layerdrawable.pdf}
	\caption{The Android framework code for loading the attribute value of \texttt{android:gravity} in the class \texttt{LayerDrawable}.}
	\label{fig:layerdrawable}
\end{figure}
\textbf{Unavailable configuration APIs.}
The Android framework loads attribute values by calling configuration APIs after parsing the XML tags in configuration files to \texttt{AttributeSet} or \texttt{TypedArray} objects.
Some statements invoking configuration APIs are introduced or removed as the Android framework evolves, resulting in an inability to load the associated configuration attribute values in a certain range of API levels.
In our empirical dataset, we found 116 (59.2\%) issues that were induced by such a type of code changes.
For example, the attribute value of \texttt{android:gravity} in Figure~\ref{fig:layerdrawable} is loaded by \texttt{LayerDrawable} to adjust the gravity for layer alignment starting from API level 23.
A navigation app OsmAnd~\cite{osmand} filed an issue in commit 1bbf578 that the attribute value of \texttt{android:gravity} is not loaded when running at an API level below 23, causing the incorrect display of graphic user interfaces. %There are 116 (59.2\%) issues falling into this issue type.

\textbf{Inconsistent configuration APIs.}~Configuration APIs in the Android framework are designed to load attribute values in specific data formats.
%The configuration APIs to load an attribute may vary across API levels.
Compatibility issues can happen when the configuration APIs to load an attribute vary across API levels.
The example in Figure~\ref{fig:configurationfile} between API levels 22 and 23 falls into this types.
Such an issue is caused by the style format attribute value of \texttt{android:color} not being loaded by the configuration API \texttt{getAttributeIntValue()} at API level 22.
The loading of \texttt{android:color} in an unsupported format can result in app crashes at API level 22. 
There are in total 42 (21.4\%) issues of this type.

\textbf{Inconsistent Android internal XML configuration files.}
The Android framework provides a set of internal XML configuration files that can be referenced by the developers as a part of their apps.
Compatibility issues can happen when there are changes in those internal XML configuration files as the Android framework evolves.
For example, QKSMS~\cite{qksms} commit 6b70a47 describes an issue caused by the internal XML configuration file \texttt{ic\_menu\_added.xml}, which was introduced at API level 23. There are 19 (9.7\%) issues falling into this issue type.

\begin{figure}[t]
	\centering
	\includegraphics[width=0.5\textwidth]{./img/datepicker.pdf}
	\caption{Code changes of loading \texttt{android:spinnersShown} in the class \texttt{DatePicker}.}
	\label{fig:datepicker}
\end{figure}
\textbf{Inconsistent attribute dependencies.}~
In the Android framework, there are dependencies across configuration attributes.
In other words, the runtime behaviors of one attribute depend on the value of other attributes.
We found nine compatibility issues that were induced by the inconsistent implementations on attribute dependencies among API levels. %of triggering conditions of invoking configuration APIs in the Android framework.
%Such changes can cause an attribute not to be loaded at specific API levels.
For example, open-keychain~\cite{openkeychain} reported an issue in commit be06c4c.
As Figure~\ref{fig:datepicker}(a) shows, developers have specified the value of \texttt{android:spinnersShown} as \texttt{true} to make the date picker widget be displayed in the spinner mode.
The attribute value cannot be loaded at API level 21 without specifying the value of \texttt{android:datePickerMode} as \texttt{spinner}, causing the date picker to be displayed in the calendar mode by default.
In this case, the app crashed at API level 21 due to a specific implementation of the date picker in the calendar mode.
The code changes in Figure~\ref{fig:datepicker}(b) show that starting from API level 21, the configuration API for \texttt{android:spinnersShown} is guarded by the conditional statement that checks whether the value of \texttt{android:datePickerMode} is \texttt{spinner}.
%Note that the existing path-insensitive analysis technique can fail to analyze the code changes falling into this type although the proportion of issues caused by such a pattern are small. 
%As the example in Figure~\ref{fig:frameworkprocess} shows, existing techniques cannot infer that \texttt{android:color} will be loaded in Line 5 without analyzing the conditional statement in Line 4. 
%Such a case can make the existing techniques derive spurious issue-inducing code changes as \texttt{android:color} is loaded in \texttt{ColorStateList} starting from API level 23.

\textbf{Inconsistent attribute usages.}
Compatibility issues can happen when there are inconsistent implementations on how the Android framework uses the attribute values after being loaded by configuration APIs.
As Figure~\ref{fig:frameworkprocess} shows, there is a change in processing the value of \texttt{android:color} (in Line 12 and 13) between API levels 21 and 22. The change avoids the \texttt{ArrayIndexOutOfBoundsException} when the Android framework parses the XML element in Figure~\ref{fig:configurationfile} at API level 22.
In total, there are nine (4.6\%) issues falling into this issue type.

\textbf{Inconsistent attribute default values.}
There is one (0.5\%) issue caused by inconsistent default values of the attribute \texttt{android:useLevel} in the XML tag \texttt{<shape>} between API levels 21 and 22, as reported in the commit a221442 of OsmAnd~\cite{osmand}.

\subsection{RQ2: Issue Symptoms}
We further analyzed the common issue symptoms as below.
Specifically, 89 (45.4\%) of the 196 issues in our empirical dataset can cause the apps to crash when triggering incompatibility-inducing XML configuration elements, as shown by the motivating example in Figure~\ref{fig:frameworkprocess}.
Another 88 issues (44.9\%) can induce an inconsistent look-and-feel across different API levels, affecting the apps' functionalities.
For example, the problem reported in the commit a221442 of OsmAnd~\cite{osmand} can force the progress bar to always show a full circle.
The remaining 19 issues (9.7\%) can cause inconsistent app behaviors beyond crashes and look-and-feel.
For example, the app Slide~\cite{slide} specified \texttt{android:requestLegacyExternalStorage} to make sure the app can still request for the external storage at an API level $\geq$ 29.
This shows that configuration compatibility issues can cause severe consequences to the app developers.

\section{Proofs}
\subsection{Proof of Theorem \ref{th:inexactLS1}}
We start from a similar argument as in \cite[proof of Therorem~2]{Blumen}. 
%proof of \cite[Theorem~2]{Blumen}. 
Set $g := 2\nabla f(x^{k-1})=2A^T(Ax^{k-1}-y)$ and $\g:=2\nablaa^{\nug} f(x^{k-1})= g+2\eg^k$ for some vector $\eg^k$ which by definition~\eqref{eq:grad} is bounded $\norm{\eg^k}\leq \nug^k$. It follows that
\ifCLASSOPTIONtwocolumn
\begin{align*} 
&\norm{y-Ax^k}^2-\norm{y-Ax^{k-1}}^2	\\
&= \langle x^k-x^{k-1},g \rangle +\norm{A(x^k-x^{k-1})}^2 \\
&\leq \langle x^k-x^{k-1},g \rangle + \MM \norm{x^k-x^{k-1}}^2, 
\end{align*}
\else
\begin{align*} 
\norm{y-Ax^k}^2-\norm{y-Ax^{k-1}}^2	&= \langle x^k-x^{k-1},g \rangle +\norm{A(x^k-x^{k-1})}^2 \\
&\leq \langle x^k-x^{k-1},g \rangle + \MM \norm{x^k-x^{k-1}}^2, 
\end{align*}
\fi
where the last inequality follows from the ULE property in Definition \ref{def:Lip}. Assuming $\MM \leq 1/\mu$, we have
\ifCLASSOPTIONtwocolumn
\begin{align*}
&\langle x^k-x^{k-1},g \rangle + \MM \norm{x^k-x^{k-1}}^2 \\
& \leq \langle x^k-x^{k-1},g \rangle + \frac{1}{\mu} \norm{x^k-x^{k-1}}^2\\
&= \langle x^k-x^{k-1},\g \rangle + \frac{1}{\mu} \norm{x^k-x^{k-1}}^2 - \langle x^k-x^{k-1},2\eg^k \rangle\\
& = \frac{1}{\mu} \norm{x^k-x^{k-1}+\frac{\mu}{2} \g }^2 - \frac{\mu}{4} \norm{\g}^2 - \langle x^k-x^{k-1},2\eg^k \rangle.
\end{align*}
\else
\begin{align*}
\langle x^k-x^{k-1},g \rangle + \MM \norm{x^k-x^{k-1}}^2 
& \leq \langle x^k-x^{k-1},g \rangle + \frac{1}{\mu} \norm{x^k-x^{k-1}}^2\\
&= \langle x^k-x^{k-1},\g \rangle + \frac{1}{\mu} \norm{x^k-x^{k-1}}^2 - \langle x^k-x^{k-1},2\eg^k \rangle\\
& = \frac{1}{\mu} \norm{x^k-x^{k-1}+\frac{\mu}{2} \g }^2 - \frac{\mu}{4} \norm{\g}^2 - \langle x^k-x^{k-1},2\eg^k \rangle.
\end{align*}
\fi
Due to the update rule of Algorithm \eqref{eq:inIP} and the inexact (fixed-precision) projection step, we have
\ifCLASSOPTIONtwocolumn
\begin{align*}
	&\norm{x^k-x^{k-1}+\frac{\mu}{2} \g }^2 \\
	&\leq  \norm{\pp_{\Cc}(x^{k-1}-\frac{\mu}{2} \g)-x^{k-1}+\frac{\mu}{2} \g }^2 +(\nup^k)^2\\
	&\leq \norm{x^\gt-x^{k-1}+\frac{\mu}{2} \g }^2 +(\nup^k)^2.
\end{align*}
\else
\begin{align*}
\norm{x^k-x^{k-1}+\frac{\mu}{2} \g }^2 
&\leq  \norm{\pp_{\Cc}(x^{k-1}-\frac{\mu}{2} \g)-x^{k-1}+\frac{\mu}{2} \g }^2 +(\nup^k)^2\\
&\leq \norm{x^\gt-x^{k-1}+\frac{\mu}{2} \g }^2 +(\nup^k)^2.
\end{align*}
\fi
The last inequality holds for any member of $\Cc$ and thus here for $x^\gt$. Therefore we can write
\ifCLASSOPTIONtwocolumn
\begin{align}
&\norm{y-Ax^k}^2-\norm{y-Ax^{k-1}}^2 \nonumber	\\
%&=\langle x^{t+1}-x^t,g \rangle + \frac{1}{\mu} \norm{x^{t+1}-x^t}^2 \nonumber\\
&\leq \frac{1}{\mu} \norm{x^\gt-x^{k-1}+\frac{\mu}{2} \g }^2 - \frac{\mu}{4} \norm{\g}^2 \nonumber\\
&\qquad - \langle x^k-x^{k-1},2\eg^k \rangle +(\frac{\nup^k}{\sqrt\mu})^2 \nonumber \\
&= \langle x^\gt-x^{k-1},\g \rangle + \frac{1}{\mu} \norm{x^\gt-x^{k-1}}^2 \nonumber\\
&\qquad - \langle x^k-x^{k-1},2\eg^k \rangle 
 +(\frac{\nup^k}{\sqrt\mu})^2\nonumber\\
%&= \langle x^\gt-x^{k-1},g \rangle + \frac{1}{\mu} \norm{x^\gt-x^{k-1}}^2 - \langle x^k-x^*,2\eg^k \rangle +\frac{\nup^k}{\mu}\nonumber\\
&\leq \langle x^\gt-x^{k-1},g \rangle + \frac{1}{\mu} \norm{x^\gt-x^{k-1}}^2 \nonumber\\
&\qquad+2\nug^k\norm{x^k-x^*} +(\frac{\nup^k}{\sqrt\mu})^2. \label{eq:p1b2}
\end{align}
\else
\begin{align}
\norm{y-Ax^k}^2-\norm{y-Ax^{k-1}}^2 \nonumber	
%&=\langle x^{t+1}-x^t,g \rangle + \frac{1}{\mu} \norm{x^{t+1}-x^t}^2 \nonumber\\
&\leq \frac{1}{\mu} \norm{x^\gt-x^{k-1}+\frac{\mu}{2} \g }^2 - \frac{\mu}{4} \norm{\g}^2 
 - \langle x^k-x^{k-1},2\eg^k \rangle +(\frac{\nup^k}{\sqrt\mu})^2 \nonumber \\
&= \langle x^\gt-x^{k-1},\g \rangle + \frac{1}{\mu} \norm{x^\gt-x^{k-1}}^2 
 - \langle x^k-x^{k-1},2\eg^k \rangle 
+(\frac{\nup^k}{\sqrt\mu})^2\nonumber\\
%&= \langle x^\gt-x^{k-1},g \rangle + \frac{1}{\mu} \norm{x^\gt-x^{k-1}}^2 - \langle x^k-x^*,2\eg^k \rangle +\frac{\nup^k}{\mu}\nonumber\\
&\leq \langle x^\gt-x^{k-1},g \rangle + \frac{1}{\mu} \norm{x^\gt-x^{k-1}}^2 
+2\nug^k\norm{x^k-x^*} +(\frac{\nup^k}{\sqrt\mu})^2. \label{eq:p1b2}
\end{align}
\fi
The last line replaces $\g= g+2\eg^k$ and uses the Cauchy-Schwartz inequality. 


Similarly we use the LLE property in Definition \ref{def:Lip} to obtain an upper bound on $ \langle x^\gt-x^{k-1},g \rangle$:
\begin{align*} 
\langle x^\gt-x^{k-1},g \rangle 	&= w^2-\norm{y-Ax^{k-1}}^2 +\norm{A(x_0-x^{k-1})}^2 \\
&\leq w^2 -\norm{y-Ax^{k-1}}^2 +\mmx\norm{x^\gt-x^{k-1}}^2,
\end{align*}
where $w=\norm{ y-Ax^\gt}$. Replacing this bound in \eqref{eq:p1b2} and cancelling $-\norm{y-Ax^{k-1}}^2$ from both sides of the inequality yields
\ifCLASSOPTIONtwocolumn
\begin{align}
&\norm{y-Ax^k}^2- 2\nug^k\norm{x^k-x^\gt}\nonumber \\ 
&\leq \left(\frac{1}{\mu}-\mmx \right)\norm{x^{k-1}-x^\gt}^2 + (\frac{\nup^k}{\sqrt\mu})^2+w^2. \label{eq:p1b3}
\end{align}
\else
\begin{align}
\norm{y-Ax^k}^2- 2\nug^k\norm{x^k-x^\gt}
\leq \left(\frac{1}{\mu}-\mmx \right)\norm{x^{k-1}-x^\gt}^2 + (\frac{\nup^k}{\sqrt\mu})^2+w^2. \label{eq:p1b3}
\end{align}
\fi
We continue to lower bound the left-hand side of this inequality:
\ifCLASSOPTIONtwocolumn
\begin{align*}
&\norm{y-Ax^k}^2- 2\nug^k\norm{x^k-x^\gt}\\
&= \norm{A(x^k-x^\gt)}^2+w^2-2\langle y-Ax^\gt, A(x^k-x^\gt)\rangle\\
&- 2\nug^k\norm{x^k-x^\gt} \\
&\geq \norm{A(x^k-x^\gt)}^2+w^2-2w \norm{A(x^k-x^\gt)}\\
&- 2\nug^k\norm{x^k-x^\gt} \\
& \geq \mmx\norm{x^k-x^\gt}^2+w^2-2(w \sqrt{\MM}+\nug^k)\norm{x^k-x^\gt}\\
&= \left(\sqrt{\mmx}\norm{x^k-x^\gt}-\frac{\nug^k}{\sqrt{\mmx}}- \sqrt{\frac{\MM}{\mmx}}w\right)^2 \\
&- (\frac{\nug^k}{\sqrt{\mmx}})^2 -(\frac{\MM}{\mmx}-1)w^2.
\end{align*}
\else
\begin{align*}
\norm{y-Ax^k}^2- 2\nug^k\norm{x^k-x^\gt}
&= \norm{A(x^k-x^\gt)}^2+w^2-2\langle y-Ax^\gt, A(x^k-x^\gt)\rangle- 2\nug^k\norm{x^k-x^\gt} \\
&\geq \norm{A(x^k-x^\gt)}^2+w^2-2w \norm{A(x^k-x^\gt)}
- 2\nug^k\norm{x^k-x^\gt} \\
& \geq \mmx\norm{x^k-x^\gt}^2+w^2-2(w \sqrt{\MM}+\nug^k)\norm{x^k-x^\gt}\\
&= \left(\sqrt{\mmx}\norm{x^k-x^\gt}-\frac{\nug^k}{\sqrt{\mmx}}- \sqrt{\frac{\MM}{\mmx}}w\right)^2 - (\frac{\nug^k}{\sqrt{\mmx}})^2 -(\frac{\MM}{\mmx}-1)w^2.
\end{align*}
\fi
The first inequality uses the Cauchy-Schwartz's and the second inequality follows from the ULE and LLE properties. Using this bound together with \eqref{eq:p1b3} we get
\ifCLASSOPTIONtwocolumn
\begin{align*}
&\left(\sqrt{\mmx}\norm{x^k-x^\gt}-\frac{\nug^k}{\sqrt{\mmx}}- \sqrt{\frac{\MM}{\mmx}}w\right)^2\\
&\leq \left(\frac{1}{\mu}-\mmx \right)\norm{x^{k-1}-x^\gt}^2 + (\frac{\nug^k}{\sqrt{\mmx}})^2+ (\frac{\nup^k}{\sqrt\mu})^2+\frac{\MM}{\mmx}w^2 \\
&\leq \left(\sqrt{\frac{1}{\mu}-\mmx} \norm{x^{k-1}-x^\gt} + \frac{\nug^k}{\sqrt{\mmx}}+ \frac{\nup^k}{\sqrt\mu}+\sqrt{\frac{\MM}{\mmx}}w \right)^2.
\end{align*}
\else
\begin{align*}
\left(\sqrt{\mmx}\norm{x^k-x^\gt}-\frac{\nug^k}{\sqrt{\mmx}}- \sqrt{\frac{\MM}{\mmx}}w\right)^2
&\leq \left(\frac{1}{\mu}-\mmx \right)\norm{x^{k-1}-x^\gt}^2 + (\frac{\nug^k}{\sqrt{\mmx}})^2+ (\frac{\nup^k}{\sqrt\mu})^2+\frac{\MM}{\mmx}w^2 \\
&\leq \left(\sqrt{\frac{1}{\mu}-\mmx} \norm{x^{k-1}-x^\gt} + \frac{\nug^k}{\sqrt{\mmx}}+ \frac{\nup^k}{\sqrt\mu}+\sqrt{\frac{\MM}{\mmx}}w \right)^2.
\end{align*}
\fi
The last inequality assumes $\mu\leq \mmx^{-1}$ which holds since we previously assumed $\mu\leq \MM^{-1}$. As a result we deduce that
\begin{align}
\norm{x^k-x^\gt}\leq \rho \norm{x^{k-1}-x^\gt} + \nut^k + 2\frac{\sqrt{\MM}}{\mmx}w \label{eq:p1b4}
\end{align}
for $\rho$ and $\nut^k$ defined in Theorem \ref{th:inexactLS1}. Applying this bound recursively (and setting $x^0=0$) completes the proof:
\begin{align*}
\norm{x^k-x^\gt}\leq \rho^k \norm{x^\gt} + \sum_{i=1}^k \rho^{k-i} \nut^i + \frac{2\sqrt{\MM}}{\mmx(1-\rho)}w.
\end{align*} 
Note that for convergence we require $\rho<1$ and therefore, a lower bound on the step size which is $\mu> (2\mmx)^{-1}$. 

\subsection{Proof of Corollary~\ref{cor:decay}}
Following the error bound \eqref{eq:errbound} derived in   Theorem~\ref{th:inexactLS1} and by setting $\nut^k\leq C r^k$ we obtain:
		\eq{
			\norm{x^{k}-x^\gt}\leq  \rho^k \left(\norm{x^\gt}+C\sum_{i=1}^k (r/\rho)^{i}  \right)+ \frac{2\sqrt{\MM}}{\mmx(1-\rho)}w,			
		}
which for $r<\rho$ it implies 		
		\eq{
			\norm{x^{k}-x^\gt}\leq 
			 \rho^k \left(\norm{x^\gt}+\frac{C}{1-r/\rho}\right)+ \frac{2\sqrt{\MM}}{\mmx(1-\rho)}w,
		}
and for $r>\rho$ implies %and following \eqref{eq:errbound} we get		
\begin{align*}
\norm{x^{k}-x^\gt}&\leq  \rho^k \norm{x^\gt}+C r^k \sum_{i=1}^k (\rho/r)^{k-i}  + \frac{2\sqrt{\MM}}{\mmx(1-\rho)}w\\
&\leq r^k \left(\norm{x^\gt}+\frac{C}{1-\rho/r}\right)+ \frac{2\sqrt{\MM}}{\mmx(1-\rho)}w,	
\end{align*}
and for $r=\rho$ we immediately get
\eq{
\norm{x^{k}-x^\gt}\leq  \rho^k \norm{x^\gt}+C k \rho^k + \frac{2\sqrt{\MM}}{\mmx(1-\rho)}w.	
}
Note that there exists a constant $c$ such that for an arbitrary small $\xi>0$ it holds $k\rho^k\leq c(\rho+\xi)^k$. Therefore we also achieve a linear convergence for the case $r=\rho$.
\subsection{Proof of Theorem \ref{th:inexactLS2}}
As before set $g= 2A^T(Ax^{k-1}-y)$ and $\g= g+2\eg^k$ for some bounded gradient error vector $\eg^k$ i.e. $\norm{\eg^k}\leq \nug^k$. Note that 
here the update rule of Algorithm \eqref{eq:inIP2} uses the  $(1+\epsilon)$-approximate projection  which by definition \eqref{eq:eproj} implies
\ifCLASSOPTIONtwocolumn
\begin{align*}
&\norm{x^k-x^{k-1}+\frac{\mu}{2} \g }^2 =  \norm{\pp^{\epsilon}_{\Cc}(x^{k-1}-\frac{\mu}{2} \g)-x^{k-1}+\frac{\mu}{2} \g }^2\\
&\leq  (1+\epsilon)^2\norm{\pp_{\Cc}(x^{k-1}-\frac{\mu}{2} \g)-x^{k-1}+\frac{\mu}{2} \g }^2\\
&\leq \norm{x^\gt-x^{k-1}+\frac{\mu}{2} \g }^2 + \phi(\epsilon)^2\frac{\mu^2}{4}\norm{\g}^2
\end{align*}
\else
\begin{align*}
\norm{x^k-x^{k-1}+\frac{\mu}{2} \g }^2 &=  \norm{\pp^{\epsilon}_{\Cc}(x^{k-1}-\frac{\mu}{2} \g)-x^{k-1}+\frac{\mu}{2} \g }^2\\
&\leq  (1+\epsilon)^2\norm{\pp_{\Cc}(x^{k-1}-\frac{\mu}{2} \g)-x^{k-1}+\frac{\mu}{2} \g }^2\\
&\leq \norm{x^\gt-x^{k-1}+\frac{\mu}{2} \g }^2 + \phi(\epsilon)^2\frac{\mu^2}{4}\norm{\g}^2
\end{align*}
\fi
where $\phi(\epsilon):=\sqrt{2\epsilon+\epsilon^2}$. For the last inequality we replace $\pp_{\Cc}(x^{k-1}-\frac{\mu}{2} \g)$ with two feasible points $x^\gt,x^{k-1}\in \Cc$. 

As a result by only replacing $\nug^k$ with $\mu\phi(\epsilon)\norm{\g}/2$, we can follow identical steps as for the proof of Theorem \ref{th:inexactLS1} up to \eqref{eq:p1b4}, revise the definition of $\nut^k:={2\nug^k}/{\mmx} + {\sqrt{\mu}\phi(\epsilon)\norm{\g}}/(2\sqrt{{\mmx}})$ and write
\ifCLASSOPTIONtwocolumn
\begin{align*}
\norm{x^k-x^\gt}\leq& \sqrt{\frac{1}{\mu\mmx}-1} \norm{x^{k-1}-x^\gt} \\
&+ \frac{2\nug^k}{\mmx} +\frac{\phi(\epsilon)}{2} \sqrt{\frac{\mu}{\mmx}}\norm{\g} + 2\frac{\sqrt{\MM}}{\mmx}w. 
\end{align*}
\else
\begin{align*}
\norm{x^k-x^\gt}\leq \sqrt{\frac{1}{\mu\mmx}-1} \norm{x^{k-1}-x^\gt} 
+ \frac{2\nug^k}{\mmx} +\frac{\phi(\epsilon)}{2} \sqrt{\frac{\mu}{\mmx}}\norm{\g} + 2\frac{\sqrt{\MM}}{\mmx}w. 
\end{align*}
\fi
Note that so far we only assumed $\mu\leq \MM^{-1}$. 

On the other hand by triangle inequality we have
\begin{align*}
	\norm{\g}&\leq \norm{g}+2\nug^k\\
	&\leq 2\norm{A^TA(x^{k-1}-x^\gt)}+2\norm{A^T(y-Ax^\gt)}+2\nug^k \\
	&\leq 2\sqrt \MM\vertiii{A}\norm{(x^{k-1}-x^\gt)}+2\vertiii{A}w+2\nug^k\\
	&\leq 2\sqrt{ 1/\mu}\vertiii{A}\norm{(x^{k-1}-x^\gt)}+2\vertiii{A}w+2\nug^k.
\end{align*}
The third inequality uses the ULE property and the last one holds since $\mu\leq \MM^{-1}$.
Therefore, we get
\ifCLASSOPTIONtwocolumn
\begin{align*}
&\norm{x^k-x^\gt}\leq
\left(\sqrt{\frac{1}{\mu\mmx}-1}+ \phi(\epsilon)\frac{\vertiii{A}}{\sqrt{\mmx}}\right) \norm{x^{k-1}-x^\gt} \\
&+ \left( \frac{2}{\mmx} +\phi(\epsilon){\sqrt{\frac{\mu}{\mmx}}}\right) \nug^k 
+ \left( 2\frac{\sqrt{\MM}}{\mmx}+ \phi(\epsilon)\sqrt{\frac{\mu}{\mmx}} \vertiii{A} \right)w. 
\end{align*}
\else
\begin{align*}
\norm{x^k-x^\gt}\leq&
\left(\sqrt{\frac{1}{\mu\mmx}-1}+ \phi(\epsilon)\frac{\vertiii{A}}{\sqrt{\mmx}}\right) \norm{x^{k-1}-x^\gt} \\
&+ \left( \frac{2}{\mmx} +\phi(\epsilon){\sqrt{\frac{\mu}{\mmx}}}\right) \nug^k 
+ \left( 2\frac{\sqrt{\MM}}{\mmx}+ \phi(\epsilon)\sqrt{\frac{\mu}{\mmx}} \vertiii{A} \right)w. 
\end{align*}
\fi
Based on assumption $\phi(\epsilon)\frac{\vertiii{A}}{\sqrt{\mmx}}\leq \delta$ of the theorem  we can deduce
\ifCLASSOPTIONtwocolumn
\begin{align*}
\norm{x^k-x^\gt}\leq&
\rho \norm{x^{k-1}-x^\gt} + \left( \frac{2}{\mmx} +\frac{\sqrt \mu}{\vertiii{A}} \delta\right) \nug^k \\
&
+ \left( 2\frac{\sqrt{\MM}}{\mmx}+\sqrt{\mu}\delta \right)w 
\end{align*}
\else
\begin{align*}
\norm{x^k-x^\gt}\leq
\rho \norm{x^{k-1}-x^\gt} + \left( \frac{2}{\mmx} +\frac{\sqrt \mu}{\vertiii{A}} \delta\right) \nug^k 
+ \left( 2\frac{\sqrt{\MM}}{\mmx}+\sqrt{\mu}\delta \right)w 
\end{align*}
\fi
where $\rho=\sqrt{\frac{1}{\mu\mmx}-1}+\delta$.

Applying this bound recursively (and setting $x^0=0$) completes the proof:
\eq{
\norm{x^{k}-x^\gt}\leq  \rho^k \norm{x^\gt}+\kappa_g \sum_{i=1}^k \rho^{k-i} \nug^i+ \frac{\kappa_w}{1-\rho}w
}
for $\kappa_g, \kappa_w$ defined in Theorem \ref{th:inexactLS2}. The condition for convergence is $\rho<1$ which implies $\delta<1$ and a lower bound on the step size which is $\mu> (\mmx+(1-\delta)^2\mmx)^{-1}$. 
		
 \end{APPENDICES}
%%%%%%%%%%%%%%%%%%%%%%%%%%%%%%%%%%%%%%%%%%%%%%%%%%%%%%%%%%%%%%%%%%%%%%%%%%%%%%%
%%%%%%%%%%%%%%%%%%%%%%%%%%%%%%%%%%%%%%%%%%%%%%%%%%%%%%%%%%%%%%%%%%%%%%%%%%%%%%%


\end{document}

