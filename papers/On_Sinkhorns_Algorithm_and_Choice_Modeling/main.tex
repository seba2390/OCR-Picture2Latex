\documentclass[nonblindrev]{informs3}
\linespread{1.0}
\setlength{\parindent}{0pt}
%%\DoubleSpacedXI % Made default 4/4/2014 at request
\OneAndAHalfSpacedXI % current default line spacing
%%\OneAndAHalfSpacedXII
%%\DoubleSpacedXII
% Recommended, but optional, packages for figures and better typesetting:
\usepackage{microtype}
\usepackage{graphicx}
\usepackage{algorithm}
\usepackage{algorithmic}
\usepackage{subfigure}
\usepackage{booktabs} % for professional tables
\usepackage{multirow}
\usepackage{changepage}


\usepackage{hyperref}
\usepackage{natbib}
 \bibpunct[, ]{(}{)}{,}{a}{}{,}%
 \def\bibfont{\small}%
 \def\bibsep{\smallskipamount}%
 \def\bibhang{24pt}%
 \def\newblock{\ }%
 \def\BIBand{and}%

\usepackage{comment}

% Attempt to make hyperref and algorithmic work together better:
\newcommand{\theHalgorithm}{\arabic{algorithm}}

\OneAndAHalfSpacedXI

%% Setup of theorem styles. Outcomment only one.
%% Preferred default is the first option.
\TheoremsNumberedThrough     % Preferred (Theorem 1, Lemma 1, Theorem 2)
%\TheoremsNumberedByChapter  % (Theorem 1.1, Lema 1.1, Theorem 1.2)
\ECRepeatTheorems

% For theorems and such
\usepackage{amsmath}
\usepackage{amssymb}
\usepackage{mathtools}
%\usepackage{amsthm}

\usepackage[capitalize,noabbrev]{cleveref}
\crefname{assumption}{Assumption}{assumptions}
%%%%%%%%%%%%%%%%%%%%%%%%%%%%%%%%
% THEOREMS
%%%%%%%%%%%%%%%%%%%%%%%%%%%%%%%%
\theoremstyle{plain}
% \newtheorem{theorem}{Theorem}[section]
% \newtheorem{proposition}[theorem]{Proposition}
% \newtheorem{lemma}[theorem]{Lemma}
% \newtheorem{corollary}[theorem]{Corollary}
% %\theoremstyle{definition}
% \newtheorem{definition}[theorem]{Definition}
 % \newtheorem{assumption}[theorem]{Assumption}
% \newtheorem{remark}[theorem]{Remark}

\newcommand{\out}{\text{out}}
\newcommand{\inn}{\text{in}}

\newcommand{\ju}[1]{{\color{magenta} [JU: #1]}}
\newcommand{\zq}[1]{{\color{blue} [ZQ: #1]}}

\begin{document}

 \RUNAUTHOR{Qu, Galichon, Ugander}

 \RUNTITLE{Sinkhorn's Algorithm and Choice Modeling}

% Full title. Sample:
% \TITLE{Bundling Information Goods of Decreasing Value}
% Enter the full title:
\TITLE{On Sinkhorn's Algorithm and Choice Modeling}

% It is OKAY to include author information, even for blind
% submissions: the style file will automatically remove it for you
% unless you've provided the [accepted] option to the icml2023
% package.

% List of affiliations: The first argument should be a (short)
% identifier you will use later to specify author affiliations
% Academic affiliations should list Department, University, City, Region, Country
% Industry affiliations should list Company, City, Region, Country

% You can specify symbols, otherwise they are numbered in order.
% Ideally, you should not use this facility. Affiliations will be numbered
% in order of appearance and this is the preferred way.
\ARTICLEAUTHORS{%
\AUTHOR{Zhaonan Qu}
\AFF{Department of Economics, Stanford University, California, USA \EMAIL{zhaonanq@stanford.edu}} 
\AUTHOR{Alfred Galichon}
\AFF{Department of Mathematics and Department of Economics, New York University, New York, USA\\ Department of Economics, Sciences Po, Paris, France \EMAIL{alfred.galichon@nyu.edu}}
\AUTHOR{Johan Ugander}
\AFF{Department of Management Science \& Engineering, Stanford University, California, USA \EMAIL{jugander@stanford.edu}}
% Enter all authors
} % end of the block

\ABSTRACT{For a broad class of choice and ranking models based on Luce's choice axiom, 
including the Bradley--Terry--Luce and Plackett--Luce models, we show that the associated maximum likelihood estimation problems are equivalent to a classic matrix balancing problem with target row and column sums. This perspective opens doors between two seemingly unrelated research areas, and allows us to unify existing algorithms in the choice modeling literature as special instances or analogs of Sinkhorn's celebrated algorithm for matrix balancing. 
We draw inspirations from these connections and resolve important open problems on the study of Sinkhorn's algorithm. We first prove the global linear convergence of Sinkhorn's algorithm for non-negative matrices whenever finite solutions to the matrix balancing problem exist. We characterize this global rate of convergence in terms of the algebraic connectivity of the bipartite graph constructed from data. Next, we also derive the sharp asymptotic rate of linear convergence, which generalizes a classic result of Knight (2008), but with a more explicit analysis that exploits an intrinsic orthogonality structure. To our knowledge, these are the first quantitative linear convergence results for Sinkhorn's algorithm for general non-negative matrices and positive marginals. 
The connections we establish in this paper between matrix balancing and choice modeling could help motivate further transmission of ideas and interesting results in both directions.
}
\KEYWORDS{Choice Modeling, Matrix Balancing, Sinkhorn's Algorithm, Algebraic Connectivity}
\maketitle

% !TEX root = ../arxiv.tex

Unsupervised domain adaptation (UDA) is a variant of semi-supervised learning \cite{blum1998combining}, where the available unlabelled data comes from a different distribution than the annotated dataset \cite{Ben-DavidBCP06}.
A case in point is to exploit synthetic data, where annotation is more accessible compared to the costly labelling of real-world images \cite{RichterVRK16,RosSMVL16}.
Along with some success in addressing UDA for semantic segmentation \cite{TsaiHSS0C18,VuJBCP19,0001S20,ZouYKW18}, the developed methods are growing increasingly sophisticated and often combine style transfer networks, adversarial training or network ensembles \cite{KimB20a,LiYV19,TsaiSSC19,Yang_2020_ECCV}.
This increase in model complexity impedes reproducibility, potentially slowing further progress.

In this work, we propose a UDA framework reaching state-of-the-art segmentation accuracy (measured by the Intersection-over-Union, IoU) without incurring substantial training efforts.
Toward this goal, we adopt a simple semi-supervised approach, \emph{self-training} \cite{ChenWB11,lee2013pseudo,ZouYKW18}, used in recent works only in conjunction with adversarial training or network ensembles \cite{ChoiKK19,KimB20a,Mei_2020_ECCV,Wang_2020_ECCV,0001S20,Zheng_2020_IJCV,ZhengY20}.
By contrast, we use self-training \emph{standalone}.
Compared to previous self-training methods \cite{ChenLCCCZAS20,Li_2020_ECCV,subhani2020learning,ZouYKW18,ZouYLKW19}, our approach also sidesteps the inconvenience of multiple training rounds, as they often require expert intervention between consecutive rounds.
We train our model using co-evolving pseudo labels end-to-end without such need.

\begin{figure}[t]%
    \centering
    \def\svgwidth{\linewidth}
    \input{figures/preview/bars.pdf_tex}
    \caption{\textbf{Results preview.} Unlike much recent work that combines multiple training paradigms, such as adversarial training and style transfer, our approach retains the modest single-round training complexity of self-training, yet improves the state of the art for adapting semantic segmentation by a significant margin.}
    \label{fig:preview}
\end{figure}

Our method leverages the ubiquitous \emph{data augmentation} techniques from fully supervised learning \cite{deeplabv3plus2018,ZhaoSQWJ17}: photometric jitter, flipping and multi-scale cropping.
We enforce \emph{consistency} of the semantic maps produced by the model across these image perturbations.
The following assumption formalises the key premise:

\myparagraph{Assumption 1.}
Let $f: \mathcal{I} \rightarrow \mathcal{M}$ represent a pixelwise mapping from images $\mathcal{I}$ to semantic output $\mathcal{M}$.
Denote $\rho_{\bm{\epsilon}}: \mathcal{I} \rightarrow \mathcal{I}$ a photometric image transform and, similarly, $\tau_{\bm{\epsilon}'}: \mathcal{I} \rightarrow \mathcal{I}$ a spatial similarity transformation, where $\bm{\epsilon},\bm{\epsilon}'\sim p(\cdot)$ are control variables following some pre-defined density (\eg, $p \equiv \mathcal{N}(0, 1)$).
Then, for any image $I \in \mathcal{I}$, $f$ is \emph{invariant} under $\rho_{\bm{\epsilon}}$ and \emph{equivariant} under $\tau_{\bm{\epsilon}'}$, \ie~$f(\rho_{\bm{\epsilon}}(I)) = f(I)$ and $f(\tau_{\bm{\epsilon}'}(I)) = \tau_{\bm{\epsilon}'}(f(I))$.

\smallskip
\noindent Next, we introduce a training framework using a \emph{momentum network} -- a slowly advancing copy of the original model.
The momentum network provides stable, yet recent targets for model updates, as opposed to the fixed supervision in model distillation \cite{Chen0G18,Zheng_2020_IJCV,ZhengY20}.
We also re-visit the problem of long-tail recognition in the context of generating pseudo labels for self-supervision.
In particular, we maintain an \emph{exponentially moving class prior} used to discount the confidence thresholds for those classes with few samples and increase their relative contribution to the training loss.
Our framework is simple to train, adds moderate computational overhead compared to a fully supervised setup, yet sets a new state of the art on established benchmarks (\cf \cref{fig:preview}).

\section{Related Work}
\label{sec:related-works}
This section includes an extensive review of related works in choice modeling and matrix balancing. Well-versed readers may skip ahead to the mathematical preliminaries (\cref{sec:formulations}) and our core results (\cref{sec:equivalence,sec:linear-convergence}).
\subsection{Choice Modeling}
Methods for aggregating choice and comparison data usually take one of two closely related approaches: maximum likelihood estimation of a statistical model or ranking according to the stationary distributions of a random walk on a Markov chain. Recent connections between maximum likelihood and spectral methods have put these two classes of approaches in increasingly close conversation with each other.

\textbf{Spectral Methods.}
The most well-known spectral method for rank aggregation is perhaps
the PageRank algorithm \citep{page1999pagerank}, which ranks web pages
based on the stationary distribution of a random walk on a
hyperlink graph. The use of stationary distributions also features in the work of \citet{dwork2001rank}, the Rank Centrality (RC) algorithm \citep{negahban2012iterative,negahban2016rank}, which generates consistent estimates for the Bradley--Terry--Luce
pairwise comparison model under assumptions on the sampling frame, and the Luce Spectral Ranking (LSR) and iterative LSR (I-LSR) algorithms of \citet{maystre2015fast} for choices from pairs as well as larger sets. Following that work, \citet{agarwal2018accelerated} proposed the Accelerated Spectral Ranking (ASR) algorithm with provably faster mixing times than RC and LSR, and better sample complexity bounds than \citet{negahban2016rank}.
\citet{knight2008sinkhorn} is an intriguing work partially motivated by \citet{page1999pagerank} that applies Sinkhorn's algorithm, which is central to the current work, to compute authority and hub scores similar to
those proposed by \citet{kleinberg1999authoritative} and \citet{tomlin2003new}, although the focus in \citet{knight2008sinkhorn} is on Markov chains rather than maximum likelihood estimation of choice models.
For ranking data, \citet{soufiani2013generalized} decompose rankings into pairwise comparisons and develop consistent estimators for Plackett--Luce models based on a generalized method of moments. Other notable works that make connections between Markov chains and 
choice modeling include \citet{blanchet2016markov} and \citet{ragain2016pairwise}.

\textbf{Maximum Likelihood Methods.}
Maximum likelihood estimation of the Bradley--Terry--Luce model 
dates back to \citet{zermelo1929berechnung}, \citet{dykstra1956note}, and \citet{ford1957solution}, which all give variants of the same iterative algorithm and prove its convergence to the MLE when the directed comparison graph is strongly connected. 
Much later, \citet{hunter2004mm} observed that their
algorithms are instances of the class of minorization-maximization (MM) algorithms, and develops MM algorithms for the Plackett--Luce model for ranking data, among others. \citet{vojnovic2020convergence} further investigate the convergence of the MM algorithm for choice models. \citet{newman2023efficient} proposes an alternative to the classical iterative algorithm for pairwise comparisons based on a reformulated moment condition, achieving impressive empirical speedups. \citet{negahban2012iterative} is arguably the first work that connects maximum likelihood estimation to Markov chains, followed by \citet{maystre2015fast}, whose spectral method is based on a balance equation interpretation of the optimality condition.
\citet{kumar2015inverting} consider the problem of inverting the stationary distribution of a Markov
chain, and embed the maximum likelihood problem of the Luce choice model into this
framework, where the MLEs parameterize the desired transition matrix. \citet{maystre2017choicerank} consider the estimation of a network choice model with similarly parameterized random walks. Lastly, a vast literature in econometrics on discrete choice also considers different aspects of the ML estimation problem. In particular, the present paper is related to the Berry--Levinsohn--Pakes (BLP) framework of \citet{berry1995automobile}, well-known in econometrics. The matrix balancing interpretation of maximum likelihood estimation of choice models that we develop in this paper connects many of the aforementioned works.

Besides the optimization of a given maximum likelihood problem, there have also been extensive studies of the
statistical properties of the maximum likelihood estimator for these choice problems. \citet{hajek2014minimax}
prove that the MLE is minimax optimal for $k$-way rankings, and
\citet{rajkumar2014statistical} show that the MLE for Bradley--Terry--Luce can recover the correct \emph{ranking} under model mis-specification
when noise is ``bounded''. As a byproduct of analysis for a context-dependent generalization of the Luce choice model, \citet{seshadri2020learning} obtain tight expected risk and tail risk bounds for the MLEs of Luce choice models (which they call MNL) and Plackett--Luce ranking models, extending and improving upon previous works by \citet{hajek2014minimax,shah2015estimation,vojnovic2016parameter}. 

Our present work is primarily concerned with the optimization aspects of the maximum likelihood problem. Nonetheless, the importance of \emph{algebraic connectivity}---as quantified by the Fiedler eigenvalue \citep{fiedler1973algebraic}---in the results of \citet{shah2015estimation,seshadri2020learning} as well as \citet{vojnovic2020convergence} provides motivations in our convergence analysis of Sinkhorn's algorithm for matrix balancing. 

Lastly, a short note on terminology. Even though a choice model based on \eqref{eq:Luce} is technically a  ``multinomial logit model'' with only intercept terms \citep{mcfadden1973conditional}, there are subtle differences. 
When \eqref{eq:Luce} is applied to model
ranking and choice data with distinct items, each observation $i$ usually consists of a possibly \emph{different}
subset $S_i$ of the universe of all alternatives, so that there is a large number of different configurations of the choice menu in the dataset. On the other hand, common applications of multinomial logit models, such as classification models in statistics and machine learning \citep{bishop2006pattern} and discrete choice models in econometrics \citep{mcfadden1973conditional}, often deal with repeated observations consisting of the \emph{same} number of alternatives. However, these alternatives now possess ``characteristics'' that vary across observations, which are often mapped \emph{parametrically} to the scores in \eqref{eq:Luce}. 
 In this paper, we primarily use the term \emph{Luce choice model} to refer to the model \eqref{eq:model}, although it is also called MNL (for multinomial logit) models in some works. We refrain from using the term MNL to avoid confusion with parametric models for featurized items used in ML and econometrics.

\subsection{Matrix Balancing}
The matrix balancing problem we study in this paper and its variants \citep{ruiz2001scaling,bradley2010algorithms} underlie a diverse range of applications from different disciplines. The  question of scaling rows and columns of a matrix $A$ so that the resulting matrix has target row and column norms $p,q$ has been studied as early as the 1930s, and continue to intrigue researchers from different backgrounds. The present paper only contains a partial survey of the vast literature on the topic. \citet{schneider1990comparative} and \citet{idel2016review} provide excellent discussions of many applications. In \cref{app:related-works}, we present concise summaries of some applications to illustrate the ubiquity of the matrix balancing problem. 

The standard iterative algorithm for the matrix balancing problem has been rediscovered independently quite a few times. As a result, it has domain-dependent names, including the iterative proportional fitting (IPF) procedure \citep{deming1940least}, biproportional fitting \citep{bacharach1965estimating} and the RAS algorithm \citep{stone1962multiple}, but is perhaps most widely known as \emph{Sinkhorn's algorithm} \citep{sinkhorn1964relationship,cuturi2013sinkhorn}. A precise description can be found in \cref{alg:scaling}.
This algorithm is also closely related to relaxation and coordinate descent type methods for solving the dual of entropy minimization problems \citep{bregman1967relaxation,cottle1986lagrangean,tseng1987relaxation,luo1992convergence}, as well as message passing and belief propagation algorithms in distributed optimization \citep{balakrishnan2004polynomial,agarwal2018accelerated}. 

The convergence behavior of Sinkhorn's algorithm has been extensively studied by \citet{sinkhorn1964relationship,bregman1967proof,lamond1981bregman,franklin1989scaling,ruschendorf1995convergence,kalantari2008complexity,knight2008sinkhorn,pukelsheim2009iterative,altschuler2017near,chakrabarty2021better,leger2021gradient}, among many others. For $A$ with strictly \emph{positive} entries, \citet{franklin1989scaling} first established the global linear convergence of Sinkhorn's algorithm in the Hilbert projective metric $d$. In particular, if $r^{(t)}$ denotes the row sum of the scaled matrix after $t$ iterations of Sinkhorn's algorithm that enforce column constraints, then 
 \begin{align}
 \label{eq:hilbert-contraction} 
   d(r^{(t)}, p) \leq  \lambda^{t} \cdot d(r^{(0)}, p)
 \end{align}
  for some $\lambda \in (0,1)$ dependent on $A$. See \citet{bushell1973hilbert} for details on the Hilbert metric. \citet{altschuler2017near} and \citet{chakrabarty2021better} show global sub-linear convergence in terms of iteration complexity bounds on the $\ell^1$ distance independent of matrix dimension. 
  
  However, the matter of convergence is more delicate when the matrix contains zero entries, and additional assumptions on the problem structure are required. For non-negative $A$, convergence is first established by \citet{sinkhorn1967concerning} in the special case of  \emph{square} $A$ and uniform $p=q=\mathbf{1}_n=\mathbf{1}_m$. Their necessary and sufficient condition is that $A$
has \emph{total support}, i.e., any non-zero entry of $A$ must be in $(A_{1\sigma(1)},A_{2\sigma(2)},\dots,A_{n\sigma(n)})$
for some permutation $\sigma$. \citet{soules1991rate} shows the convergence is linear, and \citet{knight2008sinkhorn} provides an explicit and tight \emph{asymptotic} linear convergence rate in terms of the sub-dominant (second largest) singular value of the scaled doubly stochastic matrix $D^0AD^1$. However, no asymptotic linear convergence rate is previously known for non-uniform marginals.

For general non-negative matrices and non-uniform marginals, the necessary and sufficient conditions for the matrix balancing problem that generalize that of \citet{sinkhorn1967concerning} have been studied by \citet{thionet1964note,bacharach1965estimating,brualdi1968convex,menon1968matrix,djokovic1970note,sinkhorn1974diagonal,balakrishnan2004polynomial,pukelsheim2009iterative}, and convergence of Sinkhorn's algorithm under these conditions is well-known. Connecting Sinkhorn's algorithm to dual coordinate descent for entropy minimization, \citet{luo1992convergence} show that the dual objective converges linearly with some unknown rate $\lambda$ when finite scalings $D^0,D^1$ exist. However, their result is implicit and there are no results that quantify this rate $\lambda$, even for special classes of non-negative matrices. When convergence results on positive matrices in previous works are applied to non-negative matrices, the bounds often blow up or become degenerate as soon as $\min_{ij}A_{ij} \downarrow 0$. For example, in \eqref{eq:hilbert-contraction} the contraction factor $\lambda \rightarrow 1$ when $A$ contains zero entries. In contrast, under a \emph{weaker} condition that guarantees the convergence of Sinkhorn's algorithm, \citet{leger2021gradient} gives a \emph{quantitative} global $\mathcal{O}(1/t)$ bound for non-negative matrices. It remains to reconcile the results of these works and characterize the linear rate $\lambda$ for non-negative $A$.

Our work precisely fills the gaps left by these works. The global linear convergence result in \cref{thm:global-convergence} establishes a contraction like \eqref{eq:hilbert-contraction} whenever finite scalings $D^0,D^1$ exist, and characterize $\lambda$ in terms of the algebraic connectivity. Moreover, the asymptotic linear rate in \cref{thm:convergence} directly extends the result of \citet{knight2008sinkhorn}. See \cref{tab:convergence-summary} for a detailed summary and comparison of the convergence results in previous works and this paper.

The dependence of Sinkhorn's convergence rate on spectral properties of graphs can be compared to convergence results in the  literature on decentralized optimization and gossip algorithms, where a spectral gap quantifies the convergence rate \citep{boyd2006randomized,xiao2007distributed}.
 

%!TEX root = hopfwright.tex
%

In this section we systematically recast the Hopf bifurcation problem in Fourier space. 
We introduce appropriate scalings, sequence spaces of Fourier coefficients and convenient operators on these spaces. 
To study Equation~\eqref{eq:FourierSequenceEquation} we consider Fourier sequences $ \{a_k\}$ and fix a Banach space in which these sequences reside. It is indispensable for our analysis that this space have an algebraic structure. 
The Wiener algebra of absolutely summable Fourier series is a natural candidate, which we use with minor modifications. 
In numerical applications, weighted sequence spaces with algebraic and geometric decay have been used to great effect to study periodic solutions which are $C^k$ and analytic, respectively~\cite{lessard2010recent,hungria2016rigorous}. 
Although it follows from Lemma~\ref{l:analytic} that the Fourier coefficients of any solution decay exponentially, we choose to work in a space of less regularity. 
The reason is that by working in a space with less regularity, we are better able to connect our results with the global estimates in \cite{neumaier2014global}, see Theorem~\ref{thm:UniqunessNbd2}.


%
%
%\begin{remark}
%	Although it follows from Lemma~\ref{l:analytic} that the Fourier coefficients of any solution decay exponentially, we choose to work in a space of less regularity, namely summable Fourier coefficients. This allows us to draw SOME MORE INTERESTING CONCLUSION LATER.
%	EXPLAIN WHY WE CHOOSE A NORM WITH ALMOST NO DECAY!
%	% of s Periodic solutions to Wright's equation are known to be real analytic and so their  Fourier coefficients must decay geometrically [Nussbaum].
%	% We do not use such a strong result;  any periodic solution must be continuously differentiable, by which it follows that $ \sum | c_k| < \infty$.
%\end{remark}


\begin{remark}\label{r:a0}
There is considerable redundancy in Equation~\eqref{eq:FourierSequenceEquation}. First, since we are considering real-valued solutions $y$, we assume $\c_{-k}$ is the complex conjugate of $\c_k$. This symmetry implies it suffices to consider Equation~\eqref{eq:FourierSequenceEquation} for $k \geq 0$.
Second, we may effectively ignore the zeroth Fourier coefficient of any periodic solution \cite{jones1962existence}, since it is necessarily equal to $0$. 
%In \cite{jones1962existence}, it is shown that if $y \not\equiv -1$ is a periodic solution of~\eqref{eq:Wright} with frequency $\omega$, then $ \int_0^{2\pi/\omega} y(t) dt =0$. 
		The self contained argument is as follows. 
		As mentioned in the introduction, any periodic solution to Wright's equation must satisfy $ y(t) > -1$ for all $t$. 
	By dividing Equation~\eqref{eq:Wright} by $(1+y(t))$, which never vanishes, we obtain
	\[
	\frac{d}{dt} \log (1 + y(t)) = - \alpha y(t-1).
	\]  
	Integrating over one period $L$ we derive the condition 
	$0=\int_0^L y(t) dt $.
	Hence $a_0=0$ for any periodic solution. 
	It will be shown in Theorem~\ref{thm:FourierEquivalence1} that a related argument implies that we do not need to consider Equation~\eqref{eq:FourierSequenceEquation} for $k=0$.
\end{remark}

%%%
%%%
%%%\begin{remark}\label{r:c0} 
%%%In \cite{jones1962existence}, it is shown that if $y \not\equiv -1$ is a periodic solution of~\eqref{eq:Wright} with frequency $\omega$, then $ \int_0^{2\pi/\omega} y(t) dt =0$. 
%%%PERHAPS TOO MUCH DETAIL HERE. The self contained argument is as follows.
%%%If $y \not\equiv -1$ then $y(t) \neq -1$ for all $t$, since if $y(t_0)=-1$ for some $t_0 \in \R$ then $y'(t_0)=0$ by~\eqref{eq:Wright} and in fact by differentiating~\eqref{eq:Wright} repeatedly one obtains that all derivatives of $y$ vanish at $t_0$. Hence $y \equiv -1$ by Lemma~\ref{l:analytic}, a contradiction. Now divide~\eqref{eq:Wright} by $(1+y(t))$, which never vanishes, to obtain
%%%\[
%%%  \frac{d}{dt} \log |1 + y(t)| = - \alpha y(t-1).
%%%\]  
%%%Integrating over one period we obtain $\int_0^L y(t) dt =0$.
%%%\end{remark}



%Furthermore, the condition that $y(t)$ is real forces $\c_{-k} = \overline{\c}_{k}$.  
%
We define the spaces of absolutely summable Fourier series
\begin{alignat*}{1}
	\ell^1 &:= \left\{ \{ \c_k \}_{k \geq 1} : 
    \sum_{k \geq 1} | \c_k| < \infty  \right\} , \\
	\ell^1_\bi &:= \left\{ \{ \c_k \}_{k \in \Z} : 
    \sum_{k \in \Z} | \c_k| < \infty  \right\} .
\end{alignat*} 
We identify any semi-infinite sequence $ \{ \c_k \}_{k \geq 1} \in \ell^1$ with the bi-infinite sequence $ \{ \c_k \}_{k \in \Z} \in \ell^1_\bi$ via the conventions (see Remark~\ref{r:a0})
\begin{equation}
  \c_0=0 \qquad\text{ and }\qquad \c_{-k} = \c_{k}^*. 
\end{equation}
In other word, we identify $\ell^1$ with the set
\begin{equation*}
   \ell^1_\sym := \left\{ \c \in \ell^1_\bi : 
	\c_0=0,~\c_{-k}=\c_k^* \right\} .
\end{equation*}
On $\ell^1$ we introduce the norm
\begin{equation}\label{e:lnorm}
  \| \c \| = \| \c \|_{\ell^1} := 2 \sum_{k = 1}^\infty |\c_k|.
\end{equation}
The factor $2$ in this norm is chosen to have a Banach algebra estimate.
Indeed, for $\c, \tilde{\c} \in \ell^1 \cong \ell^1_\sym$ we define
the discrete convolution 
\[
\left[ \c * \tilde{\c} \right]_k = \sum_{\substack{k_1,k_2\in\Z\\ k_1 + k_2 = k}} \c_{k_1} \tilde{\c}_{k_2} .
\]
Although $[\c*\tilde{\c}]_0$ does not necessarily vanish, we have $\{\c*\tilde{\c}\}_{k \geq 1} \in \ell^1 $ and 
\begin{equation*}
	\| \c*\tilde{\c} \| \leq \| \c \| \cdot  \| \tilde{\c} \| 
	\qquad\text{for all } \c , \tilde{\c} \in \ell^1, 
\end{equation*}
hence $\ell^1$ with norm~\eqref{e:lnorm} is a Banach algebra.

By Lemma~\ref{l:analytic} it is clear that any periodic solution of~\eqref{eq:Wright} has a well-defined Fourier series $\c \in \ell^1_\bi$. 
The next theorem shows that in order to study periodic orbits to Wright's equation we only need to study Equation~\eqref{eq:FourierSequenceEquation} 
for $k \geq 1$. For convenience we introduce the notation 
\[
G(\alpha,\omega,\c)_k=
( i \omega k + \alpha e^{ - i \omega k}) \c_k + \alpha \sum_{k_1 + k_2 = k} e^{- i \omega k_1} \c_{k_1} \c_{k_2} \qquad \text{for } k \in \N.
\]
We note that we may interpret the trivial solution $y(t)\equiv 0$ as a periodic solution of arbitrary period.
\begin{theorem}
\label{thm:FourierEquivalence1}
Let $\alpha>0$ and $\omega>0$.
If $\c \in \ell^1 \cong \ell^1_{\sym}$ solves
$G(\alpha,\omega,\c)_k =0$  for all $k \geq 1$,
then $y(t)$ given by~\eqref{eq:FourierEquation} is a periodic solution of~\eqref{eq:Wright} with period~$2\pi/\omega$.
Vice versa, if $y(t)$ is a periodic solution of~\eqref{eq:Wright} with period~$2\pi/\omega$ then its Fourier coefficients $\c \in \ell^1_\bi$ lie in $\ell^1_\sym \cong \ell^1$ and solve $G(\alpha,\omega,\c)_k =0$ for all $k \geq 1$.
\end{theorem}

\begin{proof}	
	If $y(t)$ is a periodic solution of~\eqref{eq:Wright} then it is real analytic by Lemma~\ref{l:analytic}, hence its Fourier series $\c$ is well-defined and $\c \in \ell^1_{\sym}$ by Remark~\ref{r:a0}.
	Plugging the Fourier series~\eqref{eq:FourierEquation} into~\eqref{eq:Wright} one easily derives that $\c$ solves~\eqref{eq:FourierSequenceEquation} for all $k \geq 1$.

To prove the reverse implication, assume that $\c \in \ell^1_\sym$ solves
Equation~\eqref{eq:FourierSequenceEquation} for all $k \geq 1$. Since $\c_{-k}
= \c_k^*$, Equation \eqref{eq:FourierSequenceEquation} is also satisfied for
all $k \leq -1$. It follows from the Banach algebra property and
\eqref{eq:FourierSequenceEquation} that $\{k \c_k\}_{k \in \Z} \in \ell^1_\bi$,
hence $y$, given by~\eqref{eq:FourierEquation}, is continuously differentiable.
% (and by bootstrapping one infers that $\{k^m c_k \} \in \ell^1_\bi$, 
% hence $y \in C^m$ for any $m \geq 1$).
	Since~\eqref{eq:FourierSequenceEquation} is satisfied for all $k \in \Z \setminus \{0\}$ (but not necessarily for $k=0$) one may perform the inverse Fourier transform on~\eqref{eq:FourierSequenceEquation} to conclude that
	$y$ satisfies the delay equation 
\begin{equation}\label{eq:delaywithK}
   	y'(t) = - \alpha y(t-1) [ 1 + y(t)] + C
\end{equation}
	for some constant $C \in \R$. 
   Finally, to prove that $C=0$ we argue by contradiction.
   Suppose $C \neq 0$. Then $y(t) \neq -1$ for all $t$.
   Namely, at any point where $y(t_0) =-1$ one would have $y'(t_0) = C$
   which has fixed sign,   hence it would follow that $y$ is not periodic
   ($y$ would not be able to cross $-1$ in the opposite direction, 
   preventing $y$  from being periodic).  
  We may thus divide~\eqref{eq:delaywithK} through by $1 + y(t)$ and obtain 
\begin{equation*}
	\frac{d}{dt} \log | 1 + y(t) | = - \alpha y(t-1) + \frac{C}{1+y(t)} .
\end{equation*}
	By integrating both sides of the equation over one period $L$ and by using that $\c_0=0$, we 
	obtain
	\[
	 C \int_0^L \frac{1}{1+y(t)} dt =0.
	\]
	Since the integrand is either strictly negative or strictly positive, this implies that $C=0$. Hence~\eqref{eq:delaywithK} reduces to~\eqref{eq:Wright},
	and $y$ satisfies Wright's equation. 
\end{proof}






To efficiently study Equation~\eqref{eq:FourierSequenceEquation}, we introduce the following linear operators on $ \ell^1$:
\begin{alignat*}{1}
   [K \c ]_k &:= k^{-1} \c_k  , \\ 
   [ U_\omega \c ]_k &:= e^{-i k \omega} \c_k  .
\end{alignat*}
The map $K$ is a compact operator, and it has a densely defined inverse $K^{-1}$. The domain of $K^{-1}$ is denoted by
\[
  \ell^K := \{ \c \in \ell^1 : K^{-1} \c \in \ell^1 \}.  
\]
The map $U_{\omega}$ is a unitary operator on $\ell^1$, but
it is discontinuous in $\omega$. 
With this notation, Theorem~\ref{thm:FourierEquivalence1} implies that our problem of finding a SOPS to~\eqref{eq:Wright} is equivalent to finding an $\c \in \ell^1$ such that
\begin{equation}
\label{e:defG}
  G(\alpha,\omega,\c) :=
  \left( i \omega K^{-1} + \alpha U_\omega \right) \c + \alpha \left[U_\omega \, \c \right] * \c  = 0.
\end{equation}


%In order for the solutions of Equation \ref{eq:FHat} to be isolated we need to impose a phase condition. 
%If there is a sequence $ \{ c_k \} $ which satisfies  Equation \ref{eq:FHat}, then $ y( t + \tau) = \sum_{k \in \Z} c_k e^{ i k \omega (t + \tau)}$ satisfies Wright's equation at parameter $\alpha$. 
%Fix $ \tau = - Arg[c_1] / \omega$ so that $ c_1  e^{ i \omega \tau} $ is a nonnegative real number. 
%By Proposition \ref{thm:FourierEquivalence1} it follows that $\{ c'_k \} =  \{c_k e^{ i \omega k \tau }   \}$ is a solution to Equation \ref{eq:FHat}, and furthermore that $ c'_1 = \epsilon$ for some $ \epsilon \geq 0$. 


Periodic solutions are invariant under time translation: if $y(t)$ solves Wright's equation, then so does $ y(t+\tau)$ for any $\tau \in \R$. 
We remove this degeneracy by adding a phase condition. 
Without loss of generality, if $\c \in \ell^1$ solves Equation~\eqref{e:defG}, we may assume that $\c_1 = \epsilon$ for some 
\emph{real non-negative}~$\epsilon$:
\[
  \ell^1_{\epsilon} := \{\c \in \ell^1 : \c_1 = \epsilon \} 
  \qquad \text{where } \epsilon \in \R,  \epsilon \geq 0.
\]
In the rest of our analysis, we will split elements $\c \in \ell^1$ into two parts: $\c_1$ and $\{\c_{k}\}_{k \geq 2}$.  
We define the basis elements $\e_j \in \ell^1$ for $j=1,2,\dots$ as
\[
  [\e_j]_k = \begin{cases}
  1 & \text{if } k=j, \\
  0 & \text{if } k \neq j.
  \end{cases}
\]
We note that $\| \e_j \|=2$. 
Then we can decompose
% We define
% \[
%   \tilde{\epsilon} := (\epsilon,0,0,0,\dots) \in \ell^1
% \]
% and
% For clarity when referring to sequences $\{c_{k}\}_{k \geq 2}$, we make the following definition:
% \[
% \ell^1_0  := \{ \tc \in \ell^1 : \tc_1 = 0 \}.
% \]
% With the
any $\c \in \ell^1_\epsilon$ uniquely as
\begin{equation}\label{e:aepsc}
  \c= \epsilon \e_1 + \tc \qquad \text{with}\quad 
  \tc \in \ell^1_0 := \{ \tc \in \ell^1 : \tc_1 = 0 \}.
\end{equation}
We follow the classical approach in studying Hopf bifurcations and consider 
$\c_1 = \epsilon$ to be a parameter, and then find periodic solutions with Fourier modes in $\ell^1_{\epsilon}$.
This approach rewrites the function $G: \R^2 \times \ell^K \to \ell^1$ as a function $\tilde{F}_\epsilon : \R^2 \times \ell^K_0 \to \ell^1$, where 
we denote 
\[
\ell^K_0 := \ell^1_0 \cap \ell^K.
\]
% I AM ACTUALLY NOT SURE IF YOU WANT TO DEFINE THIS WITH RANGE IN $\ell^1$
% OR WITH DOMAIN IN $\ell^1_0$ ?? IT SEEMS TO DEPEND ON WHICH GLOBAL STATEMENT YOU WANT/NEED TO MAKE!?
\begin{definition}
We define the $\epsilon$-parameterized family of  functions $\tilde{F}_\epsilon: \R^2 \times \ell^K_0  \to \ell^1$ 
by 
\begin{equation}
\label{eq:fourieroperators}
\tilde{F}_{\epsilon}(\alpha,\omega, \tc) := 
\epsilon [i \omega + \alpha e^{-i \omega}] \e_1 + 
( i \omega K^{-1} + \alpha U_{\omega}) \tc + 
\epsilon^2 \alpha e^{-i \omega}  \e_2  +
\alpha \epsilon L_\omega \tc + 
\alpha  [ U_{\omega} \tc] * \tc ,
\end{equation}
where
$L_\omega : \ell^1_0 \to \ell^1$ is given by
\[
   L_{\omega} := \sigma^+( e^{- i \omega} I + U_{\omega}) + \sigma^-(e^{i \omega} I + U_{\omega}),
\]
with $I$ the identity and  $\sigma^\pm$ the shift operators on $\ell^1$:
\begin{alignat*}{2}
\left[ \sigma^- a \right]_k &:=  a_{k+1}  , \\
\left[ \sigma^+ a \right]_k &:=  a_{k-1}  &\qquad&\text{with the convention } \c_0=0.
\end{alignat*}
The operator $ L_\omega$ is discontinuous in $\omega$ and $ \| L_\omega \| \leq 4$. 
\end{definition} 

%The maps $ \sigma^{+}$ and $ \sigma^-$ are shift up and shift down operators respectively. 
We reformulate Theorem~\ref{thm:FourierEquivalence1}  in terms of the map  $\tilde{F}$. 
We note that it follows from Lemma~\ref{l:analytic} and 
%\marginpar{Reformulate}
%one's choice of  
Equation~\eqref{eq:FourierSequenceEquation}  
%or Equation ~\eqref{eq:fourieroperators},
that the Fourier coefficients of any periodic solution of~\eqref{eq:Wright} lie in $\ell^K$.
These observations are summarized in the following theorem.
\begin{theorem}
\label{thm:FourierEquivalence2}
	Let $ \epsilon \geq 0$,  $\tc \in \ell^K_0$, $\alpha>0$ and $ \omega >0$. 
	Define $y: \R\to \R$ as 
\begin{equation}\label{e:ytc}
	y(t) = 
	\epsilon \left( e^{i \omega t }  + e^{- i \omega t }\right) 
	+  \sum_{k = 2}^\infty   \tc_k e^{i \omega k t }  + \tc_k^* e^{- i \omega k t } .
\end{equation}
%	and suppose that $ y(t) > -1$. 
	Then $y(t)$ solves~\eqref{eq:Wright} if and only if $\tilde{F}_{\epsilon}( \alpha , \omega , \tc) = 0$. 
	Furthermore, up to time translation, any periodic solution of~\eqref{eq:Wright} with period $2\pi/\omega$ is described by a Fourier series of the form~\eqref{e:ytc} with $\epsilon \geq 0$ and $\tc \in \ell^K_0$.
\end{theorem}


%We note that for $\epsilon>0$ such solutions are truly periodic, while for $\epsilon=0$ a zero of $\tilde{F}_\epsilon$ may either correspond to a periodic solution or to the trivial solution $y(t) \equiv 0$. 



% \begin{proof}
%  By Proposition \ref{thm:FourierEquivalence1}, it suffices to show that $\tilde{F}(\alpha,\omega,c) =0$ is equivalent to Equation \ref{eq:FourierSequenceEquation} being satisfied for $k \geq 1$.
%  Since Equation \ref{eq:FourierSequenceEquation} is equivalent to Equation \ref{eq:FHat}, we expand  Equation \ref{eq:FHat} by writing $ \hat{c} = \hat{\epsilon } + c$  where $ \hat{\epsilon} := (\epsilon,0,0,\dots) \in \ell^1$ as below:
%  \begin{equation}
%  0=  \left( i \omega K^{-1} + \alpha U_\omega \right) (\hat{\epsilon}+ c) + \alpha \left[U_\omega \, (\hat{\epsilon}+ c) \right] * (\hat{\epsilon}+ c) \label{eq:Intial}
%  \end{equation}
%  The RHS of Equation \ref{eq:Intial} is $ \tilde{F}(\alpha,\omega,c)$, so the theorem is proved.
% \end{proof}



Since we want to analyze a Hopf bifurcation, we will want to solve $\tilde{F}_\epsilon = 0$ for small values of~$\epsilon$. 
However, at the bifurcation point, $ D \tilde{F}_0(\pp  ,\pp , 0)$ is not invertible.
In order for our asymptotic analysis to be non-degenerate,
we work with a rescaled version of the problem. To this end, for any $\epsilon >0$, we rescale both $\tc$ and $\tilde{F}$ as follows. Let $\tc = \epsilon c$ and 
\begin{equation}\label{e:changeofvariables}
  \tilde{F}_\epsilon (\alpha,\omega,\epsilon c) = \epsilon F_\epsilon (\alpha,\omega,c).
\end{equation}
For $\epsilon>0$ the problem then reduces to finding zeros of 
\begin{equation}
\label{eq:FDefinition}
	F_\epsilon(\alpha,\omega, c) := 
	[i \omega + \alpha e^{-i \omega}] \e_1 + 
	( i \omega K^{-1} + \alpha U_{\omega}) c + 
	\epsilon \alpha e^{-i \omega} \e_2  +
	\alpha \epsilon L_\omega c + 
	\alpha \epsilon [ U_{\omega} c] * c.
\end{equation}
We denote the triple $(\alpha,\omega,c) \in \R^2 \times \ell^1_0$ by $x$.
To pinpoint the components of $x$ we use the projection operators
\[
   \pi_\alpha x = \alpha, \quad \pi_\omega x = \omega, \quad 
  \pi_c x = c \qquad\text{for any } x=(\alpha,\omega,c).
\]

After the change of variables~\eqref{e:changeofvariables} we now have an invertible Jacobian $D F_0(\pp  ,\pp , 0)$ at the bifurcation point.
On the other hand, for $\epsilon=0$ the zero finding problems for $\tilde{F}_\epsilon$ and $F_\epsilon$ are not equivalent. 
However, it follows from the following lemma that any nontrivial periodic solution having $ \epsilon=0$ must have a relatively large size when $ \alpha $ and $ \omega $ are close to the bifurcation point. 

\begin{lemma}\label{lem:Cone}
	Fix $ \epsilon \geq 0$ and $\alpha,\omega >0$. 
	Let
	\[
	b_* :=  \frac{\omega}{\alpha} - \frac{1}{2} - \epsilon  \left(\frac{2}{3}+ \frac{1}{2}\sqrt{2 + 2 |\omega-\pp| } \right).
	\]
Assume that $b_*> \sqrt{2} \epsilon$. 
Define
% \begin{equation*}%\label{e:zstar}
% 	z^{\pm}_* :=b_* \pm \sqrt{(b_*)^2- \epsilon^2 } .
% \end{equation*}
% \note[J]{Proposed change to match Lemma E.4}
\begin{equation}\label{e:zstar}
z^{\pm}_* :=b_* \pm \sqrt{(b_*)^2- 2 \epsilon^2 } .
\end{equation}
If there exists a $\tc \in \ell^1_0$ such that $\tilde{F}_\epsilon(\alpha, \omega,\tc) = 0$, then \\
\mbox{}\quad\textup{(a)} either $ \|\tc\| \leq  z_*^-$ or $ \|\tc\| \geq z_*^+  $.\\
\mbox{}\quad\textup{(b)} 
$ \| K^{-1} \tc \| \leq (2\epsilon^2+ \|\tc\|^2) / b_*$. 
\end{lemma}
\begin{proof}
	The proof follows from Lemmas~\ref{lem:gamma} and~\ref{lem:thecone} in Appendix~\ref{appendix:aprioribounds}, combined with the observation that
$\frac{\omega}{\alpha} - \gamma \geq b_*$,
% \[
%   \frac{\omega}{\alpha} - \gamma \geq b_*
%  \qquad\text{for all }
% | \alpha - \pp| \leq r_\alpha \text{ and } 
%   | \omega - \pp| \leq r_\omega.
% \]
with $\gamma$ as defined in Lemma~\ref{lem:gamma}.
\end{proof}

\begin{remark}\label{r:smalleps}
We note that for $\alpha < 2\omega$
\begin{alignat*}{1}
z^+_* &\geq   \frac{2 \omega - \alpha}{\alpha} 
- \epsilon \left(4/3+\sqrt{2 + 2 |\omega-\pp| } \, \right) + \cO(\epsilon^2)
\\[1mm]
z^-_* & \leq   \cO(\epsilon^2)
\end{alignat*}
for small $\epsilon$. 
Hence Lemma~\ref{lem:Cone} implies that for values of $(\alpha,\omega)$ near $(\pp,\pp)$ any solution has either $\|\tc\|$ of order 1 or $\|\tc\| =  \cO(\epsilon^2)$. 
The asymptotically small term bounding $z_*^-$ is explicitly calculated in Lemma~\ref{lem:ZminusBound}. 
A related consequence is that for $\epsilon=0$ there are no nontrivial solutions 
of $\tilde{F}_0(\alpha,\omega,\tc)=0$ with 
$\| \tc \| < \frac{2 \omega - \alpha}{\alpha} $. 
\end{remark}

\begin{remark}\label{r:rhobound}
In Section~\ref{s:contraction} we will work on subsets of $\ell^K_0$ of the form
\[
  \ell_\rho := \{ c \in \ell^K_0 : \|K^{-1} c\| \leq \rho \} .
\]
Part (b) of Lemma~\ref{lem:Cone} will be used in Section~\ref{s:global} to guarantee that we are not missing any solutions by considering $\ell_\rho$ (for some specific choice of $\rho$) rather than the full space $\ell^K_0$.
In particular, we infer from Remark~\ref{r:smalleps} that  small solutions (meaning roughly that $\|\tc\| \to 0$ as $\epsilon \to 0$)
satisfy $\| K^{-1} \tc \| = \cO(\epsilon^2)$.
\end{remark}

The following theorem guarantees that near the bifurcation point the problem of finding all periodic solutions is equivalent to considering the rescaled problem $F_\epsilon(\alpha,\omega,c)=0$.
\begin{theorem}
\label{thm:FourierEquivalence3}
\textup{(a)} Let $ \epsilon > 0$,  $c \in \ell^K_0$, $\alpha>0$ and $ \omega >0$. 
	Define $y: \R\to \R$ as 
\begin{equation}\label{e:yc}
	y(t) = 
	\epsilon \left( e^{i \omega t }  + e^{- i \omega t }\right) 
	+ \epsilon  \sum_{k = 2}^\infty   c_k e^{i \omega k t }  + c_k^* e^{- i \omega k t } .
\end{equation}
%	and suppose that $ y(t) > -1$. 
	Then $y(t)$ solves~\eqref{eq:Wright} if and only if $F_{\epsilon}( \alpha , \omega , c) = 0$.\\
\textup{(b)}
Let $y(t) \not\equiv 0$ be a periodic solution of~\eqref{eq:Wright} of period $2\pi/\omega$
 with Fourier coefficients $\c$.
Suppose $\alpha < 2\omega$ and $\| \c \| < \frac{2 \omega - \alpha}{\alpha} $.
Then, up to time translation, $y(t)$ is described by a Fourier series of the form~\eqref{e:yc} with $\epsilon > 0$ and $c \in \ell^K_0$.
\end{theorem}

\begin{proof}
Part (a) follows directly from Theorem~\ref{thm:FourierEquivalence2} and the  change of variables~\eqref{e:changeofvariables}.
To prove part (b) we need to exclude the possibility that there is a nontrivial solution with $\epsilon=0$. The asserted bound on the ratio of $\alpha$ and $\omega$ guarantees, by Lemma~\ref{lem:Cone} (see also Remark~\ref{r:smalleps}), that indeed $\epsilon>0$ for any nontrivial solution. 
\end{proof}

We note that in practice (see Section~\ref{s:global}) a bound on $\| \c \|$ is derived from a bound on $y$ or $y'$ using Parseval's identity.

\begin{remark}\label{r:cone}
It follows from Theorem~\ref{thm:FourierEquivalence3} and Remark~\ref{r:smalleps} that for values of $(\alpha,\omega)$ near $(\pp,\pp)$ any reasonably bounded solution satisfies $\| c\| =  O(\epsilon)$ as well as $\|K^{-1} c \| = O(\epsilon)$ asymptotically (as $\epsilon \to 0$).
These bounds will be made explicit (and non-asymptotic) for specific choices of the parameters in Section~\ref{s:global}.
\end{remark}

% We are able to rule out such large amplitude solutions using global estimates such as those in \cite{neumaier2014global}.
% Hence, near the bifurcation point, the problem of describing periodic solutions of~\eqref{eq:Wright} reduces to studying the family of zeros finding problems $F_\epsilon=0$.





%Specifically, if a solution having $ \epsilon = 0$ does in fact correspond to a nontrivial periodic solution and $\alpha  < 2\omega $, then $ \| \tilde{c} \| > 2 \omega \alpha^{-1} -1$. 
%%PERHAPS THIS NEEDS A FORMULATION AS A THEOREM AS WELL?
%%IN OTHER WORDS: ARE WE SURE WE HAVE FOUND ALL ZEROS OF $\tilde{F}_0$, I.E. ALL SOLUTIONS WITH $\epsilon=0$ NEAR THE BIFURCATION POINT? AFTER RESCALING THESE ARE INVISIBLE?
%%THERE SHOULD BE A STATEMENT ABOUT THIS SOMEWHERE! EITHER HERE OR SOME





We finish this section by defining a curve of approximate zeros $\bx_\epsilon$ of $F_\epsilon$ 
(see \cite{chow1977integral,hassard1981theory}). 
%(see \cite{chow1977integral,morris1976perturbative,hassard1981theory}). 


\begin{definition}\label{def:xepsilon}
Let
\begin{alignat*}{1}
	\balpha_\epsilon &:= \pp + \tfrac{\epsilon^2}{5} ( \tfrac{3\pi}{2} -1)  \\
	\bomega_\epsilon &:= \pp -  \tfrac{\epsilon^2}{5} \\
	\bc_\epsilon 	 &:= \left(\tfrac{2 - i}{5}\right) \epsilon \,  \e_2 \,.
\end{alignat*}
We define the approximate solution 
$ \bx_\epsilon := \left( \balpha_\epsilon , \bomega_\epsilon  , \bc_\epsilon \right)$
for all $\epsilon \geq 0$.
\end{definition}

We leave it to the reader to verify that both 
 $F_\epsilon(\pp,\pp,\bc_{\epsilon})=\cO(\epsilon^2)$ and $F_\epsilon(\bx_\epsilon)=\cO(\epsilon^2)$.
%%%	
%%%	
%%%	}{Better like this?}
%%%\annote[J]{ $F_\epsilon(\bx_0)=\cO(\epsilon^2)$ and $F_\epsilon(\bx_\epsilon)=\cO(\epsilon^2)$.}{I think we'd still need the $ \bar{c}_\epsilon$ term in $\bar{x}_0$ to be of order $ \epsilon$.}
%%%\remove[JB]{We show in Proposition A.1
%%%%\ref{prop:ApproximateSolutionWorks} 
%%% that any $ x \in \R^2 \times \ell^1_0$ which is $ \cO(\epsilon^2)$ close to $ \bar{x}_\epsilon $ will yield the estimate $F_\epsilon(x) = \cO(\epsilon^2)$.
%%%Hence choosing $\{ \pp , \pp, \bar{c}_\epsilon\}$ as our approximate solution would also have been a natural choice for performing an $\cO(\epsilon^2)$ analysis and would have simplified several of our calculations.
%%%However,} 
%%%
We choose to use the more accurate approximation 
for the $ \alpha$ and $ \omega $ components to improve our final quantitative results. 














%
% Values for $ (\alpha, \omega,c)$ which approximately solve $\tilde{F}(\alpha,\omega,c) = 0$  are computed in  \cite{chow1977integral,morris1976perturbative,hassard1981theory} and are as follows:
%  \begin{eqnarray}
%  \tilde{\alpha}( \epsilon) &:=& \pi /2 + \tfrac{\epsilon^2}{5} ( \tfrac{3\pi}{2} -1) \nonumber \\
%  \tilde{\omega}( \epsilon) &:=& \pi /2 -  \tfrac{\epsilon^2}{5} \label{eq:ScaleApprox} \\
%  \tc(\epsilon) 	  &:=& \{ \left(\tfrac{2 - i}{5}\right)  \epsilon^2 , 0,0, \dots \} \nonumber
%  \end{eqnarray}
% In Appendix \ref{sec:OperatorNorms} we illustrate an alternative method for deriving this approximation.
%
%
%
%
% We want to solve $ \tilde{F}(\alpha , \omega, \hat{c}) =0$ for small values of $ \epsilon$.
% However $ D \tilde{F}(\alpha , \omega , c)$ is not invertible at $ ( \pp , \pp , 0)$ when $ \epsilon = 0$.
% In order for our asymptotic analysis to be non-degenerate, we need to make the change of variables $ c \mapsto \epsilon c$.
% Under this change of variables, we define the function $ F$ below so that $ \tilde{F}(\alpha , \omega , \epsilon c) =\epsilon  F( \alpha , \omega , c)$.
%
%
%
% \begin{definition}
% Construct an $\epsilon$-parameterized family of densely defined functions  $F : \R^2 \oplus \ell^1 / \C \to \ell^1$ by:
% \begin{equation}
% \label{eq:FDefinition}
% 	F(\alpha,\omega, c) :=
% 	[i \omega + \alpha e^{-i \omega}]_1 +
% 	( i \omega K^{-1} + \alpha U_{\omega}) c +
% 	[\epsilon \alpha e^{-i \omega}]_2  +
% 	\alpha \epsilon L_\omega c +
% 	\alpha \epsilon [ U_{\omega} c] * c.
% \end{equation}
% \end{definition}

%%
%%
%%\begin{corollary}
%%	\label{thm:FourierEquivalence3}
%%	Fix $ \epsilon > 0$, and $ c \in \ell^1 / \C $, and $ \omega >0$. Define $y: \R\to \R$ as 
%%	\[
%%	y(t) = 
%%	\epsilon \left( e^{i \omega t }  + e^{- i \omega t }\right) 
%%	+  \epsilon  \left( \sum_{k = 2}^\infty   c_k e^{i \omega k t }  + \overline{c}_k e^{- i \omega k t } \right) 
%%	\]
%%	and suppose that $ y(t) > -1$. 
%%	Then $y(t)$ solves Wright's equation at parameter $ \alpha > 0 $ if and only if $ F( \alpha , \omega , c) = 0$ at parameter $ \epsilon$. 
%%	
%%	
%%	
%%\end{corollary}
%%
%%
%%\begin{proof}
%%	Since $ \tilde{F}(\alpha,\omega, \epsilon c) = \epsilon F( \alpha , \omega , c)$, the result follows from Theorem \ref{thm:FourierEquivalence2}.
%%\end{proof}

% If we can find $(\alpha , \omega, c)$ for which $ F( \alpha , \omega,c)=0$ at parameter $\epsilon$, then $ \tilde{F}(\alpha ,\omega, c)=0$.
% By Theorem \ref{thm:FourierEquivalence2} this amounts to finding a periodic solution to Wright's equation.
% Lastly, because we have performed the change of variables $ c \mapsto \epsilon c$, we need to  apply this change of variables to our approximate solution as well.
%
% \begin{definition}
% 	Define the approximate solution $ x( \epsilon) = \left\{ \alpha(\epsilon ) , \omega ( \epsilon ) , c(\epsilon) \right\}$ as below,  where $c(\epsilon) = \{ c_2( \epsilon) , 0 ,0 , \dots\} $.
% 	We may also write $ x_\epsilon = x(\epsilon) $.
% 	\begin{eqnarray}
% 	\alpha( \epsilon) &:=& \pi /2 + \tfrac{\epsilon^2}{5} ( \tfrac{3\pi}{2} -1) \nonumber \\
% 	\omega( \epsilon) &:=& \pi /2 -  \tfrac{\epsilon^2}{5} \label{eq:Approx} \\
% 	c_2(\epsilon) 	  &:=& \left(\tfrac{2 - i}{5}\right) \epsilon \nonumber
% 	\end{eqnarray}
%
% \end{definition}

\section{Connecting Choice Modeling and Matrix Balancing}
\label{sec:equivalence}

In this section, we formally establish the connection between choice modeling and matrix balancing. We show that
maximizing the log-likelihood \eqref{eq:log-likelihood} is equivalent to solving a canonical matrix balancing problem. We also precisely describe the correspondence between the relevant conditions in the two problems. In view of this equivalence, we show that Sinkhorn's algorithm, when applied to estimate Luce choice models, is in fact a \emph{parallelized} generalization of the classic iterative algorithm for choice models, dating back to \citet{zermelo1929berechnung,dykstra1956note,ford1957solution}, and studied extensively also by \citet{hunter2004mm,vojnovic2020convergence}.

\subsection{Maximum Likelihood Estimation of Luce Choice Model as Matrix Balancing}
\label{subsec:reformulation}
The optimality conditions for maximizing the log-likelihood \eqref{eq:log-likelihood} for each $s_j$ are given by
\begin{align*}
\partial_{s_{j}}\ell(s)=\sum_{j\mid (j,S_i)}\frac{1}{s_{j}}-\sum_{i\mid j\in S_{i}}\frac{1}{\sum_{k\in S_{i}}s_{k}} & =0.
\end{align*}
 Multiplying by $s_{j}$ and dividing by $1/n$, we have 
\begin{align}
\label{eq:optimality-original}
\frac{W_j}{n} & = \frac{1}{n} \sum_{i\mid j\in S_{i}}\frac{s_{j}}{\sum_{k\in S_{i}}s_{k}},
\end{align}
where $W_j:=|\{i\mid (j,S_i)\}|$ is the number of observations where $j$ is selected.

Note that in the special case where $S_i\equiv [n]$, i.e., every choice set contains \emph{all} items, the MLE simply reduces to the familiar empirical frequencies $\hat s_j = {W_j}/{n}$. However, when the choice sets $S_i$ vary, no closed form solution to \eqref{eq:optimality-original} exists, which is the primary motivation behind the long line of works on the algorithmic problem of solving \eqref{eq:optimality-original}. 

With varying $S_i$, we can interpret the optimality condition as requiring the \emph{observed} frequency of $j$ being chosen (left hand side) be equal to the conditional \emph{expected} probability of $j$ being chosen among all observations $i$ where it is part of the choice set $S_i$ (right hand side). In addition, note that since the optimality condition in \eqref{eq:optimality-original} only involves the \emph{frequency} of selection, distinct datasets could yield the same optimality condition and hence the same MLEs. For example, suppose that two alternatives $j$ and $k$ both appear in choice sets
$S_{i}$ and $S_{i'}$, with $j$ selected in $S_{i}$ and  $k$ selected in $S_{i'}$. Then switching
the choices in $S_{i}$ and $S_{i'}$ does not alter the likelihood and optimality conditions. This feature holds more generally with longer cycles of items and choice sets, and can be viewed as a consequence of the context-independent nature of Luce's choice axiom. In some sense, it is also the underpinning of many works in economics that estimate choice models based on \emph{marginal} sufficient statistics. A famous example is  \citet{berry1995automobile}, which estimates consumer preferences using only \emph{aggregate} market shares of products.

In practice, the choice sets $S_i$ of many observations may be identical to each other. Because \eqref{eq:optimality-original} only depends on the total winning counts of items, we may aggregate over observations with the same $S_i$:
\begin{align*}
\sum_{i\mid j\in S_{i}}\frac{s_{j}}{\sum_{k\in S_{i}}s_{k}} & = \sum_{i'\mid j\in S^\ast_{i'}} R_{i'} \cdot \frac{s_{j}}{\sum_{k\in S^\ast_{i'}}s_{k}},
\end{align*}
where each $S_{i'}^\ast$ is a unique choice set that appears in $R_{i'}\geq 1$ observations, for $i'=1,\dots,n^\ast \leq n$. By construction, $\sum_{i'=1}^{n^\ast}R_{i'} =n$. Note, however, that the selected item could vary across different appearances of $S_i^\ast$, but the optimality conditions only involve each item's winning count $W_j$. From now on, we assume this reduction and drop the $^\ast$ superscript. In other words, we assume that we observe $n$ unique choice sets, and choice set $S_i$ has \emph{multiplicity} $R_i$. The resulting problem has optimality conditions
\begin{align}
\label{eq:optimality}
{W_j} & = \sum_{i\mid j\in S_{i}}{R_i} \cdot \frac{s_{j}}{\sum_{k\in S_{i}}s_{k}}.
\end{align}
We are now ready to reformulate \eqref{eq:optimality} as a canonical matrix balancing problem. Define $p\in\mathbb{R}^{n}$ as $p_i=R_i$, i.e., the number of times choice set $S_i$ appears in the data. Define
$q\in\mathbb{R}^{m}$ as $q_j={W_j}$,
i.e., the number of times item $j$ was \emph{selected} in the data. By construction we have $\sum_i p_i=\sum_j q_j$, and $p_i,q_j>0$ whenever \cref{ass:strong-connected} holds.

Now define the $n\times m$ binary matrix $A$ by
$A_{ij}=1\{j\in S_{i}\}$, so the $i$-th row of $A$
is the indicator of which items appear in the (unique) choice set $S_i$, and the $j$-th column of $A$ is the indicator of which choice sets
item $j$ appears in. We refer to this $A$ constructed from a choice dataset as the \emph{participation matrix}. By construction, $A$ has distinct rows, but may still have identical columns. If necessary, we can also remove repeated columns by ``merging'' items and their win counts. Their estimated scores can be computed from the score of the merged item proportional to their respective win counts.

Let $D^{0}\in\mathbb{R}^{m\times m}$ be the diagonal matrix with
$D_{j}^{0}=s_{j}$ and $D^{1}\in\mathbb{R}^{n\times n}$ be the
diagonal matrix with $D_{i}^{1}={R_i}/{\sum_{k\in S_{i}}s_{k}}$,
and define the scaled matrix
\begin{align}
\label{eq:scaled-matrix}
\hat{A} & :=D^{1}AD^{0}.
\end{align}
The matrices $D^{1}$ and $D^{0}$ are scalings of rows and columns
of $A$, respectively, and
\begin{align*}
    \hat{A}_{ij} = \frac{R_i}{\sum_{k\in S_{i}}s_{k}}\cdot1\{j\in S_{i}\}\cdot s_{j}.
\end{align*}
The key observation is that the optimality condition \eqref{eq:optimality} can be rewritten as
\begin{align}
\label{eq:bridge}
\hat{A}^T \mathbf{1}_n & = q.
\end{align}
Moreover, by construction $\hat{A}$ also satisfies
\begin{align}
\label{eq:marginal}
\hat{A}\mathbf{1}_m & = p.
\end{align}
 Therefore, if $s_j$'s satisfy the optimality conditions for maximizing \eqref{eq:log-likelihood}, then $D^0,D^1$  defined above solve the matrix balancing problem in \cref{eq:scaled-matrix,eq:bridge,eq:marginal}. Moreover, the converse is also true, and we thus establish the equivalence between choice maximum likelihood estimation and matrix balancing. All omitted proofs appear in \Cref{app:proofs}.
\begin{theorem}
\label{prop:mle-scaling}
Let $p\in\mathbb{R}^{n}$ with $p_{i}=R_i$, $q\in\mathbb{R}^{m}$ with $q_{j}=W_j$, and 
$A\in\mathbb{R}^{n\times m}$
with $A_{ij}=1\{j\in S_{i}\}$ be constructed from the choice dataset. Then  $D^{0},D^{1}>0$ with $\sum_j D_j^0=1$ solves the matrix balancing problem
\begin{align}
\label{eq:equation-system}
\begin{split}
\hat{A} & =D^{1}AD^{0}\\
\hat{A}\mathbf{1}_m & =p\\
\hat{A}^{T} \mathbf{1}_n & =q
\end{split}
\end{align}
if and only if $s \in \Delta_m$ with $s_j=D^0_j$ satisfies the optimality condition \eqref{eq:optimality} of the ML estimation problem.
\end{theorem}
In particular, \eqref{eq:log-likelihood} has a unique maximizer $s$ in the interior of the probability simplex if and only if \eqref{eq:equation-system} has a unique normalized solution $D^0$ as well. The next question, naturally, is then how \cref{ass:strong-connected} and \cref{ass:weak-connected} for choice modeling are connected to \cref{ass:matrix-existence} and \cref{ass:matrix-uniqueness} for matrix balancing.
\begin{theorem}
\label{thm:necessary-and-sufficient}
Let (A,p,q) be constructed from the choice dataset as in \cref{prop:mle-scaling}, with $p,q>0$. \cref{ass:weak-connected} is equivalent to \cref{ass:matrix-uniqueness}. Furthermore, \cref{ass:strong-connected} holds if and only if $(A,p,q)$ satisfy \cref{ass:matrix-existence} and $A$ satisfies \cref{ass:matrix-uniqueness}.
\end{theorem} 
Thus when the ML estimation problem is cast as a matrix balancing problem, \cref{ass:matrix-existence} exactly characterizes the \emph{gap} between \cref{ass:weak-connected} and \cref{ass:strong-connected}. 
 We provide some intuition for \cref{thm:necessary-and-sufficient}. When we construct a triplet $(A,p,q)$ from a choice dataset, with $p$ the numbers of appearances of unique choice sets and $q$ the winning counts, \cref{ass:matrix-uniqueness} precludes the possibility of partitioning the items into two subsets that never get compared with each other, i.e., \cref{ass:weak-connected}. Then \cref{ass:matrix-existence} requires that whenever a strict subset $M\subsetneq [m]$ of objects only appear in a strict subset $N\subsetneq [n]$ of the observations, their total winning counts are \emph{strictly} smaller than the total number of these observations, i.e., there is some object $j\notin M$ that is selected in $S_i$ for some $i\in N$, which is required by \cref{ass:strong-connected}.

Interestingly, while \cref{ass:strong-connected} requires the directed comparison graph, defined by the $m\times m$ matrix of counts of item $j$ being chosen over item $k$, to be strongly connected, the corresponding conditions for the equivalent matrix balancing problem concern the $n\times m$ participation matrix $A$ and positive vectors $p,q$, which do not explicitly encode the specific \emph{choice} of each observation. This apparent discrepancy is due to the fact that $(A,p,q)$ form the \emph{sufficient statistics} of the Luce choice model. In other words, there can be more than one choice dataset with the same optimality condition \eqref{eq:optimality} and $(A,p,q)$ defining the equivalent matrix balancing problem.

\textbf{Remark.} This feature where ``marginal'' quantities constitute the sufficient statistics of a parametric model is an important one that underlies many works in economics and statistics \citep{kullback1997information,stone1962multiple,good1963maximum,birch1963maximum,theil1967economics,fienberg1970iterative,berry1995automobile,fofana2002balancing,maystre2017choicerank}. It makes the task of estimating a \emph{joint} model from marginal quantities feasible. This feature is useful because in many applications, only marginal data is available due to high sampling cost or privacy reasons. 

Having formulated a particular matrix balancing problem from the estimation problem given choice data, we may ask how one can go in the other direction. In other words, when/how can we construct a ``choice dataset'' whose sufficient statistics is a given triplet $(A,p,q)$? First off, for $(A,p,q)$ to be valid sufficient statistics of a Luce choice model, $p,q$ need to be positive integers. Moreover, $A$ has to be a binary matrix, with each row containing at least two non-zero elements (valid choice sets have at least two items). Given such a $(A,p,q)$ satisfying Assumptions \ref{ass:matrix-existence} and \ref{ass:matrix-uniqueness}, a choice dataset can be constructed efficiently. Such a procedure is described, for example, in \citet{kumar2015inverting}, where $A$ is motivated
by random walks on a graph instead of matrix balancing (\cref{sec:connections}). Their construction relies on finding the max flow on the bipartite graph $G_b$. For rational $p,q$, this maximum flow can be found efficiently in polynomial time \citep{balakrishnan2004polynomial,idel2016review}. Moreover, the maximum flow implies a matrix $A'$ satisfying \cref{ass:matrix-existence}(a), thus providing a feasibility certificate for the matrix balancing problem as well. 

We have thus closed the loop and fully established the equivalence of the maximum likelihood estimation of Luce choice models and the canonical matrix balancing problem.
\begin{corollary}
    There is a one-to-one correspondence between classes of maximum likelihood estimation problems with the same optimality conditions \eqref{eq:optimality} and canonical matrix balancing problems with $(A,p,q)$, where $A$ is a valid participation matrix and $p,q>0$ have integer entries. 
\end{corollary}
We next turn our attention to the algorithmic connections between choice modeling and matrix balancing. 
\subsection{Algorithmic Connections between Matrix Balancing and Choice Modeling}
\label{subsec:IPF}
 Given the equivalence between matrix balancing and choice modeling, we can naturally consider applying Sinkhorn's algorithm to maximize \eqref{eq:log-likelihood}. In this case, one can verify that the updates in each full iteration of \cref{alg:scaling} reduce algebraically to
\begin{align}
\label{eq:scaling-iteration}
s_{j}^{(t+1)} & =W_{j}/\sum_{i\mid j\in S_{i}}\frac{R_i}{\sum_{k\in S_{i}}s_{k}^{(t)}}
\end{align}
 in the $t$-th iteration. 
 Comparing \eqref{eq:scaling-iteration} to the optimality condition in \eqref{eq:optimality}, which we recall is given by 
\begin{align*}
 W_j & =\sum_{i\mid j\in S_{i}} R_i\frac{s_{j}}{\sum_{k\in S_{i}}s_{k}}= s_{j} \cdot \sum_{i\mid j\in S_{i}}\frac{R_i}{\sum_{k\in S_{i}}s_{k}},
 \end{align*}
we can therefore interpret the iterations as simply dividing the winning count $W_j$ by the coefficient of $s_j$ on the right repeatedly, in the hope of converging to a \emph{fixed point}. A similar intuition was given by \citet{ford1957solution} in the special case of pairwise comparisons. Indeed, the algorithm proposed by \citet{ford1957solution} is a cyclic variant of \eqref{eq:scaling-iteration} applied to pairwise comparisons. However, this connection is mainly algebraic, as the optimality condition in \citet{ford1957solution} does not admit a reformulation as the matrix balancing problem in \eqref{eq:equation-system}.

In \cref{sec:connections}, we provide further discussions on the connections of Sinkhorn's algorithm to existing frameworks and algorithms in the choice modeling literature, and connect it to distributed optimization as well. We demonstrate that many existing algorithms for Luce choice model estimation are in fact special cases or analogs of Sinkhorn's algorithm. These connections also illustrate the many interpretations of Sinkhorn's algorithm, e.g., as a distributed optimization algorithm as well as a minorization-majorization (MM) algorithm. However, compared to most algorithms for choice modeling discussed in this work, Sinkhorn's algorithm is more general as it applies to non-binary $A$ and non-integer $p,q$, and has the critical advantage of being paralellized and distributed, hence more efficient in practice.

The mathematical and algorithmic connections between matrix balancing and choice modeling we establish in this paper allow the transfusion of ideas in both directions. For example, inspired by regularized maximum likelihood estimation \citep{maystre2017choicerank}, we propose a regularized version of Sinkhorn's algorithm in \cref{subsec:regularization}, which is guaranteed to converge even when the original Sinkhorn's algorithm does not converge. Moreover, the importance of algebraic connectivity in quantifying estimation and computation efficiency in choice modeling motivates us to solve some important open problems on the convergence of Sinkhorn's algorithm. We turn to this topic next. 
\section{Linear Convergence of Sinkhorn's Algorithm for Non-negative Matrices}
\label{sec:linear-convergence}
In this section, we turn our focus to matrix balancing and study the global and asymptotic linear convergence rates of Sinkhorn's algorithm for general non-negative matrices and positive marginals. We first discuss relevant quantities and important concepts before presenting the convergence results in \cref{subsec:global-linear-convergence} and \cref{subsec:sharp-rate}.
Throughout, we use superscript $(t)$ to denote quantities after $t$ iterations of Sinkhorn's algorithm.

\subsection{Preliminaries}
\begin{table*}
\caption{Summary of some convergence results on Sinkhorn's algorithm. In \citet{franklin1989scaling}, $\kappa(A)=\frac{\theta(A)^{1/2}-1}{\theta(A)^{1/2}+1}$,
where $\theta(A)$ is the diameter of $A$ in the Hilbert metric.
The norm in \citet{knight2008sinkhorn} is not explicitly specified, and $\sigma_{2}(\hat{A})$
denotes the second largest singular value of the scaled
doubly stochastic matrix $\hat{A}$. The bound in \citet{altschuler2017near} was
originally stated as $\|r^{(t)}-p\|_{1}\protect\leq\epsilon'$
in $t=O(\epsilon'^{-2}\log(\frac{\sum_{ij}A_{ij}}{\min_{ij}A_{ij}}))$
iterations. The result in \citet{leger2021gradient} applies more generally to couplings
of probability distributions. In view of Pinsker's inequality, it implies the bound in \citet{altschuler2017near} but with a constant that is finite even when $A$ has zero entries. In the bound in \citet{knight2008sinkhorn} and our
asymptotic result, the $\lambda+\epsilon$ denotes an asymptotic rate, with the
bound valid for any $\epsilon>0$ and all $t$ sufficiently large. In our global bound, the linear rate $\lambda_{-2}(\mathcal{L})$ is the second \emph{smallest} eigenvalue of the Laplacian of the bipartite graph defined by $A$ (see \cref{subsec:graph-laplacian}), $ l=\min \{\max_j (A^T\mathbf{1}_n)_j, \max_i (A\mathbf{1}_m)_i\}$,
$c_B=\exp(-4B)$, and $B$ is a bound on the initial sub-level set, which is finite if and only if \cref{ass:matrix-existence} holds.
}
 \begin{adjustwidth}{-1.5cm}{}
\begin{centering}
\begin{tabular}{c|c|c|c|c}
 & convergence statement & $\lambda$ & $A$ & $p,q$\tabularnewline
\hline 
\citet{franklin1989scaling} & $d_{\text{Hilbert}}(r^{(t)},p)\leq\lambda^t d_{\text{Hilbert}}(r^{(0)},p)$ & $\kappa^{2}(A)$ & $A>0$, rectangular & uniform\tabularnewline
\hline 
\citet{luo1992convergence} & $g(u^{(t)},v^{(t)})-g^\ast\leq\lambda^t (g(u^{(0)},v^{(0)})-g^\ast)$ & \text{unknown} & $A\geq0$, rectangular & general\tabularnewline
\hline 
\citet{knight2008sinkhorn} & $\|D_{t+1}^{0}-D^{0}\|_{\ast}\leq(\lambda+\epsilon)\|D_{t}^{0}-D^{0}\|_{\ast}$ & $\sigma_{2}^{2}(\hat{A})$ & $A\geq0$, square  & uniform\tabularnewline
\hline 
\citet{pukelsheim2009iterative} & $\|r^{(t)}-p\|_{1}\rightarrow0$ & no rate & $A\geq0$, rectangular & general\tabularnewline
\hline 
\citet{altschuler2017near} & $\|r^{(t)}-p\|_{1}\leq c \sqrt{\frac{\lambda}{t}}$ & $\log(\frac{\sum_{ij}A_{ij}}{\min_{ij}A_{ij}})$ & $A>0$, rectangular & general\tabularnewline
\hline 
\citet{leger2021gradient} & $D_{\text{KL}}(r^{(t)}\| p) \leq\frac{\lambda}{t}$ & $D_{\text{KL}}(\hat{A}\| A)$ & $A\geq0$, continuous & general\tabularnewline
\hline 
current work, asymptotic & $\|\frac{r^{(t+1)}}{\sqrt{p}}-\sqrt{p}\|_{2}\leq(\lambda+\epsilon)\|\frac{r^{(t)}}{\sqrt{p}}-\sqrt{p}\|_{2}$ & $\lambda_{2}(\tilde{A}^T\tilde{A})$ & $A\geq0$, rectangular & general\tabularnewline
\hline 
current work, global & $g(u^{(t)},v^{(t)})-g^\ast\leq\lambda^t (g(u^{(0)},v^{(0)})-g^\ast)$ & $1-c_B\lambda_{-2}(\mathcal{L})/l$ & $A\geq0$, rectangular & general\tabularnewline
\end{tabular}
\par\end{centering}
\label{tab:convergence-summary}
\end{adjustwidth}
\end{table*}
We start with the optimization principles associated with matrix balancing and Sinkhorn's algorithm. Consider the following KL divergence (relative entropy) minimization problem 
     \begin{align}
\label{eq:relative-entropy-minimization}
\begin{split}
  \min_{\hat A\in \mathbb{R}^{n\times m}_+} & D_{\text{KL}}(\hat{A}\| A)\\
\hat{A}\mathbf{1}_m & =p\\
\hat{A}^{T}\mathbf{1}_n & =q.
\end{split}
\end{align}
It is well-known that solutions $\hat{A}=D^1AD^0$ to the matrix balancing problem with $(A,p,q)$ are minimizers of \eqref{eq:relative-entropy-minimization} \citep{ireland1968contingency,bregman1967proof}. Moreover, Sinkhorn's algorithm can be interpreted as a block coordinate descent type algorithm applied to minimize the following dual problem of \eqref{eq:relative-entropy-minimization}:
 \begin{align}
 \label{eq:log-barrier}
     g(d^0,d^1)	:=(d^1)^{T}Ad^0-\sum_{i=1}^{n}p_{i}\log d^1_{i}-\sum_{j=1}^{m}q_{j}\log d^0_{j},
\end{align} 
\citet{luo1992convergence} study the linear convergence of block coordinate descent algorithms. Their result implies that the
convergence of Sinkhorn's algorithm, measured in terms of the optimality gap of $g$, is linear with some implicit rate $\lambda>0$,
 as long as finite positive scalings $D^0,D^1$ exist for the matrix balancing problem. Minimizers $d^0,d^1$ of \eqref{eq:log-barrier} precisely give the diagonals of $D^0,D^1$. The function $g$, known to be a \emph{potential function} of Sinkhorn's algorithm, also turns out to be crucial in quantifying the global linear convergence rate in the present work. 

\textbf{Remark.} Interestingly, minimizing \eqref{eq:log-barrier} is in fact equivalent to maximizing the log-likelihood function $\ell(s)$ in \eqref{eq:log-likelihood} for valid $(A,p,q)$, because $\min_{d^1}g(d^0,d^1)=-\ell(d^0)+c$ for some $c>0$. Moreover, the optimality condition of minimizing $g$ with respect to $d^0$ reduces to the optimality condition \eqref{eq:optimality}. A detailed discussion can be found in \cref{sec:sinkhorn-MM}. This connection relates choice modeling and matrix balancing from an optimization perspective.
     
      Although convergence results on Sinkhorn's algorithm are abundant, the recent work of \citet{leger2021gradient} stands out as the first \emph{explicit} global convergence result applicable to general non-negative matrices, with a sub-linear $\mathcal O(1/t)$ bound on the KL divergence with respect to target marginals. It implies the bounds in \citet{chakrabarty2021better,altschuler2017near} but with a constant that is finite even when $A$ has zero entries. The result in \citet{leger2021gradient} applies more generally to couplings of continuous probability distributions, but when restricted to the discrete matrix balancing problem, it 
     holds under the following equivalent conditions that are weaker than \cref{ass:matrix-existence}.
\begin{assumption}[\textbf{Weak Existence}]
    \label{ass:matrix-weak-existence}
    
    \textbf{(a)} There exists a non-negative matrix $A'\in \mathbb{R}_+^{n\times m}$ that inherits all zeros of $A$ and has row and column sums $p$ and $q$. Or, equivalently,
    
\textbf{(b)} For every pair of sets of indices $N \subsetneq [n]$ and $M \subsetneq [m]$ such that $A_{ij}=0$ for $i\notin N$ and $j\in M$, $\sum_{i\in N}p_i \geq \sum_{j\in M}q_j$.
\end{assumption}

The equivalence of these conditions follows from Theorem 4 in \citet{pukelsheim2009iterative}, which also shows that they are the minimal requirements for the convergence of Sinkhorn's algorithm. \cref{ass:matrix-weak-existence}(a) precisely guarantees 
that the optimization problem \eqref{eq:relative-entropy-minimization} is feasible and bounded. It relaxes \cref{ass:matrix-existence}(a) by allowing additional zeros in the matrix $A'$. Similarly, \cref{ass:matrix-weak-existence}(b) relaxes \cref{ass:matrix-existence}(b) by allowing equality between $\sum_{i\in N}p_i$ and $\sum_{j\in M}q_j$ even when $M,N$ do not correspond to a block-diagonal structure. 

The distinction between \cref{ass:matrix-existence} and \cref{ass:matrix-weak-existence} is crucial for the matrix balancing problem and Sinkhorn's algorithm. Recall that \cref{ass:matrix-existence} guarantees the matrix balancing problem has a solution $(D^0,D^1)$, and $D^1AD^0$ is always a solution to \eqref{eq:relative-entropy-minimization}. On the other hand, the weaker condition \cref{ass:matrix-weak-existence} guarantees that \eqref{eq:relative-entropy-minimization} has a solution $\hat A$.
If indeed $\hat A$ has additional zeros relative to $A$, then no direct (finite and positive) scaling $(D^{0},D^{1})$ exists such that $\hat A=D^1AD^0$. However, the sequence of scaled matrices $\hat A^{(t)}$ from Sinkhorn's algorithm still converges to $\hat A$. In this case, the matrix balancing problem is said to have a \emph{limit} scaling, where some entries of $d^{0},d^{1}$ in Sinkhorn iterations approach 0 or $\infty$, resulting in additional zeros in $\hat A$. Below we give an example adapted from \citet{pukelsheim2009iterative}, where $p,q=(3,3)$ and the scaled matrices $\hat A^{(t)}$ converge but no direct scaling exists:
\begin{align*}
{D^{1}}^{(t)}\begin{bmatrix}3&1\\
0 & 2
\end{bmatrix}{D^{0}}^{(t)} = \begin{bmatrix}1&0\\
0 & \frac{3t}{2}
\end{bmatrix}\begin{bmatrix}3&1\\
0 & 2
\end{bmatrix}\begin{bmatrix}1&0\\
0 & \frac{1}{t}
\end{bmatrix}	\rightarrow \begin{bmatrix}3&0\\
0 & 3
\end{bmatrix}.
\end{align*}

Given these discussions, it is therefore important to clarify the convergence behaviors of Sinkhorn's algorithm in different regimes. In particular, it remains to reconcile the gap between the implicit linear convergence result of \citet{luo1992convergence} under strong existence, and the quantitative sub-linear bound of \citet{leger2021gradient} under weak existence. Furthermore, it remains to provide explicit characterizations of both the global and asymptotic (local) rates when Sinkhorn's algorithm does converge linearly. 

Our results in this section provide answers to these questions. We show that the $\mathcal O(1/t)$ rate can be sharpened to a global $\mathcal O(\lambda^t)$ bound if and only if the weak existence condition (\cref{ass:matrix-weak-existence}) is replaced by the strong existence condition (\cref{ass:matrix-existence}). Moreover, we provide an explicit global linear convergence rate $\lambda$ in terms of the \emph{algebraic connectivity}, revealing the structure-dependent nature of Sinkhorn's algorithm for problems with non-negative matrices. This generalizes the implicit result of \citet{luo1992convergence} and sheds light on how different assumptions impact Sinkhorn's convergence, which is explicitly reflected in the constants of the bound. Going further, we characterize the sharp asymptotic rate of linear convergence in terms of the second largest singular value of $\mathcal{D}(1/\sqrt{p})\cdot\hat{A}\cdot\mathcal{D}(1/\sqrt{q})$, where $\mathcal{D}$ denotes the diagonalization of a vector. This asymptotic rate reduces to that given by \citet{knight2008sinkhorn} for $m=n$ and uniform $p,q$.

The choice of convergence measure is important, and previous works have used different convergence measures. First note that after each iteration in \cref{alg:scaling}, the column constraint is always satisfied: ${A}^{(t)}\mathbf{1}_n=q$, where ${A}^{(t)}$ is the scaled matrix after $t$ iterations. 
\citet{leger2021gradient} uses the KL divergence $D_{\text{KL}}(r^{(t)}\| p)$ between the row sum $r^{(t)}={A}^{(t)}\mathbf{1}_m$ and the target row sum $p$ to measure convergence. \citet{franklin1989scaling} use the Hilbert projective metric between $r^{(t)}$ and $p$. \citet{pukelsheim2009iterative} and \citet{altschuler2017near} use the $\ell^1$ distance, which is upper bounded by the KL divergence via Pinsker's inequality. \citet{knight2008sinkhorn} focuses on the convergence of the scaling diagonal matrix $D^0=\mathcal{D}(d^0)$ to the optimal solution \emph{line}, but does not explicitly specify the norm.
Some bounds are \emph{a priori} and hold globally for all iterations, while others hold locally in a neighborhood of the optimum.
We summarize the relevant convergence results in \cref{tab:convergence-summary}. Here $\lambda_{-2}(S)$ denotes the second smallest eigenvalue of a real symmetric matrix $S$, and $\lambda_{2}(S)$ the second largest eigenvalue. In our work, we characterize the global linear convergence through the optimality gap of \eqref{eq:log-barrier}, which naturally leads to a bound on $\|r^{(t)}-p\|_1$. For the sharp asymptotic rate, we choose to use the $\ell^2$ distance $\|r^{(t)}/\sqrt{p}-\sqrt{p}\|_2$ in order to exploit an intrinsic orthogonality structure afforded by Sinkhorn's algorithm. This approach results in a novel analysis compared to \citet{knight2008sinkhorn} that most explicitly reveals the importance of spectral properties in  the rate of convergence.

\subsection{Global Linear Convergence}
\label{subsec:global-linear-convergence}
We first present the global linear convergence results. Our analysis starts with the following change of variables to transform the potential function \eqref{eq:log-barrier}:
\begin{align}
\label{eq:change-of-variables}
    u:=\log d^0,\quad v:=-\log d^1.
\end{align}
This results in the potential function $g(u,v)$ defined as
\begin{align}
\label{eq:transformed-potential}
  g(u,v)	:=\sum_{ij}A_{ij}e^{-v_{i}+u_{j}}+\sum_{i=1}^{n}p_{i}v_{i}-\sum_{j=1}^{m}q_{j}u_{j},
\end{align}
and we can verify that Sinkhorn's algorithm is equivalent to the alternating minimization algorithm \citep{bertsekas1997nonlinear,beck2013convergence} for \eqref{eq:transformed-potential}, which alternates between minimizing with respect to $u$ and $v$, holding the other block fixed:
\begin{align}
   \label{eq:alternating-minimization} u_j^{(t)}\leftarrow  \log \frac{q_j}{\sum_i A_{ij}e^{-v^{(t-1)}_i}},\quad v_i^{(t)}\leftarrow  \log \frac{p_i}{\sum_j A_{ij}e^{u^{(t)}_j}}.
\end{align}
The Hessian $\nabla^2g(u,v)$ always has $\mathbf{1}_{m+n}$ in its null space. On the surface, standard linear convergence results for first-order methods, which require strong convexity (or related properties like the Polyak--Lojasiewicz condition) of the objective function, do not apply to $g(u,v)$. However, we show that under strong existence and uniqueness conditions for the matrix balancing problem, $g(u,v)$ is in fact strongly convex when \emph{restricted} to the subspace 
\begin{align*}
   \mathbf{1}_{m+n}^\perp:= \{u\in \mathbb{R}^m,v\in \mathbb{R}^n:(u,v)^T\mathbf{1}_{m+n}=0\}.
\end{align*}
As a result, Sinkhorn's algorithm converges linearly with a rate that depends on the (restricted) condition number of its Hessian.

Before proceeding, we introduce a slew of useful definitions. Let Sinkhorn's algorithm initialize with $u^{(0)}$, and $v^{(0)}$ given by \eqref{eq:alternating-minimization}. 
Define the constant $B$ as 
\begin{align*}
   B:&= \sup_{(u,v)} \|(u,v)\|_\infty\\  \text{subject to } (u,v)^{T}&\mathbf{1}_{m+n}=0,\\
   g(u,v)&\leq g(u^{(0)},v^{(0)}).
\end{align*}
In other words, $B$ is the \emph{diameter} of the initial normalized sub-level set. We will show that $B$ is finite and that it bounds normalized Sinkhorn iterates by the \emph{coercivity} of $g(u,v)$ under \cref{ass:matrix-existence}.  
We similarly define the normalized optimal solution pair 
\begin{align}
\label{eq:normalized-optimum}
    (u^\ast,v^\ast):=\arg \min_{(u,v)\in \mathbf{1}_{m+n}^\perp} g(u,v),
\end{align}
and $g^\ast:=g(u^\ast,v^\ast)$.
Finally, define 
\begin{align*}
    l_0:= \max_j (A^T\mathbf{1}_n)_j, \quad l_1:= \max_i (A\mathbf{1}_m)_i,
\end{align*}
which are the Lipschitz constants of the two sub-blocks of $g(u,v)$. Next, define the \emph{Laplcian} matrix $\mathcal{L}$ of the bipartite graph $G_b$ as 
\begin{align*}
\mathcal{L}:&=\begin{bmatrix}\mathcal{D}({A}\mathbf{1}_{m}) & -{A}\\
-{A}^{T} & \mathcal{D}({A}^{T}\mathbf{1}_{n})
\end{bmatrix}, 
\end{align*}
and refer to the second smallest eigenvalue $\lambda_{-2}(\mathcal{L})$ as the Fiedler eigenvalue. For details on the graph Laplacian and the Fielder eigenvalue, see \cref{subsec:graph-laplacian}.

Using the above notation, we can now state one of our main contributions to the study of Sinkhorn's algorithm. 
\begin{theorem}[\textbf{Global Linear Convergence}]
\label{thm:global-convergence}
Suppose \cref{ass:matrix-existence} and \cref{ass:matrix-uniqueness} hold. For all $t>0$,
\begin{align}
\label{eq:potential convergence}
g(u^{(t+1)},v^{(t+1)}) - g^\ast \leq (1-c_B \frac{\lambda_{-2}(\mathcal{L})} {l})\left( g(u^{(t)},v^{(t)})-g^\ast \right),
    \end{align}
    where $c_B=e^{-4B}$ and $l=\min\{l_0,l_1\}$.
      
    As a consequence, we have the following bound:
    \begin{align*}
       \|r^{(t)}-p\|_{1} &\leq c'_Be^{-c_{B}\frac{\lambda_{-2}(\mathcal{L})}{\min\{l_{0},l_{1}\}}\cdot t},
    \end{align*} 
    where $(c'_B)^2=(8B\sum_{i}p_{i})$.
\end{theorem}
\cref{thm:global-convergence} immediately implies the following iteration complexity bound.
\begin{corollary}[\textbf{Iteration Complexity}]
Under \cref{ass:matrix-existence} and \cref{ass:matrix-uniqueness}, $\|r^{(t)}-p\|_1\leq \epsilon$ after
      \begin{align*}
          \mathcal{O} \left (\frac{\min\{l_{0},l_{1}\}}{\lambda_{-2}(\mathcal{L})} \cdot \log (1/\epsilon) \right )
      \end{align*}
       iterations of Sinkhorn's algorithm.
\end{corollary}

\textbf{Remark.} The ability of Sinkhorn's algorithm to exploit the strong convexity of $g(u,v)$ on $\mathbf{1}_{m+n}^\perp$ relies critically on the invariance of $g(u,v)$ under \emph{normalization}, which is an intrinsic feature of the problem that has been largely set aside in the convergence analysis so far. Recall that $u=\log d^0$ and $v=-\log d^1$, where $d^0,d^1$ are the diagonals of the scaling $(D^0,D^1)$. Scalings are only determined up to multiplication by $(1/c,c)$ for $c>0$, and the translation $(u,v)\rightarrow(u-\log c,v-\log c)$ does not alter the objective value in \eqref{eq:transformed-potential}. We may therefore impose an \emph{auxiliary} normalization $(u,v)^T\mathbf{1}_{m+n}=0$, or equivalently $\prod_j d^0_j = \prod_i d^1_i$. This normalization is easily achieved by requiring that after every update of Sinkhorn's algorithm, a normalization $(d^0/c,c d^1)$ is performed using the normalizing constant 
\begin{align}
\label{eq:normalization}
   c=\sqrt{\prod_j d^0_j /\prod_i d^1_i}.
\end{align}
See \cref{alg:scaling-normalized}. Note, however, that this normalization is only a supplementary construction in our analysis. The final convergence result applies to the original Sinkhorn's algorithm without normalization, since it does not alter the objective value. Normalization of Sinkhorn's algorithm is discussed in \citet{carlier2023sista}, although they use the asymmetric condition $u_0=0$, which does not guarantee that normalized Sinkhorn iterates stay in $\mathbf{1}_{m+n}^\perp$. 
\begin{algorithm}[tb]
\caption{Normalized Sinkhorn's Algorithm}
   \label{alg:scaling-normalized}
\begin{algorithmic}
   \STATE {\bfseries Input:}  $A, p, q,\epsilon_{\text{tol}}$.
   \STATE {\bfseries initialize} $d^{0}\in\mathbb{R}_{++}^{m}$
   \REPEAT
   \STATE $d^{1} \leftarrow  p/( A d^0)$ 
   \STATE 
  normalization  $(d^0,d^1) \leftarrow (d^0/c,c d^1),c>0$
   \STATE $d^{0}\leftarrow  q/({A}^{T} d^{1})$
   \STATE 
  normalization  $(d^0,d^1) \leftarrow (d^0/c,c d^1),c>0$ 
   \STATE 
$\epsilon\leftarrow$  update of $(d^{0},d^1)$
\UNTIL{$\epsilon<\epsilon_{\text{tol}}$}
\end{algorithmic}
\end{algorithm}

The proof of \cref{thm:global-convergence} then relies crucially on the observation that the Hessian of $g(u,v)$ at $(0,0)$ is precisely the \emph{Laplacian} $\mathcal{L}$ of the bipartite graph $G_b$. Therefore, as $(u,v)$ are \emph{bounded} throughout the iterations thanks to the coercivity of $g$, the Fiedler eigenvalue of $\mathcal{L}$ quantifies the strong convexity on $\mathbf{1}_{m+n}^\perp$. The linear convergence then follows from standard results on block coordinate descent and alternating minimization methods for strongly convex and smooth functions \citep{beck2013convergence}. Typically, the leading eigenvalue of the Hessian quantifies the smoothness \citep{luenberger1984linear}. This is given by $2\max\{l_0,l_1\}$ for $\mathcal{L}$. However, for alternating minimization methods, the better smoothness constant $\min\{l_0,l_1\}$ is available. Thus the quantity $\min\{l_{0},l_{1}\}/\lambda_{-2}(\mathcal{L})$ can be interpreted as a type of  ``condition number'' of the graph Laplacian $\mathcal{L}$. When $A$ is positive (not just non-negative), then the strong existence and uniqueness conditions are trivially satisfied, and our results continue to hold with the rate quantified by $\min\{l_{0},l_{1}\}/\lambda_{-2}(\mathcal{L})$.

\textbf{Remark.} The importance of Assumptions \ref{ass:matrix-existence} and \ref{ass:matrix-uniqueness} are clearly reflected in the bound \eqref{eq:potential convergence}.
First, note that the Fiedler eigenvalue $\lambda_{-2}(\mathcal{L})>0$ iff \cref{ass:matrix-uniqueness} holds (see \cref{subsec:graph-laplacian}). On the other hand, \cref{ass:matrix-existence} guarantees the \emph{coercivity} of $g$ on $\mathbf{1}_{m+n}^\perp$. This property ensures that $B<\infty$, and consequently, that normalized iterates stay bounded by $B$. That \cref{ass:matrix-existence} guarantees  $g(u,v)$ is coercive should be compared to the observation by \citet{hunter2004mm} that \cref{ass:strong-connected} guarantees the upper compactness (a closely related concept) of the log-likelihood function \eqref{eq:log-likelihood}.
In contrast, when only the weak existence condition (\cref{ass:matrix-weak-existence}) holds, finite minimizer of $g(u,v)$ may not exist, in which case the diameter $B$ of the initial sub-level set may become infinite.

\cref{ass:matrix-weak-existence} corresponds to the ``limit scaling'' regime of Sinkhorn's algorithm, where the scaled matrices $A^{(t)}$ are guaranteed to converge to a finite matrix $\hat A$ with the desired marginal distributions that solves \eqref{eq:relative-entropy-minimization}, and may have additional zeros compared to $A$. Under \cref{ass:matrix-weak-existence}, \citet{leger2021gradient} shows the slower $\mathcal O(1/t)$ convergence in KL divergence $D_{\text{KL}}(r^{(t)}\| p)$. We now show that this rate is tight, which fully characterizes the following convergence behavior of Sinkhorn's algorithm: whenever a direct scaling exists for the matrix balancing problem, Sinkhorn's algorithm converges linearly. If only a limit scaling exists, then convergence deteriorates to 
$\mathcal O(1/t)$.
\begin{theorem}
\label{thm:lower-bound}
       For general non-negative matrices, Sinkhorn's algorithm converges linearly
iff $(A,p,q)$ satisfy \cref{ass:matrix-existence} and \cref{ass:matrix-uniqueness}. The convergence deteriorates to sub-linear iff the weak existence condition \cref{ass:matrix-weak-existence} holds but \cref{ass:matrix-existence} fails.
\end{theorem}

The regime of sub-linear convergence also has an interpretation in the choice modeling framework. The weak existence condition \cref{ass:matrix-weak-existence}, when applied to $(A,p,q)$ constructed from a choice dataset, allows the case where some subset $S$ of items is always preferred over $S^C$, which implies, as observed already by the early work of \citet{ford1957solution}, that the log-likelihood function \eqref{eq:log-likelihood} is only maximized at the \emph{boundary} of the probability simplex, by shrinking  $s_j$ for $j\in S^C$ towards 0, i.e., $D^0_j \rightarrow 0$. Incidentally, \citet{bacharach1965estimating} also refers to the corresponding regime in matrix balancing as ``boundary solutions''. 

\subsection{Sharp Asymptotic Rate}
\label{subsec:sharp-rate}
Having established the global convergence of Sinkhorn's algorithm when finite scalings exist, we now turn to the open problem of characterizing its asymptotic linear convergence rate for non-uniform marginals. Our analysis relies on an \emph{intrinsic} orthogonality structure of Sinkhorn's algorithm instead of the auxiliary  normalization used to prove the global linear convergence above. Note that unlike the global rate, which depends on $A$, the asymptotic rate now depends on the associated solution $\hat A$ (and $p,q$), as expected.  
\begin{theorem}[\textbf{Sharp Asymptotic Rate}]
\label{thm:convergence}
Suppose $(A,p,q)$ satisfy \cref{ass:matrix-existence} and \cref{ass:matrix-uniqueness}. Let $\hat{A}$
be the unique scaled matrix with marginals $p,q$. Then
\begin{align*}
\lim_{t\rightarrow \infty} \frac{\|r^{(t+1)}/\sqrt{p}-\sqrt{p}\|_{2} }{\|r^{(t)}/\sqrt{p}-\sqrt{p}\|_{2}} = \lambda_\infty,
\end{align*}
where the asymptotic linear rate of convergence $\lambda_\infty$ is
\begin{align*}
\lambda_\infty & :=\lambda_{2}(\tilde{A}\tilde{A}^{T})=\lambda_{2}(\tilde{A}^{T}\tilde{A})\\
\tilde{A} & :=\mathcal{D}(1/\sqrt{p})\cdot\hat{A}\cdot\mathcal{D}(1/\sqrt{q}),
\end{align*}
where $\lambda_{2}(\cdot)$ denotes the second largest eigenvalue. 
\end{theorem}

Intuitively, the dependence of the linear rate of convergence on the second largest eigenvalue of $\tilde{A}^T\tilde{A}$ (and $\tilde{A}\tilde{A}^T$) is due to the fact that near the optimum $\sqrt{p}$, $\tilde{A}\tilde{A}^T$ (which is the Jacobian at $\sqrt{p}$) approximates the first order change in $r^{(t)}/\sqrt{p}$. Normally, the \emph{leading} eigenvalue quantifies this change. The unique leading eigenvalue of $\tilde{A}\tilde{A}^T$ is equal to 1 with eigenvector $\sqrt{p}$, which does not imply contraction. Fortunately, using the quantity  $r^{(t)}/\sqrt{p}$
allows us to exploit the following orthogonality structure:
\begin{align*}
(r^{(t)}/\sqrt{p}-\sqrt{p})^{T}\sqrt{p} & =\sum_{i}(r_{i}^{(t)}-p_{i})=0
\end{align*}
by virtue of Sinkhorn's algorithm preserving the quantities $r^{(t)T}\mathbf{1}_{n}$
for all $t$. Thus, the residual $r^{(t)}/\sqrt{p}-\sqrt{p}$ is always \emph{orthogonal} to
$\sqrt{p}$, which is both the leading eigenvector and the fixed point of the iteration. The convergence is then controlled by the \emph{second} largest eigenvalue of $\tilde{A}\tilde{A}^T$. This proof approach echoes that of the global linear convergence result in \cref{thm:global-convergence}, where we also exploit an orthogonality condition to obtain a meaningful bound. In \cref{thm:global-convergence} the bound depends on the second smallest eigenvalue of the Hessian, while in \cref{thm:convergence} the bound depends on the second largest eigenvalue of the Jacobian. 

In the special case of $m=n$ and $p=q=\mathbf{1}$, the asymptotic rate in \cref{thm:convergence} reduces to that in \citet{knight2008sinkhorn}. Note, however, that the convergence metric is different: we use the $\ell^2$ norm $\|r^{(t)}/\sqrt{p}-\sqrt{p}\|_2$ while \citet{knight2008sinkhorn} uses an \emph{implicit} norm that measures the convergence of $D^0$ to the solution \emph{line} due to scale invariance. Our analysis exploits the orthogonality structure of Sinkhorn's algorithm and more explicitly reveals the dependence of the convergence rate on the spectral structure of the data. 

Our results in this section are relevant in several respects. First, we clarify the gap between the $\mathcal O(1/t)$ and $\mathcal O(\lambda^t)$ convergence of Sinkhorn's algorithm: the slowdown happens if and only if Sinkhorn's algorithm converges but the canonical matrix balancing problem does not have a \emph{finite} scaling $(D^0,D^1)$. This slowdown has been observed in the literature but not systematically studied. Second, 
 we settle open problems and establish the first quantitative global linear convergence result for Sinkhorn's algorithm applied to general non-negative matrices. We also characterize the asymptotic linear rate of convergence, generalizing the result of \citet{knight2008sinkhorn} but with a novel analysis. 
  Third, our analysis reveals the importance of algebraic connectivity for the convergence of Sinkhorn's algorithm. Although an important quantity in the choice modeling literature, algebraic connectivity has not been previously used to in the analysis of Sinkhorn's algorithm. The importance of algebraic connectivity for Sinkhorn's algorithm becomes less surprising once we connect it to the distributed optimization literature in \cref{sec:connections}, where it is well-known that the spectral gap of the \emph{gossip} matrix, which defines the decentralized communication network, governs the rates of convergence.
% \vspace{-0.5em}
\section{Conclusion}
% \vspace{-0.5em}
Recent advances in multimodal single-cell technology have enabled the simultaneous profiling of the transcriptome alongside other cellular modalities, leading to an increase in the availability of multimodal single-cell data. In this paper, we present \method{}, a multimodal transformer model for single-cell surface protein abundance from gene expression measurements. We combined the data with prior biological interaction knowledge from the STRING database into a richly connected heterogeneous graph and leveraged the transformer architectures to learn an accurate mapping between gene expression and surface protein abundance. Remarkably, \method{} achieves superior and more stable performance than other baselines on both 2021 and 2022 NeurIPS single-cell datasets.

\noindent\textbf{Future Work.}
% Our work is an extension of the model we implemented in the NeurIPS 2022 competition. 
Our framework of multimodal transformers with the cross-modality heterogeneous graph goes far beyond the specific downstream task of modality prediction, and there are lots of potentials to be further explored. Our graph contains three types of nodes. While the cell embeddings are used for predictions, the remaining protein embeddings and gene embeddings may be further interpreted for other tasks. The similarities between proteins may show data-specific protein-protein relationships, while the attention matrix of the gene transformer may help to identify marker genes of each cell type. Additionally, we may achieve gene interaction prediction using the attention mechanism.
% under adequate regulations. 
% We expect \method{} to be capable of much more than just modality prediction. Note that currently, we fuse information from different transformers with message-passing GNNs. 
To extend more on transformers, a potential next step is implementing cross-attention cross-modalities. Ideally, all three types of nodes, namely genes, proteins, and cells, would be jointly modeled using a large transformer that includes specific regulations for each modality. 

% insight of protein and gene embedding (diff task)

% all in one transformer

% \noindent\textbf{Limitations and future work}
% Despite the noticeable performance improvement by utilizing transformers with the cross-modality heterogeneous graph, there are still bottlenecks in the current settings. To begin with, we noticed that the performance variations of all methods are consistently higher in the ``CITE'' dataset compared to the ``GEX2ADT'' dataset. We hypothesized that the increased variability in ``CITE'' was due to both less number of training samples (43k vs. 66k cells) and a significantly more number of testing samples used (28k vs. 1k cells). One straightforward solution to alleviate the high variation issue is to include more training samples, which is not always possible given the training data availability. Nevertheless, publicly available single-cell datasets have been accumulated over the past decades and are still being collected on an ever-increasing scale. Taking advantage of these large-scale atlases is the key to a more stable and well-performing model, as some of the intra-cell variations could be common across different datasets. For example, reference-based methods are commonly used to identify the cell identity of a single cell, or cell-type compositions of a mixture of cells. (other examples for pretrained, e.g., scbert)


%\noindent\textbf{Future work.}
% Our work is an extension of the model we implemented in the NeurIPS 2022 competition. Now our framework of multimodal transformers with the cross-modality heterogeneous graph goes far beyond the specific downstream task of modality prediction, and there are lots of potentials to be further explored. Our graph contains three types of nodes. while the cell embeddings are used for predictions, the remaining protein embeddings and gene embeddings may be further interpreted for other tasks. The similarities between proteins may show data-specific protein-protein relationships, while the attention matrix of the gene transformer may help to identify marker genes of each cell type. Additionally, we may achieve gene interaction prediction using the attention mechanism under adequate regulations. We expect \method{} to be capable of much more than just modality prediction. Note that currently, we fuse information from different transformers with message-passing GNNs. To extend more on transformers, a potential next step is implementing cross-attention cross-modalities. Ideally, all three types of nodes, namely genes, proteins, and cells, would be jointly modeled using a large transformer that includes specific regulations for each modality. The self-attention within each modality would reconstruct the prior interaction network, while the cross-attention between modalities would be supervised by the data observations. Then, The attention matrix will provide insights into all the internal interactions and cross-relationships. With the linearized transformer, this idea would be both practical and versatile.

% \begin{acks}
% This research is supported by the National Science Foundation (NSF) and Johnson \& Johnson.
% \end{acks}

\bibliography{sinkhorn-choice}
\bibliographystyle{informs2014}

%%%%%%%%%%%%%%%%%%%%%%%%%%%%%%%%%%%%%%%%%%%%%%%%%%%%%%%%%%%%%%%%%%%%%%%%%%%%%%%
%%%%%%%%%%%%%%%%%%%%%%%%%%%%%%%%%%%%%%%%%%%%%%%%%%%%%%%%%%%%%%%%%%%%%%%%%%%%%%%
% APPENDIX
%%%%%%%%%%%%%%%%%%%%%%%%%%%%%%%%%%%%%%%%%%%%%%%%%%%%%%%%%%%%%%%%%%%%%%%%%%%%%%%
%%%%%%%%%%%%%%%%%%%%%%%%%%%%%%%%%%%%%%%%%%%%%%%%%%%%%%%%%%%%%%%%%%%%%%%%%%%%%%%
\newpage
\begin{APPENDICES}
\crefalias{section}{appendix}
\section{Graph Laplacians and Algebraic Connectivity}
\label{subsec:graph-laplacian}
In this section, we introduce the quantities central to our global linear convergence analysis, especially the \emph{graph Laplacian} matrices associated with the graphs defined by a non-negative matrix $A$ and the Fielder eigenvalues.

Given a non-negative matrix $A\in \mathbb{R}_+^{n\times m}$, we define the associated (weighted) bipartite graph $G_b$ on $V\cup U$ by the adjacency matrix $A^b \in \mathbb{R}^{(m+n)\times (m+n)}$ defined as
\begin{align*}
    A^b := \begin{bmatrix}\mathbf{0} & {A}\\
{A}^{T} & \mathbf{0}
\end{bmatrix}.	
\end{align*}
The rows of $A$ correspond to vertices in $V$
with $|V|=n$, while the columns of $A$ correspond to vertices in
$U$ with $|U|=m$, and $V\cap U=\emptyset$. The matrix $A$ here is sometimes called the \emph{biadjacency matrix} of the bipartite graph.

The matrix $A$ also defines
an \emph{undirected} ``comparison'' graph $G_c$ on $m$ items. This is most easily understood when $A$ is binary and we can associate it with the participation matrix of a choice dataset, but the definition below is more general. Define the adjacency matrix $A^c \in \mathbb{R}^{m\times m}$ by
\begin{align*}
{A^c}_{jj'} & =\begin{cases}
0 & j=j'\\
(A^{T}A)_{jj'} & j\neq j',
\end{cases}
\end{align*}
If $A$ is a binary participation matrix associated with a choice dataset, then there is a (weighted) edge in $G_{c}$ between items $j$ and $j'$
if and only the two appear in some choice set together, with the edge
weight equal to the number of times of their co-occurrence. This undirected comparison graph $G_c$ is not the same as the directed comparison graph in \cref{ass:strong-connected}, since it does not encode the \emph{choice} of each observation. However, it is also an important object in choice modeling. For example, the uniqueness condition in \cref{ass:weak-connected} for choice maximum likelihood estimation has a concise graph-theoretic interpretation as it is a requirement that $G_c$ be connected.

For a (generic) undirected graph $G$ with adjacency matrix $M$, the graph Laplacian matrix (or simply the Laplacian) is defined as $L(M):=\mathcal{D}(M\mathbf{1})-M$, where recall $\mathcal{D}$ is the diagonalization of a vector. The graph Laplacian $L(M)$ is always positive semidefinite as a result of the Gershgorin circle theorem, since $L(M)$ is diagonally dominant  with positive diagonal and non-positive off-diagonals. Moreover, the Laplacian always has $\mathbf{1}$ in its null space.

For the graphs $G_b,G_c$, their Laplacians are  given respectively by 
\begin{align}
    \label{eq:Laplacians}
\mathcal{L}:&=L(A^b)=\begin{bmatrix}\mathcal{D}({A}\mathbf{1}_{m}) & -{A}\\
-{A}^{T} & \mathcal{D}({A}^{T}\mathbf{1}_{n})
\end{bmatrix} \\L:&=L(A^c)=A^TA\mathbf{1}_m-A^TA,
\end{align}
where we can
verify that for the comparison graph $G_c$, its Laplacian $L$ satisfies
\begin{align*}
L=A^c\mathbf{1}_{m}-A^c & =A^{T}A\mathbf{1}_{m}-A^{T}A.
\end{align*}
The graph Laplacian $\mathcal{L}$ based on $A^b$ and $L$ based on $A^c$ are closely connected through the identity
\begin{align*} (A^{b})^{2}&=\begin{bmatrix}AA^{T} & 0\\
0 & A^{T}A
\end{bmatrix},
\end{align*}
which implies that $L$ is the lower right block of the graph Laplacian $\mathcal{D}((A^{b})^{2}\mathbf{1}_{m+n})-(A^{b})^{2}$. Moreover, $L$ plays a central role in works on the statistical and computational efficiency in choice modeling \citep{shah2015estimation,seshadri2020learning,vojnovic2020convergence}.

An important concept in spectral graph theory is the \emph{algebraic connectivity} of a graph, quantified by the second smallest eigenvalue $\lambda_{-2}$ of the graph Laplacian matrix, also called the Fiedler eigenvalue \citep{fiedler1973algebraic,spielman2007spectral}.
% The Fiedler eigenvalue $\lambda_{-2}$ features prominently in Cheeger's inequality,
% \begin{align}
% \label{eq:Cheeger}
%   \frac{h^2(G)}{2\max_j(M\mathbf{1})_j}   \leq \lambda_{-2}(\mathcal{D}(M\mathbf{1})-M) \leq  2h(G),
% \end{align}
% where $h(G)\geq0$ is the Cheeger constant, positive iff $G$ is connected.
Intuitively, Fiedler eigenvalue quantifies how well-connected a graph is in terms of how many edges need to be removed for the graph to become disconnected. It is well-known that the multiplicity of the smallest eigenvalue of the graph Laplacian, which is 0, describes the number of connected components of a graph. The uniqueness condition for matrix balancing in \cref{ass:matrix-uniqueness} therefore guarantees that the Fiedler eigenvalue of $G_b$ is positive: $\lambda_{-2}(\mathcal{L}) >0$. This property is important for our results, since $\lambda_{-2}(\mathcal{L})$ quantifies the \emph{strong} convexity of the potential function and hence the linear convergence rate of Sinkhorn's algorithm.

\section{Further Connections to Choice Modeling and Optimization}
\label{sec:connections}
In this section, we demonstrate that our matrix balancing formulation \eqref{eq:equation-system} of the maximum likelihood problem \eqref{eq:log-likelihood} provides a unifying perspective on many existing works on choice modeling, and establishes interesting connections to distributed optimization as well. Throughout, Sinkhorn's algorithm will serve as the connecting thread. In particular, it reduces algebraically to the algorithms in \citet{zermelo1929berechnung,dykstra1956note,ford1957solution,hunter2004mm,maystre2017choicerank} in their respective choice model settings. This motivates us to provide an interpretation of Sinkhorn's algorithm as a ``minorization-maximization'' (MM) algorithm \citep{lange2000optimization}. Moreover, Sinkhorn's Algorithm is also related to the ASR algorithm of \citet{agarwal2018accelerated} for choice modeling, as they can both be viewed as message passing algorithms in distributed optimization \citep{balakrishnan2004polynomial}. Last but not least, we establish a connection between Sinkhorn's algorithm and the well-known BLP algorithm of \citet{berry1995automobile}, widely used in economics to estimate consumer preferences from data on market shares. 

\subsection{Pairwise Comparisons}
The same algorithmic idea in many works on pairwise comparisons appeared as early as \citet{zermelo1929berechnung}. For example,
\citet{dykstra1956note} gives the following update formula: 
\begin{align}
\label{eq:pairwise}
s_{j}^{(t+1)} & =W_{j}/\sum_{j\neq k}\frac{C_{jk}}{s_{j}^{(t)}+s_{k}^{(t)}},
% s_{j}^{(t+1)} & =W_{j}\cdot\left[\sum_{j\neq k}\frac{C_{jk}}{s_{j}^{(t)}+s_{k}^{(t)}}\right]^{-1},
\end{align}
 where again $W_{j}=|\{i\mid (j,S_i)\}|$ is the number of times
item $j$ is chosen (or ``wins''), and $C_{jk}$ is the number of comparisons between $j$
and $k$. \cref{ass:strong-connected} guarantees $C_{jk}>0$ for any $j,k$.  \citet{zermelo1929berechnung}
proved that under this assumption
$s^{(1)},s^{(2)},\dots$ converge to the unique maximum likelihood estimator,
and the sequence of log-likelihoods $\ell(s^{(1)}),\ell(s^{(2)}),\dots$
is monotone increasing. A cyclic version of \eqref{eq:pairwise} appeared in \citet{ford1957solution} with an independent proof of convergence. One can verify that by aggregating choice sets $S_i$ in \eqref{eq:scaling-iteration} over pairs of objects, it reduces to \eqref{eq:pairwise}. However, \eqref{eq:pairwise} as is written does not admit a matrix balancing formulation. A generalization of the algorithm of \citet{zermelo1929berechnung,ford1957solution,dykstra1956note} for pairwise comparison to ranking data was not achieved until the influential works of \citet{lange2000optimization} and \citet{hunter2004mm}. 
\subsection{MM Algorithm of \citet{hunter2004mm} for Ranking
Data} 
Motivated by the observation in \citet{lange2000optimization} that \eqref{eq:pairwise} is an instance of an minorization-maximization (MM) algorithm, the seminal work of \citet{hunter2004mm} proposed the general approach of solving ML estimation of choice models via MM algorithms, which relies on the inequality 
\begin{align*}
-\log x & \geq1-\log y-(x/y)
\end{align*}
to construct a lower bound (minorization) on the log-likelihood that has an explicit maximizer (maximization), and iterates between the two steps. \citet{hunter2004mm} develops such an algorithm for the Plackett--Luce
model for ranking data and proves its monotonicity and convergence.

Given $n$ partial rankings, where the $i$-th partial ranking on $l_{i}$ objects is indexed by $a(i,1)\rightarrow a(i,2)\rightarrow \cdots \rightarrow a(i,l_{i})$, the MM algorithm of \citet{hunter2004mm} takes the form 
\begin{align}
\label{eq:mm}
s_{k}^{(t+1)} & =\frac{w_{k}}{\sum_{i=1}^{n}\sum_{j=1}^{l_{i}-1}\delta_{ijk}[\sum_{j'=j}^{l_{i}}s_{a(i,j')}^{(t)}]^{-1}},
\end{align}
 where $\delta_{ijk}$ is the indicator that item $k$ ranks no better
than the $j$-th ranked item in the $i$-th ranking, and $w_{k}$ is the number of
rankings in which $k$ appears but is not ranked last. 
\begin{proposition}
\label{lem:mm}
Sinkhorn's algorithm applied to the ML estimation of the Plackett--Luce model is algebraically equivalent to \eqref{eq:mm}. 
\end{proposition}
Therefore, Sinkhorn's algorithm applied to the ML estimation of the Plackett--Luce model reduces algebraically to the MM algorithm of \citet{hunter2004mm}. However, \cref{alg:scaling} applies to more general choice models with minimal or no change, while the approach in \citet{hunter2004mm} requires deriving the minorization-majorization step for every new optimization objective. This was carried out, for example, in \citet{maystre2017choicerank} for a network choice model. We show in \cref{prop:choicerank} that their ChoiceRank algorithm is also a special case of (regularized) Sinkhorn's algorithm. From a computational perspective, even when algorithms are equivalent algebraically, their empirical performance can vary drastically depending
on the particular implementation. Another advantage of Sinkhorn's algorithm is that it computes \emph{all}
entries simultaneously through vector and matrix operations, while
the analytical formula in \eqref{eq:mm} is hard to parallelize. This distinction is likely behind
the discrepancy in \cref{sec:empirics} between our experiments and those in \citet{maystre2015fast},
who conclude that the MM version \eqref{eq:mm} is slower in terms of wall clock time than their Iterative Luce Spectral Ranking (I-LSR) algorithm for the Plackett--Luce model on $k$-way partial ranking data. 

\subsection{Markov Steady State Inversion and Network Choice}
\label{subsec:steady-state}
Our work is related to the works of \citet{kumar2015inverting,maystre2017choicerank} on Markov chains on graphs, where transition matrices are parameterized by node-dependent scores prescribed by Luce's choice axiom. More precisely, given a directed graph $G=(V,E)$ and $N^\out_j, N^\inn_j \subseteq V$ the neighbors with edges going out from and into $j\in V$, and a target stationary distribution $\pi$, the (unweighted) steady state inversion problem of \citet{kumar2015inverting} seeks scores $s_j$ such that the transition matrix $T_{j,k}=\frac{s_k}{\sum_{k'\in N^\out_j}s_{k'}}$ has the desired stationary distribution $\pi$. Their Theorem 13 shows that a bipartite version of this problem is equivalent to solving the ML estimation conditions \eqref{eq:optimality} of the choice model. Furthermore, one can verify that their bipartite inversion problem has the same form as \eqref{eq:bridge} in our paper, with the bipartite graph defined using $A$. Their existence condition (termed ``consistency'') is equivalent to \cref{ass:matrix-weak-existence}(a) \citep{menon1968matrix} for the matrix balancing problem. Despite these connections, the key difference in our work is the reformulation of \eqref{eq:optimality} as one involving diagonal scalings of rows and columns of $A$, which was absent in \citet{kumar2015inverting}. Consequently, they proposed a different algorithm instead of applying Sinkhorn's algorithm. 

Building on \citet{kumar2015inverting}, \citet{maystre2017choicerank} consider a similar Markov chain on $(V,E)$, where now for each edge $(j,k) \in E$ one observes a finite \emph{number} $c_{jk}$ of transitions along it, and take a maximum likelihood approach to estimate the scores $s_j$.  They show that, as one might expect, the steady state inversion problem of \citet{kumar2015inverting} is the asymptotic version of the ML estimation problem in their network choice model. 

An additional contribution of \citet{maystre2017choicerank} is the regularization of the inference problem via a Gamma prior on $s_j$'s, which eliminates the necessity of any assumptions on the choice dataset such as \cref{ass:strong-connected}. They then follow the proposal of \citet{hunter2004mm} and develop an MM algorithm for maximum likelihood estimation called ChoiceRank, the unregularized version of which can be written as follows:
\begin{align}
    \label{eq:choicrank}
    s_j^{(t+1)} = \frac{c_j^\inn}{\sum_{k\in N_j^\inn} \gamma_k^{(t)}}, \gamma_j^{(t)}=\frac{c_j^\out}{\sum_{k\in N_j^\out}s_k^{(t)}},
\end{align}
where $c_j^\inn=\sum_{k\in N_j^\inn}c_{kj}$ and $c_j^\out=\sum_{k\in N_j^\out}c_{jk}$ are the total number of observed transitions into and out of $j\in V$. 
\begin{proposition}
    \label{prop:choicerank}
   The network choice model of \citet{maystre2017choicerank} is a special case of the choice model \eqref{eq:model}, and  Sinkhorn's algorithm applied to this case reduces to an iteration algebraically equivalent to \eqref{eq:choicrank}.
\end{proposition}
We also explore the regularization approach of \citet{maystre2017choicerank} in \cref{subsec:regularization} and demonstrate that Sinkhorn's algorithm can easily accommodate this extension, resulting in a regularized version of Sinkhorn's algorithm for matrix balancing that \emph{always} converges. This is given in \cref{alg:regularized}. Once again, insights from choice modeling yield useful improvements in matrix balancing.

\subsection{Sinkhorn's Algorithm as an MM Algorithm}
\label{sec:sinkhorn-MM}
That Sinkhorn's algorithm reduces to MM algorithms when applied to various choice models is not a coincidence. 
In this section, we establish the connection between choice modeling
and matrix balancing through an optimization perspective. This connection
provides an interesting interpretation of Sinkhorn's algorithm
as optimizing a dominating function, i.e., an MM algorithm. See \citet{lange2000optimization,lange2016mm} for a discussion of the general correspondence between block
coordinate descent algorithms and MM algorithms.

First, we discuss the connection between the log-likelihood function
\eqref{eq:log-likelihood} and the dual potential function \eqref{eq:log-barrier} when $(A,p,q)$ corresponds to a choice dataset. Consider maximizing the negative
dual potential function 
\begin{align*}
h(d^{0},d^{1}):=-g(d^{0},d^{1}) & =\sum_{j=1}^{m}q_{j}\log d_{j}^{0}+\sum_{i=1}^{n}p_{i}\log d_{i}^{1}-(d^{1})^{T}Ad^{0}.
\end{align*}
For each fixed $d^{0}$, the function is concave in $d^{1}$, and
maximization with respect to $d^{1}$ yields first order conditions
\begin{align*}
d^{1} & =p/(Ad^{0}).
\end{align*}
Substuting this back into $h(d^{0},d^{1})$, we obtain 
\begin{align*}
f(d^{0}):=h(d^{0},p/(Ad^{0})) & =\sum_{j=1}^{m}q_{j}\log d_{j}^{0}+\sum_{i=1}^{n}p_{i}\log(\frac{p_{i}}{(Ad^{0})_{i}})-\sum_{i}p_{i}\\
 & =\sum_{j=1}^{m}q_{j}\log d_{j}^{0}-\sum_{i=1}^{n}p_{i}\log(Ad^{0})_{i}+\sum_{i}p_{i}\log p_{i}-\sum_{i}p_{i}.
\end{align*}
If $A$ is a valid participation matrix for a choice dataset and $p,q$
are integers, we can identify $(A,p,q)$ with a choice dataset. Each
row of the participation matrix $A$ is the indicator vector of choice
set $S_{i}$, and $d_{j}^{0}$ is the quality score. In this
case $(Ad^{0})_{i}=\sum_{k\in S_{i}}d_{k}^{0}$, so that 
\begin{align*}
\sum_{j=1}^{m}q_{j}\log d_{j}^{0}-\sum_{i=1}^{n}p_{i}\log(Ad^{0})_{i} & =\sum_{j=1}^{m}q_{j}\log d_{j}^{0}-\sum_{i=1}^{n}p_{i}\sum_{k\in S_{i}}d_{k}^{0}=\ell(d^{0}).
\end{align*}

It then follows that
\begin{align*}
\min_{d^{0},d^{1}}g(d^{0},d^{1})\Leftrightarrow\max_{d^{0},d^{1}}h(d^{0},d^{1})\Leftrightarrow\max_{d^{0}}\max_{d^{1}}h(d^{0},d^{1})\Leftrightarrow\max_{d^{0}}f(d^{0})\Leftrightarrow\max_{d^{0}}\ell(d^{0}),
\end{align*}
so that minimizing the potential function $g$ is equivalent to maximizing
the log-likelihood function $\ell$. Moreover, the first order condition of maximizing $h$ with respect to $d^1$ is 
$d^0=q/(A^Td^1)$, which when $(A,p,q)$ is identified with a choice dataset reduces to 
\begin{align*}
    q_j =\sum_{i\mid j\in S_i}p_i \frac{d^0_j}{\sum_{k\in S_i}d^0_k},
\end{align*}
which is the optimality condition \eqref{eq:optimality} of the choice model.

Next, given $d^{0(t)}$ the estimate of $d^{0}$ after the
$t$-th iteration, define the function 
\begin{align*}
f(d^{0}\mid d^{0(t)}):=h(d^{0},p/(Ad^{0(t)})) & =\sum_{j=1}^{m}q_{j}\log d_{j}^{0}+\sum_{i=1}^{n}p_{i}\log\frac{p}{Ad^{0(t)}}-\frac{p_{i}}{(Ad^{0(t)})_{i}}(Ad^{0})_{i}.
\end{align*}
We can verify that 
\begin{align*}
f(d^{0(t)}\mid d^{0(t)}) & =f(d^{0(t)})\\
f(d^{0}\mid d^{0(t)}) & \leq f(d^{0}),
\end{align*}
 so that $f(d^{0}\mid d^{0(t)})$ is a valid minorizing function of
$f(d^{0})$ \citep{lange2000optimization} that guarantees the ascent property $f(d^{0(t+1)})\geq f(d^{0(t+1)}\mid d^{0(t)})=\max_{d^{0}}f(d^{0}\mid d^{0(t)})\geq f(d^{0(t)}\mid d^{0(t)})=f(d^{0(t)})$.
The update in the maximization step
\begin{align*}
d^{0(t+1)}=\arg\max_{d^{0}}f(d^{0}\mid d^{0(t)}) & =q/A^{T}\frac{p}{Ad^{0(t)}}
\end{align*}
 is precisely one full iteration of Sinkhorn's algorithm. Note that this interpretation of Sinkhorn's algorithm does not require $A$ to be binary, and $p,q$ to be integers.

On the other hand, using the property $-\ln x\geq1-\ln y-(x/y)$,
we can \emph{directly} construct a minorizing function of $\ell$
by 
\begin{align*}
\ell(d^{0})=\sum_{j=1}^{m}q_{j}\log d_{j}^{0}-\sum_{i=1}^{n}p_{i}\log(Ad^{0})_{i} & \geq\sum_{j=1}^{m}q_{j}\log d_{j}^{0}+\sum_{i=1}^{n}p_{i}(-\frac{(Ad^{0})_{i}}{(Ad^{0(t)})_{i}}-\log(Ad^{0(t)})_{i}+1)=\ell(d^{0}\mid d^{0(t)}),
\end{align*}
 where $\ell(d^{0}\mid d^{0(t)})$ is a valid minorizing function
of $\ell$. Maximizing $\ell(d^{0}\mid d^{0(t)})$ with respect to
$d^{0}$, the update in the maximization step is
\begin{align*}
d_{j}^{0(t+1)} & =q_{j}/\sum_{i\mid j\in S_i}\frac{p_{i}}{(Ad^{0(t)})_{i}}
\end{align*}
 which again is one full iteration of Sinkhorn's algorithm applied to the Luce choice model. Moreover,
\begin{align*}
\ell(d^{0}\mid d^{0(t)})+\sum_{i}p_{i}\log p_{i}-\sum_{i}p_{i} & =\sum_{j=1}^{m}q_{j}\log d_{j}^{0}+\sum_{i=1}^{n}p_{i}(-\frac{(Ad^{0})_{i}}{(Ad^{0(t)})_{i}}-\log(Ad^{0(t)})_{i}+1)+\sum_{i}p_{i}\log p_{i}-\sum_{i}p_{i}\\
 & =\sum_{j=1}^{m}q_{j}\log d_{j}^{0}+\sum_{i=1}^{n}p_{i}(-\frac{(Ad^{0})_{i}}{(Ad^{0(t)})_{i}}+\log\frac{p_{i}}{(Ad^{0(t)})_{i}})\\
 & =f(d^{0}\mid d^{0(t)}).
\end{align*}
 Therefore, the minorizing function $\ell(d^{0}\mid d^{0(t)})$ constructed
using $-\ln x\geq1-\ln y-(x/y)$ for the log-likelihood and the minorizing
function $f(d^{0}\mid d^{0(t)})$ constructed for $\max_{d^{1}}h(d^{0},d^{1})$
are identical modulo a constant $\sum_{i}p_{i}\log p_{i}-\sum_{i}p_{i}$.
Sinkhorn's algorithm is in fact the MM algorithm corresponding to both minorizations. However, the perspective using $f(d^{0}\mid d^{0(t)})$ is more general since it applies to general $(A,p,q)$ as long as they satisfy Assumptions \cref{ass:matrix-existence} and \cref{ass:matrix-uniqueness}, whereas the MM algorithm based on $\ell(d^{0}\mid d^{0(t)})$ is designed for choice dataset, so requires $A$ to be binary.

\subsection{ Sinkhorn's Algorithm and Distributed Optimization} 
We now shift our focus to algorithms in distributed optimization, where Sinkhorn's algorithm can be interpreted as a message passing/belief propagation algorithm \citep{balakrishnan2004polynomial}. We start by observing a connection to the ASR algorithm for estimating Luce choice models \citep{agarwal2018accelerated}, which returns the same approximate ML estimators as the RC \citep{negahban2012iterative} and LSR \citep{maystre2015fast} algorithms, but has provably faster convergence. 

Consider the bipartite graph $G_b$ defined by $A$ in \cref{subsec:graph-laplacian}, which consists of choice set nodes $V$ on one hand and item nodes $U$ on the other, where there is an edge between $i\in V$ and $j\in U$ if and only if $j\in S_i$. \citet{agarwal2018accelerated} provide the following message passing interpretation of ASR on the bipartite graph: at every iteration, the item nodes send a ``message'' to their
neighboring choice set nodes consisting of each item node's current estimate of their own $s_j$; the choice set nodes then aggregate the messages they receive by summing up these estimates, and then sending back the sums to their neighboring item nodes. The item nodes use these sums to update estimates of their own $s_j$. \citet{agarwal2018accelerated} show that since the ASR algorithm is an instance of the message passing algorithm, it can be implemented in a distributed manner.% similar to the parallelization capability of Sinkhorn's algorithm. 

We now explain how Sinkhorn's algorithm is another instance of the message passing algorithm described above. Recall that $d^0$ is identified with the $s_j$'s in the Luce choice model, so that $d^{1} \leftarrow p/(Ad^{0})$ precisely corresponds to item nodes ``passing'' their current estimates to set nodes, which then sum up the received estimates and then take the weighted \emph{inverse} of this sum. Similarly, $d^0 \leftarrow q/(A^{T} d^{1})$ corresponds to choice set nodes passing their current estimates of $d^1$ back to item nodes, which then sum up the received messages and take the weighted inverse as their updated estimates of $s_j$. The main difference with ASR lies in how each item node $j$ updates its estimate of $s_j$ based on the messages it receives from neighboring set nodes. In Sinkhorn's algorithm, the update to $s_j$ is achieved by dividing $p$ by a weighted average of the \emph{inverse} of summed messages $1/\sum_{k\in S_{i}}s_{k}^{(t)}$:
\begin{align*}
  s_j^{(t+1)} \leftarrow  q_j/(A^{T} d^{1})_j=W_j/\sum_{i\mid j\in S_{i}}\frac{R_i}{\sum_{k\in S_{i}}s_{k}^{(t)}},
\end{align*}
whereas in ASR, the update is an average of the summed messages $\sum_{k\in S_{i}}s_{k}^{(t)}$ without taking their inverses first: 
\begin{align*}
    s^{(t+1)}_j \leftarrow  \frac{1}{\sum_{i\mid j\in S_{i}} R_i} \sum_{i\mid j \in S_i}  W_{ji} \sum_{k\in S_{i}}s_{k}^{(t)},
\end{align*}
where $W_{ji}$ is the number of times item $j$ is selected from all observations having choice set $S_i$, with $\sum_i W_{ji} = W_j$.

From another perspective, the two algorithms arise from different \emph{moment} conditions. While Sinkhorn's algorithm is based on the optimality condition \eqref{eq:optimality}, ASR is based on the condition
\begin{align*}
    \sum_{i\mid j\in S_{i}} R_i = \sum_{i\mid j \in S_i}  W_{ji}/\frac{s_j}{\sum_{k\in S_{i}}s_{k}},
\end{align*}
which results in an approximate instead of exact MLE. 

The message passing interpretation also provides further insights on the importance of algebraic connectivity to the convergence rate of Sinkhorn's algorithm. Graph theoretic conditions like \cref{ass:strong-connected} are related to network flow and belief propagation, and characterize how fast information can be distributed across the bipartite network with the target distributions $p,q$. It is well-known that convergence of distributed algorithms on networks depends critically on the network topology through the spectral gap of the associated averaging matrix. We can understand \cref{thm:convergence} on the asymptotic convergence rate of Sinkhorn's algorithm as a result of this flavor, although a precise equivalence is left for future works. 

\subsection{The Berry--Levinsohn--Pakes Algorithm}
Last but not least, our work is also closely related to the economics literature that studies consumer behavior based on discrete choices \citep{mcfadden1973conditional,mcfadden1978modelling,mcfadden1981econometric,berry1995automobile}. Here we discuss the particular connection with the work of Berry,
Levinsohn, and Pakes \cite{berry1995automobile}, often referred to as BLP. To estimate consumer preferences over automobiles across different markets (e.g., geographical), they propose a random utility (RUM) model indexed by individual $i$, product
$j$, and market $t$:
\begin{align*}
U_{ijt} & =\beta_{i}^{T}X_{jt}+\theta_{jt}+\epsilon_{ijt},
\end{align*}
 where $\theta_{jt}$ is an unobserved product characteristic, such as the overall popularity of certain types of cars in different regions, and $\epsilon_{ijt}$
are \emph{i.i.d.} double exponential random variables.
% \begin{align*}
% f(\epsilon) & =\exp(-\epsilon-\exp(-\epsilon))
% \end{align*}
The individual-specific coefficient $\beta_{i}$ is random with 
\begin{align*}
\beta_{i} & =Z_{i}^T\Gamma+\eta_{i}\\
\eta_{i}\mid Z_{i} & \sim\mathcal{N}(\beta,\Sigma),
\end{align*}
and the observations consist of \emph{market shares} $\hat{p}_{jt}$ of each
product $j$ in market $t$ and observable population characteristics $Z_{i}$ in each
market. Given a model with fixed $\beta,\Gamma,\Sigma$ and observations, the task is to estimate $\theta_{jt}$. 

For every value of $\theta_{jt}$, we can compute, or simulate if necessary, the \emph{expected} market shares $p_{jt}$, which is the likelihood of product $j$ being chosen in market $t$. For example, in the special case that $\beta,\Gamma,\Sigma \equiv 0$ and $\exp(\theta_{jt})\equiv s_{j}$ for all
$j,t$, i.e., (perceived) product characteristic does not vary across
markets, the expected market share reduces to the familiar formula
\begin{align*}
p_{jt} & =\frac{\exp(\theta_{jt})}{\sum_{k}\exp(\theta_{kt})}=\frac{s_{j}}{\sum_{k}s_{k}}.
\end{align*}
The generalized method of moments (GMM) approach of \citet{berry1995automobile} is to find $\theta_{jt}$ such that $p_{jt}=\hat{p}_{jt}$, i.e., the implied expected market share equals the observed share. Recall the similarity to the optimality condition \eqref{eq:optimality} of the Luce choice model. BLP propose the iteration 
\begin{align}
\label{eq:blp}
\theta_{jt}^{(m+1)} & =\theta_{jt}^{(m)}+\log\hat{p}_{jt}-\log p_{jt}(\theta^{(m)},\beta,\Gamma,\Sigma),
\end{align}
 and show that it is a \emph{contraction mapping}, whose fixed point is the desired estimates of $\theta_{jt}$. 
\begin{proposition}
    \label{prop:BLP}
    When $\beta,\Gamma,\Sigma \equiv 0$ and $\exp(\theta_{jt})\equiv s_{j}$, the GMM condition of BLP on market shares is equivalent to the optimality condition \eqref{eq:optimality} for a Luce choice model where all alternatives are available in every observation. Furthermore, the BLP algorithm is equivalent to Sinkhorn's algorithm in this model.
\end{proposition}
For a more detailed correspondence, see \citet{bonnet2022yogurts}. Importantly, \cref{prop:BLP} does not imply that the Luce choice model and Sinkhorn's algorithm is a strict special case of the BLP framework. The key difference is that BLP, and most discrete choice models in econometrics, implicitly assumes that the entire set of alternatives is always available in each observation. This assumption translates to a participation matrix $A$ in \eqref{eq:equation-system} that has 1's in all entries. In this setting, the MLE of $s_j$ is simply the empirical winning frequencies. %Note that \cref{ass:strong-connected} may still be violated in this case, highlighting again the gap between what is required for the ML estimation problem and by the matrix scaling problem.
On the other hand, while the Luce choice model allows different choice sets $S_i$ across observations, they do not include covariate information on the alternatives or decision makers, which is important in discrete choice modeling. One can reconcile this difference by relabeling alternatives with different covariates as \emph{distinct} items, and we leave investigations on further connections in this direction to future works.
\section{Regularization of Luce Choice Models and Matrix Balancing Problems}
\label{subsec:regularization}
In practice, many choice and ranking datasets may not satisfy \cref{ass:strong-connected}, which is required for the maximum likelihood estimation to be well-posed. Equivalently, for the matrix balancing problem, when a triplet $(A,p,q)$ does not satisfy \cref{ass:matrix-existence} and \cref{ass:matrix-uniqueness}, no finite scalings exist and Sinkhorn's algorithm may diverge. In this section, we discuss some regularization techniques to address these problems. They are easy to implement and require minimal modifications to Sinkhorn's algorithm. Nevertheless, they can be very useful in practice to regularize ill-posed problems. Given the equivalence between the problem of computing the MLE of Luce choice models and the problem of matrix balancing, our proposed regularization methods apply to both. 
\subsection{Regularization via Gamma Prior}
 As discussed in \cref{sec:formulations},  for a choice dataset to have a well-defined maximum likelihood estimator, it needs to satisfy \cref{ass:strong-connected}, which requires the directed comparison graph to be strongly connected. Although this condition is easy to verify, the question remains what one can do in case it does \emph{not} hold. As one possibility, we may introduce a prior on the parameters $s_j$, which serves as a regularization of the log-likelihood that results in a unique maximizer. Many priors are possible. For example, \citet{maystre2017choicerank}, following \citet{caron2012efficient}, use independent Gamma priors on $s_j$. In view of the fact that the unregularized problem and algorithm in \citet{maystre2017choicerank} is a special case of the Luce choice model and Sinkhorn's algorithm, we can also incorporate the Gamma prior to the Luce choice model \eqref{eq:log-likelihood} to address identification problems. 

More precisely, suppose now that each $s_j$ in the Luce choice model are i.i.d. Gamma$(\alpha,\beta)\propto s_j^{\alpha-1}e^{-\beta s_j}$. This leads to the following regularized log-likelihood:
\begin{align}
\label{eq:regularized-log-likelihood}
\ell^R(s):=\sum_{i=1}^{n}\log s_{j_i}-\log\sum_{k\in S_{i}}s_{k} +(\alpha-1)\sum_{j=1}^{m}\log s_j - \beta \sum_{j=1}^{m} s_j.
\end{align}
The corresponding first order condition is given by
\begin{align}
\label{eq:optimality-regularized}
\frac{W_j+\alpha-1}{n} & = \frac{1}{n} \left(\sum_{i\mid j\in S_{i}}R_i \frac{s_{j}}{\sum_{k\in S_{i}}s_{k}} +\beta s_j\right),
\end{align}
which leads to the following modified Sinkhorn's algorithm, which generalizes the ChoiceRank algorithm of \citet{maystre2017choicerank}: 
\begin{align}
\label{eq:regularized-sinkhorn}
 d^0 \leftarrow (q+\alpha -1)/ (A^Td^1+\beta),  \quad d^1 \leftarrow p/A d^0.
\end{align}
 The choice of $\beta$ determines the normalization of $s_j$. With $u_j=\log s_j$, we can show in a similar way as Theorem 2 of \citet{maystre2017choicerank} that \eqref{eq:regularized-log-likelihood} always has a unique maximizer whenever $\alpha>1$. Regarding the convergence, \citet{maystre2017choicerank} remarked that since their ChoiceRank algorithm can be viewed as an MM algorithm, it inherits the local linear convergence of MM algorithms \citep{lange2000optimization}, but ``a detailed investigation of convergence behavior is left for future works''. With the insights we develop in this paper, we can in fact provide an explanation for the validity of the Gamma priors from an optimization perspective. This perspective allows us to conclude directly that \eqref{eq:regularized-log-likelihood} always has a unique solution in the interior of the probability simplex, and that furthermore the iteration in \eqref{eq:regularized-sinkhorn} has global linear convergence. 
 Consider now the following regularized potential function
 \begin{align}
     g^R(d^0,d^1)	:= ((d^1)^{T}A+\beta (\mathbf{1}_m)^T)d^0-\sum_{i=1}^{n}p_{i}\log d^1_{i}-\sum_{j=1}^{m}(q_{j}+\alpha-1)\log d^0_{j}.
\end{align}
We can verify that by substituting the optimality condition of $d^1$ into $-g^R$, it reduces to the log posterior \eqref{eq:regularized-log-likelihood}. Moreover, the iteration \eqref{eq:regularized-sinkhorn} is precisely the alternating minimization algorithm for $g^R$. When $\alpha-1,\beta>0$, the reparameterized potential function
\begin{align*}
    \sum_{ij}(e^{-v_{i}}A_{ij}e^{u_{j}})+\beta\sum_{j}e^{u_{j}}+\sum_{i}p_{i}v_{i}-\sum_{j}(q_{j}+\alpha-1)u_{j}
\end{align*} 
is always coercive regardless of whether \cref{ass:matrix-existence} holds. Therefore, during the iterations \eqref{eq:regularized-sinkhorn}, $(u,v)$ stay \emph{bounded}. Moreover, the Hessian is 
\begin{align*} \begin{bmatrix}\sum_{j}e^{-v_{i}}A_{ij}e^{u_{j}} & -e^{-v_{i}}A_{ij}e^{u_{j}}\\
-e^{-v_{i}}A_{ij}e^{u_{j}} & \sum_{i}e^{-v_{i}}A_{ij}e^{u_{j}}+\beta e^{u_{j}}
\end{bmatrix}	\succ0,
\end{align*}
 which is now positive definite.  As a result, $g^{R}(u,v)$ is strongly convex and smooth, so that  \eqref{eq:regularized-sinkhorn} converges linearly. From the perspective of the matrix balancing problem, we have thus obtained a regularized version of Sinkhorn's algorithm, summarized in \cref{alg:regularized}, which is guaranteed to converge linearly to a finite solution $(D^1,D^0)$, even when \cref{ass:matrix-existence} does not hold for the triplet $(A,p,q)$. Moreover, the regularization also improves the convergence of Sinkhorn's algorithm even when it converges, as the regularized Hessian becomes more-behaved. This regularized algorithm could be very useful in practice to deal with real datasets that result in slow, divergent, or oscillating Sinkhorn iterations.
 \begin{algorithm}[tb]
\caption{Regularized Sinkhorn's Algorithm}
   \label{alg:regularized}
\begin{algorithmic}
   \STATE {\bfseries Input:}  $A, p, q,\alpha>1,\beta>0,\epsilon_{\text{tol}}$.
   \STATE {\bfseries initialize} $d^{0}\in\mathbb{R}_{++}^{m}$
   \REPEAT
   \STATE $d^{1} \leftarrow  p/( A d^0)$ 

   \STATE $d^{0}\leftarrow  (q+\alpha-1)/({A}^{T} d^{1}+\beta)$

   \STATE 
$\epsilon\leftarrow$  update of $(d^{0},d^1)$
\UNTIL{$\epsilon<\epsilon_{\text{tol}}$}
\end{algorithmic}
\end{algorithm}

\subsection{Regularization via Data Augmentation}
The connection between Bayesian methods and \emph{data augmentation} motivates us to also consider direct data augmentation methods. This is best illustrated in the choice modeling setting. Suppose for a choice dataset we construct participation matrix $A$, $p$ the counts of distinct choice sets, and $q$ the counts of each item being selected. We know that $(A,p,q)$ has a finite scaling solution if and only if  \cref{ass:strong-connected} holds, i.e., the directed comparison graph is strongly connected. We now propose the following modification of $(A,p,q)$ such that the resulting problem is always valid. 

First, if $A$ does not already contain a row equal to $\mathbf{1}_m^T$, i.e., containing all 1's, add this additional row to $A$. Call the resulting matrix $A'$. Then, expanding the dimension of $p$ if necessary, add $m\epsilon$ to the entry corresponding to $\mathbf{1}_m$, where we can assume for now that $\epsilon\geq 1$ is an integer.
This procedure effectively adds $m \epsilon$ ``observations'' that contain all $m$ items. For these additional observations, we let each item be selected exactly $\epsilon$ times. Luce's choice axiom guarantees that the exact choice of each artificial observation is irrelevant, and we just need to add $\epsilon \mathbf{1}_m$ to $q$. This represents augmenting each item with an additional $\epsilon$ ``wins'', resulting in the triplet $(A',p+(m\epsilon)\mathbf{e},q+\epsilon \mathbf{1}_m)$, where $\mathbf{e}$ is the one-hot indicator of the row $\mathbf{1}_m^T$ in $A'$. Now by construction, in any partition of $[m]$ into two non-empty subsets, any item from one subset is selected at least $\epsilon$ times over any item from the other subset. Therefore, \cref{ass:strong-connected} holds, and the maximum likelihood estimation problem, and equivalently the matrix balancing problem with $(A',p+(m\epsilon)\mathbf{e},q+\epsilon \mathbf{1}_m)$, is well-defined. This regularization method applies more generally to any non-negative $A$, even if it is not a participation matrix, i.e., binary. Although in the above construction based on choice dataset, $\epsilon$ is taken to be an integer, for the regularized matrix balancing problem with $(A',p+(m\epsilon)\mathbf{e},q+\epsilon \mathbf{1}_m)$, we can let $\epsilon \rightarrow 0$. %An interesting question is whether the sequence of solutions to the matrix balancing problems indexed by $\epsilon$ converges, and what point in the probability simplex it converges to, when the original triplet $(A,p,q)$ does not define a valid matrix balancing problem. 
%\zq{Future directions. Optimization: Can design regularization procedures that improves the algebraic connectivity for fixed data. Sampling: Can design sampling procedures that improve the algebraic connectivity for streaming data.}
\section{Applications of Matrix Balancing}
\label{app:related-works}
This section contains a brief survey on the applications of matrix balancing in a diverse range of disciplines.

\textbf{Traffic and Transportation Networks.} These applications are some of the earliest and most popular uses of matrix balancing. \citet{kruithof1937telefoonverkeersrekening} considered the problem of estimating new telephone traffic patterns among telephone exchanges given existing traffic volumes and marginal densities of departing and terminating traffic for each exchange when their subscribers are updated. A closely related problem in transportation networks is to use observed \emph{total} traffic flows out of each origin and into each destination to estimate \emph{detailed} traffics between origin-destination pairs \citep{carey1981method,nguyen1984estimating,sheffi1985urban,chang2021mobility}. The key idea is to find a traffic assignment satisfying the total flow constraints that is ``close'' to some known reference traffic pattern. The resulting (relative) entropy minimization principle, detailed in \eqref{eq:relative-entropy-minimization}, is an important optimization perspective on Sinkhorn's algorithm.

\textbf{Demography.} A problem similar to that in networks arises in demography. Given out-of-date inter-regional migration statistics and up-to-date net migrations from and into each region, the task is to estimate migration flows that are consistent with the marginal statistics \citep{plane1982information}.

\textbf{Economics.}
General equilibrium models in economics employ \emph{social accounting matrices}, which record the flow of funds between important (aggregate) agents in an economy at different points in time \citep{stone1962multiple,pyatt1985social}. Often accurate data is available on the total expenditure and receipts for each agent, but due to survey error or latency, detailed flows are not always consistent with these marginal statistics. Thus they need to be ``adjusted'' to satisfy consistency requirements.  Other important applications of the matrix balancing problem in economics include the estimation of gravity equations in inter-regional and international trade \citep{uribe1966information,wilson1969use,anderson2003gravity,silva2006log} and coefficient matrices in input-output models \citep{leontief1965structure,stone1971computable,bacharach1970biproportional}. In recent years, optimal transport \citep{villani2009optimal} has found great success in economics \citep{carlier2016vector,galichon2018optimal,galichon2021unreasonable,galichon2021matching}. As matrix balancing and Sinkhorn's algorithm are closely connected to optimal transport (\cref{sec:linear-convergence}), they are likely to have more applications in economics.

\textbf{Statistics.}
A contingency table encodes frequencies of subgroups of populations, where the rows and columns correspond to values of two categorical variables, such as gender and age. Similar to social accounting matrices, a common problem is to adjust out-of-date or inaccurate cell values of a table given accurate marginal frequencies. The problem is first studied by  \citet{deming1940least}, who proposed the classic iterative algorithm. \citet{ireland1968contingency} formalized its underlying entropy optimization principle, and \citet{fienberg1970iterative} analyzed its convergence. 

\textbf{Optimization and Machine Learning.} Matrix balancing plays a different but equally important role in optimization. Given a linear system $Ax=b$ with non-singular $A$, it is well-known that the convergence of first order solution methods depends on the \emph{condition number} of $A$, and an important problem is to find diagonal \emph{preconditioners} $D^1,D^0$ such that $D^1AD^0$ has smaller condition number. Although it is possible to find optimal diagonal preconditioners via semidefinite programming \citep{boyd1994linear,qu2022optimal}, matrix balancing methods remain very attractive heuristics due to their low computational costs, and continue to be an important component of modern workhorse optimization solvers \citep{ruiz2001scaling,bradley2010algorithms,knight2013fast,stellato2020osqp,gao2022hdsdp}.

In recent years, optimal transport distances have become an important tool in machine learning and optimization for measuring the similarity between probability distributions \citep{arjovsky2017wasserstein,peyre2019computational,blanchet2019robust,mohajerin2018data,kuhn2019wasserstein}. Besides appealing theoretical properties, efficient methods to approximate them in practice have also contributed to their wide adoption. This is achieved through an entropic regularization of the OT problem, which is precisely equivalent to the matrix balancing problem and solved via Sinkhorn's algorithm \citep{cuturi2013sinkhorn,altschuler2017near,dvurechensky2018computational}.   

\textbf{Political Representation.} The apportionment of representation seats based on election results has found unexpected solution in matrix balancing. A standard example consists of the matrix recording the votes each party received from different regions. The marginal constraints are that each party's total number of seats be proportional to the number of votes they receive, and similar for each region. A distinct feature of this problem is that the final apportionment matrix must have integer values, and variants of the standard algorithm that incorporate \emph{rounding} have been proposed \citep{balinski2006matrices,pukelsheim2006current,maier2010divisor}. More than just mathematical gadgets, they have found real-world implementations in Swiss cities such as Zurich \citep{pukelsheim2009iterative}.

\textbf{Markov Chains.}
Last but not least, Markov chains and related topics offer another rich set of applications for matrix balancing. \citet{schro1931uber} considered a continuous version of the following problem. Given a ``prior'' transition matrix $A$ of a Markov chain and \emph{observed} distributions $p^0,p^1$ before and after the transition, find the most probable transition matrix (or path) $\hat A$ that satisfies $\hat A p^0=p^1$. This is a variant of the matrix balancing problem and has been studied and generalized in a long line of works \citep{fortet1940resolution,beurling1960automorphism,ruschendorf1995convergence,gurvits2004classical,georgiou2015positive,friedland2017schrodinger}. Applications in marketing estimate customers' transition probabilities between different brands using market share data
\citep{theil1966quadratic}. Coming full circle back to  choice modeling, matrix balancing has also been used to rank nodes of a network. \citet{knight2008sinkhorn} explains how the inverses of left and right scalings of the adjacency matrix with uniform target marginals (stationary distributions) can be naturally interpreted as measures of their ability to attract and emit traffic. This approach is also related to the works of \citet{lamond1981bregman,kleinberg1999authoritative,tomlin2003new}.


\section{Numerical Experiments}
\label{sec:empirics}
We compare the empirical performance of Sinkhorn's algorithm with the iterative LSR (I-LSR) algorithm of \citet{maystre2015fast} on real choice datasets. Because the implementation of I-LSR by \citet{maystre2015fast}
only accommodates pairwise comparison data and partial ranking data, but does not easily generalize to multi-way choice data, we focus on data with pairwise comparisons. 

We use the
natural parameters $\log s_{j}$ (logits) instead of $s_{j}$ when computing and evaluating
the updates, as the probability of $j$ winning
over $k$ is proportional to the ratio $s_{j}/s_{k}$, so that $s_j$ are usually logarithmically spaced. To make sure that estimates
are normalized, we impose the normalization that $\sum_{j}s_{j}=m$,
the number of objects, at the end of each iteration, although due to
the logarithm scale of the convergence criterion, the choice of normalization
does not seem to significantly affect the performances of the algorithms.
%generated based on the proposal in \citet{agarwal2018accelerated}, and partial ranking data, generated with code from \citet{maystre2015fast}.

We evaluate the algorithms on five real-world datasets consisting
of partial ranking or pairwise comparison data. The NASCAR dataset
consists of ranking results of the 2002 season NASCAR races. The SUSHI
datasets consist of rankings of sushi items. The Youtube dataset consists
of pairwise comparisons between videos and which one was considered
more entertaining by users. The GIFGIF dataset similarly consists
of pairwise comparisons of GIFs that are rated based on which one
is closer to describing a specific sentiment, such as happiness and
anger. We downsampled the Youtube and GIFGIF datasets due to memory
constraints. 
\begin{table*}[t]
\begin{centering}
\hspace*{-3cm}%
\begin{tabular}{|c|c|c|c|c|c|c|c|c|}
\hline 
\multirow{2}{*}{dataset} & \multirow{2}{*}{data type} & \multirow{2}{*}{items} & \multirow{2}{*}{observations} & \multirow{2}{*}{$k$} & \multicolumn{2}{c|}{Sinkhorn} & \multicolumn{2}{c|}{I-LSR}\tabularnewline
\cline{6-9} \cline{7-9} \cline{8-9} \cline{9-9} 
 &  &  &  &  & iterations & time & iterations & time\tabularnewline
\hline 
\hline 
NASCAR & $k$-way ranking & 83 & 36 & 43 & 20 & \textbf{0.029} & 13 & 0.960\tabularnewline
\hline 
SUSHI-10 & $k$-way ranking & 10 & 5000 & 10 & 16 & \textbf{0.025} & 8 & 1.708\tabularnewline
\hline 
SUSHI-100 & $k$-way ranking & 100 & 5000 & 10 & 21 & \textbf{0.253} & 9 & 2.142\tabularnewline
\hline 
Youtube & pairwise comparison & 2156 & 28134 & 2 & 89 & \textbf{7.984} & 33 & 15.026\tabularnewline
\hline 
GIFGIF & pairwise comparison & 2503 & 6876 & 2 & 1656 & \textbf{30.97} & 315 & 138.63\tabularnewline
\hline 
\end{tabular}\hspace*{-3cm}
\par\end{centering}
\caption{Performance of iterative ML inference algorithms on five real datasets. Youtube and GIFGIF data were subsampled. 
Convergence is declared when the maximum entry-wise change of an update
is less than $10^{-8}$. At convergence, the ML estimates returned
by the algorithms have entry-wise difference of at most $10^{-10}$.}
\label{tab:empirical}
\end{table*}

In \cref{tab:empirical} we report the running time of the three algorithms on different
datasets. Convergence is declared when the maximum entry-wise change
of an update to the natural parameters $\log s_{i}$ is less than
$10^{-8}$. At convergence, the MLEs returned by the algorithms have entry-wise difference of at most $10^{-10}$. %Since
% we know that the sequence of MM algorithm must converge to a stationary
% point, and that the matrix scaling algorithm is equivalent to the
% MM algorithm for pairwise comparisons and partial rankings, we know
% that the final estimates must be the MLE. 
We see that Sinkhorn's algorithm consistently outperforms
the I-LSR algorithm in terms of convergence speed. It also has the additional advantage of being parallelized with
elementary matrix-vector operations, whereas the iterative I-LSR algorithm needs to repeatedly
compute the steady-state of a continuous-time Markov chain, which
is prone to problems of ill-conditioning. This also explains why Sinkhorn's algorithm may take more iterations but has better wall clock time, since each iteration is much less costly. On the other hand, we note
that for large datasets, particularly those with a large number of
observations or alternatives, the dimension of $A$ used
may become too large for the memory of a single machine. If this is still a problem after removing duplicate rows and columns according to \cref{sec:equivalence}, we can use distributed implementations of Sinkhorn's algorithm, which in view of its connections to message passing algorithms, is a standard procedure.
% decompose $A$ into blocks of sub-matrices, on which \cref{alg:scaling} allows for parallel computation. 
%See \cref{alg:scaling-distributed}.
\section{Proofs from \secref{sec:qkd}}
\label{app:proofs}

In \secref{sec:qkd} we show how to define the security of QKD in a
composable framework and relate this to the trace distance security
criterion introduced in \textcite{Ren05}. This composable treatment of
the security of QKD follows the literature \cite{BHLMO05,MR09}, and
the results presented in \secref{sec:qkd} may be found in
\textcite{BHLMO05,MR09} as well. The formulation of the statements
differs however from those works, since we use here the Abstract
Cryptography framework of \textcite{MR11}. So for completeness, we
provide here proofs of the main results from \secref{sec:qkd}.

\begin{proof}[Proof of \thmref{thm:qkd}]
  Recall that in \secref{sec:security.simulator} we fixed the
  simulator and show that to satisfy \eqnref{eq:qkd.security} it is
  sufficient for \eqnref{eq:qkd.security.2} to hold. Here, we will
  break \eqnref{eq:qkd.security.2} into security [\eqnref{eq:qkd.cor}]
  and correctness [\eqnref{eq:qkd.sec}], thus proving the theorem.

  Let us define $\gamma_{ABE}$ to be a state obtained from
  $\rho^{\top}_{ABE}$ [\eqnref{eq:qkd.security.tmp}] by throwing away
  the $B$ system and replacing it with a copy of $A$, i.e., \[
  \gamma_{ABE} = \frac{1}{1-p^\bot} \sum_{k_A,k_B \in \cK} p_{k_A,k_B}
  \proj{k_A,k_A} \otimes \rho^{k_A,k_B}_E.\] From the triangle
  inequality we get \begin{multline*} D(\rho^\top_{ABE},\tau_{AB} \otimes
  \rho^\top_{E}) \leq \\ D(\rho^\top_{ABE},\gamma_{ABE}) +
  D(\gamma_{ABE},\tau_{AB} \otimes \rho^\top_{E}) .\end{multline*}

Since in the states $\gamma_{ABE}$ and
$\tau_{AB} \otimes \rho^\top_{E}$ the $B$ system is a copy of the $A$
system, it does not modify the distance. Furthermore,
$\trace[B]{\gamma_{ABE}} =
\trace[B]{\rho^{\top}_{ABE}}$. Hence
\[D(\gamma_{ABE},\tau_{AB} \otimes \rho^\top_{E}) =
  D(\gamma_{AE},\tau_{A} \otimes \rho^\top_{E}) =
  D(\rho^\top_{AE},\tau_{A} \otimes \rho^\top_{E}).\]

For the other term note that
\begin{align*}
  & D(\rho^\top_{ABE},\gamma_{ABE}) \\
  & \qquad \leq \sum_{k_A,k_B} \frac{p_{k_A,k_B}}{1-p^{\bot}}
    D\left(\proj{k_A,k_B} \otimes \rho^{k_A,k_B}_E,\right. \\
  & \qquad \qquad \qquad \qquad \qquad \qquad \left.\proj{k_A,k_A} \otimes \rho^{k_A,k_B}_E \right)\\
  & \qquad = \sum_{k_A \neq k_B} \frac{p_{k_A,k_B}}{1-p^{\bot}} = \frac{1}{1-p^{\bot}}\Pr
  \left[ K_A \neq K_B \right].
\end{align*}
Putting the above together with \eqnref{eq:qkd.security.2}, we get
\begin{align*} & D(\rho_{ABE},\tilde{\rho}_{ABE}) \\
  & \qquad = (1-p^\bot)
  D(\rho^\top_{ABE},\tau_{AB} \otimes \rho^\bot_{E}) \\ & \qquad \leq \Pr
  \left[ K_A \neq K_B \right] + (1-p^\bot) D(\rho^\top_{AE},\tau_{A}
  \otimes \rho^\top_{E}). \qedhere \end{align*}
\end{proof}

\begin{proof}[Proof of \lemref{lem:robustness}]
  By construction, $\aK_\delta$ aborts with exactly the same
  probability as the real system. And because $\sigma^{\qkd}_E$
  simulates the real protocols, if we plug a converter $\pi_E$ in
  $\aK\sigma^{\qkd}_E$ which emulates the noisy channel $\aQ_q$ and
  blogs the output of the simulated authentic channel, then
  $\aK_\delta = \aK\sigma^{\qkd}_E\pi_E$. Also note that by
  construction we have
  $\aQ_q \| \aA' = \left(\aQ \| \aA\right) \pi_E$. Thus
  \begin{multline*} d\left( \pi_A^{\qkd}\pi_B^{\qkd}(\aQ_q \| \aA')
      ,\aK_\delta\right) \\ = d\left( \pi_A^{\qkd}\pi_B^{\qkd}\left(\aQ
        \| \aA\right) \pi_E , \aK\sigma^{\qkd}_E\pi_E\right). \end{multline*}

  Finally, because the converter $\pi_E$ on both the real and ideal
  systems can only decrease their distance (see
  \secref{sec:ac.systems}), the result follows.
\end{proof}


%%% Local Variables:
%%% TeX-master: "main.tex"
%%% End:

 \end{APPENDICES}
%%%%%%%%%%%%%%%%%%%%%%%%%%%%%%%%%%%%%%%%%%%%%%%%%%%%%%%%%%%%%%%%%%%%%%%%%%%%%%%
%%%%%%%%%%%%%%%%%%%%%%%%%%%%%%%%%%%%%%%%%%%%%%%%%%%%%%%%%%%%%%%%%%%%%%%%%%%%%%%


\end{document}

