\section{Conclusion}
In this paper, we develop extensive connections between matrix balancing and the estimation of a broad class of choice models. In particular, many algorithms in choice modeling can be viewed as special cases or analogs of Sinkhorn's algorithm for matrix balancing. These connections benefit both disciplines. For choice modeling, they open the door to tools and insights from a rich research area in optimization and numerical linear algebra, potentially leading to new  results on the estimation of choice models. For matrix balancing, the connections inspire us to resolve some long-standing open problems on the linear convergence of Sinkhorn's algorithm for non-negative matrices, revealing the importance of algebraic connectivity and related spectral properties. Moreover, we propose regularization methods for Sinkhorn's algorithm inspired by works from choice modeling, in order to address existence and convergence issues for matrix balancing. We believe that the connections we establish in this paper are useful for researchers from both domains and can lead to further interesting results. 

\section{Acknowledgements}
This work is supported in part by NSF CAREER Award \#2143176. We are very grateful for insightful comments and suggestions from Serina Chang, Patrick Ding, Wenzhi Gao, Guido Imbens, S{\"u}leyman Kerimov, Yongchan Kwon, Han Hong, Frederic Koehler, Pavel Shibayev, Ruoxuan Xiong, and Yinyu Ye. % and invitation to present at the INFORMS 2023 Session on Econometric, Big Data Methods and Applications to Finance. 