
\documentclass{article}
% if you need to pass options to natbib, use, e.g.:
% \PassOptionsToPackage{numbers, compress}{natbib}
% before loading nips_2017
%
% to avoid loading the natbib package, add option nonatbib:
% \usepackage[nonatbib]{nips_2017}

%\usepackage{nips_2017}
% to compile a camera-ready version, add the [final] option, e.g.:
%\usepackage[final]{nips_2017}

% use Times
\usepackage{times}
% For figures
\usepackage{graphicx} % more modern
%\usepackage{epsfig} % less modern

\usepackage[utf8]{inputenc} % allow utf-8 input
\usepackage[T1]{fontenc}    % use 8-bit T1 fonts
\usepackage{hyperref}       % hyperlinks
\usepackage{url}            % simple URL typesetting
\usepackage{booktabs}       % professional-quality tables
\usepackage{amsfonts}       % blackboard math symbols
\usepackage{nicefrac}       % compact symbols for 1/2, etc.
\usepackage{microtype}      % microtypography
\usepackage[inline]{enumitem}
\usepackage[margin=1in]{geometry}

% For citations
\usepackage{natbib}

% For algorithms
\usepackage{algorithm}
\usepackage{algorithmic}

% As of 2011, we use the hyperref package to produce hyperlinks in the
% resulting PDF.  If this breaks your system, please commend out the
% following usepackage line and replace \usepackage{icml2016} with
% \usepackage[nohyperref]{icml2016} above.
\usepackage{latexsym}
\usepackage{amsmath}
\usepackage{amsthm}
\usepackage{amssymb}
\usepackage{mathtools}
\usepackage{multirow}
\usepackage{bm}
\usepackage{color,soul}
\usepackage{lscape}
\usepackage{graphicx,epstopdf}
\usepackage{psfrag}
\usepackage{rotating}
\usepackage{outlines}
\usepackage[textsize=small]{todonotes}
\usepackage{subcaption}
\usepackage{textcomp}
\usepackage{booktabs}

\newcommand{\fix}{\marginpar{FIX}}
\newcommand{\new}{\marginpar{NEW}}

\newcommand{\sameer}[1]{\todo[color=blue!20]{\textbf{s:} #1}{}}
\newcommand{\pouya}[1]{\todo[color=green!20]{\textbf{p:} #1}{}}

\newcommand{\para}[1]{\vspace{2mm}\noindent\textbf{#1}:}

%\newcommand{\citet}[1]{\cite{#1}}
%\newcommand{\citep}[1]{\cite{#1}}

% --------------------------------------------------------------
% --------------------------------------------------------------
% --------------------------------------------------------------
\DeclarePairedDelimiter\floor{\lfloor}{\rfloor}
%\newcommand{\R}{\ensuremath{\Re}}
\newcommand{\R}{\ensuremath{\Real}}

\newcommand{\cut}[1]{}

%TODO
\newcommand{\todo}[1]{{\color{red}{\bf [TODO]:~{#1}}}}

%THEOREMS
\newtheorem{theorem}{Theorem}
\newtheorem{corollary}{Corollary}
\newtheorem{lemma}{Lemma}
\newtheorem{proposition}{Proposition}
\newtheorem{problem}{Problem}
\newtheorem{definition}{Definition}
\newtheorem{remark}{Remark}
\newtheorem{example}{Example}
\newtheorem{assumption}{Assumption}

%HANS' CONVENIENCES
\newcommand{\define}[1]{\textit{#1}}
\newcommand{\join}{\vee}
\newcommand{\meet}{\wedge}
\newcommand{\bigjoin}{\bigvee}
\newcommand{\bigmeet}{\bigwedge}
\newcommand{\jointimes}{\boxplus}
\newcommand{\meettimes}{\boxplus'}
\newcommand{\bigjoinplus}{\bigjoin}
\newcommand{\bigmeetplus}{\bigmeet}
\newcommand{\joinplus}{\join}
\newcommand{\meetplus}{\meet}
\newcommand{\lattice}[1]{\mathbf{#1}}
\newcommand{\semimod}{\mathcal{S}}
\newcommand{\graph}{\mathcal{G}}
\newcommand{\nodes}{\mathcal{V}}
\newcommand{\agents}{\{1,2,\dots,N\}}
\newcommand{\edges}{\mathcal{E}}
\newcommand{\neighbors}{\mathcal{N}}
\newcommand{\Weights}{\mathcal{A}}
\renewcommand{\leq}{\leqslant}
\renewcommand{\geq}{\geqslant}
\renewcommand{\preceq}{\preccurlyeq}
\renewcommand{\succeq}{\succcurlyeq}
\newcommand{\Rmax}{\mathbb{R}_{\mathrm{max}}}
\newcommand{\Rmin}{\mathbb{R}_{\mathrm{min}}}
\newcommand{\Rext}{\overline{\mathbb{R}}}
\newcommand{\R}{\mathbb{R}}
\newcommand{\N}{\mathbb{N}}
\newcommand{\A}{\mathbf{A}}
\newcommand{\B}{\mathbf{B}}
\newcommand{\x}{\mathbf{x}}
\newcommand{\e}{\mathbf{e}}
\newcommand{\X}{\mathbf{X}}
\newcommand{\W}{\mathbf{W}}
\newcommand{\weights}{\mathcal{W}}
\newcommand{\alternatives}{\mathcal{X}}
\newcommand{\xsol}{\bar{\mathbf{x}}}
\newcommand{\y}{\mathbf{y}}
\newcommand{\Y}{\mathbf{Y}}
\newcommand{\z}{\mathbf{z}}
\newcommand{\Z}{\mathbf{Z}}
\renewcommand{\a}{\mathbf{a}}
\renewcommand{\b}{\mathbf{b}}
\newcommand{\I}{\mathbf{I}}
\DeclareMathOperator{\supp}{supp}
\newcommand{\Par}[2]{\mathcal{P}_{{#1} \to {#2}}}
\newcommand{\Laplacian}{\mathcal{L}}
\newcommand{\F}{\mathcal{F}}
\newcommand{\inv}[1]{{#1}^{\sharp}}
\newcommand{\energy}{Q}
\newcommand{\err}{\mathrm{err}}
\newcommand{\argmin}{\mathrm{argmin}}
\newcommand{\argmax}{\mathrm{argmax}}
\newcommand\tab[1][1cm]{\hspace*{#1}}



\title{Compact Factorization of Matrices Using Generalized Round-Rank}
%Compact Representation Capabilities

\author{
	\bf Pouya Pezeshkpour\\
	%Department of Electrical Engineering\\
	University of California\\
	Irvine, CA\\\texttt{pezeshkp@uci.edu} \\
    \and
    \bf Carlos Guestrin\\
	%Department of Computer Science\\
	University of Washington \\
	Seattle, WA\\
	\texttt{guestrin@cs.uw.edu} \\
	\and
    \bf Sameer Singh \\
	%Department of Computer Science\\
	University of California\\
	Irvine, CA\\
	\texttt{sameer@uci.edu} \\
}
\date{}
\begin{document}

\maketitle

%, however primarily from a linear factorization perspective. 
\begin{abstract}
Matrix factorization is a well-studied task in machine learning for compactly representing large, noisy data. In our approach, instead of using the traditional concept of matrix rank, we define a new notion of \emph{link-rank} based on a non-linear link function used within factorization. In particular, by applying the round function on a factorization to obtain ordinal-valued matrices, we introduce \emph{generalized round-rank} ($\GRR$). We show that not only are there many full-rank matrices that are low $\GRR$, but further, that these matrices cannot be approximated well by low-rank linear factorization. We provide uniqueness conditions of this formulation and provide gradient descent-based algorithms. Finally, we present experiments on real-world datasets to demonstrate that the $\GRR$-based factorization is significantly more accurate than linear factorization, while converging faster and using lower rank representations.  
\end{abstract}

\section{Introduction}
\label{sec:intro}

Matrix factorization is a popular machine learning technique, with applications in variety of domains, such as recommendation systems~\citep{lawrence09:non-linear,salakhutdinov08:probabilistic}, natural language processing~\citep{riedel13:relation}, and computer vision~\citep{huang03:learning}.
Due to this widespread use of these models, there has been considerable theoretical analysis of the various properties of low-rank approximations of real-valued matrices, including approximation rank~\citep{alon13:the-approximate,davenport14:1-bit} and sample complexity~\citep{balcan2017optimal}.

Rather than assume real-valued data, a number of studies (particularly ones on practical applications) focus on more specific data types,
such as binary data~\citep{nickel13:logistic}, integer data~\citep{lin2009integer}, and ordinal data~\citep{koren2011ordrec,udell14:generalized}.
For such matrices, existing approaches have used different \emph{link} functions, applied in an element-wise manner to the low-rank representation~\citep{neumann16:what}, i.e. the output $\hat{Y}$ is $\link(\U^T\V)$ instead of the conventional $\U^T\V$.
These link functions have been justified from a probabilistic point of view \citep{collins01:a-generalization,salakhutdinov08:bayesian}, and have provided considerable success in  empirical settings.
However, theoretical results for linear factorization do not apply here, and thus the expressive power of the factorization models with non-linear link functions is not clear, and neither is the relation of the rank of a matrix to the link function used.
%\sameer{this might be better suited for related work, here we should focus on the fact that there's a mismatch}} introduced a generalization of PCA method to lost function for non real-valued data,
%such as binary-valued.
%\citet{salakhutdinov08:bayesian

In this paper, we first define a generalized notion of rank based on the link function $\link$, as the rank of a latent matrix before the link function is applied.
We focus on a link function that applies to factorization of integer-valued matrices: the generalized round function ($\text{GRF}$), and define the corresponding generalized round-rank ($\GRR$). 
After providing background on $\GRR$, we show that there are many low-$\GRR$ matrices that are full rank\footnote{We will refer to rank of a matrix as its \emph{linear} rank, and refer to the introduced generalized rank as \emph{link}-rank.}.
Moreover, we also study the approximation limitations of linear rank, by showing, for example, that low $\GRR$ matrices often cannot be approximated by low-rank linear matrices.
%
We define uniqueness for $\GRR$-based matrix completion, and derive its necessary and sufficient conditions. These properties demonstrate that many full linear-rank matrices can be factorized using low-rank matrices if an appropriate link function is used.
%\sameer{include other theoretical results?.\textcolor{red}{done}}
%

We also present an empirical evaluation of factorization with different link functions for matrix reconstruction and completion.
We show that using link functions is efficient compared to linear rank, in that gradient-based optimization approach learns more accurate reconstructions using a lower rank representation and fewer training samples.
We also perform experiments on matrix completion on two recommendation datasets, and demonstrate that appropriate link function outperform linear factorization, thus can play a crucial role in accurate matrix completion.


% --------------------------------------------------------------
% --------------------------------------------------------------
% --------------------------------------------------------------
\section{Link Functions and Generalized Matrix Rank}
\label{sec:link}

Here we introduce our notation for matrix factorization, and use it to introduce link functions and \emph{generalized link-rank}.
We will focus on the round function and round-rank, introduce their generalized versions, and present their properties.
%by introducing the concept of the generalized matrix, we define the generalized round-rank as a special case.


% --------------------------------------------------------------
% --------------------------------------------------------------
%\pouya{subsections can go inside the text}
\para{Rank Based Factorization}
Matrix factorization, broadly defined, is a decomposition of a matrix as a multiplication of two matrices. 
Accordingly, rank of a matrix $\Y\in \Real^{n\times m}$ defined as the smallest natural number $r$ such that:
%\begin{align}
$
\Y = \U \V^T, \text{or,}
\Y_{ij} = \sum_k \U_{ik}\V_{jk}
$, %\end{align}
where $\U\in \Real^{n\times r}$ and $\V\in \Real^{n\times r}$. 
We use $\rank(\Y)$ to indicate the rank of a matrix $\Y$. 
% --------------------------------------------------------------
% --------------------------------------------------------------
% --------------------------------------------------------------

\para{Link Functions and Link-Rank}
Since the matrix $\Y$ may be from a domain $\mathbb{V}^{n\times m}$ different from real matrices, link functions can be used to define an alternate factorization: %, as follows:
%\sameer{what is X?, what is $\theta$?..\textcolor{red}{done}}
\begin{align}
\Y = \link_\taus(\X), 
\X = \U \V^T,
%\tau \in\Lparams,
\end{align}
% --------------------------------------------------------------
where $\Y \in \mathbb{V}^{n\times m}$, $\link:\Real\rightarrow\mathbb{V}$ (applied element-wise), $\X\in \Real^{n\times m}$, $\U\in \Real^{n\times r}$, $\V\in \Real^{n\times r}$, and $\taus$ represent parameters of the link function, if any. % and $\Lparams$ is a set of all possible thresholds. 
Examples of link functions that we will study in this paper include the \emph{round} function for binary matrices, and its generalization to ordinal-valued matrices.
Link functions were introduced for matrix factorization by \citet{singh08:a-unified}, consequently \citet{udell14:generalized} presented their generalization to loss functions and regularization for \emph{abstract data types}.

%We represent the link functions, introduced in , as:
%Previously, some generalizations of link functions, losses, and regularization for  presented in .
% --------------------------------------------------------------
% --------------------------------------------------------------
%\subsection{Generalized Matrix Rank}
%\label{sec:link:gmr}
\begin{thm:def}\label{thm:rank}
    Given a matrix $\Y$ and a link function $\link_\taus$ parameterized by $\taus$, the \textbf{link-rank} $\rank_\link$ of $\Y$ is defined as the minimal rank of a real-matrix $\X$ such that,  $\Y = \link_\taus(\X)$,
    \begin{equation} \label{eq:rank}
        \rank_\link(\Y) = \min_{\X\in\Real^{n\times m}, \taus} \left\{ \rank(\X); \Y = \link_\taus(\X) \right\}
    \end{equation}
\end{thm:def}
Note that with $\link\equiv I$, i.e. $\link(x)=x$, $\rank_\link(\Y)=\rank(\Y)$. 

\para{Sign and Round Rank}
If we consider the $\text{sign}$ function as the link function, where $
%\begin{align}
\text{sign}(x)= \{ 0 \text{ if } x<0, 1 \text{ o.w.} \}$
% \begin{cases}
%               - & x < 0\\
%               + & 0 \leqslant x
%             \end{cases}
%\end{align}
(applied element-wise to the entries of the matrix), the link-rank defined above corresponds to the well-known $\text{sign-rank}$ for binary matrices~\citep{neumann2015some}: 
%\vskip -7mm
\[
\text{sign-rank}(\Y) = \min_{\X\in\Real^{n\times m}} \left\{ \rank(\X); \Y = \text{sign}(\X) \right\}.
\]
%\vskip -3mm
A variation of the $\text{sign}$ function that uses a threshold $\tau$, $\Round_\tau(x)=\{ 0 \text{ if } x<\tau, 1 \text{ o.w.} \}$ when used as a link function results in the $\text{round-rank}$ for binary matrices, i.e.
\[
\text{round-rank}_{\tau}(\Y) = \min_{\X\in\Real^{n\times m}} \left\{ \rank(\X); \Y = \Round_{\tau}(\X) \right\},
\] %
as shown in ~\citet{neumann2015some}. 
Thus, our notion of \emph{link-rank} not only unifies existing definitions of rank, but can be used for novel ones, as we will do next.

\para{Generalized Round-Rank ($\GRR$)}
\label{sec:rrabk}
Many matrix factorization applications use ordinal values, i.e $\mathbb{V}=\{0,1,\ldots,N\}$. %, and thus we explore link functions for such matrices.
For these, we define generalized round function ($\text{GRF}$): %\sameer{is this Round, or GRF, or Multi-Step?.\textcolor{red}{done}}
\begin{align}
\text{GRF}_{\tau_1,...,\tau_N}(x)=
\begin{cases}
               0\tab\tab x \leq \tau_1\\
               1\tab\tab \tau_1 < x \leq \tau_2\\
               \vdots\\
               N-1\tab \tau_{N-1} < x \leq \tau_{N}\\
               N\tab\tab \text{o.w.}
            \end{cases}
\end{align}
where its parameters $\taus\equiv\{\tau_1,...,\tau_N\}$ are thresholds (sorted in ascending order). % and $x \in R$. 
%Generalized round-rank, is a specific form of generalized matrix rank where the link function is generalized version of round function. 
Accordingly, we define \emph{generalized round-rank} ($\GRR$) for any ordinal matrix $\Y$ as: 
\[
\GRR_{\taus}(\Y)=\min_{\X\in\Real^{n\times m}} \left\{ \rank(\X); \Y = \text{GRF}_{\taus}(\X) \right\}
.
\]
%Note that, we apply $\GRF$ on the matrix $\X$ in element wise manner. Furthermore, every entries of matrix $\Y$ should be one of $\{0,...,N\}$. 
Here, we are primarily interested in exploring the utility of $\GRR$ and, in particular, compare the representation capabilities of low-$\GRR$ matrices to low-linear rank matrices.
To this end, we present the following interesting property of $\GRR$. % as following theorem.% which is proved in the supplementary materials. 


%%%%%%%%%%%%%%%%%%%%%%%%%%%%%%%%%%%%%%%%%%%%%%%%%%%%%%%%%%%%%%%%%%
\iffalse
\begin{thm:lemma}

For $A ,B \in \{0,...,N\}^{n \times m}$:
\begin{align}
\GRR_{\tau_1,...\tau_N}(A) &\leq min(n,m)\\
 \GRR_{\tau_1,...\tau_N}(A) &=\GRR_{\tau_1,...\tau_N}(A^T)\\
 \GRR_{\tau_1,...\tau_N}(A+B) &\leq \GRR_{\tau_1,...\tau_N}(A)+\GRR_{\tau_1,...\tau_N}(B)
\end{align}
Where $+$ is in the real numbers and $A+B \in \{0,...,N\}^{n \times m}$.
\end{thm:lemma}


\begin{thm:lemma}

the following decomposition holds for $\text{GRF}$:
\begin{align}
Round_{\tau_1,...\tau_N}(A)=\sum\limits_{i=1}^{N}Round_{\tau_i}(A)
\end{align}
\end{thm:lemma}

\begin{thm:lemma}

For any arbitrary subset of thresholds $T=\{\tau_{i_1},...,\tau_{i_r}\}$:
\begin{align}
\GRR_{\tau_1,...\tau_N}(A)\geq \GRR_{T}(\bar A)
\end{align}
Where $\bar A$ attained by the following transformation in matrix $A$:
\begin{align}
\bar A & =[b_{ij}]_{n\times m}\\
b_{ij} & =
\begin{cases}
	0, & \text{if } a_{ij} \in \{0,..,i_{1}-1\} \\
	1, & \text{if } a_{ij} \in\{i_{1},..,i_{2}-1\} \\
	\vdots\\
	r-1, & \text{if } a_{ij} \in\{i_{r},..,N-1\} 
\end{cases}
\end{align}
\end{thm:lemma}

\begin{thm:lemma}

Following inequality holds for GRR:
\begin{align}
GRR_{\tau_{1},...\tau_{N}}(A)\leq GRR_{\tau_1,...\tau_N,\tau_{N+1}}(A)
\end{align}
\end{thm:lemma}

\begin{thm:lemma}

Lets define the function $F: R^N \rightarrow N$ as follows:
\begin{align}
F(\tau_{1},...\tau_{N})= GRR_{\tau_1,...\tau_N}(A)
\end{align}
Where $A$ is a matrix in $\{0,...,N\}^{n\times m}$. Then we have the following inequality:
\begin{align}
F((\tau_{1}+\tau\textquotesingle_{1})/2,...\tau_{N}) \leq F(\tau_{1},...\tau_{N})+F(\tau\textquotesingle_{1},...\tau_{N})
\end{align}
\end{thm:lemma}


\begin{thm:lemma}
We have the following inequality:
\begin{align}
F(\tau_{1}+\tau\textquotesingle_{1},...,\tau_{N}+\tau\textquotesingle_{N}) \leq  F(\tau_{1},...\tau_{N})+F(\tau\textquotesingle_{1},...,\tau\textquotesingle_{N})
\end{align}
\end{thm:lemma}

\fi
%%%%%%%%%%%%%%%%%%%%%%%%%%%%%%%%%%%%%%%%%%%%%%%%%%%%%%%%%%%%%%%%%
\begin{thm:thm}
\label{thm:thresh}
For a given matrix $\Y \in \{0,\ldots,N\}^{n \times m}$, let's assume $\taus^*$ is the set of optimal thresholds, i.e. $ \GRR_{\taus^{\star}}(Y)=\text{argmin}_{\taus}\GRR_{\taus}(Y)$, then for any other $\taus'$: % with size $N$
\begin{align}
\label{thresh}
\GRR_{\taus'}(\Y) \leq N\times\GRR_{\taus^{\star}}(\Y)+1
%$.
\end{align}
%Where $T$ is the set of all possible thresholds with the size of $N$.
\begin{proof}
We provide a sketch of proof here, and include the details in the appendix.
%
%To prove above theorem, we first provide two lemmas. 
We can show that the GRR can change at most by $1$ if we add a constant to all the thresholds and does not change at all if all the thresholds are multiplied by a constant. % does not change the $\GRR$.   
%
Further, we show that there exist $\epsilon_i$ for every $i \in \{1,...,N-1\}$ such that shifting $\tau_i$ by $\epsilon_i$ does not change the GRR. %, there exists an $\epsilon_i$ which satisfies the following equality:
%\begin{align}
%\GRR_{\tau_{1},...,\tau_{i}-\epsilon_i,...,\tau_{N}}(\Y) = \GRR_{\tau_{1},...,\tau_{N}}(\Y)
%\end{align}
These properties provide a bound to the change in GRR between any two sets of thresholds. % $\mathbf{\tau}$ to $\mathbf{\tau'}$.
\cut{
\begin{thm:lemma}
\label{thm:thr}
We have the following property for \GRR: 
\begin{align}
\GRR_{\tau_{1}+c,...,\tau_{N}+c}(\Y)\leq \GRR_{\tau_1,...,\tau_N}(\Y)+1 
\end{align}
Where $c$ is a real number.
\begin{proof}
We define $\mathbf{B}$ and $\mathbf{B'}$ as follows:
\begin{align}
\mathbf{B} &=\{B|\GRF_{\tau_1,..,.\tau_N}(B)=\Y\}\\
\mathbf{B'} &=\{B'|\GRF_{\tau_1+c,...,\tau_N+c}(B')=\Y\}
\end{align}
For an arbitrary $B\in \mathbf{B}$ let's assume we have matrix $\U$ and $\V$ such that $B=\U \times \V^T$. If we add a column to the end of $\U$ and $\V$ and call them $\U'$ and $\V'$ as follows:
\begin{align}
U'&=\begin{bmatrix}
 & c\\ 
U & \vdots \\ 
 & c
\end{bmatrix}
,& V'&=\begin{bmatrix}
 & 1\\ 
V & \vdots \\ 
 & 1 
\end{bmatrix}
\end{align}
It is clear that $B'=\U'\times \V'^T \in \mathbf{B'}$. Furthermore, using the fact that $\rank(B')\leq \rank(B)+1$ we can complete the proof.
\end{proof}
\end{thm:lemma}

\begin{thm:lemma}
For any $k \in \R$, the following holds: 
\begin{align}
\GRR_{k\tau_{1},...,k\tau_{N}}(\Y) = \GRR_{\tau_{1},...\tau_{N}}(\Y)
\end{align}
\begin{proof}
If we define $\mathbf{B}$ as before, and $\mathbf{B'}$ as follows:
\begin{align}
\mathbf{B'} &=\{B'|\GRF_{k\tau_1,...,k\tau_N}(B')=\Y\}
\end{align}
For any $B\in \mathbf{B}$  it is clear that  $kB\in \mathbf{B'}$. On the other hand, for any $B'\in \mathbf{B'}$  we know that  $ B'/k\in \mathbf{B}$. By using the fact that $\rank(kB) = \rank(B)$, we can complete the proof.
\end{proof}
\end{thm:lemma}

Based on these lemmas and the fact that for any $i \in \{1,...,N-1\}$, there exist an $\epsilon_i$ which will satisfies the following equality:
\begin{align}
\label{op}
\GRR_{\tau_{1}^{\star},...,\tau_{i}^{\star}-\epsilon_i,...,\tau_{N}^{\star}}(\Y) = \GRR_{\tau_{1}^{\star},...\tau_{N}^{\star}}(\Y)
\end{align}
We can show that there exists a set of  $\epsilon_i$ $(i \in \{1,...,N-1\})$, that transform $(\tau_{1}^{\star},...\tau_{N}^{\star})$ in to $(\tau\textquotesingle_{1},...,\tau\textquotesingle_{N})$ with a set of linear combinations and a constant shift in the thresholds. In another word, it means we have $k_0,...,k_{N-1}$ in a way that (effect of constant shift will appear as the plus one in the inequality~\ref{thresh}):
\hspace{5mm}
$
T'=k_0T_0^{\star}+...+k_{N-1}T_{N-1}^{\star}
$, 
Where $T'=(\tau_{1}',...\tau_{N}')$, $T_0=(\tau_{1}^{\star},...\tau_{N}^{\star})$ and $T_i=(\tau_{1}^{\star},...,\tau_{i}^{\star}-\epsilon_i,...,\tau_{N}^{\star})$. Therefore, if we define $\mathbf{B_i}$ as follows:
\begin{align}
\mathbf{B_i}=\{B_i|\GRF_{{T_i}^{\star}}(B_i)=\Y\}
\end{align}
And considering the fact that:
\begin{align}
 \rank(k_0B_0+...+k_{N-1}B_{N-1})&\leq \sum_{j=0}^{N-1} \rank(k_{j}B_{j})\\
  &=\sum_{j=0}^{N-1} \rank(B_{j}) 
\end{align}
Finally, with Lemma~\ref{thm:thr} and equation~\ref{op} we can complete the theorem.
}
\end{proof}
\end{thm:thm}
This theorem shows that even though using a fixed set of thresholds is not optimal, the rank is still bounded in terms of $N$, and does not depend on the size of the matrix ($n$ or $m$). Other complementary lemmas are provided in appendix.
\begin{thm:rmk}
The upper bound in the theorem \ref{thm:thresh} matches the upper bound found in \citet{neumann16:what} for the case where $N=1$,
$
%\begin{align}
\GRR_{\tau '}(\Y) \leq  \GRR_{\tau^*}(\Y)+1
%\end{align}
$. %, derived in a different manner.
\end{thm:rmk}



% --------------------------------------------------------------
%%%%%------------------------------------------------
%%%%%-----------------------------------------------
\section{Comparing Generalized Round Rank to Linear Rank}
\label{sec:GRR vs LR}

Matrix factorization (MF) based on linear rank has been widely used in lots of machine learning problems like matrix completion, matrix recovery and recommendation systems. The primary advantage of matrix factorization is its ability to model data in a compact form. 
Being able to represent the same data accurately in an even more compact form, specially when we are dealing with high rank matrices, is thus quite important.
Here, we study specific aspects of exact and approximate matrix reconstruction with $\GRR$. 
In particular, we introduce matrices with high linear rank but low $\GRR$, and demonstrate the inability of linear factorization in approximating many low-$\GRR$ matrices.

%Identity and upper triangle are two examples of binary matrices with full Linear Rank. %On the other hand, we know the GRR of Identity matrix is equal to $2$ and upper %triangle matrix has GRR equal to $1$~\citep{neumann16:what}. In result, we use these %two structures to bring out the full potential of GRR in comparison to Linear Rank. %But first we present the following lemma:
\subsection{Exact Low-Rank Reconstruction}
To compare linear and $\GRR$ matrix factorization, here we identify families of matrices that have high (or full) linear rank but low (or constant) $\GRR$.
Such matrices demonstrate the primary benefit of GRR over linear rank: factorizing matrices using GRR can be significantly beneficial. % in many cases. 
%play a major role in specifying the advantage of $\GRR$ over linear rank.\sameer{remember the goal: show that there are many matrices for which the linear rank is lower bounded at a much higher value, while the GRR is constant (or low). \textcolor{red}{we know so little about GRR, for example for rounding-rank only if it is not nested its rounding rank is 1}}

As provided in \citet{neumann2015some} for round-rank (a special case of $\GRR$), $\GRR_{\taus}(\Y)\leq r(\Y)$ for any matrix $\Y\in\mathbb{V}^{n\times m}$. 
More importantly, there are many structures that lower bound the linear rank of a matrix.
For example, if we define the upper triangle number $n_U$ for matrix $\Y\in \mathbb{V}^{n\times n}$ as the size of the biggest square block which is in the form of an upper triangle matrix, then $
%\begin{align}
\rank(\Y)\geq n_U
%\end{align}
$. %\sameer{unclear.. and is this correct?}
If we define the identity number $n_I$ similarly, then
%for matrix $\Y\in \mathbb{V}^{n\times n}$ 
%as the size of the biggest square block which is in the form of an identity matrix, then 
$
%\begin{align}
\rank(\Y)\geq n_I
%\end{align}
$, %\sameer{again, unclear}. 
and similarly for matrices with a band diagonal submatrix. 
None of these lower bounds that are based on identity, upper-triangle, and band-diagonal structures apply to $\GRR$. 
In particular, as shown in \citet{neumann2015some}, identity matrices (of any size) have a constant round-rank of $2$, upper triangle matrices have round-rank of $1$, and band diagonal matrices have round-rank of $2$ (which also holds for GRR). 
Moreover, we provide another lower bound for linear rank of a matrix, which is again not applicable to $\GRR$.   
%\sameer{why are these corollaries? of the lemma?\textcolor{red}{done}}
\begin{thm:thm}
	If a matrix $\Y\in\R^{n\times m}$ contains $k$ rows, $k\leq n,k\leq m$, such that $R=\{Y_{R_1},...,Y_{R_k}\}$, two columns $C=\{j_0,j_1\}$, and:
	%\pouya{1) 2) 3)...in text}
	\begin{enumerate}[nosep]
	\item rows in $R$ are distinct from each other, i.e, $\forall i,i'\in R, \exists j,Y_{ij}\neq Y_{i'j}$,
	\item columns in $C$ are distinct from each other, i.e, $\exists i,Y_{ij_0}\neq Y_{ij_1}$, and
	\item matrix spanning $R$ and $C$ are non-zero constants, w.l.o.g. $\forall i\in R,Y_{ij_0}=Y_{ij_1}=1$,
	\end{enumerate}
	then $\rank(\Y)\geq k$. (See appendix for the proof)
	%\begin{proof}
	
	%See supplement.\sameer{brief sketch, but again, what does this give us? maybe consider omitting. \textcolor{red}{done}}
	\iffalse
		Let us assume $\rank(\Y)<k$, i.e. $\exists k'<k, \U\in\R^{k'\times n}, \V\in\R^{k'\times m}$ such that $\Y=\U^T\times\V$.
		Since the rows $R$ and the columns in $C$ are distinct, their factorizations in $\U$ and $\V$ have to also be distinct, i.e. $\forall i,i'\in R, i\neq i', \U_{i}\neq\U_{i'}$ and $\V_{j_0}\neq\V_{j_1}$.
		Furthermore, $\forall i,i'\in R, i\neq i', \not\exists a,\U_{i}=a\U_{i'}$ and $\not\exists a,\V_{j_0}=a\V_{j_1}$ for $a\neq0$, it is clear that $\U_i\cdot\V_{j_0}=\U_i\cdot\V_{j_1}=1$ (and similarly for $i,i'\in R$).

		Now consider a row $i\in R$. Since $\forall j\in C, \Y_{ij}=1$, then $\U_i\cdot\V_{j}=1$.
		As a result, $\V_j$ are distinct vectors that lie in the hyperplane spanned by $\U_i\cdot\V_j=1$.
		In other words, the hyperplane $\U_i\cdot\V_j=1$ defines a $k'$-dimensional hyperplane tangent to the unit hyper-sphere.

		Going over all the rows in $R$, we obtain constraints that $\V_j$ are distinct vectors that lie in the intersection of the hyperplanes spanned by $\U_i\cdot\V_j=1$ for all $i\in R$.
		Since all $\U_i$s are distinct, there are $k$ distinct $k'$-dimensional hyperplanes, all tangent to the unit sphere, that intersect at more than one point (since $\V_j$s are distinct).

		Since $k$ hyper-planes tangent to unit sphere can intersect at at most one point in $k'<k$ dimensional space, $\V_j$ cannot be distinct vectors. Hence, our original assumption $k'<k$ is wrong, therefore, $\rank(\Y)\geq k$.
    \fi
	%\end{proof}
\end{thm:thm}

So far, we provide examples of high linear-rank structures that do not impose any constraints on $\GRR$.
We now provide the following lemma that, in conjunction with above results, indicates that lower bounds on the linear rank can be really high for matrices if they contain low-GRR structures (like identity and upper-triangle), while the lower bound on $\GRR$ is low.
\begin{thm:lemma}
For any matrix $A$, if there exists a submatrix $A'$ in a way that $\rank(A')=R$ and $\GRR_{\taus}(A')=r$, then $\GRR_{\taus}(A)\geq  r$ and $\rank(A)\geq R$. 
%\end{align}
\begin{proof}
If we consider the linear rank as the number of independent row (column) of the matrix, consequently having a rank of $R$ for submatrix $A'$ means there exist at least $R$ independent rows in matrix $A$. Using this argument we can simply prove above inequalities.
\end{proof}
\end{thm:lemma}



% --------------------------------------------------------------
%%%%%------------------------------------------------
%%%%%-----------------------------------------------

\subsection{Approximate Low-Rank Reconstruction}

Apart from examples of high linear-rank matrices that have low GRR, we can further show that many of these matrices cannot even be \emph{approximated} by a linear factorization.
In other words, we show that there exist many matrices for which not only their linear rank is high, but further, that the linear rank approximations are poor as well, while their low GRR reconstruction is perfect.
In order to measure whether a matrix can be approximated well, we describe the notion of approximate rank (introduced by \citet{alon13:the-approximate}, we rephrase it here in our notation).
\begin{thm:def}
Given $\epsilon$, \textbf{approximate rank} of a matrix $\X$ is:\\
%\hspace{5mm}
$
\apprank(\X) = \min\{\rank(\X'): \X'\in\Real^{n\times m}, ||\X-\X'||^2_F\le\epsilon\}
$
\end{thm:def}
We extend this definition to introduce the generalized form of approximate rank as follows:
\begin{thm:def}
Given $\epsilon$ and a link function $\link$ (e.g. GRF), the \textbf{generalized approximate rank} of a matrix $\Y$ is defined as:
\hspace{5mm}
$%\begin{equation}
\apprank_\link(\Y) = \min\{\rank_\link(\Y'): \Y'\!\in\!\mathbb{V}^{n\times m}, ||\Y-\Y'||^2_F\le\epsilon\}
$. %\end{equation}
\end{thm:def}

% --------------------------------------------------------------

% --------------------------------------------------------------
For an arbitrary matrix, we can evaluate how well a linear factorization can approximate it using SVD, i.e.:
\begin{thm:thm}
\label{thm:pca}
For a matrix $\X=\U\Sigma\V^T$, where $\mathrm{diag}(\Sigma)$ are the singular values, and $\U$ and $\V$ are orthogonal matrices, then $\sum_{i=k+1}^n|\Sigma_{ii}|^2 = \min_{\Y,\rank(\Y)=k}||\X-\Y||^2_F$.
\begin{proof}
%cite or proof!
This was first introduced in ~\citet{eckart1936approximation}, and recently presented again in ~\citet{udell14:generalized}.
We omit the detailed proof, but the primary intuition is that the PCA decomposition minimizes the Frobenius norm, and $\Y=\U'\V'$, with $\U'=\U\Sigma^{\frac{1}{2}}$ and $\V'=\Sigma^{\frac{1}{2}}\V^T$.
\end{proof}
%\label{thm:pca}
\end{thm:thm}
%\item\textbf{Identity Function, $\Bm_\step\not\subset\Bm^\epsilon_\identity$ (approximately)}
For an arbitrary binary matrix $\Y$, recall that $\Round_{\tau=0}(\Y)$ is equal to $\text{sign-rank}(\Y)$. %$\rank_\step(\Y)$ (
Using above theorem, we want to show that there are binary matrices that cannot be approximated by low linear-rank matrices (for non-trivial $\epsilon$), but can be approximated well by low round-rank matrices. % 
Clearly, these results extend to ordinal matrices and their $\GRR$ approximations, the generalized form of binary case. %\sameer{why?.\textcolor{red}{I am not sure}}

Let us consider $\Y$, the identity binary matrix of size $n$, for which the singular values of $\Y$ are all $1$s. 
%~\footnote{Since $\mathrm{det}(\Y)$ is the product of the diagonal entries, $\mathrm{det}(\Y-\lambda\I)=(\Y_{00}-\lambda)(\Y_{11}-\lambda)\ldots=0$.}.
By using Theorem~\ref{thm:pca}, any linear factorization $\Y'$ of rank $k$ will have $||\Y-\Y'||_F^2\geq(n-k)$.
As a result, the identity matrix cannot be approximated by \emph{any} rank-$k$ linear factorization for $\epsilon<{n-k}$. 
On the other hand, such a matrix can be reconstructed exactly with
a rank $2$ factorization if using the round-link function, since $\text{round-rank}(\Y)=2$.
%This shows the need for a metric other than linear rank to represent high rank matrices with matrix factorization.
%
%
% --------------------------------------------------------------
%\subsection{Properties for Ordinal Matrices}
%
%\begin{enumerate}
%\item\textbf{Multi-Sigmoid Function, $\Bm_\mstep=\Bm^\epsilon_\msigmoid$ (approximately)}
%\end{enumerate}
%\textcolor{red}{explanation}
% --------------------------------------------------------------
% --------------------------------------------------------------
% --------------------------------------------------------------
%
%\subsection{Illustrative Examples}
%We illustrate this incapability of linear matrix factorization to approximate matrices . 
In Figure~\ref{fig:pca_plot}, we illustrate a number of other such matrices, i.e. they can be exactly represented by a factorization with $\GRR$ of $2$, but cannot be approximated by any compact linear factorization.
\begin{figure}
    \centering
        \includegraphics[width=0.7\columnwidth]{pca_plot.pdf}
    	\caption{Comparison of the optimal linear factorization approximation as the rank $k$ is varied for a number of matrices (of size $n\times n$), demonstrating that linear factorization is unable to approximate these matrices with low-rank. All of these matrices have a \emph{constant} generalized round-rank ($\leq 2$).}
    	\label{fig:pca_plot}
\end{figure}
\section{Matrix Completion with Generalized Round-Rank Factorization}
\label{sec:SC}

So far, we show that there are many matrices that cannot be represented compactly using conventional matrix factorization (linear), either approximately or exactly, whereas they can be reconstructed using compact matrices when using $\GRF$ as the link function.
In this section, we study properties of \emph{completion} of ordinal-valued matrices based on $\GRF$ (and the notion of rank from $\GRR$).
In particular, given a number of noise-free observations $\Omega$ from $\Y\in\{0,\ldots,N\}^{n\times m}$ and its $\GRR(\Y)=r, r\ll\min(n,m)$, the goal here is to identify $\U\in\Real^{n\times r},\V\in\Real^{m\times r}$ such that $\GRF(\U\V^T)$ completes the unobserved entries of $\Y$ accurately.

%We also present the theoretical pr


%\sameer{why? what does it mean here?\textcolor{red}{done}}
\subsection{Theoretical Results for Uniqueness}

Uniqueness in matrix completion is defined as the minimum number of entries required to recover the matrix $\Y$ with high probability, assuming that sampling of the set of observed entries is based on an specific distribution. 
%In recent years, there has been an increasing amount of literature on concept of uniqueness~\citep{eldar2012uniqueness,ge2016matrix} which shows the major role it plays in different machine learning problems like matrix completion and recommendation systems. 
%
To obtain uniqueness in $\GRR$ based factorization, we first need to introduce the interval matrix $\mathbf{\bar{X}}$. 
Based on definition of generalized round function ($\text{GRF}$) and a set of fixed thresholds, we define matrix $\mathbf{\bar X}$ to be a matrix with interval entries calculated based on entries of matrix $\Y$ and thresholds ($\tau_1,...\tau_N$). 
As an example, if an entry $\Y_{ij}$ is $k \in \{0,...,N\}$, $\mathbf{\bar X_{ij}}$ would be equal to the interval $[\tau_k,\tau_{k+1}]$. 
When entries of $\Y$ are equal to $0$ or $N$, w.l.o.g. we assume the corresponding entries in matrix $\mathbf{\bar{X}}$ are bounded. Thus, each one of matrix $\mathbf{\bar X}$'s entries must be one of the $N+1$ possible intervals based on $\text{GRF}$'s thresholds.
\begin{thm:def}
 A target matrix $\Y \in \{0,\ldots,N\}^{n \times m}$ with 1)~observed set of entries $\Omega=\{(i,j), \Y_{ij} \text{is observed}\}$, 2)~set of known thresholds ($\tau_1,...\tau_N$), and 3)~$\GRR_{\tau_1,...,\tau_N}(\Y)=r$, is called uniquely recoverable, if we can recover its unique interval matrix $\mathbf{\bar X}$ with high probability.    
\end{thm:def}

Similar to $\mathbf{\bar X}$, we introduce $\mathcal{X}^{\star}$ to be a set of all matrices that satisfy following two conditions: 1)~For the observed entries $\Omega$ of $\Y$, $\Y_{ij}=\text{GRF}_{\tau_1,...,\tau_N}(\X^{\star}_{ij})$, and 2)~linear rank of $\mathcal{X}^{\star}$ is $r$. % $\GRR_{\tau_1,...,\tau_N}(\Y)$. 
If we consider a matrix $\X \in \mathcal{X}^{\star}$ then for an arbitrary entry $\X_{ij}$ we must have $\X_{ij} \in \mathbf{\bar X}_{ij}$, where $\mathbf{\bar X}_{ij}$ is an interval containing $\X_{ij}$. %Now we are ready to define the concept of uniqueness in our model: 
Given a matrix $\X\in\mathcal{X}^{\star}$, the uniqueness conditions ensure that we would be able to recover $\mathbf{\bar X}$, using which we can uniquely recover matrix $\Y$. % and vice versa. 

%Finally, we find necessary condition for uniqueness of matrix $\bold{\bar X}$ in the case that we have already find . 

In the next theorems, we first find the necessary condition on the entries of matrix $\X$ for satisfying uniqueness of matrix $\Y$. Then, we derive the sufficient condition accordingly.
In our calculations, we assume the thresholds to be fixed and our target matrix $\Y$ be noiseless, and further, there is at least one observed entry in every column and row of matrix $\Y$. % there should be at least one observed entry.
%\sameer{unclear.. is $\X$ a set of matrices? why should $\X$ be unique? don't use lowercase for a matrix.\textcolor{red}{done}} 
%\sameer{explain assumptions before the theorem.\textcolor{red}{done}}
% --------------------------------------------------------------
% --------------------------------------------------------------

\begin{thm:thm} (Necessary Condition)
For a target matrix $\Y \in \mathbb{V}^{n \times m}$ with few observed entries and given $\GRR(\Y)=r$, we consider set of $\{\Y_{i_1j},...,\Y_{i_rj}\}$ to be the $r$ observed entries in an arbitrary column $j$ of $Y$. 
Given any matrix $\X \in \mathcal{X}^{\star}$, $\X=\U\times \V^{T}$, and taking an unobserved entry $\Y_{ij}$, we define $a_{i_kj}$ as: 
$
\U_i=\sum_{k=1}^{r} a_{i_kj} \U_{i_k} 
$, where $\U_d$ ($d \in \{1,...,n\}$) is the $d^\text{th}$ row of matrix $\U$ and $i_k$ represents the index of observed entries in $j$th column.
%(the reason behind eligibility of this definition is provided in supplement). 
Then, the necessary condition of uniqueness of $\Y$ is:
\begin{align}
\sum_{k=1}^{r}\left | a_{i_kj} \right | \leq  \epsilon\left(\frac{T_\text{min}}{T_\text{max}}\right)
\end{align}
Where $r=\GRR(\Y)$, $T_\text{min}$ and $T_\text{max}$ are the length of smallest and largest intervals and $\epsilon$ is a small constant. 
% with assumption that the right and left intervals $\text{GRF}$ do not grow to infinity
% \begin{proof}
% See supplement.%\textbf{}\sameer{need a few lines of text for a sketch. It is one of the main contributions}
% \end{proof}
\begin{proof}
We only provide a sketch of proof here, and include the details in the appendix. To achieve uniqueness we need to find a condition in which for any column of $\X$, by changing respective row of $\V$, while the value of observed entries stay in the respected intervals, the value of unobserved ones wouldn't change dramatically which result in moving to other intervals. To do so, we will calculate the maximum of the possible change for an arbitrary unobserved entry of column $j$ in matrix $\Y$. To calculate this maximum for any unobserved entry $\Y_{ij}$, we consider the row $\U_i$ as a linear combination of linearly independent rows of $\U$ (which are in respect to observed entries of $\Y$ in column $j$). Then, by finding the maximum possible change for observed entries in column $j$, based on their respective intervals, we find mentioned boundary for achieving the uniqueness.
\iffalse
To better understand the concept of uniqueness in $\GRR$ benchmark, let's first look at the uniqueness in fixed value linear matrix factorization. %.

 In fixed value matrix factorization, it is proved that to achieve uniqueness, we need at least $r=\rank(\X)$ observation in each column (other than the linearly independent columns). Therefore, if we decompose $\X$ as $\X=\U\V^T$, and decide to change only unobserved entries of $\X$ in column $j$ (in opposed to uniqueness), we need to change the $jth$ row of matrix $\V$. To do so, let's assume we change the $jth$ row to: $
[\V_{j1}+c_1,...,\V_{jr}+c_r]
$.
Now since we know $\rank(U)=r$ and assume the respective rows of $\U$ to observed entries of column $j$ in matrix $\X$ are independent (as a consequence of uniqueness), we can see that only possible value for $c_1,..., c_r$ that does not change the observed entries of $\X$ is equal to $0$ (using $\forall q\in\{1,..r\}, \sum_{j=1}^{r}U_{i_qj}\times C_j =0 $). %, which confirm the uniqueness.  
%\sameer{where did we "show" this?}

The biggest difference between factorization based on $\GRR$ and linear factorization is the fact that the observed entries of matrix $\X$ ($\Y=\GRF(\X)$) are not fixed in $\GRR$ version, and can change through the respective interval. In result, to achieve uniqueness we need to find a condition in which for any column of $\X$, by changing respective row of $\V$, while the value of observed entries stay in the respected intervals, the value of unobserved ones wouldn't change dramatically which result in moving to other intervals. To do so, we will calculate the maximum of the possible change for an arbitrary unobserved entry of column $j$ in matrix $\Y$.

Let's call the $r$ observed entries of column's $j$ of matrix $\Y$, $\Y_{i_{1}j},...,\Y_{i_{r}j}$. Similar to linear factorization, we assume that the respective rows of $\U$ to these entries are linearly independent. In result, if we represent the change in entries of $j^\text{th}$ rows of $\V$ by $c_i$, we should have:
\begin{align}
\begin{bmatrix}
\U_{i_1}\\ 
\vdots \\ 
\U_{i_r}
\end{bmatrix}
\times
\begin{bmatrix}
c_1\\ 
\vdots \\ 
c_r
\end{bmatrix}
=
\begin{bmatrix}
\epsilon_{i_1j}\\ 
\vdots \\ 
\epsilon_{i_rj}
\end{bmatrix}
\end{align}            
Where $\U_{i_k}$ is the $i_k th$ row of $\U$, and $\epsilon_{i_kj}$ is the possible change for $\X_{i_{k}j}$, based on the observed interval. Therefore:
\begin{align}
\epsilon_{i_kj} \in (\tau_{i_kj}\downarrow-\X_{i_kj},\tau_{i_kj}\uparrow-\X_{i_kj})=(\epsilon_{i_kj}^-,\epsilon_{i_kj}^+)
\end{align}
Where $\tau_{i_kj}\downarrow$ and $\tau_{i_kj}\uparrow$ are lower bound and upper bound of respective interval of $\X_{i_kj}$ calculated based on $\Y_{i_kj}$.
Now let's assume we want to find the maximum possible change for $\X_{sj}$ considering that $\Y_{sj}$ is an unobserved entry. Since $\U_{i_k}$'s are independent, there exist $a_1,..a_r$ such that
$
\U_s=\sum_{k=1}^{r}a_{i_kj}\U_{i_k}
$.
Therefore, we can show the change in entry $\X_{sj}$ as
$
A=\sum_{k=1}^{r}a_{i_kj}\epsilon_{i_kj}
$.
In result, for the maximum possible change we have:
\begin{align}
\max|A|=\max(\sum_{k=1}^{r}a_{i_kj}\epsilon_{i_kj}^{\text{sign}(a_{i_kj})},|\sum_{k=1}^{r}a_{i_kj}\epsilon_{i_kj}^{-\text{sign}(a_{i_kj})}|)
\end{align}
Where $\text{sign}(.)$ is the sign function. On the other hand: % we know:
\begin{align}
\sum_{k=1}^{r}a_{i_kj}\epsilon_{i_kj}^{\text{sign}(a_{i_kj})}+|\sum_{k=1}^{r}a_{i_kj}\epsilon_{i_kj}^{-\text{sign}(a_{i_kj})}|=\sum_{k=1}^{r}|a_{i_kj}|T_{i_kj}
\end{align}
\begin{align}
\Rightarrow \max|A| \geqslant \frac{1}{2}\sum_{k=1}^{r}|a_{i_kj}|T_{i_kj}
\end{align}
where $T_{i_kj}$ is the length of $\bold{\bar X}_{i_kj}$ (an interval entry). Clearly, to achieve the uniqueness we need $\max|A|\leq T_{sj}$. But, since the entry $\Y_{sj}$ is unobserved we do not know the value of $T_{sj}$. In result, for uniqueness in the worst case we need:
\begin{align}
\sum_{k=1}^{r}|a_{i_kj}|T_{\max} &\leq \epsilon T_{\min}\\
\Rightarrow\sum_{k=1}^{r}|a_{i_kj}| &\leq\epsilon \frac{T_{\min}}{T_{\max}}
\end{align} 
Where $T_{\min}$ and $T_{\max}$ are the smallest and the biggest interval, and $\epsilon$ is a small real constant. 
\fi
 \end{proof}
\end{thm:thm}
The same condition is necessary for matrix $\V$ as well. The necessary condition must be satisfied for all columns of matrix $\X$. 
Moreover, if the necessary condition is not satisfied, we cannot find a unique matrix $\X$, and hence a unique completion, i.e. $\Y=\text{GRF}_{\tau_1,...,\tau_N}(\X)$ where $\X \in \mathcal{X}^{\star}$.
%\sameer{put something like this above. \textcolor{red}{done}} 

%\begin{thm:rmk}
%We define sample complexity on matrix $\X$  as the minimum number of required entries to complete matrix $\X$ uniquely. As a result of $\text{GRF}$ definition, we know that each one of matrix $\X$'s entries must be chosen from one of the $N+1$ intervals, corresponding to the $\text{GRF}$'s thresholds.\sameer{what is GRF?\textcolor{red}{done}}

%Consequently, in deriving uniqueness necessary condition, we assumed the sample complexity of matrix $\X\in \bold{\bar X}$ to be equal to the case when we are dealing with fixed value matrix completion on a matrix with rank $r$ ($\GRR(Y)=r$). Based on information- theoretic sample complexity of $O(nR\log n)$ ~\citep{balcan2017optimal} for fixed value matrix completion (where $R$ is the linear rank of fixed value matrix), we can achieve sample complexity of $O(nr\log n)$ instead of $O(nR\log n)$, where $r$ is the $\GRR(Y)$ and $R$ is the linear rank of $Y$ ( if $r\ll R$ this will result in far better sample complexity). Note that, with more samples in each column the right side of necessary condition will become larger.
%\end{thm:rmk}
\begin{thm:thm} (Sufficient Condition)
Using above necessary condition, for any unobserved entry $\Y_{ij}$ of matrix $\Y$ we define $\bar\epsilon$ as minimum distance of $\X_{ij}$ with its respected interval's boundaries. Then, we will have the following inequality as sufficient condition of uniqueness:
\begin{align}
\bar\epsilon \geq \max \left(\sum_{k=1}^{r}a_{i_kj}\epsilon_{i_k}^{\text{sign}(a_{i_kj})} , \left | \sum_{k=1}^{r}a_{i_kj}\epsilon_{i_k}^{-\text{sign}(a_{i_kj})}  \right |\right)
\end{align}
where $r$ and $a_{i_kj}$ are defined as before, $\epsilon_{i_kj}^{+}$ is defined as the distance of $\X_{i_kj}$ to its upper bound, and $\epsilon_{i_kj}^{-}$ is defined as negative of the distance of $\X_{i_kj}$ to its lower bound.
% \begin{proof}
% See supplement.
% \end{proof}
\end{thm:thm}
Above sufficient condition is a direct result of necessary condition proof. %(see supplement). 
Although not tight, it guarantees the existence of unique $\mathbf{\bar X}$, and thus the complete matrix $\Y$.
% --------------------------------------------------------------
% --------------------------------------------------------------


\subsection{Gradient-Based Algorithm for $\GRR$ Factorization}

Although previous studies have used many different paradigms for matrix factorization, such as alternating minimization \citep{hardt2014understanding,jain2013low} and adaptive sampling~\citep{krishnamurthy2013low}, 
stochastic gradient descent-based (SGD) approaches have gained widespread adoption, in part due to their flexibility, scalability, and theoretical properties~\citep{de2014global}.
For linear matrix factorization, a loss function that minimizes the squared error is used, i.e. $L_{\text{linear}}=\sum (Y_{ij}- U_iV_j)^2$, where the summation is over the observed entries.
In order to prevent over-fitting, $L_2$ regularization is often incorporated.

% and \citet{marevcek2017matrix} based on interval matrix factorization. In this section we provide a novel algorithm for completion of matrix $Y \in R^{n \times m}$ given a few ones of its entries. The inputs to our algorithm are few samples from matrix $\Y$, a given rank for low-rank approximation and size of the matrix $\Y$. We have three cost functions in our algorithm which the $\sum$ in them are over observed entries and we do optimization based on Gradient Descent (GD) method: 

%1) $L_{\text{linear}}=\sum (y_{ij}- U_iV_j)^2$ which is traditional linear matrix completion cost function with Identity as the link function and gives us the \emph{Linear} result. Note that, this function has a regularization term as well. In previous studies affect of different type of method as GD, SGD with different batch size and ... have been investigated on this cost function. 
%\sameer{mention regularization here, \textcolor{red}{what other method (GD, SGD, ...)?}}

\para{Round}
We extend this framework to support $\GRR$-based factorization by defining an alternate loss function. 
In particular, with each observed entry $Y_{ij}$ and the current estimate of $\taus$, we compute the $b_{ij}^\downarrow$ and $b_{ij}^\uparrow$ as the lower and upper bounds for $X_{ij}$ with respect to the $\GRF$.
Given these, we use the following loss, $L_{\text{Round}}=\sum (b_{ij}^\downarrow - U_iV_j)_++( U_iV_j - b_{ij}^\uparrow)_+$, where $(.)_+=\max(.,0)$. Considering the regularization term as well, we apply stochastic gradient descent as before, computing gradients using a differentiable form of $\max$ with respect to $\U$, $\V$, and $\taus$.
%and $b_{ij}\downarrow$ and $b_{ij}\uparrow$ are lower bound and upper bound of observed entry $y_{ij}$ with respect to the $\GRR$ thresholds. Link function in this case is $\GRF$. 

\para{Multi-Sigmoid}
Although the above loss captures the goal of the $\GRR$-based factorization accurately, it contains both discontinuities and flat regions, and thus is difficult to optimize.
Instead, we also propose to use a smoother and noise tolerant approximation of the $\GRF$ function.
The sigmoid function, $\sigma(x)=\frac{1}{1+e^{-x}}$, for example, is often used to approximate the $\text{sign}$ function.
When used as a link function in factorization, we can further show that it approximates the $\text{sign-rank}$ well. 
\begin{thm:thm}
For any $\epsilon>0$ and matrix $\Y$, $\text{sign-rank}(\Y) = \apprank_\sigmoid(\Y)$. (See appendix for the proof)

\iffalse
\begin{proof}
By introducing $\Bm_\sigmoid^\epsilon(k)=\{\B\in\{0,1\}^{n\times m}; \apprank_\sigmoid(\B)=k\}$, i.e. the set of binary matrices whose $\apprank_\sigmoid$ is $k$, We prove the theorem by showing both directions.
$\Bm_\step\subseteq\Bm_\sigmoid$:
Any $\U,\V$ that works for $\step$ should work with $\sigmoid$ if multiplied by a very large number, i.e. take a sufficiently large $\eta$, and $\U_\sigmoid=\eta\U_\step,\V_\sigmoid=\eta\V_\step$.
Then, $\X_\sigmoid=\eta^2\X_\step$ and if we set $\lparams_\sigmoid=\eta^2\lparams_\step$, then $(\X_\sigmoid-\lparams_\sigmoid)=\eta^2(\X_\step-\lparams_\step)$, therefore will have the same sign, and $\Y_\sigmoid=\sigmoid(\X_\sigmoid)$ will be arbitrarily close to $0$ and $1$ in $\Y_\step$.

$\Bm_\sigmoid\subseteq\Bm_\step$:
Any $\U,\V$ that works for $\sigmoid$ will directly work with $\step$.
\end{proof}
\fi
\end{thm:thm}
We can similarly approximate $\GRF$ using a sum of sigmoid functions that we call $\text{Multi-sigmoid}$ defined as $\link^\msigmoid_{\taus}(x)=\sum_{d=1}^{N}\sigma(x-\tau_d)$, for which the above properties also hold.
The resulting loss function that minimizes the squared error is $L_{\text{multi-sigmoid}}=\sum(Y_{ij}-\link^\msigmoid_\taus(U_iV_j))^2$.

In our experiments, we evaluate both of our proposed loss functions, and compare their relative performance.
%3)  which is another one of our defined objective functions to approximate generalized matrix rank and gives us the \emph{Multi-Sigmoid} result, where $\link^\msigmoid$ is Multi-Sigmoid link function, a differentiable approximation of $\text{GRF}$.\sameer{more details here.\textcolor{red}{don}} The reason behind introducing $\text{Multi-sigmoid}$ cost function, is the fact that $\text{Round}$ cost function is discontinues and difficult to optimize. In result, we provide a more smooth and noise tolerant approximation of $\text{Round}$ function.
%To better understand the connection of Round function with its Sigmoid approximation, we provide following theorem. Accordingly, for an arbitrary matrix $\Y$, we represent $\text{rounding-rank}_{\tau=0}(Y)$ as $\rank_\step(\Y)$ and capture its relation to the approximate rank in the following theorem: 
%
%\para{Implementation Details}
We study variations in which the thresholds $\taus$ are either pre-fixed or updated (using $\frac{\partial}{\partial\taus}L$) during training.
All the parameters of the optimization, such as learning rate and early stopping, and the hyper-parameters of our approaches, such as regularization, are tuned on validation data.


%%%%%------------------------------------------------

%%%%%------------------------------------------------
% --------------------------------------------------------------
%\section{\textcolor{red}{Sample Complexity}}
%\label{sec:SC}

%In this section we study the Sample complexity of Generalized Round Rank Matrix Completion. Since we are going to present our main algorithm base on completion of matrix $X$ rather than matrix $Y$ in Generalized Matrix Rank frame work we first need to define Sample complexity on matrix $X$. In result of computing matrix $X$ by applying the Generalized Round function concept on the matrix $Y$, each one of its enteries are going to be one of the $N+1$ intervals, corresponding to the Generalized Round Function's thresholds. In result, Sample Complexity for Matrix $X$ defines as number of required enteires to complete matrix $X$ Uniquely.

%To find a lower bound for sample complexity of Matrix $X$, we need to prove at least one of the following propositions:

%\begin{thm:pro}
%The following upper bound holds for the possible number of matrix $X$ (Totally there exist $(N+1)^{n\times m}$ possibility for matrix $X$ base on the thresholds) which there exist a choice of fixed value from the intervals for each entery in a way that resulted matrix has Linear rank of $r$(GRR of $Y$)(\textcolor{red}{$nf(r)<<n^2$}):
%\begin{align}
%|N| \leq a(N+1)^{2nr-r^2}
%\end{align}
%Where a is a constatn value. If this Proposition holds, base on Fano's inequality argument we can find a lower bound for Sample complexity.
%\end{thm:pro}

%\begin{thm:pro}
%Degree of Freedom of interval matrix $X$ is in the order of Degree of freedom of a matrix which made by choosing a fixed value from the intervals in the matrix $X$ and has the Linear rank of $r$(\textcolor{red}{$O(nf(r))$}). To prove this Proposition we can use Reductio ad absurdum argument and consider Sample Complexity in simple case of Sign Rank .
%\end{thm:pro}

%Along the same lines, we can impracticaly shows the superiority of Sample Complexity of GMR comparing to traditional matrix completion by considering basic matrices like Upper Triangle, Identity, Diagonal .... We can see the result in the $Figure 2$


% --------------------------------------------------------------
% --------------------------------------------------------------
%\section{Optimization}
%\label{sec:algo}


% --------------------------------------------------------------
% --------------------------------------------------------------
%\subsection{Gradient Descent Algorithm}

%\begin{align}
%\Y_{ij} &= \sum_{d=1}^D\psi(\U_i\V_j-\lparams_d)\\
%\mathcal{L}(y,\Y_{ij}) &= (y-\Y_{ij})^2 + \beta||\U_i||^2_2 + \beta||\V_j||^2_2\\
%\frac{\partial}{\partial\U_i}\mathcal{L}(y,\Y_{ij}) &= -2(y-\Y_{ij})\frac{\partial}{\partial\U_i}\Y_{ij} + 2\beta\U_i\\
%\frac{\partial}{\partial\U_i}\Y_{ij} &= \sum_{d=1}^D\frac{\partial}{\partial\U_i}\psi(\U_i\V_j-\lparams_d)
 %=\V_j\sum_{d=1}^D\psi'(\U_i\V_j-\lparams_d)\\
%\frac{\partial}{\partial\V_j}\Y_{ij} &= \sum_{d=1}^D\frac{\partial}{\partial\V_j}\psi(\U_i\V_j-\lparams_d) =
 %\U_i\sum_{d=1}^D\psi'(\U_i\V_j-\lparams_d)\\
%\frac{\partial}{\partial \lparams_d}\Y_{ij} &= \frac{\partial}{\partial \lparams_d}\psi(\U_i\V_j-\lparams_d) = -\psi'(\U_i\V_j-\lparams_d)
%\end{align}

% --------------------------------------------------------------
% --------------------------------------------------------------
% \subsection{Open: Convergence}

% \begin{align}
% \rank_\link(\Y) = \min_{\X\in\R^{n\times m}, \lparams\in\Lparams} \left\{ \rank(\X); \Y = \link_\lparams(\X) \right\}
% \end{align}

% Trace norm

% Nuclear norm approximation, prove convexity?

% Sample complexity?

% \begin{align}
% (\U_\link,\V_\link) = \underset{\U\in\R^{n\times k},\V\in\R^{m\times k}}{\arg\min} \min_{\lparams\in\Lparams} ||\Y-\link_\lparams(\U^T\V)||
% \end{align}

% ALS convergence proof?

% Stochastic gradient descent algorithms

% % --------------------------------------------------------------
% % --------------------------------------------------------------
% \subsubsection{Open: Sample Complexity}

% Matrix factorization sample complexity proof.

% However, this doesn't hold for matrices with link functions since they are full rank?

% --------------------------------------------------------------
% --------------------------------------------------------------

% --------------------------------------------------------------
% --------------------------------------------------------------
\section{Experiments}
\label{sec:results}

In this section we evaluate the capabilities of our proposed $\GRR$ factorization relative to linear factorization first through variety of simulations, followed by considering \emph{smallnetflix} and \emph{MovieLens 100K}\footnote{The codes available at: \url{https://github.com/pouyapez/GRR-Matrix-Factorization}} datasets. 
Unless otherwise noted, all of evaluations are based on Root Mean Square Error (RMSE).

\paragraph{Matrix Recovery}
%\sameer{why is this important?\textcolor{red}{I am not sure}}
We first consider the problem of recovering a fully known matrix $\Y$ from its factorization, thus all entries are considered observed. 
We create three matrices in order to evaluate our approaches for recovery:
%based on prior full knowledge of all the entries. 
%For this task, we consider 
(a)~Random $10\times10$ matrix with $N=5$ that has $\GRR\leq 2$ (create by randomly generating $\taus$, $\U$, and $\V$),
%Where $T$ is a random set of thresholds with the size of $5$, and $U$ and $V$ are $10\times 2$ randomly chosen matrices. Clearly, the $\GRR$ of matrix $A$ is less than or equal to 2.
(b)~Binary upper triangle matrix with size 10 ($\GRR$ of 1), and 
(c)~Band-diagonal matrix of size 10 and bandwidth 3, which has the linear rank of $8$ and $\GRR$ of 2.
%
Figure~\ref{fig:fullrec} presents the RMSE comparison of these three matrices as training progresses. For the upper triangle and the band diagonal, we fix threshold to $\tau=0.5$. The results show that Round works far better than others by converging to zero.  Moreover, linear approach is outperformed by the Multi-sigmoid without fixed thresholds in all, demonstrating it cannot recover even simple matrices. %, and by both  fixed threshold in band-diagonal. % and by fixed threshold Round in Upper Triangle and Band Diagonal.


% --------------------------------------------------------------
% --------------------------------------------------------------
\begin{figure*}[tb]
	\centering
	\begin{subfigure}{0.31\textwidth}
	  \centering
	  %\missingfigure{}
	  \includegraphics[width=\textwidth]{Random.pdf}
	  \caption{Random matrix, k=2}
	  %\label{fig:synth:samples}
		\end{subfigure}
		\quad
		\begin{subfigure}{0.31\textwidth}
	  	\centering
	  	%\missingfigure{}
	  	\includegraphics[width=\textwidth]{uppertriangle.pdf}
	  	\caption{Upper Triangle matrix, k=1}
	 	 %\label{fig:synth:samples}
		\end{subfigure}
		\quad
		\begin{subfigure}{0.31\textwidth}
	  \centering
	  %\missingfigure{}
	  \includegraphics[width=\textwidth]{banddiagonal.pdf}
	  \caption{Band Diagonal matrix, k=2}
	\end{subfigure}
	\caption{\textbf{Matrix Recovery:} Synthetic matrices that are reconstructed using their $k$-dimensional factorization with different representations. We plot RMSE of the reconstruction vs the number of training iterations, demonstrating the efficiency of $\GRR$-based methods, especially without fixed thresholds.}
	\label{fig:fullrec}
\end{figure*}
% --------------------------------------------------------------
% --------------------------------------------------------------

\paragraph{Matrix Completion}
%\sameer{why is this important?\textcolor{red}{I am not sure}}
Instead of fully-observed matrices, we now evaluate completion of the matrix when only a few of the entries are observed. 
%Matrix Completion is the problem of recovering a target matrix $\Y$, based on low-rank approximation, from partially observed entries. 
We consider $50\times50$ upper-triangle and band-diagonal (bandwidth $10$) matrices, and sample entries from them, to illustrate how well our approaches can complete them. 
Results on held-out 20\% entries are given in Tables~\ref{tab:MC-UT} and \ref{tab:MC-BD}. 
%For upper triangle we consider $k=1$ and find the RMSE after convergence (around 5000 iteration). 
%On the other hand, for band diagonal we set $k=2$. 
In addition, we build a random matrix with size 50 and $\GRR$ 2, and present the results for this matrix in Table ~\ref{tab:MC-Ra}. 
As we can see, linear factorization in all three cases is outperformed by our proposed approaches. 
In band-diagonal, because of over-fitting of the Round approach, Multi-sigmoid performs a little better, and for upper-triangle, we achieve the best result for Round method by fixing $\tau=0.5$. % for Round method .

\paragraph{Matrix Completion on Real Data}
%
In this section we use the \emph{smallnetflix} movie ratings data for $95526$ users and $3561$ movies, where the training dataset contains $3,298,163$ ratings and validation contains $545,177$ ratings, while each one of ratings is an integer in $\{1,2,3,4,5\}$. % and we try to complete a $95526 \times 3561$ users vs movies' matrix. %\footnote{\url{http://www.select.cs.cmu.edu/code/graphlab/datasets/}}
We also evaluate on a second movie recommendation dataset, \emph{Movielens 100k}, with $100,000$ ratings from $1000$ users on $1700$ movies, with the same range as \emph{smallnetflix}.  
For this recommendation systems, in addition to RMSE, we also consider the notion of accuracy that is more appropriate for the task, calculated as the fraction of predicted ratings that are within $\pm0.5$ of the real ratings. As shown in Figure~\ref{fig:smallnet}, for \emph{smallnetflix}, linear factorization is better than Round approach from RMSE perspective, probably because linear is more robust to noise. On the other hand, Multi-sigmoid achieves better RMSE than linear method. Furthermore, both Round and Multi-sigmoid outperform the linear factorization in accuracy. \emph{Movielens} results for the percentage metric shows similar behavior as \emph{smallnetflix}, demonstrating that $\GRR$-based factorization can provide benefits to real-world applications. Furthermore, a comparison of our models with existing approaches on \emph{Movielens} dataset is provided in Table~\ref{tab:Ml}. We choose the RMSE result for smallest $k$ presented in those works. As we can see, our Multi-sigmoid method appear very competitive in comparison with other methods, while our Round approach result suffer from existence of noise in the dataset as before.   

%%%--------------------------------------------------------------------------------------------------------------------
\begin{table*}[tb]
	\centering
	\caption{Matrix completion for \textbf{Upper Triangular Matrices} ($k=1$)}
	\label{tab:MC-UT}
\begin{tabular}{lcccccccc}
\toprule
%&\multicolumn{8}{c}{Upper triangle} \\ \hline
Proportion of Observations& 10\%& 20\%& 30\%& 40\%& 50\%& 60\%& 70\%& 80\% \\ %\hline
\midrule
Linear&0.50&0.50&0.50&0.50&0.50&0.50&0.50&0.50\\ %\hline
\addlinespace
Multi-Sigmoid&0.51&0.30&0.25&0.25&0.26&0.25&0.23&0.23\\ %\hline
%Multi-Sigmoid ($\tau=0$)&0.55&0.41&0.41&0.39&0.39&0.38&0.38&0.39\\ %\hline
Multi-Sigmoid, $\tau=0.5$&0.58&0.37&0.36&0.36&0.35&0.35&0.34&0.34\\ %\hline
Round&0.46&0.34&0.27&0.25&0.26&0.21&0.20&0.16\\ %\hline
%Round ($\tau=0$)&0.71&0.69&0.73&0.72&0.72&0.69&0.67&0.70\\ %\hline
Round, $\tau=0.5$&\bf{0.38}&\bf{0.26}&\bf{0.23}&\bf{0.19}&\bf{0.15}&\bf{0.13}&\bf{0.15}&\bf{0.13}\\ %\hline
\bottomrule
\end{tabular}
\end{table*}
%%%--------------------------------------------------------------------------------------------------------------------
%%%--------------------------------------------------------------------------------------------------------------------
\begin{table*}[tb]
	\centering
	\caption{Matrix completion for \textbf{Band Diagonal Matrices} ($k=2$)}
	\label{tab:MC-BD}
\begin{tabular}{lcccccccc}
\toprule
%&\multicolumn{8}{l|}{Band Diagonal} \\ \hline
Proportion of Observations& 10\%& 20\%& 30\%& 40\%& 50\%& 60\%& 70\%& 80\% \\ \midrule
Linear&0.49&0.46&0.46&0.46&0.46&0.46&0.46&0.46\\ %\hline
\addlinespace
Multi-Sigmoid&\bf{0.39}&\bf{0.26}&\bf{0.23}&\bf{0.23}&\bf{0.22}&\bf{0.21}&\bf{0.20}&\bf{0.20}\\ %\hline
Multi-Sigmoid, $\tau=0.5$&0.48&0.49&0.33&0.31&0.30&0.29&0.29&0.29\\ %\hline
%Multi-Sigmoid ($\tau=0$)&0.57&0.35&0.36&0.33&0.35&0.32&0.32&0.32\\ %\hline
Round&0.71&0.41&0.35&0.29&0.29&0.27&0.23&0.22\\ %\hline
%Round ($\tau=0$)&0.72&0.70&0.69&0.69&0.69&0.68&0.70&0.71\\ %\hline
Round, $\tau=0.5$&0.61&0.57&0.39&0.52&0.58&0.30&0.29&0.34\\ \bottomrule
\end{tabular}
\end{table*}
%%%--------------------------------------------------------------------------------------------------------------------
%%%--------------------------------------------------------------------------------------------------------------------
\begin{table*}[tb]
	\centering
	\caption{Matrix completion with different number of samples for \textbf{Random low-GRR Matrices}}
	\label{tab:MC-Ra}
\begin{tabular}{lcccccccc}
\toprule
%&\multicolumn{8}{l|}{Random Matrix} \\ \hline
Proportion of Observations& 10\%& 20\%& 30\%& 40\%& 50\%& 60\%& 70\%& 80\% \\ \midrule
Linear&1.73&1.06&0.97&0.90&0.85&0.85&0.87&0.83\\ %\hline
\addlinespace
Multi-Sigmoid&1.92&\bf{0.53}&\bf{0.48}&\bf{0.42}&\bf{0.39}&\bf{0.38}&0.36&0.35\\ %\hline
Multi-Sigmoid (Fixed $\tau$)&1.96&1.54&1.37&1.32&1.29&1.28&1.25&1.23\\ %\hline
Round&\bf{1.49}&0.92&0.60&0.48&0.48&0.39&\bf{0.30}&\bf{0.28}\\ %\hline
Round (Fixed $\tau$)&2.44&1.50&1.50&1.43&1.36&1.39&1.44&1.34\\ \bottomrule
\end{tabular}
\end{table*}
%%%--------------------------------------------------------------------------------------------------------------------

% --------------------------------------------------------------
\begin{figure*}[tb]
	\centering
	\begin{subfigure}{0.30\textwidth}
		\centering
		%\missingfigure{}
		\includegraphics[width=\textwidth]{smallnetflix-percentage-log.pdf}
		\caption{Percentage, smallnetfix}
		%\label{fig:synth:samples}
	\end{subfigure}
	%\quad
	\begin{subfigure}{0.32\textwidth}
		\centering
		%\missingfigure{}
		\includegraphics[width=\textwidth]{smallnetflix-rmse-log.pdf}
		\caption{RMSE, smallnetflix}
		%\label{fig:synth:samples}
	\end{subfigure}
	%\quad
	\begin{subfigure}{0.32\textwidth}
		\centering
		%\missingfigure{}
		\includegraphics[width=\textwidth]{movielens-percentage-log.pdf}
		\caption{Percentage, movielens}
		%\label{fig:synth:samples}
	\end{subfigure}
	
	\caption{Performance on recommendation datasets, as $k$ in increased}
	\label{fig:smallnet}
\end{figure*}

%%%_____________________________________________________________________________
%%%--------------------------------------------------------------------------------------------------------------------
\begin{table*}[tb]
	\centering
	\caption{RMSE on Movielens-100k for a variety of models with different low-rank approximation (k).}
	\label{tab:Ml}
\begin{tabular}{lcc}
\toprule
%&\multicolumn{8}{l|}{Random Matrix} 
Models&Low-rank approximation&RMSE\\ \hline
APG~\citep{kwok2015accelerated}& k=70 & 1.037\\
AIS-Impute~\citep{kwok2015accelerated}& k=70 & 1.037\\
CWOCFI~\citep{lu2013second}& k=10 & 1.01\\
our Round& k=10 & 1.007\\
our Linear& k=10 & 0.995\\
UCMF~\citep{zhang2014information}& - & 0.948\\
our Multi-sigmoid& k=10 & 0.928\\
SVDPlusPlus~\citep{gantner2011mymedialite}& k=10 & 0.911\\
SIAFactorModel~\citep{gantner2011mymedialite}& k=10 & 0.908\\
GG~\citep{lakshminarayanan2011robust}& k=30 & 0.907\\
\bottomrule
\end{tabular}
\end{table*}
%%%-------------------------------------------------------------

%\begin{outline}
%\1 Methods
%	\2 linear
%	\2 linear with dynamic biases
%	\2 sum of sigmoid with fixed biases
%	\2 sum of sigmoid with dynamic biases
%\1 Measures
%	\2 mean absolute error
%	\2 RMSE
%\end{outline}

% --------------------------------------------------------------
% --------------------------------------------------------------
%\subsection{Synthetic Experiments}

%\begin{outline}
%\1 Rank Dependency
%	\2 Plot of reconstruction error of the various approaches as the rank is varied.
%\1 Sample Complexity
%	\2 Plot of reconstruction error of the various approaches as the number of observed samples is varied.
%\end{outline}

% --------------------------------------------------------------
% --------------------------------------------------------------
%\subsection{Collaborative Filtering}

%\begin{outline}
%\1 Data sets
%	\2 MovieLens 1m
%	\2 Netflix
%	\2 Amazon
%\1 Experiments
%	\2 For a low k, ordinal is better than linear
%	\2 ? dependence on rank
%	\2 ? sample complexity
%\end{outline}

% --------------------------------------------------------------
% --------------------------------------------------------------
%\subsection{Topic Modeling}

%\begin{outline}
%\1 Model
%\1 Datasets
%	\2 4 UCI BoWs datasets
%	\2 ? Reuters
%	\2 ? NYT
%\1 Experiments
%	\2 For a low k, ordinal is better than linear
%	\2 ? dependence on rank
%	\2 ? sample complexity
%\end{outline}


% --------------------------------------------------------------
\section{Related Work}
\label{sec:related}

There is a rich literature on matrix factorization and its applications. 
To date, a number of link functions have been used, along with different losses for each, however here we are first to focus on expressive capabilities of these link functions, in particular of the ordinal-valued matrices~\citep{singh08:a-unified,koren2011ordrec,paquet2012hierarchical,udell14:generalized}.
\citet{nickel13:logistic} addressed tensor factorization problem and showed improved performance when using a sigmoid link function. 
\citet{marevcek2017matrix} introduced the concept of matrix factorization based on interval uncertainty, which results in a similar objective as our algorithm. 
However, not only is our proposed algorithm going beyond by updating the thresholds and supporting sigmoid-based smoothing, but we present results on the representation capabilities of the round-link function. %\sameer{TODO}. 

A number of methods have approached matrix factorization from a probabilistic view, primarily describing solutions when faced with different forms of noise, resulting, interestingly, in link functions as well.
\citet{collins01:a-generalization} introduced a generalization of PCA method to loss function for non real-valued data,
such as binary-valued.
\citet{salakhutdinov08:bayesian} focused on Bayesian treatment of probabilistic matrix factorization, identifying the appropriate priors to encode various \emph{link} functions. % in which model capacity is controlled automatically by integrating over all model parameters and hyperparameters. 
On the other hand, \citet{lawrence09:non-linear} have analyzed non-linear matrix factorization based on Gaussian process and used SGD to optimize their model. 
However, these approaches do not explicitly investigate the representation capabilities, in particular, the significant difference in \emph{rank} when link functions are taken into account.

Sign-rank and its properties have been studied by \citet{nickel14:reducing,bouchard15:on-approximate,davenport14:1-bit}, and more recently, \citet{neumann2015some} provides in-depth analysis of round-rank. 
Although these have some similarity to $\GRR$, sign-rank and round-rank are limited to binary matrices, while $\GRR$ is more suitable for most practical applications, and further, we present extension of their results in this paper that apply to round-rank as well. 
Since we can view matrix factorization as a simple neural-network, research in understanding the complexity of neural networks~\citep{huang03:learning}, in particular with rectifier units~\citep{pan2016expressiveness}, is relevant, however the results differ significantly in the aspects of representation we focus on.
\section{Conclusions and Future Work}
\label{sec:conclusions}

In this paper, we demonstrated the expressive power of using link functions for matrix factorization, specifically the generalized round-rank ($\GRR$) for ordinal-value matrices.
We show that not only are there full-rank matrices that are low $\GRR$, but further, that these matrices cannot even be approximated by low linear factorization. 
Furthermore, we provide uniqueness conditions of this formulation, and provide gradient descent-based algorithms to perform such a factorization. % which are major theoretical properties of matrix completion. 
We present evaluation on synthetic and real-world datasets that demonstrate that $\GRR$-based factorization works significantly better than linear factorization: converging faster while requiring fewer observations. %Moreover,
In future work, 
%There are a number of avenues for future work.
we will investigate theoretical properties of our optimization algorithm, in particular explore convex relaxations to obtain convergence and analyze sample complexity.
We are interested in the connection of link-rank with different probabilistic interpretations, in particular, robustness to noise. %, specifically considering the approximation properties presented here if link-specific loss function is used.
Finally, we are also interested in practical applications of these ideas to different link functions and domains.
% \subsubsection*{Acknowledgments}

% Use unnumbered third level headings for the acknowledgments. All
% acknowledgments go at the end of the paper. Do not include
% acknowledgments in the anonymized submission, only in the
% final paper.


\bibliography{GRR}
\bibliographystyle{plainnat}

%////////////////////////////////

%\appendix
\newpage

\section*{Appendices}

\setcounter{section}{2}

\begin{thm:lemma}
For matrices $A ,B \in \{0,...,N\}^{n \times m}$:
\begin{align}
\GRR_{\tau_1,...\tau_N}(A) &\leq min(n,m)\\
 \GRR_{\tau_1,...\tau_N}(A) &=\GRR_{\tau_1,...\tau_N}(A^T)\\
 \GRR_{\tau_1,...\tau_N}(A+B) &\leq \GRR_{\tau_1,...\tau_N}(A)+\GRR_{\tau_1,...\tau_N}(B)
\end{align}
Where $+$ is in the real numbers and $A+B \in \{0,...,N\}^{n \times m}$.
\begin{proof}
According to definition of $\GRR$ and the fact that if $A=\GRF(C)$ then $ r(C) \leq min(n,m)$ we can conclude the first inequality. Furthermore, Since we know for any matrix $C$, $r(C)=r(C^T)$ and use the fact that if $A=\GRF(C)$ then $A^T=\GRF(C^t)$ we can show the second inequality as well. And the third inequality is the direct result of following famous inequality:
\begin{align}
r(A+b) &\leq r(A)+r(B)
\end{align}
\end{proof}
\end{thm:lemma}

\begin{thm:lemma}

the following decomposition holds for Generalized $\Round$ function: 
\begin{align}
\GRF_{\tau_1,...,\tau_N}(A)=\sum\limits_{i=1}^{N}\Round_{\tau_i}(A) 
\end{align}
\begin{proof}
Base on definition of Round Function $\sum\limits_{i=1}^{N}\Round_{\tau_i}(A)$ , counts the number of thresholds which are smaller than $A$, and this number is clearly equal to  $\GRF_{\tau_1,...,\tau_N}(A)$.
\end{proof}
\end{thm:lemma}

\begin{thm:lemma}
For any arbitrary subset of thresholds $T=\{\tau_{i_1},...,\tau_{i_r}\}$:
\begin{align}
\GRR_{\tau_1,...\tau_N}(A)\geq \GRR_{T}(\bar A)
\end{align}
Where $\bar A$ attained by the following transformation in matrix $A$:
\begin{align}
\bar A & =[b_{ij}]_{n\times m}\\
b_{ij} & =
\begin{cases}
	0, & \text{if } a_{ij} \in \{0,..,i_{1}-1\} \\
	1, & \text{if } a_{ij} \in\{i_{1},..,i_{2}-1\} \\
	\vdots\\
	r-1, & \text{if } a_{ij} \in\{i_{r},..,N-1\} 
\end{cases}
\end{align}
\begin{proof}
We define $\mathcal{B}$ and $\bar {\mathcal{B}}$ as follows:
\begin{align}
\mathcal{B} &=\{B|\GRF_{\tau_1,...,\tau_N}(B)=A\}\\
\bar {\mathcal{B}} &=\{\bar B|\GRF_{T}(\bar B)=A\} 
\end{align}
In result for any $B\in \mathcal{B}$, it is clear that $B \in \bar {\mathcal{B}}$
\end{proof}
\end{thm:lemma}

\begin{thm:lemma}

Following inequality holds for \GRR: 
\begin{align}
\GRR_{\tau_{1},...,\tau_{N}}(A)\leq \GRR_{\tau_1,...,\tau_N,\tau_{N+1}}(A) 
\end{align}
\begin{proof}
Similar to previous Lemma, if we define $\mathcal{B}$ and $\bar {\mathcal{B}}$ as follows:
\begin{align}
\mathcal{B} &=\{B|\GRF_{\tau_1,...,\tau_N}(B)=A\}\\
\bar {\mathcal{B}} &=\{\bar B|\GRF_{\tau_1,..,.\tau_N,\tau_{N+1}}(\bar B)=A\} 
\end{align}
Then it is clear that for any $\bar B\in \bar {\mathcal{B}}$, we have $\bar B \in \mathcal{B}$
\end{proof}
\end{thm:lemma}

\begin{thm:lemma}

Lets define the function $F: R^N \rightarrow N$ as follows:
\begin{align}
F(\tau_{1},...,\tau_{N})= \GRR_{\tau_1,...,\tau_N}(A) 
\end{align}
Where $A$ is a fix matrix in $\{0,...,N\}^{n\times m}$. Then we have the following inequality:
\begin{align}
F((\tau_{1}+\tau'_{1})/2,...,\tau_{N}) \leq F(\tau_{1},...,\tau_{N})+F(\tau'_{1},...,\tau_{N})
\end{align}
\begin{proof}
We define $\mathcal{B}$, $\mathcal{B}'$ and $\bar {\mathcal{B}}$ as follows:
\begin{align}
\mathcal{B} &=\{B|\GRF_{\tau_1,...,\tau_N}(B)=A\}\\
\mathcal{B}' &=\{B'|\GRF_{\tau_1',...,\tau_N}(B')=A\}\\
\bar {\mathcal{B}} &=\{\bar B|\GRF_{(\tau_{1}+\tau_{1}')/2,...,\tau_N}(\bar B)=A\} 
\end{align}
Accordingly, for any $B\in \mathcal{B}$ and $B' \in \mathcal{B}'$ we know $\frac{B+B'}{2}\in \bar {\mathcal{B}}$. Furthermore, since $\rank(\frac{B+B'}{2})=\rank(B+B')$ and $\rank(B+B') \leq \rank(B)+\rank(B')$ we can clearly prove the inequality.
\end{proof}
\end{thm:lemma}


\begin{thm:lemma}
We have the following inequality: 
\begin{align}
F(\tau_{1}+\tau'_{1},...,\tau_{N}+\tau'_{N}) \leq  F(\tau_{1},...,\tau_{N})+F(\tau'_{1},...,\tau'_{N})
\end{align}
\begin{proof}
Similar to previous Lemma, if we define $\mathcal{B}$, $\mathcal{B}'$ and $\bar {\mathcal{B}}$ as follows:
\begin{align}
\mathcal{B} &=\{B|\GRF_{\tau_1,...,\tau_N}(B)=A\}\\
\mathcal{B}' &=\{B'|\GRF_{\tau_1',...,\tau_N'}(B')=A\}\\
\bar {\mathcal{B}} &=\{\bar B|\GRF_{\tau_{1}+\tau_{1}',...,\tau_N+\tau_{N}'}(\bar B)=A\} 
\end{align}
For any $B\in \mathcal{B}$ and $B' \in \mathcal{B}'$ we know $B+B'\in \bar {\mathcal{B}}$. And since $\rank(B+B') \leq \rank(B)+\rank(B')$ we can clearly prove the inequality.
\end{proof}
\end{thm:lemma}

\begin{thm:thm}
For a given matrix $\Y \in \{0,\ldots,N\}^{n \times m}$, let's assume $\taus^*$ is the set of optimal thresholds, i.e. $ \GRR_{\taus^{\star}}(Y)=\text{argmin}_{\taus}\GRR_{\taus}(Y)$, then for any other $\taus'$: % with size $N$
\begin{align}
\label{thresh_th_ap}
\GRR_{\taus'}(\Y) \leq N\times\GRR_{\taus^{\star}}(\Y)+1
\end{align}
\begin{proof}
To prove above inequality we first need two following lemmas:
\begin{thm:lemma}
\label{thresh_ap}
We have the following inequality for \GRR: 
\begin{align}
\GRR_{\tau_{1}+c,...,\tau_{N}+c}(\Y)\leq \GRR_{\tau_1,...,\tau_N}(\Y)+1 
\end{align}
Where $c$ is a real number.
\begin{proof}
We define $\mathcal{B}$ and $\mathcal{B}'$ as follows:
\begin{align}
\mathcal{B} &=\{B|\GRF_{\tau_1,..,.\tau_N}(B)=\Y\}\\
\mathcal{B}' &=\{B'|\GRF_{\tau_1+c,...,\tau_N+c}(B')=\Y\}
\end{align}
For an arbitrary $B\in \mathcal{B}$ let's assume we have matrix $\U$ and $\V$ in a way that, $B=\U \times \V^T$. If we add a column to the end of $\U$ and a row to the and of $\V$ and call them $\U'$ and $\V'$ as follows:
\begin{align}
U'&=\begin{bmatrix}
 & c\\ 
U & \vdots \\ 
 & c
\end{bmatrix},
&V'&=\begin{bmatrix}
 & 1\\ 
V & \vdots \\ 
 & 1 
\end{bmatrix}
\end{align}
It is clear that $B'=\U'\times \V'^T \in \mathcal{B}'$. Furthermore, by using the fact that $\rank(B')\leq \rank(B)+1$ we can complete the proof.
\end{proof}
\end{thm:lemma}

\begin{thm:lemma}
For arbitrary $k \in \R$, the following equality holds: 
\begin{align}
\GRR_{k\tau_{1},...,k\tau_{N}}(\Y) = \GRR_{\tau_{1},...\tau_{N}}(\Y)
\end{align}
\begin{proof}
Similar to previous Lemma, if we define $\mathcal{B}$ and $\mathcal{B}'$ as follows:
\begin{align}
\mathcal{B} &=\{B|\GRF_{\tau_1,...\tau_N}(B)=A\}\\
\mathcal{B}' &=\{B'|\GRF_{k\tau_1,...,k\tau_N}(B')=A\}
\end{align}
For any $B\in \mathcal{B}$  it is clear that  $k\times B\in \mathcal{B}'$. On the other hand, for any $B'\in \mathcal{B}'$  we know that  $ B'/k\in \mathcal{B}$.In result, by considering the fact that $\rank(kB) = \rank(B)$, we can complete the proof .
\end{proof}
\end{thm:lemma}

 base on These lemmas and the fact that for any $i \in \{1,...,N-1\}$, there exist an $\epsilon_i$ which will satisfies the following equality:
\begin{align}
\label{equal}
\GRR_{\tau_{1},...,\tau_{i}-\epsilon_i,...,\tau_{N}}(\Y) = \GRR_{\tau_{1},...\tau_{N}}(\Y)
\end{align}
We can show that there exists a set of  $\epsilon_i$ $(i \in \{1,...,N-1\})$, that transform $(\tau_{1},...\tau_{N})$ in to $(\tau'_{1},...,\tau'_{N})$ with a set of linear combinations. In another word, it means we have $k_0,...,k_{N-1}$ in a way that:
\begin{align}
T'=k_0T_0+...+k_{N-1}T_{N-1}
\end{align}
Where $T'=(\tau_{1}',...\tau_{N}')$, $T_0=(\tau_{1},...\tau_{N})$ and $T_i=(\tau_{1},...,\tau_{i}-\epsilon_i,...,\tau_{N})$ in vector format. Therefore, if we define ${\mathcal{B}}_i$ as follows:
\begin{align}
{\mathcal{B}}_i=\{B_i|\GRF_{T_i}(B_i)=A\}
\end{align}
And considering the fact that:
\begin{align}
 \rank(k_0B+...+k_{N-1}B_{N-1})&\leq \sum_{j=0}^{N-1} \rank(k_{j}B_{j})\\
  &=\sum_{j=0}^{N-1} \rank(B_{j}) 
\end{align}
Finally, with Lemma~\ref{thresh_ap} equation~\ref{equal} we can complete the theorem.
\end{proof}
\end{thm:thm}

%%------------------------------------------------------------------------------------
\stepcounter{section}
\setcounter{section}{3}
\begin{thm:lemma}
For any matrix $A$, if there exists a submatrix $A'$ in a way that $\rank(A')=R$ and $\GRR_{\taus}(A')=r$, then $\GRR_{\taus}(A)\geq  r$ and $\rank(A)\geq R$
. \begin{proof}
If we consider the linear rank as the number of independent row (column) of the matrix, consequently having a rank of $r$ for submatrix $A'$ means there exist at least $r$ independent rows in matrix a $A$. Using this argument we can simply prove above inequalities.
\end{proof}
\end{thm:lemma}

\begin{thm:thm}
		If a matrix $\Y\in\R^{n\times m}$ contains $k$ rows, $k\leq n,k\leq m$, such that $R=\{Y_{R_1},...,Y_{R_k}\}$, two columns $C=\{j_0,j_1\}$, and:
	%\pouya{1) 2) 3)...in text}
	\begin{enumerate}
	\item rows in $R$ are distinct from each other, i.e, $\forall i,i'\in R, \exists j,Y_{ij}\neq Y_{i'j}$,
	\item columns in $C$ are distinct from each other, i.e, $\exists i,Y_{ij_0}\neq Y_{ij_1}$, and
	\item matrix spanning $R$ and $C$ are non-zero constants, w.l.o.g. $\forall i\in R,Y_{ij_0}=Y_{ij_1}=1$,
	\end{enumerate}
	then $\rank(\Y)\geq k$.
	\begin{proof}
	
				Let us assume $\rank(\Y)<k$, i.e. $\exists k'<k, \U\in\R^{k'\times n}, \V\in\R^{k'\times m}$ such that $\Y=\U^T\times\V$.
		Since the rows $R$ and the columns in $C$ are distinct, their factorizations in $\U$ and $\V$ have to also be distinct, i.e. $\forall i,i'\in R, i\neq i', \U_{i}\neq\U_{i'}$ and $\V_{j_0}\neq\V_{j_1}$.
		Furthermore, $\forall i,i'\in R, i\neq i', \not\exists a,\U_{i}=a\U_{i'}$ and $\not\exists a,\V_{j_0}=a\V_{j_1}$ for $a\neq0$, it is clear that $\U_i\cdot\V_{j_0}=\U_i\cdot\V_{j_1}=1$ (and similarly for $i,i'\in R$).

		Now consider a row $i\in R$. Since $\forall j\in C, \Y_{ij}=1$, then $\U_i\cdot\V_{j}=1$.
		As a result, $\V_j$ are distinct vectors that lie in the hyperplane spanned by $\U_i\cdot\V_j=1$.
		In other words, the hyperplane $\U_i\cdot\V_j=1$ defines a $k'$-dimensional hyperplane tangent to the unit hyper-sphere.

		Going over all the rows in $R$, we obtain constraints that $\V_j$ are distinct vectors that lie in the intersection of the hyperplanes spanned by $\U_i\cdot\V_j=1$ for all $i\in R$.
		Since all $\U_i$s are distinct, there are $k$ distinct $k'$-dimensional hyperplanes, all tangent to the unit sphere, that intersect at more than one point (since $\V_j$s are distinct).

		Since $k$ hyper-planes tangent to unit sphere can intersect at at most one point in $k'<k$ dimensional space, $\V_j$ cannot be distinct vectors. Hence, our original assumption $k'<k$ is wrong, therefore, $\rank(\Y)\geq k$.
	\end{proof}
\end{thm:thm}

%%-------------------------------------------------------------------------------------
\stepcounter{section}
\setcounter{section}{4}
\begin{thm:thm} (Necessary Condition)
For a target matrix $\Y \in \mathbb{V}^{n \times m}$ with few observed entries and given $\GRR(\Y)=r$, we consider set of $\{\Y_{i_1j},...,\Y_{i_rj}\}$ to be the $r$ observed entries in an arbitrary column $j$ of $Y$. 
Given any matrix $\X \in \mathcal{X}^{\star}$, $\X=\U\times \V^{T}$, and taking an unobserved entry $\Y_{ij}$, we define $a_{i_kj}$ as: 
$
\U_i=\sum_{k=1}^{r} a_{i_kj} \U_{i_k} 
$, where $\U_d$ ($d \in \{1,...,n\}$) is the $d^\text{th}$ row of matrix $\U$ and $i_k$ represents the index of observed entries in $j$th column.
%(the reason behind eligibility of this definition is provided in supplement). 
Then, the necessary condition of uniqueness of $\Y$ is:
\begin{align}
\sum_{k=1}^{r}\left | a_{i_kj} \right | \leq  \epsilon\left(\frac{T_\text{min}}{T_\text{max}}\right)
\end{align}
Where $r=\GRR(\Y)$, $T_\text{min}$ and $T_\text{max}$ are the length of smallest and largest intervals and $\epsilon$ is a small constant. 
\begin{proof}
To better understand the concept of uniqueness in $\GRR$ benchmark, let's first look at the uniqueness in fixed value matrix factorization (traditional definition(MF)).

 In fixed value matrix factorization, it is proved that to achieve uniqueness, we need at least $r=\rank(\X)$ observation in each column(other than the independent columns). Therefore, if we decompose $\X$ as $\X=\U\V^T$, and plan to changed only unobserved entries of $\Y$ in column $j$ (in opposed to uniqueness), we need to change the $jth$ row of matrix $\V$. To do so, let's assume we change the $jth$ row to:
 \begin{align}
[\V_{j1}+c_1,...,\V_{jr}+c_r]
\end{align}
Now since we know $\rank(U)=r$ and assume the respective rows of $\U$ to observed entries of column $j$ in matrix $\X$ are independent (this is a required assumption for uniqueness), we can show that only possible value for $c_1,..., c_r$ which does not change the observed entries of $\X$ is $0$, which confirm the uniqueness.  

The biggest difference between MF based on $\GRR$ and traditional MF is the fact that the observed entries of matrix $\X$ are not fixed in $\GRR$ version, and can change through the respective interval. In result, to achieve uniqueness we need to find a condition which for any column of $\X$, by changing respective row of $\V$, while the value of observed entries stay in the respected intervals, the value of unobserved ones wouldn't change dramatically which result in moving to other intervals. To do so, we will calculate the maximum of the possible change for an arbitrary unobserved entry of column $j$ in matrix $\Y$.

Let's call the $r$ observed entries of column's $j$ of matrix $\Y$, $\Y_{i_{1}j},...,\Y_{i_{r}j}$. Similar to MF case, we assume that the respective rows of $\U$ to these entries are independent. In result, if we represent the change in entries of $jth$ rows of $\V$ by $c_i$, we should have:
\begin{align}
\begin{bmatrix}
\U_{i_1}\\ 
\vdots \\ 
\U_{i_r}
\end{bmatrix}
\times
\begin{bmatrix}
c_1\\ 
\vdots \\ 
c_r
\end{bmatrix}
=
\begin{bmatrix}
\epsilon_{i_1j}\\ 
\vdots \\ 
\epsilon_{i_rj}
\end{bmatrix}
\end{align}            
Where $\U_{i_k}$ is the $i_k th$ row of $\U$, and $\epsilon_{i_kj}$ is the possible change for $\X_{i_{k}j}$, based on the observed interval. Therefore:
\begin{align}
\epsilon_{i_kj} \in (\tau_{i_kj}\downarrow-\X_{i_kj},\tau_{i_kj}\uparrow-\X_{i_kj})=(\epsilon_{i_kj}^-,\epsilon_{i_kj}^+)
\end{align}
Now let's assume we want to find the maximum possible change for $\X_{sj}$ considering that $\Y_{sj}$ is and unobserved entry. Since $\U_{i_k}$'s are independent, there exist $a_1,..a_r$ which:
\begin{align}
\U_s=\sum_{k=1}^{r}a_{i_kj}\U_{i_k}
\end{align}
Therefore, we can show the change in entry $\X_{sj}$ as:
\begin{align}
A=\sum_{k=1}^{r}a_{i_kj}\epsilon_{i_kj}
\end{align}
In result, for the maximum possible change we have:
\begin{align}
max|A|=max(\sum_{k=1}^{r}a_{i_kj}\epsilon_{i_kj}^{sign(a_{i_kj})},|\sum_{k=1}^{r}a_{i_kj}\epsilon_{i_kj}^{-sign(a_{i_kj})}|)
\end{align}
Where $sign(.)$ is the sign function. On the other hand we know:
\begin{align}
\sum_{k=1}^{r}a_{i_kj}\epsilon_{i_kj}^{sign(a_{i_kj})}+|\sum_{k=1}^{r}a_{i_kj}\epsilon_{i_kj}^{-sign(a_{i_kj})}|=\sum_{k=1}^{r}|a_{i_kj}|T_{i_kj}
\end{align}
\begin{align}
\Rightarrow max|A| \geqslant \frac{1}{2}\sum_{k=1}^{r}|a_{i_kj}|T_{i_kj}
\end{align}
 Where $T_{i_kj}$ is the length of the interval entry of $\mathbf{\bar X}_{i_kj}$. Clearly, to achieve the uniqueness we need $max|A|\leq T_{sj}$. But, since the entry $\X_{sj}$ is unobserved we don't know the value of $T_{sj}$. In result, for sake of uniqueness in the worst case we need:
\begin{align}
\sum_{k=1}^{r}|a_{i_kj}|T_{max} &\leq \epsilon T_{min}\\
\Rightarrow\sum_{k=1}^{r}|a_{i_kj}| &\leq\epsilon \frac{T_{min}}{T_{max}}
\end{align} 
Where $T_{min}$ and $T_{max}$ are the smallest and the biggest interval, and $\epsilon$ is a small real constant. 
 \end{proof}
\end{thm:thm}


\begin{thm:thm} (Sufficient Condition)
Using above necessary condition, for any unobserved entry $\Y_{ij}$ of matrix $\Y$ we define $\bar\epsilon$ as minimum distance of $\X_{ij}$ with its respected interval's boundaries. Than, we will have the following inequality as sufficient condition of uniqueness:
\begin{align}
\bar\epsilon \geq \max (\sum_{k=1}^{r}a_{i_kj}\epsilon_{i_k}^{\text{sign}(a_{i_kj})} , \left | \sum_{k=1}^{r}a_{i_kj}\epsilon_{i_k}^{-\text{sign}(a_{i_kj})}  \right |)
\end{align}
Where $r$ and $a_{i_kj}$ are defined as before, $\epsilon_{i_kj}^{+}$ is defined as the distance of $\X_{i_kj}$ with its respected upper bound and $\epsilon_{i_kj}^{-}$ is defined as negative of the distance of $\X_{i_kj}$ with its respected lower bound.
\begin{proof}
Sufficient condition is the direct result of Necessary Conditions proof. By combining (48) with the definition of uniqueness we can achieve the Sufficient Condition.
\end{proof}
\end{thm:thm}

\begin{thm:thm}
For any $\epsilon>0$ and matrix $\Y$, $\text{sign-rank}(\Y) = \apprank_\sigmoid(\Y)$.
\begin{proof}
Let $\Bm_\sigmoid^\epsilon(k)=\{\B\in\{0,1\}^{n\times m}; \apprank_\sigmoid(\B)=k\}$, i.e. the set of binary matrices whose $\apprank_\sigmoid$ is equal to $k$, and $\Bm_\step(k)=\{\B\in\{0,1\}^{n\times m}; \text{sign-rank}(\B)=k\}$.  We prove the theorem by showing both directions.
\textul{$\Bm_\step\subseteq\Bm_\sigmoid$}:
Any $\U,\V$ that works for $\step$ should work with $\sigmoid$ if multiplied by a very large number, i.e. take a sufficiently large $\eta$, and $\U_\sigmoid=\eta\U_\step,\V_\sigmoid=\eta\V_\step$.
Then, $\X_\sigmoid=\eta^2\X_\step$ and if we set $\lparams_\sigmoid=\eta^2\lparams_\step$, then $(\X_\sigmoid-\lparams_\sigmoid)=\eta^2(\X_\step-\lparams_\step)$, therefore will have the same sign, and $\Y_\sigmoid=\sigmoid(\X_\sigmoid)$ will be arbitrarily close to $0$ and $1$ in $\Y_\step$.
\textul{$\Bm_\sigmoid\subseteq\Bm_\step$}:
Any $\U,\V$ that works for $\sigmoid$ will directly work with $\step$.
\end{proof}
\end{thm:thm}

\begin{thm:rmk}
To extend Theorem 4.3 to multi-ordinal cases, we need to show that for any arbitrary set of thresholds in GRR, there exists another set of thresholds for multi-sigmoid function which will satisfy the condition in theorem 4.3 for multi-ordinal matrices. The procedure of proof is similar to binary cases. The only difference is the fact that after multiplying our matrices into a big enough constant, we need to choose multi-sigmoid’s thresholds in a way that will guarantee the multi-sigmoid(X) is close enough of to GRF(X)(which is equal to Y). 
\end{thm:rmk}


\end{document}
