There are many reasons for the fact that available data are,
more often than not,
incomplete.
Some of the possible reasons are:
\bi
\item Epidemics are rare events that occur at random times and in random places,
and when such rare events do occur,
public health officials and scientists may not be well-prepared to collect relevant data without advance notice.
\item Ethical and legal considerations can make the collection of data on individual population members challenging,
if not impossible:
e.g.,
if there was universal cell phone coverage and all population members carried cell phones at all times,
collecting data on contacts among population members would be straightforward by monitoring the locations of cell phones.
However,
collecting such data would violate laws that protect the privacy of population members.
\item Epidemics are not limited to urban areas with excellent infrastructure and ready access to public resources,
but may occur in remote corners of the planet:
e.g.,
the most recent outbreaks of Ebola started in remote areas of Africa.
Worse,
some areas with outbreaks may be war-torn.
As a result,
researchers may not be able to collect data by visiting areas with outbreaks without exposing themselves and others to unacceptable risks.
\item In addition, 
there are more mundane reasons for incomplete data,
such as
\bi
\item[---] {\em design-based mechanisms:} 
e.g.,
motivated by financial constraints,
researchers may sample population members, 
which implies that a sampling design determines which data are collected;
\item[---] {\em out-of-design mechanisms:} 
e.g.,
population members refuse to share data when the data are considered sensitive.
\ei
\ei
