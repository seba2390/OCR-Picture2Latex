We proceed with posterior predictions of the degree distribution of the population contact network.
While the population contact network and its degree distribution are unobserved,
posterior predictions of the degree distribution can be generated,
provided draws from the posterior distribution are available.
Probabilistic statements about the degrees of population members based on the posterior distribution are informed by the observed infectious and removal times along with the assessments of doctors of who infected whom.
Both of these sources of information help inform who was in contact with whom,
because
\bi
\item the observed infectious and removal times reveal when infected population members were infectious,
which helps narrow down the possible sources of infections;
\item an infection implies a contact.
\ei
The posterior predictions of the degree distribution shown in Figure \ref{ppc.degrees} suggest that the degree distribution is long-tailed:
The bulk of population members has no more than 10 contacts,
but some population members have as many as 80 contacts.
As pointed out before,
population members with many contacts can infect many other population members and therefore represent an important public health concern.
