\bi
\item[] Section 1 Motivation
\bi
\item[] Section 1.1 Advantages of network-based approaches to epidemics
\item[] Section 1.2 Shortcomings of existing network-based approaches
\item[] Section 1.3 Proposed network-based approach 
\item[] Section 1.4 Goal: superpopulation inference for finite populations
\item[] Section 1.5 Structure of the paper
\ei
\item[] Section 2 A network-based stochastic model of epidemics
\bi
\item[] Section 2.1 Data-generating process
\item[] Section 2.2 Parametric population models
\ei
\item[] Section 3 Shortcomings of parametric population models
\item[] Section 4 Semiparametric population model
\bi
\item[] Section 4.1 Detecting potential superspreaders
\item[] Section 4.2 Short- and long-tailed degree distributions
\ei
\item[] Section 5 Incomplete data
\bi
\item[] Section 5.1 Possible reasons for incomplete data
\item[] Section 5.2 Importance of collecting network data
\ei
\item[] Section 6 Bayesian inference
\bi
\item[] Section 6.1 Complete- and incomplete-data generating process
\item[] Section 6.2 Bayesian inference based on incomplete data
\item[] Section 6.3 Truncated Dirichlet process priors
\item[] Section 6.4 Bayesian Markov chain Monte Carlo algorithm
\item[] Section 6.5 Addressing the label-switching problem
\ei
\item[] Section 7 Simulation results
\bi
\item[] Section 7.1 Simulation results quantifying the error of estimation
\item[] Section 7.2 Simulation results quantifying the effect of network sampling
\ei
\item[] Section 8 Partially observed MERS epidemic in South Korea
\bi
\item[] Section 8.1 Data
\item[] Section 8.2 Model
\item[] Section 8.3 Computing
\item[] Section 8.4 Results
\ei
\item[] Section 9 Open questions and directions for future research
\bi
\item[] Section 9.1 What is the population of interest?
\item[] Section 9.2 Incomplete data
\item[] Section 9.3 Computational challenges arising from incomplete data
\item[] Section 9.4 Non-ignorable incomplete-data generating processes
\item[] Section 9.5 Population models capturing additional network features
\item[] Section 9.6 Time-evolving population contact networks
\ei
\ei
