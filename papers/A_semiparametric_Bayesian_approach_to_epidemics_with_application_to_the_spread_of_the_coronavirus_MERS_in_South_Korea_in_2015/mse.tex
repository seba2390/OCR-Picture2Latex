1,000 egocentric samples of size $n = 25$, $50$, $75$, $100$, $125$, $150$, $187$ are generated from the population of size $N = 187$.
For each sample, 
we estimated the model.
In addition,
we estimated the model without observations of contacts,
that is,
when $n=0$.
The MSE of the posterior median and mean of the rate of infection $\beta$ and the degree parameters $\gamma_1, \gamma_2, \gamma_3$ are shown in Figure \ref{mse.plot} when $n = 0$, $25$, $50$, $75$, $100$, $125$, $150$, $187$.
By construction of the model,
estimators of the epidemiological parameters $\eta_{E,1}$, $\eta_{E,2}$, $\eta_{I,1}$, $\eta_{I,2}$ are not expected to be sensitive to $n$---which determines how much information is available about the network parameters---and the MSE of the posterior median and mean of $\eta_{E,1}$, $\eta_{E,2}$, $\eta_{I,1}$, $\eta_{I,2}$ is indeed not sensitive to $n$ (not shown).
Figure \ref{mse.plot} demonstrates that samples of contacts reduce the MSE of posterior median and mean of $\beta$, $\gamma_1, \gamma_2, \gamma_3$:           
The MSE turns out to be highest when $n = 0$,
%A sample of size $n=25$ results in a large reduction in MSE,
%and increasing $n$ tends to decrease the MSE.
and rapidly decreases as $n$ increases.
These observations suggest that in practice samples of contacts---or functions of contacts, 
such as degrees---should be collected,
reducing posterior uncertainty.
