We discuss Bayesian inference for the parameters $\bta$ and $\btheta$ of the population model based on incomplete data.
Since interest centers on the population model,
it is natural to ask:
Under which conditions is the process that determines which data are observed ignorable for the purpose of Bayesian inference for the parameters $\bta$ and $\btheta$ of the population model?
To answer the question,
we start from first principles.
We first separate the complete-data generating process from the incomplete-data generating process:
\bi
\item The {\em complete-data generating process} is the process that generates the complete data,
that is,
the process that generates a realization $(\bx, \by)$ of $(\bX, \bY)$.
\item The {\em incomplete-data generating process} is the process that determines which subset of the complete data $(\bx, \by)$ is observed.
\ei
A failure to separate these processes can lead to misleading conclusions,
as pointed out by \citet{Ru76}, \citet{DaDi77}, \citet*{KoRoPa09}, \citet{HaGi09,GiHa17}, \citet{crane2018probabilistic}, and \citet{ScKrBu17}.

We therefore proceed as follows.
We first separate the complete-data generating process from the incomplete-data generating process in Section 6.1
and then discuss Bayesian inference based on incomplete data in Section 6.2.
We then discuss Bayesian computing in Sections 6.3, 6.4, and 6.5.
