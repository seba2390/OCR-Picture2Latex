We consider two specifications of the prior probabilities of who infected whom,
because the posterior is sensitive to the prior probabilities of who infected whom.
The reason is that the MERS data set does not contain direct observations of transmissions or contacts,
and the information on transmissions is therefore limited to two sources of indirect observations:
the assessments of doctors of who infected whom and the observed infectious and removal times,
as discussed in Section 8.1.
While the observed infectious and removal times help narrow down the possible sources of infections,
there may be many possible sources of infections left.
As a consequence,
it is not surprising that the posterior is sensitive to the choice of prior probabilities of who infected whom.
We demonstrate the sensitivity of the posterior to the choice of prior in Section 8.4 by using two specifications of prior probabilities:
\bi 
\item[(a)] If the doctors assessed that population member $i$ infected population member $j$,
we specify $\varphi(\mbox{$i$ infected $j$}) = 1$ and $\varphi(\mbox{$h$ infected $j$}) = 0$ for all infected population members $h \in \{1, \dots, 186\}\, \setminus\, \{i, j\}$.
If the doctors did not specify who infected $j$ and there are $M_j \in \{1, \dots, 185\}$ infected population members $h$ satisfying $I_h < E_j < R_h$,
we specify $\varphi(\mbox{$i$ infected $j$}) = 1\, /\, M_j$\, for all infected population members $i$ satisfying $I_i < E_j < R_i$ and $\varphi(\mbox{$h$ infected $j$}) = 0$ for all other infected population members $h$.
\item[(b)] If there are $M_j \in \{1, \dots, 185\}$ infected population members $h$ satisfying $I_h < E_j < R_h$,
we specify $\varphi(\mbox{$i$ infected $j$}) = 1\, /\, M_j$ for all infected population members $i$ satisfying $I_i < E_j < R_i$ and $\varphi(\mbox{$h$ infected $j$}) = 0$ for all other infected population members $h$.
\ei
