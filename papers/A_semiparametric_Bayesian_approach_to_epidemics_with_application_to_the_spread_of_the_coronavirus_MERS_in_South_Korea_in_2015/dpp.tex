In addition to detecting potential superspreaders,
the model can accommodate short- and long-tailed degree distributions.
To demonstrate that Dirichlet process priors can accommodate short- and long-tailed degree distributions,
we consider a population of size $N = $ 1,000 and generate three sets of degree parameters $\theta_1, \dots, \theta_{1000}$\, from the Dirichlet process prior with 
concentration parameter $\alpha=5$ and base distribution $N(-5,\, 25)$.
Figure 2 shows kernel density plots of the three sets of degree parameters $\theta_1, \dots, \theta_{1000}$ along with the expected degrees of population members.
The expected degree of population member $i$ is defined as
\bea
\mu_i(\btheta)
&=& \mbE_{\btheta}\left(\dsum_{j=1:\, j \neq i}^{1000} Y_{i,j}\right)
&=& \dsum_{j=1:\, j \neq i}^{1000} \dfrac{1}{1 + \exp(-\theta_i-\theta_j)},
& i = 1, \dots, N.
\eea
\hide{
[1] "theta:"
    Min.  1st Qu.   Median     Mean  3rd Qu.     Max. 
-16.9059 -11.1971  -9.8041  -7.9952  -2.7154  -0.2946 
[1] "mean degrees:"
    Min.  1st Qu.   Median     Mean  3rd Qu.     Max. 
 0.00000  0.00096  0.00386  3.98418  4.47277 38.69275 
      75%       80%       85%       90%       95% 
 4.472773  4.472773  4.472773  4.472773 38.692754 
[1] "theta:"
    Min.  1st Qu.   Median     Mean  3rd Qu.     Max. 
-15.2407  -6.4523  -1.7174  -3.3954   0.5948   5.9689 
[1] "mean degrees:"
    Min.  1st Qu.   Median     Mean  3rd Qu.     Max. 
  0.0007   4.3531 157.5882 175.3936 324.2115 722.5930 
     75%      80%      85%      90%      95% 
324.2115 324.2115 513.3448 513.3448 513.3448 
[1] "theta:"
   Min. 1st Qu.  Median    Mean 3rd Qu.    Max. 
-14.079 -10.760 -10.760  -7.245  -5.686   5.369 
[1] "mean degrees:"
     Min.   1st Qu.    Median      Mean   3rd Qu.      Max. 
  0.01154   0.31797   0.31797  47.77933  34.57946 259.14991 
      75%       80%       85%       90%       95% 
 34.57946 127.52472 155.90042 240.41126 248.45654 
}
Figure 2 demonstrates that the distribution of the expected degrees $\mu_1(\btheta), \dots, \mu_{1000}(\btheta)$ can be short- or long-tailed,
depending on the degree parameters $\btheta = (\theta_1, \dots, \theta_{1000})$.
The first set of degree parameters $\theta_1, \dots, \theta_{1000}$ generated from the Dirichlet process prior consists of three subsets of degree parameters,
all of them negative.
The resulting distribution of the expected degrees resembles a steep mountain with a high peak in a neighborhood of $0$ and a short upper tail.
In fact,
90\% of all population members have expected degrees of less than 5,
and the highest expected degree is less than 39,
which is much lower than the highest possible degree of 999 in a population of size 1,000.
The second set of generated degree parameters $\theta_1, \dots, \theta_{1000}$ consists of many negative degree parameters and some positive degree parameters between $0$ and $5$.
Since the log odds of the probability of a contact between two population members $i$ and $j$ is $\theta_i + \theta_j$,
population members with positive degree parameters can have high to very high expected degrees.
Figure 2 shows that the population consists of at least three subpopulations:
population members with expected degrees of less than 100;
population members with expected degrees between 100 and 200;
and population members with expected degree of more than 300.
The highest expected degree is about 722.
%which is a very high expected degree in a population of size 1,000.
The resulting distribution of the expected degrees is both multimodal and long-tailed.
The third set of generated degree parameters $\theta_1, \dots, \theta_{1000}$ resembles the second set of generated degree parameters,
in that the distribution of the expected degrees is multimodal and long-tailed.
That said,
the third set of draws is less extreme than the second set of draws:
e.g.,
the highest expected degree is about 259 rather than 722.

The examples presented above demonstrate that Dirichlet process priors with Gaussian base distributions can accomodate both short- and long-tailed degree distributions,
despite the fact that Gaussians are symmetric and unimodal distributions with light tails.
Other, non-Gaussian base distributions can be chosen.
%symmetric or asymmetric base distributions and unimodal or multimodal base distributions.
As a consequence,
the proposed semiparametric population model is flexible and can accomodate a wide range of degree distributions with countless forms and shapes,
including short- and long-tailed degree distributions.
