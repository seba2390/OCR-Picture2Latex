To demonstrate that collecting network data can reduce the posterior uncertainty about the parameters of the population model,
we consider a population consisting of $K=3$ subpopulations.
The $K=3$ subpopulations correspond to 
\bi
\item a low-degree subpopulation of size 127 with degree parameter $\gamma_1 = -3.5$;
\item a moderate-degree subpopulation of size 50 with degree parameter $\gamma_2 = -1.5$;
\item a high-degree subpopulation of size 10 with degree parameter $\gamma_3 = .5$.
\ei
% The degree parameters of population members $i$ are given by $\theta_i = \bZ_i^\top\, \bgamma$ ($i = 1, \dots, N$).
We generate 1,000 ego-centric samples of sizes $n = 25$,\, $50$,\, $75$,\, $100$,\, $125$,\, $150$,\, $187$ from the population of size $N = 187$.
We then estimate the population model from each sample of contacts along with observations of the exposure, infectious, and removal times of infected population members.
In addition,
we estimate the population model without observations of contacts,
which corresponds to a sample size of $n=0$,
using observations of the exposure, infectious, and removal times of infected population members.
% As pointed out in Section 5.2,
% collecting network data sampling helps infer the unobserved sources of infections,
% which in turn helps infer the population model.
To assess how much the posterior uncertainty about the parameters of the population model is reduced by sampling contacts,
we use the mean squared error (MSE) of the posterior median and mean of the parameters.
