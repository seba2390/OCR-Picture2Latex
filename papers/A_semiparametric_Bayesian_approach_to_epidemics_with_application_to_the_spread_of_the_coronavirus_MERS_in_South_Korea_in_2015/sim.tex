We explore the frequentist properties of Bayesian point estimators and the reduction in statistical error due to collecting network data by using simulations.
We consider a population of size 187 consisting of $K = 3$ subpopulations labeled $1$, $2$, $3$.
The three subpopulations consist of low-, moderate-, and high-degree population members.
We assign population members $i$ to subpopulations $1, 2, 3$ by sampling $\bZ_i \iid \mbox{Multinomial}(1;\, \bpi=(.4,\, .3,\, .3))$ ($i = 1, \dots, N$).
We then generate a population contact network from the population model described in Section 4 with degree parameters $\theta_i = \bZ_i^\top\, \bgamma$ ($i = 1, \dots, N$). 
Conditional on the population contact network,
an epidemic is generated by the stochastic model described in Section 2,
assuming that $I_i - E_i$ and $R_i - I_i$ are independent Gamma$(\eta_{E,1},\, \eta_{E,2})$ and Gamma$(\eta_{I,1},\, \eta_{I,2})$ random variables,
respectively ($i = 1, \dots, M$).
The data-generating values of the parameters are specified in Sections 7.1 and 7.2.
Unless stated otherwise,
we assume that the exposure, infectious, and removal times $\bE$, $\bI$, $\bR$ are observed,
whereas the transmissions $\bT$ are unobserved,
as are the population contact network $\bY$ and the indicators $\bZ$.
