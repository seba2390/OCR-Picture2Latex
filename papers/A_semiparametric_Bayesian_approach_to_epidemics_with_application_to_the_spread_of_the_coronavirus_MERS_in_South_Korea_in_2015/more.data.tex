It is worth noting that,
in addition to the assessments of doctors,
the observed infectious and removal times help inform who infected whom.
To see that,
note that we know who is infectious in a given time interval,
because we observe the infectious and removal times of infected population members.
Suppose that population member $1$ turns infectious on day 1 and recovers by day 14 and population member $2$ turns infectious on day 7,
while all other population members are non-infectious during the first two weeks of the epidemic.
Then population member $1$ must have infected population member $2$ during the first week of the epidemic.
In other words,
the observed infectious and removal times can reduce the posterior uncertainty about who infected whom,
by narrowing down the possible sources of infections.
