Markov chain Monte Carlo samples from the posterior distribution may show evidence of label-switching,
that is,
the labeling of subpopulations may have switched from iteration to iteration of the Markov chain Monte Carlo algorithm.
The label-switching problem is rooted in the fact that the likelihood function is invariant to permutations of the labels of subpopulations.
While the prior is not invariant to permutations of the labels of subpopulations,
the prior is dominated by the likelihood function when the data is informative.
As a consequence,
the labels of subpopulations may switch from iteration to iteration of the Markov chain Monte Carlo algorithm.
Label-switching can give rise to misleading conclusions about parameters that depend on the labeling of the subpopulations,
including the indicators $\bZ_1, \dots, \bZ_N$ and the degree parameters $\gamma_1, \dots, \gamma_K$ and $\theta_1 = \bZ_1^\top\, \bgamma, \dots, \theta_N = \bZ_N^\top\, \bgamma$.
The label-switching problem is a well-known problem in the literature concerned with Bayesian Markov chain Monte Carlo estimation of finite mixture models and related models.
We follow the Bayesian decision-theoretic approach of \citet{St00} to undoing the label-switching,
by minimizing the posterior expectation of a well-chosen loss function.
A loss function and relabeling algorithm are stated in \citet{ScHa13} and implemented in {\tt R} package {\tt hergm} \citep{ScLu15}.
We use them in Sections 7 and 8 to undo the label-switching in all Markov chain Monte Carlo samples generated by the Bayesian Markov chain Monte Carlo algorithm.
