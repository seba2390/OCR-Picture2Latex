We sample from the posterior distribution by combining the following Markov chain Monte Carlo steps by means of cycling or mixing \citep*{Tl94,Li08}.

\s

{\bf Concentration parameter $\alpha \mid \bpi$.}
Assuming the hyperprior of concentration parameter $\alpha$ is Gamma$(A_1,\, B_1)$,
we can update $\alpha$ by sampling
\bea
\nonumber
\alpha \mid \bpi
&\sim& \mbox{Gamma}(A_1 + K - 1,\; B_1 - \log \pi_K).
\eea

{\bf Mean parameter $\mu \mid \sigma^2,\, \bgamma$.}
Assuming the hyperprior of mean parameter $\mu$ is $N(O,\, S^2)$,
we can update $\mu$ by sampling
\bea
\nonumber
\mu \mid \sigma^2,\, \bgamma
&\sim& N\left(\dis\frac{S^{-2}\, O + \sigma^{-2} \sum_{k = 1}^K \gamma_k}{S^{-2} + K \sigma^{-2}},\; \dis\frac{1}{S^{-2} + K \sigma^{-2}}\right).
\eea

{\bf Precision parameter $1/\sigma^{2} \mid \mu,\, \bgamma$.}
Assuming the hyperprior of precision parameter $1/\sigma^{2}$ is Gamma$(A_2,\, B_2)$,
we can update $1/\sigma^{2}$ by sampling
\bea
\nonumber
1/\sigma^2 \mid \mu,\, \bgamma
&\sim& \mbox{Gamma}\left(A_2 + \dfrac{K}2,\; B_2 + \dsum\limits_{k = 1}^K \dis\frac{(\gamma_k - \mu)^2}2\right).
\eea

{\bf Mixing proportions $\bpi \mid \bZ=\bz,\, \alpha$.}
Let $N_k$ be the number of population members in subpopulation $k$.
Then $\bpi$ can be updated by sampling
\bea
\nonumber
V_k \mid \bZ=\bz,\, \alpha
&\ind& \mbox{Beta}\left(1 + N_k,\; \alpha + \dsum\limits_{j = k + 1}^K N_j\right),\; k = 1, \dots, K - 1,
\eea
then setting $V_K = 1$ and constructing $\bpi$ as follows:
\bea
\nonumber
  \pi_1 \= V_1\s
  \\
  \pi_k \= V_k\, \dis\prod_{j = 1}^{k - 1}\, (1 - V_j),\; k = 2, \dots, K.
\eea

\s

{\bf Indicators $\bZ_i \mid \bY=\by,\, \bZ_{-i} = \bz_{-i},\, \bgamma,\, \bpi$.}
We update indicators $\bZ_i$ by sampling
\bea
\nonumber
\bZ_i \mid \bY=\by,\, \bZ_{-i} = \bz_{-i},\, \bgamma,\, \bpi
&\sim& \mbox{Multinomial}(1; \pi_{i,1}, \dots, \pi_{i,K}),
\eea
where
\bea
\nonumber
\pi_{i,k} = \dis\frac{\dis \pi_k\, \prod_{j:\, j \neq i}^N p(y_{i,j} \mid Z_{ik} = 1,\, \bY=\by,\, \bZ_{-i}=\bz_{-i},\, \bgamma,\, \bpi)}{\dis\sum_{l = 1}^K \left(\pi_l\, \prod_{j:\, j \neq i}^N p(y_{i,j} \mid Z_{il} = 1,\, \bY=\by,\, \bZ_{-i}=\bz_{-i},\, \bgamma,\, \bpi)\right)}.
\eea

\s

{\bf Degree parameters $\bgamma \mid \bY=\by,\, \bZ=\bz$.}
We update $\bgamma$ by Metropolis-Hastings steps,
where proposals are generated from random-walk, independence, or autoregressive proposal distributions \citep{Tl94}.

\s

{\bf Degree parameters $\btheta$.}
The degree parameter $\theta_i$ of population member $i$ is updated by setting
\bea
\nonumber
\theta_i
\= \bZ_i^\top\, \bgamma,
&& i = 1, \dots, N.
\eea

\s

{\bf Epidemiological parameters $\bta \mid \bX = \bx$.}
We use Gibbs and Metropolis-Hastings steps for updating $\bta$ along the lines of \citet{GrWeHu10,GrWeHu11}.
\s

{\bf Unobserved contacts $Y_{i,j} \mid \bX=\bx,\, \bZ_i=\bz_i,\, \bZ_j=\bz_j,\, \bgamma,\, \bta$.}
If $Y_{i,j}$ is unobserved,
we can update $Y_{i,j}$ by sampling
\bea
Y_{i,j} \mid \bX=\bx,\, \bZ_i=\bz_i,\, \bZ_j=\bz_j,\, \bgamma,\, \bta
&\ind& \mbox{Bernoulli}(p_{i,j}),
\eea
where
\bea
p_{i,j}
\= \dfrac{\exp(-\beta\, \max(\min(E_j, R_i) - I_i, 0))\, \mbP_{\theta_i,\theta_j}(Y_{i,j}=1)}{p_{i,j}(0) + \exp(-\beta\, \max(\min(E_j, R_i) - I_i, 0))\, \mbP_{\theta_i,\theta_j}(Y_{i,j}=1)}.
\eea
Here,
$\mbP_{\theta_i,\theta_j}(Y_{i,j}=y_{i,j})$ is defined by
\bea
\mbP_{\theta_i,\theta_j}(Y_{i,j}=y_{i,j}) \= \exp(\lambda_{i,j}(\btheta)\, y_{i,j} - \psi_{i,j}(\btheta)),
\ee
where
\bea
\lambda_{i,j}(\btheta)
&=& \theta_i + \theta_j.
\eea

\s

{\bf Unobserved transmissions $\bT \mid \bE,\, \bI,\, \bR,\, \bY = \by$.}
If population member $j$ was infected and the source of infection is unknown,
we can update the source of infection by sampling from its full conditional distribution.
If $\varphi(\mbox{$i$ infected $j$})$ denotes the prior probability of the event that population member $i$ infected population member $j$,
the conditional probability of the event that population member $i$ infected population member $j$,
given everything else,
takes the form
\bea
\nonumber
\mbP(\mbox{$i$ infected $j$} \mid \bE,\, \bI,\, \bR,\, \bY=\by)
&=& \dfrac{y_{i,j}\; 1_{I_i < E_j < R_i}\; \varphi(\mbox{$i$ infected $j$})}{\sum_{h=1:\, h \neq j}^M\, y_{h,j}\; 1_{I_h < E_j < R_h}\; \varphi(\mbox{$h$ infected $j$})}.
\eea

\s

{\bf Unobserved exposure, infectious, and removal times $\bE,\, \bI,\, \bR \mid \bT,\, \bta$.}
If some or all exposure and infectious times $\bE$ and $\bI$ are unobserved,
we use the Metropolis-Hastings steps of \citet{GrWeHu10} for updating them.
If some or all removal times $\bR$ are unobserved,
we use the Gibbs and Metropolis-Hastings steps described in the Ph.D.\ thesis of \citet[][Section 4.6]{Bo14} for updating them.

