The proposed semiparametric modeling framework,
based on infinite mixture distributions and Dirichlet process priors \citep{Fe73,Teh2007},
extends to infinite populations.
That said,
we assume that the number of population members $N$ is finite and embrace a superpopulation approach to statistical inference along the lines of \citet{hartley1975super} and \citet*{ScKrBu17},
motivated by applications.

The assumption of finite $N$ is motivated by the fact that in epidemiological applications the number of population members $N$ cannot be infinite.
For example,
when the population of interest consists of all animals or all humans on earth,
the size of the population is bounded above by real-world constraints such as geography and the scarcity of natural resources:
Planet earth cannot host infinite populations of animals or humans.

Since the population of interest is finite,
the natural objective of statistical inference is to learn the stochastic process that generated the population contact network and allows infectious diseases to spread through the population of interest,
with a view to understanding and predicting epidemics in the population of interest and similar populations.
In other words,
it is natural to embrace a superpopulation approach to statistical inference,
as discussed by \citet{hartley1975super} and \citet{ScKrBu17}.
The properties of statistical procedures for superpopulation inference can be understood by developing a non-asymptotic statistical theory that relies on concentration inequalities and other non-asymptotic tools that have been embraced in high-dimensional statistics \citep[see, e.g.,][]{Wa19}.
We are not aware of non-asymptotic statistical theory for stochastic models of epidemics,
although there are asymptotic results in probability theory \citep*[e.g.,][]{Re95,BrLiTu11,BaRe13,PaPa20,Ba21} and statistical theory \citep[e.g.,][]{Br98,Br01a} based on $N \to \infty$ asymptotics.
Developing non-asymptotic statistical theory for stochastic models of epidemics constitutes an interesting direction for future research,
but is beyond the scope of our paper.
