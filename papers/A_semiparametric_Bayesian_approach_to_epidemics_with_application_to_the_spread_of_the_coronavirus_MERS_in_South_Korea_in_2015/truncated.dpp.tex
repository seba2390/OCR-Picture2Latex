To facilitate Markov chain Monte Carlo sampling from the posterior distribution,\linebreak
we approximate Dirichlet process priors by truncated Dirichlet process priors along the lines of \citet{IsJa01}.

The truncation of Dirichlet process priors takes advantage of the stick-breaking construction of Dirichlet process priors \citep{Se94,IsJa01,Teh2007}.
A stick-breaking construction of a Dirichlet process prior with base distribution $G$ and concentration parameter $\alpha$ proceeds as follows.
First,
we sample parameters from the base distribution $G$: 
\bea
\nonumber
\gamma_k &\msim\limits^{\tiny \mbox{iid}}& G,
& k = 1, 2, \dots
\eea
We then construct mixing proportions $\pi_1$, $\pi_2$, $\dots$ by first sampling
\bea
\nonumber
  V_k \mid \alpha &\msim\limits^{\tiny \mbox{iid}}& \mbox{Beta}(1, \alpha),
& k = 1, 2, \dots
\eea
and then setting
\bea
\nonumber
 \pi_1 & = & V_1\s
 \\
 \pi_k & = & V_k\, \dis\prod_{j = 1}^{k - 1} (1 - V_j),\; k = 2, 3, \dots
\eea
Last,
but not least,
we construct an infinite mixture distribution with mixing proportions $\pi_1, \pi_2, \dots$ and point masses $\delta_{\gamma_1}$, $\delta_{\gamma_2}$, $\dots$ as follows:
\bea
\nonumber
\mathscr{P}_\infty &=& \dsum_{k=1}^\infty\, \pi_k\; \delta_{\gamma_k}.
\eea
The distribution $\mathscr{P}_\infty$ is then a draw from the Dirichlet process prior with concentration parameter $\alpha$ and base distribution $G$. 

A Dirichlet process prior can be truncated by choosing a positive integer $K$ and setting $V_K = 1$,
which implies that $\pi_{K+1} = 0$, $\pi_{K+2} = 0$, $\dots$
A finite mixture distribution with mixing proportions $\pi_1, \pi_2, \dots, \pi_K$ and point masses $\delta_{\gamma_1}$, $\delta_{\gamma_2}$, $\dots$, $\delta_{\gamma_K}$ can then be constructed as follows:
\bea
\nonumber
\mathscr{P}_K &=& \dsum_{k=1}^K\, \pi_k\; \delta_{\gamma_k}.
\eea
The distribution $\mathscr{P}_K$ can be regarded as a draw from the Dirichlet process prior with concentration parameter $\alpha$ and base distribution $G$ truncated at $K$.
If $K$ is large,
the truncated Dirichlet process prior is expected to be a good approximation of the Dirichlet process prior.
Some theoretical guidance regarding the choice of $K$ can be found in \citet{IsJa01}.
In practice,
$K$ can be chosen by 
\bi
\item selecting a large positive integer, 
such as $K = N$;
\item exploiting domain knowledge;
\item choosing a value of $K$ that leads to acceptable in- or out-of-sample performance.
\ei
The truncated stick-breaking construction of $\bpi$ implies that $\bpi$ is generalized Dirichlet distributed,
which is conjugate to multinomial sampling \citep{IsJa01} and facilitates Markov chain Monte Carlo sampling from the posterior distribution.

Since the concentration parameter $\alpha$ and the parameters $\mu$ and $\sigma^2$ of the base distribution $N(\mu,\, \sigma^2)$ are unknown,
it is natural to express the uncertainty about $\alpha$, $\mu$, and $\sigma^2$ by assuming that $\alpha$, $\mu$, and $\sigma^2$ have hyperpriors.
%In other words,
%the knowledge about the hyperparameters $\alpha$, $\mu$, and $\sigma^2$ is updated in the light of the observed data:
%The Bayesian Markov chain Monte Carlo algorithm described in Section 6.4 can be used to approximate the marginal posteriors of the hyperparameters $\alpha$, $\mu$, and $\sigma^2$ given the observed data,
%although the hyperparameters are not the primary target of statistical inference.
We assume that the hyperpriors of the hyperparameters $\alpha$, $\mu$, and $1 / \sigma^{2}$ are Gamma, Gaussian, and Gamma distributions,
respectively,
which are conjugate priors and facilitate Markov chain Monte Carlo sampling from the posterior distribution.

