Since the concentration parameter $\alpha$ and the parameters $\mu$ and $\sigma^2$ of the base distribution $N(\mu,\, \sigma^2)$ are unknown,
it is natural to express the uncertainty about $\alpha$, $\mu$, and $\sigma^2$ by assuming that $\alpha$, $\mu$, and $\sigma^2$ have hyperpriors.
%In other words,
%the knowledge about the hyperparameters $\alpha$, $\mu$, and $\sigma^2$ is updated in the light of the observed data:
%The Bayesian Markov chain Monte Carlo algorithm described in Section 6.4 can be used to approximate the marginal posteriors of the hyperparameters $\alpha$, $\mu$, and $\sigma^2$ given the observed data,
%although the hyperparameters are not the primary target of statistical inference.
We assume that the hyperpriors of the hyperparameters $\alpha$, $\mu$, and $1 / \sigma^{2}$ are Gamma, Gaussian, and Gamma distributions,
respectively,
which are conjugate priors and facilitate Markov chain Monte Carlo sampling from the posterior distribution.
