%\usepackage{xstring} % For if-then-else macros
%\usepackage{esint} % for \oiint

%%% Typographic commands
\renewcommand{\bf}[1]{\textbf{#1}}
% This is an underline environment that looks nice
% needs packages 'contour' and 'ulem'
% Following advice of https://alexwlchan.net/2017/10/latex-underlines/
\usepackage{ulem}
\usepackage{contour}
\renewcommand{\ULdepth}{1.8pt}
\contourlength{0.8pt}

\renewcommand{\emph}[1]{%
  \uline{\phantom{#1}}%
    \llap{\contour{white}{#1}}%
    }

%%%%%%%%%%%%%% Hyperlinks
\usepackage{xcolor}
\definecolor{purple}{rgb}{0.75,0.05,0.35}
%\definecolor{magenta}{rgb}{0.75,0.05,0.35}
\usepackage[colorlinks = true,
            linkcolor = blue,
            urlcolor  = blue,
            citecolor = blue,
            anchorcolor = blue,
    filecolor=blue]{hyperref}

%% COMMENTS
\newcommand{\bcomment}[1]{\textcolor{purple}{{#1}}}

%%%%%%%%%%%%% ABBREVIATIONS
\newcommand{\Eqref}[1]{Eq.~\eqref{#1}}
\newcommand{\secref}[1]{Sec.~\ref{#1}}
\def \ie {\textit{i.e.}\ }
\def \eg {\textit{e.g.}\ }
\def \cf {\textit{cf.}\ }
\def \testfnc {f}
\def\ts{\tilde{\omega}}
\def\tt{\tilde{t}}
\def\eps{\varepsilon}
\def\cA{\mathcal{A}}
\def\cB{\mathcal{B}}
\def\cD{\mathcal{D}}
\def\cL{\mathcal{L}}
\def\cM{\mathcal{M}}
\def\cO{\mathcal{O}}
\def\cP{\mathcal{P}}
\def\bP{\mathbb{P}}
\def\cT{\mathcal{T}}
\def\adj{^{\dagger}}

\def\ext{\chi_{\rm{ext}}}

%%%% Math signs
%\def{\Re}{\rm{Re}}
%\def{\Im}{\mathrm{Im}}
\newcommand{\deltabar}{\delta\hspace*{-0.24em}\bar{}\hspace*{0.1em}\,}
\newcommand{\dbar}{\mathrm{d}\hspace*{-0.08em}\bar{}\hspace*{0.1em}}
\newcommand{\dd}[1]{\mathrm{d}{#1}\,}
\newcommand{\He}{\operatorname{He}}
\newcommand{\avg}[1]{\left\langle {#1} \right\rangle}
\newcommand{\yavg}[1]{\overline{#1}}
%%%%%%%% Physical quantitites

\def\deriv{\Delta}

\def\fptxy{\tau_{x_0,x_1}}
\def\xclassic{x^{\rm{cl}}}
\def\dotxclassic{\dot{x}^{\rm{cl}}}
\newcommand{\fpt}[1]{\hat{p}\left(#1\right)}
\newcommand{\trans}[1]{\hat{T}\left(#1\right)}
\newcommand{\return}[1]{\hat{R}\left(#1\right)}

\newcommand{\yfpt}[1]{\hat{p}\left(#1;[y]\right)}
\newcommand{\ytrans}[1]{\hat{T}\left(#1;[y]\right)}
\newcommand{\yreturn}[1]{\hat{R}\left(#1;[y]\right)}
\newcommand{\yinvreturn}[1]{\hat{R}^{-1}\left(#1;[y]\right)}

%\newcommand{\avgfpt}[1]{F\left(#1\right)}
%\newcommand{\havgfpt}[1]{\hat{F}\left(#1\right)}
\newcommand{\avgfpt}[1]{M\left(#1\right)}
\newcommand{\havgfpt}[1]{\hat{M}\left(#1\right)}
\newcommand{\avgtrans}[1]{T\left(#1\right)}
\newcommand{\avgreturn}[1]{R\left(#1\right)}

\newcommand{\transcoeff}[2]{%
    %\IfEqCase{#1}{%
    %    {0}{T\left(#2\right)}%
    %    {1}{T^{\prime}\left(#2\right)}%
    %    {2}{T^{\prime \prime}\left(#2\right)}%
    %}[
\hat{T}^{(#1)}\left(#2\right)%]%
}%

\newcommand{\ntranscoeff}[3]{%
 %   \IfEqCase{#1}{%
  %      {0}{T_{#2}\left(#3\right)}%
   %     {1}{T_{#2}^{\prime}\left(#3\right)}%
    %    {2}{T_{#2}^{\prime \prime}\left(#3\right)}%
    %}[
\hat{T}_{#2}^{(#1)}\left(#3\right)%]%
}%
%\newcommand{\transcoeff}[2]{T^{(#1)}\left(#2\right)}

\newcommand{\returncoeff}[2]{%
 %   \IfEqCase{#1}{%
 %       {0}{R\left(#2\right)}%
 %       {1}{R^{\prime}\left(#2\right)}%
 %       {2}{R^{\prime \prime}\left(#2\right)}%
 %   }[
\hat{R}^{(#1)}\left(#2\right)%]%
}%

%\newcommand{\invreturncoeff}[2]{{\hat{\rho}_{x_1,x_1}^{-1}}{}^{(#1)}\left(#2\right)}
\newcommand{\invreturncoeff}[2]{%
   % \IfEqCase{#1}{%
   %     {0}{R^{-1}\left(#2\right)}%
   %     {1}{(R^{-1})^{\prime}\left(#2\right)}%
   %     {2}{(R^{-1})^{\prime \prime}\left(#2\right)}%
   % }[
(\hat{R}^{-1})^{(#1)}\left(#2\right)%]%
}%

\newcommand{\fullinvreturncoeff}[2]{%
    \IfEqCase{#1}{%
        {0}{\hat{R}^{-1}\left(#2\right)}%
        {1}{(\hat{R}^{-1})^{\prime}\left(#2\right)}%
        {2}{(\hat{R}^{-1})^{\prime \prime}\left(#2\right)}%
    }[(\hat{R}^{-1})^{(#1)}\left(#2\right)]%
}%
%\newcommand{\invreturncoeff}[2]{{R^{-1;(#1)}}\left(#2\right)}

\newcommand{\corr}[1]{\hat{C}_2(#1)}

\def\target{\bar{x}_1}
\def\start{\bar{x}_0}

\def\LL{\mathcal{L}}
%\def\MGFzero{\mathcal{M}_0}
%\def\MGFone{\mathcal{M}_1}
%\def\MGFquenched{\mathcal{M}^{\mathrm{quenched}}}
\def\MGFzero{M_0}
\def\MGFone{M_1}
\def\MGFquenched{M^{\mathrm{quenched}}}


%% DIAGRAMS
\usepackage{tikz}
\usetikzlibrary{decorations.pathmorphing}
\usetikzlibrary{decorations.markings}
\usetikzlibrary{arrows,shapes,snakes,automata,backgrounds,petri}
\usetikzlibrary{calc}
\usetikzlibrary{patterns}
\usetikzlibrary{decorations.text}
% The t-styles are for text, the A-styles are amputated, the S-style
% have shifted arrow.
\tikzset{
	bareprop/.style={very thick,draw=red},
	ynoise/.style={very thick,dashed,draw=blue},
	bareproparrow/.style={very thick,draw=red, postaction={decorate},
decoration={markings,mark=at position .5 with {\arrow[draw=red]{>}}}},
}
\newcommand{\coupling}[1]{\draw[color=black,fill=black] (#1) circle (.15em);}
\newcommand{\lcoupling}[1]{\draw[color=black,fill=black] (#1) circle (.15em);
\draw[color=black] (#1) -- ++(0.1,0.2); }
\newcommand{\source}[1]{\draw[color=black,fill=black] (#1) node {$\times$};}
\newcommand{\prop}[1]{\draw[bareprop] (#1,0) -- (0,0);}

% Droplet taken from https://tex.stackexchange.com/questions/400017/how-to-draw-a-water-droplet-in-latex
\newcommand{\droplet}[2]{%
    	\coordinate (a) at (#1,#2);% Change 1.5 to change the shape of the droplet
	\node [circle,draw,fill=magenta,magenta,inner sep = 2 pt] (c) at (#1,#2+0.4) {$c$};
    \draw[magenta,fill] (a) -- (tangent cs:node=c,point={(a)},solution=1) --
    (c.center) -- (tangent cs:node=c,point={(a)},solution=2) -- cycle;
}

\newcommand{\fptprop}[2]{%
	\draw[color=magenta,thick,double] (#1,#2)--(#1+1,#2);
}

\newcommand{\transprop}[2]{%
	\draw[color=magenta,thick] (#1,#2)--(#1+1,#2);
}
\newcommand{\returnprop}[2]{%
	\draw [color=magenta,thick] (#1,#2) -- (#1+1,#2);
	\draw [color=magenta,thick, domain =(#1):(#1+1)] plot ({\x},{(#2)-sqrt(0.25-(\x-(#1+0.5))*(\x-(#1+0.5)))});
}

\newcommand{\ldroplet}[2]{%
    	\coordinate (a) at (#1,#2);% Change 1.5 to change the shape of the droplet
	\node [circle,draw,fill=black,black,inner sep = 2 pt] (c) at (#1,#2+0.4) {$c$};
    \draw[black,fill] (a) -- (tangent cs:node=c,point={(a)},solution=1) --
    (c.center) -- (tangent cs:node=c,point={(a)},solution=2) -- cycle;
}

\newcommand{\lfptprop}[2]{%
	\draw[double] (#1,#2)--(#1+1,#2);
}

\newcommand{\ltransprop}[2]{%
	\draw (#1,#2)--(#1+1,#2);
}
\newcommand{\lreturnprop}[2]{%
	\draw (#1,#2) -- (#1+1,#2);
	\draw [ domain =(#1):(#1+1)] plot ({\x},{(#2)-sqrt(0.25-(\x-(#1+0.5))*(\x-(#1+0.5)))});
}

\newcommand{\noise}[2]{%
	\draw [magenta, thick, dashed] (#1,#2) -- (#1+0.2, #2+0.5);
}

\newcommand{\avgnoise}[2]{%
	\draw [magenta, thick, dashed] (#1, 0) -- (#2+0.2,0.6) -- (#2, 0);
	\draw[color=magenta,fill=magenta] (#2+0.2,0.6) circle (.15em);
}

%%% 3d PLOT with TIKZ

\usepackage{tikz-3dplot}
