% This is samplepaper.tex, a sample chapter demonstrating the
% LLNCS macro package for Springer Computer Science proceedings;
% Version 2.20 of 2017/10/04
%
\documentclass[runningheads]{llncs}
%
\usepackage{comment}
\usepackage{graphicx}
\usepackage{cite}
\usepackage{colortbl}
\usepackage{indentfirst}
\usepackage{color}
\usepackage[utf8]{inputenc} % allow utf-8 input
\usepackage[T1]{fontenc}    % use 8-bit T1 fonts
%\usepackage{hyperref}       % hyperlinks
\usepackage{url}            % simple URL typesetting
\usepackage{booktabs}       % professional-quality tables
\usepackage{amsfonts}       % blackboard math symbols
\usepackage{nicefrac}       % compact symbols for 1/2, etc.
\usepackage{microtype}      % microtypography
%\usepackage{xcolor}         % colors
\usepackage{authblk}
\usepackage{hyperref}
\usepackage[cmyk]{xcolor}
% Used for displaying a sample figure. If possible, figure files should
% be included in EPS format.
%
% If you use the hyperref package, please uncomment the following line
% to display URLs in blue roman font according to Springer's eBook style:
% \renewcommand\UrlFont{\color{blue}\rmfamily}

\begin{document}
%
% \title{AIGCIQA2023:  A Large-scale Image Quality Assessment Database for AI Generated Images: from the Perspectives of Quality, Authenticity \\ and Correspondence }

% }


%\end{comment}
%******************

% CAMERA READY SUBMISSION
% \begin{comment}

\title{AIGCIQA2023:  A Large-scale Image Quality Assessment Database for AI Generated Images: from the Perspectives of Quality, Authenticity \\ and Correspondence}
% If the paper title is too long for the running head, you can set
% an abbreviated paper title here
%
\author{Jiarui Wang\inst{1}\and
Huiyu Duan\inst{1}\and
Jing Liu\inst{2}\and
Shi Chen\inst{3}\\
Xiongkuo Min\inst{1}$^*$, and Guangtao Zhai\inst{1}\thanks{Corresponding Authors.}
}

\authorrunning{F. Author et al.}

\institute{Shanghai Jiao Tong University, Shanghai, China \\
\email{\{wangjiarui,huiyuduan,minxiongkuo,zhaiguangtao\}@sjtu.edu.cn}
\and Tianjin University, Tianjin, China\\
\and Shanghai Second Polytechnic University, Shanghai, China\\}
% \end{comment}
%******************
\maketitle              % typeset the header of the contribution
\begin{abstract}
    
    Recent years have witnessed a rapid growth of Artificial Intelligence Generated Content (AIGC), among which with the development of text-to-image techniques, AI-based image generation has been applied to various fields.
    However, AI Generated Images (AIGIs) may have some unique distortions compared to natural images, thus many generated images are not qualified for real-world applications. 
    Consequently, it is important and significant to study subjective and objective Image Quality Assessment (IQA) methodologies for AIGIs. 
    %The human preference is a missing dimension of image quality that is not well tracked by existing mainstream evaluation metrics
    In this paper, in order to get a better understanding of the human visual preferences for AIGIs, a large-scale IQA database for AIGC is established, which is named as AIGCIQA2023. We first generate over 2000 images based on 6 state-of-the-art text-to-image generation models using 100 prompts.
    Based on these images, a well-organized subjective experiment is conducted to assess the human visual preferences for each image from three perspectives including \emph{\textbf{quality}}, \emph{\textbf{authenticity}} and \emph{\textbf{correspondence}}. 
    Finally, based on this large-scale database, we conduct a benchmark experiment to evaluate the performance of several state-of-the-art IQA metrics on our constructed database. The AIGCIQA2023 database and benchmark will be released to facilitate future research on \textcolor{magenta!100}{\url{https://github.com/wangjiarui153/AIGCIQA2023}}
    %%

\keywords{AI generated content (AIGC) \and text-to-image generation \and image quality assessment \and human visual preference }
\end{abstract}

%provides a more complete evaluation for individual text-to-image generations, making it better aligned with human preferences.
%
%
\section{Introduction}
% introduction of AIGC and GIQA
Artificial Intelligence Generated Content (AIGC) refers to the content, including texts, images, audios, or videos, \textit{etc.}, that is created or generated with the assistance of AI technology.
Many impressive AIGC models have been developed in recent years, such as ChatGPT %\cite{kulkarni11} 
and DALLE\cite{ramesh2022hierarchical}, which have been utilized in various application scenarios. 
As an important part of AIGC, AI Generated Images (AIGIs) have also gained significant attention in recent years due to advancement in generative models including Generative Adversarial Network (GAN) \cite{goodfellow2020generative}, Variational Autoencoder (VAE) \cite{kingma2013auto}, diffusion models \cite{Rombach2021HighResolutionIS}, \textit{etc.}, and language-image pre-training techniques including CLIP\cite{radford2021learning}, BLIP\cite{li2022blip}, \textit{etc.}

%current dataset
However, the development of AIGI models also raises new problems and challenges. 
One significant challenge is that not all generated images are qualified for real-world applications, which often require to be processed, adjusted, refined or filtered out before being applied to practical scenes.
However, unlike common image content, such as Natural Scene Images (NSIs)\cite{duan2017ivqad,duan2018perceptual}, screen content images\cite{min2017unified,duan2022confusing}, graphic images\cite{min2017unified,duan2022saliency}, \textit{etc.}, which generally encounters some common distortions including  noise, blur, compression, \emph{etc.} \cite{duan2023masked,duan2022develop}, AIGIs may suffer from some unique degradations such as unreal structures, unreasonable combinations, \textit{etc}. Moreover, the generated images may not correspond to the semantics of the text prompts \cite{lee2023aligning,kirstain2023pick,xu2023imagereward}.
Therefore, it is important to study the human visual preferences for AIGIs and design corresponding objective Image Quality Assessment (IQA) metrics for these images.

%current method
Many subjective IQA studies have been conducted for human captured or created images, and many objective IQA models have also been developed.
However, these models are designed for assessing low-level distortions, while AIGIs generally contain both low-level artifacts and high-level semantic degradations.
Some quantitative evaluation metrics such as Inception Score (IS)\cite{gulrajani2017improved} and Fréchet Inception Distance (FID)\cite{heusel2017gans} have been proposed to assess the performance of generative models and have been widely used to evaluate the authenticity of the generated images.
However, these methods cannot evaluate the authenticity of a single generated image, and cannot measure the correspondence between the generated images and the text-prompts.
As a new type of image content, previous IQA methods may fail to assess the image quality of AIGIs and cannot align well with human preferences due to the irregular distortions. 

%Our dataset and IQA method
To gain a better understanding of human visual preferences for AIGIs and guide the design process of corresponding objective IQA models, in this paper, we conduct a comprehensive subjective and objective IQA study for AIGIs.
We first establish a large-scale IQA database for AIGIs termed AIGCIQA2023, which contains 2,400 diverse images generated by 6 state-of-the-art AIGI models based on 100 various text prompts. 
Based on these images, a well-organized subjective experiment is conducted to assess the human visual preferences for each individual generated image from three perspectives including 
\textbf{\textit{quality}}, \textbf{\textit{authenticity}}, and \textbf{\textit{correspondence}}. Based on the constructed AIGCIQA2023 database, we evaluate the performance of several state-of-the-art IQA models and establish a new benchmark. Experimental results demonstrate that current IQA methods cannot well align with human visual preferences for AIGIs, and more efforts should be made in this research field in the future. The main contributions of this paper are summarized as follows:

\begin{itemize}


\item We propose to disentangle the human visual experience for AIGIs into three perspectives including \textbf{\textit{quality}}, \textbf{\textit{authenticity}}, and \textbf{\textit{correspondence}}.
\item Based on the above theory, we establish a novel large-scale database, \textit{i.e.,} AIGCIQA2023, to better understand the human visual preferences for AIGIs and guide the design of objective IQA models.
\item We conduct a benchmark experiment to
evaluate the performance of several current state-of-the-art IQA algorithms in measuring the quality, authenticity, and text-image correspondence of AIGIs.

\end{itemize}

The rest of the paper is organized as follows. In Section 2 we introduce the details of our constructed AIGCIQA2023 database, including the generation of AIGIs and the subjective quality assessment methodology and procedures.
In section 3 we present the benchmark experiment for current state-of-the-art IQA algorithms based on the established database. 
Section 4 concludes the whole paper and we discuss possible future research that can be conducted with the database.


\section{Database Construction and Analysis}
In order to get a better understanding of human visual preferences for AI-generated images based on text prompts, we construct a novel IQA database for AIGIs, termed AIGCIQA2023, which is a collection of generated images derived from six state-of-the-art deep generative models based on 100 text prompts, and corresponding subjective quality ratings from three different perspectives.
Then we further analyze the human visual preferences for AIGIs based on the constructed database.

\subsection{AIGI Collection}
\begin{figure}[t]
  \centering
  \includegraphics[width=0.9\textwidth]{figures/mycircle.pdf}
  \caption{Pie Chart of the ten challenge categories and ten scene categories selected from PartiPrompts\cite{yu2022scaling}. }
\end{figure}
We adopt six latest text-to-image generative models, including Glide\cite{Nichol2021GLIDETP}, Lafite\cite{Zhou_2022_CVPR}, DALLE\cite{ramesh2022hierarchical}, Stable-diffusion\cite{Rombach2021HighResolutionIS}, Unidiffuser\cite{Bao2023OneTF}, Controlnet\cite{Zhang2023AddingCC}, to produce AIGIs by using open source code and default weights.
To ensure content diversity and catch up with the practical application requirements, we collect diverse texts from the PartiPrompts website \cite{yu2022scaling} as the prompts for AIGI generation.
The text prompts can be simple, allowing generative models to produce imaginative results.
They can also be complex, which raises the challenge for generative models.
We select 10 scene categories from the prompt set, and each scene contains 10 challenge categories.
Overall, we collect 100 text prompts (10 scene categories $\times$ 10 challenge categories) from PartiPrompts\cite{yu2022scaling}.
The distribution of the selected scene and challenge categories is displayed in pie chart of Fig.1.
It can be observed that the dataset exhibits a high level of scene diversity, with images generated covering a broad range of challenges.
Then we perform the text-to-image generation based on these models and prompts. Specifically, for each prompt, we generate 4 various images randomly for each generative model. Therefore, the constructed AIGCIQA2023 database totally contains 2400 AIGIs (4 images $\times$ 6 models $\times$ 100 prompts) corresponding to 100 prompts.
\begin{figure}[t]
  \centering
  \includegraphics[width=\textwidth]{figures/samples_imgs.pdf}
  \caption{Sample images from the AIGCIQA2023 database generated by six different generative models (Glide\cite{Nichol2021GLIDETP}, Lafite\cite{Zhou_2022_CVPR}, DALLE\cite{ramesh2022hierarchical}, \
  Stable-diffusion\cite{Rombach2021HighResolutionIS}, Unidiffuser\cite{Bao2023OneTF}, Controlnet\cite{Zhang2023AddingCC}.)}
\end{figure}

\subsection{Subjective Experiment Setup}

Subjective IQA is the most reliable way to evaluate the visual quality of digital images perceived by the users. 
It is generally used to construct image quality datasets and served as the ground truth to optimize or evaluate the performance of objective quality assessment metrics. 
Due to the unnatural property of AIGIs and different text prompts having different target image spaces, it is unreasonable to just use one score, \textit{i.e.,} ``quality'' to represent human visual preferences.
In this paper, we propose to measure the human visual preferences of AIGIs from three perspectives including \textbf{\textit{quality}}, \textbf{\textit{authenticity}}, and text-image \textbf{\textit{correspondence}}.
For an image, these three visual perception perspectives are related but different.
\begin{figure}[t]
%   \centering
%   \includegraphics[width=\textwidth]{figures/corgi.png}
%   \caption{Images generated by the prompt of “a corgi” with top10 high quality and top10 low quality.}
% ~\\
% ~\\
%   \centering
%   \includegraphics[width=\textwidth]{figures/girl.png}
%   \caption{Images generated by the prompt of “a girl” with top10 high  reality and top10 low  reality.}
% ~\\
% ~\\
  \centering
  \includegraphics[width=\textwidth]{figures/IMG.pdf}
  \caption{Illustration of the images from the perspectives of quality, authenticity, and text-image correspondence.
  (a)  10 high quality examples
and 10 low quality examples of the images generated by the prompt of “a corgi”.
  (b)  10 high authenticity and
10 low authenticity examples of images generated by the prompt of “a girl”.
  (c)  10 high text-image correspondence and 10 low correspondence
examples of images generated by the prompt of “a grandmother reading a book
to her grandson and granddaughter”.
}

\end{figure}

The first dimension of AIGI evaluation is ``quality'' evaluation, \textit{i.e.,} evaluating an AIGI from its clarity, color, lightness, contrast, \textit{etc.}, which is similar to the assessment of NSIs.
During the experiment procedure, subjects are instructed to evaluate whether the image outline is clear, whether the content can be distinguished, and the richness of details, \textit{etc.} 
Fig.3 (a) shows 10 high quality examples and 10 low quality examples of the images generated by the prompt of “a corgi”.

Considering the generation nature of AIGIs, an important problem of these images is that they may not look real compared to NSIs.
Therefore, we introduce a second dimension of evaluation metrics for the generated images, \textit{i.e.,} ``authenticity'' evaluation.
For this dimension, subjects are instructed to assess the image from the authenticity aspect, \textit{i.e.,} whether it looks real or whether they can distinguish that the image is AI-generated or not. 
Fig.3 (b) shows 10 high authenticity and 10 low authenticity examples of images generated by the prompt of ``a girl''.

Since an AIGI is generated from a text, it is also important to evaluate its correspondence with the original prompt, \textit{i.e.,} the third dimension, text-image ``correspondence''. 
For this purpose, subjects are instructed to consider textual information provided with the image and then give the correspondence score from 0 to 5 to assess the relevance between the generated image and its prompt. 
Fig.3 (c) shows 10 high text-image correspondence and 10 low correspondence examples of images generated by the prompt of “a grandmother reading a book to her grandson and granddaughter''.
\begin{figure}[t]
  \centering
  \includegraphics[width=0.9\textwidth]{figures/ui.pdf}
  \caption{An example of the subjective assessment interface. The subject can  evaluate the quality of AIGIs and record the quality, authenticity, correspondence scores with the scroll bar on the right.}
\end{figure}
\vspace{-2pt}
\subsection{Subjective Experiment Procedure}
To evaluate the quality of the images in the AIGCIQA2023 and obtain Mean Opinion Scores (MOSs), a subjective experiment is conducted following the guidelines of ITU-R BT.500-14 \cite{duan2022confusing}. 
The subjects are asked to rate their visual preference degree of exhibited AIGIs from the quality, authenticity and text-image correspondence.
The AIGIs are presented in a random order on an iMac monitor with a resolution of up to 4096 × 2304, using an interface designed with Python Tkinter, as shown in Fig.4. The interface allows viewers to browse the previous and next AIGIs and rate them using a quality scale that ranges from 0 to 5, with a minimum interval of 0.01. A total of 28 graduate students (14 males and 14 females) participate in the experiment, and they are seated at a distance of around 60 cm in a laboratory  environment with normal indoor lighting.





\subsection{Subjective Data Processing}

We follow the suggestions recommended by ITU to conduct the outlier detection and subject rejection. 
The score rejection rate is 2\%.
In order to obtain the MOS for an AIGI, we first convert the raw ratings into Z-scores, then linearly scale them to the range $[0,100]$ as follows:
$$z_i{}_j=\frac{r_i{}_j-\mu_i{}_j}{\sigma_i},\quad z_{ij}'=\frac{100(z_{ij}+3)}{6},$$
$$\mu_i=\frac{1}{N_i}\sum_{j=1}^{N_i}r_i{}_j, ~~ \sigma_i=\sqrt{\frac{1}{N_i-1}\sum_{j=1}^{N_i}{(r_i{}_j-\mu_i{}_j)^2}}$$ 
where $r_{ij}$ is the raw ratings given by the $i$-th subject to the $j$-th image. $N_i$ is the number of images judged by subject $i$. 

Next, the mean opinion score (MOS) of the image j is computed by averaging the rescaled z-scores as follows:
$$MOS_j=\frac{1}{M}\sum_{i=1}^{M}z_{ij}'$$
where $MOS_j$ indicates the MOS for the $j$-th AIGI, $M$ is the number of valid subjects, and $z'_i{}_j$ are the rescaled z-scores. 
%%
%% Beginning of file 'sample.tex'
%%
%% Modified 2005 December 5
%%
%% This is a sample manuscript marked up using the
%% AASTeX v5.x LaTeX 2e macros.

%% The first piece of markup in an AASTeX v5.x document
%% is the \documentclass command. LaTeX will ignore
%% any data that comes before this command.

%% The command below calls the preprint style
%% which will produce a one-column, single-spaced document.
%% Examples of commands for other substyles follow. Use
%% whichever is most appropriate for your purposes.
%%
%%\documentclass[12pt,preprint]{aastex}

%% manuscript produces a one-column, double-spaced document:

\documentclass[manuscript]{aastex}

%% preprint2 produces a double-column, single-spaced document:

%% \documentclass[preprint2]{aastex}

%% Sometimes a paper's abstract is too long to fit on the
%% title page in preprint2 mode. When that is the case,
%% use the longabstract style option.

%% \documentclass[preprint2,longabstract]{aastex}

%% If you want to create your own macros, you can do so
%% using \newcommand. Your macros should appear before
%% the \begin{document} command.
%%
%% If you are submitting to a journal that translates manuscripts
%% into SGML, you need to follow certain guidelines when preparing
%% your macros. See the AASTeX v5.x Author Guide
%% for information.

\newcommand{\vdag}{(v)^\dagger}
\newcommand{\myemail}{skywalker@galaxy.far.far.away}

%% You can insert a short comment on the title page using the command below.

\slugcomment{Not to appear in Nonlearned J., 45.}

%% If you wish, you may supply running head information, although
%% this information may be modified by the editorial offices.
%% The left head contains a list of authors,
%% usually a maximum of three (otherwise use et al.).  The right
%% head is a modified title of up to roughly 44 characters.
%% Running heads will not print in the manuscript style.

\shorttitle{Collapsed Cores in Globular Clusters}
\shortauthors{Djorgovski et al.}

%% This is the end of the preamble.  Indicate the beginning of the
%% paper itself with \begin{document}.

\begin{document}

%% LaTeX will automatically break titles if they run longer than
%% one line. However, you may use \\ to force a line break if
%% you desire.

\title{Collapsed Cores in Globular Clusters, \\
    Gauge-Boson Couplings, and AAS\TeX\ Examples}

%% Use \author, \affil, and the \and command to format
%% author and affiliation information.
%% Note that \email has replaced the old \authoremail command
%% from AASTeX v4.0. You can use \email to mark an email address
%% anywhere in the paper, not just in the front matter.
%% As in the title, use \\ to force line breaks.

\author{S. Djorgovski\altaffilmark{1,2,3} and Ivan R. King\altaffilmark{1}}
\affil{Astronomy Department, University of California,
    Berkeley, CA 94720}

\author{C. D. Biemesderfer\altaffilmark{4,5}}
\affil{National Optical Astronomy Observatories, Tucson, AZ 85719}
\email{aastex-help@aas.org}

\and

\author{R. J. Hanisch\altaffilmark{5}}
\affil{Space Telescope Science Institute, Baltimore, MD 21218}

%% Notice that each of these authors has alternate affiliations, which
%% are identified by the \altaffilmark after each name.  Specify alternate
%% affiliation information with \altaffiltext, with one command per each
%% affiliation.

\altaffiltext{1}{Visiting Astronomer, Cerro Tololo Inter-American Observatory.
CTIO is operated by AURA, Inc.\ under contract to the National Science
Foundation.}
\altaffiltext{2}{Society of Fellows, Harvard University.}
\altaffiltext{3}{present address: Center for Astrophysics,
    60 Garden Street, Cambridge, MA 02138}
\altaffiltext{4}{Visiting Programmer, Space Telescope Science Institute}
\altaffiltext{5}{Patron, Alonso's Bar and Grill}

%% Mark off your abstract in the ``abstract'' environment. In the manuscript
%% style, abstract will output a Received/Accepted line after the
%% title and affiliation information. No date will appear since the author
%% does not have this information. The dates will be filled in by the
%% editorial office after submission.

\begin{abstract}
This is a preliminary report on surface photometry of the major
fraction of known globular clusters, to see which of them show the signs
of a collapsed core.
We also explore some diversionary mathematics and recreational tables.
\end{abstract}

%% Keywords should appear after the \end{abstract} command. The uncommented
%% example has been keyed in ApJ style. See the instructions to authors
%% for the journal to which you are submitting your paper to determine
%% what keyword punctuation is appropriate.

\keywords{globular clusters: general --- globular clusters: individual(NGC 6397,
NGC 6624, NGC 7078, Terzan 8}

%% From the front matter, we move on to the body of the paper.
%% In the first two sections, notice the use of the natbib \citep
%% and \citet commands to identify citations.  The citations are
%% tied to the reference list via symbolic KEYs. The KEY corresponds
%% to the KEY in the \bibitem in the reference list below. We have
%% chosen the first three characters of the first author's name plus
%% the last two numeral of the year of publication as our KEY for
%% each reference.


%% Authors who wish to have the most important objects in their paper
%% linked in the electronic edition to a data center may do so by tagging
%% their objects with \objectname{} or \object{}.  Each macro takes the
%% object name as its required argument. The optional, square-bracket 
%% argument should be used in cases where the data center identification
%% differs from what is to be printed in the paper.  The text appearing 
%% in curly braces is what will appear in print in the published paper. 
%% If the object name is recognized by the data centers, it will be linked
%% in the electronic edition to the object data available at the data centers  
%%
%% Note that for sources with brackets in their names, e.g. [WEG2004] 14h-090,
%% the brackets must be escaped with backslashes when used in the first
%% square-bracket argument, for instance, \object[\[WEG2004\] 14h-090]{90}).
%%  Otherwise, LaTeX will issue an error. 

\section{Introduction}

A focal problem today in the dynamics of globular clusters is core collapse.  
It has been predicted by theory for decades \citep{hen61,lyn68,spi85}, but
observation has been less alert to the phenomenon. For many years the
central brightness peak in M15 \citep{kin75,new78} seemed a unique anomaly.  
Then \citet{aur82} suggested a central peak in \object{NGC 6397}, and a 
limited photographic survey of ours \citep[Paper I]{djo84} found three more 
cases, \objectname{NGC 6624}, \objectname[M 15]{NGC 7078}, and 
\object[Cl 1938-341]{Terzan 8}), whose sharp center had often been 
remarked on \citep{can78}.  

As an example of how the new AASTeX object tagging macros work, we will
cite some of the ``Superlative'' objects mentioned in section 10 of
Trimble's (1992) review of astrophysics in the year 1991.  The youngest
star yet found was \object[\[JCC87\] IRAS 4]{IRAS 4} in \objectname{NGC 1333}.
\object{70 Oph} was found to be the longest period spectroscopic binary.
The most massive white dwarf was \object{GD 50}, estimated at 1.2 solar masses.
The first neutral hydrogen found in a globular cluster was \object{NGC 2808}
while the \objectname[SDSS J093401.92+551427.9]{I Zw 18} retained the
record for metal deficiency.  However, another low metallicitity
galaxy was \object{UGC 4483} in the \objectname{M 83} group. The
largest redshift \object[PC 1247+3406]{source} in 1991 was found at
z=4.897.  Lastly, what paper would be complete without a mention of the
\object[M1]{Crab nebula}!

\section{Observations}

%% In a manner similar to \objectname authors can provide links to dataset
%% hosted at participating data centers via the \dataset{} command.  The
%% second curly bracket argument is printed in the text while the first
%% parentheses argument serves as the valid data set identifier.  Large
%% lists of data set are best provided in a table (see Table 3 for an example).
%% Valid data set identifiers should be obtained from the data center that
%% is currently hosting the data.
%%
%% Note that AASTeX interprets everything between the curly braces in the 
%% macro as regular text, so any special characters, e.g. "#" or "_," must be 
%% preceded by a backslash. Otherwise, you will get a LaTeX error when you 
%% compile your manuscript.  Special characters do not 
%% need to be escaped in the optional, square-bracket argument.

All our observations were short direct exposures with CCD's.  We also have
a random {\it Chandra} data set \dataset{ADS/Sa.ASCA\#X/86008020} and a
neat \dataset[ADS/Sa.HST#Y0Q70101T]{HST FOS spectrum} that readers can access
via the links in the electronic edition.  Unfortunately this has nothing 
whatsoever to do with this research.  At Lick Observatory we used a TI 500$\times$500 
chip and a GEC 575$\times$385, on the 1-m Nickel reflector.  The only
filter available at Lick was red.  At CTIO we used a GEC 575$\times$385, with
$B, V,$ and $R$ filters, and an RCA 512$\times$320, with $U, B, V, R,$ and $I$
filters, on the 1.5-m reflector. In the CTIO observations we tried to
concentrate on the shortest practicable wavelengths; but faintness, reddening,
and poor short-wavelength sensitivity often kept us from observing in $U$ or
even in $B$. All four cameras had scales of the order of 0.4 arcsec/pixel, and
our field sizes were around 3 arcmin.

The CCD images are unfortunately not always suitable, for very poor
clusters or for clusters with large cores.  Since the latter are easily
studied by other means, we augmented our own CCD profiles by collecting
from the literature a number of star-count profiles 
\citep{kin68,pet76,har84,ort85}, as well as photoelectric profiles 
\citep{kin66,kin75} and electronographic profiles \citep{kro84}.
In a few cases we judged normality by eye estimates on one of the Sky
Surveys.

%% In this section, we use  the \subsection command to set off
%% a subsection.  \footnote is used to insert a footnote to the text.

%% Observe the use of the LaTeX \label
%% command after the \subsection to give a symbolic KEY to the
%% subsection for cross-referencing in a \ref command.
%% You can use LaTeX's \ref and \label commands to keep track of
%% cross-references to sections, equations, tables, and figures.
%% That way, if you change the order of any elements, LaTeX will
%% automatically renumber them.

%% This section also includes several of the displayed math environments
%% mentioned in the Author Guide.

\section{Helicity Amplitudes}

It has been realized that helicity amplitudes provide a convenient means
for Feynman diagram\footnote{Footnotes can be inserted like this.}
evaluations.  These amplitude-level techniques
are particularly convenient for calculations involving many Feynman
diagrams, where the usual trace techniques for the amplitude
squared becomes unwieldy.  Our calculations use the helicity techniques
developed by other authors \cite[]{hag86}; we briefly summarize below.

\subsection{Formalism} \label{bozomath}

%% The equation environment wil produce a numbered display equation.

A tree-level amplitude in $e^+e^-$ collisions can be expressed in
terms of fermion strings of the form
\begin{equation}
\bar v(p_2,\sigma_2)P_{-\tau}\hat a_1\hat a_2\cdots
\hat a_nu(p_1,\sigma_1) ,
\end{equation}
where $p$ and $\sigma$ label the initial $e^{\pm}$ four-momenta
and helicities $(\sigma = \pm 1)$, $\hat a_i=a^\mu_i\gamma_\nu$
and $P_\tau=\frac{1}{2}(1+\tau\gamma_5)$ is a chirality projection
operator $(\tau = \pm1)$.  The $a^\mu_i$ may be formed from particle
four-momenta, gauge-boson polarization vectors or fermion strings with
an uncontracted Lorentz index associated with final-state fermions.

%% The \notetoeditor{TEXT} command allows the author to communicate
%% information to the copy editor.  This information will appear as a
%% footnote on the printed copy for the manuscript style file.  Nothing will
%% appear on the printed copy if the preprint or
%% preprint2 style files are used.

%% The eqnarray environment produces multi-line display math. The end of
%% each line is marked with a \\. Lines will be numbered unless the \\
%% is preceded by a \nonumber command.
%% Alignment points are marked by ampersands (&). There should be two
%% ampersands (&) per line.

In the chiral \notetoeditor{Figures 1 and 2 should appear side-by-side in
print} representation the $\gamma$ matrices are expressed
in terms of $2\times 2$ Pauli matrices $\sigma$ and the unit matrix 1 as
\begin{eqnarray}
\gamma^\mu  & = &
 \left(
\begin{array}{cc}
0 & \sigma^\mu_+ \\
\sigma^\mu_- & 0
\end{array}     \right) ,
 \gamma^5= \left(
\begin{array}{cc}
-1 &   0\\
0 &   1
\end{array}     \right)  , \nonumber \\
\sigma^\mu_{\pm}  & = &   ({\bf 1} ,\pm \sigma) , \nonumber
\end{eqnarray}
giving
\begin{equation}
\hat a= \left(
\begin{array}{cc}
0 & (\hat a)_+\\
(\hat a)_- & 0
\end{array}\right), (\hat a)_\pm=a_\mu\sigma^\mu_\pm ,
\end{equation}
The spinors are expressed in terms of two-component Weyl spinors as
\begin{equation}
u=\left(
\begin{array}{c}
(u)_-\\
(u)_+
\end{array}\right), v={\bf (}\vdag_+{\bf ,}   \vdag_-{\bf )} .
\end{equation}

%% Putting eqnarrays or equations inside the mathletters environment groups
%% the enclosed equations by letter. For instance, the eqnarray below, instead
%% of being numbered, say, (4) and (5), would be numbered (4a) and (4b).
%% LaTeX the paper and look at the output to see the results.

The Weyl spinors are given in terms of helicity eigenstates
$\chi_\lambda(p)$ with $\lambda=\pm1$ by
\begin{mathletters}
\begin{eqnarray}
u(p,\lambda)_\pm & = & (E\pm\lambda|{\bf p}|)^{1/2}\chi_\lambda(p) , \\
v(p,\lambda)_\pm & = & \pm\lambda(E\mp\lambda|{\bf p}|)^{1/2}\chi
_{-\lambda}(p)
\end{eqnarray}
\end{mathletters}

%% This section contains more display math examples, including unnumbered
%% equations (displaymath environment). The last paragraph includes some
%% examples of in-line math featuring a couple of the AASTeX symbol macros.

\section{Floating material and so forth}

%% The displaymath environment will produce the same sort of equation as
%% the equation environment, except that the equation will not be numbered
%% by LaTeX.

Consider a task that computes profile parameters for a modified
Lorentzian of the form
\begin{equation}
I = \frac{1}{1 + d_{1}^{P (1 + d_{2} )}}
\end{equation}
where
\begin{displaymath}
d_{1} = \sqrt{ \left( \begin{array}{c} \frac{x_{1}}{R_{maj}}
\end{array} \right) ^{2} +
\left( \begin{array}{c} \frac{y_{1}}{R_{min}} \end{array} \right) ^{2} }
\end{displaymath}
\begin{displaymath}
d_{2} = \sqrt{ \left( \begin{array}{c} \frac{x_{1}}{P R_{maj}}
\end{array} \right) ^{2} +
\left( \begin{array}{c} \case{y_{1}}{P R_{min}} \end{array} \right) ^{2} }
\end{displaymath}
\begin{displaymath}
x_{1} = (x - x_{0}) \cos \Theta + (y - y_{0}) \sin \Theta
\end{displaymath}
\begin{displaymath}
y_{1} = -(x - x_{0}) \sin \Theta + (y - y_{0}) \cos \Theta
\end{displaymath}

In these expressions $x_{0}$,$y_{0}$ is the star center, and $\Theta$ is the
angle with the $x$ axis.  Results of this task are shown in table~\ref{tbl-1}.
It is not clear how these sorts of analyses may affect determination of
 $M_{\sun}$, but the assumption is that the alternate results
should be less than 90\degr\ out of phase with previous values.
We have no observations of \ion{Ca}{2}.
Roughly \slantfrac{4}{5} of the electronically submitted abstracts
for AAS meetings are error-free.

%% If you wish to include an acknowledgments section in your paper,
%% separate it off from the body of the text using the \acknowledgments
%% command.

%% Included in this acknowledgments section are examples of the
%% AASTeX hypertext markup commands. Use \url without the optional [HREF]
%% argument when you want to print the url directly in the text. Otherwise,
%% use either \url or \anchor, with the HREF as the first argument and the
%% text to be printed in the second.

\acknowledgments

We are grateful to V. Barger, T. Han, and R. J. N. Phillips for
doing the math in section~\ref{bozomath}.
More information on the AASTeX macros package is available \\ at
\url{http://www.aas.org/publications/aastex}.
For technical support, please write to
\email{aastex-help@aas.org}.

%% To help institutions obtain information on the effectiveness of their
%% telescopes, the AAS Journals has created a group of keywords for telescope
%% facilities. A common set of keywords will make these types of searches
%% significantly easier and more accurate. In addition, they will also be
%% useful in linking papers together which utilize the same telescopes
%% within the framework of the National Virtual Observatory.
%% See the AASTeX Web site at http://www.journals.uchicago.edu/AAS/AASTeX
%% for information on obtaining the facility keywords.

%% After the acknowledgments section, use the following syntax and the
%% \facility{} macro to list the keywords of facilities used in the research
%% for the paper.  Each keyword will be checked against the master list during
%% copy editing.  Individual instruments or configurations can be provided 
%% in parentheses, after the keyword, but they will not be verified.

{\it Facilities:} \facility{Nickel}, \facility{HST (STIS)}, \facility{CXO (ASIS)}.

%% Appendix material should be preceded with a single \appendix command.
%% There should be a \section command for each appendix. Mark appendix
%% subsections with the same markup you use in the main body of the paper.

%% Each Appendix (indicated with \section) will be lettered A, B, C, etc.
%% The equation counter will reset when it encounters the \appendix
%% command and will number appendix equations (A1), (A2), etc.

\appendix

\section{Appendix material}

Consider once again a task that computes profile parameters for a modified
Lorentzian of the form
\begin{equation}
I = \frac{1}{1 + d_{1}^{P (1 + d_{2} )}}
\end{equation}
where
\begin{mathletters}
\begin{displaymath}
d_{1} = \frac{3}{4} \sqrt{ \left( \begin{array}{c} \frac{x_{1}}{R_{maj}}
\end{array} \right) ^{2} +
\left( \begin{array}{c} \frac{y_{1}}{R_{min}} \end{array} \right) ^{2} }
\end{displaymath}
\begin{equation}
d_{2} = \case{3}{4} \sqrt{ \left( \begin{array}{c} \frac{x_{1}}{P R_{maj}}
\end{array} \right) ^{2} +
\left( \begin{array}{c} \case{y_{1}}{P R_{min}} \end{array} \right) ^{2} }
\end{equation}
\begin{eqnarray}
x_{1} & = & (x - x_{0}) \cos \Theta + (y - y_{0}) \sin \Theta \\
y_{1} & = & -(x - x_{0}) \sin \Theta + (y - y_{0}) \cos \Theta
\end{eqnarray}
\end{mathletters}

For completeness, here is one last equation.
\begin{equation}
e = mc^2
\end{equation}

%% The reference list follows the main body and any appendices.
%% Use LaTeX's thebibliography environment to mark up your reference list.
%% Note \begin{thebibliography} is followed by an empty set of
%% curly braces.  If you forget this, LaTeX will generate the error
%% "Perhaps a missing \item?".
%%
%% thebibliography produces citations in the text using \bibitem-\cite
%% cross-referencing. Each reference is preceded by a
%% \bibitem command that defines in curly braces the KEY that corresponds
%% to the KEY in the \cite commands (see the first section above).
%% Make sure that you provide a unique KEY for every \bibitem or else the
%% paper will not LaTeX. The square brackets should contain
%% the citation text that LaTeX will insert in
%% place of the \cite commands.

%% We have used macros to produce journal name abbreviations.
%% AASTeX provides a number of these for the more frequently-cited journals.
%% See the Author Guide for a list of them.

%% Note that the style of the \bibitem labels (in []) is slightly
%% different from previous examples.  The natbib system solves a host
%% of citation expression problems, but it is necessary to clearly
%% delimit the year from the author name used in the citation.
%% See the natbib documentation for more details and options.

\begin{thebibliography}{}
\bibitem[Auri\`ere(1982)]{aur82} Auri\`ere, M.  1982, \aap,
    109, 301
\bibitem[Canizares et al.(1978)]{can78} Canizares, C. R.,
    Grindlay, J. E., Hiltner, W. A., Liller, W., \&
    McClintock, J. E.  1978, \apj, 224, 39
\bibitem[Djorgovski \& King(1984)]{djo84} Djorgovski, S.,
    \& King, I. R.  1984, \apjl, 277, L49
\bibitem[Hagiwara \& Zeppenfeld(1986)]{hag86} Hagiwara, K., \&
    Zeppenfeld, D.  1986, Nucl.Phys., 274, 1
\bibitem[Harris \& van den Bergh(1984)]{har84} Harris, W. E.,
    \& van den Bergh, S.  1984, \aj, 89, 1816
\bibitem[H\`enon(1961)]{hen61} H\'enon, M.  1961, Ann.d'Ap., 24, 369
\bibitem[Heiles \& Troland(2003)]{heiles03} Heiles, C. \& Troland, T. H., 2003, \apjs, preprint doi:10.1086/381753
\bibitem[Kim, Ostricker, \& Stone(2003)]{kim03} Kim, W.-T.,  Ostriker, E., \& Stone, J. M., 2003, \apj, 599, 1157
\bibitem[King(1966)]{kin66}  King, I. R.  1966, \aj, 71, 276
\bibitem[King(1975)]{kin75}  King, I. R.  1975, Dynamics of
    Stellar Systems, A. Hayli, Dordrecht: Reidel, 1975, 99
\bibitem[King et al.(1968)]{kin68}  King, I. R., Hedemann, E.,
    Hodge, S. M., \& White, R. E.  1968, \aj, 73, 456
\bibitem[Kron et al.(1984)]{kro84} Kron, G. E., Hewitt, A. V.,
    \& Wasserman, L. H.  1984, \pasp, 96, 198
\bibitem[Lynden-Bell \& Wood(1968)]{lyn68} Lynden-Bell, D.,
    \& Wood, R.  1968, \mnras, 138, 495
\bibitem[Newell \& O'Neil(1978)]{new78} Newell, E. B.,
    \& O'Neil, E. J.  1978, \apjs, 37, 27
\bibitem[Ortolani et al.(1985)]{ort85} Ortolani, S., Rosino, L.,
    \& Sandage, A.  1985, \aj, 90, 473
\bibitem[Peterson(1976)]{pet76} Peterson, C. J.  1976, \aj, 81, 617
\bibitem[Rudnick et al.(2003)]{rudnick03} Rudnick, G. et al., 2003, \apj, 599, 847
\bibitem[Spitzer(1985)]{spi85} Spitzer, L.  1985, Dynamics of
    Star Clusters, J. Goodman \& P. Hut, Dordrecht: Reidel, 109
\bibitem[Treu et al.(2003)]{treu03} Treu, T. et al., 2003, \apj, 591, 53
\end{thebibliography}

\clearpage

%% Use the figure environment and \plotone or \plottwo to include
%% figures and captions in your electronic submission.
%% To embed the sample graphics in
%% the file, uncomment the \plotone, \plottwo, and
%% \includegraphics commands
%%
%% If you need a layout that cannot be achieved with \plotone or
%% \plottwo, you can invoke the graphicx package directly with the
%% \includegraphics command or use \plotfiddle. For more information,
%% please see the tutorial on "Using Electronic Art with AASTeX" in the
%% documentation section at the AASTeX Web site,
%% http://www.journals.uchicago.edu/AAS/AASTeX.
%%
%% The examples below also include sample markup for submission of
%% supplemental electronic materials. As always, be sure to check
%% the instructions to authors for the journal you are submitting to
%% for specific submissions guidelines as they vary from
%% journal to journal.

%% This example uses \plotone to include an EPS file scaled to
%% 80% of its natural size with \epsscale. Its caption
%% has been written to indicate that additional figure parts will be
%% available in the electronic journal.

\begin{figure}
\epsscale{.80}
\plotone{f1.eps}
\caption{Derived spectra for 3C138 \citep[see][]{heiles03}. Plots for all sources are available
in the electronic edition of {\it The Astrophysical Journal}.\label{fig1}}
\end{figure}

\clearpage

%% Here we use \plottwo to present two versions of the same figure,
%% one in black and white for print the other in RGB color
%% for online presentation. Note that the caption indicates
%% that a color version of the figure will be available online.
%%

\begin{figure}
\plottwo{f2.eps}{f2_color.eps}
\caption{A panel taken from Figure 2 of \citet{rudnick03}. 
See the electronic edition of the Journal for a color version 
of this figure.\label{fig2}}
\end{figure}

%% This figure uses \includegraphics to scale and rotate the still frame
%% for an mpeg animation.

\begin{figure}
\includegraphics[angle=90,scale=.50]{f3.eps}
\caption{Animation still frame taken from \citet{kim03}.
This figure is also available as an mpeg
animation in the electronic edition of the
{\it Astrophysical Journal}.}
\end{figure}

%% If you are not including electonic art with your submission, you may
%% mark up your captions using the \figcaption command. See the
%% User Guide for details.
%%
%% No more than seven \figcaption commands are allowed per page,
%% so if you have more than seven captions, insert a \clearpage
%% after every seventh one.

%% Tables should be submitted one per page, so put a \clearpage before
%% each one.

%% Two options are available to the author for producing tables:  the
%% deluxetable environment provided by the AASTeX package or the LaTeX
%% table environment.  Use of deluxetable is preferred.
%%

%% Three table samples follow, two marked up in the deluxetable environment,
%% one marked up as a LaTeX table.

%% In this first example, note that the \tabletypesize{}
%% command has been used to reduce the font size of the table.
%% We also use the \rotate command to rotate the table to
%% landscape orientation since it is very wide even at the
%% reduced font size.
%%
%% Note also that the \label command needs to be placed
%% inside the \tablecaption.

%% This table also includes a table comment indicating that the full
%% version will be available in machine-readable format in the electronic
%% edition.

\clearpage

\begin{deluxetable}{ccrrrrrrrrcrl}
\tabletypesize{\scriptsize}
\rotate
\tablecaption{Sample table taken from \citet{treu03}\label{tbl-1}}
\tablewidth{0pt}
\tablehead{
\colhead{POS} & \colhead{chip} & \colhead{ID} & \colhead{X} & \colhead{Y} &
\colhead{RA} & \colhead{DEC} & \colhead{IAU$\pm$ $\delta$ IAU} &
\colhead{IAP1$\pm$ $\delta$ IAP1} & \colhead{IAP2 $\pm$ $\delta$ IAP2} &
\colhead{star} & \colhead{E} & \colhead{Comment}
}
\startdata
0 & 2 & 1 & 1370.99 & 57.35    &   6.651120 &  17.131149 & 21.344$\pm$0.006  & 2
4.385$\pm$0.016 & 23.528$\pm$0.013 & 0.0 & 9 & -    \\
0 & 2 & 2 & 1476.62 & 8.03     &   6.651480 &  17.129572 & 21.641$\pm$0.005  & 2
3.141$\pm$0.007 & 22.007$\pm$0.004 & 0.0 & 9 & -    \\
0 & 2 & 3 & 1079.62 & 28.92    &   6.652430 &  17.135000 & 23.953$\pm$0.030  & 2
4.890$\pm$0.023 & 24.240$\pm$0.023 & 0.0 & - & -    \\
0 & 2 & 4 & 114.58  & 21.22    &   6.655560 &  17.148020 & 23.801$\pm$0.025  & 2
5.039$\pm$0.026 & 24.112$\pm$0.021 & 0.0 & - & -    \\
0 & 2 & 5 & 46.78   & 19.46    &   6.655800 &  17.148932 & 23.012$\pm$0.012  & 2
3.924$\pm$0.012 & 23.282$\pm$0.011 & 0.0 & - & -    \\
0 & 2 & 6 & 1441.84 & 16.16    &   6.651480 &  17.130072 & 24.393$\pm$0.045  & 2
6.099$\pm$0.062 & 25.119$\pm$0.049 & 0.0 & - & -    \\
0 & 2 & 7 & 205.43  & 3.96     &   6.655520 &  17.146742 & 24.424$\pm$0.032  & 2
5.028$\pm$0.025 & 24.597$\pm$0.027 & 0.0 & - & -    \\
0 & 2 & 8 & 1321.63 & 9.76     &   6.651950 &  17.131672 & 22.189$\pm$0.011  & 2
4.743$\pm$0.021 & 23.298$\pm$0.011 & 0.0 & 4 & edge \\
\enddata
%% Text for table notes should follow after the \enddata but before
%% the \end{deluxetable}. Make sure there is at least one \tablenotemark
%% in the table for each \tablenotetext.
\tablecomments{Table \ref{tbl-1} is published in its entirety in the 
electronic edition of the {\it Astrophysical Journal}.  A portion is 
shown here for guidance regarding its form and content.}
\tablenotetext{a}{Sample footnote for table~\ref{tbl-1} that was generated
with the deluxetable environment}
\tablenotetext{b}{Another sample footnote for table~\ref{tbl-1}}
\end{deluxetable}

%% If you use the table environment, please indicate horizontal rules using
%% \tableline, not \hline.
%% Do not put multiple tabular environments within a single table.
%% The optional \label should appear inside the \caption command.

\clearpage

\begin{table}
\begin{center}
\caption{More terribly relevant tabular information.\label{tbl-2}}
\begin{tabular}{crrrrrrrrrrr}
\tableline\tableline
Star & Height & $d_{x}$ & $d_{y}$ & $n$ & $\chi^2$ & $R_{maj}$ & $R_{min}$ &
\multicolumn{1}{c}{$P$\tablenotemark{a}} & $P R_{maj}$ & $P R_{min}$ &
\multicolumn{1}{c}{$\Theta$\tablenotemark{b}} \\
\tableline
1 &33472.5 &-0.1 &0.4  &53 &27.4 &2.065  &1.940 &3.900 &68.3 &116.2 &-27.639\\
2 &27802.4 &-0.3 &-0.2 &60 &3.7  &1.628  &1.510 &2.156 &6.8  &7.5 &-26.764\\
3 &29210.6 &0.9  &0.3  &60 &3.4  &1.622  &1.551 &2.159 &6.7  &7.3 &-40.272\\
4 &32733.8 &-1.2\tablenotemark{c} &-0.5 &41 &54.8 &2.282  &2.156 &4.313 &117.4 &78.2 &-35.847\\
5 & 9607.4 &-0.4 &-0.4 &60 &1.4  &1.669\tablenotemark{c}  &1.574 &2.343 &8.0  &8.9 &-33.417\\
6 &31638.6 &1.6  &0.1  &39 &315.2 & 3.433 &3.075 &7.488 &92.1 &25.3 &-12.052\\
\tableline
\end{tabular}
%% Any table notes must follow the \end{tabular} command.
\tablenotetext{a}{Sample footnote for table~\ref{tbl-2} that was
generated with the \LaTeX\ table environment}
\tablenotetext{b}{Yet another sample footnote for table~\ref{tbl-2}}
\tablenotetext{c}{Another sample footnote for table~\ref{tbl-2}}
\tablecomments{We can also attach a long-ish paragraph of explanatory
material to a table.}
\end{center}
\end{table}

%% If the table is more than one page long, the width of the table can vary
%% from page to page when the default \tablewidth is used, as below.  The
%% individual table widths for each page will be written to the log file; a
%% maximum tablewidth for the table can be computed from these values.
%% The \tablewidth argument can then be reset and the file reprocessed, so
%% that the table is of uniform width throughout. Try getting the widths
%% from the log file and changing the \tablewidth parameter to see how
%% adjusting this value affects table formatting.

%% The \dataset{} macro has also been applied to a few of the objects to
%% show how many observations can be tagged in a table.

\clearpage

\begin{deluxetable}{lrrrrcrrrrr}
\tablewidth{0pt}
\tablecaption{Literature Data for Program Stars}
\tablehead{
\colhead{Star}           & \colhead{V}      &
\colhead{b$-$y}          & \colhead{m$_1$}  &
\colhead{c$_1$}          & \colhead{ref}    &
\colhead{T$_{\rm eff}$}  & \colhead{log g}  &
\colhead{v$_{\rm turb}$} & \colhead{[Fe/H]} &
\colhead{ref}}
\startdata
HD 97 & 9.7& 0.51& 0.15& 0.35& 2 & \nodata & \nodata & \nodata & $-1.50$ & 2 \\
& & & & & & 5015 & \nodata & \nodata & $-1.50$ & 10 \\
\dataset[ADS/Sa.HST#O6H04VAXQ]{HD 2665} & 7.7& 0.54& 0.09& 0.34& 2 & \nodata & \nodata & \nodata & $-2.30$ & 2 \\
& & & & & & 5000 & 2.50 & 2.4 & $-1.99$ & 5 \\
& & & & & & 5120 & 3.00 & 2.0 & $-1.69$ & 7 \\
& & & & & & 4980 & \nodata & \nodata & $-2.05$ & 10 \\
HD 4306 & 9.0& 0.52& 0.05& 0.35& 20, 2& \nodata & \nodata & \nodata & $-2.70$ & 2 \\
& & & & & & 5000 & 1.75 & 2.0 & $-2.70$ & 13 \\
& & & & & & 5000 & 1.50 & 1.8 & $-2.65$ & 14 \\
& & & & & & 4950 & 2.10 & 2.0 & $-2.92$ & 8 \\
& & & & & & 5000 & 2.25 & 2.0 & $-2.83$ & 18 \\
& & & & & & \nodata & \nodata & \nodata & $-2.80$ & 21 \\
& & & & & & 4930 & \nodata & \nodata & $-2.45$ & 10 \\
HD 5426 & 9.6& 0.50& 0.08& 0.34& 2 & \nodata & \nodata & \nodata & $-2.30$ & 2 \\
\dataset[ADS/Sa.HST#O5F654010]{HD 6755} & 7.7& 0.49& 0.12& 0.28& 20, 2& \nodata & \nodata & \nodata & $-1.70$ & 2 \\
& & & & & & 5200 & 2.50 & 2.4 & $-1.56$ & 5 \\
& & & & & & 5260 & 3.00 & 2.7 & $-1.67$ & 7 \\
& & & & & & \nodata & \nodata & \nodata & $-1.58$ & 21 \\
& & & & & & 5200 & \nodata & \nodata & $-1.80$ & 10 \\
& & & & & & 4600 & \nodata & \nodata & $-2.75$ & 10 \\
\dataset[ADS/Sa.HST#O56D06010]{HD 94028} & 8.2& 0.34& 0.08& 0.25& 20 & 5795 & 4.00 & \nodata & $-1.70$ & 22 \\
& & & & & & 5860 & \nodata & \nodata & $-1.70$ & 4 \\
& & & & & & 5910 & 3.80 & \nodata & $-1.76$ & 15 \\
& & & & & & 5800 & \nodata & \nodata & $-1.67$ & 17 \\
& & & & & & 5902 & \nodata & \nodata & $-1.50$ & 11 \\
& & & & & & 5900 & \nodata & \nodata & $-1.57$ & 3 \\
& & & & & & \nodata & \nodata & \nodata & $-1.32$ & 21 \\
HD 97916 & 9.2& 0.29& 0.10& 0.41& 20 & 6125 & 4.00 & \nodata & $-1.10$ & 22 \\
& & & & & & 6160 & \nodata & \nodata & $-1.39$ & 3 \\
& & & & & & 6240 & 3.70 & \nodata & $-1.28$ & 15 \\
& & & & & & 5950 & \nodata & \nodata & $-1.50$ & 17 \\
& & & & & & 6204 & \nodata & \nodata & $-1.36$ & 11 \\
\cutinhead{This is a cut-in head}
+26\arcdeg2606& 9.7&0.34&0.05&0.28&20,11& 5980 & \nodata & \nodata &$<-2.20$ & 19 \\
& & & & & & 5950 & \nodata & \nodata & $-2.89$ & 24 \\
+26\arcdeg3578& 9.4&0.31&0.05&0.37&20,11& 5830 & \nodata & \nodata & $-2.60$ & 4 \\
& & & & & & 5800 & \nodata & \nodata & $-2.62$ & 17 \\
& & & & & & 6177 & \nodata & \nodata & $-2.51$ & 11 \\
& & & & & & 6000 & 3.25 & \nodata & $-2.20$ & 22 \\
& & & & & & 6140 & 3.50 & \nodata & $-2.57$ & 15 \\
+30\arcdeg2611& 9.2&0.82&0.33&0.55& 2 & \nodata & \nodata & \nodata & $-1.70$ & 2 \\
& & & & & & 4400 & 1.80 & \nodata & $-1.70$ & 12 \\
& & & & & & 4400 & 0.90 & 1.7 & $-1.20$ & 14 \\
& & & & & & 4260 & \nodata & \nodata & $-1.55$ & 10 \\
+37\arcdeg1458& 8.9&0.44&0.07&0.22&20,11& 5296 & \nodata & \nodata & $-2.39$ & 11 \\
& & & & & & 5420 & \nodata & \nodata & $-2.43$ & 3 \\
+58\arcdeg1218&10.0&0.51&0.03&0.36& 2 & \nodata & \nodata & \nodata & $-2.80$ & 2 \\
& & & & & & 5000 & 1.10 & 2.2 & $-2.71$ & 14 \\
& & & & & & 5000 & 2.20 & 1.8 & $-2.46$ & 5 \\
& & & & & & 4980 & \nodata & \nodata & $-2.55$ & 10 \\
+72\arcdeg0094&10.2&0.31&0.09&0.26&12 & 6160 & \nodata & \nodata & $-1.80$ & 19 \\
\sidehead{I'm a side head:}
G5--36 & 10.8& 0.40& 0.07& 0.28& 20 & \nodata & \nodata & \nodata & $-1.19$ & 21 \\
G18--54 & 10.7& 0.37& 0.08& 0.28& 20 & \nodata & \nodata & \nodata & $-1.34$ & 21 \\
G20--08 & 9.9& 0.36& 0.05& 0.25& 20,11& 5849 & \nodata & \nodata & $-2.59$ & 11 \\
& & & & & & \nodata & \nodata & \nodata & $-2.03$ & 21 \\
G20--15 & 10.6& 0.45& 0.03& 0.27& 20,11& 5657 & \nodata & \nodata & $-2.00$ & 11 \\
& & & & & & 6020 & \nodata & \nodata & $-1.56$ & 3 \\
& & & & & & \nodata & \nodata & \nodata & $-1.58$ & 21 \\
G21--22 & 10.7& 0.38& 0.07& 0.27& 20,11& \nodata & \nodata & \nodata & $-1.23$ & 21 \\
G24--03 & 10.5& 0.36& 0.06& 0.27& 20,11& 5866 & \nodata & \nodata & $-1.78$ & 11 \\
& & & & & & \nodata & \nodata & \nodata & $-1.70$ & 21 \\
G30--52 & 8.6& 0.50& 0.25& 0.27& 11 & 4757 & \nodata & \nodata & $-2.12$ & 11 \\
& & & & & & 4880 & \nodata & \nodata & $-2.14$ & 3 \\
G33--09 & 10.6& 0.41& 0.10& 0.28& 20 & 5575 & \nodata & \nodata & $-1.48$ & 11 \\
G66--22 & 10.5& 0.46& 0.16& 0.28& 11 & 5060 & \nodata & \nodata & $-1.77$ & 3 \\
& & & & & & \nodata & \nodata & \nodata & $-1.04$ & 21 \\
G90--03 & 10.4& 0.37& 0.04& 0.29& 20 & \nodata & \nodata & \nodata & $-2.01$ & 21 \\
LP 608--62\tablenotemark{a} & 10.5& 0.30& 0.07& 0.35& 11 & 6250 & \nodata &
\nodata & $-2.70$ & 4 \\
\enddata
\tablenotetext{a}{Star LP 608--62 is also known as BD+1\arcdeg 2341p.  We will
make this footnote extra long so that it extends over two lines.}
%% You can append references to a table using the \tablerefs command.
\tablerefs{
(1) Barbuy, Spite, \& Spite 1985; (2) Bond 1980; (3) Carbon et al. 1987;
(4) Hobbs \& Duncan 1987; (5) Gilroy et al. 1988: (6) Gratton \& Ortolani 1986;
(7) Gratton \& Sneden 1987; (8) Gratton \& Sneden (1988); (9) Gratton \& Sneden 1991;
(10) Kraft et al. 1982; (11) LCL, or Laird, 1990; (12) Leep \& Wallerstein 1981;
(13) Luck \& Bond 1981; (14) Luck \& Bond 1985; (15) Magain 1987;
(16) Magain 1989; (17) Peterson 1981; (18) Peterson, Kurucz, \& Carney 1990;
(19) RMB; (20) Schuster \& Nissen 1988; (21) Schuster \& Nissen 1989b;
(22) Spite et al. 1984; (23) Spite \& Spite 1986; (24) Hobbs \& Thorburn 1991;
(25) Hobbs et al. 1991; (26) Olsen 1983.}
\end{deluxetable}

%% Tables may also be prepared as separate files. See the accompanying
%% sample file table.tex for an example of an external table file.
%% To include an external file in your main document, use the \input
%% command. Uncomment the line below to include table.tex in this
%% sample file. (Note that you will need to comment out the \documentclass,
%% \begin{document}, and \end{document} commands from table.tex if you want
%% to include it in this document.)

%% \begin{table}[h]
\centering
\begin{tabular}{l|ll}
$k \setminus j$ & $2$ & $4$ \\
\hline
$2$ & $q^{-2}$ &  \\
$-2$ & $q^{2}$ & $1$ \\
\end{tabular}
\caption*{$3_1$}
\end{table}

%% The following command ends your manuscript. LaTeX will ignore any text
%% that appears after it.

\end{document}

%%
%% End of file `sample.tex'.


\subsection{AIGI Analysis from Three Perspectives}
\vspace{-2pt}

To further illustrate the differences of the three perspectives, we demonstrate several example images and their corresponding subjective ratings from three aspects in Fig.5.
For each subfigure, it can be noticed that the right AIGI outperforms the left AIGI on two evaluation dimensions but is much worse than the left AIGI on another dimension, which demonstrates that each evaluation perspective (quality, authenticity, or text-image correspondence) has its own unique perspective and value.



\begin{figure}[t]
%   \centering
%   \includegraphics[width=\textwidth]{figures/s1.png}
%   \caption{ Scores and MOSs probability distribution of quality score.}
% ~\\
% ~\\
%   \centering
%   \includegraphics[width=\textwidth]{figures/s2.png}
%   \caption{ Scores and MOSs probability distribution of reality score.}
% ~\\
% ~\\
%   \centering
%   \includegraphics[width=\textwidth]{figures/s3.png}
%   \caption{ Scores and MOSs probability distribution of relation score.}
  \centering
  \includegraphics[width=\textwidth]{figures/scores.pdf}
  \caption{ (a) MOSs distribution of quality score.
  (b) MOSs distribution of authenticity score.
  (c) MOSs distribution of correspondence score.
  (d) Distribution of the quality score.
  (e) Distribution of the authenticity score.
  (f) Distribution of the correspondence score.}
\end{figure}
Fig.6 demonstrates the MOS and score distribution for quality evaluation, authenticity evaluation, and text-image correspondence evaluation, respectively, which demonstrate the images in AIGCIQA 2023 cover a wide range of perceptual quality. 
% Fig. 11-13 displays the subjective evaluation results with top 20 high score and top 20 low score from 3 different aspects.





\section{EXPERIMENT}
\subsection{Benchmark Models}
Since the AIGIs in the proposed AIGCIQA2023 database are generated based on text prompts and have no pristine reference images, they can only be evaluated by no-reference (NR) IQA metrics.
In this paper, we select fifteen state-of-the-art IQA models for comparison. The selected models can be classified into two groups:


\begin{itemize}
\item \textbf{Handcrafted-based} models, including: NIQE\cite{mittal2012making}, BMPRI\cite{min2018blind}, BPRI\cite{min2017blind}, BRISQUE\cite{mittal2012no}, HOSA\cite{xu2016blind}, BPRI-LSSn\cite{min2017blind}, BPRI-LSSs\cite{min2017blind}, BPRI-PSS\cite{min2017blind}, QAC\cite{xue2013learning}, HIGRADE-1 and HIGRADE-2\cite{kundu2017large}. 

These models extract handcrafted features based on prior knowledge about image quality. 
%\item \textbf{SVR-based} models: HIGRADE-1 and HIGRADE-2\cite{kundu2017large} . 

%These models combine hand crafted features from a Support Vector Regression(SVR) to represent perceptual quality.
\item \textbf{Deep learning-based} models, including: CNNIQA\cite{kang2014convolutional}, WaDIQaM-NR\cite{bosse2017deep}, VGG (VGG-16 and VGG-19)\cite{simonyan2014very} and ResNet (ResNet-18 and ResNet-34)\cite{he2016deep}.  

These models characterize quality-aware information by training deep neural networks from labeled data.
\end {itemize}
%%% This declares a command \Comment
%% The argument will be surrounded by /* ... */
\SetKwComment{Comment}{/* }{ */}

\begin{algorithm}[t]
\caption{Training Scheduler}\label{alg:TS}
% \KwData{$n \geq 0$}
% \KwResult{$y = x^n$}
\LinesNumbered
\KwIn{Training data $\mathcal{D}_{train}=\{(q_i, a_i, p_i^+)\}_{i=1}^m$, \\
\qquad \quad Iteration number $L$.}
\KwOut{A set of optimal model parameters.}

\For{$l=1,\cdots, L$}{
    Sample a batch of questions $Q^{(l)}$\\
    \For{$q_i\in Q^{(l)}$}{
        $\mathcal{P}_{i}^{(l)} \gets \mathrm{arg\,max}_{p_{i,j}}(\mathrm{sim}(q_i^{en},p_{i,j}),K)$\\
        $\mathcal{P}_{Gi}^{(l)} \gets \mathcal{P}_{i}^{(l)}\cup\{p^+_i\}$\\
        Compute $\mathcal{L}^i_{retriever}$, $\mathcal{L}^i_{postranker}$, $\mathcal{L}^i_{reader}$\\ according to Eq.\ref{eq:retriever}, Eq.\ref{eq:rerank}, Eq.\ref{eq:reader}\\
    }
    % $\mathcal{L}^{(l)}_{retriever} \gets \frac{1}{|Q^{(l)}|}\sum_i\mathcal{L}^i_{retriever}$\\
    % $\mathcal{L}^{(l)}_{retriever} \gets \mathrm{Avg}(\mathcal{L}^i_{retriever})$,
    % $\mathcal{L}^{(l)}_{rerank} \gets \mathrm{Avg}(\mathcal{L}^i_{rerank})$,
    % $\mathcal{L}^{(l)}_{reader} \gets \mathrm{Avg}(\mathcal{L}^i_{reader})$\\
    % Compute $\mathcal{L}^{(l)}_{retriever}$, $\mathcal{L}^{(l)}_{rerank}$, and $\mathcal{L}^{(l)}_{reader}$ by averaging over $Q^{(l)}$\\
    $\mathcal{L}^{(l)} \gets \frac{1}{|Q^{(l)}|}\sum_i(\mathcal{L}^{i}_{retriever} + \mathcal{L}^{i}_{postranker}+ \mathcal{L}^{i}_{reader})$\\
    $\mathcal{P}^{(l)}_K\gets\{\mathcal{P}^{(l)}_i|q_i\in Q^{(l)}\}$,\quad $\mathcal{P}^{(l)}_{KG}\gets\{\mathcal{P}^{(l)}_{Gi}|q_i\in Q^{(l)}\}$\\
    Compute the coefficient $v^{(l)}$ according to Eq.~\ref{eq:v}\\
  \eIf{$ v^{(l)}=1$}{
    $\mathcal{L}^{(l)}_{final} \gets \mathcal{L}^{(l)}(\mathcal{P}_{KG}^{(l)})$\\
  }{
      $\mathcal{L}^{(l)}_{final} \gets \mathcal{L}^{(l)}(\mathcal{P}^{(l)}_{K}),$\\
    }
    Optimize $\mathcal{L}^{(l)}_{final}$
}
\end{algorithm}


%  \eIf{$ \mathcal{L}^{(l-1)}_{retriever}<\lambda$}{
%     $\mathcal{L}^{(l)}_{final} \gets \mathcal{L}^{(l)}(\mathcal{P}_K^{(l)})$\\
%   }{
%       $\mathcal{L}^{(l)}_{final} \gets \mathcal{L}^{(l)}(\mathcal{P}^{(l)}_{KG}),$\\
%     }

\subsection{Evaluation Criteria}





In this study, we utilize the following four performance evaluation criteria to evaluate
the consistency between the predicted scores and the corresponding ground-truth MOSs, including Spearman Rank Correlation Coefficient (SRCC), Pearson Linear Correlation Coefficient (PLCC), Kendall’s Rank Correlation Coefficient (KRCC), and Root Mean Squared Error (RMSE).

% The SRCC metric measures the similarity between two sets of rankings, while the PLCC metric computes the linear correlation between two groups of rankings. SRCC and PLCC are used to measures the monotonicity and accuracy of prediction respectively. The KRCC metric, on the other hand, estimates the ordinal relationship between two measured quantities. The closer the result of SRCC, PLCC and KRCC is to 1, the better the prediction performance. The RMSE measures the distance between the ground-truth MOSs and the predicted MOSs. The closer the result of RMSE is to 0, the better the prediction performance.


\subsection{ Experimental Setup}
All the benchmark models are validated on the proposed AIGCIQA2023 database. 
For traditional handcrafted-based models, they are directly evaluated based on the database.
For deep trainable models, we first randomly split the database into an 4:1 ratio for training/testing while ensuring the image with the same prompt label falls into the same set. 
The partitioning and evaluation process is repeated several times for a fair comparison while considering the computational complexity, and the average result is reported as the final performance. 
%For SVR-based models, the repeating time is 1,000, implemented by LIBSVM  with radial basis function (RBF) kernel. 
For deep learning-based models, we applied CNNIQA\cite{kang2014convolutional}, WaDIQaM-NR\cite{bosse2017deep}, VGG (VGG-16 and VGG-19)\cite{simonyan2014very} and ResNet (ResNet-18 and ResNet-34)\cite{he2016deep} to predict the MOS of image quality. 
The repeating time is 10, the training epochs are 50 with an initial learning rate of 0.0001 and batch size of 4.


\subsection{Performance Discussion}
\section{Related Work}

%We compare the TT-ADMM and TT-RALS with the existing tensor completion methods with TT-decomposition.
To solve the tensor completion problem with TT decomposition,
Wang \etal \cite{wang2016tensor} and Grasedyck \etal \cite{grasedyck2015alternating} developed
algorithms that iteratively solve minimization problems with respect
to $G_k$ for each $k = 1,\ldots,K$. Unfortunately, the adaptivity of
the TT rank is not well discussed.  \cite{wang2016tensor} assumed that
the TT rank is given. Grasedyck \etal \cite{grasedyck2015alternating} proposed a grid
search method. However, the TT rank is determined by a single
parameter (i.e., $R_1=\dots=R_{K-1}$) and the search method lacks its
generality.  Furthermore, the scalability problem remains in both
methods---they require more than $O(I^K)$ space.

Phien et al.(2016) \cite{phien2016efficient} proposed a convex
optimization method using the Schatten TT norm.  However, because they
employed an alternating-type optimization method, the global
convergence of their method is not guaranteed. Moreover, since they
maintain $X$ directly and perform the reshape of $X$ several times,
their method requires $O(I^K)$ time.

%TT-ADMM is the convex optimization method which can be guaranteed to
%converge to the global optimum via the ADMM approach and can select
%the rank adaptivity.  Also, the statistical error is theoretically
%evaluated by the value of TT rank.  Though the advantages, TT-ADMM
%cannot avoid the computational burden from controlling $X$ directly.
%
%TT-RALS is the method which can solve the computational complexity
%problem.  Since TT-RALS does not handle $X$ but controls only
%$\{G_k\}_{k=1}^K$, the computational burden is avoided.  Also, TT-RALS
%can select the TT rank adaptivity and its statistical performance is
%evaluated.  By setting the random projection parameter $s$
%sufficiently small, the time and space complexity does not increase
%exponentially as $K$ grows.

Table~\ref{tab:contribution} highlights the difference between the
existing and our methods. We emphasize that our study is the first
attempt to analyze the statistical performance of TT decomposition. In
addition, TT-RALS is only the method that both time and space
complexities do not grow exponentially in $K$.

\begin{table}[htbp]
  \centering
  {\small
  \begin{tabular}{rccccc}
    \hline
    Method & \shortstack{Global\\Convergence} & \shortstack{Rank\\Adaptivity} & \shortstack{Time\\Complexity}& \shortstack{Space\\Complexity}& \shortstack{Statistical\\Bounds}\\
    \hline
    TCAM-TT\cite{wang2016tensor}&        & & $O(nIKR^4)$ & $O(I^K)$ & \\
    ADF for TT\cite{grasedyck2015alternating}          &          & (search) &$O(KIR^3 + nKR^2)$& $O(I^K)$& \\
    SiLRTC-TT\cite{phien2016efficient}      & & \checkmark & $O(I^{3K/2})$ & $O(KI^K)$ & \\
   	\textbf{TT-ADMM}              &\checkmark & \checkmark & $O(K I^{3K/2})$ & $O(I^K)$ &\checkmark\\
    \textbf{TT-RALS}              &      &  \checkmark  & $O((n + KD^2)KI^2R^4)$ & $O(n + KI^2R^4)$ &\checkmark\\
    \hline
  \end{tabular}
  }
  \caption{Comparison of TT completion algorithms, with $R$ is a parameter for the TT rank such that $R = R_1 = \cdots = R_{K-1}$, $I = I_1 = \cdots = I_K$ is dimension, $K$ is the number of modes, $n$ is the number of observed elements, and $D$ is the dimension of random projection.}
  \label{tab:contribution}
\end{table}


%%% Local Variables:
%%% mode: latex
%%% TeX-master: "TTcomp_NIPS2017.tex"
%%% End:

The performance results of the state-of-the-art IQA models mentioned above on the proposed AIGCIQA2023 database are exhibited in Table 1, from which we can make several conclusions:
\begin{itemize}
\item The handcrafted-based methods achieve poor performance on the whole database, which indicates the extracted handcrafted features are not effective for modeling the quality representation of AIGIs. This is because most employed handcrafted features of
these methods are based on the prior knowledge learned from NSIs, which are not effective for evaluating AIGIs.

\item The deep learning-based methods achieve relatively more competitive performance results on three evaluation perspectives. However, they are still far away from satisfactory.

\item Most of the IQA models achieve better performance on quality evaluation and worse on text-image correspondence score assessment.
The reason is that the text prompts for image generation are not utilized for the IQA model training. 
This makes it more challenging for the IQA models to extract relation features from AIGIs, which inevitably leads to performance drops.
\end {itemize}


\section{Conclusion and Future Work}
In this paper, we study the human visual preference problem for AIGIs. 
We first construct a new IQA database for AIGIs, termed AIGCIQA2023, which includes 2400 AIGIs generated based on 100 various text-prompts, and corresponding subjective MOSs evaluated from three perspectives (\textit{i.e., quality, authenticity, and text-image correspondence}).
Experimental analysis demonstrates that these three dimensions can reflect different aspects of human visual preferences on AIGIs, which further manifests that the evaluation of Quality of Experience (QoE) for AIGIs should be considered from multiple dimensions.
Based on the constructed database, we evaluate the performance of several state-of-the-art IQA models and establish a new benchmark to facilitate future research.

In future work, we will further explore the human visual perception for AIGIs and develop corresponding objective evaluation models for better assessing the quality of AIGIs from the three perspectives proposed in this paper.

%\input{figures/top20}




\newpage

\bibliographystyle{splncs04}
\bibliography{ref.bib}
\end{document}








\end{document}
