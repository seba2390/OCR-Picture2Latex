\documentclass[letterpaper, 10 pt, journal, twoside]{IEEEtran}

\usepackage{amssymb}
\usepackage{amsmath}
\usepackage{amsfonts}       %
\usepackage{arydshln}
\usepackage{booktabs}       %
\usepackage{balance}
\usepackage{booktabs}
\usepackage[utf8]{inputenc} %
\usepackage[T1]{fontenc}    %
\usepackage{color, colortbl}
\usepackage{collcell}
\usepackage[pdftex, pdfstartview={FitV}, pdfpagelayout={TwoColumnLeft},bookmarksopen=true,plainpages = false, colorlinks=true, linkcolor=black, citecolor = black, urlcolor = black,filecolor=black , pagebackref=false,hypertexnames=false, plainpages=false, pdfpagelabels,bookmarks=false ]{hyperref}
\usepackage{url}            %
\usepackage{nicefrac}       %
\usepackage{microtype}      %
\usepackage{makecell}
\usepackage[font=scriptsize]{caption}
\usepackage{subcaption}
\usepackage[]{siunitx}
\usepackage{graphicx}
\usepackage{epstopdf} %
\usepackage{mathtools}
\usepackage{marginnote}
\usepackage[dvipsnames, table]{xcolor} %
\usepackage{float}
\usepackage[capitalize]{cleveref}
\usepackage[]{algorithmic}
\usepackage{makecell}
\usepackage{multirow}
\usepackage{paralist}
\floatstyle{boxed} \floatname{algorithm}{Algorithm}
\newfloat{algorithm}{t}{loa}[section]
\usepackage{xcolor}
\usepackage[sort,compress]{cite}
\usepackage{soul}
\usepackage{etoolbox}
\usepackage{pgf}
\usepackage{mathrsfs}

\usepackage{xstring}
\usepackage{color}
\usepackage{soul}
\usepackage{xspace}
\usepackage[capitalize]{cleveref}
\usepackage{dsfont}
\usepackage{bm}
\usepackage{bbm}
\usepackage{nicefrac}  %
\usepackage{amsfonts}
\usepackage{amssymb}
\usepackage{array}
\usepackage{mathtools}
\usepackage{subcaption}
\usepackage{fontawesome}
\usepackage{thmtools}
\usepackage{thm-restate}
\usepackage{lipsum}
\usepackage{pifont}
\usepackage{makecell}
\usepackage{multirow}
\usepackage{afterpage}
\usepackage{tabularx} 
\usepackage{textcase}
\usepackage{tikz}
\usepackage{url}
\usepackage[normalem]{ulem}
\usepackage{wrapfig}  %
\usepackage{textcomp}
\usepackage{xifthen}%

% A
\newacronym{afe}{AFE}{analogue front-end}
\newacronym{asic}{ASIC}{application-specific integrated circuit}
% B
\newacronym{bx}{BX}{bunch crossing}
\newacronym{be}{BE}{back-end}
% C
\newacronym{cdr}{CDR}{clock data recovery}
\newacronym{cml}{CML}{current mode logic}
\newacronym{cms}{CMS}{Compact Muon Solenoid}
\newacronym{csa}{CSA}{charge sensitive amplifier}
% D
\newacronym{da}{DA}{differential amplifier}
\newacronym{dac}{DAC}{digital-to-analogue converter}
\newacronym{daq}{DAQ}{data acquisition}
\newacronym{diff}{DIFF}{differential}
% E
\newacronym{enc}{ENC}{equivalent noise charge}
% F
\newacronym{fe}{FE}{front-end}
% H
\newacronym{hdi}{HDI}{high density interconnect}
\newacronym{hllhc}{HL-LHC}{High Luminosity LHC}
% I
\newacronym{it}{IT}{Inner Tracker}
\newacronym{ip}{IP}{intellectual property}
% L
\newacronym{lcc}{LCC}{leakage current compensation}
\newacronym{led}{LED}{light-emitting diodes}
\newacronym{lhc}{LHC}{Large Hadron Collider}
\newacronym{lin}{LIN}{linear}
\newacronym{ls3}{LS3}{Long Shutdown 3}
% M
\newacronym{mip}{MIP}{minimum ionizing particle}
\newacronym{mpv}{MPV}{most probable value}
% O
\newacronym{ot}{OT}{Outer Tracker}
% P
\newacronym{pa}{PA}{preamplifier}
\newacronym{pll}{PLL}{phase locked loop}
% R
\newacronym{rms}{RMS}{root-mean-square}
% S
\newacronym{sync}{SYNC}{synchronous}
\newacronym{shldo}{ShLDO}{shunt low-dropout}
% T
\newacronym{tbpx}{TBPX}{Tracker Barrel Pixel detector}
\newacronym{tepx}{TEPX}{Tracker Endcap Pixel detector}
\newacronym{tfpx}{TFPX}{Tracker Forward Pixel detector}
\newacronym{tia}{TIA}{transimpedance amplifier}
\newacronym{tid}{TID}{total ionizing dose}
\newacronym{tot}{TOT}{time-over-threshold}
\newacronym{tsmc}{TSMC}{Taiwan Semiconductor Manufacturing Company}
\newcommand{\figref}[1]{Fig.~\ref{#1}}
\newcommand{\tblref}[1]{Table~\ref{#1}}
\newcommand{\secref}[1]{Section~\ref{#1}}
\renewcommand{\eqref}[1]{Equation~(\ref{#1})}

\def\availableat{\url{url-published-on-acceptance}}

\newcommand{\todo}[1]{{\color{red} TODO: {#1}}}
\newcommand{\newstuff}[1]{{\color{red} CHECK: {#1}}}
%\newcommand{\todo}[1]{{}}

\newcommand{\ckp}[2]{$CK_{#1}P_{#2}$}
\newcommand{\cext}{\ckp{8}{16}$ext$}
\newcommand{\cfin}{$F_{CK_{X}P_{Y}}$}
\newcommand{\cray}{\ckp{8}{8}$ray$}
\newcommand{\csin}{$SK_{8}P_{8}$}
\newcommand{\casin}{$SK_{combined}$}
%\renewcommand{\cext}{$SK_{8}K_{8}P_{8}$}
\newcommand{\ckpnl}[2]{$CK_{#1}P_{#2}nl$}


\makeatletter
\newcommand{\Spvek}[2][r]{%
	\gdef\@VORNE{1}
	\left(\hskip-\arraycolsep%
	\begin{array}{#1}\vekSp@lten{#2}\end{array}%
	\hskip-\arraycolsep\right)}

\def\vekSp@lten#1{\xvekSp@lten#1;vekL@stLine;}
\def\vekL@stLine{vekL@stLine}
\def\xvekSp@lten#1;{\def\temp{#1}%
	\ifx\temp\vekL@stLine
	\else
	\ifnum\@VORNE=1\gdef\@VORNE{0}
	\else\@arraycr\fi%
	#1%
	\expandafter\xvekSp@lten
	\fi}
\makeatother

\begin{document}
\title{Efficient Deep Learning of Robust Policies\\ 
from MPC using Imitation and\\ Tube-Guided Data Augmentation}
\author{Andrea Tagliabue and Jonathan P.\ How%
	\thanks{The authors are with the Department of Aeronautics and Astronautics, Massachusetts Institute of Technology.
	    {\texttt{\{atagliab, jhow\}@mit.edu.}}}
    \thanks{Work funded by the Air Force Office of Scientific Research MURI FA9550-19-1-0386.}
}%

\maketitle

\begin{abstract}
Imitation Learning (IL) has been increasingly employed to generate computationally efficient policies from task-relevant demonstrations provided by Model Predictive Control (MPC). However, commonly employed IL methods are often data- and computationally-inefficient, as they require a large number of MPC demonstrations, resulting in long training times, and they produce policies with limited robustness to disturbances not experienced during training.
In this work, we propose an IL strategy to \textit{efficiently} compress a computationally expensive MPC into a deep neural network policy that is \textit{robust} to previously unseen disturbances. 
By using a robust variant of the MPC, called Robust Tube MPC, and leveraging properties from the controller, we introduce a computationally-efficient data augmentation method that enables a significant reduction of the number of MPC demonstrations and training time required to generate a robust policy. %
Our approach opens the possibility of \textit{zero-shot} transfer of a policy trained from a single MPC demonstration collected in a nominal domain, such as a simulation or a robot in a lab/controlled environment, to a new domain with previously unseen bounded model errors/perturbations. 
Numerical and experimental evaluations performed using linear and nonlinear MPC for agile flight on a multirotor show that our method outperforms strategies commonly employed in IL (such as Dataset-Aggregation (DAgger) and Domain Randomization (DR)) in terms of demonstration-efficiency, training time, and robustness to perturbations unseen during training. %
\end{abstract}

\begin{IEEEkeywords}
Imitation Learning; Data Augmentation; Robust Tube Model Predictive Control; Aerial Robotics.
\end{IEEEkeywords}
\vspace{-3ex}

\section*{Supplementary Material}
Video: \url{https://youtu.be/aWRuvy3LviI}.

\acresetall
\section{Introduction} \label{sec:introduction}
\section{Introduction}

Scientific literature is most commonly available in the form of PDFs, which pose challenges for accessibility \citep{NielsenPDFStillUnfit, Bigham2016AnUT}. When researchers, students, and other individuals who are blind or low vision (BLV) interact with scientific PDFs through screen readers, the availability of document structure tags, labeled reading order, labeled headers, and image alt-text are necessary to facilitate these interactions. However, these features must be painstakingly added by authors using proprietary software tools, and as a result, are often missing from papers. Low vision or dyslexic readers who interact with PDFs through screen magnification or text-to-speech may also find the complexity of certain academic paper PDF formats challenging, e.g., non-linear layout can interrupt the flow of text in a magnifying tool. Inaccessible paper PDFs can lead to high cognitive overload, frustration, and abandonment of reading for BLV readers. 

Unfortunately, we find that the majority of scientific PDFs lack basic accessibility features. We estimate based on a sample of \numpdfs PDFs from multiple fields of study that only around \percaccessible of paper PDFs released in the last decade satisfy all of the aforementioned accessibility requirements. 
Accessibility challenges for academic PDFs are largely due to three factors: (1) the complexity of the PDF file format, which make it less amenable to certain accessibility features, (2) the dearth of tools, especially non-proprietary tools, for creating accessible PDFs, and (3) the dependency on volunteerism from the community with minimal support or enforcement \citep{Bigham2016AnUT}. The intent of the PDF file format is to support faithful visual representation of a document for printing, a goal that is inherently divergent from that of document representation for the purposes of accessibility. Though some professional organizations like the Association for Computing Machinery (ACM) have encouraged PDF accessibility through standards and writing guidelines,\footnote{\href{https://www.acm.org/publications/authors/submissions}{https://www.acm.org/publications/authors/submissions}} uptake among academic publishers and disciplines more broadly has been limited. 

While policy changes help, the fact remains that most academic PDFs produced today, and historically, are inaccessible, yet remain as the dominant way to read those papers. A long-range solution will necessitate buy-in from multiple stakeholders---publishers, authors, readers, technologists, granting agencies, and the like. But in the interim, there are technological solutions that can be offered as a sort of ``band-aid'' to the problem. We use this paper to offer an in-depth qualitative and quantitative description of the problem as it stands, and to introduce one such technological solution: the \scially system that automatically extracts semantic information from paper PDFs and re-renders this content in the form of an accessible HTML document. Though the process is imperfect and can introduce errors, we demonstrate the ability of the rendered HTMLs to reduce cognitive load and facilitate in-paper navigation and interactions for BLV users. 

The goals and contributions of this paper are three-fold:

\begin{enumerate}
    \item We characterize the state of academic-paper PDF accessibility by estimating the degree of adherence to accessibility criteria for papers published in the last decade (2010--2019), and describe correlations between year, field of study, PDF typesetting software, and PDF accessibility.
    \item We propose an automated approach for extracting the content of academic PDFs and displaying this content in a more accessible HTML document format. We build a prototype that re-renders 12 million PDFs in HTML, and describe the design decisions, features, and quality of the renders (assessed as faithfulness to the source PDF). We perform expert grading of the rendered HTML and report an error analysis. A demo of our system is available at \href{https://scia11y.org/}{scia11y.org}, which makes available 1.5M HTML renders of open access PDFs.
    \item We conduct an exploratory user study with \numusers BLV scholars to better understand the challenges they experience when reading academic papers and how our proposed tool might augment their current workflow. During the study, we ask users to interact with the prototype and offer feedback for its improvement. We perform open coding of interviews to identify existing reading challenges, coping mechanisms, as well as positive and negative responses to prototype features. We summarize the findings of this user study into a set of design recommendations.
\end{enumerate}

Our analysis reveals that PDF accessibility adherence is low across all fields of study. Of the five accessibility criteria we assess, only \percaccessible of the PDFs we assess demonstrate full compliance. Though compliance for several criteria seems to be increasing over time, author awareness and contribution to accessibility remains low, as Alt-text has the lowest compliance of the five criteria at between 5--10\% (Alt-text is the only criterion of the five that \textit{requires} author intervention in all cases using current tools). We also find that typesetting software is strongly associated with accessibility compliance, with LaTeX and publishing software like Arbortext APP producing low compliance PDFs, while Microsoft Word is generally associated with higher compliance.


\begin{figure}[t!]
    \centering
    \includegraphics[width=\textwidth]{figures/pipeline.png}
    \caption{A schematic for creating the \scially HTML render from a paper PDF. Starting with the raw two-column PDF on the left, S2ORC \citep{lo-wang-2020-s2orc} is used to extract title, authors, abstract, section headers, body text, and references. S2ORC also identifies links between inline citations and references to figures and table objects. DeepFigures \citep{Siegel2018ExtractingSF} is used to extract figures and tables, along with their captions. The output of these two models are merged with metadata from the Semantic Scholar API. Heuristics are used to construct a table of contents, to insert figures and tables in the appropriate places in the text, and to repair broken URLs. We add HTML headers as illustrated (header tags for sections, paragraph tags for body text, and figure tags for figures and tables); highlighted components (table of contents and links in references) are not in the PDF and novel navigational features that we introduce to the HTML render. An example HTML render of parts of a paper document is show to the right (actual render is single column, which is split here for presentation).}
    \label{fig:pipeline}
    \Description{A schematic diagram showing the components of the SciA11y pipeline. An image of a paper PDF is on the left. Red boxes on the PDF image highlight the text components from the paper, with an arrow pointing to a box that says "S2ORC extracts: title, authors, abstract, section headers, body paragraphs, and references." A blue box on the PDF image highlights a figure, with an arrow pointing to a box that says "DeepFigures extracts: figures, figure captions, tables, and table titles/captions." A box below "S2ORC extracts" and "DeepFigures extracts" says "Additional content: metadata from Semantic Scholar API, table of contents, figures and tables inserted at first mention, and links between references and text." Arrows from all three boxes point into a larger box that describes the SciA11y prototype, where HTML tags are inserted around various blocks of text extracted from the PDF. On the right of all this is a screen capture of an example HTML render, showing how the semantic content from the PDF is represented as a single-column HTML page for easy reading.}
\end{figure}

To offset the reading challenges of inaccessible papers for BLV researchers, we propose and test the \scially system for rendering academic PDFs into accessible HTML documents. As shown in Figure~\ref{fig:pipeline}, our prototype integrates several machine learning text and vision models to extract the structure and semantic content of papers. The content is represented as an HTML document with headings and links for navigation, figures and tables, as well as other novel features to assist in document structure understanding. Our evaluation of the \scially system identifies common classes of extraction problems, and finds that though many papers exhibit some extraction errors, the majority (55\%) have no major problems that impact readability, and another 32\% have only some problems that impact readability.

Through our user study, we identify numerous challenges faced by BLV users when reading paper PDFs, including some that affect the whole document or limit navigation, and many that affect the ability of the reader to understand text or various elements of a paper like math content or tables. Responses to \scially were positive; participants especially liked navigation features such as headings, the table of contents, and bidirectional links between inline citations and references. Of the extraction errors in \scially, missed or incorrectly extracted headings were the most problematic, as these impact the user's ability to navigate between sections and fully trust the system. All users reported being likely to use the system in the future. When asked how the system might be integrated into their workflow, one participant replied ``I think it would become the workflow.'' Another participant said, ``for unaccessible PDFs, this is life-changing.'' We condense these findings into a set of recommendations for designing and engineering accessible reading systems (Section~\ref{sec:designrecs}). Most importantly, documents should be structured to match a reader's mental model, objects should be properly tagged, and care should be taken to reduce the reader's cognitive load and increase trust in the system. Features that emulate the external memory that visual layout provides to sighted users can be especially beneficial.

This paper is organized as follows. Following a description of related work in Section \ref{sec:related_work}, we first provide a meta-scientific analysis of the current state of academic PDF accessibility in Section \ref{sec:sos}. In Section \ref{sec:pdf2html}, we document our pipeline for converting PDF to HTML and describe the \scially prototype for rendering papers. An evaluation of HTML render quality and faithfulness is provided in Section \ref{sec:evaluation}. Section \ref{sec:user_study} describes our user study and findings. 
We recognize that no PDF extraction system is perfect, and many open research challenges remain in improving these systems. However, based on our findings, we believe \scially can dramatically improve screen reader navigation of most papers compared to PDFs, and is well-positioned to assist BLV researchers with many of their most common reading use cases. Our hope is that a system such as \scially can improve BLV researcher access to the content of academic papers, and that these design recommendations can be leveraged by others to create better, more faithful, and ultimately more usable tools and systems for scholars in the BLV community.


\section{Related Works} \label{sec:related_works}
\section{Related Work}

Online meta-learning brings together ideas from online learning, meta learning, and continual learning, with the aim of adapting quickly to each new task while \emph{simultaneously} learning how to adapt even more quickly in the future. We discuss these three sets of approaches next.


\noindent \textbf{Meta Learning:} Meta learning methods try to learn the high-level context of the data, to behave well on new tasks (\emph{Learning to learn}). These methods involve learning a metric space~\citep{koch2015siamese, vinyals2016matching, snell2017prototypical, yang2017learning}, gradient based updates~\citep{finn2017model, li2017meta, park2019meta, nichol2018first, nichol2018reptile}, or some specific architecture designs~\citep{santoro2016meta, munkhdalai2017meta, ravi2016optimization}.
In this work, we are mainly interested in gradient based meta learning methods for online learning. MAML~\citep{finn2017model} and its variants~\citep{nichol2018first, nichol2018reptile, li2017meta, park2019meta, antoniou2018train} first meta train the models in such a way that the meta parameters are close to the optimal task specific parameters (good initialization). This way, adaptation becomes faster when fine tuning from the meta parameters. However, directly adapting this approach into an online setting will require more relaxation on online learning assumptions, such as access to task boundaries and resetting back and froth from meta parameters. Our method does not require knowledge of task boundaries.




\noindent \textbf{Online Learning:} Online learning methods update their models based on the stream of data sequentially. There are various works on online learning using linear models~\citep{cesa2006prediction}, non-linear models with kernels~\citep{kivinen2004online, jin2010online}, and deep neural networks~\citep{zhou2012online}. Online learning algorithms often simply update the model on the new data, and do not consider the past knowledge of the previously seen data to do this online update more efficiently. However, the online meta learning framework, allow us to keep track of previously seen data and with the ``meta'' knowledge we can update the online weights to the new data more faster and efficiently.
\noindent \textbf{Continual Learning:} 
A number of prior works on continual learning have addressed catastrophic forgetting~\citep{mccloskey1989catastrophic,li2017learning,ratcliff1990connectionist, rajasegaran2019random, rajasegaran2020itaml}, removing the need to store all prior data during training. Our method does not address catastrophic forgetting for the meta-training phase, because we must still store all data so as to ``replay'' it for meta-training, though it may be possible to discard or sub-sample old data (which we leave to future work). However, our adaptation process is fully online. A number of works perform meta-learning for better continual learning, i.e. learning good continual learning strategies~\citep{al2017continuous,nagabandi2018deep,javed2019meta,harrison2019continuous,he2019task,beaulieu2020learning}. However, these prior methods still perform batch-mode meta-training, while our method also performs the meta-training itself incrementally online, without task boundaries.


The closest work to ours is the follow the meta-leader (FTML) method~\citep{finn19a} and other online meta-learning methods~\citep{yao2020online}. FTML is a varaint of MAML that finetunes to each new task in turn, resetting to the meta-trained parameters between every task. While this effectively accelerates acquisition of new tasks, it requires ground truth knowledge of task boundaries and, as we show in our experiments, our approach outperforms FTML \emph{even when FTML has access to task boundaries and our method does not}. Note that the memory requirements for such methods increase with the number of adaptation gradient steps, and this limitation is also shared by our approach. Online-within-online meta-learning~\cite{denevi2019online} also aims to accelerate online updates by leveraging prior tasks, but still requires knowledge of task boundaries. MOCA~\cite{harrison2020continuous} instead aims to \emph{infer} the task boundaries. In contrast, our method does not even attempt to find the task boundaries, but directly adapts without them. A number of related works also address continual learning via meta-learning, but with the aim of minimizing catastrophic forgetting~\cite{gupta2020maml, caccia2020online}. Our aim is not to address catastrophic forgetting. Our method also meta-trains from small datasets for thousands of tasks, whereas prior continual learning approaches typically focus on settings with fewer larger tasks (e.g., 10-100 tasks).


\section{Problem Statement} \label{sec:imitation}
This part describes the problem of learning a robust policy in a demonstration and computationally efficient way by imitating an \ac{MPC} expert demonstrator. Robustness and efficiency are determined by the ability to design an \ac{IL} procedure that can compensate for the covariate shifts induced by uncertainties encountered during real-world deployment while collecting demonstrations in a domain (the training domain) that presents only a subset of those uncertainty realizations. %
Our problem statement follows the one of robust \ac{IL} (e.g., DART \cite{laskey2017dart}), modified to use deterministic policies/experts and to account for the differences in uncertainties encountered in deployment and training domains. 
Additionally, we present a common approach employed to address the covariate shift issues caused by uncertainties, \ac{DR}, highlighting its limitations. 

\subsection{Assumptions and Notation}
\label{sec:transfer}
\noindent
\textbf{System Dynamics.} 
We assume the dynamics of the real system are Markovian and stochastic~\cite{sutton2018reinforcement}, and can be described by a twice continuously differentiable function $f(\cdot)$:
\begin{equation}
\label{eq:system_dynamics}
\vbf{x}_{t+1} = f(\vbf{x}_t, \vbf{u}_t) + \vbf{w}_t,
\end{equation}
where $\vbf{x}_t \in \mathbb{X} \subseteq \mathbb{R}^{n_x}$ represents the state, $\vbf{u}_t \in \mathbb{U} \subseteq \mathbb{R}^{n_u}$ the control input in the compact subsets $\mathbb{X}$, $\mathbb{U}$. $\vbf{w}_t \in \mathbb{W}_\mathcal{D} \subset \mathbb{R}^{n_x}$ is an unknown state perturbation, belonging to a compact convex set $\mathbb{W}_\mathcal{D}$ containing the origin. Stochasticity in \cref{eq:system_dynamics} is introduced by $\vbf{w}_t$, sampled from $\mathbb{W}_\mathcal{D}$ under a (possibly unknown) probability distribution, capturing the effects of noise, approximation errors in the learned policy, model changes, and other disturbances acting on the system during training or under real-world conditions at deployment. 

\noindent
\textbf{Sim2Real and Lab2Real Transfer Setup.}
Two different domains $\mathcal{D}$ are considered: a training domain $\mathcal{S}$ (\textit{source}) and a deployment domain $\mathcal{T}$ (\textit{target}). The two domains differ in their transition probabilities, and we assume that $\mathbb{W}_\mathcal{S} \subset \mathbb{W}_\mathcal{T}$, representing the fact that training is usually performed in simulation or in a controlled/lab environment under some nominal model errors/disturbances ($\vbf{w} \in \mathbb{W}_\mathcal{S}$), while at deployment a larger set of perturbations ($\vbf{w} \in \mathbb{W}_\mathcal{T}$) can be encountered. 

\noindent
\textbf{MPC Expert.} We assume that we are given an \ac{MPC} expert demonstrator that plans along an $N+1$-steps horizon. 
The expert is given the current state $\vbf{x}_t$, and $\vbf{X}_t^\text{des} \in \mathbb{X}_M^\text{des} \coloneqq \{\mathbb{X}\}_{i=0}^{M}$, representing a desired state to be reached ($M=0$), or a state trajectory to be followed ($M = N+1$). Then, the \ac{MPC} expert generates control actions by solving an \ac{OC} problem of the form: 
\begin{equation}
\begin{split}
\bar{\mathbf{X}}_t^*, \bar{\mathbf{U}}_t^*
    \in \underset{\bar{\mathbf{X}}_t, \bar{\mathbf{U}}_t}{\text{argmin}} & 
       \: J_{N,M}(\bar{\mathbf{X}}_t, \bar{\mathbf{U}}_t, \mathbf{X}^\text{des}_t) \\
    \text{subject to} \:\: & \bar{\vbf{x}}_{0|t} = \vbf{x}_t, \\
    & \bar{\mathbf{x}}_{i+1|t} = f(\bar{\mathbf{x}}_{i|t}, \bar{\mathbf{u}}_{i|t}),  \\
    & \bar{\mathbf{x}}_{i|t} \in \mathbb{X}, \:\: \bar{\mathbf{u}}_{i|t} \in \mathbb{U}, \\
    & i = 0, ..., N-1.
\end{split}
\end{equation}
where $J_{N,M}$ represents the cost to be minimized (where $N$ denotes the dependency on the planning horizon, and $M$ on the type of task, i.e., tracking or reaching a goal state), and $\bar{\mathbf{X}}_t = \{\bar{\vbf{x}}_{0|t},\dots,\bar{\vbf{x}}_{N|t}\}$ and $\bar{\vbf{U}}_t = \{\bar{\vbf{u}}_{0|t},\dots,\bar{\vbf{u}}_{N-1|t}\}$ are sequences of states and actions along the planning horizon, where the notation $\bar{\vbf{x}}_{i|t}$ indicates the planned state at the future time $t+i$, as planned at the current time $t$. 
At every timestep $t$, given $\vbf{x}_t$, the control input applied to the real system is the first element of $\bar{\vbf{U}}^*_t$, resulting in an implicit deterministic control law (policy) that we denote as $\pi_{\vbs{\theta}^*}: \mathbb{X} \times \mathbb{X}_M^\text{des} \rightarrow \mathbb{U}$.

\noindent 
\textbf{\ac{DNN} Student Policy.}
As for the \ac{MPC} expert, we model the \ac{DNN} student policy as a deterministic policy $\pi_{\vbs{\theta}}$, with parameters $\vbs{\theta}$.
When considering trajectory tracking tasks, the policy takes as input the current state and the desired reference trajectory segment, $\pi_{\vbs{\theta}} : \mathbb{X} \times \mathbb{X}_{N+1}^\text{des} \rightarrow \mathbb{U}$. When considering the task of reaching a goal state, the policy takes as input the current state, the desired goal state and the current timestep $t \in \mathbb{I}_{\geq 0}$, $\pi_{\vbs{\theta}}: \mathbb{X} \times \mathbb{X}_0^\text{des} \times \mathbb{I}_{\geq 0} \rightarrow \mathbb{U}$.




\noindent
\textbf{Transition Probabilities}. 
We denote the state transition probability under $\pi_{\vbs{\theta}}$ in a domain $\mathcal{D}$ for a given goal-reaching or trajectory tracking task
as $p_{\pi_{\vbs{\theta}}, \mathcal{D}}(\vbf{x}_{t+1}|\vbf{x}_t)$. 
We highlight that the probability of collecting a $T$-step trajectory $\boldsymbol{\xi} = \{ \vbf{x}_0, \vbf{u}_0, \vbf{x}_1, \vbf{u}_1, \dots, \vbf{x}_{T} \}$, given a policy $\pi_{\vbs{\theta}}$, depends on the deployment environment $\mathcal{D}$:  
\begin{equation}
    p(\boldsymbol{\xi}|\pi_{\vbs{\theta}}, \mathcal{D}) = p(\vbf{x}_0) \prod_{t=0}^{T-1} p_{\pi_{\vbs{\theta}}, \mathcal{D}}(\vbf{x}_{t+1}|\vbf{x}_t),
\end{equation}
where $p(\vbf{x}_0)$ represents the initial state distribution. 

\noindent
\subsection{Robust Imitation Learning Objective}
The objective of robust \ac{IL}, following~\cite{laskey2017dart}, is to find parameters $\vbs{\theta}$ of $\pi_{\vbs{\theta}}$ that minimize a distance metric $\mathcal{L}(\vbs{\theta}, \vbs{\theta}^*|\boldsymbol{\xi})$ from the \ac{MPC} expert $\pi_{\vbs{\theta}^*}$. This metric captures the differences between the actions generated by the expert $\pi_{\vbs{\theta}^*}$ and the action produced by the student $\pi_{\vbs{\theta}}$ across the distribution of trajectories induced by the student policy $\pi_{\vbs{\theta}}$, in the perturbed domain $\mathcal{T}$:
\begin{equation}
    \hat{\vbs{\theta}}^* = \text{arg}\min_{\vbs{\theta}} \mathbb{E}_{p(\boldsymbol{\xi}|\pi_{\vbs{\theta}}, \mathcal{T})}\mathcal{L}(\vbs{\theta}, \vbs{\theta}^*|\boldsymbol{\xi}).
    \label{eq:il_obj_target}
\end{equation}
The distance metric considered in this work is the \ac{MSE} loss:
\begin{equation}
\mathcal{L}(\vbs{\theta}, \vbs{\theta}^*|\boldsymbol{\xi}) = \frac{1}{T}\sum_{t=0}^{T-1}\| \pi_{\vbs{\theta}}(\vbf{x}^\text{in}_t) - \pi_{\vbs{\theta}^*}(\vbf{x}_t, \vbf{X}_t^\text{des})\|_2^2.
\end{equation}
where $\vbf{x}^\text{in}_t = \{\vbf{x}_t, \vbf{X}_t^\text{des}\}$ for trajectory tracking tasks, and $\vbf{x}^\text{in}_t = \{\vbf{x}_t, \vbf{X}_t^\text{des}, t\}$ for go-to-goal-state tasks.

\noindent
\textbf{Covariate Shift due to Sim2real and Lab2real Transfer.}
Because in practice we do not have access to the target environment, the goal of Robust IL is to try to solve~\cref{eq:il_obj_target} by finding an approximation of the optimal policy parameters $\hat{\vbs{\theta}}^*$ using data from the source environment: 
\begin{equation}
    \hat{\vbs{\theta}}^* = \text{arg}\min_{\vbs{\theta}} \mathbb{E}_{p(\boldsymbol{\xi}|\pi_{\vbs{\theta}}, \mathcal{S})}\mathcal{L}(\vbs{\theta}, \vbs{\theta}^*|\boldsymbol{\xi}).
    \label{eq:il_obj_source}
\end{equation}
The way this minimization is solved depends on the chosen \ac{IL} algorithm. The performance of the learned policy in the target and source domains can be related via: 
\begin{equation}
\begin{multlined}
\hspace*{-0.45in}    \mathbb{E}_{p(\boldsymbol{\xi}|\pi_{\vbs{\theta}}, \mathcal{T})}\mathcal{L}(\vbs{\theta}, \vbs{\theta}^*|\boldsymbol{\xi}) = \\
    \underbrace{
    \mathbb{E}_{p(\boldsymbol{\xi}|\pi_{\vbs{\theta}}, \mathcal{T})}\mathcal{L}(\vbs{\theta}, \vbs{\theta}^*|\boldsymbol{\xi}) - 
    \mathbb{E}_{p(\boldsymbol{\xi}|\pi_{\vbs{\theta}}, \mathcal{S})}\mathcal{L}(\vbs{\theta}, \vbs{\theta}^*|\boldsymbol{\xi})}_{\text{covariate shift due to transfer}} \\ + 
    \underbrace{\mathbb{E}_{p(\boldsymbol{\xi}|\pi_{\vbs{\theta}}, \mathcal{S})}\mathcal{L}(\vbs{\theta}, \vbs{\theta}^*|\boldsymbol{\xi})}_{\text{\ac{IL} objective}},
\end{multlined}
\label{eq:}
\end{equation}
which clearly shows the presence of a covariate shift induced by the transfer. The last term corresponds to the objective minimized by performing \ac{IL} in $\mathcal{S}$. Attempting to solve \cref{eq:il_obj_target} by directly optimizing \cref{eq:il_obj_source} (e.g., via \ac{BC}~\cite{pomerleau1989alvinn}) offers no assurances of finding a policy with good performance in $\mathcal{T}$. 


\subsection{Compensating Transfer Covariate Shifts via Domain Randomization.}
\label{subsec:domain_randomization}
A well-known strategy to compensate for the effects of covariate shifts between source and target domain is \ac{DR}~\cite{peng2018sim}, which modifies the transition probabilities of the source $\mathcal{S}$ by trying to ensure that the trajectory distribution in the modified training domain $\mathcal{S}_\text{DR}$ matches the one encountered in the target domain: $p(\boldsymbol{\xi}|\pi_{\vbs{\theta}}, \mathcal{S}_\text{DR}) \approx p(\boldsymbol{\xi}|\pi_{\vbs{\theta}}, \mathcal{T})$.
This is done by applying perturbations to the robot during demonstration collection, sampling perturbations $\vbf{w} \in \mathbb{W}_\text{DR}$ according to some knowledge/hypotheses on their distribution $p_\mathcal{T}(\vbf{w})$ in the target domain~\cite{peng2018sim}, obtaining the perturbed trajectory distribution $p(\boldsymbol{\xi}|\pi_{\vbs{\theta}}, \mathcal{S}, \vbf{w})$. The minimization of \cref{eq:il_obj_target} can then be approximately performed by minimizing instead:
\begin{equation}
\label{eq:il_dr_modified_source}
    \mathbb{E}_{p_\mathcal{T}(\vbf{w})}[\mathbb{E}_{p(\boldsymbol{\xi}|\pi_{\vbs{\theta}}, \mathcal{S}, \vbf{w})}\mathcal{L}(\vbs{\theta}, \vbs{\theta}^*|\boldsymbol{\xi})].
\end{equation}
This approach, however, requires the ability to apply disturbances/model changes to the system, which may be unpractical e.g., in the \textit{lab2real} setting, and may require a large number of demonstrations due to the need to sample enough state perturbations $\vbf{w}$.



\section{Efficient Learning from Linear RTMPC} \label{sec:rtmpc_linear}

\label{sec:robust_tube_mpc}
In this Section, we present the strategy to efficiently learn robust policies from \ac{MPC} when the system dynamics in \cref{eq:system_dynamics} can be well approximated by a linear model of the form:
\begin{equation}
\label{eq:linearized_dynamics}
\vbf{x}_{t+1} = \vbf{A} \vbf{x}_t + \vbf{B} \vbf{u}_t + \vbf{w}_t.
\end{equation}
First, we present the Robust Tube variant of linear MPC, \ac{RTMPC}, that we employ to collect demonstrations (\cref{subsec:rtmpc_expert}). Then, we present a strategy that leverages information available from the \ac{RTMPC} expert to compensate for the covariate shifts caused by uncertainties and mismatches between the training and deployment domains (\cref{sec:sa}). Our strategy is based on a \ac{DA} procedure that can be combined with different \ac{IL} methods (on-policy, such as \ac{DAgger} \cite{ross2011reduction}, and off-policy, such as \ac{BC}, \cite{pomerleau1989alvinn}) for improved efficiency/robustness in the policy learning procedure. The \ac{RTMPC} expert is based on \cite{mayne2005robust} but with the objective function modified to track desired trajectories, as trajectory-tracking tasks will be the focus of the experimental evaluation of policies learned from this controller (\cref{sec:evaluation_linear}). 

\subsection{Trajectory Tracking \ac{RTMPC} Expert Formulation} \label{subsec:rtmpc_expert}
\ac{RTMPC} is a type of robust \ac{MPC} that regulates the system in \cref{eq:linearized_dynamics} while ensuring satisfaction of the state and actuation constraints $\mathbb X, \mathbb U$ regardless of the disturbances $\mathbf w \in \mathbb{W}_\mathcal{T}$. 


\noindent
\textbf{Mathematical Preliminaries.} Let $\mathbb{A} \subset \mathbb{R}^{n}$ and $\mathbb{B} \subset \mathbb{R}^{n}$ be convex polytopes, and let $\mathbf{C} \in \mathbb{R}^{m \times n}$ be a linear mapping. In this context, we establish the following definition:
\begin{enumerate}[a)]
\item Linear mapping: $\mathbf C \mathbb{A} \coloneqq \{\mathbf{C} \mathbf a \in \mathbb{R}^m \:|\: \mathbf a \in \mathbb{A}\}$
\item Minkowski sum: $\mathbb{A} \oplus \mathbb{B} \coloneqq \{\mathbf a + \mathbf b \in \mathbb{R}^n \:|\: \mathbf a \in \mathbb{A}, \: \mathbf b \in \mathbb{B} \}$
\item Pontryagin diff.: $\mathbb{A} \ominus \mathbb{B} \coloneqq \{\mathbf c \in \mathbb{R}^n \:|\: \mathbf{c + b} \in \mathbb{A}, \forall \mathbf  b \in \mathbb{B} \}$. 
\end{enumerate}
\textbf{Optimization Problem.} 
At each time step $t$, trajectory tracking \ac{RTMPC} receives the current robot state $\mathbf x_t$ and a desired trajectory $\mathbf{X}^\text{des}_t = \{\xdes_{0|t},\dots,\xdes_{N|t}\}$ spanning $N+1$ steps as input. It then computes a sequence of reference (``safe'') states $\bar{\mathbf{X}}_t = \{\xsafe_{0|t},\dots,\xsafe_{N|t}\}$ and actions $\bar{\mathbf{U}}_t = \{\usafe_{0|t},\dots,\usafe_{N-1|t}\}$ that ensure constraint compliance regardless of the realization of $\mathbf{w}_t \in \mathbb{W}_\mathcal{T}$. This  is achieved by solving the following optimization problem:
\begin{align}
\label{eq:rtmpc_optimization_problem}
    \mathbf{\bar{U}}_t^*, \mathbf{\bar{X}}_t^* &= \underset{\mathbf{\bar{U}}_t, \mathbf{\bar{X}}_t}{\text{argmin}}
        \| \mathbf e_{N|t} \|^2_{\mathbf{P}_x} + 
        \sum_{i=0}^{N-1} 
            \| \mathbf e_{i|t} \|^2_{\mathbf{Q}_x} + 
            \| \mathbf u_{i|t} \|^2_{\mathbf{R}_u} \notag \\
    &\text{subject to} \:\:  \xsafe_{i+1|t} = \mathbf A \xsafe_{i|t} + \mathbf B \usafe_{i|t}, \\
    &\xsafe_{i|t} \in \mathbb{X} \ominus \mathbb{Z}, \:\: \usafe_{i|t} \in \mathbb{U} \ominus \mathbf{K} \mathbb{Z}, \notag\\
    &\vbf{x}_t \in \mathbb{Z} \oplus  \xsafe_{0|t}, \: i = 0, \dots, {N-1}  \notag
\end{align}
where $\mathbf e_{i|t} = \xsafe_{i|t} - \xdes_{i|t}$ is the tracking error. The positive definite matrices $\mathbf{Q}_x$,
$\mathbf{R}_u$ 
define the trade-off between deviations from the desired trajectory and actuation usage, while $\| \mathbf e_{N|t} \|^2_{\mathbf{P}_x}$ is the terminal cost. $\mathbf{P}_x$ 
and $\mathbf K$ are obtained by formulating an infinite horizon optimal control LQR problem using $\mathbf A$, $\mathbf B$, $\mathbf{Q}_x$ and $\mathbf{R}_u$ and by solving the associated algebraic Riccati equation \cite{aastrom2021feedback}.
To achieve recursive feasibility, we ensure a sufficiently long prediction horizon is selected, as commonly practiced \cite{kamel2017model}, while omitting the inclusion of terminal set constraints.

\noindent
\textbf{Tube and Ancillary Controller.}  A control input for the real system is generated by \ac{RTMPC} via an \textit{ancillary controller}:
\begin{equation}
\label{eq:ancillary_controller}
    \mathbf u_t = \usafe^*_{t} + \mathbf K (\mathbf{x}_t - \xsafe^*_{t}),
\end{equation}
where $\usafe^*_t = \usafe^*_{0|t}$ and $\xsafe^*_t = \xsafe^*_{0|t}$. 
As shown in \cref{fig:tube_illustration}, this controller ensures that the system remains inside a \textit{tube} (with ``cross-section'' $\mathbb{Z}$) centered around $\xsafe_t^*$ regardless of the realization of the disturbances in $\mathbb{W}_\mathcal{T}$, provided that the tube contains the initial state of the system (constraint $\vbf{x}_t \in \mathbb{Z} \oplus \xsafe_{0|t}$).
The set $\mathbb{Z}$ is a disturbance invariant set for the closed-loop system $\mathbf{A}_K := \mathbf{A + B K}$, satisfying the property that $\forall \mathbf{x}_j \in \mathbb{Z}$, $\forall \mathbf{w}_j \in \mathbb{W}_\mathcal{T}$, $\forall j \in \mathbb{N}^+$, $\mathbf{x}_{j+1} = \mathbf{A}_K \mathbf{x}_j + \mathbf{w}_j \in \mathbb{Z}$ \cite{mayne2005robust}. 
$\mathbb{Z}$ can be computed offline using $\mathbf{A}_K$ and the model of the disturbance $\mathbb{W}$ via ad-hoc analytic algorithms~\cite{borrelli2017predictive, mayne2005robust}, or can be learned from data~\cite{fan2020deep}. 

\begin{figure}
    \centering
    \includegraphics[width=0.9\columnwidth, trim={0.3in, 1.2in, 0.3in, 7.2in}, clip]{figs/tube_diagram/augmented_sampling_tube_mpc_journal.drawio.pdf}
    \caption{
    Illustration of the robust control invariant tube $\mathbb{Z}$
    centered around the optimal reference $\bar{\mathbf{x}}_0^*(\mathbf{x}_t)$
    computed by RTMPC at every state $\mathbf{x}$, for a system with state dimension $n_x = 2$.}
    \label{fig:tube_illustration}
\end{figure}


\subsection{Covariate Shift Compensation via Sampling Augmentation}
\label{sec:sa}
We observe that training a policy by collecting demonstrations in a controlled source domain $\mathcal{S}$, with the objective of deploying it in the target domain $\mathcal{T}$, may introduce a sample selection bias \cite{kouw2018introduction}, i.e., demonstrations are collected only around the nominal state distribution, and not around the distribution induced by perturbations encountered in the \textit{sim2real} and \textit{lab2real} transfer. Such selection bias is a known cause of distribution shifts \cite{kouw2018introduction}, and it is usually mitigated by re-weighting collected samples in a way that takes into account their likelihood of appearing in the target domain $\mathcal{T}$ (e.g., via importance-sampling). These approaches, however, do not apply in our case, since we do not have access to samples/demonstrations collected in $\mathcal{T}$. 


In this work, we propose to mitigate the covariate shift introduced by the policy generation procedure not only by collecting demonstrations from the \ac{RTMPC} but by using additional information computed in the controller.
Specifically, during the collection of a trajectory $\boldsymbol{\xi}$ in the source domain $\mathcal{S}$, we utilize instead the tube computed by the \ac{RTMPC} demonstrator to obtain knowledge of the states that the system may visit when subjected to perturbations. Given this information, we propose a tube-guided \ac{DA} strategy, called \acf{SA}, that samples states from the tube. The corresponding actions are provided at low computational cost by the ancillary controller of the \ac{RTMPC} expert. The collected state-actions pairs are then included in the dataset of demonstrations used to train a policy. The following paragraphs frame the tube sampling problem in the context of covariate shift reduction in \ac{IL}, and discuss tube-sampling strategies. %

\noindent 
\textbf{Tube as a Model of State Distribution Under Uncertainties.} 
The key intuition of the proposed approach is the following. We observe that, although the density $p(\boldsymbol{\xi}|\pi_{\vbs{\theta}},\mathcal{T})$ is unknown, an approximation of its support $\mathfrak{R}$, given a demonstration $\boldsymbol{\xi}$ collected in the source domain $\mathcal{S}$, is known. Such support corresponds to the tube computed by the \ac{RTMPC} when collecting $\boldsymbol{\xi}$:
\begin{equation}
    \mathfrak{R}_{\boldsymbol{\xi}^+|\pi_{\vbs{\theta}^*},\boldsymbol{\xi}} = \bigcup_{t=0}^{T-1}\{\bar{\vbf{x}}_{t}^* \oplus \mathbb{Z}\}.
    \label{eq:tube}
\end{equation}
where $\boldsymbol{\xi}^+$ is a trajectory in the tube of $\boldsymbol{\xi}$.
This is true thanks to the ancillary controller in \cref{eq:ancillary_controller}, which ensures that the system remains inside \cref{eq:tube} for every possible realization of $\vbf{w} \in \mathbb{W}_\mathcal{T}$.
The ancillary controller additionally provides a computationally efficient way to obtain the actions to apply for every state inside the tube. Let $\vbf{x}_{t,j}^+ \in \bar{\vbf{x}}_t^* \oplus \mathbb{Z}$, i.e., $\vbf{x}_{t,j}^+$ is a state inside the tube computed when the system is at $\vbf{x}_t$, then the corresponding robust control action $\vbf{u}_{t,j}^+$ is:
\begin{equation}
\vbf{u}_{t,j}^+ = \bar{\vbf{u}}_t^* +  \vbf{K}(\vbf{x}_{t,j}^+ - \bar{\vbf{x}}_t^*).
\label{eq:tubempc_feedback_policy}
\end{equation}
For every timestep $t$ in $\boldsymbol{\xi}$, extra state-action samples $(\vbf{x}_{t,j}^+, \vbf{u}_{t,j}^+)$, with $j = 1, \dots, N_s$ collected from within the tube can be used to augment the dataset employed to train the policy, obtaining a way to approximate the expected risk in the domain $\mathcal{T}$ by only having access to demonstrations collected in $\mathcal{S}$: 
\begin{equation}
\begin{multlined}
    \mathbb{E}_{p(\boldsymbol{\xi}|\pi_{\vbs{\theta}}, \mathcal{T})}\mathcal{L}(\vbs{\theta}, \vbs{\theta}^*|\boldsymbol{\xi}) \approx \\
    \mathbb{E}_{p(\boldsymbol{\xi}|\pi_{\vbs{\theta}}, \mathcal{S})}[\mathcal{L}(\vbs{\theta}, \vbs{\theta}^*|\boldsymbol{\xi}) +  \mathbb{E}_{p(\boldsymbol{\xi}^+|\pi_{\vbs{\theta}^*},\boldsymbol{\xi})}\mathcal{L}(\vbs{\theta}, \vbs{\theta}^*|\boldsymbol{\xi}^+)].
    \label{eq:sampling_augmentation}
\end{multlined}
\end{equation}
The demonstrations in the source domain $\mathcal{S}$ can be collected using existing \ac{IL} techniques, such as \ac{BC} or \ac{DAgger}.

\noindent
\textbf{Tube Approximation and Sampling Strategies.}
\begin{figure}
    \centering
    \includegraphics[width=\columnwidth]{figs/icra/sampling_extra_states_drawio_v4-cropped.pdf}
    \caption{The possible strategies to sample extra state-action pairs from an axis-aligned bounding box approximation of the tube of the \ac{RTMPC} expert: dense (left) and sparse (right). The tube is represented for a system with state dimension $n_x = 3$.}
    \label{fig:tube_sampling_strategies} %
\end{figure}
In practice, the density $p(\boldsymbol{\xi}^+|\boldsymbol{\xi}, \pi_{\vbs{\theta}^*})$ may not be available, making it difficult to establish which states to sample for \ac{DA}. We consider an adversarial approach to the problem by sampling states that may be visited under worst-case perturbations. To efficiently compute those samples, we (outer) approximate the tube $\mathbb{Z}$ with an axis-aligned bounding box $\hat{\mathbb{Z}}$. We investigate two strategies, shown in~\cref{fig:tube_sampling_strategies}, to obtains state samples $\vbf{x}_{t,j}^+$ at every state $\vbf{x}_t$ in $\boldsymbol{\xi}$:
\begin{inparaenum}[i)]
\item dense sampling: sample extra states from the vertices of $\bar{\vbf{x}}_t^*\oplus\hat{\mathbb{Z}}$. The approach produces $N_s = 2^{n_x}$ extra state-action samples. It is more conservative, as it produces more samples, but more computationally expensive.
\item sparse sampling: sample one extra state from the center of each \textit{facet} of $\bar{\vbf{x}}_t^* \oplus\hat{\mathbb{Z}}$, producing  $N_s = 2n_x$ additional state-action pairs. It is less conservative and more computationally efficient.
\end{inparaenum}



\section{Efficient Learning from Nonlinear RTMPC} \label{sec:rtmpc_nonlinear}
In this Section, we design an \ac{IL} and \ac{DA} strategy, which is an extension of the one presented in \cref{sec:rtmpc_linear}, that enables robust and efficient policy learning from an \ac{MPC} that employs nonlinear models of the form in \cref{eq:system_dynamics}. Different from \cref{sec:rtmpc_linear}, the focus here is on obtaining policies capable of reaching a desired goal state, as this will enable acrobatic maneuvers -- the scenario considered in the evaluation of policies learned from this controller (\cref{sec:evaluation_nonlinear}). %
To accomplish this, first, we use a nonlinear version of \ac{RTMPC}, based on \cite{mayne2011tube}, to collect demonstrations that account for the effects of uncertainties. This expert is summarized in \cref{subsec:rtnmpc_expert}. Second, we develop a computationally efficient tube-guided \ac{DA} strategy leveraging the ancillary controller of the nonlinear \ac{RTMPC} expert. Unfortunately, unlike in the linear \ac{RTMPC} case, nonlinear \ac{RTMPC} \cite{mayne2011tube} uses \ac{NMPC} as an ancillary controller. This limits the computational efficiency in \ac{DA}, as the generation of extra state-action samples requires solving a large \ac{NLP} associated with the ancillary \ac{NMPC} (discussed in \cref{subsec:ancillary_nonlinear_mpc}). We overcome this issue by presenting, in \cref{subsec:sensitivity}, a time-varying linear feedback law, approximation of the ancillary \ac{NMPC}, that enables efficient generation of the extra data. This is done by leveraging the sensitivity of the control input to perturbations in the states visited during an initial demonstration collection procedure. 
Finally, in \cref{subsec:rob_perf_under_approx_samples}, we address the approximation errors introduced by the sensitivity-based \ac{DA} by presenting strategies to mitigate the gap, in performance and robustness, between the learned policy and the \ac{MPC} expert.


\subsection{Nonlinear RTMPC Expert Formulation} \label{subsec:rtnmpc_expert}
Nonlinear RTMPC \cite{mayne2011tube} ensures state and actuation constraint satisfaction while controlling a nonlinear, uncertain system of the form in \cref{eq:system_dynamics}. This controller operates by solving two \acp{OCP}, one to compute a nominal safe plan, and one to track the safe plan (ancillary \ac{NMPC}).

\noindent
\textbf{Nominal Safe Planner.} \label{subsec:nmpc_nominal_plan}
The first \ac{OCP}, given an $N+1$-steps planning horizon, generates nominal safe state and action open-loop plans 
$\vbf{Z}_{\tzr} = \{\vbf{z}_{0|\tzr}, \dots, \vbf{z}_{N|\tzr}\}, \vbf{V}_{\tzr} = \{\vbf{v}_{0|\tzr},\dots, \vbf{v}_{N-1|\tzr}\}$. 
The plans are open-loop because they are generated only at time $\tzr$ when the desired state and action pair $\vbf{X}^\text{des}_{\tzr} = \{\vbf{x}_{\tzr}^e, \vbf{u}_{\tzr}^e \}$, equilibrium pair for the nominal system, changes. The nominal safe plan is obtained from: 
\begin{equation} \label{eq:nmpc_nominal}
\begin{split}
\vbf{V}_{\tzr}^*, \vbf{Z}_{\tzr}^*
    = \underset{\vbf{V}_{\tzr}, \mathbf{Z}_{\tzr}}{\text{argmin}} & 
       \: J_\text{RTNMPC}(\vbf{Z}_{\tzr}, \vbf{V}_{\tzr}, \vbf{X}_{\tzr}^\text{des}) \\
    \text{subject to} \:\: & \mathbf{z}_{i+1|\tzr} = f(\mathbf{z}_{i|\tzr}, \mathbf{v}_{i|\tzr}),  \\
    & \mathbf{z}_{i|\tzr} \in \bar{\mathbb{Z}}, \:\: \mathbf{v}_{i|\tzr} \in \bar{\mathbb{V}}, \\
    & \mathbf{z}_{0|\tzr} = \vbf{x}_{\tzr}, \:\: \mathbf{z}_{N|\tzr} = \vbf{x}_{\tzr}^e.
\end{split}
\end{equation}
$J_\text{RTNMPC} = \sum_{i=0}^{N-1} \| \vbf{z}_{i|\tzr} - \vbf{x}^e_{\tzr} \|^2_{\mathbf{Q}_z} + \| \vbf{v}_{i|{\tzr}} - \vbf{u}^e_{\tzr} \|^2_{\mathbf{R}_v}$, where $\mathbf{Q}_z$, $\mathbf{R}_v$ are positive definite. A key idea in this approach involves imposing modified state and actuation constraints $\bar{\mathbb{Z}} \subset \mathbb{X}$ and $\bar{\mathbb{V}} \subset \mathbb{U}$ so that the generated nominal safe plan is at a specific distance from state and actuation constraints. 
To be more precise, similar to the linear \ac{RTMPC} case (\cref{eq:rtmpc_optimization_problem}), the given state constraints $\mathbb{X}$ and actuation constraints $\mathbb{U}$ are tightened (made more conservative) by an amount that accounts for the spread of trajectories induced by the ancillary controller when the system is subject to uncertainties, obtaining $\bar{\mathbb{Z}} \subset \mathbb{X}$ and $\bar{\mathbb{V}} \subset \mathbb{U}$. Different from the linear case, however, analytically computing the tightened constraints is challenging. Fortunately, as highlighted in \cite{mayne2011tube}, accurately computing these sets is not needed, and an outer approximation is sufficient. This approximation can be obtained via Monte-Carlo simulations \cite{mayne2011tube} of the system under disturbances, or learned \cite{fan2020deep}.  %

\noindent
\textbf{Ancillary \ac{NMPC}.}
The second \ac{OCP} corresponds to a trajectory tracking \ac{NMPC}, that acts as an ancillary controller, to maintain the state of the uncertain system close to the reference generated by \cref{eq:nmpc_nominal}. The \ac{OCP} is:
\begin{equation} 
\begin{split} \label{eq:ancillary_nmpc_eq}
    \mathbf{\bar{U}}_t^*, \mathbf{\bar{X}}_t^*
    = \underset{\mathbf{\bar{U}}_t, \mathbf{\bar{X}}_t}{\text{argmin}} & 
        \| \mathbf e_{N|t} \|^2_{\mathbf{P}_x} \!\! + \!\!
        \sum_{i=0}^{N-1} 
            \| \mathbf e_{i|t} \|^2_{\mathbf{Q}_x} \!\!+\! 
            \| \bar{\vbf{u}}_{i|t}\!-\!\vbf{v}_{i + t|\tzr}^* \|^2_{\mathbf{R}_u} \\
    \text{subject to} \:\: &  \xsafe_{i+1|t} = f(\xsafe_{i|t}, \usafe_{i|t})  \\
    & \xsafe_{0|t} = \vbf{x}_t, \usafe_{i|t} \in \bar{\mathbb{U}} \\
\end{split}
\end{equation}
where $\mathbf e_{i|t} = \xsafe_{i|t} - \vbf{z}^*_{i + t - t_0|\tzr}$ is the state tracking error. The positive definite matrices $\mathbf{Q}_x$ and $\mathbf{R}_u$ are tuning parameters and can differ from the ones in \cref{eq:nmpc_nominal}, while $\mathbf{P}_x$ defines a terminal cost. \cref{eq:ancillary_nmpc_eq} is solved at each timestep using the current state $\vbf{x}_t$, while the action applied to the robot is $\vbf{u}_t = \bar{\vbf{u}}^*_{0|t}$. %
We note that the ancillary \ac{NMPC} can have different tuning parameters than \cref{eq:nmpc_nominal}, including the discretization time of the dynamics and the prediction horizon $N$, providing additional degrees of freedom to shape the response of the system under uncertainties.    %

A key result of the employed nonlinear \ac{RTMPC} \cite{mayne2011tube} is that the ancillary \ac{NMPC} in \cref{eq:ancillary_nmpc_eq} maintains the trajectories of the uncertain system in \cref{eq:system_dynamics} inside state and action tubes $\mathbb{T}^\text{state} \subset \mathbb{R}^{n_x}, \mathbb{T}^\text{action} \subset \mathbb{R}^{n_u}$ that contain the current nominal safe state and action trajectories $\mathbf{z}_{t|\tzr}^*$, $\mathbf{v}_{t|\tzr}^*$ from the \ac{OCP} in \cref{eq:nmpc_nominal}. The state and action tubes are used to obtain the tightened state and actuation constraints $\bar{\mathbb{Z}}$, $\bar{\mathbb{V}}$, ensuring constraint satisfaction.  %

\subsection{Solving the Ancillary \ac{NMPC}}
\label{subsec:ancillary_nonlinear_mpc}
A large portion of the computational cost of deploying or collecting demonstrations from nonlinear \ac{RTMPC} comes from the need to solve the \ac{OCP} of the ancillary NMPC (\cref{eq:ancillary_nmpc_eq}) at each timestep. In contrast, the \ac{OCP} of the nominal safe plan (\cref{eq:nmpc_nominal}) can be solved once per task (e.g., whenever the desired goal state $\vbf{X}^\text{des}_{\tzr}$ changes).%


A state-of-the-art method to solve the \ac{OCP} in \cref{eq:ancillary_nmpc_eq} that yields high-quality solutions is Multiple Shooting \cite{rawlings2017model}. In this approach, an \ac{OCP}, discretized over a time grid, is transformed in a \ac{NLP} that tries to find an optimal sequence of states and actions. This is done by forward-simulating the nonlinear dynamics on each interval (e.g., via an implicit \ac{RK} integrator) while imposing additional continuity conditions. The resulting \ac{NLP} can be solved via a \ac{SQP}, i.e., by repeatedly:
\begin{inparaenum}[i)]
\item linearizing the \ac{NLP} around a given linearization point; 
\item generating and solving a corresponding \ac{QP}, obtaining a refined linearization point for the next \ac{SQP} iteration.
\end{inparaenum}
While capable of producing high-quality solutions, \ac{SQP} methods incur large computational requirements due to the need of performing computationally-expensive system linearization and solving the associated \ac{QP} one or more times per timestep.



\subsection{Computationally-Efficient Data Augmentation using the Parametric Sensitivities} \label{subsec:sensitivity}
The tube $\mathbb{T}^\text{state}$ induced by the ancillary controller in \cref{eq:ancillary_nmpc_eq} can be used to identify relevant regions of the state space for \ac{DA}, as it approximates the support of the state distribution under uncertainties, as discussed in \cref{sec:sa}. However, generating the corresponding extra action samples using \cref{eq:ancillary_nmpc_eq} can be very computationally inefficient, as it requires solving the associated \ac{SQP} for every extra state sample,
making \ac{DA} computationally impractical, and defeating our initial objective of designing \textit{computationally efficient} \ac{DA} strategies.  

In this work, we extend our efficient \ac{DA} strategy, \acf{SA}, to efficiently learn policies from nonlinear \ac{RTMPC} by proposing instead to employ a time-varying, linear approximation of the ancillary \ac{NMPC} -- enabling efficient generation of extra state-action samples. Specifically, we observe that \cref{eq:ancillary_nmpc_eq} solves the implicit feedback law:
\begin{equation}
\label{eq:ancillary_nmpc_implicit}
\vbf{u}_t \! = \! \bar{\vbf{u}}_{0|t}^*(\vbs{\chi}_t) \! \coloneqq \! \kappa(\vbs{\chi}_t), \;  \vbs{\chi}_t \! \coloneqq \! \{\!\vbf{x}_t, \!t; \!\vbf{V}_{\tzr}^*, \vbf{Z}_{\tzr}^*\}
\end{equation}
where we have denoted the current inputs $\vbs{\chi}_t$ for convenience. Then, for each timestep of the trajectory collected during a demonstration in the source environment $\mathcal{S}$, with current ancillary \ac{NMPC} input $\tilde{\vbs{\chi}}_t = \{ \tilde{\vbf{x}}_t, \tilde{t}; \!\vbf{V}_{\tzr}^*, \vbf{Z}_{\tzr}^* \}$, we generate a local linear approximation of \cref{eq:ancillary_nmpc_implicit} by computing the first-order sensitivity of $\vbf{u}_t$ to the initial state $\vbf{x}_t$: %
\begin{equation} \label{eq:sensitivity_matrix}
\vbf{K}_{\tilde{\vbs{\chi}}_t} \!\!
\coloneqq 
\!
\left.
\frac{\partial \vbf{\bar{u}}_{0|t}^*}{\partial \vbf{x}_t} 
\right|_{{\vbs{\chi}}_t = \tilde{\vbs{\chi}}_t}
\!\!\!\!\! = \!\!
\begin{bmatrix}
\left.\dfrac{\partial{\bar{\vbf{u}}_{0|t}^*}}{\partial{\left[ \vbf{x}_t \right]_{1}}}
\right|_{\tilde{\vbs{\chi}}_t},\!\!& 
\!\!
\dots, 
\!\!
& \!\!
\left.
\dfrac{\partial{\bar{\vbf{u}}_{0|t}^*}}{\partial{[\vbf{x}_t]_{n_x}}}
\right|_{\tilde{\vbs{\chi}}_t}
\\
\end{bmatrix}.
\end{equation}
The sensitivity matrix $\vbf{K}_{\tilde{\vbs{\chi}}_t} \in \mathbb{R}^{n_u \times n_x}$, 
enables us to compute extra actions $\vbf{u}_{t,j}^+$ from states $\vbf{x}_{t,j}^+ \in \mathbb{T}^\text{state}$, with $j = 1, \dots, N_s$, sampled from the tube: 
\begin{equation}
\label{eq:approximate_ancillary_controller}
    \vbf{u}_{t,j}^+ = \vbf{\bar{u}}_{0|t}^* + \vbf{K}_{\tilde{\vbs{\chi}}_t} (\vbf{x}_{t,j}^+ - \vbf{\bar{x}}_{0|t}^*) \coloneqq \hat{\kappa}(\vbf{x}_{t,j}^+, \tilde{\vbs{\chi}}_t).
\end{equation}
The \ac{DA} procedure enabled by this approximation is computationally-efficient, as we do not need to solve an \ac{SQP} for each extra state-action sample $(\vbf{x}^+_{t,j}, \vbf{u}^+_{t,j})$ generated for \ac{DA}, and we only need to compute, once per timestep, the sensitivity matrix $\vbf{K}_{\tilde{\vbs{\chi}}_t}$.
We note that the feedback-response introduced by the so-generated extra state-action samples can be interpreted as a type of linearized neighboring feedback control \cite[\S 1.3.1]{diehl2001real}, where the linear feedback law is computed based on the trajectory $\vbs{\xi}$ executed during demonstration collection in the source environment $\mathcal{S}$, rather than a reference trajectory $\vbf{Z}_\tzr^*, \vbf{V}_\tzr^*$. We remark, additionally, that the actions computed when collecting demonstrations are obtained by solving the entire \ac{SQP}, and the sensitivity-base approximation is used only for \ac{DA}. 

\noindent
\textbf{Sensitivity Matrix Computation.} As described in \cite[\S 8.6]{rawlings2017model}, an expression to compute the sensitivity matrix in \cref{eq:sensitivity_matrix} (also called \textit{tangential predictor}) can be obtained by re-writing the \ac{NLP} in \cref{eq:ancillary_nmpc_eq} in a parametric form $\mathfrak{p}(\left[ \vbf{x}_t \right]_i)$, highlighting the dependency on scalar parameter representing the $i$-th component of the initial state $\vbf{x}_t$ (part of $\vbs{\chi}_t$). The parametric \ac{NLP} $\mathfrak{p}(\left[ \vbf{x}_t \right]_i)$ is: %
\begin{equation}
\begin{split}
\label{eq:nmpc_opt_problem} 
\vspace{-.5in}
\underset{\vbf{y}}{\text{min}} & \: F_{\vbs{\chi}_t}(\vbf{y}) \\
    \text{subject to} \:\: & G_{\vbs{\chi}_t}(\left[ \vbf{x}_t \right]_i, \vbf{y}) = \vbs{0} \\
    & H(\vbf{y}) \leq \vbs{0},
\end{split}
\end{equation}
where $\vbf{y} \in \mathbb{R}^{n_{\vbf{y}}}$ corresponds to the optimization variables in \cref{eq:ancillary_nmpc_eq}, and $F_{{\vbs{\chi}}_t}(\cdot), G_{{\vbs{\chi}}_t}(\cdot), H(\cdot)$ are, respectively, the objective function, equality, and inequality constraints in \cref{eq:ancillary_nmpc_eq}, given the current state and reference trajectory in $\vbs{\chi}_t$. Additionally, we denote the solution of \cref{eq:nmpc_opt_problem} at $\tilde{\vbs{\chi}}_t$ (computed during the collected demonstration) as $(\tilde{\vbf{y}}^*, \tilde{\vbs{\lambda}}^*, \tilde{\vbs{\mu}}^*)$, where $\tilde{\vbs{\lambda}}^*, \tilde{\vbs{\mu}}^*$ are, respectively, the Lagrange multipliers for the equality and inequality constraints at the solution found. Then, each $i$-th column of the sensitivity matrix (\cref{eq:sensitivity_matrix}) can be computed by solving the following \ac{QP} (\cite[Th. 8.16]{rawlings2017model}, and \cite[Th. 3.4 and Remark 4]{diehl2001real}): 
\begin{equation}
\begin{aligned}
\underset{\mathbf{y}}{\text{min}} \quad& \hspace{-.1in} F_{\vbs{\chi}_t,L}(\mathbf{y}; \tilde{\mathbf{y}}^*) 
+ \frac{1}{2}(\mathbf{y} - \tilde{\mathbf{y}}^*)^\top \nabla^2_{\mathbf{y}} \mathscr{L}(\tilde{\mathbf{y}}^*, \tilde{\boldsymbol{\lambda}}^*, \tilde{\boldsymbol{\mu}}^*)(\mathbf{y} - \tilde{\mathbf{y}}^*) \\
\text{s.t.} \quad & G_{\vbs{\chi}_t,L}([\mathbf{x}_t]_i, \mathbf{y}; \tilde{\mathbf{y}}^*) = \mathbf{0} \label{eq:tangential_predictor_qp} \\
& H_L(\mathbf{y}; \tilde{\mathbf{y}}^*) \leq \mathbf{0}
\end{aligned}
\end{equation}
where $F_{\vbs{\chi}_t,L}(\cdot; \tilde{\vbf{y}}^*)$, $G_{\vbs{\chi}_t,L}(\cdot; \tilde{\vbf{y}}^*)$, $H_L(\cdot; \tilde{\vbf{y}}^*)$ denote the respective functions in \cref{eq:nmpc_opt_problem} linearized at the solution found. $\nabla^2_{\vbf{y}}\mathscr{L}$ denotes the Hessian of the Lagrangian associated with \cref{eq:nmpc_opt_problem}, while the parameter is set to zero ($\left[ \vbf{x}_t \right]_i = 0$). The $i$-th column of the sensitivity matrix can be extracted from the entries of $\vbf{y}^*$, solution of \cref{eq:tangential_predictor_qp}, at the position corresponding to $\bar{\vbf{u}}_{0|t}$.
We highlight that \cref{eq:tangential_predictor_qp} can be computed efficiently, as it leverages the latest internal linearization of the \ac{KKT} conditions performed in the \ac{SQP} employed to solve \cref{eq:ancillary_nmpc_eq}, and therefore it does not require to re-execute the computationally expensive system linearization routines that are carried out at each \ac{SQP} iteration.
We note that this local approximation exists when the assumptions in \cite[Th. 8.15]{rawlings2017model} (equivalent to \cite[Th. 3.3]{diehl2001real}) and \cite[Th. 3.3, Remark 2, 4]{diehl2001real} are satisfied, i.e., that the solution $(\tilde{\vbf{y}}^*, \tilde{\vbs{\lambda}}^*, \tilde{\vbs{\mu}}^*)$ found during demonstration collection is a strongly regular \ac{KKT} point, and the assumptions, i.e., that the solution found satisfies strict complementary conditions. 
Last, extra samples are generated using \cref{eq:approximate_ancillary_controller} under the assumption that the set of active inequality constraints (i.e., the index set $p \in \{1, \dots, n_H\}$ such that $[H(\tilde{\vbs{y}}^*)]_p = 0$) does not change.

\noindent
\textbf{Generalized Tangential Predictor} A strategy that applies to the cases where strict complementary conditions do not hold, or where the extra state samples cause a change in the active set of constraints, is based on the \textit{generalized tangential predictor} \cite[\S 8.9.1]{rawlings2017model}. 
This predictor can be obtained by solving the \ac{QP} in \cref{eq:tangential_predictor_qp} with the set of equality constraints modified to be $G_{\vbs{\chi}_t,L}(\vbf{x}_{t,j}^+, \vbf{y}; \tilde{\vbf{y}}) = \vbf{0}$ \cite[Eq. 8.60]{rawlings2017model}. %
Although this approach requires solving a \ac{QP} to compute the action $\vbf{u}_{t,j}^+$ corresponding to each state $\vbf{x}_{t,j}^+$ sampled from the tube, it does not require re-generating the computationally expensive linearization performed at each \ac{SQP} iteration (and other performance optimization routines, such as condensing \cite{rawlings2017model}) nor solving the entire \ac{SQP} for multiple iterations -- resulting in a much more computationally-efficient procedure than solving the entire $\ac{SQP}$ \textit{ex-novo} for every extra state-action sample. We remark that the linearization point in \cref{eq:tangential_predictor_qp} is updated at every timestep when a full \ac{SQP} is solved as part of demonstration-collection.





\subsection{Robustness and Performance Under Approximate Samples} \label{subsec:rob_perf_under_approx_samples}
While the proposed sensitivity-based \ac{DA} strategy enables the efficient generation of extra state-action samples, it introduces approximation errors that may affect the performance and robustness of the learned policy. Here, we discuss strategies to account for these errors, reducing the gaps between the nonlinear \ac{RTMPC} expert and the learned policy in terms of robustness and performance.

\noindent
\textbf{Robustness.}
A key property of \ac{RTMPC} is the ability to explicitly account for uncertainties, including the ones introduced by the proposed sensitivity-based \ac{DA} framework, by further tightening state and actuation constraints for the nominal safe plan (\cref{eq:nmpc_nominal}). The general nonlinear formulation of the dynamics in \cref{eq:system_dynamics}, however, makes it challenging to compute an \textit{exact} additional tightening bound for state and actuation constraints. A possible avenue to establish a tightening procedure for the actuation constraints is to observe that the linear approximation of \cref{eq:ancillary_nmpc_implicit} introduces an error upper bounded by~(\cite[Th. 8.16]{rawlings2017model}):
\begin{equation}
\| \kappa(\vbs{\chi}_t) - \hat{\kappa}(\!\vbf{x}^+_{t,j}, \vbs{\chi}_t)\|  \leq D \|\vbf{x}^+_{t,j} - \vbf{x}_t\|^2
 \end{equation}
where $D$ may be obtained by considering the Lipschitz constant of the controller (e.g. \cite{krishnamoorthy2022sensitivity}). However, estimating this constant may be difficult, or computationally expensive, for large-dimensional systems, as is the case herein. 
An alternative is to update the tubes as was done in \cref{subsec:nmpc_nominal_plan}, e.g., by employing Monte-Carlo simulations of the closed-loop system, starting from an initial (possibly conservative) tightening guess and by iteratively adjusting the cross-section (size) of the tube, or by directly learning the tubes from simulations or previous (conservative) real-world deployments \cite{fan2020deep}. These procedures, when performed using the learned policy, are particularly appealing in our context, as our efficient policy learning methodologies enable rapid training/updates of the learned policy, and the computational efficiency of the policy enables rapid simulations.   



\noindent
\textbf{Performance Improvements via Fine-Tuning}
In the context of learning policies from nonlinear \ac{RTMPC}, we include in \ac{SA} an (optional) fine tuning-step. This fine-tuning step consists in training the policy with additional demonstrations, without \ac{DA}, therefore avoiding introducing further approximate samples, and having discarded the extra data used to train the policy after an initial demonstration. More specifically, the overall \ac{SA} procedure for nonlinear \ac{RTMPC} with the optional fine-tuning step consists of the following: 
\begin{enumerate}[1)]
\item Collect a single task demonstration $\vbs{\xi}$ (e.g., reach goal) from the nonlinear \ac{RTMPC} expert; 
\item Perform \ac{DA} using the parametric sensitivity (\cref{subsec:sensitivity}) and train the policy, obtaining the policy parameters $\hat{\vbs{\theta}}_0$;
\item Optional \textit{fine-tuning} step:
\begin{enumerate}[i)]
\item Discard the collected data so far, including the data generated by the \ac{DA};
\item Collect new demonstrations (e.g., using DAgger \cite{ross2011reduction} and the pre-trained policy , or \ac{BC}), without \ac{DA}, and re-train the pre-trained policy (with parameters $\hat{\vbs{\theta}}_0$) after every newly collected demonstration.
\end{enumerate}
\end{enumerate}
In this procedure, our proposed tube-guided \ac{DA} is treated as a methodology to efficiently generate an initial guess $\hat{\vbs{\theta}}_0$ of the policy parameters; the policy can then be further fine-tuned, for performance improvements, using newly collected demonstrations via on-policy (e.g, \ac{DAgger}) or off-policy methods (e.g., \ac{BC}).




\section{Application to Agile Flight} \label{sec:agile_flight}
In this Section, we tailor the proposed efficient policy learning strategies to agile flight tasks, as this will be the focus of our numerical and experimental evaluation. First, in \cref{subsec:mav_model}, we present the nonlinear model of the multirotor used to collect demonstrations in simulation in all our approaches, and used for control design. Then, in \cref{subsec:linear_mpc}, we present a \ac{RTMPC} expert for \textit{trajectory tracking} based on a \textit{linear} multirotor model and that will be used with the \ac{IL} procedure described in \cref{sec:rtmpc_linear}. Because the considered trajectories require the robot to operate around a fixed, pre-defined condition (near hover), a hover-linearized model is suitable for the design of this controller. Last, in \cref{subsec:nonlinar_mpc}, we design a nonlinear \ac{RTMPC} expert capable of performing a $360^\circ$ flip in \textit{near-minimum time} - a maneuver that demands exploitation of the full \textit{nonlinear} dynamics of the multirotor, and that requires large and careful actuation usage. This controller will be used with the \ac{IL} procedure described in \cref{sec:rtmpc_nonlinear}. 

\subsection{Nonlinear Multirotor Model} \label{subsec:mav_model}
We consider an inertial reference frame $\text{W}$ attached to the ground, and a non-inertial frame $\text{B}$ attached to the \ac{CoM} of the robot. The translational and rotational dynamics of the multirotor are:  
\begin{subequations} \label{eq:mav_model_full}
\begin{align}
    \vbsd[W]{p} & = \vbs[W]{v} \label{eq:mav_model_full:tr_kin} \\
    \vbsd[W]{v} & = m^{-1}(\vbs{R}_\text{WB} \vbs[B]{t}_\text{cmd} + \vbs[W]{f}_\text{drag} + \vbs[W]{f}_\text{ext}) - \vbs[W]{g} \label{eq:mav_model_full:tr_dyn} \\
    \vbsd{q}_\text{WB} & = \frac{1}{2} \vbs{\Omega} (\vbs[B]{\omega}) \vbs{q}_\text{WB}  \label{eq:mav_model_full:rot_kin} \\
    \vbsd[B]{\omega} &  = \vbs{I}_\text{mav}^{-1}(- \vbs[B]{\omega} \times \vbs{I}_\text{mav} \vbs[B]{\omega} + \vbs[B]{\tau}_\text{cmd} + \vbs[B]{\tau}_\text{drag}) \label{eq:mav_model_full:rot_dyn}
\end{align}
\end{subequations}
where $\boldsymbol{p}$, $\boldsymbol{v}$, $\boldsymbol{q}$, $\boldsymbol{\omega}$ are, respectively, position, velocity, attitude quaternion and angular velocity of the robot, with the prescript denoting the corresponding reference frame. The attitude quaternion $\boldsymbol{q} = [q_w, \boldsymbol{q}_v^\top]^\top$ consists of a scalar part $q_w$ and a vector part $\boldsymbol{q}_v = [q_x, q_y, q_z]^\top$ and it is unit-normalized; the associated $3 \times 3$ rotation matrix is $\boldsymbol{R} = \vbs{R}(\vbs{q})$, while
\begin{equation}
\vbs{\Omega} (\boldsymbol{\omega}) = 
\begin{bmatrix}
0 & -\boldsymbol{\omega}^\top \\
\boldsymbol{\omega} & \lfloor \boldsymbol{\omega} \rfloor_\times \\
\end{bmatrix},
\end{equation} with $\lfloor \boldsymbol{\omega} \rfloor_\times$ denoting the $3 \times 3$ skew symmetric matrix of $\boldsymbol{\omega}$. $m$ denotes the mass, $\vbs{I}_\text{mav}$ the $3 \times 3$ diagonal inertial matrix, and $\boldsymbol{g} = [0, 0, g]^\top$ the gravity vector. Aerodynamic effects are taken into account via $\boldsymbol{f}_\text{drag} = - c_{D,1} \boldsymbol{v} - c_{D,2} \|\boldsymbol{v}\| \boldsymbol{v}$ and isotropic drag torque $\boldsymbol{\tau} = - c_{D,3} \boldsymbol{\omega}$, capturing the parasitic drag produced by the motion of the robot. The robot is additionally subject to external force disturbances $\boldsymbol{f}_\text{ext}$, such as the one caused by wind or by an unknown payload. 
Last, $\boldsymbol{t}_\text{cmd} = [0, 0, t_\text{cmd}]^\top$ is the commanded thrust force, and $\boldsymbol{\tau}_\text{cmd}$ the commanded torque. These commands can be mapped to the desired thrust $f_{\text{prop},i}$ for the $i$-th propeller ($i = 1, \dots, n_p$) via a linear mapping (\textit{allocation} matrix) $\boldsymbol{\mathcal{A}}$: 
\begin{equation}
\label{eq:allocation_matrix}
    \begin{bmatrix}
    t_\text{cmd} \\
    \vbs{\tau}_\text{cmd} \\ 
    \end{bmatrix}
    = \boldsymbol{\mathcal{A}}
    \begin{bmatrix}
    f_{\text{prop},1} \\
    \vdots \\
    f_{\text{prop}, n_p}
    \end{bmatrix} = \boldsymbol{\mathcal{A}} \vbs{f}_\text{prop}.
\end{equation}
The attitude of the quadrotor is controlled via the geometric attitude controller in \cite{lee2011geometric} that generates desired torque commands $\vbs[B]{\tau}_\text{cmd}$ given a desired attitude $\vbs{R}_\text{WB}^\text{des}$, angular velocity $\vbs[B]{\omega}^\text{des}$ and acceleration $\vbsd[B]{\omega}^\text{des}$. The controllers designed in the next sections output setpoints for the attitude controller, and desired thrust $t_\text{cmd}$. 

  

\subsection{Linear \ac{RTMPC} for Trajectory Tracking} \label{subsec:linear_mpc}
The model employed by the linear \ac{RTMPC} for trajectory tracking (\cref{eq:rtmpc_optimization_problem}) is based on a simplified, hover-linearized model derived from \cref{eq:mav_reduced_model_for_nmpc}, using the approach in \cite{kamel2017linear}, but modified to account for uncertainties. First, similar to \cite{kamel2017linear}, we express the model in a yaw-fixed, gravity-aligned frame $\text{I}$ via the rotation matrix $\vbs{R}_\text{BI}$
\begin{equation}
    \begin{bmatrix}
    \phi \\
    \theta \\
    \end{bmatrix} =
    \vbs{R}_\text{BI}
    \begin{bmatrix}
    \prescript{}{I}{\phi} \\
    \prescript{}{I}{\theta} \\
    \end{bmatrix}, \hspace{1pt}
    \vbs{R}_\text{BI} =
    \begin{bmatrix}
    \cos(\psi) & \sin(\psi) \\
    -\sin(\psi) & \cos(\psi) \\
    \end{bmatrix},
\end{equation}
where the attitude has been represented, for interpretability, via the Euler angles yaw $\psi$, pitch $\theta$, roll $\phi$ (\textit{intrinsic} rotations around the $z$-$y$-$x$ such that $\vbs{R} = \vbs{R}_{z}(\psi)\vbs{R}_{y}(\theta) \vbs{R}_{x}(\phi)$, with $\vbs{R}_{l}(\alpha)$ being a rotation of $\alpha$ around the ${l}$-th axis). 
Second, as in \cite{kamel2017linear}, we assume that the closed-loop attitude dynamics can be described by a first-order dynamical system that can be identified from experiments, replacing \cref{eq:mav_model_full:rot_kin}, \cref{eq:mav_model_full:rot_dyn}. 
Last, different from \cite{kamel2017linear}, we assume $\vbs[W]{f}_\text{ext}$ in \cref{eq:mav_model_full:tr_dyn} to be an unknown disturbance/model errors that capture the uncertain parts of the model, such that $\vbs[W]{f}_\text{ext} \in \mathbb{W}$.

The controller generates tilt (roll, pitch) and thrust commands ($n_u = 3$) given the state of the robot ($n_x=8$) consisting of position, velocity, and tilt, and given the reference trajectory. 
The desired yaw is fixed (and it is tracked by the cascaded attitude controller); similarly, $\vbs[B]{\omega}^\text{des}$ and $\vbsd[B]{\omega}^\text{des}$ are set to zero.  We employ the nonlinear attitude compensation scheme in \cite{kamel2017linear}.

The controller takes into account position constraints (e.g., available 3D flight space), actuation limits, and velocity/tilt limits via $\mathbb{X}$ and $\mathbb{U}$. The cross-section of the tube $\mathbb{Z}$ is a constant outer approximation based on an axis-aligned bounding box. It is estimated via Monte-Carlo sampling, by measuring the state deviations of the closed loop linear system $\vbf{A}_K$ under the disturbances in $\mathbb{W}$.

\subsection{Nonlinear \ac{RTMPC} for Acrobatic Maneuvers} \label{subsec:nonlinar_mpc}

\noindent 
\textbf{Ancillary \ac{NMPC}.}
We start by designing the ancillary \ac{NMPC} (\cref{eq:ancillary_nmpc_eq}). The selected nominal model is the same used in the high-performance trajectory tracking \ac{NMPC} for multirotors~\cite{loquercio2019deep, mueller2013computationally}: 
\begin{equation}
\label{eq:mav_reduced_model_for_nmpc}
\begin{split}
    \vbsd[W]{v} & = m^{-1}(\vbs{R}_\text{WB} \vbs[B]{t}_\text{cmd} + \vbs[W]{f}_\text{drag}) - \vbs[W]{g} \\
    \vbsd[W]{p} & = \vbs[W]{v} \\
    \vbsd{q}_\text{WB} & = \frac{1}{2} \vbs{\Omega} (\vbs[B]{\omega}_\text{cmd})\vbs{q}_\text{WB},
\end{split}
\end{equation}
where the rotational dynamics (\cref{eq:mav_model_full:rot_dyn}) have been neglected, assuming that the cascaded attitude controller enables fast tracking of the desired angular velocity setpoint $\vbs[B]{\omega}_\text{cmd}$.  
The controller uses the state and control input: 
\begin{equation}
\label{eq:ancillary_nmpc_state_and_control_input}
    \bar{\vbf{x}} = [\vbs[W]{p}^\top, \vbs[W]{v}^\top, \vbs{q}_\text{WB}^\top ]^\top, \;\; \bar{\vbf{u}} = [t_\text{cmd}, \vbs[B]{\omega}_\text{cmd}^\top]^\top.
\end{equation} 
The angular acceleration setpoint for the attitude controller $\vbsd[B]{\omega}_\text{cmd}$ is obtained via numerical differentiation, while we do not explicitly generate an attitude setpoint (e.g., we set $\vbs{R}^\text{des}_\text{WB} = \vbs{R}_\text{WB}$) obtaining an attitude controller that only tracks desired angular velocities/accelerations.

\noindent 
\textbf{Near-Minimum Time Safe Plan Generation.}
To compute safe nominal plans for acrobatic maneuvers (by solving the \ac{OCP} in \cref{eq:nmpc_nominal}), we employ an extended version of the full nonlinear dynamic model in \cref{subsec:mav_model}. More specifically, we solve the  \ac{OCP} in \cref{eq:nmpc_nominal} by using the following state $\tilde{\vbf{z}} \in \tilde{\bar{\mathbb{Z}}}$ and control inputs $\tilde{\vbf{v}} \in \tilde{\bar{\mathbb{V}}}$: %
\begin{equation}
\label{eq:agile_fligh:nominal_dynamics}
\tilde{\vbf{z}} =
[
    \vbs[W]{p}^\top, \vbs[W]{v}^\top, \vbs{q}_\text{WB}^\top, \vbs[B]{\omega}^\top, \vbs[B]{f}_\text{prop}^\top 
]^\top
\;\; \tilde{\vbf{v}} = \vbsd[B]{f}_\text{prop},
\end{equation}
where the state has been extended to include the thrust produced by each propeller $\vbs{f}^\text{prop}$ to ensure continuity in the reference thrust, accounting for the unmodeled actuators' dynamics. As for the linear case, uncertainties are modeled by $\vbs[W]{f}_\text{ext} \in \mathbb{W}$.
The cost function captures the near-minimum time objective: 
\begin{equation}
    \tilde{J}_\text{RTNMPC} = T_f + \alpha_1\!\vbs{v}^\top\!\!\vbs{v} + \alpha_2 {\vbs{f}_\text{prop}}^\top\!\!\vbs{f}_\text{prop} + \alpha_3\!\tilde{\vbf{v}}^\top\!\!\tilde{\vbf{v}}
\end{equation}
where $T_f$ is the total time of the maneuver, while the remaining terms act as a regularizer for the optimizer, with $\alpha_i \ll T_f$ (i.e., $\alpha_i \approx 10^{-2}, ~\forall ~i$). 

We note that $\tilde{J}_\text{RTNMPC}$ contains a non-quadratic term, therefore differing from the quadratic cost employed in the safe nominal planner in \cite{mayne2011tube} (our \cref{eq:nmpc_nominal}); such cost function was chosen to automate the selection of the prediction horizon $N$ for the safe nominal plan. Our evaluation will demonstrate that the ancillary \ac{NMPC} maintains the system within a tube from the generated reference, further highlighting the flexibility of the framework. 

Additionally, we note that state and control input (\cref{eq:agile_fligh:nominal_dynamics}) have been extended compared to the ones (\cref{eq:ancillary_nmpc_state_and_control_input}) selected for the ancillary \ac{NMPC}, as emphasized by our notation $\tilde{\cdot}$. For this reason, the optimal safe nominal plan $\tilde{\vbf{Z}}_t^*$, $\tilde{\vbf{V}}_t^*$ found using the extended state needs to be mapped to the reference trajectory for the ancillary \ac{NMPC}, $\vbf{Z}_t^*$ $\vbf{V}_t^*$. This is done by simply selecting position, velocity and attitude from $\tilde{\vbf{Z}}_t^*$ to obtain $\vbf{Z}_t^*$. The thrust setpoint $t_\text{cmd}$ in $\vbf{V}_t^*$ is computed via $\vbs{\mathcal{A}}$ in \cref{eq:allocation_matrix} from $\vbs{f}_\text{prop}$ in $\tilde{\vbf{Z}}_t^*$, while the angular velocity setpoint $\vbs{\omega}_\text{cmd}$ is obtained by assuming it equal to the angular velocity $\vbs{\omega}$ in $\tilde{\vbf{Z}}_t^*$.

\noindent 
\textbf{Constraints.} The state constraint $\bar{\vbf{x}}_t \in \mathbb{X}$ encodes the maximum safe linear velocity $\vbs{v}$ and position boundaries $\vbs{p}$ of the environment, while actuation constraints $\bar{\vbf{u}}_t \in \mathbb{U}$ account for physical limits of the robot, restricting the nominal angular velocities $\vbs{\omega}_\text{cmd}$ (to prevent saturation of the onboard gyroscope), and the maximum/minimum thrust force $t_\text{cmd}$ produced by the propellers.
We impose tightened constraints on the thrust force by constraining $\vbs{f}_\text{prop}$  in $\tilde{\vbf{z}} \in \tilde{\bar{\mathbb{Z}}}$. These constraints are obtained via a conservative approach, i.e. we require a minimal thrust to generate a trajectory feasible within our position and velocity constraints. 
This was done to ensure that sufficient control authority is left to the ancillary \ac{NMPC} to account for the presence of large unknown aerodynamic effects and mismatches in the mapping from commanded thrust/actual thrust. This cautious approach enabled successful real-world execution of the maneuver without further real-world tuning. We additionally leverage the further degrees of freedom introduced by the extended state $\tilde{\vbf{z}}$ by shaping the safe plan through upper-bounding the thrust rates $\dot{\vbs{f}}_\text{prop}$ via $\tilde{\bar{\mathbb{V}}}$, although this constraint will not be enforced by the ancillary \ac{NMPC}. Last, using Monte-Carlo closed-loop simulations with disturbances sampled from $\mathbb{W}$, we verify that $\mathbb{X}$ and $\mathbb{U}$ are satisfied, and we generate a constant estimate (outer approximation, axis-aligned bounding box) of the cross-section of the tubes $\mathbb{T}^\text{state}$ and $\mathbb{T}^\text{action}$.




\noindent 
\textbf{Tube and Data Augmentation with Attitude Quaternions.}
The normalized attitude quaternion, part of the states $\bar{\vbf{x}}$, $\tilde{\vbf{z}}$ of nonlinear \ac{RTMPC}, and part of the reference $\vbf{Z}_t^*$ for the ancillary \ac{NMPC}, does not belong to a vector space, and therefore it is not trivial to describe its tube nor to generate extra samples for \ac{DA}. In this work, we employ an attitude error representation $\vbs{\epsilon} \in \mathbb{R}^3$ based on the \ac{MRP} \cite{shuster1993survey} to generate a representation that can be treated as an element of a vector space. Specifically, we use
\begin{equation}
\vbs{\epsilon}_t = \text{MRP}(\vbs{q}_t \odot {\vbs{q}^*_t}^{-1}),
\end{equation}
where $\vbs{q}_t$ is the current attitude, ${\vbs{q}_t^*}$ is the desired attitude (from the safe plan $\vbf{z}_t^*$), $\text{MRP}(\cdot)$ maps a quaternion to the corresponding three-dimensional attitude representation, while $\odot$ denotes the quaternion product.






\section{Evaluation - Learning From Linear RTMPC} \label{sec:evaluation_linear}
We start by evaluating our policy learning approach for the task of trajectory tracking using the linear \ac{RTMPC} expert. 

\subsection{Evaluation Approach and Details}
\noindent
\textbf{Simulation Environment.} Demonstration collection and policy evaluations are performed in a simulation environment implementing the nonlinear multirotor dynamics in \cref{subsec:mav_model}, discretized at $200$Hz, while the attitude controller runs at $100$Hz. The robot follows desired trajectories, starting from randomly generated initial states centered around the origin. Given the specified external disturbance magnitude bound $\mathbb{W}_\mathcal{D} = \{ f_\text{ext} \in \mathbb{R}| \underline{f}_\text{ext} \leq f_\text{ext} \leq \overline{f}_\text{ext}\}$, disturbances are applied in the domain $\mathcal{D}$ by sampling $\vbs[W]{f}_\text{ext}$ via the spherical coordinates: 
\vspace{-0.3cm}
\begin{equation} \label{eq:disturbance_sampling}
\vbs[W]{f}_\text{ext} = 
f_\text{ext} 
\begin{bmatrix}
\cos(\phi) \sin(\theta) \\
\sin(\phi) \sin(\theta) \\
\cos(\theta)
\end{bmatrix}, ~~~
\begin{aligned}
&f_\text{ext} \sim \mathcal{U}(\underline{f}_\text{ext}, \overline{f}_\text{ext}), \\
&\theta \sim \mathcal{U}(0, \pi), \\
&\phi \sim \mathcal{U}(0, 2\pi).
\end{aligned}
\end{equation}
\vspace{-0.3cm}

\noindent
\textbf{Linear \ac{RTMPC}.}
The linear \ac{RTMPC} demonstrator runs at $10$Hz. The reference fed to the expert is a sequence of desired positions and velocities for the next $3$s, discretized with a sampling time of $0.1$s; the expert uses a corresponding planning horizon of $N=30$, (resulting in a reference being a $180$-dim. vector). The controller is designed under the assumption that $\mathbb{W} = \{f_\text{ext} \in \mathbb{R} | 0 \leq f_\text{ext} \leq 0.3 mg\}$, corresponding to the safe physical limit of the actuators of the robot.

\noindent
\textbf{Policy Architecture.} The student policy is a $2$-hidden layer, fully connected \ac{DNN}, with $32$ neurons per layer, and \texttt{ReLU} activation function. The total input dimension of the \ac{DNN} is $188$ (matching the input of the expert, consisting of state and reference trajectory). The output dimension is $3$ (desired thrust and tilt expressed in an inertial frame). We rotate the tilt output of the \ac{DNN} in the body frame to avoid taking into account yaw, which is not part of the optimization problem \cite{kamel2017linear}, not causing any relevant computational cost. We additionally apply the nonlinear attitude compensation scheme as in \cite{kamel2017linear}.

\noindent
\textbf{Baselines and Training Details.} We apply the proposed \ac{SA} strategies to every demonstration collected via \ac{DAgger} or \ac{BC}, and compare their performance against the two without \ac{SA} and the two combined with \ac{DR}. 
During demonstration-collection in the source environment $\mathcal{S}$, we do not apply disturbances, setting $\mathbb{W}_\mathcal{S} = \{ \emptyset \}$, with the exception for \ac{DR}, where we sample disturbances from $\mathbb{W}_\text{DR} = \mathbb{W}_\mathcal{T}$. 
In all the methods that use \text{DAgger} we set the probability of using actions of the expert $\beta$ (a hyperparameter of DAgger \cite{ross2011reduction}) to be $1$ at the first demonstration and $0$ otherwise (as this was found to be the best-performing setup).
Demonstrations are collected with a sampling time of $0.1$s. After every collected demonstration, the policy is trained for $50$ epochs\footnote{One epoch is one pass through the entire dataset} using all the data available so far with the ADAM \cite{kingma2014adam} optimizer and a learning rate of $0.001$. The policy is then evaluated on the task for $20$ times (episodes), starting from slightly different initial states centered around the origin, in both $\mathcal{S}$ and $\mathcal{T}$. 

\noindent 
\textbf{Evaluation Metrics:}
We monitor:
\begin{enumerate}[i)]
    \item \textit{Robustness (Success Rate)}, as the percentage of episodes where the robot never violates any state constraint;
    \item \textit{Performance}, via either
        \begin{enumerate}[a)]
            \item $C_{\vbs{\xi}}({\pi_\theta}) := \sum_{t=0}^{T}\|\vbf{x}_t - \vbf{x}_t^\text{des}\|^2_{\vbf{Q}_x}+ \|\vbf{u}_t\|_{\vbf{R}_u}^2$ tracking error along the trajectory (\textit{MPC Stage Cost}); or
            \item $\|C_{\vbs{\xi}}({\pi_{{\theta}^*}}) - C_{\vbs{\xi}}(\pi_{\hat{\theta}^*})\|/\|C_{\vbs{\xi}}(\pi_{{\theta}^*})\|$ relative error between expert and policy tracking errors (\textit{Expert Gap}); 
        \end{enumerate} 
    \item \textit{Efficiency}
    \begin{enumerate}
        \item number of expert demonstrations (\textit{Num. Demonstrations Used for Training}), and
        \item wall-clock time to generate the policy (\textit{Training Time}~\footnote{Training time is the time to collect demonstrations and the time to train the policy, as measured by a wall-clock. In our evaluations, the simulated environment steps at its highest possible rate (in contrary to running at the same rate of the simulated physical system), providing an advantage to those methods that require a large number of environment interactions, such as the considered baselines.}).
    \end{enumerate}
\end{enumerate}

\subsection{Numerical Evaluation of Demonstration-Efficiency, Robustness, and Performance when Learning to Track a Single Trajectory}

\begin{figure}
\captionsetup[sub]{font=footnotesize}
\centering
\begin{subfigure}{\columnwidth}
    \centering
    \includegraphics[width=\columnwidth, trim={2cm 0 2cm 0.5cm},clip]{figs/icra/disturbance_analysis_v1/robustness.pdf}
\end{subfigure}%
    \caption{Robustness (\textit{Success Rate}) in the task of flying along an eight-shaped, $7$s long-trajectory, subject to wind-like disturbances (right,  target domain $\mathcal{T}_1$) and without (left, source domain $\mathcal{S}$), starting from different initial states. Evaluation is repeated across $8$ random seeds, $20$ times per demonstration per seed. We additionally show the $95\%$ confidence interval. The lines for the SA-based methods overlap.}
    \label{fig:single_trajectory_eval}
    \vskip-1ex
\end{figure}

\begin{table}[tb]
    \caption{Comparison of the \ac{IL} methods considered to learn a policy from RTMPC. We consider the task of tracking a trajectory in a deployment domain with wind-like disturbances ($\mathcal{T}_1$), and one under model errors ($\mathcal{T}_2$, under drag coefficient mismatch). 
    At convergence (iteration 20-30) we evaluate robustness (success rate) and performance (relative percent error between tracking error of expert and policy). Demonstration-Efficiency represents the number of demonstrations required to achieve, for the first time, full success rate. An approach is considered easy if it does not require to apply disturbances/perturbations during training (e.g., in \textit{lab2real} transfer); an approach is considered safe if does not execute actions that may cause state constraints violations (crashes) during training. *Safe in our numerical evaluation, but not guaranteed as it requires executing actions of a policy that may be partially trained.}
    \label{tab:comparison}
\newcolumntype{P}[1]{>{\centering\arraybackslash}p{#1}}
    \tiny
    \renewcommand{\tabcolsep}{1pt}
    \centering
    \begin{tabular}{|p{0.9cm}p{0.9cm}||P{0.7cm}|P{0.7cm}|P{0.7cm}|P{0.7cm}|P{0.7cm}|P{0.7cm}|P{0.7cm}|P{0.7cm}|}
    \hline
    \multicolumn{2}{|c||}{Method} &  
    \multicolumn{2}{c|}{Training} & 
    \multicolumn{2}{c|}{\makecell{Robustness\\succ. rate (\%)}} &
    \multicolumn{2}{c|}{\makecell{Performance\\expert gap (\%)}}  &
    \multicolumn{2}{c|}{\makecell{Demonstration\\Efficiency}}\\
    
        Robustif.           & Imitation     & Easy          & Safe          & $\mathcal{T}_1$                    &  $\mathcal{T}_2$           &  $\mathcal{T}_1$           &  $\mathcal{T}_2$       & $\mathcal{T}_1$        &  $\mathcal{T}_2$\\
        \hline
        \hline
        \multirow{2}{*}{-} & BC          & \textbf{Yes}  & \textbf{Yes}      & < 1                  & 100           & 24.15            & 29.47         & -         & 6 \\
                           & DAgger      & \textbf{Yes}  & No                & 98                   & 100           & 15.79    & 1.34       & 7         & 6 \\ 
        \hline
        \multirow{2}{*}{DR}& BC         & No            & \textbf{Yes}      & 95                    & 100           & 10.04             & 1.27        & 14        & 9 \\ 
         & DAgger                       & No            & No                & \textbf{100}          & 100           & \textbf{4.09}    & 1.45        & 10        & 6 \\
        \hline
        \multirow{2}{*}{SA-Dense}      & BC   & \textbf{Yes}  & \textbf{Yes}      & \textbf{100} & 100    & 25.64     & 1.34      & \textbf{1} & \textbf{1} \\
                                                & DAgger        & \textbf{Yes}  & \textbf{Yes$^*$}  & \textbf{100} & 100    & 10.21     & 1.66      & \textbf{1} & \textbf{1} \\
        \hline
        \multirow{2}{*}{\textbf{SA-Sparse}}    & BC     & \textbf{Yes}  & \textbf{Yes}      & \textbf{100} & 100    & 4.23 & \textbf{1.13}    & \textbf{1} & \textbf{1} \\
                                        & \textbf{DAgger} & \textbf{Yes}  & \textbf{Yes$^*$}   & \textbf{100} & 100    & \textbf{3.75} & \textbf{1.07}    & \textbf{1} & \textbf{1} \\
    \hline
    \end{tabular}%
\end{table}

\noindent
\textbf{Tasks Description.}
Our objective is to generate a policy from linear \ac{RTMPC} capable of tracking a $7$s long ($70$ steps), figure eight-shaped trajectory. We evaluate the considered \ac{IL} approaches in two different target domains, with wind-like disturbances ($\mathcal{T}_1$) or with model errors ($\mathcal{T}_2$). Disturbances in $\mathcal{T}_1$ are external force perturbations $\vbs{f}_\text{ext}$ sampled from $\mathbb{W}_{\mathcal{T}_1} \approx \{f_\text{ext}|0.25mg \leq f_\text{ext} \leq 0.3mg\}$. Model errors in $\mathcal{T}_2$ are applied via mismatches in the drag coefficients used between training and testing, representing uncertainties not explicitly considered during the design of the linear \ac{RTMPC}. 

\noindent
\textbf{Results.} We start by evaluating the robustness in $\mathcal{T}_1$ as a function of the number of demonstrations collected in the source domain.
The results are shown in \cref{fig:single_trajectory_eval}, highlighting that:
\begin{inparaenum}[i)]
\item while all the approaches achieve robustness (full success rate) in the source domain, \ac{SA} achieves full success rate after only a single demonstration, being $5$-$6$ times more sample efficient than the baseline methods;
\item \ac{SA} is able to achieve full robustness in the target domain, while baseline methods do not fully succeed, and converge at a much lower rate. 
\end{inparaenum}
These results emphasize the presence of a distribution shift between the source and target, which is not fully compensated for by baseline methods such as \ac{BC} due to a lack of exploration and robustness.

The performance evaluation and additional results are summarized in \cref{tab:comparison}. We highlight that in the target domain $\mathcal{T}_1$, \ac{SA}-sparse combined with \ac{DAgger} achieves the performance that is closest to the expert.
\ac{SA}-dense suffers from performance drops, potentially due to the limited capacity of the considered \ac{DNN} or challenges in training introduced by this \ac{DA}. 
\cref{tab:comparison} additionally presents the results for the target domain $\mathcal{T}_2$. Although this task is less challenging (i.e., all the approaches achieve full robustness), the proposed method (\ac{SA}-sparse) achieves the highest demonstration-efficiency and lowest expert gap, with similar trends as in $\mathcal{T}_1$. 


\noindent
\textbf{Training Time.}
\cref{fig:single_trajectory_eval} highlights that the best-performing baseline, DAgger+DR, requires about $10$ demonstrations to learn to robustly track a $7$s long trajectory, which corresponds to a total training time of $10.8$s. Among the proposed approaches, DAgger+SA-sparse instead only requires $1$ demonstration, corresponding to a training time of $3.8$s, a $64.8\%$ reduction in wall-clock time required to learn the policy. DAgger+SA-dense, instead, while requiring a single demonstration to achieve full robustness, necessitates $114$s of training time due to the large number of samples generated. 
Because of its effectiveness and greater computational efficiency, we use \ac{SA}-sparse, rather than \ac{SA}-dense, for the rest of the work. 


\begin{table}[t]
    \caption{Computation time required to generated a new action for the linear RTMPC (Expert) and the \ac{DNN} policy (Policy). The \ac{DNN} policy is $25.2$ times faster than the optimization-based expert.  Evaluation performed on an Intel i9-10920.}
    \vskip-1ex
    \label{tab:computational_cost_linear_rtmpc}    \centering
    \begin{tabular} {|C{1.5cm} |C{2.0cm}|C{0.5cm}|C{0.9cm}|C{0.5cm}|C{0.5cm}| }
    \multicolumn{2}{c}{} & \multicolumn{4} {c}{Time (ms)} \\
    \hline
        Method  & Setup & Mean & St. Dev. & Min & Max \\
        \hline 
        \hline 
        Expert (Linear) & \texttt{CVXPY} \cite{diamond2016cvxpy} &  4.28 & 0.39 & 4.21 & 16.66  \\ 
        \textbf{Policy} & PyTorch                               & $\mathbf{0.17}$ & $\mathbf{0.00}$ & $\mathbf{0.17}$ & $\mathbf{0.22}$  \\\hline
    \end{tabular}
\end{table}



\noindent
\textbf{Computation.} \cref{tab:computational_cost_linear_rtmpc} shows that the \ac{DNN} policy is about $25$ times faster than the expert. We additionally report that the average latency for the policy on an \texttt{Nvidia Jetson TX2} (CPU, PyTorch) is $1.66$ms.






\begin{figure}[t]
    \centering
    \begin{subfigure}{0.69\columnwidth}
        \centering
        \includegraphics[height=3.0cm, trim={0.9cm 0.05cm 1.0cm 0.2cm}]{figs/icra/hardware_experiments/uav_with_leaf_blower_hovering_vio_and_thrust_v2}
    \end{subfigure}
    \begin{subfigure}{0.29\columnwidth}
        \centering
        \includegraphics[height=3.0cm, trim={0.4cm 0 0cm 0},clip]{figs/icra/hardware_experiments/uav_with_leaf_blower_hovering_v3.png}
    \end{subfigure}%
    \caption{Experimental evaluation performed by hovering with and without wind disturbances produced by a leaf blower. The employed policy is trained in simulation from a \textit{single} demonstration of the desired trajectory. The wind-like disturbances produce a large position error but do not destabilize the system. The thrust decreases due to the robot being pushed up by the disturbances. The state estimate (shown in the plot) is provided by onboard VIO. The wind points approximately in the same direction as the $y$-axis. 
    }
    \label{fig:experimental_robustness_evaluation_hovering}
\vskip-4ex
\end{figure}
\vskip-2ex

\begin{figure*}
\captionsetup[sub]{font=footnotesize}
\centering
\begin{subfigure}{0.22\paperwidth}
    \centering
    \includegraphics[height=3.5cm, trim={0.5cm 5.5cm 0.25cm 5.5cm},clip]{figs/icra/hardware_experiments/lemniscate_dist_vicon_v4-cropped.pdf}
    \caption{Reference and actual trajectory}
    \label{fig:reference_and_actual_trajectory}
\end{subfigure}%
\hspace{0.1cm}
\begin{subfigure}{0.38\paperwidth}
    \centering
    \includegraphics[height=3.5cm, trim={0.5cm 2cm 1cm 2cm}, clip]{figs/icra/hardware_experiments/uav_leaf_blower_4_hz_v2_smaller_size.png}
    \caption{Time-lapse of an experiment showing the trajectory executed by the robot, and the leaf blowers used to generate disturbances}
    \label{fig:experiment}
\end{subfigure}%
\hspace{0.12cm}
\begin{subfigure}{0.20\paperwidth}
    \centering
    \includegraphics[height=3.5cm, trim={0cm, 0.4cm, 0.4cm, 0.8cm}]{figs/icra/hardware_experiments/leminscate_dist_altitude_and_thrust_vicon_v5}
    \caption{Effects of wind}
    \label{fig:thrust_and_altitude}
\end{subfigure}
    \caption{Experimental evaluation of a trajectory tracking policy learned from a \textit{single} linear \ac{RTMPC} demonstration collected in simulation, achieving \textit{zero-shot} transfer. The multirotor is able to withstand previously unseen disturbances, such as the wind produced by an array of leaf-blowers, and whose effects are clearly visible in the altitude errors (and change in commanded thrust) in \cref{fig:thrust_and_altitude}.  
    This demonstration-efficiency and robustness is enabled by Sampling-Augmentation (SA), our proposed tube-guided data augmentation strategy.}
    \label{fig:experiment_lemniscate}
\end{figure*}

\begin{figure}
    \centering
    \includegraphics[width=\columnwidth, trim={0, 8.5cm, 0, 0.1cm}, clip]{figs/lab2real_transfer/lab2real_transfer_v2.pdf}
    \caption{Example of \textit{lab2real} transfer. One expert demonstration ($1$m step in $x$-$y$-$z$)(1) is sufficient to train a policy that can reproduce the demonstration of the expert (2) and that is robust to previously unseen wind disturbances, accumulating tracking error but not drifting away (3).}
    \label{fig:lab2real_transfer}
\end{figure}

\subsection{Hardware Evaluation for Tracking a Single Trajectory from a Single Demonstration}
\noindent \textbf{Sim2Real Transfers.}
We validate the demonstration-efficiency, robustness, and performance of the proposed approach by experimentally testing policies trained after a \textit{single} demonstration collected in simulation using DAgger/BC (which operate identically since we use DAgger with $\beta=1$ for the first demonstration), combined with \ac{SA}-sparse. We use the MIT/ACL open-source snap-stack \cite{acl_snap_stack} for controlling the attitude of the MAV. The learned policy runs at $100$Hz on the onboard \texttt{Nvidia Jetson TX2} (CPU), with the reference trajectory provided at $100$Hz. State estimation is obtained via a motion capture system or onboard \ac{VIO}. 

The first task considered is to hover under wind disturbances produced by a leaf blower. The results (\cref{fig:experimental_robustness_evaluation_hovering}) highlight the ability of the system to remain stable despite the large position error caused by the wind. 

The second task is to track a figure eight-shaped trajectory, with velocities up to $3.4$m/s. We evaluate the robustness of the learned policy by applying a wind-like disturbance produced by an array of $3$ leaf blowers (\cref{fig:experiment_lemniscate}). The given position reference and the corresponding trajectory are shown in \cref{fig:reference_and_actual_trajectory}. The effects of the wind disturbances are clearly visible in the altitude errors and changes in commanded thrust in \cref{fig:reference_and_actual_trajectory} (at $t=11$s and $t=23$s). 
These experiments show that the learned policy can robustly track the desired reference, withstanding challenging perturbations unseen during the training phase. 

\noindent \textbf{Lab2Real Transfer.} We evaluate the ability of the proposed method to learn from a single demonstration collected on a real robot in a controlled environment (lab) and generalize to previously unseen disturbances (real). We do so by collecting a \ac{RTMPC} demonstration of a $1$m position-step on $x$-$y$-$z$ axes of $\text{W}$ with the multirotor, augmenting the collected demonstration with \ac{SA}-sparse, and deploying the learned policy while we apply previously unseen wind disturbances. As shown in the sequence in \cref{fig:lab2real_transfer} and in our video submission, the policy reproduces the expert demonstration and it is robust to previously unseen wind disturbances.

\begin{figure}
\captionsetup[sub]{font=footnotesize}
\centering
\begin{subfigure}{\columnwidth}
    \centering
    \includegraphics[width=\columnwidth, trim={2cm 2cm 2cm 0},clip]{figs/icra/learn_multi_traj_v2/robustness.pdf}
\end{subfigure}%
\hspace{0.1cm}
\begin{subfigure}{\columnwidth}
    \centering
    \includegraphics[width=\columnwidth, trim={2cm 0 2cm 0},clip]{figs/icra/learn_multi_traj_v2/performance.pdf}
\end{subfigure}
    \caption{Robustness (\textit{Success Rate}, top row) and performance (\textit{MPC Stage Cost}, bottom row) of the proposed approach (with $95\%$ confidence interval), as a function of the number of demonstrations used for training, for the task of learning to track previously unseen circular, eight-shaped and constant position reference trajectories, sampled from the same training distribution. The lines for SA-based methods overlap. The left column presents the results in the training domain (no disturbance), while the right column in the target domain, under wind-like perturbations (with disturbance). The proposed RTMPC-driven SA-sparse strategy learns to track multiple trajectories and generalize to unseen ones requiring fewer demonstrations. At convergence (from demonstration $20$ to $30$), DAgger+SA achieves the closes performance to the expert ($2.7\%$ \textit{expert gap}), followed by BC+SA ($3.0\%$  \textit{expert gap}). Evaluation performed using $20$ randomly sampled trajectories per demonstration, repeated across $6$ random seeds, with a prediction horizon of $N=20$ to speed up demonstration collection, and the DNN input size is adjusted accordingly.}
    \label{fig:learning_multiple_trajectories}
    \centering
    \includegraphics[width=\columnwidth, trim={1.2cm, 0.5cm, 1.2cm, 0.5cm}, clip]{figs/icra/hardware_experiments/learn_multi_traj_10_demonstrations}
    \caption{Examples of different, arbitrary chosen trajectories from the training distribution, tested in hardware experiments with and without strong wind-like disturbances produced by leaf blowers. The employed policy is trained with $10$ demonstrations (when other baseline methods have not fully converged yet, see \cref{fig:learning_multiple_trajectories}) using DAgger+SA (sparse). This highlights that sparse SA can learn multiple trajectories in a more sample-efficient way than other IL methods, retaining RTMPC's robustness and performance. The prediction horizon used is $N=20$, and the \ac{DNN} input size is adjusted accordingly.}
    \label{fig:learn_multiple_trajectories_experiment}
\end{figure}

\subsection{Numerical and Hardware Evaluation for Learning and Generalizing to Multiple Trajectories}
We evaluate the ability of the proposed approach to track multiple trajectories while generalizing to unseen ones. To do so, we define a training distribution of reference trajectories (circle, position step, eight-shape) and a distribution for these trajectory parameters (radius, velocity, position). 
During training, we sample at random a desired, $7$s long ($70$ steps) reference with randomly sampled parameters, collecting a demonstration and updating the proposed policy, while testing on a set of $20$, $7$s long trajectories randomly sampled from the defined distributions. We monitor the robustness and performance of the different methods, with force disturbances (from $\mathbb{W}_{\mathcal{T}_1}$) applied in the target domain. The results of the numerical evaluation, shown in \cref{fig:learning_multiple_trajectories}, confirm that \ac{SA}-sparse
\begin{inparaenum}[i)]
    \item achieves robustness and performance comparable to the expert in a sample efficient way, requiring fewer than half the number of demonstrations needed for the baseline approaches; 
    \item simultaneously learns to generalize to multiple trajectories randomly sampled from the training distribution.
\end{inparaenum}
The hardware evaluation, performed with DAgger augmented via SA-sparse, is shown in \cref{fig:learn_multiple_trajectories_experiment}. It confirms that the obtained policy is experimentally capable of tracking multiple trajectories under real-world disturbances/model errors.

\section{Evaluation - Learning From Nonlinear RTMPC} \label{sec:evaluation_nonlinear}
In this Section, we evaluate the ability of our method to efficiently learn acrobatic flight maneuvers using demonstrations collected from nonlinear \ac{RTMPC}. 

\subsection{Evaluation Approach}
\noindent \textbf{Task Description}
We consider the task of learning a policy capable of performing a flip, i.e., a $360^\circ$ rotation about the body-frame $x$-axis, in near-minimum time. This is a challenging maneuver, as it covers a large nonlinear envelope of the dynamics of the \ac{MAV}, and the near-minimum time objective function, combined with the need to account for uncertainties, pushes the actuators close to their physical limits. 

\noindent
\textbf{Simulation Environment.} The simulation environment for training/numerical evaluations is the same as in \cref{sec:evaluation_linear}, i.e., implements the nonlinear multirotor model in \cref{subsec:mav_model}. In the training domain (source, $\mathcal{S}$), $\mathbb{W}_\mathcal{S} = \{ \emptyset \}$, while in the deployment domain (target, $\mathcal{T}$) $\mathbb{W}_\mathcal{T} = \{f_\text{ext} | 0.001 m g \leq f_\text{ext} \leq 0.3 m g \}$, sampled according to \cref{eq:disturbance_sampling}.

\begin{table}
\caption{Parameters employed for the solution of the ancillary \ac{NMPC} using the optimization package \texttt{ACADOS} \cite{Verschueren2021}.}
\vspace*{-0.1in}
\begin{center}
\begin{tabular}{ r c }
Parameter & Value \\
\hline 
\hline 
Hessian Approximation  & Gauss-Newton \\
QP solver & Partial Condensing \texttt{HPIPM} \cite{frison2020hpipm} \\
NLP Tolerance &  $10^{-8}$ \\
QP  Tolerance & $10^{-8}$ \\
Levenberg-Marquardt & $10^{-4}$\\
Integrator Type & Implicit Runge-Kutta  \\
Max. \# Iterations QP Solver & $100$ \\
Prediction Horizon ($N$, steps) & $50$ \\
Prediction Horizon (time, seconds) & $1.0$ \\
\end{tabular}
\label{tab:acados_parameters}
\end{center}
\end{table}
\noindent \textbf{Nonlinear \ac{RTMPC}.}
We generate a safe nominal flip trajectory using \texttt{MECO-Rockit} \cite{gillis2020effortless}. Because this nominal trajectory happens in the plane spanned by the orthogonal vectors defining the $y$ and $z$ axis of the inertial reference frame $\text{W}$, for simplicity, we project the dynamics onto the $y$ and $z$ axes, resulting in a two-dimensional model of the \ac{MAV} used to generate the nominal plan. The nominal flip trajectory can therefore be obtained by setting the initial rotation around the $z$ to be $0$, and the desired final attitude to be $2 \pi$, while the remaining initial/terminal states are all set to zero. 

The ancillary \ac{NMPC} is solved using the \ac{SQP} solver \texttt{ACADOS} \cite{Verschueren2021}, and runs in simulation at $50$Hz. Sensitivities for \ac{DA} (\cref{eq:tangential_predictor_qp}) are computed using the built-in sensitivity computation in the chosen solver, \texttt{HPIPM} \cite{frison2020hpipm}.
We remark that the employed ancillary \ac{NMPC} uses the full 3D multirotor model in \cref{eq:mav_reduced_model_for_nmpc}, therefore performing 3D disturbance rejection -- a critical requirement for real-world deployments. %
For a more challenging and interesting comparison to the considered \ac{IL} baselines, the ancillary \ac{NMPC} uses the \texttt{SQP\_RTI} setting of \texttt{ACADOS}. This setting performs only a single \ac{SQP} iteration per timestep, enabling significant speed-ups in the solver, and it is often employed in real-time, embedded implementations of \ac{NMPC}. This setting creates an advantage, in terms of training time, to \ac{IL} methods that require querying the expert multiple times (the baselines of our comparison), as it speeds-up the computation time of the expert. 
The other \texttt{ACADOS} parameters given in \cref{tab:acados_parameters} were chosen as they enabled higher overall performance/accuracy in the selected acrobatic maneuver. Last, we introduce a discount factor $\gamma = 0.95$ in the stage cost of \cref{eq:ancillary_nmpc_eq} to aid the convergence of the solver.

\noindent \textbf{Student Policy}
The student policy is a $2$-hidden layer, fully connected \ac{DNN} with $\{64, 32\}$ neurons per layer, and \texttt{ReLU} activation functions. The input vector has dimension $14$, as it contains the current state ($n_x = 10$), the current time $t$, and a desired final position $\vbs{p}^\text{des}$ (fixed to the origin). To simplify the learning and \ac{DA} procedure, we enforce continuity to the quaternion input of the policy using the method in \cite[Eq. 3]{kusaka2022stateful}, avoiding the need to increase the training data/demonstrations at every timestep to account for the fact that $\vbs{q}$ and $-\vbs{q}$ encode the same orientation.



\noindent
\textbf{Baselines and Evaluation Metrics} The choice of baselines matches those in \cref{sec:evaluation_linear} (\ac{DAgger}, \ac{BC} and their combination with \ac{DR}) and so do the monitored metrics (\textit{Robustness}, \textit{Performance} and \textit{Training Time}), with the difference that performance is based on the stage cost of the ancillary \ac{NMPC} \cref{eq:ancillary_nmpc_eq}. 

\noindent \textbf{Training Details}
As in \cref{sec:evaluation_linear}, training is performed by collecting demonstrations with the multirotor starting from slightly different initial states inside the tube centered around the origin. The nominal flip maneuver is pre-generated, as the goal state $\vbf{x}^e_{\tzr}$ (with $\tzr = 0)$ does not change, and only the ancillary \ac{NMPC} is solved at every timestep. The resulting flip maneuver takes about $T_f = 2.5$s, and demonstrations are collected over an episode of length $3.0$s, at $50$Hz ($T = 150$ environment steps per demonstration).
To speed-up the demonstration collection phase, and thereby avoid excessive re-training of the policy, we collect demonstrations in batches for all the baselines, using $10$ demonstrations per batch for $20$ batches. For \ac{SA}-methods, we collect demonstrations one-by-one, and we implement the fine-tuning procedure described in \cref{subsec:rob_perf_under_approx_samples} by performing \ac{DA} with the first collected demonstration, while we do not perform \ac{DA} for the following demonstrations. Because of its computational efficiency, we always use the sensitivity-based \ac{DA} (i.e., \cref{eq:approximate_ancillary_controller}, assuming no changes in active set of constraints). Each time the policy is updated, we evaluate it $20$ times in source $\mathcal{S}$ and target $\mathcal{T}$ environments. The evaluations are averaged across $10$ different random seeds. To further speed-up training of all the methods, we update the previously trained policy using only the newly collected batch of demonstrations (or single demonstration, for \ac{SA}). All the policies are trained using the ADAM optimizer for up to $400$ epochs, but we terminate training if the validation loss (from $30\%$ of the data) does not decrease within $30$ epochs. Last, to study the effects of varying the number of samples used for \ac{DA}, we introduce \ac{SA}-$N_S$, a variant of \ac{SA} where we sample uniformly inside the tube $N_S = \{25, 50, 100\}$ samples for every timestep.

\subsection{Numerical Evaluation of Robustness, Performance, and Efficiency} 
\begin{figure}
    \centering
    \includegraphics[width=\columnwidth, trim={3cm 0.5cm 3.5cm 1cm}, clip]{figs/quadrotor_flip_v3/robustness.pdf}
    \caption{Robustness as a function of the number of training demonstrations. The \ac{IL} methods that use the proposed \ac{DA} strategy, Sampling-Augmentation (SA), overlap on the top-left part of the diagram, achieving full success rate in both the source and target domain of the training environment. Uncertainties in the target domains are applied in the form of a constant external force acting on the center of mass of the multirotor, representing large wind disturbances.}
    \label{fig:nmpc_robustness}
    \centering
    \includegraphics[width=\columnwidth, trim={3cm 0.5cm 3.5cm 1cm}, clip]{figs/quadrotor_flip_v3/training_time_vs_robustness.pdf}
    \caption{Robustness as a function of the training time. The \ac{IL} methods that use the proposed \ac{DA} and fine-tuning strategy, Sampling-Augmentation (SA), achieve full robustness under uncertainties in a fraction of the training time required by the best performing robust baseline, DAgger + DR.}
    \label{fig:nmpc_efficiency}
    \centering
    \includegraphics[width=\columnwidth, trim={3cm 0.5cm 3.5cm 1cm}, clip]{figs/quadrotor_flip_v3/performance.pdf}
    \caption{Performance as a function of the number of training demonstrations. The \ac{IL} methods that use the proposed \ac{DA} and fine-tuning strategy, Sampling-Augmentation (SA), achieve performance close to the expert in less than $10$ demonstrations, while the best-performing baseline, DAgger + DR, achieves comparable performance but in a larger number of demonstrations.}
    \label{fig:nmpc_performance}
\end{figure}
\noindent

\noindent \textbf{Comparison with Baselines.}
We start by evaluating the robustness and performance of the proposed approach as a function of the number of demonstrations collected in simulation, and as a function of the training time. 

\cref{fig:nmpc_robustness} shows the robustness of the considered method as a function of the number of expert demonstrations. It reveals that \ac{SA}-based approaches can achieve full success rate in the environment with disturbances (target, $\mathcal{T}$) and without disturbances (source, $\mathcal{S}$) after a single demonstration, while the best-performing baseline, \ac{DAgger}+\ac{DR}, requires about $60$ demonstrations to achieve full robustness in $\mathcal{S}$, and more than $100$ in $\mathcal{T}$. \ac{SA}-based methods, therefore, enable more than one order of magnitude reduction in the number of demonstrations (interactions with the environment) compared to \ac{DAgger}+\ac{DR}.
As previously observed in \cref{sec:evaluation_linear}, \ac{DAgger} alone is not robust. 
Additionally, \ac{BC} methods fail to converge, potentially due to the lack of sufficiently meaningful exploration and the forgetting caused by the iterative training strategy employed. 

\cref{fig:nmpc_efficiency} additionally shows the robustness as a function of the training time (recall, this includes demonstration collection and policy train). The results show that the demonstration-efficiency of \ac{SA}-based methods translates into significant improvements in training time, as \ac{DAgger}+\ac{DR} requires more than $3$ times the training time than \ac{SA}-based approaches. These improvements are larger for the variants of \ac{SA} that generate fewer extra samples (e.g., \ac{SA}-$25$).

Last, \cref{fig:nmpc_performance} reports the \textit{performance} as a function of the number of demonstrations. The results indicate that \ac{SA}-based methods can achieve low tracking errors even after a single demonstration. Furthermore, employing a fine-tuning phase (after the initial demonstration) proves highly advantageous in further reducing this error, thereby reducing the performance gap between policies obtained via \ac{SA} and the expert.


\noindent \textbf{Comparison of Sampling Strategies.}
\cref{tab:sa_analysis} provides a detailed comparison of performance, robustness, and training time of the different variants of \ac{SA} methods, as a function of the number of demonstrations ($1$, $2$ and $10$), and compares those with the best-performing baselines, \ac{DAgger}+\ac{DR}.  As expected, \ac{SA} methods that require fewer samples obtain significant improvements in training time, while increasing the number of samples is beneficial in reducing the mean and the variance of the expert gap, both with and without disturbances. \cref{tab:sa_analysis} additionally highlights the benefits of fine-tuning, as even methods that use few samples (e.g., SA-sparse, SA-$25$) can obtain a significant performance improvement after a single fine-tuning demonstration ($2$ demonstrations in total), while there are diminishing returns for additional fine-tuning demonstrations (e.g., $10$ demonstrations).     


\begin{table}[t]
    \caption{Computation time required to generated a new action for the nonlinear RTMPC (Expert) and the proposed learned  \ac{DNN} policy (Policy). The policy is $66.1$ times faster than the optimization-based expert. Evaluation performed on an Intel i9-10920. We note that the faster inference time than the linear case is caused by the input dimension being smaller ($14$ vs $188$)}
    \vskip-1ex
    \label{tab:computational_cost_rtnmpc}    \centering
    \begin{tabular} {|C{1.5cm} |C{1.5cm}|C{0.5cm}|C{0.9cm}|C{0.5cm}|C{0.5cm}| }
    \multicolumn{2}{c}{} & \multicolumn{4} {c}{Time (ms)} \\
    \hline
        Method  & Setup & Mean & St. Dev. & Min & Max \\
        \hline 
        \hline 
        Expert (Nonlinear)& \texttt{ACADOS} \cite{Verschueren2021} &  7.28 & 0.15 & 7.05 & 8.00  \\
        \textbf{Policy}   & PyTorch                               & $\mathbf{0.11}$ & $\mathbf{0.01}$ & $\mathbf{0.11}$ & $\mathbf{0.27}$  \\\hline
    \end{tabular}
\end{table}


\noindent
\textbf{Computation.} The computational cost of the nonlinear \ac{RTMPC} expert and the learned policy is reported in \cref{tab:computational_cost_rtnmpc}, highlighting that the policy is $66$ times faster than the expert. The average time to step the training environment is $2.1$ ms.



\begin{table}
    \tiny
    \renewcommand{\tabcolsep}{1pt}
    \centering
    \tiny

\newcommand*{\opacity}{40}%

\definecolor{hight}{HTML}{ec462e} 
\definecolor{lowt}{HTML}{76f013}  %
\newcommand*{\minvalt}{84.0}%
\newcommand*{\maxvalt}{600.0}%
\newcommand{\grdt}[1]{
    \pgfmathparse{int(round(100*(min(max(#1, \minvalt), \maxvalt)/(\maxvalt-\minvalt))-(\minvalt*(100/(\maxvalt-\minvalt)))))}
    \xdef\tempa{\pgfmathresult}
    \cellcolor{hight!\tempa!lowt!\opacity} #1
}

\definecolor{highp}{HTML}{ec462e}
\definecolor{lowp}{HTML}{76f013}
\newcommand*{\minvalp}{29}%
\newcommand*{\maxvalp}{800}%
\newcommand{\grdp}[1]{
    \pgfmathparse{int(round(100*(min(max(#1, \minvalp), \maxvalp)/(\maxvalp-\minvalp))-(\minvalp*(100/(\maxvalp-\minvalp)))))}
    \xdef\tempa{\pgfmathresult}
    \cellcolor{highp!\tempa!lowp!\opacity} #1
}

\definecolor{highr}{HTML}{ffffff}
\definecolor{lowr}{HTML}{ec462e}
\newcommand*{\minvalr}{90}%
\newcommand*{\maxvalr}{100}%
\newcommand{\grdr}[1]{
    \pgfmathparse{int(round(100*(min(max(#1, \minvalr), \maxvalr)/(\maxvalr-\minvalr))-(\minvalr*(100/(\maxvalr-\minvalr)))))}
    \xdef\tempa{\pgfmathresult}
    \cellcolor{highr!\tempa!lowr!\opacity} #1
}

\newcommand{\gradientcell}[6]{
    \ifdimcomp{#1pt}{>}{#3 pt}{#1}{
        \ifdimcomp{#1pt}{<}{#2 pt}{#1}{
            \pgfmathparse{int(round(100*(#1/(#3-#2))-(\minval*(100/(#3-#2)))))}
            \xdef\tempa{\pgfmathresult}
            \cellcolor{#5!\tempa!#4!#6} #1
    }}
}


\newcolumntype{P}[1]{>{\centering\arraybackslash}p{#1}}



\begin{tabular}{|P{0.6cm}P{1.1cm}P{0.75cm}||P{0.5cm}P{0.5cm}|P{0.5cm}P{0.5cm}|P{0.5cm}P{0.5cm}|P{0.5cm}P{0.5cm}|P{0.5cm}P{0.5cm}|}%
\hline
{}     & 
{}     & 
{}     & 
\multicolumn{4}{c|}{\makecell{Robustness\\success rate\\($\%$, $\uparrow$)}} & 
\multicolumn{4}{c|}{\makecell{Performance\\expert gap\\($\%$, $\downarrow$)}} & 
\multicolumn{2}{c|}{\makecell{Efficiency\\training time\\($s$, $\downarrow$)}} 
\\
Method & 
Robustif. & 
\# of Demonstr. & 
\multicolumn{2}{c|}{$\mathcal{S}$} & 
\multicolumn{2}{c|}{$\mathcal{T}$} & 
\multicolumn{2}{c|}{$\mathcal{S}$} & 
\multicolumn{2}{c|}{$\mathcal{T}$} &
\multicolumn{2}{c|}{$-$} 
\\
&       & {} &    mean & std & mean & std &  mean &   std &  mean &   std &        mean & std \\
\hline
\hline
DAgger & DR    & 50  &      \grdr{92} &  27 &   \grdr{82} &  39 &       \grdp{9094} & 20608 &  \grdp{3096} &  5497 & \grdt{392} &  34 \\
       &       & 100 &     100 &   0 &   \grdr{97} &  17 &        \grdp{634} &   711 &  \grdp{1277} &  4947 & \grdt{810} & 104 \\
       &       & 200 &     100 &   0 &   \grdr{99} &  10 &         \grdp{91} &    73 &   \grdp{274} &  1247 & \grdt{1970} & 154 \\
\hline
BC     & SA-sparse (18)&1& 100 &   0 &  100 &   0 &        \grdp{553} &   462 &   \grdp{211} &   228 &          \grdt{84} &   9 \\
       &       & 2   &     100 &   0 &   \grdr{97} &  17 &         \grdp{41} &    39 &   \grdp{371} &  1808 &          \grdt{88} &  10 \\
       &       & 10  &     100 &   0 &   \grdr{97} &  17 &         \grdp{33} &    42 &   \grdp{226} &   815 &         \grdt{115} &  10 \\
       & SA-25 & 1   &     100 &   0 &  100 &   0 &        \grdp{956} &   402 &   \grdp{270} &   384 &          \grdt{87} &  15 \\
       &       & 2   &     100 &   0 &  100 &   0 &        \grdp{148} &   140 &   \grdp{107} &   150 &          \grdt{90} &  14 \\
       &       & 10  &     100 &   0 &  100 &   0 &        \grdp{107} &   175 &    \grdp{90} &    83 &         \grdt{117} &  14 \\

       & SA-50 & 1   &     100 &   0 &  100 &   0 &        \grdp{421} &   193 &   \grdp{105} &   116 &         \grdt{204} &  32 \\
       &       & 2   &     100 &   0 &  100 &   0 &         \grdp{76} &    31 &   \grdp{66}&    56 &         \grdt{207} &  31 \\
       &       & 10  &     100 &   0 &  100 &   0 &         \grdp{55} &    28 &   \grdp{76}&   102 &         \grdt{235} &  31 \\

       & SA-100& 1   &     100 &   0 &  100 &   0 &        \grdp{291} &   154 &    \grdp{89} &   105 &         \grdt{339} &  72 \\
       &       & 2   &     100 &   0 &  100 &   0 &         \grdp{57} &    20 &    \grdp{76} &    85 &         \grdt{342} &  72 \\
       &       & 10  &     100 &   0 &  100 &   0 &         \grdp{33} &    21 &    \grdp{97} &   118 &         \grdt{369} &  72 \\
\hline
DAgger & SA-sparse (18)&1& 100 &   0 &  100 &   0 &        \grdp{747} &   705 &   \grdp{319} &   879 &          \grdt{85} &   6 \\
       &       & 2   &     100 &   0 &  100 &   0 &        \grdp{222} &   122 &   \grdp{142} &   219 &          \grdt{89} &   6 \\
       &       & 10  &     100 &   0 &  100 &   0 &        \grdp{29}  &    24 &   \grdp{114} &   150 &         \grdt{117} &   6 \\

       & SA-25 & 1   &     100 &   0 &  100 &   0 &        \grdp{579} &   224 &   \grdp{160} &   168 &          \grdt{92} &  12 \\
       &       & 2   &     100 &   0 &  100 &   0 &        \grdp{366} &   279 &   \grdp{122} &   182 &          \grdt{96} &  12 \\
       &       & 10  &     100 &   0 &  100 &   0 &        \grdp{110} &   115 &   \grdp{100} &   120 &         \grdt{124} &  12 \\

       & SA-50 & 1   &     100 &   0 &  100 &   0 &        \grdp{361} &   161 &    \grdp{78} &    91 &         \grdt{206} &  28 \\
       &       & 2   &    100 &   0 &  100 &   0 &         \grdp{169} &   117 &    \grdp{77} &    80 &         \grdt{210} &  28 \\
       &       & 10  &     100 &   0 &  100 &   0 &         \grdp{56} &    77 &    \grdp{82} &   107 &         \grdt{237} &  28 \\

       & SA-100 & 1   &     100 &   0 &  100 &  0 &        \grdp{309} &   133 &    \grdp{92} &   105 &         \grdt{342} &  29 \\
       &       & 2   &     100 &   0 &  100 &   0 &        \grdp{100} &    61 &    \grdp{77} &    96 &         \grdt{346} &  29 \\
       &       & 10  &     100 &   0 &  100 &   0 &         \grdp{30} &    28 &    \grdp{90} &   109 &         \grdt{373} &  29 \\
\hline
\end{tabular}

    \caption{Performance, robustness and training time for SA-based methods after $1$, $2$, and $10$ demonstrations, compared with the best performing baselines, DAgger+DR, in the environment without disturbances ($\mathcal{S}$, source), and with ($\mathcal{T}$, target). 
    Robustness is color-coded from white ($100\%$) to red ($90\%$ or below). Performance and training time are color-coded from green (fast training time, small expert gap) to red (long training time, large expert gap). The results highlight that \ac{SA}-methods achieve high robustness and close to expert performance compared to DAgger+DR, even after a single demonstration, and their performance can be further improved via additional fine-tuning demonstrations. We note that \ac{DAgger} and \ac{BC}-based approaches differ at one demonstration due to non-determinism in the training procedure.}
    \label{tab:sa_analysis}
    \vspace{-5pt}
\end{table}

\subsection{Hardware Evaluation}
We experimentally evaluate the ability of the policy to perform a flip on a real multirotor, under real-world uncertainties such as model errors (e.g., inaccurate thrust to battery voltage mappings, aerodynamic coefficients, moments of inertia) and external disturbances (e.g., ground effect). The tested policy is obtained using DAgger+SA-$25$ trained after $2$ demonstrations (the first with \ac{DA}, the second for fine-tuning), as the method represents a good trade-off between performance, robustness and training time. As in \cref{sec:evaluation_linear}, we deploy the learned policy on an onboard \texttt{Nvidia Jetson TX2}, where it runs at $100$Hz. The maneuver includes a take-off/landing phase consisting of a $1$m ramp on $x$-$y$-$z$ in $\text{W}$ and overall has a total duration of $6$s. The maneuver is repeated $5$ times in a row, to demonstrate repeatability, recording successful execution of the maneuver and successful landing at the designated location in all the cases. \cref{fig:flip_timelapse} shows a time-lapse of the different phases of the maneuver (excluding the ramp from and to the landing location). The 3D position of the robot, as well as the direction of its thrust vector, are shown for two runs in \cref{fig:rtnmc_experiment_state_traj}, highlighting the large distance and altitude traveled in a short time. \cref{fig:rtnmc_experiment_actuation} additionally shows some critical parameters of the maneuvers, such as the attitude and the angular velocity, as well as thrust and the vertical velocities. It highlights that the robot rotates at up to $11$rad/s, and the overall $360^\circ$ rotation takes about $0.5$s. Overall, these results validate our numerical analysis and highlight the robustness and performance of a policy efficiently trained from $2$ demonstrations and about $100$s of training time. Our video submission \cite{video_submission} includes an additional experiment demonstrating near-minimum time navigation from one position to another, starting and ending with velocity close to zero, using a policy trained with two demonstrations (DAgger+SA-$25$).

\begin{figure}
    \centering
    \begin{subfigure}{.5\columnwidth}
      \centering
      \includegraphics[width=\columnwidth, trim = {6.0in, 1.5in, 18.0in, 1.5in}, clip]{figs/flip_3d/flips_v2.pdf}
    \end{subfigure}%
    \begin{subfigure}{.5\columnwidth}
      \centering
      \includegraphics[width=\columnwidth, trim = {19in, 1.5in, 5.0in, 1.5in}, clip]{figs/flip_3d/flips_v2.pdf}
    \end{subfigure}%
    \caption{Acrobatic flip trajectory performed by our UAV using a policy efficiently learned from a nonlinear Robust Tube MPC. All the units are expressed in meters. The red arrows denote the direction of the thrust vector on the quadrotor, and they highlight that the flip occurs at the point of highest altitude of the maneuver. The plot shows the additional take-off/landing phase, also performed by the learned policy, taking the robot back and forth the origin and $(x, y, z) = (1, 1, 1)$, that is the point where the flip starts and ends. All the units are expressed in (m). 
    } 
    \label{fig:rtnmc_experiment_state_traj}
    \vspace{-3ex}
\end{figure}

\begin{figure}
    \centering
    \includegraphics[width=\columnwidth, trim={0.4cm 0.5cm 0.5cm 0.5cm}, clip]{figs/flip_cmds/flip_cmds_v6.pdf}
    \caption{
    Control inputs generated by the learned policy and relevant states during the real--world acrobatic flip maneuver on a multirotor, where the robot is subject to large levels of uncertainties, caused primarily by model mismatches such as thrust-to-battery voltage mappings and hard-to-model aerodynamic effects. 
    Despite the large level of uncertainties, that require the usage of the maximum thrust allowed, the maneuver is completed successfully, and the robot reaches the desired vertical velocity of \SI{-1.0}{\m \per \s} at the end of the recovery phase (and that corresponds to the initial velocity of the landing phase). 
    We highlight that the flip is performed at an angular velocity of about \SI{11.0}rad/s. 
    The actual thrust $t_\text{cmd}$ can be related to the 
    normalized thrust $\bar{t}_\text{cmd}$ via  
    $t_\text{cmd} = m g (1 + \bar{t}_\text{cmd})$,
    where $ m g $ is the weight force of the robot.}
    \label{fig:rtnmc_experiment_actuation}
\end{figure}
\vspace{-7pt}

\section{Discussion and Conclusion} \label{sec:conclusion}

This work has presented an \ac{IL} strategy to efficiently train a robust \ac{DNN} policy from \ac{MPC}. Key ideas of our work were to 
\begin{inparaenum}[(a)]
\item leverage a Robust Tube variant of \ac{MPC}, called \ac{RTMPC}, to collect demonstrations using existing \ac{IL} methods (\ac{DAgger}, \ac{BC}), and
\item to augment the collected demonstrations with extra state-and-actions sampled from the robust control invariant set (tube) of the controller, leveraging the key intuition that the tube represents an approximation of the support of the state distribution that the learned policy will encounter at deployment, when subject to uncertainties.
\end{inparaenum}
We first demonstrate these properties by efficiently training a policy that enables robust trajectory tracking on a multirotor operating at near-hover conditions. This is accomplished by leveraging a linear model of the robot and a linear \ac{RTMPC} framework. Our numerical and experimental evaluations  highlight the possibility of training a policy robust to real-world uncertainties (such as wind and model errors) after a single demonstration, collected either in simulation or directly with the real robot.

While obtaining extra data in a computationally efficient way is straightforward in a linear \ac{RTMPC} setting, since actions can be computed efficiently using the linear ancillary controller, as shown in our conference paper \cite{tagliabue2022efficient}, the same efficiency can be challenging to achieve when leveraging nonlinear variants of \ac{RTMPC}. Therefore, in this journal extension of \cite{tagliabue2022efficient} we have presented a strategy to efficiently perform tube-guided \ac{DA} when leveraging nonlinear variants of \ac{RTMPC}. The key idea, in this case, was to perform \ac{DA} by generating a linear approximation of the ancillary \ac{NMPC} employed to maintain the system under uncertainties inside the tube. We have additionally presented a fine-tuning phase that requires very few demonstrations and can significantly reduce the errors introduced by the approximation of the ancillary \ac{NMPC}. Experimental evaluations on the challenging task of performing a flip on a multirotor have demonstrated the performance of policies trained with only two demonstrations, requiring less than one-third of the training time needed by existing methods. 

While design procedures to accurately estimate tubes given uncertainty and model priors may be complex and computationally expensive, our work has also demonstrated that fixed-size approximations of the tube are sufficient to efficiently generate policies that can withstand real-world uncertainties. Additionally, depending on the scenario, designing a tube and performing tube-guided \ac{DA} may be a simpler method to learn robust policies compared to strategies based on \ac{DR}, where a user needs to carefully randomize relevant environmental parameters. This is the case, for example, when learning tubes from data, i.e., when prior knowledge on the state deviations caused by uncertainties is known, but the exact sources of those uncertainties are unknown, and therefore it is hard to identify what are the parameters that should be randomized in the popular \ac{DR} procedures. 

Overall, our work has demonstrated that it is possible to efficiently generate high-quality, robust policies from \ac{MPC}, i.e. in a way that requires few 
\begin{inparaenum}[(a)]
    \item queries to expensive optimal controllers, and
    \item environment interactions.
\end{inparaenum}
This can have important repercussions across different fields of robotics. For example, the improvements in (a) enable speed-up in the training time of policies, a convenient property when performing controller tuning or when working on robots where the model is so poorly known that the model/controller/policy needs to be modified whenever new data (e.g., obtained from the previous deployment of the learned policy) becomes available. Additionally, the improvements in (b) can open up further opportunities to imitation-learn robust policies in environments that are extremely expensive to query, such as the real world (where expensive is intended in terms of effort to run data collection), or in scenarios where the simulated environments require a large amount of computation (i.e., in fluid-dynamic simulations). %
Last, shifting gear from the \ac{IL} setting, we would like to highlight that our proposed efficient policy learning procedure can serve as a boot-strapping methodology, leveraging \textit{model and uncertainty priors}, to \textit{efficiently} obtain \textit{high-performance}, \textit{robust} initial policy guesses for subsequent policy fine-tuning steps using model-free methods, such as \ac{RL}, with the potential to significantly reduce the amount of random exploration typically required by these methods.     






\section{Acknowledgments}
We thank Kota Kondo for the help with the experimental evaluation, Prof.\ Michael Everett,  Dr.\ Dong-Ki Kim, Tong Zhao and Dr.\ Donggun Lee for feedback and discussions.

\bibliographystyle{IEEEtran} %
\bibliography{ref}
\end{document}
