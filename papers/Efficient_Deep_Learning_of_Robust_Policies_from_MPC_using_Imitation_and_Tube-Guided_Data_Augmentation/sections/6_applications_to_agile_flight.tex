In this Section, we tailor the proposed efficient policy learning strategies to agile flight tasks, as this will be the focus of our numerical and experimental evaluation. First, in \cref{subsec:mav_model}, we present the nonlinear model of the multirotor used to collect demonstrations in simulation in all our approaches, and used for control design. Then, in \cref{subsec:linear_mpc}, we present a \ac{RTMPC} expert for \textit{trajectory tracking} based on a \textit{linear} multirotor model and that will be used with the \ac{IL} procedure described in \cref{sec:rtmpc_linear}. Because the considered trajectories require the robot to operate around a fixed, pre-defined condition (near hover), a hover-linearized model is suitable for the design of this controller. Last, in \cref{subsec:nonlinar_mpc}, we design a nonlinear \ac{RTMPC} expert capable of performing a $360^\circ$ flip in \textit{near-minimum time} - a maneuver that demands exploitation of the full \textit{nonlinear} dynamics of the multirotor, and that requires large and careful actuation usage. This controller will be used with the \ac{IL} procedure described in \cref{sec:rtmpc_nonlinear}. 

\subsection{Nonlinear Multirotor Model} \label{subsec:mav_model}
We consider an inertial reference frame $\text{W}$ attached to the ground, and a non-inertial frame $\text{B}$ attached to the \ac{CoM} of the robot. The translational and rotational dynamics of the multirotor are:  
\begin{subequations} \label{eq:mav_model_full}
\begin{align}
    \vbsd[W]{p} & = \vbs[W]{v} \label{eq:mav_model_full:tr_kin} \\
    \vbsd[W]{v} & = m^{-1}(\vbs{R}_\text{WB} \vbs[B]{t}_\text{cmd} + \vbs[W]{f}_\text{drag} + \vbs[W]{f}_\text{ext}) - \vbs[W]{g} \label{eq:mav_model_full:tr_dyn} \\
    \vbsd{q}_\text{WB} & = \frac{1}{2} \vbs{\Omega} (\vbs[B]{\omega}) \vbs{q}_\text{WB}  \label{eq:mav_model_full:rot_kin} \\
    \vbsd[B]{\omega} &  = \vbs{I}_\text{mav}^{-1}(- \vbs[B]{\omega} \times \vbs{I}_\text{mav} \vbs[B]{\omega} + \vbs[B]{\tau}_\text{cmd} + \vbs[B]{\tau}_\text{drag}) \label{eq:mav_model_full:rot_dyn}
\end{align}
\end{subequations}
where $\boldsymbol{p}$, $\boldsymbol{v}$, $\boldsymbol{q}$, $\boldsymbol{\omega}$ are, respectively, position, velocity, attitude quaternion and angular velocity of the robot, with the prescript denoting the corresponding reference frame. The attitude quaternion $\boldsymbol{q} = [q_w, \boldsymbol{q}_v^\top]^\top$ consists of a scalar part $q_w$ and a vector part $\boldsymbol{q}_v = [q_x, q_y, q_z]^\top$ and it is unit-normalized; the associated $3 \times 3$ rotation matrix is $\boldsymbol{R} = \vbs{R}(\vbs{q})$, while
\begin{equation}
\vbs{\Omega} (\boldsymbol{\omega}) = 
\begin{bmatrix}
0 & -\boldsymbol{\omega}^\top \\
\boldsymbol{\omega} & \lfloor \boldsymbol{\omega} \rfloor_\times \\
\end{bmatrix},
\end{equation} with $\lfloor \boldsymbol{\omega} \rfloor_\times$ denoting the $3 \times 3$ skew symmetric matrix of $\boldsymbol{\omega}$. $m$ denotes the mass, $\vbs{I}_\text{mav}$ the $3 \times 3$ diagonal inertial matrix, and $\boldsymbol{g} = [0, 0, g]^\top$ the gravity vector. Aerodynamic effects are taken into account via $\boldsymbol{f}_\text{drag} = - c_{D,1} \boldsymbol{v} - c_{D,2} \|\boldsymbol{v}\| \boldsymbol{v}$ and isotropic drag torque $\boldsymbol{\tau} = - c_{D,3} \boldsymbol{\omega}$, capturing the parasitic drag produced by the motion of the robot. The robot is additionally subject to external force disturbances $\boldsymbol{f}_\text{ext}$, such as the one caused by wind or by an unknown payload. 
Last, $\boldsymbol{t}_\text{cmd} = [0, 0, t_\text{cmd}]^\top$ is the commanded thrust force, and $\boldsymbol{\tau}_\text{cmd}$ the commanded torque. These commands can be mapped to the desired thrust $f_{\text{prop},i}$ for the $i$-th propeller ($i = 1, \dots, n_p$) via a linear mapping (\textit{allocation} matrix) $\boldsymbol{\mathcal{A}}$: 
\begin{equation}
\label{eq:allocation_matrix}
    \begin{bmatrix}
    t_\text{cmd} \\
    \vbs{\tau}_\text{cmd} \\ 
    \end{bmatrix}
    = \boldsymbol{\mathcal{A}}
    \begin{bmatrix}
    f_{\text{prop},1} \\
    \vdots \\
    f_{\text{prop}, n_p}
    \end{bmatrix} = \boldsymbol{\mathcal{A}} \vbs{f}_\text{prop}.
\end{equation}
The attitude of the quadrotor is controlled via the geometric attitude controller in \cite{lee2011geometric} that generates desired torque commands $\vbs[B]{\tau}_\text{cmd}$ given a desired attitude $\vbs{R}_\text{WB}^\text{des}$, angular velocity $\vbs[B]{\omega}^\text{des}$ and acceleration $\vbsd[B]{\omega}^\text{des}$. The controllers designed in the next sections output setpoints for the attitude controller, and desired thrust $t_\text{cmd}$. 

  

\subsection{Linear \ac{RTMPC} for Trajectory Tracking} \label{subsec:linear_mpc}
The model employed by the linear \ac{RTMPC} for trajectory tracking (\cref{eq:rtmpc_optimization_problem}) is based on a simplified, hover-linearized model derived from \cref{eq:mav_reduced_model_for_nmpc}, using the approach in \cite{kamel2017linear}, but modified to account for uncertainties. First, similar to \cite{kamel2017linear}, we express the model in a yaw-fixed, gravity-aligned frame $\text{I}$ via the rotation matrix $\vbs{R}_\text{BI}$
\begin{equation}
    \begin{bmatrix}
    \phi \\
    \theta \\
    \end{bmatrix} =
    \vbs{R}_\text{BI}
    \begin{bmatrix}
    \prescript{}{I}{\phi} \\
    \prescript{}{I}{\theta} \\
    \end{bmatrix}, \hspace{1pt}
    \vbs{R}_\text{BI} =
    \begin{bmatrix}
    \cos(\psi) & \sin(\psi) \\
    -\sin(\psi) & \cos(\psi) \\
    \end{bmatrix},
\end{equation}
where the attitude has been represented, for interpretability, via the Euler angles yaw $\psi$, pitch $\theta$, roll $\phi$ (\textit{intrinsic} rotations around the $z$-$y$-$x$ such that $\vbs{R} = \vbs{R}_{z}(\psi)\vbs{R}_{y}(\theta) \vbs{R}_{x}(\phi)$, with $\vbs{R}_{l}(\alpha)$ being a rotation of $\alpha$ around the ${l}$-th axis). 
Second, as in \cite{kamel2017linear}, we assume that the closed-loop attitude dynamics can be described by a first-order dynamical system that can be identified from experiments, replacing \cref{eq:mav_model_full:rot_kin}, \cref{eq:mav_model_full:rot_dyn}. 
Last, different from \cite{kamel2017linear}, we assume $\vbs[W]{f}_\text{ext}$ in \cref{eq:mav_model_full:tr_dyn} to be an unknown disturbance/model errors that capture the uncertain parts of the model, such that $\vbs[W]{f}_\text{ext} \in \mathbb{W}$.

The controller generates tilt (roll, pitch) and thrust commands ($n_u = 3$) given the state of the robot ($n_x=8$) consisting of position, velocity, and tilt, and given the reference trajectory. 
The desired yaw is fixed (and it is tracked by the cascaded attitude controller); similarly, $\vbs[B]{\omega}^\text{des}$ and $\vbsd[B]{\omega}^\text{des}$ are set to zero.  We employ the nonlinear attitude compensation scheme in \cite{kamel2017linear}.

The controller takes into account position constraints (e.g., available 3D flight space), actuation limits, and velocity/tilt limits via $\mathbb{X}$ and $\mathbb{U}$. The cross-section of the tube $\mathbb{Z}$ is a constant outer approximation based on an axis-aligned bounding box. It is estimated via Monte-Carlo sampling, by measuring the state deviations of the closed loop linear system $\vbf{A}_K$ under the disturbances in $\mathbb{W}$.

\subsection{Nonlinear \ac{RTMPC} for Acrobatic Maneuvers} \label{subsec:nonlinar_mpc}

\noindent 
\textbf{Ancillary \ac{NMPC}.}
We start by designing the ancillary \ac{NMPC} (\cref{eq:ancillary_nmpc_eq}). The selected nominal model is the same used in the high-performance trajectory tracking \ac{NMPC} for multirotors~\cite{loquercio2019deep, mueller2013computationally}: 
\begin{equation}
\label{eq:mav_reduced_model_for_nmpc}
\begin{split}
    \vbsd[W]{v} & = m^{-1}(\vbs{R}_\text{WB} \vbs[B]{t}_\text{cmd} + \vbs[W]{f}_\text{drag}) - \vbs[W]{g} \\
    \vbsd[W]{p} & = \vbs[W]{v} \\
    \vbsd{q}_\text{WB} & = \frac{1}{2} \vbs{\Omega} (\vbs[B]{\omega}_\text{cmd})\vbs{q}_\text{WB},
\end{split}
\end{equation}
where the rotational dynamics (\cref{eq:mav_model_full:rot_dyn}) have been neglected, assuming that the cascaded attitude controller enables fast tracking of the desired angular velocity setpoint $\vbs[B]{\omega}_\text{cmd}$.  
The controller uses the state and control input: 
\begin{equation}
\label{eq:ancillary_nmpc_state_and_control_input}
    \bar{\vbf{x}} = [\vbs[W]{p}^\top, \vbs[W]{v}^\top, \vbs{q}_\text{WB}^\top ]^\top, \;\; \bar{\vbf{u}} = [t_\text{cmd}, \vbs[B]{\omega}_\text{cmd}^\top]^\top.
\end{equation} 
The angular acceleration setpoint for the attitude controller $\vbsd[B]{\omega}_\text{cmd}$ is obtained via numerical differentiation, while we do not explicitly generate an attitude setpoint (e.g., we set $\vbs{R}^\text{des}_\text{WB} = \vbs{R}_\text{WB}$) obtaining an attitude controller that only tracks desired angular velocities/accelerations.

\noindent 
\textbf{Near-Minimum Time Safe Plan Generation.}
To compute safe nominal plans for acrobatic maneuvers (by solving the \ac{OCP} in \cref{eq:nmpc_nominal}), we employ an extended version of the full nonlinear dynamic model in \cref{subsec:mav_model}. More specifically, we solve the  \ac{OCP} in \cref{eq:nmpc_nominal} by using the following state $\tilde{\vbf{z}} \in \tilde{\bar{\mathbb{Z}}}$ and control inputs $\tilde{\vbf{v}} \in \tilde{\bar{\mathbb{V}}}$: %
\begin{equation}
\label{eq:agile_fligh:nominal_dynamics}
\tilde{\vbf{z}} =
[
    \vbs[W]{p}^\top, \vbs[W]{v}^\top, \vbs{q}_\text{WB}^\top, \vbs[B]{\omega}^\top, \vbs[B]{f}_\text{prop}^\top 
]^\top
\;\; \tilde{\vbf{v}} = \vbsd[B]{f}_\text{prop},
\end{equation}
where the state has been extended to include the thrust produced by each propeller $\vbs{f}^\text{prop}$ to ensure continuity in the reference thrust, accounting for the unmodeled actuators' dynamics. As for the linear case, uncertainties are modeled by $\vbs[W]{f}_\text{ext} \in \mathbb{W}$.
The cost function captures the near-minimum time objective: 
\begin{equation}
    \tilde{J}_\text{RTNMPC} = T_f + \alpha_1\!\vbs{v}^\top\!\!\vbs{v} + \alpha_2 {\vbs{f}_\text{prop}}^\top\!\!\vbs{f}_\text{prop} + \alpha_3\!\tilde{\vbf{v}}^\top\!\!\tilde{\vbf{v}}
\end{equation}
where $T_f$ is the total time of the maneuver, while the remaining terms act as a regularizer for the optimizer, with $\alpha_i \ll T_f$ (i.e., $\alpha_i \approx 10^{-2}, ~\forall ~i$). 

We note that $\tilde{J}_\text{RTNMPC}$ contains a non-quadratic term, therefore differing from the quadratic cost employed in the safe nominal planner in \cite{mayne2011tube} (our \cref{eq:nmpc_nominal}); such cost function was chosen to automate the selection of the prediction horizon $N$ for the safe nominal plan. Our evaluation will demonstrate that the ancillary \ac{NMPC} maintains the system within a tube from the generated reference, further highlighting the flexibility of the framework. 

Additionally, we note that state and control input (\cref{eq:agile_fligh:nominal_dynamics}) have been extended compared to the ones (\cref{eq:ancillary_nmpc_state_and_control_input}) selected for the ancillary \ac{NMPC}, as emphasized by our notation $\tilde{\cdot}$. For this reason, the optimal safe nominal plan $\tilde{\vbf{Z}}_t^*$, $\tilde{\vbf{V}}_t^*$ found using the extended state needs to be mapped to the reference trajectory for the ancillary \ac{NMPC}, $\vbf{Z}_t^*$ $\vbf{V}_t^*$. This is done by simply selecting position, velocity and attitude from $\tilde{\vbf{Z}}_t^*$ to obtain $\vbf{Z}_t^*$. The thrust setpoint $t_\text{cmd}$ in $\vbf{V}_t^*$ is computed via $\vbs{\mathcal{A}}$ in \cref{eq:allocation_matrix} from $\vbs{f}_\text{prop}$ in $\tilde{\vbf{Z}}_t^*$, while the angular velocity setpoint $\vbs{\omega}_\text{cmd}$ is obtained by assuming it equal to the angular velocity $\vbs{\omega}$ in $\tilde{\vbf{Z}}_t^*$.

\noindent 
\textbf{Constraints.} The state constraint $\bar{\vbf{x}}_t \in \mathbb{X}$ encodes the maximum safe linear velocity $\vbs{v}$ and position boundaries $\vbs{p}$ of the environment, while actuation constraints $\bar{\vbf{u}}_t \in \mathbb{U}$ account for physical limits of the robot, restricting the nominal angular velocities $\vbs{\omega}_\text{cmd}$ (to prevent saturation of the onboard gyroscope), and the maximum/minimum thrust force $t_\text{cmd}$ produced by the propellers.
We impose tightened constraints on the thrust force by constraining $\vbs{f}_\text{prop}$  in $\tilde{\vbf{z}} \in \tilde{\bar{\mathbb{Z}}}$. These constraints are obtained via a conservative approach, i.e. we require a minimal thrust to generate a trajectory feasible within our position and velocity constraints. 
This was done to ensure that sufficient control authority is left to the ancillary \ac{NMPC} to account for the presence of large unknown aerodynamic effects and mismatches in the mapping from commanded thrust/actual thrust. This cautious approach enabled successful real-world execution of the maneuver without further real-world tuning. We additionally leverage the further degrees of freedom introduced by the extended state $\tilde{\vbf{z}}$ by shaping the safe plan through upper-bounding the thrust rates $\dot{\vbs{f}}_\text{prop}$ via $\tilde{\bar{\mathbb{V}}}$, although this constraint will not be enforced by the ancillary \ac{NMPC}. Last, using Monte-Carlo closed-loop simulations with disturbances sampled from $\mathbb{W}$, we verify that $\mathbb{X}$ and $\mathbb{U}$ are satisfied, and we generate a constant estimate (outer approximation, axis-aligned bounding box) of the cross-section of the tubes $\mathbb{T}^\text{state}$ and $\mathbb{T}^\text{action}$.




\noindent 
\textbf{Tube and Data Augmentation with Attitude Quaternions.}
The normalized attitude quaternion, part of the states $\bar{\vbf{x}}$, $\tilde{\vbf{z}}$ of nonlinear \ac{RTMPC}, and part of the reference $\vbf{Z}_t^*$ for the ancillary \ac{NMPC}, does not belong to a vector space, and therefore it is not trivial to describe its tube nor to generate extra samples for \ac{DA}. In this work, we employ an attitude error representation $\vbs{\epsilon} \in \mathbb{R}^3$ based on the \ac{MRP} \cite{shuster1993survey} to generate a representation that can be treated as an element of a vector space. Specifically, we use
\begin{equation}
\vbs{\epsilon}_t = \text{MRP}(\vbs{q}_t \odot {\vbs{q}^*_t}^{-1}),
\end{equation}
where $\vbs{q}_t$ is the current attitude, ${\vbs{q}_t^*}$ is the desired attitude (from the safe plan $\vbf{z}_t^*$), $\text{MRP}(\cdot)$ maps a quaternion to the corresponding three-dimensional attitude representation, while $\odot$ denotes the quaternion product.




