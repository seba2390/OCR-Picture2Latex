%chapter {comment}
% ****** Start of file apssamp.tex ******
%
%   This file is part of the APS files in the REVTeX 4.1 distribution.
%   Version 4.1p of REVTeX, March 2010
%
%   Copyright (c) 2009, 2010 The American Physical Society.
%
%   See the REVTeX 4 README file for restrictions and more information.
%
% TeX'ing this file requires that you have AMS-LaTeX 2.0 installed
% as well as the rest of the prerequisites for REVTeX 4.1
%
% See the REVTeX 4 README file
% It also requires running BibTeX. The commands are as follows:
%
%  1)  latex apssamp.tex
%  2)  bibtex apssamp
%  3)  latex apssamp.tex
%  4)  latex apssamp.tex
%
%\documentclass[twocolumn,showpacs,preprintnumbers,amsmath,amssymb]{revtex4}

\documentclass[%
reprint,
superscriptaddress,
%groupedaddress,
%unsortedaddress,
%runinaddress,
%frontmatterverbose,
%preprint,
%showpacs,preprintnumbers,
%nofootinbib,
%nobibnotes,
%bibnotes,
amsmath,amssymb,
aps,
%prl,
%linenumbers,
%prb,
%rmp,
%prstab,
%prstper,
%floatfix,
]{revtex4-1}

\usepackage{color}
\usepackage{graphicx}% Include figure files
\usepackage{dcolumn}% Align table columns on decimal point
\usepackage{bm}% bold math
%\usepackage{hyperref}% add hypertext capabilities
%\usepackage[mathlines]{lineno}% Enable numbering of text and display math
%\linenumbers\relax % Commence numbering lines

\newcommand{\tr}{\operatorname{tr}}
%\usepackage[showframe,%Uncomment any one of the following lines to test
%%scale=0.7, marginratio={1:1, 2:3}, ignoreall,% default settings
%%text={7in,10in},centering,
%%margin=1.5in,
%%total={6.5in,8.75in}, top=1.2in, left=0.9in, includefoot,
%%height=10in,a5paper,hmargin={3cm,0.8in},
%]{geometry}

\begin {document}
%section {title}
%\preprint{APS/123-QED}

\title{Infinite ergodic theory for three heterogeneous stochastic models with application to 
 subrecoil laser cooling }% Force line breaks with \\
%\thanks{A footnote to the article title}%

\author{Takuma Akimoto}
\email{takuma@rs.tus.ac.jp}
\affiliation{%
  Department of Physics, Tokyo University of Science, Noda, Chiba 278-8510, Japan
}%

\author{Eli Barkai}
%\email{takuma@rs.tus.ac.jp}
\affiliation{%
  Department of Physics, Bar-Ilan University, Ramat-Gan 5290002, Israel
}%

\author{G\"unter Radons}
%\email{takuma@rs.tus.ac.jp}
\affiliation{%
  Institute of Physics, Chemnitz University of Technology, 09107 Chemnitz, Germany
}%


%\collaboration{MUSO Collaboration}%\noaffiliation

\date{\today}% It is always \today, today,
%  but any date may be explicitly specified

\begin{abstract}
%We investigate the accumulation process of the momentum of an atom towards zero in three stochastic models of 
%subrecoil-laser-cooled gases. 
We compare ergodic properties of the kinetic energy for three stochastic models of subrecoil-laser-cooled gases. One model is based on a heterogeneous random walk (HRW), another is an HRW with long-range jumps 
(the exponential model), and the other is a mean-field-like approximation of the exponential model (the deterministic model).
%We derive the master equation for the heterogeneous random walk model, 
All the models show an accumulation of the momentum at zero in the long-time limit, and 
a formal steady state cannot be normalized, i.e., there exists an infinite invariant density. 
We obtain the exact form of the infinite invariant density and the scaling function for the exponential and deterministic models
%Consequently, we find that the propagator of the exponential model is in surprisingly good agreement with that of the heterogeneous 
%random walk model when a uniform distribution is used in the exponential model. %while there is a difference in the deterministic model. 
and devise a useful approximation for the momentum distribution in the HRW model. 
%We show that trajectory-to-trajectory fluctuations of time averages are universal when 
% the observable is integrable with respect to the infinite invariant density. On the other hand, when the observable is not 
% integrable, the fluctuations of the exponential and the HRW models are the same, but those of the 
% deterministic model are different. 
%We show that the infinite ergodic theory plays an important role in characterizing 
%fluctuations in the kinetic energy and the cooling efficiency of subrecoil-laser-cooled gases.  
%We discuss  a limit of a cooling efficiency.
While the models are kinetically non-identical, it is natural to wonder whether their ergodic properties share common traits, 
given that they are all described by an infinite invariant density. We show that the answer to this question depends on the type of observable under study.
If the observable is integrable, the ergodic properties such as the statistical behavior of the time averages are universal as they are described by the Darling-Kac theorem.
In contrast, for non-integrable observables, the models in general exhibit non-identical statistical laws. This implies that 
focusing on non-integrable observables, we discover non-universal features of the cooling process, that hopefully can lead to a better understanding of the particular model most suitable for a statistical description of the process. 
This result is expected to hold true for many other systems, beyond laser cooling.
%, which paves the way for constructing a fundamental theory of ergodic properties in non-stationary phenomena as well as dynamical systems. 
\end{abstract}

%\pacs{05.45.Ac, 05.40.Fb, 87.15.Vv}% PACS, the Physics and Astronomy
% Classification Scheme.
%\keywords{Suggested keywords}%Use showkeys class option if keyword
%display desired
\maketitle

%\tableofcontents

\section{Introduction}
%infinite invariant density/infinite ergodic theory
In many cases in equilibrium statistical physics, a steady-state solution of a master equation yields the 
equilibrium distribution. However, the formal steady-state solution may not be normalizable, 
 especially for non-stationary stochastic processes found in the context of anomalous diffusion and 
non-normalizable Boltzmann states \cite{van1992stochastic, Kessler2010,lutz2013, Rebenshtok2014, Holz2015, Leibovich2019,Aghion2019,aghion2020infinite,aghion2021moses,Streissnin2021}. 
 Such an unnormalized formal steady state is called an infinite invariant density, 
 which is known from deterministic dynamical systems \cite{Thaler1983,Aaronson1997}. 
Interestingly, dynamical systems with infinite invariant densities exhibit non-stationary behaviors and 
trajectory-to-trajectory fluctuations of time averages, 
whereas they are ergodic in the mathematical sense \cite{Aaronson1997}.
%; i.e.,  the measure of {\color{red}an invariant set, which is invariant under the transformation, } 
%is zero, or the measure of the complement is zero \cite{Aaronson1997}. 
%{\color{blue}Yes, the measure becomes infinite when the measure of the complement is zero and the measure is the infinite measure. These are the equivalent statement.}

The ergodic properties of dynamical systems with infinite invariant densities 
have been established in infinite ergodic theory \cite{Aaronson1997, inoue1997ratio, Thaler1998,Thaler2002, inoue2004ergodic, Akimoto2008,Akimoto2015,Sera2019,Sera2020}, where distributional limit theorems for time-averaged quantities play an important role. 
The distributional limit theorems state that time-averaged observables obtained with single trajectories show 
trajectory-to-trajectory fluctuations. The distribution function of the fluctuations depends on whether 
the observable is integrable with respect to the infinite invariant measure \cite{Aaronson1981,Akimoto2008,Akimoto2010,Akimoto2012,Akimoto2015}. This distributional behavior of time averages 
is a characteristic feature of infinite ergodic theory. Similar distributional behaviors have been 
 observed in experiments such as the fluorescence of quantum dots, diffusion in living cells, and interface fluctuations in liquid crystals 
 \cite{Brok2003,stefani2009,Golding2006,Weigel2011,Jeon2011,Hofling2013,Manzo2015,takeuchi2016}. 



%heterogeneous diffusion/subrecoil laser cooling
Subrecoil laser cooling is a powerful technique for cooling atoms \cite{cohen1990new, Bardou1994}. A key idea of this technique is to realize 
experimentally a heterogeneous random walk (HRW) of the atoms in momentum space. 
%For a homogeneous random walk a random walker never accumulates on any point. Therefore, 
In a standard cooling technique such as Doppler cooling, a biased random walk is utilized to shift the momenta of
 atoms towards zero \cite{cohen1990new}. Thus, Doppler cooling is routinely modeled using a standard 
 Fokker--Planck equation for the momentum distribution. 
In contrast to a homogeneous random walk, an HRW 
enables the accumulation of walkers at some point without an external force induced by the Doppler effect. 
In other words,  the probability of finding a random walker 
at that point converges to one in the long-time limit
 due to an ingenious trapping mechanism, that gives rise to a heterogeneous environment. Hence,  
for subrecoil laser cooling, 
instead of a biased random walk, an HRW 
plays an essential role. This was a paradigm shift for cooling and useful for cooling beyond the lowest 
limit obtained previously in standard cooling techniques \cite{cohen1990new}.  
%Because the process is an accumulation process {\color{red}toward} zero momentum, it is 
%intrinsically non-stationary. Therefore, we expect that an infinite invariant density 
%may exist and that the infinite ergodic theory will apply to time-averaged observables such as the kinetic energy 
%in subrecoil laser cooling. 
%One of the aims of this paper is to show the existence of the infinite invariant density in subrecoil laser cooling. 

%purpose
It has been recognized that infinite ergodic theory provides a fundamental theory for subrecoil-laser cooling \cite{Barkai2021,*barkai2022gas}. 
In \cite{Bardou2002} three models of subrecoil laser cooling are proposed. One is based on the 
HRW, another is obtained from the HRW model with long-range jumps
 called the exponential model, and the third is a mean-field-like approximation of the exponential model called 
the deterministic model. It is known that the infinite invariant density depends in principle on some details of the system \cite{Aghion2019,aghion2020infinite,aghion2021moses}. 
The question then remains: what elements of the infinite ergodic theory remain universal? 
 These questions with respect to the general validity of the theory are particularly important because we have at least two general classes of observables, i.e., integrable and non-integrable with respect to the infinite invariant measure. 
 To unravel the universal features of subrecoil laser cooling,  we explore here the three models of subrecoil laser cooling. %Comparison between obtained results help us elevate our understanding  of generic features of the theory, as we summarize in the end of the paper. }

\if0
Our study problem is to clarify the role of the infinite invariant density in subrecoil laser cooling. In our previous study \cite{Barkai2021,*barkai2022gas}, 
we partially provided the answer for a model of subrecoil laser cooling. 
 The role of infinite invariant densities in non-stationary processes is not  trivial, in contrast to equilibrium distributions, because 
  non-stationary processes have no steady state. Because an infinite invariant density is a ``formal steady state," 
  it plays a  role  in non-stationary processes. In dynamical systems with infinite invariant densities, 
  the infinite invariant densities are essential for obtaining a deep understanding of
  the distributional behaviors of time averages, dynamical instability of the system,
   and evolution of the density \cite{Akimoto2007, Korabel2009, Akimoto2010, Akimoto2010a, Korabel2012, Akimoto2012, Akimoto2013b,Korabel2013}. In previous studies \cite{Akimoto2020, Barkai2021,*barkai2022gas}, 
   we showed that the infinite invariant density in a stochastic process exhibiting 
 non-stationary behaviors is important to obtain ergodic properties, such as the trajectory-to-trajectory fluctuations 
 of time-averaged observables in the non-stationary process.
   The present study aims to provide an explicit role for the infinite invariant density for models of a subrecoil-laser-cooled gas. 
 The results indicate that the infinite ergodic theory 
 plays an important role in describing the fundamental {\color{red}theory} of subrecoil laser cooling. 
 \fi
 
 The rest of the paper is organized as follows. In Sec.~II, we introduce the three stochastic models of subrecoil laser cooling. 
 In Sec.~III, we introduce the master equation and the formal steady-state solution, i.e., the infinite invariant density, in the HRW model. 
 In Sections~IV and V, we present the infinite invariant densities and the distributional limit theorems for the time average of the kinetic energy in the deterministic and exponential model, respectively. While the master equations for the HRW and exponential model are different, we show that the propagators and the distributional behaviors of the time-averaged kinetic energy match very well. 
Section VI is devoted to the conclusion. In the Appendix, we give a derivation of the moments of the associated action as a function of 
time $t$.

\section{Three stochastic models}
   
   Here, we introduce the three stochastic models of subrecoil laser cooling. All the models 
   describe stochastic dynamics of the momentum of an atom. 
   
   First, the HRW model is a one-dimensional continuous-time random walk (CTRW)  in momentum space  $p$.  
   Here, we consider confinement, which is represented by reflecting boundary at $p=-p_{\max}$ and $p_{\max}$.
   The CTRW is a random walk with continuous waiting times. Usually, in the CTRW the waiting times are 
   independent and identically distributed (IID). In the HRW model, they are not  IID random variables. 
   In the HRW, 
   the waiting time between stochastic updates of momentum given $p$ is exponentially distributed with a mean waiting time $1/R(p)$. 
   After waiting the atom jolts and momentum is modified. 
   %The waiting time distribution is an exponential distribution with mean $1/R(p)$. %The jump rate $R(p)$ is not a constant near $p\cong 0$ and 
   We assume that the jump distribution $G( \Delta p)$ follows a Gaussian distribution:
   \begin{equation}
G( \Delta p)=(2\pi \sigma ^{2})^{-1/2}\exp [-\Delta p^{2}/(2\sigma ^{2})],  
\end{equation}
where $\Delta p$ is a jump of the 
momentum of an atom and $\sigma^2$ is the variance of the jumps.  
   The heterogeneous rate $R(p)$ is important to cool atoms and can be realized by velocity selective coherent population trapping in experiments \cite{AAK88}. In subrecoil laser cooling, the jump rate $R(p)$ is typically given by 
\begin{equation}
R(p)\propto |p|^\alpha 
\label{atom-laser interaction}
\end{equation}
for $|p|\to 0$ \cite{Bardou2002}, where $\alpha$ is a positive constant. 
This constant can take any value in principle \cite{KaC92}, for instance,
 $\alpha=2$ in velocity-selective coherent population trapping \cite{AAK88}. 
In what follows, we consider a specific jump rate: 
\begin{equation}
R(p)= \left\{
\begin{array}{ll}
c^{-1}|p|^\alpha  \quad &(|p| <p_0)\\
\\
c^{-1}|p_0|^\alpha &(|p| \geq p_0),
\end{array}
\right.
\label{atom-laser interaction2}
\end{equation}
where $p_0$ is the width of the jump rate dip and $c$ is a positive constant (see Fig.~\ref{traj}). 
At $p=\pm  p_{\max}$ we have reflecting boundary. 
A typical trajectory in the HRW model is shown in Fig.~\ref{traj}. 
   Since the HRW model is a non-biased random walk, the momentum will eventually reach high values.
To prevent such a situation, one considers a confinement in an experimentally realizable way. 

%%%%%%%%%%%%%%%%%%%%%%%%%%%%%%%%%%%%%%%%%%%%%%%%%%%%%%%%%%%%%%%%%%%%
%Figure 1
\begin{figure}
\includegraphics[width=.95\linewidth, angle=0]{traj-R_p_.eps}
\caption{A typical trajectory of momentum $p(t)$ in the HRW model, where $R(p)=|p|^2$, $p_0=1$, $p_{\max}=3$, and $\sigma^2=0.01$.
%$p_{\rm trap} \cong 0.035$ is shown for reference. 
%The blue and the yellow {\color{red}regions discriminated by $p_{\rm trap}$} are the trapping and the recycling region, respectively. 
The inset is a schematic illustration of the jump rate $R(p)$.}
\label{traj}
\end{figure}
%%%%%%%%%%%%%%%%%%%%%%%%%%%%%%%%%%%%%%%%%%%%%%%%%%%%%%%%%%%%%%%%%%%%

Next, we explain how we obtain the other two models, i.e., the exponential and the deterministic model, inspired by the HRW model. 
The region in momentum space can be divided into two regions, i.e., 
trapping and recycling regions \cite{Bardou2002}. 
The trapping region is defined as $|p| \leq p_{\rm trap}$, where we assume $p_{\rm trap} \ll \sigma$ and $p_{\rm trap} < p_0$. The assumption  
$p_{\rm trap} \ll \sigma$ is used in the uniform approximation stated below. 
%, the atom-laser interaction follows Eq.~(\ref{atom-laser interaction}).  
%Because the trap size $p_{\rm trap}$ can be arbitrarily small, the jump rate becomes $R(p)\cong 0$, 
%which makes the atom stay there for a long time. 
In the recycling region, the atom undergoes a non-biased random walk, which will eventually lead the atom back to 
the trapping region with the aid of the confinement. 
%In the HRW model, the trap size $p_{\rm trap}$ does not play any role. 
The jumps of a random walker are long-ranged in the trapping region in the sense that 
momentum after jumping in the trapping region is approximately independent of the previous momentum. 
Therefore, the following assumption is quite reasonable. In the exponential and the deterministic model, 
 momentum after jumping in the trapping region is assumed to be an IID random variable. 
 In particular, the probability density function (PDF) $\chi (p)$ for the momentum at every jump in the trapping region 
 is assumed to be uniform \cite{Saubamea1999,Bardou2002,bertin2008}: 
\begin{equation}
\chi  (p)= \frac{1}{2p_{\rm trap}} ~~{\rm for}~p\in[-p_{\rm trap},p_{\rm trap}].
%\int_{0}^{+\infty } {\color{red}\phi}  (v,\tau )\;d\tau =\left\langle \delta \left( v-v_{i}\right) \right\rangle ,  
\label{chi}
\end{equation}
%where $p_{\rm trap}$ is a parameter which is smaller than $p_0$ and $\sigma$ (see Fig.~\ref{traj}). 
%As in the HRW, momentum of an atom in these two models remains constant during the waiting time.
A trajectory for the exponential model is similar to that for the HRW model.
%in the sense that the momentum remains constant during the waiting time.   
%as shown in Fig.~\ref{traj},  
However, a crucial difference between the HRW model and the exponential model is in the nature of the waiting time: the waiting time is an independent random variable in the exponential model, whereas it is not in the HRW model.  
In the HRW model, momentum performs a random walk. When momentum changes due to photon scattering, the renewed momentum depends on the previous
 momentum. Hence in this sense we have a correlation of momentum that spans several jolting events. 
 On the other hand, the renewed momentum is independent of the previous momentum in the exponential model. 
 In both models, the waiting time given $p$ is an exponentially distributed random variable with rate $R(p)$. 
 Thus, the statistics of the waiting times in the two models is different, because in the HRW model they are correlated through the momentum sequence, 
 whereas in the exponential model they are not.
 %Thus, the dependences of the waiting times in the two models are different(?). 
 %There are two time scales in the dynamics of the exponential model. One is a time scale of jolting of momentum due to photon scattering. 
 However, in the exponential model, the momentum is always in the trapping region. In the HRW model, it jumps in the recycling region. In other words, 
 a time of returning to the trapping region is not taken into consideration in the exponential model. 
%Such a dependence of waiting times is also the nature of the quenched trap model (QTM) \cite{bouchaud90}, which is a random walk in a static 
%random heterogeneous environment \cite{bouchaud90}. Note that 
%the {\color{red}heterogeneous environment} in the HRW model is static but not random.

%Such a difference in the waiting time is also observed in  
%the quenched trap model (QTM) and CTRW, where waiting times are IID in the CTRW 
%but are not independent random variables in the QTM . 

 
A difference between the exponential and the deterministic model is in the coupling between the waiting time and the momentum. 
In the exponential model, momentum and waiting time are stochastically coupled. 
As for the HRW this model is a Markov model and the conditional PDF of the waiting time given the momentum $p$ follows an exponential 
distribution with mean $1/R(p)$. 
On the other hand, the deterministic model is a non-Markov model.  
The waiting time given the momentum $p$ is deterministically prescribed as $\tau (p) = 1/R(p)$ \cite{Bardou2002}. In other words, the waiting time, which is 
a random variable in the exponential model, is replaced by its mean in the deterministic model. In this sense, the deterministic 
model is a mean-field-like model of the exponential model.
Note that this implies a double meaning of $1/R(p)$: while in the HRW and in the exponential model 
it is the mean waiting time, whereas in the deterministic model it is the exact waiting time for a given momentum $p$.





\section{Heterogeneous Random Walk Model}



Here, we consider the HRW model confined to the interval $[-p_{\max},p_{\max}]$ \cite{AAK88,Bardou1994}. 
The momentum $p(t)$ at time $t$ undergoes a non-biased random walk. 
 Jumps of the momentum are attributed to photon scattering and spontaneous emissions.  
 Importantly, its jump rate $R(p)$ follows Eq.~(\ref{atom-laser interaction}) for $|p|<p_0$ \cite{Bardou1994}. 
 %In particular, we consider that the jump rate  follows Eq.~(\ref{atom-laser interaction2}) \cite{Bardou1994}. 
%In this random walk model, the momentum of an atom remains constant during the waiting time (see Fig.~\ref{traj}). 
% like the continuous-time random walk or the quenched trap model \cite{metzler00, bouchaud90} 
%In the same manner as the quenched trap model,  
In this model, the conditional PDF $q(\tilde{\tau}|p)$ of $\tilde{\tau}$ given $p$ follows the exponential distribution:
\begin{equation}
q(\tilde{\tau}|p)= R(p) \exp(-R(p) \tilde{\tau}).
\label{conditional_PDF_HRW}
\end{equation}
Clearly, the mean waiting time given $p$ explicitly depends on $p$ when $|p| <p_0$.
Thus, the random walk is heterogeneous. 
A confinement of atoms can also be achieved by Doppler cooling \cite{cohen1990new, Bardou1994}. 
However,  for simplicity, we consider reflecting boundary conditions at $p=\pm p_{\max}$. 
As will be observed later, the size of the confinement or the width of the jump rate dip 
does not affect the asymptotic behavior of the scaling function of the propagator. 
More precisely, the scaling function and fluctuations of the time-averaged energy 
do not depend on $p_{\max}$ and $p_0$. %, i.e., no $p_{\max}$ dependence. 
%We assume that the distribution of jumps of a random walker follows a Gaussian distribution $G(\Delta p)$, i.e., 
%\begin{equation}
%G( \Delta p)=(2\pi \sigma ^{2})^{-1/2}\exp [-\Delta p^{2}/(2\sigma ^{2})],  
%\end{equation}
%where $\Delta p$ is a jump of the momentum of an atom and $\sigma^2$ is the variance of the jump distribution.  
As shown in Fig.~\ref{traj}, the momentum of an atom remains 
constant for a long time when $|p|$ is small. On the other hand, 
momentum changes frequently occur when $|p|$ is away from zero. 


%There is a fundamental confusion resulting from the double use of p': p' is a coordinate and not the state value after a jump. But in the text the change of the state value is identified with p'-p, but the latter is just the difference between 2 coordinates.  In the exact determination of G~(p'|p) (see below) one has (infinitely) many different ways (due to reflections) to reach a given p' from a given p in one step. it seems that you always consider delta p to be Gaussian distributed. Only for p' inside the boundaries delta p has the meaning of the momentum change p'-p. If a boundary is involved, p'-p is no longer equal to delta p
%obviously delta p is drawn from the Gaussian, but then p+delta p can also be larger than 2*p_max. What rule is applied in such a case?

%If I understand this rule correctly, you want to jump from the mirror point p_s=2p_max-p (p reflected at p_max) in a Gaussian way (if the condition for reflection is fulfilled), but this may bring you to even larger values than p_s, further away from the boundary.

%In any case, the rule you wrote down does not capture the reflection properly (e.g. for arbitrarily large positive or negative outcomes generated by drawing from the Gaussian). Multiple reflections may occur with the drawing of one large enough Gaussian variable.

%There is an exact formula for G~(p'|p) as an infinite sum of shifted and mirrored Gaussians. It is given by
%$G(p^{\prime }|p)=\sum_{n=-\infty }^{\infty }G[(-1)^{n}p^{\prime}-p-2np_{\max }]$
%where G[x] is the Gaussian function. You gave only one of these terms.

%The exact expression is symmetric in p and p': G~(p'|p)=G~(p|p')

%Therefore (11) and (12) follows immediately and without any approximation.

\subsection{Master equation and infinite invariant density}
The HRW model is a Markov model. In general, the time evolution of the propagator of a Markov model can be described by a master equation \cite{van1992stochastic}. 
The time evolution of the probability density function (PDF) $\rho(p,t)$ of  momentum $p$ at time $t$ 
is given by the master equation with gain and loss terms:
\begin{equation}
\frac{\partial \rho \left( p,t\right) }{\partial t}=\int_{-p_{\max}}^{p_{\max}} dp^{\prime }\left[ W(p^{\prime}\rightarrow p)
\rho \left( p^{\prime },t\right) -W(p\rightarrow p^{\prime})\rho \left( p,t\right) \right],  
\label{Master}
\end{equation}
where $W(p\rightarrow p^{\prime})$ is the transition rate from $p$ to $p'$. As will be shown later, the formal steady-state solution for 
the master equation may not provide a PDF but a non-normalized density, i.e. an infinite invariant density.
Jump and transition rates can be represented as 
\begin{equation}
R(p)=\int_{-\infty}^\infty dp^{\prime }W(p\rightarrow p^{\prime })
\label{jump rate}
\end{equation}%
and
\begin{equation}
W(p\rightarrow p^{\prime })= R(p) \tilde{G}(p'|p), 
\label{transition rate0}
\end{equation}
respectively, where $ \tilde{G}(p'|p)$ is the conditional PDF of $p'$ given $p$, 
 where both the domain and the codomain of the function $ \tilde{G}(p'|p)$ are  $[-p_{\max}, p_{\max}]$ because of the confinement.  
The function $ \tilde{G}(p'|p)$ is equivalent to $G(p'-p)$ when $p+\Delta p$ does not exceed the boundary, i.e., $|p+\Delta p| < p_{\max}$, where $\Delta p$ is 
a momentum jump following the Gaussian distribution.
On the other hand, $ \tilde{G}(p'|p)$ cannot depend solely on the 
difference $p'-p$ when a random walker reaches the reflecting boundary, i.e., $|p+\Delta p| > p_{\max}$.   
In particular, we have 
\begin{equation}
\tilde{G}(p^{\prime }|p)= \sum_{n=-\infty}^\infty G(2np_{\max } -p  + (-1)^{n}p^{\prime}).
\end{equation}
Because $G(x)$ is a symmetric function (Gaussian distribution), $\tilde{G}(p^{\prime }|p)$ is symmetric in $p$ and $p'$: $\tilde{G}(p^{\prime }|p)=\tilde{G}(p|p^{\prime })$.
It follows that the master equation  (Eq.~(\ref{Master})) of the HRW model takes the following form:%
\begin{equation}
\frac{\partial \rho \left( p,t\right) }{\partial t}=-R(p)\rho \left( p,t\right) +\int_{-p_{\max}}^{p_{\max}}
dp^{\prime }\rho \left( p^{\prime },t\right) R(p^{\prime }) \tilde{G}(p|p^{\prime }).
\label{Master1}
\end{equation} 



\if0
For $|p|$ and $\sigma \ll p_{\max}$,  a random walker rarely comes close to the boundary, 
and the master equation can be approximated by
\begin{equation}
\frac{\partial \rho \left( p,t\right) }{\partial t}\simeq -R(p)\rho \left( p,t\right) +\int_{-\infty}^{\infty}
dp^{\prime }\rho \left( p^{\prime },t\right) R(p^{\prime }) G(p-p^{\prime }).
\label{Master2}
\end{equation}
As will be shown, a situation in which almost all walkers obey $|p| \ll p_{\max}$ is valid in the long-time limit. 
Because the jump distribution $G(\Delta p)$ is symmetric, i.e., $G(\Delta p)=G(-\Delta p)$, the transition rate 
satisfies $W(p\to p')/R(p) = W(p'\to p)/R(p')$, especially for $|p|$ and $|p'| \ll p_{\max}$. 
\fi

The stationary solution $\rho^*(p)$ is easily obtained from the detailed balance in Eq.~(\ref{Master}), i.e., 
\begin{equation}
 W(p^{\prime }\rightarrow p)\rho ^{\ast }\left( p^{\prime }\right)
-W(p\rightarrow p^{\prime })\rho ^{\ast }\left( p\right)  =0,
\label{detailed balance}
\end{equation}%
where $\rho^*(p)$ is the stationary solution. 
%For $|p|$ and $|p'| \ll p_{\max}$,  the conditional PDF $\tilde{G}(p|p')$ is approximately symmetric, i.e., $\tilde{G}(p|p')=\tilde{G}(p'|p)$. 
%Therefore, for $|p|, |p'| \ll p_{\max}$, detailed balance yields
As shown before, the conditional PDF $\tilde{G}(p|p')$ is symmetric, i.e., $\tilde{G}(p|p')=\tilde{G}(p'|p)$. Therefore, 
detailed balance yields 
\begin{equation}
R(p^{\prime })\rho ^{\ast }\left( p^{\prime }\right) =R(p)\rho ^{\ast}\left( p\right) ,
\end{equation}
which is fulfilled only if $R(p)\rho ^{\ast }\left( p\right) $ is constant. 
In subrecoil laser cooling, the jump rate $R(p)$ becomes a power-law form near $p\cong 0$, i.e., Eq.~(\ref{atom-laser interaction}). 
 For example, the velocity selective coherent population trapping gives $\alpha=2$ 
\cite{AAK88}, and the Raman cooling experiments realize  $\alpha=2$ and 4 by 1D square pulses and the 
Blackman pulses, respectively \cite{RBB95}.
Therefore,  for $|p| \ll p_{\max}$, the steady-state distribution $\rho^* (p)$ is formally given by
\begin{equation}
\rho^* (p) ={\rm const.}/R(p) \propto |p|^{-\alpha}.
\label{steady-state}
\end{equation}
For $\alpha\geq 1$, it cannot be normalized because of the divergence at $p=0$, and $\rho^* (p)$ 
is therefore called an infinite invariant density. Although $\rho^* (p)$ is 
the formal steady state, a steady state in the conventional sense does not exist in the system with $\alpha\geq 1$. 
  As will be shown below, a part of the infinite invariant density can be observed in the 
  propagator especially for a large time. Moreover, it will be shown that 
  $t^{1-1/\alpha} \rho (p,t)$ converges to the infinite invariant density for $t\to\infty$.
  Therefore, the infinite invariant density is not a vague solution but plays an important role in reality.  
  
  \if0 %Eli のコメントを受けて消した
For $\alpha<1$, it can be normalized, and a steady state exists.
%\begin{equation}
%\rho^* (p) %\sim \dfrac{R(p)^{-1}}{\int_{-p_{\max}}^{p_{\max}} dp' R(p')^{-1}} 
%\propto |p|^{-\alpha}\quad (p\to 0).
%\rho^* (p) = \frac{1-\alpha}2p_0^{1-\alpha}} |p|^{-\alpha}.
%\label{steady-state-norm}
%\end{equation}
%which does not depend on $D_0$. 
While this steady-state density is unbounded at $p=0$, the variance of the momentum converges 
to a non-zero constant. Thus, for subrecoil laser cooling with $\alpha<1$,
the probability of finding a non-cooled state with $p^2 > \varepsilon$ is finite for any $0<$ $\varepsilon < p_{\max}^2$. 
%does not provide a nice cooling technique.
On the other hand, because the formal steady state cannot be normalized for $\alpha \geq 1$, 
 as will be shown later,  $\rho (p,t)$ accumulates at $p=0$ in the long-time limit. In other words, the probability 
 of finding a non-cooled state becomes zero in the long-time limit. 


\begin{widetext}
For small momenta $|p|\ll 1$, the master equation, Eq.~(\ref{Master1}), can be approximated as 
\begin{equation}
\frac{\partial \rho(p,t)}{\partial t} \simeq - \rho(p,t) R(p) +  \int_{-\infty}^\infty \rho(p-p',t) R(p-p') G(p')  dp',
\end{equation} 
where  an effect of the boundary was ignored. 
By Taylor expansions of $\rho(p-p',t)$ and $r(p-p') $ with respect to $p'$, we have 
\begin{equation}
\frac{\partial \rho(p,t)}{\partial t} \simeq
 D (p)  \frac{\partial^2 \rho(p,t)}{\partial p^2}
+  2 \frac{\partial \rho(p,t)}{\partial p} \frac{\partial D(p)}{\partial p}  + \rho(p,t) \frac{\partial^2 D(p)}{\partial p^2},
%+  \frac{3 \langle \delta p^2 \rangle}{2}  \frac{\partial^2 \rho(p,t)}{\partial p^2} \frac{\partial^2 D(p)}{\partial p^2}
+O(\sigma^4),
%&\simeq& \frac{\partial^2}{\partial p^2} D(p) \rho (p,t),
\end{equation}
where $D(p) = \sigma^2 R(p)/2$.  
For $\sigma^2 \ll 1$, the master equation yields the following heterogeneous diffusion equation:
\begin{equation}
\frac{\partial \rho(p,t)}{\partial t} \simeq \frac{\partial^2}{\partial p^2} D(p) \rho (p,t).
\label{hetero-diffusion-eq}
\end{equation}
While $\sigma^2$ is assumed to be small in the derivation of the diffusion equation, the master equation is valid for any $\sigma^2$. 
The steady state of the heterogeneous diffusion equation can be easily obtained as $\rho^* (p) ={\rm const.}/D(p)$, which is
consistent with Eq.~(\ref{steady-state}). Note that the dynamics describing the heterogeneous diffusion {\color{red}is} continuous, 
whereas the dynamics of the HRW model {\color{red}is} discontinuous because of instantaneous jumps in the HRW model.
\end{widetext}
\fi

Figure~\ref{prop-hrw} shows numerical simulations of the propagator in the HRW model. 
The propagator accumulates 
near zero, and $\rho (p,t)$ around $p\cong 0$ increases with time $t$. Moreover, a power-law form, i.e., $p^{-\alpha}$, 
of the formal steady state $\rho^*(p)$ is observed, especially when $t$ is large, except for $p\cong 0$ (see also Fig.~\ref{propagator-exp}). 
Since the infinite invariant density $\rho^*(p)$ cannot be normalized, the propagator never converges to $\rho^*(p)$.

%%%%%%%%%%%%%%%%%%%%%%%%%%%%%%%%%%%%%%%%%%%%%%%%%%%%%%%%%%%%%%%%%%%%
%Figure 2
\begin{figure}
\includegraphics[width=.95\linewidth, angle=0]{prop-hrw.eps}
\caption{Time evolution of the propagator in the HRW model ($p_0=p_{\max}=1$, $\sigma=1$, and $R(p)=|p|^2$). 
Symbols with lines are the numerical results of the HRW model by simulating trajectories of random walkers. 
The solid line represents a part of a steady-state solution,  $\rho^*(p) \propto |p|^{-\alpha}$, for reference. 
The dashed lines represent plateaus around $|p|= 0$, which shift up with time $t$. 
Initial momentum is chosen uniformly on $[-1,1]$.  The number of trajectories used in this and all subsequent simulation results is $10^6$. }
\label{prop-hrw}
\end{figure}
%%%%%%%%%%%%%%%%%%%%%%%%%%%%%%%%%%%%%%%%%%%%%%%%%%%%%%%%%%%%%%%%%%%%

\if0
\begin{eqnarray}
\rho(p \pm \Delta p,t) &\simeq&   \rho(p ,t) \pm \Delta p \frac{\partial \rho(p,t)}{\partial p} 
+ \frac{1}{2}(\Delta p)^2 \frac{\partial^2 \rho(p,t)}{\partial p^2} , \\
 r (p\pm \Delta p) &\simeq&  R(p) \pm \Delta p \frac{\partial R(p)}{\partial p} 
+ \frac{1}{2}(\Delta p)^2 \frac{\partial^2 R(p)}{\partial p^2} , 
\end{eqnarray}
and 
\begin{equation}
\int_{-1}^{1} n(p')dp'=1, \int_{-1}^{1} p' n(p')dp'=0, \int_{-1}^{1} p'^2n(p')dp'= \langle \delta p^2 \rangle, 
\int_{-1}^{1} p'^4 n(p')dp'= 3\langle \delta p^2 \rangle^2, 
\end{equation}
\fi


\if0
In subrecoil laser cooling, the jump rate $R(p)$ is given by $R(p)\propto |p|^\alpha$ for $|p|\ll 1$. Thus, we assume 
 $D(p) =D_0 |p|^\alpha$. The steady-state solution $\rho_{\rm ss}(p)$ can be obtained using 
 ${\displaystyle \frac{\partial}{\partial p} D(p) \rho_{\rm ss}(p)=0}$: 
\begin{equation}
\alpha p^{\alpha -1} \rho_{\rm ss}(p) + p^{\alpha} \frac{d \rho_{\rm ss}(p)}{dp}=0.
\end{equation}
Therefore, the steady-state distribution $\rho_{\rm ss}(p)$ is given by
\begin{equation}
\rho_{\rm ss}(p) \propto |p|^{-\alpha}.
\label{steady-state}
\end{equation}
For $\alpha\geq 1$, it cannot be normalized. Thus, it is an infinite invariant density. 
For $\alpha<1$, it can be normalized as 
\begin{equation}
\rho_{\rm ss}(p) = \frac{1-\alpha}{2p_0^{1-\alpha}} |p|^{-\alpha},
\label{steady-state-norm}
\end{equation}
which does not depend on $D_0$. While it is unbounded at $p=0$, the variance of the momentum converges 
to a non-zero constant in the long-time limit. Thus, subrecoil laser cooling with $\alpha<1$ is not a suitable cooling technique.
\fi



\if0
%\section{Heterogeneous random walk model with long-range jumps}
{\color{red}Here, we introduce other stochastic models of subrecoil laser cooling. 
In experiments, the region on the momentum space can be divided into two regions, i.e., 
trapping and recycling regions \cite{Bardou2002}. In the trapping region ($p\cong 0$), the atom-laser interaction follows  
Eq.~(\ref{atom-laser interaction}), which makes the atom stay there for a long time. 
In the recycling region, the atom-laser interaction $R(p)$ becomes constant, i.e., $R(p)=$ const, which means that 
the atom undergoes a homogeneous random walk in the recycling region. With the aid of the confinement, the atom 
will eventually step back to the trapping region. 

The jump of a random walker is long-range in the trap region in the sense that 
the trap size is much smaller than $\sigma$, which is always possible in experiments 
because the trap size can be in principle arbitrarily small.
 In the trap region, 
the momentum after a jump is approximately independent of the previous momentum. Therefore, 
 the momentum can be considered to be independent and identically distributed (IID) random variables 
at every jump. 
In what follows, we assume that the momentum at every jump is an IID random variable and consider two stochastic models.
One model is called the exponential model, where momentum and waiting time are stochastically coupled, and 
the other is called the deterministic model, where they are coupled deterministically \cite{Bardou2002}. 
Here, we assume that the momentum at every jump is drawn uniformly on the interval
 $[-p_0, p_0]$, i.e., uniform approximation \cite{Bardou2002}, where $p_0$ is the trap size. 
 Thus, the PDF $\chi (p)$ for the momentum of an elementary event becomes 
\begin{equation}
\chi  (p)= \frac{1}{2p_0} ~~{\rm for}~p\in[-p_0,p_0],
%\int_{0}^{+\infty } {\color{red}\phi}  (v,\tau )\;d\tau =\left\langle \delta \left( v-v_{i}\right) \right\rangle ,  
\label{chi}
\end{equation}
where we assume $p_0\ll \sigma$. 
\fi


%Exponential model
\section{Exponential model}
In this section, we give theoretical results for the exponential model, which were already shown in our previous study 
\cite{Barkai2021,*barkai2022gas}.
 %As in the HRW model, momentum remains constant during a waiting time in the exponential model. 
 %The theoretical approach is slightly different from the previous one. 
 Here, we consider the Laplace transform of the propagator and execute the inverse transform to obtain the infinite invariant 
 density and the scaling function. 
 The derivation of the scaling function is different from the previous study \cite{Barkai2021,*barkai2022gas}, where 
 the master equation is directly solved.

%Master equation, infinite invariant density, and scaling function
\subsection{Master equation, infinite invariant density, and scaling function}
In the exponential model, the jump distribution is independent of the previous momentum unlike the HRW model. 
Therefore, for the exponential model the conditional probability $\tilde{G}(p'|p)$ in Eq.~(\ref{transition rate0}) can be replaced by a $p$-independent function 
$\chi(p')$ leading to
%the transition rate of the exponential model can be represented by $R(p)$ and $\chi(p^{\prime })$ separably: 
 \begin{equation}
 W(p\rightarrow p^{\prime })= R(p)\chi (p^{\prime }).
\label{transition rate exp}
\end{equation}
%The master equation of the exponential model can be obtained in the same way as for  the HRW model because the dynamics of the exponential model is Markovian.
%In particular, the loss term is the same as that in the HRW model and the gain term is given by $R(p)$ and $\chi(p)$.}
Inserting this into Eq.~(\ref{Master}), the
 master equation of the exponential model becomes
\begin{equation}
\frac{\partial \rho \left( p,t\right) }{\partial t}= - R(p)\rho(p,t) + \frac{1}{2p_{\rm trap}} \int_{-p_{\rm trap}}^{p_{\rm trap}} R(p') \rho(p',t) dp',
\label{Master-exp}
\end{equation}
where we used Eq.~(\ref{chi}). As a result the second term, i.e., gain term, is different from that in the HRW model, Eq.~(\ref{Master1}). 
%To obtain an analytical form of the propagator in the HRW, we consider a uniform approximation instead of solving the 
%the master equation explicitly. 
%If the momentum accumulates at zero in the long-time limit, this approximation is valid because $G(p)$ is approximately 
%uniform on $[-p_0,p_0]$ for $p_0\ll 1$. 
%This approximation can be considered to be an annealed model in the sense that 
%the waiting time does not depend on the position (momentum). In fact,  
%the waiting times are IID random variables, although they are coupled with the momenta.  
%In what follows, we consider a specific jump rate, i.e.,  $R(p)=c^{-1}|p|^\alpha$ or equivalently $\tau(p) =c |p|^{-\alpha}$, 
%where $c$ is a positive constant and $\alpha$ is a constant, which can take any value in principle \cite{KaC92}, for instance,
% $\alpha=2$ in velocity-selective coherent population trapping \cite{AAK88}. 
%There are two typical couplings between the waiting time $\tilde{\tau}$ and the momentum $p$. One is a deterministic coupling that leads to 
%the deterministic model, and the other is a stochastic coupling resulting in the so-called exponential model. 
%In the deterministic model, the waiting time $\tilde{\tau}$ is deterministically given by $\tilde{\tau} = 1/R(p)$ when the momentum 
%is $p$. On the other hand, 
In the exponential model, the momentum remains constant until the next jump, and the conditional waiting time distribution given by momentum $p$  follows an exponential distribution with mean $1/R(p)$, which is the same as in the HRW model, i.e., Eq.~(\ref{conditional_PDF_HRW}) holds also here.  
 Because the conditional waiting time distribution depends on $p$, 
the joint PDF of momentum $p$ and waiting time $\tilde{\tau}$, 
\begin{equation}
\phi  (p,\tilde{\tau} )=\left\langle \delta \left( p-p_{i}\right) \delta \left( \tilde{\tau}-\tilde{\tau}_{i}\right) \right\rangle  
\label{jpdgen}
\end{equation}
 plays an important role, where $\delta \left(.\right) $ is the $\delta$ function, $\langle \cdot \rangle$ represents the ensemble average, 
 $i$ is the $i$-th emission  event ($i=1,2, ...$), $p_i$ is the $i$th momentum, and $\tilde{\tau}_{i}$ is the $i$th waiting time. It can be expressed by 
\begin{equation}
\phi  (p,\tilde{\tau} )= q(\tilde{\tau}|p) \chi (p) ,
\label{joint-pdf exp}
\end{equation} 
where $q(\tilde{\tau}|p)$ is the conditional PDF $q(\tilde{\tau}|p)$ of waiting time $\tilde{\tau}$ given $p$, Eq.(\ref{conditional_PDF_HRW}), 
and $\chi (p)$ is given by Eq.(\ref{chi})
% and follows an exponential distribution with mean $1/R(p)$, which is the same as that in the HRW, i.e., Eq.~(\ref{conditional_PDF_HRW}). %In both models, 
%The PDF of the waiting time follows a power law: 
%\begin{equation}
%\psi (\tilde{\tau}) \sim A \tilde{\tau}^{-1 -\gamma} \quad (\tilde{\tau}\to \infty),
%\end{equation} 
%where we use $\gamma\equiv1/\alpha$ and $A=\gamma p_0^{-1} c^\gamma$ and $A=\gamma p_0^{-1} c^\gamma \Gamma (1+\gamma)$ for the deterministic and exponential models, respectively. 
%More precisely, in the deterministic model, it is expressed as 
%\begin{equation}
%\psi (\tilde{\tau}) = \gamma p_0^{-1} c^\gamma \tilde{\tau}^{-1 -\gamma} \quad (\tilde{\tau}\geq c p_0^{-\gamma^{-1}}).
%\label{waiting-time-pdf-det}
%\end{equation} 
%This is because the probability of $\tilde{\tau}<x$ is given by 
%\begin{eqnarray}
%\Pr (\tilde{\tau} \leq x)&=& \Pr (c|p|^{-\alpha} \leq x) = \Pr (|p| \geq (x/c)^{-\frac{1}{\alpha}}) \nonumber\\
%&=&  1- p_0^{-1}(x/c)^{-\frac{1}{\alpha}} 
%\end{eqnarray}
%for $x \geq cp_0^{-\alpha}$. In the exponential model, 

The unconditioned PDF of the waiting time is given by 
\begin{eqnarray}
\psi (\tilde{\tau}) = \frac{1}{2p_{\rm trap}} \int_{-p_{\rm trap}}^{p_{\rm trap}} R(p) \exp (-R(p)\tilde{\tau})dp,
\end{eqnarray} 
which follows from averaging the joint PDF,
 over the uniform density $\chi(p)$. %As shown in Ref. [{\color{blue}tell me the reference}],
By a change of variables ($y=R(p)\tilde{\tau}$), we have
\begin{eqnarray}
\psi (\tilde{\tau}) &=& \frac{c^{\frac{1}{\alpha}}\tilde{\tau}^{-1-\frac{1}{\alpha}}}{\alpha p_{\rm trap}} \int_0^{\tilde{\tau} c^{-1} p_{\rm trap}^\alpha} 
y^{\frac{1}{\alpha}} \exp (-y)dy\\
&\sim& \frac{\gamma c^{\gamma}\Gamma (1+\gamma)}{ p_{\rm trap}} \tilde{\tau}^{-1-\gamma}\quad (\tilde{\tau}\to \infty),
\end{eqnarray} 
where $\gamma=1/\alpha$.
In what follows, we assume $\gamma\leq 1$, which implies that the mean waiting time 
diverges. Therefore, as will be shown, the dynamics of $p$ becomes non-stationary.  
%While we consider a deterministic coupling; i.e, waiting time $\tau$ is not a random variable but a constant depending on $p$, 
%the power-law distribution is valid even when the coupling is stochastic. This is the same situation as in the quenched trap model. 
%\cite{bouchaud90,Akimoto2018}. 



%\subsection{Scaling function and infinite invariant density}


The exponential model is a continuous-time Markov chain, which is a special type of semi-Markov process (SMP). Therefore, 
we utilize an SMP with continuous variables to obtain analytical results for the exponential model.
In an SMP, the state value is determined by the waiting time, which is randomly 
selected, or equivalently, the waiting time is determined by the state value, which is randomly chosen. 
In the latter case, the state value is renewed according to the PDF $\chi(p)$.   
%In the HRW model, the method of determination follows the latter, and the state value (momentum) is not an IID random variable. 
%On the other hand, in the exponential model, the state values or the waiting times are IID random variables. 
In general, an SMP is characterized by the state distribution $\chi(p)$ and the joint PDF of the state value and the waiting time $\phi (p,\tau)$, Eq.~(\ref{joint-pdf exp}). 
%The exponential model is an SMP with joint PDF following Eq.~(\ref{joint-pdf exp}). 
 %The joint PDF can be written as
%\begin{equation}
%\phi  (p,\tilde{\tau} )= \delta \left( \tilde{\tau} - R(p)^{-1} \right) \chi  (p).
%\label{jpdfdet}
%\end{equation}
The deterministic model, which we will treat in Sect.~V, is identical to the SMP with a deterministic coupling 
between the state value and the waiting time. 
On the other hand, the SMP with an exponential conditional PDF of waiting times given the state 
is equivalent to the exponential model. For the SMP with $\chi(p)$ and $\phi (p,\tau)$,  
the Laplace transform of the propagator with respect to $t$ 
is obtained as in Ref.~\cite{Akimoto2020}. Applying the technique given in Ref.~\cite{Akimoto2020} to the exponential model, we find 
\begin{equation}
\hat{\rho} (p,s) = \frac{1}{s} \frac{\chi(p) - \hat{\phi}(p,s)}{1-\hat{\psi}(s)},
\label{MW-SMP}
\end{equation}
where $\hat{\phi}(p,s)$ and $\hat{\psi}(s)$ are the Laplace transforms of $\phi(p,\tilde{\tau})$ and $\psi(\tilde{\tau})$ with respect to 
$\tilde{\tau}$, respectively. Here,  initial conditions as for ordinary renewal processes were used \cite{Akimoto2020,Cox1962}.

%The theory of the SMP can be straightforwardly applied to an exponential model. In particular, the Laplace transform of 
%the propagator is given by  Eq.~(\ref{MW-SMP}). 
In the exponential model, the Laplace transform of the joint PDF is given by 
\begin{equation}
\hat{\phi}(p,s)= \frac{\chi (p) R(p)}{s+R(p)}.
\label{JPDF-SMP}
\end{equation}
If follows from Eqs.~(\ref{MW-SMP}) and (\ref{JPDF-SMP}) that $\hat{\rho} (p,s)$ becomes 
\begin{equation}
\hat{\rho} (p,s) =  \frac{ \chi(p)}{s+R(p)} \frac{1}{1-\hat{\psi}(s)}.
\end{equation}
In the long-time limit ($s\to 0$), it becomes 
\begin{equation}
\hat{\rho} (p,s) \cong  \frac{ 1}{s+c^{-1}|p|^{\alpha}} \frac{1}{2\Gamma(1-\alpha^{-1})\Gamma(1+\alpha^{-1}) (cs)^{\alpha^{-1}}},
\end{equation}
where %$a=\Gamma(1-\alpha^{-1})\Gamma(1+\alpha^{-1}) p_{\rm trap}^{-1}c^{\alpha^{-1}}$ {\color{red}and 
$\chi(p)=1/(2p_{\rm trap})$ is used. Interestingly, the Laplace transform 
of the propagator does not depend on $p_{\rm trap}$ in the long-time limit.  To obtain the exponential model from the HRW model, we assumed that
 $p_{\rm trap}$ is much smaller than $\sigma$. However, the asymptotic  behavior of the propagator is independent of $p_{\rm trap}$ in the exponential model. 
 Therefore, $p_{\rm trap}$ introduced in the HRW model can be assumed to be arbitrary small because the value of $p_{\rm trap}$ does 
 not affect the asymptotic behavior of the propagator of the exponential model. 
 %$p_{\rm trap}\ll \sigma$ without loss of generality 
 When $p_{\rm trap}\ll \sigma$, the distribution of momentum after jumping in the trapping region, i.e., $[-p_{\rm trap}, p_{\rm trap}]$, is approximately uniform.
 Therefore, 
the exponential model with the uniform approximation for $\chi(p)$ is a good approximation for the HRW model for large $t$. 
By the inverse Laplace transform, we have 
\begin{equation}
\rho(p,t)  \cong \frac{\sin (\pi \alpha^{-1})}{2\pi c^{\alpha^{-1}}  \Gamma (1+\alpha^{-1})} \int_0^t dt' e^{-c^{-1} |p|^{\alpha} (t-t')} t'^{\alpha^{-1} -1}
\label{propagator_asympt-exp}
\end{equation}
for $t\to\infty$. Through a change of variables ($u=t'/t$), we obtain 
\begin{equation}
\rho(p,t)  \cong \frac{\sin (\pi \alpha^{-1}) t^{\alpha^{-1} }}{2\pi c^{\alpha^{-1}} \Gamma (1+\alpha^{-1})} \int_0^1 du 
e^{-c^{-1} |p|^{\alpha} t(1-u)} u^{\alpha^{-1} -1}.
\label{propagator_asympt-exp2}
\end{equation}
Therefore, the cooled peak, i.e., $\rho(0,t)$, increases with $t^{\alpha^{-1}}$, which means that the probability 
of finding the cooled state ($p\cong 0$) increases with time, i.e., this is a signature of cooling.

For $|p|>0$ and $t\gg 1$, the integral in Eq.~(\ref{propagator_asympt-exp2}) can be approximated leading to
\begin{equation}
\rho(p,t)  \cong \frac{\sin (\pi \alpha^{-1}) t^{\alpha^{-1} -1}}{2\pi c^{\alpha^{-1}-1}  \Gamma (1+\alpha^{-1})} 
 \frac{1}{|p|^{\alpha} }.
\label{propagator_asympt-exp3}
\end{equation}
Furthermore, an infinite invariant density is obtained as 
\begin{equation}
  \lim_{t\to \infty} t^{1-\alpha^{-1}} \rho(p,t) = I_{\rm exp} (p) \equiv
 \frac{ \sin (\pi \alpha^{-1} ) \left\vert p\right\vert ^{-{\alpha}}}{2 \pi c^{\alpha^{-1}-1} \Gamma (1+\alpha^{-1}) }
\label{inf-d-exp}
\end{equation}
for $|p| \leq p_{\rm trap}$. The power-law form of Eq.~(\ref{inf-d-exp}), 
$I_{\rm exp} (p) \propto |p|^{-\alpha}$, in the exponential model matches with 
the infinite invariant density, Eq.~(\ref{steady-state}), in the HRW model.

Through a change of variables ($p'=t^{\alpha^{-1}} p/c^{\alpha^{-1}}$),
we obtain the rescaled propagator $\rho_{\rm res} (p',t)$. In the long-time 
limit, the rescaled propagator converges to a time-independent function $g_{\rm exp} (p')$ (scaling function):
\begin{equation}
 \rho_{\rm res} (p',t) \equiv \rho (c^{\alpha^{-1}} p'/t^{\alpha^{-1}},t) \left| \frac{dp}{dp'}\right|
\to g_{\rm exp} (p')  ,
\label{rescaling}
\end{equation}
where the scaling function is given by
\begin{equation}
 g_{\rm exp} (p') \equiv  \frac{\sin (\pi \alpha^{-1}) }{2\pi  \Gamma (1+\alpha^{-1})} \int_0^1 du 
e^{- |p'|^\alpha (1-u)} u^{\alpha^{-1} -1}.
 \label{sf-exp}
\end{equation}
This scaling function describes the propagator near $p=0$. 
This result was previously obtained by a different approach \cite{Barkai2021,*barkai2022gas}. 

Here, we are going to demonstrate 
 that the theory of the exponential model describes the asymptotic behavior of the propagator 
in the HRW model surprisingly well.  
Figure~\ref{propagator-exp} shows that the propagator for the HRW model 
%approaches zero in the long-time limit and 
is in perfect agreement with the analytical result of the exponential model, i.e., Eq.~(\ref{propagator_asympt-exp2}). 
%{\color{blue}I think that this statement is correct and the infinite invariant density is the same in between two models for arbitrary $p_{\rm trap}, p_0$ and $\sigma$, where the uniform approximation is assumed in the exponential model, because the result does not depend on $p_{\rm trap}$ in the exponential model. Therefore, the uniform density is special to reproduce the results of the HRW model. In other words, if we use another distribution for $\chi (p)$, some of the results in the HRW cannot be reproduced.} 
In the numerical simulations of the HRW model, we generated $10^8$ trajectories to obtain the propagator. 
There are two forms in the propagator. %, as in the propagator of the deterministic model. 
The propagator near $p=0$ increases with 
time $t$. On the other hand, the propagator for $p>0$ asymptotically approaches a power-law form, i.e., the infinite invariant density. 
%$t^{\gamma-1}I_{\rm exp}(p)$. %although the proportional constant depends on time $t$. 
Figure~\ref{propagator-rescale-exp} shows that the rescaled propagator of the HRW model for different times 
is well captured by the scaling function $g_{\rm exp} (p')$ without fitting parameters, where we generated $10^8$ trajectories 
to obtain the rescaled propagator. 
 Because the scaling function 
describes the details of the propagator near $p=0$ and is universal in the sense that it does not depend on $p_{\rm trap}$ in the exponential 
model, 
the dynamics of the HRW model near $p=0$ should also be universal and does not depend on the details of the jump 
distribution $G(\Delta p)$. In fact, as shown in Fig.~\ref{propagator-rescale-exp}, 
the rescaled propagator does not depend on $\sigma^2$. 
This is one of the reasons why the uniform approximation works very well. 
Moreover, because the momentum almost certainly approaches zero in the long-time limit, the assumption of $|p|\ll 1$ is correct for $t\gg 1$. 
Furthermore, it can be confirmed that Eq.~(\ref{propagator_asympt-exp2}) becomes a solution to the master equation, Eq.~(\ref{Master1}),
in the long-time limit, where the momentum at every jump is approximately renewed according to $G(\Delta p)$. 
%Our theory does not depend on the details of the uniform approximation. In other words, it is also valid for a non-uniform injection distribution such as a Gaussian distribution. 
Therefore,  the theory of the exponential well describes the propagator for the HRW model. 

%%%%%%%%%%%%%%%%%%%%%%%%%%%%%%%%%%%%%%%%%%%%%%%%%%%%%%%%%%%%%%%%%%%%
%Figure 3
\begin{figure}
\includegraphics[width=.95\linewidth, angle=0]{prop-inf-hrw-exp.eps}
\caption{Time evolution of the propagator, i.e. data from Fig.~\ref{prop-hrw}, multiplied by $t^{1-\gamma}$ in the HRW model 
for different times ($\alpha =2$,  $c=1$, $p_0=p_{\max}=1$, and $\sigma^2 =1$). 
Symbols with lines are the results of numerical simulations for the HRW model. 
The dashed lines represent the infinite invariant density, i.e., Eq.~(\ref{inf-d-exp}). The solid lines represent rescaled scaling functions, 
$t g_{\rm exp} (t^\gamma p)$. The dotted lines represent $t g_{\rm exp} (0)$ for different values of $t$. 
The initial momentum is chosen uniformly on $[-1,1]$.  %{\color{red}The number of trajectories used in the simulation is $10^6$.} 
 }
\label{propagator-exp}
\end{figure}
%%%%%%%%%%%%%%%%%%%%%%%%%%%%%%%%%%%%%%%%%%%%%%%%%%%%%%%%%%%%%%%%%%%%

%%%%%%%%%%%%%%%%%%%%%%%%%%%%%%%%%%%%%%%%%%%%%%%%%%%%%%%%%%%%%%%%%%%%
%Figure 4
\begin{figure}
\includegraphics[width=.95\linewidth, angle=0]{rescale-hrw-exp.eps}
\caption{Rescaled propagator of the HRW model for different values of $\sigma^2$  ($\alpha =2$, $c=1$, $p_0=p_{\max}=1$, and $t=10^4$).
Symbols with lines are the results of numerical simulations for the HRW model. 
The dashed solid line represents the scaling function, i.e., Eq.~(\ref{sf-exp}). 
The initial position is chosen uniformly on $[-1,1]$. %{\color{red}The number of trajectories used in the simulation is $10^6$.} 
Note that the results for different $\sigma^2$ are indistinguishable.}
%Jump distribution is  Gaussian with variance $0.1$. }
%All results for different times $t$ collapse to the master curve.}
\label{propagator-rescale-exp}
\end{figure}
%%%%%%%%%%%%%%%%%%%%%%%%%%%%%%%%%%%%%%%%%%%%%%%%%%%%%%%%%%%%%%%%%%%%

%Ensemble and time averages of observables
\subsection{Ensemble and time averages of observables }

In this subsection, we consider the ensemble average of an observable, which is defined as  
\begin{eqnarray}
\langle {\mathcal O}(p(t)) \rangle &\equiv& \int_{-p_{\rm trap}}^{p_{\rm trap}} {\mathcal O}(p) \rho(p,t)dp.
\label{ensemble-ave-def}
\end{eqnarray}
We assume that the observable is ${\mathcal O}(p) = C|p|^\beta$ and $\beta>-1$. 
For example, if $\beta=2$ we are considering the kinetic energy of atom. 
Through a change of variables ($p'=t^{\alpha^{-1}} p/c^{\alpha^{-1}}$) and using the scaling function, Eq.~(\ref{sf-exp}),
we have 
\begin{eqnarray}
\langle {\mathcal O}(p(t)) \rangle \sim \int_{- \left(\frac{t}{c}\right)^{\alpha^{-1}} p_{\rm trap}}^{\left(\frac{t}{c}\right)^{\alpha^{-1}} p_{\rm trap}}
 {\mathcal O} \left( \frac{c^{\alpha^{-1}} p'}{t^{\alpha^{-1}}} \right) g_{\rm exp} (p')dp'
\label{ensemble-ave-def-exp}
\end{eqnarray}
for $t\to\infty$. 

When $|p|^\beta$ is integrable with respect to $g_{\rm exp}(p)$, i.e., 
$\int_{-\infty}^\infty g_{\rm exp}(p) |p|^\beta dp<\infty$, $\beta$ satisfies $-1<\beta < \alpha -1$. 
%as in the deterministic model. 
In this case, the asymptotic behavior of the ensemble average becomes 
\begin{equation}
\langle {\mathcal O}(p(t)) \rangle \sim \frac{C c^{\beta \alpha^{-1}}}{t^{\beta\alpha^{-1}}} 
 \int_{-\infty}^\infty |p'|^\beta g_{\rm exp}(p')dp' \quad (t \to \infty).
\label{en-ave-scaling}
\end{equation}
On the other hand, when $|p|^\beta$ is integrable with respect to $I_{\rm exp} (p)$, i.e., 
$\int_{-p_{\rm trap}}^{p_{\rm trap}} I_{\rm exp} (p) {\mathcal O}(p) dv < \infty$,  $\beta$ satisfies $\beta > \alpha-1~(>0)$, 
implying that $|p|^\beta$ is not 
integrable with respect to the scaling function, i.e., $\int_{-\infty}^\infty g_{\rm exp}(p) |p|^\beta dp=\infty$. 
In this case, the asymptotic behavior of the ensemble average becomes 
\begin{equation}
 \langle {\mathcal O}(p(t)) \rangle \sim  t^{\alpha^{-1}-1} \int_{-p_{\rm trap}}^{p_{\rm trap}} I_{\rm exp} (p) {\mathcal O}(p) dv \quad (t \to \infty).
 \label{en-ave-infty-exp}
\end{equation}
Therefore, the asymptotic behavior becomes  
\begin{equation}
\langle {\mathcal O}(p(t)) \rangle \propto t^{-\lambda(\alpha,\beta)}\quad (t \to \infty), 
\end{equation}
and 
the integrability of the observable with respect to the scaling function or infinite invariant density 
determines the power-law exponent $\lambda(\alpha,\beta)$. 
In the case of $\beta = \alpha -1$, the integrals of the observable with respect to both the scaling function and 
infinite invariant density diverge. In this case, the integration in Eq.~(\ref{ensemble-ave-def-exp}) contains a logarithmic 
divergence for $t\to\infty$. Therefore, the leading order for $t\to\infty$ is 
\begin{equation}
\langle {\mathcal O}(p(t)) \rangle \propto t^{\alpha^{-1}-1} \ln t.
\end{equation}



The power-law exponent $\lambda(\alpha,\beta)$ in the exponential model is given by
\begin{equation}
\lambda(\alpha,\beta) = \left\{
\begin{array}{ll}
1 - \alpha^{-1}   & (\beta > \alpha-1)\\
\\
\beta \alpha^{-1} & (\beta < \alpha-1) .
\end{array}
\right.
\label{decay-exp}
\end{equation}
As will be shown later,  the decay process is universal in the sense that $\lambda(\alpha,\beta)$ 
does not depend on the three models that we consider here. Moreover, the fastest decay, which implies the maximum 
of $\lambda(\alpha,\beta)$, is realized at the transition point between integrable and non-integrable with respect to the 
infinite invariant measure, i.e., $\alpha=\beta + 1$. 
In particular, 
the fastest decay of the kinetic energy, i.e., $\beta=2$, can be achieved for $\alpha=3$, 
which suggests that the cooling efficiency, in a sense, is optimized at this point.  As shown in the previous subsection, 
the height of the cooled peak increases with $t^{\alpha^{-1}}$. Moreover, 
the half-width of the cooled peak in the momentum distribution decays with $t^{-\alpha^{-1}}$. 
If we use the half-width of the cooled peak in the momentum distribution to characterize the cooling efficiency, 
the optimized parameter is $\alpha=1$. Therefore, the most efficient cooling parameter depends on the definition of efficiency.


%Distributional behaviors of time-averaged observables in the exponential model
\subsection{Distributional characteristics of time-averaged observables}
Here, we construct a theory of the distribution of time averages in the exponential model. 
The time average of an observable ${\mathcal O}(p)$ is defined by
\begin{equation}
\overline{{\mathcal O}}(t) \equiv \frac{1}{t} \int_0^t {\mathcal O}(p(t'))dt'.
\label{ta-def}
\end{equation}
We obtain the mean and variance for two cases, when the observable
 is integrable with respect to the infinite invariant density and when it is not. In what follows, we consider kinetic energy as a specific example, i.e., ${\mathcal O}(p)=p^2$. 
The integrated value  of an observable ${\mathcal O}(p)$ denoted by ${\mathcal S}(t)$ 
can be represented by
\begin{eqnarray}
{\mathcal S}(t) &=& \int_0^t {\mathcal O}(p(t'))dt'\\
&=& \sum_{i=1}^{N(t)} \Delta {\mathcal S}_i + {\mathcal O}(p_{N(t)+1}) (t-t_{N(t)}),
\end{eqnarray}
where $\Delta {\mathcal S}_i = {\mathcal O}(p_i) \tilde{\tau}_i $, 
%$\tilde{\tau}_i$ is a random variable following the exponential distribution with mean $\tau(p_i)$, 
$N(t)$ is the number of jumps until time $t$, $p_i$ is the momentum during $[t_{i-1}, t_{i})$, and 
$t_i=\tilde{\tau}_1 + \cdots \tilde{\tau}_i$. 
The integrated value ${\mathcal S}(t)$ is a piecewise linear function of $t$ \cite{Barkai2021,*barkai2022gas} 
because ${\mathcal O}(p(t))$ is a piecewise constant function,
where $p_i$ and $\tilde{\tau}_i$ are coupled stochastically. 
%We studied distributional limit theorems of time averages in the deterministic version, i.e., the continuous accumulation process \cite{Akimoto2015, Akimoto2020}. 
The joint PDF of $\Delta {\mathcal S}_i$, $\tilde{\tau}_i$, and $p_i$ denoted by $\phi_{3} (x,\tilde{\tau},p)$ is given by 
\begin{equation}
\phi_{3} (x,\tilde{\tau},p) = \chi (p) R(p) e^{-R(p) \tilde{\tau}} \delta (x- {\mathcal O}(p)\tilde{\tau}).
\end{equation} 
The joint PDF of the integrated value of an elementary step and the waiting time $\tilde{\tau}$ is given by 
\begin{eqnarray}
\phi_{2} (x, \tilde{\tau}) &=& \int_{-p_{\rm trap}}^{p_{\rm trap}} dp \phi_{3} (x,\tilde{\tau},p) \nonumber\\
%&=& \frac{1}{2\sqrt{x\tilde{\tau}}}\sum_{p=\pm \sqrt{\frac{x}{\tilde{\tau}}}} \chi (p) R(p) e^{-R(p) \tau}\quad (\sqrt{x/\tilde{\tau}}<p_0) \nonumber\\
&=& \frac{1}{2 p_{\rm trap}\sqrt{x\tilde{\tau}}} R(\sqrt{x/\tilde{\tau}}) e^{-R(\sqrt{x/\tilde{\tau}}) \tilde{\tau}} \quad (\sqrt{x/\tilde{\tau}}<p_{\rm trap}). \nonumber
\end{eqnarray} 

Let $Q(x,t)$ be the PDF of $x={\mathcal S}(t)$ when a jump occurs exactly at time $t$; then, we have
\begin{equation}
Q(x,t) =  \int_0^x dx' \int_0^t dt'  \phi_{t} (x', t') Q(x-x', t-t') + Q_0(x,t), %\delta(x)\delta(t),
\end{equation}
where  $Q_0(x,t)=\delta(x)\delta(t)$. 
The PDF of ${\mathcal S}(t)$ at time $t$ is given by
\begin{eqnarray}
P(x,t) &=&  \int_0^x dx' \int_0^t  dt' \Phi_{2} (x', t') Q(x-x', t-t'),  %\nonumber\\&+& \Psi_{f} (x,t),
\end{eqnarray}
where
\begin{equation}
\Phi_2 (x,t) = \int_t^\infty d\tilde{\tau} \int_{-p_{\rm trap}}^{p_{\rm trap}} dp \chi (p) R(p) e^{-R(p) \tilde{\tau}} \delta (x- {\mathcal O}(p)t) .
\end{equation}
The double-Laplace transform with respect to $x$ and $t$ ($u\leftrightarrow x$ and $s\leftrightarrow t$) yields
\begin{equation}
\widehat{P}(u,s) =  \frac{\widehat{\Phi}_{2}(u,s)}{1- \widehat{\phi}_{2}(u,s)}, 
\label{montroll-weiss-like}
\end{equation}
where $\widehat{\phi}_{2}(u,s)$ and $\widehat{\Phi}_{2}(u,s)$ are 
the double-Laplace transforms of $\phi_{2} (x, \tilde{\tau} )$ and $\Phi_{2} (x,t)$, which are given by 
\begin{eqnarray}
\widehat{\phi}_{2}(u,s) &=& \int_0^\infty dx \int_0^\infty d\tau \int_{-p_{\rm trap}}^{p_{\rm trap}} dp  e^{-ux-s\tau} \phi_3(x,\tau,p)\nonumber\\
&=& \int_0^{p_{\rm trap}} \frac{c^{-1}p_{\rm trap}^{-1}p^\alpha}{s+up^2 + c^{-1}p^\alpha}dp
\label{psi_laplace_ta}
\end{eqnarray}
and
\begin{eqnarray}
\widehat{\Phi}_{2}(u,s) &=& \int_0^{p_{\rm trap}} \frac{p_{\rm trap}^{-1}}{s+up^2 + c^{-1}p^\alpha}dp, 
\label{PSI_laplace_ta}
\end{eqnarray}
respectively. Eq.~(\ref{montroll-weiss-like}) is the exact form of the PDF of ${\mathcal S}(t)$ in Laplace space.  Because 
$1-\widehat{\phi}_{2}(0,s)=s\widehat{\Phi}_{2}(0,s)$, normalization is actually satisfied, i.e., $\widehat{P}(0,s)=1/s$. 

The Laplace transform of the first moment of ${\mathcal S}(t)$ can be obtained as
\begin{equation}
-\left. \frac{\partial \widehat{P}(u,s)}{\partial u} \right|_{u=0} = 
-\frac{\widehat{\Phi}_{2}'(0,s)}{1- \widehat{\phi}_{2}(0,s)} - \frac{\widehat{\phi}_{2}'(0,s)}{s[1- \widehat{\phi}_{2}(0,s)]}. 
\label{laplace-1st-moment-exp}
\end{equation}
For $\alpha<3$, $\widehat{\phi}_{2}'(0,0)$ is finite, whereas it diverges for $\alpha\geq 3$. Therefore, $\alpha=3$ is a transition point 
at which the asymptotic behavior of $\langle {\mathcal S}(t) \rangle$ exhibits a different form.  
The asymptotic behavior of $1- \widehat{\phi}_{2}(0,s)$ for $s\to 0$ is given by 
\begin{eqnarray}
1- \widehat{\phi}_{2}(0,s) &=& s \int_0^{p_{\rm trap}} \frac{p_{\rm trap}^{-1}}{s + cp^\alpha}dp %\nonumber\\
\sim A_\alpha s^{1/\alpha},
\end{eqnarray}
where $A_\alpha$ is given by
\begin{equation}
A_\alpha = \frac{c^{1/\alpha}p_{\rm trap}^{-1}\pi}{\alpha \sin (\pi/\alpha)}.
\end{equation}
For $\alpha<3$, the leading order of Eq.~(\ref{laplace-1st-moment-exp}) is 
\begin{equation}
-\left. \frac{\partial \widehat{P}(k,s)}{\partial u} \right|_{u=0} \sim -
 \frac{\widehat{\phi}_{2}'(0,0)}{A_\alpha s^{1+\frac{1}{\alpha}}}, 
 \label{pd-P}
\end{equation}
where the first term in Eq.~(\ref{laplace-1st-moment-exp}) is ignored because $\widehat{\Phi}_{2}'(0,s)\propto s^{3/\alpha -2}$.
Therefore, the asymptotic behavior of $\langle {\mathcal S}(t) \rangle$ becomes 
\begin{equation}
\langle {\mathcal S}(t) \rangle \sim
 \frac{-\widehat{\phi}_{2}'(0,0)}{A_\alpha \Gamma(1+1/\alpha) } t^{\frac{1}{\alpha}}
 \label{1st-moment-X-a<3}
\end{equation}
for $t\to\infty$, where $-\widehat{\phi}_{2}'(0,0)=cp_{\rm trap}^{2-\alpha}/(3-\alpha)$.

For $\alpha \geq 3$, on the other hand, the asymptotic behavior of $\langle {\mathcal S}(t) \rangle$ becomes 
different from Eq.~(\ref{1st-moment-X-a<3}). 
For $\alpha >3$, the asymptotic behaviors of $-\widehat{\phi}_{2}'(0,s)$ and $-\widehat{\Phi}_{2}'(0,s)$ 
 for $s\to 0$ become 
\begin{eqnarray}
-\widehat{\phi}_{2}'(0,s) &=& \int_0^{p_{\rm trap}} \frac{cp_{\rm trap}^{-1}p^{2+\alpha}}{(s + cp^\alpha)^2}dp %\nonumber\\
\sim b_\alpha s^{3/\alpha -1}
\end{eqnarray}
and 
\begin{eqnarray}
-\widehat{\Phi}_{2}'(0,s) &=& \int_0^{p_{\rm trap}} \frac{p_{\rm trap}^{-1}p^{2}}{(s + cp^\alpha)^2}dp %\nonumber\\
\sim B_\alpha s^{3/\alpha -2},
\end{eqnarray}
where $b_\alpha$ and $B_\alpha$ are given by
\begin{equation}
b_\alpha = \frac{3c^{3/\alpha }\pi p_{\rm trap}^{-1}}{\alpha^2 \sin (3\pi/\alpha)}
\end{equation}
and
\begin{equation}
B_\alpha = \frac{(\alpha -3) \pi p_{\rm trap}^{-1}c^{3/\alpha}}{\alpha^2 \sin (3\pi/\alpha)},
\end{equation}
respectively.
Note that there is a logarithmic correction in the asymptotic behavior of $\langle {\mathcal S}(t) \rangle$ when $\alpha=3$. 
Therefore, the asymptotic behavior of $\langle {\mathcal S}(t) \rangle$ becomes 
\begin{eqnarray}
\langle {\mathcal S}(t) \rangle &\sim&
 \frac{b_\alpha+B_\alpha}{A_\alpha \Gamma(2-2/\alpha) } t^{1-\frac{2}{\alpha}}\nonumber\\
 &=& \frac{c^{2/\alpha}\sin(\pi/\alpha)}{ \Gamma(2-2/\alpha) \sin (3\pi/\alpha) } t^{1-\frac{2}{\alpha}}
 \label{1st-moment-X-a>3}
\end{eqnarray}
for $t\to\infty$. 

The Laplace transform of the second moment of ${\mathcal S}(t)$ can be obtained as
\begin{eqnarray}
\left. \frac{\partial^2 \widehat{P} (u,s)}{\partial u^2} \right|_{u=0} = 
\frac{\widehat{\Phi}_{2}''(0,s)}{1- \widehat{\phi}_{2}(0,s)} + \frac{2\widehat{\Phi}_{2}'(0,s)\widehat{\phi}_{2}'(0,s)}{[1- \widehat{\phi}_{2}(0,s)]^2} \nonumber\\
+\frac{\widehat{\Phi}_{2}(0,s)\widehat{\phi}_{2}''(0,s)}{[1- \widehat{\phi}_{2}(0,s)]^2} + 
 \frac{2\widehat{\Phi}_{2}(0,s)\widehat{\phi}_{2}'(0,s)^2}{[1- \widehat{\phi}_{2}(0,s)]^3}. 
\label{laplace-2nd-moment-exp}
\end{eqnarray}
For $\alpha<3$, the last term represents the leading term. Therefore, we have
\begin{eqnarray}
\left. \frac{\partial^2 \widehat{P}(k,s)}{\partial u^2} \right|_{u=0} \sim
 \frac{2\widehat{\phi}_{2}'(0,0)^2}{s[1- \widehat{\phi}_{2}(0,s)]^2}\sim 
  \frac{2\widehat{\phi}_{2}'(0,0)^2}{A_\alpha^2 s^{1+2/\alpha} }
\end{eqnarray}
for $s\to 0$. It follows that the asymptotic behavior of $\langle {\mathcal S}(t)^2 \rangle$ becomes 
\begin{equation}
\langle {\mathcal S}(t)^2 \rangle \sim
 \frac{2\widehat{\phi}_{2}'(0,0)^2}{A_\alpha^2 \Gamma(1+2/\alpha) } t^{\frac{2}{\alpha}}
 \label{2nd-moment-X-a<3}
\end{equation}
for $t\to\infty$. Because the ergodicity breaking (EB) parameter is given by 
\begin{equation}
{\rm EB}\equiv 
\frac{\langle \overline{\mathcal O}(t)^2 \rangle - \langle \overline{\mathcal O}(t) \rangle^2}{\langle \overline{\mathcal O}(t)\rangle^2} 
= \frac{\langle {\mathcal S}(t)^2 \rangle - \langle {\mathcal S}(t) \rangle^2}{\langle {\mathcal S}(t) \rangle^2}, 
\end{equation}
we have the EB parameter for the kinetic energy:
\begin{equation}
{\rm EB}\to 
 \frac{2 \Gamma(1+1/\alpha)^2}{\Gamma (1+2/\alpha)} -1 
 \label{eb-ML}
\end{equation}
for $t\to \infty$. This is a consequence of the Darling-Kac theorem \cite{Darling1957}. Thus, this is 
a universal result that does not depend on the subrecoil laser cooling model considered here. 

On the other hand, for $\alpha\geq 3$, all the terms in Eq.~(\ref{laplace-2nd-moment-exp}) contribute  to
the asymptotic behavior of $\langle {\mathcal S}(t)^2 \rangle$. 
For $\alpha >3$, the asymptotic behaviors of $\widehat{\Phi}_{2}''(0,s)$ and $\widehat{\phi}_{2}''(0,s)$ for $s\to 0$ become 
\begin{eqnarray}
\widehat{\phi}_{2}''(0,s) &=& \int_0^{p_{\rm trap}} \frac{2c^{-1}p_{\rm trap}^{-1}p^{4+\alpha}}{(s + c^{-1}p^\alpha)^3}dp %\nonumber\\
\sim c_\alpha s^{5/\alpha -2}
\end{eqnarray}
and 
\begin{eqnarray}
\widehat{\Phi}_{2}''(0,s) &=& \int_0^{p_{\rm trap}} \frac{2p_{\rm trap}^{-1}p^{4}}{(s + c^{-1}p^\alpha)^3}dp %\nonumber\\
\sim C_\alpha s^{5/\alpha -3},
\end{eqnarray}
where $c_\alpha$ and $C_\alpha$ are given by \begin{equation}
c_\alpha = \frac{5(-5+\alpha) \pi p_{\rm trap}^{-1} c^{5/\alpha}}{\alpha^3 \sin (5\pi /\alpha)}
\end{equation}
and 
\begin{equation}
C_\alpha = \frac{(-5+\alpha)(-5+2\alpha) \pi p_{\rm trap}^{-1} c^{5/\alpha}}{\alpha^3 \sin (5\pi /\alpha)},
\end{equation}
respectively. 
It follows that 
\begin{eqnarray}
\left. \frac{\partial^2 \widehat{P}(u,s)}{\partial u^2} \right|_{u=0} \sim
\left(\frac{c_\alpha + C_\alpha}{A_\alpha} + \frac{2B_\alpha b_\alpha}{A_\alpha^2} 
+ \frac{2b_\alpha^2}{A_\alpha^2} \right)s^{4/\alpha-3}\nonumber
\end{eqnarray}
for $s\to 0$. Therefore, in the long-time limit, 
\begin{equation}
\langle {\mathcal S}(t)^2 \rangle \sim \left(\frac{c_\alpha + C_\alpha}{A_\alpha} + \frac{2B_\alpha b_\alpha}{A_\alpha^2} 
+ \frac{2b_\alpha^2}{A_\alpha^2} \right) 
 \frac{t^{2(1-\frac{2}{\alpha})}}{\Gamma(3-4/\alpha) } ,
 \label{2nd-moment-X-a>3}
\end{equation}
 and the EB parameter becomes 
\begin{equation}
{\rm EB}\to \frac{2 \Gamma (2-2/\alpha)^2}{\alpha\Gamma (3-4/\alpha)} 
\left[ \frac{(-5+\alpha) \sin^2 (3\pi/\alpha)}{\sin (5\pi/\alpha) \sin (\pi/\alpha)} +3\right]
  -1 
  \label{EB-p2-a>3}
\end{equation}
for $t\to \infty$. Contrary to the universality in the case of $\alpha<3$, as will be shown later, 
this result is different from that in the deterministic model. 


\if0
integrable with respect to the infinite invariant density, i.e., $\int_0^1{\mathcal O}(p) I_\infty(p) dp <\infty$, where the ensemble 
average of the increment is actually finite. Therefore, the distribution of the time average follows the Mittag-Leffler distribution. 
More precisely, the normalized time averages defined by $\overline{{\mathcal O}}(t)/\langle \overline{\mathcal O}(t)\rangle$ converge in 
distribution to the Mittag-Leffler distribution:
\begin{equation}
\frac{\overline{\mathcal O}(t) }{\langle \overline{\mathcal O}(t)\rangle t} %\int_0^t  {\mathcal O}(t') dt' 
\Rightarrow M_\gamma 
\end{equation}
for $t\to\infty$, where $M_\gamma$ is a random variable. The ensemble average of the time average decays as 
$\langle \overline{\mathcal O}(t)\rangle \propto t^{\gamma-1}$ for $t\to\infty$ and, in general, 
$\langle \overline{\mathcal O}(t)^n \rangle \propto t^{n(\gamma-1)}$ for $t\to\infty$. Thus, $M_\gamma$ does not depend on time $t$ 
in the long-time limit. 
The mean of $M_\gamma$ is one by definition and the variance 
is given by 
\begin{equation}
{\rm ML}(\gamma) \equiv \frac{2\Gamma(1+\gamma)^2}{\Gamma (1+2\gamma)} -1.
\label{eb-ML}
\end{equation}
On the other hand, for $\alpha \geq 3$, observable ${\mathcal O}(p)=p^2$ is not integrable with respect to 
the infinite invariant density and the ensemble average of the increment also diverges. In this case, 
the normalized time average does not converge in distribution to $M_\gamma$ but another random variable $A_\gamma$: 
\begin{equation}
\frac{\overline{\mathcal O}(t)}{\langle \overline{\mathcal O}(t)\rangle} %\int_0^t  {\mathcal O}(t') dt' 
\Rightarrow A_\gamma 
\end{equation}
for $t\to\infty$. The ensemble average of the time average decays as 
$\langle \overline{\mathcal O}(t)\rangle \propto t^{-2\gamma}$ for $t\to\infty$ and, in general, 
$\langle \overline{\mathcal O}(t)^n \rangle \propto t^{-2n\gamma}$ for $t\to\infty$. The variance of $A_\gamma$ is given by
\begin{equation}
 {\rm A}(\gamma)\equiv \frac{6\gamma \Gamma(2-2\gamma)^2}{\Gamma(3-4\gamma)} 
\left[\frac{ 3 \Gamma (2 - 5\gamma) \Gamma (1-\gamma) }{5\gamma   \Gamma(1 -3\gamma)^2}
+ 1 \right] -1.
\label{eb-abs-inf}
\end{equation}
Since the normalized time average defined by $\overline{{\mathcal O}}(t)/\langle \overline{\mathcal O}(t)\rangle$ 
converges in distribution to $M_\gamma$ or $A_\gamma$ for $\alpha<3$ and $\alpha>3$, respectively, 
 the EB parameter, which is defined by the relative variance 
of $\overline{\mathcal O}(t)$, i.e., $\langle \overline{\mathcal O}(t)^2 \rangle/\langle \overline{\mathcal O}(t) \rangle^2 -1$, 
is given by ${\rm A}(\gamma)$ and ${\rm ML}(\gamma)$  for $\gamma<1/3$ and $\gamma>1/3$, respectively. 
As shown in Fig.~\ref{eb-gamma}, the trajectory-to-trajectory fluctuations of $\overline{\mathcal O}(t)$ surpress with  
increasing $\gamma$ for $\gamma>1/3$ and vanish for $\gamma\to 1$. On the other hand, they 
show non-trivial dependence of $\gamma$ for $\gamma<1/3$. 
We note that ${\displaystyle \lim_{\gamma\to1/3}{\rm A}(\gamma) = \lim_{\gamma\to1/3}{\rm ML}(\gamma)}$. 
\fi

%%%%%%%%%%%%%%%%%%%%%%%%%%%%%%%%%%%%%%%%%%%%%%%%%%%%%%%%%%%%%%%%%%%%
%Figure 5
\begin{figure}
\includegraphics[width=.95\linewidth, angle=0]{EB-gamma-all.eps}
\caption{EB parameter as a function of $\gamma$ ($=1/\alpha$) for the kinetic energy, i.e., ${\mathcal O}(p)=p^2$. 
Symbols are the results of numerical simulations for the HRW, deterministic, and exponential models. 
The solid line represents ${\rm A}(\gamma)$ and ${\rm ML}(\gamma)$ for $\gamma<1/3$ and $\gamma>1/3$, respectively.
The dashed line represents  Eq.~(\ref{EB-p2-a>3}). The solid line represents Eq.~(\ref{eb-ML}) and (\ref{eb-p2-det}).
 }
%All results for different times $t$ collapse to the master curve.}
\label{eb-gamma}
\end{figure}
%%%%%%%%%%%%%%%%%%%%%%%%%%%%%%%%%%%%%%%%%%%%%%%%%%%%%%%%%%%%%%%%%%%%


%section IV Stochastic model with a deterministic coupling
\section{Stochastic model with a deterministic coupling}
Here, we consider a stochastic model with a deterministic coupling, i.e., the deterministic model. 
This model is obtained by replacing the conditional PDF of the waiting time given the momentum 
by its mean. 
In this sense, this model is a mean-field-like model of the exponential model.
In the deterministic model, the conditional PDF $q(\tilde{\tau}|p)$ of $\tilde{\tau}$ given $p$ becomes deterministic:  
\begin{equation}
q(\tilde{\tau}|p) = \delta(\tilde{\tau} - R(p)^{-1}). 
\end{equation}
Using Eq.~(\ref{joint-pdf exp}) and integrating over momentum $p$ yields that the PDF of the waiting time follows a power law: 
\begin{equation}
\psi (\tilde{\tau}) = \gamma p_{\rm trap}^{-1} c^\gamma \tilde{\tau}^{-1 -\gamma} \quad (\tilde{\tau}\geq c p_{\rm trap}^{-\gamma^{-1}}).
\label{waiting-time-pdf-det}
\end{equation} 
%This is because the probability of $\tilde{\tau}<x$ is given by 
%\begin{eqnarray}
%\Pr (\tilde{\tau} \leq x)&=& \Pr (c|p|^{-\alpha} \leq x) = \Pr (|p| \geq (x/c)^{-\frac{1}{\alpha}}) \nonumber\\
%&=&  1- p_{\rm trap}^{-1}(x/c)^{-\frac{1}{\alpha}} 
%\end{eqnarray}
%for $x \geq cp_{\rm trap}^{-\alpha}$. 



\if0
The dynamics of the deterministic model {\color{red}is}
 described by a semi-Markov process (SMP) with continuous variables, which was introduced 
in our previous paper \cite{Akimoto2020}. 
In the SMP, the state value is determined by the waiting time, which is randomly 
selected, or equivalently, the waiting time is determined by the state value, which is randomly chosen. 
In the HRW model, the method of determination follows the latter, and the state value (momentum) is not an IID random variable. 
{\color{red}On the other hand, in the deterministic model, the state values or the waiting times are IID random variables. 
The SMP is characterized by the joint PDF of the state value and the waiting time. 
 The joint PDF can be written as
\begin{equation}
\phi  (p,\tilde{\tau} )= \delta \left( \tilde{\tau} - R(p)^{-1} \right) \chi  (p).
\label{jpdfdet}
\end{equation}
 \fi

\subsection{Scaling function and infinite invariant density}
%Here, we summarize the results for the SMP (for details, see Ref.~\cite{Akimoto2020}). 
%the relation between the momentum and the waiting time is given by $\tau=c|p|^{- \alpha}$. 
%($\nu -1 = - \gamma$ in the laser cooling model). 
%In the SMP, the Laplace transform of the propagator with respect to $t$ is represented by 
%\begin{equation}
%\hat{\rho} (p,s) = \frac{1}{s} \frac{\chi(p) - \hat{\phi}(p,s)}{1-\hat{\psi}(s)},
%\label{MW-SMP}
%\end{equation}
%where $\hat{\phi}(p,s)$ and $\hat{\psi}(s)$ are the Laplace transforms of $\phi(p,\tilde{\tau})$ and $\psi(\tilde{\tau})$ with respect to 
%$\tilde{\tau}$, respectively. 
The deterministic model is described by the SMP. Using Eq.~(\ref{MW-SMP}), 
we have
\begin{equation}
\hat{\rho} (p,s) = \frac{\chi(p)}{s} \frac{1 - e^{-sR(p)}}{1-\hat{\psi}(s)}.
\end{equation}
Because $\psi(\tilde{\tau})$ follows a power law, i.e., Eq.~(\ref{waiting-time-pdf-det}), the asymptotic form of the 
the Laplace transform $\hat{\psi}(s)$ for $s\to 0$ is given by 
\begin{equation}
\hat{\psi}(s) = 1 - a s^\gamma + o(s^\gamma), 
\end{equation}
where $a= \Gamma(1-\gamma) p_{\rm trap}^{-1}c^\gamma$. 
In the long-time limit, the propagator is expressed as 
\begin{equation}
\rho(p,t)  \sim 
\begin{cases}
\dfrac{\sin (\pi \gamma) }{2\pi \gamma    } \left(\dfrac{t}{c}\right)^{\gamma } \quad &(|p| \leq p_c(t))\\
\\
\dfrac{\sin (\pi \gamma)}{2 \pi\gamma  } \dfrac{t^{\gamma} - (t- t_{c}(p))^\gamma }{c^\gamma} &(|p|>p_c(t)),
\end{cases}
\label{propagator_asympt1}
\end{equation}
 where $p_{c}(t)=(t/c)^{-\gamma}$ and $t_c(p)= c|p|^{-\gamma^{-1}}$. 
We note that $\rho (p,t)$ is discontinuous at $|p|=p_{c}(t)$, in contrast to the HRW model. 
%because $\left\langle N(0)\right\rangle=1$ and $\left\langle N(t)\right\rangle=0$ when $t<0$.  
Importantly, the asymptotic behavior of the propagator, as expressed by Eq.~(\ref{propagator_asympt1}), 
does not depend on the details of the uniform approximation; i.e., $\rho(p,t)$ is independent of $p_{\rm trap}$.
For any small $\varepsilon>0$, there exists $t$ such that $p_c(t) < \varepsilon$ because $p_c(t) \to 0$ for $t\to\infty$. 
Therefore, for any small $\varepsilon>0$, the probability of $|p|>\varepsilon$ becomes zero for $t\to\infty$. 
More precisely, for $t\gg t_c(\varepsilon)$, the probability is given by
\begin{equation}
\Pr (|p|>\varepsilon) \sim \frac{ \sin (\pi \gamma)}{1-\gamma}(1-\varepsilon^{1-\gamma})t^{\gamma -1}.
\end{equation}
Therefore, the temperature of the system almost certainly approaches zero in the long-time limit. 

By changing the variables ($p'=t^{\gamma} p/c^\gamma$), we obtain the rescaled propagator $\rho_{\rm res} (p',t)$. In the long-time 
limit, the rescaled propagator converges to a time-independent function $g_{\rm det} (p')$ (scaling function):
\begin{equation}
 \rho_{\rm res} (p',t) \equiv \rho (c p'/t^{\gamma},t) \left| \frac{dp}{dp'}\right|
\to g_{\rm det} (p')  ,
\label{rescaling}
\end{equation}
where the scaling function is given by
\begin{equation}
 g_{\rm det} (p') \equiv \left\{
\begin{array}{ll}
 \dfrac{  \sin (\pi \gamma )}{2\pi c^{\gamma-1} \gamma   } ~&(|p'|<1)\\
 \\
 \dfrac{ \sin (\pi \gamma ) \{ 1 - (1-|p'|^{-\gamma^{-1}})^\gamma\}}{2\pi c^{\gamma-1} \gamma  } %|p'|^{-1 + \frac{\gamma}{1-\nu}} 
 &(|p'| \geq 1) .
 \end{array}
 \right.
 \label{master-curve}
\end{equation}
This scaling function describes the details of the propagator near $p=0$. Furthermore, an infinite invariant density
 is obtained as a formal steady state: 
\begin{equation}
 I_\infty(p) \equiv  \lim_{t\to \infty} t^{1-\gamma} \rho(p,t) 
 = \frac{ \sin (\pi \gamma ) \left\vert p\right\vert ^{-\gamma^{-1}}}{2 \pi c^\gamma  }
\label{inf-d}
\end{equation}
for $|p|<p_{\rm trap}$.  In the long-time limit, the propagator can be almost described by the infinite invariant density, 
 whereas the former is normalized and the latter is not. The 
infinite invariant density $I_\infty(p)$ is the same as the formal steady state obtained using Eq.~(\ref{steady-state}). 
However,  the propagator described by Eq.~(\ref{propagator_asympt1}) is not a solution of the master equation, 
Eq.~(\ref{Master1}).



Figure~\ref{propagator} shows the scaled propagator of the deterministic model.
In the numerical simulations, we generated $10^8$ trajectories to obtain the propagator. 
There are two forms of the propagator. For $|p|<p_c(t)$, the propagator increases with 
time $t$. For $|p|>p_c(t)$, the asymptotic form of the propagator follows the infinite invariant density $t^{\gamma-1}I_\infty(p)$. 
Because the constant $t^{\gamma -1}$ approaches zero in the long-time limit, the propagator outside $p_c(t)$ becomes zero. 
A cusp exists at $p=t_c(t)$, in contrast to the HRW and the exponential model, where no cusp exists in the propagator. 
Figure~\ref{propagator-rescale} shows numerical simulations of 
the rescaled propagators in the deterministic case for different $\chi(p)$, i.e., for uniform and Gaussian distributions. The propagators 
are compared with the scaling function $g_{\rm det} (p')$ without fitting parameters, where we generate $10^8$ trajectories 
to obtain the rescaled propagator. 
 Therefore, the scaling function 
describes the details of the propagator near $p=0$ and is universal in the sense that it does not depend on $\chi(p)$.


%%%%%%%%%%%%%%%%%%%%%%%%%%%%%%%%%%%%%%%%%%%%%%%%%%%%%%%%%%%%%%%%%%%%
%Figure 6
\begin{figure}
\includegraphics[width=.95\linewidth, angle=0]{prop-inf-det.eps}
\caption{Time evolution of the propagator multiplied by $t^{\gamma-1}$ in the deterministic model 
for different times ($\alpha=\gamma^{-1} =2, c=1$, and $p_{\rm trap}=1$). 
Symbols with lines represent the results of numerical simulations of the deterministic model. 
The dashed lines represent the infinite invariant density $I_\infty(p)$ given by Eq.~(\ref{inf-d}). 
The solid lines represent rescaled scaling functions, 
$t g_{\rm det} (t^\gamma p)$. The dotted lines represent $t g_{\rm det} (0)$ for different values of $t$. 
The initial position is chosen uniformly on $[-1,1]$.  %{\color{red}The number of trajectories used in the simulation is $10^6$.} 
 }
\label{propagator}
\end{figure}
%%%%%%%%%%%%%%%%%%%%%%%%%%%%%%%%%%%%%%%%%%%%%%%%%%%%%%%%%%%%%%%%%%%%


%%%%%%%%%%%%%%%%%%%%%%%%%%%%%%%%%%%%%%%%%%%%%%%%%%%%%%%%%%%%%%%%%%%%
%Figure 7
\begin{figure}
\includegraphics[width=.9\linewidth, angle=0]{sf-det-t-104.eps}
\caption{Rescaled propagators  for different distributions $\chi(p)$ ($\alpha=\gamma^{-1} =2$, $c=1$,  and $p_{\rm trap}=1$), where 
we consider the uniform distribution $\chi(p)=1/2$ on $p\in [-1,1]$ and the Gaussian distribution $\chi (p) = 
\exp(-p^2/2)/\sqrt{2\pi}$. 
Symbols with lines are the results of the numerical simulations of the deterministic model with $t=10^4$. 
The solid line represents the scaling function given by Eq.~(\ref{master-curve}). 
The initial position is chosen uniformly on $[-1,1]$. %{\color{red}The number of trajectories used in the simulation is $10^6$.} 
Note that the results for different $\chi(p)$ are indistinguishable.}
%Jump distribution is  Gaussian with variance $0.1$. }
%All results for different times $t$ collapse to the master curve.}
\label{propagator-rescale}
\end{figure}
%%%%%%%%%%%%%%%%%%%%%%%%%%%%%%%%%%%%%%%%%%%%%%%%%%%%%%%%%%%%%%%%%%%%


%Ensemble and time averages of observables
\subsection{Ensemble and time averages of observables}





Here, we consider the ensemble averages of observables and 
show that the scaling function and infinite invariant density play an important role. 
% in obtaining the time dependence of the ensemble average of some observables. 
%In particular, we consider the energy as the observable, which converges to zero in the long-time limit. 
%We discuss an efficient cooling situation by the relaxation process of the energy. 
In this subsection, we set $p_{\rm trap}=1$ for simplicity.
The ensemble average of an observable ${\mathcal O}(p)$ is given by Eq.~(\ref{ensemble-ave-def}), 
which can be represented using the scaling function and infinite invariant density. To verify, we divide the integral range 
as 
\begin{widetext}
%Using The ensemble average of function ${\mathcal O}(p)$ is given by 
\begin{equation}
\langle {\mathcal O}(p(t)) \rangle %\equiv \int_{-1}^1 {\mathcal O}(p) \rho(p,t)dp
= \int_{-p_c(t)}^{p_c(t)} \rho(p,t) {\mathcal O}(p) dp + \int_{|p|>p_c(t)} \rho (p,t) {\mathcal O}(p)dp.
%+  \int_{-\infty}^{-v_c(t)} p(v,t) f(v) dv.
\label{ensemble-ave}
\end{equation}
In the long-time limit, using the scaling function and infinite invariant density, we have 
\begin{equation}
\langle {\mathcal O}(p(t)) \rangle 
\cong \int_{-1}^{1} g_{\rm det} (p') {\mathcal O}(cp'/t^{\gamma}) dp' + t^{\gamma-1} \int_{|p|>p_c(t)} I_\infty (p) {\mathcal O}(p) dp,
\label{ensemble-ave2}
\end{equation}
where we applied a change of variables in the first term and used Eqs.~(\ref{propagator_asympt1}), (\ref{master-curve}), and 
(\ref{inf-d}). 
\end{widetext}


Here, we assume that ${\mathcal O}(p) \sim C|p|^\beta$ for $p\to 0$ and that it is bounded for $p\ne 0$. In particular,  
the energy and the absolute value of the momentum correspond to observables with 
$\beta=2$ and $\beta=1$, respectively. 
When $|p|^\beta$ is integrable with respect to $g_{\rm det}(p)$, i.e., 
$\int_{-\infty}^\infty g_{\rm det}(p) |p|^\beta dp<\infty$, 
%({\color{blue}I think the integral boundaries are OK because the support of $[-\infty, \infty]$.}) 
$\gamma^{-1}$ satisfies the following inequality: $-1<\beta < \gamma^{-1}-1$. 
In this case, the asymptotic behavior of the ensemble average becomes 
\begin{equation}
\langle {\mathcal O}(p(t)) \rangle \sim \frac{C c^{\beta -\gamma+1} \sin(\pi\gamma)}{\pi  \gamma(\beta+1)} 
 t^{-\beta \gamma} \quad (t \to \infty), 
\label{en-ave-scaling}
\end{equation}
where we used Eq.~(\ref{master-curve}):
\begin{equation}
\int_{-1}^{1} g_{\rm det}(p') {\mathcal O}(cp'/t^{\gamma})dp' \sim C c^\beta  \int_{-1}^1 g_{\rm det}(p') |p'|^{\beta} 
dp' t^{-\beta \gamma}
\end{equation}
for $t \to \infty$.
Note that the second term in Eq.~(\ref{ensemble-ave2}) can be ignored in the asymptotic behavior because 
$-\beta \gamma>\gamma -1$. 
On the other hand, 
when ${\mathcal O}(p)$ is integrable with respect to $I_\infty (p)$, i.e., $\int_{-1}^1 I_\infty (p) {\mathcal O}(p) dp < \infty$,  
where  $\beta$ must satisfy $\beta > \gamma^{-1}-1~ (>0)$, the asymptotic behavior of the ensemble average becomes 
\begin{equation}
 \langle {\mathcal O}(p(t)) \rangle \sim  t^{\gamma-1} \int_{-1}^1 I_\infty (p) {\mathcal O}(p) dp \quad (t \to \infty).
 \label{en-ave-infty}
\end{equation} 
Therefore, the asymptotic behavior of the ensemble average becomes proportional to $ t^{-\lambda(\alpha,\beta)}$, and 
the integrability of the observable with respect to the scaling function or infinite invariant density 
determines the power-law exponent $\lambda(\alpha,\beta)$. Note that the exponent $\gamma$ is defined as 
$\gamma=1/\alpha$. Therefore, the power-law exponent in 
decay processes of the ensemble- and time-averaged observable is universal. 

In the case of $\beta = \gamma^{-1}-1$, the integrals of the observables with respect to both the scaling function and 
infinite invariant density diverge.
%We note that there is a logarithmic correction when $\beta = \gamma^{-1}-1$. 
In this case, Eq.~(\ref{ensemble-ave2}) should be expressed as 
\begin{widetext}
\begin{equation}
\langle {\mathcal O}(p(t)) \rangle 
= \int_{-1}^{1} g_{\rm det} (p') {\mathcal O}(cp'/t^{\gamma}) dp' +
\int_{1<|p'|\leq t^\gamma/c} g_{\rm det} (p') {\mathcal O}(cp'/t^{\gamma}) dp' .
\label{ensemble-ave3}
\end{equation}
\end{widetext}
The first term decays as $t^{-\beta \gamma}$ because the integral of the observable ${\mathcal O}(p)$ from -1 to 1 
with respect to the scaling function is finite. 
Because there is a logarithmic correction in the second term, the second term yields the leading order for $t\to\infty$:
\begin{equation}
\langle {\mathcal O}(p(t)) \rangle \sim \frac{C c^{\gamma^{-1}-\gamma-1} \gamma \sin(\pi\gamma)}{\pi  } t^{\gamma-1} \ln t.
\end{equation}


Here, we discuss the decrease of the energy.  When the observable is the energy, i.e., 
${\mathcal O}(p)=p^2$, the asymptotic decay is 
\begin{equation}
\langle p(t)^2 \rangle \sim \frac{t^{-2 \gamma}}{\beta+1}\quad(t\to\infty)
\end{equation} 
or
\begin{equation}
\langle p(t)^2 \rangle \sim t^{\gamma-1} \int_{-1}^1 I_\infty (p) {\mathcal O}(p) dv \quad(t\to\infty)
\end{equation} 
for $\gamma^{-1}>3$ and $\gamma^{-1}<3$, respectively. Thus, the ensemble average of the energy approaches zero in the long-time limit. 
Interestingly, a constraint exists in the power-law exponent
$\lambda(2,\gamma)$; i.e., $\lambda(2,\gamma) \leq 2/3$, where the equality holds at $\gamma^{-1}=\alpha=3$. 
 For general observables, 
the power-law exponent is restricted as 
\begin{equation}
\lambda(\beta,\gamma) < \frac{\beta}{\beta +1}.
\end{equation}
In the case of the absolute value of the momentum, it is bounded as $\lambda(1,\gamma) < 1/2$, which is maximized 
at $\gamma^{-1}=2$. 

%Distributional behaviors of time-averaged observables in the deterministic model
\subsection{Distributional characteristics of time-averaged observables}
 Distributional limit theorems for time-averaged observables  in the SMP with continuous state variables 
 were also considered in Ref.~\cite{Akimoto2020},
where the infinite invariant density plays an important role in discriminating classes of observables. 
%The time average of an observable ${\mathcal O}(p)$ is defined by.
For the SMP, the integral of ${\mathcal O}(p(t))$ is a piecewise linear function of $t$  and is called a 
  continuous accumulation process \cite{Akimoto2015}. The 
ensemble average of an increment of one segment, i.e., 
\begin{equation}
\left\langle \int_0^{\tilde{\tau}} {\mathcal O}(p(t'))dt'\right\rangle \equiv \int_0^\infty \tilde{\tau} {\mathcal O}
\left( c^\gamma\tilde{\tau}^{-\gamma} \right) \psi (\tilde{\tau})d\tilde{\tau},
\end{equation}
may diverge for some observables. When it is finite, the distribution function of the time-averaged observable 
follows the Mittag--Leffler distribution, which is a well-known distribution in infinite ergodic theory \cite{Aaronson1997,shinkai2006lempel} 
and stochastic processes \cite{Darling1957,kasahara77,Lubelski2008,He2008, Miyaguchi2011, Miyaguchi2013, Akimoto2013a,AkimotoYamamoto2016a,Albers2018, Radice2020, Albers2022}.
On the other hand, when it diverges, other non-Mittag-Leffler limit distributions are known \cite{Akimoto2008, Akimoto2015, Albers2018, Akimoto2020, Barkai2021,*barkai2022gas, Albers2022}.
%the distribution function instead follows a distribution similar to that given in Ref.~\cite{Akimoto2015}. 
This condition of integrability of the increment can be represented by 
the integrability of the observable with respect to the infinite invariant density. 

Here, we consider energy as a specific example. The distributional limit theorems derived in Ref.~\cite{Akimoto2020} can be 
straightforwardly applied to this case. A derivation of the distributional limit theorems is given in Appendix~A. Here, we 
simply apply our previous results. 
For $\gamma<1/3$, the observable ${\mathcal O}(p)=p^2$ is 
integrable with respect to the infinite invariant density, i.e., $\int_0^1{\mathcal O}(p) I_\infty(p) dp <\infty$, where the ensemble 
average of the increment is finite. Therefore, the distribution of the time average follows the Mittag--Leffler distribution. 
More precisely,  the normalized time averages defined by $\overline{{\mathcal O}}(t)/\langle \overline{\mathcal O}(t)\rangle$ converges 
in distribution: 
\begin{equation}
\frac{\overline{\mathcal O}(t) }{\langle \overline{\mathcal O}(t)\rangle } %\int_0^t  {\mathcal O}(t') dt' 
\Rightarrow M_\gamma 
\end{equation}
for $t\to\infty$, where $M_\gamma$ is a random variable, distributed according to the Mittag-Leffler law \cite{Aaronson1997, Miyaguchi2013}.
The ensemble average of the time average decays as 
$\langle \overline{\mathcal O}(t)\rangle \propto t^{\gamma-1}$ for $t\to\infty$ and, in general, 
$\langle \overline{\mathcal O}(t)^n \rangle \propto t^{n(\gamma-1)}$ for $t\to\infty$. Thus, $M_\gamma$ does not depend on time $t$. 
in the long-time limit. 
The mean of $M_\gamma$ is one by definition and the variance 
is given by 
\begin{equation}
{\rm ML}(\gamma) \equiv \frac{2\Gamma(1+\gamma)^2}{\Gamma (1+2\gamma)} -1.
\label{eb-ML2}
\end{equation}
On the other hand, for $\gamma \geq 1/3$, the observable ${\mathcal O}(p)=p^2$ is not integrable with respect to 
the infinite invariant density, and the ensemble average of the increment also diverges. In this case, 
 the normalized time average does not converge in distribution to $M_\gamma$ but rather to another random variable $C_\gamma$ \cite{Akimoto2020}: 
\begin{equation}
\frac{\overline{\mathcal O}(t)}{\langle \overline{\mathcal O}(t)\rangle} %\int_0^t  {\mathcal O}(t') dt' 
\Rightarrow C_\gamma 
\end{equation}
for $t\to\infty$. The ensemble average of the time average decays as 
$\langle \overline{\mathcal O}(t)\rangle \propto t^{-2\gamma}$ for $t\to\infty$ and, in general, 
$\langle \overline{\mathcal O}(t)^n \rangle \propto t^{-2n\gamma}$ for $t\to\infty$. The variance of $C_\gamma$ is given by
\begin{equation}
 {\rm A}(\gamma)\equiv \frac{6\gamma \Gamma(2-2\gamma)^2}{\Gamma(3-4\gamma)} 
\left[\frac{ 3 \Gamma (2 - 5\gamma) \Gamma (1-\gamma) }{5\gamma   \Gamma(1 -3\gamma)^2}
+ 1 \right] -1.
\label{eb-p2-det}
\end{equation}
Since the distribution of the normalized time average defined by $\overline{{\mathcal O}}(t)/\langle \overline{\mathcal O}(t)\rangle$ 
converges to $M_\gamma$ or $C_\gamma$ for $\gamma<1/3$ and $\gamma>1/3$, respectively, 
 the EB parameter, which is defined by the relative variance 
of $\overline{\mathcal O}(t)$, i.e., $\langle \overline{\mathcal O}(t)^2 \rangle/\langle \overline{\mathcal O}(t) \rangle^2 -1$. 
is given by ${\rm A}(\gamma)$ and ${\rm ML}(\gamma)$  for $\gamma<1/3$ and $\gamma>1/3$, respectively. 
As shown in Fig.~\ref{eb-det}, the trajectory-to-trajectory fluctuations of $\overline{\mathcal O}(t)$ are suppressed by  
increasing $\gamma$ for $\gamma>1/3$ and vanish for $\gamma\to 1$. On the other hand, they 
show a non-trivial dependence on $\gamma$ for $\gamma<1/3$. 
We note that ${\displaystyle \lim_{\gamma\to1/3}{\rm A}(\gamma) = \lim_{\gamma\to1/3}{\rm ML}(\gamma)}$. 

%%%%%%%%%%%%%%%%%%%%%%%%%%%%%%%%%%%%%%%%%%%%%%%%%%%%%%%%%%%%%%%%%%%%
%Figure 8
\begin{figure}
\includegraphics[width=.95\linewidth, angle=0]{EB-det.eps}
\caption{EB parameter as a function of $\gamma$ for two observables ${\mathcal O}(p)=p^2$ and ${\mathcal O}(p)=I(|p|>0.5)$, 
where ${\mathcal O}(p)=I(|p|>0.5)=1$ if $|p|>0.5$ and zero otherwise. 
The solid line represents ${\rm A}(\gamma)$ and ${\rm ML}(\gamma)$ for $\gamma<1/3$ and $\gamma>1/3$, respectively.
The dashed line represents  ${\rm ML}(\gamma)$ for $\gamma<1/3$. Note that $I(|p|>0.5)$ is integrable with respect to 
$I_\infty (p)$ for all $\gamma$. 
 }
%All results for different times $t$ collapse to the master curve.}
\label{eb-det}
\end{figure}
%%%%%%%%%%%%%%%%%%%%%%%%%%%%%%%%%%%%%%%%%%%%%%%%%%%%%%%%%%%%%%%%%%%%


%%%%%%%%%%%%%%%%%%%%%%%%%%%%%%%%%%%%%%%%%%%%%%%%%%%%%%%%%%%%%%%%%%%%
%Figure 9
%\begin{figure}
%\includegraphics[width=.95\linewidth, angle=0]{summary-subrecoil.eps}
%\caption{Comparison of the infinite invariant density, the scaling function, the relaxation power-law exponent of the time-and-ensemble averaged energy, and the EB parameter in three stochastic models.}
%All results for different times $t$ collapse to the master curve.}
%\label{sum}
%\end{figure}
%%%%%%%%%%%%%%%%%%%%%%%%%%%%%%%%%%%%%%%%%%%%%%%%%%%%%%%%%%%%%%%%%%%%

%%%%%%%%%%%%%%%%%%%%%%%%%%%%%%%%%%%%%%%%%%%%%%%%%%%%%%%%%%%%%%%%%%%%
%Table 1
\begin{table*}
  \begin{tabular}{|p{30mm}|p{30mm}|p{30mm}|p{30mm}|} % l:左寄せ,c:中央揃え r:右寄せ 
    \hline  & HRW & exponential model & deterministic model \\ \hline
    model & Markov & Markov & non-Markov \\ \hline
    invariant density & $\rho^*(p) \propto |p|^{-\alpha}$ & $\rho^*(p) \propto |p|^{-\alpha}$  & $\rho^*(p) \propto |p|^{-\alpha}$ \\ \hline
    scaling function & same as in the exponential model & Eq. (\ref{sf-exp}) &  Eq.~(\ref{master-curve}) \\ \hline
    decay exponent & same as in the exponential model & Eq. (\ref{decay-exp}) &  Eq. (\ref{decay-exp}) \\ \hline
    EB (integrable) & same as in the exponential model & $\dfrac{2\Gamma(1+\gamma)^2}{\Gamma(1+2\gamma)} -1$ &  $\dfrac{2\Gamma(1+\gamma)^2}{\Gamma(1+2\gamma)} -1$ \\ \hline     
    EB (non-integrable) & same as in the exponential model & Eq. (\ref{EB-p2-a>3}) &  Eq. (\ref{eb-p2-det}) \\ \hline
    %\hline
    %\begin{minipage}{10mm}
      %\centering
      %\scalebox{0.3}{\includegraphics{summary-subrecoil.eps}}
    %\end{minipage} 
  \end{tabular}
  \caption{Comparison of the infinite invariant density, the scaling function, the relaxation power-law exponent of the time-and-ensemble averaged energy, and the EB parameter in three stochastic models.}
  \label{sum} % \ref{ラベル名}で表番号を参照
\end{table*}
%%%%%%%%%%%%%%%%%%%%%%%%%%%%%%%%%%%%%%%%%%%%%%%%%%%%%%%%%%%%%%%%%%%%

\section{Conclusion}
We investigated the accumulation process of the momentum of an atom in three stochastic models of subrecoil laser cooling.
%For low temperature ($|p|\ll 1$), we obtained a heterogeneous diffusion equation, where the diffusion coefficient depends on the momentum, as derived from the master equation. 
For the HRW and the exponential models, the formal steady state of the master equation
 cannot be normalized when $\alpha\geq 1$. For all the models, the scaled propagator defined by $t^{1-\gamma} \rho (p,t)$ 
 converges to a time-independent function, i.e., an infinite invariant density. 
In the deterministic and exponential model, we derived the exact forms of the scaling function and the 
infinite invariant density. As a result, we found universality and non-universality in all three stochastic models.
In particular, the power-law form of the infinite invariant density 
is universal in the three models, whereas there is a clear difference in the scaling functions of the 
deterministic and exponential models. A summary of the comparisons of the three stochastic models 
is presented in Table~\ref{sum}.

We numerically showed that the propagator obtained using the exponential model 
is in perfect agreement with that in the HRW model for large $t$, which means that 
 the uniform approximation used in the exponential model 
is very useful for obtaining a deeper understanding of the HRW model. 
%For all the models, the momentum of an atom accumulates at zero in the long-time limit. 
When we focus on the jumps of the momentum to the trapping region, the jump distribution can be taken as 
approximately uniform in 
the trapping region because the trap size $p_{\rm trap}$ can be arbitrarily small. We note that the uniform distribution for $\chi(p)$ is 
necessary but the value of $p_{\rm trap}$ is not relevant for reproducing the statistical behavior of the HRW model.
This is the reason why the uniform approximation can be applied to the HRW model. 
The relation between the exponential and the HRW models is similar to that between the CTRW and the quenched trap model (QTM)  \cite{bouchaud90}. 
In particular, the waiting times in the exponential model and the CTRW are IID random variables, whereas 
those in the HRW and the QTM are not. Moreover, it is known that the CTRW
is a good approximation of the QTM when the dimension is greater than two or under a bias \cite{Machta1985}.


We showed that the integrability of observables with respect to the infinite invariant density
determines the power-law-decay exponent in the decrease of the ensemble average of the observables 
in the exponential and deterministic models. 
As a result, we found that the power-law exponent has a maximum at the transition point for both models.
%, which implies the existence of a parameter $\alpha$ that yields the most efficient cooling. 
%{\color{red}Interestingly, the most efficient cooling is achieved at the transition point of the integrability of the kinetic energy with respect to the infinite invariant density.} 
Furthermore, we found that 
the integrability of the observable with respect to the infinite invariant density 
plays an important role in characterizing the trajectory-to-trajectory 
fluctuations of the time averages in the three models.  When the observables are integrable, the distribution 
is universal and described by the Mittag-Leffler distribution. %, which is valid for all the models. 
On the other hand, the distribution differs for the exponential and the deterministic model when the observables are not integrable. 
Using the EB parameter, we numerically showed that the distribution in the HRW model agrees with that in the 
exponential model even when the observable is not integrable.

\section*{Acknowledgement}
T.A. was supported by JSPS Grant-in-Aid for Scientific Research (No.~C JP18K03468). 
The support of Israel Science Foundation's grant 1898/17 is acknowledged (EB).

\appendix


\section{Simulation algorithm}

For all the models, we generate trajectories starting with uniform initial conditions. In the HRW model, the momentum jumps are
generated by random variables following a Gaussian distribution with mean 0 and variance $\sigma^2$ by the Box-Muller's method \cite{box1958note}. 
When momentum becomes $p$ after a momentum jump, the waiting time is a random variable following an exponential distribution with rate $R(p)$. 
In numerical simulations, the waiting time $\tilde{\tau}$ is generated by $\tilde{\tau} = -\ln ({\bm X}) / R(p)$, where ${\bm X}$ is a uniform random variable on $[0,1]$.  
In the HRW model, we consider the reflecting boundary condition at $p=\pm p_{\max}$. In particular, when the momentum becomes $p>p_{\max}$,
  we have  $2p_{\max} - p$. If $p<-p_{\max}$, we have $-2p_{\max} - p$. 


For the exponential and deterministic models, 
updates of the momentum are independent of the previous momentum and generated by a uniform random variable on $[-p_{\rm trap}, 
p_{\rm trap}]$. The waiting time in the exponential model is generated in the same way as in the HRW model. 
The waiting time given $p$ in the deterministic model is determined by $\tilde{\tau} =1/R(p)$.

\begin{widetext}
\section{Asymptotic solution to the master equation for the HRW model}
Here, we show that the asymptotic solution of the master equation for the exponential model, i.e., Eq.~(\ref{propagator_asympt-exp2}), 
is also a solution of the master equation for the HRW model. 
Differentiating Eq.~(\ref{propagator_asympt-exp2}) with respect to $t$ gives 
\begin{equation}
\frac{\partial \rho(p,t)}{\partial t}  
\cong - R(p) \rho (p,t) + 
\frac{\sin (\pi \alpha^{-1}) t^{\alpha^{-1}-1 }}{2\pi c^{\alpha^{-1}} \Gamma (\alpha^{-1})} \int_0^1 du 
e^{- R(p) t(1-u)} u^{\alpha^{-1} -1}.
\end{equation}
The first term is the same as that of the master equation of the HRW model, i.e., Eq.~(\ref{Master1}). For $|p|\to 0$, it becomes 
\begin{equation}
\frac{\partial \rho(p,t)}{\partial t}  
\cong - R(p) \rho (p,t) + 
\frac{\sin (\pi \alpha^{-1}) t^{\alpha^{-1}-1 }}{2\pi c^{\alpha^{-1}} \Gamma (1 + \alpha^{-1})} .
\label{propagator-derivative-approx}
\end{equation}
Using Eq.~(\ref{propagator_asympt-exp2}), we approximately calculate the second term of the master equation of the HRW model, i.e., Eq.~(\ref{Master1}).
\begin{equation}
\int_{-p_{\max}}^{p_{\max}} dp^{\prime }\rho \left( p^{\prime },t\right) R(p^{\prime }) \tilde{G}(p|p^{\prime }) \cong 
\frac{\sin (\pi \alpha^{-1}) t^{\alpha^{-1} }}{2\pi c^{\alpha^{-1}} \Gamma (1 + \alpha^{-1})} 
\int_{-\infty}^{\infty} dp^{\prime } G(p-p^{\prime })  \int_0^1 du R(p^{\prime })  e^{-R(p^{\prime })  t(1-u)} u^{\alpha^{-1} -1},
\label{master1-sub}
\end{equation}
where we assumed $|p| , |p^{\prime }| \ll p_{\max}$ and used $\tilde{G}(p|p^{\prime }) \cong G(p-p^{\prime })$. Integrating Eq.~(\ref{master1-sub}) 
by parts, we have 
\begin{equation}
 \int_0^1 du R(p^{\prime })  e^{-R(p^{\prime })  t(1-u)} u^{\alpha^{-1} -1} \cong t^{-1} - (\alpha^{-1} -1) \int_0^1 du e^{-R(p^{\prime })  t(1-u)} u^{\alpha^{-1} -2}
 \sim t^{-1}.
\end{equation}
Thus, Eq.~(\ref{master1-sub}) becomes  
\begin{equation}
\int_{-p_{\max}}^{p_{\max}} dp^{\prime }\rho \left( p^{\prime },t\right) R(p^{\prime }) \tilde{G}(p|p^{\prime }) \cong 
\frac{\sin (\pi \alpha^{-1}) t^{\alpha^{-1}-1 }}{2\pi c^{\alpha^{-1}} \Gamma (1 + \alpha^{-1})} ,
\end{equation}
which is the same as the second term of Eq.~(\ref{propagator-derivative-approx}). Here, we confirmed that Eq.~(\ref{propagator_asympt-exp2}) is a solution 
to the master equation of the HRW model under the assumption of $|p| \to 0$. 
For the HRW model, momentum converges to $p=0$ almost surely in the long-time limit.
Therefore, Eq.~(\ref{propagator_asympt-exp2}) is a solution to the master equation of the HRW model in the long-time limit. 
\end{widetext} 


\section{Derivation of the $n$th moment of ${\mathcal S}(t)$}

Here, we derive the $n$th moments of ${\mathcal S}(t)$ for $\alpha<3$ in the exponential model. 
For $\alpha<3$, $\widehat{\phi}_{2}'(0,0) <\infty$. The leading term of the Laplace transform 
of the  $n$th moment is 
\begin{eqnarray}
\left. \frac{\partial^n \widehat{P} (u,s)}{\partial u^2} \right|_{u=0} \sim
 \frac{(-1)^n n!\widehat{\phi}_{2}'(0,s)^n}{[1- \widehat{\phi}_{2}(0,s)]^n}
\label{laplace-nth-moment-exp}
\end{eqnarray}
for $s\to 0$. It follows that the asymptotic behavior of $\langle {\mathcal S}(t)^n \rangle$ becomes 
\begin{equation}
\langle {\mathcal S}(t)^n \rangle \sim
 \frac{n!\{-\widehat{\phi}_{2}'(0,0)\}^n}{A_\alpha^n \Gamma(1+n/\alpha) } t^{\frac{n}{\alpha}}
 \label{nth-moment-X-a<3}
\end{equation}
for $t\to\infty$. In the long-time limit,  the $n$th moment of ${\mathcal S}(t)/\langle {\mathcal S}(t)\rangle$
converges to $n!\Gamma(1+1/\alpha)^n/\Gamma(1+n/\alpha)$ for all $n>0$. Therefore, the random variable defined by 
$\overline{\mathcal S} \equiv {\mathcal S}(t)/\langle {\mathcal S}(t)\rangle$ does not depend on time $t$ 
in the long-time limit and follows the Mittag--Leffler distribution with exponent $1/\alpha$, where 
the Laplace transform of  the random variable $M_\alpha$ following the Mittag--Leffler distribution with exponent $1/\alpha$ is given by 
\begin{equation}
\langle e^{-s  M_\alpha} \rangle = \sum_{k=0}^\infty \frac{\Gamma(1+1/\alpha)^n}{\Gamma(1+n/\alpha)} (-s)^n.
\end{equation} 
In real space, the PDF $f_\alpha(x)$ of $M_\alpha$ becomes
\begin{equation}
f_\alpha (x) = - \frac{1}{\pi \alpha} \sum_{k=1}^\infty \frac{\Gamma (k \alpha + 1)}{k!} x^{k-1} \sin (\pi k \alpha).
\end{equation}

%\bibliography{qtm}
%merlin.mbs apsrev4-1.bst 2010-07-25 4.21a (PWD, AO, DPC) hacked
%Control: key (0)
%Control: author (8) initials jnrlst
%Control: editor formatted (1) identically to author
%Control: production of article title (-1) disabled
%Control: page (0) single
%Control: year (1) truncated
%Control: production of eprint (0) enabled
\begin{thebibliography}{58}%
\makeatletter
\providecommand \@ifxundefined [1]{%
 \@ifx{#1\undefined}
}%
\providecommand \@ifnum [1]{%
 \ifnum #1\expandafter \@firstoftwo
 \else \expandafter \@secondoftwo
 \fi
}%
\providecommand \@ifx [1]{%
 \ifx #1\expandafter \@firstoftwo
 \else \expandafter \@secondoftwo
 \fi
}%
\providecommand \natexlab [1]{#1}%
\providecommand \enquote  [1]{``#1''}%
\providecommand \bibnamefont  [1]{#1}%
\providecommand \bibfnamefont [1]{#1}%
\providecommand \citenamefont [1]{#1}%
\providecommand \href@noop [0]{\@secondoftwo}%
\providecommand \href [0]{\begingroup \@sanitize@url \@href}%
\providecommand \@href[1]{\@@startlink{#1}\@@href}%
\providecommand \@@href[1]{\endgroup#1\@@endlink}%
\providecommand \@sanitize@url [0]{\catcode `\\12\catcode `\$12\catcode
  `\&12\catcode `\#12\catcode `\^12\catcode `\_12\catcode `\%12\relax}%
\providecommand \@@startlink[1]{}%
\providecommand \@@endlink[0]{}%
\providecommand \url  [0]{\begingroup\@sanitize@url \@url }%
\providecommand \@url [1]{\endgroup\@href {#1}{\urlprefix }}%
\providecommand \urlprefix  [0]{URL }%
\providecommand \Eprint [0]{\href }%
\providecommand \doibase [0]{http://dx.doi.org/}%
\providecommand \selectlanguage [0]{\@gobble}%
\providecommand \bibinfo  [0]{\@secondoftwo}%
\providecommand \bibfield  [0]{\@secondoftwo}%
\providecommand \translation [1]{[#1]}%
\providecommand \BibitemOpen [0]{}%
\providecommand \bibitemStop [0]{}%
\providecommand \bibitemNoStop [0]{.\EOS\space}%
\providecommand \EOS [0]{\spacefactor3000\relax}%
\providecommand \BibitemShut  [1]{\csname bibitem#1\endcsname}%
\let\auto@bib@innerbib\@empty
%</preamble>
\bibitem [{\citenamefont {Van~Kampen}(1992)}]{van1992stochastic}%
  \BibitemOpen
  \bibfield  {author} {\bibinfo {author} {\bibfnamefont {N.~G.}\ \bibnamefont
  {Van~Kampen}},\ }\href@noop {} {\emph {\bibinfo {title} {Stochastic processes
  in physics and chemistry}}}\ (\bibinfo  {publisher} {Elsevier, New York},\
  \bibinfo {year} {1992})\BibitemShut {NoStop}%
\bibitem [{\citenamefont {Kessler}\ and\ \citenamefont
  {Barkai}(2010)}]{Kessler2010}%
  \BibitemOpen
  \bibfield  {author} {\bibinfo {author} {\bibfnamefont {D.~A.}\ \bibnamefont
  {Kessler}}\ and\ \bibinfo {author} {\bibfnamefont {E.}~\bibnamefont
  {Barkai}},\ }\href {\doibase 10.1103/PhysRevLett.105.120602} {\bibfield
  {journal} {\bibinfo  {journal} {Phys. Rev. Lett.}\ }\textbf {\bibinfo
  {volume} {105}},\ \bibinfo {pages} {120602} (\bibinfo {year}
  {2010})}\BibitemShut {NoStop}%
\bibitem [{\citenamefont {Lutz}\ and\ \citenamefont
  {Renzoni}(2013)}]{lutz2013}%
  \BibitemOpen
  \bibfield  {author} {\bibinfo {author} {\bibfnamefont {E.}~\bibnamefont
  {Lutz}}\ and\ \bibinfo {author} {\bibfnamefont {F.}~\bibnamefont {Renzoni}},\
  }\href@noop {} {\bibfield  {journal} {\bibinfo  {journal} {Nat. Phys.}\
  }\textbf {\bibinfo {volume} {9}},\ \bibinfo {pages} {615} (\bibinfo {year}
  {2013})}\BibitemShut {NoStop}%
\bibitem [{\citenamefont {Rebenshtok}\ \emph {et~al.}(2014)\citenamefont
  {Rebenshtok}, \citenamefont {Denisov}, \citenamefont {H\"anggi},\ and\
  \citenamefont {Barkai}}]{Rebenshtok2014}%
  \BibitemOpen
  \bibfield  {author} {\bibinfo {author} {\bibfnamefont {A.}~\bibnamefont
  {Rebenshtok}}, \bibinfo {author} {\bibfnamefont {S.}~\bibnamefont {Denisov}},
  \bibinfo {author} {\bibfnamefont {P.}~\bibnamefont {H\"anggi}}, \ and\
  \bibinfo {author} {\bibfnamefont {E.}~\bibnamefont {Barkai}},\ }\href
  {\doibase 10.1103/PhysRevLett.112.110601} {\bibfield  {journal} {\bibinfo
  {journal} {Phys. Rev. Lett.}\ }\textbf {\bibinfo {volume} {112}},\ \bibinfo
  {pages} {110601} (\bibinfo {year} {2014})}\BibitemShut {NoStop}%
\bibitem [{\citenamefont {Holz}\ \emph {et~al.}(2015)\citenamefont {Holz},
  \citenamefont {Dechant},\ and\ \citenamefont {Lutz}}]{Holz2015}%
  \BibitemOpen
  \bibfield  {author} {\bibinfo {author} {\bibfnamefont {P.~C.}\ \bibnamefont
  {Holz}}, \bibinfo {author} {\bibfnamefont {A.}~\bibnamefont {Dechant}}, \
  and\ \bibinfo {author} {\bibfnamefont {E.}~\bibnamefont {Lutz}},\ }\href
  {http://stacks.iop.org/0295-5075/109/i=2/a=23001} {\bibfield  {journal}
  {\bibinfo  {journal} {Europhys. Lett.}\ }\textbf {\bibinfo {volume} {109}},\
  \bibinfo {pages} {23001} (\bibinfo {year} {2015})}\BibitemShut {NoStop}%
\bibitem [{\citenamefont {Leibovich}\ and\ \citenamefont
  {Barkai}(2019)}]{Leibovich2019}%
  \BibitemOpen
  \bibfield  {author} {\bibinfo {author} {\bibfnamefont {N.}~\bibnamefont
  {Leibovich}}\ and\ \bibinfo {author} {\bibfnamefont {E.}~\bibnamefont
  {Barkai}},\ }\href {\doibase 10.1103/PhysRevE.99.042138} {\bibfield
  {journal} {\bibinfo  {journal} {Phys. Rev. E}\ }\textbf {\bibinfo {volume}
  {99}},\ \bibinfo {pages} {042138} (\bibinfo {year} {2019})}\BibitemShut
  {NoStop}%
\bibitem [{\citenamefont {Aghion}\ \emph {et~al.}(2019)\citenamefont {Aghion},
  \citenamefont {Kessler},\ and\ \citenamefont {Barkai}}]{Aghion2019}%
  \BibitemOpen
  \bibfield  {author} {\bibinfo {author} {\bibfnamefont {E.}~\bibnamefont
  {Aghion}}, \bibinfo {author} {\bibfnamefont {D.~A.}\ \bibnamefont {Kessler}},
  \ and\ \bibinfo {author} {\bibfnamefont {E.}~\bibnamefont {Barkai}},\ }\href
  {\doibase 10.1103/PhysRevLett.122.010601} {\bibfield  {journal} {\bibinfo
  {journal} {Phys. Rev. Lett.}\ }\textbf {\bibinfo {volume} {122}},\ \bibinfo
  {pages} {010601} (\bibinfo {year} {2019})}\BibitemShut {NoStop}%
\bibitem [{\citenamefont {Aghion}\ \emph {et~al.}(2020)\citenamefont {Aghion},
  \citenamefont {Kessler},\ and\ \citenamefont {Barkai}}]{aghion2020infinite}%
  \BibitemOpen
  \bibfield  {author} {\bibinfo {author} {\bibfnamefont {E.}~\bibnamefont
  {Aghion}}, \bibinfo {author} {\bibfnamefont {D.~A.}\ \bibnamefont {Kessler}},
  \ and\ \bibinfo {author} {\bibfnamefont {E.}~\bibnamefont {Barkai}},\
  }\href@noop {} {\bibfield  {journal} {\bibinfo  {journal} {Chaos, Solitons \&
  Fractals}\ }\textbf {\bibinfo {volume} {138}},\ \bibinfo {pages} {109890}
  (\bibinfo {year} {2020})}\BibitemShut {NoStop}%
\bibitem [{\citenamefont {Aghion}\ \emph {et~al.}(2021)\citenamefont {Aghion},
  \citenamefont {Meyer}, \citenamefont {Adlakha}, \citenamefont {Kantz},\ and\
  \citenamefont {Bassler}}]{aghion2021moses}%
  \BibitemOpen
  \bibfield  {author} {\bibinfo {author} {\bibfnamefont {E.}~\bibnamefont
  {Aghion}}, \bibinfo {author} {\bibfnamefont {P.~G.}\ \bibnamefont {Meyer}},
  \bibinfo {author} {\bibfnamefont {V.}~\bibnamefont {Adlakha}}, \bibinfo
  {author} {\bibfnamefont {H.}~\bibnamefont {Kantz}}, \ and\ \bibinfo {author}
  {\bibfnamefont {K.~E.}\ \bibnamefont {Bassler}},\ }\href@noop {} {\bibfield
  {journal} {\bibinfo  {journal} {New J. Phys.}\ }\textbf {\bibinfo {volume}
  {23}},\ \bibinfo {pages} {023002} (\bibinfo {year} {2021})}\BibitemShut
  {NoStop}%
\bibitem [{\citenamefont {Strei\ss{}nig}\ and\ \citenamefont
  {Kantz}(2021)}]{Streissnin2021}%
  \BibitemOpen
  \bibfield  {author} {\bibinfo {author} {\bibfnamefont {C.}~\bibnamefont
  {Strei\ss{}nig}}\ and\ \bibinfo {author} {\bibfnamefont {H.}~\bibnamefont
  {Kantz}},\ }\href {\doibase 10.1103/PhysRevResearch.3.013115} {\bibfield
  {journal} {\bibinfo  {journal} {Phys. Rev. Research}\ }\textbf {\bibinfo
  {volume} {3}},\ \bibinfo {pages} {013115} (\bibinfo {year}
  {2021})}\BibitemShut {NoStop}%
\bibitem [{\citenamefont {Thaler}(1983)}]{Thaler1983}%
  \BibitemOpen
  \bibfield  {author} {\bibinfo {author} {\bibfnamefont {M.}~\bibnamefont
  {Thaler}},\ }\href@noop {} {\bibfield  {journal} {\bibinfo  {journal} {Isr.
  J. Math.}\ }\textbf {\bibinfo {volume} {46}},\ \bibinfo {pages} {67}
  (\bibinfo {year} {1983})}\BibitemShut {NoStop}%
\bibitem [{\citenamefont {Aaronson}(1997)}]{Aaronson1997}%
  \BibitemOpen
  \bibfield  {author} {\bibinfo {author} {\bibfnamefont {J.}~\bibnamefont
  {Aaronson}},\ }\href@noop {} {\emph {\bibinfo {title} {An Introduction to
  Infinite Ergodic Theory}}}\ (\bibinfo  {publisher} {American Mathematical
  Society},\ \bibinfo {address} {Providence},\ \bibinfo {year}
  {1997})\BibitemShut {NoStop}%
\bibitem [{\citenamefont {Inoue}(1997)}]{inoue1997ratio}%
  \BibitemOpen
  \bibfield  {author} {\bibinfo {author} {\bibfnamefont {T.}~\bibnamefont
  {Inoue}},\ }\href@noop {} {\bibfield  {journal} {\bibinfo  {journal} {Ergod.
  Theory Dyn. Syst.}\ }\textbf {\bibinfo {volume} {17}},\ \bibinfo {pages}
  {625} (\bibinfo {year} {1997})}\BibitemShut {NoStop}%
\bibitem [{\citenamefont {Thaler}(1998)}]{Thaler1998}%
  \BibitemOpen
  \bibfield  {author} {\bibinfo {author} {\bibfnamefont {M.}~\bibnamefont
  {Thaler}},\ }\href@noop {} {\bibfield  {journal} {\bibinfo  {journal} {Trans.
  Am. Math. Soc.}\ }\textbf {\bibinfo {volume} {350}},\ \bibinfo {pages} {4593}
  (\bibinfo {year} {1998})}\BibitemShut {NoStop}%
\bibitem [{\citenamefont {Thaler}(2002)}]{Thaler2002}%
  \BibitemOpen
  \bibfield  {author} {\bibinfo {author} {\bibfnamefont {M.}~\bibnamefont
  {Thaler}},\ }\href@noop {} {\bibfield  {journal} {\bibinfo  {journal} {Ergod.
  Theory Dyn. Syst.}\ }\textbf {\bibinfo {volume} {22}},\ \bibinfo {pages}
  {1289} (\bibinfo {year} {2002})}\BibitemShut {NoStop}%
\bibitem [{\citenamefont {Inoue}(2004)}]{inoue2004ergodic}%
  \BibitemOpen
  \bibfield  {author} {\bibinfo {author} {\bibfnamefont {T.}~\bibnamefont
  {Inoue}},\ }\href@noop {} {\bibfield  {journal} {\bibinfo  {journal} {Ergod.
  Theory Dyn. Syst.}\ }\textbf {\bibinfo {volume} {24}},\ \bibinfo {pages}
  {525} (\bibinfo {year} {2004})}\BibitemShut {NoStop}%
\bibitem [{\citenamefont {Akimoto}(2008)}]{Akimoto2008}%
  \BibitemOpen
  \bibfield  {author} {\bibinfo {author} {\bibfnamefont {T.}~\bibnamefont
  {Akimoto}},\ }\href@noop {} {\bibfield  {journal} {\bibinfo  {journal} {J.
  Stat. Phys.}\ }\textbf {\bibinfo {volume} {132}},\ \bibinfo {pages} {171}
  (\bibinfo {year} {2008})}\BibitemShut {NoStop}%
\bibitem [{\citenamefont {Akimoto}\ \emph {et~al.}(2015)\citenamefont
  {Akimoto}, \citenamefont {Shinkai},\ and\ \citenamefont
  {Aizawa}}]{Akimoto2015}%
  \BibitemOpen
  \bibfield  {author} {\bibinfo {author} {\bibfnamefont {T.}~\bibnamefont
  {Akimoto}}, \bibinfo {author} {\bibfnamefont {S.}~\bibnamefont {Shinkai}}, \
  and\ \bibinfo {author} {\bibfnamefont {Y.}~\bibnamefont {Aizawa}},\
  }\href@noop {} {\bibfield  {journal} {\bibinfo  {journal} {J. Stat. Phys.}\
  }\textbf {\bibinfo {volume} {158}},\ \bibinfo {pages} {476} (\bibinfo {year}
  {2015})}\BibitemShut {NoStop}%
\bibitem [{\citenamefont {Sera}\ and\ \citenamefont {Yano}(2019)}]{Sera2019}%
  \BibitemOpen
  \bibfield  {author} {\bibinfo {author} {\bibfnamefont {T.}~\bibnamefont
  {Sera}}\ and\ \bibinfo {author} {\bibfnamefont {K.}~\bibnamefont {Yano}},\
  }\href {\doibase 10.1090/tran/7755} {\bibfield  {journal} {\bibinfo
  {journal} {Trans. Amer. Math. Soc.}\ }\textbf {\bibinfo {volume} {372}},\
  \bibinfo {pages} {3191} (\bibinfo {year} {2019})}\BibitemShut {NoStop}%
\bibitem [{\citenamefont {Sera}(2020)}]{Sera2020}%
  \BibitemOpen
  \bibfield  {author} {\bibinfo {author} {\bibfnamefont {T.}~\bibnamefont
  {Sera}},\ }\href {\doibase 10.1088/1361-6544/ab5ceb} {\bibfield  {journal}
  {\bibinfo  {journal} {Nonlinearity}\ }\textbf {\bibinfo {volume} {33}},\
  \bibinfo {pages} {1183} (\bibinfo {year} {2020})}\BibitemShut {NoStop}%
\bibitem [{\citenamefont {Aaronson}(1981)}]{Aaronson1981}%
  \BibitemOpen
  \bibfield  {author} {\bibinfo {author} {\bibfnamefont {J.}~\bibnamefont
  {Aaronson}},\ }\href@noop {} {\bibfield  {journal} {\bibinfo  {journal} {J.
  D'Analyse Math.}\ }\textbf {\bibinfo {volume} {39}},\ \bibinfo {pages} {203}
  (\bibinfo {year} {1981})}\BibitemShut {NoStop}%
\bibitem [{\citenamefont {Akimoto}\ and\ \citenamefont
  {Miyaguchi}(2010)}]{Akimoto2010}%
  \BibitemOpen
  \bibfield  {author} {\bibinfo {author} {\bibfnamefont {T.}~\bibnamefont
  {Akimoto}}\ and\ \bibinfo {author} {\bibfnamefont {T.}~\bibnamefont
  {Miyaguchi}},\ }\href@noop {} {\bibfield  {journal} {\bibinfo  {journal}
  {Phys. Rev. E}\ }\textbf {\bibinfo {volume} {82}},\ \bibinfo {pages}
  {030102(R)} (\bibinfo {year} {2010})}\BibitemShut {NoStop}%
\bibitem [{\citenamefont {Akimoto}(2012)}]{Akimoto2012}%
  \BibitemOpen
  \bibfield  {author} {\bibinfo {author} {\bibfnamefont {T.}~\bibnamefont
  {Akimoto}},\ }\href {\doibase 10.1103/PhysRevLett.108.164101} {\bibfield
  {journal} {\bibinfo  {journal} {Phys. Rev. Lett.}\ }\textbf {\bibinfo
  {volume} {108}},\ \bibinfo {pages} {164101} (\bibinfo {year}
  {2012})}\BibitemShut {NoStop}%
\bibitem [{\citenamefont {Brokmann}\ \emph {et~al.}(2003)\citenamefont
  {Brokmann}, \citenamefont {Hermier}, \citenamefont {Messin}, \citenamefont
  {Desbiolles}, \citenamefont {Bouchaud},\ and\ \citenamefont
  {Dahan}}]{Brok2003}%
  \BibitemOpen
  \bibfield  {author} {\bibinfo {author} {\bibfnamefont {X.}~\bibnamefont
  {Brokmann}}, \bibinfo {author} {\bibfnamefont {J.-P.}\ \bibnamefont
  {Hermier}}, \bibinfo {author} {\bibfnamefont {G.}~\bibnamefont {Messin}},
  \bibinfo {author} {\bibfnamefont {P.}~\bibnamefont {Desbiolles}}, \bibinfo
  {author} {\bibfnamefont {J.-P.}\ \bibnamefont {Bouchaud}}, \ and\ \bibinfo
  {author} {\bibfnamefont {M.}~\bibnamefont {Dahan}},\ }\href@noop {}
  {\bibfield  {journal} {\bibinfo  {journal} {Phys. Rev. Lett.}\ }\textbf
  {\bibinfo {volume} {90}},\ \bibinfo {pages} {120601} (\bibinfo {year}
  {2003})}\BibitemShut {NoStop}%
\bibitem [{\citenamefont {Stefani}\ \emph {et~al.}(2009)\citenamefont
  {Stefani}, \citenamefont {Hoogenboom},\ and\ \citenamefont
  {Barkai}}]{stefani2009}%
  \BibitemOpen
  \bibfield  {author} {\bibinfo {author} {\bibfnamefont {F.~D.}\ \bibnamefont
  {Stefani}}, \bibinfo {author} {\bibfnamefont {J.~P.}\ \bibnamefont
  {Hoogenboom}}, \ and\ \bibinfo {author} {\bibfnamefont {E.}~\bibnamefont
  {Barkai}},\ }\href@noop {} {\bibfield  {journal} {\bibinfo  {journal} {Phys.
  today}\ }\textbf {\bibinfo {volume} {62}},\ \bibinfo {pages} {34} (\bibinfo
  {year} {2009})}\BibitemShut {NoStop}%
\bibitem [{\citenamefont {Golding}\ and\ \citenamefont
  {Cox}(2006)}]{Golding2006}%
  \BibitemOpen
  \bibfield  {author} {\bibinfo {author} {\bibfnamefont {I.}~\bibnamefont
  {Golding}}\ and\ \bibinfo {author} {\bibfnamefont {E.~C.}\ \bibnamefont
  {Cox}},\ }\href@noop {} {\bibfield  {journal} {\bibinfo  {journal} {Phys.
  Rev. Lett.}\ }\textbf {\bibinfo {volume} {96}},\ \bibinfo {pages} {098102}
  (\bibinfo {year} {2006})}\BibitemShut {NoStop}%
\bibitem [{\citenamefont {Weigel}\ \emph {et~al.}(2011)\citenamefont {Weigel},
  \citenamefont {Simon}, \citenamefont {Tamkun},\ and\ \citenamefont
  {Krapf}}]{Weigel2011}%
  \BibitemOpen
  \bibfield  {author} {\bibinfo {author} {\bibfnamefont {A.}~\bibnamefont
  {Weigel}}, \bibinfo {author} {\bibfnamefont {B.}~\bibnamefont {Simon}},
  \bibinfo {author} {\bibfnamefont {M.}~\bibnamefont {Tamkun}}, \ and\ \bibinfo
  {author} {\bibfnamefont {D.}~\bibnamefont {Krapf}},\ }\href@noop {}
  {\bibfield  {journal} {\bibinfo  {journal} {Proc. Natl. Acad. Sci. USA}\
  }\textbf {\bibinfo {volume} {108}},\ \bibinfo {pages} {6438} (\bibinfo {year}
  {2011})}\BibitemShut {NoStop}%
\bibitem [{\citenamefont {Jeon}\ \emph {et~al.}(2011)\citenamefont {Jeon},
  \citenamefont {Tejedor}, \citenamefont {Burov}, \citenamefont {Barkai},
  \citenamefont {Selhuber-Unkel}, \citenamefont {Berg-S\o{}rensen},
  \citenamefont {Oddershede},\ and\ \citenamefont {Metzler}}]{Jeon2011}%
  \BibitemOpen
  \bibfield  {author} {\bibinfo {author} {\bibfnamefont {J.-H.}\ \bibnamefont
  {Jeon}}, \bibinfo {author} {\bibfnamefont {V.}~\bibnamefont {Tejedor}},
  \bibinfo {author} {\bibfnamefont {S.}~\bibnamefont {Burov}}, \bibinfo
  {author} {\bibfnamefont {E.}~\bibnamefont {Barkai}}, \bibinfo {author}
  {\bibfnamefont {C.}~\bibnamefont {Selhuber-Unkel}}, \bibinfo {author}
  {\bibfnamefont {K.}~\bibnamefont {Berg-S\o{}rensen}}, \bibinfo {author}
  {\bibfnamefont {L.}~\bibnamefont {Oddershede}}, \ and\ \bibinfo {author}
  {\bibfnamefont {R.}~\bibnamefont {Metzler}},\ }\href {\doibase
  10.1103/PhysRevLett.106.048103} {\bibfield  {journal} {\bibinfo  {journal}
  {Phys. Rev. Lett.}\ }\textbf {\bibinfo {volume} {106}},\ \bibinfo {pages}
  {048103} (\bibinfo {year} {2011})}\BibitemShut {NoStop}%
\bibitem [{\citenamefont {H{\"o}fling}\ and\ \citenamefont
  {Franosch}(2013)}]{Hofling2013}%
  \BibitemOpen
  \bibfield  {author} {\bibinfo {author} {\bibfnamefont {F.}~\bibnamefont
  {H{\"o}fling}}\ and\ \bibinfo {author} {\bibfnamefont {T.}~\bibnamefont
  {Franosch}},\ }\href@noop {} {\bibfield  {journal} {\bibinfo  {journal} {Rep.
  Prog. Phys.}\ }\textbf {\bibinfo {volume} {76}},\ \bibinfo {pages} {046602}
  (\bibinfo {year} {2013})}\BibitemShut {NoStop}%
\bibitem [{\citenamefont {Manzo}\ \emph {et~al.}(2015)\citenamefont {Manzo},
  \citenamefont {Torreno-Pina}, \citenamefont {Massignan}, \citenamefont
  {Lapeyre~Jr}, \citenamefont {Lewenstein},\ and\ \citenamefont
  {Parajo}}]{Manzo2015}%
  \BibitemOpen
  \bibfield  {author} {\bibinfo {author} {\bibfnamefont {C.}~\bibnamefont
  {Manzo}}, \bibinfo {author} {\bibfnamefont {J.~A.}\ \bibnamefont
  {Torreno-Pina}}, \bibinfo {author} {\bibfnamefont {P.}~\bibnamefont
  {Massignan}}, \bibinfo {author} {\bibfnamefont {G.~J.}\ \bibnamefont
  {Lapeyre~Jr}}, \bibinfo {author} {\bibfnamefont {M.}~\bibnamefont
  {Lewenstein}}, \ and\ \bibinfo {author} {\bibfnamefont {M.~F.~G.}\
  \bibnamefont {Parajo}},\ }\href@noop {} {\bibfield  {journal} {\bibinfo
  {journal} {Phys. Rev. X}\ }\textbf {\bibinfo {volume} {5}},\ \bibinfo {pages}
  {011021} (\bibinfo {year} {2015})}\BibitemShut {NoStop}%
\bibitem [{\citenamefont {Takeuchi}\ and\ \citenamefont
  {Akimoto}(2016)}]{takeuchi2016}%
  \BibitemOpen
  \bibfield  {author} {\bibinfo {author} {\bibfnamefont {K.~A.}\ \bibnamefont
  {Takeuchi}}\ and\ \bibinfo {author} {\bibfnamefont {T.}~\bibnamefont
  {Akimoto}},\ }\href@noop {} {\bibfield  {journal} {\bibinfo  {journal} {J.
  Stat. Phys.}\ }\textbf {\bibinfo {volume} {164}},\ \bibinfo {pages} {1167}
  (\bibinfo {year} {2016})}\BibitemShut {NoStop}%
\bibitem [{\citenamefont {Cohen-Tannoudji}\ and\ \citenamefont
  {Phillips}(1990)}]{cohen1990new}%
  \BibitemOpen
  \bibfield  {author} {\bibinfo {author} {\bibfnamefont {C.}~\bibnamefont
  {Cohen-Tannoudji}}\ and\ \bibinfo {author} {\bibfnamefont {W.~D.}\
  \bibnamefont {Phillips}},\ }\href@noop {} {\bibfield  {journal} {\bibinfo
  {journal} {Phys. Today}\ }\textbf {\bibinfo {volume} {43}},\ \bibinfo {pages}
  {33} (\bibinfo {year} {1990})}\BibitemShut {NoStop}%
\bibitem [{\citenamefont {Bardou}\ \emph {et~al.}(1994)\citenamefont {Bardou},
  \citenamefont {Bouchaud}, \citenamefont {Emile}, \citenamefont {Aspect},\
  and\ \citenamefont {Cohen-Tannoudji}}]{Bardou1994}%
  \BibitemOpen
  \bibfield  {author} {\bibinfo {author} {\bibfnamefont {F.}~\bibnamefont
  {Bardou}}, \bibinfo {author} {\bibfnamefont {J.~P.}\ \bibnamefont
  {Bouchaud}}, \bibinfo {author} {\bibfnamefont {O.}~\bibnamefont {Emile}},
  \bibinfo {author} {\bibfnamefont {A.}~\bibnamefont {Aspect}}, \ and\ \bibinfo
  {author} {\bibfnamefont {C.}~\bibnamefont {Cohen-Tannoudji}},\ }\href
  {\doibase 10.1103/PhysRevLett.72.203} {\bibfield  {journal} {\bibinfo
  {journal} {Phys. Rev. Lett.}\ }\textbf {\bibinfo {volume} {72}},\ \bibinfo
  {pages} {203} (\bibinfo {year} {1994})}\BibitemShut {NoStop}%
\bibitem [{\citenamefont {Barkai}\ \emph {et~al.}(2021)\citenamefont {Barkai},
  \citenamefont {Radons},\ and\ \citenamefont {Akimoto}}]{Barkai2021}%
  \BibitemOpen
  \bibfield  {author} {\bibinfo {author} {\bibfnamefont {E.}~\bibnamefont
  {Barkai}}, \bibinfo {author} {\bibfnamefont {G.}~\bibnamefont {Radons}}, \
  and\ \bibinfo {author} {\bibfnamefont {T.}~\bibnamefont {Akimoto}},\ }\href
  {\doibase 10.1103/PhysRevLett.127.140605} {\bibfield  {journal} {\bibinfo
  {journal} {Phys. Rev. Lett.}\ }\textbf {\bibinfo {volume} {127}},\ \bibinfo
  {pages} {140605} (\bibinfo {year} {2021})}\BibitemShut {NoStop}%
\bibitem [{\citenamefont {Barkai}\ \emph {et~al.}(2022)\citenamefont {Barkai},
  \citenamefont {Radons},\ and\ \citenamefont {Akimoto}}]{barkai2022gas}%
  \BibitemOpen
  \bibfield  {author} {\bibinfo {author} {\bibfnamefont {E.}~\bibnamefont
  {Barkai}}, \bibinfo {author} {\bibfnamefont {G.}~\bibnamefont {Radons}}, \
  and\ \bibinfo {author} {\bibfnamefont {T.}~\bibnamefont {Akimoto}},\
  }\href@noop {} {\bibfield  {journal} {\bibinfo  {journal} {J. Chem. Phys.}\
  }\textbf {\bibinfo {volume} {156}},\ \bibinfo {pages} {044118} (\bibinfo
  {year} {2022})}\BibitemShut {NoStop}%
\bibitem [{\citenamefont {Bardou}\ \emph {et~al.}(2002)\citenamefont {Bardou},
  \citenamefont {Bouchaud}, \citenamefont {Aspect},\ and\ \citenamefont
  {Cohen-Tannoudji}}]{Bardou2002}%
  \BibitemOpen
  \bibfield  {author} {\bibinfo {author} {\bibfnamefont {F.}~\bibnamefont
  {Bardou}}, \bibinfo {author} {\bibfnamefont {J.-P.}\ \bibnamefont
  {Bouchaud}}, \bibinfo {author} {\bibfnamefont {A.}~\bibnamefont {Aspect}}, \
  and\ \bibinfo {author} {\bibfnamefont {C.}~\bibnamefont {Cohen-Tannoudji}},\
  }\href@noop {} {\emph {\bibinfo {title} {Levy statistics and laser cooling:
  how rare events bring atoms to rest}}}\ (\bibinfo  {publisher} {Cambridge
  University Press},\ \bibinfo {year} {2002})\BibitemShut {NoStop}%
\bibitem [{\citenamefont {Aspect}\ \emph {et~al.}(1988)\citenamefont {Aspect},
  \citenamefont {Arimondo}, \citenamefont {Kaiser}, \citenamefont
  {Vansteenkiste},\ and\ \citenamefont {Cohen-Tannoudji}}]{AAK88}%
  \BibitemOpen
  \bibfield  {author} {\bibinfo {author} {\bibfnamefont {A.}~\bibnamefont
  {Aspect}}, \bibinfo {author} {\bibfnamefont {E.}~\bibnamefont {Arimondo}},
  \bibinfo {author} {\bibfnamefont {R.}~\bibnamefont {Kaiser}}, \bibinfo
  {author} {\bibfnamefont {N.}~\bibnamefont {Vansteenkiste}}, \ and\ \bibinfo
  {author} {\bibfnamefont {C.}~\bibnamefont {Cohen-Tannoudji}},\ }\href
  {\doibase 10.1103/PhysRevLett.61.826} {\bibfield  {journal} {\bibinfo
  {journal} {Phys. Rev. Lett.}\ }\textbf {\bibinfo {volume} {61}},\ \bibinfo
  {pages} {826} (\bibinfo {year} {1988})}\BibitemShut {NoStop}%
\bibitem [{\citenamefont {Kasevich}\ and\ \citenamefont {Chu}(1992)}]{KaC92}%
  \BibitemOpen
  \bibfield  {author} {\bibinfo {author} {\bibfnamefont {M.}~\bibnamefont
  {Kasevich}}\ and\ \bibinfo {author} {\bibfnamefont {S.}~\bibnamefont {Chu}},\
  }\href {\doibase 10.1103/PhysRevLett.69.1741} {\bibfield  {journal} {\bibinfo
   {journal} {Phys. Rev. Lett.}\ }\textbf {\bibinfo {volume} {69}},\ \bibinfo
  {pages} {1741} (\bibinfo {year} {1992})}\BibitemShut {NoStop}%
\bibitem [{\citenamefont {Saubam\'ea}\ \emph {et~al.}(1999)\citenamefont
  {Saubam\'ea}, \citenamefont {Leduc},\ and\ \citenamefont
  {Cohen-Tannoudji}}]{Saubamea1999}%
  \BibitemOpen
  \bibfield  {author} {\bibinfo {author} {\bibfnamefont {B.}~\bibnamefont
  {Saubam\'ea}}, \bibinfo {author} {\bibfnamefont {M.}~\bibnamefont {Leduc}}, \
  and\ \bibinfo {author} {\bibfnamefont {C.}~\bibnamefont {Cohen-Tannoudji}},\
  }\href {\doibase 10.1103/PhysRevLett.83.3796} {\bibfield  {journal} {\bibinfo
   {journal} {Phys. Rev. Lett.}\ }\textbf {\bibinfo {volume} {83}},\ \bibinfo
  {pages} {3796} (\bibinfo {year} {1999})}\BibitemShut {NoStop}%
\bibitem [{\citenamefont {Bertin}\ and\ \citenamefont
  {Bardou}(2008)}]{bertin2008}%
  \BibitemOpen
  \bibfield  {author} {\bibinfo {author} {\bibfnamefont {E.}~\bibnamefont
  {Bertin}}\ and\ \bibinfo {author} {\bibfnamefont {F.}~\bibnamefont
  {Bardou}},\ }\href@noop {} {\bibfield  {journal} {\bibinfo  {journal} {Am. J.
  Phys.}\ }\textbf {\bibinfo {volume} {76}},\ \bibinfo {pages} {630} (\bibinfo
  {year} {2008})}\BibitemShut {NoStop}%
\bibitem [{\citenamefont {Reichel}\ \emph {et~al.}(1995)\citenamefont
  {Reichel}, \citenamefont {Bardou}, \citenamefont {Dahan}, \citenamefont
  {Peik}, \citenamefont {Rand}, \citenamefont {Salomon},\ and\ \citenamefont
  {Cohen-Tannoudji}}]{RBB95}%
  \BibitemOpen
  \bibfield  {author} {\bibinfo {author} {\bibfnamefont {J.}~\bibnamefont
  {Reichel}}, \bibinfo {author} {\bibfnamefont {F.}~\bibnamefont {Bardou}},
  \bibinfo {author} {\bibfnamefont {M.~B.}\ \bibnamefont {Dahan}}, \bibinfo
  {author} {\bibfnamefont {E.}~\bibnamefont {Peik}}, \bibinfo {author}
  {\bibfnamefont {S.}~\bibnamefont {Rand}}, \bibinfo {author} {\bibfnamefont
  {C.}~\bibnamefont {Salomon}}, \ and\ \bibinfo {author} {\bibfnamefont
  {C.}~\bibnamefont {Cohen-Tannoudji}},\ }\href {\doibase
  10.1103/PhysRevLett.75.4575} {\bibfield  {journal} {\bibinfo  {journal}
  {Phys. Rev. Lett.}\ }\textbf {\bibinfo {volume} {75}},\ \bibinfo {pages}
  {4575} (\bibinfo {year} {1995})}\BibitemShut {NoStop}%
\bibitem [{\citenamefont {Akimoto}\ \emph {et~al.}(2020)\citenamefont
  {Akimoto}, \citenamefont {Barkai},\ and\ \citenamefont
  {Radons}}]{Akimoto2020}%
  \BibitemOpen
  \bibfield  {author} {\bibinfo {author} {\bibfnamefont {T.}~\bibnamefont
  {Akimoto}}, \bibinfo {author} {\bibfnamefont {E.}~\bibnamefont {Barkai}}, \
  and\ \bibinfo {author} {\bibfnamefont {G.}~\bibnamefont {Radons}},\ }\href
  {\doibase 10.1103/PhysRevE.101.052112} {\bibfield  {journal} {\bibinfo
  {journal} {Phys. Rev. E}\ }\textbf {\bibinfo {volume} {101}},\ \bibinfo
  {pages} {052112} (\bibinfo {year} {2020})}\BibitemShut {NoStop}%
\bibitem [{\citenamefont {Cox}(1962)}]{Cox1962}%
  \BibitemOpen
  \bibfield  {author} {\bibinfo {author} {\bibfnamefont {D.~R.}\ \bibnamefont
  {Cox}},\ }\href@noop {} {\emph {\bibinfo {title} {Renewal theory}}}\
  (\bibinfo  {publisher} {Methuen},\ \bibinfo {address} {London},\ \bibinfo
  {year} {1962})\BibitemShut {NoStop}%
\bibitem [{\citenamefont {Darling}\ and\ \citenamefont
  {Kac}(1957)}]{Darling1957}%
  \BibitemOpen
  \bibfield  {author} {\bibinfo {author} {\bibfnamefont {D.~A.}\ \bibnamefont
  {Darling}}\ and\ \bibinfo {author} {\bibfnamefont {M.}~\bibnamefont {Kac}},\
  }\href@noop {} {\bibfield  {journal} {\bibinfo  {journal} {Trans. Am. Math.
  Soc.}\ }\textbf {\bibinfo {volume} {84}},\ \bibinfo {pages} {444} (\bibinfo
  {year} {1957})}\BibitemShut {NoStop}%
\bibitem [{\citenamefont {Shinkai}\ and\ \citenamefont
  {Aizawa}(2006)}]{shinkai2006lempel}%
  \BibitemOpen
  \bibfield  {author} {\bibinfo {author} {\bibfnamefont {S.}~\bibnamefont
  {Shinkai}}\ and\ \bibinfo {author} {\bibfnamefont {Y.}~\bibnamefont
  {Aizawa}},\ }\href@noop {} {\bibfield  {journal} {\bibinfo  {journal} {Prog.
  Theor. Phys.}\ }\textbf {\bibinfo {volume} {116}},\ \bibinfo {pages} {503}
  (\bibinfo {year} {2006})}\BibitemShut {NoStop}%
\bibitem [{\citenamefont {Kasahara}(1977)}]{kasahara77}%
  \BibitemOpen
  \bibfield  {author} {\bibinfo {author} {\bibfnamefont {Y.}~\bibnamefont
  {Kasahara}},\ }\href@noop {} {\bibfield  {journal} {\bibinfo  {journal}
  {Publ. RIMS, Kyoto Univ.}\ }\textbf {\bibinfo {volume} {12}},\ \bibinfo
  {pages} {801} (\bibinfo {year} {1977})}\BibitemShut {NoStop}%
\bibitem [{\citenamefont {Lubelski}\ \emph {et~al.}(2008)\citenamefont
  {Lubelski}, \citenamefont {Sokolov},\ and\ \citenamefont
  {Klafter}}]{Lubelski2008}%
  \BibitemOpen
  \bibfield  {author} {\bibinfo {author} {\bibfnamefont {A.}~\bibnamefont
  {Lubelski}}, \bibinfo {author} {\bibfnamefont {I.~M.}\ \bibnamefont
  {Sokolov}}, \ and\ \bibinfo {author} {\bibfnamefont {J.}~\bibnamefont
  {Klafter}},\ }\href@noop {} {\bibfield  {journal} {\bibinfo  {journal} {Phys.
  Rev. Lett.}\ }\textbf {\bibinfo {volume} {100}},\ \bibinfo {pages} {250602}
  (\bibinfo {year} {2008})}\BibitemShut {NoStop}%
\bibitem [{\citenamefont {He}\ \emph {et~al.}(2008)\citenamefont {He},
  \citenamefont {Burov}, \citenamefont {Metzler},\ and\ \citenamefont
  {Barkai}}]{He2008}%
  \BibitemOpen
  \bibfield  {author} {\bibinfo {author} {\bibfnamefont {Y.}~\bibnamefont
  {He}}, \bibinfo {author} {\bibfnamefont {S.}~\bibnamefont {Burov}}, \bibinfo
  {author} {\bibfnamefont {R.}~\bibnamefont {Metzler}}, \ and\ \bibinfo
  {author} {\bibfnamefont {E.}~\bibnamefont {Barkai}},\ }\href@noop {}
  {\bibfield  {journal} {\bibinfo  {journal} {Phys. Rev. Lett.}\ }\textbf
  {\bibinfo {volume} {101}},\ \bibinfo {pages} {058101} (\bibinfo {year}
  {2008})}\BibitemShut {NoStop}%
\bibitem [{\citenamefont {Miyaguchi}\ and\ \citenamefont
  {Akimoto}(2011)}]{Miyaguchi2011}%
  \BibitemOpen
  \bibfield  {author} {\bibinfo {author} {\bibfnamefont {T.}~\bibnamefont
  {Miyaguchi}}\ and\ \bibinfo {author} {\bibfnamefont {T.}~\bibnamefont
  {Akimoto}},\ }\href@noop {} {\bibfield  {journal} {\bibinfo  {journal} {Phys.
  Rev. E}\ }\textbf {\bibinfo {volume} {83}},\ \bibinfo {pages} {031926}
  (\bibinfo {year} {2011})}\BibitemShut {NoStop}%
\bibitem [{\citenamefont {Miyaguchi}\ and\ \citenamefont
  {Akimoto}(2013)}]{Miyaguchi2013}%
  \BibitemOpen
  \bibfield  {author} {\bibinfo {author} {\bibfnamefont {T.}~\bibnamefont
  {Miyaguchi}}\ and\ \bibinfo {author} {\bibfnamefont {T.}~\bibnamefont
  {Akimoto}},\ }\href {\doibase 10.1103/PhysRevE.87.032130} {\bibfield
  {journal} {\bibinfo  {journal} {Phys. Rev. E}\ }\textbf {\bibinfo {volume}
  {87}},\ \bibinfo {pages} {032130} (\bibinfo {year} {2013})}\BibitemShut
  {NoStop}%
\bibitem [{\citenamefont {Akimoto}\ and\ \citenamefont
  {Miyaguchi}(2013)}]{Akimoto2013a}%
  \BibitemOpen
  \bibfield  {author} {\bibinfo {author} {\bibfnamefont {T.}~\bibnamefont
  {Akimoto}}\ and\ \bibinfo {author} {\bibfnamefont {T.}~\bibnamefont
  {Miyaguchi}},\ }\href@noop {} {\bibfield  {journal} {\bibinfo  {journal}
  {Phys. Rev. E}\ }\textbf {\bibinfo {volume} {87}},\ \bibinfo {pages} {062134}
  (\bibinfo {year} {2013})}\BibitemShut {NoStop}%
\bibitem [{\citenamefont {Akimoto}\ and\ \citenamefont
  {Yamamoto}(2016)}]{AkimotoYamamoto2016a}%
  \BibitemOpen
  \bibfield  {author} {\bibinfo {author} {\bibfnamefont {T.}~\bibnamefont
  {Akimoto}}\ and\ \bibinfo {author} {\bibfnamefont {E.}~\bibnamefont
  {Yamamoto}},\ }\href@noop {} {\bibfield  {journal} {\bibinfo  {journal} {J.
  Stat. Mech.}\ }\textbf {\bibinfo {volume} {2016}},\ \bibinfo {pages} {123201}
  (\bibinfo {year} {2016})}\BibitemShut {NoStop}%
\bibitem [{\citenamefont {Albers}\ and\ \citenamefont
  {Radons}(2018)}]{Albers2018}%
  \BibitemOpen
  \bibfield  {author} {\bibinfo {author} {\bibfnamefont {T.}~\bibnamefont
  {Albers}}\ and\ \bibinfo {author} {\bibfnamefont {G.}~\bibnamefont
  {Radons}},\ }\href {\doibase 10.1103/PhysRevLett.120.104501} {\bibfield
  {journal} {\bibinfo  {journal} {Phys. Rev. Lett.}\ }\textbf {\bibinfo
  {volume} {120}},\ \bibinfo {pages} {104501} (\bibinfo {year}
  {2018})}\BibitemShut {NoStop}%
\bibitem [{\citenamefont {Radice}\ \emph {et~al.}(2020)\citenamefont {Radice},
  \citenamefont {Onofri}, \citenamefont {Artuso},\ and\ \citenamefont
  {Pozzoli}}]{Radice2020}%
  \BibitemOpen
  \bibfield  {author} {\bibinfo {author} {\bibfnamefont {M.}~\bibnamefont
  {Radice}}, \bibinfo {author} {\bibfnamefont {M.}~\bibnamefont {Onofri}},
  \bibinfo {author} {\bibfnamefont {R.}~\bibnamefont {Artuso}}, \ and\ \bibinfo
  {author} {\bibfnamefont {G.}~\bibnamefont {Pozzoli}},\ }\href {\doibase
  10.1103/PhysRevE.101.042103} {\bibfield  {journal} {\bibinfo  {journal}
  {Phys. Rev. E}\ }\textbf {\bibinfo {volume} {101}},\ \bibinfo {pages}
  {042103} (\bibinfo {year} {2020})}\BibitemShut {NoStop}%
\bibitem [{\citenamefont {Albers}\ and\ \citenamefont
  {Radons}(2022)}]{Albers2022}%
  \BibitemOpen
  \bibfield  {author} {\bibinfo {author} {\bibfnamefont {T.}~\bibnamefont
  {Albers}}\ and\ \bibinfo {author} {\bibfnamefont {G.}~\bibnamefont
  {Radons}},\ }\href {\doibase 10.1103/PhysRevE.105.014113} {\bibfield
  {journal} {\bibinfo  {journal} {Phys. Rev. E}\ }\textbf {\bibinfo {volume}
  {105}},\ \bibinfo {pages} {014113} (\bibinfo {year} {2022})}\BibitemShut
  {NoStop}%
\bibitem [{\citenamefont {Bouchaud}\ and\ \citenamefont
  {Georges}(1990)}]{bouchaud90}%
  \BibitemOpen
  \bibfield  {author} {\bibinfo {author} {\bibfnamefont {J.}~\bibnamefont
  {Bouchaud}}\ and\ \bibinfo {author} {\bibfnamefont {A.}~\bibnamefont
  {Georges}},\ }\href@noop {} {\bibfield  {journal} {\bibinfo  {journal} {Phys.
  Rep.}\ }\textbf {\bibinfo {volume} {195}},\ \bibinfo {pages} {127} (\bibinfo
  {year} {1990})}\BibitemShut {NoStop}%
\bibitem [{\citenamefont {Machta}(1985)}]{Machta1985}%
  \BibitemOpen
  \bibfield  {author} {\bibinfo {author} {\bibfnamefont {J.}~\bibnamefont
  {Machta}},\ }\href@noop {} {\bibfield  {journal} {\bibinfo  {journal}
  {Journal of Physics A: Mathematical and General}\ }\textbf {\bibinfo {volume}
  {18}},\ \bibinfo {pages} {L531} (\bibinfo {year} {1985})}\BibitemShut
  {NoStop}%
\bibitem [{\citenamefont {Box}\ and\ \citenamefont
  {Muller}(1958)}]{box1958note}%
  \BibitemOpen
  \bibfield  {author} {\bibinfo {author} {\bibfnamefont {G.~E.}\ \bibnamefont
  {Box}}\ and\ \bibinfo {author} {\bibfnamefont {M.~E.}\ \bibnamefont
  {Muller}},\ }\href@noop {} {\bibfield  {journal} {\bibinfo  {journal} {Ann.
  Math. Statist.}\ }\textbf {\bibinfo {volume} {29}},\ \bibinfo {pages} {610}
  (\bibinfo {year} {1958})}\BibitemShut {NoStop}%
\end{thebibliography}%

\end{document}
%
% ****** End of file apssamp.tex ******

