\documentclass[journal = jpclcd]{achemso}
%,layout=twocolumn

\usepackage{amssymb}
\usepackage{enumerate}
\usepackage{multirow}
\usepackage{changepage}
\usepackage{appendix}
\usepackage{sectsty}
\usepackage{float}
\usepackage{booktabs}
\sectionfont{\centering}
\date{}
\renewcommand{\thefigure}{S\arabic{figure}}
\renewcommand{\thetable}{S\arabic{table}}

\author{Guorong Weng}
\author{Vojt\v{e}ch Vl\v{c}ek}
\affiliation{Department of Chemistry and Biochemistry, University of California, Santa Barbara, 93106, U.S.A}
\email{vlcek@ucsb.edu}

\title{Supplementary Information for  ``Quasiparticles and Band Transport in Organized Nanostructures of Donor-Acceptor Copolymers''}

\begin{document}

\maketitle

\section{Computational details}

\subsection{Geometries}

Each polymer is considered to be straight and infinitely periodic. For D-A copolymers, the conventional alkyl groups attached to the fluorene unit are replaced with hydrogen atoms for computational convenience. The geometry and lattice constans of each periodic system are fully relaxed by employing Quantum-Espresso package \cite{Giannozzi2009,Giannozzi2017} with Kohn-Sham density functional theory (DFT) within the generalized gradient approximation (GGA) \cite{Perdew1996} combined with Tkachenko-Scheffler treatment of the van der Waals interactions.\cite{Tkatchenko2009} The lattice parameters of rectangular cells used throughout this study are summarized in Table~\ref{tab:lattice_constants_and_k-point} together with the k--point meshes used to sample the Brillouin zones. In the optimization, $\pi$-$\pi$ stacking without displacement is energetically the most favorable; the structures are illustrated in Figure \ref{Fig_supercells}.

\begin{table}[H]
     \centering
     \begin{tabular}{c|c|c|c|ccc|c}
 \multirow{2}{*}{system} & \multicolumn{3}{c|}{lattice constants}                                                   & \multirow{2}{*}{\begin{tabular}[c]{@{}c@{}}k-point\\ mesh\end{tabular}} \\ \cline{2-4}
                         & a [\AA]     & b [\AA]    & c [\AA]             &                                                                         \\ \hline
 1D FBT                  & 12.774 & -     & -     & 4$\times$1$\times$1                                                                   \\
 2D FBT                  & 12.759 & 3.836 & -    & 1$\times$4$\times$1                                                                   \\
 3D FBT                  & 12.759 & 3.802 & 7.109   & 4$\times$4$\times$4                                                                   \\
 1D FB-Ox                 & 12.774 & -     & -      & 4$\times$1$\times$1                                                                   \\
 1D FB-Se                 & 12.774 & -     & -       & 4$\times$1$\times$1                                                                  
 \end{tabular}
     \caption{Lattice constants of various systems of interest and the k-point meshes for geometry optimization used in the Quantum-Espresso package.}
     \label{tab:lattice_constants_and_k-point}
 \end{table}
 
\subsection{Electronic structure and excitation energies}
The many-body calculations are performed on polymer geometries obtained from first-principles structural optimizations with 0D, 1D, 2D, and 3D periodic boundary conditions.\cite{Rozzi2006} The details of the structural relaxation are provided the previous section. The QP energies are obtained from perturbation theory using the Kohn-Sham density functional theory (DFT) with generalized gradient approximation as a starting point. The DFT step was performed using supercells that correspond to the $k$-point meshes in Table~\ref{tab:lattice_constants_and_k-point}; the supercells are illustrated in Figure~\ref{Fig_supercells}. The parameters of our DFT calculations are in Table~\ref{tab:DFTsetups1}\&\ref{tab:DFTsetups2}.
From the DFT step, we obtain a set of eigenvalues $\left\{\varepsilon^{\rm KS}\right\}$ and corresponding eigenstates $\left\{\phi\right\}$. The quasiparticle energies are computed as
\begin{equation}
    \varepsilon = \varepsilon^{\rm KS} + \left\langle \phi\middle| \hat \Sigma(\omega = \varepsilon) - \hat v_{xc} \middle| \phi\right\rangle
\end{equation}
where $v_{xc}$ is the mean-field exchange-correlation potential and $\Sigma(\omega)$ is the dynamical and non-local self-energy operator. In the space-time domain (represented by coordinates $1\equiv (r_1, t_1)$, the self-energy is approximated as \cite{Hedin1965}
\begin{equation}
    \Sigma(1,2) = iG_0(1,2)W_0(1,2^+)
\end{equation}
where $G$ is the KS Green's function, $W$ is the screened Coulomb interaction computed within the random-phase approximation, and $2^+$ is infinitesimally later than $2$.\cite{martin_reining_ceperley_2016} The total self-energy is further decomposed to the static exchange and frequency-dependent correlation contribution. The evaluation of the self-energy employs the stochastic approach in which the expectation value of the self-energy is computed through a randomized sampling of wave functions and stochastic decomposition of quantum mechanical operators.\cite{Neuhauser2014,Vlcek2018,Vlcek2019} The parameters of the stochastic $G_0W_0$ calculations are in Table. \ref{tab:GWsetups}.

For periodic systems, Brillouin-zone unfolding \cite{Popescu2012,Huang2014,Boykin2005,Boykin2007,Brooks2020} is performed to generate the band structure. Then many-body calculations are performed on LCB, AIB and UCB at the $k$-points accessible by the choice of the supercell to give QP energies that form QP bands. The exchange and correlation energies are extracted from our GW calculations. The bands structures are interpolated by cubic splines.

\begin{table}[H]
\centering
\begin{tabular}{c|c|c|c}
System          & Grid                                                                                  & \begin{tabular}[c]{@{}c@{}}Gridpoint Spacing\\ {(}bohr{)}\end{tabular}          & \begin{tabular}[c]{@{}c@{}}Cutoff\\ {(}hartree{)}\end{tabular} \\ \hline
1D              & 482$\times$76$\times$76                                                                             & \begin{tabular}[c]{@{}c@{}}dx=0.400664\\ dy=dz=0.4\end{tabular}                 & 26                                                             \\ \cline{2-4} 
2D              & 74$\times$144$\times$480                                                                            & \begin{tabular}[c]{@{}c@{}}dx=0.4\\ dy=0.402778\\ dz=0.401852\end{tabular}      & 26                                                             \\ \cline{2-4} 
3D              & 244$\times$72$\times$136                                                                            & \begin{tabular}[c]{@{}c@{}}dx=0.395264\\ dy=0.399167\\ dz=0.395112\end{tabular} & 26                                                             \\ \cline{2-4} 
single molecule & \begin{tabular}[c]{@{}c@{}}76$\times$76$\times$50 (F)\\ 80$\times$50$\times$56 (BT)\\ 56$\times$90$\times$96 (FBT)\end{tabular} & 0.4                                                                             & 28                                                            
\end{tabular}
\caption{Setups in the DFT calculations of all systems containing F and BT. Note: The system is periodic in x direction in 1D calculations but in y and z directions in 2D calculations.}
\label{tab:DFTsetups1}
\end{table}

\begin{table}[H]
\centering
\begin{tabular}{c|c|c|c}
System                                                                   & Grid        & \begin{tabular}[c]{@{}c@{}}Grid Spacing\\ {(}bohr{)}\end{tabular} & \begin{tabular}[c]{@{}l@{}}Cutoff\\ {(}hartree{)}\end{tabular} \\ \hline
1D fluorene                                                              & 320$\times$76$\times$76   & \begin{tabular}[c]{@{}c@{}}dx=0.397250\\ dy=dz=0.4\end{tabular}        & 26                                                             \\ \cline{2-4} 
\begin{tabular}[c]{@{}l@{}}O-substituted\\ 1D strand\end{tabular}   & 482$\times$74$\times$74   & \begin{tabular}[c]{@{}c@{}}dx=0.400664\\ dy=dz=0.4\end{tabular}        & 26                                                             \\ \cline{2-4} 
\begin{tabular}[c]{@{}l@{}}Se-substituted\\ 1D strand\end{tabular} & 482$\times$74$\times$74   & \begin{tabular}[c]{@{}c@{}}dx=0.400664\\ dy=dz=0.4\end{tabular}        & 26                                                             \\ \cline{2-4} 
1D polyacetylene                                                         & 126$\times$100$\times$100 & \begin{tabular}[c]{@{}c@{}}dx=0.298107\\ dx=dz=0.3\end{tabular}        & 26                                                             \\ \cline{2-4} 
1D polyethylene                                                          & 130$\times$100$\times$100 & \begin{tabular}[c]{@{}c@{}}dx=0.297060\\ dy=dz=0.3\end{tabular}        & 26                                                            
\end{tabular}
\caption{Setups in the DFT calculations of the other systems.}
\label{tab:DFTsetups2}
\end{table}

\begin{table}[H]
\centering
\begin{tabular}{p{0.5\textwidth}|c}
Parameter                             & Value                            \\ \hline
plane wave cut-off (hartree)                      & 26/28 (same in DFT calculations) \\
& \\
number of random vectors used for sparse stochastic compression           & 20000 \\
& \\
number of random vectors characterizing the screened Coulomb interaction (per each vector sampling the Green's function) & 15\\
& \\
number of vectors sampling the Green's function & 1000\\
& \\
maximum time for real-time propagation of the dynamical self-energy              & 50 a.u.                           
\end{tabular}
\caption{Setups in the GW calculations of all systems.}
\label{tab:GWsetups}
\end{table}

\section{Supplementary figures and tables}
\begin{figure}[H]
    \centering
    \includegraphics[width=.8\textwidth]{Fig_supercells.png}
    \caption{The supercells of periodic systems in our DFT and GW calculations. (a) A supercell of the 1D system containing 8 repeated units in the periodic direction, the polymer axis which is defined as the X direction. (b) An 8$\times$8 supercell of the 2D system that is periodic in two directions. The X direction is defined as in 1D and the Y axis denotes the polymer $\pi$-$\pi$ stacking direction. (c) A 4$\times$4$\times$4 supercell of the 3D system that is periodic in three directions. The X and Y directions are defined as in 2D. The Z direction the edge-to-edge stacking direction.} 
    \label{Fig_supercells}
\end{figure}



\begin{figure}[H]
    \centering
    \includegraphics[width=.8\textwidth]{Fig_expIPEA.png}
    \caption{Charge excitation energies, the highest occupied molecular orbitals (HOMO), and lowest unoccupied orbitals (LUMO) of the fluorene unit and the benzothiadiazole unit, respectively. Red and blue colors distinguish the wave function phase. The computed results are from the GW MB calculations, while the experimental results are available from the NIST database as indicated by the red dots on the energy axis.}  
    \label{Fig_IPEA}
\end{figure}

\begin{figure}[H]
    \centering
    \includegraphics[width=.8\textwidth]{Fig_energydiagram.png}
    \caption{Calculated QP energies and fundamental gaps of different systems. (a) Ionization potentials and electron affinities computed from GW for the FBT monomer, 1D FBT strand, and 2D FBT surface. The plotted frontier orbitals show that in the periodic systems, the valence band maximum state inherits the delocalized characteristic from the HOMO of the monomer and the conduction minimum state inherits the localized feature from the LUMO. Red and blue colors distinguish the wave function phase. (b) The fundamental gap as a function of the dimensionality of the system. Both the DFT gap and QP gap collapse as the system evolves from 0D (monomer) to 3D, while the QP gap shows a much more responsive contraction with respects to the system's topology. The red dots on the axes are available experimental results.\cite{Mai2015}} 
    \label{Fig_EnergyDiagram}
\end{figure}

\begin{figure}[H]
    \centering
    \includegraphics[width=.8\textwidth]{Fig_hybri.png}
    \caption{Formations of molecular orbitals of the FBT monomers from the donor and the acceptor units. Red and blue colors distinguish the wave function phase. The FBT HOMO retains the delocalized feature of the fluorene and the benzothiadiazole HOMO, which represents the signature of the lower conjugated band (LCB). The FBT LUMO, however, retain the acceptor LUMO only being highly localized on the acceptor unit, which is responsible for the formation of the acceptor impurity band (AIB) in the periodic system. The LUMO+1 behaves similarly to the HOMO, accounting for the formation of the upper conjugated band (UCB).} 
    \label{Fig_Hybri}
\end{figure}

\begin{figure}[H]
    \centering
    \includegraphics[width=.8\textwidth]{Fig_1Dorbitals.png}
    \caption{Selected orbitals from the bands of interest in the 1D FBT system. The periodicity of the orbital correspond to the crystal momentum of the state. Red and blue colors distinguish the wave function phase. The lower block presents 5 states from the LCB and they feature delocalized orbitals along the backbone. The middle block presents 5 states from AIB where the orbitals are highly localized on the acceptor unit regardless the change in periodicity. The upper block presents 5 states from the UCB, which are qualitatively similar to those from LCB (the lower block).} 
    \label{Fig_1Dorbitals}
\end{figure}


\begin{figure}[H]
    \centering
    \includegraphics[width=.8\textwidth]{Fig_AIB.png}
    \caption{Band structures of three D-A copolymers that have different heteratoms on the acceptor unit obtained from the Quantum-Espresso package. These three copolymers share very similar geometries (Table \ref{tab:3polymer_geometry_info}). The highlighted bands from the bottom to the top correspond to LCB, AIB, and UCB, which shows the presence of AIB regardless of the chemical modifications (highlighted in the chemical structures) on the acceptor.\cite{Chua2019}} 
    \label{Fig_AIB}
\end{figure}

\begin{table}[H]
     \centering
     \begin{tabular}{c|c|c|c|c}
 \multirow{2}{*}{system} & \multicolumn{2}{c|}{torsion angle ($^\circ$)} & \multicolumn{2}{c}{C-to-C distance (\AA)} \\ \cline{2-5} 
                         & $\phi_1$         & $\phi_2$        & $d_1$             & $d_2$                                       \\ \hline
 FBOX                    & 41.80            & 43.05           & 1.467             & 1.466                                   \\
 FBT                     & 42.37            & 43.92           & 1.470             & 1.471                                      \\
 FBSE                    & 41.68            & 42.75           & 1.471             & 1.471                                    
     \end{tabular}
     \caption{Torsion angles and C-to-C distances between the fluorene unit and three different acceptor units.}
     \label{tab:3polymer_geometry_info}
\end{table}


\begin{table}[H]
\centering
\begin{tabular}{c|c|c}
Decomposition                  & \multicolumn{2}{c}{Contribution {(}meV{)}} \\ \hline
single-electron interactions   & \multicolumn{2}{c}{-249}                   \\ \hline
classical Coulomb interactions & \multicolumn{2}{c}{1227}                   \\ \hline
exchange-correlation           & -292 (mean-field)     & -120 (non-local)    \\ \hline
total bandwidth                & 686 (DFT)             & 858 (GW)           
\end{tabular}
\caption{Individual contribution to the LCB width of 1D FBT.}
\label{tab:energydecomposition}
\end{table}

\begin{table}[H]
     \centering
     \begin{tabular}{c|ccc}
     \multirow{2}{*}{system} & \multicolumn{3}{c}{VBW GW/DFT {(}meV{)}}                                                \\ \cline{2-4} 
                         & \multicolumn{1}{c|}{x}            & \multicolumn{1}{c|}{y}            & z          \\ \hline
 \begin{tabular}[c]{@{}c@{}}1D \\ chain\end{tabular}              & \multicolumn{1}{c|}{858$\pm$(38)/686} & \multicolumn{1}{c|}{-}            & -          \\
 \begin{tabular}[c]{@{}c@{}}2D \\ surface\end{tabular}            & \multicolumn{1}{c|}{363$\pm$(26)/393} & \multicolumn{1}{c|}{626$\pm$(24)/625} & -          \\
 \begin{tabular}[c]{@{}c@{}}3D \\ solid\end{tabular}              & \multicolumn{1}{c|}{588$\pm$(33)/533} & \multicolumn{1}{c|}{711$\pm$(32)/661} & 43$\pm$(32)/23
     \end{tabular}
     \caption{Valence bandwidths (VBW) of different FBT systems in each periodic direction by DFT and GW.}
     \label{tab:bandwidths1}
 \end{table}

\begin{table}[H]
     \centering
     \begin{tabular}{c|ccc}
     \multirow{2}{*}{system} & \multicolumn{3}{c}{CBW GW/DFT {(}meV{)}}                                                \\ \cline{2-4} 
                         & \multicolumn{1}{c|}{x}            & \multicolumn{1}{c|}{y}            & z          \\ \hline
 \begin{tabular}[c]{@{}c@{}}1D \\ chain\end{tabular}              & \multicolumn{1}{c|}{265$\pm$(46)/182} & \multicolumn{1}{c|}{-}            & -          \\
 \begin{tabular}[c]{@{}c@{}}2D \\ surface\end{tabular}            & \multicolumn{1}{c|}{142$\pm$(29)/111} & \multicolumn{1}{c|}{115$\pm$(32)/55} & -          \\
 \begin{tabular}[c]{@{}c@{}}3D \\ solid\end{tabular}              & \multicolumn{1}{c|}{164$\pm$(34)/143} & \multicolumn{1}{c|}{123$\pm$(35)/132} & 302$\pm$(33)/256
     \end{tabular}
     \caption{Conduction bandwidths (CBW) of different FBT systems in each periodic direction by DFT and GW.}
     \label{tab:bandwidths2}
 \end{table}

\begin{figure}[H]
    \centering
    \includegraphics[width=.8\textwidth]{Fig_1DExP.png}
    \caption{Band structures with exchange and correlation energies of the 1D FBT system. Exchange and correlation energies are plotted relative to the band average. The $\Gamma$ to X portion stands for the transport in the polymer direction. (a) The QP energy and exchange energy are plotted  as a function of the crystal momentum for the highlighted lower and middle bands (colored). UCB is plotted in black. In LCB, the exchange grows more and more negative going from the $\Gamma$ point to the X point due to the increase in orbital overlaps, which indicates the exchange interactions broaden the valence bandwidth in the conjugated direction. In AIB, however, the exchange energy is almost insensitive to the change in states due to the fact that all the orbitals are highly localized. (b) The QP energy and correlation energy are plotted  as a function of the crystal momentum for the highlighted lower and middle bands (colored).  UCB is plotted in black. In both LCB and AIB, the correlation energy suppresses the bandwidth due to the fact that the higher the QP energy, the more negative the correlation energy (Figure ~\ref{Fig_CorrQPE}).} 
    \label{Fig_1DExP}
\end{figure}

 \begin{table}[H]
     \centering
     \begin{tabular}{c|c|c|c}
 \multirow{2}{*}{system} & \multicolumn{2}{c|}{Exchange energy (eV)} & \multirow{2}{*}{Difference (meV)} \\ \cline{2-3} 
            & Center state         & Boundary state                                                       \\ \hline
 1D PAE     & -14.59           & -16.65           & 2069 \\
 1D PEE     & -19.77           & -19.41           & -363                                 
     \end{tabular}
     \caption{The exchange energies of the states at the Brillouin center and boundary of the highest valence bands for 1D polyacetylene and polyethylene systems.}
     \label{tab:exchange-driven_band_broadening}
\end{table}

\begin{figure}[H]
    \centering
    \includegraphics[width=.8\textwidth]{Fig_CorrQPE.png}
    \caption{Correlation energy plotted as a function of the QP energy of the $\Gamma$ state (black) and the X state (blue) for systems: (a) 1D FBT strand, (b) 1D trans-polyacetylene, and (c) 1D polyethylene. The intersection between the curve and the straight line of the same color represents both the QP energy (x-coordinate) and the correlation energy (y-coordinate). All systems show the same rule that the correlation energy increases as the QP energy decreases.}
    \label{Fig_CorrQPE}
\end{figure}

\begin{figure}[H]
    \centering
    \includegraphics[width=.8\textwidth]{Fig_torsion.png}
    \caption{Molecular geometries of FBT single strands optimized in periodic systems with different dimensionalities. (a) The donor subunit and the acceptor subunit retain a rigid planar structure with unaltered bond lengths and bond angles in three systems. The main geometrical difference among the single strands is the torsion angles, $\phi_1$ and $\phi_2$, between the donor and the acceptor, which are slightly different from each other (Table \ref{tab:FBT_geometry_info}). (b) The average torsion angle ($\phi_1$$+$$\phi_2$)$/$2 of the single strands from the optimized 1D, 3D, and 2D system in the order of the magnitude being 43$^\circ$, 49$^\circ$, and 56$^\circ$.} 
    \label{Fig_torsion}
\end{figure}

\begin{table}[H]
\begin{tabular}{c|ccc}
\multirow{2}{*}{system} & \multicolumn{3}{c}{torsion angle ($^\circ$)}                                 \\ \cline{2-4} 
                        & \multicolumn{1}{c|}{$\phi_1$}  & \multicolumn{1}{c|}{$\phi_2$}  & average \\ \hline
strand of 1D chain      & \multicolumn{1}{c|}{42.37} & \multicolumn{1}{c|}{43.92} & 43      \\
strand from 3D solid    & \multicolumn{1}{c|}{49.79} & \multicolumn{1}{c|}{49.36} & 49      \\
strand from 2D surface  & \multicolumn{1}{c|}{55.67} & \multicolumn{1}{c|}{56.16} & 56     
\end{tabular}
\caption{Torsion angles between the fluorene unit and the benzothiadiazole unit in three FBT systems.}
\label{tab:FBT_geometry_info}
\end{table}



\begin{figure}[H]
    \centering
    \includegraphics[width=.8\textwidth]{Fig_2DExP.png}
    \caption{Band structures with exchange and correlation energies of the 2D FBT system. Exchange and correlation energies are plotted relative to the band average. The $\Gamma$ to Y portion stands for the transport in the polymer $\pi$-$\pi$ stacking direction while the $\Gamma$ to X portion stands for the transport along the polymer backbone. (a) The QP energy and exchange energy are plotted  as a function of the crystal momentum for the highlighted lower and middle bands (colored). UCB is plotted in black. In the polymer direction, the exchange causes the same effects as found in the 1D system, while in the $\pi$-$\pi$ stacking direction, the exchange behave oppositely. The exchange suppresses the both AIB and LCB widths. (b) The QP energy and correlation energy are plotted  as a function of the crystal momentum for the highlighted lower and middle bands (colored).  UCB is plotted in black.In both AIB and LCB and both directions, the correlation suppresses the widths due to the fact that the correlation energy increases as the QP energy decreases (Figure \ref{Fig_CorrQPE}).} 
    \label{Fig_2DExP}
\end{figure}



 \begin{table}[H]
     \centering
     \begin{tabular}{c|c|c|c|c}
 \multirow{2}{*}{supercell size} & \multicolumn{2}{c|}{HOMO {(}eV{)}} & \multicolumn{2}{c}{Transport gap {(}eV{)}} \\ \cline{2-5} 
                                 & DFT          & GW                  & DFT              & GW                      \\ \hline
 2$\times$2$\times$1                           & -4.51        & -5.12$\pm$(0.03)        & 1.52             & 3.02$\pm$(0.05)             \\
 4$\times$4$\times$1                           & -4.53        & -5.52$\pm$(0.03)        & 1.52             & 3.34$\pm$(0.04)             \\
 6$\times$6$\times$1                           & -4.54        & -5.45$\pm$(0.02)        & 1.45             & 3.31$\pm$(0.04)             \\interactions
 8$\times$8$\times$1                           & -4.54        & -5.48$\pm$(0.02)        & 1.52             & 3.33$\pm$(0.03)            
 \end{tabular}
     \caption{Convergence of HOMO energy levels and transport gaps to the  supercell size in 2D calculations.}
     \label{tab:convergence_to_supercell}
 \end{table}


\bibliography{MB_vdW_OC}

\end{document}
