\documentclass[journal = jpclcd]{achemso}
\usepackage{graphicx}
\usepackage{amsmath}
\usepackage{xcolor}
\usepackage{amssymb}
\usepackage{enumerate}
\usepackage{multirow}
\usepackage{changepage}
\usepackage{appendix}
\usepackage{xr}
\makeatletter
\newcommand*{\addFileDependency}[1]{
  \typeout{(#1)}
  \@addtofilelist{#1}
  \IfFileExists{#1}{}{\typeout{No file #1.}}
}
\makeatother

\newcommand*{\myexternaldocument}[1]{
    \externaldocument{#1}
    \addFileDependency{#1.tex}
    \addFileDependency{#1.aux}
}

\myexternaldocument{SI}
\listfiles

\author{Guorong Weng}
\author{Vojt\v{e}ch Vl\v{c}ek}
\affiliation{Department of Chemistry and Biochemistry, University of California, Santa Barbara, 93106, U.S.A}
\email{vlcek@ucsb.edu}

\title{Quasiparticles and Band Transport in Organized Nanostructures of Donor-Acceptor Copolymers}

\keywords{Organic semiconductor, quasiparticle excitations, donor-acceptor copolymer, impurity states,  band transport, many-body interactions}

\begin{document}

\begin{tocentry}
    \centering
    \includegraphics[width=5cm]{TOC_final.png}
    \label{fig:TOCentry}
\end{tocentry}

\date{}
\maketitle


% Keywords: Please provide a minimum of three and a maximum of seven keywords, separated by commas


% Abstract should be written in the present tense and impersonal style (i.e., avoid we), and be at most 200 words long
\begin{abstract}

The performance of organic semiconductor devices is linked to highly-ordered nanostructures of self-assembled molecules and polymers. We employ many-body perturbation theory and study the excited states in bulk compolymers. We discover that acceptors in the polymer scaffold introduce a, hitherto unrecognized, conduction impurity band. The donor units are surrounded by conjugated bands which are only mildly perturbed by the presence of acceptors. Along the polymer axis, mutual interactions among copolymer strands hinder efficient band transport, which is, however, strongly enhanced across individual chains. We find that holes are most effectively transported in the $\pi-\pi$ stacking while electrons in the impurity band follow the edge-to-edge directions. The copolymers exhibit regions with inverted transport polarity, in which electrons and holes are efficiently transported in mutually orthogonal directions.

\end{abstract}


% Text: Please use section headings and subheadings as specified below. For communications, all section headings apart from Experimental Section should be removed
% Please make the first reference to a display item bold: \textbf{Figure 1}
% Do not abbreviate Figure, Equation, etc.; display items are always singular, i.e., Figure 1 and 2.
% Equations are always singular, i.e., Equation 1 and 2, and should be inserted using the {equation} environment, not as graphics
% Please do not use footnotes in the text, additional information can be added to the Reference list.


Donor-acceptor (D-A) semiconducting copolymers represent arguably the most variable class of semiconducting materials in organic electronics.\cite{Gunes2007,Heeger2010,Facchetti2011,Wang2012} The wide range of possible donors and acceptors provides an unmatched tunability of the system's electronic and optical properties.\cite{Yuen2011,Li2012,Zhang2013} Rational device-design is, however, hampered by the complicated relationship between electronic properties and the arrangement of the molecular chains in the condensed phase (e.g., in spin-coated thin-films).\cite{Schwartz2003,Noriega2013} Experiments showed that highly-ordered \emph{nanodomains}, i.e., highly organized nm-scale regions, are widely present in solution-processed thin films. The nanodomains are composed of nanowires,\cite{Oh2009} nanosheets,\cite{Barzegar2018} and crystallites.\cite{Sirringhaus1999,Venkateshvaran2014,Luo2014,Li2016} Face-on ($\pi$-$\pi$) or edge-on stacking is the dominant arrangement of conjugated molecules leading to high charge mobilities and excellent device performance.\cite{Oh2009,Barzegar2018,Sirringhaus1999,Venkateshvaran2014,Luo2014,Li2016} The nanodomains exhibit quasiparticle bands observed by angle-resolved photoemission \cite{Hsu2015}. Hence, the high hole mobilities are explained by band-like transport\cite{Sakanoue2010,Yamashita2014} in the $\pi$-$\pi$ direction\cite{Sirringhaus1999,Luo2014,Kim2014}. However, a detailed microscopic understanding of how the structure and composition of the copolymers impact the electronic excitations is currently missing.

Answering these questions requires a theoretical investigation of the copolymers' electronic structure in the condensed phase. In principle, such simulations need to capture the non-local inter-molecular interactions \cite{Sutton2016} of electrons delocalized along the $\pi$-conjugated backbone.\cite{Hsu2015} The individual polymer chains are highly polarizable and held together by van der Waals (vdW) forces. Even in the limit of ideally crystalline systems, quantitative theoretical predictions of electronic excitations are prohibitive, and they have been limited to crystals of small molecules\cite{Norton2008,Nayak2009,Difley2010,Ryno2013,Refaely-Abramson2013,Poelking2015,Kang2016,Li2016_Blase,Sun2016,Li2018,Bhandari2018}. For polymers, the computational efforts have considered only isolated\cite{Halls1999,Cornil2003} oligomers or 1D periodic systems\cite{Bredas1984,Bredas1985,cheng2017,He2018,Bredas2018} treated by mean-field approaches, which are less expensive but do not take into account the non-local electronic correlations (governed by polarization effects). \cite{Woods2016} Further, the geometries  of the polymer strands are typically forced to be planar, i.e., they disregard actual arrangements in the highly organized domains.\cite{Cornil2003,He2018} Finally, the mean-field methods do not, in principle, provide access to quasiparticle (injected electron and hole) energies and tend to underestimate excitation energies grossly.\cite{martin_reining_ceperley_2016}

In this work, we overcome these limitations and apply state-of-the-art theoretical approaches to explain the key features of the electronic structure of D-A copolymers.
Our calculations employ many-body perturbation theory\cite{martin_reining_ceperley_2016} within the stochastic $GW$ approach.\cite{Neuhauser2014,Vlcek2017,Vlcek2018,Vlcek2019} The electron-electron interactions are computed for each excitation (i.e., no mean-field approximation is applied). In the $GW$ approximation, the interaction term takes into account a selected class of Feynman diagrams describing the electrodynamic screening, i.e., the induced charge density fluctuations. Electrons thus interact via a screened Coulomb interaction, which is non-local and time-dependent. In practice, the $GW$ method yields quasiparticle (QP) excitation energies in excellent agreement with available experimental data.\cite{Blase2011,martin_reining_ceperley_2016,Vlcek2017}

The electronic structure and QP energies of the condensed phase is determined by the properties of the constituting moieties as well as by mutual interactions among individual copolymer strands.  While these contributions are nontrivial, relations among a few key parameters govern the system's overall behavior. To illustrate this, we consider a prototypical example:  ``FBT'' and related D-A copolymers\cite{Mai2013,Mai2015,Cui2018} (see Supporting Information (SI) for the geometry optimization). Here, the fluorene moiety (F) acts as a ``donor'' (D), and benzothiadiazole (BT) acts as an ``acceptor'' (A). The isolated molecules are illustrated in the inset of \textbf{Figure~\ref{Fig1Dbstr}} and the SI. The D units are the source of delocalized electronic states. In contrast, acceptors are typically chosen so that they have a higher electron affinity than donors\cite{Zhou2012,Duan2012}, acting as strong potential wells for electrons (see \textbf{Figure~\ref{Fig_IPEA}}).  Hence, the A unit is a source of localized electrons whose wave functions have a limited spatial extent. 

\begin{figure}
    \centering
    \includegraphics[width=\linewidth]{Fig1Dbstr.png}
    \caption{Quasiparticle band structures and orbitals of selected states of a fluorene (a, b) and FBT (c, d) strands. The monomer units are shown in the inset of panels b and c. The electronic states that are delocalized over the entire polymer backbone are denoted lower and upper conjugated bands (LCB and UCB), for the highest valence and lowest conduction band. The band-edge states in fluorene (a) are formed by LCB and UCB illustrated for the Brillouin zone center, $\Gamma$,  and its boundary, X. The corresponding bands are highlighted in panel b. FBT is D-A copolymer, with the individual subunits labeled in the inset of panel c. Due to the presence of A, the bandstructure contains an acceptor impurity band (AIB) highlighted in red (c). Panel d depicts LCB, AIB, and UCB for two points in the Brillouin zone; the AIB is strongly localized on the acceptor subunit. Red and blue colors distinguish the wave function phase.} 
    \label{Fig1Dbstr}
\end{figure}

In a single copolymer strand (i.e.,  1D periodic system with repeated D and A subunits),  the quantum confinement is reduced in the direction of the polymer axis. Consequently, the fundamental gap of the infinite chain decreases with the polymerization length; for an infinite system, it is $4.08 \pm0.04$~eV, which is 1.48$\pm0.05$~eV less than for an isolated monomer (\textbf{Figure~\ref{Fig_EnergyDiagram}}).  In a condensed phase (either 2D slab or 3D bulk), the presence of neighboring strands eliminates the quantum confinement in the directions orthogonal to the polymer axis. Hence, the fundamental band gap  further decreases (Figure ~\ref{Fig_EnergyDiagram}b).

For quantitative predictions of the band gaps in the condensed phase, many-body methods turn out to be indispensable as dynamical electron-electron interactions are responsible for the non-local (inter-chain) interactions. Indeed, the ionization potential for the 2D slab computed with the s$GW$ method is 5.48$\pm0.02$~eV (Figure~\ref{Fig_EnergyDiagram}a), in excellent agreement with thin-film experiments that provide an estimate of 5.4-5.5 eV.\cite{Mai2015} The fundamental band gaps of the surface and the bulk are 3.33 and 2.23 eV (Figure~\ref{Fig_EnergyDiagram}b), and the latter is in good agreement with the experimental value of 2.32-2.44 eV.\cite{Mai2015}

The periodic copolymer arrangement supports the formation of band structures (observed experimentally, as discussed above). To characterize the principal features of the electronic states, we start with the 1D system shown in Figure~\ref{Fig1Dbstr}c. The crystal momentum is imprinted on the individual wavefunctions (Figure~\ref{Fig1Dbstr}d), which, however, retain much of their molecular character  (\textbf{Figure~\ref{Fig_Hybri}}). It is thus possible to separate the contributions of D and A  to the highest valence and lowest conduction bands responsible for the charge transport. 

The donor behavior dominates the highest valence state; it has conjugated character and delocalized $\pi$ orbitals (see more details in \textbf{Figure~\ref{Fig_1Dorbitals}}).  The top valence band is broad (its bandwidth is $0.86\pm0.04$~eV) with a parabolic dispersion near the extrema that occur at the critical points of the Brillouin zone. The near-band-edge character, together with the large bandwidth, translates to a low effective mass of $\sim0.22 m^*_e$. Such a low value is consistent with experimental results for similar (semi)conducting copolymers.\cite{Hsu2015}  We denote the highest conduction band and the lower conjugated band (LCB). The complementary ``upper'' conjugated band (UCB) is formed from $\pi^*$ orbitals, and it has much higher energy  (Figure~\ref{Fig1Dbstr}d). Both LCB and UCB are qualitatively analogous to the  band edge states in a \emph{pure} fluorene chain (Figure~\ref{Fig1Dbstr}a), i.e., the conjugated bands are only mildly perturbed by the presence of acceptor subunits. The correspondence between the electronic structures of D-A and pure donor polymers has not been noticed up to now.

In contrast, the lowest conduction band of the copolymer comprises states localized only on the acceptors (Figure~\ref{Fig_1Dorbitals}).  The acceptor band has significantly reduced width (Figure~\ref{Fig1Dbstr}c), and it appears between LCB and UCB.

In calculations with distinct A molecules, we found that the exact energy separation between the conjugated and localized states depends only a little on the choice of acceptors (see \textbf{Figure~\ref{Fig_AIB}} for details). In FBT, the separation of the conduction states is 1.11 eV; the oxygen- and selenium- substituted copolymers show slightly larger separations (1.38 eV and 1.16 eV, respectively -- see Figure~\ref{Fig_AIB}). In all cases studied, the localized state is characteristically inserted between the two conjugated bands. Based on the conceptual analogy to charge-trapping  ``in-gap'' states, we denote the lowest conduction states as the acceptor impurity band (AIB).  The formation of the localized and flat impurity band has not been described previously. One of the key findings of this communication is the recognition and distinction between the conjugated and impurity bands.

LCB, AIB, and UCB are present in the same order in 1D and in the condensed phases. While the van der Waals forces only weakly bond the individual copolymer strands, the inter-chain interactions change the band structures significantly. Besides the shift of the QP gaps (discussed above), the charge transport is critically influenced by the changes in the bandwidth. The dispersion of LCB and AIB  determines the charge transport polarity. Further,  the bandwidth is directly related to the charge carrier effective mass. To investigate the physical origin of the of the band structure changes, we will separate two main contributions: (i) the one-body electronic interactions\cite{onebodyterm} including the (classical) density-density Coulomb repulsion (\textbf{Table~\ref{tab:energydecomposition}}), and (ii) the electron-electron interactions,  which represent highly non-local and dynamical (time-dependent) quantum effects. 

The first contribution mostly depends on the local\cite{localproperties} properties of the copolymer. The electronic structure (and charge transport) strongly depend on the bond arrangement between the donor and acceptor subunits.\cite{Bredas1985}  The existence of a single bond between adjacent donors and acceptors implies large rotational freedom.  In practice, the mutual orientation of the A and D units depends on the environment. The rotational angle varies between 43$^\circ$ and 56$^\circ$ in the relaxed structures with 1D, 2D, or 3D topology (\textbf{Figure~\ref{Fig_torsion}}). Other structural variations are insignificant as the rest of the copolymer backbone is rigid, and we disregard them in the analysis.

As noted above, AIB is composed of  localized states  centered on the acceptor subunits. The corresponding wave function near the conduction edge does extent to the D-A joint appreciably (Figure~\ref{Fig1Dbstr}d). Hence, AIB is practically insensitive to the torsion angle.  In contrast, rotation away from the ideally planar geometry leads to the narrowing of conjugated states  (\textbf{Figure~\ref{Figtorsion}}b). Since the torsion angle is larger in the condensed phase than in a free-standing polymer, the hole effective mass in LCB is thus increased in bulk compared to a prediction from the 1D model.

The sensitivity of LCB is directly related to the character of the wave function near the D-A bond. Going from the low energy part of the LCB (near the X point of the Brillouin zone) to the band edge, the wavefunction develops a nodal plane across the D-A joint (Figure~\ref{Figtorsion}c). The presence of the nodes is associated with increased QP kinetic energy. A close inspection of various torsion angles reveals that the nodes across the D-A bond are suppressed when going from 1D to 3D conformation. The band edge is kinetically stabilized (Figure~\ref{Figtorsion}a), while the bottom LCB is insensitive to the rotation. As a result, the single-electron interactions promote bandwidth reduction in the condensed phase.

 \begin{figure}
    \centering
    \includegraphics[width=0.5\linewidth]{Figtorsion.png}
    \caption{The effect of the characteristic torsion angles between D and A subunits found in various condensed phases of FBT: (a) The QP energies of LCB at the Brillouin zone center, $\Gamma$, and its boundary, X as a function of the torsion angle. The QP energy of LCB is more sensitive to torsion at $\Gamma$   than that at X. (b) The QP valence bandwidth linearly decreases with the torsion angle. (c) The LCB wavefunction at the donor (D) and the acceptor (A) joint for $\Gamma$ and $X$ points of the Brillouin zone. Red and blue colors distinguish the wave function phase. At $\Gamma$, the torsion gradually destroys a nodal plane between D and A, leading to kinetic stabilization of the QP energy. Conversely, the ``bridging character'' of LCB at the X point is little affected by the increased torsion. The error bars in panels a and b represent the statistical error of the stochastic many-body calculation.}
    \label{Figtorsion}
\end{figure}

While the local properties are clearly responsible for the electronic structure modification, the non-local many-body effects are equally important and influence the excited states. These electron-electron interactions are decomposed into two principal contributions: (i) non-local exchange (due to the fermionic nature of the charge carriers), and (ii) time-dependent correlations among electrons and holes (which include vdW interactions responsible for the cohesive energy of the bulk).
The significance of the many-body treatment is illustrated by  the fact that LCB and AIB widths increase by $\sim 25\%$ and $\sim 46\%$  if the non-local and dynamic description is used instead of the common mean-field approach (e.g., in local and static density functional theory -- see \textbf{Table~\ref{tab:energydecomposition}}). 

We first inspect the behavior of the conjugated states. While the exchange interaction typically drives electron localization,\cite{Sanchez2008} it surprisingly enhances the dispersion of the delocalized bands along the polymer axis. The energies of states near the valence band maximum are stabilized much less than at the Brillouin zone boundary,  i.e., the X-point (\textbf{Figure~\ref{Fig_1DExP}}a). In the latter case, there is an increased spatial overlap with a large number of occupied orbitals, and energy decrease is observed for states near the X-point. The exchange-driven band widening is a signature of the conjugated bands, and it is not observed otherwise. To document this, we provide complementary calculations for additional polymer strands (polyacetylene and polyethylene, with and without conjugated bonds) in the SI (\textbf{Table~\ref{tab:exchange-driven_band_broadening}}). 

In general, this effect is dramatic for copolymer systems. In the absence of electronic correlation (which reduces the exchange through dynamical screening), LCB would widen by an additional 40\%. This increase can be paralleled with a (spurious) infinite-range response to hole localization observed for bare exchange interactions.\cite{Vlcek2016}

The screening contribution thus changes the picture qualitatively. It is governed by the reducible polarizability which is directly related to charge density fluctuations.\cite{martin_reining_ceperley_2016}
These correlation effects are dominated by optical (plasmon) excitation that shifts to lower energy as the crystal momentum increases (\textbf{Figure~\ref{Fig_CorrQPE}}). The states away from the band edge (i.e., closer to the Brillouin zone boundary) have energies approaching the resonant frequency of the collective charge density oscillations.  For the corresponding quasiparticle excitations, the exchange interaction is strongly attenuated and becomes short range; the QP energies shift up, and the LCB consequently narrows (Figure~\ref{Fig_1DExP}b).

In the condensed systems, the LCB and UCB  remain delocalized only along the polymer, not across the individual strands (illustrated in \textbf{Figure~\ref{Fig3Dbstr}}c). As a result, the conjugated bands can further flatten. Along the edge-to-edge direction ($\Gamma \to$Z Figure~\ref{Fig3Dbstr}a), both LCB and UCB are extremely narrow and effectively ``molecular'' in nature. Neither non-local exchange nor correlation effects contribute significantly to the quasiparticle energies in this case. In practice, any band-transport of holes in LCB is significantly hampered  along the edge-to-edge stacking direction. 

\begin{figure}
    \centering
    \includegraphics[width=.99\textwidth]{Fig3Dbstr.png}
    \caption{Quasiparticle band structure of FBT in a 3D crystalline domain with relative contributions of electronic (a) exchange and (b) correlation energies for LCB and AIB.  The contributions are given by the color code and are plotted relative to the band average. The $\Gamma \to$~Y branch corresponds to the band in the $\pi$-$\pi$ stacking direction; the $\Gamma \to$~X and $\Gamma \to$~Z branches correspond to the intra-chain and edge-on stacking directions. The inverted polarity regime is in the $\Gamma\to$~Z, where the band dispersion of AIB is much higher than for the conjugated bands. (c) Local wave function character is of the selected states at two distinct points in the Brillouin zone along the $\Gamma\to$~Z direction. The two molecules are depicted in the edge-on stacking. Both LCB and UCB remain localized on individual strands, but AIB bridges the polymer chains. Red and blue colors distinguish the wave function phase. }
    \label{Fig3Dbstr}
\end{figure}

However, the localization does not imply that the conjugated bands behave as those in an isolated strand.  Here, the band dispersion is reduced by as much as 60\% along the polymer axis compared to a free-standing copolymer. The flattening is most prominent in the 2D case (\textbf{Table~\ref{tab:bandwidths1}}). Non-local interchain correlations govern the decrease of the LCB width; they are almost twice as big as the effect of torsion between the D and A subunits. In slabs, the strong polarization effects lead to the formation of local maxima in LCB and dispersion narrowing near the $\Gamma$ point (\textbf{Figure~\ref{Fig_2DExP}}b). This indicates that in near-surface regions, the valence band-edge may not be characterized by a single crystal momentum vector, and the fundamental band-gap is likely indirect.

In contrast, cooperative interchain interactions appear along the $\pi-\pi$ stacking  ($\Gamma \to$Y in Figure~\ref{Fig3Dbstr}). As a result, the LCB dispersion is the largest along this face-on direction  (up to 710~meV). Significant bandwidth suggests a high propensity for efficient band-like transport of holes within LCB. The reason for the increased bandwidth (despite the strong on-chain localization) is twofold: first, the packing of chains in bulk is tighter; second, the high efficient screening allows greater delocalization of the $\pi$ (and $\pi^*$) orbitals above and below the conjugated framework. Both effects lead to improved interchain ``communication'', which leads to an enlarged bandwidth. In the 3D condensed phase, the LCB width is largest along the $\pi-\pi$ stacking compared to any other direction (Figures~\ref{Fig3Dbstr}a and \ref{Fig3Dbstr}b) and indicates an efficient band-transport of holes.

The lowest conduction band is very different. The impurity states are strongly localized along the copolymer axis. As a result, local and non-local interactions are insensitive to crystal momentum, and the band is narrow. Further, there are no increased interactions along the $\pi-\pi$ stacking, and the AIB electronic states thus appear to be molecule-like. Hence, the acceptor band is  flat along $\Gamma\to$~Y as well. There is thus a low likelihood of electron band-transport in the face-on or polymer axis directions. 

AIB, however, unexpectedly exhibits cooperative effects along the edge-to-edge stacking (Figure~\ref{Fig3Dbstr}c). For states near the band minimum (i.e., near $\Gamma$), the impurity wavefunction delocalizes across individual copolymer chains. In contrast, a nodal plane appears between every adjacent polymer for higher crystal momenta due to the increase of the kinetic energy towards the Brillouin zone boundary (i.e., near Z).  The associated QP energy variation leads to a relatively wide\cite{dispersion} dispersion of $\sim$300 meV in the $\Gamma\to$Z direction. Besides the kinetic contribution, the band widening is also driven by a large variation of the exchange energy. Along $\Gamma\to$Z,  the AIB thus behaves like the conjugated states in the polymer axis. These properties indicate that AIB can sustain electron transport along the edge-to-edge stacking direction.

In summary, we investigated a prototypical example of D-A copolymers, FBT, and explained its electronic structure and the propensity to band transport in the condensed phase. Our many-body calculations are in excellent agreement with available experimental data and, for the first time, they provide insight into the quasiparticle (added hole and electron) states of bulk copolymers. The results show that acceptors, which typically act as strong potential wells for electrons, form a previously unrecognized ``impurity'' band. In contrast, the donors groups are responsible for delocalized lower (valence) and upper (conduction) conjugated bands. The delocalized states and they surround the acceptor band, but they only mildly affect each other.

The intra-chain transport is negatively impacted by the condensed phase stacking, which affects the rotation between the donor and acceptors. On the other hand, electronic states delocalize across the copolymer strands and form wide bands that likely support efficient transport. Electronic correlations (responsible for the cohesive van der Waals forces) universally suppresses band dispersion, but non-local exchange interactions drive it in selected directions. 

The large width of valence bands along the $\pi-\pi$ stacking indicates that hole transport is possible in the face-on direction. Surprisingly, we observe a strong propensity for electron transport along the edge-on stacking within the acceptor impurity band. Hence, D-A copolymers exhibit an orthogonal ambipolar transport network, which has been so far reported only in heterogeneous mixtures of p-type polymer and a small n-type molecule.\cite{Huang2016,Huang2018} Our results suggest that orthogonal transport of electrons and holes can be achieved in pure D-A copolymers merely through molecular packing.

\begin{suppinfo}
The Supporting Information provides additional texts, figures and tables listed below.

Texts: computational methods and details.

Figures: 1D, 2D, and 3D supercells in computations, QP energy diagrams of molecular and periodic systems, hybridization of FBT frontier orbitals, selected orbitals of UCB, AIB, and LCB in the 1D system, Comparison of band structures of D-A copolymers with different acceptors, QP band structures with exchange and correlation energy as functions of the momentum, graphical solutions to the QP and correlation energies, represnetaion of the D-A torsion angle.

Tables: parameters in DFT and MBPT calculations, measurements of geometrical constants for different polymers, decomposition of the contribution to the valence bandwidth, bandwidths of the LCB and AIB, exchange contribution to the valence bandwidth, measurements of torsion angle for FBT strands, convergence of the IP, EA, and gap to the supercell's size.

\end{suppinfo}

\section{Acknowledgement}
The authors want to acknowledge Prof.~Thuc-Quyen Nguyen and Prof.~Guillermo Bazan for fruitful discussions. This work was supported by the NSF CAREER award through Grant No. DMR-1945098. The calculations were performed as part of the XSEDE computational Project No. TG-CHE180051. Use was made of computational facilities purchased with funds from the National Science Foundation (CNS-1725797) and administered by the Center for Scientific Computing (CSC). The CSC is supported by the California NanoSystems Institute and the Materials Research Science and Engineering Center (MRSEC; NSF DMR 1720256) at UC Santa Barbara.


\bibliography{MB_vdW_OC}




\end{document}
