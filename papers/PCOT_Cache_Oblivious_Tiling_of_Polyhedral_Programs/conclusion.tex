\section{Conclusion and Future Work}
\label{sec:conclusion}

In this paper, we have presented an empirical study of two approaches for
tiling: SLT and COT. We have developed cache oblivious code generator that
support any polyhedral computation to widen the scope of study to polyhedral
programs.  The takeaway is that COT does not save you from tuning effort for
speed, but it gives you reduced off-chip memory accesses without tuning.
However, when the speed is the primary objective, situations where good
utilization of hardware prefetcher is essential negate the benefit of COT.

It is interesting to note that the access pattern of Jacobi-style stencils,
which has been one of the main beneficiary of COT, is prefetcher-friendly, and
are not fully benefiting from COT in our study. Our code generator now offers
the option to use COT instead of other execution strategies implemented in the
polyhedral tools for a much wider class of programs than stencils.  

Since the speed of PCOT and SLT are similar, the savings off-chip memory
accesses may result in saving in energy. It would be interesting to quantify
the energy savings as a future work.  This work only investigated Single-Level
Tiling where multi-level tiling is a known alternative to improve data
locality at multiple levels of the memory hierarchy.  Further study that also
explore the use of multi-level tiling is a natural future work.


% Local Variables: ***
% TeX-master: "PACT2017.tex" ***
% fill-column: 78 ***
% End: ***
