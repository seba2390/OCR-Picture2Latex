\section{Code Generation of PCOT}
\label{sec:pcot}

%% Comment: I assume that by this point we have already explained the
%% transformations by tiling hyperplanes and the iteration space can be tiled
%% along canonic directions.

In this section, we describe our generalization of the cache oblivious code
generation to polyhedal programs: Polyhedral Cache Oblivious Tiling (PCOT).

\subsection{Approach Overview}
The input is any polyhedral loop nest that is fully permutable and hence
tiling it with hyper-rectangular tiles is a legal transformation.  We first
(in Section~\ref{sec:perfect}) describe the case for tiling all dimensions of a
perfect loop nest.  Other cases can be handled with additional pre-processing,
which we describe in Section~\ref{sec:codegen_ext}.

Figure~\ref{fig:sample_code} illustrates the structure of our generated code.
The input loop nest is replaced by a call to start the recursion as shown in 
Figure~\ref{fig:sample_code}a. The computation of the bounding box from loop nests 
are discussed in Section~\ref{sec:computingBB}.

\begin{figure*}
  \centering
    \includegraphics[width=0.98\textwidth]{codegen-example}
    \caption{The structure of generated code for Heat2D stencils, which has
      loop depth $d=3$. (a) The tileable loop nest is replaced by a call to
      start the recursion. The input is the bounding box of the loop nest. (b)
      Structure of the recursive function. The input bounding box is split
      into $2^d$ new orthants by dividing each dimention in half, and the same
      function is recursively called for each new orthant. When the orthant
      reaches the input tile size, the recursion is terminated by a call to
      the base function. (c) Code for the base function that
      performs the computation of a tile in lexicographic order.  }
  \label{fig:sample_code}
\end{figure*}



\subsection{Codegen for Perfect Loop Nests}
\label{sec:perfect}

Given a perfectly nested loop with all $d$ dimensions tilable, we seek to
generate,
\begin{inlinelist}
	\item a call to recursive function to start the recursion,
	\item a recursive function, and
	\item a base function.
\end{inlinelist}  
The recursive function takes origin and the size of the orthant as inputs,
which are $d$ dimensional vectors. For the initial call to the recursive
function, the origin and the orthant size correspond to the bounding box of
the input loop nest.  Bounding box of a domain is the smallest
hyper-rectangular shaped domain which encloses the given domain.

The recursive function visits the iteration domain in divide-and-conquer
order.  It recursively divides iteration domain into orthants until they are
smaller than an input parameter.
%The body of the recursive function has 2 main parts (
%Figure~\ref{fig:sample_code}.c), \begin{inlinelist}
%\item a call to the base function if the orthant is small enough, and
%\item divide the orthant into $2^d$ number of new orthants by dividing each
%dimension by 2 and generate recursive function call to visit each new orthant.
%\end{inlinelist}
The call to the base function is wrapped by a condition to check whether the
size of the current orthant is less than or equal to the base case threshold. 
%explaining the execution order without using wavefronts

The orthants are visited sequentially in the lexicographic order.  For
parallel execution the tasks are executed with wave-front parallelism.  We use
the OpenMP tasks for parallel execution, where each recursive function call is
annotated with {\tt omp task} pragma, and the wave-front time boundaries are
synchronized with {\tt omp taskwait}.

% To describe the execution order of orthants, let us represent new orthants
% with a hypercube of $d$ dimensions. Each subcube can be labeled using $d$
% bits.  When $d=3$, subcubes are (0,0,0),(0,0,1),...,(1,1,1). Sum of three
% bits represent the ``distance'' from (0,0,0) subcube. If the difference of
% ``distance'' of two subcubes is one, then those two subcubes are adjacent to
% and has a dependence between them.  For the sequential execution, subcubes
% can be executed in an increasing order of ``distance'' which satisfy the
% order imposed due to the dependences among subcubes. In the generated code,
% this is a sequence of recursive function calls in the increasing order of
% ``distance'' of subcubes. The recursive function call takes the origin and
% size of subcube as inputs (Figure~\ref{fig:sample_code}.c).
%
% For the parallel execution, subcubes are executed in the increasing order of
% ``distance'' where the subcubes with the same ``distance'' can be executed
% in parallel. i.e., we start with (0,0,0) after that (0,0,1), (0,1,0), and
% (1,0,0) can be executed in parallel. We spawn parallel tasks using
% \texttt{OpenMP task} directives. Then use \texttt{taskwait} directive to
% enforce the increasing order of ``distance''.

% We generate $2^d$ recursive calls with origin and size of the orthant as
% parameters to each call.  The canonic directions are legal tiling
% hyperplanes (since hyper-rectangular tiling is legal), therefore the 45
% degree wavefronts is a legal schedule to execute new orthants. All the
% orthants within the wavefront can be run in parallel. To generate parallel
% code, we group together the recursive calls within the same wavefront. Then
% insert an OpenMP \texttt{\#pragma omp task} before each recursive call and a
% \texttt{\#pragma omp taskwait} after each group to act as a thread barrier
% before moving on to the next group of tasks.
%
% This makes sure that after each group, worker threads wait till all the
% tasks in the group are finished before moving on to the next group of tasks.

The base function visits all points in the intersection of leaf orthant and
the input loop nest iteratively in lexicographical order.  The loop nest in
the base function is identical to the loop nest for a tile in SLT with
parametric tile sizes we use.  One can use any one of the available
parametrically tiled code generators~\cite{sanjay-lcpc2009,
  sanjay-kim-dtilingTR-2010,baskaran-etal-cgo10,iooss:hal2015} to generate
parametrically tiled loops and then extract the point loops.
%% removed since double-blind
% We use the point loops generated by
% DTiling~\cite{sanjay-lcpc2009,sanjay-kim-dtilingTR-2010} code generator in
% AlphaZ~\cite{yuki2013alphaz} system\footnote{Since we use the same code to
% iterate a tile in single-level tiling code and to iterate base case in cache
% oblivious code, the underlying compiler is capable of performing same set of
% optimizations for both codes.}.
%% Following is removed since this is not a feature but kind of a bug
% In divide-and-conquer style codes, size of the base case at run time can be
% less than or equal to the tuned size of the base case. When the problem size
% (at runtime) is a power of 2 of tuned base case size, then the actual base
% case size is equal to the tuned base case size, otherwise, base case size is
% smaller than the tuned base case size.  The actual base case size depends on
% the problem size parameter, therefore, it is important to use a
% parametrically tiled code generator to extract the point loops from.


\subsection{Optimizations}
\label{sec:codegen:opt}
We implement a number of optimizations to improve the speed of code.  In this
section we explain two important optimizations to exit the recursion early.

\subsubsection{Early Exit for Zero Orthants}

The zero orthants surface when we have tile sizes that are cubic bounding box with
hyper-rectangular tiles and vise versa. In this case, orthant size along all
the dimensions reaches the leaf size in different levels of the recursion. But
still our code generator generates $2^d$ sequence of recursive calls where
size some of the new orthants may be zero along dimensions where the input
orthant size is already smaller than the leaf threshold.  This is a simple
optimization where we check whether a width along anyone of the dimensions of
the orthant is zero. If it is zero then exit the recursion.
%In the recursive function, we only divide a
%dimension by two, if the size of the orthant along this dimension is larger
%than that of the tuned base case size. Therefore, it is possible that some
%dimensions are not divided in the middle, leading to new orthants with zero
%width along some dimensions. Lets assume that the origin of the orthant is
%$(0,0)$, the size of the orthant is $(32\times32)$ and the base case threshold
%is $(16\times32)$. In this case, we only divides the first dimension by 2
%(because 32 $>$ 16) and we do not divide the second dimension. This introduce
%only 2 new orthants where origin is $(0,0)$, size $(16\times32)$ and origin is
%$(16,0)$, size $(16\times32)$. But in the recursive function there are $2^2$
%recursive calls (or new orthants). The two of them corresponds to the two new
%orthants mentioned above and other two recursive calls have size of the 2nd
%dimension as 0 since we did not divide it by two because the size of it is
%already matches the base threshold. Hence, there are two recursive calls with
%the size of the orthant is 0. This optimization filter these orthants out.

\subsubsection{Early Exit for Empty Orthants}
Second optimization is due to the fact we visit the bounding box of the input
loop nest instead of the actual domain. Some kernels operate on
non-hyper-rectangular domains(i.e, Cholesky  operates on triangular domains)
and some kernels need iteration space skewing to enable hyper-rectangular
tiling (i.e., Heat2D). If the actual domain is not a hyper-rectangle then the
bounding box will have points where no computations are defined.  This may
lead to orthants outside of the original iteration space, analogous to empty
tiles in iteration space tiling. 

For example, a loop nest whose iteration space is triangular, the boundingbox
is a rectangle where the half of the points has no computation defined. When
we visit this rectangular box in divide-and-conquer order, we will end up
visiting empty orthants and we want to identify these orthants.  We generate
conditions to check whether all vertices of the orthant remain outside of the
original iteration space using \emph{isl} library~\cite{verdoolaege2010isl}.
%If that is the case, we exit the recursion early 

%We generate code to check whether the current orthant is completely outside
%the actual domain of the loop nest.  We check whether all the vertices of the
%orthant unsatisfy at least one of the constraints of the actual domain of the
%input loop nest.  If that is the case we exit the recursion early.  The
%conditions are generated and simplified using the \emph{isl}
%library~\cite{verdoolaege2010isl}.

The \texttt{checkEmpty} method at the top of the
Figure~\ref{fig:sample_code}c implements both optimizations.  Without them,
the code produces correct answers but visits many base cases which are empty.
The recursion ends either when \texttt{checkEmpty} method returns true or when
the orthant reaches its input tile size.

%\subsubsection{Optimization 3}
%If an orthant is smaller than or equal to the base size parameter, it will
% exit the recursion and compute all the points in the orthant. In this case,
% base size parameter act as an threshold.  The actual orthant size can be
% less than or equal to the provided threshold.  The actual base size depends
% on the problem size. For example if the problem size is 1024 then the
% orthant sizes are 1024, 512, 256, 128, 64, 32,... For any base size
% parameter from 32 to 63, the actual base size going to be 32. If the problem
% size is 1100, then the orthant sizes are 1100, 550, 275, 137, 68, 34,... For
% any base size parameter from 34 to 67, the actual base size going to be
% 34. Therefore, the base size we want to use may not be the actual base size
% and it heavily depend on the problem size and out of our control.
%
% We want the generated code to have the exact base case size we provide as a
% parameter(s). The solution is simple, we pad the bounding box of the
% iteration domain so that size of each dimension is multiple of its base size
% parameter and some power of two. If the size of the $i_{th}$ dimension is
% $N_i$, base case size parameter is $b_i$ then the new padded size is the
% minimum value of $b_i\times2^{k} \ge N_i$ where $k$ is an integer and
% $k \ge 0$. When you use the padded bounding box as the starting orthant,
% after $k$ levels of recursion, we reach orthants with size corresponds to
% the base case parameter sizes and then exit the recursion to compute the
% points within the orthant.  The extra points without any computations due to
% the padding is optimized by the Optimization 2.

% \subsubsection{Other optimizations}
% In addition to the main optimizations discussed above, we added
% \emph{restrict} keyword to the declarations of pointer variables. This helps
% the compiler to optimize code knowing that pointer variables are not
% referred using other aliases.
%
% We only transform the loop nests which are dominant in computations.  The
% input loop nests may contain loop nests of different dimensions (depths).
% The number of computations in the loop nests with smaller number of
% dimensions are significantly smaller than the main-loop which is the loop
% nest(s) with higher number of dimensions. We optimize only the main-loop(s)
% which is the hotspot of the input loop nest.
%
% The generated code is parametric in problem size and base case size. Base
% case size is a vector that specify the size along all the tilable
% dimensions. The parameters of recursive and base functions are passed using
% C structs to reduce the number of function parameters in higher dimensional
% programs.

\subsection{Handling Imperfect Loop Nests}
\label{sec:codegen_ext}
The discussion so far assumed perfectly nested loops as inputs.  We now extend
this to imperfect loop nests, and loop nests where subset of the $d$
dimensions are marked as tilable.

%\subsection{Handling imperfectly nested loops}
The input imperfect loop nests are converted to perfect loop nests with a
pre-processing.  This is called the embedding transformation that are used to
handle imperfectly nested loops in parametric tiled code
generation~\cite{sanjay-kim-dtilingTR-2010}. 
It involves, bringing all the statements into the same loop depth by adding
loops with one iteration as necessary. Then affine guards are added to
eliminate sequence of inner loops, which lead to ``perfect loop nest with
affine guards''.

% The conversion process includes
%the following high level steps.  First, we bring all statements in the
%imperfect loop nest into the same loop depth.  This may result in a sequence
%of inner loops.  Then, we use affine guards to eliminate the sequence of
%loops, and we call it ``perfect loop nest with affine guards''.
%%We specify options to Cloog so that the generated loop nest is already perfectly nested.

When a subset of the loops are marked as tilable, we first extract the marked
band of loops parameterized by both program input parameters as well
as untiled outer loop iterators. We apply the techniques we have discussed so
far to generate code for the extracted loop nest. In this case, the function
call to start the recursion is added as the body of outer untiled loop nest.
The inner untiled loop nests are added as the body of the point loop in the
base function.
%For the case where a subset of the loops are marked as tilable, let us assume
%that a band of loops from depth $i$ to $k$ are tilable, and loops from depth 0
%to $i-1$ and $k+1$ to $d$ are not tilable.  The first step is to extract the
%tilable band of loops from depth $i$ to $k$.  Now, we have a loop nest where
%all the dimensions are tilable therefore, we can apply the above techniques to
%generate a recursive function, base function and a function call to start the
%recursion.  The function call to start the recursion is added as the body of
%outer untiled loops nest at depth $i-1$. The inner untiled loops from depth
%$k+1$ to $d$ are set as the body of the innermost loop in the base function.

% Let's consider that we want to tile only the space dimensions (inner 2
% dimensions) of Heat-2D stencil. We extract the inner two loops and generate
% new recursive function and a base function as described in the beginning of
% this section. Now, there is a recursive function call for each outer time
% step. Therefore, we place the function call to start the recursion as the
% body of the untiled outer loop.

% Let's go through an example.  Let's consider adding another outer loop to
% the program in Listing~\ref{lst:motiv_fully}. The resulting loop nest is
% shown in Lising~\ref{lst:motiv_band}. Lets also assume that only the inner 2
% loops are tilable. In this case we extract the tilable loop nests (from
% lines \ref{lst:motiv_band:1s}-\ref{lst:motiv_band:1e}) and provide it as the
% input to the PCOT code generator. This ends up generating a recursive
% function similar to Listing~\ref{lst:motiv_recur}. The base function is
% similar to Listing~\ref{lst:motiv_base}. The recursive function call to
% start recursion will be the body of outer most untiled-loop as shown in
% Listing~\ref{lst:motiv_band_start}. The bounding box (orthant size) is a
% function of the outer loop iterator $c0$.
%

\subsection{Computing the Bounding Box}
\label{sec:computingBB}
The bounding box is a hyper-rectangle containing the iteration space of the
loop nest. It is computed by eliminating the outer loop indices from the loop
bound expressions.  There can be infinitely many bounding boxes for a given
loop nest, but we start with the tightest (smallest) bounding box among all
the possibilities.

The bounding box also plays an important role in deciding the leaf tile size.
We want the leaf tile size to have the exact value as the input tile size
parameter to the generated code. Therefore, we pad the size of bounding box
along each dimension to the minimum value of $b_i\times2^{k_i} \ge N_i$ where
$b_i$ input leaf size parameter, $N_i$ the size along the $i^{th}$ dimension
of bounding box and $k_i \ge 0$ is an integer. Now, at $k_i$th level of
recursion the orthant size along the $i$th dimension will be $b_i$. The
padding of the bounding box introduces iteration points outside of original
iteration space.  These empty points get optimized away by the optimizations
described in Section~\ref{sec:codegen:opt}


%
%The loop bounding expressions of the inner loop indices may be functions of
%outer loop indices.  We eliminate outer loop indices from bounding expressions
%starting from the outer most loop.  In lower bound expressions, we replace
%outer index with its lower bound if the sign of the index is negative, upper
%bound if the sign is positive.  In upper bound expressions, we replace index
%with its upper and lower bound if the sign is positive and negative
%respectively.  At the end all the bounding expressions are functions of input
%parameters and iterators of the loops surrounding bounding box.
%
% Let's consider the scenario where only a subset of the loops are tilable,
% then the bounding box is a function of both input parameters and indices of
% outer untiled loops. In other words, for each instance of outer untiled
% iteration, there is a instance of the bounding box. The approach is same as
% above except, we do not eliminate outer untiled loop indices in the bounding
% expressions.
%
% For a given loop nest, tilable band specifies band of loops which are
% tilable or permutable. Usually, the band is specified by the start loop
% depth and end loop depth.  Bounding box is a cuboid (or hyper-rectangle)
% containing the iteration space of the tilable band.  There can be infinitely
% many bounding boxes for a given tilable band, but we chose the tightest
% bounding box among all the possibilities. Since there are outer loops
% surrounding the tilable band of loops, the bounding box is parameterised by
% outer loop iterators. In other words, there is a bounding box for each
% instance of outer loops.
%
%When the input is a sequence of perfectly nested loops, the corresponding
%domain can be an union of polyhedra.  In this case, we compute the bounding
%box of each polyhedra individually as described above.  Then, we compute the
%upper bound along a given dimension by taking the maximum of upper bounds of
%all the polyhedra along the same dimension.  Lower bound is computed similarly
%by taking the minimum of lower bounds of all the polyhedra along a given
%dimension.  The resulting upper bounds and lower bounds define the bounding
%box of the union of polyhedra.
%
%
% Simply using the bounding box as the input orthant size to start the
% recursion may leads to issues with the actual base case size. For example,
% lets assume a cubic bounding box with size 512 along each dimension. Then
% the subsequent orthant sizes are 512, 256, 128, 64, 32, ... For any base
% size parameter from 32 to 63, the actual base size going to be 32. If the
% problem size is 550, the orthant sizes are 550, 275, 137, 68, 34,... For any
% base size parameter from 34 to 67, the actual base size going to be
% 34. Therefore, the base size we want to use may not be the actual base size
% and it heavily depends on the size of the bounding box and out of our
% control.  We can get control of it, if
%
% In this work, we use parametric tile sizes for cache oblivious tiling code.
% Existing techniques for parametric tiling~\cite{sanjay-lcpc2009,
% baskaran-etal-cgo10} have demonstrated that the parameterization does not
% incur significant performance overhead when compared to tiling by
% compile-time constants. We implemented our cache oblivious tiling code
% generator using Cloog and ISL libraries.
%

% We implemented our cache oblivius tiling code generator in the AlphaZ
% system~\cite{alphaz} which is capable of generating parametrically tiled
% code using D-Tiling~\cite{sanjay-lcpc2009}.

% Local Variables: ***
% TeX-master: "TACO2017.tex" ***
% fill-column: 78 ***
% End: ***
