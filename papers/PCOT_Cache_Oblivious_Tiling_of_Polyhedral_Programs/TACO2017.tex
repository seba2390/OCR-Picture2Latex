\documentclass[sigconf, authordraft=false]{acmart}
%\documentclass[format=acmsmall, review=true, authordraft=true]{acmart}
%\documentclass[sigplan,10pt,review,anonymous]{acmart}\settopmatter{printfolios=true,printccs=false,printacmref=false}
%\documentclass[sigplan,10pt,review,anonymous,authordraft]{acmart}\settopmatter{printfolios=true,printccs=false,printacmref=false}

\usepackage{booktabs} % For formal tables

\usepackage{subcaption}

\usepackage[ruled]{algorithm2e} % For algorithms
\renewcommand{\algorithmcfname}{ALGORITHM}
\SetAlFnt{\small}
\SetAlCapFnt{\small}
\SetAlCapNameFnt{\small}
\SetAlCapHSkip{0pt}
\IncMargin{-\parindent}

%mypackages
\usepackage{listings}
\usepackage{enumitem}
\usepackage{siunitx}
\usepackage{multirow}

%my commands
\setlist[enumerate,1]{%
  label=\arabic*.,
}

\newlist{inlinelist}{enumerate*}{1}
\setlist*[inlinelist,1]{%
  label=(\roman*),
}

\newcommand{\FIXME}[1]{{\textcolor{red}{{\tt{FIXME:}}\,\,#1}}}

%Décommenter et commenter en dessous pour supprimer les commentaires
\newcommand{\remarkauthor}[2]{{\textcolor{red}{{\tt{(#1)}}\emph{~#2}}}}
\newcommand{\tomoRemark}[1]{{\remarkauthor{tomo}{#1}}}


%\DeclareRobustCommand*\cal{\@fontswitch\relax\mathcal}

% Metadata Information
%\acmJournal{TACO}
%\acmVolume{9}
%\acmNumber{4}
%\acmArticle{39}
%\acmYear{2017}
%\acmMonth{3}
%\copyrightyear{2009}
%\acmArticleSeq{9}

% Copyright
%\setcopyright{acmcopyright}
%\setcopyright{acmlicensed}
\setcopyright{rightsretained}
%\setcopyright{usgov}
%\setcopyright{usgovmixed}
%\setcopyright{cagov}
%\setcopyright{cagovmixed}

% DOI
%\acmDOI{0000001.0000001}

% Paper history
%\received{February 2007}
%\received[revised]{March 2009}
%\received[accepted]{June 2009}


% Document starts
\begin{document}
% Title portion. Note the short title for running heads 
\title{PCOT: Cache Oblivious Tiling
of Polyhedral Programs}
%\footnote{New Paper, Not an Extension of a Conference Paper} 
\author{Waruna Ranasinghe}
\author{Nirmal Prajapati}
%\orcid{1234-5678-9012-3456}
\affiliation{%
  \institution{Colorado State University}
  \department{Department of Computer Science}
%  \streetaddress{104 Jamestown Rd}
  \city{Fort Collins}
  \state{CO}
  \postcode{80523}
  \country{USA}}
\author{Tomofumi Yuki}
\affiliation{%
  \institution{INRIA}
  \city{Rennes}
  \country{France}}
\author{Sanjay Rajopadhye}
\affiliation{%
  \institution{Colorado State University}
  \department{Department of Computer Science}
  \city{Fort Collins}
  \state{CO}
  \postcode{80523}
  \country{USA}}
\begin{abstract}
  This paper studies two variants of tiling: iteration space tiling (or loop
  blocking) and cache oblivious methods that recursively split the iteration
  space with divide-and-conquer. The key question to answer is when
  we should be using one over the other. The answer to this question is complicated
  for modern architecture due to a number of reasons.

  In this paper, we present a detailed empirical study to answer this question for a range of kernels
  that fit the polyhedral model. Our study is based on a generalized cache oblivious code generator that
  support this class, which is a superset of those supported by existing tools.
  The conclusion is that cache oblivious code is most useful when the aim is to
  have reduced off-chip memory accesses, e.g., lower energy,
  albeit certain situations that diminish its effectiveness exist. 
\end{abstract}
%
%  This paper marries two important ideas: the \textbf{polyhedral model}, a
%  framework to describe, analyze, transform and compile a class of
%  \textbf{compute-} and \textbf{data-intensive} regular programs, and
%  \textbf{cache-oblivious algorithms/tiling,} a strategy for specifying and/or
%  deriving divide-and-conquer programs that provide provable bounds on the
%  number of cache misses.  Specifically, we present a compiler framework that
%  generates, from a polyhedral input specification, codes with the
%  divide-and-conquer schedule of cache oblivious algorithms.  This is a strict
%  generalization of earlier cache oblivious code generators to a richer class
%  of programs, and of polyhedral compilers to enable them to use a
%  divide-and-conquer schedule, rather than multidimensional affine functions.
%

%
% The code below should be generated by the tool at
% http://dl.acm.org/ccs.cfm
% Please copy and paste the code instead of the example below. 
%
\begin{CCSXML}
<ccs2012>
<concept>
<concept_id>10003752.10003809.10010170.10010171</concept_id>
<concept_desc>Theory of computation~Shared memory algorithms</concept_desc>
<concept_significance>500</concept_significance>
</concept>
<concept>
<concept_id>10010147.10010169.10010170.10010171</concept_id>
<concept_desc>Computing methodologies~Shared memory algorithms</concept_desc>
<concept_significance>500</concept_significance>
</concept>
</ccs2012>
\end{CCSXML}

\ccsdesc[500]{Theory of computation~Shared memory algorithms}
\ccsdesc[500]{Computing methodologies~Shared memory algorithms}

%
% End generated code
%


\keywords{Tiling, Cache Oblivious, Polyhedral Model}


%\thanks{This work is supported by TODO}

%  Author's addresses: G. Zhou, Computer Science Department, College of
%  William and Mary; Y. Wu {and} J. A. Stankovic, Computer Science
%  Department, University of Virginia; T. Yan, Eaton Innovation Center;
%  T. He, Computer Science Department, University of Minnesota; C.
%  Huang, Google; T. F. Abdelzaher, (Current address) NASA Ames
%  Research Center, Moffett Field, California 94035.}


\maketitle

% The default list of authors is too long for headers}
%\renewcommand{\shortauthors}{W. Ranasinghe et al.}

%
\section{Introduction}\label{sec:introduction}
Images captured in weakly illuminated conditions suffer from unfavorable visual quality, which seriously impairs the performance of downstream vision tasks and practical intelligent systems, such as image classification~\cite{Classification}, object detection~\cite{Facedetection}, automatic driving~\cite{autonomousdriving}, and visual navigation~\cite{VisualNavigation}, etc. Numerous advances have been made in improving contrast and restoring details to transform low-light images into high-quality ones. Traditional approaches typically rely on optimization-based rules, with the effectiveness of these methods being highly dependent on the accuracy of hand-crafted priors. However, low-light image enhancement (LLIE) is inherently an ill-posed problem, as it presents difficulties in adapting such priors for various illumination conditions. 

With the rise of deep learning, such problems have been partially solved, learning-based methods directly learn mappings between degraded images and corresponding high-quality sharp versions, thus being more robust than traditional methods. Learning-based methods can be divided into two categories: supervised and unsupervised. The former leverage large-scale datasets and powerful neural network architectures for restoration that aim to optimize distortion metrics such as PSNR and SSIM~\cite{SSIM}, while lacking visual fidelity for human perception. Unsupervised methods have better generalization ability for unseen scenes, benefiting from their label-free characteristics. Nevertheless, they are typically unable to control the degree of enhancement and may produce visually unappealing results in some cases, such as over-enhanced or noise amplification, as with supervised ones. As shown in Fig.~\ref{fig:teaser}, both unsupervised SCI~\cite{SCI} and supervised SNRNet~\cite{SNRNet} appear incorrectly overexposed in the background region.

Therefore, following perceptual-driven approaches~\cite{VGGLoss,perceptual_driven}, deep generative models such as generative adversarial networks (GANs)~\cite{EnlightenGAN,low_light_GAN} and variational autoencoders (VAEs)~\cite{LUD_VAE} have emerged as promising approaches for LLIE. These methods aim to capture multi-modal distributions of inputs to generate output images with better perceptual fidelity. Recently, diffusion models (DMs)~\cite{diffusion_1,diffusion_2,diffusion_3} have garnered attention due to their impressive performance in image synthesis~\cite{ddpm,ddim} and restoration tasks~\cite{Refusion,D2C-SR}. DMs rely on a hierarchical architecture of denoising autoencoders that enable them to iteratively reverse a diffusion process and achieve high-quality mapping from randomly sampled Gaussian noise to target images or latent distributions~\cite{latent_diffusion}, without suffering instability and mode-collapse present in previous generative models. Although standard DMs are capable enough, there exist several challenges for some image restoration tasks, especially for LLIE. As shown in Fig.~\ref{fig:teaser}, the diffusion-based methods Palette~\cite{palette} and WeatherDiff~\cite{weatherdiff} generate images with color distortion or artifacts, since the reverse process starts from a random sampled Gaussian noise, resulting in the final result may have unexpected chaotic content due to the diversity of the sampling process, despite conditional inputs can be used to constrain the output distribution. Moreover, DMs usually require extensive computational resources and long inference times to achieve effective iterative denoising, as shown in Fig.~\ref{fig:efficiency} (a), the diffusion-based methods take more than 10 seconds to restore an image with the size of 600$\times$400.
\begin{figure}[!t]
	\centering
	\includegraphics[width=\linewidth]{Efficiency.pdf}
	\caption{Quantitative comparisons with state-of-the-art methods. (a) presents the performance, i.e., PSNR, and inference time of comparison methods on the LOLv1~\cite{RetinexNet} test set, where the image size is 600$\times$400. (b) presents numerical scores for five perceptual metrics including LPIPS~\cite{LPIPS}, FID~\cite{fid}, NIQE~\cite{NIQE}, BRISQUE~\cite{BRISQUE}, and PI~\cite{PI} (the lower the scores the better the visual quality) on the LOLv1 test set and five real-world benchmarks including DICM~\cite{DICM}, MEF~\cite{MEF}, LIME~\cite{LIME}, NPE~\cite{NPE}, and VV. It can be easily observed that our method is remarkably superior to others, achieving state-of-the-art performance while being computational efficiency.}
	\label{fig:efficiency}
\end{figure}

In this work, we propose a diffusion-based framework, named DiffLL, for robust and efficient low-light image enhancement. Specifically, we first convert the low-light image into the wavelet domain using $K$ times 2D discrete wavelet transformations (2D-DWT), which noticeably reduces the spatial dimension while avoiding information loss, resulting in an average coefficient that represents the global information of the input image and $K$ sets of high-frequency coefficients that represent sparse vertical, horizontal, and diagonal details of the image. Accordingly, we propose a wavelet-based conditional diffusion model (WCDM) that performs diffusion operations on the average coefficient instead of the original image space or latent space for the reduction of computational resource consumption and inference time. As depicted in Fig.~\ref{fig:efficiency} (a), our method achieves a speed that is over \textbf{70$\times$} faster than the previous diffusion-based method DDIM~\cite{ddim}. In addition, during the training phase of the WCDM, we perform both the forward diffusion and the denoising processes. This enables us to fully leverage the generative capacity of the diffusion mode while teaching the model to conduct stable sampling during inference, preventing content diversity even with randomly sampled noise. As shown in Fig.~\ref{fig:teaser}, our method properly improves global contrast and prevents excessive correction on the well-exposed region, without any artifacts or chaotic content appearance. Besides, perceptual scores shown in Fig.~\ref{fig:efficiency} (b) also indicate that our method generates images with better visual quality. Furthermore, the local details contained in the high-frequency coefficients are reconstructed through our well-designed high-frequency restoration modules (HFRM), where the vertical and horizontal information is utilized to complement the diagonal details for better fine-grained details restoration which contributes to the overall quality. Extensive experiments on publicly available datasets show that our method outperforms the existing state-of-the-art (SOTA) methods in terms of both distortion and perceptual metrics, and offers notable improvements in efficiency compared to previous diffusion-based methods. The application for low-light face detection also reveals the potential practical values of our method. Our contributions can be summarized as follows:
\begin{itemize}
	\item We propose a wavelet-based conditional diffusion model (WCDM) that leverages the generative ability of diffusion models and the strengths of wavelet transformation for robust and efficient low-light image enhancement. 
	\item We propose a new training strategy that enables WCDM to achieve content consistency during inference by performing both the forward diffusion and the denoising processes in the training phase. 
	\item We further design a high-frequency restoration module (HFRM) that utilizes both vertical and horizontal information to complement the diagonal details for local details reconstruction.
	\item Extensive experimental results on public real-world benchmarks show that our proposed method achieves state-of-the-art performance on both distortion metrics and perceptual quality while offering noticeable speed up compared to previous diffusion-based methods.
\end{itemize}

\section{Related Work}\label{sec:related_work}
\subsection{Low-light Image Enhancement}
To transform low-light images into visually satisfactory ones, numerous efforts have made considerable advances in improving contrast and restoring details. Traditional methods mainly adopt histogram equalization (HE)~\cite{HE} and Retinex theory~\cite{Retinex} to achieve image enhancement. HE-based methods~\cite{HE1,HE2,HE3} take effect by changing the histogram of the image to improve the contrast. Retinex-based methods~\cite{Retinex_based1,SRIE,Retinex_based2} decompose the image into an illumination map and a reflectance map, by changing the dynamic range of pixels in the illumination map and suppressing the noise in the reflectance map to improve the visual quality of the image. For example, LIME~\cite{LIME} utilized a weighted vibration model based on a prior hypothesis to adjust the estimated illumination, followed by BM3D~\cite{BM3D} as a post-processing step for noise removal.
\begin{figure*}[!t]
	\centering
	\includegraphics[width=\linewidth]{pipeline.pdf}
	\caption{The overall pipeline of our proposed DiffLL. We first convert the low-light image into the wavelet domain using 2D discrete wavelet transformation (2D-DWT) in $K$ times, resulting in an average coefficient $A_{low}^{K}$ and $K$ sets of high-frequency coefficients $\{V_{low}^{k}, H_{low}^{k}, D_{low}^{k}\}$, where $k \in [1, K]$. The proposed wavelet-based conditional model (WCDM) performs diffusion operations on the average coefficient to achieve robust and efficient restoration, in which the training and inference strategies are detailed in ALGORITHM~\ref{algo:1} and ALGORITHM~\ref{algo:2}, respectively. The forward diffusion process is only performed in the training phase and the denoising process is performed in both training and inference phases. Finally, the result at scale $k-1$ is obtained by converting the restored average coefficient and the high-frequency coefficients at scale $k$ using the 2D inverse discrete wavelet transformation (2D-IDWT).}
	\label{fig:pipeline}
\end{figure*}

Following the development of learning-based image restoration methods~\cite{restoration1,restoration2,restoration3}, LLNet~\cite{LLNet} first proposed a deep encoder-decoder network for contrast enhancement and introduced a dataset synthesis pipeline. RetinexNet~\cite{RetinexNet} introduced the first paired dataset captured in real-world scenes and combined Retinex theory with a CNN network to estimate and adjust the illumination map. EnlightenGAN~\cite{EnlightenGAN} adopted unpaired images for training for the first time by using the generative adverse network as the main framework. Zero-DCE~\cite{Zero-DCE} proposed to transform LLIE into a curve estimation problem and designed a zero-reference learning strategy for training. RUAS~\cite{RUAS} proposed a Retinex-inspired architecture search framework to discover low-light prior architectures from a compact search space. SNRNet~\cite{SNRNet} utilized signal-to-noise-ratio-aware transformers and CNN models with spatial-varying operations for restoration. A more comprehensive survey can be found in~\cite{survey}.

\subsection{Diffusion-based Image Restoration}
Diffusion-based generative models have yielded encouraging results with improvements employed in denoising diffusion probability models~\cite{ddpm}, which become more impactful in image restoration tasks such as super-resolution~\cite{IR_SDE,ResDiff}, inpainting~\cite{ddpm_inpainting}, and deblurring~\cite{ddpm_deblurring}. DDRM~\cite{ddrm} used a pre-trained diffusion model to solve any linear inverse problem, which presented superior results over recent unsupervised methods on several image restoration tasks. ShadowDiffusion~\cite{shadowdiffusion} designed an unrolling diffusion model that achieves robust shadow removal by progressively refining the results with degradation and generative priors. DR2~\cite{DR2} first used a pre-trained diffusion model for coarse degradation removal, followed by an enhancement module for finer blind face restoration. WeatherDiff~\cite{weatherdiff} proposed a patch-based diffusion model to restore images captured in adverse weather conditions, which employed a guided denoising process across the overlapping patches in the inference process.

Although diffusion-based methods can obtain restored images with better visual quality than GAN-based and VAE-based generative approaches~\cite{gan_face_restoration,VAE_inpainting}, they often suffer from high computational resource consumption and inference time due to multiple forward and backward passes through the entire network. In this paper, we propose to combine diffusion models with wavelet transformation to noticeably reduce the spatial dimension of inputs in the diffusion process, achieving more efficient and robust low-light image enhancement.

\section{Method}\label{sec:method}
The overall pipeline of our proposed DiffLL is illustrated in Fig. ~\ref{fig:pipeline}. Our approach leverages the generative ability of diffusion models and the strengths of wavelet transformation to achieve visually satisfactory image restoration and improve the efficiency of diffusion models. The wavelet transformation can halve the spatial dimensions after each transformation without sacrificing information, while other transformation techniques, such as Fast Fourier Transformation (FFT) and Discrete Cosine Transform (DCT), are unable to achieve this level of reduction and may result in information loss. Therefore, we conduct diffusion operations on the wavelet domain instead of on the image space. In this section, we begin by presenting the preliminaries of 2D discrete wavelet transformation (2D-DWT) and conventional diffusion models. Then, we introduce the wavelet-based conditional diffusion model (WCDM), which forms the core of our approach. Finally, we present the high-frequency restoration module (HFRM), a well-designed component that contributes to local details reconstruction and improvement in overall quality.
\subsection{Discrete Wavelet Transformation}\label{subsec:Discrete Wavelet Transformation}
Given a low-light image $I_{low} \in \mathbb{R}^{H \times W \times c}$, we use 2D discrete wavelet transformation (2D-DWT) with Haar wavelets~\cite{haar} to transform the input into four sub-bands, i.e.,
\begin{equation}\label{eq:1}
	\{A_{low}^{1}, V_{low}^{1}, H_{low}^{1}, D_{low}^{1}\} = \operatorname{2D-DWT}(I_{low}),
\end{equation}
where $A_{low}^{1}, V_{low}^{1}, H_{low}^{1}, D_{low}^{1} \in \mathbb{R}^{\frac{H}{2} \times \frac{W}{2} \times c}$ represent the average of the input image and high-frequency information in the vertical, horizontal, and diagonal directions, respectively. In particular, the average coefficient contains the global information of the original image, which can be treated as the downsampled version of the image, and the other three coefficients contain sparse local details. As shown in Fig.~\ref{fig:Wavelet_combination}, the images reconstructed by exchanging high-frequency coefficients still have approximately the same content as the original images, whereas the image reconstructed by replacing the average coefficient changes the global information, resulting in the largest error with the original images. Therefore, the primary focus on restoring the low-light image in the wavelet domain is to obtain the average coefficient that has natural illumination consistent with its normal-light counterpart. For this purpose, we utilize the generative capability of diffusion models to restore the average coefficient, and the remaining three high-frequency coefficients are reconstructed through the proposed HFRM to facilitate local details restoration.
\begin{figure}[!t]
	\centering
	\includegraphics[width=\linewidth]{wavelet_combination.pdf}
	\caption{Illustration of exchanging wavelet coefficients between the low-light and normal-light images. Replacing the average coefficient in reconstructed images alters global illumination while exchanging high-frequency coefficients approximately retains the same content as the original images. The quantitative results in mean error also demonstrate that the average coefficient contains richer information than the high-frequency coefficients.}
	\label{fig:Wavelet_combination}
\end{figure}

Although the spatial dimension of the wavelet component processed by the diffusion model after one wavelet transformation is four times smaller than the original image, we further perform $K-1$ times wavelet transformations on the average coefficient, i.e.,
\begin{equation}\label{eq:2}
	\{A_{low}^{k}, V_{low}^{k}, H_{low}^{k}, D_{low}^{k}\} = \operatorname{2D-DWT}(A_{low}^{k-1}),
\end{equation}
where $A_{low}^{k}, V_{low}^{k}, H_{low}^{k}, D_{low}^{k} \in \mathbb{R}^{\frac{H}{2^{k}} \times \frac{W}{2^{k}} \times c}$, $k \in [1, K]$,  and the original input image could be denoted as $A_{low}^{0}$. Subsequently, we do diffusion operations on the $A_{low}^{K}$ to further improve the efficiency, and the high-frequency coefficients $\{V_{low}^{k}, H_{low}^{k}, D_{low}^{k}\}$ are also reconstructed by the HFRM$_{k}$. In this way, our approach achieves significant decreases in inference time and computational resource consumption of the diffusion model due to $4^{K}$ times reduction in the spatial dimension.

\subsection{Conventional Conditional Diffusion Models}\label{subsec:Conventional Conditional Diffusion Models}
Diffusion models~\cite{ddim,ddpm} aim to learn a Markov Chain to gradually transform the distribution of Gaussian noise into the training data, which is generally divided into two phases: the forward diffusion and the denoising. The forward diffusion process first uses a fixed variance schedule $\{\beta_1, \beta_2, \cdots, \beta_T\}$ to progressively transform the input $\mathbf{x}_0$ into corrupted noise data $\mathbf{x}_T \sim \mathcal{N}(\mathbf{0},\mathbf{I})$ through $T$ steps, which can be formulated as:
\begin{equation}\label{eq:3}
	q(\mathbf{x}_{1:T} \mid \mathbf{x}_0)=\prod_{t=1}^T q(\mathbf{x}_t \mid \mathbf{x}_{t-1}),
\end{equation}
\begin{equation}\label{eq:4}
	q(\mathbf{x}_t \mid \mathbf{x}_{t-1})=\mathcal{N}(\mathbf{x}_t ; \sqrt{1-\beta_t} \mathbf{x}_{t-1}, \beta_t \mathbf{I}),
\end{equation}
where $\mathbf{x}_t$ and $\beta_t$ are the corrupted noise data and the pre-defined variance at time-step $t$, respectively, and $\mathcal{N}$ represents the Gaussian distribution.

The inverse process learns the Gaussian denoising transitions starting from the standard normal prior $p(\mathbf{\hat{x}}_T)=\mathcal{N}(\mathbf{\hat{x}}_T;\mathbf{0},\mathbf{I})$, and gradually denoise a randomly sampled Gaussian noise $\mathbf{\hat{x}}_T \sim \mathcal{N}(\mathbf{0},\mathbf{I})$ into a sharp result $\mathbf{\hat{x}}_0$, i.e.,
\begin{equation}\label{eq:5}
	p_\theta(\mathbf{\hat{x}}_{0:T})=p(\mathbf{\hat{x}}_T) \prod_{t=1}^T p_\theta(\mathbf{\hat{x}}_{t-1} \mid \mathbf{\hat{x}}_t).
\end{equation}
By applying the editing and data synthesis capabilities of conditional diffusion models~\cite{conditional_ddpm}, we aim to learn conditional denoising process $p_\theta(\mathbf{\hat{x}}_{0:T} \mid \tilde{\mathbf{x}})$ without changing the forward diffusion for $\mathbf{x}$, which results in high fidelity of the sampled result to the distribution conditioned on the conditional input $\tilde{\mathbf{x}}$. The Eq.\ref{eq:5} is converted to:
\begin{equation}\label{eq:6}
	p_\theta(\mathbf{\hat{x}}_{0:T} \mid \tilde{\mathbf{x}})=p(\mathbf{\hat{x}}_T) \prod_{t=1}^T p_\theta(\mathbf{\hat{x}}_{t-1} \mid \mathbf{\hat{x}}_t, \tilde{\mathbf{x}}),
\end{equation}
\begin{equation}\label{eq:7}
	p_\theta(\mathbf{\hat{x}}_{t-1} \mid \mathbf{\hat{x}}_t)=\mathcal{N}(\mathbf{\hat{x}}_{t-1};\boldsymbol{\mu}_\theta(\mathbf{\hat{x}}_t, t), \sigma_t^2 \mathbf{I}).
\end{equation}
The training objective for diffusion models is to optimize a network $\boldsymbol{\mu}_\theta(\mathbf{x}_t, \tilde{\mathbf{x}}, t)=\frac{1}{\sqrt{\alpha_t}}(\mathbf{x}_t-\frac{\beta_t}{\sqrt{1-\bar{\alpha}_t}} \boldsymbol{\epsilon}_\theta(\mathbf{x}_t, \tilde{\mathbf{x}}, t))$ that predicts $\tilde{\boldsymbol{\mu}}_t$, where the model can predict the noise vector $\boldsymbol{\epsilon}_\theta(\mathbf{x}_t, \tilde{\mathbf{x}}, t)$ by optimizing the network parameters $\theta$ of $\boldsymbol{\epsilon}_\theta$ like~\cite{ddpm}. The objective function is formulated as:
\begin{equation}\label{eq:8}
	\mathcal{L}_{diff} = E_{\mathbf{x}_0, t, \epsilon_t \sim \mathcal{N}(\mathbf{0},\mathbf{I})}[\|\epsilon_t-\epsilon_\theta(\mathbf{x}_t, \tilde{\mathbf{x}}, t)\|^2].
\end{equation}

\subsection{Wavelet-based Conditional Diffusion Models}\label{subsec:Wavelet-based Conditional Diffusion Models}
However, for some image restoration tasks, there are two challenges with the above conventional diffusion and reverse processes: 1) The forward diffusion with large time step $T$, typically set to $T=1000$, and small variance $\beta_t$ permits the assumption that the denoising process becomes close to Gaussian, which results in costly inference time and computational resources consumption. 2) Since the reverse process starts from a randomly sampled Gaussian noise, the diversity of the sampling process may lead to content inconsistency in the restored results, despite the conditioned input enforcing constraints on the distribution of the sampling results.
\begin{algorithm}[!t]
	\caption{Wavelet-based conditional diffusion model training}
	\label{algo:1} 
	\SetKwData{Left}{left}\SetKwData{This}{this}\SetKwData{Up}{up} \SetKwFunction{Union}{Union}\SetKwFunction{FindCompress}{FindCompress} \SetKwInOut{Input}{input}\SetKwInOut{Output}{output}
	
	\Input{Average coefficients of low/normal-light image pairs $A_{low}^{K}$ and $A_{high}^{K}$, denoted as $\tilde{\mathbf{x}}$ and $\mathbf{x}_0$, respectively, the time step $T$, and the number of implicit sampling step $S$.} 
	
	\While{Not converged}{
		
		\emph{\# Forward diffusion process}
		
		$t \sim \operatorname{Uniform}\{1,\cdots,T\}$
		
		$\epsilon_t \sim \mathcal{N}(\mathbf{0},\mathbf{I})$ 
		
		Perform a single gradient descent step for
		$\nabla_\theta\|\boldsymbol{\epsilon}_t-\boldsymbol{\epsilon}_\theta(\sqrt{\bar{\alpha}_t} \mathbf{x}_0+\sqrt{1-\bar{\alpha}_t} \boldsymbol{\epsilon}_t, \tilde{\mathbf{x}}, t)\|^2$
		
		\emph{\# Denoising process}
		
		$\mathbf{\hat{x}}_T \sim \mathcal{N}(\mathbf{0},\mathbf{I})$
		
		\For{$i = S:1$}{ 
			$t = (i-1) \cdot T/S + 1$
			
			$t_{\operatorname{next}} = (i-2) \cdot T/S + 1$ if $i > 1$, else $0$
			
			$\mathbf{\hat{x}}_t \leftarrow \sqrt{\bar{\alpha}_{t_{\operatorname{next}}}}(\frac{\mathbf{\hat{x}}_t-\sqrt{1-\bar{\alpha}_t} \cdot \epsilon_\theta(\mathbf{\hat{x}}_t, \tilde{\mathbf{x}}, t)}{\sqrt{\bar{\alpha}_t}}) +\sqrt{1-\bar{\alpha}_{t_{\operatorname{next}}}} \cdot \epsilon_\theta(\mathbf{\hat{x}}_t, \tilde{\mathbf{x}}, t)$
		}
		
		Perform a single gradient descent step for
		$\nabla_\theta\|\mathbf{\hat{x}}_0 - \mathbf{x}_0\|^2$}
	
	\Output{$\theta$, $\mathbf{\hat{x}}_0$ ($\hat{A}_{low}^{K}$)}
\end{algorithm}
\begin{algorithm}[!t]
	\caption{Wavelet-based conditional diffusion model inference}
	\label{algo:2} 
	\SetKwData{Left}{left}\SetKwData{This}{this}\SetKwData{Up}{up} \SetKwFunction{Union}{Union}\SetKwFunction{FindCompress}{FindCompress} \SetKwInOut{Input}{input}\SetKwInOut{Output}{output}
	
	\Input{Average coefficients of low-light image $A_{low}^{k}$, denoted as $\tilde{\mathbf{x}}$, the time step $T$, and the number of implicit sampling step $S$.} 
	\emph{\# Denoising process}
	
	$\mathbf{\hat{x}}_T \sim \mathcal{N}(\mathbf{0},\mathbf{I})$
	
	\For{$i = S:1$}{ 
		$t = (i-1) \cdot T/S + 1$
		
		$t_{\operatorname{next}} = (i-2) \cdot T/S + 1$ if $i > 1$, else $0$
		
		$\mathbf{\hat{x}}_t \leftarrow \sqrt{\bar{\alpha}_{t_{\operatorname{next}}}}(\frac{\mathbf{\hat{x}}_t-\sqrt{1-\bar{\alpha}_t} \cdot \epsilon_\theta(\mathbf{\hat{x}}_t, \tilde{\mathbf{x}}, t)}{\sqrt{\bar{\alpha}_t}}) +\sqrt{1-\bar{\alpha}_{t_{\operatorname{next}}}} \cdot \epsilon_\theta(\mathbf{\hat{x}}_t, \tilde{\mathbf{x}}, t)$    
	}
	\Output{$\mathbf{\hat{x}}_0$ ($\hat{A}_{low}^{k}$)}
\end{algorithm}

To address these problems, we propose a wavelet-based conditional diffusion model (WCDM) that converts the input low-light image $I_{low}$ into the wavelet domain and performs the diffusion process on the average coefficient, which considerably reduces the spatial dimension for efficient restoration. In addition, we perform both the forward diffusion and the denoising processes in the training phase, which aids the model in achieving stable sampling during inference and avoiding content diversity. The training approach of our wavelet-based conditional diffusion model is summarized in Algorithm~\ref{algo:1}. Specifically, we first perform the forward diffusion to optimize a noise estimator network via Eq.\ref{eq:8}, then adopt the sampled noise $\mathbf{\hat{x}}_T$ with conditioned input $\tilde{\mathbf{x}}$, i.e., the average coefficient $A_{low}^{K}$, for the denoising process, resulting in the restored coefficient $\hat{A}_{low}^{K}$. The content consistency is realized by minimizing the L2 distance between the restored coefficient and the reference coefficient $A_{high}^{K}$ which is only available during the training phase. Accordingly, the objective function Eq.\ref{eq:8} used to optimize the diffusion model is rewritten as:
\begin{equation}\label{eq:9}
	\mathcal{L}_{diff} = E_{\mathbf{x}_0, t, \epsilon_t \sim \mathcal{N}(\mathbf{0},\mathbf{I})}[\|\epsilon_t-\epsilon_\theta(\mathbf{x}_t, \tilde{\mathbf{x}}, t)\|^2] + \|\hat{A}_{low}^{K} - A_{high}^{K}\|^2.
\end{equation}
\begin{figure}[!t]
	\centering
	\includegraphics[width=\linewidth]{HFRM.pdf}
	\caption{The detailed architecture of our proposed high-frequency restoration module. Depth Conv denotes the depth-wise separable convolution.}
	\label{fig:HFRM_architecture}
\end{figure}

{\textbf{Discussion.} There exist some diffusion models, such as Latent Diffusion~\cite{latent_diffusion}, that adopt VAE to transform the original image into the feature level and perform diffusion operations in the latent space, which can also achieve spatial dimension reduction as our method. However, encoding the original image through the VAE encoder inevitably causes information loss, whereas the wavelet transformation offers the advantage of preserving all information without any sacrifice. In addition, utilizing VAE would introduce an increase in model parameters, whereas the wavelet transformation, being a linear operation, does not incur additional resource consumption. On the other hand, the data typically used for training VAE are captured under normal-light conditions, it should be retrained on the low-light dataset to prevent domain shifts and maintain effective generation capability.}

\subsection{High-Frequency Restoration Module}\label{subsec:High-Frequency Restoration Module}
For the $k$-th level wavelet sub-bands of the low-light image, the high-frequency coefficients $V_{low}^{k}$, $H_{low}^{k}$, and $D_{low}^{k}$ hold sparse representations of vertical, horizontal, and diagonal details of the image. To restore the low-light image to contain as rich information as the normal-light image, we propose a high-frequency restoration module (HFRM) to reconstruct these coefficients. As shown in Fig.~\ref{fig:HFRM_architecture}, we first use three depth-wise separable convolutions~\cite{depth_conv} for the sake of efficiency to extract the features of input coefficients, then two cross-attention layers~\cite{cross_attention} are employed to leverage the information in $V$ and $H$ to complement the details in $D$. Subsequently, inspired by~\cite{SFDNet}, we design a progressive dilation Resblock utilizing dilation convolutions for local restoration, where the first and last convolutions are utilized to extract local information, and the middle dilation convolutions are used to improve the receptive field to make better use of long-range information. By gradually increasing and decreasing the dilation rate $d$, the gridding effect can be avoided. Finally, three depth-wise separable convolutions are used to reduce channels to obtain the reconstructed $\hat{V}_{low}^{k}$, $\hat{H}_{low}^{k}$, and $\hat{D}_{low}^{k}$. The restored average coefficient and the high-frequency coefficients at scale $k$ are correspondingly converted into the output at scale $k-1$ by employing the 2D inverse discrete wavelet transformation (2D-IDWT), i.e.,
\begin{equation}\label{eq:10}
	\hat{A}_{low}^{k-1} = \operatorname{2D-IDWT}(\{\hat{A}_{low}^{k}, \hat{V}_{low}^{k}, \hat{H}_{low}^{k}, \hat{D}_{low}^{k}\}),
\end{equation}
where the $\hat{A}_{low}^{0}$ denotes the final restored image $\hat{I}_{low}$.
\begin{table*}[!t]
	\centering
	\caption{Quantitative evaluation of different methods on LOL-v1~\cite{RetinexNet}, LOLv2-real~\cite{LOLV2}, and LSRW~\cite{R2RNet} test sets. The best results are highlighted in \textbf{bold} and the second best results are \underline{underlined}. $\ast$ denotes the methods are retrained on the LOLv1 training set.}
	\resizebox{\linewidth}{!}{
		\begin{tabular}{l|c|cccc|cccc|cccc}
			\toprule
			\multirow{2}[4]{*}{Methods} & \multirow{2}[4]{*}{Reference} & \multicolumn{4}{c|}{LOLv1} & \multicolumn{4}{c|}{LOLv2-real} & \multicolumn{4}{c}{LSRW} \\
			\cmidrule{3-14} & & PSNR $\uparrow$ & SSIM $\uparrow$ & LPIPS $\downarrow$ & FID $\downarrow$ & PSNR $\uparrow$ & SSIM $\uparrow$ & LPIPS $\downarrow$ & FID $\downarrow$ & PSNR $\uparrow$ & SSIM $\uparrow$ & LPIPS $\downarrow$ & FID $\downarrow$ \\
			\midrule
			NPE & TIP' 13 & 16.970 & 0.484 & 0.400 & 104.057 & 17.333 & 0.464 & 0.396 & 100.025 & 16.188 & 0.384 & 0.440 & 90.132 \\
			SRIE & CVPR' 16 & 11.855 & 0.495 & 0.353 & 88.728 & 14.451 & 0.524 & 0.332 & 78.834 & 13.357 & 0.415 & \underline{0.399} & 69.082 \\
			LIME & TIP' 16 & 17.546 & 0.531 & 0.387 & 117.892 & 17.483 & 0.505 & 0.428 & 118.171 & 17.342 & 0.520 & 0.471 & 75.595 \\
			RetinexNet & BMVC' 18 & 16.774 & 0.462 & 0.417 & 126.266 & 17.715 & 0.652 & 0.436 & 133.905 & 15.609 & 0.414 & 0.454 & 108.350 \\
			DSLR & TMM' 20 & 14.816 & 0.572 & 0.375 & 104.428 & 17.000 & 0.596 & 0.408 & 114.306 & 15.259 & 0.441 & 0.464 & 84.930 \\
			DRBN & CVPR' 20 & 16.677 & 0.730 & 0.345 & 98.732 & 18.466 & 0.768 & 0.352 & 89.085 & 16.734 & 0.507 & 0.457 & 80.727 \\
			Zero-DCE & CVPR' 20 & 14.861 & 0.562 & 0.372 & 87.238 & 18.059 & 0.580 & 0.352 & 80.449 & 15.867 & 0.443 & 0.411 & 63.320 \\
			MIRNet & ECCV' 20 & 24.138 & 0.830 & 0.250 & 69.179 & 20.020 & 0.820 & 0.233 & \underline{49.108} & 16.470 & 0.477 & 0.430 & 93.811 \\
			EnlightenGAN & TIP' 21 & 17.606 & 0.653 & 0.372 & 94.704 & 18.676 & 0.678 & 0.364 & 84.044 & 17.106 & 0.463 & 0.406 & 69.033 \\
			ReLLIE & ACM MM' 21& 11.437 & 0.482 & 0.375 & 95.510 & 14.400 & 0.536 & 0.334 & 79.838 & 13.685 & 0.422 & 0.404 & 65.221 \\
			RUAS & CVPR' 21 & 16.405 & 0.503 & 0.364 & 101.971 & 15.351 & 0.495 & 0.395 & 94.162 & 14.271 & 0.461 & 0.501 & 78.392 \\
			DDIM$^{\ast}$ & ICLR' 21 & 16.521 & 0.776 & 0.376 & 84.071 & 15.280 & 0.788 & 0.387 & 76.387 & 14.858 & 0.486 & 0.495 & 71.812 \\
			CDEF & TMM' 22 & 16.335 & 0.585 & 0.407 & 90.620 & 19.757 & 0.630 & 0.349 & 74.055 & 16.758 & 0.465 & \underline{0.399} & 62.780 \\
			SCI & CVPR' 22 & 14.784 & 0.525 & 0.366 & 78.598 & 17.304 & 0.540 & 0.345 & 67.624 & 15.242 & 0.419 & 0.404 & \underline{56.261} \\
			URetinex-Net & CVPR' 22 & 19.842 & 0.824 & 0.237 & \underline{52.383} & 21.093 & \underline{0.858} & \underline{0.208} & 49.836 & \underline{18.271} & 0.518 & 0.419 & 66.871 \\
			SNRNet & CVPR' 22 & \underline{24.610} & \underline{0.842} & \underline{0.233} & 55.121 & 21.480 & 0.849 & 0.237 & 54.532 & 16.499 & 0.505 & 0.419 & 65.807 \\
			Uformer$^{\ast}$ & CVPR' 22 & 19.001 & 0.741 & 0.354 & 109.351 & 18.442 & 0.759 & 0.347 & 98.138 & 16.591 & 0.494 & 0.435 & 82.299 \\
			Restormer$^{\ast}$ & CVPR' 22 & 20.614 & 0.797 & 0.288 & 72.998 & \underline{24.910} & 0.851 & 0.264 & 58.649 & 16.303 & 0.453 & 0.427 & 69.219 \\
			Palette$^{\ast}$ & SIGGRAPH' 22 & 11.771 & 0.561 & 0.498 & 108.291 & 14.703 & 0.692 & 0.333 & 83.942 & 13.570 & 0.476 & 0.479 & 73.841 \\
			UHDFour$_{2\times}$ & ICLR' 23 & 23.093 & 0.821 & 0.259 & 56.912 & 21.785 & 0.854 & 0.292 & 60.837 & 17.300 & \underline{0.529} & 0.443 & 62.032 \\
			%\cdashline{1-14}[2pt/1pt]
			WeatherDiff$^{\ast}$ & TPAMI' 23 & 17.913 & 0.811 & 0.272 & 73.903 & 20.009 & 0.829 & 0.253 & 59.670 & 16.507 & 0.487 & 0.431 & 96.050 \\
			GDP & CVPR' 23 & 15.896 & 0.542 & 0.421 & 117.456 & 14.290 & 0.493 & 0.435 & 102.416 & 12.887 & 0.362 & 0.412 & 76.908 \\
			DiffLL(Ours) & - & \textbf{26.336} & \textbf{0.845} & \textbf{0.217} & \textbf{48.114} & \textbf{28.857} & \textbf{0.876} & \textbf{0.207} & \textbf{45.359} & \textbf{19.281} & \textbf{0.552} & \textbf{0.350} & \textbf{45.294} \\
			\bottomrule
	\end{tabular}}
	\label{tab:evaluation_paired}
\end{table*}

\subsection{Network Training}\label{subsec:Network Training}
Besides the objective function $\mathcal{L}_{diff}$ used to optimize the diffusion model, we also use a detail preserve loss $\mathcal{L}_{detail}$ combing MSE loss and TV loss~\cite{TV} similar to~\cite{RUAS} to reconstruct the high-frequency coefficients, i.e.,
\begin{equation}\label{eq:11}
	\begin{aligned}
		\mathcal{L}_{detail} &= \lambda_{1} \sum_{k=1}^{K} \|\{\hat{V}_{low}^{k}, \hat{H}_{low}^{k}, \hat{H}_{low}^{k}\} - \{V_{high}^{k}, H_{high}^{k}, D_{high}^{k}\}\|^2 \\& + \lambda_{2} \sum_{k=1}^{K} \operatorname{TV}(\{\hat{V}_{low}^{k}, \hat{H}_{low}^{k}, \hat{H}_{low}^{k}\}),
	\end{aligned}
\end{equation}
where $\lambda_{1}$ and $\lambda_{2}$ are the weights of each term set as 0.1 and 0.01, respectively. Moreover, we utilize a content loss $\mathcal{L}_{content}$ that combines L1 loss and SSIM loss~\cite{SSIM} to minimize the content difference between the restored image $\hat{I}_{low}$ and the reference image $I_{high}$, i.e.,
\begin{equation}\label{eq:12}
	\mathcal{L}_{content} = |\hat{I}_{low} - I_{high}|_{1} + (1 - \operatorname{SSIM}(\hat{I}_{low}, I_{high})).
\end{equation}
The total loss $\mathcal{L}_{total}$ is expressed by combing the diffusion objective function, the detail preserve loss, and the content loss as:
\begin{equation}\label{eq:13}
	\mathcal{L}_{total} = \mathcal{L}_{diff} + \mathcal{L}_{detail} + \mathcal{L}_{content}.
\end{equation}

\section{Experiment}\label{sec:experiment}
\subsection{Experimental Settings}\label{subsec:Experimental Settings}
\textbf{Implementation Details.} Our implementation is done with PyTorch. The proposed network can be converged after being trained for $1\times10^{5}$ iterations on four NVIDIA RTX 2080Ti GPUs. The Adam optimizer~\cite{Adam} is adopted for optimization. The initial learning rate is set to $1\times10^{-4}$ and decays by a factor of 0.8 after every $5\times10^{3}$ iterations. The batch size and patch size are set to 12 and 256$\times$256, respectively. The wavelet transformation scale $K$ is set to 2. For our wavelet-based conditional diffusion model, the commonly used U-Net architecture~\cite{Unet} is adopted as the noise estimator network. To achieve efficient restoration, the time step $T$ is set to 200 for the training phase, and the implicit sampling step $S$ is set to 10 for both the training and inference phases. 

\textbf{Datasets.} Our network is trained and evaluated on the LOLv1 dataset~\cite{RetinexNet}, which contains 1,000 synthetic low/normal-light image pairs and 500 real-word pairs, where 15 real-world pairs are adopted for evaluation and the remaining pairs are used for training. Additionally, we adopt 2 real-world paired datasets, including LOLv2-real~\cite{LOLV2} and LSRW~\cite{R2RNet}, to evaluate the performance of the proposed method. More specifically, the LSRW dataset contains 5,650 image pairs captured in various scenes, 5,600 image pairs are randomly selected as the training set and the remaining 50 image pairs are used for evaluation. The LOLv2-real dataset contains 789 image pairs, of which 689 pairs are for training and 100 pairs for evaluation. To validate the effectiveness of our method for high-resolution image restoration, we adopt the UHD-LL dataset~\cite{UHD_ICLR} for evaluation which contains 2,150 low/normal-light Ultra-High-Definition (UHD) image pairs with 4K resolution, where 150 pairs are used for evaluation and the rest for training. Furthermore, we evaluate the generalization ability of the proposed method to unseen scenes on five commonly used real-world unpaired benchmarks, including DICM~\cite{DICM}, LIME~\cite{LIME}, NPE~\cite{NPE}, MEF~\cite{MEF}, and VV\footnote{https://sites.google.com/site/vonikakis/datasets}.

\textbf{Metrics.} For the paired datasets, we adopt two full-reference distortion metrics PSNR and SSIM~\cite{SSIM} to evaluate the performance of the proposed method, and also two perceptual metrics LPIPS~\cite{LPIPS} and FID~\cite{fid} to measure the visual quality of the enhanced results. For the other five unpaired datasets, we use three non-reference perceptual metrics NIQE~\cite{NIQE}, BRISQUE~\cite{BRISQUE}, and PI~\cite{PI} for evaluation.

\subsection{Comparison with Existing Methods}\label{subsec:Comparison with Existing Methods}
\textbf{Comparison Methods.} In this section, we compare our proposed DiffLL with four categories of existing state-of-the-art methods, including 1) optimization-based LLIE methods NPE~\cite{NPE}, SRIE~\cite{SRIE}, LIME~\cite{LIME}, and CDEF~\cite{CDEF}, 2) learning-based LLIE methods RetinexNet~\cite{RetinexNet}, DSLR~\cite{DSLR}, DRBN~\cite{DRBN}, Zero-DCE~\cite{Zero-DCE}, MIRNet~\cite{MIRNet}, EnlightenGAN~\cite{EnlightenGAN}, ReLLIE~\cite{ReLLIE}, RUAS~\cite{RUAS},  SCI~\cite{SCI}, URetinex-Net~\cite{Uretinex-net}, SNRNet~\cite{SNRNet}, and UHDFour~\cite{UHD_ICLR}, 3) transformer-based image restoration methods Uformer~\cite{Uformer} and Restormer~\cite{Restormer}, and 4) diffusion-based methods conditional DDIM~\cite{ddim}, Palette~\cite{palette}, WeatherDiff~\cite{weatherdiff}, and GDP~\cite{GDP}. For fair comparisons, we retrain the transformer-based and diffusion-based methods, except GDP, on the LOLv1 training set using their default implementations, denoted by $^{\ast}$. UHDFour provides two versions for low-resolution and UHD LLIE, which downsample the original input image by a factor of 2 and 8 for efficient restoration, denoted as UHDFour$_{2\times}$ and UHDFour$_{8\times}$, and are trained on the LOLv1 and UHD-LL training sets, respectively. Therefore, we mainly report the performance of the UHDFour$_{2\times}$ version for fair comparisons. Furthermore, the metrics of the diffusion-based methods and our method are the mean values for five times evaluation.

\textbf{Quantitative Comparison.} We first compare our method with all comparison methods on the LOLv1~\cite{RetinexNet}, LOLv2-real~\cite{LOLV2}, and LSRW~\cite{R2RNet} test sets. As shown in Table~\ref{tab:evaluation_paired}, our method achieves SOTA quantitative performance compared with all comparison methods. Specifically, for the distortion metrics, our method achieves 1.726dB(=26.336-24.610) and 0.003(=0.845-0.842) improvements in terms of PSNR and SSIM on the LOLv1 test set compared with the second-best method SNRNet. On the LOLv2-real test set, our DiffLL obtains PSNR and SSIM of 28.857dB and 0.876 that greatly outperform the second-best approaches, i.e., Restormer and URetinex-Net, by 3.947dB(=28.857-24.910) and 0.018(0.876-0.858). On the LSRW test set, our method improves the two metrics by at least 1.01dB(=19.281-18.271) and 0.023(0.552-0.529), respectively. For perceptual metrics, i.e., LPIPS and FID, where regression-based methods and previous diffusion-based methods perform unfavorably, our method yields the lowest perceptual scores on all three datasets, which demonstrates our proposed wavelet-based diffusion model can produce restored images with satisfactory visual quality and is more appropriate for the LLIE task. In particular, for FID, most methods are unable to obtain stable metrics due to the cross-domain issue between training and test data, while on the contrary, our method obtains FID scores less than 50 on all three datasets, proving that our method can generalize well to unknown datasets.
\begin{table}[!t]
	\centering
	\caption{The average time (seconds) and GPU memory (G) costs of different methods consumed on the image with the size of 600$\times$400, 1920$\times$1080 (1080P), and 2560$\times$1440 (2K), respectively, during inference. OOM denotes the out-of-memory error and `-’ denotes unavailable.}
	\resizebox{\linewidth}{!}{
		\begin{tabular}{l|cc|cc|cc}
			\toprule
			\multirow{2}[2]{*}{Methods} & \multicolumn{2}{c|}{600$\times$400} & \multicolumn{2}{c|}{1920$\times$1080 (1080P)} & \multicolumn{2}{c}{2560$\times$1440 (2K)} \\
			& Mem.(G) $\downarrow$ & Time (s) $\downarrow$ & Mem.(G) $\downarrow$ & Time (s) $\downarrow$ & Mem.(G) $\downarrow$ & Time (s) $\downarrow$ \\
			\midrule
			MIRNet & 2.699 & 0.643 & OOM & - & OOM & - \\
			URetinex-Net & 1.656 & 0.066 & 5.416 & 0.454 & 8.699 & 0.814 \\
			SNRNet & 1.523 & 0.072 & 8.125 & 0.601 & OOM & - \\
			Uformer & 4.457 & 0.901 & OOM & - & OOM & - \\
			Restormer & 4.023 & 0.513 & OOM & - & OOM & - \\
			DDIM & 5.881 & 12.138 & OOM & - & OOM & - \\
			Palette & 6.830 & 168.515 & OOM & - & OOM & - \\
			UHDFour$_{2\times}$ & 2.228 & 0.062 & 8.263 & 0.533 & OOM & - \\
			UHDFour$_{8\times}$ & 1.423 & 0.011 & 2.017 & 0.057 & 4.956 & 0.105 \\
			WeatherDiff & 6.570 & 52.703 & 8.344 & 548.708 & OOM & - \\
			DiffLL(Ours) & 1.850 & 0.157 & 3.873 & 1.072 & 7.141 & 2.403 \\
			\bottomrule
	\end{tabular}}
	\label{tab:inference_time}
\end{table}

Besides the enhanced visual quality, efficiency is also an important indicator for image restoration tasks. We report the average time and GPU memory costs of competitive methods and our method during inference on the LOLv1 test set in Table~\ref{tab:inference_time}, where the image size is 600$\times$400. All methods are tested on an NVIDIA RTX 2080Ti GPU. We can see that our method achieves a better trade-off in efficiency and performance, averagely spending 0.157s and 1.850G memory on each image which is at a moderate level in all compared methods. Benefiting from the proposed wavelet-based diffusion model, our method is at least 70$\times$ faster than the previous diffusion-based restoration approaches and consumes at least 3$\times$ fewer computational resources. We also report the evaluation of the inference time and GPU memory consumed on the image with the size of 1920$\times$1080 (1080P) and 2560$\times$1440 (2K), where most comparison methods encounter out-of-memory errors, especially at 2K resolution. In contrast, our method spends much fewer GPU resources, which proves the potential value of our approach for large-resolution LLIE tasks.
\begin{table}[!t]
	\centering
	\caption{Quantitative evaluation of different methods on the UHD-LL~\cite{UHD_ICLR} test set. The best results are highlighted in \textbf{bold} and the second best results are \underline{underlined}. Note that UHDFour$_{2\times}$ is trained on the LOLv1~\cite{RetinexNet} training set like our method, and the methods indicated with ${\ast}$ are also retrained on it. On the other hand, UHDFour$_{8\times}$ is trained on the UHD-LL training set, which brings it stronger capability to handle ultra-high-definition image restoration. The resolution indicates the maximum size that the models can handle during inference.}
	\resizebox{0.88\linewidth}{!}{
		\begin{tabular}{l|c|ccccc}
			\toprule
			Methods & Resolution & PSNR $\uparrow$ & SSIM $\uparrow$ & LPIPS $\downarrow$ & FID $\downarrow$ \\
			\midrule
			LIME & 3840$\times$2160 & 17.361 & 0.499 & 0.455 & 52.494 \\
			DRBN & 3840$\times$2160 & 16.517 & 0.661 & 0.456 & 83.900 \\
			Zero-DCE & 3840$\times$2160 & 17.088 & 0.629 & 0.510 & 54.251 \\
			MIRNet & 1280$\times$720 & 19.395 & 0.770 & 0.403 & 62.390 \\
			EnlightenGAN & 3840$\times$2160 & 17.391 & 0.666 & 0.473 & 64.788 \\
			RUAS & 3840$\times$2160 & 11.765 & 0.647 & 0.491 & 89.438 \\
			DDIM$^{\ast}$ & 960$\times$540 & 16.820 & 0.699 & 0.445 & 59.573 \\
			SCI & 3840$\times$2160 & 15.467 & 0.586 & 0.524 & 44.093 \\
			URetinex-Net & 2560$\times$1440 & \underline{20.966} & 0.762 & \underline{0.368} & \underline{38.780} \\
			SNRNet & 1920$\times$1080 & 15.967 & 0.730 & 0.424 & 60.009 \\
			Uformer$^{\ast}$ & 960$\times$540 & 18.112 & \underline{0.793} & 0.421 & 75.229 \\
			Restormer$^{\ast}$ & 960$\times$540 & 18.072 & 0.760 & 0.446 & 53.972 \\
			Palette$^{\ast}$ & 720$\times$480 & 9.973 & 0.540 & 0.729 & 86.313 \\
			UHDFour$_{2\times}$ & 1920$\times$1080 & 11.958 & 0.459 & 0.530 & 128.927 \\
			WeatherDiff$^{\ast}$ & 1920$\times$1080 & 16.693 & 0.769 & 0.403 & 102.380 \\
			DiffLL(Ours) & 2560$\times$1440 & \textbf{21.356} & \textbf{0.803} & \textbf{0.356} & \textbf{32.686} \\
			\midrule
			UHDFour$_{8\times}$ & 3840$\times$2160 & \textbf{26.226} & \textbf{0.872} & \textbf{0.239} & \textbf{20.851} \\
			\bottomrule
	\end{tabular}}
	\label{tab:quantitative_UHD}
\end{table}

To validate that, we compare the proposed method with competitive methods that perform well in small-resolution paired datasets on the UHD-LL~\cite{UHD_ICLR} test set, including LIME, DRBN, Zero-DCE, MIRNet, EnlightenGAN, RUAS, SCI, URetinex-Net, SNRNet, Uformer, Restormer, UHDFour, DDIM, Palette, WeatherDiff, and GDP. As mentioned in Table~\ref{tab:inference_time}, most methods are unable to handle large-resolution images, we, therefore, follow~\cite{UHD_ICLR} to downsample the input to the maximum size that the models can handle and then resize the results to the original resolution for performance evaluation. As shown in Table~\ref{tab:quantitative_UHD}, our method achieves state-of-the-art quantitative performance compared with all comparison methods trained on the LOLv1 training set. In terms of distortion metrics, our method outperforms the second-best models, URetinex-Net and Uformer, by 0.390dB(=21.356-20.966) and 0.010(=0.803-0.793) in PSNR and SSIM, respectively. For perceptual metrics, i.e., LPIPS and FID, our method yields the lowest perceptual scores among all comparison methods, indicating our ability to generate images with satisfactory visual quality. Even when compared with UHDFour$_{8\times}$, which is specifically designed for the UHD LLIE task and trained on the UHD-LL training set, our method does not present significant inferiority. These findings demonstrate the generalization ability of our method for high-resolution low-light image restoration.
\begin{table*}[!t]
	\centering
	\caption{Quantitative comparison of different methods on DICM~\cite{DICM}, MEF~\cite{MEF}, LIME~\cite{LIME}, NPE~\cite{NPE}, and VV datasets. The best results are highlighted in \textbf{bold} and the second best results are \underline{underlined}. $\ast$ denotes the methods are retrained on the LOLv1~\cite{RetinexNet} training set and BRI. denotes BRISQUE~\cite{BRISQUE}.}
	\resizebox{\linewidth}{!}{
		\begin{tabular}{l|ccc|ccc|ccc|ccc|ccc|ccc}
			\toprule
			\multirow{2}[4]{*}{Methods} & \multicolumn{3}{c|}{DICM} & \multicolumn{3}{c|}{MEF} & \multicolumn{3}{c|}{LIME} & \multicolumn{3}{c|}{NPE} & \multicolumn{3}{c|}{VV} & \multicolumn{3}{c}{AVG.} \\
			\cmidrule{2-19}      & NIQE $\downarrow$ & BRI. $\downarrow$ & PI $\downarrow$ & NIQE $\downarrow$ & BRI. $\downarrow$ & PI $\downarrow$ & NIQE $\downarrow$ & BRI. $\downarrow$ & PI $\downarrow$ & NIQE $\downarrow$ & BRI. $\downarrow$ & PI $\downarrow$ & NIQE $\downarrow$ & BRI. $\downarrow$ & PI $\downarrow$ & NIQE $\downarrow$ & BRI. $\downarrow$ & PI $\downarrow$ \\
			\midrule
			LIME & 4.476 & 27.375 & 4.216 & 4.744 & 39.095 & 5.160 & 5.045 & 32.842 & 4.859 & 4.170 & 28.944 & 3.789 & 3.713 & 18.929 & 3.335 & 4.429 & 29.437 & 4.272 \\
			DRBN & 4.369 & 30.708 & 3.800 & 4.869 & 44.669 & 4.711 & 4.562 & 34.564 & 3.973 & 3.921 & 25.336 & 3.267 & 3.671 & 24.945 & 3.117 & 4.278 & 32.045 & 3.774 \\
			Zero-DCE & 3.951 & 23.350 & 3.149 & \underline{3.500} & 29.359 & \textbf{2.989} & 4.379 & 26.054 & \underline{3.239} & 3.826 & 21.835 & 2.918 & 5.080 & 21.835 & 3.307 & 4.147 & 24.487 & 3.120 \\
			MIRNet & 4.021 & 22.104 & 3.691 & 4.202 & 34.499 & 3.756 & 4.378 & 28.623 & 3.398 & 3.810 & 23.658 & 3.205 & 3.548 & 22.897 & 2.911 & 3.992 & 26.356 & 3.392 \\
			EnlightenGAN & 3.832 & 19.129 & 3.256 & 3.556 & 26.799 & 3.270 & \underline{4.249} & 22.664 & 3.381 & 3.775 & 21.157 & 2.953 & 3.689 & \textbf{14.153} & 2.749 & 3.820 & 20.780 & 3.122 \\
			RUAS & 7.306 & 46.882 & 5.700 & 5.435 & 42.120 & 4.921 & 5.322 & 34.880 & 4.581 & 7.198 & 48.976 & 5.651 & 4.987 & 35.882 & 4.329 & 6.050 & 41.748 & 5.036 \\ 
			DDIM$^{\ast}$ & 3.899 & 19.787 & 3.213 & 3.621 & 28.614 & 3.376 & 4.399 & 24.474 & 3.459 & 3.679 & 18.996 & 2.870 & 3.640 & 16.879 & 2.669 & 3.848 & 21.750 & 3.118 \\
			SCI & 4.519 & 27.922 & 3.700 & 3.608 & 26.716 & 3.286 & 4.463 & 25.170 & 3.376 & 4.124 & 28.887 & 3.534 & 5.312 & 22.800 & 3.648 & 4.405 & 26.299 & 3.509 \\
			URetinex-Net & 4.774 & 24.544 & 3.565 & 4.231 & 34.720 & 3.665 & 4.694 & 29.022 & 3.713 & 4.028 & 26.094 & 3.153 & 3.851 & 22.457 & 2.891 & 4.316 & 27.368 & 3.397 \\
			SNRNet & \underline{3.804} & 19.459 & 3.285 & 4.063 & 28.331 & 3.753 & 4.597 & 29.023 & 3.677 & 3.940 & 28.419 & 3.278 & 3.761 & 23.672 & 2.903 & 4.033 & 25.781 & 3.379 \\
			Uformer$^{\ast}$ & 3.847 & 19.657 & 3.180 & 3.935 & \underline{25.240} & 3.582 & 4.300 & 21.874 & 3.565 & \underline{3.510} & 16.239 & 2.871 & \textbf{3.274} & 19.491 & \underline{2.618} & \underline{3.773} & 20.500 & 3.163 \\
			Restormer$^{\ast}$ & 3.964 & 19.474 & 3.152 & 3.815 & 25.322 & 3.436 & 4.365 & 22.931 & 3.292 & 3.729 & 16.668 & \underline{2.862} & 3.795 & 16.974 & 2.712 & 3.934 & 20.274 & 3.091 \\
			Palette$^{\ast}$ & 4.118 & \underline{18.732} & 3.425 & 4.459 & 25.602 & 4.205 & 4.485 & 20.551 & 3.579 & 3.777 & 16.006 & 3.018 & 3.847 & 16.106 & 2.986 & 4.137 & \underline{19.399} & 3.443 \\
			UHDFour$_{2\times}$ & 4.575 & 26.926 & 3.684 & 4.231 & 29.538 & 4.124 & 4.430 & \underline{20.263} & 3.813 & 4.049 & \underline{15.934} & 3.135 & 3.867 & 15.297 & 2.894 & 4.230 & 21.592 & 3.530 \\
			WeatherDiff$^{\ast}$ & \textbf{3.773} & 20.387 & \underline{3.130} & 3.753 & 30.480 & 3.312 & 4.312 & 28.090 & 3.424 & 3.677 & 20.262 & 2.878 & \underline{3.472} & 18.070 & 2.656 & 3.797 & 23.458 & \underline{3.080} \\
			GDP & 4.358 & 19.294 & 3.552 & 4.609 & 34.859 & 4.115 & 4.891 & 27.460 & 3.694 & 4.032 & 19.527 & 3.097 & 4.683 & 20.910 & 3.431 & 4.514 & 24.410 & 3.578 \\
			DiffLL(Ours) & 3.806 & \textbf{18.584} & \textbf{3.011} & \textbf{3.427} & \textbf{24.165} & \underline{3.011} & \textbf{3.777} & \textbf{19.843} & \textbf{3.074} & \textbf{3.427} & \textbf{15.789} & \textbf{2.597} & 3.507 & \underline{14.644} & \textbf{2.562} & \textbf{3.589} & \textbf{18.605} & \textbf{2.851} \\
			\bottomrule
	\end{tabular}}
	\label{tab:evaluation_unpaired}
\end{table*}
\begin{figure*}[!t]
	\centering
	\includegraphics[width=\linewidth]{visual_paired.pdf}
	\caption{Qualitative comparison of our method and competitive methods on the LOLv1~\cite{RetinexNet} (row 1), LOLv2-real~\cite{LOLV2} (row 2), and LSRW~\cite{R2RNet} (row 3) test sets. Error-prone regions are highlighted with red boxes, best viewed by zooming in.}
	\label{fig:visual_compare_paired}
\end{figure*}
\begin{figure*}[!t]
	\centering
	\includegraphics[width=\linewidth]{visual_UHD.pdf}
	\caption{Qualitative comparison of our method and competitive methods on the UHD-LL~\cite{UHD_ICLR} test set. Error-prone regions are highlighted with red boxes, best viewed by zooming in. Note that UHDFour$_{8\times}$ is trained on the UHD-LL training set, while other comparison supervised methods and our method are trained on the simpler LOLv1~\cite{RetinexNet} training set.}
	\label{fig:visual_UHD}
\end{figure*}
\begin{figure*}[!t]
	\centering
	\includegraphics[width=\linewidth]{visual_unpaired.pdf}
	\caption{Qualitative comparison of our method and competitive methods on the DICM~\cite{DICM} (row 1), MEF~\cite{MEF} (row 2), and NPE~\cite{NPE} (row 3) datasets. Error-prone regions are highlighted with red boxes, best viewed by zooming in.}
	\label{fig:visual_compare_unpaired}
\end{figure*}

Moreover, we also compared the proposed DiffLL with competitive methods on five unpaired datasets to further validate the effectiveness of our method. Three non-reference perceptual metrics NIQE~\cite{NIQE}, BRISQUE~\cite{BRISQUE}, and PI~\cite{PI} are utilized to evaluate the visual quality of the enhanced results, the lower the metrics the better the visual quality. As shown in Table~\ref{tab:evaluation_unpaired}, our method achieves the lowest NIQE score on the MEF, LIME, and NPE datasets, and is comparable to the best methods on the DICM and VV datasets. For BRISQUE and PI scores, we achieve the best results on four datasets as well as the second-best results on the remaining dataset. In summary, our method achieves the lowest average scores on all three metrics, which proves that our method has a better generalization ability to unseen real-world scenes.

\textbf{Qualitative Comparison.}
We present qualitative comparisons of our method and state-of-the-art methods on the paired datasets in Fig.~\ref{fig:visual_compare_paired}. The images in rows 1-3 are selected from LOLv1, LOLv2-real, and LSRW test sets, respectively. The results presented in rows 1 and 3 show that DRBN, MIRNet, RUAS, and GDP are unable to properly improve the contrast of the images, yielding underexposed or overexposed results. Likewise, EnlightenGAN suffers from noise amplification, while LIME, SNRNet, Restormer, UHDFour$_{2\times}$ and WeatherDiff produce over-smoothed textures and artifacts. Moreover, as depicted in row 2, the results restored by Palette and GDP are concomitant with chaotic content and color distortion, other methods typically introduce artifacts around the lamp, whereas WeatherDiff avoids this distortion in that particular area but exhibits artifacts elsewhere. In contrast, our method effectively improves global and local contrast, reconstructs sharper details, and suppresses noise, resulting in visually satisfactory results in all cases. The visual comparisons on the UHD-LL~\cite{UHD_ICLR} test set is illustrated in Fig.~\ref{fig:visual_UHD}. As we can see that previous methods appear incorrect exposure, color distortion, noise amplification, or artifacts, thereby undermining the overall visual quality. In contrast, our method effectively improves global contrast and presents a vivid color without introducing chaotic content.
\begin{figure*}[!t]
	\centering
	\includegraphics[width=\linewidth]{face_detection.pdf}
	\caption{Comparison of face detection results before and after enhanced by different methods on the DARK FACE dataset~\cite{DarkFace}.}
	\label{fig:face_detection}
\end{figure*}

We also provide visual comparisons of our method and competitive methods on the unpaired datasets in Fig.~\ref{fig:visual_compare_unpaired}, where the images in rows 1-3 are selected from DICM, MEF, and NPE datasets, respectively. The previous learning-based LLIE methods and GDP fail to relight the building and the lantern in rows 1 and 2 compared to our method and WeatherDiff, but WeatherDiff produces undesirable content in the sky. In row 3, Zero-DCE, EnlightenGAN, and DDIM smooth the cloud texture, as Uformer and WeatherDiff produce gridding effects and messy content. Our method successfully restores the illumination of the image and reconstructs the details, which demonstrates that our method can generalize well to unseen scenes. For more qualitative results, please refer to the supplementary material.

\subsection{Low-Light Face Detection}\label{subsec:Low-Light Face Detection}
In this section, we investigate the impact of LLIE methods as pre-processing on improving the face detection task under weakly illuminated conditions. Specifically, we take the DARK FACE dataset~\cite{DarkFace} consisting of 10,000 images captured in real-world nighttime scenes for comparison, of which 6,000 images are selected as the training set, and the rest are adopted for testing. Since the bounding box labels of the test set are not publicly available, we utilize the training set for evaluation, applying our method and comparison methods as a pre-processing step, followed by the well-known learning-based face detector DSFD~\cite{DSFD}. We illustrate the precision-recall (P-R) curves in Fig.~\ref{fig:face_detection}(a) and compare the average precision (AP) under the IoU threshold of 0.5 using the official evaluation tool. As we can see that our proposed method achieves superior performance in the high recall area, and utilizing our approach as a pre-processing step leads to AP improvement from 26.4\% to 38.5\%. Observing the visual examples in Fig.~\ref{fig:face_detection}(b), our method relights the face regions while maintaining well-illuminated regions, enabling DSFD more robust in weakly illuminated scenes.

\subsection{Ablation Study}\label{subsec:Ablation Study}
In this section, we conduct a set of ablation studies to measure the impact of different component configurations employed in our method. We employ the implementation details described in Sec.~\ref{subsec:Experimental Settings} to train all models and subsequently evaluate their performances on the LOLv1~\cite{RetinexNet} test set for analysis. Detailed experiment settings are discussed below.

\textbf{Wavelet Transformation Scale.} We first validate the impact of performing diffusion processes on the average coefficient at different wavelet scales $K$. As reported in Table~\ref{tab:efficiency_ablation}, when we apply the WCDM to restore the average coefficient $A_{low}^{1}$, i.e., $K=1$, leads to overall performance improvements while resulting in the inference time increase. As the wavelet scale increases, the spatial dimension of the average coefficient decreases by $4^{K}$ relative to the original image, which causes performance degradation due to the information richness reduction but lowers the inference time. For a trade-off between performance and efficiency, we choose $K=2$ as the default setting. From another perspective, even if we perform three times wavelet transformation, i.e., $K=3$, our method is also comparable to the state-of-the-art methods reported in Table~\ref{tab:evaluation_paired}, which demonstrates that our proposed wavelet-based diffusion model can be applied to restore higher-resolution images.
\begin{table}[!t]
	\centering
	\caption{Ablation studies of various settings on the wavelet transformation scale and sampling step, please refer to the text for more details. The results using default settings are \underline{underlined}.}
	\resizebox{\linewidth}{!}{
		\begin{tabular}{c|c|ccccc}
			\toprule
			Wavelet scale & Sampling step  & PSNR $\uparrow$ & SSIM $\uparrow$ & LPIPS $\downarrow$ & FID $\downarrow$ & Time (s) $\downarrow$ \\
			\midrule
			\multirow{4}[2]{*}{$K=1$} & $S=5$ & 26.256 & 0.855 & 0.188 & 40.884 & 0.198 \\
			& $S=10$ & 26.444 & 0.858 & 0.182 & 40.143 & 0.380 \\
			& $S=20$ & 26.385 & 0.856 & 0.179 & 40.001 & 0.757 \\
			& $S=30$ & 26.211 & 0.850 & 0.190 & 41.023 & 1.256 \\
			\midrule
			\multirow{4}[2]{*}{$K=2$} & $S=5$ & 25.855 & 0.835 & 0.210 & 49.371 & 0.085 \\
			& $S=10$ & \underline{26.336} & \underline{0.845} & \underline{0.217} & \underline{48.114} & \underline{0.157} \\
			& $S=20$ & 26.246 & 0.844 & 0.212 & 48.071 & 0.285 \\
			& $S=30$ & 26.094 & 0.851 & 0.213 & 49.993 & 0.477 \\
			\midrule
			\multirow{4}[2]{*}{$K=3$} & $S=5$ & 24.408 & 0.817 & 0.283 & 61.234 & 0.066 \\
			& $S=10$ & 25.091 & 0.828 & 0.265 & 57.190 & 0.114 \\
			& $S=20$ & 25.103 & 0.826 & 0.271 & 57.227 & 0.237 \\
			& $S=30$ & 24.796 & 0.822 & 0.279 & 59.326 & 0.401 \\
			\bottomrule
	\end{tabular}}
	\label{tab:efficiency_ablation}
\end{table}

\textbf{Sampling Step.} Another factor that enables our method to be more computational efficiency is that a smaller sampling step $S$ is adopted. We present the performance of our method using different $S$ from 5 to 30 in Table~\ref{tab:efficiency_ablation}. Benefiting from we conduct the denoising process during training, the model learns how to perform denoising with different steps, making variations in sampling steps, whether smaller or larger, have no major impact on the performance of our proposed WCDM. A large sampling step contributes to generating images with better visual quality for diffusion-based image generation methods~\cite{ddpm,ddim}, while for us it only increases the inference time, which indicates that larger sampling steps, e.g., $S=1000$ in GDP~\cite{GDP} and Palette~\cite{palette}, may not be essential for diffusion-based LLIE and other relative image restoration tasks using our proposed framework.

\textbf{High-frequency Restoration Module.}
To validate the effectiveness of our designed high-frequency restoration module (HFRM), we form a baseline by using the proposed wavelet-based conditional diffusion model (WCDM) only to restore the average coefficient. As shown in rows 1-3 of Table~\ref{tab:module_ablation}, the HFRMs utilized to reconstruct the high-frequency coefficients at wavelet scales $k=1$ and $k=2$ improve the overall performance, achieving 2.142dB and 1.440dB gains in terms of PSNR, as well as 0.064 and 0.044 in terms of SSIM, respectively. Since the high-frequency coefficients at $k=1$ have richer details than the coefficients at $k=2$, the HFRM$_{1}$ is thus capable of delivering greater gains. Furthermore, we have formed two HFRM versions dubbed HFRM-v2 and HFRM-v3 to explore the effect of high-frequency coefficients complementarity. Specifically, the HFRM-v2 does not leverage the information in vertical and horizontal directions to complement the details present in the diagonal by using convolutional layers with equal parameters to replace cross-attention layers. On the other hand, the HFRM-v3 utilizes information from the diagonal direction to reinforce the vertical and horizontal details. As reported in rows 4-6 of Table~\ref{tab:module_ablation}, our default HFRM is superior to the other two versions, which demonstrates the effectiveness of high-frequency complementarity and that complementing diagonal details with information in vertical and horizontal directions being more useful than the reverse.
\begin{table}[!t]
	\centering
	\caption{Ablation studies of the effectiveness of our high-frequency restoration module. The results using default settings are \underline{underlined}.}
	\resizebox{0.75\linewidth}{!}{
		\begin{tabular}{l|cccc}
			\toprule
			& PSNR $\uparrow$ & SSIM $\uparrow$ & LPIPS $\downarrow$ & FID $\downarrow$ \\
			\midrule
			WCDM & 21.975 & 0.729 & 0.310 & 98.104 \\
			WCDM + HFRM$_{2}$  & 23.415 & 0.773 & 0.259 & 76.193 \\
			WCDM + HFRM$_{1}$ & 24.117 & 0.793 & 0.251 & 62.210 \\
			\midrule
			WCDM + HFRM-v2 & 24.145 & 0.822 & 0.266 & 67.909 \\
			WCDM + HFRM-v3 & 25.630 & 0.824 & 0.221 & 51.466 \\
			\midrule
			Default & \underline{26.336} & \underline{0.845} & \underline{0.217} & \underline{48.114} \\
			\bottomrule
	\end{tabular}}
	\label{tab:module_ablation}
\end{table}
\begin{table}[!ht]
	\centering
	\caption{Ablation studies of the loss function terms. The results using default settings are \underline{underlined}. `w/o' denotes without.}
	\resizebox{0.8\linewidth}{!}{
		\begin{tabular}{l|cccc}
			\toprule
			& PSNR $\uparrow$ & SSIM $\uparrow$ & LPIPS $\downarrow$ & FID $\downarrow$ \\
			\midrule
			w/o $\mathcal{L}_{diff}$ & 24.238 & 0.823 & 0.314 & 89.318 \\
			w/o $\mathcal{L}_{content}$ & 23.821 & 0.808 & 0.283 & 67.502 \\
			w/o $\mathcal{L}_{detail}$ & 25.413 & 0.835 & 0.249 & 57.663 \\
			\midrule
			vanilla $\mathcal{L}_{diff}$ (Eq.\ref{eq:8}) & 25.956 & 0.838 & 0.221 & 50.467 \\
			\midrule
			Default & \underline{26.306} & \underline{0.845} & \underline{0.217} & \underline{48.114} \\
			\bottomrule
	\end{tabular}}
	\label{tab:loss_ablation}
\end{table}

\textbf{Loss Function.}
To validate the effectiveness of the proposed loss functions, we conduct experiments by individually removing each component from the default setting, where the quantitative results are reported in Table~\ref{tab:loss_ablation}. As shown in row 1, the removal of the diffusion loss $\mathcal{L}_{diff}$ results in decreases in all perceptual metrics, as the generative capacity of diffusion models relies heavily on this component. The incorporation of content loss $\mathcal{L}_{content}$ yields noticeable improvements, especially the distortion metrics, which can be improved by 2.485dB and 0.037 for PSNR and SSIM, respectively, as shown in row 2 and row 5. The detail preserve loss $\mathcal{L}_{detail}$ is designed to reconstruct more image details, thus its removal causes performance degradation. However, such degradation is not significant relative to removing the content loss, which reveals the importance of the content loss. From another perspective, the content loss should be integrated with our proposed training strategy, which also illustrates its effectiveness. Furthermore, as described in Sec.~\ref{subsec:Wavelet-based Conditional Diffusion Models}, we incorporate an extra loss term into the vanilla diffusion loss, i.e., Eq.\ref{eq:8}, to promote the restoration of the average coefficient. To validate the effectiveness of the auxiliary term, we employ vanilla diffusion loss to optimize the noise estimator network. As shown in row 4, the inclusion of the auxiliary term proves to be beneficial in enhancing the visual fidelity of the restored images, thereby leading to the overall performance improved.
\begin{table}[!t]
	\centering
	\caption{Ablation studies of the training strategy of our wavelet-based conditional diffusion model. The results using default settings are \underline{underlined}.}
	\resizebox{\linewidth}{!}{
		\begin{tabular}{l|cccc}
			\toprule
			Methods & PSNR $\uparrow$ & SSIM $\uparrow$ & LPIPS $\downarrow$ & FID $\downarrow$ \\
			\midrule
			DDIM$^{\ast}$ & 16.521$_{\pm 0.544}$ & 0.776$_{\pm 0.002}$ & 0.376$_{\pm 0.001}$ & 84.071${_\pm 32.909}$ \\
			Palette$^{\ast}$ & 11.771$_{\pm 0.254}$ & 0.561$_{\pm 0.001}$ & 0.498$_{\pm 0.001}$ & 108.291${_\pm 8.019}$ \\
			WeatherDiff$^{\ast}$ & 17.913$_{\pm 0.182}$ & 0.811$_{\pm 0.001}$ & 0.272$_{\pm 0.001}$ & 73.903${_\pm 5.265}$\\
			\midrule
			Ours$_{\operatorname{FD}}$ & 18.125$_{\pm 0.212}$ & 0.789$_{\pm 0.001}$ & 0.273$_{\pm 0.002}$ & 64.676$_{\pm 5.544}$ \\
			Ours$_{\operatorname{FD+DP}}$ & \underline{26.336$_{\pm 1\times 10^{-5}}$} & \underline{0.845$_{\pm 2\times 10^{-7}}$} & \underline{0.217$_{\pm 6\times 10^{-9}}$} & \underline{48.114$_{\pm 0.006}$} \\
			\bottomrule
	\end{tabular}}
	\label{tab:training_ablation}
\end{table}
\begin{figure}[!t]
	\centering
	\includegraphics[width=\linewidth]{diversity.pdf}
	\caption{The sampling results from five evaluations, in which performing forward diffusion only in the training phase results in diverse outputs, while our training strategy contributes to generating results with consistency.}
	\label{fig:diversity}
\end{figure}

\textbf{Training Strategy.}
The content diversity caused by randomly sampled noise during the denoising process of diffusion models is undesirable for some image restoration tasks, such as LLIE, dehazing, deraining, etc. As described in Sec.~\ref{subsec:Wavelet-based Conditional Diffusion Models}, we address this issue by incorporating the denoising process (DP) in both training and inference phases to achieve consistency. As shown in Table~\ref{tab:training_ablation}, we report the means and variances of five evaluations for three diffusion-based methods, our method that performs forward diffusion (FD) only in the training phase denoted as Ours$_{\operatorname{FD}}$, and our method with the default training strategy denoted as Ours$_{\operatorname{FD+DP}}$. By incorporating DP into the training phase, our method not only achieves the best performance but also aids in circumventing content diversity and yields minimal variance, which proves the superiority of our training strategy. The restored results of five evaluations illustrated in Fig.~\ref{fig:diversity} show that performing forward diffusion only in the training phase leads to diverse results, while our default training strategy not only mitigates the appearance of chaotic content and color distortion but also facilitates generating images with consistency.

\subsection{Limitations}
Although our method is effective in restoring images captured in weakly illuminated conditions, it does not work as well in extremely low-light environments as images captured in such conditions suffer from greater information loss and are more challenging to restore. In addition, some real-time methods can be applied directly to the low-light video enhancement task, whereas our method is not yet efficient enough to be applied directly to this task. Another natural limitation of our model is its inherently limited capacity to only generalize to the LLIE task observed at training time, and investigating the effectiveness of the proposed approach for other image restoration tasks will be our future work.

\section{Conclusions}\label{sec:conclusion}
We have presented DiffLL, a diffusion-based framework for robust and efficient low-light image enhancement. Technically, we propose a wavelet-based conditional diffusion model that leverages the generative ability of diffusion models and wavelet transformation to produce visually satisfactory results while reducing inference time and computational resource consumption. Moreover, to circumvent content diversity during inference, we perform the denoising process in both the training and inference phases, enabling the model to learn stable sampling and obtain restored results with consistency. Additionally, we design a high-frequency restoration module to complement the diagonal details with vertical and horizontal information for better fine-grained details reconstruction. Experimental results on publicly available benchmarks demonstrate that our method outperforms competitors both quantitatively and qualitatively while being computationally efficient. The results of the low-light face detection also show the practical values of our method.

\section*{Acknowledgements}
This work was supported by National Natural Science Foundation of China (NSFC) under grant No.62372091 and Sichuan Science and Technology Program of China under grant No.2023NSFSC0462.

% Bibliography
\bibliographystyle{ACM-Reference-Format}
\bibliography{sample-bibliography}
% !TEX root = ../arxiv.tex

Unsupervised domain adaptation (UDA) is a variant of semi-supervised learning \cite{blum1998combining}, where the available unlabelled data comes from a different distribution than the annotated dataset \cite{Ben-DavidBCP06}.
A case in point is to exploit synthetic data, where annotation is more accessible compared to the costly labelling of real-world images \cite{RichterVRK16,RosSMVL16}.
Along with some success in addressing UDA for semantic segmentation \cite{TsaiHSS0C18,VuJBCP19,0001S20,ZouYKW18}, the developed methods are growing increasingly sophisticated and often combine style transfer networks, adversarial training or network ensembles \cite{KimB20a,LiYV19,TsaiSSC19,Yang_2020_ECCV}.
This increase in model complexity impedes reproducibility, potentially slowing further progress.

In this work, we propose a UDA framework reaching state-of-the-art segmentation accuracy (measured by the Intersection-over-Union, IoU) without incurring substantial training efforts.
Toward this goal, we adopt a simple semi-supervised approach, \emph{self-training} \cite{ChenWB11,lee2013pseudo,ZouYKW18}, used in recent works only in conjunction with adversarial training or network ensembles \cite{ChoiKK19,KimB20a,Mei_2020_ECCV,Wang_2020_ECCV,0001S20,Zheng_2020_IJCV,ZhengY20}.
By contrast, we use self-training \emph{standalone}.
Compared to previous self-training methods \cite{ChenLCCCZAS20,Li_2020_ECCV,subhani2020learning,ZouYKW18,ZouYLKW19}, our approach also sidesteps the inconvenience of multiple training rounds, as they often require expert intervention between consecutive rounds.
We train our model using co-evolving pseudo labels end-to-end without such need.

\begin{figure}[t]%
    \centering
    \def\svgwidth{\linewidth}
    \input{figures/preview/bars.pdf_tex}
    \caption{\textbf{Results preview.} Unlike much recent work that combines multiple training paradigms, such as adversarial training and style transfer, our approach retains the modest single-round training complexity of self-training, yet improves the state of the art for adapting semantic segmentation by a significant margin.}
    \label{fig:preview}
\end{figure}

Our method leverages the ubiquitous \emph{data augmentation} techniques from fully supervised learning \cite{deeplabv3plus2018,ZhaoSQWJ17}: photometric jitter, flipping and multi-scale cropping.
We enforce \emph{consistency} of the semantic maps produced by the model across these image perturbations.
The following assumption formalises the key premise:

\myparagraph{Assumption 1.}
Let $f: \mathcal{I} \rightarrow \mathcal{M}$ represent a pixelwise mapping from images $\mathcal{I}$ to semantic output $\mathcal{M}$.
Denote $\rho_{\bm{\epsilon}}: \mathcal{I} \rightarrow \mathcal{I}$ a photometric image transform and, similarly, $\tau_{\bm{\epsilon}'}: \mathcal{I} \rightarrow \mathcal{I}$ a spatial similarity transformation, where $\bm{\epsilon},\bm{\epsilon}'\sim p(\cdot)$ are control variables following some pre-defined density (\eg, $p \equiv \mathcal{N}(0, 1)$).
Then, for any image $I \in \mathcal{I}$, $f$ is \emph{invariant} under $\rho_{\bm{\epsilon}}$ and \emph{equivariant} under $\tau_{\bm{\epsilon}'}$, \ie~$f(\rho_{\bm{\epsilon}}(I)) = f(I)$ and $f(\tau_{\bm{\epsilon}'}(I)) = \tau_{\bm{\epsilon}'}(f(I))$.

\smallskip
\noindent Next, we introduce a training framework using a \emph{momentum network} -- a slowly advancing copy of the original model.
The momentum network provides stable, yet recent targets for model updates, as opposed to the fixed supervision in model distillation \cite{Chen0G18,Zheng_2020_IJCV,ZhengY20}.
We also re-visit the problem of long-tail recognition in the context of generating pseudo labels for self-supervision.
In particular, we maintain an \emph{exponentially moving class prior} used to discount the confidence thresholds for those classes with few samples and increase their relative contribution to the training loss.
Our framework is simple to train, adds moderate computational overhead compared to a fully supervised setup, yet sets a new state of the art on established benchmarks (\cf \cref{fig:preview}).
         % This file has Introduction section
%auto-ignore
\begin{figure}[t!]
\centering
\includegraphics[width=1.0\linewidth]{figures/wireishard.pdf}
% \includegraphics[width=1.0\linewidth]{figures/wiresarehard2.pdf}
\caption{\textbf{Challenges of wire segmentation.} Wires have a diverse set of appearances. Challenges include but are not limited to (a) structural complexity, (b) visibility and thickness, (c) partial occlusion by other objects, (d) camera aberration artifacts, and variations in (e) object attachment, (f) color, (g) width and (h) shape.
% \zwei{this needs to be correspondent to the attributes you mentioned}
}
\vspace{-5.5mm}
\label{fig:motivation}
\end{figure} 
\section{Background and Motivation}

\subsection{IBM Streams}

IBM Streams is a general-purpose, distributed stream processing system. It
allows users to develop, deploy and manage long-running streaming applications
which require high-throughput and low-latency online processing.

The IBM Streams platform grew out of the research work on the Stream Processing
Core~\cite{spc-2006}.  While the platform has changed significantly since then,
that work established the general architecture that Streams still follows today:
job, resource and graph topology management in centralized services; processing
elements (PEs) which contain user code, distributed across all hosts,
communicating over typed input and output ports; brokers publish-subscribe
communication between jobs; and host controllers on each host which
launch PEs on behalf of the platform.

The modern Streams platform approaches general-purpose cluster management, as
shown in Figure~\ref{fig:streams_v4_v6}. The responsibilities of the platform
services include all job and PE life cycle management; domain name resolution
between the PEs; all metrics collection and reporting; host and resource
management; authentication and authorization; and all log collection. The
platform relies on ZooKeeper~\cite{zookeeper} for consistent, durable metadata
storage which it uses for fault tolerance.

Developers write Streams applications in SPL~\cite{spl-2017} which is a
programming language that presents streams, operators and tuples as
abstractions. Operators continuously consume and produce tuples over streams.
SPL allows programmers to write custom logic in their operators, and to invoke
operators from existing toolkits. Compiled SPL applications become archives that
contain: shared libraries for the operators; graph topology metadata which tells
both the platform and the SPL runtime how to connect those operators; and
external dependencies. At runtime, PEs contain one or more operators. Operators
inside of the same PE communicate through function calls or queues. Operators
that run in different PEs communicate over TCP connections that the PEs
establish at startup. PEs learn what operators they contain, and how to connect
to operators in other PEs, at startup from the graph topology metadata provided
by the platform.

We use ``legacy Streams'' to refer to the IBM Streams version 4 family. The
version 5 family is for Kubernetes, but is not cloud native. It uses the
lift-and-shift approach and creates a platform-within-a-platform: it deploys a
containerized version of the legacy Streams platform within Kubernetes.

\subsection{Kubernetes}

Borg~\cite{borg-2015} is a cluster management platform used internally at Google
to schedule, maintain and monitor the applications their internal infrastructure
and external applications depend on. Kubernetes~\cite{kube} is the open-source
successor to Borg that is an industry standard cloud orchestration platform.

From a user's perspective, Kubernetes abstracts running a distributed
application on a cluster of machines. Users package their applications into
containers and deploy those containers to Kubernetes, which runs those
containers in \emph{pods}. Kubernetes handles all life cycle management of pods,
including scheduling, restarting and migration in case of failures.

Internally, Kubernetes tracks all entities as \emph{objects}~\cite{kubeobjects}.
All objects have a name and a specification that describes its desired state.
Kubernetes stores objects in etcd~\cite{etcd}, making them persistent,
highly-available and reliably accessible across the cluster. Objects are exposed
to users through \emph{resources}. All resources can have
\emph{controllers}~\cite{kubecontrollers}, which react to changes in resources.
For example, when a user changes the number of replicas in a
\code{ReplicaSet}, it is the \code{ReplicaSet} controller which makes sure the
desired number of pods are running. Users can extend Kubernetes through
\emph{custom resource definitions} (CRDs)~\cite{kubecrd}. CRDs can contain
arbitrary content, and controllers for a CRD can take any kind of action.

Architecturally, a Kubernetes cluster consists of nodes. Each node runs a
\emph{kubelet} which receives pod creation requests and makes sure that the
requisite containers are running on that node. Nodes also run a
\emph{kube-proxy} which maintains the network rules for that node on behalf of
the pods. The \emph{kube-api-server} is the central point of contact: it
receives API requests, stores objects in etcd, asks the scheduler to schedule
pods, and talks to the kubelets and kube-proxies on each node. Finally,
\emph{namespaces} logically partition the cluster. Objects which should not know
about each other live in separate namespaces, which allows them to share the
same physical infrastructure without interference.

\subsection{Motivation}
\label{sec:motivation}

Systems like Kubernetes are commonly called ``container orchestration''
platforms. We find that characterization reductive to the point of being
misleading; no one would describe operating systems as ``binary executable
orchestration.'' We adopt the idea from Verma et al.~\cite{borg-2015} that
systems like Kubernetes are ``the kernel of a distributed system.'' Through CRDs
and their controllers, Kubernetes provides state-as-a-service in a distributed
system. Architectures like the one we propose are the result of taking that view 
seriously.

The Streams legacy platform has obvious parallels to the Kubernetes
architecture, and that is not a coincidence: they solve similar problems.
Both are designed to abstract running arbitrary user-code across a distributed
system.  We suspect that Streams is not unique, and that there are many
non-trivial platforms which have to provide similar levels of cluster
management.  The benefits to being cloud native and offloading the platform
to an existing cloud management system are: 
\begin{itemize}
    \item Significantly less platform code.
    \item Better scheduling and resource management, as all services on the cluster are 
        scheduled by one platform.
    \item Easier service integration.
    \item Standardized management, logging and metrics.
\end{itemize}
The rest of this paper presents the design of replacing the legacy Streams 
platform with Kubernetes itself.

    % This file has Background section
%\section{SNOW Implementation on TI CC13x0}\label{sec:implementation}
\begin{figure}[!htb]
\centering
\includegraphics[width=0.49\textwidth]{figs/devices-new.eps}
\caption{Devices used in our SNOW implementation. A node is a CC1310 or CC1350 device. The BS has two USRP B200s, each having its own antenna. An antenna is approximately 2x bigger than a B200.}
\label{fig:devices}
\end{figure}
%talk about what devices are used in BS and as nodes
The original SNOW implementation in~\cite{snow_ton} uses the USRP hardware platform for both the BS and the nodes. In our implementation, we use the CC13x0 devices as SNOW nodes and USRP in the BS (Figure~\ref{fig:devices}).
%A USRP B200 device with a half-duplex radio costs approximately \$750 USD, as of today. As such, it becomes costly to deploy the SNOW network and examine its scalability. In contrast, in this work, we realize the functionality of a SNOW node in TI CC1310 LaunchPad~\cite{cc1310} that costs approximately \$30 USD, thus much cheaper and widely available to the research community to develop and deploy SNOW networks. 
For BS implementation, we adopt the open-source code provided in~\cite{snow_bs}. The BS uses two half-duplex USRP devices (Rx-Radio and Tx-Radio), each having its own antenna. Also. the BS is implemented on the GNURadio software platform that gives a high magnitude of freedom to perform baseband signal processing~\cite{gnuradio}.
In the following, we explore a number of implementation considerations and feasibility for a CC13x0 device to work as a SNOW node in practical deployments. 
First, we show how to configure a CC13x0 device to make it work as a SNOW node. We then address the practical challenges (e.g., PAPR problem, CSI estimation, and CFO estimation) associated with our CC13x0-based SNOW implementation.

\subsection{Configuring TI CC13x0}
%talk about how to configure nodes and BS
We configure the subcarrier center frequency, bandwidth, modulation, and the Tx power by setting appropriate values to the CC13x0 command inputs \code{centerFreq, rxBw, modulation}, and \code{txPower}, respectively, using {\em Code Composer Studio} (CCS) provided by Texas Instruments~\cite{snow_cots}. A graphical user interface alternative to CCS is {\em SmartRF Studio}. The MAC protocol of SNOW in CC13x0 is implemented on top of the example CSMA/CA project that comes with CCS. Note that the functionalities of a SNOW node are very simple and may be incorporated easily in the IoT devices that have both storage and computational limitations like the CC13x0 devices.

\subsection{Peak-to-Average Power Ratio Observation}\label{sec:papr}
By transmitting on a large number of subcarriers simultaneously (in the downlink), the BS suffers from a traditional OFDM problem called {\em peak-to-average power ratio (PAPR)}. PAPR of an OFDM signal is defined as the ratio of the maximum instantaneous power to its average power.
In the SNOW downlink communications (i.e., BS to nodes), after the IFFT is performed by the BS, the composite signal can be represented as
$\nonumber x(t) = \frac{1}{\sqrt{N}}\sum_{k=0}^{N-1}X_k~e^{j2 \pi f_k t},~~0 \le t \le NT.$
% \begin{equation}
% \nonumber x(t) = \frac{1}{\sqrt{N}}\sum_{k=0}^{N-1}X_k~e^{j2 \pi f_k t},~~0 \le t \le NT
% \end{equation} 
Here, $X_k$ is the modulated data symbol for node $k = \{0, 1, \cdots, N-1\}$ on subcarrier center frequency $f_k = k\Delta f$, where $\Delta f = \frac{1}{NT}$ and $T$ is the symbol period. Therefore, the PAPR may be calculated as%~\cite{jiang2008overview}
\begin{equation}
\nonumber \text{PAPR}[x(t)] = 10\log_{10}\Bigg( \frac{\max\limits_{0~ \le ~t~ \le~ NT} [|x(t)|^2 ]}{P_{\text{avg}}}\Bigg)~~dB.
\end{equation}
Here, the average power $P_{\text{avg}} = E [|x(t)|^2]$.
A node's signal detection on its subcarrier is very sensitive to the nonlinear signal processing components used in the BS, i.e., the digital-to-analog converter (DAC) and high power amplifier (HPA), which may severely impair the bit error rate (BER) in the nodes due to the induced spectral regrowth. If the HPA does not operate in the linear region with a large power back-off due to high PAPR, the out-of-band power will exceed the specified limit and introduce severe ICI~\cite{jiang2008overview}. Moreover, the in-band distortion (constellation tilting and scattering) due to high PAPR may cause severe performance degradation~\cite{kamali2012understanding}. It has been shown that the PAPR reduction results in significant power saving at the transmitters~\cite{baxley2004power}.
\begin{figure}[!htb]
\centering
\includegraphics[width=0.35\textwidth]{figs/papr/papr.eps}
\caption{PAPR distribution of D-OFDM signal in Tx-Radio.}
\label{fig:papr}
\end{figure}


As shown in Figure~\ref{fig:papr}, the PAPR in the SNOW downlink communications (for N = 64) follows the Gaussian distribution. Thus, the peak signal occurs quite rarely and the transmitted D-OFDM signal will cause the HPA to operate in the nonlinear region, resulting in a very inefficient amplification. To illustrate the power efficiency of the HPA for N = 64, let us assume the probability of the clipped D-OFDM frames is less than 0.01\%. We thus need to apply an input back-off (IBO)~\cite{baxley2004power} equivalent to the PAPR at a probability of $10^{-4}$. Here, PAPR $\approx$ 14dB or 25.12. Thus, the efficiency ($\eta = 0.5/\text{PAPR}$) of the HPA~\cite{jiang2008overview} is $\eta = 0.5/25.12 \approx 1.99\%$. Such low efficiency at the HPA motivates us to explore the high PAPR in SNOW for practical deployments.
%show some results explaining how PARP affects the BS-Node communication
Several uplink PAPR reduction techniques for single-user OFDM systems have been proposed (see survey~\cite{jiang2008overview}). However, the characteristics of the downlink PAPR in SNOW, where different data are concurrently transmitted to different nodes, are entirely different from the PAPR observed in a single-user OFDM system. To adopt an uplink PAPR reduction technique used in the single-user OFDM systems for the downlink PAPR reduction in SNOW, each node has to process the entire data frame transmitted by the BS and then demodulate its own data. However, a SNOW node has less computational power and does not apply FFT to decode its data~\cite{snow_ton}, or any other node's data. Thus, the existing PAPR reduction techniques will not work in our implementation.

%To this extent, we address the PAPR problem in SNOW by allocating a special subcarrier  called {\em downlink subcarrier} for the downlink communications.
We propose to handle the PAPR problem in SNOW by using only one subcarrier (called {\em downlink subcarrier}) for downlink communication. All the nodes use this subcarrier to receive from the BS. Namely, the Tx-Radio transmits only on one subcarrier that is not used by any node for uplink communication.
The BS may send any broadcast message, ACK, or data to the nodes using that downlink subcarrier. A node has to switch to the downlink subcarrier to listen to any broadcast message, ACK, or data.
The BS may reserve multiple subcarriers  as {\slshape backup subcarriers} for downlink communication. 
If the currently used downlink subcarrier becomes overly noisy or unreliable, it can be replaced by a backup subcarrier.
Note that the dual-radio in the BS allows it to receive concurrent packets from a set of nodes (uplink) and transmit broadcast/ACK/data packets to another set of nodes (downlink), simultaneously. 
The BS can acknowledge several nodes using a single transmission by using a bit-vector of size equals to the number of subcarriers.
If the BS receives a packet from a node operating on subcarrier $i$, it will set the $i$-th bit in the bit-vector. Upon receiving the bit-vector, that node may get an ACK by looking at the $i$-th bit of the vector. Because of the bit-vector, the downlink ACKs also scale up like the uplink traffic. In the case of different packets for different nodes, the volume of downlink traffic (compared to the uplink traffic) is also practical since the IoT applications may not require high volume downlink traffic~\cite{whitespaceSurvey}.

%A node retransmits the packet if that packet is not acknowledged in the first ACK received by that node. 
%In the following, we describe our technique below to handle a {\bf rare} case in practical SNOW deployments, and hence may be kept optional in implementation.

%\revise{Let nodes $A$ and $B$ share subcarrier $i$. The BS may receive a packet from node $B$ while preparing the ACK for node $A$'s packet. If both packets are decoded correctly, the BS acknowledges them by setting the $i$-th bit of the vector. However, if only one packet is decoded correctly, the BS resets the $i$-th bit of the vector. Thus, none of the packets are acknowledged. To compensate for this, the BS (Tx-Radio) switches to subcarrier $i$ and sends separate ACKs for nodes $A$ and $B$. On the other hand, if a node finds that its packet is not acknowledged in the downlink subcarrier, it listens to its subcarrier for a short fixed window before attempting a retransmission. A node knows about that fixed window when it joins the network. Note that {\em a very few} nodes (sharing the same subcarrier) may be involved in this scenario since the ACK generation time at the BS is very small. Other ways of addressing this issue may include the use of \emph{hash functions}, which we do not consider due to the scalability issues in hash-related collisions.}

When a node $u$ transmits to the BS, if another node $v$ sharing the same subcarrier wants to transmit, $v$ senses the channel as busy and refrain from transmitting. When the BS transmits ACK to $u$ on the downlink subcarrier using the Tx-Radio, node $v$ may also transmit to the BS. Since the Tx-Radio at that time is making a downlink transmission, it may not send the ACK upon $v$'s transmission immediately. However, the Tx-Radio can send $v$'s ACK immediately after completing its current downlink transmission. Thus, $v$ may need to wait for ACK for a little longer than the time needed to send a downlink transmission from the BS. A node may go to sleep mode or its next state right after receiving an ACK. However, if a node that has transmitted but not yet received ACK, should wait for a little longer (e.g., up to one or two downlink transmission time). Note that a very few nodes (sharing the same subcarrier) may be involved in this scenario since the ACK generation time at the BS is very small. For the same reason, the waiting time for ACK will also not be very long (e.g., up to one or two downlink transmission time). Note that this scenario is quite rare and most of the times the nodes will receive ACK immediately upon transmission.


%When a subcarrier (say, $i$) is shared by multiple nodes, the BS may receive a packet (say, from node A) before transmitting the ACK for another packet (say, from node B). In this case, both nodes A and B may be acknowledged by setting the $i$-th bit of the vector. However, if the packet from node A (or, B) is valid and the packet from node B (or, A) is invalid, the BS will reset the $i$-th bit of the vector and transmit the ACK. Thus, none of the packets are acknowledged even if one of them is valid. To compensate for that, the BS (Tx-Radio) will switch to node A's (or, B's) subcarrier and transmit an ACK packet. Thus, in our implementation, if a node finds that its packet is not acknowledged in the first valid ACK it received, before retransmission it listens to its subcarrier for a fixed amount of time. Each node may know this fixed time when it joins the network. Typically, if a subcarrier is shared by $G$ nodes, the fixed amount of time (worst case) may be set to $GD_p$ (ignoring the frequency switching time in the Tx-Radio), where $D_p$ is the time to transmit one packet. Other ways of addressing such issue may include the use of \emph{hash functions}. However, we do not explore that in our implementation for scalability issue due to hash collision.

%In the case where each subcarrier is assigned to only a node, the size of the bit-vector may be set to the total number of subcarriers. Thus, if the BS receives a packet from a node operating on subcarrier $i$, it will set the $i$-th bit in the bit-vector. Upon receiving the bit-vector in ACK subcarrier, that node can check the $i$-th bit of the vector. However, in practical deployments with thousands of nodes, a subcarrier may be shared by multiple nodes, making the creation of the bit-vector non-trivial. We cannot also use any hashing technique because of the hash collisions and scalability issues. 

\begin{figure*}[!htbp] 
    \centering
      \subfigure[RSSI under varying distance\label{fig:csi_rssi}]{
    \includegraphics[width=0.35\textwidth]{figs/csi/rssi.eps}
      }\hfill
      \subfigure[Path Loss under varying distance\label{fig:csi_pathloss}]{
        \includegraphics[width=.35\textwidth]{figs/csi/pathloss.eps}
      }\hfill
      \subfigure[BER under varying distance\label{fig:csi_ber}]{
        \includegraphics[width=.35\textwidth]{figs/csi/ber.eps}
      }
    \caption{RSSI, path loss, and BER at the SNOW BS for a TI CC1310 node.}
    \label{fig:csi}
 \end{figure*}
\subsection{Channel State Information Estimation}\label{sec:csi}

Multi-user OFDM communication requires channel estimation and tracking to ensure high data rate at the BS. One way to avoid channel estimation is to use the \emph{differential phase-shift keying (DPSK)} modulation. DPSK, however, results in a lower bitrate at the BS due to a 3dB loss in the signal-to-noise ratio (SNR)~\cite{van1995channel}. Additionally, the current SNOW design does not support DPSK modulation. SNR at the BS for each node is different in SNOW. Also, SNR of each node is affected differently due to channel conditions, deteriorating the overall bitrate in the uplinks. Thus, it requires handling of the channel estimation in SNOW.

Figure~\ref{fig:csi} shows the experimentally found received signal strength indicator (RSSI), path loss, and BER at the SNOW BS for a CC1310 device that transmits successive 1000 30-byte (payload) packets from 200 to 1000m distances, respectively, with a Tx power of 15dBm, subcarrier center frequency at 500MHz, and a bandwidth of 98kHz. Figure~\ref{fig:csi_rssi} indicates that the RSSI decreases rapidly with the increase in distance. Also, the path loss in Figure~\ref{fig:csi_pathloss} shows that it is significantly higher than the theoretical free space loss~\cite{rappaport1996wireless}. We also compare with the Okumura-Hata~\cite{rappaport1996wireless} loss to check if it fits the model, however, it does not. Finally, Figure~\ref{fig:csi_ber} confirms that the BER goes above $10^{-3}$ (which is not acceptable~\cite{rnr}) beyond 400m due to the unknown channel conditions. Figure~\ref{fig:csi_ber} also shows that the BER worsens for an increase in the subcarrier bandwidth. Thus, to make our implementation more resilient, we need to incorporate the CSI estimation in SNOW.

We calculate the CSI for each SNOW node independently on its subcarrier. We consider a slow flat-fading model~\cite{tse2005fundamentals}, where the channel conditions vary slowly with respect to a single node to BS packet duration. Note that joint-CSI estimation~\cite{jiang2007iterative, ribeiro2008uplink} in SNOW is not our design goal since it would require SNOW nodes to be strongly time-synchronized.  
Similar to IEEE 802.16e, we run CSI estimation independently for each node because of their different fading and noise characteristics. In the following, we explain the CSI estimation technique for one node on its subcarrier for each packet. The BS uses the same technique to estimate CSI for all other nodes. 

For a node, in a narrowband flat-fading subcarrier, the system is modeled as $y = Hx + w$,
% \begin{equation}
% \nonumber y = Hx + w,
% \end{equation}
where $y$, $x$, and $w$ are the receive vector, transmit vector, and noise vector, respectively. $H$ is the channel matrix. 
We model the noise as additive white Gaussian noise, i.e., a circular symmetric complex normal ($CN$) with $w \sim CN(0, W)$, where the mean is zero and noise covariance matrix $W$ is known.
%Noise is modeled as circular symmetric complex normal ($CN$) with $w \sim CN(0, W)$, where the mean is zero and noise covariance matrix $W$ is known, thus an additive white Gaussian noise. 
As the subcarrier conditions vary, we estimate the CSI on a short-term basis based on popular approach called {\em training sequence}. We use the known preamble transmitted at the beginning of each packet. $H$ is estimated using the combined knowledge of the received and the transmitted preambles. To make the estimation robust, we divide the preamble into $n$ equal parts (preamble sequence). E.g., n = 4, which is similar to the estimation in IEEE 802.11.

Let the preamble sequence be $(p_1, p_2, \cdots, p_n)$, where vector $p_i$ is transmitted as $y_i = Hp_i + w_i$.
% \begin{equation}
% \nonumber y_i = Hp_i + w_i.
% \end{equation}
Combining the received preamble sequences, we get $Y = [y_1, \cdots, y_n] = HP + W$, where 
% \begin{equation}
% \nonumber Y = [y_1, \cdots, y_n] = HP + W.
% \end{equation}
$P = [p_1, \cdots, p_n]$ and $W = [w_1, \cdots, w_n]$. With combined knowledge of $Y$ and $P$, channel matrix $H$ is estimated. Similar to the CSI estimation in the uplink communications by the BS, each node also estimates the CSI during its downlink communications. Note that the computational complexity of CSI estimation at the nodes is lightweight since each SNOW packet has a 32-bit preamble~\cite{snow_ton}, divided into four equal parts. A node thus processes a vector of only 8 bits at a time.
%during CSI estimation.




\subsection{Carrier Frequency Offset Estimation} \label{sec:cfo}

Multi-user OFDM systems are very sensitive to the CFO between the transmitters and the receiver. CFO causes the OFDM systems to lose orthogonality between subcarriers, which results in severe ICI. 
A transmitter and a receiver observe CFO due to (i) the mismatch in their local oscillator frequency as a result of hardware imperfections; (ii) the relative motion that causes a Doppler shift. 
%CFO originates in a transmitter and a receiver due to their (i) local oscillator's frequency mismatch as a result of hardware imperfections; (ii) relative motion that causes a Doppler shift. 
ICI degrades the SNR between an OFDM transmitter and a receiver, which results in significant BER. Thus, we investigate the needs for CFO estimation in our implementation.
\begin{figure}[!htb]
\centering
\includegraphics[width=0.35\textwidth]{figs/cfo/ber.eps}
\caption{BER at different $E_b/N_0$.}
\label{fig:cfo}
\end{figure}
The loss in SNR due to the CFO between the SNOW BS and a node can be estimated as 
$SNR_{loss} = 1 + \frac{1}{3}(\pi \delta f T)^2\frac{E_s}{N_0}$~\cite{nee2000ofdm}, where
% \begin{equation} \scriptsize
%  \nonumber SNR_{loss} = 1 + \frac{1}{3}(\pi \delta f T)^2\frac{E_s}{N_0} 
% \end{equation}
$\delta f$ is the frequency offset, $T$ is the symbol duration, $E_s$ is the average received subcarrier energy, and $N_0/2$ is the two-sided spectral density of the noise power.

To observe the effects of CFO, we choose two neighboring orthogonal subcarriers in the BS and send concurrent packets from two nodes at 200m distance. Each node sends successive 1000 30-byte packets. We repeat this experiment varying the transmission powers at the nodes to generate signals with different $E_b/N_0$, where $E_b$ is the average energy per bit in the received signals. 
Figure~\ref{fig:cfo} shows the BER at the BS while receiving packets from these two nodes. This figure shows that BER is nearly $10^{-3}$ even for very high $E_b/N_0$ ($\approx 40$dB), which is also very high compared to the theoretical BER~\cite{choi2000carrier}. Thus, CFO is heavily pronounced in SNOW.
The distributed and asynchronous nature of SNOW does not allow CFO estimation similar to the traditional multi-user OFDM systems.
While the USRP-based SNOW implementation provides a trivial and {\em coarse} CFO estimation, it is not robust and does not account for the mobility of the nodes~\cite{snow_ton}.
We propose a pilot-based robust CFO estimation technique, combining both coarse and finer estimations, which accounts for the mobility of the nodes as well. We use training symbols for CFO estimation in an ICI free environment for each node independently, while it joins the network by communicating with the BS using a non-overlapping {\em join subcarrier}.


We explain the CFO estimation technique between a node and the BS (uplink) on a join subcarrier $f$ based on time-domain samples. Note that the BS keeps running the G-FFT on the entire BS spectrum. We thus extract the corresponding time-domain samples of the join subcarrier by applying IFFT during a node join. The join subcarrier does not overlap with other subcarriers; hence it is ICI-free. If $f_{\text{node}}$ and $f_{\text{BS}}$ are the frequencies at a node and the BS, respectively, then their frequency offset $\delta f = f_{\text{node}}-  f_{\text{BS}}$.
For transmitted signal $x(t)$ from a node, the received signal  $y(t)$ at the BS that experiences a CFO of $\delta f$ is given by 
$y(t)  = x(t) e^{j2\pi \delta f t}$.
Similar to IEEE 802.11a, we estimate $\delta f$ based on short and long preamble approach. Note that the USRP-based implementation has considered only one preamble to estimate CFO.
In our implementation, the BS first divides a $n$-bit preamble from a node into short and long preambles of lengths $n/4$ and $3n/4$, respectively. Thus for a 32-bit preamble (typically used in SNOW), the lengths of the short and long preambles are  8 and 24, respectively. 
The short preamble and the long preamble are used for coarse and finer CFO estimation, respectively. 
Considering $\delta t_s$ as the short preamble duration and $\delta f_s$ as the coarse CFO estimation, we have
$y(t-\delta t_s)  = x(t) e^{j2\pi \delta f_s (t-\delta t_s)}.$

Since $y(t)$ and $y(t-\delta t_s)$ are known at the BS, we have
\begin{align*}
y(t-\delta t_s) y^*(t)  & = x(t) e^{j2\pi \delta f_s (t-\delta t_s)}       x^*(t) e^{-j2\pi  \delta f_s t}
                           = |x(t)|^2  e^{j 2\pi  \delta f_s -\delta t_s }.
\end{align*}
Taking angle of both sides gives us as follows.
% $$\sphericalangle  y(t-\delta t_s) y^*(t)   =  \sphericalangle     |x(t)|^2  e^{j 2\pi  \delta f_s -\delta t_s }  =      - 2\pi  \delta f_s \delta t_s$$
\begin{align*}
\sphericalangle  y(t-\delta t_s) y^*(t)   &=  \sphericalangle     |x(t)|^2  e^{j 2\pi  \delta f_s -\delta t_s } =      - 2\pi  \delta f_s \delta t_s
                                          %&=      - 2\pi  \delta f_s \delta t_s
\end{align*}
By rearranging the above equation, we get
$$\delta f_s   =  - \frac{\sphericalangle  y(t-\delta t_s) y^*(t) }{2\pi\delta t_s}.$$

Now that we have the coarse CFO $\delta f_s$, we correct each time domain sample (say, $P$) received in the long preamble as $ P_a = P_a e^{-ja \delta f_s}$, where $a = \{1, 2, \cdots, A\}$ and $A$ is the number of time-domain samples in the long preamble. Taking into account the corrected samples of the long preamble and considering $\delta t_l$ as the long preamble duration, we estimate the finer CFO as follows. 
\begin{equation} 
\delta f  =  - \frac{\sphericalangle  y(t-\delta t_l) y^*(t) }{2\pi\delta t_l} \label{eqn:finer_cfo}
\end{equation}
To this extent, considering the join subcarrier $f$, the {\slshape ppm (parts per million)} on the BS's crystal is given by $ \text{ppm}_\text{BS} = 10^6  \big(\frac{\delta f}{f}\big) $. Thus, the BS calculates $ \delta f_i$ on subcarrier $f_i$ (assigned for node $i$) as 
$\delta f_i =  \frac{(f_i * \text{ppm}_\text{BS})}{10^6}.$ The CFO between the Tx-Radio and the Rx-radio can be estimated using a basic SISO CFO estimation technique~\cite{yao2005blind}. Thus, BS also knows the CFO for the downlink.


We now explain the CFO estimation to compensate for the Doppler shift. Note that if the signal bandwidth is sufficiently narrow at a given carrier frequency and mobile velocity, the Doppler shift can be approximated as a common shift across the entire signal bandwidth~\cite{talbot2007mobility}. Thus, the Doppler shift in the join subcarrier for a node also represents the Doppler shift at its assigned subcarrier, and hence the estimated CFO in Equation (\ref{eqn:finer_cfo}) is not affected due to the Doppler Shift.
For simplicity, we consider that a node's velocity is constant and the change in Doppler shift is negligible during a single packet transmission in SNOW.
Considering $\delta f_d$ as the CFO due to the Doppler shift, $v$ as the velocity of the node, and $\theta$ as the angle of the arrived signal at the BS from the node, we have $f_d = f_i\big(\frac{v}{c}\big)\cos(\theta)$~\cite{talbot2007mobility}, where
% \begin{equation}
% 	\nonumber \delta f_d = f_i\big(\frac{v}{c}\big)\cos(\theta).
% \end{equation}
$f_i$ is the subcarrier center frequency and $c$ is the speed of light. The node itself may consider its motion as circular and approximate $\theta = \frac{\delta s}{r}$, where $\delta s$ is the amount of anticipated change in position during a packet transmission and $r$ is the {\em line-of-sight} distance between the node and BS. Thus, CFO compensation due to the Doppler shift is done at the nodes during uplink communications. In the downlink communications, the BS Tx-Radio can also compensate for the node's mobility as the node can report its Doppler shift to the BS during the uplink communications.

In summary, as the nodes asynchronously transmit, estimating joint-CFO of the subcarriers at the BS is very difficult. We thus use a simple feedback approach for proactive CFO correction in the uplink communications. Specifically, 
$\delta f_i$  estimated at the BS for subcarrier $f_i$ is given to the node (during joining process at subcarrier $f_i$).
The node may then adjust its transmitted signal based on $\delta f_i$ and $\delta f_d$, calculated as $(\delta f_i + \delta f_d)$, which will align its signal so that the BS does not need to compensate for CFO in the uplink communications. Such feedback-based proactive compensation scheme was studied before for multi-user OFDM and is also used in global system for mobile communication (GSM)~\cite{van1999time}.

\section{Handling the Near-Far Power Problem} \label{sec:near-far}
\begin{figure}[!htb]
\centering
\includegraphics[width=0.5\textwidth]{figs/near-far-flat.eps}
\caption{An illustration of the near-far power problem. B is farther from the BS than A and both transmit concurrently using the same Tx power.}
\label{fig:near-far}
\end{figure}
Wireless communication is susceptible to the near-far power problem, especially in CDMA (Code Division Multiple Access)~\cite{muqattash2003cdma}. Multi-user D-OFDM system in SNOW may also suffer from this problem. Figure~\ref{fig:near-far} illustrates the near-far power problem in SNOW. Suppose, nodes A and B are operating on two adjacent subcarriers. Node A is closer to the BS compared to node B. When both nodes A and B transmit concurrently to the BS, the received frequency domain signals from node A and B may look as shown on the right of Figure~\ref{fig:near-far}. Here, transmission from node B is severely interfered by the strong radiations of node A's transmission. As such, node B's signal may be buried under node A's signal making it difficult for the BS to decode the packet from node B. 
A typical SNOW deployment may have such scenarios if the nodes operating on adjacent subcarriers use the same transmission power and transmit concurrently at the BS from different distances. 
\begin{figure}[t]
    \centering 
      \subfigure[Avg. PDR at different Tx powers\label{fig:nf_pdr}]{
    \includegraphics[width=0.35\textwidth]{figs/nearfar/pdr.eps}
      }\hfill
      \subfigure[Avg. PDR at different Tx powers and time\label{fig:nf_time}]{
        \includegraphics[width=.35\textwidth]{figs/nearfar/pdr-time.eps}
      }
    \caption{Packet delivery ratio at different Tx powers}
    \label{fig:nf-effects}
 \end{figure}


To observe the near-far power problem in SNOW, we run experiments by choosing 3 different adjacent subcarriers, where the middle subcarrier observes the near-far power problem introduced by both subcarriers on its left and right. We place two CC1310 nodes within 20m of the BS that use the left and the right subcarrier, respectively. We use another CC1310 node that uses the middle subcarrier and is placed at different distances between 200 and 1000m from the BS. Nodes that are within 20m of the BS transmit packets continuously with a transmission power of 0dBm. At each distance, for each transmission power between 8 and 15dBm, the node that uses the middle subcarrier sends 100 rounds of 1000 consecutive packets (sends one packet then waits for the ACK and then sends another packet, and so on) to the BS and with a random interval of 0-500ms. For each transmission power level, at each distance, that node calculates its average {\em packet delivery ratio (PDR)}. PDR is defined as the ratio of the number of successfully acknowledged packets to the number of total packets sent.
We repeat the same experiments for 7 days at 9 AM, 2 PM, and 6 PM.

Figure~\ref{fig:nf_pdr} shows that the average PDR increases at each distance with the increase in the transmission power. Figure~\ref{fig:nf_time} depicts the result for 7-day experiments (only at a distance of 200m) and shows that the average PDR changes at different time of the day. Overall, Figure~\ref{fig:nf_pdr} and~\ref{fig:nf_time} confirms that the average PDR increases with the increase in the transmission power. To ensure the energy-efficiency at the nodes, i.e., to find a  transmission power  that suffices to eliminate the effects of near-far power problem, we propose an adaptive transmission power control for the SNOW design, as described below.

% To demonstrate the effects of near-far problem, we run experiments in SNOW by placing 5 nodes at different distances ranging between 200-1000m. The subcarriers assigned to the nodes are chosen in a way such that they observe the near-far problem as shown in Figure~\ref{fig:nf-effects}.  
% At each distance, a node transmits 1000 packets using a fixed transmission power (concurrently with other nodes at other distances using the same transmission power) and calculate its packet delivery ratio (PDR). A packet is correctly delivered if the node receives an ACK for that packet. At each location we vary the transmission power between 0-15dBm and repeat the same strategy. Figure~\ref{fig:nf_rssi} shows that the average RSSI at the BS increases with the increase in the transmission power, as expected. However, Figure~\ref{fig:nf_pdr} shows that the PDR at nodes at different distances vary unexpectedly. For example, node at distance 400m observes very low PDR due to the node at 200m, and so on. Thus, the near-far problem needs to be addressed in SNOW. To this extent, we propose an adaptive transmission power control in SNOW.


\subsection{Adaptive Transmission Power Control}\label{sec:atpc}
Our design objective for the adaptive Tx power control is to correlate the subcarrier-level Tx power and link quality (i.e., PDR) between each node and the BS. We thus formulate a predictive model to provide each node with a proper Tx power to make a successful transmission to the BS using its assigned subcarrier. Note that our work differs from the work in~\cite{lin2016atpc} in fundamental concepts of the network design and architecture. In~\cite{lin2016atpc}, the authors have considered a multi-hop wireless sensor network based on IEEE 802.15.4 with no concurrency between a set of transmitters and a receiver. Additionally, our model is much more simpler since we deal with single hop communications. As such, the overheads (i.e., energy consumption and latency at each node) associated with our model are fundamentally lesser than that in~\cite{lin2016atpc}, or the other techniques developed for multi-hop wireless networks~\cite{son2006experimental, li2005cone}. In the following, we describe our model.


Whenever a node is assigned a new subcarrier or observes a lower PDR, e.g., PDR below quality of service (QoS) requirements due to mobility, it runs a lightweight predictive model to determine the convenient Tx power to make successful transmissions to the BS.
Our predictive model uses an approximation function to estimate the PDR distribution at different Tx power levels. Over time, that function is modified to adapt to the node's changes. The function is built from the sample pairs of the Tx power levels and PDRs between the node and the BS via a curve-fitting approach. A node collects these samples by sending groups of packets to the BS at different Tx power levels. A node may not be assigned new subcarriers or may not observe lower PDR due to mobility (as per our CSI and CFO estimations) frequently. Thus, the overhead (e.g., energy consumption) for collecting these samples may be negligible compared to the overall network lifetime (which is several years).

Specifically, our predictive model uses two vectors: $TP$ and $L$, where $TP = \{ tp_1, tp_2, \cdots, tp_m \}$ contains $m$ different Tx power levels that the node uses to send $m$ groups of packets to the BS and $L = \{ l_1, l_2, \cdots, l_m \}$ contains the corresponding PDR values at different Tx power levels. Thus, $l_i$ represents the PDR value at Tx power level $tp_i$. We use the following linear function to correlate between Tx power and PDR.
\begin{equation}
	l(tp_i) = a~.~tp_i + b \label{eqn:linear_model}
\end{equation}
To lessen the computational overhead in the node, we adopt the {\em least square approximation} technique to determine the unknown coefficients $a$ and $b$ in Equation (\ref{eqn:linear_model}). Thus, we find the minimum of the function $S(a, b)$, where $\nonumber S(a, b) = \sum |l_i - l(tp_i)|^2.$
% \begin{equation}
% \nonumber	S(a, b) = \sum |l_i - l(tp_i)|^2.
% \end{equation}
The minimum of $S(a, b)$ is determined by taking the partial derivatives of $S(a, b)$ with respect to $a$ and $b$, respectively, and setting them to zero. Thus, $ \frac{\partial S}{\partial a} = 0$ and $\frac{\partial S}{\partial b} = 0$ give us
\begin{align}
	\nonumber a~\sum (tp_i)^2 + b~\sum tp_i &= \sum l_i.tp_i \text{ and} \\ 
  \nonumber a~\sum tp_i + b~m &= \sum l_i.
\end{align}
Simplifying the above two equations, we find the estimated values of $a$ and $b$ as follows.
\begin{equation}\nonumber
\begin{split}
	\begin{bmatrix}
		\hat{a}\\
        \hat{b}
	\end{bmatrix}
    = \frac{1}{m \sum (tp_i)^2 - (\sum tp_i)^2} \times \\
    \begin{bmatrix}
    	m \sum l_i.tp_i - \sum l_i \sum tp_i\\
    	\sum l_i \sum (tp_i)^2 - \sum l_i.tp_i \sum tp_i
    \end{bmatrix}
\end{split}
\end{equation}
Using the estimated values of $a$ and $b$, the node can calculate the appropriate Tx power as follows.
\begin{equation}\label{eqn:estimated}
tp = \big[\frac{PDR_{\text{threshold}} - \hat{b}}{\hat{a}}\big] \in TP
\end{equation}
Here, $PDR_{\text{threshold}}$ is the threshold set empirically or according to QoS requirements, and $[.]$ denotes the function that rounds the value to the nearest integer in the vector $TP$.

Now that the initial model has been established in Equation (\ref{eqn:estimated}), this needs to be updated continuously with the node's changes over time. In Equation (\ref{eqn:linear_model}), both $a$ and $b$ are functions of time that allow the node to use the latest samples to adjust the curve-fitting model dynamically. 
It is empirically found that (Figure~\ref{fig:nf_pdr}) the slope of the curve does not change much over time; hence $a$ is assumed time-invariant in the predictive model. On the other hand, the value of $b$ changes drastically over time (Figure~\ref{fig:nf_time}). Thus, Equation (\ref{eqn:linear_model}) is rewritten as follows that characterizes the actual relationship between Tx power and PDR.
\begin{equation}
	\nonumber l(tp(t)) = a.tp(t) + b(t)
\end{equation}
Now, $b(t)$ is determined by the latest Tx power and PDR pairs using the following feedback-based control equation~\cite{lin2016atpc}.
\begin{align}
	\nonumber \Delta \hat{b}(t) &= \hat{b}(t) - \hat{b}(t+1) \\
    			\nonumber	  &= \frac{\sum^K_{k=1} [PDR_{\text{threshold}} - l_k(t - 1)]}{K} \\ 
                      &= PDR_{\text{threshold}} - l(t-1) \label{eqn:control}
\end{align}
Here, $l(t-1)$ is the average value of $K$ readings denoted as 
\begin{equation}
	\nonumber l(t-1) = \frac{\sum^K_{k=1} l_k(t - 1)}{K}.
\end{equation}
Here, $l_k(t-1)$, for $k = \{1, 2, \cdots, K\}$, is one reading of PDR during the time period $t-1$ and $K$ is the number of feedback responses at time period $t-1$. Now, the error in Equation (\ref{eqn:control}) is deducted from the previous estimation; hence the new estimation of $b(t)$ can be written as: $\hat{b}(t) = \hat{b}(t-1) - \Delta \hat{b}(t)$.
Given the newly estimated $\hat{b}(t)$, the node now can set the Tx power at time $t$ as
\begin{equation}
	\nonumber tp(t) = \big[\frac{PDR_{\text{threshold}} - \hat{b}(t)}{\hat{a}}\big].
\end{equation}













    % This file has Technical section
%\section{Schedule Independent Memory Allocation}
\label{sec:sima}

We also address a memory-based limitation of polyhedral compilation
tools.  It is well known that in any parallelization (of any program), it is
essential to respect (only) the \emph{true} or flow dependences.  Other
(memory-based) dependences can be ignored if one can re-allocate memory.  In
practice, this is limited by the fact that the associated memory expansion may
be prohibitively expensive, and there has been work on mitigating this
expansion~\cite{vasilache-impact12, lefebvre-feautrier-pc98, sanjay-europar96,
  sanjay-toplas00}.  We propose a novel yet simple \emph{schedule-independent}
memory allocation strategy.  Our work also generalizes polyhedral compilation
by enabling polyhedral tools to use alternate, \emph{hybrid} schedules
consisting of affine loops for certain parts of the iteration space and
cache-oblivious divide-and-conquer schedules for others.


\subsection{Background}

In this section, we introduce the necessary background of our work. We first
give a brief description of the polyhedral representation of programs, and the
general flow of a polyhedral compiler.  Then, we discuss the legality of
tiling, which is related to the input of our code generator.

\begin{figure*}[tb]
  \centering %\vspace*{6cm}
  \includegraphics[scale=0.6]{PolyCompiler}
  \caption{\small{Polyhedral Compilation: the Polyhedral Reduced Dependence
      (hyper) Graph (PRDG) serves as the intermediate representation.
      Piecewise Quasi-Affine Functions (PQAFs) describe transformations.}}
  \label{fig:compiler}
\end{figure*}

\subsubsection{Polyhedral Compilation and Representation}

Figure~\ref{fig:compiler} shows the flow of polyhedral compilation.  First,
dependence analysis of an input program (or a ``polyhedral section'' thereof)
produces an intermediate representation (IR) in the form of~\cite{DRV-sched00}
a \emph{Polyhedral Reduced Dependence (hyper) Graph} (PRDG).  Various analyses
are performed on the PRDG to choose a number of mappings in the form of
\emph{Piecewise Quasi-Affine Functions} (PQAFs) that specify the schedule as a
set of \emph{multi-dimensional} vectors.  The PQAFs come with annotations to
indicate whether each dimension is sequential or parallel, and also whether it
is part of a \emph{tilable band}, i.e., whether tiling this band of dimensions
is legal.  The transformations may be applied to the PRDG iteratively, and
(eventually) the PRDG and QLAF are provided to a code-generator that produces
code for various targets.

% specify which dimensions are sequential, which dimensions are (and
% implicitly, also the algorithm~\cite{uday-pldi08} to get tiling hyperplanes
% and tilable dimensions for each statement which is called \emph{tilable
% band}.  Then we transform the program using tiling hyperplanes.  This
% results in a program where hyper-rectangular tiles are legal and the
% wavefront parallel execution order is a legal schedule for executing tiles.
% We also provide schedule independent memory allocations for all the
% variables.  Finally we generate code which traverse the iteration space of
% the program in divide and conquer order.  by post processing the AST of
% tiled code generated by DTiler The following section presents the schedule
% independent memory allocation for affine programs.

One of the strengths of the polyhedral model is that a parametric program may
be concisely represented with a PRDG with finite number of nodes (statements)
and edges (dependences).  The potentially unbounded sets of instances of a
statement are represented in abstract forms of integer sets, called
\emph{domains}, and dependences between them as affine functions (or
relations, which are viewed as a set-valued function) over these statement
domains.  Indeed, every edge, $e$ from node $v$ to $w$, in the PRDG is
annotated with two objects: (i) a domain, $D_e$ specifying the (subset of) the
domain, $D_v$ of its source node, where the dependence occurs, and (ii) the
affine function, $f$, such that for any point $z\in D_e$, the (set of)
point(s) in $D_w$ on which it depends is given by $f(z)$.  $D_e$ is called the
context of the edge, and $f$ is its dependence function.  We also use the
notation $f(D_e)$ to denote the set valued image of $D_e$ by $f$.

An affine function $\mathbb{Z}^n \rightarrow \mathbb{Z}^m$ may be expressed as
$f(x) = A\vec{x} + \vec{b}$, where $\vec{x}$, function domain, is an integer
vector of size $n$; $A$, linear part, is an $n\times m$ matrix; and $\vec{b}$,
constant part, is an integer vector of size $m$.  A dependence is said to be
uniform if the dependence function is only a constant offset, i.e., when the
linear part $A$ is the identity.

%  is used to get the  tilable dimensions....  The parallelism that can be
%  explored using tiles is assumed to be wavefront....??  However, we modify
%  the execution order of tiles and schedule them in a divide-and-conquer
%  fashion.... Updating the memory allocation schemes becomes important when
%  we change the order of execution of tiles.... The following Section talks
%  about Memory Allocation...

\subsubsection{Legality of Tiling}

Tiling is a well-known loop transformation for partitioning computations into
smaller, atomic (all inputs to a tile can be computed before its execution),
units called tiles~\cite{irigoin-popl88, Wolf91tiling}.  The natural legality
condition is that the dependences across tiles do not create a cycle.  In
compilers, this condition is typically expressed as fully permutability (i.e.,
dependences are non-negative direction vectors), which is a sufficient
condition.  Our transformation for cache oblivious tiling takes as inputs a
loop nest that is fully permutable.  For polyhedral programs, scheduling
techniques to expose such loop nests are available~\cite{uday-pldi08}.


\subsection{Memory Allocation}

\begin{figure}[tb]
  \centering % \vspace*{4cm}
{\small\begin{lstlisting}
for (i = 0; i < N; i++){
  S0:  X[0,i] = A[i]; } // Initialize
for (t = 1; t <= 2*N; t++){ //Note: ub is even
  for (i = 1; i < N-1; i++){
    S1: X[t%2][i] = f(X[(t-1)%2][i-1],
             X[(t-1)%2][i], X[(t-1)%2][i+1],
             X[(t-1)%2][0]);
    }
  S2: X[t%2][0] = g(X[(t-1)%2][0]); 
  S3: X[t%2][N-1] = g(X[(t-1)%2][N-1]);
}
for (i = 0; i < N; i++){
  S4: Aout[i] = X[0,i]; } // Output copy
\end{lstlisting}
}
\caption{\small{Neither Pochoir nor Autogen can handle the computation
    performed by this simple loop.  Moreover, it has a memory based dependence
    that prevents polyhedral compilers like Pluto from tiling both dimensions.
    However, the true dependences of the program admit a tilable schedule, but
    at the potentical cost of $O(N^2)$ memory.  Our scheme reduces this to
    $O(N)$}}
\label{fig:motiv}
\end{figure}


In this section, we first describe how memory based dependences prevent
tiling, using our motivating example (Fig.~\ref{fig:motiv}), and show that
simply ignoring these (false) dependences would lead to memory explosion.
After formulating our problem, we next propose a simple, schedule independent
memory allocation scheme that resolves it.

Consider the statement $\mathrm{S1}$, and note that its domain, $D_1$ is the
polyhedral set, $\{t,i~|~ 1\leq t\leq 2N \wedge 1\leq i\leq N-1 \}$.
$\mathrm{S1}$ has four true dependences (for points sufficiently far from the
boundaries), three of which are $\mathrm{S1}[t-1, i-1]$, $\mathrm{S1}[t-1, i]$
and $\mathrm{S1}[t-1, i+1]$, the typical, 1D-Jacobi stencil dependences, and
the fourth one is $\mathrm{S2}[t-1, 0]$, which is a truly affine dependence on
the most recent writer to the memory location $\mathtt{X[(t-1)\%2,0]}$ when
the statement $\langle \mathrm{S1}, [t, i]\rangle$ is being executed.  All
these dependences are captured as edges with affine \emph{functions} in the
PRDG.  In addition, there is a memory based dependence, that we must also
respect.  Consider statement $\mathrm{S2}$, whose domain, $D_2 = \{t~|~ 1\leq
t\leq 2N\}$ is just one dimensional.  The $t$-th instance of S2 \emph{(over)
  writes} $\mathtt{X[t\%2, 0]}$, therefore all computations that read the
previous value must be executed before it.  In this sense, $\mathrm{S2}[t]$
``depends on'' the set $\mathrm{S1}[t,i]$, for all $1\leq i\leq N-1$.  This
dependence (which is a \emph{relation} rather than a function) is captured by
another a special edge in the PRDG.

The only schedule that respects all these dependences is the family of lines
parallel to the $t$ axis (provided all iterations of S1 are done first).
Although this has maximal parallelism, it has very poor locality.  Note that
the Pluto scheduler does not seek maximal parallelism, but rather, to maximize
the \emph{number of linearly independent tiling hyperplanes}.  Unfortunately,
the $t=\mathrm{const}$ is the only legal tiling hyperplane for this set of
dependences, and the tilable band obtained by Pluto is only 1-dimensional.

What if we did not have the memory-based dependences, i.e., what if we ignored
the memory allocation of the original program, and stored each computed value
in a distinct memory location?  In this case, there would be no memory based
dependences, and we can indeed find another family of (actually, infinitely
many) legal tiling hyperplanes: say, the lines $i+t=\mathrm{const}$.  As a
result, if we use the mapping $(t,i) \mapsto (t,i+t)$ as our ``schedule,'' the
new loops in the transformed program would be fully permutable, and could be
legally tiled.

Thus, the problem we seek to solve is: \emph{how to avoid memory based
  dependences, but without the cost of memory expansion that it seems to
  imply}.

Memory allocation for polyhedral programs is a well studied problem, and there
are two main approaches.  One either does memory allocation after the schedule is
chosen~\cite{sanjay-europar96, degreef-memory97, lefebvre-feautrier-pc98,
  sanjay-toplas00, darte-lattice05, vasilache-impact12,
  bhaskaracharya-toplas16, bhaskaracharya-popl16} since it often leads to a
smaller memory footprint, or else uses a \emph{schedule independent} memory
allocation, based on the so called \emph{universal occupancy vectors} (UOV).
This problem is solved when the program has \emph{uniform dependences}, i.e.,
when each dependence can be described by a \emph{constant vector}, and for some
simple extensions of this~\cite{strout-etal-asplos98, sanjay-memory-2011}.

It is important to note that tiling actually modifies a schedule: the so
called, ``schedule dimensions'' become fully permutable loops, and indeed,
these loops \emph{are actually permuted} in the generated tiled code.  So,
when a \emph{tiling schedule} specified by a family of $d$ tiling hyperplanes
is finally implemented by the generated code, the actual time-stamps are not
really $d$-dimensional vectors, but rather $2d$-dimensional ones obtained as
some complicated function of these indices.  Furthermore, we will see when we
generate cache-oblivious tiled codes, these tilable loops will actually be
visited in the divide-and-conquer order of execution, as required by COT.  As
a result, finding a memory map that takes into account such a rather
complicated final schedule is a tricky problem.  We therefore seek and propose
schedule-independent memory allocations.

The intuition behind our solution is (deceptively) simple, and we first
illustrate it on our motivating example (Fig.~\ref{fig:motiv}).  Rather than
the so-called ``single assignment'' program for the entire iteration space of
the program (i.e., full memory expansion), could we find lower-dimensional
subsets, such that a single assignment memory for only these subsets is
sufficient?  A careful examination of the code reveals that the memory based
dependences arise due to statement S2, and its domain is only 1-dimensional.
So we store the results of this statement into an auxiliary array, \texttt{Y},
and modify the program so that the fourth dependence simply reads
\texttt{Y[t-1]}, rather than \texttt{X[(t-1)\%2,0]}.  For the variable,
\texttt{X}, we use the old $(t,i) \mapsto (t\%2,i)$ memory allocation that was
used in the original code.  This results in $4N$ memory, which is a polynomial
degree better than quadratic.  Of course, the challenge is how to discover
this automatically.

% PLUTO schedule give a schedule and a tiling band.  A point in a tiling band
% corresponds to a particular tile.  All points within a tile execute
% sequentially.  However, the tiles can be executed in parallel.  Consider the
% modified Jacobi-1D stencil (CHANGE THIS TO BE CONSISTENT WITH RUNNING
% EXAMPLE).  The Figure \textbf{ABC} shows the iteration space of Jacobi-1D
% example (CHANGE THIS TO BE CONSISTENT WITH RUNNING EXAMPLE).  The blue bars
% show the inputs and outputs of the program corresponding to statements
% $S_{0}$ and $S_{4}$ respectively.  The memory allocation scheme used
% initially is modulo memory allocation.  However, this modulo memory
% allocation is not affine and hence PLUTO is unable to find a schedule. We
% make this memory map to not to use modulo but use previous and current
% instead. Even then, PLUTO is unable to find a schedule such that all
% dimensions are tilable.  PLUTO will decide to tile this program with single
% assignment memory.  Identity memory allocation is known to be overkill and
% lead to inefficient codes.  Therefore, we need a schedule independent memory
% allocation scheme which guarantees that the given memory allocation is both
% legal as well as optimal for any given schedule.

% The problem of schedule independent memory allocation for Polyhedral
% programs is a partially solved problem. [Strout et al] presented an
% algorithm for schedule independent memory allocation for a class of programs
% that have only uniform dependences.  A set of programs with uniform
% dependences is a proper subset of the set of programs with affine
% dependences, the class of Polyhedral programs.  Our recursive
% divide-and-conquer code generator is applicable to all Polyhedral programs
% in general, including the ones that have truly affine non-uniform
% dependences. We therefore present a novel scheme for schedule independent
% memory allocation for all affine programs.

We now outline how this is done.  At a high level, our algorithm takes a PRDG
as input, applies some (piecewise) affine transformations to it, and outputs
the transformed PRDG together with a separate memory map for each node in the
transformed PRDG.  More specifically, it works as follows.

\begin{itemize}
\item \emph{Preprocessing.}  For each edge, $e$, in the PRDG, with context
  $D_e$, and function, $f$, we first identify whether $f$ is \emph{uniform in
    context} in the sense that, for all points, $z\in D_e$, the value of
  $z-f(z)$ is a constant vector, independent of $z$.

  For example, consider a dependence function, $(i,j) \mapsto (i-1,i-1)$
  which, maps any point $[i,j]$ in the plane to a point on the diagonal, and
  is clearly not uniform.  However, what if $D_e = \{i,j~|~i=j-1\}$?  With
  this contextual information, the dependence is actually uniform: $(i, j)
  \mapsto (i-1, j-2)$.

  All edges/dependences that are neither uniform to begin with, not uniform in
  context, are marked as \emph{truly affine}.
\item \emph{Affine Split.}  For every node, $v$, in the PRDG that has at least
  one truly affine edge $e$ incident on it, we create a new node, $v'$.  Its
  domain $D_{v'}$ is the union of $f(D_e)$ of all such incident edges.

  The edges in the PRDG are modified as follows.  All the truly affine edges
  that were incident on $v$ are now made incident on $v'$; and $v'$ has a
  single outgoing edge $e'$, annotated with $\langle D_{v'}, I \rangle$ (its
  dependence function is the identity map) and whose destination is $v$.

  It is easy to see that we have not changed the program semantics.  In
  effect, we have simply copied the value of every point in $D_v$ that was the
  target of any truly affine dependence over to a new variable $v'$, and
  ``diverted'' all the truly affine edges that used to be incident on $v$ over
  to $v'$.  Moreover, since the identity function is uniform by definition, all
  edges incident on $v$ are now either uniform, or uniform in context.
\item We now use existing UOV based methods~\cite{strout-etal-asplos98,
    sanjay-memory-2011} to choose a schedule-independent memory allocation for
  all the original nodes in the PRDG, and a \emph{single-assignment} memory
  allocation for all the newly introduced variables.
\end{itemize}

The key insight into why this leads to significant memory savings, is the fact
that in all polyhedral programs that we encountered, truly affine dependences
are almost always \emph{rank deficient}, i.e., are many-to-one mappings from
the consumer index points to the producers.  The only exceptions are either
pathological programs, or programs that do multi-dimensional data
reorganizations via bijections (e.g., matrix transpose, tensor permutations,
etc.) where here is no scope nor need to reduce the total memory footprint.
As a result, $f(D_e)$ is almost always a lower dimensional polyhedron, and
requires significantly less memory, even when stored supposedly inefficiently.


\subsection{Related Work}

Memory allocation for polyhedral programs is a well studied problem for almost
two decades.  DeGreef and Cathoor~\cite{degreef-memory97} tackled the problem
of sharing the memory across multiple arrays in the program.the so called
inet-array memory reuse problem, and proposed an ILP based solution.  Wilde
and Rajopadhye, in dealing with an intrinsically memory-inefficient functional
language Alpha~\cite{mauras1989thesis} (one can think of this as a program after
full expansion) first addressed the memory reuse for points of an iteration
space~\cite{sanjay-europar96}.  They gave necessary and sufficient conditions
for the legality of a memory allocation fucntion, which they allowed to be
``in any direction.''  but they did not provide any insight into how to choose
the mapping.  Lefebvre and Feautrier~\cite{lefebvre-feautrier-pc98} on the
other hand, considered only canonic projections, combined with a modulo
factor, but showed how to choose the mapping optimally.  Later, Quiller\'e and
Rajopadhye~\cite{sanjay-toplas00} revisited multiprojections, extended them to
quasi-affine functions, and proved a tight bound on the number of dimensions
of reuse.  They also showed that cananic projections with modulo factors was
sometimes a constant factor better, and sometimes a constant factor worse.
Darte at al.~\cite{darte-lattice05} took a fresh and elegant approach to the
problem, and formulated the conditions for legal memory allocations by
defining the \emph{conflict set}.  This led to techniques for choosing
provably optimal memory allocations, initially for non-parameterized iteration
spaces, and recently in the context of FPGA acelerators, for parametrically
tiled spaces~\cite{darte2014parametric, darte2016extended}.  Vasilache et
al.~\cite{vasilache-impact12} developed a tool to combine the scheduling and
limiting memory expansion using an ILP formulation, implemented in the
R-Stream compiler.  Recently, Bhaskaracharya et
al.~\cite{bhaskaracharya-toplas16} developed methods to optimally choose
quasi-affine memory allocations, and showed how they are beneficial for tiled
codes, especialy with live-out data.  Furthermore, they also
showed~\cite{bhaskaracharya-popl16} how to combine iner-and intra array reuse
in a unifying framework.

The other \emph{schedule independent} memory allocation was pioneered by
Strout et al.~\cite{strout-etal-asplos98}.  Here, the memory allocation is
chosen based only on the dependences, and is guaranteed to be legal,
regardless of the schedule.  This problem is solved when the program has
\emph{uniform dependences}, i.e., when each dependence can be descibed by a
\emph{constant vector}, and for some simple extensions of
this~\cite{strout-etal-asplos98, sanjay-memory-2011}.

Thies at al.~\cite{thies-pldi02} have also formulated the problem of
simultaneously choosing the schedule and memory allocation as a combined
optimization problem.
% \begin{algorithm}[h]
%   \mbox{} Input : PRDG
  
%     Output : Transformed PRDG and Memory Map
%     \begin{enumerate}
%     \item Recognize truly non-uniform dependence edges
  
%       For each node $X_{i}$ with one or more incoming non-uniform affine
%       edge(s):
%       \begin{itemize}
%       \item Create new node $X_{i}^{new}$ by applying ``Affine Split"
%         transformation
    	
%         Assertion: $X_{i} = X_{i}^{old} + X_{i}^{new}$

%       \end{itemize}
%     \item Find universal occupancy vector $(UOV)$ for all nodes with uniform
%       dependences
%     \item Construct memory mapping function for all nodes except
%       $X_{i}^{new}$ using $UOV$
% 	\item Use identity memory map for $X_{i}^{new}$ nodes
% 	\item \textbf{(SANJAY verify 4 , 5)} Apply change of basis to reduce
%    number of dimensions of the domain of new node
%  \end{enumerate}
%  \caption{SIMA: Schedule Independent Memory Allocation for Polyhedral
%  Programs}
%  \label{alg:sima}
% \end{algorithm}

% Local Variables: ***
% TeX-master: "PACT17.tex" ***
% fill-column: 78 ***
% End: ***
          % This file has Technical result on memory allocation
%\section{Our Approach}
We formulate the problem as an anisotropic diffusion process and the diffusion tensor is learned through a deep CNN directly from the given image, which guides the refinement of the output.

\begin{figure}[t]
\includegraphics[width=1.0\textwidth]{fig/CSPN_SPN2.pdf}
\caption{Comparison between the propagation process in SPN~\cite{liu2017learning} and CPSN in this work.}
\label{fig:compare}
\end{figure}

\subsection{Convolutional Spatial Propagation Network}
% demonstrate the thereom is hold when turns to be convolution.
Given a depth map $D_o \in \spa{R}^{m\times n}$ that is output from a network, and image $\ve{X} \in \spa{R}^{m\times n}$, our task is to update the depth map to a new depth map $D_n$ within $N$ iteration steps, which first reveals more details of the image, and second improves the per-pixel depth estimation results. 

\figref{fig:compare}(b) illustrates our updating operation. Formally, without loss of generality, we can embed the $D_o$ to some hidden space $\ve{H} \in \spa{R}^{m \times n \times c}$. The convolutional transformation functional with a kernel size of $k$ for each time step $t$ could be written as,
\begin{align}
    \ve{H}_{i, j, t + 1} &= \sum\nolimits_{a,b = -(k-1)/2}^{(k-1)/2} \kappa_{i,j}(a, b) \odot \ve{H}_{i-a, j-b, t} \nonumber \\
\mbox{where,~~~~}
    \kappa_{i,j}(a, b) &= \frac{\hat{\kappa}_{i,j}(a, b)}{\sum_{a,b, a, b \neq 0} |\hat{\kappa}_{i,j}(a, b)|}, \nonumber\\
    \kappa_{i,j}(0, 0) &= 1 - \sum\nolimits_{a,b, a, b \neq 0}\kappa_{i,j}(a, b)
\label{eqn:cspn}
\end{align}
where the transformation kernel $\hat{\kappa}_{i,j} \in \spa{R}^{k\times k \times c}$ is the output from an affinity network, which is spatially dependent on the input image. The kernel size $k$ is usually set as an odd number so that the computational context surrounding pixel $(i, j)$ is symmetric.
$\odot$ is element-wise product. Following~\cite{liu2017learning}, we normalize kernel weights between range of $(-1, 1)$ so that the model can be stabilized and converged by satisfying the condition $\sum_{a,b, a,b \neq 0} |\kappa_{i,j}(a, b)| \leq 1$. Finally, we perform this iteration $N$ steps to reach a stationary distribution.

% theorem, it follows diffusion with PDE 
%\addlinespace
\noindent\textbf{Correspondence to diffusion process with a partial differential equation (PDE).} \\
Similar with~\cite{liu2017learning}, here we show that our CSPN holds all the desired properties of SPN.
Formally, we can rewrite the propagation in \equref{eqn:cspn} as a process of diffusion evolution by first doing column-first vectorization of feature map $\ve{H}$ to $\ve{H}_v \in \spa{R}^{\by{mn}{c}}$.
\begin{align}
     \ve{H}_v^{t+1} = 
     \begin{bmatrix}
    1-\lambda_{0, 0}  & \kappa_{0,0}(1,0) & \cdots & 0 \\
    \kappa_{1,0}(-1,0)   & 1-\lambda_{1, 0} & \cdots & 0 \\
    \vdots & \vdots & \ddots & \vdots \\
    \vdots & \cdots & \cdots & 1-\lambda_{m,n} \\
\end{bmatrix} = \ve{G}\ve{H}_v^{t}
\label{eqn:vector}
\end{align}
where $\lambda_{i, j} = \sum_{a,b}\kappa_{i,j}(a,b)$ and $\ve{G}$ is a $\by{mn}{mn}$ transformation matrix. The diffusion process expressed with a partial differential equation (PDE) is derived as follows, 
\begin{align}
     \ve{H}_v^{t+1} &= \ve{G}\ve{H}_v^{t} = (\ve{I} - \ve{D} + \ve{A})\ve{H}_v^{t} \nonumber\\
     \ve{H}_v^{t+1} - \ve{H}_v^{t} &= - (\ve{D} - \ve{A}) \ve{H}_v^{t} \nonumber\\
     \partial_t \ve{H}_v^{t+1} &= -\ve{L}\ve{H}_v^{t}
\label{eqn:proof}
\end{align}
where $\ve{L}$ is the Laplacian matrix, $\ve{D}$ is the diagonal matrix containing all the $\lambda_{i, j}$, and $\ve{A}$ is the affinity matrix which is the off diagonal part of $\ve{G}$.

In our formulation, different from~\cite{liu2017learning} which scans the whole image in four directions~(\figref{fig:compare}(a)) sequentially, CSPN propagates a local area towards all directions at each step~(\figref{fig:compare}(b)) simultaneously, \ie with~\by{k}{k} local context, while larger context is observed when recurrent processing is performed, and the context acquiring rate is in an order of $O(kN)$.

In practical, we choose to use convolutional operation due to that it can be efficiently implemented through image vectorization, yielding real-time performance in depth refinement tasks.

Principally, CSPN could also be derived from loopy belief propagation with sum-product algorithm~\cite{kschischang2001factor}. However, since our approach adopts linear propagation, which is efficient while just a special case of pairwise potential with L2 reconstruction loss in graphical models. Therefore, to make it more accurate, we call our strategy convolutional spatial propagation in the field of diffusion process.

\begin{figure}[t]
\centering
\includegraphics[width=0.9\textwidth]{fig/hist.pdf}
\caption {(a) Histogram of RMSE with depth maps from~\cite{Ma2017SparseToDense} at given sparse depth points.  (b) Comparison of gradient error between depth maps with sparse depth replacement (blue bars) and with ours CSPN (green bars), where ours is much smaller. Check~\figref{fig:gradient} for an example. Vertical axis shows the count of pixels.}
\label{fig:hist}
\end{figure}

\subsection{Spatial Propagation with Sparse Depth Samples}
In this application, we have an additional sparse depth map $D_s$ (\figref{fig:gradient}(b)) to help estimate a depth depth map from a RGB image. Specifically, a sparse set of pixels are set with real depth values from some depth sensors, which can be used to guide our propagation process. 

Similarly, we also embed the sparse depth map $D_s = \{d_{i,j}^s\}$ to a hidden representation $\ve{H}^s$,  and we can write the updating equation of $\ve{H}$ by simply adding a replacement step after performing \equref{eqn:cspn}, 
\begin{align}
    \ve{H}_{i, j, t+1} = (1 - m_{i, j}) \ve{H}_{i, j, t+1}  +  m_{i, j} \ve{H}_{i, j}^s 
\label{eqn:cspn_sp}
\end{align}
where $m_{i, j} = \spa{I}(d_{i, j}^s > 0)$ is an indicator for the availability of sparse depth at $(i, j)$. 

In this way, we guarantee that our refined depths have the exact same value at those valid pixels in sparse depth map. Additionally, we propagate the information from those sparse depth to its surrounding pixels such that the smoothness between the sparse depths and their neighbors are maintained. 
Thirdly, thanks to the diffusion process, the final depth map is well aligned with image structures. 
This fully satisfies the desired three properties for this task which is discussed in our introduction (\ref{sec:intro}). 

% it performs a non-symmetric propagation where the information can only be diffused from the given sparse depth to others, while not the other way around.

% still follows PDE
In addition, this process is still following the diffusion process with PDE, where the transformation matrix can be built by simply replacing the rows satisfying $m_{i, j} = 1$ in $\ve{G}$ (\equref{eqn:vector}), which are corresponding to sparse depth samples, by $\ve{e}_{i + j*m}^T$. Here $\ve{e}_{i + j*m}$ is an unit vector with the value at $i + j*m$ as 1.
Therefore, the summation of each row is still $1$, and obviously the stabilization still holds in this case.

\begin{figure}[t]
\centering
\includegraphics[width=0.95\textwidth]{fig/fig2.pdf}
\caption{Comparison of depth map~\cite{Ma2017SparseToDense} with sparse depth replacement and with our CSPN \wrt smoothness of depth gradient at sparse depth points. (a) Input image. (b) Sparse depth points. (c) Depth map with sparse depth replacement. (d) Depth map with our CSPN with sparse depth points. We highlight the differences in the red box.}
\label{fig:gradient}
\end{figure}

Our strategy has several advantages over the previous state-of-the-art sparse-to-dense methods~\cite{Ma2017SparseToDense,LiaoHWKYL16}.
In \figref{fig:hist}(a), we plot a histogram of depth displacement from ground truth at given sparse depth pixels from the output of Ma \etal~\cite{Ma2017SparseToDense}. It shows the accuracy of sparse depth points cannot preserved, and some pixels could have very large displacement (0.2m), indicating that directly training a CNN for depth prediction does not preserve the value of real sparse depths provided. To acquire such property, 
one may simply replace the depths from the outputs with provided sparse depths at those pixels, however, it yields non-smooth depth gradient \wrt surrounding pixels. 
In~\figref{fig:gradient}(c), we plot such an example, at right of the figure, we compute Sobel gradient~\cite{sobel2014history} of the depth map along x direction, where we can clearly see that the gradients surrounding pixels with replaced depth values are non-smooth.
We statistically verify this in \figref{fig:hist}(b) using 500 sparse samples, the blue bars are the histogram of gradient error  at sparse pixels by comparing the gradient of the depth map with sparse depth replacement and of ground truth depth map. We can see the difference is significant, 2/3 of the sparse pixels has large gradient error.
Our method, on the other hand, as shown with the green bars in \figref{fig:hist}(b), the average gradient error is much smaller, and most pixels have zero error. In\figref{fig:gradient}(d), we show the depth gradients surrounding sparse pixels are smooth and close to ground truth, demonstrating the effectiveness of our propagation scheme. 

% Finally, in our experiments~\ref{sec:exp}, we validate the number of iterations $N$ and kernel size $k$ used for doing the CSPN.


\subsection{Complexity Analysis}
\label{subsec:time}

As formulated in~\equref{eqn:cspn}, our CSPN takes the operation of convolution, where the complexity of using CUDA with GPU for one step CSPN is $O(\log_2(k^2))$, where $k$ is the kernel size. This is because CUDA uses parallel sum reduction, which has logarithmic complexity. In addition,  convolution operation can be performed parallel for all pixels and channels, which has a constant complexity of $O(1)$. Therefore, performing $N$-step propagation, the overall complexity for CSPN is $O(\log_2(k^2)N)$, which is irrelevant to image size $(m, n)$.

SPN~\cite{liu2017learning} adopts scanning row/column-wise propagation in four directions. Using $k$-way connection and running in parallel, the complexity for one step is $O(\log_2(k))$. The propagation needs to scan full image from one side to another, thus the complexity for SPN is $O(\log_2(k)(m + n))$. Though this is already more efficient than the densely connected CRF proposed by~\cite{philipp2012dense}, whose implementation complexity with permutohedral lattice is $O(mnN)$, ours $O(\log_2(k^2)N)$ is more efficient since the number of iterations $N$ is always much smaller than the size of image $m, n$. We show in our experiments (\secref{sec:exp}), with $k=3$ and $N=12$, CSPN already outperforms SPN with a large margin (relative $30\%$), demonstrating both efficiency and effectiveness of the proposed approach.


\subsection{End-to-End Architecture}
\label{subsec:unet}
\begin{figure}[t]
\centering
\includegraphics[width=0.95\textwidth,height=0.45\textwidth]{fig/framework2.pdf}
\caption{Architecture of our networks with mirror connections for  depth estimation via transformation kernel prediction with CSPN (best view in color). Sparse depth is an optional input, which can be embedded into the CSPN to guide the depth refinement.}
\label{fig:arch}
\end{figure}

We now explain our end-to-end network architecture to predict both the transformation kernel and the depth value, which are the inputs to CSPN for depth refinement.
 As shown in \figref{fig:arch}, our network has some similarity with that from Ma \etal~\cite{Ma2017SparseToDense}, with the final CSPN layer that outputs a dense depth map.  
 
For predicting the transformation kernel $\kappa$ in \equref{eqn:cspn}, 
rather than building a new deep network for learning affinity same as Liu \etal~\cite{liu2017learning}, we branch an additional output from the given network, which shares the same feature extractor with the depth network. This helps us to save memory and time cost for joint learning of both depth estimation and transformation kernels prediction. 

Learning of affinity is dependent on fine grained spatial details of the input image. However, spatial information is weaken or lost with the down sampling operation during the forward process of the ResNet in~\cite{laina2016deeper}. Thus, we add mirror connections similar with the U-shape network~\cite{ronneberger2015u} by directed concatenating the feature from encoder to up-projection layers as illustrated by ``UpProj$\_$Cat'' layer in~\figref{fig:arch}. Notice that it is important to carefully select the end-point of mirror connections. Through experimenting three possible positions to append the connection, \ie after \textit{conv}, after \textit{bn} and after \textit{relu} as shown by the ``UpProj'' layer in~\figref{fig:arch} , we found the last position provides the best results by validating with the NYU v2 dataset (\secref{subsec:ablation}). 
In doing so, we found not only the depth output from the network is better recovered, and the results after the CSPN is additionally refined, which we will show the experiment section~(\secref{sec:exp}).
Finally we adopt the same training loss as~\cite{Ma2017SparseToDense}, yielding an end-to-end learning system.

		  %THis file has high level approach and comparison with wavefront tiling
\section{Code Generation of PCOT}
\label{sec:pcot}

%% Comment: I assume that by this point we have already explained the
%% transformations by tiling hyperplanes and the iteration space can be tiled
%% along canonic directions.

In this section, we describe our generalization of the cache oblivious code
generation to polyhedal programs: Polyhedral Cache Oblivious Tiling (PCOT).

\subsection{Approach Overview}
The input is any polyhedral loop nest that is fully permutable and hence
tiling it with hyper-rectangular tiles is a legal transformation.  We first
(in Section~\ref{sec:perfect}) describe the case for tiling all dimensions of a
perfect loop nest.  Other cases can be handled with additional pre-processing,
which we describe in Section~\ref{sec:codegen_ext}.

Figure~\ref{fig:sample_code} illustrates the structure of our generated code.
The input loop nest is replaced by a call to start the recursion as shown in 
Figure~\ref{fig:sample_code}a. The computation of the bounding box from loop nests 
are discussed in Section~\ref{sec:computingBB}.

\begin{figure*}
  \centering
    \includegraphics[width=0.98\textwidth]{codegen-example}
    \caption{The structure of generated code for Heat2D stencils, which has
      loop depth $d=3$. (a) The tileable loop nest is replaced by a call to
      start the recursion. The input is the bounding box of the loop nest. (b)
      Structure of the recursive function. The input bounding box is split
      into $2^d$ new orthants by dividing each dimention in half, and the same
      function is recursively called for each new orthant. When the orthant
      reaches the input tile size, the recursion is terminated by a call to
      the base function. (c) Code for the base function that
      performs the computation of a tile in lexicographic order.  }
  \label{fig:sample_code}
\end{figure*}



\subsection{Codegen for Perfect Loop Nests}
\label{sec:perfect}

Given a perfectly nested loop with all $d$ dimensions tilable, we seek to
generate,
\begin{inlinelist}
	\item a call to recursive function to start the recursion,
	\item a recursive function, and
	\item a base function.
\end{inlinelist}  
The recursive function takes origin and the size of the orthant as inputs,
which are $d$ dimensional vectors. For the initial call to the recursive
function, the origin and the orthant size correspond to the bounding box of
the input loop nest.  Bounding box of a domain is the smallest
hyper-rectangular shaped domain which encloses the given domain.

The recursive function visits the iteration domain in divide-and-conquer
order.  It recursively divides iteration domain into orthants until they are
smaller than an input parameter.
%The body of the recursive function has 2 main parts (
%Figure~\ref{fig:sample_code}.c), \begin{inlinelist}
%\item a call to the base function if the orthant is small enough, and
%\item divide the orthant into $2^d$ number of new orthants by dividing each
%dimension by 2 and generate recursive function call to visit each new orthant.
%\end{inlinelist}
The call to the base function is wrapped by a condition to check whether the
size of the current orthant is less than or equal to the base case threshold. 
%explaining the execution order without using wavefronts

The orthants are visited sequentially in the lexicographic order.  For
parallel execution the tasks are executed with wave-front parallelism.  We use
the OpenMP tasks for parallel execution, where each recursive function call is
annotated with {\tt omp task} pragma, and the wave-front time boundaries are
synchronized with {\tt omp taskwait}.

% To describe the execution order of orthants, let us represent new orthants
% with a hypercube of $d$ dimensions. Each subcube can be labeled using $d$
% bits.  When $d=3$, subcubes are (0,0,0),(0,0,1),...,(1,1,1). Sum of three
% bits represent the ``distance'' from (0,0,0) subcube. If the difference of
% ``distance'' of two subcubes is one, then those two subcubes are adjacent to
% and has a dependence between them.  For the sequential execution, subcubes
% can be executed in an increasing order of ``distance'' which satisfy the
% order imposed due to the dependences among subcubes. In the generated code,
% this is a sequence of recursive function calls in the increasing order of
% ``distance'' of subcubes. The recursive function call takes the origin and
% size of subcube as inputs (Figure~\ref{fig:sample_code}.c).
%
% For the parallel execution, subcubes are executed in the increasing order of
% ``distance'' where the subcubes with the same ``distance'' can be executed
% in parallel. i.e., we start with (0,0,0) after that (0,0,1), (0,1,0), and
% (1,0,0) can be executed in parallel. We spawn parallel tasks using
% \texttt{OpenMP task} directives. Then use \texttt{taskwait} directive to
% enforce the increasing order of ``distance''.

% We generate $2^d$ recursive calls with origin and size of the orthant as
% parameters to each call.  The canonic directions are legal tiling
% hyperplanes (since hyper-rectangular tiling is legal), therefore the 45
% degree wavefronts is a legal schedule to execute new orthants. All the
% orthants within the wavefront can be run in parallel. To generate parallel
% code, we group together the recursive calls within the same wavefront. Then
% insert an OpenMP \texttt{\#pragma omp task} before each recursive call and a
% \texttt{\#pragma omp taskwait} after each group to act as a thread barrier
% before moving on to the next group of tasks.
%
% This makes sure that after each group, worker threads wait till all the
% tasks in the group are finished before moving on to the next group of tasks.

The base function visits all points in the intersection of leaf orthant and
the input loop nest iteratively in lexicographical order.  The loop nest in
the base function is identical to the loop nest for a tile in SLT with
parametric tile sizes we use.  One can use any one of the available
parametrically tiled code generators~\cite{sanjay-lcpc2009,
  sanjay-kim-dtilingTR-2010,baskaran-etal-cgo10,iooss:hal2015} to generate
parametrically tiled loops and then extract the point loops.
%% removed since double-blind
% We use the point loops generated by
% DTiling~\cite{sanjay-lcpc2009,sanjay-kim-dtilingTR-2010} code generator in
% AlphaZ~\cite{yuki2013alphaz} system\footnote{Since we use the same code to
% iterate a tile in single-level tiling code and to iterate base case in cache
% oblivious code, the underlying compiler is capable of performing same set of
% optimizations for both codes.}.
%% Following is removed since this is not a feature but kind of a bug
% In divide-and-conquer style codes, size of the base case at run time can be
% less than or equal to the tuned size of the base case. When the problem size
% (at runtime) is a power of 2 of tuned base case size, then the actual base
% case size is equal to the tuned base case size, otherwise, base case size is
% smaller than the tuned base case size.  The actual base case size depends on
% the problem size parameter, therefore, it is important to use a
% parametrically tiled code generator to extract the point loops from.


\subsection{Optimizations}
\label{sec:codegen:opt}
We implement a number of optimizations to improve the speed of code.  In this
section we explain two important optimizations to exit the recursion early.

\subsubsection{Early Exit for Zero Orthants}

The zero orthants surface when we have tile sizes that are cubic bounding box with
hyper-rectangular tiles and vise versa. In this case, orthant size along all
the dimensions reaches the leaf size in different levels of the recursion. But
still our code generator generates $2^d$ sequence of recursive calls where
size some of the new orthants may be zero along dimensions where the input
orthant size is already smaller than the leaf threshold.  This is a simple
optimization where we check whether a width along anyone of the dimensions of
the orthant is zero. If it is zero then exit the recursion.
%In the recursive function, we only divide a
%dimension by two, if the size of the orthant along this dimension is larger
%than that of the tuned base case size. Therefore, it is possible that some
%dimensions are not divided in the middle, leading to new orthants with zero
%width along some dimensions. Lets assume that the origin of the orthant is
%$(0,0)$, the size of the orthant is $(32\times32)$ and the base case threshold
%is $(16\times32)$. In this case, we only divides the first dimension by 2
%(because 32 $>$ 16) and we do not divide the second dimension. This introduce
%only 2 new orthants where origin is $(0,0)$, size $(16\times32)$ and origin is
%$(16,0)$, size $(16\times32)$. But in the recursive function there are $2^2$
%recursive calls (or new orthants). The two of them corresponds to the two new
%orthants mentioned above and other two recursive calls have size of the 2nd
%dimension as 0 since we did not divide it by two because the size of it is
%already matches the base threshold. Hence, there are two recursive calls with
%the size of the orthant is 0. This optimization filter these orthants out.

\subsubsection{Early Exit for Empty Orthants}
Second optimization is due to the fact we visit the bounding box of the input
loop nest instead of the actual domain. Some kernels operate on
non-hyper-rectangular domains(i.e, Cholesky  operates on triangular domains)
and some kernels need iteration space skewing to enable hyper-rectangular
tiling (i.e., Heat2D). If the actual domain is not a hyper-rectangle then the
bounding box will have points where no computations are defined.  This may
lead to orthants outside of the original iteration space, analogous to empty
tiles in iteration space tiling. 

For example, a loop nest whose iteration space is triangular, the boundingbox
is a rectangle where the half of the points has no computation defined. When
we visit this rectangular box in divide-and-conquer order, we will end up
visiting empty orthants and we want to identify these orthants.  We generate
conditions to check whether all vertices of the orthant remain outside of the
original iteration space using \emph{isl} library~\cite{verdoolaege2010isl}.
%If that is the case, we exit the recursion early 

%We generate code to check whether the current orthant is completely outside
%the actual domain of the loop nest.  We check whether all the vertices of the
%orthant unsatisfy at least one of the constraints of the actual domain of the
%input loop nest.  If that is the case we exit the recursion early.  The
%conditions are generated and simplified using the \emph{isl}
%library~\cite{verdoolaege2010isl}.

The \texttt{checkEmpty} method at the top of the
Figure~\ref{fig:sample_code}c implements both optimizations.  Without them,
the code produces correct answers but visits many base cases which are empty.
The recursion ends either when \texttt{checkEmpty} method returns true or when
the orthant reaches its input tile size.

%\subsubsection{Optimization 3}
%If an orthant is smaller than or equal to the base size parameter, it will
% exit the recursion and compute all the points in the orthant. In this case,
% base size parameter act as an threshold.  The actual orthant size can be
% less than or equal to the provided threshold.  The actual base size depends
% on the problem size. For example if the problem size is 1024 then the
% orthant sizes are 1024, 512, 256, 128, 64, 32,... For any base size
% parameter from 32 to 63, the actual base size going to be 32. If the problem
% size is 1100, then the orthant sizes are 1100, 550, 275, 137, 68, 34,... For
% any base size parameter from 34 to 67, the actual base size going to be
% 34. Therefore, the base size we want to use may not be the actual base size
% and it heavily depend on the problem size and out of our control.
%
% We want the generated code to have the exact base case size we provide as a
% parameter(s). The solution is simple, we pad the bounding box of the
% iteration domain so that size of each dimension is multiple of its base size
% parameter and some power of two. If the size of the $i_{th}$ dimension is
% $N_i$, base case size parameter is $b_i$ then the new padded size is the
% minimum value of $b_i\times2^{k} \ge N_i$ where $k$ is an integer and
% $k \ge 0$. When you use the padded bounding box as the starting orthant,
% after $k$ levels of recursion, we reach orthants with size corresponds to
% the base case parameter sizes and then exit the recursion to compute the
% points within the orthant.  The extra points without any computations due to
% the padding is optimized by the Optimization 2.

% \subsubsection{Other optimizations}
% In addition to the main optimizations discussed above, we added
% \emph{restrict} keyword to the declarations of pointer variables. This helps
% the compiler to optimize code knowing that pointer variables are not
% referred using other aliases.
%
% We only transform the loop nests which are dominant in computations.  The
% input loop nests may contain loop nests of different dimensions (depths).
% The number of computations in the loop nests with smaller number of
% dimensions are significantly smaller than the main-loop which is the loop
% nest(s) with higher number of dimensions. We optimize only the main-loop(s)
% which is the hotspot of the input loop nest.
%
% The generated code is parametric in problem size and base case size. Base
% case size is a vector that specify the size along all the tilable
% dimensions. The parameters of recursive and base functions are passed using
% C structs to reduce the number of function parameters in higher dimensional
% programs.

\subsection{Handling Imperfect Loop Nests}
\label{sec:codegen_ext}
The discussion so far assumed perfectly nested loops as inputs.  We now extend
this to imperfect loop nests, and loop nests where subset of the $d$
dimensions are marked as tilable.

%\subsection{Handling imperfectly nested loops}
The input imperfect loop nests are converted to perfect loop nests with a
pre-processing.  This is called the embedding transformation that are used to
handle imperfectly nested loops in parametric tiled code
generation~\cite{sanjay-kim-dtilingTR-2010}. 
It involves, bringing all the statements into the same loop depth by adding
loops with one iteration as necessary. Then affine guards are added to
eliminate sequence of inner loops, which lead to ``perfect loop nest with
affine guards''.

% The conversion process includes
%the following high level steps.  First, we bring all statements in the
%imperfect loop nest into the same loop depth.  This may result in a sequence
%of inner loops.  Then, we use affine guards to eliminate the sequence of
%loops, and we call it ``perfect loop nest with affine guards''.
%%We specify options to Cloog so that the generated loop nest is already perfectly nested.

When a subset of the loops are marked as tilable, we first extract the marked
band of loops parameterized by both program input parameters as well
as untiled outer loop iterators. We apply the techniques we have discussed so
far to generate code for the extracted loop nest. In this case, the function
call to start the recursion is added as the body of outer untiled loop nest.
The inner untiled loop nests are added as the body of the point loop in the
base function.
%For the case where a subset of the loops are marked as tilable, let us assume
%that a band of loops from depth $i$ to $k$ are tilable, and loops from depth 0
%to $i-1$ and $k+1$ to $d$ are not tilable.  The first step is to extract the
%tilable band of loops from depth $i$ to $k$.  Now, we have a loop nest where
%all the dimensions are tilable therefore, we can apply the above techniques to
%generate a recursive function, base function and a function call to start the
%recursion.  The function call to start the recursion is added as the body of
%outer untiled loops nest at depth $i-1$. The inner untiled loops from depth
%$k+1$ to $d$ are set as the body of the innermost loop in the base function.

% Let's consider that we want to tile only the space dimensions (inner 2
% dimensions) of Heat-2D stencil. We extract the inner two loops and generate
% new recursive function and a base function as described in the beginning of
% this section. Now, there is a recursive function call for each outer time
% step. Therefore, we place the function call to start the recursion as the
% body of the untiled outer loop.

% Let's go through an example.  Let's consider adding another outer loop to
% the program in Listing~\ref{lst:motiv_fully}. The resulting loop nest is
% shown in Lising~\ref{lst:motiv_band}. Lets also assume that only the inner 2
% loops are tilable. In this case we extract the tilable loop nests (from
% lines \ref{lst:motiv_band:1s}-\ref{lst:motiv_band:1e}) and provide it as the
% input to the PCOT code generator. This ends up generating a recursive
% function similar to Listing~\ref{lst:motiv_recur}. The base function is
% similar to Listing~\ref{lst:motiv_base}. The recursive function call to
% start recursion will be the body of outer most untiled-loop as shown in
% Listing~\ref{lst:motiv_band_start}. The bounding box (orthant size) is a
% function of the outer loop iterator $c0$.
%

\subsection{Computing the Bounding Box}
\label{sec:computingBB}
The bounding box is a hyper-rectangle containing the iteration space of the
loop nest. It is computed by eliminating the outer loop indices from the loop
bound expressions.  There can be infinitely many bounding boxes for a given
loop nest, but we start with the tightest (smallest) bounding box among all
the possibilities.

The bounding box also plays an important role in deciding the leaf tile size.
We want the leaf tile size to have the exact value as the input tile size
parameter to the generated code. Therefore, we pad the size of bounding box
along each dimension to the minimum value of $b_i\times2^{k_i} \ge N_i$ where
$b_i$ input leaf size parameter, $N_i$ the size along the $i^{th}$ dimension
of bounding box and $k_i \ge 0$ is an integer. Now, at $k_i$th level of
recursion the orthant size along the $i$th dimension will be $b_i$. The
padding of the bounding box introduces iteration points outside of original
iteration space.  These empty points get optimized away by the optimizations
described in Section~\ref{sec:codegen:opt}


%
%The loop bounding expressions of the inner loop indices may be functions of
%outer loop indices.  We eliminate outer loop indices from bounding expressions
%starting from the outer most loop.  In lower bound expressions, we replace
%outer index with its lower bound if the sign of the index is negative, upper
%bound if the sign is positive.  In upper bound expressions, we replace index
%with its upper and lower bound if the sign is positive and negative
%respectively.  At the end all the bounding expressions are functions of input
%parameters and iterators of the loops surrounding bounding box.
%
% Let's consider the scenario where only a subset of the loops are tilable,
% then the bounding box is a function of both input parameters and indices of
% outer untiled loops. In other words, for each instance of outer untiled
% iteration, there is a instance of the bounding box. The approach is same as
% above except, we do not eliminate outer untiled loop indices in the bounding
% expressions.
%
% For a given loop nest, tilable band specifies band of loops which are
% tilable or permutable. Usually, the band is specified by the start loop
% depth and end loop depth.  Bounding box is a cuboid (or hyper-rectangle)
% containing the iteration space of the tilable band.  There can be infinitely
% many bounding boxes for a given tilable band, but we chose the tightest
% bounding box among all the possibilities. Since there are outer loops
% surrounding the tilable band of loops, the bounding box is parameterised by
% outer loop iterators. In other words, there is a bounding box for each
% instance of outer loops.
%
%When the input is a sequence of perfectly nested loops, the corresponding
%domain can be an union of polyhedra.  In this case, we compute the bounding
%box of each polyhedra individually as described above.  Then, we compute the
%upper bound along a given dimension by taking the maximum of upper bounds of
%all the polyhedra along the same dimension.  Lower bound is computed similarly
%by taking the minimum of lower bounds of all the polyhedra along a given
%dimension.  The resulting upper bounds and lower bounds define the bounding
%box of the union of polyhedra.
%
%
% Simply using the bounding box as the input orthant size to start the
% recursion may leads to issues with the actual base case size. For example,
% lets assume a cubic bounding box with size 512 along each dimension. Then
% the subsequent orthant sizes are 512, 256, 128, 64, 32, ... For any base
% size parameter from 32 to 63, the actual base size going to be 32. If the
% problem size is 550, the orthant sizes are 550, 275, 137, 68, 34,... For any
% base size parameter from 34 to 67, the actual base size going to be
% 34. Therefore, the base size we want to use may not be the actual base size
% and it heavily depends on the size of the bounding box and out of our
% control.  We can get control of it, if
%
% In this work, we use parametric tile sizes for cache oblivious tiling code.
% Existing techniques for parametric tiling~\cite{sanjay-lcpc2009,
% baskaran-etal-cgo10} have demonstrated that the parameterization does not
% incur significant performance overhead when compared to tiling by
% compile-time constants. We implemented our cache oblivious tiling code
% generator using Cloog and ISL libraries.
%

% We implemented our cache oblivius tiling code generator in the AlphaZ
% system~\cite{alphaz} which is capable of generating parametrically tiled
% code using D-Tiling~\cite{sanjay-lcpc2009}.

% Local Variables: ***
% TeX-master: "TACO2017.tex" ***
% fill-column: 78 ***
% End: ***
       % This file has the codegenerator details
%Online convex optimization with memory has emerged as an important and challenging area with a wide array of applications, see \citep{lin2012online,anava2015online,chen2018smoothed,goel2019beyond,agarwal2019online,bubeck2019competitively} and the references therein.  Many results in this area have focused on the case of online optimization with switching costs (movement costs), a form of one-step memory, e.g., \citep{chen2018smoothed,goel2019beyond,bubeck2019competitively}, though some papers have focused on more general forms of memory, e.g., \citep{anava2015online,agarwal2019online}. In this paper we, for the first time, study the impact of feedback delay and nonlinear switching cost in online optimization with switching costs. 

An instance consists of a convex action set $\mathcal{K}\subset\mathbb{R}^d$, an initial point $y_0\in\mathcal{K}$, a sequence of non-negative convex cost functions $f_1,\cdots,f_T:\mathbb{R}^d\to\mathbb{R}_{\ge0}$, and a switching cost $c:\mathbb{R}^{d\times(p+1)}\to\mathbb{R}_{\ge0}$. To incorporate feedback delay, we consider a situation where the online learner only knows the geometry of the hitting cost function at each round, i.e., $f_t$, but that the minimizer of $f_t$ is revealed only after a delay of $k$ steps, i.e., at time $t+k$.  This captures practical scenarios where the form of the loss function or tracking function is known by the online learner, but the target moves over time and measurement lag means that the position of the target is not known until some time after an action must be taken. 
To incorporate nonlinear (and potentially nonconvex) switching costs, we consider the addition of a known nonlinear function $\delta$ from $\mathbb{R}^{d\times p}$ to $\mathbb{R}^d$ to the structured memory model introduced previously.  Specifically, we have
\begin{align}
c(y_{t:t-p}) = \frac{1}{2}\|y_t-\delta(y_{t-1:t-p})\|^2,    \label{e.newswitching}
\end{align}
where we use $y_{i:j}$ to denote either $\{y_i, y_{i+1}, \cdots, y_j\}$ if $i\leq j$, or  $\{y_i, y_{i-1}, \cdots, y_j\}$ if $i > j$ throughout the paper. Additionally, we use $\|\cdot\|$ to denote the 2-norm of a vector or the spectral norm of a matrix.

In summary, we consider an online agent that interacts with the environment as follows:
% \begin{inparaenum}[(i)] 
\begin{enumerate}%[leftmargin=*]
    \item The adversary reveals a function $h_t$, which is the geometry of the $t^\mathrm{th}$ hitting cost, and a point $v_{t-k}$, which is the minimizer of the $(t-k)^\mathrm{th}$ hitting cost. Assume that $h_t$ is $m$-strongly convex and $l$-strongly smooth, and that $\arg\min_y h_t(y)=0$.
    \item The online learner picks $y_t$ as its decision point at time step $t$ after observing $h_t,$  $v_{t-k}$.
    \item The adversary picks the minimizer of the hitting cost at time step $t$: $v_t$. 
    \item The learner pays hitting cost $f_t(y_t)=h_t(y_t-v_t)$ and switching cost $c(y_{t:t-p})$ of the form \eqref{e.newswitching}.
\end{enumerate}

The goal of the online learner is to minimize the total cost incurred over $T$ time steps, $cost(ALG)=\sum_{t=1}^Tf_t(y_t)+c(y_{t:t-p})$, with the goal of (nearly) matching the performance of the offline optimal algorithm with the optimal cost $cost(OPT)$. The performance metric used to evaluate an algorithm is typically the \textit{competitive ratio} because the goal is to learn in an environment that is changing dynamically and is potentially adversarial. Formally, the competitive ratio (CR) of the online algorithm is defined as the worst-case ratio between the total cost incurred by the online learner and the offline optimal cost: $CR(ALG)=\sup_{f_{1:T}}\frac{cost(ALG)}{cost(OPT)}$.

It is important to emphasize that the online learner decides $y_t$ based on the knowledge of the previous decisions $y_1\cdots y_{t-1}$, the geometry of cost functions $h_1\cdots h_t$, and the delayed feedback on the minimizer $v_1\cdots v_{t-k}$. Thus, the learner has perfect knowledge of cost functions $f_1\cdots f_{t-k}$, but incomplete knowledge of $f_{t-k+1}\cdots f_t$ (recall that $f_t(y)=h_t(y-v_t)$).

Both feedback delay and nonlinear switching cost add considerable difficulty for the online learner compared to versions of online optimization studied previously. Delay hides crucial information from the online learner and so makes adaptation to changes in the environment more challenging. As the learner makes decisions it is unaware of the true cost it is experiencing, and thus it is difficult to track the optimal solution. This is magnified by the fact that nonlinear switching costs increase the dependency of the variables on each other. It further stresses the influence of the delay, because an inaccurate estimation on the unknown data, potentially magnifying the mistakes of the learner. 

The impact of feedback delay has been studied previously in online learning settings without switching costs, with a focus on regret, e.g., \citep{joulani2013online,shamir2017online}.  However, in settings with switching costs the impact of delay is magnified since delay may lead to not only more hitting cost in individual rounds, but significantly larger switching costs since the arrival of delayed information may trigger a very large chance in action.  To the best of our knowledge, we give the first competitive ratio for delayed feedback in online optimization with switching costs. 

We illustrate a concrete example application of our setting in the following.

\begin{example}[Drone tracking problem]
\label{example:drone} \emph{
Consider a drone with vertical speed $y_t\in\mathbb{R}$. The goal of the drone is to track a sequence of desired speeds $y^d_1,\cdots,y^d_T$ with the following tracking cost:}
\begin{equation}
    \sum_{t=1}^T \frac{1}{2}(y_t-y^d_t)^2 + \frac{1}{2}(y_t-y_{t-1}+g(y_{t-1}))^2,
\end{equation}
\emph{where $g(y_{t-1})$ accounts for the gravity and the aerodynamic drag. One example is $g(y)=C_1+C_2\cdot|y|\cdot y$ where $C_1,C_2>0$ are two constants~\cite{shi2019neural}. Note that the desired speed $y_t^d$ is typically sent from a remote computer/server. Due to the communication delay, at time step $t$ the drone only knows $y_1^d,\cdots,y_{t-k}^d$.}

\emph{This example is beyond the scope of existing results in online optimization, e.g.,~\cite{shi2020online,goel2019beyond,goel2019online}, because of (i) the $k$-step delay in the hitting cost $\frac{1}{2}(y_t-y_t^d)$ and (ii) the nonlinearity in the switching cost $\frac{1}{2}(y_t-y_{t-1}+g(y_{t-1}))^2$ with respective to $y_{t-1}$. However, in this paper, because we directly incorporate the effect of delay and nonlinearity in the algorithm design, our algorithms immediately provide constant-competitive policies for this setting.}
\end{example}


\subsection{Related Work}
This paper contributes to the growing literature on online convex optimization with memory.  
Initial results in this area focused on developing constant-competitive algorithms for the special case of 1-step memory, a.k.a., the Smoothed Online Convex Optimization (SOCO) problem, e.g., \citep{chen2018smoothed,goel2019beyond}. In that setting, \citep{chen2018smoothed} was the first to develop a constant, dimension-free competitive algorithm for high-dimensional problems.  The proposed algorithm, Online Balanced Descent (OBD), achieves a competitive ratio of $3+O(1/\beta)$ when cost functions are $\beta$-locally polyhedral.  This result was improved by \citep{goel2019beyond}, which proposed two new algorithms, Greedy OBD and Regularized OBD (ROBD), that both achieve $1+O(m^{-1/2})$ competitive ratios for $m$-strongly convex cost functions.  Recently, \citep{shi2020online} gave the first competitive analysis that holds beyond one step of memory.  It holds for a form of structured memory where the switching cost is linear:
$
    c(y_{t:t-p})=\frac{1}{2}\|y_t-\sum_{i=1}^pC_iy_{t-i}\|^2,
$
with known $C_i\in\mathbb{R}^{d\times d}$, $i=1,\cdots,p$. If the memory length $p = 1$ and $C_1$ is an identity matrix, this is equivalent to SOCO. In this setting, \citep{shi2020online} shows that ROBD has a competitive ratio of 
\begin{align}
    \frac{1}{2}\left( 1 + \frac{\alpha^2 - 1}{m} + \sqrt{\Big( 1 + \frac{\alpha^2 - 1}{m}\Big)^2 + \frac{4}{m}} \right),
\end{align}
when hitting costs are $m$-strongly convex and $\alpha=\sum_{i=1}^p\|C_i\|$. 


Prior to this paper, competitive algorithms for online optimization have nearly always assumed that the online learner acts \emph{after} observing the cost function in the current round, i.e., have zero delay.  The only exception is \citep{shi2020online}, which considered the case where the learner must act before observing the cost function, i.e., a one-step delay.  Even that small addition of delay requires a significant modification to the algorithm (from ROBD to Optimistic ROBD) and analysis compared to previous work. 

As the above highlights, there is no previous work that addresses either the setting of nonlinear switching costs nor the setting of multi-step delay. However, the prior work highlights that ROBD is a promising algorithmic framework and our work in this paper extends the ROBD framework in order to address the challenges of delay and non-linear switching costs. Given its importance to our work, we describe the workings of ROBD in detail in Algorithm~\ref{robd}. 

\begin{algorithm}[t!]
  \caption{ROBD \citep{goel2019beyond}}
  \label{robd}
\begin{algorithmic}[1]
  \STATE {\bfseries Parameter:} $\lambda_1\ge0,\lambda_2\ge0$
  \FOR{$t=1$ {\bfseries to} $T$}
  \STATE {\bfseries Input:} Hitting cost function $f_t$, previous decision points $y_{t-p:t-1}$
  \STATE $v_t\leftarrow\arg\min_yf_t(y)$
  \STATE $y_t\leftarrow\arg\min_yf_t(y)+\lambda_1c(y,y_{t-1:t-p})+\frac{\lambda_2}{2}\|y-v_t\|^2_2$
  \STATE {\bfseries Output:} $y_t$
  \ENDFOR
   
\end{algorithmic}
\end{algorithm}

Another line of literature that this paper contributes to is the growing understanding of the connection between online optimization and adaptive control. The reduction from adaptive control to online optimization with memory was first studied in \citep{agarwal2019online} to obtain a sublinear static regret guarantee against the best linear state-feedback controller, where the approach is to consider a disturbance-action policy class with some fixed horizon.  Many follow-up works adopt similar reduction techniques \citep{agarwal2019logarithmic, brukhim2020online, gradu2020adaptive}. A different reduction approach using control canonical form is proposed by \citep{li2019online} and further exploited by \citep{shi2020online}. Our work falls into this category.  The most general results so far focus on Input-Disturbed Squared Regulators, which can be reduced to online convex optimization with structured memory (without delay or nonlinear switching costs).  As we show in \Cref{Control}, the addition of delay and nonlinear switching costs leads to a significant extension of the generality of control models that can be reduced to online optimization.          % This file has the analytical model of cache misses
In this section we conduct comprehensive experiments to emphasise the effectiveness of DIAL, including evaluations under white-box and black-box settings, robustness to unforeseen adversaries, robustness to unforeseen corruptions, transfer learning, and ablation studies. Finally, we present a new measurement to test the balance between robustness and natural accuracy, which we named $F_1$-robust score. 


\subsection{A case study on SVHN and CIFAR-100}
In the first part of our analysis, we conduct a case study experiment on two benchmark datasets: SVHN \citep{netzer2011reading} and CIFAR-100 \cite{krizhevsky2009learning}. We follow common experiment settings as in \cite{rice2020overfitting, wu2020adversarial}. We used the PreAct ResNet-18 \citep{he2016identity} architecture on which we integrate a domain classification layer. The adversarial training is done using 10-step PGD adversary with perturbation size of 0.031 and a step size of 0.003 for SVHN and 0.007 for CIFAR-100. The batch size is 128, weight decay is $7e^{-4}$ and the model is trained for 100 epochs. For SVHN, the initial learinnig rate is set to 0.01 and decays by a factor of 10 after 55, 75 and 90 iteration. For CIFAR-100, the initial learning rate is set to 0.1 and decays by a factor of 10 after 75 and 90 iterations. 
%We compared DIAL to \cite{madry2017towards} and TRADES \citep{zhang2019theoretically}. 
%The evaluation is done using Auto-Attack~\citep{croce2020reliable}, which is an ensemble of three white-box and one black-box parameter-free attacks, and various $\ell_{\infty}$ adversaries: PGD$^{20}$, PGD$^{100}$, PGD$^{1000}$ and CW$_{\infty}$ with step size of 0.003. 
Results are averaged over 3 restarts while omitting one standard deviation (which is smaller than 0.2\% in all experiments). As can be seen by the results in Tables~\ref{black-and_white-svhn} and \ref{black-and_white-cifar100}, DIAL presents consistent improvement in robustness (e.g., 5.75\% improved robustness on SVHN against AA) compared to the standard AT 
%under variety of attacks 
while also improving the natural accuracy. More results are presented in Appendix \ref{cifar100-svhn-appendix}.


\begin{table}[!ht]
  \caption{Robustness against white-box, black-box attacks and Auto-Attack (AA) on SVHN. Black-box attacks are generated using naturally trained surrogate model. Natural represents the naturally trained (non-adversarial) model.
  %and applied to the best performing robust models.
  }
  \vskip 0.1in
  \label{black-and_white-svhn}
  \centering
  \small
  \begin{tabular}{l@{\hspace{1\tabcolsep}}c@{\hspace{1\tabcolsep}}c@{\hspace{1\tabcolsep}}c@{\hspace{1\tabcolsep}}c@{\hspace{1\tabcolsep}}c@{\hspace{1\tabcolsep}}c@{\hspace{1\tabcolsep}}c@{\hspace{1\tabcolsep}}c@{\hspace{1\tabcolsep}}c@{\hspace{1\tabcolsep}}c}
    \toprule
    & & \multicolumn{4}{c}{White-box} & \multicolumn{4}{c}{Black-Box}  \\
    \cmidrule(r){3-6} 
    \cmidrule(r){7-10}
    Defense Model & Natural & PGD$^{20}$ & PGD$^{100}$  & PGD$^{1000}$  & CW$^{\infty}$ & PGD$^{20}$ & PGD$^{100}$ & PGD$^{1000}$  & CW$^{\infty}$ & AA \\
    \midrule
    NATURAL & 96.85 & 0 & 0 & 0 & 0 & 0 & 0 & 0 & 0 & 0 \\
    \midrule
    AT & 89.90 & 53.23 & 49.45 & 49.23 & 48.25 & 86.44 & 86.28 & 86.18 & 86.42 & 45.25 \\
    % TRADES & 90.35 & 57.10 & 54.13 & 54.08 & 52.19 & 86.89 & 86.73 & 86.57 & 86.70 &  49.50 \\
    $\DIAL_{\kl}$ (Ours) & 90.66 & \textbf{58.91} & \textbf{55.30} & \textbf{55.11} & \textbf{53.67} & 87.62 & 87.52 & 87.41 & 87.63 & \textbf{51.00} \\
    $\DIAL_{\ce}$ (Ours) & \textbf{92.88} & 55.26  & 50.82 & 50.54 & 49.66 & \textbf{89.12} & \textbf{89.01} & \textbf{88.74} & \textbf{89.10} &  46.52  \\
    \bottomrule
  \end{tabular}
\end{table}


\begin{table}[!ht]
  \caption{Robustness against white-box, black-box attacks and Auto-Attack (AA) on CIFAR100. Black-box attacks are generated using naturally trained surrogate model. Natural represents the naturally trained (non-adversarial) model.
  %and applied to the best performing robust models.
  }
  \vskip 0.1in
  \label{black-and_white-cifar100}
  \centering
  \small
  \begin{tabular}{l@{\hspace{1\tabcolsep}}c@{\hspace{1\tabcolsep}}c@{\hspace{1\tabcolsep}}c@{\hspace{1\tabcolsep}}c@{\hspace{1\tabcolsep}}c@{\hspace{1\tabcolsep}}c@{\hspace{1\tabcolsep}}c@{\hspace{1\tabcolsep}}c@{\hspace{1\tabcolsep}}c@{\hspace{1\tabcolsep}}c}
    \toprule
    & & \multicolumn{4}{c}{White-box} & \multicolumn{4}{c}{Black-Box}  \\
    \cmidrule(r){3-6} 
    \cmidrule(r){7-10}
    Defense Model & Natural & PGD$^{20}$ & PGD$^{100}$  & PGD$^{1000}$  & CW$^{\infty}$ & PGD$^{20}$ & PGD$^{100}$ & PGD$^{1000}$  & CW$^{\infty}$ & AA \\
    \midrule
    NATURAL & 79.30 & 0 & 0 & 0 & 0 & 0 & 0 & 0 & 0 & 0 \\
    \midrule
    AT & 56.73 & 29.57 & 28.45 & 28.39 & 26.6 & 55.52 & 55.29 & 55.26 & 55.40 & 24.12 \\
    % TRADES & 58.24 & 30.10 & 29.66 & 29.64 & 25.97 & 57.05 & 56.71 & 56.67 & 56.77 & 24.92 \\
    $\DIAL_{\kl}$ (Ours) & 58.47 & \textbf{31.19} & \textbf{30.50} & \textbf{30.42} & \textbf{26.91} & 57.16 & 56.81 & 56.80 & 57.00 & \textbf{25.87} \\
    $\DIAL_{\ce}$ (Ours) & \textbf{60.77} & 27.87 & 26.66 & 26.61 & 25.98 & \textbf{59.48} & \textbf{59.06} & \textbf{58.96} & \textbf{59.20} & 23.51  \\
    \bottomrule
  \end{tabular}
\end{table}


% \begin{table}[!ht]
%   \caption{Robustness comparison of DIAL to Madry et al. and TRADES defense models on the SVHN dataset under different PGD white-box attacks and the ensemble Auto-Attack (AA).}
%   \label{svhn}
%   \centering
%   \begin{tabular}{llllll|l}
%     \toprule
%     \cmidrule(r){1-5}
%     Defense Model & Natural & PGD$^{20}$ & PGD$^{100}$ & PGD$^{1000}$ & CW$_{\infty}$ & AA\\
%     \midrule
%     $\DIAL_{\kl}$ (Ours) & $\mathbf{90.66}$ & $\mathbf{58.91}$ & $\mathbf{55.30}$ & $\mathbf{55.12}$ & $\mathbf{53.67}$  & $\mathbf{51.00}$  \\
%     Madry et al. & 89.90 & 53.23 & 49.45 & 49.23 & 48.25 & 45.25  \\
%     TRADES & 90.35 & 57.10 & 54.13 & 54.08 & 52.19 & 49.50 \\
%     \bottomrule
%   \end{tabular}
% \end{table}


\subsection{Performance comparison on CIFAR-10} \label{defence-settings}
In this part, we evaluate the performance of DIAL compared to other well-known methods on CIFAR-10. 
%To be consistent with other methods, 
We follow the same experiment setups as in~\cite{madry2017towards, wang2019improving, zhang2019theoretically}. When experiment settings are not identical between tested methods, we choose the most commonly used settings, and apply it to all experiments. This way, we keep the comparison as fair as possible and avoid reporting changes in results which are caused by inconsistent experiment settings \citep{pang2020bag}. To show that our results are not caused because of what is referred to as \textit{obfuscated gradients}~\citep{athalye2018obfuscated}, we evaluate our method with same setup as in our defense model, under strong attacks (e.g., PGD$^{1000}$) in both white-box, black-box settings, Auto-Attack ~\citep{croce2020reliable}, unforeseen "natural" corruptions~\citep{hendrycks2018benchmarking}, and unforeseen adversaries. To make sure that the reported improvements are not caused by \textit{adversarial overfitting}~\citep{rice2020overfitting}, we report best robust results for each method on average of 3 restarts, while omitting one standard deviation (which is smaller than 0.2\% in all experiments). Additional results for CIFAR-10 as well as comprehensive evaluation on MNIST can be found in Appendix \ref{mnist-results} and \ref{additional_res}.
%To further keep the comparison consistent, we followed the same attack settings for all methods.


\begin{table}[ht]
  \caption{Robustness against white-box, black-box attacks and Auto-Attack (AA) on CIFAR-10. Black-box attacks are generated using naturally trained surrogate model. Natural represents the naturally trained (non-adversarial) model.
  %and applied to the best performing robust models.
  }
  \vskip 0.1in
  \label{black-and_white-cifar}
  \centering
  \small
  \begin{tabular}{cccccccc@{\hspace{1\tabcolsep}}c}
    \toprule
    & & \multicolumn{3}{c}{White-box} & \multicolumn{3}{c}{Black-Box} \\
    \cmidrule(r){3-5} 
    \cmidrule(r){6-8}
    Defense Model & Natural & PGD$^{20}$ & PGD$^{100}$ & CW$^{\infty}$ & PGD$^{20}$ & PGD$^{100}$ & CW$^{\infty}$ & AA \\
    \midrule
    NATURAL & 95.43 & 0 & 0 & 0 & 0 & 0 & 0 &  0 \\
    \midrule
    TRADES & 84.92 & 56.60 & 55.56 & 54.20 & 84.08 & 83.89 & 83.91 &  53.08 \\
    MART & 83.62 & 58.12 & 56.48 & 53.09 & 82.82 & 82.52 & 82.80 & 51.10 \\
    AT & 85.10 & 56.28 & 54.46 & 53.99 & 84.22 & 84.14 & 83.92 & 51.52 \\
    ATDA & 76.91 & 43.27 & 41.13 & 41.01 & 75.59 & 75.37 & 75.35 & 40.08\\
    $\DIAL_{\kl}$ (Ours) & 85.25 & $\mathbf{58.43}$ & $\mathbf{56.80}$ & $\mathbf{55.00}$ & 84.30 & 84.18 & 84.05 & \textbf{53.75} \\
    $\DIAL_{\ce}$ (Ours)  & $\mathbf{89.59}$ & 54.31 & 51.67 & 52.04 &$ \mathbf{88.60}$ & $\mathbf{88.39}$ & $\mathbf{88.44}$ & 49.85 \\
    \midrule
    $\DIAL_{\awp}$ (Ours) & $\mathbf{85.91}$ & $\mathbf{61.10}$ & $\mathbf{59.86}$ & $\mathbf{57.67}$ & $\mathbf{85.13}$ & $\mathbf{84.93}$ & $\mathbf{85.03}$  & \textbf{56.78} \\
    $\TRADES_{\awp}$ & 85.36 & 59.27 & 59.12 & 57.07 & 84.58 & 84.58 & 84.59 & 56.17 \\
    \bottomrule
  \end{tabular}
\end{table}



\paragraph{CIFAR-10 setup.} We use the wide residual network (WRN-34-10)~\citep{zagoruyko2016wide} architecture. %used in the experiments of~\cite{madry2017towards, wang2019improving, zhang2019theoretically}. 
Sidelong this architecture, we integrate a domain classification layer. To generate the adversarial domain dataset, we use a perturbation size of $\epsilon=0.031$. We apply 10 of inner maximization iterations with perturbation step size of 0.007. Batch size is set to 128, weight decay is set to $7e^{-4}$, and the model is trained for 100 epochs. Similar to the other methods, the initial learning rate was set to 0.1, and decays by a factor of 10 at iterations 75 and 90. 
%For being consistent with other methods, the natural images are padded with 4-pixel padding with 32-random crop and random horizontal flip. Furthermore, all methods are trained using SGD with momentum 0.9. For $\DIAL_{\kl}$, we balance the robust loss with $\lambda=6$ and the domains loss with $r=4$. For $\DIAL_{\ce}$, we balance the robust loss with $\lambda=1$ and the domains loss with $r=2$. 
%We also introduce a version of our method that incorporates the AWP double-perturbation mechanism, named DIAL-AWP.
%which is trained using the same learning rate schedule used in ~\cite{wu2020adversarial}, where the initial 0.1 learning rate decays by a factor of 10 after 100 and 150 iterations. 
See Appendix \ref{cifar10-additional-setup} for additional details.

\begin{table}[ht]
  \caption{Black-box attack using the adversarially trained surrogate models on CIFAR-10.}
  \vskip 0.1in
  \label{black-box-cifar-adv}
  \centering
  \small
  \begin{tabular}{ll|c}
    \toprule
    \cmidrule(r){1-2}
    Surrogate (source) model & Target model & robustness \% \\
    % \midrule
    \midrule
    TRADES & $\DIAL_{\ce}$ & $\mathbf{67.77}$ \\
    $\DIAL_{\ce}$ & TRADES & 65.75 \\
    \midrule
    MART & $\DIAL_{\ce}$ & $\mathbf{70.30}$ \\
    $\DIAL_{\ce}$ & MART & 64.91 \\
    \midrule
    AT & $\DIAL_{\ce}$ & $\mathbf{65.32}$ \\
    $\DIAL_{\ce}$ & AT  & 63.54 \\
    \midrule
    ATDA & $\DIAL_{\ce}$ & $\mathbf{66.77}$ \\
    $\DIAL_{\ce}$ & ATDA & 52.56 \\
    \bottomrule
  \end{tabular}
\end{table}

\paragraph{White-box/Black-box robustness.} 
%We evaluate all defense models using Auto-Attack, PGD$^{20}$, PGD$^{100}$, PGD$^{1000}$ and CW$_{\infty}$ with step size 0.003. We constrain all attacks by the same perturbation $\epsilon=0.031$. 
As reported in Table~\ref{black-and_white-cifar} and Appendix~\ref{additional_res}, our method achieves better robustness compared to the other methods. Specifically, in the white-box settings, our method improves robustness over~\citet{madry2017towards} and TRADES by 2\% 
%using the common PGD$^{20}$ attack 
while keeping higher natural accuracy. We also observe better natural accuracy of 1.65\% over MART while also achieving better robustness over all attacks. Moreover, our method presents significant improvement of up to 15\% compared to the the domain invariant method suggested by~\citet{song2018improving} (ATDA).
%in both natural and robust accuracy. 
When incorporating 
%the double-perturbation mechanism of 
AWP, our method improves the results of $\TRADES_{\awp}$ by almost 2\%.
%and reaches state-of-the-art results for robust models with no additional data. 
% Additional results are available in Appendix~\ref{additional_res}.
When tested on black-box settings, $\DIAL_{\ce}$ presents a significant improvement of more than 4.4\% over the second-best performing method, and up to 13\%. In Table~\ref{black-box-cifar-adv}, we also present the black-box results when the source model is taken from one of the adversarially trained models. %Then, we compare our model to each one of them both as the source model and target model. 
In addition to the improvement in black-box robustness, $\DIAL_{\ce}$ also manages to achieve better clean accuracy of more than 4.5\% over the second-best performing method.
% Moreover, based on the auto-attack leader-board \footnote{\url{https://github.com/fra31/auto-attack}}, our method achieves the 1st place among models without additional data using the WRN-34-10 architecture.

% \begin{table}
%   \caption{White-box robustness on CIFAR-10 using WRN-34-10}
%   \label{white-box-cifar-10}
%   \centering
%   \begin{tabular}{lllll}
%     \toprule
%     \cmidrule(r){1-2}
%     Defense Model & Natural & PGD$^{20}$ & PGD$^{100}$ & PGD$^{1000}$ \\
%     \midrule
%     TRADES ~\cite{zhang2019theoretically} & 84.92  & 56.6 & 55.56 & 56.43  \\
%     MART ~\cite{wang2019improving} & 83.62  & 58.12 & 56.48 & 56.55  \\
%     Madry et al. ~\cite{madry2017towards} & 85.1  & 56.28 & 54.46 & 54.4  \\
%     Song et al. ~\cite{song2018improving} & 76.91 & 43.27 & 41.13 & 41.02  \\
%     $\DIAL_{\ce}$ (Ours) & $ \mathbf{90}$  & 52.12 & 48.88 & 48.78  \\
%     $\DIAL_{\kl}$ (Ours) & 85.25 & $\mathbf{58.43}$ & $\mathbf{56.8}$ & $\mathbf{56.73}$ \\
%     \midrule
%     $\DIAL_{\kl}$+AWP (Ours) & $\mathbf{85.91}$ & $\mathbf{61.1}$ & - & -  \\
%     TRADES+AWP \cite{wu2020adversarial} & 85.36 & 59.27 & 59.12 & -  \\
%     % MART + AWP & 84.43 & 60.68 & 59.32 & -  \\
%     \bottomrule
%   \end{tabular}
% \end{table}


% \begin{table}
%   \caption{White-box robustness on MNIST}
%   \label{white-box-mnist}
%   \centering
%   \begin{tabular}{llllll}
%     \toprule
%     \cmidrule(r){1-2}
%     Defense Model & Natural & PGD$^{40}$ & PGD$^{100}$ & PGD$^{1000}$ \\
%     \midrule
%     TRADES ~\cite{zhang2019theoretically} & 99.48 & 96.07 & 95.52 & 95.22 \\
%     MART ~\cite{wang2019improving} & 99.38  & 96.99 & 96.11 & 95.74  \\
%     Madry et al. ~\cite{madry2017towards} & 99.41  & 96.01 & 95.49 & 95.36 \\
%     Song et al. ~\cite{song2018improving}  & 98.72 & 96.82 & 96.26 & 96.2  \\
%     $\DIAL_{\kl}$ (Ours) & 99.46 & 97.05 & 96.06 & 95.99  \\
%     $\DIAL_{\ce}$ (Ours) & $\mathbf{99.49}$  & $\mathbf{97.38}$ & $\mathbf{96.45}$ & $\mathbf{96.33}$ \\
%     \bottomrule
%   \end{tabular}
% \end{table}


% \paragraph{Attacking MNIST.} For consistency, we use the same perturbation and step sizes. For MNIST, we use $\epsilon=0.3$ and step size of $0.01$. The natural accuracy of our surrogate (source) model is 99.51\%. Attacks results are reported in Table~\ref{black-and_white-mnist}. It is worth noting that the improvement margin is not conclusive on MNIST as it is on CIFAR-10, which is a more complex task.

% \begin{table}
%   \caption{Black-box robustness on MNIST and CIFAR-10 using naturally trained surrogate model and best performing robust models}
%   \label{black-box-mnist-cifar}
%   \centering
%   \begin{tabular}{lllllll}
%     \toprule
%     & \multicolumn{3}{c}{MNIST} & \multicolumn{3}{c}{CIFAR-10} \\
%     \cmidrule(r){2-4} 
%     \cmidrule(r){5-7}  
%     Defense Model & PGD$^{40}$ & PGD$^{100}$ & PGD$^{1000}$ & PGD$^{20}$ & PGD$^{100}$ & PGD$^{1000}$ \\
%     \midrule
%     TRADES ~\cite{zhang2019theoretically} & 98.12 & 97.86 & 97.81 & 84.08 & 83.89 & 83.8 \\
%     MART ~\cite{wang2019improving} & 98.16 & 97.96 & 97.89  & 82.82 & 82.52 & 82.47 \\
%     Madry et al. ~\cite{madry2017towards}  & 98.05 & 97.73 & 97.78 & 84.22 & 84.14 & 83.96 \\
%     Song et al. ~\cite{song2018improving} & 97.74 & 97.28 & 97.34 & 75.59 & 75.37 & 75.11 \\
%     $\DIAL_{\kl}$ (Ours) & 98.14 & 97.83 & 97.87  & 84.3 & 84.18 & 84.0 \\
%     $\DIAL_{\ce}$ (Ours)  & $\mathbf{98.37}$ & $\mathbf{98.12}$ & $\mathbf{98.05}$  & $\mathbf{89.13}$ & $\mathbf{88.89}$ & $\mathbf{88.78}$ \\
%     \bottomrule
%   \end{tabular}
% \end{table}



% \subsubsection{Ensemble attack} In addition to the white-box and black-box settings, we evaluate our method on the Auto-Attack ~\citep{croce2020reliable} using $\ell_{\infty}$ threat model with perturbation $\epsilon=0.031$. Auto-Attack is an ensemble of parameter-free attacks. It consists of three white-box attacks: APGD-CE which is a step size-free version of PGD on the cross-entropy ~\citep{croce2020reliable}. APGD-DLR which is a step size-free version of PGD on the DLR loss ~\citep{croce2020reliable} and FAB which  minimizes the norm of the adversarial perturbations, and one black-box attack: square attack which is a query-efficient black-box attack ~\citep{andriushchenko2020square}. Results are presented in Table~\ref{auto-attack}. Based on the auto-attack leader-board \footnote{\url{https://github.com/fra31/auto-attack}}, our method achieves the 1st place among models without additional data using the WRN-34-10 architecture.

%Additional results can be found in Appendix ~\ref{additional_res}.

% \begin{table}
%   \caption{Auto-Attack (AA) on CIFAR-10 with perturbation size $\epsilon=0.031$ with $\ell_{\infty}$ threat model}
%   \label{auto-attack}
%   \centering
%   \begin{tabular}{lll}
%     \toprule
%     \cmidrule(r){1-2}
%     Defense Model & AA \\
%     \midrule
%     TRADES ~\cite{zhang2019theoretically} & 53.08  \\
%     MART ~\cite{wang2019improving} & 51.1  \\
%     Madry et al. ~\cite{madry2017towards} & 51.52    \\
%     Song et al.   ~\cite{song2018improving} & 40.18 \\
%     $\DIAL_{\ce}$ (Ours) & 47.33  \\
%     $\DIAL_{\kl}$ (Ours) & $\mathbf{53.75}$ \\
%     \midrule
%     DIAL-AWP (Ours) & $\mathbf{56.78}$ \\
%     TRADES-AWP \cite{wu2020adversarial} & 56.17 \\
%     \bottomrule
%   \end{tabular}
% \end{table}


% \begin{table}[!ht]
%   \caption{Auto-Attack (AA) Robustness (\%) on CIFAR-10 with $\epsilon=0.031$ using an $\ell_{\infty}$ threat model}
%   \label{auto-attack}
%   \centering
%   \begin{tabular}{cccccc|cc}
%     \toprule
%     % \multicolumn{8}{c}{Defence Model}  \\
%     % \cmidrule(r){1-8} 
%     TRADES & MART & Madry & Song & $\DIAL_{\ce}$ & $\DIAL_{\kl}$ & DIAL-AWP  & TRADES-AWP\\
%     \midrule
%     53.08 & 51.10 & 51.52 &  40.08 & 47.33  & $\mathbf{53.75}$ & $\mathbf{56.78}$ & 56.17 \\

%     \bottomrule
%   \end{tabular}
% \end{table}

% \begin{table}[!ht]
% \caption{$F_1$-robust measurement using PGD$^{20}$ attack in white-box and black-box settings on CIFAR-10}
%   \label{f1-robust}
%   \centering
%   \begin{tabular}{ccccccc|cc}
%     \toprule
%     % \multicolumn{8}{c}{Defence Model}  \\
%     % \cmidrule(r){1-8} 
%     Defense Model & TRADES & MART & Madry & Song & $\DIAL_{\kl}$ & $\DIAL_{\ce}$ & DIAL-AWP  & TRADES-AWP\\
%     \midrule
%     White-box & 0.659 & 0.666 & 0.657 & 0.518 & $\mathbf{0.675}$  & 0.643 & $\mathbf{0.698}$ & 0.682 \\
%     Black-box & 0.844 & 0.831 & 0.846 & 0.761 & 0.847 & $\mathbf{0.895}$ & $\mathbf{0.854}$ &  0.849 \\
%     \bottomrule
%   \end{tabular}
% \end{table}

\subsubsection{Robustness to Unforeseen Attacks and Corruptions}
\paragraph{Unforeseen Adversaries.} To further demonstrate the effectiveness of our approach, we test our method against various adversaries that were not used during the training process. We attack the model under the white-box settings with $\ell_{2}$-PGD, $\ell_{1}$-PGD, $\ell_{\infty}$-DeepFool and $\ell_{2}$-DeepFool \citep{moosavi2016deepfool} adversaries using Foolbox \citep{rauber2017foolbox}. We applied commonly used attack budget 
%(perturbation for PGD adversaries and overshot for DeepFool adversaries) 
with 20 and 50 iterations for PGD and DeepFool, respectively.
Results are presented in Table \ref{unseen-attacks}. As can be seen, our approach  gains an improvement of up to 4.73\% over the second best method under the various attack types and an average improvement of 3.7\% over all threat models.


\begin{table}[ht]
  \caption{Robustness on CIFAR-10 against unseen adversaries under white-box settings.}
  \vskip 0.1in
  \label{unseen-attacks}
  \centering
%   \small
  \begin{tabular}{c@{\hspace{1.5\tabcolsep}}c@{\hspace{1.5\tabcolsep}}c@{\hspace{1.5\tabcolsep}}c@{\hspace{1.5\tabcolsep}}c@{\hspace{1.5\tabcolsep}}c@{\hspace{1.5\tabcolsep}}c@{\hspace{1.5\tabcolsep}}c}
    \toprule
    Threat Model & Attack Constraints & $\DIAL_{\kl}$ & $\DIAL_{\ce}$ & AT & TRADES & MART & ATDA \\
    \midrule
    \multirow{2}{*}{$\ell_{2}$-PGD} & $\epsilon=0.5$ & 76.05 & \textbf{80.51} & 76.82 & 76.57 & 75.07 & 66.25 \\
    & $\epsilon=0.25$ & 80.98 & \textbf{85.38} & 81.41 & 81.10 & 80.04 & 71.87 \\\midrule
    \multirow{2}{*}{$\ell_{1}$-PGD} & $\epsilon=12$ & 74.84 & \textbf{80.00} & 76.17 & 75.52 & 75.95 & 65.76 \\
    & $\epsilon=7.84$ & 78.69 & \textbf{83.62} & 79.86 & 79.16 & 78.55 & 69.97 \\
    \midrule
    $\ell_{2}$-DeepFool & overshoot=0.02 & 84.53 & \textbf{88.88} & 84.15 & 84.23 & 82.96 & 76.08 \\\midrule
    $\ell_{\infty}$-DeepFool & overshoot=0.02 & 68.43 & \textbf{69.50} & 67.29 & 67.60 & 66.40 & 57.35 \\
    \bottomrule
  \end{tabular}
\end{table}


%%%%%%%%%%%%%%%%%%%%%%%%% conference version %%%%%%%%%%%%%%%%%%%%%%%%%%%%%%%%%%%%%
\paragraph{Unforeseen Corruptions.}
We further demonstrate that our method consistently holds against unforeseen ``natural'' corruptions, consists of 18 unforeseen diverse corruption types proposed by \citet{hendrycks2018benchmarking} on CIFAR-10, which we refer to as CIFAR10-C. The CIFAR10-C benchmark covers noise, blur, weather, and digital categories. As can be shown in Figure \ref{corruption}, our method gains a significant and consistent improvement over all the other methods. Our method leads to an average improvement of 4.7\% with minimum improvement of 3.5\% and maximum improvement of 5.9\% compared to the second best method over all unforeseen attacks. See Appendix \ref{corruptions-apendix} for the full experiment results.


\begin{figure}[h]
 \centering
  \includegraphics[width=0.4\textwidth]{figures/spider_full.png}
%   \caption{Summary of accuracy over all unforeseen corruptions compared to the second and third best performing methods.}
  \caption{Accuracy comparison over all unforeseen corruptions.}
  \label{corruption}
\end{figure}


%%%%%%%%%%%%%%%%%%%%%%%%% conference version %%%%%%%%%%%%%%%%%%%%%%%%%%%%%%%%%%%%%

%%%%%%%%%%%%%%%%%%%%%%%%% Arxiv version %%%%%%%%%%%%%%%%%%%%%%%%%%%%%%%%%%%%%
% \newpage
% \paragraph{Unforeseen Corruptions.}
% We further demonstrate that our method consistently holds against unforeseen "natural" corruptions, consists of 18 unforeseen diverse corruption types proposed by \cite{hendrycks2018benchmarking} on CIFAR-10, which we refer to as CIFAR10-C. The CIFAR10-C benchmark covers noise, blur, weather, and digital categories. As can be shown in Figure  \ref{spider-full-graph}, our method gains a significant and consistent improvement over all the other methods. Our approach leads to an average improvement of 4.7\% with minimum improvement of 3.5\% and maximum improvement of 5.9\% compared to the second best method over all unforeseen attacks. Full accuracy results against unforeseen corruptions are presented in Tables \ref{corruption-table1} and \ref{corruption-table2}. 

% \begin{table}[!ht]
%   \caption{Accuracy (\%) against unforeseen corruptions.}
%   \label{corruption-table1}
%   \centering
%   \tiny
%   \begin{tabular}{lcccccccccccccccccc}
%     \toprule
%     Defense Model & brightness & defocus blur & fog & glass blur & jpeg compression & motion blur & saturate & snow & speckle noise  \\
%     \midrule
%     TRADES & 82.63 & 80.04 & 60.19 & 78.00 & 82.81 & 76.49 & 81.53 & 80.68 & 80.14 \\
%     MART & 80.76 & 78.62 & 56.78 & 76.60 & 81.26 & 74.58 & 80.74 & 78.22 & 79.42 \\
%     AT &  83.30 & 80.42 & 60.22 & 77.90 & 82.73 & 76.64 & 82.31 & 80.37 & 80.74 \\
%     ATDA & 72.67 & 69.36 & 45.52 & 64.88 & 73.22 & 63.47 & 72.07 & 68.76 & 72.27 \\
%     DIAL (Ours)  & \textbf{87.14} & \textbf{84.84} & \textbf{66.08} & \textbf{81.82} & \textbf{87.07} & \textbf{81.20} & \textbf{86.45} & \textbf{84.18} & \textbf{84.94} \\
%     \bottomrule
%   \end{tabular}
% \end{table}


% \begin{table}[!ht]
%   \caption{Accuracy (\%) against unforeseen corruptions.}
%   \label{corruption-table2}
%   \centering
%   \tiny
%   \begin{tabular}{lcccccccccccccccccc}
%     \toprule
%     Defense Model & contrast & elastic transform & frost & gaussian noise & impulse noise & pixelate & shot noise & spatter & zoom blur \\
%     \midrule
%     TRADES & 43.11 & 79.11 & 76.45 & 79.21 & 73.72 & 82.73 & 80.42 & 80.72 & 78.97 \\
%     MART & 41.22 & 77.77 & 73.07 & 78.30 & 74.97 & 81.31 & 79.53 & 79.28 & 77.8 \\
%     AT & 43.30 & 79.58 & 77.53 & 79.47 & 73.76 & 82.78 & 80.86 & 80.49 & 79.58 \\
%     ATDA & 36.06 & 67.06 & 62.56 & 70.33 & 64.63 & 73.46 & 72.28 & 70.50 & 67.31 \\
%     DIAL (Ours) & \textbf{48.84} & \textbf{84.13} & \textbf{81.76} & \textbf{83.76} & \textbf{78.26} & \textbf{87.24} & \textbf{85.13} & \textbf{84.84} & \textbf{83.93}  \\
%     \bottomrule
%   \end{tabular}
% \end{table}


% \begin{figure}[!ht]
%   \centering
%   \includegraphics[width=9cm]{figures/spider_full.png}
%   \caption{Accuracy comparison with all tested methods over unforeseen corruptions.}
%   \label{spider-full-graph}
% \end{figure}
% %%%%%%%%%%%%%%%%%%%%%%%%% Arxiv version %%%%%%%%%%%%%%%%%%%%%%%%%%%%%%%%%%%%%
%%%%%%%%%%%%%%%%%%%%%%%%% Arxiv version %%%%%%%%%%%%%%%%%%%%%%%%%%%%%%%%%%%%%

\subsubsection{Transfer Learning}
Recent works \citep{salman2020adversarially,utrera2020adversarially} suggested that robust models transfer better on standard downstream classification tasks. In Table \ref{transfer-res} we demonstrate the advantage of our method when applied for transfer learning across CIFAR10 and CIFAR100 using the common linear evaluation protocol. see Appendix \ref{transfer-learning-settings} for detailed settings.

\begin{table}[H]
  \caption{Transfer learning results comparison.}
  \vskip 0.1in
  \label{transfer-res}
  \centering
  \small
\begin{tabular}{c|c|c|c}
\toprule

\multicolumn{2}{l}{} & \multicolumn{2}{c}{Target} \\
\cmidrule(r){3-4}
Source & Defence Model & CIFAR10 & CIFAR100 \\
\midrule
\multirow{3}{*}{CIFAR10} & DIAL & \multirow{3}{*}{-} & \textbf{28.57} \\
 & AT &  & 26.95  \\
 & TRADES &  & 25.40  \\
 \midrule
\multirow{3}{*}{CIFAR100} & DIAL & \textbf{73.68} & \multirow{3}{*}{-} \\
 & AT & 71.41 & \\
 & TRADES & 71.42 &  \\
%  \midrule
% \multirow{3}{}{SVHN} & DIAL &  &  & \multirow{3}{}{-} \\
%  & Madry et al. &  &  &  \\
%  & TRADES &  &  &  \\ 
\bottomrule
\end{tabular}
\end{table}


\subsubsection{Modularity and Ablation Studies}

We note that the domain classifier is a modular component that can be integrated into existing models for further improvements. Removing the domain head and related loss components from the different DIAL formulations results in some common adversarial training techniques. For $\DIAL_{\kl}$, removing the domain and related loss components results in the formulation of TRADES. For $\DIAL_{\ce}$, removing the domain and related loss components results in the original formulation of the standard adversarial training, and for $\DIAL_{\awp}$ the removal results in $\TRADES_{\awp}$. Therefore, the ablation studies will demonstrate the effectiveness of combining DIAL on top of different adversarial training methods. 

We investigate the contribution of the additional domain head component introduced in our method. Experiment configuration are as in \ref{defence-settings}, and robust accuracy is based on white-box PGD$^{20}$ on CIFAR-10 dataset. We remove the domain head from both $\DIAL_{\kl}$, $\DIAL_{\awp}$, and $\DIAL_{\ce}$ (equivalent to $r=0$) and report the natural and robust accuracy. We perform 3 random restarts and omit one standard deviation from the results. Results are presented in Figure \ref{ablation}. All DIAL variants exhibits stable improvements on both natural accuracy and robust accuracy. $\DIAL_{\ce}$, $\DIAL_{\kl}$, and $\DIAL_{\awp}$ present an improvement of 1.82\%, 0.33\%, and 0.55\% on natural accuracy and an improvement of 2.5\%, 1.87\%, and 0.83\% on robust accuracy, respectively. This evaluation empirically demonstrates the benefits of incorporating DIAL on top of different adversarial training techniques.
% \paragraph{semi-supervised extensions.} Since the domain classifier does not require the class labels, we argue that additional unlabeled data can be leveraged in future work.
%for improved results. 

\begin{figure}[ht]
  \centering
  \includegraphics[width=0.35\textwidth]{figures/ablation_graphs3.png}
  \caption{Ablation studies for $\DIAL_{\kl}$, $\DIAL_{\ce}$, and $\DIAL_{\awp}$ on CIFAR-10. Circle represent the robust-natural accuracy without using DIAL, and square represent the robust-natural accuracy when incorporating DIAL.
  %to further investigate the impact of the domain head and loss on natural and robust accuracy.
  }
  \label{ablation}
\end{figure}

\subsubsection{Visualizing DIAL}
To further illustrate the superiority of our method, we visualize the model outputs from the different methods on both natural and adversarial test data.
% adversarial test data generated using PGD$^{20}$ white-box attack with step size 0.003 and $\epsilon=0.031$ on CIFAR-10. 
Figure~\ref{tsne1} shows the embedding received after applying t-SNE ~\citep{van2008visualizing} with two components on the model output for our method and for TRADES. DIAL seems to preserve strong separation between classes on both natural test data and adversarial test data. Additional illustrations for the other methods are attached in Appendix~\ref{additional_viz}. 

\begin{figure}[h]
\centering
  \subfigure[\textbf{DIAL} on natural logits]{\includegraphics[width=0.21\textwidth]{figures/domain_ce_test.png}}
  \hspace{0.03\textwidth}
  \subfigure[\textbf{DIAL} on adversarial logits]{\includegraphics[width=0.21\textwidth]{figures/domain_ce_adversarial.png}}
  \hspace{0.03\textwidth}
    \subfigure[\textbf{TRADES} on natural logits]{\includegraphics[width=0.21\textwidth]{figures/trades_test.png}}
    \hspace{0.03\textwidth}
    \subfigure[\textbf{TRADES} on adversarial logits]{\includegraphics[width=0.21\textwidth]{figures/trades_adversarial.png}}
  \caption{t-SNE embedding of model output (logits) into two-dimensional space for DIAL and TRADES using the CIFAR-10 natural test data and the corresponding PGD$^{20}$ generated adversarial examples.}
  \label{tsne1}
\end{figure}


% \begin{figure}[ht]
% \centering
%   \begin{subfigure}{4cm}
%     \centering\includegraphics[width=3.3cm]{figures/domain_ce_test.png}
%     \caption{\textbf{DIAL} on nat. examples}
%   \end{subfigure}
%   \begin{subfigure}{4cm}
%     \centering\includegraphics[width=3.3cm]{figures/domain_ce_adversarial.png}
%     \caption{\textbf{DIAL} on adv. examples}
%   \end{subfigure}
  
%   \begin{subfigure}{4cm}
%     \centering\includegraphics[width=3.3cm]{figures/trades_test.png}
%     \caption{\textbf{TRADES} on nat. examples}
%   \end{subfigure}
%   \begin{subfigure}{4cm}
%     \centering\includegraphics[width=3.3cm]{figures/trades_adversarial.png}
%     \caption{\textbf{TRADES} on adv. examples}
%   \end{subfigure}
%   \caption{t-SNE embedding of model output (logits) into two-dimensional space for DIAL and TRADES using the CIFAR-10 natural test data and the corresponding adversarial examples.}
%   \label{tsne1}
% \end{figure}



% \begin{figure}[ht]
% \centering
%   \begin{subfigure}{6cm}
%     \centering\includegraphics[width=5cm]{figures/domain_ce_test.png}
%     \caption{\textbf{DIAL} on nat. examples}
%   \end{subfigure}
%   \begin{subfigure}{6cm}
%     \centering\includegraphics[width=5cm]{figures/domain_ce_adversarial.png}
%     \caption{\textbf{DIAL} on adv. examples}
%   \end{subfigure}
  
%   \begin{subfigure}{6cm}
%     \centering\includegraphics[width=5cm]{figures/trades_test.png}
%     \caption{\textbf{TRADES} on nat. examples}
%   \end{subfigure}
%   \begin{subfigure}{6cm}
%     \centering\includegraphics[width=5cm]{figures/trades_adversarial.png}
%     \caption{\textbf{TRADES} on adv. examples}
%   \end{subfigure}
%   \caption{t-SNE embedding of model output (logits) into two-dimensional space for DIAL and TRADES using the CIFAR-10 natural test data and the corresponding adversarial examples.}
%   \label{tsne1}
% \end{figure}



\subsection{Balanced measurement for robust-natural accuracy}
One of the goals of our method is to better balance between robust and natural accuracy under a given model. For a balanced metric, we adopt the idea of $F_1$-score, which is the harmonic mean between the precision and recall. However, rather than using precision and recall, we measure the $F_1$-score between robustness and natural accuracy,
using a measure we call
%We named it
the
\textbf{$\mathbf{F_1}$-robust} score.
\begin{equation}
F_1\text{-robust} = \dfrac{\text{true\_robust}}
{\text{true\_robust}+\frac{1}{2}
%\cdot
(\text{false\_{robust}}+
\text{false\_natural})},
\end{equation}
where $\text{true\_robust}$ are the adversarial examples that were correctly classified, $\text{false\_{robust}}$ are the adversarial examples that were miss-classified, and $\text{false\_natural}$ are the natural examples that were miss-classified.
%We tested the proposed $F_1$-robust score using PGD$^{20}$ on CIFAR-10 dataset in white-box and black-box settings. 
Results are presented in Table~\ref{f1-robust} and demonstrate that our method achieves the best $F_1$-robust score in both settings, which supports our findings from previous sections.

% \begin{table}[!ht]
%   \caption{$F_1$-robust measurement using PGD$^{20}$ attack in white and black box settings on CIFAR-10}
%   \label{f1-robust}
%   \centering
%   \begin{tabular}{lll}
%     \toprule
%     \cmidrule(r){1-2}
%     Defense Model & White-box & Black-box \\
%     \midrule
%     TRADES & 0.65937  & 0.84435 \\
%     MART & 0.66613  & 0.83153  \\
%     Madry et al. & 0.65755 & 0.84574   \\
%     Song et al. & 0.51823 & 0.76092  \\
%     $\DIAL_{\ce}$ (Ours) & 0.65318   & $\mathbf{0.88806}$  \\
%     $\DIAL_{\kl}$ (Ours) & $\mathbf{0.67479}$ & 0.84702 \\
%     \midrule
%     \midrule
%     DIAL-AWP (Ours) & $\mathbf{0.69753}$  & $\mathbf{0.85406}$  \\
%     TRADES-AWP & 0.68162 & 0.84917 \\
%     \bottomrule
%   \end{tabular}
% \end{table}

\begin{table}[ht]
\small
  \caption{$F_1$-robust measurement using PGD$^{20}$ attack in white and black box settings on CIFAR-10.}
  \vskip 0.1in
  \label{f1-robust}
  \centering
%   \small
  \begin{tabular}{c
  @{\hspace{1.5\tabcolsep}}c @{\hspace{1.5\tabcolsep}}c @{\hspace{1.5\tabcolsep}}c @{\hspace{1.5\tabcolsep}}c
  @{\hspace{1.5\tabcolsep}}c @{\hspace{1.5\tabcolsep}}c @{\hspace{1.5\tabcolsep}}|
  @{\hspace{1.5\tabcolsep}}c
  @{\hspace{1.5\tabcolsep}}c}
    \toprule
    % \cmidrule(r){8-9}
     & TRADES & MART & AT & ATDA & $\DIAL_{\ce}$ & $\DIAL_{\kl}$ & $\DIAL_{\awp}$ & $\TRADES_{\awp}$ \\
    \midrule
    White-box & 0.659 & 0.666 & 0.657 & 0.518 & 0.660 & \textbf{0.675} & \textbf{0.698} & 0.682 \\
    Black-box & 0.844 & 0.831 & 0.845 & 0.761 & \textbf{0.890} & 0.847 & \textbf{0.854} & 0.849 \\ 
    \bottomrule
  \end{tabular}
\end{table}
   % This file has Experimental section

%the following two sections might be merged into the same file
\section{Related Work}
\label{sec:related_work}
We now provide a brief overview of related work in the areas of language grounding and transfer for reinforcement learning.
%There has been work on learning to make optimal local decisions for structured prediction problems~\cite{daume2006searn}.
%
%\newcite{branavan2010reading} looked at a similar task of building a partial model of the environment while following instructions. The differences with our work are (1) the text in their case is instructions, while we only have text describing the environment, and (2) their environment is deterministic, hence the transition function can be learned more easily. 
%
%TODO - model-based RL, value iteration, predictron.


\subsection{Grounding Language in Interactive Environments}
In recent years, there has been increasing interest in systems that can utilize textual knowledge to learn control policies. Such applications include interpreting help documentation~\fullcite{branavan2010reading}, instruction following~\fullcite{vogel2010learning,kollar2010toward,artzi2013weakly,matuszek2013learning,Andreas15Instructions} and learning to play computer games~\fullcite{branavan2011nonlinear,branavan2012learning,narasimhan2015language,he2016deep}. In all these applications, the models are trained and tested on the same domain.

Our work represents two departures from prior work on grounding. First, rather than optimizing control performance for a single domain,
we are interested in the multi-domain transfer scenario, where language 
descriptions drive generalization. Second, prior work used text in the form of strategy advice to directly learn the policy. Since the policies are typically optimized for a specific task, they may be harder to transfer across domains. Instead, we utilize text to bootstrap the induction of the environment dynamics, moving beyond task-specific strategies. 

%Previous work has explored the use of text manuals in game playing, %ranging from constructing useful features by mining patterns in %text~\cite{eisenstein2009reading}, learning a semantic interpreter %with access to limited gameplay examples~\cite{goldwasser2014learning} %to learning through reinforcement from in-game %rewards~\cite{branavan2011learning}. These efforts have demonstrated %the usefulness of exploiting domain knowledge encoded in text to learn %effective policies. However, these methods use the text to infer %directly the best strategy to perform a task. In contrast, our goal is %to learn mappings from the text to the dynamics of an environment and %separate out the learning of the strategy/motives. 

Another related line of work consists of systems that learn spatial and topographical maps of the environment for robot navigation using natural language descriptions~\fullcite{walter2013learning,hemachandra2014learning}. These approaches use text mainly containing appearance and positional information, and integrate it with other semantic sources (such as appearance models) to obtain more accurate maps. In contrast, our work uses language describing the dynamics of the environment, such as entity movements and interactions, which 
is complementary to static positional information received through state observations. Further, our goal is to help an agent learn policies that generalize over different stochastic domains, while their works consider a single domain.

%karthik: I don't see the direct relevance
%Another line of work explores using textual interactive %environments~\cite{narasimhan2015language,he2016deep} to ground %language understanding into actions taken by the system in the %environment. In these tasks, understanding of language is crucial, %without which a system would not be able to take reasonable actions. %Our motivation is different -- we take an existing set of tasks and %domains which are amenable to learning through reinforcement, and %demonstrate how to utilize textual knowledge to learn faster and more %optimal policies in both multitask and transfer setups.

\subsection{Transfer in Reinforcement Learning}
Transferring policies across domains is a challenging problem in reinforcement learning~\fullcite{konidaris2006framework,taylor2009transfer}. The main hurdle lies in learning a good mapping between the state and action spaces of different domains to enable effective transfer. Most previous approaches have either explored skill transfer~\fullcite{konidaris2007building,konidaris2012transfer} or value function/policy transfer~\fullcite{liu2006value,taylor2007transfer,taylor2007cross}. There have also been attempts at model-based transfer for RL~\fullcite{taylor2008transferring,nguyen2012transferring,gavsic2013pomdp,wang2015learning,joshi2018cross} but these methods either rely on hand-coded inter-task mappings for state and actions spaces or require significant interactions in the target task to learn an effective mapping. Our approach doesn't use any explicit mappings and can learn to predict the dynamics of a target task using its descriptions.

% Work by \newcite{konidaris2006autonomous} look at knowledge transfer by learning a mapping from sensory signals to reward functions.

A closely related line of work concerns transfer methods for deep reinforcement learning. \citeA{parisotto2016actor}  train a deep network to mimic pre-trained experts on source tasks using policy distillation. The learned parameters are then used to initialize a network on a target task to perform transfer. Rusu et al.~\citeyear{rusu2016progressive} facilitate transfer by freezing parameters learned on source tasks and adding a new set of parameters for every new target task, while using both sets to learn the new policy. Work by Rajendran et al.~\citeyear{rajendran20172t} uses attention networks to selectively transfer from a set of expert policies to a new task. \textcolor{black}{Barreto et al.~\citeyear{barreto2017successor} use features based on successor representations~\fullcite{dayan1993improving} for transfer across tasks in the same domain. Kansky~et~al.~\citeyear{kansky2017schema} learn a generative model of causal physics in order to help zero-shot transfer learning.} Our approach is orthogonal to all these directions since we use text to bootstrap transfer, and can potentially be combined with these methods to achieve more effective transfer. 

\textcolor{black}{There has also been prior work on zero-shot policy generalization for tasks with input goal specifications. \fullciteA{schaul2015universal} learn a universal value function approximator that can generalize across both states and goals. \fullcite{andreas2016modular} use policy sketches, which are annotated sequences of subgoals, in order to learn a hierarchical policy that can generalize to new goals. \fullciteA{oh2017zero} investigate zero-shot transfer for instruction following tasks, aiming to generalize to unseen instructions in the same domain. The main departure of our work compared to these is in the use of environment descriptions for generalization across domains rather than generalizing across text instructions.}

Perhaps closest in spirit to our hypothesis is the recent work by~\fullcite{harrison2017guiding}. Their approach makes use of paired instances of text descriptions and state-action information from human gameplay to learn a machine translation model. This model is incorporated into a policy shaping algorithm to better guide agent exploration. Although the motivation of language-guided control policies is similar to ours, their work considers transfer across tasks in a single domain, and requires human demonstrations to learn a policy.

\textcolor{black}{
\subsection{Using Task Features for Transfer}
The idea of using task features/dictionaries for zero-shot generalization has been explored previously in the context of image classification. \fullciteA{kodirov2015unsupervised} learn a joint feature embedding space between domains and also induce the corresponding projections onto this space from different class labels. 
\fullciteA{kolouri2018joint} learn a joint dictionary across visual features and class attributes using sparse coding techniques. \fullciteA{romera2015embarrassingly} model the relationship between input features, task attributes and classes as a linear model to achieve efficient yet simple zero-shot transfer for classification. \fullciteA{socher2013zero} learn a joint semantic representation space for images and associated words to perform zero-shot transfer.}

\textcolor{black}{
Task descriptors have also been explored in zero-shot generalization for control policies. \fullciteA{sinapov2015learning} use task meta-data as features to learn a mapping between pairs of tasks. This mapping is then used to select the most relevant source task to transfer a policy from. \fullciteA{isele2016using} build on the ELLA framework~\fullcite{ruvolo2013ella,ammar2014online}, and their key idea is to maintain two shared linear bases across tasks -- one for the policy ($L$) and the other for task descriptors ($D$). Once these bases are learned on a set of source tasks, they can be used to predict policy parameters for a new task given its corresponding descriptor. 
% The training scheme is similar to Actor-mimic scheme~\cite{parisotto2016actor} -- for each task, an expert policy is trained separately and then distilled into policy parameters dependent on the shared basis $L$. 
In these lines of work, the task features were either manually engineered or directly taken from the underlying system parameters defining the dynamics of the environment. In contrast, our framework only requires access to crowd-sourced textual descriptions, alleviating the need for expert domain knowledge.}





% A major difference in our work is that we utilize natural language descriptions of different environments to bootstrap transfer, requiring less exploration in the new task.

% using a policy distillation~\cite{parisotto2016actor,rusu2016progressive,yin2017knowledge} or selective attention over expert networks learnt in the source tasks~\cite{rajendran20172t}. Though these approaches provide some benefits, they still suffer from the requirement of efficiently exploring the new environment to learn how to transfer their existing policies. In contrast, we utilize natural language descriptions of different environments to bootstrap transfer, leading to more focused exploration in the target task. 


% Describe amn in detail




   % This file has Related Work section
%\mySection{Related Works and Discussion}{}
\label{chap3:sec:discussion}

In this section we briefly discuss the similarities and differences of the model presented in this chapter, comparing it with some related work presented earlier (Chapter \ref{chap1:artifact-centric-bpm}). We will mention a few related studies and discuss directly; a more formal comparative study using qualitative and quantitative metrics should be the subject of future work.

Hull et al. \citeyearpar{hull2009facilitating} provide an interoperation framework in which, data are hosted on central infrastructures named \textit{artifact-centric hubs}. As in the work presented in this chapter, they propose mechanisms (including user views) for controlling access to these data. Compared to choreography-like approach as the one presented in this chapter, their settings has the advantage of providing a conceptual rendezvous point to exchange status information. The same purpose can be replicated in this chapter's approach by introducing a new type of agent called "\textit{monitor}", which will serve as a rendezvous point; the behaviour of the agents will therefore have to be slightly adapted to take into account the monitor and to preserve as much as possible the autonomy of agents.

Lohmann and Wolf \citeyearpar{lohmann2010artifact} abandon the concept of having a single artifact hub \cite{hull2009facilitating} and they introduce the idea of having several agents which operate on artifacts. Some of those artifacts are mobile; thus, the authors provide a systematic approach for modelling artifact location and its impact on the accessibility of actions using a Petri net. Even though we also manipulate mobile artifacts, we do not model artifact location; rather, our agents are equipped with capabilities that allow them to manipulate the artifacts appropriately (taking into account their location). Moreover, our approach considers that artifacts can not be remotely accessed, this increases the autonomy of agents.

The process design approach presented in this chapter, has some conceptual similarities with the concept of \textit{proclets} proposed by Wil M. P. van der Aalst et al. \citeyearpar{van2001proclets, van2009workflow}: they both split the process when designing it. In the model presented in this chapter, the process is split into execution scenarios and its specification consists in the diagramming of each of them. Proclets \cite{van2001proclets, van2009workflow} uses the concept of \textit{proclet-class} to model different levels of granularity and cardinality of processes. Additionally, proclets act like agents and are autonomous enough to decide how to interact with each other.

The model presented in this chapter uses an attributed grammar as its mathematical foundation. This is also the case of the AWGAG model by Badouel et al. \citeyearpar{badouel14, badouel2015active}. However, their model puts stress on modelling process data and users as first class citizens and it is designed for Adaptive Case Management.

To summarise, the proposed approach in this chapter allows the modelling and decentralized execution of administrative processes using autonomous agents. In it, process management is very simply done in two steps. The designer only needs to focus on modelling the artifacts in the form of task trees and the rest is easily deduced. Moreover, we propose a simple but powerful mechanism for securing data based on the notion of accreditation; this mechanism is perfectly composed with that of artifacts. The main strengths of our model are therefore : 
\begin{itemize}
	\item The simplicity of its syntax (process specification language), which moreover (well helped by the accreditation model), is suitable for administrative processes;
	\item The simplicity of its execution model; the latter is very close to the blockchain's execution model \cite{hull2017blockchain, mendling2018blockchains}. On condition of a formal study, the latter could possess the same qualities (fault tolerance, distributivity, security, peer autonomy, etc.) that emanate from the blockchain;
	\item Its formal character, which makes it verifiable using appropriate mathematical tools;
	\item The conformity of its execution model with the agent paradigm and service technology.
\end{itemize}
In view of all these benefits, we can say that the objectives set for this thesis have indeed been achieved. However, the proposed model is perfectible. For example, it can be modified to permit agents to respond incrementally to incoming requests as soon as any prefix of the extension of a bud is produced. This makes it possible to avoid the situation observed on figure \ref{chap3:fig:execution-figure-4} where the associated editor is informed of the evolution of the subtree resulting from $C$ only when this one is closed. All the criticisms we can make of the proposed model in particular, and of this thesis in general, have been introduced in the general conclusion (page \pageref{chap5:general-conclusion}) of this manuscript.



    % This file has Discussion section
% \vspace{-0.5em}
\section{Conclusion}
% \vspace{-0.5em}
Recent advances in multimodal single-cell technology have enabled the simultaneous profiling of the transcriptome alongside other cellular modalities, leading to an increase in the availability of multimodal single-cell data. In this paper, we present \method{}, a multimodal transformer model for single-cell surface protein abundance from gene expression measurements. We combined the data with prior biological interaction knowledge from the STRING database into a richly connected heterogeneous graph and leveraged the transformer architectures to learn an accurate mapping between gene expression and surface protein abundance. Remarkably, \method{} achieves superior and more stable performance than other baselines on both 2021 and 2022 NeurIPS single-cell datasets.

\noindent\textbf{Future Work.}
% Our work is an extension of the model we implemented in the NeurIPS 2022 competition. 
Our framework of multimodal transformers with the cross-modality heterogeneous graph goes far beyond the specific downstream task of modality prediction, and there are lots of potentials to be further explored. Our graph contains three types of nodes. While the cell embeddings are used for predictions, the remaining protein embeddings and gene embeddings may be further interpreted for other tasks. The similarities between proteins may show data-specific protein-protein relationships, while the attention matrix of the gene transformer may help to identify marker genes of each cell type. Additionally, we may achieve gene interaction prediction using the attention mechanism.
% under adequate regulations. 
% We expect \method{} to be capable of much more than just modality prediction. Note that currently, we fuse information from different transformers with message-passing GNNs. 
To extend more on transformers, a potential next step is implementing cross-attention cross-modalities. Ideally, all three types of nodes, namely genes, proteins, and cells, would be jointly modeled using a large transformer that includes specific regulations for each modality. 

% insight of protein and gene embedding (diff task)

% all in one transformer

% \noindent\textbf{Limitations and future work}
% Despite the noticeable performance improvement by utilizing transformers with the cross-modality heterogeneous graph, there are still bottlenecks in the current settings. To begin with, we noticed that the performance variations of all methods are consistently higher in the ``CITE'' dataset compared to the ``GEX2ADT'' dataset. We hypothesized that the increased variability in ``CITE'' was due to both less number of training samples (43k vs. 66k cells) and a significantly more number of testing samples used (28k vs. 1k cells). One straightforward solution to alleviate the high variation issue is to include more training samples, which is not always possible given the training data availability. Nevertheless, publicly available single-cell datasets have been accumulated over the past decades and are still being collected on an ever-increasing scale. Taking advantage of these large-scale atlases is the key to a more stable and well-performing model, as some of the intra-cell variations could be common across different datasets. For example, reference-based methods are commonly used to identify the cell identity of a single cell, or cell-type compositions of a mixture of cells. (other examples for pretrained, e.g., scbert)


%\noindent\textbf{Future work.}
% Our work is an extension of the model we implemented in the NeurIPS 2022 competition. Now our framework of multimodal transformers with the cross-modality heterogeneous graph goes far beyond the specific downstream task of modality prediction, and there are lots of potentials to be further explored. Our graph contains three types of nodes. while the cell embeddings are used for predictions, the remaining protein embeddings and gene embeddings may be further interpreted for other tasks. The similarities between proteins may show data-specific protein-protein relationships, while the attention matrix of the gene transformer may help to identify marker genes of each cell type. Additionally, we may achieve gene interaction prediction using the attention mechanism under adequate regulations. We expect \method{} to be capable of much more than just modality prediction. Note that currently, we fuse information from different transformers with message-passing GNNs. To extend more on transformers, a potential next step is implementing cross-attention cross-modalities. Ideally, all three types of nodes, namely genes, proteins, and cells, would be jointly modeled using a large transformer that includes specific regulations for each modality. The self-attention within each modality would reconstruct the prior interaction network, while the cross-attention between modalities would be supervised by the data observations. Then, The attention matrix will provide insights into all the internal interactions and cross-relationships. With the linearized transformer, this idea would be both practical and versatile.

% \begin{acks}
% This research is supported by the National Science Foundation (NSF) and Johnson \& Johnson.
% \end{acks}    % This file has Conclusion and Future Work sections


\bibliographystyle{ACM-Reference-Format}
\bibliography{TACO2017}

\appendix
\section{Schedule Independent Memory Allocation}
\label{sec:sima}

We also address a memory-based limitation of polyhedral compilation
tools.  It is well known that in any parallelization (of any program), it is
essential to respect (only) the \emph{true} or flow dependences.  Other
(memory-based) dependences can be ignored if one can re-allocate memory.  In
practice, this is limited by the fact that the associated memory expansion may
be prohibitively expensive, and there has been work on mitigating this
expansion~\cite{vasilache-impact12, lefebvre-feautrier-pc98, sanjay-europar96,
  sanjay-toplas00}.  We propose a novel yet simple \emph{schedule-independent}
memory allocation strategy.  Our work also generalizes polyhedral compilation
by enabling polyhedral tools to use alternate, \emph{hybrid} schedules
consisting of affine loops for certain parts of the iteration space and
cache-oblivious divide-and-conquer schedules for others.


\subsection{Background}

In this section, we introduce the necessary background of our work. We first
give a brief description of the polyhedral representation of programs, and the
general flow of a polyhedral compiler.  Then, we discuss the legality of
tiling, which is related to the input of our code generator.

\begin{figure*}[tb]
  \centering %\vspace*{6cm}
  \includegraphics[scale=0.6]{PolyCompiler}
  \caption{\small{Polyhedral Compilation: the Polyhedral Reduced Dependence
      (hyper) Graph (PRDG) serves as the intermediate representation.
      Piecewise Quasi-Affine Functions (PQAFs) describe transformations.}}
  \label{fig:compiler}
\end{figure*}

\subsubsection{Polyhedral Compilation and Representation}

Figure~\ref{fig:compiler} shows the flow of polyhedral compilation.  First,
dependence analysis of an input program (or a ``polyhedral section'' thereof)
produces an intermediate representation (IR) in the form of~\cite{DRV-sched00}
a \emph{Polyhedral Reduced Dependence (hyper) Graph} (PRDG).  Various analyses
are performed on the PRDG to choose a number of mappings in the form of
\emph{Piecewise Quasi-Affine Functions} (PQAFs) that specify the schedule as a
set of \emph{multi-dimensional} vectors.  The PQAFs come with annotations to
indicate whether each dimension is sequential or parallel, and also whether it
is part of a \emph{tilable band}, i.e., whether tiling this band of dimensions
is legal.  The transformations may be applied to the PRDG iteratively, and
(eventually) the PRDG and QLAF are provided to a code-generator that produces
code for various targets.

% specify which dimensions are sequential, which dimensions are (and
% implicitly, also the algorithm~\cite{uday-pldi08} to get tiling hyperplanes
% and tilable dimensions for each statement which is called \emph{tilable
% band}.  Then we transform the program using tiling hyperplanes.  This
% results in a program where hyper-rectangular tiles are legal and the
% wavefront parallel execution order is a legal schedule for executing tiles.
% We also provide schedule independent memory allocations for all the
% variables.  Finally we generate code which traverse the iteration space of
% the program in divide and conquer order.  by post processing the AST of
% tiled code generated by DTiler The following section presents the schedule
% independent memory allocation for affine programs.

One of the strengths of the polyhedral model is that a parametric program may
be concisely represented with a PRDG with finite number of nodes (statements)
and edges (dependences).  The potentially unbounded sets of instances of a
statement are represented in abstract forms of integer sets, called
\emph{domains}, and dependences between them as affine functions (or
relations, which are viewed as a set-valued function) over these statement
domains.  Indeed, every edge, $e$ from node $v$ to $w$, in the PRDG is
annotated with two objects: (i) a domain, $D_e$ specifying the (subset of) the
domain, $D_v$ of its source node, where the dependence occurs, and (ii) the
affine function, $f$, such that for any point $z\in D_e$, the (set of)
point(s) in $D_w$ on which it depends is given by $f(z)$.  $D_e$ is called the
context of the edge, and $f$ is its dependence function.  We also use the
notation $f(D_e)$ to denote the set valued image of $D_e$ by $f$.

An affine function $\mathbb{Z}^n \rightarrow \mathbb{Z}^m$ may be expressed as
$f(x) = A\vec{x} + \vec{b}$, where $\vec{x}$, function domain, is an integer
vector of size $n$; $A$, linear part, is an $n\times m$ matrix; and $\vec{b}$,
constant part, is an integer vector of size $m$.  A dependence is said to be
uniform if the dependence function is only a constant offset, i.e., when the
linear part $A$ is the identity.

%  is used to get the  tilable dimensions....  The parallelism that can be
%  explored using tiles is assumed to be wavefront....??  However, we modify
%  the execution order of tiles and schedule them in a divide-and-conquer
%  fashion.... Updating the memory allocation schemes becomes important when
%  we change the order of execution of tiles.... The following Section talks
%  about Memory Allocation...

\subsubsection{Legality of Tiling}

Tiling is a well-known loop transformation for partitioning computations into
smaller, atomic (all inputs to a tile can be computed before its execution),
units called tiles~\cite{irigoin-popl88, Wolf91tiling}.  The natural legality
condition is that the dependences across tiles do not create a cycle.  In
compilers, this condition is typically expressed as fully permutability (i.e.,
dependences are non-negative direction vectors), which is a sufficient
condition.  Our transformation for cache oblivious tiling takes as inputs a
loop nest that is fully permutable.  For polyhedral programs, scheduling
techniques to expose such loop nests are available~\cite{uday-pldi08}.


\subsection{Memory Allocation}

\begin{figure}[tb]
  \centering % \vspace*{4cm}
{\small\begin{lstlisting}
for (i = 0; i < N; i++){
  S0:  X[0,i] = A[i]; } // Initialize
for (t = 1; t <= 2*N; t++){ //Note: ub is even
  for (i = 1; i < N-1; i++){
    S1: X[t%2][i] = f(X[(t-1)%2][i-1],
             X[(t-1)%2][i], X[(t-1)%2][i+1],
             X[(t-1)%2][0]);
    }
  S2: X[t%2][0] = g(X[(t-1)%2][0]); 
  S3: X[t%2][N-1] = g(X[(t-1)%2][N-1]);
}
for (i = 0; i < N; i++){
  S4: Aout[i] = X[0,i]; } // Output copy
\end{lstlisting}
}
\caption{\small{Neither Pochoir nor Autogen can handle the computation
    performed by this simple loop.  Moreover, it has a memory based dependence
    that prevents polyhedral compilers like Pluto from tiling both dimensions.
    However, the true dependences of the program admit a tilable schedule, but
    at the potentical cost of $O(N^2)$ memory.  Our scheme reduces this to
    $O(N)$}}
\label{fig:motiv}
\end{figure}


In this section, we first describe how memory based dependences prevent
tiling, using our motivating example (Fig.~\ref{fig:motiv}), and show that
simply ignoring these (false) dependences would lead to memory explosion.
After formulating our problem, we next propose a simple, schedule independent
memory allocation scheme that resolves it.

Consider the statement $\mathrm{S1}$, and note that its domain, $D_1$ is the
polyhedral set, $\{t,i~|~ 1\leq t\leq 2N \wedge 1\leq i\leq N-1 \}$.
$\mathrm{S1}$ has four true dependences (for points sufficiently far from the
boundaries), three of which are $\mathrm{S1}[t-1, i-1]$, $\mathrm{S1}[t-1, i]$
and $\mathrm{S1}[t-1, i+1]$, the typical, 1D-Jacobi stencil dependences, and
the fourth one is $\mathrm{S2}[t-1, 0]$, which is a truly affine dependence on
the most recent writer to the memory location $\mathtt{X[(t-1)\%2,0]}$ when
the statement $\langle \mathrm{S1}, [t, i]\rangle$ is being executed.  All
these dependences are captured as edges with affine \emph{functions} in the
PRDG.  In addition, there is a memory based dependence, that we must also
respect.  Consider statement $\mathrm{S2}$, whose domain, $D_2 = \{t~|~ 1\leq
t\leq 2N\}$ is just one dimensional.  The $t$-th instance of S2 \emph{(over)
  writes} $\mathtt{X[t\%2, 0]}$, therefore all computations that read the
previous value must be executed before it.  In this sense, $\mathrm{S2}[t]$
``depends on'' the set $\mathrm{S1}[t,i]$, for all $1\leq i\leq N-1$.  This
dependence (which is a \emph{relation} rather than a function) is captured by
another a special edge in the PRDG.

The only schedule that respects all these dependences is the family of lines
parallel to the $t$ axis (provided all iterations of S1 are done first).
Although this has maximal parallelism, it has very poor locality.  Note that
the Pluto scheduler does not seek maximal parallelism, but rather, to maximize
the \emph{number of linearly independent tiling hyperplanes}.  Unfortunately,
the $t=\mathrm{const}$ is the only legal tiling hyperplane for this set of
dependences, and the tilable band obtained by Pluto is only 1-dimensional.

What if we did not have the memory-based dependences, i.e., what if we ignored
the memory allocation of the original program, and stored each computed value
in a distinct memory location?  In this case, there would be no memory based
dependences, and we can indeed find another family of (actually, infinitely
many) legal tiling hyperplanes: say, the lines $i+t=\mathrm{const}$.  As a
result, if we use the mapping $(t,i) \mapsto (t,i+t)$ as our ``schedule,'' the
new loops in the transformed program would be fully permutable, and could be
legally tiled.

Thus, the problem we seek to solve is: \emph{how to avoid memory based
  dependences, but without the cost of memory expansion that it seems to
  imply}.

Memory allocation for polyhedral programs is a well studied problem, and there
are two main approaches.  One either does memory allocation after the schedule is
chosen~\cite{sanjay-europar96, degreef-memory97, lefebvre-feautrier-pc98,
  sanjay-toplas00, darte-lattice05, vasilache-impact12,
  bhaskaracharya-toplas16, bhaskaracharya-popl16} since it often leads to a
smaller memory footprint, or else uses a \emph{schedule independent} memory
allocation, based on the so called \emph{universal occupancy vectors} (UOV).
This problem is solved when the program has \emph{uniform dependences}, i.e.,
when each dependence can be described by a \emph{constant vector}, and for some
simple extensions of this~\cite{strout-etal-asplos98, sanjay-memory-2011}.

It is important to note that tiling actually modifies a schedule: the so
called, ``schedule dimensions'' become fully permutable loops, and indeed,
these loops \emph{are actually permuted} in the generated tiled code.  So,
when a \emph{tiling schedule} specified by a family of $d$ tiling hyperplanes
is finally implemented by the generated code, the actual time-stamps are not
really $d$-dimensional vectors, but rather $2d$-dimensional ones obtained as
some complicated function of these indices.  Furthermore, we will see when we
generate cache-oblivious tiled codes, these tilable loops will actually be
visited in the divide-and-conquer order of execution, as required by COT.  As
a result, finding a memory map that takes into account such a rather
complicated final schedule is a tricky problem.  We therefore seek and propose
schedule-independent memory allocations.

The intuition behind our solution is (deceptively) simple, and we first
illustrate it on our motivating example (Fig.~\ref{fig:motiv}).  Rather than
the so-called ``single assignment'' program for the entire iteration space of
the program (i.e., full memory expansion), could we find lower-dimensional
subsets, such that a single assignment memory for only these subsets is
sufficient?  A careful examination of the code reveals that the memory based
dependences arise due to statement S2, and its domain is only 1-dimensional.
So we store the results of this statement into an auxiliary array, \texttt{Y},
and modify the program so that the fourth dependence simply reads
\texttt{Y[t-1]}, rather than \texttt{X[(t-1)\%2,0]}.  For the variable,
\texttt{X}, we use the old $(t,i) \mapsto (t\%2,i)$ memory allocation that was
used in the original code.  This results in $4N$ memory, which is a polynomial
degree better than quadratic.  Of course, the challenge is how to discover
this automatically.

% PLUTO schedule give a schedule and a tiling band.  A point in a tiling band
% corresponds to a particular tile.  All points within a tile execute
% sequentially.  However, the tiles can be executed in parallel.  Consider the
% modified Jacobi-1D stencil (CHANGE THIS TO BE CONSISTENT WITH RUNNING
% EXAMPLE).  The Figure \textbf{ABC} shows the iteration space of Jacobi-1D
% example (CHANGE THIS TO BE CONSISTENT WITH RUNNING EXAMPLE).  The blue bars
% show the inputs and outputs of the program corresponding to statements
% $S_{0}$ and $S_{4}$ respectively.  The memory allocation scheme used
% initially is modulo memory allocation.  However, this modulo memory
% allocation is not affine and hence PLUTO is unable to find a schedule. We
% make this memory map to not to use modulo but use previous and current
% instead. Even then, PLUTO is unable to find a schedule such that all
% dimensions are tilable.  PLUTO will decide to tile this program with single
% assignment memory.  Identity memory allocation is known to be overkill and
% lead to inefficient codes.  Therefore, we need a schedule independent memory
% allocation scheme which guarantees that the given memory allocation is both
% legal as well as optimal for any given schedule.

% The problem of schedule independent memory allocation for Polyhedral
% programs is a partially solved problem. [Strout et al] presented an
% algorithm for schedule independent memory allocation for a class of programs
% that have only uniform dependences.  A set of programs with uniform
% dependences is a proper subset of the set of programs with affine
% dependences, the class of Polyhedral programs.  Our recursive
% divide-and-conquer code generator is applicable to all Polyhedral programs
% in general, including the ones that have truly affine non-uniform
% dependences. We therefore present a novel scheme for schedule independent
% memory allocation for all affine programs.

We now outline how this is done.  At a high level, our algorithm takes a PRDG
as input, applies some (piecewise) affine transformations to it, and outputs
the transformed PRDG together with a separate memory map for each node in the
transformed PRDG.  More specifically, it works as follows.

\begin{itemize}
\item \emph{Preprocessing.}  For each edge, $e$, in the PRDG, with context
  $D_e$, and function, $f$, we first identify whether $f$ is \emph{uniform in
    context} in the sense that, for all points, $z\in D_e$, the value of
  $z-f(z)$ is a constant vector, independent of $z$.

  For example, consider a dependence function, $(i,j) \mapsto (i-1,i-1)$
  which, maps any point $[i,j]$ in the plane to a point on the diagonal, and
  is clearly not uniform.  However, what if $D_e = \{i,j~|~i=j-1\}$?  With
  this contextual information, the dependence is actually uniform: $(i, j)
  \mapsto (i-1, j-2)$.

  All edges/dependences that are neither uniform to begin with, not uniform in
  context, are marked as \emph{truly affine}.
\item \emph{Affine Split.}  For every node, $v$, in the PRDG that has at least
  one truly affine edge $e$ incident on it, we create a new node, $v'$.  Its
  domain $D_{v'}$ is the union of $f(D_e)$ of all such incident edges.

  The edges in the PRDG are modified as follows.  All the truly affine edges
  that were incident on $v$ are now made incident on $v'$; and $v'$ has a
  single outgoing edge $e'$, annotated with $\langle D_{v'}, I \rangle$ (its
  dependence function is the identity map) and whose destination is $v$.

  It is easy to see that we have not changed the program semantics.  In
  effect, we have simply copied the value of every point in $D_v$ that was the
  target of any truly affine dependence over to a new variable $v'$, and
  ``diverted'' all the truly affine edges that used to be incident on $v$ over
  to $v'$.  Moreover, since the identity function is uniform by definition, all
  edges incident on $v$ are now either uniform, or uniform in context.
\item We now use existing UOV based methods~\cite{strout-etal-asplos98,
    sanjay-memory-2011} to choose a schedule-independent memory allocation for
  all the original nodes in the PRDG, and a \emph{single-assignment} memory
  allocation for all the newly introduced variables.
\end{itemize}

The key insight into why this leads to significant memory savings, is the fact
that in all polyhedral programs that we encountered, truly affine dependences
are almost always \emph{rank deficient}, i.e., are many-to-one mappings from
the consumer index points to the producers.  The only exceptions are either
pathological programs, or programs that do multi-dimensional data
reorganizations via bijections (e.g., matrix transpose, tensor permutations,
etc.) where here is no scope nor need to reduce the total memory footprint.
As a result, $f(D_e)$ is almost always a lower dimensional polyhedron, and
requires significantly less memory, even when stored supposedly inefficiently.


\subsection{Related Work}

Memory allocation for polyhedral programs is a well studied problem for almost
two decades.  DeGreef and Cathoor~\cite{degreef-memory97} tackled the problem
of sharing the memory across multiple arrays in the program.the so called
inet-array memory reuse problem, and proposed an ILP based solution.  Wilde
and Rajopadhye, in dealing with an intrinsically memory-inefficient functional
language Alpha~\cite{mauras1989thesis} (one can think of this as a program after
full expansion) first addressed the memory reuse for points of an iteration
space~\cite{sanjay-europar96}.  They gave necessary and sufficient conditions
for the legality of a memory allocation fucntion, which they allowed to be
``in any direction.''  but they did not provide any insight into how to choose
the mapping.  Lefebvre and Feautrier~\cite{lefebvre-feautrier-pc98} on the
other hand, considered only canonic projections, combined with a modulo
factor, but showed how to choose the mapping optimally.  Later, Quiller\'e and
Rajopadhye~\cite{sanjay-toplas00} revisited multiprojections, extended them to
quasi-affine functions, and proved a tight bound on the number of dimensions
of reuse.  They also showed that cananic projections with modulo factors was
sometimes a constant factor better, and sometimes a constant factor worse.
Darte at al.~\cite{darte-lattice05} took a fresh and elegant approach to the
problem, and formulated the conditions for legal memory allocations by
defining the \emph{conflict set}.  This led to techniques for choosing
provably optimal memory allocations, initially for non-parameterized iteration
spaces, and recently in the context of FPGA acelerators, for parametrically
tiled spaces~\cite{darte2014parametric, darte2016extended}.  Vasilache et
al.~\cite{vasilache-impact12} developed a tool to combine the scheduling and
limiting memory expansion using an ILP formulation, implemented in the
R-Stream compiler.  Recently, Bhaskaracharya et
al.~\cite{bhaskaracharya-toplas16} developed methods to optimally choose
quasi-affine memory allocations, and showed how they are beneficial for tiled
codes, especialy with live-out data.  Furthermore, they also
showed~\cite{bhaskaracharya-popl16} how to combine iner-and intra array reuse
in a unifying framework.

The other \emph{schedule independent} memory allocation was pioneered by
Strout et al.~\cite{strout-etal-asplos98}.  Here, the memory allocation is
chosen based only on the dependences, and is guaranteed to be legal,
regardless of the schedule.  This problem is solved when the program has
\emph{uniform dependences}, i.e., when each dependence can be descibed by a
\emph{constant vector}, and for some simple extensions of
this~\cite{strout-etal-asplos98, sanjay-memory-2011}.

Thies at al.~\cite{thies-pldi02} have also formulated the problem of
simultaneously choosing the schedule and memory allocation as a combined
optimization problem.
% \begin{algorithm}[h]
%   \mbox{} Input : PRDG
  
%     Output : Transformed PRDG and Memory Map
%     \begin{enumerate}
%     \item Recognize truly non-uniform dependence edges
  
%       For each node $X_{i}$ with one or more incoming non-uniform affine
%       edge(s):
%       \begin{itemize}
%       \item Create new node $X_{i}^{new}$ by applying ``Affine Split"
%         transformation
    	
%         Assertion: $X_{i} = X_{i}^{old} + X_{i}^{new}$

%       \end{itemize}
%     \item Find universal occupancy vector $(UOV)$ for all nodes with uniform
%       dependences
%     \item Construct memory mapping function for all nodes except
%       $X_{i}^{new}$ using $UOV$
% 	\item Use identity memory map for $X_{i}^{new}$ nodes
% 	\item \textbf{(SANJAY verify 4 , 5)} Apply change of basis to reduce
%    number of dimensions of the domain of new node
%  \end{enumerate}
%  \caption{SIMA: Schedule Independent Memory Allocation for Polyhedral
%  Programs}
%  \label{alg:sima}
% \end{algorithm}

% Local Variables: ***
% TeX-master: "PACT17.tex" ***
% fill-column: 78 ***
% End: ***


\end{document}
