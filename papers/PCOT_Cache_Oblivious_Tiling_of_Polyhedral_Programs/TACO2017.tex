\documentclass[sigconf, authordraft=false]{acmart}
%\documentclass[format=acmsmall, review=true, authordraft=true]{acmart}
%\documentclass[sigplan,10pt,review,anonymous]{acmart}\settopmatter{printfolios=true,printccs=false,printacmref=false}
%\documentclass[sigplan,10pt,review,anonymous,authordraft]{acmart}\settopmatter{printfolios=true,printccs=false,printacmref=false}

\usepackage{booktabs} % For formal tables

\usepackage{subcaption}

\usepackage[ruled]{algorithm2e} % For algorithms
\renewcommand{\algorithmcfname}{ALGORITHM}
\SetAlFnt{\small}
\SetAlCapFnt{\small}
\SetAlCapNameFnt{\small}
\SetAlCapHSkip{0pt}
\IncMargin{-\parindent}

%mypackages
\usepackage{listings}
\usepackage{enumitem}
\usepackage{siunitx}
\usepackage{multirow}

%my commands
\setlist[enumerate,1]{%
  label=\arabic*.,
}

\newlist{inlinelist}{enumerate*}{1}
\setlist*[inlinelist,1]{%
  label=(\roman*),
}

\newcommand{\FIXME}[1]{{\textcolor{red}{{\tt{FIXME:}}\,\,#1}}}

%Décommenter et commenter en dessous pour supprimer les commentaires
\newcommand{\remarkauthor}[2]{{\textcolor{red}{{\tt{(#1)}}\emph{~#2}}}}
\newcommand{\tomoRemark}[1]{{\remarkauthor{tomo}{#1}}}


%\DeclareRobustCommand*\cal{\@fontswitch\relax\mathcal}

% Metadata Information
%\acmJournal{TACO}
%\acmVolume{9}
%\acmNumber{4}
%\acmArticle{39}
%\acmYear{2017}
%\acmMonth{3}
%\copyrightyear{2009}
%\acmArticleSeq{9}

% Copyright
%\setcopyright{acmcopyright}
%\setcopyright{acmlicensed}
\setcopyright{rightsretained}
%\setcopyright{usgov}
%\setcopyright{usgovmixed}
%\setcopyright{cagov}
%\setcopyright{cagovmixed}

% DOI
%\acmDOI{0000001.0000001}

% Paper history
%\received{February 2007}
%\received[revised]{March 2009}
%\received[accepted]{June 2009}


% Document starts
\begin{document}
% Title portion. Note the short title for running heads 
\title{PCOT: Cache Oblivious Tiling
of Polyhedral Programs}
%\footnote{New Paper, Not an Extension of a Conference Paper} 
\author{Waruna Ranasinghe}
\author{Nirmal Prajapati}
%\orcid{1234-5678-9012-3456}
\affiliation{%
  \institution{Colorado State University}
  \department{Department of Computer Science}
%  \streetaddress{104 Jamestown Rd}
  \city{Fort Collins}
  \state{CO}
  \postcode{80523}
  \country{USA}}
\author{Tomofumi Yuki}
\affiliation{%
  \institution{INRIA}
  \city{Rennes}
  \country{France}}
\author{Sanjay Rajopadhye}
\affiliation{%
  \institution{Colorado State University}
  \department{Department of Computer Science}
  \city{Fort Collins}
  \state{CO}
  \postcode{80523}
  \country{USA}}
\begin{abstract}
  This paper studies two variants of tiling: iteration space tiling (or loop
  blocking) and cache oblivious methods that recursively split the iteration
  space with divide-and-conquer. The key question to answer is when
  we should be using one over the other. The answer to this question is complicated
  for modern architecture due to a number of reasons.

  In this paper, we present a detailed empirical study to answer this question for a range of kernels
  that fit the polyhedral model. Our study is based on a generalized cache oblivious code generator that
  support this class, which is a superset of those supported by existing tools.
  The conclusion is that cache oblivious code is most useful when the aim is to
  have reduced off-chip memory accesses, e.g., lower energy,
  albeit certain situations that diminish its effectiveness exist. 
\end{abstract}
%
%  This paper marries two important ideas: the \textbf{polyhedral model}, a
%  framework to describe, analyze, transform and compile a class of
%  \textbf{compute-} and \textbf{data-intensive} regular programs, and
%  \textbf{cache-oblivious algorithms/tiling,} a strategy for specifying and/or
%  deriving divide-and-conquer programs that provide provable bounds on the
%  number of cache misses.  Specifically, we present a compiler framework that
%  generates, from a polyhedral input specification, codes with the
%  divide-and-conquer schedule of cache oblivious algorithms.  This is a strict
%  generalization of earlier cache oblivious code generators to a richer class
%  of programs, and of polyhedral compilers to enable them to use a
%  divide-and-conquer schedule, rather than multidimensional affine functions.
%

%
% The code below should be generated by the tool at
% http://dl.acm.org/ccs.cfm
% Please copy and paste the code instead of the example below. 
%
\begin{CCSXML}
<ccs2012>
<concept>
<concept_id>10003752.10003809.10010170.10010171</concept_id>
<concept_desc>Theory of computation~Shared memory algorithms</concept_desc>
<concept_significance>500</concept_significance>
</concept>
<concept>
<concept_id>10010147.10010169.10010170.10010171</concept_id>
<concept_desc>Computing methodologies~Shared memory algorithms</concept_desc>
<concept_significance>500</concept_significance>
</concept>
</ccs2012>
\end{CCSXML}

\ccsdesc[500]{Theory of computation~Shared memory algorithms}
\ccsdesc[500]{Computing methodologies~Shared memory algorithms}

%
% End generated code
%


\keywords{Tiling, Cache Oblivious, Polyhedral Model}


%\thanks{This work is supported by TODO}

%  Author's addresses: G. Zhou, Computer Science Department, College of
%  William and Mary; Y. Wu {and} J. A. Stankovic, Computer Science
%  Department, University of Virginia; T. Yan, Eaton Innovation Center;
%  T. He, Computer Science Department, University of Minnesota; C.
%  Huang, Google; T. F. Abdelzaher, (Current address) NASA Ames
%  Research Center, Moffett Field, California 94035.}


\maketitle

% The default list of authors is too long for headers}
%\renewcommand{\shortauthors}{W. Ranasinghe et al.}

%
\section{Introduction}
% 0) how task-oriented dialogue is related to IR
% Task-oriented dialogue systems offer a natural and effective interface for users to complete specific tasks, such as booking flight tickets and providing intelligent customer services. Such systems often receive and process user utterances, and respond to users interactively. 
Task-oriented dialogue systems offer a natural and effective interface for users to seek information and complete complex tasks in an interactive manner. Such systems often collect users' preferences in the course of dialogue before issuing the final query to the knowledge base (such as booking a flight ticket). There are also some works \cite{hixon2015learning, saha2018complex} viewing the task-oriented dialogue task as a context-aware, multi-turn question answering (QA) task in which a user can interact with the system in multi-turn contexts and the system also has access to the knowledge base.

% 1)DM定义和其重要意义
 Different from open-domain conversational systems which are often modeled in an end-to-end manner, task-oriented dialogue systems are generally composed of several cascaded processes, as shown in Figure \ref{fig:dialog_system}, including natural language understanding (NLU), dialogue management (DM), and natural language generation (NLG).
Dialogue management, which is in charge of selecting actions in response to user inputs, plays a central role in task-oriented dialogue systems \cite{williams2007partially, ge2dialogue}. It takes as input the user intent which is analyzed by NLU, interacts with knowledge base, and decides the next system action. Sometimes NLU and DM can be coupled together as a single module which can be trained end-to-end to read directly from user utterance and produce system action. The system action produced by DM will be translated into a natural language utterance by NLG \cite{wen2015semantically} to interact with users. 
%By affecting the final response through action, DM makes an emphatic effect on a user's impression of the system. Belief state and dialog policy are the two most basic concepts to define a dialog manager \cite{thomson2013statistical}. 


% FIG: 对话系统简图
\begin{figure}
  \includegraphics[width=0.6\textwidth]{fig/dialog_system.png}
  \caption{The processing flow of task-oriented a dialogue system. Natural language understanding (NLU) parses the user utterance and extracts structured semantic information from the utterance, dialogue management receives the semantic information and decides the next dialogue act that the system should take, and natural language generation (NLG) translates the dialogue act to a natural language response. In some cases, NLU and DM can be coupled together as a single module, and the semantic information produced by NLG is often unstructured in this situation, such as the output of neural network.
  }
  \label{fig:dialog_system}
\end{figure}


%% 下面这些段落,跟后面的related work太重复了!!!

%Generally speaking, there are three different mainstream approaches to deal with the dialog management problem: \emph{rule-based} models, \emph{bayesian network} models and \emph{neural} network models.

% 2)rule-based 模型
%As the name implies, \emph{rule-based} models construct dialog manager mainly by hand-crafted rules, with different frameworks.
%\cite{mctear1998modelling} formulated DM as a pre-defined flow-diagram, in which the nodes represent explicit dialogue states and corresponding actions, and the transitions are triggered by different user inputs. This model requires rich human cost and thus can not be easily scaled to large domain. To solve this problem, \cite{goddeau1996form} proposed a slot-based model which redefines dialog state as an aggregation of domain slots and their values which users can talk about, e.g., \emph{cuisine} in a restaurant domain system. For each specific set of filled slots, an action is assigned to give final response. The idea of representing dialog state by slots eases scalability problem. An continuation of this model, the \emph{information state model} \cite{larsson2000information}\cite{bos2003dipper}, expanded the meaning of dialogue state $S_t$ at time $t$ which is the accumulation of information presented up to that time point. Dialog action along with user utterance are used to update dialog state which is then used for dialogue manager to choose next action.
%It is now widely used in industrial applications because of high accuracy. However, rule-based systems are not easy to transfer to another domain or task since the rules are usually tailored to a specific scenario. And they have to exhaust as many patterns as possible, and are sensitive to the errors from upstream modules. Since patterns are rigid, the variety and diversity of language is not well reflected. In spite of its weakness, the basic structure of \emph{rule-based} methods gives enlightenment on how to build dialogue manager.
% However, {\em rule-based} approaches have apparent defects on robustness and domain adaptation, and lack the variety of natural language. The need for hand-crafted rules also makes it expensive to build a {\em rule-based} system \cite{schatzmann2006survey}. 

% paragraph 3 介绍bayesian方法,包括MDP和POMDP。
%Bayesian network approach takes the first step to model the statistical uncertainty in dialogue manager. \cite{levin1997stochastic} propose MDP-based dialgue management model. One basic assumption of the model is that only a portion of user intent and dialog information is observable, thus we have to infer it from limited observations. Another assumption is \emph{Markov assumption}, which assumes that the new state $s^{t+1}$ is depended only on last turn state $s^t$ and system action $a^t$. The notion of \emph{belief state} is introduced which is a distribution over all possible dialog states at each time point. In each dialog turn, the model updates \emph{belief state} rather than a deterministic selection of state. 
%\shortcite{roy2000spoken} proposed a statistical DM framework based on Partially Observable Markov Decision Processes (POMDPs) which are more robust by taking environment observation $o^{t+1}$ into consideration.
%By modeling uncertainty in dialogue, Bayesian network models are more robust on errors produced by ASR and NLU modules. It is also easier to generalize to a new domain by automatic state update and policy learning. However, the states in Bayesian network approaches is hand-crafted which still require rich human cost. And the one-hot representation of state leads to obvious information loss during dialogue.

%%不太直接相关
%As a subtask of DM, dialogue state tracking (DST) which analyzes the state of each turn in dialogues has been extensively studied \cite{williams2010incremental, henderson2013deep, henderson2014word}. Dialogue management, instead of only analyzing at which state the dialogue is, needs to predict the next dialogue action given the history information, 
%thus more complicated than DST.

%\cite{henderson2014word} proposed a RNN-based approach for DST which can be trained end-to-end, and obtained impressive performance.
%Nonetheless, this work didn't resolve the full DM problem, and RNN with single vector state has a lack in representational capability of history information.

% 新加:从long term dependency的角度解释上面的模型为何不能很好地处理对话

% neural network model
%Neural network models has been shown to advance many tasks in Natural Language Learning. Recently, many neural network-based DM approaches are proposed to tackle the above problems. By utilizing the representation learning capability of neural network, dialogue manager can directly take user utterances as inputs rather than NLU results, mitigating the influence of NLU errors. \cite{henderson2013deep, henderson2014word} proposed a word-based dialogue state tracking model, taking ASR results with confidence score as inputs. For each specific slot, an recurrent neural network is used to maintain a distributed representation of state during dialog. 

In order to decide the next action a dialogue system should take, dialogue management, particularly in task-oriented dialogue systems, should deal with the dialogue context information. It needs to access not only local utterances, but also the global information about what has been addressed several turns ago. The global history information, which is often referred to as \emph{dialogue state}, is a key factor in dialogue systems. Based on the dialogue state, the dialogue manager then produces system action according to its policy. The task of dialogue management is sometimes divided into two subtasks, namely {\it dialogue state tracking} which maintains dialogue history information, and {\it dialogue policy} which selects the next system action based on the dialogue state. 

%% Rule-based;Bayesian Network based;Neural network based
Early methods of modeling dialogue management are mostly rule-based, in which the state update and dialogue policy process are manually defined, but these methods did not take into account the probability uncertainty in dialogue. Bayesian network methods \cite{paek2005markov, williams2007partially} formulated dialogue management as a probabilistic graphical model which models the conditional dependency between different states, and each specific state is bound with an action to be taken, but the definition of dialogue state still need manually-crafted rules. Recently, many neural network methods have been proposed for dialogue management due to their capability of semantic representation and automatic feature extraction, and obtain state-of-the-art performance on many dialogue tasks \cite{ge2015dialogue, serban2016building}. More specifically, most neural dialogue models are RNN (Recurrent Neural Network) based which takes as input user utterance and system response at each dialogue turn, and the hidden state of RNN is utilized as the representation of dialogue state \cite{henderson2014word, williams2017hybrid}.

However, despite of the success of RNN on various text modeling tasks, simple RNN is proven to have poor performance on dialogue tasks \cite{williams2017hybrid} due to the single hidden state vector used in RNN and thus the defect of modeling long-range contexts. Hierarchical RNN structures ~\cite{serban2016hierarchical} and memory networks~\cite{weston2014memory,dodge2015evaluating, bordes2016learning} are feasible solutions to this issue, but existing neural models still lack an explicit memorization of the history semantics of the entire dialogue session: the dialogue act types, semantic slots, and the values of the slots are not explicitly processed during the interaction.

Another important issue is to extract semantic information from user utterance when combining NLU and DM together, which is the case in most end-to-end dialogue systems. Such semantic information is critical for dialogue state update. Existing methods either extract information from predefined features (such as POS and NER tags) by heuristic rules \cite{henderson2014word}, or from pretrained word embeddings by neural network encoder \cite{mrkvsic2016neural}.
%Many semantic features, including POS and NER, have been utilized to represent user intent. Pre-trained word embeddings are semantically-favorable and have been proven to be useful in many NLP tasks. \emph{Neural belief tracker} utilized averaged word vectors to build n-gram feature for dialog state tracking. 
However, words in user utterance have different importance for updating dialogue states and predicting the next action, which is not taken into consideration by previous methods. For example, in a user utterance \emph{I want to book a table in Beijing Hotel}, the word \emph{book} apparently contributes more than the word \emph{want} to the user intent. Furthermore, each word contributes differently to different slots, e.g., word \emph{British} is more related to slot \emph{Cuisine} while \emph{north} is more related to \emph{Location}, as shown in Figure \ref{fig:word_attn}.

% 从上述的缺点(long-range dependency)引出memory network。不像之前那样从MC切入。
%Memory network-based model\cite{graves2014neural, weston2014memory, sukhbaatar2015end}, recently achieves state-of-the-art results in many tasks such as machine comprehension, is adept in handling long-range dependencies by an extra memory structure.
%%为什么要从MC这个问题切入呢?我觉得很奇怪,很突兀,直接说现在DM存在什么不足。。。memory是一种途径,已经有一个什么工作,这样是不是更好?
%Inspired by this, here are already some preliminary works \cite{dodge2015evaluating, bordes2016learning} using memory network to solve dialogue problems by formulating the dialogue task into a MC problem using end-to-end memory network (MENN2N) \cite{sukhbaatar2015end}. 
%However, a direct transfer from memory network for MC task into dialogue modeling is inappropriate in that answers in MC tasks are usually simple and are selected from a fixed set while it is impossible to generate a complete answer set in dialogue management.
%Hence a memory-based model specialize in dialogue is needed.

% paragraph 5 提出我们的模型    
To address the above issues, we propose a novel Memory-Augmented Dialogue management model (MAD) which attentively receives user utterances as input and predicts the next dialogue act\footnote{The dialogue act can be translated into a natural language utterance by a language generator, as shown in \cite{wen2015semantically}.
%% cite the EMNLP paper
}. 
Dialogue act is composed of two parts in our model: {\it dialogue act type} and {\it slot-value pairs}, as shown in Table \ref{tab:dialogact} . 
Dialogue act type indicates the intent type such as {\it Query} or {\it Recommendation}, which is a high-level representation of dialogue act. Slot-value pairs denote key elements of a task, and represent the key semantic information supplied by the user during the interaction, which also indicate the state of the dialogue.


We design two memory modules, namely a slot-value memory and an external memory, which can be written (or updated) and read, to enhance the ability of modeling history semantics of dialogues. %Both of the write and read operation applied on the two memories are controlled by a memory controller. 
A memory controller is introduced to control the write and read operations to the two memories.
The slot-value memory explicitly memorizes and updates the values of the semantic slots during interaction. The write to the slot-value memory units, each corresponding to a slot, is implemented by a slot-level attention mechanism. In this manner, the slot-value memory provides an observable and interpretable representation of the dialogue state. The external memory serves as a supplement to the single hidden state of a RNN structure and provides a better capacity to store more historical dialogue information. A complete dialogue act (consisting of dialogue act type and slot-value pairs) for the next interaction is predicted based on the slot-value memory and external memory.



\begin{table}[htbp]
\small
\centering
\begin{tabular}{|c|c|c|c|c|c|}
  \hline
    {\bf Utterance} & \multicolumn{5}{c|}{How about a {\em British} restaurant in {\em north} part of town.}\\
  \hline
    {\bf Dialogue act type}&\multicolumn{5}{c|}{\em Query}\\
  \hline
    {\bf Slot-value pairs}&\multicolumn{5}{c|}{Cuisine={\em British}, Location={\em Paris}}\\
  \hline
    \multirow{2}{*}{\bf Mask} (auxiliary) & Rating & Cuisine & Price & Service & Location\\
  \cline{2-6}
    & 0 & 1 & 0 & 0 & 1\\
  \hline
\end{tabular}
\caption{\label{tab:dialogact}An example of dialogue act for a given utterance. Dialogue act type is a high-level representation of an utterance. Slot-value pairs are the task-specific semantic elements that are mentioned in an utterance.}
\label{appendix-da-example}
\end{table}


%%简单描述一下如何工作的
%Slot-value memory maintains the values of specific slot while external memory holds other semantic information.

%We introduce 2 memories  network into sequence-to-sequence dialogue model inspired by the motivation that using memory to expand the representational capability. 

Our contributions are summarized as follows:
\begin{itemize}
    \item We propose a novel memory-augmented dialogue management model by introducing two memory networks. The slot-value memory network maintains the values of semantic slots during interaction, and the external-memory augments the single state representation of the recurrent networks. Both memory modules enable the model to access not only local utterances, but also the global semantics of the entire dialogue session. %Our experiments on two datasets show the effectiveness of the model.
    
    \item We propose an attention mechanism for updating the dialogue state. In particular, the model first computes a weight distribution over all words in a user utterance for each slot. Then, the weighted representation of the utterance is used to update the memory unit for each slot.
    
    \item The model can offer more observable and interpretable results in that the slot-value memory can track the change of dialogue states explicitly.
    
\end{itemize}


\section{Related Work}
\label{sec:rel-work}

% classification
The role of dialogue management (DM) is to launch the next interaction through predicting the next action the system should take, or by generating an utterance directly in response to user's query. The previous studies on DM can be broadly classified into three types: rule-based models, Bayesian network models, and neural models.

% 1 rule-based
Rule-based approaches date back to very early dialogue systems~\cite{weizenbaum1966eliza}. Several architectures are proposed to formulate the process of dialogue management. The \emph{flow diagram} approach \cite{mctear1998modelling} used a finite-state machine to model state transition in dialogue, where the state represents a certain dialogue status, and the transition between states is triggered by the corresponding type of a user utterance.  \emph{Slot-filling} approaches \cite{goddeau1996form} expanded the definition of dialogue state to an aggregation of slots and values. In such models, user can talk about each slot by issuing constraints and requesting the values of slots, and the dialogue state will be updated as long as a user provides new values for the slots during interaction.
%%下面这句话是什么意思啊???
%The \emph{agenda-based} model takes into account the hierarchical task structure of some dialogue, and allows developers to define sub-tasks in a dialogue system. 
Though \emph{rule-based} DM models work well in some applications, these approaches have apparent difficulties in task and domain adaptation~\cite{zukerman2001predictive} because the rules are usually tailored to a specific scenario. Due to the nature of hand-crafted rules, the variety and diversity of language is not well addressed. The need for hand-crafted rules also makes it expensive to build a {\em rule-based} system. 
%Even so, the framework of rule-based systems, especially the notion of \emph{belief state} and \emph{policy}, inspired the development of statistical approaches.

% 2 Bayesian network
Bayesian network approaches are proposed to address the issues of rule-based methods. Dialogue management was firstly formalized as a Markov decision process (MDP) \cite{levin1998using} under the Markov assumption \cite{paek2005markov}, in which the new state $s_t$ at turn $t$ is only conditioned on the previous state $s_{t-1}$ and system action $a_{t-1}$. MDP models the uncertainty in dialogue and becomes more robust to the errors induced by speech recognition and NLU. Partially observable Markov decision processes (POMDP) \cite{williams2007partially} provides a more principled way in that it takes environment observation $o_t$ into consideration. On the top of this framework, state transition and dialogue policy are trained using reinforcement learning. However, the POMDP model becomes difficult to train for the domains with large state space. An improved version of POMDP - Hidden Information State (HIS)  \cite{young2007hidden} is proposed to address this problem by grouping dialogue states into partitions. Another key problem in building Bayesian dialogue model is the lack of training corpus, thus user simulation \cite{schatzmann2006survey} is employed to enhance the training procedure, where dialogue data can be collected through interactions between a user simulator and a target system.
In spite of the success of Bayesian network methods, designing an appropriate reward function and manually crafting features limit the applicability of such approaches. As a noticeable defect, the state in these approaches is still manually defined, requiring a large amount of human labor.

% 3 neural network
A variety of neural models have recently been applied for the dialogue management task. Since the process of a dialogue session naturally follows a sequence-to-sequence learning problem at the turn level, recurrent neural network (RNN) is proposed to model the process \cite{henderson2014word, mrkvsic2016neural, wen2017latent}. At each turn, RNN takes as input the structured semantic representation produced by NLU (or raw user utterance when combining NLU and DM together) and predicts system action, where the hidden state of RNN is utilized as the representation of a dialogue state. 
There are also some neural end-to-end models which directly take dialogue context as input and generate natural language response \cite{shang2015neural, li2016deep, serban2016building, serban2017hierarchical} in open-domain conversational systems.
However, due to the vanishing gradient problem and the limited ability of state representation, RNN is difficult to capture the long-range context in dialogue. Hybrid Code Networks \cite{williams2017hybrid} proposes to handle the state representation problem by combining rule-based and RNN-based models together, while the performance is still highly dependent on the hand-crafted rules.

%% 下面这句话,看起来是评价神经网络的,跟DM有什么关系呢?
%Thus GRU and LSTM, which are variations of recurrent neural networks, are then introduced in dialogue system to ease the long-term dependency problem by introducing controllable gates and additional memory layers \cite{williams2016end}. 


%%Pre-trained word embeddings can provide abundant semantic information, but each word contributes differently on state update, which is not taken into account.
% A critical issue in existing approaches for DM is the incapability of modeling the long-range history semantics in the entire dialogue session. 


% FIG: word重要性不同
\begin{figure}
  \includegraphics[width=0.5\textwidth]{fig/word_attn.png}
  \caption{Slot-level attention: word mentions in user utterance are mapped to semantic slots such as {\it rating, cuisine, price, service,} and {\it location}.
  }%%
  \label{fig:word_attn}
\end{figure}


%% 下面这段逻辑太混乱了。。。
% 4 memory network
Memory network provides a principled approach for modeling long-range dependency and making multi-hop reasoning, which has advanced many NLP tasks such as machine translation \cite{wang2016memory} and question answering \cite{sukhbaatar2015end}. Neural turing machines \cite{graves2014neural} was proposed to augment existing neural models with additional memory units to solve complicated tasks. It is analogous to a Turing machine but is differentiable end-to-end. \cite{weston2014memory} proposed \emph{fully supervised memory networks} which employ supervision signal not only from answer labels but also from pre-specified supporting facts. \cite{sukhbaatar2015end} proposed \emph{end-to-end memory networks} (MEMN2N) which can be trained end-to-end without any intervention on which supporting fact should be used during training. \emph{Dynamic memory network} proposed by \cite{kumar2016ask} uses a sentence-level attention mechanism to update its internal memory during multi-hop inference. \emph{Key-value memory network} \cite{miller2016key} encodes prior knowledge by introducing a key memory structure which stores facts to address to the relevant memory value. There are already some works which introduced memory network into the task of dialogue management \cite{perez2016dialog} where memory networks are straightforwardly applied in a machine reading manner. In comparison, our model is better to model the long-range history semantics of the dialogue session by memorizing and updating the dialogue act types and the values of semantic slots explicitly, which is implemented through a slot-value memory and an external memory.


% 5. semantic feature in utterance
Extracting semantic information from user utterance is a key issue in task-oriented dialogue systems when combining NLU and DM together.
Early methods used hand-crafted rules and semantic features, including NER and POS tags, to construct semantic features for user utterance. \cite{henderson2014word} proposed to use the speech recognition confidence score as an additional feature. \cite{serban2016building, serban2017hierarchical} used hierarchical RNN models, where the user utterance is processed by a word-level RNN, and utterances are sequentially connected through an utterance-level RNN. \cite{mrkvsic2016neural} proposed to use convolutional neural network (CNN) model for semantic feature extraction. However, existing approaches did not consider the fact that words in an utterance contribute differently to different slots,  which is important for updating the dialogue state.


% 5 attention
%Attention mechanism has been shown to benefit many NLP tasks such as machine translation and [?] and conversation [?]. The basic idea is different part in inputs contributes differently on the final object. Most attention-based model has a decoder procedure which has to make sequential decisions. It takes the encoder results as an static memory and get the weighted-sum of memory cells as a context on each step of decoding. As mentioned above, DM module can be coupled with NLU by directly reading user utterance. In order to extract slot-correlate information in dialog state, the model should learning to attend to important words from utterance for different slot. To the best of our knowledge, there is still not much work reported on how to attend to important words for specific slot.
%By the representation capability of memory network, these model can better tackling the history information dependency problem. However, due to the restrict of MEMN2N model, these works produce answer by response ranking from an answer pool. For some sophisticated tasks, the huge and complex answer DA space makes it impossible to build a complete answer set. sentences by selection rather than generation, which makes it rather tough for domain adaptation and extension.

\section{Memory-augmented Dialogue Management with Slot-Attention}

% 整体模型图
\begin{figure}
  \includegraphics[width=0.6\textwidth]{fig/model.png}
  \caption{Memory-augmented Dialogue Management (MAD): At each dialogue turn $t$, the model takes as input the current user utterance and the previous system response, and predicts the next dialogue act. The slot-value memory is updated with an attentive read of the user utterance by a slot-level attention mechanism while the external memory is read and updated by the controller. The memory controller along with the two memory modules will predict the next dialogue act of the system by a classifier.
  }
  \label{fig:model}
\end{figure}


\subsection{Task Definition}
\label{sec:taskdef}
%%这里最好用数学符号,形式化定义一下对话管理的任务
% 简要介绍任务

% 下面是修改后的内容,先是一句文字描述,然后形式化定义
This paper deals with task-oriented dialogue management. We start by defining the input and output of our model. At the current turn ($t$) of a dialogue, given a user utterance along with the system response of the previous turn ($t-1$), the task of dialogue management module is to predict the next system dialogue act that will be utilized to generate a natural language utterance. This procedure can be formalized as follows:
\begin{equation*}
    P_{\theta}({DA}_t|x_1, y_1,..., x_{t-1}, y_{t-1}, x_{t})
\end{equation*}
where $x_t$ and $y_{t-1}$ are the user utterance at the current turn and system response at the previous turn, respectively, and ${DA}_t$ is the next dialogue act which can be used to generate system response. $\theta$ represents the parameters of the model. The next system response $y_t$ will be generated from $DA_t$ by a natural language generator, which is beyond the scope of this paper.

To exemplify the concept of {\it dialogue act} in our model, we take the task of restaurant reservation as an example, as shown in Table \ref{tab:dialogact}. Dialogue act ($DA$) is composed of two elements: {\em dialogue act type} and {\em slot-value pairs}. Dialogue act type is a general description of user intents, such as {\em Query} where the user may search for some information, and {\em Recommend} where the user may ask for some recommendations. A slot-value pair represents a filled value for a slot
\footnote{Generally speaking, a slot in task-oriented dialogue systems is a category of semantic features, which defines some key attribute or element for accomplishing a task. } %% price, number 能算作实体??
, such as Location={\em north}, Price={\em expensive} and Cuisine={\em British}. The slot-value pairs are usually regarded as the state representation in many dialogue state tracking studies \cite{henderson2014word}. During the interaction, the filled value for each slot may be provided or updated by the user, and correspondingly, the dialogue state changes. For instance, when the user says {\em How about a British restaurant in north part of town.}, two slot-value pairs, Cuisine={\em British} and Location={\em north}, will be updated. 
%% 下面这句话完全没说清楚
However, not all slot-value pairs which are mentioned in the context are to be addressed in the dialogue act of system response. We thus introduce an auxiliary variable {\em Mask}, which is a one-hot vector with dimension $n^s$ which is the number of slots, to decide which slot-value pairs are to be included in the next dialogue act. 
%%Mask 最好举一个句子的例子,MASK 这个概念不那么好懂
As shown in Table \ref{tab:dialogact}, the slots appeared in dialogue act are only Cuisine and Location, and their mask value is set to 1. In previous dialogue turns, the value of other slots may have already been mentioned, but their value is useless for the system response of this turn, and their Mask value is 0.
Generally speaking, a dialogue act can be viewed as the structured semantic representation of a natural language sentence.


% A DA sample can be "Recommend(area=Paris, restaurant type=Chinese)", indicating an intent of recommending a Chinese restaurant at Paris.
%%咱们就不能从读论文中学习到人家怎么描述的吗??
%The DA in our model is composed of three aspects{\em act type}, which is the dialogue act type of a question; {\em slot value}, the values of each slots and {\em mask}, indicating which slots are to be addressed in the sentence. 

%%这个表的数据在哪里?表的里面的数据严重不行!!


%%一定要介绍清楚DA是什么,给例子
%%一般人的不懂什么是DA

%It can maintain the history information in a dialogue more efficiently using two additional memory structure, and give appropriate response representation in each turn.
%The problem we are dealing with is formalized as a typical DM task: In each turn  of a dialog, we are given a post sentence from user along with the system response in last turn, and then produce a logical form of response sentence, such as dialogue Act (DA) \cite{}.
%The final response sentences could be easily generated by an existing NLG module using DA, which is not studied in this paper.
%要形式化说清楚任务是什么,DA是什么;


\subsection{Overview}

% 介绍基本变量
%先说一个central idea;具体想法再说细节
%In order to better model the history context during dialogue (),
%%解决的主要问题是什么
As shown in Figure \ref{fig:model}, the memory-augmented dialogue management model has two novel memory components, namely slot-value memory $(M^S,M_t^V)$ and external memory $M_t^E$. The slot-value memory consists of a static slot memory ($M^S$) and a dynamic value memory ($M_t^V$) where one memory unit $M^S$(i) in $M^S$ is mapped to a unique unit $M^V$(i) in $M_t^V$. $M^S$ remains unchanged during the interaction, while $M_t^V$ and $M_t^E$ is updated at each turn $t$.
We also design an RNN-based memory controller which controls read and write of the slot-value memory and external memory. 
The slot-value memory is updated with an attentive read of the user utterance by a slot-level attention mechanism while the external memory is read and updated by the controller. The memory controller along with the two memory modules will predict the next dialogue act of the system by a set of classifiers.
%These structures are designed to model the long-range history semantics of a dialogue session.
%%这两句话应该放在3.3 ;3.4说
%The slot-value memory tracks the dialogue state by storing and updating the values of the semantic slots during interaction. The external memory augments the conventional state presentation of RNN and stores more history information about the dialogue context.
%%这个地方,ext mem到底保存了哪些东西要说明一下

%%接下来要说这两个mem为什么要有,有什么用的
% 增加:简要解释两个memory的组成和作用。

%%这几句话放在这里不突兀吗?是要从概念层面说明其作用,而不是从符号层面
%Both S-V and external memory are composed of column vectors $(M^S(i),M_t^V(i))$ and $M_t^E(j)$.
%Each column of S-V memory corresponds to a specific slot in domain, with $M^S(i)$ and $M_t^V(i)$ represent the name and value of slot $i$, respectively.
%The interpretability of external memory is not as strong as S-V memory.
%It is a latent semantic representation of dialogue history.


%As illustrated in Figure \ref{fig:model}, the memory-augmented dialogue manager is based build upon general RNN architecture, with two extra memory networks, $M_t^E\in\mathbb{R}^{m\times{n}}$ and $M_t^K,M_t^V\in\mathbb{R}^{m\times{n_s}}$, where $m$ is the dimension of column vector, $n$ is the number of vector in external memory and $n_s$ is the number of slots. 
%These memories are meant to maintain the dialogue state respectively and enhancing the representation capability of single vector state $S_t$ of original RNN architecture.


Let $x_t=(\mathbf{e}_1^x,...,\mathbf{e}_{n_{x,t}}^x)$ and $y_{t-1}=(\mathbf{e}_1^y,...,\mathbf{e}_{n_{y,t-1}}^y)$\footnote{Note that $y_{t-1}$ is the system response at turn $t-1$ while $y_t$ is to be generated with a predicted $DA_t$.} denote the word embedding sequence of the user utterance at turn $t$ and the preceding system response at turn $t-1$, respectively, where $\mathbf{e}_i^x,\mathbf{e}_j^y\in\mathbb{R}^m$ are word embeddings, $n_{x,t}$ and $n_{y,t-1}$ are the lengths of two sequences. At each turn $t$, our model works in the following procedure:
\\
\\{\bf 1. Memory Read}:
The controller reads information from the value memory and external memory. The read of $M_t^V$ is conditioned on the controller state ($S_{t-1}$) and the value memory ($M_{t-1}^V$) at the previous turn, and the slot memory, formally as follows:
\begin{equation}
    \mathbf{r}_t^{V}= \mathbf{read^v}(S_{t-1}, M^S, M_{t-1}^V),\\
\end{equation}
and the read of the external memory conditions on the controller state and the external memory at the previous turn:
\begin{equation}
    \mathbf{r}_t^E= \mathbf{read^e}(S_{t-1}, M_{t-1}^E).
\end{equation}
Inspired by \cite{graves2014neural}, we introduce content-based addressing for memory read. $\mathbf{r}_t^{V},\mathbf{r}_t^E\in\mathbb{R}^m$ are content vectors read from the slot-value memory and the external memory, respectively.
\\
\\{\bf 2. Controller State Update}:
The controller state $S_{t-1}$ is then updated by the information read from the value memory and the external memory, and the content from $x_t$ and $y_{t-1}$:
\begin{equation}
    \mathbf{S}_t=\mathbf{GRU}(\mathbf{S}_{t-1}, [x_t;y_{t-1}; \mathbf{r}_t^V; \mathbf{r}_t^E])
\end{equation}
where GRU stands for gated recurrent units \cite{cho2014learning}, and $[\cdot;\cdot]$ denotes the concatenation of vectors. 
For simplicity, an utterance ($x_t/y_{t-1}$) is represented by the averaged word embeddings but more elaborated representation models are also applicable.
\\
\\{\bf 3. Memory Write}:
Memory vectors in $M_{t}^V$ and $M_{t}^E$ are updated based on $S_t$ and their previous values:
\begin{gather}
    M_t^V=\mathbf{write^v}(S_t, M^S, M_{t-1}^V)\\
    M_t^E=\mathbf{write^e}(S_t, M_{t-1}^E)
\end{gather}
The output at turn $t$ is obtained  based on $S_t$ and $M_t^V$. The output consists of the elements of a dialogue act, that is, the dialogue act type, slot-value pairs and a mask. Note that the slot memory $M^S$ is static and does not need to be updated.
%%task definition如果没有说明act type,slot 这里一定要重复说明。

%Details of the three operations will be presented in the following subsections.
%$read(\cdot)$ and $write(\cdot)$ are memory operations which will be introduced later.

\subsection{Slot-Value Memory}
\label{sec:sv-memory}
%The Slot-Value Memory in our model is illuminated by the Key-Value Memory Network proposed by \cite{miller2016key}, while our model is not identical to it.
The slot-value memory tracks the dialogue state by storing and updating the value of each semantic slot during interaction. It is composed of two components: slot memory and value memory, and both of them are composed of the same number ($n_s$) of column vectors. The slot memory is kept constant during the dialogue, with each column vector $M^S(i)$ corresponding to a semantic slot $i$. The semantic slots are like {\it Location, Price, or Cuisine}. 
% Each slot memory unit is initialized to zzzzzz and remains unchanged during training. The 
Inspired by \cite{miller2016key}, each slot memory unit $M^S$(i) in our model acts as the index, which helps to locate the content in $M_t^V$. In our proposed model, we further apply the slot memory unit to extracting slot-relevant information from user utterance.
Thus we keep $M^S$ unchanged during training and test time, and $M^S(i)$ is initialized by the averaged embeddings of words in slot $i$.

The value memory stores the value of each slot $i$ in $M_t^V(i)$. During the dialogue, the value of a slot may be added into the memory when a new slot is mentioned, or an old value can be updated to a new value of a previously mentioned slot. That is, each memory unit in the value memory stores the latest value (may be empty) of a semantic slot.
\\ \\
{\bf Read from the slot-value memory}
In our model, the main function of the slot-value memory is to trace the latest value of each slot, which is critical for predicting the slot-value pairs in the dialogue act. However, the effect of the slot-value memory on the state update of the controller is not straightforward. Thus, we employ a simple method for the read from the slot-value memory, which is the average of the vectors in the value memory:
\begin{equation}
    \mathbf{r}_t^V=\frac{1}{n_s}\sum_i{M_{t-1}^V(i)},
\end{equation}
where $n_s$ is the number of slots.
\\ \\
{\bf Write to the slot-value memory}
%
The write to $M_t^V$(i) depends on slot addressing which decides how much information should be updated for each slot when giving a user utterance. Ideally, the value memory is supposed to update its values for all slots that are mentioned in a user utterance. 
For example, when user inputs an utterance "{\em I want a Chinese restaurant}",  the model updates slot {\it Cuisine} with a new value {\em Chinese}.

Inspired by \cite{graves2014neural,miller2016key}, we apply a slot addressing technique to decide the amount of information that should be updated to each value memory vector of the corresponding slot given a user utterance:
\begin{equation}
    M_t^V(i)={\beta_t^i}{\mathbf{c}_t^i}+(1-\beta_t^i)M_{t-1}^V(i)
\end{equation} 
The first term is new information obtained from the attentive representation ($\mathbf{c}_t^i$) of utterance $x_t$ and the second term is the old information maintained. The attentive representation $\mathbf{c}_t^i$ of utterance $x_t$, described soon later, essentially decides the relatedness of the user utterance to slot $i$.
$\beta_t^i$ is a gate which controls how much $M_t^V$ should be updated, and it depends on the attentive read $\mathbf{c}_t^i$ and the last system response $\mathbf{y}_{t-1}$:
\begin{equation} \label{eq:updategate}
    \beta_t^i= {\rm sigmoid}(W_i^c([\mathbf{y}_{t-1};\mathbf{c}_t^i])+b_i^c)
\end{equation}
%   
If utterance $x_t$ mentions slot $i$, $\beta_t^i$ will be large, and the corresponding value memory unit $M_t^V(i)$ will be updated substantially, otherwise much less information will be updated with a smaller $\beta_t^i$. In order to better train these $\beta_t^i$, we employ additional supervision on the weight, as defined in $\mathcal{L}^{att}$ (see Eq. \ref{eq:loss-att}).


% slot-level注意力机制
\begin{figure}
  \includegraphics[width=0.7\textwidth]{fig/slot_attn.png}
  \caption{
  Slot-level attention mechanism for updating the slot-value memory.For each slot $i$, the attention score $\alpha_{i,j}$ for each word $j$ is calculated based on word embeddings $e_j$ and slot memory $M^S(i)$. Context vector $c_i$ is the weighted sum of word embeddings of the utterance. Finally, the value memory is updated based on the previous value vector and the context vector. Note that the attention mechanism is applied on each slot $i$.
  }
  \label{fig:slot_attn}
\end{figure}


\subsection{Slot-level Attention}
The context vector $\mathbf{c}_t^i$ in the above section is an attentive representation of utterance $x_t$, conditioned on the $i$-th slot vector. 
More formally, for an user utterance $x_t=(\mathbf{e}_1^x,..,\mathbf{e}_{n_{x,t}}^x)$, we compute attention weights ($\alpha_{i,1},...\alpha_{i,j},...,\alpha_{i,{n_{x,t}}}$) where each weight indicates the similarity of a word embedding $e^x_j$ to a slot memory unit $M^S(i)$, as follows:

\begin{equation}
    \mathbf{c}_t^i=\sum_{j=1}^{n_{x,t}}\alpha_{i,j}\mathbf{e}^x_j \\
\end{equation}
\begin{equation}\label{eq:attnweight}
    \alpha_{i,j}=\frac{\exp({d}_{i,j})}{\sum_{k=1}^{n_{x,t}}\exp({d}_{i,k})} \\
\end{equation}
\begin{equation}
    {d}_{i,j}=MLP([M^S(i), \mathbf{e}^x_j])
\end{equation}
%% 这个MLP最好直接写公式,不然还得说细节,几层啥的
%where $MLP_a(\cdot)$ is a MLP network to score the similarity between an utterance word $\mathbf{e}^x_j$ and a slot memory vector $M^S(i)$.
%%说一下细节
%The content vector $\mathbf{c}_t^i$ will be used for slot addressing to update the corresponding value vector of the value memory.
For the previous example, the weight between word {\em Chinese} and slot {\em Cuisine} will be large, while the weights between other words and this slot will be much smaller. The learning of $\alpha_{i,j}$ is also supervised as shown in $\mathcal{L}^{att}$ (see Eq. \ref{eq:loss-att}). 


\subsection{External Memory}

The external memory is used to augment the representation capacity of the single state of RNN \city{henderson2014word}, and it is sometimes referred to as {\em memory state} \cite{wang2016memory} in other works. Varies from the slot-value memory, external memory is not endowed with explicit semantic meaning in our framework. The external memory $M_t^E$ consists of $n_e$ columns of $m$-dimensional unit vectors, which are to be read and written to during dialogue controlled by the memory controller.
\\{\bf Read}
The read vector $\mathbf{r}_t^E$ at turn $t$ is a weighted sum of the memory units:
\begin{equation}
    \mathbf{r}_t^E=\sum_{i=1}^{n_e}\mathbf{w}_t^r(i){\cdot}M_{t-1}^E(i)
\end{equation}
where $n_e$ is the number of external memory units. And the weight $\mathbf{w}_t^r\in\mathbb{R}^{n_e}$ is given by
\begin{equation}
\label{eq-read-weight-ext}
    \mathbf{w}_t^r={\bm{g}_t^r}{\cdot}{\mathbf{w}_{t-1}^r}+(\mathbf{1}-\bm{g}_t^r){\cdot}{\widetilde{\mathbf{w}}_t^r}
\end{equation}
where $\mathbf{g}_t^r{\in}\mathbb{R}^{n_e}$ is an update gate which controls the amount of $\mathbf{w}_{t-1}^r$ to be updated, and $\widetilde{\mathbf{w}}_t^r$ is a weight controlling new information to read from $M_{t-1}^E$ conditioned on the state of the controller $\mathbf{S}_{t-1}$.
\begin{gather}
    \mathbf{g}_t^r=\sigma({W_g^r}{\mathbf{S}_{t-1}}) \\
    \widetilde{\mathbf{w}}_t^r={\rm softmax}(\mathbf{v}^\top[M_{t-1}^E(i);\mathbf{S}_{t-1}]) 
\end{gather}
%
{\bf Write}
There are two operations during the write to the external memory: $erase$ and $add$. $erase$ controls how much old information should be removed from the memory and $add$ controls the addition of new information. Formally, 
\begin{equation}
    M_t^E(i)=M_{t-1}^E(i)(\mathbf{1}-\theta(i){\cdot}\bm{\mu}_t^e)+\theta(i){\cdot}\bm{\mu}_t^a
\end{equation}
%
where the first term is the left information after erased by vector $\bm{\mu}_t^e \in \mathbb{R}^m$, and the second is new information added by vector $\bm{\mu}_t^a \in \mathbb{R}^m$. The scalar $\theta(i)=\bm{w}_t^r(i)$, the read weight on memory unit $i$, as defined in Eq. \ref{eq-read-weight-ext}.

%The erasing is determined by $erase~vector$ $\mu_t^e\in\mathbb{R}^m$ which decides the amount of content that will be removed from each column vector. 
%This is given by
%\begin{equation}
%    \widetilde{M}_t^E(i)=M_{t-1}^E(i)(\mathbf{1}-\mu_t^ew_t^r(i))
%\end{equation}
%where $\widetilde{M}_t^E(i)$ represents the remaining content.\\
%The $add$ operation is to control the amount of information of $add~vector$ $\mu_t^a\in\mathbb{R}^m$ to be updated into memory

Both {\it erase vector} and {\it add vector} are obtained conditioned on the state of the controller $S_t$, as follows:
\begin{gather}
    \bm{\mu}_t^e=\sigma(W^e\bm{S}_t)\\
    \bm{\mu}_t^a=\sigma(W^a\bm{S}_t)
\end{gather}


% 整体模型图
\begin{figure}
  \includegraphics[width=0.7\textwidth]{fig/model2.png}
  \caption{Dialogue act prediction of MAD: ${DAT}_t$ is the dialogue act type of system response at turn $t$. ${Mask}_t$ is the mask for slot-value pairs at turn $t$, and the color of each mask block indicates its value, with white indicating 1 and black for 0. $v_t^i$ represents the value of slot $i$. The prediction of ${Mask}_t^i$ and $v_t^i$ are both based on $M_t^E$(i).
  }%% 这个color指什么color,白色,黑色,还是什么???
  \label{fig:model2}
\end{figure}


% Dialogue Act 生成
\subsection{Dialogue Act Prediction}
\label{sec:prediction}

As illustrated in Figure. \ref{fig:model2}, our memory-augmented network predicts a dialogue act as follows: first, the dialogue act type is predicted via $P_t^{dat}$; second, each slot is associated with a binary classifier ($P_t^{m,i}$) that decides whether the $i-$th slot should be included in the final dialogue act; third, if a slot $i$ is selected, the value of the slot is predicted by $P_t^i$. The final dialogue act can be assembled by these predicted results. 
%. Since a dialogue act is composed of a dialogue act type, and one or more slot-value pairs, we conduct the prediction on the following variables:
%
%
\\
    \\ {\bf Predicting dialogue act type}: this classifier outputs a distribution over dialogue act types such as {\em Inform}, {\em Request}, and {\em Recommendation}.
    It is implemented by a MLP conditioned on the controller state and all memory units:
    \begin{align}
    \label{predict:dat}
        P_t^{dat}(dat|\bm{S}_t,M_t^E, M_t^V)=MLP([\bm{S}_t;M_t^E(1);...;M_t^E(n_e);M_t^V(1);...;M_t^V(n_s)])
    \end{align}
    where $dat$ is one of all the dialogue act types.
    %This prediction is to inform the other modules about what is the intent of the next interaction to be launched.
%
%
    \\ {\bf Predicting a slot}: there is a slot mask which controls the slots to be included in the final dialogue act.
    There is a binary classifier for each slot $i$ conditioned on the controller state $S_t$, external memory $M_t^E$ and its corresponding value memory unit $M_t^V(i)$):
    \begin{equation}
    \label{predict:mask}
        P_t^{m,i}(z|\bm{S}_t)=MLP([M_t^E(1);...M_t^E(n_e)])
    \end{equation}
    where $z \in \{0,1\}$, $z=1$ indicates that slot $i$ should be included in the next dialogue act. 
%
    \\ {\bf Predicting the value of a slot}: once we obtain which slot should be included in the dialogue act, we need to decide which value of the slot should be mentioned. This is given by the classifier which estimates a probability distribution over all the values for a slot:
    \begin{equation}
    \label{predict:sv}
        P_t^{i}(v_j^{i}|\bm{S}_t)=MLP([M^S(i), M_t^V(i)])
    \end{equation}
    where $v_j^{i}$ is all the values of slot $i$. %Thus, for each slot, the slot memory vector and the corresponding value memory vector are input to the classifier to predict which specific values have been addressed up to the current state.
%  
%   

% 增加:mask的作用,如何与slot pair搭配使用
%%这个地方还需要讲一下mask是怎么用的,和slot value pair的预测一起用么?具体如何搭配使用的,没有说清楚
%%这么设计的motivation???


%%【注意,很多MLP,但是要说其中的细节,或者直接用公式代替,不说MLP】

% 损失函数
\subsection{Loss Function}
We adopt cross entropy as our objective function.
There are three terms in the function corresponding to the prediction of dialogue act types ($\mathcal{L}^{dat}$), slot-value pairs ($\mathcal{L}^{v}$), and slot mask ($\mathcal{L}^{m}$), as presented in the previous section.

%The first loss is defined on DA type, with the second and third on slot value pairs and masks.
%% 这几个公式,变量符号全部都改成跟3.5一致;而且,应该还有sigma over t才对吧?
%%就是必须用3.5节的符号来定义这些loss
The loss function is defined as follows:
\begin{equation}
\label{eq:loss}
    \mathcal{L}=\mathcal{L}^{dat}+\gamma \sum_i\mathcal{L}^m(i)+\lambda\sum_i\mathcal{L}^v(i)
\end{equation}
where
%%顺序要跟前面的调整为一致!!!
\begin{gather}
    \mathcal{L}^{dat}=-\sum_t\sum_{k=1}^{n_{dat}}[\hat{P}_t^{dat}(dat_k) {\rm ln}{P_t^{dat}(dat_k)}] \\  %%应该有两个sigma
    \mathcal{L}^m(i)=-\sum_t\sum_{z \in \{0,1\}}[\hat{P}_t^{m,i}(z){\rm ln}{P_t^{m,i}(z)}] \\ %三个sigma\\
    \mathcal{L}^v(i)=-\sum_t \sum_{k=1}^{n_{i}}[\hat{P}_t^{i}(v_k^i){\rm ln}{P_t^{i}(v_k^i)}] %%这个公式应该有三个simga
\end{gather}
%%要说清楚之前没有介绍过的变量,以及gold分布怎么得来的;
where $n_{dat}$ is the number of dialogue act types,
$n_i$ is the number of values for slot $i$,
$\hat{P}_t^*$ are the gold distributions obtained from the training data, and $P_t^*$ are defined in the preceding subsection.
%
$\lambda$ and $\gamma$ are hyper-parameters.

% attention 和 update gate的监督
%What's more, we found that applying attention supervision on early stage of training is important to improve model performance.
%That is because the parameters related to memory reading and writing are trained based on the content vectors obtained from user utterance, which is controlled by attention parameters.
%Thus a pre-training for attention model can help to reduce the unnecessary loss on early stage and avoid rapping in local optimum.
%We also use cross entropy loss for attention supervision, which is applied on both attention vector and value memory update weight.

%%这些中间的监督,就必须说清楚,标注是怎么得到的。
% 
Furthermore, we found that performance improvement can be observed when applying weak heuristic supervision on the intermediate variables, and the supervision signal can be easily obtained by simple string matching rules. This is a common practice for training sophisticated neural networks \cite{liu2016att-superversion, kiddon2016globally}. More specifically, we apply extra supervision on the update gate of the value memory (see Eq. \ref{eq:updategate}) and the attention weight of an utterance (see Eq. \ref{eq:attnweight}). %%这里最好引用一下coling2016周昊讲过的那个attention sup
Those intermediate supervision is applied with a two-stage training schema: firstly, we pretrain our model only with the heuristic loss ($\mathcal{L}^{att}$, see below) for several epochs, and then train the model further with the loss ($\mathcal{L}$) defined by Eq. \ref{eq:loss} for the remaining epochs.


The heuristic supervision loss is defined as follows:
\begin{align}
\label{eq:loss-att}
    \mathcal{L}^{h}=-\sum_t\sum_i\sum_{j=1}^{n_{x,t}}[\hat{\alpha}_{i,j}^t{\rm ln}{\alpha}_{i,j}^t]\notag\\
    -\sum_t\sum_i[\hat{\beta}_t^i{\rm ln}{\beta}_t^i+(1-\hat{\beta}_t^i){\rm ln}(1-{\beta}_t^i)]
\end{align}
%%请把这个公式直接修改为用前面的变量符号!!!并说明如何得到gold weight
% 变量符号已经修改
where $n_{x,t}$ is the number of words in $x_t$ at turn $t$ and $i$ is the slot index.

Note that $\hat{\alpha}_{i,k}^t$ and $\hat{\beta}_t^i$ represent the gold distributions of the update and attention weights, respectively.
For each word $w_j$ of utterance $x_t$, if $w_j$ appears in the values of slot $i$,  $\hat{\alpha}_{i,j}^t=1$ and $\hat{\beta}_t^i=1$, otherwise
$\hat{\alpha}_{i,j}^t=0$ and
$\hat{\beta}_t^i=0$. This means that if a value of a slot appears in the utterance, the value (also the word) should be attended w.r.t. that slot, and the update weight should be equal to 1. By this way, the value memory of the corresponding slot can be updated accordingly.
%%
%$\hat{\beta}_t^i=1$ while %$\hat{\alpha}_{i,k}^t|_{k\neq j}=0$.








\section{Experiment}
%%所有的we做主语的修改为过去式,在实验部分
\subsection{Data Preparation}
% 数据概括
We first evaluated our memory augmented dialogue management model on two synthetic datasets adopted from the \emph{dialog bAbI} dataset\cite{bordes2016learning} and \emph{the Second Dialogue State Tracking Challenge} dataset \cite{henderson2014second}, which are originally proposed for end-to-end dialogue systems and dialogue state tracking task. However, both of the above two datasets are small-scale. To better assess the performance of our proposed model on large-scale datasets, we collected a new Chinese dialogue management dataset consisting of real conversations from the flight booking domain.

\subsubsection{DMBD: Dialogue Management bAbI Dataset}
%简要介绍
The original {\em dialogue bAbI} dataset (DBD) is designed to evaluate the performance of end-to-end dialogue systems on the task of restaurant reservation. In \cite{bordes2016learning}, the task is formulated as a machine comprehension task by applying the MEMN2N \cite{sukhbaatar2015end} model, considering the dialogue context and last user utterance as story and question respectively, and the system response is selected from a fixed answer set. The DBD dataset is composed of five manually constructed subtasks: {\em issuing API calls, updating API calls, displaying, providing extra information} and {\em full dialogue}, to examine the system performance on different tasks, in which the {\em full dialogue} is a combination of the first four tasks. The data for these tasks were collected through a simulator which is based on an underlying knowledge base along with some manually-crafted natural language patterns, where the simulator rules can be utilized by us to perform dialogue act annotations. For more details of DBD, please refer to \cite{bordes2016learning}. 


\begin{table}[htbp]
\begin{center}
\begin{tabular}{| c | c | c |}
  \hline
    \multicolumn{2}{|c|}{\bf Informable slots} & \multicolumn{1}{c|}{\bf Requestable slots}\\
  \hline
    Name & \#Value & Name\\
  \hline
    Cuisine & 10 & Address\\
    Location & 10 & Telephone\\
    Price & 3 &\\
    Size & 4 & \\
  \hline
\end{tabular}
\end{center}
\caption{\label{tab:babi-ontology}Ontologies of the DMBD dataset. An {\em informable slot} means that user can provide values to the slot to constrain a query to KB; while a {\em requestable slot} can only be queried from KB without any user provided value.}
\end{table}


%%下面这段我试着写一下
Since the dialogue act types and slot-value pairs are not annotated in DBD, we have to do this by ourselves to train our model. Fortunately, we can easily annotate the system response utterances because the original data is generated with an underlying knowledge base and some simple natural language patterns. We thus did reverse engineering by conducting automatic annotations with manually-crafted rules utilizing the knowledge base of DBD to label the dialogue act type and slot-value pairs for each utterance. This processed dataset for dialogue management is termed as {\it Dialogue Management bAbI Dataset} ({\bf DMBD}) in the following sections.


% 数据处理后的统计
In DMBD, the original user and system utterances are reserved to serve as the input of each turn of dialogue, while the output is changed from system utterance to its dialogue act, as detailed in Table \ref{tab:dialogact}. The resulting DMBD dataset has fifteen dialogue act types, four informable slots and two requestable slots, as seen in Table \ref{tab:babi-ontology}.
An informable slot means that user can provide values to the slot to constrain a query to KB; while a requestable slot can only be queried from KB without any user provided value. Note that DMBD shares the same KB with DBD. As the requestable slots are only used for issuing API calls, in our implementation, we design a special informable slot called {\em Ask Slot}, which tracks the slots that are to be queried. The values of {\em Ask Slot} are the names of requestable slots.

% \footnote{We solve the problem of requestable slots by introducing an special informable slot with the name of requestable slot as its values.???%%这句话什么意思啊

%% 下面这些细节,对于审稿人更容易理解你的论文有帮助吗?如果没有,为什么要写?
%% 请多站在读者的角度组织内容!!!
%As the requestable slots are only used for issuing API calls, we design a special informable slot {\em ask slot}, which tracks the slots that are to be queried.
%{\em ask slot} has only two values, {\em address} and {\em telephone}, which are exactly the two requestable slots.
%Note that this processing helps to facilitate our model to process the two kinds of slots uniformly.

\subsubsection{DM-DSTC: Dialogue Management of the Second Dialogue State Tracking Challenge dataset}
The dialogues in the above DMBD are collected via a simulator which employs hand-crafted templates, and are thus more or less synthetic. In order to evaluate the performance of our model on real-world dialogue corpus, we conducted another experiment based on DSTC2 which is a real world dialogue dataset, and it is also about the task of restaurant reservation.

% 数据改造
The original DSTC2 dataset is for dialogue state tracking, in which the output at each turn is the filled slots and their values which have already been presented by the user so far. The dialogue act of the system utterance is also annotated and is thus directly utilized as model output.
We thus transform the original DSTC2 dataset to our settings for dialogue management, referred to as {\bf DM-DSTC}. The ontologies of dialogue act type and slot in the original dataset are directly reused in the DN-DSTC.

% 数据细节
The resulting DM-DSTC is composed of four informable and nine requestable slots, and the average value number of informable slots is 54, which is much higher than that of DMBD, and the enhanced complexity of DM-DSTC dataset reflects the characteristics of real-world data which is more stochastic and noisy. We also created a special slot for requestable slots in this experiment as we did in the DMBD experiment. Some statistics of DM-DSTC are shown in Table \ref{tab:dstc-ontology}.


\begin{table}[htbp]
\begin{center}
\begin{tabular}{| c | c | c |}
  \hline
    \multicolumn{2}{|c|}{\bf Informable slots} & \multicolumn{1}{c|}{\bf Requestable slots}\\
  \hline
    Name & \#Value & Name\\
  \hline
    Food & 91 & Addr, Area, Food\\
    Pricerange & 3 & Phone, Pricerange\\
    Res\_name & 113 & Postcode, Signature\\
    Area & 5 & Res\_name\\
  \hline
\end{tabular}
\caption{\label{tab:dstc-ontology}Ontology of the DM-DSTC dataset. The {\em Res\_name} indicates restaurant name. The average value number of informable slots is 54 which is much higher than that of DMBD dataset. The enhanced complexity of DM-DSTC reflects the characteristics of real-world dialogue data.}
\end{center}
\end{table}

\subsubsection{ALDM: Alibaba Dialogue Management Dataset}
\label{sec:aldm-dataset}
The sizes of the above two datasets are limited, we thus propose ALDM to test our model's performance on large-scale dataset.
ALDM is a Chinese dataset, consisting of real conversations from the flight-booking domain, in which the system is supposed to acquire departure city, arrive city and departure date information from the user to book a flight ticket. To better fit our model, the departure date values in the corpus are preprocessed into an uniform MM.DD format, e.g., {\em 12.25} for {\em 25th, Dec.}. ALDM is much larger than the other two datasets, where there are 15,330 sessions for training, 7,665 for validation, and 3,832 for test. 
On average, there are 5 turns in a session. The average sentence length is 4, and particularly, most of the user responses have only one word as users only provide the departure or arrival city, or the departure data.
One difference to the other two datasets exists in that the departure city slot and the arrive city slot share the same value list, which raises additional difficulty to require the model to identify which slot the city name in the user utterance should be filled in. To handle this issue, the model should be able to fill slots conditioned on the dialogue context. For example, if the user responds with {\em Beijing} to the last system response {\em Where are you flying from?}, the value of {\em Beijing} should be filled in the {\em departure city}. Another difference is that there are not requestable slots due to the fact that ALDM is system-driven.


\begin{table}[htbp]
\begin{center}
\begin{tabular}{| c | c | c |}
  \hline
    {\bf DA type} & \multicolumn{2}{c|}{\bf Informable Slots}\\
  \hline
    ask\_dep\_loc & Name & \#Value\\
  \cline{2-3}
    ask\_arr\_loc & Dep\_city & 174 \\
    ask\_dep\_date & Arr\_city & 174 \\
    offer, end & Date & 100 \\
  \hline
\end{tabular}
\caption{\label{tab:aldm-ontology}Ontology of the ALDM dataset. The {\em ask\_} DA type means the system is asking the user for information, {\em offer} means the system is giving recommendation and {\em end} means the dialogue session is done. {\em Dep\_city} and {\em Arr\_city} represent the slot of departure city and arrive city respectively, and they share the same value list. The value of {\em Date} slot is transformed into a uniform MM.DD format.}
\end{center}
\end{table}

As shown in Table \ref{tab:aldm-ontology}, ALDM is composed of 3 informable slots, and the average value number is 150, which is remarkably larger than those of the above two datasets. And there are 5 dialogue act types as shown in Table \ref{tab:aldm-ontology}.


\subsection{Experimental Setup}
Our model is implemented with Tensorflow \cite{abadi2016tensorflow}. The word embeddings used in each dataset were pretrained on their own dialogue corpora, where there are 15,000 sessions in DMBD (3,000 per each task), 2,118 sessions in DM-DSTC and 26,827 sessions in ALDM, using the GloVe algorithm \cite{pennington2014glove}. The dimensions of word embeddings, memory column vectors, and state vectors were all set to 128, and there are 8 columns in the external memory. 
We first pretrain our model with the heuristic loss $\mathcal{L}^h$ (see Eq. \ref{eq:loss-att}) for 2 epochs and then continue to train it using $\mathcal{L}$ in Eq. \ref{eq:loss}.

The parameters $\gamma$ amd $\lambda$ in $\mathcal{L}$ are not constant during training.
More specifically, in the first 7 epochs, $\lambda$ increases linearly from 0 to 1 while $\gamma$ remains zero, and in the following 7 epoches $\gamma$ also rises from 0 to 1 linearly with $\lambda$ unchanged. The reason for this setting is that the process of the value update in the slot-value memory has strong influence on the training of other components.
All the other parameters are initialized with a random uniform distribution $\mathcal{N}(0, 1)$.

We used the train/valid/test partition of the original DBD for each task, where there are 1,000 sessions in each set; and the partition of DM-DSTC is 1412/353/353. For ALDM, we split the dataset into 15,330/7,665/3,832.

We trained our model using ADAM \cite{kingma2014adam} with a learning rate which is set to 0.002, and the momentum parameters ${\beta}_1 = 0.9$ and ${\beta}_2 = 0.999$. For each dataset, the model is trained with at most 15 epochs. We use the model parameter with the lowest validation loss for test.


\subsection{Baseline}

We included two types of baselines in the evaluation. The first type is to select a sentence as answer from a predefined candidate answer set in a machine comprehension manner, as described in \cite{bordes2016learning}. The second type is to predict a structured dialogue act, the same as our model, where the models need to make predictions over all combinations of dialogue act type and slot-value pairs.

In the baselines of the first type, each candidate answer sentence is a natural language utterance, which lexicalizes\footnote{Lexicalizing a dialogue act means converting the act from formal semantic representation to a natural language utterance.} an underlying dialogue act. However, the candidate answer set is not complete, where not all possible combinations of dialogue act type and slot-value pairs are included. In other words, the size of the answer space in the first type is less than that in the second type. Thus, the first setting is therefore easier than the second one. 


The baselines of the first type, which select an utterance from a predefined candidate answer set \cite{bordes2016learning}, are listed as follows:
\begin{itemize}

\item {\bf TF-IDF}: A TF-IDF matching algorithm\cite{salton1986introduction} which computes a cosine similarity score between the input (the whole dialogue history) and a candidate sentence, and the sentence with the highest score is selected as the final answer. Both the input and the candidate sentence are represented by the average of bag-of-word vectors.
    
\item {\bf TF-IDF(+ type)}: An enhanced version of TF-IDF by introducing additional entity type features.

\item {\bf Supervised Ebd}: An information retrieval model based on trainable word embeddings. The similarity score between an input and a candidate sentence is the inner product of their averaged word embeddings. The  is trained with a margin ranking loss \cite{bai2009supervised}.
%% 确定这个叫做supervised吗?体现在哪里呢?

\item {\bf MEMN2N}: Standard end-to-end memory networks \cite{sukhbaatar2015end, bordes2016learning}. It stores the dialogue history information in a memory network and chooses a response by running multi-hop reasoning upon the history. 

\item {\bf MEMN2N(+ match)}: A variant of MEMN2N which included additional features about entity types.
\end{itemize}


The baselines of the second type, which predict a structured dialogue act, the same as our proposed model, are as follows:

\begin{itemize}

\item {\bf MEM}: A memory network model which predicts dialogue act. For each output structure (DA type, slot-value, and mask), a MEMN2N is introduced to make prediction.

\item {\bf RNN}: A recurrent neural network model with turn-level input and output. The dialogue act predictions (type and slot-value) are based on the hidden state $\bm{S}_t$ at each time step $t$.

\item {\bf MAD - SM}: A variant of our proposed model without the slot-value memory. Those predictions involving the slot-value memory are modified to using only the memory controller state $\bm{S}_t$ to make prediction.

\item {\bf MAD - Attn}: A variant of our model without the slot-level attention mechanism. In this setting, the averaged word embeddings of an utterance is used to update the slot-value memory.

\item {\bf MAD - EM}: A variant of our model without the external memory. The predictions involving the external memory are modified to using the memory controller state $\bm{S}_t$ only, just as MAD-SM.
\\
\end{itemize}

It should be noted that the MEMN2N and MEM baseline take as input a context-question pair at each round, which means they have to make calculation on the cumulated dialogue context at each turn. Thus with the increasing of the dialogue context, there is an exponential increase in the computation complexity. While for our model, the context information is stored in the memory network, and the computation time in each turn is basically the same.


\subsection{Performance on DMBD}

In this section, we evaluated the performance of our model and the baselines on the DMBD dataset. The prediction accuracy on both turn-level and session-level evaluation is reported, similar to \cite{bordes2016learning}. Based on the distribution defined in Section \ref{sec:prediction}, our model chooses a dialogue act with the maximal probability as output, respectively for {\em DA type}, {\em slot-value} and {\em mask}. Note here that for  DA type and mask, the prediction is judged as correct only if the output matches the target. 
As mentioned in Section \ref{sec:taskdef}, mask is an auxiliary variable helping to filter the undesired slot-value pairs in a predicted dialogue act.
Thus for the prediction of slot-value, we only need to correctly predict those slot-value pairs whose mask value is 1.
% While the prediction of slot-value is judged to be correct only if those slot-value pairs which are to be addressed in the final dialogue act are correctly predicted.
%% 这句话没说清:是说预测的 包含 在target里就算对,还是什么???
Finally, the overall dialogue act is correct only if its DA type, slot-value and mask are all correctly predicted. And a dialogue session is correct only if all the dialogue acts in the session are correctly predicted. We termed this session-level evaluation. 

% final results for DMBD
\begin{table*}
\centering
\begin{tabular}{c | c c | c c | c c | c c | c c }
    \hline
        \multirow{2}{*}{Metrics} & \multicolumn{2}{c|}{1 Issuing } & \multicolumn{2}{c|}{2 Updating} & \multicolumn{2}{c|}{3 Displaying} & \multicolumn{2}{c|}{4 Providing} & \multicolumn{2}{c}{5 Full} \\ 
        & \multicolumn{2}{c|}{API calls} & \multicolumn{2}{c|}{API calls} & \multicolumn{2}{c|}{options} & \multicolumn{2}{c|}{options} & \multicolumn{2}{c}{dialogs} \\
    \hline
        {TF-IDF (no type)} & 5.6 & (0) & 3.4 & (0) & 8.0 & (0) & 9.5 & (0) & 4.6 & (0)\\
        {TF-IDF (+ type)} & 22.4 & (0) & 16.4 & (0) & 8.0 & (0) & 17.8 & (0) & 8.1 & (0)\\
        {Nearest Neighbor} & 55.1 & (0) & 68.3 & (0) & 58.8 & (0) & 28.6 & (0) & 57.1 & (0)\\
        {Supervised Ebd} & 100 & (100) & 68.4 & (0) & 64.9 & (0) & 57.2 & (0) & 75.4 & (0)\\
        {MEMN2N (no match)} & 99.9 & (99.6) & 100 & (100) & 74.9 & (2.0) & 59.5 & (3.0) & 96.1 & (49.4)\\
        {MEMN2N (+ match)} & 100 & (100) & 98.3 & (83.9) & 74.9 & (0.0) & 100 & (100) & 93.4 & (19.7)\\
    \hline
        {MEM} & 47.4 & (0.1) & 61.1 & (0.1) & 24.6 & (0.1) & 56.7 & (0.8) & 25.2 & (0.1) \\
        {RNN} & 80.6 & (0.1) & 45.5 & (0.0) & 30.0 & (0.0) & 57.2 & (0.0) & 3.7 & (0.0) \\
        % {MAD - SM} & 77.2 & (0.2) & 78.9 & (0) & 70.7 & (0) & 57.3 & (0.1) & 59.6 & (0) \\
        % {MAD - Attn} & 82.7 & (0.5) & 79.0 & (0) & 73.9 & (0) & 57.3 & (0.1) & 67.7 & (0) \\
        % {MAD - EM} & 95.2 & (78.2) & 57.4 & (0.2) & 40.5 & (0.0) & 1.0 & (1.0) & 3.1 & (0.0)\\
    \hline
        {MAD} & 99.0 & (94.2) & 100 & (100) & {\bf 99.1} & {\bf (90.6)} & 100 & (100) & {\bf 99.9} & {\bf (97.8)}\\
    \hline
\end{tabular}
\caption{\label{tab:result} The accuracy across all tasks and methods. The numbers in brackets are the accuracy at the session level, and numbers without brackets are at the turn level. A session is correct only if all the sentences in the session are predicted correctly.}
\end{table*}

\subsubsection{Overall Performance Analysis} 



% 在 DMBD 上的评价
We first evaluated our proposed model based on the overall accuracy of dialogue act prediction, as shown in Table \ref{tab:result}. The results of baselines of the first type are reprinted from their original paper \cite{bordes2016learning}, because the partitions of training/validation/test data are the same as ours, and the results are hence directly comparable. Both turn-level and session-level results on all the five tasks are reported. 
%% 把实验分析结果和结论分成item,清晰的point
%% 先说最重要的,然后说次要的,逐条,一个item只说一个要点
We have the following observations:
\begin{itemize}
    \item 
    MAD obtains the best performance on most of the tasks. The model obtains an accuracy of about 100\% at both turn and session-level evaluation, which shows the effectiveness of our proposed model. While in Task 1, MAD is at the second place, where Supervised Ebd and MEMN2N (+match) methods obtains 100\% accuracy at both turn and session-level evaluation, which is 1\% higher than ours. MAD's defect on Task 1 can be attributed to a potential rule in Task 1, where if the user doesn't provide enough values to form a query, the agent will request for the value of slots in a fixed order. For example in task 1, the agent requests for slots in an order of ({\em Cuisine} $\rightarrow$ {\em Location} $\rightarrow$ {\em Size} $\rightarrow$ {\em Price}). However, this order rule is not essential for a practica application, where the agent can request for values in an arbitrary order as long as it can obtain all necessary values.
%% 你是想说,数据是按照固定的slot order 产生的,
%% 预测的序却可能不一样,但只要最后任务、结果能完成就ok??上面这段要重新写一下

    \item
    % As for the variants of MAD, MAD-SM, which ablates the slot-value memory module, obtains degraded performance compared to MAD. MAD-Attn, which removes the slot-level attention mechanism, works worse than MAD but still slightly better than MAD-SM on each task. The performance of MAD-EM drops even more than MAD-SM on all tasks except for Task 1. The RNN model, which can be regarded as MAD without slot-memory and external memory, performs even worse on most of the 5 tasks.

\end{itemize}


\subsubsection{Fine-grained Performance Analysis}

To better understand how the slot-value memory and the external memory influence the performance, we further analyzed the fine-grained prediction accuracy of MAD and its variants in addition to the overall dialogue act prediction. Evaluation on the fine-grained predictions is shown in Table \ref{tab:subtask}. We have the following observations:

%% 我建议不要以模型配置领头这么去写,因为这样跟前的很多重复
%% 相反:以主要结论和观点去说,被比较的对象只是顺便带出来的,要特别注意 每段的第一句话

\begin{itemize}
    \item The variants of MAD, MAD-SM, which ablates the slot-value memory module, obtains degraded performance on overall accuracy compared to MAD. MAD-Attn, which removes the slot-level attention mechanism, works worse than MAD but still slightly better than MAD-SM on each task. The performance of MAD-EM drops even more than MAD-SM on all tasks except for Task 1. The RNN model, which can be regarded as MAD without slot-memory and external memory, performs even worse on most of the 5 tasks.

    \item The fine-grained results demonstrate the effectiveness of our proposed model more specifically. Here we can see that the accuracy of MAD on both slot-value and mask is 100\%, while the prediction on DA type has very few errors. The high accuracy of slot-value prediction indicates that the slot addressing and the attentive question representation work well, which is attributed to the slot-value memory and attention supervision we applied. The contribution of the external memory is also shown by the high performance of DA type and mask prediction.
    
    \item The slot-value memory leads to significant improvements in slot-value accuracy. In our model, the role of the slot-value memory is to extract semantic information about slots during the dialogue, thus the ability of tracking slot-value information should decrease if the slot-value memory is removed. As shown in Table \ref{tab:subtask}, the prediction accuracy of MAD-SM on slot-value drops much from 100\% to around 30\%. However, the performance on dialogue act type and mask prediction are not heavily affected, and the accuracy is still above 90\%.
    
    \item The slot-level attention mechanism we applied on semantic information extraction influences the performance remarkably. In MAD-Attn, the slot-level attention mechanism is removed, and the value update is based on averaged word embeddings of user utterance. Intuitively, the update of the slot-value memory is not able to concentrate on relevant words without attention mechanism, thus the performance of slot-value prediction must be heavily influenced. The experiment results also support our hypothesis, where the accuracy of slot-value prediction degrades remarkably, but is still better than that of MAD-SM since MAD-Attn retains the slot-value memory. The attention mechanism affects dialogue act type and mask prediction very slightly.
    
    \item The external memory significantly improves the performance of DA type and mask accuracy by enhancing the representation capacity of the original RNN state. In MAD-EM, the external memory is removed, and those predictions involving the external memory, that is the prediction of DA type and mask, are changed to use the memory controller state, which is identical to the hidden state in a RNN model. Compared to MAD, the accuracy of MAD-EM on DA type and slot-value prediction decreases heavily. This is attributed to the enhanced representation capacity, meaning that the model can do better in capturing longer term temporal dependencies in dialogue.
\end{itemize}

% \begin{itemize}
%     \item 
% {\bf MAD}: Here we can see that the accuracy of MAD on both slot value and mask is 100\%, while the prediction on DA type has very few errors. The high accuracy of slot value prediction indicates that the slot addressing and the attentive question representation work well, which is attributed to the attention supervision we applied.

%     \item 

% {\bf MAD-SM}: The slot-value memory in our model is in charge of maintaining semantic information about slots during dialogue, thus the capability of tracking slot-value information should be negatively affected if the slot-value memory is removed. As shown in Table \ref{tab:subtask}, the prediction accuracy of MAD-SM on slot-value declines much from 100\% to around 30\%. However, the performance on dialogue act type and mask prediction are not heavily affected, and the accuracy is still above 90\%.


%     \item 
% {\bf MAD-Attn}: The slot-level attention mechanism is removed in this setting, and the value update is based on averaged word embeddings of user utterance. Intuitively, the update of the slot-value memory is not able to concentrate on relevant words without attention mechanism, thus the performance of slot-value prediction must be heavily influenced. The experiment results support our hypothesis, where the the accuracy of slot-value prediction is sharply reduced, but still better than that of MAD-SM since it retains the slot-value memory. The effect of attention on dialogue act type and mask is slight and their accuracy is still close to 100\%.


%     \item 
% {\bf MAD-EM}: Compared to MAD, the performance of MAD-EM on dialogue type and mask prediction decreases more than that on slot-value. This can be explained by the function of the external memory. In MAD, the external memory is used to enhance the representation capability of single hidden state of recurrent neural network. As the semantic information of slot-values is mainly captured and represented by the slot-value memory, the external memory is supposed to handle other semantic information, such as the dependency of context and current system action. Thus the external memory affects more on dialogue act type and mask than slot-value pair, which is indeed reflected by the experiment results.
% \end{itemize}


From the above analysis, we can see that the effect of the slot-value memory is mainly on predicting slot-value, while the effect of the external memory is on predicting dialogue act type and mask.
%%下面这句没有看懂
% 这里是说我们的两个memory network并不是简单的引入两个不相关的memory单元,从测试结果看它们两个在训练的过程中是互相影响的。
However, the influence of the modules on the performance is more complex. We can see from Table \ref{tab:subtask} that DA type and mask accuracy will also decrease if the slot-value memory is removed, and so will slot-value accuracy when we remove external memory. This means the two memory networks in our model are coupled correlatively by the memory controller and can affect the performance of each other.


% The results in Table \ref{tab:result} demonstrate that our model outperforms most baselines across all the tasks at both sentence and session level, except for MEMN2N(+ match) in task 1.
% 系统询问slot约束的时候,需要按照固定顺序来,比如口味->价格->人数->
% The reason may be that in the task setting, if the user doesn't provide enough values, the system should ask for a value in a fixed order.
% 


\begin{table}[htbp]
\small
\begin{center}
\begin{tabular}{ c | c | c c | c c | c c | c c | c c }
  \hline
    \multicolumn{2}{c|}{\bf Task} & \multicolumn{2}{c|}{1} & \multicolumn{2}{c|}{2} & \multicolumn{2}{c|}{3} & \multicolumn{2}{c|}{4} & \multicolumn{2}{c}{5}\\
  \hline
  \multirow{4}{*}{\bf DA type} & MAD-SM & 93.9 & (65.9) & 100 & (100) & 95.6 & (58.2) & 100 & (100) & 90.9 & (11.9) \\
%   \cline{2-12}
    & MAD-EM & 95.8 & (80.5) & 65.7 & (3.5) & 56.3 & (5.8) & 100 & (100) & 17.8 & (0) \\
%   \cline{2-12}
    & MAD-Attn & {\bf 99.5} & (96.9) & 100 & (100) & 99.0 & (90.3) & 100 & (100) & {\bf 99.9} & (98.6) \\
%   \cline{2-12}
    & MAD & 99.0 & (94.2) & 100 & (100) & {\bf 99.1} & (90.6) & 100 & (100) & 99.9 & (97.8) \\
  \hline
  \multirow{4}{*}{\bf slot-value} & MAD-SM & 21.1 & (0.3) & 22.3 & (0) & 18.4 & (0) & 40.3 & (0.1) & 20.9 & (0) \\
%   \cline{2-12}
    & MAD-EM & 100 & (100) & 95.3 & (65.8) & 27.5 & (0.1) & 100 & (100) & 22.6 & (0) \\
%   \cline{2-12}
    & MAD-Attn & 26.8 & (0.5) & 24.8 & (0) & 27.5 & (0) & 41.3 & (0.1) & 31.4 & (0) \\
%   \cline{2-12}
    & MAD & 100 & (100) & 100 & (100) & 100 & (100) & 100 & (100) & 100 & (100) \\
  \hline
  \multirow{4}{*}{\bf mask} & MAD-SM & 1.0 & (1.0) & 100 & (100) & 99.9 & (99.9) & 100 & (100) & 98.8 & (6) \\
%   \cline{2-12}
    & MAD-EM & 99.1 & (88.8) & 87.8 & (2.6) & 87.8 & (16.4) & 100 & (100) & 66.8 & (0) \\
%   \cline{2-12}
    & MAD-Attn & 100 & (100) & 100 & (100) & 100 & (100) & 100 & (100) & 100 & (100) \\
%   \cline{2-12}
    & MAD & 100 & (100) & 100 & (100) & 100 & (100) & 100 & (100) & 100 & (100) \\
  \hline
  \multirow{4}{*}{\bf Overall} & MAD-SM & 77.2 & (0.2) & 78.9 & (0) & 70.7 & (0) & 57.3 & (0.1) & 59.6 & (0) \\
%   \cline{2-12}
    & MAD-EM & 95.2 & (78.2) & 57.4 & (0.2) & 40.5 & (0.0) & 1.0 & (1.0) & 3.1 & (0.0)\\
%   \cline{2-12}
    & MAD-Attn & 82.7 & (0.5) & 79.0 & (0) & 73.9 & (0) & 57.3 & (0.1) & 67.7 & (0) \\
%   \cline{2-12}
    & MAD & {\bf 99.0} & (94.2) & 100 & (100) & {\bf 99.1} & (90.6) & 100 & (100) & {\bf 99.9} & (97.8)\\
  \hline
\end{tabular}

% \begin{tabular}{ c | c | c c | c c | c c | c c | c c }
%   \hline
%     \multicolumn{2}{c|}{\bf Task} & \multicolumn{2}{c|}{1} & \multicolumn{2}{c|}{2} & \multicolumn{2}{c|}{3} & \multicolumn{2}{c|}{4} & \multicolumn{2}{c}{5}\\
%   \hline
%     \multirow{3}{*}{\bf MAD-SM} & DA type & 93.9 & (65.9) & 100 & (100) & 95.6 & (58.2) & 100 & (100) & 90.9 & (11.9) \\
%   \cline{2-12}
%     & slot-value & 21.1 & (0.3) & 22.3 & (0) & 18.4 & (0) & 40.3 & (0.1) & 20.9 & (0) \\
%   \cline{2-12}
%     & mask & 1.0 & (1.0) & 100 & (100) & 99.9 & (99.9) & 100 & (100) & 98.8 & (6) \\
%   \hline
%     \multirow{3}{*}{\bf MAD-EM} & DA type & 95.8 & (80.5) & 65.7 & (3.5) & 56.3 & (5.8) & 100 & (100) & 17.8 & (0) \\
%   \cline{2-12}
%     & slot-value & 100 & (100) & 95.3 & (65.8) & 27.5 & (0.1) & 100 & (100) & 22.6 & (0) \\
%   \cline{2-12}
%     & mask & 99.1 & (88.8) & 87.8 & (2.6) & 87.8 & (16.4) & 100 & (100) & 66.8 & (0) \\
%   \hline
%     \multirow{3}{*}{\bf MAD-Attn} & DA type & 99.5 & (96.9) & 100 & (100) & 99.0 & (90.3) & 100 & (100) & 99.9 & (98.6) \\
%   \cline{2-12}
%     & slot-value & 26.8 & (0.5) & 24.8 & (0) & 27.5 & (0) & 41.3 & (0.1) & 31.4 & (0) \\
%   \cline{2-12}
%     & mask & 100 & (100) & 100 & (100) & 100 & (100) & 100 & (100) & 100 & (100) \\
%   \hline
%     \multirow{3}{*}{\bf MAD} & DA type & 99.0 & (94.2) & 100 & (100) & 99.1 & (90.6) & 100 & (100) & 99.9 & (97.8) \\
%   \cline{2-12}
%     & slot-value & 100 & (100) & 100 & (100) & 100 & (100) & 100 & (100) & 100 & (100) \\
%   \cline{2-12}
%     & mask & 100 & (100) & 100 & (100) & 100 & (100) & 100 & (100) & 100 & (100) \\
%   \hline
% \end{tabular}

\end{center}
\caption{\label{tab:subtask}Fine-grained performance on the DMBD dataset. We tested the performance of our proposed model and three of its variations on both turn and session level, where for each model the dialogue act type, slot-value, mask and overall prediction accuracy on each task is reported. The highest accuracy on turn level which is lower than 100\% is in bold font.}
\end{table}


% %%%%%%%%%% 下面的是DSTC实验
\subsection{Performance on DM-DSTC}
\label{sec:per-dm-dstc}
Although our proposed model obtains good results on DMBD, it should be noted that the performance reflected by the above results are somehow optimistic due to two facts: First, these dialogues are generated by rules, which are much simpler than real dialogue data. Second, the number of slots and values in DMBD is quite small, while in real applications the number may become very large.

To assess the performance of our proposed model on real dialogue data, we conducted another experiment on DM-DSTC. Different from DMBD, there is only one task in the DM-DSTC dataset. We only reported the results of the methods which predict dialogue act as output. It should be pointed out that in this new dataset, many values in dialogue act annotation didn't appear exactly in user utterances (such as {\em asian oriental}), thus for those values we can not provide precise attention supervision, which will affect the performance of slot-level attention. 
%% 这个Name到底有什么用,需不需要说,前面也有这个问题
Moreover, the {\em Res\_name} slot in this dataset degrades the accuracy because its value does not appear in the dialogue context, and is queried from a knowledge base conditioned on previous search constraints, which is not consistent with our model setting. We reported the fine-grained and overall accuracy at the turn level and session level, as shown in Table \ref{tab:dstc}.

% The positive effect of external memory is also shown in both Table \ref{tab:result} and \ref{tab:dstc}, comparing {\bf MAD}-EM with {\bf MAD} model.
% It is because the external memory enhanced the capability for representing history information.


\begin{table}[htbp]
\small
\begin{center}
\begin{tabular}{ c | c c | c c | c c | c c}
  \hline
    Metrics & \multicolumn{2}{c|}{DA type} & \multicolumn{2}{c|}{slot-value} & \multicolumn{2}{c|}{mask} & \multicolumn{2}{c}{All}\\
  \hline
    MEM & 62.5 & (9.9) & 14.2 & (0.0) & 71.0 & (0.1) & 0 & (0.0)\\   
    RNN & 50.9 & (0.3) & 14.3 & (0.1) & 61.8 & (0.3) & 0.1 & (0.0) \\
    MAD-SM & 64.1 & (13.6) & 11.6 & (0.1) & 81.6 & (0.4) & 17.1 & (0.1)\\
    MAD-Attn & {\bf 64.6} & (12.5) & 18.5 & (0.1) & 80.8 & (1.0) & 16.9 & (0.0)\\
    MAD-EM & 44.9 & (2.3) & 17.5 & (0.1) & 69.7 & (0) & 5.7 & (0.0)\\
  \hline
    {\bf MAD} & 63.8 & (11.0) & {\bf 27.3} & (0.1) & {\bf 82.1} & (1.3) & {\bf 18.8} & (0)\\ %TODO 等试验跑出来更新
  \hline
\end{tabular}
\end{center}
\caption{\label{tab:dstc}Fine-grained and overall accuracy on the DM-DSTC dataset. The number in brackets are the accuracy at the session level, and number without brackets are at the turn level.}
\end{table}
%% MEM模型改为用一个MEMN2N预测所有slot,现在是给每个slot分配一个MEMN2N
%% 看了这个数据,审稿人最重要的质疑是slot value 为什么MEM那么高;咱们的低




The results in Table \ref{tab:dstc} demonstrate our model is still comparable to the vanilla memory network model. Compared to MEM and RNN, our proposed method obtains higher accuracy on turn-level overall prediction, as well as the dialogue act type and mask prediction.
Although MEM's accuracy on DA type , slot-value and mask prediction is slightly lower than ours, its overall accuracy on turn-level is far less than our proposed model. This can be attributed to the framework of MEM, where its DA type, mask and slot-value prediction is trained separately, while in our model these three tasks are trained.
For the variants of MAD, the experiment results are consistent with what we observed in DMBD. MAD-SM obtains lower accuracy on slot-value prediction compared to MAD, while maintains similar accuracy on DA type and mask. For MAD-Attn, the result is similar to MAD-SM when compared to MAD, but its accuracy on slot-value prediction is obviously higher than that of MAD-SM since it maintains the slot-value memory network. MAD-EM, which removes the external memory, obtains significantly lower accuracy on the prediction of dialogue act type and mask, and its accuracy on slot-value prediction is also reduced.

We can see that the performance of slot-value prediction is the bottleneck of promoting overall accuracy. That can be attributed to the data feature of DM-DSTC, where many values of slots does not appear precisely in the user utterance, which makes it hard to acquire accurate attention supervision, thus the model's capacity of extracting semantic features from user utterance is negatively influenced. For the prediction of DA type and mask, although the result is far better than that of slot-value, the accuracy is still not so high as that in DMBD. This can be attributed to the characteristics of real-world data, where there exists much more probability uncertainty and noise than DMBD. More specifically, in different sessions, the DA type of agent response varies much even it is given the same dialogue context. What's more, the agent response in original DSTC2 dataset is conditioned on the knowledge base query result which is not provided, and this also restricts our model's ability on predicting DA type and mask.
%More specifically, the dialogue between user and agent does not always follow a fixed pattern like that in DMBD, sometimes the custom service people can even take different dialogue actions given the same dialogue context.

%However the classification of {\em DA type} and {\mask}, defined in Section \ref{sec:prediction}, are both based on $S_t$ and $M_t^E$.
% 可能是因为st和me存储的信息足够做mask这样的二分类,但是做da_type这样的多分类还不够
%The reason may be that the information stored in both state and externam memory is sufficient for simple binary-classification like predicting mask, but not for multi-class classification, such as the prediction of {\em DA type}.


% 删掉,用上面老师写的描述
% Those model are evaluated on original DBD dataset, in which the outputs are chosen from answer set.
% While the answer sentence are composed mainly by DA type key words and slot values such as {\em api\_call~indian~madrid~six~expensive}.
% Hence the complexity of the original end-to-end dialogue tasks and our dialogue management tasks are basically the same,

%%上面两个能抄一下么?no type,type的区别都要说清楚啊
% MEMN2N is the end-to-end memory networks \cite{sukhbaatar2015end}, which solves the dialog task in a machine comprehension-like way.
%%必须要逐个写上每个baseline,加引用;可以简单一两句,把别人的论文抄过来

% The results in Table \ref{tab:result} demonstrate that our model outperforms the baselines in value and mask prediction by getting 1.00 accuracy on each tasks.
% It indicates that the question addressing and 

% The value prediction depends directly on value memory, which is controlled mainly by question addressing, which selects the most relevant value for specific slot; and slot addressing, controlling the amount of data to be updated.
% Under ideal conditions, if a value $w_j$ that belongs to slot $i$ appears in user utterance, its word vector should be directly assigned to $M_t^V(i)$.


% 在阿里数据集上的实验,
% TODO: 暂时简称ALDM
\subsection{Performance on ALDM}
% TODO 补充memdm三个变种baseline的数据
\begin{table}[htbp]
\small
\begin{center}
\begin{tabular}{ c | c c | c c | c c | c c}
  \hline
    Metrics & \multicolumn{2}{c|}{DA type} & \multicolumn{2}{c|}{Slot-value} & \multicolumn{2}{c|}{Mask} & \multicolumn{2}{c}{All}\\
  \hline
    MEM & 64.9 & (1.4) & 73.5 & (0.0) & 100.0 & (100.0) & 0.0 & (0.0)\\
    RNN & 60.0 & (0.0) & 80.0 & (0.0) & 100.0 & (100.0) & 40.0 & (0.0)\\
    MAD-SM & 60.3 & (0.0) & 80.0 & (0.0) & 100.0 & (100.0) & 40.3 & (0.0)\\
    MAD-Attn & 76.4 & (15.7) & 100.0 & (100.0) & 100.0 & (100.0) & 76.4 & (17.1)\\
    MAD-EM & 76.4 & (15.4) & 98.6 & (92.8) & 100.0 & (100.0) & 74.9 & (14.2)\\
  \hline
    {\bf MAD} & {\bf 76.7} & (16.3) & 100.0 & (100.0) & 100.0 & (100.0) & {\bf 76.7} & (16.3)\\
  \hline
\end{tabular}
\end{center}
\caption{\label{tab:aldm}Fine-grained and overall accuracy on the ALDM dataset. The number in 
bracket is the accuracy at the session level, and the number without bracket is at the turn level.}
\end{table}

We reported the results of the methods which can output a structured dialogue act as we did in Section \ref{sec:per-dm-dstc}. The mask prediction is relatively simple for ALDM in which most of the slot values only appear in the last system response, and thus all the models have an accuracy of 100\%. Therefore, the following analysis will be focused on the DA type and slot-value.

A difference of ALDM compared to the other two datasets is that ALDM is more system-driven, which makes it hard for our model to correctly predict the order of {\em ask\_} DA type, %%有哪些ask DA 啊?不列出来怎么知道order是什么概念呢?
For instance, {\em ask\_dep\_Loc} is only based on the currently filled slots. If the departure location is provided by the user, the system can ask for either the arrive location or the departure date in the next turn, which makes the next DA type difficult to predict. Thus the DA type accuracy is not as good as that in DMBD. However, when $N-1$ slots is already filled (N is the total number of slots to complete a booking task), the next slot to be asked is determinate. Thus, the dialogue state still has impact on DA type prediction, which is shown by the results of MAD-SM and RNN in which the two models removed the slot-value memory.

Although the average number of the slot values in ALDM is much larger than that in the other two dataset, we still obtain high slot-value accuracy. This can be attributed to the high data quality of ALDM which is carefully cleaned before training. By removing the slot-value memory (RNN and MAD-SM) we can see that the slot-value accuracy decreases remarkably, which shows the ability of slot-value memory for maintaining dialogue states. As it can be seen from Table \ref{tab:aldm}, the slot-value accuracy of our full model is the same as that of {\em MAD-Attn}. This is because of the nature of the ALDM dataset that the user responses are mainly one-word sentences, which makes no difference between the models with/without attention mechanism.


% 对于departure city和arrive city两个slot的分析
\begin{table}[htbp]
\small
\begin{center}
\begin{tabular}{ c | c c }
  \hline
    Metrics & {Departure-City} & {Arrive-City}\\
  \hline
    MEM & 2.7 & 4.1  \\ 
    RNN & 0.2  & 0.1  \\
    MAD-SM & 0.5 & 0.3 \\
    MAD-Attn & 100.0  & 100.0  \\
    MAD-EM & 96.5  & 96.2  \\
  \hline
    {\bf MAD} & 100.0  & 100.0 \\
  \hline
\end{tabular}
\end{center}
\caption{\label{tab:departure-arrive}Prediction accuracy on the departure city and arrive city slots. The number in bracket is the accuracy at the session level, and that without bracket at the turn level.}
%%why the others have extremely low scores??审稿人一定会问的!!!
\end{table}

To verify the model's ability to combine context information in slot filling, we further analyzed the prediction accuracy on the {\em Departure\_City} slot and the {\em Arrive\_City} slot. As described in Section \ref{sec:aldm-dataset}, they share the same value list. The ability of identifying values from different slots is mainly controlled by the update gate ${\beta}_t^i$ as defined in Section \ref{sec:sv-memory}. Slot-value memory dominates the prediction of the next slot values, which can be seen from the results of MAD-SM, RNN, and MEM in Table  \ref{tab:departure-arrive}. The results drop dramatically when removing the slot-value memory (RNN and MAD-EM).
For MEM, although its accuracy is higher than that of RNN and MAD-EM, it's still much lower than our proposed model. This is because that 1) the city number is too large for MEM to predict, and 2) MEM fails to identify which slot the value belongs to.


\subsection{Parameter Tuning}

Generally speaking, the performance of neural network models is highly correlated with the number of parameters. There are many important hyper-parameters in our model, including the dimensions of the slot-value memory and external memory, and the number of column vectors in the external memory. We evaluated the influence of these hyper-parameters on performance. The following experiments were performed on the DM-DSTC dataset.


% column of external memory - acc
\begin{figure}
  \includegraphics[width=0.6\textwidth]{fig/n_e.jpg}
  \caption{Fine-grained prediction accuracy on DMBD with different $n_e$ (the number of column vectors in the external memory). The optimal number is 8.
  }
  \label{fig:n_e}
\end{figure}


First, we studied how the performance is influenced by the number of column vectors in the external memory $n_e$. The number $n_e$ varies from 3 to 9, with a step size of 1. We studied the accuracy change on dialogue act type, slot-value, and mask, as shown in Figure \ref{fig:n_e}. For predicting dialogue act type and mask, the optimal $n_e$ is 8 and the optimal accuracy is significantly better than others. For predicting slot-values, although the optimal $n_e$ is 4 with an accuracy of 0.331, the accuracy is almost the same (from 0.321 to 0.331) when varying $n_e$ from 4 to 8. 


\begin{figure}
  \includegraphics[width=0.6\textwidth]{fig/em_dim.jpg}
  \caption{Accuracy change on DMBD with different dimensions of the column vectors in the external memory. The optimal number is 128.
  }
  \label{fig:dimension}
\end{figure}


Second, we studied the influence of the dimension of column vectors, as shown in Figure. \ref{fig:dimension}. The dimension number in our experiment ranges from 32 to 256 with a step size of 32. The accuracy of dialogue act type and mask is highly correlated, whose best accuracy are both obtained with the dimension of 128. While the optimal value for slot-value accuracy is obtained with the dimension of 64.

%% 是不是跟前面说的自相矛盾,还是不要的好 %%而且写得也不好
%%As aforementioned, the slot-value memory and external memory is designed to augment the memory capacity of recurrent networks. As can be seen, increasing the number of column vectors or the dimension of memory units is supposed to increase the memory capacity of the model. The above experiments show that the prediction of slot-value prefers smaller external memory, while dialogue act type and mask prefer larger. This phenomenon can be explained based on the architecture of our model. In our framework, information save in slot-value memory is more specific, while the information in external memory is not that specific, thus there may be overlap between them. As described in Figure. \ref{fig:model}, the prediction of system dialogue act is based on controller state, slot-value memory and external memory, while different aspects attend on different modules. The act type and mask is predicted on external memory as defined in Eq. \ref{predict:dat} and Eq. \ref{predict:mask}, and slot-value is predicted based on slot-value memory, seen in Eq. \ref{predict:sv}. When we increase the parameter amount in external memory, the training of slot-value memory can be damaged, which results in a lower accuracy on slot-value prediction. For the other two aspects, higher accuracy can be obtained with more information.

%% 我觉得分析一下也很好,但是 先把自己的宏观逻辑理清楚!
%%1. 要强调 有memory比 没有好 (之前的哪些结果)
%%2. 要强调 memory不是越多越好,也不是越少越好;然后说为什么 


\subsection{Visualization Analysis}
% attention可视化展示
Figure \ref{fig:attention} illustrates an example of the slot-level attention mechanism. 
For each slot, the model generates a distribution over the words of an utterance. Each row is thus a probability distribution over words, where the largest probability corresponds to the word that should be attended mostly. 
%For example we obtained an ideal attention result.
For utterance {\em "can you book a table with British cuisine for six people in Madrid in an expensive price range"}, 
for slot {\em Cuisine}, the most attended word is {\em British}, while for slot {\em Price }, the word is {\em expensive}, and for slot {\em Number}, the word is {\em six}. 
Note that the weight of {\em $<Rating,british>$} is also large, which is wrong intuitively in that Rating information has not yet been mentioned. However, this kind of wrong attention weight does not have influence on model performance. 
% To alleviate the effect of such errors on predicting the next dialogue act, we further have a slot mask to control which slot should be included in the next dialogue act.
In other words, the inclusion of a slot-value pair in the predicted dialogue act is decided by two distributions: the value distribution and the slot mask distribution for a slot, as mentioned in Section \ref{sec:prediction}. The effect of faulty attention will be filtered out by mask when deciding which slots are to be addressed in final dialogue act. %....%%指向方法部分.


\begin{figure}
\centering
\includegraphics[width=\textwidth]{fig/attention.png}
\caption{Attention visualization. For each slot, the attention weights (in a row) are a distribution over the words of an utterance. For utterance {\em "can you book a table with british cuisine for six people in madrid in an expensive price range."} the predicted slot-value pairs are $<cuisine, british>$, $<number, six>$, $<location,madrid>$, and $<price, expensive>$. }
\label{fig:attention}
\end{figure}


% dialog example 展示
Figure \ref{fig:full-dialog} illustrates the change of the dialogue state and the predicted next dialogue act in an exemplar dialogue session. We visualized the values stored in the slot-value memory and shown the next dialogue act type predicted by the model.
At each turn, the model computes an update gate $\beta_t^i$ (Eq.\ref{eq:updategate}) for each slot $i$.
If a certain value of slot $i$ appears in user utterance $x_t$, $\beta_t^i$ increases, and the color of the corresponding cell becomes darker.
The darkness of a cell represents the value of $\beta_t^i \in [0, 1]$, which is calculated independently for each slot $i$ at each turn $t$.
The value in each cell is computed by Eq. \ref{predict:sv} and we only output the value for slot $i$ if $\beta_{\tau}^i > 0.5$ for some turn $\tau$. 
These values compose a search constraint at each turn.
%Previous values are predicted mostly based on initial $M^V$ values and are meaningless.
In the exemplar dialogue session, each value in user utterance is captured by the attention mechanism of a user utterance, 
and its values are filled into $M^V$ with large $\beta_t^i$s.

For instance, when the user asks {\em can you book a table in a cheap price range in london?}, the {\em price} slot is filled with the value of {\em cheap }, and the {\em location} slot is filled with the value of {\em london}. The model predicts the next dialogue act {\em ask\_cuisine} which prompts the user on the preference of {\em cuisine}. As the user supplied new information with the utterance {\em with french food}, the {\em cuisine} slot is filled with the value of {\em french}. At this state, the model predicts the next dialogue act  {\em ask\_people} which should ask the user about how many people are involved. As the dialogue proceeds, the slot-value memory explicitly tracks the dialogue state, and the next dialogue act is also predicted according to the state. 

\begin{figure}
  \small
  \renewcommand\floatpagefraction{.9}   
\renewcommand\topfraction{.9}  
\renewcommand\bottomfraction{.9}  
\renewcommand\textfraction{.1}
  \includegraphics[width=0.8\textwidth]{fig/dialog-example.png}
  \caption{An example of DA prediction for a dialogue session. $\bm{x}$ represents user utterance and ${y}$ system response.The values of slots at each turn are predicted by Eq. \ref{predict:sv}. The color darkness of each cell represents the value of ${\beta}_t^i$ defined in Eq. \ref{eq:updategate}. Darker colors indicate larger values.}
  \label{fig:full-dialog}
\end{figure}


% \section{Discussion}


%%%上面几段话,距离写得很清楚还有很大距离;读者需要费很多功夫去理解,每个步骤应该是讲清楚物理意义。才能让人follow

\section{Conclusion}
In this paper, we present a memory augmented dialogue management model for capturing long-range dialogue semantics by explicitly memorizing and updating the dialogue act types and slot-value pairs during interactions in task-oriented dialogue systems. 
The model employs two memory modules, namely the slot-value memory and external memory, to address the history semantics during the entire dialogue session. The slot-value memory tracks the dialogue state by memorizing and updating the values of semantic slots, and the external memory augments the single state representation of RNN by storing more context information.
We also propose a slot-level attention mechanism for attentive read of a user utterance to update the slot-value memory. The attention mechanism helps to extract the slot-related information that is addressed in a user utterance. 
Through the attention mechanism and the memory modules, our proposed model can better interpret the dialogue context in a more observable and explainable way, which also helps to predict the next dialogue act given the current dialogue state.
Results show that our model is better than the state-of-the-art baselines, and moreover, the model can offer more observable dialogue semantics by presenting
predicted slot-value pairs at each dialogue turn.
We believe that research on interactive IR may benefit from our work, particularly from the idea of enhancing the interpretability of dialogue management.

%Future work will be conducted to analyze the semantic information stored in the external memory, reducing the semantic overlap between slot-value and external memory.


% Bibliography
\bibliographystyle{ACM-Reference-Format}
\bibliography{sample-bibliography}


Reinforcement learning has achieved great success in areas such as Game-playing \citep{silver2018general,vinyals2019grandmaster}, robotics \cite{kober2013reinforcement}, large language models \citep{ouyang2022training}, etc.
However, due to safety concerns or physical limitations, in some real-world reinforcement learning problems, we must consider additional constraints that may influence the optimal policy and the learning process \citep{garcia2015comprehensive}.
% For example, a robotic arm must not take actions that may cause harm to itself or the environments.
A standard framework to handle such cases is the constrained Markov Decision Process (CMDP) \citep{altman1999constrained}.
Within the CMDP framework, the agent has to maximize
the expected cumulative reward while
obeying a finite number of constraints, which are usually in the form of expected cumulative cost criteria.

However, we are sometimes concerned with the problem with a continuum of constraints.
For example,
the constraints we meet might be time-evolving or subject to uncertain parameters, which
cannot be formulated as an ordinary CMDP
(see Examples \ref{Example_Time_Evolving} and  \ref{Example_Uncertain}).
In this paper we would study a generalized CMDP  
to address the above problem.  Because the constraints are not only infinite-number but also lie
in a continuous set,
the generalization is not trivial. Fortunately, we find that we can borrow the idea behind semi-infinite programming (SIP) \citep{remez1934determination, hettich1993semi} to deal with the semi-infinite constraints.
Accordingly, we propose \emph{semi-infinitely constrained Markov decision processes} (SICMDPs)
as a novel complement to the ordinary CMDP framework.
%More specifically,  an SICMDP model %, we consider 
%contains a continuum of constraints whereas an ordinary CMDP contains a finite number of constraints. 

%This generalization is natural but not trivial. However, we can brows the idea  
%The idea is quite natural and can be backtracked
%to the practice of extending linear programming to linear semi-infinite programming (LSIP) %\cite{remez1934determination, GobernaLSIO1998}.
%In addition, 
%As a complementary approach to the ordinary CMDP framework, 
%SICMDP can be used to model these problems  which cannot be described by a finite number of constraints
%that are not covered by .
%For example,
%the restrictions we consider can be time-evolving or subject to uncertain parameters
%, thus
%cannot be described by a finite number of constraints but a continuum of constraints 
%(see Examples \ref{Example_Time_Evolving} and  \ref{Example_Uncertain}).

We also present two reinforcement learning algorithms to solve SICMDPs called SI-CRL and SI-CPO, respectively.
SI-CRL is a model-based reinforcement learning algorithm designed for tabular cases, and SI-CPO is a policy optimization algorithm for non-tabular cases.
% and analyze its performance both theoretically and empirically.
The main challenge is that we need to deal with a continuum of constraints, thus reinforcement learning algorithms for ordinary CMDPs do not work anymore.
In SI-CRL, we tackle this difficulty by first transforming the reinforcement learning problem to an equivalent LSIP problem, which can then be solved using methods in the LSIP literature like the dual exchange methods \citep{Hu1990,reemtsen1998numerical}.
In SI-CPO, we resort to the idea of cooperative stochastic approximation developed in \cite{lan2020algorithms, wei2020comirror}.
As far as we know, we are the first to introduce tools from semi-infinitely programming (SIP) into the reinforcement learning community for solving constrained reinforcement learning problems.

% To the best of our knowledge, we are the first to apply tools from semi-infinitely programming (SIP) to solve reinforcement learning problems.
Furthermore, we give theoretical analysis for both SI-CRL and SI-CPO.
We decompose the error of SI-CRL into two parts: the statistical error from approximating the true SICMDP with an offline dataset and the optimization error due to the fact that the solution of the LSIP problem obtained by the dual exchange method is inexact.
On the optimization side, we show that the iteration complexity of SI-CRL is $O\left(\left\{\mathrm{diam}(Y)L\sqrt{|\gS|^2|\gA|m}/\left[(1-\gamma)\epsilon\right]\right\}^m\right)$.
On the statistical side, we show that the sample complexity of SI-CRL is $\widetilde O\left(\frac{|S|^2|A|^2}{\epsilon^2(1-\gamma)^3}\right)$ if the offline dataset is generated by a generative model, and $\widetilde O\left(\frac{|S||A|}{\nu_{\min} \epsilon^2(1-\gamma)^3}\right)$ if the dataset is generated by a probability measure $\nu$ as considered in \cite{chen2019information}.
Here $\widetilde O$ means that all logarithm terms are discarded.
For SI-CPO, things become a little more complicated because other than the statistical error and the optimization error, we also need to consider the function approximation error, which comes from imperfect policy parametrizations.
It is shown if the function approximation error can be controlled to $O(\epsilon)$ order, the iteration complexity of SI-CPO is $\widetilde{O}\left(\frac{1}{\epsilon^2(1-\gamma)^6}\right)$ and the sample complexity of SI-CPO is $\widetilde{O}(\frac{1}{\epsilon^4(1-\gamma)^{10}})$.
Here our iteration complexity bound is equivalent to a typical $\widetilde O(1/\sqrt{T})$ global convergence rate.

We perform a set of numerical experiments to illustrate the SICMDP model and validate our proposed algorithms.
Specifically, we examine two numerical examples, namely the discharge of sewage and ship route planning.
Through the discharge of sewage example, we show the advantage of the SICMDP framework over the CMDP baseline obtained by naive discretization in modeling realistic sequential decision-making problems.
Moreover, we demonstrate the effectiveness of the SI-CRL and SI-CPO algorithms in such tabular environments. 
In the ship route planning example, we illustrate the benefits of the SICMDP framework and the ability of the SI-CPO algorithm to address complex continuous control tasks involving continuous state spaces with modern deep reinforcement learning techniques.

% In summary, our contributions are listed as follows.
% First, we present the SICMDP model, which can be viewed as a generalization of the ordinary CMDP model.
% Second, we propose an algorithm to perform reinforcement learning for SICMDPs, which is called SI-CRL, and we believe that we are the first to apply tools from SIP
% to solve reinforcement learning problems.
% Third, we give a theoretical analysis of SI-CRL and identify both its sample complexity and iteration complexity.
% In addition, we perform numerical experiments to illustrate the SICMDP model and validate the SI-CRL algorithm.
% \{This paragraph can be removed!!! \}




         % This file has Introduction section
\section{Motivations for Empirical Study}
\label{sec:motivations}
The key question that we try to answer is when and why we should use standard
iteration space tiling over cache oblivious tiling.  The two approaches
perform similar partitioning of the iteration space, but the schedules given
to the partitions are different.  Theoretically, cache oblivious code seems to
have advantages over iteration space tiling.  However, many factors complicate
the actual performance, which made our initial experiments difficult to
interpret.  In this section, we describe the obstacles between the theory and
practice we have identified.

We use Single-Level Tiling (SLT) for iteration space tiling, and Cache
Oblivious Tiling (COT) for cache oblivious techniques in this
paper, which are further described in Section~\ref{sec:background}.

\paragraph{Recursion Overhead} This is a well-known overhead of
COT~\cite{yotov2007experimental}.  The recursion introduces overheads, such as
function call overhead, and increased register pressure.  Furthemore, the
functions force inter-procedural analysis/optimization, known to be more
difficult for compilers well.  Thus, the leaf tiles must be ``sufficiently
large'' to avoid excessive overhead due to the recursion.

 \paragraph{Recursive Split Constraints the Tile Sizes} In typical cache
 oblivious algorithms, the problem is recursively split into halves in each
 dimension. This is in fact a rather coarse-grained exploration of the
 hierarchical partitioning of the iteration space. For instance, if the
 current problem size is $B^3$, then the next sub-problem would be
 $(\frac{B}{2})^3$.  If the best problem size for utilizing a level of cache
 is $(B-x)^3$ where $x\ll \frac{B}{2}$ then the subproblems due to
 divide-and-conquer will not match the best.  This is another factor that
 necessitates fine tuning of leaf tile sizes even for COT, since the utilization
 rate of L1 cache has strong impact on performance.  

%\paragraph{COT Leads to Imbalanced Tiles} Current COT tools recursively split
%the problem into halves in each dimension.  If the original bounds are not
%powers of two, every power-of-two leaf will be paired with a non-power-of-two
%leaf.  Since leaf tile sizes are often carefully tuned, thismeans that half
%the leaves will be suboptimal.  Our code generator incorporates a simple
%optimization that ensures that such suboptimal leaf nodes only occur at the
%boundaries of the iteration space.

\paragraph{COT has more Conflict Misses} The divide-and-conquer execution
order may negatively affect cache interference, especially with high
dimensional data.  This happens when the memory is allocated such that the
accesses are contiguous along some direction in the iteration space (typically
along innermost canonical axis).  With lexicographic order of execution, this
contiguity is largely preserved in the tiled execution.  However,
divide-and-conquer executes neighboring tiles in all dimensions, and many of
those tiles access some distant location in memory.  In contrast to accessing
contiguous regions of memory, accessing various segments of the memory
increases the chances of conflicts.

\paragraph{Hardware Prefetching}  Modern architectures are equipped with
hardware prefetchers that can bring data to the L1 cache. When
having sufficient locality at L2 or LLC makes the program compute-bound, then
the latency to L2/LLC can be hidden by the prefetcher. For such programs, it is
unnecessary to tile for the fastest cache, and larger tiles targeting slower
caches improve performance by maximizing prefetcher
effectiveness~\cite{mehta2016turbotiling}. When the primary objective is speed,
the leaf tiles for COT should also be large, which negates the benefit of
divide-and-conquer, as the leafs are already targeting slower caches.
Prefetching have little impact on parallel executions, since prefetching is
bandwidth limited. When multiple cores try to prefetch at the same time,
the bandwidth limit is quickly reached, and the latency hiding effect is
lost. Furthermore, smaller tile sizes are better for parallel execution for
load balancing  reasons.


These factors limit the effectiveness of COT in various ways and are also
closely tied to the characteristics of the computation. Our empirical study
illustrate the impact of these factors on polyhedral computations.

% Local Variables: ***
% TeX-master: "TACO2017.tex" ***
% fill-column: 78 ***
% End: ***
 

% Panoptic segmentation

% 3D segmentation

% Multi-object tracking

% Online 3D panoptic:

% PanopticFusion: (IROS 2019)
% https://arxiv.org/pdf/1903.01177.pdf
%
% - most similar to ours
% - PSPNet + M-RCNN + 2D fusion
% - volumetric mapping, 
% - greedy matching with IoU -> optimal only with 0.5 threshold
% - voxel & class weighting
% - CRF regularisation
%
% - good:
%
% - bad:
%  - CRF post-processing step
%  - greedy data-association
%    - can't be tuned for lower overlap ratios -> has to have high framerate, large changes in viewpoint could break this
%    - IoU: sensitive to 2D labels projecting over object borders (CRF and voxel weighting seem to alleviate this)

% Voxblox++: (Robotics & automation letters 2019)
% https://arxiv.org/pdf/1903.00268.pdf
% https://github.com/ethz-asl/voxblox-plusplus
%
% - M-RCNN + geometric segmentation + fusion 
% - data association of geometric segments with 3D overlap (no. points inside volume), fixed threshold for min number of points
% - instance label is assigned to a segment based on highest overlap
% - only one detected segment per reference label, as in PanopticFusion and Ours
% - TSDF Integration 
%
% good: 
% - because of geometric segmentation objects with no associated semantic class can also be segmented
% bad:
% - two different object segment types -> confusing, overly complicated ?
% - quite inaccurate (fixed below)

% Reconstructing Interactive 3D Scenes by Panoptic Mapping and CAD Model Alignments (ICRA 2021)
% https://arxiv.org/pdf/2103.16095.pdf
% https://github.com/hmz-15/Interactive-Scene-Reconstruction
%
% - based heavily on Voxblox++, much more accurate
% - Scene-graph ("contact graph") for mapping object relations
% - Search & replace voxels with CAD models, with geometrical and physical constraints
% - Object 6D pose
% - Format for robot interaction
%
% - Segmentation: bilateral fusion of geomatric and semantic segments -> reduce segmentation noise compared to Voxblox++
% - Fusion: triplet count improves consistency over Voxblox++ pairwise count strategy (take semantic label into account in addition to instance and geometry)
% - Fusion: instance labels are also combined if there is enough overlap with common geometric label for long enough time
%   - this means multiple detections can match the same reference unlike ours, voxblox++ and PanopticFusion ?
%

% Panoptic-MOPE: (ROBOTICS AND AUTOMATION LETTERS 2020)
% https://ieeexplore.ieee.org/stamp/stamp.jsp?tp=&arnumber=8977356
% https://github.com/hoangcuongbk80/Object-RPE/tree/panoptic-mope
%
% - novel RGB-D semantic segmentation model + M-RCNN
% - camera tracking based on "addaptively weighted optimization of geometric, appearance, and semantic cues"
% - surfel map: 
%   - how does it scale ? authors satate they tested on room-sized environments, but could be applied in larger scale as well ...
%     - could maybe be applied as VO in a SLAM algorithm ...
%   - demo only on a small pallet + surroundings, might not be applicable in large-scale SLAM

% US VS THEM:
%
% - based heavily on PanopticFusion, with modifications:
%   - instead of greedy data-association (which seems to be the case in others as well), we solve LAP (JPDA?)
%     - overlap threshold can be tuned, which renders the algorithm more flexible
%     - could be extended to dynamic tracking ?
%   - multiple options for association likelihood
%   - outlier rejection (either clustering or probabilistic)
%   - test different options for decreasing processing time
%   - no post-processing
%
% - model-agnostic:
%   - completely separated from segmentation
%   - does not care how point clouds are obtained -> applicable for LIDAR segmentation (e.g. EfficientLPS) as well
%
% - also agnostic to localisation method
%   - could, however, be utilised to find landmark locations / poses

% More compact version of this paragraph to introduction to save space?
%Panoptic segmentation -- proposed in \cite{panoptic_segmentation} -- aims to solve the unified task of semantic- and instance segmentation. Semantic classes are separated to \textit{stuff} -- amorphous, unquantifiable regions like sky, road or floor -- and \textit{things} -- quantifiable objects. The distinction between the two can vary depending on the application, but a semantic class can only belong to one or another. The article also proposes a unified panoptic evaluation metric, coined \textbf{Panoptic Quality} (PQ). Many 2D approaches to panoptic segmentation -- \textit{e.g.} \cite{panopticfpn,seamless,panoptic_deeplab,efficientps} -- have since been proposed. Deep neural networks for performing semantic- or instance segmentation directly on the 3D reconstruction -- \textit{e.g.} on \cite{scannet,s3dis,paris_lille_3d} -- have also been proposed, but since they require the reconstructed 3D scene, they are mostly offline approaches and therefore out of scope for this work. Some recent works also apply panoptic segmentation to point clouds -- \textit{e.g.} methods in the SemanticKITTI panoptic segmentation competition \cite{semantic_kitti} -- mostly aimed at segmenting LiDAR output. They are suitable for online processing, but similar to RGB-D images require a method for tracking object instances persistent in both time and space. In fact, our proposed method, as well as some others mentioned in this work, could use segmented LiDAR point clouds as an input similarly to RGB-D images.

PanopticFusion \cite{panopticfusion} is the first work to propose online integration of panoptic image segmentations to a 3D reconstruction. They integrate point clouds generated from segmented images to a TSDF voxel volume \cite{tsdf,voxblox} by greedily matching detected segments with the reconstruction and regulating each voxel's corresponding instance with a weighting function. Semantic labels are inferred in a bayesian manner based on confidence scores provided by the segmentation model. They also apply a Conditional Random Field (CRF) to regularise the reconstruction, improving results significantly. Voxblox++ \cite{voxblox++} -- introduced later the same year -- is a similar approach that also integrates segmented RGB-D images into a TSDF volume. It leverages geometric segmentation of depth images to improve instance segmentation accuracy. Both geometric and semantic segments are used to compute a pair-wise weight, which is used to greedily match them with segments in the reconstruction. Because of the geometric segmentation, the method allows segmentation of objects with no known semantic class in addition to objects recognised by the instance segmentation model. 

Recently, \cite{interactive_3d_scenes} built upon the idea of Voxblox++. They apply Voxblox++ for 3D instance integration, with two small but effective modifications: the pair-wise weight is replaced by a triplet weight that also takes semantic labels into account in the fusion, and -- in addition to geometric segments -- instance segments are fused if they overlap by a significant amount. The article introduces a method for searching and aligning CAD models to reconstructed objects based on geometry and semantic class, as well as geometrical and physical rules. With the CAD models, a contact graph and interactive virtual scene are reconstructed to allow a robot to simulate its interaction with the environment. SceneGraphFusion \cite{scenegraphfusion} is another approach that forms a scene graph online from a stream of RGB-D images, but unlike the above-mentioned approach, it generates the graph with a deep neural network, after which the panoptic labels for geometrically segmented portions of the 3D reconstruction are produced a side product.

Panoptic-MOPE \cite{panoptic_mope} is another recent approach, which integrates sequences of RGB-D images into a surfel reconstruction. Unlike other mentioned approaches -- which assume the camera pose either known or estimated elsewhere -- it also tracks camera movements based on geometric-, appearance- and semantic cues. The method also applies a novel RGB-D panoptic segmentation model. Although it is only tested on room-sized environments, the authors claim it could be scaled to larger environments as well.    % This file has Background section
%\section{SNOW Implementation on TI CC13x0}\label{sec:implementation}
\begin{figure}[!htb]
\centering
\includegraphics[width=0.49\textwidth]{figs/devices-new.eps}
\caption{Devices used in our SNOW implementation. A node is a CC1310 or CC1350 device. The BS has two USRP B200s, each having its own antenna. An antenna is approximately 2x bigger than a B200.}
\label{fig:devices}
\end{figure}
%talk about what devices are used in BS and as nodes
The original SNOW implementation in~\cite{snow_ton} uses the USRP hardware platform for both the BS and the nodes. In our implementation, we use the CC13x0 devices as SNOW nodes and USRP in the BS (Figure~\ref{fig:devices}).
%A USRP B200 device with a half-duplex radio costs approximately \$750 USD, as of today. As such, it becomes costly to deploy the SNOW network and examine its scalability. In contrast, in this work, we realize the functionality of a SNOW node in TI CC1310 LaunchPad~\cite{cc1310} that costs approximately \$30 USD, thus much cheaper and widely available to the research community to develop and deploy SNOW networks. 
For BS implementation, we adopt the open-source code provided in~\cite{snow_bs}. The BS uses two half-duplex USRP devices (Rx-Radio and Tx-Radio), each having its own antenna. Also. the BS is implemented on the GNURadio software platform that gives a high magnitude of freedom to perform baseband signal processing~\cite{gnuradio}.
In the following, we explore a number of implementation considerations and feasibility for a CC13x0 device to work as a SNOW node in practical deployments. 
First, we show how to configure a CC13x0 device to make it work as a SNOW node. We then address the practical challenges (e.g., PAPR problem, CSI estimation, and CFO estimation) associated with our CC13x0-based SNOW implementation.

\subsection{Configuring TI CC13x0}
%talk about how to configure nodes and BS
We configure the subcarrier center frequency, bandwidth, modulation, and the Tx power by setting appropriate values to the CC13x0 command inputs \code{centerFreq, rxBw, modulation}, and \code{txPower}, respectively, using {\em Code Composer Studio} (CCS) provided by Texas Instruments~\cite{snow_cots}. A graphical user interface alternative to CCS is {\em SmartRF Studio}. The MAC protocol of SNOW in CC13x0 is implemented on top of the example CSMA/CA project that comes with CCS. Note that the functionalities of a SNOW node are very simple and may be incorporated easily in the IoT devices that have both storage and computational limitations like the CC13x0 devices.

\subsection{Peak-to-Average Power Ratio Observation}\label{sec:papr}
By transmitting on a large number of subcarriers simultaneously (in the downlink), the BS suffers from a traditional OFDM problem called {\em peak-to-average power ratio (PAPR)}. PAPR of an OFDM signal is defined as the ratio of the maximum instantaneous power to its average power.
In the SNOW downlink communications (i.e., BS to nodes), after the IFFT is performed by the BS, the composite signal can be represented as
$\nonumber x(t) = \frac{1}{\sqrt{N}}\sum_{k=0}^{N-1}X_k~e^{j2 \pi f_k t},~~0 \le t \le NT.$
% \begin{equation}
% \nonumber x(t) = \frac{1}{\sqrt{N}}\sum_{k=0}^{N-1}X_k~e^{j2 \pi f_k t},~~0 \le t \le NT
% \end{equation} 
Here, $X_k$ is the modulated data symbol for node $k = \{0, 1, \cdots, N-1\}$ on subcarrier center frequency $f_k = k\Delta f$, where $\Delta f = \frac{1}{NT}$ and $T$ is the symbol period. Therefore, the PAPR may be calculated as%~\cite{jiang2008overview}
\begin{equation}
\nonumber \text{PAPR}[x(t)] = 10\log_{10}\Bigg( \frac{\max\limits_{0~ \le ~t~ \le~ NT} [|x(t)|^2 ]}{P_{\text{avg}}}\Bigg)~~dB.
\end{equation}
Here, the average power $P_{\text{avg}} = E [|x(t)|^2]$.
A node's signal detection on its subcarrier is very sensitive to the nonlinear signal processing components used in the BS, i.e., the digital-to-analog converter (DAC) and high power amplifier (HPA), which may severely impair the bit error rate (BER) in the nodes due to the induced spectral regrowth. If the HPA does not operate in the linear region with a large power back-off due to high PAPR, the out-of-band power will exceed the specified limit and introduce severe ICI~\cite{jiang2008overview}. Moreover, the in-band distortion (constellation tilting and scattering) due to high PAPR may cause severe performance degradation~\cite{kamali2012understanding}. It has been shown that the PAPR reduction results in significant power saving at the transmitters~\cite{baxley2004power}.
\begin{figure}[!htb]
\centering
\includegraphics[width=0.35\textwidth]{figs/papr/papr.eps}
\caption{PAPR distribution of D-OFDM signal in Tx-Radio.}
\label{fig:papr}
\end{figure}


As shown in Figure~\ref{fig:papr}, the PAPR in the SNOW downlink communications (for N = 64) follows the Gaussian distribution. Thus, the peak signal occurs quite rarely and the transmitted D-OFDM signal will cause the HPA to operate in the nonlinear region, resulting in a very inefficient amplification. To illustrate the power efficiency of the HPA for N = 64, let us assume the probability of the clipped D-OFDM frames is less than 0.01\%. We thus need to apply an input back-off (IBO)~\cite{baxley2004power} equivalent to the PAPR at a probability of $10^{-4}$. Here, PAPR $\approx$ 14dB or 25.12. Thus, the efficiency ($\eta = 0.5/\text{PAPR}$) of the HPA~\cite{jiang2008overview} is $\eta = 0.5/25.12 \approx 1.99\%$. Such low efficiency at the HPA motivates us to explore the high PAPR in SNOW for practical deployments.
%show some results explaining how PARP affects the BS-Node communication
Several uplink PAPR reduction techniques for single-user OFDM systems have been proposed (see survey~\cite{jiang2008overview}). However, the characteristics of the downlink PAPR in SNOW, where different data are concurrently transmitted to different nodes, are entirely different from the PAPR observed in a single-user OFDM system. To adopt an uplink PAPR reduction technique used in the single-user OFDM systems for the downlink PAPR reduction in SNOW, each node has to process the entire data frame transmitted by the BS and then demodulate its own data. However, a SNOW node has less computational power and does not apply FFT to decode its data~\cite{snow_ton}, or any other node's data. Thus, the existing PAPR reduction techniques will not work in our implementation.

%To this extent, we address the PAPR problem in SNOW by allocating a special subcarrier  called {\em downlink subcarrier} for the downlink communications.
We propose to handle the PAPR problem in SNOW by using only one subcarrier (called {\em downlink subcarrier}) for downlink communication. All the nodes use this subcarrier to receive from the BS. Namely, the Tx-Radio transmits only on one subcarrier that is not used by any node for uplink communication.
The BS may send any broadcast message, ACK, or data to the nodes using that downlink subcarrier. A node has to switch to the downlink subcarrier to listen to any broadcast message, ACK, or data.
The BS may reserve multiple subcarriers  as {\slshape backup subcarriers} for downlink communication. 
If the currently used downlink subcarrier becomes overly noisy or unreliable, it can be replaced by a backup subcarrier.
Note that the dual-radio in the BS allows it to receive concurrent packets from a set of nodes (uplink) and transmit broadcast/ACK/data packets to another set of nodes (downlink), simultaneously. 
The BS can acknowledge several nodes using a single transmission by using a bit-vector of size equals to the number of subcarriers.
If the BS receives a packet from a node operating on subcarrier $i$, it will set the $i$-th bit in the bit-vector. Upon receiving the bit-vector, that node may get an ACK by looking at the $i$-th bit of the vector. Because of the bit-vector, the downlink ACKs also scale up like the uplink traffic. In the case of different packets for different nodes, the volume of downlink traffic (compared to the uplink traffic) is also practical since the IoT applications may not require high volume downlink traffic~\cite{whitespaceSurvey}.

%A node retransmits the packet if that packet is not acknowledged in the first ACK received by that node. 
%In the following, we describe our technique below to handle a {\bf rare} case in practical SNOW deployments, and hence may be kept optional in implementation.

%\revise{Let nodes $A$ and $B$ share subcarrier $i$. The BS may receive a packet from node $B$ while preparing the ACK for node $A$'s packet. If both packets are decoded correctly, the BS acknowledges them by setting the $i$-th bit of the vector. However, if only one packet is decoded correctly, the BS resets the $i$-th bit of the vector. Thus, none of the packets are acknowledged. To compensate for this, the BS (Tx-Radio) switches to subcarrier $i$ and sends separate ACKs for nodes $A$ and $B$. On the other hand, if a node finds that its packet is not acknowledged in the downlink subcarrier, it listens to its subcarrier for a short fixed window before attempting a retransmission. A node knows about that fixed window when it joins the network. Note that {\em a very few} nodes (sharing the same subcarrier) may be involved in this scenario since the ACK generation time at the BS is very small. Other ways of addressing this issue may include the use of \emph{hash functions}, which we do not consider due to the scalability issues in hash-related collisions.}

When a node $u$ transmits to the BS, if another node $v$ sharing the same subcarrier wants to transmit, $v$ senses the channel as busy and refrain from transmitting. When the BS transmits ACK to $u$ on the downlink subcarrier using the Tx-Radio, node $v$ may also transmit to the BS. Since the Tx-Radio at that time is making a downlink transmission, it may not send the ACK upon $v$'s transmission immediately. However, the Tx-Radio can send $v$'s ACK immediately after completing its current downlink transmission. Thus, $v$ may need to wait for ACK for a little longer than the time needed to send a downlink transmission from the BS. A node may go to sleep mode or its next state right after receiving an ACK. However, if a node that has transmitted but not yet received ACK, should wait for a little longer (e.g., up to one or two downlink transmission time). Note that a very few nodes (sharing the same subcarrier) may be involved in this scenario since the ACK generation time at the BS is very small. For the same reason, the waiting time for ACK will also not be very long (e.g., up to one or two downlink transmission time). Note that this scenario is quite rare and most of the times the nodes will receive ACK immediately upon transmission.


%When a subcarrier (say, $i$) is shared by multiple nodes, the BS may receive a packet (say, from node A) before transmitting the ACK for another packet (say, from node B). In this case, both nodes A and B may be acknowledged by setting the $i$-th bit of the vector. However, if the packet from node A (or, B) is valid and the packet from node B (or, A) is invalid, the BS will reset the $i$-th bit of the vector and transmit the ACK. Thus, none of the packets are acknowledged even if one of them is valid. To compensate for that, the BS (Tx-Radio) will switch to node A's (or, B's) subcarrier and transmit an ACK packet. Thus, in our implementation, if a node finds that its packet is not acknowledged in the first valid ACK it received, before retransmission it listens to its subcarrier for a fixed amount of time. Each node may know this fixed time when it joins the network. Typically, if a subcarrier is shared by $G$ nodes, the fixed amount of time (worst case) may be set to $GD_p$ (ignoring the frequency switching time in the Tx-Radio), where $D_p$ is the time to transmit one packet. Other ways of addressing such issue may include the use of \emph{hash functions}. However, we do not explore that in our implementation for scalability issue due to hash collision.

%In the case where each subcarrier is assigned to only a node, the size of the bit-vector may be set to the total number of subcarriers. Thus, if the BS receives a packet from a node operating on subcarrier $i$, it will set the $i$-th bit in the bit-vector. Upon receiving the bit-vector in ACK subcarrier, that node can check the $i$-th bit of the vector. However, in practical deployments with thousands of nodes, a subcarrier may be shared by multiple nodes, making the creation of the bit-vector non-trivial. We cannot also use any hashing technique because of the hash collisions and scalability issues. 

\begin{figure*}[!htbp] 
    \centering
      \subfigure[RSSI under varying distance\label{fig:csi_rssi}]{
    \includegraphics[width=0.35\textwidth]{figs/csi/rssi.eps}
      }\hfill
      \subfigure[Path Loss under varying distance\label{fig:csi_pathloss}]{
        \includegraphics[width=.35\textwidth]{figs/csi/pathloss.eps}
      }\hfill
      \subfigure[BER under varying distance\label{fig:csi_ber}]{
        \includegraphics[width=.35\textwidth]{figs/csi/ber.eps}
      }
    \caption{RSSI, path loss, and BER at the SNOW BS for a TI CC1310 node.}
    \label{fig:csi}
 \end{figure*}
\subsection{Channel State Information Estimation}\label{sec:csi}

Multi-user OFDM communication requires channel estimation and tracking to ensure high data rate at the BS. One way to avoid channel estimation is to use the \emph{differential phase-shift keying (DPSK)} modulation. DPSK, however, results in a lower bitrate at the BS due to a 3dB loss in the signal-to-noise ratio (SNR)~\cite{van1995channel}. Additionally, the current SNOW design does not support DPSK modulation. SNR at the BS for each node is different in SNOW. Also, SNR of each node is affected differently due to channel conditions, deteriorating the overall bitrate in the uplinks. Thus, it requires handling of the channel estimation in SNOW.

Figure~\ref{fig:csi} shows the experimentally found received signal strength indicator (RSSI), path loss, and BER at the SNOW BS for a CC1310 device that transmits successive 1000 30-byte (payload) packets from 200 to 1000m distances, respectively, with a Tx power of 15dBm, subcarrier center frequency at 500MHz, and a bandwidth of 98kHz. Figure~\ref{fig:csi_rssi} indicates that the RSSI decreases rapidly with the increase in distance. Also, the path loss in Figure~\ref{fig:csi_pathloss} shows that it is significantly higher than the theoretical free space loss~\cite{rappaport1996wireless}. We also compare with the Okumura-Hata~\cite{rappaport1996wireless} loss to check if it fits the model, however, it does not. Finally, Figure~\ref{fig:csi_ber} confirms that the BER goes above $10^{-3}$ (which is not acceptable~\cite{rnr}) beyond 400m due to the unknown channel conditions. Figure~\ref{fig:csi_ber} also shows that the BER worsens for an increase in the subcarrier bandwidth. Thus, to make our implementation more resilient, we need to incorporate the CSI estimation in SNOW.

We calculate the CSI for each SNOW node independently on its subcarrier. We consider a slow flat-fading model~\cite{tse2005fundamentals}, where the channel conditions vary slowly with respect to a single node to BS packet duration. Note that joint-CSI estimation~\cite{jiang2007iterative, ribeiro2008uplink} in SNOW is not our design goal since it would require SNOW nodes to be strongly time-synchronized.  
Similar to IEEE 802.16e, we run CSI estimation independently for each node because of their different fading and noise characteristics. In the following, we explain the CSI estimation technique for one node on its subcarrier for each packet. The BS uses the same technique to estimate CSI for all other nodes. 

For a node, in a narrowband flat-fading subcarrier, the system is modeled as $y = Hx + w$,
% \begin{equation}
% \nonumber y = Hx + w,
% \end{equation}
where $y$, $x$, and $w$ are the receive vector, transmit vector, and noise vector, respectively. $H$ is the channel matrix. 
We model the noise as additive white Gaussian noise, i.e., a circular symmetric complex normal ($CN$) with $w \sim CN(0, W)$, where the mean is zero and noise covariance matrix $W$ is known.
%Noise is modeled as circular symmetric complex normal ($CN$) with $w \sim CN(0, W)$, where the mean is zero and noise covariance matrix $W$ is known, thus an additive white Gaussian noise. 
As the subcarrier conditions vary, we estimate the CSI on a short-term basis based on popular approach called {\em training sequence}. We use the known preamble transmitted at the beginning of each packet. $H$ is estimated using the combined knowledge of the received and the transmitted preambles. To make the estimation robust, we divide the preamble into $n$ equal parts (preamble sequence). E.g., n = 4, which is similar to the estimation in IEEE 802.11.

Let the preamble sequence be $(p_1, p_2, \cdots, p_n)$, where vector $p_i$ is transmitted as $y_i = Hp_i + w_i$.
% \begin{equation}
% \nonumber y_i = Hp_i + w_i.
% \end{equation}
Combining the received preamble sequences, we get $Y = [y_1, \cdots, y_n] = HP + W$, where 
% \begin{equation}
% \nonumber Y = [y_1, \cdots, y_n] = HP + W.
% \end{equation}
$P = [p_1, \cdots, p_n]$ and $W = [w_1, \cdots, w_n]$. With combined knowledge of $Y$ and $P$, channel matrix $H$ is estimated. Similar to the CSI estimation in the uplink communications by the BS, each node also estimates the CSI during its downlink communications. Note that the computational complexity of CSI estimation at the nodes is lightweight since each SNOW packet has a 32-bit preamble~\cite{snow_ton}, divided into four equal parts. A node thus processes a vector of only 8 bits at a time.
%during CSI estimation.




\subsection{Carrier Frequency Offset Estimation} \label{sec:cfo}

Multi-user OFDM systems are very sensitive to the CFO between the transmitters and the receiver. CFO causes the OFDM systems to lose orthogonality between subcarriers, which results in severe ICI. 
A transmitter and a receiver observe CFO due to (i) the mismatch in their local oscillator frequency as a result of hardware imperfections; (ii) the relative motion that causes a Doppler shift. 
%CFO originates in a transmitter and a receiver due to their (i) local oscillator's frequency mismatch as a result of hardware imperfections; (ii) relative motion that causes a Doppler shift. 
ICI degrades the SNR between an OFDM transmitter and a receiver, which results in significant BER. Thus, we investigate the needs for CFO estimation in our implementation.
\begin{figure}[!htb]
\centering
\includegraphics[width=0.35\textwidth]{figs/cfo/ber.eps}
\caption{BER at different $E_b/N_0$.}
\label{fig:cfo}
\end{figure}
The loss in SNR due to the CFO between the SNOW BS and a node can be estimated as 
$SNR_{loss} = 1 + \frac{1}{3}(\pi \delta f T)^2\frac{E_s}{N_0}$~\cite{nee2000ofdm}, where
% \begin{equation} \scriptsize
%  \nonumber SNR_{loss} = 1 + \frac{1}{3}(\pi \delta f T)^2\frac{E_s}{N_0} 
% \end{equation}
$\delta f$ is the frequency offset, $T$ is the symbol duration, $E_s$ is the average received subcarrier energy, and $N_0/2$ is the two-sided spectral density of the noise power.

To observe the effects of CFO, we choose two neighboring orthogonal subcarriers in the BS and send concurrent packets from two nodes at 200m distance. Each node sends successive 1000 30-byte packets. We repeat this experiment varying the transmission powers at the nodes to generate signals with different $E_b/N_0$, where $E_b$ is the average energy per bit in the received signals. 
Figure~\ref{fig:cfo} shows the BER at the BS while receiving packets from these two nodes. This figure shows that BER is nearly $10^{-3}$ even for very high $E_b/N_0$ ($\approx 40$dB), which is also very high compared to the theoretical BER~\cite{choi2000carrier}. Thus, CFO is heavily pronounced in SNOW.
The distributed and asynchronous nature of SNOW does not allow CFO estimation similar to the traditional multi-user OFDM systems.
While the USRP-based SNOW implementation provides a trivial and {\em coarse} CFO estimation, it is not robust and does not account for the mobility of the nodes~\cite{snow_ton}.
We propose a pilot-based robust CFO estimation technique, combining both coarse and finer estimations, which accounts for the mobility of the nodes as well. We use training symbols for CFO estimation in an ICI free environment for each node independently, while it joins the network by communicating with the BS using a non-overlapping {\em join subcarrier}.


We explain the CFO estimation technique between a node and the BS (uplink) on a join subcarrier $f$ based on time-domain samples. Note that the BS keeps running the G-FFT on the entire BS spectrum. We thus extract the corresponding time-domain samples of the join subcarrier by applying IFFT during a node join. The join subcarrier does not overlap with other subcarriers; hence it is ICI-free. If $f_{\text{node}}$ and $f_{\text{BS}}$ are the frequencies at a node and the BS, respectively, then their frequency offset $\delta f = f_{\text{node}}-  f_{\text{BS}}$.
For transmitted signal $x(t)$ from a node, the received signal  $y(t)$ at the BS that experiences a CFO of $\delta f$ is given by 
$y(t)  = x(t) e^{j2\pi \delta f t}$.
Similar to IEEE 802.11a, we estimate $\delta f$ based on short and long preamble approach. Note that the USRP-based implementation has considered only one preamble to estimate CFO.
In our implementation, the BS first divides a $n$-bit preamble from a node into short and long preambles of lengths $n/4$ and $3n/4$, respectively. Thus for a 32-bit preamble (typically used in SNOW), the lengths of the short and long preambles are  8 and 24, respectively. 
The short preamble and the long preamble are used for coarse and finer CFO estimation, respectively. 
Considering $\delta t_s$ as the short preamble duration and $\delta f_s$ as the coarse CFO estimation, we have
$y(t-\delta t_s)  = x(t) e^{j2\pi \delta f_s (t-\delta t_s)}.$

Since $y(t)$ and $y(t-\delta t_s)$ are known at the BS, we have
\begin{align*}
y(t-\delta t_s) y^*(t)  & = x(t) e^{j2\pi \delta f_s (t-\delta t_s)}       x^*(t) e^{-j2\pi  \delta f_s t}
                           = |x(t)|^2  e^{j 2\pi  \delta f_s -\delta t_s }.
\end{align*}
Taking angle of both sides gives us as follows.
% $$\sphericalangle  y(t-\delta t_s) y^*(t)   =  \sphericalangle     |x(t)|^2  e^{j 2\pi  \delta f_s -\delta t_s }  =      - 2\pi  \delta f_s \delta t_s$$
\begin{align*}
\sphericalangle  y(t-\delta t_s) y^*(t)   &=  \sphericalangle     |x(t)|^2  e^{j 2\pi  \delta f_s -\delta t_s } =      - 2\pi  \delta f_s \delta t_s
                                          %&=      - 2\pi  \delta f_s \delta t_s
\end{align*}
By rearranging the above equation, we get
$$\delta f_s   =  - \frac{\sphericalangle  y(t-\delta t_s) y^*(t) }{2\pi\delta t_s}.$$

Now that we have the coarse CFO $\delta f_s$, we correct each time domain sample (say, $P$) received in the long preamble as $ P_a = P_a e^{-ja \delta f_s}$, where $a = \{1, 2, \cdots, A\}$ and $A$ is the number of time-domain samples in the long preamble. Taking into account the corrected samples of the long preamble and considering $\delta t_l$ as the long preamble duration, we estimate the finer CFO as follows. 
\begin{equation} 
\delta f  =  - \frac{\sphericalangle  y(t-\delta t_l) y^*(t) }{2\pi\delta t_l} \label{eqn:finer_cfo}
\end{equation}
To this extent, considering the join subcarrier $f$, the {\slshape ppm (parts per million)} on the BS's crystal is given by $ \text{ppm}_\text{BS} = 10^6  \big(\frac{\delta f}{f}\big) $. Thus, the BS calculates $ \delta f_i$ on subcarrier $f_i$ (assigned for node $i$) as 
$\delta f_i =  \frac{(f_i * \text{ppm}_\text{BS})}{10^6}.$ The CFO between the Tx-Radio and the Rx-radio can be estimated using a basic SISO CFO estimation technique~\cite{yao2005blind}. Thus, BS also knows the CFO for the downlink.


We now explain the CFO estimation to compensate for the Doppler shift. Note that if the signal bandwidth is sufficiently narrow at a given carrier frequency and mobile velocity, the Doppler shift can be approximated as a common shift across the entire signal bandwidth~\cite{talbot2007mobility}. Thus, the Doppler shift in the join subcarrier for a node also represents the Doppler shift at its assigned subcarrier, and hence the estimated CFO in Equation (\ref{eqn:finer_cfo}) is not affected due to the Doppler Shift.
For simplicity, we consider that a node's velocity is constant and the change in Doppler shift is negligible during a single packet transmission in SNOW.
Considering $\delta f_d$ as the CFO due to the Doppler shift, $v$ as the velocity of the node, and $\theta$ as the angle of the arrived signal at the BS from the node, we have $f_d = f_i\big(\frac{v}{c}\big)\cos(\theta)$~\cite{talbot2007mobility}, where
% \begin{equation}
% 	\nonumber \delta f_d = f_i\big(\frac{v}{c}\big)\cos(\theta).
% \end{equation}
$f_i$ is the subcarrier center frequency and $c$ is the speed of light. The node itself may consider its motion as circular and approximate $\theta = \frac{\delta s}{r}$, where $\delta s$ is the amount of anticipated change in position during a packet transmission and $r$ is the {\em line-of-sight} distance between the node and BS. Thus, CFO compensation due to the Doppler shift is done at the nodes during uplink communications. In the downlink communications, the BS Tx-Radio can also compensate for the node's mobility as the node can report its Doppler shift to the BS during the uplink communications.

In summary, as the nodes asynchronously transmit, estimating joint-CFO of the subcarriers at the BS is very difficult. We thus use a simple feedback approach for proactive CFO correction in the uplink communications. Specifically, 
$\delta f_i$  estimated at the BS for subcarrier $f_i$ is given to the node (during joining process at subcarrier $f_i$).
The node may then adjust its transmitted signal based on $\delta f_i$ and $\delta f_d$, calculated as $(\delta f_i + \delta f_d)$, which will align its signal so that the BS does not need to compensate for CFO in the uplink communications. Such feedback-based proactive compensation scheme was studied before for multi-user OFDM and is also used in global system for mobile communication (GSM)~\cite{van1999time}.

\section{Handling the Near-Far Power Problem} \label{sec:near-far}
\begin{figure}[!htb]
\centering
\includegraphics[width=0.5\textwidth]{figs/near-far-flat.eps}
\caption{An illustration of the near-far power problem. B is farther from the BS than A and both transmit concurrently using the same Tx power.}
\label{fig:near-far}
\end{figure}
Wireless communication is susceptible to the near-far power problem, especially in CDMA (Code Division Multiple Access)~\cite{muqattash2003cdma}. Multi-user D-OFDM system in SNOW may also suffer from this problem. Figure~\ref{fig:near-far} illustrates the near-far power problem in SNOW. Suppose, nodes A and B are operating on two adjacent subcarriers. Node A is closer to the BS compared to node B. When both nodes A and B transmit concurrently to the BS, the received frequency domain signals from node A and B may look as shown on the right of Figure~\ref{fig:near-far}. Here, transmission from node B is severely interfered by the strong radiations of node A's transmission. As such, node B's signal may be buried under node A's signal making it difficult for the BS to decode the packet from node B. 
A typical SNOW deployment may have such scenarios if the nodes operating on adjacent subcarriers use the same transmission power and transmit concurrently at the BS from different distances. 
\begin{figure}[t]
    \centering 
      \subfigure[Avg. PDR at different Tx powers\label{fig:nf_pdr}]{
    \includegraphics[width=0.35\textwidth]{figs/nearfar/pdr.eps}
      }\hfill
      \subfigure[Avg. PDR at different Tx powers and time\label{fig:nf_time}]{
        \includegraphics[width=.35\textwidth]{figs/nearfar/pdr-time.eps}
      }
    \caption{Packet delivery ratio at different Tx powers}
    \label{fig:nf-effects}
 \end{figure}


To observe the near-far power problem in SNOW, we run experiments by choosing 3 different adjacent subcarriers, where the middle subcarrier observes the near-far power problem introduced by both subcarriers on its left and right. We place two CC1310 nodes within 20m of the BS that use the left and the right subcarrier, respectively. We use another CC1310 node that uses the middle subcarrier and is placed at different distances between 200 and 1000m from the BS. Nodes that are within 20m of the BS transmit packets continuously with a transmission power of 0dBm. At each distance, for each transmission power between 8 and 15dBm, the node that uses the middle subcarrier sends 100 rounds of 1000 consecutive packets (sends one packet then waits for the ACK and then sends another packet, and so on) to the BS and with a random interval of 0-500ms. For each transmission power level, at each distance, that node calculates its average {\em packet delivery ratio (PDR)}. PDR is defined as the ratio of the number of successfully acknowledged packets to the number of total packets sent.
We repeat the same experiments for 7 days at 9 AM, 2 PM, and 6 PM.

Figure~\ref{fig:nf_pdr} shows that the average PDR increases at each distance with the increase in the transmission power. Figure~\ref{fig:nf_time} depicts the result for 7-day experiments (only at a distance of 200m) and shows that the average PDR changes at different time of the day. Overall, Figure~\ref{fig:nf_pdr} and~\ref{fig:nf_time} confirms that the average PDR increases with the increase in the transmission power. To ensure the energy-efficiency at the nodes, i.e., to find a  transmission power  that suffices to eliminate the effects of near-far power problem, we propose an adaptive transmission power control for the SNOW design, as described below.

% To demonstrate the effects of near-far problem, we run experiments in SNOW by placing 5 nodes at different distances ranging between 200-1000m. The subcarriers assigned to the nodes are chosen in a way such that they observe the near-far problem as shown in Figure~\ref{fig:nf-effects}.  
% At each distance, a node transmits 1000 packets using a fixed transmission power (concurrently with other nodes at other distances using the same transmission power) and calculate its packet delivery ratio (PDR). A packet is correctly delivered if the node receives an ACK for that packet. At each location we vary the transmission power between 0-15dBm and repeat the same strategy. Figure~\ref{fig:nf_rssi} shows that the average RSSI at the BS increases with the increase in the transmission power, as expected. However, Figure~\ref{fig:nf_pdr} shows that the PDR at nodes at different distances vary unexpectedly. For example, node at distance 400m observes very low PDR due to the node at 200m, and so on. Thus, the near-far problem needs to be addressed in SNOW. To this extent, we propose an adaptive transmission power control in SNOW.


\subsection{Adaptive Transmission Power Control}\label{sec:atpc}
Our design objective for the adaptive Tx power control is to correlate the subcarrier-level Tx power and link quality (i.e., PDR) between each node and the BS. We thus formulate a predictive model to provide each node with a proper Tx power to make a successful transmission to the BS using its assigned subcarrier. Note that our work differs from the work in~\cite{lin2016atpc} in fundamental concepts of the network design and architecture. In~\cite{lin2016atpc}, the authors have considered a multi-hop wireless sensor network based on IEEE 802.15.4 with no concurrency between a set of transmitters and a receiver. Additionally, our model is much more simpler since we deal with single hop communications. As such, the overheads (i.e., energy consumption and latency at each node) associated with our model are fundamentally lesser than that in~\cite{lin2016atpc}, or the other techniques developed for multi-hop wireless networks~\cite{son2006experimental, li2005cone}. In the following, we describe our model.


Whenever a node is assigned a new subcarrier or observes a lower PDR, e.g., PDR below quality of service (QoS) requirements due to mobility, it runs a lightweight predictive model to determine the convenient Tx power to make successful transmissions to the BS.
Our predictive model uses an approximation function to estimate the PDR distribution at different Tx power levels. Over time, that function is modified to adapt to the node's changes. The function is built from the sample pairs of the Tx power levels and PDRs between the node and the BS via a curve-fitting approach. A node collects these samples by sending groups of packets to the BS at different Tx power levels. A node may not be assigned new subcarriers or may not observe lower PDR due to mobility (as per our CSI and CFO estimations) frequently. Thus, the overhead (e.g., energy consumption) for collecting these samples may be negligible compared to the overall network lifetime (which is several years).

Specifically, our predictive model uses two vectors: $TP$ and $L$, where $TP = \{ tp_1, tp_2, \cdots, tp_m \}$ contains $m$ different Tx power levels that the node uses to send $m$ groups of packets to the BS and $L = \{ l_1, l_2, \cdots, l_m \}$ contains the corresponding PDR values at different Tx power levels. Thus, $l_i$ represents the PDR value at Tx power level $tp_i$. We use the following linear function to correlate between Tx power and PDR.
\begin{equation}
	l(tp_i) = a~.~tp_i + b \label{eqn:linear_model}
\end{equation}
To lessen the computational overhead in the node, we adopt the {\em least square approximation} technique to determine the unknown coefficients $a$ and $b$ in Equation (\ref{eqn:linear_model}). Thus, we find the minimum of the function $S(a, b)$, where $\nonumber S(a, b) = \sum |l_i - l(tp_i)|^2.$
% \begin{equation}
% \nonumber	S(a, b) = \sum |l_i - l(tp_i)|^2.
% \end{equation}
The minimum of $S(a, b)$ is determined by taking the partial derivatives of $S(a, b)$ with respect to $a$ and $b$, respectively, and setting them to zero. Thus, $ \frac{\partial S}{\partial a} = 0$ and $\frac{\partial S}{\partial b} = 0$ give us
\begin{align}
	\nonumber a~\sum (tp_i)^2 + b~\sum tp_i &= \sum l_i.tp_i \text{ and} \\ 
  \nonumber a~\sum tp_i + b~m &= \sum l_i.
\end{align}
Simplifying the above two equations, we find the estimated values of $a$ and $b$ as follows.
\begin{equation}\nonumber
\begin{split}
	\begin{bmatrix}
		\hat{a}\\
        \hat{b}
	\end{bmatrix}
    = \frac{1}{m \sum (tp_i)^2 - (\sum tp_i)^2} \times \\
    \begin{bmatrix}
    	m \sum l_i.tp_i - \sum l_i \sum tp_i\\
    	\sum l_i \sum (tp_i)^2 - \sum l_i.tp_i \sum tp_i
    \end{bmatrix}
\end{split}
\end{equation}
Using the estimated values of $a$ and $b$, the node can calculate the appropriate Tx power as follows.
\begin{equation}\label{eqn:estimated}
tp = \big[\frac{PDR_{\text{threshold}} - \hat{b}}{\hat{a}}\big] \in TP
\end{equation}
Here, $PDR_{\text{threshold}}$ is the threshold set empirically or according to QoS requirements, and $[.]$ denotes the function that rounds the value to the nearest integer in the vector $TP$.

Now that the initial model has been established in Equation (\ref{eqn:estimated}), this needs to be updated continuously with the node's changes over time. In Equation (\ref{eqn:linear_model}), both $a$ and $b$ are functions of time that allow the node to use the latest samples to adjust the curve-fitting model dynamically. 
It is empirically found that (Figure~\ref{fig:nf_pdr}) the slope of the curve does not change much over time; hence $a$ is assumed time-invariant in the predictive model. On the other hand, the value of $b$ changes drastically over time (Figure~\ref{fig:nf_time}). Thus, Equation (\ref{eqn:linear_model}) is rewritten as follows that characterizes the actual relationship between Tx power and PDR.
\begin{equation}
	\nonumber l(tp(t)) = a.tp(t) + b(t)
\end{equation}
Now, $b(t)$ is determined by the latest Tx power and PDR pairs using the following feedback-based control equation~\cite{lin2016atpc}.
\begin{align}
	\nonumber \Delta \hat{b}(t) &= \hat{b}(t) - \hat{b}(t+1) \\
    			\nonumber	  &= \frac{\sum^K_{k=1} [PDR_{\text{threshold}} - l_k(t - 1)]}{K} \\ 
                      &= PDR_{\text{threshold}} - l(t-1) \label{eqn:control}
\end{align}
Here, $l(t-1)$ is the average value of $K$ readings denoted as 
\begin{equation}
	\nonumber l(t-1) = \frac{\sum^K_{k=1} l_k(t - 1)}{K}.
\end{equation}
Here, $l_k(t-1)$, for $k = \{1, 2, \cdots, K\}$, is one reading of PDR during the time period $t-1$ and $K$ is the number of feedback responses at time period $t-1$. Now, the error in Equation (\ref{eqn:control}) is deducted from the previous estimation; hence the new estimation of $b(t)$ can be written as: $\hat{b}(t) = \hat{b}(t-1) - \Delta \hat{b}(t)$.
Given the newly estimated $\hat{b}(t)$, the node now can set the Tx power at time $t$ as
\begin{equation}
	\nonumber tp(t) = \big[\frac{PDR_{\text{threshold}} - \hat{b}(t)}{\hat{a}}\big].
\end{equation}













    % This file has Technical section
%\section{Schedule Independent Memory Allocation}
\label{sec:sima}

We also address a memory-based limitation of polyhedral compilation
tools.  It is well known that in any parallelization (of any program), it is
essential to respect (only) the \emph{true} or flow dependences.  Other
(memory-based) dependences can be ignored if one can re-allocate memory.  In
practice, this is limited by the fact that the associated memory expansion may
be prohibitively expensive, and there has been work on mitigating this
expansion~\cite{vasilache-impact12, lefebvre-feautrier-pc98, sanjay-europar96,
  sanjay-toplas00}.  We propose a novel yet simple \emph{schedule-independent}
memory allocation strategy.  Our work also generalizes polyhedral compilation
by enabling polyhedral tools to use alternate, \emph{hybrid} schedules
consisting of affine loops for certain parts of the iteration space and
cache-oblivious divide-and-conquer schedules for others.


\subsection{Background}

In this section, we introduce the necessary background of our work. We first
give a brief description of the polyhedral representation of programs, and the
general flow of a polyhedral compiler.  Then, we discuss the legality of
tiling, which is related to the input of our code generator.

\begin{figure*}[tb]
  \centering %\vspace*{6cm}
  \includegraphics[scale=0.6]{PolyCompiler}
  \caption{\small{Polyhedral Compilation: the Polyhedral Reduced Dependence
      (hyper) Graph (PRDG) serves as the intermediate representation.
      Piecewise Quasi-Affine Functions (PQAFs) describe transformations.}}
  \label{fig:compiler}
\end{figure*}

\subsubsection{Polyhedral Compilation and Representation}

Figure~\ref{fig:compiler} shows the flow of polyhedral compilation.  First,
dependence analysis of an input program (or a ``polyhedral section'' thereof)
produces an intermediate representation (IR) in the form of~\cite{DRV-sched00}
a \emph{Polyhedral Reduced Dependence (hyper) Graph} (PRDG).  Various analyses
are performed on the PRDG to choose a number of mappings in the form of
\emph{Piecewise Quasi-Affine Functions} (PQAFs) that specify the schedule as a
set of \emph{multi-dimensional} vectors.  The PQAFs come with annotations to
indicate whether each dimension is sequential or parallel, and also whether it
is part of a \emph{tilable band}, i.e., whether tiling this band of dimensions
is legal.  The transformations may be applied to the PRDG iteratively, and
(eventually) the PRDG and QLAF are provided to a code-generator that produces
code for various targets.

% specify which dimensions are sequential, which dimensions are (and
% implicitly, also the algorithm~\cite{uday-pldi08} to get tiling hyperplanes
% and tilable dimensions for each statement which is called \emph{tilable
% band}.  Then we transform the program using tiling hyperplanes.  This
% results in a program where hyper-rectangular tiles are legal and the
% wavefront parallel execution order is a legal schedule for executing tiles.
% We also provide schedule independent memory allocations for all the
% variables.  Finally we generate code which traverse the iteration space of
% the program in divide and conquer order.  by post processing the AST of
% tiled code generated by DTiler The following section presents the schedule
% independent memory allocation for affine programs.

One of the strengths of the polyhedral model is that a parametric program may
be concisely represented with a PRDG with finite number of nodes (statements)
and edges (dependences).  The potentially unbounded sets of instances of a
statement are represented in abstract forms of integer sets, called
\emph{domains}, and dependences between them as affine functions (or
relations, which are viewed as a set-valued function) over these statement
domains.  Indeed, every edge, $e$ from node $v$ to $w$, in the PRDG is
annotated with two objects: (i) a domain, $D_e$ specifying the (subset of) the
domain, $D_v$ of its source node, where the dependence occurs, and (ii) the
affine function, $f$, such that for any point $z\in D_e$, the (set of)
point(s) in $D_w$ on which it depends is given by $f(z)$.  $D_e$ is called the
context of the edge, and $f$ is its dependence function.  We also use the
notation $f(D_e)$ to denote the set valued image of $D_e$ by $f$.

An affine function $\mathbb{Z}^n \rightarrow \mathbb{Z}^m$ may be expressed as
$f(x) = A\vec{x} + \vec{b}$, where $\vec{x}$, function domain, is an integer
vector of size $n$; $A$, linear part, is an $n\times m$ matrix; and $\vec{b}$,
constant part, is an integer vector of size $m$.  A dependence is said to be
uniform if the dependence function is only a constant offset, i.e., when the
linear part $A$ is the identity.

%  is used to get the  tilable dimensions....  The parallelism that can be
%  explored using tiles is assumed to be wavefront....??  However, we modify
%  the execution order of tiles and schedule them in a divide-and-conquer
%  fashion.... Updating the memory allocation schemes becomes important when
%  we change the order of execution of tiles.... The following Section talks
%  about Memory Allocation...

\subsubsection{Legality of Tiling}

Tiling is a well-known loop transformation for partitioning computations into
smaller, atomic (all inputs to a tile can be computed before its execution),
units called tiles~\cite{irigoin-popl88, Wolf91tiling}.  The natural legality
condition is that the dependences across tiles do not create a cycle.  In
compilers, this condition is typically expressed as fully permutability (i.e.,
dependences are non-negative direction vectors), which is a sufficient
condition.  Our transformation for cache oblivious tiling takes as inputs a
loop nest that is fully permutable.  For polyhedral programs, scheduling
techniques to expose such loop nests are available~\cite{uday-pldi08}.


\subsection{Memory Allocation}

\begin{figure}[tb]
  \centering % \vspace*{4cm}
{\small\begin{lstlisting}
for (i = 0; i < N; i++){
  S0:  X[0,i] = A[i]; } // Initialize
for (t = 1; t <= 2*N; t++){ //Note: ub is even
  for (i = 1; i < N-1; i++){
    S1: X[t%2][i] = f(X[(t-1)%2][i-1],
             X[(t-1)%2][i], X[(t-1)%2][i+1],
             X[(t-1)%2][0]);
    }
  S2: X[t%2][0] = g(X[(t-1)%2][0]); 
  S3: X[t%2][N-1] = g(X[(t-1)%2][N-1]);
}
for (i = 0; i < N; i++){
  S4: Aout[i] = X[0,i]; } // Output copy
\end{lstlisting}
}
\caption{\small{Neither Pochoir nor Autogen can handle the computation
    performed by this simple loop.  Moreover, it has a memory based dependence
    that prevents polyhedral compilers like Pluto from tiling both dimensions.
    However, the true dependences of the program admit a tilable schedule, but
    at the potentical cost of $O(N^2)$ memory.  Our scheme reduces this to
    $O(N)$}}
\label{fig:motiv}
\end{figure}


In this section, we first describe how memory based dependences prevent
tiling, using our motivating example (Fig.~\ref{fig:motiv}), and show that
simply ignoring these (false) dependences would lead to memory explosion.
After formulating our problem, we next propose a simple, schedule independent
memory allocation scheme that resolves it.

Consider the statement $\mathrm{S1}$, and note that its domain, $D_1$ is the
polyhedral set, $\{t,i~|~ 1\leq t\leq 2N \wedge 1\leq i\leq N-1 \}$.
$\mathrm{S1}$ has four true dependences (for points sufficiently far from the
boundaries), three of which are $\mathrm{S1}[t-1, i-1]$, $\mathrm{S1}[t-1, i]$
and $\mathrm{S1}[t-1, i+1]$, the typical, 1D-Jacobi stencil dependences, and
the fourth one is $\mathrm{S2}[t-1, 0]$, which is a truly affine dependence on
the most recent writer to the memory location $\mathtt{X[(t-1)\%2,0]}$ when
the statement $\langle \mathrm{S1}, [t, i]\rangle$ is being executed.  All
these dependences are captured as edges with affine \emph{functions} in the
PRDG.  In addition, there is a memory based dependence, that we must also
respect.  Consider statement $\mathrm{S2}$, whose domain, $D_2 = \{t~|~ 1\leq
t\leq 2N\}$ is just one dimensional.  The $t$-th instance of S2 \emph{(over)
  writes} $\mathtt{X[t\%2, 0]}$, therefore all computations that read the
previous value must be executed before it.  In this sense, $\mathrm{S2}[t]$
``depends on'' the set $\mathrm{S1}[t,i]$, for all $1\leq i\leq N-1$.  This
dependence (which is a \emph{relation} rather than a function) is captured by
another a special edge in the PRDG.

The only schedule that respects all these dependences is the family of lines
parallel to the $t$ axis (provided all iterations of S1 are done first).
Although this has maximal parallelism, it has very poor locality.  Note that
the Pluto scheduler does not seek maximal parallelism, but rather, to maximize
the \emph{number of linearly independent tiling hyperplanes}.  Unfortunately,
the $t=\mathrm{const}$ is the only legal tiling hyperplane for this set of
dependences, and the tilable band obtained by Pluto is only 1-dimensional.

What if we did not have the memory-based dependences, i.e., what if we ignored
the memory allocation of the original program, and stored each computed value
in a distinct memory location?  In this case, there would be no memory based
dependences, and we can indeed find another family of (actually, infinitely
many) legal tiling hyperplanes: say, the lines $i+t=\mathrm{const}$.  As a
result, if we use the mapping $(t,i) \mapsto (t,i+t)$ as our ``schedule,'' the
new loops in the transformed program would be fully permutable, and could be
legally tiled.

Thus, the problem we seek to solve is: \emph{how to avoid memory based
  dependences, but without the cost of memory expansion that it seems to
  imply}.

Memory allocation for polyhedral programs is a well studied problem, and there
are two main approaches.  One either does memory allocation after the schedule is
chosen~\cite{sanjay-europar96, degreef-memory97, lefebvre-feautrier-pc98,
  sanjay-toplas00, darte-lattice05, vasilache-impact12,
  bhaskaracharya-toplas16, bhaskaracharya-popl16} since it often leads to a
smaller memory footprint, or else uses a \emph{schedule independent} memory
allocation, based on the so called \emph{universal occupancy vectors} (UOV).
This problem is solved when the program has \emph{uniform dependences}, i.e.,
when each dependence can be described by a \emph{constant vector}, and for some
simple extensions of this~\cite{strout-etal-asplos98, sanjay-memory-2011}.

It is important to note that tiling actually modifies a schedule: the so
called, ``schedule dimensions'' become fully permutable loops, and indeed,
these loops \emph{are actually permuted} in the generated tiled code.  So,
when a \emph{tiling schedule} specified by a family of $d$ tiling hyperplanes
is finally implemented by the generated code, the actual time-stamps are not
really $d$-dimensional vectors, but rather $2d$-dimensional ones obtained as
some complicated function of these indices.  Furthermore, we will see when we
generate cache-oblivious tiled codes, these tilable loops will actually be
visited in the divide-and-conquer order of execution, as required by COT.  As
a result, finding a memory map that takes into account such a rather
complicated final schedule is a tricky problem.  We therefore seek and propose
schedule-independent memory allocations.

The intuition behind our solution is (deceptively) simple, and we first
illustrate it on our motivating example (Fig.~\ref{fig:motiv}).  Rather than
the so-called ``single assignment'' program for the entire iteration space of
the program (i.e., full memory expansion), could we find lower-dimensional
subsets, such that a single assignment memory for only these subsets is
sufficient?  A careful examination of the code reveals that the memory based
dependences arise due to statement S2, and its domain is only 1-dimensional.
So we store the results of this statement into an auxiliary array, \texttt{Y},
and modify the program so that the fourth dependence simply reads
\texttt{Y[t-1]}, rather than \texttt{X[(t-1)\%2,0]}.  For the variable,
\texttt{X}, we use the old $(t,i) \mapsto (t\%2,i)$ memory allocation that was
used in the original code.  This results in $4N$ memory, which is a polynomial
degree better than quadratic.  Of course, the challenge is how to discover
this automatically.

% PLUTO schedule give a schedule and a tiling band.  A point in a tiling band
% corresponds to a particular tile.  All points within a tile execute
% sequentially.  However, the tiles can be executed in parallel.  Consider the
% modified Jacobi-1D stencil (CHANGE THIS TO BE CONSISTENT WITH RUNNING
% EXAMPLE).  The Figure \textbf{ABC} shows the iteration space of Jacobi-1D
% example (CHANGE THIS TO BE CONSISTENT WITH RUNNING EXAMPLE).  The blue bars
% show the inputs and outputs of the program corresponding to statements
% $S_{0}$ and $S_{4}$ respectively.  The memory allocation scheme used
% initially is modulo memory allocation.  However, this modulo memory
% allocation is not affine and hence PLUTO is unable to find a schedule. We
% make this memory map to not to use modulo but use previous and current
% instead. Even then, PLUTO is unable to find a schedule such that all
% dimensions are tilable.  PLUTO will decide to tile this program with single
% assignment memory.  Identity memory allocation is known to be overkill and
% lead to inefficient codes.  Therefore, we need a schedule independent memory
% allocation scheme which guarantees that the given memory allocation is both
% legal as well as optimal for any given schedule.

% The problem of schedule independent memory allocation for Polyhedral
% programs is a partially solved problem. [Strout et al] presented an
% algorithm for schedule independent memory allocation for a class of programs
% that have only uniform dependences.  A set of programs with uniform
% dependences is a proper subset of the set of programs with affine
% dependences, the class of Polyhedral programs.  Our recursive
% divide-and-conquer code generator is applicable to all Polyhedral programs
% in general, including the ones that have truly affine non-uniform
% dependences. We therefore present a novel scheme for schedule independent
% memory allocation for all affine programs.

We now outline how this is done.  At a high level, our algorithm takes a PRDG
as input, applies some (piecewise) affine transformations to it, and outputs
the transformed PRDG together with a separate memory map for each node in the
transformed PRDG.  More specifically, it works as follows.

\begin{itemize}
\item \emph{Preprocessing.}  For each edge, $e$, in the PRDG, with context
  $D_e$, and function, $f$, we first identify whether $f$ is \emph{uniform in
    context} in the sense that, for all points, $z\in D_e$, the value of
  $z-f(z)$ is a constant vector, independent of $z$.

  For example, consider a dependence function, $(i,j) \mapsto (i-1,i-1)$
  which, maps any point $[i,j]$ in the plane to a point on the diagonal, and
  is clearly not uniform.  However, what if $D_e = \{i,j~|~i=j-1\}$?  With
  this contextual information, the dependence is actually uniform: $(i, j)
  \mapsto (i-1, j-2)$.

  All edges/dependences that are neither uniform to begin with, not uniform in
  context, are marked as \emph{truly affine}.
\item \emph{Affine Split.}  For every node, $v$, in the PRDG that has at least
  one truly affine edge $e$ incident on it, we create a new node, $v'$.  Its
  domain $D_{v'}$ is the union of $f(D_e)$ of all such incident edges.

  The edges in the PRDG are modified as follows.  All the truly affine edges
  that were incident on $v$ are now made incident on $v'$; and $v'$ has a
  single outgoing edge $e'$, annotated with $\langle D_{v'}, I \rangle$ (its
  dependence function is the identity map) and whose destination is $v$.

  It is easy to see that we have not changed the program semantics.  In
  effect, we have simply copied the value of every point in $D_v$ that was the
  target of any truly affine dependence over to a new variable $v'$, and
  ``diverted'' all the truly affine edges that used to be incident on $v$ over
  to $v'$.  Moreover, since the identity function is uniform by definition, all
  edges incident on $v$ are now either uniform, or uniform in context.
\item We now use existing UOV based methods~\cite{strout-etal-asplos98,
    sanjay-memory-2011} to choose a schedule-independent memory allocation for
  all the original nodes in the PRDG, and a \emph{single-assignment} memory
  allocation for all the newly introduced variables.
\end{itemize}

The key insight into why this leads to significant memory savings, is the fact
that in all polyhedral programs that we encountered, truly affine dependences
are almost always \emph{rank deficient}, i.e., are many-to-one mappings from
the consumer index points to the producers.  The only exceptions are either
pathological programs, or programs that do multi-dimensional data
reorganizations via bijections (e.g., matrix transpose, tensor permutations,
etc.) where here is no scope nor need to reduce the total memory footprint.
As a result, $f(D_e)$ is almost always a lower dimensional polyhedron, and
requires significantly less memory, even when stored supposedly inefficiently.


\subsection{Related Work}

Memory allocation for polyhedral programs is a well studied problem for almost
two decades.  DeGreef and Cathoor~\cite{degreef-memory97} tackled the problem
of sharing the memory across multiple arrays in the program.the so called
inet-array memory reuse problem, and proposed an ILP based solution.  Wilde
and Rajopadhye, in dealing with an intrinsically memory-inefficient functional
language Alpha~\cite{mauras1989thesis} (one can think of this as a program after
full expansion) first addressed the memory reuse for points of an iteration
space~\cite{sanjay-europar96}.  They gave necessary and sufficient conditions
for the legality of a memory allocation fucntion, which they allowed to be
``in any direction.''  but they did not provide any insight into how to choose
the mapping.  Lefebvre and Feautrier~\cite{lefebvre-feautrier-pc98} on the
other hand, considered only canonic projections, combined with a modulo
factor, but showed how to choose the mapping optimally.  Later, Quiller\'e and
Rajopadhye~\cite{sanjay-toplas00} revisited multiprojections, extended them to
quasi-affine functions, and proved a tight bound on the number of dimensions
of reuse.  They also showed that cananic projections with modulo factors was
sometimes a constant factor better, and sometimes a constant factor worse.
Darte at al.~\cite{darte-lattice05} took a fresh and elegant approach to the
problem, and formulated the conditions for legal memory allocations by
defining the \emph{conflict set}.  This led to techniques for choosing
provably optimal memory allocations, initially for non-parameterized iteration
spaces, and recently in the context of FPGA acelerators, for parametrically
tiled spaces~\cite{darte2014parametric, darte2016extended}.  Vasilache et
al.~\cite{vasilache-impact12} developed a tool to combine the scheduling and
limiting memory expansion using an ILP formulation, implemented in the
R-Stream compiler.  Recently, Bhaskaracharya et
al.~\cite{bhaskaracharya-toplas16} developed methods to optimally choose
quasi-affine memory allocations, and showed how they are beneficial for tiled
codes, especialy with live-out data.  Furthermore, they also
showed~\cite{bhaskaracharya-popl16} how to combine iner-and intra array reuse
in a unifying framework.

The other \emph{schedule independent} memory allocation was pioneered by
Strout et al.~\cite{strout-etal-asplos98}.  Here, the memory allocation is
chosen based only on the dependences, and is guaranteed to be legal,
regardless of the schedule.  This problem is solved when the program has
\emph{uniform dependences}, i.e., when each dependence can be descibed by a
\emph{constant vector}, and for some simple extensions of
this~\cite{strout-etal-asplos98, sanjay-memory-2011}.

Thies at al.~\cite{thies-pldi02} have also formulated the problem of
simultaneously choosing the schedule and memory allocation as a combined
optimization problem.
% \begin{algorithm}[h]
%   \mbox{} Input : PRDG
  
%     Output : Transformed PRDG and Memory Map
%     \begin{enumerate}
%     \item Recognize truly non-uniform dependence edges
  
%       For each node $X_{i}$ with one or more incoming non-uniform affine
%       edge(s):
%       \begin{itemize}
%       \item Create new node $X_{i}^{new}$ by applying ``Affine Split"
%         transformation
    	
%         Assertion: $X_{i} = X_{i}^{old} + X_{i}^{new}$

%       \end{itemize}
%     \item Find universal occupancy vector $(UOV)$ for all nodes with uniform
%       dependences
%     \item Construct memory mapping function for all nodes except
%       $X_{i}^{new}$ using $UOV$
% 	\item Use identity memory map for $X_{i}^{new}$ nodes
% 	\item \textbf{(SANJAY verify 4 , 5)} Apply change of basis to reduce
%    number of dimensions of the domain of new node
%  \end{enumerate}
%  \caption{SIMA: Schedule Independent Memory Allocation for Polyhedral
%  Programs}
%  \label{alg:sima}
% \end{algorithm}

% Local Variables: ***
% TeX-master: "PACT17.tex" ***
% fill-column: 78 ***
% End: ***
          % This file has Technical result on memory allocation
%\section{Approach}
\begin{figure}[t]
\centering
\resizebox{0.48\textwidth}{!}{ 
  \includegraphics[width=\textwidth]{figures/workflow.PNG}
}
  \caption{Workflow of \system}
  \label{fig:workflow}
\end{figure}

Figure ~\ref{fig:workflow} shows the overall workflow of \system. The triggers for using \system are usually alert(s) from automated anomaly detection, or sometimes an SRE engineer's suspicion. There are three major steps: constructing the service  dependency graph, constructing the event causality graph,  and root cause ranking. The outputs are the root causes ranked by the likelihood. To support fast human investigation experience, we build an interactive UI as shown in  Figure~\ref{fig:UI}: the service dependency, events with causal links and additional details such as raw metrics or the developer contact (of a code deployment event) are presented to the user for next steps. As an  offline part of human investigation, we label/collect a data set, perform validation, and summarize the knowledge for further improvement on all incidents on a daily basis. %as validations and heterogeneous graph learning (HGL)~\cite{qiao2020heterogeneous} to synthesize the knowledge from existing cases in order to further improve the system.

\subsection{Constructing Service Dependency Graph}
\label{sec:appgraph}

The construction of the service dependency graph starts with the initial alerted or suspicious service(s), denoted as $I$. For example, in Figure ~\ref{fig:ex1_dep}, $I=\{\textit{Checkout}\}$. $I$ can contain multiple services based on the range of the trigger alerts or suspicions. We maintain domain service lists where domain-level alerts can be triggered because there is no clear service-level indication.

At the back end, \system maintains a global service dependency graph $G_{global}$ via distributed tracing and log analysis. The directed edge from nodes $A$ to $B$ (two services or system components) in the dependency graph indicates a service invocation or other forms of dependency. In Figure~\ref{fig:ex1_dep}, the black arrows indicate such edges. Bi-directional edges and cycles between the services can be possible and exist. In this work, the global dependency graph is updated daily.%by extracting from one day's total site traffic.

The service dependency (sub)graph $G$ is constructed using $G_{global}$ and $I$. An extended service list $L$ is first constructed by traversing each service in $I$ over $G_{global}$ for a radius range $r$. Each service $u \in L$ can be traversed by at least one service $v \in I$ within $r$ steps: $L=\{u|\exists v\in I, \ dist(u,v)\le r\ or\ dist(v,u)\le r\}$. Then, the service dependency subgraph $G$ is constructed by the nodes in $L$ and the edges between them in $G_{global}$. In our current implementation, $r$ is set to $2$, since this dependency graph may be dynamically extended in the next steps based on events' detail for longer issue chains or additional dependencies.

\subsection{Constructing Event Causality Graph}
\label{sec:causality}

In the second step, \system collects all supported events for each service in $G$ and constructs the causal links between events. 

\subsubsection{Collecting Events}

Table~\ref{tab:events} presents some example event types and detection techniques for \system's production implementation. For detection techniques, ``De Facto'' indicates that the event can be directly collected via a specific API or storage. %The detection can be done passively at the back end continuously then store anomaly events in different databases; or done actively by pulling data and run detection on the fly to save compute resources. 
The detection either runs passively in the back end to reduce delay and improve accuracy, or runs actively for only the services within the dependency graph range to save resources. %For example, low-level error signals or logs are detected actively since they are too many to scale. 

There are three major categories of events: performance metrics, status logs, and developer activities:
\begin{itemize}
    \item \emph{Performance metrics} represent an anomaly of monitored time series metrics. For example, high CPU usage indicates that the service is causing high CPU usage on a certain machine. In this category, most events are continuously and passively detected and stored. %For high CPU usage, threshold indicates the event is created when CPU usage is higher than certain predefined value. TPS spike indicates a spike in transaction per second, since TPS is a moving average value, we use some statistical model learned from historical data to detect such events.
    \item \emph{Status logs} are caused by abnormal system status, such as spike of HTTP error code metrics while accessing other services' endpoints. Different types of error metrics are important and supported in \system, including third-party APIs. For example, Bad Host indicates abnormal patterns on some machines running the service, and can be detected by a  clustering-based ML approach.%Markdown indicates that the whole service is down. 
    \item \emph{Developer activities} are the events generated when a certain activity of developers is triggered, such as code deployment and config change.
\end{itemize}

\begin{table}[t]
\centering
\caption{List of example event types used in \system}
\resizebox{0.4\textwidth}{!}{ 
\begin{tabular}{|c|c|c|}
\hline
Type                                & Event Type                  & Detection Technique  \\ \hline
\multirow{6}{*}{Performance Metrics} & High GC (Overhead)      & Rule-based        \\ \cline{2-3} 
                                    & High CPU Usage          & Rule-based        \\ \cline{2-3} 
%                                    & Out of Memory           & Rule-based        \\ \cline{2-3} 
%                                    & LB Connection Stacking  & Statistical Model \\ \cline{2-3} 
                                    & Latency Spike           & Statistical Model \\ \cline{2-3} 
                                    & TPS Spike               & Statistical Model \\ \cline{2-3} 
                                    & Database Anomaly        & ML Model          \\ \cline{2-3} 
                                    & Business Metric Anomaly & ML Model          \\ \hline
\multirow{4}{*}{Status Logs}        & WebAPI Error            & Statistical Model \\ \cline{2-3} 
                                    & Internal Error          & Statistical Model \\ \cline{2-3} 
                                    & ServiceClient Error     & Statistical Model \\ \cline{2-3} 
                                    & Bad Host                & ML Model          \\ \hline %\cline{2-3} 
%                                    & Hystrix Circuit Break   & De Facto          \\ \hline
\multirow{3}{*}{Developer Activities} & Code Deployment         & De Facto          \\ \cline{2-3} 
                                    & Configuration Change    & De Facto          \\ \cline{2-3} 
                                    & Execute URL             & De Facto          \\ \hline
\end{tabular}
}
\label{tab:events}
\end{table}

In Groot, there are more than a dozen event types such as \emph{Latency Spike} as listed in the column 2 of Table~\ref{tab:events}. 
Each event type is characterized by three aspects: $Name$ indicates the name of this event type; $Lookback Period$ %\footnote{In Figure~\ref{fig:ex2_n1}, there are two periods, 1 day indicates the look-back range if the model has already finished deployment, 4 days indicates the range if the model deployment is still ongoing(incremental deployment).} 
indicates the time range to look back (from the time when the use of \system is triggered) for collecting events of this event type;  $PropertyType$ indicates the types of the properties that an event of this event type should hold. 
$PropertType$  is characterized by a vector of pairs, each of which indicates the string type for a property's name and the primitive type for the property's value such as string, integer, and float. 
Formally, an event type is defined as a tuple: 
$ET = <Name, Lookback Period, PropertyType>$ 
where 
$PropertyType = <(string, \textit{type}_1), ..., (string, \textit{type}_{n})>$ ($n$ is the number of properties that an event of this event type holds). 
%

Each event of a certain event type $ET$ is characterized by four aspects:
$\textit{Service}$ indicates the service name that the event belongs to; $\textit{Type}$ indicates $ET$'s $\textit{Name}$;  $\textit{StartTime}$ indicates the time when the event happens; $\textit{Properties}$ indicates the properties that the event  holds.
Formally, an event is defined as a tuple: 
$e = <Service, Type, StartTime, Properties>$ 
where $Properties$ is an instantiation of $ET$'s  $PropertyType$. 


%and each event is defined as $e = \{<\textit{Property}_i, \textit{value}_i>\}$. Each event type serves as a template for the event instantiation. such as a string, an integer, a float or a set of primitive types while $\textit{value}$ is limited to primitive data types. 
%
%Each event is defined as a sequence of property-value pairs where the set size is $n$.

For example, in Figure~\ref{fig:example1}, the generated event for \emph{Latency Spike in DataCenter-A} in \emph{Service-C} would be $<``\textit{Service-C}'', ``\textit{Latency\ Spike}'', \textit{2021/08/01-12:36:04}, <(``\textit{DataCenter}'',``\textit{DC-1}''),  ...>>$. %So for each service in $G$, we detect/collect and filter the events within specified time range of the alert.

\subsubsection{Constructing Causal Link}

After collecting all events on all services in $G$, in this step, causal links between these events are constructed for RCA ranking. The causal links (red arrows) in Figure~\ref{fig:ex1_cas} are such examples. A causal link represents that the source event can possibly be caused by the target event. SRE knowledge is engineered into rules and used to create causal links between the pairs of events. %As shown in Figure~\ref{fig:example2}, there are two categories of rules: basic rules and conditional rules. 

A rule for constructing a causal link is defined as a tuple:  $Rule = <Target\mbox{-}Type,  Source\mbox{-}Events, Target\mbox{-}Events, Direction,\\ Target\mbox{-}Service,  Condition>$  ($Condition$ can be optionally specified). $Target\mbox{-}Type$ indicates the type of the rule, being either $Static$ or $Dynamic$ (explained further later). $Source\mbox{-}Events$ indicates the type of the causal link's source event ($Source\mbox{-}Events$ are listed in the names of the rules shown in Figures~\ref{fig:ex2_n1},~\ref{fig:ex2_n2} and~\ref{fig:dynamic_example}).   $Target\mbox{-}Events$ indicates the type of the causal link's target event. $Direction$ indicates the direction of the casual link between the target event and source event. $Target\mbox{-}Service$ indicates the service that the target event should belong to. Note that $Target\mbox{-}Service$ in $Static$ rules can be  $Self$, which indicates that the target event would be within the same service as the source event, or $Outgoing$/$Incoming$, which indicates that the target event would belong to the downstream/upstream services of the service that the source event belongs to in $G$.

\begin{figure}[t]
\centering
\includegraphics[width=0.56\columnwidth]{figures/example3.png}
\caption{Example of dynamic rule}
\label{fig:dynamic_example}
\end{figure}

There are two categories of special rules. The first category is \emph{dynamic} rules (i.e., rules whose $Target\mbox{-}Type$  is set to $Dynamic$) to support dynamic dependencies. Here $Target\mbox{-}Service$ does not indicate any of the three possible options listed earlier but indicates the name of the target service that \system would need to create. For example, live DB dependencies are not available due to different tech stacks and high volume. In Figure~\ref{fig:dynamic_example}, a DB issue (DB Markdown) is shown in \emph{Service-A}. Based on the listed \emph{dynamic} rule, \system creates a new ``service'' \emph{DB-1} in $G$, a new event ``Issues'' that belongs to \emph{DB-1}, and a causal link between the two events.  In practice, the SRE teams use dynamic rules to cover a lot of third-party services and database issues since the live dependencies are not easy to maintain.  %However through the internal error messages and dynamic rules, \system is still able to handle these dependencies. %we can still support external inferences. 

The second category of special rules is \emph{conditional} rules. \emph{Conditional} rules are used when some prerequisite conditions should be satisfied before a certain causal link is created. In these rules, $Condition$ is specified with a boolean predicate. As shown in Figure~\ref{fig:ex2_n2}, the SRE teams believe \emph{Latency Spike} events from different services are related only when both events happen within the same data center. Based on this observation, \system would first evaluate the predicate in $Condition$ and build only the causal link when the predicate is true. A conditional rule overwrites the basic rule on the same source-target event pair.

When constructing causal links, \system first applies the \emph{dynamic} rules so that dynamic dependencies and events are first created at once. Then for every event in the initial services (denoted as $I$), if the rule conditions are satisfied, one or many causal links are created from this event to other events from the same or upstream/downstream services. When a causal link is created, the step is repeated recursively for the target event (as a new origin) to create new causal links. After no new causal links are created, the construction of the event causality graph is finished.

% When \system constructs the causal links, \system first processes all dynamic rules as they may create new event nodes in the graph. %\system enumerates the dynamic rules on each existing event node and also on the newly added nodes (There could also be rules applicable to the newly added nodes) until no new event nodes can be created. 


%Each rule is defined as a predicate containing both events' property-value pair. If the predicate evaluates to be true between two events, then we would add the edge in the causality graph. For example, in Figure~\ref{fig:example1}, the rule used to establish the edge between \emph{GC overhead in RNO} and \emph{Latency increase in LVS, RNO, SLC} would be like this: Suppose we are now determining whether there should be a link from event $u$ to event $v$, then this rule would be $u.\text{pool} = v.\text{pool}\ and\ u.\text{type} = ``\text{High GC Overhead}"\ and\ v.\text{type} = ``\text{Latency increase}"\ and\ u.\text{center} \cap v.\text{center} \ne \emptyset$ which holds true for these two events. Each causality link is also associated with a weight which represents the likelihood of causality - we set all initial values as $1.0$. Overtime these value are updated by the statistical analysis result of the collected data set.


\subsection{Root Cause Ranking}
Finally, \system ranks and recommends the most probable root causes from the event causality graph. Similar to how search engines infer the importance of pages by page links, we customize the PageRank \cite{manning2010introduction} algorithm to calculate the root cause ranking; the customized algorithm is named as GrootRank. The input is the event causality graph from the previous step. Each edge is associated with a weighted score for weighted propagation. The default value is set as $1$, and is set lower for alerts with high false-positive rates. 

Based on the observation that dangling nodes are more likely to be the root cause, we customize the personalization vector as $P_n = f_n $ or $P_d = 1$, where $P_d$ is the personalization score for dangling nodes, and $P_n$ is for the remaining nodes; and $f_n$ is a value smaller than 1 to enhance the propagation between dangling nodes. In our work, the parameter setting is $f_n = 0.5$, $\alpha = 0.85$, $max_{iter} = 100$ (which are parameters for the PageRank algorithm). Figure \ref{fig:person} illustrates an example. The grey circles are the events collected from three services and one database. The grey arrows are the dependency links and the red ones are the causal links with the weight of $1$. Both of the PageRank and GrootRank algorithms detect $event 5$ (DB issue) as the root cause, which is expected and correct. However, the PageRank algorithm ranks $event 4$ higher than $event 3$. But $event 3$ of $\textit{Service-C}$ is more likely to be the second most possible root cause (besides $event 5$), because the scores on dangling nodes are propagated to all others equally in each iteration. We can see that $event 3$ is correctly ranked as second using the GrootRank algorithm.

The second step of GrootRank is to break the tied results from the previous step. The tied results are due to the fact that the event graph can contain multiple disconnected sub-graphs with the same shape. We design two techniques to untie the ranking: 
\begin{figure}[t]
\centering
  \includegraphics[width=0.8\columnwidth]{figures/personalvector.png}
  \caption{Example of personalization vector customization}
  \label{fig:person}
\end{figure}

\begin{figure}[t]
\centering
  \includegraphics[width=0.8\columnwidth]{figures/accessdistance.png}
  \caption{Example of using access distance to untie the ranking results}
  \label{fig:untie}
\end{figure}
\begin{enumerate}
\item For each joint event, the access distance (sum) is calculated from the initial anomaly service(s) to the service where the event belongs to. If any ``access'' is not reachable, the distance is set as $d_m+1$ where $d_m$ is the maximum possible distance. The one with shorter access distance (sum) would be ranked higher and vice versa. Figure \ref{fig:untie} presents an example, where \emph{Service-A} and \emph{Service-B} are both initial anomaly services. Since \system suspects that $event 2$ is caused by either $event 3$ or $event 1$ with the same weight. The scores of $event 3$ and $event 1$ are tied. Then, $event 3$ has a score of $1$ (i.e., $0+1$) and $event 1$ has a score of 2 (i.e., $0+2$), since it is not reachable by \emph{Service-B}). Therefore, $event 3$ is ranked first and logical. 
\item For the remaining joint results with the same access distances, \system continues to untie by using the historical root cause frequency of the event types under the same trigger conditions (e.g., checkout domain alerts). This frequency information is generated from the manually labeled dataset. A more frequently occurred root cause type is ranked higher.% than the less frequent ones.
\end{enumerate}


\subsection{Rule Customization Management}

While \system users create or update the rules,  there could be overlaps, inconsistencies, or even conflicts being introduced such as the example in Figure~\ref{fig:ex2_n2}. \system uses two graphs to manage the rule relationships and avoid conflicts for users. One graph is to represent the link rules between events in the same service (\emph{Same-Graph}) while the other is to represent links between different services (\emph{Diff-Graph}). The nodes in these two graphs are the event types defined in Section~\ref{sec:causality}. There are three statuses between each (directional) pair of event types: (1) no rule, (2) only basic rule, and (3) conditional rule (since it overwrites the basic rule). In \emph{Same-Graph}, \system does not allow self-loop as it does not build links between an event and itself.
% but it is possible that we build links between different services with the same event type.

When rule change happens, existing rules are enumerated to build edges in \emph{Same-Graph} and \emph{Diff-Graph} based on $Target\mbox{-}Events$ and $Target\mbox{-}Service$. Based on the users' operation of 
% \begin{itemize}
%     \item 
    (1) ``remove a rule'',  \system removes the corresponding edge on the graphs;
    % \item 
    (2) ``add/update a rule'',  \system checks whether there are existing edges between the given event types, and then warns the users for possible overwrites. 
    % The users can also combine the conditional rules.   % while users are adding basic rules between event types if there are existing conditional rules between them.
    If there are no conflicts, \system just adds/updates edges between the event types.
    % \item Add conditional rules. We would first alert the possible overwrite. Then if users are about to add new conditional rules on the top of existing conditional rules, we would ask the users to combine these two conditions to add a new one. We then build or change all corresponding edges to status 3.
% \end{itemize} 

After all changes, \system extracts the rules from the graphs by converting each edge to a single rule. These rules are automatically implemented, and then tested against our labeled data set. The \system users need to review the changes with validation reports before the changes go online.

% Note that currently we don't check the consistencies between dynamic rules as we cannot process the dynamic event types, but this could be solved in the future by using nodes with symbolic values to represent such event types. 		  %THis file has high level approach and comparison with wavefront tiling
\section{Code Generation of PCOT}
\label{sec:pcot}

%% Comment: I assume that by this point we have already explained the
%% transformations by tiling hyperplanes and the iteration space can be tiled
%% along canonic directions.

In this section, we describe our generalization of the cache oblivious code
generation to polyhedal programs: Polyhedral Cache Oblivious Tiling (PCOT).

\subsection{Approach Overview}
The input is any polyhedral loop nest that is fully permutable and hence
tiling it with hyper-rectangular tiles is a legal transformation.  We first
(in Section~\ref{sec:perfect}) describe the case for tiling all dimensions of a
perfect loop nest.  Other cases can be handled with additional pre-processing,
which we describe in Section~\ref{sec:codegen_ext}.

Figure~\ref{fig:sample_code} illustrates the structure of our generated code.
The input loop nest is replaced by a call to start the recursion as shown in 
Figure~\ref{fig:sample_code}a. The computation of the bounding box from loop nests 
are discussed in Section~\ref{sec:computingBB}.

\begin{figure*}
  \centering
    \includegraphics[width=0.98\textwidth]{codegen-example}
    \caption{The structure of generated code for Heat2D stencils, which has
      loop depth $d=3$. (a) The tileable loop nest is replaced by a call to
      start the recursion. The input is the bounding box of the loop nest. (b)
      Structure of the recursive function. The input bounding box is split
      into $2^d$ new orthants by dividing each dimention in half, and the same
      function is recursively called for each new orthant. When the orthant
      reaches the input tile size, the recursion is terminated by a call to
      the base function. (c) Code for the base function that
      performs the computation of a tile in lexicographic order.  }
  \label{fig:sample_code}
\end{figure*}



\subsection{Codegen for Perfect Loop Nests}
\label{sec:perfect}

Given a perfectly nested loop with all $d$ dimensions tilable, we seek to
generate,
\begin{inlinelist}
	\item a call to recursive function to start the recursion,
	\item a recursive function, and
	\item a base function.
\end{inlinelist}  
The recursive function takes origin and the size of the orthant as inputs,
which are $d$ dimensional vectors. For the initial call to the recursive
function, the origin and the orthant size correspond to the bounding box of
the input loop nest.  Bounding box of a domain is the smallest
hyper-rectangular shaped domain which encloses the given domain.

The recursive function visits the iteration domain in divide-and-conquer
order.  It recursively divides iteration domain into orthants until they are
smaller than an input parameter.
%The body of the recursive function has 2 main parts (
%Figure~\ref{fig:sample_code}.c), \begin{inlinelist}
%\item a call to the base function if the orthant is small enough, and
%\item divide the orthant into $2^d$ number of new orthants by dividing each
%dimension by 2 and generate recursive function call to visit each new orthant.
%\end{inlinelist}
The call to the base function is wrapped by a condition to check whether the
size of the current orthant is less than or equal to the base case threshold. 
%explaining the execution order without using wavefronts

The orthants are visited sequentially in the lexicographic order.  For
parallel execution the tasks are executed with wave-front parallelism.  We use
the OpenMP tasks for parallel execution, where each recursive function call is
annotated with {\tt omp task} pragma, and the wave-front time boundaries are
synchronized with {\tt omp taskwait}.

% To describe the execution order of orthants, let us represent new orthants
% with a hypercube of $d$ dimensions. Each subcube can be labeled using $d$
% bits.  When $d=3$, subcubes are (0,0,0),(0,0,1),...,(1,1,1). Sum of three
% bits represent the ``distance'' from (0,0,0) subcube. If the difference of
% ``distance'' of two subcubes is one, then those two subcubes are adjacent to
% and has a dependence between them.  For the sequential execution, subcubes
% can be executed in an increasing order of ``distance'' which satisfy the
% order imposed due to the dependences among subcubes. In the generated code,
% this is a sequence of recursive function calls in the increasing order of
% ``distance'' of subcubes. The recursive function call takes the origin and
% size of subcube as inputs (Figure~\ref{fig:sample_code}.c).
%
% For the parallel execution, subcubes are executed in the increasing order of
% ``distance'' where the subcubes with the same ``distance'' can be executed
% in parallel. i.e., we start with (0,0,0) after that (0,0,1), (0,1,0), and
% (1,0,0) can be executed in parallel. We spawn parallel tasks using
% \texttt{OpenMP task} directives. Then use \texttt{taskwait} directive to
% enforce the increasing order of ``distance''.

% We generate $2^d$ recursive calls with origin and size of the orthant as
% parameters to each call.  The canonic directions are legal tiling
% hyperplanes (since hyper-rectangular tiling is legal), therefore the 45
% degree wavefronts is a legal schedule to execute new orthants. All the
% orthants within the wavefront can be run in parallel. To generate parallel
% code, we group together the recursive calls within the same wavefront. Then
% insert an OpenMP \texttt{\#pragma omp task} before each recursive call and a
% \texttt{\#pragma omp taskwait} after each group to act as a thread barrier
% before moving on to the next group of tasks.
%
% This makes sure that after each group, worker threads wait till all the
% tasks in the group are finished before moving on to the next group of tasks.

The base function visits all points in the intersection of leaf orthant and
the input loop nest iteratively in lexicographical order.  The loop nest in
the base function is identical to the loop nest for a tile in SLT with
parametric tile sizes we use.  One can use any one of the available
parametrically tiled code generators~\cite{sanjay-lcpc2009,
  sanjay-kim-dtilingTR-2010,baskaran-etal-cgo10,iooss:hal2015} to generate
parametrically tiled loops and then extract the point loops.
%% removed since double-blind
% We use the point loops generated by
% DTiling~\cite{sanjay-lcpc2009,sanjay-kim-dtilingTR-2010} code generator in
% AlphaZ~\cite{yuki2013alphaz} system\footnote{Since we use the same code to
% iterate a tile in single-level tiling code and to iterate base case in cache
% oblivious code, the underlying compiler is capable of performing same set of
% optimizations for both codes.}.
%% Following is removed since this is not a feature but kind of a bug
% In divide-and-conquer style codes, size of the base case at run time can be
% less than or equal to the tuned size of the base case. When the problem size
% (at runtime) is a power of 2 of tuned base case size, then the actual base
% case size is equal to the tuned base case size, otherwise, base case size is
% smaller than the tuned base case size.  The actual base case size depends on
% the problem size parameter, therefore, it is important to use a
% parametrically tiled code generator to extract the point loops from.


\subsection{Optimizations}
\label{sec:codegen:opt}
We implement a number of optimizations to improve the speed of code.  In this
section we explain two important optimizations to exit the recursion early.

\subsubsection{Early Exit for Zero Orthants}

The zero orthants surface when we have tile sizes that are cubic bounding box with
hyper-rectangular tiles and vise versa. In this case, orthant size along all
the dimensions reaches the leaf size in different levels of the recursion. But
still our code generator generates $2^d$ sequence of recursive calls where
size some of the new orthants may be zero along dimensions where the input
orthant size is already smaller than the leaf threshold.  This is a simple
optimization where we check whether a width along anyone of the dimensions of
the orthant is zero. If it is zero then exit the recursion.
%In the recursive function, we only divide a
%dimension by two, if the size of the orthant along this dimension is larger
%than that of the tuned base case size. Therefore, it is possible that some
%dimensions are not divided in the middle, leading to new orthants with zero
%width along some dimensions. Lets assume that the origin of the orthant is
%$(0,0)$, the size of the orthant is $(32\times32)$ and the base case threshold
%is $(16\times32)$. In this case, we only divides the first dimension by 2
%(because 32 $>$ 16) and we do not divide the second dimension. This introduce
%only 2 new orthants where origin is $(0,0)$, size $(16\times32)$ and origin is
%$(16,0)$, size $(16\times32)$. But in the recursive function there are $2^2$
%recursive calls (or new orthants). The two of them corresponds to the two new
%orthants mentioned above and other two recursive calls have size of the 2nd
%dimension as 0 since we did not divide it by two because the size of it is
%already matches the base threshold. Hence, there are two recursive calls with
%the size of the orthant is 0. This optimization filter these orthants out.

\subsubsection{Early Exit for Empty Orthants}
Second optimization is due to the fact we visit the bounding box of the input
loop nest instead of the actual domain. Some kernels operate on
non-hyper-rectangular domains(i.e, Cholesky  operates on triangular domains)
and some kernels need iteration space skewing to enable hyper-rectangular
tiling (i.e., Heat2D). If the actual domain is not a hyper-rectangle then the
bounding box will have points where no computations are defined.  This may
lead to orthants outside of the original iteration space, analogous to empty
tiles in iteration space tiling. 

For example, a loop nest whose iteration space is triangular, the boundingbox
is a rectangle where the half of the points has no computation defined. When
we visit this rectangular box in divide-and-conquer order, we will end up
visiting empty orthants and we want to identify these orthants.  We generate
conditions to check whether all vertices of the orthant remain outside of the
original iteration space using \emph{isl} library~\cite{verdoolaege2010isl}.
%If that is the case, we exit the recursion early 

%We generate code to check whether the current orthant is completely outside
%the actual domain of the loop nest.  We check whether all the vertices of the
%orthant unsatisfy at least one of the constraints of the actual domain of the
%input loop nest.  If that is the case we exit the recursion early.  The
%conditions are generated and simplified using the \emph{isl}
%library~\cite{verdoolaege2010isl}.

The \texttt{checkEmpty} method at the top of the
Figure~\ref{fig:sample_code}c implements both optimizations.  Without them,
the code produces correct answers but visits many base cases which are empty.
The recursion ends either when \texttt{checkEmpty} method returns true or when
the orthant reaches its input tile size.

%\subsubsection{Optimization 3}
%If an orthant is smaller than or equal to the base size parameter, it will
% exit the recursion and compute all the points in the orthant. In this case,
% base size parameter act as an threshold.  The actual orthant size can be
% less than or equal to the provided threshold.  The actual base size depends
% on the problem size. For example if the problem size is 1024 then the
% orthant sizes are 1024, 512, 256, 128, 64, 32,... For any base size
% parameter from 32 to 63, the actual base size going to be 32. If the problem
% size is 1100, then the orthant sizes are 1100, 550, 275, 137, 68, 34,... For
% any base size parameter from 34 to 67, the actual base size going to be
% 34. Therefore, the base size we want to use may not be the actual base size
% and it heavily depend on the problem size and out of our control.
%
% We want the generated code to have the exact base case size we provide as a
% parameter(s). The solution is simple, we pad the bounding box of the
% iteration domain so that size of each dimension is multiple of its base size
% parameter and some power of two. If the size of the $i_{th}$ dimension is
% $N_i$, base case size parameter is $b_i$ then the new padded size is the
% minimum value of $b_i\times2^{k} \ge N_i$ where $k$ is an integer and
% $k \ge 0$. When you use the padded bounding box as the starting orthant,
% after $k$ levels of recursion, we reach orthants with size corresponds to
% the base case parameter sizes and then exit the recursion to compute the
% points within the orthant.  The extra points without any computations due to
% the padding is optimized by the Optimization 2.

% \subsubsection{Other optimizations}
% In addition to the main optimizations discussed above, we added
% \emph{restrict} keyword to the declarations of pointer variables. This helps
% the compiler to optimize code knowing that pointer variables are not
% referred using other aliases.
%
% We only transform the loop nests which are dominant in computations.  The
% input loop nests may contain loop nests of different dimensions (depths).
% The number of computations in the loop nests with smaller number of
% dimensions are significantly smaller than the main-loop which is the loop
% nest(s) with higher number of dimensions. We optimize only the main-loop(s)
% which is the hotspot of the input loop nest.
%
% The generated code is parametric in problem size and base case size. Base
% case size is a vector that specify the size along all the tilable
% dimensions. The parameters of recursive and base functions are passed using
% C structs to reduce the number of function parameters in higher dimensional
% programs.

\subsection{Handling Imperfect Loop Nests}
\label{sec:codegen_ext}
The discussion so far assumed perfectly nested loops as inputs.  We now extend
this to imperfect loop nests, and loop nests where subset of the $d$
dimensions are marked as tilable.

%\subsection{Handling imperfectly nested loops}
The input imperfect loop nests are converted to perfect loop nests with a
pre-processing.  This is called the embedding transformation that are used to
handle imperfectly nested loops in parametric tiled code
generation~\cite{sanjay-kim-dtilingTR-2010}. 
It involves, bringing all the statements into the same loop depth by adding
loops with one iteration as necessary. Then affine guards are added to
eliminate sequence of inner loops, which lead to ``perfect loop nest with
affine guards''.

% The conversion process includes
%the following high level steps.  First, we bring all statements in the
%imperfect loop nest into the same loop depth.  This may result in a sequence
%of inner loops.  Then, we use affine guards to eliminate the sequence of
%loops, and we call it ``perfect loop nest with affine guards''.
%%We specify options to Cloog so that the generated loop nest is already perfectly nested.

When a subset of the loops are marked as tilable, we first extract the marked
band of loops parameterized by both program input parameters as well
as untiled outer loop iterators. We apply the techniques we have discussed so
far to generate code for the extracted loop nest. In this case, the function
call to start the recursion is added as the body of outer untiled loop nest.
The inner untiled loop nests are added as the body of the point loop in the
base function.
%For the case where a subset of the loops are marked as tilable, let us assume
%that a band of loops from depth $i$ to $k$ are tilable, and loops from depth 0
%to $i-1$ and $k+1$ to $d$ are not tilable.  The first step is to extract the
%tilable band of loops from depth $i$ to $k$.  Now, we have a loop nest where
%all the dimensions are tilable therefore, we can apply the above techniques to
%generate a recursive function, base function and a function call to start the
%recursion.  The function call to start the recursion is added as the body of
%outer untiled loops nest at depth $i-1$. The inner untiled loops from depth
%$k+1$ to $d$ are set as the body of the innermost loop in the base function.

% Let's consider that we want to tile only the space dimensions (inner 2
% dimensions) of Heat-2D stencil. We extract the inner two loops and generate
% new recursive function and a base function as described in the beginning of
% this section. Now, there is a recursive function call for each outer time
% step. Therefore, we place the function call to start the recursion as the
% body of the untiled outer loop.

% Let's go through an example.  Let's consider adding another outer loop to
% the program in Listing~\ref{lst:motiv_fully}. The resulting loop nest is
% shown in Lising~\ref{lst:motiv_band}. Lets also assume that only the inner 2
% loops are tilable. In this case we extract the tilable loop nests (from
% lines \ref{lst:motiv_band:1s}-\ref{lst:motiv_band:1e}) and provide it as the
% input to the PCOT code generator. This ends up generating a recursive
% function similar to Listing~\ref{lst:motiv_recur}. The base function is
% similar to Listing~\ref{lst:motiv_base}. The recursive function call to
% start recursion will be the body of outer most untiled-loop as shown in
% Listing~\ref{lst:motiv_band_start}. The bounding box (orthant size) is a
% function of the outer loop iterator $c0$.
%

\subsection{Computing the Bounding Box}
\label{sec:computingBB}
The bounding box is a hyper-rectangle containing the iteration space of the
loop nest. It is computed by eliminating the outer loop indices from the loop
bound expressions.  There can be infinitely many bounding boxes for a given
loop nest, but we start with the tightest (smallest) bounding box among all
the possibilities.

The bounding box also plays an important role in deciding the leaf tile size.
We want the leaf tile size to have the exact value as the input tile size
parameter to the generated code. Therefore, we pad the size of bounding box
along each dimension to the minimum value of $b_i\times2^{k_i} \ge N_i$ where
$b_i$ input leaf size parameter, $N_i$ the size along the $i^{th}$ dimension
of bounding box and $k_i \ge 0$ is an integer. Now, at $k_i$th level of
recursion the orthant size along the $i$th dimension will be $b_i$. The
padding of the bounding box introduces iteration points outside of original
iteration space.  These empty points get optimized away by the optimizations
described in Section~\ref{sec:codegen:opt}


%
%The loop bounding expressions of the inner loop indices may be functions of
%outer loop indices.  We eliminate outer loop indices from bounding expressions
%starting from the outer most loop.  In lower bound expressions, we replace
%outer index with its lower bound if the sign of the index is negative, upper
%bound if the sign is positive.  In upper bound expressions, we replace index
%with its upper and lower bound if the sign is positive and negative
%respectively.  At the end all the bounding expressions are functions of input
%parameters and iterators of the loops surrounding bounding box.
%
% Let's consider the scenario where only a subset of the loops are tilable,
% then the bounding box is a function of both input parameters and indices of
% outer untiled loops. In other words, for each instance of outer untiled
% iteration, there is a instance of the bounding box. The approach is same as
% above except, we do not eliminate outer untiled loop indices in the bounding
% expressions.
%
% For a given loop nest, tilable band specifies band of loops which are
% tilable or permutable. Usually, the band is specified by the start loop
% depth and end loop depth.  Bounding box is a cuboid (or hyper-rectangle)
% containing the iteration space of the tilable band.  There can be infinitely
% many bounding boxes for a given tilable band, but we chose the tightest
% bounding box among all the possibilities. Since there are outer loops
% surrounding the tilable band of loops, the bounding box is parameterised by
% outer loop iterators. In other words, there is a bounding box for each
% instance of outer loops.
%
%When the input is a sequence of perfectly nested loops, the corresponding
%domain can be an union of polyhedra.  In this case, we compute the bounding
%box of each polyhedra individually as described above.  Then, we compute the
%upper bound along a given dimension by taking the maximum of upper bounds of
%all the polyhedra along the same dimension.  Lower bound is computed similarly
%by taking the minimum of lower bounds of all the polyhedra along a given
%dimension.  The resulting upper bounds and lower bounds define the bounding
%box of the union of polyhedra.
%
%
% Simply using the bounding box as the input orthant size to start the
% recursion may leads to issues with the actual base case size. For example,
% lets assume a cubic bounding box with size 512 along each dimension. Then
% the subsequent orthant sizes are 512, 256, 128, 64, 32, ... For any base
% size parameter from 32 to 63, the actual base size going to be 32. If the
% problem size is 550, the orthant sizes are 550, 275, 137, 68, 34,... For any
% base size parameter from 34 to 67, the actual base size going to be
% 34. Therefore, the base size we want to use may not be the actual base size
% and it heavily depends on the size of the bounding box and out of our
% control.  We can get control of it, if
%
% In this work, we use parametric tile sizes for cache oblivious tiling code.
% Existing techniques for parametric tiling~\cite{sanjay-lcpc2009,
% baskaran-etal-cgo10} have demonstrated that the parameterization does not
% incur significant performance overhead when compared to tiling by
% compile-time constants. We implemented our cache oblivious tiling code
% generator using Cloog and ISL libraries.
%

% We implemented our cache oblivius tiling code generator in the AlphaZ
% system~\cite{alphaz} which is capable of generating parametrically tiled
% code using D-Tiling~\cite{sanjay-lcpc2009}.

% Local Variables: ***
% TeX-master: "TACO2017.tex" ***
% fill-column: 78 ***
% End: ***
       % This file has the codegenerator details
%\section{The \MakeLowercase{i}W\MakeLowercase{inr}NFL model}
\label{sec:model}

In this section we are going to present the data we used to develop our in-game probability model as well as the design details of {\method}. 

{\bf Data: }In order to perform our analysis we utilize a dataset collected from NFL's Game Center for all the regular season games between the seasons 2009 and 2016. 
We access the data using the Python {\tt nflgame} API \cite{nflgame}. 
The dataset includes detailed play-by-play information for every game that took place during these seasons. 
This information is used to obtain the state of the game that will drive the design of {\method}. 
In total, we collected information for 2,048 regular season games and a total of 338,294 snaps/plays. 

{\bf Model: }
{\method} is based on a logistic regression model that calculates the probability of the home team winning given the current status of the game as: 

\begin{equation}
\Pr(H=1| \mathbf{x})= \frac{\exp(\mathbf{\weight}^T\cdot\mathbf{x})}{1+\exp(\mathbf{\weight}^T\cdot\mathbf{x})}
\label{eq:reg}
\end{equation}
where $H$ is the dependent random variable of our model representing whether the home team wins or not, $\mathbf{x}$ is the vector with the independent variables, while the coefficient vector $\mathbf{\weight}$ includes the weights for each independent variable and is estimated using the corresponding data.  
For a game of infinite duration a linear model could be a very good approximation.  
However, the boundary effects from the finite duration of a game create several non-linearities \cite{winston2012mathletics}.  
For this reason, we enhance our model - using the same set of features - with a Support Vector Machine classifier with radial kernel for the last three minutes of regulation.  
In order to obtain a probability output from the SVM classifier, we further use Platt's scaling \cite{platt1999probabilistic}: 

\begin{equation}
\Pr(H=1| \mathbf{x})= \frac{1}{1+\exp{(Af(x)+B)}}
\label{eq:platt}
\end{equation}
where $f(x)$ is the uncalibrated value produced by the SVM classifier: 

\begin{equation}
f(x) = \sum_{i} (\alpha_i y_i k(\mathbf{x}_i\cdot\mathbf{x}))+ b
\label{eq:svm}
\end{equation}
where $k(\mathbf{x},\mathbf{x}')$ is the kernel used for the SVM.   
Figure \ref{fig:iwinrNFL} depicts the simple flow chart of {\method}. 


\begin{figure}[t]
\begin{center}
\includegraphics[scale=0.35]{plots/iwinrNFL.pdf}%\vspacecap
 \caption{{\method} includes a linear and a non-linear component.}
 \label{fig:iwinrNFL}
\end{center}
\end{figure}

In order to describe the status of the game we use the following variables:

\begin{enumerate}
\item {\bf Ball Possession Team:} This binary feature captures whether the home or the visiting team has the ball possession
\item {\bf Score Differential:} This feature captures the current score differential (home - visiting)
\item {\bf Timeouts Remaining:} This feature is represented by two independent variables - one for the home and one for the away team - and they capture the number of timeouts remaining for each of the teams
%\item {\bf Quarter:} This feature captures the current quarter of the game
%\item {\bf Time Remaining:} This feature captures the time (in seconds) remaining for the current quarter to end
\item {\bf Time Elapsed: } This feature captures the time elapsed since the beginning of the game
\item {\bf Down:} This feature represents the down of the team in possession
\item {\bf Field Position:} This feature captures the distance covered by the team in possession from their own yard line
\item {\bf Yards-to-go:} This variables represents the number of yards needed for a first down
\item {\bf Ball Possession Time: } This variable captures the time that the offensive unit of the home team is on the field 
\item {\bf Ranking Differential: } This variable represents the difference of the win percentage for the two team (home - visiting)
\end{enumerate}

The last independent variable is representative of the power ranking difference between the two teams. 
Most of the existing models that include such a variable are using the Vegas line spread for each game.  
We choose not to do so for the following reason.  
The objective of the Vegas line is not to predict game outcomes but rather distribute money across the different bets.  
Exactly because of this objective the line is changing during the week before the game.  
While this line can change due to new information for the competing teams (e.g., injury updates), the line is mainly changing when a particular team has accumulated the majority of the bets. 
In this case it will also be hard to choose which line to use (e.g., the opening, the closing or some average of them).  
Therefore, we choose to use the win percentage differential of the two teams as an indicator of their strength (even though this has its own issues given the uneven schedule in NFL).  
However, note that if one would like to use the point spread as a variable this can be easily incorporated in the model. 
Table \ref{tab:iwinrnfl} presents the coefficients of the logistic regression model of {\method} with standardized independent variables for better comparisons. 


\begin{table}[ht]
\begin{center}
\def\sym#1{\ifmmode^{#1}\else\(^{#1}\)\fi}
\begin{tabular}{l*{1}{c}}
\toprule
                    &\multicolumn{1}{c}{(1)}\\
                    &\multicolumn{1}{c}{Winner}\\
\midrule
Possession Team (H)         &      0.41\sym{***}\\
                    &     (49.19)         \\
\addlinespace
Score Differential           &      3.59\sym{***}\\
                    &    (247.34)         \\
\addlinespace
Home Timeouts           &     0.12\sym{***}\\
                    &      (8.74)         \\
\addlinespace
Away Timeouts           &     -0.11\sym{***}\\
                    &    (-12.47)         \\
\addlinespace
Ball Possession Time  &     -0.05.\\
                    &    (-1.66)         \\
\addlinespace
Time Lapsed       &   -0.05.\\
                    &      (-1.66)         \\
\addlinespace
Down                &   -0.01         \\
                    &      (0.04)         \\
\addlinespace
Field Position            &   0.02\sym{**} \\
                    &      (2.71)         \\
\addlinespace
Yards-to-go                &  -0.01         \\
                    &      (0.23)         \\
\addlinespace
Rating differential         &       0.75\sym{***}\\
                    &     (80.47)         \\
\addlinespace
Intercept            &       0.57\sym{*}\\
                    &    (2.09)         \\
\midrule
Observations        &      338,294         \\
\bottomrule
\multicolumn{2}{l}{\footnotesize \textit{t} statistics in parentheses}\\
\multicolumn{2}{l}{\footnotesize \sym{$_.$} \(p<0.1\), \sym{*} \(p<0.05\), \sym{**} \(p<0.01\), \sym{***} \(p<0.001\)}\\
\end{tabular}
\end{center}
\caption{Standardized logisitic regression coefficients for {\method}.}
\label{tab:iwinrnfl}
\end{table}


As we can see, as one might have expected the current scoring differential exhibits the strongest correlation with the in-game win probability.  
The only factors that do not appear to be statistically significant predictors of the dependent variable are the down and the yards-to-go. 
Even though the corresponding coefficients are negative as one might have expected (e.g., being at an earlier down gives you more chances to advance the ball), they are not significant in estimating the win probability. 
On the contrary, all else being equal timeouts appear to be quiet important since they can help a team stop the clock, while teams with better win percentage appear to have an advantage as well, since this can be a sign of a better team. 
In the following section we provide a detailed evaluation of {\method}.         % This file has the analytical model of cache misses

\section{Experiments}\label{sec:experiments}
We validate our approach using multiple datasets containing real-life data from the fields of criminal risk assessment, credit, lending, and college admissions. In each of the datasets we select a binary feature and treat it as the protected attribute (e.g., race or gender), which is the feature we require our trained classifier to behave fairly upon. Our proposed method performs well on all of these datasets, succeeding in removing unfairness almost entirely, at a very modest price in terms of accuracy.


\begin{table*}[h]
\centering
\resizebox{\textwidth}{!}{
\def\arraystretch{1.2}

\begin{tabular}{c c c | c | c | c || c | c | c || c | c | c |}

\cline{4-12}
&&&
\multicolumn{9}{ c| }{\textbf{COMPAS Dataset}}
\\ \cline{4-12}
&&&
\multicolumn{3}{ c|| }{\textbf{FPR Considerations}}&
\multicolumn{3}{ c|| }{\textbf{FNR Considerations}}&
\multicolumn{3}{ c| }{\textbf{Both Considerations}}
\\ \cline{4-12}
&&&
 $\mathbf{Acc.}$ &  $\mathbf{D_{FPR}}$ &  $\mathbf{D_{FNR}}$ &  $\mathbf{Acc.}$ &  $\mathbf{D_{FPR}}$ &  $\mathbf{D_{FNR}}$ &  $\mathbf{Acc.}$ &  $\mathbf{D_{FPR}}$ &  $\mathbf{D_{FNR}}$
\\  \cline{4-12}
\vspace*{-0.5ex}
\\ \cline{1-2} \cline{4-12}
\multicolumn{1}{ |c  }{} &
\multicolumn{1}{ c|  }{  \textbf{Our Method (AVD Penalizers)}}  &&
$\mathbf{0.660}$    &  $\mathbf{0.01}$  &  $0.04$ &
$\mathbf{0.653}$    &  $0.02$   &  $\mathbf{0.04}$ &
$\mathbf{0.654}$    &  $\mathbf{0.02}$  &  $\mathbf{0.04}$
\\ \cline{1-2} \cline{4-12}
\multicolumn{1}{ |c  }{} &
\multicolumn{1}{ c|  }{  \textbf{Our Method (SD Penalizers)}}  &&
$\mathbf{0.664}$    &  $\mathbf{0.02}$  &  $0.09$ &
$\mathbf{0.661}$    &  $0.05$   &  $\mathbf{0.03}$ &
$\mathbf{0.661}$    &  $\mathbf{0.02}$  &  $\mathbf{0.03}$
\\ \cline{1-2} \cline{4-12}
\multicolumn{1}{ |c  }{} &
\multicolumn{1}{ c|  }{  Zafar et al.~(\citeyear{disparatemistreatment})}  &&
$0.660$    &   $0.06$    &   $0.14$  &
$0.662$    &   $0.03$    &   $0.10$  &
$0.661$    &   $0.03$    &   $0.11$
\\ \cline{1-2} \cline{4-12}
\multicolumn{1}{ |c  }{} &
\multicolumn{1}{ c|  }{  Zafar et al. Baseline~(\citeyear{disparatemistreatment})}  &&
$0.643$    &   $0.03$    &   $0.11$  &
$0.660$    &   $0.00$    &   $0.07$  &
$0.660$    &   $0.01$    &   $0.09$
\\ \cline{1-2} \cline{4-12}
\multicolumn{1}{ |c  }{} &
\multicolumn{1}{ c|  }{  Hardt et al.~(\citeyear{hardt})}  &&
$0.659$    &  $0.02$    &   $0.08$  &
$0.653$    &  $0.06$   &    $0.01$  &
$0.645$    &  $0.01$   &    $0.01$
\\ \cline{1-2} \cline{4-12}
\multicolumn{1}{ |c  }{} &
\multicolumn{1}{ c|  }{  \textbf{Vanilla Regularized Logistic Regression}}  &&
$\mathbf{0.672}$    &   $\mathbf{0.20}$    &   $\mathbf{0.30}$  &
$\mathbf{0.672}$    &   $\mathbf{0.20}$    &   $\mathbf{0.30}$  &
$\mathbf{0.672}$    &   $\mathbf{0.20}$    &   $\mathbf{0.30}$
\\ \cline{1-2} \cline{4-12}
\end{tabular}
}
\vspace{3mm}
\caption{Performance comparison on the COMPAS dataset. For the approaches in bold -- Accuracy, FPR difference and FNR difference are evaluated on the test set, averaging over five runs and using a 70-30 training/test split. The performance of the remaining three approaches is stated as reported in Zafar et al.~(\citeyear{disparatemistreatment}).} \label{table:comparison_results}
\end{table*}



\begin{figure*}[b]
  \includegraphics[scale=0.6]{compas0-400.png}
  \caption{COMPAS Dataset. Accuracy, FPR difference ($\mathbf{D_{FPR}}$), and FNR difference ($\mathbf{D_{FNR}}$) (all evaluated on the test set) of the learned classifier, as a function of the weight $c=c_1 = c_2 \geq 0$ placed on the fairness penalizer terms. On the left we use the Absolute Value Difference (AVD) penalizer, and the Squared Difference (SD) penalizer on the right, both as presented in Section~\ref{regularization}. ``Relaxed FPR/FNR Diff.'' plots the value of the relevant penalization term.} %In this particular run, parameters chosen for the absolute value relaxation were: $c=80, q_c=60$, and for the squared relaxation: $c=220, q_c=30$.}
  \label{fig:compas}
\end{figure*}


\subsection{Implementation}
\textbf{Our method} 
%We instantiate our method in the following way: Given dataset $Q$, we split it randomly into a training set $S$ (which we will use for learning) and a test set $T$ (which we will only use for reporting performance). 
For the purpose of comparison with  Zafar et al.~(\citeyear{disparatemistreatment}) and Hardt et al.~\cite{hardt} on the COMPAS data, we use a parameter $c$ to induce three possible combinations of weights on the FPR and FNR penalization terms: $c = c_1$ and $c_2 = 0$; $c_1 = 0$ and $c = c_2$; and $c = c_1 = c_2$. For the other three datasets, we consider only $c = c_1 = c_2$.\footnote{The reason for varying the values of $c$ in the training phase is since we shifted to a proxy problem, in which we rely on the distance from the decision boundary rather the actual classifications. 
%Our hope is that there is no need for a worst-case cross validation between all of the combinations of $c_1, c_2, c_3$, and that the training scheme we propose is sufficient. 
It is possible, of course, that even better results are attainable using our scheme with other combinations of $c_1, c_2$, and $q$.} To explore the accuracy/fairness trade-off curve for the relaxed optimization problem~(\ref{eq:2}), we train for different values of $c$, starting at $c=0$ (which is just standard logistic regression), and growing gradually.



Given a dataset $Q$ and fixing a $d_1, d_2 \in \{0, 1\}$ of interest, we use the following training scheme:
\begin{enumerate}
\item Split $Q$ at random into training set $S$ and test set $T$.
\item For each $c$, perform cross-validation on $S$ to select the corresponding best value $q_c$ for the regularization parameter.
\item For each $(c,q_c)$, let $\theta_c = \argmin\limits_{\theta} \text{Proxy}(\theta;S,c,c,q_c)$.
\item Select $\theta^* \in \argmin\limits_{\theta_c} \text{Objective}(\theta_c;S,d_1,d_2)$.
\item Evaluate performance using $\theta^*$ on test set $T$.
\end{enumerate}
We report the average of five such runs, each with a fresh training-test split.




%We instantiate our method by solving the relaxed optimization problem~(\ref{eq:2}), in place of the original, non-convex problem~(\ref{eq:1}).  
%We test our approach with three different combinations of weights on the penalization terms:
%\katrina{What are the $d$, and how are they related to the $c$s?}
%\begin{enumerate}
%\item FPR considerations only: $d_1 = 1, d_2 = 0$.
%\item FNR considerations only: $d_1 = 0, d_2 = 1$.
%\item Both FPR, FNR considerations, assigned similar significance: $d_1 = 1, d_2 = 1$.
%\end{enumerate}
%One could, of course, pick any other combination of the FPR and FNR penalty weights.

%\katrina{I don't understand how the below is distinct from the list above}
%Learning is done by training the parameters of a logistic regressor to solve~\ref{eq:2}, while picking the value of $c_1, %c_2$ as the following:
%\begin{enumerate}
%\item FPR considerations only: $c_1 = c \geq 0$, $c_2 = 0$.
%\item FNR considerations only: $c_1 = 0$, $c_2 = c \geq 0$.
%\item Both FPR, FNR considerations, assigned similar significance: $c_1 = c_2 = c \geq 0$
%\end{enumerate}



% We then cross-validate to pick the best $c_3$ (the weight on the standard $\ell_2$-regularization term) given $c$.\footnote{The reason for varying the values of $c$ in the training phase is since we shifted to a proxy problem, in which we rely on the distance from the decision boundary rather the actual classifications. 
%Our hope is that there is no need for a worst-case cross validation between all of the combinations of $c_1, c_2, c_3$, and that the training scheme we propose is sufficient. 
%It is possible, of course, that even better results are attainable using our scheme with other combinations of $c_1, c_2, c_3$.} For each such combination, we report results as the averages of multiple \katrina{how many?} different runs, each time splitting data randomly into training and test sets.
%\yahav{We need to shorten this description.}

We solve the relaxed convex optimization problem using the CVXPY solver. Due to stability issues with large training sets, we use a train/test split of 30-70 on the larger datasets, rather than 70-30 as on the COMPAS dataset\footnote{The code implementing our method can be found at https://github.com/jjgold012/lab-project-fairness}.

%
%
%We then report the results (as evaluated on the test set) attained by a regressor $\theta \in \mathbb{R}^d$ that minimizes (on the training set $S$) a weighted combination of the $0$-$1$ loss and the differences in FPR and FNR across populations:
%\begin{equation*}
%\begin{aligned}
%&\underset{\theta}{\text{argmin}}
%& & L_{S}^{0\text{-}1}(\theta) \\
%&&& + d_1|FPR_{A=0}(\theta;S)-FPR_{A=1}(\theta;S)| \\
%&&& + d_2|FNR_{A=0}(\theta;S)-FNR_{A=1}(\theta;S)|
%\end{aligned}
%\end{equation*}
%
%\katrina{What is $d_1$ vs. $c_1$ etc.?}



%For classification, we decided use a standard cut-off threshold of $c=0.5$. There are of course, further possible interactions between the FPR, FNR considerations, and picking a certain cut-off level. These are not straightforward, since  these interactions are data-specific. 



%allows for flexibility in picking the values of $c_1, c_2$, which reflect the significance we wish to place on the objectives of achieving accuracy, equal FPR, and equal FNR. As for $c_3$, we will want to find the value of it that achieves the best results, for any combined objective of accuracy and fairness defined by a specific selection of $c_1,c_2$. Therefore, given a specific selection of $c_1, c_2$, we apply cross-validation to select the value of $c_3$. 




We briefly describe the other algorithmic approaches to which we compare:\\
\textbf{Zafar et al.}~(\citeyear{disparatemistreatment}) performs optimization by considering a proxy for the bias: the covariance between the samples' sensitive attributes and the signed distance between the feature vectors of misclassified users and the classifier decision boundary.\\
\textbf{Zafar et al. Baseline}~(\citeyear{disparatemistreatment}) tries to enforce equal FP/FN rates on the different groups by introducing different penalties for misclassified data points with different sensitive attribute values during the training phase.\\
\textbf{Hardt et al.}~(\citeyear{hardt}) performs post-processing on a standard trained (unfair) logistic regressor, picking different decision thresholds for different groups, and possibly adding randomization.


\subsection{Experimental Results}

In what follows, we use the following notation, given a trained classifier $\hat{Y}$:
\begin{align*}
\mathbf{D_{FPR}}&=\left|FPR_{A=0}(\hat{Y})-FPR_{A=1}(\hat{Y})\right| \\ 
\mathbf{D_{FNR}}&=\left|FNR_{A=0}(\hat{Y})-FNR_{A=1}(\hat{Y})\right|
\end{align*}
The values $FPR_{A=0}(\hat{Y})$, $FPR_{A=1}(\hat{Y})$, $FNR_{A=0}(\hat{Y})$, $FNR_{A=1}(\hat{Y})$ are reported as evaluated on the test set.

\paragraph{The COMPAS Dataset\footnote{https://github.com/propublica/compas-analysis}} The Correctional Offender Management Profiling for Alternative Sanctions (COMPAS) records from Broward County, Florida 2013-2014, made available online by ProPublica, are perhaps the best-studied data in the context of fairness.  The goal in this scenario is to successfully predict recidivism within two years, based on features such as age, gender, race, number of prior offenses, and charge degree. The dataset contains 5,278 samples. The protected attribute in this scenario is race, where $A$ indicates black or white. We filtered the dataset using the same features as Zafar et al.~(\citeyear{disparatemistreatment}), to allow for comparison.

%\begin{table}[h]
%\centering
%\begin{tabularx}{\columnwidth}{c|c|c|c}
%\hline
%  &  Recid. ($y = 1$)        & No Recid.  ($y = 0$)       & Total \\ \hline
%Black &  $ 1661   $ & $ 1514 $ &  $ 3175 $ \\ \hline
%White &  $ 822   $  & $1281  $ &  $ 2103 $ \\ \hline
%Total &  $ 2483  $  & $2795 $ &  $ 5278 $ \\\hline
%\end{tabularx}
%\caption{Statistics of the ProPublica COMPAS data.} \label{table:compas-stats}
%\label{tab:stats}
%\end{table}
%\vspace{-1em}

%\begin{table}[h]
%\centering
%\begin{tabularx}{\columnwidth}{c|c}
%\hline
%Feature  &  Description \\ \hline
%Age Category &  $<25$, between $25$ and $45$, $>45$ \\
%Gender &  Male or Female \\
%Race &  White or Black \\
%Priors Count &  0--37 \\
%Charge Degree &  Misconduct or Felony \\
%\hline
%2-year-recid. & Whether or not the  \\
%(target feature)  & defendant recidivated within two years
%\end{tabularx}
%\caption{Description of features used from ProPublica COMPAS data.} \label{table:compas-features}
%\label{tab:features}
%\end{table}




\begin{table*}[t]
\centering
\caption{A description of the datasets used, along with parameters of the training procedure used for each.}
\label{table:datasets_description}
\begin{adjustbox}{max width=\textwidth}
\begin{tabular}{|l|l|l|l|l|l|l|l|}
\hline
\textbf{Dataset} & \textbf{No. Samples} & \textbf{No. Features} & \textbf{Train/Test Split} & \textbf{No. Repetitions} & \textbf{No. Folds in CV} & \textbf{Protected Feature} & \textbf{Target Variable} \\ \hline
COMPAS           & 5,278                     & 5                          & 70-30                     & 5                        & 5                                 & Race                       & 2-Year-Recidivism        \\ \hline
Adult            & 30,162                    & 10                         & 30-70                     & 5                        & 5                                 & Gender                     & Income Over/Under 50K    \\ \hline
Default          & 30,000                    & 23                         & 30-70                     & 5                        & 3                                 & Gender                     & Defaulting On Payments   \\ \hline
Admissions       & 20,839                    & 17                         & 30-70                     & 5                        & 3                                 & Race                       & Passing Bar Exam         \\ \hline
\end{tabular}
\end{adjustbox}
\end{table*}


\begin{table*}[t]
\centering
\resizebox{\textwidth}{!}{
\def\arraystretch{1.2}

\begin{tabular}{c c c | c | c | c || c | c | c || c | c | c |}

\cline{4-12}
&&&
\multicolumn{3}{ c|| }{\textbf{Adult Dataset}}&
\multicolumn{3}{ c|| }{\textbf{Default Dataset}}&
\multicolumn{3}{ c| }{\textbf{Admissions Dataset}}
\\ \cline{4-12}
%&&&
%\multicolumn{3}{ c|| }{\textbf{Both Considerations}}&
%\multicolumn{3}{ c|| }{\textbf{Both Considerations}}&
%\multicolumn{3}{ c| }{\textbf{Both Considerations}}
%\\ \cline{4-12}
&&&
 $\mathbf{Acc.}$ &  $\mathbf{D_{FPR}}$ &  $\mathbf{D_{FNR}}$ &  $\mathbf{Acc.}$ &  $\mathbf{D_{FPR}}$ &  $\mathbf{D_{FNR}}$ &  $\mathbf{Acc.}$ &  $\mathbf{D_{FPR}}$ &  $\mathbf{D_{FNR}}$
\\  \cline{4-12}
\vspace*{-0.5ex}
\\ \cline{1-2} \cline{4-12}
\multicolumn{1}{ |c  }{} &
\multicolumn{1}{ c|  }{  \textbf{Our Method (AVD Penalizers)}}  &&
$\mathbf{0.776}$    &  $\mathbf{0.00}$  &  $\mathbf{0.04}$ &
$\mathbf{0.807}$    &  $\mathbf{0.00}$   &  $\mathbf{0.01}$ &
$\mathbf{0.950}$    &  $\mathbf{0.01}$  &  $\mathbf{0.00}$
\\ \cline{1-2} \cline{4-12}
\multicolumn{1}{ |c  }{} &
\multicolumn{1}{ c|  }{  \textbf{Our Method (SD Penalizers)}}  &&
$\mathbf{0.783}$    &  $\mathbf{0.00}$  &  $\mathbf{0.09}$ &
$\mathbf{0.806}$    &  $\mathbf{0.01}$   &  $\mathbf{0.02}$ &
$\mathbf{0.950}$    &  $\mathbf{0.00}$  &  $\mathbf{0.00}$
\\ \cline{1-2} \cline{4-12}
\multicolumn{1}{ |c  }{} &
\multicolumn{1}{ c|  }{  \textbf{Vanilla Regularized Logistic Regression}}  &&
$\mathbf{0.800}$    &   $\mathbf{0.08}$    &   $\mathbf{0.39}$  &
$\mathbf{0.807}$    &   $\mathbf{0.01}$    &   $\mathbf{0.05}$  &
$\mathbf{0.951}$    &   $\mathbf{0.16}$    &   $\mathbf{0.02}$
\\ \cline{1-2} \cline{4-12}
\end{tabular}
}
\vspace{3mm}
\caption{Performance on the Adult, Loan Default, and Admissions datasets, penalizing for both FPR and FNR difference. Accuracy, FPR difference and FNR difference are evaluated on the test set, averaging over five runs and using a 30-70 training/test split.} \label{table:comparison_results_rest}
\end{table*}


In Table~\ref{table:comparison_results}, we compare the performance of our approach with that of three other techniques from the literature. Each method was trained based on logistic regression.  As a basis for comparison, we also present the performance of vanilla logistic regression, absent fairness considerations, with the regularization parameter selected via cross-validation.\footnote{Zafar et al.~(\citeyear{disparatemistreatment}) do not incorporate regularization in any of the approaches they report.}
%Results are reported as the averages of 5 different runs \katrina{Is that still correct?}, each time splitting data evenly and randomly into training and test sets. 
Results for Zafar et al., Zafar et al. baseline, and Hardt et al. appear here as reported in Zafar et al.~(\citeyear{disparatemistreatment}).\footnote{Our method selects the classifier based on the training set only and reports its performance over the test set. Results for the three other approaches, reported by Zafar et al.~(\citeyear{disparatemistreatment}), are based on tuning parameters after seeing the trade-off curve over the test set, and reporting according to the best selection of these parameters.}
%\katrina{Perhaps here is the right place for a footnote about the discrepancy with the Zafar baseline}

We find that the vanilla logistic regressor (absent fairness considerations) results in significant unfairness, as $\mathbf{D_{FPR}}=0.20$, and $\mathbf{D_{FNR}}=0.30$. The overall accuracy of this classifier measured on the test set was $0.672$.\footnote{Zafar et al.~(\citeyear{disparatemistreatment}) report a slightly different baseline of: Accuracy = 0.668, $\mathbf{D_{FPR}}=0.18$, $\mathbf{D_{FNR}}=0.30$.} Our SD penalization approach empirically achieves approximately the same accuracy as the Zafar et al.~(\citeyear{disparatemistreatment}) approach, with significantly better fairness. It is difficult to compare fairness-accuracy tradeoffs with the Hardt et al.~(\citeyear{hardt}) approach, since their accuracy is significantly lower than ours. A more direct comparison is possible by noting that our learned classifier can be post-processed to improve its fairness at a direct cost to accuracy. Hence, we can achieve accuracy of $0.659$ with $\mathbf{D_{FPR}} = \mathbf{D_{FNR}} = 0.01$, which compares very favorably with the Hardt et al. accuracy rate of 0.645 given the same FPR and FNR rates.\footnote{For completeness, we note that using a 50-50 training-test split (again not using the test set for parameter selection), our method (SD, both considerations) produces a classifier that provides: Accuracy = 0.659, $\mathbf{D_{FPR}} = 0.01, \mathbf{D_{FNR}} = 0.05$. This classifier can be post-processed to achieve rates of: Accuracy = 0.655, $\mathbf{D_{FPR}} = \mathbf{D_{FNR}} = 0.01$.}

Figure \ref{fig:compas} illustrates the accuracy/fairness trade-offs achievable using our scheme. Increasing the weight $c$ on the proxy fairness penalizers results in reducing their magnitude. The figure also illustrates how our relaxed penalizers succeed in tracking the real FPR and FNR differences. 
%
%
%\katrina{Must rewrite the following paragraph}
%We observe that our method succeeds in eliminating unfairness almost completely on the COMPAS dataset, while retaining most of the accuracy, when compared to the vanilla logistic regression. We achieve very low difference rates when penalizing for achieving each of the FPR and FNR criteria individually, and also for both. We achieve preferable results comparing to Zafar et al. and Zafar et al. baseline in all 3 scenarios, and also comparing to Hardt et al. in the settings of false positive/false negative considerations only. In the setting of both considerations - The Hardt et al. method removes a larger portion of the unfairness, however it results in major accuracy loss as it achieves accuracy rate of 0.645 in comparison to our method which results in accuracy of 0.665, retaining most of the original accuracy rate while removing most of the unfairness.




%The Hardt et al.~\cite{hardt} approach as reported removes a smaller portion of the bias in the different scenarios, however for FP/FN constraints alone, it provides higher accuracy rates. The Zafar et al.~(\citeyear{disparatemistreatment}) approach as reported retains significant bias (in most cases), but in some cases  achieves slightly superior accuracy rates to the methods above. 

%These performance comparisons are incomplete in the sense that each of the compared techniques has the potential to trade off between accuracy and fairness, using some degree of parameter tuning; what we report here is only one point on the achievable trade-off frontier for each algorithm. The ``correct'' trade-off, and, in particular, the best manner in which to weigh unfairness in the FPR against unfairness in the FNR, are matters of opinion. We have chosen to report our method's performance under parameters designed to very aggressively mitigate unfairness, at some cost to the accuracy.

%It would certainly be desirable to evaluate these and other approaches to fair learning on other datasets and on different tasks, particularly on larger datasets, which might afford both greater accuracy and better bias-reduction. The present empirical evaluations, however, suggest that our regularization-based approach provides a new tool worthy of consideration---we succeed in almost entirely eliminating bias on the hold-out set, at a modest price in terms of accuracy.

%Due to the fact that our true objective includes the original non-convex penalization terms, our approach does not carry any formal guarantees. However, the ease of implementation, generality, and empirical results are encouraging. Figure~\ref{fig:test1} illustrates the rate of convergence to a fair, accurate classifier on this dataset.
%In terms of computation costs, given that at each iteration we must calculate the gradient according to the FPR and FNR regularizers, we are required to predict the labels for the entire training set at each step. 
%However, this does not pose a computational burden, as it is already required by the (classic) gradient descent algorithm in our logistic regressor fitting scheme. Furthermore, when given a sufficiently large dataset (one or two orders of magnitude larger than the one currently available for the COMPAS scores data), this could be relaxed to sampling only a mini-batch of samples from the training data set at each iteration (much as is done in stochastic gradient descent).






\subsection{Additional Datasets}


Table~\ref{table:datasets_description} provides summary statistics on each of the datasets on which we tested our approach. We also briefly describe the datasets below. 


{\bf The Adult Dataset}\footnote{http://archive.ics.uci.edu/ml/datasets/Adult} is based on 1994 US Census data. The task we consider is to predict whether the income of each individual is over or under 50K dollars per year, based on features such as occupation, marital status, and education. The protected attribute selected in this task is gender. 

{\bf The Loan Default Dataset}\footnote{{\scriptsize https://archive.ics.uci.edu/ml/datasets/default+of+credit+card+clients}}
contains data regrading Taiwanese credit card users. The task we consider is to predict whether an individual will default on payments, based on features such as history of past payments, age, and the amount of given credit. The protected attribute is gender.

{\bf The Admissions Dataset}\footnote{http://www2.law.ucla.edu/sander/Systemic/Data.htm}
contains records of law school students who went on to take the bar exam. The task we consider is to predict whether a student will pass the exam based on features such as LSAT score, undergraduate GPA, and family income. The protected attribute is set to race.

Table~\ref{table:comparison_results_rest} describes the performance of our approach on these datasets, and Figures~\ref{fig:adult},~\ref{fig:default}, and~\ref{fig:lawschool} illustrate the fairness-accuracy trade-offs we achieve in each context. Overall, we see that unfairness is nearly eliminated while accuracy remains quite high. The dataset on which accuracy suffers most under our approach is the Adult dataset, which is also the dataset on which the vanilla regression is the most unfair.


\begin{figure*}[]
  \includegraphics[scale=0.6]{adult0-800.png}
  \caption{Adult Dataset. Fairness-Accuracy tradeoffs, as in Figure~\ref{fig:compas}.}
  \label{fig:adult}  
\end{figure*}



\begin{figure*}[]
  \includegraphics[scale=0.6]{default0-50.png}
  \caption{Loan Default Dataset. Fairness-Accuracy tradeoffs, as in Figure~\ref{fig:compas}.}
  \label{fig:default}
\end{figure*}



\begin{figure*}[]
  \includegraphics[scale=0.6]{admissions0-400.png}
  \caption{Admissions Dataset. Fairness-Accuracy tradeoffs, as in Figure~\ref{fig:compas}.}
  \label{fig:lawschool}
\end{figure*}


   % This file has Experimental section

%the following two sections might be merged into the same file
\subsection{Multitask Learning}

MTL has been succesfully used in different domains, including CV \cite{UberNet,MaskRCNN}. Some challenges appear when applying it \cite{Caruana}: \textit{learning speed} differences between tasks and  deciding \textit{what to share} according to the \textit{relatedness} between tasks in the multitask architecture \cite{Stitch, AdaptiveFeatureSharing}.

\subsection{Semantic Segmentation}

Semantic segmentation aims at partitioning parts of images belonging to the same semantic class, typically via pixel-wise classification. Fully convolutional networks (FCN) \cite{FCN} have improved both accuracy and speed for dense prediction problems by using only convolutional layers. Upsampling layers allow a segmentation output size equal to the input and skip connections add finer details. Other approaches add post-processing steps \cite{DeeplabCRF}, learnable \textit{deconvolution} layers \cite{ Deconv} or global context \cite{ParseNet}.

\subsection{Object Detection}

Object detection aims at finding in an image all instances of objects and classifying them in a number of classes. Faster R-CNN \cite{FasterRCNN} was the first to give close to real-time performance. YOLO \cite{YOLO} avoids the generation of region proposals for increased speed. SSD \cite{SSD} avoids fully-connected layers for speed and takes features at different levels for improved accuracy. 

%\cite{SpeedAccuracy} reviews the speed vs. accuracy trade-off for different object detectors.   % This file has Related Work section
%In this paper, 2D and 3D CNN models were used to generate pelvic sCTs from T1-weighted MR images. Our sCT generation methods were fully automated, requiring no deformable registration or manual segmentation of bone tissues. As shown in Figure~\ref{fig3}, the 2D and 3D CNN models generated high quality sCTs. MAE curves shown in Figure~\ref{fig4} indicated that both models could precisely estimate soft-tissue HU values but had difficulty in reproducing air and high-density bone tissues. 

The MAEs within the body contour across all patients were 40.5 $\pm$ 5.4 HU and 37.6 $\pm$ 5.1 HU for the 2D and 3D models, respectively. The time required for generating a pelvic sCT using our CNN models was about 5.5 s. Our MAE results are comparable to previous studies. Kim $et \ al.$\cite{RN41} presented a voxel-based weighted summation method that produced an MAE of 74.3 $\pm$ 3.9 HU. However, manual contouring of bone tissues required for this method can be tedious and time-consuming. An MAE of 40.5 $\pm$ 8.2 HU was achieved by Dowling $et \ al.$\cite{RN11} using an average MRI-CT atlas from 38 patients. Andreasen $et \ al.$\cite{RN42} reported an MAE of 54 $\pm$ 8 HU using an atlas-based method with pattern recognition, and its prediction time was about 20.8 min. Another random forest model proposed by Andreasen $et \ al.$\cite{RN43} generated sCTs with an MAE of 58 $pm$ 9 HU. A hybrid method suggested by Siversson $et \ al.$ \cite{RN45} obtained an MAE of 36.5 $\pm$ 4.1 HU when ignoring errors introduced by gas cavities. This hybrid method was implemented in the cloud-based commercial software MriPlanner (Spectronic Medical AB, Helsingborg, Sweden), which required 50 to 80 min to generate a sCT.\cite{RN45} The patch-based 3D context-aware generative adversarial network presented by Nie $et \ al.$\cite{RN26} achieved an MAE of 39.0 $\pm$ 4.6 HU. 

Our CNN models reproduced low-density bone as shown in Figure ~\ref{fig4}. The bone-region DSCs were 0.81 $\pm$ 0.04 and 0.82 $\pm$ 0.04 from the 2D and 3D models, respectively. These results are comparable to reported DSC results of 0.79 $\pm$ 0.12\cite{RN10} and 0.91$\pm$0.03{\cite{RN11}}, where the authors compared bone contours manually drawn on the sCT and CT.

It was feasible to train the proposed 3D model with 16 image volumes from scratch. Results of the Wilcoxon signed-rank tests shown in Table~\ref{tab1} demonstrated a statistically significant improvement in overall MAE, bone DSC, and bone precision of the 3D model compared to the 2D model. However, as shown in Figure~\ref{fig4}, the 2D model seemed to perform better in estimating the high-density bone HU values. It should be noted that smaller overall MAEs do not guarantee improved sCT dose calculation and patient positioning performance. While the models performed well, we will continue to acquire more patient data to potentially improve model accuracy and further test model differences.

As this was a retrospective study, the MR image voxel sizes were not matched, resulting in different voxel intensities between images. This may have affected the sCT generation accuracy although we applied intensity normalization. A potential study could examine how voxel size variations affects sCT estimation. 

The proposed 3D model can be implemented on a 12 GB GPU to process volumetric images with dimensions of 256 $\times$ 256 $\times$ 30. More GPU memory would be required to process higher resolution 3D images. Considering the limited access to multi-GPU systems, a 3D architecture with fewer convolutional layers could be considered to deal with higher resolutions. However, the performance could be affected by the reduced parameters and smaller receptive fields of the less complex model. Another approach would be to extract 30-slice sub-volumes from CT and MR images for training the 3D model. The sCT could then be generated by averaging 30-slice sCT sub-volumes produced by the model. 

A number of techniques could be investigated for improving model performance.  Nie $et \ al.$\cite{RN26} showed that introducing an additional adversarial discriminator improved overall sCT quality. The same approach could be adapted in our proposed 2D and 3D CNN models.  Non-rigid deformation\cite{RN44} could also be applied to both CT and MR images in the process of the on-the-fly data augmentation to produce more training pairs. Multiple MR images acquired with different sequences could be fed into models to provide more information for distinguishing different tissues. Multi-GPU systems with more memory would enable the exploration of larger batch sizes for training CNN models, which could reduce variances in gradient estimation and accelerate the training. 

    % This file has Discussion section

\begin{comment}
\begin{figure}
\includegraphics[width=\linewidth]{figs/beyond_tss_lesion.pdf}
\caption[]{End-to-End runtime lesion study of the entire MNIST dataset and the FMA featurized music dataset. Each of DROP's contributions provides a runtime improvement.}
\label{fig:beyond_lesion}
\end{figure}
\end{comment}



\section{Conclusion}
\label{sec:conclusion}

Advanced data analytics techniques must scale to rising data volumes. 
DR techniques offer a powerful toolkit when processing these datasets, with PCA frequently outperforming popular techniques in exchange for high computational cost. 
In response, we propose DROP, a new dimensionality reduction optimizer. 
DROP combines progressive sampling, progress estimation, and online aggregation to identify high quality low dimensional bases via PCA without processing the entire dataset by balancing the runtime of downstream tasks and achieved dimensionality. 
Thus, DROP provides a first step in bridging the gap between quality and efficiency in end-to-end DR for downstream \red{analytics}. 

%We revisit canonical operators for time series dimensionality reduction and the measurement study of~\cite{keogh-study}, and show that PCA is more effective than popular alternatives in the data mining literature often by a margin of over $2\times$ on average on gold-standard time series benchmark data sets with respect to output data dimension. More surprisingly, we empirically demonstrate that a small number of samples are sufficient to accurately characterize directions of maximum variance and obtain a high-quality low-dimensional transformation.


    % This file has Conclusion and Future Work sections


\bibliographystyle{ACM-Reference-Format}
\bibliography{TACO2017}

\appendix
\section{Schedule Independent Memory Allocation}
\label{sec:sima}

We also address a memory-based limitation of polyhedral compilation
tools.  It is well known that in any parallelization (of any program), it is
essential to respect (only) the \emph{true} or flow dependences.  Other
(memory-based) dependences can be ignored if one can re-allocate memory.  In
practice, this is limited by the fact that the associated memory expansion may
be prohibitively expensive, and there has been work on mitigating this
expansion~\cite{vasilache-impact12, lefebvre-feautrier-pc98, sanjay-europar96,
  sanjay-toplas00}.  We propose a novel yet simple \emph{schedule-independent}
memory allocation strategy.  Our work also generalizes polyhedral compilation
by enabling polyhedral tools to use alternate, \emph{hybrid} schedules
consisting of affine loops for certain parts of the iteration space and
cache-oblivious divide-and-conquer schedules for others.


\subsection{Background}

In this section, we introduce the necessary background of our work. We first
give a brief description of the polyhedral representation of programs, and the
general flow of a polyhedral compiler.  Then, we discuss the legality of
tiling, which is related to the input of our code generator.

\begin{figure*}[tb]
  \centering %\vspace*{6cm}
  \includegraphics[scale=0.6]{PolyCompiler}
  \caption{\small{Polyhedral Compilation: the Polyhedral Reduced Dependence
      (hyper) Graph (PRDG) serves as the intermediate representation.
      Piecewise Quasi-Affine Functions (PQAFs) describe transformations.}}
  \label{fig:compiler}
\end{figure*}

\subsubsection{Polyhedral Compilation and Representation}

Figure~\ref{fig:compiler} shows the flow of polyhedral compilation.  First,
dependence analysis of an input program (or a ``polyhedral section'' thereof)
produces an intermediate representation (IR) in the form of~\cite{DRV-sched00}
a \emph{Polyhedral Reduced Dependence (hyper) Graph} (PRDG).  Various analyses
are performed on the PRDG to choose a number of mappings in the form of
\emph{Piecewise Quasi-Affine Functions} (PQAFs) that specify the schedule as a
set of \emph{multi-dimensional} vectors.  The PQAFs come with annotations to
indicate whether each dimension is sequential or parallel, and also whether it
is part of a \emph{tilable band}, i.e., whether tiling this band of dimensions
is legal.  The transformations may be applied to the PRDG iteratively, and
(eventually) the PRDG and QLAF are provided to a code-generator that produces
code for various targets.

% specify which dimensions are sequential, which dimensions are (and
% implicitly, also the algorithm~\cite{uday-pldi08} to get tiling hyperplanes
% and tilable dimensions for each statement which is called \emph{tilable
% band}.  Then we transform the program using tiling hyperplanes.  This
% results in a program where hyper-rectangular tiles are legal and the
% wavefront parallel execution order is a legal schedule for executing tiles.
% We also provide schedule independent memory allocations for all the
% variables.  Finally we generate code which traverse the iteration space of
% the program in divide and conquer order.  by post processing the AST of
% tiled code generated by DTiler The following section presents the schedule
% independent memory allocation for affine programs.

One of the strengths of the polyhedral model is that a parametric program may
be concisely represented with a PRDG with finite number of nodes (statements)
and edges (dependences).  The potentially unbounded sets of instances of a
statement are represented in abstract forms of integer sets, called
\emph{domains}, and dependences between them as affine functions (or
relations, which are viewed as a set-valued function) over these statement
domains.  Indeed, every edge, $e$ from node $v$ to $w$, in the PRDG is
annotated with two objects: (i) a domain, $D_e$ specifying the (subset of) the
domain, $D_v$ of its source node, where the dependence occurs, and (ii) the
affine function, $f$, such that for any point $z\in D_e$, the (set of)
point(s) in $D_w$ on which it depends is given by $f(z)$.  $D_e$ is called the
context of the edge, and $f$ is its dependence function.  We also use the
notation $f(D_e)$ to denote the set valued image of $D_e$ by $f$.

An affine function $\mathbb{Z}^n \rightarrow \mathbb{Z}^m$ may be expressed as
$f(x) = A\vec{x} + \vec{b}$, where $\vec{x}$, function domain, is an integer
vector of size $n$; $A$, linear part, is an $n\times m$ matrix; and $\vec{b}$,
constant part, is an integer vector of size $m$.  A dependence is said to be
uniform if the dependence function is only a constant offset, i.e., when the
linear part $A$ is the identity.

%  is used to get the  tilable dimensions....  The parallelism that can be
%  explored using tiles is assumed to be wavefront....??  However, we modify
%  the execution order of tiles and schedule them in a divide-and-conquer
%  fashion.... Updating the memory allocation schemes becomes important when
%  we change the order of execution of tiles.... The following Section talks
%  about Memory Allocation...

\subsubsection{Legality of Tiling}

Tiling is a well-known loop transformation for partitioning computations into
smaller, atomic (all inputs to a tile can be computed before its execution),
units called tiles~\cite{irigoin-popl88, Wolf91tiling}.  The natural legality
condition is that the dependences across tiles do not create a cycle.  In
compilers, this condition is typically expressed as fully permutability (i.e.,
dependences are non-negative direction vectors), which is a sufficient
condition.  Our transformation for cache oblivious tiling takes as inputs a
loop nest that is fully permutable.  For polyhedral programs, scheduling
techniques to expose such loop nests are available~\cite{uday-pldi08}.


\subsection{Memory Allocation}

\begin{figure}[tb]
  \centering % \vspace*{4cm}
{\small\begin{lstlisting}
for (i = 0; i < N; i++){
  S0:  X[0,i] = A[i]; } // Initialize
for (t = 1; t <= 2*N; t++){ //Note: ub is even
  for (i = 1; i < N-1; i++){
    S1: X[t%2][i] = f(X[(t-1)%2][i-1],
             X[(t-1)%2][i], X[(t-1)%2][i+1],
             X[(t-1)%2][0]);
    }
  S2: X[t%2][0] = g(X[(t-1)%2][0]); 
  S3: X[t%2][N-1] = g(X[(t-1)%2][N-1]);
}
for (i = 0; i < N; i++){
  S4: Aout[i] = X[0,i]; } // Output copy
\end{lstlisting}
}
\caption{\small{Neither Pochoir nor Autogen can handle the computation
    performed by this simple loop.  Moreover, it has a memory based dependence
    that prevents polyhedral compilers like Pluto from tiling both dimensions.
    However, the true dependences of the program admit a tilable schedule, but
    at the potentical cost of $O(N^2)$ memory.  Our scheme reduces this to
    $O(N)$}}
\label{fig:motiv}
\end{figure}


In this section, we first describe how memory based dependences prevent
tiling, using our motivating example (Fig.~\ref{fig:motiv}), and show that
simply ignoring these (false) dependences would lead to memory explosion.
After formulating our problem, we next propose a simple, schedule independent
memory allocation scheme that resolves it.

Consider the statement $\mathrm{S1}$, and note that its domain, $D_1$ is the
polyhedral set, $\{t,i~|~ 1\leq t\leq 2N \wedge 1\leq i\leq N-1 \}$.
$\mathrm{S1}$ has four true dependences (for points sufficiently far from the
boundaries), three of which are $\mathrm{S1}[t-1, i-1]$, $\mathrm{S1}[t-1, i]$
and $\mathrm{S1}[t-1, i+1]$, the typical, 1D-Jacobi stencil dependences, and
the fourth one is $\mathrm{S2}[t-1, 0]$, which is a truly affine dependence on
the most recent writer to the memory location $\mathtt{X[(t-1)\%2,0]}$ when
the statement $\langle \mathrm{S1}, [t, i]\rangle$ is being executed.  All
these dependences are captured as edges with affine \emph{functions} in the
PRDG.  In addition, there is a memory based dependence, that we must also
respect.  Consider statement $\mathrm{S2}$, whose domain, $D_2 = \{t~|~ 1\leq
t\leq 2N\}$ is just one dimensional.  The $t$-th instance of S2 \emph{(over)
  writes} $\mathtt{X[t\%2, 0]}$, therefore all computations that read the
previous value must be executed before it.  In this sense, $\mathrm{S2}[t]$
``depends on'' the set $\mathrm{S1}[t,i]$, for all $1\leq i\leq N-1$.  This
dependence (which is a \emph{relation} rather than a function) is captured by
another a special edge in the PRDG.

The only schedule that respects all these dependences is the family of lines
parallel to the $t$ axis (provided all iterations of S1 are done first).
Although this has maximal parallelism, it has very poor locality.  Note that
the Pluto scheduler does not seek maximal parallelism, but rather, to maximize
the \emph{number of linearly independent tiling hyperplanes}.  Unfortunately,
the $t=\mathrm{const}$ is the only legal tiling hyperplane for this set of
dependences, and the tilable band obtained by Pluto is only 1-dimensional.

What if we did not have the memory-based dependences, i.e., what if we ignored
the memory allocation of the original program, and stored each computed value
in a distinct memory location?  In this case, there would be no memory based
dependences, and we can indeed find another family of (actually, infinitely
many) legal tiling hyperplanes: say, the lines $i+t=\mathrm{const}$.  As a
result, if we use the mapping $(t,i) \mapsto (t,i+t)$ as our ``schedule,'' the
new loops in the transformed program would be fully permutable, and could be
legally tiled.

Thus, the problem we seek to solve is: \emph{how to avoid memory based
  dependences, but without the cost of memory expansion that it seems to
  imply}.

Memory allocation for polyhedral programs is a well studied problem, and there
are two main approaches.  One either does memory allocation after the schedule is
chosen~\cite{sanjay-europar96, degreef-memory97, lefebvre-feautrier-pc98,
  sanjay-toplas00, darte-lattice05, vasilache-impact12,
  bhaskaracharya-toplas16, bhaskaracharya-popl16} since it often leads to a
smaller memory footprint, or else uses a \emph{schedule independent} memory
allocation, based on the so called \emph{universal occupancy vectors} (UOV).
This problem is solved when the program has \emph{uniform dependences}, i.e.,
when each dependence can be described by a \emph{constant vector}, and for some
simple extensions of this~\cite{strout-etal-asplos98, sanjay-memory-2011}.

It is important to note that tiling actually modifies a schedule: the so
called, ``schedule dimensions'' become fully permutable loops, and indeed,
these loops \emph{are actually permuted} in the generated tiled code.  So,
when a \emph{tiling schedule} specified by a family of $d$ tiling hyperplanes
is finally implemented by the generated code, the actual time-stamps are not
really $d$-dimensional vectors, but rather $2d$-dimensional ones obtained as
some complicated function of these indices.  Furthermore, we will see when we
generate cache-oblivious tiled codes, these tilable loops will actually be
visited in the divide-and-conquer order of execution, as required by COT.  As
a result, finding a memory map that takes into account such a rather
complicated final schedule is a tricky problem.  We therefore seek and propose
schedule-independent memory allocations.

The intuition behind our solution is (deceptively) simple, and we first
illustrate it on our motivating example (Fig.~\ref{fig:motiv}).  Rather than
the so-called ``single assignment'' program for the entire iteration space of
the program (i.e., full memory expansion), could we find lower-dimensional
subsets, such that a single assignment memory for only these subsets is
sufficient?  A careful examination of the code reveals that the memory based
dependences arise due to statement S2, and its domain is only 1-dimensional.
So we store the results of this statement into an auxiliary array, \texttt{Y},
and modify the program so that the fourth dependence simply reads
\texttt{Y[t-1]}, rather than \texttt{X[(t-1)\%2,0]}.  For the variable,
\texttt{X}, we use the old $(t,i) \mapsto (t\%2,i)$ memory allocation that was
used in the original code.  This results in $4N$ memory, which is a polynomial
degree better than quadratic.  Of course, the challenge is how to discover
this automatically.

% PLUTO schedule give a schedule and a tiling band.  A point in a tiling band
% corresponds to a particular tile.  All points within a tile execute
% sequentially.  However, the tiles can be executed in parallel.  Consider the
% modified Jacobi-1D stencil (CHANGE THIS TO BE CONSISTENT WITH RUNNING
% EXAMPLE).  The Figure \textbf{ABC} shows the iteration space of Jacobi-1D
% example (CHANGE THIS TO BE CONSISTENT WITH RUNNING EXAMPLE).  The blue bars
% show the inputs and outputs of the program corresponding to statements
% $S_{0}$ and $S_{4}$ respectively.  The memory allocation scheme used
% initially is modulo memory allocation.  However, this modulo memory
% allocation is not affine and hence PLUTO is unable to find a schedule. We
% make this memory map to not to use modulo but use previous and current
% instead. Even then, PLUTO is unable to find a schedule such that all
% dimensions are tilable.  PLUTO will decide to tile this program with single
% assignment memory.  Identity memory allocation is known to be overkill and
% lead to inefficient codes.  Therefore, we need a schedule independent memory
% allocation scheme which guarantees that the given memory allocation is both
% legal as well as optimal for any given schedule.

% The problem of schedule independent memory allocation for Polyhedral
% programs is a partially solved problem. [Strout et al] presented an
% algorithm for schedule independent memory allocation for a class of programs
% that have only uniform dependences.  A set of programs with uniform
% dependences is a proper subset of the set of programs with affine
% dependences, the class of Polyhedral programs.  Our recursive
% divide-and-conquer code generator is applicable to all Polyhedral programs
% in general, including the ones that have truly affine non-uniform
% dependences. We therefore present a novel scheme for schedule independent
% memory allocation for all affine programs.

We now outline how this is done.  At a high level, our algorithm takes a PRDG
as input, applies some (piecewise) affine transformations to it, and outputs
the transformed PRDG together with a separate memory map for each node in the
transformed PRDG.  More specifically, it works as follows.

\begin{itemize}
\item \emph{Preprocessing.}  For each edge, $e$, in the PRDG, with context
  $D_e$, and function, $f$, we first identify whether $f$ is \emph{uniform in
    context} in the sense that, for all points, $z\in D_e$, the value of
  $z-f(z)$ is a constant vector, independent of $z$.

  For example, consider a dependence function, $(i,j) \mapsto (i-1,i-1)$
  which, maps any point $[i,j]$ in the plane to a point on the diagonal, and
  is clearly not uniform.  However, what if $D_e = \{i,j~|~i=j-1\}$?  With
  this contextual information, the dependence is actually uniform: $(i, j)
  \mapsto (i-1, j-2)$.

  All edges/dependences that are neither uniform to begin with, not uniform in
  context, are marked as \emph{truly affine}.
\item \emph{Affine Split.}  For every node, $v$, in the PRDG that has at least
  one truly affine edge $e$ incident on it, we create a new node, $v'$.  Its
  domain $D_{v'}$ is the union of $f(D_e)$ of all such incident edges.

  The edges in the PRDG are modified as follows.  All the truly affine edges
  that were incident on $v$ are now made incident on $v'$; and $v'$ has a
  single outgoing edge $e'$, annotated with $\langle D_{v'}, I \rangle$ (its
  dependence function is the identity map) and whose destination is $v$.

  It is easy to see that we have not changed the program semantics.  In
  effect, we have simply copied the value of every point in $D_v$ that was the
  target of any truly affine dependence over to a new variable $v'$, and
  ``diverted'' all the truly affine edges that used to be incident on $v$ over
  to $v'$.  Moreover, since the identity function is uniform by definition, all
  edges incident on $v$ are now either uniform, or uniform in context.
\item We now use existing UOV based methods~\cite{strout-etal-asplos98,
    sanjay-memory-2011} to choose a schedule-independent memory allocation for
  all the original nodes in the PRDG, and a \emph{single-assignment} memory
  allocation for all the newly introduced variables.
\end{itemize}

The key insight into why this leads to significant memory savings, is the fact
that in all polyhedral programs that we encountered, truly affine dependences
are almost always \emph{rank deficient}, i.e., are many-to-one mappings from
the consumer index points to the producers.  The only exceptions are either
pathological programs, or programs that do multi-dimensional data
reorganizations via bijections (e.g., matrix transpose, tensor permutations,
etc.) where here is no scope nor need to reduce the total memory footprint.
As a result, $f(D_e)$ is almost always a lower dimensional polyhedron, and
requires significantly less memory, even when stored supposedly inefficiently.


\subsection{Related Work}

Memory allocation for polyhedral programs is a well studied problem for almost
two decades.  DeGreef and Cathoor~\cite{degreef-memory97} tackled the problem
of sharing the memory across multiple arrays in the program.the so called
inet-array memory reuse problem, and proposed an ILP based solution.  Wilde
and Rajopadhye, in dealing with an intrinsically memory-inefficient functional
language Alpha~\cite{mauras1989thesis} (one can think of this as a program after
full expansion) first addressed the memory reuse for points of an iteration
space~\cite{sanjay-europar96}.  They gave necessary and sufficient conditions
for the legality of a memory allocation fucntion, which they allowed to be
``in any direction.''  but they did not provide any insight into how to choose
the mapping.  Lefebvre and Feautrier~\cite{lefebvre-feautrier-pc98} on the
other hand, considered only canonic projections, combined with a modulo
factor, but showed how to choose the mapping optimally.  Later, Quiller\'e and
Rajopadhye~\cite{sanjay-toplas00} revisited multiprojections, extended them to
quasi-affine functions, and proved a tight bound on the number of dimensions
of reuse.  They also showed that cananic projections with modulo factors was
sometimes a constant factor better, and sometimes a constant factor worse.
Darte at al.~\cite{darte-lattice05} took a fresh and elegant approach to the
problem, and formulated the conditions for legal memory allocations by
defining the \emph{conflict set}.  This led to techniques for choosing
provably optimal memory allocations, initially for non-parameterized iteration
spaces, and recently in the context of FPGA acelerators, for parametrically
tiled spaces~\cite{darte2014parametric, darte2016extended}.  Vasilache et
al.~\cite{vasilache-impact12} developed a tool to combine the scheduling and
limiting memory expansion using an ILP formulation, implemented in the
R-Stream compiler.  Recently, Bhaskaracharya et
al.~\cite{bhaskaracharya-toplas16} developed methods to optimally choose
quasi-affine memory allocations, and showed how they are beneficial for tiled
codes, especialy with live-out data.  Furthermore, they also
showed~\cite{bhaskaracharya-popl16} how to combine iner-and intra array reuse
in a unifying framework.

The other \emph{schedule independent} memory allocation was pioneered by
Strout et al.~\cite{strout-etal-asplos98}.  Here, the memory allocation is
chosen based only on the dependences, and is guaranteed to be legal,
regardless of the schedule.  This problem is solved when the program has
\emph{uniform dependences}, i.e., when each dependence can be descibed by a
\emph{constant vector}, and for some simple extensions of
this~\cite{strout-etal-asplos98, sanjay-memory-2011}.

Thies at al.~\cite{thies-pldi02} have also formulated the problem of
simultaneously choosing the schedule and memory allocation as a combined
optimization problem.
% \begin{algorithm}[h]
%   \mbox{} Input : PRDG
  
%     Output : Transformed PRDG and Memory Map
%     \begin{enumerate}
%     \item Recognize truly non-uniform dependence edges
  
%       For each node $X_{i}$ with one or more incoming non-uniform affine
%       edge(s):
%       \begin{itemize}
%       \item Create new node $X_{i}^{new}$ by applying ``Affine Split"
%         transformation
    	
%         Assertion: $X_{i} = X_{i}^{old} + X_{i}^{new}$

%       \end{itemize}
%     \item Find universal occupancy vector $(UOV)$ for all nodes with uniform
%       dependences
%     \item Construct memory mapping function for all nodes except
%       $X_{i}^{new}$ using $UOV$
% 	\item Use identity memory map for $X_{i}^{new}$ nodes
% 	\item \textbf{(SANJAY verify 4 , 5)} Apply change of basis to reduce
%    number of dimensions of the domain of new node
%  \end{enumerate}
%  \caption{SIMA: Schedule Independent Memory Allocation for Polyhedral
%  Programs}
%  \label{alg:sima}
% \end{algorithm}

% Local Variables: ***
% TeX-master: "PACT17.tex" ***
% fill-column: 78 ***
% End: ***


\end{document}
