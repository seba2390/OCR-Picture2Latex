\section{Background}
\label{sec:background}

In this section, we present necessary background on tiling, cache oblivious
divide-and-conquer execution, and define the terminology used in this paper.

\subsection{Tiling}
Tiling is a well-known loop transformation for partitioning computations into
smaller, atomic (all inputs to a tile can be computed before its execution),
units called tiles~\cite{irigoin-popl88, Wolf91tiling}. The partitioning into
tiles improves data locality by altering the execution order of the operations.
Tiling also exposes coarse-grained parallelism making it the core transformation
for polyhedral automatic parallelizers such as Pluto~\cite{uday-pldi08}.

 The natural legality condition of tiling is that the dependences across tiles do not
create a cycle.  In compilers, this condition is typically expressed as fully
permutability (i.e., dependences are non-negative direction vectors), which is
a sufficient condition. In the rest of this paper, we assume that the programs
have been transformed to expose loop nests that satisfy this condition.  For
polyhedral programs, scheduling techniques to expose such loop nests are
available~\cite{uday-pldi08}.

\subsection{Cache Oblivious Tiling}
Cache oblivious algorithms~\cite{prokop-thesis99, frigo-etal-focs99} are based
on recursive formulation into smaller subproblems (divide-and-conquer). The
main argument is that as the problem sizes are recursively made smaller, a
subproblem is going to fit on some level of the memory hierarchy that may be
caches, main memory, etc. This class of algorithms is expected to take
advantage of all memory hierarchies through this strategy.

Cache Oblivious Tiling is a specialization of such algorithms based on tiling.
Tiles after a level of tiling can be tiled again with smaller tile
sizes to realize the divide-and-conquer execution pattern. COT may be viewed as
hierarchical tiling, except that the number of tiling levels are determined at
run-time through divide-and-conquer~\cite{carter1995hierarchical}. 

The key effect of multi-level tiling is to change the execution order of the
tiles. As illustrated in Figure~\ref{fig:tile-order}, the grouping of smaller
tiles; forming larger tiles increasing intra-tile reuse; is what accounts for
better cache utilization.


\begin{figure}
\centering
  \begin{subfigure}{0.2\textwidth}
    \includegraphics[width=\columnwidth]{wavefront-tile-order}
    \caption{Single-Level Tiling \label{fig:wavefront-tile-order}}
  \end{subfigure}
  \begin{subfigure}{0.2\textwidth}
    \includegraphics[width=\columnwidth]{cot-tile-order}
    \caption{Two-Level Tiling \label{fig:cot-tile-order}}
  \end{subfigure}
  \caption{\label{fig:tile-order}Execution order of the smallest tiles under single- and two-level
tiling. Two-level tiling may be viewed as Cache Oblivious Tiling where the
recursion reached the base case after one recursive step. With hierarchical tiling,
neighboring tiles form a larger tile that increase intra-tile reuse.
\vspace{-0.2cm}
}
\end{figure}


\subsection{Terminology}\label{sec:terminology}
We introduce a few terms, in addition to COT in the above.
%, that are used throughout this paper.

\paragraph{Single-Level Tiling} SLT is when loop tiling is applied once to
improve data locality with respect to a level of cache. The reason we restrict
to SLT is because SLT and COT expose the same number of tuning parameters. It
is known that even cache oblivious algorithms require the performance of the
leaf subproblems to be tuned to have good performance. The tuning effort
required is similar to SLT with the same number of tuning parameters, which
would not be the case if you apply multi-level tiling that multiplies the
number of tuning parameters.  

\paragraph{Tile Size} We use tile size interchangeably with the base case
threshold (and leaf tile size) in COT. Whenever tile sizes are discussed for
COT, it refers to the size of the leaf tile sizes, which is its tuning parameter. 

\paragraph{Off-Chip Accesses} OCA are all load accesses that read from the main
memory.  This includes the accesses due to Last-Level Cache misses due to load
operations, as well as those arising from prefetching instructions.

%\subsection{Benefits of Cache Oblivious Tiling}
%For compute-bound programs, the performance (speed) of SLT and COT is expected
%be similar.  As demonstrated by Zou and Rajopadhye~\cite{zou2015rajopadhye}
%for compute-bound stencil computations, reducing the amount of off-chip memory
%accesses (reduced Last Level Cache misses) does not always translate to
%performance. The latency penalty for LLC misses are mostly hidden by the
%computation, and thus further reducing the penalty have little impact on
%speed.
%
%%\FIXME{I am contradicting the
%%TurboTiling paper here - I think what happens is that for TurboTiling the
%%performance gains are not from reduced LLC misses, but more optimized execution
%%of tiles due to larger tiles. We should check this.}
%
%Therefore, the execution times of recursively tiled programs are not expected
%to be any better than those with just a single level of tiling if the tile sizes are
%well-tuned. In fact, we expect that the recursion to introduce overhead of
%various forms, such as functional call overhead, increased register pressure,
%increased difficulty for the compiler as the recursive functions cannot be
%inlined.
%
%However, the increased intra-tile data reuse improves the data locality with
%respect to lower levels of the memory hierarchy. In effect, what can be
%expected for COT with compute-bound programs is \emph{reduced off-chip memory
%accesses with similar performance as SLT, without the need to explicitly tune
%for LLC}. This what we validate in our experiments presented in
%Section~\ref{sec:experiments}.
%
%The two scenarios where COT is useful are memory-bound computations and energy
%efficient computing. When the program is memory-bound to the extent latency
%hiding is not possible, then improved data locality in all levels are expected
%to translate to speed improvements. Another important benefit of reduced LLC
%misses\footnote{In fact, COT reduces L2 cache misses as well, but we focus on
%LLC misses that are associated with the more costly off-chip memory accesses.}
%is energy. Energy consumption due to off-chip memory accesses is a
%non-negligible component of the total energy consumption of a computing
%platform, and significant reduction on off-chip memory accesses can be
%beneficial for energy~\cite{zou2015rajopadhye,mikami2011evaluation}.
%

%\section{Our Approach}
We formulate the problem as an anisotropic diffusion process and the diffusion tensor is learned through a deep CNN directly from the given image, which guides the refinement of the output.

\begin{figure}[t]
\includegraphics[width=1.0\textwidth]{fig/CSPN_SPN2.pdf}
\caption{Comparison between the propagation process in SPN~\cite{liu2017learning} and CPSN in this work.}
\label{fig:compare}
\end{figure}

\subsection{Convolutional Spatial Propagation Network}
% demonstrate the thereom is hold when turns to be convolution.
Given a depth map $D_o \in \spa{R}^{m\times n}$ that is output from a network, and image $\ve{X} \in \spa{R}^{m\times n}$, our task is to update the depth map to a new depth map $D_n$ within $N$ iteration steps, which first reveals more details of the image, and second improves the per-pixel depth estimation results. 

\figref{fig:compare}(b) illustrates our updating operation. Formally, without loss of generality, we can embed the $D_o$ to some hidden space $\ve{H} \in \spa{R}^{m \times n \times c}$. The convolutional transformation functional with a kernel size of $k$ for each time step $t$ could be written as,
\begin{align}
    \ve{H}_{i, j, t + 1} &= \sum\nolimits_{a,b = -(k-1)/2}^{(k-1)/2} \kappa_{i,j}(a, b) \odot \ve{H}_{i-a, j-b, t} \nonumber \\
\mbox{where,~~~~}
    \kappa_{i,j}(a, b) &= \frac{\hat{\kappa}_{i,j}(a, b)}{\sum_{a,b, a, b \neq 0} |\hat{\kappa}_{i,j}(a, b)|}, \nonumber\\
    \kappa_{i,j}(0, 0) &= 1 - \sum\nolimits_{a,b, a, b \neq 0}\kappa_{i,j}(a, b)
\label{eqn:cspn}
\end{align}
where the transformation kernel $\hat{\kappa}_{i,j} \in \spa{R}^{k\times k \times c}$ is the output from an affinity network, which is spatially dependent on the input image. The kernel size $k$ is usually set as an odd number so that the computational context surrounding pixel $(i, j)$ is symmetric.
$\odot$ is element-wise product. Following~\cite{liu2017learning}, we normalize kernel weights between range of $(-1, 1)$ so that the model can be stabilized and converged by satisfying the condition $\sum_{a,b, a,b \neq 0} |\kappa_{i,j}(a, b)| \leq 1$. Finally, we perform this iteration $N$ steps to reach a stationary distribution.

% theorem, it follows diffusion with PDE 
%\addlinespace
\noindent\textbf{Correspondence to diffusion process with a partial differential equation (PDE).} \\
Similar with~\cite{liu2017learning}, here we show that our CSPN holds all the desired properties of SPN.
Formally, we can rewrite the propagation in \equref{eqn:cspn} as a process of diffusion evolution by first doing column-first vectorization of feature map $\ve{H}$ to $\ve{H}_v \in \spa{R}^{\by{mn}{c}}$.
\begin{align}
     \ve{H}_v^{t+1} = 
     \begin{bmatrix}
    1-\lambda_{0, 0}  & \kappa_{0,0}(1,0) & \cdots & 0 \\
    \kappa_{1,0}(-1,0)   & 1-\lambda_{1, 0} & \cdots & 0 \\
    \vdots & \vdots & \ddots & \vdots \\
    \vdots & \cdots & \cdots & 1-\lambda_{m,n} \\
\end{bmatrix} = \ve{G}\ve{H}_v^{t}
\label{eqn:vector}
\end{align}
where $\lambda_{i, j} = \sum_{a,b}\kappa_{i,j}(a,b)$ and $\ve{G}$ is a $\by{mn}{mn}$ transformation matrix. The diffusion process expressed with a partial differential equation (PDE) is derived as follows, 
\begin{align}
     \ve{H}_v^{t+1} &= \ve{G}\ve{H}_v^{t} = (\ve{I} - \ve{D} + \ve{A})\ve{H}_v^{t} \nonumber\\
     \ve{H}_v^{t+1} - \ve{H}_v^{t} &= - (\ve{D} - \ve{A}) \ve{H}_v^{t} \nonumber\\
     \partial_t \ve{H}_v^{t+1} &= -\ve{L}\ve{H}_v^{t}
\label{eqn:proof}
\end{align}
where $\ve{L}$ is the Laplacian matrix, $\ve{D}$ is the diagonal matrix containing all the $\lambda_{i, j}$, and $\ve{A}$ is the affinity matrix which is the off diagonal part of $\ve{G}$.

In our formulation, different from~\cite{liu2017learning} which scans the whole image in four directions~(\figref{fig:compare}(a)) sequentially, CSPN propagates a local area towards all directions at each step~(\figref{fig:compare}(b)) simultaneously, \ie with~\by{k}{k} local context, while larger context is observed when recurrent processing is performed, and the context acquiring rate is in an order of $O(kN)$.

In practical, we choose to use convolutional operation due to that it can be efficiently implemented through image vectorization, yielding real-time performance in depth refinement tasks.

Principally, CSPN could also be derived from loopy belief propagation with sum-product algorithm~\cite{kschischang2001factor}. However, since our approach adopts linear propagation, which is efficient while just a special case of pairwise potential with L2 reconstruction loss in graphical models. Therefore, to make it more accurate, we call our strategy convolutional spatial propagation in the field of diffusion process.

\begin{figure}[t]
\centering
\includegraphics[width=0.9\textwidth]{fig/hist.pdf}
\caption {(a) Histogram of RMSE with depth maps from~\cite{Ma2017SparseToDense} at given sparse depth points.  (b) Comparison of gradient error between depth maps with sparse depth replacement (blue bars) and with ours CSPN (green bars), where ours is much smaller. Check~\figref{fig:gradient} for an example. Vertical axis shows the count of pixels.}
\label{fig:hist}
\end{figure}

\subsection{Spatial Propagation with Sparse Depth Samples}
In this application, we have an additional sparse depth map $D_s$ (\figref{fig:gradient}(b)) to help estimate a depth depth map from a RGB image. Specifically, a sparse set of pixels are set with real depth values from some depth sensors, which can be used to guide our propagation process. 

Similarly, we also embed the sparse depth map $D_s = \{d_{i,j}^s\}$ to a hidden representation $\ve{H}^s$,  and we can write the updating equation of $\ve{H}$ by simply adding a replacement step after performing \equref{eqn:cspn}, 
\begin{align}
    \ve{H}_{i, j, t+1} = (1 - m_{i, j}) \ve{H}_{i, j, t+1}  +  m_{i, j} \ve{H}_{i, j}^s 
\label{eqn:cspn_sp}
\end{align}
where $m_{i, j} = \spa{I}(d_{i, j}^s > 0)$ is an indicator for the availability of sparse depth at $(i, j)$. 

In this way, we guarantee that our refined depths have the exact same value at those valid pixels in sparse depth map. Additionally, we propagate the information from those sparse depth to its surrounding pixels such that the smoothness between the sparse depths and their neighbors are maintained. 
Thirdly, thanks to the diffusion process, the final depth map is well aligned with image structures. 
This fully satisfies the desired three properties for this task which is discussed in our introduction (\ref{sec:intro}). 

% it performs a non-symmetric propagation where the information can only be diffused from the given sparse depth to others, while not the other way around.

% still follows PDE
In addition, this process is still following the diffusion process with PDE, where the transformation matrix can be built by simply replacing the rows satisfying $m_{i, j} = 1$ in $\ve{G}$ (\equref{eqn:vector}), which are corresponding to sparse depth samples, by $\ve{e}_{i + j*m}^T$. Here $\ve{e}_{i + j*m}$ is an unit vector with the value at $i + j*m$ as 1.
Therefore, the summation of each row is still $1$, and obviously the stabilization still holds in this case.

\begin{figure}[t]
\centering
\includegraphics[width=0.95\textwidth]{fig/fig2.pdf}
\caption{Comparison of depth map~\cite{Ma2017SparseToDense} with sparse depth replacement and with our CSPN \wrt smoothness of depth gradient at sparse depth points. (a) Input image. (b) Sparse depth points. (c) Depth map with sparse depth replacement. (d) Depth map with our CSPN with sparse depth points. We highlight the differences in the red box.}
\label{fig:gradient}
\end{figure}

Our strategy has several advantages over the previous state-of-the-art sparse-to-dense methods~\cite{Ma2017SparseToDense,LiaoHWKYL16}.
In \figref{fig:hist}(a), we plot a histogram of depth displacement from ground truth at given sparse depth pixels from the output of Ma \etal~\cite{Ma2017SparseToDense}. It shows the accuracy of sparse depth points cannot preserved, and some pixels could have very large displacement (0.2m), indicating that directly training a CNN for depth prediction does not preserve the value of real sparse depths provided. To acquire such property, 
one may simply replace the depths from the outputs with provided sparse depths at those pixels, however, it yields non-smooth depth gradient \wrt surrounding pixels. 
In~\figref{fig:gradient}(c), we plot such an example, at right of the figure, we compute Sobel gradient~\cite{sobel2014history} of the depth map along x direction, where we can clearly see that the gradients surrounding pixels with replaced depth values are non-smooth.
We statistically verify this in \figref{fig:hist}(b) using 500 sparse samples, the blue bars are the histogram of gradient error  at sparse pixels by comparing the gradient of the depth map with sparse depth replacement and of ground truth depth map. We can see the difference is significant, 2/3 of the sparse pixels has large gradient error.
Our method, on the other hand, as shown with the green bars in \figref{fig:hist}(b), the average gradient error is much smaller, and most pixels have zero error. In\figref{fig:gradient}(d), we show the depth gradients surrounding sparse pixels are smooth and close to ground truth, demonstrating the effectiveness of our propagation scheme. 

% Finally, in our experiments~\ref{sec:exp}, we validate the number of iterations $N$ and kernel size $k$ used for doing the CSPN.


\subsection{Complexity Analysis}
\label{subsec:time}

As formulated in~\equref{eqn:cspn}, our CSPN takes the operation of convolution, where the complexity of using CUDA with GPU for one step CSPN is $O(\log_2(k^2))$, where $k$ is the kernel size. This is because CUDA uses parallel sum reduction, which has logarithmic complexity. In addition,  convolution operation can be performed parallel for all pixels and channels, which has a constant complexity of $O(1)$. Therefore, performing $N$-step propagation, the overall complexity for CSPN is $O(\log_2(k^2)N)$, which is irrelevant to image size $(m, n)$.

SPN~\cite{liu2017learning} adopts scanning row/column-wise propagation in four directions. Using $k$-way connection and running in parallel, the complexity for one step is $O(\log_2(k))$. The propagation needs to scan full image from one side to another, thus the complexity for SPN is $O(\log_2(k)(m + n))$. Though this is already more efficient than the densely connected CRF proposed by~\cite{philipp2012dense}, whose implementation complexity with permutohedral lattice is $O(mnN)$, ours $O(\log_2(k^2)N)$ is more efficient since the number of iterations $N$ is always much smaller than the size of image $m, n$. We show in our experiments (\secref{sec:exp}), with $k=3$ and $N=12$, CSPN already outperforms SPN with a large margin (relative $30\%$), demonstrating both efficiency and effectiveness of the proposed approach.


\subsection{End-to-End Architecture}
\label{subsec:unet}
\begin{figure}[t]
\centering
\includegraphics[width=0.95\textwidth,height=0.45\textwidth]{fig/framework2.pdf}
\caption{Architecture of our networks with mirror connections for  depth estimation via transformation kernel prediction with CSPN (best view in color). Sparse depth is an optional input, which can be embedded into the CSPN to guide the depth refinement.}
\label{fig:arch}
\end{figure}

We now explain our end-to-end network architecture to predict both the transformation kernel and the depth value, which are the inputs to CSPN for depth refinement.
 As shown in \figref{fig:arch}, our network has some similarity with that from Ma \etal~\cite{Ma2017SparseToDense}, with the final CSPN layer that outputs a dense depth map.  
 
For predicting the transformation kernel $\kappa$ in \equref{eqn:cspn}, 
rather than building a new deep network for learning affinity same as Liu \etal~\cite{liu2017learning}, we branch an additional output from the given network, which shares the same feature extractor with the depth network. This helps us to save memory and time cost for joint learning of both depth estimation and transformation kernels prediction. 

Learning of affinity is dependent on fine grained spatial details of the input image. However, spatial information is weaken or lost with the down sampling operation during the forward process of the ResNet in~\cite{laina2016deeper}. Thus, we add mirror connections similar with the U-shape network~\cite{ronneberger2015u} by directed concatenating the feature from encoder to up-projection layers as illustrated by ``UpProj$\_$Cat'' layer in~\figref{fig:arch}. Notice that it is important to carefully select the end-point of mirror connections. Through experimenting three possible positions to append the connection, \ie after \textit{conv}, after \textit{bn} and after \textit{relu} as shown by the ``UpProj'' layer in~\figref{fig:arch} , we found the last position provides the best results by validating with the NYU v2 dataset (\secref{subsec:ablation}). 
In doing so, we found not only the depth output from the network is better recovered, and the results after the CSPN is additionally refined, which we will show the experiment section~(\secref{sec:exp}).
Finally we adopt the same training loss as~\cite{Ma2017SparseToDense}, yielding an end-to-end learning system.

		  %THis file has high level approach

%\FIXME{Following subsections should move to code gen section.}
%
%\subsection{Parametric Tiling}
%In this paper, we use parametric tile sizes for both Single-Level Tiling and
%Cache Oblivious Tiling. Existing techniques for parametric
%tiling~\cite{sanjay-lcpc2009, baskaran-etal-cgo10} have demonstrated that the
%parameterization does not incur significant performance overhead when compared
%to tiling by compile-time constants. 
%
%
%\subsection{Bounding Box of Tilable Band}
%For a given loop nest, tilable band specifies band of loops which are tilable
%or permutable. Usually, the band is specified by the start loop depth and end
%loop depth.  Bounding box is a cuboid (or hyper-rectangle) containing the
%iteration space of the tilable band.  There can be infinitely many bounding
%boxes for a given tilable band, but we chose the tightest bounding box among
%all the possibilities. Since there are outer loops surrounding the tilable
%band of loops, the bounding box is parameterised by outer loop iterators. In
%other words, there is a bounding box for each instance of outer loops. 
%


%In this section, we introduce the necessary background of our work. We first
%give a brief description of the polyhedral representation of programs, and the
%general flow of a polyhedral compiler.  Then, we discuss the legality of
%tiling, which is related to the input of our code generator.
%
%\begin{figure*}[tb]
%  \centering %\vspace*{6cm}
%  \includegraphics[scale=0.6]{figures/PolyCompiler}
%  \caption{\small{Polyhedral Compilation: the Polyhedral Reduced Dependence
%      (hyper) Graph (PRDG) serves as the intermediate representation.
%      Piecewise Quasi-Affine Functions (PQAFs) describe transformations.}}
%  \label{fig:compiler}
%\end{figure*}
%
%\subsection{Polyhedral Compilation and Representation}
%
%Figure~\ref{fig:compiler} shows the flow of polyhedral compilation.  First,
%dependence analysis of an input program (or a ``polyhedral section'' thereof)
%produces an intermediate representation (IR) in the form of~\cite{DRV-sched00}
%a \emph{Polyhedral Reduced Dependence (hyper) Graph} (PRDG).  Various analyses
%are performed on the PRDG to choose a number of mappings in the form of
%\emph{Piecewise Quasi-Affine Functions} (PQAFs) that specify the schedule as a
%set of \emph{multi-dimensional} vectors.  The PQAFs come with annotations to
%indicate whether each dimension is sequential or parallel, and also whether it
%is part of a \emph{tilable band}, i.e., whether tiling this band of dimensions
%is legal.  The transformations may be applied to the PRDG iteratively, and
%(eventually) the PRDG and QLAF are provided to a code-generator that produces
%code for various targets.
%
%One of the strengths of the polyhedral model is that a parametric program may
%be concisely represented with a PRDG with finite number of nodes (statements)
%and edges (dependences).  The potentially unbounded sets of instances of a
%statement are represented in abstract forms of integer sets, called
%\emph{domains}, and dependences between them as affine functions (or
%relations, which are viewed as a set-valued function) over these statement
%domains.  Indeed, every edge, $e$ from node $v$ to $w$, in the PRDG is
%annotated with two objects: (i) a domain, $D_e$ specifying the (subset of) the
%domain, $D_v$ of its source node, where the dependence occurs, and (ii) the
%affine function, $f$, such that for any point $z\in D_e$, the (set of)
%point(s) in $D_w$ on which it depends is given by $f(z)$.  $D_e$ is called the
%context of the edge, and $f$ is its dependence function.  We also use the
%notation $f(D_e)$ to denote the set valued image of $D_e$ by $f$.
%
%An affine function $\mathbb{Z}^n \rightarrow \mathbb{Z}^m$ may be expressed as
%$f(x) = A\vec{x} + \vec{b}$, where $\vec{x}$, function domain, is an integer
%vector of size $n$; $A$, linear part, is an $n\times m$ matrix; and $\vec{b}$,
%constant part, is an integer vector of size $m$.  A dependence is said to be
%uniform if the dependence function is only a constant offset, i.e., when the
%linear part $A$ is the identity.
%

%Why we do not expect execution time of COT to
%outperform wavefront tiling irrespective of lower LLC misses count.

% \begin{algorithm}[h]
%   \mbox{}

%   \begin{enumerate}
%   \item Extract Polyhedral Representation from C program
%   \item Determine per statement $s$ target mapping/schedule that specifies,
%     \begin{enumerate}
%     \item sequential dimensions
%     \item tilable band(s)
%     \end{enumerate}
%   \item Transform IR to above schedule
% 	\item Choose memory mappings
% 	\item Generate cache oblivious code
%   \end{enumerate}
%   \caption{PCOT: Polyhedral Cache Oblivious Tiling}
%   \label{alg:pcot}
% \end{algorithm}

% Local Variables: ***
% TeX-master: "PACT17.tex" ***
% fill-column: 78 ***
% End: ***
