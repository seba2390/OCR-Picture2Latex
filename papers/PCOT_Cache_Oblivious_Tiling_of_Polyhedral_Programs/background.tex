\section{Background}
\label{sec:background}

In this section, we present necessary background on tiling, cache oblivious
divide-and-conquer execution, and define the terminology used in this paper.

\subsection{Tiling}
Tiling is a well-known loop transformation for partitioning computations into
smaller, atomic (all inputs to a tile can be computed before its execution),
units called tiles~\cite{irigoin-popl88, Wolf91tiling}. The partitioning into
tiles improves data locality by altering the execution order of the operations.
Tiling also exposes coarse-grained parallelism making it the core transformation
for polyhedral automatic parallelizers such as Pluto~\cite{uday-pldi08}.

 The natural legality condition of tiling is that the dependences across tiles do not
create a cycle.  In compilers, this condition is typically expressed as fully
permutability (i.e., dependences are non-negative direction vectors), which is
a sufficient condition. In the rest of this paper, we assume that the programs
have been transformed to expose loop nests that satisfy this condition.  For
polyhedral programs, scheduling techniques to expose such loop nests are
available~\cite{uday-pldi08}.

\subsection{Cache Oblivious Tiling}
Cache oblivious algorithms~\cite{prokop-thesis99, frigo-etal-focs99} are based
on recursive formulation into smaller subproblems (divide-and-conquer). The
main argument is that as the problem sizes are recursively made smaller, a
subproblem is going to fit on some level of the memory hierarchy that may be
caches, main memory, etc. This class of algorithms is expected to take
advantage of all memory hierarchies through this strategy.

Cache Oblivious Tiling is a specialization of such algorithms based on tiling.
Tiles after a level of tiling can be tiled again with smaller tile
sizes to realize the divide-and-conquer execution pattern. COT may be viewed as
hierarchical tiling, except that the number of tiling levels are determined at
run-time through divide-and-conquer~\cite{carter1995hierarchical}. 

The key effect of multi-level tiling is to change the execution order of the
tiles. As illustrated in Figure~\ref{fig:tile-order}, the grouping of smaller
tiles; forming larger tiles increasing intra-tile reuse; is what accounts for
better cache utilization.


\begin{figure}
\centering
  \begin{subfigure}{0.2\textwidth}
    \includegraphics[width=\columnwidth]{wavefront-tile-order}
    \caption{Single-Level Tiling \label{fig:wavefront-tile-order}}
  \end{subfigure}
  \begin{subfigure}{0.2\textwidth}
    \includegraphics[width=\columnwidth]{cot-tile-order}
    \caption{Two-Level Tiling \label{fig:cot-tile-order}}
  \end{subfigure}
  \caption{\label{fig:tile-order}Execution order of the smallest tiles under single- and two-level
tiling. Two-level tiling may be viewed as Cache Oblivious Tiling where the
recursion reached the base case after one recursive step. With hierarchical tiling,
neighboring tiles form a larger tile that increase intra-tile reuse.
\vspace{-0.2cm}
}
\end{figure}


\subsection{Terminology}\label{sec:terminology}
We introduce a few terms, in addition to COT in the above.
%, that are used throughout this paper.

\paragraph{Single-Level Tiling} SLT is when loop tiling is applied once to
improve data locality with respect to a level of cache. The reason we restrict
to SLT is because SLT and COT expose the same number of tuning parameters. It
is known that even cache oblivious algorithms require the performance of the
leaf subproblems to be tuned to have good performance. The tuning effort
required is similar to SLT with the same number of tuning parameters, which
would not be the case if you apply multi-level tiling that multiplies the
number of tuning parameters.  

\paragraph{Tile Size} We use tile size interchangeably with the base case
threshold (and leaf tile size) in COT. Whenever tile sizes are discussed for
COT, it refers to the size of the leaf tile sizes, which is its tuning parameter. 

\paragraph{Off-Chip Accesses} OCA are all load accesses that read from the main
memory.  This includes the accesses due to Last-Level Cache misses due to load
operations, as well as those arising from prefetching instructions.

%\subsection{Benefits of Cache Oblivious Tiling}
%For compute-bound programs, the performance (speed) of SLT and COT is expected
%be similar.  As demonstrated by Zou and Rajopadhye~\cite{zou2015rajopadhye}
%for compute-bound stencil computations, reducing the amount of off-chip memory
%accesses (reduced Last Level Cache misses) does not always translate to
%performance. The latency penalty for LLC misses are mostly hidden by the
%computation, and thus further reducing the penalty have little impact on
%speed.
%
%%\FIXME{I am contradicting the
%%TurboTiling paper here - I think what happens is that for TurboTiling the
%%performance gains are not from reduced LLC misses, but more optimized execution
%%of tiles due to larger tiles. We should check this.}
%
%Therefore, the execution times of recursively tiled programs are not expected
%to be any better than those with just a single level of tiling if the tile sizes are
%well-tuned. In fact, we expect that the recursion to introduce overhead of
%various forms, such as functional call overhead, increased register pressure,
%increased difficulty for the compiler as the recursive functions cannot be
%inlined.
%
%However, the increased intra-tile data reuse improves the data locality with
%respect to lower levels of the memory hierarchy. In effect, what can be
%expected for COT with compute-bound programs is \emph{reduced off-chip memory
%accesses with similar performance as SLT, without the need to explicitly tune
%for LLC}. This what we validate in our experiments presented in
%Section~\ref{sec:experiments}.
%
%The two scenarios where COT is useful are memory-bound computations and energy
%efficient computing. When the program is memory-bound to the extent latency
%hiding is not possible, then improved data locality in all levels are expected
%to translate to speed improvements. Another important benefit of reduced LLC
%misses\footnote{In fact, COT reduces L2 cache misses as well, but we focus on
%LLC misses that are associated with the more costly off-chip memory accesses.}
%is energy. Energy consumption due to off-chip memory accesses is a
%non-negligible component of the total energy consumption of a computing
%platform, and significant reduction on off-chip memory accesses can be
%beneficial for energy~\cite{zou2015rajopadhye,mikami2011evaluation}.
%

%\section{Approach}
\begin{figure}[t]
\centering
\resizebox{0.48\textwidth}{!}{ 
  \includegraphics[width=\textwidth]{figures/workflow.PNG}
}
  \caption{Workflow of \system}
  \label{fig:workflow}
\end{figure}

Figure ~\ref{fig:workflow} shows the overall workflow of \system. The triggers for using \system are usually alert(s) from automated anomaly detection, or sometimes an SRE engineer's suspicion. There are three major steps: constructing the service  dependency graph, constructing the event causality graph,  and root cause ranking. The outputs are the root causes ranked by the likelihood. To support fast human investigation experience, we build an interactive UI as shown in  Figure~\ref{fig:UI}: the service dependency, events with causal links and additional details such as raw metrics or the developer contact (of a code deployment event) are presented to the user for next steps. As an  offline part of human investigation, we label/collect a data set, perform validation, and summarize the knowledge for further improvement on all incidents on a daily basis. %as validations and heterogeneous graph learning (HGL)~\cite{qiao2020heterogeneous} to synthesize the knowledge from existing cases in order to further improve the system.

\subsection{Constructing Service Dependency Graph}
\label{sec:appgraph}

The construction of the service dependency graph starts with the initial alerted or suspicious service(s), denoted as $I$. For example, in Figure ~\ref{fig:ex1_dep}, $I=\{\textit{Checkout}\}$. $I$ can contain multiple services based on the range of the trigger alerts or suspicions. We maintain domain service lists where domain-level alerts can be triggered because there is no clear service-level indication.

At the back end, \system maintains a global service dependency graph $G_{global}$ via distributed tracing and log analysis. The directed edge from nodes $A$ to $B$ (two services or system components) in the dependency graph indicates a service invocation or other forms of dependency. In Figure~\ref{fig:ex1_dep}, the black arrows indicate such edges. Bi-directional edges and cycles between the services can be possible and exist. In this work, the global dependency graph is updated daily.%by extracting from one day's total site traffic.

The service dependency (sub)graph $G$ is constructed using $G_{global}$ and $I$. An extended service list $L$ is first constructed by traversing each service in $I$ over $G_{global}$ for a radius range $r$. Each service $u \in L$ can be traversed by at least one service $v \in I$ within $r$ steps: $L=\{u|\exists v\in I, \ dist(u,v)\le r\ or\ dist(v,u)\le r\}$. Then, the service dependency subgraph $G$ is constructed by the nodes in $L$ and the edges between them in $G_{global}$. In our current implementation, $r$ is set to $2$, since this dependency graph may be dynamically extended in the next steps based on events' detail for longer issue chains or additional dependencies.

\subsection{Constructing Event Causality Graph}
\label{sec:causality}

In the second step, \system collects all supported events for each service in $G$ and constructs the causal links between events. 

\subsubsection{Collecting Events}

Table~\ref{tab:events} presents some example event types and detection techniques for \system's production implementation. For detection techniques, ``De Facto'' indicates that the event can be directly collected via a specific API or storage. %The detection can be done passively at the back end continuously then store anomaly events in different databases; or done actively by pulling data and run detection on the fly to save compute resources. 
The detection either runs passively in the back end to reduce delay and improve accuracy, or runs actively for only the services within the dependency graph range to save resources. %For example, low-level error signals or logs are detected actively since they are too many to scale. 

There are three major categories of events: performance metrics, status logs, and developer activities:
\begin{itemize}
    \item \emph{Performance metrics} represent an anomaly of monitored time series metrics. For example, high CPU usage indicates that the service is causing high CPU usage on a certain machine. In this category, most events are continuously and passively detected and stored. %For high CPU usage, threshold indicates the event is created when CPU usage is higher than certain predefined value. TPS spike indicates a spike in transaction per second, since TPS is a moving average value, we use some statistical model learned from historical data to detect such events.
    \item \emph{Status logs} are caused by abnormal system status, such as spike of HTTP error code metrics while accessing other services' endpoints. Different types of error metrics are important and supported in \system, including third-party APIs. For example, Bad Host indicates abnormal patterns on some machines running the service, and can be detected by a  clustering-based ML approach.%Markdown indicates that the whole service is down. 
    \item \emph{Developer activities} are the events generated when a certain activity of developers is triggered, such as code deployment and config change.
\end{itemize}

\begin{table}[t]
\centering
\caption{List of example event types used in \system}
\resizebox{0.4\textwidth}{!}{ 
\begin{tabular}{|c|c|c|}
\hline
Type                                & Event Type                  & Detection Technique  \\ \hline
\multirow{6}{*}{Performance Metrics} & High GC (Overhead)      & Rule-based        \\ \cline{2-3} 
                                    & High CPU Usage          & Rule-based        \\ \cline{2-3} 
%                                    & Out of Memory           & Rule-based        \\ \cline{2-3} 
%                                    & LB Connection Stacking  & Statistical Model \\ \cline{2-3} 
                                    & Latency Spike           & Statistical Model \\ \cline{2-3} 
                                    & TPS Spike               & Statistical Model \\ \cline{2-3} 
                                    & Database Anomaly        & ML Model          \\ \cline{2-3} 
                                    & Business Metric Anomaly & ML Model          \\ \hline
\multirow{4}{*}{Status Logs}        & WebAPI Error            & Statistical Model \\ \cline{2-3} 
                                    & Internal Error          & Statistical Model \\ \cline{2-3} 
                                    & ServiceClient Error     & Statistical Model \\ \cline{2-3} 
                                    & Bad Host                & ML Model          \\ \hline %\cline{2-3} 
%                                    & Hystrix Circuit Break   & De Facto          \\ \hline
\multirow{3}{*}{Developer Activities} & Code Deployment         & De Facto          \\ \cline{2-3} 
                                    & Configuration Change    & De Facto          \\ \cline{2-3} 
                                    & Execute URL             & De Facto          \\ \hline
\end{tabular}
}
\label{tab:events}
\end{table}

In Groot, there are more than a dozen event types such as \emph{Latency Spike} as listed in the column 2 of Table~\ref{tab:events}. 
Each event type is characterized by three aspects: $Name$ indicates the name of this event type; $Lookback Period$ %\footnote{In Figure~\ref{fig:ex2_n1}, there are two periods, 1 day indicates the look-back range if the model has already finished deployment, 4 days indicates the range if the model deployment is still ongoing(incremental deployment).} 
indicates the time range to look back (from the time when the use of \system is triggered) for collecting events of this event type;  $PropertyType$ indicates the types of the properties that an event of this event type should hold. 
$PropertType$  is characterized by a vector of pairs, each of which indicates the string type for a property's name and the primitive type for the property's value such as string, integer, and float. 
Formally, an event type is defined as a tuple: 
$ET = <Name, Lookback Period, PropertyType>$ 
where 
$PropertyType = <(string, \textit{type}_1), ..., (string, \textit{type}_{n})>$ ($n$ is the number of properties that an event of this event type holds). 
%

Each event of a certain event type $ET$ is characterized by four aspects:
$\textit{Service}$ indicates the service name that the event belongs to; $\textit{Type}$ indicates $ET$'s $\textit{Name}$;  $\textit{StartTime}$ indicates the time when the event happens; $\textit{Properties}$ indicates the properties that the event  holds.
Formally, an event is defined as a tuple: 
$e = <Service, Type, StartTime, Properties>$ 
where $Properties$ is an instantiation of $ET$'s  $PropertyType$. 


%and each event is defined as $e = \{<\textit{Property}_i, \textit{value}_i>\}$. Each event type serves as a template for the event instantiation. such as a string, an integer, a float or a set of primitive types while $\textit{value}$ is limited to primitive data types. 
%
%Each event is defined as a sequence of property-value pairs where the set size is $n$.

For example, in Figure~\ref{fig:example1}, the generated event for \emph{Latency Spike in DataCenter-A} in \emph{Service-C} would be $<``\textit{Service-C}'', ``\textit{Latency\ Spike}'', \textit{2021/08/01-12:36:04}, <(``\textit{DataCenter}'',``\textit{DC-1}''),  ...>>$. %So for each service in $G$, we detect/collect and filter the events within specified time range of the alert.

\subsubsection{Constructing Causal Link}

After collecting all events on all services in $G$, in this step, causal links between these events are constructed for RCA ranking. The causal links (red arrows) in Figure~\ref{fig:ex1_cas} are such examples. A causal link represents that the source event can possibly be caused by the target event. SRE knowledge is engineered into rules and used to create causal links between the pairs of events. %As shown in Figure~\ref{fig:example2}, there are two categories of rules: basic rules and conditional rules. 

A rule for constructing a causal link is defined as a tuple:  $Rule = <Target\mbox{-}Type,  Source\mbox{-}Events, Target\mbox{-}Events, Direction,\\ Target\mbox{-}Service,  Condition>$  ($Condition$ can be optionally specified). $Target\mbox{-}Type$ indicates the type of the rule, being either $Static$ or $Dynamic$ (explained further later). $Source\mbox{-}Events$ indicates the type of the causal link's source event ($Source\mbox{-}Events$ are listed in the names of the rules shown in Figures~\ref{fig:ex2_n1},~\ref{fig:ex2_n2} and~\ref{fig:dynamic_example}).   $Target\mbox{-}Events$ indicates the type of the causal link's target event. $Direction$ indicates the direction of the casual link between the target event and source event. $Target\mbox{-}Service$ indicates the service that the target event should belong to. Note that $Target\mbox{-}Service$ in $Static$ rules can be  $Self$, which indicates that the target event would be within the same service as the source event, or $Outgoing$/$Incoming$, which indicates that the target event would belong to the downstream/upstream services of the service that the source event belongs to in $G$.

\begin{figure}[t]
\centering
\includegraphics[width=0.56\columnwidth]{figures/example3.png}
\caption{Example of dynamic rule}
\label{fig:dynamic_example}
\end{figure}

There are two categories of special rules. The first category is \emph{dynamic} rules (i.e., rules whose $Target\mbox{-}Type$  is set to $Dynamic$) to support dynamic dependencies. Here $Target\mbox{-}Service$ does not indicate any of the three possible options listed earlier but indicates the name of the target service that \system would need to create. For example, live DB dependencies are not available due to different tech stacks and high volume. In Figure~\ref{fig:dynamic_example}, a DB issue (DB Markdown) is shown in \emph{Service-A}. Based on the listed \emph{dynamic} rule, \system creates a new ``service'' \emph{DB-1} in $G$, a new event ``Issues'' that belongs to \emph{DB-1}, and a causal link between the two events.  In practice, the SRE teams use dynamic rules to cover a lot of third-party services and database issues since the live dependencies are not easy to maintain.  %However through the internal error messages and dynamic rules, \system is still able to handle these dependencies. %we can still support external inferences. 

The second category of special rules is \emph{conditional} rules. \emph{Conditional} rules are used when some prerequisite conditions should be satisfied before a certain causal link is created. In these rules, $Condition$ is specified with a boolean predicate. As shown in Figure~\ref{fig:ex2_n2}, the SRE teams believe \emph{Latency Spike} events from different services are related only when both events happen within the same data center. Based on this observation, \system would first evaluate the predicate in $Condition$ and build only the causal link when the predicate is true. A conditional rule overwrites the basic rule on the same source-target event pair.

When constructing causal links, \system first applies the \emph{dynamic} rules so that dynamic dependencies and events are first created at once. Then for every event in the initial services (denoted as $I$), if the rule conditions are satisfied, one or many causal links are created from this event to other events from the same or upstream/downstream services. When a causal link is created, the step is repeated recursively for the target event (as a new origin) to create new causal links. After no new causal links are created, the construction of the event causality graph is finished.

% When \system constructs the causal links, \system first processes all dynamic rules as they may create new event nodes in the graph. %\system enumerates the dynamic rules on each existing event node and also on the newly added nodes (There could also be rules applicable to the newly added nodes) until no new event nodes can be created. 


%Each rule is defined as a predicate containing both events' property-value pair. If the predicate evaluates to be true between two events, then we would add the edge in the causality graph. For example, in Figure~\ref{fig:example1}, the rule used to establish the edge between \emph{GC overhead in RNO} and \emph{Latency increase in LVS, RNO, SLC} would be like this: Suppose we are now determining whether there should be a link from event $u$ to event $v$, then this rule would be $u.\text{pool} = v.\text{pool}\ and\ u.\text{type} = ``\text{High GC Overhead}"\ and\ v.\text{type} = ``\text{Latency increase}"\ and\ u.\text{center} \cap v.\text{center} \ne \emptyset$ which holds true for these two events. Each causality link is also associated with a weight which represents the likelihood of causality - we set all initial values as $1.0$. Overtime these value are updated by the statistical analysis result of the collected data set.


\subsection{Root Cause Ranking}
Finally, \system ranks and recommends the most probable root causes from the event causality graph. Similar to how search engines infer the importance of pages by page links, we customize the PageRank \cite{manning2010introduction} algorithm to calculate the root cause ranking; the customized algorithm is named as GrootRank. The input is the event causality graph from the previous step. Each edge is associated with a weighted score for weighted propagation. The default value is set as $1$, and is set lower for alerts with high false-positive rates. 

Based on the observation that dangling nodes are more likely to be the root cause, we customize the personalization vector as $P_n = f_n $ or $P_d = 1$, where $P_d$ is the personalization score for dangling nodes, and $P_n$ is for the remaining nodes; and $f_n$ is a value smaller than 1 to enhance the propagation between dangling nodes. In our work, the parameter setting is $f_n = 0.5$, $\alpha = 0.85$, $max_{iter} = 100$ (which are parameters for the PageRank algorithm). Figure \ref{fig:person} illustrates an example. The grey circles are the events collected from three services and one database. The grey arrows are the dependency links and the red ones are the causal links with the weight of $1$. Both of the PageRank and GrootRank algorithms detect $event 5$ (DB issue) as the root cause, which is expected and correct. However, the PageRank algorithm ranks $event 4$ higher than $event 3$. But $event 3$ of $\textit{Service-C}$ is more likely to be the second most possible root cause (besides $event 5$), because the scores on dangling nodes are propagated to all others equally in each iteration. We can see that $event 3$ is correctly ranked as second using the GrootRank algorithm.

The second step of GrootRank is to break the tied results from the previous step. The tied results are due to the fact that the event graph can contain multiple disconnected sub-graphs with the same shape. We design two techniques to untie the ranking: 
\begin{figure}[t]
\centering
  \includegraphics[width=0.8\columnwidth]{figures/personalvector.png}
  \caption{Example of personalization vector customization}
  \label{fig:person}
\end{figure}

\begin{figure}[t]
\centering
  \includegraphics[width=0.8\columnwidth]{figures/accessdistance.png}
  \caption{Example of using access distance to untie the ranking results}
  \label{fig:untie}
\end{figure}
\begin{enumerate}
\item For each joint event, the access distance (sum) is calculated from the initial anomaly service(s) to the service where the event belongs to. If any ``access'' is not reachable, the distance is set as $d_m+1$ where $d_m$ is the maximum possible distance. The one with shorter access distance (sum) would be ranked higher and vice versa. Figure \ref{fig:untie} presents an example, where \emph{Service-A} and \emph{Service-B} are both initial anomaly services. Since \system suspects that $event 2$ is caused by either $event 3$ or $event 1$ with the same weight. The scores of $event 3$ and $event 1$ are tied. Then, $event 3$ has a score of $1$ (i.e., $0+1$) and $event 1$ has a score of 2 (i.e., $0+2$), since it is not reachable by \emph{Service-B}). Therefore, $event 3$ is ranked first and logical. 
\item For the remaining joint results with the same access distances, \system continues to untie by using the historical root cause frequency of the event types under the same trigger conditions (e.g., checkout domain alerts). This frequency information is generated from the manually labeled dataset. A more frequently occurred root cause type is ranked higher.% than the less frequent ones.
\end{enumerate}


\subsection{Rule Customization Management}

While \system users create or update the rules,  there could be overlaps, inconsistencies, or even conflicts being introduced such as the example in Figure~\ref{fig:ex2_n2}. \system uses two graphs to manage the rule relationships and avoid conflicts for users. One graph is to represent the link rules between events in the same service (\emph{Same-Graph}) while the other is to represent links between different services (\emph{Diff-Graph}). The nodes in these two graphs are the event types defined in Section~\ref{sec:causality}. There are three statuses between each (directional) pair of event types: (1) no rule, (2) only basic rule, and (3) conditional rule (since it overwrites the basic rule). In \emph{Same-Graph}, \system does not allow self-loop as it does not build links between an event and itself.
% but it is possible that we build links between different services with the same event type.

When rule change happens, existing rules are enumerated to build edges in \emph{Same-Graph} and \emph{Diff-Graph} based on $Target\mbox{-}Events$ and $Target\mbox{-}Service$. Based on the users' operation of 
% \begin{itemize}
%     \item 
    (1) ``remove a rule'',  \system removes the corresponding edge on the graphs;
    % \item 
    (2) ``add/update a rule'',  \system checks whether there are existing edges between the given event types, and then warns the users for possible overwrites. 
    % The users can also combine the conditional rules.   % while users are adding basic rules between event types if there are existing conditional rules between them.
    If there are no conflicts, \system just adds/updates edges between the event types.
    % \item Add conditional rules. We would first alert the possible overwrite. Then if users are about to add new conditional rules on the top of existing conditional rules, we would ask the users to combine these two conditions to add a new one. We then build or change all corresponding edges to status 3.
% \end{itemize} 

After all changes, \system extracts the rules from the graphs by converting each edge to a single rule. These rules are automatically implemented, and then tested against our labeled data set. The \system users need to review the changes with validation reports before the changes go online.

% Note that currently we don't check the consistencies between dynamic rules as we cannot process the dynamic event types, but this could be solved in the future by using nodes with symbolic values to represent such event types. 		  %THis file has high level approach

%\FIXME{Following subsections should move to code gen section.}
%
%\subsection{Parametric Tiling}
%In this paper, we use parametric tile sizes for both Single-Level Tiling and
%Cache Oblivious Tiling. Existing techniques for parametric
%tiling~\cite{sanjay-lcpc2009, baskaran-etal-cgo10} have demonstrated that the
%parameterization does not incur significant performance overhead when compared
%to tiling by compile-time constants. 
%
%
%\subsection{Bounding Box of Tilable Band}
%For a given loop nest, tilable band specifies band of loops which are tilable
%or permutable. Usually, the band is specified by the start loop depth and end
%loop depth.  Bounding box is a cuboid (or hyper-rectangle) containing the
%iteration space of the tilable band.  There can be infinitely many bounding
%boxes for a given tilable band, but we chose the tightest bounding box among
%all the possibilities. Since there are outer loops surrounding the tilable
%band of loops, the bounding box is parameterised by outer loop iterators. In
%other words, there is a bounding box for each instance of outer loops. 
%


%In this section, we introduce the necessary background of our work. We first
%give a brief description of the polyhedral representation of programs, and the
%general flow of a polyhedral compiler.  Then, we discuss the legality of
%tiling, which is related to the input of our code generator.
%
%\begin{figure*}[tb]
%  \centering %\vspace*{6cm}
%  \includegraphics[scale=0.6]{figures/PolyCompiler}
%  \caption{\small{Polyhedral Compilation: the Polyhedral Reduced Dependence
%      (hyper) Graph (PRDG) serves as the intermediate representation.
%      Piecewise Quasi-Affine Functions (PQAFs) describe transformations.}}
%  \label{fig:compiler}
%\end{figure*}
%
%\subsection{Polyhedral Compilation and Representation}
%
%Figure~\ref{fig:compiler} shows the flow of polyhedral compilation.  First,
%dependence analysis of an input program (or a ``polyhedral section'' thereof)
%produces an intermediate representation (IR) in the form of~\cite{DRV-sched00}
%a \emph{Polyhedral Reduced Dependence (hyper) Graph} (PRDG).  Various analyses
%are performed on the PRDG to choose a number of mappings in the form of
%\emph{Piecewise Quasi-Affine Functions} (PQAFs) that specify the schedule as a
%set of \emph{multi-dimensional} vectors.  The PQAFs come with annotations to
%indicate whether each dimension is sequential or parallel, and also whether it
%is part of a \emph{tilable band}, i.e., whether tiling this band of dimensions
%is legal.  The transformations may be applied to the PRDG iteratively, and
%(eventually) the PRDG and QLAF are provided to a code-generator that produces
%code for various targets.
%
%One of the strengths of the polyhedral model is that a parametric program may
%be concisely represented with a PRDG with finite number of nodes (statements)
%and edges (dependences).  The potentially unbounded sets of instances of a
%statement are represented in abstract forms of integer sets, called
%\emph{domains}, and dependences between them as affine functions (or
%relations, which are viewed as a set-valued function) over these statement
%domains.  Indeed, every edge, $e$ from node $v$ to $w$, in the PRDG is
%annotated with two objects: (i) a domain, $D_e$ specifying the (subset of) the
%domain, $D_v$ of its source node, where the dependence occurs, and (ii) the
%affine function, $f$, such that for any point $z\in D_e$, the (set of)
%point(s) in $D_w$ on which it depends is given by $f(z)$.  $D_e$ is called the
%context of the edge, and $f$ is its dependence function.  We also use the
%notation $f(D_e)$ to denote the set valued image of $D_e$ by $f$.
%
%An affine function $\mathbb{Z}^n \rightarrow \mathbb{Z}^m$ may be expressed as
%$f(x) = A\vec{x} + \vec{b}$, where $\vec{x}$, function domain, is an integer
%vector of size $n$; $A$, linear part, is an $n\times m$ matrix; and $\vec{b}$,
%constant part, is an integer vector of size $m$.  A dependence is said to be
%uniform if the dependence function is only a constant offset, i.e., when the
%linear part $A$ is the identity.
%

%Why we do not expect execution time of COT to
%outperform wavefront tiling irrespective of lower LLC misses count.

% \begin{algorithm}[h]
%   \mbox{}

%   \begin{enumerate}
%   \item Extract Polyhedral Representation from C program
%   \item Determine per statement $s$ target mapping/schedule that specifies,
%     \begin{enumerate}
%     \item sequential dimensions
%     \item tilable band(s)
%     \end{enumerate}
%   \item Transform IR to above schedule
% 	\item Choose memory mappings
% 	\item Generate cache oblivious code
%   \end{enumerate}
%   \caption{PCOT: Polyhedral Cache Oblivious Tiling}
%   \label{alg:pcot}
% \end{algorithm}

% Local Variables: ***
% TeX-master: "PACT17.tex" ***
% fill-column: 78 ***
% End: ***
