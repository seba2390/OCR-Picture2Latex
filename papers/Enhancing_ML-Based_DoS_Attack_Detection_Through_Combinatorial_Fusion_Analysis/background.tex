\begin{comment}
\begin{table*}[ht!]
\caption{Types of network traffic in the LYCOS-IDS dataset and their training and testing sets.}
\label{tab:lycos3}
\centering
\begin{small}
\begin{tabular}{lll}
Traffic & Training & Test\\ \hline
Benign & 330474 & 110,158 \\
DoS\_hulk & 119241 & 39,747 \\
Portscan & 119197 & 39,732 \\
DDoS & 71761 & 23,920 \\
DoS\_goldeneye & 5073 & 1,691 \\
DoS\_slowloris & 4255 & 1,418 \\
DoS\_slowhttptest & 3649 & 1,216 \\
FTP\_patator & 3001 & 1,000 \\
SSH\_patator & 2218 & 739 \\
Webattack\_bruteforce & 1020 & 340 \\
Bot & 550 & 183 \\
Webattack\_xss & 489 & 163 \\
Webattack\_sql\_injection & 9 & 3 \\
Heartbleed & 7 & 2 \\
\hline
Total & 660944 &  220,312 \\
\end{tabular}
\end{small}
\end{table*}
\end{comment}

Combinatorial Fusion Analysis (CFA) is an emerging approach for combining results from multiple scoring systems. These scoring systems can be described by a set of statistical attributes or variables at the data level or by a group of algorithms or models at the computational, informatics level. In this paper, we utilize the CFA approach to combine the results from six candidate ML models, namely Linear Discriminant Analysis(LDA), Quadratic Discriminant Analysis (QDA), SVM, K-Nearest Neighbors (k-NN), Decision Trees (DT) and Random Forest (RF).

Here, the CFA approach is a process of ensemble reinforcement learning, where all possible rank and score combinations are considered to find the optimal combination of the candidate models. We apply this approach to address the problem of DoS detection. To determine the most effective model fusion performance, our methodology utilizes the emerging CFA approaches which involve considering both the score and rank functions for each model or scoring system and applying various metrics such as average score combination, average rank combination, and weighted combination that factor in the diversity strength of each scoring system. We also employ a 2-model permutation technique as part of our analysis. 

%Overall, our paper presents a comprehensive methodology for combining the results from multiple ML using the emerging approach of CFA which has been applied to a wide range of problems in various fields, including finance, healthcare, and social sciences. For further in-depth information regarding the CFA, we suggest referring to follow good resources \cite{hsu2003raf, frank2005comparing,hsucombining, xu1992methods, andresini2021insomnia}.

\begin{table}[h!]
\caption{Types of network traffic in the LYCOS-IDS dataset and their training and testing sets.}
\label{tab:lycos3}
\centering
\begin{tabular}{lll}
Traffic & Training & Test\\ \hline
Benign & 330474 & 110,158 \\
DoS\_hulk & 119241 & 39,747 \\
Portscan & 119197 & 39,732 \\
DDoS & 71761 & 23,920 \\
DoS\_goldeneye & 5073 & 1,691 \\
DoS\_slowloris & 4255 & 1,418 \\
DoS\_slowhttptest & 3649 & 1,216 \\
FTP\_patator & 3001 & 1,000 \\
SSH\_patator & 2218 & 739 \\
Webattack\_bruteforce & 1020 & 340 \\
Bot & 550 & 183 \\
Webattack\_xss & 489 & 163 \\
Webattack\_sql\_injection & 9 & 3 \\
Heartbleed & 7 & 2 \\
\hline
Total & 660944 &  220,312 \\
\end{tabular}
\end{table}