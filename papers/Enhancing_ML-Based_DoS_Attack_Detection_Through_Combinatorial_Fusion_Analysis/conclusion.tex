This work uses the fusion approach to enhance ML-based DoS attack detection. Specifically, we targeted low-profile attack detection, an area where single models faltered. By merging multiple ML models with advanced CFA metrics, we improved precision, recall, and F1-score. This produced a highly effective combined model that nearly perfectly detected all attacks, including low-profile ones, achieving 100\% recall. Unlike most single models, our approach succeeded in this. CFA methods hold promise for boosting DoS attack detection. Future work can refine algorithms, explore more scoring systems, and integrate emerging tech for stronger defense. Advancing in this field ensures secure online services despite evolving threats.

%In this study, we explored the effectiveness of the combinatorial fusion approach for improving the detection of DoS attacks. We also paid special attention to identifying a model that could detect low-profile attacks since the single models explored could not do that. By combining multiple ML models using one of the advanced CFA metrics, our fusion achieved enhanced precision, recall, and F1-score performance. Remarkably, this approach yielded a highly effective combined model capable of detecting all attacks with near-perfect scores, including low-profile attacks. Notably, our model achieved a 100\% recall in identifying low-profile attacks. This is something that almost all the single models could not do. CFA approaches offer a promising avenue for enhancing DoS attack detection capabilities. Further research can focus on refining algorithms, exploring additional scoring systems, and integrating emerging technologies to detect and strengthen the defense against DoS attacks. By continuously advancing our understanding and techniques in this field, we can ensure the availability and security of online services in the face of evolving threats.