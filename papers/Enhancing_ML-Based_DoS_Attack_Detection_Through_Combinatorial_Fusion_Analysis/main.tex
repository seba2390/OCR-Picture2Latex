\documentclass[conference]{IEEEtran}
%% IEEE CNS addition:
\makeatletter
\def\ps@headings{%
\def\@oddhead{\mbox{}\scriptsize\rightmark \hfil \thepage}%
\def\@evenhead{\scriptsize\thepage \hfil \leftmark\mbox{}}%
\def\@oddfoot{}%
\def\@evenfoot{}}
\makeatother
\pagestyle{empty}




\IEEEoverridecommandlockouts
% The preceding line is only needed to identify funding in the first footnote. If that is unneeded, please comment it out.

\usepackage{geometry}
 \geometry{
 top = 0.75in,
 bottom=1.1in,
 right=0.625in,
 left=0.625in,
 }

%\usepackage[left=0.62in,right=0.62in,top=0.75in, bottom=1in]{geometry}

 \setlength{\columnsep}{0.245in}

%\usepackage{breqn}
\usepackage{cite}
\usepackage{amsmath,amssymb,amsfonts}
\usepackage{algorithmic}
\usepackage{graphicx}
\usepackage{textcomp}
\usepackage{xcolor}
\usepackage{comment}
\usepackage{tabularx}

\usepackage{caption}
\usepackage{subcaption}
\usepackage{multirow}
\captionsetup[subfigure]{justification=centering}
\usepackage{multicol}
\usepackage{array}
\usepackage{float} 
\usepackage{adjustbox}

\def\BibTeX{{\rm B\kern-.05em{\sc i\kern-.025em b}\kern-.08em
    T\kern-.1667em\lower.7ex\hbox{E}\kern-.125emX}}

\makeatletter % changes the catcode of @ to 11
\newcommand{\linebreakand}{%
  \end{@IEEEauthorhalign}
  \hfill\mbox{}\par
  \mbox{}\hfill\begin{@IEEEauthorhalign}
}
\makeatother % changes the catcode of @ back to 12


\begin{document}

\title{%Exploring ML Explainability Towards DoS Detection using Combinatorial Fusion.
%DoS Attack Detection Improved: Combining Multiple ML Models using Combinatorial Fusion
%Leveraging Combinatorial Fusion for Enhanced ML-based DoS Attack Detection
%Unleashing the Power of Combinatorial Fusion
Enhancing ML-Based DoS Attack Detection Through Combinatorial Fusion Analysis

% Combinatorial Fusion-based Enhancement of ML Prediction Quality for Bitcoin Price
%DoS Attack Detection Improved: Combining Multiple ML Models using Combinatorial Fusion Technology.
%{\footnotesize \textsuperscript{*}Note: Sub-titles are not captured in Xplore and should not be used}
%\thanks{Identify applicable funding agency here. If none, delete this.}
}

\author{\IEEEauthorblockN{Evans Owusu}
\IEEEauthorblockA{%\textit{Computer \& Information Science} \\
\textit{Fordham Univ, NY USA}\\
%NY USA \\
eowusu3@fordham.edu}
\and
\IEEEauthorblockN{Mohamed Rahouti}
\IEEEauthorblockA{%\textit{Computer \& Information Science} \\
\textit{Fordham Univ., NY USA}\\
%NY, USA \\
mrahouti@fordham.edu}
\and
\IEEEauthorblockN{D. Frank Hsu}
\IEEEauthorblockA{%\textit{Computer \& Information Science} \\
%\textit{Laboratory of Informatics and Data Mining}\\
\textit{Fordham Univ., NY USA}\\
%NY, USA \\
hsu@fordham.edu}

\and

%\linebreakand % <----- NOTE HERE, breaking after the third one!

\IEEEauthorblockN{Kaiqi Xiong}
\IEEEauthorblockA{%\textit{Cyber Florida} \\
\textit{Cyber Florida, FL USA}\\
%FL, USA \\
xiongk@usf.edu}
\and
\IEEEauthorblockN{Yufeng Xin}
\IEEEauthorblockA{%\textit{Renaissance Computing Institute} \\
%\textit{Univ. of North Carolina at Chapel Hill}\\
\textit{RENCI, NC USA}\\
%NC, USA \\
yxin@renci.org}
}


\maketitle

\begin{abstract}
%Detecting and mitigating Denial-of-Service (DoS) attacks is crucial for ensuring the availability and security of online services. While various machine learning (ML) models have been utilized for DoS attack detection, there is a need for innovative approaches to improving their performance. In this paper, we propose adopting a cutting-edge approach called combinatorial fusion, which leverages recently developed algorithms to combine multiple ML models for improved DoS attack detection. Our methodology involves advanced score combination, rank combination, weighted combination techniques, and the diversity strength of scoring systems. Through rigorous performance evaluations, we showcase the efficacy of the combinatorial fusion approach. Our evaluations encompass key metrics such as detection precision, recall, and F1-score, providing comprehensive insights into the interpretability and effectiveness of our approach. We highlight the challenge faced by individual models in classifying low-profiled attacks while excelling in other attack types. To overcome this limitation, model fusion techniques were used to create a comprehensive model capable of addressing both low-profiled attacks and other traffic types. Furthermore, our findings highlight the potential of this approach for enhancing DoS attack detection capabilities and contributing to the development of more robust defense mechanisms.
Mitigating Denial-of-Service (DoS) attacks is vital for online service security and availability. While machine learning (ML) models are used for DoS attack detection, new strategies are needed to enhance their performance. We suggest an innovative method, combinatorial fusion, which combines multiple ML models using advanced algorithms. This includes score and rank combinations, weighted techniques, and diversity strength of scoring systems. Through rigorous evaluations, we demonstrate the effectiveness of this fusion approach, considering metrics like precision, recall, and F1-score. We address the challenge of low-profiled attack classification by fusing models to create a comprehensive solution. Our findings emphasize the potential of this approach to improve DoS attack detection and contribute to stronger defense mechanisms.
\end{abstract}

\begin{IEEEkeywords}
%Denial of Service, combinatorial fusion, machine learning, rank, score
Cognitive diversity (CD), combinatorial fusion analysis (CFA), Denial of Services (DoS), machine learning (ML), rank function, rank-score characteristic (RSC) function, and score function
\end{IEEEkeywords}

\section{Introduction} % \leavevmode
% \\
% \\
% \\
% \\
% \\
\section{Introduction}
\label{introduction}

AutoML is the process by which machine learning models are built automatically for a new dataset. Given a dataset, AutoML systems perform a search over valid data transformations and learners, along with hyper-parameter optimization for each learner~\cite{VolcanoML}. Choosing the transformations and learners over which to search is our focus.
A significant number of systems mine from prior runs of pipelines over a set of datasets to choose transformers and learners that are effective with different types of datasets (e.g. \cite{NEURIPS2018_b59a51a3}, \cite{10.14778/3415478.3415542}, \cite{autosklearn}). Thus, they build a database by actually running different pipelines with a diverse set of datasets to estimate the accuracy of potential pipelines. Hence, they can be used to effectively reduce the search space. A new dataset, based on a set of features (meta-features) is then matched to this database to find the most plausible candidates for both learner selection and hyper-parameter tuning. This process of choosing starting points in the search space is called meta-learning for the cold start problem.  

Other meta-learning approaches include mining existing data science code and their associated datasets to learn from human expertise. The AL~\cite{al} system mined existing Kaggle notebooks using dynamic analysis, i.e., actually running the scripts, and showed that such a system has promise.  However, this meta-learning approach does not scale because it is onerous to execute a large number of pipeline scripts on datasets, preprocessing datasets is never trivial, and older scripts cease to run at all as software evolves. It is not surprising that AL therefore performed dynamic analysis on just nine datasets.

Our system, {\sysname}, provides a scalable meta-learning approach to leverage human expertise, using static analysis to mine pipelines from large repositories of scripts. Static analysis has the advantage of scaling to thousands or millions of scripts \cite{graph4code} easily, but lacks the performance data gathered by dynamic analysis. The {\sysname} meta-learning approach guides the learning process by a scalable dataset similarity search, based on dataset embeddings, to find the most similar datasets and the semantics of ML pipelines applied on them.  Many existing systems, such as Auto-Sklearn \cite{autosklearn} and AL \cite{al}, compute a set of meta-features for each dataset. We developed a deep neural network model to generate embeddings at the granularity of a dataset, e.g., a table or CSV file, to capture similarity at the level of an entire dataset rather than relying on a set of meta-features.
 
Because we use static analysis to capture the semantics of the meta-learning process, we have no mechanism to choose the \textbf{best} pipeline from many seen pipelines, unlike the dynamic execution case where one can rely on runtime to choose the best performing pipeline.  Observing that pipelines are basically workflow graphs, we use graph generator neural models to succinctly capture the statically-observed pipelines for a single dataset. In {\sysname}, we formulate learner selection as a graph generation problem to predict optimized pipelines based on pipelines seen in actual notebooks.

%. This formulation enables {\sysname} for effective pruning of the AutoML search space to predict optimized pipelines based on pipelines seen in actual notebooks.}
%We note that increasingly, state-of-the-art performance in AutoML systems is being generated by more complex pipelines such as Directed Acyclic Graphs (DAGs) \cite{piper} rather than the linear pipelines used in earlier systems.  
 
{\sysname} does learner and transformation selection, and hence is a component of an AutoML systems. To evaluate this component, we integrated it into two existing AutoML systems, FLAML \cite{flaml} and Auto-Sklearn \cite{autosklearn}.  
% We evaluate each system with and without {\sysname}.  
We chose FLAML because it does not yet have any meta-learning component for the cold start problem and instead allows user selection of learners and transformers. The authors of FLAML explicitly pointed to the fact that FLAML might benefit from a meta-learning component and pointed to it as a possibility for future work. For FLAML, if mining historical pipelines provides an advantage, we should improve its performance. We also picked Auto-Sklearn as it does have a learner selection component based on meta-features, as described earlier~\cite{autosklearn2}. For Auto-Sklearn, we should at least match performance if our static mining of pipelines can match their extensive database. For context, we also compared {\sysname} with the recent VolcanoML~\cite{VolcanoML}, which provides an efficient decomposition and execution strategy for the AutoML search space. In contrast, {\sysname} prunes the search space using our meta-learning model to perform hyperparameter optimization only for the most promising candidates. 

The contributions of this paper are the following:
\begin{itemize}
    \item Section ~\ref{sec:mining} defines a scalable meta-learning approach based on representation learning of mined ML pipeline semantics and datasets for over 100 datasets and ~11K Python scripts.  
    \newline
    \item Sections~\ref{sec:kgpipGen} formulates AutoML pipeline generation as a graph generation problem. {\sysname} predicts efficiently an optimized ML pipeline for an unseen dataset based on our meta-learning model.  To the best of our knowledge, {\sysname} is the first approach to formulate  AutoML pipeline generation in such a way.
    \newline
    \item Section~\ref{sec:eval} presents a comprehensive evaluation using a large collection of 121 datasets from major AutoML benchmarks and Kaggle. Our experimental results show that {\sysname} outperforms all existing AutoML systems and achieves state-of-the-art results on the majority of these datasets. {\sysname} significantly improves the performance of both FLAML and Auto-Sklearn in classification and regression tasks. We also outperformed AL in 75 out of 77 datasets and VolcanoML in 75  out of 121 datasets, including 44 datasets used only by VolcanoML~\cite{VolcanoML}.  On average, {\sysname} achieves scores that are statistically better than the means of all other systems. 
\end{itemize}


%This approach does not need to apply cleaning or transformation methods to handle different variances among datasets. Moreover, we do not need to deal with complex analysis, such as dynamic code analysis. Thus, our approach proved to be scalable, as discussed in Sections~\ref{sec:mining}. \label{sec:introduction}


\section{Related Work} \section{Related Work}
\label{sec:related_work}
We now provide a brief overview of related work in the areas of language grounding and transfer for reinforcement learning.
%There has been work on learning to make optimal local decisions for structured prediction problems~\cite{daume2006searn}.
%
%\newcite{branavan2010reading} looked at a similar task of building a partial model of the environment while following instructions. The differences with our work are (1) the text in their case is instructions, while we only have text describing the environment, and (2) their environment is deterministic, hence the transition function can be learned more easily. 
%
%TODO - model-based RL, value iteration, predictron.


\subsection{Grounding Language in Interactive Environments}
In recent years, there has been increasing interest in systems that can utilize textual knowledge to learn control policies. Such applications include interpreting help documentation~\fullcite{branavan2010reading}, instruction following~\fullcite{vogel2010learning,kollar2010toward,artzi2013weakly,matuszek2013learning,Andreas15Instructions} and learning to play computer games~\fullcite{branavan2011nonlinear,branavan2012learning,narasimhan2015language,he2016deep}. In all these applications, the models are trained and tested on the same domain.

Our work represents two departures from prior work on grounding. First, rather than optimizing control performance for a single domain,
we are interested in the multi-domain transfer scenario, where language 
descriptions drive generalization. Second, prior work used text in the form of strategy advice to directly learn the policy. Since the policies are typically optimized for a specific task, they may be harder to transfer across domains. Instead, we utilize text to bootstrap the induction of the environment dynamics, moving beyond task-specific strategies. 

%Previous work has explored the use of text manuals in game playing, %ranging from constructing useful features by mining patterns in %text~\cite{eisenstein2009reading}, learning a semantic interpreter %with access to limited gameplay examples~\cite{goldwasser2014learning} %to learning through reinforcement from in-game %rewards~\cite{branavan2011learning}. These efforts have demonstrated %the usefulness of exploiting domain knowledge encoded in text to learn %effective policies. However, these methods use the text to infer %directly the best strategy to perform a task. In contrast, our goal is %to learn mappings from the text to the dynamics of an environment and %separate out the learning of the strategy/motives. 

Another related line of work consists of systems that learn spatial and topographical maps of the environment for robot navigation using natural language descriptions~\fullcite{walter2013learning,hemachandra2014learning}. These approaches use text mainly containing appearance and positional information, and integrate it with other semantic sources (such as appearance models) to obtain more accurate maps. In contrast, our work uses language describing the dynamics of the environment, such as entity movements and interactions, which 
is complementary to static positional information received through state observations. Further, our goal is to help an agent learn policies that generalize over different stochastic domains, while their works consider a single domain.

%karthik: I don't see the direct relevance
%Another line of work explores using textual interactive %environments~\cite{narasimhan2015language,he2016deep} to ground %language understanding into actions taken by the system in the %environment. In these tasks, understanding of language is crucial, %without which a system would not be able to take reasonable actions. %Our motivation is different -- we take an existing set of tasks and %domains which are amenable to learning through reinforcement, and %demonstrate how to utilize textual knowledge to learn faster and more %optimal policies in both multitask and transfer setups.

\subsection{Transfer in Reinforcement Learning}
Transferring policies across domains is a challenging problem in reinforcement learning~\fullcite{konidaris2006framework,taylor2009transfer}. The main hurdle lies in learning a good mapping between the state and action spaces of different domains to enable effective transfer. Most previous approaches have either explored skill transfer~\fullcite{konidaris2007building,konidaris2012transfer} or value function/policy transfer~\fullcite{liu2006value,taylor2007transfer,taylor2007cross}. There have also been attempts at model-based transfer for RL~\fullcite{taylor2008transferring,nguyen2012transferring,gavsic2013pomdp,wang2015learning,joshi2018cross} but these methods either rely on hand-coded inter-task mappings for state and actions spaces or require significant interactions in the target task to learn an effective mapping. Our approach doesn't use any explicit mappings and can learn to predict the dynamics of a target task using its descriptions.

% Work by \newcite{konidaris2006autonomous} look at knowledge transfer by learning a mapping from sensory signals to reward functions.

A closely related line of work concerns transfer methods for deep reinforcement learning. \citeA{parisotto2016actor}  train a deep network to mimic pre-trained experts on source tasks using policy distillation. The learned parameters are then used to initialize a network on a target task to perform transfer. Rusu et al.~\citeyear{rusu2016progressive} facilitate transfer by freezing parameters learned on source tasks and adding a new set of parameters for every new target task, while using both sets to learn the new policy. Work by Rajendran et al.~\citeyear{rajendran20172t} uses attention networks to selectively transfer from a set of expert policies to a new task. \textcolor{black}{Barreto et al.~\citeyear{barreto2017successor} use features based on successor representations~\fullcite{dayan1993improving} for transfer across tasks in the same domain. Kansky~et~al.~\citeyear{kansky2017schema} learn a generative model of causal physics in order to help zero-shot transfer learning.} Our approach is orthogonal to all these directions since we use text to bootstrap transfer, and can potentially be combined with these methods to achieve more effective transfer. 

\textcolor{black}{There has also been prior work on zero-shot policy generalization for tasks with input goal specifications. \fullciteA{schaul2015universal} learn a universal value function approximator that can generalize across both states and goals. \fullcite{andreas2016modular} use policy sketches, which are annotated sequences of subgoals, in order to learn a hierarchical policy that can generalize to new goals. \fullciteA{oh2017zero} investigate zero-shot transfer for instruction following tasks, aiming to generalize to unseen instructions in the same domain. The main departure of our work compared to these is in the use of environment descriptions for generalization across domains rather than generalizing across text instructions.}

Perhaps closest in spirit to our hypothesis is the recent work by~\fullcite{harrison2017guiding}. Their approach makes use of paired instances of text descriptions and state-action information from human gameplay to learn a machine translation model. This model is incorporated into a policy shaping algorithm to better guide agent exploration. Although the motivation of language-guided control policies is similar to ours, their work considers transfer across tasks in a single domain, and requires human demonstrations to learn a policy.

\textcolor{black}{
\subsection{Using Task Features for Transfer}
The idea of using task features/dictionaries for zero-shot generalization has been explored previously in the context of image classification. \fullciteA{kodirov2015unsupervised} learn a joint feature embedding space between domains and also induce the corresponding projections onto this space from different class labels. 
\fullciteA{kolouri2018joint} learn a joint dictionary across visual features and class attributes using sparse coding techniques. \fullciteA{romera2015embarrassingly} model the relationship between input features, task attributes and classes as a linear model to achieve efficient yet simple zero-shot transfer for classification. \fullciteA{socher2013zero} learn a joint semantic representation space for images and associated words to perform zero-shot transfer.}

\textcolor{black}{
Task descriptors have also been explored in zero-shot generalization for control policies. \fullciteA{sinapov2015learning} use task meta-data as features to learn a mapping between pairs of tasks. This mapping is then used to select the most relevant source task to transfer a policy from. \fullciteA{isele2016using} build on the ELLA framework~\fullcite{ruvolo2013ella,ammar2014online}, and their key idea is to maintain two shared linear bases across tasks -- one for the policy ($L$) and the other for task descriptors ($D$). Once these bases are learned on a set of source tasks, they can be used to predict policy parameters for a new task given its corresponding descriptor. 
% The training scheme is similar to Actor-mimic scheme~\cite{parisotto2016actor} -- for each task, an expert policy is trained separately and then distilled into policy parameters dependent on the shared basis $L$. 
In these lines of work, the task features were either manually engineered or directly taken from the underlying system parameters defining the dynamics of the environment. In contrast, our framework only requires access to crowd-sourced textual descriptions, alleviating the need for expert domain knowledge.}





% A major difference in our work is that we utilize natural language descriptions of different environments to bootstrap transfer, requiring less exploration in the new task.

% using a policy distillation~\cite{parisotto2016actor,rusu2016progressive,yin2017knowledge} or selective attention over expert networks learnt in the source tasks~\cite{rajendran20172t}. Though these approaches provide some benefits, they still suffer from the requirement of efficiently exploring the new environment to learn how to transfer their existing policies. In contrast, we utilize natural language descriptions of different environments to bootstrap transfer, leading to more focused exploration in the target task. 


% Describe amn in detail




 \label{sec:related}


%\section{Research Problem and Goals} \section{Background and Motivation}

\subsection{IBM Streams}

IBM Streams is a general-purpose, distributed stream processing system. It
allows users to develop, deploy and manage long-running streaming applications
which require high-throughput and low-latency online processing.

The IBM Streams platform grew out of the research work on the Stream Processing
Core~\cite{spc-2006}.  While the platform has changed significantly since then,
that work established the general architecture that Streams still follows today:
job, resource and graph topology management in centralized services; processing
elements (PEs) which contain user code, distributed across all hosts,
communicating over typed input and output ports; brokers publish-subscribe
communication between jobs; and host controllers on each host which
launch PEs on behalf of the platform.

The modern Streams platform approaches general-purpose cluster management, as
shown in Figure~\ref{fig:streams_v4_v6}. The responsibilities of the platform
services include all job and PE life cycle management; domain name resolution
between the PEs; all metrics collection and reporting; host and resource
management; authentication and authorization; and all log collection. The
platform relies on ZooKeeper~\cite{zookeeper} for consistent, durable metadata
storage which it uses for fault tolerance.

Developers write Streams applications in SPL~\cite{spl-2017} which is a
programming language that presents streams, operators and tuples as
abstractions. Operators continuously consume and produce tuples over streams.
SPL allows programmers to write custom logic in their operators, and to invoke
operators from existing toolkits. Compiled SPL applications become archives that
contain: shared libraries for the operators; graph topology metadata which tells
both the platform and the SPL runtime how to connect those operators; and
external dependencies. At runtime, PEs contain one or more operators. Operators
inside of the same PE communicate through function calls or queues. Operators
that run in different PEs communicate over TCP connections that the PEs
establish at startup. PEs learn what operators they contain, and how to connect
to operators in other PEs, at startup from the graph topology metadata provided
by the platform.

We use ``legacy Streams'' to refer to the IBM Streams version 4 family. The
version 5 family is for Kubernetes, but is not cloud native. It uses the
lift-and-shift approach and creates a platform-within-a-platform: it deploys a
containerized version of the legacy Streams platform within Kubernetes.

\subsection{Kubernetes}

Borg~\cite{borg-2015} is a cluster management platform used internally at Google
to schedule, maintain and monitor the applications their internal infrastructure
and external applications depend on. Kubernetes~\cite{kube} is the open-source
successor to Borg that is an industry standard cloud orchestration platform.

From a user's perspective, Kubernetes abstracts running a distributed
application on a cluster of machines. Users package their applications into
containers and deploy those containers to Kubernetes, which runs those
containers in \emph{pods}. Kubernetes handles all life cycle management of pods,
including scheduling, restarting and migration in case of failures.

Internally, Kubernetes tracks all entities as \emph{objects}~\cite{kubeobjects}.
All objects have a name and a specification that describes its desired state.
Kubernetes stores objects in etcd~\cite{etcd}, making them persistent,
highly-available and reliably accessible across the cluster. Objects are exposed
to users through \emph{resources}. All resources can have
\emph{controllers}~\cite{kubecontrollers}, which react to changes in resources.
For example, when a user changes the number of replicas in a
\code{ReplicaSet}, it is the \code{ReplicaSet} controller which makes sure the
desired number of pods are running. Users can extend Kubernetes through
\emph{custom resource definitions} (CRDs)~\cite{kubecrd}. CRDs can contain
arbitrary content, and controllers for a CRD can take any kind of action.

Architecturally, a Kubernetes cluster consists of nodes. Each node runs a
\emph{kubelet} which receives pod creation requests and makes sure that the
requisite containers are running on that node. Nodes also run a
\emph{kube-proxy} which maintains the network rules for that node on behalf of
the pods. The \emph{kube-api-server} is the central point of contact: it
receives API requests, stores objects in etcd, asks the scheduler to schedule
pods, and talks to the kubelets and kube-proxies on each node. Finally,
\emph{namespaces} logically partition the cluster. Objects which should not know
about each other live in separate namespaces, which allows them to share the
same physical infrastructure without interference.

\subsection{Motivation}
\label{sec:motivation}

Systems like Kubernetes are commonly called ``container orchestration''
platforms. We find that characterization reductive to the point of being
misleading; no one would describe operating systems as ``binary executable
orchestration.'' We adopt the idea from Verma et al.~\cite{borg-2015} that
systems like Kubernetes are ``the kernel of a distributed system.'' Through CRDs
and their controllers, Kubernetes provides state-as-a-service in a distributed
system. Architectures like the one we propose are the result of taking that view 
seriously.

The Streams legacy platform has obvious parallels to the Kubernetes
architecture, and that is not a coincidence: they solve similar problems.
Both are designed to abstract running arbitrary user-code across a distributed
system.  We suspect that Streams is not unique, and that there are many
non-trivial platforms which have to provide similar levels of cluster
management.  The benefits to being cloud native and offloading the platform
to an existing cloud management system are: 
\begin{itemize}
    \item Significantly less platform code.
    \item Better scheduling and resource management, as all services on the cluster are 
        scheduled by one platform.
    \item Easier service integration.
    \item Standardized management, logging and metrics.
\end{itemize}
The rest of this paper presents the design of replacing the legacy Streams 
platform with Kubernetes itself.

 \label{sec:background}

\section{Methodology} \label{sec:methodology} \section{Evaluation}
\label{sec:eval}

This section evaluates the performance, area and power of the \ZF architecture demonstrating how it improves over the state-of-the-art  DaDianNao accelerator~\cite{DaDiannao}. 
Section~\ref{sec:eval:method} details the experimental methodology. 
Section~\ref{sec:eval:performance} evaluates the performance of \ZF. 
Sections~\ref{sec:eval:area} and \ref{sec:eval:power} evaluate the area and power of \ZF, and Section~\ref{sec:add-ineffectual} considers the removal of non-zero neurons.

%
%

%

\subsection{Methodology}
\label{sec:eval:method}


%
%
\begin{table}[t!]
\centering
\begin{tabular}{|l|l|l|}
\hline
\textbf{Network} & \pbox{5cm}{\textbf{Conv.} \\ \textbf{Layers}} & \textbf{Source} \\ \hline \hline
alex     	 & 5 & Caffe: bvlc\_reference\_caffenet \\ \hline
google 	 & 59 & Caffe: bvlc\_googlenet \\ \hline
nin 		 & 12 & Model Zoo: NIN-imagenet \\ \hline
vgg19  	 & 16 & Model Zoo: VGG 19-layer \\ \hline
cnnM  & 5 & Model Zoo: VGG\_CNN\_M\_2048 \\ \hline
cnnS  & 5 & Model Zoo: VGG\_CNN\_S \\ \hline
\end{tabular}
\caption{Networks used}
\label{table:networks}
\end{table}

The evaluation uses the set of popular~\cite{AlexNIPS2012}, and state-of-the-art convolutional neural networks~\cite{ILSVRC15}\cite{nin}\cite{vgg}\cite{vgg19} shown in Table \ref{table:networks}. 
These networks perform image classification on the ILSVRC12 dataset~\cite{ILSVRC15}, which contains $256\times256$ images across 1000 classes. 
The experiments use a randomly selected set of 1000 images, one from each class. The networks are available, pre-trained for Caffe, either as part of the distribution or at the Caffe Model Zoo~\cite{model-zoo}.

%
%
We created a cycle accurate simulator of the baseline accelerator and \ZF. 
The simulator integrates with the Caffe framework~\cite{caffe} to enable on-the-fly validation of the layer ouput neurons. 
%
%
%
%
The area and power characteristics of \ZF and \BASE are measured 
with synthesized implementations. The two designs are implemented 
in Verilog and synthesized via the Synopsis Design 
Compiler~\cite{synopsys_site} with the TSMC 65nm library. 
The NBin, NBout, and \ZF offset SRAM buffers were modeled using 
the Artisan single-ported register file memory compiler~\cite{artisan} 
using double-pumping to allow a read and write per cycle. The eDRAM 
area and energy was modeled with \textit{Destiny}~\cite{destiny}.
%



%
%
%
%
%

%
%

%
%
%
%

%
%
%
%
%
%
%
%
%
%
%
%
%
 

% Polishing stopped here.
\section{Evaluation} \section{Evaluation}
\label{sec:evaluation}
\begin{table*}[!t]
\begin{center}
%\small
\caption {Benchmarks and applications for the study of the application-level resilience}
\vspace{-5pt}
\label{tab:benchmark}
\tiny
\begin{tabular}{|p{1.7cm}|p{7.5cm}|p{4cm}|p{2.5cm}|}
\hline
\textbf{Name} 	& \textbf{Benchmark description} 		& \textbf{Execution phase for evaluation}  			& \textbf{Target data objects}             \\ \hline \hline
CG (NPB)             & Conjugate Gradient, irregular memory access (input class S)   & The routine conj\_grad in the main computation loop  & The arrays $r$ and $colidx$     \\\hline
MG (NPB)    	       & Multi-Grid on a sequence of meshes (input class S)             & The routine mg3P in the main computation loop & The arrays $u$ and $r$ 	\\ \hline
FT (NPB)             & Discrete 3D fast Fourier Transform (input class S)            & The routine fftXYZ in the main computation loop  & The arrays $plane$ and $exp1$    \\ \hline
BT (NPB)             & Block Tri-diagonal solver (input class S)         		& The routine x\_solve in the main computation loop & The arrays $grid\_points$ and $u$	\\ \hline
SP (NPB)             & Scalar Penta-diagonal solver (input class S)         		& The routine x\_solve in the main computation loop & The arrays $rhoi$ and $grid\_points$  \\ \hline
LU (NPB)            & Lower-Upper Gauss-Seidel solver (input class S)        	& The routine ssor 	& The arrays $u$ and $rsd$ \\ \hline \hline
LULESH~\cite{IPDPS13:LULESH} & Unstructured Lagrangian explicit shock hydrodynamics (input 5x5x5) & 
The routine CalcMonotonicQRegionForElems 
& The arrays $m\_elemBC$ and $m\_delv\_zeta$ \\ \hline
AMG2013~\cite{anm02:amg} & An algebraic multigrid solver for linear systems arising from problems on unstructured grids (we use  GMRES(10) with AMG preconditioner). We use a compact version from LLNL with input matrix $aniso$. & The routine hypre\_GMRESSolve & The arrays $ipiv$ and $A$   \\ \hline
%$hierarchy.levels[0].R.V$ \\ \hline
\end{tabular}
\end{center}
\vspace{-5pt}
\end{table*}

%We evaluate the effectiveness of ARAT, and 
%We use ARAT to study the application-level resilience.
%The goal is to demonstrate 
%that aDVF can be a very useful metric to quantify the resilience of data objects
%at the application level. 
We study 12 data objects from six benchmarks of the NAS parallel benchmark (NPB) suite (we use SNU\_NPB-1.0.3) and 4 data objects from two scientific applications. 
%which is a c version of NPB 3.3, but ARAT can work for Fortran.
Those data objects are chosen to be representative: they have various data access patterns and participate in various execution phases.  
%For the benchmarks, we use CLASS S as the input problems and use the default compiler options of NPB.
For those benchmarks and applications, we use their default compiler options, and use gcc 4.7.3 and LLVM 3.4.2 for trace generation.
To count the algorithm-level fault masking, we use the default convergence thresholds (or the fault tolerance levels) for those benchmarks.
Table~\ref{tab:benchmark} gives 
%for->on by anzheng
detailed information on the benchmarks and applications.
The maximum fault propagation path for aDVF analysis is set to 10 by default.
%the value shadowing threshold is set as 0.01 (except for BT, we use $1 \times 10^{-6}$).
%These value shadowing thresholds are chosen such that any error corruption
%that results in the operand's value variance less than 1\% (for the threshold 0.01) or 0.0001\% (for the threshold $1 \times 10^{-6}$) during the 
%trace analysis does not impact the outcome correctness of six benchmarks.
%LU: check the newton-iteration residuals against the tolerance levels
%SP: check the newton-iteration residuals against the tolerance levels
%BT: check the newton-iteration residuals against the tolerance levels

\subsection{Resilience Modeling Results}
%We use ARAT to calculate aDVF values of 16 data objects. 
Figure~\ref{fig:aDVF_3tiers_profiling}
shows the aDVF results and breaks them down into the three levels 
(i.e., the operation-level, fault propagation level, and algorithm-level).
Figure~\ref{fig:aDVF_3classes_profiling} shows the 
%for->of by anzheng
results for the analyses at the levels of the operation and fault propagation,
and further breaks down the results into 
the three classes (i.e., the value overwriting, logical and comparison operations,
and value shadowing). %based on the reasons of the fault masking.
We have multiple interesting findings from the results.

\begin{figure*}
	\centering
        \includegraphics[width=0.8\textwidth]{three_tiers_gray.pdf}
% * <azguolu@gmail.com> 2017-03-23T03:20:28.808Z:
%
% ^.
        \vspace{-5pt}
        \caption{The breakdown of aDVF results based on the three level analysis. The $x$ axis is the data object name.}
        \vspace{-8pt}
        \label{fig:aDVF_3tiers_profiling}
\end{figure*}


\begin{figure*}
	\centering
	\includegraphics[width=0.8\textwidth]{three_types_gray.pdf}
	\vspace{-5pt}
	\caption{The breakdown of aDVF results based on the three classes of fault masking. The $x$ axis is the data object name. \textit{zeta} and \textit{elemBC} in LULESH are \textit{m\_delv\_zeta} and \textit{m\_elemBC} respectively.} % Anzheng
	\vspace{-5pt}
	\label{fig:aDVF_3classes_profiling}
    %\vspace{-5pt}
\end{figure*}

(1) Fault masking is common across benchmarks and applications.
Several data objects (e.g., $r$ in CG, and $exp1$ and $plane$ in FT)
have aDVF values close to 1 in Figure~\ref{fig:aDVF_3tiers_profiling}, 
which indicates that most of operations working on these data objects
have fault masking.
However, a couple of data objects have much less intensive fault masking.
For example, the aDVF value of $colidx$ in CG is 0.28 (Figure~\ref{fig:aDVF_3tiers_profiling}). 
Further study reveals that $colidx$ is an array to store column indexes of sparse matrices, and there is few operation-level or fault propagation-level fault masking  (Figure~\ref{fig:aDVF_3classes_profiling}).
The corruption of it can easily cause segmentation fault caught by the
algorithm-level analysis. 
$grid\_points$ in SP and BT also have a relatively small aDVF value (0.14 and 0.38 for SP and BT respectively in Figure~\ref{fig:aDVF_3tiers_profiling}).
Further study reveals that $grid\_points$ defines input problems for SP and BT. 
A small corruption of $grid\_points$ 
%change->changes by anzheng
can easily cause major changes in computation
caught by the fault propagation analysis. 

The data object $u$ in BT also has a relatively small aDVF value (0.82 in Figure~\ref{fig:aDVF_3tiers_profiling}).
Further study reveals that $u$ is read-only in our target code region
for matrix factorization and Jacobian, neither of which is friendly
for fault masking.
Furthermore, the major fault masking for $u$ comes from value shadowing,
and value shadowing only happens in a couple of the least significant bits 
of the operands that reference $u$, which further reduces the value of aDVF.
%also reduces fault masking.

(2) The data type is strongly correlated with fault masking.
Figure~\ref{fig:aDVF_3tiers_profiling} reveals that the integer data objects ($colidx$ in CG, $grid\_points$ in BT and SP, $m\_elemBC$ in LULESH) appear to be 
more sensitive to faults than the floating point data objects 
($u$ and $r$ in MG, $exp1$ and $plane$ in FT, $u$ and $rsd$ in LU, $m\_delv\_zeta$ in LULESH, and $rhoi$ in SP).
In HPC applications, the integer data objects are commonly employed to
define input problems and bound computation boundaries (e.g., $colidx$ in CG and $grid\_points$ in BT), 
or track computation status (e.g., $m\_elemBC$ in LULESH). Their corruption 
%these integer data objects
is very detrimental to the application correctness. 

(3) Operation-level fault masking is very common.
For many data objects, the operation-level fault masking contributes 
more than 70\% of the aDVF values. For $r$ in CG, $exp1$ in FT, and $rhoi$ in SP,
the contribution of the operation-level fault masking is close to 99\% (Figure~\ref{fig:aDVF_3tiers_profiling}).

Furthermore, the value shadowing is a very common operation level fault masking,
especially for floating point data objects (e.g., $u$ and $r$ in BT, $m\_delv\_zeta$ in LULESH, and $rhoi$ in SP in Figure~\ref{fig:aDVF_3classes_profiling}).
This finding has a very important indication for studying the application resilience.
In particular, the values of a data object can be different across different input problems. If the values of the data object are different, 
then the number of fault masking events due to the value shadowing will be different. 
Hence, we deduce that the application resilience
can be correlated with the input problems,
because of the correlation between the value shadowing and input problems. 
We must consider the input problems when studying the application resilience.
This conclusion is consistent with a very recent work~\cite{sc16:guo}.

(4) The contribution of the algorithm-level fault masking to the application resilience can be nontrivial.
For example, the algorithm-level fault masking contributes 19\% of the aDVF value for $u$ in MG and 27\% for $plane$ in FT (Figure~\ref{fig:aDVF_3tiers_profiling}).
The large contribution of algorithm-level fault masking in MG is consistent with
the results of existing work~\cite{mg_ics12}. 
For FT (particularly 3D FFT), the large contribution of algorithm-level fault masking in $plane$ (Figure~\ref{fig:aDVF_3tiers_profiling})
comes from frequent transpose and 1D FFT computations that average out 
or overwrite the data corruption.
CG, as an iterative solver, is known to have the algorithm-level fault masking
because of the iterative nature~\cite{2-shantharam2011characterizing}.
Interestingly, the algorithm-level fault masking in CG contributes most to the resilience of $colidx$ which is a vulnerable integer data object (Figure~\ref{fig:aDVF_3tiers_profiling}).

%Our study reveals the algorithm-level fault masking of CG from
%two perspectives. First, $a$ in CG, which is an array for intermediate results,
%has few algorithm-level fault masking (0.008\%);
%Second, $x$ in CG, which is a result vector, has 5.4\% of the aDVF value coming from the algorithm-level fault masking.
%This result indicates that the effects of the algorithm-level fault masking
%are not uniform across data objects. 

(5) Fault masking at the fault propagation level is small.
For all data objects, the contribution of the fault masking at the level of fault propagation is less than 5\% (Figure~\ref{fig:aDVF_3tiers_profiling}).
For 6 data objects ($r$ and $colidx$ in CG, $grid\_points$ and $u$ in BT, and 
$grid\_points$ and $rhoi$ in SP),  there is no fault masking at the level of fault propagation.
In combination with the finding 4, we conclude that once the fault
is propagated, it is difficult to mask it because of the contamination of
more data objects after fault propagation, and only the algorithm semantics can tolerate  propagated faults well. 
%This finding is consistent with our sensitivity analysis. 

(6) Fault masking by logical and comparison operations is small,
%For all data objects, the fault masking contributions due to logical and comparison operations are very small, 
comparing with the contributions of value shadowing and overwriting (Figure~\ref{fig:aDVF_3classes_profiling}). 
Among all data objects, 
the logical and comparison operations in $grid\_points$ in BT contribute the most (25\% contribution in Figure~\ref{fig:aDVF_fine_profiling}), 
because of intensive ICmp operations (integer comparison). %logical OR and SHL (left shifting).


(7) The resilience varies across data objects. %within the same application.
This fact is especially pronounced in two data objects $colidx$ and $r$ in CG (Figure~\ref{fig:aDVF_3tiers_profiling}).
 $colidx$ has aDVF much smaller than $r$, which means $colidx$ is much less resilient than $r$ (see finding 1 for a detailed analysis on $colidx$). 
Furthermore, $colidx$ and $r$ have different algorithm-level
fault masking (see finding 4 for a detailed analysis).

\begin{comment}
\textbf{Finding 7: The resilience of the same data objects varies across different applications.}
This fact is especially pronounced in BT and SP.
BT and SP address the same numerical problem but with different algorithms.
BT and SP have the same data objects, $qs$ and $rhoi$, but
$qs$ manifests different resilience in BT and SP.
This result is interesting, because it indicates that by using
different algorithms, we have opportunities to
improve the resilience of data objects.
\end{comment}

To further investigate the reasons for fault masking, 
we break down the aDVF results at the granularity of LLVM instructions,
based on the analyses at the levels of operation and fault propagation.
The results are shown in Figure~\ref{fig:aDVF_fine_profiling}.
%Because of the space limitation, 
%we only show one data object per benchmark, but each selected data object has the most diverse fault masking events within the corresponding benchmark.
%Based on Figure~\ref{fig:aDVF_fine_profiling}, we have another interesting finding.

(8) Arithmetic operations make a lot of contributions to fault masking.
%For $r$ in CG, $r$ in MG, $exp1$ in FT, $u$ in BT, $qs$ in SP, and $u$ in LU,
%the arithmetic operations, FMul (100\%), Add (16\%), FMul (85\%), 
%FMul (94\%), FMul (28\%), and FAdd (50\%)
For $r$ in CG, $u$ in BT, $plane$ and $exp1$ in FT, $m\_elemBC$ in LULESH, 
arithmetic operations (addition, multiplication, and division) contribute to almost 100\% of the fault masking (Figure~\ref{fig:aDVF_fine_profiling}).  
%(at the operation level and the fault propagation level).
%For $qs$ in SP and $u$ in LU, the store operation also makes
%important contributions as the arithmetic operations because of value overwriting.

\begin{figure*}
	\centering
	\includegraphics[width=0.77\textheight, height=0.23\textheight]{pie_chart.pdf}
	\vspace{-10pt}
	\caption{Breakdown of the aDVF results based on the analyses at the levels of operation and fault propagation}
    \vspace{-10pt}
	\label{fig:aDVF_fine_profiling}
\end{figure*}


\subsection{Sensitivity Study}
\label{sec:eval_sen}
%\textbf{change the fault propagation threshold and study the sensitivity of analysis to the threshold}
ARAT uses 10 as the default fault propagation analysis threshold. 
The fault propagation analysis will not go beyond 10 operations. Instead,
we will use deterministic fault injection after 10 operations. 
In this section, we study the impact of this threshold on the modeling accuracy. We use a range of threshold values and examine how the aDVF value varies and whether
the identification of fault masking varies. 
Figure~\ref{fig:sensitivity_error_propagation} shows the results for 
%add , after BT by anzheng
multiple data objects in CG, BT, and SP.
We perform the sensitivity study for all 16 data objects.
%in six benchmarks and two applications.
Due to the page space limitation, we only show the results for three data objects,
but we summarize the sensitivity study results for all data objects in this section.
%but other data objects in all benchmarks have the same trend.

Our results reveal that the identification of fault masking by tracking fault propagation is not significantly 
affected by the fault propagation analysis threshold. Even if we use a rather large threshold (50), 
the variation of aDVF values is 4.48\% on average among all data objects,
and the variation at each of the three levels of analysis (the operation level, fault propagation level,  and algorithm level) is less than 5.2\% on average. 
In fact, using a threshold value of 5 is sufficiently accurate in most of the cases (14 out of 16 data objects).
This result is consistent with our finding 5 (i.e., fault masking at the fault propagation level is small). %in most benchmarks).
However, we do find a data object ($m\_elementBC$ in LULESH) %and $exp1$ in FT) 
showing relatively high-sensitive (up to 15\% variation) to the threshold. For this uncommon data object, using 50 as the fault propagation path is sufficient. 

%In other words, even though using a larger threshold value can identify more error masking by tracking error 
%propagation, the implicit error masking induced by the error propagation is very limited.

\begin{figure}
		\begin{center}
		\includegraphics[width=0.48\textwidth,height=0.11\textheight]{sensi_study_gray.pdf}
		\vspace{-15pt}
		\caption{Sensitivity study for fault propagation threshold}
		\label{fig:sensitivity_error_propagation}
		\end{center}
\vspace{-15pt}
\end{figure}


\begin{comment}
\subsection{Comparison with the Traditional Random Fault Injection}
%\textbf{compare with the traditional fault injection to verify accuracy}
To show the effectiveness of our resilience modeling, we compare traditional random fault injection
and our analytical modeling. Figure~\ref{fig:comparison_fi} and Table~\ref{tab:comparison} show the results.
The figure shows the success rate of all random fault injection. The ``success'' means the application
outcome is verified successfully by the benchmarks and the execution does not have any segfault. The success rate is used as a metric
to evaluate the application resilience.

We use a data-oriented approach to perform random fault injection.
In particular, given a data object, for each fault injection test we trigger a bit flip
in an operand of a random instruction, and this operand must be a reference to the
target data object. We develop a tool based on PIN~\cite{pintool} to implement the above fault injection functionality.
For each data object, we conduct five sets of random fault injection tests, 
and each set has 200 tests (in total 1000 tests per data object). 
We show the results for CG and FT in this section, but we find that
the conclusions we draw from CG and FT are also valid for the other four benchmarks.


%\begin{table*}
%\label{tab:success_rate}
%\begin{centering}
%\renewcommand\arraystretch{1.1}
%\begin{tabular}{|c|c|c|c|c|c|c|}
%\hline 
%Success Rate (Difference) & Test set 1 & Test set 2 & Test set 3 & Test set 4 & Test set 5 & Average\tabularnewline
%\hline 
%\hline 
%CG-a & 66.1\% (11.7\%) & 68.5\% (15.7\%) & 56.7\% (4.21\%) & 61.3\% (3.57\%) & 43.3\% (26.8\%) & 59.2\%\tabularnewline
%\hline 
%CG-x & 99.2\% (2.2\%) & 98.6\% (1.5\%) & 96.5\% (0.63\%) & 97.8\% (0.64\%) & 93.6\% (3.7\%) & 97.1\%\tabularnewline
%\hline 
%CG-colidx & 36.8\% (12.7\%) & 49.6\% (17.8\%) & 40.2\% (4.6\%) & 52.6\% (24.9\%) & 31.4\% (25.4\%) & 42.1\%\tabularnewline
%\hline 
%FT-exp1 & 52.7\% (1.4\%) & 22.6\% (56.5\%) & 78.5\% (51.0\%) & 60.7\% (16.7\%) & 45.4\% (12.7\%) & 51.9\%\tabularnewline
%\hline 
%FT-plane & 82.1\% (2.5\%) & 79.3\% (5.6\%) & 99.5\% (18.2\%) & 93.2\% (10.7\%) & 66.8\% (20.6\%) & 84.2\%\tabularnewline
%\hline 
%\end{tabular}
%\par\end{centering}
%\caption{XXXXX}
%\end{table*}


\begin{table*}
\begin{centering}
\caption{\small The results for random fault injection. The numbers in parentheses for each set of tests (200 tests per set) are the success rate difference from the average success rate of 1000 fault injection tests.}
\label{tab:comparison}
\renewcommand\arraystretch{1.1}
\begin{tabular}{|c|p{2.2cm}|p{2.2cm}|p{2.2cm}|p{2.2cm}|p{2.2cm}|p{1.8cm}|}
\hline 
       %& Test set 1 & Test set 2 & Test set 3 & Test set 4 & Test set 5 & Average\tabularnewline
       & \hspace{13pt} Test set 1 \hspace{1pt}/  & \hspace{13pt} Test set 2 \hspace{1pt}/ & \hspace{13pt} Test set 3 \hspace{1pt}/ & \hspace{13pt} Test set 4 \hspace{1pt}/ & \hspace{13pt} Test set 5 \hspace{1pt}/ & Ave. of all test / \\
       & success rate (diff.) & success rate (diff.) & success rate (diff.) & success rate (diff.) & success rate (diff.) & \hspace{5pt} success rate \\
\hline 
\hline 
CG-a & 66.1\% (6.9\%) & 68.5\% (9.3\%) & 56.7\% (-2.5\%) & 61.3\% (2.1\%) & 43.3\% (-15.9\%) & 59.2\%\tabularnewline
\hline 
CG-x & 99.2\% (2.1\%) & 98.6\% (1.5\%) & 96.5\% (-0.6\%) & 97.8\% (0.7\%) & 93.6\% (-3.5\%) & 97.1\%\tabularnewline
\hline 
CG-colidx & 36.8\% (-5.3\%) & 49.6\% (7.5\%) & 40.2\% (-2.0\%) & 52.6\% (10.5\%) & 31.4\% (-10.7\%) & 42.1\%\tabularnewline
\hline 
FT-exp1 & 52.7\% (0.8\%) & 22.6\% (-29.3\%) & 78.5\% (26.6\%) & 60.7\% (8.8\%) & 45.4\% (-6.5\%) & 51.9\%\tabularnewline
\hline 
FT-plane & 82.1\% (-2.1\%) & 79.3\% (-4.9\%) & 99.5\% (15.3\%) & 93.2\% (9.0\%) & 66.8\% (-17.4\%) & 84.2\%\tabularnewline
\hline 
\end{tabular}
\par\end{centering}
\vspace{-0.4cm}
\end{table*}

\begin{figure}
	\begin{center}
		\includegraphics[width=0.48\textwidth,keepaspectratio]{verifi-study.png}
		\caption{The traditional random fault injection vs. ARAT}
		\label{fig:comparison_fi}
	\end{center}
\vspace{-0.7cm}
\end{figure}


We first notice from Table~\ref{tab:comparison} that 
%across 5 sets of random fault injection tests, there are big variances (up to 55.9\% in $exp1$ of FT) in terms of the success rate. 
the results of 5 test sets can be quite different from each other and from 1000 random fault inject tests (up to 29.3\%).
1000 fault injection tests provide better statistical significance than 200 fault injection tests.
We expect 1000 fault injection tests potentially provide higher accuracy to quantify the application resilience.
The above result difference is clearly an indication to the randomness of fault injection, and there
is no guarantee on the random fault injection accuracy.

%In Figure~\ref{fig:comparison_fi}, 
We compare the success rate of 1000 fault inject tests with the aDVF value (Fig.~\ref{fig:comparison_fi}). 
We find that the order of the success rate of the three data objects in CG (i.e., $colidx < a < x$) and the two data objects in FT 
(i.e., $exp1 < plane$) is the same as the order of the aDVF values of these data objects. 
%In fact, 1000 fault injection tests
%account for \textcolor{blue}{\textbf{xxx\%}} of total memory references to the data object,
%and provide better resilience quantification than 200 fault injection tests.
The same order (or the same resilience trend)
%between our approach and the random fault injection based on a large number of tests 
is a demonstration of the effectiveness of our approach.
Note that the values of the aDVF and success rate %for a data object
cannot be exactly the same (even if we have sufficiently large numbers of random fault injection), 
because aDVF and random fault injection quantify
the resilience based on different metrics.
Also, the random fault injection can miss some fault masking events that can be captured by our approach.

\end{comment} \label{sec:evaluation}

\section{Conclusion} % \vspace{-0.5em}
\section{Conclusion}
% \vspace{-0.5em}
Recent advances in multimodal single-cell technology have enabled the simultaneous profiling of the transcriptome alongside other cellular modalities, leading to an increase in the availability of multimodal single-cell data. In this paper, we present \method{}, a multimodal transformer model for single-cell surface protein abundance from gene expression measurements. We combined the data with prior biological interaction knowledge from the STRING database into a richly connected heterogeneous graph and leveraged the transformer architectures to learn an accurate mapping between gene expression and surface protein abundance. Remarkably, \method{} achieves superior and more stable performance than other baselines on both 2021 and 2022 NeurIPS single-cell datasets.

\noindent\textbf{Future Work.}
% Our work is an extension of the model we implemented in the NeurIPS 2022 competition. 
Our framework of multimodal transformers with the cross-modality heterogeneous graph goes far beyond the specific downstream task of modality prediction, and there are lots of potentials to be further explored. Our graph contains three types of nodes. While the cell embeddings are used for predictions, the remaining protein embeddings and gene embeddings may be further interpreted for other tasks. The similarities between proteins may show data-specific protein-protein relationships, while the attention matrix of the gene transformer may help to identify marker genes of each cell type. Additionally, we may achieve gene interaction prediction using the attention mechanism.
% under adequate regulations. 
% We expect \method{} to be capable of much more than just modality prediction. Note that currently, we fuse information from different transformers with message-passing GNNs. 
To extend more on transformers, a potential next step is implementing cross-attention cross-modalities. Ideally, all three types of nodes, namely genes, proteins, and cells, would be jointly modeled using a large transformer that includes specific regulations for each modality. 

% insight of protein and gene embedding (diff task)

% all in one transformer

% \noindent\textbf{Limitations and future work}
% Despite the noticeable performance improvement by utilizing transformers with the cross-modality heterogeneous graph, there are still bottlenecks in the current settings. To begin with, we noticed that the performance variations of all methods are consistently higher in the ``CITE'' dataset compared to the ``GEX2ADT'' dataset. We hypothesized that the increased variability in ``CITE'' was due to both less number of training samples (43k vs. 66k cells) and a significantly more number of testing samples used (28k vs. 1k cells). One straightforward solution to alleviate the high variation issue is to include more training samples, which is not always possible given the training data availability. Nevertheless, publicly available single-cell datasets have been accumulated over the past decades and are still being collected on an ever-increasing scale. Taking advantage of these large-scale atlases is the key to a more stable and well-performing model, as some of the intra-cell variations could be common across different datasets. For example, reference-based methods are commonly used to identify the cell identity of a single cell, or cell-type compositions of a mixture of cells. (other examples for pretrained, e.g., scbert)


%\noindent\textbf{Future work.}
% Our work is an extension of the model we implemented in the NeurIPS 2022 competition. Now our framework of multimodal transformers with the cross-modality heterogeneous graph goes far beyond the specific downstream task of modality prediction, and there are lots of potentials to be further explored. Our graph contains three types of nodes. while the cell embeddings are used for predictions, the remaining protein embeddings and gene embeddings may be further interpreted for other tasks. The similarities between proteins may show data-specific protein-protein relationships, while the attention matrix of the gene transformer may help to identify marker genes of each cell type. Additionally, we may achieve gene interaction prediction using the attention mechanism under adequate regulations. We expect \method{} to be capable of much more than just modality prediction. Note that currently, we fuse information from different transformers with message-passing GNNs. To extend more on transformers, a potential next step is implementing cross-attention cross-modalities. Ideally, all three types of nodes, namely genes, proteins, and cells, would be jointly modeled using a large transformer that includes specific regulations for each modality. The self-attention within each modality would reconstruct the prior interaction network, while the cross-attention between modalities would be supervised by the data observations. Then, The attention matrix will provide insights into all the internal interactions and cross-relationships. With the linearized transformer, this idea would be both practical and versatile.

% \begin{acks}
% This research is supported by the National Science Foundation (NSF) and Johnson \& Johnson.
% \end{acks} \label{sec:conclusion}

%\section*{Acknowledgment}
%The authors would like to acknowledge the National Science Foundation (NSF) for partially sponsoring the work under grants \# 2236280, \#1633978, \#1620871, \#1620862, and \#1636622, and BBN/GPO project \#1936 through an NSF/CNS grant. We also thank the Florida Center for Cybersecurity (Cyber Florida) for a seed grant. The views and conclusions contained herein are those of the authors and should not be interpreted as necessarily representing the official policies, either expressed or implied by NSF.

%\section*{References}
\bibliographystyle{IEEEtran}
\bibliography{refs}


%\section{Follow-up Experiments} \input{Follow-up}

%\section{Comments}
%\textcolor{red}{
%\begin{itemize}
    %\item Page 4: you only show 4 models, but initially you had 6 models.
    %\textcolor{green}{EVANS: COULD NOT FIND THIS}
    %\item Formula 1 is incomplete, missing key details.
    %\item Page 5, Fig 1: unclear. Please elaborate on it. The first five scoring systems are almost the same, you probably have an error there. If there is no error, this complete coincidence might be an issue. \\
    %\textcolor{green}{EVANS: I have done several experiments and research and it turns out that the models are giving full probability or confidence for almost all cases. This is a good thing because it tells us that the models are very confident in their predictions. However, for the purpose of what we are doing here, I am not fully sure if we still want to proceed with probability scores, due to the extremely large score ties.}
    %\\We have 6 models, but only two of them can be seen on the figure. \\ \textcolor{green}{EVANS: COULD NOT FIND THIS}
    %\item Double-check the weighted combination formulas.
    %\item Is it possible to measure the weighted combination by performance?
    %\item Page 5, rank combination vs. score combination: Based on our understanding, the smaller the better the rank is because we take the inverse of the diversity strength. \\
    %\textcolor{green}{EVANS: Right, we are already doing this}
    %\item Table 3: suspicious, Must be investigated. \\ 
    %\textcolor{green}{EVANS: TABLE 3 SHOULD BE CORRECT.}
    %\item Tables 3 and 4: We may need performance evaluation instead. \\
    %\textcolor{green}{EVANS: TABLE 3 AND 4 ARE EXACTLY FOR PERFORMANCE EVALUATION}
    %\item How did you calculate the precision of each class? Please clarify this. \\
    %\textcolor{green}{EVANS: This called is the performance evaluation of a multi class classification problem. I can create a section and provide insights into this}
    %\item Figs 2 and 3: What do you mean by "under 20 results"? and what are you trying to show? \\
    %\textcolor{green}{EVANS: Those are the first 20 data items. I have changed the captions to reflect this}
    %\item Table 5: Is it a summary of Tables 3 and 4? \\
    %\textcolor{green}{EVANS: Table 5 is a summary of Tables 2 and 3. We can keep either of them.} 
    %\item Table 6: What is it for? It's unclear. Is it the performance evaluation of the score combination of all models?
    %\textcolor{green}{EVANS: Yes. that is the performance evaluation of the score combination of all models} 
    %\item The evaluation does NOT say what the top intruders are/it does not tell the performance. 
    %\textcolor{green}{EVANS: working on this now} 
%\end{itemize}}



\end{document}