%Denial-of-Service (DoS) attacks pose a significant threat to the availability and stability of online services, making server protection a paramount concern for organizations worldwide \cite{gupta2021distributed}. As the digital landscape continues to evolve, the sophistication and frequency of these attacks have increased, necessitating robust countermeasures to safeguard the networked infrastructure. While numerous machine learning (ML) models have been employed in this domain, there is a pressing need to explore innovative approaches that can improve their detection capabilities and overall performance \cite{khalaf2019comprehensive}.

DoS attacks threaten online service availability and stability, making server protection crucial \cite{gupta2021distributed}. With evolving digital threats, more sophisticated and frequent attacks require strong countermeasures to secure networked infrastructure \cite{rahouti2021synguard}. While many ML models have been used, there's a need for innovative approaches to enhance detection and performance. %\cite{khalaf2019comprehensive}.

While ML has shown promise in various cybersecurity applications, including denial DoS detection, it is important to recognize its limitations \cite{mittal2022deep}. Among these limitations, with regard to DoS detection, is the interpretability and explainability. Specifically, some ML algorithms, especially deep learning models, are considered black boxes, making it difficult to interpret their decision-making process. This lack of interpretability and explainability can hinder trust and make it challenging to understand why a particular decision or prediction (e.g., whether a networking flow is malicious or benign) was made.

To overcome such limitations, a holistic approach combining ML with other techniques, network-level mitigations, anomaly detection, and expert knowledge can be employed to enhance the accuracy and robustness of DoS detection systems. In this paper, we introduce a novel methodology based on leveraging the cutting-edge approach of combinatorial fusion, which harnesses recently developed algorithms and techniques for model fusion. Our objective is to combine multiple ML models using combinatorial fusion analysis to achieve superior DoS attack detection performance and interpretability.

The key contributions of this paper are summarized as follows.
\begin{itemize}
    \item %The landscape of DoS attack detection models: 
    Examine the key ML models commonly used for DoS attack detection. Such an examination serves as the ground truth for integrating the combinatorial fusion analysis (CFA) approach into our ML-based DoS detection.
    \item CFA approach: Develop an innovative approach based on CFA for combining multiple ML models in DoS attack detection. %Leveraging such an approach, we integrate the outputs of these models to achieve optimal detection performance. 
    Our methodology encompasses various techniques, including advanced score combination, rank combination, weighted combination, and the consideration of diversity strength across multiple scoring systems.
    \item Conduct performance evaluations to assess the effectiveness of our CFA. Through comparative analyses, we showcase the significant improvements achieved in terms of detection accuracy, false positive rates, and overall precision.
\end{itemize}

By leveraging the power of the recently developed combinatorial fusion approach and its associated algorithms, our study aims to push the boundaries of DoS attack detection. We anticipate that our findings will contribute to the advancement of more robust and effective defense mechanisms against DoS attacks, bolstering the uninterrupted availability and security of critical online services.

%The rest of this paper is organized as follows. Section \ref{sec:related} summarizes the state-of-the art studies related to this work. Next, Section \ref{sec:methodology} discusses the research problem and methodology utilized in this work. Section \ref{sec:evaluation} presents the evaluation setups and key findings to demonstrate the efficiency of the proposed methodology. Last, Section \ref{sec:discussion} provides further discussions on the evaluation results, and  Section \ref{sec:conclusion} concludes this work.


