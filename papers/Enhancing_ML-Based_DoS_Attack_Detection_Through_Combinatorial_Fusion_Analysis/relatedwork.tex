%In recent years, the detection of DoS attacks has garnered significant attention in the field of cybersecurity. Various approaches and techniques have been proposed to enhance the accuracy and efficiency of DoS attack detection \cite{zhijun2020low}. Next, we present a summary of the state-of-the-art, closely related to the research of this paper, for DoS detection in networking environments. The escalation of DoS attacks has led researchers and service providers to focus more on investigating and addressing these events \cite{david2021discriminating}. Traditional network layer detection methods, such as examining packet headers or traffic patterns, may not be sufficient to effectively detect and mitigate modern DoS attacks. As a result, there has been a surge of research and development of novel detection and mitigation approaches \cite{chin2018kernel, zheng2018realtime, rahouti2021synguard, liang2019empirical}.

DoS attack detection continuously gains attention in cybersecurity, with diverse methods proposed to enhance accuracy \cite{zhijun2020low}. Escalating DoS attacks prompted increased focus by researchers \cite{david2021discriminating}. %Traditional methods like examining headers or patterns may fall short against modern attacks, spurring novel approaches \cite{rahouti2021synguard}. 
Many studies employ ML algorithms for DoS attack detection, using models like SVMs, Random Forests, and Naive Bayes. These models analyze network patterns, using features like packet rates, sizes, and volume to differentiate normal and attack traffic \cite{ali2023machine}. Further, CNNs, RNNs, and variants extract intricate patterns and temporal dependencies from network traffic \cite{mittal2022deep}, showing promise in accurate attack identification. Ensemble methods like Bagging, Boosting, and Stacking also gained traction, combining models to enhance prediction \cite{deepa2019design}. These approaches counter individual model limitations, boosting accuracy and robustness.

%Numerous studies have utilized ML algorithms for DoS attack detection. Traditional ML models, such as Support Vector Machines (SVMs), Random Forests, and Naive Bayes, have been employed to classify network traffic patterns and identify anomalous behavior indicative of DoS attacks. These models leverage features such as packet arrival rates, packet size distributions, and traffic volume to differentiate normal and attack traffic \cite{ali2023machine}.

%Further, with the advancements in deep learning, researchers have explored the use of deep neural networks for DoS attack detection \cite{zainudin2022efficient}. Convolutional Neural Networks (CNNs), Recurrent Neural Networks (RNNs), and their variants have been applied to extract intricate patterns and temporal dependencies in network traffic data. These deep learning models have shown promising results in accurately identifying DoS attacks \cite{mittal2022deep, sanchez2021feature, yuan2017deepdefense}. Ensemble learning techniques have also gained attention for their ability to combine multiple models to improve prediction performance. Ensemble methods such as Bagging, Boosting, and Stacking have been employed in DoS attack detection to leverage the strengths of diverse ML models. These approaches aim to mitigate individual models' limitations and enhance the detection system's overall accuracy and robustness \cite{deepa2019design}.

%Advances in deep learning led to using neural networks for DoS attack detection \cite{zainudin2022efficient}. 

In recent years, model fusion techniques, such as combinatorial fusion analysis (CFA), have emerged as practical means of combining multiple ML models' performance \cite{hsu2006combinatorial, hsu2010rank}. These approaches leverage algorithms and metrics specifically designed to integrate the outputs of different models, considering their diversity and individual performance. The goal is to leverage diverse models' collective knowledge and expertise to achieve higher accuracy and detection rates \cite{hurley2020multi}. While previous studies have remarkably contributed to DoS attack detection, our work focuses on the emerging CFA. To the best of our knowledge, this is the first work that adopts CFA for examining the performance of ML models-based DoS attack detection. By combining the strengths of multiple ML models using advanced fusion techniques, we aim to achieve more accurate and reliable detection results.