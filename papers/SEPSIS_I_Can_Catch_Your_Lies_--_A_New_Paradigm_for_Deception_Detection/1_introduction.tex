\vspace{-4mm}
\section{Defining Deception -- Inspiration from Psychology}

According to \cite{schuiling2004deceive}, deception is a behavior observed in various species and is considered an evolutionary adaptive trait. \cite{depaulo1998everyday}  assert that deception is an integral part of social interactions, with the \ul{majority of humans engaging in deceptive acts at least once or twice a day}. While most instances of deception are relatively minor, there is a frequent association between deception and egregious norm violations, such as theft, murder, and attempts to evade punishment for such crimes. Consequently, researchers have long been interested in identifying behaviors that can differentiate between truthful and deceitful communications.

%Numerous studies have explored the behavioral indicators of lying; however, no solitary behavior or amalgamation of behaviors possesses the unequivocal capability to ascertain deceptive communication. 
Numerous studies have delved into describing the behavioral indicators of deceit. However, no single behavior or combination of behaviors has been found to possess the definitive ability to accurately determine deceptive communication. %The empirical evidence supporting the role of specific individual behaviors during deception often presents contradictory findings \cite{depaulo1985deceiving}\cite{kraut1980humans}\cite{vrij2000detecting}. One possible explanation for these contradictions in the literature on deception cues is the inadequate distinction made by researchers between distinct subtypes of deception. 
The empirical evidence supporting the significance of specific individual behaviors in deception often presents conflicting findings \cite{depaulo1985deceiving, kraut1980humans, vrij2000detecting}. One possible explanation for these contradictions in the literature regarding deception cues is the insufficient differentiation made by researchers between distinct subtypes of deception.

% \vspace{-1mm}
%\begin{strip}
In the realm of psychology research, a consensus has yet to be reached regarding the classification of various types of deception. Nevertheless, we discovered that the framework outlined in Hample's work \cite{hample1982empirical}, visually described in figure ~\ref{fig:sepsis}, provides a viable foundation for constructing NLP models. 
Hample, et al, 1982 categorize deception into three distinct forms: \emph{lies of omission}, \emph{lies of commission}, and \emph{lies of influence}. For the purpose of our study, we focus solely on investigating \emph{lies of omission}. It is worth noting that the NLP community has extensively explored the fact verification problem, which is primarily associated with \emph{lies of commission}. Conversely, \emph{lies of omission} have received comparatively less attention. In this paper, we present a comprehensive study on lies of omission, which, to the best of our knowledge, is the first of its kind. 
%\vj{Don't think we need to talk about lies of influence here, last sentence could be deleted.}On the other hand, \emph{lies of influence} represent a significantly more intricate phenomenon that necessitates the utilization of large-scale conversation corpora for a thorough examination.

%In the field of psychology research, there is no consensus regarding the categorization of different types of deception. However, we found the framework as described in \cite{hample1982empirical} (~\cref{fig:sepsis}) to be descriptive and \textcolor{brown}{feasible to build NLP models on it}. 
%To study deception, it is important to differentiate between different approaches or methods used to deceive others. One method, known as omission, involves deliberately withholding relevant information, while the other method, referred to as commission, entails fabricating false information \cite{hample1982empirical}. From the standpoint of the NLP and AI community, fact verification has been extensively studied; however, lies of omission have received less attention. In this paper, we introduce an extensive study on lies of omission, which is, to our knowledge, the first of its kind.
\vspace{-2mm}

\begin{tcolorbox}[colback=blue!5!white,colframe=blue!75!black,title=\footnotesize{\textbf{\textsc{\ul{Our Contributions}}}: \textls[-10]{SEPSIS dataset, MTL framework utilizing dataless LLM merging, unveiling the relationship between deception and propaganda}.}]
% - pertaining questions]
\begin{itemize}
[leftmargin=1mm]
\setlength\itemsep{0em}
\begin{spacing}{0.90}
\vspace{-2mm}
\item[\ding{224}] {\footnotesize \fontfamily{phv}\fontsize{8}{9}\selectfont{ 
%Building upon the definition of deception from psychology, we apply NLP techniques, such as mask infilling and paraphrasing for data augmentation.
%, to mitigate the intricate problem of deception detection, promoting interdisciplinary research.
We present a pioneering study on the phenomenon of lies of omission.
}}
% \end{spacing}}
\vspace{-1mm}
\item[\ding{224}] {\footnotesize 
{\fontfamily{phv}\fontsize{8}{9}\selectfont{
%Dataset curation of around 876K samples with $4$ objective layers described in Figures \ref{fig:sepsis} \& \ref{fig: architecture_MaskInfilling} descriptive layers (annotating reasons for tagging it with the \textit{type of omission} and \textit{color of lies}).
We introduce the SEPSIS corpus and associated resources. The SEPSIS corpus (876,784 data points) incorporates four layers of annotation, including type, color, intention, and topic.}
}}

\vspace{-1mm}
\item[\ding{224}] {\footnotesize 
{\fontfamily{phv}\fontsize{8}{9}\selectfont
{We introduce an MTL pipeline for SEPSIS classification. The MTL pipeline leverages the dataless merging of fine-tuned Language Models (LMs) and further incorporates a tailored loss function specific to each layer, addressing different sub-problems.}
}}

\vspace{-1mm}
\item[\ding{224}] {\footnotesize 
{\fontfamily{phv}\fontsize{8}{9}\selectfont
{Finally, the paper reveals a significant correlation between deception and propaganda techniques.}
}}

\vspace{-6mm}
\end{spacing}
\end{itemize}
\end{tcolorbox}
\vspace{-3mm}
