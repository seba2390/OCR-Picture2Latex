%\documentclass{article}
%\usepackage[most]{tcolorbox}
%\usepackage{lipsum}
%\begin{document}
%\lipsum[2]
\begin{tcbraster}[raster columns=2,raster equal height]
%%%%%%%%%
\begin{tcolorbox}[nobeforeafter, title=box 1,colback=brown!5!white,colframe=brown!75!black,title=\footnotesize{\textbf{\textsc{Propaganda Technique Definition}}}]
%\lipsum[2]
\begin{itemize}
[leftmargin=1mm]
\setlength\itemsep{0em}
\begin{spacing}{0.90}
%\vspace{-2.2mm}

%\vspace{-2.2mm}
\item[\ding{224}] {\footnotesize 
{\fontfamily{phv}\fontsize{8}{9}\selectfont
{\textbf{Straw Men:}Substituting an opponent’s proposition with a similar one, which is then refuted in place of the original proposition.}
}}

\item[\ding{224}] {\footnotesize 
{\fontfamily{phv}\fontsize{8}{9}\selectfont
{\textbf{Doubt:}Questioning the credibility of someone or something.}
}}

\vspace{-2.2mm}
\item[\ding{224}] {\footnotesize 
{\fontfamily{phv}\fontsize{8}{9}\selectfont
{\textbf{Obfuscation:} Using words that are deliberately not clear, so that the audience may have their own interpretations.}
}}

\vspace{-2.2mm}
\item[\ding{224}] {\footnotesize 
{\fontfamily{phv}\fontsize{8}{9}\selectfont
{\textbf{Reductio ad Hitlerum:} An attempt to invalidate someone else’s argument on the basis that the same idea was promoted.}
}}

\vspace{-2.2mm}
\item[\ding{224}] {\footnotesize 
{\fontfamily{phv}\fontsize{8}{9}\selectfont
{\textbf{Black and White Fallacy:}Using words that depict the fallacy of leaping from the undesirability of one proposition to the truth of an extreme opposite.
}
}}

\vspace{-6mm}
\end{spacing}
\end{itemize}
%\end{tcolorbox}



%%%%%%%%%%

\end{tcolorbox}
%\begin{tcolorbox}[nobeforeafter, title=box 2]
\begin{tcolorbox}[ nobeforeafter, title=box 2, colback=teal!5!white,colframe=teal!75!black,title=\footnotesize{\textbf{\textsc{Propaganda through Deception}}}]
\begin{itemize}
[leftmargin=1mm]
\setlength\itemsep{0em}
\begin{spacing}{0.90}
\vspace{-2.2mm}
\item[\ding{224}] {\footnotesize 
{\fontfamily{phv}\fontsize{8}{9}\selectfont
{\textbf{Straw Men:} Straw Men maps mostly with speculation but sometimes with opinion too. It maps with both black and white lie of layer 2 in most cases and gaining advantage in layer 3 and political in layer 4. }
}}


\vspace{-2.2mm}
\item[\ding{224}] {\footnotesize 
{\fontfamily{phv}\fontsize{8}{9}\selectfont
{\textbf{Doubt:} Doubt maps mostly with speculation in layer 1, black lie in layer 2, gaining advantage in layer 3 and political in layer 4.}
}}

\vspace{-2.2mm}
\item[\ding{224}] {\footnotesize 
{\fontfamily{phv}\fontsize{8}{9}\selectfont
{\textbf{Obfuscation:} This technique maps mostly with speculation in layer 1, red lie in layer 2, gaining advantage in layer 3 and political in layer 4. }
}}

\vspace{-2.2mm}
\item[\ding{224}] {\footnotesize 
{\fontfamily{phv}\fontsize{8}{9}\selectfont
{\textbf{Reductio ad Hitlerum:} This technique maps with speculation and distrotion in layer1, black lies and occasional white lies in layer 2. Layer 3 and layer 4 are primarily associated with gaining advantage and politics, respectively. }
}}

\vspace{-2.2mm}
\item[\ding{224}] {\footnotesize 
{\fontfamily{phv}\fontsize{8}{9}\selectfont
{\textbf{Black and White Fallacy:} This technique predominantly involves speculation and opinion, with elements of black lies in the second layer. In the third layer, it is mostly aligned with gaining advantage but occasionally tied to protecting oneself and political and educational in layer 4.
}
}}

\vspace{-6mm}
\end{spacing}
\end{itemize}
%\end{tcolorbox}


%\lipsum[2]
\end{tcolorbox}
\end{tcbraster}
%\lipsum[2]
%\end{document}
