%\documentclass{article}
%\usepackage[most]{tcolorbox}
%\usepackage{lipsum}
%\begin{document}
%\lipsum[2]
\begin{tcbraster}[raster columns=2,raster equal height]
%%%%%%%%%
\begin{tcolorbox}[nobeforeafter, title=box 1,colback=brown!5!white,colframe=brown!75!black,title=\footnotesize{\textbf{\textsc{Propaganda Technique Definition}}}]
%\lipsum[2]
\begin{itemize}
[leftmargin=1mm]
\setlength\itemsep{0em}
\begin{spacing}{0.90}
\vspace{-2.2mm}
\item[\ding{224}] {\footnotesize \fontfamily{phv}\fontsize{8}{9}\selectfont{\textbf{Flag Waving:} Playing on strong national feeling (or to any group, e.g., race, gender, etc) to justify or promote an action or an idea.
}}
% \end{spacing}}
\vspace{-2.2mm}
\item[\ding{224}] {\footnotesize 
{\fontfamily{phv}\fontsize{8}{9}\selectfont{\textbf{Slogans:}A brief and striking phrase that may include labeling and stereotyping.
}
}}

\vspace{-2.2mm}
\item[\ding{224}] {\footnotesize 
{\fontfamily{phv}\fontsize{8}{9}\selectfont
{\textbf{Appeal to fear - prejudices:}Seeking to build support for an idea by instilling anxiety and/or panic in the population towards an alternative.}
}}
\vspace{-2.2mm}
\item[\ding{224}] {\footnotesize 
{\fontfamily{phv}\fontsize{8}{9}\selectfont
{\textbf{Exaggeration-Minimization}: Either representing something in an excessive manner: making things larger, better, worse (e.g., the best of the best) or making something seem less important or smaller than it really is (e.g., saying that an insult was actually just a joke).}
}}

\vspace{-2.2mm}
\item[\ding{224}] {\footnotesize 
{\fontfamily{phv}\fontsize{8}{9}\selectfont
{\textbf{Repetition:} Repeating the same message over and over again so that the audience will eventually accept it.}
}}

\vspace{-2.2mm}
\item[\ding{224}] {\footnotesize 
{\fontfamily{phv}\fontsize{8}{9}\selectfont
{\textbf{Name Calling Labelling:} Labeling the object of the propaganda campaign as something that the target audience fears, hates, finds undesirable, or loves or praises. }
}}

\vspace{-2.2mm}
\item[\ding{224}] {\footnotesize 
{\fontfamily{phv}\fontsize{8}{9}\selectfont
{\textbf{Bandwagon:} Attempting to persuade the target audience to join in and take the course of action because “everyone else is taking the same action.”}
}}

\vspace{-2.2mm}
\item[\ding{224}] {\footnotesize 
{\fontfamily{phv}\fontsize{8}{9}\selectfont
{\textbf{Loaded Language:} Using specific words and phrases with strong emotional implications (either positive or negative) to influence an audience. }
}}

\vspace{-2.2mm}
\item[\ding{224}] {\footnotesize 
{\fontfamily{phv}\fontsize{8}{9}\selectfont
{\textbf{Casual Oversimplification:} Assuming a single cause or reason when there are actually multiple causes for an issue.}
}}

\vspace{-2.2mm}
\item[\ding{224}] {\footnotesize 
{\fontfamily{phv}\fontsize{8}{9}\selectfont
{\textbf{Red herring:} Introducing irrelevant material to the issue being discussed so that everyone’s attention is diverted away from the points made.}
}}


\vspace{-2.2mm}
\item[\ding{224}] {\footnotesize 
{\fontfamily{phv}\fontsize{8}{9}\selectfont
{\textbf{Appeal to authority:} Stating that a claim is true simply because a valid authority or expert on the issue said it was true.}
}}

\vspace{-2.2mm}
\item[\ding{224}] {\footnotesize 
{\fontfamily{phv}\fontsize{8}{9}\selectfont
{\textbf{Thought terminating cliches:} Words or phrases that discourage critical thought and meaningful discussion about a given topic.}
}}

\vspace{-2.2mm}
\item[\ding{224}] {\footnotesize 
{\fontfamily{phv}\fontsize{8}{9}\selectfont
{\textbf{Whataboutism:} A technique that attempts to discredit an opponent’s position by charging them with hypocrisy without directly disproving their argument.}
}}




\vspace{-6mm}
\end{spacing}
\end{itemize}
%\end{tcolorbox}



%%%%%%%%%%

\end{tcolorbox}
%\begin{tcolorbox}[nobeforeafter, title=box 2]
\begin{tcolorbox}[ nobeforeafter, title=box 2, colback=teal!5!white,colframe=teal!75!black,title=\footnotesize{\textbf{\textsc{Propaganda through Deception}}}]
\begin{itemize}
[leftmargin=1mm]
\setlength\itemsep{0em}
\begin{spacing}{0.90}
\vspace{-2.2mm}
\item[\ding{224}] {\footnotesize \fontfamily{phv}\fontsize{8}{9}\selectfont{\textbf{Flag Waving:} Flag waving maps to speculation in layer 1, black lies in layer 2, gaining advantage in layer 3, and religious aspects in layer 4.
}}
% \end{spacing}}
\vspace{-2.2mm}
\item[\ding{224}] {\footnotesize 
{\fontfamily{phv}\fontsize{8}{9}\selectfont{\textbf{Slogans:} This technique is mostly mapped with speculation in layer1, white lie in layer 2, political in layer 3 and gaining advantage in layer 4.
}
}}

\vspace{-2.2mm}
\item[\ding{224}] {\footnotesize 
{\fontfamily{phv}\fontsize{8}{9}\selectfont
{\textbf{Appeal to fear - prejudices:} This technqiue primarily corresponds to speculation in layer 1, black lie in layer 2, political in layer 3 and gaining advantage in layer 4.}
}}

\vspace{-2.2mm}
\item[\ding{224}] {\footnotesize 
{\fontfamily{phv}\fontsize{8}{9}\selectfont
{\textbf{Exaggeration-Minimization}: In the Layers of Omission, Exaggeration or Minimization is mostly mapped to speculation in layer 1, black lie in layer 2, political in layer 3 and gaining advantage in layer 4.}
}}

\vspace{-2.2mm}
\item[\ding{224}] {\footnotesize 
{\fontfamily{phv}\fontsize{8}{9}\selectfont
{\textbf{Repetition:} Repetition is mostly mapped to Speculation, Black lie, intention of gaining advantage and in political influence.}
}}

\vspace{-2.2mm}
\item[\ding{224}] {\footnotesize 
{\fontfamily{phv}\fontsize{8}{9}\selectfont
{\textbf{Name Calling Labelling:} Name Calling or Labelling is largely mapped to speculation in layer 1, black lie in layer 2, gaining advantage in layer 3 and political in layer 4.}
}}

\vspace{-2.2mm}
\item[\ding{224}] {\footnotesize 
{\fontfamily{phv}\fontsize{8}{9}\selectfont
{\textbf{Bandwagon:} Bandwagon is mostly mapped to speculation in layer 1. It is mapped with both white and gray lie in layer 2. It is mapped with protecting oneself in layer 3 and education in layer 4. }
}}

\vspace{-2.2mm}
\item[\ding{224}] {\footnotesize 
{\fontfamily{phv}\fontsize{8}{9}\selectfont
{\textbf{Loaded Language:} Loaded Language is mapped mostly with speculation in layer 1, black lie in layer 2, gaining advantage in layer 3 and political in layer 4. }
}}

\vspace{-2.2mm}
\item[\ding{224}] {\footnotesize 
{\fontfamily{phv}\fontsize{8}{9}\selectfont
{\textbf{Casual Oversimplification:} Causal Oversimplification is mapped mostly with speculation in layer 1, with black lie and in some cases with red lie in layer 2, gaining advantage in layer 3 and political in layer 4. }
}}

\vspace{-2.2mm}
\item[\ding{224}] {\footnotesize 
{\fontfamily{phv}\fontsize{8}{9}\selectfont
{\textbf{Red herring:} In layer 1, Red Herring corresponds to both speculation and opinion. Layer 2 primarily associates it with black lies, occasionally with white lies. In layer 3, it largely aligns with gaining advantage, while layer 4 relates to political aspects.}
}}


\vspace{-2.2mm}
\item[\ding{224}] {\footnotesize 
{\fontfamily{phv}\fontsize{8}{9}\selectfont
{\textbf{Appeal to authority:} This technique largely maps with opinion and with speculation too. In the 2nd layer, it maps with black and gray lies and with gaining advantage in 3rd layer and political in 4th layer. }
}}

\vspace{-2.2mm}
\item[\ding{224}] {\footnotesize 
{\fontfamily{phv}\fontsize{8}{9}\selectfont
{\textbf{Thought terminating cliches:} This technique mostly maps with speculation in layer 1, gray and black lie in layer 2, gaining advantage in layer 3 and political in layer 4.}
}}

\vspace{-2.2mm}
\item[\ding{224}] {\footnotesize 
{\fontfamily{phv}\fontsize{8}{9}\selectfont
{\textbf{Whataboutism:} Whataboutism mostly maps with speculation in layer 1, black lie in layer 2, gaining advantage in layer 3 and political in layer 4. }
}}


\vspace{-6mm}
\end{spacing}
\end{itemize}
%\end{tcolorbox}


%\lipsum[2]
\end{tcolorbox}
\end{tcbraster}
%\lipsum[2]
%\end{document}
