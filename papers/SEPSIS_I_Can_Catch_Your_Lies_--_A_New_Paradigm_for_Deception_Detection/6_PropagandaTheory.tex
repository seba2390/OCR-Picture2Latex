\vspace{-3mm}
\section{Dissecting Propaganda through the Lens of Deception}
\vspace{-1.5mm}
As mentioned earlier, numerous studies have explored the behavioral indicators of lying, but there is hardly any consensus on categorization. However, the focus of this paper specifically revolves around investigating \emph{lies of omission} and their connection to related research within the scientific community. Notably, there are works that have extensively examined the analysis of \emph{propaganda} through language \cite{da-san-martino-etal-2019-fine,martino2020survey}.

%\vspace{-1mm}
\begin{figure}[h]
\vspace{-3mm}  
\centering
  \centering
\includegraphics[width=0.75\columnwidth]{Image/Circos_LoadedLanguage.png}
\vspace{-1.5mm}
  \caption{The Circos presents the co-occurrence of all the layers of deception with a propaganda technique named \emph{loaded language}.}
  \label{fig:loaded_language}
\vspace{-4mm}  
\end{figure}
% \vspace{-3mm}

Our scientific curiosity led us to further investigate the specific types of \emph{lies of omission} employed in strategizing particular propaganda, such as \textit{exaggeration} and/or \textit{red herring}. To conduct this study, we utilized the propaganda datasets introduced by \cite{da-san-martino-etal-2019-fine} and applied the SEPSIS classifier, as discussed in section ~\ref{sec:sepsis_classifier} on the data. Through the analysis of these experiments, we made intriguing discoveries, including: (i) \emph{the prevalence of political topic in loaded language compared to other propaganda types}, (ii) \emph{the close association between the intention of gaining advantage and Name Calling}, and (iii) \emph{the complexity underlying causal simplification as a form of speculation.} A Circos \cite{Flourish} example is presented in Fig. \ref{fig:loaded_language} for a propaganda technique named
\textit{loaded language} (cf. Appendix \ref{sec: Propaganda} for Circos diagrams corresponding to propaganda techniques). Therefore, we firmly believe that our research on SEPSIS not only stands on its own but also acts as a bridge, facilitating a deeper understanding of deception. 


%(i) \emph{the prevalence of biased language in slogans compared to other propaganda types}, (ii) \emph{the close association between distortion and whataboutism}, and (iii) \emph{the complexity underlying causal simplification as a form of speculation.}


\begin{comment}
The propaganda model is a conceptual model in political economy advanced by Edward S. Herman and Noam Chomsky to explain how propaganda and systemic biases function in corporate mass media. The model seeks to explain how populations are manipulated and how consent for economic, social, and political policies, both foreign and domestic, is "manufactured" in the public mind due to this propaganda \cite{}. %cite Wikipedia page.
Propaganda aims at influencing people’s mindsets with the purpose of advancing a specific agenda.
% How are we connecting it to deception


%Data

Other research work involved releasing data and categorizing them into 20 propaganda techniques \cite{}. We use data from XYZ \cite{} that is tagged with $20$ different propaganda techniques.

% data processing
\url{https://propaganda.math.unipd.it/fine-grained-propaganda-emnlp.html}

%MTL to predict which kind of deception






%20 circos over here, 4 of them should come here, rest in the appendix
\end{comment}










