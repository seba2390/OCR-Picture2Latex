% This must be in the first 5 lines to tell arXiv to use pdfLaTeX, which is strongly recommended.
\pdfoutput=1
% In particular, the hyperref package requires pdfLaTeX in order to break URLs across lines.

\documentclass[11pt]{article}

% Remove the "review" option to generate the final version.
\usepackage{acl}

% Standard package includes
\usepackage{times}
\usepackage{latexsym}

% For proper rendering and hyphenation of words containing Latin characters (including in bib files)
\usepackage[T1]{fontenc}
% For Vietnamese characters
% \usepackage[T5]{fontenc}
% See https://www.latex-project.org/help/documentation/encguide.pdf for other character sets

% This assumes your files are encoded as UTF8
\usepackage[utf8]{inputenc}

% This is not strictly necessary and may be commented out.
% However, it will improve the layout of the manuscript,
% and will typically save some space.
\usepackage{microtype}

% This is also not strictly necessary and may be commented out.
% However, it will improve the aesthetics of text in
% the typewriter font.
\usepackage{inconsolata}
\usepackage{multirow}
\usepackage{arydshln} % for dotted lines
\usepackage{color, soul, colortbl}
%\usepackage[table,xcdraw]{xcolor}
\usepackage{graphicx}
%\usepackage[table]{xcolor}
\usepackage{booktabs}
\usepackage[most]{tcolorbox}
\usepackage[export]{adjustbox} % for the valign option
\usepackage{pifont}
\usepackage{booktabs}
\usepackage{multirow}
\usepackage{subcaption}
\usepackage{graphicx}
\usepackage{tikz}
\usetikzlibrary{shapes.geometric, arrows}
\usetikzlibrary{decorations.markings}
\usepackage{soul}
\usepackage{wrapfig,graphicx,lipsum}% http://ctan.org/pkg/{wrapfig,graphicx,lipsum}
\usepackage[T1]{fontenc}
\usepackage[utf8]{inputenc}
%\usepackage[italian]{babel}
\usepackage{lmodern}
\usepackage{extsizes}
\usepackage{mathptmx}
\usepackage{cuted}
\usepackage{flushend}
\usepackage{float}
\usepackage{multicol}
\usepackage{anyfontsize}
\usepackage[draft,textsize=footnotesize,textwidth=15mm]{todonotes}
%\usepackage[usenames,dvipsnames]{color}
\usepackage{enumitem}
\setlist{leftmargin=5.5mm}
\usepackage{setspace}
\usepackage{cleveref}




\usepackage{algorithm}
\usepackage{algpseudocode}
\usepackage{color}
\DeclareRobustCommand{\hlpink}[1]{{\sethlcolor{pink}\hl{#1}}}
\DeclareRobustCommand{\hlgreen}[1]{{\sethlcolor{green}\hl{#1}}}


% Added by pankaj
\usepackage{siunitx,tabularx,ragged2e,booktabs,caption}

\makeatletter
\newcommand{\DrawLine}{%
  \begin{tikzpicture}
  \path[use as bounding box] (0,0) -- (\linewidth,0);
  \draw[color=blue!75!black,dashed,dash phase=.5pt]
        (0-\kvtcb@leftlower-\kvtcb@boxsep,0)--
        (\linewidth+\kvtcb@rightlower+\kvtcb@boxsep,0);
  \end{tikzpicture}%
  }
\makeatother

\usepackage[draft,textsize=footnotesize,textwidth=15mm]{todonotes}
%\usepackage[usenames,dvipsnames]{color}
\usepackage{color}
\usepackage{soul}
\DeclareRobustCommand{\hlpink}[1]{{\sethlcolor{pink}\hl{#1}}}
\DeclareRobustCommand{\hlgreen}[1]{{\sethlcolor{green}\hl{#1}}}


\newcommand\ac[1]{\todo[author=AC,color=blue!40]{#1}}
\newcommand\acil[1]{\todo[author=AC,color=blue!40,inline]{#1}}
\newcommand\acb[1]{\textcolor{blue}{#1}}
\newcommand\act[1]{\textcolor{blue}{#1}}

%--Vinija---
\newcommand\vj[1]{\todo[author=VJ,color=orange!40]{#1}}
\newcommand\vjil[1]{\todo[author=VJ,color=orange!40,inline]{#1}}
\newcommand\vjt[1]{\textcolor{orange}{#1}}


%--Amitava---
\newcommand\ad[1]{\todo[author=AD,color=purple!40]{#1}}
\newcommand\adil[1]{\todo[author=AD,color=purple!40,inline]{#1}}
\newcommand\adt[1]{\textcolor{purple}{#1}}


% If the title and author information does not fit in the area allocated, uncomment the following
%
\setlength\titlebox{5.75 cm}
%
% and set <dim> to something 5cm or larger.

\title{
\vspace{-20mm}
\begin{center}
\includegraphics[width=0.75\textwidth]{Image/SEPSIS_bannerv2.pdf}
\end{center}
% [SEPSIS] I can catch your lies: A new paradigm for Deception Detection
SEPSIS: I Can Catch Your Lies -- A New Paradigm for Deception Detection
}

% Author information can be set in various styles:
% For several authors from the same institution:
% \author{Author 1 \and ... \and Author n \\
%         Address line \\ ... \\ Address line}
% if the names do not fit well on one line use
%         Author 1 \\ {\bf Author 2} \\ ... \\ {\bf Author n} \\
% For authors from different institutions:
% \author{Author 1 \\ Address line \\  ... \\ Address line
%         \And  ... \And
%         Author n \\ Address line \\ ... \\ Address line}
% To start a separate ``row'' of authors use \AND, as in
% \author{Author 1 \\ Address line \\  ... \\ Address line
%         \AND
%         Author 2 \\ Address line \\ ... \\ Address line \And
%         Author 3 \\ Address line \\ ... \\ Address line}

\author{\textbf{Anku Rani}$^{1}$ \quad \textbf{Dwip Dalal}$^{2}$ \quad \textbf{Shreya Gautam}$^{3}$ \quad \textbf{Pankaj Gupta}$^{4}$ \\
\textbf{Vinija Jain}\dag\textsuperscript{5,6} \quad
\textbf{Aman Chadha}\dag\textsuperscript{5,6} \quad \textbf{Amit Sheth}$^{1}$ \quad \textbf{Amitava Das}$^{1}$ \quad \\
$^{1}$University of South Carolina, USA \quad     
$^{2}$IIT Gandhinagar, India \quad \\
$^{3}$BIT Mesra, India \quad
$^{4}$DTU, India \quad 
$^{5}$Stanford University, USA \quad  
$^{6}$Amazon AI, USA \\
% \setttsize{}
\tt  arani@mailbox.sc.edu \quad
\tt amitava@mailbox.sc.edu
}

\begin{document}
\maketitle
\renewcommand{\thefootnote}{\fnsymbol{footnote}}
\footnotetext[2]{Work does not relate to the position at Amazon.}
\renewcommand*{\thefootnote}{\arabic{footnote}}
\setcounter{footnote}{0}
\begin{abstract}

Deception is the intentional practice of twisting information. It is a nuanced societal practice deeply intertwined with human societal evolution, characterized by a multitude of facets. This research explores the problem of deception through the lens of psychology, employing a framework that categorizes deception into three forms: \emph{lies of omission}, \emph{lies of commission}, and \emph{lies of influence.} The primary focus of this study is specifically on investigating only \emph{lies of omission.} We propose a novel framework for deception detection leveraging NLP techniques. We curated an annotated dataset of $876,784$ samples by amalgamating a popular large-scale fake news dataset and scraped news headlines from the Twitter handle of "\emph{Times of India}", a well-known Indian news media house. Each sample has been labeled with four layers, namely: (i) the \ul{type of omission} \textit{(speculation, bias, distortion, sounds factual,} and \textit{opinion)}, (ii) \ul{colors of lies} \textit{(black, white, grey,} and \textit{red)}, and (iii) the \ul{intention of such lies} \textit{(to influence, gain social prestige,} etc\textit{)} (iv) \ul{topic of lies} \textit{(political, educational, religious, racial,} and \textit{ethnicity)}. We present a novel multi-task learning [MTL] pipeline that leverages the dataless merging of fine-tuned language models to address the deception detection task mentioned earlier. Our proposed model achieved an impressive F1 score of 0.87, demonstrating strong performance across all layers including the \textit{type}, \textit{color}, \textit{intent}, and \textit{topic} aspects of deceptive content.
Finally, our research aims to explore the relationship between \emph{lies of omission} and \emph{propaganda} techniques. To accomplish this, we conducted an in-depth analysis, uncovering compelling findings. For instance, our analysis revealed a significant correlation between \emph{loaded language} and \emph{opinion}, shedding light on their interconnectedness. 
To encourage further research in this field, we will be making our SEPSIS models and dataset available with the MIT License, making it favorable for open-source research and commercial use.

\end{abstract}



\begin{figure}[t]
\begin{center}
   \includegraphics[width=1.0\linewidth]{figures/nas_comp_v3}
\end{center}
   \vspace{-4mm}
   \caption{The comparison between NetAdaptV2 and related works. The number above a marker is the corresponding total search time measured on NVIDIA V100 GPUs.}
\label{fig:nas_comparison}
\end{figure}

\section{Introduction}
\label{sec:introduction}

Neural architecture search (NAS) applies machine learning to automatically discover deep neural networks (DNNs) with better performance (e.g., better accuracy-latency trade-offs) by sampling the search space, which is the union of all discoverable DNNs. The search time is one key metric for NAS algorithms, which accounts for three steps: 1) training a \emph{super-network}, whose weights are shared by all the DNNs in the search space and trained by minimizing the loss across them, 2) training and evaluating sampled DNNs (referred to as \emph{samples}), and 3) training the discovered DNN. Another important metric for NAS is whether it supports non-differentiable search metrics such as hardware metrics (e.g., latency and energy). Incorporating hardware metrics into NAS is the key to improving the performance of the discovered DNNs~\cite{eccv2018-netadapt, Tan2018MnasNetPN, cai2018proxylessnas, Chen2020MnasFPNLL, chamnet}.


There is usually a trade-off between the time spent for the three steps and the support of non-differentiable search metrics. For example, early reinforcement-learning-based NAS methods~\cite{zoph2017nasreinforcement, zoph2018nasnet, Tan2018MnasNetPN} suffer from the long time for training and evaluating samples. Using a super-network~\cite{yu2018slimmable, Yu_2019_ICCV, autoslim_arxiv, cai2020once, yu2020bignas, Bender2018UnderstandingAS, enas, tunas, Guo2020SPOS} solves this problem, but super-network training is typically time-consuming and becomes the new time bottleneck. The gradient-based methods~\cite{gordon2018morphnet, liu2018darts, wu2018fbnet, fbnetv2, cai2018proxylessnas, stamoulis2019singlepath, stamoulis2019singlepathautoml, Mei2020AtomNAS, Xu2020PC-DARTS} reduce the time for training a super-network and training and evaluating samples at the cost of sacrificing the support of non-differentiable search metrics. In summary, many existing works either have an unbalanced reduction in the time spent per step (i.e., optimizing some steps at the cost of a significant increase in the time for other steps), which still leads to a long \emph{total} search time, or are unable to support non-differentiable search metrics, which limits the performance of the discovered DNNs.

In this paper, we propose an efficient NAS algorithm, NetAdaptV2, to significantly reduce the \emph{total} search time by introducing three innovations to \emph{better balance} the reduction in the time spent per step while supporting non-differentiable search metrics:

\textbf{Channel-level bypass connections (mainly reduce the time for training and evaluating samples, Sec.~\ref{subsec:channel_level_bypass_connections})}: Early NAS works only search for DNNs with different numbers of filters (referred to as \emph{layer widths}). To improve the performance of the discovered DNN, more recent works search for DNNs with different numbers of layers (referred to as \emph{network depths}) in addition to different layer widths at the cost of training and evaluating more samples because network depths and layer widths are usually considered independently. In NetAdaptV2, we propose \emph{channel-level bypass connections} to merge network depth and layer width into a single search dimension, which requires only searching for layer width and hence reduces the number of samples.

\textbf{Ordered dropout (mainly reduces the time for training a super-network, Sec.~\ref{subsec:ordered_droput})}: We adopt the idea of super-network to reduce the time for training and evaluating samples. In previous works, \emph{each} DNN in the search space requires one forward-backward pass to train. As a result, training multiple DNNs in the search space requires multiple forward-backward passes, which results in a long training time. To address the problem, we propose \emph{ordered dropout} to jointly train multiple DNNs in a \emph{single} forward-backward pass, which decreases the required number of forward-backward passes for a given number of DNNs and hence the time for training a super-network.

\textbf{Multi-layer coordinate descent optimizer (mainly reduces the time for training and evaluating samples and supports non-differentiable search metrics, Sec.~\ref{subsec:optimizer}):} NetAdaptV1~\cite{eccv2018-netadapt} and MobileNetV3~\cite{Howard_2019_ICCV}, which utilizes NetAdaptV1, have demonstrated the effectiveness of the single-layer coordinate descent (SCD) optimizer~\cite{book2020sze} in discovering high-performance DNN architectures. The SCD optimizer supports both differentiable and non-differentiable search metrics and has only a few interpretable hyper-parameters that need to be tuned, such as the per-iteration resource reduction. However, there are two shortcomings of the SCD optimizer. First, it only considers one layer per optimization iteration. Failing to consider the joint effect of multiple layers may lead to a worse decision and hence sub-optimal performance. Second, the per-iteration resource reduction (e.g., latency reduction) is limited by the layer with the smallest resource consumption (e.g., latency). It may take a large number of iterations to search for a very deep network because the per-iteration resource reduction is relatively small compared with the network resource consumption. To address these shortcomings,  we propose the \emph{multi-layer coordinate descent (MCD) optimizer} that considers multiple layers per optimization iteration to improve performance while reducing search time and preserving the support of non-differentiable search metrics.

Fig.~\ref{fig:nas_comparison} (and Table~\ref{tab:nas_result}) compares NetAdaptV2 with related works. NetAdaptV2 can reduce the search time by up to $5.8\times$ and $2.4\times$ on ImageNet~\cite{imagenet_cvpr09} and NYU Depth V2~\cite{nyudepth} respectively and discover DNNs with better performance than state-of-the-art NAS works. Moreover, compared to NAS-discovered MobileNetV3~\cite{Howard_2019_ICCV}, the discovered DNN has $1.8\%$ higher accuracy with the same latency.



\begin{figure}[!tbh]
\centering
\includegraphics[width=\columnwidth, height=3.75cm]{Image/SEPSIS_final.pdf}
\vspace{-5mm}
\caption{The figure represents the categorization of the SEPSIS corpus across all layers. The 1$^{st}$ layer represents \textit{type of omission} and its respective categories, 2$^{nd}$ layer represents colors of lies, 3$^{rd}$ layer represents the intent of lies, and 4$^{th}$ layer represents the topic of lies. }
% An example for all classes across layers is written for ease of understanding. }
%\acil{Increase figure size after removing surrounding whitespace within the image.}
\label{fig:sepsis}
\vspace{-5mm}
\end{figure}
%\vspace{-2.5mm}

\section{Introducing SEPSIS: A novel corpus on \ul{lies of omission}}
\label{sec:introduction}
\vspace{-1.5mm}
We are delighted to introduce the \textbf{SEPSIS} corpus (\textbf{S}p\textbf{E}culation o\textbf{P}inion bia\textbf{S} d\textbf{I}\textbf{S}tortion), explicitly curated for \emph{lies of omission}. This novel resource will significantly enhance the study and analysis of deceptive communication by focusing on the deliberate exclusion of information. Figure \ref{fig:sepsis} offers a concise visual depiction that effectively summarizes the categorization we present in the SEPSIS. 
%We take great pride in presenting the SEPSIS corpus (SpEculation oPinion biaS dIStortion), a pioneering corpus specifically dedicated to lies of omission. This novel resource will enrich the study and analysis of deceptive communication by honing in on the deliberate omission of information.
%Figure ~\ref{fig:sepsis} provides a visual representation that succinctly summarizes the innovative categorization we have developed for the creation of SEPSIS. This novel categorization scheme plays a crucial role in organizing and classifying the corpus, facilitating a comprehensive understanding of the lies of omission. 
%In the subsequent paragraphs, we outline scientific questions that serve as motivations for our research. We elucidate how the pursuit of these questions has influenced the development of our annotation schema, which plays a fundamental role in the design of our research framework.
In the subsequent paragraphs, we present a collection of scientific inquiries along with their corresponding answers, which serve as the driving force behind our research. Furthermore, we delve into the influence of these questions on the development of our annotation schema, which lays the groundwork for our research framework.

\vspace{2mm}
\noindent
\textls[0]{\textbf{\ul{Is there a specific dialogue act that individuals employ for lies of omission?}} Within the classical switchboard corpus \cite{godfrey1992switchboard}, 
 %\ac{citations needed} 
there exist 42 well-defined dialogue acts. Following extensive deliberation and analysis, we have reached the conclusion that individuals often utilize dialogue acts such as \emph{speculation, opinion, bias, and distortions} when engaging in deceptive behavior. 
%Through rigorous discussion and debate, we derived a conclusion that dialogue acts commonly employed by individuals when they engage in deceptive behavior are \emph{speculation, opinion, bias, and distortions}. 
%These acts serve as figurative dialogue acts utilized by individuals to conceal their lies through encryption \cite{elaad2003effects} while they want to drop some information.
These dialogue acts function as figurative communication techniques employed by individuals to mask their deceit through encryption \cite{elaad2003effects}, particularly when they desire to disclose certain information selectively.}

\vspace{-3mm}
\begin{itemize}
[leftmargin=1mm]
\setlength\itemsep{0em}
\item[\ding{93}] \fontsize{9}{10}\selectfont{\textbf{Speculation} entails conjecturing without ample evidence.
%entails hypothesizing or .For example, utilizing verbs like "reports," "could," and "maybe" in a text can render it speculative.}
\vspace{-1mm}
\item[\ding{93}] \textbf{Opinion} is a subjective viewpoint formed without relying on factually accepted knowledge.
%Opinion} is a subjective viewpoint or assessment formed without relying on factual or widely accepted knowledge. Instances of statements appearing as opinions typically differ from person to person.
\vspace{-1mm}
\item[\ding{93}] \textbf{Bias} refers to unfair prejudice towards a particular individual or group. 
%refers to a predisposition or unfair prejudice towards a particular individual or group. Instances of bias can be classified based on factors such as race, ethnicity, religious beliefs, secularism, and so forth.
\vspace{-1mm}
\item[\ding{93}] \textbf{Distortion} is the act of twisting something away from its genuine, inherent, or initial condition.
%is the act of twisting or modifying something away from its genuine, inherent, or initial condition. It manifests in statements that appear superficial or illogical.

\vspace{-1mm}
\item[\ding{93}] Lastly, we define \textbf{sounds factual} as a statement that seems factual but may not be true.}
%It includes instances of information given by reports of claims by various organizations.}
\end{itemize}


\vspace{-7mm}
%%%%%%%%%%%%%
\begin{tcolorbox}[enhanced,attach boxed title to top right={yshift=-3mm,yshifttext=-1mm},
  colback=blue!5!white,colframe=blue!75!black,colbacktitle=red!80!black,
  title=$1^{st}$ level: type of omission,fonttitle=\bfseries,
  boxed title style={size=small,colframe=red!50!black},left=0pt, right=0pt]

  \begin{spacing}{0.8}
  \textbf{\ul{\footnotesize Speculation:}} {\fontfamily{lmss}\fontsize{8}{10}\selectfont{Biden warned the US does not have 'resources to win WW3' as tensions rise in the Middle East.}}
  \end{spacing}

  \vspace{-2.5mm}
  \DrawLine
  %\vspace{-2mm}
  
  \begin{spacing}{0.8}
  \textbf{\ul{\footnotesize Opinion:} }{\fontfamily{lmss}\fontsize{8}{10}\selectfont
 Poll: Trump receives low overall approval rating but praise for strong economy.}\end{spacing}
  \vspace{-2.5mm}
  \DrawLine
  \begin{spacing}{0.8}
  \textbf{\ul{\footnotesize Bias:} }{
  \fontfamily{lmss}\fontsize{8}{10}\selectfont
  Russia lauds India for following own interests on energy issue.}
  \end{spacing} 

  \vspace{-2.5mm}
  \DrawLine
  %\vspace{-2mm}
  
  \begin{spacing}{0.8}
  \textbf{\ul{\footnotesize Distortion:} }{\fontfamily{lmss}\fontsize{8}{10}\selectfont
  Republic TV: Jama Masjid in dark due to non-payment of electricity bills over four crores.}
  \end{spacing}

  \vspace{-2.5mm}
  \DrawLine
  %\vspace{-2mm}
  
  \begin{spacing}{0.8}
  \textbf{\ul{\footnotesize Sounds\, Factual:} }{\fontfamily{lmss}\fontsize{8}{10}\selectfont
  A US government study confirms most face recognition systems are racist.}\end{spacing} 
  \vspace{-0.75mm}
\end{tcolorbox}


%%%%%%%%%%%%%%%%%%%%%%


\vspace{-1mm}
\noindent
\textbf{\ul{What has been omitted?}} In the study of lies of omission, it is crucial to determine what information has been deliberately omitted. To address this, we draw inspiration from journalism, where the use of the 5W framework is common. The 5W framework consists of the questions \textit{who, what, when, where, and why} which are considered fundamental in information gathering and problem-solving. These questions are frequently utilized in journalism and police investigations \cite{10.2307/1023893, stofer2009sports, silverman, su2019study, smarts_2017,article_2023}. As an example:

\vspace{-1.5mm}  
\begin{tcolorbox}[left=0pt, right=0pt]
\small
\vspace{-1mm}
\{Hillary Clinton\}\textsubscript{\textcolor{blue}{who\textsubscript{1}}} announces \{Global Climate Resilience Fund\}\textsubscript{\textcolor{blue}{what}} for \{women\}\textsubscript{\textcolor{blue}{who\textsubscript{2}}}  
to\{tackle climate change\}\textsubscript{\textcolor{blue}{why}}
\vspace{-2.5mm}
\end{tcolorbox} 

\noindent
\textbf{\ul{What is the vulnerability of the uttered lie?}} In the realm of deception research, it is of utmost importance to comprehend and quantify the susceptibility of lies. One approach involves categorizing lies into different colors, namely \emph{black, red, white, and gray} \cite{ratliff2011behavioral,depaulo2004many}. Each color represents a distinct type of lie with varying levels of vulnerability, as detailed below:

%%%%%%%%%%%%%%%%%%%%%%%%%%%%%%%%%%%%




\vspace{-3mm}
\begin{itemize}
[leftmargin=1mm]
\setlength\itemsep{0em}
\item[\ding{93}] \fontsize{9}{10}\selectfont{A \textbf{black lie} is about simple and callous selfishness. Typically uttered when there is no benefit to others, its sole intention is to extricate oneself from trouble.}
\vspace{-1.5mm}
\item[\ding{93}] A \textbf{white lie} prioritizes others' welfare over personal interests, reflecting an altruistic nature.

% is characterized by its altruistic nature as it prioritizes the welfare of others, even if it entails sacrificing one's own interests to some extent. is one which is altruistic as it seeks to help others first, even at some cost to oneself.
\vspace{-1mm}
\item[\ding{93}] \textbf{Gray lies} exhibits dual behavior, partially benefiting others and partially benefiting oneself depending on the viewpoint. 
%They are partially told to help others and partially to help oneself, depending on the perspective of the sentence.
\vspace{-1mm}
\item[\ding{93}] \textbf{Red lies} are spoken from a hatred and revenge perspective against individuals or groups. 
%These are driven by the motive to harm others even at the expense of harming oneself.
\end{itemize}

\vspace{-7mm}
\begin{tcolorbox}[enhanced,attach boxed title to top right={yshift=-3mm,yshifttext=-1mm},
  colback=blue!5!white,colframe=blue!75!black,colbacktitle=red!80!black,
  title=$2^{nd}$ level: colors of lie,fonttitle=\bfseries,
  boxed title style={size=small,colframe=red!50!black},left=0pt, right=0pt ]

  \begin{spacing}{0.8}\textbf{\ul{\footnotesize Red:} }{\fontfamily{lmss}\fontsize{8}{10}\selectfont
  Donald Trump's congratulatory post for North Korea's WHO membership sparks outrage and controversy.}
  \end{spacing} 

  \vspace{-2.5mm}
  \DrawLine
  %\vspace{-2mm}

  \begin{spacing}{0.8}
  \textbf{\ul{\footnotesize Black:} }{\fontfamily{lmss}\fontsize{8}{10}\selectfont
FTX collapse: Former CEO Sam Bankman-Fried urges court to toss charges.}
  \end{spacing}

  \vspace{-2.5mm}
  \DrawLine
  %\vspace{-2mm}

  \begin{spacing}{0.8}
  \textbf{\ul{\footnotesize White:} }{\fontfamily{lmss}\fontsize{8}{10}\selectfont 
% After Hope Hicks told the House Intelligence Committee that she has told white lies on behalf of President Trump, news organizations around the world made this admission a top story.
% another example
% That day, Michael Jordan’s mother changed Nike’s history forever: ‘Even if you don’t like it, you’re going to listen to them’!
An apple a day slashes frailty risk by 20 percent, but Study points otherwise.
}
  \end{spacing} 

  \vspace{-2.5mm}
  \DrawLine
  %\vspace{-2mm}
  
  \begin{spacing}{0.8}
  \textbf{\ul{\footnotesize Gray:} }{\fontfamily{lmss}\fontsize{8}{10}\selectfont
Hillary Clinton Announces Global Climate Resilience Fund For Women To Tackle Climate Change.}
  \end{spacing}
  
  \vspace{-1mm}
\end{tcolorbox}



\begin{comment}
\begin{table}[ht]
\resizebox{\columnwidth}{!}{
\begin{tabular}{|llll|}
\hline
\multicolumn{4}{|c|}{\cellcolor[HTML]{FFCCC9}\textbf{SEPSIS at a glance}}  \\ \hline
\multicolumn{1}{|l|}{\textbf{\begin{tabular}[c]{@{}l@{}}Data\\ Sources\end{tabular}}} & \multicolumn{1}{l|}{\textbf{No. of datapoints}} & \multicolumn{1}{l|}{\textbf{Number of Paraphrases}} & \textbf{\begin{tabular}[c]{@{}l@{}}Number of augmented data points \\ through prompt engineering\end{tabular}} \\ \hline
\multicolumn{1}{|l|}{\textbf{Times of India}}                                         & \multicolumn{1}{l|}{2333}                       & \multicolumn{1}{l|}{43344}                          & 23567                                                                                                          \\ \hline
\multicolumn{1}{|l|}{\textbf{ISOT dataset}}                                           & \multicolumn{1}{l|}{2245}                       & \multicolumn{1}{l|}{34569}                          & 58902                                                                                                          \\ \hline
\multicolumn{1}{|l|}{\textbf{Total}}                                                  & \multicolumn{1}{l|}{}                           & \multicolumn{1}{l|}{}                               &                                                                                                                \\ \hline
\end{tabular}
}
\caption{SEPSIS corpus statistics}
\label{tab:SEPSIS_corpus}
\end{table}

\end{comment}


\vspace{-2mm}
\noindent
\textbf{\ul{What is the intent of the lie?}}
%\textcolor{red}{we need a text here and associated intent of lies examples}
%It is important to study the intent of lies to understand what certain deceptive texts intend to achieve. 
Studying the intent of lies helps to comprehend deceptive language's objectives. We have thus categorized lies into different intents as shown below. 

%So we followed an extensive approach to categorize lies into different intents such as \emph{gaining advantage, protecting themselves}, etc.

\vspace{-3mm}
\begin{tcolorbox}[enhanced,attach boxed title to top right={yshift=-3mm,yshifttext=-1mm},
  colback=blue!5!white,colframe=blue!75!black,colbacktitle=red!80!black,
  title=$3^{rd}$ level: intent of lie,fonttitle=\bfseries,
  boxed title style={size=small,colframe=red!50!black},left=0pt, right=0pt ]
  
\begin{spacing}{0.8}
\textbf{\ul{\footnotesize Gaining\,Advantage:} }{\fontfamily{lmss}\fontsize{8}{10}\selectfont
  Elizabeth Holmes ordered dinners for Theranos staff but made sure they weren't delivered until after 8 p.m. so they worked late: book.
  }
\end{spacing}  

  \vspace{-2.5mm}
  \DrawLine
  %\vspace{-2mm}

  \begin{spacing}{0.8}
  \textbf{\ul{\footnotesize Protecting\,Themselves:} }{\fontfamily{lmss}\fontsize{8}{10}\selectfont
ChatGPT creator Sam Altman testifies to US Congress on AI risks.}
  \end{spacing} 

  \vspace{-2.5mm}
  \DrawLine
  %\vspace{-2mm}

  \begin{spacing}{0.8}
  \textbf{\ul{\footnotesize Avoiding\,Embarrassment:} }{\fontfamily{lmss}\fontsize{8}{10}\selectfont 
Trump’s Suggestion That Disinfectants Could Be Used to Treat Coronavirus Prompts Aggressive Pushback, was Sarcastic?}
  \end{spacing} 

  \vspace{-2.5mm}
  \DrawLine
  %\vspace{-2mm}

  \begin{spacing}{0.8}
  \textbf{\ul{\footnotesize Gaining\,Esteem:} }{\fontfamily{lmss}\fontsize{8}{10}\selectfont
Sasan Goodarzi, the CEO of software giant Intuit, which has avoided mass layoffs, says tech firms axed jobs because they misread the pandemic.}
  \end{spacing} 

  \vspace{-2.5mm}
  \DrawLine
  %\vspace{-2mm}

  \begin{spacing}{0.8}
 \textbf{\ul{\footnotesize Protecting\,Others:} }{\fontfamily{lmss}\fontsize{8}{10}\selectfont 
Nobel Laureate Malala Urges U.S. To Bolster Support For Afghan Girls, Women!}
  \end{spacing}
  
  \vspace{-2.5mm}
  \DrawLine
  %\vspace{-2mm}

\begin{spacing}{0.8}
 \textbf{\ul{\footnotesize Defaming\,Esteem:} }{\fontfamily{lmss}\fontsize{8}{10}\selectfont 
Taiwan war would be ‘devastating,’ warns US Defense Secretary Lloyd Austin as he criticizes China at Shangri-La security summit.}
  \end{spacing}
  \vspace{-1mm}
\end{tcolorbox}

\vspace{-3mm}

\vspace{-2mm}
\begin{itemize}
[leftmargin=1mm]
\setlength\itemsep{0em}
\item[\ding{93}] \fontsize{9}{10}\selectfont{\textbf{Intent of Gaining Advantage} can be used as an act of intentionally providing false information or misleading others to gain an unfair advantage over them. 
%It involves distorting facts or creating a false narrative to secure personal or professional benefits at the expense of others.
\vspace{-1mm}
\item[\ding{93}] \textbf{Intent of Protecting Themselves} can be used as a means of self-preservation or self-defense when an individual feels threatened or vulnerable.
%It results in withholding or distorting information to shield oneself from potential harm, consequences, or legal implications. 
\vspace{-1 mm}
\item[\ding{93}] \textbf{Intent of Avoiding Embarrassment} can be employed to evade situations that may lead to embarrassment, humiliation, or social discomfort.
%These can involve providing false information to avoid potential judgment with the intention to maintain a positive self-image.
\vspace{-1mm}

\item[\ding{93}] \textbf{Intent of Gaining Esteem} can be utilized to enhance one's reputation, social status, or personal image.

%It requires fabricating hyperbole achievements, qualities, or experiences to gain admiration, respect, or validation from others. The aim is to elevate oneself in the eyes of others and be perceived in a more favorable light.

\item[\ding{93}] \textbf{Intent of Protecting Others} can be used as a means of preservation for others when a group or community feels threatened or vulnerable.
%It results in withholding or distorting information to shield it from potential harm, consequences, or legal implications and benefit others. 
\item[\ding{93}] \textbf{Intent of Defaming Esteem} intends to damage reputation by spreading false information or rumors.}
%aims at defaming someone's esteem and involves spreading false information or rumors to damage an individual's reputation or public perception. It is motivated by a desire to discredit or tarnish someone's image.


\end{itemize}


\vspace{-2mm}


\noindent
\textbf{\ul{What is the topic of lie?}}
To study deception further and to understand its topical influence, this research categorizes different topics of lies such as political, educational, etc.

\vspace{-3mm}
\begin{itemize}
[leftmargin=1mm]
\setlength\itemsep{0em}

\item[\ding{93}] \fontsize{9}{10}\selectfont{\textbf{Political} deception occurs by the deliberate use of statements by political entities to manipulate public opinion.

%gain power, or advance their agenda. It involves spreading misleading promises or distorting facts for political gain.}
\vspace{-1mm}
\item[\ding{93}] \fontsize{9}{10}\selectfont{\textbf{Educational} deception occurs by the deliberate use of statements by academic entities to manipulate opinion, directed especially towards the younger population.
%It involves spreading misleading policies for economic gain.
\vspace{-1mm}
\item[\ding{93}] \textbf{Racial} deception occurs when individuals intentionally misrepresent their racial identity or engage in deception driven by racial motives.}
%This can encompass falsely asserting a different racial heritage, manipulating others through racial stereotypes, or partaking in impersonation based on race.
\vspace{-1mm}
\item[\ding{93}]\textbf{Religious} deception involves the act of deceiving others by misrepresenting one's religious beliefs.}

%, for engaging in fraudulent practices and personal gain.
\vspace{-1mm}
\item[\ding{93}] \textbf{Ethnic} deception refers to the act of intentionally manipulating one's ethnic identity by targeting specific ethnic groups.
%and exploiting stereotypes, and cultural appropriation for personal or political reasons.
\end{itemize}


\vspace{-6mm}
\begin{tcolorbox}[enhanced,attach boxed title to top right={yshift=-3mm,yshifttext=-1mm},
  colback=blue!5!white,colframe=blue!75!black,colbacktitle=red!80!black,
  title=$4^{th}$ level: topic of lie,fonttitle=\bfseries,
  boxed title style={size=small,colframe=red!50!black},left=0pt, right=0pt ]
  
\begin{spacing}{0.8}
  \textbf{\ul{\footnotesize Political:} }{\fontfamily{lmss}\fontsize{8}{10}\selectfont
 No elections safe from AI, deep fake photos, videos of politicians to become common, warns former Google boss.}\end{spacing}

  \vspace{-2.5mm}
  \DrawLine

  \begin{spacing}{0.8}
  \textbf{\ul{\footnotesize Educational:} }{\fontfamily{lmss}\fontsize{8}{10}\selectfont 
Hundreds gather at Florida school board meeting over Disney movie controversy: 'Your policies are not protecting us from anything.}\end{spacing} 

  \vspace{-2.5mm}
  \DrawLine

  \begin{spacing}{0.8}
  \textbf{\ul{\footnotesize Religious:} }{\fontfamily{lmss}\fontsize{8}{10}\selectfont
Pope: Christianity, Islam share common commitment to good life.}\end{spacing}

  \vspace{-2.5mm}
  \DrawLine

  \begin{spacing}{0.8}
  \textbf{\ul{\footnotesize Racial:} }{\fontfamily{lmss}\fontsize{8}{10}\selectfont
  Why shouldn’t a mixed-race actress play Egyptian queen Cleopatra?}\end{spacing}

  \vspace{-2.5mm}
  \DrawLine

  \begin{spacing}{0.8}
 \textbf{\ul{\footnotesize Ethnicity:} }{\fontfamily{lmss}\fontsize{8}{10}\selectfont 
Egyptians complain over Netflix depiction of Cleopatra as black.}\end{spacing}
  
  \vspace{-1mm}
  
\end{tcolorbox}
\vspace{-2mm}

%%%%%%%%%%%%%%%%%%%%%


\section{SEPSIS: Data Sources, Annotation, and Agreement}
\vspace{-1mm}  
At the outset, we engaged in the crowd annotation of 5,000 sentences through Amazon Mechanical Turk (AMT), employing four layers of deception. Subsequently, we applied data augmentation techniques as detailed in section ~\ref{sec:data_augmentation}, culminating in a total of 8,76,784 data points.

\subsection{Data Sources}
\vspace{-0.5mm}  

\begin{table*}[ht]
\centering
\resizebox{\textwidth}{!}{%
\begin{tabular}{lccccccccccccccccccccccccc}
\toprule
                                        & \multicolumn{5}{c}{\textbf{Lies of omission}}                                                           & \multicolumn{4}{c}{\centering \textbf{Color of lies}}             & \multicolumn{6}{c}{\textbf{Intent of Lies}}        
                                        % & \multicolumn{6}{c}{\textbf{Traits of lies}}
                                        
                                        \\
                                        % \midrule
                                        \toprule
                                        
                                        
                                        & \multicolumn{1}{p{1.2cm}}{\centering \textbf{Specula-}
                                        \textbf{tion}
                                        }                                        

                                        % \multicolumn{1}{p{2.2cm}}{\centering \textbf{Avoiding} \textbf{Embarrassment}}
                                        
                                        
                                        & \textbf{Bias}
                                        & 
                                        \multicolumn{1}{p{1cm}}{\centering \textbf{Distor-}
                                        \textbf{tion}
                                        }   
                                        
                                        & \multicolumn{1}{p{1.35cm}}{\centering \textbf{Opinion}} 
                                        & \multicolumn{1}{p{1.35cm}}{\centering \textbf{Sounds}
                                        \\ \textbf{Factual}}
                                        
                                        & 
                                        
                                        \textbf{Black} & \textbf{White} & \textbf{Grey} & \textbf{Red} 
                                        
                                        & 
                                        
                                        \multicolumn{1}{p{1.5cm}}{\centering \textbf{Gaining} \\ \textbf{Advantage}} & 
                                        \multicolumn{1}{p{1.6cm}}{\centering \textbf{Protecting} \\ \textbf{Themselves}} & 
                                        \multicolumn{1}{p{2.2cm}}{\centering \textbf{Avoiding} \textbf{Embarrassment}} & \multicolumn{1}{p{1.3cm}}{\centering \textbf{Gaining} \\ \textbf{Esteem}} & \multicolumn{1}{p{1.4cm}}{\centering \textbf{Protecting} \\ \textbf{Others}} 
                                        & \multicolumn{1}{p{1.4cm}}{\centering \textbf{Defaming} \\ \textbf{Esteem}}

                                        
                                        \\  \hline 
% \multicolumn{1}{l}{\textbf{Tweet}}     & 0.598                   & 0.552            & 0.539                  & 0.54               & 0.679                      & 0.771             & 0.727             & 0.691            & 0.766           & 0.710        & 0.672        & 0.612        & 0.684        & 0.557        & 0.529        

% % & 0.840        & 0.712        & 0.772        & 0.636        & 0.592
% \\

% \multicolumn{1}{l}{\textbf{Fake News}} &  0.639 &
% 0.581 & 0.603 & 0.513 & 0.647 &  0.798 & 0.765 & 0.715 
% & 0.812   & 0.679 & 0.73 & 0.658 & 0.597 & 0.617 & 0.616


\multicolumn{1}{l}{\textbf{Tweet}} &
\cellcolor{yellow!50}0.678 &
\cellcolor{yellow!50}0.632 &
\cellcolor{yellow!50}0.619 &
\cellcolor{yellow!50}0.62 &
% {\color{colorname}Text to be colored}
\cellcolor{blue!50}{\textcolor{white}{0.759}} &
\cellcolor{green!50}0.831 &
\cellcolor{green!50}0.807 &
\cellcolor{blue!50}{\textcolor{white}{0.771}} &
\cellcolor{green!50}0.846 &
\cellcolor{blue!50}\textcolor{white}{0.790} &
\cellcolor{blue!50}{\textcolor{white}{0.752}} &
\cellcolor{yellow!50}0.692 &
\cellcolor{blue!50}{\textcolor{white}{0.744}} &
\cellcolor{yellow!50}0.637 &
\cellcolor{yellow!50}0.609 \\

\multicolumn{1}{l}{\textbf{Fake News}} &
\cellcolor{blue!50}{\textcolor{white}{0.719}} &
\cellcolor{yellow!50}0.661 &
\cellcolor{yellow!50}{0.683} &
\cellcolor{yellow!50}0.603 &
\cellcolor{blue!50}{\textcolor{white}{0.727}} &
\cellcolor{green!50}0.878 &
\cellcolor{green!50}0.845 &
\cellcolor{green!50}0.811 &
\cellcolor{green!50}0.892 &
\cellcolor{blue!50}{\textcolor{white}{0.759}} &
\cellcolor{green!50}0.81 &
\cellcolor{blue!50}{\textcolor{white}{0.738}} &
\cellcolor{yellow!50}0.677 &
\cellcolor{blue!50}{\textcolor{white}{0.709}} &
\cellcolor{yellow!50}{0.681} \\

% & 0.823        & 0.754        & 0.715        & 0.678  
% & 0.649


\bottomrule   
\end{tabular}%
} 
\vspace{-1mm}
\caption{Kappa score representation for layer 1: \textit{type of omission} %(\emph{speculation, bias, Distortion, opinion, sounds factual}), 
layer 2: \textit{colors of lies}, 
%(\emph{Black, White, grey, and red}), 
and layer 3: \textit{Intent of lies}.
%(\emph{Gaining Advantage, Protecting themselves, Avoiding Embarrassment, Gaining esteem, Protecting others and Defaming esteem}).
Kappa score for the layer 4 topic of lies
%(\emph{religious, educational, racial, ethnicity, and political} is present) 
can be found in Appendix \ref{sec: Data Annotation}.}
\vspace{-3.5mm}
\label{tab: Kappa score}
\end{table*}

In terms of data sources, we have identified two distinct categories of interest. The first category focuses on the presence of omissions in factual data, specifically news data. The second category examines the involvement of omissions in fake news data. To address these categories, we have selected data sources from two prominent outlets: (a) Times of India \cite{timesofindia} Twitter handle, the renowned news agency in India, and (b) Information Security and Object Technology (ISOT) fake news dataset \cite{ISOTFakeNewsDataset}. More information on these sources can be found in the appendix \ref{sec:data sources}. A detailed analysis of the SEPSIS corpus and the results can be found in Appendix \ref{sec: data analysis}.
%Data acquisition is done primarily from two sources Times of India \cite{} which is a popular Indian media house and a popular fake news data set. \cite{}. It is a compilation of several thousand fake news and truthful articles, obtained from different legitimate news sites and sites flagged as unreliable by \emph{"Politifact.com"}. From this dataset a sample of 2605 datapoint is taken from the one that is flagged as fake news.


%\url{https://onlineacademiccommunity.uvic.ca/isot/wp-content/uploads/sites/7295/2023/02/ISOT_Fake_News_Dataset_ReadMe.pdf}

\vspace{-1mm}
\subsection{Data Annotation}
\vspace{-1mm}  
% For annotation we chose to utilize the Amazon Mechanical Turk (AMT) service. This platform offers a quick and cost-effective solution for annotating large-scale datasets, although it is important to acknowledge its susceptibility to errors.
We have chosen to leverage the Amazon Mechanical Turk (AMT) service for annotation purposes, which provides a rapid and cost-effective solution for annotating large-scale datasets, albeit acknowledging its susceptibility to errors. To ensure the annotation quality, we implemented rigorous checks and measures throughout the entire annotation process. The dataset was annotated at the sentence level using a multi-class annotation approach, allowing each individual feature to be assigned multiple categories during the annotation process. For instance, a statement could be tagged as both speculative and sounding factual, recognizing the possibility for it to either be a verifiable fact or contain speculative elements that satisfy both possibilities. A comprehensive account of the overall annotation process is provided in Appendix \ref{sec: data cleaning}. Notably, during the initial layer of annotation, if a particular text appeared to be factual, we refrained from annotating the specific type, intent, and influence of the lie since it was treated as a fact.

\subsection{Inter Annotator Agreement and Quality}
\label{sec:iaa_score}
To ensure quality control in the AMT annotations, we performed in-house annotation on 1000 data points. This in-house dataset was utilized to assess the quality of annotations provided by individual annotators. Based on this assessment, we selected the appropriate set of annotators on the AMT platform. For the annotation task on AMT, we offered a compensation of \$1 for each sentence annotation across all the four layers.

We obtained four annotations per sentence and subsequently consolidated the data using an improved voting technique, as suggested in \cite{hovy-etal-2013-learning}, which has been empirically shown to outperform majority voting. To assess the level of agreement in the annotated corpus, we also calculated the Cohen Kappa score \cite{cohen1960coefficient}. Since there are multiple categories for a given sentence, we report class-wise agreement scores. The overall agreement score is presented in Table \ref{tab: Kappa score}. An overview of data points is presented in Table \ref{tab:SEPSIS_corpus}. To understand how features across these four layers are dependent on each other we present six heatmaps in Appendix \ref{sec: data analysis}.



%The comprehensive examination of various categories of lies reveals a complex and nuanced understanding, posing challenges in the annotation process. Consequently, the Lies of Omission category received relatively lower scores. Conversely, the intent of lie and color of lies categories proved comparatively more accessible for comprehension, interpretation, and detection, leading to higher scores in this layer. Taking into account the inherently subjective nature of the matter at hand, we deem our inter-annotation scores to be fairly commendable.

%\textcolor{red}{we need some text to describe what to take away from agreement scores.} \textcolor{green}{Is this okay?}

\begin{table}[!ht]
\centering
\vspace{-1mm}
\resizebox{0.95\columnwidth}{!}{%
\begin{tabular}{cccccl}
% \toprule
% \multicolumn{6}{c}{\cellcolor[HTML]{FFFFFF}\textbf{SEPSIS Corpus Statistics}}                                                                                                                                                                                                                                                                                                                                                                                                                                                                \\ 
\toprule
\textbf{Data Source} & \multicolumn{1}{c}{\textbf{\begin{tabular}[c]{@{}c@{}}Sentences\end{tabular}}} & \multicolumn{1}{c}{\textbf{\begin{tabular}[c]{@{}c@{}}+  Paraphrasing\end{tabular}}} & \multicolumn{1}{c}{\textbf{\begin{tabular}[c]{@{}c@{}}+  Mask Infilling\end{tabular}} } \\ \toprule
\textbf{Tweets}       &       $2495$                                                                                       &                    $12475$                                                                                                          &      $389105$                                                                                                                            &                                                                                                                                                   \\
\textbf{Fake News}   &      $2605$                                                                                        &     $13025$                                                                                                                         &    $487829$                                                                                                                              &                                                                                                                                                   \\ \midrule
\textbf{Total}       &       $5100$                                                                                       &      $25500$                                                                                                                        &     $876784$                                                                                                                             &                                                                                                                                                   \\ \bottomrule
\end{tabular}
}
%\caption{SEPSIS corpus statistics represents the count of data points across tweets and fake news. The number of sentences represents the original data point that was scrapped and annotated directly. The \textit{+paraphrasing} and \textit{+mask infilling} columns denote the count of unique augmented data points using the respective techniques.

\caption {Number of original sentences and augmented sentences using \textit{paraphrasing} and \textit{mask infilling}.}
\label{tab:SEPSIS_corpus}
\vspace{-5mm}
\end{table}

\begin{comment}
We addressed several issues in our dataset to improve the reliability of the annotations and reduce redundancies. Many data points were assigned the correct reason behind their choice but were assigned the wrong category. For example, certain statements were placed under the 'distortion' category, but the provided reason suggested their categorization to be 'speculation.' Additionally, we re-annotated several incomplete entries where annotators had entered the statement category but missed providing further information on the color of lies or intent of lies.
\end{comment}



%\subsection{SEPSIS Corpus - statistics}

\section{Data Augmentation}
\vspace{-1mm}
\label{sec:data_augmentation}
It is widely acknowledged that neural network-based techniques have a high demand for data. To address this data requirement, data augmentation has almost become a standard practice in the AI community \cite{van2001art,shorten2021text,liu2020survey}. 
%\ad{we need lot of citations here to support this statement, specific citations from NLP}. 
We have utilized three methods for data augmentation here: (i) paraphrasing, (ii) 5W masking followed by infilling \cite{gao-etal-2022-mask}.

\subsection{Paraphrasing Deceptive Datapoints}
\vspace{-1mm}
% The motivation behind paraphrasing deceptive data is as follows. Textual deceptive data may appear in various different textual forms in real life, owing to variations in the writing styles of different news publishing houses. Incorporating such variations is essential to developing a strong benchmark to ensure a holistic evaluation 
The motivation for paraphrasing deceptive data stems from the diverse manifestations of textual deceptive content in real-world scenarios, often influenced by variations in writing styles among different news publishing outlets. It is vital to incorporate these variations in order to establish a robust benchmark that facilitates comprehensive evaluation and analysis (cf. Figure \ref{fig: paraphrase} in Appendix \ref{sec:paraphrase-evaluation} for examples).
 
Undoubtedly, manual generation of possible paraphrases is ideal; however, this process is time-consuming and labor-intensive. On the other hand, automatic paraphrasing has garnered significant attention recently \cite{niu2020unsupervised, nicula2021automated, witteveen2019paraphrasing, nighojkar2021improving}. We used GPT-3.5 \cite{brown2020language} (specifically the \textit{text-davinci-003} variant) \cite{brown2020language} model as it generates linguistically diverse, grammatically correct, and a maximum number of considerable paraphrases, i.e., 5 in this case. This is the best-performing model for data augmentation using paraphrasing \cite{rani2023factify5wqa}. Additionally, we conducted experiments with Pegasus \cite{zhang2020pegasus} and T5 (T5-Large) \cite{raffel2020exploring} models, but GPT-3.5 (\texttt{text-davinci-003} variant) \cite{brown2020language} outperformed them, as indicated in Appendix \ref{sec:paraphrase-evaluation}. We gathered a total of 25,500 unique paraphrased deceptive data points through this method. 
%by inputting the data into GPT-3.5. and prompting it to generate five paraphrases. 
% Upon cross-checking the generated data points, all paraphrases are unique.

At this stage, several important questions arise: (i) \emph{What is the accuracy of the paraphrases generated?} (ii) \emph{How do they differ from or distort the original content?} To address these questions, we have conducted extensive experiments and obtained empirical answers. However, due to space limitations, please refer to Appendix \ref{sec:paraphrase-evaluation} for details of our experiments and conclusions. We have evaluated the paraphrase modules based on three key dimensions: \textit{(i) \ul{Coverage}: number of considerable paraphrase generations, (ii) \ul{Correctness}: correctness of these generations, and (iii) \ul{Diversity}: linguistic diversity in these generations}.



\subsection{Synthetic Data Augmentation using 5W Specific Mask Infilling}
\vspace{-1mm}
As mentioned previously in section ~\ref{sec:introduction}, our hypothesis revolves around the possible omission of the 5W (who, what, when, where, and why) for deceits. With this in mind, we developed a pipeline to detect the presence of the 5W and subsequently replace them with deceptive/null information generated from a language model. In the subsequent subsections, we will present our methodology for designing 5W semantic role labeling and mask filling techniques to address 5W omission.

\begin{comment}
\begin{tcolorbox}[colback=blue!5!white,colframe=blue!75!black,title=Data augmentation pipeline]
We achieved this task in the following steps:

\begin{enumerate}
    \item STEP 1: Paraphrase the tweets and the claims, generating 5 paraphrases for each using GPT3 Text Da Vinci Model

    \item STEP 2: Pass the paraphrased texts through the AllenNLP model to generate the 5Ws for each

    \item STEP 3: MASK the Ws and generate top 3 words for each masked W.

    \item STEP 4: Replace the Ws with the words and generate new sentences with each prompt.

    \item STEP 5: List all the newly generated sentences in a single file.
    
\end{enumerate}
\end{tcolorbox}
\end{comment}

\vspace{-1mm}
\begin{figure}[!tbh]
\vspace{-1mm}
\centering
\includegraphics[width=1\columnwidth, trim={0cm 0cm 0cm 0cm}]{Image/Prompt_vizv7.pdf}
\vspace{-6mm}
\caption{Architecture representation for the process of leveraging mask infilling using RoBERTa \cite{liu2019roberta} for creating the deception dataset.}
\label{fig: architecture_MaskInfilling}
\vspace{-2.5mm}
\end{figure}
% \vspace{-2mm}

\noindent
\textbf{5W Semantic Role Labeling:}
Identification of the functional semantic roles played by various words or phrases in a given sentence is known as semantic role labeling (SRL). SRL is a well-explored area within the NLP community. There are quite a few off-the-shelf tools available: (i) Stanford SRL \cite{manning2014stanford}, (ii) AllenNLP \cite{allennlpsrl}, etc. A typical SRL system initially identifies the verbs in a given sentence and subsequently associates all the related words/phrases with the verb through relational projection, assigning them appropriate roles. Thematic roles are generally marked by standard roles defined by the Proposition Bank (generally referred to as PropBank) \cite{palmer2005proposition}, such as: \textit{Arg0, Arg1, Arg2}, and so on. We propose a mapping mechanism to map these PropBank arguments to 5W semantic roles (look at the conversion table \ref{tab:5w-map-SRL}, in appendix).


\begin{comment}
Semantic role labelling (SRL) is a natural language processing technique that involves identifying the functions of different words or phrases in a sentence. This helps to determine the meaning of the sentence by revealing the relationships between the entities in the sentence. For example, in the sentence "\textit{Moderna’s lawsuits against Pfizer-BioNTech show COVID-19 vaccines were in the works before the pandemic started,}" \textit{Moderna} would be labeled as the \textit{agent} and \textit{Pfizer-BioNTech} would be labelled as the \textit{patient}.


Using the generated paraphrases, we identify 5Ws(Who, What, When, Where, Why) using the mapping between Semantic Role Labels and 5Ws as explained by \cite{rani2023factify5wqa} \cite{}. A typical SRL system first identifies verbs in a given sentence  and then marks all the related words/phrases haven relational projection with the verb and assigns appropriate roles. Thematic roles are generally marked by standard roles defined by the Proposition Bank (generally referred to as PropBank) \cite{palmer2005proposition}, such as: \textit{Arg0, Arg1, Arg2}, and so on. A mapping mechanism to map these PropBank arguments to 5W semantic roles is described in the conversion table \ref{tab:5w-map-SRL}.
\end{comment}


\begin{comment}
After the mapping is done, a detailed analysis for the presence of each of the 5W is conducted which is summarized in figure \ref{fig: 5W_presence}.




We have leveraged 4 masking LLMs, (bert-base-uncased \cite{}, roberta \cite{}, XLNet \cite{} and ELECTRA \cite{}) to mask and replace the 5Ws generated by the AllenNLP model for the paraphrased sentences.

We have used “fill-mask” to get the output. Fill-mask is a technique within the Hugging Face Transformers Library that allows masked language modeling tasks. It is used in some NLP models to predict missing words in a sentence. In this approach, the model is given a sentence with one or more masked words, represented by special tokens such as "[MASK]", and the goal is to predict the most likely word(s) to fill in the blanks.

The top three generations from each of these models were considered for final data augmentation. The selection of the top three words was based on a metric called x.
\end{comment}

\noindent
\textbf{5W Slot Filling:} Building upon our hypothesis, it is plausible for individuals to deliberately omit any of the given W to transform a statement into a lie of omission. Therefore, once we detect the presence of the Ws, our objective is to generate variations of the original statement by selectively omitting specific Ws. For this purpose, we train a masked LLM as depicted in the Figure \ref{fig: architecture_MaskInfilling}. For the 5W slot-filling task we have experimented with five models: (i) MPNet \cite{song2020mpnet}
, (ii) ELECTRA \cite{clark2020electra},
(iii) RoBERTa \cite{liu2019roberta}, (iv) ALBERT \cite{lan2019albert}, and (v) BERT \cite{devlin2018bert}.

RoBERTa \cite{liu2019roberta}, a language model that leverages large-scale pre-training and removes the next sentence prediction objective, significantly enhancing language understanding. With its transformer architecture and fine-tuning, it predicts the original masked tokens in an \textit{input sequence X} by maximizing the likelihood of the true masked tokens given the predicted \textit{probabilities P}. Considering the scenario where all the Ws are present in a sentence, it is feasible to generate five variations. At this juncture, a crucial question arises: is there a high likelihood that the generated sentences deviate substantially from the original deceptive input? To substantiate we have calculated BLEU \cite{papineni2002bleu} score and MoverScore \cite{zhao-etal-2019-moverscore} between the original input and all the perturbed generations, reported in Table~\ref{tab:Evaluation_MaskInfilling}. 










%In this scenario, when a W is masked and passed to the model, it fills the corresponding span with content that would potentially generate deceptive data. This approach allows us to achieve our objective of generating lies of omission and performed the best based on the \aacb{BLEU score described in Table \ref{tab:Evaluation_MaskInfilling}.}
%Through our observations, we have found that pronouns and elliptic references are frequently used as fillers. This approach allows us to achieve our objective of generating lies of omission.


% \vspace{-2mm}
\begin{table}[h]
\centering
\vspace{-1mm}
\resizebox{0.95\columnwidth}{!}{
\begin{tabular}{lcc}
\toprule
\multicolumn{1}{l}{\textbf{Model}} & \textbf{BLEU Score} & \textbf{MoverScore} \\ \toprule
RoBERTa-base                       & \bf{0.7457}  & \bf{0.7216}          \\ 
MPNet-base               & 0.7329       & 0.7194     \\ 
ELECTRA-large-generator     & 0.7225   &0.7129           \\ 
BERT-base-uncase                   & 0.7222   &0.712           \\ 
ALBERT-large-v2                    & 0.7116    &0.703          \\ 
\bottomrule
\end{tabular}}
\vspace{-1mm}
\caption{BLEU Score for various models for mask infilling. RoBERTa performed the best.}
\label{tab:Evaluation_MaskInfilling}
\vspace{-5mm}
\end{table}
\vspace{-2mm}



\begin{comment}
\subsection{Synthetic Data Augmentation using Prompt Engineering}
Prompt engineering is the practice of creating powerful prompts in order to create and optimize natural language processing (NLP) models. For augmenting data in this study, we need to come up with a prompt that could generate deceptive datapoint. We use PromptScource(Bach et al., 2022) from the BigScience community to use the existing prompts or to create new prompts. PromptSource is a toolkit for creating, sharing, and using natural language prompts. Prompts are saved in standalone structured files and written in a simple templating language called Jinja. We leverage
available English prompts on PromptSource and came up with a list of the following prompts.

\begin{itemize}

 \item a
 \item b
 \item c

\end{itemize}

We used these manually selected prompts on four GPT models (GPT-2 \cite{}, GPT-3 \cite{}, GPT-3.5 \cite{}, GPT-4 \cite{}) to generate more data points.

Data augmentation using prompt engineering generated a total of additional 2,20,678 data points.
\end{comment}
% \vspace{-1mm}
\section{Designing the SEPSIS Classifier}
\label{sec:sepsis_classifier}
\vspace{-1mm}

%Explain architecture
%Diagram of architecture
%why is that architecture novel and best for MTL
%Table for evaluation number

\begin{figure*}[!htp]
\centering
\vspace{-3mm}
\includegraphics[width=0.86\textwidth, trim={0cm 0cm 0cm 0cm}]{Image/Arch.pdf}
\vspace{-2mm}
\caption{Multi-task learning architecture delineating the process of an input text going through labeling along four dimensions: (i) types of omission, (ii) colors of lie, (iii) intention of lie, and (iv) topic of lie. Here, DB Loss stands for Distribution-balanced Loss and CE loss stands for Cross Entropy loss (cf. Appendix \ref{sec:MTL_loss_function}).}
\label{fig:MTL}
\vspace{-3.5mm}
\end{figure*}

\textls[-10]{SEPSIS, by its design, is a multitask-multilabel problem requiring the application of Multitask Learning (MTL) techniques. In general MTL framework utilizes a shared representation for all the tasks. It has been observed by several researchers \cite{parisotto2015actor, rusu2015policy, yu2020gradient, fifty2021efficiently} that shared representation has its own limitations and further effects on learning task-specific loss functions. In our approach, we introduced two specific innovations, detailed in subsequent sections. Using the MTL model (Fig. \ref{fig:MTL}), we achieved a score of 0.81 F1 score on the human-annotated dataset (5000 samples) and 0.87 F1 score on the SEPSIS dataset (0.8M data points). Fig.~\ref{fig:MTL_result} shows the F1 score across deception classes on the SEPSIS dataset (cf. Appendix \ref{sec:MTL}).}


%\textbf{Merging fine-tuned LLMs:} We adopt the idea proposed in  \cite{jin2022dataless}, where a fine-tuned Language Model (LLM) specifically trained for a sub-task (in our case, the layer of deception) was combined with other similar fine-tuned LLMs. This merging process facilitated the creation of a shared representation, resulting in improved efficiency and empirically better performance. 

%\textbf{Tailored loss function:} Secondly, we conducted experiments with a wide range of loss functions to identify the most suitable one for each specific sub-task. Through this exploration, we discovered the optimal loss function that yielded the best results for the given sub-task.

\begin{comment}
The following subsection provides an overview of our novel multitask architecture for deception detection. It delves into the specific loss function, LLM and architecture employed,  highlighting its significance in training our model effectively. Moreover, we present a detailed account of the rigorous experiments carried out using four different permutations of LLM and encoder networks over 4 metrics accuracy, precision, recall, and F-score.


Multi-task learning (MTL) has emerged as a powerful paradigm for training deep neural networks to simultaneously perform multiple related tasks. In this paper, we propose a multi-task learning-based architecture for predicting five different tasks of the Deception dataset. The main advantage of using multi-task learning is the ability to leverage shared information across tasks, leading to improved model generalization and increased efficiency in training and inference. By jointly training multiple tasks, the model learns useful representations that are transferable to other related tasks, leading to better overall performance \cite{caruana1997multitask}. 

% Our approach takes a sentence as input, which is converted into an embedding. Making the rich embedding, it's a two-stage process, i.e., performing the "Dataless Knowledge Fusion" technique on it, i.e., merging fine-tuned models that originate from pre-trained language models
% with the same architecture and pre-trained weights. As it helps in better performance and generalizes on out-of-domain data. Once the embeddings are obtained, stage 2 is to convert them into latent representation using transformer encoders. The representation is that carried to five task-specific heads, four of which are multilabel and one is multiclass output. 



Our methodology takes a sentence as input and converts it into a latent embedding. The process of creating this rich embedding involves a two-stage approach. Firstly, we leverage the "Dataless Knowledge Fusion" technique \cite{jin2022dataless}, which merges fine-tuned models sharing the same architecture and pre-trained weights, resulting in enhanced performance and improved generalization capabilities, particularly when dealing with out-of-domain data \cite{jin2022dataless}. Once the word embeddings are obtained, the second stage involves converting them into a latent representation using the transformer encoder module. This representation is then propagated through four task-specific multilabel heads.
\end{comment}


\vspace{-1.5mm}
\begin{figure}[H]
\centering
\vspace{-2mm}
\includegraphics[width=0.49\textwidth, trim={0cm 0cm 0cm 0cm}]{Image/plot.pdf}
\vspace{-7mm}
\caption{SEPSIS's F1 score for all classes of deception varies from 0.81 to 0.94. We have reported accuracy, precision, and recall as well (cf. Appendix \ref{sec:experimental setup}, tab \ref{tab:overall_exp}).}
\label{fig:MTL_result}
\vspace{-4mm}
\end{figure}
\vspace{-2mm}

\noindent
\vspace{-5mm}
\subsection{Merging Finetuned LMs Brings Power!}
\vspace{-1mm}
Drawing inspiration from \cite{jin2022dataless}, we incorporated techniques for merging multiple fine-tuned LMs, a process referred to as \emph{dataless merging}. During our experimentation with various LMs, we found that T5 performed exceptionally well for our specific case, and was also the best LM for dataless merging as emphasized in \cite{jin2022dataless}. For the four layers of deception, we fine-tuned four T5 models using the data outlined in Table ~\ref{tab:SEPSIS_corpus}. These models are denoted as T5\textsubscript{layer1}, T5\textsubscript{layer2}, T5\textsubscript{layer3}, and T5\textsubscript{layer4}. By leveraging the methodology proposed in \cite{jin2022dataless}, we merged these fine-tuned T5 models to achieve a better-shared representation tailored to our specific objectives.  Figure ~\ref{fig:MTL} visually depicts the merging process via an architecture diagram. The code for reproducing experiments can be found at \url{https://bit.ly/3FglMtB}.
%In many cases, LLMs are trained using domain-specific datasets, which can limit their performance when applied to out-of-domain cases. To address this challenge, we employ a fine-tuning approach on the T5-base model for each specific task, resulting in a total of five finetuned T5-based models (one model corresponding to one task). To leverage these models in our Multitask learning architecture, we employ Dataless Knowledge Fusion on these five finetuned T5-models into a single, more generalized model that exhibits improved performance in multitask learning (from here referred \textit{merged-fine-tuned-T5}).
\noindent
\vspace{-3mm}
\subsection{Tailored Loss Function}
\vspace{-2mm}

\textls[-10]{During our exploration for suitable sub-task loss functions, we experimented with several available options, including (i) cross-entropy loss, (ii) focal loss \cite{lin2017focal}, (iii) dice loss \cite{li2019dice}, and (iv) distribution-balanced loss (DB) \cite{huang-etal-2021-balancing}. After a thorough evaluation, we observed that distribution-balanced loss yielded the best performance for layer 1, cross-entropy loss was most effective for layer 2, focal loss performed well for layer 3, and dice loss was the optimal choice for layer 4. For a comprehensive overview of the results and an in-depth discussion of different loss functions, please refer to the Appendix \ref{sec:MTL_loss_function}.}

\begin{comment}
\begin{itemize}
    \item Cross-Entropy Loss
    
    \item Focal Loss  effectively tackles class imbalance by assigning higher weights to challenging examples and down-weighting easier ones. By emphasizing learning from difficult instances, it enables the model to concentrate on improving performance specifically for the more challenging classes.
    
    \item Dice Loss 
\end{itemize}
\end{comment}



\begin{comment}
\subsection{Experiments}
We experimented with four different variations:
\begin{itemize}
    \item T5-base + LSTM
    \item T5-base + Transformer
    \item \textit{merged-fine-tuned-T5} + LSTM
    \item \textit{merged-fine-tuned-T5} + Transformer
\end{itemize}







Our experimental analysis conducted on the Deception dataset demonstrates the effectiveness of the proposed multi-task learning architecture. To assess the model's performance, we evaluated several variations, including LSTM and Transformer models, both with and without model merging. We utilized essential evaluation metrics such as accuracy, precision, recall, and F1 score to provide comprehensive insights into the efficacy and effectiveness of the approach. The results obtained per task are listed in the table \ref{tab:loss_MTL}.

\textcolor{red}{this section should finish with a big fat table detailing all the experiments and results.}
\end{comment}



\vspace{-3mm}
\section{Dissecting Propaganda through the Lens of Deception}
\vspace{-1.5mm}
As mentioned earlier, numerous studies have explored the behavioral indicators of lying, but there is hardly any consensus on categorization. However, the focus of this paper specifically revolves around investigating \emph{lies of omission} and their connection to related research within the scientific community. Notably, there are works that have extensively examined the analysis of \emph{propaganda} through language \cite{da-san-martino-etal-2019-fine,martino2020survey}.

%\vspace{-1mm}
\begin{figure}[h]
\vspace{-3mm}  
\centering
  \centering
\includegraphics[width=0.75\columnwidth]{Image/Circos_LoadedLanguage.png}
\vspace{-1.5mm}
  \caption{The Circos presents the co-occurrence of all the layers of deception with a propaganda technique named \emph{loaded language}.}
  \label{fig:loaded_language}
\vspace{-4mm}  
\end{figure}
% \vspace{-3mm}

Our scientific curiosity led us to further investigate the specific types of \emph{lies of omission} employed in strategizing particular propaganda, such as \textit{exaggeration} and/or \textit{red herring}. To conduct this study, we utilized the propaganda datasets introduced by \cite{da-san-martino-etal-2019-fine} and applied the SEPSIS classifier, as discussed in section ~\ref{sec:sepsis_classifier} on the data. Through the analysis of these experiments, we made intriguing discoveries, including: (i) \emph{the prevalence of political topic in loaded language compared to other propaganda types}, (ii) \emph{the close association between the intention of gaining advantage and Name Calling}, and (iii) \emph{the complexity underlying causal simplification as a form of speculation.} A Circos \cite{Flourish} example is presented in Fig. \ref{fig:loaded_language} for a propaganda technique named
\textit{loaded language} (cf. Appendix \ref{sec: Propaganda} for Circos diagrams corresponding to propaganda techniques). Therefore, we firmly believe that our research on SEPSIS not only stands on its own but also acts as a bridge, facilitating a deeper understanding of deception. 


%(i) \emph{the prevalence of biased language in slogans compared to other propaganda types}, (ii) \emph{the close association between distortion and whataboutism}, and (iii) \emph{the complexity underlying causal simplification as a form of speculation.}


\begin{comment}
The propaganda model is a conceptual model in political economy advanced by Edward S. Herman and Noam Chomsky to explain how propaganda and systemic biases function in corporate mass media. The model seeks to explain how populations are manipulated and how consent for economic, social, and political policies, both foreign and domestic, is "manufactured" in the public mind due to this propaganda \cite{}. %cite Wikipedia page.
Propaganda aims at influencing people’s mindsets with the purpose of advancing a specific agenda.
% How are we connecting it to deception


%Data

Other research work involved releasing data and categorizing them into 20 propaganda techniques \cite{}. We use data from XYZ \cite{} that is tagged with $20$ different propaganda techniques.

% data processing
\url{https://propaganda.math.unipd.it/fine-grained-propaganda-emnlp.html}

%MTL to predict which kind of deception






%20 circos over here, 4 of them should come here, rest in the appendix
\end{comment}











\section{Related Work}

Online meta-learning brings together ideas from online learning, meta learning, and continual learning, with the aim of adapting quickly to each new task while \emph{simultaneously} learning how to adapt even more quickly in the future. We discuss these three sets of approaches next.


\noindent \textbf{Meta Learning:} Meta learning methods try to learn the high-level context of the data, to behave well on new tasks (\emph{Learning to learn}). These methods involve learning a metric space~\citep{koch2015siamese, vinyals2016matching, snell2017prototypical, yang2017learning}, gradient based updates~\citep{finn2017model, li2017meta, park2019meta, nichol2018first, nichol2018reptile}, or some specific architecture designs~\citep{santoro2016meta, munkhdalai2017meta, ravi2016optimization}.
In this work, we are mainly interested in gradient based meta learning methods for online learning. MAML~\citep{finn2017model} and its variants~\citep{nichol2018first, nichol2018reptile, li2017meta, park2019meta, antoniou2018train} first meta train the models in such a way that the meta parameters are close to the optimal task specific parameters (good initialization). This way, adaptation becomes faster when fine tuning from the meta parameters. However, directly adapting this approach into an online setting will require more relaxation on online learning assumptions, such as access to task boundaries and resetting back and froth from meta parameters. Our method does not require knowledge of task boundaries.




\noindent \textbf{Online Learning:} Online learning methods update their models based on the stream of data sequentially. There are various works on online learning using linear models~\citep{cesa2006prediction}, non-linear models with kernels~\citep{kivinen2004online, jin2010online}, and deep neural networks~\citep{zhou2012online}. Online learning algorithms often simply update the model on the new data, and do not consider the past knowledge of the previously seen data to do this online update more efficiently. However, the online meta learning framework, allow us to keep track of previously seen data and with the ``meta'' knowledge we can update the online weights to the new data more faster and efficiently.
\noindent \textbf{Continual Learning:} 
A number of prior works on continual learning have addressed catastrophic forgetting~\citep{mccloskey1989catastrophic,li2017learning,ratcliff1990connectionist, rajasegaran2019random, rajasegaran2020itaml}, removing the need to store all prior data during training. Our method does not address catastrophic forgetting for the meta-training phase, because we must still store all data so as to ``replay'' it for meta-training, though it may be possible to discard or sub-sample old data (which we leave to future work). However, our adaptation process is fully online. A number of works perform meta-learning for better continual learning, i.e. learning good continual learning strategies~\citep{al2017continuous,nagabandi2018deep,javed2019meta,harrison2019continuous,he2019task,beaulieu2020learning}. However, these prior methods still perform batch-mode meta-training, while our method also performs the meta-training itself incrementally online, without task boundaries.


The closest work to ours is the follow the meta-leader (FTML) method~\citep{finn19a} and other online meta-learning methods~\citep{yao2020online}. FTML is a varaint of MAML that finetunes to each new task in turn, resetting to the meta-trained parameters between every task. While this effectively accelerates acquisition of new tasks, it requires ground truth knowledge of task boundaries and, as we show in our experiments, our approach outperforms FTML \emph{even when FTML has access to task boundaries and our method does not}. Note that the memory requirements for such methods increase with the number of adaptation gradient steps, and this limitation is also shared by our approach. Online-within-online meta-learning~\cite{denevi2019online} also aims to accelerate online updates by leveraging prior tasks, but still requires knowledge of task boundaries. MOCA~\cite{harrison2020continuous} instead aims to \emph{infer} the task boundaries. In contrast, our method does not even attempt to find the task boundaries, but directly adapts without them. A number of related works also address continual learning via meta-learning, but with the aim of minimizing catastrophic forgetting~\cite{gupta2020maml, caccia2020online}. Our aim is not to address catastrophic forgetting. Our method also meta-trains from small datasets for thousands of tasks, whereas prior continual learning approaches typically focus on settings with fewer larger tasks (e.g., 10-100 tasks).

\section{Conclusion and Future Work}\label{sec:conclusion}
In this paper, we report an approach which adopts reinforcement learning algorithms to solve the problem of robustness-guided falsification of CPS systems. We implement our approach in a prototype tool and conduct preliminary evaluations with a widely adopted CPS system. The evaluation results show that our method can reduce the number of episodes to find the falsifying input. As a future work, we plan to extend the current work to explore more reinforcement learning algorithms and evaluate our methods on more CPS benchmarks. 


\bibliography{custom}
%\bibliographystyle{acl_natbib}

\newpage
\onecolumn
\section*{Frequently Asked Questions (FAQs)}\label{sec:FAQs}

\begin{enumerate}
    %\item What was the hypothesis for selecting the Times of India, all news in the times of India isn't fake?
    %\begin{description}
    %\item \textbf{Ans.} - 
    %\end{description}

    %\item Are there any ethical considerations related to deception?
    %\begin{description}
    %\item \textbf{Ans.} - research studies, interpersonal relationships, professional settings
    % \end{description}

    %\item Can deception be justified?
    %\begin{description}
    %\item \textbf{Ans.} - In situations like law enforcement activities conducted covertly, strategic military tactics, protecting vulnerable individuals from harm, carefully weighing the ethical considerations and potential consequences.
    %\end{description}

    %\item How do lies of omission differ from other forms of deception?
    %\begin{description}
    %\item \textbf{Ans.} - 
    %\end{description}

    %\item What are the challenges in studying the lies of omission?
    %\begin{description}
    %\item \textbf{Ans.} - the arbitrary nature of what constitutes an omission, cognitive, linguistic and social factors involved in deceptive communication
    % \end{description}

    %\item Among lies of Omission, how does one differentiate between a gray lie and a white lie?
    %\begin{description}
    %\item \textbf{Ans.} - 
    %\end{description}

    %\item What are some of the ethical considerations and challenges involved in this work?
    %\begin{description}
    %\item \textbf{Ans.} - Some of the ethical considerations involved in this research include the potential for false positives and false negatives in deception detection, the potential for bias in the annotated dataset, and the potential for misuse of the technology to suppress free speech or target individuals unfairly.
    % \end{description}

    % \item How can this research be used to improve the transparency of automated fact-checking systems?
    % \begin{description}
    % \item \textbf{Ans.} - This research can be used to improve the transparency of automated fact-checking systems by providing a more nuanced understanding of the different types of deception that can occur in online content.  By flagging false or misleading information, organizations can be held accountable for their actions and fact-checking systems can become more effective at identifying misleading information.  
    % \end{description}

    % \item What are some of the potential implications of this research for the future of journalism  /  How can this research be used to combat the spread of misinformation and disinformation?
    % \begin{description}
    % \item \textbf{Ans.} - This work has important implications for the future of journalism by highlighting the need for more accurate and reliable ways to identify lies of omission in online content. By incorporating the insights gained from this research into their reporting, journalists can become more effective at exposing false or misleading information and promoting transparency and accountability in the media.  
    % \end{description}

    \item 
    [\ding{93}] {\fontfamily{lmss} \selectfont \textbf{What were the specific instructions provided to the annotators and the criteria used for selecting them in the crowd annotation process of 5000 sentences through AMT?}}
    
    \vspace{-3mm}
    \begin{description}
    \item[\ding{224}] The annotation pipeline outlines a step-by-step approach to deception detection based on different layers, as shown in Figure 1. To ensure reliable annotations, the dataset source was kept undisclosed from the annotators. 
    %The annotators, who were primarily students aged 18-21, were selected indefinitely. 
    Notably, for sentences categorized as "Sounds Factual," no additional annotations were made apart from missing W's.
    \end{description}

    % \item  What were the quality control measures implemented during the annotation process?
    % \begin{description}
    % \item 
    % \end{description}


    \item[\ding{93}] {\fontfamily{lmss} \selectfont \textbf{How were the loss functions determined, specifically for each task head?}}

    \vspace{-3mm}
    \begin{description}
    \item[\ding{224}] The selection of loss functions for each task head was based on the characteristics of the class distribution for that specific task. If the class distribution was imbalanced, loss functions designed to handle such scenarios were chosen. Detailed explanations and experimental results supporting the choice of each loss function can be found in the appendix section \ref{sec:MTL}.
    \end{description}

    % Shreya ans:
    \item[\ding{93}] {\fontfamily{lmss} \selectfont \textbf{Why RoBERTa was finally chosen as our baseline model for the Mask Infilling task?}}

    \vspace{-3mm}
    \begin{description}
    \item[\ding{224}] Our experimentation in comparison to other state-of-the-art language models like RoBERTa-base, MPNet-base, ELECTRA-large-generator, BERT-base-uncase, and ALBERT-large-v2 revealed a higher Bilingual Evaluation Understudy (BLEU) score using RoBERTa. The selection of RoBERTa as the preferred model for the mask infilling task, based on its highest BLEU score, implies that RoBERTa's generated outputs exhibited a greater resemblance to the desired reference outputs. This characteristic of RoBERTa's performance is particularly advantageous for generating deceptive sentences that closely resemble reference sentences. By leveraging RoBERTa's capabilities, the task of producing deceptive sentences can be effectively achieved with a higher degree of fidelity to the reference sentences.
    \end{description}
    
    \item[\ding{93}] {\fontfamily{lmss} \selectfont \textbf{Why was the T5 base model chosen for model merging, and how was its performance evaluated?}}

    \vspace{-3mm}
    \begin{description}
    \item[\ding{224}] The selection of the T5 base model for model merging involved extensive experimentation and evaluation of various language models (LLMs), such as RoBERTa, T5, and DeBERTa. Our evaluation aimed to identify the LLM that would deliver the best performance for our specific case. Initially, we assessed the individual performance of each LLM by utilizing them in the architecture to generate word embeddings, without employing model merging or fine-tuning. However, there was no significant improvement in scores observed for RoBERTa and DeBERTa when compared to using the LLM as-is (without merging) or with model merging. In contrast, the T5 model demonstrated an additional 4-5\% improvement after applying Dataless Knowledge Fusion.
    \end{description}

    \item[\ding{93}] {\fontfamily{lmss} \selectfont \textbf{What are the details of the train-test validation split and other hyperparameters used for replicating the experiments?}} 

    \vspace{-3mm}
    \begin{description}
    \item[\ding{224}] The dataset was divided into an 80-20 train-test split, where 80\% of the data was used for training and 20\% for testing. To assess the model's performance, we employed 5-fold cross-validation.The train-test split was meticulously crafted to ensure that each sentence and its augmented versions are exclusively present in either the train set or the test set, but never in both. This careful arrangement guarantees the absence of any sentence overlap (i.e. sentence "S" present in train split and paraphrased version of sentence "S" present in test spilt), maintaining the integrity of the data and enhancing the overall quality of the split. The train-test split of the dataset would be made available along with all the hyperparameters of the code on GitHub for replication of the results.
    \end{description}


    \item[\ding{93}] {\fontfamily{lmss} \selectfont \textbf{What motivated the use of data augmentations and multi-task learning, and what improvement was achieved?}} 

\vspace{-3mm}
\begin{description}
\item[\ding{224}] In our initial experiment, without employing multi-task learning and data augmentation, we achieved an average accuracy score of 0.758 (averaged across all classes). Recognizing correlations between the classes, we introduced multi-task learning to capitalize on these relationships. To further enhance the model's robustness, we applied data augmentation. The improvements in average accuracy are detailed in the appendix table \ref{tab:overall_exp}. The code for reproducing experiments can be found at  \url{https://anonymous.4open.science/r/deception_MTL-60DB/}.
\end{description}


    

\end{enumerate}

\newpage
\onecolumn
\appendix
\renewcommand{\thesubsection}{\Alph{section}.\arabic{subsection}}
\renewcommand{\thesection}{\Alph{section}}
\setcounter{section}{0}

\section*{Appendix}\label{sec:appendix}
This section provides supplementary material in the form of additional examples, implementation details, etc. to bolster the reader's understanding of the concepts presented in this work.

\section{Lies of omission -- across cultures}\label{sec:app-A}
Instances of lies of omission can be discovered in ancient literature from diverse cultures across the globe. In order to stimulate further discussion and provide motivation, we will present (in the appendix - due to obvious space limitation) two specific examples—one from the Western tradition and another from the Eastern tradition. These examples serve to highlight the prevalence and significance of lies of omission in literature and emphasize the need for deeper exploration of this phenomenon.

\noindent
\textbf{The merchant of Venice}: In Shakespeare's play, Antonio, an antisemitic merchant, borrows money from the Jewish moneylender Shylock in order to assist his friend in pursuing a relationship with Portia. Antonio can't repay the loan, and without mercy, Shylock demands a pound of his flesh as collateral. At this critical moment, Portia, who is now married to Antonio's friend, disguises herself as a lawyer and intervenes to save Antonio. Though the agreement allows Shylock to claim a pound of flesh, he must ensure that not a single drop of blood is shed, as causing harm to a Christian is strictly forbidden by law.

\noindent
\textbf{Mahabharata} - \emph{Ashwathama hatho, naro va kunjaro va}: This story is derived from an ancient Indian epic \emph{"The Mahabharta"}. In this excerpt, \emph{Ashwathama} is an elephant. \emph{Ashwathama} was also the name of the son of Guru Dronacharya. Yudhishtir, one of the Pandavas and \emph{Dharmraj} (which means he would never lie), faces the daunting task of confronting his unbeatable mentor, Guru Dronacharya, from whom he and his brothers had learned the art of warfare. Reluctant to engage in direct combat against his beloved teacher, Yudhishtir follows the advice of Lord Krishna and employs a strategy of omission. He announces the death of Ashwathama, but discreetly adds the words "naro va kunjaro va," indicating that it is actually a question whether the deceased Ashwathama is a human or an elephant. While Yudhishtir technically did not prevaricate, the news of his son's supposed demise deeply affects Guru Dronacharya, causing him to lose his will to fight and making it easier for Yudhishtir to overcome him. The story highlights Yudhishtir's adherence to his principles of truthfulness while employing a clever tactic of omission to gain an advantage in the battle.

\begin{comment}
\textcolor{red}{random------------}
\cite{ratliff2011behavioral} defines deception to be an outcome of omission and commission. To study deception using natural language processing techniques, in this research, we focus on omission because for ease of implementation. Referring to work on linking 5W (who, what, when, where, why) with semantic role labeling \cite{rani2023factify5wqa}, we study the omissions of these Ws and further use prompt engineering to augment deceptive data.

% \subsection{Definition of lies}

\textbf{Lies of commission} occur when someone takes the facts and embellishes them to produce an account of what never happened as it is typically more flattering. on the other hand, \textbf{lies of omission} occur when people generally leave out crucial information like facts from the victim. Thus the task of identifying such types of lies could be studied by identifying 5W(Who, What, When, Where, Why) and then finding missing ones. Therefore, we study omission in depth.

%2nd layer of lie definitions

\end{comment}

\section{Dataset Curation}
This contains additional information on data sources, data cleaning, annotation, and Inter annotator agreement
\subsection{Data Sources}\label{sec:data sources}
Information Security and Object Technology (ISOT) fake news dataset \cite{ISOTFakeNewsDataset}: This dataset contains two types of articles fake and real news. This dataset was collected from real-world sources; the truthful articles were obtained by crawling articles from Reuters.com (News website). As for the fake news articles, they were collected from different sources. The fake news articles were collected from unreliable websites that were flagged by Politifact (a fact-checking organization in the USA) and Wikipedia. For this research, the fake news dataset is leveraged. The data source has a file named “Fake.csv” which contains more than 12,600 articles from different fake news outlet resources. Each article contains the following information: article title, text, type, and the date the article was published on. We chose 2500 data points randomly from this set for this research.


\subsection{Data Cleaning and Annotation Quality check}\label{sec: data cleaning}
Data cleaning involves two iterations, data set preparation, and a human-level review of the manual annotations. The process involved the removal of URLs and unnecessary internet taxonomy with the aim of increasing data quality. To further increase the quality of data for human understanding, we reviewed the annotations manually by following the below-mentioned steps:
\vspace{-3mm}
\begin{itemize}
\setlength\itemsep{0em}
    \item Accounting for multiple annotations against a single field by the same annotator by getting rid of one of the two annotations along the lines of the definitions formulated at the start of the process.
   
    \item Filling in for fields annotated by the first entity and missed by the second entity by accounting for the gaps by building along the lines of definitions established earlier. 
    Correcting typographical errors implicating a similar meaning.
    
    \item Overriding annotations for a couple of data items where the reviewer found them overwhelmingly wrong.
\end{itemize}
\vspace{-3mm}

\subsection{Inter Annotator Agreement
}\label{sec: Data Annotation}
In the ~\cref{sec:iaa_score} we have reported inter-annotator scores for all the 3 layers in ~\cref{tab: Kappa score}. In addition, here we are reporting inter-annotator agreement for the topic of lie in the ~\cref{tab:iaa_topic_of_lie}.

\begin{table}[!tbh]
\centering
\resizebox{0.7\textwidth}{!}{
\begin{tabular}{lcccccc}
\toprule
          & Political & Educational & Religious & Ethnicity & Racial & Others \\
\midrule
Twitter   &\cellcolor{green!50} 0.82      &\cellcolor{blue!50} 0.78        & \cellcolor{green!50}0.81      &\cellcolor{blue!50} 0.73      & \cellcolor{blue!50}0.76   & \cellcolor{blue!50}0.72   \\
Fake News & \cellcolor{green!50}0.87      &\cellcolor{green!50} 0.84        &\cellcolor{green!50} 0.85      &\cellcolor{blue!50} 0.77      & \cellcolor{green!50}0.82   &\cellcolor{blue!50} 0.79  
\\
\bottomrule
\end{tabular}
}
\label{tab:iaa_topic_of_lie}
\caption{Inter Annotator Agreement score for Topic of Lies.}
\end{table}
%\newpage


\vspace{-2mm}
\subsection{Data Analysis of SEPSIS Corpus and Insights}\label{sec: data analysis}

This section contains a thorough analysis of the entire corpus.


\noindent
\textbf{Word representation of the sepsis corpus}: We have utilized two different data sources to understand the frequency of words, we present the word clouds in fig \ref{fig:fakenews-5k} and fig \ref{fig:tweets-5k}. An interesting insight is figure \ref{fig:fakenews-5k} represents US news and figure \ref{fig:tweets-5k} represents the Indian media house.

\begin{figure}[H]
  \centering
  \begin{subfigure}[b]{0.3\textwidth}
    \includegraphics[width=\linewidth, trim={1cm 1cm 1cm 1cm}]{Image/fakenews_5k.pdf}
    \caption{Word cloud of data collected from ISOT fake news.}
    \label{fig:fakenews-5k}
  \end{subfigure}
    \hspace{2cm}
    \begin{subfigure}[b]{0.3\textwidth}\includegraphics[width=\linewidth, trim={1cm 1cm 1cm 1cm}]{Image/tweets_5k.pdf}
    \caption{Word cloud of data collected from Times of India.}
    \label{fig:tweets-5k}
  \end{subfigure}
  \label{fig:combined_figure}
\end{figure}


%%%%%%%%%%
\noindent
\textbf{Statistics on categories across entire corpus:}
We further present the percentage of each feature across the entire dataset as represented in table \ref{tab: SEPSIS_breakup}.

%Type of Omission




%%enter value counts % for all -- @shreya

% Please add the following required packages to your document preamble:
% \usepackage{multirow}
\begin{table}[H]
\centering
\resizebox{0.5\columnwidth}{!}{
\begin{tabular}{clcc}
\toprule
\textbf{Layers of Deception}                                                                     & \textbf{\begin{tabular}[c]{@{}l@{}}Categories within the layer\end{tabular}} & \textbf{\begin{tabular}[c]{@{}c@{}}Number of datapoints\end{tabular}} & \textbf{Percentage} \\ \toprule
\multirow{5}{*}{\textbf{\begin{tabular}[c]{@{}c@{}}Layer 1:\\ \\ Type of Omission\end{tabular}}} & Speculation                                                                     & 311754                                                                        & 35.56\%                   \\
                                                                                                 & Bias                                                                            & 72268                                                                        & 8.24\%                   \\
                                                                                                 & Distortion                                                                      & 150249                                                                        & 17.14\%                   \\
                                                                                                 & Opinion                                                                         & 154590                                                                        & 17.63\%                   \\
                                                                                                 & Sounds Factual                                                                  & 187923                                                                        & 21.43\%                   \\ \hline
\multirow{4}{*}{\textbf{\begin{tabular}[c]{@{}c@{}}Layer 2:\\ Colors of Lies\end{tabular}}}      & Black                                                                           & 322634                                                                        & 45.31\%                   \\
                                                                                                 & White                                                                           & 90019                                                                        & 12.64\%                   \\
                                                                                                 & Gray                                                                            & 182161                                                                        & 25.58\%                   \\
                                                                                                 & Red                                                                             & 117245                                                                        & 16.47\%                   \\ \hline
\multirow{6}{*}{\textbf{\begin{tabular}[c]{@{}c@{}}Layer 3:\\ Intent of Lies\end{tabular}}}      & Gaining Advantage                                                               & 332661                                                                        & 47.73\%                  \\
                                                                                                 & Protecting Themselves                                                           & 202395                                                                        & 29.04\%                  \\
                                                                                                 & Gaining Esteem                                                                  & 124197                                                                        & 17.96\%                  \\
                                                                                                 & Avoiding Embarrasment                                                           & 24505                                                                        & 3.52\%                  \\
                                                                                                 & Defaming Esteem                                                                 & 6938                                                                        & 1.00\%                  \\
                                                                                                 & Protecting Others                                                               & 5236                                                                        & 0.75\%                  \\ \hline
\multirow{6}{*}{\textbf{\begin{tabular}[c]{@{}c@{}}Layer 4:\\ Topic of Lies\end{tabular}}}       & Political                                                                       & 546780                                                                        & 72.36\%                  \\
                                                                                                 & Educational                                                                     & 109596                                                                        & 14.50\%                  \\
                                                                                                 & Ethnicity                                                                       & 29343                                                                        & 3.88\%                  \\
                                                                                                 & Religious                                                                       & 27575                                                                        & 3.64\%                  \\
                                                                                                 & Racial                                                                          & 27354                                                                        & 3.61\%                  \\
                                                                                                 & Others                                                                          & 15250                                                                        & 2.01\%                  \\ \bottomrule
\end{tabular}
}
\caption{Breakup of SEPSIS datapoints over layers of deception and categories within each layer.}
\label{tab: SEPSIS_breakup}
\end{table}





%%%%%%%

\noindent
\textbf{Percentage presence of 5Ws across all datapoints}:
Since we utilize 5W-based mask infilling, we also present \% of 5Ws across the entire dataset. and the statistics around it can be found in the table \ref{tab:percentage_presence} below.

\begin{table}[H]
%\begin{wraptable}{R}{8cm}
\centering
\resizebox{0.8\columnwidth}{!}{%
\begin{tabular}{@{}lccccc@{}}
\toprule
{}           &  \textbf{Who}  & \textbf{What} & \textbf{Why} & \textbf{When} & \textbf{Where}\\ \midrule
\% presence of 5W for tweets from Times of India          &   34.84\% & 53.06\% & 1.02\% & 6.31\% & 4.77\%      \\
\% presence of 5W from ISOT fake news dataset             &   36.40\% & 52.73\% & 1.41\% & 6.30\% & 3.16\%   \\
\bottomrule
\end{tabular}%
}
\caption{\% of 5Ws across the entire dataset.}
\label{tab:percentage_presence}
\end{table}




\noindent
\textbf{Co-occurence percentage}: The four layers are connected to the input sentence. To study the co-occurrence across all categories and layers, we present them in heatmaps as described in fig \ref{fig:heatmap_layers}.


When analyzing lies of omission and colors of lies, we observe a strong correlation between speculation and black lies. Additionally, a significant majority of speculative texts can be categorized as political in nature. This association becomes even more apparent when we delve into the Intent of Lie on Lies of Omission. It is evident that the primary objective behind the creation of speculative texts is to gain an advantage. Black lies, in particular, are frequently employed for this purpose. It is noteworthy that political texts predominantly consist of black lies, serving as a means to gain an advantage.


\begin{figure*}[!h]
    \begin{subfigure}[b]{0.5\textwidth}
    \centering
        \includegraphics[width=0.8\textwidth]{heatmaps/SBDO_color.pdf}
        %\caption{}
        \caption{Lies of Omission-Colors of Lie.}
    \end{subfigure}
    \begin{subfigure}[b]{0.5\textwidth}
    \centering
        \includegraphics[width=0.8\textwidth]{heatmaps/SBDO_intent.pdf}
        %\caption{}
        \caption{Lies of Omission-Intent of Lie.}
    \end{subfigure}
    
    \begin{subfigure}[b]{0.5\textwidth}
    \centering
        \includegraphics[width=0.8\textwidth]{heatmaps/SBDO_genre.pdf}
        \caption{Type of omission-Topic of lie.}
        %\caption{Lies of Omission-Influence of Lie}
    % \label{fig: vitc}
    \end{subfigure}            
    \begin{subfigure}[b]{0.5\textwidth}
    \centering
        \includegraphics[width=0.8\textwidth]{heatmaps/color_intent.pdf}
        \caption{Colors of Lie-Intent of Lie.}
    \end{subfigure}
    
    \begin{subfigure}[b]{0.5\textwidth}
    \centering
        \includegraphics[width=0.8\textwidth]{heatmaps/color_genre.pdf}

        \caption{Colors of Lie-Influence of Lie.}
    \end{subfigure}
    \begin{subfigure}[b]{0.5\textwidth}
    \centering
        \includegraphics[width=0.8\textwidth]{heatmaps/intent_genre.pdf}
        %\caption{}
        \caption{Intent of Lie-Influence of Lie.}
    \end{subfigure}
    
    \caption{The heatmaps provide a concise overview of the interconnections and overlaps among various layers of Lies. Numbers represents \% overlap. }
    \label{fig:heatmap_layers}
\end{figure*}


\newpage
\section{Data Augmentation}
For data augmentation, we have used two techniques (i) Paraphrasing and (ii) 5W Mask Infilling.
We provide additional information on these techniques in the following subsection.
\subsection{Paraphrasing Deceptive Datapoints}\label{sec:paraphrase-evaluation}


The underlying drive for paraphrasing textual assertions stems from the need to address variations that exist in real-life written content. The same textual claim might take on several different shapes since different news publishing companies use a variety of writing techniques. It is essential to create a solid standard for a thorough examination by taking these variations into account ( example in Figure \ref{fig: paraphrase}).

\begin{figure}[!tbh]
\centering
\resizebox{\columnwidth}{!}{%
\fbox
{%
    \parbox{\columnwidth}{%
    \fontsize{9}{10}\selectfont{
    \textcolor{blue}{Sasan Goodarzi, the CEO of software giant Intuit, which has avoided mass layoffs, says tech firms axed jobs because they misread the pandemic.}}
    \\
    \textbf{Prphr 1:} Sasan Goodarzi, the CEO of Intuit, a software giant that refrained from massive layoffs, explains that tech companies terminated employees due to their misinterpretation of the pandemic.
     \\
    \textbf{Prphr 2:} Intuit's CEO, Sasan Goodarzi, highlights that unlike other tech firms, the software giant avoided extensive job cuts as they correctly understood the impact of the pandemic.
     \\
    \textbf{Prphr 3:} The pandemic was misinterpreted by tech companies, leading them to lay off employees, according to Sasan Goodarzi, CEO of Intuit, a software giant that took a different approach and did not resort to mass layoffs.
     \\
    \textbf{Prphr 4:} Sasan Goodarzi, the CEO of Intuit, a software giant, asserts that tech companies made a mistake by laying off staff members because they failed to comprehend the true nature of the pandemic.
     \\
    \textbf{Prphr 5:} In contrast to tech firms that made the wrong call and downsized their workforce, Intuit, led by CEO Sasan Goodarzi, correctly assessed the pandemic and refrained from mass layoffs.
    }
    }%
}

%\vspace{-3mm}
\caption{Deceptive paraphrased data obtained using \texttt{text-davinci-003} \cite{brown2020language}.}
\label{fig: paraphrase}
\end{figure}


\begin{wrapfigure}{R}{8cm}
\centering
\resizebox{0.5\columnwidth}{!}{
\includegraphics[width=0.85\columnwidth, height=8cm, trim={0cm 0.5cm 0cm 0cm}]{Image/figure_score_paraphrase.pdf}}
\caption{A higher diversity score depicts an increase in the number of generated paraphrases and linguistic variations in those generated paraphrases.}
%\vspace{-2mm}
\label{fig: parr}
\end{wrapfigure}

 To generate multiple paraphrases for a given claim, we employ state-of-the-art (SoTA) models. When selecting the appropriate paraphrase model from a list of available options, our main consideration is to ensure that the generated paraphrases exhibit both linguistic correctness and rich diversity. The process we follow to achieve this can be outlined as follows: Let's assume we have a claim denoted as $c$. Using a paraphrasing model, we generate $n$ paraphrases, resulting in a set of paraphrases $p_1^c$, $p_2^c$, ..., $p_n^c$. Subsequently, we conduct pairwise comparisons between these paraphrases and the original claim $c$, giving us comparisons such as $c-p_1^c$, $c-p_2^c$, ..., $c-p_n^c$. At this stage, we identify the examples that exhibit entailment, selecting only those for further consideration. To determine entailment, we utilize RoBERTa Large \cite{liu2019roberta}, a state-of-the-art model trained on the SNLI task \cite{bowman2015large}.



However, it is important to consider various secondary factors when evaluating paraphrase models. For instance, one model may generate a limited number of paraphrase variations compared to others, but those variations might be more accurate and consistent. Therefore, we took into account three key dimensions in our evaluation: \textit{(i) the number of meaningful paraphrase generations, (ii) the correctness of those generations, and (iii) the linguistic diversity exhibited by the generated paraphrases}. In our experiments, we explored the capabilities of three available models: (a) Pegasus \cite{zhang2020pegasus}, (b) T5 (T5-Large) \cite{raffel2020exploring}, and (c) GPT-3 (specifically, the \texttt{text-davinci-003} variant) \cite{brown2020language}. Based on empirical observations and analysis, we found that GPT-3 consistently outperformed the other models. To ensure transparency regarding our experimental process, we provide a detailed description of the aforementioned evaluation dimensions as follows.

\begin{table}[H]
%\begin{wraptable}{L}{15cm}
\centering
\resizebox{0.5\columnwidth}{!}{%
\begin{tabular}{@{}lcccc@{}}
\toprule
Model           &  Coverage  & Correctness & Diversity \\ \midrule
Pegasus          &   31.98   &    93.23\%       &     3.53      \\
T5               &  30.09       &      84.56\%       &    3.04       \\
GPT-3 &   35.19   &   89.67\%      &     7.39      \\
\bottomrule
\end{tabular}
}
\caption{Experimental results of automatic paraphrasing models based on three factors: \textit{(i) coverage, (ii) correctness and (iii) diversity}; GPT-3 (\texttt{text-davinci-003}) can be seen as the most performant.}
\label{tab:my-table}
\end{table}
%\end{wraptable}
%\vspace{-4mm}

\textbf{Coverage - Generating a substantial number of paraphrases:} Our objective is to generate up to five paraphrases for each given claim. After generating the paraphrases, we employ the concept of minimum edit distance (MED) \cite{wagner1974string} to assess the similarity between the paraphrase candidates and the original claim (with word-level units instead of individual characters). If the MED exceeds a threshold of ±2 for a particular paraphrase candidate (e.g., $c-p_1^c$), we consider it as a viable paraphrase and retain it for further evaluation. However, if the MED is within the threshold, we discard that particular paraphrase. By employing this setup, we evaluated all three models to determine which one generates the highest number of meaningful paraphrases.

\textbf{Correctness - Ensuring correctness in the generated paraphrases:} Following the initial filtration step, we conducted pairwise entailment assessments using the RoBERTa Large model \cite{liu2019roberta}, which is a state-of-the-art model trained on the SNLI dataset \cite{bowman2015large}. We retained only those paraphrase candidates that were identified as entailed by the RoBERTa Large model.


\textbf{Diversity - Ensuring linguistic diversity in the generated paraphrases:} Our focus was to select a model that could produce paraphrases with greater linguistic diversity. To assess the dissimilarities between the generated paraphrase claims, we compared pairs such as $c-p_n^c$, $p_1^c-p_n^c$, $p_2^c-p_n^c$, ..., $p_{n-1}^c-p_n^c$ for each paraphrase. We repeated this process for all other paraphrases and calculated the average dissimilarity score. Since there is no specific metric to measure dissimilarity, we utilized the inverse of the BLEU score \cite{papineni2002bleu}. This allowed us to gauge the linguistic diversity exhibited by a given model. Based on these experiments, we observed that the \texttt{text-davinci-003} variant performed the best in terms of linguistic diversity. The results of the experiment are presented in the table below. Moreover, we prioritized the selection of a model that maximized linguistic variations, and \texttt{text-davinci-003} excelled in this regard as well. The diversity vs. chosen models plot is illustrated in Figure ~\ref{fig: parr}.




\subsection{Data Augmentation using 5W Mask Infilling}
This mapping describes how Propbank roles are mapped to 5Ws(Who, What, When, Where, Why). We have used this mapping for mask infilling.
\begin{table}[H]
%\begin{wraptable}{R}{8cm}
\centering
\resizebox{0.5\columnwidth}{!}{
\begin{tabular}{ccccccc}
\toprule 
\textbf{PropBank Role }& \textbf{Who} & \textbf{What} & \textbf{When} & \textbf{Where} & \textbf{Why} & \textbf{How} \\
\midrule
\textbf{ARG0} & \textbf{84.48} & 0.00 & 3.33 & 0.00 & 0.00 & 0.00 \\
\textbf{ARG1} & 10.34 & \textbf{53.85} & 0.00 & 0.00 & 0.00 & 0.00 \\
\textbf{ARG2} & 0.00 & 9.89 & 0.00 & 0.00 & 0.00 & 0.00 \\
\textbf{ARG3} & 0.00 & 0.00 & 0.00 & 22.86 & 0.00 & 0.00 \\
\textbf{ARG4} & 0.00 & 3.29 & 0.00 & 34.29 & 0.00 & 0.00 \\
\textbf{ARGM-TMP} & 0.00 & 1.09 & \textbf{60.00} & 0.00 & 0.00 & 0.00 \\
\textbf{ARGM-LOC} & 0.00 & 1.09 & 10.00 & \textbf{25.71} & 0.00 & 0.00 \\
\textbf{ARGM-CAU} & 0.00 & 0.00 & 0.00 & 0.00 & \textbf{100.00} & 0.00 \\
\textbf{ARGM-ADV} & 0.00 & 4.39 & 20.00 & 0.00 & 0.00 & 0.06 \\
\textbf{ARGM-MNR} & 0.00 & 3.85 & 0.00 & 8.57 & 0.00 & \textbf{90.91} \\
\textbf{ARGM-MOD} & 0.00 & 4.39 & 0.00 & 0.00 & 0.00 & 0.00 \\
\textbf{ARGM-DIR} & 0.00 & 0.01 & 0.00 & 5.71 & 0.00 & 3.03 \\
\textbf{ARGM-DIS} & 0.00 & 1.65 & 0.00 & 0.00 & 0.00 & 0.00 \\
\textbf{ARGM-NEG} & 0.00 & 1.09 & 0.00 & 0.00 & 0.00 & 0.00 \\
\bottomrule
\end{tabular}
}
\caption{A mapping table from PropBank \cite{palmer2005proposition} {(\textit{Arg0, Arg1, ...})} to 5W {(\textit{Who, What, When, Where, and Why})}.}
\label{tab:5w-map-SRL}
\end{table}
%\end{wraptable}


\section{Multi-Task Learning}\label{sec:MTL}

In this section, we delve into the specific architectural choices, experimental setup, and the formulation of the loss function employed for multi-task learning frameworks: The SEPSIS Classifier. By exploring the intricacies of this approach, we aim to shed light on the systematic integration of multiple tasks into a unified learning framework, ultimately enabling the model to effectively leverage synergistic information across layers of Deception. 

\subsection{Architectural Discussion}


 Multi-task learning (MTL) has emerged as a powerful paradigm for training deep neural networks to perform multiple related tasks simultaneously. In this paper, we propose a multi-task learning-based architecture for predicting four different tasks of the Deception dataset. The main advantage of using multi-task learning is the ability to leverage shared information across tasks, leading to improved model generalization and increased efficiency in training and inference. By jointly training multiple tasks, the model learns useful representations that are transferable to other related tasks, leading to better overall performance \cite{caruana1997multitask}. 

 \subsubsection{Dataless Knowledge Fusion}

 In many cases, LLMs are trained using domain-specific datasets, which can limit their performance when applied to out-of-domain cases. To address this challenge, we employ a fine-tuning approach on the T5-base model for each specific task, resulting in a total of four finetuned T5-based models (one model corresponding to one task). To leverage these models in our Multitask learning architecture, we employ Dataless Knowledge Fusion \cite{jin2022dataless} on these four finetuned T5-models into a single, more generalized model that exhibits improved performance in multitask learning (from here referred \textit{merged-fine-tuned-T5}).

 \subsubsection{Methodology}

 Our methodology takes a sentence as input and converts it into a latent embedding. The process of creating this rich embedding involves a two-stage approach. Firstly, we leverage the model-merging technique \cite{jin2022dataless}, which merges fine-tuned models sharing the same architecture and pre-trained weights, resulting in enhanced performance and improved generalization capabilities, particularly when dealing with out-of-domain data \cite{jin2022dataless}. Once the word embeddings are obtained from this merged model, the second stage involves converting them into a latent representation using the transformer encoder module. This representation is then propagated through four task-specific multilabel heads to obtain the output labels for each of the layers of Deception. 

 

%\subsubsection{Dataless Knowledge Fusion}

%Define dataless knowledge fusion
%\subsubsection{Rationale behind first using T5 and then dataless knowledge fusion and then encoder}

\subsection{Loss Functions}\label{sec:MTL_loss_function}
This section contains an in-depth discussion of different loss functions that we used for different tasks of MTL architecture.

\subsubsection{Cross-Entropy Loss}
Cross entropy loss, also known as log loss or logistic loss, is a commonly used loss function in machine learning, particularly in classification tasks. It measures the dissimilarity between the predicted probabilities of classes and the true labels of the data. The log loss function penalizes incorrect predictions more strongly, meaning that as the predicted probability deviates further from the true label, the loss increases. The loss approaches zero when the predicted probability aligns with the true label.

For the SEPSIS classifier, i.e., multi-label classification task with n classes, the cross-entropy loss is calculated as the average of the individual binary cross-entropy losses for each class. 
\begin{equation}
L_{B C E}= \begin{cases}-\log \left(p_i^k\right) & \text { if } y_i^k=1 \\ -\log \left(1-p_i^k\right) & \text { otherwise }\end{cases}
\end{equation}

where,
\begin{itemize}
\item $y^k=$ $\left[y_1^k, \ldots, y_C^k\right] \in\{0,1\}^C(C$ is the number of classes),
\item{$p_i^k$ is the predicted probability distribution across the classes}
\end{itemize}


\subsubsection{Focal Loss}

Focal loss is a modification of the cross entropy loss that addresses the issue of class imbalance in multi-class classification tasks \cite{lin2017focal}. In the standard multi-class cross-entropy loss, all classes are treated equally, which can be problematic when dealing with imbalanced datasets where certain classes have a much smaller representation. Focal loss aims to down-weight the contribution of well-classified examples and focuses more on difficult and misclassified examples. The focal loss for multi-label classification is defined as follows:
\begin{equation}
L_{F L}= \begin{cases}-\left(1-p_i^k\right)^\gamma \log \left(p_i^k\right) & \text { if } y_i^k=1 \\ -\left(p_i^k\right)^\gamma \log \left(1-p_i^k\right) & \text { otherwise }\end{cases}
\end{equation}

where:
\begin{itemize}
\item{$p_i^k$ is the predicted probability distribution across the classes}
\item{\(\gamma\) is the focusing parameter that controls the degree of down weighting. It is usually set to a value greater than 0. We used \(\gamma\) = 2 in our experiment.}
\end{itemize}
The focal loss formula introduces the term \((1 - p_{i})^{\gamma}\) which acts as a modulating factor. This factor down weights well-classified examples 
\(p_i^k\) close to 1 and assigns them a lower contribution to the loss. The focusing parameter gamma controls how much the loss is down-weighted. Higher values of gamma place more emphasis on difficult examples.
By incorporating the focal loss into the training objective, the model can effectively handle class imbalance and focus more on challenging examples.

\subsubsection{Dice Loss}
The Dice loss is a similarity-based loss function commonly used in image segmentation tasks and data-imbalanced multi-class classification problems. It measures the overlap or similarity between predicted and true labels. For multi-label classification, the Dice loss can be defined as follows:
\begin{equation}
L_{DL} = 1 - \frac{2 \sum_{i=1}^C y_i^k \cdot p_i^k+\epsilon}{\sum_{i=1}^C y_i^k+\sum_{i=1}^C p_i^k+\epsilon}
\end{equation}

\begin{itemize}
\item{C is the number of classes}
\item{\(y_i^k\) represents the true label for class C, which can be either 0 or 1 for each label.}
\item{\(p_i^k\) represents the predicted probability or output for class c}

\end{itemize}
The formula calculates the Dice coefficient for each example by summing the products of the true labels \(y_i^k\) and predicted probabilities \(p_i^k\) for each class C. The numerator represents the intersection between the predicted and true labels, while the denominator represents the sum of the predicted and true labels, which corresponds to the union of the two sets. By subtracting the Dice coefficient from 1, we obtain the Dice loss.

By using the Dice loss, the model is encouraged to focus on correctly identifying and predicting the minority classes, as the loss is computed based on the intersection and sum of true and predicted labels for each class. This property is especially valuable in data-imbalanced settings, as it helps to alleviate the bias towards majority classes and improve the model's ability to capture and predict the minority classes accurately.

\subsubsection{Distribution-balanced Loss}

The distribution-balanced (DB) loss function is a promising solution for addressing class imbalance and label dependency in multilabel text classification tasks. Unlike traditional approaches such as resampling and re-weighting, which often lead to oversampling common labels, the DB loss function tackles these challenges directly. By inherently considering the class distribution and label linkage, it offers a more effective alternative for achieving balanced training.

According to \cite{huang-etal-2021-balancing}, the application of the DB loss function has demonstrated superior performance compared to commonly used loss functions in multi-label scenarios. This novel approach addresses the problem of class imbalance, where certain labels are significantly underrepresented, and considers the relationship and dependencies between different labels. By striking a balance between these factors, the DB loss function ensures that the training process is fair and unbiased, resulting in improved accuracy and robustness in multilabel text classification tasks. 

For multi-label classification, the Distribution-balanced loss can be defined as follows:
\begin{equation}
L_{D B}= \begin{cases}-\hat{r}_{D B}\left(1-q_i^k\right)^\gamma \log \left(q_i^k\right) & \text { if } y_i^k=1 \\ -\hat{r}_{D B} \frac{1}{\lambda}\left(q_i^k\right)^\gamma \log \left(1-q_i^k\right) & \text { otherwise }\end{cases}
\end{equation}

where:
\begin{itemize}
\item{C is the number of classes}
\item $\hat{r}_{D B}=\alpha+\sigma\left(\beta \times\left(r_{D B}-\mu\right)\right)$ $\rightarrow$ $r_{D B}= \frac{\frac{1}{C} \frac{1}{n_i}} {\frac{1}{C} \sum_{y_i^k=1} \frac{1}{n_i}}$
\item{\(y_{i}\) represents the true label }
\item {$\lambda$ scale factor}
\end{itemize}

The distribution-balanced loss combines rebalanced weighting and negative-tolerant regularization (NTR) to address key challenges in multi-label scenarios. It effectively reduces redundant information arising from label co-occurrence, which is crucial in such tasks. Additionally, the loss explicitly assigns lower weights to negative instances that are considered "easy-to-classify," thereby improving the model's ability to handle these instances effectively. \cite{wu2020distribution}



% The formula calculates the Dice coefficient for each example by summing the products of the true labels \((y_{i,c})\) and predicted probabilities \((p_{i,c})\) for each class c. The numerator represents the intersection between the predicted and true labels, while the denominator represents the sum of the predicted and true labels, which corresponds to the union of the two sets. By subtracting the Dice coefficient from 1, we obtain the Dice loss.

% The Dice loss encourages the model to produce higher values for the overlap between predicted and true labels, resulting in better segmentation accuracy. Minimizing the Dice loss leads to optimizing the model parameters to generate segmentation masks that closely match the ground truth masks.



\subsubsection{Rationale for choosing loss function for the particular task.}

The selection of specific loss functions for each task is driven by various factors and considerations. 

\begin{enumerate}

    \item \textbf{Distribution-balanced loss function for Types of Omission:} Due to the strong multi-label nature and skewed distribution of the Types of Omission layer, the Distribution-balanced loss function is utilized \cite{huang-etal-2021-balancing}. This loss function is specifically designed to handle extreme multi-label scenarios and skewed class distributions, providing a more balanced and effective training process for the model.


    \item \textbf{Cross Entropy loss for Color of Lie}:
The Color of Lie layer is relatively class-wise balanced. In such cases, the Cross-Entropy loss is a commonly used and standard loss function. It is well-suited for balanced class distributions and helps the model effectively learn and classify the color of lies.

    \item \textbf{Focal loss for Intent of Lie:}  The Intent of Lie layer is a class-imbalanced scenario. In such situations, the Focal loss has shown to perform well. Focal loss down-weights easy examples and focuses more on hard, misclassified examples, which helps in addressing class imbalance and improving the model's performance on classification of minority classes.

    \item \textbf{Dice loss for Topic of Lie:} The Topic of Lie layer is also a class-imbalanced scenario. The Dice loss has demonstrated effectiveness in handling class imbalance. Hence we used the Dice loss for this layer so that, the model can better capture and predict the minority topics. 


The rationale behind selecting focal loss for the Intent of lie and Dice loss for the topic of lie is based on experimentation. Initially, we tried the opposite combination, which resulted in an F1 score of 0.85 for the Intent of lie and a score of 0.85 for the topic of lie. However, in the current configuration, we achieved improved performance with an F1 score of 0.87 for the Intent of lie and a score of 0.86 for the topic of lie. Therefore, after careful evaluation, we opted for focal loss and Dice loss for their respective categories to maximize overall performance.



\end{enumerate}



\subsection{Experimental results} \label{sec:experimental setup}
For overall experiments, we had 4 setups broadly.
\begin{itemize}

\item{T5 with LSTM encoder combined with no model merging}
\item{T5 with LSTM encoder combined with model merging}
\item{T5 with transformer encoder combined with no model merging}
\item{T5 with transformer encoder combined with model merging}

\end{itemize}
We used accuracy, precision, recall, and F1 score for evaluating the performance of our model. T5 with transformer encoder combined with model merging performed the best and results on these metrics for all experiments are presented in table \ref{tab:overall_exp}.


% Please add the following required packages to your document preamble:
% \usepackage{multirow}
\begin{table}[!tbh]
\resizebox{\columnwidth}{!}{
\begin{tabular}{l|l|l|llllllll||llllllll}
	\toprule
\multirow{2}{*}{}      & \multirow{2}{*}{\textbf{SEPSIS}}         & \multirow{2}{*}{\textbf{Labels}} & \multicolumn{8}{c||}{\textbf{Without Model Merging}}                                                                                                                                                                                                                                     & \multicolumn{8}{c|}{\textbf{With Model Merging}}                                                                                                                                                                                                                                        \\ \cline{4-19} 
                       &                                 &                         & \multicolumn{2}{c|}{\textbf{Accuracy} \%}                                         & \multicolumn{2}{l|}{\textbf{Precision}}                                       & \multicolumn{2}{l|}{\textbf{Recall}}                                            & \multicolumn{2}{l||}{\textbf{F1-Score}}                     & \multicolumn{2}{l|}{\textbf{Accuracy} \%}                                         & \multicolumn{2}{l}{\textbf{Precision}}                                         & \multicolumn{2}{l|}{\textbf{Recall}}                                            & \multicolumn{2}{l|}{\textbf{F1-Score}}                     \\
                       
                       \toprule
                       \multirow{21}{*}{\textbf{\parbox[c]{2.5cm}{\centering T5 with \\ LSTM \\ encoder}}}


 & \multirow{5}{*}{Type of Omission}     & Speculation             & \multicolumn{1}{l|}{82.58} & \multicolumn{1}{l|}{\multirow{5}{*}{80.25}} & \multicolumn{1}{l|}{0.78} & \multicolumn{1}{l|}{\multirow{5}{*}{0.77}} & \multicolumn{1}{l|}{0.83} & \multicolumn{1}{l|}{\multirow{5}{*}{0.80}} & \multicolumn{1}{l|}{0.8}  & \multirow{5}{*}{0.78} & \multicolumn{1}{l|}{86.15} & \multicolumn{1}{l|}{\multirow{5}{*}{82.89}} & \multicolumn{1}{l|}{0.84} & \multicolumn{1}{l|}{\multirow{5}{*}{0.81}} & \multicolumn{1}{l|}{0.85} & \multicolumn{1}{l|}{\multirow{5}{*}{0.83}} & \multicolumn{1}{l|}{0.84} & \multirow{5}{*}{0.82} \\ \cline{3-4} \cline{6-6} \cline{8-8} \cline{10-10} \cline{12-12} \cline{14-14} \cline{16-16} \cline{18-18}
                       &                                 & Opinion                 & \multicolumn{1}{l|}{80.76} & \multicolumn{1}{l|}{}                       & \multicolumn{1}{l|}{0.80} & \multicolumn{1}{l|}{}                      & \multicolumn{1}{l|}{0.79} & \multicolumn{1}{l|}{}                      & \multicolumn{1}{l|}{0.79} &                       & \multicolumn{1}{l|}{82.54} & \multicolumn{1}{l|}{}                       & \multicolumn{1}{l|}{0.82} & \multicolumn{1}{l|}{}                      & \multicolumn{1}{l|}{0.81} & \multicolumn{1}{l|}{}                      & \multicolumn{1}{l|}{0.81} &                       \\ \cline{3-4} \cline{6-6} \cline{8-8} \cline{10-10} \cline{12-12} \cline{14-14} \cline{16-16} \cline{18-18}
                       &                                 & Bais                    & \multicolumn{1}{l|}{74.92} & \multicolumn{1}{l|}{}                       & \multicolumn{1}{l|}{0.73} & \multicolumn{1}{l|}{}                      & \multicolumn{1}{l|}{0.76} & \multicolumn{1}{l|}{}                      & \multicolumn{1}{l|}{0.74} &                       & \multicolumn{1}{l|}{77.39} & \multicolumn{1}{l|}{}                       & \multicolumn{1}{l|}{0.75} & \multicolumn{1}{l|}{}                      & \multicolumn{1}{l|}{0.80} & \multicolumn{1}{l|}{}                      & \multicolumn{1}{l|}{0.77} &                       \\ \cline{3-4} \cline{6-6} \cline{8-8} \cline{10-10} \cline{12-12} \cline{14-14} \cline{16-16} \cline{18-18}
                       &                                 & Distortion              & \multicolumn{1}{l|}{79.51} & \multicolumn{1}{l|}{}                       & \multicolumn{1}{l|}{0.75} & \multicolumn{1}{l|}{}                      & \multicolumn{1}{l|}{0.78} & \multicolumn{1}{l|}{}                      & \multicolumn{1}{l|}{0.76} &                       & \multicolumn{1}{l|}{81.87} & \multicolumn{1}{l|}{}                       & \multicolumn{1}{l|}{0.8}  & \multicolumn{1}{l|}{}                      & \multicolumn{1}{l|}{0.82} & \multicolumn{1}{l|}{}                      & \multicolumn{1}{l|}{0.81} &                       \\ \cline{3-4} \cline{6-6} \cline{8-8} \cline{10-10} \cline{12-12} \cline{14-14} \cline{16-16} \cline{18-18}
                       &                                 & Sound Factual           & \multicolumn{1}{l|}{83.50} & \multicolumn{1}{l|}{}                       & \multicolumn{1}{l|}{0.79} & \multicolumn{1}{l|}{}                      & \multicolumn{1}{l|}{0.83} & \multicolumn{1}{l|}{}                      & \multicolumn{1}{l|}{0.81} &                       & \multicolumn{1}{l|}{86.48} & \multicolumn{1}{l|}{}                       & \multicolumn{1}{l|}{0.83} & \multicolumn{1}{l|}{}                      & \multicolumn{1}{l|}{0.86} & \multicolumn{1}{l|}{}                      & \multicolumn{1}{l|}{0.84} &                       \\ \cline{2-19} 
                       & \multirow{4}{*}{Color of Lie}   & White                   & \multicolumn{1}{l|}{85.68} & \multicolumn{1}{l|}{\multirow{4}{*}{86.37}} & \multicolumn{1}{l|}{0.83} & \multicolumn{1}{l|}{\multirow{4}{*}{0.84}} & \multicolumn{1}{l|}{0.86} & \multicolumn{1}{l|}{\multirow{4}{*}{0.84}} & \multicolumn{1}{l|}{0.84} & \multirow{4}{*}{0.84} & \multicolumn{1}{l|}{88.95} & \multicolumn{1}{l|}{\multirow{4}{*}{88.84}} & \multicolumn{1}{l|}{0.86} & \multicolumn{1}{l|}{\multirow{4}{*}{0.87}} & \multicolumn{1}{l|}{0.88} & \multicolumn{1}{l|}{\multirow{4}{*}{0.88}} & \multicolumn{1}{l|}{0.87} & \multirow{4}{*}{0.87} \\ \cline{3-4} \cline{6-6} \cline{8-8} \cline{10-10} \cline{12-12} \cline{14-14} \cline{16-16} \cline{18-18}
                       &                                 & Grey                    & \multicolumn{1}{l|}{84.50} & \multicolumn{1}{l|}{}                       & \multicolumn{1}{l|}{0.87} & \multicolumn{1}{l|}{}                      & \multicolumn{1}{l|}{0.83} & \multicolumn{1}{l|}{}                      & \multicolumn{1}{l|}{0.85} &                       & \multicolumn{1}{l|}{86.38} & \multicolumn{1}{l|}{}                       & \multicolumn{1}{l|}{0.89} & \multicolumn{1}{l|}{}                      & \multicolumn{1}{l|}{0.85} & \multicolumn{1}{l|}{}                      & \multicolumn{1}{l|}{0.87} &                       \\ \cline{3-4} \cline{6-6} \cline{8-8} \cline{10-10} \cline{12-12} \cline{14-14} \cline{16-16} \cline{18-18}
                       &                                 & Red                     & \multicolumn{1}{l|}{86.87} & \multicolumn{1}{l|}{}                       & \multicolumn{1}{l|}{0.84} & \multicolumn{1}{l|}{}                      & \multicolumn{1}{l|}{0.83} & \multicolumn{1}{l|}{}                      & \multicolumn{1}{l|}{0.83} &                       & \multicolumn{1}{l|}{88.20} & \multicolumn{1}{l|}{}                       & \multicolumn{1}{l|}{0.87} & \multicolumn{1}{l|}{}                      & \multicolumn{1}{l|}{0.89} & \multicolumn{1}{l|}{}                      & \multicolumn{1}{l|}{0.88} &                       \\ \cline{3-4} \cline{6-6} \cline{8-8} \cline{10-10} \cline{12-12} \cline{14-14} \cline{16-16} \cline{18-18}
                       &                                 & Black                   & \multicolumn{1}{l|}{88.43} & \multicolumn{1}{l|}{}                       & \multicolumn{1}{l|}{0.82} & \multicolumn{1}{l|}{}                      & \multicolumn{1}{l|}{0.85} & \multicolumn{1}{l|}{}                      & \multicolumn{1}{l|}{0.83} &                       & \multicolumn{1}{l|}{91.83} & \multicolumn{1}{l|}{}                       & \multicolumn{1}{l|}{0.87} & \multicolumn{1}{l|}{}                      & \multicolumn{1}{l|}{0.90} & \multicolumn{1}{l|}{}                      & \multicolumn{1}{l|}{0.88} &                       \\ \cline{2-19} 
                       & \multirow{6}{*}{Intent of lie}  & Gaining Advantage       & \multicolumn{1}{l|}{87.62} & \multicolumn{1}{l|}{\multirow{6}{*}{83.69}} & \multicolumn{1}{l|}{0.85} & \multicolumn{1}{l|}{\multirow{6}{*}{0.84}} & \multicolumn{1}{l|}{0.83} & \multicolumn{1}{l|}{\multirow{6}{*}{0.79}} & \multicolumn{1}{l|}{0.84} & \multirow{6}{*}{0.81} & \multicolumn{1}{l|}{91.08} & \multicolumn{1}{l|}{\multirow{6}{*}{86.12}} & \multicolumn{1}{l|}{0.87} & \multicolumn{1}{l|}{\multirow{6}{*}{0.84}} & \multicolumn{1}{l|}{0.89} & \multicolumn{1}{l|}{\multirow{6}{*}{0.85}} & \multicolumn{1}{l|}{0.88} & \multirow{6}{*}{0.84} \\ \cline{3-4} \cline{6-6} \cline{8-8} \cline{10-10} \cline{12-12} \cline{14-14} \cline{16-16} \cline{18-18}
                       &                                 & Protecting Themselves   & \multicolumn{1}{l|}{84.87} & \multicolumn{1}{l|}{}                       & \multicolumn{1}{l|}{0.86} & \multicolumn{1}{l|}{}                      & \multicolumn{1}{l|}{0.81} & \multicolumn{1}{l|}{}                      & \multicolumn{1}{l|}{0.83} &                       & \multicolumn{1}{l|}{88.23} & \multicolumn{1}{l|}{}                       & \multicolumn{1}{l|}{0.84} & \multicolumn{1}{l|}{}                      & \multicolumn{1}{l|}{0.88} & \multicolumn{1}{l|}{}                      & \multicolumn{1}{l|}{0.86} &                       \\ \cline{3-4} \cline{6-6} \cline{8-8} \cline{10-10} \cline{12-12} \cline{14-14} \cline{16-16} \cline{18-18}
                       &                                 & Gaining Esteem          & \multicolumn{1}{l|}{82.97} & \multicolumn{1}{l|}{}                       & \multicolumn{1}{l|}{0.82} & \multicolumn{1}{l|}{}                      & \multicolumn{1}{l|}{0.77} & \multicolumn{1}{l|}{}                      & \multicolumn{1}{l|}{0.79} &                       & \multicolumn{1}{l|}{84.49} & \multicolumn{1}{l|}{}                       & \multicolumn{1}{l|}{0.85} & \multicolumn{1}{l|}{}                      & \multicolumn{1}{l|}{0.83} & \multicolumn{1}{l|}{}                      & \multicolumn{1}{l|}{0.84} &                       \\ \cline{3-4} \cline{6-6} \cline{8-8} \cline{10-10} \cline{12-12} \cline{14-14} \cline{16-16} \cline{18-18}
                       &                                 & Avoiding Embarrassment  & \multicolumn{1}{l|}{80.91} & \multicolumn{1}{l|}{}                       & \multicolumn{1}{l|}{0.84} & \multicolumn{1}{l|}{}                      & \multicolumn{1}{l|}{0.79} & \multicolumn{1}{l|}{}                      & \multicolumn{1}{l|}{0.81} &                       & \multicolumn{1}{l|}{82.97} & \multicolumn{1}{l|}{}                       & \multicolumn{1}{l|}{0.83} & \multicolumn{1}{l|}{}                      & \multicolumn{1}{l|}{0.80} & \multicolumn{1}{l|}{}                      & \multicolumn{1}{l|}{0.81} &                       \\ \cline{3-4} \cline{6-6} \cline{8-8} \cline{10-10} \cline{12-12} \cline{14-14} \cline{16-16} \cline{18-18}
                       &                                 & Defaming Esteem         & \multicolumn{1}{l|}{82.06} & \multicolumn{1}{l|}{}                       & \multicolumn{1}{l|}{0.83} & \multicolumn{1}{l|}{}                      & \multicolumn{1}{l|}{0.75} & \multicolumn{1}{l|}{}                      & \multicolumn{1}{l|}{0.79} &                       & \multicolumn{1}{l|}{83.87} & \multicolumn{1}{l|}{}                       & \multicolumn{1}{l|}{0.81} & \multicolumn{1}{l|}{}                      & \multicolumn{1}{l|}{0.84} & \multicolumn{1}{l|}{}                      & \multicolumn{1}{l|}{0.82} &                       \\ \cline{3-4} \cline{6-6} \cline{8-8} \cline{10-10} \cline{12-12} \cline{14-14} \cline{16-16} \cline{18-18}
                       &                                 & Protecting others       & \multicolumn{1}{l|}{80.11} & \multicolumn{1}{l|}{}                       & \multicolumn{1}{l|}{0.75} & \multicolumn{1}{l|}{}                      & \multicolumn{1}{l|}{0.79} & \multicolumn{1}{l|}{}                      & \multicolumn{1}{l|}{0.77} &                       & \multicolumn{1}{l|}{82.11} & \multicolumn{1}{l|}{}                       & \multicolumn{1}{l|}{0.79} & \multicolumn{1}{l|}{}                      & \multicolumn{1}{l|}{0.81} & \multicolumn{1}{l|}{}                      & \multicolumn{1}{l|}{0.8}  &                       \\ \cline{2-19} 
                       & \multirow{6}{*}{Topic of Lies} & Political               & \multicolumn{1}{l|}{88.70} & \multicolumn{1}{l|}{\multirow{6}{*}{83.60}} & \multicolumn{1}{l|}{0.82} & \multicolumn{1}{l|}{\multirow{6}{*}{0.81}} & \multicolumn{1}{l|}{0.86} & \multicolumn{1}{l|}{\multirow{6}{*}{0.82}} & \multicolumn{1}{l|}{0.84} & \multirow{6}{*}{0.81} & \multicolumn{1}{l|}{91.88} & \multicolumn{1}{l|}{\multirow{6}{*}{86.13}} & \multicolumn{1}{l|}{0.86} & \multicolumn{1}{l|}{\multirow{6}{*}{0.83}} & \multicolumn{1}{l|}{0.88} & \multicolumn{1}{l|}{\multirow{6}{*}{0.84}} & \multicolumn{1}{l|}{0.87} & \multirow{6}{*}{0.83} \\ \cline{3-4} \cline{6-6} \cline{8-8} \cline{10-10} \cline{12-12} \cline{14-14} \cline{16-16} \cline{18-18}
                       &                                 & Educational             & \multicolumn{1}{l|}{83.98} & \multicolumn{1}{l|}{}                       & \multicolumn{1}{l|}{0.84} & \multicolumn{1}{l|}{}                      & \multicolumn{1}{l|}{0.81} & \multicolumn{1}{l|}{}                      & \multicolumn{1}{l|}{0.82} &                       & \multicolumn{1}{l|}{86.79} & \multicolumn{1}{l|}{}                       & \multicolumn{1}{l|}{0.85} & \multicolumn{1}{l|}{}                      & \multicolumn{1}{l|}{0.86} & \multicolumn{1}{l|}{}                      & \multicolumn{1}{l|}{0.85} &                       \\ \cline{3-4} \cline{6-6} \cline{8-8} \cline{10-10} \cline{12-12} \cline{14-14} \cline{16-16} \cline{18-18}
                       &                                 & Regilious               & \multicolumn{1}{l|}{84.18} & \multicolumn{1}{l|}{}                       & \multicolumn{1}{l|}{0.81} & \multicolumn{1}{l|}{}                      & \multicolumn{1}{l|}{0.85} & \multicolumn{1}{l|}{}                      & \multicolumn{1}{l|}{0.83} &                       & \multicolumn{1}{l|}{84.98} & \multicolumn{1}{l|}{}                       & \multicolumn{1}{l|}{0.85} & \multicolumn{1}{l|}{}                      & \multicolumn{1}{l|}{0.83} & \multicolumn{1}{l|}{}                      & \multicolumn{1}{l|}{0.84} &                       \\ \cline{3-4} \cline{6-6} \cline{8-8} \cline{10-10} \cline{12-12} \cline{14-14} \cline{16-16} \cline{18-18}
                       &                                 & Ethnicity               & \multicolumn{1}{l|}{79.29} & \multicolumn{1}{l|}{}                       & \multicolumn{1}{l|}{0.83} & \multicolumn{1}{l|}{}                      & \multicolumn{1}{l|}{0.75} & \multicolumn{1}{l|}{}                      & \multicolumn{1}{l|}{0.79} &                       & \multicolumn{1}{l|}{83.84} & \multicolumn{1}{l|}{}                       & \multicolumn{1}{l|}{0.81} & \multicolumn{1}{l|}{}                      & \multicolumn{1}{l|}{0.82} & \multicolumn{1}{l|}{}                      & \multicolumn{1}{l|}{0.81} &                       \\ \cline{3-4} \cline{6-6} \cline{8-8} \cline{10-10} \cline{12-12} \cline{14-14} \cline{16-16} \cline{18-18}
                       &                                 & Racial                  & \multicolumn{1}{l|}{81.85} & \multicolumn{1}{l|}{}                       & \multicolumn{1}{l|}{0.77} & \multicolumn{1}{l|}{}                      & \multicolumn{1}{l|}{0.82} & \multicolumn{1}{l|}{}                      & \multicolumn{1}{l|}{0.79} &                       & \multicolumn{1}{l|}{83.16} & \multicolumn{1}{l|}{}                       & \multicolumn{1}{l|}{0.80} & \multicolumn{1}{l|}{}                      & \multicolumn{1}{l|}{0.79} & \multicolumn{1}{l|}{}                      & \multicolumn{1}{l|}{0.79} &                       \\ \cline{3-4} \cline{6-6} \cline{8-8} \cline{10-10} \cline{12-12} \cline{14-14} \cline{16-16} \cline{18-18}
                       &                                 & Other                   & \multicolumn{1}{l|}{76.95} & \multicolumn{1}{l|}{}                       & \multicolumn{1}{l|}{0.72} & \multicolumn{1}{l|}{}                      & \multicolumn{1}{l|}{0.77} & \multicolumn{1}{l|}{}                      & \multicolumn{1}{l|}{0.74} &                       & \multicolumn{1}{l|}{81.90} & \multicolumn{1}{l|}{}                       & \multicolumn{1}{l|}{0.76} & \multicolumn{1}{l|}{}                      & \multicolumn{1}{l|}{0.79} & \multicolumn{1}{l|}{}                      & \multicolumn{1}{l|}{0.77} &                       \\ 
                       % \hline
\midrule
% \hline
&                                  & Speculation              & \multicolumn{1}{l|}{85.67} & \multicolumn{1}{l|}{}                        & \multicolumn{1}{l|}{0.83} & \multicolumn{1}{l|}{}                       & \multicolumn{1}{l|}{0.81} & \multicolumn{1}{l|}{}                       & \multicolumn{1}{l|}{0.82} &                        & \multicolumn{1}{l|}{\cellcolor[HTML]{FFFE65}\textbf{89.91}} & \multicolumn{1}{l|}{\cellcolor[HTML]{FFFE65}}                                 & \multicolumn{1}{l|}{\cellcolor[HTML]{FFFE65}\textbf{0.86}} & \multicolumn{1}{l|}{\cellcolor[HTML]{FFFE65}}                                & \multicolumn{1}{l|}{\cellcolor[HTML]{FFFE65}\textbf{0.88}} & \multicolumn{1}{l|}{\cellcolor[HTML]{FFFE65}}                                & \multicolumn{1}{l|}{\cellcolor[HTML]{FFFE65}\textbf{0.87}} & \cellcolor[HTML]{FFFE65}                                \\ \cline{3-4} \cline{6-6} \cline{8-8} \cline{10-10} \cline{12-12} \cline{14-14} \cline{16-16} \cline{18-18}
                                                                                               &                                  & Opinion                  & \multicolumn{1}{l|}{83.40} & \multicolumn{1}{l|}{}                        & \multicolumn{1}{l|}{0.80} & \multicolumn{1}{l|}{}                       & \multicolumn{1}{l|}{0.82} & \multicolumn{1}{l|}{}                       & \multicolumn{1}{l|}{0.81} &                        & \multicolumn{1}{l|}{\cellcolor[HTML]{FFFE65}\textbf{87.09}} & \multicolumn{1}{l|}{\cellcolor[HTML]{FFFE65}}                                 & \multicolumn{1}{l|}{\cellcolor[HTML]{FFFE65}\textbf{0.84}} & \multicolumn{1}{l|}{\cellcolor[HTML]{FFFE65}}                                & \multicolumn{1}{l|}{\cellcolor[HTML]{FFFE65}\textbf{0.83}} & \multicolumn{1}{l|}{\cellcolor[HTML]{FFFE65}}                                & \multicolumn{1}{l|}{\cellcolor[HTML]{FFFE65}\textbf{0.83}} & \cellcolor[HTML]{FFFE65}                                \\ \cline{3-4} \cline{6-6} \cline{8-8} \cline{10-10} \cline{12-12} \cline{14-14} \cline{16-16} \cline{18-18}
                                                                                               &                                  & Bais                     & \multicolumn{1}{l|}{76.30} & \multicolumn{1}{l|}{}                        & \multicolumn{1}{l|}{0.77} & \multicolumn{1}{l|}{}                       & \multicolumn{1}{l|}{0.75} & \multicolumn{1}{l|}{}                       & \multicolumn{1}{l|}{0.76} &                        & \multicolumn{1}{l|}{\cellcolor[HTML]{FFFE65}\textbf{80.49}} & \multicolumn{1}{l|}{\cellcolor[HTML]{FFFE65}}                                 & \multicolumn{1}{l|}{\cellcolor[HTML]{FFFE65}\textbf{0.79}} & \multicolumn{1}{l|}{\cellcolor[HTML]{FFFE65}}                                & \multicolumn{1}{l|}{\cellcolor[HTML]{FFFE65}\textbf{0.83}} & \multicolumn{1}{l|}{\cellcolor[HTML]{FFFE65}}                                & \multicolumn{1}{l|}{\cellcolor[HTML]{FFFE65}\textbf{0.81}} & \cellcolor[HTML]{FFFE65}                                \\ \cline{3-4} \cline{6-6} \cline{8-8} \cline{10-10} \cline{12-12} \cline{14-14} \cline{16-16} \cline{18-18}
                                                                                               &                                  & Distortion               & \multicolumn{1}{l|}{80.44} & \multicolumn{1}{l|}{}                        & \multicolumn{1}{l|}{0.81} & \multicolumn{1}{l|}{}                       & \multicolumn{1}{l|}{0.79} & \multicolumn{1}{l|}{}                       & \multicolumn{1}{l|}{0.8}  &                        & \multicolumn{1}{l|}{\cellcolor[HTML]{FFFE65}\textbf{85.77}} & \multicolumn{1}{l|}{\cellcolor[HTML]{FFFE65}}                                 & \multicolumn{1}{l|}{\cellcolor[HTML]{FFFE65}\textbf{0.83}} & \multicolumn{1}{l|}{\cellcolor[HTML]{FFFE65}}                                & \multicolumn{1}{l|}{\cellcolor[HTML]{FFFE65}\textbf{0.85}} & \multicolumn{1}{l|}{\cellcolor[HTML]{FFFE65}}                                & \multicolumn{1}{l|}{\cellcolor[HTML]{FFFE65}\textbf{0.84}} & \cellcolor[HTML]{FFFE65}                                \\ \cline{3-4} \cline{6-6} \cline{8-8} \cline{10-10} \cline{12-12} \cline{14-14} \cline{16-16} \cline{18-18}
                                                                                               & \multirow{-5}{*}{Type of Omission}     & Sound Factual            & \multicolumn{1}{l|}{85.32} & \multicolumn{1}{l|}{\multirow{-5}{*}{82.22}} & \multicolumn{1}{l|}{0.84} & \multicolumn{1}{l|}{\multirow{-5}{*}{0.81}} & \multicolumn{1}{l|}{0.80} & \multicolumn{1}{l|}{\multirow{-5}{*}{0.79}} & \multicolumn{1}{l|}{0.82} & \multirow{-5}{*}{0.80} & \multicolumn{1}{l|}{\cellcolor[HTML]{FFFE65}\textbf{88.23}} & \multicolumn{1}{l|}{\multirow{-5}{*}{\cellcolor[HTML]{FFFE65}\textbf{86.30}}} & \multicolumn{1}{l|}{\cellcolor[HTML]{FFFE65}\textbf{0.86}} & \multicolumn{1}{l|}{\multirow{-5}{*}{\cellcolor[HTML]{FFFE65}\textbf{0.84}}} & \multicolumn{1}{l|}{\cellcolor[HTML]{FFFE65}\textbf{0.89}} & \multicolumn{1}{l|}{\multirow{-5}{*}{\cellcolor[HTML]{FFFE65}\textbf{0.86}}} & \multicolumn{1}{l|}{\cellcolor[HTML]{FFFE65}\textbf{0.87}} & \multirow{-5}{*}{\cellcolor[HTML]{FFFE65}\textbf{0.84}} \\ \cline{2-19} 
                                                                                               &                                  & White                    & \multicolumn{1}{l|}{87.36} & \multicolumn{1}{l|}{}                        & \multicolumn{1}{l|}{0.88} & \multicolumn{1}{l|}{}                       & \multicolumn{1}{l|}{0.86} & \multicolumn{1}{l|}{}                       & \multicolumn{1}{l|}{0.87} &                        & \multicolumn{1}{l|}{\cellcolor[HTML]{FFFE65}\textbf{91.23}} & \multicolumn{1}{l|}{\cellcolor[HTML]{FFFE65}}                                 & \multicolumn{1}{l|}{\cellcolor[HTML]{FFFE65}\textbf{0.90}} & \multicolumn{1}{l|}{\cellcolor[HTML]{FFFE65}}                                & \multicolumn{1}{l|}{\cellcolor[HTML]{FFFE65}\textbf{0.89}} & \multicolumn{1}{l|}{\cellcolor[HTML]{FFFE65}}                                & \multicolumn{1}{l|}{\cellcolor[HTML]{FFFE65}\textbf{0.90}} & \cellcolor[HTML]{FFFE65}                                \\ \cline{3-4} \cline{6-6} \cline{8-8} \cline{10-10} \cline{12-12} \cline{14-14} \cline{16-16} \cline{18-18}
                                                                                               &                                  & Grey                     & \multicolumn{1}{l|}{89.05} & \multicolumn{1}{l|}{}                        & \multicolumn{1}{l|}{0.88} & \multicolumn{1}{l|}{}                       & \multicolumn{1}{l|}{0.84} & \multicolumn{1}{l|}{}                       & \multicolumn{1}{l|}{0.86} &                        & \multicolumn{1}{l|}{\cellcolor[HTML]{FFFE65}\textbf{94.53}} & \multicolumn{1}{l|}{\cellcolor[HTML]{FFFE65}}                                 & \multicolumn{1}{l|}{\cellcolor[HTML]{FFFE65}\textbf{0.92}} & \multicolumn{1}{l|}{\cellcolor[HTML]{FFFE65}}                                & \multicolumn{1}{l|}{\cellcolor[HTML]{FFFE65}\textbf{0.88}} & \multicolumn{1}{l|}{\cellcolor[HTML]{FFFE65}}                                & \multicolumn{1}{l|}{\cellcolor[HTML]{FFFE65}\textbf{0.90}} & \cellcolor[HTML]{FFFE65}                                \\ \cline{3-4} \cline{6-6} \cline{8-8} \cline{10-10} \cline{12-12} \cline{14-14} \cline{16-16} \cline{18-18}
                                                                                               &                                  & Red                      & \multicolumn{1}{l|}{88.41} & \multicolumn{1}{l|}{}                        & \multicolumn{1}{l|}{0.86} & \multicolumn{1}{l|}{}                       & \multicolumn{1}{l|}{0.85} & \multicolumn{1}{l|}{}                       & \multicolumn{1}{l|}{0.85} &                        & \multicolumn{1}{l|}{\cellcolor[HTML]{FFFE65}\textbf{93.45}} & \multicolumn{1}{l|}{\cellcolor[HTML]{FFFE65}}                                 & \multicolumn{1}{l|}{\cellcolor[HTML]{FFFE65}\textbf{0.91}} & \multicolumn{1}{l|}{\cellcolor[HTML]{FFFE65}}                                & \multicolumn{1}{l|}{\cellcolor[HTML]{FFFE65}\textbf{0.92}} & \multicolumn{1}{l|}{\cellcolor[HTML]{FFFE65}}                                & \multicolumn{1}{l|}{\cellcolor[HTML]{FFFE65}\textbf{0.92}} & \cellcolor[HTML]{FFFE65}                                \\ \cline{3-4} \cline{6-6} \cline{8-8} \cline{10-10} \cline{12-12} \cline{14-14} \cline{16-16} \cline{18-18}
                                                                                               & \multirow{-4}{*}{Color of Lie}   & Black                    & \multicolumn{1}{l|}{91.62} & \multicolumn{1}{l|}{\multirow{-4}{*}{89.11}} & \multicolumn{1}{l|}{0.89} & \multicolumn{1}{l|}{\multirow{-4}{*}{0.88}} & \multicolumn{1}{l|}{0.85} & \multicolumn{1}{l|}{\multirow{-4}{*}{0.85}} & \multicolumn{1}{l|}{0.87} & \multirow{-4}{*}{0.86} & \multicolumn{1}{l|}{\cellcolor[HTML]{FFFE65}\textbf{96.17}} & \multicolumn{1}{l|}{\multirow{-4}{*}{\cellcolor[HTML]{FFFE65}\textbf{93.84}}} & \multicolumn{1}{l|}{\cellcolor[HTML]{FFFE65}\textbf{0.94}} & \multicolumn{1}{l|}{\multirow{-4}{*}{\cellcolor[HTML]{FFFE65}\textbf{0.92}}} & \multicolumn{1}{l|}{\cellcolor[HTML]{FFFE65}\textbf{0.93}} & \multicolumn{1}{l|}{\multirow{-4}{*}{\cellcolor[HTML]{FFFE65}\textbf{0.91}}} & \multicolumn{1}{l|}{\cellcolor[HTML]{FFFE65}\textbf{0.94}} & \multirow{-4}{*}{\cellcolor[HTML]{FFFE65}\textbf{0.92}} \\ \cline{2-19} 
                                                                                               &                                  & Gaining Advantage        & \multicolumn{1}{l|}{89.35} & \multicolumn{1}{l|}{}                        & \multicolumn{1}{l|}{0.88} & \multicolumn{1}{l|}{}                       & \multicolumn{1}{l|}{0.86} & \multicolumn{1}{l|}{}                       & \multicolumn{1}{l|}{0.87} &                        & \multicolumn{1}{l|}{\cellcolor[HTML]{FFFE65}\textbf{92.54}} & \multicolumn{1}{l|}{\cellcolor[HTML]{FFFE65}}                                 & \multicolumn{1}{l|}{\cellcolor[HTML]{FFFE65}\textbf{0.91}} & \multicolumn{1}{l|}{\cellcolor[HTML]{FFFE65}}                                & \multicolumn{1}{l|}{\cellcolor[HTML]{FFFE65}\textbf{0.93}} & \multicolumn{1}{l|}{\cellcolor[HTML]{FFFE65}}                                & \multicolumn{1}{l|}{\cellcolor[HTML]{FFFE65}\textbf{0.92}} & \cellcolor[HTML]{FFFE65}                                \\ \cline{3-4} \cline{6-6} \cline{8-8} \cline{10-10} \cline{12-12} \cline{14-14} \cline{16-16} \cline{18-18}
                                                                                               &                                  & Protecting Themselves    & \multicolumn{1}{l|}{88.74} & \multicolumn{1}{l|}{}                        & \multicolumn{1}{l|}{0.86} & \multicolumn{1}{l|}{}                       & \multicolumn{1}{l|}{0.85} & \multicolumn{1}{l|}{}                       & \multicolumn{1}{l|}{0.85} &                        & \multicolumn{1}{l|}{\cellcolor[HTML]{FFFE65}\textbf{90.78}} & \multicolumn{1}{l|}{\cellcolor[HTML]{FFFE65}}                                 & \multicolumn{1}{l|}{\cellcolor[HTML]{FFFE65}\textbf{0.89}} & \multicolumn{1}{l|}{\cellcolor[HTML]{FFFE65}}                                & \multicolumn{1}{l|}{\cellcolor[HTML]{FFFE65}\textbf{0.90}} & \multicolumn{1}{l|}{\cellcolor[HTML]{FFFE65}}                                & \multicolumn{1}{l|}{\cellcolor[HTML]{FFFE65}\textbf{0.89}} & \cellcolor[HTML]{FFFE65}                                \\ \cline{3-4} \cline{6-6} \cline{8-8} \cline{10-10} \cline{12-12} \cline{14-14} \cline{16-16} \cline{18-18}
                                                                                               &                                  & Gaining Esteem           & \multicolumn{1}{l|}{85.67} & \multicolumn{1}{l|}{}                        & \multicolumn{1}{l|}{0.85} & \multicolumn{1}{l|}{}                       & \multicolumn{1}{l|}{0.82} & \multicolumn{1}{l|}{}                       & \multicolumn{1}{l|}{0.83} &                        & \multicolumn{1}{l|}{\cellcolor[HTML]{FFFE65}\textbf{88.56}} & \multicolumn{1}{l|}{\cellcolor[HTML]{FFFE65}}                                 & \multicolumn{1}{l|}{\cellcolor[HTML]{FFFE65}\textbf{0.88}} & \multicolumn{1}{l|}{\cellcolor[HTML]{FFFE65}}                                & \multicolumn{1}{l|}{\cellcolor[HTML]{FFFE65}\textbf{0.86}} & \multicolumn{1}{l|}{\cellcolor[HTML]{FFFE65}}                                & \multicolumn{1}{l|}{\cellcolor[HTML]{FFFE65}\textbf{0.87}} & \cellcolor[HTML]{FFFE65}                                \\ \cline{3-4} \cline{6-6} \cline{8-8} \cline{10-10} \cline{12-12} \cline{14-14} \cline{16-16} \cline{18-18}
                                                                                               &                                  & Avoiding Embarrassment   & \multicolumn{1}{l|}{83.25} & \multicolumn{1}{l|}{}                        & \multicolumn{1}{l|}{0.82} & \multicolumn{1}{l|}{}                       & \multicolumn{1}{l|}{0.83} & \multicolumn{1}{l|}{}                       & \multicolumn{1}{l|}{0.82} &                        & \multicolumn{1}{l|}{\cellcolor[HTML]{FFFE65}\textbf{87.19}} & \multicolumn{1}{l|}{\cellcolor[HTML]{FFFE65}}                                 & \multicolumn{1}{l|}{\cellcolor[HTML]{FFFE65}\textbf{0.85}} & \multicolumn{1}{l|}{\cellcolor[HTML]{FFFE65}}                                & \multicolumn{1}{l|}{\cellcolor[HTML]{FFFE65}\textbf{0.88}} & \multicolumn{1}{l|}{\cellcolor[HTML]{FFFE65}}                                & \multicolumn{1}{l|}{\cellcolor[HTML]{FFFE65}\textbf{0.86}} & \cellcolor[HTML]{FFFE65}                                \\ \cline{3-4} \cline{6-6} \cline{8-8} \cline{10-10} \cline{12-12} \cline{14-14} \cline{16-16} \cline{18-18}
                                                                                               &                                  & Defaming Esteem          & \multicolumn{1}{l|}{83.46} & \multicolumn{1}{l|}{}                        & \multicolumn{1}{l|}{0.83} & \multicolumn{1}{l|}{}                       & \multicolumn{1}{l|}{0.82} & \multicolumn{1}{l|}{}                       & \multicolumn{1}{l|}{0.82} &                        & \multicolumn{1}{l|}{\cellcolor[HTML]{FFFE65}\textbf{86.88}} & \multicolumn{1}{l|}{\cellcolor[HTML]{FFFE65}}                                 & \multicolumn{1}{l|}{\cellcolor[HTML]{FFFE65}\textbf{0.85}} & \multicolumn{1}{l|}{\cellcolor[HTML]{FFFE65}}                                & \multicolumn{1}{l|}{\cellcolor[HTML]{FFFE65}\textbf{0.84}} & \multicolumn{1}{l|}{\cellcolor[HTML]{FFFE65}}                                & \multicolumn{1}{l|}{\cellcolor[HTML]{FFFE65}\textbf{0.84}} & \cellcolor[HTML]{FFFE65}                                \\ \cline{3-4} \cline{6-6} \cline{8-8} \cline{10-10} \cline{12-12} \cline{14-14} \cline{16-16} \cline{18-18}
                                                                                               & \multirow{-6}{*}{Intent of lie}  & Protecting others        & \multicolumn{1}{l|}{81.16} & \multicolumn{1}{l|}{\multirow{-6}{*}{86.09}} & \multicolumn{1}{l|}{0.80} & \multicolumn{1}{l|}{\multirow{-6}{*}{0.85}} & \multicolumn{1}{l|}{0.79} & \multicolumn{1}{l|}{\multirow{-6}{*}{0.84}} & \multicolumn{1}{l|}{0.79} & \multirow{-6}{*}{0.84} & \multicolumn{1}{l|}{\cellcolor[HTML]{FFFE65}\textbf{85.04}} & \multicolumn{1}{l|}{\multirow{-6}{*}{\cellcolor[HTML]{FFFE65}\textbf{88.49}}} & \multicolumn{1}{l|}{\cellcolor[HTML]{FFFE65}\textbf{0.83}} & \multicolumn{1}{l|}{\multirow{-6}{*}{\cellcolor[HTML]{FFFE65}\textbf{0.87}}} & \multicolumn{1}{l|}{\cellcolor[HTML]{FFFE65}\textbf{0.84}} & \multicolumn{1}{l|}{\multirow{-6}{*}{\cellcolor[HTML]{FFFE65}\textbf{0.88}}} & \multicolumn{1}{l|}{\cellcolor[HTML]{FFFE65}\textbf{0.83}} & \multirow{-6}{*}{\cellcolor[HTML]{FFFE65}\textbf{0.87}} \\ \cline{2-19} 
                                                                                               &                                  & Political                & \multicolumn{1}{l|}{90.59} & \multicolumn{1}{l|}{}                        & \multicolumn{1}{l|}{0.88} & \multicolumn{1}{l|}{}                       & \multicolumn{1}{l|}{0.86} & \multicolumn{1}{l|}{}                       & \multicolumn{1}{l|}{0.87} &                        & \multicolumn{1}{l|}{\cellcolor[HTML]{FFFE65}\textbf{94.16}} & \multicolumn{1}{l|}{\cellcolor[HTML]{FFFE65}}                                 & \multicolumn{1}{l|}{\cellcolor[HTML]{FFFE65}\textbf{0.93}} & \multicolumn{1}{l|}{\cellcolor[HTML]{FFFE65}}                                & \multicolumn{1}{l|}{\cellcolor[HTML]{FFFE65}\textbf{0.90}} & \multicolumn{1}{l|}{\cellcolor[HTML]{FFFE65}}                                & \multicolumn{1}{l|}{\cellcolor[HTML]{FFFE65}\textbf{0.91}} & \cellcolor[HTML]{FFFE65}                                \\ \cline{3-4} \cline{6-6} \cline{8-8} \cline{10-10} \cline{12-12} \cline{14-14} \cline{16-16} \cline{18-18}
                                                                                               &                                  & Educational              & \multicolumn{1}{l|}{86.77} & \multicolumn{1}{l|}{}                        & \multicolumn{1}{l|}{0.87} & \multicolumn{1}{l|}{}                       & \multicolumn{1}{l|}{0.88} & \multicolumn{1}{l|}{}                       & \multicolumn{1}{l|}{0.87} &                        & \multicolumn{1}{l|}{\cellcolor[HTML]{FFFE65}\textbf{90.66}} & \multicolumn{1}{l|}{\cellcolor[HTML]{FFFE65}}                                 & \multicolumn{1}{l|}{\cellcolor[HTML]{FFFE65}\textbf{0.90}} & \multicolumn{1}{l|}{\cellcolor[HTML]{FFFE65}}                                & \multicolumn{1}{l|}{\cellcolor[HTML]{FFFE65}\textbf{0.87}} & \multicolumn{1}{l|}{\cellcolor[HTML]{FFFE65}}                                & \multicolumn{1}{l|}{\cellcolor[HTML]{FFFE65}\textbf{0.88}} & \cellcolor[HTML]{FFFE65}                                \\ \cline{3-4} \cline{6-6} \cline{8-8} \cline{10-10} \cline{12-12} \cline{14-14} \cline{16-16} \cline{18-18}
                                                                                               &                                  & Regilious                & \multicolumn{1}{l|}{85.46} & \multicolumn{1}{l|}{}                        & \multicolumn{1}{l|}{0.84} & \multicolumn{1}{l|}{}                       & \multicolumn{1}{l|}{0.84} & \multicolumn{1}{l|}{}                       & \multicolumn{1}{l|}{0.84} &                        & \multicolumn{1}{l|}{\cellcolor[HTML]{FFFE65}\textbf{87.83}} & \multicolumn{1}{l|}{\cellcolor[HTML]{FFFE65}}                                 & \multicolumn{1}{l|}{\cellcolor[HTML]{FFFE65}\textbf{0.87}} & \multicolumn{1}{l|}{\cellcolor[HTML]{FFFE65}}                                & \multicolumn{1}{l|}{\cellcolor[HTML]{FFFE65}\textbf{0.85}} & \multicolumn{1}{l|}{\cellcolor[HTML]{FFFE65}}                                & \multicolumn{1}{l|}{\cellcolor[HTML]{FFFE65}\textbf{0.86}} & \cellcolor[HTML]{FFFE65}                                \\ \cline{3-4} \cline{6-6} \cline{8-8} \cline{10-10} \cline{12-12} \cline{14-14} \cline{16-16} \cline{18-18}
                                                                                               &                                  & Ethnicity                & \multicolumn{1}{l|}{84.69} & \multicolumn{1}{l|}{}                        & \multicolumn{1}{l|}{0.84} & \multicolumn{1}{l|}{}                       & \multicolumn{1}{l|}{0.85} & \multicolumn{1}{l|}{}                       & \multicolumn{1}{l|}{0.84} &                        & \multicolumn{1}{l|}{\cellcolor[HTML]{FFFE65}\textbf{88.67}} & \multicolumn{1}{l|}{\cellcolor[HTML]{FFFE65}}                                 & \multicolumn{1}{l|}{\cellcolor[HTML]{FFFE65}\textbf{0.86}} & \multicolumn{1}{l|}{\cellcolor[HTML]{FFFE65}}                                & \multicolumn{1}{l|}{\cellcolor[HTML]{FFFE65}\textbf{0.87}} & \multicolumn{1}{l|}{\cellcolor[HTML]{FFFE65}}                                & \multicolumn{1}{l|}{\cellcolor[HTML]{FFFE65}\textbf{0.86}} & \cellcolor[HTML]{FFFE65}                                \\ \cline{3-4} \cline{6-6} \cline{8-8} \cline{10-10} \cline{12-12} \cline{14-14} \cline{16-16} \cline{18-18}
                                                                                               &                                  & Racial                   & \multicolumn{1}{l|}{81.84} & \multicolumn{1}{l|}{}                        & \multicolumn{1}{l|}{0.83} & \multicolumn{1}{l|}{}                       & \multicolumn{1}{l|}{0.82} & \multicolumn{1}{l|}{}                       & \multicolumn{1}{l|}{0.82} &                        & \multicolumn{1}{l|}{\cellcolor[HTML]{FFFE65}\textbf{85.89}} & \multicolumn{1}{l|}{\cellcolor[HTML]{FFFE65}}                                 & \multicolumn{1}{l|}{\cellcolor[HTML]{FFFE65}\textbf{0.87}} & \multicolumn{1}{l|}{\cellcolor[HTML]{FFFE65}}                                & \multicolumn{1}{l|}{\cellcolor[HTML]{FFFE65}\textbf{0.84}} & \multicolumn{1}{l|}{\cellcolor[HTML]{FFFE65}}                                & \multicolumn{1}{l|}{\cellcolor[HTML]{FFFE65}\textbf{0.85}} & \cellcolor[HTML]{FFFE65}                                \\ \cline{3-4} \cline{6-6} \cline{8-8} \cline{10-10} \cline{12-12} \cline{14-14} \cline{16-16} \cline{18-18}
\multirow{-21}{*}{\textbf{\begin{tabular}[c]{@{}c@{}}T5 with\\ Transformer \\ Encoder\end{tabular}}}
 & \multirow{-6}{*}{Topic of Lies} & Other                    & \multicolumn{1}{l|}{79.18} & \multicolumn{1}{l|}{\multirow{-6}{*}{85.87}} & \multicolumn{1}{l|}{0.78} & \multicolumn{1}{l|}{\multirow{-6}{*}{0.85}} & \multicolumn{1}{l|}{0.78} & \multicolumn{1}{l|}{\multirow{-6}{*}{0.85}} & \multicolumn{1}{l|}{0.78} & \multirow{-6}{*}{0.85} & \multicolumn{1}{l|}{\cellcolor[HTML]{FFFE65}\textbf{82.34}} & \multicolumn{1}{l|}{\multirow{-6}{*}{\cellcolor[HTML]{FFFE65}\textbf{88.26}}} & \multicolumn{1}{l|}{\cellcolor[HTML]{FFFE65}\textbf{0.84}} & \multicolumn{1}{l|}{\multirow{-6}{*}{\cellcolor[HTML]{FFFE65}\textbf{0.87}}} & \multicolumn{1}{l|}{\cellcolor[HTML]{FFFE65}\textbf{0.81}} & \multicolumn{1}{l|}{\multirow{-6}{*}{\cellcolor[HTML]{FFFE65}\textbf{0.86}}} & \multicolumn{1}{l|}{\cellcolor[HTML]{FFFE65}\textbf{0.82}} & \multirow{-6}{*}{\cellcolor[HTML]{FFFE65}\textbf{0.86}} \\ \hline
                    
\end{tabular}
}
\caption{Experiment results: The table showcases the results obtained from different experiments using varying encoder architectures, namely LSTM and Transformer. The term "Without Model Merging" refers to the utilization of the T5-3b model without any fine-tuning. Conversely, the term "With Model Merging" signifies the fine-tuning of four T5 models, each corresponding to a distinct layer, followed by Dataless Knowledge fusion. \cite{jin2022dataless}}\
\label{tab:overall_exp}
\end{table}









% % Please add the following required packages to your document preamble:
% % \usepackage{multirow}
% \begin{table}[!tbh]
% \resizebox{\columnwidth}{!}{
% \begin{tabular}{l|l|l|llllllll||llllllll}
% 	\toprule
% \multirow{2}{*}{}      & \multirow{2}{*}{\textbf{SEPSIS}}         & \multirow{2}{*}{\textbf{Labels}} & \multicolumn{8}{c||}{\textbf{Without Model Merging}}                                                                                                                                                                                                                                     & \multicolumn{8}{c|}{\textbf{With Model Merging}}                                                                                                                                                                                                                                        \\ \cline{4-19} 
%                        &                                 &                         & \multicolumn{2}{c|}{\textbf{Accuracy} \%}                                         & \multicolumn{2}{l|}{\textbf{Precision}}                                       & \multicolumn{2}{l|}{\textbf{Recall}}                                            & \multicolumn{2}{l||}{\textbf{F1-Score}}                     & \multicolumn{2}{l|}{\textbf{Accuracy} \%}                                         & \multicolumn{2}{l}{\textbf{Precision}}                                         & \multicolumn{2}{l|}{\textbf{Recall}}                                            & \multicolumn{2}{l|}{\textbf{F1-Score}}                     \\
                       
%                        \toprule
%                        \multirow{21}{*}{\textbf{\parbox[c]{2.5cm}{\centering T5 with \\ LSTM \\ encoder}}}


%  & \multirow{5}{*}{Type of Omission}     & Speculation             & \multicolumn{1}{l|}{82.58} & \multicolumn{1}{l|}{\multirow{5}{*}{80.25}} & \multicolumn{1}{l|}{0.78} & \multicolumn{1}{l|}{\multirow{5}{*}{0.77}} & \multicolumn{1}{l|}{0.83} & \multicolumn{1}{l|}{\multirow{5}{*}{0.80}} & \multicolumn{1}{l|}{0.8}  & \multirow{5}{*}{0.78} & \multicolumn{1}{l|}{86.15} & \multicolumn{1}{l|}{\multirow{5}{*}{82.89}} & \multicolumn{1}{l|}{0.84} & \multicolumn{1}{l|}{\multirow{5}{*}{0.81}} & \multicolumn{1}{l|}{0.85} & \multicolumn{1}{l|}{\multirow{5}{*}{0.83}} & \multicolumn{1}{l|}{0.84} & \multirow{5}{*}{0.82} \\ \cline{3-4} \cline{6-6} \cline{8-8} \cline{10-10} \cline{12-12} \cline{14-14} \cline{16-16} \cline{18-18}
%                        &                                 & Opinion                 & \multicolumn{1}{l|}{80.76} & \multicolumn{1}{l|}{}                       & \multicolumn{1}{l|}{0.80} & \multicolumn{1}{l|}{}                      & \multicolumn{1}{l|}{0.79} & \multicolumn{1}{l|}{}                      & \multicolumn{1}{l|}{0.79} &                       & \multicolumn{1}{l|}{82.54} & \multicolumn{1}{l|}{}                       & \multicolumn{1}{l|}{0.82} & \multicolumn{1}{l|}{}                      & \multicolumn{1}{l|}{0.81} & \multicolumn{1}{l|}{}                      & \multicolumn{1}{l|}{0.81} &                       \\ \cline{3-4} \cline{6-6} \cline{8-8} \cline{10-10} \cline{12-12} \cline{14-14} \cline{16-16} \cline{18-18}
%                        &                                 & Bais                    & \multicolumn{1}{l|}{74.92} & \multicolumn{1}{l|}{}                       & \multicolumn{1}{l|}{0.73} & \multicolumn{1}{l|}{}                      & \multicolumn{1}{l|}{0.76} & \multicolumn{1}{l|}{}                      & \multicolumn{1}{l|}{0.74} &                       & \multicolumn{1}{l|}{77.39} & \multicolumn{1}{l|}{}                       & \multicolumn{1}{l|}{0.75} & \multicolumn{1}{l|}{}                      & \multicolumn{1}{l|}{0.80} & \multicolumn{1}{l|}{}                      & \multicolumn{1}{l|}{0.77} &                       \\ \cline{3-4} \cline{6-6} \cline{8-8} \cline{10-10} \cline{12-12} \cline{14-14} \cline{16-16} \cline{18-18}
%                        &                                 & Distortion              & \multicolumn{1}{l|}{79.51} & \multicolumn{1}{l|}{}                       & \multicolumn{1}{l|}{0.75} & \multicolumn{1}{l|}{}                      & \multicolumn{1}{l|}{0.78} & \multicolumn{1}{l|}{}                      & \multicolumn{1}{l|}{0.76} &                       & \multicolumn{1}{l|}{81.87} & \multicolumn{1}{l|}{}                       & \multicolumn{1}{l|}{0.8}  & \multicolumn{1}{l|}{}                      & \multicolumn{1}{l|}{0.82} & \multicolumn{1}{l|}{}                      & \multicolumn{1}{l|}{0.81} &                       \\ \cline{3-4} \cline{6-6} \cline{8-8} \cline{10-10} \cline{12-12} \cline{14-14} \cline{16-16} \cline{18-18}
%                        &                                 & Sound Factual           & \multicolumn{1}{l|}{83.50} & \multicolumn{1}{l|}{}                       & \multicolumn{1}{l|}{0.79} & \multicolumn{1}{l|}{}                      & \multicolumn{1}{l|}{0.83} & \multicolumn{1}{l|}{}                      & \multicolumn{1}{l|}{0.81} &                       & \multicolumn{1}{l|}{86.48} & \multicolumn{1}{l|}{}                       & \multicolumn{1}{l|}{0.83} & \multicolumn{1}{l|}{}                      & \multicolumn{1}{l|}{0.86} & \multicolumn{1}{l|}{}                      & \multicolumn{1}{l|}{0.84} &                       \\ \cline{2-19} 
%                        & \multirow{4}{*}{Color of Lie}   & White                   & \multicolumn{1}{l|}{85.68} & \multicolumn{1}{l|}{\multirow{4}{*}{86.37}} & \multicolumn{1}{l|}{0.83} & \multicolumn{1}{l|}{\multirow{4}{*}{0.84}} & \multicolumn{1}{l|}{0.86} & \multicolumn{1}{l|}{\multirow{4}{*}{0.84}} & \multicolumn{1}{l|}{0.84} & \multirow{4}{*}{0.84} & \multicolumn{1}{l|}{88.95} & \multicolumn{1}{l|}{\multirow{4}{*}{88.84}} & \multicolumn{1}{l|}{0.86} & \multicolumn{1}{l|}{\multirow{4}{*}{0.87}} & \multicolumn{1}{l|}{0.88} & \multicolumn{1}{l|}{\multirow{4}{*}{0.88}} & \multicolumn{1}{l|}{0.87} & \multirow{4}{*}{0.87} \\ \cline{3-4} \cline{6-6} \cline{8-8} \cline{10-10} \cline{12-12} \cline{14-14} \cline{16-16} \cline{18-18}
%                        &                                 & Grey                    & \multicolumn{1}{l|}{84.50} & \multicolumn{1}{l|}{}                       & \multicolumn{1}{l|}{0.87} & \multicolumn{1}{l|}{}                      & \multicolumn{1}{l|}{0.83} & \multicolumn{1}{l|}{}                      & \multicolumn{1}{l|}{0.85} &                       & \multicolumn{1}{l|}{86.38} & \multicolumn{1}{l|}{}                       & \multicolumn{1}{l|}{0.89} & \multicolumn{1}{l|}{}                      & \multicolumn{1}{l|}{0.85} & \multicolumn{1}{l|}{}                      & \multicolumn{1}{l|}{0.87} &                       \\ \cline{3-4} \cline{6-6} \cline{8-8} \cline{10-10} \cline{12-12} \cline{14-14} \cline{16-16} \cline{18-18}
%                        &                                 & Red                     & \multicolumn{1}{l|}{86.87} & \multicolumn{1}{l|}{}                       & \multicolumn{1}{l|}{0.84} & \multicolumn{1}{l|}{}                      & \multicolumn{1}{l|}{0.83} & \multicolumn{1}{l|}{}                      & \multicolumn{1}{l|}{0.83} &                       & \multicolumn{1}{l|}{88.20} & \multicolumn{1}{l|}{}                       & \multicolumn{1}{l|}{0.87} & \multicolumn{1}{l|}{}                      & \multicolumn{1}{l|}{0.89} & \multicolumn{1}{l|}{}                      & \multicolumn{1}{l|}{0.88} &                       \\ \cline{3-4} \cline{6-6} \cline{8-8} \cline{10-10} \cline{12-12} \cline{14-14} \cline{16-16} \cline{18-18}
%                        &                                 & Black                   & \multicolumn{1}{l|}{88.43} & \multicolumn{1}{l|}{}                       & \multicolumn{1}{l|}{0.82} & \multicolumn{1}{l|}{}                      & \multicolumn{1}{l|}{0.85} & \multicolumn{1}{l|}{}                      & \multicolumn{1}{l|}{0.83} &                       & \multicolumn{1}{l|}{91.83} & \multicolumn{1}{l|}{}                       & \multicolumn{1}{l|}{0.87} & \multicolumn{1}{l|}{}                      & \multicolumn{1}{l|}{0.90} & \multicolumn{1}{l|}{}                      & \multicolumn{1}{l|}{0.88} &                       \\ \cline{2-19} 
%                        & \multirow{6}{*}{Intent of lie}  & Gaining Advantage       & \multicolumn{1}{l|}{87.62} & \multicolumn{1}{l|}{\multirow{6}{*}{83.69}} & \multicolumn{1}{l|}{0.85} & \multicolumn{1}{l|}{\multirow{6}{*}{0.84}} & \multicolumn{1}{l|}{0.83} & \multicolumn{1}{l|}{\multirow{6}{*}{0.79}} & \multicolumn{1}{l|}{0.84} & \multirow{6}{*}{0.81} & \multicolumn{1}{l|}{91.08} & \multicolumn{1}{l|}{\multirow{6}{*}{86.12}} & \multicolumn{1}{l|}{0.87} & \multicolumn{1}{l|}{\multirow{6}{*}{0.84}} & \multicolumn{1}{l|}{0.89} & \multicolumn{1}{l|}{\multirow{6}{*}{0.85}} & \multicolumn{1}{l|}{0.88} & \multirow{6}{*}{0.84} \\ \cline{3-4} \cline{6-6} \cline{8-8} \cline{10-10} \cline{12-12} \cline{14-14} \cline{16-16} \cline{18-18}
%                        &                                 & Protecting Themselves   & \multicolumn{1}{l|}{84.87} & \multicolumn{1}{l|}{}                       & \multicolumn{1}{l|}{0.86} & \multicolumn{1}{l|}{}                      & \multicolumn{1}{l|}{0.81} & \multicolumn{1}{l|}{}                      & \multicolumn{1}{l|}{0.83} &                       & \multicolumn{1}{l|}{88.23} & \multicolumn{1}{l|}{}                       & \multicolumn{1}{l|}{0.84} & \multicolumn{1}{l|}{}                      & \multicolumn{1}{l|}{0.88} & \multicolumn{1}{l|}{}                      & \multicolumn{1}{l|}{0.86} &                       \\ \cline{3-4} \cline{6-6} \cline{8-8} \cline{10-10} \cline{12-12} \cline{14-14} \cline{16-16} \cline{18-18}
%                        &                                 & Gaining Esteem          & \multicolumn{1}{l|}{82.97} & \multicolumn{1}{l|}{}                       & \multicolumn{1}{l|}{0.82} & \multicolumn{1}{l|}{}                      & \multicolumn{1}{l|}{0.77} & \multicolumn{1}{l|}{}                      & \multicolumn{1}{l|}{0.79} &                       & \multicolumn{1}{l|}{84.49} & \multicolumn{1}{l|}{}                       & \multicolumn{1}{l|}{0.85} & \multicolumn{1}{l|}{}                      & \multicolumn{1}{l|}{0.83} & \multicolumn{1}{l|}{}                      & \multicolumn{1}{l|}{0.84} &                       \\ \cline{3-4} \cline{6-6} \cline{8-8} \cline{10-10} \cline{12-12} \cline{14-14} \cline{16-16} \cline{18-18}
%                        &                                 & Avoiding Embarrassment  & \multicolumn{1}{l|}{80.91} & \multicolumn{1}{l|}{}                       & \multicolumn{1}{l|}{0.84} & \multicolumn{1}{l|}{}                      & \multicolumn{1}{l|}{0.79} & \multicolumn{1}{l|}{}                      & \multicolumn{1}{l|}{0.81} &                       & \multicolumn{1}{l|}{82.97} & \multicolumn{1}{l|}{}                       & \multicolumn{1}{l|}{0.83} & \multicolumn{1}{l|}{}                      & \multicolumn{1}{l|}{0.80} & \multicolumn{1}{l|}{}                      & \multicolumn{1}{l|}{0.81} &                       \\ \cline{3-4} \cline{6-6} \cline{8-8} \cline{10-10} \cline{12-12} \cline{14-14} \cline{16-16} \cline{18-18}
%                        &                                 & Defaming Esteem         & \multicolumn{1}{l|}{82.06} & \multicolumn{1}{l|}{}                       & \multicolumn{1}{l|}{0.83} & \multicolumn{1}{l|}{}                      & \multicolumn{1}{l|}{0.75} & \multicolumn{1}{l|}{}                      & \multicolumn{1}{l|}{0.79} &                       & \multicolumn{1}{l|}{83.87} & \multicolumn{1}{l|}{}                       & \multicolumn{1}{l|}{0.81} & \multicolumn{1}{l|}{}                      & \multicolumn{1}{l|}{0.84} & \multicolumn{1}{l|}{}                      & \multicolumn{1}{l|}{0.82} &                       \\ \cline{3-4} \cline{6-6} \cline{8-8} \cline{10-10} \cline{12-12} \cline{14-14} \cline{16-16} \cline{18-18}
%                        &                                 & Protecting others       & \multicolumn{1}{l|}{80.11} & \multicolumn{1}{l|}{}                       & \multicolumn{1}{l|}{0.75} & \multicolumn{1}{l|}{}                      & \multicolumn{1}{l|}{0.79} & \multicolumn{1}{l|}{}                      & \multicolumn{1}{l|}{0.77} &                       & \multicolumn{1}{l|}{82.11} & \multicolumn{1}{l|}{}                       & \multicolumn{1}{l|}{0.79} & \multicolumn{1}{l|}{}                      & \multicolumn{1}{l|}{0.81} & \multicolumn{1}{l|}{}                      & \multicolumn{1}{l|}{0.8}  &                       \\ \cline{2-19} 
%                        & \multirow{6}{*}{Topic of Lies} & Political               & \multicolumn{1}{l|}{88.70} & \multicolumn{1}{l|}{\multirow{6}{*}{83.60}} & \multicolumn{1}{l|}{0.82} & \multicolumn{1}{l|}{\multirow{6}{*}{0.81}} & \multicolumn{1}{l|}{0.86} & \multicolumn{1}{l|}{\multirow{6}{*}{0.82}} & \multicolumn{1}{l|}{0.84} & \multirow{6}{*}{0.81} & \multicolumn{1}{l|}{91.88} & \multicolumn{1}{l|}{\multirow{6}{*}{86.13}} & \multicolumn{1}{l|}{0.86} & \multicolumn{1}{l|}{\multirow{6}{*}{0.83}} & \multicolumn{1}{l|}{0.88} & \multicolumn{1}{l|}{\multirow{6}{*}{0.84}} & \multicolumn{1}{l|}{0.87} & \multirow{6}{*}{0.83} \\ \cline{3-4} \cline{6-6} \cline{8-8} \cline{10-10} \cline{12-12} \cline{14-14} \cline{16-16} \cline{18-18}
%                        &                                 & Educational             & \multicolumn{1}{l|}{83.98} & \multicolumn{1}{l|}{}                       & \multicolumn{1}{l|}{0.84} & \multicolumn{1}{l|}{}                      & \multicolumn{1}{l|}{0.81} & \multicolumn{1}{l|}{}                      & \multicolumn{1}{l|}{0.82} &                       & \multicolumn{1}{l|}{86.79} & \multicolumn{1}{l|}{}                       & \multicolumn{1}{l|}{0.85} & \multicolumn{1}{l|}{}                      & \multicolumn{1}{l|}{0.86} & \multicolumn{1}{l|}{}                      & \multicolumn{1}{l|}{0.85} &                       \\ \cline{3-4} \cline{6-6} \cline{8-8} \cline{10-10} \cline{12-12} \cline{14-14} \cline{16-16} \cline{18-18}
%                        &                                 & Regilious               & \multicolumn{1}{l|}{84.18} & \multicolumn{1}{l|}{}                       & \multicolumn{1}{l|}{0.81} & \multicolumn{1}{l|}{}                      & \multicolumn{1}{l|}{0.85} & \multicolumn{1}{l|}{}                      & \multicolumn{1}{l|}{0.83} &                       & \multicolumn{1}{l|}{84.98} & \multicolumn{1}{l|}{}                       & \multicolumn{1}{l|}{0.85} & \multicolumn{1}{l|}{}                      & \multicolumn{1}{l|}{0.83} & \multicolumn{1}{l|}{}                      & \multicolumn{1}{l|}{0.84} &                       \\ \cline{3-4} \cline{6-6} \cline{8-8} \cline{10-10} \cline{12-12} \cline{14-14} \cline{16-16} \cline{18-18}
%                        &                                 & Ethnicity               & \multicolumn{1}{l|}{79.29} & \multicolumn{1}{l|}{}                       & \multicolumn{1}{l|}{0.83} & \multicolumn{1}{l|}{}                      & \multicolumn{1}{l|}{0.75} & \multicolumn{1}{l|}{}                      & \multicolumn{1}{l|}{0.79} &                       & \multicolumn{1}{l|}{83.84} & \multicolumn{1}{l|}{}                       & \multicolumn{1}{l|}{0.81} & \multicolumn{1}{l|}{}                      & \multicolumn{1}{l|}{0.82} & \multicolumn{1}{l|}{}                      & \multicolumn{1}{l|}{0.81} &                       \\ \cline{3-4} \cline{6-6} \cline{8-8} \cline{10-10} \cline{12-12} \cline{14-14} \cline{16-16} \cline{18-18}
%                        &                                 & Racial                  & \multicolumn{1}{l|}{81.85} & \multicolumn{1}{l|}{}                       & \multicolumn{1}{l|}{0.77} & \multicolumn{1}{l|}{}                      & \multicolumn{1}{l|}{0.82} & \multicolumn{1}{l|}{}                      & \multicolumn{1}{l|}{0.79} &                       & \multicolumn{1}{l|}{83.16} & \multicolumn{1}{l|}{}                       & \multicolumn{1}{l|}{0.80} & \multicolumn{1}{l|}{}                      & \multicolumn{1}{l|}{0.79} & \multicolumn{1}{l|}{}                      & \multicolumn{1}{l|}{0.79} &                       \\ \cline{3-4} \cline{6-6} \cline{8-8} \cline{10-10} \cline{12-12} \cline{14-14} \cline{16-16} \cline{18-18}
%                        &                                 & Other                   & \multicolumn{1}{l|}{76.95} & \multicolumn{1}{l|}{}                       & \multicolumn{1}{l|}{0.72} & \multicolumn{1}{l|}{}                      & \multicolumn{1}{l|}{0.77} & \multicolumn{1}{l|}{}                      & \multicolumn{1}{l|}{0.74} &                       & \multicolumn{1}{l|}{81.90} & \multicolumn{1}{l|}{}                       & \multicolumn{1}{l|}{0.76} & \multicolumn{1}{l|}{}                      & \multicolumn{1}{l|}{0.79} & \multicolumn{1}{l|}{}                      & \multicolumn{1}{l|}{0.77} &                       \\ 
%                        % \hline
% \midrule
% % \hline
% &                                  & Speculation              & \multicolumn{1}{l|}{85.67} & \multicolumn{1}{l|}{}                        & \multicolumn{1}{l|}{0.83} & \multicolumn{1}{l|}{}                       & \multicolumn{1}{l|}{0.81} & \multicolumn{1}{l|}{}                       & \multicolumn{1}{l|}{0.82} &                        & \multicolumn{1}{l|}{\cellcolor[HTML]{FFFE65}\textbf{89.91}} & \multicolumn{1}{l|}{\cellcolor[HTML]{FFFE65}}                                 & \multicolumn{1}{l|}{\cellcolor[HTML]{FFFE65}\textbf{0.86}} & \multicolumn{1}{l|}{\cellcolor[HTML]{FFFE65}}                                & \multicolumn{1}{l|}{\cellcolor[HTML]{FFFE65}\textbf{0.88}} & \multicolumn{1}{l|}{\cellcolor[HTML]{FFFE65}}                                & \multicolumn{1}{l|}{\cellcolor[HTML]{FFFE65}\textbf{0.87}} & \cellcolor[HTML]{FFFE65}                                \\ \cline{3-4} \cline{6-6} \cline{8-8} \cline{10-10} \cline{12-12} \cline{14-14} \cline{16-16} \cline{18-18}
%                                                                                                &                                  & Opinion                  & \multicolumn{1}{l|}{83.40} & \multicolumn{1}{l|}{}                        & \multicolumn{1}{l|}{0.80} & \multicolumn{1}{l|}{}                       & \multicolumn{1}{l|}{0.82} & \multicolumn{1}{l|}{}                       & \multicolumn{1}{l|}{0.81} &                        & \multicolumn{1}{l|}{\cellcolor[HTML]{FFFE65}\textbf{87.09}} & \multicolumn{1}{l|}{\cellcolor[HTML]{FFFE65}}                                 & \multicolumn{1}{l|}{\cellcolor[HTML]{FFFE65}\textbf{0.84}} & \multicolumn{1}{l|}{\cellcolor[HTML]{FFFE65}}                                & \multicolumn{1}{l|}{\cellcolor[HTML]{FFFE65}\textbf{0.83}} & \multicolumn{1}{l|}{\cellcolor[HTML]{FFFE65}}                                & \multicolumn{1}{l|}{\cellcolor[HTML]{FFFE65}\textbf{0.83}} & \cellcolor[HTML]{FFFE65}                                \\ \cline{3-4} \cline{6-6} \cline{8-8} \cline{10-10} \cline{12-12} \cline{14-14} \cline{16-16} \cline{18-18}
%                                                                                                &                                  & Bais                     & \multicolumn{1}{l|}{76.30} & \multicolumn{1}{l|}{}                        & \multicolumn{1}{l|}{0.77} & \multicolumn{1}{l|}{}                       & \multicolumn{1}{l|}{0.75} & \multicolumn{1}{l|}{}                       & \multicolumn{1}{l|}{0.76} &                        & \multicolumn{1}{l|}{\cellcolor[HTML]{FFFE65}\textbf{80.49}} & \multicolumn{1}{l|}{\cellcolor[HTML]{FFFE65}}                                 & \multicolumn{1}{l|}{\cellcolor[HTML]{FFFE65}\textbf{0.79}} & \multicolumn{1}{l|}{\cellcolor[HTML]{FFFE65}}                                & \multicolumn{1}{l|}{\cellcolor[HTML]{FFFE65}\textbf{0.83}} & \multicolumn{1}{l|}{\cellcolor[HTML]{FFFE65}}                                & \multicolumn{1}{l|}{\cellcolor[HTML]{FFFE65}\textbf{0.81}} & \cellcolor[HTML]{FFFE65}                                \\ \cline{3-4} \cline{6-6} \cline{8-8} \cline{10-10} \cline{12-12} \cline{14-14} \cline{16-16} \cline{18-18}
%                                                                                                &                                  & Distortion               & \multicolumn{1}{l|}{80.44} & \multicolumn{1}{l|}{}                        & \multicolumn{1}{l|}{0.81} & \multicolumn{1}{l|}{}                       & \multicolumn{1}{l|}{0.79} & \multicolumn{1}{l|}{}                       & \multicolumn{1}{l|}{0.8}  &                        & \multicolumn{1}{l|}{\cellcolor[HTML]{FFFE65}\textbf{85.77}} & \multicolumn{1}{l|}{\cellcolor[HTML]{FFFE65}}                                 & \multicolumn{1}{l|}{\cellcolor[HTML]{FFFE65}\textbf{0.83}} & \multicolumn{1}{l|}{\cellcolor[HTML]{FFFE65}}                                & \multicolumn{1}{l|}{\cellcolor[HTML]{FFFE65}\textbf{0.85}} & \multicolumn{1}{l|}{\cellcolor[HTML]{FFFE65}}                                & \multicolumn{1}{l|}{\cellcolor[HTML]{FFFE65}\textbf{0.84}} & \cellcolor[HTML]{FFFE65}                                \\ \cline{3-4} \cline{6-6} \cline{8-8} \cline{10-10} \cline{12-12} \cline{14-14} \cline{16-16} \cline{18-18}
%                                                                                                & \multirow{-5}{*}{Type of Omission}     & Sound Factual            & \multicolumn{1}{l|}{85.32} & \multicolumn{1}{l|}{\multirow{-5}{*}{82.22}} & \multicolumn{1}{l|}{0.84} & \multicolumn{1}{l|}{\multirow{-5}{*}{0.81}} & \multicolumn{1}{l|}{0.80} & \multicolumn{1}{l|}{\multirow{-5}{*}{0.79}} & \multicolumn{1}{l|}{0.82} & \multirow{-5}{*}{0.80} & \multicolumn{1}{l|}{\cellcolor[HTML]{FFFE65}\textbf{88.23}} & \multicolumn{1}{l|}{\multirow{-5}{*}{\cellcolor[HTML]{FFFE65}\textbf{86.30}}} & \multicolumn{1}{l|}{\cellcolor[HTML]{FFFE65}\textbf{0.86}} & \multicolumn{1}{l|}{\multirow{-5}{*}{\cellcolor[HTML]{FFFE65}\textbf{0.84}}} & \multicolumn{1}{l|}{\cellcolor[HTML]{FFFE65}\textbf{0.89}} & \multicolumn{1}{l|}{\multirow{-5}{*}{\cellcolor[HTML]{FFFE65}\textbf{0.86}}} & \multicolumn{1}{l|}{\cellcolor[HTML]{FFFE65}\textbf{0.87}} & \multirow{-5}{*}{\cellcolor[HTML]{FFFE65}\textbf{0.84}} \\ \cline{2-19} 
%                                                                                                &                                  & White                    & \multicolumn{1}{l|}{87.36} & \multicolumn{1}{l|}{}                        & \multicolumn{1}{l|}{0.88} & \multicolumn{1}{l|}{}                       & \multicolumn{1}{l|}{0.86} & \multicolumn{1}{l|}{}                       & \multicolumn{1}{l|}{0.87} &                        & \multicolumn{1}{l|}{\cellcolor[HTML]{FFFE65}\textbf{91.23}} & \multicolumn{1}{l|}{\cellcolor[HTML]{FFFE65}}                                 & \multicolumn{1}{l|}{\cellcolor[HTML]{FFFE65}\textbf{0.90}} & \multicolumn{1}{l|}{\cellcolor[HTML]{FFFE65}}                                & \multicolumn{1}{l|}{\cellcolor[HTML]{FFFE65}\textbf{0.89}} & \multicolumn{1}{l|}{\cellcolor[HTML]{FFFE65}}                                & \multicolumn{1}{l|}{\cellcolor[HTML]{FFFE65}\textbf{0.90}} & \cellcolor[HTML]{FFFE65}                                \\ \cline{3-4} \cline{6-6} \cline{8-8} \cline{10-10} \cline{12-12} \cline{14-14} \cline{16-16} \cline{18-18}
%                                                                                                &                                  & Grey                     & \multicolumn{1}{l|}{89.05} & \multicolumn{1}{l|}{}                        & \multicolumn{1}{l|}{0.88} & \multicolumn{1}{l|}{}                       & \multicolumn{1}{l|}{0.84} & \multicolumn{1}{l|}{}                       & \multicolumn{1}{l|}{0.86} &                        & \multicolumn{1}{l|}{\cellcolor[HTML]{FFFE65}\textbf{94.53}} & \multicolumn{1}{l|}{\cellcolor[HTML]{FFFE65}}                                 & \multicolumn{1}{l|}{\cellcolor[HTML]{FFFE65}\textbf{0.92}} & \multicolumn{1}{l|}{\cellcolor[HTML]{FFFE65}}                                & \multicolumn{1}{l|}{\cellcolor[HTML]{FFFE65}\textbf{0.88}} & \multicolumn{1}{l|}{\cellcolor[HTML]{FFFE65}}                                & \multicolumn{1}{l|}{\cellcolor[HTML]{FFFE65}\textbf{0.90}} & \cellcolor[HTML]{FFFE65}                                \\ \cline{3-4} \cline{6-6} \cline{8-8} \cline{10-10} \cline{12-12} \cline{14-14} \cline{16-16} \cline{18-18}
%                                                                                                &                                  & Red                      & \multicolumn{1}{l|}{88.41} & \multicolumn{1}{l|}{}                        & \multicolumn{1}{l|}{0.86} & \multicolumn{1}{l|}{}                       & \multicolumn{1}{l|}{0.85} & \multicolumn{1}{l|}{}                       & \multicolumn{1}{l|}{0.85} &                        & \multicolumn{1}{l|}{\cellcolor[HTML]{FFFE65}\textbf{93.45}} & \multicolumn{1}{l|}{\cellcolor[HTML]{FFFE65}}                                 & \multicolumn{1}{l|}{\cellcolor[HTML]{FFFE65}\textbf{0.91}} & \multicolumn{1}{l|}{\cellcolor[HTML]{FFFE65}}                                & \multicolumn{1}{l|}{\cellcolor[HTML]{FFFE65}\textbf{0.92}} & \multicolumn{1}{l|}{\cellcolor[HTML]{FFFE65}}                                & \multicolumn{1}{l|}{\cellcolor[HTML]{FFFE65}\textbf{0.92}} & \cellcolor[HTML]{FFFE65}                                \\ \cline{3-4} \cline{6-6} \cline{8-8} \cline{10-10} \cline{12-12} \cline{14-14} \cline{16-16} \cline{18-18}
%                                                                                                & \multirow{-4}{*}{Color of Lie}   & Black                    & \multicolumn{1}{l|}{91.62} & \multicolumn{1}{l|}{\multirow{-4}{*}{89.11}} & \multicolumn{1}{l|}{0.89} & \multicolumn{1}{l|}{\multirow{-4}{*}{0.88}} & \multicolumn{1}{l|}{0.85} & \multicolumn{1}{l|}{\multirow{-4}{*}{0.85}} & \multicolumn{1}{l|}{0.87} & \multirow{-4}{*}{0.86} & \multicolumn{1}{l|}{\cellcolor[HTML]{FFFE65}\textbf{96.17}} & \multicolumn{1}{l|}{\multirow{-4}{*}{\cellcolor[HTML]{FFFE65}\textbf{93.84}}} & \multicolumn{1}{l|}{\cellcolor[HTML]{FFFE65}\textbf{0.94}} & \multicolumn{1}{l|}{\multirow{-4}{*}{\cellcolor[HTML]{FFFE65}\textbf{0.92}}} & \multicolumn{1}{l|}{\cellcolor[HTML]{FFFE65}\textbf{0.93}} & \multicolumn{1}{l|}{\multirow{-4}{*}{\cellcolor[HTML]{FFFE65}\textbf{0.91}}} & \multicolumn{1}{l|}{\cellcolor[HTML]{FFFE65}\textbf{0.94}} & \multirow{-4}{*}{\cellcolor[HTML]{FFFE65}\textbf{0.92}} \\ \cline{2-19} 
%                                                                                                &                                  & Gaining Advantage        & \multicolumn{1}{l|}{89.35} & \multicolumn{1}{l|}{}                        & \multicolumn{1}{l|}{0.88} & \multicolumn{1}{l|}{}                       & \multicolumn{1}{l|}{0.86} & \multicolumn{1}{l|}{}                       & \multicolumn{1}{l|}{0.87} &                        & \multicolumn{1}{l|}{\cellcolor[HTML]{FFFE65}\textbf{92.54}} & \multicolumn{1}{l|}{\cellcolor[HTML]{FFFE65}}                                 & \multicolumn{1}{l|}{\cellcolor[HTML]{FFFE65}\textbf{0.91}} & \multicolumn{1}{l|}{\cellcolor[HTML]{FFFE65}}                                & \multicolumn{1}{l|}{\cellcolor[HTML]{FFFE65}\textbf{0.93}} & \multicolumn{1}{l|}{\cellcolor[HTML]{FFFE65}}                                & \multicolumn{1}{l|}{\cellcolor[HTML]{FFFE65}\textbf{0.92}} & \cellcolor[HTML]{FFFE65}                                \\ \cline{3-4} \cline{6-6} \cline{8-8} \cline{10-10} \cline{12-12} \cline{14-14} \cline{16-16} \cline{18-18}
%                                                                                                &                                  & Protecting Themselves    & \multicolumn{1}{l|}{88.74} & \multicolumn{1}{l|}{}                        & \multicolumn{1}{l|}{0.86} & \multicolumn{1}{l|}{}                       & \multicolumn{1}{l|}{0.85} & \multicolumn{1}{l|}{}                       & \multicolumn{1}{l|}{0.85} &                        & \multicolumn{1}{l|}{\cellcolor[HTML]{FFFE65}\textbf{90.78}} & \multicolumn{1}{l|}{\cellcolor[HTML]{FFFE65}}                                 & \multicolumn{1}{l|}{\cellcolor[HTML]{FFFE65}\textbf{0.89}} & \multicolumn{1}{l|}{\cellcolor[HTML]{FFFE65}}                                & \multicolumn{1}{l|}{\cellcolor[HTML]{FFFE65}\textbf{0.90}} & \multicolumn{1}{l|}{\cellcolor[HTML]{FFFE65}}                                & \multicolumn{1}{l|}{\cellcolor[HTML]{FFFE65}\textbf{0.89}} & \cellcolor[HTML]{FFFE65}                                \\ \cline{3-4} \cline{6-6} \cline{8-8} \cline{10-10} \cline{12-12} \cline{14-14} \cline{16-16} \cline{18-18}
%                                                                                                &                                  & Gaining Esteem           & \multicolumn{1}{l|}{85.67} & \multicolumn{1}{l|}{}                        & \multicolumn{1}{l|}{0.85} & \multicolumn{1}{l|}{}                       & \multicolumn{1}{l|}{0.82} & \multicolumn{1}{l|}{}                       & \multicolumn{1}{l|}{0.83} &                        & \multicolumn{1}{l|}{\cellcolor[HTML]{FFFE65}\textbf{88.56}} & \multicolumn{1}{l|}{\cellcolor[HTML]{FFFE65}}                                 & \multicolumn{1}{l|}{\cellcolor[HTML]{FFFE65}\textbf{0.88}} & \multicolumn{1}{l|}{\cellcolor[HTML]{FFFE65}}                                & \multicolumn{1}{l|}{\cellcolor[HTML]{FFFE65}\textbf{0.86}} & \multicolumn{1}{l|}{\cellcolor[HTML]{FFFE65}}                                & \multicolumn{1}{l|}{\cellcolor[HTML]{FFFE65}\textbf{0.87}} & \cellcolor[HTML]{FFFE65}                                \\ \cline{3-4} \cline{6-6} \cline{8-8} \cline{10-10} \cline{12-12} \cline{14-14} \cline{16-16} \cline{18-18}
%                                                                                                &                                  & Avoiding Embarrassment   & \multicolumn{1}{l|}{83.25} & \multicolumn{1}{l|}{}                        & \multicolumn{1}{l|}{0.82} & \multicolumn{1}{l|}{}                       & \multicolumn{1}{l|}{0.83} & \multicolumn{1}{l|}{}                       & \multicolumn{1}{l|}{0.82} &                        & \multicolumn{1}{l|}{\cellcolor[HTML]{FFFE65}\textbf{87.19}} & \multicolumn{1}{l|}{\cellcolor[HTML]{FFFE65}}                                 & \multicolumn{1}{l|}{\cellcolor[HTML]{FFFE65}\textbf{0.85}} & \multicolumn{1}{l|}{\cellcolor[HTML]{FFFE65}}                                & \multicolumn{1}{l|}{\cellcolor[HTML]{FFFE65}\textbf{0.88}} & \multicolumn{1}{l|}{\cellcolor[HTML]{FFFE65}}                                & \multicolumn{1}{l|}{\cellcolor[HTML]{FFFE65}\textbf{0.86}} & \cellcolor[HTML]{FFFE65}                                \\ \cline{3-4} \cline{6-6} \cline{8-8} \cline{10-10} \cline{12-12} \cline{14-14} \cline{16-16} \cline{18-18}
%                                                                                                &                                  & Defaming Esteem          & \multicolumn{1}{l|}{83.46} & \multicolumn{1}{l|}{}                        & \multicolumn{1}{l|}{0.83} & \multicolumn{1}{l|}{}                       & \multicolumn{1}{l|}{0.82} & \multicolumn{1}{l|}{}                       & \multicolumn{1}{l|}{0.82} &                        & \multicolumn{1}{l|}{\cellcolor[HTML]{FFFE65}\textbf{86.88}} & \multicolumn{1}{l|}{\cellcolor[HTML]{FFFE65}}                                 & \multicolumn{1}{l|}{\cellcolor[HTML]{FFFE65}\textbf{0.85}} & \multicolumn{1}{l|}{\cellcolor[HTML]{FFFE65}}                                & \multicolumn{1}{l|}{\cellcolor[HTML]{FFFE65}\textbf{0.84}} & \multicolumn{1}{l|}{\cellcolor[HTML]{FFFE65}}                                & \multicolumn{1}{l|}{\cellcolor[HTML]{FFFE65}\textbf{0.84}} & \cellcolor[HTML]{FFFE65}                                \\ \cline{3-4} \cline{6-6} \cline{8-8} \cline{10-10} \cline{12-12} \cline{14-14} \cline{16-16} \cline{18-18}
%                                                                                                & \multirow{-6}{*}{Intent of lie}  & Protecting others        & \multicolumn{1}{l|}{81.16} & \multicolumn{1}{l|}{\multirow{-6}{*}{86.09}} & \multicolumn{1}{l|}{0.80} & \multicolumn{1}{l|}{\multirow{-6}{*}{0.85}} & \multicolumn{1}{l|}{0.79} & \multicolumn{1}{l|}{\multirow{-6}{*}{0.84}} & \multicolumn{1}{l|}{0.79} & \multirow{-6}{*}{0.84} & \multicolumn{1}{l|}{\cellcolor[HTML]{FFFE65}\textbf{85.04}} & \multicolumn{1}{l|}{\multirow{-6}{*}{\cellcolor[HTML]{FFFE65}\textbf{88.49}}} & \multicolumn{1}{l|}{\cellcolor[HTML]{FFFE65}\textbf{0.83}} & \multicolumn{1}{l|}{\multirow{-6}{*}{\cellcolor[HTML]{FFFE65}\textbf{0.87}}} & \multicolumn{1}{l|}{\cellcolor[HTML]{FFFE65}\textbf{0.84}} & \multicolumn{1}{l|}{\multirow{-6}{*}{\cellcolor[HTML]{FFFE65}\textbf{0.88}}} & \multicolumn{1}{l|}{\cellcolor[HTML]{FFFE65}\textbf{0.83}} & \multirow{-6}{*}{\cellcolor[HTML]{FFFE65}\textbf{0.87}} \\ \cline{2-19} 
%                                                                                                &                                  & Political                & \multicolumn{1}{l|}{90.59} & \multicolumn{1}{l|}{}                        & \multicolumn{1}{l|}{0.88} & \multicolumn{1}{l|}{}                       & \multicolumn{1}{l|}{0.86} & \multicolumn{1}{l|}{}                       & \multicolumn{1}{l|}{0.87} &                        & \multicolumn{1}{l|}{\cellcolor[HTML]{FFFE65}\textbf{94.16}} & \multicolumn{1}{l|}{\cellcolor[HTML]{FFFE65}}                                 & \multicolumn{1}{l|}{\cellcolor[HTML]{FFFE65}\textbf{0.93}} & \multicolumn{1}{l|}{\cellcolor[HTML]{FFFE65}}                                & \multicolumn{1}{l|}{\cellcolor[HTML]{FFFE65}\textbf{0.90}} & \multicolumn{1}{l|}{\cellcolor[HTML]{FFFE65}}                                & \multicolumn{1}{l|}{\cellcolor[HTML]{FFFE65}\textbf{0.91}} & \cellcolor[HTML]{FFFE65}                                \\ \cline{3-4} \cline{6-6} \cline{8-8} \cline{10-10} \cline{12-12} \cline{14-14} \cline{16-16} \cline{18-18}
%                                                                                                &                                  & Educational              & \multicolumn{1}{l|}{86.77} & \multicolumn{1}{l|}{}                        & \multicolumn{1}{l|}{0.87} & \multicolumn{1}{l|}{}                       & \multicolumn{1}{l|}{0.88} & \multicolumn{1}{l|}{}                       & \multicolumn{1}{l|}{0.87} &                        & \multicolumn{1}{l|}{\cellcolor[HTML]{FFFE65}\textbf{90.66}} & \multicolumn{1}{l|}{\cellcolor[HTML]{FFFE65}}                                 & \multicolumn{1}{l|}{\cellcolor[HTML]{FFFE65}\textbf{0.90}} & \multicolumn{1}{l|}{\cellcolor[HTML]{FFFE65}}                                & \multicolumn{1}{l|}{\cellcolor[HTML]{FFFE65}\textbf{0.87}} & \multicolumn{1}{l|}{\cellcolor[HTML]{FFFE65}}                                & \multicolumn{1}{l|}{\cellcolor[HTML]{FFFE65}\textbf{0.88}} & \cellcolor[HTML]{FFFE65}                                \\ \cline{3-4} \cline{6-6} \cline{8-8} \cline{10-10} \cline{12-12} \cline{14-14} \cline{16-16} \cline{18-18}
%                                                                                                &                                  & Regilious                & \multicolumn{1}{l|}{85.46} & \multicolumn{1}{l|}{}                        & \multicolumn{1}{l|}{0.84} & \multicolumn{1}{l|}{}                       & \multicolumn{1}{l|}{0.84} & \multicolumn{1}{l|}{}                       & \multicolumn{1}{l|}{0.84} &                        & \multicolumn{1}{l|}{\cellcolor[HTML]{FFFE65}\textbf{87.83}} & \multicolumn{1}{l|}{\cellcolor[HTML]{FFFE65}}                                 & \multicolumn{1}{l|}{\cellcolor[HTML]{FFFE65}\textbf{0.87}} & \multicolumn{1}{l|}{\cellcolor[HTML]{FFFE65}}                                & \multicolumn{1}{l|}{\cellcolor[HTML]{FFFE65}\textbf{0.85}} & \multicolumn{1}{l|}{\cellcolor[HTML]{FFFE65}}                                & \multicolumn{1}{l|}{\cellcolor[HTML]{FFFE65}\textbf{0.86}} & \cellcolor[HTML]{FFFE65}                                \\ \cline{3-4} \cline{6-6} \cline{8-8} \cline{10-10} \cline{12-12} \cline{14-14} \cline{16-16} \cline{18-18}
%                                                                                                &                                  & Ethnicity                & \multicolumn{1}{l|}{84.69} & \multicolumn{1}{l|}{}                        & \multicolumn{1}{l|}{0.84} & \multicolumn{1}{l|}{}                       & \multicolumn{1}{l|}{0.85} & \multicolumn{1}{l|}{}                       & \multicolumn{1}{l|}{0.84} &                        & \multicolumn{1}{l|}{\cellcolor[HTML]{FFFE65}\textbf{88.67}} & \multicolumn{1}{l|}{\cellcolor[HTML]{FFFE65}}                                 & \multicolumn{1}{l|}{\cellcolor[HTML]{FFFE65}\textbf{0.86}} & \multicolumn{1}{l|}{\cellcolor[HTML]{FFFE65}}                                & \multicolumn{1}{l|}{\cellcolor[HTML]{FFFE65}\textbf{0.87}} & \multicolumn{1}{l|}{\cellcolor[HTML]{FFFE65}}                                & \multicolumn{1}{l|}{\cellcolor[HTML]{FFFE65}\textbf{0.86}} & \cellcolor[HTML]{FFFE65}                                \\ \cline{3-4} \cline{6-6} \cline{8-8} \cline{10-10} \cline{12-12} \cline{14-14} \cline{16-16} \cline{18-18}
%                                                                                                &                                  & Racial                   & \multicolumn{1}{l|}{81.84} & \multicolumn{1}{l|}{}                        & \multicolumn{1}{l|}{0.83} & \multicolumn{1}{l|}{}                       & \multicolumn{1}{l|}{0.82} & \multicolumn{1}{l|}{}                       & \multicolumn{1}{l|}{0.82} &                        & \multicolumn{1}{l|}{\cellcolor[HTML]{FFFE65}\textbf{85.89}} & \multicolumn{1}{l|}{\cellcolor[HTML]{FFFE65}}                                 & \multicolumn{1}{l|}{\cellcolor[HTML]{FFFE65}\textbf{0.87}} & \multicolumn{1}{l|}{\cellcolor[HTML]{FFFE65}}                                & \multicolumn{1}{l|}{\cellcolor[HTML]{FFFE65}\textbf{0.84}} & \multicolumn{1}{l|}{\cellcolor[HTML]{FFFE65}}                                & \multicolumn{1}{l|}{\cellcolor[HTML]{FFFE65}\textbf{0.85}} & \cellcolor[HTML]{FFFE65}                                \\ \cline{3-4} \cline{6-6} \cline{8-8} \cline{10-10} \cline{12-12} \cline{14-14} \cline{16-16} \cline{18-18}
% \multirow{-21}{*}{\textbf{\begin{tabular}[c]{@{}c@{}}T5 with\\ Transformer \\ Encoder\end{tabular}}}
%  & \multirow{-6}{*}{Topic of Lies} & Other                    & \multicolumn{1}{l|}{79.18} & \multicolumn{1}{l|}{\multirow{-6}{*}{85.87}} & \multicolumn{1}{l|}{0.78} & \multicolumn{1}{l|}{\multirow{-6}{*}{0.85}} & \multicolumn{1}{l|}{0.78} & \multicolumn{1}{l|}{\multirow{-6}{*}{0.85}} & \multicolumn{1}{l|}{0.78} & \multirow{-6}{*}{0.85} & \multicolumn{1}{l|}{\cellcolor[HTML]{FFFE65}\textbf{82.34}} & \multicolumn{1}{l|}{\multirow{-6}{*}{\cellcolor[HTML]{FFFE65}\textbf{88.26}}} & \multicolumn{1}{l|}{\cellcolor[HTML]{FFFE65}\textbf{0.84}} & \multicolumn{1}{l|}{\multirow{-6}{*}{\cellcolor[HTML]{FFFE65}\textbf{0.87}}} & \multicolumn{1}{l|}{\cellcolor[HTML]{FFFE65}\textbf{0.81}} & \multicolumn{1}{l|}{\multirow{-6}{*}{\cellcolor[HTML]{FFFE65}\textbf{0.86}}} & \multicolumn{1}{l|}{\cellcolor[HTML]{FFFE65}\textbf{0.82}} & \multirow{-6}{*}{\cellcolor[HTML]{FFFE65}\textbf{0.86}} \\ \hline
                    
% \end{tabular}
% }
% \caption{Experiment results: The table showcases the results obtained from different experiments using varying encoder architectures, namely LSTM and Transformer. The term "Without Model Merging" refers to the utilization of the T5-3b model without any fine-tuning. Conversely, the term "With Model Merging" signifies the fine-tuning of four T5 models, each corresponding to a distinct layer, followed by Dataless Knowledge fusion. \cite{jin2022dataless}}\
% \label{tab:overall_exp}
% \end{table}

%\newpage
\newpage
\section{Propaganda Techniques}\label{sec: Propaganda}
%\vspace{-5mm}
Propaganda techniques are strategies used to manipulate and influence people's opinions, emotions, and behavior in order to promote a particular agenda or ideology \cite{da-san-martino-etal-2019-fine, martino2020survey}. These techniques are often employed in mass media, advertising, politics, and public relations. While they can vary in their specific methods, we present definitions of 18 propaganda techniques that we have used in this study in the left box in the subsequent section. In the box on the right side, we present insights from propaganda techniques through deception.
%\newpage

%\documentclass{article}
%\usepackage[most]{tcolorbox}
%\usepackage{lipsum}
%\begin{document}
%\lipsum[2]
\begin{tcbraster}[raster columns=2,raster equal height]
%%%%%%%%%
\begin{tcolorbox}[nobeforeafter, title=box 1,colback=brown!5!white,colframe=brown!75!black,title=\footnotesize{\textbf{\textsc{Propaganda Technique Definition}}}]
%\lipsum[2]
\begin{itemize}
[leftmargin=1mm]
\setlength\itemsep{0em}
\begin{spacing}{0.90}
\vspace{-2.2mm}
\item[\ding{224}] {\footnotesize \fontfamily{phv}\fontsize{8}{9}\selectfont{\textbf{Flag Waving:} Playing on strong national feeling (or to any group, e.g., race, gender, etc) to justify or promote an action or an idea.
}}
% \end{spacing}}
\vspace{-2.2mm}
\item[\ding{224}] {\footnotesize 
{\fontfamily{phv}\fontsize{8}{9}\selectfont{\textbf{Slogans:}A brief and striking phrase that may include labeling and stereotyping.
}
}}

\vspace{-2.2mm}
\item[\ding{224}] {\footnotesize 
{\fontfamily{phv}\fontsize{8}{9}\selectfont
{\textbf{Appeal to fear - prejudices:}Seeking to build support for an idea by instilling anxiety and/or panic in the population towards an alternative.}
}}
\vspace{-2.2mm}
\item[\ding{224}] {\footnotesize 
{\fontfamily{phv}\fontsize{8}{9}\selectfont
{\textbf{Exaggeration-Minimization}: Either representing something in an excessive manner: making things larger, better, worse (e.g., the best of the best) or making something seem less important or smaller than it really is (e.g., saying that an insult was actually just a joke).}
}}

\vspace{-2.2mm}
\item[\ding{224}] {\footnotesize 
{\fontfamily{phv}\fontsize{8}{9}\selectfont
{\textbf{Repetition:} Repeating the same message over and over again so that the audience will eventually accept it.}
}}

\vspace{-2.2mm}
\item[\ding{224}] {\footnotesize 
{\fontfamily{phv}\fontsize{8}{9}\selectfont
{\textbf{Name Calling Labelling:} Labeling the object of the propaganda campaign as something that the target audience fears, hates, finds undesirable, or loves or praises. }
}}

\vspace{-2.2mm}
\item[\ding{224}] {\footnotesize 
{\fontfamily{phv}\fontsize{8}{9}\selectfont
{\textbf{Bandwagon:} Attempting to persuade the target audience to join in and take the course of action because “everyone else is taking the same action.”}
}}

\vspace{-2.2mm}
\item[\ding{224}] {\footnotesize 
{\fontfamily{phv}\fontsize{8}{9}\selectfont
{\textbf{Loaded Language:} Using specific words and phrases with strong emotional implications (either positive or negative) to influence an audience. }
}}

\vspace{-2.2mm}
\item[\ding{224}] {\footnotesize 
{\fontfamily{phv}\fontsize{8}{9}\selectfont
{\textbf{Casual Oversimplification:} Assuming a single cause or reason when there are actually multiple causes for an issue.}
}}

\vspace{-2.2mm}
\item[\ding{224}] {\footnotesize 
{\fontfamily{phv}\fontsize{8}{9}\selectfont
{\textbf{Red herring:} Introducing irrelevant material to the issue being discussed so that everyone’s attention is diverted away from the points made.}
}}


\vspace{-2.2mm}
\item[\ding{224}] {\footnotesize 
{\fontfamily{phv}\fontsize{8}{9}\selectfont
{\textbf{Appeal to authority:} Stating that a claim is true simply because a valid authority or expert on the issue said it was true.}
}}

\vspace{-2.2mm}
\item[\ding{224}] {\footnotesize 
{\fontfamily{phv}\fontsize{8}{9}\selectfont
{\textbf{Thought terminating cliches:} Words or phrases that discourage critical thought and meaningful discussion about a given topic.}
}}

\vspace{-2.2mm}
\item[\ding{224}] {\footnotesize 
{\fontfamily{phv}\fontsize{8}{9}\selectfont
{\textbf{Whataboutism:} A technique that attempts to discredit an opponent’s position by charging them with hypocrisy without directly disproving their argument.}
}}




\vspace{-6mm}
\end{spacing}
\end{itemize}
%\end{tcolorbox}



%%%%%%%%%%

\end{tcolorbox}
%\begin{tcolorbox}[nobeforeafter, title=box 2]
\begin{tcolorbox}[ nobeforeafter, title=box 2, colback=teal!5!white,colframe=teal!75!black,title=\footnotesize{\textbf{\textsc{Propaganda through Deception}}}]
\begin{itemize}
[leftmargin=1mm]
\setlength\itemsep{0em}
\begin{spacing}{0.90}
\vspace{-2.2mm}
\item[\ding{224}] {\footnotesize \fontfamily{phv}\fontsize{8}{9}\selectfont{\textbf{Flag Waving:} Flag waving maps to speculation in layer 1, black lies in layer 2, gaining advantage in layer 3, and religious aspects in layer 4.
}}
% \end{spacing}}
\vspace{-2.2mm}
\item[\ding{224}] {\footnotesize 
{\fontfamily{phv}\fontsize{8}{9}\selectfont{\textbf{Slogans:} This technique is mostly mapped with speculation in layer1, white lie in layer 2, political in layer 3 and gaining advantage in layer 4.
}
}}

\vspace{-2.2mm}
\item[\ding{224}] {\footnotesize 
{\fontfamily{phv}\fontsize{8}{9}\selectfont
{\textbf{Appeal to fear - prejudices:} This technqiue primarily corresponds to speculation in layer 1, black lie in layer 2, political in layer 3 and gaining advantage in layer 4.}
}}

\vspace{-2.2mm}
\item[\ding{224}] {\footnotesize 
{\fontfamily{phv}\fontsize{8}{9}\selectfont
{\textbf{Exaggeration-Minimization}: In the Layers of Omission, Exaggeration or Minimization is mostly mapped to speculation in layer 1, black lie in layer 2, political in layer 3 and gaining advantage in layer 4.}
}}

\vspace{-2.2mm}
\item[\ding{224}] {\footnotesize 
{\fontfamily{phv}\fontsize{8}{9}\selectfont
{\textbf{Repetition:} Repetition is mostly mapped to Speculation, Black lie, intention of gaining advantage and in political influence.}
}}

\vspace{-2.2mm}
\item[\ding{224}] {\footnotesize 
{\fontfamily{phv}\fontsize{8}{9}\selectfont
{\textbf{Name Calling Labelling:} Name Calling or Labelling is largely mapped to speculation in layer 1, black lie in layer 2, gaining advantage in layer 3 and political in layer 4.}
}}

\vspace{-2.2mm}
\item[\ding{224}] {\footnotesize 
{\fontfamily{phv}\fontsize{8}{9}\selectfont
{\textbf{Bandwagon:} Bandwagon is mostly mapped to speculation in layer 1. It is mapped with both white and gray lie in layer 2. It is mapped with protecting oneself in layer 3 and education in layer 4. }
}}

\vspace{-2.2mm}
\item[\ding{224}] {\footnotesize 
{\fontfamily{phv}\fontsize{8}{9}\selectfont
{\textbf{Loaded Language:} Loaded Language is mapped mostly with speculation in layer 1, black lie in layer 2, gaining advantage in layer 3 and political in layer 4. }
}}

\vspace{-2.2mm}
\item[\ding{224}] {\footnotesize 
{\fontfamily{phv}\fontsize{8}{9}\selectfont
{\textbf{Casual Oversimplification:} Causal Oversimplification is mapped mostly with speculation in layer 1, with black lie and in some cases with red lie in layer 2, gaining advantage in layer 3 and political in layer 4. }
}}

\vspace{-2.2mm}
\item[\ding{224}] {\footnotesize 
{\fontfamily{phv}\fontsize{8}{9}\selectfont
{\textbf{Red herring:} In layer 1, Red Herring corresponds to both speculation and opinion. Layer 2 primarily associates it with black lies, occasionally with white lies. In layer 3, it largely aligns with gaining advantage, while layer 4 relates to political aspects.}
}}


\vspace{-2.2mm}
\item[\ding{224}] {\footnotesize 
{\fontfamily{phv}\fontsize{8}{9}\selectfont
{\textbf{Appeal to authority:} This technique largely maps with opinion and with speculation too. In the 2nd layer, it maps with black and gray lies and with gaining advantage in 3rd layer and political in 4th layer. }
}}

\vspace{-2.2mm}
\item[\ding{224}] {\footnotesize 
{\fontfamily{phv}\fontsize{8}{9}\selectfont
{\textbf{Thought terminating cliches:} This technique mostly maps with speculation in layer 1, gray and black lie in layer 2, gaining advantage in layer 3 and political in layer 4.}
}}

\vspace{-2.2mm}
\item[\ding{224}] {\footnotesize 
{\fontfamily{phv}\fontsize{8}{9}\selectfont
{\textbf{Whataboutism:} Whataboutism mostly maps with speculation in layer 1, black lie in layer 2, gaining advantage in layer 3 and political in layer 4. }
}}


\vspace{-6mm}
\end{spacing}
\end{itemize}
%\end{tcolorbox}


%\lipsum[2]
\end{tcolorbox}
\end{tcbraster}
%\lipsum[2]
%\end{document}


%\documentclass{article}
%\usepackage[most]{tcolorbox}
%\usepackage{lipsum}
%\begin{document}
%\lipsum[2]
\begin{tcbraster}[raster columns=2,raster equal height]
%%%%%%%%%
\begin{tcolorbox}[nobeforeafter, title=box 1,colback=brown!5!white,colframe=brown!75!black,title=\footnotesize{\textbf{\textsc{Propaganda Technique Definition}}}]
%\lipsum[2]
\begin{itemize}
[leftmargin=1mm]
\setlength\itemsep{0em}
\begin{spacing}{0.90}
%\vspace{-2.2mm}

%\vspace{-2.2mm}
\item[\ding{224}] {\footnotesize 
{\fontfamily{phv}\fontsize{8}{9}\selectfont
{\textbf{Straw Men:}Substituting an opponent’s proposition with a similar one, which is then refuted in place of the original proposition.}
}}

\item[\ding{224}] {\footnotesize 
{\fontfamily{phv}\fontsize{8}{9}\selectfont
{\textbf{Doubt:}Questioning the credibility of someone or something.}
}}

\vspace{-2.2mm}
\item[\ding{224}] {\footnotesize 
{\fontfamily{phv}\fontsize{8}{9}\selectfont
{\textbf{Obfuscation:} Using words that are deliberately not clear, so that the audience may have their own interpretations.}
}}

\vspace{-2.2mm}
\item[\ding{224}] {\footnotesize 
{\fontfamily{phv}\fontsize{8}{9}\selectfont
{\textbf{Reductio ad Hitlerum:} An attempt to invalidate someone else’s argument on the basis that the same idea was promoted.}
}}

\vspace{-2.2mm}
\item[\ding{224}] {\footnotesize 
{\fontfamily{phv}\fontsize{8}{9}\selectfont
{\textbf{Black and White Fallacy:}Using words that depict the fallacy of leaping from the undesirability of one proposition to the truth of an extreme opposite.
}
}}

\vspace{-6mm}
\end{spacing}
\end{itemize}
%\end{tcolorbox}



%%%%%%%%%%

\end{tcolorbox}
%\begin{tcolorbox}[nobeforeafter, title=box 2]
\begin{tcolorbox}[ nobeforeafter, title=box 2, colback=teal!5!white,colframe=teal!75!black,title=\footnotesize{\textbf{\textsc{Propaganda through Deception}}}]
\begin{itemize}
[leftmargin=1mm]
\setlength\itemsep{0em}
\begin{spacing}{0.90}
\vspace{-2.2mm}
\item[\ding{224}] {\footnotesize 
{\fontfamily{phv}\fontsize{8}{9}\selectfont
{\textbf{Straw Men:} Straw Men maps mostly with speculation but sometimes with opinion too. It maps with both black and white lie of layer 2 in most cases and gaining advantage in layer 3 and political in layer 4. }
}}


\vspace{-2.2mm}
\item[\ding{224}] {\footnotesize 
{\fontfamily{phv}\fontsize{8}{9}\selectfont
{\textbf{Doubt:} Doubt maps mostly with speculation in layer 1, black lie in layer 2, gaining advantage in layer 3 and political in layer 4.}
}}

\vspace{-2.2mm}
\item[\ding{224}] {\footnotesize 
{\fontfamily{phv}\fontsize{8}{9}\selectfont
{\textbf{Obfuscation:} This technique maps mostly with speculation in layer 1, red lie in layer 2, gaining advantage in layer 3 and political in layer 4. }
}}

\vspace{-2.2mm}
\item[\ding{224}] {\footnotesize 
{\fontfamily{phv}\fontsize{8}{9}\selectfont
{\textbf{Reductio ad Hitlerum:} This technique maps with speculation and distrotion in layer1, black lies and occasional white lies in layer 2. Layer 3 and layer 4 are primarily associated with gaining advantage and politics, respectively. }
}}

\vspace{-2.2mm}
\item[\ding{224}] {\footnotesize 
{\fontfamily{phv}\fontsize{8}{9}\selectfont
{\textbf{Black and White Fallacy:} This technique predominantly involves speculation and opinion, with elements of black lies in the second layer. In the third layer, it is mostly aligned with gaining advantage but occasionally tied to protecting oneself and political and educational in layer 4.
}
}}

\vspace{-6mm}
\end{spacing}
\end{itemize}
%\end{tcolorbox}


%\lipsum[2]
\end{tcolorbox}
\end{tcbraster}
%\lipsum[2]
%\end{document}








%%%%%%%%%%%%%%%%%%%




\begin{figure*}[htbp]
    \begin{subfigure}[b]{0.50\textwidth}
    \centering
        \includegraphics[width=1.1\textwidth]{circos/Circos_AppealtoAuthority.png}
        \caption{Layers of Deception-Appeal to Authority}
    \end{subfigure}
    \begin{subfigure}[b]{0.50\textwidth}
    \centering
        \includegraphics[width=0.8\textwidth]{circos/Circos_StrawMen.png}
        \caption{Layers of Deception-Straw Men}
    \end{subfigure}    
    \begin{subfigure}[b]{0.50\textwidth}
    \centering
        \includegraphics[width=0.9\textwidth]{circos/Circos_bandwagan.png}
        \caption{Layers of Deception-Bandwagon}
    \end{subfigure}
    \begin{subfigure}[b]{0.50\textwidth}
    \centering
        \includegraphics[width=0.95\textwidth]{circos/Circos_Doubt.png}
        \caption{Layers of Deception-Doubt}
    % \label{fig: vitc}
    \end{subfigure}
    \begin{subfigure}[b]{0.50\textwidth}
    \centering
        \includegraphics[width=1\textwidth]{circos/Circos_BlackWhiteFallacy.png}
        \caption{Layers of Deception-Black and white Fallacy}
    % \label{fig: vitc}
    \end{subfigure}
    \begin{subfigure}[b]{0.50\textwidth}
    \centering
        \includegraphics[width=0.99\textwidth]{circos/Circos_flagwaving.png}
        \caption{Layers of Deception-Flag Waving}
    % \label{fig: vitc}
    \end{subfigure}
\end{figure*}    
\begin{figure*}[htbp]    
    \begin{subfigure}[b]{0.50\textwidth}
    \centering
        \includegraphics[width=0.9\textwidth]{circos/Circos_NameCalling.png}
        \caption{Layers of Deception-Name Calling}
    % \label{fig: vitc}
    \end{subfigure}
    \begin{subfigure}[b]{0.50\textwidth}
    \centering
        \includegraphics[width=0.9\textwidth]{circos/Circos_Obfuscation.png}
        \caption{Layers of Deception-Obfuscation}
    % \label{fig: vitc}
    \end{subfigure}
    \begin{subfigure}[b]{0.50\textwidth}
    \centering
        \includegraphics[width=0.9\textwidth]{circos/Circos_RedHerring.png}
        \caption{Layers of Deception-Red Herring}
    % \label{fig: vitc}
    \end{subfigure}
    \begin{subfigure}[b]{0.50\textwidth}
    \centering
        \includegraphics[width=\textwidth]{circos/Circos_Reducto.png}
        \caption{Layers of Deception-Reductio ad Hitlerum}
    \end{subfigure}
    \begin{subfigure}[b]{0.50\textwidth}
    \centering
        \includegraphics[width=\textwidth]{circos/Circos_Slogans2x.png}
        \caption{Layers of Deception-Slogans}
    \end{subfigure}    
    \begin{subfigure}[b]{0.50\textwidth}
    \centering
        \includegraphics[width=1.3\textwidth]{circos/Circos_ThoughtTerminatingCliches2x.png}
        \caption{Layers of Deception-Thought terminating cliches}
    \end{subfigure}
\end{figure*}

\begin{figure*}[htbp]
    \begin{subfigure}[b]{0.50\textwidth}
    \centering
        \includegraphics[width=\textwidth]{circos/Circos_Whataboutism.png}
        \caption{Layers of Deception-Whataboutism}
    % \label{fig: vitc}
    \end{subfigure}
    \begin{subfigure}[b]{0.50\textwidth}
    \centering
        \includegraphics[width=\textwidth]{circos/Circos_Repeatition.png}
        \caption{Layers of Deception-Repetition}
    % \label{fig: vitc}
    \end{subfigure}
    \begin{subfigure}[b]{0.50\textwidth}
    \centering
        \includegraphics[width=1\textwidth]{circos/Circos_OverSimplification.png}
        \caption{Layers of Deception-Casual Oversimplification}
    % \label{fig: vitc}
    \end{subfigure}
    \begin{subfigure}[b]{0.50\textwidth}
    \centering
        \includegraphics[width=0.7\textwidth]{circos/Circos_LoadedLanguage.png}
        \caption{Layers of Deception-Loaded Language}
    % \label{fig: vitc}
    \end{subfigure}
    \begin{subfigure}[b]{0.50\textwidth}
    \centering
        \includegraphics[width=0.95\textwidth]{circos/Circos_Exaggeration.png}
        \caption{Layers of Deception-Exaggeration}
    % \label{fig: vitc}
    \end{subfigure}
    \begin{subfigure}[b]{0.50\textwidth}
    \centering
        \includegraphics[width=1.1\textwidth]{circos/Circos_Appealtofear.png}
        \caption{Layers of Deception-Appeal to fear}
    % \label{fig: vitc}
    \end{subfigure}
\end{figure*}




\end{document}