
\begin{figure}[!tbh]
\centering
\includegraphics[width=\columnwidth, height=3.75cm]{Image/SEPSIS_final.pdf}
\vspace{-5mm}
\caption{The figure represents the categorization of the SEPSIS corpus across all layers. The 1$^{st}$ layer represents \textit{type of omission} and its respective categories, 2$^{nd}$ layer represents colors of lies, 3$^{rd}$ layer represents the intent of lies, and 4$^{th}$ layer represents the topic of lies. }
% An example for all classes across layers is written for ease of understanding. }
%\acil{Increase figure size after removing surrounding whitespace within the image.}
\label{fig:sepsis}
\vspace{-5mm}
\end{figure}
%\vspace{-2.5mm}

\section{Introducing SEPSIS: A novel corpus on \ul{lies of omission}}
\label{sec:introduction}
\vspace{-1.5mm}
We are delighted to introduce the \textbf{SEPSIS} corpus (\textbf{S}p\textbf{E}culation o\textbf{P}inion bia\textbf{S} d\textbf{I}\textbf{S}tortion), explicitly curated for \emph{lies of omission}. This novel resource will significantly enhance the study and analysis of deceptive communication by focusing on the deliberate exclusion of information. Figure \ref{fig:sepsis} offers a concise visual depiction that effectively summarizes the categorization we present in the SEPSIS. 
%We take great pride in presenting the SEPSIS corpus (SpEculation oPinion biaS dIStortion), a pioneering corpus specifically dedicated to lies of omission. This novel resource will enrich the study and analysis of deceptive communication by honing in on the deliberate omission of information.
%Figure ~\ref{fig:sepsis} provides a visual representation that succinctly summarizes the innovative categorization we have developed for the creation of SEPSIS. This novel categorization scheme plays a crucial role in organizing and classifying the corpus, facilitating a comprehensive understanding of the lies of omission. 
%In the subsequent paragraphs, we outline scientific questions that serve as motivations for our research. We elucidate how the pursuit of these questions has influenced the development of our annotation schema, which plays a fundamental role in the design of our research framework.
In the subsequent paragraphs, we present a collection of scientific inquiries along with their corresponding answers, which serve as the driving force behind our research. Furthermore, we delve into the influence of these questions on the development of our annotation schema, which lays the groundwork for our research framework.

\vspace{2mm}
\noindent
\textls[0]{\textbf{\ul{Is there a specific dialogue act that individuals employ for lies of omission?}} Within the classical switchboard corpus \cite{godfrey1992switchboard}, 
 %\ac{citations needed} 
there exist 42 well-defined dialogue acts. Following extensive deliberation and analysis, we have reached the conclusion that individuals often utilize dialogue acts such as \emph{speculation, opinion, bias, and distortions} when engaging in deceptive behavior. 
%Through rigorous discussion and debate, we derived a conclusion that dialogue acts commonly employed by individuals when they engage in deceptive behavior are \emph{speculation, opinion, bias, and distortions}. 
%These acts serve as figurative dialogue acts utilized by individuals to conceal their lies through encryption \cite{elaad2003effects} while they want to drop some information.
These dialogue acts function as figurative communication techniques employed by individuals to mask their deceit through encryption \cite{elaad2003effects}, particularly when they desire to disclose certain information selectively.}

\vspace{-3mm}
\begin{itemize}
[leftmargin=1mm]
\setlength\itemsep{0em}
\item[\ding{93}] \fontsize{9}{10}\selectfont{\textbf{Speculation} entails conjecturing without ample evidence.
%entails hypothesizing or .For example, utilizing verbs like "reports," "could," and "maybe" in a text can render it speculative.}
\vspace{-1mm}
\item[\ding{93}] \textbf{Opinion} is a subjective viewpoint formed without relying on factually accepted knowledge.
%Opinion} is a subjective viewpoint or assessment formed without relying on factual or widely accepted knowledge. Instances of statements appearing as opinions typically differ from person to person.
\vspace{-1mm}
\item[\ding{93}] \textbf{Bias} refers to unfair prejudice towards a particular individual or group. 
%refers to a predisposition or unfair prejudice towards a particular individual or group. Instances of bias can be classified based on factors such as race, ethnicity, religious beliefs, secularism, and so forth.
\vspace{-1mm}
\item[\ding{93}] \textbf{Distortion} is the act of twisting something away from its genuine, inherent, or initial condition.
%is the act of twisting or modifying something away from its genuine, inherent, or initial condition. It manifests in statements that appear superficial or illogical.

\vspace{-1mm}
\item[\ding{93}] Lastly, we define \textbf{sounds factual} as a statement that seems factual but may not be true.}
%It includes instances of information given by reports of claims by various organizations.}
\end{itemize}


\vspace{-7mm}
%%%%%%%%%%%%%
\begin{tcolorbox}[enhanced,attach boxed title to top right={yshift=-3mm,yshifttext=-1mm},
  colback=blue!5!white,colframe=blue!75!black,colbacktitle=red!80!black,
  title=$1^{st}$ level: type of omission,fonttitle=\bfseries,
  boxed title style={size=small,colframe=red!50!black},left=0pt, right=0pt]

  \begin{spacing}{0.8}
  \textbf{\ul{\footnotesize Speculation:}} {\fontfamily{lmss}\fontsize{8}{10}\selectfont{Biden warned the US does not have 'resources to win WW3' as tensions rise in the Middle East.}}
  \end{spacing}

  \vspace{-2.5mm}
  \DrawLine
  %\vspace{-2mm}
  
  \begin{spacing}{0.8}
  \textbf{\ul{\footnotesize Opinion:} }{\fontfamily{lmss}\fontsize{8}{10}\selectfont
 Poll: Trump receives low overall approval rating but praise for strong economy.}\end{spacing}
  \vspace{-2.5mm}
  \DrawLine
  \begin{spacing}{0.8}
  \textbf{\ul{\footnotesize Bias:} }{
  \fontfamily{lmss}\fontsize{8}{10}\selectfont
  Russia lauds India for following own interests on energy issue.}
  \end{spacing} 

  \vspace{-2.5mm}
  \DrawLine
  %\vspace{-2mm}
  
  \begin{spacing}{0.8}
  \textbf{\ul{\footnotesize Distortion:} }{\fontfamily{lmss}\fontsize{8}{10}\selectfont
  Republic TV: Jama Masjid in dark due to non-payment of electricity bills over four crores.}
  \end{spacing}

  \vspace{-2.5mm}
  \DrawLine
  %\vspace{-2mm}
  
  \begin{spacing}{0.8}
  \textbf{\ul{\footnotesize Sounds\, Factual:} }{\fontfamily{lmss}\fontsize{8}{10}\selectfont
  A US government study confirms most face recognition systems are racist.}\end{spacing} 
  \vspace{-0.75mm}
\end{tcolorbox}


%%%%%%%%%%%%%%%%%%%%%%


\vspace{-1mm}
\noindent
\textbf{\ul{What has been omitted?}} In the study of lies of omission, it is crucial to determine what information has been deliberately omitted. To address this, we draw inspiration from journalism, where the use of the 5W framework is common. The 5W framework consists of the questions \textit{who, what, when, where, and why} which are considered fundamental in information gathering and problem-solving. These questions are frequently utilized in journalism and police investigations \cite{10.2307/1023893, stofer2009sports, silverman, su2019study, smarts_2017,article_2023}. As an example:

\vspace{-1.5mm}  
\begin{tcolorbox}[left=0pt, right=0pt]
\small
\vspace{-1mm}
\{Hillary Clinton\}\textsubscript{\textcolor{blue}{who\textsubscript{1}}} announces \{Global Climate Resilience Fund\}\textsubscript{\textcolor{blue}{what}} for \{women\}\textsubscript{\textcolor{blue}{who\textsubscript{2}}}  
to\{tackle climate change\}\textsubscript{\textcolor{blue}{why}}
\vspace{-2.5mm}
\end{tcolorbox} 

\noindent
\textbf{\ul{What is the vulnerability of the uttered lie?}} In the realm of deception research, it is of utmost importance to comprehend and quantify the susceptibility of lies. One approach involves categorizing lies into different colors, namely \emph{black, red, white, and gray} \cite{ratliff2011behavioral,depaulo2004many}. Each color represents a distinct type of lie with varying levels of vulnerability, as detailed below:

%%%%%%%%%%%%%%%%%%%%%%%%%%%%%%%%%%%%




\vspace{-3mm}
\begin{itemize}
[leftmargin=1mm]
\setlength\itemsep{0em}
\item[\ding{93}] \fontsize{9}{10}\selectfont{A \textbf{black lie} is about simple and callous selfishness. Typically uttered when there is no benefit to others, its sole intention is to extricate oneself from trouble.}
\vspace{-1.5mm}
\item[\ding{93}] A \textbf{white lie} prioritizes others' welfare over personal interests, reflecting an altruistic nature.

% is characterized by its altruistic nature as it prioritizes the welfare of others, even if it entails sacrificing one's own interests to some extent. is one which is altruistic as it seeks to help others first, even at some cost to oneself.
\vspace{-1mm}
\item[\ding{93}] \textbf{Gray lies} exhibits dual behavior, partially benefiting others and partially benefiting oneself depending on the viewpoint. 
%They are partially told to help others and partially to help oneself, depending on the perspective of the sentence.
\vspace{-1mm}
\item[\ding{93}] \textbf{Red lies} are spoken from a hatred and revenge perspective against individuals or groups. 
%These are driven by the motive to harm others even at the expense of harming oneself.
\end{itemize}

\vspace{-7mm}
\begin{tcolorbox}[enhanced,attach boxed title to top right={yshift=-3mm,yshifttext=-1mm},
  colback=blue!5!white,colframe=blue!75!black,colbacktitle=red!80!black,
  title=$2^{nd}$ level: colors of lie,fonttitle=\bfseries,
  boxed title style={size=small,colframe=red!50!black},left=0pt, right=0pt ]

  \begin{spacing}{0.8}\textbf{\ul{\footnotesize Red:} }{\fontfamily{lmss}\fontsize{8}{10}\selectfont
  Donald Trump's congratulatory post for North Korea's WHO membership sparks outrage and controversy.}
  \end{spacing} 

  \vspace{-2.5mm}
  \DrawLine
  %\vspace{-2mm}

  \begin{spacing}{0.8}
  \textbf{\ul{\footnotesize Black:} }{\fontfamily{lmss}\fontsize{8}{10}\selectfont
FTX collapse: Former CEO Sam Bankman-Fried urges court to toss charges.}
  \end{spacing}

  \vspace{-2.5mm}
  \DrawLine
  %\vspace{-2mm}

  \begin{spacing}{0.8}
  \textbf{\ul{\footnotesize White:} }{\fontfamily{lmss}\fontsize{8}{10}\selectfont 
% After Hope Hicks told the House Intelligence Committee that she has told white lies on behalf of President Trump, news organizations around the world made this admission a top story.
% another example
% That day, Michael Jordan’s mother changed Nike’s history forever: ‘Even if you don’t like it, you’re going to listen to them’!
An apple a day slashes frailty risk by 20 percent, but Study points otherwise.
}
  \end{spacing} 

  \vspace{-2.5mm}
  \DrawLine
  %\vspace{-2mm}
  
  \begin{spacing}{0.8}
  \textbf{\ul{\footnotesize Gray:} }{\fontfamily{lmss}\fontsize{8}{10}\selectfont
Hillary Clinton Announces Global Climate Resilience Fund For Women To Tackle Climate Change.}
  \end{spacing}
  
  \vspace{-1mm}
\end{tcolorbox}



\begin{comment}
\begin{table}[ht]
\resizebox{\columnwidth}{!}{
\begin{tabular}{|llll|}
\hline
\multicolumn{4}{|c|}{\cellcolor[HTML]{FFCCC9}\textbf{SEPSIS at a glance}}  \\ \hline
\multicolumn{1}{|l|}{\textbf{\begin{tabular}[c]{@{}l@{}}Data\\ Sources\end{tabular}}} & \multicolumn{1}{l|}{\textbf{No. of datapoints}} & \multicolumn{1}{l|}{\textbf{Number of Paraphrases}} & \textbf{\begin{tabular}[c]{@{}l@{}}Number of augmented data points \\ through prompt engineering\end{tabular}} \\ \hline
\multicolumn{1}{|l|}{\textbf{Times of India}}                                         & \multicolumn{1}{l|}{2333}                       & \multicolumn{1}{l|}{43344}                          & 23567                                                                                                          \\ \hline
\multicolumn{1}{|l|}{\textbf{ISOT dataset}}                                           & \multicolumn{1}{l|}{2245}                       & \multicolumn{1}{l|}{34569}                          & 58902                                                                                                          \\ \hline
\multicolumn{1}{|l|}{\textbf{Total}}                                                  & \multicolumn{1}{l|}{}                           & \multicolumn{1}{l|}{}                               &                                                                                                                \\ \hline
\end{tabular}
}
\caption{SEPSIS corpus statistics}
\label{tab:SEPSIS_corpus}
\end{table}

\end{comment}


\vspace{-2mm}
\noindent
\textbf{\ul{What is the intent of the lie?}}
%\textcolor{red}{we need a text here and associated intent of lies examples}
%It is important to study the intent of lies to understand what certain deceptive texts intend to achieve. 
Studying the intent of lies helps to comprehend deceptive language's objectives. We have thus categorized lies into different intents as shown below. 

%So we followed an extensive approach to categorize lies into different intents such as \emph{gaining advantage, protecting themselves}, etc.

\vspace{-3mm}
\begin{tcolorbox}[enhanced,attach boxed title to top right={yshift=-3mm,yshifttext=-1mm},
  colback=blue!5!white,colframe=blue!75!black,colbacktitle=red!80!black,
  title=$3^{rd}$ level: intent of lie,fonttitle=\bfseries,
  boxed title style={size=small,colframe=red!50!black},left=0pt, right=0pt ]
  
\begin{spacing}{0.8}
\textbf{\ul{\footnotesize Gaining\,Advantage:} }{\fontfamily{lmss}\fontsize{8}{10}\selectfont
  Elizabeth Holmes ordered dinners for Theranos staff but made sure they weren't delivered until after 8 p.m. so they worked late: book.
  }
\end{spacing}  

  \vspace{-2.5mm}
  \DrawLine
  %\vspace{-2mm}

  \begin{spacing}{0.8}
  \textbf{\ul{\footnotesize Protecting\,Themselves:} }{\fontfamily{lmss}\fontsize{8}{10}\selectfont
ChatGPT creator Sam Altman testifies to US Congress on AI risks.}
  \end{spacing} 

  \vspace{-2.5mm}
  \DrawLine
  %\vspace{-2mm}

  \begin{spacing}{0.8}
  \textbf{\ul{\footnotesize Avoiding\,Embarrassment:} }{\fontfamily{lmss}\fontsize{8}{10}\selectfont 
Trump’s Suggestion That Disinfectants Could Be Used to Treat Coronavirus Prompts Aggressive Pushback, was Sarcastic?}
  \end{spacing} 

  \vspace{-2.5mm}
  \DrawLine
  %\vspace{-2mm}

  \begin{spacing}{0.8}
  \textbf{\ul{\footnotesize Gaining\,Esteem:} }{\fontfamily{lmss}\fontsize{8}{10}\selectfont
Sasan Goodarzi, the CEO of software giant Intuit, which has avoided mass layoffs, says tech firms axed jobs because they misread the pandemic.}
  \end{spacing} 

  \vspace{-2.5mm}
  \DrawLine
  %\vspace{-2mm}

  \begin{spacing}{0.8}
 \textbf{\ul{\footnotesize Protecting\,Others:} }{\fontfamily{lmss}\fontsize{8}{10}\selectfont 
Nobel Laureate Malala Urges U.S. To Bolster Support For Afghan Girls, Women!}
  \end{spacing}
  
  \vspace{-2.5mm}
  \DrawLine
  %\vspace{-2mm}

\begin{spacing}{0.8}
 \textbf{\ul{\footnotesize Defaming\,Esteem:} }{\fontfamily{lmss}\fontsize{8}{10}\selectfont 
Taiwan war would be ‘devastating,’ warns US Defense Secretary Lloyd Austin as he criticizes China at Shangri-La security summit.}
  \end{spacing}
  \vspace{-1mm}
\end{tcolorbox}

\vspace{-3mm}

\vspace{-2mm}
\begin{itemize}
[leftmargin=1mm]
\setlength\itemsep{0em}
\item[\ding{93}] \fontsize{9}{10}\selectfont{\textbf{Intent of Gaining Advantage} can be used as an act of intentionally providing false information or misleading others to gain an unfair advantage over them. 
%It involves distorting facts or creating a false narrative to secure personal or professional benefits at the expense of others.
\vspace{-1mm}
\item[\ding{93}] \textbf{Intent of Protecting Themselves} can be used as a means of self-preservation or self-defense when an individual feels threatened or vulnerable.
%It results in withholding or distorting information to shield oneself from potential harm, consequences, or legal implications. 
\vspace{-1 mm}
\item[\ding{93}] \textbf{Intent of Avoiding Embarrassment} can be employed to evade situations that may lead to embarrassment, humiliation, or social discomfort.
%These can involve providing false information to avoid potential judgment with the intention to maintain a positive self-image.
\vspace{-1mm}

\item[\ding{93}] \textbf{Intent of Gaining Esteem} can be utilized to enhance one's reputation, social status, or personal image.

%It requires fabricating hyperbole achievements, qualities, or experiences to gain admiration, respect, or validation from others. The aim is to elevate oneself in the eyes of others and be perceived in a more favorable light.

\item[\ding{93}] \textbf{Intent of Protecting Others} can be used as a means of preservation for others when a group or community feels threatened or vulnerable.
%It results in withholding or distorting information to shield it from potential harm, consequences, or legal implications and benefit others. 
\item[\ding{93}] \textbf{Intent of Defaming Esteem} intends to damage reputation by spreading false information or rumors.}
%aims at defaming someone's esteem and involves spreading false information or rumors to damage an individual's reputation or public perception. It is motivated by a desire to discredit or tarnish someone's image.


\end{itemize}


\vspace{-2mm}


\noindent
\textbf{\ul{What is the topic of lie?}}
To study deception further and to understand its topical influence, this research categorizes different topics of lies such as political, educational, etc.

\vspace{-3mm}
\begin{itemize}
[leftmargin=1mm]
\setlength\itemsep{0em}

\item[\ding{93}] \fontsize{9}{10}\selectfont{\textbf{Political} deception occurs by the deliberate use of statements by political entities to manipulate public opinion.

%gain power, or advance their agenda. It involves spreading misleading promises or distorting facts for political gain.}
\vspace{-1mm}
\item[\ding{93}] \fontsize{9}{10}\selectfont{\textbf{Educational} deception occurs by the deliberate use of statements by academic entities to manipulate opinion, directed especially towards the younger population.
%It involves spreading misleading policies for economic gain.
\vspace{-1mm}
\item[\ding{93}] \textbf{Racial} deception occurs when individuals intentionally misrepresent their racial identity or engage in deception driven by racial motives.}
%This can encompass falsely asserting a different racial heritage, manipulating others through racial stereotypes, or partaking in impersonation based on race.
\vspace{-1mm}
\item[\ding{93}]\textbf{Religious} deception involves the act of deceiving others by misrepresenting one's religious beliefs.}

%, for engaging in fraudulent practices and personal gain.
\vspace{-1mm}
\item[\ding{93}] \textbf{Ethnic} deception refers to the act of intentionally manipulating one's ethnic identity by targeting specific ethnic groups.
%and exploiting stereotypes, and cultural appropriation for personal or political reasons.
\end{itemize}


\vspace{-6mm}
\begin{tcolorbox}[enhanced,attach boxed title to top right={yshift=-3mm,yshifttext=-1mm},
  colback=blue!5!white,colframe=blue!75!black,colbacktitle=red!80!black,
  title=$4^{th}$ level: topic of lie,fonttitle=\bfseries,
  boxed title style={size=small,colframe=red!50!black},left=0pt, right=0pt ]
  
\begin{spacing}{0.8}
  \textbf{\ul{\footnotesize Political:} }{\fontfamily{lmss}\fontsize{8}{10}\selectfont
 No elections safe from AI, deep fake photos, videos of politicians to become common, warns former Google boss.}\end{spacing}

  \vspace{-2.5mm}
  \DrawLine

  \begin{spacing}{0.8}
  \textbf{\ul{\footnotesize Educational:} }{\fontfamily{lmss}\fontsize{8}{10}\selectfont 
Hundreds gather at Florida school board meeting over Disney movie controversy: 'Your policies are not protecting us from anything.}\end{spacing} 

  \vspace{-2.5mm}
  \DrawLine

  \begin{spacing}{0.8}
  \textbf{\ul{\footnotesize Religious:} }{\fontfamily{lmss}\fontsize{8}{10}\selectfont
Pope: Christianity, Islam share common commitment to good life.}\end{spacing}

  \vspace{-2.5mm}
  \DrawLine

  \begin{spacing}{0.8}
  \textbf{\ul{\footnotesize Racial:} }{\fontfamily{lmss}\fontsize{8}{10}\selectfont
  Why shouldn’t a mixed-race actress play Egyptian queen Cleopatra?}\end{spacing}

  \vspace{-2.5mm}
  \DrawLine

  \begin{spacing}{0.8}
 \textbf{\ul{\footnotesize Ethnicity:} }{\fontfamily{lmss}\fontsize{8}{10}\selectfont 
Egyptians complain over Netflix depiction of Cleopatra as black.}\end{spacing}
  
  \vspace{-1mm}
  
\end{tcolorbox}
\vspace{-2mm}

%%%%%%%%%%%%%%%%%%%%%


\section{SEPSIS: Data Sources, Annotation, and Agreement}
\vspace{-1mm}  
At the outset, we engaged in the crowd annotation of 5,000 sentences through Amazon Mechanical Turk (AMT), employing four layers of deception. Subsequently, we applied data augmentation techniques as detailed in section ~\ref{sec:data_augmentation}, culminating in a total of 8,76,784 data points.

\subsection{Data Sources}
\vspace{-0.5mm}  

\begin{table*}[ht]
\centering
\resizebox{\textwidth}{!}{%
\begin{tabular}{lccccccccccccccccccccccccc}
\toprule
                                        & \multicolumn{5}{c}{\textbf{Lies of omission}}                                                           & \multicolumn{4}{c}{\centering \textbf{Color of lies}}             & \multicolumn{6}{c}{\textbf{Intent of Lies}}        
                                        % & \multicolumn{6}{c}{\textbf{Traits of lies}}
                                        
                                        \\
                                        % \midrule
                                        \toprule
                                        
                                        
                                        & \multicolumn{1}{p{1.2cm}}{\centering \textbf{Specula-}
                                        \textbf{tion}
                                        }                                        

                                        % \multicolumn{1}{p{2.2cm}}{\centering \textbf{Avoiding} \textbf{Embarrassment}}
                                        
                                        
                                        & \textbf{Bias}
                                        & 
                                        \multicolumn{1}{p{1cm}}{\centering \textbf{Distor-}
                                        \textbf{tion}
                                        }   
                                        
                                        & \multicolumn{1}{p{1.35cm}}{\centering \textbf{Opinion}} 
                                        & \multicolumn{1}{p{1.35cm}}{\centering \textbf{Sounds}
                                        \\ \textbf{Factual}}
                                        
                                        & 
                                        
                                        \textbf{Black} & \textbf{White} & \textbf{Grey} & \textbf{Red} 
                                        
                                        & 
                                        
                                        \multicolumn{1}{p{1.5cm}}{\centering \textbf{Gaining} \\ \textbf{Advantage}} & 
                                        \multicolumn{1}{p{1.6cm}}{\centering \textbf{Protecting} \\ \textbf{Themselves}} & 
                                        \multicolumn{1}{p{2.2cm}}{\centering \textbf{Avoiding} \textbf{Embarrassment}} & \multicolumn{1}{p{1.3cm}}{\centering \textbf{Gaining} \\ \textbf{Esteem}} & \multicolumn{1}{p{1.4cm}}{\centering \textbf{Protecting} \\ \textbf{Others}} 
                                        & \multicolumn{1}{p{1.4cm}}{\centering \textbf{Defaming} \\ \textbf{Esteem}}

                                        
                                        \\  \hline 
% \multicolumn{1}{l}{\textbf{Tweet}}     & 0.598                   & 0.552            & 0.539                  & 0.54               & 0.679                      & 0.771             & 0.727             & 0.691            & 0.766           & 0.710        & 0.672        & 0.612        & 0.684        & 0.557        & 0.529        

% % & 0.840        & 0.712        & 0.772        & 0.636        & 0.592
% \\

% \multicolumn{1}{l}{\textbf{Fake News}} &  0.639 &
% 0.581 & 0.603 & 0.513 & 0.647 &  0.798 & 0.765 & 0.715 
% & 0.812   & 0.679 & 0.73 & 0.658 & 0.597 & 0.617 & 0.616


\multicolumn{1}{l}{\textbf{Tweet}} &
\cellcolor{yellow!50}0.678 &
\cellcolor{yellow!50}0.632 &
\cellcolor{yellow!50}0.619 &
\cellcolor{yellow!50}0.62 &
% {\color{colorname}Text to be colored}
\cellcolor{blue!50}{\textcolor{white}{0.759}} &
\cellcolor{green!50}0.831 &
\cellcolor{green!50}0.807 &
\cellcolor{blue!50}{\textcolor{white}{0.771}} &
\cellcolor{green!50}0.846 &
\cellcolor{blue!50}\textcolor{white}{0.790} &
\cellcolor{blue!50}{\textcolor{white}{0.752}} &
\cellcolor{yellow!50}0.692 &
\cellcolor{blue!50}{\textcolor{white}{0.744}} &
\cellcolor{yellow!50}0.637 &
\cellcolor{yellow!50}0.609 \\

\multicolumn{1}{l}{\textbf{Fake News}} &
\cellcolor{blue!50}{\textcolor{white}{0.719}} &
\cellcolor{yellow!50}0.661 &
\cellcolor{yellow!50}{0.683} &
\cellcolor{yellow!50}0.603 &
\cellcolor{blue!50}{\textcolor{white}{0.727}} &
\cellcolor{green!50}0.878 &
\cellcolor{green!50}0.845 &
\cellcolor{green!50}0.811 &
\cellcolor{green!50}0.892 &
\cellcolor{blue!50}{\textcolor{white}{0.759}} &
\cellcolor{green!50}0.81 &
\cellcolor{blue!50}{\textcolor{white}{0.738}} &
\cellcolor{yellow!50}0.677 &
\cellcolor{blue!50}{\textcolor{white}{0.709}} &
\cellcolor{yellow!50}{0.681} \\

% & 0.823        & 0.754        & 0.715        & 0.678  
% & 0.649


\bottomrule   
\end{tabular}%
} 
\vspace{-1mm}
\caption{Kappa score representation for layer 1: \textit{type of omission} %(\emph{speculation, bias, Distortion, opinion, sounds factual}), 
layer 2: \textit{colors of lies}, 
%(\emph{Black, White, grey, and red}), 
and layer 3: \textit{Intent of lies}.
%(\emph{Gaining Advantage, Protecting themselves, Avoiding Embarrassment, Gaining esteem, Protecting others and Defaming esteem}).
Kappa score for the layer 4 topic of lies
%(\emph{religious, educational, racial, ethnicity, and political} is present) 
can be found in Appendix \ref{sec: Data Annotation}.}
\vspace{-3.5mm}
\label{tab: Kappa score}
\end{table*}

In terms of data sources, we have identified two distinct categories of interest. The first category focuses on the presence of omissions in factual data, specifically news data. The second category examines the involvement of omissions in fake news data. To address these categories, we have selected data sources from two prominent outlets: (a) Times of India \cite{timesofindia} Twitter handle, the renowned news agency in India, and (b) Information Security and Object Technology (ISOT) fake news dataset \cite{ISOTFakeNewsDataset}. More information on these sources can be found in the appendix \ref{sec:data sources}. A detailed analysis of the SEPSIS corpus and the results can be found in Appendix \ref{sec: data analysis}.
%Data acquisition is done primarily from two sources Times of India \cite{} which is a popular Indian media house and a popular fake news data set. \cite{}. It is a compilation of several thousand fake news and truthful articles, obtained from different legitimate news sites and sites flagged as unreliable by \emph{"Politifact.com"}. From this dataset a sample of 2605 datapoint is taken from the one that is flagged as fake news.


%\url{https://onlineacademiccommunity.uvic.ca/isot/wp-content/uploads/sites/7295/2023/02/ISOT_Fake_News_Dataset_ReadMe.pdf}

\vspace{-1mm}
\subsection{Data Annotation}
\vspace{-1mm}  
% For annotation we chose to utilize the Amazon Mechanical Turk (AMT) service. This platform offers a quick and cost-effective solution for annotating large-scale datasets, although it is important to acknowledge its susceptibility to errors.
We have chosen to leverage the Amazon Mechanical Turk (AMT) service for annotation purposes, which provides a rapid and cost-effective solution for annotating large-scale datasets, albeit acknowledging its susceptibility to errors. To ensure the annotation quality, we implemented rigorous checks and measures throughout the entire annotation process. The dataset was annotated at the sentence level using a multi-class annotation approach, allowing each individual feature to be assigned multiple categories during the annotation process. For instance, a statement could be tagged as both speculative and sounding factual, recognizing the possibility for it to either be a verifiable fact or contain speculative elements that satisfy both possibilities. A comprehensive account of the overall annotation process is provided in Appendix \ref{sec: data cleaning}. Notably, during the initial layer of annotation, if a particular text appeared to be factual, we refrained from annotating the specific type, intent, and influence of the lie since it was treated as a fact.

\subsection{Inter Annotator Agreement and Quality}
\label{sec:iaa_score}
To ensure quality control in the AMT annotations, we performed in-house annotation on 1000 data points. This in-house dataset was utilized to assess the quality of annotations provided by individual annotators. Based on this assessment, we selected the appropriate set of annotators on the AMT platform. For the annotation task on AMT, we offered a compensation of \$1 for each sentence annotation across all the four layers.

We obtained four annotations per sentence and subsequently consolidated the data using an improved voting technique, as suggested in \cite{hovy-etal-2013-learning}, which has been empirically shown to outperform majority voting. To assess the level of agreement in the annotated corpus, we also calculated the Cohen Kappa score \cite{cohen1960coefficient}. Since there are multiple categories for a given sentence, we report class-wise agreement scores. The overall agreement score is presented in Table \ref{tab: Kappa score}. An overview of data points is presented in Table \ref{tab:SEPSIS_corpus}. To understand how features across these four layers are dependent on each other we present six heatmaps in Appendix \ref{sec: data analysis}.



%The comprehensive examination of various categories of lies reveals a complex and nuanced understanding, posing challenges in the annotation process. Consequently, the Lies of Omission category received relatively lower scores. Conversely, the intent of lie and color of lies categories proved comparatively more accessible for comprehension, interpretation, and detection, leading to higher scores in this layer. Taking into account the inherently subjective nature of the matter at hand, we deem our inter-annotation scores to be fairly commendable.

%\textcolor{red}{we need some text to describe what to take away from agreement scores.} \textcolor{green}{Is this okay?}

\begin{table}[!ht]
\centering
\vspace{-1mm}
\resizebox{0.95\columnwidth}{!}{%
\begin{tabular}{cccccl}
% \toprule
% \multicolumn{6}{c}{\cellcolor[HTML]{FFFFFF}\textbf{SEPSIS Corpus Statistics}}                                                                                                                                                                                                                                                                                                                                                                                                                                                                \\ 
\toprule
\textbf{Data Source} & \multicolumn{1}{c}{\textbf{\begin{tabular}[c]{@{}c@{}}Sentences\end{tabular}}} & \multicolumn{1}{c}{\textbf{\begin{tabular}[c]{@{}c@{}}+  Paraphrasing\end{tabular}}} & \multicolumn{1}{c}{\textbf{\begin{tabular}[c]{@{}c@{}}+  Mask Infilling\end{tabular}} } \\ \toprule
\textbf{Tweets}       &       $2495$                                                                                       &                    $12475$                                                                                                          &      $389105$                                                                                                                            &                                                                                                                                                   \\
\textbf{Fake News}   &      $2605$                                                                                        &     $13025$                                                                                                                         &    $487829$                                                                                                                              &                                                                                                                                                   \\ \midrule
\textbf{Total}       &       $5100$                                                                                       &      $25500$                                                                                                                        &     $876784$                                                                                                                             &                                                                                                                                                   \\ \bottomrule
\end{tabular}
}
%\caption{SEPSIS corpus statistics represents the count of data points across tweets and fake news. The number of sentences represents the original data point that was scrapped and annotated directly. The \textit{+paraphrasing} and \textit{+mask infilling} columns denote the count of unique augmented data points using the respective techniques.

\caption {Number of original sentences and augmented sentences using \textit{paraphrasing} and \textit{mask infilling}.}
\label{tab:SEPSIS_corpus}
\vspace{-5mm}
\end{table}

\begin{comment}
We addressed several issues in our dataset to improve the reliability of the annotations and reduce redundancies. Many data points were assigned the correct reason behind their choice but were assigned the wrong category. For example, certain statements were placed under the 'distortion' category, but the provided reason suggested their categorization to be 'speculation.' Additionally, we re-annotated several incomplete entries where annotators had entered the statement category but missed providing further information on the color of lies or intent of lies.
\end{comment}



%\subsection{SEPSIS Corpus - statistics}
