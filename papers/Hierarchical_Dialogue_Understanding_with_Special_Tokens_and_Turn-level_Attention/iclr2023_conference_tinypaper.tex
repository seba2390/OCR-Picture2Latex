
\documentclass{article} % For LaTeX2e
\usepackage{iclr2023_conference_tinypaper,times}

% Optional math commands from https://github.com/goodfeli/dlbook_notation.
%%%%% NEW MATH DEFINITIONS %%%%%

\usepackage{amsmath,amsfonts,bm}

% Mark sections of captions for referring to divisions of figures
\newcommand{\figleft}{{\em (Left)}}
\newcommand{\figcenter}{{\em (Center)}}
\newcommand{\figright}{{\em (Right)}}
\newcommand{\figtop}{{\em (Top)}}
\newcommand{\figbottom}{{\em (Bottom)}}
\newcommand{\captiona}{{\em (a)}}
\newcommand{\captionb}{{\em (b)}}
\newcommand{\captionc}{{\em (c)}}
\newcommand{\captiond}{{\em (d)}}

% Highlight a newly defined term
\newcommand{\newterm}[1]{{\bf #1}}


% Figure reference, lower-case.
\def\figref#1{figure~\ref{#1}}
% Figure reference, capital. For start of sentence
\def\Figref#1{Figure~\ref{#1}}
\def\twofigref#1#2{figures \ref{#1} and \ref{#2}}
\def\quadfigref#1#2#3#4{figures \ref{#1}, \ref{#2}, \ref{#3} and \ref{#4}}
% Section reference, lower-case.
\def\secref#1{section~\ref{#1}}
% Section reference, capital.
\def\Secref#1{Section~\ref{#1}}
% Reference to two sections.
\def\twosecrefs#1#2{sections \ref{#1} and \ref{#2}}
% Reference to three sections.
\def\secrefs#1#2#3{sections \ref{#1}, \ref{#2} and \ref{#3}}
% Reference to an equation, lower-case.
% \def\eqref#1{equation~\ref{#1}}
 \def\eqref#1{(\ref{#1})}
% Reference to an equation, upper case
\def\Eqref#1{Equation~\ref{#1}}
% A raw reference to an equation---avoid using if possible
\def\plaineqref#1{\ref{#1}}
% Reference to a chapter, lower-case.
\def\chapref#1{chapter~\ref{#1}}
% Reference to an equation, upper case.
\def\Chapref#1{Chapter~\ref{#1}}
% Reference to a range of chapters
\def\rangechapref#1#2{chapters\ref{#1}--\ref{#2}}
% Reference to an algorithm, lower-case.
\def\algref#1{algorithm~\ref{#1}}
% Reference to an algorithm, upper case.
\def\Algref#1{Algorithm~\ref{#1}}
\def\twoalgref#1#2{algorithms \ref{#1} and \ref{#2}}
\def\Twoalgref#1#2{Algorithms \ref{#1} and \ref{#2}}
% Reference to a part, lower case
\def\partref#1{part~\ref{#1}}
% Reference to a part, upper case
\def\Partref#1{Part~\ref{#1}}
\def\twopartref#1#2{parts \ref{#1} and \ref{#2}}

\def\ceil#1{\lceil #1 \rceil}
\def\floor#1{\lfloor #1 \rfloor}
\def\1{\bm{1}}
\newcommand{\train}{\mathcal{D}}
\newcommand{\valid}{\mathcal{D_{\mathrm{valid}}}}
\newcommand{\test}{\mathcal{D_{\mathrm{test}}}}

\def\eps{{\epsilon}}


% Random variables
\def\reta{{\textnormal{$\eta$}}}
\def\ra{{\textnormal{a}}}
\def\rb{{\textnormal{b}}}
\def\rc{{\textnormal{c}}}
\def\rd{{\textnormal{d}}}
\def\re{{\textnormal{e}}}
\def\rf{{\textnormal{f}}}
\def\rg{{\textnormal{g}}}
\def\rh{{\textnormal{h}}}
\def\ri{{\textnormal{i}}}
\def\rj{{\textnormal{j}}}
\def\rk{{\textnormal{k}}}
\def\rl{{\textnormal{l}}}
% rm is already a command, just don't name any random variables m
\def\rn{{\textnormal{n}}}
\def\ro{{\textnormal{o}}}
\def\rp{{\textnormal{p}}}
\def\rq{{\textnormal{q}}}
\def\rr{{\textnormal{r}}}
\def\rs{{\textnormal{s}}}
\def\rt{{\textnormal{t}}}
\def\ru{{\textnormal{u}}}
\def\rv{{\textnormal{v}}}
\def\rw{{\textnormal{w}}}
\def\rx{{\textnormal{x}}}
\def\ry{{\textnormal{y}}}
\def\rz{{\textnormal{z}}}

% Random vectors
\def\rvepsilon{{\mathbf{\epsilon}}}
\def\rvtheta{{\mathbf{\theta}}}
\def\rva{{\mathbf{a}}}
\def\rvb{{\mathbf{b}}}
\def\rvc{{\mathbf{c}}}
\def\rvd{{\mathbf{d}}}
\def\rve{{\mathbf{e}}}
\def\rvf{{\mathbf{f}}}
\def\rvg{{\mathbf{g}}}
\def\rvh{{\mathbf{h}}}
\def\rvu{{\mathbf{i}}}
\def\rvj{{\mathbf{j}}}
\def\rvk{{\mathbf{k}}}
\def\rvl{{\mathbf{l}}}
\def\rvm{{\mathbf{m}}}
\def\rvn{{\mathbf{n}}}
\def\rvo{{\mathbf{o}}}
\def\rvp{{\mathbf{p}}}
\def\rvq{{\mathbf{q}}}
\def\rvr{{\mathbf{r}}}
\def\rvs{{\mathbf{s}}}
\def\rvt{{\mathbf{t}}}
\def\rvu{{\mathbf{u}}}
\def\rvv{{\mathbf{v}}}
\def\rvw{{\mathbf{w}}}
\def\rvx{{\mathbf{x}}}
\def\rvy{{\mathbf{y}}}
\def\rvz{{\mathbf{z}}}

% Elements of random vectors
\def\erva{{\textnormal{a}}}
\def\ervb{{\textnormal{b}}}
\def\ervc{{\textnormal{c}}}
\def\ervd{{\textnormal{d}}}
\def\erve{{\textnormal{e}}}
\def\ervf{{\textnormal{f}}}
\def\ervg{{\textnormal{g}}}
\def\ervh{{\textnormal{h}}}
\def\ervi{{\textnormal{i}}}
\def\ervj{{\textnormal{j}}}
\def\ervk{{\textnormal{k}}}
\def\ervl{{\textnormal{l}}}
\def\ervm{{\textnormal{m}}}
\def\ervn{{\textnormal{n}}}
\def\ervo{{\textnormal{o}}}
\def\ervp{{\textnormal{p}}}
\def\ervq{{\textnormal{q}}}
\def\ervr{{\textnormal{r}}}
\def\ervs{{\textnormal{s}}}
\def\ervt{{\textnormal{t}}}
\def\ervu{{\textnormal{u}}}
\def\ervv{{\textnormal{v}}}
\def\ervw{{\textnormal{w}}}
\def\ervx{{\textnormal{x}}}
\def\ervy{{\textnormal{y}}}
\def\ervz{{\textnormal{z}}}

% Random matrices
\def\rmA{{\mathbf{A}}}
\def\rmB{{\mathbf{B}}}
\def\rmC{{\mathbf{C}}}
\def\rmD{{\mathbf{D}}}
\def\rmE{{\mathbf{E}}}
\def\rmF{{\mathbf{F}}}
\def\rmG{{\mathbf{G}}}
\def\rmH{{\mathbf{H}}}
\def\rmI{{\mathbf{I}}}
\def\rmJ{{\mathbf{J}}}
\def\rmK{{\mathbf{K}}}
\def\rmL{{\mathbf{L}}}
\def\rmM{{\mathbf{M}}}
\def\rmN{{\mathbf{N}}}
\def\rmO{{\mathbf{O}}}
\def\rmP{{\mathbf{P}}}
\def\rmQ{{\mathbf{Q}}}
\def\rmR{{\mathbf{R}}}
\def\rmS{{\mathbf{S}}}
\def\rmT{{\mathbf{T}}}
\def\rmU{{\mathbf{U}}}
\def\rmV{{\mathbf{V}}}
\def\rmW{{\mathbf{W}}}
\def\rmX{{\mathbf{X}}}
\def\rmY{{\mathbf{Y}}}
\def\rmZ{{\mathbf{Z}}}

% Elements of random matrices
\def\ermA{{\textnormal{A}}}
\def\ermB{{\textnormal{B}}}
\def\ermC{{\textnormal{C}}}
\def\ermD{{\textnormal{D}}}
\def\ermE{{\textnormal{E}}}
\def\ermF{{\textnormal{F}}}
\def\ermG{{\textnormal{G}}}
\def\ermH{{\textnormal{H}}}
\def\ermI{{\textnormal{I}}}
\def\ermJ{{\textnormal{J}}}
\def\ermK{{\textnormal{K}}}
\def\ermL{{\textnormal{L}}}
\def\ermM{{\textnormal{M}}}
\def\ermN{{\textnormal{N}}}
\def\ermO{{\textnormal{O}}}
\def\ermP{{\textnormal{P}}}
\def\ermQ{{\textnormal{Q}}}
\def\ermR{{\textnormal{R}}}
\def\ermS{{\textnormal{S}}}
\def\ermT{{\textnormal{T}}}
\def\ermU{{\textnormal{U}}}
\def\ermV{{\textnormal{V}}}
\def\ermW{{\textnormal{W}}}
\def\ermX{{\textnormal{X}}}
\def\ermY{{\textnormal{Y}}}
\def\ermZ{{\textnormal{Z}}}

% Vectors
\def\vzero{{\bm{0}}}
\def\vone{{\bm{1}}}
\def\vmu{{\bm{\mu}}}
\def\vtheta{{\bm{\theta}}}
\def\va{{\bm{a}}}
\def\vb{{\bm{b}}}
\def\vc{{\bm{c}}}
\def\vd{{\bm{d}}}
\def\ve{{\bm{e}}}
\def\vf{{\bm{f}}}
\def\vg{{\bm{g}}}
\def\vh{{\bm{h}}}
\def\vi{{\bm{i}}}
\def\vj{{\bm{j}}}
\def\vk{{\bm{k}}}
\def\vl{{\bm{l}}}
\def\vm{{\bm{m}}}
\def\vn{{\bm{n}}}
\def\vo{{\bm{o}}}
\def\vp{{\bm{p}}}
\def\vq{{\bm{q}}}
\def\vr{{\bm{r}}}
\def\vs{{\bm{s}}}
\def\vt{{\bm{t}}}
\def\vu{{\bm{u}}}
\def\vv{{\bm{v}}}
\def\vw{{\bm{w}}}
\def\vx{{\bm{x}}}
\def\vy{{\bm{y}}}
\def\vz{{\bm{z}}}

% Elements of vectors
\def\evalpha{{\alpha}}
\def\evbeta{{\beta}}
\def\evepsilon{{\epsilon}}
\def\evlambda{{\lambda}}
\def\evomega{{\omega}}
\def\evmu{{\mu}}
\def\evpsi{{\psi}}
\def\evsigma{{\sigma}}
\def\evtheta{{\theta}}
\def\eva{{a}}
\def\evb{{b}}
\def\evc{{c}}
\def\evd{{d}}
\def\eve{{e}}
\def\evf{{f}}
\def\evg{{g}}
\def\evh{{h}}
\def\evi{{i}}
\def\evj{{j}}
\def\evk{{k}}
\def\evl{{l}}
\def\evm{{m}}
\def\evn{{n}}
\def\evo{{o}}
\def\evp{{p}}
\def\evq{{q}}
\def\evr{{r}}
\def\evs{{s}}
\def\evt{{t}}
\def\evu{{u}}
\def\evv{{v}}
\def\evw{{w}}
\def\evx{{x}}
\def\evy{{y}}
\def\evz{{z}}

% Matrix
\def\mA{{\bm{A}}}
\def\mB{{\bm{B}}}
\def\mC{{\bm{C}}}
\def\mD{{\bm{D}}}
\def\mE{{\bm{E}}}
\def\mF{{\bm{F}}}
\def\mG{{\bm{G}}}
\def\mH{{\bm{H}}}
\def\mI{{\bm{I}}}
\def\mJ{{\bm{J}}}
\def\mK{{\bm{K}}}
\def\mL{{\bm{L}}}
\def\mM{{\bm{M}}}
\def\mN{{\bm{N}}}
\def\mO{{\bm{O}}}
\def\mP{{\bm{P}}}
\def\mQ{{\bm{Q}}}
\def\mR{{\bm{R}}}
\def\mS{{\bm{S}}}
\def\mT{{\bm{T}}}
\def\mU{{\bm{U}}}
\def\mV{{\bm{V}}}
\def\mW{{\bm{W}}}
\def\mX{{\bm{X}}}
\def\mY{{\bm{Y}}}
\def\mZ{{\bm{Z}}}
\def\mBeta{{\bm{\beta}}}
\def\mPhi{{\bm{\Phi}}}
\def\mLambda{{\bm{\Lambda}}}
\def\mSigma{{\bm{\Sigma}}}

% Tensor
\DeclareMathAlphabet{\mathsfit}{\encodingdefault}{\sfdefault}{m}{sl}
\SetMathAlphabet{\mathsfit}{bold}{\encodingdefault}{\sfdefault}{bx}{n}
\newcommand{\tens}[1]{\bm{\mathsfit{#1}}}
\def\tA{{\tens{A}}}
\def\tB{{\tens{B}}}
\def\tC{{\tens{C}}}
\def\tD{{\tens{D}}}
\def\tE{{\tens{E}}}
\def\tF{{\tens{F}}}
\def\tG{{\tens{G}}}
\def\tH{{\tens{H}}}
\def\tI{{\tens{I}}}
\def\tJ{{\tens{J}}}
\def\tK{{\tens{K}}}
\def\tL{{\tens{L}}}
\def\tM{{\tens{M}}}
\def\tN{{\tens{N}}}
\def\tO{{\tens{O}}}
\def\tP{{\tens{P}}}
\def\tQ{{\tens{Q}}}
\def\tR{{\tens{R}}}
\def\tS{{\tens{S}}}
\def\tT{{\tens{T}}}
\def\tU{{\tens{U}}}
\def\tV{{\tens{V}}}
\def\tW{{\tens{W}}}
\def\tX{{\tens{X}}}
\def\tY{{\tens{Y}}}
\def\tZ{{\tens{Z}}}


% Graph
\def\gA{{\mathcal{A}}}
\def\gB{{\mathcal{B}}}
\def\gC{{\mathcal{C}}}
\def\gD{{\mathcal{D}}}
\def\gE{{\mathcal{E}}}
\def\gF{{\mathcal{F}}}
\def\gG{{\mathcal{G}}}
\def\gH{{\mathcal{H}}}
\def\gI{{\mathcal{I}}}
\def\gJ{{\mathcal{J}}}
\def\gK{{\mathcal{K}}}
\def\gL{{\mathcal{L}}}
\def\gM{{\mathcal{M}}}
\def\gN{{\mathcal{N}}}
\def\gO{{\mathcal{O}}}
\def\gP{{\mathcal{P}}}
\def\gQ{{\mathcal{Q}}}
\def\gR{{\mathcal{R}}}
\def\gS{{\mathcal{S}}}
\def\gT{{\mathcal{T}}}
\def\gU{{\mathcal{U}}}
\def\gV{{\mathcal{V}}}
\def\gW{{\mathcal{W}}}
\def\gX{{\mathcal{X}}}
\def\gY{{\mathcal{Y}}}
\def\gZ{{\mathcal{Z}}}

% Sets
\def\sA{{\mathbb{A}}}
\def\sB{{\mathbb{B}}}
\def\sC{{\mathbb{C}}}
\def\sD{{\mathbb{D}}}
% Don't use a set called E, because this would be the same as our symbol
% for expectation.
\def\sF{{\mathbb{F}}}
\def\sG{{\mathbb{G}}}
\def\sH{{\mathbb{H}}}
\def\sI{{\mathbb{I}}}
\def\sJ{{\mathbb{J}}}
\def\sK{{\mathbb{K}}}
\def\sL{{\mathbb{L}}}
\def\sM{{\mathbb{M}}}
\def\sN{{\mathbb{N}}}
\def\sO{{\mathbb{O}}}
\def\sP{{\mathbb{P}}}
\def\sQ{{\mathbb{Q}}}
\def\sR{{\mathbb{R}}}
\def\sS{{\mathbb{S}}}
\def\sT{{\mathbb{T}}}
\def\sU{{\mathbb{U}}}
\def\sV{{\mathbb{V}}}
\def\sW{{\mathbb{W}}}
\def\sX{{\mathbb{X}}}
\def\sY{{\mathbb{Y}}}
\def\sZ{{\mathbb{Z}}}

% Entries of a matrix
\def\emLambda{{\Lambda}}
\def\emA{{A}}
\def\emB{{B}}
\def\emC{{C}}
\def\emD{{D}}
\def\emE{{E}}
\def\emF{{F}}
\def\emG{{G}}
\def\emH{{H}}
\def\emI{{I}}
\def\emJ{{J}}
\def\emK{{K}}
\def\emL{{L}}
\def\emM{{M}}
\def\emN{{N}}
\def\emO{{O}}
\def\emP{{P}}
\def\emQ{{Q}}
\def\emR{{R}}
\def\emS{{S}}
\def\emT{{T}}
\def\emU{{U}}
\def\emV{{V}}
\def\emW{{W}}
\def\emX{{X}}
\def\emY{{Y}}
\def\emZ{{Z}}
\def\emSigma{{\Sigma}}

% entries of a tensor
% Same font as tensor, without \bm wrapper
\newcommand{\etens}[1]{\mathsfit{#1}}
\def\etLambda{{\etens{\Lambda}}}
\def\etA{{\etens{A}}}
\def\etB{{\etens{B}}}
\def\etC{{\etens{C}}}
\def\etD{{\etens{D}}}
\def\etE{{\etens{E}}}
\def\etF{{\etens{F}}}
\def\etG{{\etens{G}}}
\def\etH{{\etens{H}}}
\def\etI{{\etens{I}}}
\def\etJ{{\etens{J}}}
\def\etK{{\etens{K}}}
\def\etL{{\etens{L}}}
\def\etM{{\etens{M}}}
\def\etN{{\etens{N}}}
\def\etO{{\etens{O}}}
\def\etP{{\etens{P}}}
\def\etQ{{\etens{Q}}}
\def\etR{{\etens{R}}}
\def\etS{{\etens{S}}}
\def\etT{{\etens{T}}}
\def\etU{{\etens{U}}}
\def\etV{{\etens{V}}}
\def\etW{{\etens{W}}}
\def\etX{{\etens{X}}}
\def\etY{{\etens{Y}}}
\def\etZ{{\etens{Z}}}

% The true underlying data generating distribution
\newcommand{\pdata}{p_{\rm{data}}}
% The empirical distribution defined by the training set
\newcommand{\ptrain}{\hat{p}_{\rm{data}}}
\newcommand{\Ptrain}{\hat{P}_{\rm{data}}}
% The model distribution
\newcommand{\pmodel}{p_{\rm{model}}}
\newcommand{\Pmodel}{P_{\rm{model}}}
\newcommand{\ptildemodel}{\tilde{p}_{\rm{model}}}
% Stochastic autoencoder distributions
\newcommand{\pencode}{p_{\rm{encoder}}}
\newcommand{\pdecode}{p_{\rm{decoder}}}
\newcommand{\precons}{p_{\rm{reconstruct}}}

\newcommand{\laplace}{\mathrm{Laplace}} % Laplace distribution

\newcommand{\E}{\mathbb{E}}
\newcommand{\Ls}{\mathcal{L}}
\newcommand{\R}{\mathbb{R}}
\newcommand{\emp}{\tilde{p}}
\newcommand{\lr}{\alpha}
\newcommand{\reg}{\lambda}
\newcommand{\rect}{\mathrm{rectifier}}
\newcommand{\softmax}{\mathrm{softmax}}
\newcommand{\sigmoid}{\sigma}
\newcommand{\softplus}{\zeta}
\newcommand{\KL}{D_{\mathrm{KL}}}
\newcommand{\Var}{\mathrm{Var}}
\newcommand{\standarderror}{\mathrm{SE}}
\newcommand{\Cov}{\mathrm{Cov}}
% Wolfram Mathworld says $L^2$ is for function spaces and $\ell^2$ is for vectors
% But then they seem to use $L^2$ for vectors throughout the site, and so does
% wikipedia.
\newcommand{\normlzero}{L^0}
\newcommand{\normlone}{L^1}
\newcommand{\normltwo}{L^2}
\newcommand{\normlp}{L^p}
\newcommand{\normmax}{L^\infty}

\newcommand{\parents}{Pa} % See usage in notation.tex. Chosen to match Daphne's book.

\DeclareMathOperator*{\argmax}{arg\,max}
\DeclareMathOperator*{\argmin}{arg\,min}

\DeclareMathOperator{\sign}{sign}
\DeclareMathOperator{\Tr}{Tr}
\let\ab\allowbreak

\newcommand{\norm}[2]{\left\| #1 \right\|_{#2}}

\newcommand{\zz}[1]{\textcolor{blue}{ [{\em Zhihui:} #1]}}
\newcommand{\jz}[1]{\textcolor{red}{ [{\em JZ:} #1]}}
% \newcommand{\td}[1]{\textcolor{blue}{ [{\em TD:} #1]}}
\newcommand{\jj}[1]{\textcolor{pink}{ [{\em JJ:} #1]}}

\usepackage{hyperref}
\usepackage{url}
\usepackage{booktabs}
\usepackage{multirow}
\usepackage{graphicx} 
\usepackage{fixltx2e}
\usepackage{wrapfig}
\usepackage{caption}
\newcommand*{\myalign}[2]{\multicolumn{1}{#1}{#2}}
\newcommand{\LX}[1]{\textcolor{teal}{[Xiao: #1]}}
\newcommand{\FZ}[1]{\textcolor{red}{[Fuzhao: #1]}}
\newcommand*\samethanks[1][\value{footnote}]{\footnotemark[#1]}

\title{Hierarchical Dialogue Understanding with Special Tokens and Turn-level Attention}

% Authors must not appear in the submitted version. They should be hidden
% as long as the \iclrfinalcopy macro remains commented out below.
% Non-anonymous submissions will be rejected without review.

\author{Xiao Liu\thanks{This work was done during the author's time as a student at National University of Singapore}, Jian Zhang\thanks{Equal contributions}, Heng Zhang\samethanks, Fuzhao Xue, Yang You \\
Department of Computer Science, National University of Singapore\\
\texttt{\{liuxiao, zhang.jian, zhang\_heng, f.xue\}@u.nus.edu,} \\ \texttt{youy@comp.nus.edu.sg} \\
}
% The \author macro works with any number of authors. There are two commands
% used to separate the names and addresses of multiple authors: \And and \AND.
%
% Using \And between authors leaves it to \LaTeX{} to determine where to break
% the lines. Using \AND forces a linebreak at that point. So, if \LaTeX{}
% puts 3 of 4 authors names on the first line, and the last on the second
% line, try using \AND instead of \And before the third author name.

\newcommand{\fix}{\marginpar{FIX}}
\newcommand{\new}{\marginpar{NEW}}

\iclrfinalcopy % Uncomment for camera-ready version, but NOT for submission.
\begin{document}


\maketitle

\begin{abstract}
Compared with standard text, understanding dialogue is more challenging for machines as the dynamic and unexpected semantic changes in each turn. To model such inconsistent semantics, we propose a \emph{simple but effective} \textbf{Hi}erarchical \textbf{Dialog}ue Understanding model, HiDialog. Specifically, we first insert multiple special tokens into a dialogue and propose the turn-level attention to learn turn embeddings hierarchically. Then, a heterogeneous graph module is leveraged to polish the learned embeddings. We evaluate our model on various dialogue understanding tasks including dialogue relation extraction, dialogue emotion recognition, and dialogue act classification. Results show that our simple approach achieves state-of-the-art performance on all three tasks above. All our source code is publicly available at \url{https://github.com/ShawX825/HiDialog}.
\end{abstract}

%%%%%%%
\begin{figure}[t]
\begin{center}
   \includegraphics[width=1.0\linewidth]{figures/nas_comp_v3}
\end{center}
   \vspace{-4mm}
   \caption{The comparison between NetAdaptV2 and related works. The number above a marker is the corresponding total search time measured on NVIDIA V100 GPUs.}
\label{fig:nas_comparison}
\end{figure}

\section{Introduction}
\label{sec:introduction}

Neural architecture search (NAS) applies machine learning to automatically discover deep neural networks (DNNs) with better performance (e.g., better accuracy-latency trade-offs) by sampling the search space, which is the union of all discoverable DNNs. The search time is one key metric for NAS algorithms, which accounts for three steps: 1) training a \emph{super-network}, whose weights are shared by all the DNNs in the search space and trained by minimizing the loss across them, 2) training and evaluating sampled DNNs (referred to as \emph{samples}), and 3) training the discovered DNN. Another important metric for NAS is whether it supports non-differentiable search metrics such as hardware metrics (e.g., latency and energy). Incorporating hardware metrics into NAS is the key to improving the performance of the discovered DNNs~\cite{eccv2018-netadapt, Tan2018MnasNetPN, cai2018proxylessnas, Chen2020MnasFPNLL, chamnet}.


There is usually a trade-off between the time spent for the three steps and the support of non-differentiable search metrics. For example, early reinforcement-learning-based NAS methods~\cite{zoph2017nasreinforcement, zoph2018nasnet, Tan2018MnasNetPN} suffer from the long time for training and evaluating samples. Using a super-network~\cite{yu2018slimmable, Yu_2019_ICCV, autoslim_arxiv, cai2020once, yu2020bignas, Bender2018UnderstandingAS, enas, tunas, Guo2020SPOS} solves this problem, but super-network training is typically time-consuming and becomes the new time bottleneck. The gradient-based methods~\cite{gordon2018morphnet, liu2018darts, wu2018fbnet, fbnetv2, cai2018proxylessnas, stamoulis2019singlepath, stamoulis2019singlepathautoml, Mei2020AtomNAS, Xu2020PC-DARTS} reduce the time for training a super-network and training and evaluating samples at the cost of sacrificing the support of non-differentiable search metrics. In summary, many existing works either have an unbalanced reduction in the time spent per step (i.e., optimizing some steps at the cost of a significant increase in the time for other steps), which still leads to a long \emph{total} search time, or are unable to support non-differentiable search metrics, which limits the performance of the discovered DNNs.

In this paper, we propose an efficient NAS algorithm, NetAdaptV2, to significantly reduce the \emph{total} search time by introducing three innovations to \emph{better balance} the reduction in the time spent per step while supporting non-differentiable search metrics:

\textbf{Channel-level bypass connections (mainly reduce the time for training and evaluating samples, Sec.~\ref{subsec:channel_level_bypass_connections})}: Early NAS works only search for DNNs with different numbers of filters (referred to as \emph{layer widths}). To improve the performance of the discovered DNN, more recent works search for DNNs with different numbers of layers (referred to as \emph{network depths}) in addition to different layer widths at the cost of training and evaluating more samples because network depths and layer widths are usually considered independently. In NetAdaptV2, we propose \emph{channel-level bypass connections} to merge network depth and layer width into a single search dimension, which requires only searching for layer width and hence reduces the number of samples.

\textbf{Ordered dropout (mainly reduces the time for training a super-network, Sec.~\ref{subsec:ordered_droput})}: We adopt the idea of super-network to reduce the time for training and evaluating samples. In previous works, \emph{each} DNN in the search space requires one forward-backward pass to train. As a result, training multiple DNNs in the search space requires multiple forward-backward passes, which results in a long training time. To address the problem, we propose \emph{ordered dropout} to jointly train multiple DNNs in a \emph{single} forward-backward pass, which decreases the required number of forward-backward passes for a given number of DNNs and hence the time for training a super-network.

\textbf{Multi-layer coordinate descent optimizer (mainly reduces the time for training and evaluating samples and supports non-differentiable search metrics, Sec.~\ref{subsec:optimizer}):} NetAdaptV1~\cite{eccv2018-netadapt} and MobileNetV3~\cite{Howard_2019_ICCV}, which utilizes NetAdaptV1, have demonstrated the effectiveness of the single-layer coordinate descent (SCD) optimizer~\cite{book2020sze} in discovering high-performance DNN architectures. The SCD optimizer supports both differentiable and non-differentiable search metrics and has only a few interpretable hyper-parameters that need to be tuned, such as the per-iteration resource reduction. However, there are two shortcomings of the SCD optimizer. First, it only considers one layer per optimization iteration. Failing to consider the joint effect of multiple layers may lead to a worse decision and hence sub-optimal performance. Second, the per-iteration resource reduction (e.g., latency reduction) is limited by the layer with the smallest resource consumption (e.g., latency). It may take a large number of iterations to search for a very deep network because the per-iteration resource reduction is relatively small compared with the network resource consumption. To address these shortcomings,  we propose the \emph{multi-layer coordinate descent (MCD) optimizer} that considers multiple layers per optimization iteration to improve performance while reducing search time and preserving the support of non-differentiable search metrics.

Fig.~\ref{fig:nas_comparison} (and Table~\ref{tab:nas_result}) compares NetAdaptV2 with related works. NetAdaptV2 can reduce the search time by up to $5.8\times$ and $2.4\times$ on ImageNet~\cite{imagenet_cvpr09} and NYU Depth V2~\cite{nyudepth} respectively and discover DNNs with better performance than state-of-the-art NAS works. Moreover, compared to NAS-discovered MobileNetV3~\cite{Howard_2019_ICCV}, the discovered DNN has $1.8\%$ higher accuracy with the same latency.


\section{Related Work}
Our work draws from, and improves upon, several research threads.

\textbf{Sustainability.}~\citet{srba2016stack} conducted a case study on why StackOverflow, the largest and oldest of the sites in \CQA{StackExchange} network, is failing. They shed some insights into knowledge market failure such as novice and negligent users generating low quality content perpetuating the decline of the market. However, they do not provide a systematic way to understand and prevent failures in these markets.~\citet{wu2016} introduced a framework for understanding the user strategies in a knowledge market---revealing the importance of diverse user strategies for sustainable markets. In this paper, we present an alternative model that provides many interesting insights including knowledge market sustainability.

\textbf{Activity Dynamics.}~\citet{walk2016} modeled user-level activity dynamics in \CQA{StackExchange} using two factors: intrinsic activity decay, and positive peer influence. However, the model proposed there does not reveal the collective platform dynamics, and the eventual success or failure of a platform.~\citet{abufouda2017} developed two models for predicting the interaction decay of community members in online social communities. Similar to~\citet{walk2016}, these models accommodate user-level dynamics, whereas we concentrate on the collective platform dynamics.~\citet{wu2011} proposed a discrete generalized beta distribution (DGBD) model that reveals several insights into the collective platform dynamics, notably the concept of a size-dependent distribution. In this paper, we improve upon the concept of a size-dependent distribution.  

\textbf{Economic Perspective.} \citet{Kumar2010} proposed an economic view of CQA platforms, where they concentrated on the growth of two types of users in a market setting: users who provide questions, and users who provide answers. In this paper, we concentrate on a subsequent problem---the ``relation'' between user growth and content generation in a knowledge market.~\citet{butler2001} proposed a resource-based theory of sustainable social structures. While they treat members as resources, like we do, our model differs in that it concentrates on a market setting, instead of a network setting, and takes the complex content dependency of the platform into consideration. Furthermore, our model provides a systematic way to understand successes and failures of knowledge markets, which none of these models provide.  

\textbf{Scale Study.}~\citet{lin2017} examined Reddit communities to characterize the effect of user growth in voting patterns, linguistic patterns, and community network patterns. Their study reveals that these patterns do not change much after a massive growth in the size of the user community.~\citet{tausczik2017} investigated the effects of crowd size on solution quality in StackExchange communities. Their study uncovers three distinct levels of group size in the crowd that affect solution quality: topic audience size, question audience size, and number of contributors. In this paper, we examine the consequence of scale on knowledge markets from a different perspective by using a set of health metrics.

\textbf{Stability.} Successes and failures of platforms have been studied from the perspective of user retention and stability~\cite{patil2013, garcia2013, kapoor2014, ellis2016}. Notably,~\citet{patil2013} studied the dynamics of group stability based on the average increase or decrease in member growth. Our paper examines stability in a different manner---namely, by considering the relative exchangeability of users as a function of scale.

\textbf{User Growth.} Successes and failures of user communities have also been widely studied from the perspective of user growth~\cite{Kumar2006, Backstrom2006, kairam2012, Ribeiro2014, zang2016}.~\citet{kairam2012} examined diffusion and non-diffusion growth to design models that predict the longevity of social groups.~\citet{Ribeiro2014} proposed a daily active user prediction model which classifies membership based websites as sustainable and unsustainable. While this perspective is important, we argue that studying the successes and failures of communities based on content production can perhaps be more meaningful~\cite{kraut2014, zhu2014, zhu2014niche}.

\textbf{Modeling CQA Websites.} There is a rich body of work that extensively analyzed CQA websites~\cite{Adamic2008, chen2010, anderson2012, wang2013, srba2016}, along with user behavior~\cite{zhang2007, liu2011, pal2012, hanrahan2012, upadhyay2017}, roles~\cite{furtado2013, kumar2016}, and content generation~\cite{baezaYates2015, Yang2015, ferrara2017}. Notably,~\citet{Yang2015} noted the \emph{scalability problem} of CQA---namely, the volume of questions eventually subsumes the capacity of the answerers within the community. Understanding and modeling this phenomenon is one of the goals of this paper.


\section{Proposed Method: SyMFM6D}

We propose a deep multi-directional fusion approach called SyMFM6D that estimates the 6D object poses of all objects in a cluttered scene based on multiple RGB-D images while considering object symmetries. 
In this section, we define the task of multi-view 6D object pose estimation and present our multi-view deep fusion architecture.

\begin{figure*}[tbh]
  \vspace{2mm}
  \centering
  \includegraphics[page=1, trim = 5mm 40mm 5mm 42mm, clip,  width=1.0\linewidth]{figures/SyMFM6D_architecture4_2.pdf}
   \caption{Network architecture of SyMFM6D which fuses $N$ RGB-D input images. Our method converts the $N$ depth images to a single point cloud which is processed by an encoder-decoder point cloud network. The $N$ RGB images are processed by an encoder-decoder CNN. Every hierarchy contains a point-to-pixel fusion module and a pixel-to-point fusion module for deep multi-directional multi-view fusion. We utilize three MLPs with four layers each to regress 3D keypoint offsets, center point offsets, and semantic labels based on the final features. The 6D object poses are computed as in \cite{pvn3d} based on mean shift clustering and least-squares fitting. We train our network by minimizing our proposed symmetry-aware multi-task loss function using precomputed object symmetries. $N_p$ is the number of points in the point cloud. $H$ and $W$ are height and width of the RGB images.}
   \label{fig_architecture}
   \vspace{-2mm}
\end{figure*}


6D object pose estimation describes the task of predicting a rigid transformation $\boldsymbol p = [\boldsymbol R |  \boldsymbol t] \in SE(3)$ which transforms the coordinates of an observed object from the object coordinate system into the camera coordinate system. This transformation is called 6D object pose because it is composed of a 3D rotation $\boldsymbol R \in SO(3)$ and a 3D translation $\boldsymbol t \in \mathbb{R}^3$. 
The designated aim of our approach is to jointly estimate the 6D poses of all objects in a given cluttered scene using multiple RGB-D images which depict the scene from multiple perspectives. We assume the 3D models of the objects and the camera poses to be known as proposed by \cite{mv6d}.



\subsection{Network Overview}

Our symmetry-aware multi-view network consists of three stages which are visualized in \cref{fig_architecture}. 
The first stage receives one or multiple RGB-D images and extracts visual features as well as geometric features which are fused to a joint representation of the scene. 
The second stage performs a detection of predefined 3D keypoints and an instance semantic segmentation.
Based on the keypoints and the information to which object the keypoints belong, we compute the 6D object poses with a least-squares fitting algorithm \cite{leastSquares} in the third stage.



\subsection{Multi-View Feature Extraction}

To efficiently predict keypoints and semantic labels, the first stage of our approach learns a compact representation of the given scene by extracting and merging features from all available RGB-D images in a deep multi-directional fusion manner. For that, we first separate the set of RGB images $\text{RGB}_1, ..., \text{RGB}_N$ from their corresponding depth images $\text{Dpt}_1$, ..., $\text{Dpt}_N$. The $N$ depth images are converted into point clouds, transformed into the coordinate system of the first camera, and merged to a single point cloud using the known camera poses as in \cite{mv6d}. 
Unlike \cite{mv6d}, we employ a point cloud network based on RandLA-Net \cite{hu2020randla} with an encoder-decoder architecture using skip connections.
The point cloud network learns geometric features from the fused point cloud and considers visual features from the multi-directional point-to-pixel fusion modules as described in \cref{sec_multi_view_fusion}.

The $N$ RGB images are independently processed by a CNN with encoder-decoder architecture using the same weights for all $N$ views. The CNN learns visual features while considering geometric features from the multi-directional pixel-to-point fusion modules. We followed \cite{ffb6d} and build the encoder upon a ResNet-34 \cite{resnet} pretrained on ImageNet~\cite{imagenet} and the decoder upon a PSPNet \cite{pspnet}. 

After the encoding and decoding procedures including several multi-view feature fusions, we collect the visual features from each view corresponding to the final geometric feature map and concatenate them. The output is a compact feature tensor containing the relevant information about the entire scene which is used for keypoint detection and instance semantic segmentation as described in \cref{sec_keypoint_detection_and_segmentation}.


\begin{figure*}[tbh]
  \vspace{2mm} 
  \centering  
\begin{subfigure}[b]{0.48\textwidth}
  \includegraphics[page=1, trim = 1mm 6mm 6mm 6mm, clip,  width=1.0\linewidth]{figures/p2r_8.pdf}
   \caption{Point-to-pixel fusion module.~~~~}
   \label{fig_pt2px_fusion}
\end{subfigure}
\begin{subfigure}[b]{0.48\textwidth}
  \centering  
  \includegraphics[page=1, trim = 1mm 6mm 6mm 6mm, clip,  width=1.0\linewidth]{figures/r2p_8.pdf}
   \caption{Pixel-to-point fusion module.~~~~~}
   \label{fig_px2pt_fusion}
   \end{subfigure}
      \caption{Overview of our proposed multi-directional multi-view fusion modules. They combine pixel-wise visual features and point-wise geometric features by exploiting the correspondence between pixels and points using the nearest neighbor algorithm. We compute the resulting features using multiple shared MLPs with a single layer and max-pooling.
      For simplification, we depict an example with $N=2$ views and $K_\text{i}=K_\text{p}=3$ nearest neighbors. The points of ellipsis (...) illustrate the generalization for an arbitrary number of views $N$. Please refer to \cite{ffb6d} for better understanding the basic operations.
      }
   \label{fig_fusion_modules}
   \vspace{-1mm}
\end{figure*}



\subsection{Multi-View Feature Fusion}
\label{sec_multi_view_fusion}
In order to efficiently fuse the visual and geometric features from multiple views, we extend the fusion modules of FFB6D~\cite{ffb6d} from bi-directional fusion to \emph{multi-directional fusion}. We present two types of multi-directional fusion modules which are illustrated in \cref{fig_fusion_modules}.
Both types of fusion modules take the pixel-wise visual feature maps and the point-wise geometric feature maps from each view, combine them, and compute a new feature map.
This process requires a correspondence between pixel-wise and point-wise features which we obtain by computing an XYZ map for each RGB feature map based on the depth data of each pixel using the camera intrinsic matrix as in \cite{ffb6d}. To deal with the changing dimensions at different layers, we use the centers of the convolutional kernels as new coordinates of the feature maps and resize the XYZ map to the same size using nearest interpolation as proposed in \cite{ffb6d}.

The \emph{point-to-pixel} fusion module in \cref{fig_pt2px_fusion} computes a 
fused feature map $\bb F_\text{f}$ based on the image features $\bb F_{\text{i}}(v)$ of all views $v \in \{1, \ldots, N\}$.
We collect the $K_\text{p}$ nearest point features $\bb F_{\text{p}_k}(v)$ with $k \in \{1, \ldots, K_\text{p}\}$ from the point cloud for each pixel-wise feature and each view independently by computing the nearest neighbors according to the Euclidean distance in the XYZ map. Subsequently, we process them by a shared MLP before aggregating them by max-pooling, i.e.,
\begin{align} 
    \widetilde{\bb F}_{\text{p}}(v) = \max_{k \in \{1, \ldots, K_\text{p}\}} 
    \Big( \text{MLP}_\text{p}(\bb F_{\text{p}_k}(v)) \Big).
    \label{eq_p2r}
\end{align}
Finally, we apply a second shared MLP to fuse all features $\bb F_\text{i}$ and 
$\widetilde{\bb F}_{\text{p}}$ as 
$\bb F_{\text{f}} = \text{MLP}_\text{fp}(\widetilde{\bb F}_{\text{p}} \oplus \bb F_\text{i})$ where $\oplus$ denotes the concatenate operation.


The \emph{pixel-to-point} fusion module in \cref{fig_px2pt_fusion} collects the $K_\text{i}$ nearest image features $\bb F_{\text{i}_k}(\textrm{i2v}(i_k))$ with $k\in\{1, ..., K_\text{i}\}$. $\textrm{i2v}(i_k)$ is a mapping that maps the index of an image feature to its corresponding view. This procedure is performed for each point feature vector $\bb F_\text{p}(n)$.
We aggregate the collected image features by max-pooling and apply a shared MLP, i.e.,
\begin{align}
    \widetilde{\bb F}_{\text{i}} = \text{MLP}_\text{i} 
    \left( \max_{k \in \{1, \ldots, K_\text{i}\}} 
    \Big( \bb F_{\text{i}_k}(\textrm{i2v}(i_k)) \Big)  
    \right).
    \label{eq_r2p}
\end{align}
One more shared MLP fuses the resulting image features $\widetilde{\bb F}_{\text{i}}$ with the point features $\bb F_\text{p}$ as 
$\bb F_{\text{f}} = \text{MLP}_\text{fi}(\widetilde{\bb F}_{\text{i}} \oplus \bb F_\text{p})$.




\subsection{Keypoint Detection and Segmentation}
\label{sec_keypoint_detection_and_segmentation}
The second stage of our SyMFM6D network contains modules for 3D keypoint detection and instance semantic segmentation following \cite{mv6d}. However, unlike \cite{mv6d}, we use the SIFT-FPS algorithm \cite{lowe1999sift} as proposed by FFB6D \cite{ffb6d} to define eight target keypoints for each object class. SIFT-FPS yields keypoints with salient features which are easier to detect.
Based on the extracted features, we apply two shared MLPs to estimate the translation offsets from each point of the fused point cloud to each target keypoint and to each object center.
We obtain the actual point proposals by adding the translation offsets to the respective points of the fused point cloud. 
Applying the mean shift clustering algorithm \cite{cheng1995meanshift} results in predictions for the keypoints and the object centers.
We employ one more shared MLP 
for estimating the object class of each point in the fused point cloud as in \cite{pvn3d}.



\subsection{6D Pose Computation via Least-Squares Fitting}

Following \cite{pvn3d}, we use the least-squares fitting algorithm \cite{leastSquares} to compute the 6D poses of all objects based on the estimated keypoints. As the $M$ estimated keypoints $\boldsymbol{\widehat{k}}_1, ..., \boldsymbol{\widehat{k}}_M$ are in the coordinate system of the first camera and the target keypoints $\boldsymbol k_1, ..., \boldsymbol k_M$ are in the object coordinate system, least-squares fitting calculates the rotation matrix $\boldsymbol R$ and the translation vector $\boldsymbol t$ of the 6D pose by minimizing the squared loss
\begin{equation}
    L_\text{Least-squares} = \sum_{i=1}^M \norm{\boldsymbol{\widehat{k}_i} - (\boldsymbol R \boldsymbol k_i + \boldsymbol t)}_2^2.
\end{equation}



\subsection{Symmetry-aware Keypoint Detection}

Most related work, including \cite{pvn3d, ffb6d}, and \cite{mv6d} does not specifically consider object symmetries. 
However, symmetries lead to ambiguities in the predicted keypoints as multiple 6D poses can have the same visual and geometric appearance. 
Therefore, we introduce a novel symmetry-aware training procedure for the 3D keypoint detection including a novel symmetry-aware objective function to make the network predicting either the original set of target keypoints for an object or a rotated version of the set corresponding to one object symmetry. Either way, we can still apply the least-squares fitting which efficiently computes an estimate of the target 6D pose or a rotated version corresponding to an object symmetry. To do so, we precompute the set $\boldsymbol{S}_I$ of all rotational symmetric transformations for the given object instance $I$ with a stochastic gradient
descent algorithm \cite{sgdr}.
Given the known mesh of an object and an initial estimate for the symmetry axis, we transform the object mesh along the symmetry axis estimate and optimize the symmetry axis iteratively by minimizing the ADD-S metric \cite{hinterstoisser2012model}.
Reflectional symmetries which can be represented as rotational symmetries are handled as rotational symmetries. 
Other reflectional symmetries are ignored, since the reflection cannot be expressed as an Euclidean transformation.
To consider continuous rotational symmetries, we discretize them into 16 discrete rotational symmetry transformations.

We extend the keypoints loss function of \cite{pvn3d} to become symmetry-aware such that it predicts the keypoints of the closest symmetric transformation, i.e. 
\begin{equation}
    L_\text{kp}(\mathcal{I}) = \frac{1}{N_I} 
    \min_{\boldsymbol{S} \in \boldsymbol{S}_I} 
    \sum_{i \in \mathcal{I}} \sum_{j=1}^M 
    \norm{\boldsymbol{x}_{ij} - \boldsymbol{S}\boldsymbol{\widehat{x}}_{ij}}_2, 
\label{eq_keypoint_loss}
\end{equation}
where $N_I$ is the number of points in the point cloud for object instance $I$, $M$ is the number of target keypoints per object, and $\mathcal{I}$ is the set of all point indices that belong to object instance $I$.  
The vector $\boldsymbol{\widehat{x}}_{ij}$ is the predicted keypoint offset for the $i$-th point and the $j$-th keypoint while $\boldsymbol{x}_{ij}$ is the corresponding ground truth. 



\subsection{Objective Function}

We train our network by minimizing the multi-task loss function
\begin{equation}
 \label{eq_total_loss}
    L_\text{multi-task} = \lambda_1 L_\text{kp} 
    + \lambda_2 L_\text{semantic}  
    +  \lambda_3 L_\text{cp},
\end{equation}
where $L_\text{kp}$ is our symmetry-aware keypoint loss from \cref{eq_keypoint_loss}.
$L_\text{cp}$ is an L1 loss for the center point prediction, $L_\text{semantic}$ is a Focal loss \cite{focalLoss} for the instance semantic segmentation, and $\lambda_1=2$, $\lambda_2=1$, and $\lambda_3=1$ are the weights for the individual loss functions as in \cite{ffb6d}.

% !TEX root = ../top.tex
% !TEX spellcheck = en-US

\section{Experiments}
\label{sec:experiments}

In this section, we first evaluate our framework on the SPEED dataset, and then introduce the SwissCube dataset, which contains accurate 3D mesh and physically-modeled astronomical objects, and perform thorough ablation studies on it. We further show results on real images of the same satellite. Finally, to demonstrate the generality of our approach we evaluate it on the standard Occluded-LINEMOD dataset depicting small depth variations. 
% \WJ{"CubeSat" leaks author nationalities and even institution (it's an EPFL project). In my community, this would be perceived negatively during peer review. I'd suggest using a temporary name ("NanoSat") with an asterisk/footnote saying that the dataset name is temporarily anonymized as not to reveal authorship.} \YH{I am not sure if it is in CV, while it does not hurt to change to a temporary name.} \MS{I agree, but I suggest "CubeSat", which is the standard term for this type of satellite.}

% \yh{
We train our model starting from a backbone pre-trained on ImageNet~\cite{Deng09}, and, for any 6D pose dataset, feed it 3M unique training samples obtained via standard online data augmentation strategies, such as random shift, scale, and rotation. To evaluate the accuracy, we will report the individual performance under different depth ranges, using the standard ADI-0.1d~\cite{Hu19a,Hu20a} accuracy metrics, which encodes the percentage of samples whose 3D reconstruction error is below 10\% of the object diameter. On the SPEED dataset, however, we use a different metric, as we do not have access to the 3D SPEED model, making the computation of ADI impossible. Instead, we use the metric from the competition, that is, ${\bf e}_{\bf q}+{\bf e}_{\bf t}$, where ${\bf e}_{\bf q}$ is the angular error between the ground-truth quaternion and the predicted one, and ${\bf e}_{\bf t}$ is the normalized translation error. Furthermore, because the depth distribution of SPEED is not uniform, with only few images depicting the satellite at a large distance from the camera, we only report the average error on the whole test set, as in the competition.
% }
The source code and dataset are publicly available at \href{https://github.com/cvlab-epfl/wide-depth-range-pose}{https://github.com/cvlab-epfl/wide-depth-range-pose}.

\subsection{Evaluation on the SPEED Dataset}
Although the SPEED dataset has several drawbacks, discussed in Section~\ref{sec:related}, it remains a valuable benchmark, and we thus begin by evaluating our method on it. As the test annotations are not publicly available, and the competition is not ongoing, we divide the training set into two parts, 10K images for training and the remaining 2K ones for testing.
We evaluate the two top-performing methods from the competition,~\cite{Chen19DLR} (DLR) 
% \MS{Don't they have a better name?}
and~\cite{Hu19a} (SegDriven-Z), on these new splits using the publicly-available code, and find their errors to be of similar magnitude to the ones reported online during the challenge.
Note that our method, as DLR and SegDriven-Z, uses the 3D model to define the keypoints whose image location we predict. We therefore exploit a method of~\cite{Hartley00} to first reconstruct the satellite from the dataset. 

Table~\ref{tab:speed_stoa} compares our results to those of the two top-performing methods on this dataset. Note that DLR combines the results of 6 pose estimation networks, followed by an additional pose refinement strategy to improve accuracy. We therefore also report the results of our method with and without this pose refinement strategy. Note, however, that we still use a single pose estimation network. Furthermore, for our method, we report the results of two separate networks trained at different input resolutions. 
At the resolution of 960$\times$, we outperform the two state-of-the-art methods, while our architecture is much smaller and much faster. To further speed up our approach, we train a network at a third (640$\times$) of the raw image resolution. This network remains on par with DLR but runs 20+ times faster.
% \MS{What resolution does Chen use? If they use 960, I would tend to turn this the other way around: Say that, at the same resolution, we outperform the two state-of-the-art methods, while our architecture is much smaller and much faster. To further speed up our approach, we train a network at a third of the raw image resolution. This network remains on par with Chen but runs 50 times faster.}
% \YH{As they use two networks, we can not compare the resolution directly. In more detail. they train a 768x768 detector first and resize all the detected bounding box to 768x768 again to feed into the next pose network. And they use the second pose network 6 times for results ensemble.}
% The faster version can already on par with the top performer but runs 10+ times faster. Our slower version performs the best and still runs 5+ times faster than the competitors.
% \WJ{Can you explain the rationale for these versions with different resolutions? Just speed? Wasn't clear from the text.}\YH{Yes, mainly for speed, also for GPU memory consumption, especially during training.}
% \WJ{You could more prominently point out speed as one of the benefits in the introduction. Practical usage in an autonomous satellite will require low-latency low-compute answers.}\YH{Fixed}

% ~\footnote{\href{https://github.com/BoChenYS/satellite-pose-estimation}{https://github.com/BoChenYS/satellite-pose-estimation}}$^{,}$\footnote{\href{https://github.com/cvlab-epfl/segmentation-driven-pose}{https://github.com/cvlab-epfl/segmentation-driven-pose}}

%  
 
% !TEX root = ../top.tex
% !TEX spellcheck = en-US

\begin{figure}[t]
    \begin{center}
    \includegraphics[width=0.6\linewidth]{./fig/swisscube_statistics/swisscube_statistics.pdf}
    % \fbox{\rule{0pt}{2in} \rule{0.25\linewidth}{0pt}}
    \end{center}
    \vspace{-6mm}
    \caption{{\bf The depth distribution of the target object in datasets.} 
    Nearly 80\% of the SPEED dataset is located in the depth range of 1 to 5 times the object diameter. As contrast, our Swisscube dataset is uniformly distributed among the depth range approximately. Here, the depth is in the unit of times of the diameter of the target object, and note that, we do not have the accurate 3D object model of SPEED, its diameter is approximately computed from a 3D reconstruction. \YH{Suppose to remove this figure, no space}
    }
    \label{fig:swisscube_statistics}
\end{figure}


 
% !TEX root = ../top.tex
% !TEX spellcheck = en-US

\begin{table}
    \centering
    \scalebox{0.8}{
    \begin{small}
    % \rowcolors{2}{white}{gray!10}
    \begin{tabular}{cccccc}
        \toprule
        &&	\multicolumn{2}{c}{Accuracy} & \multirow{2}{*}{Model Size} & \multirow{2}{*}{FPS}\\
        && Raw & Refinement & \\
        \midrule
        \multicolumn{2}{l}{SegDriven-Z~\cite{Hu19a}} & 0.022 & - & 89.2 M & 3.1 \\
        \multicolumn{2}{l}{DLR~\cite{Chen19DLR}} & 0.017 & 0.012 & 176.2 M & 0.7 \\
        \midrule
        \multirow{2}{*}{\bf Ours} 
        & 640$\times$ & 0.018 & 0.013 & {\bf 51.5 M} & {\bf 35} \\
        & 960$\times$ & {\bf 0.016} & {\bf 0.010} & {\bf 51.5 M} & 18 \\
        % {\bf Ours} & {\bf 51.5 M} & {\bf $\sim$ 45 ms} \\
        % Chen {\it etc.} & 48.4 + 21.3 $\times$ 6 = 176.2 M & $\sim$ 1500 ms \\
        % SegDriven-Z & 44.6 + 44.6 = 89.2 M & $\sim$ 300 ms \\ 
        % \midrule
        \bottomrule
    \end{tabular}
    \end{small}
    }
    \vspace{-3mm}
    \caption{{\bf Comparison with the state of the art on SPEED.} Our method outperforms the two top-performing methods in the challenge and is much faster and lighter.}
    \label{tab:speed_stoa}
\end{table} 
\subsection{Evaluation on the SwissCube Dataset}
To facilitate the evaluation of 6D object pose estimation methods in the wide-depth-range scenario, we
introduce a novel SwissCube dataset. The renderings in this dataset account for the
precise 3D shape of the satellite and include realistic models of the star backdrop, Sun, Earth,
and target satellite, including the effects of global illumination, mainly
glossy reflection of the Sun and Earth from the satellite's surface.
To create the 3D model of the SwissCube, we modeled every mechanical part from
raw CAD files, including solar panels, antennas, and screws, and we
carefully assigned material parameters to each part.

The renderings feature a space environment based on the relative placement and
sizes of the Earth and Sun. Correct modeling of the Earth is most important, as
it is often directly observed in the images and significantly affects the
appearance of the satellite via inter-reflection. We extract a high-resolution
spectral texture of the Earth's surface and atmosphere from published data
products acquired by the NASA Visible Infrared Imaging Radiometer Suite (VIIRS)
instrument. These images account for typical cloud coverage and provide
accurate spectral color information on 6 wavelength bands. Illumination from
the Sun is also modeled spectrally using the extraterrestrial solar irradiance
spectrum. The spectral simulation performed using the open source Mitsuba 2
renderer~\cite{Nimier19} finally produces an RGB output that
can be ingested by standard computer vision tools.

% !TEX root = ../top.tex
% !TEX spellcheck = en-US

\begin{figure}[t]
    \begin{center}
    \includegraphics[width=0.6\linewidth]{./fig/render_setting/render_setting.pdf}
    % \fbox{\rule{0pt}{2in} \rule{0.25\linewidth}{0pt}}
    \end{center}
    \vspace{-6mm}
    \caption{{\bf Settings for physical rendering of SwissCube.} We physically model the Sun, the Earth, and the complex illumination conditions that can occur in space.}
    \label{fig:render_setting}
\end{figure}

The renderings also include a backdrop of galaxies, nebulae, and star clusters
based on the HYG database star catalog~\cite{hygdatabase} containing around
120K astronomical objects along with information about position and
brightness. The irradiance due to astronomical objects is orders of magnitude
below that of the Sun. To increase the diversity of the dataset, and to ensure
that the network ultimately learns to ignore such details, we boost the
brightness of astronomical objects in renderings to make them more apparent.

Following these steps, we place the SwissCube into its actual orbit located
approximately 700 km above the Earth's surface along with a virtual observer
positioned in a slightly elevated orbit. We render sequences with different
relative velocities, distances and angles. To this end, we use a wide field-of-view (100$^{\circ}$) camera whose distance to the target ranges uniformly between $1d$ to $10d$, where $d$ indicates the diameter of the SwissCube without taking the antennas into accounts.
% \MS{Do you use a wide field-of-view camera? With what angle? Does the diameter $d$ include the antennas, or is it just the cube edge length?}
% \YH{Yes, we use the virtual camera with a FOV of 100. the diameter is computed from only the cube body and does not take the antennas into accounts. And, we treat the Swisscube as an asymmetrical object.}
The high-level
setup is illustrated in Fig.~\ref{fig:render_setting}. Note that the renderings
are essentially black when the SwissCube passes into the earth's shadow, and we
detect and remove such configurations.

We generate 500 scenes each consisting of a 100-frame sequence, for a total of
50K images. We take 40K images from 400 scenes for training and the 10K
image from the remaining 100 scenes for testing. 
%We make the depth range of the %CubeSat dataset approximately uniformly distributed from 1d to 10d, as
We render the images at a 1024$\times$1024
resolution, a few of which are shown in Fig.~\ref{fig:results_demo}. During network processing, we resize the
input to 512$\times$512. 
%Although higher input resolution often means higher
%accuracy, as shown by the SPEED experiments, we will focus on this resolution
%setting for a detailed ablation study in this experiment. 
We report the ADI-0.1d accuracy at three
depth ranges, which we refer to as {\it near}, {\it medium}, and {\it far}, corresponding to the depth ranges [1d-4d],
[4d-7d], and [7d-10d], respectively.

% !TEX root = ../top.tex
% !TEX spellcheck = en-US

\begin{figure*}[t]
    \begin{center}
    % \fbox{\rule{0pt}{1in} \rule{0.25\linewidth}{0pt}}
    \includegraphics[width=0.135\linewidth]{./fig/results_demo/_data_swisscube_20200922_hu_test_seq_000482_000000_rgb_000006.png}
    \includegraphics[width=0.135\linewidth]{./fig/results_demo/_data_swisscube_20200922_hu_test_seq_000401_000000_rgb_000011.png}
    \includegraphics[width=0.135\linewidth]{./fig/results_demo/_data_swisscube_20200922_hu_test_seq_000424_000000_rgb_000082.png}
    \includegraphics[width=0.135\linewidth]{./fig/results_demo/_data_swisscube_20200922_hu_test_seq_000406_000000_rgb_000052.png}
    \includegraphics[width=0.135\linewidth]{./fig/results_demo/_data_swisscube_20200922_hu_test_seq_000410_000000_rgb_000038.png}
    \includegraphics[width=0.135\linewidth]{./fig/results_demo/_data_swisscube_20200922_hu_test_seq_000417_000000_rgb_000058.png} \\
    \includegraphics[width=0.135\linewidth]{./fig/results_demo/_data_swisscube_20200922_hu_test_seq_000482_000000_rgb_000006_pred.png}
    \includegraphics[width=0.135\linewidth]{./fig/results_demo/_data_swisscube_20200922_hu_test_seq_000401_000000_rgb_000011_pred.png}
    \includegraphics[width=0.135\linewidth,trim=0 100 200 100, clip]{./fig/results_demo/_data_swisscube_20200922_hu_test_seq_000424_000000_rgb_000082_pred.png}
    \includegraphics[width=0.135\linewidth,trim=0 100 200 100, clip]{./fig/results_demo/_data_swisscube_20200922_hu_test_seq_000406_000000_rgb_000052_pred.png}
    \includegraphics[width=0.135\linewidth,trim=100 50 200 250, clip]{./fig/results_demo/_data_swisscube_20200922_hu_test_seq_000410_000000_rgb_000038_pred.png}
    \includegraphics[width=0.135\linewidth,trim=250 200 50 100, clip]{./fig/results_demo/_data_swisscube_20200922_hu_test_seq_000417_000000_rgb_000058_pred.png}
    \end{center}
    \vspace{-6mm}
    \caption{{\bf Qualitative results on the SwissCube dataset.} Our method yields accurate pose estimates at all scales.}
    \label{fig:results_demo}
\end{figure*}

\subsubsection{Effect of our Ensemble-Aware Sampling}
We first evaluate the effectiveness of our ensemble-aware sampling strategy, further comparing our approach with the single-scale baseline SegDriven~\cite{Hu19a}, which uses the same backbone as us. Note that the original SegDriven method did not rely on a detector to zoom in on the object, but was extended with a YOLOv3~\cite{Redmon18} one in the SPEED competition, resulting in the SegDriven-Z approach evaluated above. For our comparison on the SwissCube dataset to be fair, we therefore also report the results of SegDriven-Z.
% \MS{Could we also evaluate Chen on this dataset? This would be more convincing, although probably too late.} \YH{We had the result, I will add it back.}
Moreover, we also evaluate the top performer on the SPEED dataset, DLR~\cite{Chen19DLR}, on our dataset.

Fig.~\ref{fig:param_study} demonstrates the effectiveness of our sampling strategy.
Our results with different $\lambda$ values, which controls the ensemble-aware sampling, show that large values, such as $\lambda>10$, yield lower accuracies. With such large values, our sampling strategy degenerates to the one commonly-used in FPN-based object detectors. This therefore evidences the importance of encouraging every pyramid level to produce valid estimates at more than a single object scale. 
%adopts is much inferior to other settings. That big $\lambda$ makes every pyramid level working on unoverlapped training instances, making different pyramid levels uncombinable during inference for a specific instance. On the other hand, the case of 
Note also that $\lambda=0$, which corresponds to distributing every training instance uniformly to all levels, does not yield the best results, suggesting that forcing every level to produce high-accuracy at all the scales is sub-optimal. In other words, each level should perform well in a reasonable scale range, but these ranges should overlap across the pyramid levels. 
%The imposing of large variation difficulties to every pyramid level makes their performance deteriorate, leading to a worse fusion accuracy. 
This is achieved approximately with $\lambda=1$, which we will use in the following experiments.

Table~\ref{tab:parameters_study} summarizes the comparison results with other baselines. Because it does not explicitly handle scale, SegDriven performs poorly on far objects. This is improved by the detector used in SegDiven-Z. However, the performance of this two-stage approach remains much worse than that of our framework.
Our method outperforms DLR as well, even though our method is 20+ times faster than DLR.
% , independently of the hyper-parameter value $\lambda$, controlling the ensemble-aware sampling. 
Fig.~\ref{fig:results_demo} depicts a few rendered images and corresponding poses estimated with our approach. 
% \MS{I would tend to show this at the end of the first subsection, and potentially compare with SegDriven-Z.}

 
% !TEX root = ../top.tex
% !TEX spellcheck = en-US

\begin{table}
    \centering
    \scalebox{0.8}{
    \begin{small}
    % \rowcolors{2}{white}{gray!10}
    \begin{tabular}{lcccc}
    \toprule
    & Near & Medium & Far & All  \\
    \midrule
    SegDriven~\cite{Hu19a} &  41.1 & 22.9 & 7.1 & 21.8 \\ 
    SegDriven-Z~\cite{Hu19a} &  52.6 & 45.4 & 29.4 & 43.2 \\ 
    DLR~\cite{Chen19DLR} & 63.8 & 47.8 & 28.9 & 46.8 \\
    {\bf Ours} & {\bf 65.2} & {\bf 48.7} & {\bf 31.9} & {\bf 47.9} \\
    \bottomrule
    \end{tabular}
    \end{small}
    }
    \vspace{-3mm}
    \caption{\bf Our method outperforms all baselines on SwissCube.}
        % Our multi-scale framework outperforms the single-scale baseline SegDriven~\cite{Hu19a} and its zoomed version (SegDriven-Z) significantly, and also DLR~\cite{Chen19DLR}, the top performer on SPEED dataset.
    \label{tab:parameters_study}
\end{table} 

 
% !TEX root = ../top.tex
% !TEX spellcheck = en-US

\begin{table}
    \centering
    \scalebox{0.8}{
    \begin{small}
    % \rowcolors{2}{white}{gray!10}
    \begin{tabular}{lcccc}
    \toprule
    & Near & Medium & Far & All  \\
    \midrule
    SegDriven~\cite{Hu19a} &  41.1 & 22.9 & 7.1 & 21.8 \\ 
    SegDriven-Z~\cite{Hu19a} &  52.6 & 45.4 & 29.4 & 43.2 \\ 
    DLR~\cite{Chen19DLR} & 63.8 & 47.8 & 28.9 & 46.8 \\
    {\bf Ours} & {\bf 65.2} & {\bf 48.7} & {\bf 31.9} & {\bf 47.9} \\
    \bottomrule
    \end{tabular}
    \end{small}
    }
    \vspace{-3mm}
    \caption{\bf Our method outperforms all baselines on SwissCube.}
        % Our multi-scale framework outperforms the single-scale baseline SegDriven~\cite{Hu19a} and its zoomed version (SegDriven-Z) significantly, and also DLR~\cite{Chen19DLR}, the top performer on SPEED dataset.
    \label{tab:parameters_study}
\end{table} 

 
% !TEX root = ../top.tex
% !TEX spellcheck = en-US

\begin{table}
    \centering
    \scalebox{0.8}{
    \begin{small}
    % \rowcolors{2}{white}{gray!10}
    \begin{tabular}{ccccc}
    \toprule
    & Near & Medium & Far & All \\
    \midrule
    L1 & 0 & 25.2 & \underline{31.8} & 19.5 \\
    L2 & 36.5 & \underline{48.4} & 27.7 & 38.2 \\
    L3 & \underline{62.3} & 47.4 & 19.9 & \underline{42.6} \\
    L4 & 59.2 & 20.2 & 1.7 & 26.3 \\
    L5 & 25.5 & 0.9 & 0 & 8.3 \\
    \midrule
    {\bf Fusion} & {\bf 65.2} & {\bf 48.7} & {\bf 31.9} & {\bf 47.9} \\
    \bottomrule
    \end{tabular}
    \end{small}
    }
    \vspace{-3mm}
    \caption{{\bf Effect of the multi-scale fusion.} Each pyramid level favors a specific depth range, which our multi-scale fusion strategy leverages to outperform every individual level.}
    \label{tab:fusion_effect}
\end{table} 


\subsubsection{Effect of our Multi-Scale Fusion}

To better understand the role of each pyramid level during multi-scale fusion, we study the accuracy obtained using the predictions of each individual pyramid level.
Intuitively, we expect the levels with a larger receptive field (feature maps with low spatial resolution) to perform well for close objects, and those with a small receptive field (feature maps with high spatial resolution) to produce better results far-away ones. While the results in Table~\ref{tab:fusion_effect} confirm this intuition for Levels L1, L2 and L3, we observe that the performance degrades at L4 and L5. We believe this to be due to the very low spatial resolution of the corresponding feature maps, 8$\times$8, and 4$\times$4, respectively, making it difficult for these levels to output precise poses. Nevertheless, the accuracy after multi-scale fusion outperforms every individual level, and we leave the study of a different number of pyramid levels to future work.
% \MS{This suggests that we should probably just stop at L3...}\YH{Although the performance of L4 and L5 alone is bad, we are not sure if the L4 or L5 can contribute to the final loss via ensemble. We need more experiments to verify it, so leave it as it is right now.}

%performance with the results combined only from feature cells within each level's segmentation mask. Table~\ref{tab:fusion_effect} shows the results. In intuition, levels with larger reception fields perform better for closer objects and vice versa. However, we find that this is not always true. The performance of level 4 on near objects can not match the one on level 3, and level 5 becomes even more worse. Note that, the spatial feature dimensions for L1, L2, L3, L4, and L5 are 64$\times$64, 32$\times$32, 16$\times$16, 8$\times$8, and 4$\times$4, respectively. Although L4, especially L5, has larger reception fields, the lower spatial resolution makes them less discriminable against 2D keypoints and introduces more visual noises for each cell. Nevertheless, the accuracy after multi-scale fusion outperforms every single level and we leave the study of a different number of pyramid levels to future work.

\subsubsection{Effect of the 3D Loss}

 
% !TEX root = ../top.tex
% !TEX spellcheck = en-US

\begin{table}
    \centering
    \scalebox{0.8}{
    \begin{small}
    % \rowcolors{2}{white}{gray!10}
    \begin{tabular}{cccccccccc}
    \toprule
    & Near & Medium & Far & All \\
    \midrule
    2D loss & 64.6 & 42.0 & 24.0 & 43.1   \\
    {\bf 3D loss} & {\bf 65.2} & {\bf 48.7} & {\bf 31.9} & {\bf 47.9} \\
    \midrule
    Delta & +0.6 & +6.7 & +7.9 & +4.8 \\
    \bottomrule
    \end{tabular}
    \end{small}
    }
    \vspace{-3mm}
    \caption{{\bf Effect of the 3D loss.} The proposed 3D loss outperforms the 2D one in every depth ranges. The farther the object, the more obvious the advantage of the 3D loss.}
    \label{tab:error_3d_vs_2d}
\end{table} 

% !TEX root = ../top.tex
% !TEX spellcheck = en-US

\begin{figure}[t]
\centering
\includegraphics[width=0.6\linewidth]{./fig/error_vs_positions/error_vs_pos.pdf}
\vspace{-3mm}
\caption{\small {\bf Pose error as a function of the object position.} The performance of the 2D loss clearly degrades for objects near the image center, whereas that of our 3D loss doesn't. See Fig.~\ref{fig:cube_problem}(b) for the underlying geometry. Note that as the object moves closer to the image boundary, it becomes truncated, which degrades the performance of both losses.}
\label{fig:error_vs_positions}
\end{figure}

%The popular 2D reprojection loss has server problems in the wide-depth-range scenarios as discussed in Fig.~\ref{fig:cube_problem}. To fairly compare the proposed 3D loss against the 2D loss, we train our framework two times from the same initial states and with the same other settings except for the adopted regression loss. 
In Table~\ref{tab:error_3d_vs_2d}, we compare the results obtained by training our approach with either the commonly-used 2D reprojection loss or our loss function in 3D space. Note that our 3D loss outperforms the 2D one in all depth ranges, and the farther the object, the larger the gap between the results of the two loss functions.
In Fig.~\ref{fig:error_vs_positions}, we plot the average accuracy as a function of the object image location. The performance of the 2D loss degrades significantly when the object is located near the image center, whereas the accuracy of our 3D loss remains stable for most object positions. Note that, The reason both of them become worse in the right part of the figure is due to the object truncation by image borders.

\subsection{Results on Real Images}

In Fig.~\ref{fig:domain_adaptation}, we illustrate the performance of our approach on real images. Note that these real images were not captured in space but in a lab environment using a mock-up model of the target and an OptiTrack motion capture system to obtain ground-truth pose information for a few images. We then fine-tuned our model pre-trained on our synthetic SwissCube dataset using only 20 real images with pose annotations. Because this procedure only requires small amounts of annotated real data, it would be applicable in an actual mission, where images can be sent to the ground, annotated manually, and the updated network parameters uploaded back to space.
%Although our CubeSat dataset is rendered by a computer, thanks to its high realism, it can be easily adapted to real data. For the real data, we obtain it by capturing a real-size mock-up of the target. We use a simple finetune~\cite{1}, which is a very basic domain adaptation technique, to adapt our model to the read data. shows some real results on two different satellites, CubeSat and VESPA as well. Although the real data is not captured from the ``real'' space and we are sure we can find better domain adaptation methods, it shines a bright light for the preparation of the real launching in the future.

%  
% !TEX root = ../top.tex
% !TEX spellcheck = en-US

\begin{table}
    \centering
    \begin{small}
    % \rowcolors{2}{white}{gray!10}
    \begin{tabular}{lcccc}
        \toprule
        &	Near & Medium & Far & All\\
        \midrule
        {\bf Ours} & {\bf 60.8} & {\bf 51.6} & {\bf 35.1} & {\bf 49.0} \\
        Chen {\it etc.} & 56.7 & 48.2 & 32.8 & 46.1 \\
        SegDriven-Z & 52.6 & 45.4 & 29.4 & 43.2 \\ 
        % \midrule
        \bottomrule
    \end{tabular}
    \end{small}
    \vspace{-3mm}
    \caption{{\bf Comparison with the state of the art on CubeSat.} bla bla bla bla bla bla bla bla bla bla bla bla bla bla bla bla bla bla bla bla bla bla bla bla bla bla bla bla bla bla bla bla bla bla bla bla a bla bla bla bla bla bla bla bla bla bla bla bla bla bla bla bla bla bla bla bla bla bla bla bla bla bla bla bla bla bla bla bla bla bla bla bla bla bla bla bla v, much more faster as shown in Table~\ref{tab:swisscube_stoa}.}
    \label{tab:swisscube_stoa}
\end{table} 
% !TEX root = ../top.tex
% !TEX spellcheck = en-US

\begin{figure}[t]
\centering
\includegraphics[width=0.29\linewidth,trim=450 380 400 400,clip]{fig/real_results/im_36_12_723405.jpg}
\includegraphics[width=0.29\linewidth,trim=450 380 400 400,clip]{fig/real_results/im_409_50_004279.jpg}
\includegraphics[width=0.29\linewidth,trim=450 380 400 400,clip]{fig/real_results/im_880_97_080135.jpg}
% \begin{tabular}{cc}
%     \fbox{\rule{0pt}{1.5in} \rule{0.4\linewidth}{0pt}} &
%     \fbox{\rule{0pt}{1.5in} \rule{0.4\linewidth}{0pt}} \\
%     (a) SwissCube & (b) VESPA\\
% \end{tabular}
\vspace{-3mm}
\caption{\small {\bf Qualitative results on real data.} Our model easily adapts to real data, using as few as 20 annotated images.
}
\label{fig:domain_adaptation}
\end{figure}
 
% !TEX root = ../top.tex
% !TEX spellcheck = en-US

\begin{table}
    \centering
    \scalebox{0.8}{
    % \rowcolors{2}{white}{gray!10}
    \begin{small}
    \begin{tabular}{L{4em}C{3em}C{5em}C{3em}C{3em}}
        \toprule
        & PVNet & SimplePnP & Hybrid & {\bf Ours}\\
        \midrule
        Ape    & 15.8 & 19.2 & 20.9 &  {\bf 22.3} \\
        Can    & 63.3 & 65.1 & 75.3 &  {\bf 77.8} \\
        Cat    & 16.7 & 18.9 & 24.9 &  {\bf 25.1} \\
        Driller& 65.7 & 69.0 & 70.2 &  {\bf 70.6} \\
        Duck   & 25.2 & 25.3 & 27.9 &  {\bf 30.2} \\
    Eggbox$^*$ & 50.2 & 52.0 & 52.4 &  {\bf 52.5} \\
    Glue$^*$   & 49.6 & 51.4 & 53.8 &  {\bf 54.9} \\
       Holepun.& 39.7 & 45.6 & 54.2 &  {\bf 55.6} \\
        \midrule
        Avg.   & 40.8 & 43.3 & 47.5 &  {\bf 48.6} \\
        \bottomrule
    \end{tabular}
    \end{small}
    }
    \vspace{-3mm}
    \caption{{\bf Comparison on Occluded-LINEMOD.} We compare our results with those of PVNet~\cite{Peng19a}, SimplePnP~\cite{Hu20a} and Hybrid~\cite{Song20a}. Symmetry objects are denoted with ``$^*$''.}
    \label{tab:occ_linemod_stoa}
\end{table} 

\subsection{Evaluation on Occluded-LINEMOD}

Finally, to demonstrate that our approach is general, and thus applies to datasets depicting small depth variations, we evaluate it on the standard Occluded-LINEMOD dataset~\cite{Krull15}. Following~\cite{Hu20a}, we use the raw images at resolution 640$\times$480 as input to our network, train our model on the LINEMOD~\cite{Hinterstoisser12b} dataset and test it on Occluded-LINEMOD without overlapped data. Although our framework supports multi-object training, for the evaluation to be fair, we train one model for each object type and compare it with methods not relying on another refinement procedure.
Considering the small depth variations in this dataset, we remove the two pyramid levels with the largest reception fields from our framework, leaving only ${\cal F}_1$, ${\cal F}_2$ and ${\cal F}_3$. As shown in Table~\ref{tab:occ_linemod_stoa}, our model outperforms the state of the art even in this general 6D object pose estimation scenario.

%shows the comparison results of our framework against the state-of-the-art methods. It shows that our multi-scale fusion framework also works pretty well in general 6D object pose estimation.


\section{Conclusion}
In this work, we present a novel strategy for addressing few-shot open-set recognition. We frame the few-shot open-set classification task as a meta-learning problem similar to \cite{peeler}, but unlike their strategy, we do not solely rely on thresholding softmax scores to indicate the openness of a sample. We argue that existing thresholding type FSOSR methods \cite{peeler,snatcher} rely heavily on the choice of a carefully tuned threshold to achieve good performance. Additionally, the proclivity of softmax to overfit to unseen classes makes it an unreliable choice as an open-set indicator, especially when there is a dearth of samples. Instead, we propose to use a reconstruction of exemplar images as a key signal to detect out-of-distribution samples. 
The learned embedding which is used to classify the sample is further modulated to ensure a proficient gap between the seen and unseen class clusters in the feature space. Finally, the modulated embedding, the softmax score, and the quality reconstructed exemplar are jointly utilized to cognize if the sample is in-distribution or out-of-distribution. 
The enhanced performance of our framework is verified empirically over a wide variety of few-shot tasks and the results establish it as the new state-of-the-art. In the future, we would like to extend this approach to more cross-domain few-shot tasks, including videos.
\vspace{-2em}
\section{Acknowledgement}
This work was partially supported by US National Science Foundation grant 2008020 and US Office of Naval Research grants N00014-19-1-2264 and N00014-18-1-2252.
\vspace{-1em}
% \LX{TODO: fix citation; Main experiment part}

% \LX{@Fuzhao, can we put ablation and analysis in the appendix?}
% \FZ{Yes}

% \LX{@Fuzhao, should we include related work part and put it in the appendix?}
% \FZ{Yes}

% \FZ{Use ChatGPT to help your writing.}

% \FZ{You can see the template I sent to you. That would give you more space to place your figure and tables.}


% \clearpage
\section*{Contribution}
This work has been expanded upon from a course project that was undertaken by the key authors, Xiao Liu, Jian Zhang, and Heng Zhang, during their time as students at the National University of Singapore. Fuzhao Xue, the teaching assistant, and project mentor played a role in guiding and shaping the outcome of this work. We are grateful to Yang You for providing us with valuable instructions and computational resources. 

\section*{Acknowledgement}
Yang You's research group is being sponsored by NUS startup grant (Presidential Young Professorship), Singapore MOE Tier-1 grant, ByteDance grant, ARCTIC grant, SMI grant and Alibaba grant. Our sincere appreciation also goes out to Min-Yen Kan, our course lecturer, for his invaluable suggestions during our enrollment.

% This work was extended from a course project during key authors' (Xiao Liu, Jian Zhang, Heng Zhang) time as students at the National University of Singapore. Fuzhao Xue was the teaching assistant and projector mentor of this work.


\section*{URM Statement}
The authors acknowledge that the first author of this work meets the URM criteria of ICLR 2023 Tiny Papers Track.

\bibliography{iclr2023_conference_tinypaper}
\bibliographystyle{iclr2023_conference_tinypaper}

% % !TEX root = ../supp.tex
% !TEX spellcheck = en-US

\section{Physical Rendering of SwissCube}
\label{sec:appendix}

Although the European Space Agency has organized a satellite pose estimation challenge and released the SPEED satellite dataset, the unavailability of the target 3D model makes the pose accuracy not depending on the pose estimation method alone. Furthermore, the limited varieties of lighting also make it soon saturated and less discriminative, as discussed in Section~\ref{sec:related}.

To fully demonstrate the effectiveness of our method in space, we introduce the Swisscube satellite dataset. Swisscube is a Cubesat-type satellite which was designed at EPFL and launched in 2009. Given the accurate CAD files and material properties of each component of it, we synthesize photorealistic images using physically based rendering~\cite{xx}.

 
% !TEX root = ../top.tex
% !TEX spellcheck = en-US

\begin{table}
    \centering
    \begin{small}
    % \rowcolors{2}{white}{gray!10}
    \begin{tabular}{lcc}
        \toprule
        &	SPEED & {\bf CubeSat}\\
        \midrule
        % Synthetic  &12k & 40k \\
        % Real       & 5 & 300 \\
        % \midrule
        Size & 12k & 40k \\
        Accurate 3D model   &\xmark  & \cmark  \\
        Complex lighting    &\xmark  & \cmark \\
        Physical modeling    &\xmark  & \cmark \\
        Colors        &\xmark  & \cmark \\
        Sequences         &\xmark  & \cmark  \\
        Depth distribution & non-uniform & uniform \\
        \bottomrule
    \end{tabular}
    \end{small}
    % \vspace{-3mm}
    \caption{{\bf CubeSat dataset.} bla bla bla bla bla bla bla bla bla bla bla bla bla bla bla bla bla bla bla bla bla bla bla bla bla bla bla bla bla bla bla bla bla bla bla bla bla bla bla bla bla bla bla bla bla bla bla bla bla bla bla bla bla bla bla bla bla bla bla bla bla bla bla bla bla bla bla bla bla bla bla bla bla bla bla bla bla bla bla bla }
    \label{tab:swisscube_vs_speed}
\end{table} 

% -----------------------------------
% Physically-based spectral rendering

\subsubsection{Physically-based spectral rendering}

In this section we provide a high-level description of the Swisscube satellite dataset which collects 40'000  physically-based synthetic images. While the SPEED satellite dataset images were produced using an OpenGL-based RGB rendering framework, we opted for a physically-based approach, where every element of the rendering pipeline were carefully modeled to mimic reality.

While the RGB model is often used to render color images, using tristimulus RGB colors in the rendering simulation generally yields non-physical results. For instance, surface reflectance properties of an object can be highly dependent on the wavelengths, which won't be accurately reproduced with RGB values. In a spectral renderer, colors are represented as spectral power distributions, resulting in improved accuracy especially when measured spectral data is available. For the Swisscube dataset, using a spectral renderer was a necessity as it was a requirement to correctly model the spectral responses of the solar irradiance emission, material reflectance properties and Earth surface radiance. Although the rendering simulation uses spectral colors, the resulting images will be converted to RGB images.

Relying on a physically-based rendering pipeline also gives us more control on the dynamic range of the output images. Thus we were able to accurately reproduce highlights orders of magnitude brighter than darker region of the images. An appropriate gamma curve could then be applied to produce images that can be viewed on regular displays.

To achieve all of this, we build our pipeline around the Mitsuba 2 renderer [???] which is a highly modular open-source framework that supports spectral rendering.

% -------------------
% The 3D / CAD model

\subsubsection{Accurate 3D model from CAD data}

For this dataset, we modelled every mechanical parts of the SwissCube, such as solar panels, antennas, and screws based directly on the raw CAD files. We carefully assign material reflection properties to each part. Given the physically-based nature of the pipeline, it would be possible to use efficient material acquisition technique such as [???] in the future for better results. Due to confidential reasons, we only release the mesh geometry data of the combined SwissCube without separable pieces to the public, which is enough for perfect registration.

% Citation for material acquistion technique: https://rgl.epfl.ch/publications/Dupuy2018Adaptive

% --------------------
% Physically-based Sun

\subsubsection{Modeling a physically-based Sun emitter}

In order to correctly model the illumination from the Sun, we leveraged the vast literature in astrophysics. As the target object will be placed above the Earth atmosphere, it is not enough to use specular solar irradiance measurements made at ground surface [???] as those will be be affected by the highly variable and absorbing constituents of the Earth atmosphere. Instead, we rely on the air mass zero reference spectrum [???], also known as extraterrestrial solar irradiance, mainly based on data from satellites and space shuttle missions. Figure \ref{fig:swisscube_sun_spectrum} show its spectral power distribution. We then use a point light source to represent the Sun, placed at the correct distance to the Earth. Note that is was necessary to scale the Sun irradiance to account for its surface area.

% Groud surface citation: https://www.osapublishing.org/ao/abstract.cfm?uri=ao-21-3-554
% ASTM Standard Extraterrestrial Spectrum Reference: https://www.nrel.gov/grid/solar-resource/spectra-astm-e490.html

% !TEX root = ../top.tex
% !TEX spellcheck = en-US

\begin{figure}[t]
    \begin{center}
    \includegraphics[width=1.0\linewidth]{./fig/swisscube_sun_spectrum/sun_spectrum_plot.jpeg}
    % \fbox{\rule{0pt}{2in} \rule{0.25\linewidth}{0pt}}
    \end{center}
    \vspace{-6mm}
    \caption{{\bf Air mass zero solar spectral power distribution.} 
    }
    \label{fig:swisscube_sun_spectrum}
\end{figure}


% ---------------------------
% Modeling the stars / galaxy

\subsubsection{Modeling the stars and galaxies}

We also added galaxies and other astronomical objects to our pipeline as we believe those could distract the learning algorithm. Based on the HYG database star catalogue [???], we could generate a high-resolution environment map that we later used as an second emitter. The HYG database contains around 220 thousands astronomical objects, mostly galaxies but also star clusters and Nebulae along with information regarding their position and brightness. Figure \ref{fig:swisscube_stars_envmap} shows the astronomical object projected on a spherical coordinate 2D map with their respective physically-based brightness.

Compared to the sun illumination, the irradiance coming from the stars is orders of magnitude lower. On the other hand, to maximize the diversities of the generated data, we decided to increase the actual brightness of each star in the galaxies to make them more apparent in rendering, which we think is a beneficial perturbation for a dataset.

% !TEX root = ../top.tex
% !TEX spellcheck = en-US

\begin{figure}[t]
    \begin{center}
    \includegraphics[width=1.0\linewidth]{./fig/swisscube_stars_envmap/stars_envmap.jpeg}
    % \fbox{\rule{0pt}{2in} \rule{0.25\linewidth}{0pt}}
    \end{center}
    \vspace{-6mm}
    \caption{{\bf Astronomical objects environment map based on the HYG dataset.} 
    }
    \label{fig:swisscube_stars_envmap}
\end{figure}


% HYG database link: http://www.astronexus.com/hyg

% ---------------------------------------------------
% Modeling the Earth radiance using the VIIRS dataset

\subsubsection{Spectral Earth radiance using the VIIRS dataset}

Properly modeling the Earth is very important here as it often occupies a large portion of the images. Moreover, the Sun light reflecting off the Earth is drastically affecting the illumination of the target object. In our pipeline, the Earth is a represented as a very large sphere, reflecting light coming from the Sun emitter onto the target object or directly towards the camera. Based on the NASA Visible Infrared Imaging Radiometer Suite (VIIRS) Level-1B data products [????], we generated a spectral radiance texture to model the reflectance of the Earth and its atmosphere. The VIIRS data products are produced by whiskbroom scanning radiometers on satellites orbiting around the Earth at a nominal altitude of 829 km, providing a full daily coverage of the Earth. These data products include 6 bands in the visible spectrum with high spatial resolution which we could use to generate a spectral reflectance texture. Figure \ref{fig:swisscube_earth_renders} shows the result of this process compared to the use of a simple Earth albedo texture available from the NASA website [????].

% !TEX root = ../supp.tex
% !TEX spellcheck = en-US

\begin{figure}[t]
    \centering
    \begin{tabular}{ccc}
    \includegraphics[height=2.4cm]{fig/swisscube_earth_renders_L1/earth_L1_albedo_texture.png}&
    \includegraphics[height=2.4cm]{fig/swisscube_earth_renders_L1/earth_L1_spectral_texture.png}&
    \includegraphics[height=2.4cm]{fig/swisscube_earth_renders_L1/earth_L1_DSCOVR.png}\\
    (a)&(b)&(c)\\
    \end{tabular}
    \vspace{-3mm}
    \caption{\small {\bf Rendered Earth compared to DSCOVR photograph.} (a) Ground-level albedo texture doesn't account for the scattering effects introduced by the atmosphere, resulting in over saturated colors when viewed from space. (b) Rendering of the Earth using our spectral texture based on the VIIRS data products. (c) Ground-truth real photograph taken by the DSCOVR satellite at the L1 Lagrange point.}
    \label{fig:swisscube_earth_renders}
\end{figure}

% VIIRS website: https://earthdata.nasa.gov/earth-observation-data/near-real-time/download-nrt-data/viirs-nrt#ed-corrected-reflectance
% Earth albedo texture: https://visibleearth.nasa.gov/images/57735/the-blue-marble-land-surface-ocean-color-sea-ice-and-clouds/57737l
% Real image DSCOVR website: https://www.nesdis.noaa.gov/content/dscovr-deep-space-climate-observatory

\subsubsection{Bring everything together at real scale}

We placed all those elements in a virtual scene at the real scale to ensure high fidelity images and produce a more comprehensive dataset. We could then generate sequences of images, simulating various docking procedures by varying camera and target vehicle poses and respective speed. Each sequence contains 100 consecutive images.



To achieve highest reflection of the real world, we place the SwissCube in the actual-working orbits about 700 km above the Earth's surface during rendering and model most of the space-borne items, such as the Sun, the Earth, and galaxies, physically. 

Depth range.

\subsubsection{Dataset generation and specs}

We generate 400 sequences in total, each sequence contains 100 consecutive images with random angular speed between 0.xx to 0.xx.

Fig.~\ref{fig:swisscube_vs_speed} shows some examples of the generated SwissCube dataset.

bla bla bla bla bla bla bla bla bla bla bla bla bla bla bla bla bla bla bla bla bla bla bla bla bla bla bla bla bla bla bla bla bla bla bla bla bla bla bla bla bla bla 

As discussed Section~\ref{sec:related}. Table~\ref{tab:swisscube_vs_speed} shows the comparison between SwissCube and SPEED dataset. Note that, in SPEED dataset, there are 2998 more synthetic images and also 300 more real images in the test set. However, Their ground truth labels are not accessible, so here we do not take them into account.

\YH{TODO}

From the total 400 sequences of images in the SwissCube dataset, we take all the images from the first 300 sequences as our training set and that from the last 100 as the test. In this subsection, we will evaluate the methods' performance in different depth ranges, that is, the whole depth range will be divided to three regions denoted as {\it Near}, {\it Medium} and {\it Far}, which correspond to depth range [1d-4d], [4d-7d] and [7d-10d], respectively.

\end{document}

