
\documentclass{article} % For LaTeX2e
\usepackage{iclr2023_conference_tinypaper,times}

% Optional math commands from https://github.com/goodfeli/dlbook_notation.
%%%%% NEW MATH DEFINITIONS %%%%%

\usepackage{amsmath,amsfonts,bm}
\usepackage{xifthen}

% Highlight a newly defined term
\newcommand{\newterm}[1]{{\bf #1}}

\def\eps{{\epsilon}}


% Utility for ticks 
\newcommand{\cmark}{\ding{51}}%
\newcommand{\xmark}{\ding{55}}%

% Theorem styles 
\theoremstyle{definition}
\newtheorem{theorem}{Theorem}[section]
\newtheorem{definition}{Definition}[section]
% \newtheorem{remark}{Remark}[theorem] %numbered remark
\newtheorem*{remark}{Remark} %unnumbered remark
\newtheorem{lemma}{Lemma}[section]
\newtheorem{prop}{Proposition}[section]
\newtheorem{corollary}{Corollary}[theorem]
\newtheorem{conjecture}{Conjecture}[section]
\newtheorem{assumption}{Assumption}[section]

\newtheorem{manualtheoreminner}{Theorem}
\newenvironment{manualtheorem}[1]{%
  \renewcommand\themanualtheoreminner{#1}%
  \manualtheoreminner
}{\endmanualtheoreminner}


% Math helper - standard function
\DeclareMathOperator*{\argmax}{arg\,max}
\DeclareMathOperator*{\argmin}{arg\,min}
\DeclareMathOperator{\support}{support}
\DeclareMathOperator{\MAX}{MAX}
\DeclareMathOperator{\term}{\texttt{term}}
\DeclareMathOperator*{\logsumexp}{log-sum-exp}
\DeclareMathOperator*{\TV}{TV}
\newcommand{\norm}[1]{\left\lVert#1\right\rVert}
\DeclarePairedDelimiter\set\{\}
\DeclarePairedDelimiter\abs{\lvert}{\rvert}%
\newcommand*{\mytop}{\mathrel{\scalebox{0.5}{$\top$}}}
\newcommand*{\mybot}{\mathrel{\scalebox{0.5}{$\bot$}}}
\newcommand*{\mydiese}{\mathrel{\scalebox{0.5}{$\#$}}}
\newcommand*{\myplus}{\mathrel{\scalebox{0.5}{$+$}}}
\newcommand*{\myminus}{\mathrel{\scalebox{0.5}{$-$}}}
\newcommand*{\bmg}{\bm{\gamma}}
\newcommand*{\bml}{\bm{\lambda}}

% MDP notation
\renewcommand{\S}{\mathcal{S}}
\newcommand{\X}{\mathcal{X}}
\newcommand{\A}{\mathcal{A}}
\newcommand{\T}{\mathcal{T}}
\newcommand{\M}{\mathcal{M}}
\newcommand{\B}{\mathcal{B}}
\newcommand{\Bset}{\mathfrak{B}}
\newcommand{\Dist}{\mathscr{P}}
\newcommand{\D}{\mathcal{D}}
\newcommand{\Real}{\mathbb{R}}
\renewcommand{\P}{\mathcal{P}}
\newcommand{\E}{\mathop{\mathbb{E}}}
\renewcommand{\H}{\mathcal{H}}
% \newcommand{\R}{\mathcal{R}}
% \newcommand{\C}{\mathcal{C}}

% Extended MDP notation
\newcommand{\Pstar}{p^{\star}}
\newcommand{\Rstar}{\bm{r}^{\star}}
\newcommand{\Cstar}{C^{\star}}
% \newcommand{\rmax}{\textsc{Rmax}}
\newcommand{\rmax}{r_{\mytop}}
\newcommand{\cmax}{\textsc{Cmax}}

\newcommand{\mstar}{m^{\star}}
\newcommand{\mhat}{\hat{m}}
\newcommand{\mopt}{m^{\star}}

\newcommand{\Phat}{\hat{p}}
\newcommand{\Rhat}{\hat{\bm{r}}}
\newcommand{\Chat}{\hat{C}}

% Math helper - custom function
\newcommand{\expwrtpi}[1]{\E_{\pi} [\sum_{t=0}^{\infty} \gamma^t #1(s_t, a_t)]}
\newcommand{\expangle}[1]{\langle #1  \rangle}

% helper function for return and constraints

% for value function, takes arguments:
% #1: policy 
% #2: the function of interest, R or C_i
% #3 (optional): the MDP for which this is estimated
\newcommand{\V}[3]{ %
    \ifthenelse{\isempty{#3}}%
    {V^{#1}(#2)}% #3 is empty 
    {V^{#1}_{#3}(#2)}%
}

\newcommand{\Q}[3]{
    \ifthenelse{\isempty{#3}}
    {Q^{#1}(#2)}% #3 is empty 
    {Q^{#1}_{#3}(#2)}%
}


\newcommand{\Adv}[3]{
    \ifthenelse{\isempty{#3}}
    {A^{#1}(#2)}% #3 is empty 
    {A^{#1}_{#3}(#2)}%
}

% careful diff notation
% 1: pi
% 2: R/C
% 3: M
\newcommand{\J}[3]{
    \ifthenelse{\isempty{#3}}
    {\mathcal{J}^{#1}_{#2}}% #3 is empty -> eg V^{\pi}(x ; R)
    {\mathcal{J}^{#1}_{#3,#2}}% -? eg V^{\pi}_{M}(x ; C)
    % {J_{#2}(#1)}% #3 is empty 
    % {J_{#2}(#1, #3)} %
}



\newcommand{\MRkern}{%
  \mkern-6.5mu
  \mathchoice{}{}{\mkern0.2mu}{\mkern0.5mu}%
}

% for value function, takes arguments:
% #1: policy 
% #2: the function of interest, R or C_i
% #3 (optional): the MDP for which this is estimated
% #4: variables to be given input (x) or (x,a)
\newcommand{\val}[4]{ %
    \ifthenelse{\isempty{#3}}%
    {v^{#1}_{#2}(#4)}% #3 is empty -> eg V^{\pi}(x ; R)
    {v^{#1}_{#3,#2}(#4)}% -? eg V^{\pi}_{M}(x ; C)
    % {V^{#1}_{#3}(#4 ;#2)}% -? eg V^{\pi}_{M}(x ; C)
    % {V_{#2}(#4 ; #1)}% #3 is empty -> eg V_R(x ; \pi)
    % {V_{#2}(#4 ;#1, #3)}% -? eg V_C(x ; \pi, M)
    % {#2 \MRkern V^{#1}_{#3}(#4)}% -? eg V^{\pi}_{M}(x ; C) # combines the letter V and R together
}

\newcommand{\qval}[4]{
    \ifthenelse{\isempty{#3}}
    {q^{#1}_{#2}(#4)}% #3 is empty -> eg V^{\pi}(x ; R)
    {q^{#1}_{#3,#2}(#4)}% -? eg V^{\pi}_{M}(x ; C)
    % {Q^{#1}(#4 ; #2)}% #3 is empty -> eg Q^{\pi}(x,a ; R)
    % {Q^{#1}_{#3}(#4 ;#2)}% -? eg Q^{\pi}_{M}(x,a ; C)
    % {Q_{#2}(#4 ; #1)}% #3 is empty -> eg Q_R(x,a ; \pi)
    % {Q_{#2}(#4 ;#1, #3)}% -? eg Q_C(x,a ; \pi, M)
}
\DeclareMathOperator*{\advantage}{Adv}

\newcommand{\adv}[4]{
    \ifthenelse{\isempty{#3}}
    {\advantage^{#1}_{#2}(#4)}% #3 is empty -> eg V^{\pi}(x ; R)
    {\advantage^{#1}_{#3,#2}(#4)}% -? eg V^{\pi}_{M}(x ; C)
    % {A^{#1}(#4 ; #2)}% #3 is empty -> eg Q^{\pi}(x,a ; R)
    % {A^{#1}_{#3}(#4 ;#2)}% -? eg Q^{\pi}_{M}(x,a ; C)
    % {A_{#2}(#4 ; #1)}% #3 is empty -> eg A_R(x,a ; \pi)
    % {A_{#2}(#4 ;#1, #3)}% -? eg A_C(x,a ; \pi, M)
}




\newcommand{\ci}{C}

\newcommand{\pib}{\pi_{b}}
\newcommand{\piopt}{\pi^{*}}
\newcommand{\pie}{\pi_{t}}

\newcommand{\lR}{\lambda_{R}}
\newcommand{\lC}{\lambda_{C}}
\newcommand{\ephi}{e_{\phi}}

\newcommand{\pr}{\text{Pr}}
\newcommand{\IS}{\text{IS}}
\newcommand{\CI}{\text{CI}}


% SPIBB symbols 
\newcommand{\EpsPib}{(\pi_b, e, \epsilon)}

\usepackage{hyperref}
\usepackage{url}
\usepackage{booktabs}
\usepackage{multirow}
\usepackage{graphicx} 
\usepackage{fixltx2e}
\usepackage{wrapfig}
\usepackage{caption}
\newcommand*{\myalign}[2]{\multicolumn{1}{#1}{#2}}
\newcommand{\LX}[1]{\textcolor{teal}{[Xiao: #1]}}
\newcommand{\FZ}[1]{\textcolor{red}{[Fuzhao: #1]}}
\newcommand*\samethanks[1][\value{footnote}]{\footnotemark[#1]}

\title{Hierarchical Dialogue Understanding with Special Tokens and Turn-level Attention}

% Authors must not appear in the submitted version. They should be hidden
% as long as the \iclrfinalcopy macro remains commented out below.
% Non-anonymous submissions will be rejected without review.

\author{Xiao Liu\thanks{This work was done during the author's time as a student at National University of Singapore}, Jian Zhang\thanks{Equal contributions}, Heng Zhang\samethanks, Fuzhao Xue, Yang You \\
Department of Computer Science, National University of Singapore\\
\texttt{\{liuxiao, zhang.jian, zhang\_heng, f.xue\}@u.nus.edu,} \\ \texttt{youy@comp.nus.edu.sg} \\
}
% The \author macro works with any number of authors. There are two commands
% used to separate the names and addresses of multiple authors: \And and \AND.
%
% Using \And between authors leaves it to \LaTeX{} to determine where to break
% the lines. Using \AND forces a linebreak at that point. So, if \LaTeX{}
% puts 3 of 4 authors names on the first line, and the last on the second
% line, try using \AND instead of \And before the third author name.

\newcommand{\fix}{\marginpar{FIX}}
\newcommand{\new}{\marginpar{NEW}}

\iclrfinalcopy % Uncomment for camera-ready version, but NOT for submission.
\begin{document}


\maketitle

\begin{abstract}
Compared with standard text, understanding dialogue is more challenging for machines as the dynamic and unexpected semantic changes in each turn. To model such inconsistent semantics, we propose a \emph{simple but effective} \textbf{Hi}erarchical \textbf{Dialog}ue Understanding model, HiDialog. Specifically, we first insert multiple special tokens into a dialogue and propose the turn-level attention to learn turn embeddings hierarchically. Then, a heterogeneous graph module is leveraged to polish the learned embeddings. We evaluate our model on various dialogue understanding tasks including dialogue relation extraction, dialogue emotion recognition, and dialogue act classification. Results show that our simple approach achieves state-of-the-art performance on all three tasks above. All our source code is publicly available at \url{https://github.com/ShawX825/HiDialog}.
\end{abstract}

%%%%%%%
\section{Introduction}

Scientific literature is most commonly available in the form of PDFs, which pose challenges for accessibility \citep{NielsenPDFStillUnfit, Bigham2016AnUT}. When researchers, students, and other individuals who are blind or low vision (BLV) interact with scientific PDFs through screen readers, the availability of document structure tags, labeled reading order, labeled headers, and image alt-text are necessary to facilitate these interactions. However, these features must be painstakingly added by authors using proprietary software tools, and as a result, are often missing from papers. Low vision or dyslexic readers who interact with PDFs through screen magnification or text-to-speech may also find the complexity of certain academic paper PDF formats challenging, e.g., non-linear layout can interrupt the flow of text in a magnifying tool. Inaccessible paper PDFs can lead to high cognitive overload, frustration, and abandonment of reading for BLV readers. 

Unfortunately, we find that the majority of scientific PDFs lack basic accessibility features. We estimate based on a sample of \numpdfs PDFs from multiple fields of study that only around \percaccessible of paper PDFs released in the last decade satisfy all of the aforementioned accessibility requirements. 
Accessibility challenges for academic PDFs are largely due to three factors: (1) the complexity of the PDF file format, which make it less amenable to certain accessibility features, (2) the dearth of tools, especially non-proprietary tools, for creating accessible PDFs, and (3) the dependency on volunteerism from the community with minimal support or enforcement \citep{Bigham2016AnUT}. The intent of the PDF file format is to support faithful visual representation of a document for printing, a goal that is inherently divergent from that of document representation for the purposes of accessibility. Though some professional organizations like the Association for Computing Machinery (ACM) have encouraged PDF accessibility through standards and writing guidelines,\footnote{\href{https://www.acm.org/publications/authors/submissions}{https://www.acm.org/publications/authors/submissions}} uptake among academic publishers and disciplines more broadly has been limited. 

While policy changes help, the fact remains that most academic PDFs produced today, and historically, are inaccessible, yet remain as the dominant way to read those papers. A long-range solution will necessitate buy-in from multiple stakeholders---publishers, authors, readers, technologists, granting agencies, and the like. But in the interim, there are technological solutions that can be offered as a sort of ``band-aid'' to the problem. We use this paper to offer an in-depth qualitative and quantitative description of the problem as it stands, and to introduce one such technological solution: the \scially system that automatically extracts semantic information from paper PDFs and re-renders this content in the form of an accessible HTML document. Though the process is imperfect and can introduce errors, we demonstrate the ability of the rendered HTMLs to reduce cognitive load and facilitate in-paper navigation and interactions for BLV users. 

The goals and contributions of this paper are three-fold:

\begin{enumerate}
    \item We characterize the state of academic-paper PDF accessibility by estimating the degree of adherence to accessibility criteria for papers published in the last decade (2010--2019), and describe correlations between year, field of study, PDF typesetting software, and PDF accessibility.
    \item We propose an automated approach for extracting the content of academic PDFs and displaying this content in a more accessible HTML document format. We build a prototype that re-renders 12 million PDFs in HTML, and describe the design decisions, features, and quality of the renders (assessed as faithfulness to the source PDF). We perform expert grading of the rendered HTML and report an error analysis. A demo of our system is available at \href{https://scia11y.org/}{scia11y.org}, which makes available 1.5M HTML renders of open access PDFs.
    \item We conduct an exploratory user study with \numusers BLV scholars to better understand the challenges they experience when reading academic papers and how our proposed tool might augment their current workflow. During the study, we ask users to interact with the prototype and offer feedback for its improvement. We perform open coding of interviews to identify existing reading challenges, coping mechanisms, as well as positive and negative responses to prototype features. We summarize the findings of this user study into a set of design recommendations.
\end{enumerate}

Our analysis reveals that PDF accessibility adherence is low across all fields of study. Of the five accessibility criteria we assess, only \percaccessible of the PDFs we assess demonstrate full compliance. Though compliance for several criteria seems to be increasing over time, author awareness and contribution to accessibility remains low, as Alt-text has the lowest compliance of the five criteria at between 5--10\% (Alt-text is the only criterion of the five that \textit{requires} author intervention in all cases using current tools). We also find that typesetting software is strongly associated with accessibility compliance, with LaTeX and publishing software like Arbortext APP producing low compliance PDFs, while Microsoft Word is generally associated with higher compliance.


\begin{figure}[t!]
    \centering
    \includegraphics[width=\textwidth]{figures/pipeline.png}
    \caption{A schematic for creating the \scially HTML render from a paper PDF. Starting with the raw two-column PDF on the left, S2ORC \citep{lo-wang-2020-s2orc} is used to extract title, authors, abstract, section headers, body text, and references. S2ORC also identifies links between inline citations and references to figures and table objects. DeepFigures \citep{Siegel2018ExtractingSF} is used to extract figures and tables, along with their captions. The output of these two models are merged with metadata from the Semantic Scholar API. Heuristics are used to construct a table of contents, to insert figures and tables in the appropriate places in the text, and to repair broken URLs. We add HTML headers as illustrated (header tags for sections, paragraph tags for body text, and figure tags for figures and tables); highlighted components (table of contents and links in references) are not in the PDF and novel navigational features that we introduce to the HTML render. An example HTML render of parts of a paper document is show to the right (actual render is single column, which is split here for presentation).}
    \label{fig:pipeline}
    \Description{A schematic diagram showing the components of the SciA11y pipeline. An image of a paper PDF is on the left. Red boxes on the PDF image highlight the text components from the paper, with an arrow pointing to a box that says "S2ORC extracts: title, authors, abstract, section headers, body paragraphs, and references." A blue box on the PDF image highlights a figure, with an arrow pointing to a box that says "DeepFigures extracts: figures, figure captions, tables, and table titles/captions." A box below "S2ORC extracts" and "DeepFigures extracts" says "Additional content: metadata from Semantic Scholar API, table of contents, figures and tables inserted at first mention, and links between references and text." Arrows from all three boxes point into a larger box that describes the SciA11y prototype, where HTML tags are inserted around various blocks of text extracted from the PDF. On the right of all this is a screen capture of an example HTML render, showing how the semantic content from the PDF is represented as a single-column HTML page for easy reading.}
\end{figure}

To offset the reading challenges of inaccessible papers for BLV researchers, we propose and test the \scially system for rendering academic PDFs into accessible HTML documents. As shown in Figure~\ref{fig:pipeline}, our prototype integrates several machine learning text and vision models to extract the structure and semantic content of papers. The content is represented as an HTML document with headings and links for navigation, figures and tables, as well as other novel features to assist in document structure understanding. Our evaluation of the \scially system identifies common classes of extraction problems, and finds that though many papers exhibit some extraction errors, the majority (55\%) have no major problems that impact readability, and another 32\% have only some problems that impact readability.

Through our user study, we identify numerous challenges faced by BLV users when reading paper PDFs, including some that affect the whole document or limit navigation, and many that affect the ability of the reader to understand text or various elements of a paper like math content or tables. Responses to \scially were positive; participants especially liked navigation features such as headings, the table of contents, and bidirectional links between inline citations and references. Of the extraction errors in \scially, missed or incorrectly extracted headings were the most problematic, as these impact the user's ability to navigate between sections and fully trust the system. All users reported being likely to use the system in the future. When asked how the system might be integrated into their workflow, one participant replied ``I think it would become the workflow.'' Another participant said, ``for unaccessible PDFs, this is life-changing.'' We condense these findings into a set of recommendations for designing and engineering accessible reading systems (Section~\ref{sec:designrecs}). Most importantly, documents should be structured to match a reader's mental model, objects should be properly tagged, and care should be taken to reduce the reader's cognitive load and increase trust in the system. Features that emulate the external memory that visual layout provides to sighted users can be especially beneficial.

This paper is organized as follows. Following a description of related work in Section \ref{sec:related_work}, we first provide a meta-scientific analysis of the current state of academic PDF accessibility in Section \ref{sec:sos}. In Section \ref{sec:pdf2html}, we document our pipeline for converting PDF to HTML and describe the \scially prototype for rendering papers. An evaluation of HTML render quality and faithfulness is provided in Section \ref{sec:evaluation}. Section \ref{sec:user_study} describes our user study and findings. 
We recognize that no PDF extraction system is perfect, and many open research challenges remain in improving these systems. However, based on our findings, we believe \scially can dramatically improve screen reader navigation of most papers compared to PDFs, and is well-positioned to assist BLV researchers with many of their most common reading use cases. Our hope is that a system such as \scially can improve BLV researcher access to the content of academic papers, and that these design recommendations can be leveraged by others to create better, more faithful, and ultimately more usable tools and systems for scholars in the BLV community.

\section{Related Work}\label{related_work}
% \subsubsection{Multi-Turn Dialogue Understanding}
\textbf{Multi-Turn Dialogue Understanding}. Various tasks and corresponding benchmarks are proposed to evaluate the capacities of dialogue understanding models. Dialogue-based relation extraction (RE) is a classification task that assigns a pair of entities a relation label in a dialogue. Focusing on the word level,  \cite{xue2021gdpnet} constructed a multi-view graph with words in the dialogue as nodes and proposed Dynamic Time Warping Pooling to automatically select words in interest. SimpleRE \citep{SimpleRE} designed a novel input sequence format and utilized a Relation Refinement Gate to filter the semantic representation which is later fed into the classifier. TUCORE-GCN \citep{zahiri:18a} used a heterogeneous dialogue graph to encode the interaction between speakers, arguments, and turns across the dialogues.
% \citet{christopoulou-etal-2019-connecting} constructed an edge-oriented graph model to encode the dialogue as a graph with nodes and edges of different functions and applied an inference mechanism on the graph edges to recognize the internal relationship. By constructing a mention-level graph and an entity-level graph, \citet{zahiri:18a} reasoned the relation between entities by path inference.

Emotion Recognition in Conversation (ERC) has been extensively studied in the research community. It aims to attach an emotional label to every turn in a given dialogue. \cite{kratzwald2018deep} customized the recurrent neural network with bidirectional processing to solve the problem of emotion classification. \cite{majumder2019dialoguernn} leveraged the Recurrent Neural Network to extract the information of the party states and use it to predict the emotion in conversations with two speakers. On top of the recurrent neural network, COSMIC \cite{ghosal2020cosmic} models the commonsense knowledge, mental states, events, and actions to enhance emotion detection in dialogue. 
%Towards solving the context propagation problems in the recurrent neural network-based methods, \citet{ghosal-etal-2019-dialoguegcn} proposed a graph-based method that models the utterances as nodes and the speakers' dependency as edges. 

Deep learning-based methods have been extensively studied in recent works \citep{lee-dernoncourt-2016-sequential, chen2018dialogue, raheja2019dialogue} regarding Dialogue Act classification (DAC). \cite{chen2018dialogue} introduced a relation layer into the shared hierarchical encoder to model the interaction between the tasks of dialog act recognition and sentiment classification.
%Combining recurrent neural networks and convolutional neural networks, \citet{lee-dernoncourt-2016-sequential} incorporated the preceding texts while classifying the act.


% \subsubsection{Context-Aware Representation Learning}
\textbf{Context-Aware Representation Learning}. To address dynamics and semantic changes in multi-turn dialogue, previous works extend pre-trained large language models to learn context-aware representations for turns \citep{lee2021graph, DialogXL, DCM, chapuis2020hierarchical}. TUCORE-GCN \citep{lee2021graph} proposes the turn attention module, masking out distant turns to learn the contextual embeddings. Instead of adding extra modules, DialogXL \citep{DialogXL} targets the encoder and incorporates four self-attention mechanisms to different attention heads to capture diverse dialog-aware information. Similarly, such dialogue-oriented self-attention can also be found in MDFN \citep{MDFN} where it is defined as utterance-aware and speaker-aware channels. However, most of them involve an additional pre-training stage \citep{DialogXL, DCM, chapuis2020hierarchical}. 
%These efforts focus on the turn-level modeling but ignored the gap between the pre-training objective and dialogue understanding tasks. Also, though achieving preferable performance on limited datasets or tasks, these methods have not led to a unified solution to multi-turn dialogue understanding. 
\section{Method}
Fig.~\ref{fig:framework} presents the illustration of the proposed \frameworkName.
In this section,  
we start by providing the problem definition of online CIL. Then, we describe the definition of the online prototype, the proposed online prototype equilibrium, and the proposed adaptive prototypical feedback. Finally, we propose an online prototype learning framework.

\subsection{Problem Definition}
Formally, online CIL considers a continuous sequence of tasks from a single-pass data stream $\mathfrak{D}=\left\{\mathcal{D}_1, \ldots, \mathcal{D}_T \right\} $, where $\mathcal{D}_t = \left\{ x_{i}, y_{i} \right\} ^{N_t}_{i=1} $ is the dataset of task $t$, and $T$ is the total number of tasks. Dataset $\mathcal{D}_t$ contains $N_t$ labeled samples, $y_{i}$ is the class label of sample $x_{i}$ and $y_{i} \in \mathcal{C}_t$, where $\mathcal{C}_t$ is the class set of task $t$ and the class sets of different tasks are disjoint. 
For replay-based methods, a memory bank is used to store a small subset of seen data, and we also maintain a memory bank $\mathcal{M}$ in our method.
At each time step of task $t$, the model receives a mini-batch data $X \cup X^\mathrm{b}$ for training, where $X$ and $X^\mathrm{b}$ are drawn from the i.i.d distribution $\mathcal{D}_t$ and the memory bank $\mathcal{M}$, respectively. 
Moreover, we adopt the single-head evaluation setup~\cite{EWC}, where a unified classifier must choose labels from all seen classes at inference due to unavailable task identifiers. 
The goal of online CIL is to train a unified model on data seen only once while predicting well on both new and old classes.

\subsection{Online Prototype Definition}
Prior to introducing the online prototypes, we first present the network architecture of our \frameworkName. Suppose that the model consists of three components: an encoder network $f$, a projection head $g$, and a classifier $\varphi$. Each sample $x$ in incoming data $X$ (a mini-batch data from new classes) is mapped to a projected vectorial embedding (representation) $\mathbf{z}$ by encoder $f$ and projector $g$:
\begin{align}
\label{eq:cal_z}
    \mathbf{z} = g(f(\operatorname{aug}(x);\theta_f);\theta_g),
\end{align}
where $\operatorname{aug}$ represents the data augmentation operation, $\theta_f$ and $\theta_g$ represent the parameters of $f$ and $g$, respectively, and $\mathbf{z}$ is $\ell_2$-normalized. 
Similar to Eq.~\eqref{eq:cal_z}, we use $\mathbf{z}^\mathrm{b}$ to denote the representation of replay data $X^\mathrm{b}$ (a mini-batch data from seen classes in the memory bank). 

At each time step of task $t$, the online prototype of each class is defined as the mean representation in a mini-batch:
\begin{align}
\label{eq:cal_p}
    \mathbf{p}_i = \frac{1}{n_i}\sum\nolimits_j\mathbf{z}_j\cdot \mathbbm{1}\{y_j = i\},
\end{align}
where $n_i$ is the number of samples for class $i$ in a mini-batch, and $\mathbbm{1}$ is the indicator function. 
We can get a set of $K$ online prototypes  in $X$, $\mathcal{P} = \left\{ \mathbf{p}_{i} \right\} ^{K}_{i=1}$, and a set of $K^\mathrm{b}$ online prototypes in $X^\mathrm{b}$, $\mathcal{P}^\mathrm{b} = \left\{ \mathbf{p}_i^\mathrm{b} \right\} ^{K^\mathrm{b}}_{i=1}$.
Note that $K = |\mathcal{P}| \leq |\mathcal{C}_t|$ and $K^\mathrm{b} = |\mathcal{P}^\mathrm{b}| \leq \sum_{i=1}^{t}|\mathcal{C}_i| $, where $|\cdot|$ denotes the cardinal number.



\subsection{Online Prototype Equilibrium}
The introduced online prototypes can provide representative features and avoid class-unrelated information.  
These characteristics are exactly the key to counteracting shortcut learning in online CL.
Besides, maintaining the discrimination among seen classes is also essential to mitigate catastrophic forgetting.
Based on these, we attempt to learn representative features of each class by pulling online prototypes $\mathcal{P}$ and their augmented views $\widehat{\mathcal{P}}$ closer in the embedding space, and learn discriminative features between classes by pushing online prototypes of different classes away, formally defined as a contrastive loss:
\begin{align}
\label{eq:proto_infoNCE}
    \ell(\mathcal{P},\widehat{\mathcal{P}})\!=\!
    % \frac{-1}{K}
    \frac{-1}{|\mathcal{P}|}\sum_{i=1}^{|\mathcal{P}|}\!\log\! 
    \tfrac
    {\exp \big(\tfrac{{\mathbf{p}_i^\mathrm{T} \widehat{\mathbf{p}}_i}}{\tau}\big)}
    {
    \sum\limits_{j} \exp \big(\tfrac{{\mathbf{p}_i^\mathrm{T} \widehat{\mathbf{p}}_j}}{\tau}\big)
    +\!
    \sum\limits_{\substack{j \neq i}} \exp \big(\tfrac{{\mathbf{p}_i^\mathrm{T} \mathbf{p}_j}}{\tau}\big) 
    },
\end{align}
where 
$\tau$ is the temperature hyper-parameter, 
$\mathcal{P}$ and $\widehat{\mathcal{P}}$ are $\ell_2$-normalized. To compute the contrastive loss across all positive pairs in both $(\mathcal{P}, \widehat{\mathcal{P}})$ and $(\widehat{\mathcal{P}}, \mathcal{P})$, we define $\mathcal{L}_{\mathrm{pro}}$ as the final contrastive loss over online prototypes:
\begin{align}
    \mathcal{L}_{\mathrm{pro}}(\mathcal{P},\widehat{\mathcal{P}}) = 
    \frac{1}{2}
    \left[\ell(\mathcal{P}, \widehat{\mathcal{P}}) + \ell(\widehat{\mathcal{P}}, \mathcal{P})\right].
\end{align}



Considering the learning of new classes and the consolidation of learned knowledge simultaneously in online CL, we propose Online Prototype Equilibrium (\methodname) to 
learn representative and discriminative features on both new and seen classes by employing $\mathcal{L}_{\mathrm{pro}}$:
\begin{equation}
    \begin{aligned}
    \mathcal{L}_{\mathrm{\methodname}}
    &=
    \mathcal{L}^{\mathrm{new}}_{\mathrm{pro}}(\mathcal{P},\widehat{\mathcal{P}}) + \mathcal{L}^{\mathrm{seen}}_{\mathrm{pro}}(\mathcal{P}^\mathrm{b},\widehat{\mathcal{P}}^\mathrm{b}),
    \end{aligned}
\end{equation}
where
$\mathcal{L}^{\mathrm{new}}_{\mathrm{pro}}$ focuses on learning knowledge from \emph{new} classes, and $\mathcal{L}^{\mathrm{seen}}_{\mathrm{pro}}$ is dedicated to preserving learned knowledge of all \emph{seen} classes.
\textit{This process is similar to a zero-sum game, 
and \methodname aims to achieve the equilibrium to play a win-win game.}
Concretely,
as the model learns, the knowledge of new classes is gained and added to the prototypes over the memory bank $\mathcal{M}$, causing $\mathcal{L}^{\mathrm{seen}}_{\mathrm{pro}}$ gradually changes to the equilibrium that separates all seen classes well, including new ones. 
This variation is crucial to mitigate forgetting and is consistent with the goal of CIL.



\subsection{Adaptive Prototypical Feedback} 
Although \methodname can bring an overall equilibrium, it tends to treat each class \emph{equally}. 
In fact, the degree of confusion varies among classes, 
and the model should focus purposefully on confused classes to consolidate learned knowledge. 
To this end, we propose Adaptive Prototypical Feedback (\dataaugname) with the feedback of online prototypes to sense the classes that are prone to be misclassified and then enhance their decision boundaries.
 
For each class pair in the memory bank $\mathcal{M}$, \dataaugname calculates the distances between online prototypes of all seen classes from the previous time step, showing the class confusion status by these distances. The closer the two prototypes are, the easier to be misclassified. 
Based on this analysis, 
our idea is to enhance the boundaries for those classes. Therefore, we convert the prototype distance matrix to a probability distribution $P$ over the classes via a symmetric Gaussian kernel, defined as follows:
\begin{align}
\label{eq:cal_d}
    P_{i, j} \propto \exp (-\| \mathbf{p}_i^\mathrm{b} - \mathbf{p}_j^\mathrm{b} \|_2^2),
\end{align}
where $i,j \in \{ 1, \ldots, |\mathcal{P}^\mathrm{b}| \}$ and $i \neq j$. 
Then, 
all probabilities are normalized to a probability mass function that sums to one.
\dataaugname returns probabilities to $\mathcal{M}$ for guiding the next sampling process and enhancing decision boundaries of easily misclassified classes. 


Our adaptive prototypical feedback is implemented as a sampling-based mixup. Specifically, 
\dataaugname adaptively selects more samples from easily misclassified classes in $\mathcal{M}$ for mixup~\cite{Mixup} according to the probability distribution $P$. 
Considering not over-penalizing the equilibrium of current online prototypes, we introduce a two-stage sampling strategy for replay data $X^\mathrm{b}$ of size $m$. 
First, we select $n_{\mathrm{\dataaugname}}$ samples  
with $P$, and a larger $P_{a,b}$ means more sampling from classes $a$ and $b$. Here, $n_{\mathrm{\dataaugname}} = \alpha \cdot m$, and $\alpha$ is the ratio of \dataaugname.
Second, the remaining $m-n_{\mathrm{\dataaugname}}$ samples are uniformly randomly selected from the entire memory bank to avoid the model only focusing on easily misclassified classes and disrupting the established equilibrium. 




\subsection{Overall Framework of \frameworkName}
The overall structure of \frameworkName is shown in Fig.~\ref{fig:framework}. \frameworkName comprises two key components based on proposed online prototypes: Online Prototype Equilibrium (\methodname) and Adaptive Prototypical Feedback (\dataaugname). 
With the two components, 
the model can learn representative features against shortcut learning, and 
all seen classes maintain discriminative when learning new classes. 
However, classes may not be compact, because the online prototypes cannot cover full instance-level information.
To further achieve intra-class compactness, 
we employ supervised contrastive learning~\cite{SupCL} to learn instance-wise representations:
\begin{equation}
\begin{aligned}
    \mathcal{L}_{\mathrm{INS}}
    &=
    \sum_{i=1}^{2N} \frac{-1}{\left|I_i\right|} \sum_{j \in I_i} \log \frac{\exp \left(\mathrm{sim}(\mathbf{z}_i, \mathbf{z}_j) / \tau^{\prime}\right)}{\sum\limits_{k \neq i} \exp \left(\mathrm{sim}(\mathbf{z}_i, \mathbf{z}_k) / \tau^{\prime}\right)}
    \\
    &+
    \sum_{i=1}^{2m} \frac{-1}{\left|I_i^{\mathrm{b}}\right|} \sum_{j \in I_i^{\mathrm{b}}} \log \frac{\exp (\mathrm{sim}(\mathbf{z}_i^{\mathrm{b}}, \mathbf{z}_j^{\mathrm{b}}) / \tau^{\prime})}{\sum\limits_{k \neq i} \exp \left(\mathrm{sim}(\mathbf{z}_i^{\mathrm{b}}, \mathbf{z}_k^{\mathrm{b}}) / \tau^{\prime}\right)},
\end{aligned}
\end{equation}
where $I_i=\left\{j \in\{1, \ldots, 2 N\} \mid j \neq i, y_j=y_i\right\}$ and $I_i^\mathrm{{b}}=\left\{j \in\{1, \ldots, 2m\} \mid j \neq i, y_j^\mathrm{b}=y_i^\mathrm{b}\right\}$ are the set of positive samples indexes to $\mathbf{z}_i$ and $\mathbf{z}_i^\mathrm{{b}}$, respectively. $y_i^\mathrm{b}$ is the class label of input $x_i^\mathrm{b}$ from $X^\mathrm{b}$. $N$ is the batch size of $X$. $\tau^{\prime}$ is the temperature hyperparameter.
The similarity function $\mathrm{sim}$ is computed in the same way as Eq.~(9) in OCM~\cite{OCM}.

Thus, the total loss of our \frameworkName framework is given as:
\begin{align}
    \mathcal{L}_{\mathrm{\frameworkName}}=\mathcal{L}_{\mathrm{\methodname}} + \mathcal{L}_{\mathrm{INS}} + \mathcal{L}_{\mathrm{CE}},
\end{align}
where $\mathcal{L}_{\mathrm{CE}} = \mathrm{CE}(y^\mathrm{b}, \varphi(f(\operatorname{aug}(x^\mathrm{b}))))$ is the cross-entropy loss; see Appendix~\ref{appendix:algorithm} for detailed training algorithms.

Following other replay-based methods~\cite{ER, SCR, OCM}, we update the memory bank in each time step by uniformly randomly selecting samples from $X$ to push into $\mathcal{M}$ and, if $\mathcal{M}$ is full, pulling an equal number of samples out of $\mathcal{M}$.


\section{Experiments}
\subsection{Datasets and Metrics}

%dialogue task分四种, intent prediction, slot-filling, semantic parsing, and dialogue state tracking. 然后我们完成的数据集任务是relation extraction, emotional recognition, speech act classification,不太确定是不是都能解决

\textbf{DialogRE} \citep{yu-etal-2020-dialogue} is a relation extraction task based on 1,788 dialogues from the Friends transcript. Each pair of arguments can be classified as one of 36 possible relation types. For each of the 10,168 human-annotated entity pairs, the trigger words are also provided. % such as neighbours or siblings. (subject, object, relation type) triplets

\textbf{EmoryNLP} \citep{zahiri:18a} is an emotion detection task based on 12,606 utterances from the Friends transcript. Each utterance can be classified as one of seven emotions, e.g., joyful, scared. 
%The label is human-annotated based on the dialogue context. 

\textbf{DailyDialog} \citep{DailyDialog} is a dialogue database containing 13,118 simple English dialogues. Each utterance can be assigned an emotion label from seven categories (anger, surprise, etc.). 
%targets both emotion detection and act classification. DailyDialog The label is human-annotated based on the dialogue context.  (Inform, Questions, Directives, Commissive) 

\textbf{MELD} \citep{poria-etal-2019-meld} is an emotion detection task based on 13,000 sentences from the Friends transcript. Each utterance can be classified as one of eight emotions, such as sad, disgust. % or neutral. 

%The label is human-annotated based on the multimodal cues of the dialogue, including visual, audio and textual information. 
\begin{table}[t]
  % \setlength{\tabcolsep}{3.5pt}
% \footnotesize
% \scriptsize
  \centering
  \vspace{-0.35cm}
  \begin{tabular}{lllllll}
    \toprule
  \multirow{2}{*}{\textbf{Method}} &\multirow{2}{*}{\textbf{MELD}} & \multirow{2}{*}{\textbf{ENLP}} & \multirow{2}{*}{\textbf{DDialog}} & \multirow{2}{*}{\textbf{MRDA}} & \multicolumn{2}{c}{\textbf{DialogRE-Test}}      \\
    % \textbf{Method} &\textbf{MELD} & \textbf{ENLP} & \textbf{DDialog} &\textbf{MRDA} & \multicolumn{2}{c}{\textbf{DialogRE}}      \\
    %\cmidrule(r){2-5}
    \cmidrule(l){6-7}   
    % \cmidrule(l){2-2}  \cmidrule(l){3-3} \cmidrule(l){4-4} \cmidrule(l){5-5}  \cmidrule(l){6-7}  
      &  & & & & $F1$  & $F1_c$  \\
    \midrule
    % BERT &61.50	& 34.17	& 54.09& 91.0 & 58.5	& 53.2\\
    % +HiDialog & + 1.78	& +0.63 & +5.55 & +0.3 & + &\textbf{59.8}\\
    PHT  &61.90&-&	60.14&	\textbf{92.4}&- &- \\
    DialogXL  & 62.41 & 34.73 & 54.93 & -& - & -  \\
    % \midrule
    RoBERTa$_s$& 64.19&38.03	&61.65	&91.3	&71.3 & 63.7\\
+Intra-turn &\textbf{65.64}\textsubscript{\textcolor{green}{+1.45}}& \textbf{38.13}\textsubscript{\textcolor{green}{+0.1}} &\textbf{61.83}\textsubscript{\textcolor{green}{+0.28}}&91.5\textsubscript{\textcolor{green}{+0.2}} & \textbf{74.4}\textsubscript{\textcolor{green}{+3.1}}&\textbf{66.6}\textsubscript{\textcolor{green}{+2.9}}\\
    \bottomrule
  \end{tabular}
  % \vspace{-0.2cm}
  % \caption{All methods performance on 5 multi-turn dialogue-based understanding datasets: MELD, EmoryNLP (Weighted-F1), DailyDialog (Micro-F1), MRDA (Top-1 Acc.), DialogRE (F1 and F1$_c$), averaged over five runs. Performance gains over the RoBERTa$_s$ are highlighted in green.}
    \caption{All methods performance on 5 multi-turn dialogue-based understanding datasets: MELD, EmoryNLP, DailyDialog, MRDA, DialogRE, averaged over five runs. Performance gains over the RoBERTa$_s$ are highlighted in green.}
  \label{tab:exp-mtr}
  \vspace{-0.5cm}
\end{table}


\begin{minipage}[]{0.48\linewidth}
\footnotesize
\setlength{\tabcolsep}{7pt}
\begin{center}
\begin{tabular}{l c c} 
 \toprule
  {\textbf{Method}} & \textbf{F1} & \textbf{F1$_c$} \\
 \midrule
HiDialog                      & 77.1        & 68.2       \\ 
w/o attention mask & 76.5 (-0.6) & 67.9 (-0.3) \\
w/o special tokens & 75.6 (-1.5) & 67.4 (-0.8) \\
only intra-turn     & 74.4 (-2.7) & 66.6 (-1.6) \\
\bottomrule
\end{tabular}
\end{center}
\captionof{table}{Ablation Study on HiDialog components on DialogRE to evaluate the individual effect of turn-level attention, turn-level special tokens, and graph module. } %\textit{Ablation study. Turn-level} is omitted for brevity.}
\label{tab:ablation_structure}
\end{minipage}
\hspace*{0.1cm}
\begin{minipage}[]{0.48\linewidth}
\footnotesize
\setlength{\tabcolsep}{7pt}
\begin{center}
% \vspace{-0.5cm}
\begin{tabular}{lccc} 
\toprule
\textbf{Method} & \textbf{I} &  \textbf{II} &  \textbf{III}\\
\midrule
BERT & 42.5  & 60.7 & 65.6 \\
% BERT$_s$ & 46.5 & 61.5 & 69.4 \\
GDPNet  & 47.4  &59.8  &68.1 \\
RoBERTa$_s$ & 57.4 & 69.3 & 79.6 \\
TUCORE-GCN  &62.3  &\textbf{71.3}  &79.9 \\
\midrule
HiDialog & \textbf{76.6}  & 70.5	& \textbf{80.9}  \\
w/o graph module &65.5 & 69.9 & 79.4\\
\bottomrule
\end{tabular}
\end{center}
% \vspace{-0.3cm}
\captionof{table}{All methods performance on DialogRE. We break down the performance into three groups (I) asymmetric inverse relations, (II) symmetric inverse relations, and (III) others.}
\label{tab:symmetric}
\vspace{0.5cm}
\end{minipage}

\textbf{MRDA} \citep{MRDA} is a dialogue act task based on 75 hours of real-life meeting transcript. Each sentence is assigned a general dialogue act (topic change, repeat, etc.) and a specific dialogue act (apology, suggestion, etc.).  % explanation sympathy

\textbf{Metrics}. For DialogRE, F1 and F1$_c$ are used as evaluation metrics. F1$_c$ modifies F1 by taking an early part of the dialogue as input \cite{yu-etal-2020-dialogue}. For MELD and EmoryNLP, we use weighted-F1 as metrics. For DailyDialog, the Micro-F1 score excluding the neutral class is used as the metric. 
\subsection{Results and Analysis}
\textbf{Overall Performance}. We first evaluated HiDialog on the Dialogue Relation Extract (DRE) dataset, DialogRE \citep{yu-etal-2020-dialogue} and the Emotion Recognition in Conversation (ERC) dataset, MELD \citep{poria-etal-2019-meld}. We selected BERT \citep{bertbase}, GDPNet \citep{xue2021gdpnet}, RoBERTa$_s$ \citep{yu-etal-2020-dialogue}, SimpleRE \citep{SimpleRE}, and TUCORE-GCN \citep{lee2021graph} as baselines. As reported in Table \ref{tab:exp-re}, HiDialog established new state-of-the-art results on both datasets. 
On the DialogRE test set, HiDialog surpassed the previous SOTA, TUCORE-GCN, by 4\% in F1 and 2.3\% in F1$_c$. On the MELD dataset, HiDialog outperformed TUCORE-GCN by 1.5\% in weighted F1. 
%, surpassing the previous SOTA by 4\% in F1 and 2.3\% in F1$_c$ on the DialogRE test set, by 1.5\% in weighted F1 score on the MELD test set. 


\textbf{Towards Generality}.
% In view of the simplicity and effectiveness of the intra-turn modeling, it is expected to have a general use for further work on dialogue understanding. To validate this idea, we incorporate our intra-turn modeling into the baseline encoder, without any extra components such as the inter-turn module or speaker embeddings. For a fair comparison, only the global $[CLS]$ token embedding from the encoder output is fed into a softmax classifier to make a prediction. 
Our intra-turn modeling's simplicity suggests its potential as a valuable solution for enhancing dialogue understanding without the need for extra pre-training. To assess this claim, we integrated it into the baseline encoder without any additional components, such as an inter-turn module or speaker embeddings. For fair comparisons, only the encoder's global $[CLS]$ token was used in a softmax classifier for prediction.
%Our intra-turn modeling's simplicity and effectiveness suggest its potential as a valuable solution for enhancing dialogue understanding, without additional pre-training. 

% \begin{wraptable}{r}{8.2cm}
%   \setlength{\tabcolsep}{3.5pt}
% % \footnotesize
% \scriptsize
%   \centering
%   \vspace{-0.35cm}
%   \begin{tabular}{lllllll}
%     \toprule
%   \multirow{2}{*}{\textbf{Method}} &\multirow{2}{*}{\textbf{MELD}} & \multirow{2}{*}{\textbf{ENLP}} & \multirow{2}{*}{\textbf{DDialog}} & \multirow{2}{*}{\textbf{MRDA}} & \multicolumn{2}{c}{\textbf{DialogRE}}      \\
%     % \textbf{Method} &\textbf{MELD} & \textbf{ENLP} & \textbf{DDialog} &\textbf{MRDA} & \multicolumn{2}{c}{\textbf{DialogRE}}      \\
%     %\cmidrule(r){2-5}
%     \cmidrule(l){6-7}   
%     % \cmidrule(l){2-2}  \cmidrule(l){3-3} \cmidrule(l){4-4} \cmidrule(l){5-5}  \cmidrule(l){6-7}  
%       &  & & & & $F1$  & $F1_c$  \\
%     \midrule
%     % BERT &61.50	& 34.17	& 54.09& 91.0 & 58.5	& 53.2\\
%     % +HiDialog & + 1.78	& +0.63 & +5.55 & +0.3 & + &\textbf{59.8}\\
%     PHT  &61.90&-&	60.14&	\textbf{92.4}&- &- \\
%     DialogXL  & 62.41 & 34.73 & 54.93 & -& - & -  \\
%     % \midrule
%     RoBERTa$_s$& 64.19&38.03	&61.65	&91.3	&71.3 & 63.7\\
% +Intra-turn &\textbf{65.64}\textsubscript{\textcolor{green}{+1.45}}& \textbf{38.13}\textsubscript{\textcolor{green}{+0.1}} &\textbf{61.83}\textsubscript{\textcolor{green}{+0.28}}&91.5\textsubscript{\textcolor{green}{+0.2}} & \textbf{74.4}\textsubscript{\textcolor{green}{+3.1}}&\textbf{66.6}\textsubscript{\textcolor{green}{+2.9}}\\
%     \bottomrule
%   \end{tabular}
%   \vspace{-0.2cm}
%   % \caption{All methods performance on 5 multi-turn dialogue-based understanding datasets: MELD, EmoryNLP (Weighted-F1), DailyDialog (Micro-F1), MRDA (Top-1 Acc.), DialogRE (F1 and F1$_c$), averaged over five runs. Performance gains over the RoBERTa$_s$ are highlighted in green.}
%     \caption{All methods performance on 5 multi-turn dialogue-based understanding datasets: MELD, EmoryNLP, DailyDialog, MRDA, DialogRE, averaged over five runs. Performance gains over the RoBERTa$_s$ are highlighted in green.}
%   \label{tab:exp-mtr}
%   \vspace{-0.35cm}
% \end{wraptable}



% \begin{minipage}[]{0.5\linewidth}
% % \begin{figure}[h]
% % \vspace{-0.2cm}
% \footnotesize
% \centering
% \includegraphics[width=7cm]{sections/length.pdf}
% \vspace{-0.5cm}
% \captionof{figure}{Analysis of robustness of HiDialog tackling increasing utterance length compared to baseline TUCORE-GCN on DialogRE dataset.}
% \vspace{0.6cm}
% \label{fig:length}
% % \end{figure}
% \end{minipage}



We conducted the experiment on 5 datasets from 3 different tasks: DRE (DialogRE), ERC (MELD, EmoryNLP \citep{zahiri:18a}, DailyDialog \citep{DailyDialog}), and Dialogue Act Classification (MRDA \citep{MRDA}). We chose RoBERTa$_s$, Pretrained Hierarchical Transformer (PHT) \citep{chapuis2020hierarchical}, and DialogXL \citep{DialogXL} as baselines. Compared to PHT and DialogXL, both of which require additional pre-training to address the domain adaption gap, the performance of proposed intra-turn modeling is surprisingly good in all 5 datasets (Table \ref{tab:exp-mtr}). 

% Moreover, we conducted an ablation study and analysis, which further reveals HiDialog is good at handling asymmetric relations and robust against increasing utterance length (see Appendix \ref{ap:ablation}). 

% \begin{wraptable}{r}{7cm}
% % \scriptsize
% % \vspace{-0.5cm}
% % \setlength{\tabcolsep}{3.5pt}
% \begin{center}

% \begin{tabular}{l c c} 
%  \toprule
%   {\textbf{Method}} & \textbf{F1} & \textbf{F1$_c$} \\
%  \midrule
% HiDialog                      & 77.1        & 68.2       \\ 
% w/o attention mask & 76.5 (-0.6) & 67.9 (-0.3) \\
% w/o special tokens & 75.6 (-1.5) & 67.4 (-0.8) \\
% only intra-turn     & 74.4 (-2.7) & 66.6 (-1.6) \\
% \bottomrule
% \end{tabular}
% \end{center}
% % \vspace{-0.2cm}
% \caption{Ablation Study on HiDialog components on DialogRE to evaluate the individual effect of turn-level attention, turn-level special tokens, and graph module. } %\textit{Ablation study. Turn-level} is omitted for brevity.}
% \label{tab:ablation_structure}
% % \vspace{-0.2cm}
% \end{wraptable}



\textbf{Ablation study on components.} We conducted an ablation study on DialogRE to evaluate key components in HiDialog: turn-level attention, turn-level special tokens, and inter-turn module (Table \ref{tab:ablation_structure}). First, after we removed the turn-level attention mask, the performance slightly dropped. In this case, these special tokens are able to aggregate information from the entire sequence, thus they are not context-aware at the turn level. We experimented with removing intra-turn modeling, resulting in only one difference from the final HiDialog: here we used an average of corresponding token embeddings for initialization. The $F1$ score decreases by 1.5\% and the $F1_c$ score declines by 0.8\%.

\textbf{Analysis of relations}. We grouped the test set of DialogRE according to the relation types into three subsets: (I) asymmetric, when a relation type differs from its inversion (e.g. \textit{children} and \textit{parents});  (II) symmetric, when a relation type is the same as its inversion (e.g. \textit{spouse}); (III) other, when a relation type does not have inversion (e.g. \textit{age}). We compared the performance of our model with baselines and report the results in Table \ref{tab:symmetric}. As we can observe, there is a great performance increase in the asymmetric subset while the F1 score drops moderately for symmetric relations. This trend reverses when we remove the graph module in our method (i.e. symmetric $>$ asymmetric). 
% \clearpage

\textbf{Analysis of robustness against increasing utterance length.} With the hierarchical aggregation in HiDialog, each turn-level special token is enforced to capture intra-turn critical information regardless of the whole dialogue. This nature enables our method to handle dialogues of various lengths. Thus, we further divided the samples in the DialogRE test set into six groups according to their lengths and compared HiDialog against the previous SOTA, TUCORE-GCN. As shown in Figure \ref{fig:length}, our method consistently outperforms TUCORE-GCN in all groups, where the largest performance gap can be found in the group with less than 100 tokens. Moreover, TUCORE-GCN shows a great drop with an increase of length (i.e., from $[400,500)$ to $[500,+\infty)$), while HiDialog maintains decent performance for long sequences.

 \begin{figure}[t]
\centering
% \vspace{-0.2cm}
\footnotesize
\centering
\includegraphics[width=7cm]{sections/length.pdf}
% \vspace{-0.5cm}
\captionof{figure}{Analysis of robustness of HiDialog tackling increasing utterance length compared to baseline TUCORE-GCN on DialogRE dataset.}
% \vspace{0.6cm}
\label{fig:length}
\end{figure}


%Note that both PHT and DialogXL are pre-trained methods that require extra computation
%Moreover, it outperforms the current SOTA by 1.3\% in $F1$ and 0.7\% in $F1_c$ on the DialogRE dataset, by 0.3\% in weighted F1 on MELD.
%HiDialog performs well in managing asymmetric relations and handling longer utterances, as revealed in our ablation study (see Appendix \ref{ap:ablation}). It provides an effective solution for bridging the gap between pre-training on general corpora and dialogue understanding without additional computational costs or training data while maintaining good performance. 
%Considering that our turn-level attention is easy to adopt and does not introduce any parameters to the base encoder, we believe it can be used as a strong baseline or plug-in module for future work in the community.
%Our turn-level attention mechanism can be effortlessly integrated into the base encoder without introducing any new parameters. Thus, we anticipate that it will serve as a compelling baseline or plug-in module for upcoming research in this field.
In this work, we demonstrate that it's possible to distill huge models trained on large datasets to obtain much smaller models that perform well on paralinguistic speech tasks.
The distillation uses only \textbf{7\% of the training data} and is entirely from public sources. The models we obtain are between 22MB and 314MB, and achieve between \textbf{90\% and 96\% of the larger CAP12 accuracy on 6 of 7 tasks}. These models are between \textbf{1\% and 15\% the size} of the original model. We release the model to allow the research community to benefit from the practical applications of self-supervised representations for paralinguistic speech.
% \LX{TODO: fix citation; Main experiment part}

% \LX{@Fuzhao, can we put ablation and analysis in the appendix?}
% \FZ{Yes}

% \LX{@Fuzhao, should we include related work part and put it in the appendix?}
% \FZ{Yes}

% \FZ{Use ChatGPT to help your writing.}

% \FZ{You can see the template I sent to you. That would give you more space to place your figure and tables.}


% \clearpage
\section*{Contribution}
This work has been expanded upon from a course project that was undertaken by the key authors, Xiao Liu, Jian Zhang, and Heng Zhang, during their time as students at the National University of Singapore. Fuzhao Xue, the teaching assistant, and project mentor played a role in guiding and shaping the outcome of this work. We are grateful to Yang You for providing us with valuable instructions and computational resources. 

\section*{Acknowledgement}
Yang You's research group is being sponsored by NUS startup grant (Presidential Young Professorship), Singapore MOE Tier-1 grant, ByteDance grant, ARCTIC grant, SMI grant and Alibaba grant. Our sincere appreciation also goes out to Min-Yen Kan, our course lecturer, for his invaluable suggestions during our enrollment.

% This work was extended from a course project during key authors' (Xiao Liu, Jian Zhang, Heng Zhang) time as students at the National University of Singapore. Fuzhao Xue was the teaching assistant and projector mentor of this work.


\section*{URM Statement}
The authors acknowledge that the first author of this work meets the URM criteria of ICLR 2023 Tiny Papers Track.

\bibliography{iclr2023_conference_tinypaper}
\bibliographystyle{iclr2023_conference_tinypaper}

% % !TEX root = ../supp.tex
% !TEX spellcheck = en-US

\section{Physical Rendering of SwissCube}
\label{sec:appendix}

Although the European Space Agency has organized a satellite pose estimation challenge and released the SPEED satellite dataset, the unavailability of the target 3D model makes the pose accuracy not depending on the pose estimation method alone. Furthermore, the limited varieties of lighting also make it soon saturated and less discriminative, as discussed in Section~\ref{sec:related}.

To fully demonstrate the effectiveness of our method in space, we introduce the Swisscube satellite dataset. Swisscube is a Cubesat-type satellite which was designed at EPFL and launched in 2009. Given the accurate CAD files and material properties of each component of it, we synthesize photorealistic images using physically based rendering~\cite{xx}.

 
% !TEX root = ../top.tex
% !TEX spellcheck = en-US

\begin{table}
    \centering
    \begin{small}
    % \rowcolors{2}{white}{gray!10}
    \begin{tabular}{lcc}
        \toprule
        &	SPEED & {\bf CubeSat}\\
        \midrule
        % Synthetic  &12k & 40k \\
        % Real       & 5 & 300 \\
        % \midrule
        Size & 12k & 40k \\
        Accurate 3D model   &\xmark  & \cmark  \\
        Complex lighting    &\xmark  & \cmark \\
        Physical modeling    &\xmark  & \cmark \\
        Colors        &\xmark  & \cmark \\
        Sequences         &\xmark  & \cmark  \\
        Depth distribution & non-uniform & uniform \\
        \bottomrule
    \end{tabular}
    \end{small}
    % \vspace{-3mm}
    \caption{{\bf CubeSat dataset.} bla bla bla bla bla bla bla bla bla bla bla bla bla bla bla bla bla bla bla bla bla bla bla bla bla bla bla bla bla bla bla bla bla bla bla bla bla bla bla bla bla bla bla bla bla bla bla bla bla bla bla bla bla bla bla bla bla bla bla bla bla bla bla bla bla bla bla bla bla bla bla bla bla bla bla bla bla bla bla bla }
    \label{tab:swisscube_vs_speed}
\end{table} 

% -----------------------------------
% Physically-based spectral rendering

\subsubsection{Physically-based spectral rendering}

In this section we provide a high-level description of the Swisscube satellite dataset which collects 40'000  physically-based synthetic images. While the SPEED satellite dataset images were produced using an OpenGL-based RGB rendering framework, we opted for a physically-based approach, where every element of the rendering pipeline were carefully modeled to mimic reality.

While the RGB model is often used to render color images, using tristimulus RGB colors in the rendering simulation generally yields non-physical results. For instance, surface reflectance properties of an object can be highly dependent on the wavelengths, which won't be accurately reproduced with RGB values. In a spectral renderer, colors are represented as spectral power distributions, resulting in improved accuracy especially when measured spectral data is available. For the Swisscube dataset, using a spectral renderer was a necessity as it was a requirement to correctly model the spectral responses of the solar irradiance emission, material reflectance properties and Earth surface radiance. Although the rendering simulation uses spectral colors, the resulting images will be converted to RGB images.

Relying on a physically-based rendering pipeline also gives us more control on the dynamic range of the output images. Thus we were able to accurately reproduce highlights orders of magnitude brighter than darker region of the images. An appropriate gamma curve could then be applied to produce images that can be viewed on regular displays.

To achieve all of this, we build our pipeline around the Mitsuba 2 renderer [???] which is a highly modular open-source framework that supports spectral rendering.

% -------------------
% The 3D / CAD model

\subsubsection{Accurate 3D model from CAD data}

For this dataset, we modelled every mechanical parts of the SwissCube, such as solar panels, antennas, and screws based directly on the raw CAD files. We carefully assign material reflection properties to each part. Given the physically-based nature of the pipeline, it would be possible to use efficient material acquisition technique such as [???] in the future for better results. Due to confidential reasons, we only release the mesh geometry data of the combined SwissCube without separable pieces to the public, which is enough for perfect registration.

% Citation for material acquistion technique: https://rgl.epfl.ch/publications/Dupuy2018Adaptive

% --------------------
% Physically-based Sun

\subsubsection{Modeling a physically-based Sun emitter}

In order to correctly model the illumination from the Sun, we leveraged the vast literature in astrophysics. As the target object will be placed above the Earth atmosphere, it is not enough to use specular solar irradiance measurements made at ground surface [???] as those will be be affected by the highly variable and absorbing constituents of the Earth atmosphere. Instead, we rely on the air mass zero reference spectrum [???], also known as extraterrestrial solar irradiance, mainly based on data from satellites and space shuttle missions. Figure \ref{fig:swisscube_sun_spectrum} show its spectral power distribution. We then use a point light source to represent the Sun, placed at the correct distance to the Earth. Note that is was necessary to scale the Sun irradiance to account for its surface area.

% Groud surface citation: https://www.osapublishing.org/ao/abstract.cfm?uri=ao-21-3-554
% ASTM Standard Extraterrestrial Spectrum Reference: https://www.nrel.gov/grid/solar-resource/spectra-astm-e490.html

% !TEX root = ../top.tex
% !TEX spellcheck = en-US

\begin{figure}[t]
    \begin{center}
    \includegraphics[width=1.0\linewidth]{./fig/swisscube_sun_spectrum/sun_spectrum_plot.jpeg}
    % \fbox{\rule{0pt}{2in} \rule{0.25\linewidth}{0pt}}
    \end{center}
    \vspace{-6mm}
    \caption{{\bf Air mass zero solar spectral power distribution.} 
    }
    \label{fig:swisscube_sun_spectrum}
\end{figure}


% ---------------------------
% Modeling the stars / galaxy

\subsubsection{Modeling the stars and galaxies}

We also added galaxies and other astronomical objects to our pipeline as we believe those could distract the learning algorithm. Based on the HYG database star catalogue [???], we could generate a high-resolution environment map that we later used as an second emitter. The HYG database contains around 220 thousands astronomical objects, mostly galaxies but also star clusters and Nebulae along with information regarding their position and brightness. Figure \ref{fig:swisscube_stars_envmap} shows the astronomical object projected on a spherical coordinate 2D map with their respective physically-based brightness.

Compared to the sun illumination, the irradiance coming from the stars is orders of magnitude lower. On the other hand, to maximize the diversities of the generated data, we decided to increase the actual brightness of each star in the galaxies to make them more apparent in rendering, which we think is a beneficial perturbation for a dataset.

% !TEX root = ../top.tex
% !TEX spellcheck = en-US

\begin{figure}[t]
    \begin{center}
    \includegraphics[width=1.0\linewidth]{./fig/swisscube_stars_envmap/stars_envmap.jpeg}
    % \fbox{\rule{0pt}{2in} \rule{0.25\linewidth}{0pt}}
    \end{center}
    \vspace{-6mm}
    \caption{{\bf Astronomical objects environment map based on the HYG dataset.} 
    }
    \label{fig:swisscube_stars_envmap}
\end{figure}


% HYG database link: http://www.astronexus.com/hyg

% ---------------------------------------------------
% Modeling the Earth radiance using the VIIRS dataset

\subsubsection{Spectral Earth radiance using the VIIRS dataset}

Properly modeling the Earth is very important here as it often occupies a large portion of the images. Moreover, the Sun light reflecting off the Earth is drastically affecting the illumination of the target object. In our pipeline, the Earth is a represented as a very large sphere, reflecting light coming from the Sun emitter onto the target object or directly towards the camera. Based on the NASA Visible Infrared Imaging Radiometer Suite (VIIRS) Level-1B data products [????], we generated a spectral radiance texture to model the reflectance of the Earth and its atmosphere. The VIIRS data products are produced by whiskbroom scanning radiometers on satellites orbiting around the Earth at a nominal altitude of 829 km, providing a full daily coverage of the Earth. These data products include 6 bands in the visible spectrum with high spatial resolution which we could use to generate a spectral reflectance texture. Figure \ref{fig:swisscube_earth_renders} shows the result of this process compared to the use of a simple Earth albedo texture available from the NASA website [????].

% !TEX root = ../supp.tex
% !TEX spellcheck = en-US

\begin{figure}[t]
    \centering
    \begin{tabular}{ccc}
    \includegraphics[height=2.4cm]{fig/swisscube_earth_renders_L1/earth_L1_albedo_texture.png}&
    \includegraphics[height=2.4cm]{fig/swisscube_earth_renders_L1/earth_L1_spectral_texture.png}&
    \includegraphics[height=2.4cm]{fig/swisscube_earth_renders_L1/earth_L1_DSCOVR.png}\\
    (a)&(b)&(c)\\
    \end{tabular}
    \vspace{-3mm}
    \caption{\small {\bf Rendered Earth compared to DSCOVR photograph.} (a) Ground-level albedo texture doesn't account for the scattering effects introduced by the atmosphere, resulting in over saturated colors when viewed from space. (b) Rendering of the Earth using our spectral texture based on the VIIRS data products. (c) Ground-truth real photograph taken by the DSCOVR satellite at the L1 Lagrange point.}
    \label{fig:swisscube_earth_renders}
\end{figure}

% VIIRS website: https://earthdata.nasa.gov/earth-observation-data/near-real-time/download-nrt-data/viirs-nrt#ed-corrected-reflectance
% Earth albedo texture: https://visibleearth.nasa.gov/images/57735/the-blue-marble-land-surface-ocean-color-sea-ice-and-clouds/57737l
% Real image DSCOVR website: https://www.nesdis.noaa.gov/content/dscovr-deep-space-climate-observatory

\subsubsection{Bring everything together at real scale}

We placed all those elements in a virtual scene at the real scale to ensure high fidelity images and produce a more comprehensive dataset. We could then generate sequences of images, simulating various docking procedures by varying camera and target vehicle poses and respective speed. Each sequence contains 100 consecutive images.



To achieve highest reflection of the real world, we place the SwissCube in the actual-working orbits about 700 km above the Earth's surface during rendering and model most of the space-borne items, such as the Sun, the Earth, and galaxies, physically. 

Depth range.

\subsubsection{Dataset generation and specs}

We generate 400 sequences in total, each sequence contains 100 consecutive images with random angular speed between 0.xx to 0.xx.

Fig.~\ref{fig:swisscube_vs_speed} shows some examples of the generated SwissCube dataset.

bla bla bla bla bla bla bla bla bla bla bla bla bla bla bla bla bla bla bla bla bla bla bla bla bla bla bla bla bla bla bla bla bla bla bla bla bla bla bla bla bla bla 

As discussed Section~\ref{sec:related}. Table~\ref{tab:swisscube_vs_speed} shows the comparison between SwissCube and SPEED dataset. Note that, in SPEED dataset, there are 2998 more synthetic images and also 300 more real images in the test set. However, Their ground truth labels are not accessible, so here we do not take them into account.

\YH{TODO}

From the total 400 sequences of images in the SwissCube dataset, we take all the images from the first 300 sequences as our training set and that from the last 100 as the test. In this subsection, we will evaluate the methods' performance in different depth ranges, that is, the whole depth range will be divided to three regions denoted as {\it Near}, {\it Medium} and {\it Far}, which correspond to depth range [1d-4d], [4d-7d] and [7d-10d], respectively.

\end{document}

