\section{Assumptions for security}
\label{sec:attacks}

The security of a quantum cryptographic protocol relies on assumptions about the physics of the devices that are employed to implement the protocol.  In this section, we discuss these assumptions. For concreteness, we focus on the case of QKD, for which we describe the full set of assumptions in \secref{sec:attacks:assumptionlist}.  We then explain why these assumptions are needed and to what extent they are justified in \secref{sec:attacks:necessity}. Experimental work in QKD has shown however that the assumptions are often very difficult to meet, and are actually not met in many cases. This fact can be exploited by quantum hacking attacks, which are described in \ref{sec:attacks:hacking}. Finally, in Section~\ref{sec:attacks:countermeasures}, we discuss countermeasures against these attacks. 

\subsection{Standard assumptions for QKD} \label{sec:attacks:assumptionlist}

The security of QKD protocols usually relies on the following assumptions.

\begin{enumerate}
  \item \label{item_qm} All devices used by Alice and Bob, as well as the communication channels connecting them, are correctly and completely\footnote{The completeness of quantum theory can be derived from their correctness;  see \secref{sec:completeness}.} described by quantum theory.
    \item \label{item_res} The channel that Alice and Bob use to exchange classical messages is authentic, i.e., it is impossible for an adversary to modify messages or insert new ones. 
  \item \label{item_conv} The devices that Alice and Bob use locally to execute the steps of the protocol, e.g., for preparing and measuring quantum systems, do exactly what they are instructed to do.
\end{enumerate}

As already indicated earlier, due to the lack of proof techniques, additional assumptions had been introduced in the past. A prominent example is the \emph{i.i.d.}\ assumption, which demands that the quantum channel connecting Alice and Bob be described by a sequence of identical and independently distributed maps. Physically, this means that an adversary's interception strategy is such that each signal sent from Alice to Bob is modified in the same manner and independently of the other signals. Security under the i.i.d.\ assumption is called security against \emph{collective attacks}~\cite[see also \secref{sec:qkd.other.models}]{BM97b,BBBvdGM02}. Another assumption, which  usually comes on top of the i.i.d.\ assumption, is that Eve only stores classical data, which she obtains by individually measuring the pieces of information she gained from each signal sent from Alice to Bob. Since it is difficult to argue why an adversary should be restricted in that particular way, the corresponding security guarantee is rather weak. It is usually referred to as security against \emph{individual attacks}~\cite[see \secref{sec:qkd.other.models}]{Fuchsetal1997,Lutkenhaus2000}. 

Most modern security proofs do however not require such additional assumptions, i.e., they are based entirely on Assumptions~\ref{item_qm}--\ref{item_conv} above. This means, in particular, that the quantum channel connecting Alice and Bob can be arbitrary, and may even be entirely controlled by Eve. In this case, one talks about security against \emph{general attacks}, \emph{coherent attacks}, or \emph{joint attacks}. Sometimes the term  \emph{unconditional security} appeared in the literature~\cite{SBCDLP09}, but it is important to keep in mind that the assumptions listed above are still necessary.

\subsection{Necessity and justification of assumptions} \label{sec:attacks:necessity}

Assumption~\ref{item_qm} is often implicit, for it is a prerequisite to even describe the cryptographic scheme. It justifies the use of the formalism of quantum theory to model the different systems, such as the communication channel, including any possible attacks on them. The assumption thus captures the main idea behind quantum cryptography, namely that an adversary is limited by the laws of quantum theory.  The other two assumptions ensure that the experimental implementation follows the theoretical prescription that enters the security definition (Definition~\ref{def:security}), namely the description of the protocol $\pi_{AB}$ and the used resources. In particular, Assumption~\ref{item_res} guarantees that the resources shared between Alice and Bob fulfil the theoretical specifications~$\aR$, which in the case of QKD includes the classical authentic communication channel. Assumption~\ref{item_conv} guarantees that the steps prescribed by the protocol~$\pi_{A B}$ are correctly executed.

Assumption~\ref{item_qm} is widely accepted \--- and proving it wrong would represent a major breakthrough in physics. Nevertheless, it has been shown that there exist QKD protocols that only rely on the weaker assumption of \emph{no-signalling}~\cite{BHK05}.  

Assumption~\ref{item_res} demands that an authentic communication channel is set up between Alice and Bob. There exist information-theoretically secure protocols that achieve this, provided that Alice and Bob share a weak secret key~\cite[see also \secref{sec:smt.auth}]{RW03,DW09,ACLV19}.  Assumption~\ref{item_res} can thus be met by the use of such authentication protocols (see also~\secref{sec:intro} as well as standard textbooks on classical cryptography)

Although Assumption~\ref{item_conv} sounds rather natural, and is in fact required for almost any cryptographic scheme, including any classical one, it is rather challenging to meet.  Numerous quantum hacking experiments, which have been conducted over the past few years, have shown that many implementations of QKD failed to satisfy this assumption. To illustrate this problem, we describe selected examples of such attacks in the following subsection.

\subsection{Quantum hacking attacks} \label{sec:attacks:hacking}          

We start with the \emph{photon number splitting attack}~\cite{Brassardetal2000}, which targets optical implementations of QKD that use individual photons as quantum information carriers. Suppose, for concreteness, that Alice and Bob implement the BB84 protocol~\cite{BB84} by encoding the qubits into the polarisation degree of freedom of individual photons. Specifically, Alice may use a single-photon source that emits photons with a polarisation that she can choose. The BB84 protocol\footnote{This protocol is explained in more detail in \secref{sec:securityproofs}, where a security proof is also sketched.} requires her to send in each round at random a state from one orthonormal basis, say $\{\ket{h}, \ket{v}\}$, where $\ket{h}$ may be realised by a horizontally polarised photon and $\ket{v}$ by a vertically polarised one, or from a complementary basis $\{\ket{d^+}, \ket{d^-}\}$, where $\ket{d^+} = \smash{\frac{1}{\sqrt{2}}} (\ket{h} + \ket{v})$ and $\ket{d^-} = \smash{\frac{1}{\sqrt{2}}} (\ket{h} - \ket{v})$. It may now happen that, in an experimental implementation, the source sometimes accidentally emits two photons at once, which then carry the same polarisation. The states emitted in the four cases are thus $\ket{h} \otimes \ket{h}$, $\ket{v} \otimes \ket{v}$, $\ket{d^+} \otimes \ket{d^+}$, and $\ket{d^-} \otimes \ket{d^-}$.

Before describing the actual attack, we first give a simple information-theoretic argument for why this is problematic. Note first that one single photon carries no information about the choice of the basis made by Alice. Indeed, for either of the basis choices, the density operator describing the photon is maximally mixed, i.e., $\frac{1}{2} \proj{h} + \frac{1}{2} \proj{v} = \frac{1}{2} \proj{d^+} + \frac{1}{2} \proj{d^-} =  \frac{1}{2} \mathbf{1}$. This is however no longer the case for a pulse consisting of two photons, i.e., 
\begin{align}
  \frac{1}{2} \proj{h}^{\otimes 2} + \frac{1}{2} \proj{v}^{\otimes 2} \neq \frac{1}{2} \proj{d^+}^{\otimes 2} + \frac{1}{2} \proj{d^-}^{\otimes 2} \ .
\end{align}
Hence, if the source accidentally emits two equally polarised photons instead of one, it reveals information about Alice's basis choice, which it shouldn't. 

It is therefore not surprising that such two-photon pulses can be exploited by an adversary to attack the system. Eve, who intercepts the channel, may split the two-photon pulse into two, keep one of the photons and send the other one to Bob. The latter thus receives photons in exactly the way prescribed by the protocol, and hence does not notice the interception. Eve, meanwhile, may measure the photons she captured. In principle, if Eve had quantum memory, she could even wait with the measurement until Alice announces the basis choice to Bob, and hence always gain full information about the polarisation state that Alice prepared. 

While the photon number splitting attack exploits an imperfection of the sender (namely that it sometimes emits two identically polarised photons instead of one), many quantum attacks are targeted towards the receiver. An example is the \emph{time-shift attack}~\cite{Makarovetal2006,qi2007time,Zhaoetal2008}, which exploits inaccuracies of the photon detectors. In order to avoid dark counts, the photon detectors are often set up such that they only count photons that arrive within a small time window around the time when a signal is expected to arrive. Furthermore, Bob's receiver device may consist of more than one detector, e.g., one for each possible polarisation state. The time windows of the different detectors are then never perfectly synchronised. This means that there are times at which the receiver is more sensitive to signals with respect to one polarisation than another. Eve may therefore, by appropriately delaying the signals sent from Alice and Bob, bias the detected signals towards one or the other polarisation, and thus gain information about what Bob measures. While this information may be partial, it can, together with the error correction information that is available to Eve, be sufficient to infer the final key. 

Another attack that is targeted towards the receiver is the \emph{detector blinding attack} ~\cite{Makarov2009,WKRFNW11,LWWESM10,GLLSKM11}, where the adversary tries to control the detectors by illuminating them with bright laser light.  In a QKD implementation that uses the encoding of information into the polarisation of individual photons, the detectors are usually configured such they can optimally detect single photon pulses. That is, they should click whenever the incoming pulse contains a photon, and not click if the pulse is empty.  However, the behaviour of such detectors may be rather different in a regime where the incoming pulses contain many photons. For example, it could be that they always click when they are exposed to bright light with a particular intensity, and they may never click for another intensity. Hence, by sending in light with appropriately chosen polarisation and intensity, Eve may gain immediate control over the clicks of Bob's detector. To exploit this for an attack, Eve may mimic Bob's receiver, i.e., intercept the photons sent from Alice and measure them in a randomly chosen basis, as Bob would do. She then sends bright light to Bob to ensure that he obtains the same detector clicks as if he had directly obtained Alice's photons. This works particularly well for implementations that use a \emph{passive basis choice}, i.e., where Bob's measurement basis is not  provided as an input, but rather made by the detection device itself. In this case, an adversary can essentially remote-control Bob and thus get hold of the entire key. 

Yet another hacking strategy are \emph{Trojan-horse attacks}~\cite{Vakhitov2001,GisinFaselKraus2006}. Here the idea is to send a bright laser pulse via the optical fibre into Alice or Bob's component to extract information about its internal settings. Depending on the sender and receiver hardware which is used, measuring the reflection of the pulse can allow Eve, for instance, to determine the basis choices made by Alice and Bob.

In some optical implementations of QKD, e.g., in the \emph{plug-and-play}~\cite{Muller97} or the \emph{circular-type}~\cite{Nishioka2002} system, Alice does not have a photon source but instead encodes information by modulating an incoming signal from Bob before sending it back to him. The signal thus travels twice in opposite directions through the same optical links, which helps reducing fluctuations due to birefringence  and environmental noise. The two-fold use of the (insecure) channel however opens additional possibilities of attacks~\cite{GisinFaselKraus2006}. A prominent example is the \emph{phase-remapping attack}~\cite{FQTL07,XQL10}. It exploits the fact that the modulator used by Alice to encode information into the signal coming from Bob acts on that signal during a particular time interval. In the attack, the adversary slightly advances or delays the signal on its way from Bob to Alice, so that it no longer lies fully within that time interval. The modulation by Alice will then be incomplete, which means that the encoding of the information in the signal differs from what is foreseen by the protocol. This can in turn be exploited by Eve in an intercept-and-resend attack on the signal returned from Alice to Bob. 


\subsection{Countermeasures against quantum hacking} \label{sec:attacks:countermeasures}

The attacks described here have in common that they all exploit a breakdown of Assumption~\ref{item_conv}. Specifically, in the case of the photon-splitting attack, the device used by Alice sends out more information than it is supposed to. In the case of the time-shift attack, it is Bob's measurement device  whose measurement operators are not constant over time and can even be partially controlled by Eve. Finally, in the case of the detector blinding attack on systems with passive basis choice, Eve even takes over control of the randomness used to choose the basis.

A seemingly obvious countermeasure to prevent such attacks is to manufacture sources and detectors that meet the theoretical specifications. That is, one would need a perfect single-photon source, as well as detectors that are perfectly efficient and only measure photon pulses in a specified parameter regime. Such requirements are however unrealistic --- the devices used in experiments will always, at least slightly, deviate from these specifications. 

The other possibility is to develop cryptographic protocols and  security proofs that tolerate imperfections of the devices~\cite{GLLP04}.  This has been done in particular for the attacks described above. To prevent photon number splitting attacks, an efficient countermeasure is the \emph{decoy-state} method~\cite{Hwang2003,Wang2005,Loetal2005}. The idea here is that Alice sometimes deliberately sends multi-photon pulses. Alice and Bob can  then check statistically whether an adversary captured them. Another possibility is to use protocols where Alice's encoding of information has the property that, even when one photon is extracted from a pulse, the information about what Alice sent is still partial~\cite{SARG,TamakiLo,SYK14}. In the case of time-shift attacks, it is sufficient to characterise the maximum bias in the detector efficiencies that can be introduced and account for it in the security proofs. Finally, for the detector blinding attacks, a possible countermeasure is to add tests to the protocol, such as a monitoring of the photocurrent, in order to detect those~\cite{Yuanetal2010}.  

The main problem with such countermeasures is however that the space of possible imperfections is hard to characterise. The above are just a few examples of attacks, and many others have been proposed, and sometimes even demonstrated to work successfully in experiments. For example, an adversary may exploit imperfections in the randomness that Alice and Bob use for choosing their measurement basis. To prevent such attacks, one may again extend the protocols such that they can tolerate imperfect randomness (see \secref{sec:alternative.randomness}). 



The last decade has thus seen an arms race between designers and attackers of quantum cryptographic schemes. A possible way out of this unsatisfactory situation is \emph{device-independent cryptography}. Here the idea is to replace Assumption~\ref{item_conv} by something much weaker. Namely, one requires that the devices used by Alice and Bob do not unintentionally send information out to an adversary, and that the classical processing of information done by Alice and Bob is correct. Crucially, however,  one does no longer demand that the sources and detectors used by Alice and Bob work according to their specifications. The way this can work is explained in \secref{sec:alternative.di}. 

%%% Local Variables:
%%% TeX-master: "main.tex"
%%% End:
