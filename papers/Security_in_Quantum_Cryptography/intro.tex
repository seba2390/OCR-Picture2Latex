\section{Security from physical principles}
\label{sec:intro}

Communication theory is concerned with the task of making information
available to different parties. The sender of a message $x$ wants
$x$ to become accessible to a designated set of recipients. In
\emph{cryptography}, one adds to this a somewhat opposite requirement
\--- that of restricting the availability of information. The sender
of~$x$ also wishes to have a guarantee that $x$ remains inaccessible
to \emph{adversaries}, i.e., parties other than the intended recipients.
The term \emph{security} refers to this additional guarantee.

Testing whether a communication protocol works correctly is easy. It
suffices to compare the message~$x$ sent with the received
one. Testing security, however, is more subtle. To ensure that an
adversary cannot read~$x$, one needs to exclude all physically
possible eavesdropping strategies.  Since there are infinitely many such strategies it is not possible, at least not by direct experiments, to prove  that a cryptographic scheme is secure \---
although a successful hacking experiment would of course show the
opposite.

But the situation is not as hopeless as this sounds. Security can be established indirectly, provided that one is ready to make certain assumptions about the capabilities of the adversaries. Clearly, the weaker these assumptions are, the more confident we can be that they apply to any realistic adversary, and hence that a cryptographic scheme based on them is actually secure. %However, if they are too weak, it will no longer be possible to derive security claims from them.

The security of most cryptographic schemes used today relies on computational hardness assumptions. They correspond to constraints on the adversaries' computational resources. For example, it is assumed that adversaries do not have the capacity to factor large integers~\cite{RSA78}. This is a relatively strong assumption, justified merely by the belief that the currently known algorithms for factoring cannot be substantially improved in the foreseeable future \--- and that  quantum computers, powerful enough to run Shor's (efficient) factoring algorithm~\cite{Shor97}, cannot be built. Cryptographic schemes whose security is based on assumptions of this type are commonly termed \emph{computationally secure}.

In contrast to this, the main assumption that enters quantum cryptography is that adversaries are subject to the laws of quantum mechanics.\footnote{To prove security, one usually also requires that adversaries cannot manipulate the local devices (such as senders and receivers) of the legitimate parties. But, remarkably, this (seemingly necessary) requirement can be weakened \--- this is the topic of device-independent cryptography, which we discuss in~\secref{sec:open.di}.}  This assumption completely substitutes computational hardness assumptions, i.e., security holds even if the adversaries can use unbounded computational resources to process their information.\footnote{Though one may naturally also consider computationally secure quantum cryptography, which we do in, e.g., \secref{sec:computational} and \secref{sec:open.computational}.} To distinguish this from computational security, the resulting security is sometimes termed \emph{information-theoretic}, reflecting the fact that it can be defined in terms of purely information-theoretic concepts~\cite{Shannon49}.

\subsection{Completeness of quantum theory} \label{sec:completeness}

The assumption that adversaries are subject to the laws of quantum mechanics appears to be  rather straightforward to justify. Indeed, quantum mechanics is one of our best tested physical theories. As of yet, no  experiment has been able to detect deviations from its predictions. Of particular relevance for cryptography are non-classical features of quantum mechanics, such as entanglement between remote subsystems, which have been tested by Bell experiments~\cite{FreedmanClauser,Aspect81,Aspect82,Tittel,Weihs,Rowe,Giustina13,Christensen,Hensen,Giustina15,Shalm,Rosenfeld}. However, the assumption  that enters quantum cryptography not only concerns the \emph{correctness} of quantum mechanics (as one may naively think), but also its \emph{completeness}. This is an important point, and we therefore devote this entire subsection to it. 

Quantum mechanics is a \emph{non-deterministic} theory in the following sense. Even if we know, for instance, the polarisation direction $\psi$ of a photon to arbitrary accuracy, the theory will not in general allow us to predict with certainty the outcome $z$ of a polarisation measurement of, say, the vertical versus the horizontal direction. The statement that we can obtain from quantum mechanics may even be completely uninformative. For example, if the polarisation $\psi$ before the measurement was diagonal, the theory merely tells us that a measurement of the vertical versus the horizontal direction will yield both possible outcomes~$z$ with equal probability. 

It is  conceivable that non-determinism is just a limitation of current quantum theory, rather than a fundamental property of nature. This would mean that there could exist another theory that gives better predictions. In the example above, it could be that the photon, in addition to its polarisation state~$\psi$, has certain not yet discovered properties $\lambda$ on which the measurement outcome~$z$ depends.  A theory that takes into account $\lambda$ could then yield more informative predictions for~$z$ than quantum mechanics. If this were the case then quantum mechanics could not be considered a complete theory. 

Quantum cryptography is built on the use of physical systems, such as photons, as information carriers. The incompleteness of quantum mechanics would hence imply that the theory does not give a full account of all  information contained in these systems. This would have severe consequences for security claims. For example, a  cryptographic scheme for transmitting a confidential message~$x$ may be claimed to be secure on the grounds that the quantum state $\psi$ of the information carriers  gathered by an adversary  is independent of~$x$. Nonetheless, it could still be that the adversary's information carriers have an extra property, $\lambda$, which is not described by quantum theory and hence not included in $\psi$. The independence of $\psi$ from~$x$ is then not sufficient to guarantee that the adversary cannot learn the secret message.

A possible way around this problem is to simply \emph{assume} that no adversary can access properties of physical systems, like $\lambda$ in the above example, which are not captured by their quantum state~$\psi$. But such an assumption seems to be similarly difficult to justify as, for instance, the non-existence of an efficient factoring algorithm. The fact that we have not yet been able to discover $\lambda$ does not mean that it does not exist (or that it cannot be discovered). 

Fortunately, the problem can be resolved in a more fundamental manner. The solution is based on a long sequence of work dating back to~\textcite{Born26,EPR35},  where the question regarding the completeness of quantum mechanics was raised. The central insight resulting from this work was that the set of possible theories that could improve the predictions of quantum mechanics is highly constrained. For example, no such theory can yield deterministic predictions, based on additional parameters~$\lambda$, unless it is non-local\footnote{This concept will be briefly discussed in the context of device-independent cryptography in \secref{sec:alternative.di}.}~\cite{Bell64} and contextual \cite{Bell66,KocSpe67}. More recently, it has been shown that no theory can improve the predictions of quantum mechanics unless it violates the requirement that measurement settings can be chosen freely, i.e., independently of other parameters of the theory~\cite{CR11}.\footnote{More precisely, according to~\textcite{BellFree}, variables are ``free'' if they ``have implications only in their future light cones.'' In other words, they are uncorrelated to anything outside their causal future. This notion has sometimes also been called ``free will''~\cite{Conway2006}.} The completeness of quantum mechanics is hence implied by the assumption that physics does not prevent us from making free choices \--- an assumption that appears to be unavoidable in cryptography anyway~\cite{ER14}. 

\subsection{Correctness of quantum-theoretic description}

In the previous section we have seen that the security of quantum cryptography crucially relies on the completeness of quantum mechanics, but that the latter can be derived from the requirement that one can make free choices.  It is of course still necessary to assume that quantum mechanics is correct, in the sense that it accurately describes the  hardware used for implementing a cryptographic protocol. But since  quantum mechanics consists of a set of different rules,  we should  be more specific about what this correctness assumption really means. 

Quantum cryptographic protocols are usually described within the framework of quantum information theory~\cite{nielsen2010quantum}, which provides the necessary formalism to talk about information carriers and operations on them. Any information carrier is modelled as a quantum system $S$ with an associated Hilbert space $\mathcal{H}_S$, and the information encoded in~$S$ corresponds to its state. In the case of ``classical'' information,  the different values~$x$ of a variable with range~$\mathcal{X}$ are represented by different elements from a fixed orthonormal basis  $\{\ket{x}\}_{x \in \mathcal{X}}$ of $\mathcal{H}_S$. If the marginal state of  a system~$S$ has the form  $\rho_{S} = \sum_{x} p_{x} \proj{x}$  this means that $S$ carries the value~$x$ with probability~$p_x$. 
%The interpretation of a state of the form $\rho_S = \sum_{x \in \mathcal{X}} p_x  \proj{x}$ is then that system $S$ carries the value~$x$ with probability~$p_x$. 
Any processing of information (including, for instance, a measurement)  corresponds to a change of the state of the involved information carriers, and is represented mathematically by a  trace-preserving, completely positive map.\footnote{We refer to standard textbooks in quantum information theory, such as~\textcite{nielsen2010quantum}, for a description of these concepts. An argument that justifies their use in the context of cryptography can be found in \textcite{RenesRenner2020}.}

The modelling of real-world implementations in terms of these rather abstract information-theoretic notions is a highly non-trivial task. To illustrate this, take for example an optical scheme for quantum key distribution, where information is communicated by an encoding in the polarisation of individual photons. This suggests a description where each photon sent over the optical channel is regarded as an individual quantum system. However, photons are just excitations of the electromagnetic field and thus a priori not objects with their own identity. (That is, they are indistinguishable.) A solution to this problem could be that one ``labels'' the photons by the time at which they are sent out, i.e., photons sent at  different times are regarded as different quantum systems, $S$. But there could be more than one photon emitted at a particular time, and these different photons could or could not have the same frequency. One may now choose to take this into account by modelling the photon number and their frequency as internal degrees of freedoms of the system $S$. Or, one could choose the frequency to be an additional system label, so that photons with different frequencies are regarded as different systems. 

This example shows that the translation of an actual physical setup into the language of quantum information theory is prone to mistakes and certainly not unique. Nonetheless, it is critical for security \--- if done incorrectly, the security statements, which are derived within quantum information theory, are vacuous. Particular care must be taken to ensure that no information carriers that are present in an implementation are omitted.  A realistic photon source may, for instance, sometimes emit two instead of only one photon whose polarisation encodes the same value, and this second photon may be accessible to an eavesdropper  (see Section~\secref{sec:attacks:hacking}). This possibility must therefore be included in the quantum information-theoretic description of that photon source.  If it was not, it would represent a \emph{side channel} to the adversary that is not accounted for by the security proof. Side channels may also occur in other components, such as photon detectors, for instance. We refer to the review of~\cite{SBCDLP09} for a general discussion of these practical aspects of quantum cryptography. 

% \bigskip

\subsection{Overview of this review}

In this review we will focus on the information-theoretic layer of security proofs, i.e., we will presume that we have a correct quantum information-theoretic description of the cryptographic hardware. The existence of such a description is indeed a standard assumption made for security proofs and usually termed \emph{device-dependence}. It contrasts \emph{device-independent} cryptography, where this assumption is considerably relaxed (see~\secref{sec:alternative.di} for a brief discussion). 

We will start in the next section by introducing general concepts from cryptography. From then on we will largely focus on Quantum Key Distribution (QKD), which currently is the most widespread application of  quantum cryptography. It is also an excellent concrete example to discuss security definitions, the underlying assumptions, as well as proof techniques. Towards the end of this review we will explain how these notions apply to cryptographic tasks other than key distribution.

%We conclude this first section with a short summary. The security of quantum cryptography relies on a purely physical principle, namely that the cryptographic equipment is correctly described by the laws of quantum mechanics. Hence, breaking a quantum cryptographic scheme that is correctly implemented would necessarily mean that one has discovered new physics. 

%Suppose that a sender, Alice, would like to transmit a message $x$ to a receiver, Bob. To achieve this, they may proceed according to a certain protocol, which tells  them how this task can be achieved with the resources they have to their availability (e.g., a noisy communication channel). 



%%% Local Variables:
%%% TeX-master: "main.tex"
%%% End:
