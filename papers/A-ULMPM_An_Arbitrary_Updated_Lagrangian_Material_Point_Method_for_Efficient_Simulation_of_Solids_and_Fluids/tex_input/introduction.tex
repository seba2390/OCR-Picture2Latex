\section{INTRODUCTION}
\label{sec:introduction}
The Material Point Method (MPM) family of discretizations~\cite{Sulsky:1994:Particle}, such as Fluid Implicit Particle (FLIP)~\cite{Brackbill:1988:Flip} and Particle-in-Cell (PIC)~\cite{Sulsky:1995:PIC}, emerged as an effective choice for simulating various materials and gained popularity in visual effects (VFX) for providing high-fidelity physics simulations of snow~\cite{Stomakhin:2013:MPMsnow}, sand~\cite{Klar:2016:Drucker,Daviet:2016:Semi-implicit}, phase change~\cite{Stomakhin:2014:Augmented,gao2018gpu}, viscoelasticity~\cite{Ram:2015:Material,yue:2015:mpm-plasticity,Su:2021:USOSVL}, viscoplasticity~\cite{Fang:2019:Silly}, elastoplasticity~\cite{Gao:2017:AGIMPM}, fluid structure interactions~\cite{Fang:2020:IQ-MPM}, fracture~\cite{Wolper:2019:CD-MPM,hegemann:2013:ductile}, fluid-sediment mixtures~\cite{Tampubolon:2017:Multi,gao2018animating}, baking and cooking~\cite{Ding:2019:Thermomechanical}, and diffusion-driven phenomena~\cite{XUE:2020:NF}.
In contrast to Lagrangian mesh-based methods, such as the Finite Element Method (FEM)~\cite{Zienkiewicz:1977:FEM,Sifakis:2012:FEM}, and pure particle-based methods, such as Smoothed Particle Hydrodynamics (SPH)~\cite{Desbrun:1996:SPH,Liu:2008:Overview}, MPM merges the advantages of both Lagrangian and Eulerian approaches and automatically supports dynamic topology changes such as material splitting and merging. 
It uses Lagrangian particles to carry material states, while the background grid acts as an Eulerian ``scratch pad'' for computing the divergence of stress and performing spatial/temporal numerical integration. 
The use of a background grid allows for regular numerical stencils, benefiting from cache-locality, while the use of particles avoids the numerical dissipation issues characteristics of Eulerian grid-based schemes.

\begin{figure*}[t]
\vspace*{-4mm}
\begin{center}
\includegraphics[width=\textwidth]{image/2d_rotating_mls_2.png}\\
\vspace{-1mm}
MLS-EMPM: $t=\{0.00833,\:0.01667,\:0.02500,\:0.03333,\:0.04167\}$s\\
\includegraphics[width=\textwidth]{image/2d_rotating_mls_tl_2.png}\\
\vspace{-1mm}
MLS-TLMPM: $t=\{0.00833,\:0.02500,\:0.04167,\:0.06250,\:0.07917\}$s\\
\includegraphics[width=\textwidth]{image/2d_rotating_mls_tl_update_speed.png}\\
\vspace{-3mm}
MLS-$A$-ULMPM: $t=\{0.00833,\:0.02500,\:0.04167,\:0.06250,\:0.07917\}$s\\
\vspace{-4mm}
\end{center}
\caption{\textbf{Eliminating numerical fracture.} Our $A$-ULMPM (bottom row) and TLMPM (middle row) for solid simulation capture the appealing hyperelastic rotation, preserve the angular momentum for long simulation periods, and completely eliminate numerical fracture.
Traditional EMPM (top row) suffers from severe numerical fracture in solid simulations when large deformations occur, while TLMPM (middle row) does not. 
 EMPM updates configurations simultaneously with respect to particles, while TLMPM does not update configurations at all. $A$-ULMPM only updates configurations whenever the shape changes reach to our predefined criterion (see equation~\eqref{eq:update_conf}). The background grid represents the configuration linking to the present particle dynamics.}
\vspace*{-4mm}
\label{fig:2D_rotating}
\end{figure*}

Conventional MPM discretization employs an \emph{Eulerian} formulation, which measures stress and strain and computes derivatives and integrals with respect to the Eulerian coordinates (i.e., the ``current'' configuration). 
Eulerian MPM (EMPM) has been acknowledged as a powerful tool for physics-based simulations~\cite{jiang2016material}, particularly if the system involves large deformations, such as fluid-like motion. 
However, it suffers from a number of shortcomings such as the \emph{cell-crossing instability}. 
Many studies have shown that when cell-crossing occurs, the MPM solutions can either be non-convergent or reduce the convergence rate when refining grid, with the spatial convergence rate varying between first and second order~\cite{W:2007:improved} (see Figure~\ref{fig:ConvPlot}).
The deficiency of cell-cross instability has been reduced in the latest MPM formulations, such as the Affine Particle-In-Cell (APIC) method~\cite{Jiang:2017:APIC}, the generalized interpolation MPM (GIMP)~\cite{Bardenhagen:2004:GIMPM, Gao:2017:AGIMPM}, and the Convected Particles Domain Interpolation (CPDI)~\cite{Sadeghirad:2011:CPDI}. 
Among them, the APIC approach has been widely adopted in computer graphics. The central idea behind APIC is to retain the filtering property of PIC, but reduce dissipation by interpolating more information, such as the velocity and its \emph{derivatives}, aiming to conserve linear and angular momentum. 
An improved APIC, namely PolyPIC, was proposed in~\cite{Fu:2017:PolyPIC} that allows for locally \emph{high-order approximations}, rather than approximations to the grid velocity field.
Later on, a \emph{moving least squares} MPM formulation (MLS-MPM)~\cite{Hu:2018:Moving} was developed by introducing the MLS technique to elevate the accuracy of the internal force evaluation and velocity derivatives. 
 
 Although EMPM discretizations have been improved to a certain degree via the aforementioned MLS techniques, they still suffer from \emph{numerical fracture} that occurs when particles move far enough from the cell where they are originally located, and a gap of one cell or more is created between them. 
 Such non-physical fracture depends only on the background grid resolution and is not related to any other issue that would ultimately limit the accuracy of the EMPM to model actual physical fracture of materials under large deformations (see Figure~\ref{fig:2D_rotating}), particularly in solid simulations.
 
 As a counterpart to Eulerian formulations, \emph{total Lagrangian formulations} offer a promising alternative for avoiding the cell-crossing instability and numerical fracture in MPM~\cite{De:2020:TLMPM,De:2021:TLMPM}. 
 Unlike EMPM, total Lagrangian MPM (TLMPM) measures stress and strain and computes derivatives and integrals with respect to the \emph{original} configuration at time $t^0$ (similar to traditional FEM~\cite{Sifakis:2012:FEM,Zienkiewicz:1977:FEM}). 
 By doing this, no matter what the deformation, the reference configuration being always the same, there is neither any cell-crossing instability nor any numerical fracture. 
 Despite its high efficacy in solid simulations, traditional TLMPM fails to model extremely large deformations that arise in fluid simulations. 
 We show a 2D droplet example in Figure~\ref{fig:2D_droplet}. TLMPM is not able to capture the dynamics of splashes because the interpolation kernel and its derivatives in TLMPM only reflect the fixed topology at time $t^0$ and do not support extreme topologically changing dynamics. 
%\subsection{Contributions}
\begin{figure*}[t]
\vspace*{-4mm}
\begin{center}
\includegraphics[width=\textwidth]{image/Kernel_TL.png}\\
\vspace{-1mm}
Kernel-TLMPM: $t=\{0.14167,\:0.20833,\:0.47083,\:0.74167,\:1.31667,\:1.88750,\:2.04583\}$s\\
\includegraphics[width=\textwidth]{image/Kernel_EMPM.png}\\
\vspace{-1mm}
Kernel-EMPM: $t=\{0.14167,\:0.20833,\:0.47083,\:0.74167,\:1.31667,\:1.88750,\:2.06250\}$s\\
\includegraphics[width=\textwidth]{image/MLS_TL_V0_0.png}\\
\vspace{-1mm}
MLS-TLMPM: $t=\{0.14167,\:0.20833,\:0.47083,\:0.74167,\:1.31667,\:1.88750,\:2.06250\}$s\\
\includegraphics[width=\textwidth]{image/MLS_Update_PCT_0.03_V0_0.png}\\
\vspace{-1mm}
MLS-$A$-ULMPM: $t=\{0.14167,\:0.20833,\:0.47083,\:0.74167,\:1.31667,\:1.88750,\:2.06250\}$s\\
\includegraphics[width=\textwidth]{image/MLS_Update_Per_Step.png}\\
\vspace{-2mm}
MLS-EMPM: $t=\{0.14167,\:0.20833,\:0.47083,\:0.74167,\:1.31667,\:1.88750,\:2.06250\}$s\\
\vspace{-4mm}
\end{center}
\caption{\textbf{2D Droplet.} 
We integrate $A$-ULMPM with traditional Kernel-MPM~\cite{Stomakhin:2013:MPMsnow} (see Appendix~\ref{sec:A-ULMPM-kernel}) and MLS-MPM (see Section~\ref{sec:A_UL_MLS_MPM}). 
Although TLMPM can eliminate numerical fracture in solid simulations as shown in Figure~\ref{fig:2D_rotating}, it fails to capture very large deformations, such as fluid-like motion (see rows 1 and 3) in both Kernel-EMPM and MLS-EMPM. Our proposed $A$-ULMPM automatically updates configurations to produce similarly detailed dynamics as those with EMPM for fluid simulation (see rows 4 and 5). Background grid represents the configuration linking to the present particle dynamics.}
\vspace*{-3mm}
\label{fig:2D_droplet}
\end{figure*}

To alleviate the cell-crossing instability and numerical fracture while allowing for large deformations, we present an arbitrary updated Lagrangian discretization of MPM ($A$-ULMPM) that spans from total Lagrangian formulations to Eulerian formulations. Unlike EMPM and TLMPM, $A$-ULMPM allows the configuration to be updated \emph{adaptively} for measuring physical states including stress, strain, interpolation kernels and their derivatives. 
As a natural extension, we revisit the APIC~\cite{Jiang:2017:APIC} and MLS-MPM~\cite{Hu:2018:Moving} formulations and integrate these two methods in $A$-ULMPM. We augment the accuracy of velocity rasterization by leveraging both the local velocity and its first-order derivatives. 
Our theoretical derivations focus on a nodal discretized Lagrangian (see equation~\eqref{eq:nodal_L}), instead of the weak form discretization in MLS-MPM~\cite{Hu:2018:Moving}, and naturally lead to a modified MLS-MPM in $A$-ULMPM, which can recover MLS-MPM by using a completely Eulerian formulation.  
$A$-ULMPM does not require significant changes to traditional EMPM, and is more computationally efficient since it only updates interpolation kernels and their derivatives when large topology changes occur. To summarize, our main contributions are as follows:
\vspace{-3mm}
\begin{enumerate}
    \item An arbitrary updated Lagrangian MPM ($A$-ULMPM) that spans discretizations from TLMPM to EMPM and avoids the cell-crossing instability and numerical fracture;
    \item An easy-to-implement criterion that automatically updates the reference topology to enable fluid-like simulations;
    \item Integration of APIC and MLS-MPM in $A$-ULMPM to allow angular momentum conservation and efficiency by reconstructing the interpolation kernel only during topology updates;
    \item End-to-end 3D simulations of stretching and twisting hyperelastic solids, splashing liquids, and multi-material interactions to highlight the benefits of our $A$-ULMPM framework.
\end{enumerate}