\section{$A$-ULMPM Formulation for Discrete Systems}
\label{sec:A_UL_nodal}

We now discretize the arbitrary updated Lagrangian formulation described in Section~\ref{sec:arbitrary-updated-lagrangian}. A similar derivation of a total Lagrangian formulation for thermoelasticity was given in~\cite{Xue:2019:NL_L}.  
We use particles to track material information. Below we describe a general discretization methodology, which can either be implemented in pure particle-based discretizations such as SPH~\cite{Desbrun:1996:SPH} or MPM~\cite{jiang2016material}.

The continuum body is considered to be divided into a $N_{\text{dim}}$-dimensional Euclidean space ($\mathbb{E}^N_{\text{dim}}$), in which $N_{\text{dim}}$ denotes the number of dimensions due to spatial discretization, for the purpose of generality, we use \emph{nodes} to discretize the continuum body. These nodes can be particles in pure particle-based methods, such as Smoothed Particle Hydrodynamics(SPH)~\cite{Desbrun:1996:SPH} and grids in pure grid-based methods, such as finite difference method (FDM). Define $i$th node's position vector as $\bs{q}_i$ and its volume as $V_i$. 
We interpolate the Lagrangian of $i$ node at time instant $t_n$ ($\f{L}_i^n$) by its neighboring nodes $j$ at configuration at $t_s$ as follows
\begin{eqnarray}
    \f{L}_i^n=\sum_j \left[\frac{1}{2}\rho_j^0 (\bs{\dot{q}}_j^n)^T\bs{\dot{q}}^n_j -\Psi(\bs{F}_j^{0n})\right]W_{ij}^s V_j^s
    \label{eq:nodal_L}
\end{eqnarray}
We substitute $\f{L}_i^n$ to equation~\eqref{eq:lagrangian} and equation~\eqref{eq:AUL} and get
\begin{enumerate}
\item Kinetic term:
\begin{eqnarray}
     \frac{d}{dt}\left(\frac{\p \f{L}_i^n}{\p \bs{\dot{q}}_i^n}\right)=\sum_j \rho_j^0 \bs{\dot{q}}_j^n W_{ij}^s V_j^s =\sum_j \rho_j^0  W_{ij}^s V_j^s\bs{\ddot{q}}_i^n
\end{eqnarray}
\item Deformation term:
\begin{eqnarray}
    \frac{\p \f{L}_i^n}{\p \bs{q}_i^n}=-\sum_j \frac{\p \Psi(\bs{F}_j^{0n})}{\p\bs{q}_i^n} W_{ij}^s V_j^s= 
    -\sum_j\mathbb{P}_j^{s}  \frac{\p (\bs{F}_j^{sn})^T}{\p \bs{q}_i^s} W_{ij}^s V_j^ss
\label{eq:defromation}
\end{eqnarray}
where $\mathbb{P}_j^s=\mathbb{P}_j^0(\bs{F}_j^{0s})^T/J_j^{0s}$. 
\end{enumerate}
The equation motion of node $i$ at $t_n$ is given by
\begin{equation}
    \sum_j \rho_j^n  W_{ij}^s V_j^s\bs{\ddot{q}}_i^n+\sum_j\mathbb{P}_j^s(\bs{F}_j^{0s})^T  \frac{\p (\bs{F}_j^{0s})^T}{\p \bs{q}_i^s} W_{ij}^s V_j^s=\bs{0}
    \label{eq:dis_L_eq}
\end{equation}
One may have noticed that a further explicit expression of equation~\eqref{eq:dis_L_eq} relies on the concrete formulation of $\bs{F}_j^{0s}$.

\subsubsection{MLS-based gradient operator}
We introduce a MLS-based gradient operator in~\cite{Xue:2019:NL_L} (See Appendix A) that locally minimizes the error of a certain position over its neighboring regime.  
Following the same notation in equation~\eqref{eq:dis_L_eq}, deformation gradient of $j$th node at time instant $t_n$ with respect to an arbitrary configuration (reference configuration) at $t_s$ are given by
\begin{eqnarray}
    \bs{F}_j^{sn}=\left(\sum_{k}(\bs{q}_k^n-\bs{q}_j^n)\otimes \bs{r}_{jk}^s W_{jk}^s V_k^s\right)\left(\sum_{j}\bs{r}_{jk}^s\otimes\bs{r}_{jk}^s W_{jk}^s V_k^s \right)^{-1}
\label{eq:gradient_qp_0s}
\end{eqnarray}
where $k$ represents nodes within the neighboring regime of center node $j$ and $\bs{r}_{jk}^s=\bs{q}_j^s-\bs{q}_j^s$.
\subsubsection{Derivation of $\bs{F}_j^{sn}$}
Based on equation~\eqref{eq:dFdq} in Appendix A, We further evaluate the derivation of $\bs{F}_j^{sn}$ with respect to $\bs{q}_i^s$. 
\begin{eqnarray}
 \frac{\p \bs{F}_j^{sn}}{\p\bs{q}_i^s}=\left[\sum_{k}\left(\delta_{ik}^n-\delta_{ji}^n\right)\otimes \bs{r}_{jk}^s W_{jk}^s V_k^s \right]\left[\sum_{k}\bs{r}_{jk}^s\otimes \bs{r}_{jk}^s W_{jk}^s V_k^s \right]^{-1}
 \label{eq:dFjdqi}
\end{eqnarray}
\subsubsection{Equation of Motion with Arbitrary Updated Lagrangian}
Substituting equation~\eqref{eq:dFjdqi} to equation~\eqref{eq:dis_L_eq} yields the following equation of motion of node $i$ at $t_n$ 
\begin{equation}
\sum_j \rho_j^s  W_{ij}^s V_j^s\bs{\ddot{q}}_i^n+
\sum_j\left[\mathbb{P}_j^s\mathbb{K}_j^s+\mathbb{P}_i^s\mathbb{K}_i^s\right]\bs{r}_{ij}^s W_{ij}^s V_j^s=\bs{0}
\label{eq:EOM1}
\vspace{-3mm}
\end{equation}
where $\mathbb{K}_j^s=\left[\sum_{k}\bs{r}_{jk}^s\otimes \bs{r}_{jk}^s W_{jk}^s V_k^s \right]^{-1}$. We take advantage of $V_j^s=J_i^sV_j^0$ and $\rho_j^s=\rho_0/J_s$ and equation~\eqref{eq:EOM1} can be rewritten by
\begin{eqnarray}
\sum_j \rho_j^0  W_{ij}^s V_j^0\bs{\ddot{q}}_i^n+
\sum_j\left[\mathbb{P}_j^0(\bs{F}_j^{0s})^T\mathbb{K}_j^s+\mathbb{P}_i^0(\bs{F}_i^{0s})^T\mathbb{K}_i^s\right]\bs{r}_{ij}^s W_{ij}^s V_j^0=\bs{0}
\label{eq:EOM2}
\vspace{-3mm}
\end{eqnarray}
Equation~\eqref{eq:EOM2} unifies a general discretization formulation that allows to introduce arbitrarily intermediate configuration $\bs{F}_{0s}$ spanning from total Lagrangian formulation to updated Lagrangian formulations in a nodal discretized system. 
Specifically, we have
\begin{enumerate}
    \item Total Lagrangian formulation: $s=0$ 
\begin{eqnarray}
\sum_j \rho_j^0  W_{ij}^0 V_j^0\bs{\ddot{q}}_i^n+
\sum_j\left(\mathbb{P}_j^0\mathbb{K}_j^0+\mathbb{P}_i^0\mathbb{K}_i^0\right)\bs{r}_{ij}^0W_{ij}^0 V_j^0=\bs{0}
\label{eq:PD_TLSPH}
\end{eqnarray}

\item Eulerian formulation: $s=n$ 

\begin{eqnarray}
\sum_j \rho_j^0  W_{ij}^n V_j^0\bs{\ddot{q}}_i^n+
\sum_j\left[\mathbb{P}_j^0(\bs{F}_j^{0n})^T\mathbb{K}_j^n+\mathbb{P}_i^0(\bs{F}_i^{0n})^T\mathbb{K}_i^n\right]\bs{r}_{ij}^n W_{ij}^n V_j^0=\bs{0}
\label{eq:co_SPH}
\end{eqnarray}
\end{enumerate}
There is no need to update $W_{ij}$, $\bs{r}_{ij}$ and $\bs{K}$ matrices in equation~\eqref{eq:PD_TLSPH}, while they have to be updated at each time step in equation~\eqref{eq:co_SPH}. In comparison with pure particle-based discretization methods, equation~\eqref{eq:PD_TLSPH} recovers state-based Peridynamcis~\cite{Javili:2019:PD, Xue:2018:PD} and Total Lagrangian SPH~\cite{Ganzenmuller:2015:Similarity}, while equation~\eqref{eq:co_SPH} is the associated Eulerian formulation with respect to equation~\eqref{eq:PD_TLSPH} and recovers the correct SPH method~\cite{Vidal:2007:Stabilized}, where the stress term ($\mathbf{P}\bs{F}^T$) was replaced by Cauchy stress. 
The above formulations are derived based on a pure nodal discretization so that equation~\eqref{eq:EOM2} can be directly used for systems discretizated by particles. 
However, neighbouring particle searching is still needed whenever there is a need to update $W_{ij}$, $\bs{r}_{ij}$, and $\mathbb{K}$ matrices. 
The specific discussions regarding pure particle-based methods are outside the scope of this paper. 
