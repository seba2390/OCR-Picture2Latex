%\vspace{-4mm}
\section{$A$-ULMPM Formulation}
\label{sec:A_UL_MLS_MPM}

Starting from a nodal Lagrangian formulation, we derive an alternate expression for equation~\eqref{eq:Lagrangian GE} in the hybrid particle-grid framework of MPM. 
Interestingly, our formulation leads to a new MLS-MPM method (MLS-$A$-ULMPM) whose variant recovers MLS-MPM~\cite{Hu:2018:Moving} in the Eulerian setting.  
We use subscript $i$ to denote quantities on grid nodes, subscript $p$ to denote quantities on particles, and subscript $s$ to denote the intermediate configuration map. 
Table~\ref{tab:notation} summarizes the notation used in this section.
\begin{figure*}
    \centering
    \includegraphics[width=.98\textwidth]{image/EMPM_snow_2.png}
     \vspace{-4mm}
     \caption{\textbf{Snow bunnies break over a wedge.} 
    Our $A$-ULMPM framework captures rich interactions of several snow bunnies smashing and scattering after falling on a solid wedge, demonstrating the extreme deformations that our method can capture, similar to its Eulerian counterparts proposed in prior works.} 
    \label{fig:snow}
    \vspace{-3mm}
\end{figure*}




\subsection{Grid Setting}

We use a global grid that covers the entire computational domain. Each object has its own \emph{configuration map} $\phi$ that maps points $\bs{x}$ within the object to specific locations $\phi(\bs{x})$ in the global grid. To reduce the associated memory overhead, we implement the global grid using sparsity aware data structures~\cite{Setaluri:2014:SSP}.%will be updated when criteria is satisfied (see section~\ref{sec:update_configuration}). 
%In practical simulations, global grid is only used for collision detecting. we embed all the individual configuration map to this global grid so that it can be directly integrate to any existing MPM. 
%========================

\subsection{P2G Rasterization}

In contrast to EMPM, the velocity gradients $\nabla_s\bs{v}_p$ and weights $W_{pi}^s$ are all evaluated with respect to the configuration map $\phi$ at time $t_s$. We use a first order Taylor expansion to reconstruct particle velocities and rasterize particle momentum to the grid as follows:
\begin{equation}
    m_i^s \bs{v}_i^n=\sum_p m_p \left(\bs{v}_p^n+ \nabla_s\bs{v}_p \bs{r}_{ip}^s\right) W_{pi}^s
\end{equation}
where $\bs{r}_{ip}^s=\left(\bs{q}_p^s-\bs{q}_i^s\right)$ and $\nabla_s \bs{v}_p^n=\frac{\p \bs{v}_p^n}{\p \bs{q}_s}$, which is further expanded out in equation~\eqref{eq:gradientVp}.
The mass/velocity at grid nodes are given by:
\begin{equation}
    m_i^s=\sum_p m_p W_{pi}^s,\quad \bs{v}_i^n= \frac{m_i^s \bs{v}_i^n} {\sum_p m_p W_{pi}^s}
\end{equation}
We also rasterize particle positions to the grid to simplify the theoretical derivations for the deformation gradient in equation~\eqref{eq:MPM_gradient_1} and internal forces in equation~\eqref{eq:EOM1} as shown below:
\begin{equation}
  \bs{q}_i^n=\frac{\sum_p\bs{q}_p^n W_{ij}^s}{\sum_p W_{ij}^s} 
 \label{eq:p2g_position}
\end{equation}
The idea of introducing a first-order Taylor expansion to enhance velocity rasterization is not new and can also be found in APIC~\cite{jiang:2015:apic} and MLS-MPM~\cite{Hu:2018:Moving}. 
%===========================================

\subsection{Grid Momentum Update} 
\subsubsection{Nodal Lagrangian}
We interpolate the Lagrangian $\f{L}_i^{n+1}$ at grid node $i$ using values  associated with its nearby particles $p$ as follows:
\begin{eqnarray}
    \f{L}_i^{n+1}=\sum_p \left[\frac{1}{2}\rho_j^0 \left(\bs{\dot{q}}_p^{n+1}\right)^T\bs{\dot{q}}^{n+1}_p -\Psi\left(\bs{F}_p^{0{n+1}}\right)\right]W_{ip}^s V_p^s
    \label{eq:nodal_L}
\end{eqnarray}
Substituting the expression for $\f{L}_i^{n+1}$ to equations~\eqref{eq:lagrangian} and~\eqref{eq:AUL} gives:
\begin{enumerate}
\item Kinetic term:
\begin{eqnarray}
     \frac{d}{dt}\left(\frac{\p \f{L}_i^{n+1}}{\p \bs{\dot{q}}_i^{n+1}}\right)=\sum_p \rho_p^0 \bs{\dot{q}}_p^{n+1} W_{ip}^s V_p^s =\sum_p \rho_p^0  W_{ip}^s V_p^s\bs{\ddot{q}}_p^{n+1}=m_i^s\bs{\ddot{q}}_p^{n+1}
\end{eqnarray}
\item Deformation term:
\begin{eqnarray}
    \frac{\p \f{L}_i^{n+1}}{\p \bs{q}_i^n}=-\sum_p \frac{\p \Psi(\bs{F}_p^{0{n+1}})}{\p\bs{q}_i^n} W_{ip}^s V_p^s= 
    -\sum_p\mathbb{P}_p^{s}  \frac{\p (\bs{F}_p^{s(n+1)})^T}{\p \bs{q}_i^s} W_{ip}^s V_p^s
\label{eq:defromation}
\end{eqnarray}
where $\mathbb{P}_j^s=\mathbb{P}_j^0(\bs{F}_j^{0s})^T/J_j^{0s}$. 
\end{enumerate}
\begin{figure*}[h]
    \centering
    \includegraphics[width=.98\textwidth]{image/water_in_sphere.png}
     \vspace{-4mm}
     \caption{\textbf{Liquid bunny splash inside a ball.} A liquid bunny is dropped inside a spherical container and undergoes vibrant and dynamic splashes, demonstrating that our proposed $A$-ULMPM framework can capture the rich motion of incompressible fluids similar to existing Eulerian approaches proposed in prior works.}
    \label{fig:water}
    \vspace{-3mm}
\end{figure*}

Thus, the equation of motion for grid node $i$ at time $t_n$ is given by:
\begin{equation}
    m_i^s\bs{\ddot{q}}_i^{n+1}+\sum_j\mathbb{P}_p^s(\bs{F}_p^{0s})^T  \frac{\p (\bs{F}_p^{s(n+1)})^T}{\p \bs{q}_i^s} W_{ip}^s V_j^s=\bs{0}
    \label{eq:dis_L_eq}
\end{equation}
The reader may have noticed that an explicit expression of equation~\eqref{eq:dis_L_eq} relies on a concrete formulation of $\bs{F}_p^{s(n+1)}$ and its derivative ${\p (\bs{F}_p^{s(n+1)})}/{\p \bs{q}_i^s}$. 
For example, in kernel-MPM~\cite{Stomakhin:2013:MPMsnow}, the shape function is unitized to simplify the internal force evaluation in equation~\eqref{eq:dis_L_eq} (see Appendix~\ref{sec:A-ULMPM-kernel}). 

\subsubsection{MLS-based gradient operator}
We use the gradient operator based on \emph{moving least squares} (MLS) that was proposed in~\cite{Xue:2019:NL_L} (see Appendix~\ref{sec:mls_gradient}) that locally minimizes the error of a certain position over its neighborhood.  
Following the same notation in equation~\eqref{eq:dis_L_eq}, the deformation gradient of particle $p$ at time $t_n$ relative to an arbitrary configuration map at time $t_s$ is given as:
\begin{eqnarray}
    \bs{F}_p^{s(n+1)}=\left(\sum_{k}(\bs{q}_k^{n+1}-\bs{q}_p^{n+1})\otimes \bs{r}_{pk}^s W_{pk}^s\right)\left(\sum_{k}\bs{r}_{pk}^s\otimes\bs{r}_{pk}^s W_{pk}^s\right)^{-1}
\label{eq:gradient_qp_0s}
\end{eqnarray}
where $\bs{r}_{pk}^s=\bs{q}_k^s-\bs{q}_p^s$.
\subsubsection{Evaluation of $\p\bs{F}_p^{s(n+1)}/\p\bs{q}_i^s$}
Based on equation~\eqref{eq:dFdq} in Appendix~\ref{sec:mls_gradient}, we evaluate the derivative of $\bs{F}_p^{sn}$ with respect to $\bs{q}_i^s$ as: 
\begin{eqnarray}
\begin{small}
 \frac{\p \bs{F}_p^{s(n+1)}}{\p\bs{q}_i^s}=\left(\sum_{k}\left[\delta_{ik}^{n+1}-\delta_{ip}^{n+1}\right)\otimes \bs{r}_{pk}^s W_{pk}^s \right]\left[\sum_{k}\bs{r}_{pk}^s\otimes \bs{r}_{pk}^s W_{pk}^s V_k^s \right]^{-1}
 \end{small}
 \label{eq:dFjdqi}
\end{eqnarray}
\subsubsection{MPM Equation of Motion with Arbitrary Updated Lagrangian}
Substituting equation~\eqref{eq:dFjdqi} to equation~\eqref{eq:dis_L_eq} yields the following equation of motion for grid node $i$ at time $t_n$:
\begin{equation}
m_i^s\bs{\ddot{q}}_i^{n+1}+
\sum_p\left[\mathbb{P}_p^s\mathbb{K}_p^s+\mathbb{P}_i^s\mathbb{K}_i^s\right]\bs{r}_{ip}^s W_{ip}^s V_p^s=\bs{0}
\label{eq:EOM1}
\vspace{-3mm}
\end{equation}
where $\mathbb{P}_p^s=\mathbb{P}_p^0(\bs{F}_p^{s(n+1)})^T$ and $\mathbb{P}_i^s=\mathbb{P}_i^0(\bs{F}_i^{s(n+1)})^T$. $\mathbb{K}_p^s$ and $\mathbb{K}_i^s$ are given by:
\begin{equation*}
    \mathbb{K}_p^s=\left(\sum_i \bs{r}_{pi}^s\otimes\bs{r}_{pi}^s W_{pi}^s V_i^s\right)^{-1},\quad \mathbb{K}_i^s=\left(\sum_p \bs{r}_{ip}^s\otimes\bs{r}_{ip}^s W_{ip}^s V_p^s\right)^{-1}. 
\end{equation*}
We take advantage of equation~\eqref{eq:p2g_position} to simplify the term $\mathbb{P}_i^0(\bs{F}_i^{0s})^T\mathbb{K}_i^s$ in equation~\eqref{eq:EOM1} as follows:
$$\sum_p\mathbb{P}_i^s\mathbb{K}_i^s\bs{r}_{ip}^sW_{ip}^sV_p^s=\mathbb{P}_i^s\mathbb{K}_i^s\sum_p\bs{r}_{ip}^sW_{ip}^sV_p^s=0$$ %\left({\sum_p \bs{q}_p^s W_{ip}^s}-{\sum_p \bs{q}_i^sW_{ip}^s}\right)V_p^0=\bs{0}.$$
Setting $V_p^s=J_p^sV_p^0$, equation~\eqref{eq:EOM1} can be rewritten as follows:
\begin{eqnarray}
m_i^s\bs{\ddot{q}}_i^{n+1}+
\sum_p \mathbb{P}_p^0(\bs{F}_p^{0s})^T\mathbb{K}_p^s\bs{r}_{ip}^s W_{ip}^s V_p^0=\bs{0}
\label{eq:EOM2}
\vspace{-3mm}
\end{eqnarray}
Equation~\eqref{eq:EOM2} is a general MPM formulation that allows the use of arbitrary intermediate configurations. It spans from total Lagrangian formulations to updated Lagrangian formulations. Specifically:
\begin{enumerate}
    \item Setting $s=0$ yields total Lagrangian MPM:
\begin{eqnarray}
m_i^0\bs{\ddot{q}}_i^{n+1}+
\sum_p \mathbb{P}_p^0\mathbb{K}_p^0\bs{r}_{ip}^0 W_{ip}^0 V_p^0=\bs{0}
\label{eq:TLMPM}
\end{eqnarray}

\item Setting $s=n+1$ yields Eulerian MPM:

\begin{eqnarray}
m_i^n\bs{\ddot{q}}_i^{n+1}+
\sum_p \mathbb{P}_p^0(\bs{F}_p^{0(n+1)})^T\mathbb{K}_p^{n+1}\bs{r}_{ip}^{n+1} W_{ip}^{n+1} V_p^0=\bs{0}
\label{eq:EMPM}
\end{eqnarray}
\end{enumerate}
In the total Lagrangian formulation, there is no need to update $W_{ip}$, $\bs{r}_{ip}$, $m_i$, and $\mathbb{K}_p$ matrices in equation~\eqref{eq:TLMPM}, while they require an update at every time step in the Eulerian setting (see equation~\eqref{eq:EMPM}). 
\begin{figure*}[h!]
    \centering
    \includegraphics[width=\textwidth]{image/bunnies_MLS_EMPM.png}\\
    \includegraphics[width=\textwidth]{image/bunnies_MLS_TL.png}\\
     \vspace{-4mm}
     \caption{\textbf{Hyperelastic bunny yo-yo.} Under the effects of gravity, a hyperelastic bunny yo-yo breaks mid-way due to numerical fracture, when simulated with MLS-EMPM (top), while MLS-$A$-ULMPM (bottom) robustly captures the stretching motion of the elastic cord to pull the bunnies back upwards.}
    \label{fig:strechy_bunny_1}
        \vspace{-3mm}
\end{figure*}
\vspace{-3mm}
\subsection{Grid Velocity and Position Update}
With explicit time stepping, the velocity is updated as $\bs{v}_i^{n+1}=\bs{\hat{v}}_i^{n+1}$, where $\bs{\hat{v}}_i^{n+1}$ is given by: 
\begin{equation}
\bs{\hat{v}}_i^{n+1}=\bs{v}_i^n-\frac{\dt}{m_i}\sum_p\mathbb{P}_p^0(\bs{F}_p^{0s})^T\mathbb{K}_p^s\bs{r}_{ip}^s W_{ip}^s V_j^0\\
\label{eq:grid_Vel_update_explicit}
\end{equation}
For a semi-implicit update, we follow~\cite{Stomakhin:2013:MPMsnow, Hu:2018:Moving} and take an implicit step on the velocity update by utilizing the Hessian of $\f{L}^{n+1}$ with respective to $\bs{q}_i^{n+1}$. 
The action of this Hessian on an arbitrary increment $\delta \bs{q}^s$ is given as follows:
 \begin{equation}
     -\delta \bs{f}_i=\sum_pV_p^0 \mathbb{A}_p^s(\bs{F}_p^{0s})^T\mathbb{K}_p^s\bs{r}_{ip}^s W_{ip}^s V_j^0
     \label{eq:Hessian_L_derivative}
 \end{equation}
 where $\mathbb{A}_p^s$ is given by: 
 \begin{equation}
     \mathbb{A}_p^s=\frac{\p ^2 \f{L}^{n+1}}{\p \bs{F}_p^{0s} \p\bs{F}_p^{0s}}:\sum_j\delta \bs{q}_j^s \mathbb{K}_p^s\bs{r}_{ip}^s W_{ip}^s V_j^0 \bs{F}_p^{0s}.
 \end{equation}
 We linearize the implicit system with one step of Newton's method, which provides the following symmetric system for $\bs{\hat{v}}_i^{n+1}$:
 \begin{equation}
 \sum_j\left(\bs{I}\delta_{ij}+\frac{\dt^2}{m_i}\frac{\p ^2 \f{L}^{n+1}}{\p \bs{q}_i^s \p\bs{q}_j^s}\right) \bs{v}_j^{n+1}=\bs{\hat{v}}_i^{n+1}
 %\sum_j\left(\bs{I}\delta_{ij}-\frac{\dt^2}{m_i}\frac{\p \bs{f}_i^n}{ \p\bs{q}_j}\right)\bs{v}_j^{n+1}=\bs{\hat{v}}_i^{n+1}
 \label{eq:grid_Vel_update_implicit}
 \end{equation}
 where $\bs{I}$ is the identity matrix and $\bs{\hat{v}}_i^{n+1}$ is given in equation~\eqref{eq:grid_Vel_update_explicit}. We update rasterized positions at grid nodes as shown below:
 \begin{equation}
  \bs{q}_i^{n+1}=\bs{q}_i^n+\dt\bs{{v}}_i^{n+1}
 \end{equation}

\subsection{G2P Velocity and Position Transfer} 
We project $\bs{q}_i^{n+1}$ to the global grid and process grid-based collisions~\cite{Stomakhin:2013:MPMsnow} to compute $\bs{v}_i^{n+1}$, which is transferred back to particles using the APIC method~\cite{jiang:2015:apic}. %More specifics can be found in~\cite{Jiang:2017:APIC}. 
The specific updates for $\bs{q}_p^{n+1}$ and $\bs{v}_p^{n+1}$ are given below:
\begin{equation}
    \bs{v}_p^{n+1}=\sum_i\bs{v}_i^{n+1}W_{pi}^s,\quad \bs{q}_p^{n+1}=\bs{q}_p^n+\dt \bs{v}_p^{n+1}
\end{equation}
\subsection{Update Particle Deformation Gradients}
We first update particle deformation gradients with respect to the latest configuration at time $t_s$ and use the following MLS-based gradient operator (see equation~\eqref{eq:gradient_qp_0s}):
\begin{equation}
    \bs{F}_p^{s(n+1)}=\frac{\p\bs{q}^{n+1}_p}{\p\bs{q}^s_p}=
    \left(\sum_{i}(\bs{q}_i^{n+1}-\bs{q}_p^{n+1})\otimes \bs{r}_{ip}^s W_{pi}^s V_i^s\right)\mathbb{K}_p^s
\label{eq:MPM_gradient_1}
\end{equation}
Substituting $\bs{q}_i^{n+1}=\bs{q}_i^n+\dt\bs{v}_i^{n+1}$ and $\bs{q}_p^{n+1}=\bs{q}_i^n+\dt\bs{v}_p^{n+1}$ to equation~\eqref{eq:MPM_gradient_1} gives:
\begin{eqnarray}
\begin{split}
  \bs{F}_p^{s(n+1)}=  &\sum_i\left[\left(\bs{q}_i^n+\bs{v}_i^{n+1}\dt\right)-\left(\bs{q}_p^n+\dt\bs{v}_p^{n+1}\right)\right]\otimes \bs{r}_{pi}^s W_{pi}^s V_i^s \mathbb{K}_p^s\\
   % &=\sum_i\left(\bs{q}_i^n-\bs{q}_p^n\right)\otimes \bs{r}_{pi}^s W_{pi}^s V_i^s \mathbb{K}_p^s+\dt\sum_i\left(\bs{v}_i^{n+1}-\bs{v}_p^{n+1}\right)\otimes \bs{r}_{pi}^s W_{pi}^s V_i^s \mathbb{K}_p^s \\
   =  &\bs{F}_p^{sn}+\dt\nabla_s\bs{v}_p^{n+1}\\
\end{split}
\label{eq:MPM_gradient_s(n+1)}
\end{eqnarray}
where $\nabla_s\bs{v}_p^{n+1}$ is given by:
\begin{equation}
\nabla_s\bs{v}_p^{n+1}=\sum_i(\bs{v}_i^{n+1}-\bs{v}_p^{n+1})\otimes \bs{r}_{pi}^s W_{pi}^s V_i^s \mathbb{K}_p^s
\label{eq:gradientVp}
\end{equation}
Using the chain rule, the update for $\bs{F}_p^{0(n+1)}$ is given by:
\begin{equation}
    \bs{F}_p^{0(n+1)}=\frac{\p \bs{q}_p^{n+1}}{\p\bs{q}_p^{s}}\frac{\p\bs{q}_p^{s}}{\p\bs{q}_p^{0}}
    =\bs{F}_p^{s(n+1)}\bs{F}_p^{0s}
\end{equation}

\begin{figure*}
    \centering
    \includegraphics[width=\textwidth]{image/fsi_v2_4.jpg}
    \includegraphics[width=\textwidth]{image/fsi_v1_4.jpg}
    \vspace{-7.5mm}
    \caption{\textbf{Liquid bunny falling on a hyperelastic bowl.} (Top) Our proposed $A$-ULMPM framework can robustly capture the vivid dynamic responses of fluid-solid interactions and preserves the bowl shape. (Bottom) In contrast, the bowl fractures and fails to hold the water when simulated with MLS-EMPM.}.
    \label{fig:FSI}
    \vspace{-6mm}
\end{figure*}

\subsection{Update Grid Configuration Map}\label{sec:update_configuration}
We designed the configuration update criterion based on an intuitive assumption that more severe deformation is accompanied with large change of $J$ in the next time step $t^{n+1}$ with respect to the latest configuration $t^s$. 
Using this observation, our criterion is defined as a measure for the amount of particle deformation as follows:
\begin{equation}
    \delta J_p=\|J_p^{s(n+1)}-J_p^{ss}\|
    \label{eq:update_conf}
\end{equation}
where $J_p^{s{n+1}}=\text{det}(\bs{F}_p^{s(n+1)})$ and $J_p^{ss}=1$. 
We mark particles with $\delta J_p\geq \epsilon$ as indicators where large deformation is taking place and count the total amount of marked particles ($n_{mp}$). If $n_{mp}/n_p\geq \eta$, where $n_p$ is the total number of particles, we update $W_{ip}^s$, $\mathbb{K}_p^s$, and $\bs{F}_p^{0s}$. Otherwise, these variables remain the same until a new update occurs. $\epsilon$ and $\eta$ are user-defined parameters to adjust the update frequency. 
%{For solid simulations, we set $\epsilon=0.5$ and $\eta=0.5$, while for fluid simulations, we set $\epsilon=0.1$ and $\eta=0.01$.}
\subsection{Similarities with  MLS-EMPM and APIC}
While our $A$-ULMPM framework is new to computer graphics, setting $s=n$ for the configuration map shares some similarities with MLS-EMPM~\cite{Hu:2018:Moving} and APIC~\cite{jiang:2015:apic}. 
We prove that the MLS-gradient of velocity is equivalent to the APIC-gradient of velocity by leveraging the properties of interpolation weights $\sum_i \bs{r}_{pi} W_{pi} =\bs{0}$~\cite{Jiang:2017:APIC} as follows:
\begin{eqnarray}
\begin{split}
    \nabla\bs{v}_p=&\sum_i(\bs{v}_i-\bs{v}_p)\otimes\bs{r}_{pi} W_{ip}V_i
    \mathbb{K}_p\\
    =&\underbrace{\sum_i\bs{v}_i\otimes\bs{r}_{pi} W_{ip}V_i
    \mathbb{K}_p}_{\textbf{APIC:} \nabla \bs{v}_p}-\underbrace{\bs{v}_p\otimes\sum_i\bs{r}_{pi} W_{ip}V_i
    \mathbb{K}_p}_{\bs{0}}
\end{split}
\end{eqnarray}
Our internal force in equation~\eqref{eq:grid_Vel_update_explicit} has the same pattern as that in original MLS-EMPM~\cite{Hu:2018:Moving}, which is derived from the weak-form element-free Galerkin (EFG) framework. The $\mathbb{K}_p$ matrices are simplified by $\frac{4}{\dx^2}\bs{I}$ for quadratic $W_{ip}$ and $\frac{3}{\dx^2}\bs{I}$ for cubic $W_{ip}$. This simplification comes from the properties of splines, and is not generally true for other interpolation functions. 
\vspace{-4mm}

%\tx{Only when particles are located at cell center, our $\mathbb{K}_p$ recovers the above two  cases.}
