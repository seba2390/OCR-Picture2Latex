\section{Overview of Different Formulations for Continuum Mechanics}
\label{sec:governing_equation}
In this section, we briefly revisit the Lagrangian framework from continuum mechanics for describing the governing equations of motion and summarize the different formulations that can be derived depending on the choice of the reference configuration. 
%We then introduce an intermediate configuration at $t=t_s$ between the initial configuration $t=t_0$ and the current configuration $t=t_n$ as shown in Figure~\ref{fig:configrations}, and derive the associated governing equation at $t_s$ for continuum mechanics aiming to present  an arbitrary updated Lagrangian expression of continuum mechanics. 
\begin{figure}[h]
\centering
\begin{overpic}[width=.45\textwidth]{image/ball_impact_2.png}
    \put(8,0){$t=t_0$: $\bs{q}_0$}
    \put(42,0){$t=t_s$: $\bs{q}_s$}
     \put(75,0){$t=t_n$: $\bs{q}_n$}
    \put(25,35){$\bs{F}_{0s}=\frac{\p\bs{q}_s}{\p\bs{q}_0}$}
    \put(55,35){$\bs{F}_{sn}=\frac{\p\bs{q}_n}{\p\bs{q}_s}$}
    \put(40,43){$\bs{F}_{0n}=\frac{\p\bs{q}_n}{\p\bs{q}_0}$}
\end{overpic}
\vspace{-2mm}
\caption{\textbf{Elastic ball.} The deformation gradient can be described with respect to the initial configuration (total Lagrangian formulation) at time $t_0$ as $\bs{F}_{0s}$ and $\bs{F}_{0n}$, or with respect to an intermediate configuration at time $t_s$ (updated Lagrangian formulation) as $\bs{F}_{sn}$, where $\bs{q}$ is the displacement.}
\label{fig:configrations}
\vspace{-4mm}
\end{figure}
\subsection{Lagrangian Framework}
The Lagrangian $\f{L}$ for holonomic systems is defined as:
\begin{equation}
\f{L}(\bs{q},\dot{\bs{q}})=\f{K}(\bs{\dot{q}})-\f{U}(\bs{q}) %,\:\:\f{U}=\Pi_{int}+\Pi_{ext}
\end{equation}
where $\bs{q}$ is the generalized displacement, $\bs{\dot{q}}$ is the generalized velocity, $\f{K}(\dot{\bs{q}})$ is the kinetic energy and $\f{U}(\bs{q})$ is the potential energy. By omitting the energy due to external body forces and traction for simplicity, the kinetic and potential energies can be defined as:
\begin{equation}
\f{K}(\bs{\dot{q}})=\frac{1}{2}\int_{\f{B}}\bs{\dot{q}}^T\bs{\dot{q}} dV,\quad
\f{U}(\bs{q})=\int_{\f{B}}\rho_0\Psi(\bs{F})dV
\end{equation}
where $\rho_0$ is the material density at time $t_0$, 
$\Psi(\bs{F})$ denotes the Helmholtz free energy per unit mass in homogeneous materials, and $\bs{F}$ is the deformation gradient tensor. Consequently, the Lagrangian density function can be defined as follows:
\begin{equation}
     \bar{\f{L}}(\bs{q},\bs{\dot{q}}, \bs{F})=\frac{1}{2}\rho_0\bs{\dot{q}}^T\bs{\dot{q}} -\Psi(\bs{F})
\end{equation}
Based on the Lagrangian framework~\cite{Zienkiewicz:1977:FEM}, the governing equations of motion at time $t_n$ can be described as:
\begin{equation}
   \frac{d}{dt}\left(\frac{\p \bar{\f{L}}}{\p \bs{\dot{q}}_n}\right)-\frac{\p \bar{\f{L}}}{\p \bs{q}_n}=\bs{0}
    \label{eq:lagrangian}
\end{equation}
where
\begin{equation}
\frac{d}{dt}\left(\frac{\p \bar{\f{L}}}{\p \bs{\dot{q}}_n}\right)=\rho_0 \bs{\ddot{q}}_n, \quad \frac{\p \bar{\f{L}}}{\p \bs{q}_n}=\frac{\p\Psi(\bs{F})}{\p\bs{F}}\frac{\p\bs{F}}{\p \bs{q}_n}.%=\mathbb{P}\frac{\p\bs{F}}{\p \bs{q}},
\end{equation}
Note that the term ${\p\Psi(\bs{F})}/{\p\bs{F}}$ represents a stress tensor that is determined by the material constitutive model, and the term ${\p \bs{F}}/{\p\bs{q}}$ can be further expressed in terms of a divergence operator. These two terms together define the internal force as the material deforms. %We leave the $\frac{\p\Psi(\bs{F})}{\p\bs{F}}$ and  $\frac{\p\bs{F}}{\p \bs{q}}$ in equation~\eqref{}
%and $\mathbb{P}$ represents the \emph{first Piola–Kirchhoff stress}. 

\subsection{Total Lagrangian Formulation}

In this formulation, the stress and strain in the material are measured relative to the original configuration at time $t_0$ (see Figure~\ref{fig:configrations}), such that the deformation gradient tensor $\bs{F}$ is the displacement derivative of $\bs{q}_n$ with respect to $\bs{q}_0$. Substituting $\bs{F}_{0n}$ to $\p \bar{\f{L}}/\p \bs{q}_n$ gives:
 $$\frac{\p \bar{\f{L}}}{\p \bs{q}_n}=\frac{\p\Psi(\bs{F})}{\p\bs{F}_{0n}}\frac{\p}{\p \bs{q}_n}\left(\frac{\p \bs{q}_n}{\p \bs{q}_0}\right)=\nabla_0\cdot\mathbb{P}$$
 and have the following governing equation of motion:
    \begin{equation}
      \rho_0 \bs{\ddot{q}}_n=\nabla_{0}\cdot\mathbb{P}_0 
      \label{eq:Lagrangian GE}
    \end{equation}
where the divergence operator $\nabla_0$ is also evaluated with respect to the original configuration at time $t_0$. In the total Lagrangian formulation, the stress tensor $\mathbb{P}_0$ is the \emph{first Piola–Kirchhoff stress}. 

\subsection{Eulerian Formulation}

In this formulation, the stress and strain in the material are measured relative to the current configuration at time $t_n$ (see Figure~\ref{fig:configrations}). Thus, the expression for ${\p \bar{\f{L}}}/{\p \bs{q}}$ can be expanded as follows:
$$\frac{\p \bar{\f{L}}}{\p \bs{q}_n}=
\frac{\p\Psi(\bs{F})}{\p\bs{F}_{0n}}\frac{\p \bs{F}_{0n}}{\p \bs{q}_n}=
\frac{\p}{\p \bs{q}_n}\left(\frac{\p\Psi(\bs{F})}{\p\bs{F}_{0n}}\frac{\p \bs{q}_n}{\p \bs{q}_0}\right)=\nabla_n\cdot\left(\mathbb{P}_0\bs{F}_{0n}^T\right)$$
Besides, $\rho_0$ can be mapped to $\rho_n$ using the relation $\rho_0=J_{0n}\rho_n$, where $\rho_n$ is the density at time $t_n$ and $J_{0n}=\text{det}(\bs{F}_{0n})$. Consequently, the Eulerian formulation gives the following equation of motion:
\begin{equation}
    \rho_n\bs{\ddot{q}}=\nabla_n\cdot \mathbb{P}_n%{\left(\frac{1}{J_{0n}}\mathbb{P}_0\bs{F}_{0n}^T\right)}
    \label{eq:Euleraian GE}
\end{equation}
where $\mathbb{P}_n=\mathbb{P}_0\bs{F}_{0n}^T/{J_{0n}}$ provides the definition of the \emph{Cauchy stress}.

\subsection{Arbitrary Updated Lagrangian Formulation}
\label{sec:arbitrary-updated-lagrangian}

A general formulation for measuring stress and strain with respect to an arbitrary reference configuration at time $t_s$ has been derived in~\cite{Zienkiewicz:1977:FEM, Shabana:2018:computational}. In this formulation, the expression for ${\p \bar{\f{L}}}/{\p \bs{q}}$ can be expanded as follows:
%\begin{eqnarray}
%\frac{\p \bar{\f{L}}}{\p \bs{q}_n}=
%\frac{\p\Psi(\bs{F})}{\p\bs{F}_{0n}}\frac{\p \bs{F}_{0n}}{\p \bs{q}_n}=
%\frac{\p}{\p \bs{q}_s}\left(\frac{\p\Psi(\bs{F})}{\p\bs{F}_{0n}}\frac{\p \bs{q}_n}{\p \bs{q}_0}\frac{\p \bs{q}_s}{\p\bs{q}_n}\right)=\nabla_s\cdot\left(\mathbb{P}_0\bs{F}_{0s}^T\right)
%\label{eq:AUL}
%\end{eqnarray}
\begin{eqnarray}
\frac{\p \bar{\f{L}}}{\p \bs{q}_n}=
\frac{\p\Psi(\bs{F})}{\p\bs{F}_{0n}}\frac{\p \bs{F}_{0n}}{\p \bs{q}_n}=
\frac{\p\Psi(\bs{F})}{\p\bs{F}_{0n}}\frac{\p \bs{F}_{0n}}{\p \bs{F}_{sn}}\frac{\p \bs{F}_{sn}}{\p \bs{q}_n}=\nabla_s\cdot\left(\mathbb{P}_0\bs{F}_{0s}^T\right)
\label{eq:AUL}
\end{eqnarray}
Similar to the Eulerian formulation, we map $\rho_0$ to $\rho_s$ using the relation $\rho_s=J_{0s}\rho_s$, where $\rho_s$ represents the density at time $t_s$ and $J_{0s}=\text{det}(\bs{F}_{0s})$. Consequently, the arbitrary updated Lagrangian formulation gives the following governing equation of motion:
\begin{equation}
    \rho_s\bs{\ddot{q}}=\nabla_s\cdot\mathbb{P}_s%{\left(\frac{1}{J_{0s}}\mathbb{P}_0\bs{F}_{0s}^T\right)}
    \label{eq:generalized EG}
\end{equation}
where $\mathbb{P}_s=\mathbb{P}_0\bs{F}_{0s}^T/{J_{0s}}$ defines a stress measured at time $t_s$. 
By setting $s=0$ and using the defining properties of the initial configuration $J_{00}=1$ and $\bs{F}_{00}=\bs{I}$, where $\bs{I}$ is the identity matrix, equation~\eqref{eq:generalized EG} recovers the total Lagrangian formulation in equation~\eqref{eq:Lagrangian GE}. Likewise, setting $s=n$ yields the Eulerian formulation in equation~\eqref{eq:Euleraian GE}. 


\begin{comment}
\tx{Known $\bs{\sigma}=p\bs{I}$, where $p$-pressure at $t_n$}
\begin{equation}
    \rho_n\bs{\ddot{q}}=\nabla_n\cdot\underbrace{\left(\frac{1}{J_{0n}}\mathbb{P}_0\bs{F}_{0n}^T\right)}_{\bs{\sigma}-in-MPM}
    \label{eq:Euleraian GE}
\end{equation}
\begin{enumerate}
\item Total Lagrangian
 \begin{equation}
      \rho_0 \bs{\ddot{q}}_n=\nabla_{0}\cdot\mathbb{P}_0 
      \label{eq:Lagrangian GE}
    \end{equation}
$$\sum_p V_p J_{0n}\bs{\sigma}\bs{F}_{on}^{-T} \nabla_0 W_{ij}^0$$

\item Updated Lagrangian--reference $t=t_s$
\begin{equation}
    \rho_s\bs{\ddot{q}}=\nabla_s\cdot{\left(\frac{1}{J_{0s}}\mathbb{P}_0\bs{F}_{0s}^T\right)}
    \label{eq:generalized EG}
\end{equation}
$$\frac{1}{J_{0s}}J_{0n}\bs{\sigma}\bs{F}_{on}^{-T}\bs{F}_{0s}^T$$
$$\frac{J_{0n}}{J_{0s}}\bs{\sigma}\frac{\p \bs{q}_0}{\p \bs{q}_n}\frac{\p \bs{q}_s}{\p \bs{q}_0}$$
$$\sum_p V_p \det(\bs{F}_{sn})\bs{\sigma}\underbrace{\frac{\p \bs{q}_n}{\p \bs{q}_s}^{-T}}_{\bs{F}_{sn}^{-T}}  \nabla_s W_{ij}^s$$

$$\bs{q}_{n+1}=\bs{q}_n+\bs{v}_n\dt$$
$$\frac{\p \bs{q}_{n+1}}{\p \bs{q}_s}=\frac{\p\bs{q}_s+\Delta\bs{q}}{\p \bs{q}_s}+\nabla_s\bs{v}_n\dt=\bs{F}_{sn}+\nabla_s\bs{v}_n\dt$$

$$\bs{F}_{s,n+1}=\frac{\p\bs{q}_n}{\p \bs{q}_s}+\nabla_s\bs{v}_n\dt=\bs{F}_{sn}+\nabla_s\bs{v}_n\dt$$
\item reference $t=0$
$$\bs{F}_{0,n+1}=\underbrace{(\bs{F}_{sn}+\nabla_s\bs{v}_n\dt)}_{\frac{\p\bs{q}_{n+1}}{\p\bs{q}_s}}\frac{\p \bs{q}_s}{\p \bs{q}_0}$$
\end{enumerate}
\end{comment}

