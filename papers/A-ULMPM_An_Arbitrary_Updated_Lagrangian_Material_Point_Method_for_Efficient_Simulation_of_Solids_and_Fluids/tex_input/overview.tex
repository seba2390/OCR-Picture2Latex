\section{Method Overview}
\label{sec:overview}
%
\begin{figure}[h]
%\vspace{-3mm}
\centering
\includegraphics[width=\columnwidth]{image/AULMPMoverview.jpg}
\vspace{-7mm}
\caption{Overview of our proposed method.
The red and green arrows represent the velocity and force vectors at the nodes of the background grid. The blue grids represent the previous configuration while the green grids denote the new configuration according to our criterion for updating the state.}
\label{fig:A_ULMPM_overview}
\vspace{-3mm}
\end{figure}
%
\begin{figure*}[h]
    \vspace{-4mm}
    \centering
    \includegraphics[width=\textwidth]{image/EMPM_Inversible.png} \\
   MLS-EMPM: $t=\{0.0008,\:0.0026,\:0.0032,\:0.0035,\: 0.04,\:0.1\}$s\\
    \includegraphics[width=\textwidth]{image/TL_Inversible.png}
    MLS-$A$-ULMPM: $t=\{0.0008,\:0.0026,\:0.0032,\:0.0035,\: 0.04,\:0.1\}$s\\
    \vspace{-4mm}
    \caption{The kinematic constraint on one end of the beam is released after a certain time. MLS-EMPM (row 1) fails to recover from the twisting deformation, while  MLS-$A$-ULMPM (row 2) produces realistic elastic response and recovers from the elastic deformation induced by the pulling and twisting motion.} 
    \label{fig:bar_twisting_inv_deformation}
    \vspace{-5mm}
\end{figure*}
%
Our method introduces a nodal discretized Lagrangian into hybrid grid-to-particle discretization for MPM and takes advantage of the zero cell-crossing error and zero numerical fracture properties of TLMPM. It also allows for updating the reference configuration in an adaptive manner to ensure robustness during extreme material deformations, such as those arising in fluid simulations. 
In contrast to prior MPM formulations (see equations~\eqref{eq:Lagrangian GE} and~\eqref{eq:Euleraian GE}), our method adopts adaptive reference configurations (see grids at time $t_s$ in Figure~\ref{fig:configrations}) to measure stress, strain, interpolation kernels and their derivatives. Data from particles is first transferred to grids (P2G) in their latest configuration. 
Next, the forces are evaluated and the velocities are updated on these grids. Subsequently, the updated grid velocities are interpolated back to the particles (G2P) and used to update the particle positions following the APIC method~\cite{jiang:2015:apic}. 
The velocity gradients and deformation gradients are then updated on the particles leveraging information on the grids. 
Our method uses a novel criterion (see equation~\eqref{eq:update_conf}) for automatically determining when to update the grid configuration. 
This prevents data transfers between the particles and grids from accumulating errors as the interpolation functions are only updated when necessary. Figure~\ref{fig:A_ULMPM_overview} provides a high-level overview of our method and the essential algorithmic steps are summarized below: % 
\begin{enumerate}
\item{\textbf{P2G Rasterization:}}
Velocities on particles are reconstructed by first-order Taylor expansion, and mass and momentum from particles at time $t_n$ are transferred to grids at time $t_s$. 
\item{\textbf{Grid Momentum Update:}} 
Grid momentum is updated using explicit or implicit schemes. Note that forces are evaluated with respect to the latest configuration at time $t_s$. 
\item{\textbf{G2P Velocity Transfer:}}
Use APIC~\cite{jiang:2015:apic} to transfer velocities from the grid to particles. 
\item{\textbf{Update Particle Positions:}}
Particle positions are updated with their new velocities.
\item{\textbf{Update Particle Deformation Gradients:}}
Use the MLS gradient operator to update the particle deformation gradient and account for plasticity, if it occurs.
\item{\textbf{Update Grid Configuration:}}
Check if the deformation is extreme according to the update criterion in equation~\eqref{eq:update_conf}. In case of a large deformation, update weights between particles and the grid and the $\mathbb{K}$ matrices on particles. 
\end{enumerate} 

\subsection{Terminology}
We show integration of our arbitrary updated Lagrangian formulation with Kernel-MPM~\cite{Stomakhin:2013:MPMsnow} in Appendix~\ref{sec:A-ULMPM-kernel} and MLS-MPM~\cite{Hu:2018:Moving} in Section~\ref{sec:A_UL_MLS_MPM}. In Kernel-MPM, the interpolation is performed using shape functions and stress divergence is evaluated by derivatives of the shape function, while MLS-MPM utilizes APIC particle-to-grid velocity rasterization and uses MLS shape functions to derive the internal force term. Using these definitions, our terminology for different methods is provided as follows:
\begin{enumerate}
    %\vspace{-1mm}
    \item {\textbf{Kernel-MPM}:}
    In kernel-MPM, we have kernel-EMPM and kernel-TLMPM represent the kernel-MPM in Eulerian and total Lagrangian frameworks, respectively. Kernel-EMPM was presented in~\cite{Stomakhin:2013:MPMsnow} and kernel-TLMPM was present in~\cite{De:2020:TLMPM}. These two methods can be recovered in our kernel-$A$-ULMPM framework by setting $s=n$ for EMPM and $s=0$ for TLMPM. 
    \item{\textbf{MLS-MPM}:}
    In MLS-MPM, we have MLS-EMPM and MLS-TLMPM represent MLS-MPM in Eulerian and total Lagrangian frameworks, respectively. 
    MLS-EMPM was presented in MLS-MPM~\cite{Hu:2018:Moving}.
    Besides, MLS-EMPM and MLS-TLMPM can be recovered in MLS-$A$-ULMPM by setting $s=n$ for MLS-EMPM and $s=0$ for MLS-TLMPM. 
\end{enumerate}
\begin{table}[h!]
\caption{Physical quantities stored on particles and grid nodes.}
\vspace{-3mm}
\begin{tabular}{ccc}
\hline
\textbf{Particle} & \textbf{Description} & \textbf{Grid}\\
\hline
$\bs{q}_p$ & position & $\bs{q}_i$ \\
$\bs{v}_p$ & velocity & $\bs{v}_i$ \\
\multirow{2}{*}{$\bs{F}_p^{0n}$} & deformation gradient at $t_n$ \\
                                & with respect to configuration at $t_0$&{$--$}\\
\multirow{2}{*}{$\bs{F}_p^{0s}$} & deformation gradient at $t_s$ \\
                                & with respect to configuration at $t_0$&$--$\\
\multirow{2}{*}{$\bs{F}_p^{sn}$} & deformation gradient at $t_n$ \\
                                & with respect to configuration at $t_s$&$--$\\
%$\bs{F}_p^{sn}$ & deformation gradient at $t_n$  w.r.t. Conf. at  $t_s$&$--$\\
$\mathbb{P}_p^{s}$ &Stress tensor &$--$\\
\multirow{2}{*}{$J_p^{sn}$}& volume change at $t_n$  \\
                           & with respect to configuration at $t_s$&$--$\\
$--$ & force & $\bs{f}_i$  \\
$V_p$ & volume & $--$ \\
$m_p$ &  mass & $m_i$ \\
%$\tl{\square}_p$& algorithmic qualities& $\tl{\square}_i$\\
\hline
\end{tabular}
\label{tab:notation}
\vspace{-7mm}
\end{table}