\section{RELATED WORK}
\label{sec:related-work}

In this work, we only review prior work related to MPM~\cite{jiang2016material} since our focus is on MPM. However, we note that there are several established methods for particle-based simulations, including SPH~\cite{Desbrun:1996:SPH}, position-based dynamics~\cite{Muller:2004:Point-based,muller:2007:pbd,macklin:2014:PBDs}, linear complementarity formulations~\cite{Erleben:2013:LCP}, and geometric computing techniques~\cite{deGoes:2015:PPIFS,Sin:2009:Voronoi}.

\subsection{MPM in Graphics}
The MPM discretization has been widely adopted in computer graphics applications~\cite{jiang2016material}. 
The seminal work of Zhu and Bridson~\cite{Zhu:2005:Animating} first introduced the FLIP method for sand simulation. Subsequent works further explored its strength in simulating a broader spectrum of material behaviors including snow~\cite{Stomakhin:2013:MPMsnow}, granular materials~\cite{Daviet:2016:Semi-implicit,Klar:2016:Drucker,Tampubolon:2017:Multi,gao2018animating}, foam~\cite{Ram:2015:Material,yue:2015:mpm-plasticity}, complex fluids~\cite{Fang:2019:Silly,Gao:2017:AGIMPM}, cloth, hair and fiber collisions~\cite{Jiang:2017:Anisotropic,Fei:2017:MSLHI,fei:2018:cloth}, fracture~\cite{Wolper:2019:CD-MPM,Wolper:2020:anisompm} and phase change~\cite{Stomakhin:2014:Augmented,gao2018gpu,Su:2021:USOSVL}. We also note the related works of~\cite{mcadams:2009:incomp} for hair simulation, \cite{sifakis:2008:incomp} for cloth simulation, \cite{narain:2010:sand} for sand simulation and~\cite{patkar:2013:bubble} for bubble simulation, which bear similarities to MPM due to their hybrid nature.
Various works have improved or modified aspects of the standard MPM techniques commonly used in graphics~\cite{Fang:2020:IQ-MPM,Yue:2018:HG,XUE:2020:NF, Ding:2019:Thermomechanical}.
Among them, notably, Jiang et al.~\shortcite{jiang:2015:apic,Jiang:2017:APIC} proposed an Affine Particle-In-Cell (APIC) approach that conserves angular momentum and prevents visual artifacts such as noise, instability, clumping and volume loss/gain existing in both FLIP and PIC methods. Furthermore, APIC was enhanced in~\cite{Fu:2017:PolyPIC,Hu:2018:Moving} to improve the kinetic energy conservation in particle/grid transfers. 
\begin{figure*}[h]
    \vspace{-4mm}
    \centering
    \includegraphics[width=\textwidth]{image/Twist_Affine_EMPM_2.png}\\
     MLS-EMPM: Time instants from left to right $t=\{0,\:0.0005,\:0.0009,\:0.0016,\: 0.002,\:0.0027\}$s\\
    \includegraphics[width=\textwidth]{image/Twist_Affine_TL_2.png}\\
     MLS-$A$-ULMPM:  Time instants from left to right $t=\{0,\:0.0005,\:0.0009,\:0.0016,\: 0.002,\:0.0027\}$s\\
     \vspace{-4mm}
    \caption{\textbf{3D Twisting column.} The two ends of a rectangular beam are kinematically separated while twisting the beam. Standard MLS-EMPM (top row) suffers from severe numerical fracture. In contrast, our MLS-$A$-ULMPM (bottom row) nicely captures the rich twisting surface and preserves column shape.}
    \label{fig:bar_twisting}
    \vspace{-3mm}
\end{figure*}

\subsection{MPM in Engineering}

In the engineering community, MPM was first introduced in~\cite{Sulsky:1994:Particle} as an extension of the FLIP method~\cite{Brackbill:1988:Flip}, and substantial improvements and variants have been proposed thereafter, including experimental validation for studying dynamic anticrack propagation in snow avalanches~\cite{gaume:2018:dynamic} and the use of MPM for designing differentiable physics engines for robotics applications~\cite{Hu:2019:ChainQueen}.
Different strategies for updating the stress were compared to investigate the energy conservation error in MPM in~\cite{Bardenhagen:2002:Energy}. The quadrature error and cell-crossing error of MPM was investigated in~\cite{Steffen:2008:Analysis}.
A generalized interpolation material point method (GIMPM)~\cite{Bardenhagen:2004:GIMPM} was proposed to obtain a smoother field representation by combining the shape functions of the grid with the particle characteristic function. 
The cell-crossing error in the MPM discretization was alleviated by introducing the local \emph{tangent} affine deformation of particles in the convected particle domain method (CPDI)~\cite{Sadeghirad:2011:CPDI}. 
However, CPDI does not completely remove the numerical fracture issue due to gaps between particles and grid cells. 
Recently, the standard MPM was reformulated with respect to the initial topology, which provides the total Lagrangian formulation for MPM (TLMPM)~\cite{De:2020:TLMPM}, which has been proven to completely eliminate the numerical fracture issue and cell-crossing instability and has been further extended in~\cite{De:2021:TLMPM} to support multi-body contacts. 


%%%%====================================
\begin{comment}
\subsection{Moving Least Squares in MPM}
As a widely used fitting scheme, Moving Least Squares (MLS) has been utilized to boost the computational accuracy of various discretization methods~\cite{Lu:1994:newEFG, Zhang:2009:improvedEFG, Atluri:1998:New, Bessa:2014:RKPM, Zienkiewicz:1974:Least}. 
In the MPM discretization, MLS techniques have successfully demonstrated the ability of improving the solution accuracy in large deformation simulations~\cite{W:2007:improved, Sulsky:2016:improving}. 
However, the formulation requires the inversion of the moment matrix on the solution of a system of equations, therefore, this is both expensive and there is a possibility that the moment matrix may be singular or ill-conditioned.  
Spurious values due to the ill-conditioning of moment matrix were demonstrated in the background gird~\cite{W:2007:improved}. 
To circumvent this issue, Gram Schmidt orthogonalization method was used to formulate a set of orthogonal basis functions to facilitate rapid solution in the PolyPIC~\cite{Fu:2017:PolyPIC} and the Improved MPM~\cite{Tran:2019:Improved}.
This idea has also been used in the Element-free Galerkin method~\cite{Liew:2006:Boundary,Zhang:2009:improvedEFG,Zhang:2014:improved} to overcome the instability. 
Besides, a noteworthy integration of MLS and MPM was proposed in~\cite{Hu:2018:Moving}, which unifies APIC and PolyPIC and can be treated as a modified EFG method. 
\end{comment}