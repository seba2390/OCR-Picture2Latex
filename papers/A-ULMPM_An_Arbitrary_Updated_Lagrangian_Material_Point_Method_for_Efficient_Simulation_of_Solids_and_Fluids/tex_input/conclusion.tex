\section{Discussion and Conclusion}
\label{sec:conclusion}

\subsection{Limitations and Future Work} 
Our model has generated a large number of compelling examples, but there remains much work to be done. Parameters to adjust the configuration update frequency were tuned by hand, and it would be interesting to calibrate them to measured models. Since each object has it own configuration map, we only briefly investigated contact by projecting particle positions to a global grid and processing grid-based collisions~\cite{Stomakhin:2013:MPMsnow}. By doing this, we observed slight self-penetration in the 3D twisting beam example (see Figure~\ref{fig:bar_twisting}). This could be addressed by further introducing contact algorithms, such as~\cite{Jiang:2017:Anisotropic, Han:2019:Hybrid}.
While we did not experience a need for excessively small time steps given the low stiffness of the materials we considered, deforming materials with high wave speed, such as steel, could benefit from a fully implicit discretization.
Though side-to-side comparisons with Eulerian MLS-MPM shows that our $A$-ULMPM framework produces similar fluid-like dynamics as EMPM and captures rich interactions, a slight energy loss in $A$-ULMPM framework was observed. 
This could be addressed by introducing more accurate configuration update criteria. 
Finally, while our focus was on the material responses of water, snow, and hyperelastic solids, it would be interesting to investigate other materials and multi-physics coupling problems with $A$-ULMPM. 

\subsection{Conclusion}
We proposed an arbitrary updated Lagrangian Material Point Method ($A$-ULMPM), which combines advantages of Total Lagrangian frameworks~\cite{De:2021:TLMPM,De:2020:TLMPM} and Eulerian frameworks~\cite{Stomakhin:2013:MPMsnow, Hu:2018:Moving} in an adaptive fashion. $A$-ULMPM avoids the cell-crossing instability and numerical fracture in solid simulations, while still allowing for large deformations that arise in fluid-like simulations. 
It can be easily integrated with any existing MPM framework and builds a foundation for devising various MPM schemes, such as PolyPIC~\cite{Fu:2017:PolyPIC}, for enhanced accuracy and visual vividness. 


%Future work can explore many interesting avenues. Parameters to adjust the configuration update frequency were tuned by hand, and it would be interesting to calibrate them to measured models. Since each object has it own configuration map, we only briefly investigated contact by projecting particle positions to a global grid and processing grid-based collisions~\cite{Stomakhin:2013:MPMsnow}. By doing this, we observed slight self-penetration in the 3D twisting beam example (see Figure~\ref{fig:bar_twisting}). This could be addressed by further introducing contact algorithms, such as~\cite{Jiang:2017:Anisotropic, Han:2019:Hybrid}.
%While we did not experience a need for excessively small time steps given the low stiffness of the materials we considered, deforming materials with high wave speed, such as steel, could benefit from a fully implicit discretization. 
%Finally, while our focus was on the material responses of water, snow, and hyperelastic solids, it would be interesting to investigate other materials and multi-physics coupling problems with $A$-ULMPM. 