\appendix
\label{sec:appendix}
\section{MLS Gradient}
\label{sec:mls_gradient}
Consider a domain {($\Omega_0$)} in Euclidean space $\mathbb{E}$. Let $\bs{q}_i\in \Omega_{0} \subset \mathbb{E}$ denote the geometric position of a certain point in the domain and $\bs{q}_j\in \Omega_{0} \subset \mathbb{E}$ denote another point which is close to the position $\bs{q}_i$.
Therefore, any physical quantity at each point in space can be defined as $\bs{\varphi}(\bs{q}_i)$ and following a first-order Taylor approximation gives: 
\begin{equation}
\bs{\varphi}(\bs{q}_j)\cong \bs{\varphi}(\bs{q}_i) +\nabla{\bs{\varphi}}_{ij} \bs{r}_{ij}
\label{eq:Taylor1st}
\end{equation}
where $\bs{r}_{ij}=\bs{q}_j-\bs{q}_i$. A weighted error $E_i$ of equation~\eqref{eq:Taylor1st} is defined within its associated domain $\Omega_i$ as follows:
\begin{equation}
E_i=\int_{j\in\Omega_i}\|{\bs{\varphi}(\bs{q}_j)-\bs{\varphi}(\bs{q}_i)}-\nabla{\bs{\varphi}_{ij}} \bs{r}_{ij}\|^2W_{ij}dV
\label{eq:error_i}
\end{equation}
where $\Omega_i$ represents the neighboring domain of $\bs{q}_i$; $W_{ij}$ represents the influence from $j$ to $i$ and is a function of $\left\|\bs{r}_{ij}\right\|$. To minimize the error $E_i$ and get the best approximation, the Least Squares Technique is exploited by introducing $\frac{\partial E_i}{\partial \nabla \bs{\phi}_i}=0$:
\begin{equation}
\frac{\partial}{\partial\nabla \bs{\phi}_i}\int_{j\in\Omega_i}\left({\bs{\varphi}(\bs{q}_j)-\bs{\varphi}(\bs{q}_i)}-\nabla{\bs{\varphi}_{ij}} \bs{r}_{ij}\right)^2W_{ij}dV=0 
\end{equation}
Therefore, we have a MLS-gradient operator : 
\begin{eqnarray}
\nabla\bs{\varphi}(\bs{q}_i)=\left(\int_{j\in\Omega_i}\bs{\varphi}(\bs{q})_{ij}\otimes \bs{r}_{ij} W_{ij} dV\right)\left(\int_{j\in\Omega_i}\bs{r}_{ij}\otimes\bs{r}_{ij} W_{ij} dV \right)^{-1}
\label{eq:gradient_int}
\end{eqnarray}
where $\bs{\varphi}(\bs{q})_{ij}=\bs{\varphi}(\bs{q})_{j}-\bs{\varphi}(\bs{q})_{i}$
and its discrete form is given by:
\begin{eqnarray}
\nabla\bs{\varphi}(\bs{q}_i)=\left(\sum_{j\in\Omega_i}\bs{\varphi}(\bs{q})_{ij}\otimes \bs{r}_{ij} W_{ij} V_j\right)\left(\sum_{j\in\Omega_i}\bs{r}_{ij}\otimes\bs{r}_{ij} W_{ij} V_j \right)^{-1}
\label{eq:gradient_sum}
\end{eqnarray}
where $V_j$ denotes volume distributed at $\bs{q}_j$. 
Moreover, the derivation of $\nabla\bs{\varphi}(\bs{q}_i)$ with respect to $\bs{q}_k$ is given as follows:
\begin{eqnarray}
\begin{split}
&\frac{\p\nabla\bs{\varphi}(\bs{q}_i)}{\p \bs{\varphi}(\bs{q}_k)}=\left(\sum_{j\in\Omega_i}\frac{\bs{\varphi}(\bs{q})_{ij}}{\p\bs{\varphi}(\bs{q}_{k})}\otimes \bs{r}_{ij} W_{ij} V_j\right)\left(\sum_{j\in\Omega_i}\bs{r}_{ij}\otimes\bs{r}_{ij} W_{ij} V_j \right)^{-1} \\
&=\left(\sum_{j\in\Omega_i}\left(\delta_{jk}-\delta_{ik}\right)\otimes \bs{r}_{ij} W_{ij} V_j\right)\left(\sum_{j\in\Omega_i}\bs{r}_{ij}\otimes\bs{r}_{ij} W_{ij} V_j \right)^{-1}\\
\end{split}
\label{eq:dFdq}
\end{eqnarray}
where $\delta$ is the Kronecker delta function.


\section{$A$-ULMPM Integration with kernel-MPM}
\label{sec:A-ULMPM-kernel}
We briefly describe the integration of our $A$-ULMPM with traditional MPM~\cite{Stomakhin:2013:MPMsnow} as follows:
\begin{enumerate}
\item{Transfer Particle to Grid.}
\begin{equation}
    m_i^s \bs{v}_i^n=\sum_p m_p \bs{v}_p^n W_{pi}^s,\quad m_i^s=\sum_p m_p W_{pi}^s,\quad \bs{v}_i^n= \frac{m_i^s \bs{v}_i^n}{m_i}
\end{equation}
\item{Update Grid Momentum.} 
\begin{equation}
\begin{split}
    &\bs{\hat{v}}_i^{n+1}=\bs{v}_i^n-\frac{\dt}{m_i}\sum_p\mathbb{P}_p^0(\bs{F}_p^{0s})^T \nabla^sW_{ip}^s V_j^0\\
    &\bs{q}_i^{n+1}=\bs{q}_i^n+\dt\bs{\hat{v}}_i^{n+1}\\
\end{split}
\label{eq:kernel_grid_Vel_update_explicit}
\end{equation}
\item{Transfer Grid Velocity to Particles.} 
We project $\bs{q}_i^{n+1}$ for all objects to the global grid and consider grid-based collisions~\cite{Stomakhin:2013:MPMsnow} to obtain $\bs{v}_i^{n+1}$. We transfer $\bs{v}_i^{n+1}$ to particles using a weighted combination of PIC and FLIP, as described in~\cite{Stomakhin:2013:MPMsnow}. 
\item{Update Particle Deformation Gradient.}
\begin{eqnarray}
  \bs{F}_p^{s(n+1)}=    &\bs{F}_p^{sn}+\dt\nabla_s\bs{v}_p^{n+1}\\
\label{eq:kernel_gradient_s(n+1)}
\end{eqnarray}
where $\nabla_s\bs{v}_p^{n+1}$ is given by:

\begin{equation}
\nabla_s\bs{v}_p^{n+1}=\sum_i\bs{v}_i^{n+1} \nabla^sW_{pi}^s
\label{eq:kernelgradientVp}
\end{equation}
By differential chain rule, the update of $\bs{F}_p^{0(n+1)}$ is given by:
\begin{equation}
    \bs{F}_p^{0(n+1)}=\frac{\p \bs{q}_p^{n+1}}{\p\bs{q}_p^{s}}\frac{\p\bs{q}_p^{s}}{\p\bs{q}_p^{0}}
    =\bs{F}_p^{s(n+1)}\bs{F}_p^{0s}
\end{equation}
\end{enumerate}
