\section{Results}
\label{sec:result}

Accompanying this article, we open-source our code for running 3D examples with our proposed $A$-ULMPM framework (see Section~\ref{sec:A_UL_MLS_MPM}). MLS-based MPM, including MLS-EMPM, MLS-TLMPM, and MLS-$A$-ULMPM, was applied to simulate all our 3D simulations. 
Besides the advantage of removing numerical fracture and the cell-crossing instability in solid and fluid simulations, $A$-ULMPM has the practical convenience of using the same numerical implementation to adaptively update the configuration map, without having to switch between TLMPM and EMPM.
We use the example of a falling elastic ball (see Figure~\ref{fig:configrations}) to evaluate the computational cost of explicit (see equation~\eqref{eq:grid_Vel_update_explicit}) and implicit (see equation~\eqref{eq:grid_Vel_update_implicit}) Euler schemes. 
As shown in the first two rows of Table~\ref{tab:timings}, the computational cost of these two schemes are comparable. 
Given the low stiffness in the materials, we utilize the explicit Euler scheme in all our 3D examples and did not experience any need for excessively small time steps.  
For our 3D solid simulations, we set $\epsilon=0.5$ and $\eta=0.1$, and observed that no configuration update was required, showing that $A$-ULMPM inherits the advantage of TLMPM for solid simulation. 
Due to foreseeable large deformations that arise when simulating fluid-like materials, such as water splashes and snow scattering, we set $\epsilon=0.01$ and $\eta=0.01$ to update the configuration maps more frequently. 
Even so, our $A$-ULMPM scheme reduces the configuration update overhead by $4.27\times$ -- $31.52\times$ in fluid-like simulations (see Table~\ref{tab:timings2}), compared to  standard EMPM. 
Table~\ref{tab:timings} summarizes the specific timings for all our examples. 

\begin{table}[h!]
\vspace{-3mm}
\caption{All simulations were run on Machine 1: Intel(R) Core(TM) i7-8750H CPU @ 2.20GHz and Machine 2: Intel(R) Xeon(R) CPU E5-1620 v4 @ 3.50GHz. {Simulation time is measured in average seconds per time step}. \textbf{Grid:} The number of occupied voxels in the background sparse grid. \textbf{Particle:} The total number of MPM particles in the simulation.}
\vspace{-4mm}
\begin{center}
\setlength{\tabcolsep}{0.1mm}{
\begin{tabular}{lcccc}
\hline 
\textbf{Simulation} & \textbf{Time} &  \textbf{Machine} &  \textbf{Grid} & \textbf{Particle} \\
\hline 
2D Elastic ball (Fig.\ref{fig:configrations})           & 0.0148    & 1     & 16.4K     & 20K      \\
2D Elastic ball (implicit)                              & 0.0165    & 1     & 16.4K     & 20K     \\
2D Rotation (Fig.\ref{fig:2D_rotating}(1st row))        & 0.0571    & 1     & 16.4K     & 42K    \\
2D Rotation (Fig.~\ref{fig:2D_rotating}(2nd row))       & 0.0571    & 1     & 16.4K     & 42K      \\
2D Rotation (Fig.~\ref{fig:2D_rotating}(3rd row))       & 0.0571    & 1     & 16.4K     & 42K     \\
2D Droplet (Fig.~\ref{fig:2D_droplet}(1st row))  & 0.0154    & 1     & 16.4K     & 5K       \\
2D Droplet (Fig.~\ref{fig:2D_droplet}(2nd row))   & 0.0108    & 1     & 16.4K     & 5K    \\
2D Droplet (Fig.~\ref{fig:2D_droplet}(3rd row))         & 0.0194    & 1     & 16.4K     & 5K       \\
2D Droplet (Fig.~\ref{fig:2D_droplet}(4th row))     & 0.0146    & 1     & 16.4K     & 5K       \\
2D Droplet (Fig.~\ref{fig:2D_droplet}(5th row))          & 0.0199    & 1     & 16.4K     & 5K       \\
Twisting bar (Fig.\ref{fig:bar_twisting}(1st row))        & 0.627     & 1     & 1.0M      & 193.3K   \\
Twisting bar (Fig.\ref{fig:bar_twisting}(2nd row))   & 0.624     & 1     & 1.0M      & 193.3K   \\
Twisting bar (Fig.\ref{fig:bar_twisting_inv_deformation}(1st row))        & 0.125     & 2  & 131.1K   & 48.3K \\
Twisting bar (Fig.\ref{fig:bar_twisting_inv_deformation}(2nd row))   & 0.124     & 2  & 131.1K   & 48.3K \\
Stretchy yo-yo (Fig.\ref{fig:strechy_bunny_1}(1st row))  & 1.314 &  2    &  4.2M     &  179.3K \\
Stretchy yo-yo (Fig.\ref{fig:strechy_bunny_1}(2nd row))     & 1.310 &  2    &  4.2M     &  179.3K \\
Snow bunny (Fig.\ref{fig:snow}) & 0.965     &  2    &  524.3K   &  116.3K \\
Snow bunny (EMPM in video)                 & 0.971     &  2    &  523.4K   &  116.3K \\
Water bunny (Fig.\ref{fig:water})       & 0.341     &  2    &  2.1M     &  96.3K \\
Water bunny (EMPM in video)                        & 0.381     &  2    &  2.1M     &  96.3K \\
FSI (Fig.\ref{fig:FSI}(1st row)) & 0.934         & 1 & 4.2M &  169.1K\\
FSI (Fig.\ref{fig:FSI}(2nd row))      & 0.939  & 1 & 4.2M & 169.1K\\
\hline
\end{tabular}
}
\end{center}
\label{tab:timings}
\vspace{-5mm}
\end{table}

\begin{table}[h!]
\caption{Configuration update cost in fluid-like simulations. \textbf{$\epsilon$} and  \textbf{$\eta$}: user-defined parameters to adjust the update frequency.  \textbf{$\tau$}: average update times per 104 time steps. \textbf{Cost}: average run-time.}
\vspace{-4mm}
\begin{center}
\setlength{\tabcolsep}{0.5mm}{
\begin{tabular}{lcccc}
\hline 
\textbf{Simulation} & \textbf{$\epsilon$} &  \textbf{$\eta$} &  \textbf{$\tau$} & \textbf{Cost} \\
\hline 
Snow bunny ($A$-ULMPM in Fig.\ref{fig:snow})          &   0.01      &0.01         & 24.33    &   0.112    \\
Snow bunny (EMPM in video)          & N/A        & N/A        &   104  &  0.479     \\
Water bunny ($A$-ULMPM in Fig.\ref{fig:water})          &  0.01       & 0.01        &  3.31   &     0.021  \\
Water bunny (EMPM in video)       & N/A        & N/A        &   104  &  0.662   \\
\hline
\end{tabular}
}
\end{center}
\label{tab:timings2}
\vspace{-5.5mm}
\end{table}


\subsection{2D Simulations for Solids and Fluids}
\subsubsection{Numerical fracture and cell-crossing instability}
We simulated a 2D rotating hyperelastic plate to demonstrate that EMPM suffers from the cell-crossing instability, which leads to severe numerical fractures, while TLMPM can completely eliminate these artifacts. Figure~\ref{fig:2D_rotating} shows that our $A$-ULMPM framework and TLMPM integrated with MLS-MPM captures the appealing hyperelastic rotation, preserving the angular momentum for long simulation periods while completely eliminating numerical fractures. However, traditional MLS-MPM~\cite{Hu:2018:Moving} that uses an Eulerian approach suffers from severe numerical fractures when large deformations occur. 

We quantitatively evaluate the spatial accuracy of our method on the rotating hyperelastic plate example by varying the grid resolution $\Delta x$. We utilized numerical results with resolution $256\times 256$ as the benchmark solution (see Figure~\ref{fig:2D_rotating}) and discretize sampling examples with spatial resolution $2^i\times2^i,\:i\in\{2,\ldots,7\}$ for the grid. 
We fixed the particle number $n_p=41943$ for each case and ran simulations up to a total simulation time of $0.025s$ with a time step size of $10^{-4}s$. The error for numerical simulations is defined as:
\begin{equation}
{E}=\sqrt{\frac{\left|\bs{\phi}_{i}-\bs{\phi}_{8}\right|^2}{n_p}}
    \label{eq:error}
\end{equation}
where $\bs{\phi}_{i}$ is the variable evaluated by lower grid resolutions, while $\bs{\phi}_8$ is the value at resolution $256^2$ (benchmark). 
Figure~\ref{fig:ConvPlot} shows the convergence plots for the average particle displacement and velocity for $A$-ULMPM integrated with MLS-MPM $(s=0)$, $A$-ULMPM integrated with MLS-MPM $(s\neq 0)$, $A$-ULMPM integrated with MLS-MPM (black) $(s=n)$, and $A$-ULMPM integrated with kernel-MPM $(s=0)$, where $A$-ULMPM ($s=0$) recovers TLMPM and $A$-ULMPM ($s=n$) recovers EMPM, as described in Section~\ref{sec:A_UL_MLS_MPM}. 
In general,  TLMPM can produce more accurate simulations since the cell-crossing error is completely eliminated, while EMPM fails to converge when the cell-crossing error is significant. $A$-ULMPM with $s\neq n $ exhibits second-order accuracy while its integration with MLS-MPM has higher solution accuracy than that with kernel-MPM.
\begin{figure}[h!]
\vspace{-3mm}
\begin{overpic}[width=.18\textwidth]{image/DispCon}
    \put(30,10){\small slope$=2.08$}
    \put(40,-15){\small$\log_{2}(\Delta x)$}
    \put(-22,32){\small\rotatebox{90}{$\log_{2} (E)$}}
    \put(-19,0){\small $-17.6$}
    \put(-19,15){\small $-11.2$}
    \put(-19,30){\small $-10.2$}
    \put(-16,45){\small $-9.6$}
    \put(-16,60){\small $-9.2$}
    \put(-16,75){\small  $-8.7$}
    \put(0,-8){\small $-7$}
    \put(15,-8){\small $-6$}
    \put(35,-8){\small $-5$}
    \put(55,-8){\small $-4$}
    \put(75,-8){\small $-3$}
    \put(90,-8){\small $-2$}
\end{overpic}
\hspace{8mm}
\begin{overpic}[width=.18\textwidth]{image/VelCon} 
    \put(30,10){\small slope$=1.93$}
    \put(40,-15){\small$\log_{2}(\Delta x)$}
    \put(-22,30){\small\rotatebox{90}{$\log_{2} (E)$}}
    \put(-16,0){\small $-3.9$}
    \put(-16,15){\small $-2.9$}
    \put(-16,30){\small $-2.3$}
    \put(-16,45){\small $-1.9$}
    \put(-16,60){\small $-1.6$}
    \put(-16,75){\small  $-1.3$}
    \put(0,-8){\small $-7$}
    \put(15,-8){\small $-6$}
    \put(35,-8){\small $-5$}
    \put(55,-8){\small $-4$}
    \put(75,-8){\small $-3$}
    \put(90,-8){\small $-2$}
\end{overpic}
\vspace{3mm}
\caption{Log-log plots of error (labeled $E$) vs. grid mesh size $\Delta x$.
(Left) Error of the displacement, (right) error of the velocity using $A$-ULMPM integrated with MLS-MPM $(s=0)$ (red), $A$-ULMPM integrated with MLS-MPM $(s\neq 0)$ (blue), $A$-ULMPM integrated with MLS-MPM (black) $(s=n)$, and $A$-ULMPM integrated with kernel-MPM (black) $(s=0)$. Our $A$-ULMPM framework recovers TLMPM ($s=0$) and EMPM ($s=n$).   
Comparing the slopes and solution accuracy shows EMPM is non-convergent when grid-crossing occurs, while $A$-ULMPM with $s=0$ and $s\neq n$ provides convergent simulations with second order accuracy for displacement and velocity with small $\Delta x$.}
\label{fig:ConvPlot}
\vspace{-5mm}
\end{figure}
\subsubsection{2D droplet}
We ran several simulations of falling droplets to compare our implementations in the $A$-ULMPM framework. As shown in Figure~\ref{fig:2D_droplet}, although TLMPM has benefits over EMPM for solid simulations (see Figure~\ref{fig:2D_rotating}), it fails to achieve detailed free surfaces in fluid simulations, as shown in Figure~\ref{fig:2D_droplet}, since the topology changes significantly in fluid-like simulations. 
$A$-ULMPM automatically updates configuration maps to produce similar fluid dynamics as EMPM and captures rich interactions. Moreover, our integration with MLS-MPM (see Section~\ref{sec:A_UL_MLS_MPM}) yields more energetic behavior compared to an integration with kernel-MPM (see Appendix~\ref{sec:A-ULMPM-kernel}). 

\subsection{Bar Twisting}
We imposed torsion and stretch boundary conditions at the two ends of a hyperelastic beam to showcase that $A$-ULMPM allows large deformations in solids without non-physical fractures. As shown in Figure~\ref{fig:bar_twisting}, traditional EMPM fails to preserve the shape of the beam during the twisting and pulling, while $A$-ULMPM can readily handle the challenging invertible elasticity when one end of the beam is released after twisting. Figure~\ref{fig:bar_twisting_inv_deformation} shows that $A$-ULMPM is capable of robustly recovering the beam shape after extreme elastic deformations, while particles in EMPM cluster into one irreversible (or plastic) thin string blocking the ``recovery''  of elasticity. 
\vspace{-1.5mm}
\subsection{Stretchy Yo-Yo} 
Next, we tossed a stretchy Stanford bunny yo-yo with gravity forces. Our $A$-ULMPM hyperelastic bunny demonstrates rich elastic responses and realistic bouncing dynamics, while EMPM breaks due to severe numerical fracture (see Figure~\ref{fig:strechy_bunny_1}). 
Our $A$-ULMPM framework can perfectly handle hyperelastic deformation under severe bending (see Figure~\ref{fig:strechy_bunny_zoom_in} (left)) and stretching (see Figure~\ref{fig:strechy_bunny_zoom_in} (right)).
\begin{figure}[h!]
 \vspace{-4mm}
    \centering
    \includegraphics[width=\columnwidth]{image/bunnies_TL_f94.png}
     \vspace{-7.5mm}
    \caption{A closer view of the hyperelastic bunny yo-yo from Figure~\ref{fig:strechy_bunny_1}.}
    \label{fig:strechy_bunny_zoom_in}
    \vspace{-5mm}
\end{figure}
% \vspace{-1.5mm}
\subsection{Fluid-like Bunny Simulation} 
We simulated the fluid-like behavior of different materials using our $A$-ULMPM framework, as described in Section~\ref{sec:A_UL_MLS_MPM}, such as elastoplastic snow~\cite{Stomakhin:2013:MPMsnow} and weakly compressible water. 
We dropped two copies of the snow Stanford bunny with different orientations to a solid wedge, as shown in Figure~\ref{fig:snow}. $A$-ULMPM captures the vivid snow smashing and scattering after the bunnies fall on the wedge, similar to its Eulerian counterparts proposed in prior works~\cite{Stomakhin:2013:MPMsnow} (see side-by-side comparisons in our video). 
We simulated a water Stanford bunny falling inside a spherical container (see Figure~\ref{fig:water}), showing the extreme large deformations and energetic splashes.   
 $A$-ULMPM does not update the configuration maps at each time step, in contrast to EMPM, so it is naturally more efficient in fluid-like simulations (see Table~\ref{tab:timings2}).
 
\subsection{Fluid-Structure Interaction}

Our $A$-ULMPM framework can also simulate realistic fluid-structure interaction problems, as shown in Figure~\ref{fig:FSI}.
\begin{figure}[h!]
\vspace{-4mm}
    \centering
    \includegraphics[width=.49\columnwidth]{image/0020.png}
    \includegraphics[width=.49\columnwidth]{image/0042.png}
    \vspace{-3mm}
    \caption{A closer view of hyperelastic deformations in the fluid-structure interaction example from Figure~\ref{fig:FSI}. (Left) The bowl breaks in EMPM due to numerical fractures, while (right) $A$-ULMPM maintains a leakproof bowl. }
        \vspace{-5mm}
    \label{fig:fsi_zoom_in}
\end{figure}
\vspace{1.5mm}
We dropped a water Stanford bunny to a hyperelastic bowl. $A$-ULMPM captures rich fluid-structure interactions, displaying advantages of both TLMPM and EMPM (see Figure~\ref{fig:fsi_zoom_in} (right)), and highlighting that $A$-ULMPM can adaptively handle both fluid and solid simulations without having to switch between TLMPM and EMPM.
In contrast, EMPM suffers from numerical fractures. As shown in Figure~\ref{fig:FSI}, the hyperelastic bowl fractures, causing the water to spill below.
