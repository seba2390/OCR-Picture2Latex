% SIAM Supplemental File Template
\documentclass[supplement,onefignum,onetabnum]{siamart190516} % review

% SIAM Shared Information Template
% This is information that is shared between the main document and any
% supplement. If no supplement is required, then this information can
% be included directly in the main document.


% Packages and macros go here
\usepackage{lipsum}
\usepackage{amsfonts}
\usepackage{graphicx}
\usepackage{epstopdf}
\usepackage{algorithmic}
\ifpdf
  \DeclareGraphicsExtensions{.eps,.pdf,.png,.jpg}
\else
  \DeclareGraphicsExtensions{.eps}
\fi

% Add a serial/Oxford comma by default.
\newcommand{\creflastconjunction}{, and~}

% Used for creating new theorem and remark environments
\newsiamremark{remark}{Remark}
\newsiamremark{hypothesis}{Hypothesis}
\crefname{hypothesis}{Hypothesis}{Hypotheses}
\newsiamthm{claim}{Claim}

% Sets running headers as well as PDF title and authors
\headers{Speeding up Python-based Lagrangian Particle Simulations}{C. Kehl, E. van Sebille and A. Gibson}

% Title. If the supplement option is on, then "Supplementary Material"
% is automatically inserted before the title.
\title{Speeding up Python-based Lagrangian Fluid-Flow Particle Simulations via Dynamic Collection Data Structures\thanks{Submitted to the SIAM Journal on Scientific Computing editors 2021-01-25.
\funding{This work was funded by the European Research Council (ERC) under TOPIOS project (grant no.~715386). Simulations were carried out on the SURF's Dutch National e-Infrastructure (project no. 16371 and 2019.034).}}} %<at \textbf{Angus}: please insert your funding here>

% Authors: full names plus addresses.
\author{Christian Kehl
\and Erik van Sebille\thanks{Institute for Marine and Atmospheric Research, Utrecht University, the Netherlands 
  (\email{c.kehl@uu.nl}, \email{e.vansebille@uu.nl}).}
\and Angus Gibson\thanks{The Australian National University (\email{angus.gibson@anu.edu.au}).}
%\and Jane E. Smith\footnotemark[3]
}

\usepackage{amsopn}
\DeclareMathOperator{\diag}{diag}


%%% Local Variables: 
%%% mode:latex
%%% TeX-master: "ex_article"
%%% End: 


\externaldocument{ex_article}

% Optional PDF information
\ifpdf
\hypersetup{
  pdftitle={Supplementary Materials: Speeding up Python-based Lagrangian Fluid-Flow Particle Simulations via Dynamic Collection Data Structures},
  pdfauthor={C. Kehl, E. van Sebille and A. Gibson}
}
\fi

\definecolor{newcolor}{rgb}{.8,.349,.1}

\RequirePackage[UKenglish]{babel}
\selectlanguage{UKenglish}

% declare the path(s) where your graphic files are
%\graphicspath{{./small/}{./original/}}
%\graphicspath{{./img/}}
\graphicspath{{./}}
% and their extensions so you won't have to specify these with
% every instance of \includegraphics
\DeclareGraphicsExtensions{.jpg,.jpeg,.png,.eps}
\RequirePackage[center,tight,footnotesize]{subfigure}
\RequirePackage[acronyms,shortcuts]{glossaries}
\RequirePackage[normalem]{ulem}
% A
\newacronym{afe}{AFE}{analogue front-end}
\newacronym{asic}{ASIC}{application-specific integrated circuit}
% B
\newacronym{bx}{BX}{bunch crossing}
\newacronym{be}{BE}{back-end}
% C
\newacronym{cdr}{CDR}{clock data recovery}
\newacronym{cml}{CML}{current mode logic}
\newacronym{cms}{CMS}{Compact Muon Solenoid}
\newacronym{csa}{CSA}{charge sensitive amplifier}
% D
\newacronym{da}{DA}{differential amplifier}
\newacronym{dac}{DAC}{digital-to-analogue converter}
\newacronym{daq}{DAQ}{data acquisition}
\newacronym{diff}{DIFF}{differential}
% E
\newacronym{enc}{ENC}{equivalent noise charge}
% F
\newacronym{fe}{FE}{front-end}
% H
\newacronym{hdi}{HDI}{high density interconnect}
\newacronym{hllhc}{HL-LHC}{High Luminosity LHC}
% I
\newacronym{it}{IT}{Inner Tracker}
\newacronym{ip}{IP}{intellectual property}
% L
\newacronym{lcc}{LCC}{leakage current compensation}
\newacronym{led}{LED}{light-emitting diodes}
\newacronym{lhc}{LHC}{Large Hadron Collider}
\newacronym{lin}{LIN}{linear}
\newacronym{ls3}{LS3}{Long Shutdown 3}
% M
\newacronym{mip}{MIP}{minimum ionizing particle}
\newacronym{mpv}{MPV}{most probable value}
% O
\newacronym{ot}{OT}{Outer Tracker}
% P
\newacronym{pa}{PA}{preamplifier}
\newacronym{pll}{PLL}{phase locked loop}
% R
\newacronym{rms}{RMS}{root-mean-square}
% S
\newacronym{sync}{SYNC}{synchronous}
\newacronym{shldo}{ShLDO}{shunt low-dropout}
% T
\newacronym{tbpx}{TBPX}{Tracker Barrel Pixel detector}
\newacronym{tepx}{TEPX}{Tracker Endcap Pixel detector}
\newacronym{tfpx}{TFPX}{Tracker Forward Pixel detector}
\newacronym{tia}{TIA}{transimpedance amplifier}
\newacronym{tid}{TID}{total ionizing dose}
\newacronym{tot}{TOT}{time-over-threshold}
\newacronym{tsmc}{TSMC}{Taiwan Semiconductor Manufacturing Company}

\begin{document}

\maketitle


\section{Memory Access Patterns for Contiguous Arrays}
\label{sec:supplemental:memaccess}

$N x M$ arrays,  Arrays can be aligned in two patterns: in a usual and simple matter (1) the major axis represents the structure attributes aligned contiguously in memory, while the minor axis represents the multiple array items. Alternatively (2) the major axis can represent the array items contiguously aligned in memory, whereas the attributes follow the minor axis. Formally, with $N = |I|$, $I = \{x_{0}, x_{1}, ..., x_{n}\}$, and $M = |A|$ for $x(a)_{i} \forall a \in A$ (with $A$ representing or item structure \textit{attributes}), the arrays can structured as $N x M$ (1) or as (2) $M x N$. This has performance implications for the functional evaluation of the array, differing between a traditional sequential and a modern vectorized evaluation. This is shown in \cref{fig:appendix:memaccess:aos} for $N x M$ arrays (named \textit{Array of Structures (AoS)}, and in \cref{fig:appendix:memaccess:soa} for $M x N$ arrays (named \textit{Structure of Arrays (SoA)}.

\begin{figure}[htbp]
  \centering
  \includegraphics[keepaspectratio, width=0.90\columnwidth]{AoS_MemoryAccess}
  \caption{Memory access patterns related to a simple particle advection function on an $N x M$ array of structures, for sequential and vectorized evaluation (listed as assembly mnemonics).}
  \label{fig:appendix:memaccess:aos}
\end{figure}

\begin{figure}[htbp]
  \centering
  \includegraphics[keepaspectratio, width=0.90\columnwidth]{SoA_MemoryAccess}
  \caption{Memory access patterns related to a simple particle advection function on an $M x N$ structure of arrays, for sequential and vectorized evaluation (listed as assembly mnemonics).}
  \label{fig:appendix:memaccess:soa}
\end{figure}

Comparing the memory access patterns in terms of performance, we see that in traditional sequential processing the \textit{SoA} pattern is faster, as it only requires 1 cache update for each iteration whereas it requires 4 updates with an \textit{SoA} pattern. In modern computing architectures, each iteration actually executes between 2 (for multi-threaded CPUs) and 32 (for GPU warps) iterations simultaneously, thus a view on per-iteration (i.e. per-item) computations is invalid. This simultaneous computation is referred to as \textit{Vectorization} (on CPUs) or \textit{Simultaneous Multi-Processing (SMP)} (on general processors). Comparing the numbers for a 2-item evaluation, \textit{AoS} requires 2 cache updates and \textit{SoA} requires 8 cache updates when assuming a sequential processing. For vectorized processing though, evaluating 2 items requires 14 updates for \textit{AoS} and just 4 updates for \textit{SoA}. Furthermore, the number of cache changes is approximately $M \times \frac{N}{|PU|}$ (where $|PU|$ is the number of processing units), hence scaling favourably with the number of processing units and threads (in software). This naturally makes \textit{SoA} superior in performance for modern processors for $M << N$. Conversely, where $N \leq M$ one can just switch both array axes. Lastly, in practice, the layout change can be easily achieved by changing from FORTRAN-contiguous to C-contiguous order in Python arrays (or vice versa), which equally is achieved by matrix transposition. Obviously, establishing the correct item order on allocation and keeping the order static is vital, as a per-iteration matrix transposition consumes vastly more processing cycles than is saved by the \textit{SoA} evaluation.

%\begin{equation}
%  \label{eq:suppa}
%  a^2 + b^2 = c^2.
%\end{equation}
%You can also reference equations such as \cref{eq:matrices,eq:bb} 
%from the main article in this supplement.

%\begin{theorem}
%  An example theorem.
%\end{theorem}
 
%\begin{lemma}
%  An example lemma.
%\end{lemma}

%Here is an example citation: \cite{KoMa14}.

%\begin{table}[htbp]
%{\footnotesize
%  \caption{Example table}  \label{tab:foo}
%\begin{center}
%  \begin{tabular}{|c|c|c|} \hline
%   Species & \bf Mean & \bf Std.~Dev. \\ \hline
%    1 & 3.4 & 1.2 \\
%    2 & 5.4 & 0.6 \\ \hline
%  \end{tabular}
%\end{center}
%}
%\end{table}


\bibliographystyle{siamplain}
\bibliography{ParticlePerformance}


\end{document}
