\section{Proof of Lemma~\ref{lem:eigenvalue_main}: lower-bound for smallest singular values}\label{apdx:proofNguyen}

First we state a special case of Corollary 1.17
from~\citep{nguyen2017repulsion}. Let $N_I(X) $, denote the number of
eigenvalues of $X$ in the interval $I$.
\begin{corollary}\label{cor:Nguyen}
  Let $M'$ be a deterministic symmetric matrix in $\Snn$. Let $G'$ be a
  random symmetric matrix with entries $G'_{ij}$ drawn i.i.d.\ from
  $\N(0,1)$ for
  $i \geq j$ (in particular, independent of $M'$.) Then, for given $0 < \gamma < 1$, there exists a
  constant $c = c(\gamma)$ such that for any $\eps > 0$ and $k \geq 1$,
  with $I$ being the interval,
  $[-\frac{\eps k}{\sqrt{n}}, \frac{\eps k}{\sqrt{n}}]$,
$$
\Pr{N_I(M'+G') \geq k } \leq n^k \left(\frac{c\eps}{\sqrt{2\pi}}\right)^{(1-\gamma)k^2/2}.
$$
\end{corollary}
We can use the above corollary to prove
Lemma~\ref{lem:eigenvalue_main}.
\begin{proof} %[Proof of Lemma \ref{lem:eigenvalue_main}]
	In our case, entries of $G$ have variance $\sG^2$. Thus, set $G = \sG G'$, and set $\bar M = \sG M'$.
  From Corollary~\ref{cor:Nguyen}, we get
  \begin{align*}
	  N_{\sG I}(\bar M+G) = N_I(M'+G') < k
  \end{align*}
  with probability
  at least $1 - n^k \left(\frac{c\eps}{\sqrt{2\pi}}\right)^{(1-\gamma)k^2/2}$. In this event,
  $\sigma_{n-(k-1)}(\bar M+G) \geq \frac{\eps k}{\sqrt{n}} \sG$. Choose
  $\gamma =\frac{1}{2}$, and $\eps = \frac{1}{2 c}$. Substituting this we get with
  probability at least
  $1 - \exp\left( - \frac{k^2}{8} \log( 8 \pi) + k \log (n)\right)$
  that
$$
\sigma_{n-(k-1)}(\bar M+G) \geq \frac{k}{2c \sqrt{n}}\sG.
$$
Hence,
$\sum_{i=1}^{k} \sigma_{n-(i-1)}\left(\bar M+G\right)^2 \geq
\sigma_{n-(k-1)}\left(\bar M+G\right)^2 \geq \frac{k^2}{\const n}
 \sG^2$, for some absolute constant $\const = 4c^2$.
\end{proof}