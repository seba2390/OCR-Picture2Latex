\section{Proofs for Section~\ref{sec:exact}}\label{app:exact}

\begin{proof}[Proof of Lemma~\ref{lem:global}]
	Necessary and sufficient optimality conditions for~\eqref{eq:prob_fx} are: $\nabla f(X) \succeq 0$ and $\nabla f(X)X = 0$. Let $U$ be an SOSP for~\eqref{eq:prob_fx_U} with $\rank(U) < k$  and define $X = UU^T$. Then, $\nabla g(U) = 2\nabla f(UU^T)U = 0$ and $\nabla^2 g(U) \succeq 0$. The first statement readily shows that $\nabla f(X)X = 0$. The Hessians of $f$ and $g$ are related by:% \TODO{define $\nabla g[U]$ notation}:
	\begin{align*}
		\frac{1}{2}\nabla^2 g(U)[\dot U] & = \nabla f(UU^T)\dot U + \nabla^2 f(UU^T)[U\dot U^T + \dot U U^T]U.
	\end{align*}
	Since $\rank(U) < k$, there exists a vector $z \in \Rk$ such that $Uz = 0$ and $\|z\|_2 = 1$. For any $x \in \Rn$, set $\dot U = xz^T$ so that $U\dot U^T + \dot U U^T = 0$. Using second-order stationarity of $U$, we find:
	\begin{align*}
		0 \leq \frac{1}{2}\ip{\dot U}{\nabla^2 g(U)[\dot U]} & = \ip{xz^T}{\nabla f(UU^T)xz^T} = x^T \nabla f(UU^T) x.
	\end{align*}
	This holds for all $x \in \Rn$, hence $\nabla f(UU^T) \succeq 0$ and $X = UU^T$ is optimal for~\eqref{eq:prob_fx}. Since~\eqref{eq:prob_fx} is a relaxation of~\eqref{eq:prob_fx_U}, it follows that $U$ is optimal for~\eqref{eq:prob_fx_U}.
\end{proof}



\begin{proof}[Proof of Lemma~\ref{lem:rank_deficient}]
	Let $U$ be any FOSP of~\eqref{eq:penalty_factored} and consider the linear operator $\calA \colon \Snn \to \Rm$ defined by $\calA(X)_i = \ip{A_i}{X}$. By first-order stationarity, we have: 
	\begin{align*}
		\nabla L_\mu(U) & = 2\left( C + 2\mu \calA^*(\calA(UU^T) - b) \right)U = 0.
	\end{align*}
	%	Assume for contradiction that $U$ has rank $k$. Then,
	Hence, the nullity of $C + 2\mu \calA^*(\calA(UU^T) - b)$ (the dimension of its kernel) satisfies:
	\begin{align}
		\rank(U) \leq \nulll(C + 2\mu \calA^*(\calA(UU^T) - b)) \leq \max_{y \in \Rm} \nulll(C + \calA^*(y)).
		\label{eq:maxnulliny}
	\end{align}
	The maximum over $y$ is indeed attained since the function $\nulll$ takes integer values in $0, \ldots, n$. Say the maximum evaluates to $\ell$. Then, for some $y$, $M \triangleq C + \calA^*(y)$ has nullity $\ell$. Hence,
	\begin{align*}
		C & = M - \calA^*(y) \in \calN_\ell + \im \calA^*,
	\end{align*}
	where $\calN_\ell$ is the manifold of symmetric matrices of size $n$ and nullity $\ell$, $\im \calA^*$ is the range of $\calA^*$ and the plus is a set-sum. More generally, assuming the maximum in~\eqref{eq:maxnulliny} is $p$ or more, then
	\begin{align*}
		C & \in \calM_p \triangleq \bigcup_{\ell = p, \ldots, n} \calN_\ell + \im \calA^*.
	\end{align*}
	The manifold $\calN_\ell$ has dimension $\frac{n(n+1)}{2} - \frac{\ell(\ell+1)}{2}$~\citep[Prop.~2.1(i)]{helmke1995matrixlsq}, while $\im \calA^*$ has dimension at most $m$. Hence, $$\dim \calM_p \leq m + \max_{\ell = p, \ldots, n} \dim \calN_\ell = m + \frac{n(n+1)}{2} - \frac{p(p+1)}{2}.$$ Since $C$ is in $\Snn$ and $\dim \Snn = \frac{n(n+1)}{2}$, almost no $C$ lives in $\calM_p$ if $\dim\calM_p < \dim \Snn$, which is the case if $\frac{p(p+1)}{2} > m$. Stated differently: $\rank(U) \leq p$, and for almost all $C \in \Snn$, $\frac{p(p+1)}{2} \leq m$. To conclude, require that $k$ is strictly larger than any $p$ which satisfies $\frac{p(p+1)}{2} \leq m$.
\end{proof}
