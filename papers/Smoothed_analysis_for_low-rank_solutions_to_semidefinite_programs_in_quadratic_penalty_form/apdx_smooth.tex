\section{Proofs for Section~\ref{sec:smooth}}\label{app:smooth}

\begin{proof}[Proof of Lemma \ref{lem:compact_optimal_approx}]
	The gradient and Hessian of $L_\mu$~\eqref{eq:penalty_factored}, with $\vr \triangleq \vr(U) = \calA(UU^T)-b$, are:
	\begin{align}
		\nabla L_\mu(U) & = 2\left( C + 2\mu\calA^*(\vr) \right)U,\label{eq:gradLmu}\\
		\nabla^2 L_\mu(U)[\dot U] & = 2\left( C + 2\mu\calA^*(\vr) \right)\dot U + 4\mu\calA^*(\calA(\dot U U^T + U\dot U^T))U. \label{eq:HessianLmu}
	\end{align}
	Since $U$ is an $(\eps, \gamma)$-SOSP, it holds for all $\dot U \in \Rnk$ with $\|\dot U\|_F = 1$ that:
	\begin{align}
		-\frac{\gamma\sqrt{\eps}}{2} & \leq \frac{1}{2} \ip{\dot U}{\nabla^2 L_\mu(U)[\dot U]} = \ip{C + 2\mu\calA^*(\vr)}{\dot U \dot U^T} + \mu \left\|\calA(\dot U U^T + U \dot U^T)\right\|_2^2.
		 \label{eq:HessianLmuip}
	\end{align}
	We now construct specific $\dot U$'s to exploit the fact that $U$ is almost rank deficient. Let $z \in \R^k$ be a right singular vector of $U$ such that $\|Uz\|_2 =\sigma_k(U)$ (that is, $z$ is associated to the least singular value of $U$ and $\|z\|_2 = 1$.) For any $x \in \Rn$ with $\|x\|_2 = 1$, introduce $\dot U = xz^T$ in~\eqref{eq:HessianLmuip}:
	\begin{align*}
	 	-\frac{\gamma\sqrt{\eps}}{2} & \leq x^T(C + 2\mu\calA^*(\vr))x + \mu \left\|\calA(\dot U U^T + U \dot U^T)\right\|_2^2.
	\end{align*}
	The last term is easily controlled:
	\begin{align*}
		\left\|\calA(\dot U U^T + U \dot U^T)\right\|_2 \leq 2 \|\calA\| \|U\dot U^T\|_F = 2 \|\calA\| \|Uzx^T\|_F \leq 2 \|\calA\|\|Uz\|_2 \|x\|_2 = 2 \|\calA\|\sigma_k(U).
	\end{align*}
	Let $x$ be an eigenvector of $C + 2\mu\calA^*(\vr)$ associated to its least eigenvalue and combine the last two statements together with the assumption on $\sigma_k(U)$ to find:
	\begin{align}
		\lambda_{\min}(C + 2\mu\calA^*(\vr)) \geq -\frac{\gamma\sqrt{\eps}}{2} - 4\mu\|\calA\|^2\sigma_k^2(U) \geq -\gamma\sqrt{\eps}.
		\label{eq:lambdaminC2mu}
	\end{align}
	This inequality is key to bound the optimality gap. For this part, we rely on the fact that $L_\mu(U) = F_\mu(UU^T)$ and $F_\mu$ is convex on $\Snn$~\eqref{eq:penalty_sdp}. Specifically, let $\tilde X$ be a global optimum for $F_\mu$ (assuming it exists), and set $X = UU^T$. Then, $\nabla F_\mu(X) = C + 2\mu\calA^*(\vr), \nabla L_\mu(U) = 2\nabla F_\mu(X)U$ and:
	\begin{align*}
%		L_\mu(\tilde U) - L_\mu(U) & = 
		F_\mu(\tilde X) - F_\mu(X) & \geq \ip{\nabla F_\mu(X)}{\tilde X - X}  = \ip{C+2\mu\calA^*(\vr)}{\tilde X} - \frac{1}{2}\ip{\nabla L_\mu(U)}{U} \\
								   & \geq -\gamma \sqrt{\eps} \operatorname{Tr}(\tilde X) - \frac{1}{2}\eps \|U\|_F.
	\end{align*}
	In the last step, we used~\eqref{eq:lambdaminC2mu} as well as approximate first-order stationarity .
%	
%	The objective is convex in $UU^T$, we have:
%	\begin{align*}
%		&\tLsm(U^*) -\tLsm(U) \\ &=  \ip{C+\tG}{X^*-X} + \mu \left[ \|\calA(X^*)-\vb\|_2^2+(\ip{S}{X^*}-\bz)^2 -\|\calA(X)-\vb\|_2^2-(\ip{S}{X}-\bz)^2  \right]  \\ &\geq \ip{C+\tG+2\mu \left[\calA^*(\vr)+ (\ip{S}{X}-\bz)S \right] }{X^*-X} \\
%		&= \ip{C + 2\mu\calA^*(\vr)}{X^*} - \ip{(C+\tG+2\mu\left[\calA^*(\vr)+ (\ip{S}{X}-\bz)S \right])U}{U} \\
%		& \geq -\gamma \sqrt{\epsilon} \trace{X^*} - \eps \|U\|_F.
%	\end{align*}
%	Step $(i)$ follows from using Lemma~\ref{lem:compact_optimal_approx} and the gradient magnitude, $\| (C+\tG+2\mu \left[\calA^*(\vr)+ (\ip{S}{X}-\bz)S \right])U\|_F$, is less than $\eps$. \TODO{comment that we only use first order at the very end, not for the lambda min}
\end{proof}


\begin{proof}[Proof of Proposition \ref{prop:compactSDPconsraintA0}]
	One direction is elementary: if there exists $A_0 \succ 0$ and $b_0 \geq 0$ such that $\ip{A_0}{X} = b_0$ for all $X \in \calC$, then,
	\begin{align*}
		\forall X \in \calC, \quad \trace{X} = \ip{I_n}{X} \leq \lambda_{\min}(A_0)^{-1} \ip{A_0}{X} = \lambda_{\min}(A_0)^{-1} b_0.
	\end{align*}
	Thus, the trace of $X \succeq 0$ is bounded, and it follows that $\calC$ is compact. Furthermore: if $b_0 = 0$, then $\calC = \{0\}$; and if $b_0 > 0$, then $0 \notin \calC$.
	
	To prove the other direction, assume $\calC$ is non-empty and compact. If $\calC = \{0\}$, let $A_0 = I_n, b_0 = 0$. Now assume $\calC \neq \{0\}$. The SDP comes in a primal-dual pair:
	\begin{align*}
	\min_{X\in\Snn} \ip{C}{X} \quad & \textrm{ s.t. } \quad \calA(X) = b, \ X \succeq 0, \tag{P} \label{eq:proofP}\\
	\max_{y\in\R^m} \ip{b}{y} \quad & \textrm{ s.t. } \quad C \succeq \calA^*(y). \tag{D} \label{eq:proofD}
	\end{align*}
	It is well known that if~\eqref{eq:proofD} is infeasible, then~\eqref{eq:proofP} is unbounded or infeasible~\citep[Thm.~4.1(a)]{wolkowicz1981optimization}. Since we assume $\calC$ is non-empty, this simplifies to: if~\eqref{eq:proofD} is infeasible, then~\eqref{eq:proofP} is unbounded. The contrapositive states: if~\eqref{eq:proofP} is bounded, then~\eqref{eq:proofD} is feasible. By our compactness assumption on $\calC$, we know that~\eqref{eq:proofP} is bounded for all $C \in \Snn$. Thus,~\eqref{eq:proofD} is feasible for any $C$. In particular, take $C = -I_n$: there exists $-y \in \R^m$ such that $A_0 \triangleq \calA^*(y) \succeq I_n$. Furthermore,
	\begin{align*}
		\forall X \in \calC, \quad \ip{A_0}{X} = \ip{\calA^*(y)}{X} = \ip{y}{\calA(X)} = \ip{y}{b} \triangleq b_0.
	\end{align*}
	Since there exists $X \neq 0$ in $\calC$, it follows that $b_0 > 0$.
\end{proof}

\begin{proof}[Proof of Theorem \ref{thm:compact_eps_fosp}]
	Using~\eqref{eq:gradLmu}, $U$ is an $\eps$-FOSP of the perturbed problem if and only if $\| (M+\tG)U\|_F \leq \frac{\eps}{2}$, where $M = C + 2\mu\calA^*(\vr)$.
	Let $U = P \Sigma Q^T$ be a thin SVD of $U$ ($P$ is $n\times k$ with orthonormal columns; $Q$ is $k\times k$ orthogonal). Then,
	\begin{align*}
	\| (M+\tG)U\|_F & = \| (M+\tG) P \Sigma \|_F \\
	& \geq \sigma_k(U)\| (M+\tG) P\|_F \\
	& \geq  \sigma_k(U) \sqrt{\sum_{i=1}^k \sigma_{n-(i-1)}(M+\tG)^2}.
	\end{align*}
	Hence, we control the smallest singular value of $U$ in terms of $\eps$ and the $k$ smallest singular values of $M+G$:
	\begin{align}
		\sigma_k(U) & \leq \frac{\eps}{2\sqrt{\sum_{i=1}^{k}\sigma_{n-(i-1)}(M+\tG)^2}}.
		\label{eq:keyboundsigmak}
	\end{align}
%	In order to upper-bound $\sigma_k(U)$, we need to lower-bound the denominator.
	%We need to lower-bound $\sum_{i=1}^{k}\sigma_{n-(i-1)}(M+G)^2$ to get an upperbound on $\sigma_k(U)$.
	The next lemma helps lower-bound the denominator---it follows from Theorem 1.16 and Corollary 1.17 in \citep{nguyen2017repulsion}; see proof in Appendix~\ref{apdx:proofNguyen}.
	\begin{lemma}\label{lem:eigenvalue_main}
		Let $\bar M$ be a fixed symmetric matrix of size $n$. Let $G$ be a symmetric Gaussian matrix of size $n$, independent of $\bar M$, with diagonal and upper-triangular entries sampled independently from $\N(0,\sG^2)$. There exists an absolute constant $\const$ such that:
		\begin{align*}
		\Pr{\sum_{i=1}^{k} \sigma_{n-(i-1)}\left(\bar M+G\right)^2 <  \frac{k^2}{\const n} \sG^2 } \leq \exp\left( - \frac{k^2}{8} \log(8 \pi) + k \log (n)\right).
		\end{align*}
	\end{lemma}
	We cannot use Lemma~\ref{lem:eigenvalue_main} directly, as in our case $M$ is not statistically independent of $G$. Indeed, $M$ depends on $U$ through the residue $\vr = \vr(U)$ and $U$ is an $\eps$-FOSP: a feature that depends on $G$.
	To resolve this, we cover the set of possible $M$'s with a net, under the assumption that $\vr$ is bounded. Lemma~\ref{lem:eigenvalue_main} provides a bound for each $\bar M$ in this net. This can be extended to hold for all $\bar M$'s in the net simultaneously via a union bound. By taking a sufficiently dense net, we can then infer that $M$ is necessarily close to one of these $\bar M$'s, and conclude.
	
	Let $\calE$ be the event (on $G$) that $\|\vr\|_2  \leq B$ for all $\eps$-FOSPs of the perturbed problem.
	Conditioned on $\calE$, we have
	\begin{align*}
		\|M - C\|_F & = 2\mu\|\calA^*(\vr)\|_F \leq 2\mu B \|\calA\|,
	\end{align*}
	where $\|\calA\|$ is defined in~\eqref{eq:normofA}.
	As a result, $M$ lies in a ball of center $C$ and radius $2 \mu B\|\calA\|$ in an affine subspace of dimension $\rank(\calA)$.  A unit-ball in Frobenius norm in $d$ dimensions admits an $\varepsilon$-net of $(1+2/\varepsilon)^d$ points~\citep[Cor.~4.2.13]{vershynin2016high}. Thus, we can pick a net with
	%\footnote{Lemma 1.18 in Philippe Rigollet's lecture notes, \url{http://www-math.mit.edu/~rigollet/PDFs/RigNotes15.pdf}: can pack unit ball in 2-norm with at most $(3/\varepsilon)^d$ points.}
	$\left( 1 + \frac{4 \mu  B\|\calA\|}{\sG} \sqrt{\frac{4\const n}{k^2}} \right)^{\rank(\calA)}$ points in such a way that, independently of $\vr$, there exists a point $\bar M$ in the net satisfying:
	\begin{align}
		\| \bar M - M \|_F \leq \sqrt{\frac{k^2}{4\const n}} \sG = \frac{k}{2\sqrt{\const n}}\sG.
		\label{eq:epscover}
	\end{align}
	Let $T \colon \Snn \to \Rk$ be defined by $T_q(A) = (\sigma_{n-q+1}(A), \ldots, \sigma_n(A))^T$, that is: $T$ extracts the $q$ smallest singular values of $A$, in order. Then,
	\begin{align*}
		\| \bar M - M \|_F & = \| (\bar M+G) - (M+G) \|_F \\
						   & \geq \| T_n(\bar M+G) - T_n(M + G) \|_2 \\
						   & \geq \| T_k(\bar M+G) - T_k(M + G) \|_2 \\
						   & \geq \| T_k(\bar M+G) \|_2 - \| T_k(M + G) \|_2,
	\end{align*}
	where the first inequality follows from~\citep[Ex.~IV.3.5]{bhatia2013matrix}. Hence,
	\begin{align}
		\sqrt{\sum_{i=1}^{k}\sigma_{n-(i-1)}(M+\tG)^2} \geq \sqrt{\sum_{i=1}^{k}\sigma_{n-(i-1)}(\bar M+\tG)^2} - \| \bar M - M \|_F. \label{eq:foo}
	\end{align}
	Now, taking a union bound for $\calE$ and for Lemma~\ref{lem:eigenvalue_main} over each $\bar M$ in the net, we get~\eqref{eq:epscover} and
	\begin{align}
		\sqrt{ \sum_{i=1}^{k} \sigma_{n-(i-1)}\left(\bar M+G\right)^2 } \geq \frac{k}{\sqrt{\const n}} \sG
		\label{eq:bar}
	\end{align}
	with probability at least
	$$
		1 - \exp \left( - \frac{k^2}{8}\log(8\pi) + k\log(n) + \rank(\calA) \cdot \log\left( 1 + \frac{4 \mu  B\|\calA\|}{\sG} \sqrt{\frac{4\const n}{k^2}} \right) \right) - \delta.
	$$
	Inside the log, we can safely replace $k$ with 1, as this only hurts the probability. Then, the result holds with probability at least
	$$
		1 - \exp \left( - \frac{k^2}{8}\log(8\pi) + k\log(n) + \rank(\calA) \cdot \log\left( 1 + \frac{8 \mu  B\|\calA\|}{\sG} \sqrt{\const n} \right) \right) - \delta.
	$$
	We aim to pick $k$ so as to ensure
	\begin{align*}
		\exp \left( - \frac{k^2}{8}\log(8\pi) + k\log(n) + \rank(\calA) \cdot \log\left( 1 + \frac{8 \mu  B\|\calA\|}{\sG} \sqrt{\const n} \right) \right) \leq \delta'.
	\end{align*}
	This is a quadratic condition of the form
	\begin{align*}
		-ak^2 + bk + c \leq \log(\delta')
	\end{align*}
	for some $a, b > 0$, $c \geq 0$. Since $k$ is positive we get, $k \geq \frac{b+\sqrt{a (c+\log(1/\delta'))}}{a}$, which is satisfied for, $$ k \geq  3\left[\log\left(\frac{n}{\delta'}\right) + \sqrt{ \rank(\calA)   \log\left( 1 + \frac{8 \mu B\|\calA\| \sqrt{\const  n}}{\sG } \right) }  \right].$$
	
	
%	Hence, the probability of failure conditioned on $\calE$ is bounded by
%	$$
%	\frac{1}{1-\delta}\exp \left( -\frac{k^2}{8}\log(8\pi) + k\log(n) + \rank(\calA) \cdot \log\left(2 \mu \frac{B_1\|\calA\|_{F,2}+ B_2 \|S\|_F }{\sG } \sqrt{\frac{4\const n}{k^2}}\right) \right).
%	$$
%	The total failure probability obeys:
%	\begin{multline*}
%	\Pr{failure} \leq \Pr{failure \vert \calE}\Pr{\calE} + \Pr{\calE^c} \\ \leq \exp \left( -\frac{k^2}{8}\log(8\pi) + k\log(n) + \rank(\calA) \cdot \log\left(2 \mu \frac{B_1\|\calA\|_{F,2}+ B_2 \|S\|_F }{\sG } \sqrt{\frac{4\const n}{k^2}}\right) \right) + \delta.
%	\end{multline*}
	Combining~\eqref{eq:keyboundsigmak},~\eqref{eq:epscover},~\eqref{eq:foo} and~\eqref{eq:bar}, we find:
	\begin{align*}
		\sigma_k(U) \leq \frac{ \epsilon}{\sG } \frac{2\sqrt{\const n}}{k}
	\end{align*}
	 with probability at least $1-\delta-\delta'$.
%	
%	\TODO{ What follows ought to be used somewhere: make sure I didn't forget}
%	
%	By assumption, $\Pr{\calE} \geq 1-\delta$. Hence,
%	\begin{multline*}
%	\Pr{\left. \sum_{i=1}^{k} \sigma_{n-(i-1)}\left(\bar M+G\right)^2 <  \frac{k^2}{4\const n} \cdot \sG^2 \, \right\vert \calE} \\
%	\leq \frac{1}{1-\delta} \Pr{\sum_{i=1}^{k} \sigma_{n-(i-1)}\left(\bar M+G\right)^2 <  \frac{k^2}{4\const n} \cdot \sG^2 } \\
%	\leq \frac{1}{1-\delta}\exp\left( - \frac{k^2}{8} \log(8 \pi) + k \log (n)\right).
%	\end{multline*}
\end{proof}


\begin{proof}[Proof of Lemma \ref{lem:residues_compact}]
	If $U = 0$, the bounds clearly hold: assume $U \neq 0$ in what follows.
	Using $\nabla \tilde L_\mu(U) = 2( C + 2\mu\tilde \calA^*(\tilde \vr) )U$, the definition of $\eps$-FOSP reads:
	\begin{align*}
		\frac{\eps}{2} \geq \left\| \left( C + 2\mu\tilde \calA^*(\tilde \vr) \right) U \right\|_F.
	\end{align*}
	Combining this with $\|A\|_F \geq \frac{1}{\|B\|_F} \ip{A}{B}$ for $B\neq 0$ (Cauchy--Schwarz) gives:
	\begin{align*}
		\frac{\eps}{2} \geq \frac{1}{\|U\|_F} \ip{\left( C + 2\mu\tilde \calA^*(\tilde \vr) \right) U}{U}.
	\end{align*}
	This can be further developed as:
	\begin{align}
		\frac{\eps \|U\|_F}{2} & \geq \ip{C + 2\mu\tilde \calA^*(\tilde \vr)}{UU^T} \nonumber\\
		& = \ip{C}{UU^T} + 2\mu \ip{\tilde\vr}{\tilde\calA(UU^T)} \nonumber\\
		& = \ip{C}{UU^T} + 2\mu \ip{\tilde \calA(UU^T) - \tilde \vb}{\tilde \calA(UU^T)}. \label{eq:residueboundintermediate}
	\end{align}
	At this point, we separate the constraint $(A_0, b_0)$ from the rest, using the usual definition for $(\calA, \vb)$ which capture constraints $1, \ldots, m$:
	\begin{align*}
		\frac{\eps \|U\|_F}{2} & \geq \ip{C}{UU^T}+2\mu\left( \ip{\calA(UU^T)-\vb}{\calA(UU^T)}+ \left(\ip{A_0}{UU^T}-b_0\right)\ip{A_0}{UU^T} \right) \\
		& \geq \ip{C}{UU^T} + 2\mu\left( \|\calA(UU^T)\|_2^2-\|\vb\|_2 \|\calA(UU^T)\|_2 + \left(\ip{A_0}{UU^T}-b_0\right)\ip{A_0}{UU^T} \right).
	\end{align*}
	Let $y =\|\calA(UU^T)\|_2$. Then the above inequality holds when
	$$
	y^2 -\|\vb\|_2 y +\frac{1}{2\mu}\left( \ip{C}{UU^T}-\frac{\eps \|U\|_F}{2} \right)+  \left(\ip{A_0}{UU^T}-b_0\right)\ip{A_0}{UU^T}  \leq 0.
	$$
	For this to happen we need the above quadratic to have real roots. This requires:
	\begin{align*}
	\frac{1}{4}\|\vb\|_2^2 & \geq \frac{1}{2\mu}\left( \ip{C}{UU^T}-\frac{\eps \|U\|_F}{2} \right)+  (\ip{A_0}{UU^T}-\bz)\ip{A_0}{UU^T} \\
	& \geq\frac{1}{2\mu}\left( -\|CU\|_F \|U\|_F-\frac{\eps \|U\|_F}{2} \right)+  \lambda_{\min}(A_0)^2 \|U\|_F^4-\bz \lambda_{\max}(A_0) \|U\|_F^2 \\
	& \geq \lambda_{\min}(A_0)^2 \|U\|_F^4 -\frac{\|C\|_2}{2\mu}\|U\|_F^2 - \bz \lambda_{\max}(A_0) \|U\|_F^2 - \frac{\eps}{4\mu}\|U\|_F,
	\end{align*}
	where we used that for any two matrices $A$ and $B$, it holds that $\|AB\|_F \leq \|A\|_2 \|B\|_F$.
	Focus on the last two terms of the last inequality. We distinguish two cases. Either
	\begin{align*}
		\bz \lambda_{\max}(A_0) \|U\|_F^2 + \frac{\eps}{4\mu}\|U\|_F \geq  \frac{3}{2}\bz \lambda_{\max}(A_0)\|U\|_F^2,
	\end{align*}
	in which case $\|U\|_F \leq \frac{\eps}{2\mu \bz \lambda_{\max}(A_0)}$ (assuming $b_0 > 0$). Or the opposite holds, and:
	\begin{align*}
		\frac{1}{4} \|\vb\|_2^2 & \geq \lambda_{\min}(A_0)^2 \|U\|_F^4 - \left(\frac{\|C\|_2}{2\mu} + \frac{3}{2}\bz \lambda_{\max}(A_0)\right)\|U\|_F^2.
	\end{align*}
%	If $\|\vb\|_2 = 0$, then $\|U\|_F^2 \leq \frac{1}{\lambda_{\min}(A_0)^2}\left(\frac{\|C\|_2}{2\mu} + \frac{3}{2}\bz \lambda_{\max}(A_0)\right)$. Otherwise,
	This is a quadratic inequality in $y = \|U\|_F^2$ of the form $ay^2 - by - c \leq 0$ with coefficients $a> 0$ and $b, c \geq 0$. Such a quadratic always has at least one real root, so that $y \leq \frac{b + \sqrt{b^2 + 4ac}}{2a}$. Furthermore, $\sqrt{b^2 + 4ac} \leq \sqrt{b^2 + (\sqrt{4ac})^2 + 2b\sqrt{4ac}} = b + \sqrt{4ac}$. Hence, $y \leq \frac{b}{a} + \sqrt{\frac{c}{a}}$, which means:
	\begin{align*}
		\|U\|_F^2 & \leq \frac{1}{\lambda_{\min}(A_0)^2} \left(\frac{\|C\|_2}{2\mu} + \frac{3}{2}\bz \lambda_{\max}(A_0)\right) + \frac{\|\vb\|_2}{2\lambda_{\min}(A_0)}.
	\end{align*}
	Accounting for the two distinguished cases, we find:
	\begin{align*}
		\|U\|_F^2 & \leq \max\left\{ \left(\frac{\eps}{2\mu \bz \lambda_{\max}(A_0)}\right)^2, \frac{1}{\lambda_{\min}(A_0)^2} \left(\frac{\|C\|_2}{2\mu} + \frac{3}{2}\bz \lambda_{\max}(A_0)\right) + \frac{\|\vb\|_2}{2\lambda_{\min}(A_0)} \right\}.
	\end{align*}
	
	We now bound the residues (generically) in terms of $\|U\|_F$, using submultiplicativity for $\|UU^T\|_F \leq \|U\|_F^2$ and the definition of $\|\calA\|$~\eqref{eq:normofA}:
	\begin{align*}
		\|\vr\|_2 = \| \calA(UU^T) - \vb \|_2 \leq \|\calA\| \|UU^T\|_F  + \|\vb\|_2 \leq \|\calA\| \|U\|_F^2 + \|\vb\|_2.
	\end{align*}
	Evidently, the same bound holds for $\tilde\calA, \tilde \vb, \tilde \vr$.
\end{proof}

\begin{proof}[Proof of Theorem \ref{thm:optimal_approx_compact}]
%	\TODO{need to be careful what uses $\calA, \vb$ and what uses $\tilde \calA, \tilde \vb$.}
	
	By Lemma~\ref{lem:residues_compact}, for a problem perturbed with $G$, the residues of all $\eps$-FOSPs, $\|\tilde \vr\|_2$, are bounded as:
	\begin{align*}
		\|\tilde \calA\| \max\left\{ \left(\frac{\eps}{2\mu \bz \lambda_{\max}(A_0)}\right)^2, \frac{1}{\lambda_{\min}(A_0)^2} \left(\frac{\|C+G\|_2}{2\mu} + \frac{3}{2}\bz \lambda_{\max}(A_0)\right) + \frac{\|\vb\|_2}{2\lambda_{\min}(A_0)} \right\} + \|\tilde \vb\|_2
	\end{align*}
	With probability at least $1 - \delta$, $\|C + G\|_2 \leq \|C\|_2 + 3\sG\left(\sqrt{n} + \sqrt{2\log(1/\delta)}\right)$. Hence, Theorem~\ref{thm:compact_eps_fosp} applies with this $\delta$ and
	\begin{align*}
		B = \|\tilde \calA\| \max\left\{ \left(\frac{\eps}{2\mu \bz \lambda_{\max}(A_0)}\right)^2, \frac{1}{\lambda_{\min}(A_0)^2} \left(\frac{\|C\|_2 + 3\sG\sqrt{n}}{2\mu} + \frac{3}{2}\bz \lambda_{\max}(A_0)\right) + \frac{\|\vb\|_2}{2\lambda_{\min}(A_0)} \right\} + \|\tilde \vb\|_2.
	\end{align*}
	Hence, with $k$ as prescribed in that theorem for a given $\delta' =\delta \in (0, 1)$, with probability at least $1 - 2\delta$, it holds that
	\begin{align*}
		\sigma_k(U) \leq \frac{ 2\epsilon}{\sG } \frac{\sqrt{\const n}}{k}
	\end{align*}
	for any $\eps$-FOSP. Lemma~\ref{lem:compact_optimal_approx} requires $\sigma_k^2(U) \leq \frac{\gamma \sqrt{\eps}}{8 \mu\|\calA\|^2}$. Hence, we choose: $\eps \leq \left(\frac{\gamma k^2 \sigma_G^2 }{ 32\const n  \mu \|\calA\|^2}\right)^{\sfrac{2}{3}}$, and with probability at least $1-2\delta$  hypothesis of Lemma \ref{lem:compact_optimal_approx} is satisfied. Let $\tilde X$ be a global optimum for $\tilde F_\mu$, then the optimality gap obeys:
\begin{align*}
\tilde F_\mu(UU^T) - \tilde F_\mu(\tilde X) & \leq \gamma \sqrt{\eps} \operatorname{Tr}(\tilde X) + \frac{1}{2}\eps \|U\|_F.
\end{align*}
\end{proof}

\subsection{Proof of section 4.2}
\begin{proof}[Proof of Lemma \ref{lem:residues}]
With probability at least $1-\delta$, $\sigma_1(G) \leq 3\sG\sqrt{n} $. In that event, for $\sG \leq \frac{\lambda_{n}(C)}{6\sqrt{n \log(n/\delta)}}$, we have $C+G \succeq \frac{\lambda_{n}(C)}{2}I$.

$U$ is an $\eps$-FOSP of \eqref{eq:smoothed} implies $\|2(C+\tG+2\mu \calA^*(\vr))U\|_F \leq \eps$. \begin{align*}
\frac{\eps}{2} &\geq \left\| \left(C+\tG+2\mu \calA^*(\vr) \right)U \right\|_F \\
&\geq \frac{1}{\|U\|_F}  \ip{C+\tG+2\mu \calA^*(\vr)}{UU^T} .
\end{align*} Hence, \begin{align*} 
\frac{\eps \|U\|_F}{2}  &\geq \ip{C+\tG}{UU^T} + 2\mu\ip{ \calA^*(\vr)}{UU^T}  \\
&\geq  \frac{\lambda_n(C)}{2}\|U\|_F^2+2\mu\ip{\vr}{\calA(UU^T)} \\
&\geq \frac{\lambda_n(C)}{2}\|U\|_F^2+2\mu( \|\calA(UU^T)\|_2^2 - \|\vb\|_2\|\calA(UU^T)\|_2).
 \end{align*}
The above inequality is  a quadratic in $y=\|\calA(UU^T)\|_2$: $y^2 -y \|\vb\|_2 + \frac{1}{2\mu} \left(\frac{\lambda_n(C)}{2}\|U\|_F^2 -\frac{\eps \|U\|_F}{2}\right) \leq 0$. If $\frac{\eps \|U\|_F}{2} \geq \frac{\lambda_n(C)}{4}\|U\|_F^2$, then  $\|U\| _F \leq \frac{2\eps} {\lambda_n(C)}$. Else, for the above inequality to hold we need the quadratic to have real roots.
\begin{align*}
\|\vb\|_2^2 &\geq 4 \cdot 1 \cdot \frac{1}{2\mu} \left( \frac{\lambda_n(C)}{2}\|U\|_F^2 -\frac{\eps \|U\|_F}{2} \right) \\
&\geq \frac{2}{\mu} \frac{\lambda_n(C)}{4}\|U\|_F^2.
\end{align*}
The last inequality follows from  $\frac{\eps \|U\|_F}{2} \leq \frac{\lambda_n(C)}{4}\|U\|_F^2$. Hence, $\|U\|_F^2 \leq \max \left \{ \left( \frac{2\eps} {\lambda_n(C)}\right)^2, \frac{2\mu}{\lambda_n(C)} \|\vb\|_2^2 \right \}$. Hence,
\begin{multline*}
\|\vr\|_2 = \| \calA(UU^T) -\vb\|_2 \leq \|\calA(UU^T)\|_2 +\|\vb\|_2 \leq \|\calA\| \|UU^T\|_F +\|\vb\|_2 \\ \leq \|\calA\| \max \left \{ \left( \frac{2\eps} {\lambda_n(C)}\right)^2, \frac{2\mu}{\lambda_n(C)} \|\vb\|_2^2 \right \}+\|\vb\|_2.
\end{multline*}
\end{proof}
