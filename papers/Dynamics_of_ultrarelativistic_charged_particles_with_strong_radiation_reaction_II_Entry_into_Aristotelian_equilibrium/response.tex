\documentclass{article}
\usepackage[utf8]{inputenc}
\usepackage{amsfonts}
\usepackage{graphicx}
\usepackage{amsmath}
\usepackage{color}

\begin{document}

Dear Editor, \\

We thank the referee for helpful comments and respond to each below, indicating changes.

\begin{quote}\textit{
1. In section IV.A the authors claim that equilibrium is not reached
and that particle energy grows in a case when parameters are chosen in
such a way that $C_3 > 1$. I think the authors could add a figure
demonstrating the solution of motion equations in that case. Is there
a physical explanation why equilibrium cannot be reached in that case?
Is there always a drift towards the region where condition $C_3 < 1$
is satisfied?}\end{quote}

The explanation is indeed that there is always a drift towards the region where $C_3 \ll 1$ is satisfied.  We have added the requested figure showing this behavior and included a more detailed caption describing the motion.

\begin{quote}\textit{2. In section IV.C the authors consider a circular field
configuration. It seems that this configuration is very similar to the
case where electric and magnetic fields are parallel but rotate with
some angular frequency. Such field configuration appears e.g. in
magnetic nodes of a st circular plane wave. This exemplary
configuration is well-known to laser-plasma community and analogous
equilibrium solutions were originally found in special case ($B = 0$)
by Zeldovich [Y. B. Zel’dovich, Sov. Phys. Usp 18, 79 (1975)] and
later in general case ($B \neq 0$) [I. Yu. Kostyukov, E. N. Nerush,
Phys. Plasmas 23, 093119 (2016)]. It would be interesting to obtain
the result using the approach developed by the authors or at least
comment whether it's even possible.\\
\\
3. In section VI the authors claim that exploring regime where
quasi-static assumption breaks is outside the scope of the paper.
Nevertheless it's indeed really interesting in regards to laser
configurations. There are several configurations known where radiation
reaction leads to non-equilibrium solution, i.e. solutions with
growing particle energy. The first configuration corresponds to a
plasma accelerator (see [I. Yu. Kostyukov, E. N. Nerush, and A. G.
Litvak, Phys. Rev. Spec. Top.–Accel. Beams 15, 111001 (2012)] and [A.
A. Golovanov, E. N. Nerush, and I. Yu. Kostyukov, New J. Phys. 24,
033011 (2022)]), where radiation reaction does not completely stops
the acceleration, but slows it down. The interesting feature of these
solution is that transverse momentum (with regards to direction of
accelerating field) is constant when averaged over a betatron
oscillations. The second configuration is a running plane wave (see
[A. Di Piazza, Lett. Math. Phys. 83, 305–313 (2008); Y. Hadad et.al.
Phys Rev D 82, 096012 (2010); J. E. Gunn and J. P. Ostriker,
Astrophys. J. 165, 523 (1971); M. Grewing, E. Schrüfer, and H.
Heintzmann, Z. Phys. A: Hadrons Nucl. 260, 375–384 (1973).]) where
radiation reaction also leads to growth of the particle energy. In
this scenario transverse momentum also stays constant on average over
a wave period. Another example is motion in standing plane wave [A.
Gonoskov et.al., Phys. Rev. Lett. 113, 014801 (2014)], where in
radiation-dominated regime particle also gain energy. And finally in
the case of parallel uniform fields, which authors consider in Sec.
V.A, it's also clear that transverse momentum stays constants, while
particle energy grows. So it seems that conserving of transverse
momentum (at least on average) might be a signature of an Aristotelian
equilibrium, when energy is not balanced, i.e. Eq. (25) is invalid. So
can the authors comment on whether it's possible to explore this
regime in a manner similar to the one used in the manuscript?}\end{quote}

These are extremely interesting questions, and we wish we had a satisfactory answer for the referee.  However, rather than hazard a guess now, we would prefer to take our time and fully study the literature the referee has mentioned.  Since we come primarily from an astrophysics and relativity background, we are unfamiliar with the results and terminology of the plasma physics community.  We are especially grateful for the long and detailed reading list provided by the referee, but it will take some time for us to digest these results and understand the connection to our methods.  As such, we would prefer not to comment on these issues in the present paper.  We hope the referee understands.\\
\\
All non-trivial changes to the manuscript are indicated in red.

\end{document}