\begin{abstract}
Machine learning models are typically evaluated by computing similarity with reference annotations and trained by maximizing similarity with such.
Especially in the biomedical domain, annotations are subjective and suffer from low inter- and intra-rater reliability.
Since annotations only reflect one interpretation of the \emph{real world}, this can lead to sub-optimal predictions even though the model achieves high similarity scores. 
Here, the theoretical concept of \eac{PGT} is introduced.
\eac{PGT} marks the point beyond which an increase in similarity with the \emph{reference annotation} stops translating to better \eac{RWMP}.
Additionally, a quantitative technique to approximate \eac{PGT} by computing inter- and intra-rater reliability is proposed.
Finally, four categories of \eac{PGT}\emph{-aware} strategies to evaluate and improve model performance are reviewed.
%\keywords{Shared Autonomy, Shared Control Teleoperation, Human-Robot Teaming, Telerobotics, Human-Robot Collaboration, Human Performance Augmentation}
\keywords{Physical Human-Robot Interaction, Telerobotics, Rehabilitation Robotics, Personal Robots, Human Performance Augmentation}

\end{abstract}
