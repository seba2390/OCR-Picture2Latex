\begin{abstract}
Machine learning models are typically evaluated by computing similarity with reference annotations and trained by maximizing similarity with such.
Especially in the biomedical domain, annotations are subjective and suffer from low inter- and intra-rater reliability.
Since annotations only reflect one interpretation of the \emph{real world}, this can lead to sub-optimal predictions even though the model achieves high similarity scores. 
Here, the theoretical concept of \eac{PGT} is introduced.
\eac{PGT} marks the point beyond which an increase in similarity with the \emph{reference annotation} stops translating to better \eac{RWMP}.
Additionally, a quantitative technique to approximate \eac{PGT} by computing inter- and intra-rater reliability is proposed.
Finally, four categories of \eac{PGT}\emph{-aware} strategies to evaluate and improve model performance are reviewed.
%%
%% The code below is generated by the tool at http://dl.acm.org/ccs.cfm.
%% Please copy and paste the code instead of the example below.
%%

\begin{CCSXML}
<ccs2012>
   <concept>
       <concept_id>10003120.10003121.10003124.10010865</concept_id>
       <concept_desc>Human-centered computing~Graphical user interfaces</concept_desc>
       <concept_significance>500</concept_significance>
       </concept>
   <concept>
       <concept_id>10010147.10010371.10010352</concept_id>
       <concept_desc>Computing methodologies~Animation</concept_desc>
       <concept_significance>500</concept_significance>
       </concept>
 </ccs2012>
\end{CCSXML}

\ccsdesc[500]{Human-centered computing~Graphical user interfaces}
\ccsdesc[500]{Computing methodologies~Animation}


%%
%% Keywords. The author(s) should pick words that accurately describe
%% the work being presented. Separate the keywords with commas.
\keywords{Kinetic typography, Animation, Motion transfer}
\end{abstract}
