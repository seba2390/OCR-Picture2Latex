\subsection{Improving Annotation Quality}
One mitigation strategy consists in improving the annotation quality to increase the \emph{annotation's} validity.
This allows shifting \ac{PGT} to the \emph{"right"}, compare \Cref{fig:peak}.
Therefore, the decline in \eac{RWMP} can be expected to occur at a higher level of similarity.
Here, a prominent approach marks the ensembling of annotation entities, for example, by employing consensus voting \citep{klebanov2010some,MaierHein2016, qing-etal-2014-empirical,yang2022neural}.
Interactive \emph{Human in the loop} approaches represent an active field of research trying to develop means to leverage the complementary potentials of machines and humans  \citep{Mosqueira-Rey2022,BUDD2021102062}.
Rädsch et al.\ demonstrate that annotation instructions should also be carefully formulated \citep{radsch2022labeling} to avoid erroneous expert annotations.
