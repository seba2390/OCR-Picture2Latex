\begin{figure}[ht]
\includegraphics[width=1.0\linewidth]{body/fig/micros.png}
\caption{
Dependency on thresholding in light sheet fluorescence microscopy -- slices depicting human neurons from the public \emph{SHANEL} dataset \citep{zhao2020cellular} at different thresholds.
\emph{Left}: the neurons at a low threshold.
\emph{Middle}: the neurons at a medium threshold overlayed with the annotation of the expert neurobiologist.
Colors represent the instances (neurons).
\emph{Right}: the neurons at a high threshold.
Depending on the arbitrary intensity threshold setting, the objects appear in varying sizes.
This effect is particularly pronounced in light sheet microscopy, as the illumination is perpendicular to the objective.
Consequently, the annotations can appear on the full spectrum from \emph{over-} to \emph{under-segmented}.
Notably, only one of the neurons encircled in \emph{magenta} and \emph{lime} is annotated, while neither appears in the \emph{left} image.
This illustrates the subjectivity of human-curated \emph{labels} and thus the discrepancy between \emph{annotations} and \emph{ground truth}.
}
\label{fig:micros}
\end{figure}

These are particularly pronounced in light microscopy.
In fluorescence microscopy, objects appear smaller or larger in size depending on the setting of the threshold intensity; cf. \Cref{fig:micros}.
Consequently, the volumes of the resulting annotations are a function of the threshold setting.
Furthermore, thresholding stands in complex interaction with biological parameters, such as the quality of the applied clearing, the tissue's autofluorescence, or fluorophore amplification \citep{cai2019panoptic}.


It is important to note that comparable issues arise for other imaging techniques.
For instance, \emph{Magnetic Resonance} or \emph{Computed Tomography} images frequently suffer from artifacts induced due to motion, field inhomogeneity, etc.
This should be taken into account when interpreting volumetric measures, such as the popular \eac{DSC}.

