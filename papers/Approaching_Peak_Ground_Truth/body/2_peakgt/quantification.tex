\subsection{Quantitative Approximation}
Where is \eac{PGT} located?
In other words, at which point should one stop optimizing similarity metrics?

The location of \eac{PGT} is determined by two factors.
First, the similarity metrics' ability to capture the \emph{real-world} problem and second, the validity of our annotation entities.
Thus, we propose the following multi-step process to narrow down the location of \eac{PGT}:

First, the \eac{ML} problem needs to be formalized to properly represent the \emph{real-world} problem by selecting an appropriate metric, cf. \Cref{sec:metric_selection}.
Second, we determine the validity of the annotation.
As usual, there is no direct measure for validity; we need to approximate it.
% TODO find validity citation
Therefore, we determine the inter-rater reliability to serve as the lower- and the intra-rater reliability to serve as the upper bound.
Now, the \emph{real} \eac{PGT} with high likelihood lies between these two values.
Here, it is important to note that systematic measurement errors (\emph{lack of validity}) will lead to overestimation of the upper and lower bound.
Consequently, one can escape the \eac{PGT} problem by combining perfect annotations with perfect similarity metric(s).

In a practical example, high inter-rater reliability suggests there is strong agreement between annotators.
Consequently, we expect \eac{PGT} at a higher similarity between model output and \emph{reference}.
On the other hand, high intra-rater reliability indicates strong consistency of the annotation entity.
Exceeding the intra-rater reliability indicates the model started to (randomly) fit the annotation noise.
