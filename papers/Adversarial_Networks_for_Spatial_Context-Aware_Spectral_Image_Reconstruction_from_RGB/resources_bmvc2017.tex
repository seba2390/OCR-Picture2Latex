\begin{figure*}
	\begin{center}
		\fbox{\rule{0pt}{2in} \rule{.9\linewidth}{0pt}}
	\end{center}
	\caption{Example of a short caption, which should be centered.}
	\label{fig:short}
\end{figure*}

%_________________________________________

\begin{table}
	\begin{center}
		\begin{tabular}{|l|c|}
			\hline
			Method & Frobnability \\
			\hline\hline
			Theirs & Frumpy \\
			Yours & Frobbly \\
			Ours & Makes one's heart Frob\\
			\hline
		\end{tabular}
	\end{center}
	\caption{Results.   Ours is better.}
\end{table}

%_________________________________________
\begin{figure}
	\begin{tabular}{ccc}
		\bmvaHangBox{\fbox{\parbox{2.7cm}{~\\[2.8mm]
					\rule{0pt}{1ex}\hspace{2.24mm}\includegraphics[width=2.33cm]{images/eg1_largeprint.png}\\[-0.1pt]}}}&
		\bmvaHangBox{\fbox{\includegraphics[width=2.8cm]{images/eg1_largeprint.png}}}&
		\bmvaHangBox{\fbox{\includegraphics[width=5.6cm]{images/eg1_2up.png}}}\\
		(a)&(b)&(c)
	\end{tabular}
	\caption{It is often a good idea for the first figure to attempt to
		encapsulate the article, complementing the abstract.  This figure illustrates
		the various print and on-screen layouts for which this paper format has
		been optimized: (a) traditional BMVC print format; (b) on-screen
		single-column format, or large-print paper; (c) full-screen two column, or
		2-up printing. }
	\label{fig:teaser}
\end{figure}
%_________________________________________
Saying ``this builds on the work of Lucy Smith [1]'' does not say
that you 

%_________________________________________
 {\em require} 
  using {\tt pdflatex},
  {\bf overlength will not be reviewed}. 
   useful {\em aide	memoire} in 
%_________________________________________
in~\cite{Authors06b}
shortened to ``\etal'' (not ``{\em et.\ al.}'' as ``{\em et}'' is
a complete word.)  The provided \verb'\etal' macro is a useful {\em aide
	memoire} in this regard
We use {\tt natbib}, so citations in random order are nicely sorted:
\cite{Alpher03,Alpher02,Authors06b,Authors06}.  


%_________________________________________

ences (e.g.\ this line is $210.5$), although in
most cases one would expect that the approximate location ($210$ in the
previous example) will be adequate.

%_________________________________________
