\section{Case $k \geq 3$ is \NP-complete}\label{sec:NP}

\begin{problem}[3D matching]
Let $U,V,W$ be disjoint finite sets such that $|U|=|V|=|W|=k$ and let $T$ be a subset of $U\times V\times W$. Is there a set $M\subset T$ with $|M|=k$, such that for any $x\in U\cup V\cup W$ there is exactly one hyperedge $e\in M$ such that $x\in e$.
\end{problem}
\begin{fact}[\cite{b:karp}]
3D matching is \NP-complete.
\end{fact}

\begin{theorem}
The problem of 3D matching can be polynomially reduced to \POISMST problem for 3 graphs. 
\end{theorem}
\begin{proof}
Without loss of generality we assume that every element of $U,V$ and $W$ is element of at least one hyperedge in $T$, otherwise the original 3D matching trivially has no solution. 

We define graphs $G_1,G_2,G_3$ and $H$ where $H = G_1 \cap G_2 \cap G_3$ forming a "sunflower" intersection, that is $H = G_i \cap G_j$ for each $i \neq j$. We associate $G_1$ with $U$, $G_2$ with $V$ and $G_3$ with $W$. 

First put a central vertex $c$ in $H$. For each hyperedge $e\in T$, put a vertex $v_e\in H$ and connect it to $c$ by an edge in $H$ of weight 1. For each element $x\in U$, put a vertex $v_x$ into the exclusive part of $G_1$ ($G_1 \backslash H$) and for every $e \in T$ such that $x \in e$, connect $v_e$ and $v_x$ by an edge of weight 0. Do the same for $V$ and $W$ with graphs $G_2$ and $G_3$ respectively. By construction these graphs form the required "sunflower" configuration. 

The structure of the graph $H$ can be alternatively described as follows. The intersection $H$ contains exactly a star with center $c$ and all edges of weight 1 where each ray represents a different element from $T$. 

Let us focus on $G_1$ and $U$, for the other graphs and sets the arguments are symmetrical. The graph $G_1$ is composed of the central star and exclusive vertices representing elements of the associated set $U$. Every vertex representing an element $x$ is connected via edges of weight 0 to all vertices representing the hyperedges that contain $x$. So for every element $x$, $v_x$ is a center of a weight-0 star in $G_1$. All of these weight-0 stars are disjoint as in each hyperedge there is at most one element from $U$. Since all the edges of weight 0 form an acyclic subgraph of $G_1$, every solution of this \SMST instance must contain all of them. Let $S$ be a solution of the \SMST problem. As for each $x \in U$, the $v_x$ is in the same component as $c$ in $G_1$, it must also be in the same component of $S[E(G_1)]$ and therefore at least one edge $cv_e \in S$ for some hyperedge $e$ such that $x \in e$. If it happened that $cv_f \in S$ for some other hyperedge $f$ with $x \in f$, then $cv_e,v_ev_x,v_xv_f,v_fc$ form a cycle in $G_1$ and we get a contradiction. 

This means that the hyperedges represented by the edges (where $e$ is represented by edge $cv_e$) of weight 1 in $S$ are a solution of the 3D-matching. This is true as each $x \in U$ belongs to exactly one of the hyperedges from $S$ and the same applies to every $y \in V$ and every $z \in W$. 

On the other hand, let $M$ be a solution of the 3D-matching. Then we can construct a solution of the \SMST by simply picking all the edges of weight 0 and all the edges of weight 1 that represent the hyperedgesedges from $M$. As previously, we observe that everything in $G_1$ is connected into a single component. If we only consider the edges of weigh 0 on the other hand, then for each $x,y \in U$ the vertices $v_x$ and $v_y$ are in distinct components and can only be connected via the central star. Therefore any solution must connect $G_1$ into a single component using at least $|U|$ edges of weight 1. Since $|U| = |M|$, the solution of the \SMST constructed from $M$ is clearly minimal. \qed
\end{proof}

\begin{corollary}
The problem \SMST and its variants \OISMST and \POISMST are \NP-complete for 3 and more graphs.
\end{corollary}