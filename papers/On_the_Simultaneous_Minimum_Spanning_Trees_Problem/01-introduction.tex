\section{Introduction} 
The problem of finding a minimum spanning tree is one of the most important and most well-studied problems in graph algorithms. We consider a variant of this problem inspired by the following motivation. 

In a Sunflower land, there is a capital city and several smaller cities around it. In the past, there was a telecommunication company based in the capital city, but it is now bankrupt. The inhabitants of each of the small cities want to establish their own telecommunication company that would connect all of the houses in their city as well as all of the houses in the capital. The representatives of each city meet to coordinate their soon-to-be networks so that they all agree on the capital and can split the cost of covering the capital evenly. However, all of the companies are so afraid of bankruptcy that none of them would accept a solution that would cost them a single dollar more than necessary. Is it always possible to plan all of the networks so that all of the companies reach their goal simultaneously while each of the individual costs is minimized? How hard is it to find such a plan, if it exists, or recognize that it does not exist?

\begin{problem}[Simultaneous Minimum Spanning Trees]
Let $k$ be a positive integer and let $G_1=(V_1, E_1),$ $G_2=(V_2, E_2), \ldots, G_k=(V_k, E_k)$ be graphs and $w$ a non-negative weight function of all of their edges ($w:{\bigcup}_{i=1}^{k} E_i \rightarrow \R^{+}_0$) such that there is a graph $\bar{G}$ satisfying that $\bar{G} = G_i[\bar{V}]$ for any $i$ from 1 to $k$, where $\bar{V} = V_i \cap V_j$ for any $i \neq j$ from 1 to $k$ (i.e. the graphs together form a ``sunflower'' shape with no lateral edges). Find minimum spanning trees $T_i\subseteq G_i$, such that they all coincide on $\bar{G}$, or answer $NO$ if there are no such spanning trees. We shall abbreviate this problem as \SMST.
\end{problem}

Note that the $T_i$'s do not have to induce a spanning tree on $\bar{G}$, nor does the union of $T_i$'s have to be acyclic on the union of all of the $G_i$'s. Indeed both of these situations necessarily happen in solutions of some instances of the \SMST problem. Unlike the minimum spanning tree problem, the \SMST problem does not always have a solution.

As an example, let $G_1$ be a triangle $xzy$, let $G_2$ be a triangle $xwy$ and let $xy$ be the heavies edge. Although $G_1 \cap G_2$ induces a connected graph (edge $xy$), we have a unique solution $\{xz,zy,yw,wx\}$ which is not connected on $G_1 \cap G_2$ and is not acyclic on $G_1 \cup G_2$. Furthermore, if we remove any light edge, e.g. $xz$, then there is no solution. 

We show that \SMST is an \NP-complete problem already for a fairly small number of graphs (more than 2) and even when limited to simplified instances. We present a scheme that allows us to solve any \SMST for two graphs in polynomial time using a tandem of reductions and multiple runs of matroid intersection algorithm. 

\subsection{Preliminaries}
The problem of finding a minimum spanning tree for a single graph has been studied thoroughly since Bor\r uvka \cite{b:boruvka}, Jarn\' ik-Prim \cite{b:jarnik}\cite{b:prim} and Kruskal \cite{b:kruskal}. See \cite{b:mst} for more details. Currently, the optimal algorithm is known \cite{b:mstopt}, but its asymptotics is still an open problem. 

We do not distinguish instances where the input graph is connected from instances where it is disconnected. The inclusion of disconnected instances is natural as many constructions work just as well under such circumstances. Furthermore, usual incremental and iterative approaches typically work on subsets of the input graph and it is therefore not strictly clear whether they maintain a spanning tree or a spanning forest. For convenience, we define the usual term {\em spanning tree} as a maximal acyclic subgraph. In doing so we include the disconnected case, where the more proper term would be {\em spanning forest}. 

We focus mainly on the Kruskal's algorithm and use its known properties. Kruskal's algorithm starts by sorting the edges in a non-decreasing order (by weight) or obtains the edges in a non-decreasing order on input. Then it processes all the edges one by one in sorted order while greedily maintaining maximum acyclic subgraph which we refer to as {\em partial spanning tree}. 

\begin{definition}
Consider the run of Kruskal's algorithm. A {\em stage} is a collection of steps in which the algorithm processes edges of the same weight. 
\end{definition}

\begin{fact}\label{fact:kruskal}
Let $G = (V,E,w)$ be a graph with weighted edges. Then all of the following holds for Kruskal's algorithm applied to graph $G$ and a non-decreasing order of edges $\pi$: 
\begin{itemize}
    \item Kruskal's algorithm is complete (finishes) and correct (answers correctly) for any non-decreasing $\pi$, although the created spanning trees might be different. 
    \item Let $T$ be a minimum spanning tree of $G$ and let $\pi_T$ be the non-decreasing order such that all edges from $T$ are ordered before all edges of the same weight that are not from $T$. Then Kruskal's algorithm using $\pi_T$ outputs exactly $T$. 
    \item After every stage, components of the partial spanning tree span across the same vertices for all non-decreasing $\pi$. 
    \item Edges added to the partial spanning tree in each stage depend only on their ordering, not on the edges chosen in the previous stages. 
    \item Kruskal's algorithm accesses $\pi$ in a read-once fashion, accepting or refusing each edge before accessing the next one. 
\end{itemize}
\end{fact}