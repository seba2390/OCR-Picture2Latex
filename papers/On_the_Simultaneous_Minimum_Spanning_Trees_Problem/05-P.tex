\section{Case $k = 2$ is in \PP}

In this section we show that the general \SMST problem is polynomially solvable. We progress via a tandem of reductions. We already know that the general \SMST can be solved using an algorithm for \OISMST for a cost of some polynomial factor. We further reduce instances of \OISMST to tasks that are more orderly and symmetrical in some sense. We then use this to reduce the task to \POISMST. Finally, we show that solving \POISMST can be reduced to a problem of intersection of two matroids, which is a polynomial problem for two graphs. As an intermediate step, we will also solve the \SST problem by reduction to a matroid intersection problem. 

\begin{definition}
Let $G_1$ and $G_2$ be two graphs intersecting in a common induced subgraph and let $F$ be a subset of edges of $G_1$ and $G_2$. We say that $F$ is {\em simultaneously acyclic} if $F$ restricted to each of the two graphs $G_1$ and $G_2$ forms an acyclic subgraph. 
\end{definition}

\subsection{Reduction of 2-graph \OISMST to 2-graph \POISMST}\label{sec:reductioncap01smst}

For technical reasons we first want to get rid of all edges of weight 1 that cross the boundary in between the intersection and one of the exclusive parts. 

\begin{observation}\label{obs:OInotransedges}
Every \OISMST instance can be transformed to an instance where all of the edges of weight 1 have either both ends in the intersection or both ends in an exclusive part of one graph. This transformation at most doubles the number of edges and vertices. 
\end{observation}
\begin{proof}
This can be achieved by a simple operation that shifts the edges into the exclusive parts. We take each edge $xy$ of weight 1 such that $x$ is in the intersection and $y$ in the exclusive part of one of the two graphs. We subdivide $xy$ into two edges $xz$ and $zy$ where the vertex $z$ lies in the exclusive part of the relevant graph. We set the weight of $xz$ to 0 and the weight of $zy$ to 1. Since the vertex $z$ has degree two, the 0-weight edge $xz$ is an element of each solution of the new \OISMST. It is now easy to see that we can construct the solution of the original \OISMST instance from any solution of the new instance by removing $xz$ and substitution of $zy$ with $xy$ (if it is part of the solution). \qed
\end{proof}

Another issue is that each of the two graphs may require a different number of edges of weight 1, while each edge from the intersection would increase the size of both solutions. 

\begin{observation}\label{obs:OIsamesizesolutions}
Every instance of \OISMST with two graphs $G_1$ and $G_2$ can be transformed into an instance where every minimum spanning tree of $G_1$ and every minimum spanning tree of $G_2$ contain the same number of edges of weight 1. This transformation at most doubles the number of edges and vertices. 
\end{observation}
\begin{proof}
Let $G$ denote the union of $G_1$ and $G_2$ and let $\bar{G}$ denote their intersection. From the properties of the \SKA (lemma \ref{lem:ska}) we know that we can determine beforehand the components of $G_1$ and $G_2$ after all the edges of weight 0 are processed and after all the edges of weight 1 are processed. We also know that in order to compute the restriction of the solution to the edges of weight 1 we do not need to know the exact choice of edges of weight 0, they are in fact independent. By considering the number of components of $G_1$ and $G_2$ just after processing all the edges of weight 0 and after processing all edges, we deduce how many edges of weight 1 must be added into the minimum spanning tree of each graph, which is equal to the difference of the two values. 

Suppose that the solution of \OISMST must contain $j_1$ edges of weight 1 from the graph $G_1$ and $j_2$ edges of weight 1 from the graph $G_2$. If $j_1 = j_2$ then we do not need to modify the instance, otherwise without loss of generality $j_1 > j_2$. We pick an arbitrary vertex $v$ from the exclusive part of $G_2$ and extend $G_2$ by $j_1-j_2$ leaves attached to $v$. All the leaves are new vertices and lie in the exclusive part of $G_2$; and all of the new edges have weight 1. Every spanning tree of $G_2$ must now contain all of these edges, while every solution of the original instance can be extended by exactly these edges. After this modification, $j_1 = j_2'$ where $j_2'$ denotes the new number of weight-one edges in the graph $G_2$ after modification. Note that this construction also works for the case  $j_2 = 0$, although this can be solved directly using \SKA. \qed
\end{proof}

\begin{lemma}\label{lem:OItoPOI}
The \OISMST problem for $k = 2$ is polynomially reducible to \POISMST problem for $k=2$ of asymptotically at most quadratic size. Furthermore if the set of edges of weight 0 of the original \OISMST instance is simultaneously acyclic, then the same is true for the new \POISMST instance. 
\end{lemma}
\begin{proof}
Let us have an instance of \OISMST and let $G_1,G_2$ denote the two graphs and $\bar{G}$ their intersection. Using the previous observation we can assume without loss of generality that all of the weight-1 edges have either both ends contained in $\bar{G}$ or both ends contained in the exclusive part of one of the two graphs; and that there exists a positive integer $j$ such that every solution of the \OISMST constrained to both $G_1$ or $G_2$ has exactly $j$ edges of weight 1. This increases the size of the problem by a small multiplicative constant. 

We modify the problem so that all of the edges from the exclusive parts are removed and equivalently modeled by gadgets that have edges of weight 1 only in $\bar{G}$. To do this, we consider all pairs $e=(e_1,e_2),f=(f_1,f_2)$ of edges of weight 1 such that $e$ is from the exclusive part of $G_1$ and $f$ is from the exclusive part of $G_2$. We create two new vertices $x^{ef}_1,x^{ef}_2$ in $\bar{G}$ and add edges $e_1x^{ef}_1,e_2x^{ef}_2,f_1x^{ef}_1,f_2x^{ef}_2$ of weight 0 and an edge $x^{ef}_1,x^{ef}_2$ of weight 1. After processing all pairs, we delete all the edges of weight 1 from the exclusive parts. 

Let $M$ be a solution of the modified instance of \OISMST (which is in fact \POISMST). First we observe that whenever $x^{ef}_1x^{ef}_2 \in M$ for some removed edges $e$ and $f$ then $x^{eg}_1x^{eg}_2 \notin M$ for any $g \neq f$ as otherwise $e_1,x^{ef}_1,x^{ef}_2,e_2,x^{eg}_2,x^{eg}_1,e_1$ forms a cycle in $M[G_1]$. To get a solution of the original instance, we remove all the extra edges of weight 0 and replace each edge $x^{ef}_1x^{ef}_2$ by edges $e$ and $f$. Let us denote the resulting set of edges $M'$. Consider the graph $G_1$ and the components defined by $M'$ restricted to $G_1$. It is easy to see that the components are the same as in $M$ with the exception of the new vertices which are now isolated. Also, the total weight of $M'$ restricted to each graph (of the original instance) is the same as the total weight of $M$ restricted to each graph (of the modified instance). We conclude that $M'$ is a minimum simultaneous spanning tree.

\begin{figure}
    \centering
    \includegraphics{fig-POIgadget.pdf}
    \caption{Gadget replacing pairs of edges}
    \label{fig:my_label}
\end{figure}

On the other hand, let $\bar{M}$ be a solution of the original instance. Since there is the same amount of edges of weight 1 in $\bar{M}$ restricted to $G_1$ and $G_2$, we can pair all of the edges from $\bar{M}$ of weight 1 that are in the exclusive parts of $G_1$ and $G_2$. We can now replace each pair of edges $e$ and $f$ by $x^{ef}_1x^{ef}_2$. After adding all the new edges of weight 0, we get a solution of the modified instance. Therefore the total cost of $\bar{M}$ is at most a total cost of the given solution. 

Supposing that the original edges of weight 0 form a simultaneously acyclic set, we observe that the same is true after the reduction, as each new cycle contains an edge of weight 1. Furthermore we added at most a constant number of edges and vertices for all of the pairs of original edges, obtaining a problem of asymptotically at most quadratic size compared to the input problem. \qed
\end{proof}

\subsection{Matroids}

\begin{definition}
A {\em matroid} $M$ is a pair $(E,I)$ where $E$ is a set of elements and $I$ is a family of independent sets (subsets of $E$) satisfying the following properties: 

\begin{enumerate}
    \item $\emptyset \in I$ 
    \item $\forall X,Y$ s.t. $ X \in I$ and $ Y \subset X$ : $Y \in I$
    \item $\forall X,Y \in I$ s.t. $|X| > |Y|$ : $\exists x \in X\setminus Y$ s.t. $Y \cup \{x\} \in I$
\end{enumerate}
\end{definition}

\begin{definition}
Let $G$ be a graph with a set of edges $E$ and $I$ be a set of all acyclic subsets of $E$. Then $(E,I)$ is a {\em graphic matroid} of $G$. 
\end{definition}

\begin{fact}
For any graph $G$ (possibly multigraph with loops), the graphic matroid of $G$ is a matroid and maximal independent sets of this matroid are exactly all possible spanning trees of $G$. 
\end{fact}

\begin{definition}
A {\em matroid intersection problem} of two matroids $(E,I_1)$ and $(E,I_2)$ on the same set of elements $E$ is the problem of finding a maximum subset of $E$ s.t. it is independent in both matroids. 
\end{definition}

\begin{fact}[\cite{b:edmonds}]
For a set $E$ and two matroids $(E,I_1)$ and $(E,I_2)$ given as independence oracles, the matroid intersection problem is solvable in polynomial time and polynomially many oracle queries. 
\end{fact}

\begin{fact}[\cite{b:graphicintersection}]
There are specialized algorithms for graphic matroid intersection problem. 
\end{fact}

\begin{lemma}\label{lem:grmatroid}
Let $G$ be a graph with edges divided into two disjoint subsets $F$ and $\bar{E}$ where $F$ is acyclic and $\bar{E} = E(G) \backslash F$. Let $I$ be a set of all subsets $X$ of $\bar{E}$ such that $F \cup X$ is an acyclic subgraph of $G$. Then $(\bar{E},I)$ is a graphic matroid. 
\end{lemma}
\begin{proof}
Let $H$ denote $G$ with all edges from $F$ contracted; we keep all the parallel edges and loops. We observe that the graphic matroid of $H$ is exactly $(\bar{E},I)$. \qed
\end{proof}

\subsection{Polynomiality}

\begin{theorem}\label{thm:SSTinP}
\SST $\in$ \PP for any number of graphs. 
\end{theorem}
\begin{proof}
To solve \SST, it suffices to use Kruskal's algorithm (or any other MST algorithm) to first take a minimum spanning tree of the intersection, and then extend this partial solution to each individual graph using only exclusive edges. Clearly each exclusive edge may only create a cycle in its respective graph. On the other hand we are never forced to take an exclusive edge closing a cycle (in fact, Kruskal's algorithm refuses such edges by definition). \qed
\end{proof}

%\begin{proof}
%Let us set the weights of all edges to 1. We now have a \OISMST instance with the property that the set of edges of weight 0 is empty and therefore simultaneously acyclic. We use the reduction from Lemma \ref{lem:OItoPOI} obtaining a new \POISMST instance in which the edges of weight 0 are also simultaneously acyclic. Finally, the previous Lemma \ref{lem:acyclic01} shows we can solve this instance in polynomial time using a matroid intersection algorithm. \qed
%\end{proof}
%We would like to remark that while maximum independent sets of a graphic matroid correspond to spanning trees, in general, the \SST problem cannot be solved using the intersection of matroids directly. As an example, consider two graphs where $G_1$ is a single edge $xy$ and $G_2$ is a set of triangles $xyw_i$ for $i$ going from 1 to an arbitrarily high value. If we were to look for a maximum set of edges which is simultaneously acyclic (that is to use intersection of two graphic matroids), we would get the set of all edges except $xy$. This set clearly does not give a spanning tree on $G_1$, nor can it be successfully extended. On the other hand we have plenty of correct solutions, all of roughly half the size. This is why we need to ensure that the number of edges of any solution cannot be exceeded by any simultaneously acyclic set of edges. This is achieved by moving all the edges into the intersection, using auxiliary edges to keep the structure of the original problem. 

\begin{lemma}\label{lem:acyclic01}
Let $I$ be an instance of \POISMST for two graphs such that the edges of weight 0 form a simultaneously acyclic set. Then $I$ can be solved in polynomial time using a matroid intersection algorithm. 
\end{lemma}
\begin{proof}
For each of the two graphs $G_i$ for $i \in \{1,2\}$ we define $F_i$ as the set containing all edges of weight 0 and $\bar{E}$ the set of all edges of weight 1. Let $I_i$ be a set of all subsets $X$ of $\bar{E}$ such that $X \cup F_i$ is acyclic in $G_i$ and let $M_i$ denote the pair $(\bar{E},I_i)$. According to Lemma \ref{lem:grmatroid} each $M_i$ is a matroid. Furthermore, both of the matroids are defined on the same ground set $\bar{E}$.

Let $F = F_1 \cup F_2$ be all the edges of weight 0. By lemma \ref{lem:selfred}, $F$ can be extended to a solution of the \POISMST by a suitable subset of $\bar{E}$. We can now use a (graphic) matroid intersection algorithm to find a set $X$ which is a maximum subset of $\bar{E}$ independent in both matroids $M_1$ and $M_2$. Therefore $X$ is the maximum subset of $\bar{E}$ that extends $F$ so that $X \cup F$ is simultaneously acyclic. If $X \cup F$ restricted to $G_1$ and $G_2$ spans all components, we output $X \cup F$, otherwise we answer "no". This is the same as to compare the size of $X \cup F$ to the size it should have.

Clearly if there exists a solution of the given \POISMST instance, then according to Lemma~\ref{lem:selfred} there exists a solution $Y$ extending the set $F$. The set of edges $Y \backslash F$ is an independent set in both matroids $M_1$ and $M_2$ and therefore $X$ exists and is of size $|Y \backslash F|$. This means that $X \cup F$ is a simultaneous spanning tree and the algorithm answers correctly. On the other hand, if no solution exists, then the set $X \cup F$ restricted to either $G_1$ or $G_2$ is acyclic but does not connect all the vertices connected in the original graph. We recognize this case and answer "no" correctly. \qed
\end{proof}

\begin{lemma}\label{lem:POIinP}
\POISMST $\in$ \PP for two graphs.
\end{lemma}
\begin{proof}
Let $I$ be an instance of the \POISMST problem. We show that we can solve $I$ using a (graphic) matroid intersection algorithm. 

First suppose that the edges of weight 0 are not simultaneously acyclic. We simply restrict $I$ to edges of weight 0, which gives us an instance of \SST. We can solve this instance in polynomial time according to Theorem \ref{thm:SSTinP}. If we obtain answer "no", then according to Lemma \ref{lem:selfred} there is no solution and we also answer "no". 

Suppose we get a solution $X$. Then, by Lemma \ref{lem:selfred}, we may delete all the edges of weight 0 except the edges from $X$ and further assume that the edges of weight 0 are simultaneously acyclic. We use Lemma~\ref{lem:acyclic01} to solve this reduced instance in polynomial time. \qed
\end{proof}

\begin{theorem}\label{thm:SMSTinP}
\SMST $\in$ \PP for two graphs.
\end{theorem}
\begin{proof}
Let us have an instance of the \SMST problem. According to Lemma \ref{lem:01reduction}, every instance of \SMST can be solved by solving at most $O(m)$ \OISMST problems, where $m$ denotes the number of edges on input. 

Any \OISMST can be polynomially reduced to \POISMST as shown in Lemma \ref{lem:OItoPOI}; and according to Lemma \ref{lem:POIinP}, each \POISMST instance can be solved in polynomial time. \qed
\end{proof}

\subsection{Complexity}

Let us have an instance of \SMST and let $n$ denote the number of vertices, $m$ the number of edges and $w$ the number of weights in the given instance. We proceed according to Theorem~\ref{thm:SMSTinP}. 

The \SMST problem is first decomposed into $(w-1)$ \OISMST subproblems. We observe that each edge in these subproblems is either already fixed as a part of the solution of \SMST or appears for the first time. The first kind of edges can be bound as at most $O(n)$ per \OISMST subproblem, as they must form a simultaneously acyclic set. The second kind can be bound as at most $O(m)$ over all of the \OISMST subproblems. 

Each of the \OISMST subproblems is reduced to a \POISMST problem of asymptotically at most quadratic size (by Lemma \ref{lem:OItoPOI}). Using the simultaneous acyclicity of edges of weight 0 we can use the approach of Lemma~\ref{lem:acyclic01} in all but the first subproblem, and use the Lemma~\ref{lem:POIinP} to solve the first subproblem. Therefore we solve at most $w$ (graphic) matroid intersection problems during the whole process and one instance of \SST problem. The final complexity depends on the choice of algorithms used to solve the matroid intersection problems and the \SST. 

Furthermore, if $w$ asymptotically approaches $m$, then some weight values have few representatives and more direct methods from Observation~\ref{cor:trivialstages} and Observation~\ref{obs:PIOSMST} using \SKA may be applied to reduce the complexity. 
