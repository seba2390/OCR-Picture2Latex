\section{Simultaneous Kruskal's algorithm}

Consider a \SMST task for a given $k$ and graphs $G_1, G_2, \dots, G_k$. Let us denote the union of all $G_i$s as $G$ and their intersection as $\bar{G}$. Suppose we order all the edges of $G$ in a non-decreasing order $\pi$ which we call a {\em universal order} and denote $\pi[E(G_i)]$ the restrictions of $\pi$ to edges in $G_i$ for every $i$. For a set of edges $F$, we also say that a universal order is {\em F-preferring} if all the edges from $F$ are ordered before any other edges of the same weight. 

Consider the following construction. First, we fix an arbitrary non-decreasing universal order $\pi$. We simulate $k$ independent instances of Kruskal's algorithm, $K_1,K_2,...,K_k$ where the job of each $K_i$ is to find a minimum spanning tree $T_i$ of $G_i$ using the order $\pi[E(G_i)]$, not considering the other instances. In parallel with the instances of the Kruskal's algorithm we try to incrementally build a simultaneous minimum spanning tree. 

In the beginning, we start with an empty simultaneous spanning tree $T$ and process all the edges one by one according to the universal order. We present each edge $e$ to all instances $K_i$ such that $e \in G_i$. If we assume a sunflower intersection, we can rephrase this in the following way: if $e \in \bar{G}$ then we present $e$ to all instances and if $e \notin \bar{G}$ then $e \in G_j$ for some unique $j$ and we present $e$ only to one instance $K_j$. If every invoked $K_i$ adds $e$ to its local $T_i$, we also add $e$ to $T$. If every invoked instance $K_i$ refuses to add $e$ to its local $T_i$, we also throw $e$ away. If the invoked instances do not agree, we fail. If the algorithm processes all edges without failing, we output $T$ as a solution. 

We call this construction {\em Simultaneous Kruskal's algorithm} or \SKA in short. There are two natural versions of the \SKA. If \SKA expects the universal order $\pi$ on input, then it is a deterministic algorithm. Alternatively, \SKA may be formulated as a non-deterministic algorithm which guesses a correct universal order which avoids failure (if any such order exists), then we speak of a non-deterministic simultaneous Kruskal's algorithm or \NSKA in short. We naturally extend the definition of a stage from Kruskal's algorithm to the \NISKA as the collection of steps in which the algorithm processes edges of the same weight. 

\begin{lemma}\label{lem:ska}
Let $I$ be an instance of the \SMST problem. Then all of the following holds for simultaneous Kruskal's algorithm: 

\begin{itemize}
    \item \NSKA is complete (finishes) and correct (answers correctly).  
    \item Let $T$ be a solution of $I$ and a let $\pi_T$ be a $T$-preferring universal order. Then \SKA using $\pi_T$ outputs exactly $T$. 
    \item After every successful stage of \SKA and \NSKA, components of the partial simultaneous spanning tree after restriction to any $G_i$ span across the same vertices for all choices of universal order $\pi$. 
    \item Edges added to the partial simultaneous spanning tree in each stage depend only on their ordering, not on the edges chosen in the previous stages. 
    \item \SKA accesses $\pi$ in a read-once fashion, accepting or refusing each edge before accessing the next one. 
\end{itemize}
\end{lemma}
\begin{proof}
Let us first prove the second point. Suppose we run the \SKA using the $T$-preferring universal order. Let us analyze the behavior of an arbitrary $K_i$. Let $T_i$ denote the restriction of $T$ to $G_i$. By definition $T_i$ is a minimum spanning tree of $G_i$ and $\pi[G_i]$ is a $T_i$-preferring order. From the properties of the Kruskal's algorithm (fact \ref{fact:kruskal}) we know that $K_i$ constructs exactly $T_i$. Since every $K_i$ would construct exactly $T_i$ should it run on its own, we observe that all the invoked instances $K_j$ accept each edge if and only if it belongs to $T$, and the whole algorithm never fails. At the end of the computation the algorithm gives exactly $T$ as a solution. 

To prove correctness, let us first suppose that the \NSKA terminates with success. Then the set $T$ on output is a union of the local spanning trees from all $K_i$ algorithms. Since each algorithm $K_j$ processes all the edges from $G_j$ in a non-decreasing order of weight and Kruskal's algorithm is sound, each of the local spanning trees is a minimum spanning tree. Thus, $T$ is a solution of the \SMST problem. If \NSKA terminates with a failure, then from the second point it follows that there was no solution $T$, as otherwise \NSKA guesses a $T$-preferring universal order and terminates successfully.

The last three points are simple observations extending the facts \ref{fact:kruskal} into simultaneous setting using the previous two points.\qed
\end{proof}
