% File tacl2021v1.tex
% Dec. 15, 2021

% The English content of this file was modified from various *ACL instructions
% by Lillian Lee and Kristina Toutanova
%
% LaTeXery is mostly all adapted from acl2018.sty.

\documentclass[11pt,a4paper]{article}
\usepackage{times,latexsym}
\usepackage{url}
\usepackage[T1]{fontenc}
\usepackage{bbm}
\usepackage{makecell}
\usepackage{graphicx}
\usepackage{amsmath}
\usepackage{multirow}
\usepackage{bigstrut}
\usepackage{subfigure}
\usepackage{diagbox}
\usepackage{footnote}
\usepackage{mathtools}
\usepackage{epstopdf}
\usepackage[dvipsnames,table,xcdraw]{xcolor}

%% Package options:
%% Short version: "hyperref" and "submission" are the defaults.
%% More verbose version:
%% Most compact command to produce a submission version with hyperref enabled
%%    \usepackage[]{tacl2021v1}
%% Most compact command to produce a "camera-ready" version
%%    \usepackage[acceptedWithA]{tacl2021v1}
%% Most compact command to produce a double-spaced copy-editor's version
%%    \usepackage[acceptedWithA,copyedit]{tacl2021v1}
%
%% If you need to disable hyperref in any of the above settings (see Section
%% "LaTeX files") in the TACL instructions), add ",nohyperref" in the square
%% brackets. (The comma is a delimiter in case there are multiple options specified.)

\usepackage[acceptedWithA]{tacl2021v1}
% \setlength\titlebox{10cm} % <- for Option 2 below

%%%% Material in this block is specific to generating TACL instructions
\usepackage{xspace,mfirstuc,tabulary}
\usepackage{subfigure}
\usepackage{graphicx}
\newcommand{\dateOfLastUpdate}{Dec. 15, 2021}
\newcommand{\styleFileVersion}{tacl2021v1}

\newcommand{\ex}[1]{{\sf #1}}

\newif\iftaclinstructions
\taclinstructionsfalse % AUTHORS: do NOT set this to true
\iftaclinstructions
\renewcommand{\confidential}{}
\renewcommand{\anonsubtext}{(No author info supplied here, for consistency with
TACL-submission anonymization requirements)}
\newcommand{\instr}
\fi

%
\iftaclpubformat % this "if" is set by the choice of options
\newcommand{\taclpaper}{final version\xspace}
\newcommand{\taclpapers}{final versions\xspace}
\newcommand{\Taclpaper}{Final version\xspace}
\newcommand{\Taclpapers}{Final versions\xspace}
\newcommand{\TaclPapers}{Final Versions\xspace}
\else
\newcommand{\taclpaper}{submission\xspace}
\newcommand{\taclpapers}{{\taclpaper}s\xspace}
\newcommand{\Taclpaper}{Submission\xspace}
\newcommand{\Taclpapers}{{\Taclpaper}s\xspace}
\newcommand{\TaclPapers}{Submissions\xspace}
\fi

\newcommand{\NAME}[0]{\textsc{RR}}
%%%% End TACL-instructions-specific macro block
%%%%

\usepackage{amsmath}
\DeclareMathOperator*{\argmax}{arg\,max}
\DeclareMathOperator*{\argmin}{arg\,min}

\title{Rethinking with Retrieval: Faithful Large Language Model Inference}

% Author information does not appear in the pdf unless the "acceptedWithA" option is given

% The author block may be formatted in one of two ways:

% Option 1. Author’s address is underneath each name, centered.
\iffalse
\author{
  Template Author1\Thanks{The {\em actual} contributors to this instruction
    document and corresponding template file are given in Section
    \ref{sec:contributors}.} 
  \\
  Template Affiliation1/Address Line 1
  \\
  Template Affiliation1/Address Line 2
  \\
  Template Affiliation1/Address Line 2
  \\
  \texttt{template.email1example.com}
  \And
  Template Author2 
  \\
  Template Affiliation2/Address Line 1
  \\
  Template Affiliation2/Address Line 2
  \\
  Template Affiliation2/Address Line 2
  \\
  \texttt{template.email2@example.com}
}
\fi

\author{Hangfeng He$^\dag$\thanks{\, Part of this work was done while the author was at the University of Pennsylvania.}\qquad Hongming Zhang$^\ddagger$ \qquad Dan Roth$^\mathsection$ \\
  $^\dag$University of Rochester\qquad
  $^\ddagger$Tencent AI Lab, Seattle \qquad
  $^\mathsection$University of Pennsylvania\\
\texttt{hanfeng.he@rochester.edu}, \texttt{hongmzhang@global.tencent.com} \\ \texttt{danroth@seas.upenn.edu}
}

% % Option 2.  Author’s address is linked with superscript
% % characters to its name, author names are grouped, centered.

% \author{
%   Template Author1\Thanks{The {\em actual} contributors to this instruction
%     document and corresponding template file are given in Section
%     \ref{sec:contributors}.}$^\diamond$ 
%   \and
%   Template Author2$^\dagger$
%   \\
%   \ \\
%   $^\diamond$Template Affiliation1/Address Line 1
%   \\
%   Template Affiliation1/Address Line 2
%   \\
%   Template Affiliation1/Address Line 2
%   \\
%   \texttt{template.email1example.com}
%   \\
%   \ \\
%   \\
%   $^\dagger$Template Affiliation2/Address Line 1
%   \\
%   Template Affiliation2/Address Line 2
%   \\
%   Template Affiliation2/Address Line 2
%   \\
%   \texttt{template.email2@example.com}
% }

\date{}

\begin{document}
\maketitle
\begin{abstract}
Despite the success of large language models (LLMs) in various natural language processing (NLP) tasks, the stored knowledge in these models may inevitably be incomplete, out-of-date, or incorrect. This motivates the need to utilize external knowledge to assist LLMs. Unfortunately, current methods for incorporating external knowledge often require additional training or fine-tuning, which can be costly and may not be feasible for LLMs. To address this issue, we propose a novel post-processing approach, \textit{rethinking with retrieval} (\NAME{}), which retrieves relevant external knowledge based on the decomposed reasoning steps obtained from the chain-of-thought (CoT) prompting. This lightweight approach does not require additional training or fine-tuning and is not limited by the input length of LLMs. We evaluate the effectiveness of \NAME{} through extensive experiments with GPT-3 on three complex reasoning tasks: commonsense reasoning, temporal reasoning, and tabular reasoning. Our results show that \NAME{} can produce more faithful explanations and improve the performance of LLMs.\footnote{Our code is publicly available at \url{https://github.com/HornHehhf/RR}.}

\end{abstract}

\section{Introduction} 
The problem of finding a minimum spanning tree is one of the most important and most well-studied problems in graph algorithms. We consider a variant of this problem inspired by the following motivation. 

In a Sunflower land, there is a capital city and several smaller cities around it. In the past, there was a telecommunication company based in the capital city, but it is now bankrupt. The inhabitants of each of the small cities want to establish their own telecommunication company that would connect all of the houses in their city as well as all of the houses in the capital. The representatives of each city meet to coordinate their soon-to-be networks so that they all agree on the capital and can split the cost of covering the capital evenly. However, all of the companies are so afraid of bankruptcy that none of them would accept a solution that would cost them a single dollar more than necessary. Is it always possible to plan all of the networks so that all of the companies reach their goal simultaneously while each of the individual costs is minimized? How hard is it to find such a plan, if it exists, or recognize that it does not exist?

\begin{problem}[Simultaneous Minimum Spanning Trees]
Let $k$ be a positive integer and let $G_1=(V_1, E_1),$ $G_2=(V_2, E_2), \ldots, G_k=(V_k, E_k)$ be graphs and $w$ a non-negative weight function of all of their edges ($w:{\bigcup}_{i=1}^{k} E_i \rightarrow \R^{+}_0$) such that there is a graph $\bar{G}$ satisfying that $\bar{G} = G_i[\bar{V}]$ for any $i$ from 1 to $k$, where $\bar{V} = V_i \cap V_j$ for any $i \neq j$ from 1 to $k$ (i.e. the graphs together form a ``sunflower'' shape with no lateral edges). Find minimum spanning trees $T_i\subseteq G_i$, such that they all coincide on $\bar{G}$, or answer $NO$ if there are no such spanning trees. We shall abbreviate this problem as \SMST.
\end{problem}

Note that the $T_i$'s do not have to induce a spanning tree on $\bar{G}$, nor does the union of $T_i$'s have to be acyclic on the union of all of the $G_i$'s. Indeed both of these situations necessarily happen in solutions of some instances of the \SMST problem. Unlike the minimum spanning tree problem, the \SMST problem does not always have a solution.

As an example, let $G_1$ be a triangle $xzy$, let $G_2$ be a triangle $xwy$ and let $xy$ be the heavies edge. Although $G_1 \cap G_2$ induces a connected graph (edge $xy$), we have a unique solution $\{xz,zy,yw,wx\}$ which is not connected on $G_1 \cap G_2$ and is not acyclic on $G_1 \cup G_2$. Furthermore, if we remove any light edge, e.g. $xz$, then there is no solution. 

We show that \SMST is an \NP-complete problem already for a fairly small number of graphs (more than 2) and even when limited to simplified instances. We present a scheme that allows us to solve any \SMST for two graphs in polynomial time using a tandem of reductions and multiple runs of matroid intersection algorithm. 

\subsection{Preliminaries}
The problem of finding a minimum spanning tree for a single graph has been studied thoroughly since Bor\r uvka \cite{b:boruvka}, Jarn\' ik-Prim \cite{b:jarnik}\cite{b:prim} and Kruskal \cite{b:kruskal}. See \cite{b:mst} for more details. Currently, the optimal algorithm is known \cite{b:mstopt}, but its asymptotics is still an open problem. 

We do not distinguish instances where the input graph is connected from instances where it is disconnected. The inclusion of disconnected instances is natural as many constructions work just as well under such circumstances. Furthermore, usual incremental and iterative approaches typically work on subsets of the input graph and it is therefore not strictly clear whether they maintain a spanning tree or a spanning forest. For convenience, we define the usual term {\em spanning tree} as a maximal acyclic subgraph. In doing so we include the disconnected case, where the more proper term would be {\em spanning forest}. 

We focus mainly on the Kruskal's algorithm and use its known properties. Kruskal's algorithm starts by sorting the edges in a non-decreasing order (by weight) or obtains the edges in a non-decreasing order on input. Then it processes all the edges one by one in sorted order while greedily maintaining maximum acyclic subgraph which we refer to as {\em partial spanning tree}. 

\begin{definition}
Consider the run of Kruskal's algorithm. A {\em stage} is a collection of steps in which the algorithm processes edges of the same weight. 
\end{definition}

\begin{fact}\label{fact:kruskal}
Let $G = (V,E,w)$ be a graph with weighted edges. Then all of the following holds for Kruskal's algorithm applied to graph $G$ and a non-decreasing order of edges $\pi$: 
\begin{itemize}
    \item Kruskal's algorithm is complete (finishes) and correct (answers correctly) for any non-decreasing $\pi$, although the created spanning trees might be different. 
    \item Let $T$ be a minimum spanning tree of $G$ and let $\pi_T$ be the non-decreasing order such that all edges from $T$ are ordered before all edges of the same weight that are not from $T$. Then Kruskal's algorithm using $\pi_T$ outputs exactly $T$. 
    \item After every stage, components of the partial spanning tree span across the same vertices for all non-decreasing $\pi$. 
    \item Edges added to the partial spanning tree in each stage depend only on their ordering, not on the edges chosen in the previous stages. 
    \item Kruskal's algorithm accesses $\pi$ in a read-once fashion, accepting or refusing each edge before accessing the next one. 
\end{itemize}
\end{fact}
\section{Related Work}

\paragraph{Enhancing LMs through retrieval.} Retrieval-enhanced LMs have received significant attention as a means of improving performance through the incorporation of external knowledge. For example, the k-most similar training contexts can be retrieved to improve the estimation of the next word distribution in both the training stage \cite{borgeaud2021improving} and the inference stage \cite{khandelwal2019generalization}. Furthermore, search query generators have been adopted to generate search queries for search engines to retrieve relevant documents \cite{komeili2022internet, shuster2022language, thoppilan2022lamda}. Other approaches have utilized retrieved documents as the additional context in generation tasks \cite{joshi2020contextualized, guu2020retrieval, lewis2020retrieval}. \citet{nakano2021webgpt} instead use human feedback in a text-based web-browsing environment. Among these previous works, \citet{khandelwal2019generalization} is most closely related to our approach. However, they focus on improving local inference by using the nearest neighbor datastore constructed from training data, whereas we focus on conducting faithful inference using external knowledge. In contrast to other aforementioned approaches, which require training or fine-tuning to incorporate retrieved knowledge, we propose a post-processing method for leveraging retrieved knowledge without additional training or fine-tuning.

\paragraph{Incorporating external knowledge into LMs.} Significant effort has been devoted to leveraging external knowledge to improve the reasoning ability of LMs. Previous work has incorporated external knowledge sources such as WordNet \cite{miller1995wordnet} and ConceptNet \cite{speer2017conceptnet} to enhance LMs for tabular reasoning tasks \cite{neeraja2021incorporating, varun2022trans}. Explicit rules have also been added to inputs to improve reasoning ability over implicit knowledge \cite{talmor2020leap}. In addition, explicit knowledge from Wikidata \cite{vrandevcic2014wikidata} and implicit knowledge in LLMs have been integrated into a transformer \cite{vaswani2017attention} for visual question answering \cite{gui2021kat}. \citet{nye2021improving} instead introduces a symbolic reasoning module to improve coherence and consistency in LLMs. Among these previous works, \citet{nye2021improving} is the most relevant to our approach. Still, they focus on incorporating logical constraints to improve coherence and consistency, whereas we aim to improve the faithfulness of explanations through the use of external knowledge. In contrast to other aforementioned approaches that incorporate external knowledge before generation and require additional training or fine-tuning, our proposal leverages external knowledge in a post-processing manner to enhance LMs without additional training or fine-tuning.

\paragraph{Uncovering latent Knowledge in LLMs.} There has been a line of work exploring the knowledge hidden within LLMs for reasoning. This has included the use of careful prompting to encourage LLMs to generate explanations in the reasoning process, such as through chain of thought prompting in few-shot \cite{wei2022chain} or zero-shot \cite{kojima2022large} learning, or through the use of scratchpads for intermediate computation \cite{nye2022show}. In addition, various methods based on sampling a diverse set of reasoning paths in LLMs have been proposed, including training verifiers to judge the correctness of model completions \cite{cobbe2021training}, calibrating model predictions based on the reliability of the explanations \cite{ye2022unreliability}, and promoting self-consistency over diverse reasoning paths \cite{wang2022self}. \citet{zelikman2022star} instead iteratively bootstrap the ability of LLMs to generate high-quality rationales from a few initial examples. \citet{liu2022generated} further propose generating knowledge from LLMs, which is then used as additional input to improve commonsense reasoning. In contrast to this line of work, our proposal focuses on leveraging external knowledge to enhance LLMs, while they aim to explore the knowledge hidden within LLMs.







\section{Rethinking with Retrieval}
\label{sec:framework}

LLMs have been shown to generate incorrect supporting facts from time to time, even when they accurately capture the perspective needed to answer a question. This phenomenon highlights intrinsic issues in the way LLMs store and retrieve knowledge, including (1) the presence of out-of-date, incorrect, or missing relevant knowledge in the pre-training corpus; (2) incorrect memorization of relevant knowledge during pre-training; and (3) incorrect retrieval of relevant knowledge during the inference stage. To address these issues, we propose the use of \NAME{}, which leverages external knowledge through the retrieval of relevant information based on decomposed reasoning steps.

\paragraph{Overview.} Given a query $Q$, we utilize chain-of-thought prompting to generate a diverse set of reasoning paths $R_1, R_2, \cdots R_N$, where each reasoning path $R_i$ consists of an explanation $E_i$ followed by a prediction $P_i$. After that, we retrieve relevant knowledge $K_1, \cdots K_M$ from a suitable knowledge base $\mathcal{KB}$ to support the explanation in each reasoning path, and select the prediction $\hat{P}$ that is most faithful to this knowledge. To better illustrate our proposal, we use ``\textit{Did Aristotle use a laptop?}'' as a running example in this work.

\paragraph{Chain-of-thought prompting.}  In contrast to standard prompting, CoT prompting \cite{wei2022chain} includes demonstrations of step-by-step reasoning examples in the prompt
to produce a series of short sentences that capture the reasoning process. For instance, given the question ``\textit{Did Aristotle use a laptop?}'', CoT prompting aims to generate the complete reasoning path ``Aristotle died in 322 BC. The first laptop was invented in 1980. Thus, Aristotle did not use a laptop. So the answer is no.'' rather than simply outputs ``No.'' Empirical results show that CoT prompting significantly improves the performance of LLMs on many multi-step reasoning tasks. Therefore, we adopt CoT prompting to obtain both explanation $E$ and prediction $P$ for the query $Q$. 

\paragraph{Sampling diverse reasoning paths.} Similar to \citet{wang2022self}, we sample a diverse set of reasoning paths $R_1, R_2, \cdots R_N$ rather than only considering the greedy path as in \citet{wei2022chain}. For the question ``\textit{Did Aristotle use a laptop?}'', the potential reasoning paths can be as follows:
\begin{itemize}
    \item[($R_1$)] Aristotle died in 2000. The first laptop was invented in 1980. Thus, Aristotle used a laptop. So the answer is yes.
    \item[($R_2$)] Aristotle died in 322BC. The first laptop was invented in 2000. Thus, Aristotle did not use a laptop. So the answer is no.
    \item[($R_3$)] Aristotle died in 322BC. The first laptop was invented in 1980. Thus, Aristotle did not use a laptop. So the answer is no.
\end{itemize}

\paragraph{Knowledge retrieval.} Different knowledge bases can be used to address different tasks. For example, to address the question ``\textit{Did Aristotle use a laptop?}'', we can use Wikipedia as the external knowledge base $\mathcal{KB}$. Information retrieval techniques can be applied to retrieve the relevant knowledge $K_1, \cdots K_M$ from Wikipedia based on the decomposed reasoning steps. Ideally, we would obtain the following two paragraphs from Wikipedia for this question:
\begin{itemize}
    \item[($K_1$)] Aristotle (384–322 BC) was a Greek philosopher and polymath during the Classical period in Ancient Greece. ...
    \item[($K_2$)] The Epson HX-20, the first laptop computer, was invented in 1980. ...
\end{itemize}

\paragraph{Faithful inference.} 
The faithfulness of each reasoning path $R_i$ can be estimated using a function $f_{\mathcal{KB}}(R_i)$, which is based on relevant knowledge $K_1, \cdots, K_M$ retrieved from the knowledge base $\mathcal{KB}$. The final prediction is obtained through the application of the following inference procedure\footnote{Note that this is the basic version of faithful inference, and further variations can be found in Section \ref{subsec:variations}.}:
\begin{equation}
\hat{P} = \argmax_{P_i \in\{P_1, \cdots, P_N\}} \sum_{i=1}^N \mathbbm{1}(P_i=P) f_{\mathcal{KB}}(R_i),
\label{eq:inference}
\end{equation}
where $P_i$ denotes the corresponding prediction in the reasoning path $R_i$. This inference procedure is designed to identify the most faithful prediction $\hat{P}$ to the knowledge base among all predictions in the $N$ reasoning paths. For instance, in the running example, given reasoning paths $R_1, R_2, R_3$ and the retrieved knowledge $K_1, K_2$, the above inference procedure would output the prediction ``So the answer is no.'', as it is supported by both $R_2$ and $R_3$ and has a higher faithfulness score compared to the prediction ``So the answer is yes.'', which is only supported by $R_1$. 


\section{Experimental Setups}
We evaluate the proposed model with several baselines to show how well the model can produce the appropriately inflected form of a given lemma.

\subsection{Corpus}
We conduct experiments in a Japanese-to-English translation task with the ASPEC corpus \citep{nakazawa-etal-2016-aspec}.
This corpus consists of abstracts from scientific articles, which tend to contain many technical terms.
Such words are rare and hard for the model to learn the correct translation, and thus this corpus fits the typical use-case of lexically constrained translation.
We use the initial 1M sentence pairs from the training split for training.

\subsection{Word Dictionary}
In this study, lexical constraints in translation are introduced through a source-to-target word dictionary. We construct the dictionary automatically from the ASPEC corpus through the following procedure.

First, we obtain the word alignment by feeding the first 1M sentence pairs of the training split and validation/test splits to \texttt{GIZA++}.\footnote{\url{https://github.com/moses-smt/giza-pp}} We tokenize Japanese sentences with \texttt{Mecab}\footnote{\url{https://taku910.github.io/mecab/}} and English sentences with \texttt{spaCy}.\footnote{\url{https://spacy.io/}} We then construct a phrase table and extract only those with more than 100 occurrences.
Then, we split the dictionary into noun and verb entries to facilitate the analysis of the results and remove noise. If both the Japanese and English phrases are noun phrases, the entry is registered in the noun dictionary. If the Japanese phrase is a nominal verb\footnote{The nominal verb (\ja{サ変動詞}) is the most productive class of verb in Japanese and many new or technical terms fall into this category ({\it e.g.}, \ja{最適化する}-{\it optimize}, \ja{過学習する}-{\it overfit}).} and English is a verb, the entry is registered in the verb dictionary.
In this study, we evaluate the model's ability to inflect a provided lemma. Lemmas for the target language (English) are obtained with \texttt{spaCy}.

\subsection{Models}
As the baseline, we implement a Transformer \citep{NIPS2017_3f5ee243} translation model based on \texttt{AllenNLP} \citep{Gardner2017AllenNLP}.
We configure the model in the Transformer-base setting and sentences are tokenized using \texttt{sentencepiece} \citep{kudo-2018-subword}, which has a shared source-target vocabulary of about 16k sub-words.
The overviews of lexically constrained models are summarized in \fig{fig:baseline}.

\minisection{Placeholder (PH)}
In the placeholder method, the model is trained to translate sentences with a placeholder token and pass that through to the translation. In our experiments, we use different placeholder tokens \texttt{[NOUN]} and \texttt{[VERB]} for nouns and verbs.
Predicted placeholder tokens are replaced by the pre-specified term in the post-processing step.
We evaluate three types of placeholder baselines, each of which differs in what inflected form the target placeholder token is replaced with: {\bf PH (oracle)}, where the pre-specified term is embedded in the same form as in the reference; {\bf PH (lemma)}, always the lemma form; {\bf PH (common)}, the most common inflected forms in the training data, which are the singular form for \texttt{[NOUN]} and the past tense form for \texttt{[VERB]}. The results of PH (lemma) and PH (common) are provided as naive baselines to give a sense of how difficult predicting the correct inflected form is.

We also provide a baseline that performs word inflection through an external resource ({\bf PH (morph)}).
As in \citet{tamchyna-etal-2017-modeling}, words that need inflection are followed by morphological tags, and word formation is realized through an external resource.
We use \texttt{LemmInflect}\footnote{\url{https://github.com/bjascob/LemmInflect}} to decompose the dictionary entries with their lemma and part-of-speech tags and to recover the inflected word form.
As this model uses an external resource to perform inflection, it is not directly comparable with our proposed models but we provide its results as an oracle baseline.

\minisection{Code-switching (CH)}
The code-switching model replaces a phrase in a source sentence with the corresponding target phrase according to a bilingual dictionary.\footnote{\citet{dinu-etal-2019-training} utilize source factors that indicate which tokens are code-switched, but we observe no significant difference by adding source factors. Therefore, we simply report the results from the model with minimal components.}
{\bf CH (oracle)} uses the same target words as in the reference, and {\bf CH (lemma)} uses the lemma form.

\minisection{Proposed Model}
We implement our proposed model described in \cref{sec:approach} on top of the placeholder baseline model.
Compared to the baseline, our proposed model has three additional modules: the target context encoder, target character embeddings, and character-level decoder.
The embedding and hidden sizes are all set to 512, which is the same as in the Transformer-base model.
The additional encoder and decoder have two layers, and the feedforward dimension is 1024.


Note that, for all the models, we restrict the number of constraints to at most one in each sentence as an initial investigation. This favors the placeholder-based models as handling more than one placeholder introduces additional complexity in the system and tends to degrade the performance, while the code-switching methods suffer less from multiple constraints \citep{song-etal-2019-code}.
We leave experiments with multiple constraints to future work.


\begin{figure}[t]
\centering
\includegraphics[width=14.0cm]{data/baselines.png}
\caption{The preprocessing of the lexically constrained baseline models.}
\label{fig:baseline}
\end{figure}



\subsection{Training with Lexical Constraints}
To apply lexical constraints, the models are trained with data augmentation.
Augmented data is created for all sentences that contain any of the source and target phrases found in the dictionary entries.
To control the amount of augmented data to around 10\% of the original training data, we restrict the dictionary entries to infrequent ones.
The restriction to infrequent phrases also simulates real-word use-cases, where user-specified terms are often rare words that typical NMT models struggle with in translation.
Specifically, we restrict the noun entries to ones with a count at most 20, and the verb entries to 2000.
The threshold is chosen to balance the amount of noun and verb entries in the augmented data.


\subsection{Optimization}
We optimize the models using Adam \citep{Kingma2015AdamAM} with the Noam learning rate scheduler with 8000 warmup steps \citep{NIPS2017_3f5ee243}. The training is stopped when the validation BLEU score does not improve for 3 epochs.

For our proposed model, we found that optimizing the word-level modules and character-level modules separately stabilizes the training process and improves the translation quality.
We first train a normal placeholder model, use the weights to initialize those of our proposed model, and then only update the parameters of the additional modules.
In this second training stage, we use the loss value as validation metric and stop the training when the lowest value is updated for 5 epochs.

\begin{table*}[t]
\centering
\scalebox{1.0}{
\begin{tabular}{c|m{0.4\linewidth}|m{0.4\linewidth}}
\Xhline{2\arrayrulewidth}
  & \multicolumn{1}{c|}{Wrong supporting facts} & \multicolumn{1}{c}{Wrong inference} \bigstrut[t] \bigstrut[b] \\ \hline
Question & Was Lil Jon's top ranked Billboard song a collaboration with a member of The Lox?
 & Would the top of Mount Fuji stick out of the Sea of Japan? 
 \bigstrut[t] \bigstrut[b] \\ \hline
Answer & \multicolumn{1}{c|}{No} & \multicolumn{1}{c}{Yes} \bigstrut[t] \bigstrut[b]  \\ \hline
Facts & Lil Jon's highest ranked billboard song was Yeah; Yeah was a collaboration between Lil Jon, Usher, and Ludacris; The Lox is a rap trio consisting of: Styles P, Sheek Louch, and Jadakiss. & The average depth of the Sea of Japan is  5,748 feet (1,752 metres) and its maximum depth is 12,276 feet (3,742 metres); Mount Fuji is 3,776.24 metres (12,389.2 ft) tall.  \bigstrut[t] \bigstrut[b]  \\ \hline
GPT-3 & \textcolor{red}{Lil Jon's top ranked Billboard song was "Get Low"} with the East Side Boyz. The Lox is not a member of the East Side Boyz. Thus, Lil Jon's top ranked Billboard song was not a collaboration with a member of The Lox. So the answer is no.  & Mount Fuji is 3,776 meters tall. The Sea of Japan is about 3,741 meters deep. \textcolor{red}{Thus, the top of Mount Fuji would not stick out of the Sea of Japan.} So the answer is no.
 \bigstrut[t] \bigstrut[b]  \\ \Xhline{2\arrayrulewidth}
\end{tabular}}
\caption{Examples of incorrect outputs from GPT-3 with CoT prompting.} 
\label{table:gpt-3-outputs}
\end{table*}




\section{Analysis}

In this section, we perform a thorough analysis to gain a deeper understanding of \NAME{}.

\subsection{Limitations of LLMs in Reasoning}
\label{subsec:limitations}
In this subsection, we present an analysis of GPT-3 with CoT prompting on the StrategyQA dataset. Upon closer examination of the outputs of GPT-3, we observed that it can provide reasonable explanations and correct predictions for a number of questions. For example, when given the question ``\textit{Will the Albany in Georgia reach a hundred thousand occupants before the one in New York?}'', GPT-3 produced the following output:

\begin{quote}
\textcolor{cyan}{The Albany in New York has a population of about 98,000. The Albany in Georgia has a population of about 77,000.} \textcolor{YellowOrange}{Thus, the Albany in New York is more populous than the Albany in Georgia.} \textcolor{LimeGreen}{So the answer is no.}
\end{quote}

The above output consists of three components: (1) supporting facts (in cyan) that are based on a particular perspective, (2) chaining arguments (in orange), and (3) a prediction (in green). Components (1) and (2) contribute to the explanation. Overall, the output exhibits a high level of quality. However, we also observed that GPT-3 may occasionally produce incorrect supporting facts for its explanations or make incorrect inferences for its predictions, despite generally being able to identify suitable perspectives. 

\paragraph{Wrong supporting facts.} As shown in Table \ref{table:gpt-3-outputs}, GPT-3 provides the incorrect supporting fact for Lil Jon's top-ranked Billboard song, stating that it was ``Get Low'' instead of the correct answer, ``Yeah''. However, it does have the correct perspective on how to answer the question, ``\textit{Was Lil Jon’s top ranked Billboard song a collaboration with a member of
The Lox?}''.

\paragraph{Wrong inference.} As shown in Table \ref{table:gpt-3-outputs}, GPT-3 makes an incorrect inference, stating that the top of Mount Fuji ``would not stick out'' of the Sea of Japan, rather than the correct answer, ``would stick out''. However, it does provide correct supporting facts based on the appropriate perspective for the question, ``\textit{Would the top of Mount Fuji stick out of the Sea of Japan?}''. 

\subsection{Ablation Study}


\begin{table}[t]
\centering
\scalebox{0.92}{
\begin{tabular}{c|c|c}
\hline
Retrieval & Commonsense & Tabular  \bigstrut[t]  \\ \hline
Query-based & 73.36 & 36.69 \bigstrut[t] \bigstrut[b] \\ \hline
Decomposition-based & {\bf 77.73} & {\bf 39.05}  \bigstrut[t] \bigstrut[b] \\ \hline
\end{tabular}}
\caption{Comparison of query-based and decomposition-based retrieval on commonsense and tabular reasoning.
}
\label{table:retrieval-analysis}
\end{table} 

\paragraph{Importance of decomposition-based retrieval.} In our proposed method, we retrieve relevant external knowledge based on the decomposed reasoning steps rather than the original query. To further investigate the impact of this choice, we conducted additional experiments in which we used the original query for knowledge retrieval while keeping other aspects of our method unchanged. As shown in Table \ref{table:retrieval-analysis}, the results for these experiments are poor for both commonsense and temporal reasoning, indicating the importance of using decomposition-based retrieval in our approach.

\begin{table}[t]
\centering
\scalebox{1.0}{
\begin{tabular}{c|c}
\hline
Knowledge & Tabular  \bigstrut[t]  \\ \hline
External & 79.92 \bigstrut[t] \bigstrut[b] \\ \hline
Background &  84.75 \bigstrut[t] \bigstrut[b] \\ \hline
Background + External & {\bf 84.83}  \bigstrut[t] \bigstrut[b] \\ \hline
\end{tabular}}
\caption{Performance of \NAME{} with different types of knowledge on tabular reasoning: external only, background only, and a combination of both. External knowledge refers to WordNet and ConceptNet, while background knowledge refers to the tables.
}
\label{table:knowledge-analysis}
\end{table} 
\begin{figure*}[t]
		\centering
        \hspace{0.01in}
		\subfigure[Accuracy of predictions]{
			\centering
			\includegraphics[scale=0.35]{figures/answers_accuracy_model_sizes.pdf}
			\label{fig:answer-accuracy-model-size}}
        \hspace{0.01in} 
        	\subfigure[Faithfulness of explanations]{
			\centering
		\includegraphics[scale=0.35]{figures/reasons_factuality_model_sizes.pdf}
			\label{fig:reason-factuality-model-size}}
		\caption{The effect of LM size on the performance of our proposed method (Variant II) and CoT prompting. We use various sizes of OPT models, with the exception of the 175B model, which is GPT-3.}
\label{fig:model-size}
\end{figure*}
\paragraph{The impact of different types of knowledge.} For tabular reasoning, we use both external knowledge (WordNet and ConceptNet) and background knowledge (tables) in our experiments. In this section, we further examine the effect of different types of knowledge on the performance of our proposed method. As shown in Table \ref{table:knowledge-analysis}, the additional improvement gained by incorporating Wikidata and ConceptNet in addition to tables is limited, indicating that GPT-3 already captures many word-level relations in these external knowledge sources. In addition, the observed significant improvement in tabular reasoning from using tables alone suggests that our proposed method can also effectively leverage background knowledge.

\subsection{Variations of the Proposed Approach}
\label{subsec:variations}
\paragraph{Basic approach: Weighting outputs.} In Section \ref{sec:framework}, we present a basic version of our proposal for taking advantage of external knowledge. Our basic approach involves \textit{weighting outputs as individual units} and using a \textit{voting} mechanism to select the best-supported prediction. We can also directly choose the best-supported output, which includes both an explanation and a prediction, without using voting. For example, in the running example of ``\textit{Did Aristotle use a laptop?}'' (see more in Section \ref{sec:framework}), the third reasoning path $R_3$ is the output most supported by the knowledge paragraphs $K_1$ and $K_2$.

\paragraph{Variant \uppercase\expandafter{\romannumeral1}: Fact selection.} The first variant of our approach involves selecting facts from the outputs of LLMs based on external knowledge. For example, consider the running example of ``\textit{Did Aristotle use a laptop?}'', where we only have access to the first two reasoning paths, $R_1$ and $R_2$. In this case, the first sentence in $R_2$ and the second sentence in $R_1$ are supported by knowledge $K_1$ and $K_2$, respectively. Therefore, the first variant would output the first sentence in $R_2$ and the second sentence in $R_1$ as the supporting facts.

\paragraph{Variant \uppercase\expandafter{\romannumeral2}: Fact generation.} The second variant of our approach involves generating facts based on both the outputs of LLMs and external knowledge. For example, consider the running example of ``\textit{Did Aristotle use a laptop?}'', where we only have access to the first reasoning path $R_1$. The second sentence in $R_1$ is supported by the second knowledge paragraph $K_2$. However, the first sentence is not supported by any evidence paragraphs. We can generate questions about the first sentence, such as ``When did Aristotle die?'' and use the first knowledge paragraph $K_1$ to generate a new fact: ``Aristotle died in 322BC.''. As a result, the second variant would output the generated fact ``Aristotle died in 322 BC.'' and the second sentence in $R_1$ as the supporting facts.

\paragraph{Inference with supporting facts.} For the two variants of our approach, we only have the supporting facts and need to perform a final inference step to obtain the corresponding prediction. One option for this inference is to use LLMs, but they can be costly \cite{brown2020language} or difficult to use \cite{zhang2022opt}. An alternative is to use an off-the-shelf model for inference with supporting facts, such as UnifiedQA \cite{khashabi2020unifiedqa, khashabi2022unifiedqa}. As discussed in Appendix \ref{subsec:inference-comparison}, UnifiedQA is more robust to noisy supporting facts than GPT-3. We thus use the second version of UnifiedQA, UnifiedQA-v2 \cite{khashabi2022unifiedqa}, for the final step of inference.

\paragraph{Experimental settings.} In this part, we focus on commonsense reasoning and use the \textit{evidence paragraphs} provided in StrategyQA as the relevant knowledge, rather than the retrieved paragraphs discussed in Section \ref{subsec:commonsense}. To evaluate the quality of the explanations, we adopt the best metric for factual consistency evaluation in \citet{honovich2022true}. For simplicity, we use the pre-trained NLI model released by \citet{nie2020adversarial} to compute the NLI-based metric, rather than fine-tuning T5-11B \cite{raffel2020exploring} ourselves. The implementation details of the two variants can be found in Appendix \ref{subsec:variants-implementation}.

\begin{table}[t]
\centering
\scalebox{0.88}{
\begin{tabular}{c|c|c}
\hline
Methods & Accuracy (\%) & Faithfulness (\%) \bigstrut[t]  \\ \hline
CoT prompting & 65.94 & 38.73 \bigstrut[t] \bigstrut[b] \\ \hline \hline
Basic (w/o voting) & 76.86 & 50.02 \bigstrut[t] \bigstrut[b] \\ \hline
Variant \uppercase\expandafter{\romannumeral1} & {\bf 78.60} & 54.11 \bigstrut[t] \bigstrut[b] \\ \hline
Variant \uppercase\expandafter{\romannumeral2} & {\bf 78.60} & {\bf 54.54} \bigstrut[t] \bigstrut[b] \\ \hline
\end{tabular}}
\caption{Comparison of various variations of \NAME{} and the CoT prompting baseline on StrategyQA using evidence paragraphs.
}
\label{table:proposal-variants}
\end{table} 



\paragraph{Results.} Table \ref{table:proposal-variants} illustrates that the fact selection and fact generation variants of our proposal improve the faithfulness of the supporting facts in explanations, leading to increased prediction accuracy compared to the basic approach without voting. Across all variations of our proposal, we observe significant improvements in both prediction accuracy and the faithfulness of explanations when compared to the CoT prompting baseline.

The incorporation of a voting mechanism leads to an increased prediction accuracy of $79.91\%$ for the basic approach. Comparison with the performance (i.e., $77.73\%$) of the same approach using retrieved paragraphs rather than evidence paragraphs in Table \ref{table:gpt3-results} demonstrates that retrieved paragraphs are also effective for our proposal, as both significantly outperform the voting baseline, self-consistency (i.e., $73.36\%$), as shown in Table \ref{table:gpt3-results}.

It is noteworthy that UnifiedQA performs poorly on StrategyQA, achieving an accuracy of only $58.95\%$. However, when provided with gold supporting facts in StrategyQA, UnifiedQA demonstrates excellent performance with an accuracy of $90.83\%$. This suggests that UnifiedQA is suitable for last-step inference, but not effective for answering questions in StrategyQA.

\subsection{Impact of the Size of LMs}
\label{subsec:model-size}
In this subsection, we examine the effect of the size of LMs on the performance of our proposed method, specifically in the context of the fact generation variant. We compare the performance of our method using various sizes of OPT models \cite{zhang2022opt} in addition to GPT-3 (175B) using the same experimental setup as in Section \ref{subsec:variations}. As shown in Figure~\ref{fig:model-size}, our proposed method (Variant II) consistently outperforms CoT prompting in terms of both prediction accuracy and the faithfulness of explanations, even when using smaller LMs.


\section{Conclusion and Future Work}
In this study, we point out that the traditional placeholder translation method embeds the specified term into the generated translation without considering the context of the placeholder token, which potentially leads to grammatically incorrect translations.
To address this shortcoming, we proposed a flexible placeholder translation model that handles inflection when the specified term is given in the form of a lemma.
In the experiment of the Japanese-to-English translation task, we showed that the proposed model can inflect user-specified terms more accurately than the code-switching method.

Future work includes testing the proposed method on morphologically-rich languages or extending the model to handle more than one placeholder in a sentence.
Also, the proposed model still has room for improvement to learn inflection.
It is possible that we can improve the model by exploiting monolingual corpora in the target language to provide additional training signals for learning the correct inflection in context.


\clearpage
\bibliography{new}
\bibliographystyle{acl_natbib}

\clearpage
\appendix
\begin{table*}[ht]
\centering
\scalebox{1.0}{
\begin{tabular}{m{1.0\linewidth}}
\hline
{\bf Q:} who was governor of minnesota when maathaad maathaadu mallige was released? \bigstrut[t] \\
{\bf A:} Maathaad Maathaadu Mallige was released on 24 August 2007. Tim Pawlenty served as the 39th governor of Minnesota from 2003 to 2011. Thus, Tim Pawlenty was governor of minnesota when maathaad maathaadu mallige was released. {\bf So the answer is Tim Pawlenty.} \\
{\bf Q:} who was us president during the costa rican civil war? \bigstrut[t]  \\
{\bf A:} The Costa Rican civil war was a civil war in Costa Rica from 12 March to 24 April 1948. Harry S. Truman was the 33rd president of the United States, serving from 1945 to 1953. Thus, Harry S. Truman was us president during the costa rican civil war. {\bf So the answer is Harry S. Truman.} \\
{\bf Q:} who was governor of oregon when the collector was released? \bigstrut[t]\\
{\bf A:} The Collector premiered at the Cannes Film Festival on May 20, 1965. Mark Hatfield served as the 29th governor of Oregon from 1959 to 1967. Thus, Mark Hatfield was governor of oregon when the collector was released. {\bf So the answer is Mark Hatfield.}\\
{\bf Q:} who was governor of oregon when shanghai noon was released? \bigstrut[t] \\
{\bf A:} Shanghai Noon was released on May 26, 2000. John Kitzhaber served as the 35th governor of Oregon from 1995 to 2003. Thus, John Kitzhaber was governor of oregon when shanghai noon was released. {\bf So the answer is John Kitzhaber.}\\
{\bf Q:} who was us president when john andrew shulze was a teenager? \bigstrut[t] \\
{\bf A:} John Andrew Shulze was born on July 19, 1775. A teenager is someone who is between 13 and 19 years old. George Washington served as the first president of the United States from 1789 to 1797. Thus, George Washington was us president when john andrew shulze was a teenager. {\bf So the answer is George Washington.}\\
{\bf Q:} who was us president during the seventh coalition? \bigstrut[t] \\
{\bf A:} The War of the Seventh Coalition was from 20 March to 8 July 1815. James Madison served as the fourth president of the United States from 1809 to 1817. Thus, James Madison was us president during the seventh coalition. {\bf So the answer is James Madison.} \bigstrut[b]\\
\hline
\end{tabular}}
\caption{The CoT prompt for temporal reasoning.
}
\label{table:temporal-prompt}
\end{table*} 

\section{Appendix}
\label{sec:appendix}

In this section, we provide additional details on our experimental setup. Further information can be found in our code.

\subsection{Detailed Prompts}
\label{subsec:prompts}

We adopt the same CoT prompt for commonsense reasoning (i.e., StrategyQA) as those presented in \citet{wei2022chain}. The CoT prompt for temporal reasoning is provided in Table \ref{table:temporal-prompt}. For tabular reasoning, we adopt the method of \citet{brown2020language} for converting NLI into QA for RTE \cite{dagan2005pascal}, and randomly sample $6$ examples from the training data to construct the prompt, as shown in Table \ref{table:tabular-prompt}. The few-shot prompt utilizes the same exemplars as the CoT prompt and does not involve CoT reasoning processes.

\subsection{Description of Faithfulness Functions}
\label{subsec:faithfulness-functions}
For a sentence $s$, we denote its MPNet similarity, entailment score, and contradiction score as $M(s)$, $E(s)$, and $C(s)$, respectively. In our experiments, the corresponding thresholds for these scores are $T_m = 0.5$, $T_e = 0.6$, and $T_c = 0.99$. Given the entailment scores, contradiction scores, and MPNet similarities of all supporting facts (denoted as $S$) in the explanation of a reasoning path $R$, different faithfulness functions $f_{\mathcal{KB}}(\cdot)$ can be adopted in different settings as follows:
\begin{itemize}
    \item[(1)] $f_{\mathcal{KB}}(R) = \sum_{s \in S}
 [M(s) \times (M(s) >= T_m) + E(s) \times (M(s) < T_m)  - C(s)]$
    \item[(2)] $f_{\mathcal{KB}}(R) = \sum_{s \in S} [M(s) + E(s)]$
    \item[(3)] $f_{\mathcal{KB}}(R) = \sum_{s \in S} [E(s) \times (E(s) >= T_e) - C(s) \times (C(s) >= T_c)]$
\end{itemize}

In Section \ref{sec:experiments}, we employ function (1) for commonsense and tabular reasoning. For temporal reasoning, we use function (2) as the distinct nature of sentences converted from temporal relations leads to unreliable contradiction scores. In Sections \ref{subsec:variations}-\ref{subsec:model-size}, we use function (3) for commonsense reasoning with evidence paragraphs, as the high quality of the relevant knowledge negates the need for the complementary use of the MPNet similarity to improve the entailment score.


\subsection{Comparison of Retrieval Systems}
\label{subsec:retrieval-comparison}

For commonsense reasoning, we utilized different retrieval systems in \citet{karpukhin2020dense} to retrieve relevant paragraphs from Wikipedia. The performance of BM25, DPR, and BM25+DPR were $77.73\%$, $58.52\%$, and $77.29\%$, respectively, indicating that BM25 is the best choice in our case.

\iffalse
\subsection{Converting Temporal Relations to Sentences}
\label{subsec:temporal-templates}
\subsection{Converting Word Relations to Sentences}
\label{subsec:word-templates}
\fi

\subsection{Implementation Details for the Two Variants of \NAME{}}
\label{subsec:variants-implementation}

\paragraph{Fact selection implementation details.} In this work, we utilize the information present in the top-ranked output produced by our basic approach as a guide. To this end, we apply a greedy clustering algorithm to group the sentences from all outputs into distinct topic categories based on the cosine similarity of their MPNet sentence embeddings. For each fact in the top-ranked output of our basic approach, we identify the fact with the highest faithfulness within the same topic group and replace it in the output. The faithfulness of a fact is calculated using the $f_{\mathcal{KB}}$ function by replacing the supporting facts with a single fact.

\paragraph{Fact generation implementation details.} In this part, we generate questions for the named entities present in each fact of the top-ranked output produced by our basic approach, and retrieve the corresponding answers from the evidence paragraphs using UnifiedQA. We employ the question generation model described in \citet{deutsch2021towards}, which has been shown to be more extractive compared to other models as demonstrated in \citet{fabbri2021qafacteval}. We adopt the question filtering approach proposed in \citet{honovich2021q2} using an off-the-shelf extractive QA model (ktrapeznikov/albert-xlarge-v2-squad-v2 from Hugging Face \cite{wolf2020transformers}). We then use an off-the-shelf model (MarkS/bart-base-qa2d from Hugging Face) to convert the generated QA pairs into declarative sentences. We apply simple rules based on the entailment and contradiction scores of the selected facts from the fact selection variant and the generated declarative sentences to obtain the final generated facts.

\subsection{Comparison of Different Inference Methods with Supporting Facts}
\label{subsec:inference-comparison}

In our experiments, we utilize UnifiedQA for the final step of inference in both variants. However, it is worth noting that GPT-3 could also be used for this purpose. As shown in Table \ref{table:inference-comparison}, we observe that UnifiedQA performs better at inference with generated facts, while GPT-3 with CoT prompting performs better with empty or gold facts. This suggests that UnifiedQA is more robust to noisy inputs compared to GPT-3. Additionally, both UnifiedQA and GPT-3 with CoT prompting significantly outperform GPT-3 with zero-shot prompting, indicating that the CoT prompting is also beneficial for the final step of inference.



\begin{table}
\centering
\scalebox{0.85}{
\begin{tabular}{c|c|c}
 & Methods & Accuracy (\%) \bigstrut[b] \\ \hline
\multirow{3}{*}{Empty facts} &  GPT-3 (zero-shot) & 58.08 \bigstrut[t] \\ 
 & GPT-3 (CoT) & {\bf 65.94} \\
 & UnifiedQA & 58.95 \bigstrut[b] \\ \hline
 \multirow{3}{*}{Gold facts} & GPT-3 (zero-shot) & 81.66 \bigstrut[t] \\ 
 & GPT-3 (CoT) & {\bf 91.70} \\
 & UnifiedQA & 90.83 \bigstrut[b] \\ \hline
 \multirow{3}{*}{Generated facts} & GPT-3 (zero-shot) & 69.87 \bigstrut[t] \\ 
 & GPT-3 (CoT) & 76.42 \\
 & UnifiedQA & {\bf 78.60} \bigstrut[b]  \\ \Xhline{2\arrayrulewidth}
\end{tabular}}
\caption{Comparison of different inference methods on empty, gold, and generated facts.
}
\label{table:inference-comparison}
\end{table} 

\begin{table*}
\centering
\scalebox{1.0}{
\begin{tabular}{m{1.0\linewidth}}
\hline
Charles Sumner Tainter was Born on April 25, 1854   ( 1854-04-25 )   Watertown, Massachusetts, U.S..  Charles Sumner Tainter was Died on April 20, 1940   ( 1940-04-21 )  (aged 85)  San Diego, California, U.S..  The Nationality of Charles Sumner Tainter are American.  The Known for of Charles Sumner Tainter are Photophone, phonograph Father Of The Speaking Machine. \bigstrut[t] \\
{\bf Question:} Charles Sumner Tainter never left the state of Massachusetts. True or False?\\
{\bf Answer:} Charles Sumner Tainter was died in San Diego, California, U.S.. California is a state. Thus, Charles Sumner Tainter has left the state of Massachusetts. {\bf So the answer is false.} \\
The Region of Curitiba are South.  The Elevation of Curitiba are 934.6 m (3,066.3 ft).  The Density of Curitiba are 4,062/km 2  (10,523/sq mi).  The Metro density of Curitiba are 210.9/km 2  (546.2/sq mi). \bigstrut[t] \\
{\bf Question:} Curitiba is above sea level. True or False?\\
{\bf Answer:} The elevation of Curitiba are 934.6 m (3,066.3 ft). Elevation is a hypernym of level. Thus, Curitiba is above sea level. {\bf So the answer is true.}\\
Charles (Prince of Wales) was Born on 14 November 1948 ( 1948-11-14 )   (age 70)  Buckingham Palace, London, England.  The Spouse of Charles (Prince of Wales) are Lady Diana Spencer ( m.   1981 ;  div.   1996 )  , and Camilla Parker Bowles  ( m.   2005 ).  The Issue of Charles (Prince of Wales) are Prince William, Duke of Cambridge , and Prince Harry, Duke of Sussex. \bigstrut[t] \\
{\bf Question:} Charles was born in 1948 and has been married twice. True or False?\\
{\bf Answer:} Charles (Prince of Wales) was Born on 14 November 1948. The Spouse of Charles (Prince of Wales) are Lady Diana Spencer ( m.   1981 ;  div.   1996 )  , and Camilla Parker Bowles  ( m.   2005 ). Married is related to spouse. Thus, Charles was born in 1948 and has been married twice. {\bf So the answer is true.}\\
The Born of Idris Elba are 6 September 1972  (age 46)   Hackney, London, England.  The Residence of Idris Elba are London.  The Other names of Idris Elba are DJ Big Driis, Big Driis the Londoner, Big Driis, and 7 Dub.  The Occupation of Idris Elba are Actor, producer, director, musician, and DJ. \bigstrut[t] \\
{\bf Question:} Idris Elba is an English entertainer. True or False?\\
{\bf Answer:} The residence of Idris Elba is London. English is related to London. The occupation of Idris Elba are actor, producer, director, musician, and DJ. Actor is a hyponym of entertainer. Musician is a hyponym of entertainer. DJ is an entertainer. Thus, Idris Elba is an English entertainer. {\bf So the answer is true.} \\
The Breed of Jean, the Vitagraph Dog are Scotch Collie.  The Sex of Jean, the Vitagraph Dog are Female.  The Born of Jean, the Vitagraph Dog are 1902 Eastport, Maine.  The Years active of Jean, the Vitagraph Dog are 1909 - 1916. \bigstrut[t] \\
{\bf Question:} Jean, the Vitagraph Dog was a Golden Retriever which perform in circus. True or False?\\
{\bf Answer:} The Breed of Jean, the Vitagraph Dog are Scotch Collie. Collie is a hyponym of dog. Retriever is a hyponym of dog. Thus, Jean, the Vitagraph Dog was not a Golden Retriever which perform in circus. {\bf So the answer is false.}\\
The Studio of Hydrograd are Sphere Studios, North Hollywood, Los Angeles.  The Genre of Hydrograd are Hard rock.  The Label of Hydrograd are Roadrunner.  The Producer of Hydrograd are Jay Ruston. \bigstrut[t]\\
{\bf Question:} Hydrograd is in the rap genre. True or False?\\
{\bf Answer:} The Genre of Hydrograd are Hard rock. Rap is distinct from rock. Thus, Hydrograd is not in the rap genre. {\bf So the answer is false.} \bigstrut[b]\\
\hline
\end{tabular}}
\caption{The CoT prompt for tabular reasoning.
}
\label{table:tabular-prompt}
\end{table*} 

\end{document}


