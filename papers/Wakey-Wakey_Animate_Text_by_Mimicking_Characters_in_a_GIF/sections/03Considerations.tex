\section{Design Considerations}
\rev{
Motivated to lower the barrier in creating kinetic typography, we explore motion transfer techniques. Instead of tweaking keyframe configurations from scratch, users may specify the desired animation effect based on a reference GIF. With numerous online GIF instances, users may derive more diverse animation effects compared with using template-based tools.

One major design consideration is to {\it support both direct generation and fine-grain refinement} (\textbf{C1}). We expect our approach can generalize to various user requirements, including casual use as in online-messaging and professional editing like video-making. In addition to producing one-off results, the tool should allow refinement over fine-grain configurations of each frame. This is because motion transfer inherently introduces uncertainties in the generated result from motion transfer, which may violate user preference.

Additionally, we strive to {\it empower creators with interpretable algorithmic parameters} (\textbf{C2}). We hope the system supports iterative refinement, which necessitates providing explanations for the generation process so that users can provide feedback and make adjustments at every step of the generation process. By combining human experience and supervision, we seek to achieve higher quality and more consistent generations in line with users' expectations.
}
% To guide our design of an authoring tool, we make design considerations listed below.

% \begin{itemize}[leftmargin=1em]
    % \item \textbf {C1: Generate kinetic typography based on driving GIFs.} Witnessing the limitations of existing works, we try to let users specify the desired animation effect based on a reference GIF instead of tweaking keyframe configurations from scratch. With numerous online GIF instances, users may derive more diverse animation effects compared with using template-based tools.
    % \item \textbf{Support both on-the-fly generation and fine-grain refinement.} We expect our approach can generalize to various user requirements, including casual use and professional editing. Instead of producing one-off results or letting users control everything, we try not to set a boundary and allow dynamic workflows for different usage scenarios.

    % since kinetic typography has broad applications and permeates our daily lives.  
    % \item \textbf{C3: Expose algorithmic parameters in an intuitive way.} We hope the system to support iterative refinement, which necessitates providing explanations for the generation process so that users can provide feedback and make adjustments at every step of the generation process. By combining human experience and supervision, we strive to achieve higher quality and more consistent generations in line with users' expectations.
    % \item \textbf{C3: Support fine-grain refinement.}  We expect the system to support interactive refinement, allowing users to provide feedback and make adjustments at every step of the generation process. By combining human experience and supervision, we strive to achieve higher quality and more consistent generations in line with users' expectations.
% \end{itemize}



%   C1: Simplify the authoring procedure. To address the current challenge of time-consuming and experience-dependent animated text creation, our tool aims to provide a simple and user-friendly generation process that is clear and intuitive to use, even for non-technical users and instant applications.

% old c1 simplify c2 personalize c3 fine-grain

%    g1 - simply authoring procedure
%    g2 - useful for different user profiles

