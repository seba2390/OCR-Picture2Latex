\section{Workshop}
\label{sec:workshop}
We organized a workshop to evaluate the utility of our method.

\subsection{Demonstrations}
To elicit in-depth discussions in the workshop, we designed and implemented several demonstrations of potential application scenarios, including a video opening, an online messaging widget, and an emotional word cloud.

\paragraph{Video opening}

Vlogs (Video blogs) are becoming a prevalent form to share personal experiences creatively.
To make a vlog stand out, an engaging opening animation can help grab viewers' attention and set the tone for the rest of the video.
Leveraging results generated by our method, we created a vlog opening as an example of its practical application in daily content creation.
As shown in \autoref{fig: video} A, D1 is a vlog opening with the Puppy Maltese theme, showcasing different states of the character and introducing ``My Day'', which suggests the video topic. 
We envision that integrating animated text with consistent motions enhances both explanatory and entertaining values. 
Similarly, users can use our method to create and personalize animated text in their video creations across different themes and contents.

\paragraph{Browser Widget for Online Chatting}
Informed by the previous efforts in enhancing emotion communication in online messaging~\cite{Wang2004communicate, malik2009communicating, aoki2022emoballon}, we implemented a light-weighted Chrome extension to facilitate the real-time creation of kinetic typography (see \autoref{fig: video} B).
It has a simplified interface compared with Wakey-wakey, which features real-time generation and removes the human interaction module.
Users may upload an anchor GIF, type down the text, configure the color and font, and then directly obtain the generated kinetic typography in the GIF format.


\paragraph{Emotional Animated Word Cloud}
The word cloud is a common visualization technique to summarize text data, where the text size represents the word frequency.
\rev{Xie \ea~\cite{xie2023emordle} coined the word ``emordle'' representing animated word clouds that suggest underlying emotions.
Based on our approach, we generated an ``emordle'' by transferring the animated scheme of a bumpy cartoon pig from the MGif dataset.
Instead of using the parsed control points, we transformed the text anchors for each text element that constitute the word cloud. 
In other words, we replace the vector control points with the anchor points of each word in the proposed approach. Note that one word has one anchor point at its central position.}
The generated word cloud example is displayed in \autoref{fig: wccase}.

\begin{figure}[t]
  \centering
  \includegraphics[width=\linewidth]{figures/emordle.pdf}
  \caption{Application of animated word cloud that delivers a certain emotion with the animation.}
  \label{fig: wccase}
  \Description{In Figure 12, there are two rows of keyframes showing the anchor GIF and the corresponding animated word cloud.}
    % \vspace{-0.1in}
\end{figure}



\subsection{Protocol}
The workshop proceeded in the following four stages.

\paragraph{Briefing}
The briefing session took 10 minutes, during which we introduced the background of our work and briefly explained our method. We then used a demo video to demonstrate the functions of the tool and illustrate the generation process.

\paragraph{Demonstration} 
We spent another 10 minutes displaying the video opening, the online messaging widget, and the animated word cloud to demonstrate potential application scenarios. We also introduced the installation and use of the plugin through a demo.

\paragraph{Self-creation}
We then allow users to freely use and explore the system and widget to create their kinetic typography for 20 minutes. 

\paragraph{Post-interview}
After trying \tool, participants are asked to complete a questionnaire with six questions on a 7-point Likert scale about its usability, including covering \textit{Practicality}, \textit{Customization}, \textit{Pleasure}, \textit{Efficiency}, and the \textit{Intuitiveness} for widget and interface respectively.
We also raised open-ended questions following a structured template in order to understand their perceptions of the authoring process as well as the generated kinetic typography.




\subsection{Participants}
Both designers and general users are invited to obtain feedback from different perspectives in the evaluation.
We recruited 20 participants through our personal network and advertisements on social media, with 7 females and 13 males.
There are 3 professional designers: P1 works on user experience design; P2 engages in self-media creation as a blogger and vlogger; P3 is a digital painter (P3).
The rest are graduate students majoring in data science at a local university (denoted as P4–P20 in ascending time order of their interviews).
Among all participants, 13 have seen kinetic typography before, such as in online memes, short videos, slides, and advertisements. 
Only two participants have experience in creating kinetic typography with other software (P2, P3). 

\subsection{Results}
Here we present both the quantitative and qualitative results.

\subsubsection{Observations.}
Nonetheless, when it comes to the time required for creation, there is substantial dissent, as our concurrency fell short, leading to extended completion times for some users.
In spite of this setback, the system garners favorable acknowledgment for its expressiveness, user-friendliness, intuitiveness, and real-world applicability.
To elevate the overall user experience, refinements in creation duration and concurrency should be considered.
\rev{As for the usage of $\alpha$ or $e$, most users were generally satisfied with the empirical default value. However, two users with a design background occasionally fine-tuned this parameter to derive better results.
In terms of the manual adjustments on control points, users without a design background barely adjusted the control points. Three users with design backgrounds occasionally make manual adjustments, about 8/14 frames, with an average duration of 2.1/5.8 min for one kinetic typography. Users tended to adjust key points in GIFs when there were noticeable detection errors. Otherwise, they normally increased $\alpha$ to mitigate unexpected deformations, though less motion preserving.}

\subsubsection{Usability Ratings.}

\autoref{fig: quant_res} shows the distribution of users' subjective ratings of the six usability questions.
More than half of users concurred that the tool was intuitive, tailored, and delivers an enjoyable experience during usage.
 To be more specific, 45\% users strongly agreed that our work is pleasant and interesting.
 They valued the innovative animated text design, the straightforward comprehension of emotions portrayed, and the uncomplicated tool configurations for selecting animation approaches.
 They also observed that blending textual meanings with animations facilitated a beneficial expression of the content.



\begin{figure}[t]
  \centering
  \includegraphics[width=\linewidth]{figures/user-study.pdf}
  \caption{Quantitative Results of the usefulness of our method. We measured the \textit{practicality}, \textit{customization}, \textit{pleasure}, \textit{efficiency} and the \textit{intuitiveness} for the browser widget and authoring interface respectively.}
  \label{fig: quant_res}
  \Description{Figure 13 shows the overall chart is a stacked bar chart with seven categories ranging from "strongly disagree" to "strongly agree". Vertically, it represents "practicality", "customization", "pleasure", "efficiency", "widget", and "interface". The x-axis is marked with percentages and the specific statistics are indicated on the bars. Our approach has been recognized for its usefulness.}
    % \vspace{-0.1in}
\end{figure}

\subsubsection{User Feedback.}
We summarize the following insights and implications for future improvements from the users' feedback.

$\diamond$ \underline{On mixed-initiative authoring}. Participants expressed their agreement with the balance we stroke between user involvement and automatic generation. P3 said: ``{\it supporting quick automatic generation while also offering optional customization and improvement from users}''. Participants were optimistic about the mixed-initiative way, as ``{\it intelligence improves efficiency while users enhance quality and creativity, as the imagination of users cannot be dismissed}'' (P10).

$\diamond$ \underline{On personalization.}
19 participants (except P11) valued customization highly and believed it helps incorporate their own ideas and shape a unique personal style, while three designers showed an ``{\it innate aversion to preconceived solutions}'' (P1). Among them, 17 believed that customization is also crucial in the context of animated text, which helps to ``{\it express opinions and feelings more effectively and precisely}'' (P10), as well as ``{\it making the creations more specific and memorable}'' (P18).

% 新颖性,趣味性、启发性
$\diamond$ \underline{On motion transfer.}
Users expressed their agreement with the novelty, interest, and inspiration of our approach.
Seven participants mentioned ``novel'', where P9 commented ``{\it it is pleasantly surprising}''.
Four participants mentioned ``interesting'', and 3 participants found it inspiring.
P2 found {``\it the generated results inspiring and insightful for my design progress''}.
P1 said: ``{\it I appreciate the smart use of motion transfer, as the interpretation of motion is subjective, but you use memes as an intermediate medium whose ability to convey emotions is validated through wide applications, making the generated results meet subjective expectations. In addition, using text as a vector container for transformation, where the container can be liquefied, allows for slight deformation, thus in support of more subtle emotional expression.}''

$\diamond$ \underline{On application scenarios}.
Participants brainstormed multiple application scenarios in their personal life with kinetic typography, including online chatting, social media post, website banners, presentations, subtitle enhancement, and E-invitation. 
\rev{P13 also imagined that in order to attract interest, animated text may be used as a teaching instrument for introducing words to little children.}
P1 identified some challenges in the application of the animated text. ``{\it When used alone, the interpretation by users can be ambiguous. There are challenges in accessibility, readability, as well as efficiency of perception and recognition}''.
He suggested that we can ``{\it emphasize the combination with animated images, which can enhance contextual effects and create a synergistic interaction greater than the sum of the separate parts}''.

$\diamond$ \underline{On future improvements}. 
First, Participants suggested a possible enhancement in the guidance of interactions, preferably by introducing recommendations for interactive operations. 
The operations may be clearer ``{\it with the help of some icons and text}'' (P16), and ``{\it the generation efficiency can be improved by recommending interactive behaviors. In addition, it would be very helpful to support automatic modifications of other frames after modifying one frame in key point correction and glyph refinement}'' (P2).
Besides, integrating external resources may enrich the generated results.
P11 suggested ``{\it incorporating language models to generate using instructions enhances its convenience}''.
P3 commented that ``{\it integrating meme and artistic font libraries helps generate richer and more artistic results}''.



