\section{Introduction}
Nowadays, kinetic typography, \ie, animated text or motion text, has become common in daily life.
These vibrant artifacts can be observed in movies, website widgets, and online memes, \rev{such as the lyric video \textit{Skyfall}~\cite{skyfall} and the main title sequence of the movie \textit{Spider-Man}.}
% A classic example is \rev{\textit{Sir Jr.}, the lively lamp replacing the letter ``i'' in most Pixar movie openings~\cite{luxos}.
Kinetic typography is effective for expressing emotional content, creating characters, and capturing or directing attention~\cite{shannon1998kinetic, lee2002engine}.
And there have been fruitful investigations of its application scenarios, including animated visualization~\cite{xie2023emordle}, instant messaging~\cite{gaylord2015body,kim2016yo}, ambient displays~\cite{minakuchi2008kinetic}, and captioning~\cite{Lee07EmotiveCaptioning}.

However, it remains non-trivial to craft the animation for text elements.
Leveraging commercial animation software~\cite{motion, aftereffects} or programming toolkits~\cite{lee2002engine} one may tweak the configuration of text in each animation keyframe,\eg, color, positions of the anchor point, and the transition between keyframes like a slow--in easing function.
Orchestrating these low-level parameters for a meaningful animation such as a melting scene requires careful considerations like which part of the text element to move, where to move, at what speed, \etc~  
As such, this process remains challenging and time-consuming with the large design space.
Previous studies~\cite{yeo2008kim, Lee07EmotiveCaptioning} tried to alleviate the authoring burden by designing a suite of templates.
While categorizing animated effects allows a one-click or even automatic generation, this approach suffers from limited customizability.
For instance, when one hopes to impress viewers with a refreshing presentation title, a pre-defined jumping effect may be inferior to a customized motion of breakdance.


% making visual representation meaningful to people~\cite{romat2020dear}

% We value the uniqueness and personalization and hope to  can be critical for user experience~\cite{viegas09participatory, laura2013anytype}.
Valuing uniqueness and personalization in digital communication~\cite{viegas09participatory, laura2013anytype}, we are motivated to find a sweet spot between automaticity and agency in kinetic typography tools.
Inspired by the recent advances in artificial intelligence, where a head image can talk by mimicking the motion of a driving video, \eg,~\cite{zhou2020makeittalk, hong2022depth}, we explore transferring existing animation designs to text.
The flourishing of GIFs on the web offers myriad high-quality animation references that imply emotions and humor, which can enrich the expressiveness of kinetic typography and make it easy for creators to specify desired effects.
However, existing approaches are not directly applicable to our goal.
On the one hand, research in motion transfer hardly attends to the non-photorealistic domains~\cite{siarohin2019first,xu2022motion}, especially for kinetic typography.
On the other hand, relevant research in text stylization focus on static text (\eg,~\cite{iluz2023word,xu2007calligraphic}), where the animation remains largely under-explored.
% Such insufficiencies in existing approaches call for a more efficient way to generate bespoke kinetic typography.

% With the recent advances of AI-generated content~\cite{anantrasirichai2022artificial, cetinic2022understanding}, we explore the opportunity to generate kinetic typography automatically, which is in line with the call for exploiting machine creativity in design~\cite{di2023doom}.
% To enhance the output diversity, we are motivated to use a motion transfer technique to create animated text with semantic movements aligning to an anchoring GIF, which has been ubiquitous with meaningful movements.
% Existing studies on motion transfer demonstrated effectiveness with compelling applications in talking heads and body movements~\cite{hong2022depth, xu2022motion}, \etc, yet little scholarly attention has been paid to the non-photorealistic domains~\cite{siarohin2019first, zhou2020makeittalk}, especially for kinetic typography.
% On the other hand, relevant research in graphics focused on static semantic text (\eg,~\cite{iluz2023word,xu2007calligraphic}) where the animation remains largely under-explored.

% participatory~\cite{viegas09participatory}; provoke exploration~\cite{laura2013anytype}

We propose a mixed-initiative framework for creating kinetic typography based on a driving GIF with a moving character.
On the machine side, the motion of the driving GIF is represented as the trajectories of several key points, which are extracted and guide the positional changes in the control points of the target text.
 On the human side, people can steer the mapping process by directly manipulating these points to refine the automatically computed positions of each point, resulting in a more desirable output.
 Based on the proposed framework, we develop an interactive interface for creating kinetic typography.
 We perform a series of evaluation studies to evaluate the usefulness and effectiveness of our approach.
 First, we demonstrate how individual components of the proposed framework contribute to the final result and test several cases.
 Second, a questionnaire study shows evidence that the output is both aesthetically pleasing and similar to the driving GIF.
 Third, we organize a workshop with general users and expert designers to evaluate the utility of our approach.




In summary, our work contributes to the following three aspects.
\begin{itemize}[leftmargin=2.2em]
    \item (Technique) An automatic approach to transfer the animation scheme from an anchor GIF to vector text.
    \item (Application) A prototype authoring tool for generating bespoke kinetic typography, which supports various scenarios, \eg, design prototyping and instant messaging.
    \item (Evaluation) A questionnaire study validating our transfer approach and a workshop demonstrating the usefulness of the authoring tool.
\end{itemize}