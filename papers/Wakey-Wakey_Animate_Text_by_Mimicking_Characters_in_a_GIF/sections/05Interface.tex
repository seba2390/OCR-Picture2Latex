

\section{Authoring Interface}
\label{sec:authoring_tool}
% Leveraging the self-explainability of our approach, 
Based on the proposed framework, we implement a mixed-initiative authoring tool called Wakey-Wakey\footnote{\rev{The name suggests that the authoring tool awakes static text and makes it lively.}} that allows fine-grain adjustment for more natural and aesthetic results (see \autoref{fig: interface}).
This section offers a step-by-step guide showing how to interactively generate animated text with our tool referring to the interface.



%in line with the practice of previous works~\cite{gatys2016image, mao2022intelligent}.


The user-oriented process mainly consists of three steps: text and GIF input, key point correction, and glyph refinement, each supported by a view: \textit{Input View}, \textit{Correction View}, and \textit{Refinement View}.
While the input step is mandatory, the correction and refinement stages are optional. This allows for simple end-to-end personalized generation, as well as interactive improvement.

\paragraph{Input View.}
Users first input the text and customize its static appearance with the global typeface and color (\autoref{fig: interface} A).
They can upload and preview the driving GIF through a button (\autoref{fig: interface} B) .
The section will record and list the recent upload history. 
After clicking ``Next'', the system will process the input with our method, and both intermediate and final results will be displayed in the other two views. Users can easily obtain the generated animated text here without any additional effort.

\paragraph{Correction View.}
Users can then drag the displayed the key points from the motion trajectory extraction module to a suitable location at a specific frame, as shown in \autoref{fig: interface} D.
Special attention can be paid to the key point trajectory with a corresponding colored button above. 
Two thumbnails (\autoref{fig: interface} C) enable switching between the anchor GIF and the extracted key points for correction.
As the corresponding key points share the same color, users can learn how the mapping is.
% The correction on the GIF can also improve the generation of FOMM. 
Users are supported to ensure better quality by maintaining reasonable motion trajectory to drive the generation of the vector animated text.

\paragraph{Refinement View.}
Users may configure the parameters $\alpha$ and $e$, and select whether the kinetic typography is generated by aligning to key points from the anchor GIF or the extracted key points (\autoref{fig: interface} E).
The bottom panel (\autoref{fig: interface} G) enables the users to adjust the glyph by dragging the control points.
And the final GIF of kinetic typography is displayed in the middle panel (\autoref{fig: interface} F).

