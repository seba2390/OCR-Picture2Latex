%%
%% This is file `sample-sigconf.tex',
%% generated with the docstrip utility.
%%
%% The original source files were:
%%
%% samples.dtx  (with options: `sigconf')
%% 
%% IMPORTANT NOTICE:
%% 
%% For the copyright see the source file.
%% 
%% Any modified versions of this file must be renamed
%% with new filenames distinct from sample-sigconf.tex.
%% 
%% For distribution of the original source see the terms
%% for copying and modification in the file samples.dtx.
%% 
%% This generated file may be distributed as long as the
%% original source files, as listed above, are part of the
%% same distribution. (The sources need not necessarily be
%% in the same archive or directory.)
%%
%% Commands for TeXCount
%TC:macro \cite [option:text,text]
%TC:macro \citep [option:text,text]
%TC:macro \citet [option:text,text]
%TC:envir table 0 1
%TC:envir table* 0 1
%TC:envir tabular [ignore] word
%TC:envir displaymath 0 word
%TC:envir math 0 word
%TC:envir comment 0 0
%%
%%
%% The first command in your LaTeX source must be the \documentclass command.
\documentclass[sigconf, screen, authorversion]{acmart}
%% NOTE that a single column version may be required for 
%% submission and peer review. This can be done by changing
%% the \doucmentclass[...]{acmart} in this template to 
%% \documentclass[manuscript,screen]{acmart}
%% 
%% To ensure 100% compatibility, please check the white list of
%% approved LaTeX packages to be used with the Master Article Template at
%% https://www.acm.org/publications/taps/whitelist-of-latex-packages 
%% before creating your document. The white list page provides 
%% information on how to submit additional LaTeX packages for 
%% review and adoption.
%% Fonts used in the template cannot be substituted; margin 
%% adjustments are not allowed.
%%
%%
%% \BibTeX command to typeset BibTeX logo in the docs
\AtBeginDocument{%
  \providecommand\BibTeX{{%
    \normalfont B\kern-0.5em{\scshape i\kern-0.25em b}\kern-0.8em\TeX}}}

%% Rights management information.  This information is sent to you
%% when you complete the rights form.  These commands have SAMPLE
%% values in them; it is your responsibility as an author to replace
%% the commands and values with those provided to you when you
%% complete the rights form.
\copyrightyear{2023}
\acmYear{2023}
\setcopyright{acmlicensed}\acmConference[UIST '23]{The 36th Annual ACM Symposium on User Interface Software and Technology}{October 29-November 1, 2023}{San Francisco, CA, USA}
\acmBooktitle{The 36th Annual ACM Symposium on User Interface Software and Technology (UIST '23), October 29-November 1, 2023, San Francisco, CA, USA}
\acmPrice{15.00}
\acmDOI{10.1145/3586183.3606813}
\acmISBN{979-8-4007-0132-0/23/10}
%% These commands are for a PROCEEDINGS abstract or paper.
\acmConference[UIST '2023]{The ACM Symposium on User Interface Software and Technology}{Oct 29--Nov 1,
  2023}{California Bay Area, USA}
%
%  Uncomment \acmBooktitle if th title of the proceedings is different
%  from ``Proceedings of ...''!
%
%\acmBooktitle{Woodstock '18: ACM Symposium on Neural Gaze Detection,
%  June 03--05, 2018, Woodstock, NY} 

\newcommand{\zzy}[1]{\textcolor{ACMBlue}{#1}}


%%
%% Submission ID.
%% Use this when submitting an article to a sponsored event. You'll
%% receive a unique submission ID from the organizers
%% of the event, and this ID should be used as the parameter to this command.
\acmSubmissionID{2679}

%%
%% For managing citations, it is recommended to use bibliography
%% files in BibTeX format.
%%
%% You can then either use BibTeX with the ACM-Reference-Format style,
%% or BibLaTeX with the acmnumeric or acmauthoryear sytles, that include
%% support for advanced citation of software artefact from the
%% biblatex-software package, also separately available on CTAN.
%%
%% Look at the sample-*-biblatex.tex files for templates showcasing
%% the biblatex styles.
%%

%%
%% The majority of ACM publications use numbered citations and
%% references.  The command \citestyle{authoryear} switches to the
%% "author year" style.
%%
%% If you are preparing content for an event
%% sponsored by ACM SIGGRAPH, you must use the "author year" style of
%% citations and references.
%% Uncommenting
%% the next command will enable that style.
%%\citestyle{acmauthoryear}

% ====================================================================
% Packages
% ====================================================================
\usepackage[utf8]{inputenc} % allow utf-8 input
\usepackage[T1]{fontenc}    % use 8-bit T1 fonts
\usepackage{hyperref}       % hyperlinks
\usepackage{url}    % simple URL typesetting
\usepackage{booktabs}       % professional-quality tables
\usepackage{amsfonts}       % blackboard math symbols
\usepackage{nicefrac}       % compact symbols for 1/2, etc.
\usepackage{microtype}      % microtypography
\usepackage{lineno}
\usepackage{natbib}
\usepackage{pifont}
\usepackage{fleqn}
\usepackage{amsmath,amssymb}
\usepackage{graphicx}
\usepackage{multicol,graphicx,xcolor}
\usepackage{tikz}
\usepackage{verbatim}
\usepackage{bm}
\usetikzlibrary{shapes.misc, positioning}
\usetikzlibrary{matrix, calc, positioning, arrows,shapes,backgrounds, arrows.meta, quotes}
\usepackage{gensymb,stmaryrd,mathtools,textcomp,xspace,grffile}
\usetikzlibrary{shapes.geometric,fit,matrix,positioning,shapes.multipart}
\usetikzlibrary{decorations.markings}
\usetikzlibrary{shadows,decorations.pathreplacing,fadings}
\pgfdeclarelayer{background}
\pgfsetlayers{background,main}
\usetikzlibrary{topaths, calc,3d}
\usetikzlibrary{fit}
\usepackage{color}

\usepackage{bm}		% Bold maths symbols, including upright Greek
\usepackage{pdflscape}	% Landscape pages
\usepackage{ae,aecompl}
%\usepackage{newtxtext}
%\usepackage{newtxmath}
\usepackage[section]{placeins}
\usepackage{float}
\usepackage[normalem]{ulem}
\usepackage[toc,page]{appendix}
\usepackage{longtable}

\usepackage[normalem]{ulem}

\usepackage{tabularx}
    \newcolumntype{L}{>{\raggedright\arraybackslash}X}

\usepackage{caption}
\usepackage{subcaption}

%%
%% end of the preamble, start of the body of the document source.
\begin{document}

\documentclass{article}

\usepackage{authblk}
\usepackage{orcidlink}

\title{A Relationship Between Spin and Geometry}
\author{Peter T. J. Bradshaw\,\orcidlink{0000-0001-9938-8460}\thanks{ucapptj@ucl.ac.uk, orcid.org/0000-0001-9938-8460}}
\affil{\textit{Department of Physics and Astronomy, University College London, London, WC1E 6BT}}
\date{}

\begin{document}
\maketitle
\end{document}
\correspondingauthor{Samuel N. Quinn}
\email{squinn@cfa.harvard.edu}

\author[0000-0002-8964-8377]{Samuel N. Quinn}
\affiliation{\cfa}

\author[0000-0003-3182-5569]{Saul Rappaport}
\affiliation{\MIT}

\author[0000-0001-7246-5438]{Andrew Vanderburg}
\affiliation{\wisconsin}

\author[0000-0003-3773-5142]{Jason D. Eastman}
\affiliation{\cfa}

\author[0000-0002-6916-8130]{Lorne A. Nelson}
\affiliation{\bishops}

\author[0000-0003-3988-3245]{Thomas L. Jacobs}
\amateur
\affiliation{12812 SE 69th Place Bellevue, WA 98006, USA}

\author[0000-0002-8527-2114]{Daryll M. LaCourse}
\amateur
\affiliation{7507 52nd Place NE Marysville, WA 98270, USA}

\author{Allan R. Schmitt}
\amateur
\affiliation{616 W. 53rd. St., Apt. 101, Minneapolis, MN 55419, USA}

\author{Perry Berlind} 
\affiliation{\cfa}

\author[0000-0002-2830-5661]{Michael L. Calkins} 
\affiliation{\cfa}

\author[0000-0002-9789-5474]{Gilbert A. Esquerdo} 
\affiliation{\cfa}

\author[0000-0001-8638-0320]{Andrew W. Howard}
\affiliation{\caltech}

\author[0000-0002-0531-1073]{Howard Isaacson}
\affiliation{\berkeley}

\author[0000-0001-9911-7388]{David W. Latham}
\affiliation{\cfa}

% \author{others?}
% \noaffiliation{}


%%
%% The abstract is a short summary of the work to be presented in the
%% article.
\begin{abstract}
  With appealing visual effects, kinetic typography (animated text) has prevailed in movies, advertisements, and social media.
However, it remains challenging and time-consuming to craft its animation scheme.
We propose an automatic framework to transfer the animation scheme of a rigid body on a given meme GIF to text in vector format.
First, the trajectories of key points on the GIF anchor are extracted and mapped to the text's control points based on local affine transformation.
Then the temporal positions of the control points are optimized to maintain the text topology.
We also develop an authoring tool that allows intuitive human control in the generation process.
A questionnaire study provides evidence that the output results are aesthetically pleasing and well preserve the animation patterns in the original GIF, \rev{where participants were impressed by a similar emotional semantics of the original GIF}.
In addition, we evaluate the utility and effectiveness of our approach through a workshop with general users and designers.

% 另一个思路是从丰富多媒体体验出发,参考brook leaves home

% A survey study provides evidence that the
% semantics behind the animation is preserved after transferring to
% text.
\end{abstract}

%\keywords{Shared Autonomy, Shared Control Teleoperation, Human-Robot Teaming, Telerobotics, Human-Robot Collaboration, Human Performance Augmentation}
\keywords{Physical Human-Robot Interaction, Telerobotics, Rehabilitation Robotics, Personal Robots, Human Performance Augmentation}


%% A "teaser" image appears between the author and affiliation
%% information and the body of the document, and typically spans the
%% page.
\begin{teaserfigure}
  \includegraphics[width=\textwidth]{figures/teaser-compressed.pdf}
  \caption{We introduce an approach to revive static text by transferring the motion of characters in a driving GIF.}
  \Description{Figure 1 is a row-wise illustration of animations. The first row is the driving GIF with a sketched puppy character arranged horizontally by frame. The second and third rows are also arranged by frame and depict corresponding movements in the texts “UIST” and “2023”, respectively.}
  \label{fig:teaser}
\end{teaserfigure}

% \received{20 February 2007}
% \received[revised]{12 March 2009}
% \received[accepted]{5 June 2009}

%%
%% This command processes the author and affiliation and title
%% information and builds the first part of the formatted document.
\maketitle

% UIST papers are of variable length. Paper length must be based on the weight of the contribution. A new idea presented in a compact format is more likely to be accepted than the same idea in a long format, and shorter, more focused papers are encouraged. As a guideline, please consider papers in the range of ~7,500-10,000 words.

\section{Introduction}
Nowadays, kinetic typography, \ie, animated text or motion text, has become common in daily life.
These vibrant artifacts can be observed in movies, website widgets, and online memes, \rev{such as the lyric video \textit{Skyfall}~\cite{skyfall} and the main title sequence of the movie \textit{Spider-Man}.}
% A classic example is \rev{\textit{Sir Jr.}, the lively lamp replacing the letter ``i'' in most Pixar movie openings~\cite{luxos}.
Kinetic typography is effective for expressing emotional content, creating characters, and capturing or directing attention~\cite{shannon1998kinetic, lee2002engine}.
And there have been fruitful investigations of its application scenarios, including animated visualization~\cite{xie2023emordle}, instant messaging~\cite{gaylord2015body,kim2016yo}, ambient displays~\cite{minakuchi2008kinetic}, and captioning~\cite{Lee07EmotiveCaptioning}.

However, it remains non-trivial to craft the animation for text elements.
Leveraging commercial animation software~\cite{motion, aftereffects} or programming toolkits~\cite{lee2002engine} one may tweak the configuration of text in each animation keyframe,\eg, color, positions of the anchor point, and the transition between keyframes like a slow--in easing function.
Orchestrating these low-level parameters for a meaningful animation such as a melting scene requires careful considerations like which part of the text element to move, where to move, at what speed, \etc~  
As such, this process remains challenging and time-consuming with the large design space.
Previous studies~\cite{yeo2008kim, Lee07EmotiveCaptioning} tried to alleviate the authoring burden by designing a suite of templates.
While categorizing animated effects allows a one-click or even automatic generation, this approach suffers from limited customizability.
For instance, when one hopes to impress viewers with a refreshing presentation title, a pre-defined jumping effect may be inferior to a customized motion of breakdance.


% making visual representation meaningful to people~\cite{romat2020dear}

% We value the uniqueness and personalization and hope to  can be critical for user experience~\cite{viegas09participatory, laura2013anytype}.
Valuing uniqueness and personalization in digital communication~\cite{viegas09participatory, laura2013anytype}, we are motivated to find a sweet spot between automaticity and agency in kinetic typography tools.
Inspired by the recent advances in artificial intelligence, where a head image can talk by mimicking the motion of a driving video, \eg,~\cite{zhou2020makeittalk, hong2022depth}, we explore transferring existing animation designs to text.
The flourishing of GIFs on the web offers myriad high-quality animation references that imply emotions and humor, which can enrich the expressiveness of kinetic typography and make it easy for creators to specify desired effects.
However, existing approaches are not directly applicable to our goal.
On the one hand, research in motion transfer hardly attends to the non-photorealistic domains~\cite{siarohin2019first,xu2022motion}, especially for kinetic typography.
On the other hand, relevant research in text stylization focus on static text (\eg,~\cite{iluz2023word,xu2007calligraphic}), where the animation remains largely under-explored.
% Such insufficiencies in existing approaches call for a more efficient way to generate bespoke kinetic typography.

% With the recent advances of AI-generated content~\cite{anantrasirichai2022artificial, cetinic2022understanding}, we explore the opportunity to generate kinetic typography automatically, which is in line with the call for exploiting machine creativity in design~\cite{di2023doom}.
% To enhance the output diversity, we are motivated to use a motion transfer technique to create animated text with semantic movements aligning to an anchoring GIF, which has been ubiquitous with meaningful movements.
% Existing studies on motion transfer demonstrated effectiveness with compelling applications in talking heads and body movements~\cite{hong2022depth, xu2022motion}, \etc, yet little scholarly attention has been paid to the non-photorealistic domains~\cite{siarohin2019first, zhou2020makeittalk}, especially for kinetic typography.
% On the other hand, relevant research in graphics focused on static semantic text (\eg,~\cite{iluz2023word,xu2007calligraphic}) where the animation remains largely under-explored.

% participatory~\cite{viegas09participatory}; provoke exploration~\cite{laura2013anytype}

We propose a mixed-initiative framework for creating kinetic typography based on a driving GIF with a moving character.
On the machine side, the motion of the driving GIF is represented as the trajectories of several key points, which are extracted and guide the positional changes in the control points of the target text.
 On the human side, people can steer the mapping process by directly manipulating these points to refine the automatically computed positions of each point, resulting in a more desirable output.
 Based on the proposed framework, we develop an interactive interface for creating kinetic typography.
 We perform a series of evaluation studies to evaluate the usefulness and effectiveness of our approach.
 First, we demonstrate how individual components of the proposed framework contribute to the final result and test several cases.
 Second, a questionnaire study shows evidence that the output is both aesthetically pleasing and similar to the driving GIF.
 Third, we organize a workshop with general users and expert designers to evaluate the utility of our approach.




In summary, our work contributes to the following three aspects.
\begin{itemize}[leftmargin=2.2em]
    \item (Technique) An automatic approach to transfer the animation scheme from an anchor GIF to vector text.
    \item (Application) A prototype authoring tool for generating bespoke kinetic typography, which supports various scenarios, \eg, design prototyping and instant messaging.
    \item (Evaluation) A questionnaire study validating our transfer approach and a workshop demonstrating the usefulness of the authoring tool.
\end{itemize}
\section{Background \& Related Work}
In this section, we provide background information on typography and review existing research on kinetic typography, text stylization, and guided animation generation with an anchor.

\subsection{Preliminaries on Digital Typography}
Typography is defined as the art and technique of organizing text in a way that is easy to read, comprehend, and visually pleasing while presented.
In general, the visual appearance of a digital letter is determined by its \textit{font}, which is a particular size, weight, and style of a \textit{typeface}.
The typeface is a set of designed characters or letters, named glyphs, such as \rev{\texttt{Courier New}, {\fontfamily{ptm}\selectfont Times New Roman}, and {\fontfamily{pbk}\selectfont Bookman Old}}.
Internally, a typeface is represented in the raster domain or vector domain.
As bitmap fonts may become distorted or blurred with mosaic-like jagged edges at high resolution, we chose to adopt a vector-based typeface--the TrueType~\cite{penney1996truetype} font, which describes glyphs with quadratic bezier curves.


\subsection{Kinetic Typography}
Kinetic typography enriches animated user interface~\cite{chang1993animation} and digital media, which has received scholarly interest since the 1990s~\cite{shannon1998kinetic}. Most recently, Xie~\ea~\cite{xie2023emordle} summarized a design space of kinetic typography concerning changes in style, shape, position, and scale.
Compared with static text, kinetic typography is more competent in guiding attention~\cite{borzyskowski2004animated, minakuchi2008kinetic} and communicating emotions or semantics with the paralinguistic clues underlying animation~\cite{malik2009communicating, Lee07EmotiveCaptioning}.
Accordingly, there has been a series of works seeking to lower the burden of creating kinetic typography.
Kinetic Typography Engine~\cite{lee2002engine} set the basis of modern animation software (\eg, Adobe After Effects~\cite{aftereffects} and TypeMonkey~\cite{typemonkey}) with frame-based low-level specifications and a library of common effects.
The specification concerns text properties like position, rotation, \etc~
And the library was composed of functional time filters like oscillation.
TextAlive~\cite{kato2015textalive} featured kinetic typography synchronized with audio signals in video editing.

A stream of work investigated tools for average users rather than professional designers, where reducing efforts in animation configurations is a primary goal.
These works normally predefined a suite of animated effects and support selection or automatic matching under various contexts.
Instant messaging has been most studied,~\eg,~\cite{gaylord2015atim,yeo2008kim,forlizzi2003kinedit, minakuchi2005kinetic}.
For instance, Kinedit~\cite{forlizzi2003kinedit} allowed users to integrate text animation into a line of words.
Minakuchi and Tanaka~\cite{minakuchi2005kinetic} conceptualized an automatic composer that analyzes the semantics of text and queries suitable motions from a static repository to amplify its meanings.
Other scenarios include emotional animation for lyric videos~\cite{vy2008enact} and dynamic display based on viewers' emotions~\cite{lim2022study}.
These works suffered from \rev{the number of animated effects provided}.
For instance, there is hardly any consideration of transforming the text shape, which is common in animation~\cite{thomas1995illusion} yet requires by-frame editing.
In comparison, our work takes advantage of the ubiquitous online memes or stickers and can scrape their animation schemes to a random text with reliable transformation on its outlines.


\subsection{Text Stylization}
Our work closely relates to the task of text style transfer and semantic typography in text stylization, an area widely studied in computer vision/graphics to make a given text visually appealing.

Similar to our workflow, text style transfer concerns transferring the style of a given source (font samples, natural image/video) into text.
Some works explored propagating the design of a few stylized letters to others, such as typeface geometry~\cite{phan2015flexyfont} and glyph decorations~\cite{wang2019typography}.
Other works followed the general workflow of neural style transfer~\cite{gatys2016image} and viewed style as local neural patterns of the input image/video,~\eg, \cite{mao2022intelligent, yang2021shape, men2019dyntypo}.
In contrast, our work deforms the vectorized outline of the text to match the reference GIF.
% 还需要补充说明我们的变形和他们的异同,特别是vector base的,生成式的 
% 有没有使用通用算法或者创新
% ~\cite{berio2022strokestyles}
% Existing motion transfer models are mostly designed specifically for a certain field, requiring prior knowledge of the object. 
We propose to vivify text by animating it in the way of a cartoon character, which diverges from their focus on learning image patch-based features.
Additionally, compared with kinetic typography, text style transfer emphasizes the artistic effect rather than an affective impact and typically produces static output.

Semantic typography amplifies the semantic meanings through visual cues in typography, which is also our goal.
Xu and Kaplan~\cite{xu2007calligraphic} proposed calligraphic packing, which deforms letters in a word to fit a given shape, which was improved by Zou~\ea~\cite{zou2016legible}.
In contrast to the intense deformation in letters, Word-As-Image~\cite{iluz2023word} stroke the balance of transformation on both sides, preserving the original font's style and legibility while ensuring the semantic implication, which was constrained by a pre-trained Stable Diffusion model~\cite{rombach2022high}.
Other approaches operate in the raster domain and leveraged external icons to replace parts of a text~\cite{tendulkar2019trick, zhang2017synthesizing}.
Our work differentiates from these works in that we imply semantics/emotion via animation of the text geometry rather than its static appearance, where the continuity between frames is considered.
To the best of our knowledge, this work is the first attempt to incorporate semantics in generating kinetic typography.

%%%%



\subsection{Guided Animation Generation}
As we aim to produce emotionally or semantically resonant kinetic typography based on a given text, relevant constraints need to be introduced in the animation generation process.
Some works infer motions directly from a given still image, concerning features like texture~\cite{chuang2005animating, kazi2014draco, lai2016data}, status in a motion cycle~\cite{xu2008animating}, periodic patterns~\cite{halperin2021endless}, \etc~
These methods are unsuitable for our goal because a text usually appears with no background and is not equipped with equivalently rich properties for motion inference.

Motion transfer has been a standard task in computer vision, which is to generate a video based on a source image and a driven video by learning the motion from the driving video while preserving the appearance of the source image.
Monkey-Net~\cite{siarohin2019animating} was the first model-free approach to transfer motions of arbitrary objects by aligning key points between the source and target domain.
FOMM~\cite{siarohin2019first} further enhanced it with local affine transformations on the extracted key points.
It is one of the state-of-the-art models and we adapted it to fit the vector-based text.
Specifically, we maintained the text legibility by regularizing motion anchors with the distance change in the Laplacian coordinate.
Our method shares the same idea to preserve the structural information as DAM~\cite{tao2022structure}, which introduced a latent root anchor to model the structure of objects.
Different from our focus on the text, most existing datasets and models concern talking heads and human posture (\eg,~\cite{chan2019everybody, zhou2020makeittalk, siarohin2021motion, hong2022depth, smith2023tog}) and do not yield desired results on texts where legibility matters (see \autoref{sec:method}).
Our exploration of kinetic typography contributes to a unique case of cross-domain motion transfer.


In addition to fully automatic approaches, mixed-initiative interfaces for animation authoring have been investigated.
Users may specify the intended effect with sketch-based demonstration~\cite{kazi2014draco, xing2016energy, kazi2016motionamplifiers, willett2018mixed}, gestures~\cite{arora2019magicalhands}, or examples~\cite{dvoroznak2017example}.
% For instance, Energy Brushes~\cite{xing2016energy} inferred the animation base on the user's sketch gestures that coarsely define the underlying forces, such as water drift or heat ascension.
% Willett~\ea~\cite{willett2018mixed} presented a tool for animating visual elements of a static picture, where users scribbled sample objects and their motion direction, and the system applied the motion to similar objects.
Pose2Pose~\cite{willett2020pose2pose} supports creating cartoon character animation by minimizing the design efforts through clustering postures and automatically matching the stylized postures designed by the artists to the driving video. 
Most similar to our work, Live Sketch~\cite{su2018live} leveraged motion transfer to let novice users create animated sketches, where users are required to define control points in both the source and target domain. 
Our approach also allows users to specify their desired animation effect through a GIF, which is easy to access online.
However, the key points in the driving video are automatically extracted and automatically mapped to the target domain.
For a finer-grain control, users can  adjust the extracted key points and internal parameters.



\section{Design Considerations}
\rev{
Motivated to lower the barrier in creating kinetic typography, we explore motion transfer techniques. Instead of tweaking keyframe configurations from scratch, users may specify the desired animation effect based on a reference GIF. With numerous online GIF instances, users may derive more diverse animation effects compared with using template-based tools.

One major design consideration is to {\it support both direct generation and fine-grain refinement} (\textbf{C1}). We expect our approach can generalize to various user requirements, including casual use as in online-messaging and professional editing like video-making. In addition to producing one-off results, the tool should allow refinement over fine-grain configurations of each frame. This is because motion transfer inherently introduces uncertainties in the generated result from motion transfer, which may violate user preference.

Additionally, we strive to {\it empower creators with interpretable algorithmic parameters} (\textbf{C2}). We hope the system supports iterative refinement, which necessitates providing explanations for the generation process so that users can provide feedback and make adjustments at every step of the generation process. By combining human experience and supervision, we seek to achieve higher quality and more consistent generations in line with users' expectations.
}
% To guide our design of an authoring tool, we make design considerations listed below.

% \begin{itemize}[leftmargin=1em]
    % \item \textbf {C1: Generate kinetic typography based on driving GIFs.} Witnessing the limitations of existing works, we try to let users specify the desired animation effect based on a reference GIF instead of tweaking keyframe configurations from scratch. With numerous online GIF instances, users may derive more diverse animation effects compared with using template-based tools.
    % \item \textbf{Support both on-the-fly generation and fine-grain refinement.} We expect our approach can generalize to various user requirements, including casual use and professional editing. Instead of producing one-off results or letting users control everything, we try not to set a boundary and allow dynamic workflows for different usage scenarios.

    % since kinetic typography has broad applications and permeates our daily lives.  
    % \item \textbf{C3: Expose algorithmic parameters in an intuitive way.} We hope the system to support iterative refinement, which necessitates providing explanations for the generation process so that users can provide feedback and make adjustments at every step of the generation process. By combining human experience and supervision, we strive to achieve higher quality and more consistent generations in line with users' expectations.
    % \item \textbf{C3: Support fine-grain refinement.}  We expect the system to support interactive refinement, allowing users to provide feedback and make adjustments at every step of the generation process. By combining human experience and supervision, we strive to achieve higher quality and more consistent generations in line with users' expectations.
% \end{itemize}



%   C1: Simplify the authoring procedure. To address the current challenge of time-consuming and experience-dependent animated text creation, our tool aims to provide a simple and user-friendly generation process that is clear and intuitive to use, even for non-technical users and instant applications.

% old c1 simplify c2 personalize c3 fine-grain

%    g1 - simply authoring procedure
%    g2 - useful for different user profiles


\begin{figure*}[h]
  \centering
  \includegraphics[width=\textwidth]{figures/workflow.pdf}
  \caption{\rev{Overview of our approach. Inputs are a driving GIF and static text. The output is kinetic typography echoing the GIF's animations. The motion trajectory extraction module captures the key points in the driving GIF. The key point alignment module aligns the control points of the vectorized text to the key points. The position optimization module regularizes the text outline. And the User Interaction module allows human intervention on the intermediate key points and final results. }}
  \label{fig: overview}
  \Description{Figure 2 shows a conceptual diagram of the proposed method. The input is a GIF and a static text, and the output is kinetic typography. There are four modules. The motion trajectory extraction module receives the GIF and rasterizes static text as input and outputs the key points. This is input to the key point alignment module along with the control points obtained by vectorizing the static text, and output the deformed control points, which will be input to the position optimization module and output the final result. There is also a user interaction module pointing to key points and final control points for the adjustment.}
    % \vspace{-0.1in}
\end{figure*}
\section{Framework}
\label{sec:method}
In this section, we introduce a general framework for transferring the motion of a given GIF to a static text.
\subsection{Overview}
\autoref{fig: overview} illustrates our framework. 
\rev{The computational pipeline takes in a driving GIF and static text as input and outputs the kinetic typography. Users can tweak the intermediate key points and control points in the generated result (\textbf{C1}).}

Internally, the input text is represented in the TrueType format~\cite{penney1996truetype}.
It is first converted into an image and fed into a FOMM model~\cite{siarohin2019first} together with the anchor GIF to obtain the trajectory of the motion key points at each frame $X_i^f$, where the model identifies $N$ key points, and the GIF consists of $F$ frames, \ie, $i=1, ..., N$, $f=1, ..., F$.
The input text is also parsed to the initial control point set $C_j^0$ of its glyphs, with a total of $M$ control points, \ie, $j=1, ..., M$.
A local affine transformation is applied to both the initial control point set $C^0$ and the key point set trajectory $X^f$ to obtain the motion trajectory of the control point set $C_j^f$.
The updated control point trajectory $C_j^{\prime f}$ is attained through position optimization.
Subsequently, a vectorized glyph sequence is generated, culminating in the creation of animated text in vector form. 
Through the user interaction module, $X_i^f$ and $C_j^{\prime f}$ can be directly manipulated, and users can control some hyperparameters (\textbf{C2}).


\begin{figure*}[t!]
  \centering
  \includegraphics[width=\textwidth]{figures/interface.pdf}
  \caption{Wakey-Wakey: An authoring interface to interactively create anchor-based kinetic typography. There are three views: input view, correction view, and refinement view. (A) Input and preview the text, where font and color can be specified. (B) Upload a driving GIF. (C) Preview the matching of key points between the text and GIF at each frame. (D) Directly manipulate key points locations. (E) Fine-tune the hyperparameter. (F) Preview result GIF. (G) Refine the text control points at each frame. }
  \label{fig: interface}
  \Description{Figure 3 is a screenshot of the authoring interface for creating kinetic typography decomposed into seven areas. There are an input view, a correction view, and a refinement view. The input view allows for the input and customization of text (A) and the upload and preview of GIF (B). The correction view displays the key points in the animated image (C) and supports drag for correction (D). The refinement view contains parameter input boxes (E), supports control point editing (G), and displays the final result (F).}
    % \vspace{-0.1in}
\end{figure*}
\subsection{Motion Trajectory Extraction}
\label{subsec: fomm}
     We convert the static text to an image, and input it along with the anchor GIF to obtain the trajectory of motion key points. We adopted motion transfer to support the fast and flexible generation of kinetic typography.
As the object in the GIF usually differs from the text in shape, we need to separate the appearance and extract the motion trajectories of key points from the source GIF.
% Most existing motion transfer models are designed for a specific field, requiring prior knowledge of the object. 

FOMM~\cite{siarohin2019first} is applied for key points extraction in our task. It is a self-supervised method using a framework that decouples appearance and motion, which effectively enriches the possible transferable motions to support motion transfer within any object category.
% FOMM features its key point detector, which learns the transformation information of key points and their surroundings in a self-supervised way. 
To address the problem of large differences in key points between the driving frame $\mathbf{D}$ and the source image $\mathbf{S}$, the FOMM model introduces an abstract reference frame $\mathbf{R}$ and obtains $\mathcal{T}_{\mathbf{S}\leftarrow \mathbf{D}}$ by separately calculating $\mathcal{T}_{\mathbf{S}\leftarrow \mathbf{R}}$ and $\mathcal{T}_{\mathbf{D}\leftarrow \mathbf{R}}^{-1}$.
$$
\mathcal{T}_{\mathbf{S} \leftarrow \mathbf{D}}=\mathcal{T}_{\mathbf{S} \leftarrow \mathbf{R}} \circ \mathcal{T}_{\mathbf{R} \leftarrow \mathbf{D}}=\mathcal{T}_{\mathbf{S} \leftarrow \mathbf{R}} \circ \mathcal{T}_{\mathbf{D} \leftarrow \mathbf{R}}^{-1},
$$
where $\mathcal{T}_{\mathbf{A}\leftarrow \mathbf{B}}$ denotes the mapping from the image $B$ to $A$.

In the implementation, $\mathcal{T}_{\mathbf{S}\leftarrow \mathbf{R}}$ and $\mathcal{T}_{\mathbf{D}\leftarrow \mathbf{R}}$ are obtained by key points detection in $\mathbf{S}$ and $\mathbf{D}$, respectively, which supports us to extract the key point trajectories from both the source and generated pixel-based text GIFs. Either of the two trajectories of the key points can be applied to drive the subsequent generation, and we use $X_i^f$ to represent the selected key point trajectory for simplicity. The separate detection mode also supports the relative generation ($\mathcal{T}_{\mathbf{S}_t\leftarrow \mathbf{S}_1}$ to deform from the source image) following a similar mindset, in addition to the absolute way ($\mathcal{T}_{\mathbf{S}_t\leftarrow \mathbf{D}_t}$ to deform from the corresponding frame of the source GIF directly).
%making it easy to switch between absolute (directly deforming from the corresponding frame of the source GIF) and relative (deforming from the source image) generation way.

In our implementation, we utilized the pre-trained FOMM model on the MGif dataset\rev{~\cite{siarohin2019animating}}, which has shown good performance in key point detection. 
\rev{Following the pre-trained model, the features extracted for each frame are estimated independently, and the number of key points is set to 10.}
However, to further enhance the integration of emotion into generation and analysis, we gather and create a dataset of Puppy Maltese~\cite{linedog} with 77 emotional-labeled GIFs and use it to fine-tune the model. This is done to better cater to the needs of the subsequent case studies and user surveys.

Due to the difficulty of FOMM in achieving good performance in motion transfer across different categories of objects, we only extract intermediate results from the key point detection module and redesign the subsequent generation steps based on our task.
We compare our result with the rasterized output of FOMM in a crowdsourcing study introduced in \autoref{sec:crowd}.


\subsection{Key Point Alignment}
\label{sec: keypointalignment}
A local affine transformation is applied to align the motion trajectories of key points to the control points. It introduces non-linearity to preserve local information better and achieve richer deformation. 
In our task, each key point extracted from each frame from the GIF is considered a local region, and the local affine transformation matrix set is obtained by computing the translational transformation of each key point in adjacent frames. The global nonlinear transformation is then calculated using a distance-weighted interpolation-based approach with the matrix set.
$$
\left[\begin{array}{c}
C_{j}^{f+1} \\
1 \\
\end{array}\right]
=\sum_{i=1}^{N} w_{i}(C_{j})\cdot\left[\begin{array}{c}
C_{j}^0 \\
1 \\
\end{array}\right] \left[\begin{array}{cc}
\mathcal{I} & 0 \\
(X_{i}^{f} - X_{i}^1)^{T} & 1
\end{array}\right],
$$
$$
w_i(C_{j})=\frac{1 / \|C_{j}^0 - X_{i}^1\|^{e}}{\sum_i 1 /\|C_{j}^0 - X_{i}^1\|^{e}}.
$$
$C_{j}^f$ and $X_{i}^f$ denote the control point $j$ and the key point $i$ at frame $f$, respectively. The control point's position at each frame is calculated in reference to the key point at the first frame to achieve global stability. $\mathcal{I}$ is a 2nd-order identity matrix. $w_i$ is a weight function for a control point with respect to the key point $i$.

The weight decays according to the inverse of the $e$-th power of the relative distance from $X_i$ to $C_j$, where $e$ controls the locality of the affine transformations, \ie, the degree to which each affine transformation affects the target point.
\begin{figure*}[t]
  \centering
  \includegraphics[width=\textwidth]{figures/alpha-test.pdf}
  \caption{Comparison of the generation results with $\alpha = 0, 2, 4$. Increasing $\alpha$ enhances the smoothness of the glyph, but an excessive value of $\alpha$ may negatively impact the amplitude of the motion.}
  \label{fig: ablation1}
  \Description{Figure 4 contains three blocks showing cases applying different parameter alpha. The vertical axis represents different values of the parameter alpha (0, 2, 4), and the horizontal axis shows the generated kinetic typography from three words "sleepy", "thanks", and "wakey" arranged by frame.}
    % \vspace{-0.1in}
\end{figure*}
\subsection{Position Optimization}
To alleviate inappropriate deformation of glyphs caused by changes in the relative position of the control points, we optimize the positions of the control points by frame based on the Laplacian coordinate, which generally describes the relative positions on the surface using the neighbor information.
For a control point $j$ at frame $f$, its Laplacian coordinate $L_j^f$ is calculated as
$$
\begin{aligned}
    L_{j}^f &= \sum_{k \in N_{j}}\omega_{jk}^f\left(C_{k}^f-C_{j}^f\right) =\sum_{k \in N_{j}}\omega_{jk}^f C_{k}^f - C_{j}^f,
\end{aligned}
$$
where $C_j^f$ and $C_k^f$ denote the Cartesian coordinate of the control point $j$ and $k$, respectively. $N_{j}$ is the set of $K$-nearest neighboring control points with the smallest Euclidean distance to the control point $j$, which is calculated based on the initial control points set $C^0$, invariant to changes in $f$. $\omega_{jk}^f$ denotes the weight of the neighbor point $k$ in the Laplacian representation of the control point $j$, where
$$
\omega_{jk}^f=\frac{1 / \|C^{f}_{k} - C^{f}_{j}\|^{2}}{\sum_{k \in N_j} 1 /\|C_{k}^f - C_{j}^f\|^{2}}.
$$

Considering the inhomogeneity of the distribution of discrete sampling points, we use the aforementioned distance-based weights to describe the detailed location information better.
Further, we design the following objective function  $\mathcal{L}_{\text{total}}$ to optimize the coordinates of the sequence of control points obtained by frame.
$$
\begin{aligned}
 \mathcal{L}_{\text{total}} &= \alpha \cdot  \mathcal{L}_{\text{glyph}} + \mathcal{L}_{\text{motion}}, \ \alpha\in[0,+\infty),\\
 \mathcal{L}_{\text{glyph}} &= \sum_{j=1}^{M}\left\|L_{j}^f-L_{j}^0\right\|^{e},\  
 \mathcal{L}_{\text{motion}} = \sum_{j=1}^{M}\left\|C_j^f - C_j^{f\prime}\right\|^{e}.\\
 % C1, C2, ..., C_n & = \arg\min\mathcal{L}_{\text{total}}.
\end{aligned}
$$

$L_{j}^f$ and $L_{j}^0$ denote the Laplacian coordinates of the control point $j$ in frame $f$ and $0$, respectively. $C_j^f$ and $C_j^f\prime$ denote the coordinates of the control points before and after optimization.
$\mathcal{L}_{\text{glyph}}$ measures how much the local shape details are preserved, which is computed as the sum of the distance of Laplacian coordinates between the optimized and initial control points. 
$\mathcal{L}_{\text{motion}}$ measures how much of the motions are preserved, as the sum of the distance of the control points before and after optimization, i.e., minimizing edit distance.
$\alpha$ is a hyperparameter representing the trade-off between the two loss functions.
The larger $\alpha$ is, the more details of the initial glyph and the less motion are preserved.
As for the norm $e$, the larger it is, the locality is more regulated, which leads to stronger deformation.
We use a $K$-dimensional tree to accelerate the nearest-neighbor search, where the parameter is empirically set: $K=3$.
\rev{While we employ frame-by-frame optimization, we note it is worth introducing global temporal regularization terms in the loss function to promote smoothness and consistency. 
% such as a penalty term that penalizes abrupt changes or high-frequency fluctuations between consecutive frames.
}

\subsection{User Interaction}
The user interaction module allows direct manipulation of the computed positions of the key points and control points at each frame, \ie, $\{X_i^f\}$ and $\{C_j^f\}$, $\forall i \in [1, N]\cap \mathbb{N}$, $j\in [1,M]\cap\mathbb{N}$, $f\in [1, F]\cap\mathbb{N}$.
In this way, creators of kinetic typography can participate in the motion transfer process and adjust the final results according to their needs.
The hyperparameter $\alpha$ in the position optimization stage can also be adjusted for different texts, as illustrated in \autoref{sec:ablation}.




\section{Authoring Interface}
\label{sec:authoring_tool}
% Leveraging the self-explainability of our approach, 
Based on the proposed framework, we implement a mixed-initiative authoring tool called Wakey-Wakey\footnote{\rev{The name suggests that the authoring tool awakes static text and makes it lively.}} that allows fine-grain adjustment for more natural and aesthetic results (see \autoref{fig: interface}).
This section offers a step-by-step guide showing how to interactively generate animated text with our tool referring to the interface.



%in line with the practice of previous works~\cite{gatys2016image, mao2022intelligent}.


The user-oriented process mainly consists of three steps: text and GIF input, key point correction, and glyph refinement, each supported by a view: \textit{Input View}, \textit{Correction View}, and \textit{Refinement View}.
While the input step is mandatory, the correction and refinement stages are optional. This allows for simple end-to-end personalized generation, as well as interactive improvement.

\paragraph{Input View.}
Users first input the text and customize its static appearance with the global typeface and color (\autoref{fig: interface} A).
They can upload and preview the driving GIF through a button (\autoref{fig: interface} B) .
The section will record and list the recent upload history. 
After clicking ``Next'', the system will process the input with our method, and both intermediate and final results will be displayed in the other two views. Users can easily obtain the generated animated text here without any additional effort.

\paragraph{Correction View.}
Users can then drag the displayed the key points from the motion trajectory extraction module to a suitable location at a specific frame, as shown in \autoref{fig: interface} D.
Special attention can be paid to the key point trajectory with a corresponding colored button above. 
Two thumbnails (\autoref{fig: interface} C) enable switching between the anchor GIF and the extracted key points for correction.
As the corresponding key points share the same color, users can learn how the mapping is.
% The correction on the GIF can also improve the generation of FOMM. 
Users are supported to ensure better quality by maintaining reasonable motion trajectory to drive the generation of the vector animated text.

\paragraph{Refinement View.}
Users may configure the parameters $\alpha$ and $e$, and select whether the kinetic typography is generated by aligning to key points from the anchor GIF or the extracted key points (\autoref{fig: interface} E).
The bottom panel (\autoref{fig: interface} G) enables the users to adjust the glyph by dragging the control points.
And the final GIF of kinetic typography is displayed in the middle panel (\autoref{fig: interface} F).


\section{Implementation}
\rev{\tool{}\footnote{Source code available at \url{https://github.com/KeriYuu/Wakey-Wakey}.} was implemented as a client/server web application.
The front end was built with \texttt{Vue} for user interactions.
The computational framework for generating kinetic typography was implemented in Python.
An automatic generation takes around 300ms/frame (CPU: Intel i7 4.9 GHz).
The \texttt{Flask} framework is used to handle the messaging between the front end and the back end.
}





\section{Method Analysis}
\label{sec:crowd}
Due to the absence of pre-defined ``ground truth'' and lack of standard metrics in the nascent area of motion transfer for quantitative assessment, we empirically evaluated our approach by (1) analyzing the impacts introduced by each component, (2) comparing the automatically generated result from different styles of GIFs and typefaces, and (3) conducting questionnaire studies to understand how general people perceive the outputs based on several cases.


\subsection{\rev{Effects of Components}}
\label{sec:ablation}
\rev{We evaluated the effect of each component in the workflow} to analyze how our adaption to FOMM and the introduced human interventions can improve the generated result, including local position optimization, vectorized text representation, key point correction, and glyph refinement.

% 

\paragraph{Local Position Optimization}
The position optimization module is introduced to preserve the local shape of each glyph better.
\autoref{fig: ablation1} demonstrates three motion transfer results with $\alpha$ set to $0, 2, 4$. 
$\alpha$ is the weight of $\mathcal{L}_{\text{glyph}}$, which controls the degree of preservation of local shape. As can be seen, when alpha is set to 0, i.e., without local position optimizing, some local parts of the glyphs are unsatisfactory, such as the ``p'' in ``sleep'', the ``t'' and ``k'' in ``thanks'', and the ``a'' and ``k'' in ``wakey''.
With the increment of $\alpha$, the glyph becomes smoother. However, when alpha is too large, it may cause too much preservation of the original glyph and result in a loss of motion, for example, the ``w'' and ``k'' of ``wakey'' when $\alpha = 4$. Through experiments, we find a suitable default value of 2.
% which is the weight of $\mathcal{L}_{\text{glyph}}$.
% Without position optimization, due to changes in the relative position of points after local affine transformation, adjacent letters may become entangled and compressed together, such as the text ``orte'' shown in \autoref{fig: ablation2}, with excessive deformation of individual letters and narrow separation between them. 
% However, with position optimization, the independence and integrity of the letters are enhanced, the deformation of individual letters is more moderate, and the boundaries between letters are clearer and more distinct, resulting in stronger legibility.


\label{sec:case}

\begin{figure}[t]
  \centering
  \includegraphics[width=\linewidth]{figures/case-study.pdf}
  \caption{A comparison with the FOMM model~\cite{siarohin2019first}. Our approach operates on the control points of vectorized text, which improves legibility.}
  \label{fig: case}
  \Description{Figure 5 contains three rows of animation keyframes. From top to bottom, they are the driving GIF, the rasterized animated text generated by the FOMM model, and the vectorized kinetic typography generated by our method.}
    % \vspace{-0.1in}
\end{figure}
\paragraph{Vectorized Text Representation.}
% Users can select anchor GIFs with expected semantics to drive the corresponding generations of animated text, such as emotions like anger. 
Instead of directly employing existing image-based motion transfer models, our approach operates on the control points of text glyphs.
As shown in \autoref{fig: case}, the output results based on pixels are not stable enough.
For example, in the pixel-based glyph generated by FOMM, the letter ``y'' of ``angry'' has breaks and extra noisy strokes.
As the anchor GIF is hardly the targeted category of kinetic typography, using FOMM for motion transfer does not produce satisfactory results. 
In contrast, our method better preserves the integrity and legibility of the glyph and produces a more stable frame sequence.



\paragraph{Key Point Correction.}
The key points $\{X_i^f\}$ detected by the model may not always be accurate, which can result in unexpected deformations in the generated animated text that rely on these key points. 
Our approach allows users to interactively correct the key points, thereby obtaining a more desirable generation that aligns with their expectations.
As shown in \autoref{fig: ablation2}, by analyzing the preceding and following frames, we can find that in the fourth frame of the pixel-based animated text image sequence generated by FOMM, the key point marked in red noticeably shifts towards the right. 
This caused an excessive deformation towards the right in the lower right part of the letter ``W'' in the vector-based animated text generated with this key point. 
By dragging the key point towards the left to an area consistent with the preceding and following frames, the deformation of the generated glyph appears more reasonable and smoother.


\begin{figure}[h]
  \centering
  \includegraphics[width=\linewidth]{figures/alignment.pdf}
  \caption{Comparison of the output before (top) and after (bottom) key point correction. The automatic mapping fails when the highlighted red key point shifts to another location in two consecutive frames, which yields distortion in the text.}
  \label{fig: ablation2}
  \Description{In Figure 5, the left part shows the key points before and after correction, and the right part shows the corresponding generated vector text.}
    % \vspace{-0.1in}
\end{figure}


\paragraph{Glyph Refinement.}
The control point sequence $\{C_j^f\}$ can be manually updated for fine-grain refinement.
Through the authoring interface, users are supported to drag the control points and preview the result immediately.
As shown in \autoref{fig: ablation3}, the sharp corners inside the first letter ``p'' affect the glyph aesthetics, where the highlighted left line segment is tilted to the left and needs to be adjusted. 
By moving the three control points in the sharp corner area to the right and adjusting the relative positions of the three points, the refinement process is done. 
It is evident that the updated glyph achieves a better effect through simple and immediate dragging.


\begin{figure}[h]
  \centering
  \includegraphics[width=\linewidth]{figures/glyph-refinement.pdf}
  \caption{Manual refinement of the text control points. By dragging the distorted control points, one may intuitively refine the output kinetic typography at a fine-grained level.}
  \label{fig: ablation3}
  \Description{In Figure 7, the left and right are the glyphs before and after the glyph refinement, and the center shows the control points that were adjusted.}
    % \vspace{-0.1in}
\end{figure}




\subsection{\rev{Generalizability}}
\label{sec:generalizability}
\begin{figure*}[h]
  \centering
  \includegraphics[width=0.8\textwidth]{figures/complexity.pdf}
  \caption{\rev{A comparison of results from driving GIFs of different complexities in background and target object(s).}}
  \label{fig: complexity}
  \Description{In Figure 8, there are four blocks displaying keyframes of the generated results of “UIST” from different complexities of GIF inputs.}
\end{figure*}

\begin{figure*}[h]
  \centering
  \includegraphics[width=\textwidth]{figures/font.pdf}
  \caption{\rev{A comparison of results from typefaces of different categories and average number of control points for the 26 English alphabets (Avg. \#Control). The first column shows four key frames of the driving GIF. Each column in the rest shows the font information and the corresponding motion transfer result.} }
  \label{fig: font}
  \Description{In Figure 9, there are eleven columns. The first column shows four keyframes of the driving GIF. The rest ten columns show corresponding keyframes based on different types of typefaces. Three additional rows in each column present the class of the typeface, the name of the typeface, and the average number of control points for 26 English alphabets in the typeface.}
\end{figure*}


\rev{
Drawing from our experience, we reflect on the generalizability of our approach in terms of the driving GIF and the input typeface.

In general, \tool{} can accommodate input GIFs with a clean background and a simple-shape moving rigid body, such as instances from the Puppy Maltese dataset.
This is because our implementation adopts the pre-trained FOMM model based on the MGif dataset, which features a white background and one cartoon animal. 
Seen from \autoref{fig: complexity}, the automatic key point extraction may fail and cause large distortion when there are multiple moving objects, or the moving object exhibits complex patterns.
A complex background also threatens the reliability of extracted key points in the driving GIFs, such as a clip from natural videos.
While these issues can be addressed by manual correction, we also note that the motion trajectory extraction module can be improved by unsupervised training on a larger dataset with representative cases or using a large universal model.

As for the input fonts, our approach empirically performs well for typefaces with more than 5 control points in a glyph.
The more control points encapsulated in the typeface, the more likely that the position optimization can maintain its legibility.
\autoref{fig: font} showcases the automatic generation results for ten typefaces of common classes.
Typefaces with the most control point number also yield the most smooth results, including Fredericka and Cedarville.
However, there might be strong deformation for handwriting-styled typefaces, potentially due to their high flexibility.

}

\subsection{Questionnaire Study}




We conducted two questionnaire studies to evaluate the effectiveness of our approach.
Specifically, we seek to understand (1) whether our approach convincingly transfers the motion, and (2) to what extent the semantics of the original GIF can be preserved.

\subsubsection{Setup} The questionnaires are distributed on Qualtrics.
\rev{Participants are required to complete Study I before Study II}. And the questions appear in a random order in each study. 
We used meaningless pseudo-words from the Lorem Ipsum corpus~\cite{lorem} as input text in order to minimize the influence of text content.
\rev{For driving GIFs, we used the Puppy Maltese dataset to generate cases. To avoid confounding effects, the driving GIFs are non-repetitive. And we employed the typeface ``Akronim'' for it has over 300 control points, which may lead to satisfying results without human intervention and therefore suitable for our scenario requiring batch generation.}

\paragraph{Study I: Motion Transfer}
The first study aimed to evaluate the overall quality of the output kinetic typography.
As no quantitative metric is available in our task, we obtained subjective assessments by asking the participants to rate the similarity between the driving GIF and the output kinetic typography and their aesthetics.
On the one hand, the similarity between the source and target is the primary goal in motion transfer.
On the other hand, aesthetics is a common pursuit in animation design.

% 丢去结论 First, whether the motion of the anchor GIF is effectively transferred, thus validating the feasibility of the technique; 
% second, whether the animated text is visually appealing, confirming the applicability of the generated results.


% Following the opinion that it is just as valid to believe that an artwork is aesthetically pleasing without providing a specific reason, as it is to attribute its aesthetic appeal to a particular visual quality~\cite{AestheticPrinciples}, we take ``aesthetically pleasing'' as the aesthetic metric, and use "move similarly" to assess the motion.
A sample question is shown in \autoref{fig: crowd} A.
When designing the questionnaires, we tried to familiarize participants with simple and concrete questions.
For instance, we asked whether a GIF is ``aesthetically pleasing'' to align participants' appraisal of the aesthetic property to their feelings~\cite{AestheticPrinciples}\rev{.}
For each question, the animated text and the corresponding anchor GIF are displayed, and participants are asked to rate them on a 7-point Likert scale (0--strongly disagree to 6--strongly agree) for aesthetics and motion similarity, respectively. 
% From the dataset, we selected 6 driving GIFs labeled corresponding to Ekman's six basic emotions~\cite{ekman1999basic}, and the other 14 driving GIFs are selected randomly, making a total of twenty GIFs.
20 driving GIFs were randomly selected.
For each driving GIF, we set two questions, one for the baseline--rasterized kinetic typography generated with FOMM, and one for the experimental group--vectorized kinetic typography with our approach.
Therefore, a questionnaire consists of 40 questions.
Considering the influence of the font, participants are asked to rate the aesthetics of the static font before viewing the main body of the questionnaire.

\begin{figure*}[h!]
  \centering
  \includegraphics[width=\textwidth]{figures/crowd.pdf}
  \caption{Example questions and results in two questionnaire studies (N=33). (A) Study I: Subjective ratings for the aesthetic property of the output kinetic typography and similarity between the target and source GIF. (B) Study II: Perception of emotions underlying kinetic typography. (C) The average ratings and standard errors of cases in Study I, where our approach outperforms the baseline FOMM in motion similarity and result aesthetics. (D) The average ratings and standard errors of the pairwise (pleasure, arousal) ratings for sample GIFs, where the driving GIF and corresponding kinetic typography posit in adjacent areas. \rev{Ratings of the same emotion are encoded with colors of a similar hue.}}
  \label{fig: crowd}
  \Description{Figure 10 contains four figures demonstrating the questionnaire design and results. (A) A snapshot of Questionnaire I. The top shows the anchor GIF and the corresponding kinetic typography. The bottom is a 7-point scale for rating motion similarity and aesthetic pleasure. (B) A snapshot of Questionnaire II. The top shows a GIF, and the bottom is a slider scale for rating pleasure and arousal. (C) The results of Questionnaire I show that our method is superior to FOMM in terms of aesthetic appeal and movement similarity. (D) The results of Questionnaire II show that the anchor GIF of a certain emotion and its corresponding generated kinetic typography are located closely on a two-dimensional coordinate system with pleasure and arousal as the x and y axes, respectively.}
    % \vspace{-0.1in}
\end{figure*}
\paragraph{Study II: Semantic Preservation}
% user emotion
% picture emotion
% lorem
% 防止上下题情感的影响,random

The second study aimed to verify whether the learned animation can preserve the semantics of the driving GIF.
We tackled the problem from the perspective of emotion, which is an integral part of semantics.
Specifically, we focused on Ekman's six basic emotions~\cite{ekman1999basic}, \ie~sadness, happiness, fear, anger, surprise, and disgust.

\autoref{fig: crowd} B illustrates a sample question. We adopted the Affective Slider~\cite{betella2016affective} for participants to self-report their emotions, which consist of two dimensions: pleasure and arousal from the extent 1 to 100.
Pleasure means the degree of positivity or negativity of an individual's emotional state.
Arousal corresponds to the level of physiological activation or stimulation in an individual's emotional state.
For each basic emotion, we selected two GIFs as anchors \rev{according to the pre-defined labels in the dataset.}
%resulting in a total of 12 animations used in the questionnaire.
A question comprises one kinetic typography, where we required participants to assess the perceived emotions.
Hence, there were 12 questions.

% The corresponding animated text was generated for each anchor GIF, resulting in twelve questions assessing the emotions conveyed by the twelve animations using the AS scale. 
% The intention was to compare the consistency of the emotion expressed in the source motion pictures and the animated text driven by them.




\subsubsection{Participants}
We recruited participants from a local university by posting advertisements on social media. Each participant is paid £3.5 for completing the questionnaire.
A total of 33 people signed up for the questionnaire study. 
Most participants were between 18--24 years old, with 15 females and 18 males.
%indicating that the study primarily targeted a younger demographic 
% who is more familiar with contemporary digital trends and communication.
% The gender distribution among the participants showed a male-to-female ratio of 4:6, suggesting a fairly balanced representation of both genders, albeit with a slightly higher proportion of female participants.
In addition, all participants reported using emojis frequently in daily communication, where 
14 (42\%) reported to use emojis \textit{very often}.

\subsubsection{Result Analysis} % The results initially validated our approach. 
\paragraph{Study I. Motion Transfer}
Participants spent an average of 10.4 minutes in completing the 20 questions (std=6.2, ranging from 3.5 to 28).
We deemed all the responses valid.
% \rev{The average aesthetic score on the static text was 3.79.}
Seen from \autoref{fig: crowd} C, participants generally recognized the aesthetics and the similarity of movement between the driving GIFs and the animated text in our approach.
\rev{For our approach, t}he average score on the aesthetics was 4.71 (std=1.21), and the motion similarity was 4.85 (std=1.06), where both scores exceeded 4, \ie, somewhat agree, on the 7-point scale. \rev{For the baseline, the average aesthetic score was 2.34 (std=1.60) and the average similarity score was 3.31 (std=1.59).}
Compared to the baseline method, our approach obtained significantly higher scores (significance level $\alpha=0.001$, Student's t--test), which suggests that our method outperforms the FOMM in terms of motion transfer, making the generated animations more visually appealing and more similar to the source motion.
This finding echoes our ablation study on the vectorized text representation, where we identified certain glitches in the pixel-based methods.
% In addition, the average score for the font was 3.788, indicating that the participants found the font aesthetically pleasing. 
Moreover, after adding the dynamic motion, the aesthetics of the text improves compared to static text with an average aesthetics score of 3.79, further validating the effectiveness of our method.
% As for user comments, several participants provided comments stating that our model performed much better than the baseline FOMM. 
% These comments support the quantitative results and emphasize the superiority of our model in terms of motion transfer and visual appeal.

In summary, Study I verified that our model could achieve better motion transfer compared to the baseline. 
The improved aesthetics and motion similarity scores, along with the user comments, demonstrated the effectiveness and applicability of our method in generating visually appealing and motion-consistent animated text.
% \begin{table}[ht]
% \centering 
% \begin{tabular}{ lll } 
% \toprule
%  & aesthetical pleasure & motion similarity\\
% \midrule
% FOMM & 2.34 ($\pm$1.60) & 3.31 ($\pm$1.59)\\
% Ours & \textbf{4.71 ($\pm$1.21)} & \textbf {4.85 ($\pm$1.06)} \\
% \bottomrule
% \end{tabular} 
% \caption{mean and std} %title of the table 
% \label{tab:evaluation} 
% \end{table} 
\begin{figure*}[h]
  \centering
  \includegraphics[width=\textwidth]{figures/workshop.pdf}
  \caption{Demonstrations in the workshop. (A) A video opening featuring synchronous animation of textual descriptions and the cartoon character. (B) A browser plugin for the automatic generation of kinetic typography based on our approach.}
  \label{fig: video}
  \Description{Figure 11 shows two usage scenarios labeled (A) and (B). (A) Six frames of the video opening made with kinetic typography generated by our method. (B) The browser plugin interface with the top input area above and the bottom preview and output area.}
    % \vspace{-0.1in}
\end{figure*}
\paragraph{Study II. Emotion Preservation}
Participants spent an average of 8.5 minutes to complete the questionnaire, with a maximum of 30 minutes and a minimum of 2 minutes (std=8.9).
The results of study II are shown in \autoref{fig: crowd} B, where the scatterplot maps the average scores of the (pleasure, arousal) pairs.
The attached error bars indicate the standard error of each dimension with their lengths, where vertical for arousal and horizontal for pleasure.

Inspecting the diagram, we could see that the data points for the same emotion under both the driving GIF and Text are projected on adjacent areas, revealing that the expression of emotion has also been successfully transferred through the motion transfer. 
Furthermore, different emotion classes largely varied.
For example, there is a significant difference between happy and sad emotions, demonstrating the effectiveness of our method in preserving the emotional semantics of the anchor GIFs.
One could see that the emotions of disgust and anger are very close, and the demarcation is not obvious, with only the ordering of pleasure being altered. 
This suggested that the emotional expressions in these two emotions might be similar, making it harder for users to differentiate them clearly.
In addition, the variances in the arousal dimension were generally greater, possibly due to the users' perception of arousal being more ambiguous or subjective, leading to a wider range of responses.
In contrast, the arousal and pleasure scores for the text are more neutral (around 50 points). 
Although the emotional expression has been learned and transferred, it is not as strong as in the source motion pictures. 
It might result from the limitations of the method or the inherent difference between text and the figure-like domain in representing emotions.



In summary, results from Study II showed that our method could effectively preserve the emotional semantics when transferring animations from the driving GIF to a text. However, some emotions may not be as distinct as they are in the original anchor GIFs. And users' perceptions of arousal might be more ambiguous. \rev{While the questionnaire shows the success of emotion transfer on the particular typeface being used, more studies are needed to validate similar mechanisms for other fonts, as we did not eliminate the influence on emotion perception from the typeface.}


% Based on this analysis, the participant group primarily consisted of young adults with diverse gender representation and a high familiarity with emojis. 
% This demographic might have contributed to a better understanding of the emotions conveyed by the animations and the effectiveness of the motion-transfer method in preserving the semantics of the anchor GIFs. 
% However, it would be beneficial to include a broader age range in future studies to ensure the generalizability of the results across different age groups.
\section{Workshop}
\label{sec:workshop}
We organized a workshop to evaluate the utility of our method.

\subsection{Demonstrations}
To elicit in-depth discussions in the workshop, we designed and implemented several demonstrations of potential application scenarios, including a video opening, an online messaging widget, and an emotional word cloud.

\paragraph{Video opening}

Vlogs (Video blogs) are becoming a prevalent form to share personal experiences creatively.
To make a vlog stand out, an engaging opening animation can help grab viewers' attention and set the tone for the rest of the video.
Leveraging results generated by our method, we created a vlog opening as an example of its practical application in daily content creation.
As shown in \autoref{fig: video} A, D1 is a vlog opening with the Puppy Maltese theme, showcasing different states of the character and introducing ``My Day'', which suggests the video topic. 
We envision that integrating animated text with consistent motions enhances both explanatory and entertaining values. 
Similarly, users can use our method to create and personalize animated text in their video creations across different themes and contents.

\paragraph{Browser Widget for Online Chatting}
Informed by the previous efforts in enhancing emotion communication in online messaging~\cite{Wang2004communicate, malik2009communicating, aoki2022emoballon}, we implemented a light-weighted Chrome extension to facilitate the real-time creation of kinetic typography (see \autoref{fig: video} B).
It has a simplified interface compared with Wakey-wakey, which features real-time generation and removes the human interaction module.
Users may upload an anchor GIF, type down the text, configure the color and font, and then directly obtain the generated kinetic typography in the GIF format.


\paragraph{Emotional Animated Word Cloud}
The word cloud is a common visualization technique to summarize text data, where the text size represents the word frequency.
\rev{Xie \ea~\cite{xie2023emordle} coined the word ``emordle'' representing animated word clouds that suggest underlying emotions.
Based on our approach, we generated an ``emordle'' by transferring the animated scheme of a bumpy cartoon pig from the MGif dataset.
Instead of using the parsed control points, we transformed the text anchors for each text element that constitute the word cloud. 
In other words, we replace the vector control points with the anchor points of each word in the proposed approach. Note that one word has one anchor point at its central position.}
The generated word cloud example is displayed in \autoref{fig: wccase}.

\begin{figure}[t]
  \centering
  \includegraphics[width=\linewidth]{figures/emordle.pdf}
  \caption{Application of animated word cloud that delivers a certain emotion with the animation.}
  \label{fig: wccase}
  \Description{In Figure 12, there are two rows of keyframes showing the anchor GIF and the corresponding animated word cloud.}
    % \vspace{-0.1in}
\end{figure}



\subsection{Protocol}
The workshop proceeded in the following four stages.

\paragraph{Briefing}
The briefing session took 10 minutes, during which we introduced the background of our work and briefly explained our method. We then used a demo video to demonstrate the functions of the tool and illustrate the generation process.

\paragraph{Demonstration} 
We spent another 10 minutes displaying the video opening, the online messaging widget, and the animated word cloud to demonstrate potential application scenarios. We also introduced the installation and use of the plugin through a demo.

\paragraph{Self-creation}
We then allow users to freely use and explore the system and widget to create their kinetic typography for 20 minutes. 

\paragraph{Post-interview}
After trying \tool, participants are asked to complete a questionnaire with six questions on a 7-point Likert scale about its usability, including covering \textit{Practicality}, \textit{Customization}, \textit{Pleasure}, \textit{Efficiency}, and the \textit{Intuitiveness} for widget and interface respectively.
We also raised open-ended questions following a structured template in order to understand their perceptions of the authoring process as well as the generated kinetic typography.




\subsection{Participants}
Both designers and general users are invited to obtain feedback from different perspectives in the evaluation.
We recruited 20 participants through our personal network and advertisements on social media, with 7 females and 13 males.
There are 3 professional designers: P1 works on user experience design; P2 engages in self-media creation as a blogger and vlogger; P3 is a digital painter (P3).
The rest are graduate students majoring in data science at a local university (denoted as P4–P20 in ascending time order of their interviews).
Among all participants, 13 have seen kinetic typography before, such as in online memes, short videos, slides, and advertisements. 
Only two participants have experience in creating kinetic typography with other software (P2, P3). 

\subsection{Results}
Here we present both the quantitative and qualitative results.

\subsubsection{Observations.}
Nonetheless, when it comes to the time required for creation, there is substantial dissent, as our concurrency fell short, leading to extended completion times for some users.
In spite of this setback, the system garners favorable acknowledgment for its expressiveness, user-friendliness, intuitiveness, and real-world applicability.
To elevate the overall user experience, refinements in creation duration and concurrency should be considered.
\rev{As for the usage of $\alpha$ or $e$, most users were generally satisfied with the empirical default value. However, two users with a design background occasionally fine-tuned this parameter to derive better results.
In terms of the manual adjustments on control points, users without a design background barely adjusted the control points. Three users with design backgrounds occasionally make manual adjustments, about 8/14 frames, with an average duration of 2.1/5.8 min for one kinetic typography. Users tended to adjust key points in GIFs when there were noticeable detection errors. Otherwise, they normally increased $\alpha$ to mitigate unexpected deformations, though less motion preserving.}

\subsubsection{Usability Ratings.}

\autoref{fig: quant_res} shows the distribution of users' subjective ratings of the six usability questions.
More than half of users concurred that the tool was intuitive, tailored, and delivers an enjoyable experience during usage.
 To be more specific, 45\% users strongly agreed that our work is pleasant and interesting.
 They valued the innovative animated text design, the straightforward comprehension of emotions portrayed, and the uncomplicated tool configurations for selecting animation approaches.
 They also observed that blending textual meanings with animations facilitated a beneficial expression of the content.



\begin{figure}[t]
  \centering
  \includegraphics[width=\linewidth]{figures/user-study.pdf}
  \caption{Quantitative Results of the usefulness of our method. We measured the \textit{practicality}, \textit{customization}, \textit{pleasure}, \textit{efficiency} and the \textit{intuitiveness} for the browser widget and authoring interface respectively.}
  \label{fig: quant_res}
  \Description{Figure 13 shows the overall chart is a stacked bar chart with seven categories ranging from "strongly disagree" to "strongly agree". Vertically, it represents "practicality", "customization", "pleasure", "efficiency", "widget", and "interface". The x-axis is marked with percentages and the specific statistics are indicated on the bars. Our approach has been recognized for its usefulness.}
    % \vspace{-0.1in}
\end{figure}

\subsubsection{User Feedback.}
We summarize the following insights and implications for future improvements from the users' feedback.

$\diamond$ \underline{On mixed-initiative authoring}. Participants expressed their agreement with the balance we stroke between user involvement and automatic generation. P3 said: ``{\it supporting quick automatic generation while also offering optional customization and improvement from users}''. Participants were optimistic about the mixed-initiative way, as ``{\it intelligence improves efficiency while users enhance quality and creativity, as the imagination of users cannot be dismissed}'' (P10).

$\diamond$ \underline{On personalization.}
19 participants (except P11) valued customization highly and believed it helps incorporate their own ideas and shape a unique personal style, while three designers showed an ``{\it innate aversion to preconceived solutions}'' (P1). Among them, 17 believed that customization is also crucial in the context of animated text, which helps to ``{\it express opinions and feelings more effectively and precisely}'' (P10), as well as ``{\it making the creations more specific and memorable}'' (P18).

% 新颖性,趣味性、启发性
$\diamond$ \underline{On motion transfer.}
Users expressed their agreement with the novelty, interest, and inspiration of our approach.
Seven participants mentioned ``novel'', where P9 commented ``{\it it is pleasantly surprising}''.
Four participants mentioned ``interesting'', and 3 participants found it inspiring.
P2 found {``\it the generated results inspiring and insightful for my design progress''}.
P1 said: ``{\it I appreciate the smart use of motion transfer, as the interpretation of motion is subjective, but you use memes as an intermediate medium whose ability to convey emotions is validated through wide applications, making the generated results meet subjective expectations. In addition, using text as a vector container for transformation, where the container can be liquefied, allows for slight deformation, thus in support of more subtle emotional expression.}''

$\diamond$ \underline{On application scenarios}.
Participants brainstormed multiple application scenarios in their personal life with kinetic typography, including online chatting, social media post, website banners, presentations, subtitle enhancement, and E-invitation. 
\rev{P13 also imagined that in order to attract interest, animated text may be used as a teaching instrument for introducing words to little children.}
P1 identified some challenges in the application of the animated text. ``{\it When used alone, the interpretation by users can be ambiguous. There are challenges in accessibility, readability, as well as efficiency of perception and recognition}''.
He suggested that we can ``{\it emphasize the combination with animated images, which can enhance contextual effects and create a synergistic interaction greater than the sum of the separate parts}''.

$\diamond$ \underline{On future improvements}. 
First, Participants suggested a possible enhancement in the guidance of interactions, preferably by introducing recommendations for interactive operations. 
The operations may be clearer ``{\it with the help of some icons and text}'' (P16), and ``{\it the generation efficiency can be improved by recommending interactive behaviors. In addition, it would be very helpful to support automatic modifications of other frames after modifying one frame in key point correction and glyph refinement}'' (P2).
Besides, integrating external resources may enrich the generated results.
P11 suggested ``{\it incorporating language models to generate using instructions enhances its convenience}''.
P3 commented that ``{\it integrating meme and artistic font libraries helps generate richer and more artistic results}''.




\section{Discussion}
\label{sec:discussion}
\rev{We summarize the implications of our investigation, reflect on our limitations, and discuss promising directions for future research.}

\subsection{Implication}
\rev{
\paragraph{Create animated effects with model-free motion transfer.}
We contribute a novel approach in the emerging area of AI-generated content to design motion graphics using prevalent cross-domain GIFs as references.
Participants in the workshop acknowledge the ease of guiding animation generation with reference GIFs.
Despite various properties to coordinate in animation design, the underlying workflow of \tool{} helps users to author in a top-down manner instead of tweaking every details.
Beyond template engines, model-free motion transfer helps create more diversified effects.

\paragraph{Support human-AI collaboration with interpretable features.}
According to the user feedback in the workshop, the extracted key point helps them understand the causes of misalignment.
While most of our participants are unaware of the internal mechanism, they are able to correct the flaws introduced by the black-box model and intervene in the generation process to produce more desirable results.
In designing AI-empowered authoring tools, interpretable features are practical entry points for humans to inject requirements into the content creation process.


\paragraph{Consider design requirements of users at different levels.}
\tool{} supports both one-off generation and fine-grain control.
On the one hand, there are default values underlying algorithms to cater to the fast-generation need of causal users.
On the other hand, configurable parameters and the vector representation of generated results are also exposed for further adjustment.
When developing authoring support for prevailing artifacts, such as kinetic typography or data visualization, it is important to consider the requirements of different user profiles to make the authoring tool more useful.
}

\subsection{Limitation}
% 文字没有复杂的内部结构,丧失很多语义信息。可以结合word-as-image,动态+外形一起表达语义
% 变形过大的时候轮廓还是会不光滑:扩充控制点;用三角面片的形式来优化
% 对动图物体有要求:rigid body
% We pose three primary limitations of the proposed approach, including the deformation stability between frames, semantic expressiveness of text, and the utilization of interaction data. Accordingly, we propose potential solutions to address the identified issues.
% Drawing upon our reflections, we propose possible directions for future exploration.



\paragraph{Deformation stability.} 
When the motion amplitude of the character in the driving GIF is too large, excessive deformation may occur in the glyph, especially for fonts with only a few control points.
Severe deformation can result in distorted glyphs with discontinued outlines and low legibility.
This may be addressed by expanding the number of control points along the predefined glyph outline~\cite{iluz2023word}, using the triangulated mesh representation of glyphs~\cite{desbrun1999implicit}, \rev{and introducing global penalty in the loss function}.
\rev{In addition, as discussed in \autoref{sec:generalizability}, the automatic pipeline may fail for over-complicated GIF styles or typefaces, which requires more generalized models in motion trajectory extraction.}  
% First, When the deformation is too large, the similarity between the motion structure of the source motion and the glyph is low, or the model is not accurate enough in detecting key points.
% It may lead to excessive deformation, confusing relative positions of control points, sharp corners of the glyph, distortion, and other unsmooth conditions. 

% To overcome this problem, we can try to apply algorithms that expand control points for glyphs with fewer control points to generate smoother lines during generation time, while at refinement time, reducing the set of control points accordingly for glyphs with more control points. 
% This provides the right number of control points for interactive user adjustment. Additionally, optimization can be done in the form of triangular face slices.

\paragraph{\rev{Motion semantic perseverance.}} 
\rev{
As with other cross-domain motion transfer problems, when generated result may not preserve the original semantics, as text generally lacks comparable internal structures with GIFs, where the length of a text strongly influences the success.
With our approach, a ``goodbye'' may crawl like a snake but hardly a giraffe swinging its neck.
In addition, our objective function prioritizes the overall deformation of the exterior and may neglect the local and independent deformation of the interior.
For instance, the generated result may learn the waving gesture but miss the delicate eye blinking.}


\paragraph{\rev{Animation expressiveness.}}
We leverage motion transfer on text control points to deform the shape of text elements and mimic the general animated effects in the driving GIF.
However, in addition to learning the shape-deformation patterns, kinetic typography also concerns other properties~\cite{xie2023emordle}.
Future works may explore incorporating visual properties like colors and designing an integrated environment for a more flexible authoring experience.
% Compared to images, text generally lacks complex internal structure, which may not be as expressive in animation.
% As such, when transferring motion from a character to text, the full semantic information may be lost.
% Another impact is that it may cause our generated content to prioritize the overall deformation of the exterior while not giving enough attention to the local and independent deformation of the interior. As a result, there may be a loss of semantic information for both appearance and motion.
% For semantic enhancement, one possible solution is to further incorporate information from the input text into the glyph, expressing semantics with both the action and appearance.
% Additionally, adding color-changing channels over time may also be helpful.

% \paragraph{Utilization of interaction data.}
% \rev{Wakey-Wakey provided users with interactive modules for fine-tuning, where they could directly drag the control points.
% However, we did not fully exploit the value of user interaction logs. 
% First, the corrected results can be further used for model training and algorithm improvement.}
% Second, a recommendation model for interactive operations can be trained with historical operation data to improve users' experience.


% The effectiveness of a motion transfer technique can be evaluated based on several criteria. Here are some possible factors to consider:
    % Reduction of Motion: The primary goal of motion transfer techniques is to reduce the amount of motion transfer between two objects or surfaces. The effectiveness of a motion transfer technique can be evaluated by measuring the amount of motion transferred before and after the technique is applied.
    % Consistency of Results: The effectiveness of a motion transfer technique can be evaluated by assessing whether it consistently reduces motion transfer across different conditions, such as varying temperatures or loads.
    % Durability: The effectiveness of a motion transfer technique can be evaluated by measuring its durability over time, particularly under conditions that would cause traditional techniques to fail, such as high loads or extreme temperatures.
    % Compatibility with Other Materials: The effectiveness of a motion transfer technique can be evaluated by assessing its compatibility with other materials or coatings that may need to be applied to the same surface.
    % Ease of Application: The effectiveness of a motion transfer technique can also be evaluated based on how easy it is to apply and integrate into existing systems, particularly in industrial or manufacturing applications.

\subsection{Future Work}
% Here we list several promising directions for future works.

% \textit{Broaden the controlled parameter space for kinetic typography.}
% In recent years we have witnessed the emergence of \textit{variable fonts}, whose parameterized properties like weight and italic extent can be accessed within a continuous range.
% Our approach leverages the pre-existed control points to perform motion transfer.
% Apart from directly transforming the text outline, future works may search in the design space of variable fonts in the matching stage and produce more legible results.

% \textit{Generalize motion transfer for vector graphics.}
\rev{This work demonstrates a novice-friendly approach to creating text animation through cross-domain motion transfer.
While we focus on texts, it is also interesting to explore arbitrary anthropomorphized shapes, such as the dancing mushrooms in Disney's \textit{Fantasia}~\cite{fantasia} or sketches of monsters~\cite{su2018live,smith2023tog}.
Unlike human postures constrained by bones and flesh, motion graphics enjoy higher flexibility for exaggerating effects, which share similarities with text.
We note that the outline optimization should shift the focus from maintaining text legibility to shape semantics. 
Recent advances in large language-vision models like Stable Diffusion~\cite{rombach2022high} may help to regularize undesired artifacts.
In addition, animated data storytelling (\eg,~\cite{wang2021infomotion,Shu21Gif}) remains an exciting avenue for integrating expressive motions on visual marks.
As illustrated in the case of an animated word cloud (see \autoref{fig: wccase}), the positions of individual visual marks in a visualization can be regarded as the control points of a text.
Using our approach, it is possible to transfer motion into visualizations, which may extend existing visual vocabularies and further facilitate the comprehension of abstract data and the expression of emotions~\cite{lan21kineticharts, xie2023emordle}. }


\section{Conclusion}
In this study, we explore the opportunity to create expressive kinetic typography based on a driving GIF with character motions.
Based on the unique characteristics of text, we propose a framework that adapts existing motion transfer models to the vector domain.
Specifically, we animate text based on their control points predefined in font specification.
To mimic the motion while maintaining fair legibility, the by-frame positions of each control point are regularized by the extracted motion key point in the driving GIF and neighboring control points.
We also introduce an interaction module that allows human-computer collaboration, where humans can steer the intermediate results and guide the generation of kinetic typography.
Based on the framework, we developed a mixed-initiative authoring tool and a browser widget featuring automatic generation.
A questionnaire study (N=33) initially validated the effectiveness of our approach.
Participants generally recognized that the results were animated in a desirable manner, both aesthetically pleasing and semantically resonant.
Moreover, we evaluated the novel transfer-based kinetic typography tools by organizing a workshop (N=20) with both general users and professional designers.
People were positive about the interactive system and showed interest in employing our tools for various scenarios.


%%
%% The acknowledgments section is defined using the "acks" environment
%% (and NOT an unnumbered section). This ensures the proper
%% identification of the section in the article metadata, and the
%% consistent spelling of the heading.

\begin{acks}
This research was supported by the Natural Science Foundation of China (NSFC No.62202105), Shanghai Municipal Science and Technology (No. 21ZR1403300 and No. 21YF1402900), and Hong Kong Research Grants Council General Research Fund 16210722.
We thank the anonymous reviewers, participants in the user studies, Zhan Wang, Ziyue Lin, Dr. Xinhuan Shu, Dr. Jiaxiong Hu, and Prof. Zhenjie Zhao for valuable feedback.
\end{acks}

%%
%% The next two lines define the bibliography style to be used, and
%% the bibliography file.
\bibliographystyle{ACM-Reference-Format}
\bibliography{reference}

\chapter{Supplementary Material}
\label{appendix}

In this appendix, we present supplementary material for the techniques and
experiments presented in the main text.

\section{Baseline Results and Analysis for Informed Sampler}
\label{appendix:chap3}

Here, we give an in-depth
performance analysis of the various samplers and the effect of their
hyperparameters. We choose hyperparameters with the lowest PSRF value
after $10k$ iterations, for each sampler individually. If the
differences between PSRF are not significantly different among
multiple values, we choose the one that has the highest acceptance
rate.

\subsection{Experiment: Estimating Camera Extrinsics}
\label{appendix:chap3:room}

\subsubsection{Parameter Selection}
\paragraph{Metropolis Hastings (\MH)}

Figure~\ref{fig:exp1_MH} shows the median acceptance rates and PSRF
values corresponding to various proposal standard deviations of plain
\MH~sampling. Mixing gets better and the acceptance rate gets worse as
the standard deviation increases. The value $0.3$ is selected standard
deviation for this sampler.

\paragraph{Metropolis Hastings Within Gibbs (\MHWG)}

As mentioned in Section~\ref{sec:room}, the \MHWG~sampler with one-dimensional
updates did not converge for any value of proposal standard deviation.
This problem has high correlation of the camera parameters and is of
multi-modal nature, which this sampler has problems with.

\paragraph{Parallel Tempering (\PT)}

For \PT~sampling, we took the best performing \MH~sampler and used
different temperature chains to improve the mixing of the
sampler. Figure~\ref{fig:exp1_PT} shows the results corresponding to
different combination of temperature levels. The sampler with
temperature levels of $[1,3,27]$ performed best in terms of both
mixing and acceptance rate.

\paragraph{Effect of Mixture Coefficient in Informed Sampling (\MIXLMH)}

Figure~\ref{fig:exp1_alpha} shows the effect of mixture
coefficient ($\alpha$) on the informed sampling
\MIXLMH. Since there is no significant different in PSRF values for
$0 \le \alpha \le 0.7$, we chose $0.7$ due to its high acceptance
rate.


% \end{multicols}

\begin{figure}[h]
\centering
  \subfigure[MH]{%
    \includegraphics[width=.48\textwidth]{figures/supplementary/camPose_MH.pdf} \label{fig:exp1_MH}
  }
  \subfigure[PT]{%
    \includegraphics[width=.48\textwidth]{figures/supplementary/camPose_PT.pdf} \label{fig:exp1_PT}
  }
\\
  \subfigure[INF-MH]{%
    \includegraphics[width=.48\textwidth]{figures/supplementary/camPose_alpha.pdf} \label{fig:exp1_alpha}
  }
  \mycaption{Results of the `Estimating Camera Extrinsics' experiment}{PRSFs and Acceptance rates corresponding to (a) various standard deviations of \MH, (b) various temperature level combinations of \PT sampling and (c) various mixture coefficients of \MIXLMH sampling.}
\end{figure}



\begin{figure}[!t]
\centering
  \subfigure[\MH]{%
    \includegraphics[width=.48\textwidth]{figures/supplementary/occlusionExp_MH.pdf} \label{fig:exp2_MH}
  }
  \subfigure[\BMHWG]{%
    \includegraphics[width=.48\textwidth]{figures/supplementary/occlusionExp_BMHWG.pdf} \label{fig:exp2_BMHWG}
  }
\\
  \subfigure[\MHWG]{%
    \includegraphics[width=.48\textwidth]{figures/supplementary/occlusionExp_MHWG.pdf} \label{fig:exp2_MHWG}
  }
  \subfigure[\PT]{%
    \includegraphics[width=.48\textwidth]{figures/supplementary/occlusionExp_PT.pdf} \label{fig:exp2_PT}
  }
\\
  \subfigure[\INFBMHWG]{%
    \includegraphics[width=.5\textwidth]{figures/supplementary/occlusionExp_alpha.pdf} \label{fig:exp2_alpha}
  }
  \mycaption{Results of the `Occluding Tiles' experiment}{PRSF and
    Acceptance rates corresponding to various standard deviations of
    (a) \MH, (b) \BMHWG, (c) \MHWG, (d) various temperature level
    combinations of \PT~sampling and; (e) various mixture coefficients
    of our informed \INFBMHWG sampling.}
\end{figure}

%\onecolumn\newpage\twocolumn
\subsection{Experiment: Occluding Tiles}
\label{appendix:chap3:tiles}

\subsubsection{Parameter Selection}

\paragraph{Metropolis Hastings (\MH)}

Figure~\ref{fig:exp2_MH} shows the results of
\MH~sampling. Results show the poor convergence for all proposal
standard deviations and rapid decrease of AR with increasing standard
deviation. This is due to the high-dimensional nature of
the problem. We selected a standard deviation of $1.1$.

\paragraph{Blocked Metropolis Hastings Within Gibbs (\BMHWG)}

The results of \BMHWG are shown in Figure~\ref{fig:exp2_BMHWG}. In
this sampler we update only one block of tile variables (of dimension
four) in each sampling step. Results show much better performance
compared to plain \MH. The optimal proposal standard deviation for
this sampler is $0.7$.

\paragraph{Metropolis Hastings Within Gibbs (\MHWG)}

Figure~\ref{fig:exp2_MHWG} shows the result of \MHWG sampling. This
sampler is better than \BMHWG and converges much more quickly. Here
a standard deviation of $0.9$ is found to be best.

\paragraph{Parallel Tempering (\PT)}

Figure~\ref{fig:exp2_PT} shows the results of \PT sampling with various
temperature combinations. Results show no improvement in AR from plain
\MH sampling and again $[1,3,27]$ temperature levels are found to be optimal.

\paragraph{Effect of Mixture Coefficient in Informed Sampling (\INFBMHWG)}

Figure~\ref{fig:exp2_alpha} shows the effect of mixture
coefficient ($\alpha$) on the blocked informed sampling
\INFBMHWG. Since there is no significant different in PSRF values for
$0 \le \alpha \le 0.8$, we chose $0.8$ due to its high acceptance
rate.



\subsection{Experiment: Estimating Body Shape}
\label{appendix:chap3:body}

\subsubsection{Parameter Selection}
\paragraph{Metropolis Hastings (\MH)}

Figure~\ref{fig:exp3_MH} shows the result of \MH~sampling with various
proposal standard deviations. The value of $0.1$ is found to be
best.

\paragraph{Metropolis Hastings Within Gibbs (\MHWG)}

For \MHWG sampling we select $0.3$ proposal standard
deviation. Results are shown in Fig.~\ref{fig:exp3_MHWG}.


\paragraph{Parallel Tempering (\PT)}

As before, results in Fig.~\ref{fig:exp3_PT}, the temperature levels
were selected to be $[1,3,27]$ due its slightly higher AR.

\paragraph{Effect of Mixture Coefficient in Informed Sampling (\MIXLMH)}

Figure~\ref{fig:exp3_alpha} shows the effect of $\alpha$ on PSRF and
AR. Since there is no significant differences in PSRF values for $0 \le
\alpha \le 0.8$, we choose $0.8$.


\begin{figure}[t]
\centering
  \subfigure[\MH]{%
    \includegraphics[width=.48\textwidth]{figures/supplementary/bodyShape_MH.pdf} \label{fig:exp3_MH}
  }
  \subfigure[\MHWG]{%
    \includegraphics[width=.48\textwidth]{figures/supplementary/bodyShape_MHWG.pdf} \label{fig:exp3_MHWG}
  }
\\
  \subfigure[\PT]{%
    \includegraphics[width=.48\textwidth]{figures/supplementary/bodyShape_PT.pdf} \label{fig:exp3_PT}
  }
  \subfigure[\MIXLMH]{%
    \includegraphics[width=.48\textwidth]{figures/supplementary/bodyShape_alpha.pdf} \label{fig:exp3_alpha}
  }
\\
  \mycaption{Results of the `Body Shape Estimation' experiment}{PRSFs and
    Acceptance rates corresponding to various standard deviations of
    (a) \MH, (b) \MHWG; (c) various temperature level combinations
    of \PT sampling and; (d) various mixture coefficients of the
    informed \MIXLMH sampling.}
\end{figure}


\subsection{Results Overview}
Figure~\ref{fig:exp_summary} shows the summary results of the all the three
experimental studies related to informed sampler.
\begin{figure*}[h!]
\centering
  \subfigure[Results for: Estimating Camera Extrinsics]{%
    \includegraphics[width=0.9\textwidth]{figures/supplementary/camPose_ALL.pdf} \label{fig:exp1_all}
  }
  \subfigure[Results for: Occluding Tiles]{%
    \includegraphics[width=0.9\textwidth]{figures/supplementary/occlusionExp_ALL.pdf} \label{fig:exp2_all}
  }
  \subfigure[Results for: Estimating Body Shape]{%
    \includegraphics[width=0.9\textwidth]{figures/supplementary/bodyShape_ALL.pdf} \label{fig:exp3_all}
  }
  \label{fig:exp_summary}
  \mycaption{Summary of the statistics for the three experiments}{Shown are
    for several baseline methods and the informed samplers the
    acceptance rates (left), PSRFs (middle), and RMSE values
    (right). All results are median results over multiple test
    examples.}
\end{figure*}

\subsection{Additional Qualitative Results}

\subsubsection{Occluding Tiles}
In Figure~\ref{fig:exp2_visual_more} more qualitative results of the
occluding tiles experiment are shown. The informed sampling approach
(\INFBMHWG) is better than the best baseline (\MHWG). This still is a
very challenging problem since the parameters for occluded tiles are
flat over a large region. Some of the posterior variance of the
occluded tiles is already captured by the informed sampler.

\begin{figure*}[h!]
\begin{center}
\centerline{\includegraphics[width=0.95\textwidth]{figures/supplementary/occlusionExp_Visual.pdf}}
\mycaption{Additional qualitative results of the occluding tiles experiment}
  {From left to right: (a)
  Given image, (b) Ground truth tiles, (c) OpenCV heuristic and most probable estimates
  from 5000 samples obtained by (d) MHWG sampler (best baseline) and
  (e) our INF-BMHWG sampler. (f) Posterior expectation of the tiles
  boundaries obtained by INF-BMHWG sampling (First 2000 samples are
  discarded as burn-in).}
\label{fig:exp2_visual_more}
\end{center}
\end{figure*}

\subsubsection{Body Shape}
Figure~\ref{fig:exp3_bodyMeshes} shows some more results of 3D mesh
reconstruction using posterior samples obtained by our informed
sampling \MIXLMH.

\begin{figure*}[t]
\begin{center}
\centerline{\includegraphics[width=0.75\textwidth]{figures/supplementary/bodyMeshResults.pdf}}
\mycaption{Qualitative results for the body shape experiment}
  {Shown is the 3D mesh reconstruction results with first 1000 samples obtained
  using the \MIXLMH informed sampling method. (blue indicates small
  values and red indicates high values)}
\label{fig:exp3_bodyMeshes}
\end{center}
\end{figure*}

\clearpage



\section{Additional Results on the Face Problem with CMP}

Figure~\ref{fig:shading-qualitative-multiple-subjects-supp} shows inference results for reflectance maps, normal maps and lights for randomly chosen test images, and Fig.~\ref{fig:shading-qualitative-same-subject-supp} shows reflectance estimation results on multiple images of the same subject produced under different illumination conditions. CMP is able to produce estimates that are closer to the groundtruth across different subjects and illumination conditions.

\begin{figure*}[h]
  \begin{center}
  \centerline{\includegraphics[width=1.0\columnwidth]{figures/face_cmp_visual_results_supp.pdf}}
  \vspace{-1.2cm}
  \end{center}
	\mycaption{A visual comparison of inference results}{(a)~Observed images. (b)~Inferred reflectance maps. \textit{GT} is the photometric stereo groundtruth, \textit{BU} is the Biswas \etal (2009) reflectance estimate and \textit{Forest} is the consensus prediction. (c)~The variance of the inferred reflectance estimate produced by \MTD (normalized across rows).(d)~Visualization of inferred light directions. (e)~Inferred normal maps.}
	\label{fig:shading-qualitative-multiple-subjects-supp}
\end{figure*}


\begin{figure*}[h]
	\centering
	\setlength\fboxsep{0.2mm}
	\setlength\fboxrule{0pt}
	\begin{tikzpicture}

		\matrix at (0, 0) [matrix of nodes, nodes={anchor=east}, column sep=-0.05cm, row sep=-0.2cm]
		{
			\fbox{\includegraphics[width=1cm]{figures/sample_3_4_X.png}} &
			\fbox{\includegraphics[width=1cm]{figures/sample_3_4_GT.png}} &
			\fbox{\includegraphics[width=1cm]{figures/sample_3_4_BISWAS.png}}  &
			\fbox{\includegraphics[width=1cm]{figures/sample_3_4_VMP.png}}  &
			\fbox{\includegraphics[width=1cm]{figures/sample_3_4_FOREST.png}}  &
			\fbox{\includegraphics[width=1cm]{figures/sample_3_4_CMP.png}}  &
			\fbox{\includegraphics[width=1cm]{figures/sample_3_4_CMPVAR.png}}
			 \\

			\fbox{\includegraphics[width=1cm]{figures/sample_3_5_X.png}} &
			\fbox{\includegraphics[width=1cm]{figures/sample_3_5_GT.png}} &
			\fbox{\includegraphics[width=1cm]{figures/sample_3_5_BISWAS.png}}  &
			\fbox{\includegraphics[width=1cm]{figures/sample_3_5_VMP.png}}  &
			\fbox{\includegraphics[width=1cm]{figures/sample_3_5_FOREST.png}}  &
			\fbox{\includegraphics[width=1cm]{figures/sample_3_5_CMP.png}}  &
			\fbox{\includegraphics[width=1cm]{figures/sample_3_5_CMPVAR.png}}
			 \\

			\fbox{\includegraphics[width=1cm]{figures/sample_3_6_X.png}} &
			\fbox{\includegraphics[width=1cm]{figures/sample_3_6_GT.png}} &
			\fbox{\includegraphics[width=1cm]{figures/sample_3_6_BISWAS.png}}  &
			\fbox{\includegraphics[width=1cm]{figures/sample_3_6_VMP.png}}  &
			\fbox{\includegraphics[width=1cm]{figures/sample_3_6_FOREST.png}}  &
			\fbox{\includegraphics[width=1cm]{figures/sample_3_6_CMP.png}}  &
			\fbox{\includegraphics[width=1cm]{figures/sample_3_6_CMPVAR.png}}
			 \\
	     };

       \node at (-3.85, -2.0) {\small Observed};
       \node at (-2.55, -2.0) {\small `GT'};
       \node at (-1.27, -2.0) {\small BU};
       \node at (0.0, -2.0) {\small MP};
       \node at (1.27, -2.0) {\small Forest};
       \node at (2.55, -2.0) {\small \textbf{CMP}};
       \node at (3.85, -2.0) {\small Variance};

	\end{tikzpicture}
	\mycaption{Robustness to varying illumination}{Reflectance estimation on a subject images with varying illumination. Left to right: observed image, photometric stereo estimate (GT)
  which is used as a proxy for groundtruth, bottom-up estimate of \cite{Biswas2009}, VMP result, consensus forest estimate, CMP mean, and CMP variance.}
	\label{fig:shading-qualitative-same-subject-supp}
\end{figure*}

\clearpage

\section{Additional Material for Learning Sparse High Dimensional Filters}
\label{sec:appendix-bnn}

This part of supplementary material contains a more detailed overview of the permutohedral
lattice convolution in Section~\ref{sec:permconv}, more experiments in
Section~\ref{sec:addexps} and additional results with protocols for
the experiments presented in Chapter~\ref{chap:bnn} in Section~\ref{sec:addresults}.

\vspace{-0.2cm}
\subsection{General Permutohedral Convolutions}
\label{sec:permconv}

A core technical contribution of this work is the generalization of the Gaussian permutohedral lattice
convolution proposed in~\cite{adams2010fast} to the full non-separable case with the
ability to perform back-propagation. Although, conceptually, there are minor
differences between Gaussian and general parameterized filters, there are non-trivial practical
differences in terms of the algorithmic implementation. The Gauss filters belong to
the separable class and can thus be decomposed into multiple
sequential one dimensional convolutions. We are interested in the general filter
convolutions, which can not be decomposed. Thus, performing a general permutohedral
convolution at a lattice point requires the computation of the inner product with the
neighboring elements in all the directions in the high-dimensional space.

Here, we give more details of the implementation differences of separable
and non-separable filters. In the following, we will explain the scalar case first.
Recall, that the forward pass of general permutohedral convolution
involves 3 steps: \textit{splatting}, \textit{convolving} and \textit{slicing}.
We follow the same splatting and slicing strategies as in~\cite{adams2010fast}
since these operations do not depend on the filter kernel. The main difference
between our work and the existing implementation of~\cite{adams2010fast} is
the way that the convolution operation is executed. This proceeds by constructing
a \emph{blur neighbor} matrix $K$ that stores for every lattice point all
values of the lattice neighbors that are needed to compute the filter output.

\begin{figure}[t!]
  \centering
    \includegraphics[width=0.6\columnwidth]{figures/supplementary/lattice_construction}
  \mycaption{Illustration of 1D permutohedral lattice construction}
  {A $4\times 4$ $(x,y)$ grid lattice is projected onto the plane defined by the normal
  vector $(1,1)^{\top}$. This grid has $s+1=4$ and $d=2$ $(s+1)^{d}=4^2=16$ elements.
  In the projection, all points of the same color are projected onto the same points in the plane.
  The number of elements of the projected lattice is $t=(s+1)^d-s^d=4^2-3^2=7$, that is
  the $(4\times 4)$ grid minus the size of lattice that is $1$ smaller at each size, in this
  case a $(3\times 3)$ lattice (the upper right $(3\times 3)$ elements).
  }
\label{fig:latticeconstruction}
\end{figure}

The blur neighbor matrix is constructed by traversing through all the populated
lattice points and their neighboring elements.
% For efficiency, we do this matrix construction recursively with shared computations
% since $n^{th}$ neighbourhood elements are $1^{st}$ neighborhood elements of $n-1^{th}$ neighbourhood elements. \pg{do not understand}
This is done recursively to share computations. For any lattice point, the neighbors that are
$n$ hops away are the direct neighbors of the points that are $n-1$ hops away.
The size of a $d$ dimensional spatial filter with width $s+1$ is $(s+1)^{d}$ (\eg, a
$3\times 3$ filter, $s=2$ in $d=2$ has $3^2=9$ elements) and this size grows
exponentially in the number of dimensions $d$. The permutohedral lattice is constructed by
projecting a regular grid onto the plane spanned by the $d$ dimensional normal vector ${(1,\ldots,1)}^{\top}$. See
Fig.~\ref{fig:latticeconstruction} for an illustration of the 1D lattice construction.
Many corners of a grid filter are projected onto the same point, in total $t = {(s+1)}^{d} -
s^{d}$ elements remain in the permutohedral filter with $s$ neighborhood in $d-1$ dimensions.
If the lattice has $m$ populated elements, the
matrix $K$ has size $t\times m$. Note that, since the input signal is typically
sparse, only a few lattice corners are being populated in the \textit{slicing} step.
We use a hash-table to keep track of these points and traverse only through
the populated lattice points for this neighborhood matrix construction.

Once the blur neighbor matrix $K$ is constructed, we can perform the convolution
by the matrix vector multiplication
\begin{equation}
\ell' = BK,
\label{eq:conv}
\end{equation}
where $B$ is the $1 \times t$ filter kernel (whose values we will learn) and $\ell'\in\mathbb{R}^{1\times m}$
is the result of the filtering at the $m$ lattice points. In practice, we found that the
matrix $K$ is sometimes too large to fit into GPU memory and we divided the matrix $K$
into smaller pieces to compute Eq.~\ref{eq:conv} sequentially.

In the general multi-dimensional case, the signal $\ell$ is of $c$ dimensions. Then
the kernel $B$ is of size $c \times t$ and $K$ stores the $c$ dimensional vectors
accordingly. When the input and output points are different, we slice only the
input points and splat only at the output points.


\subsection{Additional Experiments}
\label{sec:addexps}
In this section, we discuss more use-cases for the learned bilateral filters, one
use-case of BNNs and two single filter applications for image and 3D mesh denoising.

\subsubsection{Recognition of subsampled MNIST}\label{sec:app_mnist}

One of the strengths of the proposed filter convolution is that it does not
require the input to lie on a regular grid. The only requirement is to define a distance
between features of the input signal.
We highlight this feature with the following experiment using the
classical MNIST ten class classification problem~\cite{lecun1998mnist}. We sample a
sparse set of $N$ points $(x,y)\in [0,1]\times [0,1]$
uniformly at random in the input image, use their interpolated values
as signal and the \emph{continuous} $(x,y)$ positions as features. This mimics
sub-sampling of a high-dimensional signal. To compare against a spatial convolution,
we interpolate the sparse set of values at the grid positions.

We take a reference implementation of LeNet~\cite{lecun1998gradient} that
is part of the Caffe project~\cite{jia2014caffe} and compare it
against the same architecture but replacing the first convolutional
layer with a bilateral convolution layer (BCL). The filter size
and numbers are adjusted to get a comparable number of parameters
($5\times 5$ for LeNet, $2$-neighborhood for BCL).

The results are shown in Table~\ref{tab:all-results}. We see that training
on the original MNIST data (column Original, LeNet vs. BNN) leads to a slight
decrease in performance of the BNN (99.03\%) compared to LeNet
(99.19\%). The BNN can be trained and evaluated on sparse
signals, and we resample the image as described above for $N=$ 100\%, 60\% and
20\% of the total number of pixels. The methods are also evaluated
on test images that are subsampled in the same way. Note that we can
train and test with different subsampling rates. We introduce an additional
bilinear interpolation layer for the LeNet architecture to train on the same
data. In essence, both models perform a spatial interpolation and thus we
expect them to yield a similar classification accuracy. Once the data is of
higher dimensions, the permutohedral convolution will be faster due to hashing
the sparse input points, as well as less memory demanding in comparison to
naive application of a spatial convolution with interpolated values.

\begin{table}[t]
  \begin{center}
    \footnotesize
    \centering
    \begin{tabular}[t]{lllll}
      \toprule
              &     & \multicolumn{3}{c}{Test Subsampling} \\
       Method  & Original & 100\% & 60\% & 20\%\\
      \midrule
       LeNet &  \textbf{0.9919} & 0.9660 & 0.9348 & \textbf{0.6434} \\
       BNN &  0.9903 & \textbf{0.9844} & \textbf{0.9534} & 0.5767 \\
      \hline
       LeNet 100\% & 0.9856 & 0.9809 & 0.9678 & \textbf{0.7386} \\
       BNN 100\% & \textbf{0.9900} & \textbf{0.9863} & \textbf{0.9699} & 0.6910 \\
      \hline
       LeNet 60\% & 0.9848 & 0.9821 & 0.9740 & 0.8151 \\
       BNN 60\% & \textbf{0.9885} & \textbf{0.9864} & \textbf{0.9771} & \textbf{0.8214}\\
      \hline
       LeNet 20\% & \textbf{0.9763} & \textbf{0.9754} & 0.9695 & 0.8928 \\
       BNN 20\% & 0.9728 & 0.9735 & \textbf{0.9701} & \textbf{0.9042}\\
      \bottomrule
    \end{tabular}
  \end{center}
\vspace{-.2cm}
\caption{Classification accuracy on MNIST. We compare the
    LeNet~\cite{lecun1998gradient} implementation that is part of
    Caffe~\cite{jia2014caffe} to the network with the first layer
    replaced by a bilateral convolution layer (BCL). Both are trained
    on the original image resolution (first two rows). Three more BNN
    and CNN models are trained with randomly subsampled images (100\%,
    60\% and 20\% of the pixels). An additional bilinear interpolation
    layer samples the input signal on a spatial grid for the CNN model.
  }
  \label{tab:all-results}
\vspace{-.5cm}
\end{table}

\subsubsection{Image Denoising}

The main application that inspired the development of the bilateral
filtering operation is image denoising~\cite{aurich1995non}, there
using a single Gaussian kernel. Our development allows to learn this
kernel function from data and we explore how to improve using a \emph{single}
but more general bilateral filter.

We use the Berkeley segmentation dataset
(BSDS500)~\cite{arbelaezi2011bsds500} as a test bed. The color
images in the dataset are converted to gray-scale,
and corrupted with Gaussian noise with a standard deviation of
$\frac {25} {255}$.

We compare the performance of four different filter models on a
denoising task.
The first baseline model (`Spatial' in Table \ref{tab:denoising}, $25$
weights) uses a single spatial filter with a kernel size of
$5$ and predicts the scalar gray-scale value at the center pixel. The next model
(`Gauss Bilateral') applies a bilateral \emph{Gaussian}
filter to the noisy input, using position and intensity features $\f=(x,y,v)^\top$.
The third setup (`Learned Bilateral', $65$ weights)
takes a Gauss kernel as initialization and
fits all filter weights on the train set to minimize the
mean squared error with respect to the clean images.
We run a combination
of spatial and permutohedral convolutions on spatial and bilateral
features (`Spatial + Bilateral (Learned)') to check for a complementary
performance of the two convolutions.

\label{sec:exp:denoising}
\begin{table}[!h]
\begin{center}
  \footnotesize
  \begin{tabular}[t]{lr}
    \toprule
    Method & PSNR \\
    \midrule
    Noisy Input & $20.17$ \\
    Spatial & $26.27$ \\
    Gauss Bilateral & $26.51$ \\
    Learned Bilateral & $26.58$ \\
    Spatial + Bilateral (Learned) & \textbf{$26.65$} \\
    \bottomrule
  \end{tabular}
\end{center}
\vspace{-0.5em}
\caption{PSNR results of a denoising task using the BSDS500
  dataset~\cite{arbelaezi2011bsds500}}
\vspace{-0.5em}
\label{tab:denoising}
\end{table}
\vspace{-0.2em}

The PSNR scores evaluated on full images of the test set are
shown in Table \ref{tab:denoising}. We find that an untrained bilateral
filter already performs better than a trained spatial convolution
($26.27$ to $26.51$). A learned convolution further improve the
performance slightly. We chose this simple one-kernel setup to
validate an advantage of the generalized bilateral filter. A competitive
denoising system would employ RGB color information and also
needs to be properly adjusted in network size. Multi-layer perceptrons
have obtained state-of-the-art denoising results~\cite{burger12cvpr}
and the permutohedral lattice layer can readily be used in such an
architecture, which is intended future work.

\subsection{Additional results}
\label{sec:addresults}

This section contains more qualitative results for the experiments presented in Chapter~\ref{chap:bnn}.

\begin{figure*}[th!]
  \centering
    \includegraphics[width=\columnwidth,trim={5cm 2.5cm 5cm 4.5cm},clip]{figures/supplementary/lattice_viz.pdf}
    \vspace{-0.7cm}
  \mycaption{Visualization of the Permutohedral Lattice}
  {Sample lattice visualizations for different feature spaces. All pixels falling in the same simplex cell are shown with
  the same color. $(x,y)$ features correspond to image pixel positions, and $(r,g,b) \in [0,255]$ correspond
  to the red, green and blue color values.}
\label{fig:latticeviz}
\end{figure*}

\subsubsection{Lattice Visualization}

Figure~\ref{fig:latticeviz} shows sample lattice visualizations for different feature spaces.

\newcolumntype{L}[1]{>{\raggedright\let\newline\\\arraybackslash\hspace{0pt}}b{#1}}
\newcolumntype{C}[1]{>{\centering\let\newline\\\arraybackslash\hspace{0pt}}b{#1}}
\newcolumntype{R}[1]{>{\raggedleft\let\newline\\\arraybackslash\hspace{0pt}}b{#1}}

\subsubsection{Color Upsampling}\label{sec:color_upsampling}
\label{sec:col_upsample_extra}

Some images of the upsampling for the Pascal
VOC12 dataset are shown in Fig.~\ref{fig:Colour_upsample_visuals}. It is
especially the low level image details that are better preserved with
a learned bilateral filter compared to the Gaussian case.

\begin{figure*}[t!]
  \centering
    \subfigure{%
   \raisebox{2.0em}{
    \includegraphics[width=.06\columnwidth]{figures/supplementary/2007_004969.jpg}
   }
  }
  \subfigure{%
    \includegraphics[width=.17\columnwidth]{figures/supplementary/2007_004969_gray.pdf}
  }
  \subfigure{%
    \includegraphics[width=.17\columnwidth]{figures/supplementary/2007_004969_gt.pdf}
  }
  \subfigure{%
    \includegraphics[width=.17\columnwidth]{figures/supplementary/2007_004969_bicubic.pdf}
  }
  \subfigure{%
    \includegraphics[width=.17\columnwidth]{figures/supplementary/2007_004969_gauss.pdf}
  }
  \subfigure{%
    \includegraphics[width=.17\columnwidth]{figures/supplementary/2007_004969_learnt.pdf}
  }\\
    \subfigure{%
   \raisebox{2.0em}{
    \includegraphics[width=.06\columnwidth]{figures/supplementary/2007_003106.jpg}
   }
  }
  \subfigure{%
    \includegraphics[width=.17\columnwidth]{figures/supplementary/2007_003106_gray.pdf}
  }
  \subfigure{%
    \includegraphics[width=.17\columnwidth]{figures/supplementary/2007_003106_gt.pdf}
  }
  \subfigure{%
    \includegraphics[width=.17\columnwidth]{figures/supplementary/2007_003106_bicubic.pdf}
  }
  \subfigure{%
    \includegraphics[width=.17\columnwidth]{figures/supplementary/2007_003106_gauss.pdf}
  }
  \subfigure{%
    \includegraphics[width=.17\columnwidth]{figures/supplementary/2007_003106_learnt.pdf}
  }\\
  \setcounter{subfigure}{0}
  \small{
  \subfigure[Inp.]{%
  \raisebox{2.0em}{
    \includegraphics[width=.06\columnwidth]{figures/supplementary/2007_006837.jpg}
   }
  }
  \subfigure[Guidance]{%
    \includegraphics[width=.17\columnwidth]{figures/supplementary/2007_006837_gray.pdf}
  }
   \subfigure[GT]{%
    \includegraphics[width=.17\columnwidth]{figures/supplementary/2007_006837_gt.pdf}
  }
  \subfigure[Bicubic]{%
    \includegraphics[width=.17\columnwidth]{figures/supplementary/2007_006837_bicubic.pdf}
  }
  \subfigure[Gauss-BF]{%
    \includegraphics[width=.17\columnwidth]{figures/supplementary/2007_006837_gauss.pdf}
  }
  \subfigure[Learned-BF]{%
    \includegraphics[width=.17\columnwidth]{figures/supplementary/2007_006837_learnt.pdf}
  }
  }
  \vspace{-0.5cm}
  \mycaption{Color Upsampling}{Color $8\times$ upsampling results
  using different methods, from left to right, (a)~Low-resolution input color image (Inp.),
  (b)~Gray scale guidance image, (c)~Ground-truth color image; Upsampled color images with
  (d)~Bicubic interpolation, (e) Gauss bilateral upsampling and, (f)~Learned bilateral
  updampgling (best viewed on screen).}

\label{fig:Colour_upsample_visuals}
\end{figure*}

\subsubsection{Depth Upsampling}
\label{sec:depth_upsample_extra}

Figure~\ref{fig:depth_upsample_visuals} presents some more qualitative results comparing bicubic interpolation, Gauss
bilateral and learned bilateral upsampling on NYU depth dataset image~\cite{silberman2012indoor}.

\subsubsection{Character Recognition}\label{sec:app_character}

 Figure~\ref{fig:nnrecognition} shows the schematic of different layers
 of the network architecture for LeNet-7~\cite{lecun1998mnist}
 and DeepCNet(5, 50)~\cite{ciresan2012multi,graham2014spatially}. For the BNN variants, the first layer filters are replaced
 with learned bilateral filters and are learned end-to-end.

\subsubsection{Semantic Segmentation}\label{sec:app_semantic_segmentation}
\label{sec:semantic_bnn_extra}

Some more visual results for semantic segmentation are shown in Figure~\ref{fig:semantic_visuals}.
These include the underlying DeepLab CNN\cite{chen2014semantic} result (DeepLab),
the 2 step mean-field result with Gaussian edge potentials (+2stepMF-GaussCRF)
and also corresponding results with learned edge potentials (+2stepMF-LearnedCRF).
In general, we observe that mean-field in learned CRF leads to slightly dilated
classification regions in comparison to using Gaussian CRF thereby filling-in the
false negative pixels and also correcting some mis-classified regions.

\begin{figure*}[t!]
  \centering
    \subfigure{%
   \raisebox{2.0em}{
    \includegraphics[width=.06\columnwidth]{figures/supplementary/2bicubic}
   }
  }
  \subfigure{%
    \includegraphics[width=.17\columnwidth]{figures/supplementary/2given_image}
  }
  \subfigure{%
    \includegraphics[width=.17\columnwidth]{figures/supplementary/2ground_truth}
  }
  \subfigure{%
    \includegraphics[width=.17\columnwidth]{figures/supplementary/2bicubic}
  }
  \subfigure{%
    \includegraphics[width=.17\columnwidth]{figures/supplementary/2gauss}
  }
  \subfigure{%
    \includegraphics[width=.17\columnwidth]{figures/supplementary/2learnt}
  }\\
    \subfigure{%
   \raisebox{2.0em}{
    \includegraphics[width=.06\columnwidth]{figures/supplementary/32bicubic}
   }
  }
  \subfigure{%
    \includegraphics[width=.17\columnwidth]{figures/supplementary/32given_image}
  }
  \subfigure{%
    \includegraphics[width=.17\columnwidth]{figures/supplementary/32ground_truth}
  }
  \subfigure{%
    \includegraphics[width=.17\columnwidth]{figures/supplementary/32bicubic}
  }
  \subfigure{%
    \includegraphics[width=.17\columnwidth]{figures/supplementary/32gauss}
  }
  \subfigure{%
    \includegraphics[width=.17\columnwidth]{figures/supplementary/32learnt}
  }\\
  \setcounter{subfigure}{0}
  \small{
  \subfigure[Inp.]{%
  \raisebox{2.0em}{
    \includegraphics[width=.06\columnwidth]{figures/supplementary/41bicubic}
   }
  }
  \subfigure[Guidance]{%
    \includegraphics[width=.17\columnwidth]{figures/supplementary/41given_image}
  }
   \subfigure[GT]{%
    \includegraphics[width=.17\columnwidth]{figures/supplementary/41ground_truth}
  }
  \subfigure[Bicubic]{%
    \includegraphics[width=.17\columnwidth]{figures/supplementary/41bicubic}
  }
  \subfigure[Gauss-BF]{%
    \includegraphics[width=.17\columnwidth]{figures/supplementary/41gauss}
  }
  \subfigure[Learned-BF]{%
    \includegraphics[width=.17\columnwidth]{figures/supplementary/41learnt}
  }
  }
  \mycaption{Depth Upsampling}{Depth $8\times$ upsampling results
  using different upsampling strategies, from left to right,
  (a)~Low-resolution input depth image (Inp.),
  (b)~High-resolution guidance image, (c)~Ground-truth depth; Upsampled depth images with
  (d)~Bicubic interpolation, (e) Gauss bilateral upsampling and, (f)~Learned bilateral
  updampgling (best viewed on screen).}

\label{fig:depth_upsample_visuals}
\end{figure*}

\subsubsection{Material Segmentation}\label{sec:app_material_segmentation}
\label{sec:material_bnn_extra}

In Fig.~\ref{fig:material_visuals-app2}, we present visual results comparing 2 step
mean-field inference with Gaussian and learned pairwise CRF potentials. In
general, we observe that the pixels belonging to dominant classes in the
training data are being more accurately classified with learned CRF. This leads to
a significant improvements in overall pixel accuracy. This also results
in a slight decrease of the accuracy from less frequent class pixels thereby
slightly reducing the average class accuracy with learning. We attribute this
to the type of annotation that is available for this dataset, which is not
for the entire image but for some segments in the image. We have very few
images of the infrequent classes to combat this behaviour during training.

\subsubsection{Experiment Protocols}
\label{sec:protocols}

Table~\ref{tbl:parameters} shows experiment protocols of different experiments.

 \begin{figure*}[t!]
  \centering
  \subfigure[LeNet-7]{
    \includegraphics[width=0.7\columnwidth]{figures/supplementary/lenet_cnn_network}
    }\\
    \subfigure[DeepCNet]{
    \includegraphics[width=\columnwidth]{figures/supplementary/deepcnet_cnn_network}
    }
  \mycaption{CNNs for Character Recognition}
  {Schematic of (top) LeNet-7~\cite{lecun1998mnist} and (bottom) DeepCNet(5,50)~\cite{ciresan2012multi,graham2014spatially} architectures used in Assamese
  character recognition experiments.}
\label{fig:nnrecognition}
\end{figure*}

\definecolor{voc_1}{RGB}{0, 0, 0}
\definecolor{voc_2}{RGB}{128, 0, 0}
\definecolor{voc_3}{RGB}{0, 128, 0}
\definecolor{voc_4}{RGB}{128, 128, 0}
\definecolor{voc_5}{RGB}{0, 0, 128}
\definecolor{voc_6}{RGB}{128, 0, 128}
\definecolor{voc_7}{RGB}{0, 128, 128}
\definecolor{voc_8}{RGB}{128, 128, 128}
\definecolor{voc_9}{RGB}{64, 0, 0}
\definecolor{voc_10}{RGB}{192, 0, 0}
\definecolor{voc_11}{RGB}{64, 128, 0}
\definecolor{voc_12}{RGB}{192, 128, 0}
\definecolor{voc_13}{RGB}{64, 0, 128}
\definecolor{voc_14}{RGB}{192, 0, 128}
\definecolor{voc_15}{RGB}{64, 128, 128}
\definecolor{voc_16}{RGB}{192, 128, 128}
\definecolor{voc_17}{RGB}{0, 64, 0}
\definecolor{voc_18}{RGB}{128, 64, 0}
\definecolor{voc_19}{RGB}{0, 192, 0}
\definecolor{voc_20}{RGB}{128, 192, 0}
\definecolor{voc_21}{RGB}{0, 64, 128}
\definecolor{voc_22}{RGB}{128, 64, 128}

\begin{figure*}[t]
  \centering
  \small{
  \fcolorbox{white}{voc_1}{\rule{0pt}{6pt}\rule{6pt}{0pt}} Background~~
  \fcolorbox{white}{voc_2}{\rule{0pt}{6pt}\rule{6pt}{0pt}} Aeroplane~~
  \fcolorbox{white}{voc_3}{\rule{0pt}{6pt}\rule{6pt}{0pt}} Bicycle~~
  \fcolorbox{white}{voc_4}{\rule{0pt}{6pt}\rule{6pt}{0pt}} Bird~~
  \fcolorbox{white}{voc_5}{\rule{0pt}{6pt}\rule{6pt}{0pt}} Boat~~
  \fcolorbox{white}{voc_6}{\rule{0pt}{6pt}\rule{6pt}{0pt}} Bottle~~
  \fcolorbox{white}{voc_7}{\rule{0pt}{6pt}\rule{6pt}{0pt}} Bus~~
  \fcolorbox{white}{voc_8}{\rule{0pt}{6pt}\rule{6pt}{0pt}} Car~~ \\
  \fcolorbox{white}{voc_9}{\rule{0pt}{6pt}\rule{6pt}{0pt}} Cat~~
  \fcolorbox{white}{voc_10}{\rule{0pt}{6pt}\rule{6pt}{0pt}} Chair~~
  \fcolorbox{white}{voc_11}{\rule{0pt}{6pt}\rule{6pt}{0pt}} Cow~~
  \fcolorbox{white}{voc_12}{\rule{0pt}{6pt}\rule{6pt}{0pt}} Dining Table~~
  \fcolorbox{white}{voc_13}{\rule{0pt}{6pt}\rule{6pt}{0pt}} Dog~~
  \fcolorbox{white}{voc_14}{\rule{0pt}{6pt}\rule{6pt}{0pt}} Horse~~
  \fcolorbox{white}{voc_15}{\rule{0pt}{6pt}\rule{6pt}{0pt}} Motorbike~~
  \fcolorbox{white}{voc_16}{\rule{0pt}{6pt}\rule{6pt}{0pt}} Person~~ \\
  \fcolorbox{white}{voc_17}{\rule{0pt}{6pt}\rule{6pt}{0pt}} Potted Plant~~
  \fcolorbox{white}{voc_18}{\rule{0pt}{6pt}\rule{6pt}{0pt}} Sheep~~
  \fcolorbox{white}{voc_19}{\rule{0pt}{6pt}\rule{6pt}{0pt}} Sofa~~
  \fcolorbox{white}{voc_20}{\rule{0pt}{6pt}\rule{6pt}{0pt}} Train~~
  \fcolorbox{white}{voc_21}{\rule{0pt}{6pt}\rule{6pt}{0pt}} TV monitor~~ \\
  }
  \subfigure{%
    \includegraphics[width=.18\columnwidth]{figures/supplementary/2007_001423_given.jpg}
  }
  \subfigure{%
    \includegraphics[width=.18\columnwidth]{figures/supplementary/2007_001423_gt.png}
  }
  \subfigure{%
    \includegraphics[width=.18\columnwidth]{figures/supplementary/2007_001423_cnn.png}
  }
  \subfigure{%
    \includegraphics[width=.18\columnwidth]{figures/supplementary/2007_001423_gauss.png}
  }
  \subfigure{%
    \includegraphics[width=.18\columnwidth]{figures/supplementary/2007_001423_learnt.png}
  }\\
  \subfigure{%
    \includegraphics[width=.18\columnwidth]{figures/supplementary/2007_001430_given.jpg}
  }
  \subfigure{%
    \includegraphics[width=.18\columnwidth]{figures/supplementary/2007_001430_gt.png}
  }
  \subfigure{%
    \includegraphics[width=.18\columnwidth]{figures/supplementary/2007_001430_cnn.png}
  }
  \subfigure{%
    \includegraphics[width=.18\columnwidth]{figures/supplementary/2007_001430_gauss.png}
  }
  \subfigure{%
    \includegraphics[width=.18\columnwidth]{figures/supplementary/2007_001430_learnt.png}
  }\\
    \subfigure{%
    \includegraphics[width=.18\columnwidth]{figures/supplementary/2007_007996_given.jpg}
  }
  \subfigure{%
    \includegraphics[width=.18\columnwidth]{figures/supplementary/2007_007996_gt.png}
  }
  \subfigure{%
    \includegraphics[width=.18\columnwidth]{figures/supplementary/2007_007996_cnn.png}
  }
  \subfigure{%
    \includegraphics[width=.18\columnwidth]{figures/supplementary/2007_007996_gauss.png}
  }
  \subfigure{%
    \includegraphics[width=.18\columnwidth]{figures/supplementary/2007_007996_learnt.png}
  }\\
   \subfigure{%
    \includegraphics[width=.18\columnwidth]{figures/supplementary/2010_002682_given.jpg}
  }
  \subfigure{%
    \includegraphics[width=.18\columnwidth]{figures/supplementary/2010_002682_gt.png}
  }
  \subfigure{%
    \includegraphics[width=.18\columnwidth]{figures/supplementary/2010_002682_cnn.png}
  }
  \subfigure{%
    \includegraphics[width=.18\columnwidth]{figures/supplementary/2010_002682_gauss.png}
  }
  \subfigure{%
    \includegraphics[width=.18\columnwidth]{figures/supplementary/2010_002682_learnt.png}
  }\\
     \subfigure{%
    \includegraphics[width=.18\columnwidth]{figures/supplementary/2010_004789_given.jpg}
  }
  \subfigure{%
    \includegraphics[width=.18\columnwidth]{figures/supplementary/2010_004789_gt.png}
  }
  \subfigure{%
    \includegraphics[width=.18\columnwidth]{figures/supplementary/2010_004789_cnn.png}
  }
  \subfigure{%
    \includegraphics[width=.18\columnwidth]{figures/supplementary/2010_004789_gauss.png}
  }
  \subfigure{%
    \includegraphics[width=.18\columnwidth]{figures/supplementary/2010_004789_learnt.png}
  }\\
       \subfigure{%
    \includegraphics[width=.18\columnwidth]{figures/supplementary/2007_001311_given.jpg}
  }
  \subfigure{%
    \includegraphics[width=.18\columnwidth]{figures/supplementary/2007_001311_gt.png}
  }
  \subfigure{%
    \includegraphics[width=.18\columnwidth]{figures/supplementary/2007_001311_cnn.png}
  }
  \subfigure{%
    \includegraphics[width=.18\columnwidth]{figures/supplementary/2007_001311_gauss.png}
  }
  \subfigure{%
    \includegraphics[width=.18\columnwidth]{figures/supplementary/2007_001311_learnt.png}
  }\\
  \setcounter{subfigure}{0}
  \subfigure[Input]{%
    \includegraphics[width=.18\columnwidth]{figures/supplementary/2010_003531_given.jpg}
  }
  \subfigure[Ground Truth]{%
    \includegraphics[width=.18\columnwidth]{figures/supplementary/2010_003531_gt.png}
  }
  \subfigure[DeepLab]{%
    \includegraphics[width=.18\columnwidth]{figures/supplementary/2010_003531_cnn.png}
  }
  \subfigure[+GaussCRF]{%
    \includegraphics[width=.18\columnwidth]{figures/supplementary/2010_003531_gauss.png}
  }
  \subfigure[+LearnedCRF]{%
    \includegraphics[width=.18\columnwidth]{figures/supplementary/2010_003531_learnt.png}
  }
  \vspace{-0.3cm}
  \mycaption{Semantic Segmentation}{Example results of semantic segmentation.
  (c)~depicts the unary results before application of MF, (d)~after two steps of MF with Gaussian edge CRF potentials, (e)~after
  two steps of MF with learned edge CRF potentials.}
    \label{fig:semantic_visuals}
\end{figure*}


\definecolor{minc_1}{HTML}{771111}
\definecolor{minc_2}{HTML}{CAC690}
\definecolor{minc_3}{HTML}{EEEEEE}
\definecolor{minc_4}{HTML}{7C8FA6}
\definecolor{minc_5}{HTML}{597D31}
\definecolor{minc_6}{HTML}{104410}
\definecolor{minc_7}{HTML}{BB819C}
\definecolor{minc_8}{HTML}{D0CE48}
\definecolor{minc_9}{HTML}{622745}
\definecolor{minc_10}{HTML}{666666}
\definecolor{minc_11}{HTML}{D54A31}
\definecolor{minc_12}{HTML}{101044}
\definecolor{minc_13}{HTML}{444126}
\definecolor{minc_14}{HTML}{75D646}
\definecolor{minc_15}{HTML}{DD4348}
\definecolor{minc_16}{HTML}{5C8577}
\definecolor{minc_17}{HTML}{C78472}
\definecolor{minc_18}{HTML}{75D6D0}
\definecolor{minc_19}{HTML}{5B4586}
\definecolor{minc_20}{HTML}{C04393}
\definecolor{minc_21}{HTML}{D69948}
\definecolor{minc_22}{HTML}{7370D8}
\definecolor{minc_23}{HTML}{7A3622}
\definecolor{minc_24}{HTML}{000000}

\begin{figure*}[t]
  \centering
  \small{
  \fcolorbox{white}{minc_1}{\rule{0pt}{6pt}\rule{6pt}{0pt}} Brick~~
  \fcolorbox{white}{minc_2}{\rule{0pt}{6pt}\rule{6pt}{0pt}} Carpet~~
  \fcolorbox{white}{minc_3}{\rule{0pt}{6pt}\rule{6pt}{0pt}} Ceramic~~
  \fcolorbox{white}{minc_4}{\rule{0pt}{6pt}\rule{6pt}{0pt}} Fabric~~
  \fcolorbox{white}{minc_5}{\rule{0pt}{6pt}\rule{6pt}{0pt}} Foliage~~
  \fcolorbox{white}{minc_6}{\rule{0pt}{6pt}\rule{6pt}{0pt}} Food~~
  \fcolorbox{white}{minc_7}{\rule{0pt}{6pt}\rule{6pt}{0pt}} Glass~~
  \fcolorbox{white}{minc_8}{\rule{0pt}{6pt}\rule{6pt}{0pt}} Hair~~ \\
  \fcolorbox{white}{minc_9}{\rule{0pt}{6pt}\rule{6pt}{0pt}} Leather~~
  \fcolorbox{white}{minc_10}{\rule{0pt}{6pt}\rule{6pt}{0pt}} Metal~~
  \fcolorbox{white}{minc_11}{\rule{0pt}{6pt}\rule{6pt}{0pt}} Mirror~~
  \fcolorbox{white}{minc_12}{\rule{0pt}{6pt}\rule{6pt}{0pt}} Other~~
  \fcolorbox{white}{minc_13}{\rule{0pt}{6pt}\rule{6pt}{0pt}} Painted~~
  \fcolorbox{white}{minc_14}{\rule{0pt}{6pt}\rule{6pt}{0pt}} Paper~~
  \fcolorbox{white}{minc_15}{\rule{0pt}{6pt}\rule{6pt}{0pt}} Plastic~~\\
  \fcolorbox{white}{minc_16}{\rule{0pt}{6pt}\rule{6pt}{0pt}} Polished Stone~~
  \fcolorbox{white}{minc_17}{\rule{0pt}{6pt}\rule{6pt}{0pt}} Skin~~
  \fcolorbox{white}{minc_18}{\rule{0pt}{6pt}\rule{6pt}{0pt}} Sky~~
  \fcolorbox{white}{minc_19}{\rule{0pt}{6pt}\rule{6pt}{0pt}} Stone~~
  \fcolorbox{white}{minc_20}{\rule{0pt}{6pt}\rule{6pt}{0pt}} Tile~~
  \fcolorbox{white}{minc_21}{\rule{0pt}{6pt}\rule{6pt}{0pt}} Wallpaper~~
  \fcolorbox{white}{minc_22}{\rule{0pt}{6pt}\rule{6pt}{0pt}} Water~~
  \fcolorbox{white}{minc_23}{\rule{0pt}{6pt}\rule{6pt}{0pt}} Wood~~ \\
  }
  \subfigure{%
    \includegraphics[width=.18\columnwidth]{figures/supplementary/000010868_given.jpg}
  }
  \subfigure{%
    \includegraphics[width=.18\columnwidth]{figures/supplementary/000010868_gt.png}
  }
  \subfigure{%
    \includegraphics[width=.18\columnwidth]{figures/supplementary/000010868_cnn.png}
  }
  \subfigure{%
    \includegraphics[width=.18\columnwidth]{figures/supplementary/000010868_gauss.png}
  }
  \subfigure{%
    \includegraphics[width=.18\columnwidth]{figures/supplementary/000010868_learnt.png}
  }\\[-2ex]
  \subfigure{%
    \includegraphics[width=.18\columnwidth]{figures/supplementary/000006011_given.jpg}
  }
  \subfigure{%
    \includegraphics[width=.18\columnwidth]{figures/supplementary/000006011_gt.png}
  }
  \subfigure{%
    \includegraphics[width=.18\columnwidth]{figures/supplementary/000006011_cnn.png}
  }
  \subfigure{%
    \includegraphics[width=.18\columnwidth]{figures/supplementary/000006011_gauss.png}
  }
  \subfigure{%
    \includegraphics[width=.18\columnwidth]{figures/supplementary/000006011_learnt.png}
  }\\[-2ex]
    \subfigure{%
    \includegraphics[width=.18\columnwidth]{figures/supplementary/000008553_given.jpg}
  }
  \subfigure{%
    \includegraphics[width=.18\columnwidth]{figures/supplementary/000008553_gt.png}
  }
  \subfigure{%
    \includegraphics[width=.18\columnwidth]{figures/supplementary/000008553_cnn.png}
  }
  \subfigure{%
    \includegraphics[width=.18\columnwidth]{figures/supplementary/000008553_gauss.png}
  }
  \subfigure{%
    \includegraphics[width=.18\columnwidth]{figures/supplementary/000008553_learnt.png}
  }\\[-2ex]
   \subfigure{%
    \includegraphics[width=.18\columnwidth]{figures/supplementary/000009188_given.jpg}
  }
  \subfigure{%
    \includegraphics[width=.18\columnwidth]{figures/supplementary/000009188_gt.png}
  }
  \subfigure{%
    \includegraphics[width=.18\columnwidth]{figures/supplementary/000009188_cnn.png}
  }
  \subfigure{%
    \includegraphics[width=.18\columnwidth]{figures/supplementary/000009188_gauss.png}
  }
  \subfigure{%
    \includegraphics[width=.18\columnwidth]{figures/supplementary/000009188_learnt.png}
  }\\[-2ex]
  \setcounter{subfigure}{0}
  \subfigure[Input]{%
    \includegraphics[width=.18\columnwidth]{figures/supplementary/000023570_given.jpg}
  }
  \subfigure[Ground Truth]{%
    \includegraphics[width=.18\columnwidth]{figures/supplementary/000023570_gt.png}
  }
  \subfigure[DeepLab]{%
    \includegraphics[width=.18\columnwidth]{figures/supplementary/000023570_cnn.png}
  }
  \subfigure[+GaussCRF]{%
    \includegraphics[width=.18\columnwidth]{figures/supplementary/000023570_gauss.png}
  }
  \subfigure[+LearnedCRF]{%
    \includegraphics[width=.18\columnwidth]{figures/supplementary/000023570_learnt.png}
  }
  \mycaption{Material Segmentation}{Example results of material segmentation.
  (c)~depicts the unary results before application of MF, (d)~after two steps of MF with Gaussian edge CRF potentials, (e)~after two steps of MF with learned edge CRF potentials.}
    \label{fig:material_visuals-app2}
\end{figure*}


\begin{table*}[h]
\tiny
  \centering
    \begin{tabular}{L{2.3cm} L{2.25cm} C{1.5cm} C{0.7cm} C{0.6cm} C{0.7cm} C{0.7cm} C{0.7cm} C{1.6cm} C{0.6cm} C{0.6cm} C{0.6cm}}
      \toprule
& & & & & \multicolumn{3}{c}{\textbf{Data Statistics}} & \multicolumn{4}{c}{\textbf{Training Protocol}} \\

\textbf{Experiment} & \textbf{Feature Types} & \textbf{Feature Scales} & \textbf{Filter Size} & \textbf{Filter Nbr.} & \textbf{Train}  & \textbf{Val.} & \textbf{Test} & \textbf{Loss Type} & \textbf{LR} & \textbf{Batch} & \textbf{Epochs} \\
      \midrule
      \multicolumn{2}{c}{\textbf{Single Bilateral Filter Applications}} & & & & & & & & & \\
      \textbf{2$\times$ Color Upsampling} & Position$_{1}$, Intensity (3D) & 0.13, 0.17 & 65 & 2 & 10581 & 1449 & 1456 & MSE & 1e-06 & 200 & 94.5\\
      \textbf{4$\times$ Color Upsampling} & Position$_{1}$, Intensity (3D) & 0.06, 0.17 & 65 & 2 & 10581 & 1449 & 1456 & MSE & 1e-06 & 200 & 94.5\\
      \textbf{8$\times$ Color Upsampling} & Position$_{1}$, Intensity (3D) & 0.03, 0.17 & 65 & 2 & 10581 & 1449 & 1456 & MSE & 1e-06 & 200 & 94.5\\
      \textbf{16$\times$ Color Upsampling} & Position$_{1}$, Intensity (3D) & 0.02, 0.17 & 65 & 2 & 10581 & 1449 & 1456 & MSE & 1e-06 & 200 & 94.5\\
      \textbf{Depth Upsampling} & Position$_{1}$, Color (5D) & 0.05, 0.02 & 665 & 2 & 795 & 100 & 654 & MSE & 1e-07 & 50 & 251.6\\
      \textbf{Mesh Denoising} & Isomap (4D) & 46.00 & 63 & 2 & 1000 & 200 & 500 & MSE & 100 & 10 & 100.0 \\
      \midrule
      \multicolumn{2}{c}{\textbf{DenseCRF Applications}} & & & & & & & & &\\
      \multicolumn{2}{l}{\textbf{Semantic Segmentation}} & & & & & & & & &\\
      \textbf{- 1step MF} & Position$_{1}$, Color (5D); Position$_{1}$ (2D) & 0.01, 0.34; 0.34  & 665; 19  & 2; 2 & 10581 & 1449 & 1456 & Logistic & 0.1 & 5 & 1.4 \\
      \textbf{- 2step MF} & Position$_{1}$, Color (5D); Position$_{1}$ (2D) & 0.01, 0.34; 0.34 & 665; 19 & 2; 2 & 10581 & 1449 & 1456 & Logistic & 0.1 & 5 & 1.4 \\
      \textbf{- \textit{loose} 2step MF} & Position$_{1}$, Color (5D); Position$_{1}$ (2D) & 0.01, 0.34; 0.34 & 665; 19 & 2; 2 &10581 & 1449 & 1456 & Logistic & 0.1 & 5 & +1.9  \\ \\
      \multicolumn{2}{l}{\textbf{Material Segmentation}} & & & & & & & & &\\
      \textbf{- 1step MF} & Position$_{2}$, Lab-Color (5D) & 5.00, 0.05, 0.30  & 665 & 2 & 928 & 150 & 1798 & Weighted Logistic & 1e-04 & 24 & 2.6 \\
      \textbf{- 2step MF} & Position$_{2}$, Lab-Color (5D) & 5.00, 0.05, 0.30 & 665 & 2 & 928 & 150 & 1798 & Weighted Logistic & 1e-04 & 12 & +0.7 \\
      \textbf{- \textit{loose} 2step MF} & Position$_{2}$, Lab-Color (5D) & 5.00, 0.05, 0.30 & 665 & 2 & 928 & 150 & 1798 & Weighted Logistic & 1e-04 & 12 & +0.2\\
      \midrule
      \multicolumn{2}{c}{\textbf{Neural Network Applications}} & & & & & & & & &\\
      \textbf{Tiles: CNN-9$\times$9} & - & - & 81 & 4 & 10000 & 1000 & 1000 & Logistic & 0.01 & 100 & 500.0 \\
      \textbf{Tiles: CNN-13$\times$13} & - & - & 169 & 6 & 10000 & 1000 & 1000 & Logistic & 0.01 & 100 & 500.0 \\
      \textbf{Tiles: CNN-17$\times$17} & - & - & 289 & 8 & 10000 & 1000 & 1000 & Logistic & 0.01 & 100 & 500.0 \\
      \textbf{Tiles: CNN-21$\times$21} & - & - & 441 & 10 & 10000 & 1000 & 1000 & Logistic & 0.01 & 100 & 500.0 \\
      \textbf{Tiles: BNN} & Position$_{1}$, Color (5D) & 0.05, 0.04 & 63 & 1 & 10000 & 1000 & 1000 & Logistic & 0.01 & 100 & 30.0 \\
      \textbf{LeNet} & - & - & 25 & 2 & 5490 & 1098 & 1647 & Logistic & 0.1 & 100 & 182.2 \\
      \textbf{Crop-LeNet} & - & - & 25 & 2 & 5490 & 1098 & 1647 & Logistic & 0.1 & 100 & 182.2 \\
      \textbf{BNN-LeNet} & Position$_{2}$ (2D) & 20.00 & 7 & 1 & 5490 & 1098 & 1647 & Logistic & 0.1 & 100 & 182.2 \\
      \textbf{DeepCNet} & - & - & 9 & 1 & 5490 & 1098 & 1647 & Logistic & 0.1 & 100 & 182.2 \\
      \textbf{Crop-DeepCNet} & - & - & 9 & 1 & 5490 & 1098 & 1647 & Logistic & 0.1 & 100 & 182.2 \\
      \textbf{BNN-DeepCNet} & Position$_{2}$ (2D) & 40.00  & 7 & 1 & 5490 & 1098 & 1647 & Logistic & 0.1 & 100 & 182.2 \\
      \bottomrule
      \\
    \end{tabular}
    \mycaption{Experiment Protocols} {Experiment protocols for the different experiments presented in this work. \textbf{Feature Types}:
    Feature spaces used for the bilateral convolutions. Position$_1$ corresponds to un-normalized pixel positions whereas Position$_2$ corresponds
    to pixel positions normalized to $[0,1]$ with respect to the given image. \textbf{Feature Scales}: Cross-validated scales for the features used.
     \textbf{Filter Size}: Number of elements in the filter that is being learned. \textbf{Filter Nbr.}: Half-width of the filter. \textbf{Train},
     \textbf{Val.} and \textbf{Test} corresponds to the number of train, validation and test images used in the experiment. \textbf{Loss Type}: Type
     of loss used for back-propagation. ``MSE'' corresponds to Euclidean mean squared error loss and ``Logistic'' corresponds to multinomial logistic
     loss. ``Weighted Logistic'' is the class-weighted multinomial logistic loss. We weighted the loss with inverse class probability for material
     segmentation task due to the small availability of training data with class imbalance. \textbf{LR}: Fixed learning rate used in stochastic gradient
     descent. \textbf{Batch}: Number of images used in one parameter update step. \textbf{Epochs}: Number of training epochs. In all the experiments,
     we used fixed momentum of 0.9 and weight decay of 0.0005 for stochastic gradient descent. ```Color Upsampling'' experiments in this Table corresponds
     to those performed on Pascal VOC12 dataset images. For all experiments using Pascal VOC12 images, we use extended
     training segmentation dataset available from~\cite{hariharan2011moredata}, and used standard validation and test splits
     from the main dataset~\cite{voc2012segmentation}.}
  \label{tbl:parameters}
\end{table*}

\clearpage

\section{Parameters and Additional Results for Video Propagation Networks}

In this Section, we present experiment protocols and additional qualitative results for experiments
on video object segmentation, semantic video segmentation and video color
propagation. Table~\ref{tbl:parameters_supp} shows the feature scales and other parameters used in different experiments.
Figures~\ref{fig:video_seg_pos_supp} show some qualitative results on video object segmentation
with some failure cases in Fig.~\ref{fig:video_seg_neg_supp}.
Figure~\ref{fig:semantic_visuals_supp} shows some qualitative results on semantic video segmentation and
Fig.~\ref{fig:color_visuals_supp} shows results on video color propagation.

\newcolumntype{L}[1]{>{\raggedright\let\newline\\\arraybackslash\hspace{0pt}}b{#1}}
\newcolumntype{C}[1]{>{\centering\let\newline\\\arraybackslash\hspace{0pt}}b{#1}}
\newcolumntype{R}[1]{>{\raggedleft\let\newline\\\arraybackslash\hspace{0pt}}b{#1}}

\begin{table*}[h]
\tiny
  \centering
    \begin{tabular}{L{3.0cm} L{2.4cm} L{2.8cm} L{2.8cm} C{0.5cm} C{1.0cm} L{1.2cm}}
      \toprule
\textbf{Experiment} & \textbf{Feature Type} & \textbf{Feature Scale-1, $\Lambda_a$} & \textbf{Feature Scale-2, $\Lambda_b$} & \textbf{$\alpha$} & \textbf{Input Frames} & \textbf{Loss Type} \\
      \midrule
      \textbf{Video Object Segmentation} & ($x,y,Y,Cb,Cr,t$) & (0.02,0.02,0.07,0.4,0.4,0.01) & (0.03,0.03,0.09,0.5,0.5,0.2) & 0.5 & 9 & Logistic\\
      \midrule
      \textbf{Semantic Video Segmentation} & & & & & \\
      \textbf{with CNN1~\cite{yu2015multi}-NoFlow} & ($x,y,R,G,B,t$) & (0.08,0.08,0.2,0.2,0.2,0.04) & (0.11,0.11,0.2,0.2,0.2,0.04) & 0.5 & 3 & Logistic \\
      \textbf{with CNN1~\cite{yu2015multi}-Flow} & ($x+u_x,y+u_y,R,G,B,t$) & (0.11,0.11,0.14,0.14,0.14,0.03) & (0.08,0.08,0.12,0.12,0.12,0.01) & 0.65 & 3 & Logistic\\
      \textbf{with CNN2~\cite{richter2016playing}-Flow} & ($x+u_x,y+u_y,R,G,B,t$) & (0.08,0.08,0.2,0.2,0.2,0.04) & (0.09,0.09,0.25,0.25,0.25,0.03) & 0.5 & 4 & Logistic\\
      \midrule
      \textbf{Video Color Propagation} & ($x,y,I,t$)  & (0.04,0.04,0.2,0.04) & No second kernel & 1 & 4 & MSE\\
      \bottomrule
      \\
    \end{tabular}
    \mycaption{Experiment Protocols} {Experiment protocols for the different experiments presented in this work. \textbf{Feature Types}:
    Feature spaces used for the bilateral convolutions, with position ($x,y$) and color
    ($R,G,B$ or $Y,Cb,Cr$) features $\in [0,255]$. $u_x$, $u_y$ denotes optical flow with respect
    to the present frame and $I$ denotes grayscale intensity.
    \textbf{Feature Scales ($\Lambda_a, \Lambda_b$)}: Cross-validated scales for the features used.
    \textbf{$\alpha$}: Exponential time decay for the input frames.
    \textbf{Input Frames}: Number of input frames for VPN.
    \textbf{Loss Type}: Type
     of loss used for back-propagation. ``MSE'' corresponds to Euclidean mean squared error loss and ``Logistic'' corresponds to multinomial logistic loss.}
  \label{tbl:parameters_supp}
\end{table*}

% \begin{figure}[th!]
% \begin{center}
%   \centerline{\includegraphics[width=\textwidth]{figures/video_seg_visuals_supp_small.pdf}}
%     \mycaption{Video Object Segmentation}
%     {Shown are the different frames in example videos with the corresponding
%     ground truth (GT) masks, predictions from BVS~\cite{marki2016bilateral},
%     OFL~\cite{tsaivideo}, VPN (VPN-Stage2) and VPN-DLab (VPN-DeepLab) models.}
%     \label{fig:video_seg_small_supp}
% \end{center}
% \vspace{-1.0cm}
% \end{figure}

\begin{figure}[th!]
\begin{center}
  \centerline{\includegraphics[width=0.7\textwidth]{figures/video_seg_visuals_supp_positive.pdf}}
    \mycaption{Video Object Segmentation}
    {Shown are the different frames in example videos with the corresponding
    ground truth (GT) masks, predictions from BVS~\cite{marki2016bilateral},
    OFL~\cite{tsaivideo}, VPN (VPN-Stage2) and VPN-DLab (VPN-DeepLab) models.}
    \label{fig:video_seg_pos_supp}
\end{center}
\vspace{-1.0cm}
\end{figure}

\begin{figure}[th!]
\begin{center}
  \centerline{\includegraphics[width=0.7\textwidth]{figures/video_seg_visuals_supp_negative.pdf}}
    \mycaption{Failure Cases for Video Object Segmentation}
    {Shown are the different frames in example videos with the corresponding
    ground truth (GT) masks, predictions from BVS~\cite{marki2016bilateral},
    OFL~\cite{tsaivideo}, VPN (VPN-Stage2) and VPN-DLab (VPN-DeepLab) models.}
    \label{fig:video_seg_neg_supp}
\end{center}
\vspace{-1.0cm}
\end{figure}

\begin{figure}[th!]
\begin{center}
  \centerline{\includegraphics[width=0.9\textwidth]{figures/supp_semantic_visual.pdf}}
    \mycaption{Semantic Video Segmentation}
    {Input video frames and the corresponding ground truth (GT)
    segmentation together with the predictions of CNN~\cite{yu2015multi} and with
    VPN-Flow.}
    \label{fig:semantic_visuals_supp}
\end{center}
\vspace{-0.7cm}
\end{figure}

\begin{figure}[th!]
\begin{center}
  \centerline{\includegraphics[width=\textwidth]{figures/colorization_visuals_supp.pdf}}
  \mycaption{Video Color Propagation}
  {Input grayscale video frames and corresponding ground-truth (GT) color images
  together with color predictions of Levin et al.~\cite{levin2004colorization} and VPN-Stage1 models.}
  \label{fig:color_visuals_supp}
\end{center}
\vspace{-0.7cm}
\end{figure}

\clearpage

\section{Additional Material for Bilateral Inception Networks}
\label{sec:binception-app}

In this section of the Appendix, we first discuss the use of approximate bilateral
filtering in BI modules (Sec.~\ref{sec:lattice}).
Later, we present some qualitative results using different models for the approach presented in
Chapter~\ref{chap:binception} (Sec.~\ref{sec:qualitative-app}).

\subsection{Approximate Bilateral Filtering}
\label{sec:lattice}

The bilateral inception module presented in Chapter~\ref{chap:binception} computes a matrix-vector
product between a Gaussian filter $K$ and a vector of activations $\bz_c$.
Bilateral filtering is an important operation and many algorithmic techniques have been
proposed to speed-up this operation~\cite{paris2006fast,adams2010fast,gastal2011domain}.
In the main paper we opted to implement what can be considered the
brute-force variant of explicitly constructing $K$ and then using BLAS to compute the
matrix-vector product. This resulted in a few millisecond operation.
The explicit way to compute is possible due to the
reduction to super-pixels, e.g., it would not work for DenseCRF variants
that operate on the full image resolution.

Here, we present experiments where we use the fast approximate bilateral filtering
algorithm of~\cite{adams2010fast}, which is also used in Chapter~\ref{chap:bnn}
for learning sparse high dimensional filters. This
choice allows for larger dimensions of matrix-vector multiplication. The reason for choosing
the explicit multiplication in Chapter~\ref{chap:binception} was that it was computationally faster.
For the small sizes of the involved matrices and vectors, the explicit computation is sufficient and we had no
GPU implementation of an approximate technique that matched this runtime. Also it
is conceptually easier and the gradient to the feature transformations ($\Lambda \mathbf{f}$) is
obtained using standard matrix calculus.

\subsubsection{Experiments}

We modified the existing segmentation architectures analogous to those in Chapter~\ref{chap:binception}.
The main difference is that, here, the inception modules use the lattice
approximation~\cite{adams2010fast} to compute the bilateral filtering.
Using the lattice approximation did not allow us to back-propagate through feature transformations ($\Lambda$)
and thus we used hand-specified feature scales as will be explained later.
Specifically, we take CNN architectures from the works
of~\cite{chen2014semantic,zheng2015conditional,bell2015minc} and insert the BI modules between
the spatial FC layers.
We use superpixels from~\cite{DollarICCV13edges}
for all the experiments with the lattice approximation. Experiments are
performed using Caffe neural network framework~\cite{jia2014caffe}.

\begin{table}
  \small
  \centering
  \begin{tabular}{p{5.5cm}>{\raggedright\arraybackslash}p{1.4cm}>{\centering\arraybackslash}p{2.2cm}}
    \toprule
		\textbf{Model} & \emph{IoU} & \emph{Runtime}(ms) \\
    \midrule

    %%%%%%%%%%%% Scores computed by us)%%%%%%%%%%%%
		\deeplablargefov & 68.9 & 145ms\\
    \midrule
    \bi{7}{2}-\bi{8}{10}& \textbf{73.8} & +600 \\
    \midrule
    \deeplablargefovcrf~\cite{chen2014semantic} & 72.7 & +830\\
    \deeplabmsclargefovcrf~\cite{chen2014semantic} & \textbf{73.6} & +880\\
    DeepLab-EdgeNet~\cite{chen2015semantic} & 71.7 & +30\\
    DeepLab-EdgeNet-CRF~\cite{chen2015semantic} & \textbf{73.6} & +860\\
  \bottomrule \\
  \end{tabular}
  \mycaption{Semantic Segmentation using the DeepLab model}
  {IoU scores on the Pascal VOC12 segmentation test dataset
  with different models and our modified inception model.
  Also shown are the corresponding runtimes in milliseconds. Runtimes
  also include superpixel computations (300 ms with Dollar superpixels~\cite{DollarICCV13edges})}
  \label{tab:largefovresults}
\end{table}

\paragraph{Semantic Segmentation}
The experiments in this section use the Pascal VOC12 segmentation dataset~\cite{voc2012segmentation} with 21 object classes and the images have a maximum resolution of 0.25 megapixels.
For all experiments on VOC12, we train using the extended training set of
10581 images collected by~\cite{hariharan2011moredata}.
We modified the \deeplab~network architecture of~\cite{chen2014semantic} and
the CRFasRNN architecture from~\cite{zheng2015conditional} which uses a CNN with
deconvolution layers followed by DenseCRF trained end-to-end.

\paragraph{DeepLab Model}\label{sec:deeplabmodel}
We experimented with the \bi{7}{2}-\bi{8}{10} inception model.
Results using the~\deeplab~model are summarized in Tab.~\ref{tab:largefovresults}.
Although we get similar improvements with inception modules as with the
explicit kernel computation, using lattice approximation is slower.

\begin{table}
  \small
  \centering
  \begin{tabular}{p{6.4cm}>{\raggedright\arraybackslash}p{1.8cm}>{\raggedright\arraybackslash}p{1.8cm}}
    \toprule
    \textbf{Model} & \emph{IoU (Val)} & \emph{IoU (Test)}\\
    \midrule
    %%%%%%%%%%%% Scores computed by us)%%%%%%%%%%%%
    CNN &  67.5 & - \\
    \deconv (CNN+Deconvolutions) & 69.8 & 72.0 \\
    \midrule
    \bi{3}{6}-\bi{4}{6}-\bi{7}{2}-\bi{8}{6}& 71.9 & - \\
    \bi{3}{6}-\bi{4}{6}-\bi{7}{2}-\bi{8}{6}-\gi{6}& 73.6 &  \href{http://host.robots.ox.ac.uk:8080/anonymous/VOTV5E.html}{\textbf{75.2}}\\
    \midrule
    \deconvcrf (CRF-RNN)~\cite{zheng2015conditional} & 73.0 & 74.7\\
    Context-CRF-RNN~\cite{yu2015multi} & ~~ - ~ & \textbf{75.3} \\
    \bottomrule \\
  \end{tabular}
  \mycaption{Semantic Segmentation using the CRFasRNN model}{IoU score corresponding to different models
  on Pascal VOC12 reduced validation / test segmentation dataset. The reduced validation set consists of 346 images
  as used in~\cite{zheng2015conditional} where we adapted the model from.}
  \label{tab:deconvresults-app}
\end{table}

\paragraph{CRFasRNN Model}\label{sec:deepinception}
We add BI modules after score-pool3, score-pool4, \fc{7} and \fc{8} $1\times1$ convolution layers
resulting in the \bi{3}{6}-\bi{4}{6}-\bi{7}{2}-\bi{8}{6}
model and also experimented with another variant where $BI_8$ is followed by another inception
module, G$(6)$, with 6 Gaussian kernels.
Note that here also we discarded both deconvolution and DenseCRF parts of the original model~\cite{zheng2015conditional}
and inserted the BI modules in the base CNN and found similar improvements compared to the inception modules with explicit
kernel computaion. See Tab.~\ref{tab:deconvresults-app} for results on the CRFasRNN model.

\paragraph{Material Segmentation}
Table~\ref{tab:mincresults-app} shows the results on the MINC dataset~\cite{bell2015minc}
obtained by modifying the AlexNet architecture with our inception modules. We observe
similar improvements as with explicit kernel construction.
For this model, we do not provide any learned setup due to very limited segment training
data. The weights to combine outputs in the bilateral inception layer are
found by validation on the validation set.

\begin{table}[t]
  \small
  \centering
  \begin{tabular}{p{3.5cm}>{\centering\arraybackslash}p{4.0cm}}
    \toprule
    \textbf{Model} & Class / Total accuracy\\
    \midrule

    %%%%%%%%%%%% Scores computed by us)%%%%%%%%%%%%
    AlexNet CNN & 55.3 / 58.9 \\
    \midrule
    \bi{7}{2}-\bi{8}{6}& 68.5 / 71.8 \\
    \bi{7}{2}-\bi{8}{6}-G$(6)$& 67.6 / 73.1 \\
    \midrule
    AlexNet-CRF & 65.5 / 71.0 \\
    \bottomrule \\
  \end{tabular}
  \mycaption{Material Segmentation using AlexNet}{Pixel accuracy of different models on
  the MINC material segmentation test dataset~\cite{bell2015minc}.}
  \label{tab:mincresults-app}
\end{table}

\paragraph{Scales of Bilateral Inception Modules}
\label{sec:scales}

Unlike the explicit kernel technique presented in the main text (Chapter~\ref{chap:binception}),
we didn't back-propagate through feature transformation ($\Lambda$)
using the approximate bilateral filter technique.
So, the feature scales are hand-specified and validated, which are as follows.
The optimal scale values for the \bi{7}{2}-\bi{8}{2} model are found by validation for the best performance which are
$\sigma_{xy}$ = (0.1, 0.1) for the spatial (XY) kernel and $\sigma_{rgbxy}$ = (0.1, 0.1, 0.1, 0.01, 0.01) for color and position (RGBXY)  kernel.
Next, as more kernels are added to \bi{8}{2}, we set scales to be $\alpha$*($\sigma_{xy}$, $\sigma_{rgbxy}$).
The value of $\alpha$ is chosen as  1, 0.5, 0.1, 0.05, 0.1, at uniform interval, for the \bi{8}{10} bilateral inception module.


\subsection{Qualitative Results}
\label{sec:qualitative-app}

In this section, we present more qualitative results obtained using the BI module with explicit
kernel computation technique presented in Chapter~\ref{chap:binception}. Results on the Pascal VOC12
dataset~\cite{voc2012segmentation} using the DeepLab-LargeFOV model are shown in Fig.~\ref{fig:semantic_visuals-app},
followed by the results on MINC dataset~\cite{bell2015minc}
in Fig.~\ref{fig:material_visuals-app} and on
Cityscapes dataset~\cite{Cordts2015Cvprw} in Fig.~\ref{fig:street_visuals-app}.


\definecolor{voc_1}{RGB}{0, 0, 0}
\definecolor{voc_2}{RGB}{128, 0, 0}
\definecolor{voc_3}{RGB}{0, 128, 0}
\definecolor{voc_4}{RGB}{128, 128, 0}
\definecolor{voc_5}{RGB}{0, 0, 128}
\definecolor{voc_6}{RGB}{128, 0, 128}
\definecolor{voc_7}{RGB}{0, 128, 128}
\definecolor{voc_8}{RGB}{128, 128, 128}
\definecolor{voc_9}{RGB}{64, 0, 0}
\definecolor{voc_10}{RGB}{192, 0, 0}
\definecolor{voc_11}{RGB}{64, 128, 0}
\definecolor{voc_12}{RGB}{192, 128, 0}
\definecolor{voc_13}{RGB}{64, 0, 128}
\definecolor{voc_14}{RGB}{192, 0, 128}
\definecolor{voc_15}{RGB}{64, 128, 128}
\definecolor{voc_16}{RGB}{192, 128, 128}
\definecolor{voc_17}{RGB}{0, 64, 0}
\definecolor{voc_18}{RGB}{128, 64, 0}
\definecolor{voc_19}{RGB}{0, 192, 0}
\definecolor{voc_20}{RGB}{128, 192, 0}
\definecolor{voc_21}{RGB}{0, 64, 128}
\definecolor{voc_22}{RGB}{128, 64, 128}

\begin{figure*}[!ht]
  \small
  \centering
  \fcolorbox{white}{voc_1}{\rule{0pt}{4pt}\rule{4pt}{0pt}} Background~~
  \fcolorbox{white}{voc_2}{\rule{0pt}{4pt}\rule{4pt}{0pt}} Aeroplane~~
  \fcolorbox{white}{voc_3}{\rule{0pt}{4pt}\rule{4pt}{0pt}} Bicycle~~
  \fcolorbox{white}{voc_4}{\rule{0pt}{4pt}\rule{4pt}{0pt}} Bird~~
  \fcolorbox{white}{voc_5}{\rule{0pt}{4pt}\rule{4pt}{0pt}} Boat~~
  \fcolorbox{white}{voc_6}{\rule{0pt}{4pt}\rule{4pt}{0pt}} Bottle~~
  \fcolorbox{white}{voc_7}{\rule{0pt}{4pt}\rule{4pt}{0pt}} Bus~~
  \fcolorbox{white}{voc_8}{\rule{0pt}{4pt}\rule{4pt}{0pt}} Car~~\\
  \fcolorbox{white}{voc_9}{\rule{0pt}{4pt}\rule{4pt}{0pt}} Cat~~
  \fcolorbox{white}{voc_10}{\rule{0pt}{4pt}\rule{4pt}{0pt}} Chair~~
  \fcolorbox{white}{voc_11}{\rule{0pt}{4pt}\rule{4pt}{0pt}} Cow~~
  \fcolorbox{white}{voc_12}{\rule{0pt}{4pt}\rule{4pt}{0pt}} Dining Table~~
  \fcolorbox{white}{voc_13}{\rule{0pt}{4pt}\rule{4pt}{0pt}} Dog~~
  \fcolorbox{white}{voc_14}{\rule{0pt}{4pt}\rule{4pt}{0pt}} Horse~~
  \fcolorbox{white}{voc_15}{\rule{0pt}{4pt}\rule{4pt}{0pt}} Motorbike~~
  \fcolorbox{white}{voc_16}{\rule{0pt}{4pt}\rule{4pt}{0pt}} Person~~\\
  \fcolorbox{white}{voc_17}{\rule{0pt}{4pt}\rule{4pt}{0pt}} Potted Plant~~
  \fcolorbox{white}{voc_18}{\rule{0pt}{4pt}\rule{4pt}{0pt}} Sheep~~
  \fcolorbox{white}{voc_19}{\rule{0pt}{4pt}\rule{4pt}{0pt}} Sofa~~
  \fcolorbox{white}{voc_20}{\rule{0pt}{4pt}\rule{4pt}{0pt}} Train~~
  \fcolorbox{white}{voc_21}{\rule{0pt}{4pt}\rule{4pt}{0pt}} TV monitor~~\\


  \subfigure{%
    \includegraphics[width=.15\columnwidth]{figures/supplementary/2008_001308_given.png}
  }
  \subfigure{%
    \includegraphics[width=.15\columnwidth]{figures/supplementary/2008_001308_sp.png}
  }
  \subfigure{%
    \includegraphics[width=.15\columnwidth]{figures/supplementary/2008_001308_gt.png}
  }
  \subfigure{%
    \includegraphics[width=.15\columnwidth]{figures/supplementary/2008_001308_cnn.png}
  }
  \subfigure{%
    \includegraphics[width=.15\columnwidth]{figures/supplementary/2008_001308_crf.png}
  }
  \subfigure{%
    \includegraphics[width=.15\columnwidth]{figures/supplementary/2008_001308_ours.png}
  }\\[-2ex]


  \subfigure{%
    \includegraphics[width=.15\columnwidth]{figures/supplementary/2008_001821_given.png}
  }
  \subfigure{%
    \includegraphics[width=.15\columnwidth]{figures/supplementary/2008_001821_sp.png}
  }
  \subfigure{%
    \includegraphics[width=.15\columnwidth]{figures/supplementary/2008_001821_gt.png}
  }
  \subfigure{%
    \includegraphics[width=.15\columnwidth]{figures/supplementary/2008_001821_cnn.png}
  }
  \subfigure{%
    \includegraphics[width=.15\columnwidth]{figures/supplementary/2008_001821_crf.png}
  }
  \subfigure{%
    \includegraphics[width=.15\columnwidth]{figures/supplementary/2008_001821_ours.png}
  }\\[-2ex]



  \subfigure{%
    \includegraphics[width=.15\columnwidth]{figures/supplementary/2008_004612_given.png}
  }
  \subfigure{%
    \includegraphics[width=.15\columnwidth]{figures/supplementary/2008_004612_sp.png}
  }
  \subfigure{%
    \includegraphics[width=.15\columnwidth]{figures/supplementary/2008_004612_gt.png}
  }
  \subfigure{%
    \includegraphics[width=.15\columnwidth]{figures/supplementary/2008_004612_cnn.png}
  }
  \subfigure{%
    \includegraphics[width=.15\columnwidth]{figures/supplementary/2008_004612_crf.png}
  }
  \subfigure{%
    \includegraphics[width=.15\columnwidth]{figures/supplementary/2008_004612_ours.png}
  }\\[-2ex]


  \subfigure{%
    \includegraphics[width=.15\columnwidth]{figures/supplementary/2009_001008_given.png}
  }
  \subfigure{%
    \includegraphics[width=.15\columnwidth]{figures/supplementary/2009_001008_sp.png}
  }
  \subfigure{%
    \includegraphics[width=.15\columnwidth]{figures/supplementary/2009_001008_gt.png}
  }
  \subfigure{%
    \includegraphics[width=.15\columnwidth]{figures/supplementary/2009_001008_cnn.png}
  }
  \subfigure{%
    \includegraphics[width=.15\columnwidth]{figures/supplementary/2009_001008_crf.png}
  }
  \subfigure{%
    \includegraphics[width=.15\columnwidth]{figures/supplementary/2009_001008_ours.png}
  }\\[-2ex]




  \subfigure{%
    \includegraphics[width=.15\columnwidth]{figures/supplementary/2009_004497_given.png}
  }
  \subfigure{%
    \includegraphics[width=.15\columnwidth]{figures/supplementary/2009_004497_sp.png}
  }
  \subfigure{%
    \includegraphics[width=.15\columnwidth]{figures/supplementary/2009_004497_gt.png}
  }
  \subfigure{%
    \includegraphics[width=.15\columnwidth]{figures/supplementary/2009_004497_cnn.png}
  }
  \subfigure{%
    \includegraphics[width=.15\columnwidth]{figures/supplementary/2009_004497_crf.png}
  }
  \subfigure{%
    \includegraphics[width=.15\columnwidth]{figures/supplementary/2009_004497_ours.png}
  }\\[-2ex]



  \setcounter{subfigure}{0}
  \subfigure[\scriptsize Input]{%
    \includegraphics[width=.15\columnwidth]{figures/supplementary/2010_001327_given.png}
  }
  \subfigure[\scriptsize Superpixels]{%
    \includegraphics[width=.15\columnwidth]{figures/supplementary/2010_001327_sp.png}
  }
  \subfigure[\scriptsize GT]{%
    \includegraphics[width=.15\columnwidth]{figures/supplementary/2010_001327_gt.png}
  }
  \subfigure[\scriptsize Deeplab]{%
    \includegraphics[width=.15\columnwidth]{figures/supplementary/2010_001327_cnn.png}
  }
  \subfigure[\scriptsize +DenseCRF]{%
    \includegraphics[width=.15\columnwidth]{figures/supplementary/2010_001327_crf.png}
  }
  \subfigure[\scriptsize Using BI]{%
    \includegraphics[width=.15\columnwidth]{figures/supplementary/2010_001327_ours.png}
  }
  \mycaption{Semantic Segmentation}{Example results of semantic segmentation
  on the Pascal VOC12 dataset.
  (d)~depicts the DeepLab CNN result, (e)~CNN + 10 steps of mean-field inference,
  (f~result obtained with bilateral inception (BI) modules (\bi{6}{2}+\bi{7}{6}) between \fc~layers.}
  \label{fig:semantic_visuals-app}
\end{figure*}


\definecolor{minc_1}{HTML}{771111}
\definecolor{minc_2}{HTML}{CAC690}
\definecolor{minc_3}{HTML}{EEEEEE}
\definecolor{minc_4}{HTML}{7C8FA6}
\definecolor{minc_5}{HTML}{597D31}
\definecolor{minc_6}{HTML}{104410}
\definecolor{minc_7}{HTML}{BB819C}
\definecolor{minc_8}{HTML}{D0CE48}
\definecolor{minc_9}{HTML}{622745}
\definecolor{minc_10}{HTML}{666666}
\definecolor{minc_11}{HTML}{D54A31}
\definecolor{minc_12}{HTML}{101044}
\definecolor{minc_13}{HTML}{444126}
\definecolor{minc_14}{HTML}{75D646}
\definecolor{minc_15}{HTML}{DD4348}
\definecolor{minc_16}{HTML}{5C8577}
\definecolor{minc_17}{HTML}{C78472}
\definecolor{minc_18}{HTML}{75D6D0}
\definecolor{minc_19}{HTML}{5B4586}
\definecolor{minc_20}{HTML}{C04393}
\definecolor{minc_21}{HTML}{D69948}
\definecolor{minc_22}{HTML}{7370D8}
\definecolor{minc_23}{HTML}{7A3622}
\definecolor{minc_24}{HTML}{000000}

\begin{figure*}[!ht]
  \small % scriptsize
  \centering
  \fcolorbox{white}{minc_1}{\rule{0pt}{4pt}\rule{4pt}{0pt}} Brick~~
  \fcolorbox{white}{minc_2}{\rule{0pt}{4pt}\rule{4pt}{0pt}} Carpet~~
  \fcolorbox{white}{minc_3}{\rule{0pt}{4pt}\rule{4pt}{0pt}} Ceramic~~
  \fcolorbox{white}{minc_4}{\rule{0pt}{4pt}\rule{4pt}{0pt}} Fabric~~
  \fcolorbox{white}{minc_5}{\rule{0pt}{4pt}\rule{4pt}{0pt}} Foliage~~
  \fcolorbox{white}{minc_6}{\rule{0pt}{4pt}\rule{4pt}{0pt}} Food~~
  \fcolorbox{white}{minc_7}{\rule{0pt}{4pt}\rule{4pt}{0pt}} Glass~~
  \fcolorbox{white}{minc_8}{\rule{0pt}{4pt}\rule{4pt}{0pt}} Hair~~\\
  \fcolorbox{white}{minc_9}{\rule{0pt}{4pt}\rule{4pt}{0pt}} Leather~~
  \fcolorbox{white}{minc_10}{\rule{0pt}{4pt}\rule{4pt}{0pt}} Metal~~
  \fcolorbox{white}{minc_11}{\rule{0pt}{4pt}\rule{4pt}{0pt}} Mirror~~
  \fcolorbox{white}{minc_12}{\rule{0pt}{4pt}\rule{4pt}{0pt}} Other~~
  \fcolorbox{white}{minc_13}{\rule{0pt}{4pt}\rule{4pt}{0pt}} Painted~~
  \fcolorbox{white}{minc_14}{\rule{0pt}{4pt}\rule{4pt}{0pt}} Paper~~
  \fcolorbox{white}{minc_15}{\rule{0pt}{4pt}\rule{4pt}{0pt}} Plastic~~\\
  \fcolorbox{white}{minc_16}{\rule{0pt}{4pt}\rule{4pt}{0pt}} Polished Stone~~
  \fcolorbox{white}{minc_17}{\rule{0pt}{4pt}\rule{4pt}{0pt}} Skin~~
  \fcolorbox{white}{minc_18}{\rule{0pt}{4pt}\rule{4pt}{0pt}} Sky~~
  \fcolorbox{white}{minc_19}{\rule{0pt}{4pt}\rule{4pt}{0pt}} Stone~~
  \fcolorbox{white}{minc_20}{\rule{0pt}{4pt}\rule{4pt}{0pt}} Tile~~
  \fcolorbox{white}{minc_21}{\rule{0pt}{4pt}\rule{4pt}{0pt}} Wallpaper~~
  \fcolorbox{white}{minc_22}{\rule{0pt}{4pt}\rule{4pt}{0pt}} Water~~
  \fcolorbox{white}{minc_23}{\rule{0pt}{4pt}\rule{4pt}{0pt}} Wood~~\\
  \subfigure{%
    \includegraphics[width=.15\columnwidth]{figures/supplementary/000008468_given.png}
  }
  \subfigure{%
    \includegraphics[width=.15\columnwidth]{figures/supplementary/000008468_sp.png}
  }
  \subfigure{%
    \includegraphics[width=.15\columnwidth]{figures/supplementary/000008468_gt.png}
  }
  \subfigure{%
    \includegraphics[width=.15\columnwidth]{figures/supplementary/000008468_cnn.png}
  }
  \subfigure{%
    \includegraphics[width=.15\columnwidth]{figures/supplementary/000008468_crf.png}
  }
  \subfigure{%
    \includegraphics[width=.15\columnwidth]{figures/supplementary/000008468_ours.png}
  }\\[-2ex]

  \subfigure{%
    \includegraphics[width=.15\columnwidth]{figures/supplementary/000009053_given.png}
  }
  \subfigure{%
    \includegraphics[width=.15\columnwidth]{figures/supplementary/000009053_sp.png}
  }
  \subfigure{%
    \includegraphics[width=.15\columnwidth]{figures/supplementary/000009053_gt.png}
  }
  \subfigure{%
    \includegraphics[width=.15\columnwidth]{figures/supplementary/000009053_cnn.png}
  }
  \subfigure{%
    \includegraphics[width=.15\columnwidth]{figures/supplementary/000009053_crf.png}
  }
  \subfigure{%
    \includegraphics[width=.15\columnwidth]{figures/supplementary/000009053_ours.png}
  }\\[-2ex]




  \subfigure{%
    \includegraphics[width=.15\columnwidth]{figures/supplementary/000014977_given.png}
  }
  \subfigure{%
    \includegraphics[width=.15\columnwidth]{figures/supplementary/000014977_sp.png}
  }
  \subfigure{%
    \includegraphics[width=.15\columnwidth]{figures/supplementary/000014977_gt.png}
  }
  \subfigure{%
    \includegraphics[width=.15\columnwidth]{figures/supplementary/000014977_cnn.png}
  }
  \subfigure{%
    \includegraphics[width=.15\columnwidth]{figures/supplementary/000014977_crf.png}
  }
  \subfigure{%
    \includegraphics[width=.15\columnwidth]{figures/supplementary/000014977_ours.png}
  }\\[-2ex]


  \subfigure{%
    \includegraphics[width=.15\columnwidth]{figures/supplementary/000022922_given.png}
  }
  \subfigure{%
    \includegraphics[width=.15\columnwidth]{figures/supplementary/000022922_sp.png}
  }
  \subfigure{%
    \includegraphics[width=.15\columnwidth]{figures/supplementary/000022922_gt.png}
  }
  \subfigure{%
    \includegraphics[width=.15\columnwidth]{figures/supplementary/000022922_cnn.png}
  }
  \subfigure{%
    \includegraphics[width=.15\columnwidth]{figures/supplementary/000022922_crf.png}
  }
  \subfigure{%
    \includegraphics[width=.15\columnwidth]{figures/supplementary/000022922_ours.png}
  }\\[-2ex]


  \subfigure{%
    \includegraphics[width=.15\columnwidth]{figures/supplementary/000025711_given.png}
  }
  \subfigure{%
    \includegraphics[width=.15\columnwidth]{figures/supplementary/000025711_sp.png}
  }
  \subfigure{%
    \includegraphics[width=.15\columnwidth]{figures/supplementary/000025711_gt.png}
  }
  \subfigure{%
    \includegraphics[width=.15\columnwidth]{figures/supplementary/000025711_cnn.png}
  }
  \subfigure{%
    \includegraphics[width=.15\columnwidth]{figures/supplementary/000025711_crf.png}
  }
  \subfigure{%
    \includegraphics[width=.15\columnwidth]{figures/supplementary/000025711_ours.png}
  }\\[-2ex]


  \subfigure{%
    \includegraphics[width=.15\columnwidth]{figures/supplementary/000034473_given.png}
  }
  \subfigure{%
    \includegraphics[width=.15\columnwidth]{figures/supplementary/000034473_sp.png}
  }
  \subfigure{%
    \includegraphics[width=.15\columnwidth]{figures/supplementary/000034473_gt.png}
  }
  \subfigure{%
    \includegraphics[width=.15\columnwidth]{figures/supplementary/000034473_cnn.png}
  }
  \subfigure{%
    \includegraphics[width=.15\columnwidth]{figures/supplementary/000034473_crf.png}
  }
  \subfigure{%
    \includegraphics[width=.15\columnwidth]{figures/supplementary/000034473_ours.png}
  }\\[-2ex]


  \subfigure{%
    \includegraphics[width=.15\columnwidth]{figures/supplementary/000035463_given.png}
  }
  \subfigure{%
    \includegraphics[width=.15\columnwidth]{figures/supplementary/000035463_sp.png}
  }
  \subfigure{%
    \includegraphics[width=.15\columnwidth]{figures/supplementary/000035463_gt.png}
  }
  \subfigure{%
    \includegraphics[width=.15\columnwidth]{figures/supplementary/000035463_cnn.png}
  }
  \subfigure{%
    \includegraphics[width=.15\columnwidth]{figures/supplementary/000035463_crf.png}
  }
  \subfigure{%
    \includegraphics[width=.15\columnwidth]{figures/supplementary/000035463_ours.png}
  }\\[-2ex]


  \setcounter{subfigure}{0}
  \subfigure[\scriptsize Input]{%
    \includegraphics[width=.15\columnwidth]{figures/supplementary/000035993_given.png}
  }
  \subfigure[\scriptsize Superpixels]{%
    \includegraphics[width=.15\columnwidth]{figures/supplementary/000035993_sp.png}
  }
  \subfigure[\scriptsize GT]{%
    \includegraphics[width=.15\columnwidth]{figures/supplementary/000035993_gt.png}
  }
  \subfigure[\scriptsize AlexNet]{%
    \includegraphics[width=.15\columnwidth]{figures/supplementary/000035993_cnn.png}
  }
  \subfigure[\scriptsize +DenseCRF]{%
    \includegraphics[width=.15\columnwidth]{figures/supplementary/000035993_crf.png}
  }
  \subfigure[\scriptsize Using BI]{%
    \includegraphics[width=.15\columnwidth]{figures/supplementary/000035993_ours.png}
  }
  \mycaption{Material Segmentation}{Example results of material segmentation.
  (d)~depicts the AlexNet CNN result, (e)~CNN + 10 steps of mean-field inference,
  (f)~result obtained with bilateral inception (BI) modules (\bi{7}{2}+\bi{8}{6}) between
  \fc~layers.}
\label{fig:material_visuals-app}
\end{figure*}


\definecolor{city_1}{RGB}{128, 64, 128}
\definecolor{city_2}{RGB}{244, 35, 232}
\definecolor{city_3}{RGB}{70, 70, 70}
\definecolor{city_4}{RGB}{102, 102, 156}
\definecolor{city_5}{RGB}{190, 153, 153}
\definecolor{city_6}{RGB}{153, 153, 153}
\definecolor{city_7}{RGB}{250, 170, 30}
\definecolor{city_8}{RGB}{220, 220, 0}
\definecolor{city_9}{RGB}{107, 142, 35}
\definecolor{city_10}{RGB}{152, 251, 152}
\definecolor{city_11}{RGB}{70, 130, 180}
\definecolor{city_12}{RGB}{220, 20, 60}
\definecolor{city_13}{RGB}{255, 0, 0}
\definecolor{city_14}{RGB}{0, 0, 142}
\definecolor{city_15}{RGB}{0, 0, 70}
\definecolor{city_16}{RGB}{0, 60, 100}
\definecolor{city_17}{RGB}{0, 80, 100}
\definecolor{city_18}{RGB}{0, 0, 230}
\definecolor{city_19}{RGB}{119, 11, 32}
\begin{figure*}[!ht]
  \small % scriptsize
  \centering


  \subfigure{%
    \includegraphics[width=.18\columnwidth]{figures/supplementary/frankfurt00000_016005_given.png}
  }
  \subfigure{%
    \includegraphics[width=.18\columnwidth]{figures/supplementary/frankfurt00000_016005_sp.png}
  }
  \subfigure{%
    \includegraphics[width=.18\columnwidth]{figures/supplementary/frankfurt00000_016005_gt.png}
  }
  \subfigure{%
    \includegraphics[width=.18\columnwidth]{figures/supplementary/frankfurt00000_016005_cnn.png}
  }
  \subfigure{%
    \includegraphics[width=.18\columnwidth]{figures/supplementary/frankfurt00000_016005_ours.png}
  }\\[-2ex]

  \subfigure{%
    \includegraphics[width=.18\columnwidth]{figures/supplementary/frankfurt00000_004617_given.png}
  }
  \subfigure{%
    \includegraphics[width=.18\columnwidth]{figures/supplementary/frankfurt00000_004617_sp.png}
  }
  \subfigure{%
    \includegraphics[width=.18\columnwidth]{figures/supplementary/frankfurt00000_004617_gt.png}
  }
  \subfigure{%
    \includegraphics[width=.18\columnwidth]{figures/supplementary/frankfurt00000_004617_cnn.png}
  }
  \subfigure{%
    \includegraphics[width=.18\columnwidth]{figures/supplementary/frankfurt00000_004617_ours.png}
  }\\[-2ex]

  \subfigure{%
    \includegraphics[width=.18\columnwidth]{figures/supplementary/frankfurt00000_020880_given.png}
  }
  \subfigure{%
    \includegraphics[width=.18\columnwidth]{figures/supplementary/frankfurt00000_020880_sp.png}
  }
  \subfigure{%
    \includegraphics[width=.18\columnwidth]{figures/supplementary/frankfurt00000_020880_gt.png}
  }
  \subfigure{%
    \includegraphics[width=.18\columnwidth]{figures/supplementary/frankfurt00000_020880_cnn.png}
  }
  \subfigure{%
    \includegraphics[width=.18\columnwidth]{figures/supplementary/frankfurt00000_020880_ours.png}
  }\\[-2ex]



  \subfigure{%
    \includegraphics[width=.18\columnwidth]{figures/supplementary/frankfurt00001_007285_given.png}
  }
  \subfigure{%
    \includegraphics[width=.18\columnwidth]{figures/supplementary/frankfurt00001_007285_sp.png}
  }
  \subfigure{%
    \includegraphics[width=.18\columnwidth]{figures/supplementary/frankfurt00001_007285_gt.png}
  }
  \subfigure{%
    \includegraphics[width=.18\columnwidth]{figures/supplementary/frankfurt00001_007285_cnn.png}
  }
  \subfigure{%
    \includegraphics[width=.18\columnwidth]{figures/supplementary/frankfurt00001_007285_ours.png}
  }\\[-2ex]


  \subfigure{%
    \includegraphics[width=.18\columnwidth]{figures/supplementary/frankfurt00001_059789_given.png}
  }
  \subfigure{%
    \includegraphics[width=.18\columnwidth]{figures/supplementary/frankfurt00001_059789_sp.png}
  }
  \subfigure{%
    \includegraphics[width=.18\columnwidth]{figures/supplementary/frankfurt00001_059789_gt.png}
  }
  \subfigure{%
    \includegraphics[width=.18\columnwidth]{figures/supplementary/frankfurt00001_059789_cnn.png}
  }
  \subfigure{%
    \includegraphics[width=.18\columnwidth]{figures/supplementary/frankfurt00001_059789_ours.png}
  }\\[-2ex]


  \subfigure{%
    \includegraphics[width=.18\columnwidth]{figures/supplementary/frankfurt00001_068208_given.png}
  }
  \subfigure{%
    \includegraphics[width=.18\columnwidth]{figures/supplementary/frankfurt00001_068208_sp.png}
  }
  \subfigure{%
    \includegraphics[width=.18\columnwidth]{figures/supplementary/frankfurt00001_068208_gt.png}
  }
  \subfigure{%
    \includegraphics[width=.18\columnwidth]{figures/supplementary/frankfurt00001_068208_cnn.png}
  }
  \subfigure{%
    \includegraphics[width=.18\columnwidth]{figures/supplementary/frankfurt00001_068208_ours.png}
  }\\[-2ex]

  \subfigure{%
    \includegraphics[width=.18\columnwidth]{figures/supplementary/frankfurt00001_082466_given.png}
  }
  \subfigure{%
    \includegraphics[width=.18\columnwidth]{figures/supplementary/frankfurt00001_082466_sp.png}
  }
  \subfigure{%
    \includegraphics[width=.18\columnwidth]{figures/supplementary/frankfurt00001_082466_gt.png}
  }
  \subfigure{%
    \includegraphics[width=.18\columnwidth]{figures/supplementary/frankfurt00001_082466_cnn.png}
  }
  \subfigure{%
    \includegraphics[width=.18\columnwidth]{figures/supplementary/frankfurt00001_082466_ours.png}
  }\\[-2ex]

  \subfigure{%
    \includegraphics[width=.18\columnwidth]{figures/supplementary/lindau00033_000019_given.png}
  }
  \subfigure{%
    \includegraphics[width=.18\columnwidth]{figures/supplementary/lindau00033_000019_sp.png}
  }
  \subfigure{%
    \includegraphics[width=.18\columnwidth]{figures/supplementary/lindau00033_000019_gt.png}
  }
  \subfigure{%
    \includegraphics[width=.18\columnwidth]{figures/supplementary/lindau00033_000019_cnn.png}
  }
  \subfigure{%
    \includegraphics[width=.18\columnwidth]{figures/supplementary/lindau00033_000019_ours.png}
  }\\[-2ex]

  \subfigure{%
    \includegraphics[width=.18\columnwidth]{figures/supplementary/lindau00052_000019_given.png}
  }
  \subfigure{%
    \includegraphics[width=.18\columnwidth]{figures/supplementary/lindau00052_000019_sp.png}
  }
  \subfigure{%
    \includegraphics[width=.18\columnwidth]{figures/supplementary/lindau00052_000019_gt.png}
  }
  \subfigure{%
    \includegraphics[width=.18\columnwidth]{figures/supplementary/lindau00052_000019_cnn.png}
  }
  \subfigure{%
    \includegraphics[width=.18\columnwidth]{figures/supplementary/lindau00052_000019_ours.png}
  }\\[-2ex]




  \subfigure{%
    \includegraphics[width=.18\columnwidth]{figures/supplementary/lindau00027_000019_given.png}
  }
  \subfigure{%
    \includegraphics[width=.18\columnwidth]{figures/supplementary/lindau00027_000019_sp.png}
  }
  \subfigure{%
    \includegraphics[width=.18\columnwidth]{figures/supplementary/lindau00027_000019_gt.png}
  }
  \subfigure{%
    \includegraphics[width=.18\columnwidth]{figures/supplementary/lindau00027_000019_cnn.png}
  }
  \subfigure{%
    \includegraphics[width=.18\columnwidth]{figures/supplementary/lindau00027_000019_ours.png}
  }\\[-2ex]



  \setcounter{subfigure}{0}
  \subfigure[\scriptsize Input]{%
    \includegraphics[width=.18\columnwidth]{figures/supplementary/lindau00029_000019_given.png}
  }
  \subfigure[\scriptsize Superpixels]{%
    \includegraphics[width=.18\columnwidth]{figures/supplementary/lindau00029_000019_sp.png}
  }
  \subfigure[\scriptsize GT]{%
    \includegraphics[width=.18\columnwidth]{figures/supplementary/lindau00029_000019_gt.png}
  }
  \subfigure[\scriptsize Deeplab]{%
    \includegraphics[width=.18\columnwidth]{figures/supplementary/lindau00029_000019_cnn.png}
  }
  \subfigure[\scriptsize Using BI]{%
    \includegraphics[width=.18\columnwidth]{figures/supplementary/lindau00029_000019_ours.png}
  }%\\[-2ex]

  \mycaption{Street Scene Segmentation}{Example results of street scene segmentation.
  (d)~depicts the DeepLab results, (e)~result obtained by adding bilateral inception (BI) modules (\bi{6}{2}+\bi{7}{6}) between \fc~layers.}
\label{fig:street_visuals-app}
\end{figure*}


\end{document}
\endinput
%%
%% End of file `sample-sigconf.tex'.
