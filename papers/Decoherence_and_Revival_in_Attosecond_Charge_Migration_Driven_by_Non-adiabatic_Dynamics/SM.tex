% Use only LaTeX2e, calling the article.cls class and 12-point type.

% Users of the {thebibliography} environment or BibTeX should use the
% scicite.sty package, downloadable from *Science* at
% www.sciencemag.org/about/authors/prep/TeX_help/ .
% This package should properly format in-text
% reference calls and reference-list numbers.

\documentclass[12pt]{article}
\usepackage{scicite}
\usepackage{graphicx}           % Include figure files
\usepackage{epstopdf}
\usepackage{subfigure}
\usepackage{graphicx}
\DeclareGraphicsExtensions{.eps,.ps}
\usepackage{color}
\usepackage{pstricks}           % e.g. colors can be used in the text
\usepackage{dcolumn}            % Align table columns on decimal point
\usepackage{bm}                 % bold math
\usepackage{enumerate}
\usepackage{rotating}
\usepackage{acronym}
\usepackage{scalefnt}
\usepackage{epsfig}
\usepackage{amsfonts}
\usepackage{cite}
\usepackage{graphicx}
\usepackage{braket}
\usepackage{amsmath}
\usepackage{amssymb}
\usepackage{mathrsfs}
\usepackage{bbold}
\usepackage{pdflscape}
\usepackage{url}
%\usepackage{caption}

% Use times if you have the font installed; otherwise, comment out the
% following line.

\usepackage{times}

\def\fnm#1{$^{#1}$}
\def\deg{$^{\rm o}$}
% The preamble here sets up a lot of new/revised commands and
% environments.  It's annoying, but please do *not* try to strip these
% out into a separate .sty file (which could lead to the loss of some
% information when we convert the file to other formats).  Instead, keep
% them in the preamble of your main LaTeX source file.


% The following parameters seem to provide a reasonable page setup.
\topmargin 0.0cm
\oddsidemargin 0.0cm
\textwidth 17cm
\textheight 21cm
\footskip 1.0cm


%The next command sets up an environment for the abstract to your paper.
\newenvironment{sciabstract}{%
\begin{quote} \bf}
{\end{quote}}


% If your reference list includes text notes as well as references,
% include the following line; otherwise, comment it out.

\renewcommand\refname{References and Notes}
\renewcommand{\theequation}{S\arabic{equation}}

% The following lines set up an environment for the last note in the
% reference list, which commonly includes acknowledgments of funding,
% help, etc.  It's intended for users of BibTeX or the {thebibliography}
% environment.  Users who are hand-coding their references at the end
% using a list environment such as {enumerate} can simply add another
% item at the end, and it will be numbered automatically.

\newcounter{lastnote}
\newenvironment{scilastnote}{%
\setcounter{lastnote}{\value{enumiv}}%
\addtocounter{lastnote}{+1}%
\begin{list}%
{\arabic{lastnote}.}
{\setlength{\leftmargin}{.22in}}
{\setlength{\labelsep}{.5em}}}
{\end{list}}


% Include your paper's title here

\title{Supplementary Material for
\\ \bf Decoherence and Revival in Attosecond Charge Migration Driven by Non-adiabatic Dynamics}

% Place the author information here.  Please hand-code the contact
% information and notecalls; do *not* use \footnote commands.  Let the
% author contact information appear immediately below the author names
% as shown.  We would also prefer that you don't change the type-size
% settings shown here.

\author
{Danylo Matselyukh$^1$, Victor Despr\'e$^2$, Nikolay V. Golubev$^3$, \\
Alexander I. Kuleff$^{2,\ast}$, Hans Jakob W{\"o}rner$^{1,\ast}$\\
 \\
\normalsize{$^{1}$ Laboratorium f\"{u}r Physikalische Chemie, ETH Z\"{u}rich, 8093 Z\"{u}rich, Switzerland}\\
\normalsize{$^{2}$ Theoretische Chemie, Physikalisch-Chemisches Institut (PCI), Universit\"at Heidelberg,}\\
\normalsize{69120 Heidelberg, Germany}\\
\normalsize{$^{3}$ Laboratory of Theoretical Physical Chemistry, Institut des Sciences et Ing\'enierie Chimiques,} \\
\normalsize{EPF Lausanne, 1015 Lausanne, Switzerland}\\
\normalsize{$^\ast$ \textbf{Corresponding authors. E-mails: alexander.kuleff@pci.uni-heidelberg.de, hwoerner@ethz.ch}}\\
\\
}


% Include the date command, but leave its argument blank.

\date{}

%%%%%%%%%%%%%%%%% END OF PREAMBLE %%%%%%%%%%%%%%%%


\begin{document}

% Double-space the manuscript.
\baselineskip24pt

% Make the title.
\maketitle

\noindent{\bf This PDF file includes:}\\
\noindent Experimental methods\\
Theoretical modelling \\
Figures S1 to S10 \\
Tables S1 to S4 \\
%References ...\\

\newpage
\tableofcontents

\renewcommand{\thefigure}{S\arabic{figure}}
\renewcommand{\thetable}{S\arabic{table}}
\renewcommand{\thesection}{S\arabic{section}}

\clearpage
\section{Experimental methods}

\subsection{Experimental setup}

An amplified carrier-envelope-phase (CEP) stabilized titanium:sapphire laser system (FEMTOPOWER V CEP) was used to deliver 25~fs, 1.5~mJ near-infrared (NIR) pulses centered at 790~nm with a full width at half maximum (FWHM) of 70~nm. These pulses were spectrally broadened in a 1~m-long, helium-filled hollow-core fibre. The gas pressure was adjusted such that the transmission efficiency of the fiber did not drop when helium gas was introduced, producing a spectrum spanning $>$350~nm with a central wavelength of 780~nm. Due to the third-order spectral phase imprinted onto the pulse by the broadening process \cite{jarque2018}, the pulse is compressible to 5.2~fs using broadband chirped mirrors and glass wedges (measured using a home-built, second-harmonic d-scan device). 

Once compressed, the $>$500~$\mu$J pulse was used to drive high-harmonic generation in a finite gas target filled with helium. A parabolic collimating mirror with a hole was used to separate the transmitted fundamental beam from its high harmonics, reflecting the diverging driving field while allowing the high-harmonic attosecond pulses to pass through. A 200~nm-thin zirconium foil was used to isolate the soft-X-ray pulse from the weak residual driving field before recombination. To avoid damaging the gold-coated toroidal refocusing mirror with the shorter-wavelength portion of the driving field, the filtering was performed directly in front of the toroid. In addition, two Nb$_2$O$_5$-coated mirrors were employed as partial beamsplitters to lower the driving field intensity, preventing the Zr filter from melting. A second parabolic mirror with a center hole placed after the toroidal allowed the attosecond high-harmonic pulse to pass through, collinearly recombining it with the recycled driving field while focusing the latter. The delay between the pump and the probe pulse is actively stabilised with a He-Ne interferometer and additionally monitored with white-light interferometry resulting in a delay jitter of $<$25~as. Additional details on the basic experimental setup are given in \cite{huppert15a}. 

The two pulses were focused into a 1~cm-long gas cell which was filled with the target gas. The spatial overlap between the pulses was optimized by maximizing the absorption signal from strong-field-ionized krypton. Time-zero was defined using the onset of the Kr$^+$ M$_{4,5}$-edge absorption signal. By recycling the driving pulse after high-harmonic generation, any additional nonlinear process that occurs inside the HHG target affects not only the generated harmonics but also the NIR pump pulse, which contributes to achieving an optimal temporal stability between the strong-field pump and the soft-X-ray (SXR) probe pulse. 

The transmitted spectrum was recorded using a CCD-based soft-X-ray spectrometer with sub-100~meV resolution at 100~eV, achieved by using a backlit in-vacuum CCD camera with a flat-field grating. The background signal from the residual driving field was minimized by using another 200-nm-thin Zr filter preceded by a pinhole which spatially removed a large portion of driving field, thus protecting the metallic filter from damage.

\subsection{Data acquisition and processing}

The active interferometric stabilization of our beamline allows the delay to be set to a specific value repeatedly. As such, each ATAS measurement performed on our beamline can include multiple scans of the same delay time resulting in a higher sensitivity. For the measurement presented in this work, two scans were preformed over a delay range of $-20$~fs to 70~fs with a step size of 250~as. For each scan and delay, exposures of 2 seconds were taken with the pump arm blocked and then unblocked. Before the measurement was started, a set of reference measurements were performed to establish the source of the different background signals and to allow for their subtraction. First, both the pump and probe arms were blocked and the dark counts of the CCD camera were captured. This exposure was used as a background for frames with the NIR arm blocked. Next, the probe arm was unblocked, providing a reference SXR spectrum to allow for the absolute absorption to be computed. Finally, since the pump pulse is sufficiently short and intense to produce harmonics in the target gas, only the pump arm was unblocked and the target filled with gas to provide a background for measurements with the pump arm unblocked.

To maximize the resolution of our spectrometer, as required to resolve the fine structure of the silane L$_{2,3}$ absorption spectrum, the curved nature of the spectral images produced by the flat field grating needs to be compensated for \cite{harada99a}. This is done by taking an average of the reference measurements over all delays and analyzing each row of the resulting reference image individually. For each row, the centre of mass of different absorption features is calculated. A cubic fit is performed to determine the position of the absorption features as a function of the vertical position on the CCD. This function is then used to shift the rows with respect to each other such that the absorption features line up with each other, allowing for a corrected, higher resolution spectral flux to be calculated. The horizontal pixel axis is converted into the photon energy axis by adjusting parameters in the grating equation until optimum overlap is achieved with the literature absorption spectrum of silane \cite{puettner97a}, resulting in Figure S1 (and Fig.~1C in the main text).

\begin{figure}
\includegraphics[width=0.85\textwidth]{Figures/SupFig0.pdf}
\label{figs1}
\caption{\textbf{Attosecond soft-X-ray spectra and absorbance of silane.} The black dotted and solid lines are the measured probe-pulse spectra before and after the silane is introduced into the target gas cell, respectively. The absorbance of the unpumped silane gas can therefore be calculated and is shown as the purple line. 
%The wave-like structures in the spectrum are artefacts due to CEP fluctuations of the IR driving field. 
As the CCD detector is made of silicon, its quantum efficiency drops by a factor of 2 above $\sim$100~eV which is evident in the black lines. This effect is eliminated when calculating the absorbance and, therefore, does not affect the results shown in all other figures, including the main text.}
\end{figure}

Despite the CEP stabilization of the oscillator and amplifier, slight fluctuations in the pulse energy can still lead to fluctuations in the CEP of the few-cycle pulse, resulting in significant fluctuations of the probe pulse spectrum, especially in its cut-off region. These fluctuations happen on a sub-second time scale, therefore, subsequent exposures that measure the pump-induced change in absorption often have different spectra. The spectral signature of these CEP fluctuations is a shift in energy of the harmonic features in the spectrum, as well as a global change in the intensity of the harmonic flux. If left untreated when evaluating the absorption, the result, visible in Fig.~S2, is a random and significant fluctuation of the baseline around the harmonic cut-off, which transforms into a checkerboard-like pattern for photon energies that fall lower in energy (in our case for energies below 105 eV). Various solutions to this problem have been presented in the literature, making use of Fourier filtering \cite{stooss2019}, laser intensity referencing \cite{Volkov2019}, or taking advantage of spectral regions that are expected to not exhibit any changes in optical density (OD) \cite{timmers19}. 

\begin{figure}
\includegraphics[width=1.0\textwidth]{Figures/SupFig1.pdf}
\label{figs2}
\caption{\textbf{Overview of SVD filtering.} Unfiltered $\Delta$OD transient spectrum as obtained from the experiment (left), $\Delta$OD transient spectrum as obtained by removing four singular vectors among the leading ten singular vectors (center), and the residual of the filtering corresponding to the removed singular vectors (right).}
\end{figure}

In this work, we present an approach which takes advantage of the multiple scans that can be performed with our instrument. Based on the random nature of the fluctuations, we apply singular-value decomposition (SVD) to decompose our ATAS spectrum into orthogonal singular vectors which can be processed independently. The SVD can be seen as grouping correlated changes of the OD into the same singular vectors. Any changes to the measured OD that are caused by our pump pulse will have a well-determined relationship to the delay that shall be identical in subsequent measurements, while changes to the OD that are due to CEP fluctuations will instead have random fluctuations with respect to the delay. By performing the SVD on a matrix formed by concatenating subsequent scans rather than averaging over them, the singular delay vectors of different scans can be compared and their covariance can be evaluated. By setting a cutoff for the minimum allowed covariance, only singular vectors that have significant dependence on the delay can be selected and through simple matrix multiplication, be recombined to form an SVD-filtered ATAS spectrum which can now be averaged over the different scans. Noise has the effect of mixing the different singular vectors with each other, especially for those that have a small singular value, making it difficult to classify these with a simple covariance analysis. As a result, we have only filtered the first ten singular vectors, of which four were removed, and kept all singular vectors up to the 200th to avoid removing singular vectors with a small weighting in the ATAS spectrum but a significant effect. The result and the residual are shown in Fig.~S2. Once the singular vectors are obtained, they form an orthogonal space and therefore can be applied to any other spectrum. In this work, we have also performed the same processing to our static spectrum, shown in the main text (Fig.~1C), despite only having a single exposure of this spectrum. This technique has the advantage that no assumption is made about the spectral shape of the fluctuations nor the spectral regions that are expected to exhibit a signal, allowing the method to be applied to any system. In addition, all operations are linear which allows for rapid processing and transparent interpretation. However, one must be careful not to over-interpret the physical meaning of the individual singular vectors. 

\subsection{Time-Frequency analysis}

As is the case for many pump-probe techniques, ATAS experiments can invert the role of the pump and probe pulses in order to investigate either strong-field initiated dynamics, the focus of this paper, or instead, the interactions of core-excited states with a VIS/IR field. It is not possible to only induce one of these phenomena and, as a result, ATAS measurements often exhibit both. Due to the finite duration of the pulses, signals arising in the region of temporal overlap can originate from either phenomenon and require detailed analysis to classify and interpret them. As both phenomena can produce rapidly oscillating signals with few-femtosecond periods, a detailed time-frequency analysis is best able to separate the contributions. 

\begin{figure}
\includegraphics[width=1.0\textwidth]{Figures/SupFig3.pdf}
%\label{figs3}
\caption{\label{Gabor}\textbf{Time-frequency analysis of the attosecond transient-absorption spectra.} The top-left panel shows the pseudo Wigner-Ville distribution of the $\Delta$OD at 105.55~eV. The vertical white lines indicate the three frequencies at which Gabor filters are applied to the entire transient-absorption data set. The results of these Gabor filters are shown in the three panels below. The amplitudes are shown above the phases, sharing a common color scale. Amplitude thresholding has been applied to the phases to ease the identification of the phase of the relevant signals. For completeness, the top right panel shows the unthresholded phase of the 0.72~PHz Gabor filter, exhibiting a pronounced change in structure for delays above 10~fs (highlighted by a white horizontal line). 
}
\end{figure}

The top-left panel of Figure~\ref{Gabor} shows a pseudo Wigner-Ville distribution of the $\Delta$OD signal at 105.55~eV -- the region of the spectrum exhibiting the largest-amplitude high-frequency oscillations. The strongest signal is found at negative time delays with a central frequency of 0.65~PHz. We can investigate the spectral distribution of this signal by applying a Gabor filter at 0.65~PHz on the transient spectrum (lower left panel of Fig.~\ref{Gabor}). This frequency is present over a broad spectral range at negative time delays, without exhibiting any distinct spectral peaks or large phase variations. These characteristics allow the assignment of this signal to the interaction of core-excited silane with the strong VIS/IR pulse \cite{chen13}. The coupling of light-induced sates (LIS) and/or dark states, with the bright states excited by the SXR pulse, results in a strong modulation of the final population of the core-excited state at twice the laser-field frequency. Although is 0.65~PHz is the second harmonic of a frequency that is red-shifted compared to the measured central frequency of our pump spectrum the significant third-order phase, introduced by the hollow-core fibre compressor and measured with a home-built second harmonic dispersion-scan setup (see Fig.~\ref{fig:FROG}E and F), results in a pre-pulse shoulder which exhibits a central frequency of 0.33~PHz, matching the frequency of the signal. 

From further examination of Fig.~\ref{Gabor}, we find, at positive time delays, oscillations with frequencies $>$0.7~PHz. Part of these signals originate from the same core-excited state -- IR coupling as discussed before, producing spectrally broad peaks with slowly varying phase. These signals dominate the Gabor phase up to a delay of 10~fs, however, the Gabor amplitude starts to exhibit narrow peaks already at these early delays. These narrow peaks are assigned to the excitation of a coherent superposition of electronic states into a common core-excited final state. As the pump and probe pulses lose temporal overlap and the initial decoherence of the electronic superposition sets it, both spectrally broad and narrow signals disappear. At this delay, the Gabor phase undergoes a significant change in structure. Due to the loss of pulse overlap and the IR now preceding the soft-X-ray pulse, no coupling of the core-excited states with the VIS/IR pulse is possible, resulting in the complete disappearance of the spectrally-broad signatures. This, in turn, makes the quantum beats of the electronic coherence the sole source of high-frequency oscillations in the transient spectrum, allowing the phase profile of the quantum beats to be observed. The phase exhibits two regions of monotonic (but step-like) increase/decrease, matching well the results of our ATAS simulations (see Fig.~5 of the main text), and supporting the assignment of the 0.7-0.8~PHz oscillations between 35 and 60~fs to the revival of the electronic coherence. As a result, the time window chosen to best isolate and represent the charge migration with minimal contribution of the LIS coupling was 10 -- 70~fs and is used for Fig.~4 of the main text. 

\subsection{Characterization of the pump pulse}

Unlike optical parametric amplification, HHG is an inefficient process that requires very high electric field intensities. As a result, multiple strategies have been devised to maximize the experimental conversion efficiency by recycling the optical driving field including HHG in enhancement cavities and split-mirror attosecond interferometers \cite{Bernhardt2012,Kienberger2004}. In the case of our experiment, we have made use of a rather unique technique which combines the efficiency of the split-mirror recycling with the advantages that separate pump and probe beamlines offer, which are normally achieved by placing beamsplitters before the HHG target. By placing a parabolic mirror with a drilled hole after the HHG target, the pump is split from the probe, simultaneously recycling the VIS/NIR while also allowing the beams to follow different paths where it becomes easier to spectrally shape the pulses. 

\begin{figure}
    \centering
    \includegraphics[width = 0.85\textwidth]{Figures/SupFig5.pdf}
    \caption{\textbf{Effect of HHG on the transmitted pump pulse}  A,B) TG-FROG measurements of the pump pulse with (A) or without (B) helium in the gas cell, taken after transmission through a 1~mm window (exit of the vacuum chamber). The difference of the two measurements is shown in C. D) effect of HHG on the spectrum of the transmitted IR few-cycle pulse. The the grey spectrum was measured without helium in the HHG gas cell, whereas the red was recorded with helium present. The effect, shown in black by taking a difference of the two spectra, is very small and is dominated by a $\sim$4~nm blueshift, consistent with panel C. E) result of second-harmonic-generation dispersion-scan (d-scan) measurement of the laser pulse performed in front of the experimental chamber. The residual third-order phase can be seen from the slanted SHG signal as well as in the reconstructed phase in F which also shows the reconstructed spectral intensity of the 5.2~fs pulse. The inset shows the temporal intensity profile of the pulse.}
    \label{fig:FROG}
\end{figure}

However, such a design also combines the disadvantages of the split mirror design (the creation of an annular pump-beam), and the separated beampaths (lower temporal stability). The issue of temporal delay between the pump and probe beams is resolved by employing a dual optical interferometer \cite{Huppert2015}. In this section we shall therefore discuss the effect of the HHG process and the drilled mirror on the pump beam. 

Unlike optical vortices, which exhibit a node at the centre of the beam's transverse mode in both the far and the near field, annular beams, which are created by applying a binary spatial mask to a regular Gaussian-like beam, do not exhibit a central node when focused. Therefore, the annular beam created by our drilled mirrors still exhibits a clean focus in our transient-absorption target, suffering only from a slight drop in maximum intensity and the generation of weak radial wings (using the convolution theorem we can think of this as arising due to the convolution of the unmasked focal spot with a sinc$^2$ function -- the Fourier transform of a top-hat function). A detailed analysis of these effects and their experimental application can be found in \cite{Gaumnitz2018}. The beam sizes, focal lengths, and mask sizes are almost identical to those used in our experiment.

The possible perturbation of the VIS/IR pulse in the HHG step is therefore the main source of concern. Although the process of HHG itself does not play an important part in reshaping the VIS/IR pulse, laser-plasma interactions of the driving field with free charges generated through strong-field ionization in the target are known and even exploited to increase the efficiency of HHG through phase matching. Nevertheless, it has also been shown that in cases of plasma defocusing intense enough to cause significant changes to the beam profile, HHG is not phase matched and no longer occurs \cite{Lai2011}. The use of 800~nm pulses and helium gas targets both reduce the rate of ionisation compared to longer wavelength drivers and other gases, therefore, such effects are most unlikely in our experiment. This is in agreement with the fact that no change in beam profile at the transient-absorption target is observed when performing HHG in helium.

While the spatial reshaping of the pulse by HHG is always weak, its spectral reshaping happens more readily and might also influence the results of our ATAS experiment. When HHG is performed in argon, the broadening of the spectrum is observed through self-steepening. On the other hand, helium, which is significantly harder to ionize and has an order-of-magnitude lower third-order susceptibility, exhibited no appreciable change in spectrum when the HHG gas was cycled on and off (see Fig.~\ref{fig:FROG}). The total change amounted to a blueshift of the spectrum by only $\sim$4~nm. Nonetheless, to further ensure that our pulses were not being modified by the HHG process, we performed transient-grating frequency-resolved optical-gating (TG-FROG) measurements on our pump pulses just before the transient-absorption target. Due to the fact that the pulse needs to be extracted from the experimental vacuum chamber in order to perform the TG-FROG measurement, it becomes positively chirped while passing through a window and therefore produces a rather complex FROG trace as shown in Fig.~\ref{fig:FROG}. When the HHG gas is cycled on and off, no significant changes to the FROG trace occur and only calculating the difference of the two FROG traces reveals any (Fig.~\ref{fig:FROG}A,B). In the difference plot (Fig.~\ref{fig:FROG}C) we see a slight intensity redistribution, however, the rich structure caused by the positive chirp remains completely unchanged, indicating that no significant changes to the spectral phase have occurred (Fig.~\ref{fig:FROG}D). For small redistributions of spectral intensity, one can expect negligible changes to the location of nodes in a TG-FROG trace, which is indeed the case in our results. We therefore conclude that the transmission of the few-cycle pump pulse the HHG medium has a negligible effect on its temporal structure. Finally, we also show a d-scan measurement and reconstruction of the few-cycle pulse performed in front of the vacuum chamber in Fig.~\ref{fig:FROG}E,F. The d-scan reconstruction in panel F agrees well with the measured spectrum in panel D. The temporal profile of the reconstructed pulse is shown in panel F with its 5.2~fs FWHM duration. The weak prepulse arises from the residual third-order chirp and has no influence on the dynamics reported in the main text, as discussed in section S1.3.

\section{Theoretical modelling}

\subsection{Excitation spectrum}

The UV absorption spectrum of silane was computed at the EOM-CCSD/aug-cc-pVTZ level and is presented in Fig.~\ref{UVspec}, the list of the states can be found in Table~\ref{Tab_excitation}. Each line corresponds to an excited eigenstate of the system with an intensity determined by the respective transition-dipole moment from the electronic ground state. The spectrum is normalized to the strongest transition in this energy range. Due to the tetrahedral symmetry of silane, many of the states are doubly or triply degenerate. In addition, there are a number of dark states in this energy range, which cannot be populated directly from the ground state, since they have very small or zero (by symmetry) transition-dipole moments. In the experiment, however, the excitation is performed by a strong IR field, i.e., through a multiphoton absorption, and many of the single-photon-forbidden transitions will be nominally allowed. Nevertheless, we do not expect that the spectrum will change substantially, see also Sec.~\ref{Sec:SFE}. 

\begin{figure}
\begin{center}
\includegraphics[width=14cm]{Figures/UV_spec_V2.pdf}
\end{center}
\caption{\label{UVspec}\textbf{UV absorption spectrum of silane computed at EOM-CCSD/aug-cc-pVTZ level}. The intensity of each line reflects the square of the transition-dipole moment associated with the given state. The positions of the dark states obtained in this energy region are indicated with dashed lines and the degeneracy of each state is given in parentheses. The spectrum has been convoluted with a Lorentzian with FWHM of 50~meV to simulate the spectral broadening. The states that have been identified as responsible for the electronic coherence observed in the experiment are depicted in red.}
\end{figure}

The spectrum was used to identify the states that could be responsible for the 0.72-0.765~PHz signal observed in the experiment. This signal has a period of 1.31-1.39~fs and thus corresponds to an energy splitting of 2.96-3.14~eV between the coherently populated states. Moreover, our analysis (see main text) shows that the observed beating is between a lower-lying valence state and an upper state with a Rydberg character from the $s$-series. A pair of states that satisfy these conditions is depicted in red in Fig.~\ref{UVspec}. The calculated energy gap between these two states is 2.83~eV, in good agreement with the experimental value. 

\begin{table}
\centering
\caption{\label{Tab_excitation}Electronic states present in the excitation spectrum shown in Fig.~\ref{UVspec}.}
\vspace{0.3cm}
\begin{tabular}{cccccc} 
 \hline \hline
 Numbering & Energy / eV & Degeneracy & Symmetry & Leading excitation 2t$_2\rightarrow$ \\ 
 \hline
 1 & 9.34   & 3 & T$_1$ & 3t$_2$ \\ 
 2 & 9.46   & 3 & T$_2$ & 4a$_1$ \\
 3 & 9.58   & 2 & E     & 3t$_2$ \\
 4 & 9.95   & 1 & A$_1$ & 3t$_2$ \\ 
 5 & 10.43  & 3 & T$_2$ & 3t$_2$ \\
 6 & 10.70  & 3 & T$_2$ & 4p (t$_2$) \\
 7 & 10.85  & 1 & A$_1$ & 4p (t$_2$) \\
 8 & 10.88  & 3 & T$_1$ & 4p (t$_2$) \\
 9 & 10.89  & 2 & E     & 4p (t$_2$) \\
 10 & 11.27 & 3 & T$_1$ & 3d (e) \\
 11 & 11.50 & 3 & T$_2$ & 3d (e) \\
 12 & 12.15 & 3 & T$_1$ & 3d (t$_2$) \\
 13 & 12.21 & 2 & E     & 3d (t$_2$) \\
 14 & 12.29 & 3 & T$_2$ & 5s (a$_1$) \\
 15 & 12.32 & 1 & A$_1$ & 3d (t$_2$) \\
 16 & 12.54 & 3 & T$_2$ & 3d (t$_2$) \\
 \hline
\end{tabular}
\end{table}

\subsection{Potential-energy surfaces and vibronic-coupling Hamiltonian}

In order to study the evolution of the system after the coherent population of this pair of states, we performed a quantum dynamics simulation taking into account the coupled motion of both electrons and nuclei. For this purpose, we used the well established and very powerful method of vibronic-coupling Hamiltonian \cite{koppel1984multimode}. In this approach, a Taylor expansion around a particular geometry point (usually the Franck-Condon point) is performed within a basis of quasi-diabatic states. The polynomial expansion is performed in terms of dimensionless normal coordinates, $Q$. The vibronic-coupling Hamiltonian can then be written as
%
\begin{equation}
\mathbf{H} = \pmb{\tau}_N + \pmb{\nu}_0 + \mathbf{W},
\end{equation}
%
where $\pmb{\tau}_N$ and $\pmb{\nu}_0$ denote the diagonal kinetic and the ground-state potential-energy matrices, respectively, while the matrix $\mathbf{W}$ contains the \textit{diabatic} states and the couplings between them.

Using a harmonic approximation for the vibrational modes, $\pmb{\tau}_N$ and $\pmb{\nu}_0$ can then be written as
%
\begin{eqnarray}
\pmb{\tau}_N = -\frac{1}{2}\sum_i\omega_i\frac{\partial^2}{\partial Q_i^2}\mathbf{1}, \\
\pmb{\nu}_0 = \frac{1}{2}\sum_i\omega_i Q_i^2\mathbf{1},
\end{eqnarray}
%
with $\omega_i$ being the frequency of mode $i$, and $\mathbf{1}$ denoting the unit matrix. In the present study, we used a quadratic model for the potential matrix $\mathbf{W}$, i.e., the Taylor expansion is truncated after the quadratic term, and, therefore, the matrix elements take the form
%
\begin{eqnarray}
W_{jj}&=&E_j+\sum_{i} \kappa_i^jQ_i+\frac{1}{2}\sum_{i}\gamma_i^jQ_i^2 \\
W_{jk}&=&\sum_{i}\lambda_i^{j,k}Q_i
\end{eqnarray}
%
In the above expressions, $E_j$ is the vertical excitation energy of state $j$, $\kappa_i^j$ and $\gamma_i^j$ are the linear and quadratic coupling parameters, respectively, of state $j$ for normal mode $i$, and $\lambda_i^{j,k}$ is the linear coupling parameter between states $j$ and $k$ by the normal mode $i$. The possible linear coupling parameters are determined through symmetry selection rules \cite{worth2004beyond}. The values of these parameters can be obtained through least-squares fitting to the \textit{adiabatic} potential energy surfaces (PES), obtained by solving the electronic eigenvalue problem at fixed nuclear geometry along the vibrational modes. With such a constructed vibronic-coupling Hamiltonian, we can propagate the nuclear wavepackets created by multiphoton excitation of the molecule by the strong IR pulse.

\begin{table}
\centering
\caption{\label{Tab1}Vibrational modes of silane.}
\vspace{0.3cm}
\begin{tabular}{ccc} 
 \hline\hline
 Energy / eV & Symmetry & Label \\ 
 \hline
 0.114 & t$_2$ & $\nu_1$ \\ 
 0.121 & e & $\nu_2$ \\
 0.276 & a$_1$ & $\nu_3$ \\
 0.277 & t$_2$ & $\nu_4$ \\ 
 \hline
\end{tabular}
\end{table}

The adiabatic PES along all the vibrational modes were computed at EOM-CCSD/aug-cc-pVTZ level of theory. The vibrational modes of silane are listed in Table~\ref{Tab1} and the PES along these modes are plotted in Fig.~\ref{pes}. 

\begin{figure}[ht!]
\centering
\includegraphics[width=0.7\textwidth]{Figures/PES.pdf}
\caption{\label{pes}\textbf{PES along the different vibrational modes of silane}. For the description of the modes, see Table~\ref{Tab1}. The color code is the same as in the main text, i.e., blue for the A-state, purple for B, and red for C. }
\end{figure}

The aim of our vibronic-coupling Hamiltonian model is to obtain a good description of the non-adiabatic dynamics initiated by the population of the pair of triply degenerate states identified previously and marked in red in Fig.~\ref{UVspec} (states B and C). Multiple tests, including different numbers of states and modes, were performed. We found that, due to its Rydberg character, the higher-lying triply degenerate state (C) is very weakly coupled to the other states and thus can be treated as isolated. This is in contrast to the lower-lying group of valence states. Due to the strong non-adiabatic coupling between these states, a good representation of the lower triply degenerate state (B) in the diabatic basis requires the inclusion of several other states, energetically lying both below and above it. Those states are depicted in blue (state A) and dark grey (others) in Fig.~\ref{pes}. As some of the states are degenerate, the final model includes 15 states, which are listed in Table~\ref{Tab2}. Due to the shapes of the PES along modes $\nu_1$ and $\nu_2$, it was not possible to obtain a satisfactory vibronic-coupling model accounting also for these vibrations and thus they were excluded from the vibronic-coupling Hamiltonian used to analyze the non-adiabatic dynamics of silane. As every degree of freedom can be a source of decoherence, the exclusion of some modes may in principle lead to an overestimation of the coherence time. As seen from Table~\ref{Tab1}, however, the frequencies of modes $\nu_1$ and $\nu_2$ are more than two times lower than $\nu_3$ and $\nu_4$ and thus are expected to have a much smaller influence on the dephasing of the initially created electronic coherence compared to modes $\nu_3$ and $\nu_4$. As we will see, this hypothesis is confirmed by the very good agreement of the theoretical results with the experimental observations. The vibronic-coupling Hamiltonian used to simulate the non-adiabatic dynamics triggered by the IR pulse, therefore, includes 15 states and 4 vibrational modes, the totally symmetric mode $\nu_3$ and the triply degenerate mode $\nu_4$.

\begin{table}
\centering
\caption{\label{Tab2}Electronic states present in the vibronic-coupling Hamiltonian and their calculated vertical excitation energies.}
\vspace{0.3cm}
\begin{tabular}{cccccc} 
 \hline \hline
 Numbering & Energy / eV & Degeneracy & Symmetry & Leading excitation 2t$_2\rightarrow$ & Label used\\ 
 \hline
 1 & 9.34   & 3 & T$_1$ & 3t$_2$ & A\\ 
 2 & 9.46   & 3 & T$_2$ & 4a$_1$ & B\\
 3 & 9.58   & 2 & E     & 3t$_2$ & \\
 4 & 9.95   & 1 & A$_1$ & 3t$_2$ & \\ 
 5 & 10.43  & 3 & T$_2$ & 3t$_2$ & \\ 
 14 & 12.29 & 3 & T$_2$ & 5s(a$_1$) & C\\ 
 \hline
\end{tabular}
\end{table}

\subsection{Non-adiabatic dynamics and construction of transient-absorption spectrum}

The vibronic-coupling Hamiltonian described above was used to propagate the nuclear wave packets on the coupled manifold of diabatic electronic states with the help of the Multi-Configuration Time-Dependent Hartree (MCTDH) method \cite{meyer1990multi}. MCTDH is a powerful grid-based method for numerical integration of the time-dependent Schr\"odinger equation, particularly suitable for treating multidimensional problems \cite{meyer2009multidimensional}. The Heidelberg MCTDH package \cite{mctdh} was used for the present calculations. 

Two types of calculations were performed. In the first type of calculations, the initial state was chosen to be a coherent superposition of the states denoted as B and C in Table~\ref{Tab2}. The results of these calculations are shown in Figs.~3 and 4 of the main text and used to analyze and interpret the experimental observations. To account for the possible population of other states during the multiphoton excitation step, a second type of calculations were performed, whereby all 15 states shown in Table~\ref{Tab2} were equally populated. The results of the latter calculations will be discussed below.

The coherence between each pair of electronic states can be extracted from the time-dependent overlap of the nuclear wave packets evolving on the corresponding PES (see, e.g., Ref.~\cite{despre2018charge})
%
\begin{equation}
\label{eq:coherence}
\chi_{jk}(t)=\int d \mathbf{Q}\, \chi^*_j(\mathbf{Q},t) \chi_k(\mathbf{Q},t),
\end{equation}  
%
where $\chi_j(\mathbf{Q},t)$ is the time-dependent nuclear wave packet, corresponding to state $j$, and $\mathbf{Q}$ denotes the nuclear degrees of freedom. The quantities $\chi_{jk}(t)$ are also important ingredients for computing the attosecond transient absorption spectrum, needed for having a direct comparison between theory and experiment. Using time-dependent perturbation theory and the Condon approximation, the ATAS can be obtained via the following expression for the transient absorption cross section (for a complete derivation, see the Supplemental Material of Ref.~\cite{golubev2020ATAS}): 
%
\begin{equation}
\label{eq:ATAS_cs}
\begin{split}
  \sigma (\omega,\tau) = \frac{4\pi\omega}{c} 
  & \text{Im} \sum_j \sum_k 
    \chi_{jk}(\tau)
    \sum_f \langle \Phi_j | \hat{\mu} | \Phi_f \rangle
           \langle \Phi_f | \hat{\mu} | \Phi_k \rangle \\ & \times \left(
      \frac{1}{\tilde{E}_f - E_j - \omega}
    + \frac{1}{\tilde{E}^*_f - E_k + \omega}
    \right),
\end{split}
\end{equation}
%
where $\omega$ is the photon energy, $\tau$ is the delay between pump and probe pulses, $c$ is the speed of light in vacuum, $\langle \Phi_j | \hat{\mu} | \Phi_f \rangle$ and $\langle \Phi_f | \hat{\mu} | \Phi_k \rangle$ denote transition-dipole matrix elements between initial $\{j,k\}$ and final $f$ electronic states with corresponding energies $E_{\{j,k,f\}}$ computed at the equilibrium nuclear geometry $\mathbf{Q}=0$. When $j=k$, the nuclear wave-packet overlaps $\chi_{jk}(\tau)$ give the populations of the initial electronic states. The energies of the final states are chosen complex, $\tilde{E}_f = E_f - i\frac{\Gamma}{2}$ to account for the spectral broadening. 

\begin{figure}[h!]
\includegraphics[width=1.0\textwidth]{Figures/SupFig2.pdf}
%\label{figs3}
\caption{\label{comparison}\textbf{Comparison between Fourier-filtered $\Delta$OD and MCTDH calculations.} The experimental transient spectrum has been Fourier-filtered to suppress signals resulting from purely vibrational dynamics and is shown in the first two panels. The signal frequencies left after the application of the Fourier filter are displayed in the top-right; the second panel has been passed through a narrower acceptance range filter and thus exhibits less noise. The right-most panel shows a simulated transient spectrum based on the results of the MCTDH calculations. The average of the calculated spectra at each photon energy has been subtracted from the simulation to allow for an easier comparison between the three figures.}
\end{figure}

In order to obtain the ATAS, the transitions between the initial and final states need to be determined. The initial states consists of the states present in the vibronic-coupling Hamiltonian that get a significant population during the MCTDH propagation. As the ground-state equilibrium geometry of the neutral system is assumed for the simulation of the ATAS, the degeneracy is not lifted and a single initial state can be associated to each group of degenerate states. Therefore, the three initial states included in the simulation of ATAS are the ones located at 9.34~eV (A), 9.46~eV (B), and 12.29~eV (C), respectively. As core excitations from Si(2\textit{p}) are used in the experiment to probe the non-adiabatic dynamics in those states, the set of final states is formed by core excitations from Si(2\textit{p}). %Assuming that the valence excitation has a negligible impact on the core transition, and focusing on the specific transition at 105.5~eV, the final states are obtained by shifting each initial one by 105.5~eV. Therefore, the ensemble of final states consists of three states located at 114.84 eV, 114,96 eV and 117.79 eV.



As we were not able to compute the transition-dipole moments between the initial valence-excited states and the doubly-excited final states, their values were determined using the experimental results. We want to point out, however, that the particular values that were chosen will not have an impact on the probed dynamics in the valence-excited region, but will only change the respective intensities of the different lines and the energy they appear at in the transient spectrum. The obtained theoretical ATAS is plotted in the right panel of Fig.~\ref{comparison}. The experimental results are presented in the two left-hand plots of the figure for comparison. We see that the ATAS trace reproduces the main experimentally observed features well. The early-time coherence, manifesting itself as a quantum beating in the absorption of the initially populated states, is lost in about 10~fs and a revival is observed at around 50~fs. 

\begin{table}
\centering
\caption{\label{Tab3}Final electronic states $f_i$ and transition dipoles from the valence-excited states (A,B,C) used for computing ATAS. The negative signs in the right-hand column indicate which transition dipole elements were negated to reproduce the correct phase relationship between quantum beats. The units of the transition dipoles are arbitrary and only the relative signs and magnitudes matter.}
\vspace{0.3cm}
\begin{tabular}{ccccc} 
 \hline \hline
 Energy / eV & A $\to$ F$_i$ & B $\to$ F$_i$ & C $\to$ F$_i$ \\ 
 \hline
 105.35      & 1.0       & 0.0 &    0.7 \\
 105.60      & 0.7       & 0.0 & -0.5 \\
 105.85      & 1.0       & 0.0 &    0.7 \\
 \hline
 105.55      & 0.0       & 1.0 &    0.7 \\
 105.80      & 0.0       & 0.7 & -0.5 \\
 106.05      & 0.0       & 1.0 &    0.5 \\
 106.25      & 0.0       & 0.7 & -0.5 \\
 106.60      & 0.0       & 0.7 &    0.5 \\
 106.75      & 0.0       & 0.7 & -0.5 \\
 107.05      & 0.0       & 0.5 & -0.5 \\
 \hline
\end{tabular}
\end{table}

Our analysis shows that the second coherence is created after the wave packet propagating on the initially populated state at 9.46~eV (B) is transferred to the lower-lying dark state (A) through conical intersections (CI) located close to the Franck-Condon point. These results are particularly important, as conical intersections were expected to destroy the electronic coherence \cite{arnold2017electronic}. We see, however, that in this case, not only is the initial coherence not destroyed by the splitting of the wave packet at the CI, but a new one is created between the state populated through the CI and the upper Rydberg-like state (C). These results show that the effect of conical intersections on the coherence is case-dependent and can destroy \cite{arnold2017electronic}, conserve, and even create \cite{kowalewski15a} new coherences in the system. 

As the exact state of the system after the pump-pulse is unknown and, in addition, may vary from shot to shot and over the interaction volume, we found it important to verify that the MCTDH results were robust with respect to changes to the initial conditions. Robustness to the initial population of the states is established by performing MCTDH simulations in which all states (and degenerate sub-states) in the model are initially equally populated. As is the case for the simulations shown in the main text (Figs. 3 and 4), the robustness with respect to the initial phase of the wavepackets is also simultaneously investigated by performing ten separate simulations in which the initial phase is randomized. Randomizations that show an initial coherence drop of $<$20\% are selected for propagation. This criterion restricted the total range of the phase randomisation to 600~mrad. The result of the ten simulations are shown in Fig~\ref{robustness} where they are coherently averaged.

\begin{figure}[h!]
\includegraphics[width=1.0\textwidth]{Figures/SupFig6.pdf}
\caption{\label{robustness}\textbf{Influence of other electronic states and their initial phases on the electronic dynamics} Here, we present the results of the first limiting case (mentioned in the main text), in which all states were initially populated. The robustness of the predicted coherence and population dynamics with respect to variations of the initial phase was investigated by averaging the results over 10 phase-randomised simulations. A: Coherence between states A (blue), B (purple), 3 (grey), or 4 (green) with state C. B: Simulated $\Delta$OD at 105.5 eV obtained by coherently adding all coherences between the valence states and the C state. C: Population dynamics of states A (blue), B (purple), 3 (grey), and 4 (green).}
\end{figure}

Figure~\ref{robustness}C shows that the population dynamics of the system quickly approach those of the calculation in which only the B and C states were initially populated ("BC calculation"). This is a consequence of the strong non-adiabatic coupling, which induced rapid population transfer from states 3 and 4 to states A and B within the first 10~fs. After about 20~fs, both simulations show that the majority of the valence population is in the A and B states, and is nearly equally distributed between them. As in the BC calculation, the simulation with all states initially populated shows that the coherence is preserved after the non-adiabatic population transfer and a revival of the quantum beat takes place between 40 and 50~fs. In fact, the revival in Fig~\ref{robustness}B occurs earlier than in Fig.~4B, while still being dominated by the B-C and A-C coherences, giving an even better agreement with the experimental results. The lower amplitude and weaked and/or delayed revivals of the 3-C and 4-C coherences is clear in Fig~\ref{robustness}A. 

These results therefore show that the silane molecule is very efficient at producing A-C and B-C quantum beats, a property that derives from the high density of CIs that funnel the populations of the valence-excited states into the lowest-lying excited states, to the quasi-degeneracy of its stretching modes and, most importantly, to the clock-like nature of the dynamics in the C-state, which lacks significant coupling to the other Rydberg states. The robustness of the results, combined with the excellent agreement between theory and experiment, serve as an \textit{a posteriori} validation of the hypothesis that our vibronic-coupling Hamiltonian model, even with its reduced dimensionality, is able to describe well all important features of the non-adiabatic dynamics of the system initiated by the IR pulse. 

\subsection{Semi-classical analysis of electronic coherences} 
 
In order to understand the mechanism of dephasing and revival of the electronic coherence observed in our experiment and reproduced by the accurate MCTDH simulations, we performed a theoretical analysis using a simple semi-classical model reported previously in Ref.~\cite{golubev2020tga}. We approximate the quantum wave packet $\chi_{j}(\mathbf{Q},t)$ propagating on the corresponding electronic state $j$ by a single Gaussian function
%
\begin{equation}
\label{eq:GWP}
\chi_{j}(\mathbf{Q},t) \approx \chi^G_{j}(\mathbf{Q},t) = 
    c_j(t) N \exp\left\{
        -\frac{1}{2\hbar}[\mathbf{Q}-\mathbf{Q}_{j}(t)]^{2}
        +\frac{i}{\hbar}\mathbf{P}_{j}(t)^{T}\cdot\lbrack\mathbf{Q}-\mathbf{Q}_{j}(t)]
        +\frac{i}{\hbar}\gamma_{j}(t)
    \right\},
\end{equation}
%
where $\mathbf{Q}$ and $\mathbf{P}$ are mass- and frequency-scaled coordinates and momenta, $\gamma_{j}(t)$ is the time-dependent phase of the wave packet, $N$ is the normalization constant, and $c_j(t)$ represents the time-dependent population of the electronic state. To establish the correspondence between the quantum wave packet $\chi_{j}(\mathbf{Q},t)$ and its semi-classical representation $\chi^G_{j}(\mathbf{Q},t)$, we set the position $\mathbf{Q}_j$ and momentum $\mathbf{P}_j$ of the Gaussian by computing the expectation values of the corresponding operators for the quantum wave packet evolving on the electronic state $j$
%
\begin{eqnarray*}
    &&\mathbf{Q}_j = \int d \mathbf{Q}\, \chi^*_{j}(\mathbf{Q},t) \mathbf{Q} \chi_{j}(\mathbf{Q},t),\\
    &&\mathbf{P}_j = -i \hbar \int d \mathbf{Q}\, \chi^*_{j}(\mathbf{Q},t) \nabla \chi_{j}(\mathbf{Q},t).
\end{eqnarray*}
%
The simple Gaussian form of the wave packet, Eq.~(\ref{eq:GWP}), allows us to express the electronic coherence between states $j$ and $k$, Eq.~(\ref{eq:coherence}), analytically as
%
\begin{equation}
\label{eq:SC_coh}
    \chi_{jk}(t)=c_j(t) c_k(t) e^{-d(t)^{2}/4\hbar}e^{iS(t)/\hbar},
\end{equation}
%
where
%
\begin{equation}
\label{eq:ph_dist}
    d(t)=\sqrt{|\mathbf{Q}_{j}(t)-\mathbf{Q}_{k}(t)|^{2}+|\mathbf{P}_{j}(t)-\mathbf{P}_{k}(t)|^{2}}
\end{equation}
%
is the distance in coordinate and momentum space between the centers of the two Gaussian wave packets and $S(t)$ is the phase responsible for the electronic oscillations. Further, we will focus on the absolute value of the electronic coherence, i.e., on the quantity $|\chi_{jk}(t)|$, thus omitting the consideration of the phase $S(t)$.

It is seen from Eqs.~(\ref{eq:SC_coh}) and (\ref{eq:ph_dist}) that the electronic coherence between a pair of electronic states $j$ and $k$ can be decomposed in three different components~\cite{vacher2017}: (I) the product of populations of the corresponding electronic states, (II) the overlap of the two nuclear wave packets in coordinate space, i.e., their spatial overlap, and (III) the overlap of the wave packets in momentum space, which is referred to as dephasing. Importantly, the semi-classical coherence can be written as a simple product of these contributions
%
\begin{equation}
    |\chi_{jk}(t)|=c_j(t) c_k(t)
                 e^{-|\mathbf{Q}_{j}(t)-\mathbf{Q}_{k}(t)|^{2}/4\hbar}
                 e^{-|\mathbf{P}_{j}(t)-\mathbf{P}_{k}(t)|^{2}/4\hbar}=(p_jp_k)^{1/2}
                 \mbox{O}_{jk}\mbox{D}_{jk},
\end{equation}
%
which allows the intuitive interpretation of coherence in terms of populations of the involved electronic states, and positions and momenta of the corresponding nuclear wave packets propagating on those states (see Fig.~4 of the main text).

\subsection{Simulations of strong-field excitation} \label{Sec:SFE}

In order to get insights into the pump step and the resulting initial populations of the silane excited states, we performed simulations by numerically solving the time-dependent Schr\"odinger equation in which the interaction of a neutral silane molecule with the pump laser was treated within the dipole approximation. Starting from the ground state, we perturb the system with a linearly polarized chirped Gaussian electric field
%
\begin{equation}
\label{eq:field}
    E(t)=E_0 \exp{\left[ -\frac{(t-t_{0})^2}{2 \sigma^2} \right]} \cos{((\omega_0+bt) t)},
\end{equation}
%
where $t_0$ is the center of the pulse, $\sigma$ is the pulse width, $E_0$ is the field strength, $\omega_0$ is the central frequency, and $b$ denotes the parameter associated with a linear variation of the instantaneous frequency of the pulse in time. The energy levels and the corresponding transition-dipole moments were computed at the EOM-CCSD/aug-cc-pVTZ level of theory. All electronic states lying below 15.3~eV have been taken into account.

The evolution of the populations of the excited states driven by a 5.2-fs full-width-at-half-maximum (FWHM $=2 \sqrt{2 \ln{2}}\sigma$) laser pulse with peak intensity of $I_0=c \epsilon_0 E_0^2/2= 10^{14}$ W/cm$^2$ and wavelength spanning the range from 600~nm to 950~nm (at $\pm 3\sigma$) is shown in Figure~\ref{pop_transf}. The middle panel of the figure illustrates the situation when only couplings between the ground and the excited states are present, while the transitions between the excited states themselves are disabled. It is seen that in this case the external field induces the oscillations of the populations between the ground and the excited states during the action of the pulse. However, the system remains almost completely in the ground state as soon as the external field is terminated. The bottom panel of Figure~\ref{pop_transf} depicts the population transfer dynamics for the case when all the couplings between the states are taken into account. As one can see, the external field initiates highly nonlinear population transfer from the ground to all included electronically excited states of the system. The latter indicates the fact that multiphoton excitation plays a central role in the population of excited states of silane by the pump pulse.

\begin{figure}[h!]
\centering
\includegraphics[width=0.7\textwidth]{Figures/SFexcitation.pdf}
\caption{\label{pop_transf}\textbf{Population transfer in neutral silane driven by the interaction with the pump pulse.} Top panel: Laser pulse obtained through Eq.~(\ref{eq:field}) using the following parameters: $I_0=c \epsilon_0 E_0^2/2= 10^{14}$~W/cm$^2$, FWHM $=2 \sqrt{2 \ln{2}}\sigma$ = 5.2~fs, $t_0 = 10$~fs, the wavelength spans a range from 950 nm at -3$\sigma$ to 600~nm at +3$\sigma$. Middle panel: Evolution of populations of 60 lowest excited states of a silane molecule driven by such a laser pulse. Only couplings between the ground and the excited states are present while the transitions between the excited states themselves are disabled. Bottom panel: Same as above but all the couplings between the states are taken into account. The energy levels and the corresponding transition dipole moments are computed at the EOM-CCSD/aug-cc-pVTZ level of theory.}
\end{figure}

Our simulations %by both EOM-CCSD and RT-TDDFT approaches 
demonstrate that the interaction of a neutral silane molecule with the intense pump field leads to a nonlinear population transfer from the ground state to the manifold of electronically excited states. Importantly, we found that the distribution of populations between the excited states is very sensitive to the variations of the laser pulse parameters. However, these calculations had to be carried out in the fixed-nuclei approximation, such that they neglected nuclear motion and the associated non-adiabatic dynamics. These additional effects are addressed in the following subsection.

\subsection{Modeling of multi-cycle pump-pulse effects}

As has been shown in Sec.~\ref{Sec:SFE}, the strong-field excitation is difficult to precisely model because of its high sensitivity to the details of the pump pulse and the fixed-nuclei approximation. In this section we shall, therefore, start from the conclusion of Sec.~\ref{Sec:SFE}, i.e., that all states can, in principle, be significantly populated by the pump pulse but explicitly treat the time evolution of this excitation, in analogy with the recently studied case of strong-field ionization (SFI) \cite{pabst16a}. In that case, due to the highly nonlinear nature of SFI, the majority of the ionization is restricted to the peaks of the electric field in each half-cycle. As a result, a train of wavepackets is launched in the cationic states at a frequency of twice the fundamental driving field. Applying the same temporal concept to strong-field excitation, a train of wavepackets is launched in the valence- and Rydberg-excited states. In contrast to strong-field ionization, however, no photoelectron is emitted (which can lead to entanglement and a loss of coherence) \cite{pabst11a}, therefore, it is only necessary to treat the interference of the molecular wavepackets. 

Let us first consider the electronic wavepacket interference. The phase of the excited electronic states with respect to the ground state at the moment of the creation of a new excited-state wavepacket determines whether the latter will interfere constructively or destructively with the preexisting one. This phase evolves at a frequency that is proportional to the energy difference of the ground and excited state. Since new wavepackets are produced at twice the frequency of the driving field, the population of states with energies close to a $2n\omega$ will interfere constructively and build up over the duration of a multi-cycle pulse while states with energies close to the odd-integer multiples of $\omega$ will experience destructive interference and will therefore be suppressed. 

Our D-scan reconstruction of the pump pulse shows that up to four half-cycles can contribute to the strong-field excitation with relative intensities of 2:3:3:2. Hence, the energy dependence of the pump-pulse's excitation efficiency, due to its multi-cycle nature, can be found by calculating the square amplitude of the sum of four cosine oscillations at a relative delay of 1.3~fs (corresponding to the center wavelength of 780~nm) as a function of the oscillating frequency. This is equivalent to finding the squared Fourier transform of a sequence of four delta functions spaced by 1.3~fs (equal to the product of a rectangular function and a Dirac comb in the time domain). By making use of the convolution theorem we find that the Fourier transform of such a function is the convolution of a sinc function and a Dirac comb in the frequency domain. Indeed, Fig.~\ref{MulticycleFig} which present the excitation efficiency of the pump pulse, display a comb of sinc-squared functions with three nodes (one less than the number of pulses) between subsequent maxima. The proximity of the A, B, and C states to the peaks of excitation efficiency further explains why these coherences survive the multi-cycle excitation, whereas other possible coherences do not. This result additionally supports our experiment-based assignment of the attosecond electron dynamics to the B-C and A-C coherences.

Vibrational dynamics have the potential, through the displacement of the nuclear wavepacket away from the Frank-Condon region between half-cycles, to reduce the coherence of the electronic wavepackets at the moment of their creation and therefore halt the constructive interference that leads to the energetically selective excitation. In the case of silane, however, this effect is relatively small because the half cycle of the pump pulse is short in comparison with the vibrational period. It is estimated that such effects would become observable with longer-wavelength pump pulses of at least 4~\(\mu\)m. To confirm the lack of effect of the nuclear dynamics on the coherence buildup, MCTDH calculations were run in which multiple wavepackets were consecutively launched from the Frank-Condon region into all contained electronic states. These showed that the absolute value of the initial coherence is only reduced by 20\% at the time of the following population injection, enabling a buildup of the electronic coherence over multiple cycles of the pump pulse. Furthermore, the multiple injection MCTDH simulations showed an increase of the A (+25\%), B (+50\%), and C (+50\%) states with no significant change to the temporal evolution of their quantum beats (which can be seen in the middle panel of Fig.~\ref{MulticycleFig}). In fact, since the multi-injection simulation lowers the relative population of the higher-energy valence states 3, 4 and 5 by 20\%, 70\% and 99\% respectively, the total quantum beat of the system becomes even more heavily dominated by the 0.7-0.8~PHz signal of the A-B and A-C coherences (see right-hand panel of Fig.~\ref{MulticycleFig}), leading to a simulated OD signal that resembles the experimentally measured quantum beats even more closely. 
We therefore conclude that, when treated with the strong-field-excitation model described here, the few-, but still multi-cycle nature of our pump-pulse has the potential to energetically selecting well-defined subsets of electronic coherences. The $2n\omega$ spacing of the energy windows of excitation is accordingly reflected in the higher likelihood of observing a quantum beat with an energy in the vicinity of $2\omega$. These results imply that the lack of strong signals at 0.65~PHz and 0.58~PHz in the experimental results could not only be due to very fast non-adiabatic depopulation of these states but also due to the energetically selective preparation of electronic coherences. 
\begin{figure}[h!]
\centering
\includegraphics[width=\textwidth]{Figures/SupFig7.pdf} \label{MulticycleFig}
\caption{\textbf{Inter-half-cycle interference of strong-field-prepared wavepackets} Left: Energy dependence of the excitation efficiency of an electronic state due to the interference of wavepackets created in four consecutive half-cycles of a 780~nm pump pulse. The black lines indicate the energy and transition-dipole moment of the electronically excited states of silane. Middle: Comparison between the MCTDH simulated quantum beats of a simulation with a single wavepacket initialization (black, same as Fig.~S8) and a simulation in which four identical wavepackets were launched in each electronic state with delays of 1.3~fs (red). Right: FFT of the middle panel.}
\end{figure}

\clearpage

%%-------------------- style1
\bibliography{attoH2On,attobib,biblio_SOM_Theory_Silane, extra}
\bibliographystyle{Science}

%-------------------- style2



\end{document}
