

The scene recognition is shaped by the fact that a few objects are common, but most objects are scene-specific in one given scene image. The most common objects like walls, and floor are most frequently appeared in the indoor scene. Other objects tend to be scene-specific, which means their frequencies can be quite distinctive among the various scene images, and we defined these objects as discriminative objects.

We adopt a pipeline to estimate the object distribution among all scene classes and depict the object distribution across all scenes in Fig. \ref{fig:obj_scene}. The most frequently appearing objects vary in the different scenes, e.g., in the bedroom, the bed, table, painting, lamp, and pillow are the most frequently appearing objects. However, in the bathroom, the floor, bathtub, mirror, and toilet are the most frequently appearing objects. In addition, there are some discriminative objects in the specific scene like the bed and pillow mostly are in the bedroom, while the toilet and bathtub mostly are in the bathroom. Meanwhile, there are some non-discriminative across various scene, e.g., the floor is a typical non-discriminative object that appeared in many different scenes with a high probability like a bathroom or a wet bar.

\begin{figure*}[htbp]
        \centering
        \subfigure[bedroom]{
        \begin{minipage}{0.3\textwidth}
        \centering
          \includegraphics[width= 1\textwidth]{Fig//bedroom.png}
          \label{exp:mnist:1}
        \end{minipage}}
        \subfigure[bathroom]{
        \begin{minipage}{0.3\textwidth}
        \centering
          \includegraphics[width= 1\textwidth]{Fig//bathroom.png}
          \label{exp:mnist:2}
        \end{minipage}}
        \subfigure[wet bar]{
        \begin{minipage}{0.3\textwidth}
        \centering
          \includegraphics[width= 1\textwidth]{Fig//wet_bar.png}
          \label{exp:mnist:3}
        \end{minipage}}
        \caption{The object model (OM) shows that the distribution of the objects varies in different scenes. Specifically, Fig. (a) shows the probability of many representative objects like the bed, lamp, and pillow is particularly high in the bedroom. Therefore, the scene is more likely to be identified as the bedroom. Fig. (b) is similar to Fig. (a), which also contains many discriminative objects like bathtub and toilet, and these objects are easily linked to the bathroom. Fig. (c) shows that there are only some common objects like the floor, ceiling, and door that have a high probability of appearing in the wet bar, so the scene of the wet bar is relatively difficult to be recognized when only considering objects.}
        \label{fig:obj_scene}
\end{figure*}


We first use the object model to detect or segment the objects appeared in the given image. The result will be a long vector whose elements are almost all 0 except those objects detected by the algorithm are set to 1. This vector dismiss the relation between different objects. Therefore, we'll consider using object pairs to improve the recognition accuracy in the subsequently section.

\subsection{\textbf{Object Model(OM)}}


Given a set of images $I_{c_i}$ from a scene category $c_j$, the conditional probability of object $o_i$ on the scene $c_j$ is:
\begin{equation}
\begin{split}
p\left(o_{i} | c_{j}\right) = N_{o_i} / N_{o_{total}} \\
i \in [0,N_{objs}], j \in [0,N_{scenes}]
\end{split}
\end{equation}

where $N_{o_i}$ is the total number of i-th object $o_i$ appears in $I_{c_i}$ and $N_{o_{total}}$ is total number of objects appear in $I_{c_i}$. Also, $N_{objs}$, and $N_{sce}$ represent the number of objects in the pretrained model and number of scenes we have in the dataset respectively. We deploy the object model or scene parsing model to detect objects in the given scene for every scene image. Finally, we combine all the results and obtain the $p(o|c)$ matrix as follows:

\begin{equation}
\begin{split}
p\left(o | c\right) =
\begin{bmatrix}
p\left(o_{1} | c_{1}\right) & \cdots & p\left(o_{1} | c_{j}\right) & \cdots & p\left(o_{1} | c_{m}\right) \\
\vdots &  &  &  & \vdots \\
p\left(o_{i} | c_{1}\right) & \cdots & p\left(o_{i} | c_{j}\right) & \cdots & p\left(o_{i} | c_{m}\right) \\
\vdots &  &  &  & \vdots \\
p\left(o_{n} | c_{1}\right) & \cdots & p\left(o_{n} | c_{j}\right) & \cdots  & p\left(o_{n} | c_{m}\right)
\end{bmatrix}
\end{split}
\end{equation}


Next, the  $p_(c_{j}|o_{i})$, the probability of scene $c_{j}$ given an object $o_{i}$, can be derived by the Bayes Rule and Law of Total Probability.
\begin{equation}
p\left(c_{j} | o_{i}\right)=\frac{p\left(o_{i} | c_{j}\right) p\left(c_{j}\right)}{p(o_{i})}
\end{equation}


\begin{equation}
P(o_{i})=\sum_{j} P\left(o_{i} \cap c_{j}\right)=\sum_{j} P\left(o_{i} | c_{j}\right) P\left(c_{j}\right)
\end{equation}

\begin{equation}
p\left(c_{j} | o_{i}\right)=\frac{p\left(o_{i} | c_{j}\right) p\left(c_{j}\right)}{\sum_{j} p\left(o_{i} | c_{j}\right) p\left(c_{j}\right)}
\end{equation}

where $p(c_j)$ is the probability of scene $c_j$ appears in the whole dataset.

\subsection{\textbf{Object Pair Model (OPM)}}

\begin{figure*}
        \centering
        \subfigure[OPM (bedroom)]{
        \begin{minipage}{0.22\textwidth}
        \centering
          \includegraphics[width= 1.2\textwidth]{Fig//bedroom_joint.png}
          \label{exp:mnist:1}
        \end{minipage}}
        \subfigure[The distribution of object pairs in bedroom]{
        \begin{minipage}{0.22\textwidth}
        \centering
          \includegraphics[width= 0.9\textwidth]{Fig//bedroom_joint_rank.eps}
          \label{exp:mnist:2}
        \end{minipage}}
        \centering
        \subfigure[DOPM (bedroom)]{
        \begin{minipage}{0.22\textwidth}
        \centering
         \includegraphics[width= 1.2\textwidth]{Fig//bedroom_joint_dis.png}
         \label{exp:mnist:1}
        \end{minipage}}
        \subfigure[The distribution of discriminative object pairs of bedroom]{
        \begin{minipage}{0.22\textwidth}
        \centering
         \includegraphics[width= 0.9\textwidth]{Fig//bedroom_joint_dis_rank.eps}
         \label{exp:mnist:2}
        \end{minipage}}
        \caption{Object pair model (OPM) have much more information than object model (OM). (a) shows the object pairs probability distribution of bedroom. (b) shows the object pairs which have top high probability to appear in bedroom. Objects distribution varies in different scenes. (a) shows the probability of many representative objects like bed, lamp, and pillow is very high in bedroom so this scene can be more easily to be recognized. (b) is similar with (a) which also contains many discriminative objects like bathtub and toilet, and these objects are always lined to bathroom. (c) shows that in wet bar only some common objects like floor, ceiling and door have high probability to appear, so this scene is hard to recognize when only considering objects.}
        \label{Object Pairs in Scene}
\end{figure*}

%Given a set of images $I_{c_i}$ from a scene category $c_j$, the conditional probability of object $o_i$ on the scene $c_j$ is:
%\begin{equation}
%p\left(o_{i} | c_{j}\right) = N_{o_i} / N_{o_{total}}
%\end{equation}
%
%where $N_{o_i}$ is the total number of i-th object appears in $I_{c_i}$ and $N_{o_{total}}$ is total number of objects appear in $I_{c_i}$, we deploy the Yolov3 to detect objects for every scene image.

Assume the statistical independence of each object, we obtain the joint conditional probability $p\left(o_{h}, o_{i} | c_{j}\right)$ of $o_h$ and $o_i$ appear in the scene $c_j$:
\begin{equation}
\begin{split}
p\left(o_{h}, o_{i} | c_{j}\right) = p\left(o_{h} | c_{j}\right) p\left(o_{i} | c_{j}\right)\label{joint probability} \\
h, i \in [0,N_{objs}], j \in [0,N_{scenes}]
\end{split}
\end{equation}

where $N_{objs}$, and $N_{scenes}$ represents the number of objects in the pretrained model and number of scenes we have in the dataset respectively. According to the joint probability equation, we obtain the joint probability matrix $p(o,o|c_j)$ by calculating each object pairs defined as follows:

\begin{equation}
\begin{split}
& p\left(o_{}, o_{} | c_{j}\right) = \\
& \begin{bmatrix}
p\left(o_{1}, o_{1} | c_{j}\right) & \cdots & p\left(o_{1},o_{k}| c_{j}\right) & \cdots & p\left(o_{1}, o_{m}| c_{j}\right) \\
\vdots &  & \vdots &  & \vdots \\
p\left(o_{j}, o_{1} | c_{j}\right) & \cdots & p\left(o_{j},o_{k}| c_{j}\right) & \cdots & p\left(o_{j},o_{m}| c_{j}\right) \\
\vdots &  &  &  & \vdots \\
p\left(o_{n}, o_{1} | c_{j}\right) & \cdots & p\left(o_{n},o_{k}| c_{j}\right) & \cdots  & p\left(o_{n}, o_{m} | c_{j}\right)
\end{bmatrix}
\end{split}
\end{equation}

Similarly, the  $p_(c_{j}|o_{h},o_{i})$, the probability of scene class $c_{j}$ given an object pair ($o_{h},o_{i}$), can be derived by the Bayes Rule and Law of Total Probability.
\begin{equation}
p\left(c_{j} | o_{h},o_{i}\right)=\frac{p\left(o_{h},o_{i} | c_{j}\right) p\left(c_{j}\right)}{p(o_{h},o_{i})}
\end{equation}

\begin{equation}
P(o_{h},o_{i})=\sum_{j} P\left(o_{h},o_{i} \cap c_{j}\right)=\sum_{j} P\left(o_{h},o_{i} | c_{j}\right) P\left(c_{j}\right)
\end{equation}

\begin{equation}
p\left(c_{j} | o_{h},o_{i}\right)=\frac{p\left(o_{h},o_{i} | c_{j}\right) p\left(c_{j}\right)}{\sum_{j} p\left(o_{h},o_{i} | c_{j}\right) p\left(c_{j}\right)}
\end{equation}



%\begin{figure}
%  \centering
%  \tabcolsep=0pt
%  \begin{tabular}{ll}
%  \includegraphics[height=1.7in]{Fig//ratioclustering.eps} &
%  \includegraphics[height=1.7in]{Fig//sizeclustering.eps}
%  \end{tabular}
%  \caption{Clustering results of all bounding boxes from  the dataset. (left) shows the ratio of width to height are clustered into 4 categories, (right) shows the width and height of bounding box are clustered into 6 categories.}
%  \label{fig:Clusterkmeans} %% label for entire figure
%\end{figure}

\subsection{\textbf{DOM and DOPM}}
The discriminative object model (DOM) or discriminative object pair model (DOPM) is chosen by selecting the objects or object pairs with the largest difference value. First, the scene classes $c \in C$ are ranked based on the posterior probability $p(c|o_i)$. The std(c) is the ranking function, e.g., r(1) is the largest probability and r(c) is the lowest probability. Then, the difference among the posterior probability is calculated as $dis(o_i)$:

\begin{equation}
\operatorname{dis}\left(o_{i}\right)=\max _{r \in 1, \ldots, C-1}(r(i)-r(i+1))
\end{equation}
The $dis(o_i)$ is the discriminative index for measuring the importance of the object appear in the specific scene. Last, the top N objects are selected as the discriminative objects and the rest are regarded as the non-discriminative object for scene understanding. Therefore, we filter out the non-discriminative objects based on the difference value r(i), e.g., if $r(i) \le r_{th}$, we define it as non-discriminative object in the specific scene. Also, we count the total number of times an object appears in a single scene as $N_{obj}$ and if $N_{obj} \le N_{th}$ we regard it as noise. We define the $r_{th}$ as the discriminative value threshold and $N_{th}$ as the noise threshold.


\subsection{\textbf{COM and COPM}}

We combine the object model, and the PlaceCNN model as a combined object model (COM) depicted in Fig. \ref{fig:combined_model}. For the models with only object information like object model with $X_{obj-80},X_{obj-150}$, we define these as the object model $\Phi_{om-80}$ and $\Phi_{om-150}$ models respectively, representing the object model pretrained on 80 classes objects and 150 object classes respectively.
The feature representations of the above-mentioned object model will be fed into the two fully connected layers for scene understanding. In summary, we have different model settings. Also, there are $\Phi_{opm-80}$, and $\Phi_{opm-150}$ models, by incorporating the joint probability of object pairs in the OPM. Also, we combine the PlaceCNN model of the ResNet baseline with OM or OPM as the combined model, and with a different of settings as $\Phi_{com-80}$, $\Phi_{com-150}$, $\Phi_{copm-80}$, $\Phi_{copm-150}$, representing the combined object model (COM) with 80 and 150 classes and combined object pair model (COPM) with 80 and 150 classes respectively. Moreover, we introduce the discriminative objects and non-discriminative objects in the $\Phi_{dopm-80}$, $\Phi_{dopm-150}$ respectively.

