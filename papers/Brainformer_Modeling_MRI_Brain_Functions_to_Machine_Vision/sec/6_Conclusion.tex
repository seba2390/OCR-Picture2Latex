\section{Conclusions and Discussions}
In this paper, we have investigated the feasibility of transferring human brain activities to the vision models via fMRI signals. In our proposed Brainformer, we introduce the concept of the fMRI feature technique that explores the local patterns of the signals. Additionally, the Brain 3D Voxel Embedding is proposed to preserve the 3D information that fMRI signals have been missed. The Multi-scale fMRI Transformer is presented to learn the features of a long signal. We also introduce Brain fMRI Guidance Loss inspired by the brain's mechanisms to transfer the vision semantic information of humans into the vision models. The empirical experiments on various benchmarks demonstrated that Brainformer is competitive with another SOTA method that uses text modality for knowledge transfer.

\noindent
\textbf{Limitations and Future Works}. 
Inspired by prior studies in neuroscience, we have developed our approach and demonstrated the efficiency of fMRI for training vision models. However, the proposed approach can potentially consist of minor limitations. 
First, although the NSD dataset is one of the large-scale datasets in the neuroscience field, compared to large-scale datasets such as ImageNet \cite{imagenet} in the computer vision field, this data is relatively small due to time and human efforts. However, we have demonstrated the performance of Brainformer with respect to the number of training data as in Section \ref{sec:abl_amount_data}. The result shows that if the amount of data is sufficient, the performance of Brainformer can be better and potentially surpass the previous methods. Second, the primary goal of this study is to present a new perspective on how to involve human recognition mechanisms in training vision learning models. The experimental configurations such as utilizing CLIP \cite{clip}, ConvNext \cite{liu2022convnet}, SwinTransformer \cite{liu2021swin} or even methods for downstream tasks, i.e., object detection, instance segmentation, semantic segmentation, and human brain response prediction, are conducted to prove our hypothesis on a fair basis. We leave further experiments with other settings for future studies.

\noindent
\textbf{Broader Impacts}.
Brainformer has a bi-directional impact on computer vision and the neuroscience field. In particular, we have illustrated how brain activities could help the vision model. Besides, Brainformer has potential benefits for neuroscientists to study brain activities, especially human cognition. Given a pair of images and corresponding fMRI, neuroscientists can utilize Brainformer as a valuable tool to explore which neurons inside the brain are highly activated with respect to the input image or particular objects inside, thus uncovering potential novel patterns of the brain.
