\documentclass[aps,floatfix,prl,superscriptaddress]{revtex4}
\usepackage{graphicx,subfigure,amsmath,bm}
\usepackage{dcolumn}   % needed for some tables
\usepackage{relsize}  % for math
\usepackage{verbatim}   % for math
\usepackage{mathtools}
\usepackage{color}
\DeclareMathOperator\arctanh{arctanh}

\begin{document}
\title{Supplemental Material \\ Estimating Hidden Asymptomatics, Herd Immunity Threshold and Lockdown Effects using a COVID-19 Specific Model}
\author{Shaurya Kaushal}
\affiliation{
	Jawaharlal Nehru Centre for Advanced Scientific Research, Jakkur, Bangalore  560064, India}

\author{Abhineet Singh Rajput}
\affiliation{Indian Institute of Science, CV Raman Rd, Bengaluru, Karnataka, India 560012.}
\author{ Soumyadeep Bhattacharya}
\affiliation{Sankhya Sutra Labs, Manyata Embassy Business Park, Bengaluru, Karnataka, India 560045.}
\author{M. Vidyasagar}
\affiliation{Indian Institute of Technology Hyderabad, Kandi 502285, India}
\author{Aloke Kumar}
\affiliation{Indian Institute of Science, CV Raman Rd, Bengaluru, Karnataka, India 560012.}
\author{Meher K. Prakash}
\email{meher@jncasr.ac.in}
\affiliation{
	Jawaharlal Nehru Centre for Advanced Scientific Research, Jakkur, Bangalore  560064, India}
\affiliation{VNIR Biotechnologies Pvt Ltd, Bangalore Bioinnovation Center, Helix Biotech Park,  Electronic City Phase I, Bangalore 560100.}
\author{Santosh Ansumali}
\affiliation{
	Jawaharlal Nehru Centre for Advanced Scientific Research, Jakkur, Bangalore  560064, India}
\affiliation{Sankhya Sutra Labs, Manyata Embassy Business Park, Bengaluru, Karnataka, India 560045.}

\date{\today}
\maketitle

\section{I. ANALYTICAL SOLUTION, APPROXIMATING LOGARITHM}
The first order differential equations governing the dynamics of the system before lock down are:
\begin{align}
\label{Basic}
\begin{split}
\frac{d A}{dt} &= S  \, \alpha_0 \left(  I+   A\right) - \delta  A-\gamma A
%+\beta_1(t) E
\\
\frac{d   I }{dt} &=\delta  A-\gamma I \\
\frac{d S}{dt} &= -\alpha_0S  \, \left(I+A\right)\\
\frac{d R}{dt} &= \gamma \, (I+A) 
%\frac{d D}{dt} &= \gamma_{\rm D} \, I,
\end{split}
\end{align}
We define a new variable $M$, such that $M=A+I$. The dynamics of $M$ is given by
\begin{align}
\label{ApI}
\frac{dM}{dt} &= \alpha_0 \left(S M \right) - \gamma M 
\\
\frac{dM}{dS} &= \frac{1}{r_0 S} -1  
\end{align}
where $r_0$ is the basic reproduction number given by $r_0 = \alpha_0/\gamma$.
\begin{align}
	M = 1 - S + \frac{1}{r_0} \log \left(\frac{S}{S_{0}}\right)
\end{align}
where $S_0$ is the fraction of people who are susceptible at time(t)=0, and is a number very close to 1.
Using this relation in the evolution equation of S, gives:
\begin{align}
\frac{dS}{dt} = - S \alpha_0 \left( 1 - S + \frac{1}{r_0} \log \left(\frac{S}{S_{0}}\right) \right)
\end{align}
At this point, in order to extract a integrable exact solution, an approximation for the logarithm in the RHS is required. The two ways of approximating logarithm are
\begin{align}
\begin{split}
\text{Approximation 1} &: \qquad \log(Z) \approx Z-1
\\
\text{Approximation 2} &: \qquad \log(Z) \approx (Z-1) \left( \frac{w_1}{Z} +  w_2\right)
\end{split}  
\end{align}   
where, $w_1,w_2$are weights such that $w_1 + w_2 = 1$. 
\\
As $(S/S_{0})$ lies between $(0,1)$, we are only interested in $Z$ in the range $(0,1)$. The comparison between the two approximations is illustrated in FIG.\ref{fig1}.
\begin{figure}
	\includegraphics[width=8cm]{fig1s.png}
	\caption{\label{fig1}Comparison of $\log(Z)$ with its two approximations. Approximation 1 being $\log(Z) \approx (Z-1)$ and approximation 2 being $\log(Z) \approx (Z-1)\left(w_1/Z + w_2 \right) $ with, $w_1=1/5$ and $w_2=4/5$.}
\end{figure}
Approximation 2, simplifies the differential equation to:
\begin{align}
\frac{dS}{dt} = \alpha_0 S^2  - \alpha_0 S - \frac{\alpha_0}{r_0} \left[\frac{S-S_0}{S_0}\right]\left(w_1 S_0 + w_2 S\right)
\end{align}
which can be simply written in the form
\begin{align}
\frac{dS}{dt} = a S^2 + bS + d
\end{align}
where, $ a = (S_0 r_0 -  w_2)/ r_0, \; 
b=   \left(r_0 +  w_1-  w_2\right)/r_0, \,  d=  w_1/{r_0}$.
\\
The solution upon integrating is
\begin{align}
\frac{1}{\sqrt{-b^2 + 4ad}} \left( 2 \tan^{-1} \left( \frac{b + 2as}{\sqrt{-b^2 + 4ad}} \right) \right) \Bigg |_{S_{0}}^{S} = t
\end{align}
where the integration variable is `$s$'.
As $b^2>4ad$, the equation can be rewritten as
\begin{align}
\frac{2}{\sqrt{b^2 - 4ad}}(-i) \> \> \arctan \left((i)\frac{-b-2as}{\sqrt{b^2-4ad}} \right) \Bigg |_{S_{0}}^{S} = t 
\end{align}
Using the identity $-i \arctan(ix) = \arctanh (x)$ 
\begin{align}
\frac{2}{\sqrt{b^2-4ad}} \>\> \arctanh \left(\frac{-b-2as}{\sqrt{b^2-4ad}} \right)\Bigg |_{S_{0}}^{S} = t
\end{align}
Using the identity: $\arctanh(x) = \frac{1}{2} \log \left(\frac{x+1}{x-1}\right)$
\begin{align}
\log \left( \frac{-b - 2as + \sqrt{b^2-4ad}}{-b-2as - \sqrt{b^2-4ad}} \right) \Bigg |_{S_{0}}^{S}  = \left( \sqrt{b^2-4ad}\right)t
\end{align}
Thus, using approximation 2 gives us an analytically tractable solution for the susceptible population
\begin{equation}
\frac{S}{S_0} = \frac{h   (1+ h_2 \exp(h \, \alpha_0 t)) }{2a   \left(1-h_2 \exp(h \, \alpha_0 t)\right)}
+\frac{b}{2a}
\end{equation}
where, $h_2 =    (2\, a-b-h)/(2\, a-b+h)$,  and $h,b$ are constants such that $h=\sqrt{b^2-4ad}$. 
  
\begin{figure}[h]
	\subfigure[Numerical solution of the SAIR system]{
	\includegraphics[scale=0.15]{fig2as.png}}
\subfigure[Analytical solution of the SAIR system using approximation 2]{
	\includegraphics[scale=0.15]{fig2bs.png}}
	\caption{\label{fig2} The figure illustrates the ability of the analytical solution found in section I, to correctly capture the dynamics of the SAIR model. The parameters used for these plots are, $\alpha=0.25, \gamma = 0.027, \delta_1 = 0.036$, which are in reasonable range of real time parameter values for COVID19 (discussed in section II). The initial conditions is one infected person in a million people. } 
\end{figure}

\section{II. Parameter Estimation}
In this section we discuss the estimation procedure for the parameters ($\alpha_0, \gamma, \delta$). The analytical solutions for the infected(active) and recovered populations is known for both pre(discussed in section I) and post lock-down scenarios. These analytical solutions are then fit onto the real time data for several countries, to give us an estimate of the parameters relevant to COVID19. We begin with the post lock-down scenarios as the solutions are rather straight forward. After a '$\epsilon$' number of days post lockdown, the recovery rate is given by
\begin{equation}
\label{rdot_post}
\dot{R} =  \left[ \gamma (A+I)_{\rm lock + \epsilon} \right] \exp \{ -\gamma t\}
\end{equation}
and the infection is given by
\begin{equation}
\label{I_post}
I = \exp\{ -\delta_1 t \} \left[ I_{\rm lock+\epsilon} + \left( \frac{\delta(A+I)_{\rm lock+\epsilon}}{\delta_1-\gamma} \right) \left( \exp\{(\delta_1-\gamma)t \}-1\right)\right]
\end{equation}
Using Eq. \eqref{rdot_post} and real time recovery rate data for COVID19, the parameter $\gamma$ can be estimated as shown in FIG.\ref{fig3}a. Using Eq. \eqref{I_post} and real time infection rate data, parameter $\delta_1$ can be estimated as shown in FIG. \ref{fig3}b.   
\begin{figure}[h]
	\subfigure[]{
	\includegraphics[scale=0.16]{fig3as.png}}
    \subfigure[]{
	\includegraphics[scale=0.16]{fig3bs.png}}
	\caption{\label{fig3}Estimation of parameters $\gamma$ and $\delta_1$ from post lock-down data}
\end{figure}
\begin{figure}[h]
	\includegraphics[scale=0.16]{fig4as.png} 
	\includegraphics[scale=0.16]{fig4bs.png}
	\caption{\label{fig4}Estimation of parameters $\alpha_0$ and $\beta_1$}
\end{figure}
The evolution of infections pre-lockdown and in early time limit is given by
\begin{align}
\label{pre_lock_infection}
I = \exp \{ -\delta_1 t \} \left[I_0 +  \int_0^t ds  \left( \frac{ \delta \exp\{ \delta_1 s \}}{ r_0} \right)
\left( k - g\tilde{S}(s) \right) \right]
\end{align}
where $k=(r_0-1)$ and $g=(S_0r_0-1)$. The solution post lock-down is given by
\begin{align}
\label{infections_post_lock}
I = \exp \{-\gamma t\}&(I_{\rm lock} - L) +  
\\
& L \exp\{ (-\delta_1 + \beta_1(1-{H}(t-\epsilon)))t \}
\end{align}
where, 
\begin{align}
L =  \frac{\delta \left( A_{\rm lock} \> \exp \{ \epsilon(\beta_1 - \delta_1) {H}(t-\epsilon) \} \right) } {(\gamma-\delta_1)+\beta_1(1-{H}(t-\epsilon))}
\end{align}
\begin{table} %add [H] placement to break table across pages
	\caption{\label{table1} Parameters extracted by fitting the solutions to the model we developed to the 3-day average data from the different countries.}
	\begin{ruledtabular}
		\begin{tabular}{|l|l|l|l|c|}
			Country & \pmb{$\alpha_0$} & \pmb{$\gamma$} & \pmb{$\delta_1$} &\pmb{$\beta_1$} 
			\\ \hline
			\textbf{France}    & $0.234 \pm 0.01$     & $0.034 \pm 0.002$ & $0.037 \pm 0.004$  & $0.15 \pm 0.007$         
			\\ 
			\textbf{Spain}     & $0.28 \pm 0.008$     & $0.05 \pm 0.003$   & $0.055 \pm 0.002$ & $0.2 \pm 0.01$                                    
			\\ 
			\textbf{Italy}     & $0.25 \pm 0.009$      & $0.042 \pm 0.002$  & $0.047 \pm 0.002$   &  $0.14 \pm 0.009$                                    
			\\ 
			\textbf{Switzerland} & $0.29 \pm 0.01$   &   $0.03 \pm 0.004$         & $0.05 \pm 0.003$     &  $0.053 \pm 0.004$                                  
			\\ 
		\end{tabular}
	\end{ruledtabular}
\end{table}
Now, using the above mentioned equations for infection rate and real time data for different countries, we estimate the parameters $\alpha_0$ and $\beta_1$, as shown in FIG.\ref{fig4}.
\begin{figure}
	\includegraphics[scale=0.15]{fig5s.png}
	\caption{\label{fig5} The ratio of asymptomatics to the infected population as a function of time, and for a no-lockdown scenario. }
\end{figure}
\begin{table} 
	\caption{\label{table2} The details of the peak of infections extracted using relevant parameters for the COVID-19 dynamics in the different countries and under a hypothetical no-lockdown scenario}
	\begin{ruledtabular}
		\begin{tabular}{|l|l|l|l|c|}
			Country &  $I_{\rm max}$ & $(A+I)_{\rm max}$ & Ic when $I=I_{\rm max}$ & Ic when $(A+I)=(A+I)_{\rm max}$
			\\ \hline
			\textbf{France}    &  $6 \%$   & $56 \%$ & $12 \%$   &  $6 \%$       
			\\ 
			\textbf{Spain}     & $4.3 \%$     & $53 \%$   & $8.2 \%$  & $4.1 \%$
			\\
			\textbf{Italy} & $4 \%$  &   $51 \%$  &  $7.7 \%$    & $4 \%$  
			\\  
			\textbf{Switzerland} & $5.6 \%$   &    $52 \%$  &  $10 \%$    &  $4.5 \%$  
			\\ 
		\end{tabular}
	\end{ruledtabular}
\end{table}

\end{document}