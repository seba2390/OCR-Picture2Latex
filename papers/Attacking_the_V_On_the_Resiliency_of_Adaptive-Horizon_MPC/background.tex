\section{V-Formation}
\label{sec:background}

We consider the problem of bringing a flock of $B$ birds from a random initial configuration to an organized V-formation.  Recently, Lukina et al.~\cite{lukina2016arxiv} have modeled this problem as a deterministic Markov Decision Process (MDP) $\mathcal{M}$, where the goal was to generate actions that caused $\mathcal{M}$ to reach a desired state. 

In our case, $\mathcal{M}$ is an MDP.  The \emph{state} of each bird in the flock is modeled using 4 variables: a 2-dimensional vector $\xv$ denoting the position of the bird in a 2D space, and a 2-dimensional vector $\vv$ denoting the velocity of the bird. Thus, the state space of $\M$ is $\mathbb{R}^{4B}$ representing a flock of $B$ birds.  The \emph{control actions} of each bird are 2-dimensional accelerations $\va$ and 2-dimensional position displacements $\vd$ (see discussion of $\va$ and $\vd$ below). Both are random variables.

Let $\xv_i(t),\vv_i(t),\va_i(t)$, and $\vd_i(t)$ denote the position, velocity, acceleration, and displacement of the $i$-th bird at time $t$, respectively.
%
%Here, each state of $\mathcal{M}$, at discrete time $t$, are of the form $(\xv_i(t),\vv_i(t))$, $1\,{\leqslant}\,i \,{\leqslant}\,N$, where $\xv_i(t)$ and $\vv_i(t)$ are vectors with $N$ elements (for an $N$-bird flock). Each element is 2-dimensional, representing positions and velocities, respectively. 
%
Then, the transition relation of the MDP $\mathcal{M}$ is given as follows:
%
\vspace*{-1mm}\begin{eqnarray}
\label{eq:trans}
 \xv_i(t + 1) &=& \xv_i(t) + \vv_i(t+1)\label{eq:x} + \vd_i(t) \qquad \forall~i\,{\in}\,\{1,\ldots,B\}, \nonumber\\
 \vv_i(t + 1) &=& \vv_i(t) + \va_i(t).\label{eq:v} %\\[-6mm]
\end{eqnarray}
%
Once the current acceleration and displacement are sampled, the next state is uniquely determined by~(\ref{eq:trans}) from the current state in $\M$~\cite{lukina2016arxiv}.

The problem of whether we can go from a random flock to a V-formation is a reachability question.  The reachability goal is the set of states representing a V-formation. A key assumption in~\cite{lukina2016arxiv} was that the reachability goal can be specified using a fitness function $J$,   
which assigns a non-negative real (fitness) value to each state in $\M$. 
%Moreover, $J$ was defined so that $J(s)\,{<}\,{\varphi}$, for a small $\varphi$, for exactly those states $s$ that correspond to a V-formation. {This is mentioned after eq.2} 
%Each state of $\mathcal{M}$ is represented by a fitness function reflecting essential motives for birds to fly in a V-formation in the first place.

The fitness of a state was determined by the following three terms: 
%% using three terms that measured if: (a)~birds have a clear visual field not blocked by any bird in front of them (\emph{clear view}); (b)~birds have the same velocity (\emph{velocity matching}); and
%% (c)~birds are located in space so that they are positioned in the upwash region of the bird(s) in front of them (\emph{upwash benefit}). 
\begin{itemize}
	\vspace*{-1.4mm}\item \emph{Clear View} ($\CV$). A bird's visual field is a cone with 
    angle $\theta$ that can be blocked by the wings of other birds. 
    The clear-view metric is defined by accumulating the percentage of a bird's visual 
    field that is blocked by other birds. The CV for the flock is the sum of the clear-view
    metric of all birds. The minimum value of $\CV$ 
    is $\CV^*{=}\,0$, and this value is attained in a perfect V-formation where all birds have clear view.
    
	\vspace*{1mm}\item \emph{Velocity Matching} ($\VM$). $\VM$ is defined as the  
	difference between the velocity of a given bird and all other birds, 
    summed up over all birds in the flock. The minimum
    value for $\VM$ is $\VM^*{=}\,0$, and this value is attained in a perfect
    V-formation where all birds have the same 
    velocity.
    
	\vspace*{1mm}\item \emph{Upwash Benefit} ($\UB$). The trailing upwash is 
    generated near the wingtips of a bird, while downwash is generated near 
    the center of a bird.  An upwash measure $um$ is defined on the 2D space using a 
    Gaussian-like model that peaks at the appropriate upwash and downwash regions. 
    %{Jesse: I commented this out as $um$ could be larger than 1 due to downwash} By design, the maximum $um$ for a bird is~1. 
    For bird~$i$ with upwash $um_i$, the upwash-benefit metric $\UB_i$ is defined as
    $1\,{-}um_i$, and $\UB$ for the flock is the sum of all $\UB_i$ $\forall~i\,{\in}\,\{1,\ldots,B\}$.
    The upwash benefit $\UB$ of a flock in V-formation is $\UB^*\,{=}\,1$, as all birds, except for the leader, have 
    minimum upwash-benefit metric ($\UB_i=0, um_i=1$),
    while the leader has upwash-benefit metric of $1$ ($\UB_i=1, um_i=0$).
\end{itemize}

\noindent{}Let $s=\{\xv_i, \vv_i\}_{i=1}^B$ be a state of a flock with $B$ birds. Given the above metrics, the overall fitness (cost) metric $J$ 
is of a sum-of-squares combination of $\VM$, $\CV$, and $\UB$ defined as follows:
\vspace*{-1mm}
\begin{align}
J(s) = (\CV(s)-\CV^*)^2 + (\VM(s)-\VM^*)^2 +(\UB(s)-\UB^*)^2.
\label{eq:fitness}
\end{align}
A state $s^{*}$ is considered to be a V-formation whenever $J(s^{*})\,{<}\,\varphi$, for a certain small threshold $\varphi$.

Given the above flocking model, the goal is to bring the flock from any configuration to a V-formation. Recall that we had two sets of control variables: accelerations $\vec{a}$ and displacements $\vec{d}$ for each bird of the flock.
We consider the scenario where the accelerations are under the control of one agent (the controller), and the displacements (position perturbations) are under the control of a second malicious agent (the attacker). This partition of the actions of the MDP into disjoint sets gives rise to a stochastic game on an MDP, which is described next.