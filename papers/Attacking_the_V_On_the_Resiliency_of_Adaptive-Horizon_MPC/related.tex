\vspace*{-0.6ex}
\section{Related Work}
\label{sec:related}

In the field of CPS security, one of the most widely studied attacks is \emph{sensor spoofing}.  When sensors measurements are compromised, state estimation becomes challenging, which inspired a considerable amount of work on attack-resilient state estimation~\cite{DBLP:journals/tac/FawziTD14,Bullo13:TAC,Pajic14:ICCPS,Pajic15:ICCPS,UAVspoofing}.
In these approaches, resilience to attacks is typically achieved by assuming the presence of redundant sensors, or coding sensor outputs.
In our work, we do not consider sensor spoofing attacks, but assume the attacker gets control of the displacement vectors (for some of the birds/drones).  We have not explicitly stated the mechanism by which an attacker obtains this capability, but it is easy to envision ways (radio controller, attack via physical medium, or other channels~\cite{savage}) for doing so.

Adaptive control, and its special case of adaptive model predictive control, typically refers to the aspect of the controller updating its process model that it uses to compute the control action. The field of adaptive control is concerned with the discrepancy between the actual process and its model used by the controller. In our adaptive-horizon MPC, we adapt the lookahead horizon employed by the MPC, and not the model itself.  Hence, the work in this paper is orthogonal to what is done in adaptive control~\cite{adaptive_control,adaptive_mpc}.

A key focus in CPS security has also been detection of attacks. For example, recent work considers displacement-based attacks on formation flight~\cite{aiaa2016}, but it primarily concerned with detecting which UAV was attacked using an unknown-input-observer based approach. We are not concerned with detecting attacks, but establishing that the adaptive nature of our controller provides attack-resilience for free. Moreover, in our setting, for both the attacker the and controller the state of the plant is completely observable.

We are unaware of any work that uses statistical model checking to evaluate the resilience of adaptive controllers against (certain classes of) attacks.