% This is based on the LLNCS.DEM the demonstration file of
% the LaTeX macro package from Springer-Verlag
% for Lecture Notes in Computer Science,
% version 2.4 for LaTeX2e as of 16. April 2010
%
% See http://www.springer.com/computer/lncs/lncs+authors?SGWID=0-40209-0-0-0
% for the full guidelines.
%
\documentclass{llncs}

\usepackage[table,pdftex,dvipsnames]{xcolor}
\usepackage[cmex10]{amsmath}
%\usepackage{todonotes}
\usepackage{url}
%\usepackage{algorithm}
\usepackage[caption=false,font=footnotesize]{subfig}
\usepackage{amsmath}
\usepackage{booktabs}
\usepackage{cite}
\usepackage[utf8]{inputenc}
\usepackage{stfloats}
\usepackage{amssymb}
\usepackage[ruled,linesnumbered,lined,boxed,commentsnumbered]{algorithm2e}
\usepackage{tikz}
\usepackage{multirow}
\usepackage{makecell}

\usetikzlibrary{shapes,arrows,positioning,calc}
\setcounter{tocdepth}{3}
\usepackage{tikz}

\usepackage{float}

\usepackage{xargs}                      % Use more than one optional parameter in a new commands

% flocking notations
\newcommand{\xv}{{\boldsymbol {x}}}
\newcommand{\vv}{{\boldsymbol {v}}}
\newcommand{\va}{{\boldsymbol {a}}}
\newcommand{\vd}{{\boldsymbol {d}}}
\newcommand{\VM}{{\it VM}}
\newcommand{\CV}{{\it CV}}
\newcommand{\UB}{{\it UB}}
% MDP notations
\newcommand{\M}{{\mathcal{M}}}

\usepackage[acronym]{glossaries}

\usepackage[colorinlistoftodos,prependcaption,textsize=tiny]{todonotes}
\newcommandx{\unsure}[2][1=]{\todo[linecolor=red,backgroundcolor=red!25,bordercolor=red,#1]{#2}}
\newcommandx{\change}[2][1=]{\todo[linecolor=blue,backgroundcolor=blue!25,bordercolor=blue,#1]{#2}}
\newcommandx{\info}[2][1=]{\todo[linecolor=OliveGreen,backgroundcolor=OliveGreen!25,bordercolor=OliveGreen,#1]{#2}}
\newcommandx{\improvement}[2][1=]{\todo[linecolor=Plum,backgroundcolor=Plum!25,bordercolor=Plum,#1]{#2}}
\newcommandx{\thiswillnotshow}[2][1=]{\todo[disable,#1]{#2}}
\DeclareMathOperator*{\argmin}{\arg\min}

\pagestyle{plain}   % to add page numbering-remove for final version

\begin{document}

\title{Attacking the V:\\ 
On the Resiliency of Adaptive-Horizon MPC}

%% \title{LEREC: Level-Based Receding-Horizon Control for Attack-Resilient Formations}
%
\titlerunning{Attacking flight formations using global controller}  % abbreviated title (for running head)
%                                     also used for the TOC unless
%                                     \toctitle is used
%
%\author{}
\author{Scott A. Smolka\inst{1} \and Ashish Tiwari\inst{2} \and Lukas Esterle\inst{3} \and Anna Lukina\inst{3} \\ Junxing Yang\inst{1} \and Radu Grosu\inst{1,3}}
%
%% \authorrunning{omitted for review}%Lukina, Esterle, Hirsch, Bartocci, Yang, Tiwari, Smolka, Grosu} % abbreviated author list (for running head)
%
%%%% list of authors for the TOC (use if author list has to be modified)
\tocauthor{}
%
%\institute{}
\institute{ Department of Computer Science, Stony Brook University, USA \\
    \and SRI International, USA \\
    \and Cyber-Physical Systems Group, Technische Universit\"at Wien, Austria
	}

\maketitle        % typeset the title of the contribution

\begin{abstract}
%% The abstract should summarize the contents of the paper
%% using at least 70 and at most 150 words. It will be set in 9-point
%% font size and be inset 1.0 cm from the right and left margins.
%% There will be two blank lines before and after the Abstract. \dots
We introduce the concept of a \emph{V-formation game} between a controller and an attacker, where controller's goal is to maneuver the plant (a simple model of flocking dynamics) into a V-formation, and the goal of the attacker is to prevent the controller from doing so.  
%% The controller can attain its goal by minimizing a certain flock-wide fitness 
%% function $J$, which is (almost) zero exactly when V-formation has been reached.  
%% Conversely, the attacker seeks to maximize $J$. 
%% We formalize V-formation games in terms of a Markov Decision Process (MDP) in which the controller and attacker jointly determine the transition probabilities.
Controllers in V-formation games utilize a new formulation of model-predictive control we call \emph{Adaptive-Horizon MPC} (AMPC), giving them extraordinary power: we prove that under certain controllability assumptions, an AMPC controller is able to attain V-formation with probability~1.
%% find a sequence of control actions (flock-wide accelerations) that brings the MDP to %% a V-formation goal state with probability one.

We define several classes of attackers, including those that in one move can remove
%%  a small number 
$R$ birds from the flock, or introduce random displacement
%% (perturbation)
into flock dynamics. 
%%again by selecting a small number of victim agents.  
We consider both \emph{naive attackers}, whose strategies are purely probabilistic, and \emph{AMPC-enabled attackers}, putting them on par strategically with the controllers.
%% in V-formation games.
While an AMPC-enabled controller is expected to win every game with probability~1, in practice, it is \emph{resource-constrained}: its maximum prediction horizon and the maximum number of game execution steps are fixed.  Under these conditions, an attacker has a much better chance of winning a V-formation game.

Our extensive performance evaluation of V-formation games uses statistical model checking to estimate the probability
%% by which 
an attacker can thwart the controller.  Our results show that for the bird-removal game with $R\,{=}\,1$, the controller almost always wins (restores the flock to a V-formation). For $R\,{=}\,2$, the game outcome critically depends on which two birds are removed.
%% : as long as the two removed birds are not adjacent to one another, the controller 
%% wins (is resilient).  
For the displacement game, our results again demonstrate that an intelligent attacker, i.e.~one that uses AMPC in this case, significantly outperforms its naive counterpart
%% that simply and uniformly at random
that randomly executes its attack.
%% \keywords{computational geometry, graph theory, Hamilton cycles}
\end{abstract}

	% !TEX root = ../arxiv.tex

Unsupervised domain adaptation (UDA) is a variant of semi-supervised learning \cite{blum1998combining}, where the available unlabelled data comes from a different distribution than the annotated dataset \cite{Ben-DavidBCP06}.
A case in point is to exploit synthetic data, where annotation is more accessible compared to the costly labelling of real-world images \cite{RichterVRK16,RosSMVL16}.
Along with some success in addressing UDA for semantic segmentation \cite{TsaiHSS0C18,VuJBCP19,0001S20,ZouYKW18}, the developed methods are growing increasingly sophisticated and often combine style transfer networks, adversarial training or network ensembles \cite{KimB20a,LiYV19,TsaiSSC19,Yang_2020_ECCV}.
This increase in model complexity impedes reproducibility, potentially slowing further progress.

In this work, we propose a UDA framework reaching state-of-the-art segmentation accuracy (measured by the Intersection-over-Union, IoU) without incurring substantial training efforts.
Toward this goal, we adopt a simple semi-supervised approach, \emph{self-training} \cite{ChenWB11,lee2013pseudo,ZouYKW18}, used in recent works only in conjunction with adversarial training or network ensembles \cite{ChoiKK19,KimB20a,Mei_2020_ECCV,Wang_2020_ECCV,0001S20,Zheng_2020_IJCV,ZhengY20}.
By contrast, we use self-training \emph{standalone}.
Compared to previous self-training methods \cite{ChenLCCCZAS20,Li_2020_ECCV,subhani2020learning,ZouYKW18,ZouYLKW19}, our approach also sidesteps the inconvenience of multiple training rounds, as they often require expert intervention between consecutive rounds.
We train our model using co-evolving pseudo labels end-to-end without such need.

\begin{figure}[t]%
    \centering
    \def\svgwidth{\linewidth}
    \input{figures/preview/bars.pdf_tex}
    \caption{\textbf{Results preview.} Unlike much recent work that combines multiple training paradigms, such as adversarial training and style transfer, our approach retains the modest single-round training complexity of self-training, yet improves the state of the art for adapting semantic segmentation by a significant margin.}
    \label{fig:preview}
\end{figure}

Our method leverages the ubiquitous \emph{data augmentation} techniques from fully supervised learning \cite{deeplabv3plus2018,ZhaoSQWJ17}: photometric jitter, flipping and multi-scale cropping.
We enforce \emph{consistency} of the semantic maps produced by the model across these image perturbations.
The following assumption formalises the key premise:

\myparagraph{Assumption 1.}
Let $f: \mathcal{I} \rightarrow \mathcal{M}$ represent a pixelwise mapping from images $\mathcal{I}$ to semantic output $\mathcal{M}$.
Denote $\rho_{\bm{\epsilon}}: \mathcal{I} \rightarrow \mathcal{I}$ a photometric image transform and, similarly, $\tau_{\bm{\epsilon}'}: \mathcal{I} \rightarrow \mathcal{I}$ a spatial similarity transformation, where $\bm{\epsilon},\bm{\epsilon}'\sim p(\cdot)$ are control variables following some pre-defined density (\eg, $p \equiv \mathcal{N}(0, 1)$).
Then, for any image $I \in \mathcal{I}$, $f$ is \emph{invariant} under $\rho_{\bm{\epsilon}}$ and \emph{equivariant} under $\tau_{\bm{\epsilon}'}$, \ie~$f(\rho_{\bm{\epsilon}}(I)) = f(I)$ and $f(\tau_{\bm{\epsilon}'}(I)) = \tau_{\bm{\epsilon}'}(f(I))$.

\smallskip
\noindent Next, we introduce a training framework using a \emph{momentum network} -- a slowly advancing copy of the original model.
The momentum network provides stable, yet recent targets for model updates, as opposed to the fixed supervision in model distillation \cite{Chen0G18,Zheng_2020_IJCV,ZhengY20}.
We also re-visit the problem of long-tail recognition in the context of generating pseudo labels for self-supervision.
In particular, we maintain an \emph{exponentially moving class prior} used to discount the confidence thresholds for those classes with few samples and increase their relative contribution to the training loss.
Our framework is simple to train, adds moderate computational overhead compared to a fully supervised setup, yet sets a new state of the art on established benchmarks (\cf \cref{fig:preview}).
         
    \section{Background and Motivation}

\subsection{IBM Streams}

IBM Streams is a general-purpose, distributed stream processing system. It
allows users to develop, deploy and manage long-running streaming applications
which require high-throughput and low-latency online processing.

The IBM Streams platform grew out of the research work on the Stream Processing
Core~\cite{spc-2006}.  While the platform has changed significantly since then,
that work established the general architecture that Streams still follows today:
job, resource and graph topology management in centralized services; processing
elements (PEs) which contain user code, distributed across all hosts,
communicating over typed input and output ports; brokers publish-subscribe
communication between jobs; and host controllers on each host which
launch PEs on behalf of the platform.

The modern Streams platform approaches general-purpose cluster management, as
shown in Figure~\ref{fig:streams_v4_v6}. The responsibilities of the platform
services include all job and PE life cycle management; domain name resolution
between the PEs; all metrics collection and reporting; host and resource
management; authentication and authorization; and all log collection. The
platform relies on ZooKeeper~\cite{zookeeper} for consistent, durable metadata
storage which it uses for fault tolerance.

Developers write Streams applications in SPL~\cite{spl-2017} which is a
programming language that presents streams, operators and tuples as
abstractions. Operators continuously consume and produce tuples over streams.
SPL allows programmers to write custom logic in their operators, and to invoke
operators from existing toolkits. Compiled SPL applications become archives that
contain: shared libraries for the operators; graph topology metadata which tells
both the platform and the SPL runtime how to connect those operators; and
external dependencies. At runtime, PEs contain one or more operators. Operators
inside of the same PE communicate through function calls or queues. Operators
that run in different PEs communicate over TCP connections that the PEs
establish at startup. PEs learn what operators they contain, and how to connect
to operators in other PEs, at startup from the graph topology metadata provided
by the platform.

We use ``legacy Streams'' to refer to the IBM Streams version 4 family. The
version 5 family is for Kubernetes, but is not cloud native. It uses the
lift-and-shift approach and creates a platform-within-a-platform: it deploys a
containerized version of the legacy Streams platform within Kubernetes.

\subsection{Kubernetes}

Borg~\cite{borg-2015} is a cluster management platform used internally at Google
to schedule, maintain and monitor the applications their internal infrastructure
and external applications depend on. Kubernetes~\cite{kube} is the open-source
successor to Borg that is an industry standard cloud orchestration platform.

From a user's perspective, Kubernetes abstracts running a distributed
application on a cluster of machines. Users package their applications into
containers and deploy those containers to Kubernetes, which runs those
containers in \emph{pods}. Kubernetes handles all life cycle management of pods,
including scheduling, restarting and migration in case of failures.

Internally, Kubernetes tracks all entities as \emph{objects}~\cite{kubeobjects}.
All objects have a name and a specification that describes its desired state.
Kubernetes stores objects in etcd~\cite{etcd}, making them persistent,
highly-available and reliably accessible across the cluster. Objects are exposed
to users through \emph{resources}. All resources can have
\emph{controllers}~\cite{kubecontrollers}, which react to changes in resources.
For example, when a user changes the number of replicas in a
\code{ReplicaSet}, it is the \code{ReplicaSet} controller which makes sure the
desired number of pods are running. Users can extend Kubernetes through
\emph{custom resource definitions} (CRDs)~\cite{kubecrd}. CRDs can contain
arbitrary content, and controllers for a CRD can take any kind of action.

Architecturally, a Kubernetes cluster consists of nodes. Each node runs a
\emph{kubelet} which receives pod creation requests and makes sure that the
requisite containers are running on that node. Nodes also run a
\emph{kube-proxy} which maintains the network rules for that node on behalf of
the pods. The \emph{kube-api-server} is the central point of contact: it
receives API requests, stores objects in etcd, asks the scheduler to schedule
pods, and talks to the kubelets and kube-proxies on each node. Finally,
\emph{namespaces} logically partition the cluster. Objects which should not know
about each other live in separate namespaces, which allows them to share the
same physical infrastructure without interference.

\subsection{Motivation}
\label{sec:motivation}

Systems like Kubernetes are commonly called ``container orchestration''
platforms. We find that characterization reductive to the point of being
misleading; no one would describe operating systems as ``binary executable
orchestration.'' We adopt the idea from Verma et al.~\cite{borg-2015} that
systems like Kubernetes are ``the kernel of a distributed system.'' Through CRDs
and their controllers, Kubernetes provides state-as-a-service in a distributed
system. Architectures like the one we propose are the result of taking that view 
seriously.

The Streams legacy platform has obvious parallels to the Kubernetes
architecture, and that is not a coincidence: they solve similar problems.
Both are designed to abstract running arbitrary user-code across a distributed
system.  We suspect that Streams is not unique, and that there are many
non-trivial platforms which have to provide similar levels of cluster
management.  The benefits to being cloud native and offloading the platform
to an existing cloud management system are: 
\begin{itemize}
    \item Significantly less platform code.
    \item Better scheduling and resource management, as all services on the cluster are 
        scheduled by one platform.
    \item Easier service integration.
    \item Standardized management, logging and metrics.
\end{itemize}
The rest of this paper presents the design of replacing the legacy Streams 
platform with Kubernetes itself.

  
    We briefly recall the framework of statistical inference via empirical risk minimization.
Let $(\bbZ, \calZ)$ be a measurable space.
Let $Z \in \bbZ$ be a random element following some unknown distribution $\Prob$.
Consider a parametric family of distributions $\calP_\Theta := \{P_\theta: \theta \in \Theta \subset \reals^d\}$ which may or may not contain $\Prob$.
We are interested in finding the parameter $\theta_\star$ so that the model $P_{\theta_\star}$ best approximates the underlying distribution $\Prob$.
For this purpose, we choose a \emph{loss function} $\score$ and minimize the \emph{population risk} $\risk(\theta) := \Expect_{Z \sim \Prob}[\score(\theta; Z)]$.
Throughout this paper, we assume that
\begin{align*}
     \theta_\star = \argmin_{\theta \in \Theta} L(\theta)
\end{align*}
uniquely exists and satisfies $\theta_\star \in \text{int}(\Theta)$, $\nabla_\theta L(\theta_\star) = 0$, and $\nabla_\theta^2 L(\theta_\star) \succ 0$.

\myparagraph{Consistent loss function}
We focus on loss functions that are consistent in the following sense.

\begin{customasmp}{0}\label{asmp:proper_loss}
    When the model is \emph{well-specified}, i.e., there exists $\theta_0 \in \Theta$ such that $\Prob = P_{\theta_0}$, it holds that $\theta_0 = \theta_\star$.
    We say such a loss function is \emph{consistent}.
\end{customasmp}

In the statistics literature, such loss functions are known as proper scoring rules \citep{dawid2016scoring}.
We give below two popular choices of consistent loss functions.

\begin{example}[Maximum likelihood estimation]
    A widely used loss function in statistical machine learning is the negative log-likelihood $\score(\theta; z) := -\log{p_\theta(z)}$ where $p_\theta$ is the probability mass/density function for the discrete/continuous case.
    When $\Prob = P_{\theta_0}$ for some $\theta_0 \in \Theta$,
    we have $L(\theta) = \Expect[-\log{p_\theta(Z)}] = \kl(p_{\theta_0} \Vert p_\theta) - \Expect[\log{p_{\theta_0}(Z)}]$ where $\kl$ is the Kullback-Leibler divergence.
    As a result, $\theta_0 \in \argmin_{\theta \in \Theta} \kl(p_{\theta_0} \Vert p_\theta) = \argmin_{\theta \in \Theta} L(\theta)$.
    Moreover, if there is no $\theta$ such that $p_\theta \txtover{a.s.}{=} p_{\theta_0}$, then $\theta_0$ is the unique minimizer of $L$.
    We give in \Cref{tab:glms} a few examples from the class of generalized linear models (GLMs) proposed by \citet{nelder1972generalized}.
\end{example}

\begin{example}[Score matching estimation]
    Another important example appears in \emph{score matching} \citep{hyvarinen2005estimation}.
    Let $\bbZ = \reals^\tau$.
    Assume that $\Prob$ and $P_\theta$ have densities $p$ and $p_\theta$ w.r.t the Lebesgue measure, respectively.
    Let $p_\theta(z) = q_\theta(z) / \Lambda(\theta)$ where $\Lambda(\theta)$ is an unknown normalizing constant. We can choose the loss
    \begin{align*}
        \score(\theta; z) := \Delta_z \log{q_\theta(z)} + \frac12 \norm{\nabla_z \log{q_\theta(z)}}^2 + \text{const}.
    \end{align*}
    Here $\Delta_z := \sum_{k=1}^p \partial^2/\partial z_k^2$ is the Laplace operator.
    Since \cite[Thm.~1]{hyvarinen2005estimation}
    \begin{align*}
        L(\theta) = \frac12 \Expect\left[ \norm{\nabla_z q_\theta(z) - \nabla_z p(z)}^2 \right],
    \end{align*}
    we have, when $p = p_{\theta_0}$, that $\theta_0 \in \argmin_{\theta \in \Theta} L(\theta)$.
    In fact, when $q_\theta > 0$ and there is no $\theta$ such that $p_\theta \txtover{a.s.}{=} p_{\theta_0}$, the true parameter $\theta_0$ is the unique minimizer of $L$ \cite[Thm.~2]{hyvarinen2005estimation}.
\end{example}

\myparagraph{Empirical risk minimization}
Assume now that we have an i.i.d.~sample $\{Z_i\}_{i=1}^n$ from $\Prob$.
To learn the parameter $\theta_\star$ from the data, we minimize the empirical risk to obtain the \emph{empirical risk minimizer}
\begin{align*}
    \theta_n \in \argmin_{\theta \in \Theta} \left[ L_n(\theta) := \frac1n \sum_{i=1}^n \score(\theta; Z_i) \right].
\end{align*}
This applies to both maximum likelihood estimation and score matching estimation. 
In \Cref{sec:main_results}, we will prove that, with high probability, the estimator $\theta_n$ exists and is unique under a generalized self-concordance assumption.

\begin{figure}
    \centering
    \includegraphics[width=0.45\textwidth]{graphs/logistic-dikin} %0.4
    \caption{Dikin ellipsoid and Euclidean ball.}
    \label{fig:logistic_dikin}
\end{figure}

\myparagraph{Confidence set}
In statistical inference, it is of great interest to quantify the uncertainty in the estimator $\theta_n$.
In classical asymptotic theory, this is achieved by constructing an asymptotic confidence set.
We review here two commonly used ones, assuming the model is well-specified.
We start with the \emph{Wald confidence set}.
It holds that $n(\theta_n - \theta_\star)^\top H_n(\theta_n) (\theta_n - \theta_\star) \rightarrow_d \chi_d^2$, where $H_n(\theta) := \nabla^2 L_n(\theta)$.
Hence, one may consider a confidence set $\{\theta: n(\theta_n - \theta)^\top H_n(\theta_n) (\theta_n - \theta) \le q_{\chi_d^2}(\delta) \}$ where $q_{\chi_d^2}(\delta)$ is the upper $\delta$-quantile of $\chi_d^2$.
The other is the \emph{likelihood-ratio (LR) confidence set} constructed from the limit $2n [L_n(\theta_\star) - L_n(\theta_n)] \rightarrow_d \chi_d^2$, which is known as the Wilks' theorem \citep{wilks1938large}.
These confidence sets enjoy two merits: 1) their shapes are an ellipsoid (known as the \emph{Dikin ellipsoid}) which is adapted to the optimization landscape induced by the population risk; 2) they are asymptotically valid, i.e., their coverages are exactly $1 - \delta$ as $n \rightarrow \infty$.
However, due to their asymptotic nature, it is unclear how large $n$ should be in order for it to be valid.

Non-asymptotic theory usually focuses on developing finite-sample bounds for the \emph{excess risk}, i.e., $\Prob(L(\theta_n) - L(\theta_\star) \le C_n(\delta)) \ge 1 - \delta$.
To obtain a confidence set, one may assume that the population risk is twice continuously differentiable and $\lambda$-strongly convex.
Consequently, we have $\lambda \norm{\theta_n - \theta_\star}_2^2 / 2 \le L(\theta_n) - L(\theta_\star)$ and thus we can consider the confidence set $\calC_{\text{finite}, n}(\delta) := \{\theta: \norm{\theta_n - \theta}_2^2 \le 2C_n(\delta)/\lambda\}$.
Since it originates from a finite-sample bound, it is valid for fixed $n$, i.e., $\Prob(\theta_\star \in \calC_{\text{finite}, n}(\delta)) \ge 1 - \delta$ for all $n$; however, it is usually conservative, meaning that the coverage is strictly larger than $1 - \delta$.
Another drawback is that its shape is a Euclidean ball which remains the same no matter which loss function is chosen.
We illustrate this phenomenon in \Cref{fig:logistic_dikin}.
Note that a similar observation has also been made in the bandit literature \citep{faury2020improved}.

We are interested in developing finite-sample confidence sets.
However, instead of using excess risk bounds and strong convexity, we construct in \Cref{sec:main_results} the Wald and LR confidence sets in a non-asymptotic fashion, under a generalized self-concordance condition.
These confidence sets have the same shape as their asymptotic counterparts while maintaining validity for fixed $n$.
These new results are achieved by characterizing the critical sample size enough to enter the asymptotic regime.

    \section{The Adaptive-Horizon MPC Algorithm} % {Attack Strategies}
\label{sec:ampc}

We now present our new \emph{adaptive-horizon} \emph{model-predictive-control} algorithm, we call AMPC. We will use this algorithm as the controller strategy in the stochastic game on MDPs. We will also consider attack strategies that use AMPC. Since AMPC is an adaptive MPC procedure based on particle-swarm optimization (PSO), we first
briefly present background material on MPC and PSO.
%\todo[inline]{The rest of this section does not belong here. It should be moved in the description of the controller. Similar text can be used then for the advanced attacker.}

\subsection{Background on Model-Predictive Control}
%One way to find an action that can take a flock to a V-formation is based on using model-predictive control (MPC).  
Model-predictive control (MPC) determines the control action at current time $t$ by looking $h$ steps into the future and finding the best $h$-length sequence of control actions that can take the system from its current state $s(t)$ to a new state that has the lowest fitness. (Since we assume existence of a fitness metric $J$ that we are trying to minimize, we specialize the description of MPC to this case.) 
If ${s}_{\va^h}(t+h)$ denotes the state reached from state ${s}(t)$ in time $h$ following the actions $\va^h$ of length $h$, then in the MPC approach, at each time step $t$, the following minimization is performed to find the optimal set of actions
%\begin{align}
%&\textbf{opt-$\va$}^{h}(t)=\{\textbf{opt-$\va$}_i^{h}(t)\}_{i=1}^{b}=\argmin_{\va^h(t)}J(\boldsymbol{c}_{\va^h}(t+h)).
%\label{eq:opt}
%\end{align}
\begin{align}
&\textbf{opt-$\va$}^{h}(t)=\argmin_{\va^h(t)}J(s_{\va^h}(t+h)).
\label{eq:opt}
\end{align}
Since the model is an approximation of the system, only the first action $\va(t) = \textbf{opt-$\va$}^{1}(t)$ is applied as the action at time $t$,
and the remaining future $h-1$ actions found by the optimizer are ignored.  After the control action $\va(t)$ is applied, the system is left to evolve, and the process is repeated at $t\,{+}\,1, t\,{+}\,2,$ and so on.
%
%
%\vspace*{-1mm}
%\begin{align}
%J(\boldsymbol{c}(t),\va^h(t),{h}) = (\CV(\boldsymbol{c}_{\va}^{h}(t))-\CV^*)^2 &+ 
%(\VM(\boldsymbol{c}_{\va}^{h}(t))-\VM^*)^2 \nonumber \\ & +(\UB(\boldsymbol{c}_{\va}^{h}(t))-\UB^*)^2,
%\label{eq:fitness}
%\end{align}
%\noindent{}where ${h}$ is the length of the receding prediction horizon (RPH), $\va^h(t)$ is a sequence of accelerations of length ${h}$, and $\boldsymbol{c}_{\va}^{h}(t)$ is 
%the configuration reached after applying $\va^h(t)$ to $\boldsymbol{c}(t)$.
%
%

The MPC approach can be used for achieving a V-formation, as was outlined in~\cite{yang2016bda,yang2016love}.  These earlier works, however, did not use an adaptive dynamic window, and did not consider the adversarial control problem.
%We perform a single flock-wide minimization of $J$ at each time-step $t$ to obtain an optimal plan of  length $h$ of acceleration actions:\vspace*{-3mm}
%
%\begin{align}
%&\textbf{opt-$\va$}^{h}(t)=\{\textbf{opt-$\va$}_i^{h}(t)\}_{i=1}^{b}=\argmin_{\va^h(t)}J(\boldsymbol{c}(t),\va^h(t),{h}).
%\label{eq:opt}
%\end{align}
%\vspace*{-3mm}\noindent{} 
%
%We apply the first acceleration $\va_i(t) = \textbf{opt-$\va$}_i^{1}(t)$ as the optimal acceleration for bird $i$ %at time $t$. 
%

In MPC, optimization problem (\ref{eq:opt}) is additionally subject to constraints that bound the set of possible actions and states. For example, in our flocking model, the
magnitude of velocity and acceleration for each of the $B$ birds is bounded: $||\vv_i(t)||\,{\leqslant}\,\vv_{max},
||\va^h_i(t)||\,{\leqslant}\,\rho||\vv_i(t)||$ $\forall$ $i\,{\in}\,\{1,\ldots,B\}$,
where $\vv_{max}$ is a predefined constant and $\rho\,{\in}\,(0,1)$. 
% The initial state is selected following the given initial distribution. In the investigation of V-formation conducted in~\cite{lukina2016arxiv}, the initial positions and velocities of each bird are selected at random 
% within certain ranges, and limited such that the distance between any 
% two birds is greater than a (collision) constant $d_{min}$, and small
% enough so that each bird finds itself in the upwash or downwash region of at least one other bird. 

We use a \emph{particle-swarm-optimization algorithm} to solve the optimization problems generated by the MPC procedure.
%\todo[inline]{I added the following assuming we want to refer to the ARES approach - this might to be replaced if we want to use MPC instead. (I am not sure what Jesse used for his experiments right now)}
%\todo[inline]{Anna: as far as I understand Jesse is using MPC level-based with receding horizon, however, there are no PSO clones or important splitting}

% ASHISH: I think we should comment out the following; otherwise, we will look to be proposing something very close to TACAS paper.
%To solve the optimization problem, Lukina et al.~\cite{lukina2016arxiv} present an approach using \emph{Particle Swarm Optimization (PSO)}~\cite{Kennedy95particleswarm} to find potential next actions, combined with the idea of \emph{Importance Splitting}~\cite{kahn1951} in order to increase the chance of reaching a V-formation. Furthermore, they introduce adaptive horizons to allow for temporary worse situation in the flock enabling them to overcome local minima in the fitness function. Additionally, they introduced an adaptive number of particles for the PSO allowing them to better exploit the heuristics of PSO in combination with Importance Splitting as well as achieve a speed-up in the performance of their approach. However, their approach generates a plan offline to traverses the deterministic MDP in order to reach a specific state. In contrast, we present an approach to control the flock of birds online. The resilience of our approach is demonstrated in the presence of adversaries and noise.
 
    \subsection{Background on Particle Swarm Optimization}
\label{sec:swarmOptimization}

Particle Swarm Optimization (PSO) is a randomized approximation algorithm for determining the parameters that minimize a possibly nonlinear and possibly discontinuous cost (or fitness) function. PSO was first introduced by~\cite{Kennedy95particleswarm}. In an interesting twist of events, PSO took its original inspiration from bird flocking.


%As in~\cite{lukina2016arxiv}, our controller-PSO uses ``acceleration birds'' (these are the particles in the swarm). They should not be confused with the actual flocking birds. 
The PSO procedure is best described using the metaphor of a swarm of insects collaboratively trying to find the location of food. The insects, also called particles, live in the space defined by all possible valuations of the unknown parameters (of the optimization problem).  The food is located at the position where the objective function is minimized. PSO works by having a swarm of particles, which have the same goal of finding food (the reward) without knowing its location. Each particle is informed about its distance to the food (value of the objective function). The PSO algorithm repeatedly redistributes each particle towards the one closest to the food, with a speed proportional to the distance separating them, until all particles converge to the same position. 


AMPC employs Matlab's toolbox $\texttt{particleswarm}$, which performs the classical version of PSO. A swarm of $p$ particles is sampled uniformly at random within a given bound on their positions and velocities. In the bird flocking example, if we try to find acceleration vectors by optimization over horizon $h$, then {\em{one}} ``particle'' represents $h$ 2-dimensional vectors for each of the $B$ birds, along with a vector of values that determine how these $h\cdot B$ acceleration vectors will be updated. 
%
%In the games we are considering each particle represents either a flock of bird-accelerations sequence $\{\va_i^{h}\}_{i=1}^b$, or a a flock of displacements sequence $\{\vd_i^{h}\}_{i=1}^b$, where $h$ is the current length of the receding horizon. 
After choosing a neighborhood of random size for each particle $j$, $j\,{\in}\,\{1,\ldots,p\}$, PSO computes the value of the given fitness function for each particle, and stores two vectors for each particle $j$: its so-far personal-best position $\mathbf{x}_{P}^j(t)$, and the position of its fittest neighbor $\mathbf{x}_{G}^j(t)$. The positions and velocities of the particle swarm $j\,{\in}\,\{1,\ldots,p\}$ are updated the following way:

\vspace*{-4mm}
\begin{align}
\mathbf{v}^j(t+1) = \omega\cdot\mathbf{v}^j(t) &+ y_1\cdot \mathbf{u_1}(t+1)\otimes(\mathbf{x}_{P}^j(t)-\mathbf{x}^j(t))  \nonumber \\
&+ y_2\cdot \mathbf{u_2}(t+1)\otimes(\mathbf{x}_{G}^j(t)-\mathbf{x}^j(t)),
\label{eq:swarm}
\end{align}

\vspace*{-1mm}\noindent{}where $\omega$ is an \emph{inertia weight}, which quantifies the trade-off between global and local exploration of the swarm (the value of $\omega$ is proportional to the exploration range); $y_1$ and $y_2$ are the \emph{self adjustment} and the \emph{social adjustment}, respectively; $\mathbf{u_1},\mathbf{u_2}\,{\in}\,{\rm Uniform}(0,1)$ are random variables; and $\otimes$ is the vector dot product, that is, $\forall$ random vector $\mathbf{z}$: $(\mathbf{z}_1,\ldots,\mathbf{z}_b)\otimes(\mathbf{x}_1^j,\ldots,\mathbf{x}_b^j)=(\mathbf{z}_1\mathbf{x}_1^j,\ldots,\mathbf{z}_b\mathbf{x}_b^j)$.


If the value of the fitness computed at each step of the PSO
for $\mathbf{x}^j(t+1)\,{=}\,\mathbf{x}^j(t)\,{+}\,\mathbf{v}^j(t+1)$ falls below the one for $\mathbf{x}_{P}^j(t)$, then $\mathbf{x}^j(t+1)$ is reassigned to $\mathbf{x}_{P}^j(t+1)$. A global best for the next iteration is determined as the particle with the best fitness among $j\,{\in}\,\{1,\ldots,p\}$. The stopping criterion of the PSO algorithm is either reaching the maximum number of iterations set in advance, or reaching the set time bound, or satisfying the minimum criterion. 

PSO can be used to solve any optimization problem. We use it to solve the optimization problem generated in the MPC approach. In a V-formation game, it can be used to obtain the birds' best accelerations, or even the best displacements, at each time step -- depending on whether MPC/PSO is being used by the controller or the attacker.

\emph{Remark}. We assume that PSO is fair, in the sense that it has a chance to sample all the points in the parameter space, and therefore it has the chance to find the optimal solution with probability one, given enough time.

%In a similar spirit, our advanced-attacker-PSO uses so called "displacement birds" (particles).



    	\subsection{The Main Algorithm of AMPC}
\label{sec:lerec}
% Inspired by the ARES algorithm in~\cite{lukina2016arxiv}, we propose a level-based receding-horizon model-predictive control algorithm we call LEREC. 
We propose the main algorithm of AMPC. %short for level-based Adaptive-horizon Model-Predictive Control {Jesse: this is already mentioned in the beginning of the section}. 
This algorithm performs step-by-step control of a given MDP $\M$ by looking $h$ steps ahead and predicting the next best state to move to.
We use PSO to identify the potentially best actions $\va^h$ in the current state achieving the optimal value of the fitness function in the next state.  For bird flocking, the fitness function, \texttt{Fitness}$(\M,\va^h,h)$ of $\va^h$ is defined as the minimum fitness metric $J$ obtained within $h$ steps by applying $\va^h$ on $\M$. Formally, we have
\vspace*{-2mm}
\begin{align}
\texttt{Fitness}(\M,\va^h,h) = \min_{1\leqslant \tau \leqslant h}{J(s_{\va^h}^\tau)}
\end{align}
where $s_{\va^h}^\tau$ is the state after apply the $\tau$th action of $\va^h$ on $\M$. For horizon $h$, PSO searches for the best sequence of 2-dimensional acceleration vector of length $h$, thus having $2Bh$ parameters to be optimized. The number of particles used in PSO is proportional to the number of parameters, i.e., $p = 2\beta B h$.
%was defined in equation~(\ref{eq:fitness}). 

The pseudocode for the AMPC algorithm is given in Algorithm~\ref{alg:lerec}. A novel feature of AMPC is that, unlike classical MPC that uses a fixed horizon $h$, AMPC adaptively chooses an $h$ depending on whether it is able to reach a fitness value that is lower than the current fitness by our chosen quanta $\Delta_i$, $\forall~i\,{\in}\,\{0,\ldots,m\}$.
%we would have had to resort to local optima without any guarantee of reaching a stable state.

%Following~\cite{lukina2016arxiv} we introduce level-based horizon as a way to overcome shortcomings of MPC and increase effectiveness of the optimization process. 
%To overcome the shortcoming of MPC and in order to increase the effectiveness of the optimization process, we introduce 
AMPC is hence an adaptive MPC procedure that uses level-based horizons.
It employs PSO to identify the potentially best next actions.
%for our flock.
If the chosen actions improve (decrease) the fitness of the next state $J(s_{k+h})$, $\forall~k\,{\in}\,\{0,\ldots,m\cdot h_{\mathit{max}}\}$, in comparison to the fitness of the previous state $J(s_k)$ by the predefined $\Delta_i$, the controller considers these actions to be worthy of leading the flock towards or keeping it in the V-formation.%
\footnote{We focus our attention on bird flocking, since the details generalize naturally to other MDPs that come with a fitness metric.}

In this case, the controller applies the actions to each bird and transitions to the next state of the MDP.
The threshold $\Delta_i$ determines the next level $\ell_i\,{=}\,J(s_{k+\widehat{h}})$ of the algorithm, where $\widehat{h} \leqslant h$ is the horizon with the best fitness. The prediction horizon $h$ is increased iteratively if the fitness has not been decreased enough. Upon reaching a new level, the horizon is reset to one (see Algorithm~\ref{alg:lerec}). Having a horizon $\widehat{h}\,{>}\,1$ means it will take multiple transitions in the MDP in order to reach a solution with improved fitness. However, when finding such a solution with $\widehat{h}\,{>}\,1$, we only apply the first action to transition the MDP to the next state. This is explained by the need to allow the other player (environment or an adversary) to apply their action before we obtain the actual next state. 
%adjustment faster and deal with unforeseen situations arising during runtime.
%
If no new level is reached within $h_{\mathit{max}}$ horizons, the first action of the best $\va^h$ using horizon $h_{\mathit{max}}$ is applied. 

The dynamic threshold $\Delta_i$ is defined as in~\cite{lukina2016arxiv}. Its initial value $\Delta_0$ is obtained by dividing the fitness range to be covered into $m$ equal parts, that is, $\Delta_0\,{=}\,(\ell_0\,{-}\,\ell_m)\,{/}\,m$, where $\ell_0\,{=}\,J(s_0)$ and $\ell_m\,{=}\,\varphi$. Subsequently, $\Delta_i$ is determined by the previously reached level $\ell_{i-1}$, as $\Delta_i\,{=}\,\ell_{i-1}{/}(m\,{-}\,i\,{+}\,1)$. This way AMPC advances only if $\ell_i\,{=}\,J(s_{k+\widehat{h}})$ is at least $\Delta_i$ apart from $\ell_{i-1}\,{=}\,J(s_{k})$.

This approach allows us to force PSO to 
escape from a local minimum, even if this implies passing over a bump, by gradually increasing the exploration horizon $h$. We assume that the MDP is controllable and that the set $G$ of good states is not empty, which means, that from any state, it is possible to reach a state whose fitness decreased by at least $\Delta_i$. 
%Figure~\ref{fig:approach} illustrates our approach.
Algorithm~\ref{alg:lerec} illustrates our approach.

%, which terminates if the stable state has been reached or the time elapsed.
%\begin{figure}[t]
%\centering
%	%%%%%%%%%%%%%%%%%%% colors
\definecolor{ShineSky}{rgb}{0, 0.9, 1}
\definecolor{ShineGrass}{rgb}{0.85, 0.88, 0.59}
\definecolor{InfosysDarkGrey}{gray}{0.4}
\definecolor{InfosysLightGrey}{gray}{0.6}
\definecolor{TuWienBlue}{cmyk}{1,0.38,0,0.15}
\definecolor{TuInfRed}{cmyk}{0,1,1,0}

%%%%%%%%%%%%%%%% blocks
\tikzstyle{conf} = [draw, fill=Yellow!20, circle, node distance=0.7cm]
\tikzstyle{inv} = [draw, circle, node distance=0.7cm]
\tikzstyle{arrow} = [thick,->,>=stealth]
\tikzstyle{pso} = [rectangle, rounded corners, minimum width=0.5cm, minimum 
height=0.5cm,text centered, draw=black, fill=ShineSky!20]
\tikzstyle{fit} = [rectangle, rounded corners, minimum width=0.5cm, minimum 
height=0.5cm,text centered, draw=black, fill=ShineSky!20]
\tikzstyle{decision} = [diamond, aspect=1, minimum width=1cm, text 
centered, draw=black, fill=ShineSky!20]
\tikzstyle{output} = [coordinate]

\begin{tikzpicture}[auto, node distance=2cm,>=latex', cross/.style={path 
	picture={ 
		\draw[black]
		(path picture bounding box.south east) -- (path picture bounding 
		box.north west) (path picture bounding box.south west) -- (path picture 
		bounding box.north east);
	}}]
	
	%%%%%%%%%%%%%% configuration
	\node (start) [conf] {};
    \node [above = .2mm of start,text width=.3cm,align = center]{\scriptsize{$s_0$}};
	
	%%%%%%%%%%%%% PSO	
	\node(pso1) [pso, right=3mm of start] {\scriptsize{PSO}};

	\draw [->] (start) -- (pso1);
	
	\node (c1n) [conf, right = 6mm of pso1] {};

	\draw [->] (pso1) -- node {\scriptsize{$\va^{h}$}} (c1n);
	
	%%%%%%%%%%%%%%%%%%%%% Fitness
	
	\node(fit1) [fit, right=4mm of c1n] {\scriptsize{${J}$}};

	\draw [->] (c1n) -- (fit1);
	
	%%%%%%%%%%%%%%%%%% Conditions
	\node (dec1) [decision, right = 0.4cm of fit1] {\scriptsize{$\ell_{i-1} - 
	J > \Delta$}}; 
	\draw[->] (fit1) -- node{\scriptsize{}}(dec1);
    
    \node (fitCheck) [decision, aspect=3, below = 5mm of dec1]{\scriptsize{$J < \widehat{J}$}};
    
	\node (dec2) [decision,aspect=3, below = 5mm of fitCheck] {\scriptsize{${h} < 
	{h}_{max}$}};

% 	\node (applyMaxA) [rectangle, text centered, draw=black, rounded corners, fill=gray!50, below = 5mm of dec2] {\scriptsize{\textbf{Apply $a^h_1$}}};

 \node (setJh) [pso,text width=1cm, right =6mm of dec1]{\scriptsize{$\widehat{h}:=h$}\\\scriptsize{$\widehat{J}:=J$}};
    
    \node (step) [pso,right = 4mm of setJh]{\scriptsize{Apply $a^{\widehat{h}}_1$}};
	
	\node (incH) [pso,left = 10mm of dec2]{\scriptsize{$h\mathrel{++}$}};
	
% 	\node (dec5) [decision,aspect=3, below = 4mm of dec2] {\scriptsize{${p} < 
% 	{p}_{max}$}};
	
% 	\node (incP) [rectangle, text centered, draw=black, text width=1.5cm, 
% 	rounded corners, fill=gray!30] at (incH |- dec5){\scriptsize{$h:=1;$}\\\scriptsize{$p\mathrel{+}=p_{inc};$}};
	
 	\node (replace) [pso,text width=1cm] at (setJh |- fitCheck) {\scriptsize{$\widehat{J} \mathrel{:}= J$}\\\scriptsize{$\widehat{h} := h$}};
%     
	\node (incL) [coordinate,below = 4.5cm of pso1]{};
    
    
%     \node (dec4) [decision,aspect=3, below = 10mm of dec2] 
% 	{\scriptsize{$i < m$}};

%     \node (dec3) [decision,aspect=3] at (stuff |- dec4)
% 	{\scriptsize{$\ell_i > \varphi$}};

    \node (stuff) [pso,text width=1.2cm,align=left,right = 4mm of step] {\scriptsize{$\ell_i:=\widehat{J}$}\\\scriptsize{$i\mathrel{++}$}\\\scriptsize{$h:=1$}\\\scriptsize{$\widehat{J}:=inf$}};
    
    
    \node (setP) [pso, text width=1.5cm] at (incL |- replace){\scriptsize{$p\mathrel{:}=n \cdot h \cdot 2$}};

%     \node (resH) [rectangle, text centered, draw=black, rounded corners, 
% 	fill=gray!30, above = 3mm of incL]{\scriptsize{$h:=1$}};

	
  
	%%%%%%%%%%%%%% Arrows
	\draw[->](dec1.south) -- node(arr3){\scriptsize{No}} (fitCheck.north);
    \draw[->](fitCheck.south) -- node[right]{\scriptsize{No}}(dec2.north);
    \draw[->](dec1.east) -- node(arrApply){\scriptsize{Yes}}(setJh.west);
    \draw[->](setJh.east) -- (step);
	
	%\draw[->](stuff.south) -- (dec3.north);
    
	%\draw[->](dec3.west) -- node(arr1)[above]{\scriptsize{Yes}} (dec4.east);
	%\draw[->] (dec4.west) -- node(arr5)[above]{\scriptsize{Yes}} (incL.east);
% 	\draw[->] (incL.north) -- (resH.south);
%   \draw[->] (resH.north) -- (pso1.south);

	%\draw[->] (incL.north) -- (pso1.south);
    
    \draw[->] (incL.north) -- (setP.south);
    \draw[->] (setP.north) -- (pso1.south);
    
	%\draw[->](dec5.west) -- node(arr6)[above,near start]{\scriptsize{Yes}} 
	%(resH.east);
    \draw[->](dec2.west) -- node(arr6)[above,near start]{\scriptsize{Yes}} 
	(incH.east);
	\draw[->](incH.west) --  (incH -| incL);%(psoBox.south);
    %\draw[->](dec2.south) -- node[right]{\scriptsize{No}}(applyMaxA);
    
%     \draw[->](applyMaxA) -- (dec4);
	
	\draw[->](step.east) -- (stuff);
    \draw[->](fitCheck.east) -- node[above,near start]{\scriptsize{Yes}} (replace.west);
    \draw[->](replace.south) |- (dec2.east);

% 	\node(V) [below = 4mm of dec3,text width=2cm,align = 
% 	center]{\scriptsize{\textbf{Stable state}}};
% 	\draw[->] (dec3.south) --node[right]{\scriptsize{No}} (V);
% 	\node(time) [below = 4mm of dec4,text width=2cm,align = 
% 	center]{\scriptsize{\textbf{Timeout}}};
	
% 	\draw[->] (dec4.south) --node[right]{\scriptsize{No}} (time);

	\draw[-] (stuff.south) |- (incL.east);
	\draw[->] (dec2.south) --node[right]{\scriptsize{No}} ++(0,-.5) -|  (step.south);
	
	\end{tikzpicture}
%\vspace*{-2mm}
%\caption{Graphical representation of AMPC.}
%	\label{fig:approach}
    %\vspace*{-5mm}
%\end{figure}

%% \SetAlgoSkip{}
%%%%% inner loop for finding next best action
\begin{algorithm}[b]
	\SetKwFunction{Fitness}{Fitness}
	\SetKwFunction{particleswarm}{particleswarm}
	\SetKwInOut{Input}{Input}
	\SetKwInOut{Output}{Output}
    
    $[\va^{h},\M^{h}] \leftarrow$ \particleswarm{$\mathcal{M},{p},h$}; \textit{// use 
    PSO in 
    order to determine best next action for the MDP $\mathcal{M}$ with horizon $h$\\}
    ${J}(s_{k+h})\leftarrow$ 
    \Fitness{$\M^h,\va^{h},{h}$};
    \textit{// calculate cost function if applying the first optimal action of horizon sequence of length $h$}\\
    \If{${J}(s_{k+h})\leqslant\widehat{J}$}
    {
   % $\Delta\leftarrow{J}(s_{i})/(m-i);$ \textit{// new level-threshold}\\
   		$\widehat{J}\leftarrow J(s_{k+h});$
        $\widehat{h}\leftarrow h;$ \textit{// store the horizon that has given the best cost so far}
    }
    
	\caption{RecedeHorizon ($\M,h,\widehat{J}$)}
	\label{alg:rhp}
\end{algorithm}
\setlength{\floatsep}{0.1cm}
%\SetAlgoSkip{}
%%%%%%%%% outer loop of level-based control
\begin{algorithm}[t]
	\SetKwFunction{particleswarm}{particleswarm}
    \SetKwFunction{Fitness}{Fitness}
    \SetKwFunction{RecedeHorizon}{RecedeHorizon}
	\SetKwInOut{Input}{Input}
	\SetKwInOut{Output}{Output}
    \DontPrintSemicolon
	\Input{$\M,\varphi,{h}_{\mathit{max}},m,B, \texttt{Fitness}$}
	\Output{$\{\va^i\}_{1\leqslant i\leqslant\,m}$ \textit{// optimal control sequence}}
	\BlankLine
	Initialize $\ell_0\leftarrow J(s_0)$; $\widehat{J}\leftarrow\inf$; ${p}\leftarrow 2\beta B h$; $i\leftarrow 1$;  ${h}\leftarrow 1$; $\Delta_0\leftarrow (\ell_0 - \varphi)/m$;
	\BlankLine
	\While{($\ell_{i-1} > \varphi)$ $\land$ $(i < m)$}
	{
     	\textit{// find and apply first best action out of the horizon sequence of length $h$}\;
        $[\va^{h},\widehat{J}]\leftarrow$\particleswarm{$\Fitness,\M,p,h$};\\

        \eIf{$\ell_{i-1}-\widehat{J} > \Delta_i \lor h = h_{\mathit{max}}$} 
		{
        \textit{// if a new level or the maximum horizon is reached}\;
        	$\va^i\leftarrow \va^{h}_1$;
            $\M\leftarrow\M^{\va^i}$;
            \textit{// apply the action and move to the next state}\;
			$\ell_i\leftarrow J(s(\M))$; \textit{// update $\ell_i$ with the fitness of the current state}\;
            $\Delta_i\leftarrow \ell_i/(m-i)$; \textit{// update the threshold on reaching the next level}\;
            $i \leftarrow i + 1$;
			${h} \leftarrow 1$;
			${p} \leftarrow 2\beta B h$; \textit{// update parameters}\;
		}
		{
			${h} \leftarrow {h} + 1$;
            $p\leftarrow 2\beta B h$; \textit{// increase the horizon}\;
		}
	}
	\caption{AMPC: Adaptive Model-Predictive Control}
	\label{alg:lerec}
\end{algorithm}
\setlength{\floatsep}{1cm}



\begin{theorem}[AMPC Convergence]
\label{thm:ampc}
Given an MDP $\M\,{=}\,(S,A,T,J)$ with positive and continuous fitness function $J$, and a nonempty set of target states $G\,{\subset}\,S$ with $G\,{=}\,\{s\,|\,J(s)\,{<}\,\varphi\}$. If the transition relation $T$ is controllable with actions in $A$, then there is a finite maximum horizon $h_{\mathit{max}}$ and a finite number of execution steps $m$, such that AMPC is able to find a sequence of actions $a_1,\ldots,a_m$ that brings a state in $S$ to a state in $G$ with probability one.
\end{theorem}

\begin{proof}
In each (macro) step of horizon length $h$, from level $\ell_{i-1}\,{=}\,J(s_k)$ to level $\ell_i\,{=}\,J(s_{k+\widehat{h}})$, AMPC decreases the distance to $\varphi$ by $\Delta_i\,{\geqslant}\,\Delta$, where $\Delta\,{>}\,0$ is fixed by the number of steps $m$ chosen in advance. Hence, AMPC converges to a state in $G$ in a finite number of steps, for a properly chosen $m$. AMPC is able to decrease the fitness in a macro step by $\Delta_i$ by the controllability assumption and the fairness assumption about the PSO algorithm. Since AMPC is a randomized algorithm, the result is probabilistic.
\end{proof}

Note that AMPC is a general procedure that performs adaptive MPC using PSO for dynamical systems that are controllable, come with a fitness metric, and have at least one optimal solution. In an adversarial situation two players have opposing objectives. The question arises what one player assumes about the other when computing its own action, which we discuss next.

    \section{Implementation: Ring Abstraction}
\label{sec:implement}
\subsection{Distributed \mbox{$G_t$} in QMC Solver}
\label{distributedG4}
Before introducing the communication phase of the ring abstraction layer,
it is important to understand how the authors distributed the large device array $G_t$ across MPI ranks.
%
Original $G_t$ was compared, and $G^d_t$ versions were distributed
(Figure~\ref{fig:compare_original_distributed_g4}). 


In the original $G_t$ implementation, the measurements---which were computed by matrix-matrix multiplication---are distributed statically and independently over the MPI ranks to avoid
inter-node communications. Each MPI rank keeps its partial copy of $G_{t,i}$ to accumulate 
measurements within a rank, where $i$ is the rank index. 
After all the measurements are finished, a reduction step is 
taken to accumulate $G_{t,i}$ across all MPI ranks into a final and complete
$G_t$ in the root MPI rank. The size of the $G_{t,i}$ in each rank is 
the same size as the final and complete $G_t$. 

With the distributed $G^d_t$ implementation, this large device array 
$G_t$ was evenly partitioned across all MPI ranks; each portion of it is local to each MPI rank.
%
Instead of keeping its partial copy of $G_t$, 
each rank now keeps an instance of $G^d_{t,i}$ to accumulate measurements 
of a portion or sub-slice of the final and complete $G_t$, where the notation
$d$ in $G^d_t$  refers to the distributed version, and $i$ means the $i$-th rank.
%
The $G^d_{t,i}$ size in each rank is 
reduced to $1/p$ of the size of the final and complete $G_t$, comparing the same configuration 
in original $G_t$ implementation, where $p$ is the number of MPI ranks used. 
%
For example, in Figure~\ref{fig:distributed_g4}, there are four ranks, and rank $i$
now only keeps $G^d_{t,i}$, which is one-fourth the size of the original $G_t$ array size.
% and contains values indexing from range of $[0, ..., N/4)$ in original $G_t$ array where $N$ is the 
% number of entries in  $G_t$  when viewed as a one-dimensional array.

To compute the final and complete $G^d_{t,i}$ for the distributed $G^d_t$ implementation, 
each rank must see every $G_{\sigma,i}$ from all ranks. 
%
In other words, each rank must broadcast the
locally generated $G_{\sigma,i}$ to the remainder of the other ranks at every measurement step. 
%
To efficiently perform this ``all-to-all'' broadcast, a ring abstraction layer was built (Section. \ref{section:ring_algorithm}), which circulates
all $G_{\sigma,i}$ across all ranks.

% over high-speed GPUs interconnect (GPUDirect RDMA) to facilitate the communication phase.

% \begin{figure}
% \centering
% \subfloat[Original $G_t$ implementation.]
%     {\includegraphics[width=\columnwidth]{original_g4.pdf}}\label{fig:original_g4}

% \subfloat[Distributed $G_t$ implementation.]
%     {\includegraphics[width=0.99\columnwidth]{distributed_g4.pdf} \label{fig:distributed_g4}}

% \caption{Comparison of the original $G_t$ vs. the distributed $G^d_t$ implementation. Each rank contains one GPU resource.}
% \label{fig:compare_original_distributed_g4} 
% \end{figure} 

\begin{figure}
\centering
     \begin{subfigure}[b]{\columnwidth}
         \centering
         \includegraphics[width=\textwidth]{images/original_g4.pdf}
         \caption{Original $G_t$ implementation.}
         \label{fig:original_g4}
     \end{subfigure}
     
    \begin{subfigure}[b]{\columnwidth}
         \centering
         \includegraphics[width=\textwidth]{images/distributed_g4.pdf}
         \caption{Distributed $G_t$ implementation.}
         \label{fig:distributed_g4}
     \end{subfigure}
     
\caption{Comparison of the original $G_t$ vs. the distributed $G^d_t$ implementation. Each rank contains one GPU resource.}
\label{fig:compare_original_distributed_g4}
\end{figure}

\subsection{Pipeline Ring Algorithm}
\label{section:ring_algorithm}
A pipeline ring algorithm was implemented that broadcasts the $G_{\sigma}$ 
array circularly during every measurement. 
%
The algorithm (Algorithm \ref{alg:ring_algorithm_code}) is 
visualized in Figure~\ref{fig:ring_algorithm_figure}.

\begin{algorithm}
\SetAlgoLined
    generateGSigma(gSigmaBuf)\; \label{lst:line:generateG2}
    updateG4(gSigmaBuf)\;       \label{lst:line:updateG4}
    %\texttt{\\}
    {$i\leftarrow 0$}\;         \label{lst:line:initStart}
    {$myRank \leftarrow worldRank$}\;          \label{lst:line:initRankId}
    {$ringSize \leftarrow mpiWorldSize$}\;      \label{lst:line:initRingSize}
    {$leftRank \leftarrow (myRank - 1 + ringSize) \: \% \: ringSize $}\;
    {$rightRank \leftarrow (myRank + 1 + ringSize) \: \% \: ringSize $}\;
    sendBuf.swap(gSigmaBuf)\;           \label{lst:line:initEnd}
    \While{$i< ringSize$}{
        MPI\_Irecv(recvBuf, source=leftRank, tag = recvTag, recvRequest)\; \label{lst:line:Irecv}
        MPI\_Isend(sendBuf, source=rightRank, tag = sendTag, sendRequest)\; \label{lst:line:Isend}
        MPI\_Wait(recvRequest)\;        \label{lst:line:recevBuffWait}
        
        updateG4(recvBuf)\;             \label{lst:line:updateG4_loop}
        
        MPI\_Wait(sendRequest)\;        \label{lst:line:sendBuffWait}
        
        sendBuf.swap(recvBuf)\;         \label{lst:line:swapBuff}
        i++\;
    }
\caption{Pipeline ring algorithm}
\label{alg:ring_algorithm_code}
\end{algorithm}

\begin{figure}
	\centering
	\includegraphics[width=\columnwidth, trim=0 5cm 0 0, clip]{images/ring_algorithm.pdf}
	\caption{Workflow of ring algorithm per iteration. }
	\label{fig:ring_algorithm_figure}
\end{figure}

At the start of every new measurement, a single-particle Green's function $G_{\sigma}$
 (Line~\ref{lst:line:generateG2}) is generated 
and then used to update $G^d_{t,i}$ (Line~\ref{lst:line:updateG4})
via the formula in Eq.~(\ref{eq:G4}).
%
% Different from original method that performs 
% full matrix-matrix multiplication (Equation~(\ref{eq:G4})), the current ring algorithm only performs partial
% matrix-matrix multiplication that contributes to $G^d_{t,i}$, a subslice of the final and complete $G_t$.
%
Between Lines \ref{lst:line:initStart} to \ref{lst:line:initEnd}, the algorithm 
initializes the indices
of left and right neighbors and prepares the sending message buffer from the
previously generated $G_{\sigma}$ buffer. 
%
The processes are organized as a ring so that the first and last rank are considered to be neighbors to each other. 
%
A \textit{swap} operation is used to avoid unnecessary memory copies for \textit{sendBuf} preparation.
%
A walker-accumulator thread allocates an additional \textit{recvBuf} buffer of the same size 
as \textit{gSigmaBuf} to hold incoming 
\textit{gSigmaBuf} buffer from \textit{leftRank}. 

The \textit{while} loop is the core part of the pipeline ring algorithm. 
%
For every iteration, each thread in a rank 
receives a $G_{\sigma}$ buffer from its left neighbor rank and sends a $G_{\sigma}$ buffer to its right neighbor rank. 
A synchronization step (Line~\ref{lst:line:recevBuffWait}) is performed
afterward to ensure that each rank receives a new buffer to update the 
local $G^d_{t,i}$ (Line~\ref{lst:line:updateG4_loop}). 
%
Another synchronization step
follows to ensure that all send requests are finalized 
(Line~\ref{lst:line:sendBuffWait}). Lastly, another \textit{swap} operation is used to exchange
content pointers between \textit{sendBuf} and \textit{recvBuf} to avoid unnecessary memory copy and prepare
for the next iteration of communication.
%
In the multi-threaded version (Section~\ref{subsec:multi-thread}), the thread of index, \textit{i}, only communicates with
	threads of index, \textit{i}, in neighbor ranks, and each thread allocates two buffers: \textit{sendBuff} and \textit{recvBuff}.

The \textit{while} loop will be terminated after $\mbox{\textit{ringSize}} - 1$ steps. By that time, 
each locally generated $G_{\sigma,i}$ will have traveled across all MPI ranks and
updated $G^d_{t,i}$ in all ranks. Eventually, each $G_{\sigma,i}$ reaches
to the left neighbor of its birth rank. For example, $G_{\sigma,0}$ generated from rank $0$ will end 
in last rank in the ring communicator.

Additionally, if the $G_t$ is too large to be stored in one node, 
it is optional to accumulate all $G^d_{t,i}$
at the end of all measurements. 
%
Instead, a parallel write into the file system could be taken.

\subsubsection{Sub-Ring Optimization.}

A sub-ring optimization strategy is further proposed to reduce message communication
times if the large device array $G_t$ can fit in fewer devices. 
%
The sub-ring algorithm is visualized in Figure~\ref{fig:subring_algorithm_figure}.

For the ring algorithm (Section~\ref{section:ring_algorithm}), the size of the ring communicator
(\textit{mpiWorldSize}) is set to the same size of the global \mbox{\texttt{MPI\_COMM\_WORLD}}, and thus the size of $G_t$ is equally 
distributed across all MPI ranks.

However, to complete the update to $G^d_{t,i}$ in one measurement, 
one $G_{\sigma,i}$
must travel \textit{mpiWorldSize} ranks. In total, 
there are \textit{mpiWorldSize} numbers of $G_{\sigma,i}$
being sent and received concurrently in one measurement 
in the global
\mbox{\texttt{MPI\_COMM\_WORLD}} 
communicator. If the size of $G^d_{t,i}$ is relatively small per rank, then this will cause high communication overhead.

If $G_t$ can be distributed and fitted in fewer devices, then a shorter travel distance is required 
for $G_{\sigma,i}$, thus reducing the communication overhead. One reduction
step was performed at the end of all measurements to accumulate $G^d_{t,s_i}$, 
where $s_i$ means $i$-th rank on the $s$-th sub-ring.

At the beginning of MPI initialization, the global \mbox{\texttt{MPI\_COMM\_WORLD}} was partitioned  into several new sub-ring communicators by using \mbox{\texttt{MPI\_Comm\_split}}. 
% where each new communicator represents conceptually a subring. 
The new
communicator information was passed to the DCA++ concurrency class by substituting the original global 
\mbox{\texttt{MPI\_COMM\_WORLD}} with this new communicator. Now, only a few minor modifications
are needed to transform the ring algorithm (Algorithm~\ref{alg:ring_algorithm_code})
to sub-ring Algorithm~\ref{alg:sub_ring_algorithm}. In Line~\ref{lst:line:initRankId}, \textit{myRank} is 
initialized to \textit{subRingRank} instead of \textit{worldRank}, where 
\textit{subRingRank} is the rank index in the local sub-ring communicator. 
%
In Line~\ref{lst:line:initRingSize}, \textit{ringSize} is initialized to \textit{subRingSize}
instead of \textit{mpiWorldSize}, where \textit{subRingSize} is the
size of the new communicator.
%
The general ring algorithm is a special case for the sub-ring algorithm because the
\textit{subRingSize} of the general ring algorithm is equal to \textit{mpiWorldSize}, and
there is only one sub-ring group throughout all MPI ranks.


\LinesNumberedHidden
\begin{algorithm}
    {$\mbox{\textit{myRank}} \leftarrow \mbox{\textit{subRingRank}}$}\;         
    {$\mbox{\textit{ringSize}} \leftarrow \mbox{\textit{subRingSize}}$}\;      
\caption{Modified ring algorithm to support sub-ring communication}
\label{alg:sub_ring_algorithm}
\end{algorithm}


\begin{figure}
	\centering
	\includegraphics[width=\columnwidth, trim=0 5cm 0 0, clip]{images/subring_alg.pdf}
	\caption{Workflow of sub-ring algorithm per iteration. Every consecutive $S$ rank forms a sub-ring communicator, 
	and no communication occurs between sub-ring communicators until all measurements are finished. Here, $S$ is the number of ranks in a sub-ring.}
	\label{fig:subring_algorithm_figure}
\end{figure}

\subsubsection{Multi-Threaded Ring Communication.}
\label{subsec:multi-thread}
To take advantage of the multi-threaded QMC model already in DCA++, 
multi-threaded ring communication support was further implemented in the ring algorithm.
%
Figure~\ref{fig:dca_overview} shows that in the original DCA++ method,
each walker-accumulator
thread in a rank is independent of each other, and all the threads in a 
rank synchronize only after all rank-local measurements are finished.
%
Moreover, during every measurement, each walker-accumulator thread
generates its own thread-private $G_{\sigma, i}$ to update $G_t$. 
%

The multi-threaded ring algorithm now allows concurrent message exchange so that threads of same rank-local thread index exchange their thread-private $G_{\sigma, i}$. 
%
Conceptually, there are $k$ parallel and independent rings, where $k$ 
is number of threads per rank, because threads of the same local thread ID
form a closed ring. 
%
For example, a thread of index $0$ in rank $0$ will send its $G_\sigma$ to 
the thread of index $0$ in rank $1$ and receive another $G_\sigma$ from thread index of $0$ 
from last rank in the ring algorithm.
%

The only changes in the ring algorithm are offsetting the tag values 
(\texttt{recvTag} and \texttt{sendTag}) by the thread index value. For example,
Lines~\ref{lst:line:Irecv} and ~\ref{lst:line:Isend} from 
Algorithm~\ref{alg:ring_algorithm_code} are modified to Algorithm~\ref{alg:multi_threaded_ring}.

\LinesNumberedHidden
\begin{algorithm}
        MPI\_Irecv(recvBuf, source=leftRank, tag = recvTag + threadId, recvRequest)\; 
        MPI\_Isend(sendBuf, source=rightRank, tag = sendTag + threadId, sendRequest)\;
\caption{Modified ring algorithm to support multi-threaded ring}
\label{alg:multi_threaded_ring}
\end{algorithm}

To efficiently send and receive $G_\sigma$, each thread
will allocate one additional \textit{recvBuff} to hold incoming 
\textit{gSigmaBuf} buffer from \textit{leftRank} and perform send/receive efficiently.
%
In the original DCA++ method, there are $k$ numbers of buffers of $G_\sigma$ 
size per rank, and in the multi-threaded ring method, there are $2k$
numbers of buffers of $G_\sigma$ size per rank, where $k$ is number of 
threads per rank.

	%!TEX ROOT = ../../centralized_vs_distributed.tex

\section{{\titlecap{the centralized-distributed trade-off}}}\label{sec:numerical-results}

\revision{In the previous sections we formulated the optimal control problem for a given controller architecture
(\ie the number of links) parametrized by $ n $
and showed how to compute minimum-variance objective function and the corresponding constraints.
In this section, we present our main result:
%\red{for a ring topology with multiple options for the parameter $ n $},
we solve the optimal control problem for each $ n $ and compare the best achievable closed-loop performance with different control architectures.\footnote{
\revision{Recall that small (large) values of $ n $ mean sparse (dense) architectures.}}
For delays that increase linearly with $n$,
\ie $ f(n) \propto n $, 
we demonstrate that distributed controllers with} {few communication links outperform controllers with larger number of communication links.}

\textcolor{subsectioncolor}{Figure~\ref{fig:cont-time-single-int-opt-var}} shows the steady-state variances
obtained with single-integrator dynamics~\eqref{eq:cont-time-single-int-variance-minimization}
%where we compare the standard multi-parameter design 
%with a simplified version \tcb{that utilizes spatially-constant feedback gains
and the quadratic approximation~\eqref{eq:quadratic-approximation} for \revision{ring topology}
with $ N = 50 $ nodes. % and $ n\in\{1,\dots,10\} $.
%with $ N = 50 $, $ f(n) = n $ and $ \tau_{\textit{min}} = 0.1 $.
%\autoref{fig:cont-time-single-int-err} shows the relative error, defined as
%\begin{equation}\label{eq:relative-error}
%	e \doteq \dfrac{\optvarx-\optvar}{\optvar}
%\end{equation}
%where $ \optvar $ and $ \optvarx $ denote the the optimal and sub-optimal scalar variances, respectively.
%The performance gap is small
%and becomes negligible for large $ n $.
{The best performance is achieved for a sparse architecture with  $ n = 2 $ 
in which each agent communicates with the two closest pairs of neighboring nodes. 
This should be compared and contrasted to nearest-neighbor and all-to-all 
communication topologies which induce higher closed-loop variances. 
Thus, 
the advantage of introducing additional communication links diminishes 
beyond}
{a certain threshold because of communication delays.}

%For a linear increase in the delay,
\textcolor{subsectioncolor}{Figure~\ref{fig:cont-time-double-int-opt-var}} shows that the use of approximation~\eqref{eq:cont-time-double-int-min-var-simplified} with $ \tilde{\gvel}^* = 70 $
identifies nearest-neighbor information exchange as the {near-optimal} architecture for a double-integrator model
with ring topology. 
This can be explained by noting that the variance of the process noise $ n(t) $
in the reduced model~\eqref{eq:x-dynamics-1st-order-approximation}
is proportional to $ \nicefrac{1}{\gvel} $ and thereby to $ \taun $,
according to~\eqref{eq:substitutions-4-normalization},
making the variance scale with the delay.

%\mjmargin{i feel that we need to comment about different results that we obtained for CT and DT double-intergrator dynamics (monotonic deterioration of performance for the former and oscillations for the latter)}
\revision{\textcolor{subsectioncolor}{Figures~\ref{fig:disc-time-single-int-opt-var}--\ref{fig:disc-time-double-int-opt-var}}
show the results obtained by solving the optimal control problem for discrete-time dynamics.
%which exhibit similar trade-offs.
The oscillations about the minimum in~\autoref{fig:disc-time-double-int-opt-var}
are compatible with the investigated \tradeoff~\eqref{eq:trade-off}:
in general, 
the sum of two monotone functions does not have a unique local minimum.
Details about discrete-time systems are deferred to~\autoref{sec:disc-time}.
Interestingly,
double integrators with continuous- (\autoref{fig:cont-time-double-int-opt-var}) ad discrete-time (\autoref{fig:disc-time-double-int-opt-var}) dynamics
exhibits very different trade-off curves,
whereby performance monotonically deteriorates for the former and oscillates for the latter.
While a clear interpretation is difficult because there is no explicit expression of the variance as a function of $ n $,
one possible explanation might be the first-order approximation used to compute gains in the continuous-time case.
%which reinforce our thesis exposed in~\autoref{sec:contribution}.

%\begin{figure}
%	\centering
%	\includegraphics[width=.6\linewidth]{cont-time-double-int-opt-var-n}
%	\caption{Steady-state scalar variance for continuous-time double integrators with $ \taun = 0.1n $.
%		Here, the \tradeoff is optimized by nearest-neighbor interaction.
%	}
%	\label{fig:cont-time-double-int-opt-var-lin}
%\end{figure}
}

\begin{figure}
	\centering
	\begin{minipage}[l]{.5\linewidth}
		\centering
		\includegraphics[width=\linewidth]{random-graph}
	\end{minipage}%
	\begin{minipage}[r]{.5\linewidth}
		\centering
		\includegraphics[width=\linewidth]{disc-time-single-int-random-graph-opt-var}
	\end{minipage}
	\caption{Network topology and its optimal {closed-loop} variance.}
	\label{fig:general-graph}
\end{figure}

Finally,
\autoref{fig:general-graph} shows the optimization results for a random graph topology with discrete-time single integrator agents. % with a linear increase in the delay, $ \taun = n $.
Here, $ n $ denotes the number of communication hops in the ``original" network, shown in~\autoref{fig:general-graph}:
as $ n $ increases, each agent can first communicate with its nearest neighbors,
then with its neighbors' neighbors, and so on. For a control architecture that utilizes different feedback gains for each communication link
	(\ie we only require $ K = K^\top $) we demonstrate that, in this case, two communication hops provide optimal closed-loop performance. % of the system.}

Additional computational experiments performed with different rates $ f(\cdot) $ show that the optimal number of links increases for slower rates: 
for example, 
the optimal number of links is larger for $ f(n) = \sqrt{n} $ than for $ f(n) = n $. 
\revision{These results are not reported because of space limitations.}   
    \section{Related Work}\label{sec:related}
 
The authors in \cite{humphreys2007noncontact} showed that it is possible to extract the PPG signal from the video using a complementary metal-oxide semiconductor camera by illuminating a region of tissue using through external light-emitting diodes at dual-wavelength (760nm and 880nm).  Further, the authors of  \cite{verkruysse2008remote} demonstrated that the PPG signal can be estimated by just using ambient light as a source of illumination along with a simple digital camera.  Further in \cite{poh2011advancements}, the PPG waveform was estimated from the videos recorded using a low-cost webcam. The red, green, and blue channels of the images were decomposed into independent sources using independent component analysis. One of the independent sources was selected to estimate PPG and further calculate HR, and HRV. All these works showed the possibility of extracting PPG signals from the videos and proved the similarity of this signal with the one obtained using a contact device. Further, the authors in \cite{10.1109/CVPR.2013.440} showed that heart rate can be extracted from features from the head as well by capturing the subtle head movements that happen due to blood flow.

%
The authors of \cite{kumar2015distanceppg} proposed a methodology that overcomes a challenge in extracting PPG for people with darker skin tones. The challenge due to slight movement and low lighting conditions during recording a video was also addressed. They implemented the method where PPG signal is extracted from different regions of the face and signal from each region is combined using their weighted average making weights different for different people depending on their skin color. 
%

There are other attempts where authors of \cite{6523142,6909939, 7410772, 7412627} have introduced different methodologies to make algorithms for estimating pulse rate robust to illumination variation and motion of the subjects. The paper \cite{6523142} introduces a chrominance-based method to reduce the effect of motion in estimating pulse rate. The authors of \cite{6909939} used a technique in which face tracking and normalized least square adaptive filtering is used to counter the effects of variations due to illumination and subject movement. 
The paper \cite{7410772} resolves the issue of subject movement by choosing the rectangular ROI's on the face relative to the facial landmarks and facial landmarks are tracked in the video using pose-free facial landmark fitting tracker discussed in \cite{yu2016face} followed by the removal of noise due to illumination to extract noise-free PPG signal for estimating pulse rate. 

Recently, the use of machine learning in the prediction of health parameters have gained attention. The paper \cite{osman2015supervised} used a supervised learning methodology to predict the pulse rate from the videos taken from any off-the-shelf camera. Their model showed the possibility of using machine learning methods to estimate the pulse rate. However, our method outperforms their results when the root mean squared error of the predicted pulse rate is compared. The authors in \cite{hsu2017deep} proposed a deep learning methodology to predict the pulse rate from the facial videos. The researchers trained a convolutional neural network (CNN) on the images generated using Short-Time Fourier Transform (STFT) applied on the R, G, \& B channels from the facial region of interests.
The authors of \cite{osman2015supervised, hsu2017deep} only predicted pulse rate, and we extended our work in predicting variance in the pulse rate measurements as well.

All the related work discussed above utilizes filtering and digital signal processing to extract PPG signals from the video which is further used to estimate the PR and PRV.  %
The method proposed in \cite{kumar2015distanceppg} is person dependent since the weights will be different for people with different skin tone. In contrast, we propose a deep learning model to predict the PR which is independent of the person who is being trained. Thus, the model would work even if there is no prior training model built for that individual and hence, making our model robust. 

%
    % \vspace{-0.5em}
\section{Conclusion}
% \vspace{-0.5em}
Recent advances in multimodal single-cell technology have enabled the simultaneous profiling of the transcriptome alongside other cellular modalities, leading to an increase in the availability of multimodal single-cell data. In this paper, we present \method{}, a multimodal transformer model for single-cell surface protein abundance from gene expression measurements. We combined the data with prior biological interaction knowledge from the STRING database into a richly connected heterogeneous graph and leveraged the transformer architectures to learn an accurate mapping between gene expression and surface protein abundance. Remarkably, \method{} achieves superior and more stable performance than other baselines on both 2021 and 2022 NeurIPS single-cell datasets.

\noindent\textbf{Future Work.}
% Our work is an extension of the model we implemented in the NeurIPS 2022 competition. 
Our framework of multimodal transformers with the cross-modality heterogeneous graph goes far beyond the specific downstream task of modality prediction, and there are lots of potentials to be further explored. Our graph contains three types of nodes. While the cell embeddings are used for predictions, the remaining protein embeddings and gene embeddings may be further interpreted for other tasks. The similarities between proteins may show data-specific protein-protein relationships, while the attention matrix of the gene transformer may help to identify marker genes of each cell type. Additionally, we may achieve gene interaction prediction using the attention mechanism.
% under adequate regulations. 
% We expect \method{} to be capable of much more than just modality prediction. Note that currently, we fuse information from different transformers with message-passing GNNs. 
To extend more on transformers, a potential next step is implementing cross-attention cross-modalities. Ideally, all three types of nodes, namely genes, proteins, and cells, would be jointly modeled using a large transformer that includes specific regulations for each modality. 

% insight of protein and gene embedding (diff task)

% all in one transformer

% \noindent\textbf{Limitations and future work}
% Despite the noticeable performance improvement by utilizing transformers with the cross-modality heterogeneous graph, there are still bottlenecks in the current settings. To begin with, we noticed that the performance variations of all methods are consistently higher in the ``CITE'' dataset compared to the ``GEX2ADT'' dataset. We hypothesized that the increased variability in ``CITE'' was due to both less number of training samples (43k vs. 66k cells) and a significantly more number of testing samples used (28k vs. 1k cells). One straightforward solution to alleviate the high variation issue is to include more training samples, which is not always possible given the training data availability. Nevertheless, publicly available single-cell datasets have been accumulated over the past decades and are still being collected on an ever-increasing scale. Taking advantage of these large-scale atlases is the key to a more stable and well-performing model, as some of the intra-cell variations could be common across different datasets. For example, reference-based methods are commonly used to identify the cell identity of a single cell, or cell-type compositions of a mixture of cells. (other examples for pretrained, e.g., scbert)


%\noindent\textbf{Future work.}
% Our work is an extension of the model we implemented in the NeurIPS 2022 competition. Now our framework of multimodal transformers with the cross-modality heterogeneous graph goes far beyond the specific downstream task of modality prediction, and there are lots of potentials to be further explored. Our graph contains three types of nodes. while the cell embeddings are used for predictions, the remaining protein embeddings and gene embeddings may be further interpreted for other tasks. The similarities between proteins may show data-specific protein-protein relationships, while the attention matrix of the gene transformer may help to identify marker genes of each cell type. Additionally, we may achieve gene interaction prediction using the attention mechanism under adequate regulations. We expect \method{} to be capable of much more than just modality prediction. Note that currently, we fuse information from different transformers with message-passing GNNs. To extend more on transformers, a potential next step is implementing cross-attention cross-modalities. Ideally, all three types of nodes, namely genes, proteins, and cells, would be jointly modeled using a large transformer that includes specific regulations for each modality. The self-attention within each modality would reconstruct the prior interaction network, while the cross-attention between modalities would be supervised by the data observations. Then, The attention matrix will provide insights into all the internal interactions and cross-relationships. With the linearized transformer, this idea would be both practical and versatile.

% \begin{acks}
% This research is supported by the National Science Foundation (NSF) and Johnson \& Johnson.
% \end{acks}

%SECTIONS

\bibliographystyle{splncs03}
\bibliography{cavBib}

\end{document}
