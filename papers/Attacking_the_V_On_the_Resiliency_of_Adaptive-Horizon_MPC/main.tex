% This is based on the LLNCS.DEM the demonstration file of
% the LaTeX macro package from Springer-Verlag
% for Lecture Notes in Computer Science,
% version 2.4 for LaTeX2e as of 16. April 2010
%
% See http://www.springer.com/computer/lncs/lncs+authors?SGWID=0-40209-0-0-0
% for the full guidelines.
%
\documentclass{llncs}

\usepackage[table,pdftex,dvipsnames]{xcolor}
\usepackage[cmex10]{amsmath}
%\usepackage{todonotes}
\usepackage{url}
%\usepackage{algorithm}
\usepackage[caption=false,font=footnotesize]{subfig}
\usepackage{amsmath}
\usepackage{booktabs}
\usepackage{cite}
\usepackage[utf8]{inputenc}
\usepackage{stfloats}
\usepackage{amssymb}
\usepackage[ruled,linesnumbered,lined,boxed,commentsnumbered]{algorithm2e}
\usepackage{tikz}
\usepackage{multirow}
\usepackage{makecell}

\usetikzlibrary{shapes,arrows,positioning,calc}
\setcounter{tocdepth}{3}
\usepackage{tikz}

\usepackage{float}

\usepackage{xargs}                      % Use more than one optional parameter in a new commands

% flocking notations
\newcommand{\xv}{{\boldsymbol {x}}}
\newcommand{\vv}{{\boldsymbol {v}}}
\newcommand{\va}{{\boldsymbol {a}}}
\newcommand{\vd}{{\boldsymbol {d}}}
\newcommand{\VM}{{\it VM}}
\newcommand{\CV}{{\it CV}}
\newcommand{\UB}{{\it UB}}
% MDP notations
\newcommand{\M}{{\mathcal{M}}}

\usepackage[acronym]{glossaries}

\usepackage[colorinlistoftodos,prependcaption,textsize=tiny]{todonotes}
\newcommandx{\unsure}[2][1=]{\todo[linecolor=red,backgroundcolor=red!25,bordercolor=red,#1]{#2}}
\newcommandx{\change}[2][1=]{\todo[linecolor=blue,backgroundcolor=blue!25,bordercolor=blue,#1]{#2}}
\newcommandx{\info}[2][1=]{\todo[linecolor=OliveGreen,backgroundcolor=OliveGreen!25,bordercolor=OliveGreen,#1]{#2}}
\newcommandx{\improvement}[2][1=]{\todo[linecolor=Plum,backgroundcolor=Plum!25,bordercolor=Plum,#1]{#2}}
\newcommandx{\thiswillnotshow}[2][1=]{\todo[disable,#1]{#2}}
\DeclareMathOperator*{\argmin}{\arg\min}

\pagestyle{plain}   % to add page numbering-remove for final version

\begin{document}

\title{Attacking the V:\\ 
On the Resiliency of Adaptive-Horizon MPC}

%% \title{LEREC: Level-Based Receding-Horizon Control for Attack-Resilient Formations}
%
\titlerunning{Attacking flight formations using global controller}  % abbreviated title (for running head)
%                                     also used for the TOC unless
%                                     \toctitle is used
%
%\author{}
\author{Scott A. Smolka\inst{1} \and Ashish Tiwari\inst{2} \and Lukas Esterle\inst{3} \and Anna Lukina\inst{3} \\ Junxing Yang\inst{1} \and Radu Grosu\inst{1,3}}
%
%% \authorrunning{omitted for review}%Lukina, Esterle, Hirsch, Bartocci, Yang, Tiwari, Smolka, Grosu} % abbreviated author list (for running head)
%
%%%% list of authors for the TOC (use if author list has to be modified)
\tocauthor{}
%
%\institute{}
\institute{ Department of Computer Science, Stony Brook University, USA \\
    \and SRI International, USA \\
    \and Cyber-Physical Systems Group, Technische Universit\"at Wien, Austria
	}

\maketitle        % typeset the title of the contribution

\begin{abstract}
%% The abstract should summarize the contents of the paper
%% using at least 70 and at most 150 words. It will be set in 9-point
%% font size and be inset 1.0 cm from the right and left margins.
%% There will be two blank lines before and after the Abstract. \dots
We introduce the concept of a \emph{V-formation game} between a controller and an attacker, where controller's goal is to maneuver the plant (a simple model of flocking dynamics) into a V-formation, and the goal of the attacker is to prevent the controller from doing so.  
%% The controller can attain its goal by minimizing a certain flock-wide fitness 
%% function $J$, which is (almost) zero exactly when V-formation has been reached.  
%% Conversely, the attacker seeks to maximize $J$. 
%% We formalize V-formation games in terms of a Markov Decision Process (MDP) in which the controller and attacker jointly determine the transition probabilities.
Controllers in V-formation games utilize a new formulation of model-predictive control we call \emph{Adaptive-Horizon MPC} (AMPC), giving them extraordinary power: we prove that under certain controllability assumptions, an AMPC controller is able to attain V-formation with probability~1.
%% find a sequence of control actions (flock-wide accelerations) that brings the MDP to %% a V-formation goal state with probability one.

We define several classes of attackers, including those that in one move can remove
%%  a small number 
$R$ birds from the flock, or introduce random displacement
%% (perturbation)
into flock dynamics. 
%%again by selecting a small number of victim agents.  
We consider both \emph{naive attackers}, whose strategies are purely probabilistic, and \emph{AMPC-enabled attackers}, putting them on par strategically with the controllers.
%% in V-formation games.
While an AMPC-enabled controller is expected to win every game with probability~1, in practice, it is \emph{resource-constrained}: its maximum prediction horizon and the maximum number of game execution steps are fixed.  Under these conditions, an attacker has a much better chance of winning a V-formation game.

Our extensive performance evaluation of V-formation games uses statistical model checking to estimate the probability
%% by which 
an attacker can thwart the controller.  Our results show that for the bird-removal game with $R\,{=}\,1$, the controller almost always wins (restores the flock to a V-formation). For $R\,{=}\,2$, the game outcome critically depends on which two birds are removed.
%% : as long as the two removed birds are not adjacent to one another, the controller 
%% wins (is resilient).  
For the displacement game, our results again demonstrate that an intelligent attacker, i.e.~one that uses AMPC in this case, significantly outperforms its naive counterpart
%% that simply and uniformly at random
that randomly executes its attack.
%% \keywords{computational geometry, graph theory, Hamilton cycles}
\end{abstract}

	Reinforcement learning has achieved great success in areas such as Game-playing \citep{silver2018general,vinyals2019grandmaster}, robotics \cite{kober2013reinforcement}, large language models \citep{ouyang2022training}, etc.
However, due to safety concerns or physical limitations, in some real-world reinforcement learning problems, we must consider additional constraints that may influence the optimal policy and the learning process \citep{garcia2015comprehensive}.
% For example, a robotic arm must not take actions that may cause harm to itself or the environments.
A standard framework to handle such cases is the constrained Markov Decision Process (CMDP) \citep{altman1999constrained}.
Within the CMDP framework, the agent has to maximize
the expected cumulative reward while
obeying a finite number of constraints, which are usually in the form of expected cumulative cost criteria.

However, we are sometimes concerned with the problem with a continuum of constraints.
For example,
the constraints we meet might be time-evolving or subject to uncertain parameters, which
cannot be formulated as an ordinary CMDP
(see Examples \ref{Example_Time_Evolving} and  \ref{Example_Uncertain}).
In this paper we would study a generalized CMDP  
to address the above problem.  Because the constraints are not only infinite-number but also lie
in a continuous set,
the generalization is not trivial. Fortunately, we find that we can borrow the idea behind semi-infinite programming (SIP) \citep{remez1934determination, hettich1993semi} to deal with the semi-infinite constraints.
Accordingly, we propose \emph{semi-infinitely constrained Markov decision processes} (SICMDPs)
as a novel complement to the ordinary CMDP framework.
%More specifically,  an SICMDP model %, we consider 
%contains a continuum of constraints whereas an ordinary CMDP contains a finite number of constraints. 

%This generalization is natural but not trivial. However, we can brows the idea  
%The idea is quite natural and can be backtracked
%to the practice of extending linear programming to linear semi-infinite programming (LSIP) %\cite{remez1934determination, GobernaLSIO1998}.
%In addition, 
%As a complementary approach to the ordinary CMDP framework, 
%SICMDP can be used to model these problems  which cannot be described by a finite number of constraints
%that are not covered by .
%For example,
%the restrictions we consider can be time-evolving or subject to uncertain parameters
%, thus
%cannot be described by a finite number of constraints but a continuum of constraints 
%(see Examples \ref{Example_Time_Evolving} and  \ref{Example_Uncertain}).

We also present two reinforcement learning algorithms to solve SICMDPs called SI-CRL and SI-CPO, respectively.
SI-CRL is a model-based reinforcement learning algorithm designed for tabular cases, and SI-CPO is a policy optimization algorithm for non-tabular cases.
% and analyze its performance both theoretically and empirically.
The main challenge is that we need to deal with a continuum of constraints, thus reinforcement learning algorithms for ordinary CMDPs do not work anymore.
In SI-CRL, we tackle this difficulty by first transforming the reinforcement learning problem to an equivalent LSIP problem, which can then be solved using methods in the LSIP literature like the dual exchange methods \citep{Hu1990,reemtsen1998numerical}.
In SI-CPO, we resort to the idea of cooperative stochastic approximation developed in \cite{lan2020algorithms, wei2020comirror}.
As far as we know, we are the first to introduce tools from semi-infinitely programming (SIP) into the reinforcement learning community for solving constrained reinforcement learning problems.

% To the best of our knowledge, we are the first to apply tools from semi-infinitely programming (SIP) to solve reinforcement learning problems.
Furthermore, we give theoretical analysis for both SI-CRL and SI-CPO.
We decompose the error of SI-CRL into two parts: the statistical error from approximating the true SICMDP with an offline dataset and the optimization error due to the fact that the solution of the LSIP problem obtained by the dual exchange method is inexact.
On the optimization side, we show that the iteration complexity of SI-CRL is $O\left(\left\{\mathrm{diam}(Y)L\sqrt{|\gS|^2|\gA|m}/\left[(1-\gamma)\epsilon\right]\right\}^m\right)$.
On the statistical side, we show that the sample complexity of SI-CRL is $\widetilde O\left(\frac{|S|^2|A|^2}{\epsilon^2(1-\gamma)^3}\right)$ if the offline dataset is generated by a generative model, and $\widetilde O\left(\frac{|S||A|}{\nu_{\min} \epsilon^2(1-\gamma)^3}\right)$ if the dataset is generated by a probability measure $\nu$ as considered in \cite{chen2019information}.
Here $\widetilde O$ means that all logarithm terms are discarded.
For SI-CPO, things become a little more complicated because other than the statistical error and the optimization error, we also need to consider the function approximation error, which comes from imperfect policy parametrizations.
It is shown if the function approximation error can be controlled to $O(\epsilon)$ order, the iteration complexity of SI-CPO is $\widetilde{O}\left(\frac{1}{\epsilon^2(1-\gamma)^6}\right)$ and the sample complexity of SI-CPO is $\widetilde{O}(\frac{1}{\epsilon^4(1-\gamma)^{10}})$.
Here our iteration complexity bound is equivalent to a typical $\widetilde O(1/\sqrt{T})$ global convergence rate.

We perform a set of numerical experiments to illustrate the SICMDP model and validate our proposed algorithms.
Specifically, we examine two numerical examples, namely the discharge of sewage and ship route planning.
Through the discharge of sewage example, we show the advantage of the SICMDP framework over the CMDP baseline obtained by naive discretization in modeling realistic sequential decision-making problems.
Moreover, we demonstrate the effectiveness of the SI-CRL and SI-CPO algorithms in such tabular environments. 
In the ship route planning example, we illustrate the benefits of the SICMDP framework and the ability of the SI-CPO algorithm to address complex continuous control tasks involving continuous state spaces with modern deep reinforcement learning techniques.

% In summary, our contributions are listed as follows.
% First, we present the SICMDP model, which can be viewed as a generalization of the ordinary CMDP model.
% Second, we propose an algorithm to perform reinforcement learning for SICMDPs, which is called SI-CRL, and we believe that we are the first to apply tools from SIP
% to solve reinforcement learning problems.
% Third, we give a theoretical analysis of SI-CRL and identify both its sample complexity and iteration complexity.
% In addition, we perform numerical experiments to illustrate the SICMDP model and validate the SI-CRL algorithm.
% \{This paragraph can be removed!!! \}




         
    
% Panoptic segmentation

% 3D segmentation

% Multi-object tracking

% Online 3D panoptic:

% PanopticFusion: (IROS 2019)
% https://arxiv.org/pdf/1903.01177.pdf
%
% - most similar to ours
% - PSPNet + M-RCNN + 2D fusion
% - volumetric mapping, 
% - greedy matching with IoU -> optimal only with 0.5 threshold
% - voxel & class weighting
% - CRF regularisation
%
% - good:
%
% - bad:
%  - CRF post-processing step
%  - greedy data-association
%    - can't be tuned for lower overlap ratios -> has to have high framerate, large changes in viewpoint could break this
%    - IoU: sensitive to 2D labels projecting over object borders (CRF and voxel weighting seem to alleviate this)

% Voxblox++: (Robotics & automation letters 2019)
% https://arxiv.org/pdf/1903.00268.pdf
% https://github.com/ethz-asl/voxblox-plusplus
%
% - M-RCNN + geometric segmentation + fusion 
% - data association of geometric segments with 3D overlap (no. points inside volume), fixed threshold for min number of points
% - instance label is assigned to a segment based on highest overlap
% - only one detected segment per reference label, as in PanopticFusion and Ours
% - TSDF Integration 
%
% good: 
% - because of geometric segmentation objects with no associated semantic class can also be segmented
% bad:
% - two different object segment types -> confusing, overly complicated ?
% - quite inaccurate (fixed below)

% Reconstructing Interactive 3D Scenes by Panoptic Mapping and CAD Model Alignments (ICRA 2021)
% https://arxiv.org/pdf/2103.16095.pdf
% https://github.com/hmz-15/Interactive-Scene-Reconstruction
%
% - based heavily on Voxblox++, much more accurate
% - Scene-graph ("contact graph") for mapping object relations
% - Search & replace voxels with CAD models, with geometrical and physical constraints
% - Object 6D pose
% - Format for robot interaction
%
% - Segmentation: bilateral fusion of geomatric and semantic segments -> reduce segmentation noise compared to Voxblox++
% - Fusion: triplet count improves consistency over Voxblox++ pairwise count strategy (take semantic label into account in addition to instance and geometry)
% - Fusion: instance labels are also combined if there is enough overlap with common geometric label for long enough time
%   - this means multiple detections can match the same reference unlike ours, voxblox++ and PanopticFusion ?
%

% Panoptic-MOPE: (ROBOTICS AND AUTOMATION LETTERS 2020)
% https://ieeexplore.ieee.org/stamp/stamp.jsp?tp=&arnumber=8977356
% https://github.com/hoangcuongbk80/Object-RPE/tree/panoptic-mope
%
% - novel RGB-D semantic segmentation model + M-RCNN
% - camera tracking based on "addaptively weighted optimization of geometric, appearance, and semantic cues"
% - surfel map: 
%   - how does it scale ? authors satate they tested on room-sized environments, but could be applied in larger scale as well ...
%     - could maybe be applied as VO in a SLAM algorithm ...
%   - demo only on a small pallet + surroundings, might not be applicable in large-scale SLAM

% US VS THEM:
%
% - based heavily on PanopticFusion, with modifications:
%   - instead of greedy data-association (which seems to be the case in others as well), we solve LAP (JPDA?)
%     - overlap threshold can be tuned, which renders the algorithm more flexible
%     - could be extended to dynamic tracking ?
%   - multiple options for association likelihood
%   - outlier rejection (either clustering or probabilistic)
%   - test different options for decreasing processing time
%   - no post-processing
%
% - model-agnostic:
%   - completely separated from segmentation
%   - does not care how point clouds are obtained -> applicable for LIDAR segmentation (e.g. EfficientLPS) as well
%
% - also agnostic to localisation method
%   - could, however, be utilised to find landmark locations / poses

% More compact version of this paragraph to introduction to save space?
%Panoptic segmentation -- proposed in \cite{panoptic_segmentation} -- aims to solve the unified task of semantic- and instance segmentation. Semantic classes are separated to \textit{stuff} -- amorphous, unquantifiable regions like sky, road or floor -- and \textit{things} -- quantifiable objects. The distinction between the two can vary depending on the application, but a semantic class can only belong to one or another. The article also proposes a unified panoptic evaluation metric, coined \textbf{Panoptic Quality} (PQ). Many 2D approaches to panoptic segmentation -- \textit{e.g.} \cite{panopticfpn,seamless,panoptic_deeplab,efficientps} -- have since been proposed. Deep neural networks for performing semantic- or instance segmentation directly on the 3D reconstruction -- \textit{e.g.} on \cite{scannet,s3dis,paris_lille_3d} -- have also been proposed, but since they require the reconstructed 3D scene, they are mostly offline approaches and therefore out of scope for this work. Some recent works also apply panoptic segmentation to point clouds -- \textit{e.g.} methods in the SemanticKITTI panoptic segmentation competition \cite{semantic_kitti} -- mostly aimed at segmenting LiDAR output. They are suitable for online processing, but similar to RGB-D images require a method for tracking object instances persistent in both time and space. In fact, our proposed method, as well as some others mentioned in this work, could use segmented LiDAR point clouds as an input similarly to RGB-D images.

PanopticFusion \cite{panopticfusion} is the first work to propose online integration of panoptic image segmentations to a 3D reconstruction. They integrate point clouds generated from segmented images to a TSDF voxel volume \cite{tsdf,voxblox} by greedily matching detected segments with the reconstruction and regulating each voxel's corresponding instance with a weighting function. Semantic labels are inferred in a bayesian manner based on confidence scores provided by the segmentation model. They also apply a Conditional Random Field (CRF) to regularise the reconstruction, improving results significantly. Voxblox++ \cite{voxblox++} -- introduced later the same year -- is a similar approach that also integrates segmented RGB-D images into a TSDF volume. It leverages geometric segmentation of depth images to improve instance segmentation accuracy. Both geometric and semantic segments are used to compute a pair-wise weight, which is used to greedily match them with segments in the reconstruction. Because of the geometric segmentation, the method allows segmentation of objects with no known semantic class in addition to objects recognised by the instance segmentation model. 

Recently, \cite{interactive_3d_scenes} built upon the idea of Voxblox++. They apply Voxblox++ for 3D instance integration, with two small but effective modifications: the pair-wise weight is replaced by a triplet weight that also takes semantic labels into account in the fusion, and -- in addition to geometric segments -- instance segments are fused if they overlap by a significant amount. The article introduces a method for searching and aligning CAD models to reconstructed objects based on geometry and semantic class, as well as geometrical and physical rules. With the CAD models, a contact graph and interactive virtual scene are reconstructed to allow a robot to simulate its interaction with the environment. SceneGraphFusion \cite{scenegraphfusion} is another approach that forms a scene graph online from a stream of RGB-D images, but unlike the above-mentioned approach, it generates the graph with a deep neural network, after which the panoptic labels for geometrically segmented portions of the 3D reconstruction are produced a side product.

Panoptic-MOPE \cite{panoptic_mope} is another recent approach, which integrates sequences of RGB-D images into a surfel reconstruction. Unlike other mentioned approaches -- which assume the camera pose either known or estimated elsewhere -- it also tracks camera movements based on geometric-, appearance- and semantic cues. The method also applies a novel RGB-D panoptic segmentation model. Although it is only tested on room-sized environments, the authors claim it could be scaled to larger environments as well.  
    %!TEX root = main.tex
\section{Problem Definition and Notations}
\label{sec:problem}







% In this section, we will first describe key concepts and notations used in this paper, and formally define our problem. Then we will use a case study to make our idea of story tree more concrete.

% \subsection{Problem Definition and Notations}
% \label{subsec:problem-define}

We first present some definitions of key concepts in the top-down hierarchy: \textit{topic} $\rightarrow$ \textit{story} $\rightarrow$ \textit{event} to be used in this paper.

\begin{definition}
  \textit{Event}: an event $\mathcal{E}$ is a set of one or several documents that contain highly similar information.
\end{definition}

\begin{definition}
  \textit{Story}: a story $\mathcal{S}$ is a tree of events that revolve around a group of specific persons and happen at certain places during specific times. A directed edge from event $\mathcal{E}_1$ to $\mathcal{E}_2$ indicates a temporal evolution or a logical connection from $\mathcal{E}_1$ to $\mathcal{E}_2$.
\end{definition}

\begin{definition}
  \textit{Topic}: a topic consists of a set of stories that are highly correlated or similar to each other.
  \vspace{-1mm}
\end{definition}


Each topic may contain multiple story trees, and each story tree consists of multiple logically connected events.
In our work, events (instead of news documents) are the smallest atomic units. Each event is also assumed to belong to a single story and contains partial information about that story.
For instance, considering the topic \textit{American presidential election}, \textit{2016 U.S. presidential election} is a story within this topic, and  \textit{Trump and Hilary's first television debate} is an event within this story.


We now introduce some notations and describe our problem formally. Given a news document stream $D = \{ \mathcal{D}_1, \mathcal{D}_2, \ldots, \mathcal{D}_t,\ldots \}$, where $\mathcal{D}_t$ is the set of news documents collected on time period $t$, our objective is to: a) cluster all news documents $D$ into a set of events $E = \{ \mathcal{E}_1, \ldots, \mathcal{E}_{|E|} \}$, and b) connect the extracted events to form a set of stories $S = \{ \mathcal{S}_1, ..., \mathcal{S}_{|S|} \}$. Each story $\mathcal{S} = (E, L)$ contains a set of events $E$ and a set of links $L$, where $L_{i,j} := <\mathcal{E}_i, \mathcal{E}_j>$ denotes a directed link from event $\mathcal{E}_i$ to $\mathcal{E}_j$, which indicates a temporal evolution or logical connection relationship.

%We now illustrate our problem with an example. (A example Fig) Fig... shows ...
Furthermore, we require the events and story trees to be extracted in an online or incremental manner. That is, we extract events from each $\mathcal D_t$ individually when the news corpus $\mathcal D_t$ arrives in time period $t$, and \emph{merge} the discovered events into the existing story trees that were found at time $t-1$. This is a unique strength of our scheme as compared to prior work, since we do not need to repeatedly process older documents and can deliver  a set of evolving yet logically consistent story trees to users.  

% \subsection{Case Study}
% \label{subsec:case-study}

\begin{figure}
\includegraphics[width=3.4in]{figure/StoryStructures}
\caption{Different structures to characterize a story.}
\vspace{-2mm}
\label{fig:storyStructures}
\vspace{-2mm}
\end{figure}

For example, Fig.~\ref{fig:CaseStudy} illustrates the story tree of ``2016 U.S. presidential election''. The story contains $20$ nodes, where each node indicates an event in 2016 U.S. election, and each link indicates a temporal evolution or a logical connection between two events. %For example, event $19$ says America votes to elect new president, and event $20$ says Donald Trump is elected president. 
The index number on each node represents the event sequence over the timeline. There are $6$ paths within this story tree, where the path $1 \rightarrow 20$ indicates the whole presidential election process, branch $3 \rightarrow 6$ is about Hilary's health conditions, branch $7 \rightarrow 13$ talks about television debates, $14 \rightarrow 18$ depicts the investigation into Hilary's ``mail door'', etc. As we can see, by modeling the evolutionary and logical structure of a story into a story tree, users can easily grasp the logic of news stories and learn the main information quickly. 


Let us represent each story by an empty root node $s$ from which the story is originated, and denote each event by an event node $e$. The events in a story can be organized in one of the following four structures shown in Fig. \ref{fig:storyStructures}: a) a flat structure that does not include dependencies between events; b) a timeline structure that organizes events by their timestamps; c) a graph structure that checks the connection between all pairs of events and maintains a subset of most strong connections; d) a tree structure, which represents a story's evolving structure by a tree.  

Compared with a tree structure, sorting events by timestamps omits the logical connection between events, while using directed acyclic graphs to model event dependencies without considering the evolving consistency of the whole story can leads to unnecessary connections between events.
Through extensive user experience studies in Sec.~\ref{sec:eval}, we show that tree structures are the most effective way to represent breaking news stories as compared to other structures, including the more complex graph structures. 

    \section{The Adaptive-Horizon MPC Algorithm} % {Attack Strategies}
\label{sec:ampc}

We now present our new \emph{adaptive-horizon} \emph{model-predictive-control} algorithm, we call AMPC. We will use this algorithm as the controller strategy in the stochastic game on MDPs. We will also consider attack strategies that use AMPC. Since AMPC is an adaptive MPC procedure based on particle-swarm optimization (PSO), we first
briefly present background material on MPC and PSO.
%\todo[inline]{The rest of this section does not belong here. It should be moved in the description of the controller. Similar text can be used then for the advanced attacker.}

\subsection{Background on Model-Predictive Control}
%One way to find an action that can take a flock to a V-formation is based on using model-predictive control (MPC).  
Model-predictive control (MPC) determines the control action at current time $t$ by looking $h$ steps into the future and finding the best $h$-length sequence of control actions that can take the system from its current state $s(t)$ to a new state that has the lowest fitness. (Since we assume existence of a fitness metric $J$ that we are trying to minimize, we specialize the description of MPC to this case.) 
If ${s}_{\va^h}(t+h)$ denotes the state reached from state ${s}(t)$ in time $h$ following the actions $\va^h$ of length $h$, then in the MPC approach, at each time step $t$, the following minimization is performed to find the optimal set of actions
%\begin{align}
%&\textbf{opt-$\va$}^{h}(t)=\{\textbf{opt-$\va$}_i^{h}(t)\}_{i=1}^{b}=\argmin_{\va^h(t)}J(\boldsymbol{c}_{\va^h}(t+h)).
%\label{eq:opt}
%\end{align}
\begin{align}
&\textbf{opt-$\va$}^{h}(t)=\argmin_{\va^h(t)}J(s_{\va^h}(t+h)).
\label{eq:opt}
\end{align}
Since the model is an approximation of the system, only the first action $\va(t) = \textbf{opt-$\va$}^{1}(t)$ is applied as the action at time $t$,
and the remaining future $h-1$ actions found by the optimizer are ignored.  After the control action $\va(t)$ is applied, the system is left to evolve, and the process is repeated at $t\,{+}\,1, t\,{+}\,2,$ and so on.
%
%
%\vspace*{-1mm}
%\begin{align}
%J(\boldsymbol{c}(t),\va^h(t),{h}) = (\CV(\boldsymbol{c}_{\va}^{h}(t))-\CV^*)^2 &+ 
%(\VM(\boldsymbol{c}_{\va}^{h}(t))-\VM^*)^2 \nonumber \\ & +(\UB(\boldsymbol{c}_{\va}^{h}(t))-\UB^*)^2,
%\label{eq:fitness}
%\end{align}
%\noindent{}where ${h}$ is the length of the receding prediction horizon (RPH), $\va^h(t)$ is a sequence of accelerations of length ${h}$, and $\boldsymbol{c}_{\va}^{h}(t)$ is 
%the configuration reached after applying $\va^h(t)$ to $\boldsymbol{c}(t)$.
%
%

The MPC approach can be used for achieving a V-formation, as was outlined in~\cite{yang2016bda,yang2016love}.  These earlier works, however, did not use an adaptive dynamic window, and did not consider the adversarial control problem.
%We perform a single flock-wide minimization of $J$ at each time-step $t$ to obtain an optimal plan of  length $h$ of acceleration actions:\vspace*{-3mm}
%
%\begin{align}
%&\textbf{opt-$\va$}^{h}(t)=\{\textbf{opt-$\va$}_i^{h}(t)\}_{i=1}^{b}=\argmin_{\va^h(t)}J(\boldsymbol{c}(t),\va^h(t),{h}).
%\label{eq:opt}
%\end{align}
%\vspace*{-3mm}\noindent{} 
%
%We apply the first acceleration $\va_i(t) = \textbf{opt-$\va$}_i^{1}(t)$ as the optimal acceleration for bird $i$ %at time $t$. 
%

In MPC, optimization problem (\ref{eq:opt}) is additionally subject to constraints that bound the set of possible actions and states. For example, in our flocking model, the
magnitude of velocity and acceleration for each of the $B$ birds is bounded: $||\vv_i(t)||\,{\leqslant}\,\vv_{max},
||\va^h_i(t)||\,{\leqslant}\,\rho||\vv_i(t)||$ $\forall$ $i\,{\in}\,\{1,\ldots,B\}$,
where $\vv_{max}$ is a predefined constant and $\rho\,{\in}\,(0,1)$. 
% The initial state is selected following the given initial distribution. In the investigation of V-formation conducted in~\cite{lukina2016arxiv}, the initial positions and velocities of each bird are selected at random 
% within certain ranges, and limited such that the distance between any 
% two birds is greater than a (collision) constant $d_{min}$, and small
% enough so that each bird finds itself in the upwash or downwash region of at least one other bird. 

We use a \emph{particle-swarm-optimization algorithm} to solve the optimization problems generated by the MPC procedure.
%\todo[inline]{I added the following assuming we want to refer to the ARES approach - this might to be replaced if we want to use MPC instead. (I am not sure what Jesse used for his experiments right now)}
%\todo[inline]{Anna: as far as I understand Jesse is using MPC level-based with receding horizon, however, there are no PSO clones or important splitting}

% ASHISH: I think we should comment out the following; otherwise, we will look to be proposing something very close to TACAS paper.
%To solve the optimization problem, Lukina et al.~\cite{lukina2016arxiv} present an approach using \emph{Particle Swarm Optimization (PSO)}~\cite{Kennedy95particleswarm} to find potential next actions, combined with the idea of \emph{Importance Splitting}~\cite{kahn1951} in order to increase the chance of reaching a V-formation. Furthermore, they introduce adaptive horizons to allow for temporary worse situation in the flock enabling them to overcome local minima in the fitness function. Additionally, they introduced an adaptive number of particles for the PSO allowing them to better exploit the heuristics of PSO in combination with Importance Splitting as well as achieve a speed-up in the performance of their approach. However, their approach generates a plan offline to traverses the deterministic MDP in order to reach a specific state. In contrast, we present an approach to control the flock of birds online. The resilience of our approach is demonstrated in the presence of adversaries and noise.
 
    \subsection{Background on Particle Swarm Optimization}
\label{sec:swarmOptimization}

Particle Swarm Optimization (PSO) is a randomized approximation algorithm for determining the parameters that minimize a possibly nonlinear and possibly discontinuous cost (or fitness) function. PSO was first introduced by~\cite{Kennedy95particleswarm}. In an interesting twist of events, PSO took its original inspiration from bird flocking.


%As in~\cite{lukina2016arxiv}, our controller-PSO uses ``acceleration birds'' (these are the particles in the swarm). They should not be confused with the actual flocking birds. 
The PSO procedure is best described using the metaphor of a swarm of insects collaboratively trying to find the location of food. The insects, also called particles, live in the space defined by all possible valuations of the unknown parameters (of the optimization problem).  The food is located at the position where the objective function is minimized. PSO works by having a swarm of particles, which have the same goal of finding food (the reward) without knowing its location. Each particle is informed about its distance to the food (value of the objective function). The PSO algorithm repeatedly redistributes each particle towards the one closest to the food, with a speed proportional to the distance separating them, until all particles converge to the same position. 


AMPC employs Matlab's toolbox $\texttt{particleswarm}$, which performs the classical version of PSO. A swarm of $p$ particles is sampled uniformly at random within a given bound on their positions and velocities. In the bird flocking example, if we try to find acceleration vectors by optimization over horizon $h$, then {\em{one}} ``particle'' represents $h$ 2-dimensional vectors for each of the $B$ birds, along with a vector of values that determine how these $h\cdot B$ acceleration vectors will be updated. 
%
%In the games we are considering each particle represents either a flock of bird-accelerations sequence $\{\va_i^{h}\}_{i=1}^b$, or a a flock of displacements sequence $\{\vd_i^{h}\}_{i=1}^b$, where $h$ is the current length of the receding horizon. 
After choosing a neighborhood of random size for each particle $j$, $j\,{\in}\,\{1,\ldots,p\}$, PSO computes the value of the given fitness function for each particle, and stores two vectors for each particle $j$: its so-far personal-best position $\mathbf{x}_{P}^j(t)$, and the position of its fittest neighbor $\mathbf{x}_{G}^j(t)$. The positions and velocities of the particle swarm $j\,{\in}\,\{1,\ldots,p\}$ are updated the following way:

\vspace*{-4mm}
\begin{align}
\mathbf{v}^j(t+1) = \omega\cdot\mathbf{v}^j(t) &+ y_1\cdot \mathbf{u_1}(t+1)\otimes(\mathbf{x}_{P}^j(t)-\mathbf{x}^j(t))  \nonumber \\
&+ y_2\cdot \mathbf{u_2}(t+1)\otimes(\mathbf{x}_{G}^j(t)-\mathbf{x}^j(t)),
\label{eq:swarm}
\end{align}

\vspace*{-1mm}\noindent{}where $\omega$ is an \emph{inertia weight}, which quantifies the trade-off between global and local exploration of the swarm (the value of $\omega$ is proportional to the exploration range); $y_1$ and $y_2$ are the \emph{self adjustment} and the \emph{social adjustment}, respectively; $\mathbf{u_1},\mathbf{u_2}\,{\in}\,{\rm Uniform}(0,1)$ are random variables; and $\otimes$ is the vector dot product, that is, $\forall$ random vector $\mathbf{z}$: $(\mathbf{z}_1,\ldots,\mathbf{z}_b)\otimes(\mathbf{x}_1^j,\ldots,\mathbf{x}_b^j)=(\mathbf{z}_1\mathbf{x}_1^j,\ldots,\mathbf{z}_b\mathbf{x}_b^j)$.


If the value of the fitness computed at each step of the PSO
for $\mathbf{x}^j(t+1)\,{=}\,\mathbf{x}^j(t)\,{+}\,\mathbf{v}^j(t+1)$ falls below the one for $\mathbf{x}_{P}^j(t)$, then $\mathbf{x}^j(t+1)$ is reassigned to $\mathbf{x}_{P}^j(t+1)$. A global best for the next iteration is determined as the particle with the best fitness among $j\,{\in}\,\{1,\ldots,p\}$. The stopping criterion of the PSO algorithm is either reaching the maximum number of iterations set in advance, or reaching the set time bound, or satisfying the minimum criterion. 

PSO can be used to solve any optimization problem. We use it to solve the optimization problem generated in the MPC approach. In a V-formation game, it can be used to obtain the birds' best accelerations, or even the best displacements, at each time step -- depending on whether MPC/PSO is being used by the controller or the attacker.

\emph{Remark}. We assume that PSO is fair, in the sense that it has a chance to sample all the points in the parameter space, and therefore it has the chance to find the optimal solution with probability one, given enough time.

%In a similar spirit, our advanced-attacker-PSO uses so called "displacement birds" (particles).



    	\subsection{The Main Algorithm of AMPC}
\label{sec:lerec}
% Inspired by the ARES algorithm in~\cite{lukina2016arxiv}, we propose a level-based receding-horizon model-predictive control algorithm we call LEREC. 
We propose the main algorithm of AMPC. %short for level-based Adaptive-horizon Model-Predictive Control {Jesse: this is already mentioned in the beginning of the section}. 
This algorithm performs step-by-step control of a given MDP $\M$ by looking $h$ steps ahead and predicting the next best state to move to.
We use PSO to identify the potentially best actions $\va^h$ in the current state achieving the optimal value of the fitness function in the next state.  For bird flocking, the fitness function, \texttt{Fitness}$(\M,\va^h,h)$ of $\va^h$ is defined as the minimum fitness metric $J$ obtained within $h$ steps by applying $\va^h$ on $\M$. Formally, we have
\vspace*{-2mm}
\begin{align}
\texttt{Fitness}(\M,\va^h,h) = \min_{1\leqslant \tau \leqslant h}{J(s_{\va^h}^\tau)}
\end{align}
where $s_{\va^h}^\tau$ is the state after apply the $\tau$th action of $\va^h$ on $\M$. For horizon $h$, PSO searches for the best sequence of 2-dimensional acceleration vector of length $h$, thus having $2Bh$ parameters to be optimized. The number of particles used in PSO is proportional to the number of parameters, i.e., $p = 2\beta B h$.
%was defined in equation~(\ref{eq:fitness}). 

The pseudocode for the AMPC algorithm is given in Algorithm~\ref{alg:lerec}. A novel feature of AMPC is that, unlike classical MPC that uses a fixed horizon $h$, AMPC adaptively chooses an $h$ depending on whether it is able to reach a fitness value that is lower than the current fitness by our chosen quanta $\Delta_i$, $\forall~i\,{\in}\,\{0,\ldots,m\}$.
%we would have had to resort to local optima without any guarantee of reaching a stable state.

%Following~\cite{lukina2016arxiv} we introduce level-based horizon as a way to overcome shortcomings of MPC and increase effectiveness of the optimization process. 
%To overcome the shortcoming of MPC and in order to increase the effectiveness of the optimization process, we introduce 
AMPC is hence an adaptive MPC procedure that uses level-based horizons.
It employs PSO to identify the potentially best next actions.
%for our flock.
If the chosen actions improve (decrease) the fitness of the next state $J(s_{k+h})$, $\forall~k\,{\in}\,\{0,\ldots,m\cdot h_{\mathit{max}}\}$, in comparison to the fitness of the previous state $J(s_k)$ by the predefined $\Delta_i$, the controller considers these actions to be worthy of leading the flock towards or keeping it in the V-formation.%
\footnote{We focus our attention on bird flocking, since the details generalize naturally to other MDPs that come with a fitness metric.}

In this case, the controller applies the actions to each bird and transitions to the next state of the MDP.
The threshold $\Delta_i$ determines the next level $\ell_i\,{=}\,J(s_{k+\widehat{h}})$ of the algorithm, where $\widehat{h} \leqslant h$ is the horizon with the best fitness. The prediction horizon $h$ is increased iteratively if the fitness has not been decreased enough. Upon reaching a new level, the horizon is reset to one (see Algorithm~\ref{alg:lerec}). Having a horizon $\widehat{h}\,{>}\,1$ means it will take multiple transitions in the MDP in order to reach a solution with improved fitness. However, when finding such a solution with $\widehat{h}\,{>}\,1$, we only apply the first action to transition the MDP to the next state. This is explained by the need to allow the other player (environment or an adversary) to apply their action before we obtain the actual next state. 
%adjustment faster and deal with unforeseen situations arising during runtime.
%
If no new level is reached within $h_{\mathit{max}}$ horizons, the first action of the best $\va^h$ using horizon $h_{\mathit{max}}$ is applied. 

The dynamic threshold $\Delta_i$ is defined as in~\cite{lukina2016arxiv}. Its initial value $\Delta_0$ is obtained by dividing the fitness range to be covered into $m$ equal parts, that is, $\Delta_0\,{=}\,(\ell_0\,{-}\,\ell_m)\,{/}\,m$, where $\ell_0\,{=}\,J(s_0)$ and $\ell_m\,{=}\,\varphi$. Subsequently, $\Delta_i$ is determined by the previously reached level $\ell_{i-1}$, as $\Delta_i\,{=}\,\ell_{i-1}{/}(m\,{-}\,i\,{+}\,1)$. This way AMPC advances only if $\ell_i\,{=}\,J(s_{k+\widehat{h}})$ is at least $\Delta_i$ apart from $\ell_{i-1}\,{=}\,J(s_{k})$.

This approach allows us to force PSO to 
escape from a local minimum, even if this implies passing over a bump, by gradually increasing the exploration horizon $h$. We assume that the MDP is controllable and that the set $G$ of good states is not empty, which means, that from any state, it is possible to reach a state whose fitness decreased by at least $\Delta_i$. 
%Figure~\ref{fig:approach} illustrates our approach.
Algorithm~\ref{alg:lerec} illustrates our approach.

%, which terminates if the stable state has been reached or the time elapsed.
%\begin{figure}[t]
%\centering
%	%%%%%%%%%%%%%%%%%%% colors
\definecolor{ShineSky}{rgb}{0, 0.9, 1}
\definecolor{ShineGrass}{rgb}{0.85, 0.88, 0.59}
\definecolor{InfosysDarkGrey}{gray}{0.4}
\definecolor{InfosysLightGrey}{gray}{0.6}
\definecolor{TuWienBlue}{cmyk}{1,0.38,0,0.15}
\definecolor{TuInfRed}{cmyk}{0,1,1,0}

%%%%%%%%%%%%%%%% blocks
\tikzstyle{conf} = [draw, fill=Yellow!20, circle, node distance=0.7cm]
\tikzstyle{inv} = [draw, circle, node distance=0.7cm]
\tikzstyle{arrow} = [thick,->,>=stealth]
\tikzstyle{pso} = [rectangle, rounded corners, minimum width=0.5cm, minimum 
height=0.5cm,text centered, draw=black, fill=ShineSky!20]
\tikzstyle{fit} = [rectangle, rounded corners, minimum width=0.5cm, minimum 
height=0.5cm,text centered, draw=black, fill=ShineSky!20]
\tikzstyle{decision} = [diamond, aspect=1, minimum width=1cm, text 
centered, draw=black, fill=ShineSky!20]
\tikzstyle{output} = [coordinate]

\begin{tikzpicture}[auto, node distance=2cm,>=latex', cross/.style={path 
	picture={ 
		\draw[black]
		(path picture bounding box.south east) -- (path picture bounding 
		box.north west) (path picture bounding box.south west) -- (path picture 
		bounding box.north east);
	}}]
	
	%%%%%%%%%%%%%% configuration
	\node (start) [conf] {};
    \node [above = .2mm of start,text width=.3cm,align = center]{\scriptsize{$s_0$}};
	
	%%%%%%%%%%%%% PSO	
	\node(pso1) [pso, right=3mm of start] {\scriptsize{PSO}};

	\draw [->] (start) -- (pso1);
	
	\node (c1n) [conf, right = 6mm of pso1] {};

	\draw [->] (pso1) -- node {\scriptsize{$\va^{h}$}} (c1n);
	
	%%%%%%%%%%%%%%%%%%%%% Fitness
	
	\node(fit1) [fit, right=4mm of c1n] {\scriptsize{${J}$}};

	\draw [->] (c1n) -- (fit1);
	
	%%%%%%%%%%%%%%%%%% Conditions
	\node (dec1) [decision, right = 0.4cm of fit1] {\scriptsize{$\ell_{i-1} - 
	J > \Delta$}}; 
	\draw[->] (fit1) -- node{\scriptsize{}}(dec1);
    
    \node (fitCheck) [decision, aspect=3, below = 5mm of dec1]{\scriptsize{$J < \widehat{J}$}};
    
	\node (dec2) [decision,aspect=3, below = 5mm of fitCheck] {\scriptsize{${h} < 
	{h}_{max}$}};

% 	\node (applyMaxA) [rectangle, text centered, draw=black, rounded corners, fill=gray!50, below = 5mm of dec2] {\scriptsize{\textbf{Apply $a^h_1$}}};

 \node (setJh) [pso,text width=1cm, right =6mm of dec1]{\scriptsize{$\widehat{h}:=h$}\\\scriptsize{$\widehat{J}:=J$}};
    
    \node (step) [pso,right = 4mm of setJh]{\scriptsize{Apply $a^{\widehat{h}}_1$}};
	
	\node (incH) [pso,left = 10mm of dec2]{\scriptsize{$h\mathrel{++}$}};
	
% 	\node (dec5) [decision,aspect=3, below = 4mm of dec2] {\scriptsize{${p} < 
% 	{p}_{max}$}};
	
% 	\node (incP) [rectangle, text centered, draw=black, text width=1.5cm, 
% 	rounded corners, fill=gray!30] at (incH |- dec5){\scriptsize{$h:=1;$}\\\scriptsize{$p\mathrel{+}=p_{inc};$}};
	
 	\node (replace) [pso,text width=1cm] at (setJh |- fitCheck) {\scriptsize{$\widehat{J} \mathrel{:}= J$}\\\scriptsize{$\widehat{h} := h$}};
%     
	\node (incL) [coordinate,below = 4.5cm of pso1]{};
    
    
%     \node (dec4) [decision,aspect=3, below = 10mm of dec2] 
% 	{\scriptsize{$i < m$}};

%     \node (dec3) [decision,aspect=3] at (stuff |- dec4)
% 	{\scriptsize{$\ell_i > \varphi$}};

    \node (stuff) [pso,text width=1.2cm,align=left,right = 4mm of step] {\scriptsize{$\ell_i:=\widehat{J}$}\\\scriptsize{$i\mathrel{++}$}\\\scriptsize{$h:=1$}\\\scriptsize{$\widehat{J}:=inf$}};
    
    
    \node (setP) [pso, text width=1.5cm] at (incL |- replace){\scriptsize{$p\mathrel{:}=n \cdot h \cdot 2$}};

%     \node (resH) [rectangle, text centered, draw=black, rounded corners, 
% 	fill=gray!30, above = 3mm of incL]{\scriptsize{$h:=1$}};

	
  
	%%%%%%%%%%%%%% Arrows
	\draw[->](dec1.south) -- node(arr3){\scriptsize{No}} (fitCheck.north);
    \draw[->](fitCheck.south) -- node[right]{\scriptsize{No}}(dec2.north);
    \draw[->](dec1.east) -- node(arrApply){\scriptsize{Yes}}(setJh.west);
    \draw[->](setJh.east) -- (step);
	
	%\draw[->](stuff.south) -- (dec3.north);
    
	%\draw[->](dec3.west) -- node(arr1)[above]{\scriptsize{Yes}} (dec4.east);
	%\draw[->] (dec4.west) -- node(arr5)[above]{\scriptsize{Yes}} (incL.east);
% 	\draw[->] (incL.north) -- (resH.south);
%   \draw[->] (resH.north) -- (pso1.south);

	%\draw[->] (incL.north) -- (pso1.south);
    
    \draw[->] (incL.north) -- (setP.south);
    \draw[->] (setP.north) -- (pso1.south);
    
	%\draw[->](dec5.west) -- node(arr6)[above,near start]{\scriptsize{Yes}} 
	%(resH.east);
    \draw[->](dec2.west) -- node(arr6)[above,near start]{\scriptsize{Yes}} 
	(incH.east);
	\draw[->](incH.west) --  (incH -| incL);%(psoBox.south);
    %\draw[->](dec2.south) -- node[right]{\scriptsize{No}}(applyMaxA);
    
%     \draw[->](applyMaxA) -- (dec4);
	
	\draw[->](step.east) -- (stuff);
    \draw[->](fitCheck.east) -- node[above,near start]{\scriptsize{Yes}} (replace.west);
    \draw[->](replace.south) |- (dec2.east);

% 	\node(V) [below = 4mm of dec3,text width=2cm,align = 
% 	center]{\scriptsize{\textbf{Stable state}}};
% 	\draw[->] (dec3.south) --node[right]{\scriptsize{No}} (V);
% 	\node(time) [below = 4mm of dec4,text width=2cm,align = 
% 	center]{\scriptsize{\textbf{Timeout}}};
	
% 	\draw[->] (dec4.south) --node[right]{\scriptsize{No}} (time);

	\draw[-] (stuff.south) |- (incL.east);
	\draw[->] (dec2.south) --node[right]{\scriptsize{No}} ++(0,-.5) -|  (step.south);
	
	\end{tikzpicture}
%\vspace*{-2mm}
%\caption{Graphical representation of AMPC.}
%	\label{fig:approach}
    %\vspace*{-5mm}
%\end{figure}

%% \SetAlgoSkip{}
%%%%% inner loop for finding next best action
\begin{algorithm}[b]
	\SetKwFunction{Fitness}{Fitness}
	\SetKwFunction{particleswarm}{particleswarm}
	\SetKwInOut{Input}{Input}
	\SetKwInOut{Output}{Output}
    
    $[\va^{h},\M^{h}] \leftarrow$ \particleswarm{$\mathcal{M},{p},h$}; \textit{// use 
    PSO in 
    order to determine best next action for the MDP $\mathcal{M}$ with horizon $h$\\}
    ${J}(s_{k+h})\leftarrow$ 
    \Fitness{$\M^h,\va^{h},{h}$};
    \textit{// calculate cost function if applying the first optimal action of horizon sequence of length $h$}\\
    \If{${J}(s_{k+h})\leqslant\widehat{J}$}
    {
   % $\Delta\leftarrow{J}(s_{i})/(m-i);$ \textit{// new level-threshold}\\
   		$\widehat{J}\leftarrow J(s_{k+h});$
        $\widehat{h}\leftarrow h;$ \textit{// store the horizon that has given the best cost so far}
    }
    
	\caption{RecedeHorizon ($\M,h,\widehat{J}$)}
	\label{alg:rhp}
\end{algorithm}
\setlength{\floatsep}{0.1cm}
%\SetAlgoSkip{}
%%%%%%%%% outer loop of level-based control
\begin{algorithm}[t]
	\SetKwFunction{particleswarm}{particleswarm}
    \SetKwFunction{Fitness}{Fitness}
    \SetKwFunction{RecedeHorizon}{RecedeHorizon}
	\SetKwInOut{Input}{Input}
	\SetKwInOut{Output}{Output}
    \DontPrintSemicolon
	\Input{$\M,\varphi,{h}_{\mathit{max}},m,B, \texttt{Fitness}$}
	\Output{$\{\va^i\}_{1\leqslant i\leqslant\,m}$ \textit{// optimal control sequence}}
	\BlankLine
	Initialize $\ell_0\leftarrow J(s_0)$; $\widehat{J}\leftarrow\inf$; ${p}\leftarrow 2\beta B h$; $i\leftarrow 1$;  ${h}\leftarrow 1$; $\Delta_0\leftarrow (\ell_0 - \varphi)/m$;
	\BlankLine
	\While{($\ell_{i-1} > \varphi)$ $\land$ $(i < m)$}
	{
     	\textit{// find and apply first best action out of the horizon sequence of length $h$}\;
        $[\va^{h},\widehat{J}]\leftarrow$\particleswarm{$\Fitness,\M,p,h$};\\

        \eIf{$\ell_{i-1}-\widehat{J} > \Delta_i \lor h = h_{\mathit{max}}$} 
		{
        \textit{// if a new level or the maximum horizon is reached}\;
        	$\va^i\leftarrow \va^{h}_1$;
            $\M\leftarrow\M^{\va^i}$;
            \textit{// apply the action and move to the next state}\;
			$\ell_i\leftarrow J(s(\M))$; \textit{// update $\ell_i$ with the fitness of the current state}\;
            $\Delta_i\leftarrow \ell_i/(m-i)$; \textit{// update the threshold on reaching the next level}\;
            $i \leftarrow i + 1$;
			${h} \leftarrow 1$;
			${p} \leftarrow 2\beta B h$; \textit{// update parameters}\;
		}
		{
			${h} \leftarrow {h} + 1$;
            $p\leftarrow 2\beta B h$; \textit{// increase the horizon}\;
		}
	}
	\caption{AMPC: Adaptive Model-Predictive Control}
	\label{alg:lerec}
\end{algorithm}
\setlength{\floatsep}{1cm}



\begin{theorem}[AMPC Convergence]
\label{thm:ampc}
Given an MDP $\M\,{=}\,(S,A,T,J)$ with positive and continuous fitness function $J$, and a nonempty set of target states $G\,{\subset}\,S$ with $G\,{=}\,\{s\,|\,J(s)\,{<}\,\varphi\}$. If the transition relation $T$ is controllable with actions in $A$, then there is a finite maximum horizon $h_{\mathit{max}}$ and a finite number of execution steps $m$, such that AMPC is able to find a sequence of actions $a_1,\ldots,a_m$ that brings a state in $S$ to a state in $G$ with probability one.
\end{theorem}

\begin{proof}
In each (macro) step of horizon length $h$, from level $\ell_{i-1}\,{=}\,J(s_k)$ to level $\ell_i\,{=}\,J(s_{k+\widehat{h}})$, AMPC decreases the distance to $\varphi$ by $\Delta_i\,{\geqslant}\,\Delta$, where $\Delta\,{>}\,0$ is fixed by the number of steps $m$ chosen in advance. Hence, AMPC converges to a state in $G$ in a finite number of steps, for a properly chosen $m$. AMPC is able to decrease the fitness in a macro step by $\Delta_i$ by the controllability assumption and the fairness assumption about the PSO algorithm. Since AMPC is a randomized algorithm, the result is probabilistic.
\end{proof}

Note that AMPC is a general procedure that performs adaptive MPC using PSO for dynamical systems that are controllable, come with a fitness metric, and have at least one optimal solution. In an adversarial situation two players have opposing objectives. The question arises what one player assumes about the other when computing its own action, which we discuss next.

    \newcommand{\anoise}{{\mathcal{AN}}}
\newcommand{\pnoise}{{\mathcal{PN}}}
\section{Stochastic Games for V-Formation}
\label{sec:sgv}

We describe the specialization of the stochastic-game verification problem to
V-formation.  In particular, we present the AMPC-based control strategy for reaching a V-formation, and the various attacker strategies against which we evaluate the resilience of our controller.

The MDP $\M$ for V-formation was presented in Section~\ref{sec:background}. The state variables of the MDP are the positions and velocities of the birds, and the control variables (defining the actions) are the accelerations and displacements. In the transition relation given in equation~(\ref{eq:v}), the attacker chooses the displacement $\vec{d}(t)$ it needs to manipulate the position of the birds,
whereas the controller chooses the acceleration $\vec{a}(t)$ to apply. Together, the pair $(\vec{a}(t),\vec{d}(t))$ defines the action that transforms one MDP state to another. We now define the controller's and attacker's strategies.

\subsection{Controller's Adaptive Strategies}

Given current state $(\vec{x}(t),\vec{v}(t))$, the controller's strategy $\sigma_C$ returns a probability distribution on the space of all possible accelerations (for all birds).  As mentioned above, this probability distribution is specified implicitly via a randomized algorithm that returns an actual acceleration (again for all birds).  This randomized algorithm is the AMPC algorithm, which inherits its randomization from the randomized PSO procedure it deploys.  

When the controller computes an acceleration, it assumes that the attacker does {\em{not}} introduce any disturbances; i.e., the controller uses the following model:
\vspace*{-4mm}\begin{eqnarray}
 \xv_i(t + 1) &=& \xv_i(t) + \vv_i(t+1) \qquad \forall~i\,{\in}\,\{1,\ldots,B\}, \nonumber \\
 \vv_i(t + 1) &=& \vv_i(t) + \va_i(t), \label{eq:noattack} %\\[-6mm]
\end{eqnarray}
where $\va(t)$ is the only control variable. Note that the controller chooses its next action $\va(t)$ based on the current configuration $(\xv(t),\vv(t))$ of the flock using MPC. The current configuration may have been influenced by the disturbance $\vec{d}(t-1)$ introduced by the attacker in the previous time step.  Hence, the current state need not be the state predicted by the controller when performing MPC in step $t-1$. Moreover, depending on the severity of the attacker action $\vec{d}(t-1)$, the AMPC procedure dynamically adapts its behavior, i.e.\ the choice of horizon $h$, in order to enable the controller to pick the best control action $\vec{a}(t)$ in response.

\subsection{Attacker's Strategies}

We are interested in evaluating the resilience of our V-formation controller when it is threatened by an attacker that can remove a certain number of birds from the flock, or manipulate a certain number of birds by taking control of their actuators (modeled by the displacement term in equation~(\ref{eq:trans})).
We assume that the attack lasts for a limited amount of time, after which the controller attempts to bring the system back into the good set of states. When there is no attack, the system behavior is the one given by equation~(\ref{eq:noattack}).

Note that there can be many different criteria for evaluating the success of an attack,  %(see Remark~\ref{remark:criteria})
but in our experiments, the controller is declared the winner if it can bring the flock to V-formation.
We consider three attack strategies (but see the future work discussion in Section~\ref{sec:conclusion}), each of which defines a V-formation game.

\vspace*{-0.5mm}\paragraph{\bf Remove Birds Game.}
In an RBG, the attacker selects a subset of $R$ birds, where $R\,{\ll}\,B$, and removes them from the flock.  The removal of bird $i$ from the flock at time $t\,{=}\,0$ can be simulated in our framework by allowing the attacker to set the displacement $\vd_i(0)$ for bird $i$ to $\infty$.  We assume that the flock is in a V-formation at time $t\,{=}\,0$.  
Thus, the goal of the controller is to bring the flock back into a V-formation consisting of $B\,{-}\,R$ birds.
%he controller needs to find the best adjustments in velocity $a_i$ for all remaining birds $i \in N - R$ during its turn. %$i \in N \wedge i \notin R$.
%Essentially, this results in a single-move game for the adversary. 
In an RBG, the attacker plays only one move.
When picking birds, the attacker is able to decide which birds will have the greatest negative impact on the flock's fitness when removed from the flock. Apart from seeing if the controller can bring the flock back to a V-formation, we also analyze the time it takes the controller to do so. 
%return to a v-formation for $R \leq \lceil\log(N)\rceil$ and 

% \todo[inline]{SAS: I would only suggest that the size R of the subset of
% birds removed from the flock (of size N) be such that R << N.  O/w I am
% not sure how interesting this game is.  Jesse has simulation results for
% R=1 and N=7.  Also, we should consider this game with and without process
% noise (PN), as Jesse has shown that the resiliency of the flock to remain
% in a V is highly dependent on the magnitude of PN.  It does very well with
% no PN or small PN, but resilience seems to degrade with increasing PN.}
%
%\begin{theorem}
%For any birds picked by the attacker, where $\left\vert{N - R}\right\vert \geq 3$, the planner can find 
%accelerations for each remaining bird in $N$ that will finally lead to a state $s^{*}$ such that cost 
%$J(s^{*})\{\leqslant}\,\varphi$.
%\end{theorem}

\vspace*{-0.5mm}\paragraph{\bf Random Displacement Game.}
In an RDG, the attacker chooses the displacement vector for a fixed number $R$ of birds uniformly from the space $[0,M]\times[0,2\pi]$. This means that the magnitude of the displacement vector is picked from the interval $[0,M]$, and the direction of the displacement vector is picked from the interval $[0,2\pi]$. We vary $M$ in our experiments. The $R$ birds that are picked in different steps are not necessarily the same, as the attacker makes this choice uniformly at random at runtime as well.
%In our second game, each player has control over all birds in the flock. The flock starts in a V-formation. However, both players have different goals and strategies. While the controller wants to keep the flock in a V-formation, the adversarial player tries to disrupt the V. Both players use the same planning approach but the controller tries to minimize the fitness function while the adversary tries to maximize the fitness in each step.
%In our second game, the adversarial player introduces malicious birds into the flock. These birds are controlled by the other player and hence can perturb the flock. To do so, the adversary adds small amounts of noise to this bird to distract the flock and disturb the v-formation. If this alone is not successful, the adversary can use a greater amount of noise to achieve the goal. However, this allows the controller to identify the adversary and henceforth ignore the malicious bird. 
The game starts from an initial V-formation. The attacker is allowed a fixed number of moves, say $20$, after which the displacement vector is identically $0$ for all birds.  The controller, which has been running in parallel with the attacker, is then tasked with moving the flock back to a V-formation, if necessary.
%
\vspace*{-0.5mm}\paragraph{\bf{AMPC Game.}}
An AMPC game is similar to an RDG except that the attacker does not use a uniform distribution to determine the displacement vector. The attacker is advanced and calculates the displacement (that will be the worst for the controller) using the AMPC procedure. See Figure~\ref{fig:ampc}.  In detail, the attacker applies AMPC, but assumes the controller applies zero acceleration. Thus, the attacker uses the following model of the flock dynamics:
\vspace*{-1mm}\begin{eqnarray}
 \xv_i(t + 1) &=& \xv_i(t) + \vv_i(t+1) + \vd_i(t) \qquad \forall~i\,{\in}\,\{1,\ldots,B\}, \nonumber \\
 \vv_i(t + 1) &=& \vv_i(t). \label{eq:attack} %\\[-6mm]
\end{eqnarray}
Note that the attacker is still allowed to have $\vd_i(t)$ be nonzero for a small number $R$ of birds. However, it can choose which $R$ birds it picks in each step.  It uses the AMPC procedure to simultaneously pick the $R$ birds and their displacements.
%Being a fair game, both players have the same capabilities. This means the controller as well as the adversary are able to use receding horizons to try to predict the best moves for their individual birds.

%\begin{theorem}
%
%\end{theorem}

%\paragraph{\bf Game 3.}%: Interior Lines}
% In our third game the adversary has only access to a specific subset of the birds. One could consider the attacker to add a set of malicious birds $M$ to the existing flock $N$.  Additionally we assume the controller is able to detect the attacker and hence the adversarial player needs to wait for the opportune moment to perform the actual attack. This means, the adversarial player can disrupt the V-formation slightly but only has one single move to interrupt and perturb the V-formation permanently. 
% \todo[inline] {Lukas: some important questions: the ATTACKER-ARES only controls the malicious birds and the CTL-ARES only the 'good' birds. however, does the CTL-ARES consider the malicious birds in its planning as 'good' birds? same for the ATTACKER-ARES. To me it would make sense, that the ATTACKER-ARES knows which ones are malicious birds and which ones are 'good' birds, but the CTL-ARES does not. So the CTL-ARES would consider ALL birds ($M \cup N$) but only controls the 'good' ones ($N$) -- i hope this makes any sense.}
%The third game is very similar to the second. However, when performing the final move, the attacker can decide whether it is more beneficial to introduce noise with a great magnitude to the flock or simply remove a specific number of birds from the flock. Again, we consider this a fair game where both players are able to use receding horizons do identify potential moves. Furthermore, we allow the adversary to remove up to $\log(N)$ birds from the flock.
%\subsection{Implementation: the Game is on}
%\label{sec:implementation}
%
%\todo[inline]{The following section would be the new implementation of our algorithm that deals with stochastic MDP and two-player games.}
%
% For this work, we extended the original \emph{deterministic Markov Decision Process} presented by Lukina et al.~\cite{lukina2016arxiv} to a \emph{classical MDP}~\cite{russellnorvig} by adding noise to the transition relation of the MDP. By doing so, we improved the original model and made it more realistic.
%
%We added and analyzed two different types of noises, processing noise ($\pnoise$) and actuator noise ($\anoise$). $\pnoise$ is applied to the position of each bird in our flock and changes the transition relation as follows
%\vspace*{-1mm}\begin{eqnarray*}
%\label{eq:pnoise_model}
% \xv_i(t + 1) &=& \xv_i(t) + \vv_i(t+1) + \pnoise %\label{eq:x_anoise},\\
% \vv_i(t + 1) &=& \vv_i(t) + \va_i(t) \label{eq:v_anoise},\\[-6mm]
%\end{eqnarray*}
%where $\pnoise \sim \mathcal{N}(0, \sigma^2)$. Here, $\sigma$ 

%In contrast, actuator noise is added to the acceleration action of the transition relation.
%\vspace*{-1mm}\begin{eqnarray*}
%\label{eq:model}
 %\xv_i(t + 1) &=& \xv_i(t) + \vv_i(t+1)\label{eq:x_anoise},\\
 %\vv_i(t + 1) &=& \vv_i(t) + \va_i(t) + \anoise\label{eq:v_anoise},\\[-6mm]
%\end{eqnarray*}

%\noindent where $\anoise \sim \mathcal{N}(0, \sigma^2)$. For our experiments we tried different $\sigma$, i.e. $\sigma = 0.05, 0.1, 0.2, 0.25$ and $0.3$.

%\begin{remark}\label{remark:criteria}
%Even though we use reaching V-formation as our success criterion (for the controller), we could have also used other criteria to decide if the attacker has been successful. For example, one could have used following criteria.
%
%\begin{itemize}
%\item \emph{Energy attack} is considered successful when a flock is not traveling in a V-formation for a certain amount of time. 
%
%\vspace*{1mm}\item \emph{Collisions} occur when two birds are in dangerous proximity from each other. This may happen through spoofing of existing birds or adversarial birds deliberately trying to lead to collisions with the others.
%
%\vspace*{1mm}\item \emph{Heading change} brings success, when the entire flock is diverged from its original direction (mission target) by a certain degree. 
%\end{itemize}
%\end{remark}

\begin{theorem}[AMPC resilience in a C-A game]
\label{thm:resilience}
Given a controller-attacker game, there is a finite maximum horizon $h_{\mathit{max}}$ and a finite maximum number of game-execution steps $m$ such that AMPC controller will win the controller-attacker game in $m$ steps with probability one.
\end{theorem}

\begin{proof}
Since the flock MDP (defined by Equation~6) is controllable, the PSO algorithm we use is fair, and the attack has a bounded duration, the proof of the theorem follows from Theorem~\ref{thm:ampc}. 
\end{proof}

\begin{remark}
While Theorem~\ref{thm:resilience} states that the controller is expected to win with probability one, we expect winning probability to be possibly lower than one in many cases because: (1)~the maximum horizon $h_{\mathit{max}}$ is fixed in advance, and so is (2) the maximum number of execution steps $m$; (3) the underlying PSO algorithm is also run with bounded number of particles and time.
\end{remark}

	\begin{table}[t!]
\centering
\caption{Voice conversion \& F0 manipulation results. MOS results are reported with 95\% confidence interval. VDE, and FFE are reported for F0 manipulation while PER, WER, EER, and MOS are reported for voice conversion. Notice, for VDE, and FFE higher is the better since F0 was flattened.}
\label{tab:conv}

\resizebox{1\columnwidth}{!}{
\begin{tabular}{c@{~} | c@{~} | c@{~}c@{~} | c@{~} | c@{~} ||  c@{~}c@{~} }
\toprule
\multirow{2}{*}{Dataset} & \multirow{2}{*}{Method} & \multicolumn{4}{c||}{Voice Conversion} & \multicolumn{2}{c}{F0 Manipulation} \\
\cmidrule{3-8}
& & PER~$\downarrow$ & WER~$\downarrow$ & EER~$\downarrow$ & MOS~$\uparrow$ & VDE~$\uparrow$ & FFE~$\uparrow$ \\
\midrule
VCTK & GT  & 17.16 & 4.32 & 3.25 & 4.11$\pm$0.29 & -- & -- \\
\midrule 
\multirow{3}{*}{LJ}
% & ASR-TTS   & 50.74  & --     & 66.08 & 32.96 & 1.46 \\
& CPC       & 22.22 	& 16.11 		& 0.46 		& 3.57$\pm$0.15 		& \bf 46.68 & \bf 48.71\\
& HuBERT    & \bf 19.09 & \bf 12.23 & \bf 0.31  & \bf 3.71$\pm$0.24 & 39.20 		& 48.42\\
& VQ-VAE    & 40.88 	& 36.96 		& 9.65 		& 2.90$\pm$0.17 		& 10.54 	& 12.08 \\
\midrule 
\multirow{3}{*}{VCTK} 
% & ASR-TTS   & 68.88  & --    & 41.77 & 13.55 & 6.48 \\
& CPC       &  23.58 		& 15.98 		& \bf 4.83  &  3.42 $\pm$ 0.24 		& \bf 25.29 & \bf 26.97 \\
& HuBERT    &  \bf 20.85 	& \bf 12.72 & 6.01  		& \bf  3.58 $\pm$ 0.28 	& 23.46 	& 26.67 \\
& VQ-VAE    & 36.88  		& 29.44 		& 11.56 		& 3.08 $\pm$ 0.34 		& 7.03  	& 7.80  \\
\bottomrule
\end{tabular}}
\vspace{-0.4cm}
\end{table}

\vspace{-0.1cm}
\section{Results}
\vspace{-0.1cm}
Our results cover
% We report results for 
three different settings: (i) speech reconstruction experiments; (ii) speaker conversion and F0 manipulation; (iii) bitrate analysis with subjective tests for speech codec evaluation. We employ two datasets: LJ~\cite{ljspeech17} single speaker dataset and VCTK~\cite{vctk} multi-speaker dataset. All datasets were resampled to a 16kHz sample rate.

% \paragraph*{Implementation Details.}
% \smallskip
\noindent{\bf Implementation Details\quad} 
\label{sec:impl}
We follow the same setup as in~\cite{lakhotia2021generative}. For CPC, we used the model from~\cite{Riviere2020}, which was trained on a ``clean'' 6k hour sub-sample of the LibriLight dataset~\cite{Kahn2020,Riviere2020}. We extract a downsampled representation from an intermediate layer with a 256-dimensional embedding and a hop size of 160 audio samples. For HuBERT we used a \textsc{Base} 12 transformer-layer model trained for two iterations~\cite{hsu2020hubert} on 960 hours of LibriSpeech corpus~\cite{Panayotov2015}. 
% This model encodes every 320 raw audio samples into a 768-dimensional vector. 
This model downsamples the raw audio $\times320$ into a sequence of 768-dimensional vectors. Similarly to~\cite{lakhotia2021generative}, activations were extracted from the sixth layer.

%CPC: We use a dictionary of 100 units, leading to a bitrate of 700bps.
%HuBERT: A dictionary of 100 units is used, leading to a bitrate of 350bps. 
%VQVE: The VQ-VAE discrete code operates at a bitrate of 800bps.
% For both CPC and HuBERT, the k-means algorithm is applied to convert continuous frames to discrete codes, using the LibriSpeech clean-100h~\cite{Panayotov2015} dataset. 
For CPC and HuBERT, the k-means algorithm is trained on LibriSpeech clean-100h~\cite{Panayotov2015} dataset to convert continuous frames to discrete codes. We quantize both learned representations with $K=100$ centroids. Leading to a bitrate of 700bps for CPC and 350bps for HuBERT.

% VQ-VAE
Similarly to CPC models, we trained the VQ-VAE content encoder model on the ``clean'' 6K hours subset from the LibriLight dataset. We use an encoder operating on the raw signal to extract discrete units, similar to~\cite{jukebox}. In addition, ``random restarts'' were performed when the mean usage of a codebook vector fell below a predetermined threshold. Finally, we used HiFiGAN (architecture and objective) as the decoder instead of a simple convolutional decoder, as it improved the overall audio quality. This model encodes the raw audio into a sequence of discrete tokens from 256 possible tokens~\cite{garbacea2019low} with a hop size of 160 raw audio samples. The VQ-VAE discrete code operates at a bitrate of 800bps. We additionally experimented with 100 discrete units for VQ-VAE, however results were the best for 256. This finding is consistent with~\cite{garbacea2019low}.

% verification model
The speaker verification network uses the architecture proposed in~\cite{heigold2016end}. It was trained on the VoxCeleb2~\cite{voxceleb2} dataset, achieving a 7.4\% Equal Error Rate (EER) for speaker verification on the test split of the VoxCeleb1~\cite{Nagrani17} dataset.

% pitch
Only a single F0 representation is considered across all evaluated models, trained on the VCTK dataset.
% The F0 is extracted from the raw audio using YAAPT~\cite{yaapt} algorithm, using a window size of 20ms and a 5ms hop. 
The F0 is extracted from the raw audio using a window size of 20ms and a 5ms hop. 
As a result, the F0 sequence is sampled at 200Hz. 
% We apply the quantization described at Sec.~\ref{sec:method}, using a pitch codebook of $K'=20$ tokens and an encoder that downsamples the pitch by $\times16$. 
The quantization described at Sec.~\ref{sec:method}, is applied using an F0 codebook of $K'=20$ tokens and an encoder that downsamples the signal by $\times16$. Hence, the discrete F0 representation is sampled at 12.5Hz, leading to a bitrate of 65bps. The final bitrate of the evaluated codecs is the sum of the pitch code bitrate with the content code bitrate.

% \paragraph*{Evaluation Metrics}
% \smallskip
\noindent{\bf Evaluation Metrics\quad} 
We consider both subjective and objective evaluation metrics. For subjective tests, we report the Mean Opinion Scores (MOS). In which human evaluators rate the naturalness of audio samples on a scale of 1--5. Each experiment, included 50 randomly selected samples rated by 30 raters. For objective evaluation, we consider: (i) Equal Error Rate~(EER) as an automatic speaker verification metric obtained using a pre-trained speaker verification network. We report EER between test utterances and enrolled speakers; (ii) Voicing Decision Error (VDE)~\cite{nakatani2008method}, which measures the portion of frames with voicing decision error; (iii) F0 Frame Error (FFE)~\cite{chu2009reducing}, measures the percentage of frames that contain a deviation of more than 20\% in pitch value or have a voicing decision error; (iv) Word Error Rate (WER) and Phoneme Error Rate (PER), proxy metrics to the intelligibility of the generated audio. We used a pre-trained ASR network~\cite{baevski2020wav2vec} on both reconstructed and converted samples to calculate both metrics. %To generate target phonemes, the g2p-en~\cite{g2pE2019} Grapheme2Phoneme module was used.

% \vspace{-0.1cm}
% \smallskip
\noindent{\bf Reconstruction \& Conversion}
% \vspace{-0.1cm}
We start by reporting the reconstruction performance. Results are summarized in Table~\ref{tab:recon}. When considering the intelligibility of the reconstructed signal HuBERT reaches the lowest PER and WER scores across all models, where both CPC and HuBERT are superior to VQ-VAE. However, when considering F0 reconstruction VQ-VAE outperforms both HuBERT and CPC by a significant margin. This results are somewhat intuitive, bearing in mind VQ-VAE objective is to fully reconstruct the input signal. In terms of subjective evaluation, all models reach similar MOS scores, with one exception of CPC on LJ. 

%Notice, since the same F0 units are used for each method, this result implies the VQ-VAE units contain some information about the F0 of the signal, enabling better reconstruction. Regarding speaker information, the CPC gets the lowest EER. 

To better evaluate the disentanglement properties of each method with respect to speaker identity and F0, we conducted an additional set of experiments aiming at speaker conversion and F0 manipulation. For voice conversion, we converted each test utterance into five random target speakers. Next, we employed a speaker verification network, which extracts \emph{d-vector} representation to evaluate speaker-converted utterances' similarity to real speaker utterances (low error-rate indicates good conversion), providing measurement to the speaker identity's disentanglement from the evaluated coding method. The error-rate is reported between converted test utterances and enrolled speakers. For the LJ speech single speaker dataset, we converted samples from the VCTK dataset to the single speaker and enrolled all VCTK speakers together with the single speaker. Results are summarized in Table~\ref{tab:conv} (left). Unlike resynthesis results, on voice conversion CPC and HuBERT outperform VQ-VAE on both LJ and VCTK datasets, indicating VQ-VAE contains more information about the speaker in the encoded units, hence producing more artifacts. Notice, this also affects WER, PER, and the overall subjective quality (MOS). 

Next, to evaluate the presence of F0 in the discrete units, we flattened the F0 units before synthesizing the signal and calculated VDE and FFE with respect to the original F0 values. F0 flattening was done by setting the speakers' mean F0 value across all voiced frames. In this experiment, we expected units that contain F0 information to be better at F0 reconstruction over disentangled units. Results are summarized in Table~\ref{tab:conv} (right). Notice VQ-VAE can still reconstruct the F0 almost at the same level as when using the original F0 as conditioning (5.2 vs 7.03, and 5.59 vs 7.8), in contrast to CPC and HuBERT.

\begin{figure}[t!]
\centering
\includegraphics[width=0.65\columnwidth, trim={50 20 70 20}]{figures/codec_2.pdf}
% \caption{MUSHRA subjective listening test results as a function of bitrate per second for various methods. Purple dots denote the baseline methods, and green dots the proposed SSL based method.} 
\caption{MUSHRA subjective quality results as a function of bitrate per second. Purple dots denote the baseline methods, and green dots the proposed SSL based method.} 
\label{fig:codec}
\vspace{-0.5cm}
\end{figure}

% \vspace{-0.1cm}
% \smallskip
\noindent{\bf Speech Codec}
Our final experiment evaluates the obtained speech units as a low bitrate speech codec. 
% Therefore, we evaluate how the performance varies as a function of the number of discrete units. Changing the number of units is equivalent to varying the bitrate of the encoded signal. 
We use a subjective MUSHRA-type listening test~\cite{series2014method} to measure the perceived quality of the proposed speech codec with regard to its bitrate constraints. In MUSHRA evaluations, listeners are presented with a labeled uncompressed signal for reference, a set of test samples to rate, a copy of the uncompressed reference, and a low-quality anchor. Listeners are asked to rate each test utterance and the copy of the uncompressed reference with respect to the labeled reference in a scale of 1-100.

The experiment is performed on the VCTK dataset~\cite{vctk}. For evaluation, we used 20 utterances from 5 speakers. The set of speakers in the test data is disjoint with those in the training data. For this experiment, HuBERT models with 50, 100, and 200 units were trained as described in Sec.~\ref{sec:impl}. For comparison, we included other speech codecs in our evaluation: Opus~\cite{valin2012definition} wideband at 9 kbps VBR, Codec2~\cite{rowe2011codec} at 2.4 kbps and LPCNet~\cite{valin2019real} operating at 1.6 kbps. The LPCNet model was trained from scratch on the VCTK dataset following the experimental setup in~\cite{valin2019real}. The VQ-VAE model employs the HiFiGAN decoder trained on the LibriLight dataset to match the amount of data reported in~\cite{garbacea2019low}. We compressed the anchor sample with Speex~\cite{valin2016speex} at 4 kbps as a low anchor. Fig.~\ref{fig:codec} depicts the results. HuBERT with 50 units reaches the best MUSHRA score while its bitrate is only 365bps, which is significantly lower than the baseline methods.   
    \textbf{Related work}:
% Object detection related datasets/algo in non-medical domain
% Locally labeled CXR dataset
A few CXR datasets have localized abnormality annotations \cite{shih2019augmenting,filice2020crowdsourcing,jaeger2014two} that are curated manually. These are high quality gold standard ground truth datasets but tend to be smaller in scale (< 30,000 images) and have a narrow coverage, with typically only 1-2 labels. In addition, since most labeling efforts only have abnormality semantics attached, no direct relationships with the affected anatomical locations are available. 

%MEHDI: repeated concepts from above. I am removing the following: 

%The lack of anatomic semantics in the annotation is a limitation for complex multi-modal clinical reasoning work, e.g., differential diagnosis, since clinicians often integrate information along anatomical lines, and for downstream report generation tasks, which often requires describing not only the abnormality but also correctly communicate the location of the abnormalities (and medical devices) to the receiving clinicians. 

Two recent CXR datasets have labels for anatomies described in the reports. In \cite{datta2020dataset}, a small manually annotated dataset (2000 reports) included 10 abnormalities that are individually associated with 29 unique spatial locations (anatomies) at the report level. Another CXR dataset has automatically extracted abnormality and anatomy labels as disconnected concepts that are only correlated at the study level from  160,000 reports using a supervised NLP algorithm \cite{bustos2020padchest}. This was trained on a smaller set of manually annotated data. Neither datasets contain localized annotations for the associated CXR images, nor any comparison relation annotations between sequential exams, both of which are available in the Chest ImaGenome dataset. In Table \ref{tab:related}, we present a comparison of our Chest ImagGenome dataset with other datasets available in the literature.

% Table -- Kashyap

% MEdical imaging datasets to go here: Discussed that we will only focus on cxr datasets that are available for this paper. 
% \caption{\color{red} Kashyap, feel free to continue with the table. We should remove the questionmarks and add a line for our dataset (since all others are not graph). For longer text, using abbreviations and explaining them in the caption often works better. If fill in the values is not possible, it is better to remove the table altogether.}


\begin{table}[t!]
\caption{Summary of existing chest X-ray datasets}
\resizebox{\textwidth}{!}{%
\begin{tabular}{@{}lllllllll@{}}
\toprule
\textbf{Dataset} & \textbf{Annotation Level} & \textbf{Annotation Method} & \textbf{Num Labels} & \textbf{Anatomy Labeled} & \textbf{Graph} & \textbf{Dataset Size} & \textbf{Temporal Labels} & \textbf{Reports} \\ \midrule
SIIM-ACR Pneumothorax Segmentation \cite{filice2020crowdsourcing} & Segmentation & Manual + augmented & 1 & No & No & 12,047 & No & No \\
RSNA Pneumonia Detection Challenge   \cite{shih2019augmenting} & Bounding Boxes & Manual & 1 & No & No & 30,000 & No & No \\
Indiana University Chest X-ray collection \cite{demner2016preparing} & Global & Automated & 10 & No & No & 3,813 & No & Yes \\
NIH CXR dataset \cite{wang2017chestx} & Global & Automated & 14 & No & No & 112,120 & No & No \\
PLCO \cite{team2000prostate} & Global & Automated & 24 & Yes & No & 236,000 & Yes & No \\
Stanford CheXpert \cite{irvin2019chexpert} & Global & Automated & 14 & No & No & 224,316 & No & No \\
MIMIC-CXR \cite{johnson2019mimic} & Global & Automated & 14 & No & No & 377,110 & No & Yes \\
Dutta \cite{datta2020dataset} & Global & Manual & 10 & Yes & Yes & 2,000 & No & Yes \\
PadChest \cite{bustos2020padchest} & Global & Manual + automated & 297 & Yes & No & 160,868 & No & Yes \\
Montgomery County Chest X-ray   \cite{jaeger2014two} & Segmentation & Manual & 1 & Yes & No & 138 & No & No \\
Shenzen Hospital Chest X-ray   \cite{jaeger2014two} & Segmentation & Manual & 1 & Yes & No & 662 & No & No \\  \hline \hline
\textbf{Chest ImaGenome} & Bounding Boxes & Automated & 131 & Yes & Yes & 242,072 & Yes & Yes \\
\bottomrule
\end{tabular}%
}
\label{tab:related}
\vspace{-0.4cm}
\end{table}
% removed (Derived from MIMIC-CXR \cite{johnson2019mimic}) % makes table really small

    
\begin{comment}
\begin{figure}
\includegraphics[width=\linewidth]{figs/beyond_tss_lesion.pdf}
\caption[]{End-to-End runtime lesion study of the entire MNIST dataset and the FMA featurized music dataset. Each of DROP's contributions provides a runtime improvement.}
\label{fig:beyond_lesion}
\end{figure}
\end{comment}



\section{Conclusion}
\label{sec:conclusion}

Advanced data analytics techniques must scale to rising data volumes. 
DR techniques offer a powerful toolkit when processing these datasets, with PCA frequently outperforming popular techniques in exchange for high computational cost. 
In response, we propose DROP, a new dimensionality reduction optimizer. 
DROP combines progressive sampling, progress estimation, and online aggregation to identify high quality low dimensional bases via PCA without processing the entire dataset by balancing the runtime of downstream tasks and achieved dimensionality. 
Thus, DROP provides a first step in bridging the gap between quality and efficiency in end-to-end DR for downstream \red{analytics}. 

%We revisit canonical operators for time series dimensionality reduction and the measurement study of~\cite{keogh-study}, and show that PCA is more effective than popular alternatives in the data mining literature often by a margin of over $2\times$ on average on gold-standard time series benchmark data sets with respect to output data dimension. More surprisingly, we empirically demonstrate that a small number of samples are sufficient to accurately characterize directions of maximum variance and obtain a high-quality low-dimensional transformation.




%SECTIONS

\bibliographystyle{splncs03}
\bibliography{cavBib}

\end{document}
