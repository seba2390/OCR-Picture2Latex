\vspace*{-0.6ex}
\section{Conclusions}
\label{sec:conclusion}

We have introduced AMPC, a new model-predictive controller that unlike MPC, comes with provable convergence guarantees. The key innovation of AMPC is that it dynamically adapts its receding horizon (RH) to get out of local minima. In each prediction step, AMPC calls
%% particle swarm optimizer
PSO with an optimal RH and corresponding number of particles. We used AMPC as a
bird-flocking controller whose goal is to achieve V-formation despite various forms of attacks, including bird-removal, bird-position-perturbation, and advanced AMPC-based attacks.  We quantified the resiliency of AMPC to such attacks using statistical model checking.  Our results show that AMPC is able to adapt to the severity of an attack by dynamically changing its horizon size and the number of particles used by PSO to completely recover from the attack, given a sufficiently long horizon and execution time (ET).  The intelligence of an attacker, however, makes a difference in the outcome of a game if RH and ET are bounded before the game begins.

Future work includes the consideration of additional forms of attacks, including: \emph{Energy attack}, when the flock is not traveling in a V-formation for a certain amount of time; \emph{Collisions}, when two birds are dangerously close to each other due to sensor spoofing or adversarial birds; and \emph{Heading change}, when the flock is diverted from its original destination (mission target) by a certain degree.

%2. adaptation helps to guarantee convergence, but here we observe that it can be used to obtain resilience to attacks.Effectively, if an optimization procedure can come out of local minima, it can be used o develop controllers that are resilient to attacks.
%3. stochastic model checking used to evaluate resilience of controllers to physical attacks on CPSs.
%4. attacker can employ sophisticated (adaptive) strategies (based on AMPC) to defeat controllers.
%5. Future work includes other atttacker strategies?  If so, which ones?  What else?
%Looking in the future, one of the most important for current CPS industry issue to address is attack detection and evaluation of the appropriate action plan, which we foresee to be possible to perform using our AMPC algorithm. This might require to consider distributed control of the birds. Moreover, in this work, due to the defined problem setting, we analyzed the resilience of our algorithm under the removal of one or two birds. In more general case with $B\,{>}\,7$ birds, we can use statistical approach to approximate the costs of removing $R\,{>}\,2$ birds form the V-formation.