\section{Related Work}
\label{sec:related_work}

In this section, we briefly review works related to our study. In general, the related work can be mainly grouped into the following categories.

The first category related to this paper is traditional recommendation techniques. Recommender systems assist users by supplying a list of items that might interest users. Efforts have been made on offering meaningful recommendations to users. Collaborative filtering\cite{linden2003amazon} is the most successful and the most widely used technique, which is based on the hypothesis that people often get the best recommendations from someone with similar tastes to themselves\cite{breese1998empirical}. Another common approach is content-based filtering\cite{mooney2000content}, which tries to recommend items with similar properties to those that a user ordered in the past. Knowledge-based systems\cite{akerkar2010knowledge} recommend items based on specific domain knowledge about how certain item features meet users’ needs and preferences and how the item is useful for the user. Hybrid recommender systems are based on the combination of the above mentioned two or more types of techniques\cite{burke2002hybrid}. The other topic closely related to this category is deep learning based recommender system, which is able to effectively capture the non-linear and non-trivial user-item relationships, and enables the codification of more complex abstractions as data representations in the higher layers\cite{zhang2017deep}. For instance, Nguyen et al.\cite{nguyen2017personalized} proposed a personalized tag recommender system based on CNN. It utilizes constitutional and max-pooling layer to get visual features from patches of images. Wu et al.\cite{wu2016personal} designed a session-based recommendation model for real-world e-commerce website. It utilizes the basic RNN to predict what user will buy next based on the click histories. This method helps balance the tradeoff between computation costs and prediction accuracy.

The second category is about reinforcement learning for recommendations, which is different with the traditional item recommendations. In this paper, we consider the recommending procedure as sequential interactions between users and recommender agent; and leverage reinforcement learning to automatically learn the optimal recommendation strategies. Indeed, reinforcement learning have been widely examined in recommendation field. The MDP-Based CF model in Shani et al.\cite{shani2005mdp} can be viewed as approximating a partial observable MDP (POMDP) by using a finite rather than unbounded window of past history to define the current state. To reduce the high computational and representational complexity of POMDP, three strategies have been developed: value function approximation\cite{hauskrecht1997incremental}, policy based optimization \cite{ng2000pegasus,poupart2005vdcbpi}, and stochastic sampling \cite{kearns2002sparse}. Furthermore, Mahmood et al.\cite{mahmood2009improving} adopted the reinforcement learning technique to observe the responses of users in a conversational recommender, with the aim to maximize a numerical cumulative reward function modeling the benefit that users get from each recommendation session. Taghipour et al.\cite{taghipour2007usage,taghipour2008hybrid} modeled web page recommendation as a Q-Learning problem and learned to make recommendations from web usage data as the actions rather than discovering explicit patterns from the data. The system inherits the intrinsic characteristic of reinforcement learning which is in a constant learning process. Sunehag et al.\cite{sunehag2015deep} introduced agents that successfully address sequential decision problems with high-dimensional combinatorial slate-action spaces.  
