\section{Exclusive call center problem}%
\label{s:problem}

In order to avoid the details of particular MPM systems, we 
  introduce an abstract scenario called the \emph{exclusive call center problem.}
It consists of a call center that has $k$ operators capable
  of receiving calls (i.e., the call center has
  communication capacity $k$).
The call center promises exclusivity to a single organization.
This might be desirable to ensure high quality of service, 
  for legal reasons, or to prevent the accidental leak of trade or business 
  secrets to callers of a different organization.
When a caller issues a call, an automatic \emph{answering machine} $M$ routes 
  the call to an available operator who then processes the call.
If $M$ receives more calls than there are available operators, 
  then $M$ routes as many calls as it can, and notifies the remaining callers
  that all operators are busy.

While the above seems reasonable, the call center in question 
  is greedy and wishes to oversubscribe its resources by contracting with a 
  second organization---thereby violating its exclusivity agreement.
This poses two problems for the call center.
First, $M$ cannot determine to which organization a call belongs; only an
  operator is in a position of making that distinction.
Second, with the current decision logic of $M$ (route to available operators,
  notify remaining callers that operators are busy), a group of callers from 
  the same organization can collectively determine that they are not being given 
  exclusive access to the call center (e.g., by placing $k$ calls 
  and noticing that not all are picked up).
Given these issues and the limit of $k$ operators (which is publicly known), 
  can the call center do anything to maintain the illusion of exclusivity?

The first observation that the call center's CEO makes is that while there 
  are $k$ operators, there is no guarantee that all of them are available at 
  any given point in time.
After all, operators are human and take breaks.
This, the CEO believes, opens the door to some level of plausible deniability.
In particular, if $M$ gives a caller from organization $O_1$ a busy signal it 
  could mean: 
\begin{myenumerate}
\item All $k$ operators are busy handling other callers from $O_1$.
\item Some operators are busy handling callers from $O_1$ and the 
  remaining operators are on a break.
\item Some operators are busy handling callers from $O_1$, some are busy
  handling callers from $O_2$, and some are on a break.
\end{myenumerate}

Possibility 1 is the expected scenario of a high-efficiency trustworthy
  call center.
Possibility 2 is an unwanted outcome since it is inefficient, but
  it does not violate the contractual agreement.
Possibility 3, however, violates the promise of exclusivity.
The goal of the call center is to design $M$ such that it is 
  hard for either of the two organizations and their callers (assume no
  coordination between organizations) to infer that possibility 3 is the one 
  taking place.
As we alluded to earlier, the key challenge is that $M$ cannot distinguish 
  between callers (and determine to which organization they belong), and 
  therefore cannot selectively lie to keep a consistent set of responses.
We thus ask whether there exists an $M$ that can leverage the proposed
  ambiguity to fool the organizations into thinking they are exclusive.
%In other words, is there a \emph{private answering machine} $M$?

We think of $M$ as acting in rounds, where in each round, $M$ receives
  a set of calls $C$.
We seek two informal properties from $M$.
\begin{myitemize}
\item \textbf{Liveness}: eventually a caller in $C$ gets to 
  talk to an operator.
\item \textbf{Privacy}: it is computationally hard for any colluding subset of 
  callers $S \subseteq C$ (some of whom may get to speak to operators) to 
  distinguish between a scenario where $S = C$ and a scenario 
  where $S \subset C$ (i.e., it is difficult for the colluding subset of 
  callers to determine whether they are the only callers or not).
\end{myitemize}

The liveness guarantee is needed for $M$ to be useful, but also 
  to rule out a trivial solution: if $M$ never puts anyone through to an
  operator, then the probability that any colluding set of callers $S$ can
  distinguish between $S = C$ and $S \subset C$ is 1/2.

\paragraph{Security game.}
To define privacy and liveness more formally, we use a security game played
  between an adversary $\Att$ and a challenger parameterized by
  a polynomial time answering machine $M$ and a security parameter $\lambda$.
$M$ takes as input a subset of callers $C$ from the set of all possible callers
  $\mathbb{C}$, a communication capacity $k$, and a random string $r$, where 
  $k = poly(\lambda), |\mathbb{C}| = poly(\lambda), |r| = poly(\lambda)$.
$M$ outputs a set of callers $U \subseteq C$, such that $|U| \leq k$.

\begin{enumerate}
\item $\Att$ is given oracle access to $M$, and can issue a 
  $poly(\lambda)$ number of queries to $M$ with arbitrary inputs $C$, $k$, $r$.
For each query, $\Att$ can observe the corresponding result 
  $U \leftarrow M(C, k, r)$. 

\item Challenger samples a random bit $b$ uniformly in $\{0, 1\}$,
  and a random string $r$ uniformly in ${\{0, 1\}}^\lambda$.

\item $\Att$ picks a set of callers $S$ (where $S \subset \mathbb{C}$) and 
  positive integer $k$, and sends them to the challenger.

\item Challenger sets $C = S$ if $b = 0$, and $C = S \cup \{e\}$ if 
  $b = 1$ (where $e$ is a uniform random element from the set $\mathbb{C} - S$).

\item Challenger calls $M(C, k, r)$ to obtain $U \subseteq C$ where $|U| \leq k$.

\item Finally, the challenger removes $e$ from $U$ (if it is present) and returns 
  the result ($U - \{e\}$) to $\Att$.\label{game:remove}

\item $\Att$ outputs its guess $b'$, and wins the security game if $b = b'$.
\end{enumerate}

In summary, the adversary's goal in the game is to determine if 
  the challenger is communicating with the uncompromised caller $e$ after 
  compromising all of the other callers (represented by $S$).

\definition[Private answering machine]{An answering machine $M$ guarantees 
  privacy}\label{def:private-machine}
  if in the above security game with parameter $\lambda$, 
  for all PPT algorithms $\Att$, there exists a
  \emph{negligible function}%
%
\footnote{A function $f: \mathbb{N}\rightarrow\mathbb{R}$ is negligible if
    there exists an integer $c$ such that for all positive polynomials $poly$
    and all $x$ greater than $c$, $|f(x)| < 1/poly(x)$.}
%
  $\textrm{negl}$ such that: 
  $|\Pr[b = b'] - 1/2| \leq \textrm{negl}(\lambda)$, where
  the probability is over the random coins of $M$ and the challenger.

\definition[Live answering machine]{An answering 
  machine $M$ guarantees liveness}\label{def:liveness} if given security
  parameter $\lambda$, for any set of callers $C$, positive 
  communication capacity $k$, and random 
  string $r$, the probability that $M(C, k, r)$ outputs a non-empty set is non-negligible in $\lambda$. 
Here $|C| = poly(\lambda), k = poly(\lambda), |r| = poly(\lambda)$,
  and the probability is over the random coins of $M$.
