%\usepackage[pdftex]{graphicx}
\usepackage{booktabs}  % for tables
\usepackage{dcolumn}   % for tables
%\usepackage{hhline}   % for tables
\usepackage{multirow}  % for tables
\usepackage{rotating}
\usepackage{xspace}
\usepackage{hyphenat}  % supplies \hyp{}, which tells tex that it can 
		       % hyphenate at an existing hyphen
%\usepackage{xcolor}
%\usepackage[dvipsnames]{xcolor}
\usepackage{enumerate}

%\usepackage[square,comma,numbers,sort&compress]{natbib}

%\usepackage{ulem}      % for strikethrough and underlining
%\mw{above package turns italics to underlines.}

\usepackage{wrapfig}
\usepackage{textcomp}
%\usepackage{lastpage}
\usepackage{tabularx}
\usepackage{pifont}
%\usepackage{natbib}
%\usepackage{url}

\usepackage{wrapfig}

\usepackage{ulem}
\normalem


\usepackage{colortbl} % colorful columns and rows
\usepackage{array}    % needed for 'b' argument in tabular preamble

\usepackage{dblfloatfix}

%\usepackage{algorithm}
%\usepackage{algorithmic}
%\floatname{algorithm}{Pseudocode}
\usepackage{algpseudocode}
\algrenewcomment[1]{\hfill// #1}%
% Hack algpseudocode to be more Python-like
\algnotext{EndFunction}
\algnotext{EndFor}
\algnotext{EndIf}

\usepackage{amsmath,amscd}
\usepackage{amssymb}
\usepackage{amsfonts}
\usepackage{amsthm}

\algrenewcommand\algorithmicindent{1em}
\algrenewcommand{\algorithmiccomment}[1]{{\color{CadetBlue}\hfill// #1}}

\usepackage[noeka]{mathrmletter}

\renewcommand{\ttdefault}{cmtt}

\newif\ifextended
\newif\iflongbatching
\newif\ifsubmission
\newif\ifelementary

\ifx\buildextended\undefined
\else
    \extendedtrue
\fi

\newcommand{\textred}[1]{\textcolor{red}{#1}}
%\newcommand{\textcolor}[2]{\begingroup \color{#1} #2\endgroup}
\ifx\noeditingmarks\undefined
   \newcommand{\pgwrapper}[3]{\begingroup \color{#1} #2: #3 \endgroup}
   \newcommand{\pgwrapperb}[1]{\textbf{#1}}
\else
   \newcommand{\pgwrapperb}[1]{}
   \newcommand{\pgwrapper}[3]{}
\fi

\def\hn{\usefont{OT1}{phv}{mc}{n}\selectfont}
\def\hb{\usefont{OT1}{phv}{bc}{n}\selectfont}
\newcommand{\mpfont}{\hn\scriptsize}

\ifx\noeditingmarks\undefined
    \newcommand{\MPworker}[2]{{\color{#1}\vrule\vrule}{\marginpar{\color{#1}\mpfont #2}}}
\else
    \newcommand{\MPworker}[2]{}
\fi

\ifx\noeditingmarks\undefined
    \newcommand{\changebars}[2]{%
    [{\color{magenta}\em \begingroup {#1} \endgroup}][{\color{magenta}\sout{#2}}]}
\else
    \newcommand{\changebars}[2]{#1}
\fi

\newcommand{\changebarsii}[2]{#1}

\setlength{\marginparwidth}{15mm}
\setlength{\marginparsep}{0.35mm}

\newcommand{\Att}{\mathcal{A}}

\theoremstyle{definition}
\newtheorem{theorem}{Theorem}
\newtheorem{definition}{Definition}
\newtheorem{assumption}{Assumption}
\newtheorem{lemma}{Lemma}
\newtheorem{claim}[lemma]{Claim} 
\newtheorem{corollary}[lemma]{Corollary}


% customize thanks symbols
\makeatletter
\renewcommand*{\@fnsymbol}[1]{\ensuremath{\ifcase#1\or \star\or \dagger\or \ddagger\or
   \mathsection\or \mathparagraph\or \|\or **\or \dagger\dagger
   \or \ddagger\ddagger \else\@ctrerr\fi}}
\makeatother

\newcommand{\circledone}{\ding{192}\xspace}
\newcommand{\circledtwo}{\ding{193}\xspace}
\newcommand{\circledthree}{\ding{194}\xspace}
\newcommand{\circledfour}{\ding{195}\xspace}
\newcommand{\circledfive}{\ding{196}\xspace}
\newcommand{\filledone}{\ding{202}\xspace}
\newcommand{\filledtwo}{\ding{203}\xspace}
\newcommand{\filledthree}{\ding{204}\xspace}
\newcommand{\filledfour}{\ding{205}\xspace}
\newcommand{\filledfive}{\ding{206}\xspace}

\makeatletter
\def\imod#1{\allowbreak\mkern10mu({\operator@font mod}\,\,#1)}
\makeatother

\makeatletter
\setlength{\@fptop}{0pt}
\makeatother


%\def\compactify{\leftmargin=\parindent \itemsep=0.01pt \topsep=0.01pt \partopsep=0pt \parsep=0.01pt}
\def\compactify{\itemsep=0in \topsep=2pt \parsep=0.00in \partopsep=0pt
\leftmargin=2em}
\let\latexusecounter=\usecounter
\newenvironment{CompactItemize}
  {\def\usecounter{\compactify\latexusecounter}
   \begin{itemize}\addtolength{\itemsep}{-0.075in}}
  {\end{itemize}\let\usecounter=\latexusecounter}
\newenvironment{CompactEnumerate}
  {\def\usecounter{\compactify\latexusecounter}
   \begin{enumerate}}
  {\end{enumerate}\let\usecounter=\latexusecounter}

\newenvironment{myitemize}%
  {\begin{list}{\labelitemi}{\itemsep3pt \topsep3pt \parsep0.00in
  \partopsep=3pt \leftmargin1.2em}}%
  {\end{list}}
\newenvironment{myitemize2}%
  {\begin{list}{\labelitemi}{\itemsep1pt \topsep2pt \parsep0.00in
  \partopsep=0pt \leftmargin1.2em}}%
  {\end{list}}
\newenvironment{myitemize4}%
  {\begin{list}{\labelitemi}{\itemsep2pt \topsep2pt \parsep0.00in
  \partopsep=0pt \leftmargin1.2em}}%
  {\end{list}}
\newenvironment{myitemize5}%
  {\begin{list}{\threequartdash}{\itemsep3pt \topsep3pt \parsep0.00in
  \partopsep=3pt \leftmargin1.5em}}%
  {\end{list}}

%\newenvironment{myitemize}%
%  {\begin{list}{\labelitemi}{\itemsep4pt \topsep10pt \parsep0.00in
%  \partopsep=0pt}}%
%  {\end{list}}
\newenvironment{myenumerate}
  {\def\usecounter{\compactify\latexusecounter}
   \begin{enumerate}}
  {\end{enumerate}\let\usecounter=\latexusecounter}

\def\compactsortof{\itemsep=0in \topsep=2pt \parsep=0.00in \partopsep=0pt
\leftmargin=1.7em}
\newenvironment{myenumerate2}
  {\def\usecounter{\compactsortof\latexusecounter}
   \begin{enumerate}}
  {\end{enumerate}\let\usecounter=\latexusecounter}

%\def\compactsortof{\itemsep=3pt \topsep3pt \parsep=0ex \partopsep=0pt
%\leftmargin=1.55em}
\newenvironment{myenumerate3}
  {\def\usecounter{\compactsortof\latexusecounter}
   \begin{enumerate}}
  {\end{enumerate}\let\usecounter=\latexusecounter}

\def\compactsqueeze{\itemsep=0pt \topsep0pt \parsep=0ex \partopsep=0pt
\leftmargin=1.63em}
\newenvironment{myenumerate4}
  {\def\usecounter{\compactsqueeze\latexusecounter}
   \begin{enumerate}}
  {\end{enumerate}\let\usecounter=\latexusecounter}


\newcounter{saveenumi}

\newcommand{\astskip}{\smallskip\noindent\parbox{\linewidth}
			{\center*\hspace{2.5em}*\hspace{2.5em}*\medskip\smallskip}}

% uncomment to use regular paragraphs
%\def\normalpar{}

\ifx\normalpar\undefined
  \newcommand{\mypar}[1]{\textbf{#1}}
\else
  \newcommand{\mypar}[1]{\paragraph{#1}}
\fi

\def\discretionaryslash{\discretionary{/}{}{/}}
{\catcode`\/\active
\gdef\URLprepare{\catcode`\/\active\let/\discretionaryslash
        \def~{\char`\~}}}%
\def\URL{\bgroup\URLprepare\realURL}%
\def\realURL#1{\tt #1\egroup}%

% Local Variables:
% tex-command: "gmake;:"
% tex-main-file: "icing.ltx"
% tex-dvi-view-command: "gmake preview;:"
% End:
