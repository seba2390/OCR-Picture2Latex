\section{Building answering machines}%
\label{s:solutions}

We discuss two straw man proposals to highlight the challenge of
  constructing an answering machine that meets 
  Definitions~\ref{def:private-machine} and~\ref{def:liveness}.

\paragraph{Straw man $M_1$:} 
\begin{myitemize}
\item \textbf{Input}: $C, k, r$
\item $\pi \leftarrow$ uniform pseudorandom permutation of $C$ according to $r$
\item \textbf{Output}: the first $\min(k, |C|)$ elements from $\pi$
\end{myitemize}
This is not secure.
Let $X$ be the random variable describing the cardinality of the set returned
  to $\Att$, namely $|U - \{e\}|$.
Assuming that $k \leq |C|$, $\Pr[ X < k \,|\, b = 0 ] = 0$ and 
  $\Pr[ X < k \,| \,b = 1 ] = k / |C|$.
As a result, $\Att$ can, by simply counting the elements in $U - \{e\}$,
  distinguish between $b=0$ and $b=1$ with non-negligible advantage.

\paragraph{Straw man $M_2$:}
\begin{myitemize}
\item \textbf{Input}: $C, k, r$
\item $\pi \leftarrow$ uniform pseudorandom permutation of $C$ according to $r$
\item Sample $m \in [0, \min(k, |C|)]$ uniformly at random
\item \textbf{Output}: the first $m$ elements from $\pi$
\end{myitemize}

This is also not secure. Let $k = 1$ and $|S| = 1$.
The probability that the challenger returns to $\Att$ 
  an empty set is higher when $b=1$ due to
  Line~\ref{game:remove} in the security game and the way we construct $M_2$.
Again, let $X$ be the random variable describing the cardinality of the set 
  returned to $\Att$.
In particular, $\Pr[X < 1 \, | \, b = 0] = 1/2$, whereas 
  $\Pr[ X < 1 \, | \, b = 1] = 3/4$.
As a result, $\Att$ can distinguish between $b=0$ and $b=1$ with non-negligible
  advantage.
More generally, since $X$ is drawn from a uniform distribution when $b=0$, 
  the probability mass function (pmf) for $X$ (assuming $k \leq |C|$) is: 

\begin{equation*}
\begin{split}
  f(x) & = \begin{cases}
    \frac{1}{k+1} & \mbox{for  $0 \leq x \leq k$} \\
    0 & \mbox{otherwise}
    \end{cases}
\end{split}
\end{equation*}

On the other hand, if $b=1$, the pmf for $X$ is:

\begin{equation*}
\begin{split}
  f(x) & = \begin{cases}
    \frac{1}{k+1} + \frac{1}{2(k+1)} & \mbox{for $x = 0$}\\
  \frac{1}{k+1} & \mbox{for $1 \leq x \leq k-1$} \\
  \frac{1}{2(k+1)} & \mbox{for $x = k$}\\
    0 & \mbox{otherwise}
    \end{cases}
\end{split}
\end{equation*}

An adversary $\Att$ can leverage the difference in these pmfs 
  to distinguish between $b=0$ and $b=1$ with non-negligible advantage.

We could sample $m$ and permute $C$ non-uniformly, but the effect of 
  Line~\ref{game:remove} is large enough for $\Att$'s advantage to remain 
  non-negligible ($M$ must output a non-empty set with non-negligible 
  probability to satisfy liveness).
As a result, building an $M$ that guarantees privacy and liveness without
  a bound on the cardinality of $\mathbb{C}$ seems hard.
Below we give a construction under a relaxed setting.

\subsection{Machine with a bound set of callers}%
\label{s:solutions:bounded}

We now discuss the construction of an answering machine that provides privacy 
  and liveness under the assumption that the there is fixed upper bound on
  the number of possible callers ($|\mathbb{C}|$) and this bound is known 
  in advance to $M$ (the machine still does not know which callers belong to a 
  particular organization).
As a result, we assume that each element $e$ in $\mathbb{C}$ can be uniquely 
  mapped to an integer in the range $[1, |\mathbb{C}|]$ with the map 
  $id(e)$, and that this mapping is known to $M$.
The $id$ map can be set arbitrarily if the identity of all
  potential callers is known ahead of time, or populated dynamically as calls are 
  processed (a new caller $e$ is assigned a randomly unused integer in 
  $[1, |\mathbb{C}|]$ and this value is returned every time that $e$ calls). 
 
\paragraph{Private and live answering machine $M_3$:}
\begin{myitemize}
\item \textbf{Input}: $C, k, r$
\item $U \leftarrow \varnothing$
\item $\forall_{e \in C}, \forall_{0 \leq i < k}$, if $id(e) \equiv (r + i) \mod{|\mathbb{C}|}$, add
  $e$ to $U$
\item \textbf{Output}: $U$
\end{myitemize}

In other words, $M_3$ precomputes a schedule mapping callers to rounds: in 
  each round a set of $k$ callers will be serviced (the input $r$ is 
  the current round).
If a caller happens to call during a round that has been allocated for it,
  it will be added to the set $U$ (i.e., its call will be handled).
Otherwise, the call will not be answered.

Machine $M_3$ guarantees liveness because for every caller $e$, every $k$ out of 
  $|\mathbb{C}|$ rounds are assigned to $e$; since 
  $|\mathbb{C}| = poly(\lambda)$, this occurs with non-negligible probability.
Machine $M_3$ guarantees privacy because the response given to $\Att$ 
  at the end of the game (Step~\ref{game:remove} in the security game) depends 
  only on $r$ and not on $b$.
As a result this response is exactly the same when $b = 0$ and $b=1$; observing 
  this response gives no advantage to $\Att$.
