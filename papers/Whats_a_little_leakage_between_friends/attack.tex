\section{Compromised friend attack}%
\label{s:attack}

The exclusive call center problem is the scenario encountered by users 
  in MPM systems who communicate with compromised friends.
Clients in these systems can only handle a fixed number of 
  concurrent conversations in one round (this maps to the $k$ operators in 
  the call center problem), which opens the door to an attack
  that we call the \emph{compromised friend} (CF) attack.
An adversary---via compromised friends---can dial (or start a conversation 
  through any other means supported by the MPM system) a 
  client and observe whether the client responds or not.
If the client does not have a private answering machine, 
  the adversary can distinguish between a scenario where
  the client is talking to some honest client (i.e., the adversary's 
  subset of callers is not the full set, $S \subset C$), and a scenario
  where the client is not ($S = C$).
This can leak one bit of information that opens the door to existing attacks.

\paragraph{Intersection, disclosure, and hitting set attacks.}
There is a large literature of traffic analysis attacks that uncover
  patterns of communication by observing when users send and receive messages.
For example, intersection attacks~\cite{raymond00traffic} can be used to 
  narrow down the possible recipients of a message when users communicate
  with a single friend, while disclosure~\cite{agrawal03disclosure} and 
  hitting set~\cite{kesdogan04hitting} attacks can handle the case where 
  users communicate with multiple friends.
There are also statistical variants of these attacks~\cite{danezis03statistical}.

MPM systems purportedly avoid these attacks by requiring clients to always
  be online, continuously sending and retrieving messages; the client sends 
  dummy requests if the user is idle.
Unfortunately, the CF attack allows an adversary to guess whether a client 
  is sending dummy messages or not with non-negligible advantage.
An adversary can therefore target a set of potential senders and
  recipients with a CF attack, making these systems vulnerable to
  traffic analysis.
Note that the CF attack can be achieved in another way:
If a pair of friends is currently communicating at a rate
  of $r$ messages per round ($r < k$), and they wish to increase this
  rate to improve their throughput, this is the moral equivalent of 
  dialing (since it consumes a client's limited communication capacity).

\paragraph{Difficulty of conducting a CF attack in practice.}
There are some challenges in composing a CF attack with existing attacks.
First, depending on the answering mechanism of the MPM system, an adversary 
  might need to conduct a CF attack many times before learning anything useful 
  (recall from Section~\ref{s:problem} that while the adversary has 
  non-negligible advantage, it might still be small).
However, existing MPM systems~(e.g.,~\cite{angel16unobservable,
  vandenhoof15vuvuzela, lazar16alpenhorn}) currently implement an answering 
  mechanism that leaks information with a single CF attack.
Second, the CF attack requires actively targeting users on a given round, 
  which may limit the number of observations that are available to an adversary.
Last, this attack requires compromising users' actual friends 
  or it requires the use of phishing attacks to fool users into befriending 
  malicious users.
