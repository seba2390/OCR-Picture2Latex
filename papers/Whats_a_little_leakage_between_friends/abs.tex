\begin{abstract}
This paper introduces a new attack on recent messaging systems that protect 
  communication metadata.
The main observation is that if an adversary manages to compromise a user's
  friend, it can use this compromised friend to learn information
  about the user's \emph{other} ongoing conversations.
Specifically, the adversary learns whether a user is sending other messages or
  not, which opens the door to existing intersection and disclosure attacks.
To formalize this \emph{compromised friend attack}, we present an abstract 
  scenario called the \emph{exclusive call center problem} that 
  captures the attack's root cause, and demonstrates that it is 
  independent of the particular design or
  implementation of existing metadata-private messaging systems.
We then introduce a new primitive called a \emph{private answering machine} 
  that can prevent the attack.
Unfortunately, building a secure and efficient instance of this primitive 
  under only computational hardness assumptions does not appear possible.
Instead, we give a construction under the assumption that users 
  can bound their maximum number of friends and are okay leaking 
  this information.
\end{abstract}
