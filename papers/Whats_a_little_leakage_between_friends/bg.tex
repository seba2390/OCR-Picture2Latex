\section{Background}%
\label{s:bg}

The goal of metadata-private messaging systems~\cite{vandenhoof15vuvuzela,
  kwon17atom, alexopoulos17mcmix, tyagi17stadium, angel16unobservable}
  is to allow a pair (or group) of \emph{friends} to exchange bidirectional 
  messages without leaking metadata to any party besides the sender and the 
  recipient.
A pair of users are friends if they have previously shared a secret, either
  out-of-band (e.g., in person at a coffee shop), or in-band with 
  an \emph{add friend} protocol~\cite{lazar16alpenhorn}.
Users in these systems exchange a fixed number of messages with their friends 
  in discrete time epochs called rounds; users participate in every round even 
  if they are idle.
This ensures that an attacker that monitors the network cannot tell when users
  are actively communicating with their friends or starting/stopping 
  conversations.
This also places a bound on the number of active conversations that a user can 
  have at any time; we refer to this as the client's 
  \emph{communication capacity}.

Once a client reaches its communication capacity, it cannot send 
  messages to other friends until it ends an existing conversation.
As a result, clients use a separate \emph{dialing protocol} to 
  coordinate the start and end of conversations.
In a dialing protocol, a client sends a short message (a few bits) 
  to a friend regardless of whether the friend's client has reached its communication 
  capacity.
The dialing message is sufficient to notify a user that one of their friends 
  wishes to communicate, and to agree on a round to start the 
  conversation~\cite{lazar16alpenhorn}.
There are multiple ways in which a client can react to a dialing message.
Some natural choices are:

\begin{myitemize}
\item If the client has not reached its communication capacity, it can 
  automatically accept the call and start a new conversation.
\item The client could prompt the user (similar to calling a friend in Skype), 
  who can choose to accept or reject the call.
\item If at capacity, the client could randomly end an existing conversation 
  to make room for a new one.
\end{myitemize}

Each of these choices is problematic. 
If the client's communication capacity is $1$ (as in some of the existing
  systems~\cite{vandenhoof15vuvuzela, tyagi17stadium}) and the client 
  automatically accepts calls, then any of the client's friends can easily 
  learn when the client is \emph{not} active in a conversation simply by 
  calling.
Leaving the choice to the users is slightly better since the user can choose to 
  ignore or delay accepting some calls, but their choices can still 
  inadvertently lead to intersection attacks.
Ending conversations randomly hurts usability and might still leak information.
The goal of the next section is to formalize the desired properties of 
  the client's answering mechanism.
