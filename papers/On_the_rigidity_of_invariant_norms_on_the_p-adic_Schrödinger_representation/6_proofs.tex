%%%%%%%%%%%%%%%%%%%%%%%%%%%%%%%%%%%%---------------------------
\section{Proofs of the main results}
%%%%%%%%%%%%%%%%%%%%%%%%%%%%%%%%%%%%---------------------------
Theorems \ref{thm_strong_minimality_Z_p} will be proved in section $6$.
In this section we explain how to derive Theorem \ref{thm_Rigidity} and Proposition \ref{topologically_irreducible} from it, and perform easy reduction steps towards the proof in section $6$.

%%%%%%%%%%%%%%---------------
\subsection{Proof of Proposition \ref{topologically_irreducible}}
%%%%%%%%%%%%%%---------------

    \begin{proof}
    Let $\alpha\in\Grassmannian$.
    $(1)$. By Theorem \ref{thm_strong_minimality}, the norm $\norm{}_\alpha$ is locally maximal at $f$, for any $f\in \completion{\schw}{\norm{}_\alpha}$ with $\norm{f}_\alpha=1$.
    By Theorem \ref{Thm_strong_irreducibility} it is enough to show that $\completion{\schw}{\norm{}_\alpha}$ is topologically irreducible.
    Let $W$ be a proper closed sub-representation of $\completion{\schw}{\norm{}_\alpha}$.
    The quotient norm  on $\completion{\schw}{\norm{}_\alpha}/W$ induces an invariant semi-norm $\norm{}$ on $\schw$ that is dominated by $\norm{}_\alpha$.
    Since $\schw$ is irreducible, $\norm{}$ is a norm and by Theorem \ref{thm_strong_minimality}, $\norm{}=r\cdot \norm{}_\alpha$ for some $r>0$.
    Thus, $\norm{}$ is actually a norm and $W=0$.
    
    $(2)$. The claim is clear for $\completion{\schw}{\supnorm{}}=C_0(\Q_p^d)$.
    We will show that the general case follows from this one.
    Let $g\in \mathrm{Sp}_{2d}(\Q_p)$ such that $\alpha=Pg$, and let $T_g$ be a corresponding intertwining operator, normalized such that $\norm{}_\alpha=\supnorm{T_g(\cdot )}$ on $\schw$.
    Then $T_g$ extends to an isometric isomorphism $T_g:\completion{\schw}{\norm{}_\alpha}\map C_0(\Q_p^d)$.
    Although $T_g$ is not $\heis$-equivariant, it satisfies 
    \[T_g([w,t]f)=[wg,t]T_g(f).\]
    In particular, $T_g(f)$ is a smooth vector in $C_0(\Q_p^d)$ if and only if $f$ is a smooth vector in $\completion{\schw}{\norm{}_\alpha}$.
    
    $(3).$ Let $T:\completion{\schw}{\norm{}_\alpha}\map\completion{\schw}{\norm{}_\beta}$ be a continuous $\heis$-equivariant map.
    By the previous part, the restriction of $T$ to $\schw\subset \completion{\schw}{\norm{}_\alpha}$ is an $\heis$-equivariant map $T':\schw\map \schw$.
    By Schur's lemma for smooth representations, $T'$ is multiplication by a constant.
    If this constant is non-zero, it means that $\norm{}_\alpha$ and $\norm{}_\beta$, considered on $\schw$, are homothetic.
    By Proposition \ref{prop_norms_Grassmannian} this could only be the case if $\alpha=\beta$.
    Thus, if $\alpha\neq \beta$, $T=0$.
    If $\alpha=\beta$ then by continuity, $T$ is a multiplication by a scalar.
    \end{proof}
    
    
%%%%%%%%%%%%%%---------------
\subsection{Proof of Theorem \ref{thm_Rigidity}}
%%%%%%%%%%%%%%---------------

    \begin{proof}
    Let $(B,\norm{})$ be an irreducible Banach representation of $\heis$.
    Assume we are in the first case, and let $F:B\map \completion{\schw}{\norm{}_\alpha}$ be a continuous map of representations.
    Assume that $F$ is non-zero.
    Since $B$ is topologically irreducible, the kernel of $F$ is zero, so $F$ is injective. 
    By Proposition \ref{topologically_irreducible} and Theorem \ref{Thm_strong_irreducibility}, $F$ is surjective.
    Thus, $F$ is an isomorphism.
    The norm $\norm{F^{-1}(\cdot)}$ is an $\heis$-invariant norm on $\completion{\schw}{\norm{}_\alpha}$ that is dominated by $\norm{}_\alpha$.
    By theorem \ref{thm_strong_minimality}, there exists $r>0$ such that $r\cdot \norm{F^{-1}(\cdot)}=\norm{}_\alpha$.
    Replacing $\norm{}$ by $r\cdot\norm{}$, $F$ becomes an isometry.    
        
    Assume we are in the second case and let $F:\completion{\schw}{\norm{}_\alpha}\map B$ a continuous map of representations.
    Assume that $F$ is non-zero.
    The norm $\norm{F(\cdot)}$ is an $\heis$-invariant norm on $\completion{\schw}{\norm{}_\alpha}$ that is dominated by $\norm{}_\alpha$.
    By Theorem \ref{thm_strong_minimality}, there exists $r>0$ such that $r\cdot\norm{F(\cdot)}=\norm{}_\alpha$.
    Replacing $\norm{}$ by $r\cdot \norm{}$, $F$ becomes an isometry.
    The image of $F$ is therefore a closed sub-representation of $B$, and since $B$ is topologically irreducible, $F$ is surjective.
    \end{proof} 

%%%%%%%%%%%%%%---------------
\subsection{Reduction steps}
%%%%%%%%%%%%%%---------------
The goal of this section is to show that Theorem \ref{thm_strong_minimality} and Theorem \ref{thm_strong_minimality_Z_p} follow from the particular case of Theorem \ref{thm_strong_minimality_Z_p} with $d=1$.

\begin{prop}
If Theorem \ref{thm_strong_minimality} holds for the sup norm then it holds for $\norm{}_\alpha$ for any $\alpha\in\Grassmannian$.
\end{prop} 
    \begin{proof}
    Let $Pg=\alpha\in\Grassmannian$, where $g\in \mathrm{Sp}_{2d}(\Q_p)$, and let $T_g$ be an intertwining operator such that $\norm{}_\alpha=\supnorm{T_g(\cdot)}$.
    Let $\norm{}$ be an $\heis$-invariant norm on $\schw$, dominated by $\norm{}_\alpha$.
    
    The operators $T_g$ and $(T_g)^{-1}$ act on $\invNorms$ and preserve order.
    In particular, $\norm{(T_g)^{-1}(\cdot)}$ is an $\heis$-invariant norm, dominated by the sup norm.
    By assumption, $\norm{(T_g)^{-1}(\cdot)}=r\cdot \supnorm{}$ for some $r>0$.
    Thus, $\norm{}=r\cdot \norm{}_\alpha$.    
    \end{proof} 

Next we show that Theorem \ref{thm_strong_minimality} for the sup norm follows from Theorem \ref{thm_strong_minimality_Z_p}.
\begin{prop}
Assume that Theorem \ref{thm_strong_minimality_Z_p} holds.
Let $\norm{}$ be an $\heis$-invariant norm on $\schw$ that is dominated by the sup norm.
Then $\norm{}=r\cdot \supnorm{}$ for some $r>0$.
\end{prop} 
    \begin{proof}
    For any $n\in\N$ we denote $V_n=\schw(p^{-n}\Z_p^d)$ and think about $V_n$ as the subspace of $\schw(\Q_p^d)$ of functions supported on the disc $p^{-n}\Z_p^d$.
    The restriction of $\norm{}$ to $V_n$ is invariant under translations by $p^{-n}\Z_p^d$ and multiplication by smooth characters.
    By Theorem \ref{thm_strong_minimality_Z_p} and an obvious change of variables, there exists $r_n>0$ such that $\norm{f}=r_n\cdot\supnorm{f}$ for any $f\in V_n$.
    The function $\textbf{1}_{\Z_p^d}(x)$ lies in any of the $V_n$, so the numbers $(r_n)_{n\in\N}$ must be equal to the same $r$.
    Then $\norm{f}=r\cdot \supnorm{f}$ for any compactly supported function $f$.
    \end{proof}   
    
\begin{prop}
If Theorem \ref{thm_strong_minimality_Z_p} holds for $\Z_p$ then it holds for $\Z_p^d$ for any $d$.
\end{prop} 
    \begin{proof}
    The proof is by induction, the case $d=1$ being assumed to be true.
    Let $d>1$ and assume that Theorem \ref{thm_strong_minimality_Z_p} holds for $d-1$.
    Let $\norm{}$ be a norm on $\schw(\Z_p^d)$ that is invariant under translations and multiplication by smooth characters, dominated by the sup norm and normalized on $\textbf{1}_{\Z_p^d}(x)$.
    By Proposition \ref{prop_weak_minimality} it is enough to show that $\norm{}\leq \supnorm{}$.
    The latter follows if we show that for any $n$, $\norm{\textbf{1}_{p^n\Z_p^d}(x)}=1$, where $\textbf{1}_{p^n\Z_p^d}(x)$ is the characteristic function $p^n\Z_p^d$.
    
    Let $0<n\in \N$.
    Let $P_d$ by the projection $\alpha:\Z_p^d\map \Z_p$ given by $\alpha(a_1,...,a_d)=a_d$, and denote by $P_d^*$ the induced map $P_d^*:\schw(\Z_p)\map \schw(\Z_p^d)$.
    It is easy to see that the norm $\norm{P_d^*(\cdot)}$ on $\schw(\Z_p)$ is invariant under translations and multiplication by smooth characters, dominated by the sup norm and normalized at $\textbf{1}_{\Z_p}(x)$.
    Thus, $\norm{P_d^*(f)}=\supnorm{f}$ for any $f\in \schw(\Z_p)$.
    In particular,
    \[\norm{P_d^*(\textbf{1}_{p^n\cdot\Z_p}(x))}=1,\]
    Note that
    \[P_d^*(\textbf{1}_{p^n\cdot\Z_p}(x))=\textbf{1}_{\Z_p^{d-1}\times (p^n\cdot\Z_p)}(x).\]
    
    Now, consider the Projection $\beta:\Z_p^{d-1}\times (p^n\cdot \Z_p)\map\Z_p^{d-1}$ given by $\beta(a_1,...,a_{d-1},a_d)=(a_1,..,a_{d-1})$, and the induced map $\beta^*:C(\Z_p^{d-1})\map C(\Z_p^d)$.
    By the previous lemma, the norm $\norm{\beta^*(\cdot)}$ on $\schw(\Z_p^{d-1})$ is invariant under translations and multiplication by smooth character and dominated by the sup norm.
    Since 
    \[\beta^*(\textbf{1}_{\Z_p^{d-1}}(x))=\textbf{1}_{\Z_p^{d-1}\times(p^n\cdot \Z_p)}(x)\]
    and since $\norm{\textbf{1}_{\Z_p^{d-1}\times(p^n\cdot \Z_p)}(x)}=1$, we deduce that $\norm{\beta^*(\cdot)}$ is normalized on $\textbf{1}_{\Z_p^{d-1}}$.
    Therefore $\norm{\beta^*(\cdot)}=\supnorm{}$.
    In particular,
    \[\norm{\beta^*(\textbf{1}_{p^n\cdot \Z_p^{d-1}}(x))}=1.\]
    Note that 
    \[\beta^*(\textbf{1}_{p^n\cdot \Z_p^{d-1}}(x))=\textbf{1}_{p^n\cdot \Z_p^d}(x).\]
    Thus, we proved that $\norm{\textbf{1}_{p^n\cdot \Z_p^d}(x)}=1$.
    \end{proof} 

It remains to prove Theorem \ref{thm_strong_minimality_Z_p} for $\Z_p$.
This is done in the next section.



