%%%%%%%%%%%%%%%%%%%%%%%%%%%%%%%%%%%%---------------------------
\section{Further discussion about $\mathcal{H}$-invariant norms on $\mathcal{S}$}
%%%%%%%%%%%%%%%%%%%%%%%%%%%%%%%%%%%%---------------------------
This section is motivated by the search for other minimal invariant norms on $\schw$.
In addition, Theorem \ref{prop_independence_minimal_norms} is a generalization of the discontinuity of the Fourier transform proved in \cite{ophir2016q} to finite families of intertwining operators.

By now we have constructed two types of $\heis$-invariant norms on $\schw$: the family of minimal norms $\braces{\norm{}_\alpha\ |\ \alpha\in \Grassmannian}$, and, for each non-zero $f\in\schw$, the maximal invariant norm at $f$ which we denoted by $\norm{}_f$.
The latter belong to the maximal equivalence class of $\heis$-invariant norms.
Given any subset $I\subset \Grassmannian$, we can form the norm $\sup_{\alpha \in I}\norm{}_\alpha$.
The supremum exists since all the $\norm{}_\alpha$, being normalized at $\textbf{1}_{\Z_p^d}(x)$, are bounded from above by the maximal invariant norm at $\textbf{1}_{p^d\Z_p}(x)$.
In this section we consider finite families $I\subset \Grassmannian$ and the norms
\[\norm{}_I:=\max_{\alpha\in I}\norm{}_\alpha,\]
and answer the question: are there new minimal norms that lie below $\norm{}_I$?
We show that the answer is negative.
In fact, we will show the following.
\begin{thm}\label{thm_classification_small_norms}
$\ $
    \begin{enumerate}
    \item Let $I,J\subset \Grassmannian$ be distinct finite subsets. Then $\norm{}_I$ and $\norm{}_J$ are not equivalent.
    \item Let $\norm{}$ be an $\heis$-invariant norm which is dominated by $\norm{}_I$, where $I\subset \Grassmannian$ is finite.
    Then there exists $J\subset I$ such that $\norm{}$ is equivalent to $\norm{}_J$.
    \item If $I_1,I_2\subset \Grassmannian$ are finite and disjoint, there does not exist any $\heis$-invariant norm on $\schw$ which is dominated by both $\norm{}_{I_1}$ and by $\norm{}_{I_2}$.
    \end{enumerate}
\end{thm} 
Clearly, $(1)$ implies that the $J\subset I$ in $(2)$ is unique, and $(1)$ and $(2)$ imply $(3)$.
If $L_\alpha$ is the unit ball of $\norm{}_{\alpha}$, the meaning of $(3)$ is that if we put $L_I=\bigcap_{\alpha\in I}L_\alpha$, then $L_{I_1}+L_{I_2}=\schw$.

We will also show that any norm of the form $\norm{}_I$, where $I\subset\Grassmannian$ is finite, is equivalent to a norm which is locally maximal at some vector.

To prove these results, we introduce a  notion of independence of norms.
%%%%%%%%%%%%%%---------------
\subsection{Independence of norms}
%%%%%%%%%%%%%%---------------
The setting in this sub-section is general.
Let $V$ be a vector space over $\C_p$.
\begin{prop}\label{Prop_independence_two_norms}
Let $\norm{}_1,\norm{}_2$ be two norms on $V$.
The following are equivalent.
    \begin{enumerate}
    \item There exists no (non-zero) seminorm on $V$ which is dominated by both $\norm{}_1$ and $\norm{}_2$.
    \item The diagonal map
    \[V\map \completion{V}{\norm{}_1}\oplus\completion{V}{\norm{}_2}\]
    has a dense image, where the norm on the right hand side is $(v,w)\mapsto \max(\norm{v}_1,\norm{w}_2)$.
    \item Let $L_1,L_2$ be the closed unit balls of $\norm{}_1,\norm{}_2$ respectively.
    Then $L_1+L_2=V$.
    \end{enumerate}
\end{prop} 
\begin{proof}
We will show that each of $(1)$ and $(2)$ is equivalent to $(3)$.
If 
To show that $(1)$ and $(3)$ are equivalent, note that the gauge of $L_1+L_2$ is either zero, if $L_1+L_2=V$, or defines a non-zero seminorm $\norm{}'$ on $V$.
The seminorm $\norm{}'$ is dominated by both $\norm{}_1$ and $\norm{}_2$, and any seminorm that is dominated by both $\norm{}_1$ and $\norm{}_2$ is also dominated by $\norm{}'$.
From this it follows that $(1)$ and $(3)$ are equivalent.

We now show that $(2)$ and $(3)$ are equivalent.
It is easy to see that $(2)$ is equivalent to the statement that for any $w\in V$ and $\eps>0$ there exists $v\in V$ such that $\norm{v-w}_1<\eps$ and $\norm{v}_2<\eps$.
This statement is equivalent to the claim that any $w\in V$ can be written as $w=v_1+v_2$ with $\norm{v_1}_1<\eps$ and $\norm{v_2}_2<\eps$, and this is equivalent to $(3)$.
\end{proof} 

\begin{defn}
We say that two norms $\norm{}_1,\norm{}_2$ on $V$ are \textit{independent} if one of the equivalent conditions of the previous proposition is satisfied.
\end{defn} 

\begin{defn}
We say that the norms $\norm{}_1,...,\norm{}_n$ on $V$ are independent if for any $1\leq i\leq n$ the two norms:
\[\norm{}_i \ \ \ \text{and}\ \ \   \max_{\substack{1\leq j\leq n \\ j\neq i}}\norm{}_j\]
are independent.
\end{defn} 

Note that if $\norm{}_1,...,\norm{}_n$ are independent, so is any subset of them.

\begin{prop}\label{prop_independence_general}
Let $\norm{}_1,...,\norm{}_n$ be norms on $V$.
The following are equivalent.
    \begin{enumerate}
    \item $\norm{}_1,...,\norm{}_n$ are independent.
    \item The diagonal embedding
    \[V\xmap{\ \triangle\ }\bigoplus_{i=1}^n\completion{V}{\norm{}_i}\]
    has a dense image.
    \item For any two disjoint sets $I,J\subset \braces{1,2,..,n}$ the norms 
    \[\max_{i\in I}\norm{}_i\ \ \ \text{and}\ \ \ \max_{j\in J}\norm{}_j\]
    are independent.
    \end{enumerate}
\end{prop} 
    \begin{proof}
    As $(1)$ is a particular case of $(3)$, it remains to show $(1)\Rightarrow(2)\Rightarrow(3)$.
    We will prove this by induction on $n$.
    The case $n=2$ is essentially Proposition \ref{Prop_independence_two_norms}.
    
    Assume $(1)$.
    By the assumption and Proposition \ref{Prop_independence_two_norms}, the diagonal map
    \[V\map \completion{V}{\norm{}_1}\oplus\completion{V}{\max_{1<i\leq n}\norm{}_i}\]
    has a dense image.
    The norms $\norm{}_2,...,\norm{}_n$ are also independent and by the induction hypothesis the map
    \[\completion{V}{\max_{1<i\leq n}\norm{}_i}\map \bigoplus_{1<i\leq n}\completion{V}{\norm{}_i}\]
    is an isomorphism.
    Thus, $V\map \bigoplus_{1\leq i\leq n}\completion{V}{\norm{}_i}$ also has a dense image.
    
    Assume $(2)$ and let $I,J\subset \braces{1,...,n}$ be non-empty disjoint subsets.
    We may assume that $I\cup J=\braces{1,...,n}$.
    Denote $\norm{}_I=\displaystyle\max_{i\in I}\norm{}_i$ and $\norm{}_J=\displaystyle\max_{j\in J}\norm{}_j$.
    Consider the maps:
    \[V\xmap{\triangle} \completion{V}{\norm{}_I}\oplus\completion{V}{\norm{}_J}\map \bigoplus_{1\leq i\leq n}\completion{V}{\norm{}_i}.\]
    By the induction hypothesis, the second arrow is an isometric isomorphism. The first arrow $\Delta$ therefore has a dense image, so $(3)$ follows from Proposition \ref{Prop_independence_two_norms}.
    \end{proof} 

The following Proposition is left as an exercise to the reader.
\begin{prop}\label{prop_ind_invariant_norms}
Assume that $V$ is an irreducible representation of a group $G$, and $\norm{}_1,\norm{}_2\in\Norms(V)^G$.
Then $\norm{}_1$ and $\norm{}_2$ are independent if and only if there exists no $G$-invariant norm on $V$ that is dominated by both $\norm{}_1$ and $\norm{}_2$.
\end{prop} 



%%%%%%%%%%%%%%---------------
\subsection{Proofs of the claims in this section}
%%%%%%%%%%%%%%---------------

\begin{prop}\label{prop_independence_minimal_norms}
Let $I\subset \Grassmannian$ be a finite subset.
The norms $\braces{\norm{}_\alpha\ |\ \alpha\in I}$ are independent.
\end{prop} 
    \begin{proof}
    The proof is by induction on the size of the set $I$.
    If $|I|=1$ there is nothing to prove.
    Assume that $|I|=n>1$.
    Let $\alpha\in I$, we need to show that the two norms
    \[\norm{}_\alpha\ \ \ \text{and}\ \ \ \norm{}_{I\backslash\braces{\alpha}}:=\max_{\beta\in I\backslash\braces{\alpha}}\norm{}_\beta\]
    are independent.
    By Theorem \ref{thm_strong_minimality} and Proposition \ref{prop_ind_invariant_norms} it is enough to prove that $\norm{}_{I\backslash\braces{\alpha}}$ does not dominate $\norm{}_\alpha$.
    Suppose, for a contradiction, that $\norm{}_\alpha \dominated \norm{}_{I\backslash\braces{\alpha}}$.
    By the induction hypothesis, there is an isometry
    \[\completion{\schw}{I\backslash\braces{\alpha}}\xmap{\sim} \bigoplus_{\beta\in I\backslash\braces{\alpha}}\completion{\schw}{\norm{}_\beta}.\]
    Thus, we obtain a non-zero map 
    \[\bigoplus_{\beta\in I\backslash\braces{\alpha}}\completion{\schw}{\norm{}_\beta}\map \completion{\schw}{\norm{}_\alpha}.\]
    Then there exists $\beta\in I\backslash\braces{\alpha}$ such that the reduced map $\completion{\schw}{\norm{}_\beta}\map \completion{\schw}{\norm{}_\alpha}$ is non-zero.
    By Proposition \ref{topologically_irreducible} we have $\alpha=\beta$, a contradiction.    
    \end{proof} 

\begin{cor}
Let $I\subset \Grassmannian$ be a finite subset.
The norm $\norm{}_I$ is equivalent to a locally maximal norm (with respect to some vector).
\end{cor} 
    \begin{proof}
    Since the norms $\braces{\norm{}_\alpha\ |\ \alpha\in I}$ are independent, it follows by Proposition \ref{prop_independence_general} that
    \[\completion{\schw}{\norm{}}\simeq \bigoplus_{\alpha\in I}\completion{\schw}{\norm{}_\alpha}\]
    are isomorphic Banach representations (and even isometrically isomorphic).
    By Proposition \ref{topologically_irreducible}, the spaces $\braces{\completion{\schw}{\norm{}_\alpha}\ |\ \alpha\in I}$ are pairwise non-isomorphic.
    By Proposition \ref{prop_small_representations}, $\bigoplus_{\alpha\in I}\completion{\schw}{\norm{}_\alpha}$ has a strongly cyclic vector, and by Theorem \ref{thm_strongly_cyclic_spaces}, $\norm{}$ is equivalent to a locally maximal norm.
    \end{proof} 

\begin{proof}[Proof of Theorem \ref{thm_classification_small_norms}]
    As already noted, $(3)$ follows from $(1)$ and $(2)$.
    $(1)$ follows from the fact that the $\norm{}_\alpha$ are independent (Proposition \ref{prop_independence_minimal_norms}) and by Proposition \ref{prop_independence_general}.
    We prove $(2)$ by induction on the size of $I$.
    When $|I|=1$ the claim follows from Theorem \ref{thm_strong_minimality}.
    Assume that $|I|=n>1$, and that the claim is true for all subsets of $\Grassmannian$ of size $<n$.
    Let $\norm{}$ be an $\heis$-invariant norm on $\schw$ that is dominated by $\norm{}_I$.
    Then $\norm{}$ extends to an $\heis$-invariant seminorm on the completion $\completion{\schw}{\norm{}_I}$, which, by the independence of the $\norm{}_\alpha$, is isometrically isomorphic to $\bigoplus_{\alpha\in I}\completion{\schw}{\norm{}_\alpha}$ via the diagonal embedding.
    By Proposition \ref{prop_small_representations}, the kernel of $\norm{}$ is of the form $\bigoplus_{\alpha\in K}\completion{\schw}{\norm{}_{\alpha}}$ for some subset $K\subset I$.
    Using the diagonal embedding, this means that $\norm{}$ is already dominated by $\norm{}_{I\backslash K}$.
    If $K$ is non-empty, then $|I\backslash K|<|I|$ and the claim is true by the induction hypothesis.
    Assume that $K$ is empty.
    Then $\norm{}$ is a norm on $\bigoplus_{\alpha\in I}\completion{\schw}{\norm{}_{\alpha}}$.
    We want to show that in this case $\norm{}$ is equivalent to $\norm{}_I$.
    Choose $\alpha\in I$ and denote $J=I\backslash\braces{\alpha}$.
    By Theorem \ref{thm_strong_minimality}, the restriction of $\norm{}$ to the component $\completion{\schw}{\norm{}_\alpha}$ is of the form $r_\alpha\cdot \norm{}_\alpha$.
    Similarly, the seminorm on $\completion{\schw}{\norm{}_\alpha}$, obtained from $\norm{}$ by taking the quotient of $\bigoplus_{\alpha\in I}\completion{\schw}{\norm{}_\alpha}$ by $\bigoplus_{\beta\in J}\completion{\schw}{\norm{}_\beta}$ is of the form $s_\alpha\cdot \norm{}_\alpha$.
    Clearly, $s_\alpha \leq r_\alpha$.
    We claim that $0<s_\alpha$.
    By the induction hypothesis, the restriction of $\norm{}$ to the component $\bigoplus_{\beta\in J}\completion{\schw}{\norm{}_\beta}$ is equivalent to $\norm{}_J$.
    It follows that $\bigoplus_{\beta\in J}\completion{\schw}{\norm{}_\beta}$ is a closed subspace of $\bigoplus_{\alpha\in I}\completion{\schw}{\norm{}_\alpha}$ with respect to the topology induced by $\norm{}$.
    Therefore, $s_\alpha\cdot \norm{}_\alpha$ is a norm, so $s_\alpha>0$.
    This is true for any $\alpha\in I$, so
    \[\max_{\alpha\in I}(s_\alpha\cdot\norm{}_\alpha)\leq\norm{}'\leq \max_{\alpha\in I}(r_\alpha\cdot\norm{}_\alpha),\] 
    which shows that $\norm{}$ is equivalent to $\norm{}_I$.
   \end{proof} 





    
    
    
%%%%%%%%%%%%%%---------------
\subsection{Open questions}
%%%%%%%%%%%%%%---------------
We conclude with some open questions that we find interesting.
\begin{question}
Does there exists an $\heis$-invariant norm on $\schw$ which does not dominate any of the norms $\norm{}_\alpha$, for $\alpha\in\Grassmannian$?
\end{question}
We find this question especially interesting, regardless of the answer.
If the answer is negative, the spaces $\braces{\completion{\schw}{\norm{}_\alpha}\ |\ \alpha\in \Grassmannian}$ form a complete list of the irreducible completions of $\schw$.
If the answer is positive, constructing such norms will require new ideas that could be useful in the study of Banach representations of $p$-adic groups.
In the latter case, we also ask
\begin{question}
Does there exist another $\heis$-invariant norm on $\schw$, the completion by which is an (strongly) irreducible Banach representations?
\end{question}

The last section gives a complete picture of those norms which are dominated by some $\norm{}_I$, for a finite subset $I\subset \Grassmannian$.
When $I$ is not finite, we can still define the norm $\norm{}_I$ as before.
Now it seems reasonable to consider the topology of $\Grassmannian$.

\begin{question}
Let $I_1,I_2$ be closed and disjoint subsets of $\Grassmannian$.
    \begin{enumerate}
    \item Are the norms $\norm{}_{I_1}$ and $\norm{}_{I_2}$ independent?
    \item Is there a simple description of the completion $\completion{\schw}{\norm{}_{I_1}}$ in terms of the completions $\completion{\schw}{\norm{}_\alpha}$ for $\alpha\in I_1$?
    \end{enumerate}
\end{question}
Finally, taking $I=\Grassmannian$, we ask
\begin{question}
Does the norm $\sup_{\alpha\in\Grassmannian}\norm{}_\alpha$ belong to the maximal equivalence class of $\heis$-invariant norms on $\schw$?
\end{question}

    
    
    