%%%%%%%%%%%%%%%%%%%%%%%%%%%%%%%%%%%%---------------------------
\section{A reminder on $p$-adic Heisenberg groups and Schrödinger representations}
%%%%%%%%%%%%%%%%%%%%%%%%%%%%%%%%%%%%---------------------------
In this section we recall the classical theory of smooth irreducible representations of Heisenberg groups over $\Q_p$.
This section is based on \cite{kudla_notes,van1978smooth}.
Throughout this section, the following are fixed: $p$ is a prime number and $\Q_p$ is the field of $p$-adic numbers, $d\geq 1$ is an integer and $C$ is an algebraically closed field of characteristic zero.

%%%%%%%%%%%%%%---------------
\subsection{The Heisenberg group over $\mathbb{Q}_p$ and its smooth representations}
%%%%%%%%%%%%%%---------------
Let $W=\Q_p^d\oplus \Q_p^d$ and denote by $\omega$ the symplectic form on $W$ given by $\omega((x_1,y_1),(x_2,y_2))=x_1\dotprod y_2-y_1\dotprod x_2$, where $a\dotprod b$, for $a,b\in\Q_p^d$, is the standard scalar product.

We denote by $\heis=\heis_{2d+1}(\Q_p)$ the $2d+1$-dimensional Heisenberg group.
Its underlying set is $W\times \Q_p$ and the multiplication is given by 
\[[w_1,t_1]\cdot [w_2,t_2]=[w_1+w_2,t_1+t_2+\frac{1}{2}\omega(w_1,w_2)].\]
One easily verifies that the center of $\heis$, which is also its commutator subgroup, is $Z:=\braces{[0,t]\ |\ t\in \Q_p}$, and that $\heis/Z\simeq W=\Q_p^{2d}$.
In particular, $\heis$ is a two step nilpotent group.

As a topological group, $\heis$ inherits a topology from the topology of $\Q_p$.
This makes $\heis$ a totally disconnected (t.d.) and locally compact topological group.

Recall that a representation $(V,\pi)$ of a t.d. group $G$ over $C$ is said to be smooth if the stabilizer $Stab_G(v)$ in $G$ of any vector $v\in V$ is open.
A smooth representation of $G$ is called admissible if for any open compact subgroup $K\subset G$ the sub-space $V^K$ of vectors fixed by $K$ is finite dimensional.

By Schur's lemma, if $(V,\pi)$ is a smooth irreducible representation of $\heis$, the center of $\heis$ acts on $V$ via a character $\psi$, called \textit{the central character} of $\rho$.
We identity the center of $\heis$ with $\Q_p$ and view $\psi$ as a character $\psi:(\Q_p,+)\map C^\times$.
Since $\pi$ is smooth, the kernel of $\psi$ is an open subgroup of $\Q_p$ and we say that $\psi$ is a smooth character.

The classification of smooth irreducible representations of $\heis$ is well known, and we recall it.
If $\psi$ is trivial, the action of $\heis$ factors through an abelian quotient, and $V$ is $1$-dimensional.
Assume that $\psi$ is non-trivial.
We construct a representation $\rho_\psi$, called the Schrödinger representation of $\heis$, which has central character $\psi$.

Let $\schw=\schw(\Q_p^d)$ be the space of Schwartz functions, that is functions $f:\Q_p^d\map C$ which are locally constant and compactly supported.
It is an infinite dimensional vector space over $C$.
Define a representation $\rho_\psi$ of $\heis$ on $\schw$ as follows.
Let $w=(a,b)$ with $a,b\in\Q_p^d$.
Then
\[(\rho_{\psi}([w,t])f)(x)=\psi\sog{t+\frac{1}{2}a\dotprod b+b\dotprod x}\cdot f(x+a).\]

\begin{thm}[Smooth Stone-von Neumann]
Let $\psi$ be a non-trivial smooth character of $(\Q_p,+)$.
    \begin{enumerate}
    \item The representation $\rho_\psi$ is a smooth, irreducible and admissible representation of $\heis$ and has central character $\psi$.
    \item Let $(V,\pi)$ be a smooth representation of $\heis$.
    Assume that the center of $\heis$ acts via the character $\psi$.
    Then $V$ decomposes as a direct sum of sub-representations, each isomorphic to $\rho_\psi$.
    \end{enumerate}
\end{thm} 

It is important to note that in the Schrödinger representation, the Heisenberg group acts on $\schw$ by translations and by multiplication by the smooth characters of $\Q_p^d$.
A smooth character of $\Q_p^d$ is a homomorphism $\alpha:(\Q_p^d,+)\map C^\times$ with an open kernel.
By definition, any translation appears as an action of an element of the Heisenberg group. 
It is also true that if $\alpha:\Q_p^d\map C^{\times}$ is a smooth character, there exists an element $[0,b,0]$ whose action is multiplication by $\alpha$.
Indeed, if $\psi$ is a non-trivial smooth character of $\Q_p$, any smooth character of $\Q_p^d$ is of the form $\psi\circ \lambda$, for some $\lambda$ in the dual space of $\Q_p^d$.

%%%%%%%%%%%%%%---------------
\subsection{Automorphisms of the Heisenberg group and intertwining operators on the Schrödinger representation}
%%%%%%%%%%%%%%---------------
Let $J=\begin{pmatrix}0 & I_d\\ -I_d & 0 \end{pmatrix}$ and $\mathrm{Sp}_{2d}(\Q_p)$ be the symplectic group
\[\mathrm{Sp}_{2d}(\Q_p)=\braces{g\in \mathrm{GL}_{2d}(\Q_p)\ |\ gJg^t=J}.\]
Thinking about the vectors in $\Q_p^d\oplus\Q_p^d$ as row vectors, an element $g\in \mathrm{Sp}_{2d}(\Q_p)$ acts on $W$ by right multiplication: $w\mapsto wg$ and preserves the symplectic form $\omega$.
This defines a right action of the symplectic group on the Heisenberg group by automorphisms as follows:
\[[w,t]\cdot g=[wg,t].\]
These automorphisms are continuous and their restriction to the center $Z=\braces{[0,t]\ |\ t\in\Q_p}$ is the identity.
Moreover, any continuous automorphism of $\heis$ whose restriction to the center is the identity is a composition of a conjugation and an automorphism coming from the symplectic group.
    \begin{remark}
    This is not true for Heisenberg groups over extension fields of $\Q_p$, and this is the only reason why we restrict to $\Q_p$.
    \end{remark} 

Let $g\in \mathrm{Sp}_{2d}(\Q_p)$.
Define a new representation $\rho_{g,\psi}$ on $\schw$ by 
\[\rho_{g,\psi}([w,t])f=\rho_\psi([w,t]\cdot g)f\]
for any $[w,t]\in\heis$ and $f\in\schw$.
The representation $\rho_{g,\psi}$ is smooth, irreducible and has $\psi$ as its central character.
Thus, by the Stone-von Neumann theorem, $\rho_\psi\simeq \rho_{g,\psi}$, so there exists an invertible linear operator $T_g$ on $\schw$ such that 
\[ \rho_\psi([w,t])\circ T_g=T_g\circ\rho_{g,\psi}([w,t])\]
for any $[w,t]\in \heis$.
By Schur's lemma, $T_g$ is unique up to a multiplicative constant.
It also follows that $T_{g_1\cdot g_2}$ is equal, up to a constant, to $T_{g_1}\circ T_{g_2}$, so $g\mapsto T_g$ is a projective representation, called the Weil representation.
There is an explicit formula for the operators $T_g$.
Let $g\in \mathrm{Sp}_{2d}(\Q_p)$ and write it as 
\[g=\twomat{a}{b}{c}{d}\]
where $a,b,c,d\in M_d(\Q_p)$ are square matrices.
\begin{prop}[\cite{kudla_notes}, Proposition 2.3]\label{prop_intertwining_formula}
Let $g$ as above and $T_g$ an intertwining operator corresponding to $g$.
There is a unique choice of a $\C_p$-valued Haar distribution $d\mu$ on $Im(c)$ such that 
\[T_g(f)(x)=\intop_{Im(c)}\psi\sog{\frac{1}{2}(xa)\dotprod (xb)-(xb) \dotprod y+\frac{1}{2}y\dotprod (yd)}\cdot f(xa+y)\ d\mu(y).\]
Here, $Im(c)$ is the space $\braces{vc\ |\ v\in\Q_p^d}$.
\end{prop} 
Note that if $g=J$, the formula gives, up to normalization, the usual Fourier transform. 
If $c=0$, we get the operation of multiplication by a quadratic exponential accompanied by the dilation $x\mapsto xa$. 