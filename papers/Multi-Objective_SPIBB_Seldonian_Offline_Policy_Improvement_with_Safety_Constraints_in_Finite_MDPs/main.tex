%%%%%%%% ICML 2021 EXAMPLE LATEX SUBMISSION FILE %%%%%%%%%%%%%%%%%

\documentclass{article}

% Recommended, but optional, packages for figures and better typesetting:
\usepackage{microtype}
\usepackage{graphicx}
% \usepackage{subfigure} % has conflict with subfig/subcaption
\usepackage{booktabs} % for professional tables

% hyperref makes hyperlinks in the resulting PDF.
% If your build breaks (sometimes temporarily if a hyperlink spans a page)
% please comment out the following usepackage line and replace
\usepackage{hyperref}

% -------- Custom packages 


\usepackage{url}            % simple URL typesetting
\usepackage{amsfonts}       % blackboard math symbols
\usepackage{amsthm}
\usepackage{mathrsfs}
\usepackage{nicefrac}       % compact symbols 
\usepackage{microtype}      % microtypography
\usepackage{booktabs} % for professional tables

\usepackage[utf8]{inputenc} % allow utf-8 input
\usepackage[T1]{fontenc}    % use 8-bit T1 fonts

\usepackage{tikz}
\usepackage{amsmath}
\usepackage{amssymb}
\usepackage{graphicx}
\usepackage{tabularx}
\usepackage{mathtools}
\usepackage{tcolorbox}
\usepackage{cancel}
\usepackage{pifont} % for checkmarks and cross

% Use the give algorithmic style files isntead?
%\usepackage{algorithm2e}
% \usepackage[linesnumbered,lined,algoruled,boxed,commentsnumbered]{algorithm2e}
% \usepackage{algpseudocode,algorithm,algorithmicx}

\usepackage{multirow}
\usepackage{color}
\usepackage{mathtools}
% \usepackage{todonotes}
\usepackage{wrapfig}
\usepackage{caption}
\usepackage{bbm}
\usepackage{bm}
\usepackage{esvect}
\usepackage[normalem]{ulem}
% \usepackage{subcaption}

% \usepackage[lite]{mtpro2}


\usepackage{xcolor} %
\usepackage{xargs}
\usepackage{rotating}
\usepackage{longtable}
\usepackage{adjustbox}

\usepackage{cleveref}
% \usepackage{subfig}
\usepackage{subcaption}
\usepackage{enumitem}

% \captionsetup[subfigure]{subrefformat=simple,labelformat=simple}
% \renewcommand\thesubfigure{(\alph{subfigure})}

% overwrite cref name for section for saving space, format: crefname{content-type}{singlur}{plural}
% \crefname{section}{Sec.}{Sec.}
% \crefname{equation}{Eq.}{Eq.}
% \crefname{figure}{Fig.}{Fig.}


\newcolumntype{Y}{>{\raggedright\arraybackslash}p}
	
\usepackage{contour}
\usepackage{ulem}

\contourlength{1pt}

\renewcommand{\ULdepth}{1.8pt}
\contourlength{0.8pt}

\newcommand{\myuline}[1]{%
  \uline{\phantom{#1}}%
  \llap{\contour{white}{#1}}%
}


% footnotes
\newcommand\rlf[1]{{\color{red}\footnote{{\color{red}From Romain: #1}}}}

% inline commands for notes
\newcommand\rli[1]{{\color{red} #1}}
\newcommand\romain[1]{{\color{red} #1}}
\newcommand\joelle[1]{{\color{green} #1}}
\newcommand\phil[1]{{\color{cyan} #1}}

\newcommand\harsh[1]{{\color{blue} #1}}
\newcommand\htodo[1]{{\color{blue} TODO: #1\\}}
\newcommand\hnote[1]{{\color{blue} NOTE: #1\\}}


% Attempt to make hyperref and algorithmic work together better:
\newcommand{\theHalgorithm}{\arabic{algorithm}}


%SLJ: standard prettier hypersetup:
\hypersetup{
	plainpages=false,
	%pdfpagelabels=true,          % Prevents duplicate labels
	colorlinks=true,              % Use link colors
	linkcolor=blue,               % Color for normal internal links
	anchorcolor=blue,             % Color for anchor text
	citecolor=blue,               % Color for bibliographical citations in text
	filecolor=blue,               % Color for URLs which open local files
	pagecolor=blue,               % Color for links to other pages
	urlcolor=blue,                % Color for linked URLs
	pdfview=FitH,                 % Fit the width of the document
	pdfstartview=FitH,            % Fit the width of the document
	pdfpagelayout=SinglePage      % Page-down goes to next page
}


%%%%% NEW MATH DEFINITIONS %%%%%

\usepackage{amsmath,amsfonts,bm}
\usepackage{xifthen}

% Highlight a newly defined term
\newcommand{\newterm}[1]{{\bf #1}}

\def\eps{{\epsilon}}


% Utility for ticks 
\newcommand{\cmark}{\ding{51}}%
\newcommand{\xmark}{\ding{55}}%

% Theorem styles 
\theoremstyle{definition}
\newtheorem{theorem}{Theorem}[section]
\newtheorem{definition}{Definition}[section]
% \newtheorem{remark}{Remark}[theorem] %numbered remark
\newtheorem*{remark}{Remark} %unnumbered remark
\newtheorem{lemma}{Lemma}[section]
\newtheorem{prop}{Proposition}[section]
\newtheorem{corollary}{Corollary}[theorem]
\newtheorem{conjecture}{Conjecture}[section]
\newtheorem{assumption}{Assumption}[section]

\newtheorem{manualtheoreminner}{Theorem}
\newenvironment{manualtheorem}[1]{%
  \renewcommand\themanualtheoreminner{#1}%
  \manualtheoreminner
}{\endmanualtheoreminner}


% Math helper - standard function
\DeclareMathOperator*{\argmax}{arg\,max}
\DeclareMathOperator*{\argmin}{arg\,min}
\DeclareMathOperator{\support}{support}
\DeclareMathOperator{\MAX}{MAX}
\DeclareMathOperator{\term}{\texttt{term}}
\DeclareMathOperator*{\logsumexp}{log-sum-exp}
\DeclareMathOperator*{\TV}{TV}
\newcommand{\norm}[1]{\left\lVert#1\right\rVert}
\DeclarePairedDelimiter\set\{\}
\DeclarePairedDelimiter\abs{\lvert}{\rvert}%
\newcommand*{\mytop}{\mathrel{\scalebox{0.5}{$\top$}}}
\newcommand*{\mybot}{\mathrel{\scalebox{0.5}{$\bot$}}}
\newcommand*{\mydiese}{\mathrel{\scalebox{0.5}{$\#$}}}
\newcommand*{\myplus}{\mathrel{\scalebox{0.5}{$+$}}}
\newcommand*{\myminus}{\mathrel{\scalebox{0.5}{$-$}}}
\newcommand*{\bmg}{\bm{\gamma}}
\newcommand*{\bml}{\bm{\lambda}}

% MDP notation
\renewcommand{\S}{\mathcal{S}}
\newcommand{\X}{\mathcal{X}}
\newcommand{\A}{\mathcal{A}}
\newcommand{\T}{\mathcal{T}}
\newcommand{\M}{\mathcal{M}}
\newcommand{\B}{\mathcal{B}}
\newcommand{\Bset}{\mathfrak{B}}
\newcommand{\Dist}{\mathscr{P}}
\newcommand{\D}{\mathcal{D}}
\newcommand{\Real}{\mathbb{R}}
\renewcommand{\P}{\mathcal{P}}
\newcommand{\E}{\mathop{\mathbb{E}}}
\renewcommand{\H}{\mathcal{H}}
% \newcommand{\R}{\mathcal{R}}
% \newcommand{\C}{\mathcal{C}}

% Extended MDP notation
\newcommand{\Pstar}{p^{\star}}
\newcommand{\Rstar}{\bm{r}^{\star}}
\newcommand{\Cstar}{C^{\star}}
% \newcommand{\rmax}{\textsc{Rmax}}
\newcommand{\rmax}{r_{\mytop}}
\newcommand{\cmax}{\textsc{Cmax}}

\newcommand{\mstar}{m^{\star}}
\newcommand{\mhat}{\hat{m}}
\newcommand{\mopt}{m^{\star}}

\newcommand{\Phat}{\hat{p}}
\newcommand{\Rhat}{\hat{\bm{r}}}
\newcommand{\Chat}{\hat{C}}

% Math helper - custom function
\newcommand{\expwrtpi}[1]{\E_{\pi} [\sum_{t=0}^{\infty} \gamma^t #1(s_t, a_t)]}
\newcommand{\expangle}[1]{\langle #1  \rangle}

% helper function for return and constraints

% for value function, takes arguments:
% #1: policy 
% #2: the function of interest, R or C_i
% #3 (optional): the MDP for which this is estimated
\newcommand{\V}[3]{ %
    \ifthenelse{\isempty{#3}}%
    {V^{#1}(#2)}% #3 is empty 
    {V^{#1}_{#3}(#2)}%
}

\newcommand{\Q}[3]{
    \ifthenelse{\isempty{#3}}
    {Q^{#1}(#2)}% #3 is empty 
    {Q^{#1}_{#3}(#2)}%
}


\newcommand{\Adv}[3]{
    \ifthenelse{\isempty{#3}}
    {A^{#1}(#2)}% #3 is empty 
    {A^{#1}_{#3}(#2)}%
}

% careful diff notation
% 1: pi
% 2: R/C
% 3: M
\newcommand{\J}[3]{
    \ifthenelse{\isempty{#3}}
    {\mathcal{J}^{#1}_{#2}}% #3 is empty -> eg V^{\pi}(x ; R)
    {\mathcal{J}^{#1}_{#3,#2}}% -? eg V^{\pi}_{M}(x ; C)
    % {J_{#2}(#1)}% #3 is empty 
    % {J_{#2}(#1, #3)} %
}



\newcommand{\MRkern}{%
  \mkern-6.5mu
  \mathchoice{}{}{\mkern0.2mu}{\mkern0.5mu}%
}

% for value function, takes arguments:
% #1: policy 
% #2: the function of interest, R or C_i
% #3 (optional): the MDP for which this is estimated
% #4: variables to be given input (x) or (x,a)
\newcommand{\val}[4]{ %
    \ifthenelse{\isempty{#3}}%
    {v^{#1}_{#2}(#4)}% #3 is empty -> eg V^{\pi}(x ; R)
    {v^{#1}_{#3,#2}(#4)}% -? eg V^{\pi}_{M}(x ; C)
    % {V^{#1}_{#3}(#4 ;#2)}% -? eg V^{\pi}_{M}(x ; C)
    % {V_{#2}(#4 ; #1)}% #3 is empty -> eg V_R(x ; \pi)
    % {V_{#2}(#4 ;#1, #3)}% -? eg V_C(x ; \pi, M)
    % {#2 \MRkern V^{#1}_{#3}(#4)}% -? eg V^{\pi}_{M}(x ; C) # combines the letter V and R together
}

\newcommand{\qval}[4]{
    \ifthenelse{\isempty{#3}}
    {q^{#1}_{#2}(#4)}% #3 is empty -> eg V^{\pi}(x ; R)
    {q^{#1}_{#3,#2}(#4)}% -? eg V^{\pi}_{M}(x ; C)
    % {Q^{#1}(#4 ; #2)}% #3 is empty -> eg Q^{\pi}(x,a ; R)
    % {Q^{#1}_{#3}(#4 ;#2)}% -? eg Q^{\pi}_{M}(x,a ; C)
    % {Q_{#2}(#4 ; #1)}% #3 is empty -> eg Q_R(x,a ; \pi)
    % {Q_{#2}(#4 ;#1, #3)}% -? eg Q_C(x,a ; \pi, M)
}
\DeclareMathOperator*{\advantage}{Adv}

\newcommand{\adv}[4]{
    \ifthenelse{\isempty{#3}}
    {\advantage^{#1}_{#2}(#4)}% #3 is empty -> eg V^{\pi}(x ; R)
    {\advantage^{#1}_{#3,#2}(#4)}% -? eg V^{\pi}_{M}(x ; C)
    % {A^{#1}(#4 ; #2)}% #3 is empty -> eg Q^{\pi}(x,a ; R)
    % {A^{#1}_{#3}(#4 ;#2)}% -? eg Q^{\pi}_{M}(x,a ; C)
    % {A_{#2}(#4 ; #1)}% #3 is empty -> eg A_R(x,a ; \pi)
    % {A_{#2}(#4 ;#1, #3)}% -? eg A_C(x,a ; \pi, M)
}




\newcommand{\ci}{C}

\newcommand{\pib}{\pi_{b}}
\newcommand{\piopt}{\pi^{*}}
\newcommand{\pie}{\pi_{t}}

\newcommand{\lR}{\lambda_{R}}
\newcommand{\lC}{\lambda_{C}}
\newcommand{\ephi}{e_{\phi}}

\newcommand{\pr}{\text{Pr}}
\newcommand{\IS}{\text{IS}}
\newcommand{\CI}{\text{CI}}


% SPIBB symbols 
\newcommand{\EpsPib}{(\pi_b, e, \epsilon)}


% ----------------------------------------------------------------
%               Conference specific packages 
% ----------------------------------------------------------------

% Use the following line for the initial blind version submitted for review:
\usepackage[final]{doc/neurips_2021}

\title{Multi-Objective SPIBB: Seldonian Offline Policy Improvement with Safety Constraints in Finite MDPs}

% ------ update running title also


\author{%
  Harsh Satija\\
  McGill University, Mila \\
  \texttt{harsh.satija@mail.mcgill.ca} \\
  % examples of more authors
   \And
  Philip S. Thomas \\ 
  University of Massachusetts \\ 
  \texttt{pthomas@cs.umass.edu} \\
   \AND
   Joelle Pineau \\
   McGill University, Mila, Facebook AI Research \\
   \texttt{jpineau@cs.mcgill.ca} \\
   \And
   Romain Laroche \\ 
   Microsoft Research \\
   \texttt{romain.laroche@microsoft.com}
  % \And
  % Coauthor \\
  % Affiliation \\
  % Address \\
  % \texttt{email} \\
}


\begin{document}

\maketitle

% ------------------------------------------------------------
%               Abstract
% ------------------------------------------------------------


\begin{abstract}
We study the problem of Safe Policy Improvement (SPI) under constraints in the offline Reinforcement Learning (RL) setting. We consider the scenario where: (i) we have a dataset collected under a known baseline policy, (ii) multiple reward signals are received from the environment inducing as many objectives to optimize. 
We present an SPI formulation for this RL setting that takes into account the preferences of the algorithm's user for handling the trade-offs for different reward signals while ensuring that the new policy performs at least as well as the baseline policy along each individual objective. 
We build on traditional SPI algorithms and propose a novel method based on Safe Policy Iteration with Baseline Bootstrapping~\citep[SPIBB,][]{laroche2017safe} 
that provides high probability guarantees on the performance of the agent in the true environment.
We show the effectiveness of our method on a synthetic grid-world safety task as well as in a real-world critical care context to learn a policy for the administration of IV fluids and vasopressors to treat sepsis.

\end{abstract}



% ------------------------------------------------------------
%               MAIN BODY  
% ------------------------------------------------------------
% INTRODUCTION
\section{Introduction}  \label{sec:introduction}

\newcommand\inexpIntro[3]{#1?(#2,#3).}
\newcommand\rinexpIntro[3]{*#1?(#2,#3).}
\newcommand\outexpIntro[3]{#1!(#2,#3).}
\newcommand\outatomIntro[3]{#1!(#2,#3)}

We propose a fully automated method for proving termination of \(\pi\)-calculus processes.
Although there have been a lot of studies on termination analysis for the \(\pi\)-calculus
and related calculi~\cite{Deng06IC,Demangeon07,SangiorgiTermination,KobayashiHybrid,Yoshida04IC,DBLP:journals/jlp/DemangeonHS10,Venet98SAS}, most of them have been rather theoretical,
and there have been surprisingly little efforts in developing  fully automated termination
verification methods and tools based on them. To our knowledge,
Kobayashi's \typical{}~\cite{TyPiCal,KobayashiHybrid} is the only exception that
can prove termination of \(\pi\)-calculus processes (extended with natural numbers)
fully automatically, but its termination analysis is quite limited (see Section~\ref{sec:relatedwork}).

Our method is based on a reduction to termination analysis for sequential programs:
we translate a \(\pi\)-calculus process \(P\) to a sequential program \(S_P\), so that
if \(S_P\) is terminating, so is \(P\). The reduction allows us to use
powerful, mature methods and tools
for termination analysis of sequential programs~\cite{heizmann2016ultimate,freqterm,DBLP:conf/lics/PodelskiR04,Kuwahara2014Termination,DBLP:journals/cacm/CookPR11}.

The idea of the translation is to convert a chain of communications on replicated input
channels to a chain of recursive function calls of the target sequential program.
Let us consider the following Fibonacci process:
\begin{align*}
    & \rinexpIntro{\fib}{n}{r}
        \ifexp{n<2}{ \soutatom{r}{1} \\ &\quad}
                   { \nuexp{s_1} \nuexp{s_2} (\outatomIntro{\fib}{n-1}{s_1} \PAR \outatomIntro{\fib}{n-2}{s_2} \PAR \sinexp{s_1}{x}\sinexp{s_2}{y}\soutatom{r}{x+y}) \\}
    & \PAR \outatomIntro{\fib}{m}{r}
\end{align*}
Here, the process
$\rinexpIntro{\fib}{n}{r} \ldots$ is a function server that computes the \(n\)-th Fibonacci number
in parallel and returns the result to \(r\),
and $\outatom{\fib}{m}{r}$ sends a request for computing the \(m\)-th Fibonacci number;
those who are not familiar with the syntax of the \(\pi\)-calculus may wish to consult
Section~\ref{sec:targetlanguage} first.
To prove that the process above is terminating for any integer \(m\),
it suffices to show that there is no infinite chain of communications on $\fib$:
\[
    \fib(m,r) \to \fib(m_1,r_1) \to \fib(m_2,r_2) \to \cdots.
\]
We convert the process above to the following program:\footnote{The actual translation
  given later is a little more complex.}
\begin{verbatim}
 let rec fib(n) = if n<2 then () else (fib(n-1) [] fib(n-2)) in
 fib(m)
\end{verbatim}
Here, \texttt{[]} represents the non-deterministic choice.
Note that, although the calculation of Fibonacci numbers is not preserved,
for each chain of communications on \texttt{fib}, there is a corresponding
sequence of recursive calls:
\[
\mathtt{fib}(m) \to \mathtt{fib}(m_1) \to \mathtt{fib}(m_2) \to \cdots.
\]
Thus, the termination of the sequential program above implies the termination of
the original process.
As shown in the example above, (i) each communication on a replicated input channel
is converted to a function call, (ii) each communication on a non-replicated input
channel is just removed (or, in the actual translation, replaced by a call of
a trivial function defined by \(f(\seq{x})=(\,)\)), and (iii) parallel composition
is replaced by a non-deterministic choice.
We formalize the translation outlined above and prove its correctness.

The basic translation sketched above sometimes loses too much information.
For example, consider the following process:
\begin{align*}
    & \rinexpIntro{\pre}{n}{r} \soutatom{r}{n-1} \\
    & \PAR \rinexpIntro{f}{n}{r} \ifexp{n<0}{ \soutatom{r}{1} }
                                       { \nuexp{s} (\outatomIntro{\pre}{n}{s} \PAR \sinexp{s}{x}\outatomIntro{f}{x}{r}) } \\
    & \PAR \outatomIntro{f}{m}{r}
\end{align*}
The translation sketched above would yield:
\begin{verbatim}
  let pred(n) = n-1 in
  let rec f(n) = if n<0 then () else (pred(n) [] f(*)) in
  f(m)
\end{verbatim}
Here, \texttt{*} represents a non-deterministic integer: since we have removed
the input $\sinatom{s}{x}$, we do not have information about the value of \( x \).
As a result, the sequential program above is non-terminating, although the original
process is terminating.
To remedy this problem, we also refine the basic translation above by using a refinement
type system for the \(\pi\)-calculus. Using the refinement type system,
we can infer that the value of \(x\) in the original process is less than \(n\),
so that we can refine the definition of \texttt{f} to:
\begin{verbatim}
 let rec f(n) = ... else (pred(n) [] let x=* in assume(x<n);f(x))
\end{verbatim}
The target program is now terminating, from which
we can deduce that the original process is also terminating.
We have implemented an automated tool based on the refined translation above.

The contributions of this paper are summarized as follows.
\begin{itemize}
\item The formalization of the basic translation from the \(\pi\)-calculus
  (extended with integers) to sequential programs, and a proof of its correctness.
\item The formalization of a refined translation based on a refinement type system.
\item An implementation of the refined translation, including automated refinement type
  inference based on CHC solving, and experiments to evaluate the effectiveness of
  our method.
\end{itemize}

The rest of this paper is structured as follows.
Section~\ref{sec:targetlanguage} introduces the source and target languages
of our translation.
Section~\ref{sec:approach} 
formalizes the basic translation, and proves its correctness.
Section~\ref{sec:refinement} refines the basic translation by using a refinement type system.
Section~\ref{sec:implementation} reports an implementation and experiments.
Section~\ref{sec:relatedwork} discusses related work,
and Section~\ref{sec:conclusion} concludes the paper.


% RELATED WORK       
The industry standard for pose edition is to create rigs, a collection of pieces of software designed to manipulate a character's skeleton. The rig describes the skeleton's bones, how they relate to each other, are constrained in their possible motion and are deformed. These rules are loosely specified and creating a good rig requires a detailed understanding of physics and anatomy, as well as technical and artistic skills. Rigging is thus a time consuming task even for experienced animators, and even more so in large scale productions which often require a different in-depth rig for each character in the cast.
Previous work has helped alleviate this difficulty by providing efficient tools to speed up/and or ease the rigging process, relying on inverse kinematics or data-driven methods.
\subsection{Character pose design}
\subsubsection{Inverse Kinematics (IK)}
IK solvers are a family of methods commonly used in robotics, engineering and computer graphics, in which the parameterization of a kinematic chain is determined from the position of its end effector.
They are a staple tool in pose design software, ensuring the respect of elementary constraints during pose edition. Their de-facto role is to guarantee the length of the limbs, and in some cases to enforce the orientation angle range of a joint.
Many IK solutions have been studied over the years \cite{aristidou_inverse_2018}; usually revolving around approximated linearizations or heuristics. 

Numerical methods require a set of iterations to achieve a satisfactory solution formulated by a cost function to be minimized.
IK solutions can generally be divided into three sub-categories: Jacobian \cite{Siciliano_Handbook_Robot_2007}, Newtonians \cite{cohen_ik_1996} and Heuristics. Most software implement heuristic methods such as Cyclic Coordinate Descent (CCD) \cite{wang_ccd_1991} or 
Forward-Backward Reaching IK (FABRIK) \cite{aristidou_fabrik:_2011} due to their simplicity and extensibility. 

The main drawback of 
these solvers is that they manipulate kinematic chains without taking into account many morphological aspects that make a pose more or less plausible. They offer a first level of help to users but are not sufficient to guarantee a realistic pose. Many joints constraints are dependent on each other and require subjective, human-made approximations.

\subsubsection{Data-driven pose edition}
Data-driven methods offer promising opportunities to solve these approximations. Using real-life data can help in modelling the complex inter-dependencies of skeletons and providing users with smarter edition tools.
While it is still an early field of research, some solutions have been studied. Wu \etal \cite{wu_posing_2009} propose a method for natural character posing from a large motion database. It employs adaptive KD-clustering to select a representative frame from a database and sparse approximations to accelerate training and posing. 
Huang \etal in \cite{Huang_IK_MGDM_2017} present a method based on the formulation of multi-variate Gaussian distribution models (MGDMs), which learn the joint constraints of a kinematic skeleton from motion capture data. 

Some work has also been dedicated to finding new editing interfaces. \modify{}{Instead of the usual setup manipulating joints directly, Guay \etal \cite{guay_line_2013} articulate a framework based on the conceptual "line of action" which describes the overall pose dynamics. They provide a mathematical definition of the line of action, and a interface in which the software modifies the pose to follow a user-provided line. In the same line of though} Garcia \etal \cite{garcia_sketching_2019} propose \modify{a method transforming doodle of trajectories (position and orientation over time) }{a virtual reality-based interface where the user's hands motion (position and orientation over time) are transformed} into sequences of actions and then into detailed character animations using a dataset of parametrized motion clips automatically fitted to the trajectory. 

% ==> DL et Latent Space. 
\subsection{Neural modelling of human motion}
Neural networks have received a great amount of attention over the last decade and shown impressive result in modelling complex data. Human motion has not been spared and deep learning methods have proven their capability of generating realistic motion in a number of difficult cases. 

The literature in neural-based animation include example in user-controlled character navigation \cite{Holden2017} and interactions with the environment \cite{starke_neural_2019}. 
Holden \etal \cite{Holden2020} also show that neural networks can be used to replace parts of existing data-driven methods, improving their scalability potential.
More recently, some work has also focused on improving smaller parts of the animation pipeline rather than replacing it completely. Berson et al. \cite{berson_intuitive_2020} leverage neural networks to provide an interactive system to edit facial animation. 

% Wrap up
Data-driven IK and pose editing can relieve animators from time-consuming, back-and-forth pose adjustments by applying constraints extracted from real-world data. Recently, neural-network-based approaches have demonstrated their ability to model the intricacies of human motion while scaling to large amount of data and retaining a fast inference time. In this paper we seek to take advantage of these properties to create an efficient posing tool, intuitively usable even by a inexperienced user.

% BACKGROUND and METHODOLOGY

% Panoptic segmentation

% 3D segmentation

% Multi-object tracking

% Online 3D panoptic:

% PanopticFusion: (IROS 2019)
% https://arxiv.org/pdf/1903.01177.pdf
%
% - most similar to ours
% - PSPNet + M-RCNN + 2D fusion
% - volumetric mapping, 
% - greedy matching with IoU -> optimal only with 0.5 threshold
% - voxel & class weighting
% - CRF regularisation
%
% - good:
%
% - bad:
%  - CRF post-processing step
%  - greedy data-association
%    - can't be tuned for lower overlap ratios -> has to have high framerate, large changes in viewpoint could break this
%    - IoU: sensitive to 2D labels projecting over object borders (CRF and voxel weighting seem to alleviate this)

% Voxblox++: (Robotics & automation letters 2019)
% https://arxiv.org/pdf/1903.00268.pdf
% https://github.com/ethz-asl/voxblox-plusplus
%
% - M-RCNN + geometric segmentation + fusion 
% - data association of geometric segments with 3D overlap (no. points inside volume), fixed threshold for min number of points
% - instance label is assigned to a segment based on highest overlap
% - only one detected segment per reference label, as in PanopticFusion and Ours
% - TSDF Integration 
%
% good: 
% - because of geometric segmentation objects with no associated semantic class can also be segmented
% bad:
% - two different object segment types -> confusing, overly complicated ?
% - quite inaccurate (fixed below)

% Reconstructing Interactive 3D Scenes by Panoptic Mapping and CAD Model Alignments (ICRA 2021)
% https://arxiv.org/pdf/2103.16095.pdf
% https://github.com/hmz-15/Interactive-Scene-Reconstruction
%
% - based heavily on Voxblox++, much more accurate
% - Scene-graph ("contact graph") for mapping object relations
% - Search & replace voxels with CAD models, with geometrical and physical constraints
% - Object 6D pose
% - Format for robot interaction
%
% - Segmentation: bilateral fusion of geomatric and semantic segments -> reduce segmentation noise compared to Voxblox++
% - Fusion: triplet count improves consistency over Voxblox++ pairwise count strategy (take semantic label into account in addition to instance and geometry)
% - Fusion: instance labels are also combined if there is enough overlap with common geometric label for long enough time
%   - this means multiple detections can match the same reference unlike ours, voxblox++ and PanopticFusion ?
%

% Panoptic-MOPE: (ROBOTICS AND AUTOMATION LETTERS 2020)
% https://ieeexplore.ieee.org/stamp/stamp.jsp?tp=&arnumber=8977356
% https://github.com/hoangcuongbk80/Object-RPE/tree/panoptic-mope
%
% - novel RGB-D semantic segmentation model + M-RCNN
% - camera tracking based on "addaptively weighted optimization of geometric, appearance, and semantic cues"
% - surfel map: 
%   - how does it scale ? authors satate they tested on room-sized environments, but could be applied in larger scale as well ...
%     - could maybe be applied as VO in a SLAM algorithm ...
%   - demo only on a small pallet + surroundings, might not be applicable in large-scale SLAM

% US VS THEM:
%
% - based heavily on PanopticFusion, with modifications:
%   - instead of greedy data-association (which seems to be the case in others as well), we solve LAP (JPDA?)
%     - overlap threshold can be tuned, which renders the algorithm more flexible
%     - could be extended to dynamic tracking ?
%   - multiple options for association likelihood
%   - outlier rejection (either clustering or probabilistic)
%   - test different options for decreasing processing time
%   - no post-processing
%
% - model-agnostic:
%   - completely separated from segmentation
%   - does not care how point clouds are obtained -> applicable for LIDAR segmentation (e.g. EfficientLPS) as well
%
% - also agnostic to localisation method
%   - could, however, be utilised to find landmark locations / poses

% More compact version of this paragraph to introduction to save space?
%Panoptic segmentation -- proposed in \cite{panoptic_segmentation} -- aims to solve the unified task of semantic- and instance segmentation. Semantic classes are separated to \textit{stuff} -- amorphous, unquantifiable regions like sky, road or floor -- and \textit{things} -- quantifiable objects. The distinction between the two can vary depending on the application, but a semantic class can only belong to one or another. The article also proposes a unified panoptic evaluation metric, coined \textbf{Panoptic Quality} (PQ). Many 2D approaches to panoptic segmentation -- \textit{e.g.} \cite{panopticfpn,seamless,panoptic_deeplab,efficientps} -- have since been proposed. Deep neural networks for performing semantic- or instance segmentation directly on the 3D reconstruction -- \textit{e.g.} on \cite{scannet,s3dis,paris_lille_3d} -- have also been proposed, but since they require the reconstructed 3D scene, they are mostly offline approaches and therefore out of scope for this work. Some recent works also apply panoptic segmentation to point clouds -- \textit{e.g.} methods in the SemanticKITTI panoptic segmentation competition \cite{semantic_kitti} -- mostly aimed at segmenting LiDAR output. They are suitable for online processing, but similar to RGB-D images require a method for tracking object instances persistent in both time and space. In fact, our proposed method, as well as some others mentioned in this work, could use segmented LiDAR point clouds as an input similarly to RGB-D images.

PanopticFusion \cite{panopticfusion} is the first work to propose online integration of panoptic image segmentations to a 3D reconstruction. They integrate point clouds generated from segmented images to a TSDF voxel volume \cite{tsdf,voxblox} by greedily matching detected segments with the reconstruction and regulating each voxel's corresponding instance with a weighting function. Semantic labels are inferred in a bayesian manner based on confidence scores provided by the segmentation model. They also apply a Conditional Random Field (CRF) to regularise the reconstruction, improving results significantly. Voxblox++ \cite{voxblox++} -- introduced later the same year -- is a similar approach that also integrates segmented RGB-D images into a TSDF volume. It leverages geometric segmentation of depth images to improve instance segmentation accuracy. Both geometric and semantic segments are used to compute a pair-wise weight, which is used to greedily match them with segments in the reconstruction. Because of the geometric segmentation, the method allows segmentation of objects with no known semantic class in addition to objects recognised by the instance segmentation model. 

Recently, \cite{interactive_3d_scenes} built upon the idea of Voxblox++. They apply Voxblox++ for 3D instance integration, with two small but effective modifications: the pair-wise weight is replaced by a triplet weight that also takes semantic labels into account in the fusion, and -- in addition to geometric segments -- instance segments are fused if they overlap by a significant amount. The article introduces a method for searching and aligning CAD models to reconstructed objects based on geometry and semantic class, as well as geometrical and physical rules. With the CAD models, a contact graph and interactive virtual scene are reconstructed to allow a robot to simulate its interaction with the environment. SceneGraphFusion \cite{scenegraphfusion} is another approach that forms a scene graph online from a stream of RGB-D images, but unlike the above-mentioned approach, it generates the graph with a deep neural network, after which the panoptic labels for geometrically segmented portions of the 3D reconstruction are produced a side product.

Panoptic-MOPE \cite{panoptic_mope} is another recent approach, which integrates sequences of RGB-D images into a surfel reconstruction. Unlike other mentioned approaches -- which assume the camera pose either known or estimated elsewhere -- it also tracks camera movements based on geometric-, appearance- and semantic cues. The method also applies a novel RGB-D panoptic segmentation model. Although it is only tested on room-sized environments, the authors claim it could be scaled to larger environments as well.
% 
\section{The Proposed Method}
\label{sec:Method}
% 开头阐述模型图
% 如图所示,模型的训练分为两阶段,大数据集预训练以及小数据集finetune。在预训练阶段,posterior encoder 以 linear spectrogram作为输入,输出隐变量Z,隐变量Z送入Phoneme predictor。Phoneme predictor 输出Phoneme probability,与Phoneme Look Up Table 相乘得到Phoneme embedding。该embeddding作为歌声合成输入的一部分。
% 2.1 阐述预训练框架
%     标注精良的单歌手歌声数据集例如Opencpop,往往很难有较大的规模,因为标注十分耗费人力。并且由于单歌手的音域固定,所训练出的歌声合成模型很难拥有较广的音域,从而丧失了和真人演唱相比,可能具有的音域优势。为了能够利用大的歌声合成数据集以提高歌声合成系统的音域表现,受xxx文章的启发,我们基于proposed方法的框架,采用了melody-unsupervision在大的歌声合成数据集上做预训练,但又与该文章有着不同。由于我们预训练阶段使用的歌声合成数据集没有详细的标注,仅仅只有text、phonemes,但并没有wav在时间上的详细标注,因此我们希望能够借助ASR的训练方式,通过Phoneme Predictor预测出每一个frame的音素注意力向量 。该音素注意力向量与Phonemes lookup table相乘,最终得到了frame-level的Phonemes embedding。此外,我们从wav中提取出连续的音高,并且将其量化为乐谱音高,经过embedding layer后得到frame-level的pitch embedding。同时由于Opensinger是一个多歌手的数据集,我们使用了基于ECAPA-TDNN的speaker encoder去建模不同歌手的音色信息。
% 2.1.1 Phonemes predictor
% 由于数据集没有Phonemes 的time alignments信息,因此我们采用ASR的训练方法,在pronunciation 层面上使用CTC Loss对Phonemes predictor进行训练。具体而言,Phonemes predictor包含两层FFT Blocks ,一层线性层。线性层将hidden channels 映射到Phonemes 总数对应的类别数。对于线性层的输出,我们取softmax后得到每一个phonemes的概率 p,作为注意力向量与phonemes lookup table相乘得到phonemes embedding。同时我们对p取log,与ground truth标签计算CTC Loss。值得注意的是,与xxx工作不同,由于预训练时没有duration信息,我们的phonemes predictor输出的注意力向量是frame-level的。
% 2.1.2 阐述fine-tune架构
%     finetune架构的搭建整体基于proposed 方法,在proposed 方法的基础上,提出了 d-durator 和bi-flow,提升了我们模型的性能。
% 具体来说,在finetune阶段,模型读取预训练模型的checkpoints,并且在OpenCpop上进行finetune。由于事先在大规模数据集进行了预训练,模型的可合成音域得到了提升。
% 2.2 阐述可微分的上采样模块
% 以前的歌声合成模型,大多采用简单的复制操作将phoneme-level信息转变为frame-level信息,导致模型存在韵律问题。受xxx工作启发,我们提出了可学习的时长预测器,包含一个时长预测器以及可学习的上采样层。时长预测器输出每一个phonemes占发声总时长的比例,该比例与note duration相乘,送入可微分的上采样层。该上采样层 takes a phoneme hidden sequence as input, and outputs a sequence of prior distribution at the frame level。 Compared to simply repeating each phoneme hidden sequence with the predicted duration in a hard way, the differentiable upsampling layer enables more flexible duration adjustment for each phoneme. Also, the differentiable upsampling layer makes the phoneme to frame expansion differentiable, and thus can be jointly optimized with other modules in the TTS system.
% 2.3 阐述双向flow层
% 原先的工作中,flow层在训练阶段将复杂的后验分布映射为简单的先验分布,并且在推理阶段将简单的先验分布转换为复杂的后验分布。但这里存在着training和inference时候的mismatch,也就是 train in backward direction but infer in forward direction。因此我们在训练的时候提出了一个双向的flow模型,对posterior encoder预测的后验分布以及prior encoder预测的先验分布进行转换,并计算loss。值得一提的是,我们发现将flow model两边的kl loss
The training stage of the proposed model consists of two steps: the multi-singer pre-training step and the single-singer fine-tuning step. The architecture of the proposed model is illustrated in Fig.\ref{fig: architecture}, which consists of a prior encoder, a posterior encoder, and a decoder together with a discriminator.
The proposed model is designed from our previous work \cite{zhou22f_interspeech} with the following modifications.
The posterior encoder utilizes a phoneme predictor to predict frame-level phoneme probabilities in the pre-training step. 
The prior encoder adds a speaker encoder to model the timbre variations, replaces the length regulator with a differentiable duration regulator to improve the rhythm naturalness, and upgrades the flow module to be bi-directional to improve the sound quality.

% As illustrated in Fig.\ref{fig: architecture}, the training of the proposed model consists of two stages: the pre-training stage and the fine-tuning stage. 

% In the pre-training stage, the posterior encoder takes the linear spectrogram as input and predicts the latent representation $z$. 
% The phoneme predictor estimates the frame-level phoneme probability $p$ given the latent representation $z$. 
% We multiply $p$ with the phoneme lookup table to get the phoneme embedding. The pitch is extracted from the waveform and we quantified it into note pitch. 

% we reload the checkpoint of the pre-train model and resume training in the OpenCpop\cite{wang2022opencpop} datasets. 


\subsection{The Melody-Unsupervised Multi-Singer Pre-Training Step}
Since the multi-singer training data has no phonemic timing information, in the pre-training step, this work utilizes the automatic speech recognition (ASR) training strategy to train a phoneme predictor in the posterior encoder and predict the frame-level phoneme probabilities $p$.
The probability vectors are multiplied with the phoneme look-up table to obtain the frame-level phoneme embeddings.
In addition, the continuous pitch $f_{0}$ is estimated from the audio and quantized into the note pitch.
The note pitch is passed through the embedding layer to obtain frame-level pitch embeddings.
Moreover, we apply a speaker encoder to extract frame-level speaker embeddings to model the timbre variations of different singers.
Since the pre-training step directly deals with estimated pitch values and focuses on enhancing the vocal range, the pitch predictor, the energy predictor and the duration-related modules are dropped during the pre-training step.


\subsubsection{Phoneme predictor}
We train the phoneme predictor using the connectionist temporal classification (CTC) \cite{graves2006connectionist} loss. 
It contains two layers of FFT blocks and one linear layer.
The linear layer maps the hidden channels to the number of phoneme categories. 
We obtain the probability vector $p$ after taking the softmax operation on the linear layer's output, then multiply it with the phoneme look-up table to get frame-level phoneme embeddings.
Meanwhile, we take the log function of $p$ to compute the CTC loss with the ground truth phoneme sequences. 

\subsubsection{Speaker encoder}
This work adopts one of the state-of-the-art speaker recognition models, i.e. ECAPA-TDNN \cite{desplanques2020ecapa}, as the speaker encoder. Its advanced network architecture and attentive statistics pooling layer have shown great effectiveness in both speaker recognition \cite{desplanques2020ecapa} and voice conversion \cite{guo2022improving,li2022hierarchical}. The speaker encoder is configured as the one with 512 channels in Table 1 of \cite{desplanques2020ecapa}, and it extracts 192-dimensional frame-level speaker embeddings from the audio's Mel-Spectrograms. These embeddings are given as the speaker condition in the multi-singer pre-training step. 

\subsection{The Single-Singer Fine-Tuning Step}
In the fine-tuning step, it loads the pre-trained model parameters, then utilizes the single-singer Opencpop dataset to fine-tune model parameters.
It uses the phoneme and note-pitch annotations provided by the dataset to derive the phoneme and pitch embeddings, instead of using the phoneme probability vectors and quantized f0 values in the pre-training step.
Note that the phoneme and note-pitch annotations are at the phoneme level, rather than the frame level, such that a duration regulator after the note encoder is necessary to up-sample the embedding vectors into the frame level.
As for the speaker embedding, we use the pre-trained speaker encoder to extract an averaged speaker embedding over the Opencpop dataset, then utilize it as a fixed speaker condition during the fine-tuning step.
Moreover, the energy predictor and the pitch predictor join the fine-tuning process to enhance the expressiveness and pitch accurateness of the synthesized samples, following our previous work\cite{zhou22f_interspeech}.
% The fine-tuning architecture is built based on the proposed method \cite{zhou22f_interspeech}, and on top of the proposed method, a differentiable duration predictor and bi-directional flow model are proposed to improve the performance of the synthesized singing voice.
% Specifically, in the fine-tuning phase, the model reloads the checkpoints of the pre-trained model and performs inference on OpenCpop.
The synthesizable vocal range of the model is enhanced due to the multi-singer pre-training on a large-scale dataset.

\subsection{Differentiable Duration Regulator}
Most previous SVS systems simply replicate each phoneme hidden representation with the predicted duration in a hard way, which may degrade the rhythm naturalness.
Inspired by \cite{tan2022naturalspeech}, we leverage a differentiable duration regulator, which contains a duration predictor and a differentiable up-sampling layer. 
The duration predictor outputs the ratio of each phoneme to the corresponding note duration, then the ratio is multiplied by the note duration and fed to the differentiable up-sampling layer.
The differentiable up-sampling layer leverages the predicted duration to learn a projection matrix to extend the phoneme hidden sequence from the phoneme level to the frame level.
It makes the phoneme-to-frame expansion differentiable and thus can be jointly optimized with other modules in the system.

\subsection{Bi-directional Flow}
In the previous work \cite{zhou22f_interspeech}, the flow model maps the complex posterior distribution to the simple prior distribution in the training stage while operating reversely in the inference stage.
This process suffers from the mismatch problem between the training and inference stages.
Therefore, we leverage a bi-directional flow module\cite{tan2022naturalspeech} during training, which bridges the complex posterior distribution and the simple prior distribution bi-directionally to alleviate the mismatch issue in the inference stage.
% not only maps the complex posterior distribution to the simple prior distribution during training, but also maps the simple prior distribution to the complex posterior distribution, in order to improve the quality of the synthesized singing voice.
It is worth noting that we observed that the system can easily fail to train and encounter gradient explosion when the KL losses on both sides of the flow contribute equally, so we define the reverse KL loss weight as 0.5.






% SYNTHETIC EXPERIMENTS
% -------------------------------------------------------
%               Tabular experiments
% -------------------------------------------------------
\section{Synthetic Experiments}
\label{sec:synthetic-experiments}

The main benefits of working in a synthetic domain are: (i) we can evaluate the performance on the true MDP instead of relying on off-policy evaluation (OPE) methods, (ii) we have control over the quality of the dataset. We test both 
MO-SPIBB (\ref{eq:s-opt}) and MO-HCPI (\ref{eq:h-opt}) on a variety of parameters: the amount of data, quality of baseline and different user reward scalarizations. 

% ------------ Env description ---------------
\textbf{Env details:} 
We take a standard CMDP benchmark \citep{leike2017ai, chow2018lyapunov} which consists of a $10\times10$ grid. From any state, the agent can move to the adjoining cells in the 4 directions using the 4 actions. 
The transitions are stochastic, with some probability $\alpha$ (generated randomly for each state-action for every environment instance) the agent is successfully able to reach the next state, and with $(1-\alpha)$ the agent stays in the current state. 
The agent starts at the bottom-right corner, and the goal is to reach the opposite corner (top-left). The pits are spawned randomly with some uniform probability ($\eta_{pit}=0.3$) for each cell.  
The reward vector consists of two rewards signals. A primary reward $r_0$ that is related the goal and is +1000.0 on reaching the goal and -1.0 at every other time-step. The secondary reward $r_1$ is related to pits, for which the agent gets -1.0 for any action taken in the pit.
The constraint threshold for this CMDP is $-2.0$ and $\gamma = 0.99$. Maximum length of an episode is $200$ steps. Therefore, the task objective is to reach the goal in the least number of steps, such that the agent does not spend more than $2$ time-steps in the pit cells. 

% ------------ Dataset generation ---------------
\textbf{Dataset collection procedure:} 
For every random CMDP generated, we first find the optimal policy $\piopt$ by using the procedure described in \Cref{app:cmdp-solver}. The baseline policy is generated using a convex combination of the optimal policy and a uniform random policy ($\pi_{rand}$), i.e., $\pib = \rho \piopt + (1 - \rho) \pi_{rand}$, where  $\rho$ controls how close $\pib$'s performance is to $\piopt$. Different datasets with varying sizes and $\rho$ are then collected under $\pib$ and given as input to the methods.

% ------------ Baselines ---------------
\textbf{Baselines:} 
We compare against the following baselines:
\begin{itemize}[leftmargin=*, topsep=0pt]
    \item \myuline{Linearized}: This baseline transforms the rewards into a single scalar using $\bml$ and then applies the traditional policy improvement methods on the linearized objective, i.e,  $\argmax_{\pi \in \Pi} \J{\pi}{\bml}{\mhat}$.
    
    % \item \myuline{Adv-Linearized}: This method has the same objective as the Linearized baseline, with the additional constraints based on advantage estimators built from $\mhat$:
    % \begin{align}
    %     \argmax_{\pi \in \Pi} &\langle \pi(\cdot|x) \qval{\pi}{\bml}{\mhat}{x,\cdot} \rangle \quad  \forall x \in \X \\  
    %     \text{s.t.} \quad    
    %     &\forall i\in [d], \; \sum_{a \in \A} \pi(a|x) \adv{\pib}{i}{\mhat}{x, a} \geq 0 \nonumber. 
    % \end{align}
    \item \myuline{Adv-Linearized}: This method has the same objective as the Linearized baseline, with the additional constraints based on advantage estimators built from $\mhat$, i.e. $\forall x \in \X$:
    % \begin{align}
    %     \argmax_{\pi \in \Pi} \langle \pi(\cdot|x) , \qval{\pi}{\bml}{\mhat}{x,\cdot} \rangle \quad   
    %     &\text{s.t.} \quad    
    %     \forall k\in [d], \; \sum_{a \in \A} \pi(a|x) \adv{\pib}{k}{\mhat}{x, a} \geq 0 .
    % \end{align}
    \begin{align}
        \argmax_{\pi \in \Pi} &\langle \pi(\cdot|x) , \qval{\pi}{\bml}{\mhat}{x,\cdot} \rangle \\   
        \text{s.t.} \quad    
        &\forall k\in [d], \; \sum_{a \in \A} \pi(a|x) \adv{\pib}{k}{\mhat}{x, a} \geq 0 . \nonumber
    \end{align}
    
    
    
\end{itemize}


% ------------ Evaluation ---------------
\textbf{Evaluation:} 
Using $\mopt$, we can directly calculate the returns for any solution policy. Only tracking the scalarized objective can be misleading, so we track the following metrics:
\begin{itemize}[leftmargin=*, topsep=0pt,]
    \item \myuline{Improvement over $\pib$}: This denotes the difference between the scalarized return of the solution policy and the baseline policy, i.e., $\J{\pi}{\bml}{\mopt} - \J{\pib}{\bml}{\mopt}$.
    Mean improvement over $\pib$ captures on average improvement over $\pib$ in terms of the scalarized objective.
    
    \item \myuline{Failure-rate:} 
    The failure rate over $n$ runs captures the number of times, on average, the solution policy ends up violating the safety constraints in \Cref{eq:general-safety-constraints}, and thus performs worse than the baseline. In the context of this task, safety constraints are violated if either the agent takes longer to reach the goal, or it steps into more number of pits compared to $\pib$.
\end{itemize}

We test on different combinations of user preference $(\bml)$ and baseline's quality $(\rho)$ on 100 randomly generated CMDPs, where $\lambda_i \in \{0, 1\}$, $\rho \in \{0.1, 0.4, 0.7, 0.9\}$ and %the number of trajectories 
$|D| \in \{ 10, 50, 500, 2000\}$.
We evaluate under two settings: 
(i) we use a fixed set of parameters across different $(\bml, \rho)$ combinations, where we run \ref{eq:s-opt} with $\epsilon \in \{0.01, 0.1, 1.0\}$ and \ref{eq:h-opt} with Doubly Robust IS estimator \citep{jiang2015doubly} and Student’s t-test concentration inequality; 
(ii) we treat them as hyper-parameters that can be optimized for a particular $(\bml,\rho)$ combination. The best hyper-parameters are tuned in a single environment instance and then they are used to benchmark the results on 100 random CMDPs. 
% More details about the range of hyper-parameters considered are given in \Cref{app:cmdp-best-param-results}.


% ------------ Combined figure ---------------

% \begin{figure}[t]
% \makebox[1\linewidth][c]{%
% \centering
% % \begin{subfigure}[b]{0.48\columnwidth}
% \begin{subfigure}[b]{0.5\textwidth}
%     \includegraphics[width=1\textwidth]{doc/figures/random-mdps/delta_01_latex_nf.pdf}
%     \caption{Fixed hyper-parameters}
%     \label{fig:delta-params-mean} 
% \end{subfigure}
% \hfill
% \begin{subfigure}[b]{0.5\textwidth}
%     \includegraphics[width=1\textwidth]{doc/figures/random-mdps/bench_latex_nf.pdf}
%     \caption{Optimized hyper-parameters.}
%     \label{fig:best-params-mean}
% \end{subfigure}
% }
% \caption[]{
% \small
% Results on 100 random CMDPs for different $\bml$ and $\rho$ combinations with $\delta=0.1$. The different agents are represented by different markers and colored lines. Each point on the plot denotes the mean (with standard error bars) for 12 different $\bml,\rho$ combinations for the 100 randomly generated CMDPs (1200 datapoints). 
% The x-axis denotes the amount of data the agents were trained on. 
% The y-axis for left subplot in each sub-figure represents the improvement over baseline and the right subplot denotes the failure rate. The dotted black line in the right subplots represents the high-confidence parameter $\delta=0.1$.
% \Cref{fig:delta-params-mean} denotes when the hyper-parameters are fixed $\epsilon=\{0.01, 0.1, 1.0\}$ and $\IS=$ Doubly Robust (DR) estimator with student's t-test concentration inequality. 
% \Cref{fig:best-params-mean} is the version with tuned hyper-parameters for each combination.
% \label{fig:cmdp-combined-results}}
% \vskip -0.1in
% \end{figure}

\begin{figure}[t]
\centering
% \begin{subfigure}[b]{0.48\columnwidth}
\begin{subfigure}[b]{0.7\textwidth}
    \includegraphics[width=1\textwidth]{doc/figures/random-mdps/delta_01_latex_nf.pdf}
    \caption{Fixed hyper-parameters}
    \label{fig:delta-params-mean} 
\end{subfigure}
\hfill
\begin{subfigure}[b]{0.7\textwidth}
    \includegraphics[width=1\textwidth]{doc/figures/random-mdps/bench_latex_nf.pdf}
    \caption{Optimized hyper-parameters.}
    \label{fig:best-params-mean}
\end{subfigure}
\caption[]{
\small
Results on 100 random CMDPs for different $\bml$ and $\rho$ combinations with $\delta=0.1$. The different agents are represented by different markers and colored lines. Each point on the plot denotes the mean (with standard error bars) for 12 different $\bml,\rho$ combinations for the 100 randomly generated CMDPs (1200 datapoints). 
The x-axis denotes the amount of data the agents were trained on. 
The y-axis for left subplot in each sub-figure represents the improvement over baseline and the right subplot denotes the failure rate. The dotted black line in the right subplots represents the high-confidence parameter $\delta=0.1$.
\Cref{fig:delta-params-mean} denotes when the hyper-parameters are fixed $\epsilon=\{0.01, 0.1, 1.0\}$ and $\IS=$ Doubly Robust (DR) estimator with student's t-test concentration inequality. 
\Cref{fig:best-params-mean} is the version with tuned hyper-parameters for each combination.
\label{fig:cmdp-combined-results}}
\vskip -0.1in
\end{figure}


% ------------ Results ---------------
\textbf{Results:} 
The mean results with fixed parameters and $\delta=0.1$ can be found in \Cref{fig:delta-params-mean}. 
% Tell changes with data size for the condensed figure
The high failure rate of Linearized baseline, regardless of the size of the dataset, is expected as it optimizes the scalarized reward directly and is agnostic of the individual rewards. Adv-Linearized performs better, but in the low data-regime, we see a high failure rate that eventually decreases as the size of dataset increases. This is expected because with more data, more reliable advantage functions estimates are calculated that are representative of the underlying CMDP. 
Compared to the baselines, both \ref{eq:s-opt} and \ref{eq:h-opt} maintain a failure rate below the required confidence parameter $\delta$, regardless of the amount of data.
Also, as the size of dataset increases, we see an increase in improvement over $\pib$, that makes sense as the methods only deviate from baseline when they are sure of the performance guarantees. We expect \ref{eq:s-opt} to violate the constraints with increasing value of $\epsilon$, as it relaxes the constraint on the policy-class (\Cref{eq:spibb-policy-constraint}) and leads to a looser guarantee on performance. This again is reflected in our experiments where \ref{eq:s-opt} with $\epsilon=1.0$ has a higher failure-rate than $\epsilon=0.1$.
We observed similar trends for different $\delta$ values.
A more detailed plot corresponding to different $\bml$ and $\rho$ combinations as well as results for a riskier value of $\delta=0.9$ are given in \Cref{app:cmdp-fixed-param-results}.


The results with optimized hyper-parameters can be found in \Cref{fig:best-params-mean}.
We notice that when the $\epsilon$ parameter is tuned properly, \ref{eq:s-opt} has better performance in terms of improvement over $\pib$ for the same amount of samples when compared to \ref{eq:h-opt}, while still ensuring the failure rate is less than $\delta$. These observations are consistent with the results in the single-objective setting in the original SPIBB works~\citep{laroche2017safe, nadjahi2019safe}.
The general trends and observations from the fixed-parameter case are also valid here.  Additional details, including results for $\bml, \rho$ combinations, hyper-parameters considered and qualitative analysis can be found in \Cref{app:cmdp-best-param-results}. 

We also compare our methods against \cite{le2019batch} in \Cref{app:lag-baseline}. We show the advantage of our approach over \cite{le2019batch}, particularly in the low-data regime, where our methods can improve over the baseline policy while ensuring a low failure rate. 
% The method in \cite{le2019batch} exhibits similar trends to Adv-Linearized baseline, where in the low data setting it has high-failure rate, which decreases as the size of dataset decreases. 
This makes sense as the method in \cite{le2019batch} relies on the concentrability coefficient which can be arbitrarily high in the low data setting, and therefore their performance guarantees do not hold anymore.
We also provide experiments on the scalability of methods with the number of objectives $d$ in \Cref{app:cmdp-scaling-experiments}.


% HEALTHCARE EXPERIMENTS
% \pagebreak

\section{Real-world experiment}
\label{sec:sepsis-experiments}


% --------
% Give the brief task description (SEPSIS),  the context of RL for sepsis,
% --------
In order to validate the applicability of our methods on a real-world  task, 
we consider recent works on %the improvement of 
sepsis management via RL, where we only have access to a pre-collected patient dataset and goal is to recommend treatment strategies for patients with sepsis in the ICU \citep{komorowski2018artificial,tang2020clinician}.
% \citep{raghu2017deep, komorowski2018artificial,tang2020clinician}.
Sepsis is defined as a life-threatening organ dysfunction caused by a dysregulated host response to an infection \citep{singer2016third}.
The main treatment method of sepsis involves the repeated administration of intravenous (IV) fluids and vasopressors, but how to manage their appropriate doses at the patient level is still a key clinical challenge \citep{rhodes2017surviving}. 
%\citep{rhodes2017surviving, byrne2017fluid}. 


% --------
% Here how we aim to use this Sepsis task
% --------
% The problem is safety-critical as our methods need to be cautious about using the data that was possibly collected under unobservable confounders and other biases that can lead to biased model estimates. 
% For instance, a study by \citet{ji2020trajectory} of the model used in \citet{komorowski2018artificial} found that the learned model suffers from two major kinds of limitations when it comes to clinically implausible behavior.
% The first kind of challenges are related to the inaccurate modelling assumptions such as discretization of time, and are not the focus of this work. 
% The second aspect, where we believe our methodology can help, is to prevent unexpectedly aggressive treatments resulting from small sample sizes.
% We propose to do so by incorporating safety constraints to prevent recommending the treatments decisions that were never or rarely performed in the dataset. 
% 

The problem is safety-critical as our methods need to be cautious about using the data that was possibly collected under unobservable confounders and that can lead to biased model estimates. 
For instance, a study by \citet{ji2020trajectory} of the model used in \citet{komorowski2018artificial} found that the learned model suggests clinically implausible behavior in the form of unexpectedly aggressive treatments.
We show that our methodology can be applied here to prevent such behavior that results from small sample sizes. We propose to do so by incorporating safety constraints to prevent recommending the treatment decisions that were never or rarely performed in the dataset. 

% against such propose to do so by incorporating safety constraints to prevent recommending the treatments decisions that were never or rarely performed in the dataset. 

% We believe our methodology here can help to prevent clinically implausible behavior that
% We believe our methodology can help prevent unexpectedly aggressive treatments resulting from small sample sizes.



% --------
% Data/Cohort 
% --------
% \textbf{Data and Cohort design:} 
% We use the publicly available ICU dataset MIMIC-III  \cite{johnson2016mimic}, with the setup described by \citet{komorowski2018artificial, tang2020clinician} and build on top of their data pre-processing methodology
% % The cohort is defined by adults fulfilling the sepsis-3 criteria \citep{singer2016third}
% , that includes prescription of antibiotics, lab work of bodily fluids and a Sequential Organ Failure Assessment (SOFA) score $\geq 2$ \citep{singer2016third}. 
% After applying the exclusion criteria \citep{komorowski2018artificial} we are left with a cohort of 20,954 unique patients. 
% --------
% Methodology recap 
% --------
% \textbf{MDP construction:} 
% We follow the MDP construction procedure by \citet{komorowski2018artificial, tang2020clinician} and very briefly describe the methodology here

\textbf{Data and MDP Construction:} 
We use the publicly available ICU dataset MIMIC-III  \citep{johnson2016mimic}, with the setup described by \citet{komorowski2018artificial, tang2020clinician} and build on top of their data pre-processing and MDP construction methodology.\footnote{A caveat here is regarding the underlying assumption that the MDP construction methodology by \citet{komorowski2018artificial, tang2020clinician} maintains the Markovian property in the discretized state-space.} 
This leaves us with a cohort of 20,954 unique patients. 
% The patient data is discretized into 4-hour windows, each of which is pre-processed to be treated as a single time-step.
The state-space consisting of 48 clinical variables summarizing features like demographics, physiological condition, laboratory values, etc., is discretized using a k-means based clustering algorithm to map the states to 750 clusters.
The actions include administration of IV fluids and vasopressors, which are categorized into 5 dosage bins each, leading to a total of $|\A|=25$. The $\gamma$ is set to $0.99$. The reward is based on patient mortality. The agent gets a reward, $r_0$, of $\pm 100$ at the end of the episode based on the survival of the patient. More details can be found in \Cref{app:sepsis-dataset}.


In the original work, the rare state-actions taken by the clinicians (state-action pairs observed infrequently in the training set) are removed from the dataset. Instead of removing them,
% in the pre-processing, 
we define an additional reward, $r_1$, based on the rarity of the state-action pair. We define rare state-action pairs to be those that are taken less than 10 times throughout training dataset, and the agent gets a reward of $-10$ for every such rare state-action taken, i.e.,  $r_1(x,a) = -10.0 \text{ if } \texttt{Count}(x,a) < 10$.
The final task objective then becomes to suggest treatments that handles the trade-off between prioritizing improving the survival vs prioritizing commonly used treatment decisions.


% --------
% Evaluation 
% --------
\textbf{Evaluation:} 
We compare our approach with the same baselines from \Cref{sec:synthetic-experiments} on different $\bml$ combinations.  
We run our methods for 10 runs with different random seeds, where for each run the cohort dataset was split into train/valid/test sets in the ratios of 0.7/0.1/0.2.
We evaluate the performance of the solution policies returned by different methods on the test sets using two different OPE methods, Doubly Robust (DR) \citep{jiang2015dependence} and Weighted Doubly Robust (WDR) \citep{thomas2016data}.
We acknowledge that these methods are a proxy of the actual performance of the deployed policies. Hence, these results should not be misinterpreted as us claiming that the policies returned by our methods are now ready to be used in the ICU. 
% and much more details would be needed before quantifying a policy's performance in the actual clinical setting. Hence, 

% -- Result table

\begin{table*}[ht!]
\centering
\caption{Performance of various methods using DR and WDR estimators with mean and standard deviation on 10 random splits of the cohort dataset. The red cells denote the corresponding safety constraint violation, i.e, either $\mathcal{J}_{0}^{\pi} < \mathcal{J}_{0}^{\pib}$ or $-\mathcal{J}_{1}^{\pi} > -\mathcal{J}_{1}^{\pib}$.}
\label{table:sepsis-best-results}
\begin{adjustbox}{max width=1\textwidth,center}
\begin{tabular}{cccccc}
\toprule
\multicolumn{1}{c}{User preference $(\bml)$} & \multicolumn{1}{c}{Policy} & \multicolumn{2}{c}{Survival return ($\mathcal{J}_0$)} & \multicolumn{2}{c}{Rare-treatment return ($- \mathcal{J}_1$)} \\
\hline
& & DR & WDR & DR & WDR  \\  \cline{3-6}
& Clinician's ($\pib$) & 64.78 $\pm$ 0.90 & 64.78 $\pm$ 0.90          & 13.58 $\pm$ 0.19 & 13.58 $\pm$ 0.19  \\
\midrule % \hline \hline
\multirow{4}{*}{$[\lambda_0=1, \lambda_1 = 0]$} 
& Linearized & 97.68 $\pm$ 0.22 & 97.58 $\pm$ 0.20   & \textcolor{red}{27.64 $\pm$ 1.11 }& \textcolor{red}{27.84 $\pm$ 1.09 } \\ 
& Adv-Linearized  & 91.62 $\pm$ 0.46 & 92.68 $\pm$ 0.23   & \textcolor{red}{15.18 $\pm$ 0.59 }& 13.56 $\pm$ 0.42 \\
& \ref{eq:s-opt}   & 66.11 $\pm$ 0.87 & 66.05 $\pm$ 0.86   & 13.42 $\pm$ 0.20 & 13.46 $\pm$ 0.20   \\
& \ref{eq:h-opt} & 65.95 $\pm$ 0.00 & 65.95 $\pm$ 0.00   & 13.37 $\pm$ 0.00 & 13.37 $\pm$ 0.00  \\
\midrule 
\multirow{4}{*}{$[\lambda_0=1, \lambda_1 = 1]$}
& Linearized & 87.17 $\pm$ 0.48 & 89.11 $\pm$ 0.37   & 2.41 $\pm$ 0.47 & 1.52 $\pm$ 0.41\\
& Adv-Linearized  & 86.77 $\pm$ 0.49 & 88.58 $\pm$ 0.25   & 2.53 $\pm$ 0.50 & 1.57 $\pm$ 0.43  \\
& \ref{eq:s-opt}  & 86.77 $\pm$ 0.49 & 88.58 $\pm$ 0.25   & 2.53 $\pm$ 0.50 & 1.57 $\pm$ 0.43   \\
& \ref{eq:h-opt} & 86.37 $\pm$ 0.00 & 88.03 $\pm$ 0.00   & 2.58 $\pm$ 0.00 & 1.43 $\pm$ 0.00  \\
\midrule 
\multirow{4}{*}{$[\lambda_0=0, \lambda_1 = 0]$}
& Linearized & \textcolor{red}{-89.39 $\pm$ 0.43} & \textcolor{red}{-90.90 $\pm$ 0.29 }  & \textcolor{red}{22.99 $\pm$ 0.40 }& \textcolor{red}{22.81 $\pm$ 0.30 }  \\ 
& Adv-Linearized  & \textcolor{red}{60.27 $\pm$ 0.49} & \textcolor{red}{61.44 $\pm$ 0.85 }  & \textcolor{red}{18.40 $\pm$ 0.27 }& \textcolor{red}{15.36 $\pm$ 0.58 }  \\
& \ref{eq:s-opt} & 67.73 $\pm$ 0.82 & 67.22 $\pm$ 0.88   & 13.24 $\pm$ 0.24 & 13.55 $\pm$ 0.33  \\
& \ref{eq:h-opt} & 65.95 $\pm$ 0.00 & 65.95 $\pm$ 0.00   & 13.37 $\pm$ 0.00 & 13.37 $\pm$ 0.00  \\
\midrule %\hline \hline
\multirow{4}{*}{$[\lambda_0=0, \lambda_1 = 1]$}
& Linearized & \textcolor{red}{58.27 $\pm$ 2.18} & \textcolor{red}{60.52 $\pm$ 2.07 }  & 0.04 $\pm$ 0.03 & 0.02 $\pm$ 0.01  \\ 
& Adv-Linearized  & 76.05 $\pm$ 0.65 & 76.85 $\pm$ 0.72   & 0.07 $\pm$ 0.05 & 0.04 $\pm$ 0.03  \\
& \ref{eq:s-opt}  & 76.07 $\pm$ 0.65 & 76.87 $\pm$ 0.73   & 0.07 $\pm$ 0.05 & 0.04 $\pm$ 0.03  \\
& \ref{eq:h-opt} & 76.54 $\pm$ 0.00 & 77.55 $\pm$ 0.00   & 0.09 $\pm$ 0.00 & 0.05 $\pm$ 0.00  \\
\bottomrule 
\end{tabular}
\end{adjustbox}
\vskip -0.1in
\end{table*}


% --------
%  Results
% --------
\textbf{Results:}
We refer to the return associated with the mortality reward ($r_0$) as survival return ($\mathcal{J}_{0}$), and the negative return associated with rare state-action reward ($r_1$) as rare-treatment return ($- \mathcal{J}_{1}$). Higher survival return implies more successful discharges, and lower rare-treatment return implies more adherence to common practice treatment decisions.
We present the results on survival and rare-treatment returns 
% with mean and standard deviation for the 10 runs 
in \Cref{table:sepsis-best-results}. 
% As expected, we observe the Linearized baseline violates most of the constraints across different $\bml$. The Adv-Linearized baseline performs better than Linearized, but still ends up violating some constraints, possibly due to unreliable estimates.
% We observe the Linearized baseline violates most of the constraints across different $\bml$. The Adv-Linearized baseline performs a bit better, but still ends up violating some constraints, possibly due to unreliable estimates.
% We expect both \ref{eq:s-opt} and \ref{eq:h-opt} to respect the safety constraints 
% irrespective of the $\bml$, and indeed they are able to do so.
As expected, we observe both the Linearized and Adv-Linearized baselines violates constraints across different $\bml$, whereas \ref{eq:s-opt} and \ref{eq:h-opt} are able to respect the safety constraints irrespective of the $\bml$.\footnote{In \Cref{table:sepsis-best-results}, $\bml =[1,1]$ represents a rare case of reward scalarization that allows all the methods to find a good solution policy that satisfies the constraints.  In general, it is difficult to find such scalarization parameters as seen in synthetic experiments (\Cref{app:cmdp-fixed-param-results}).}
The validation set was used to tune the hyper-parameters, and we report how the performance varies with different hyper-parameters in \Cref{app:sepsis-hyperparams}. 




% --------
% Qualitative Results 
% --------
\textbf{Qualitative Analysis:} 
We conclude with a qualitative analysis of the policies returned from our setting and the traditional RL approach of maximizing just the survival return. 
% We calculate how many rare-actions are recommended by different solution policies and compare them with the most common actions taken by the clinicians.
% For each state, for the action recommended by a solution policy, we calculate the frequency with which that state-action was observed in the training data and calculate the percentage of time that state-action pair was observed among all the possible actions taken from that state.
% Across all the states, the actions suggested by the traditional single-objective RL baseline are observed only 3\% of the time on average (5.3 observations per state). Whereas, the actions most commonly chosen by the clinicians  are observed 51.4\% of the time on average (138.2 observations per state). We study this behavior for two of the policies returned by MO-SPIBB that deviate the most from the baseline: for the policy returned by \ref{eq:s-opt} ($\bml=[1,1]$) the recommended actions are observed 24.8\% of time on average (61.0 observations per state) and for  \ref{eq:s-opt} ($\bml=[0,1]$) the recommended actions are observed 23.4\% of times (56.14 observations per state).
%
\citet{ji2020trajectory} found that the RL-policies for sepsis-management task usually end up recommending aggressive treatments, particularly high vasopressor doses for states where the common practice 
(according to most frequent action chosen by the clinician for that state) 
is to give no vasopressors at all. The common practice involves giving zero vasopressors for 722 of the 750 states. However, the policy returned by the traditional single-objective RL baseline recommends vasopressors in 562 (77.84\%) of those 722 states, with 295 of those recommendations being large doses
% , where large doses are defined as the dosages belonging in the upper 50th percentile of nonzero amounts 
(upper 50th percentile of nonzero amounts  or $>0.2$ $\mu$g/kg/min). 
We compare these statistics for two of the policies returned by MO-SPIBB that deviate the most from $\pib$. 
The policy returned by \ref{eq:s-opt} ($\bml=[1,1]$) recommends vasopressors in only 93 of those states (12.88 \%), with 47 of those recommendations belonging to high dosages. The other policy, \ref{eq:s-opt} ($\bml=[0,1]$), recommends vasopressors in 134 (18.56 \%) of those states and 70 of those recommendations fall in large dosages.
Therefore, the policies returned by our approach, even when they deviate from the baseline, are less aggressive in recommending rare treatments. 
In \Cref{app:sepsis-qual-analysis}, we present an additional qualitative analysis that demonstrates our methods recommend lesser rare-action treatments than the traditional single-objective RL approach.

% This is on argument against why include rare-action as costs.
An argument can be made against the case when all rare state-action pairs are removed from the training data itself. This will ensure that any learned policy will have near 0 rare-treatment return. However, it is not always clear how to define the cut-off criteria for rare-actions, and it might be possible that some of these rare state-action pairs are actually crucial for finding a better policy. 
For instance, we did an experiment where we assigned state-actions pairs with frequency $<100$ to be rare state-action pairs and filtered those from the training set. The clinician's performance on the test set using a DR estimator for survival return is 65.95. % (and $C$: 59.63).
In this case, the traditional single-objective RL baseline gives the survival return of 11.26, %(and $C$: 18.40), 
which shows that removing such transitions from the dataset actually hampers the solution quality. Our approach of assigning a separate reward for rare state-action pairs is able to find a solution with a survival return of 86.75 %(and $C$: 24.66)
even in this scenario.






% CONCLUSION
In this paper, 2D and 3D CNN models were used to generate pelvic sCTs from T1-weighted MR images. Our sCT generation methods were fully automated, requiring no deformable registration or manual segmentation of bone tissues. As shown in Figure~\ref{fig3}, the 2D and 3D CNN models generated high quality sCTs. MAE curves shown in Figure~\ref{fig4} indicated that both models could precisely estimate soft-tissue HU values but had difficulty in reproducing air and high-density bone tissues. 

The MAEs within the body contour across all patients were 40.5 $\pm$ 5.4 HU and 37.6 $\pm$ 5.1 HU for the 2D and 3D models, respectively. The time required for generating a pelvic sCT using our CNN models was about 5.5 s. Our MAE results are comparable to previous studies. Kim $et \ al.$\cite{RN41} presented a voxel-based weighted summation method that produced an MAE of 74.3 $\pm$ 3.9 HU. However, manual contouring of bone tissues required for this method can be tedious and time-consuming. An MAE of 40.5 $\pm$ 8.2 HU was achieved by Dowling $et \ al.$\cite{RN11} using an average MRI-CT atlas from 38 patients. Andreasen $et \ al.$\cite{RN42} reported an MAE of 54 $\pm$ 8 HU using an atlas-based method with pattern recognition, and its prediction time was about 20.8 min. Another random forest model proposed by Andreasen $et \ al.$\cite{RN43} generated sCTs with an MAE of 58 $pm$ 9 HU. A hybrid method suggested by Siversson $et \ al.$ \cite{RN45} obtained an MAE of 36.5 $\pm$ 4.1 HU when ignoring errors introduced by gas cavities. This hybrid method was implemented in the cloud-based commercial software MriPlanner (Spectronic Medical AB, Helsingborg, Sweden), which required 50 to 80 min to generate a sCT.\cite{RN45} The patch-based 3D context-aware generative adversarial network presented by Nie $et \ al.$\cite{RN26} achieved an MAE of 39.0 $\pm$ 4.6 HU. 

Our CNN models reproduced low-density bone as shown in Figure ~\ref{fig4}. The bone-region DSCs were 0.81 $\pm$ 0.04 and 0.82 $\pm$ 0.04 from the 2D and 3D models, respectively. These results are comparable to reported DSC results of 0.79 $\pm$ 0.12\cite{RN10} and 0.91$\pm$0.03{\cite{RN11}}, where the authors compared bone contours manually drawn on the sCT and CT.

It was feasible to train the proposed 3D model with 16 image volumes from scratch. Results of the Wilcoxon signed-rank tests shown in Table~\ref{tab1} demonstrated a statistically significant improvement in overall MAE, bone DSC, and bone precision of the 3D model compared to the 2D model. However, as shown in Figure~\ref{fig4}, the 2D model seemed to perform better in estimating the high-density bone HU values. It should be noted that smaller overall MAEs do not guarantee improved sCT dose calculation and patient positioning performance. While the models performed well, we will continue to acquire more patient data to potentially improve model accuracy and further test model differences.

As this was a retrospective study, the MR image voxel sizes were not matched, resulting in different voxel intensities between images. This may have affected the sCT generation accuracy although we applied intensity normalization. A potential study could examine how voxel size variations affects sCT estimation. 

The proposed 3D model can be implemented on a 12 GB GPU to process volumetric images with dimensions of 256 $\times$ 256 $\times$ 30. More GPU memory would be required to process higher resolution 3D images. Considering the limited access to multi-GPU systems, a 3D architecture with fewer convolutional layers could be considered to deal with higher resolutions. However, the performance could be affected by the reduced parameters and smaller receptive fields of the less complex model. Another approach would be to extract 30-slice sub-volumes from CT and MR images for training the 3D model. The sCT could then be generated by averaging 30-slice sCT sub-volumes produced by the model. 

A number of techniques could be investigated for improving model performance.  Nie $et \ al.$\cite{RN26} showed that introducing an additional adversarial discriminator improved overall sCT quality. The same approach could be adapted in our proposed 2D and 3D CNN models.  Non-rigid deformation\cite{RN44} could also be applied to both CT and MR images in the process of the on-the-fly data augmentation to produce more training pairs. Multiple MR images acquired with different sequences could be fed into models to provide more information for distinguishing different tissues. Multi-GPU systems with more memory would enable the exploration of larger batch sizes for training CNN models, which could reduce variances in gradient estimation and accelerate the training. 



% ACKNOWLEDGEMENTS
\section{Acknowledgements}

Luca Herranz-Celotti was supported by the Natural Sciences and Engineering Research Council of Canada through the Discovery Grant from professor Jean Rouat, and by CHIST-ERA IGLU. We thank Compute Canada for the clusters used to perform the experiments and NVIDIA for the donation of two GPUs. We thank Wolfgang Maass for the opportunity to visit the Institute of Theoretical Computer Science, Guillaume Bellec, Darjan Salaj and Franz Scherr, for their invaluable insights on learning with surrogate gradients, and Maryam Hosseini, Ahmad El Ferdaoussi and Guillaume Bellec for their feedback on the article.

% ------------------------------------------------------------
%               BIBLIOGRAPHY 
% ------------------------------------------------------------

% \bibliographystyle{plainnat}
\bibliographystyle{apalike}
\bibliography{doc/lib}



%%%%%%%%%%%%%%%%%%%%%%%%%%%%%%%%%%%%%%%%%%%%%%%%%%%%%%%%%%%%
\section*{Checklist}

\begin{enumerate}

\item For all authors...
\begin{enumerate}
  \item Do the main claims made in the abstract and introduction accurately reflect the paper's contributions and scope?
    \answerYes{See \Cref{sec:spibb-w-constraints,sec:hcpi-w-constraints} for the methodology and theoretical claims, and \Cref{sec:synthetic-experiments,sec:sepsis-experiments} for the empirical results.}
  
  \item Did you describe the limitations of your work?
    \answerYes{See \Cref{sec:introduction} for the limitations and scope of this work.}
  
  \item Did you discuss any potential negative societal impacts of your work?
    \answerYes{
    % In this work, we propose methodology and algorithms for \textit{safe} policy improvement of RL agents in the multi-objective setting.
    As we mentioned in \Cref{sec:introduction,sec:problem-formulation}, our goal is to maximize the objective specified by the user while ensuring that the solution policy avoids causing harmful effects after deployment in the true environment in comparison to the existing baseline policy. 
    % This allows the algorithm practitioner to experiment with different reward design strategies in safety-critical settings without worrying about the risks of ill-defined scalarizations. 
    % As with other Seldonian algorithms, we do not address any safety risks concerning data privacy or security, and additional caution should be exercised related to these topics.  
    We aim to bridge the gap between traditional RL methods and high-stake real-world applications, but, as with any general technology, we acknowledge that some RL applications can have the potential of misuse, and our methods do not prevent that.
    }

    
  \item Have you read the ethics review guidelines and ensured that your paper conforms to them?
    \answerYes{We discuss the societal impacts in the point above. We use pre-existing publicly available data and libraries and give more details about them in the Point 4 below.}
\end{enumerate}

\item If you are including theoretical results...
\begin{enumerate}
  \item Did you state the full set of assumptions of all theoretical results?
    \answerYes{See \Cref{sec:setting} for the assumption regarding access to baseline policy.}
	\item Did you include complete proofs of all theoretical results?
    \answerYes{The complete proofs are provided in the  \Cref{app:spibb-additional-details} and \Cref{app:hcpi-details}.}
\end{enumerate}

\item If you ran experiments...
\begin{enumerate}
  \item Did you include the code, data, and instructions needed to reproduce the main experimental results (either in the supplemental material or as a URL)?
    \answerYes{The code required to produce the results is provided in the supplementary material.}
    
  \item Did you specify all the training details (e.g., data splits, hyperparameters, how they were chosen)?
    \answerYes{The training details are mentioned in  \Cref{sec:synthetic-experiments,sec:sepsis-experiments} in the main text, and  \Cref{app:additional-details-for-synthetic-exp,app:sepsis-details}.}
    
	\item Did you report error bars (e.g., with respect to the random seed after running experiments multiple times)?
    \answerYes{The details about the error bars are provided in \Cref{sec:synthetic-experiments,sec:sepsis-experiments}.}
    
	\item Did you include the total amount of compute and the type of resources used (e.g., type of GPUs, internal cluster, or cloud provider)?
    \answerYes{Details about the compute and resources can be found in \cref{app:additional-details-for-synthetic-exp,app:sepsis-details}.
    }
\end{enumerate}

\item If you are using existing assets (e.g., code, data, models) or curating/releasing new assets...
\begin{enumerate}
  \item If your work uses existing assets, did you cite the creators?
    \answerYes{See \Cref{sec:synthetic-experiments,sec:sepsis-experiments} for the appropriate references.}
  
  \item Did you mention the license of the assets?
    \answerYes{We use the \citep[MIMIC-III,][]{johnson2016mimic} dataset and provide the appropriate reference. The explicit link to the license is here: \url{https://physionet.org/content/mimiciii/view-license/1.4/}. }
  
  \item Did you include any new assets either in the supplemental material or as a URL?
    \answerYes{We provide the accompanying code for running the experiments in the supplemental material.}
    
  \item Did you discuss whether and how consent was obtained from people whose data you're using/curating?
    \answerNA{We are using publicly available libraries and dataset.}
    
  \item Did you discuss whether the data you are using/curating contains personally identifiable information or offensive content?
    \answerNA{We are using the publicly available dataset  \citep[MIMIC-III,][]{johnson2016mimic} that already deidentifies the data in accordance with Health Insurance Portability and Accountability Act (HIPAA) standards using structured data cleansing and date shifting. We refer to the original text for more details on the deidentification process.
    }
\end{enumerate}

\item If you used crowdsourcing or conducted research with human subjects...
\begin{enumerate}
  \item Did you include the full text of instructions given to participants and screenshots, if applicable?
    \answerNA{}
  \item Did you describe any potential participant risks, with links to Institutional Review Board (IRB) approvals, if applicable?
    \answerNA{}
  \item Did you include the estimated hourly wage paid to participants and the total amount spent on participant compensation?
    \answerNA{}
\end{enumerate}

\end{enumerate}



%%%%%%%%%%%%%%%%%%%%%%%%%%%%%%%%%%%%%%%%%%%%%%%%%%%%%%%%%%%%

% \appendix



% ------------------------------------------------------------
%               APPENDIX 
% ------------------------------------------------------------
\clearpage
\onecolumn
\appendix

% Naive construction
\section{Scalarized safety constraints}
\label{app:naive-construction}

Instead of having constraints of the form in \Cref{eq:general-safety-constraints}, it is possible to define the constraints in terms of the scalarized objective directly, i.e., 
\begin{align*}
    \mathbb{P}\left(    \sum_{k\in[d]} \lambda_i \J{\pi}{i}{\mstar}  - \sum_{k\in[d]} \lambda_i \J{\pib}{i}{\mstar} > - \zeta \Big| \D \right) > 1 -\delta.
\end{align*}
 

Without loss of generality, if we assume there are only two objective ($d=2$), then satisfying the above constraint implies:
\begin{align*}
    \lambda_0 \left(\J{\pi}{0}{\mstar} - \J{\pib}{0}{\mstar} \right) + \lambda_1 \left(\J{\pi}{1}{\mstar} - \J{\pib}{1}{\mstar}  \right) \geq 0 .
\end{align*}
Consider a scenario where the solution policy performs poorly w.r.t. the second objective, i.e, $\lambda_1 (\J{\pi}{1}{\mstar}  - \J{\pib}{1}{\mstar}) < 0$, however the the improvement in the first objective is very large $\lambda_0 (\J{\pi}{0}{\mstar} - \J{\pib}{0}{\mstar}) >> 0$. In this case, even though the linearized cumulative constraint regarding the performance improvement is being satisfied, it fails to guarantee the improvement across each individual objectives.


% As an informal proof, notice that the new constraint in \Cref{eq:general-task-objective} will now be $\lR \J{\pi}{R}{\mstar} - \lC  \J{\pi}{C}{\mstar} \geq \lR \J{\pib}{R}{\mstar} - \lC \J{\pib}{C}{\mstar}$. 

% Satisfying this constraint implies that  $\lR \left( \J{\pi}{R}{\mstar} - \J{\pib}{R}{\mstar} \right) \bm{+} \lC \left( \J{\pib}{C}{\mstar}  - \J{\pi}{C}{\mstar} \right) \geq 0 $.

% Now, assume that if the solution policy actually performs poorly w.r.t. $C$, i.e., $(\J{\pib}{C}{\mstar}  - \J{\pi}{C}{\mstar}) < 0$, but the performance in rewards is very large i.e, $(\J{\pi}{R}{\mstar} - \J{\pib}{R}{\mstar}) >> 0$, then even though the linearized constraint regarding performance is being satisfied, it still violates the individual cost constraint.

% % Error bounds
% \section{Error bounds}
\label{app:err-bounds}

Using the same notation from \cite{nadjahi2019safe}, we have:

\begin{align}
    \norm{P(.|x,a) - \hat{P}(.|x,a)}_{1} &\leq e_P(x,a),  \label{eq:eP}
    \\ \abs{R(x,a) - \hat{R}(x,a)} &\leq e_P(x,a) R_{\max}, \label{eq:eR}
    \\ \abs{C_i(x,a) - \hat{C_{i}}(x,a)} &\leq e_P(x,a) C_{i_{\max}}, \quad (\forall i),   \label{eq:eC}
    \\ \abs{Q^{\pi_b}_{M*}(x,a) - Q^{\pi_b}_{\hat{M}}(x,a)} &\leq e_Q(x,a) V_{\max},  \label{eq:eQ}
    \\ \abs{{Qc}^{\pi_b}_{i, M*}(x,a) - Q^{\pi_b}_{i, \hat{M}}(x,a)} &\leq e_Q(x,a) {Vc}_{\max},  \label{eq:eQc}
\end{align}
where,
\begin{align}
    e_P(x,a) &= \sqrt{\frac{2}{N_{\D}(x,a)} \log\frac{2 \abs{\X}\abs{A}2^{\abs{\X}}}{\delta}}, 
    \\ e_Q(x,a) &= \sqrt{\frac{2}{N_{\D}(x,a)} \log\frac{2 \abs{\X}\abs{A}}{\delta}}.
\end{align}


% % SPIBB-details
% \section{SPIBB formulation details}
\label{app:spibb-background}


% ------------------------------------------------------
%                       SPIBB 
% ------------------------------------------------------


% An alternate objective that might be of interest is the following: 
% \begin{align}
%     \max_{\pi} &\quad J(\pi, \hat{M}) \label{eq:alt-batch-pi}\\ 
%     \texttt{s.t.} &\quad J(\pi, M) \geq J(\pi_b, M) - \xi, \forall M \in \Xi,
%     \nonumber \\
%     &\quad J(C_i, \pi, \hat{M}) \leq d_i, \forall i, \nonumber \\
%     &\quad J(C_i, \pi, M) \leq f(d_i, \xi), \forall i. \nonumber
% \end{align}

% The main difference between Eq.~\ref{eq:batch-pi} and Eq.~\ref{eq:alt-batch-pi} is that in Eq.~\ref{eq:alt-batch-pi} the constraint satisfaction is a hard constraint with respect to the estimated MDP, and for constraint satisfaction in the admissible MDPs we would like it to be some function of $\xi$-like quantity (to be decided). 


% \harsh{Hypothesis: One of the problems with the CMDPs formulation is that sometimes the algorithms finds the solution that is just below the constraint threshold. While this is okay in certain scenarios it ends up being more like a equality constraint. In this scenario, user can specify that the goal is to increase safety/reduce constraints, while making sure the performance doesn't suffer much.}


The SPIBB algorithm \citep{laroche2017safe} takes input a dataset of trajectories under the baseline policy $\D$, as well as needs access to the baseline policy $\pib$ to do return a improved policy based on the guarantees parameters provided by the user.

The high level methodology of the SPIBB algorithm is along the following lines. The first step is to  build a bootstrap set $\Bset$, a set of state-action pairs with counts less than $N_\Lambda$ (hyper-parameter). When the state-action pair has low confidence (rarely encountered in the dataset), SPIBB will bootstrap the baseline policy for it. Here bootstrapping means that the agent will rely on the baseline policy and copy the probabilities for that state-action pair, 
\begin{equation*}
    \pi^{\odot}_{spibb}(a|x) = \pi_b(a|x) \text{ if } (x,a) \in \Bset.
\end{equation*}
For the non-bootstrapped actions, SPIBB uses the greedy update but restricted to the non-bootstrapped actions:
\begin{equation*}
    \pi^{(i)}_{spibb}(x , \argmax_{a \mid a \notin \Bset} Q^{(i)}(x,a)) = \sum_{a \mid a \notin \Bset} \pi_b(a|x)
\end{equation*}
SPIBB is based on the binary classification of the bootstrapped set: either a pair either belongs to it and cannot affect the new policy, or it does not and the policy is changed completely for that pair. 

In Soft-SPIBB \citep{nadjahi2019safe}, authors  the objective that allows slight policy changes for uncertain state-action pairs while remaining safe. The authors constrain the class of policies to $(\pi_b, e, \epsilon)$-constrained policies: 
\begin{equation}
    \sum_a e(x,a)\ |\pi(a|x) - \pi_{b}(a|x)| \leq \epsilon,  \forall x \in \X,
\end{equation}
where $e: \X \times \A \rightarrow \Real$ is an error function on the state-action value function $Q$ and $\epsilon$ is a hyper-parameter that controls the deviation from the baseline policy. The second constraint is to improve the performance of the new policy w.r.t. the baseline, i.e. to a find $\pi_b$-advantageous policy:
\begin{equation}
\label{eq:spibb-performance-constraint}
\sum_a \pi(a|x) A^{\pi_b}(x, a) \geq 0, \forall x \in \X,
\end{equation}
where $A^{\pi_b}$ denotes the advantage function: $A^{\pi}(x,a) = Q^{\pi}(x,a) - V^{\pi}(x)$. Under these two conditions, i.e., if the policy is $\pi_b$-advantageous and $(\pi_b, e_Q, \epsilon)$-constrained then it is possible to show  can have the safety guarantees.  The new policy improvement can be viewed as:
    \begin{align}
        \label{eq:soft-spibb-obj}
        \pi_{new} &= \argmax_{\Pi} \langle \pi(\cdot|x) Q^{\pi_b}(x,\cdot) \rangle \quad \forall x \in \X  \\
        \texttt{s.t.} &\sum_a e_Q(x,a)\ |\pi(a|x) - \pi_{b}(a|x)| \leq \epsilon, \nonumber \\ 
        &\sum_a \pi(a|x) = 1 . \nonumber
    \end{align}


% more details
As such they are working with a constrained optimization problem/LP, that they propose to solve in every iteration of the policy improvement. The exact solution of the LP can be found with simplex or mixed integer programming. The authors also propose an approximate method for calculating the solution.
However, these policy constraints can be too conservative, they use $e_P$, to relax the policy space and use another assumption to guarantee safety bounds. 
    

% SPIBB-theorems
\section{SPIBB - Additional details}
\label{app:spibb-additional-details}  
% ----------------------------------------------------
%               Concentration Bounds
% ----------------------------------------------------
\subsection{Concentration Bounds} 
\label{app:error_bounds}

The difference between an estimated parameter and the true one can be bounded using concentration bounds (or equivalently, Hoeffding's inequality) applied to the state-action counts $n_{\mathcal{D}}(x,a)$ in dataset $\mathcal{D}$~\citep{petrik2016safe, laroche2017safe}. Specifically, the following inequalities hold with probability at least $1-\delta = 1 - \delta' - \delta''$ for any state-action pair $(x,a) \in \mathcal{X} \times \mathcal{A}$:
\begin{align}
	\lVert p^\star(\cdot|x,a)-\hat{p}(\cdot|x,a)\rVert_1 &\leq e_p(x,a),\\
	\forall k\in[d], \lvert r^\star_k(x,a)-\hat{r}_k(x,a)\rvert &\leq e_r(x,a)r_{\mytop} \label{eq:bound-Q-2}
\end{align}
where: 
\begin{align}
    e_p(x,a) &:= \sqrt{\cfrac{2}{n_{\mathcal{D}}(x,a)}\log\cfrac{2|\mathcal{X}||\mathcal{A}|2^{|\mathcal{X}|}}{\delta'}} \label{eq:error-function-P-2} \\
    e_r(x,a) &:= \sqrt{\cfrac{2}{n_{\mathcal{D}}(x,a)}\log\cfrac{2|\mathcal{X}||\mathcal{A}|d}{\delta''}}. \label{eq:error-function-Q-2}
\end{align}

    The two inequalities can be proved similarly to \citep[Proposition 9]{petrik2016safe}. We only detail the proof for \eqref{eq:bound-Q-2}: for any $(x,a) \in \mathcal{X} \times \mathcal{A}$, and from the two-sided Hoeffding's inequality, 
    \begin{align*}
        &\mathbb{P} \left( \forall (x,a), \big\lvert r^\star_k(x,a) -\hat{r}_k(x,a) \big\rvert > e_r(x,a)r_{\mytop} \right) 
        \\
        &\qquad\qquad = \mathbb{P} \left( \forall (x,a), \frac{\big\lvert r^\star_k(x,a)-\hat{r}_k(x,a) \big\rvert}{2 V_{max}} > \sqrt{\frac{1}{2 n_\mathcal{D}(x,a)} \log \frac{2 |\mathcal{X}||\mathcal{A}|d}{\delta''}} \right) \\
        &\qquad\qquad \leq 2 \exp \left( -2 n_\mathcal{D}(x,a) \frac{1}{2 n_\mathcal{D}(x,a)} \log \frac{2 |\mathcal{X}||\mathcal{A}|}{\delta''}  \right) \\
        &\qquad\qquad \leq \frac{\delta''}{| \mathcal{X} | | \mathcal{A} | d}
    \end{align*}
    
    By summing all $|\mathcal{X}| |\mathcal{A}|d $ state-action-reward tuples error probabilities lower than $\frac{\delta''}{| \mathcal{X} | | \mathcal{A} |d }$, we obtain \eqref{eq:bound-Q-2}. If we choose $e(x,a)=e_p(x,a)=e_r(x,a)$, we get that:
    \begin{align}
        \cfrac{2}{n_{\mathcal{D}}(x,a)}\log\cfrac{2|\mathcal{X}||\mathcal{A}|2^{|\mathcal{X}|}}{\delta'} &= \cfrac{2}{n_{\mathcal{D}}(x,a)}\log\cfrac{2|\mathcal{X}||\mathcal{A}|d}{\delta''} \\
        \cfrac{2^{|\mathcal{X}|}}{\delta'} &= \cfrac{d}{\delta''} \\
        \delta'' &= d\delta' 2^{-|\mathcal{X}|}
    \end{align}
    
    It means that $\delta = \delta' + \delta'' = \delta'(1+d2^{-|\mathcal{X}|})$. The cost in terms of approximation is therefore linear with a very small slope, inside square root of log, which means that it will basically have an insignificant impact on the concentration bound.


% ----------------------------------------------------
%               Advantage Constraints
% ----------------------------------------------------
\subsection{Need of advantageous constraints}
\label{app:spibb-need-of-advatangeous}



\begin{prop}
\label{prop:mo-sipbb-advantageous}
The advantageous constraints in \ref{eq:s-opt} ensure that performance constraints w.r.t. the individual returns are respected in $\mhat$, i.e., $ \forall k \in [d],\; \J{\pi}{k}{\mhat} - \J{\pib}{k}{\mhat} \ge 0$.
\end{prop}

\begin{proof}
For the $k$\textsuperscript{th} reward function, we can estimate the advantage function in an MDP $m$ as:
\begin{align*}
    \adv{\pib}{k}{m}{x,a} = \qval{\pib}{k}{m}{x,a} - \val{\pib}{k}{m}{x}
\end{align*}
Similarly, let $\rho^{\pi}_{m}(x)$ denote the normalized discounted future state distribution:
\begin{align*}
    \rho^{\pi}_{m}(x) &= (1-\gamma)\sum_{t=0}^{\infty} \gamma^t \mathbb{P}(X_t=x | \pi, X_0=x_0),
\end{align*}
where $X_{t} \sim p(\cdot | X_{t-1}, A_{t-1}), A_{t-1} \sim \pi(\cdot|X_{t-1})$.
From Performance Difference Lemma \citep{kakade2002approximately}, we have the following result:
\begin{align}
    \label{eq:spibb-prop-adv}
    \J{\pi}{k}{\mhat} - \J{\pib}{k}{\mhat} &= \sum_{x \in \X} \rho^{\pi}_{\mhat}(x) \underbrace{\sum_{a \in \A} \pi(a|x) \adv{\pib}{k}{\mhat}{x,a}}_{\text{advantage constraint}}
\end{align}

The first term in the above equation $\rho^{\pi}_{\mhat}(x) \ge 0$ for any $x \in \X$. The second term is the advantage constraint in the construction of \ref{eq:s-opt}.
Therefore, any solution of \ref{eq:s-opt} satisfies $\sum_{a \in \A} \pi(a|x) \adv{\pib}{k}{\mhat}{x,a} \ge 0, \forall x \in \X$.

As both the terms in \Cref{eq:spibb-prop-adv} are $\ge 0 \; \forall x \in \X$, this implies $ \J{\pi}{k}{\mhat} - \J{\pib}{k}{\mhat} \ge 0$.

\end{proof}



% ------- Old prop
% \begin{prop}
% \label{prop:spibb-r-advatangeous}
% If $\pi'$ is the solution to problem \Cref{eq:s-opt} without the $R$-constraints, i.e. without $\sum_a \pi(a|x) \Adv{\pib}{R}{\mhat}(x, a) \geq 0$ term, then $\pi'$ won't necessarily $\pib$-advantageous in $\mhat$ with respect to $\pib$ regarding reward performance $R$:
% \begin{equation*}
%     \sum_a \pi'(a|x) \Adv{\pib}{R}{\mhat}(x, a) \not\geq 0, \forall x \in \X.
% \end{equation*}
% \end{prop}
% \begin{proof}
% \htodo{Write this as a proof by contradiction, assume $\lR, \lC \geq 0$ and advantageous wrt costs are true, and then continue}
% Since $\pi'$ is the solution of \Cref{eq:s-opt} (without $R$-constraints) and $\pib$ also lies in the solution space $\Pi$, we have for any $x \in \X$:
% \begin{align*}
%     &\langle \pi'(\cdot|x) Q_{\lambda}^{\pi_b}(x,\cdot) \rangle \ge \langle \pib(\cdot|x) Q_{\lambda}^{\pib}(x,\cdot) \rangle \\ 
%     &\sum_{a \in \A} \left( \lR \pi'(a|x) \Q{\pib}{R}{\mhat}(x,a ) - \sum_i \left( \lC \pi'(a|x) \Q{\pib}{\ci}{\mhat}(x,a) \right) \right) \ge \\ 
%     &\quad \quad \sum_{a \in \A} \left( \lR \pib(a|x) \Q{\pib}{R}{\mhat}(x,a ) - \sum_i \left( \lC \pib(a|x) \Q{\pib}{\ci}{\mhat}(x,a) \right) \right)
% \intertext{Taking all the terms w.r.t $R$ on left and the rest on R.H.S, we get:}
%     &\lR \left( \sum_{a \in \A}\pi'(a|x) \Q{\pib}{R}{\mhat}(x,a)  -  \underbrace{\sum_{a \in \A} \pib(a|x) \Q{\pib}{R}{\mhat}(x,a)}_{\V{\pib}{R}{\mhat}(x)}  \right) \ge \\ 
%     &\quad \quad \sum_i \lC \left( \sum_{a \in \A} \pi'(a|x) \Q{\pib}{\ci}{\mhat}(x,a)  -  \underbrace{\sum_{a \in \A} \pib(a|x) \Q{\pib}{\ci}{\mhat}(x,a)}_{\V{\pib}{\ci}{\mhat}(x)}  \right)
% \end{align*}
% Using the definition of the advantage function we have:
% \begin{align*}
%     \lR \left( \sum_{a \in \A}\pi'(a|x) \Adv{\pib}{R}{\mhat}(x,a) \right) 
%     &\ge \sum_i \lC \left( \underbrace{\sum_{a \in \A} \pi'(a|x) \Adv{\pib}{\ci}{\mhat}(x,a)}_{\le 0} \right) 
% \end{align*}
% We have $\Adv{\pib}{\ci}{\mhat}(x,a) \le 0 \; \forall i$ by construction, and $\lR \ge 0$ and $\lC \ge 0$. When $\lR > 0,  \lC >0$ and the term on R.H.S. is negative $<0$, then that implies that it might not longer be necessary that $\pib$-advantageous property holds true in this case and the term $\sum_{a \in \A}\pi'(a|x) \Adv{\pib}{R}{\mhat}(x,a)$ can be negative.

% \textit{Remark:} In the case of Soft-SPIBB, $\lC = 0, \lR = 1$, and as such the R.H.S. is always 0 and we have:
% \begin{align*}
% \left( \sum_{a \in \A}\pi'(a|x) \Adv{\pib}{R}{\mhat}(x,a) \right) &\ge 0.
% \end{align*}

% \end{proof}



% ----------------------------------------------------
%               Soft-SPIBB 1-step
% ----------------------------------------------------
\subsection{MO-SPIBB Results}
\label{app:mo-spibb-prop}

Using the results from \Cref{app:error_bounds} and \Cref{app:spibb-need-of-advatangeous}, we can directly apply the Soft-SPIBB theorems to individual objectives in \ref{eq:s-opt}. For instance, we get the following result about 1-step policy improvement guarantees directly from Theorem 1 of Soft-SPIBB:

\begin{prop}
The policy $\pi$ returned from solving the \ref{eq:s-opt} satisfies the following property in every state $x$ with probability at least $(1 - \delta)$:
\begin{align}
    \forall k \in [d], \val{\pi}{k}{\mopt}{x} - \val{\pib}{k}{\mopt}{x} \geq -\frac{\epsilon v_{\text{max}}}{1-\gamma},
\end{align}
where $v_{\text{max}} \le \frac{\rmax}{1-\gamma}$ is the maximum of the value function.
\end{prop}

\begin{proof}
We will show the policy returned by \ref{eq:s-opt} satisfies both the properties required for applying the Theorem 1 of Soft-SPIBB:


\begin{itemize}
    \item $\pi$ is $(\pib, \epsilon, e)$-constrained: This is equivalent to $\sum_{a \in \A} e(x,a)\ |\pi(a|x) - \pib(a|x)| \leq \epsilon$, that is true by construction.
    
    \item $\pib$-advantageous in $\mhat$: For $k$\textsuperscript{th} reward function, this is equivalent to $\J{\pi}{k}{\mhat} - \J{\pib}{k}{\mhat} \ge 0$, which is also true from construction.
\end{itemize}
From there, the exact statement of Theorem 1 can be applied directly to get the above result.
\end{proof}





% \begin{proof}

% \textbf{For rewards:} 
% We first use  \Cref{prop:spibb-r-advatangeous} to show that we need additional constraints for $\pib$-advantageous property to hold true for $R$. Once we have added those to \Cref{eq:s-opt}, we can use the exact same structure of the proof of Theorem 1 of \cite{nadjahi2019safe}. 

% \textbf{For constraints:} 
% From Proposition 1 of \cite{nadjahi2019safe}, we have the following for any constraint $C_i$:
% \begin{align}
%     {Vc}^{\pi}_{i, M^*}(x) - {Vc}^{\pi_b}_{i, M^*}(x)  &= {Qc}^{\pi_b}_{M^*} (\pi - \pi_b) 
%     d^{\pi}_{M^*} \nonumber \\
%     &= \left(Qc^{\pi_b}_{M^*} - Qc^{\pi_b}_{\hat{M}} + Qc^{\pi_b}_{\hat{M}} \right)(\pi - \pi_b) d^{\pi}_{M^*} \nonumber \\ 
%     &= \left(Qc^{\pi_b}_{M^*} - Qc^{\pi_b}_{\hat{M}}\right) (\pi - \pi_b) d^{\pi}_{M^*}  +  Qc^{\pi_b}_{\hat{M}}(\pi - \pi_b)d^{\pi}_{M^*} \label{eq:thm1-proof}.
% \end{align}

% We will now show the first term is bounded by $\frac{\epsilon {Vc}_{i, \max}}{(1-\gamma)}$ using the Holder's inequality.

% \begin{align}
%     \norm{\left(Qc^{\pi_b}_{M^*} - Qc^{\pi_b}_{\hat{M}}\right) (\pi - \pi_b) d^{\pi}_{M^*}}_{\infty} &= \norm{\left(Qc^{\pi_b}_{M^*} - Qc^{\pi_b}_{\hat{M}}\right) (\pi - \pi_b)}_{\infty} \norm{d^{\pi}_{M^*}}_{1} \\ 
%     &= \max_{x} \sum_{a} \left(Qc^{\pi_b}_{M^*} - Qc^{\pi_b}_{\hat{M}}\right) (\pi - \pi_b) \frac{1}{(1 - \gamma)}\\
%     &\leq \frac{\epsilon {Vc}_{\max}}{(1 - \gamma)}.
% \end{align}

% The last line comes from the $(\pi_b, e_Q, \epsilon)$-constrained property and Eq.~\ref{eq:eQc}. The next part is to show the second part of Eq.~\ref{eq:thm1-proof} is negative. All the terms of $d^{\pi}_{M^*}$ are positive so if all the terms of vector $Qc^{\pi_b}_{\hat{M}}(\pi - \pi_b) \leq 0$ we can upper bound the result by the first term. For each $x \in \X$, we have:
% \begin{align*}
%     Qc^{\pi_b}_{\hat{M}}(\pi - \pi_b) &= \sum_a Qc^{\pi_b}_{\hat{M}}(\pi(a|x) - \pi_b(a|x)) \\ 
%     &= \sum_a Qc^{\pi_b}_{\hat{M}}\pi(a|x) - {Vc}^{\pi_b}_{\hat{M}}(x) \\ 
%     &= \sum_{a} {Ac}^{\pi_b}_{i, \hat{M}}(x,a) \pi(a|x) \\ 
%     &\leq 0
% \end{align*}
% The last inequality comes from the construction in the Eq.~\ref{eq:s-opt}. This concludes the proof.

% \end{proof}

% ----- multi-step stuff

% It is possible to search over the class of $(\pib, e_P, \epsilon)$-constrained policies, where $e_p$ is the error bound over the transition function. Using the procedure in \cite{nadjahi2019safe} it is possible to guarantee safety bounds under the Assumption~\ref{assm:eP-bounded}.

% \begin{assumption}
% \label{assm:eP-bounded}
% There exists a constant $\kappa < \frac{1}{\gamma}$ such that, for all state-action pairs $(x,a) \in \X \times \A$ the following inequality holds:
% \begin{equation}
%     \label{eq:assm-eP}
%     \sum_{x', a'} e_P(x', a') \pi_b(a'|x') P^{*}(x'|x,a) \leq \kappa e_P(x,a).
% \end{equation}
% \end{assumption}

% \begin{prop}[Multi-step PI]
% \label{thm:eP-guarantee}
%     Under Assumption~\ref{assm:eP-bounded}, the policy $\pi$ returned by the policy iteration step given by \ref{eq:s-opt} satisfies satisfies the following inequalities in every state $x$ with probability at least $(1 - \delta)$:
%     \begin{align*}
%         \V{\pi}{R}{\mopt}(x) - \V{\pib}{R}{\mopt} (x) &\geq 
%         % (\V{\pi}{R}{\mhat}(x) - \V{\pib}{R}{\mhat}(x)) 
%         - 2 \norm{d^{\pib}_{\mopt}(\cdot|x)  
%         -  d^{\pib}_{\mhat}(\cdot|x) }_1 V^{R}_{\max} 
%         - \frac{1 + \gamma}{(1 - \gamma)^2 (1 - \kappa \gamma)}\epsilon V^R_{\max}, \\ 
%         \V{\pi}{\ci}{\mopt}(x) - \V{\pib}{\ci}{\mopt} (x) &\leq 2 \norm{d^{\pib}_{\mopt}(\cdot|x) + d^{\pib}_{\mhat}(\cdot|x) }_1 V^{\ci}_{\max} 
%         + \frac{1 + \gamma}{(1 - \gamma)^2 (1 - \kappa \gamma)}\epsilon V^{\ci}_{\max}  \tag{$\forall i$}.
%     \end{align*}
% \end{prop}


% The first inequality for $V^R$ follows directly from Theorem 2 of \cite{nadjahi2019safe}. We will show the proof for the second part, i.e, the extension with constraints using the same approach.

% \begin{proof}

% From \citep[Theorem 2]{nadjahi2019safe} we have under the Assumption ~\ref{assm:eP-bounded}, any $(\pib, e_P, \eps)$-constrained policy $\pi$ satisfies the following inequality for every state-action pair $(x, a)$ with probability at least $1-\delta$:
% \begin{align*}
%       \V{\pi}{R}{\mopt}(x) - \V{\pib}{R}{\mopt} (x) &\geq \V{\pi}{R}{\mhat}(x) - \V{\pib}{R}{\mhat} (x) - 2 \norm{d^{\pib}_{\mopt}(\cdot|x) -  d^{\pib}_{\mhat}(\cdot|x) }_1 V^{R}_{\max} \\
%         &\quad - \frac{1 + \gamma}{(1 - \gamma)^2 (1 - \kappa \gamma)}\epsilon V^R_{\max}.
% \end{align*}

% We will show that the first term $(\V{\pi}{R}{\mhat}(x) - \V{\pib}{R}{\mhat}(x))$ is positive, and use that to lower bound the expression above. From \citet[Proposition 1]{nadjahi2019safe} we have: 
% \begin{align*}
%     V^{\pi_1} - V^{\pi_2} &= Q^{\pi_2} (\pi_1 - \pi_2) d^{\pi_1}. 
% \end{align*}
% Substituting $\pi_1 =\pi$ and $\pi_2 =\pi_{b}$, we have:
% \begin{align*}
%     V^{\pi} - V^{\pi_b} &= Q^{\pi_b}(\pi - \pi_b) d^{\pi}
% \end{align*}
% As the term $d^\pi(x) \geq 0$ for any $x$, we need to show the term $Q^{\pi_b}(\pi - \pi_b)$ is positive.  For any $x \in \X$, we have:
% \begin{align*}
%     Q^{\pi_b}_{\hat{M}}(\pi - \pi_b)(x) &= \sum_a Q^{\pi_b}_{\hat{M}}(x,a)(\pi(a|x) - \pi_b(a|x)) \\ 
%     &= \sum_a Q^{\pi_b}_{\hat{M}}(x,a)\pi(a|x) - {V}^{\pi_b}_{\hat{M}}(x) \\ 
%     &= \sum_{a} {A}^{\pi_b}_{\hat{M}}(x,a) \pi(a|x) \\ 
%     &\geq 0
% \end{align*}
% The last inequality comes from \Cref{lemma:spibb-r-advatangeous}. This concludes the proof for the rewards. 

% The proof for the constraints follows the same procedure, but now we will upper bound the difference in performance. From \cite[Theorem 2]{nadjahi2019safe} for constraints we get, for any $i \in \{1, \dots, m\}$:
% \begin{align*}
%       \V{\pi}{\ci}{\mopt}(x) - \V{\pib}{\ci}{\mopt} (x) &\leq - \Big( \V{\pi}{\ci}{\mhat}(x) - \V{\pib}{\ci}{\mhat} (x) \Big) + 2 \norm{d^{\pib}_{\mopt}(\cdot|x) +  d^{\pib}_{\mhat}(\cdot|x) }_1 V^{\ci}_{\max} \\
%         &\quad - \frac{1 + \gamma}{(1 - \gamma)^2 (1 - \kappa \gamma)}\epsilon V^{\ci}_{\max}.
% \end{align*}
% As in the previous case, if we can show the term $(\V{\pi}{\ci}{\mhat}(x) - \V{\pib}{\ci}{\mhat} (x) )$ is negative, and use that to bound the expression. As in Section~\ref{app:proof-thm1}, we have: 
% \begin{align*}
%     \V{\pi}{\ci}{\mhat}(x) - \V{\pib}{\ci}{\mhat} (x) &\leq 0,  
% \end{align*}
% that again comes from using the construction in \ref{eq:s-opt}. This concludes the proof for constraints.
    
    
% \end{proof}


% HCPI-details
\section{HCPI - Additional details}
\label{app:hcpi-details}

\paragraph{Concentration Inequalities:} We experimented with the following concentration inequalities \citep{thomas2015highImprovement}:
\begin{itemize}[leftmargin=*]
    \item Extension of Empirical Bernstein \citep{maurer2009empirical}: This is the extension of
    Maurer \& Pontil's empirical Bernstein (MPeB) inequality. From Theorem 1 of \cite{thomas2015highEvaluation}: Let $X_1, \dots, X_n$ denote $n$ independent real-valued random variables, such that for each $i \in \{1,\dots,n\}$, we have $\pr(0 \le X_i) = 1,  \E[X_i] \le \mu$, and some fixed real-valued threshold $c_i > 0$. Let $\delta > 0$ and $Y_i = \min\{X_i, c_i\}$, then with probability at least $(1-\delta)$:
    \begin{align}
        \mu &\ge \sum_{i=1}^{n}\left(\frac{1}{c_i}\right)^{-1} \sum_{i=1}^{n} \frac{Y_i}{c_i} - \sum_{i=1}^{n}\left(\frac{1}{c_i}\right)^{-1} \frac{7n \ln(2/\delta)}{3n-1} - 
        \sum_{i=1}^{n}\left(\frac{1}{c_i}\right)^{-1} \sqrt{\frac{\ln(2/\delta)}{n-1} \sum_{i,j=1}^{n}\left(\frac{Y_i}{c_i} - \frac{Y_j}{c_j}\right)^2}.
    \end{align}
    In context of this paper, for the $k$\textsuperscript{th} reward function, $X_i$ denotes the $\IS$ estimated return for that trajectory, i.e., $\IS_k(\tau_i,\pi_t, \pib)$. Here, $c_i$ is a hyper-parameter that needs to be tuned. In \cite{thomas2015highImprovement}, a fixed value of $c$ is used for all $c_i$. 
    
    \item Student's t-test \citep{walpole1993probability}: This is an approximate concentration inequality that is based on the assumption that the mean returns are distributed normally. For $k$\textsuperscript{th} reward, the \Cref{eq:hcope-R-lower-bound} can be written as:
    \begin{align}
    \pr \Big( \J{\pi_t}{k}{\mopt} \ge \IS_{k}(\D, \pi_t, \pib) - \frac{\hat{\sigma}_k}{\sqrt{|\D|}}t_{1-\delta/d, |\D|-1} \Big) \ge 1 - \delta/d, 
    \end{align}
    where $\hat{\sigma}_k$ is the sample standard deviation:
    \begin{align}
    \hat{\sigma}_k &= \sqrt{\frac{1}{|\D|-1} \sum_{i=1}^{|\D|}(\IS(\tau_i, \pi_t, \pib) - \overline{\IS} )^2 },
    \end{align}
    and $\overline{\IS} = \frac{1}{|\D|}\sum_{i=1}^{|\D|} \IS(\tau_i,\pi_t,\pib)$ and $t_{1-\delta/d, |\D|-1}$ is the $100(1-\delta/d)$ percentile of the student t-distribution with $|\D|-1$ degrees of freedom. 
    
\end{itemize}

We experimented with both MPeB Extension (with $c=0.5$) and Student's t-test inequalities and found that the solutions returned by the former to be very conservative. Therefore, we use t-test in all of our experiments. Even though the t-test's assumption (normally distributed returns) is technically false, it's a reasonable assumption due to central limit theorem. 
% (it's an assumption used in almost all scientific research when computing p-values)
The consequence is that the failure rate (the chance of deploying an unsafe policy) can, in theory, be higher than desired, though, in practice, that's unlikely.


\paragraph{Regularization:} 
For small problems, \ref{eq:h-opt} can be solved with methods like CMA-ES \citep{hansen2006cma}. 
% In practice, a policy is first derived using the traditional offline RL methods based on $D_{tr}$ and is regularized using $\pib$ to obtain a set of candidate policies. 
% 
For stochastic policies, 
as the optimization problem in \ref{eq:h-opt} is difficult to solve directly, we need to resort to a regularization based heuristic \citep{thomas2015highImprovement, laroche2017safe}. Let $\pi_t$ denote the solution policy found using $\D_{tr}$ using any of the traditional offline RL methods. A set of candidate policies is built using the baseline policy: $\pi_{\text{Cand}} = \{ (1-\alpha)\pi_t + \alpha \pi_b\}$, where $\alpha \in \set{0.0, 0.1, \dots, 0.9}$ is the regularization hyper-parameter. 
The best performing candidate policy that satisfies the safety-test (the performance constraints based on $\D_s$) is then returned.
If none of the candidate policies satisfy the safety-test, the baseline policy is returned.

For finding $\pi_t$, we experimented with both the Linearized and Adv-Linearized baselines in \Cref{sec:synthetic-experiments} and found that Adv-Linearized worked better (higher improvement over $\pib$ while failure rate $<\delta$). Therefore in our experiments, we first find $\pi_t$ using Adv-Linearized and then regularize it using $\pib$ to build the set of candidate policies $\pi_{\text{Cand}}$. 


\paragraph{Safety-guarantees:}
We get the safety guarantees related to \ref{eq:h-opt} directly from \cite{thomas2015highImprovement, Thomas2019}.  The constraints of \ref{eq:h-opt} define the new safety-test that ensures a candidate policy will only be returned if the individual performance guarantees corresponding to each reward function are satisfied. This procedure will only make error in the scenario where the performance constraint related to $k$\textsuperscript{th} is satisfied, i.e, $( \IS_{k}(\D_{s}, \pi, \pib) - \CI_k(\D_s, \delta/d) \geq \mu_k)$,  but in practice the policy is not good enough $(\J{\pi}{k}{\mopt} < \mu_k)$. By transitivity this implies $\J{\pi}{k}{\mopt} < \big( \IS_{k}(\D_{s}, \pi, \pib) - \CI_k(\D_s, \delta/d) \big)$, which from \Cref{eq:hcope-R-lower-bound} we know can only occur with probability at most $\delta/d$. Using the union bound, we know that cumulative probability of the union of any of these $d$ possible scenarios is $\leq \delta$.


\paragraph{Computational cost:} Compared to regular HCPI, there is an increase in computational cost proportional to the number of reward functions $d$. The value and advantage functions estimation cost increases by a factor of $d$: respectively $\mathcal{O}(d|\X|^3)$ and $\mathcal{O}(d|\A||\X|^2)$, the $\IS$ estimation also increases by factor of $d$, and the computational cost for safety-test also increases by $d$: $\mathcal{O}(d|\D|)$. 


% ---------------------------------------------------
%                   Old Stuff 
% ---------------------------------------------------


% \subsection{Chernoff-Hoeffding's Inequality for upper and lower bounds}
% \label{app:hcpi-hoeffding-bounds}


% Let $X_1, \dots, X_n$ be $n$ independent random variables such that $\pr(X_i \in [a_i, b_i])=1$. Let $S_n = \frac{1}{n} \sum_i X_i$ denote the empirical mean and $\E[S_n]$ be the true mean. Then using Chernoff-Hoeffding's Inequality \cite{hagerup1990guided} we have  for any $t>0$, we have:
% \begin{equation*}
%     \pr (|S_n - \E[S_n]| \ge t) \le 2 \exp^{-\frac{2t^2}{\sum_{i=1}^{n}(b_i - a_i)^2 }}
% \end{equation*}

% To invert this bound, set $\delta = 2 \exp^{-\frac{2t^2}{\sum_{i=1}^{n}(b_i - a_i)^2 }} \in (0,1)$. Solving for $t$, we see that with probability at least $1-\delta$,
% \begin{align*}
%     |S_n - \E [S_n]| &\leq \sqrt{\frac{\ln(\frac{1}{\delta}) \sum_{i=1}^n(b_i - a_i)^2}{2n^2}} \\ 
%     \left| \frac{1}{n}\sum{X_i} - \E \left[ \frac{1}{n} \sum_{i} X_i\right] \right| &\le  \sqrt{\frac{\ln(\frac{1}{\delta}) \sum_{i=1}^n(b_i - a_i)^2}{2n^2}}.
% \end{align*}

% From the above inequality we get:
% \begin{align*}
%     \pr \left(\E \left[ \frac{1}{n} \sum_{i} X_i\right] \ge \frac{1}{n}\sum{X_i} -    \sqrt{\frac{\ln(\frac{1}{\delta}) \sum_{i=1}^n(b_i - a_i)^2}{2n^2}} \right) \ge 1 - \delta,
%     \intertext{and}
%     \pr \left(\E \left[ \frac{1}{n} \sum_{i} X_i\right]  \le   \frac{1}{n}\sum{X_i} + \sqrt{\frac{\ln(\frac{1}{\delta}) \sum_{i=1}^n(b_i - a_i)^2}{2n^2}} \right) \ge 1 - \delta.
% \end{align*}



% \harsh{Proof of Hoeffding's from: \url{http://www.stat.cmu.edu/~arinaldo/Teaching/36709/S19/Scribed_Lectures/Jan29_Tudor.pdf}
% }

% \subsection{Proof of \Cref{prop:hcpi-safety-guarantee}}
% \label{app:proof-h-opt-safety-guarantee}

% \begin{prop}[Safety guarantees with HCPI]
% \label{prop:hcpi-safety-guarantee}
% When the test set $\D_s$ is big enough to build reliable high-confidence lower bounds, i.e., $\forall i\in [d] \; \IS_{i}(\D_s, \pi_t, \pi_b) \approx \IS_{i}(\D, \pi_t, \pi_b)$, the policy $\pi$ returned by \Cref{eq:h-opt} will only violate the safety guarantees with probability at most $\delta$.
% \end{prop}

% The constraints of \Cref{eq:h-opt} define the new safety-test that ensures a candidate policy will only be returned if the individual performance guarantees are satisfied. This procedure will only make error in either of the following scenarios:
% \begin{itemize} %enumerate?
%     \item The performance constraint related to $R$ is satisfied $( \IS_{R}(\D_{s}, \pi, \pib) - \CI_R(\D_s, \delta) \geq b_R)$,  but in practice the policy is not good enough $(\J{\pi}{R}{\mopt} < b_R)$. By transitivity this implies $\J{\pi}{R}{\mopt} < \big( \IS_{R}(\D_{s}, \pi, \pib) - \CI_R(\D_s, \delta) \big)$, which from \Cref{eq:hcope-R-lower-bound} we know can only occur with probability at most $\delta/2$.
    
%     \item If the cost performance constraint is satisfied $( \IS_{C}(\D_{s}, \pi, \pib) + \CI_C(\D_s, \delta) \leq b_C )$ but the policy ends up violating the constraint in practice $(\J{\pi}{C}{\mopt} > b_{C})$. Again, by transitivity this means that $\J{\pi}{C}{\mopt} > \big( \IS_{C}(\D_{s}, \pi, \pib) + \CI_C(\D_s, \delta) \leq b_C \big)$, and that can only happen with probability at most $\delta/2$ because of \Cref{eq:hcope-C-upper-bound} .
% \end{itemize}

% Using the union bound, we know that cumulative probability of the union of either of these two events is $\leq \delta$.






% CMDP experiments extra details 
\section{Additional details for synthetic CMDP experiments}
\label{app:additional-details-for-synthetic-exp}


% ------------------------------------------------------
%               Solving CMDP
% ------------------------------------------------------
\subsection{Solving CMDP}
\label{app:cmdp-solver}

Constrained-MDPs~\citep{altman1999constrained} are MDPs with multiple rewards where $r_0$ is the main objective, and $r_1, \dots, r_{n-1}$ are the reward signals that are used to enforce some behavior or constraints. 

Let $\J{\pi}{i}{m}(\mu)$ denote the total expected discount reward under $r_i$ in an MDP $m$, when $\pi$ is followed from an initial state chosen at random from $\mu$, the initial state distribution. 
For some given reals $c_1, \dots, c_n$ (each corresponding to $r_i$), the CMDP optimization problem is to find the policy that maximizes the $\J{\pi}{0}{m}(\mu)$ subject to the constraints $\J{\pi}{i}{\mopt}(\mu) \le c_i$ :
\begin{align}
    \label{eq:cmdp-obj}
    &\max_\pi \J{\pi}{0}{m}(\mu) \\ 
        \quad \text{ s.t. } & \J{\pi}{i}{m}(\mu) \le c_i, \, \forall i \in \{1,\dots,n-1\}. \nonumber
\end{align}
    
% CMDPs have many interesting properties that make it different from regular MDPs. One particular property is the lack of deterministic optimal policies that makes reward shaping (or reduction to regular) not always possible .

The Dual LP based algorithm for solving CMDP is based on the occupation measure w.r.t. the optimal policy $\piopt$. For any policy $\pi$ and initial state $x_0 \sim \mu(\cdot)$, the occupancy measure is described as:
\begin{align*}
    \rho^{\pi}(x,a) &= \E \left[ \sum_{t=0}^{\infty} \gamma^t \mathbbm{1}\{x_t=x, a_t=a\} \Big| x_0, \pi \right], \forall x \in \X, \forall a \in \A. 
\end{align*}
The occupation measure at any state $x \in X$ is defined as $\sum_{a} \rho^{\pi}(x,a)$.  From \citep[Chapter 9]{altman1999constrained}, the problem of finding the optimal policy for a CMDP can be solved by the solving the following LP problem: %$2|\X||\A| +1 $
\begin{align*}
    \max_{\rho}  &\quad \sum_{x \in \X, a \in \A} \rho(x,a) r_0(x,a) \\  
    \texttt{s.t.}  &\quad \sum_{x \in \X, a \in \A} \rho(x,a) r_i(x,a) \leq c_i, \; \forall i \in \{1,\dots,n-1\}.
\end{align*}
 As $\rho$ is the occupation measure it also needs to satisfy the following constraints $\forall x \in \X$:
\begin{align*}
    \rho(x,a) &\geq 0,  \quad \forall a \in \A \\
    \sum_{x_p \in \X, a \in \A} \rho(x_p,a) (\mathbbm{1}\{ x_p = x\} - p(x|x_p,a))  &= \mathbbm{1}\{ x=x_0 \}
\end{align*}
The above constraints originate from the conservation of probability mass of a stationary distribution on a Markov process.  The state-action visitations should satisfy the single-step transpose Bellman recurrence relation:
\begin{equation*}
    \rho^{\pi}(x,a) = (1-\gamma) \mu(x) \pi(a|x) + \gamma \cdot p_{T}^{\pi} \rho^{\pi}(x,a), 
\end{equation*}
where transpose policy transition operator $p_{T}^{\pi}$ is a linear operator and is the mathematical transpose (or adjoint) of $p^{\pi}$ in the sense that $<y, p^{\pi}x> = <p_{T}^{\pi} y, x>$ for any $x, y$:
\begin{equation*}
    p_{T}^{\pi} \rho(x,a) \doteq \pi(a|s)\sum_{\tilde{x}, \tilde{a}} p(x|\tilde{x}, \tilde{a}) \rho(\tilde{x}, \tilde{a})
\end{equation*}
% Therefore the total number of constraints are 3|X||A|+1
% The above relations are used to define the \textit{transpose Bellman operator} \citep{nachum2019algaedice}:
% \begin{equation*}
%     \mathcal{B}^{T}_{\pi}(\rho)(x',a') \doteq \gamma \sum_{x,a} \pi(a'|x') P(x'|x,a) \rho(x,a) + (1-\gamma) \mu_{x_0}(x') \pi(a'|x').
% \end{equation*}

In conclusion, the complete dual problem can be written as:
\begin{align}
    \label{eq:cmdp-opt}
    \max_{\rho: \X \times \A \rightarrow \mathbb{R}_{+}}  &\quad \sum_{x \in \X, a \in \A} \rho(x,a) r_0(x,a) \\  
    \texttt{s.t.}  &\quad \sum_{x \in \X, a \in \A} \rho(x,a) r_i(x,a) \leq c_i,  \; \forall i \in \{1,\dots,n-1\}, \nonumber \\
    % &\quad \rho(x',a') = \sum_{x,a} \pi(a'|x') P(x'|x,a) \rho(x,a) + (1-\gamma) \mu_{x_0}(x') \pi(a'|x'), \tag{$\forall (x',a') \in \X \times \A$}\\
    % &\quad \sum_{a'} \rho(x',a') = \sum_{x,a} P(x'|x,a) \rho(x,a) + \mu_{x_0}(x'). \tag{$\forall x' \in \X$}
    &\quad \sum_{a} \rho(x,a) = \sum_{\tilde{x},\tilde{a}} p(x|\tilde{x},\tilde{a}) \rho(\tilde{x},\tilde{a}) + \mu(x). \tag{$\forall x \in \X$}
\end{align}


The solution of the above problem $\rho^{\star}$ gives the optimal (stochastic) policy of the form:
\begin{align*}
    \piopt(a|x) &= \frac{\rho^{\star}(x,a)}{\sum_a \rho^{\star}(x,a)} , \forall x \in \X, \forall a \in \A. 
\end{align*}


% ------------------------------------------------------
%               Fixed param details
% ------------------------------------------------------
\subsection{Additional results with fixed hyper-parameters}
\label{app:cmdp-fixed-param-results}

\Cref{fig:delta-0x1-params-grid} gives the individual plots for different $\bml, \rho$ combinations corresponding to the plot in \Cref{fig:delta-params-mean}. This is the fixed parameters setting in \Cref{sec:synthetic-experiments} where the same set of parameters are used across different $\bml, \rho$ combinations. Here, we run \ref{eq:s-opt} with $\epsilon \in \{0.01, 0.1, 1.0\}$ and \ref{eq:h-opt} with Doubly Robust IS estimator \citep{jiang2015doubly} and student's t-test.  
% The failure rate is less than $\delta$ for all the cases except for $\lR=0,\lC=1,\rho=0.9$ \cref{fig:delta-params-grid}, where only \ref{eq:h-opt} ends up with a failure rate $> \delta$. The only way \ref{eq:h-opt} can violate the constraints is either due to bad IS estimator or approximate CI that are not representative of the underlying distribution. By using an exact CI like Bernstein's inequality, we can make sure the failure rate is $< \delta$, however that CI is too conservative, and always returns the baseline as the solution for any of the $\lR, \lC, \rho$ combination.
The mean results with $\delta=0.9$ can be found in \Cref{fig:extra-delta-0.9-params-mean}. A more detailed plot containing the $\bml, \rho$ wise breakdown can be found in \Cref{fig:extra-delta-0.9-params-grid}.


\begin{figure*}
  \includegraphics[width=\textwidth]{doc/figures/random-mdps/delta_0x1_grid_sem.png}
  \caption{Results on random CMDPs with fixed parameters and $\delta=0.1$. 
  The different agents are represented by different markers and color lines. 
  Each point on the grid, corresponding to a $\bml, \rho$ combination, denotes the mean (with standard error bars) for the 100 randomly generated CMDPs. 
  The x-axis denotes the amount of data the agents were trained on. They y-axis for the top subplot in a grid cell represents the improvement over baseline and the y-axis for bottom subplot in a grid cell denotes the failure rate.
  The dotted black line represents the high-confidence parameter $\delta=0.1$.
  }
  \label{fig:delta-0x1-params-grid}
\end{figure*}


\begin{figure}
    \centering
  \includegraphics[scale=0.4]{doc/figures/random-mdps/delta_0x9_mean_sem.png}
  \caption{
  Mean results on random CMDPs with fixed parameters and $\delta=0.9$. 
  The different agents are represented by different markers and color lines. 
  Each point on the plot denotes the mean (with standard error bars) for 12 different $\bml,\rho$ combinations for the 100 randomly generated CMDPs (1200 datapoints). 
  The x-axis denotes the amount of data the agents were trained on. They y-axis for the left subplot represents the improvement over baseline and the y-axis for the right subplot in a grid cell denotes the failure rate.
  The dotted black line represents the high-confidence parameter $\delta=0.9$.
  }
  \label{fig:extra-delta-0.9-params-mean}
\end{figure}


\begin{figure*}
  \includegraphics[width=\textwidth]{doc/figures/random-mdps/delta_0x9_grid_sem.png}
  \caption{
  Results on random CMDPs with fixed parameters and $\delta=0.9$. 
  The different agents are represented by different markers and color lines. 
  Each point on the grid, corresponding to a $\bml, \rho$ combination, denotes the mean (with standard error bars) for the 100 randomly generated CMDPs. 
  The x-axis denotes the amount of data the agents were trained on. They y-axis for the top subplot in a grid cell represents the improvement over baseline and the y-axis for bottom subplot in a grid cell denotes the failure rate.
  The dotted black line represents the high-confidence parameter $\delta=0.9$.
  }
  \label{fig:extra-delta-0.9-params-grid}
\end{figure*}



% ------------------------------------------------------
%               Best params
% ------------------------------------------------------
\subsection{Additional results with tuned hyper-parameters}
\label{app:cmdp-best-param-results}

\Cref{fig:best-params-grid} gives the individual plots for different $\bml, \rho$ combinations corresponding to the plot in \Cref{fig:best-params-mean}.
The best hyper-parameters are tuned in a single environment and then are used to benchmark the results on 100 random CMDPs. The following procedure is used for selecting the best hyper-parameter candidates: We first generate a random CMDP and run different hyper-parameters on that environment instance. Next, we filter the candidates that violate the safety-constraint in that CMDP instance. From the remaining candidates, we select the one that yields the highest improvement over $\pib$. 

For \ref{eq:s-opt}, we searched for $\epsilon \in \{1e^{-4}, 1e^{-3}, 1e^{-2}, 1e^{-1}, 0.5, 1.0, 2.0, 5.0\}$. For \ref{eq:h-opt}, we used student's t-test with the following $\IS$ estimators: Importance Sampling (IS), Per Decision IS (PDIS), Weighted IS, Weighted PDIS and Doubly Robust (DR) \citep{precup2000eligibility, jiang2015doubly}.




\begin{figure*}
  \includegraphics[width=\textwidth]{doc/figures/random-mdps/benchmark_best_params_delta_0x1_grid.png}
  \caption{
  Results on 100 random CMDPs for different $\bml, \rho$ combinations with best $\epsilon, \IS$ combination for $\delta=0.1$.
  The different agents are represented by different markers and color lines. 
  Each point on the grid, corresponding to a $\bml, \rho$ combination, denotes the mean (with standard error bars) for the 100 randomly generated CMDPs. 
  The x-axis denotes the amount of data the agents were trained on. They y-axis for the top subplot in a grid cell represents the improvement over baseline and the y-axis for bottom subplot in a grid cell denotes the failure rate.
  The dotted black line represents the high-confidence parameter $\delta=0.1$.
  }
  \label{fig:best-params-grid}
\end{figure*}


% Just looking from the mean across different combinations we can say that \ref{eq:s-opt} performs better on average, on average, in terms of improvement while satisfying $< \delta$ failure rate. However, from \Cref{fig:best-params-mean}, we see that choosing the best hyper-param just based on 1 run can sometimes lead to an aggressive $\epsilon$ that makes \ref{eq:s-opt} having higher failure rate $> \delta$ for a few combinations (compared to \ref{eq:h-opt} that violates it only for one combination). Maybe if we chose hyper-params smartly we can optimize further? Does that add any value?


We plot the results based on the optimized hyper-parameters for a single CMDP in \Cref{fig:10x10-gridworld-seed-0} . Here, we plot the individual performance w.r.t $r_0$ (goal reward) and $r_1$ (pit reward) for multiple agents along with the baseline's performance.  Instead of working with surrogate measures, we investigate the returns for both $\J{\pi}{r_0}{\mopt}$ and $- \J{\pi}{r_1}{\mopt}$, and see what kind of scenarios lead to violation (all the returns are normalized in $[0,1]$). In \Cref{fig:10x10-gridworld-seed-0}, the intersection of the red and blue lines denotes the performance of the baseline in the true MDP.  As we observed in the mean plots, the Linearized baseline violate most of constraints for all the dataset sizes. The Adv-Linearized baseline violates the constraints mostly for low data settings ($\blacktriangledown$ marker with darker shades). There are more violations for higher values of $\rho$ as the $\pib$ gets better and the task gets tougher. We can observe that both \ref{eq:s-opt} and \ref{eq:h-opt} based agents (denoted by $\star$ and $\blacksquare$ markers) never leave the top-left quadrant and consistently satisfy the constraints. We also observe that the deviation from the origin increases with the increase in dataset size (represented via color of the agent).

\begin{figure*}
  \includegraphics[width=\textwidth]{doc/figures/random-mdps/qual_analys_best_params.png}
  \caption{
  Results on a random $10 \times 10$ synthetic CMDP. Each $\bml$ and $\rho$ combination represents a different setting denoted by the corresponding cell in the grid. The different agents are represented by different markers and the color of the marker denotes the amount of data the agent was trained on. 
  The x-axis for individual plots are normalized $- \mathcal{J}^{\pi}_{\mopt, r_1}$ returns (for pits), and y-axis are normalized $\mathcal{J}^{\pi}_{\mopt, r_0}$ returns (for goal).
  The red line denotes the performance of the baseline w.r.t. $- \mathcal{J}^{\pib}_{\mopt, r_1}$, and the blue line for $\mathcal{J}^{\pib}_{\mopt, r_0}$. For each plot in the grid, only the points in the top-left quadrant (defined by baseline's performance via red and blue lines) satisfy the constraint for that task.
  }
  \label{fig:10x10-gridworld-seed-0}
\end{figure*}


% ------------------------------------------------------
%               Lagrangian experiments
% ------------------------------------------------------
\subsection{Comparison with \cite{le2019batch}}
\label{app:lag-baseline}



\begin{figure}[t]
\centering
\begin{subfigure}[b]{1\textwidth}
    \includegraphics[width=1\textwidth]{doc/figures/lag-baseline/fig_1.png}
    \caption{Comparisons of Lagrangian \citep{le2019batch} with $\eta = 0.01$ and MO-SPIBB (\ref{eq:s-opt}) with $\epsilon=0.1$.}
    \label{fig:lag-only-single-sopt} 
\end{subfigure}
\\
\begin{subfigure}[b]{1\textwidth}
    \includegraphics[width=1\textwidth]{doc/figures/lag-baseline/fig_all.png}
    \caption{MO-SPIBB (\cref{eq:s-opt}) and Lagrangian \citep{le2019batch} comparisons across different hyper-parameters.}
    \label{fig:lag-multiple-sopt}
\end{subfigure}
% }
\caption[]{
\small
Results on 100 random CMDPs for different $\bml$ and $\rho$ combinations with $\delta=0.1$. The different agents are represented by different markers and colored lines. Each point on the plot denotes the mean (with standard error bars) for 12 different $\bml,\rho$ combinations for the 100 randomly generated CMDPs (1200 datapoints).  The x-axis denotes the amount of data the agents were trained on. 
The y-axis for left subplot in each sub-figure represents the improvement over baseline and the right subplot denotes the failure rate. The dotted black line in the right subplots represents the high-confidence parameter $\delta=0.1$.
\Cref{fig:lag-only-single-sopt} denotes the case when MO-SPIBB (\ref{eq:s-opt}) is run with $\epsilon=0.1$, MO-HCPI (\ref{eq:h-opt}) with $\IS=$ Doubly Robust (DR) estimator with student's t-test concentration inequality, and Lagrangian \citep{le2019batch} with $\eta = 0.01$ . 
\Cref{fig:lag-multiple-sopt} shows how MO-SPIBB and Lagrangian perform across different hyper-parameters.
\label{fig:lag-combined-results}}
\vskip -0.1in
\end{figure}



We test the method by \cite{le2019batch} (henceforth referred to as Lagrangian) in the synthetic navigation CMDP task described in \Cref{sec:synthetic-experiments}. In \Cref{fig:lag-only-single-sopt}, we present the results for the best performing Lagrangian baseline on 100 random CMDPs for different $\bml$ and $\rho$ combinations with $\delta=0.1$. Similar to \Cref{fig:delta-params-mean}, we provide a more detailed plot of how the Lagrangian baseline performs with different hyper-parameters in the above setting in \Cref{fig:lag-multiple-sopt}.


% In the figure, each point on the plot denotes the mean (with standard error bars) for 12 different ,  combinations for the 10 randomly generated CMDPs (120 data points). The x-axis denotes the amount of data the agents were trained on. The y-axis for the left subplot represents the improvement over baseline and the right subplot denotes the failure rate. The dotted black line in the right subplot represents the high-confidence parameter . 
% The MO-SPIBB (\cref{eq:s-opt}) is run with $\epsilon=0.1$ and MO-HCPI (\cref{eq:h-opt}) with $\IS$ = Doubly Robust estimator with student’s t-test concentration inequality. Similar to \Cref{fig:delta-params-mean}, we provide a more detailed plot of how the Lagrangian baseline performs with different hyper-parameters in the above setting in \Cref{fig:lag-multiple-sopt}.

\textbf{Results:} As expected, we observe that the Lagrangian baseline has a high failure rate, particularly in the low-data setting. 
This makes sense as the guarantees provided by \cite{le2019batch} are of the form $\mathcal{J}^\pi_{k,m^{\star}} - \mathcal{J}^{\pi_{b}}_{k, m^{\star}} \geq - \frac{C}{(1-\gamma)^{3/2}}$ (Theorem 4.4 of \cite{le2019batch}), where $C$ is a term that depends on a constant that comes from the Concentrability assumption (Assumption 1 of \cite{le2019batch}). This assumption upper bounds the ratio between the future state-action distributions of any non-stationary policy and the baseline policy under which the dataset was generated by some constant. In other words, it makes assumptions on the quality of the data gathered under the baseline policy. Unfortunately, this assumption cannot be verified in practice, and it is unclear how to get a tractable estimate of this constant. As such, this constant can be arbitrarily large (even infinite) when the baseline policy fails to cover the support of all non-stationary policies, for instance, when the baseline policy is not exploratory enough or when the size of the dataset is small. Hence, we observe a high failure rate of \cite{le2019batch} in the experiments, especially in the low data setting. Compared to \cite{le2019batch}, our performance guarantees do not make any assumptions on the quality of the dataset or the baseline. Therefore, our approach can ensure a low failure rate even in the low-data regime.
%  in the low data setting the concentrability coefficient can be arbitrarily high, and therefore the performance guarantees provided by Le et al. do not hold anymore. As the size of the dataset increases, we observe that the failure rate of the Le et al. starts decreasing, which seems reasonable because with more data more reliable MDP parameters are estimated and the baseline policy now covers more support of the space of all non-stationary policies required for the concentrability assumption to be valid. In contrast, both the MO-SPIBB and MO-SPIBB can ensure low failure rates even in low-data scenarios.




\textbf{Implementation details and Hyper-parameters:} We build on top of the publicly available code of \cite{le2019batch} released by the authors and extend it to our setting. 
% In the accompanied code, we also provide a standalone Jupyter notebook Lagrange\_agent.ipynb that contains the implementation of Algorithm 2 of Le et al. (Section 1 of the notebook). 
We are confident that our implementation is correct as we made sure it passes various sanity tests such as convergence of the primal-dual gap and feasibility on access to true MDP parameters.

The algorithm in \cite{le2019batch} (Algorithm 2, Constrained Batch Policy Learning) requires the following hyper-parameters:

\begin{itemize}
    \item Online Learning Subroutine: We use the same online learning algorithm as used by the authors in their experiments, i.e. Exponentiated Gradient \citep{kivinen1997exponentiated}.
    
    \item Duality gap $\omega$ : This denotes the primal-dual gap or the early termination condition. We tried the values in $\set{0.01, 0.001}$ and fix the value to $0.01$.
    
    \item Number of iterations: This parameter denotes the number of iterations for which the Lagrange coefficients should be updated. We experimented in the range $\set{100, 250, 500}$ and set this to $250$.
    
    \item Norm bound $B$: The bound on the norm of Lagrange coefficients vector. We tried the values in $\set{1, 10, 50, 100}$ and fixed it $10$.
    
    \item Learning rate $\eta$: This parameter denotes the learning rate for the update of the Lagrange coefficients via the online learning subroutine. We found that this is the most sensitive variable and we tried with values in $\set{0.005, 0.01, 0.05, 0.1, 0.5, 1.0, 5.0}$. For the final experiments, we benchmark with three different values $(0.01, 0.1, 1.0)$ as mentioned in the \Cref{fig:lag-multiple-sopt}.
    
\end{itemize}

We would like to point out that the hyper-parameter tuning for the Lagrangian baseline can be particularly challenging as in the low-data setting none of the combinations of the above hyper-parameters can ensure a low failure rate even though the duality gap has converged. 

% Environments in Le et al.: Le et al. test their approach on two domains: a grid-world domain under safety constraint, and a high-dimensional car racing domain. The car racing domain takes the raw pixel image tensor as input, and as we mentioned in the limitations, it is out of the scope of our work. We would like to highlight that the grid-world domain and empirical methodology in Le et al. are considerably weaker than the approach we take in our work with respect to the safety constraints. This can be observed in Figure 2 (middle) of Le et al. where even the online-RL (equivalent to the Linearized baseline in our case) also has no constraint violation. Moreover, they do not experiment with the different sizes of the dataset  or the quality of the baseline under which the dataset was collected . Compared to that, we base our grid-world environments on the standard CMDP safety benchmarks [2,3], have comparisons against different dataset sizes and baseline quality parameters, and also explicitly calculate the failure rate.

The above experiments show the advantage of our approach over \cite{le2019batch}, particularly in the low-data safety-critical tasks, where our methods can improve over the baseline policy while ensuring a low failure rate. 






% ------------------------------------------------------
%               Scaling experiments
% ------------------------------------------------------
\subsection{Scaling experiments with number of objectives $d$}
\label{app:cmdp-scaling-experiments}

We experimented with the different number of objectives $d$ to validate if the trends we observed for \ref{eq:s-opt} and \ref{eq:h-opt} in \Cref{sec:synthetic-experiments} also extend to $d>2$. 
In the CMDP formulation, as there can only be one primary reward, we extend the CMDP to include more than 1 type of pits. The extended CMDP now has $d-1$ different kinds of pits and corresponding reward functions, where the agent gets a pit reward of $-1$ if the agent steps into a cell containing that particular kind of pit. We relax the CMDP threshold to $c_i = -10.0$ as the CMDP problem gets harder with more number of pits, and a lower threshold makes the problem of finding $\piopt$ of a random CMDP easier. Therefore, the task objective for the agent in the extended CMDP is to reach the goal in the least amount of steps, such that it can only step into at most 10 pits of every different type. 

We use the same experiment methodology from \Cref{sec:synthetic-experiments}. As the focus is to see how the trends scale with $d$, we fix the $\bml$, with $\lambda_0=1.0$ and the rest of $\lambda_{\ge 1}=0.0$.  We compare \ref{eq:s-opt} and \ref{eq:h-opt} over different $|\D|\in \{ 10, 50, 500, 2000\}$, $\rho \in \{0.1, 0.4, 0.7, 0.9\}$, the fixed set of parameters: $\IS$=DR, $\CI=$student's t-test, $\epsilon\in \{0.001, 0.01, 0.1, 1.0\}$, and $\delta=0.1$.

The results over 10 random CMDPs with fixed parameters can be found in \Cref{fig:scale-exp-10-runs-fixed-delta}. We notice that the trends from \Cref{sec:synthetic-experiments} case still carry till $d\le 1+16$, where for some value of $\epsilon$, \ref{eq:s-opt} can lead to better improvement in $\pib$ while still having failure rate $<\delta$. However, $d > 1+16$ we see there are no obvious trends and both \ref{eq:s-opt} and \ref{eq:h-opt} tend to become very conservative and returning the baseline becomes the best solution choice.


\begin{figure*}
  \includegraphics[width=\textwidth]{doc/figures/random-mdps/scale_exp_10runs_grid.png}
  \caption{Scaling with $d$ with results on a 10 random CMDPs and $\delta=0.1$. The different agents are represented by different markers and color. Each point on the graph denotes the mean for 100 runs, the standard errors is denoted by the error bars. The x-axis denotes the amount of data the agents were trained on. They y-axis for the top plot in a grid represents the improvement over baseline and the y-axis for bottom plot denotes the failure rate.
  }
  \label{fig:scale-exp-10-runs-fixed-delta}
\end{figure*}


% ------------------------------------------------------
%               Miscellaneuos details
% ------------------------------------------------------
\subsection{Additional details}

For the experiments in \Cref{sec:synthetic-experiments}, on an Intel(R) Xeon(R) Gold 6230 CPU (2.10GHz), the baselines take around 3 seconds to run, and both \ref{eq:s-opt} and \ref{eq:h-opt} take about 5 seconds.

% Sepsis extra details
\section{Additional details for sepsis experiments}
\label{app:sepsis-details}


\subsection{Sepsis data and cohort details}
\label{app:sepsis-dataset}

We followed the pre-processing methodology from \cite{tang2020clinician, komorowski2018artificial} and we refer the reader to the original work for more details. 


The dosage of prescribed IV fluids and vasopressors is converted into discrete variables to be used as actions for the constructed MDP. Each type of action (IV or vasopressor) is divided into 4 bins (each representing one quantile), and an additional action for "No drug" (0 dose) is also introduced. As such, the $|\A| = 5 \times 5$.
The cohort statistics can be found in \Cref{table:cohort-stats}. The patient data consists of 48 dimensional time-series with features representing attributes such as demographics, vitals and lab work results (\Cref{table:sepsis-features-summary}). 
The patient data is discretized into 4-hour windows, each of which is pre-processed to be treated as a single time-step. The state-space and is discretized using a k-means based clustering algorithm to map the states to $750$ clusters. Two additional absorbing states are added for death and survival ($|\X|=752$).

\begin{table}[h]
    \centering
    \caption{Cohort statistics after following the data pre-processing methodology from \cite{tang2020clinician, komorowski2018artificial}.}
    \label{table:cohort-stats}
    \vskip 0.1in
    \begin{tabular}{lrrrr}
    \toprule
     Survivors &     N & \% Female & Mean Age & Hours in ICU \\
    \midrule
     Survivors & 18066 &    44.5\% &     64.1 &         56.6 \\
    Non-survivors &  2888 &    42.9\% &     68.8 &         60.9 \\
    \bottomrule
\end{tabular}
\end{table}

% The duration of an episode consists of 28 hours data before the onset of sepsis, and 52 hours data after that, leading to a maximum of 20 time-steps for each trajectory. 

\begin{table*}[h]
    \centering
    \caption{Summary of the patient state features from \citep[][Table 3]{tang2020clinician}.}
    \label{table:sepsis-features-summary}
    \begin{tabular}{lp{9cm}}
        \toprule
        Demographics/Static &  Age, Gender, SOFA, Shock Index, Elixhauser, SIRS, Re-admission, GCS - Glasgow Coma Scale \\
        \midrule
        Lab values & Albumin, Arterial pH, Calcium, Glucose, Hemoglobin, Magnesium, PTT - Partial Thromboplastin Time, Potassium, SGPT - Serum Glutamic-Pyruvic Transaminase, Arterial Blood Gas, Blood Urea Nitrogen, Chloride, Bicarbonate, International Normalized Ratio, Sodium, Arterial Lactate, CO2, Creatinine, Ionised Calcium, Prothrombin Time, Platelets Count, SGOT - Serum Glutamic-Oxaloacetic Transaminase, Total bilirubin, White Blood Cell Count \\
        \midrule
        Vital signs & Diastolic Blood Pressure, Systolic Blood Pressure, Mean Blood Pressure, PaCO2, PaO2, FiO2, PaO/FiO2 ratio, Respiratory Rate, Temperature (Celsius), Weight (kg), Heart Rate, SpO2 \\
        \midrule
        Intake and output events & Fluid Output - 4 hourly period, Total Fluid Output, Mechanical Ventilation \\
        \bottomrule
    \end{tabular}

\end{table*}




\subsection{Performance on changing hyper-parameters}
\label{app:sepsis-hyperparams}

For the experiments in \Cref{sec:sepsis-experiments}, we treat $\delta$ as a hyper-parameter. For \ref{eq:s-opt} instead of searching over both $\delta$ and $\epsilon$, we follow the strategy proposed in Soft-SPIBB: fix the $\delta=1.0$ and only search over $\epsilon$. 
For \ref{eq:h-opt}, we found that only DR and WDR gave reliable off-policy estimates so report the results with both of them with different $\delta$. As in previous sections, we used student's t-test as the choice of concentration inequality for \ref{eq:h-opt}.



% ------------------------------------------------------
%                   S-OPT
% ------------------------------------------------------
\subsubsection{\ref{eq:s-opt} parameters}

Here, we fix $\delta=1.0$ and try with different values of the hyper-parameter $\epsilon$ and directly report the results directly on the test set. The results are presented in \Cref{table:app-s-opt-test}.


% ------------------------------------------------------
%                   H-OPT
% ------------------------------------------------------
\subsubsection{\ref{eq:h-opt} parameters}


We run with different values of the hyper-parameter $\delta$ and directly report the results directly on the test set for different $\IS$ estimators. The results for DR estimator are presented in \Cref{table:app-hopt-DR-Adv} and for WDR estimator are presented in \Cref{table:app-hopt-WDR-Adv}.




% ------------------------------------------------------
%                   Additional Details 
% ------------------------------------------------------


\subsection{Additional qualitative Analysis}
\label{app:sepsis-qual-analysis}
We calculate how many rare-actions are recommended by different solution policies and compare them with the most common actions taken by the clinicians.
For each state, for the action recommended by a solution policy, we calculate the frequency with which that state-action was observed in the training data and calculate the percentage of time that state-action pair was observed among all the possible actions taken from that state.
Across all the states, the actions suggested by the traditional single-objective RL baseline are observed only 3\% of the time on average (5.3 observations per state). Whereas, the actions most commonly chosen by the clinicians  are observed 51.4\% of the time on average (138.2 observations per state). We study this behavior for two of the policies returned by MO-SPIBB that deviate the most from the baseline: for the policy returned by \ref{eq:s-opt} ($\bml=[1,1]$) the recommended actions are observed 24.8\% of time on average (61.0 observations per state) and for  \ref{eq:s-opt} ($\bml=[0,1]$) the recommended actions are observed 23.4\% of times (56.14 observations per state).


\subsection{Additional details}

For the experiments in \Cref{sec:sepsis-experiments}, on an Intel(R) Xeon(R) Gold 6230 CPU (2.10GHz), running the Linearized baseline takes around 30 seconds, Adv-Linearized takes around 60 seconds, \ref{eq:s-opt} take about 90-120 seconds and \ref{eq:h-opt} takes about 90 seconds.


% ------------------------------------------------------
%                   Tables
% ------------------------------------------------------

% ------- S-OPT ---------
\begin{table}[h]
    \centering
    \caption{
    Performance of various \ref{eq:s-opt} policy candidates (with different $\epsilon$) using DR and WDR estimation with standard errors on 10 random splits of the TEST dataset. 
    The red cells denote the corresponding safety constraint violation, i.e, either $\mathcal{J}_{0}^{\pi} < \mathcal{J}_{0}^{\pib}$ or $-\mathcal{J}_{1}^{\pi} > -\mathcal{J}_{1}^{\pib}$.}
    \label{table:app-s-opt-test}
    \vskip 0.1in
    % \small
    % \begin{longtable}{cccccc}
    \begin{adjustbox}{max width=1\textwidth,center}
    \begin{tabular}{cccccc}
    % \begin{longtable}{Y{1cm}Y{1cm}Y{2cm}Y{2cm}Y{2cm}Y{2cm}}
    \toprule
    \multicolumn{1}{c}{User preferences $(\bml)$} & \multicolumn{1}{c}{Policy} & \multicolumn{2}{c}{Survival return ($\mathcal{J}_0$)} & \multicolumn{2}{c}{Rare-treatment return ($- \mathcal{J}_1$)} \\
    \hline
    & & DR & WDR & DR & WDR  \\  \cline{3-6}
    & Clinician's ($\pib$) & 64.78 $\pm$ 0.90 & 64.78 $\pm$ 0.90          & 13.58 $\pm$ 0.19 & 13.58 $\pm$ 0.19  \\
    \midrule
    % ------ [1, 0] -------
    \multirow{4}{*}{$[\lambda_0=1.0, \lambda_1 = 0.0]$} 
& Linearized & 97.68 $\pm$ 0.22 & 97.58 $\pm$ 0.20   & \textcolor{red}{27.64 $\pm$ 1.11 }& \textcolor{red}{27.84 $\pm$ 1.09 } \\ 
& \ref{eq:s-opt}, $\epsilon=0.0$  & 64.78 $\pm$ 0.90 & 64.78 $\pm$ 0.90   & 13.58 $\pm$ 0.19 & 13.58 $\pm$ 0.19  \\
& \ref{eq:s-opt}, $\epsilon=0.001$  & 64.91 $\pm$ 0.90 & 64.91 $\pm$ 0.90   & 13.56 $\pm$ 0.19 & 13.56 $\pm$ 0.19  \\
& \ref{eq:s-opt}, $\epsilon=0.01$  & 66.11 $\pm$ 0.87 & 66.05 $\pm$ 0.86   & 13.42 $\pm$ 0.20 & 13.46 $\pm$ 0.20  \\
& \ref{eq:s-opt}, $\epsilon=0.1$  & 73.70 $\pm$ 0.84 & 71.96 $\pm$ 0.69   & 12.30 $\pm$ 0.39 & \textcolor{red}{13.80 $\pm$ 0.33 }  \\
& \ref{eq:s-opt}, $\epsilon=0.5$  & 78.19 $\pm$ 0.54 & 81.01 $\pm$ 0.36   & \textcolor{red}{16.21 $\pm$ 0.49 }& 13.10 $\pm$ 0.31  \\
& \ref{eq:s-opt}, $\epsilon=1.0$  & 84.03 $\pm$ 0.48 & 87.11 $\pm$ 0.33   & \textcolor{red}{15.54 $\pm$ 0.59 }& 12.17 $\pm$ 0.59  \\
& \ref{eq:s-opt}, $\epsilon=2.5$  & 90.05 $\pm$ 0.25 & 91.37 $\pm$ 0.20   & \textcolor{red}{15.35 $\pm$ 0.72 }& 13.53 $\pm$ 0.56  \\
& \ref{eq:s-opt}, $\epsilon=5.0$  & 91.58 $\pm$ 0.49 & 92.66 $\pm$ 0.28   & \textcolor{red}{15.39 $\pm$ 0.59 }& \textcolor{red}{13.71 $\pm$ 0.38 }\\
& \ref{eq:s-opt}, $\epsilon=10.0$  & 91.64 $\pm$ 0.47 & 92.68 $\pm$ 0.23   & \textcolor{red}{15.19 $\pm$ 0.59 }& 13.56 $\pm$ 0.42  \\
& \ref{eq:s-opt}, $\epsilon=\infty$  & 91.62 $\pm$ 0.46 & 92.68 $\pm$ 0.23   & \textcolor{red}{15.18 $\pm$ 0.59 }& 13.56 $\pm$ 0.42  \\
    \midrule
    % --------- [1, 1] -------------
    \multirow{4}{*}{$[\lambda_0=1.0, \lambda_1 = 1.0]$}
    & Linearized & 87.17 $\pm$ 0.48 & 89.11 $\pm$ 0.37   & 2.41 $\pm$ 0.47 & 1.52 $\pm$ 0.41\\
& \ref{eq:s-opt}, $\epsilon=0.0$  & 64.78 $\pm$ 0.90 & 64.78 $\pm$ 0.90   & 13.58 $\pm$ 0.19 & 13.58 $\pm$ 0.19\\
& \ref{eq:s-opt}, $\epsilon=0.001$  & 64.90 $\pm$ 0.90 & 64.90 $\pm$ 0.90   & 13.53 $\pm$ 0.19 & 13.54 $\pm$ 0.19\\
& \ref{eq:s-opt}, $\epsilon=0.01$  & 66.02 $\pm$ 0.88 & 65.94 $\pm$ 0.87   & 13.15 $\pm$ 0.20 & 13.20 $\pm$ 0.20\\
& \ref{eq:s-opt}, $\epsilon=0.1$  & 74.34 $\pm$ 0.78 & 72.04 $\pm$ 0.87   & 9.32 $\pm$ 0.29 & 10.48 $\pm$ 0.45\\
& \ref{eq:s-opt}, $\epsilon=0.5$  & 76.47 $\pm$ 0.50 & 78.42 $\pm$ 0.41   & 7.61 $\pm$ 0.44 & 5.02 $\pm$ 0.17\\
& \ref{eq:s-opt}, $\epsilon=1.0$  & 81.39 $\pm$ 0.46 & 84.54 $\pm$ 0.36   & 4.64 $\pm$ 0.40 & 2.38 $\pm$ 0.22 \\
& \ref{eq:s-opt}, $\epsilon=2.5$  & 86.26 $\pm$ 0.33 & 88.09 $\pm$ 0.24   & 1.98 $\pm$ 0.28 & 1.14 $\pm$ 0.27  \\
& \ref{eq:s-opt}, $\epsilon=5.0$  & 86.76 $\pm$ 0.47 & 88.55 $\pm$ 0.22   & 2.52 $\pm$ 0.48 & 1.55 $\pm$ 0.41\\
& \ref{eq:s-opt}, $\epsilon=10.0$  & 86.77 $\pm$ 0.49 & 88.58 $\pm$ 0.25   & 2.53 $\pm$ 0.50 & 1.57 $\pm$ 0.43  \\
& \ref{eq:s-opt}, $\epsilon=\infty$  & 86.77 $\pm$ 0.49 & 88.58 $\pm$ 0.25   & 2.53 $\pm$ 0.50 & 1.57 $\pm$ 0.43  \\
    \midrule
    % --------- [0, 0] ------------
    \multirow{4}{*}{$[\lambda_0=0.0, \lambda_1 = 0.0]$}
    & Linearized & \textcolor{red}{-89.39 $\pm$ 0.43} & \textcolor{red}{-90.90 $\pm$ 0.29 }  & \textcolor{red}{22.99 $\pm$ 0.40 }& \textcolor{red}{22.81 $\pm$ 0.30 }  \\ 
    & \ref{eq:s-opt}, $\epsilon=0.0$  & 64.78 $\pm$ 0.90 & 64.78 $\pm$ 0.90   & 13.58 $\pm$ 0.19 & 13.58 $\pm$ 0.19\\
& \ref{eq:s-opt}, $\epsilon=0.001$  & 64.80 $\pm$ 0.90 & 64.80 $\pm$ 0.90   & 13.57 $\pm$ 0.19 & 13.57 $\pm$ 0.19\\
& \ref{eq:s-opt}, $\epsilon=0.01$  & 64.92 $\pm$ 0.90 & 64.92 $\pm$ 0.90   & 13.50 $\pm$ 0.19 & 13.51 $\pm$ 0.19\\
& \ref{eq:s-opt}, $\epsilon=0.1$  & 65.78 $\pm$ 0.89 & 65.70 $\pm$ 0.88   & 13.20 $\pm$ 0.20 & 13.25 $\pm$ 0.20  \\
& \ref{eq:s-opt}, $\epsilon=0.5$  & 67.73 $\pm$ 0.82 & 67.22 $\pm$ 0.88   & 13.24 $\pm$ 0.24 & 13.55 $\pm$ 0.33\\
& \ref{eq:s-opt}, $\epsilon=1.0$  & 69.12 $\pm$ 0.75 & 67.90 $\pm$ 0.84   & 13.57 $\pm$ 0.27 & \textcolor{red}{14.39 $\pm$ 0.44 }\\
& \ref{eq:s-opt}, $\epsilon=2.5$  & 71.00 $\pm$ 0.63 & 68.28 $\pm$ 0.46   & \textcolor{red}{14.27 $\pm$ 0.30 }& \textcolor{red}{15.73 $\pm$ 0.40 }  \\
& \ref{eq:s-opt}, $\epsilon=5.0$  & 71.95 $\pm$ 0.54 & 69.27 $\pm$ 0.63   & \textcolor{red}{15.29 $\pm$ 0.39 }& \textcolor{red}{16.12 $\pm$ 0.70 }  \\
& \ref{eq:s-opt}, $\epsilon=10.0$  & 72.73 $\pm$ 0.64 & 71.17 $\pm$ 0.65   & \textcolor{red}{16.59 $\pm$ 0.37 }& \textcolor{red}{16.21 $\pm$ 0.41 }\\
& \ref{eq:s-opt}, $\epsilon=\infty$  & \textcolor{red}{60.27 $\pm$ 0.49} & \textcolor{red}{61.44 $\pm$ 0.85 }  & \textcolor{red}{18.40 $\pm$ 0.27 }& \textcolor{red}{15.36 $\pm$ 0.58 }  \\
    \midrule
    % ------- [0,1]
    \multirow{4}{*}{$[\lambda_0=0.0, \lambda_1 = 1.0]$}
& Linearized & \textcolor{red}{58.27 $\pm$ 2.18} & \textcolor{red}{60.52 $\pm$ 2.07 }  & 0.04 $\pm$ 0.03 & 0.02 $\pm$ 0.01  \\ 
    & \ref{eq:s-opt}, $\epsilon=0.0$  & 64.78 $\pm$ 0.90 & 64.78 $\pm$ 0.90   & 13.58 $\pm$ 0.19 & 13.58 $\pm$ 0.19  \\
& \ref{eq:s-opt}, $\epsilon=0.001$  & 64.83 $\pm$ 0.90 & 64.83 $\pm$ 0.90   & 13.52 $\pm$ 0.19 & 13.52 $\pm$ 0.19  \\
 & \ref{eq:s-opt}, $\epsilon=0.01$  & 65.36 $\pm$ 0.88 & 65.27 $\pm$ 0.88   & 12.96 $\pm$ 0.19 & 13.01 $\pm$ 0.19  \\
 & \ref{eq:s-opt}, $\epsilon=0.1$  & 71.35 $\pm$ 0.96 & 69.29 $\pm$ 0.92   & 7.75 $\pm$ 0.19 & 8.30 $\pm$ 0.18\\
 & \ref{eq:s-opt}, $\epsilon=0.5$  & 71.01 $\pm$ 0.72 & 71.30 $\pm$ 0.68   & 2.54 $\pm$ 0.37 & 1.50 $\pm$ 0.11  \\
 & \ref{eq:s-opt}, $\epsilon=1.0$  & 74.19 $\pm$ 0.57 & 76.11 $\pm$ 0.57   & 0.90 $\pm$ 0.14 & 0.34 $\pm$ 0.09  \\
 & \ref{eq:s-opt}, $\epsilon=2.5$  & 76.42 $\pm$ 0.61 & 77.20 $\pm$ 0.72   & 0.10 $\pm$ 0.06 & 0.06 $\pm$ 0.04  \\
 & \ref{eq:s-opt}, $\epsilon=5.0$  & 76.08 $\pm$ 0.65 & 76.87 $\pm$ 0.74   & 0.07 $\pm$ 0.05 & 0.05 $\pm$ 0.03\\
 & \ref{eq:s-opt}, $\epsilon=10.0$  & 76.07 $\pm$ 0.65 & 76.87 $\pm$ 0.73   & 0.07 $\pm$ 0.05 & 0.04 $\pm$ 0.03\\
 & \ref{eq:s-opt}, $\epsilon=\infty$  & 76.05 $\pm$ 0.65 & 76.85 $\pm$ 0.72   & 0.07 $\pm$ 0.05 & 0.04 $\pm$ 0.03\\
 \bottomrule
    \addtocounter{table}{-1} % to decrease the counter 
    % \end{longtable}
    \end{tabular}
    \end{adjustbox}
\end{table}


% ------- DR ---------
\begin{table}[h]
    \centering
    \caption{
    Performance of various \ref{eq:h-opt} policy candidates (with different $\delta$) using $\IS=$ DR estimator with standard errors on 10 random splits of the TEST dataset. 
    The red cells denote the corresponding safety constraint violation, i.e, either $\mathcal{J}_{0}^{\pi} < \mathcal{J}_{0}^{\pib}$ or $-\mathcal{J}_{1}^{\pi} > -\mathcal{J}_{1}^{\pib}$.}
    \label{table:app-hopt-DR-Adv}
    \vskip 0.1in
    \begin{adjustbox}{max width=1\textwidth,center}
    \begin{tabular}{cccccc}
    % \begin{longtable}{cccccc}
    \toprule
    \multicolumn{1}{c}{User preferences $(\bml)$} & \multicolumn{1}{c}{Policy} & \multicolumn{2}{c}{Survival return ($\mathcal{J}_0$)} & \multicolumn{2}{c}{Rare-treatment return ($- \mathcal{J}_1$)} \\
    \hline
    & & DR & WDR & DR & WDR  \\  \cline{3-6}
    & Clinician's ($\pib$) & 64.78 $\pm$ 0.90 & 64.78 $\pm$ 0.90          & 13.58 $\pm$ 0.19 & 13.58 $\pm$ 0.19  \\
    \midrule
    % ------ [1, 0] -------
    \multirow{4}{*}{$[\lambda_0=1.0, \lambda_1 = 0.0]$} 
    & Linearized & 97.68 $\pm$ 0.22 & 97.58 $\pm$ 0.20   & \textcolor{red}{27.64 $\pm$ 1.11 }& \textcolor{red}{27.84 $\pm$ 1.09 } \\ 
    & \ref{eq:h-opt}, $\delta=0.1$  & 65.95 $\pm$ 0.00 & 65.95 $\pm$ 0.00   & 13.37 $\pm$ 0.00 & 13.37 $\pm$ 0.00\\
    & \ref{eq:h-opt},  $\delta=0.3$  & 65.95 $\pm$ 0.00 & 65.95 $\pm$ 0.00   & 13.37 $\pm$ 0.00 & 13.37 $\pm$ 0.00\\
    & \ref{eq:h-opt}, $\delta=0.5$  & 65.95 $\pm$ 0.00 & 65.95 $\pm$ 0.00   & 13.37 $\pm$ 0.00 & 13.37 $\pm$ 0.00\\
    & \ref{eq:h-opt}, $\delta=0.7$  & 65.95 $\pm$ 0.00 & 65.95 $\pm$ 0.00   & 13.37 $\pm$ 0.00 & 13.37 $\pm$ 0.00\\
    & \ref{eq:h-opt}, $\delta=0.9$  & 65.95 $\pm$ 0.00 & 65.95 $\pm$ 0.00   & 13.37 $\pm$ 0.00 & 13.37 $\pm$ 0.00\\
    \midrule
    % --------- [1, 1] -------------
    \multirow{4}{*}{$[\lambda_0=1.0, \lambda_1 = 1.0]$}
    & Linearized & 87.17 $\pm$ 0.48 & 89.11 $\pm$ 0.37   & 2.41 $\pm$ 0.47 & 1.52 $\pm$ 0.41\\
    & \ref{eq:h-opt}, $\delta=0.1$  & 86.37 $\pm$ 0.00 & 88.03 $\pm$ 0.00   & 2.58 $\pm$ 0.00 & 1.43 $\pm$ 0.00\\
    & \ref{eq:h-opt}, $\delta=0.3$  & 86.37 $\pm$ 0.00 & 88.03 $\pm$ 0.00   & 2.58 $\pm$ 0.00 & 1.43 $\pm$ 0.00\\
    & \ref{eq:h-opt}, $\delta=0.5$  & 86.37 $\pm$ 0.00 & 88.03 $\pm$ 0.00   & 2.58 $\pm$ 0.00 & 1.43 $\pm$ 0.00\\
    & \ref{eq:h-opt}, $\delta=0.7$  & 86.37 $\pm$ 0.00 & 88.03 $\pm$ 0.00   & 2.58 $\pm$ 0.00 & 1.43 $\pm$ 0.00\\
    & \ref{eq:h-opt}, $\delta=0.9$  & 86.37 $\pm$ 0.00 & 88.03 $\pm$ 0.00   & 2.58 $\pm$ 0.00 & 1.43 $\pm$ 0.00  \\
    \midrule
    % --------- [0, 0] ------------
    \multirow{4}{*}{$[\lambda_0=0.0, \lambda_1 = 0.0]$}
    & Linearized & \textcolor{red}{-89.39 $\pm$ 0.43} & \textcolor{red}{-90.90 $\pm$ 0.29 }  & \textcolor{red}{22.99 $\pm$ 0.40 }& \textcolor{red}{22.81 $\pm$ 0.30 }  \\ 
    & \ref{eq:h-opt}, $\delta=0.1$  & 65.95 $\pm$ 0.00 & 65.95 $\pm$ 0.00   & 13.37 $\pm$ 0.00 & 13.37 $\pm$ 0.00  \\
    & \ref{eq:h-opt}, $\delta=0.3$  & 65.95 $\pm$ 0.00 & 65.95 $\pm$ 0.00   & 13.37 $\pm$ 0.00 & 13.37 $\pm$ 0.00\\
    & \ref{eq:h-opt}, $\delta=0.5$  & 65.95 $\pm$ 0.00 & 65.95 $\pm$ 0.00   & 13.37 $\pm$ 0.00 & 13.37 $\pm$ 0.00\\
    & \ref{eq:h-opt}, $\delta=0.7$  & 68.28 $\pm$ 0.00 & \textcolor{red}{63.25 $\pm$ 0.00 }  & \textcolor{red}{14.16 $\pm$ 0.00 }& \textcolor{red}{16.41 $\pm$ 0.00 } \\
    & \ref{eq:h-opt}, $\delta=0.9$  & 68.28 $\pm$ 0.00 & \textcolor{red}{63.25 $\pm$ 0.00 }  & \textcolor{red}{14.16 $\pm$ 0.00 }& \textcolor{red}{16.41 $\pm$ 0.00 }\\
    \midrule
    % ------- [0,1]
    \multirow{4}{*}{$[\lambda_0=0.0, \lambda_1 = 1.0]$}
    & Linearized & \textcolor{red}{58.27 $\pm$ 2.18} & \textcolor{red}{60.52 $\pm$ 2.07 }  & 0.04 $\pm$ 0.03 & 0.02 $\pm$ 0.01  \\ 
    & \ref{eq:h-opt}, $\delta=0.1$  & 76.54 $\pm$ 0.00 & 77.55 $\pm$ 0.00   & 0.09 $\pm$ 0.00 & 0.05 $\pm$ 0.00\\
    & \ref{eq:h-opt}, $\delta=0.3$  & 76.54 $\pm$ 0.00 & 77.55 $\pm$ 0.00   & 0.09 $\pm$ 0.00 & 0.05 $\pm$ 0.00\\
    & \ref{eq:h-opt}, $\delta=0.5$  & 76.54 $\pm$ 0.00 & 77.55 $\pm$ 0.00   & 0.09 $\pm$ 0.00 & 0.05 $\pm$ 0.00\\
    & \ref{eq:h-opt}, $\delta=0.7$  & 76.54 $\pm$ 0.00 & 77.55 $\pm$ 0.00   & 0.09 $\pm$ 0.00 & 0.05 $\pm$ 0.00\\
    & \ref{eq:h-opt}, $\delta=0.9$  & 76.54 $\pm$ 0.00 & 77.55 $\pm$ 0.00   & 0.09 $\pm$ 0.00 & 0.05 $\pm$ 0.00  \\
    \bottomrule
    \addtocounter{table}{-1} % to decrease the counter 
    % \end{longtable}
    \end{tabular}
    \end{adjustbox}
\end{table}




% ------- WDR ------------
\begin{table}[h]
    \centering
    \caption{
    Performance of various \ref{eq:h-opt} policy candidates (with different $\delta$) using $\IS=$Weighed DR (WDR) estimator with standard errors on 10 random splits of the TEST dataset. 
    The red cells denote the corresponding safety constraint violation, i.e, either $\mathcal{J}_{0}^{\pi} < \mathcal{J}_{0}^{\pib}$ or $-\mathcal{J}_{1}^{\pi} > -\mathcal{J}_{1}^{\pib}$.}
    \label{table:app-hopt-WDR-Adv}
    \vskip 0.1in
    \begin{adjustbox}{max width=1\textwidth,center}
    \begin{tabular}{cccccc}
    \toprule
    \multicolumn{1}{c}{User preferences $(\bml)$} & \multicolumn{1}{c}{Policy} & \multicolumn{2}{c}{Survival return ($\mathcal{J}_0$)} & \multicolumn{2}{c}{Rare-treatment return ($- \mathcal{J}_1$)} \\
    \hline
    & & DR & WDR & DR & WDR  \\  \cline{3-6}
    & Clinician's ($\pib$) & 64.78 $\pm$ 0.90 & 64.78 $\pm$ 0.90          & 13.58 $\pm$ 0.19 & 13.58 $\pm$ 0.19  \\
    \midrule
    % ------ [1, 0] -------
    \multirow{4}{*}{$[\lambda_0=1.0, \lambda_1 = 0.0]$} 
    & Linearized & 97.68 $\pm$ 0.22 & 97.58 $\pm$ 0.20   & \textcolor{red}{27.64 $\pm$ 1.11 }& \textcolor{red}{27.84 $\pm$ 1.09 } \\ 
    & \ref{eq:h-opt}, $\delta=0.1$  & 65.95 $\pm$ 0.00 & 65.95 $\pm$ 0.00   & 13.37 $\pm$ 0.00 & 13.37 $\pm$ 0.00\\
    & \ref{eq:h-opt}, $\delta=0.3$  & 65.95 $\pm$ 0.00 & 65.95 $\pm$ 0.00   & 13.37 $\pm$ 0.00 & 13.37 $\pm$ 0.00\\
    & \ref{eq:h-opt}, $\delta=0.5$  & 65.95 $\pm$ 0.00 & 65.95 $\pm$ 0.00   & 13.37 $\pm$ 0.00 & 13.37 $\pm$ 0.00\\
    & \ref{eq:h-opt}, $\delta=0.7$  & 65.95 $\pm$ 0.00 & 65.95 $\pm$ 0.00   & 13.37 $\pm$ 0.00 & 13.37 $\pm$ 0.00\\
    & \ref{eq:h-opt}, $\delta=0.9$  & 91.39 $\pm$ 0.00 & 92.61 $\pm$ 0.00   & \textcolor{red}{15.41 $\pm$ 0.00 }& \textcolor{red}{13.89 $\pm$ 0.00 } \\
    \midrule
    % --------- [1, 1] -------------
    \multirow{4}{*}{$[\lambda_0=1.0, \lambda_1 = 1.0]$}
    & Linearized & 87.17 $\pm$ 0.48 & 89.11 $\pm$ 0.37   & 2.41 $\pm$ 0.47 & 1.52 $\pm$ 0.41\\
    & \ref{eq:h-opt}, $\delta=0.1$  & 86.37 $\pm$ 0.00 & 88.03 $\pm$ 0.00   & 2.58 $\pm$ 0.00 & 1.43 $\pm$ 0.00\\
    & \ref{eq:h-opt}, $\delta=0.3$  & 86.37 $\pm$ 0.00 & 88.03 $\pm$ 0.00   & 2.58 $\pm$ 0.00 & 1.43 $\pm$ 0.00\\
    & \ref{eq:h-opt}, $\delta=0.5$  & 86.37 $\pm$ 0.00 & 88.03 $\pm$ 0.00   & 2.58 $\pm$ 0.00 & 1.43 $\pm$ 0.00  \\
    & \ref{eq:h-opt}, $\delta=0.7$  & 86.37 $\pm$ 0.00 & 88.03 $\pm$ 0.00   & 2.58 $\pm$ 0.00 & 1.43 $\pm$ 0.00\\
    & \ref{eq:h-opt}, $\delta=0.9$  & 86.37 $\pm$ 0.00 & 88.03 $\pm$ 0.00   & 2.58 $\pm$ 0.00 & 1.43 $\pm$ 0.00\\
    \midrule
    % --------- [0, 0] ------------
    \multirow{4}{*}{$[\lambda_0=0.0, \lambda_1 = 0.0]$}
    & Linearized & \textcolor{red}{-89.39 $\pm$ 0.43} & \textcolor{red}{-90.90 $\pm$ 0.29 }  & \textcolor{red}{22.99 $\pm$ 0.40 }& \textcolor{red}{22.81 $\pm$ 0.30 }  \\ 
    & \ref{eq:h-opt}, $\delta=0.1$  & 65.95 $\pm$ 0.00 & 65.95 $\pm$ 0.00   & 13.37 $\pm$ 0.00 & 13.37 $\pm$ 0.00  \\
    & \ref{eq:h-opt}, $\delta=0.3$  & 65.95 $\pm$ 0.00 & 65.95 $\pm$ 0.00   & 13.37 $\pm$ 0.00 & 13.37 $\pm$ 0.00\\
    & \ref{eq:h-opt}, $\delta=0.5$  & 65.95 $\pm$ 0.00 & 65.95 $\pm$ 0.00   & 13.37 $\pm$ 0.00 & 13.37 $\pm$ 0.00  \\
    & \ref{eq:h-opt}, $\delta=0.7$  & 65.95 $\pm$ 0.00 & 65.95 $\pm$ 0.00   & 13.37 $\pm$ 0.00 & 13.37 $\pm$ 0.00\\
    & \ref{eq:h-opt}, $\delta=0.9$  & 65.95 $\pm$ 0.00 & 65.95 $\pm$ 0.00   & 13.37 $\pm$ 0.00 & 13.37 $\pm$ 0.00  \\
    \midrule
    % ------- [0,1]
    \multirow{4}{*}{$[\lambda_0=0.0, \lambda_1 = 1.0]$}
    & Linearized & \textcolor{red}{58.27 $\pm$ 2.18} & \textcolor{red}{60.52 $\pm$ 2.07 }  & 0.04 $\pm$ 0.03 & 0.02 $\pm$ 0.01  \\ 
    & \ref{eq:h-opt}, $\delta=0.1$  & 76.54 $\pm$ 0.00 & 77.55 $\pm$ 0.00   & 0.09 $\pm$ 0.00 & 0.05 $\pm$ 0.00\\
    & \ref{eq:h-opt}, $\delta=0.3$  & 76.54 $\pm$ 0.00 & 77.55 $\pm$ 0.00   & 0.09 $\pm$ 0.00 & 0.05 $\pm$ 0.00\\
    & \ref{eq:h-opt}, $\delta=0.5$  & 76.54 $\pm$ 0.00 & 77.55 $\pm$ 0.00   & 0.09 $\pm$ 0.00 & 0.05 $\pm$ 0.00\\
    & \ref{eq:h-opt}, $\delta=0.7$  & 76.54 $\pm$ 0.00 & 77.55 $\pm$ 0.00   & 0.09 $\pm$ 0.00 & 0.05 $\pm$ 0.00\\
    & \ref{eq:h-opt}, $\delta=0.9$  & 76.54 $\pm$ 0.00 & 77.55 $\pm$ 0.00   & 0.09 $\pm$ 0.00 & 0.05 $\pm$ 0.00\\
    \bottomrule
    \addtocounter{table}{-1} % to decrease the counter 
    \end{tabular}
    \end{adjustbox}
\end{table}





%%%%%%%%%%%%%%%%%%%%%%%%%%%%%%%%%%%%%%%%%%%%%%%%%%%%%%%%%%%%%%%%%%%%%%%%%%%%%%%
%%%%%%%%%%%%%%%%%%%%%%%%%%%%%%%%%%%%%%%%%%%%%%%%%%%%%%%%%%%%%%%%%%%%%%%%%%%%%%%


\end{document}

