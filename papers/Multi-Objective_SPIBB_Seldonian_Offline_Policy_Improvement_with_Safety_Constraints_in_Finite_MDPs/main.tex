%%%%%%%% ICML 2021 EXAMPLE LATEX SUBMISSION FILE %%%%%%%%%%%%%%%%%

\documentclass{article}

% Recommended, but optional, packages for figures and better typesetting:
\usepackage{microtype}
\usepackage{graphicx}
% \usepackage{subfigure} % has conflict with subfig/subcaption
\usepackage{booktabs} % for professional tables

% hyperref makes hyperlinks in the resulting PDF.
% If your build breaks (sometimes temporarily if a hyperlink spans a page)
% please comment out the following usepackage line and replace
\usepackage{hyperref}

% -------- Custom packages 


\usepackage{url}            % simple URL typesetting
\usepackage{amsfonts}       % blackboard math symbols
\usepackage{amsthm}
\usepackage{mathrsfs}
\usepackage{nicefrac}       % compact symbols 
\usepackage{microtype}      % microtypography
\usepackage{booktabs} % for professional tables

\usepackage[utf8]{inputenc} % allow utf-8 input
\usepackage[T1]{fontenc}    % use 8-bit T1 fonts

\usepackage{tikz}
\usepackage{amsmath}
\usepackage{amssymb}
\usepackage{graphicx}
\usepackage{tabularx}
\usepackage{mathtools}
\usepackage{tcolorbox}
\usepackage{cancel}
\usepackage{pifont} % for checkmarks and cross

% Use the give algorithmic style files isntead?
%\usepackage{algorithm2e}
% \usepackage[linesnumbered,lined,algoruled,boxed,commentsnumbered]{algorithm2e}
% \usepackage{algpseudocode,algorithm,algorithmicx}

\usepackage{multirow}
\usepackage{color}
\usepackage{mathtools}
% \usepackage{todonotes}
\usepackage{wrapfig}
\usepackage{caption}
\usepackage{bbm}
\usepackage{bm}
\usepackage{esvect}
\usepackage[normalem]{ulem}
% \usepackage{subcaption}

% \usepackage[lite]{mtpro2}


\usepackage{xcolor} %
\usepackage{xargs}
\usepackage{rotating}
\usepackage{longtable}
\usepackage{adjustbox}

\usepackage{cleveref}
% \usepackage{subfig}
\usepackage{subcaption}
\usepackage{enumitem}

% \captionsetup[subfigure]{subrefformat=simple,labelformat=simple}
% \renewcommand\thesubfigure{(\alph{subfigure})}

% overwrite cref name for section for saving space, format: crefname{content-type}{singlur}{plural}
% \crefname{section}{Sec.}{Sec.}
% \crefname{equation}{Eq.}{Eq.}
% \crefname{figure}{Fig.}{Fig.}


\newcolumntype{Y}{>{\raggedright\arraybackslash}p}
	
\usepackage{contour}
\usepackage{ulem}

\contourlength{1pt}

\renewcommand{\ULdepth}{1.8pt}
\contourlength{0.8pt}

\newcommand{\myuline}[1]{%
  \uline{\phantom{#1}}%
  \llap{\contour{white}{#1}}%
}


% footnotes
\newcommand\rlf[1]{{\color{red}\footnote{{\color{red}From Romain: #1}}}}

% inline commands for notes
\newcommand\rli[1]{{\color{red} #1}}
\newcommand\romain[1]{{\color{red} #1}}
\newcommand\joelle[1]{{\color{green} #1}}
\newcommand\phil[1]{{\color{cyan} #1}}

\newcommand\harsh[1]{{\color{blue} #1}}
\newcommand\htodo[1]{{\color{blue} TODO: #1\\}}
\newcommand\hnote[1]{{\color{blue} NOTE: #1\\}}


% Attempt to make hyperref and algorithmic work together better:
\newcommand{\theHalgorithm}{\arabic{algorithm}}


%SLJ: standard prettier hypersetup:
\hypersetup{
	plainpages=false,
	%pdfpagelabels=true,          % Prevents duplicate labels
	colorlinks=true,              % Use link colors
	linkcolor=blue,               % Color for normal internal links
	anchorcolor=blue,             % Color for anchor text
	citecolor=blue,               % Color for bibliographical citations in text
	filecolor=blue,               % Color for URLs which open local files
	pagecolor=blue,               % Color for links to other pages
	urlcolor=blue,                % Color for linked URLs
	pdfview=FitH,                 % Fit the width of the document
	pdfstartview=FitH,            % Fit the width of the document
	pdfpagelayout=SinglePage      % Page-down goes to next page
}


%%%%% NEW MATH DEFINITIONS %%%%%

\usepackage{amsmath,amsfonts,bm}

% Mark sections of captions for referring to divisions of figures
\newcommand{\figleft}{{\em (Left)}}
\newcommand{\figcenter}{{\em (Center)}}
\newcommand{\figright}{{\em (Right)}}
\newcommand{\figtop}{{\em (Top)}}
\newcommand{\figbottom}{{\em (Bottom)}}
\newcommand{\captiona}{{\em (a)}}
\newcommand{\captionb}{{\em (b)}}
\newcommand{\captionc}{{\em (c)}}
\newcommand{\captiond}{{\em (d)}}

% Highlight a newly defined term
\newcommand{\newterm}[1]{{\bf #1}}


% Figure reference, lower-case.
\def\figref#1{figure~\ref{#1}}
% Figure reference, capital. For start of sentence
\def\Figref#1{Figure~\ref{#1}}
\def\twofigref#1#2{figures \ref{#1} and \ref{#2}}
\def\quadfigref#1#2#3#4{figures \ref{#1}, \ref{#2}, \ref{#3} and \ref{#4}}
% Section reference, lower-case.
\def\secref#1{section~\ref{#1}}
% Section reference, capital.
\def\Secref#1{Section~\ref{#1}}
% Reference to two sections.
\def\twosecrefs#1#2{sections \ref{#1} and \ref{#2}}
% Reference to three sections.
\def\secrefs#1#2#3{sections \ref{#1}, \ref{#2} and \ref{#3}}
% Reference to an equation, lower-case.
% \def\eqref#1{equation~\ref{#1}}
 \def\eqref#1{(\ref{#1})}
% Reference to an equation, upper case
\def\Eqref#1{Equation~\ref{#1}}
% A raw reference to an equation---avoid using if possible
\def\plaineqref#1{\ref{#1}}
% Reference to a chapter, lower-case.
\def\chapref#1{chapter~\ref{#1}}
% Reference to an equation, upper case.
\def\Chapref#1{Chapter~\ref{#1}}
% Reference to a range of chapters
\def\rangechapref#1#2{chapters\ref{#1}--\ref{#2}}
% Reference to an algorithm, lower-case.
\def\algref#1{algorithm~\ref{#1}}
% Reference to an algorithm, upper case.
\def\Algref#1{Algorithm~\ref{#1}}
\def\twoalgref#1#2{algorithms \ref{#1} and \ref{#2}}
\def\Twoalgref#1#2{Algorithms \ref{#1} and \ref{#2}}
% Reference to a part, lower case
\def\partref#1{part~\ref{#1}}
% Reference to a part, upper case
\def\Partref#1{Part~\ref{#1}}
\def\twopartref#1#2{parts \ref{#1} and \ref{#2}}

\def\ceil#1{\lceil #1 \rceil}
\def\floor#1{\lfloor #1 \rfloor}
\def\1{\bm{1}}
\newcommand{\train}{\mathcal{D}}
\newcommand{\valid}{\mathcal{D_{\mathrm{valid}}}}
\newcommand{\test}{\mathcal{D_{\mathrm{test}}}}

\def\eps{{\epsilon}}


% Random variables
\def\reta{{\textnormal{$\eta$}}}
\def\ra{{\textnormal{a}}}
\def\rb{{\textnormal{b}}}
\def\rc{{\textnormal{c}}}
\def\rd{{\textnormal{d}}}
\def\re{{\textnormal{e}}}
\def\rf{{\textnormal{f}}}
\def\rg{{\textnormal{g}}}
\def\rh{{\textnormal{h}}}
\def\ri{{\textnormal{i}}}
\def\rj{{\textnormal{j}}}
\def\rk{{\textnormal{k}}}
\def\rl{{\textnormal{l}}}
% rm is already a command, just don't name any random variables m
\def\rn{{\textnormal{n}}}
\def\ro{{\textnormal{o}}}
\def\rp{{\textnormal{p}}}
\def\rq{{\textnormal{q}}}
\def\rr{{\textnormal{r}}}
\def\rs{{\textnormal{s}}}
\def\rt{{\textnormal{t}}}
\def\ru{{\textnormal{u}}}
\def\rv{{\textnormal{v}}}
\def\rw{{\textnormal{w}}}
\def\rx{{\textnormal{x}}}
\def\ry{{\textnormal{y}}}
\def\rz{{\textnormal{z}}}

% Random vectors
\def\rvepsilon{{\mathbf{\epsilon}}}
\def\rvtheta{{\mathbf{\theta}}}
\def\rva{{\mathbf{a}}}
\def\rvb{{\mathbf{b}}}
\def\rvc{{\mathbf{c}}}
\def\rvd{{\mathbf{d}}}
\def\rve{{\mathbf{e}}}
\def\rvf{{\mathbf{f}}}
\def\rvg{{\mathbf{g}}}
\def\rvh{{\mathbf{h}}}
\def\rvu{{\mathbf{i}}}
\def\rvj{{\mathbf{j}}}
\def\rvk{{\mathbf{k}}}
\def\rvl{{\mathbf{l}}}
\def\rvm{{\mathbf{m}}}
\def\rvn{{\mathbf{n}}}
\def\rvo{{\mathbf{o}}}
\def\rvp{{\mathbf{p}}}
\def\rvq{{\mathbf{q}}}
\def\rvr{{\mathbf{r}}}
\def\rvs{{\mathbf{s}}}
\def\rvt{{\mathbf{t}}}
\def\rvu{{\mathbf{u}}}
\def\rvv{{\mathbf{v}}}
\def\rvw{{\mathbf{w}}}
\def\rvx{{\mathbf{x}}}
\def\rvy{{\mathbf{y}}}
\def\rvz{{\mathbf{z}}}

% Elements of random vectors
\def\erva{{\textnormal{a}}}
\def\ervb{{\textnormal{b}}}
\def\ervc{{\textnormal{c}}}
\def\ervd{{\textnormal{d}}}
\def\erve{{\textnormal{e}}}
\def\ervf{{\textnormal{f}}}
\def\ervg{{\textnormal{g}}}
\def\ervh{{\textnormal{h}}}
\def\ervi{{\textnormal{i}}}
\def\ervj{{\textnormal{j}}}
\def\ervk{{\textnormal{k}}}
\def\ervl{{\textnormal{l}}}
\def\ervm{{\textnormal{m}}}
\def\ervn{{\textnormal{n}}}
\def\ervo{{\textnormal{o}}}
\def\ervp{{\textnormal{p}}}
\def\ervq{{\textnormal{q}}}
\def\ervr{{\textnormal{r}}}
\def\ervs{{\textnormal{s}}}
\def\ervt{{\textnormal{t}}}
\def\ervu{{\textnormal{u}}}
\def\ervv{{\textnormal{v}}}
\def\ervw{{\textnormal{w}}}
\def\ervx{{\textnormal{x}}}
\def\ervy{{\textnormal{y}}}
\def\ervz{{\textnormal{z}}}

% Random matrices
\def\rmA{{\mathbf{A}}}
\def\rmB{{\mathbf{B}}}
\def\rmC{{\mathbf{C}}}
\def\rmD{{\mathbf{D}}}
\def\rmE{{\mathbf{E}}}
\def\rmF{{\mathbf{F}}}
\def\rmG{{\mathbf{G}}}
\def\rmH{{\mathbf{H}}}
\def\rmI{{\mathbf{I}}}
\def\rmJ{{\mathbf{J}}}
\def\rmK{{\mathbf{K}}}
\def\rmL{{\mathbf{L}}}
\def\rmM{{\mathbf{M}}}
\def\rmN{{\mathbf{N}}}
\def\rmO{{\mathbf{O}}}
\def\rmP{{\mathbf{P}}}
\def\rmQ{{\mathbf{Q}}}
\def\rmR{{\mathbf{R}}}
\def\rmS{{\mathbf{S}}}
\def\rmT{{\mathbf{T}}}
\def\rmU{{\mathbf{U}}}
\def\rmV{{\mathbf{V}}}
\def\rmW{{\mathbf{W}}}
\def\rmX{{\mathbf{X}}}
\def\rmY{{\mathbf{Y}}}
\def\rmZ{{\mathbf{Z}}}

% Elements of random matrices
\def\ermA{{\textnormal{A}}}
\def\ermB{{\textnormal{B}}}
\def\ermC{{\textnormal{C}}}
\def\ermD{{\textnormal{D}}}
\def\ermE{{\textnormal{E}}}
\def\ermF{{\textnormal{F}}}
\def\ermG{{\textnormal{G}}}
\def\ermH{{\textnormal{H}}}
\def\ermI{{\textnormal{I}}}
\def\ermJ{{\textnormal{J}}}
\def\ermK{{\textnormal{K}}}
\def\ermL{{\textnormal{L}}}
\def\ermM{{\textnormal{M}}}
\def\ermN{{\textnormal{N}}}
\def\ermO{{\textnormal{O}}}
\def\ermP{{\textnormal{P}}}
\def\ermQ{{\textnormal{Q}}}
\def\ermR{{\textnormal{R}}}
\def\ermS{{\textnormal{S}}}
\def\ermT{{\textnormal{T}}}
\def\ermU{{\textnormal{U}}}
\def\ermV{{\textnormal{V}}}
\def\ermW{{\textnormal{W}}}
\def\ermX{{\textnormal{X}}}
\def\ermY{{\textnormal{Y}}}
\def\ermZ{{\textnormal{Z}}}

% Vectors
\def\vzero{{\bm{0}}}
\def\vone{{\bm{1}}}
\def\vmu{{\bm{\mu}}}
\def\vtheta{{\bm{\theta}}}
\def\va{{\bm{a}}}
\def\vb{{\bm{b}}}
\def\vc{{\bm{c}}}
\def\vd{{\bm{d}}}
\def\ve{{\bm{e}}}
\def\vf{{\bm{f}}}
\def\vg{{\bm{g}}}
\def\vh{{\bm{h}}}
\def\vi{{\bm{i}}}
\def\vj{{\bm{j}}}
\def\vk{{\bm{k}}}
\def\vl{{\bm{l}}}
\def\vm{{\bm{m}}}
\def\vn{{\bm{n}}}
\def\vo{{\bm{o}}}
\def\vp{{\bm{p}}}
\def\vq{{\bm{q}}}
\def\vr{{\bm{r}}}
\def\vs{{\bm{s}}}
\def\vt{{\bm{t}}}
\def\vu{{\bm{u}}}
\def\vv{{\bm{v}}}
\def\vw{{\bm{w}}}
\def\vx{{\bm{x}}}
\def\vy{{\bm{y}}}
\def\vz{{\bm{z}}}

% Elements of vectors
\def\evalpha{{\alpha}}
\def\evbeta{{\beta}}
\def\evepsilon{{\epsilon}}
\def\evlambda{{\lambda}}
\def\evomega{{\omega}}
\def\evmu{{\mu}}
\def\evpsi{{\psi}}
\def\evsigma{{\sigma}}
\def\evtheta{{\theta}}
\def\eva{{a}}
\def\evb{{b}}
\def\evc{{c}}
\def\evd{{d}}
\def\eve{{e}}
\def\evf{{f}}
\def\evg{{g}}
\def\evh{{h}}
\def\evi{{i}}
\def\evj{{j}}
\def\evk{{k}}
\def\evl{{l}}
\def\evm{{m}}
\def\evn{{n}}
\def\evo{{o}}
\def\evp{{p}}
\def\evq{{q}}
\def\evr{{r}}
\def\evs{{s}}
\def\evt{{t}}
\def\evu{{u}}
\def\evv{{v}}
\def\evw{{w}}
\def\evx{{x}}
\def\evy{{y}}
\def\evz{{z}}

% Matrix
\def\mA{{\bm{A}}}
\def\mB{{\bm{B}}}
\def\mC{{\bm{C}}}
\def\mD{{\bm{D}}}
\def\mE{{\bm{E}}}
\def\mF{{\bm{F}}}
\def\mG{{\bm{G}}}
\def\mH{{\bm{H}}}
\def\mI{{\bm{I}}}
\def\mJ{{\bm{J}}}
\def\mK{{\bm{K}}}
\def\mL{{\bm{L}}}
\def\mM{{\bm{M}}}
\def\mN{{\bm{N}}}
\def\mO{{\bm{O}}}
\def\mP{{\bm{P}}}
\def\mQ{{\bm{Q}}}
\def\mR{{\bm{R}}}
\def\mS{{\bm{S}}}
\def\mT{{\bm{T}}}
\def\mU{{\bm{U}}}
\def\mV{{\bm{V}}}
\def\mW{{\bm{W}}}
\def\mX{{\bm{X}}}
\def\mY{{\bm{Y}}}
\def\mZ{{\bm{Z}}}
\def\mBeta{{\bm{\beta}}}
\def\mPhi{{\bm{\Phi}}}
\def\mLambda{{\bm{\Lambda}}}
\def\mSigma{{\bm{\Sigma}}}

% Tensor
\DeclareMathAlphabet{\mathsfit}{\encodingdefault}{\sfdefault}{m}{sl}
\SetMathAlphabet{\mathsfit}{bold}{\encodingdefault}{\sfdefault}{bx}{n}
\newcommand{\tens}[1]{\bm{\mathsfit{#1}}}
\def\tA{{\tens{A}}}
\def\tB{{\tens{B}}}
\def\tC{{\tens{C}}}
\def\tD{{\tens{D}}}
\def\tE{{\tens{E}}}
\def\tF{{\tens{F}}}
\def\tG{{\tens{G}}}
\def\tH{{\tens{H}}}
\def\tI{{\tens{I}}}
\def\tJ{{\tens{J}}}
\def\tK{{\tens{K}}}
\def\tL{{\tens{L}}}
\def\tM{{\tens{M}}}
\def\tN{{\tens{N}}}
\def\tO{{\tens{O}}}
\def\tP{{\tens{P}}}
\def\tQ{{\tens{Q}}}
\def\tR{{\tens{R}}}
\def\tS{{\tens{S}}}
\def\tT{{\tens{T}}}
\def\tU{{\tens{U}}}
\def\tV{{\tens{V}}}
\def\tW{{\tens{W}}}
\def\tX{{\tens{X}}}
\def\tY{{\tens{Y}}}
\def\tZ{{\tens{Z}}}


% Graph
\def\gA{{\mathcal{A}}}
\def\gB{{\mathcal{B}}}
\def\gC{{\mathcal{C}}}
\def\gD{{\mathcal{D}}}
\def\gE{{\mathcal{E}}}
\def\gF{{\mathcal{F}}}
\def\gG{{\mathcal{G}}}
\def\gH{{\mathcal{H}}}
\def\gI{{\mathcal{I}}}
\def\gJ{{\mathcal{J}}}
\def\gK{{\mathcal{K}}}
\def\gL{{\mathcal{L}}}
\def\gM{{\mathcal{M}}}
\def\gN{{\mathcal{N}}}
\def\gO{{\mathcal{O}}}
\def\gP{{\mathcal{P}}}
\def\gQ{{\mathcal{Q}}}
\def\gR{{\mathcal{R}}}
\def\gS{{\mathcal{S}}}
\def\gT{{\mathcal{T}}}
\def\gU{{\mathcal{U}}}
\def\gV{{\mathcal{V}}}
\def\gW{{\mathcal{W}}}
\def\gX{{\mathcal{X}}}
\def\gY{{\mathcal{Y}}}
\def\gZ{{\mathcal{Z}}}

% Sets
\def\sA{{\mathbb{A}}}
\def\sB{{\mathbb{B}}}
\def\sC{{\mathbb{C}}}
\def\sD{{\mathbb{D}}}
% Don't use a set called E, because this would be the same as our symbol
% for expectation.
\def\sF{{\mathbb{F}}}
\def\sG{{\mathbb{G}}}
\def\sH{{\mathbb{H}}}
\def\sI{{\mathbb{I}}}
\def\sJ{{\mathbb{J}}}
\def\sK{{\mathbb{K}}}
\def\sL{{\mathbb{L}}}
\def\sM{{\mathbb{M}}}
\def\sN{{\mathbb{N}}}
\def\sO{{\mathbb{O}}}
\def\sP{{\mathbb{P}}}
\def\sQ{{\mathbb{Q}}}
\def\sR{{\mathbb{R}}}
\def\sS{{\mathbb{S}}}
\def\sT{{\mathbb{T}}}
\def\sU{{\mathbb{U}}}
\def\sV{{\mathbb{V}}}
\def\sW{{\mathbb{W}}}
\def\sX{{\mathbb{X}}}
\def\sY{{\mathbb{Y}}}
\def\sZ{{\mathbb{Z}}}

% Entries of a matrix
\def\emLambda{{\Lambda}}
\def\emA{{A}}
\def\emB{{B}}
\def\emC{{C}}
\def\emD{{D}}
\def\emE{{E}}
\def\emF{{F}}
\def\emG{{G}}
\def\emH{{H}}
\def\emI{{I}}
\def\emJ{{J}}
\def\emK{{K}}
\def\emL{{L}}
\def\emM{{M}}
\def\emN{{N}}
\def\emO{{O}}
\def\emP{{P}}
\def\emQ{{Q}}
\def\emR{{R}}
\def\emS{{S}}
\def\emT{{T}}
\def\emU{{U}}
\def\emV{{V}}
\def\emW{{W}}
\def\emX{{X}}
\def\emY{{Y}}
\def\emZ{{Z}}
\def\emSigma{{\Sigma}}

% entries of a tensor
% Same font as tensor, without \bm wrapper
\newcommand{\etens}[1]{\mathsfit{#1}}
\def\etLambda{{\etens{\Lambda}}}
\def\etA{{\etens{A}}}
\def\etB{{\etens{B}}}
\def\etC{{\etens{C}}}
\def\etD{{\etens{D}}}
\def\etE{{\etens{E}}}
\def\etF{{\etens{F}}}
\def\etG{{\etens{G}}}
\def\etH{{\etens{H}}}
\def\etI{{\etens{I}}}
\def\etJ{{\etens{J}}}
\def\etK{{\etens{K}}}
\def\etL{{\etens{L}}}
\def\etM{{\etens{M}}}
\def\etN{{\etens{N}}}
\def\etO{{\etens{O}}}
\def\etP{{\etens{P}}}
\def\etQ{{\etens{Q}}}
\def\etR{{\etens{R}}}
\def\etS{{\etens{S}}}
\def\etT{{\etens{T}}}
\def\etU{{\etens{U}}}
\def\etV{{\etens{V}}}
\def\etW{{\etens{W}}}
\def\etX{{\etens{X}}}
\def\etY{{\etens{Y}}}
\def\etZ{{\etens{Z}}}

% The true underlying data generating distribution
\newcommand{\pdata}{p_{\rm{data}}}
% The empirical distribution defined by the training set
\newcommand{\ptrain}{\hat{p}_{\rm{data}}}
\newcommand{\Ptrain}{\hat{P}_{\rm{data}}}
% The model distribution
\newcommand{\pmodel}{p_{\rm{model}}}
\newcommand{\Pmodel}{P_{\rm{model}}}
\newcommand{\ptildemodel}{\tilde{p}_{\rm{model}}}
% Stochastic autoencoder distributions
\newcommand{\pencode}{p_{\rm{encoder}}}
\newcommand{\pdecode}{p_{\rm{decoder}}}
\newcommand{\precons}{p_{\rm{reconstruct}}}

\newcommand{\laplace}{\mathrm{Laplace}} % Laplace distribution

\newcommand{\E}{\mathbb{E}}
\newcommand{\Ls}{\mathcal{L}}
\newcommand{\R}{\mathbb{R}}
\newcommand{\emp}{\tilde{p}}
\newcommand{\lr}{\alpha}
\newcommand{\reg}{\lambda}
\newcommand{\rect}{\mathrm{rectifier}}
\newcommand{\softmax}{\mathrm{softmax}}
\newcommand{\sigmoid}{\sigma}
\newcommand{\softplus}{\zeta}
\newcommand{\KL}{D_{\mathrm{KL}}}
\newcommand{\Var}{\mathrm{Var}}
\newcommand{\standarderror}{\mathrm{SE}}
\newcommand{\Cov}{\mathrm{Cov}}
% Wolfram Mathworld says $L^2$ is for function spaces and $\ell^2$ is for vectors
% But then they seem to use $L^2$ for vectors throughout the site, and so does
% wikipedia.
\newcommand{\normlzero}{L^0}
\newcommand{\normlone}{L^1}
\newcommand{\normltwo}{L^2}
\newcommand{\normlp}{L^p}
\newcommand{\normmax}{L^\infty}

\newcommand{\parents}{Pa} % See usage in notation.tex. Chosen to match Daphne's book.

\DeclareMathOperator*{\argmax}{arg\,max}
\DeclareMathOperator*{\argmin}{arg\,min}

\DeclareMathOperator{\sign}{sign}
\DeclareMathOperator{\Tr}{Tr}
\let\ab\allowbreak

\newcommand{\norm}[2]{\left\| #1 \right\|_{#2}}

\newcommand{\zz}[1]{\textcolor{blue}{ [{\em Zhihui:} #1]}}
\newcommand{\jz}[1]{\textcolor{red}{ [{\em JZ:} #1]}}
% \newcommand{\td}[1]{\textcolor{blue}{ [{\em TD:} #1]}}
\newcommand{\jj}[1]{\textcolor{pink}{ [{\em JJ:} #1]}}


% ----------------------------------------------------------------
%               Conference specific packages 
% ----------------------------------------------------------------

% Use the following line for the initial blind version submitted for review:
\usepackage[final]{doc/neurips_2021}

\title{Multi-Objective SPIBB: Seldonian Offline Policy Improvement with Safety Constraints in Finite MDPs}

% ------ update running title also


\author{%
  Harsh Satija\\
  McGill University, Mila \\
  \texttt{harsh.satija@mail.mcgill.ca} \\
  % examples of more authors
   \And
  Philip S. Thomas \\ 
  University of Massachusetts \\ 
  \texttt{pthomas@cs.umass.edu} \\
   \AND
   Joelle Pineau \\
   McGill University, Mila, Facebook AI Research \\
   \texttt{jpineau@cs.mcgill.ca} \\
   \And
   Romain Laroche \\ 
   Microsoft Research \\
   \texttt{romain.laroche@microsoft.com}
  % \And
  % Coauthor \\
  % Affiliation \\
  % Address \\
  % \texttt{email} \\
}


\begin{document}

\maketitle

% ------------------------------------------------------------
%               Abstract
% ------------------------------------------------------------


\begin{abstract}
We study the problem of Safe Policy Improvement (SPI) under constraints in the offline Reinforcement Learning (RL) setting. We consider the scenario where: (i) we have a dataset collected under a known baseline policy, (ii) multiple reward signals are received from the environment inducing as many objectives to optimize. 
We present an SPI formulation for this RL setting that takes into account the preferences of the algorithm's user for handling the trade-offs for different reward signals while ensuring that the new policy performs at least as well as the baseline policy along each individual objective. 
We build on traditional SPI algorithms and propose a novel method based on Safe Policy Iteration with Baseline Bootstrapping~\citep[SPIBB,][]{laroche2017safe} 
that provides high probability guarantees on the performance of the agent in the true environment.
We show the effectiveness of our method on a synthetic grid-world safety task as well as in a real-world critical care context to learn a policy for the administration of IV fluids and vasopressors to treat sepsis.

\end{abstract}



% ------------------------------------------------------------
%               MAIN BODY  
% ------------------------------------------------------------
% INTRODUCTION
% \leavevmode
% \\
% \\
% \\
% \\
% \\
\section{Introduction}
\label{introduction}

AutoML is the process by which machine learning models are built automatically for a new dataset. Given a dataset, AutoML systems perform a search over valid data transformations and learners, along with hyper-parameter optimization for each learner~\cite{VolcanoML}. Choosing the transformations and learners over which to search is our focus.
A significant number of systems mine from prior runs of pipelines over a set of datasets to choose transformers and learners that are effective with different types of datasets (e.g. \cite{NEURIPS2018_b59a51a3}, \cite{10.14778/3415478.3415542}, \cite{autosklearn}). Thus, they build a database by actually running different pipelines with a diverse set of datasets to estimate the accuracy of potential pipelines. Hence, they can be used to effectively reduce the search space. A new dataset, based on a set of features (meta-features) is then matched to this database to find the most plausible candidates for both learner selection and hyper-parameter tuning. This process of choosing starting points in the search space is called meta-learning for the cold start problem.  

Other meta-learning approaches include mining existing data science code and their associated datasets to learn from human expertise. The AL~\cite{al} system mined existing Kaggle notebooks using dynamic analysis, i.e., actually running the scripts, and showed that such a system has promise.  However, this meta-learning approach does not scale because it is onerous to execute a large number of pipeline scripts on datasets, preprocessing datasets is never trivial, and older scripts cease to run at all as software evolves. It is not surprising that AL therefore performed dynamic analysis on just nine datasets.

Our system, {\sysname}, provides a scalable meta-learning approach to leverage human expertise, using static analysis to mine pipelines from large repositories of scripts. Static analysis has the advantage of scaling to thousands or millions of scripts \cite{graph4code} easily, but lacks the performance data gathered by dynamic analysis. The {\sysname} meta-learning approach guides the learning process by a scalable dataset similarity search, based on dataset embeddings, to find the most similar datasets and the semantics of ML pipelines applied on them.  Many existing systems, such as Auto-Sklearn \cite{autosklearn} and AL \cite{al}, compute a set of meta-features for each dataset. We developed a deep neural network model to generate embeddings at the granularity of a dataset, e.g., a table or CSV file, to capture similarity at the level of an entire dataset rather than relying on a set of meta-features.
 
Because we use static analysis to capture the semantics of the meta-learning process, we have no mechanism to choose the \textbf{best} pipeline from many seen pipelines, unlike the dynamic execution case where one can rely on runtime to choose the best performing pipeline.  Observing that pipelines are basically workflow graphs, we use graph generator neural models to succinctly capture the statically-observed pipelines for a single dataset. In {\sysname}, we formulate learner selection as a graph generation problem to predict optimized pipelines based on pipelines seen in actual notebooks.

%. This formulation enables {\sysname} for effective pruning of the AutoML search space to predict optimized pipelines based on pipelines seen in actual notebooks.}
%We note that increasingly, state-of-the-art performance in AutoML systems is being generated by more complex pipelines such as Directed Acyclic Graphs (DAGs) \cite{piper} rather than the linear pipelines used in earlier systems.  
 
{\sysname} does learner and transformation selection, and hence is a component of an AutoML systems. To evaluate this component, we integrated it into two existing AutoML systems, FLAML \cite{flaml} and Auto-Sklearn \cite{autosklearn}.  
% We evaluate each system with and without {\sysname}.  
We chose FLAML because it does not yet have any meta-learning component for the cold start problem and instead allows user selection of learners and transformers. The authors of FLAML explicitly pointed to the fact that FLAML might benefit from a meta-learning component and pointed to it as a possibility for future work. For FLAML, if mining historical pipelines provides an advantage, we should improve its performance. We also picked Auto-Sklearn as it does have a learner selection component based on meta-features, as described earlier~\cite{autosklearn2}. For Auto-Sklearn, we should at least match performance if our static mining of pipelines can match their extensive database. For context, we also compared {\sysname} with the recent VolcanoML~\cite{VolcanoML}, which provides an efficient decomposition and execution strategy for the AutoML search space. In contrast, {\sysname} prunes the search space using our meta-learning model to perform hyperparameter optimization only for the most promising candidates. 

The contributions of this paper are the following:
\begin{itemize}
    \item Section ~\ref{sec:mining} defines a scalable meta-learning approach based on representation learning of mined ML pipeline semantics and datasets for over 100 datasets and ~11K Python scripts.  
    \newline
    \item Sections~\ref{sec:kgpipGen} formulates AutoML pipeline generation as a graph generation problem. {\sysname} predicts efficiently an optimized ML pipeline for an unseen dataset based on our meta-learning model.  To the best of our knowledge, {\sysname} is the first approach to formulate  AutoML pipeline generation in such a way.
    \newline
    \item Section~\ref{sec:eval} presents a comprehensive evaluation using a large collection of 121 datasets from major AutoML benchmarks and Kaggle. Our experimental results show that {\sysname} outperforms all existing AutoML systems and achieves state-of-the-art results on the majority of these datasets. {\sysname} significantly improves the performance of both FLAML and Auto-Sklearn in classification and regression tasks. We also outperformed AL in 75 out of 77 datasets and VolcanoML in 75  out of 121 datasets, including 44 datasets used only by VolcanoML~\cite{VolcanoML}.  On average, {\sysname} achieves scores that are statistically better than the means of all other systems. 
\end{itemize}


%This approach does not need to apply cleaning or transformation methods to handle different variances among datasets. Moreover, we do not need to deal with complex analysis, such as dynamic code analysis. Thus, our approach proved to be scalable, as discussed in Sections~\ref{sec:mining}.

% RELATED WORK       
\section{Related Work}
%\mz{We lack a comparison to this paper: https://arxiv.org/abs/2305.14877}
%\anirudh{refine to be more on-topic?}
\iffalse
\paragraph{In-Context Learning} As language models have scaled, the ability to learn in-context, without any weight updates, has emerged. \cite{brown2020language}. While other families of large language models have emerged, in-context learning remains ubiquitous \cite{llama, bloom, gptneo, opt}. Although such as HELM \cite{helm} have arisen for systematic evaluation of \emph{models}, there is no systematic framework to our knowledge for evaluating \emph{prompting methods}, and validating prompt engineering heuristics. The test-suite we propose will ensure that progress in the field of prompt-engineering is structured and objectively evaluated. 

\paragraph{Prompt Engineering Methods} Researchers are interested in the automatic design of high performing instructions for downstream tasks. Some focus on simple heuristics, such as selecting instructions that have the lowest perplexity \cite{lowperplexityprompts}. Other methods try to use large language models to induce an instruction when provided with a few input-output pairs \cite{ape}. Researchers have also used RL objectives to create discrete token sequences that can serve as instructions \cite{rlprompt}. Since the datasets and models used in these works have very little intersection, it is impossible to compare these methods objectively and glean insights. In our work, we evaluate these three methods on a diverse set of tasks and models, and analyze their relative performance. Additionally, we recognize that there are many other interesting angles of prompting that are not covered by instruction engineering \cite{weichain, react, selfconsistency}, but we leave these to future work.

\paragraph{Analysis of Prompting Methods} While most prompt engineering methods focus on accuracy, there are many other interesting dimensions of performance as well. For instance, researchers have found that for most tasks, the selection of demonstrations plays a large role in few-shot accuracy \cite{whatmakesgoodicexamples, selectionmachinetranslation, knnprompting}. Additionally, many researchers have found that even permuting the ordering of a fixed set of demonstrations has a significant effect on downstream accuracy \cite{fantasticallyorderedprompts}. Prompts that are sensitive to the permutation of demonstrations have been shown to also have lower accuracies \cite{relationsensitivityaccuracy}. Especially in low-resource domains, which includes the large public usage of in-context learning, these large swings in accuracy make prompting less dependable. In our test-suite we include sensitivity metrics that go beyond accuracy and allow us to find methods that are not only performant but reliable.

\paragraph{Existing Benchmarks} We recognize that other holistic in-context learning benchmarks exist. BigBench is a large benchmark of 204 tasks that are beyond the capabilities of current LLMs. BigBench seeks to evaluate the few-shot abilities of state of the art large language models, focusing on performance metrics such as accuracy \cite{bigbench}. Similarly, HELM is another benchmark for language model in-context learning ability. Rather than only focusing on performance, HELM branches out and considers many other metrics such as robustness and bias \cite{helm}. Both BigBench and HELM focus on ranking different language model, while fix a generic instruction and prompt format. We instead choose to evaluate instruction induction / selection methods over a fixed set of models. We are the first ever evaluation script that compares different prompt-engineering methods head to head. 
\fi

\paragraph{In-Context Learning and Existing Benchmarks} As language models have scaled, in-context learning has emerged as a popular paradigm and remains ubiquitous among several autoregressive LLM families \cite{brown2020language, llama, bloom, gptneo, opt}. Benchmarks like BigBench \cite{bigbench} and HELM \cite{helm} have been created for the holistic evaluation of these models. BigBench focuses on few-shot abilities of state-of-the-art large language models, while HELM extends to consider metrics like robustness and bias. However, these benchmarks focus on evaluating and ranking \emph{language models}, and do not address the systematic evaluation of \emph{prompting methods}. Although contemporary work by \citet{yang2023improving} also aims to perform a similar systematic analysis of prompting methods, they focus on simple probability-based prompt selection while we evaluate a broader range of methods including trivial instruction baselines, curated manually selected instructions, and sophisticated automated instruction selection.

\paragraph{Automated Prompt Engineering Methods} There has been interest in performing automated prompt-engineering for target downstream tasks within ICL. This has led to the exploration of various prompting methods, ranging from simple heuristics such as selecting instructions with the lowest perplexity \cite{lowperplexityprompts}, inducing instructions from large language models using a few annotated input-output pairs \cite{ape}, to utilizing RL objectives to create discrete token sequences as prompts \cite{rlprompt}. However, these works restrict their evaluation to small sets of models and tasks with little intersection, hindering their objective comparison. %\mz{For paragraphs that only have one work in the last line, try to shorten the paragraph to squeeze in context.}

\paragraph{Understanding in-context learning} There has been much recent work attempting to understand the mechanisms that drive in-context learning. Studies have found that the selection of demonstrations included in prompts significantly impacts few-shot accuracy across most tasks \cite{whatmakesgoodicexamples, selectionmachinetranslation, knnprompting}. Works like \cite{fantasticallyorderedprompts} also show that altering the ordering of a fixed set of demonstrations can affect downstream accuracy. Prompts sensitive to demonstration permutation often exhibit lower accuracies \cite{relationsensitivityaccuracy}, making them less reliable, particularly in low-resource domains.

Our work aims to bridge these gaps by systematically evaluating the efficacy of popular instruction selection approaches over a diverse set of tasks and models, facilitating objective comparison. We evaluate these methods not only on accuracy metrics, but also on sensitivity metrics to glean additional insights. We recognize that other facets of prompting not covered by instruction engineering exist \cite{weichain, react, selfconsistency}, and defer these explorations to future work. 

% BACKGROUND and METHODOLOGY
\section{Background and Motivation}

\subsection{IBM Streams}

IBM Streams is a general-purpose, distributed stream processing system. It
allows users to develop, deploy and manage long-running streaming applications
which require high-throughput and low-latency online processing.

The IBM Streams platform grew out of the research work on the Stream Processing
Core~\cite{spc-2006}.  While the platform has changed significantly since then,
that work established the general architecture that Streams still follows today:
job, resource and graph topology management in centralized services; processing
elements (PEs) which contain user code, distributed across all hosts,
communicating over typed input and output ports; brokers publish-subscribe
communication between jobs; and host controllers on each host which
launch PEs on behalf of the platform.

The modern Streams platform approaches general-purpose cluster management, as
shown in Figure~\ref{fig:streams_v4_v6}. The responsibilities of the platform
services include all job and PE life cycle management; domain name resolution
between the PEs; all metrics collection and reporting; host and resource
management; authentication and authorization; and all log collection. The
platform relies on ZooKeeper~\cite{zookeeper} for consistent, durable metadata
storage which it uses for fault tolerance.

Developers write Streams applications in SPL~\cite{spl-2017} which is a
programming language that presents streams, operators and tuples as
abstractions. Operators continuously consume and produce tuples over streams.
SPL allows programmers to write custom logic in their operators, and to invoke
operators from existing toolkits. Compiled SPL applications become archives that
contain: shared libraries for the operators; graph topology metadata which tells
both the platform and the SPL runtime how to connect those operators; and
external dependencies. At runtime, PEs contain one or more operators. Operators
inside of the same PE communicate through function calls or queues. Operators
that run in different PEs communicate over TCP connections that the PEs
establish at startup. PEs learn what operators they contain, and how to connect
to operators in other PEs, at startup from the graph topology metadata provided
by the platform.

We use ``legacy Streams'' to refer to the IBM Streams version 4 family. The
version 5 family is for Kubernetes, but is not cloud native. It uses the
lift-and-shift approach and creates a platform-within-a-platform: it deploys a
containerized version of the legacy Streams platform within Kubernetes.

\subsection{Kubernetes}

Borg~\cite{borg-2015} is a cluster management platform used internally at Google
to schedule, maintain and monitor the applications their internal infrastructure
and external applications depend on. Kubernetes~\cite{kube} is the open-source
successor to Borg that is an industry standard cloud orchestration platform.

From a user's perspective, Kubernetes abstracts running a distributed
application on a cluster of machines. Users package their applications into
containers and deploy those containers to Kubernetes, which runs those
containers in \emph{pods}. Kubernetes handles all life cycle management of pods,
including scheduling, restarting and migration in case of failures.

Internally, Kubernetes tracks all entities as \emph{objects}~\cite{kubeobjects}.
All objects have a name and a specification that describes its desired state.
Kubernetes stores objects in etcd~\cite{etcd}, making them persistent,
highly-available and reliably accessible across the cluster. Objects are exposed
to users through \emph{resources}. All resources can have
\emph{controllers}~\cite{kubecontrollers}, which react to changes in resources.
For example, when a user changes the number of replicas in a
\code{ReplicaSet}, it is the \code{ReplicaSet} controller which makes sure the
desired number of pods are running. Users can extend Kubernetes through
\emph{custom resource definitions} (CRDs)~\cite{kubecrd}. CRDs can contain
arbitrary content, and controllers for a CRD can take any kind of action.

Architecturally, a Kubernetes cluster consists of nodes. Each node runs a
\emph{kubelet} which receives pod creation requests and makes sure that the
requisite containers are running on that node. Nodes also run a
\emph{kube-proxy} which maintains the network rules for that node on behalf of
the pods. The \emph{kube-api-server} is the central point of contact: it
receives API requests, stores objects in etcd, asks the scheduler to schedule
pods, and talks to the kubelets and kube-proxies on each node. Finally,
\emph{namespaces} logically partition the cluster. Objects which should not know
about each other live in separate namespaces, which allows them to share the
same physical infrastructure without interference.

\subsection{Motivation}
\label{sec:motivation}

Systems like Kubernetes are commonly called ``container orchestration''
platforms. We find that characterization reductive to the point of being
misleading; no one would describe operating systems as ``binary executable
orchestration.'' We adopt the idea from Verma et al.~\cite{borg-2015} that
systems like Kubernetes are ``the kernel of a distributed system.'' Through CRDs
and their controllers, Kubernetes provides state-as-a-service in a distributed
system. Architectures like the one we propose are the result of taking that view 
seriously.

The Streams legacy platform has obvious parallels to the Kubernetes
architecture, and that is not a coincidence: they solve similar problems.
Both are designed to abstract running arbitrary user-code across a distributed
system.  We suspect that Streams is not unique, and that there are many
non-trivial platforms which have to provide similar levels of cluster
management.  The benefits to being cloud native and offloading the platform
to an existing cloud management system are: 
\begin{itemize}
    \item Significantly less platform code.
    \item Better scheduling and resource management, as all services on the cluster are 
        scheduled by one platform.
    \item Easier service integration.
    \item Standardized management, logging and metrics.
\end{itemize}
The rest of this paper presents the design of replacing the legacy Streams 
platform with Kubernetes itself.


% \section{Evaluation}
\label{sec:eval}

This section evaluates the performance, area and power of the \ZF architecture demonstrating how it improves over the state-of-the-art  DaDianNao accelerator~\cite{DaDiannao}. 
Section~\ref{sec:eval:method} details the experimental methodology. 
Section~\ref{sec:eval:performance} evaluates the performance of \ZF. 
Sections~\ref{sec:eval:area} and \ref{sec:eval:power} evaluate the area and power of \ZF, and Section~\ref{sec:add-ineffectual} considers the removal of non-zero neurons.

%
%

%

\subsection{Methodology}
\label{sec:eval:method}


%
%
\begin{table}[t!]
\centering
\begin{tabular}{|l|l|l|}
\hline
\textbf{Network} & \pbox{5cm}{\textbf{Conv.} \\ \textbf{Layers}} & \textbf{Source} \\ \hline \hline
alex     	 & 5 & Caffe: bvlc\_reference\_caffenet \\ \hline
google 	 & 59 & Caffe: bvlc\_googlenet \\ \hline
nin 		 & 12 & Model Zoo: NIN-imagenet \\ \hline
vgg19  	 & 16 & Model Zoo: VGG 19-layer \\ \hline
cnnM  & 5 & Model Zoo: VGG\_CNN\_M\_2048 \\ \hline
cnnS  & 5 & Model Zoo: VGG\_CNN\_S \\ \hline
\end{tabular}
\caption{Networks used}
\label{table:networks}
\end{table}

The evaluation uses the set of popular~\cite{AlexNIPS2012}, and state-of-the-art convolutional neural networks~\cite{ILSVRC15}\cite{nin}\cite{vgg}\cite{vgg19} shown in Table \ref{table:networks}. 
These networks perform image classification on the ILSVRC12 dataset~\cite{ILSVRC15}, which contains $256\times256$ images across 1000 classes. 
The experiments use a randomly selected set of 1000 images, one from each class. The networks are available, pre-trained for Caffe, either as part of the distribution or at the Caffe Model Zoo~\cite{model-zoo}.

%
%
We created a cycle accurate simulator of the baseline accelerator and \ZF. 
The simulator integrates with the Caffe framework~\cite{caffe} to enable on-the-fly validation of the layer ouput neurons. 
%
%
%
%
The area and power characteristics of \ZF and \BASE are measured 
with synthesized implementations. The two designs are implemented 
in Verilog and synthesized via the Synopsis Design 
Compiler~\cite{synopsys_site} with the TSMC 65nm library. 
The NBin, NBout, and \ZF offset SRAM buffers were modeled using 
the Artisan single-ported register file memory compiler~\cite{artisan} 
using double-pumping to allow a read and write per cycle. The eDRAM 
area and energy was modeled with \textit{Destiny}~\cite{destiny}.
%



%
%
%
%
%

%
%

%
%
%
%

%
%
%
%
%
%
%
%
%
%
%
%
%


% SYNTHETIC EXPERIMENTS
% -------------------------------------------------------
%               Tabular experiments
% -------------------------------------------------------
\section{Synthetic Experiments}
\label{sec:synthetic-experiments}

The main benefits of working in a synthetic domain are: (i) we can evaluate the performance on the true MDP instead of relying on off-policy evaluation (OPE) methods, (ii) we have control over the quality of the dataset. We test both 
MO-SPIBB (\ref{eq:s-opt}) and MO-HCPI (\ref{eq:h-opt}) on a variety of parameters: the amount of data, quality of baseline and different user reward scalarizations. 

% ------------ Env description ---------------
\textbf{Env details:} 
We take a standard CMDP benchmark \citep{leike2017ai, chow2018lyapunov} which consists of a $10\times10$ grid. From any state, the agent can move to the adjoining cells in the 4 directions using the 4 actions. 
The transitions are stochastic, with some probability $\alpha$ (generated randomly for each state-action for every environment instance) the agent is successfully able to reach the next state, and with $(1-\alpha)$ the agent stays in the current state. 
The agent starts at the bottom-right corner, and the goal is to reach the opposite corner (top-left). The pits are spawned randomly with some uniform probability ($\eta_{pit}=0.3$) for each cell.  
The reward vector consists of two rewards signals. A primary reward $r_0$ that is related the goal and is +1000.0 on reaching the goal and -1.0 at every other time-step. The secondary reward $r_1$ is related to pits, for which the agent gets -1.0 for any action taken in the pit.
The constraint threshold for this CMDP is $-2.0$ and $\gamma = 0.99$. Maximum length of an episode is $200$ steps. Therefore, the task objective is to reach the goal in the least number of steps, such that the agent does not spend more than $2$ time-steps in the pit cells. 

% ------------ Dataset generation ---------------
\textbf{Dataset collection procedure:} 
For every random CMDP generated, we first find the optimal policy $\piopt$ by using the procedure described in \Cref{app:cmdp-solver}. The baseline policy is generated using a convex combination of the optimal policy and a uniform random policy ($\pi_{rand}$), i.e., $\pib = \rho \piopt + (1 - \rho) \pi_{rand}$, where  $\rho$ controls how close $\pib$'s performance is to $\piopt$. Different datasets with varying sizes and $\rho$ are then collected under $\pib$ and given as input to the methods.

% ------------ Baselines ---------------
\textbf{Baselines:} 
We compare against the following baselines:
\begin{itemize}[leftmargin=*, topsep=0pt]
    \item \myuline{Linearized}: This baseline transforms the rewards into a single scalar using $\bml$ and then applies the traditional policy improvement methods on the linearized objective, i.e,  $\argmax_{\pi \in \Pi} \J{\pi}{\bml}{\mhat}$.
    
    % \item \myuline{Adv-Linearized}: This method has the same objective as the Linearized baseline, with the additional constraints based on advantage estimators built from $\mhat$:
    % \begin{align}
    %     \argmax_{\pi \in \Pi} &\langle \pi(\cdot|x) \qval{\pi}{\bml}{\mhat}{x,\cdot} \rangle \quad  \forall x \in \X \\  
    %     \text{s.t.} \quad    
    %     &\forall i\in [d], \; \sum_{a \in \A} \pi(a|x) \adv{\pib}{i}{\mhat}{x, a} \geq 0 \nonumber. 
    % \end{align}
    \item \myuline{Adv-Linearized}: This method has the same objective as the Linearized baseline, with the additional constraints based on advantage estimators built from $\mhat$, i.e. $\forall x \in \X$:
    % \begin{align}
    %     \argmax_{\pi \in \Pi} \langle \pi(\cdot|x) , \qval{\pi}{\bml}{\mhat}{x,\cdot} \rangle \quad   
    %     &\text{s.t.} \quad    
    %     \forall k\in [d], \; \sum_{a \in \A} \pi(a|x) \adv{\pib}{k}{\mhat}{x, a} \geq 0 .
    % \end{align}
    \begin{align}
        \argmax_{\pi \in \Pi} &\langle \pi(\cdot|x) , \qval{\pi}{\bml}{\mhat}{x,\cdot} \rangle \\   
        \text{s.t.} \quad    
        &\forall k\in [d], \; \sum_{a \in \A} \pi(a|x) \adv{\pib}{k}{\mhat}{x, a} \geq 0 . \nonumber
    \end{align}
    
    
    
\end{itemize}


% ------------ Evaluation ---------------
\textbf{Evaluation:} 
Using $\mopt$, we can directly calculate the returns for any solution policy. Only tracking the scalarized objective can be misleading, so we track the following metrics:
\begin{itemize}[leftmargin=*, topsep=0pt,]
    \item \myuline{Improvement over $\pib$}: This denotes the difference between the scalarized return of the solution policy and the baseline policy, i.e., $\J{\pi}{\bml}{\mopt} - \J{\pib}{\bml}{\mopt}$.
    Mean improvement over $\pib$ captures on average improvement over $\pib$ in terms of the scalarized objective.
    
    \item \myuline{Failure-rate:} 
    The failure rate over $n$ runs captures the number of times, on average, the solution policy ends up violating the safety constraints in \Cref{eq:general-safety-constraints}, and thus performs worse than the baseline. In the context of this task, safety constraints are violated if either the agent takes longer to reach the goal, or it steps into more number of pits compared to $\pib$.
\end{itemize}

We test on different combinations of user preference $(\bml)$ and baseline's quality $(\rho)$ on 100 randomly generated CMDPs, where $\lambda_i \in \{0, 1\}$, $\rho \in \{0.1, 0.4, 0.7, 0.9\}$ and %the number of trajectories 
$|D| \in \{ 10, 50, 500, 2000\}$.
We evaluate under two settings: 
(i) we use a fixed set of parameters across different $(\bml, \rho)$ combinations, where we run \ref{eq:s-opt} with $\epsilon \in \{0.01, 0.1, 1.0\}$ and \ref{eq:h-opt} with Doubly Robust IS estimator \citep{jiang2015doubly} and Student’s t-test concentration inequality; 
(ii) we treat them as hyper-parameters that can be optimized for a particular $(\bml,\rho)$ combination. The best hyper-parameters are tuned in a single environment instance and then they are used to benchmark the results on 100 random CMDPs. 
% More details about the range of hyper-parameters considered are given in \Cref{app:cmdp-best-param-results}.


% ------------ Combined figure ---------------

% \begin{figure}[t]
% \makebox[1\linewidth][c]{%
% \centering
% % \begin{subfigure}[b]{0.48\columnwidth}
% \begin{subfigure}[b]{0.5\textwidth}
%     \includegraphics[width=1\textwidth]{doc/figures/random-mdps/delta_01_latex_nf.pdf}
%     \caption{Fixed hyper-parameters}
%     \label{fig:delta-params-mean} 
% \end{subfigure}
% \hfill
% \begin{subfigure}[b]{0.5\textwidth}
%     \includegraphics[width=1\textwidth]{doc/figures/random-mdps/bench_latex_nf.pdf}
%     \caption{Optimized hyper-parameters.}
%     \label{fig:best-params-mean}
% \end{subfigure}
% }
% \caption[]{
% \small
% Results on 100 random CMDPs for different $\bml$ and $\rho$ combinations with $\delta=0.1$. The different agents are represented by different markers and colored lines. Each point on the plot denotes the mean (with standard error bars) for 12 different $\bml,\rho$ combinations for the 100 randomly generated CMDPs (1200 datapoints). 
% The x-axis denotes the amount of data the agents were trained on. 
% The y-axis for left subplot in each sub-figure represents the improvement over baseline and the right subplot denotes the failure rate. The dotted black line in the right subplots represents the high-confidence parameter $\delta=0.1$.
% \Cref{fig:delta-params-mean} denotes when the hyper-parameters are fixed $\epsilon=\{0.01, 0.1, 1.0\}$ and $\IS=$ Doubly Robust (DR) estimator with student's t-test concentration inequality. 
% \Cref{fig:best-params-mean} is the version with tuned hyper-parameters for each combination.
% \label{fig:cmdp-combined-results}}
% \vskip -0.1in
% \end{figure}

\begin{figure}[t]
\centering
% \begin{subfigure}[b]{0.48\columnwidth}
\begin{subfigure}[b]{0.7\textwidth}
    \includegraphics[width=1\textwidth]{doc/figures/random-mdps/delta_01_latex_nf.pdf}
    \caption{Fixed hyper-parameters}
    \label{fig:delta-params-mean} 
\end{subfigure}
\hfill
\begin{subfigure}[b]{0.7\textwidth}
    \includegraphics[width=1\textwidth]{doc/figures/random-mdps/bench_latex_nf.pdf}
    \caption{Optimized hyper-parameters.}
    \label{fig:best-params-mean}
\end{subfigure}
\caption[]{
\small
Results on 100 random CMDPs for different $\bml$ and $\rho$ combinations with $\delta=0.1$. The different agents are represented by different markers and colored lines. Each point on the plot denotes the mean (with standard error bars) for 12 different $\bml,\rho$ combinations for the 100 randomly generated CMDPs (1200 datapoints). 
The x-axis denotes the amount of data the agents were trained on. 
The y-axis for left subplot in each sub-figure represents the improvement over baseline and the right subplot denotes the failure rate. The dotted black line in the right subplots represents the high-confidence parameter $\delta=0.1$.
\Cref{fig:delta-params-mean} denotes when the hyper-parameters are fixed $\epsilon=\{0.01, 0.1, 1.0\}$ and $\IS=$ Doubly Robust (DR) estimator with student's t-test concentration inequality. 
\Cref{fig:best-params-mean} is the version with tuned hyper-parameters for each combination.
\label{fig:cmdp-combined-results}}
\vskip -0.1in
\end{figure}


% ------------ Results ---------------
\textbf{Results:} 
The mean results with fixed parameters and $\delta=0.1$ can be found in \Cref{fig:delta-params-mean}. 
% Tell changes with data size for the condensed figure
The high failure rate of Linearized baseline, regardless of the size of the dataset, is expected as it optimizes the scalarized reward directly and is agnostic of the individual rewards. Adv-Linearized performs better, but in the low data-regime, we see a high failure rate that eventually decreases as the size of dataset increases. This is expected because with more data, more reliable advantage functions estimates are calculated that are representative of the underlying CMDP. 
Compared to the baselines, both \ref{eq:s-opt} and \ref{eq:h-opt} maintain a failure rate below the required confidence parameter $\delta$, regardless of the amount of data.
Also, as the size of dataset increases, we see an increase in improvement over $\pib$, that makes sense as the methods only deviate from baseline when they are sure of the performance guarantees. We expect \ref{eq:s-opt} to violate the constraints with increasing value of $\epsilon$, as it relaxes the constraint on the policy-class (\Cref{eq:spibb-policy-constraint}) and leads to a looser guarantee on performance. This again is reflected in our experiments where \ref{eq:s-opt} with $\epsilon=1.0$ has a higher failure-rate than $\epsilon=0.1$.
We observed similar trends for different $\delta$ values.
A more detailed plot corresponding to different $\bml$ and $\rho$ combinations as well as results for a riskier value of $\delta=0.9$ are given in \Cref{app:cmdp-fixed-param-results}.


The results with optimized hyper-parameters can be found in \Cref{fig:best-params-mean}.
We notice that when the $\epsilon$ parameter is tuned properly, \ref{eq:s-opt} has better performance in terms of improvement over $\pib$ for the same amount of samples when compared to \ref{eq:h-opt}, while still ensuring the failure rate is less than $\delta$. These observations are consistent with the results in the single-objective setting in the original SPIBB works~\citep{laroche2017safe, nadjahi2019safe}.
The general trends and observations from the fixed-parameter case are also valid here.  Additional details, including results for $\bml, \rho$ combinations, hyper-parameters considered and qualitative analysis can be found in \Cref{app:cmdp-best-param-results}. 

We also compare our methods against \cite{le2019batch} in \Cref{app:lag-baseline}. We show the advantage of our approach over \cite{le2019batch}, particularly in the low-data regime, where our methods can improve over the baseline policy while ensuring a low failure rate. 
% The method in \cite{le2019batch} exhibits similar trends to Adv-Linearized baseline, where in the low data setting it has high-failure rate, which decreases as the size of dataset decreases. 
This makes sense as the method in \cite{le2019batch} relies on the concentrability coefficient which can be arbitrarily high in the low data setting, and therefore their performance guarantees do not hold anymore.
We also provide experiments on the scalability of methods with the number of objectives $d$ in \Cref{app:cmdp-scaling-experiments}.


% HEALTHCARE EXPERIMENTS
% \pagebreak

\section{Real-world experiment}
\label{sec:sepsis-experiments}


% --------
% Give the brief task description (SEPSIS),  the context of RL for sepsis,
% --------
In order to validate the applicability of our methods on a real-world  task, 
we consider recent works on %the improvement of 
sepsis management via RL, where we only have access to a pre-collected patient dataset and goal is to recommend treatment strategies for patients with sepsis in the ICU \citep{komorowski2018artificial,tang2020clinician}.
% \citep{raghu2017deep, komorowski2018artificial,tang2020clinician}.
Sepsis is defined as a life-threatening organ dysfunction caused by a dysregulated host response to an infection \citep{singer2016third}.
The main treatment method of sepsis involves the repeated administration of intravenous (IV) fluids and vasopressors, but how to manage their appropriate doses at the patient level is still a key clinical challenge \citep{rhodes2017surviving}. 
%\citep{rhodes2017surviving, byrne2017fluid}. 


% --------
% Here how we aim to use this Sepsis task
% --------
% The problem is safety-critical as our methods need to be cautious about using the data that was possibly collected under unobservable confounders and other biases that can lead to biased model estimates. 
% For instance, a study by \citet{ji2020trajectory} of the model used in \citet{komorowski2018artificial} found that the learned model suffers from two major kinds of limitations when it comes to clinically implausible behavior.
% The first kind of challenges are related to the inaccurate modelling assumptions such as discretization of time, and are not the focus of this work. 
% The second aspect, where we believe our methodology can help, is to prevent unexpectedly aggressive treatments resulting from small sample sizes.
% We propose to do so by incorporating safety constraints to prevent recommending the treatments decisions that were never or rarely performed in the dataset. 
% 

The problem is safety-critical as our methods need to be cautious about using the data that was possibly collected under unobservable confounders and that can lead to biased model estimates. 
For instance, a study by \citet{ji2020trajectory} of the model used in \citet{komorowski2018artificial} found that the learned model suggests clinically implausible behavior in the form of unexpectedly aggressive treatments.
We show that our methodology can be applied here to prevent such behavior that results from small sample sizes. We propose to do so by incorporating safety constraints to prevent recommending the treatment decisions that were never or rarely performed in the dataset. 

% against such propose to do so by incorporating safety constraints to prevent recommending the treatments decisions that were never or rarely performed in the dataset. 

% We believe our methodology here can help to prevent clinically implausible behavior that
% We believe our methodology can help prevent unexpectedly aggressive treatments resulting from small sample sizes.



% --------
% Data/Cohort 
% --------
% \textbf{Data and Cohort design:} 
% We use the publicly available ICU dataset MIMIC-III  \cite{johnson2016mimic}, with the setup described by \citet{komorowski2018artificial, tang2020clinician} and build on top of their data pre-processing methodology
% % The cohort is defined by adults fulfilling the sepsis-3 criteria \citep{singer2016third}
% , that includes prescription of antibiotics, lab work of bodily fluids and a Sequential Organ Failure Assessment (SOFA) score $\geq 2$ \citep{singer2016third}. 
% After applying the exclusion criteria \citep{komorowski2018artificial} we are left with a cohort of 20,954 unique patients. 
% --------
% Methodology recap 
% --------
% \textbf{MDP construction:} 
% We follow the MDP construction procedure by \citet{komorowski2018artificial, tang2020clinician} and very briefly describe the methodology here

\textbf{Data and MDP Construction:} 
We use the publicly available ICU dataset MIMIC-III  \citep{johnson2016mimic}, with the setup described by \citet{komorowski2018artificial, tang2020clinician} and build on top of their data pre-processing and MDP construction methodology.\footnote{A caveat here is regarding the underlying assumption that the MDP construction methodology by \citet{komorowski2018artificial, tang2020clinician} maintains the Markovian property in the discretized state-space.} 
This leaves us with a cohort of 20,954 unique patients. 
% The patient data is discretized into 4-hour windows, each of which is pre-processed to be treated as a single time-step.
The state-space consisting of 48 clinical variables summarizing features like demographics, physiological condition, laboratory values, etc., is discretized using a k-means based clustering algorithm to map the states to 750 clusters.
The actions include administration of IV fluids and vasopressors, which are categorized into 5 dosage bins each, leading to a total of $|\A|=25$. The $\gamma$ is set to $0.99$. The reward is based on patient mortality. The agent gets a reward, $r_0$, of $\pm 100$ at the end of the episode based on the survival of the patient. More details can be found in \Cref{app:sepsis-dataset}.


In the original work, the rare state-actions taken by the clinicians (state-action pairs observed infrequently in the training set) are removed from the dataset. Instead of removing them,
% in the pre-processing, 
we define an additional reward, $r_1$, based on the rarity of the state-action pair. We define rare state-action pairs to be those that are taken less than 10 times throughout training dataset, and the agent gets a reward of $-10$ for every such rare state-action taken, i.e.,  $r_1(x,a) = -10.0 \text{ if } \texttt{Count}(x,a) < 10$.
The final task objective then becomes to suggest treatments that handles the trade-off between prioritizing improving the survival vs prioritizing commonly used treatment decisions.


% --------
% Evaluation 
% --------
\textbf{Evaluation:} 
We compare our approach with the same baselines from \Cref{sec:synthetic-experiments} on different $\bml$ combinations.  
We run our methods for 10 runs with different random seeds, where for each run the cohort dataset was split into train/valid/test sets in the ratios of 0.7/0.1/0.2.
We evaluate the performance of the solution policies returned by different methods on the test sets using two different OPE methods, Doubly Robust (DR) \citep{jiang2015dependence} and Weighted Doubly Robust (WDR) \citep{thomas2016data}.
We acknowledge that these methods are a proxy of the actual performance of the deployed policies. Hence, these results should not be misinterpreted as us claiming that the policies returned by our methods are now ready to be used in the ICU. 
% and much more details would be needed before quantifying a policy's performance in the actual clinical setting. Hence, 

% -- Result table

\begin{table*}[ht!]
\centering
\caption{Performance of various methods using DR and WDR estimators with mean and standard deviation on 10 random splits of the cohort dataset. The red cells denote the corresponding safety constraint violation, i.e, either $\mathcal{J}_{0}^{\pi} < \mathcal{J}_{0}^{\pib}$ or $-\mathcal{J}_{1}^{\pi} > -\mathcal{J}_{1}^{\pib}$.}
\label{table:sepsis-best-results}
\begin{adjustbox}{max width=1\textwidth,center}
\begin{tabular}{cccccc}
\toprule
\multicolumn{1}{c}{User preference $(\bml)$} & \multicolumn{1}{c}{Policy} & \multicolumn{2}{c}{Survival return ($\mathcal{J}_0$)} & \multicolumn{2}{c}{Rare-treatment return ($- \mathcal{J}_1$)} \\
\hline
& & DR & WDR & DR & WDR  \\  \cline{3-6}
& Clinician's ($\pib$) & 64.78 $\pm$ 0.90 & 64.78 $\pm$ 0.90          & 13.58 $\pm$ 0.19 & 13.58 $\pm$ 0.19  \\
\midrule % \hline \hline
\multirow{4}{*}{$[\lambda_0=1, \lambda_1 = 0]$} 
& Linearized & 97.68 $\pm$ 0.22 & 97.58 $\pm$ 0.20   & \textcolor{red}{27.64 $\pm$ 1.11 }& \textcolor{red}{27.84 $\pm$ 1.09 } \\ 
& Adv-Linearized  & 91.62 $\pm$ 0.46 & 92.68 $\pm$ 0.23   & \textcolor{red}{15.18 $\pm$ 0.59 }& 13.56 $\pm$ 0.42 \\
& \ref{eq:s-opt}   & 66.11 $\pm$ 0.87 & 66.05 $\pm$ 0.86   & 13.42 $\pm$ 0.20 & 13.46 $\pm$ 0.20   \\
& \ref{eq:h-opt} & 65.95 $\pm$ 0.00 & 65.95 $\pm$ 0.00   & 13.37 $\pm$ 0.00 & 13.37 $\pm$ 0.00  \\
\midrule 
\multirow{4}{*}{$[\lambda_0=1, \lambda_1 = 1]$}
& Linearized & 87.17 $\pm$ 0.48 & 89.11 $\pm$ 0.37   & 2.41 $\pm$ 0.47 & 1.52 $\pm$ 0.41\\
& Adv-Linearized  & 86.77 $\pm$ 0.49 & 88.58 $\pm$ 0.25   & 2.53 $\pm$ 0.50 & 1.57 $\pm$ 0.43  \\
& \ref{eq:s-opt}  & 86.77 $\pm$ 0.49 & 88.58 $\pm$ 0.25   & 2.53 $\pm$ 0.50 & 1.57 $\pm$ 0.43   \\
& \ref{eq:h-opt} & 86.37 $\pm$ 0.00 & 88.03 $\pm$ 0.00   & 2.58 $\pm$ 0.00 & 1.43 $\pm$ 0.00  \\
\midrule 
\multirow{4}{*}{$[\lambda_0=0, \lambda_1 = 0]$}
& Linearized & \textcolor{red}{-89.39 $\pm$ 0.43} & \textcolor{red}{-90.90 $\pm$ 0.29 }  & \textcolor{red}{22.99 $\pm$ 0.40 }& \textcolor{red}{22.81 $\pm$ 0.30 }  \\ 
& Adv-Linearized  & \textcolor{red}{60.27 $\pm$ 0.49} & \textcolor{red}{61.44 $\pm$ 0.85 }  & \textcolor{red}{18.40 $\pm$ 0.27 }& \textcolor{red}{15.36 $\pm$ 0.58 }  \\
& \ref{eq:s-opt} & 67.73 $\pm$ 0.82 & 67.22 $\pm$ 0.88   & 13.24 $\pm$ 0.24 & 13.55 $\pm$ 0.33  \\
& \ref{eq:h-opt} & 65.95 $\pm$ 0.00 & 65.95 $\pm$ 0.00   & 13.37 $\pm$ 0.00 & 13.37 $\pm$ 0.00  \\
\midrule %\hline \hline
\multirow{4}{*}{$[\lambda_0=0, \lambda_1 = 1]$}
& Linearized & \textcolor{red}{58.27 $\pm$ 2.18} & \textcolor{red}{60.52 $\pm$ 2.07 }  & 0.04 $\pm$ 0.03 & 0.02 $\pm$ 0.01  \\ 
& Adv-Linearized  & 76.05 $\pm$ 0.65 & 76.85 $\pm$ 0.72   & 0.07 $\pm$ 0.05 & 0.04 $\pm$ 0.03  \\
& \ref{eq:s-opt}  & 76.07 $\pm$ 0.65 & 76.87 $\pm$ 0.73   & 0.07 $\pm$ 0.05 & 0.04 $\pm$ 0.03  \\
& \ref{eq:h-opt} & 76.54 $\pm$ 0.00 & 77.55 $\pm$ 0.00   & 0.09 $\pm$ 0.00 & 0.05 $\pm$ 0.00  \\
\bottomrule 
\end{tabular}
\end{adjustbox}
\vskip -0.1in
\end{table*}


% --------
%  Results
% --------
\textbf{Results:}
We refer to the return associated with the mortality reward ($r_0$) as survival return ($\mathcal{J}_{0}$), and the negative return associated with rare state-action reward ($r_1$) as rare-treatment return ($- \mathcal{J}_{1}$). Higher survival return implies more successful discharges, and lower rare-treatment return implies more adherence to common practice treatment decisions.
We present the results on survival and rare-treatment returns 
% with mean and standard deviation for the 10 runs 
in \Cref{table:sepsis-best-results}. 
% As expected, we observe the Linearized baseline violates most of the constraints across different $\bml$. The Adv-Linearized baseline performs better than Linearized, but still ends up violating some constraints, possibly due to unreliable estimates.
% We observe the Linearized baseline violates most of the constraints across different $\bml$. The Adv-Linearized baseline performs a bit better, but still ends up violating some constraints, possibly due to unreliable estimates.
% We expect both \ref{eq:s-opt} and \ref{eq:h-opt} to respect the safety constraints 
% irrespective of the $\bml$, and indeed they are able to do so.
As expected, we observe both the Linearized and Adv-Linearized baselines violates constraints across different $\bml$, whereas \ref{eq:s-opt} and \ref{eq:h-opt} are able to respect the safety constraints irrespective of the $\bml$.\footnote{In \Cref{table:sepsis-best-results}, $\bml =[1,1]$ represents a rare case of reward scalarization that allows all the methods to find a good solution policy that satisfies the constraints.  In general, it is difficult to find such scalarization parameters as seen in synthetic experiments (\Cref{app:cmdp-fixed-param-results}).}
The validation set was used to tune the hyper-parameters, and we report how the performance varies with different hyper-parameters in \Cref{app:sepsis-hyperparams}. 




% --------
% Qualitative Results 
% --------
\textbf{Qualitative Analysis:} 
We conclude with a qualitative analysis of the policies returned from our setting and the traditional RL approach of maximizing just the survival return. 
% We calculate how many rare-actions are recommended by different solution policies and compare them with the most common actions taken by the clinicians.
% For each state, for the action recommended by a solution policy, we calculate the frequency with which that state-action was observed in the training data and calculate the percentage of time that state-action pair was observed among all the possible actions taken from that state.
% Across all the states, the actions suggested by the traditional single-objective RL baseline are observed only 3\% of the time on average (5.3 observations per state). Whereas, the actions most commonly chosen by the clinicians  are observed 51.4\% of the time on average (138.2 observations per state). We study this behavior for two of the policies returned by MO-SPIBB that deviate the most from the baseline: for the policy returned by \ref{eq:s-opt} ($\bml=[1,1]$) the recommended actions are observed 24.8\% of time on average (61.0 observations per state) and for  \ref{eq:s-opt} ($\bml=[0,1]$) the recommended actions are observed 23.4\% of times (56.14 observations per state).
%
\citet{ji2020trajectory} found that the RL-policies for sepsis-management task usually end up recommending aggressive treatments, particularly high vasopressor doses for states where the common practice 
(according to most frequent action chosen by the clinician for that state) 
is to give no vasopressors at all. The common practice involves giving zero vasopressors for 722 of the 750 states. However, the policy returned by the traditional single-objective RL baseline recommends vasopressors in 562 (77.84\%) of those 722 states, with 295 of those recommendations being large doses
% , where large doses are defined as the dosages belonging in the upper 50th percentile of nonzero amounts 
(upper 50th percentile of nonzero amounts  or $>0.2$ $\mu$g/kg/min). 
We compare these statistics for two of the policies returned by MO-SPIBB that deviate the most from $\pib$. 
The policy returned by \ref{eq:s-opt} ($\bml=[1,1]$) recommends vasopressors in only 93 of those states (12.88 \%), with 47 of those recommendations belonging to high dosages. The other policy, \ref{eq:s-opt} ($\bml=[0,1]$), recommends vasopressors in 134 (18.56 \%) of those states and 70 of those recommendations fall in large dosages.
Therefore, the policies returned by our approach, even when they deviate from the baseline, are less aggressive in recommending rare treatments. 
In \Cref{app:sepsis-qual-analysis}, we present an additional qualitative analysis that demonstrates our methods recommend lesser rare-action treatments than the traditional single-objective RL approach.

% This is on argument against why include rare-action as costs.
An argument can be made against the case when all rare state-action pairs are removed from the training data itself. This will ensure that any learned policy will have near 0 rare-treatment return. However, it is not always clear how to define the cut-off criteria for rare-actions, and it might be possible that some of these rare state-action pairs are actually crucial for finding a better policy. 
For instance, we did an experiment where we assigned state-actions pairs with frequency $<100$ to be rare state-action pairs and filtered those from the training set. The clinician's performance on the test set using a DR estimator for survival return is 65.95. % (and $C$: 59.63).
In this case, the traditional single-objective RL baseline gives the survival return of 11.26, %(and $C$: 18.40), 
which shows that removing such transitions from the dataset actually hampers the solution quality. Our approach of assigning a separate reward for rare state-action pairs is able to find a solution with a survival return of 86.75 %(and $C$: 24.66)
even in this scenario.






% CONCLUSION
\mySection{Related Works and Discussion}{}
\label{chap3:sec:discussion}

In this section we briefly discuss the similarities and differences of the model presented in this chapter, comparing it with some related work presented earlier (Chapter \ref{chap1:artifact-centric-bpm}). We will mention a few related studies and discuss directly; a more formal comparative study using qualitative and quantitative metrics should be the subject of future work.

Hull et al. \citeyearpar{hull2009facilitating} provide an interoperation framework in which, data are hosted on central infrastructures named \textit{artifact-centric hubs}. As in the work presented in this chapter, they propose mechanisms (including user views) for controlling access to these data. Compared to choreography-like approach as the one presented in this chapter, their settings has the advantage of providing a conceptual rendezvous point to exchange status information. The same purpose can be replicated in this chapter's approach by introducing a new type of agent called "\textit{monitor}", which will serve as a rendezvous point; the behaviour of the agents will therefore have to be slightly adapted to take into account the monitor and to preserve as much as possible the autonomy of agents.

Lohmann and Wolf \citeyearpar{lohmann2010artifact} abandon the concept of having a single artifact hub \cite{hull2009facilitating} and they introduce the idea of having several agents which operate on artifacts. Some of those artifacts are mobile; thus, the authors provide a systematic approach for modelling artifact location and its impact on the accessibility of actions using a Petri net. Even though we also manipulate mobile artifacts, we do not model artifact location; rather, our agents are equipped with capabilities that allow them to manipulate the artifacts appropriately (taking into account their location). Moreover, our approach considers that artifacts can not be remotely accessed, this increases the autonomy of agents.

The process design approach presented in this chapter, has some conceptual similarities with the concept of \textit{proclets} proposed by Wil M. P. van der Aalst et al. \citeyearpar{van2001proclets, van2009workflow}: they both split the process when designing it. In the model presented in this chapter, the process is split into execution scenarios and its specification consists in the diagramming of each of them. Proclets \cite{van2001proclets, van2009workflow} uses the concept of \textit{proclet-class} to model different levels of granularity and cardinality of processes. Additionally, proclets act like agents and are autonomous enough to decide how to interact with each other.

The model presented in this chapter uses an attributed grammar as its mathematical foundation. This is also the case of the AWGAG model by Badouel et al. \citeyearpar{badouel14, badouel2015active}. However, their model puts stress on modelling process data and users as first class citizens and it is designed for Adaptive Case Management.

To summarise, the proposed approach in this chapter allows the modelling and decentralized execution of administrative processes using autonomous agents. In it, process management is very simply done in two steps. The designer only needs to focus on modelling the artifacts in the form of task trees and the rest is easily deduced. Moreover, we propose a simple but powerful mechanism for securing data based on the notion of accreditation; this mechanism is perfectly composed with that of artifacts. The main strengths of our model are therefore : 
\begin{itemize}
	\item The simplicity of its syntax (process specification language), which moreover (well helped by the accreditation model), is suitable for administrative processes;
	\item The simplicity of its execution model; the latter is very close to the blockchain's execution model \cite{hull2017blockchain, mendling2018blockchains}. On condition of a formal study, the latter could possess the same qualities (fault tolerance, distributivity, security, peer autonomy, etc.) that emanate from the blockchain;
	\item Its formal character, which makes it verifiable using appropriate mathematical tools;
	\item The conformity of its execution model with the agent paradigm and service technology.
\end{itemize}
In view of all these benefits, we can say that the objectives set for this thesis have indeed been achieved. However, the proposed model is perfectible. For example, it can be modified to permit agents to respond incrementally to incoming requests as soon as any prefix of the extension of a bud is produced. This makes it possible to avoid the situation observed on figure \ref{chap3:fig:execution-figure-4} where the associated editor is informed of the evolution of the subtree resulting from $C$ only when this one is closed. All the criticisms we can make of the proposed model in particular, and of this thesis in general, have been introduced in the general conclusion (page \pageref{chap5:general-conclusion}) of this manuscript.





% ACKNOWLEDGEMENTS
\documentclass{article}

\begin{document}

\section{Acknowledgements}

The author would like to thank B. J. Hiley, M. Hajtanian, and D. Nellist for their insightful conversations and support.

\end{document}

% ------------------------------------------------------------
%               BIBLIOGRAPHY 
% ------------------------------------------------------------

% \bibliographystyle{plainnat}
\bibliographystyle{apalike}
\bibliography{doc/lib}



%%%%%%%%%%%%%%%%%%%%%%%%%%%%%%%%%%%%%%%%%%%%%%%%%%%%%%%%%%%%
\section*{Checklist}

\begin{enumerate}

\item For all authors...
\begin{enumerate}
  \item Do the main claims made in the abstract and introduction accurately reflect the paper's contributions and scope?
    \answerYes{See \Cref{sec:spibb-w-constraints,sec:hcpi-w-constraints} for the methodology and theoretical claims, and \Cref{sec:synthetic-experiments,sec:sepsis-experiments} for the empirical results.}
  
  \item Did you describe the limitations of your work?
    \answerYes{See \Cref{sec:introduction} for the limitations and scope of this work.}
  
  \item Did you discuss any potential negative societal impacts of your work?
    \answerYes{
    % In this work, we propose methodology and algorithms for \textit{safe} policy improvement of RL agents in the multi-objective setting.
    As we mentioned in \Cref{sec:introduction,sec:problem-formulation}, our goal is to maximize the objective specified by the user while ensuring that the solution policy avoids causing harmful effects after deployment in the true environment in comparison to the existing baseline policy. 
    % This allows the algorithm practitioner to experiment with different reward design strategies in safety-critical settings without worrying about the risks of ill-defined scalarizations. 
    % As with other Seldonian algorithms, we do not address any safety risks concerning data privacy or security, and additional caution should be exercised related to these topics.  
    We aim to bridge the gap between traditional RL methods and high-stake real-world applications, but, as with any general technology, we acknowledge that some RL applications can have the potential of misuse, and our methods do not prevent that.
    }

    
  \item Have you read the ethics review guidelines and ensured that your paper conforms to them?
    \answerYes{We discuss the societal impacts in the point above. We use pre-existing publicly available data and libraries and give more details about them in the Point 4 below.}
\end{enumerate}

\item If you are including theoretical results...
\begin{enumerate}
  \item Did you state the full set of assumptions of all theoretical results?
    \answerYes{See \Cref{sec:setting} for the assumption regarding access to baseline policy.}
	\item Did you include complete proofs of all theoretical results?
    \answerYes{The complete proofs are provided in the  \Cref{app:spibb-additional-details} and \Cref{app:hcpi-details}.}
\end{enumerate}

\item If you ran experiments...
\begin{enumerate}
  \item Did you include the code, data, and instructions needed to reproduce the main experimental results (either in the supplemental material or as a URL)?
    \answerYes{The code required to produce the results is provided in the supplementary material.}
    
  \item Did you specify all the training details (e.g., data splits, hyperparameters, how they were chosen)?
    \answerYes{The training details are mentioned in  \Cref{sec:synthetic-experiments,sec:sepsis-experiments} in the main text, and  \Cref{app:additional-details-for-synthetic-exp,app:sepsis-details}.}
    
	\item Did you report error bars (e.g., with respect to the random seed after running experiments multiple times)?
    \answerYes{The details about the error bars are provided in \Cref{sec:synthetic-experiments,sec:sepsis-experiments}.}
    
	\item Did you include the total amount of compute and the type of resources used (e.g., type of GPUs, internal cluster, or cloud provider)?
    \answerYes{Details about the compute and resources can be found in \cref{app:additional-details-for-synthetic-exp,app:sepsis-details}.
    }
\end{enumerate}

\item If you are using existing assets (e.g., code, data, models) or curating/releasing new assets...
\begin{enumerate}
  \item If your work uses existing assets, did you cite the creators?
    \answerYes{See \Cref{sec:synthetic-experiments,sec:sepsis-experiments} for the appropriate references.}
  
  \item Did you mention the license of the assets?
    \answerYes{We use the \citep[MIMIC-III,][]{johnson2016mimic} dataset and provide the appropriate reference. The explicit link to the license is here: \url{https://physionet.org/content/mimiciii/view-license/1.4/}. }
  
  \item Did you include any new assets either in the supplemental material or as a URL?
    \answerYes{We provide the accompanying code for running the experiments in the supplemental material.}
    
  \item Did you discuss whether and how consent was obtained from people whose data you're using/curating?
    \answerNA{We are using publicly available libraries and dataset.}
    
  \item Did you discuss whether the data you are using/curating contains personally identifiable information or offensive content?
    \answerNA{We are using the publicly available dataset  \citep[MIMIC-III,][]{johnson2016mimic} that already deidentifies the data in accordance with Health Insurance Portability and Accountability Act (HIPAA) standards using structured data cleansing and date shifting. We refer to the original text for more details on the deidentification process.
    }
\end{enumerate}

\item If you used crowdsourcing or conducted research with human subjects...
\begin{enumerate}
  \item Did you include the full text of instructions given to participants and screenshots, if applicable?
    \answerNA{}
  \item Did you describe any potential participant risks, with links to Institutional Review Board (IRB) approvals, if applicable?
    \answerNA{}
  \item Did you include the estimated hourly wage paid to participants and the total amount spent on participant compensation?
    \answerNA{}
\end{enumerate}

\end{enumerate}



%%%%%%%%%%%%%%%%%%%%%%%%%%%%%%%%%%%%%%%%%%%%%%%%%%%%%%%%%%%%

% \appendix



% ------------------------------------------------------------
%               APPENDIX 
% ------------------------------------------------------------
\clearpage
\onecolumn
\appendix

% Naive construction
\section{Scalarized safety constraints}
\label{app:naive-construction}

Instead of having constraints of the form in \Cref{eq:general-safety-constraints}, it is possible to define the constraints in terms of the scalarized objective directly, i.e., 
\begin{align*}
    \mathbb{P}\left(    \sum_{k\in[d]} \lambda_i \J{\pi}{i}{\mstar}  - \sum_{k\in[d]} \lambda_i \J{\pib}{i}{\mstar} > - \zeta \Big| \D \right) > 1 -\delta.
\end{align*}
 

Without loss of generality, if we assume there are only two objective ($d=2$), then satisfying the above constraint implies:
\begin{align*}
    \lambda_0 \left(\J{\pi}{0}{\mstar} - \J{\pib}{0}{\mstar} \right) + \lambda_1 \left(\J{\pi}{1}{\mstar} - \J{\pib}{1}{\mstar}  \right) \geq 0 .
\end{align*}
Consider a scenario where the solution policy performs poorly w.r.t. the second objective, i.e, $\lambda_1 (\J{\pi}{1}{\mstar}  - \J{\pib}{1}{\mstar}) < 0$, however the the improvement in the first objective is very large $\lambda_0 (\J{\pi}{0}{\mstar} - \J{\pib}{0}{\mstar}) >> 0$. In this case, even though the linearized cumulative constraint regarding the performance improvement is being satisfied, it fails to guarantee the improvement across each individual objectives.


% As an informal proof, notice that the new constraint in \Cref{eq:general-task-objective} will now be $\lR \J{\pi}{R}{\mstar} - \lC  \J{\pi}{C}{\mstar} \geq \lR \J{\pib}{R}{\mstar} - \lC \J{\pib}{C}{\mstar}$. 

% Satisfying this constraint implies that  $\lR \left( \J{\pi}{R}{\mstar} - \J{\pib}{R}{\mstar} \right) \bm{+} \lC \left( \J{\pib}{C}{\mstar}  - \J{\pi}{C}{\mstar} \right) \geq 0 $.

% Now, assume that if the solution policy actually performs poorly w.r.t. $C$, i.e., $(\J{\pib}{C}{\mstar}  - \J{\pi}{C}{\mstar}) < 0$, but the performance in rewards is very large i.e, $(\J{\pi}{R}{\mstar} - \J{\pib}{R}{\mstar}) >> 0$, then even though the linearized constraint regarding performance is being satisfied, it still violates the individual cost constraint.

% % Error bounds
% \section{Error bounds}
\label{app:err-bounds}

Using the same notation from \cite{nadjahi2019safe}, we have:

\begin{align}
    \norm{P(.|x,a) - \hat{P}(.|x,a)}_{1} &\leq e_P(x,a),  \label{eq:eP}
    \\ \abs{R(x,a) - \hat{R}(x,a)} &\leq e_P(x,a) R_{\max}, \label{eq:eR}
    \\ \abs{C_i(x,a) - \hat{C_{i}}(x,a)} &\leq e_P(x,a) C_{i_{\max}}, \quad (\forall i),   \label{eq:eC}
    \\ \abs{Q^{\pi_b}_{M*}(x,a) - Q^{\pi_b}_{\hat{M}}(x,a)} &\leq e_Q(x,a) V_{\max},  \label{eq:eQ}
    \\ \abs{{Qc}^{\pi_b}_{i, M*}(x,a) - Q^{\pi_b}_{i, \hat{M}}(x,a)} &\leq e_Q(x,a) {Vc}_{\max},  \label{eq:eQc}
\end{align}
where,
\begin{align}
    e_P(x,a) &= \sqrt{\frac{2}{N_{\D}(x,a)} \log\frac{2 \abs{\X}\abs{A}2^{\abs{\X}}}{\delta}}, 
    \\ e_Q(x,a) &= \sqrt{\frac{2}{N_{\D}(x,a)} \log\frac{2 \abs{\X}\abs{A}}{\delta}}.
\end{align}


% % SPIBB-details
% \section{SPIBB formulation details}
\label{app:spibb-background}


% ------------------------------------------------------
%                       SPIBB 
% ------------------------------------------------------


% An alternate objective that might be of interest is the following: 
% \begin{align}
%     \max_{\pi} &\quad J(\pi, \hat{M}) \label{eq:alt-batch-pi}\\ 
%     \texttt{s.t.} &\quad J(\pi, M) \geq J(\pi_b, M) - \xi, \forall M \in \Xi,
%     \nonumber \\
%     &\quad J(C_i, \pi, \hat{M}) \leq d_i, \forall i, \nonumber \\
%     &\quad J(C_i, \pi, M) \leq f(d_i, \xi), \forall i. \nonumber
% \end{align}

% The main difference between Eq.~\ref{eq:batch-pi} and Eq.~\ref{eq:alt-batch-pi} is that in Eq.~\ref{eq:alt-batch-pi} the constraint satisfaction is a hard constraint with respect to the estimated MDP, and for constraint satisfaction in the admissible MDPs we would like it to be some function of $\xi$-like quantity (to be decided). 


% \harsh{Hypothesis: One of the problems with the CMDPs formulation is that sometimes the algorithms finds the solution that is just below the constraint threshold. While this is okay in certain scenarios it ends up being more like a equality constraint. In this scenario, user can specify that the goal is to increase safety/reduce constraints, while making sure the performance doesn't suffer much.}


The SPIBB algorithm \citep{laroche2017safe} takes input a dataset of trajectories under the baseline policy $\D$, as well as needs access to the baseline policy $\pib$ to do return a improved policy based on the guarantees parameters provided by the user.

The high level methodology of the SPIBB algorithm is along the following lines. The first step is to  build a bootstrap set $\Bset$, a set of state-action pairs with counts less than $N_\Lambda$ (hyper-parameter). When the state-action pair has low confidence (rarely encountered in the dataset), SPIBB will bootstrap the baseline policy for it. Here bootstrapping means that the agent will rely on the baseline policy and copy the probabilities for that state-action pair, 
\begin{equation*}
    \pi^{\odot}_{spibb}(a|x) = \pi_b(a|x) \text{ if } (x,a) \in \Bset.
\end{equation*}
For the non-bootstrapped actions, SPIBB uses the greedy update but restricted to the non-bootstrapped actions:
\begin{equation*}
    \pi^{(i)}_{spibb}(x , \argmax_{a \mid a \notin \Bset} Q^{(i)}(x,a)) = \sum_{a \mid a \notin \Bset} \pi_b(a|x)
\end{equation*}
SPIBB is based on the binary classification of the bootstrapped set: either a pair either belongs to it and cannot affect the new policy, or it does not and the policy is changed completely for that pair. 

In Soft-SPIBB \citep{nadjahi2019safe}, authors  the objective that allows slight policy changes for uncertain state-action pairs while remaining safe. The authors constrain the class of policies to $(\pi_b, e, \epsilon)$-constrained policies: 
\begin{equation}
    \sum_a e(x,a)\ |\pi(a|x) - \pi_{b}(a|x)| \leq \epsilon,  \forall x \in \X,
\end{equation}
where $e: \X \times \A \rightarrow \Real$ is an error function on the state-action value function $Q$ and $\epsilon$ is a hyper-parameter that controls the deviation from the baseline policy. The second constraint is to improve the performance of the new policy w.r.t. the baseline, i.e. to a find $\pi_b$-advantageous policy:
\begin{equation}
\label{eq:spibb-performance-constraint}
\sum_a \pi(a|x) A^{\pi_b}(x, a) \geq 0, \forall x \in \X,
\end{equation}
where $A^{\pi_b}$ denotes the advantage function: $A^{\pi}(x,a) = Q^{\pi}(x,a) - V^{\pi}(x)$. Under these two conditions, i.e., if the policy is $\pi_b$-advantageous and $(\pi_b, e_Q, \epsilon)$-constrained then it is possible to show  can have the safety guarantees.  The new policy improvement can be viewed as:
    \begin{align}
        \label{eq:soft-spibb-obj}
        \pi_{new} &= \argmax_{\Pi} \langle \pi(\cdot|x) Q^{\pi_b}(x,\cdot) \rangle \quad \forall x \in \X  \\
        \texttt{s.t.} &\sum_a e_Q(x,a)\ |\pi(a|x) - \pi_{b}(a|x)| \leq \epsilon, \nonumber \\ 
        &\sum_a \pi(a|x) = 1 . \nonumber
    \end{align}


% more details
As such they are working with a constrained optimization problem/LP, that they propose to solve in every iteration of the policy improvement. The exact solution of the LP can be found with simplex or mixed integer programming. The authors also propose an approximate method for calculating the solution.
However, these policy constraints can be too conservative, they use $e_P$, to relax the policy space and use another assumption to guarantee safety bounds. 
    

% SPIBB-theorems
\section{SPIBB - Additional details}
\label{app:spibb-additional-details}  
% ----------------------------------------------------
%               Concentration Bounds
% ----------------------------------------------------
\subsection{Concentration Bounds} 
\label{app:error_bounds}

The difference between an estimated parameter and the true one can be bounded using concentration bounds (or equivalently, Hoeffding's inequality) applied to the state-action counts $n_{\mathcal{D}}(x,a)$ in dataset $\mathcal{D}$~\citep{petrik2016safe, laroche2017safe}. Specifically, the following inequalities hold with probability at least $1-\delta = 1 - \delta' - \delta''$ for any state-action pair $(x,a) \in \mathcal{X} \times \mathcal{A}$:
\begin{align}
	\lVert p^\star(\cdot|x,a)-\hat{p}(\cdot|x,a)\rVert_1 &\leq e_p(x,a),\\
	\forall k\in[d], \lvert r^\star_k(x,a)-\hat{r}_k(x,a)\rvert &\leq e_r(x,a)r_{\mytop} \label{eq:bound-Q-2}
\end{align}
where: 
\begin{align}
    e_p(x,a) &:= \sqrt{\cfrac{2}{n_{\mathcal{D}}(x,a)}\log\cfrac{2|\mathcal{X}||\mathcal{A}|2^{|\mathcal{X}|}}{\delta'}} \label{eq:error-function-P-2} \\
    e_r(x,a) &:= \sqrt{\cfrac{2}{n_{\mathcal{D}}(x,a)}\log\cfrac{2|\mathcal{X}||\mathcal{A}|d}{\delta''}}. \label{eq:error-function-Q-2}
\end{align}

    The two inequalities can be proved similarly to \citep[Proposition 9]{petrik2016safe}. We only detail the proof for \eqref{eq:bound-Q-2}: for any $(x,a) \in \mathcal{X} \times \mathcal{A}$, and from the two-sided Hoeffding's inequality, 
    \begin{align*}
        &\mathbb{P} \left( \forall (x,a), \big\lvert r^\star_k(x,a) -\hat{r}_k(x,a) \big\rvert > e_r(x,a)r_{\mytop} \right) 
        \\
        &\qquad\qquad = \mathbb{P} \left( \forall (x,a), \frac{\big\lvert r^\star_k(x,a)-\hat{r}_k(x,a) \big\rvert}{2 V_{max}} > \sqrt{\frac{1}{2 n_\mathcal{D}(x,a)} \log \frac{2 |\mathcal{X}||\mathcal{A}|d}{\delta''}} \right) \\
        &\qquad\qquad \leq 2 \exp \left( -2 n_\mathcal{D}(x,a) \frac{1}{2 n_\mathcal{D}(x,a)} \log \frac{2 |\mathcal{X}||\mathcal{A}|}{\delta''}  \right) \\
        &\qquad\qquad \leq \frac{\delta''}{| \mathcal{X} | | \mathcal{A} | d}
    \end{align*}
    
    By summing all $|\mathcal{X}| |\mathcal{A}|d $ state-action-reward tuples error probabilities lower than $\frac{\delta''}{| \mathcal{X} | | \mathcal{A} |d }$, we obtain \eqref{eq:bound-Q-2}. If we choose $e(x,a)=e_p(x,a)=e_r(x,a)$, we get that:
    \begin{align}
        \cfrac{2}{n_{\mathcal{D}}(x,a)}\log\cfrac{2|\mathcal{X}||\mathcal{A}|2^{|\mathcal{X}|}}{\delta'} &= \cfrac{2}{n_{\mathcal{D}}(x,a)}\log\cfrac{2|\mathcal{X}||\mathcal{A}|d}{\delta''} \\
        \cfrac{2^{|\mathcal{X}|}}{\delta'} &= \cfrac{d}{\delta''} \\
        \delta'' &= d\delta' 2^{-|\mathcal{X}|}
    \end{align}
    
    It means that $\delta = \delta' + \delta'' = \delta'(1+d2^{-|\mathcal{X}|})$. The cost in terms of approximation is therefore linear with a very small slope, inside square root of log, which means that it will basically have an insignificant impact on the concentration bound.


% ----------------------------------------------------
%               Advantage Constraints
% ----------------------------------------------------
\subsection{Need of advantageous constraints}
\label{app:spibb-need-of-advatangeous}



\begin{prop}
\label{prop:mo-sipbb-advantageous}
The advantageous constraints in \ref{eq:s-opt} ensure that performance constraints w.r.t. the individual returns are respected in $\mhat$, i.e., $ \forall k \in [d],\; \J{\pi}{k}{\mhat} - \J{\pib}{k}{\mhat} \ge 0$.
\end{prop}

\begin{proof}
For the $k$\textsuperscript{th} reward function, we can estimate the advantage function in an MDP $m$ as:
\begin{align*}
    \adv{\pib}{k}{m}{x,a} = \qval{\pib}{k}{m}{x,a} - \val{\pib}{k}{m}{x}
\end{align*}
Similarly, let $\rho^{\pi}_{m}(x)$ denote the normalized discounted future state distribution:
\begin{align*}
    \rho^{\pi}_{m}(x) &= (1-\gamma)\sum_{t=0}^{\infty} \gamma^t \mathbb{P}(X_t=x | \pi, X_0=x_0),
\end{align*}
where $X_{t} \sim p(\cdot | X_{t-1}, A_{t-1}), A_{t-1} \sim \pi(\cdot|X_{t-1})$.
From Performance Difference Lemma \citep{kakade2002approximately}, we have the following result:
\begin{align}
    \label{eq:spibb-prop-adv}
    \J{\pi}{k}{\mhat} - \J{\pib}{k}{\mhat} &= \sum_{x \in \X} \rho^{\pi}_{\mhat}(x) \underbrace{\sum_{a \in \A} \pi(a|x) \adv{\pib}{k}{\mhat}{x,a}}_{\text{advantage constraint}}
\end{align}

The first term in the above equation $\rho^{\pi}_{\mhat}(x) \ge 0$ for any $x \in \X$. The second term is the advantage constraint in the construction of \ref{eq:s-opt}.
Therefore, any solution of \ref{eq:s-opt} satisfies $\sum_{a \in \A} \pi(a|x) \adv{\pib}{k}{\mhat}{x,a} \ge 0, \forall x \in \X$.

As both the terms in \Cref{eq:spibb-prop-adv} are $\ge 0 \; \forall x \in \X$, this implies $ \J{\pi}{k}{\mhat} - \J{\pib}{k}{\mhat} \ge 0$.

\end{proof}



% ------- Old prop
% \begin{prop}
% \label{prop:spibb-r-advatangeous}
% If $\pi'$ is the solution to problem \Cref{eq:s-opt} without the $R$-constraints, i.e. without $\sum_a \pi(a|x) \Adv{\pib}{R}{\mhat}(x, a) \geq 0$ term, then $\pi'$ won't necessarily $\pib$-advantageous in $\mhat$ with respect to $\pib$ regarding reward performance $R$:
% \begin{equation*}
%     \sum_a \pi'(a|x) \Adv{\pib}{R}{\mhat}(x, a) \not\geq 0, \forall x \in \X.
% \end{equation*}
% \end{prop}
% \begin{proof}
% \htodo{Write this as a proof by contradiction, assume $\lR, \lC \geq 0$ and advantageous wrt costs are true, and then continue}
% Since $\pi'$ is the solution of \Cref{eq:s-opt} (without $R$-constraints) and $\pib$ also lies in the solution space $\Pi$, we have for any $x \in \X$:
% \begin{align*}
%     &\langle \pi'(\cdot|x) Q_{\lambda}^{\pi_b}(x,\cdot) \rangle \ge \langle \pib(\cdot|x) Q_{\lambda}^{\pib}(x,\cdot) \rangle \\ 
%     &\sum_{a \in \A} \left( \lR \pi'(a|x) \Q{\pib}{R}{\mhat}(x,a ) - \sum_i \left( \lC \pi'(a|x) \Q{\pib}{\ci}{\mhat}(x,a) \right) \right) \ge \\ 
%     &\quad \quad \sum_{a \in \A} \left( \lR \pib(a|x) \Q{\pib}{R}{\mhat}(x,a ) - \sum_i \left( \lC \pib(a|x) \Q{\pib}{\ci}{\mhat}(x,a) \right) \right)
% \intertext{Taking all the terms w.r.t $R$ on left and the rest on R.H.S, we get:}
%     &\lR \left( \sum_{a \in \A}\pi'(a|x) \Q{\pib}{R}{\mhat}(x,a)  -  \underbrace{\sum_{a \in \A} \pib(a|x) \Q{\pib}{R}{\mhat}(x,a)}_{\V{\pib}{R}{\mhat}(x)}  \right) \ge \\ 
%     &\quad \quad \sum_i \lC \left( \sum_{a \in \A} \pi'(a|x) \Q{\pib}{\ci}{\mhat}(x,a)  -  \underbrace{\sum_{a \in \A} \pib(a|x) \Q{\pib}{\ci}{\mhat}(x,a)}_{\V{\pib}{\ci}{\mhat}(x)}  \right)
% \end{align*}
% Using the definition of the advantage function we have:
% \begin{align*}
%     \lR \left( \sum_{a \in \A}\pi'(a|x) \Adv{\pib}{R}{\mhat}(x,a) \right) 
%     &\ge \sum_i \lC \left( \underbrace{\sum_{a \in \A} \pi'(a|x) \Adv{\pib}{\ci}{\mhat}(x,a)}_{\le 0} \right) 
% \end{align*}
% We have $\Adv{\pib}{\ci}{\mhat}(x,a) \le 0 \; \forall i$ by construction, and $\lR \ge 0$ and $\lC \ge 0$. When $\lR > 0,  \lC >0$ and the term on R.H.S. is negative $<0$, then that implies that it might not longer be necessary that $\pib$-advantageous property holds true in this case and the term $\sum_{a \in \A}\pi'(a|x) \Adv{\pib}{R}{\mhat}(x,a)$ can be negative.

% \textit{Remark:} In the case of Soft-SPIBB, $\lC = 0, \lR = 1$, and as such the R.H.S. is always 0 and we have:
% \begin{align*}
% \left( \sum_{a \in \A}\pi'(a|x) \Adv{\pib}{R}{\mhat}(x,a) \right) &\ge 0.
% \end{align*}

% \end{proof}



% ----------------------------------------------------
%               Soft-SPIBB 1-step
% ----------------------------------------------------
\subsection{MO-SPIBB Results}
\label{app:mo-spibb-prop}

Using the results from \Cref{app:error_bounds} and \Cref{app:spibb-need-of-advatangeous}, we can directly apply the Soft-SPIBB theorems to individual objectives in \ref{eq:s-opt}. For instance, we get the following result about 1-step policy improvement guarantees directly from Theorem 1 of Soft-SPIBB:

\begin{prop}
The policy $\pi$ returned from solving the \ref{eq:s-opt} satisfies the following property in every state $x$ with probability at least $(1 - \delta)$:
\begin{align}
    \forall k \in [d], \val{\pi}{k}{\mopt}{x} - \val{\pib}{k}{\mopt}{x} \geq -\frac{\epsilon v_{\text{max}}}{1-\gamma},
\end{align}
where $v_{\text{max}} \le \frac{\rmax}{1-\gamma}$ is the maximum of the value function.
\end{prop}

\begin{proof}
We will show the policy returned by \ref{eq:s-opt} satisfies both the properties required for applying the Theorem 1 of Soft-SPIBB:


\begin{itemize}
    \item $\pi$ is $(\pib, \epsilon, e)$-constrained: This is equivalent to $\sum_{a \in \A} e(x,a)\ |\pi(a|x) - \pib(a|x)| \leq \epsilon$, that is true by construction.
    
    \item $\pib$-advantageous in $\mhat$: For $k$\textsuperscript{th} reward function, this is equivalent to $\J{\pi}{k}{\mhat} - \J{\pib}{k}{\mhat} \ge 0$, which is also true from construction.
\end{itemize}
From there, the exact statement of Theorem 1 can be applied directly to get the above result.
\end{proof}





% \begin{proof}

% \textbf{For rewards:} 
% We first use  \Cref{prop:spibb-r-advatangeous} to show that we need additional constraints for $\pib$-advantageous property to hold true for $R$. Once we have added those to \Cref{eq:s-opt}, we can use the exact same structure of the proof of Theorem 1 of \cite{nadjahi2019safe}. 

% \textbf{For constraints:} 
% From Proposition 1 of \cite{nadjahi2019safe}, we have the following for any constraint $C_i$:
% \begin{align}
%     {Vc}^{\pi}_{i, M^*}(x) - {Vc}^{\pi_b}_{i, M^*}(x)  &= {Qc}^{\pi_b}_{M^*} (\pi - \pi_b) 
%     d^{\pi}_{M^*} \nonumber \\
%     &= \left(Qc^{\pi_b}_{M^*} - Qc^{\pi_b}_{\hat{M}} + Qc^{\pi_b}_{\hat{M}} \right)(\pi - \pi_b) d^{\pi}_{M^*} \nonumber \\ 
%     &= \left(Qc^{\pi_b}_{M^*} - Qc^{\pi_b}_{\hat{M}}\right) (\pi - \pi_b) d^{\pi}_{M^*}  +  Qc^{\pi_b}_{\hat{M}}(\pi - \pi_b)d^{\pi}_{M^*} \label{eq:thm1-proof}.
% \end{align}

% We will now show the first term is bounded by $\frac{\epsilon {Vc}_{i, \max}}{(1-\gamma)}$ using the Holder's inequality.

% \begin{align}
%     \norm{\left(Qc^{\pi_b}_{M^*} - Qc^{\pi_b}_{\hat{M}}\right) (\pi - \pi_b) d^{\pi}_{M^*}}_{\infty} &= \norm{\left(Qc^{\pi_b}_{M^*} - Qc^{\pi_b}_{\hat{M}}\right) (\pi - \pi_b)}_{\infty} \norm{d^{\pi}_{M^*}}_{1} \\ 
%     &= \max_{x} \sum_{a} \left(Qc^{\pi_b}_{M^*} - Qc^{\pi_b}_{\hat{M}}\right) (\pi - \pi_b) \frac{1}{(1 - \gamma)}\\
%     &\leq \frac{\epsilon {Vc}_{\max}}{(1 - \gamma)}.
% \end{align}

% The last line comes from the $(\pi_b, e_Q, \epsilon)$-constrained property and Eq.~\ref{eq:eQc}. The next part is to show the second part of Eq.~\ref{eq:thm1-proof} is negative. All the terms of $d^{\pi}_{M^*}$ are positive so if all the terms of vector $Qc^{\pi_b}_{\hat{M}}(\pi - \pi_b) \leq 0$ we can upper bound the result by the first term. For each $x \in \X$, we have:
% \begin{align*}
%     Qc^{\pi_b}_{\hat{M}}(\pi - \pi_b) &= \sum_a Qc^{\pi_b}_{\hat{M}}(\pi(a|x) - \pi_b(a|x)) \\ 
%     &= \sum_a Qc^{\pi_b}_{\hat{M}}\pi(a|x) - {Vc}^{\pi_b}_{\hat{M}}(x) \\ 
%     &= \sum_{a} {Ac}^{\pi_b}_{i, \hat{M}}(x,a) \pi(a|x) \\ 
%     &\leq 0
% \end{align*}
% The last inequality comes from the construction in the Eq.~\ref{eq:s-opt}. This concludes the proof.

% \end{proof}

% ----- multi-step stuff

% It is possible to search over the class of $(\pib, e_P, \epsilon)$-constrained policies, where $e_p$ is the error bound over the transition function. Using the procedure in \cite{nadjahi2019safe} it is possible to guarantee safety bounds under the Assumption~\ref{assm:eP-bounded}.

% \begin{assumption}
% \label{assm:eP-bounded}
% There exists a constant $\kappa < \frac{1}{\gamma}$ such that, for all state-action pairs $(x,a) \in \X \times \A$ the following inequality holds:
% \begin{equation}
%     \label{eq:assm-eP}
%     \sum_{x', a'} e_P(x', a') \pi_b(a'|x') P^{*}(x'|x,a) \leq \kappa e_P(x,a).
% \end{equation}
% \end{assumption}

% \begin{prop}[Multi-step PI]
% \label{thm:eP-guarantee}
%     Under Assumption~\ref{assm:eP-bounded}, the policy $\pi$ returned by the policy iteration step given by \ref{eq:s-opt} satisfies satisfies the following inequalities in every state $x$ with probability at least $(1 - \delta)$:
%     \begin{align*}
%         \V{\pi}{R}{\mopt}(x) - \V{\pib}{R}{\mopt} (x) &\geq 
%         % (\V{\pi}{R}{\mhat}(x) - \V{\pib}{R}{\mhat}(x)) 
%         - 2 \norm{d^{\pib}_{\mopt}(\cdot|x)  
%         -  d^{\pib}_{\mhat}(\cdot|x) }_1 V^{R}_{\max} 
%         - \frac{1 + \gamma}{(1 - \gamma)^2 (1 - \kappa \gamma)}\epsilon V^R_{\max}, \\ 
%         \V{\pi}{\ci}{\mopt}(x) - \V{\pib}{\ci}{\mopt} (x) &\leq 2 \norm{d^{\pib}_{\mopt}(\cdot|x) + d^{\pib}_{\mhat}(\cdot|x) }_1 V^{\ci}_{\max} 
%         + \frac{1 + \gamma}{(1 - \gamma)^2 (1 - \kappa \gamma)}\epsilon V^{\ci}_{\max}  \tag{$\forall i$}.
%     \end{align*}
% \end{prop}


% The first inequality for $V^R$ follows directly from Theorem 2 of \cite{nadjahi2019safe}. We will show the proof for the second part, i.e, the extension with constraints using the same approach.

% \begin{proof}

% From \citep[Theorem 2]{nadjahi2019safe} we have under the Assumption ~\ref{assm:eP-bounded}, any $(\pib, e_P, \eps)$-constrained policy $\pi$ satisfies the following inequality for every state-action pair $(x, a)$ with probability at least $1-\delta$:
% \begin{align*}
%       \V{\pi}{R}{\mopt}(x) - \V{\pib}{R}{\mopt} (x) &\geq \V{\pi}{R}{\mhat}(x) - \V{\pib}{R}{\mhat} (x) - 2 \norm{d^{\pib}_{\mopt}(\cdot|x) -  d^{\pib}_{\mhat}(\cdot|x) }_1 V^{R}_{\max} \\
%         &\quad - \frac{1 + \gamma}{(1 - \gamma)^2 (1 - \kappa \gamma)}\epsilon V^R_{\max}.
% \end{align*}

% We will show that the first term $(\V{\pi}{R}{\mhat}(x) - \V{\pib}{R}{\mhat}(x))$ is positive, and use that to lower bound the expression above. From \citet[Proposition 1]{nadjahi2019safe} we have: 
% \begin{align*}
%     V^{\pi_1} - V^{\pi_2} &= Q^{\pi_2} (\pi_1 - \pi_2) d^{\pi_1}. 
% \end{align*}
% Substituting $\pi_1 =\pi$ and $\pi_2 =\pi_{b}$, we have:
% \begin{align*}
%     V^{\pi} - V^{\pi_b} &= Q^{\pi_b}(\pi - \pi_b) d^{\pi}
% \end{align*}
% As the term $d^\pi(x) \geq 0$ for any $x$, we need to show the term $Q^{\pi_b}(\pi - \pi_b)$ is positive.  For any $x \in \X$, we have:
% \begin{align*}
%     Q^{\pi_b}_{\hat{M}}(\pi - \pi_b)(x) &= \sum_a Q^{\pi_b}_{\hat{M}}(x,a)(\pi(a|x) - \pi_b(a|x)) \\ 
%     &= \sum_a Q^{\pi_b}_{\hat{M}}(x,a)\pi(a|x) - {V}^{\pi_b}_{\hat{M}}(x) \\ 
%     &= \sum_{a} {A}^{\pi_b}_{\hat{M}}(x,a) \pi(a|x) \\ 
%     &\geq 0
% \end{align*}
% The last inequality comes from \Cref{lemma:spibb-r-advatangeous}. This concludes the proof for the rewards. 

% The proof for the constraints follows the same procedure, but now we will upper bound the difference in performance. From \cite[Theorem 2]{nadjahi2019safe} for constraints we get, for any $i \in \{1, \dots, m\}$:
% \begin{align*}
%       \V{\pi}{\ci}{\mopt}(x) - \V{\pib}{\ci}{\mopt} (x) &\leq - \Big( \V{\pi}{\ci}{\mhat}(x) - \V{\pib}{\ci}{\mhat} (x) \Big) + 2 \norm{d^{\pib}_{\mopt}(\cdot|x) +  d^{\pib}_{\mhat}(\cdot|x) }_1 V^{\ci}_{\max} \\
%         &\quad - \frac{1 + \gamma}{(1 - \gamma)^2 (1 - \kappa \gamma)}\epsilon V^{\ci}_{\max}.
% \end{align*}
% As in the previous case, if we can show the term $(\V{\pi}{\ci}{\mhat}(x) - \V{\pib}{\ci}{\mhat} (x) )$ is negative, and use that to bound the expression. As in Section~\ref{app:proof-thm1}, we have: 
% \begin{align*}
%     \V{\pi}{\ci}{\mhat}(x) - \V{\pib}{\ci}{\mhat} (x) &\leq 0,  
% \end{align*}
% that again comes from using the construction in \ref{eq:s-opt}. This concludes the proof for constraints.
    
    
% \end{proof}


% HCPI-details
\section{HCPI - Additional details}
\label{app:hcpi-details}

\paragraph{Concentration Inequalities:} We experimented with the following concentration inequalities \citep{thomas2015highImprovement}:
\begin{itemize}[leftmargin=*]
    \item Extension of Empirical Bernstein \citep{maurer2009empirical}: This is the extension of
    Maurer \& Pontil's empirical Bernstein (MPeB) inequality. From Theorem 1 of \cite{thomas2015highEvaluation}: Let $X_1, \dots, X_n$ denote $n$ independent real-valued random variables, such that for each $i \in \{1,\dots,n\}$, we have $\pr(0 \le X_i) = 1,  \E[X_i] \le \mu$, and some fixed real-valued threshold $c_i > 0$. Let $\delta > 0$ and $Y_i = \min\{X_i, c_i\}$, then with probability at least $(1-\delta)$:
    \begin{align}
        \mu &\ge \sum_{i=1}^{n}\left(\frac{1}{c_i}\right)^{-1} \sum_{i=1}^{n} \frac{Y_i}{c_i} - \sum_{i=1}^{n}\left(\frac{1}{c_i}\right)^{-1} \frac{7n \ln(2/\delta)}{3n-1} - 
        \sum_{i=1}^{n}\left(\frac{1}{c_i}\right)^{-1} \sqrt{\frac{\ln(2/\delta)}{n-1} \sum_{i,j=1}^{n}\left(\frac{Y_i}{c_i} - \frac{Y_j}{c_j}\right)^2}.
    \end{align}
    In context of this paper, for the $k$\textsuperscript{th} reward function, $X_i$ denotes the $\IS$ estimated return for that trajectory, i.e., $\IS_k(\tau_i,\pi_t, \pib)$. Here, $c_i$ is a hyper-parameter that needs to be tuned. In \cite{thomas2015highImprovement}, a fixed value of $c$ is used for all $c_i$. 
    
    \item Student's t-test \citep{walpole1993probability}: This is an approximate concentration inequality that is based on the assumption that the mean returns are distributed normally. For $k$\textsuperscript{th} reward, the \Cref{eq:hcope-R-lower-bound} can be written as:
    \begin{align}
    \pr \Big( \J{\pi_t}{k}{\mopt} \ge \IS_{k}(\D, \pi_t, \pib) - \frac{\hat{\sigma}_k}{\sqrt{|\D|}}t_{1-\delta/d, |\D|-1} \Big) \ge 1 - \delta/d, 
    \end{align}
    where $\hat{\sigma}_k$ is the sample standard deviation:
    \begin{align}
    \hat{\sigma}_k &= \sqrt{\frac{1}{|\D|-1} \sum_{i=1}^{|\D|}(\IS(\tau_i, \pi_t, \pib) - \overline{\IS} )^2 },
    \end{align}
    and $\overline{\IS} = \frac{1}{|\D|}\sum_{i=1}^{|\D|} \IS(\tau_i,\pi_t,\pib)$ and $t_{1-\delta/d, |\D|-1}$ is the $100(1-\delta/d)$ percentile of the student t-distribution with $|\D|-1$ degrees of freedom. 
    
\end{itemize}

We experimented with both MPeB Extension (with $c=0.5$) and Student's t-test inequalities and found that the solutions returned by the former to be very conservative. Therefore, we use t-test in all of our experiments. Even though the t-test's assumption (normally distributed returns) is technically false, it's a reasonable assumption due to central limit theorem. 
% (it's an assumption used in almost all scientific research when computing p-values)
The consequence is that the failure rate (the chance of deploying an unsafe policy) can, in theory, be higher than desired, though, in practice, that's unlikely.


\paragraph{Regularization:} 
For small problems, \ref{eq:h-opt} can be solved with methods like CMA-ES \citep{hansen2006cma}. 
% In practice, a policy is first derived using the traditional offline RL methods based on $D_{tr}$ and is regularized using $\pib$ to obtain a set of candidate policies. 
% 
For stochastic policies, 
as the optimization problem in \ref{eq:h-opt} is difficult to solve directly, we need to resort to a regularization based heuristic \citep{thomas2015highImprovement, laroche2017safe}. Let $\pi_t$ denote the solution policy found using $\D_{tr}$ using any of the traditional offline RL methods. A set of candidate policies is built using the baseline policy: $\pi_{\text{Cand}} = \{ (1-\alpha)\pi_t + \alpha \pi_b\}$, where $\alpha \in \set{0.0, 0.1, \dots, 0.9}$ is the regularization hyper-parameter. 
The best performing candidate policy that satisfies the safety-test (the performance constraints based on $\D_s$) is then returned.
If none of the candidate policies satisfy the safety-test, the baseline policy is returned.

For finding $\pi_t$, we experimented with both the Linearized and Adv-Linearized baselines in \Cref{sec:synthetic-experiments} and found that Adv-Linearized worked better (higher improvement over $\pib$ while failure rate $<\delta$). Therefore in our experiments, we first find $\pi_t$ using Adv-Linearized and then regularize it using $\pib$ to build the set of candidate policies $\pi_{\text{Cand}}$. 


\paragraph{Safety-guarantees:}
We get the safety guarantees related to \ref{eq:h-opt} directly from \cite{thomas2015highImprovement, Thomas2019}.  The constraints of \ref{eq:h-opt} define the new safety-test that ensures a candidate policy will only be returned if the individual performance guarantees corresponding to each reward function are satisfied. This procedure will only make error in the scenario where the performance constraint related to $k$\textsuperscript{th} is satisfied, i.e, $( \IS_{k}(\D_{s}, \pi, \pib) - \CI_k(\D_s, \delta/d) \geq \mu_k)$,  but in practice the policy is not good enough $(\J{\pi}{k}{\mopt} < \mu_k)$. By transitivity this implies $\J{\pi}{k}{\mopt} < \big( \IS_{k}(\D_{s}, \pi, \pib) - \CI_k(\D_s, \delta/d) \big)$, which from \Cref{eq:hcope-R-lower-bound} we know can only occur with probability at most $\delta/d$. Using the union bound, we know that cumulative probability of the union of any of these $d$ possible scenarios is $\leq \delta$.


\paragraph{Computational cost:} Compared to regular HCPI, there is an increase in computational cost proportional to the number of reward functions $d$. The value and advantage functions estimation cost increases by a factor of $d$: respectively $\mathcal{O}(d|\X|^3)$ and $\mathcal{O}(d|\A||\X|^2)$, the $\IS$ estimation also increases by factor of $d$, and the computational cost for safety-test also increases by $d$: $\mathcal{O}(d|\D|)$. 


% ---------------------------------------------------
%                   Old Stuff 
% ---------------------------------------------------


% \subsection{Chernoff-Hoeffding's Inequality for upper and lower bounds}
% \label{app:hcpi-hoeffding-bounds}


% Let $X_1, \dots, X_n$ be $n$ independent random variables such that $\pr(X_i \in [a_i, b_i])=1$. Let $S_n = \frac{1}{n} \sum_i X_i$ denote the empirical mean and $\E[S_n]$ be the true mean. Then using Chernoff-Hoeffding's Inequality \cite{hagerup1990guided} we have  for any $t>0$, we have:
% \begin{equation*}
%     \pr (|S_n - \E[S_n]| \ge t) \le 2 \exp^{-\frac{2t^2}{\sum_{i=1}^{n}(b_i - a_i)^2 }}
% \end{equation*}

% To invert this bound, set $\delta = 2 \exp^{-\frac{2t^2}{\sum_{i=1}^{n}(b_i - a_i)^2 }} \in (0,1)$. Solving for $t$, we see that with probability at least $1-\delta$,
% \begin{align*}
%     |S_n - \E [S_n]| &\leq \sqrt{\frac{\ln(\frac{1}{\delta}) \sum_{i=1}^n(b_i - a_i)^2}{2n^2}} \\ 
%     \left| \frac{1}{n}\sum{X_i} - \E \left[ \frac{1}{n} \sum_{i} X_i\right] \right| &\le  \sqrt{\frac{\ln(\frac{1}{\delta}) \sum_{i=1}^n(b_i - a_i)^2}{2n^2}}.
% \end{align*}

% From the above inequality we get:
% \begin{align*}
%     \pr \left(\E \left[ \frac{1}{n} \sum_{i} X_i\right] \ge \frac{1}{n}\sum{X_i} -    \sqrt{\frac{\ln(\frac{1}{\delta}) \sum_{i=1}^n(b_i - a_i)^2}{2n^2}} \right) \ge 1 - \delta,
%     \intertext{and}
%     \pr \left(\E \left[ \frac{1}{n} \sum_{i} X_i\right]  \le   \frac{1}{n}\sum{X_i} + \sqrt{\frac{\ln(\frac{1}{\delta}) \sum_{i=1}^n(b_i - a_i)^2}{2n^2}} \right) \ge 1 - \delta.
% \end{align*}



% \harsh{Proof of Hoeffding's from: \url{http://www.stat.cmu.edu/~arinaldo/Teaching/36709/S19/Scribed_Lectures/Jan29_Tudor.pdf}
% }

% \subsection{Proof of \Cref{prop:hcpi-safety-guarantee}}
% \label{app:proof-h-opt-safety-guarantee}

% \begin{prop}[Safety guarantees with HCPI]
% \label{prop:hcpi-safety-guarantee}
% When the test set $\D_s$ is big enough to build reliable high-confidence lower bounds, i.e., $\forall i\in [d] \; \IS_{i}(\D_s, \pi_t, \pi_b) \approx \IS_{i}(\D, \pi_t, \pi_b)$, the policy $\pi$ returned by \Cref{eq:h-opt} will only violate the safety guarantees with probability at most $\delta$.
% \end{prop}

% The constraints of \Cref{eq:h-opt} define the new safety-test that ensures a candidate policy will only be returned if the individual performance guarantees are satisfied. This procedure will only make error in either of the following scenarios:
% \begin{itemize} %enumerate?
%     \item The performance constraint related to $R$ is satisfied $( \IS_{R}(\D_{s}, \pi, \pib) - \CI_R(\D_s, \delta) \geq b_R)$,  but in practice the policy is not good enough $(\J{\pi}{R}{\mopt} < b_R)$. By transitivity this implies $\J{\pi}{R}{\mopt} < \big( \IS_{R}(\D_{s}, \pi, \pib) - \CI_R(\D_s, \delta) \big)$, which from \Cref{eq:hcope-R-lower-bound} we know can only occur with probability at most $\delta/2$.
    
%     \item If the cost performance constraint is satisfied $( \IS_{C}(\D_{s}, \pi, \pib) + \CI_C(\D_s, \delta) \leq b_C )$ but the policy ends up violating the constraint in practice $(\J{\pi}{C}{\mopt} > b_{C})$. Again, by transitivity this means that $\J{\pi}{C}{\mopt} > \big( \IS_{C}(\D_{s}, \pi, \pib) + \CI_C(\D_s, \delta) \leq b_C \big)$, and that can only happen with probability at most $\delta/2$ because of \Cref{eq:hcope-C-upper-bound} .
% \end{itemize}

% Using the union bound, we know that cumulative probability of the union of either of these two events is $\leq \delta$.






% CMDP experiments extra details 
\section{Additional details for synthetic CMDP experiments}
\label{app:additional-details-for-synthetic-exp}


% ------------------------------------------------------
%               Solving CMDP
% ------------------------------------------------------
\subsection{Solving CMDP}
\label{app:cmdp-solver}

Constrained-MDPs~\citep{altman1999constrained} are MDPs with multiple rewards where $r_0$ is the main objective, and $r_1, \dots, r_{n-1}$ are the reward signals that are used to enforce some behavior or constraints. 

Let $\J{\pi}{i}{m}(\mu)$ denote the total expected discount reward under $r_i$ in an MDP $m$, when $\pi$ is followed from an initial state chosen at random from $\mu$, the initial state distribution. 
For some given reals $c_1, \dots, c_n$ (each corresponding to $r_i$), the CMDP optimization problem is to find the policy that maximizes the $\J{\pi}{0}{m}(\mu)$ subject to the constraints $\J{\pi}{i}{\mopt}(\mu) \le c_i$ :
\begin{align}
    \label{eq:cmdp-obj}
    &\max_\pi \J{\pi}{0}{m}(\mu) \\ 
        \quad \text{ s.t. } & \J{\pi}{i}{m}(\mu) \le c_i, \, \forall i \in \{1,\dots,n-1\}. \nonumber
\end{align}
    
% CMDPs have many interesting properties that make it different from regular MDPs. One particular property is the lack of deterministic optimal policies that makes reward shaping (or reduction to regular) not always possible .

The Dual LP based algorithm for solving CMDP is based on the occupation measure w.r.t. the optimal policy $\piopt$. For any policy $\pi$ and initial state $x_0 \sim \mu(\cdot)$, the occupancy measure is described as:
\begin{align*}
    \rho^{\pi}(x,a) &= \E \left[ \sum_{t=0}^{\infty} \gamma^t \mathbbm{1}\{x_t=x, a_t=a\} \Big| x_0, \pi \right], \forall x \in \X, \forall a \in \A. 
\end{align*}
The occupation measure at any state $x \in X$ is defined as $\sum_{a} \rho^{\pi}(x,a)$.  From \citep[Chapter 9]{altman1999constrained}, the problem of finding the optimal policy for a CMDP can be solved by the solving the following LP problem: %$2|\X||\A| +1 $
\begin{align*}
    \max_{\rho}  &\quad \sum_{x \in \X, a \in \A} \rho(x,a) r_0(x,a) \\  
    \texttt{s.t.}  &\quad \sum_{x \in \X, a \in \A} \rho(x,a) r_i(x,a) \leq c_i, \; \forall i \in \{1,\dots,n-1\}.
\end{align*}
 As $\rho$ is the occupation measure it also needs to satisfy the following constraints $\forall x \in \X$:
\begin{align*}
    \rho(x,a) &\geq 0,  \quad \forall a \in \A \\
    \sum_{x_p \in \X, a \in \A} \rho(x_p,a) (\mathbbm{1}\{ x_p = x\} - p(x|x_p,a))  &= \mathbbm{1}\{ x=x_0 \}
\end{align*}
The above constraints originate from the conservation of probability mass of a stationary distribution on a Markov process.  The state-action visitations should satisfy the single-step transpose Bellman recurrence relation:
\begin{equation*}
    \rho^{\pi}(x,a) = (1-\gamma) \mu(x) \pi(a|x) + \gamma \cdot p_{T}^{\pi} \rho^{\pi}(x,a), 
\end{equation*}
where transpose policy transition operator $p_{T}^{\pi}$ is a linear operator and is the mathematical transpose (or adjoint) of $p^{\pi}$ in the sense that $<y, p^{\pi}x> = <p_{T}^{\pi} y, x>$ for any $x, y$:
\begin{equation*}
    p_{T}^{\pi} \rho(x,a) \doteq \pi(a|s)\sum_{\tilde{x}, \tilde{a}} p(x|\tilde{x}, \tilde{a}) \rho(\tilde{x}, \tilde{a})
\end{equation*}
% Therefore the total number of constraints are 3|X||A|+1
% The above relations are used to define the \textit{transpose Bellman operator} \citep{nachum2019algaedice}:
% \begin{equation*}
%     \mathcal{B}^{T}_{\pi}(\rho)(x',a') \doteq \gamma \sum_{x,a} \pi(a'|x') P(x'|x,a) \rho(x,a) + (1-\gamma) \mu_{x_0}(x') \pi(a'|x').
% \end{equation*}

In conclusion, the complete dual problem can be written as:
\begin{align}
    \label{eq:cmdp-opt}
    \max_{\rho: \X \times \A \rightarrow \mathbb{R}_{+}}  &\quad \sum_{x \in \X, a \in \A} \rho(x,a) r_0(x,a) \\  
    \texttt{s.t.}  &\quad \sum_{x \in \X, a \in \A} \rho(x,a) r_i(x,a) \leq c_i,  \; \forall i \in \{1,\dots,n-1\}, \nonumber \\
    % &\quad \rho(x',a') = \sum_{x,a} \pi(a'|x') P(x'|x,a) \rho(x,a) + (1-\gamma) \mu_{x_0}(x') \pi(a'|x'), \tag{$\forall (x',a') \in \X \times \A$}\\
    % &\quad \sum_{a'} \rho(x',a') = \sum_{x,a} P(x'|x,a) \rho(x,a) + \mu_{x_0}(x'). \tag{$\forall x' \in \X$}
    &\quad \sum_{a} \rho(x,a) = \sum_{\tilde{x},\tilde{a}} p(x|\tilde{x},\tilde{a}) \rho(\tilde{x},\tilde{a}) + \mu(x). \tag{$\forall x \in \X$}
\end{align}


The solution of the above problem $\rho^{\star}$ gives the optimal (stochastic) policy of the form:
\begin{align*}
    \piopt(a|x) &= \frac{\rho^{\star}(x,a)}{\sum_a \rho^{\star}(x,a)} , \forall x \in \X, \forall a \in \A. 
\end{align*}


% ------------------------------------------------------
%               Fixed param details
% ------------------------------------------------------
\subsection{Additional results with fixed hyper-parameters}
\label{app:cmdp-fixed-param-results}

\Cref{fig:delta-0x1-params-grid} gives the individual plots for different $\bml, \rho$ combinations corresponding to the plot in \Cref{fig:delta-params-mean}. This is the fixed parameters setting in \Cref{sec:synthetic-experiments} where the same set of parameters are used across different $\bml, \rho$ combinations. Here, we run \ref{eq:s-opt} with $\epsilon \in \{0.01, 0.1, 1.0\}$ and \ref{eq:h-opt} with Doubly Robust IS estimator \citep{jiang2015doubly} and student's t-test.  
% The failure rate is less than $\delta$ for all the cases except for $\lR=0,\lC=1,\rho=0.9$ \cref{fig:delta-params-grid}, where only \ref{eq:h-opt} ends up with a failure rate $> \delta$. The only way \ref{eq:h-opt} can violate the constraints is either due to bad IS estimator or approximate CI that are not representative of the underlying distribution. By using an exact CI like Bernstein's inequality, we can make sure the failure rate is $< \delta$, however that CI is too conservative, and always returns the baseline as the solution for any of the $\lR, \lC, \rho$ combination.
The mean results with $\delta=0.9$ can be found in \Cref{fig:extra-delta-0.9-params-mean}. A more detailed plot containing the $\bml, \rho$ wise breakdown can be found in \Cref{fig:extra-delta-0.9-params-grid}.


\begin{figure*}
  \includegraphics[width=\textwidth]{doc/figures/random-mdps/delta_0x1_grid_sem.png}
  \caption{Results on random CMDPs with fixed parameters and $\delta=0.1$. 
  The different agents are represented by different markers and color lines. 
  Each point on the grid, corresponding to a $\bml, \rho$ combination, denotes the mean (with standard error bars) for the 100 randomly generated CMDPs. 
  The x-axis denotes the amount of data the agents were trained on. They y-axis for the top subplot in a grid cell represents the improvement over baseline and the y-axis for bottom subplot in a grid cell denotes the failure rate.
  The dotted black line represents the high-confidence parameter $\delta=0.1$.
  }
  \label{fig:delta-0x1-params-grid}
\end{figure*}


\begin{figure}
    \centering
  \includegraphics[scale=0.4]{doc/figures/random-mdps/delta_0x9_mean_sem.png}
  \caption{
  Mean results on random CMDPs with fixed parameters and $\delta=0.9$. 
  The different agents are represented by different markers and color lines. 
  Each point on the plot denotes the mean (with standard error bars) for 12 different $\bml,\rho$ combinations for the 100 randomly generated CMDPs (1200 datapoints). 
  The x-axis denotes the amount of data the agents were trained on. They y-axis for the left subplot represents the improvement over baseline and the y-axis for the right subplot in a grid cell denotes the failure rate.
  The dotted black line represents the high-confidence parameter $\delta=0.9$.
  }
  \label{fig:extra-delta-0.9-params-mean}
\end{figure}


\begin{figure*}
  \includegraphics[width=\textwidth]{doc/figures/random-mdps/delta_0x9_grid_sem.png}
  \caption{
  Results on random CMDPs with fixed parameters and $\delta=0.9$. 
  The different agents are represented by different markers and color lines. 
  Each point on the grid, corresponding to a $\bml, \rho$ combination, denotes the mean (with standard error bars) for the 100 randomly generated CMDPs. 
  The x-axis denotes the amount of data the agents were trained on. They y-axis for the top subplot in a grid cell represents the improvement over baseline and the y-axis for bottom subplot in a grid cell denotes the failure rate.
  The dotted black line represents the high-confidence parameter $\delta=0.9$.
  }
  \label{fig:extra-delta-0.9-params-grid}
\end{figure*}



% ------------------------------------------------------
%               Best params
% ------------------------------------------------------
\subsection{Additional results with tuned hyper-parameters}
\label{app:cmdp-best-param-results}

\Cref{fig:best-params-grid} gives the individual plots for different $\bml, \rho$ combinations corresponding to the plot in \Cref{fig:best-params-mean}.
The best hyper-parameters are tuned in a single environment and then are used to benchmark the results on 100 random CMDPs. The following procedure is used for selecting the best hyper-parameter candidates: We first generate a random CMDP and run different hyper-parameters on that environment instance. Next, we filter the candidates that violate the safety-constraint in that CMDP instance. From the remaining candidates, we select the one that yields the highest improvement over $\pib$. 

For \ref{eq:s-opt}, we searched for $\epsilon \in \{1e^{-4}, 1e^{-3}, 1e^{-2}, 1e^{-1}, 0.5, 1.0, 2.0, 5.0\}$. For \ref{eq:h-opt}, we used student's t-test with the following $\IS$ estimators: Importance Sampling (IS), Per Decision IS (PDIS), Weighted IS, Weighted PDIS and Doubly Robust (DR) \citep{precup2000eligibility, jiang2015doubly}.




\begin{figure*}
  \includegraphics[width=\textwidth]{doc/figures/random-mdps/benchmark_best_params_delta_0x1_grid.png}
  \caption{
  Results on 100 random CMDPs for different $\bml, \rho$ combinations with best $\epsilon, \IS$ combination for $\delta=0.1$.
  The different agents are represented by different markers and color lines. 
  Each point on the grid, corresponding to a $\bml, \rho$ combination, denotes the mean (with standard error bars) for the 100 randomly generated CMDPs. 
  The x-axis denotes the amount of data the agents were trained on. They y-axis for the top subplot in a grid cell represents the improvement over baseline and the y-axis for bottom subplot in a grid cell denotes the failure rate.
  The dotted black line represents the high-confidence parameter $\delta=0.1$.
  }
  \label{fig:best-params-grid}
\end{figure*}


% Just looking from the mean across different combinations we can say that \ref{eq:s-opt} performs better on average, on average, in terms of improvement while satisfying $< \delta$ failure rate. However, from \Cref{fig:best-params-mean}, we see that choosing the best hyper-param just based on 1 run can sometimes lead to an aggressive $\epsilon$ that makes \ref{eq:s-opt} having higher failure rate $> \delta$ for a few combinations (compared to \ref{eq:h-opt} that violates it only for one combination). Maybe if we chose hyper-params smartly we can optimize further? Does that add any value?


We plot the results based on the optimized hyper-parameters for a single CMDP in \Cref{fig:10x10-gridworld-seed-0} . Here, we plot the individual performance w.r.t $r_0$ (goal reward) and $r_1$ (pit reward) for multiple agents along with the baseline's performance.  Instead of working with surrogate measures, we investigate the returns for both $\J{\pi}{r_0}{\mopt}$ and $- \J{\pi}{r_1}{\mopt}$, and see what kind of scenarios lead to violation (all the returns are normalized in $[0,1]$). In \Cref{fig:10x10-gridworld-seed-0}, the intersection of the red and blue lines denotes the performance of the baseline in the true MDP.  As we observed in the mean plots, the Linearized baseline violate most of constraints for all the dataset sizes. The Adv-Linearized baseline violates the constraints mostly for low data settings ($\blacktriangledown$ marker with darker shades). There are more violations for higher values of $\rho$ as the $\pib$ gets better and the task gets tougher. We can observe that both \ref{eq:s-opt} and \ref{eq:h-opt} based agents (denoted by $\star$ and $\blacksquare$ markers) never leave the top-left quadrant and consistently satisfy the constraints. We also observe that the deviation from the origin increases with the increase in dataset size (represented via color of the agent).

\begin{figure*}
  \includegraphics[width=\textwidth]{doc/figures/random-mdps/qual_analys_best_params.png}
  \caption{
  Results on a random $10 \times 10$ synthetic CMDP. Each $\bml$ and $\rho$ combination represents a different setting denoted by the corresponding cell in the grid. The different agents are represented by different markers and the color of the marker denotes the amount of data the agent was trained on. 
  The x-axis for individual plots are normalized $- \mathcal{J}^{\pi}_{\mopt, r_1}$ returns (for pits), and y-axis are normalized $\mathcal{J}^{\pi}_{\mopt, r_0}$ returns (for goal).
  The red line denotes the performance of the baseline w.r.t. $- \mathcal{J}^{\pib}_{\mopt, r_1}$, and the blue line for $\mathcal{J}^{\pib}_{\mopt, r_0}$. For each plot in the grid, only the points in the top-left quadrant (defined by baseline's performance via red and blue lines) satisfy the constraint for that task.
  }
  \label{fig:10x10-gridworld-seed-0}
\end{figure*}


% ------------------------------------------------------
%               Lagrangian experiments
% ------------------------------------------------------
\subsection{Comparison with \cite{le2019batch}}
\label{app:lag-baseline}



\begin{figure}[t]
\centering
\begin{subfigure}[b]{1\textwidth}
    \includegraphics[width=1\textwidth]{doc/figures/lag-baseline/fig_1.png}
    \caption{Comparisons of Lagrangian \citep{le2019batch} with $\eta = 0.01$ and MO-SPIBB (\ref{eq:s-opt}) with $\epsilon=0.1$.}
    \label{fig:lag-only-single-sopt} 
\end{subfigure}
\\
\begin{subfigure}[b]{1\textwidth}
    \includegraphics[width=1\textwidth]{doc/figures/lag-baseline/fig_all.png}
    \caption{MO-SPIBB (\cref{eq:s-opt}) and Lagrangian \citep{le2019batch} comparisons across different hyper-parameters.}
    \label{fig:lag-multiple-sopt}
\end{subfigure}
% }
\caption[]{
\small
Results on 100 random CMDPs for different $\bml$ and $\rho$ combinations with $\delta=0.1$. The different agents are represented by different markers and colored lines. Each point on the plot denotes the mean (with standard error bars) for 12 different $\bml,\rho$ combinations for the 100 randomly generated CMDPs (1200 datapoints).  The x-axis denotes the amount of data the agents were trained on. 
The y-axis for left subplot in each sub-figure represents the improvement over baseline and the right subplot denotes the failure rate. The dotted black line in the right subplots represents the high-confidence parameter $\delta=0.1$.
\Cref{fig:lag-only-single-sopt} denotes the case when MO-SPIBB (\ref{eq:s-opt}) is run with $\epsilon=0.1$, MO-HCPI (\ref{eq:h-opt}) with $\IS=$ Doubly Robust (DR) estimator with student's t-test concentration inequality, and Lagrangian \citep{le2019batch} with $\eta = 0.01$ . 
\Cref{fig:lag-multiple-sopt} shows how MO-SPIBB and Lagrangian perform across different hyper-parameters.
\label{fig:lag-combined-results}}
\vskip -0.1in
\end{figure}



We test the method by \cite{le2019batch} (henceforth referred to as Lagrangian) in the synthetic navigation CMDP task described in \Cref{sec:synthetic-experiments}. In \Cref{fig:lag-only-single-sopt}, we present the results for the best performing Lagrangian baseline on 100 random CMDPs for different $\bml$ and $\rho$ combinations with $\delta=0.1$. Similar to \Cref{fig:delta-params-mean}, we provide a more detailed plot of how the Lagrangian baseline performs with different hyper-parameters in the above setting in \Cref{fig:lag-multiple-sopt}.


% In the figure, each point on the plot denotes the mean (with standard error bars) for 12 different ,  combinations for the 10 randomly generated CMDPs (120 data points). The x-axis denotes the amount of data the agents were trained on. The y-axis for the left subplot represents the improvement over baseline and the right subplot denotes the failure rate. The dotted black line in the right subplot represents the high-confidence parameter . 
% The MO-SPIBB (\cref{eq:s-opt}) is run with $\epsilon=0.1$ and MO-HCPI (\cref{eq:h-opt}) with $\IS$ = Doubly Robust estimator with student’s t-test concentration inequality. Similar to \Cref{fig:delta-params-mean}, we provide a more detailed plot of how the Lagrangian baseline performs with different hyper-parameters in the above setting in \Cref{fig:lag-multiple-sopt}.

\textbf{Results:} As expected, we observe that the Lagrangian baseline has a high failure rate, particularly in the low-data setting. 
This makes sense as the guarantees provided by \cite{le2019batch} are of the form $\mathcal{J}^\pi_{k,m^{\star}} - \mathcal{J}^{\pi_{b}}_{k, m^{\star}} \geq - \frac{C}{(1-\gamma)^{3/2}}$ (Theorem 4.4 of \cite{le2019batch}), where $C$ is a term that depends on a constant that comes from the Concentrability assumption (Assumption 1 of \cite{le2019batch}). This assumption upper bounds the ratio between the future state-action distributions of any non-stationary policy and the baseline policy under which the dataset was generated by some constant. In other words, it makes assumptions on the quality of the data gathered under the baseline policy. Unfortunately, this assumption cannot be verified in practice, and it is unclear how to get a tractable estimate of this constant. As such, this constant can be arbitrarily large (even infinite) when the baseline policy fails to cover the support of all non-stationary policies, for instance, when the baseline policy is not exploratory enough or when the size of the dataset is small. Hence, we observe a high failure rate of \cite{le2019batch} in the experiments, especially in the low data setting. Compared to \cite{le2019batch}, our performance guarantees do not make any assumptions on the quality of the dataset or the baseline. Therefore, our approach can ensure a low failure rate even in the low-data regime.
%  in the low data setting the concentrability coefficient can be arbitrarily high, and therefore the performance guarantees provided by Le et al. do not hold anymore. As the size of the dataset increases, we observe that the failure rate of the Le et al. starts decreasing, which seems reasonable because with more data more reliable MDP parameters are estimated and the baseline policy now covers more support of the space of all non-stationary policies required for the concentrability assumption to be valid. In contrast, both the MO-SPIBB and MO-SPIBB can ensure low failure rates even in low-data scenarios.




\textbf{Implementation details and Hyper-parameters:} We build on top of the publicly available code of \cite{le2019batch} released by the authors and extend it to our setting. 
% In the accompanied code, we also provide a standalone Jupyter notebook Lagrange\_agent.ipynb that contains the implementation of Algorithm 2 of Le et al. (Section 1 of the notebook). 
We are confident that our implementation is correct as we made sure it passes various sanity tests such as convergence of the primal-dual gap and feasibility on access to true MDP parameters.

The algorithm in \cite{le2019batch} (Algorithm 2, Constrained Batch Policy Learning) requires the following hyper-parameters:

\begin{itemize}
    \item Online Learning Subroutine: We use the same online learning algorithm as used by the authors in their experiments, i.e. Exponentiated Gradient \citep{kivinen1997exponentiated}.
    
    \item Duality gap $\omega$ : This denotes the primal-dual gap or the early termination condition. We tried the values in $\set{0.01, 0.001}$ and fix the value to $0.01$.
    
    \item Number of iterations: This parameter denotes the number of iterations for which the Lagrange coefficients should be updated. We experimented in the range $\set{100, 250, 500}$ and set this to $250$.
    
    \item Norm bound $B$: The bound on the norm of Lagrange coefficients vector. We tried the values in $\set{1, 10, 50, 100}$ and fixed it $10$.
    
    \item Learning rate $\eta$: This parameter denotes the learning rate for the update of the Lagrange coefficients via the online learning subroutine. We found that this is the most sensitive variable and we tried with values in $\set{0.005, 0.01, 0.05, 0.1, 0.5, 1.0, 5.0}$. For the final experiments, we benchmark with three different values $(0.01, 0.1, 1.0)$ as mentioned in the \Cref{fig:lag-multiple-sopt}.
    
\end{itemize}

We would like to point out that the hyper-parameter tuning for the Lagrangian baseline can be particularly challenging as in the low-data setting none of the combinations of the above hyper-parameters can ensure a low failure rate even though the duality gap has converged. 

% Environments in Le et al.: Le et al. test their approach on two domains: a grid-world domain under safety constraint, and a high-dimensional car racing domain. The car racing domain takes the raw pixel image tensor as input, and as we mentioned in the limitations, it is out of the scope of our work. We would like to highlight that the grid-world domain and empirical methodology in Le et al. are considerably weaker than the approach we take in our work with respect to the safety constraints. This can be observed in Figure 2 (middle) of Le et al. where even the online-RL (equivalent to the Linearized baseline in our case) also has no constraint violation. Moreover, they do not experiment with the different sizes of the dataset  or the quality of the baseline under which the dataset was collected . Compared to that, we base our grid-world environments on the standard CMDP safety benchmarks [2,3], have comparisons against different dataset sizes and baseline quality parameters, and also explicitly calculate the failure rate.

The above experiments show the advantage of our approach over \cite{le2019batch}, particularly in the low-data safety-critical tasks, where our methods can improve over the baseline policy while ensuring a low failure rate. 






% ------------------------------------------------------
%               Scaling experiments
% ------------------------------------------------------
\subsection{Scaling experiments with number of objectives $d$}
\label{app:cmdp-scaling-experiments}

We experimented with the different number of objectives $d$ to validate if the trends we observed for \ref{eq:s-opt} and \ref{eq:h-opt} in \Cref{sec:synthetic-experiments} also extend to $d>2$. 
In the CMDP formulation, as there can only be one primary reward, we extend the CMDP to include more than 1 type of pits. The extended CMDP now has $d-1$ different kinds of pits and corresponding reward functions, where the agent gets a pit reward of $-1$ if the agent steps into a cell containing that particular kind of pit. We relax the CMDP threshold to $c_i = -10.0$ as the CMDP problem gets harder with more number of pits, and a lower threshold makes the problem of finding $\piopt$ of a random CMDP easier. Therefore, the task objective for the agent in the extended CMDP is to reach the goal in the least amount of steps, such that it can only step into at most 10 pits of every different type. 

We use the same experiment methodology from \Cref{sec:synthetic-experiments}. As the focus is to see how the trends scale with $d$, we fix the $\bml$, with $\lambda_0=1.0$ and the rest of $\lambda_{\ge 1}=0.0$.  We compare \ref{eq:s-opt} and \ref{eq:h-opt} over different $|\D|\in \{ 10, 50, 500, 2000\}$, $\rho \in \{0.1, 0.4, 0.7, 0.9\}$, the fixed set of parameters: $\IS$=DR, $\CI=$student's t-test, $\epsilon\in \{0.001, 0.01, 0.1, 1.0\}$, and $\delta=0.1$.

The results over 10 random CMDPs with fixed parameters can be found in \Cref{fig:scale-exp-10-runs-fixed-delta}. We notice that the trends from \Cref{sec:synthetic-experiments} case still carry till $d\le 1+16$, where for some value of $\epsilon$, \ref{eq:s-opt} can lead to better improvement in $\pib$ while still having failure rate $<\delta$. However, $d > 1+16$ we see there are no obvious trends and both \ref{eq:s-opt} and \ref{eq:h-opt} tend to become very conservative and returning the baseline becomes the best solution choice.


\begin{figure*}
  \includegraphics[width=\textwidth]{doc/figures/random-mdps/scale_exp_10runs_grid.png}
  \caption{Scaling with $d$ with results on a 10 random CMDPs and $\delta=0.1$. The different agents are represented by different markers and color. Each point on the graph denotes the mean for 100 runs, the standard errors is denoted by the error bars. The x-axis denotes the amount of data the agents were trained on. They y-axis for the top plot in a grid represents the improvement over baseline and the y-axis for bottom plot denotes the failure rate.
  }
  \label{fig:scale-exp-10-runs-fixed-delta}
\end{figure*}


% ------------------------------------------------------
%               Miscellaneuos details
% ------------------------------------------------------
\subsection{Additional details}

For the experiments in \Cref{sec:synthetic-experiments}, on an Intel(R) Xeon(R) Gold 6230 CPU (2.10GHz), the baselines take around 3 seconds to run, and both \ref{eq:s-opt} and \ref{eq:h-opt} take about 5 seconds.

% Sepsis extra details
\section{Additional details for sepsis experiments}
\label{app:sepsis-details}


\subsection{Sepsis data and cohort details}
\label{app:sepsis-dataset}

We followed the pre-processing methodology from \cite{tang2020clinician, komorowski2018artificial} and we refer the reader to the original work for more details. 


The dosage of prescribed IV fluids and vasopressors is converted into discrete variables to be used as actions for the constructed MDP. Each type of action (IV or vasopressor) is divided into 4 bins (each representing one quantile), and an additional action for "No drug" (0 dose) is also introduced. As such, the $|\A| = 5 \times 5$.
The cohort statistics can be found in \Cref{table:cohort-stats}. The patient data consists of 48 dimensional time-series with features representing attributes such as demographics, vitals and lab work results (\Cref{table:sepsis-features-summary}). 
The patient data is discretized into 4-hour windows, each of which is pre-processed to be treated as a single time-step. The state-space and is discretized using a k-means based clustering algorithm to map the states to $750$ clusters. Two additional absorbing states are added for death and survival ($|\X|=752$).

\begin{table}[h]
    \centering
    \caption{Cohort statistics after following the data pre-processing methodology from \cite{tang2020clinician, komorowski2018artificial}.}
    \label{table:cohort-stats}
    \vskip 0.1in
    \begin{tabular}{lrrrr}
    \toprule
     Survivors &     N & \% Female & Mean Age & Hours in ICU \\
    \midrule
     Survivors & 18066 &    44.5\% &     64.1 &         56.6 \\
    Non-survivors &  2888 &    42.9\% &     68.8 &         60.9 \\
    \bottomrule
\end{tabular}
\end{table}

% The duration of an episode consists of 28 hours data before the onset of sepsis, and 52 hours data after that, leading to a maximum of 20 time-steps for each trajectory. 

\begin{table*}[h]
    \centering
    \caption{Summary of the patient state features from \citep[][Table 3]{tang2020clinician}.}
    \label{table:sepsis-features-summary}
    \begin{tabular}{lp{9cm}}
        \toprule
        Demographics/Static &  Age, Gender, SOFA, Shock Index, Elixhauser, SIRS, Re-admission, GCS - Glasgow Coma Scale \\
        \midrule
        Lab values & Albumin, Arterial pH, Calcium, Glucose, Hemoglobin, Magnesium, PTT - Partial Thromboplastin Time, Potassium, SGPT - Serum Glutamic-Pyruvic Transaminase, Arterial Blood Gas, Blood Urea Nitrogen, Chloride, Bicarbonate, International Normalized Ratio, Sodium, Arterial Lactate, CO2, Creatinine, Ionised Calcium, Prothrombin Time, Platelets Count, SGOT - Serum Glutamic-Oxaloacetic Transaminase, Total bilirubin, White Blood Cell Count \\
        \midrule
        Vital signs & Diastolic Blood Pressure, Systolic Blood Pressure, Mean Blood Pressure, PaCO2, PaO2, FiO2, PaO/FiO2 ratio, Respiratory Rate, Temperature (Celsius), Weight (kg), Heart Rate, SpO2 \\
        \midrule
        Intake and output events & Fluid Output - 4 hourly period, Total Fluid Output, Mechanical Ventilation \\
        \bottomrule
    \end{tabular}

\end{table*}




\subsection{Performance on changing hyper-parameters}
\label{app:sepsis-hyperparams}

For the experiments in \Cref{sec:sepsis-experiments}, we treat $\delta$ as a hyper-parameter. For \ref{eq:s-opt} instead of searching over both $\delta$ and $\epsilon$, we follow the strategy proposed in Soft-SPIBB: fix the $\delta=1.0$ and only search over $\epsilon$. 
For \ref{eq:h-opt}, we found that only DR and WDR gave reliable off-policy estimates so report the results with both of them with different $\delta$. As in previous sections, we used student's t-test as the choice of concentration inequality for \ref{eq:h-opt}.



% ------------------------------------------------------
%                   S-OPT
% ------------------------------------------------------
\subsubsection{\ref{eq:s-opt} parameters}

Here, we fix $\delta=1.0$ and try with different values of the hyper-parameter $\epsilon$ and directly report the results directly on the test set. The results are presented in \Cref{table:app-s-opt-test}.


% ------------------------------------------------------
%                   H-OPT
% ------------------------------------------------------
\subsubsection{\ref{eq:h-opt} parameters}


We run with different values of the hyper-parameter $\delta$ and directly report the results directly on the test set for different $\IS$ estimators. The results for DR estimator are presented in \Cref{table:app-hopt-DR-Adv} and for WDR estimator are presented in \Cref{table:app-hopt-WDR-Adv}.




% ------------------------------------------------------
%                   Additional Details 
% ------------------------------------------------------


\subsection{Additional qualitative Analysis}
\label{app:sepsis-qual-analysis}
We calculate how many rare-actions are recommended by different solution policies and compare them with the most common actions taken by the clinicians.
For each state, for the action recommended by a solution policy, we calculate the frequency with which that state-action was observed in the training data and calculate the percentage of time that state-action pair was observed among all the possible actions taken from that state.
Across all the states, the actions suggested by the traditional single-objective RL baseline are observed only 3\% of the time on average (5.3 observations per state). Whereas, the actions most commonly chosen by the clinicians  are observed 51.4\% of the time on average (138.2 observations per state). We study this behavior for two of the policies returned by MO-SPIBB that deviate the most from the baseline: for the policy returned by \ref{eq:s-opt} ($\bml=[1,1]$) the recommended actions are observed 24.8\% of time on average (61.0 observations per state) and for  \ref{eq:s-opt} ($\bml=[0,1]$) the recommended actions are observed 23.4\% of times (56.14 observations per state).


\subsection{Additional details}

For the experiments in \Cref{sec:sepsis-experiments}, on an Intel(R) Xeon(R) Gold 6230 CPU (2.10GHz), running the Linearized baseline takes around 30 seconds, Adv-Linearized takes around 60 seconds, \ref{eq:s-opt} take about 90-120 seconds and \ref{eq:h-opt} takes about 90 seconds.


% ------------------------------------------------------
%                   Tables
% ------------------------------------------------------

% ------- S-OPT ---------
\begin{table}[h]
    \centering
    \caption{
    Performance of various \ref{eq:s-opt} policy candidates (with different $\epsilon$) using DR and WDR estimation with standard errors on 10 random splits of the TEST dataset. 
    The red cells denote the corresponding safety constraint violation, i.e, either $\mathcal{J}_{0}^{\pi} < \mathcal{J}_{0}^{\pib}$ or $-\mathcal{J}_{1}^{\pi} > -\mathcal{J}_{1}^{\pib}$.}
    \label{table:app-s-opt-test}
    \vskip 0.1in
    % \small
    % \begin{longtable}{cccccc}
    \begin{adjustbox}{max width=1\textwidth,center}
    \begin{tabular}{cccccc}
    % \begin{longtable}{Y{1cm}Y{1cm}Y{2cm}Y{2cm}Y{2cm}Y{2cm}}
    \toprule
    \multicolumn{1}{c}{User preferences $(\bml)$} & \multicolumn{1}{c}{Policy} & \multicolumn{2}{c}{Survival return ($\mathcal{J}_0$)} & \multicolumn{2}{c}{Rare-treatment return ($- \mathcal{J}_1$)} \\
    \hline
    & & DR & WDR & DR & WDR  \\  \cline{3-6}
    & Clinician's ($\pib$) & 64.78 $\pm$ 0.90 & 64.78 $\pm$ 0.90          & 13.58 $\pm$ 0.19 & 13.58 $\pm$ 0.19  \\
    \midrule
    % ------ [1, 0] -------
    \multirow{4}{*}{$[\lambda_0=1.0, \lambda_1 = 0.0]$} 
& Linearized & 97.68 $\pm$ 0.22 & 97.58 $\pm$ 0.20   & \textcolor{red}{27.64 $\pm$ 1.11 }& \textcolor{red}{27.84 $\pm$ 1.09 } \\ 
& \ref{eq:s-opt}, $\epsilon=0.0$  & 64.78 $\pm$ 0.90 & 64.78 $\pm$ 0.90   & 13.58 $\pm$ 0.19 & 13.58 $\pm$ 0.19  \\
& \ref{eq:s-opt}, $\epsilon=0.001$  & 64.91 $\pm$ 0.90 & 64.91 $\pm$ 0.90   & 13.56 $\pm$ 0.19 & 13.56 $\pm$ 0.19  \\
& \ref{eq:s-opt}, $\epsilon=0.01$  & 66.11 $\pm$ 0.87 & 66.05 $\pm$ 0.86   & 13.42 $\pm$ 0.20 & 13.46 $\pm$ 0.20  \\
& \ref{eq:s-opt}, $\epsilon=0.1$  & 73.70 $\pm$ 0.84 & 71.96 $\pm$ 0.69   & 12.30 $\pm$ 0.39 & \textcolor{red}{13.80 $\pm$ 0.33 }  \\
& \ref{eq:s-opt}, $\epsilon=0.5$  & 78.19 $\pm$ 0.54 & 81.01 $\pm$ 0.36   & \textcolor{red}{16.21 $\pm$ 0.49 }& 13.10 $\pm$ 0.31  \\
& \ref{eq:s-opt}, $\epsilon=1.0$  & 84.03 $\pm$ 0.48 & 87.11 $\pm$ 0.33   & \textcolor{red}{15.54 $\pm$ 0.59 }& 12.17 $\pm$ 0.59  \\
& \ref{eq:s-opt}, $\epsilon=2.5$  & 90.05 $\pm$ 0.25 & 91.37 $\pm$ 0.20   & \textcolor{red}{15.35 $\pm$ 0.72 }& 13.53 $\pm$ 0.56  \\
& \ref{eq:s-opt}, $\epsilon=5.0$  & 91.58 $\pm$ 0.49 & 92.66 $\pm$ 0.28   & \textcolor{red}{15.39 $\pm$ 0.59 }& \textcolor{red}{13.71 $\pm$ 0.38 }\\
& \ref{eq:s-opt}, $\epsilon=10.0$  & 91.64 $\pm$ 0.47 & 92.68 $\pm$ 0.23   & \textcolor{red}{15.19 $\pm$ 0.59 }& 13.56 $\pm$ 0.42  \\
& \ref{eq:s-opt}, $\epsilon=\infty$  & 91.62 $\pm$ 0.46 & 92.68 $\pm$ 0.23   & \textcolor{red}{15.18 $\pm$ 0.59 }& 13.56 $\pm$ 0.42  \\
    \midrule
    % --------- [1, 1] -------------
    \multirow{4}{*}{$[\lambda_0=1.0, \lambda_1 = 1.0]$}
    & Linearized & 87.17 $\pm$ 0.48 & 89.11 $\pm$ 0.37   & 2.41 $\pm$ 0.47 & 1.52 $\pm$ 0.41\\
& \ref{eq:s-opt}, $\epsilon=0.0$  & 64.78 $\pm$ 0.90 & 64.78 $\pm$ 0.90   & 13.58 $\pm$ 0.19 & 13.58 $\pm$ 0.19\\
& \ref{eq:s-opt}, $\epsilon=0.001$  & 64.90 $\pm$ 0.90 & 64.90 $\pm$ 0.90   & 13.53 $\pm$ 0.19 & 13.54 $\pm$ 0.19\\
& \ref{eq:s-opt}, $\epsilon=0.01$  & 66.02 $\pm$ 0.88 & 65.94 $\pm$ 0.87   & 13.15 $\pm$ 0.20 & 13.20 $\pm$ 0.20\\
& \ref{eq:s-opt}, $\epsilon=0.1$  & 74.34 $\pm$ 0.78 & 72.04 $\pm$ 0.87   & 9.32 $\pm$ 0.29 & 10.48 $\pm$ 0.45\\
& \ref{eq:s-opt}, $\epsilon=0.5$  & 76.47 $\pm$ 0.50 & 78.42 $\pm$ 0.41   & 7.61 $\pm$ 0.44 & 5.02 $\pm$ 0.17\\
& \ref{eq:s-opt}, $\epsilon=1.0$  & 81.39 $\pm$ 0.46 & 84.54 $\pm$ 0.36   & 4.64 $\pm$ 0.40 & 2.38 $\pm$ 0.22 \\
& \ref{eq:s-opt}, $\epsilon=2.5$  & 86.26 $\pm$ 0.33 & 88.09 $\pm$ 0.24   & 1.98 $\pm$ 0.28 & 1.14 $\pm$ 0.27  \\
& \ref{eq:s-opt}, $\epsilon=5.0$  & 86.76 $\pm$ 0.47 & 88.55 $\pm$ 0.22   & 2.52 $\pm$ 0.48 & 1.55 $\pm$ 0.41\\
& \ref{eq:s-opt}, $\epsilon=10.0$  & 86.77 $\pm$ 0.49 & 88.58 $\pm$ 0.25   & 2.53 $\pm$ 0.50 & 1.57 $\pm$ 0.43  \\
& \ref{eq:s-opt}, $\epsilon=\infty$  & 86.77 $\pm$ 0.49 & 88.58 $\pm$ 0.25   & 2.53 $\pm$ 0.50 & 1.57 $\pm$ 0.43  \\
    \midrule
    % --------- [0, 0] ------------
    \multirow{4}{*}{$[\lambda_0=0.0, \lambda_1 = 0.0]$}
    & Linearized & \textcolor{red}{-89.39 $\pm$ 0.43} & \textcolor{red}{-90.90 $\pm$ 0.29 }  & \textcolor{red}{22.99 $\pm$ 0.40 }& \textcolor{red}{22.81 $\pm$ 0.30 }  \\ 
    & \ref{eq:s-opt}, $\epsilon=0.0$  & 64.78 $\pm$ 0.90 & 64.78 $\pm$ 0.90   & 13.58 $\pm$ 0.19 & 13.58 $\pm$ 0.19\\
& \ref{eq:s-opt}, $\epsilon=0.001$  & 64.80 $\pm$ 0.90 & 64.80 $\pm$ 0.90   & 13.57 $\pm$ 0.19 & 13.57 $\pm$ 0.19\\
& \ref{eq:s-opt}, $\epsilon=0.01$  & 64.92 $\pm$ 0.90 & 64.92 $\pm$ 0.90   & 13.50 $\pm$ 0.19 & 13.51 $\pm$ 0.19\\
& \ref{eq:s-opt}, $\epsilon=0.1$  & 65.78 $\pm$ 0.89 & 65.70 $\pm$ 0.88   & 13.20 $\pm$ 0.20 & 13.25 $\pm$ 0.20  \\
& \ref{eq:s-opt}, $\epsilon=0.5$  & 67.73 $\pm$ 0.82 & 67.22 $\pm$ 0.88   & 13.24 $\pm$ 0.24 & 13.55 $\pm$ 0.33\\
& \ref{eq:s-opt}, $\epsilon=1.0$  & 69.12 $\pm$ 0.75 & 67.90 $\pm$ 0.84   & 13.57 $\pm$ 0.27 & \textcolor{red}{14.39 $\pm$ 0.44 }\\
& \ref{eq:s-opt}, $\epsilon=2.5$  & 71.00 $\pm$ 0.63 & 68.28 $\pm$ 0.46   & \textcolor{red}{14.27 $\pm$ 0.30 }& \textcolor{red}{15.73 $\pm$ 0.40 }  \\
& \ref{eq:s-opt}, $\epsilon=5.0$  & 71.95 $\pm$ 0.54 & 69.27 $\pm$ 0.63   & \textcolor{red}{15.29 $\pm$ 0.39 }& \textcolor{red}{16.12 $\pm$ 0.70 }  \\
& \ref{eq:s-opt}, $\epsilon=10.0$  & 72.73 $\pm$ 0.64 & 71.17 $\pm$ 0.65   & \textcolor{red}{16.59 $\pm$ 0.37 }& \textcolor{red}{16.21 $\pm$ 0.41 }\\
& \ref{eq:s-opt}, $\epsilon=\infty$  & \textcolor{red}{60.27 $\pm$ 0.49} & \textcolor{red}{61.44 $\pm$ 0.85 }  & \textcolor{red}{18.40 $\pm$ 0.27 }& \textcolor{red}{15.36 $\pm$ 0.58 }  \\
    \midrule
    % ------- [0,1]
    \multirow{4}{*}{$[\lambda_0=0.0, \lambda_1 = 1.0]$}
& Linearized & \textcolor{red}{58.27 $\pm$ 2.18} & \textcolor{red}{60.52 $\pm$ 2.07 }  & 0.04 $\pm$ 0.03 & 0.02 $\pm$ 0.01  \\ 
    & \ref{eq:s-opt}, $\epsilon=0.0$  & 64.78 $\pm$ 0.90 & 64.78 $\pm$ 0.90   & 13.58 $\pm$ 0.19 & 13.58 $\pm$ 0.19  \\
& \ref{eq:s-opt}, $\epsilon=0.001$  & 64.83 $\pm$ 0.90 & 64.83 $\pm$ 0.90   & 13.52 $\pm$ 0.19 & 13.52 $\pm$ 0.19  \\
 & \ref{eq:s-opt}, $\epsilon=0.01$  & 65.36 $\pm$ 0.88 & 65.27 $\pm$ 0.88   & 12.96 $\pm$ 0.19 & 13.01 $\pm$ 0.19  \\
 & \ref{eq:s-opt}, $\epsilon=0.1$  & 71.35 $\pm$ 0.96 & 69.29 $\pm$ 0.92   & 7.75 $\pm$ 0.19 & 8.30 $\pm$ 0.18\\
 & \ref{eq:s-opt}, $\epsilon=0.5$  & 71.01 $\pm$ 0.72 & 71.30 $\pm$ 0.68   & 2.54 $\pm$ 0.37 & 1.50 $\pm$ 0.11  \\
 & \ref{eq:s-opt}, $\epsilon=1.0$  & 74.19 $\pm$ 0.57 & 76.11 $\pm$ 0.57   & 0.90 $\pm$ 0.14 & 0.34 $\pm$ 0.09  \\
 & \ref{eq:s-opt}, $\epsilon=2.5$  & 76.42 $\pm$ 0.61 & 77.20 $\pm$ 0.72   & 0.10 $\pm$ 0.06 & 0.06 $\pm$ 0.04  \\
 & \ref{eq:s-opt}, $\epsilon=5.0$  & 76.08 $\pm$ 0.65 & 76.87 $\pm$ 0.74   & 0.07 $\pm$ 0.05 & 0.05 $\pm$ 0.03\\
 & \ref{eq:s-opt}, $\epsilon=10.0$  & 76.07 $\pm$ 0.65 & 76.87 $\pm$ 0.73   & 0.07 $\pm$ 0.05 & 0.04 $\pm$ 0.03\\
 & \ref{eq:s-opt}, $\epsilon=\infty$  & 76.05 $\pm$ 0.65 & 76.85 $\pm$ 0.72   & 0.07 $\pm$ 0.05 & 0.04 $\pm$ 0.03\\
 \bottomrule
    \addtocounter{table}{-1} % to decrease the counter 
    % \end{longtable}
    \end{tabular}
    \end{adjustbox}
\end{table}


% ------- DR ---------
\begin{table}[h]
    \centering
    \caption{
    Performance of various \ref{eq:h-opt} policy candidates (with different $\delta$) using $\IS=$ DR estimator with standard errors on 10 random splits of the TEST dataset. 
    The red cells denote the corresponding safety constraint violation, i.e, either $\mathcal{J}_{0}^{\pi} < \mathcal{J}_{0}^{\pib}$ or $-\mathcal{J}_{1}^{\pi} > -\mathcal{J}_{1}^{\pib}$.}
    \label{table:app-hopt-DR-Adv}
    \vskip 0.1in
    \begin{adjustbox}{max width=1\textwidth,center}
    \begin{tabular}{cccccc}
    % \begin{longtable}{cccccc}
    \toprule
    \multicolumn{1}{c}{User preferences $(\bml)$} & \multicolumn{1}{c}{Policy} & \multicolumn{2}{c}{Survival return ($\mathcal{J}_0$)} & \multicolumn{2}{c}{Rare-treatment return ($- \mathcal{J}_1$)} \\
    \hline
    & & DR & WDR & DR & WDR  \\  \cline{3-6}
    & Clinician's ($\pib$) & 64.78 $\pm$ 0.90 & 64.78 $\pm$ 0.90          & 13.58 $\pm$ 0.19 & 13.58 $\pm$ 0.19  \\
    \midrule
    % ------ [1, 0] -------
    \multirow{4}{*}{$[\lambda_0=1.0, \lambda_1 = 0.0]$} 
    & Linearized & 97.68 $\pm$ 0.22 & 97.58 $\pm$ 0.20   & \textcolor{red}{27.64 $\pm$ 1.11 }& \textcolor{red}{27.84 $\pm$ 1.09 } \\ 
    & \ref{eq:h-opt}, $\delta=0.1$  & 65.95 $\pm$ 0.00 & 65.95 $\pm$ 0.00   & 13.37 $\pm$ 0.00 & 13.37 $\pm$ 0.00\\
    & \ref{eq:h-opt},  $\delta=0.3$  & 65.95 $\pm$ 0.00 & 65.95 $\pm$ 0.00   & 13.37 $\pm$ 0.00 & 13.37 $\pm$ 0.00\\
    & \ref{eq:h-opt}, $\delta=0.5$  & 65.95 $\pm$ 0.00 & 65.95 $\pm$ 0.00   & 13.37 $\pm$ 0.00 & 13.37 $\pm$ 0.00\\
    & \ref{eq:h-opt}, $\delta=0.7$  & 65.95 $\pm$ 0.00 & 65.95 $\pm$ 0.00   & 13.37 $\pm$ 0.00 & 13.37 $\pm$ 0.00\\
    & \ref{eq:h-opt}, $\delta=0.9$  & 65.95 $\pm$ 0.00 & 65.95 $\pm$ 0.00   & 13.37 $\pm$ 0.00 & 13.37 $\pm$ 0.00\\
    \midrule
    % --------- [1, 1] -------------
    \multirow{4}{*}{$[\lambda_0=1.0, \lambda_1 = 1.0]$}
    & Linearized & 87.17 $\pm$ 0.48 & 89.11 $\pm$ 0.37   & 2.41 $\pm$ 0.47 & 1.52 $\pm$ 0.41\\
    & \ref{eq:h-opt}, $\delta=0.1$  & 86.37 $\pm$ 0.00 & 88.03 $\pm$ 0.00   & 2.58 $\pm$ 0.00 & 1.43 $\pm$ 0.00\\
    & \ref{eq:h-opt}, $\delta=0.3$  & 86.37 $\pm$ 0.00 & 88.03 $\pm$ 0.00   & 2.58 $\pm$ 0.00 & 1.43 $\pm$ 0.00\\
    & \ref{eq:h-opt}, $\delta=0.5$  & 86.37 $\pm$ 0.00 & 88.03 $\pm$ 0.00   & 2.58 $\pm$ 0.00 & 1.43 $\pm$ 0.00\\
    & \ref{eq:h-opt}, $\delta=0.7$  & 86.37 $\pm$ 0.00 & 88.03 $\pm$ 0.00   & 2.58 $\pm$ 0.00 & 1.43 $\pm$ 0.00\\
    & \ref{eq:h-opt}, $\delta=0.9$  & 86.37 $\pm$ 0.00 & 88.03 $\pm$ 0.00   & 2.58 $\pm$ 0.00 & 1.43 $\pm$ 0.00  \\
    \midrule
    % --------- [0, 0] ------------
    \multirow{4}{*}{$[\lambda_0=0.0, \lambda_1 = 0.0]$}
    & Linearized & \textcolor{red}{-89.39 $\pm$ 0.43} & \textcolor{red}{-90.90 $\pm$ 0.29 }  & \textcolor{red}{22.99 $\pm$ 0.40 }& \textcolor{red}{22.81 $\pm$ 0.30 }  \\ 
    & \ref{eq:h-opt}, $\delta=0.1$  & 65.95 $\pm$ 0.00 & 65.95 $\pm$ 0.00   & 13.37 $\pm$ 0.00 & 13.37 $\pm$ 0.00  \\
    & \ref{eq:h-opt}, $\delta=0.3$  & 65.95 $\pm$ 0.00 & 65.95 $\pm$ 0.00   & 13.37 $\pm$ 0.00 & 13.37 $\pm$ 0.00\\
    & \ref{eq:h-opt}, $\delta=0.5$  & 65.95 $\pm$ 0.00 & 65.95 $\pm$ 0.00   & 13.37 $\pm$ 0.00 & 13.37 $\pm$ 0.00\\
    & \ref{eq:h-opt}, $\delta=0.7$  & 68.28 $\pm$ 0.00 & \textcolor{red}{63.25 $\pm$ 0.00 }  & \textcolor{red}{14.16 $\pm$ 0.00 }& \textcolor{red}{16.41 $\pm$ 0.00 } \\
    & \ref{eq:h-opt}, $\delta=0.9$  & 68.28 $\pm$ 0.00 & \textcolor{red}{63.25 $\pm$ 0.00 }  & \textcolor{red}{14.16 $\pm$ 0.00 }& \textcolor{red}{16.41 $\pm$ 0.00 }\\
    \midrule
    % ------- [0,1]
    \multirow{4}{*}{$[\lambda_0=0.0, \lambda_1 = 1.0]$}
    & Linearized & \textcolor{red}{58.27 $\pm$ 2.18} & \textcolor{red}{60.52 $\pm$ 2.07 }  & 0.04 $\pm$ 0.03 & 0.02 $\pm$ 0.01  \\ 
    & \ref{eq:h-opt}, $\delta=0.1$  & 76.54 $\pm$ 0.00 & 77.55 $\pm$ 0.00   & 0.09 $\pm$ 0.00 & 0.05 $\pm$ 0.00\\
    & \ref{eq:h-opt}, $\delta=0.3$  & 76.54 $\pm$ 0.00 & 77.55 $\pm$ 0.00   & 0.09 $\pm$ 0.00 & 0.05 $\pm$ 0.00\\
    & \ref{eq:h-opt}, $\delta=0.5$  & 76.54 $\pm$ 0.00 & 77.55 $\pm$ 0.00   & 0.09 $\pm$ 0.00 & 0.05 $\pm$ 0.00\\
    & \ref{eq:h-opt}, $\delta=0.7$  & 76.54 $\pm$ 0.00 & 77.55 $\pm$ 0.00   & 0.09 $\pm$ 0.00 & 0.05 $\pm$ 0.00\\
    & \ref{eq:h-opt}, $\delta=0.9$  & 76.54 $\pm$ 0.00 & 77.55 $\pm$ 0.00   & 0.09 $\pm$ 0.00 & 0.05 $\pm$ 0.00  \\
    \bottomrule
    \addtocounter{table}{-1} % to decrease the counter 
    % \end{longtable}
    \end{tabular}
    \end{adjustbox}
\end{table}




% ------- WDR ------------
\begin{table}[h]
    \centering
    \caption{
    Performance of various \ref{eq:h-opt} policy candidates (with different $\delta$) using $\IS=$Weighed DR (WDR) estimator with standard errors on 10 random splits of the TEST dataset. 
    The red cells denote the corresponding safety constraint violation, i.e, either $\mathcal{J}_{0}^{\pi} < \mathcal{J}_{0}^{\pib}$ or $-\mathcal{J}_{1}^{\pi} > -\mathcal{J}_{1}^{\pib}$.}
    \label{table:app-hopt-WDR-Adv}
    \vskip 0.1in
    \begin{adjustbox}{max width=1\textwidth,center}
    \begin{tabular}{cccccc}
    \toprule
    \multicolumn{1}{c}{User preferences $(\bml)$} & \multicolumn{1}{c}{Policy} & \multicolumn{2}{c}{Survival return ($\mathcal{J}_0$)} & \multicolumn{2}{c}{Rare-treatment return ($- \mathcal{J}_1$)} \\
    \hline
    & & DR & WDR & DR & WDR  \\  \cline{3-6}
    & Clinician's ($\pib$) & 64.78 $\pm$ 0.90 & 64.78 $\pm$ 0.90          & 13.58 $\pm$ 0.19 & 13.58 $\pm$ 0.19  \\
    \midrule
    % ------ [1, 0] -------
    \multirow{4}{*}{$[\lambda_0=1.0, \lambda_1 = 0.0]$} 
    & Linearized & 97.68 $\pm$ 0.22 & 97.58 $\pm$ 0.20   & \textcolor{red}{27.64 $\pm$ 1.11 }& \textcolor{red}{27.84 $\pm$ 1.09 } \\ 
    & \ref{eq:h-opt}, $\delta=0.1$  & 65.95 $\pm$ 0.00 & 65.95 $\pm$ 0.00   & 13.37 $\pm$ 0.00 & 13.37 $\pm$ 0.00\\
    & \ref{eq:h-opt}, $\delta=0.3$  & 65.95 $\pm$ 0.00 & 65.95 $\pm$ 0.00   & 13.37 $\pm$ 0.00 & 13.37 $\pm$ 0.00\\
    & \ref{eq:h-opt}, $\delta=0.5$  & 65.95 $\pm$ 0.00 & 65.95 $\pm$ 0.00   & 13.37 $\pm$ 0.00 & 13.37 $\pm$ 0.00\\
    & \ref{eq:h-opt}, $\delta=0.7$  & 65.95 $\pm$ 0.00 & 65.95 $\pm$ 0.00   & 13.37 $\pm$ 0.00 & 13.37 $\pm$ 0.00\\
    & \ref{eq:h-opt}, $\delta=0.9$  & 91.39 $\pm$ 0.00 & 92.61 $\pm$ 0.00   & \textcolor{red}{15.41 $\pm$ 0.00 }& \textcolor{red}{13.89 $\pm$ 0.00 } \\
    \midrule
    % --------- [1, 1] -------------
    \multirow{4}{*}{$[\lambda_0=1.0, \lambda_1 = 1.0]$}
    & Linearized & 87.17 $\pm$ 0.48 & 89.11 $\pm$ 0.37   & 2.41 $\pm$ 0.47 & 1.52 $\pm$ 0.41\\
    & \ref{eq:h-opt}, $\delta=0.1$  & 86.37 $\pm$ 0.00 & 88.03 $\pm$ 0.00   & 2.58 $\pm$ 0.00 & 1.43 $\pm$ 0.00\\
    & \ref{eq:h-opt}, $\delta=0.3$  & 86.37 $\pm$ 0.00 & 88.03 $\pm$ 0.00   & 2.58 $\pm$ 0.00 & 1.43 $\pm$ 0.00\\
    & \ref{eq:h-opt}, $\delta=0.5$  & 86.37 $\pm$ 0.00 & 88.03 $\pm$ 0.00   & 2.58 $\pm$ 0.00 & 1.43 $\pm$ 0.00  \\
    & \ref{eq:h-opt}, $\delta=0.7$  & 86.37 $\pm$ 0.00 & 88.03 $\pm$ 0.00   & 2.58 $\pm$ 0.00 & 1.43 $\pm$ 0.00\\
    & \ref{eq:h-opt}, $\delta=0.9$  & 86.37 $\pm$ 0.00 & 88.03 $\pm$ 0.00   & 2.58 $\pm$ 0.00 & 1.43 $\pm$ 0.00\\
    \midrule
    % --------- [0, 0] ------------
    \multirow{4}{*}{$[\lambda_0=0.0, \lambda_1 = 0.0]$}
    & Linearized & \textcolor{red}{-89.39 $\pm$ 0.43} & \textcolor{red}{-90.90 $\pm$ 0.29 }  & \textcolor{red}{22.99 $\pm$ 0.40 }& \textcolor{red}{22.81 $\pm$ 0.30 }  \\ 
    & \ref{eq:h-opt}, $\delta=0.1$  & 65.95 $\pm$ 0.00 & 65.95 $\pm$ 0.00   & 13.37 $\pm$ 0.00 & 13.37 $\pm$ 0.00  \\
    & \ref{eq:h-opt}, $\delta=0.3$  & 65.95 $\pm$ 0.00 & 65.95 $\pm$ 0.00   & 13.37 $\pm$ 0.00 & 13.37 $\pm$ 0.00\\
    & \ref{eq:h-opt}, $\delta=0.5$  & 65.95 $\pm$ 0.00 & 65.95 $\pm$ 0.00   & 13.37 $\pm$ 0.00 & 13.37 $\pm$ 0.00  \\
    & \ref{eq:h-opt}, $\delta=0.7$  & 65.95 $\pm$ 0.00 & 65.95 $\pm$ 0.00   & 13.37 $\pm$ 0.00 & 13.37 $\pm$ 0.00\\
    & \ref{eq:h-opt}, $\delta=0.9$  & 65.95 $\pm$ 0.00 & 65.95 $\pm$ 0.00   & 13.37 $\pm$ 0.00 & 13.37 $\pm$ 0.00  \\
    \midrule
    % ------- [0,1]
    \multirow{4}{*}{$[\lambda_0=0.0, \lambda_1 = 1.0]$}
    & Linearized & \textcolor{red}{58.27 $\pm$ 2.18} & \textcolor{red}{60.52 $\pm$ 2.07 }  & 0.04 $\pm$ 0.03 & 0.02 $\pm$ 0.01  \\ 
    & \ref{eq:h-opt}, $\delta=0.1$  & 76.54 $\pm$ 0.00 & 77.55 $\pm$ 0.00   & 0.09 $\pm$ 0.00 & 0.05 $\pm$ 0.00\\
    & \ref{eq:h-opt}, $\delta=0.3$  & 76.54 $\pm$ 0.00 & 77.55 $\pm$ 0.00   & 0.09 $\pm$ 0.00 & 0.05 $\pm$ 0.00\\
    & \ref{eq:h-opt}, $\delta=0.5$  & 76.54 $\pm$ 0.00 & 77.55 $\pm$ 0.00   & 0.09 $\pm$ 0.00 & 0.05 $\pm$ 0.00\\
    & \ref{eq:h-opt}, $\delta=0.7$  & 76.54 $\pm$ 0.00 & 77.55 $\pm$ 0.00   & 0.09 $\pm$ 0.00 & 0.05 $\pm$ 0.00\\
    & \ref{eq:h-opt}, $\delta=0.9$  & 76.54 $\pm$ 0.00 & 77.55 $\pm$ 0.00   & 0.09 $\pm$ 0.00 & 0.05 $\pm$ 0.00\\
    \bottomrule
    \addtocounter{table}{-1} % to decrease the counter 
    \end{tabular}
    \end{adjustbox}
\end{table}





%%%%%%%%%%%%%%%%%%%%%%%%%%%%%%%%%%%%%%%%%%%%%%%%%%%%%%%%%%%%%%%%%%%%%%%%%%%%%%%
%%%%%%%%%%%%%%%%%%%%%%%%%%%%%%%%%%%%%%%%%%%%%%%%%%%%%%%%%%%%%%%%%%%%%%%%%%%%%%%


\end{document}

