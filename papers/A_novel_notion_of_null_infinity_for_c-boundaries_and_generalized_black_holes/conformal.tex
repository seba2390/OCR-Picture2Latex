\section{Conformal extensions vs $c$-completion}
\label{conf}

The standard, mathematically precise definitions of {\em null infinity} and {\em black hole} are based on the notion of  conformal boundary of a spacetime as introduced by R. Penrose \cite{PenroseAsymptoticStructure1963,PenroseConformalInfinity1964}. We wish to generalize these notions to the context of $c$-boundaries, and accordingly we start by revisiting the relation between these two boundaries, which comprises some of the main results in \cite[Section 4]{Floresfinaldefinitioncausal2011}.

Henceforth, the $c$-completion $\overline{M}$ of $(M,g)$ will always be understood to be endowed both with the chronological relations and the chronological topology as defined in the previous section.
%Finally, we will particularize this study to the case of globally hyperbolic spacetimes.

%In order to obtain consistency on our definition of black holes, it is important to compare it with previous defintion obtained with the conformal boundary. In \cite[Section 4]{Floresfinaldefinitioncausal2011} the authors make a detailed comparison between both constructions, the conformal and the c-completion. Along this section, we will review the main points of their studies.

%\smallskip
%
%Let us begin by recalling the concept of conformal boundary.

\begin{definition}\label{def:envelopment}
  Given a strongly causal spacetime $(M,g)$, a {\em conformal extension} of $(M,g)$ is an open embedding $i:(M,g) \hookrightarrow (\tilde{M},\tilde{g})$ of $(M,g)$ into some strongly causal spacetime $(\tilde{M},\tilde{g})$ preserving time-orientation, for which there exists a strictly positive function $\Omega\in C^{\infty}(M)$ satisfying
    \begin{equation}
\label{conformalfactor1}
i^{*}\tilde{g}=\Omega^2 g.
\end{equation}
$\Omega$ is called the {\em conformal factor} of the conformal extension $i$.

The {\em conformal completion} of $(M,g)$ with respect to the conformal extension $i$ is defined as the (topological) closure $\overline{M}_i:=\overline{i(M)}\subset \tilde{M}$, and the associated {\em conformal boundary} as the (topological) boundary $\partial_i M:=\overline{i(M)}\setminus i(M)$.

Finally, we denote by $\overline{M}_i^*$ (resp. $\partial_i^* M$) the set of all the {\em accessible points} of the conformal completion (resp. conformal boundary), that is, the set of points on $\overline{M}_i$ (resp. $\partial_i M$) which are endpoints\footnote{Here, and throughout this section, the term ``endpoint'' is taken in the usual spacetime sense.} of timelike curves contained in $M$.
\end{definition}

\begin{remark}\label{rmk1}
{\em A few comments about Definition \ref{def:envelopment} are in order.
\begin{enumerate}
\item Note that we {\em do not} require, in this definition, that a given conformal boundary should have any regularity; even if it is piecewise smooth, we do not demand that the conformal factor extends smoothly to the boundary. As it stands, the definition is too weak to be useful in most applications, and has to be supplemented according to specific needs. In particular, the definition we give here is much more general than the ones usually found in the literature, which require at least $C^1$ smoothness of the boundary, extendibility of the conformal factor to the boundary and a host of other properties (see, e.g., Ch. 11 of \cite{WaldGeneralRelativity1984}). We too shall add some extra assumptions in what follows, but will shall do so gradually as needed, and the added assumptions will still be fairly general and comprise most concrete strongly causal examples in the literature.
\item If $(M,g)$ is extendible as a strongly causal spacetime, i.e., it is {\em isometric} to a proper open subset of a larger strongly causal spacetime $(\tilde{M}, \tilde{g})$, then the latter gives a conformal extension of the former with conformal factor $\Omega \equiv 1$. To avoid this kind of triviality, one usually assumes, in concrete applications, that $(M,g)$ is in some sense ``maximal'' in the strongly causal class.
\item The ``larger'' spacetime used in a conformal extension may well be $(M,g)$ itself. As a standard example, consider the following. Let $M = \mathbb{R}^2$ with the flat metric
\[
g = -dudv,
\]
and the time orientation such that both $\partial_u$ and $\partial_v$ are future-directed null vector fields. This spacetime is geodesically complete, and hence inextendible. Define
\[
i: (u,v) \in M=\mathbb{R}^2 \mapsto (\arctan u,\arctan v) \in \mathbb{R}^2
\]
and consider the smooth function
\[
\label{conformalfactor2}
\Omega : (u,v) \in \mathbb{R}^2 \mapsto \cos u \cos v \in \mathbb{R}.
\]
Then it is easy to check that $(\mathbb{R}^2,g)$ is a (nontrivial) conformal extension of $(M,g)$ such that the image of $M$ by $i$ is the open square $Q:= (-\pi/2, \pi/2)^2$, with conformal factor $\Omega \circ i$. Note that $\Omega$ vanishes on the boundary of $Q$ in $\mathbb{R}^2$. Thus, $(M,g)$ is conformally extended via a mapping onto an open set of itself.
\item Let $i:(M,g) \hookrightarrow (\tilde{M},\tilde{g})$ be a conformal extension of $(M,g)$. Since $i$ is a diffeomorphism when viewed as a map onto its open image $i(M)\subset \tilde{M}$, consider its (smooth) inverse $i^{-1}: i(M) \rightarrow M$. Then $(i(M),(i ^{-1})^{\ast} g)$ is a spacetime on its own right, and moreover it is isometric to $(M,g)$ by construction. Hence, there is no loss of generality in regarding $M \subseteq \tilde{M}$ and $i$ to be the inclusion map. We shall often do so in what follows, and will then abuse of notation by referring to $(\tilde{M},\tilde{g})$ itself as a conformal extension and discard any reference to the map $i$. In this case, we write the condition (\ref{conformalfactor1}) as
\[
\label{conformalfactor3}
\tilde{g}|_{M} = \Omega^2 g.
\]

\item Even upon identifying isometric spacetimes as in the previous item, conformal extensions, and also the associated conformal boundaries, are clearly far from being unique. But exploiting its relationship with the $c$-boundary, as we shall see below, allows one to build its accessible part in an essentially unique fashion.
% \item[5)] The existence of a conformal extension, as well as the conformal extension itself, are conformally invariant notions in the following sense. If $(\tilde{M},\tilde{g})$ is a conformal extension of $(M,g)$ with conformal factor $\Omega \in C^{\infty}(M)$, then it is also a conformal extension (with the same conformal boundary) of the spacetime $(M,\omega^2 g)$, where $\omega \in C^{\infty}(M)$ is a strictly positive function, with conformal factor $\Omega/\omega$. {\em Note, however, and this will be important in the next section, that the extendibility of the conformal factor to the boundary, and the values that such an extension might have there, are not a conformally invariant notion, depending rather sensitively on how $\Omega$ and $\omega$ are related.}
\end{enumerate}
}
\end{remark}

A given conformal completion will always be assumed to be endowed with the induced topology from $\tilde{M}$ and with a chronological (resp. causal) relation defined as follows: two points $p,q\in \overline{M}_i$ are chronologically (resp. causally) related, $p\ll_i q$ (resp. $p \leq _i q$) if there exists a smooth future-directed timelike (resp. causal) curve $\gamma:[a,b]\rightarrow \overline{M}_i$ from $p$ to $q$ with $\gamma|_{(a,b)}\subset M$ (see \cite[Section 4.1]{Floresfinaldefinitioncausal2011} for additional discussion on these choices).

\medskip

As one might expect, in general the $c$-completion differs from a given conformal one. In fact, despite its name, the conformal completion is {\em not} conformally invariant (see, for instance, \cite[Figure 10]{Floresfinaldefinitioncausal2011}). However, (and this is one of the main points of this section) under some additional conditions on the conformal extension, it is possible that both completions coincide. We shall presently discuss some of these conditions.

%The first requirement allows to ensure that the envelopment encapsulate all the possible boundary points of $M$.
\begin{definition}\label{def:chronologically complete}
Let $(M,g)$ be a spacetime. A conformal extension $(\tilde{M},\tilde{g})$ is said to be {\em chronologically complete} if any timelike curve $\gamma:[a,b)\rightarrow M$ which is inextendible in $M$ has an endpoint $p$ on the associated conformal boundary.
\end{definition}
Clearly, this condition parallels the point \ref{item:c-completioncompleta} in Theorem  \ref{thm:mainc-completion}, and is meant to ensure that the conformal completion has ``enough points''. (Otherwise, by suitably deleting points in a conformal completion, we have a highly non-unique construction which will then become geometrically useless.)

Another important requirement is that, around the points $p$ of the (accessible) conformal boundary, the causal structure of $(\tilde{M},\tilde{g})$ is adapted to that of $(M,g)$. To formalize this idea, let us begin with the concept of {\em timelike deformable points} (see \cite[Definition 4.10]{Floresfinaldefinitioncausal2011})

\begin{definition}\label{def:deformable}
Consider a continuous curve $\gamma:[a,b]\rightarrow \overline{M}_i$ such that $\gamma|_{[a,b)}$ is a future-directed smooth timelike curve contained in $M$ and $\gamma(b)\in \partial_i M$. Then, $\gamma$ is {\em future deformably timelike} if there exists a neighborhood $U=\tilde{U}\cap \overline{M}_i$ of $\gamma(b)$ (where $\tilde{U}$ is an open set of $\tilde{M}$) such that $\gamma(a)\ll_i \omega$ for all $\omega\in U$. (The notion of {\em past deformably timelike} is analogous.)

  An accessible point $p\in \partial_i^* M$ is {\em timelike deformable} if all the TIPs and TIFs associated to $p$ (i.e., TIPs and TIFs defined by timelike curves on $M$ with endpoint $p$) are intersections with $M$ of the chronological pasts or futures in $(\tilde{M},\tilde{g})$ of timelike deformable curves in $\overline{M}_i$.
\end{definition}

The previous notion ensures only that the {\em chronological} relation of the extension is well-behaved with respect to that of $(M,g)$; the following definition extends such good behaviour to the {\em causal} relation.

\begin{definition}\label{def:transitive}
  An accessible point $p\in \partial_i^* M$ is {\em (locally) timelike transitive} if it admits a neighborhood $V=\overline{M}_i\cap \tilde{V}$ ($\tilde{V}$ open in $\tilde{M}$) such that for any $q,q'\in V$:
  \begin{itemize}
  \item $q\ll_i p \leq_i q' \Rightarrow q\ll_i q'$.

  \item $q\leq_i p \ll_i q'\Rightarrow q\ll_i q'$.
  \end{itemize}
\end{definition}
 If an accessible point $p\in \partial_i^* M$ satisfies both timelike deformability and timelike transitivity, then we say that $p$ is {\it regularly accessible}. If all the accessible boundary points are regularly accessible, we will just say that $\partial_i^* M$ itself is {\it regularly accessible}.

\smallskip

We are now ready to present the main result in \cite[Section 4]{Floresfinaldefinitioncausal2011}.

\begin{theorem}\cite[Theorem 4.16]{Floresfinaldefinitioncausal2011} \label{thm:causaltoconformal} Let $i:(M,g) \hookrightarrow (\tilde{M},\tilde{g})$ be a chronologically complete extension of $(M,g)$. If the accessible conformal boundary $\partial_i^* M$ is regularly accessible, then the accessible part $\overline{M}_i^*$ of the conformal completion and the $c$-completion $\overline{M}$ are equivalent in the following precise sense:
\begin{enumerate}[label=(\roman*)]
  \item \label{thmcausaltoconformal-defPsi} There exists a homeomorphism $\Psi:\overline{M}\rightarrow \overline{M}_i^*$, which maps boundary to boundary.
  %is well defined and bijective. In fact, given $(P,F)\in \overline{M}$, $\Psi((P,F))$ will be the endpoint\footnote{??A cual endpoint te refieres?? Hay dos, no?} of any timelike curve defining $P$ or $F$.
  %There exists a map $\Psi:\overline{M}\rightarrow \overline{M}_i^*$ which is well defined and bijective. In fact, given $(P,F)\in \overline{M}$, $\Psi((P,F))$ will be the endpoint\footnote{??A cual endpoint te refieres?? Hay dos, no?} of any timelike curve defining $P$ or $F$.
%  \item $\Psi$ is a homeomorphism.
  \item \label{thmcausaltoconformal-chrniso} $\Psi$ is a chronological isomorphism (i.e., $\Psi$ and $\Psi^{-1}$  preserve the chronological relations).
  \end{enumerate}
\end{theorem}





















%
%\smallskip
%
%If $(M,g)$ is globally hyperbolic, then the regular accessibility can be obtained from more natural causal conditions. Moreover, in this case one is also able to deduce some important structural properties of the conformal boundary and the $c$-boundary which do not apply for general strongly causal spacetimes. Let us begin introducing the basic concepts we are going to use on the rest of this section.
%
%\begin{definition}
%\label{def1}
%Let $A$ be a subset of a spacetime $(M,g)$.
%\begin{itemize}
%\item[a)] $A$ is {\em causally convex} if any causal curve segment in $(M,g)$ with endpoints in $A$ is entirely contained in $A$,
%\item[b)] $A$ is {\em future-precompact} (resp. {\em past-precompact} ) is there exists a compact set $K \subset M$ such that $A \subset I^-(K)$ (resp. $A \subset I^+(K)$).
%\end{itemize}
%The {\em future boundary} (resp. {\em past boundary}) of $A$ is\footnote{We have adopted the following convention: the $\hat{ }$ (resp. $\check{ }$ ) is a superscript for future (resp. past) $c$-boundaries and $^{+}$ (resp. $^{-}$) is a superscript for future (resp. past) boundaries in $M$ of {\em subsets} thereof.}
%\[
%\partial ^+ A := I^+(A) \cap \partial A \mbox{ (resp. $\partial ^- A := I^-(A) \cap \partial A$)}.
%\]
%\end{definition}
%
%Note that any PIP (resp. PIF) is future-precompact (resp. past-precompact) in $(M,g)$. Moreover, if $(M,g)$ is globally hyperbolic and $A \subset M$ is both future- and past-precompact, then it is indeed precompact, which motivates this terminology.
%
%\begin{proposition}
%\label{convexitypastfuture} Let $A$ be a subset of a spacetime $(M,g)$. Then, the following statements hold:
%\begin{enumerate}[label=(\roman*)]
%\item Suppose $A$ is open. Then, $\partial ^+ A = \emptyset$ (resp. $\partial ^- A = \emptyset$) if and only if $A$ is a future set (resp. past set).
%\item If $A$ is causally convex, then $\partial ^{\pm} A$, if non-empty, are achronal $C^0$ hypersurfaces in $(M,g)$. If, in addition, $A$ is open, then $A \cup \partial ^{\pm} A$ is causally convex.
%\item If $A$ is either a future or a past set in $(M,g)$, then $A$ is causally convex.
%\end{enumerate}
%\end{proposition}
%\begin{proof}
%  \textit{(i)} The ``if'' part is immediate. For the converse, assume that $\partial ^+ A = \emptyset$, and let $p \in A, q \in I^+(p)$. Pick any future-directed timelike curve $\alpha:[0,1] \rightarrow M$ such that $\alpha(0) =p$ and $\alpha(1) =q$. If $\alpha$ ever leaves $A$, then it must intersect $\partial ^+ A$, a contradiction. Thus $q \in A$, and we conclude that $A$ is a future set. Reasoning time-dually, we establish that $A$ is a past set when $\partial ^- A = \emptyset$.
%
%\smallskip
%
%\textit{(ii)} We give the proof for the future boundary only, since the past case follows by time-duality. Suppose $\partial ^{+} A$ is not achronal, and let $p, q \in \partial ^{+} A$ such that $p\ll_g q$. Thus, we can pick open sets $U,V \subset I^+(A)$ such that $p \in U$, $q \in V$ and
%\[
%p' \in U, q' \in V \Longrightarrow p' \ll_g q'.
%\]
%Since $p,q \in \partial A$ we can choose $p' \in U \cap (M \setminus A)$ and $q' \in V \cap A$. We can also choose $p'' \in A \cap I^-(p')$. But then $p' \in I^+(p'') \cap I^-(q')\cap (M \setminus A)$, and therefore $A$ cannot be causally convex, contrary to our assumption.
%
%To show that $\partial ^{+} A$ is a $C^0$ hypersurface, since it is achronal we only need to show that $\partial ^{+}A \cap edge(\partial ^+ A) = \emptyset$. Let $p \in \partial ^{+}A$. Then, in particular $p \in I^+(q)$ for some $q \in A$. Pick any future-directed timelike curve $\gamma: [0,1] \rightarrow M$ such that $\gamma(0) \in I^+(q)\cap I^-(p)$ and $\gamma(1) \in I^+(p)$. In particular, $\gamma[0,1] \subset I^+(A)$. The achronality of $\partial ^{+} A$ implies that $\gamma(1) \notin A$. We claim that $\gamma(0) \in A$. If not, noting that $I^+(\gamma(0)) \cap A \neq \emptyset$ because $p \in I^+(\gamma(0)) \cap \partial A$, we could then juxtapose a future-directed timelike curve from $q \in A$ to $\gamma(0)$ with another future-directed timelike curve from $\gamma(0)$ to some point in $A$, thereby obtaining a future-directed timelike curve with endpoints in $A$ but not entirely containing therein. This in turn would contradict the fact that $A$ is causally convex. We conclude that $\gamma$ intersects $\partial ^{+}A$, and therefore $p \notin edge(\partial ^{+} A)$.
%
%Now assume $A$ is open and $A \cup \partial ^{+} A$ is not causally convex. Consider (say) a future-directed causal curve segment $\alpha:[0,1] \rightarrow M$ with $\alpha(1),\alpha(0) \in A\cup \partial ^{+} A$ and $\alpha(t_0) \neq A\cup \partial ^{+} A$ for some $t_0 \in (0,1)$. Since $A$ is open, we necessarily have $\alpha(0),\alpha(1) \in I^+(A)$. So we can pick $p \in A$ such that $p \ll_g \alpha(0)$. Thus $q \ll_g \alpha(t_0)$. But any future-directed timelike curve from $p$ to $\alpha(t_0)$ must intersect $\partial ^{+} A$, and hence $p \ll_g r \ll_g \alpha(t_0)$ for some $r \in \partial ^+A$. Now, either $\alpha(1) \in \partial ^+ A$ or else $\alpha(1) \in A$, in which case $\alpha|_{(t_0,1)}$ must intersect $\partial ^+ A$ anyway. In any case, $\alpha(t_0) < q$ for some $ q \in \partial ^+ A$, and then $r \ll_g q$, contradicting the achronality of $\partial ^+ A$.
%
%\textit{(iii)} If $A$ is a past (resp. future) set, then $\partial A \equiv \partial ^+A$ (resp. $\partial A \equiv \partial ^-A$), and $\partial A$ is well-known to be achronal by standard results in causal theory. Since $A$ is open, the fact that $A$ is causally convex now follows from $(ii)$.
%\end{proof}
%
%
%\begin{proposition}
%\label{convexitygh1}
%Assume that $(M,g)$ is globally hyperbolic, and let $U \subset M$ be an open, connected, globally hyperbolic subset. Then, the following statements are equivalent.
%\begin{enumerate}[label=(\roman*)]
%\item $U$ is causally convex.
%\item Any achronal set in $(U,g|_U)$ is achronal in $(M,g)$.
%\item Any Cauchy hypersurface $S$ in $(U,g|_U)$ is achronal in $(M,g)$.
%\end{enumerate}
%\end{proposition}
%\begin{proof}
%$(i) \Longrightarrow (ii) \Longrightarrow (iii)$ are immediate. Therefore, we only need to prove $(iii) \Longrightarrow (i)$. Let $S \subset U$ be a Cauchy hypersurface in $(U,g|_U)$. Let $\alpha:[0,1] \rightarrow M$ be a future-directed causal curve with endpoints in $U$. Since $U$ is open, there will be no loss of generality in our argument if we assume that $\alpha(0),\alpha(1) \in I^+(S,U)$ and that $\alpha$ is timelike. Suppose $\alpha(t_0) \notin U$ for some $t_0 \in (0,1)$. Also, write
%\begin{eqnarray}
%s_0 &:=& \inf \{ s \in [0,1] \, : \, \alpha[s,1] \subset U \}, \\
%r_0 &:=& \sup \{ t \in [0,1] \, : \, \alpha[0,t] \subset U \}.
%\end{eqnarray}
%Then the restrictions $\alpha_+:= \alpha|_{[0,r_0)}$ and $\alpha_{-}:= \alpha|_{(s_0,1]}$ are future- and past-inextendible in $(U,g|_U)$, respectively, and $r_0 < t_0 < s_0$. But then $\alpha_{-}$ must intersect $S$ at some $s_0< s'$, say, so
%\[
%p \ll_g \alpha(0) \ll_g \alpha(s')
%\]
%for some $p \in S$, which violates the achronality of $S$ in $(M,g)$. Therefore $\alpha[0,1] \subset U$.
%\end{proof}
%
%\begin{theorem}
%\label{convexitygh2}
%Assume that $(M,g)$ is globally hyperbolic, and let $U \subset M$ be an (non-empty) open, connected, causally convex, future-precompact and globally hyperbolic set. Then $\partial ^+ U$ is homeomorphic to a Cauchy hypersurface $S \subset U$ in $(U,g|_U)$.
%\end{theorem}
%
%\begin{proof}
%$S$ is a connected $C^0$ hypersurface in $M$. By Proposition \ref{convexitygh1}, it is achronal in $(M,g)$. Let $K \subset M$ be some compact set such that $U \subset I^-(K)$. In particular, $U$ cannot be a future set, and hence $\partial ^+ U$ is also a non-empty achronal $C^0$ hypersurface by Proposition \ref{convexitypastfuture} (i)-(ii). Let $X:M \rightarrow TM$ be a complete future-directed timelike smooth vector field in $(M,g)$, and denote its flow by $\phi$. We shall define a continuous mapping $\varphi : S \rightarrow \partial ^+ U$ as follows. Given $p \in S$, the future-inextendible timelike curve
%\[
%t \in [0, +\infty) \mapsto \phi_t(p)
%\]
%starting at $p$ must leave the compact set $J^+(p)\cap J^-(K)$, and hence must leave $U$. When its does, it will intersect $ \partial ^+ U$, and since this set is achronal, will do so at a unique point, which we define as $\varphi(p)$. The mapping thus defined is obviously one-to-one, and hence open by Invariance of Domain. Therefore, in other to show that $\varphi$ is the desired homeomorphism we only need to show that it is onto. Accordingly, let $q \in \partial ^+ U$, and consider the inextendible timelike curve
%\[
%\gamma : s \in (-\infty, +\infty) \mapsto \phi_s(q).
%\]
%Then, for some $\varepsilon >0$, $\gamma[-\varepsilon,0] \subset I^+(U)$.
%
%We claim that $\gamma(-\varepsilon) \in U$. If not, noting that $I^+(\gamma(-\varepsilon)) \cap U \neq \emptyset$ because $q \in I^+(\gamma(-\varepsilon)) \cap \partial U$, we could then juxtapose a future-directed timelike curve from some $r \in U$ to $\gamma(-\varepsilon)$ with another future-directed timelike curve from $\gamma(\varepsilon)$ to some point in $U$, thereby obtaining a future-directed timelike curve with endpoints in $U$ but not entirely containing therein. This in turn would contradict the fact that $U$ is causally convex. We conclude that $\gamma$ intersects $U$, and hence must intersect $S$. If this intersection occurred at the future of $q$ this would violate the achronality of $\partial ^+ U$, since we saw above that the future of any point in $S$ intersects $\partial ^+ U$. Therefore, for some $s_0 < 0$, $\phi_{s_0}(q) \in S$. But then we of course have $\varphi (\phi_{s_0}(q)) \equiv q$, thus concluding the proof.
%\end{proof}
%
%Both the causal convexity and the future/past-precompactness will be sufficient conditions to ensure the regular accessibility of the conformal boundary. In this sense, we introduce the following definition, adapted  from Ref. \cite{Olaf}\footnote{Jony: Referencia de Olaf}).
%
%\begin{definition}
%\label{nested}
%A conformal extension $(\tilde{M},\tilde{g})$ of a globally hyperbolic spacetime $(M,g)$ is {\em future-nesting} if
%\begin{itemize}
%\item[N1)] $(\tilde{M},\tilde{g})$ is also globally hyperbolic,
%\item[N2)] $M \subset \tilde{M}$ is causally convex and future-precompact.
%\end{itemize}
%An analogous definition follows for {\em past-nesting}.
%\end{definition}
%\cambiosn{Proposition \ref{convexitypastfuture} and Theorem \ref{convexitygh2} immediately give the following regularity result.
%\begin{corollary}
%\label{convexitygh3}
%If $(\tilde{M},\tilde{g})$ is a future-nesting conformal extension of a globally hyperbolic spacetime $(M,g)$, then the corresponding future conformal boundary $\partial ^+M$ is an achronal $C^0$ hypersurface in $(\tilde{M},\tilde{g})$ homeomorphic to a Cauchy hypersurface in $(M,g)$.
%\end{corollary}}
%
%
%\begin{remark}\label{rmk2}
%{\em The standard conformal extensions in the Einstein cylinder of Minkowski spacetime, past-incomplete Friedmann-Robertson-Walker models with a non-negative cosmological constant, and de Sitter spacetime are all future-nesting in this sense. The standard conformal extension of the Schwarzschild-Kruskal plane is of course also future-nesting.}
%\end{remark}
%
%Before giving our main result for future and past-nesting conformal extensions, we will need the following technical lemma.
%
%\begin{lemma}
%\label{technical}
%Let $(\tilde{M},\tilde{g})$ be a future-nesting conformal extension of a globally hyperbolic spacetime $(M,g)$. Given $p \in \partial ^+M$, $q \in I^-(p, \tilde{M})$ and any future-directed timelike curve $\alpha:[0,1] \rightarrow \tilde{M}$ from $q$ to $p$, there exists some $0<\varepsilon <1$ for which $\alpha[1-\varepsilon,1) \subset M$.
%\end{lemma}
%\begin{proof}
%Since $p \in I^+(M,\tilde{M})$ and the latter set is open, for some $0< \varepsilon <1$, $\alpha[1- \varepsilon,1] \subset I^+(M,\tilde{M})$ by continuity. Let $t_0 \in [1-\varepsilon,1)$. Since $p \in I^+(\alpha(t_0),\tilde{M})$, $I^+(\alpha(t_0))\cap M \neq \emptyset$ since $p$ is a boundary point. Choose a future-inextendible timelike curve $\beta:[0,a) \rightarrow \tilde{M}$ starting at $\alpha(t_0)$ and intersecting $M$. Pick a compact set $K \subset \tilde{M}$ for which $M \subset J^-(K,\tilde{M})$. The global hyperbolicity of $(\tilde{M},\tilde{g})$ forces $\beta$ to leave the compact set $J^+(\alpha(t_0),\tilde{M})\cap J^-(K, \tilde{M})$, and hence leaves $M$. Thus, if $\alpha(t_0) \notin M$, we could choose a timelike curve segment in $\tilde{M}$ with endpoints in $M$ which leaves $M$, violating the causal convexity of $M$. Therefore, $\alpha(t_0) \in M$, and we then conclude that $\alpha[1-\varepsilon,1) \subset M$.
%\end{proof}
%
%\begin{theorem}
%\label{thm:futurenestingprincipal}
%Let $(\tilde{M},\tilde{g})$ be a future and past-nesting conformal extension of a globally hyperbolic spacetime $(M,g)$. Then
%\begin{enumerate}[label=(\roman*)]
%
%\item $(\tilde{M},\tilde{g})$ is a chronologically complete conformal extension. Indeed, every inextendible causal curve in $(M,g)$ has both a future and a past endpoint on $\partial_{i}^{*}M$.
%
%\item  $\partial_{i}^{*}M$ is regularly accessible, and hence the conformal and causal completions coincide.
%
%
%\item The future (resp. past) conformal boundary $\partial^+ M$ is an achronal $C^0$ hypersurface in $(\tilde{M},\tilde{g})$ homeomorphic to a Cauchy hypersurface in $(M,g)$.
%\end{enumerate}
% \end{theorem}
% \begin{proof}
%  In order to prove \textit{(i)}, let $\gamma$ be an inextendible future-directed causal curve in $(M,g)$. We shall deal with future endpoints, since the past-endpoint case proceeds analogously. The curve $\tilde{\gamma}=i(\gamma)$ is also future-directed and causal in $(\tilde{M}, \tilde{g})$, and so it defines an IP. Moreover, by our nesting assumptions, $\tilde{\gamma}$ is contained in the compact set $J^{+}(\tilde{\gamma}(0),\tilde{M})\cap J^-(K,\tilde{M})\subset \tilde{M}$ for some compact subset $K\subset \tilde{M}$, and hence it is actually a PIP on $\tilde{M}$. In particular, there exists $p\in \tilde{M}$ such that $I^-(\tilde{\gamma},\tilde{M})=I^-(p,\tilde{M})$, where $p$ is a future endpoint of $\tilde{\gamma}$. As $\tilde{\gamma}\subset i(M)$, it follows that $p\in \partial_{i}^{*} M$, i.e., $\tilde{\gamma}$ has an endpoint in $\overline{i(M)}$, as desired.
%
%   \smallskip
%
%Finally, observe that the regular accessibility of the boundary points described in \textit{(ii)} is a direct consequence of Lemma \ref{technical}, while the structural properties in \textit{(iii)} follow from Theorem \ref{convexitygh2}. \end{proof}
%


%%% Local Variables:
%%% mode: latex
%%% TeX-master: "nullinfinityV5"
%%% End:
