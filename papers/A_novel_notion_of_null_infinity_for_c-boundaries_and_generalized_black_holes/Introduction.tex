\section{Introduction}

Since its introduction by Geroch, Kronheimer and Penrose in 1972 \cite{GerochIdealPointsSpaceTime1972}, the {\em causal boundary} (or {\em $c$-boundary}) of a given strongly causal spacetime $(M,g)$ has asserted itself as a method for attaching ideal points to $M$ which, although somewhat more abstract than conformal boundaries, is (unlike the latter) deeply ingrained in the very causal structure of $(M,g)$. In particular, this means that the $c$-boundary is, in a strong sense, an inescapable feature of the (conformally invariant aspects of the) spacetime geometry.

By contrast, from a purely geometric perspective, the existence of a conformal boundary seems to impose an {\em ad hoc} restriction on $(M,g)$: it relies on the existence of a suitable open conformal embedding into a larger spacetime $(\tilde{M},\tilde{g})$, inducing a piecewise $C^1$ (often required to be $C^{\infty}$) boundary $\partial M$ with fairly restrictive properties first introduced by Roger Penrose in 1963 as a geometric description of the asymptotic properties of the gravitational field \cite{PenroseAsymptoticStructure1963,PenroseConformalInfinity1964} (see also, e.g., \cite{HawkingLargeScaleStructure1975,WaldGeneralRelativity1984,Frauendiener2004} and their references for general accounts and further technical details and \cite{ChruscielConformalboundaryextensions2010} for a analysis of the uniqueness and some criteria for existence of conformal boundaries). It is unclear to us how generic are such properties, and indeed the very existence, of conformal boundaries.

To be sure, it is well-known that among solutions of the Einstein field equations with various signs of the cosmological constant, the seminal work of Friedrich \cite{Friedrich1,Friedrich2,Friedrich3} shows that the existence of a conformal {\em future null infinity} ${\cal J}^+$ in the sense of Penrose is a {\em stable} feature under (non-linear) perturbations for a large class of initial data for the Einstein field equations, either vacuum or suitable matter content \cite{Friedrich2,ChruschielDelayExistenceAsymptoticallySimple2002}. However, even in simple but key examples such as plane waves, which also have great physical interest, the conformal boundary may either not exist altogether, or else have counterintuitive properties \cite{Florescausalboundarywavetype2008}. Although such solutions are indeed fairly special and probably too idealized from a physical standpoint, to the best of our knowledge the absence of a sensible conformal boundary they feature might well turn out to be a stable feature by itself (see, for example, \cite{Friedrich3} and references therein for an extensive discussion of this difficult issue).

Be as it may, it cannot be gainsaid that conformal boundaries are extremely convenient in physical applications. Whenever available, they present an exceedingly elegant and well-motivated means of addressing a number of key physical notions. One of the most beautiful applications of conformal boundaries in physics is in the classical theory of black holes. These are often informally described as spacetime regions from which causal communication with ``external observers'' is barred. Now, since the precise mathematical surrogate of ``causal communication'' is fairly clear in Lorentzian geometry, any purported rigorous definition of black holes is tantamount to giving a mathematically accurate and physically useful model of the term ``external observer''.

In standard treatments, these ``observers'' have been equated with the points on the future null infinity ${\cal J}^+$ in the sense of Penrose indicated above. (Indeed, the future part of conformal boundary is often taken to consist only of such points.) However, as also pointed out before, the properties of the conformal null infinity which are useful in mathematical Relativity can be rather specialized, abstracted as they are from very specific spacetimes which are solutions of the Einstein field equation.

Therefore, it can be of great interest to take advantage of the universality and geometric naturalness of the causal boundary construction, while making it as amenable as possible to concrete applications. The original causal boundary construction \cite{GerochIdealPointsSpaceTime1972} presented a number of undesirable properties even in the simplest cases. Fortunately, the main glitches in the original proposal have been addressed and solved by a suitable redefinition of the $c$-boundary, initially by Szabados \cite{SzabadosCausalboundarystrongly1988,SzabadosCausalboundarystrongly1989} and Marolf and Ross \cite{Marolfnewrecipecausal2003}, and further streamlined in a quite satisfactory way by Flores in \cite{Florescausalboundaryspacetimes2007}, and by Flores, Herrera and S\'{a}nchez in \cite{Floresfinaldefinitioncausal2011} (see also the references of the latter paper and in \cite{Garcia-ParradoCausalstructurescausal2005} for a detailed discussion of previous/alternative proposals by other authors).

Our purpose in this paper is to show how one can use the $c$-boundary construction to give a completely general definition of future null infinity ${\cal J}^+$ which is purely geometric, and valid for any strongly causal spacetime $(M^{n+1},g)$ with $n \geq 2$. Since the definition we give relies only on the $c$-boundary, one does not need to assume any field equations, specific dimensions or special asymptotic properties of $(M,g)$. %\cambiosn{
Still, our approach here strongly suggests that, at least in many concrete physical situations in which the conformal null infinity plays a role, the latter will coincide with the $c$-boundary version. If so, we do not loose any of the good properties of the conformal boundaries (see \cite{CostaFloresHerrera2018-2}).
%}

We also use our notion of null infinity to define the concomitant notion of a black hole in this general context. It is a matter of no little surprise that some non-trivial properties of black holes can indeed be extended to this general setting. A simple but important example we discuss here is a classic result in black hole theory (see, e.g., Prop. 12.2.2 of \cite{WaldGeneralRelativity1984}: under physically reasonable geometric conditions on the spacetime, any closed trapped surface remains inside the black hole region thereof. Other minor structural features characterizing black holes are also discussed here and shown to apply even in this very general situation.

It should be emphasized that these extended notions are not meant to provide {\em substitutes} for the standard, conformal notions which are so useful in physical applications, but rather {\em complements} to them. Finer aspects of physical interest, such as the detailed conditions characterizing (null) asymptotic flatness, or key results like the so-called ``peeling property'' of the Weyl tensor in gravitational wave theory \cite{SachsPeeling1961} are best described in terms of conformal boundaries, precisely because the definition of the latter structure has been fine-tuned for a long time to achieve this. But an array of extra assumptions have to be added, albeit physically tested and well-motivated. Indeed, the merest glance at the definition of asymptotic flatness in Wald's book \cite{WaldGeneralRelativity1984}, for example, reveals the number of assumptions one has to make for the notion of infinity to work at its fullest power.

The point here is that even in much less stringent situations we may still hope for usefully doing as much with less. The results about black holes proved here offer a case in point, but we also provide examples (in section \ref{ppwaves}) which are of considerable physical interest (indeed, a class of the so-called {\em generalized plane waves}) but where conformal notions are out of the question, since a conformal boundary (with good properties) simply may not exist!

The generality we envisage here, however, comes at a price. In order to obtain good results from our notions of null infinity and black holes, a number of {\em seemingly} rather technical assumptions and definitions are needed along the way. This is because a number of annoying counterexamples may be found for most of the otherwise intuitive statements which hold in the conformal case. We have endeavoured to present these extra hypotheses clearly, and while it may not at all be obvious to the reader, the specific assumptions we adopt are actually very mild and hold for all but some rather artificial examples. However, including a detailed discussion here would enormously complicate the paper, so we analyze them in depth elsewhere \cite{CostaFloresHerrera2018-2}. In particular it will be shown on that juncture that most assumptions in this paper are automatically satisfied for a large class of spacetimes where an interesting conformal boundary exists.

The rest of this paper is organized as follows.

In section \ref{prelim}, we introduce the notation and briefly review some of the basic definitions and results about the $c$-boundary of a strongly causal spacetime which will be used in the later parts of the paper.

% In sections \ref{top1} and \ref{top2} we show how the CLT can be introduced on the $c$-boundaries, and prove that it retains all the main properties of the chronological topology.

% In \ref{compare} we discuss in more detail the relationship of the CLT with the chronological topology. In particular, we prove that the CLT coincides with the chronological topology if and only if the latter is Hausdorff, and give a class of physically relevant examples where this occurs.

 In section \ref{conf}, we briefly review the results in \cite{Floresfinaldefinitioncausal2011} about the relation between the c-boundaries and the conformal ones (when the latter are suitably defined). %\cambiosn{
 %In particular, we show that when the $c$-boundary is endowed with a suitable topology and causal structure, then it is homeomorphic to the part of the conformal boundary which is accessible by future-directed timelike curves.
 %}

%A similar result has also been obtained for the chronological topology in \cite{Final} and \cite{olaf}. The approach we adopt in this section has been inspired by the one in \cite{olaf}: we essentially borrow from its author the notions of a {\em future-precompact} subset and {\em future-nesting} conformal extensions we use in section \ref{conf}, although he uses a somewhat different terminology to describe these\footnote{Indeed, in Ref. \cite{olaf} a Hausdorff topology was also introduced on the $c$-boundary following a suggestion in \cite{GKP}. However, that topology is introduced in a fairly technical manner, and its relation to the CLT and its naturalness remain unclear to us.}

 In section \ref{bh}, a definition is given of null infinity for $c$-boundaries of strongly causal spacetimes. This definition is directly inspired by the classical notion of null infinity given in terms of the conformal boundary, plus the needed extra technicalities. We also discuss therein some of its general properties.
 %which includes the points at infinity arising in the standard (conformal) definitions whenever a conformal boundary (again suitably defined) is present. We also discuss therein some of its properties.

This extended notion of null infinity is then used in section \ref{bh2} to introduce a general notion of black hole, and we explore some of its more elementary properties. It is here that, under very mild assumptions %\cambiosn{
(taken for granted in the usual, conformal setting \cite{CostaFloresHerrera2018-2}) which {\em do not} include any notion of asymptotic flatness, we are able to prove that any closed trapped surface in $(M,g)$ has to be inside the black hole region (Corollary \ref{trappedcor}), which as mentioned before is a key result in (standard) black hole theory.

In section \ref{ppwaves} the previous results are applied to give a description of null infinity for a class of generalized plane waves, which may {\em not} admit conformal boundaries, and it is shown that these spacetimes do not admit black holes (see \cite{Florescausalboundarywavetype2008,HubenyRagamaniNoHorizons,SenovillaNoBHinPPWaves2003} for related results).In Ref. \cite{SenovillaNoBHinPPWaves2003}, especially, the author shows that under certain mild additional geometric restrictions on generalized wave-type spacetimes, closed trapped surfaces do not exist therein\footnote{We thank Jos\'{e} M. M. Senovilla for bringing this reference to our attention.}. That (purely quasi-local) result is interestingly complementary to our own results here: in the context of Corollary \ref{trappedcor} we too may conclude, from a different perspective, that insofar as any trapped surfaces would be inside a black hole region, they do not exist in the wave-type spacetimes considered here.

Finally, in an appendix we discuss an example which shows that the clauses in Definitions \ref{scri} and \ref{ample} are not only natural when interpreted from the classical conformal approach, but cannot be discarded in favor of less technical-looking ones. 


% Finally, we remark that the results in this paper are all independent of any field equations, and work for all dimensions equal to or greater than 2.

%%% Local Variables:
%%% mode: latex
%%% TeX-master: "CostaFloresHerreraNullInfty"
%%% End:
