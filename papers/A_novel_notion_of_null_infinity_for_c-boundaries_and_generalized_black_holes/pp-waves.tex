\section{Application to generalized plane waves}\label{ppwaves}

We are going to study in this section the null infinity of, and prove the absence of black holes in, the class of {\em generalized plane waves}. This family of spacetimes has been intensely studied in the literature, and is especially relevant for us here because conformal boundaries are not always available therein, but $c$-boundaries do, since they are strongly causal under mild assumptions \cite[Section 3]{0264-9381-20-11-322}. Hence, definitions of null infinity and black holes based on the latter become a natural alternative. In that case, the classical notions of null infinity and/or black holes do not make sense. (For a different perspective on the same issue, see \cite{SenovillaNoBHinPPWaves2003}.) Let us begin by reviewing some generalities about these spacetimes.

A {\em generalized plane wave} is any spacetime $(M^{n+1},g)$ of the form
\begin{equation}\label{pfw}
M=M_0 \times \mathbb{R}^2,\qquad
g(\cdot,\cdot) = g_0(\cdot,\cdot) + 2dudv + H(x,u)du^2,
\end{equation}
where $g_0$ is a smooth Riemannian metric on the ($n-1$)-dimensional manifold $M_0$, the variables $(v,u)$ are the standard Cartesian coordinates of $\mathbb{R}^2$, and $H:M_0 \times \mathbb{R} \rightarrow \mathbb{R}$ is some smooth real function.

The vector field $\partial_v$ is parallel (i.e.,
covariantly constant) and null, and the time-orientation will
always be chosen which makes it past-directed. In particular, its integral curves are past-directed null geodesics.

Consider any future-directed
causal curve segment $t \in [a,b] \mapsto \gamma(s)=(x(s),v(s),u(s))$ in $(M,g)$. Since the gradient $\nabla \,u \equiv \partial_v$ is past-directed null, we have
\begin{equation}
\label{cute}
\dot{(u\circ \gamma)}(s) = g(\nabla \, u (\gamma(s)), \dot \gamma(s)) = g(\partial_v|_{\gamma(s)},\dot \gamma(s)) = \dot u(s) \geq 0,
\end{equation}
and the inequality is strict whenever $\dot \gamma(s)$ is timelike. Integrating Eq. (\ref{cute}), we get
\[
u(b) - u(a)\geq 0,
\]
with strict inequality unless $\gamma$ is a null pregeodesic without conjugate points contained in a $u\equiv\hbox{constant}$ hypersurface. If the latter is not the case, then $\gamma$ leaves any such hypersurface where it starts, and in particular $\gamma$ cannot be a closed curve.

This calculation reveals, in particular, that each $u\equiv\hbox{constant}$ hypersurface is null and achronal and $(M,g)$ does not admits any closed timelike curves, i.e., it is chronological. The null geodesic generators of any such hypersurface coincide with the maximal integral curves of $\partial_{v}$ therein, and achronality implies that any null geodesic inside these hypersurfaces coincide with such generators.

Finally, a closer analysis of the geodesic equations easily shows that any null geodesic inside a $u\equiv\hbox{constant}$ hypersurface is {\em injective} and {\em complete}. We conclude that {\em every generalized plane wave is causal}; moreover, the vector $\partial_v$ is complete and all its maximal integral curves are null geodesic lines.


Fixing some local coordinates $x_1, \dots, x_{n-1}$ for the Riemannian
part $M_0$,
%it is straightforward to compute the
%Christoffel symbols of $g$, and
%
%thus,
%to relate the Levi-Civita connections $\nabla, \nabla^0$ for $M$
%and $M_0$, respectively (see \cite{CFSgrg}). We remark the following
%facts:
%\begin{itemize}
%\item $M_0$ is totally geodesic, i.e., $\nabla_{\partial i}
%\partial_j = \nabla^0_{\partial_{i}} \partial_j $, $i,j=1, \dots, n-1$.
%\item The non-zero curvature coefficients are:
%\[
%R^i_{jkl} = (R_{g_0})^i_{jkl}, \quad R^i_{uuk} = \frac{1}{2}(g_0)^{il}(Hess_0H)_{lk}, \quad R^v_{iuj} = -\frac{1}{2}(Hess_0H)_{ij},
%\]
%where we have used $i,j,k,l, \ldots$ for the spatial coordinates, and $Hess_0H$ denotes the Hessian of $H = H( . , u)$ with respect to the metric $g_0$.
%
%\item The Ricci tensor
%$Ric$ of $(M,g)$ and ${\rm
%Ric}^{0}$ of $(M_0,g_0)$ satisfy
%\[
%Ric = \sum_{i,j=1}^{n-1} R^{0}_{i j} d x_i \otimes d x_j
%-\frac{1}{2}\Delta_{x}H du \otimes du .
%\]
%Thus, $Ric$ is zero if and only if both the Riemannian
%Ricci tensor Ric$^{0}$ and the transverse Laplacian $\Delta_{0}H$
%vanish.
%\end{itemize}
%
%From the direct computation of Christoffel symbols of a generalized plane wave, it is straightforward
%
%to write the geodesic equations
%in local coordinates. Remarkably,
the three geodesic equations for
a curve $\gamma(s)= (x(s), v(s), u(s))$, $s\in (a,b)$, can be solved in
the following three steps  \cite[Proposition 3.1]{CandelaGeneralPlaneFronted2003}:
\begin{enumerate}
\item[(a)] $u(s)$ is any affine function, $u(s) = u_0 + s \Delta
u$, for some constant $\Delta u\in {\mathbb R}$;

\item[(b)] $x = x(s)$ is a solution on $M_0$ of
\[
D_s\dot x = - {\rm grad}_x V_{\Delta}(x(s),s) \quad \mbox{for all
$s \in \ (a,b)$,}
\]
where $D_s$ denotes the covariant derivative and $V_{\Delta}$ is
defined as:
\[
V_{\Delta}(x,s) := -\ \frac{(\Delta u)^2}{2}\ H(x, u_0 + s \Delta
u);
\]

\item[(c)] finally, with a fixed $v_{0}$ and an $s_0\in (a,b)$,
$v(s)$ can be computed from
\[
v(s) = v_0 + \frac{1}{2 \Delta u} \int_{s_0}^s \left( E_{\gamma} -
g_0(\dot x(\sigma), \dot x(\sigma)) + 2
V_{\Delta}(x(\sigma), \sigma)\right) d\sigma,
\]
where $E_{\gamma}=g(\dot{\gamma}(s), \dot{\gamma}(s))$ is a
constant (if $\Delta u = 0$ then $v = v(s)$ is affine).
\end{enumerate}
In particular, if we fix some $(\overline{x},\overline{u})\in M_0\times {\mathbb R}$, then the curve
\[
\gamma_{\overline{x},\overline{u}}(s)=(\overline{x},-s,\overline{u})\quad\hbox{(resp. $\beta_{\overline{x},\overline{u}}(s)=(\overline{x},s,\overline{u}))$,}\quad s\in {\mathbb R},
\]
is a future-directed (resp. past-directed) null geodesic line.

The $c$-boundary for generalized plane waves has been systematically studied in \cite{Florescausalboundarywavetype2008}, where it was shown that its structure strongly depends on the growth of $H$ at infinity. Some of these asymptotic behaviours, which will be also relevant in our discussion, are the following:
\begin{definition} A function ${\cal F}:M_0\times{\mathbb R}\rightarrow {\mathbb R}$ is said to be:
\begin{itemize}
%\item[(i)] {\em superquadratic} if $M_0$ is unbounded and contains a sequence of points $\{x_m\}_m\in  M_0$ and $\hat{x}\in M_0$ such that
%$d(x_m,\hat{x})\rightarrow\infty$ and
%\[
%R_1 d(x_m,\hat{x})^{2+\epsilon}+R_0\leq F(x_m,u)\quad\forall u\in {\mathbb R},
%\]
%for some $\epsilon, R_1, R_0\in {\mathbb R}$ with $\epsilon, R_1 > 0$.
%
\item[(1)] {\em at most quadratic}, if there exist $\hat{x}\in M_0$ and positive continuous functions $R_0(u), R_1(u) > 0$ such that
\[
{\cal F}(x,u)\leq R_1(u)d(x,\hat{x})^2 + R_0(u),\quad\forall (x,u)\in M_0\times {\mathbb R};
\]
\item[(2)] {\em $\lambda$-asymptotically quadratic, with} $\lambda>0$, if $M_0$ is non-compact and there
exist $\hat{x}\in M_0$, continuous functions $R_0(u), R_1(u)>0$ and a constant $R_0^-\in {\mathbb R}$ such that:
\[
\frac{\lambda^2d(x,\hat{x})^2+R_0^-}{u^2+1}\leq {\cal F}(x,u)\leq R_1(u)d(x,\hat{x})^2 + R_0(u),\quad\forall (x,u)\in M_0\times {\mathbb R}.
\]
%\item[(ii)] {\em subquadratic} if there exist $\hat{x}\in M_0$ and continuous functions $R_0(u), R_1(u)(\geq 0)$,
%$p(u)<2$ such that:
%\[
%F(x,u)\leq R_1(u)d^{p(u)}(x, \hat{x}) + R_0(u),\quad\forall (x,u)\in M_0\times {\mathbb R}.
%\]
\end{itemize}
\end{definition}
Let us recall now some noteworthy statements about the structure of the $c$-boundary for generalized plane waves (see \cite[Theorems 7.9 and 8.2]{Florescausalboundarywavetype2008}; note that $F$ in that reference corresponds with $-H$ here). From now on, we will assume that $(M_0,g_0)$ {\em is geodesically complete}:
\begin{theorem}\label{t} Let $(M,g)$ be a generalized plane wave as in (\ref{pfw}). Then, the following assertions hold.
\begin{itemize}
%\item[(i)] If $-H$ is superquadratic and $H$ is at most quadratic then $(M,g)$ is neither future nor past-distinguishing. In particular, it does not admit a c-boundary.
\item[(i)] If $|H|$ is at most quadratic, then the future (resp. past) $c$-boundary of $(M,g)$ {\em contains} a
%locally lightlike\footnote{Ivan: Eso que es??? No creo que lo hayan definido....}
copy $L^+$ (resp. $L^-$) of ${\mathbb R}$ plus the ideal point $i^+$ (resp. $i^-$)\footnote{Here, by $i^{+}$ and $i^{-}$ we are denoting the pairs $(M,\emptyset)$ and $(\emptyset,M)$, and the entire manifold $M$ is a terminal set.}. Every ideal point $\overline{u}\in L^{+}$ (resp. $\overline{u}'\in L^{-}$) can be identified with the IP $I^{-}(\gamma_{\overline{x},\overline{u}})$ for any $\overline{x}\in M_0$ (resp. the IF $I^{+}(\gamma_{\overline{x}',\overline{u}'})$, for any $\overline{x}'\in M_0$).

    The (total) $c$-boundary of $(M,g)$ {\em contains} a subset which can be identified with the quotient space $$((L^{+}\cup \{i^{+}\})\cup (L^{-}\cup \{i^{-}\}))/R,$$ where $R$ is the equivalence relation obtained by symmetrizing the following relation:\footnote{From \cite[Remark 7.10]{Florescausalboundarywavetype2008}, a pair $(\overline{x},\overline{u})$ cannot be related with more than one pair $(\overline{x}',\overline{u}')$, and viceversa.}
    \[
    (\overline{x},\overline{u})R(\overline{x}',\overline{u}')\quad\Leftrightarrow\quad (I^{-}(\gamma_{\overline{x},\overline{u}}),I^{+}(\gamma_{\overline{x}',\overline{u}'}))\in\overline{M}.
    \]
    %with eventual identifications between their points originated by the eventual formation of pairs\footnote{???????????????????} of the form $(I^{-}(\gamma_{\overline{x},\overline{v}}),I^{+}(\gamma_{\overline{x}',\overline{v}'}))$, for some $\overline{v}<\overline{v}'$.

\item[(ii)] If $-H$ is $\lambda$-asymptotically quadratic for some
$\lambda>1/2$, then the future, past and total $c$-boundaries not only contains the structures described in (i), but necessarily {\em coincide} with them.

%\item[(iii)] In the limit case $\lambda=1/2$ there exists an explicit example with higher dimensional boundary, showing that the $1$-dimensional c-boundary can no longer be expected for $\lambda\leq 1/2$.

%\item[(ii)] If $-H$ subquadratic and $(M_0,g_0)$ is complete, then $(M,g)$ is globally hyperbolic. In this case, the structure of the spacetime suggests a c-boundary with two pieces which resemble in some sense the Cauchy hypersurfaces.
    %--notice that the Cauchy hypersurfaces are necessarily noncompact and, at least when $M$ is noncompact, one could expect that some portion of boundary were higher dimensional, even of dimension
%$(n + 1)$.

%\item[(v)] For plane waves with $-H$ having negative eigenvalues the spacetime is conformal to a region of
%${\mathbb L}^{n+2}$ bounded by two lightlike hyperplanes. Nevertheless, the causal and conformal boundaries differ in this case: the former has two connected
%pieces (a future boundary and a past one); the latter, which is necessarily compact, is connected and includes implicitly properties at spacelike infinity.
\end{itemize}
\end{theorem}

%\begin{remark} {\em In previous theorem, the set $L^{+}\cup \{i^{+}\}$ corresponds with (part of) the future c-boundary of $(M,g)$, and every ideal point $\overline{v}\in L^{+}$ is associated to the IP $P=I^{-}(\gamma_{\overline{x},\overline{v}})$, for any $\overline{x}\in M_0$. Analogously, the set $L^{-}\cup \{i^-\}$ corresponds with (part of) the past c-boundary, and every ideal point $\overline{v}'\in L^{-}$ is associated to the IF $F=I^{+}(\gamma_{\overline{x}',\overline{v}'})$, for any $\overline{x}'\in M_0$. The identifications cited above correspond to the eventual formation of pairs of the form $(I^{-}(\gamma_{\overline{x},\overline{v}}),I^{+}(\gamma_{\overline{x}',\overline{v}'}))$, for some $\overline{v}<\overline{v}'$.}
%\end{remark}


Now, we are ready to obtain the (future) null infinity for these spacetimes according to Definition \ref{scri}. To simplify the study, we will restrict our attention to {\em geodesically complete} generalized plane waves. This property is guaranteed, for instance, if $H(x,u)\equiv H(x)$ is at most quadratic (see \cite[Corollary 3.4]{CandelaGeneralPlaneFronted2003}), but there are pretty more situations were it holds. We will also assume that the null rays $\gamma_{\overline{x},\overline{u}}$ are future-regular.
%\footnote{\cambiosn{En realidad, solo necesitamos que se verifique la propiedad $\uparrow \gamma_{\overline{x},\overline{u}}=\uparrow I^{-}(\gamma_{\overline{x},\overline{u}})$. Creo que si nos restringimos a esta propiedad, podemos eliminar la condicion de "properly causal" de todo el paper, puesto que solo la necesitabamos para los resultados que vienen. Esta propiedad se verifica con bastante generalidad, y debe ser facil de comprobar en ejemplos concretos.}}
%\cambiosn{We will also assume that it is strongly properly causal, which again is guaranteed if $H(x,u)\equiv H(x)$.}
\begin{corollary} Let $(M,g)$ be a geodesically complete generalized plane wave whose null rays $\gamma_{\overline{x},\overline{u}}$ are future-regular. Then, the following assertions hold:
\begin{itemize}
\item[(i)] If $|H|$ is at most quadratic, then ${\cal J}^+$ {\em contains} all the pairs of the form $(P,F)$, with $P\neq\emptyset$, which appear in case (i) of Theorem \ref{t}.
%and ${\cal J}^-$ contain copies $L^+$ and $L^-$ of ${\mathbb R}$, resp.
%If $-H$ is $\lambda$-asymptotically quadratic for some $\lambda>1/2$, and $(M_0,g_0)$ is complete, then the null infinity ${\cal J}$ coincides with the c-boundary with the ideal points $i^{+}$, $i^{-}$, removed; that is, ${\cal J}$ is formed by two copies $L^+$, $L^-$ of ${\mathbb R}$, with eventual identifications between the points of the copies. Moreover, in this case ${\cal J}$ is ample.
\item[(ii)] If $-H$ is $\lambda$-asymptotically quadratic for some
$\lambda>1/2$, then ${\cal J}^+$ not only contains, but also {\em coincides} with the structure described in previous case (i).
%If $-H$ subquadratic and $(M_0,g_0)$ is complete, then ${\cal J}$ is the disjoint union of ${\cal J}^+$ and ${\cal J}^{-}$, and ${\cal J}^+$ and ${\cal J}^-$ contain copies $L^+$ and $L^-$ of ${\mathbb R}$, resp.
\end{itemize}
\end{corollary}

\begin{proof}
(i) Recall that $\gamma_{\overline{x},\overline{u}}=(\overline{x},-s,\overline{u})$, is a future-directed null geodesic line for every $(\overline{x},\overline{u})\in M_0\times {\mathbb R}$. Since $\uparrow \gamma_{\overline{x},\overline{u}}=\uparrow I^-(\gamma_{\overline{x},\overline{u}})$, the curve $\gamma_{\overline{x},\overline{u}}$ has a future endpoint of the form $(I^-(\gamma_{\overline{x},\overline{u}}),F)$, where either $F=I^+(\gamma_{\overline{x}',\overline{u}'})$ or $F=\emptyset$.
Moreover, since $(M,g)$ is assumed to be geodesically complete, any other inextendible future-directed null geodesic $\alpha$ with future endpoint $(I^-(\gamma_{\overline{x},\overline{u}}),F)$ is complete. Hence, $(I^{-}(\gamma_{\overline{x},\overline{u}}),F)$ belongs to ${\cal J}^+$ for every $(\overline{x},\overline{u})\in M_0\times {\mathbb R}$.
%For ${\cal J}^{-}$ we reason similarly. The ample character of ${\cal J}$ is direct from the definition.

The argument for the case (ii) is totally analogous.
\end{proof}

\noindent As a direct consequence we can now deduce the absence of BH for these spacetimes.
\begin{corollary}\label{cc}
 If $(M,g)$ is a geodesically complete generalized plane wave whose null rays $\gamma_{\overline{x},\overline{u}}$ are future-regular, then it does not contain black holes. In particular, causally continuous, geodesically complete generalized plane waves have no black holes.
 %the thesis follows for any causally continuous, geodesically complete generalized plane waves.}
\end{corollary}

\begin{proof}
Given an arbitrary event $p_0=(x_0,v_0,u_0)\in M$, it suffices to show that $p_0\in I^-({\cal J}^+)$. From the proof of the previous theorem, $(I^{-}(\gamma_{\overline{x},\overline{u}}),F)\in {\cal J}^+$ for all $(\overline{x},\overline{u})\in M_0\times {\mathbb R}$. Moreover, $p_0=(x_0,v_0,u_0)\in I^-((I^{-}(\gamma_{\overline{x},\overline{u}}),F))$ if and only if $u_0<\overline{u}$. Hence, $p_0\in I^-({\cal J}^+)$ whenever $u_0<\overline{u}$. In conclusion, $M\subset I^-({\cal J}^+)$, and thus, $B^+\subset M\setminus I^-({\cal J}^+)=\emptyset$.

For the last assertion, just recall that any causally continuous spacetime is strongly properly causal (Proposition \ref{prop:causalcontinuity}), and thus, the null rays $\gamma_{\overline{x},\overline{u}}$ are future-regular (Definition \ref{def:sproperlycausal}).
\end{proof}

\begin{remark} {\em In \cite[Thm. 6.9]{EHRLICH_1992} (see also \cite{minguzzi12:causal_kam}) the authors provide mild conditions under which a plane wave is causally continuous. So, according to Corollary \ref{cc}, these conditions joined to the previously cited ones for geodesic completeness ensure that a plane wave has no black holes.
}
\end{remark}



%\cambiosj{As a final comment on this section, we note that there exist (non trivial) causally continuous generalized plane waves. In fact, we recall the following result of Ehrlich and Emch (see \cite[Thm. 6.9]{EHRLICH_1992}; and also \cite{minguzzi12:causal_kam} for further conditions on causal continuity.)}
%
%\cambiosj{\begin{theorem}
%Let $(M,g)$ be a generalized plane wave as in \eqref{pfw} satisfying:
%\begin{enumerate}[label=(\alph*)]
%	\item $M=\mathbb{R}^4$ and $g_0$ is the $2$-dimensional Euclidean metric.
%	\item The function $H$ takes the form \[H(y,z,u)=f(u)(y^2-z^2)+2g(u)yz\] for $f,g\in \mathcal{C}^2(\mathbb{R})$ two real functions satisfying that
%	\[\int_{-\infty}^{\infty}\left(f^2(u)+g^2(u)\right)du<\infty.\]
%	\item $(M,g)$ is a vacuum solution.
%\end{enumerate}
%Then, $(M,g)$ is causally continuous.
% \end{theorem}}



%\cambiosj{OJO: TODO ESTO DEBE DESAPARECER AHORA.
%The properly causal hypothesis can be removed from previous result if, instead, we assume
%%strong causality (ensured if $-H(x,u)$ is at most quadratic; see [Flores, Sanchez, CQG03]),
%the timelike convergence condition (which is equivalent to $K(x)\geq 0$, being $K(x)$ the curvature of $(M_0,g_0)$ at $x$, and $\Delta_x H(x,u)$ for all $(x,u)\in M_0\times {\mathbb R}$; see [Flores, Sanchez, CQG03]), and ${\cal J}^+\neq\emptyset$. In fact, the following result is a direct consequence of Theorem \ref{completeness2}\footnote{\cambiosj{Jony: Cuidado aquí. Entiendo que es aquí donde habrá que imponer algo sobre strongly properly causal...}}:
%\begin{corollary}
%If $(M,g)$ is a geodesically complete generalized plane wave which is strongly causal\footnote{\cambiosn{Aqui la hipotesis strongly causal aparece explicitamente porque asi aparecia en el Theorem \ref{completeness2}. No obstante, se trata de una hipotesis que en realidad se supone en todo momento, ya que estamos trabajando con el borde causal.}} and satisfies the TCC and ${\cal J}^+\neq\emptyset$, then it does not contain black holes.
%\end{corollary}
%}

%\cambiosn{
%\noindent {\em Proof.} From Theorem \ref{completeness2} it suffices to show that ${\cal J}^+\neq\emptyset$, which is ensured by the proof of previous corollary. \qcd
%}

%On the other hand, we know that $\gamma_{\overline{x},\overline{u}}$ is a future-directed null geodesic line for every $(\overline{x},\overline{u})$. Moreover, since $(M,g)$ is assumed to be geodesically complete, any other inextendible future-directed null geodesic $\gamma$ with $I^{-}(\gamma)=I^{-}(\gamma_{x_0,u_0})$ is also complete. Hence, $I^{-}(\gamma_{\overline{x},\overline{u}})$ belons to ${\cal J}^+$, and the proof is over. \qcd



%
%First, observe that condition
%$I^{-}(\gamma)=I^{-}(\gamma_{x_0,u_0})$ implies $u(s)=u(0)+s \Delta u rightarrow u_0$. Assume by contradiction that
%
%
%
%So, if $\Delta u=0$, necessarily $u(s)\equiv u(0)=u_0$, $v(s)=v(0)-s$ and $x(s)\equiv x(0)$, and thus, $\gamma$ is complete. So, assume by contradiction that $\Delta u>0$.
%
%........
%
%
%Taking into account that $u(s)$ is affine, the domain of definition of $\gamma$ must be some interval $(s_0,s_*)\subset\R$ with $s_*<\infty$. Since $(M,g)$ is complete, the trajectories.... are also complete. Hence, $x(s)\rightarrow x_*$ as $s\nearrow s_*$. Hence, $\gamma(s)\rightarrow (x_*,-s_*,u_0)$, and thus, $I^{-}(\gamma)$ is a PIP, in contradiction with the identity $I^{-}(\gamma)=I^{-}(\gamma_{x_0,u_0})$. \qcd

\section*{Appendix}
\label{sec:some-examples}
In Definitions \ref{scri} and \ref{ample} we have included some clauses that, even though natural when interpreted from the classical conformal approach viewpoint, might conceivably be discarded in favor of less technical-looking ones. The following construction shows that this is not the case if the thesis of Theorem \ref{main} is to be preserved. In fact, we will display a globally hyperbolic spacetime which is not past-complete and where (unsurprisingly) Theorem \ref{main} fails. The example also shows that the Theorem \ref{main} is also false if past-completeness is assumed but condition (ii) on Definition \ref{scri} is removed. This will show in particular that, even if with strong causality requirements on the spacetime, one cannot expect to conserve Theorem \ref{main} if the past-completeness condition is removed from the notion of regularity in Definition \ref{ample}.
%In fact, by means of the \cambiosj{two} firsts examples we show that Thm. \ref{main} is no longer true if the past-complete condition\footnote{\cambiosn{¿Algun ejemplo que pruebe que tampoco podemos quitar la condicion ample?}} is removed from the notion of regularity for ${\cal J}^+$ (Definition \ref{ample}).


\begin{note}
\emph{The construction considered below is given by making simple modifications of Minkowski spacetime. It is not difficult to realize that the future c-boundaries of the resulting spacetimes are always Hausdorff. In particular, from Proposition \ref{lema:auxiliar}, the set $\widetilde{I^+(C)}$ will be closed for any compact set $C$ in the spacetime.}
\end{note}

%The examples are based on Minkowski spacetime, ensuring a quite regular main structure. For instance, it is a simple exercise to see that, given a compact set $K$, $\widetilde{I^{+}(K)}=J^{+}(K)$.

% \cambiosj{In Theorem \ref{main} the future null infinity is assumed to be regular in the sense of Definition \ref{ample}. The first, very simple, example shows that this result fails when the past-complete condition is removed from this notion of regularity.}

% \cambiosj{In the following example we construct a globally hyperbolic spacetime which is not past-complete and, not surprisingly, Theorem \ref{main} fails in this case. This shows that, even if we additionally assume some causality requirement on the spacetime, we cannot expect to conserve Theorem \ref{main} if we remove the past-completeness condition from the notion of regularity in Definition \ref{ample}.}

% \smallskip

Consider the $3$-dimensional Minkowski spacetime
\[
\mathbb{L}^3 = (\mathbb{R}^3, dx^2+dy^2-dt^{2}).
\]
Isometrically compactify its (cartesian) $x$-coordinate as a circle $\mathbb{S}^1$ in order to obtain a new (flat) Lorentzian manifold $\tilde{M}$ whose spatial sections are $2$-dimensional cylinders. Note that the future-complete null lines of $\tilde{M}$ consist of straight lines of $\tilde{M}$ with spatial component parallel to the $y$-axis. In fact, the spatial component of any other inextendible lightlike geodesic $\sigma$ in $\tilde{M}$ will ``waste time'' spinning around the spatial cylinder, and eventually, two points of $\sigma$ will become chronologically related by some timelike curve $c$ (see Figure \ref{fig:1}).

\begin{figure}
\centering

  \setlength{\unitlength}{1bp}%
  \begin{picture}(99.80, 230.32)(0,0)
  \put(-13,0){\includegraphics{fig1NI.pdf}}
  \put(76.52,124.51){\rotatebox{0.00}{\fontsize{8.54}{10.24}\selectfont \smash{\makebox[0pt][l]{$1$}}}}
   \put(96.52,109.76){\fontsize{8.54}{10.24}\selectfont $\mathbf{x}$}
  \put(8.37,124.28){\rotatebox{0.00}{\fontsize{8.54}{10.24}\selectfont \smash{\makebox[0pt][l]{$-1$}}}}
  \put(69.36,78.76){\fontsize{8.54}{10.24}\selectfont $\sigma$}
  \put(14.36,10.76){\fontsize{8.54}{10.24}\selectfont $\sigma(t_0)$}
  \put(26.36,67.76){\fontsize{8.54}{10.24}\selectfont $c$}
  \put(69.36,202.76){\fontsize{8.54}{10.24}\selectfont $\sigma(t_1)$}
   \put(42.36,218.76){\fontsize{8.54}{10.24}\selectfont $\mathbf{y}$}
  \end{picture}%


  % \begin{picture}(230.32, 99.80)(0,0)
  % \put(0,0){\includegraphics{Figuras/fig1NI.pdf}}
  % % \put(123.36,82.76){\fontsize{8.54}{10.24}\selectfont $1$}
  % % \put(64.36,78.76){\fontsize{8.54}{10.24}\selectfont $\sigma$}
  % % \put(14.36,12.76){\fontsize{8.54}{10.24}\selectfont $\sigma(t_0)$}
  % % \put(160.36,62.76){\fontsize{8.54}{10.24}\selectfont $c$}
  % %  \put(187.36,75.76){\fontsize{8.54}{10.24}\selectfont $\sigma(t_1)$}
  % %  \put(121.60,12.12){\fontsize{8.54}{10.24}\selectfont $-1$}
  % \end{picture}%
% \else
%   \setlength{\unitlength}{1bp}%
%   \begin{picture}(230.32, 99.80)(0,0)
%   \put(0,0){\includegraphics{Figuras/fig1NI}}
%   \put(123.36,84.76){\fontsize{8.54}{10.24}\selectfont $1$}
%   \put(121.60,10.12){\fontsize{8.54}{10.24}\selectfont $-1$}
%   \end{picture}%
%   \fi
  \caption{\label{fig:1} Representation of the spatial component of the spacetime $\tilde{M}$. If the spatial component of a lightlike geodesic $\sigma$ in $\tilde{M}$ is not parallel to the $y$-axis, then two points of $\sigma$ will eventually become chronologically related in $\tilde{M}$ by some timelike curve $c$.}


  %In fact. assume both that $\gamma$ is a null ray with $\gamma(t)=(t,\sigma(t))$, and that $\sigma$ behaves as in the figure. The null character of the curve ensures that $t_1-t_0={\rm length}(\sigma|_{[t_0,t_1]})$. Now observe that, from construction, ${\rm length}(\sigma|_{[t_0,t_1]})>{\rm length}(c)=d(\sigma(t_0),\sigma(t_1))$. Therefore, and due the characterization of the chronological relation in static spacetimes, $\sigma(t_0)\ll \sigma(t_1)$, i.e., $\gamma$ is no longer a ray in $\tilde{M}$.}
\end{figure}

Next, consider the compact set
\[
C=\{(0,y,0)\, : \, -1\leq y \leq 1\},
\]
and define

  \[
\Sigma:= \{(0,y,1)\, : \, y \in \mathbb{R}\}\cap J^+(C,\tilde{M}).
    \]
Consider the Lorentzian manifold $(M,g)$, where $M:=\tilde{M}\setminus J^+(\Sigma,\tilde{M})$ and $g$ is the induced metric on $M$ from $\tilde{M}$ (see figures  \ref{fig:3}  and \ref{fig:2} for illustrations of the projections of $M$ onto the $y=0$ and $x=0$ planes, resp.).

  The $c$-boundary of $(M,g)$ is the disjoint union of the future and past c-boundaries, each being formed by pairs with an empty $F$ or $P$ component, resp. In particular, $(M,g)$ is globally hyperbolic according to \cite[Theorem 3.29]{Floresfinaldefinitioncausal2011}. Moreover, the pairs $(P,\emptyset)$ of the future c-boundary can be separated in two classes: (a) those pairs defined by inextensible timelike curves with divergent component $y$, and (b) those pairs identifiable with the points of the (topological) boundary of $J^+(\Sigma,\tilde{M})$. It readily follows that the former points are in $\mathcal{J}^{+}$, since there exist complete null rays defining the corresponding TIPs (here we can proceed just as in Minkowski spacetime). However, the latter pairs belong to $\partial M\setminus \mathcal{J}^+$, since the null rays defining such TIPs are incomplete.  In any case, the following properties hold:

\begin{figure}
\centering
  \setlength{\unitlength}{1bp}%
  \begin{picture}(227.21, 140.19)(0,0)
    \put(0,0){\includegraphics{fig2NI.pdf}}
  \put(205.82,35.23){\fontsize{11.16}{13.00}\selectfont $\mathbf{y}$}
  \put(86.47,24.23){\fontsize{9.16}{11.00}\selectfont $C$}
  \put(134.95,63.63){\fontsize{9.16}{11.00}\selectfont $\Sigma$}
  \put(99.32,90.85){\fontsize{9.16}{11.00}\selectfont $J^+(\Sigma,\tilde{M})$}
  \put(115.03,123.78){\fontsize{11.75}{13.50}\selectfont $\mathbf{t}$}
  \end{picture}%
  \caption{\label{fig:3} Representation of the intersection of $(M,g)$ with the plane $x=0$. This is a standard plane with both the grey area and the set $\Sigma$ removed. All future-complete null rays in $M$ departing from points of the form $(t,0,y)$ are contained in this plane. In particular, any null ray departing from $C$ intersects $\Sigma$, and thus, there are no future-complete null $C$-rays.}
\end{figure}

\begin{figure}
\centering
  \setlength{\unitlength}{1bp}%
  \begin{picture}(216.58, 213.83)(0,0)
  \put(0,0){\includegraphics{fig3NI.pdf}}
  \put(104.74,199.52){\fontsize{11.07}{13.28}\selectfont $\mathbf{t}$}
  \put(191.92,99.81){\fontsize{11.07}{13.28}\selectfont $\mathbf{x}$}
  \put(161.92,99.81){\fontsize{11.07}{13.28}\selectfont $1$}
   \put(42.92,99.81){\fontsize{11.07}{13.28}\selectfont $-1$}
  \put(107.35,172.11){\fontsize{9.38}{9.86}\selectfont $J^+(\Sigma,\tilde{M})$}
  \put(107.71,135.42){\fontsize{9.38}{9.86}\selectfont $\Sigma$}
  \put(107.07,99.46){\fontsize{9.38}{9.86}\selectfont $C$}
  \end{picture}%
  \caption{\label{fig:2} Representation of the intersection of $(M,g)$ with the plane $y=0$. Here, the projections of $\Sigma$ and $C$ are points, and the lines $x=-1$, $x=1$ of this plane are identified.}
\end{figure}

\begin{itemize}
\item ${\cal J}^+$ is ample: Let $K\subset M$ be any compact set. In order to show the existence of points in ${\cal J}^+$ not contained in the closed set $\widetilde{I^+(K)}$, let $(x_0,y_0,t_0)$ be a point in $M$ such that both $K$ and $J^+(\Sigma)$ are contained in $I^+(x_0,y_0,t_0)$. For any $\epsilon>0$ the null line $\gamma(t)=(x_0,y_0+t,t_0-\epsilon+t)$ is contained in $M$ (since it does not intersect $J^+(\Sigma)$), and defines a point $(P,\emptyset)\in {\cal J}^+$ with $(x_0,y_0,t_{0})\not \in P=I^-(\gamma)$. Moreover, the pair $(P,\emptyset)$ is not contained in $\widetilde{I^{+}(K)}$. Indeed, otherwise there would exist
some $q\in K$ such that $I^{-}(q)\subset P$. But by construction,
$q\in I^{+}(x_0,y_0,t_{0})$, and thus, $(x_0,y_0,t_{0})\in P$, in contradiction with the properties of $\gamma$.

    %which is not contained in $J^+(K)$ (otherwise, the curve $\gamma$ would intersect $I^+(K)$, an absurd.)

  % For the Hausdorff condition, just observe that the c-completion on $M$ shares some properties of the well-known c-completion of the Minkowski spacetime, including its Hausdorff and AN1 character (in fact, the c-completion of the Minkowski spacetime coincides with its conformal completion.)

\item $\mathcal{J}^{+}$ is not past-complete: Consider the null line $\sigma$ in $M$ given by $\sigma(t)=(1/2,1+t,t)$, which defines a pair $(P,\emptyset)\in \mathcal{J}^+$ with $P=I^-(\sigma)$. By construction, $\partial P$ is a plane containing $\sigma$ and lying on the boundary of $J^+(\Sigma,\tilde{M})$. In particular, the future null line $\sigma'(t)=(0,1+t,t)$ is a null geodesic generator of $\partial P$. However, the pair $(P',\emptyset)\in \overline{M}$, $P'=I^-(\sigma')$ (which is associated with the point $(0,2,1)$) is not included in $\mathcal{J}^{+}$. Therefore $\mathcal{J}^{+}$ is not past-complete.
  % By abuse of notation, we will write $I^-((1,0,2))$ for the past of any inextendible future timelike curve on $M$ with endpoint $(1,0,2)$. By construction, two boundary points $(I^-((1,0,2)),\emptyset)\in \partial M\setminus {\cal J}^+$ (no future-complete null ray arrives at $(1,0,2)$) and $(I^-(\sigma),\emptyset)\in {\cal J}^+$ with $I^-((1,0,2))\subset I^-(\sigma)$. Therefore, ${\cal J}^+$ is not past-complete.
  % \item \cambiosj{$C$ is not fully covered by a black hole:\footnote{\cambiosj{OJO, AQUI JONY HA DETECTADO UN ERROR QUE DEBEMOS COMENTAR!!! NO ESTA CLARO QUE EMANE UN FUTURE COMPLETE NULL RAY DESDE Q.}} Let us show that} the intersection $I^+(C)\cap  \mathcal{J}^+$ is non-empty. In fact, for each point $p\in C$ there exists a point $q\in I^+(p)\cap M$ with time component equal to $1$, from which a future-complete null ray emanates. Intuitively, if we move slightly along the $x$-axis, we can avoid the set $J^+(\Sigma)$, showing the existence of future-complete null rays starting at $q$ (see Figure \ref{fig:2}).
  \item There are no future-complete null C-rays, since any such null C-ray in $\tilde{M}$ must intersect $J^{+}(\Sigma,\tilde{M})$ (see Figure \ref{fig:3}).
  % For the closeness of $L(I^+(C))$ just recall from previous point that the topology on $\overline{M}$ is AN1.


    \end{itemize}
Thus, in order to violate the thesis of Theorem \ref{main}, we need to show that $C$ is not entirely contained inside the black hole.
Our construction, however still does not satisfy this property, since all causal curves emanating from $C$ intersect $J^{+}(\Sigma,\tilde{M})$, whose boundary is not contained in $\mathcal{J}^+$.

To complete the construction, consider then a conformally rescaled metric $\mathfrak{g}=\Omega\, g$ such that: (i) all the boundary points in $J^+(\Sigma,\tilde{M})\setminus\Sigma$ are in $\mathcal{J}^{+}$ with $\mathfrak{g}$ and (ii) all the null $C$-rays are still incomplete. In order to obtain an appropriate conformal factor $\Omega$, recall a classic result due to Clarke \cite{Clarkegeodesiccompletenesscausal1971} ensuring the existence of a conformal factor $\tilde{\Omega}$ such that all lightlike geodesics on $(M,\tilde{\Omega} g)$ are complete. Now define $\Omega$ thus: (i) $\Omega\equiv 1$ in $I^{-}_{1/4}(\Sigma)$ and (ii) $\Omega\equiv \tilde{\Omega}$ in $M\setminus I^{-}(\Sigma)$, where $I^{-}_{a}(\cdot)$ denotes the chronological past computed with the metric $-adt^2+dx^2+dy^{2}$ (see Figure \ref{fig:4}).
This conformal factor ensures that all the future-directed lightlike geodesics with endpoint in $J^+(\Sigma,\tilde{M})\setminus \Sigma$ (and so, contained in a region where $\Omega\equiv \tilde{\Omega}$) are complete. In particular, $\left(J^+(\Sigma,\tilde{M})\setminus \Sigma\right)\subset \mathcal{J}^+$.

The situation with $\Sigma$ is quite peculiar: there exist future-complete null rays with endpoints on $\Sigma$ (so condition (i) on Defn. \ref{scri} holds) but not all null geodesics with endpoints on $\Sigma$ are complete (consider for instance the null rays emanating from $C$.) Hence, $\Sigma$ does not intersect $\mathcal{J}^+$ \textit{just because condition (ii) on Defn. \ref{scri} fails}. But now, since $\Sigma$ does not intersect $\mathcal{J}^+$, $(M,\Omega\,g)$ in turn fails to be past-complete, just by the same previous arguments. However, all the points in $C$ are now visible, as we can connect them with boundary points of $\left(J^+(\Sigma,\tilde{M})\setminus \Sigma\right)$ by means of a future-directed timelike curves.

\begin{figure}
\centering
\ifpdf
  \setlength{\unitlength}{1bp}%
  \begin{picture}(216.58, 213.83)(0,0)
  \put(0,0){\includegraphics{fig4NI.pdf}}
  % \put(104.74,199.52){\fontsize{11.07}{13.28}\selectfont $T$}
  % \put(191.92,97.81){\fontsize{11.07}{13.28}\selectfont $X$}
  % \put(111.35,172.11){\fontsize{7.38}{8.86}\selectfont $J^+(\Sigma)$}
  % \put(107.71,139.41){\fontsize{7.38}{8.86}\selectfont $\Sigma$}
  % \put(105.07,99.46){\fontsize{7.38}{8.86}\selectfont $C$}
  \put(90.52,48.36){\fontsize{13.38}{16.86}\selectfont $\Omega\equiv 1$}
  \put(190.52,100.36){\fontsize{11.38}{14.86}\selectfont $\mathbf{x}$}
     \put(104.52,205.36){\fontsize{11.38}{14.86}\selectfont $\mathbf{t}$}
  \put(35.76,133.07){\fontsize{13.38}{16.86}\selectfont $\Omega=\tilde{\Omega}$}
  \end{picture}%
\else
  \setlength{\unitlength}{1bp}%
  \begin{picture}(216.58, 213.83)(0,0)
  \put(0,0){\includegraphics{fig4NI}}
  % \put(104.74,199.52){\fontsize{11.07}{13.28}\selectfont $T$}
  % \put(191.92,97.81){\fontsize{11.07}{13.28}\selectfont $X$}
  % \put(111.35,172.11){\fontsize{7.38}{8.86}\selectfont $J^+(\Sigma)$}
  % \put(107.71,139.41){\fontsize{7.38}{8.86}\selectfont $\Sigma$}
  % \put(105.07,99.46){\fontsize{7.38}{8.86}\selectfont $C$}
  \put(94.52,52.36){\fontsize{7.38}{8.86}\selectfont $\Omega\equiv 1$}
  \put(36.76,133.07){\fontsize{7.38}{8.86}\selectfont $\Omega=\tilde{\Omega}$}
  \end{picture}%
  \fi

  \caption{\label{fig:4} Illustration of the behaviour of the conformal factor $\Omega$ in the section $y=0$ of the spacetime.}
\end{figure}


\begin{remark}
 \emph{Observe that the previous example also shows that condition (ii) on Definition \ref{scri} too necessary to obtain Theorem \ref{main}. In fact, as we have mentioned, if we remove such a condition then the points of $\Sigma$ will also belong to $\mathcal{J}^{+}$, and the spacetime is actually past-complete. However, it is still true that there are no future-complete null $C$-rays. }
\end{remark}

%     \vspace{3cm}

%  \noindent    However, there are no future-complete null $C$-rays, since any such null $C$-ray in $\tilde{M}$ intersects $J^+(\Sigma,\tilde{M})$ (see Figure \ref{fig:3}). \cambiosj{In conclusion, Theorem \ref{main} is false in this case.}}






% \cambiosj{In the last example we will show that Theorem \ref{main} does not hold if condition (ii) is removed from the notion of future null infinity $\mathcal{J}^+$ (Definition \ref{scri}), but still assuming regularity of $\mathcal{J}^+$.}
% % \begin{example}
% %   {\em  \cambiosj{The spacetime considered here is a slight conformal modification of the previous one. In particular, it will remain globally hyperbolic with the same c-boundary structure as before. However, the resulting null infinity $\mathcal{J}^+$ will be different, since it is not conformally invariant.}

% %     A classic result by Clarke (see \cite{Clarkegeodesiccompletenesscausal1971}) ensures the existence of a conformal factor $\Omega \in C^{\infty}(M)$ on $M$ such that $(M,\Omega^2 g)$ is null geodesically complete. \cambiosj{Let $\tilde{\Omega}:M\rightarrow [0,1]$ be another smooth function satisfying: (i) $\tilde{\Omega}\equiv 0$ in $I^{-}_{1/4}(\Sigma)$ and (ii) $\tilde{\Omega}\equiv 1$ in $M\setminus I^{-}_{1/2}(\Sigma)$, where $I^{-}_{a}(\cdot)$ denotes the chronological past computed with the metric $-adt^2+dx^2+dy^{2}$.}
% % \footnote{Esto esta incompleto, ¿no? \cambiosj{Nota de Jony: Ahora ya sí creo que no...}}
% % %   \[
% % %   \tilde{\Omega}(t,0,y)=0,\quad \tilde{\Omega}(t,x,y)=1\;\; \hbox{$\forall$ $t\geq 2x+1$ and $x<0$, and $\forall$ $t\geq -2x+1$ and $x>0$.}
% % % \]
% % Consider he metric $\hat{g}=\hat{\Omega}^{2}g=\left(\tilde{\Omega}\Omega+(1-\tilde{\Omega}) \right)^2 g$ on $M$, \cambiosj{This metric coincides with $g$ on the set $I^{-}_{1/4}(\Sigma)$, which includes the trapezium between $C$ and $\Sigma$ in Figure \ref{fig:3}}. Hence, the null rays departing from points in $C$ parallel to the $y$-axis are still future-incomplete.

% %  \cambiosj{However the set $\mathcal{J}^{+}$ contains all the pairs $(P,\emptyset)\in \partial M$, and so, it is necessarily past-complete. In fact, $\mathcal{J}^{+}$ includes the null infinity points of previous spacetime $(M,g)$, since its complete null lines are also complete in $(M,\hat{g})$. Moreover, $\mathcal{J}^{+}$ includes the pairs $(P,\emptyset)$ identifiable with points at the boundary of $J^+(\Sigma,\tilde{M})$. In fact, these points are reachable by future null rays whose intersection with $I^{-}_{1/2}(\Sigma)$ is empty, and so, they belong to a region on $M$ where the conformal factor $\hat{\Omega}$ coincides with $\Omega$; that is, these null rays in $(M,g)$ become complete null lines in $(M,\hat{g})$, since $\hat{g}=\Omega^2g$ around them (recall the definition of $\Omega$.)}

% % Summarizing, the future null infinity is ample (reasoning as in previous example) and past-complete. However, there are not complete null rays emanating from $C$, showing that Theorem \ref{main} is false in this case.

% %   }
% % \end{example}


%%% Local Variables:
%%% mode: latex
%%% TeX-master: "CostaFloresHerreraNullInfty"
%%% End:
