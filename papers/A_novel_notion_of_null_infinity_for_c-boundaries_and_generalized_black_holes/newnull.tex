\section{Null infinity on $c$-boundaries}\label{bh}

We wish to introduce a notion of null infinity on $c$-boundaries which remains valid even when a conformal boundary does not exist.
%This notion will be defined so as to include, in a definite sense \cambiosn{(cf. Corollary \ref{voila} below),} the conformal notion of null infinity, whenever the latter is available.
Again, we will start with some general considerations, which will motivate our main results and definitions, as well as set up the appropriate notation.

First, let us give a more precise description of what we shall mean by {\em null infinity} here in the case when conformal boundaries are present. We will try and capture only some minimal aspects of the original Penrose's definition \cite{PenroseAsymptoticStructure1963,PenroseConformalInfinity1964,Frauendiener2004}, without some of the more technical and specific assumptions needed in physical applications (such as those given, for example, in Refs. \cite{HawkingLargeScaleStructure1975,WaldGeneralRelativity1984}).

\begin{definition}
\label{conformalinfinity1}
Let $i: (M,g) \hookrightarrow (\tilde{M},\tilde{g})$ be a conformal extension of the spacetime $(M,g)$ with conformal boundary $\partial_i M$ and conformal factor $\Omega$. An accessible point $p \in \partial^{*}_i M$ is said to {\em be at infinity} if there exist an open set $\tilde{U} \ni p$ of $\tilde{M}$ and a $C^{\infty}$ extension\footnote{It is possible, and may be interesting for certain purposes, to consider a lower - say, $C^1$-differentiability class without deep changes in the main results. Nevertheless, since this would make the following discussion a little more cumbersome, we adopt for simplicity a stronger regularity assumption.} $\tilde{\Omega}$ of $\Omega$ to $M \cup \tilde{U}$ such that $\tilde{\Omega}(p) =0$. Such a point at infinity is said to be {\em regular} if in addition $d\tilde{\Omega}(p) \neq 0$. The set of all regular points at infinity is the {\em conformal null infinity} ({\em associated with the conformal extension} $(\tilde{M},\tilde{g})$) and will be denoted by ${\cal J}_c$. We will say that a point $p\in {\cal J}_c$ belongs to the {\em future} (resp. {\em past}) {\em null infinity}, denoted by ${\cal J}_c^+$, (resp. ${\cal J}_c^-$) if $p$ is future (resp. past) accessible (recall Defn. \ref{def:envelopment})
\end{definition}

\begin{remark}
\label{rmk3}
{\em The following points about Definition \ref{conformalinfinity1} are relevant.
\begin{itemize}
\item[1)] The notions of a point $p \in \partial^{*}_i M$ being at infinity and/or being regular there clearly do not depend on the particular extension $\tilde{\Omega}$ of the conformal factor.
\item[2)] The notion of a point being at infinity is {\em not} a conformally invariant one. For example, if $\Omega$ can be smoothly extended to a point $p \in \partial^{*}_i M$ but the extension does not vanish there, we may rescale $g$ by (the restriction of) a conformal factor $\omega^2 \in C^{\infty}(\tilde{M})$ which diverges at $p$ suitably slowly so that $\Omega/\omega$ is still $C^{\infty}$. The point $p$ then becomes a point at infinity with respect to $(\tilde{M}, \tilde{g})$ when the latter is viewed as a conformal extension of $(M,\omega^2g)$. Conversely, points which are at infinity for a certain spacetime may fail to be so for others in the same conformal class. Indeed, the familiar example of $2d$ hyperbolic space already illustrates these remarks: points at the boundary of the unit disc in $\mathbb{R}^2$ are at infinity if we view the Euclidean plane as a conformal extension of the Poincar\'{e} disc model of the hyperbolic plane, but not if viewed as a conformal (actually isometric) extension of the (restriction of the) Euclidean metric on the disc.
\item[3)] As an example of this definition, we may look back at the example in item 3. of Remark \ref{rmk1}. There, the set of points at infinity consists of the union of the two segments $u=\pi/2,\, v>-\pi/2$ and $v=\pi/2, \, u> -\pi/2$. The point $u=v= \pi/2$ is the only non-regular point. This point is not the future endpoint of any null geodesic in $(M,g)$, but all {\em regular} points at infinity are endpoints of null geodesic {\em rays}.
    %\cambiosn{This latter feature is actually general for future-nesting conformal envelopments, as the next result, which also summarizes the features of points at infinity relevant to us here, shows.}
\end{itemize}
}
\end{remark}

\begin{theorem}
\label{conformalinfinity2}
Let $(\tilde{M},\tilde{g})$ be a conformal extension of the spacetime $(M,g)$ with conformal boundary $\partial_iM$ and conformal factor $\Omega$. Let $p \in \partial^{*}_iM$. Then the following statements hold.
\begin{itemize}
\item[i)] Suppose $p$ is at infinity, and let $\alpha:[0,A) \rightarrow M$ ($0<A \leq +\infty$) be a future-inextendible null geodesic in $(M,g)$ (analogously for past-directions). If $\alpha$ has a future endpoint at $p$ when viewed as a causal curve in $(\tilde{M}, \tilde{g})$, then $A=+\infty$ and $\alpha$ is future-complete. In other words, any null geodesic in $(M,g)$ with a future endpoint at infinity is future-complete.
\item[ii)] Conversely, suppose the conformal factor extends in a $C^{\infty}$ fashion to $M\cup \tilde{U}$ where $\tilde{U} \ni p$ is open in $\tilde{M}$. If $p$ is a future (resp. past) endpoint of a future-complete (resp. past-complete) null geodesic in $(M,g)$, then $p$ is at infinity.
\item[iii)] Suppose $(M,g)$ is inextendible (as a spacetime\footnote{Again, this result will still work {\em mutatis mutandis} if one only assumes that $(M,g)$ cannot be extended as a $C^1$ spacetime. On the other hand, there are trivial counterexamples to this result without the assumption of inextendibility, obtained by taking $(M,g)$ to be a suitable open subset of Minkowski with rough boundary.}). Then ${\cal J}_c$ is a smooth hypersurface in $(\tilde{M},\tilde{g})$.
    %%\cambiosn{
%    If in addition $(M,g)$ is globally hyperbolic and $(\tilde{M},\tilde{g})$ is future-nesting (resp. past-nesting), then for each $p \in {\cal J}_c^+$ there exists a future-complete (resp. past-complete) null geodesic ray $\gamma:[0,+\infty) \rightarrow M$ in $(M,g)$ with future (resp. past) endpoint $p$.
%    %}
\end{itemize}
\end{theorem}
\begin{proof} We shall present our arguments for future directions, since the past directed case is again analogous. \\
$(i)$ Using Definition \ref{conformalinfinity1} we can assume that the conformal factor extends in a $C^{\infty}$ fashion to $M\cup \tilde{U}$, where $\tilde{U} \ni p$ is open in $\tilde{M}$, and $\tilde{\Omega}(p)=0$. We shall, for notational simplicity, continue to refer to this extended conformal factor as $\Omega$. Denote by $\tilde{\nabla}$ the Levi-Civita connection on $(\tilde{M},\tilde{g})$. Using the fact that $\tilde{g}|_M = (\Omega|_M)^2 g$ and that $\alpha$ is a null geodesic in $(M,g)$, we get
\begin{equation}
\label{geodesic}
\tilde{\nabla}_{\dot{\alpha} }\dot{\alpha} = \nabla_{\dot{\alpha}}\dot{\alpha} + 2\frac{(\Omega \circ \alpha)'}{(\Omega \circ \alpha)} \dot{\alpha} \equiv f \cdot \dot{\alpha},
\end{equation}
where $f:[0,A) \rightarrow \mathbb{R}$ is the smooth function given by
\[
f := 2\frac{(\Omega \circ \alpha)'}{\Omega \circ \alpha}.
\]
Exercise 3.19 in p. 95 of \cite{ONeillSemiRiemannianGeometryApplications1983} now implies that $\alpha$ is a null pregeodesic in $(\tilde{M},\tilde{g})$. Let $h:[0,a) \rightarrow [0,A)$ ($0<a \leq +\infty$) be an increasing reparametrization of $\alpha$ such that $\hat{\alpha}:= \alpha \circ h$ is a geodesic in $(\tilde{M},\tilde{g})$. Since it has an endpoint in $p \in \partial^{*}_i M \subset \tilde{M}$, it is actually extendible as a null geodesic in $(\tilde{M},\tilde{g})$, and in particular $a <+\infty$. Denote by $\beta:[0,a] \rightarrow \tilde{M}$ the null $\tilde{g}$-geodesic segment extending $\hat{\alpha}$, with $\beta(a) \equiv p$. Since $\Omega(p) =0$, and $\Omega$ is $C^1$, by the mean value theorem there exists a number $k>0$ such that
\begin{equation}
\label{lipschitz}
(\Omega \circ \beta)(s) \leq k (a - s), \forall s \in [0,a].
\end{equation}
However, by exercise 3.19 of \cite{ONeillSemiRiemannianGeometryApplications1983} we must have
\[
h'' + (f \circ h) (h')^2 =0
\]
on $[0,a)$, which when we substitute the definition of $f$ gives, multiplying through $(\Omega \circ \alpha \circ h)^2$,
\begin{eqnarray}
0 &=& (\Omega \circ \alpha \circ h)^2 h'' + 2 (\Omega \circ \alpha \circ h)[(\Omega \circ \alpha)'\circ h)h ']h' \\ \nonumber
&=& (\Omega \circ \alpha \circ h)^2 h'' + 2 (\Omega \circ \alpha \circ h)(\Omega \circ \alpha \circ h)'h' \\ \nonumber
&=& (\Omega \circ \hat{\alpha})^2 h'' + 2(\Omega \circ \hat{\alpha})(\Omega \circ \hat{\alpha})'h' \\ \nonumber
&\equiv & [(\Omega \circ \hat{\alpha})^2 h']'.
\end{eqnarray}
Hence, for some constant $c \in \mathbb{R}$,
\begin{equation}
\label{relation}
(\Omega \circ \hat{\alpha})^2 h' = c.
\end{equation}
(We deduce that $c>0$, since $\Omega \circ \hat{\alpha} >0$ and $h$ is strictly increasing.) Therefore, for each $0\leq t <a$, we have
\begin{equation}
\label{integral}
h(t) = \int_0^t h'(s) ds = c \int_0^t \frac{ds}{[(\Omega \circ \hat{\alpha})(s)]^2} \geq \frac{c}{k^2} \int_0^t \frac{ds}{(a -s)^2},
\end{equation}
where we have used (\ref{lipschitz}) to get the last inequality. We thus conclude that for each $0\leq t <a$,
\[
A \geq h(t) \geq \frac{c}{k^2}\left( \frac{1}{a-t} - \frac{1}{a} \right).
\]
Since the right-hand side of this inequality diverges when $t \rightarrow a$, we conclude that $A \equiv +\infty$, proving $(i)$.

\smallskip

$(ii)$ Using the same notation as in $(i)$, assume, by way of contradiction, that $\alpha$ is future-complete in $(M,g)$, that is, that $A=+\infty$, but that $\Omega(p) \neq 0$. Then we may take $(\Omega \circ \hat{\alpha})$ to be bounded from below by some positive constant $B$ in Eq. (\ref{integral}), whence we conclude that
\[
h(t) \leq c/B, \, \forall t \in [0,a),
\]
and hence that $A \leq c/B$, a contradiction.

\smallskip

$(iii)$ Again, we may focus without loss of generality on the future part ${\cal J}^+_c$ of the conformal boundary. Let $p \in {\cal J}^+_c$. By definition, there exists an open neighborhood $\tilde{U}\ni p$ in $\tilde{M}$ to which $\Omega$ extends as a $C^{\infty}$ function (which we again denote by $\Omega$). Let $\tilde{\nabla}\Omega$ denote the (metrically) associated gradient vector field in $M\cup \tilde{U}$. Since $\tilde{\nabla}\Omega (p) \neq 0$, we can assume (shrinking it if necessary) that $\tilde{U}$ is a connected coordinate neighborhood contained in $I^+(M, \tilde{M})$, with a coordinate system denoted by $(x^1, \ldots x^{dim M})$, say, such that $\partial _{x^1} = \tilde{\nabla}\Omega|_{\tilde{U}}$. Then the set
\[
S := \{q \in \tilde{U} \, : \, x^1(p) = \Omega(q) =0 \}
\]
is a $C^{\infty}$ embedded codimension 1 submanifold (i.e., a hypersurface) of $\tilde{M}$ containing ${\cal J}^+_c \cap \tilde{U}$ and closed in $\tilde{U}$. Denote by $\tilde{U}_+$ the connected open subset of $\tilde{U}$ in which $\Omega >0$. Now, $M \cap \tilde{U}$ is obviously (non-empty and) contained in $\tilde{U}_+$, and if it were {\em properly} contained therein, then $(M\cup \tilde{U}_{+}, (1/\Omega^2) \tilde{g} |_{M\cup \tilde{U}_{+}})$ would be a non-trivial extension of $(M,g)$, contrary to our assumption. Therefore, $M \cap \tilde{U} = \tilde{U}_+$, and we then conclude that $\partial ^+ M \cap \tilde{U}\equiv S$, and hence that $S = {\cal J}^+_c \cap \tilde{U}$. This in turn establishes the first statement.

%\cambiosn{
%For the second claim, assume that $(M,g)$ is also globally hyperbolic and $(\tilde{M},\tilde{g})$ is future-nesting. Using the notation and results in the previous paragraph, let ${\cal N} \subset \tilde{U}$ be a normal neighborhood of $p$ which is the diffeomorphic image by the exponential map $\tilde{\exp}_p$ of $(\tilde{M},\tilde{g})$ of some open Euclidean ball $B \subset T_p\tilde{M}$. Now, $I^-(p, \tilde{M})\cap {\cal N} = \tilde{\exp}_p(B \cap N^-_p)$, where $N_p^- \subset T_p\tilde{M}$ is the past null cone (including the origin). Moreover, $T_pS \equiv \tilde{\nabla}\Omega ^{\perp}$ is a hyperplane in $T_p\tilde{M}$, and hence cannot coincide with the past null cone $N_p^-$. We can then pick $v \in B\cap N_p^- \setminus T_pS$. Then, either $d\Omega_p(v)<0$ or $d\Omega_p(v)>0$. In the first case, however, we'd also have $d\Omega_p(w)<0$ for some past-directed {\em timelike} vector, by continuity. Thus, for some $\delta>0$, the past-directed timelike geodesic $\gamma_w:[0,\delta] \rightarrow \tilde{M}$ with initial velocity $w$ would satisfy $\Omega (\gamma_w(t)) <0$ for each $t \in (0, \delta)$, and hence its final portions would be entirely outside $M$, contradicting Lemma \ref{technical}.
%}

%\cambiosn{
%We conclude that $d\Omega_p(v)>0$. Then, for some $\varepsilon >0$, we will have, for the past-directed null geodesic $\gamma_v:[0,\varepsilon] \rightarrow \tilde{M}$ with initial tangent vector $v$, that $\gamma_v(0,\varepsilon]\subset M$. By standard results in causality theory, $\gamma_v|_{(0,\varepsilon)}$ is a globally achronal (since it is a null generator of the past cone) null pregeodesic in $M$. Therefore, it admits a {\em future-directed} reparametrization as a null, affinely parametrized geodesic ray $\alpha:[0,+\infty) \rightarrow M$ in $(M,g)$, which is the sought-for ray. (It is future-complete by $(i)$, since it has an endpoint at infinity.)
%}
\end{proof}

We now turn to a generalized notion of null infinity. 
%In doing so, we shall take the regular conformal infinity ${\cal J}_c$ as our model. This will also allow us to define a generalized notion of black hole which applies to any strongly causal spacetime.
Let again a strongly causal spacetime $(M,g)$ with $c$-completion $\overline{M}$ and $c$-boundary $\partial M$ be given. Wherever confusion might arise between conformal and $c$-completions, we use lowercase letters $p,q,r, \ldots$ to denote points on $\overline{M}_i$, while pairs $(P,F)$ are used to denote elements of the $c$-completion $\overline{M}$. The chronological future/past of a subset $U$ on $\overline{M}$ (resp. $\overline{M}_i$) will be denoted by $I^{\pm}(U)$ (resp. $I_i^{\pm}(U)$).

% {\em Henceforth, whenever a topological statement is made about the $c$-completion, the CLT $\tau_c$ on $\hat{M}$ is always to be understood.}

% The first issue we must clarify is the following. Given a future-directed causal curve $\alpha:[0,a) \rightarrow M$, what does it mean to say that it will have ``an endpoint on $\partial_{+}M$''?. Given the natural inclusion
% \[
% i:p \in M \hookrightarrow I^-(p) \in \hat{M}
% \]
% whose main properties we have already established (cf. Theorem \ref{thething}), we naturally look at the curve $i\circ \alpha$ in $\hat{M}$. We shall then say, following the standard definition for spacetimes, that $P \in \hat{M}$ is a {\em future endpoint} of $\alpha$ if for any $\hat{U} \ni P$ open set in $\tau_c$, there exists $t_0 \in [0,a)$ such that
% \[
% i\circ \alpha(t) \in \hat{U}, \, \forall t \in [t_0,a).
% \]
% A more convenient characterization is then given as follows.
% \begin{proposition}
% \label{endpointconvenience}
% For a future-directed causal curve $\alpha:[0,a) \rightarrow M$ ($a \leq +\infty$) and $P \in \hat{M}$, the following are equivalent.
% \begin{itemize}
% \item[1)] $P$ is a future endpoint of $\alpha$.
% \item[2)] For any sequence $(t_k)$ in $[0,a)$ converging to $a$, we have $i\circ \alpha(t_k) \rightarrow P$ in $\tau_c$.
% \item[3)] $P = I^-(\alpha)$.
% \end{itemize}
% In particular, if such an endpoint exists, then it is unique, and
% \[
% P \in \partial_+M \Longleftrightarrow \mbox{ $\alpha$ is future-inextendible}.
% \]
% \end{proposition}
% {\em Proof.} The equivalence of $(1)$ and $(2)$ follows immediately from the fact that $\tau_c$ is metrizable, as does the uniqueness of any putative future endpoint. The last claim follows immediately from $(3)$ and Proposition \ref{PipST} (2). Thus, we only need to show the equivalence $(2)\Longleftrightarrow (3)$. Note that this is equivalent to showing, for an arbitrarily given sequence $(t_k)$ in $[0,a)$ converging to $a$, that $\overline{I^-(\alpha)}$ is the Hausdorff closed limit of $I^-(\alpha(t_k))$.

% It is clear that $\limsup (I^-(\alpha(t_k)) \subset \overline{I^-(\alpha)}$. Now, given $x \in \overline{I^-(\alpha)}$, and $U\ni x$ open, pick $x' \in U \cap I^-(x)$ and $t_0 \in [0,a)$ such that $\alpha(t_0) \in U\cap I^+(x') \ni x$. For large enough $k \in \mathbb{N}$, we shall have $t_0 < t_k$, whence
% \[
% x' \ll_g \alpha(t_0) \leq _g \alpha(t_k) \Longrightarrow x'\in I^-(\alpha(t_k)).
% \]
% hence, $I^-(\alpha(t_k))$ intersects $U$ for all large enough $k$, whence we conclude that $x \in \liminf(I^-(\alpha(t_k))$. This concludes the proof.
% \qcd

\begin{definition}
\label{scri}
The {\em future null infinity} of $M$, denoted as ${\cal J}^+$, is the set of points $(P,F) \in \partial  M$ such that
\begin{enumerate}[label=(\roman*)]
\item $\exists$ a future-complete and future-regular null ray $\eta:[0,+\infty) \rightarrow M$ with $(P,F)$ as a future endpoint of $\eta$,

\item $\mbox{every future-inextedible null geodesic with future endpoint $(P,F)$ is fu\-tu\-re-com\-ple\-te.}$
\end{enumerate}
A time-dual analogous definition is immediate for the \emph{past null infinity} ${\cal J}^-$.
\end{definition}

% \cambios{\begin{convention}
%   Along the present section, unless stated otherwise, we will always assume that $(M,g)$ is  \textit{strongly} properly causal Lorentz manifold (see Defn. \ref{def:properly-causal})\footnote{Nota de Jony: Creo que ambos conceptos no son equivalente...REVISAR}.
% \end{convention}}

%\cambiosn{
%Theorems \ref{thm:causaltoconformal}  and \ref{conformalinfinity2} provide the following result:
%\begin{corollary}
%\label{voila}
%Let $(\tilde{M},\tilde{g})$ be a future-nesting conformal extension of a globally hyperbolic spacetime $(M,g)$, and consider the homeomorphism $\Psi:\overline{M}\rightarrow \overline{M}_i^*$ given in Theorem \ref{thm:causaltoconformal} \ref{thmcausaltoconformal-defPsi} taking $\partial M$ onto the accessible conformal boundary $\partial^*_i M$. Then
%\[
%\Psi^{-1}({\cal J}^+_c) \subset {\cal J}^+.
%\]
%\end{corollary}
%\begin{proof}
%  Let $p \in {\cal J}^+_c$. From Theorem \ref{conformalinfinity2} (iii), we know that $p$ is the future endpoint in $\overline{M}_i$ of a future-complete null ray $\eta:[0,+\infty) \rightarrow M$. Let us write $P=I^-(\eta)$ and recall that global hyperbolicity and Thm. \ref{thm:causaltoconformal} \ref{thmcausaltoconformal-defPsi} together give $(P,\emptyset)\in \partial M$ and $\Psi(p)\equiv (P,\emptyset)$. From the definition of chronological limit, it then follows that $(P,\emptyset)$ is the future-endpoint of the curve $\eta$ on $\overline{M}$. Thus, $(P,\emptyset)$ satisfies clause (i) of Definition \ref{scri}. To show that it also satisfies the clause (ii), consider any future-inextendible null geodesic $\alpha:[0,A) \rightarrow M$ ($A\leq +\infty$) such that $I^-(\alpha) = P$. Again, $(P,\emptyset)$ is the future endpoint of $\alpha$ on $\overline{M}$. Since the map in Thm. \ref{thm:causaltoconformal} \ref{thmcausaltoconformal-defPsi} is a homeomorphism, $p$ is also the future endpoint of $\alpha$ on $\overline{M}_i$. But then $\alpha$ must be future-complete by Theorem \ref{conformalinfinity2} (i). Thus, $(P,\emptyset)$ also satisfies clause (ii) as claimed, and the proof is complete.
%\end{proof}
%}

\begin{remark}\label{rem:1}
  \em Condition (i) in the previous definition ensures, in particular, that for any $(P,F)\in \mathcal{J}^{+}$, $P\neq \emptyset$ (recall Proposition \ref{prop:strongfirst}).
\end{remark}
%\begin{remark}
%\label{rmk4}\footnote{\cambiosj{Si no me equivoco, este Remark no se menciona en ningún lado. Creo que lo dejaría como texto fuera de ningún remark (sí necesito el remark anterior).}}
%{\em
Definition \ref{scri} is meant to capture the standard physical notion of ``distant observers'' in the absence of a conformal boundary. This is clearly a {\em geometric} rather than just {\em conformal} notion, as shows up here in the geodesic completeness requirements. It seems that we do not miss the standard points at infinity at least when they are regular and the conformal extension is well-behaved. However, the example in Remark \ref{rmk1} (3) still shows that points at conformal infinity which are {\em not} regular do not need to be in ${\cal J}^+$ (modulo $\Psi$) as defined. Indeed, the point $u=v=\pi/2$ in that example is a point at conformal infinity which is not regular, and the corresponding element in $\partial  M$ is $(M,\emptyset)$ itself (the first component is a TIP in this case!). But $M$ is not the chronological past of any future-directed {\em null} ray (so clause (i) is not satisfied), although it is the past of (infinitely many) future-directed {\em timelike} rays, and clause (ii) in Definition \ref{scri} is trivially satisfied. The reason why we include the clause (i) in Definition \ref{scri} is because we are interested only in the {\em null} infinity here. Clause (ii), on the other hand, seems physically justified if our ``distant observers'' are not to ``see up close'' any (potentially ``fatal''!) ``naked singularity''.
%}
%}
%\end{remark}


%%% Local Variables:
%%% mode: latex
%%% TeX-master: "nullinfinityV5"
%%% End:
