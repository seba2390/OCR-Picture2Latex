	\NeedsTeXFormat{LaTeX2e}
\documentclass[11pt]{article}
% \makeatletter
% \DeclareRobustCommand*\cal{\@fontswitch\relax\mathcal}
% \DeclareRobustCommand*\rm{\@fontswitch\relax\mathrm}
% \DeclareRobustCommand*\it{\@fontswitch\relax\mathit}
% \DeclareRobustCommand*\sc{\@fontswitch\relax\mathsc}
% \makeatother
%\usepackage[brazil]{babel}
%\usepackage[latin1]{inpu{}enc}
\usepackage{amsfonts}
\usepackage{amsthm}
\usepackage{amssymb}
\usepackage[dvips]{graphicx,color}
\usepackage{graphicx}
%\usepackage{enumitem}
\usepackage[margin=1.3in]{geometry}
%\usepackage[all]{xy}
\usepackage{xcolor}
\usepackage[utf8]{inputenc}
\usepackage{amsmath}
\usepackage{indentfirst}
\usepackage[dvips]{graphicx,color}
\usepackage{authblk}
\usepackage{enumitem}
%\usepackage{soul}
% \usepackage{graphicx}
%\usepackage[all]{xy}
%\usepackage{epsfig}
%\usepackage{a4a4}
%\usepackage{psfrag}
%\usepackage{refcheck}
\newcommand{\tauH}{\tau_H}
\newcommand{\tauC}{\tau_{C}}
\newcommand{\calA}{\mathcal{A}}
\newcommand{\calB}{\mathcal{B}}
\newcommand{\calC}{\mathcal{C}}
\newcommand{\calD}{\mathcal{D}}
\newcommand{\calE}{\mathcal{E}}
\newcommand{\calF}{\mathcal{F}}
\newcommand{\calG}{\mathcal{G}}
\newcommand{\calH}{\mathcal{H}}
\newcommand{\calI}{\mathcal{I}}
\newcommand{\calK}{\mathcal{K}}
\newcommand{\calL}{\mathcal{L}}
\newcommand{\calM}{\mathcal{M}}
\newcommand{\calN}{\mathcal{N}}
\newcommand{\calP}{\mathcal{P}}
\newcommand{\calR}{\mathcal{R}}
\newcommand{\calT}{\mathcal{T}}
\newcommand{\Liminf}{{\rm lim\,inf}^o}
\newcommand{\Limsup}{{\rm lim\,sup}^o}
\newcommand{\calU}{\mathcal{U}}
\newcommand{\Z}{\mathbb{Z}}
\newcommand{\C}{\mathbb{C}}
\newcommand{\T}{\mathbb{T}}
\newcommand{\R}{\mathbb{R}}
\newcommand{\fun}{{\cal F}}
\newcommand{\paral}{{\cal X}}
\newcommand{\killi}{{\cal Y}}
\newcommand{\cambios}[1]{#1}
\newcommand{\cambiosn}[1]{\textcolor{red}{#1}}
\newcommand{\cambiosj}[1]{\textcolor{blue}{#1}}
\newcommand{\qcd}{\begin{flushright} $\Box$ \end{flushright}}
\newcommand{\Cl}[1]{\overline{#1}^{\overline{M}}}
\newcommand{\ClM}[1]{\overline{#1}^{M}}
\newcommand{\Lc}{L_{chr}}
\def\be{\begin{eqnarray}}
\def\ee{\end{eqnarray}}
\def\beq{\begin{equation}}
\def\eeq{\end{equation}}
\def\e{\epsilon}
\def\l{\lambda}
\def\g{\gamma}
\def\p{\partial}
\def\t{\tilde}
\def\G{\Gamma}
\def\N{{\cal N}}
\def\R{{\cal R}}
\def\RR{{\bf R^3}}
\def\({\left (}
\def\){\right )}
\def\S{{\cal S}}
\def\pr{{(\phi_{,r})}}
\def\tr{\mathrm{tr}}
\def\ra{\rightarrow}
\graphicspath{{Figuras/}}
\newtheorem{theorem}{Theorem}[section]
\newtheorem{lemma}[theorem]{Lemma}
\newtheorem*{note}{Note}
\newtheorem{proposition}[theorem]{Proposition}
\newtheorem{corollary}[theorem]{Corollary}
\newtheorem{definition}[theorem]{Definition}
\newtheorem{remark}[theorem]{Remark}
\newtheorem{convention}[theorem]{Convention}
\newtheorem{example}[theorem]{Example}
\newtheorem{conjecture}[theorem]{Conjecture}
\title{A novel notion of null infinity for c-boundaries and generalized black holes}
\author[1]{I.P. Costa e Silva}
%\email{pontual.ivan@ufsc.br}
%\address{Department of Mathematics,\\
%University of Miami, Coral Gables, FL 33124, USA.}
\author[2]{J.L. Flores}
%\email{floresj@uma.es}
%\address{Departamento de \'Algebra, Geometr\'{i}a y Topolog\'{i}a,\\ Facultad de Ciencias, Universidad de M\'alaga \\ Campus Teatinos, 29071 M\'alaga, Spain}
\author[3]{J. Herrera}
%\address{Department of Mathematics,
%Universidade Federal de Santa Catarina, 88.040-900 Florian\'{o}polis-SC, Brazil.}
%\address{Universidade Federal de Santa Catarina, Brazil}
%\address{Universidad de M\'alaga, Spain}
%\date{March 16th, 2016}
%\newcommand{\qed}{\nobreak \ifvmode \relax \else
%	\ifdim\lastskip<1.5em \hskip-\lastskip
%	\hskip1.5em plus0em minus0.5em \fi \nobreak
%	\vrule height0.75em width0.5em depth0.25em\fi}
%\newcommand{\qcd}{\begin{flushright} $\Box$ \end{flushright}}
%\usepackage[cp850]{inputenc}
%\usepackage{amsfonts,a4a4}
\affil[1]{\small{{\textit{Department of Mathematics\\
Universidade Federal de Santa Catarina, 88.040-900 Florian\'{o}polis-SC, Brazil.}}}}
\affil[2]{\small{\textit{Departamento de \'Algebra, Geometr\'{i}a y Topolog\'{i}a\\ Facultad de Ciencias, Universidad de M\'alaga\\ Campus Teatinos, 29071 M\'alaga, Spain.}}}
\affil[3]{\small{\textit{Departamento de Matemáticas, Edificio Albert Einstein, Universidad de Córdoba\\ Campus de Rabanales, 14071 Córdoba, Spain}}}
\begin{document}

\maketitle

	\begin{abstract}
\noindent
%We show that when a spacetime $M$ is globally hyperbolic, a {\em metrizable} topology, the {\em closed limit topology} (CLT) introduced by F. Hausdorff himself in the 1950's in set theory can be advantageously adopted on the Geroch-Kronheimer-Penrose causal completion (or $c$-completion) of $M$, retaining essentially all the good properties of the so-called {\em chronological topology}. The latter topology was introduced by one of us (J.L.F.) in \cite{Florescausalboundaryspacetimes2007} and its many desirable properties were described in detail in Ref. \cite{Floresfinaldefinitioncausal2011}. However, with respect to separability properties, the chronological topology is generally only $T_1$, but not $T_2$ (i.e., Hausdorff), even for globally hyperbolic spacetimes. For $c$-completions of the latter class, the CLT is shown to be strictly finer than the chronological topology, and coincide with it if and only if the chronological topology is $T_2$. We discuss how the usefulness of the CLT in the context of (future) $c$-completions of globally hyperbolic spacetimes stems from a simple but key observation made in a recent paper by G. Galloway and C. Vega \cite{GallowayHausdorffClosedLimits2017}: achronal boundaries are preserved by Hausdorff closed limits. As a mandatSory test of CLT's usefulness, we show that if the globally hyperbolic spacetime $M$ admits a {\em conformal} boundary, defined in such broad terms as to include all the standard examples in the literature, then the latter is {\em homeomorphic} to the causal boundary endowed with the CLT. In another application,
  %After reviewing the relationship between conformal extensions of spacetimes and their causal boundaries,
  We give new definitions of null infinity and black hole in terms of causal boundaries, applicable to any strongly causal spacetime $(M,g)$. These are meant to extend the standard ones given in terms of conformal boundaries, and use the new definitions to prove a classic result in black hole theory for this more general context: if the null infinity is {\em regular} (i.e. well behaved in a suitable sense) and $(M,g)$ obeys the null convergence condition, then any closed trapped surface in $(M,g)$ has to be inside the black hole region. As an illustration of this general construction, we apply it to the class of {\em generalized plane waves}, where the conformal null infinity is not always well-defined. In particular, it is shown that (generalized) black hole regions do {\em not} exist in a large family of these spacetimes.
%Since its introduction by Geroch, Kronheimer and Penrose in 1972, the causal boundary of a given strongly causal spacetime $M$ has asserted itself as a method for attaching ideal points to $M$ which, although somewhat more abstract than conformal boundaries, is (unlike the latter) deeply ingrained in the very causal structure of $M$, and hence it is, in a strong sense, an inescapable feature of the (conformally invariant aspects of the) spacetime geometry. However, this remark only applies to causal boundaries as sets endowed with a chronological relation (in the sense of \cite{Harris1}) extending that of $M$, {\em not} to its topology: indeed, there are a number of inequivalent ``good'' choices for the topology of the so-called $c$-completion (i.e., $M$ together with its attached causal boundary). Arguably, one of the most natural choices is the so-called {\em chronological topology}, introduced by one of us (J.L.F.) in \cite{JoseChronological} and its many desirable properties are described in detail in Ref. \cite{Final}. However, with respect to separability properties, these topologies are generally only $T_1$, but not $T_2$ (i.e., Hausdorff), even for globally hyperbolic spacetimes. Nevertheless, in this work we show that specifically when $M$ is globally hyperbolic, a quite natural alternative, {\em metrizable} topology, the {\em closed limit topology} (CLT) introduced by F. Hausdorff himself in the 1950's in more general set-theoretic contexts can be advantageously adopted with on the $c$-completion, retaining essentially all the good properties of the chronological topology. In particular, CLT is shown to be strictly finer than the chronological topology, and coincide with the latter if and only if it is $T_2$. We discuss how the usefulness of the CLT in the context of $c$-completions of spacetimes is the direct consequence of a simple but key observation made in a recent paper by G. Galloway and C. Vega \cite{GV}: achronal boundaries are preserved by Hausdorff closed limits. We also shown that when the globally hyperbolic spacetime $M$ admits a {\em conformal} boundary, defined in a fairly broad context which includes all the standard examples in the literature, the latter is homeomorphic to the causal boundary with the CLT. The latter result
		
\end{abstract}

\IEEEraisesectionheading{\section{Introduction}}

\IEEEPARstart{V}{ision} system is studied in orthogonal disciplines spanning from neurophysiology and psychophysics to computer science all with uniform objective: understand the vision system and develop it into an integrated theory of vision. In general, vision or visual perception is the ability of information acquisition from environment, and it's interpretation. According to Gestalt theory, visual elements are perceived as patterns of wholes rather than the sum of constituent parts~\cite{koffka2013principles}. The Gestalt theory through \textit{emergence}, \textit{invariance}, \textit{multistability}, and \textit{reification} properties (aka Gestalt principles), describes how vision recognizes an object as a \textit{whole} from constituent parts. There is an increasing interested to model the cognitive aptitude of visual perception; however, the process is challenging. In the following, a challenge (as an example) per object and motion perception is discussed. 



\subsection{Why do things look as they do?}
In addition to Gestalt principles, an object is characterized with its spatial parameters and material properties. Despite of the novel approaches proposed for material recognition (e.g.,~\cite{sharan2013recognizing}), objects tend to get the attention. Leveraging on an object's spatial properties, material, illumination, and background; the mapping from real world 3D patterns (distal stimulus) to 2D patterns onto retina (proximal stimulus) is many-to-one non-uniquely-invertible mapping~\cite{dicarlo2007untangling,horn1986robot}. There have been novel biology-driven studies for constructing computational models to emulate anatomy and physiology of the brain for real world object recognition (e.g.,~\cite{lowe2004distinctive,serre2007robust,zhang2006svm}), and some studies lead to impressive accuracy. For instance, testing such computational models on gold standard controlled shape sets such as Caltech101 and Caltech256, some methods resulted $<$60\% true-positives~\cite{zhang2006svm,lazebnik2006beyond,mutch2006multiclass,wang2006using}. However, Pinto et al.~\cite{pinto2008real} raised a caution against the pervasiveness of such shape sets by highlighting the unsystematic variations in objects features such as spatial aspects, both between and within object categories. For instance, using a V1-like model (a neuroscientist's null model) with two categories of systematically variant objects, a rapid derogate of performance to 50\% (chance level) is observed~\cite{zhang2006svm}. This observation accentuates the challenges that the infinite number of 2D shapes casted on retina from 3D objects introduces to object recognition. 

Material recognition of an object requires in-depth features to be determined. A mineralogist may describe the luster (i.e., optical quality of the surface) with a vocabulary like greasy, pearly, vitreous, resinous or submetallic; he may describe rocks and minerals with their typical forms such as acicular, dendritic, porous, nodular, or oolitic. We perceive materials from early age even though many of us lack such a rich visual vocabulary as formalized as the mineralogists~\cite{adelson2001seeing}. However, methodizing material perception can be far from trivial. For instance, consider a chrome sphere with every pixel having a correspondence in the environment; hence, the material of the sphere is hidden and shall be inferred implicitly~\cite{shafer2000color,adelson2001seeing}. Therefore, considering object material, object recognition requires surface reflectance, various light sources, and observer's point-of-view to be taken into consideration.


\subsection{What went where?}
Motion is an important aspect in interpreting the interaction with subjects, making the visual perception of movement a critical cognitive ability that helps us with complex tasks such as discriminating moving objects from background, or depth perception by motion parallax. Cognitive susceptibility enables the inference of 2D/3D motion from a sequence of 2D shapes (e.g., movies~\cite{niyogi1994analyzing,little1998recognizing,hayfron2003automatic}), or from a single image frame (e.g., the pose of an athlete runner~\cite{wang2013learning,ramanan2006learning}). However, its challenging to model the susceptibility because of many-to-one relation between distal and proximal stimulus, which makes the local measurements of proximal stimulus inadequate to reason the proper global interpretation. One of the various challenges is called \textit{motion correspondence problem}~\cite{attneave1974apparent,ullman1979interpretation,ramachandran1986perception,dawson1991and}, which refers to recognition of any individual component of proximal stimulus in frame-1 and another component in frame-2 as constituting different glimpses of the same moving component. If one-to-one mapping is intended, $n!$ correspondence matches between $n$ components of two frames exist, which is increased to $2^n$  for one-to-any mappings. To address the challenge, Ullman~\cite{ullman1979interpretation} proposed a method based on nearest neighbor principle, and Dawson~\cite{dawson1991and} introduced an auto associative network model. Dawson's network model~\cite{dawson1991and} iteratively modifies the activation pattern of local measurements to achieve a stable global interpretation. In general, his model applies three constraints as it follows
\begin{inlinelist}
	\item \textit{nearest neighbor principle} (shorter motion correspondence matches are assigned lower costs)
	\item \textit{relative velocity principle} (differences between two motion correspondence matches)
	\item \textit{element integrity principle} (physical coherence of surfaces)
\end{inlinelist}.
According to experimental evaluations (e.g.,~\cite{ullman1979interpretation,ramachandran1986perception,cutting1982minimum}), these three constraints are the aspects of how human visual system solves the motion correspondence problem. Eom et al.~\cite{eom2012heuristic} tackled the motion correspondence problem by considering the relative velocity and the element integrity principles. They studied one-to-any mapping between elements of corresponding fuzzy clusters of two consecutive frames. They have obtained a ranked list of all possible mappings by performing a state-space search. 



\subsection{How a stimuli is recognized in the environment?}

Human subjects are often able to recognize a 3D object from its 2D projections in different orientations~\cite{bartoshuk1960mental}. A common hypothesis for this \textit{spatial ability} is that, an object is represented in memory in its canonical orientation, and a \textit{mental rotation} transformation is applied on the input image, and the transformed image is compared with the object in its canonical orientation~\cite{bartoshuk1960mental}. The time to determine whether two projections portray the same 3D object
\begin{inlinelist}
	\item increase linearly with respect to the angular disparity~\cite{bartoshuk1960mental,cooperau1973time,cooper1976demonstration}
	\item is independent from the complexity of the 3D object~\cite{cooper1973chronometric}
\end{inlinelist}.
Shepard and Metzler~\cite{shepard1971mental} interpreted this finding as it follows: \textit{human subjects mentally rotate one portray at a constant speed until it is aligned with the other portray.}



\subsection{State of the Art}

The linear mapping transformation determination between two objects is generalized as determining optimal linear transformation matrix for a set of observed vectors, which is first proposed by Grace Wahba in 1965~\cite{wahba1965least} as it follows. 
\textit{Given two sets of $n$ points $\{v_1, v_2, \dots v_n\}$, and $\{v_1^*, v_2^* \dots v_n^*\}$, where $n \geq 2$, find the rotation matrix $M$ (i.e., the orthogonal matrix with determinant +1) which brings the first set into the best least squares coincidence with the second. That is, find $M$ matrix which minimizes}
\begin{equation}
	\sum_{j=1}^{n} \vert v_j^* - Mv_j \vert^2
\end{equation}

Multiple solutions for the \textit{Wahba's problem} have been published, such as Paul Davenport's q-method. Some notable algorithms after Davenport's q-method were published; of that QUaternion ESTimator (QU\-EST)~\cite{shuster2012three}, Fast Optimal Attitude Matrix \-(FOAM)~\cite{markley1993attitude} and Slower Optimal Matrix Algorithm (SOMA)~\cite{markley1993attitude}, and singular value decomposition (SVD) based algorithms, such as Markley’s SVD-based method~\cite{markley1988attitude}. 

In statistical shape analysis, the linear mapping transformation determination challenge is studied as Procrustes problem. Procrustes analysis finds a transformation matrix that maps two input shapes closest possible on each other. Solutions for Procrustes problem are reviewed in~\cite{gower2004procrustes,viklands2006algorithms}. For orthogonal Procrustes problem, Wolfgang Kabsch proposed a SVD-based method~\cite{kabsch1976solution} by minimizing the root mean squared deviation of two input sets when the determinant of rotation matrix is $1$. In addition to Kabsch’s partial Procrustes superimposition (covers translation and rotation), other full Procrustes superimpositions (covers translation, uniform scaling, rotation/reflection) have been proposed~\cite{gower2004procrustes,viklands2006algorithms}. The determination of optimal linear mapping transformation matrix using different approaches of Procrustes analysis has wide range of applications, spanning from forging human hand mimics in anthropomorphic robotic hand~\cite{xu2012design}, to the assessment of two-dimensional perimeter spread models such as fire~\cite{duff2012procrustes}, and the analysis of MRI scans in brain morphology studies~\cite{martin2013correlation}.

\subsection{Our Contribution}

The present study methodizes the aforementioned mentioned cognitive susceptibilities into a cognitive-driven linear mapping transformation determination algorithm. The method leverages on mental rotation cognitive stages~\cite{johnson1990speed} which are defined as it follows
\begin{inlinelist}
	\item a mental image of the object is created
	\item object is mentally rotated until a comparison is made
	\item objects are assessed whether they are the same
	\item the decision is reported
\end{inlinelist}.
Accordingly, the proposed method creates hierarchical abstractions of shapes~\cite{greene2009briefest} with increasing level of details~\cite{konkle2010scene}. The abstractions are presented in a vector space. A graph of linear transformations is created by circular-shift permutations (i.e., rotation superimposition) of vectors. The graph is then hierarchically traversed for closest mapping linear transformation determination. 

Despite of numerous novel algorithms to calculate linear mapping transformation, such as those proposed for Procrustes analysis, the novelty of the presented method is being a cognitive-driven approach. This method augments promising discoveries on motion/object perception into a linear mapping transformation determination algorithm.




\section{Preliminaries: the c-boundary construction}\label{prelim}

Throughout this paper, $(M,g)$ will denote a {\it spacetime}, i.e. a connected, time-oriented $C^{\infty}$ Lorentzian manifold $(M,g)$. We shall assume that the reader is familiar with standard facts in Lorentzian geometry and causal theory as given in the basic references \cite{BeemGlobalLorentzianGeometry1996,HawkingLargeScaleStructure1975,ONeillSemiRiemannianGeometryApplications1983}.
%with signature
%$(-,+,\ldots,+)$.
%A  tangent vector $v\in T_{p}M$ on a point $p\in M$  is
%named {\em timelike} (resp. {\em lightlike}; {\em causal}) if
%$g(v,v)<0$ (resp. $g(v,v)=0$, $v\neq 0$; $v$ is either timelike or
%lightlike). Accordingly, a smooth curve $\gamma:I\rightarrow M$
%($I$ real interval) is called {\em timelike} (resp. {\em
%lightlike}; {\em causal}) if its tangent vector $\dot{\gamma}(s)$ is timelike (resp.
%lightlike; causal) for all $s$. We will assume along the paper that all spacetimes are  {\em
%time-oriented}, i.e. they are endowed with a continuous, globally
%defined, timelike vector field $X$. A {\em time-orientation}
%$X$ distributes the causal tangent vectors $v\in T_{p}M$ in two
%cones, each one containing future $g(v,X(p))<0$ or past-directed
%$g(v,X(p))>0$ causal vectors. So, a causal curve $\gamma(s)$ is
%said {\em future-directed} (resp. {\em past-directed}) if
%$g(\dot{\gamma}(s),X(\gamma(s)))<0$ (resp.
%$g(\dot{\gamma}(s),X(\gamma(s)))>0$) for all $s$.

%Timelike and causal curves define the so-called conformal structure of the spacetime $(M,g)$ given by the following two relations:
%given two points $p,q\in M$, we will say that they are {\em chronologically related}, denoted by $p\ll q$
%(resp. {\em causally related}, $p\leq q$) if there exists some
%future-directed timelike (resp. causal) curve from $p$ to $q$ (the
%case $p=q$ is allowed when $p\leq q$). The {\em
%chronological past} (resp. {\em future}) of $p$, $I^{-}(p)$ (resp. $I^{+}(p)$)
%is defined as:
%\[
%I^{-}(p)=\{q\in M: q\ll p\}\qquad(I^{+}(p)=\{q\in M: p\ll q\}).
%\]

%A spacetime is  {\em distinguishing} if any of its points is
%characterized by its past and future, {\em strongly causal} if it
%does not admit neither closed nor ``almost closed'' causal curves,
%and {\em globally hyperbolic} if it admits a {\em Cauchy
%hypersurface}, i.e. a topological hypersurface that is met exactly
%once by every inextendible timelike curve. Here, global
%hyperbolicity is the most restrictive causality condition, while
%distinguishing is the most general one.

The causal completion of $(M,g)$ is by now a standard construction put forth for the first time in a seminal paper by Geroch, Kronheimer and Penrose \cite{GerochIdealPointsSpaceTime1972}. The underlying idea is simple enough: to add \textit{ideal points} to the original spacetime, comprising the so called \textit{causal completion} (or \textit{$c$-completion} for short), in such a way that any inextendible timelike curve in $(M,g)$ has endpoints. The set of these ideal endpoints form the associated {\em causal boundary}, or {\em $c$-boundary} for short. This construction is in addition conformally invariant in the sense that two spacetimes in the same conformal class have identical $c$-boundaries. Finally, it is applicable to any strongly causal spacetime, independently of their asymptotic properties.

For the convenience of the reader and to set up the notation we shall use throughout, we briefly review the c-boundary construction here, referring to \cite{GerochIdealPointsSpaceTime1972} and \cite{Floresfinaldefinitioncausal2011} for further details and proofs. Let us start with some basic
notions.

A non-empty subset $P\subset M$ is called a {\em past set} if it
coincides with its chronological past, i.e. $P=I^{-}(P)$. (In particular, every past set is open.) A past set that cannot be written as the
union of two proper subsets, both of which are also past sets, is
said to be an {\em indecomposable past} set (IP). It can be shown that an IP either coincides with the past of some point of the spacetime, i.e., $P=I^{-}(p)$ for $p\in M$, or else $P=I^{-}(\gamma)$ for some inextendible future-directed
timelike curve $\gamma$. In the former case, $P$ is said to be a {\em proper indecomposable past
set} (PIP), and in the latter case $P$ is said to be a {\em
terminal indecomposable past set} (TIP). These two classes of IPs are disjoint.

Another useful technical definition is the following. The {\em common past} of a given set $S\subset
M$ is defined by \[\downarrow S:=I^{-}(\{p\in M:\;\; p\ll
q\;\;\forall q\in S\}).\]
The corresponding
definitions for {\em future sets}, IFs, TIFs,
PIFs, {\em common future}, etc., are obtained just by interchanging the roles of past and
future, and will always be understood.

The set of all IPs constitutes the so-called {\it future $c$-completion} of $(M,g)$, denoted by $\hat{M}$. If $(M,g)$ is strongly causal, then $M$ can naturally be viewed as a subset of $\hat{M}$ by identifying every point $p\in M$ with its respective PIP, namely $I^-(p)$. Indeed, it is well-known \cite{BeemGlobalLorentzianGeometry1996} that a strongly causal $(M,g)$ is {\em distinguishing}, i.e.,
\[
\forall p,q \in M, \, I^{\pm}(p)=I^{\pm}(q) \Longrightarrow p=q,
\]
so that the inclusion $i: p\in M \hookrightarrow I^-(p) \in \hat{M}$ is indeed one-to-one.

{\em Throughout this paper, unless otherwise explicitly stated, we shall assume that $(M,g)$ is strongly causal.}

The {\it future $c$-boundary} $\hat{\partial} M$ of $(M,g)$ is defined as the set of all its TIPs. Therefore, upon identifying $M$ with its image in $\hat{M}$ by the natural inclusion as outlined above,
\[
\hat{\partial} M \equiv \hat{M} \setminus M.
\]
The definitions of \textit{past $c$-completion} $\check{M}$ and {\it past $c$-boundary} $\check{\partial}M$ of $(M,g)$ are readily defined in a time-dual fashion using IFs.

Now, one would expect that the {\em (total) $c$-completion} of $(M,g)$ could be obtained by ``joining together'' in some suitable sense the past and future $c$-completions. However, naive attempts to do so meet with some surprisingly thorny technical issues (again, see \cite{Floresfinaldefinitioncausal2011} and references therein for a discussion). To circumvent these issues, certain clever manipulations are required which were first carried out by Szabados \cite{SzabadosCausalboundarystrongly1988, SzabadosCausalboundarystrongly1989}, with important improvements by Marolf and Ross \cite{Marolfnewrecipecausal2003}.

First, we introduce the so-called {\em Szabados relation} (or \textit{$S$-relation}) between IPs and IFs: an IP $P$ and an IF $F$ are {\em S-related}, $P\sim_{S}F$, if $P$ is a maximal IP inside
$\downarrow F$ and $F$ is a maximal IF inside $\uparrow P$. In particular, for any $p \in M$, it can be shown that $I^-(p) \sim_{S} I^+(p)$. Then, we have the following

\begin{definition}\label{d1} The {\em (total) c-completion} $\overline{M}$ is
formed by all
the pairs $(P,F)$ formed by $P\in \hat{M}\cup\{\emptyset\}$ and
$F\in \check{M}\cup\{\emptyset\}$ such that either
\begin{itemize}
\item[i)] both $P$ and $F$ are non-empty and $P\sim_{S}F$; or
\item[ii)] $P=\emptyset$, $F \neq \emptyset$ and there is no $P'\neq \emptyset$ such that $P'\sim_{S}F$; or
\item[iii)] $F=\emptyset$, $P \neq \emptyset$ and there is no $F'\neq\emptyset$ such that $P\sim_{S}F'$.
\end{itemize}
The original manifold $M$ is then identified with the set $\{(I^{-}(p),I^{+}(p)): p\in M\}$, and the {\em c-boundary} is defined as $\partial M\equiv\overline{M}\setminus M$.
\end{definition}
\begin{remark}\label{r0} \emph{We will systematically use the following fact, which is easy to check: given any IP $P \neq \emptyset$ (resp. any IF $F \neq \emptyset$), there always exists $F \in \check{M}\cup\{\emptyset\}$ (resp. $P \in \hat{M}\cup\{\emptyset\}$) such that $(P,F) \in \overline{M}$.
}
\end{remark}

Having defined the set structure of the $c$-completion, the next step is to extend the chronological relation in $(M,g)$ to the $c$-completion as follows:

\begin{equation}
(P,F)\ll
(P',F')\;\;\iff\;\; F\cap P'\neq\emptyset. \label{eq:7}
\end{equation}

Moreover, two pairs $(P,F)$,
$(P',F')$ are {\em causally related}, denoted by $(P,F)\leq (P',F')$, if
$F'\subset F$ and $P\subset P'$. Finally, these pairs said to be {\em
horismotically related} if they are causally, but not
chronologically related\footnote{\label{fn:1}These notions for causal and horismotical relations are not totally satifactory in general (see for instance \cite[Section 3.2]{Marolfnewrecipecausal2003} for more discussion on this issue). However, under the additional hypotheses we will assume later in this paper, they can be adopted without problems.}.

In concrete applications, it is also important to introduce a suitable topology on the $c$-completion $\overline{M}$.
%\cambiosn{Unfortunately, there is no consensus in the literature on how this should be done.}
The original topology considered in \cite{GerochIdealPointsSpaceTime1972}, although Hausdorff, was plagued by a number of technical problems and failed to yield sensible results even in simple cases. (An extensive discussion of these issues with further pertinent references on alternative proposed topologies on the $c$-completion can be found in Ref. \cite{Floresfinaldefinitioncausal2011}.) Our choice here, for reasons which are exhaustively discussed in \cite{Floresfinaldefinitioncausal2011}, is the so-called {\em chronological topology} on the $c$-completion.

The chronological topology is more conveniently defined by means of a so-called {\em limit operator} (see for instance \cite{FloresHausdorffseparabilityboundaries2016}). Let us briefly recall this notion here. Let $X$ be any set. Denote by ${\cal S}_X$ the set of all (infinite) sequences in $X$ (including their subsequences), and by $\mathbb{P}(X)$ the power set of all subsets of $X$. A {\em limit operator} on $X$ is any mapping $L:{\cal S}_X \rightarrow \mathbb{P}(X)$ such that if $\sigma \in {\cal S}_X$ is a sequence in $X$ and $\sigma '$ is a subsequence of $\sigma$, then $L(\sigma) \subset L(\sigma ')$. If $L$ is one such limit operator, the associated {\em derived topology} $\tau _{L}$ is defined via its closed sets: by definition, a subset $C \subset X$ is closed in $\tau _L$ if and only if $L(\sigma) \subset C$ for any sequence $\sigma$ of elements of $C$. The following facts about the topological space $(X,\tau _L)$ thus defined are readily verified.
\begin{itemize}
\item[LO1)] If $x \in L(\sigma)$ for a sequence $\sigma \in {\cal S}_X$, then $\sigma$ converges to $x$ with respect to $\tau_L$.
\item[LO2)] The topological space $(X,\tau _L)$ is {\em sequential}, i.e., a set $C \subset X$ is closed if and only if it is sequentially closed therein.
\end{itemize}
A limit operator $L$ is said of {\em first order} if the converse of (LO1) also holds, i.e. if the following equivalence is satisfied:
  \begin{equation}
x\in L(\sigma)\iff\sigma \hbox{ converges to $x$ with respect to $\tau_{L}$}.\label{eq:8}
\end{equation}


Recall also that with any sequence $\{A_n\}_{n \in \mathbb{N}}$ of subsets of $X$ we can associate the {\em Hausdorff inferior and superior limits} of sets as
\begin{eqnarray}
\mathrm{LI}(A_{n})\equiv
\liminf(A_{n})&:=&\cup_{n=1}^{\infty}\cap_{k=n}^{\infty}A_{k} \\
\mathrm{LS}(A_{n})\equiv
\limsup(A_{n})&:=&\cap_{n=1}^{\infty}\cup_{k=n}^{\infty}A_{k}.
\end{eqnarray}
Clearly one always has $\mathrm{LI}(A_{n}) \subset \mathrm{LS}(A_{n})$. Moreover, if $\{A_m\}$ is any subsequence,
\[
\mathrm{LI}(A_{n}) \subset \mathrm{LI}(A_{m}) \subset \mathrm{LS}(A_{m}) \subset \mathrm{LS}(A_{n}),
\]
Simple examples show that these inclusions are usually strict.

Next, we define the {\em future chronological limit operator} $\hat{L}$ on $\hat{M}$ as follows. Given a sequence $\sigma=\{P_{n}\}_{n}\subset \hat{M}$ of IPs and $P\in \hat{M}$, we set

\begin{equation}
\label{eq:3}
P\in \hat{L}(\sigma)\iff \left\{\begin{array}{l}
                                 P\subset \mathrm{LI}(\sigma)\\
                                 P \hbox{ is a maximal IP in }\mathrm{LS}(\sigma).
\end{array}\right.
\end{equation}
Again, by simply interchanging past and future sets we may analogously define the {\it past chronological limit operator} $\check{L}$ on $\check{M}$. Then, the {\em future (resp. past) chronological topology on $\hat{M}$ (resp. $\check{M}$)} is the derived topology associated to the limit operator $\hat{L}$ (resp. $\check{L}$).

\smallskip

We are now ready to define the chronological topology on the full $c$-boundary. To this end, define a limit operator $L$ on $\overline{M}$ as follows: given a sequence
$\sigma=\{(P_{n},F_{n})\}\subset\overline{M}$, put

\begin{equation}
\label{eq:4}
(P,F)\in L(\sigma)\iff \left\{
  \begin{array}{l}
    P\in \hat{L}(P_{n}) \hbox{ if $P\neq \emptyset$}\\
F\in\check{L}(F_n) \hbox{ if $F\neq \emptyset$}.
  \end{array}
\right.
\end{equation}
By definition, the {\em chronological topology on $\overline{M}$} is the derived topology $\tau_L$ associated to the limit operator $L$ defined in (\ref{eq:4}).

\begin{remark}\emph{
    In general, the limit operator defined in (\ref{eq:4}) is not of first order (see, for instance, \cite[Figure 7]{FloresHausdorffseparabilityboundaries2016}). However, this property can be proven to hold under some mild and general hypotheses \cite{FloresGromovCauchycausal2013, AkeSpacetimecoveringscasual2017}, valid in most cases of physical interest. Accordingly, throughout this paper we will implicitly assume that this property holds, and so, the equivalence (\ref{eq:8}) will be systematically used.
}
\end{remark}


The following result, whose proof is given in \cite{Floresfinaldefinitioncausal2011}, summarizes the key properties of the chronological topology.

\begin{theorem}
\label{thm:mainc-completion}
Let $(M,g)$ be a strongly causal spacetime and consider its associated $c$-completion $\overline{M}$ endowed with the chronological relations and chronological topology defined in (\ref{eq:7}) and (\ref{eq:4}), respectively. Then, the following statements hold.

\begin{enumerate}[label=(\roman*)]
\item \label{item:mismatopologia}The inclusion $M\hookrightarrow \overline{M}$ is continuous, with an open dense image. In particular, the topology induced on $M$ by the chronological topology on $\overline{M}$ coincides with the original, manifold topology.
\item The chronological topology is second-countable and $T_1$-separable, but not necessarily Hausdorff.
\item \label{item:chains} Let $\{x_n\}\subset M$ be a {\em future (resp. past) chain}, i.e., a sequence satisfying that $x_n\ll x_{n+1}$ (resp. $x_{n+1}\ll x_{n}$) for all $n$. Then,
\[
\begin{array}{c}
L(\left\{ x_{n} \right\})=\left\{ (P,F)\in \overline{M}: P=I^-(\left\{ x_n \right\}) \right\}
\\
\hbox{(resp. $L(\left\{ x_{n} \right\})=\left\{ (P,F)\in \overline{M}: F=I^+(\left\{ x_n \right\}) \right\})$.}
\end{array}
\]
%\footnote{\cambiosn{NOTA IVAN:FIJAOS QUE EL TERMINO ``LIMIT POINT'' YA TIENE UN SIGNIFICADO EN TOPOLOGI�A QUE NOS ES %EXACTAMENTE LO QUE QUEREMOS. HIZE UNOS CAMBIOS PARA ELIMINAR EL TERMINO Y A VER SI NO METI LA PATA.}}

\item \label{item:c-completioncompleta} The c-completion is {\em complete} in the following sense: given any (future or past) chain $\{x_n\}\subset M$, necessarily $L(\left\{ x_{n} \right\}) \neq \emptyset$, i.e. any (future or past) chain converges in $\overline{M}$ (cf. (\ref{eq:8})).
    %\cambiosn{In particular, any inextendible timelike curve $\gamma$ on $M$ has two endpoints in $\overline{M}$.}

\item The sets $I^{\pm}((P,F),\overline{M})$ are open for all $(P,F)\in \overline{M}$.
\end{enumerate}

 \end{theorem}

 Let $\gamma:[a,b) \rightarrow M$ be an arbitrary future(resp. past)-directed causal curve. We want to extend the usual notion of {\em future (resp. past) endpoint} of $\gamma$ to points on $\overline{M}$. Concretely, we say that a pair $(P,F) \in \overline{M}$ is a {\em future (resp. past) endpoint} of $\gamma$ if for any increasing sequence $\left\{ t_{n} \right\} \subset [a,b)$ with $t_{n}\nearrow b$, we have $(P,F) \in L(\left\{ \gamma (t_{n}) \right\})$.

 The following immediate consequence of Theorem \ref{thm:mainc-completion} describes when a point $(P,F)\in\overline{M}$ is an endpoint of a causal curve, and will be very important in later sections. We shall often deal only with future endpoints, since the corresponding statements for past endpoints are easy to obtain from time-duality and will always be understood.

\begin{corollary}
\label{cor:endpoints}
Suppose $(M,g)$ is strongly causal spacetime with $c$-completion $\overline{M}$ as above. Let $\gamma:[a,b) \rightarrow M$ be a future-directed causal curve in $M$ and $(P,F) \in \overline{M}$. Then, the following statements hold.
\begin{enumerate}[label=(\roman*)]
\item \label{item:endpoints1}If $P=I^-(p)$ and $F=I^+(p)$ for some $p \in M$, i.e., if $(P,F)$ is a point of $M$ via its natural inclusion, then $(P,F)$ is a future endpoint of $\gamma$ (in the extended sense) if, and only if, for any increasing sequence $\left\{ t_{n} \right\} \subset [a,b)$ with $t_{n}\nearrow b$, we have $\gamma (t_{n}) \rightarrow p$ in $M$. (In order words, $(P,F)$ is a future endpoint of $\gamma$ in the extended sense iff $p$ is an endpoint in the ordinary, spacetime sense.)
\item \label{item:endpoints2} If $\gamma$ is a future (resp. past) directed {\em timelike} curve, then
\begin{equation}
\label{keyeq1}
(P,F)\in\overline{M} \mbox{ is a future endpoint of $\gamma$} \Longleftrightarrow P=I^-(\gamma).
\end{equation}
Moreover, when $\gamma$ is timelike, then it always has some endpoint on $\overline{M}$.
\item \label{item:endpoint3}If the future-directed causal curve $\gamma$ has $(P,F)$ as future endpoint with $P\neq \emptyset$, then $P=I^{-}(\gamma)$.
\end{enumerate}
\end{corollary}

\begin{proof}
$(i)$ By definition, $(P,F)$ is a future endpoint of $\gamma$ (in the extended sense) if, and only if, for any increasing sequence $\left\{ t_{n} \right\} \subset [a,b)$ with $t_{n}\nearrow b$,
\[
(P,F) \in L(\left\{ \gamma (t_{n}) \right\}) \Longleftrightarrow (I^-(\gamma (t_{n})),I^+(\gamma (t_{n}))) \rightarrow (P,F) \mbox{ in $\overline{M}$}\Longleftrightarrow \gamma (t_{n}) \rightarrow p \mbox{ in $M$},
\]
where we have used the definition of endpoint in the extended sense for the first equivalence, the fact that the limit operator is first-order (cf. (\ref{eq:8})) to obtain the second equivalence, and the fact that the induced topology on $M$ is the manifold topology according to item \ref{item:mismatopologia} of Thm. \ref{thm:mainc-completion} to get the third equivalence.
\\
$(ii)$ If $\gamma$ is timelike, then for any increasing sequence $\left\{ t_{n} \right\} \subset [a,b)$ with $t_{n}\nearrow b$ the corresponding sequence $\left\{ \gamma (t_{n})\right\}$ is a future chain in $M$, (\ref{keyeq1}) follows immediately from items \ref{item:chains} and \ref{item:c-completioncompleta} of Thm. \ref{thm:mainc-completion}. \\
$(iii)$ Suppose $(P,F)$ is a future endpoint of the causal curve $\gamma$ with $P\neq \emptyset$, and take again any increasing sequence $\left\{ t_{n} \right\} \subset [a,b)$ with $t_{n}\nearrow b$, so that $(P,F) \in L(\left\{ \gamma (t_{n}) \right\})$. According to the definition of the limit operator $L$, Eq. (\ref{eq:4}), we have that
\[
P \in \hat{L}(\left\{ I^-(\gamma (t_{n}) \right\}),
\]
which in turn means, according to Eq. (\ref{eq:3}), that
\[
P\subset \mathrm{LI}(\left\{ I^-(\gamma (t_{n}) \right\}) \subset I^-(\gamma).
\]
Now, clearly $I^-(\gamma)\subset \mathrm{LS}(\left\{ I^-(\gamma (t_{n}) \right\})$, and the maximality of $P$ as an IP
therein, as required by (\ref{eq:3}), now yields that $P=I^-(\gamma)$, as desired.
\end{proof}

\begin{remark}\label{r}{\em The situation when $\gamma$ is causal but not timelike is more involved. The converse of item \ref{item:endpoint3} is no longer true (see \cite[Section 3.5]{Floresfinaldefinitioncausal2011}). Indeed, inextendible causal curves may not have endpoints in the c-completion. It is therefore natural to consider the following definition (compare with  \cite[Definition 3.33]{Floresfinaldefinitioncausal2011}).}
\end{remark}
%
\begin{definition}\label{def:properly-causal}
  A $c$-completion $\overline{M}$ is said to be {\em properly causal} if any future or past-inextendible causal curve in $M$ has an endpoint on $\overline{M}$. \end{definition}
  % \cambiosn{We will say that $\overline{M}$ is {\em strongly properly causal} if given $(P,F)\in\overline{M}$ and $\eta$ a future-directed (resp. past-directed) causal curve, we have:}\footnote{\cambiosn{No haría falta la nocion de "strongly properly causal" si suponemos la topologia de primer orden, verdad?.}}
%     \begin{equation}
% (P,F)\hbox{ is an endpoint of $\eta$}\iff P=I^-(\eta)\hbox{ (resp. $F=I^+(\eta)$)}.\label{eq:1}
% \end{equation}
% \end{definition}
% \noindent \cambiosn{Clearly, strongly properly causal implies properly causal.}
\noindent Some conditions ensuring the properly causal property were summarized in \cite[Theorem 3.35]{Floresfinaldefinitioncausal2011}. For instance, a spacetime is properly causal if it is globally hyperbolic. In order to obtain more general results, we will take a special look at \cite[Theorem 3.35 (iii)]{Floresfinaldefinitioncausal2011}. In fact, that theorem motivates the following definition.

\begin{definition} A future-inextendible (resp. past-inextendible) causal curve $\gamma:[a,b)\rightarrow M$ is {\em future-regular} (resp. {\em past-regular}) if
 \[
 \uparrow \gamma = \uparrow I^{-}(\gamma)\quad
    \left({\rm resp.} \downarrow \gamma = \downarrow I^{+}(\gamma) \right).
\]
 \end{definition}
\begin{remark}\label{r2}{\em
  Clearly, the inclusion $\uparrow \gamma \subset \uparrow I^{-}(\gamma)$ holds for any future-inextendible causal curve $\gamma$. Moreover, any future-inextendible timelike curve is future-regular (with time-dual analogous statements). In a globally hyperbolic spacetime the common future $\uparrow I^-(\gamma)$ of any future-directed causal curve $\gamma$ is empty. In particular, any future-inextendible causal curve is future-regular. Analogously, past-inextendible causal curves are always past-regular in a globally hyperbolic spacetime. On the other hand, there are very simple examples where future(past)-regularity fails. For instance, take the flat $2d$ Minkowski spacetime $(\mathbb{R}^2, -dt^2+dx^2)$ and delete the spacelike half-axis $t=0,x\geq 0$. Then, the null geodesic generator of the past of the origin $(0,0)$ on the side of positive $x$ is not future-regular.}
\end{remark}
\noindent The following result shows that only a few very specific causal curves can fail to be regular (see also \cite[Prop. 3.2]{Florescausalboundarywavetype2008}.

  \begin{proposition}
    Let $\gamma:[a,b)\rightarrow M$ be a future-inextendible (resp. past-inextendible) causal curve. If there is no $c\in [a,b)$ such that $\gamma|_{(c,b)}$ is a null geodesic ray, then $\gamma$ is future(resp. past)-regular.
  \end{proposition}
  \begin{proof}
     Let us focus on the future case (the past case is proved analogously). From the hypothesis, there exists a sequence $\left\{ t_{n} \right\}$ with $t_{n}\nearrow b$ such that $\gamma(t_{n})\ll \gamma(t_{n+1})$ for all $n$. Therefore, we can construct a timelike curve $\eta$ by concatenating timelike curves connecting consecutive points of the sequence. By construction, $\eta$ and $\gamma$ have the same past and the same common future, and the result follows from the fact that any timelike curve $\gamma$ is future-regular.
  \end{proof}

\noindent On the other hand, we shall often need to work with null geodesic rays; to make them treatable, we introduce the following definition:
  \begin{definition}
\label{def:sproperlycausal} \noindent A c-completion $\overline{M}$ is {\em strongly properly causal} if any future-inextendible (resp. past-inextendible) null geodesic ray is future(resp. past)-regular.
  \end{definition}
\noindent Clearly, strong proper causality implies proper causality (see \cite[Theorem 3.35 (iii)]{Floresfinaldefinitioncausal2011}). One might ask whether there is any natural causal condition on $(M,g)$ which implies strong proper causality. Note, however, that the flat $2d$ spacetime we construct in Remark \ref{r2} is {\em stably causal}, since the coordinate $t$ gives a time-function therein. Yet, it turns out that {\em causal continuity}, a causal condition immediately stronger than stable causality in the causal ladder (see Section 3.3 of Ch. 3 in \cite{BeemGlobalLorentzianGeometry1996}), holds in $M$:

\begin{proposition}\label{prop:causalcontinuity}
  The $c$-completion of any causally continuous spacetime $(M,g)$ is strongly properly causal.
\end{proposition}
\begin{proof}
Suppose, by way of contradiction, that $(M,g)$ is causality continuous but there exists a future-inextendible null geodesic ray $\gamma:[a,b) \rightarrow M$ which is not future-regular. Since we always have $\uparrow \gamma \subset \uparrow I^{-}(\gamma)$ (cf. Remark \ref{r2}), there exists $p \in \uparrow I^{-}(\gamma)\setminus \uparrow \gamma$. Let $q \in I^-(p)$ such that $I^{-}(\gamma) \subset I^-(q)$, and pick $q \ll q'\ll p$. Fix $t_0 \in [a,b)$. Since any sequence $\{x_n\}$ in $I^-(\gamma(t_0)) \subset I^-(q)$ such that $x_n \rightarrow \gamma(t_0)$ is in $I^-(q)$, we infer that $\gamma(t_0) \in \overline{I^-(q)}$. The causal continuity now implies (see Lemmas 3.42 and 3.43 of \cite{Minguzzicausalhierarchyspacetimes2008}) that $ q \in \overline{I^+(\gamma(t_0))}$, and hence $q' \in I^+(\gamma(t_0))$. Since $t_0$ is arbitrary, we conclude that $\gamma[a,b) \subset I^-(q')$. But then one concludes that $p \in \uparrow \gamma$, in contradiction.
\end{proof}

The importance of strong proper causality lies in that it provides the following characterization for the endpoints of any causal curve (which extends the equivalence (\ref{keyeq1}) for timelike curves):
\begin{proposition}\label{prop:strongfirst}
  Let $\overline{M}$  be the c-completion of a strongly causal spacetime $(M,g)$, and assume that $\overline{M}$ is strongly properly causal. A pair $(P,F)$ is endpoint of a future(resp. past)-directed, future(resp. past)-regular causal curve if, and only if, $P=I^{-}(\gamma)$ (resp. $F=I^{+}(\gamma)$).
\end{proposition}
\begin{proof}
Apply the same ideas as in the proofs of Theorem \ref{thm:mainc-completion} and \cite[Theor. 3.35 (iii)]{Floresfinaldefinitioncausal2011}.
\end{proof}



%%% Local Variables:
%%% mode: latex
%%% TeX-master: "nullinfinityV5"
%%% End:


\section{Conformal extensions vs $c$-completion}
\label{conf}

The standard, mathematically precise definitions of {\em null infinity} and {\em black hole} are based on the notion of  conformal boundary of a spacetime as introduced by R. Penrose \cite{PenroseAsymptoticStructure1963,PenroseConformalInfinity1964}. We wish to generalize these notions to the context of $c$-boundaries, and accordingly we start by revisiting the relation between these two boundaries, which comprises some of the main results in \cite[Section 4]{Floresfinaldefinitioncausal2011}.

Henceforth, the $c$-completion $\overline{M}$ of $(M,g)$ will always be understood to be endowed both with the chronological relations and the chronological topology as defined in the previous section.
%Finally, we will particularize this study to the case of globally hyperbolic spacetimes.

%In order to obtain consistency on our definition of black holes, it is important to compare it with previous defintion obtained with the conformal boundary. In \cite[Section 4]{Floresfinaldefinitioncausal2011} the authors make a detailed comparison between both constructions, the conformal and the c-completion. Along this section, we will review the main points of their studies.

%\smallskip
%
%Let us begin by recalling the concept of conformal boundary.

\begin{definition}\label{def:envelopment}
  Given a strongly causal spacetime $(M,g)$, a {\em conformal extension} of $(M,g)$ is an open embedding $i:(M,g) \hookrightarrow (\tilde{M},\tilde{g})$ of $(M,g)$ into some strongly causal spacetime $(\tilde{M},\tilde{g})$ preserving time-orientation, for which there exists a strictly positive function $\Omega\in C^{\infty}(M)$ satisfying
    \begin{equation}
\label{conformalfactor1}
i^{*}\tilde{g}=\Omega^2 g.
\end{equation}
$\Omega$ is called the {\em conformal factor} of the conformal extension $i$.

The {\em conformal completion} of $(M,g)$ with respect to the conformal extension $i$ is defined as the (topological) closure $\overline{M}_i:=\overline{i(M)}\subset \tilde{M}$, and the associated {\em conformal boundary} as the (topological) boundary $\partial_i M:=\overline{i(M)}\setminus i(M)$.

Finally, we denote by $\overline{M}_i^*$ (resp. $\partial_i^* M$) the set of all the {\em accessible points} of the conformal completion (resp. conformal boundary), that is, the set of points on $\overline{M}_i$ (resp. $\partial_i M$) which are endpoints\footnote{Here, and throughout this section, the term ``endpoint'' is taken in the usual spacetime sense.} of timelike curves contained in $M$.
\end{definition}

\begin{remark}\label{rmk1}
{\em A few comments about Definition \ref{def:envelopment} are in order.
\begin{enumerate}
\item Note that we {\em do not} require, in this definition, that a given conformal boundary should have any regularity; even if it is piecewise smooth, we do not demand that the conformal factor extends smoothly to the boundary. As it stands, the definition is too weak to be useful in most applications, and has to be supplemented according to specific needs. In particular, the definition we give here is much more general than the ones usually found in the literature, which require at least $C^1$ smoothness of the boundary, extendibility of the conformal factor to the boundary and a host of other properties (see, e.g., Ch. 11 of \cite{WaldGeneralRelativity1984}). We too shall add some extra assumptions in what follows, but will shall do so gradually as needed, and the added assumptions will still be fairly general and comprise most concrete strongly causal examples in the literature.
\item If $(M,g)$ is extendible as a strongly causal spacetime, i.e., it is {\em isometric} to a proper open subset of a larger strongly causal spacetime $(\tilde{M}, \tilde{g})$, then the latter gives a conformal extension of the former with conformal factor $\Omega \equiv 1$. To avoid this kind of triviality, one usually assumes, in concrete applications, that $(M,g)$ is in some sense ``maximal'' in the strongly causal class.
\item The ``larger'' spacetime used in a conformal extension may well be $(M,g)$ itself. As a standard example, consider the following. Let $M = \mathbb{R}^2$ with the flat metric
\[
g = -dudv,
\]
and the time orientation such that both $\partial_u$ and $\partial_v$ are future-directed null vector fields. This spacetime is geodesically complete, and hence inextendible. Define
\[
i: (u,v) \in M=\mathbb{R}^2 \mapsto (\arctan u,\arctan v) \in \mathbb{R}^2
\]
and consider the smooth function
\[
\label{conformalfactor2}
\Omega : (u,v) \in \mathbb{R}^2 \mapsto \cos u \cos v \in \mathbb{R}.
\]
Then it is easy to check that $(\mathbb{R}^2,g)$ is a (nontrivial) conformal extension of $(M,g)$ such that the image of $M$ by $i$ is the open square $Q:= (-\pi/2, \pi/2)^2$, with conformal factor $\Omega \circ i$. Note that $\Omega$ vanishes on the boundary of $Q$ in $\mathbb{R}^2$. Thus, $(M,g)$ is conformally extended via a mapping onto an open set of itself.
\item Let $i:(M,g) \hookrightarrow (\tilde{M},\tilde{g})$ be a conformal extension of $(M,g)$. Since $i$ is a diffeomorphism when viewed as a map onto its open image $i(M)\subset \tilde{M}$, consider its (smooth) inverse $i^{-1}: i(M) \rightarrow M$. Then $(i(M),(i ^{-1})^{\ast} g)$ is a spacetime on its own right, and moreover it is isometric to $(M,g)$ by construction. Hence, there is no loss of generality in regarding $M \subseteq \tilde{M}$ and $i$ to be the inclusion map. We shall often do so in what follows, and will then abuse of notation by referring to $(\tilde{M},\tilde{g})$ itself as a conformal extension and discard any reference to the map $i$. In this case, we write the condition (\ref{conformalfactor1}) as
\[
\label{conformalfactor3}
\tilde{g}|_{M} = \Omega^2 g.
\]

\item Even upon identifying isometric spacetimes as in the previous item, conformal extensions, and also the associated conformal boundaries, are clearly far from being unique. But exploiting its relationship with the $c$-boundary, as we shall see below, allows one to build its accessible part in an essentially unique fashion.
% \item[5)] The existence of a conformal extension, as well as the conformal extension itself, are conformally invariant notions in the following sense. If $(\tilde{M},\tilde{g})$ is a conformal extension of $(M,g)$ with conformal factor $\Omega \in C^{\infty}(M)$, then it is also a conformal extension (with the same conformal boundary) of the spacetime $(M,\omega^2 g)$, where $\omega \in C^{\infty}(M)$ is a strictly positive function, with conformal factor $\Omega/\omega$. {\em Note, however, and this will be important in the next section, that the extendibility of the conformal factor to the boundary, and the values that such an extension might have there, are not a conformally invariant notion, depending rather sensitively on how $\Omega$ and $\omega$ are related.}
\end{enumerate}
}
\end{remark}

A given conformal completion will always be assumed to be endowed with the induced topology from $\tilde{M}$ and with a chronological (resp. causal) relation defined as follows: two points $p,q\in \overline{M}_i$ are chronologically (resp. causally) related, $p\ll_i q$ (resp. $p \leq _i q$) if there exists a smooth future-directed timelike (resp. causal) curve $\gamma:[a,b]\rightarrow \overline{M}_i$ from $p$ to $q$ with $\gamma|_{(a,b)}\subset M$ (see \cite[Section 4.1]{Floresfinaldefinitioncausal2011} for additional discussion on these choices).

\medskip

As one might expect, in general the $c$-completion differs from a given conformal one. In fact, despite its name, the conformal completion is {\em not} conformally invariant (see, for instance, \cite[Figure 10]{Floresfinaldefinitioncausal2011}). However, (and this is one of the main points of this section) under some additional conditions on the conformal extension, it is possible that both completions coincide. We shall presently discuss some of these conditions.

%The first requirement allows to ensure that the envelopment encapsulate all the possible boundary points of $M$.
\begin{definition}\label{def:chronologically complete}
Let $(M,g)$ be a spacetime. A conformal extension $(\tilde{M},\tilde{g})$ is said to be {\em chronologically complete} if any timelike curve $\gamma:[a,b)\rightarrow M$ which is inextendible in $M$ has an endpoint $p$ on the associated conformal boundary.
\end{definition}
Clearly, this condition parallels the point \ref{item:c-completioncompleta} in Theorem  \ref{thm:mainc-completion}, and is meant to ensure that the conformal completion has ``enough points''. (Otherwise, by suitably deleting points in a conformal completion, we have a highly non-unique construction which will then become geometrically useless.)

Another important requirement is that, around the points $p$ of the (accessible) conformal boundary, the causal structure of $(\tilde{M},\tilde{g})$ is adapted to that of $(M,g)$. To formalize this idea, let us begin with the concept of {\em timelike deformable points} (see \cite[Definition 4.10]{Floresfinaldefinitioncausal2011})

\begin{definition}\label{def:deformable}
Consider a continuous curve $\gamma:[a,b]\rightarrow \overline{M}_i$ such that $\gamma|_{[a,b)}$ is a future-directed smooth timelike curve contained in $M$ and $\gamma(b)\in \partial_i M$. Then, $\gamma$ is {\em future deformably timelike} if there exists a neighborhood $U=\tilde{U}\cap \overline{M}_i$ of $\gamma(b)$ (where $\tilde{U}$ is an open set of $\tilde{M}$) such that $\gamma(a)\ll_i \omega$ for all $\omega\in U$. (The notion of {\em past deformably timelike} is analogous.)

  An accessible point $p\in \partial_i^* M$ is {\em timelike deformable} if all the TIPs and TIFs associated to $p$ (i.e., TIPs and TIFs defined by timelike curves on $M$ with endpoint $p$) are intersections with $M$ of the chronological pasts or futures in $(\tilde{M},\tilde{g})$ of timelike deformable curves in $\overline{M}_i$.
\end{definition}

The previous notion ensures only that the {\em chronological} relation of the extension is well-behaved with respect to that of $(M,g)$; the following definition extends such good behaviour to the {\em causal} relation.

\begin{definition}\label{def:transitive}
  An accessible point $p\in \partial_i^* M$ is {\em (locally) timelike transitive} if it admits a neighborhood $V=\overline{M}_i\cap \tilde{V}$ ($\tilde{V}$ open in $\tilde{M}$) such that for any $q,q'\in V$:
  \begin{itemize}
  \item $q\ll_i p \leq_i q' \Rightarrow q\ll_i q'$.

  \item $q\leq_i p \ll_i q'\Rightarrow q\ll_i q'$.
  \end{itemize}
\end{definition}
 If an accessible point $p\in \partial_i^* M$ satisfies both timelike deformability and timelike transitivity, then we say that $p$ is {\it regularly accessible}. If all the accessible boundary points are regularly accessible, we will just say that $\partial_i^* M$ itself is {\it regularly accessible}.

\smallskip

We are now ready to present the main result in \cite[Section 4]{Floresfinaldefinitioncausal2011}.

\begin{theorem}\cite[Theorem 4.16]{Floresfinaldefinitioncausal2011} \label{thm:causaltoconformal} Let $i:(M,g) \hookrightarrow (\tilde{M},\tilde{g})$ be a chronologically complete extension of $(M,g)$. If the accessible conformal boundary $\partial_i^* M$ is regularly accessible, then the accessible part $\overline{M}_i^*$ of the conformal completion and the $c$-completion $\overline{M}$ are equivalent in the following precise sense:
\begin{enumerate}[label=(\roman*)]
  \item \label{thmcausaltoconformal-defPsi} There exists a homeomorphism $\Psi:\overline{M}\rightarrow \overline{M}_i^*$, which maps boundary to boundary.
  %is well defined and bijective. In fact, given $(P,F)\in \overline{M}$, $\Psi((P,F))$ will be the endpoint\footnote{??A cual endpoint te refieres?? Hay dos, no?} of any timelike curve defining $P$ or $F$.
  %There exists a map $\Psi:\overline{M}\rightarrow \overline{M}_i^*$ which is well defined and bijective. In fact, given $(P,F)\in \overline{M}$, $\Psi((P,F))$ will be the endpoint\footnote{??A cual endpoint te refieres?? Hay dos, no?} of any timelike curve defining $P$ or $F$.
%  \item $\Psi$ is a homeomorphism.
  \item \label{thmcausaltoconformal-chrniso} $\Psi$ is a chronological isomorphism (i.e., $\Psi$ and $\Psi^{-1}$  preserve the chronological relations).
  \end{enumerate}
\end{theorem}





















%
%\smallskip
%
%If $(M,g)$ is globally hyperbolic, then the regular accessibility can be obtained from more natural causal conditions. Moreover, in this case one is also able to deduce some important structural properties of the conformal boundary and the $c$-boundary which do not apply for general strongly causal spacetimes. Let us begin introducing the basic concepts we are going to use on the rest of this section.
%
%\begin{definition}
%\label{def1}
%Let $A$ be a subset of a spacetime $(M,g)$.
%\begin{itemize}
%\item[a)] $A$ is {\em causally convex} if any causal curve segment in $(M,g)$ with endpoints in $A$ is entirely contained in $A$,
%\item[b)] $A$ is {\em future-precompact} (resp. {\em past-precompact} ) is there exists a compact set $K \subset M$ such that $A \subset I^-(K)$ (resp. $A \subset I^+(K)$).
%\end{itemize}
%The {\em future boundary} (resp. {\em past boundary}) of $A$ is\footnote{We have adopted the following convention: the $\hat{ }$ (resp. $\check{ }$ ) is a superscript for future (resp. past) $c$-boundaries and $^{+}$ (resp. $^{-}$) is a superscript for future (resp. past) boundaries in $M$ of {\em subsets} thereof.}
%\[
%\partial ^+ A := I^+(A) \cap \partial A \mbox{ (resp. $\partial ^- A := I^-(A) \cap \partial A$)}.
%\]
%\end{definition}
%
%Note that any PIP (resp. PIF) is future-precompact (resp. past-precompact) in $(M,g)$. Moreover, if $(M,g)$ is globally hyperbolic and $A \subset M$ is both future- and past-precompact, then it is indeed precompact, which motivates this terminology.
%
%\begin{proposition}
%\label{convexitypastfuture} Let $A$ be a subset of a spacetime $(M,g)$. Then, the following statements hold:
%\begin{enumerate}[label=(\roman*)]
%\item Suppose $A$ is open. Then, $\partial ^+ A = \emptyset$ (resp. $\partial ^- A = \emptyset$) if and only if $A$ is a future set (resp. past set).
%\item If $A$ is causally convex, then $\partial ^{\pm} A$, if non-empty, are achronal $C^0$ hypersurfaces in $(M,g)$. If, in addition, $A$ is open, then $A \cup \partial ^{\pm} A$ is causally convex.
%\item If $A$ is either a future or a past set in $(M,g)$, then $A$ is causally convex.
%\end{enumerate}
%\end{proposition}
%\begin{proof}
%  \textit{(i)} The ``if'' part is immediate. For the converse, assume that $\partial ^+ A = \emptyset$, and let $p \in A, q \in I^+(p)$. Pick any future-directed timelike curve $\alpha:[0,1] \rightarrow M$ such that $\alpha(0) =p$ and $\alpha(1) =q$. If $\alpha$ ever leaves $A$, then it must intersect $\partial ^+ A$, a contradiction. Thus $q \in A$, and we conclude that $A$ is a future set. Reasoning time-dually, we establish that $A$ is a past set when $\partial ^- A = \emptyset$.
%
%\smallskip
%
%\textit{(ii)} We give the proof for the future boundary only, since the past case follows by time-duality. Suppose $\partial ^{+} A$ is not achronal, and let $p, q \in \partial ^{+} A$ such that $p\ll_g q$. Thus, we can pick open sets $U,V \subset I^+(A)$ such that $p \in U$, $q \in V$ and
%\[
%p' \in U, q' \in V \Longrightarrow p' \ll_g q'.
%\]
%Since $p,q \in \partial A$ we can choose $p' \in U \cap (M \setminus A)$ and $q' \in V \cap A$. We can also choose $p'' \in A \cap I^-(p')$. But then $p' \in I^+(p'') \cap I^-(q')\cap (M \setminus A)$, and therefore $A$ cannot be causally convex, contrary to our assumption.
%
%To show that $\partial ^{+} A$ is a $C^0$ hypersurface, since it is achronal we only need to show that $\partial ^{+}A \cap edge(\partial ^+ A) = \emptyset$. Let $p \in \partial ^{+}A$. Then, in particular $p \in I^+(q)$ for some $q \in A$. Pick any future-directed timelike curve $\gamma: [0,1] \rightarrow M$ such that $\gamma(0) \in I^+(q)\cap I^-(p)$ and $\gamma(1) \in I^+(p)$. In particular, $\gamma[0,1] \subset I^+(A)$. The achronality of $\partial ^{+} A$ implies that $\gamma(1) \notin A$. We claim that $\gamma(0) \in A$. If not, noting that $I^+(\gamma(0)) \cap A \neq \emptyset$ because $p \in I^+(\gamma(0)) \cap \partial A$, we could then juxtapose a future-directed timelike curve from $q \in A$ to $\gamma(0)$ with another future-directed timelike curve from $\gamma(0)$ to some point in $A$, thereby obtaining a future-directed timelike curve with endpoints in $A$ but not entirely containing therein. This in turn would contradict the fact that $A$ is causally convex. We conclude that $\gamma$ intersects $\partial ^{+}A$, and therefore $p \notin edge(\partial ^{+} A)$.
%
%Now assume $A$ is open and $A \cup \partial ^{+} A$ is not causally convex. Consider (say) a future-directed causal curve segment $\alpha:[0,1] \rightarrow M$ with $\alpha(1),\alpha(0) \in A\cup \partial ^{+} A$ and $\alpha(t_0) \neq A\cup \partial ^{+} A$ for some $t_0 \in (0,1)$. Since $A$ is open, we necessarily have $\alpha(0),\alpha(1) \in I^+(A)$. So we can pick $p \in A$ such that $p \ll_g \alpha(0)$. Thus $q \ll_g \alpha(t_0)$. But any future-directed timelike curve from $p$ to $\alpha(t_0)$ must intersect $\partial ^{+} A$, and hence $p \ll_g r \ll_g \alpha(t_0)$ for some $r \in \partial ^+A$. Now, either $\alpha(1) \in \partial ^+ A$ or else $\alpha(1) \in A$, in which case $\alpha|_{(t_0,1)}$ must intersect $\partial ^+ A$ anyway. In any case, $\alpha(t_0) < q$ for some $ q \in \partial ^+ A$, and then $r \ll_g q$, contradicting the achronality of $\partial ^+ A$.
%
%\textit{(iii)} If $A$ is a past (resp. future) set, then $\partial A \equiv \partial ^+A$ (resp. $\partial A \equiv \partial ^-A$), and $\partial A$ is well-known to be achronal by standard results in causal theory. Since $A$ is open, the fact that $A$ is causally convex now follows from $(ii)$.
%\end{proof}
%
%
%\begin{proposition}
%\label{convexitygh1}
%Assume that $(M,g)$ is globally hyperbolic, and let $U \subset M$ be an open, connected, globally hyperbolic subset. Then, the following statements are equivalent.
%\begin{enumerate}[label=(\roman*)]
%\item $U$ is causally convex.
%\item Any achronal set in $(U,g|_U)$ is achronal in $(M,g)$.
%\item Any Cauchy hypersurface $S$ in $(U,g|_U)$ is achronal in $(M,g)$.
%\end{enumerate}
%\end{proposition}
%\begin{proof}
%$(i) \Longrightarrow (ii) \Longrightarrow (iii)$ are immediate. Therefore, we only need to prove $(iii) \Longrightarrow (i)$. Let $S \subset U$ be a Cauchy hypersurface in $(U,g|_U)$. Let $\alpha:[0,1] \rightarrow M$ be a future-directed causal curve with endpoints in $U$. Since $U$ is open, there will be no loss of generality in our argument if we assume that $\alpha(0),\alpha(1) \in I^+(S,U)$ and that $\alpha$ is timelike. Suppose $\alpha(t_0) \notin U$ for some $t_0 \in (0,1)$. Also, write
%\begin{eqnarray}
%s_0 &:=& \inf \{ s \in [0,1] \, : \, \alpha[s,1] \subset U \}, \\
%r_0 &:=& \sup \{ t \in [0,1] \, : \, \alpha[0,t] \subset U \}.
%\end{eqnarray}
%Then the restrictions $\alpha_+:= \alpha|_{[0,r_0)}$ and $\alpha_{-}:= \alpha|_{(s_0,1]}$ are future- and past-inextendible in $(U,g|_U)$, respectively, and $r_0 < t_0 < s_0$. But then $\alpha_{-}$ must intersect $S$ at some $s_0< s'$, say, so
%\[
%p \ll_g \alpha(0) \ll_g \alpha(s')
%\]
%for some $p \in S$, which violates the achronality of $S$ in $(M,g)$. Therefore $\alpha[0,1] \subset U$.
%\end{proof}
%
%\begin{theorem}
%\label{convexitygh2}
%Assume that $(M,g)$ is globally hyperbolic, and let $U \subset M$ be an (non-empty) open, connected, causally convex, future-precompact and globally hyperbolic set. Then $\partial ^+ U$ is homeomorphic to a Cauchy hypersurface $S \subset U$ in $(U,g|_U)$.
%\end{theorem}
%
%\begin{proof}
%$S$ is a connected $C^0$ hypersurface in $M$. By Proposition \ref{convexitygh1}, it is achronal in $(M,g)$. Let $K \subset M$ be some compact set such that $U \subset I^-(K)$. In particular, $U$ cannot be a future set, and hence $\partial ^+ U$ is also a non-empty achronal $C^0$ hypersurface by Proposition \ref{convexitypastfuture} (i)-(ii). Let $X:M \rightarrow TM$ be a complete future-directed timelike smooth vector field in $(M,g)$, and denote its flow by $\phi$. We shall define a continuous mapping $\varphi : S \rightarrow \partial ^+ U$ as follows. Given $p \in S$, the future-inextendible timelike curve
%\[
%t \in [0, +\infty) \mapsto \phi_t(p)
%\]
%starting at $p$ must leave the compact set $J^+(p)\cap J^-(K)$, and hence must leave $U$. When its does, it will intersect $ \partial ^+ U$, and since this set is achronal, will do so at a unique point, which we define as $\varphi(p)$. The mapping thus defined is obviously one-to-one, and hence open by Invariance of Domain. Therefore, in other to show that $\varphi$ is the desired homeomorphism we only need to show that it is onto. Accordingly, let $q \in \partial ^+ U$, and consider the inextendible timelike curve
%\[
%\gamma : s \in (-\infty, +\infty) \mapsto \phi_s(q).
%\]
%Then, for some $\varepsilon >0$, $\gamma[-\varepsilon,0] \subset I^+(U)$.
%
%We claim that $\gamma(-\varepsilon) \in U$. If not, noting that $I^+(\gamma(-\varepsilon)) \cap U \neq \emptyset$ because $q \in I^+(\gamma(-\varepsilon)) \cap \partial U$, we could then juxtapose a future-directed timelike curve from some $r \in U$ to $\gamma(-\varepsilon)$ with another future-directed timelike curve from $\gamma(\varepsilon)$ to some point in $U$, thereby obtaining a future-directed timelike curve with endpoints in $U$ but not entirely containing therein. This in turn would contradict the fact that $U$ is causally convex. We conclude that $\gamma$ intersects $U$, and hence must intersect $S$. If this intersection occurred at the future of $q$ this would violate the achronality of $\partial ^+ U$, since we saw above that the future of any point in $S$ intersects $\partial ^+ U$. Therefore, for some $s_0 < 0$, $\phi_{s_0}(q) \in S$. But then we of course have $\varphi (\phi_{s_0}(q)) \equiv q$, thus concluding the proof.
%\end{proof}
%
%Both the causal convexity and the future/past-precompactness will be sufficient conditions to ensure the regular accessibility of the conformal boundary. In this sense, we introduce the following definition, adapted  from Ref. \cite{Olaf}\footnote{Jony: Referencia de Olaf}).
%
%\begin{definition}
%\label{nested}
%A conformal extension $(\tilde{M},\tilde{g})$ of a globally hyperbolic spacetime $(M,g)$ is {\em future-nesting} if
%\begin{itemize}
%\item[N1)] $(\tilde{M},\tilde{g})$ is also globally hyperbolic,
%\item[N2)] $M \subset \tilde{M}$ is causally convex and future-precompact.
%\end{itemize}
%An analogous definition follows for {\em past-nesting}.
%\end{definition}
%\cambiosn{Proposition \ref{convexitypastfuture} and Theorem \ref{convexitygh2} immediately give the following regularity result.
%\begin{corollary}
%\label{convexitygh3}
%If $(\tilde{M},\tilde{g})$ is a future-nesting conformal extension of a globally hyperbolic spacetime $(M,g)$, then the corresponding future conformal boundary $\partial ^+M$ is an achronal $C^0$ hypersurface in $(\tilde{M},\tilde{g})$ homeomorphic to a Cauchy hypersurface in $(M,g)$.
%\end{corollary}}
%
%
%\begin{remark}\label{rmk2}
%{\em The standard conformal extensions in the Einstein cylinder of Minkowski spacetime, past-incomplete Friedmann-Robertson-Walker models with a non-negative cosmological constant, and de Sitter spacetime are all future-nesting in this sense. The standard conformal extension of the Schwarzschild-Kruskal plane is of course also future-nesting.}
%\end{remark}
%
%Before giving our main result for future and past-nesting conformal extensions, we will need the following technical lemma.
%
%\begin{lemma}
%\label{technical}
%Let $(\tilde{M},\tilde{g})$ be a future-nesting conformal extension of a globally hyperbolic spacetime $(M,g)$. Given $p \in \partial ^+M$, $q \in I^-(p, \tilde{M})$ and any future-directed timelike curve $\alpha:[0,1] \rightarrow \tilde{M}$ from $q$ to $p$, there exists some $0<\varepsilon <1$ for which $\alpha[1-\varepsilon,1) \subset M$.
%\end{lemma}
%\begin{proof}
%Since $p \in I^+(M,\tilde{M})$ and the latter set is open, for some $0< \varepsilon <1$, $\alpha[1- \varepsilon,1] \subset I^+(M,\tilde{M})$ by continuity. Let $t_0 \in [1-\varepsilon,1)$. Since $p \in I^+(\alpha(t_0),\tilde{M})$, $I^+(\alpha(t_0))\cap M \neq \emptyset$ since $p$ is a boundary point. Choose a future-inextendible timelike curve $\beta:[0,a) \rightarrow \tilde{M}$ starting at $\alpha(t_0)$ and intersecting $M$. Pick a compact set $K \subset \tilde{M}$ for which $M \subset J^-(K,\tilde{M})$. The global hyperbolicity of $(\tilde{M},\tilde{g})$ forces $\beta$ to leave the compact set $J^+(\alpha(t_0),\tilde{M})\cap J^-(K, \tilde{M})$, and hence leaves $M$. Thus, if $\alpha(t_0) \notin M$, we could choose a timelike curve segment in $\tilde{M}$ with endpoints in $M$ which leaves $M$, violating the causal convexity of $M$. Therefore, $\alpha(t_0) \in M$, and we then conclude that $\alpha[1-\varepsilon,1) \subset M$.
%\end{proof}
%
%\begin{theorem}
%\label{thm:futurenestingprincipal}
%Let $(\tilde{M},\tilde{g})$ be a future and past-nesting conformal extension of a globally hyperbolic spacetime $(M,g)$. Then
%\begin{enumerate}[label=(\roman*)]
%
%\item $(\tilde{M},\tilde{g})$ is a chronologically complete conformal extension. Indeed, every inextendible causal curve in $(M,g)$ has both a future and a past endpoint on $\partial_{i}^{*}M$.
%
%\item  $\partial_{i}^{*}M$ is regularly accessible, and hence the conformal and causal completions coincide.
%
%
%\item The future (resp. past) conformal boundary $\partial^+ M$ is an achronal $C^0$ hypersurface in $(\tilde{M},\tilde{g})$ homeomorphic to a Cauchy hypersurface in $(M,g)$.
%\end{enumerate}
% \end{theorem}
% \begin{proof}
%  In order to prove \textit{(i)}, let $\gamma$ be an inextendible future-directed causal curve in $(M,g)$. We shall deal with future endpoints, since the past-endpoint case proceeds analogously. The curve $\tilde{\gamma}=i(\gamma)$ is also future-directed and causal in $(\tilde{M}, \tilde{g})$, and so it defines an IP. Moreover, by our nesting assumptions, $\tilde{\gamma}$ is contained in the compact set $J^{+}(\tilde{\gamma}(0),\tilde{M})\cap J^-(K,\tilde{M})\subset \tilde{M}$ for some compact subset $K\subset \tilde{M}$, and hence it is actually a PIP on $\tilde{M}$. In particular, there exists $p\in \tilde{M}$ such that $I^-(\tilde{\gamma},\tilde{M})=I^-(p,\tilde{M})$, where $p$ is a future endpoint of $\tilde{\gamma}$. As $\tilde{\gamma}\subset i(M)$, it follows that $p\in \partial_{i}^{*} M$, i.e., $\tilde{\gamma}$ has an endpoint in $\overline{i(M)}$, as desired.
%
%   \smallskip
%
%Finally, observe that the regular accessibility of the boundary points described in \textit{(ii)} is a direct consequence of Lemma \ref{technical}, while the structural properties in \textit{(iii)} follow from Theorem \ref{convexitygh2}. \end{proof}
%


%%% Local Variables:
%%% mode: latex
%%% TeX-master: "nullinfinityV5"
%%% End:


\section{Null infinity on $c$-boundaries}\label{bh}

We wish to introduce a notion of null infinity on $c$-boundaries which remains valid even when a conformal boundary does not exist.
%This notion will be defined so as to include, in a definite sense \cambiosn{(cf. Corollary \ref{voila} below),} the conformal notion of null infinity, whenever the latter is available.
Again, we will start with some general considerations, which will motivate our main results and definitions, as well as set up the appropriate notation.

First, let us give a more precise description of what we shall mean by {\em null infinity} here in the case when conformal boundaries are present. We will try and capture only some minimal aspects of the original Penrose's definition \cite{PenroseAsymptoticStructure1963,PenroseConformalInfinity1964,Frauendiener2004}, without some of the more technical and specific assumptions needed in physical applications (such as those given, for example, in Refs. \cite{HawkingLargeScaleStructure1975,WaldGeneralRelativity1984}).

\begin{definition}
\label{conformalinfinity1}
Let $i: (M,g) \hookrightarrow (\tilde{M},\tilde{g})$ be a conformal extension of the spacetime $(M,g)$ with conformal boundary $\partial_i M$ and conformal factor $\Omega$. An accessible point $p \in \partial^{*}_i M$ is said to {\em be at infinity} if there exist an open set $\tilde{U} \ni p$ of $\tilde{M}$ and a $C^{\infty}$ extension\footnote{It is possible, and may be interesting for certain purposes, to consider a lower - say, $C^1$-differentiability class without deep changes in the main results. Nevertheless, since this would make the following discussion a little more cumbersome, we adopt for simplicity a stronger regularity assumption.} $\tilde{\Omega}$ of $\Omega$ to $M \cup \tilde{U}$ such that $\tilde{\Omega}(p) =0$. Such a point at infinity is said to be {\em regular} if in addition $d\tilde{\Omega}(p) \neq 0$. The set of all regular points at infinity is the {\em conformal null infinity} ({\em associated with the conformal extension} $(\tilde{M},\tilde{g})$) and will be denoted by ${\cal J}_c$. We will say that a point $p\in {\cal J}_c$ belongs to the {\em future} (resp. {\em past}) {\em null infinity}, denoted by ${\cal J}_c^+$, (resp. ${\cal J}_c^-$) if $p$ is future (resp. past) accessible (recall Defn. \ref{def:envelopment})
\end{definition}

\begin{remark}
\label{rmk3}
{\em The following points about Definition \ref{conformalinfinity1} are relevant.
\begin{itemize}
\item[1)] The notions of a point $p \in \partial^{*}_i M$ being at infinity and/or being regular there clearly do not depend on the particular extension $\tilde{\Omega}$ of the conformal factor.
\item[2)] The notion of a point being at infinity is {\em not} a conformally invariant one. For example, if $\Omega$ can be smoothly extended to a point $p \in \partial^{*}_i M$ but the extension does not vanish there, we may rescale $g$ by (the restriction of) a conformal factor $\omega^2 \in C^{\infty}(\tilde{M})$ which diverges at $p$ suitably slowly so that $\Omega/\omega$ is still $C^{\infty}$. The point $p$ then becomes a point at infinity with respect to $(\tilde{M}, \tilde{g})$ when the latter is viewed as a conformal extension of $(M,\omega^2g)$. Conversely, points which are at infinity for a certain spacetime may fail to be so for others in the same conformal class. Indeed, the familiar example of $2d$ hyperbolic space already illustrates these remarks: points at the boundary of the unit disc in $\mathbb{R}^2$ are at infinity if we view the Euclidean plane as a conformal extension of the Poincar\'{e} disc model of the hyperbolic plane, but not if viewed as a conformal (actually isometric) extension of the (restriction of the) Euclidean metric on the disc.
\item[3)] As an example of this definition, we may look back at the example in item 3. of Remark \ref{rmk1}. There, the set of points at infinity consists of the union of the two segments $u=\pi/2,\, v>-\pi/2$ and $v=\pi/2, \, u> -\pi/2$. The point $u=v= \pi/2$ is the only non-regular point. This point is not the future endpoint of any null geodesic in $(M,g)$, but all {\em regular} points at infinity are endpoints of null geodesic {\em rays}.
    %\cambiosn{This latter feature is actually general for future-nesting conformal envelopments, as the next result, which also summarizes the features of points at infinity relevant to us here, shows.}
\end{itemize}
}
\end{remark}

\begin{theorem}
\label{conformalinfinity2}
Let $(\tilde{M},\tilde{g})$ be a conformal extension of the spacetime $(M,g)$ with conformal boundary $\partial_iM$ and conformal factor $\Omega$. Let $p \in \partial^{*}_iM$. Then the following statements hold.
\begin{itemize}
\item[i)] Suppose $p$ is at infinity, and let $\alpha:[0,A) \rightarrow M$ ($0<A \leq +\infty$) be a future-inextendible null geodesic in $(M,g)$ (analogously for past-directions). If $\alpha$ has a future endpoint at $p$ when viewed as a causal curve in $(\tilde{M}, \tilde{g})$, then $A=+\infty$ and $\alpha$ is future-complete. In other words, any null geodesic in $(M,g)$ with a future endpoint at infinity is future-complete.
\item[ii)] Conversely, suppose the conformal factor extends in a $C^{\infty}$ fashion to $M\cup \tilde{U}$ where $\tilde{U} \ni p$ is open in $\tilde{M}$. If $p$ is a future (resp. past) endpoint of a future-complete (resp. past-complete) null geodesic in $(M,g)$, then $p$ is at infinity.
\item[iii)] Suppose $(M,g)$ is inextendible (as a spacetime\footnote{Again, this result will still work {\em mutatis mutandis} if one only assumes that $(M,g)$ cannot be extended as a $C^1$ spacetime. On the other hand, there are trivial counterexamples to this result without the assumption of inextendibility, obtained by taking $(M,g)$ to be a suitable open subset of Minkowski with rough boundary.}). Then ${\cal J}_c$ is a smooth hypersurface in $(\tilde{M},\tilde{g})$.
    %%\cambiosn{
%    If in addition $(M,g)$ is globally hyperbolic and $(\tilde{M},\tilde{g})$ is future-nesting (resp. past-nesting), then for each $p \in {\cal J}_c^+$ there exists a future-complete (resp. past-complete) null geodesic ray $\gamma:[0,+\infty) \rightarrow M$ in $(M,g)$ with future (resp. past) endpoint $p$.
%    %}
\end{itemize}
\end{theorem}
\begin{proof} We shall present our arguments for future directions, since the past directed case is again analogous. \\
$(i)$ Using Definition \ref{conformalinfinity1} we can assume that the conformal factor extends in a $C^{\infty}$ fashion to $M\cup \tilde{U}$, where $\tilde{U} \ni p$ is open in $\tilde{M}$, and $\tilde{\Omega}(p)=0$. We shall, for notational simplicity, continue to refer to this extended conformal factor as $\Omega$. Denote by $\tilde{\nabla}$ the Levi-Civita connection on $(\tilde{M},\tilde{g})$. Using the fact that $\tilde{g}|_M = (\Omega|_M)^2 g$ and that $\alpha$ is a null geodesic in $(M,g)$, we get
\begin{equation}
\label{geodesic}
\tilde{\nabla}_{\dot{\alpha} }\dot{\alpha} = \nabla_{\dot{\alpha}}\dot{\alpha} + 2\frac{(\Omega \circ \alpha)'}{(\Omega \circ \alpha)} \dot{\alpha} \equiv f \cdot \dot{\alpha},
\end{equation}
where $f:[0,A) \rightarrow \mathbb{R}$ is the smooth function given by
\[
f := 2\frac{(\Omega \circ \alpha)'}{\Omega \circ \alpha}.
\]
Exercise 3.19 in p. 95 of \cite{ONeillSemiRiemannianGeometryApplications1983} now implies that $\alpha$ is a null pregeodesic in $(\tilde{M},\tilde{g})$. Let $h:[0,a) \rightarrow [0,A)$ ($0<a \leq +\infty$) be an increasing reparametrization of $\alpha$ such that $\hat{\alpha}:= \alpha \circ h$ is a geodesic in $(\tilde{M},\tilde{g})$. Since it has an endpoint in $p \in \partial^{*}_i M \subset \tilde{M}$, it is actually extendible as a null geodesic in $(\tilde{M},\tilde{g})$, and in particular $a <+\infty$. Denote by $\beta:[0,a] \rightarrow \tilde{M}$ the null $\tilde{g}$-geodesic segment extending $\hat{\alpha}$, with $\beta(a) \equiv p$. Since $\Omega(p) =0$, and $\Omega$ is $C^1$, by the mean value theorem there exists a number $k>0$ such that
\begin{equation}
\label{lipschitz}
(\Omega \circ \beta)(s) \leq k (a - s), \forall s \in [0,a].
\end{equation}
However, by exercise 3.19 of \cite{ONeillSemiRiemannianGeometryApplications1983} we must have
\[
h'' + (f \circ h) (h')^2 =0
\]
on $[0,a)$, which when we substitute the definition of $f$ gives, multiplying through $(\Omega \circ \alpha \circ h)^2$,
\begin{eqnarray}
0 &=& (\Omega \circ \alpha \circ h)^2 h'' + 2 (\Omega \circ \alpha \circ h)[(\Omega \circ \alpha)'\circ h)h ']h' \\ \nonumber
&=& (\Omega \circ \alpha \circ h)^2 h'' + 2 (\Omega \circ \alpha \circ h)(\Omega \circ \alpha \circ h)'h' \\ \nonumber
&=& (\Omega \circ \hat{\alpha})^2 h'' + 2(\Omega \circ \hat{\alpha})(\Omega \circ \hat{\alpha})'h' \\ \nonumber
&\equiv & [(\Omega \circ \hat{\alpha})^2 h']'.
\end{eqnarray}
Hence, for some constant $c \in \mathbb{R}$,
\begin{equation}
\label{relation}
(\Omega \circ \hat{\alpha})^2 h' = c.
\end{equation}
(We deduce that $c>0$, since $\Omega \circ \hat{\alpha} >0$ and $h$ is strictly increasing.) Therefore, for each $0\leq t <a$, we have
\begin{equation}
\label{integral}
h(t) = \int_0^t h'(s) ds = c \int_0^t \frac{ds}{[(\Omega \circ \hat{\alpha})(s)]^2} \geq \frac{c}{k^2} \int_0^t \frac{ds}{(a -s)^2},
\end{equation}
where we have used (\ref{lipschitz}) to get the last inequality. We thus conclude that for each $0\leq t <a$,
\[
A \geq h(t) \geq \frac{c}{k^2}\left( \frac{1}{a-t} - \frac{1}{a} \right).
\]
Since the right-hand side of this inequality diverges when $t \rightarrow a$, we conclude that $A \equiv +\infty$, proving $(i)$.

\smallskip

$(ii)$ Using the same notation as in $(i)$, assume, by way of contradiction, that $\alpha$ is future-complete in $(M,g)$, that is, that $A=+\infty$, but that $\Omega(p) \neq 0$. Then we may take $(\Omega \circ \hat{\alpha})$ to be bounded from below by some positive constant $B$ in Eq. (\ref{integral}), whence we conclude that
\[
h(t) \leq c/B, \, \forall t \in [0,a),
\]
and hence that $A \leq c/B$, a contradiction.

\smallskip

$(iii)$ Again, we may focus without loss of generality on the future part ${\cal J}^+_c$ of the conformal boundary. Let $p \in {\cal J}^+_c$. By definition, there exists an open neighborhood $\tilde{U}\ni p$ in $\tilde{M}$ to which $\Omega$ extends as a $C^{\infty}$ function (which we again denote by $\Omega$). Let $\tilde{\nabla}\Omega$ denote the (metrically) associated gradient vector field in $M\cup \tilde{U}$. Since $\tilde{\nabla}\Omega (p) \neq 0$, we can assume (shrinking it if necessary) that $\tilde{U}$ is a connected coordinate neighborhood contained in $I^+(M, \tilde{M})$, with a coordinate system denoted by $(x^1, \ldots x^{dim M})$, say, such that $\partial _{x^1} = \tilde{\nabla}\Omega|_{\tilde{U}}$. Then the set
\[
S := \{q \in \tilde{U} \, : \, x^1(p) = \Omega(q) =0 \}
\]
is a $C^{\infty}$ embedded codimension 1 submanifold (i.e., a hypersurface) of $\tilde{M}$ containing ${\cal J}^+_c \cap \tilde{U}$ and closed in $\tilde{U}$. Denote by $\tilde{U}_+$ the connected open subset of $\tilde{U}$ in which $\Omega >0$. Now, $M \cap \tilde{U}$ is obviously (non-empty and) contained in $\tilde{U}_+$, and if it were {\em properly} contained therein, then $(M\cup \tilde{U}_{+}, (1/\Omega^2) \tilde{g} |_{M\cup \tilde{U}_{+}})$ would be a non-trivial extension of $(M,g)$, contrary to our assumption. Therefore, $M \cap \tilde{U} = \tilde{U}_+$, and we then conclude that $\partial ^+ M \cap \tilde{U}\equiv S$, and hence that $S = {\cal J}^+_c \cap \tilde{U}$. This in turn establishes the first statement.

%\cambiosn{
%For the second claim, assume that $(M,g)$ is also globally hyperbolic and $(\tilde{M},\tilde{g})$ is future-nesting. Using the notation and results in the previous paragraph, let ${\cal N} \subset \tilde{U}$ be a normal neighborhood of $p$ which is the diffeomorphic image by the exponential map $\tilde{\exp}_p$ of $(\tilde{M},\tilde{g})$ of some open Euclidean ball $B \subset T_p\tilde{M}$. Now, $I^-(p, \tilde{M})\cap {\cal N} = \tilde{\exp}_p(B \cap N^-_p)$, where $N_p^- \subset T_p\tilde{M}$ is the past null cone (including the origin). Moreover, $T_pS \equiv \tilde{\nabla}\Omega ^{\perp}$ is a hyperplane in $T_p\tilde{M}$, and hence cannot coincide with the past null cone $N_p^-$. We can then pick $v \in B\cap N_p^- \setminus T_pS$. Then, either $d\Omega_p(v)<0$ or $d\Omega_p(v)>0$. In the first case, however, we'd also have $d\Omega_p(w)<0$ for some past-directed {\em timelike} vector, by continuity. Thus, for some $\delta>0$, the past-directed timelike geodesic $\gamma_w:[0,\delta] \rightarrow \tilde{M}$ with initial velocity $w$ would satisfy $\Omega (\gamma_w(t)) <0$ for each $t \in (0, \delta)$, and hence its final portions would be entirely outside $M$, contradicting Lemma \ref{technical}.
%}

%\cambiosn{
%We conclude that $d\Omega_p(v)>0$. Then, for some $\varepsilon >0$, we will have, for the past-directed null geodesic $\gamma_v:[0,\varepsilon] \rightarrow \tilde{M}$ with initial tangent vector $v$, that $\gamma_v(0,\varepsilon]\subset M$. By standard results in causality theory, $\gamma_v|_{(0,\varepsilon)}$ is a globally achronal (since it is a null generator of the past cone) null pregeodesic in $M$. Therefore, it admits a {\em future-directed} reparametrization as a null, affinely parametrized geodesic ray $\alpha:[0,+\infty) \rightarrow M$ in $(M,g)$, which is the sought-for ray. (It is future-complete by $(i)$, since it has an endpoint at infinity.)
%}
\end{proof}

We now turn to a generalized notion of null infinity. 
%In doing so, we shall take the regular conformal infinity ${\cal J}_c$ as our model. This will also allow us to define a generalized notion of black hole which applies to any strongly causal spacetime.
Let again a strongly causal spacetime $(M,g)$ with $c$-completion $\overline{M}$ and $c$-boundary $\partial M$ be given. Wherever confusion might arise between conformal and $c$-completions, we use lowercase letters $p,q,r, \ldots$ to denote points on $\overline{M}_i$, while pairs $(P,F)$ are used to denote elements of the $c$-completion $\overline{M}$. The chronological future/past of a subset $U$ on $\overline{M}$ (resp. $\overline{M}_i$) will be denoted by $I^{\pm}(U)$ (resp. $I_i^{\pm}(U)$).

% {\em Henceforth, whenever a topological statement is made about the $c$-completion, the CLT $\tau_c$ on $\hat{M}$ is always to be understood.}

% The first issue we must clarify is the following. Given a future-directed causal curve $\alpha:[0,a) \rightarrow M$, what does it mean to say that it will have ``an endpoint on $\partial_{+}M$''?. Given the natural inclusion
% \[
% i:p \in M \hookrightarrow I^-(p) \in \hat{M}
% \]
% whose main properties we have already established (cf. Theorem \ref{thething}), we naturally look at the curve $i\circ \alpha$ in $\hat{M}$. We shall then say, following the standard definition for spacetimes, that $P \in \hat{M}$ is a {\em future endpoint} of $\alpha$ if for any $\hat{U} \ni P$ open set in $\tau_c$, there exists $t_0 \in [0,a)$ such that
% \[
% i\circ \alpha(t) \in \hat{U}, \, \forall t \in [t_0,a).
% \]
% A more convenient characterization is then given as follows.
% \begin{proposition}
% \label{endpointconvenience}
% For a future-directed causal curve $\alpha:[0,a) \rightarrow M$ ($a \leq +\infty$) and $P \in \hat{M}$, the following are equivalent.
% \begin{itemize}
% \item[1)] $P$ is a future endpoint of $\alpha$.
% \item[2)] For any sequence $(t_k)$ in $[0,a)$ converging to $a$, we have $i\circ \alpha(t_k) \rightarrow P$ in $\tau_c$.
% \item[3)] $P = I^-(\alpha)$.
% \end{itemize}
% In particular, if such an endpoint exists, then it is unique, and
% \[
% P \in \partial_+M \Longleftrightarrow \mbox{ $\alpha$ is future-inextendible}.
% \]
% \end{proposition}
% {\em Proof.} The equivalence of $(1)$ and $(2)$ follows immediately from the fact that $\tau_c$ is metrizable, as does the uniqueness of any putative future endpoint. The last claim follows immediately from $(3)$ and Proposition \ref{PipST} (2). Thus, we only need to show the equivalence $(2)\Longleftrightarrow (3)$. Note that this is equivalent to showing, for an arbitrarily given sequence $(t_k)$ in $[0,a)$ converging to $a$, that $\overline{I^-(\alpha)}$ is the Hausdorff closed limit of $I^-(\alpha(t_k))$.

% It is clear that $\limsup (I^-(\alpha(t_k)) \subset \overline{I^-(\alpha)}$. Now, given $x \in \overline{I^-(\alpha)}$, and $U\ni x$ open, pick $x' \in U \cap I^-(x)$ and $t_0 \in [0,a)$ such that $\alpha(t_0) \in U\cap I^+(x') \ni x$. For large enough $k \in \mathbb{N}$, we shall have $t_0 < t_k$, whence
% \[
% x' \ll_g \alpha(t_0) \leq _g \alpha(t_k) \Longrightarrow x'\in I^-(\alpha(t_k)).
% \]
% hence, $I^-(\alpha(t_k))$ intersects $U$ for all large enough $k$, whence we conclude that $x \in \liminf(I^-(\alpha(t_k))$. This concludes the proof.
% \qcd

\begin{definition}
\label{scri}
The {\em future null infinity} of $M$, denoted as ${\cal J}^+$, is the set of points $(P,F) \in \partial  M$ such that
\begin{enumerate}[label=(\roman*)]
\item $\exists$ a future-complete and future-regular null ray $\eta:[0,+\infty) \rightarrow M$ with $(P,F)$ as a future endpoint of $\eta$,

\item $\mbox{every future-inextedible null geodesic with future endpoint $(P,F)$ is fu\-tu\-re-com\-ple\-te.}$
\end{enumerate}
A time-dual analogous definition is immediate for the \emph{past null infinity} ${\cal J}^-$.
\end{definition}

% \cambios{\begin{convention}
%   Along the present section, unless stated otherwise, we will always assume that $(M,g)$ is  \textit{strongly} properly causal Lorentz manifold (see Defn. \ref{def:properly-causal})\footnote{Nota de Jony: Creo que ambos conceptos no son equivalente...REVISAR}.
% \end{convention}}

%\cambiosn{
%Theorems \ref{thm:causaltoconformal}  and \ref{conformalinfinity2} provide the following result:
%\begin{corollary}
%\label{voila}
%Let $(\tilde{M},\tilde{g})$ be a future-nesting conformal extension of a globally hyperbolic spacetime $(M,g)$, and consider the homeomorphism $\Psi:\overline{M}\rightarrow \overline{M}_i^*$ given in Theorem \ref{thm:causaltoconformal} \ref{thmcausaltoconformal-defPsi} taking $\partial M$ onto the accessible conformal boundary $\partial^*_i M$. Then
%\[
%\Psi^{-1}({\cal J}^+_c) \subset {\cal J}^+.
%\]
%\end{corollary}
%\begin{proof}
%  Let $p \in {\cal J}^+_c$. From Theorem \ref{conformalinfinity2} (iii), we know that $p$ is the future endpoint in $\overline{M}_i$ of a future-complete null ray $\eta:[0,+\infty) \rightarrow M$. Let us write $P=I^-(\eta)$ and recall that global hyperbolicity and Thm. \ref{thm:causaltoconformal} \ref{thmcausaltoconformal-defPsi} together give $(P,\emptyset)\in \partial M$ and $\Psi(p)\equiv (P,\emptyset)$. From the definition of chronological limit, it then follows that $(P,\emptyset)$ is the future-endpoint of the curve $\eta$ on $\overline{M}$. Thus, $(P,\emptyset)$ satisfies clause (i) of Definition \ref{scri}. To show that it also satisfies the clause (ii), consider any future-inextendible null geodesic $\alpha:[0,A) \rightarrow M$ ($A\leq +\infty$) such that $I^-(\alpha) = P$. Again, $(P,\emptyset)$ is the future endpoint of $\alpha$ on $\overline{M}$. Since the map in Thm. \ref{thm:causaltoconformal} \ref{thmcausaltoconformal-defPsi} is a homeomorphism, $p$ is also the future endpoint of $\alpha$ on $\overline{M}_i$. But then $\alpha$ must be future-complete by Theorem \ref{conformalinfinity2} (i). Thus, $(P,\emptyset)$ also satisfies clause (ii) as claimed, and the proof is complete.
%\end{proof}
%}

\begin{remark}\label{rem:1}
  \em Condition (i) in the previous definition ensures, in particular, that for any $(P,F)\in \mathcal{J}^{+}$, $P\neq \emptyset$ (recall Proposition \ref{prop:strongfirst}).
\end{remark}
%\begin{remark}
%\label{rmk4}\footnote{\cambiosj{Si no me equivoco, este Remark no se menciona en ningún lado. Creo que lo dejaría como texto fuera de ningún remark (sí necesito el remark anterior).}}
%{\em
Definition \ref{scri} is meant to capture the standard physical notion of ``distant observers'' in the absence of a conformal boundary. This is clearly a {\em geometric} rather than just {\em conformal} notion, as shows up here in the geodesic completeness requirements. It seems that we do not miss the standard points at infinity at least when they are regular and the conformal extension is well-behaved. However, the example in Remark \ref{rmk1} (3) still shows that points at conformal infinity which are {\em not} regular do not need to be in ${\cal J}^+$ (modulo $\Psi$) as defined. Indeed, the point $u=v=\pi/2$ in that example is a point at conformal infinity which is not regular, and the corresponding element in $\partial  M$ is $(M,\emptyset)$ itself (the first component is a TIP in this case!). But $M$ is not the chronological past of any future-directed {\em null} ray (so clause (i) is not satisfied), although it is the past of (infinitely many) future-directed {\em timelike} rays, and clause (ii) in Definition \ref{scri} is trivially satisfied. The reason why we include the clause (i) in Definition \ref{scri} is because we are interested only in the {\em null} infinity here. Clause (ii), on the other hand, seems physically justified if our ``distant observers'' are not to ``see up close'' any (potentially ``fatal''!) ``naked singularity''.
%}
%}
%\end{remark}


%%% Local Variables:
%%% mode: latex
%%% TeX-master: "nullinfinityV5"
%%% End:


\section{Generalized black holes}\label{bh2}
We now use the new notion of null infinity to define a generalized notion of black hole. There is, of course, little choice for this particular definition once we have defined ${\cal J}^+$.

\begin{definition}
\label{defbh} Let $(M,g)$ be a spacetime. We shall say that a point $p \in M$ is {\em visible from (the future null)  infinity} ${\cal J}^+$ if there exists a future-inextendible null geodesic $\alpha$ starting at $p$ \cambios{and with future endpoint} some $(P,F) \in {\cal J}^+$. We denote the set of all points of $M$ visible from infinity by $V_{\infty}$. If ${\cal J}^+ \neq \emptyset$ (or equivalently, if $V_{\infty} \neq \emptyset$), then the {\em black hole (region)} of $(M,g)$ and the {\em (future) event horizon} are, respectively,
\[
B^+ := M\setminus J^-(V_{\infty}),
\]
and
\[
H^+:= \partial B^+.
\]
\end{definition}
\noindent (We require that ${\cal J}^+ \neq \emptyset$ in this definition to avoid that $M = B^+$.)

\begin{remark}
\label{rmk5}
{\em Given any point $p \in V_{\infty}$, and a future-complete null geodesic $\alpha:[0,+\infty) \rightarrow M$ starting
at $p$ with endpoint at ${\cal J}^+$, we have $I^-(\alpha) = I^-(\alpha|_{[t,+\infty)})$ for any $t \in [0,+\infty)$, and therefore any point along $\alpha$ is also visible from infinity.}
\end{remark}

Some of the more basic properties of black holes in this context are summarized in the following proposition. We particularly call the reader's attention to BH4), which precisely clinches the notion of a black hole as a region where causal communication with infinity (i.e., ``distant observers'') is forbidden.
\begin{proposition}
\label{bhprops}
The following properties of the black hole region $B^+$ and the event horizon $H^+$ hold.
\begin{itemize}
\item[BH1)] $I^+(B^+) \subset int(B^+)$. In particular, the interior $int(B^+)$ of the black hole is a future set. Moreover, the event horizon is the common boundary between $int(B^+)$ and the past set $I^-(V_{\infty})$, i.e., $M$ is the disjoint union
\begin{equation}
\label{equality}
M = int(B^+) \dot{\cup} H^+ \dot{\cup}I^-(V_{\infty}).
\end{equation}
In particular, $H^+$ is an {\em achronal boundary} (and hence \cite{ONeillSemiRiemannianGeometryApplications1983,PenroseDifferentialTopology1972} a closed $C^0$ hypersurface) in $M$.
\item[BH2)] For any  $p \in H^+\cap J^-(V_{\infty})$ there exists a future-complete null geodesic ray $\eta \subset H^+$ starting at $p$ with future endpoint at ${\cal J}^+$.
\item[BH3)] $\overline{B^+} = M \setminus I^-(V_{\infty}) = M \setminus I^-({\cal J}^+)$.
\item[BH4)] $p \in M \setminus B^+$ if and only if there exists a future-directed causal curve in $M$ starting at $p$ with a future endpoint on ${\cal J}^+$.
\end{itemize}
\end{proposition}
\begin{proof}
  \textit{BH1):} Let $q,p \in M$, with $q \in B^+$ and $q \ll_g p$. Then,
\[
q \notin J^-(V_{\infty}) \Longrightarrow p \notin  J^-(V_{\infty}) \Longrightarrow p \in B^+,
\]
but this actually proves that $I^+(q) \subset B^+$, and since $I^+(q)$ is open and contains $p$, that $p \in int(B^+)$. Now, $\overline{B^+} = int(B^+)\dot{\cup} H^+$, and
\[
p \notin \overline{B^+} \Longrightarrow \exists U \ni p \mbox{ open with }U \cap B^+ =\emptyset \Longrightarrow U \subset J^-(V_{\infty}) \Longrightarrow p \in I^-(V_{\infty}).
\]
That is, $M \setminus \overline{B^+} \subset I^-(V_{\infty})$. These implications are easily reversed, so also $I^-(V_{\infty}) \subset M \setminus \overline{B^+}$. Therefore, $M\setminus\overline{B}^+=I^{-}(V_{\infty})$, whence (\ref{equality}) follows.

\smallskip

\textit{BH2):} Let $p \in H^+\cap J^-(V_{\infty})$, and let $\alpha$ be a future-directed causal curve segment starting at $p$ and ending at some point $q \in V_{\infty}$. By the definition of visible from infinity, there exists a future-complete null geodesic $\gamma$ starting at $q$ with a future endpoint $(P,F) \in {\cal J}^+$. Let $\eta$ be the juxtaposition of these two curves. This is a future-inextendible causal curve starting at $p$. Now, we claim that its image has to be globally achronal. Otherwise, we could find some $r \in \gamma$ with $p \ll_g r$. But $H^+ \equiv \partial I^-(V_{\infty})$ by BH1), and $r \in J^-(V_{\infty})$ (cf. Remark \ref{rmk5}), so this would mean that $p \in I^-(V_{\infty})\cap \partial I^-(V_{\infty})$, which is impossible. But since $\eta$ is globally achronal, it can be affinely reparametrized as null geodesic ray (which we still call $\eta$). Since $\gamma$ is future-complete, so is $\eta$. Clearly, $(P,F)$ is also an endpoint for $\eta$ which thus has a future endpoint on ${\cal J}^+$, as desired.

\smallskip

\textit{BH3):} The equality $\overline{B^+} = M \setminus I^-(V_{\infty})$ follows immediately from (\ref{equality}). Let $p \in I^-(V_{\infty})$. Then for some $q \in I^+(p)$, there exist a future-directed  null geodesic $\gamma$ starting at $q$ and with an endpoint $(P,F) \in {\cal J}^+$. Moreover, taking into account that $P\neq \emptyset$ (recall Remark \ref{rem:1}), item \ref{item:endpoint3} of Corollary \ref{cor:endpoints} implies that $P=I^{-}(\gamma)$. But this means $p \in P=I^-(\gamma)$ and hence, from \eqref{eq:7}
\[
(I^-(p),I^+(p))\ll (P,F) \in {\cal J}^+,
\]
that is, $(I^-(p),I^+(p)) \in I^-({\cal J}^+)\cap M$. Therefore  we have
\[
I^-(V_{\infty}) \subset I^-({\cal J}^+) \cap M.
\]
Conversely, let $p\equiv (I^-(p),I^+(p)) \ll (P,F) \in {\cal J}^+$. This means that $p \in P=I^-(\gamma)$ for some future-complete  null ray $\gamma$, and so $p\ll_g q$ for some $q \in \gamma$. But then $q$ is visible from infinity (by Remark \ref{rmk5}), so $p \in I^-(V_{\infty})$. This now implies the opposite inclusion \[ I^-({\cal J}^+) \cap M\subset I^-(V_{\infty}), \] and hence $M \setminus I^-(V_{\infty}) = M \setminus I^-({\cal J}^+)$.

\smallskip

\textit{BH4):} Let $\gamma:[0,A) \rightarrow M$ ($A \leq +\infty$) be a future-inextendible causal curve starting at $p =\gamma(0)$ and with a future endpoint $(P,F) \in {\cal J}^+$. Either $\gamma$ is a (necessarily future-complete) null geodesic, in which case $p$ is visible from infinity, or else $p \in I^-(\gamma)$. But then $(I^-(p),I^+(p)) \in I^-({\cal J}^+)$, and hence $p \in I^-(V_{\infty})$ from the proof of \textit{BH3)}. In any case $p \in J^-(V_{\infty})$ and so $p \notin B^+$.

Conversely, if $p \notin B^+$ then there exists a future-complete null geodesic $\gamma:[0,+\infty) \rightarrow M$ with $p \in J^-(\gamma(0))$
such that some $(P,F)\in {\cal J}^+$ is endpoint of $\gamma$. We can therefore juxtapose a future-directed causal curve from $p$ to $\gamma(0)$ with $\gamma$ to obtain a future-inextendible causal curve $\beta$ starting at $p$. Clearly, $(P,F)$ is also endpoint of $\beta$.
\end{proof}

At least in some important cases, one can draw a direct connection between the absence of a black hole region and nonspacelike geodesic completeness.

\begin{proposition}
\label{completeness1}
Suppose that $(M,g)$ is globally hyperbolic and future null complete, with non-compact Cauchy hypersurfaces. Then ${\cal J}^+ \neq \emptyset$ but $B^+ = \emptyset$.
\end{proposition}
\begin{proof}
Let $p \in M$. Assume first that $\partial I^+(p)$ is compact. In this case, a standard argument (see, e.g., the proof of Theorem 61, Ch.14, p. 437 of \cite{ONeillSemiRiemannianGeometryApplications1983}) implies that $\partial I^+(p)$ is homeomorphic to a given Cauchy hypersurface $S\subset M$, which is absurd. Therefore, $\partial I^+(p)$ is non-compact, and again by simple arguments using limit curves (cf. the proof of Prop. 8.18, Ch. 8, p. 289 of \cite{BeemGlobalLorentzianGeometry1996}) we conclude that there exists an inextendible future-directed null geodesic ray $\eta$ starting at $p$. Now, consider the TIP $P=I^{-}(\eta)$. By Remark \ref{r0}, we can pick $F \in \check{M}\cup\{\emptyset\}$ such that $(P,F) \in \overline{M}$ (actually, $F\equiv \emptyset$ will do). Since $P$ is a TIP, $(P,F) \in \partial M$. By Remark \ref{r}, global hyperbolicity guarantees that $\eta$ is future-regular, and Proposition \ref{prop:strongfirst} now implies that $(P,F)$ is a future endpoint of $\eta$. Finally, null geodesic completeness and the future-regularity of $\eta$ now imply that $(P,F) \in {\cal J}^+$. We conclude that $p \notin B^+$ by Proposition \ref{bhprops} BH4).
\end{proof}

If we assume some extra natural geometric conditions on $(M,g)$, we can show that the general conclusion of the previous proposition can be extended to strongly causal spacetimes.

\begin{theorem}
\label{completeness2}
Suppose that the strongly causal spacetime $(M^{n+1},g)$, with $n\geq 2$, satisfies the following conditions:
\begin{itemize}
\item[(a)] $(M,g)$ is timelike and null geodesically complete;
\item[(b)] $(M,g)$ satisfies the {\em timelike convergence condition}, i.e., $Ric(v,v)\geq 0$ for any timelike $v\in TM$;
\item[(c)] ${\cal J}^+ \neq \emptyset$;
  \item[(d)] $\overline{M}$ is strongly properly causal (recall this holds, in particular, if $(M,g)$ is causally continuous - cf. Prop. \ref{prop:causalcontinuity}.)
\end{itemize}
Then $B^+ = \emptyset$.
\end{theorem}
\begin{proof}
Suppose, by way of contradiction, that $(a)-(d)$ do hold, but $B^+ \neq \emptyset$. In this case, Proposition \ref{bhprops} BH1) also implies that ${\rm int} (B^+) \neq \emptyset$; therefore, we can pick $p \in {\rm int} (B^+)$. Now, if there exists a future-directed null ray $\eta$ starting at $p$, from $(a)$ and $(d)$ it must be both future-complete and future-regular. Therefore, arguing exactly as in the end of the proof of Proposition \ref{completeness1} with strong proper causality in lieu of global hyperbolicity, we would conclude that $p \notin B^+$, a contradiction. But if such a ray does not exist, then $p$ is a {\em future trapped set}, i.e., its future horismos $E^+(p):= J^+(p) \setminus I^+(p)$ is compact, again by the proof of  of \cite[Prop. 8.18, Ch. 8, p. 289]{BeemGlobalLorentzianGeometry1996}. Therefore, since $(M,g)$ is strongly causal, the latter result together with \cite[Theor. 8.13]{BeemGlobalLorentzianGeometry1996} imply that $(M,g)$ admits a causal line intersecting $E^+(p)$, $\gamma$ say.

We claim that $\gamma$ is actually a {\em timelike} line. Without loss of generality we may assume that $\gamma$ is future-directed. Now, Proposition \ref{bhprops} BH1) means that $int(B^+)$ is a future set, and hence $E^+(p)$ is contained in ${\rm int}(B^+)$. Therefore, we can pick some $x \in {\rm int}(B^+) \cap \gamma$. If $\gamma$ were null, then its portion to the future of $x$ would be a future-directed null geodesic ray, that can be considered complete and future-regular, that is, (again arguing exactly as in the proof of Proposition \ref{completeness1}) with future endpoint on ${\cal J}^+$, which again contradicts Proposition \ref{bhprops} BH4). Thus, $\gamma$ has to be timelike.

However, conditions $(a)$ and $(b)$, together with the existence of a (complete) timelike geodesic line mean that we can apply the Lorentzian Splitting Theorem (see, e.g., Ch. 14 of \cite{BeemGlobalLorentzianGeometry1996} and references therein) to conclude that $(M,g)$ is actually {\em isometric} to a product spacetime $(\mathbb{R}\times S,-dt^2\oplus h)$, where $(S,h)$ is a complete Riemannian manifold.

Now, Theorem 3.67 in Ref. \cite{BeemGlobalLorentzianGeometry1996} implies that $(M,g)$ is actually globally hyperbolic, with Cauchy hypersurfaces homeomorphic to $S$. Hence $S$ cannot be non-compact, for in that case Proposition \ref{completeness1} would mean that $B^+ \equiv \emptyset$, contrary to our assumption. We conclude that $S$ is compact.

But then no future-inextendible null geodesic can be achronal, i.e., a geodesic ray; so that in this case we would have ${\cal J}^+ \equiv \emptyset$, contradicting $(c)$. This final contradiction ends the proof.
\end{proof}


In order to get deeper results out of our black hole definition, we need to refine it. We shall need some additonal conditions ensuring that future null infinity is ``good enough''. The following definition is meant to encode this.
\begin{definition}
\label{ample}
The future null infinity ${\cal J}^+$ is said to be:
\begin{itemize}
\item[(A1)] {\em Ample} if for any compact set $C \subset M$, and for any connected component ${\cal J}^+_0$ of ${\cal J}^+$,
%${\cal J}^+_0\cap (\overline{M} \setminus \overline{I^+ (C)})$ is non-empty.
${\cal J}^+_0\cap (\overline{M} \setminus \widetilde{I^+ (C)})$ is a non-empty open set, where
   \begin{equation}
\widetilde{I^+(C)}:=\{(P,F)\in \overline{M}: I^-(x)\subset P \mbox{ for some $x\in C$}\}.\label{eq:2}
\end{equation}
(Roughly speaking, no connected component of future null infinity can be entirely contained in the future of a compact set.)

% \item[(A2)] {\em Past-complete} if for any $(P,F) \in {\cal J}^+$ and $(P',F') \in \partial M$ with $\cambiosj{\emptyset\neq} P' \subset P$ \cambiosj{and $F'\cap P=\emptyset$,} we have $(P',F') \in {\cal J}^+$. \cambiosn{(Roughly speaking, any element of the $c$-boundary which is in the horismotic past of ${\cal J}^+$ must also be in ${\cal J}^+$.)}\footnote{\cambiosn{Igualmente, habria sido bonito poder sustituir $P' \subset P$ por $(P',F')\leq (P,F)$, pero, de nuevo, parece que no funciona.}}

\item[(A2)] \emph{Past-complete} if given $(P,F)\in \mathcal{J}^+$, any $(P',F')\in\partial M$ with $P'=I^-(\eta)$, being $\eta$ a future-directed inextendible null geodesic generator of $\partial P$, also belong to $\mathcal{J}^+$.

% \item[(A3)] \textit{Hausdorff-compatible:} if for any pairs $(P,F),(P',F')\in \partial M$ with $(P,F)\in {\cal J}^+$ and $P,P'$ Hausdorff-related, then $(P',F')$ also belong to ${\cal J}^+$. That is, if two boundary points are topologically \textit{close} and one is included in the null infinity, then the other is also included.\footnote{Jony: Esto creo que es fácil de construir. Si lo veis conveniente se puede meter el ejemplo.}

\end{itemize}
We will say that the future null infinity ${\cal J}^+$ is \emph{regular} if it is both ample and past-complete.
\end{definition}
% \begin{remark}
%  INCLUIR ALGUNAS MOTIVACIONES DE LAS DEFINICIONES ANTERIORES.
% \end{remark}
Condition (A1) in the previous definition means that the future of compact sets cannot encopass the whole future null infinity, and (A2) that ${\cal J}^+$ contains any point on the future boundary which may lie in its past. Together, they mean that the future null infinity is "big" in a precise sense. These assumptions are not really restrictive when some classical examples of physical interest are considered. In fact, on the one hand, the condition that the {\em conformal} $\mathcal{J}^+$ escapes from the future of any compact set holds, for instance, in many standard solutions of Einstein field equation, with vanishing cosmological constant, admitting a conformal completion for which conformal null infinity $\mathcal{J}^+$ is a null hypersurface in the extended spacetime having past-complete null generators, such as those in the Kerr-Newman family with suitable parameters. It also holds in some solutions to the Einstein fields equation with a negative cosmological constant, such as the Schwarzschild-Anti de Sitter spacetime, for example. On the other hand, it fails on, say, the Schwarzschild-de Sitter solutions, and will tend to fail, more generally, on globally hyperbolic spacetimes with compact Cauchy hypersurfaces. In any case, the seemingly technical assertion on the open character of the intersection considered in (A1) holds in very general situations, as becomes apparent from the following result:

% \cambiosn{The following result shows that the open condition for ${\cal J}^+_0\cap (\overline{M} \setminus \widetilde{I^+ (C)})$ in (A1) is not very restrictive, since it is guaranteed just if $\hat{M}$ is Hausdorff.}

%An important comparison with the analogous situation for conformal boundaries can be made. For this, we need first the following technical lemma:

\begin{proposition}
\label{lema:auxiliar}
If $\hat{M}$ is Hausdorff, then $\widetilde{I^+(C)}$ is a closed set.
\end{proposition}
\begin{proof}
Let $\left\{(P_n,F_n)  \right\}_{n}\subset \widetilde{I^+(C)}$ be a sequence and consider $(P,F)\in L(\left\{ (P_{n},F_n) \right\}_{n})$. Our aim is to prove that $(P,F)\in \widetilde{I^{+}(C)}$. For any $n$, let $x_{n}\in C$ be a point so $I^-(x_n)\subset P_{n}$. Observe that, as $C$ is compact, we can assume (up to a subsequence) that $\{x_{n}\}_{n}$ converges in $C$ to a point, say $x^{*}\in C$.

From such a convergence it follows (see \cite[Remark 3.17]{Floresfinaldefinitioncausal2011}) that $I^{-}(x^{*})\subset \mathrm{LI} (\left\{ I^-(x_{n}) \right\}_{n})$, and so, that $I^{-}(x^{*})\subset \mathrm{LI} (\left\{ P_n\right\}_{n})$. From standard arguments involving Zorn's Lemma, and up to a subsequence, we can ensure the existence of a IP $\overline{P}$ so $I^{-}(x^{*})\subset \overline{P}$ and $\overline{P}$ is maximal on $\mathrm{LS} (\left\{ P_{n} \right\}_{n})$, i.e., $\overline{P}\in \hat{L}(\left\{ P_{n} \right\}_{n})$. As $\hat{M}$ is Hausdorff and $P\in \hat{L}(\left\{ P_n \right\}_{n})$, $P=\overline{P}\supset I^{-}(x^{*})$, proving that $(P,F)\in \widetilde{I^+(C)}$.
\end{proof}

Condition (A2) in Definition \ref{ample} is comparatively more restrictive than (A1). As we will see on the Appendix, this condition seems unavoidable if we are to obtain the following theorem, insofar as imposing stronger requirements on the causality of the underlying spacetime will not help one to evade it.


%\begin{proposition}
%\label{voila2}
%Let $(\tilde{M},\tilde{g})$ be a future-nesting globally hyperbolic conformal extension of a globally hyperbolic spacetime $(M,g)$. Assume in addition that ${\cal J}^+_c$ is a smooth {\em null} hypersurface in $(\tilde{M},\tilde{g})$ whose null geodesic generators are past-inextendible in $I^+(M,\tilde{M})$. Then:
%\begin{itemize}
%\item[a)] For any compact set $C \subset M$, and for any connected component ${\cal I}^+_0$ of ${\cal J}_c^+$, ${\cal I}^+_0\cap [(\partial_i^+M \cup M) \setminus I^+ (C, \tilde{M})]\neq \emptyset$.
%\item[b)] If $p \in \partial^+M$, $q \in {\cal J}_c^+$, and $p \leq _{\tilde{g}} q$, then $p \in {\cal J}_c^+$.
%\item[c)] If the future null infinity ${\cal J}^+$ of $(M,g)$ is connected, then it is regular.
%\end{itemize}
%\end{proposition}
%
%\begin{proof}
%  $(a)$ Given any $p \in {\cal I}^+_0$, the past-inextendible null generator of ${\cal J}^+_c$ starting at $p$ must eventually leave the compact set $J^+(C,\tilde{M})\cap J^-(p,\tilde{M}) \subset I^+(M,\tilde{M})$ since $(\tilde{M},\tilde{g})$ is strongly causal, and will then leave $I^+ (C, \tilde{M})$ while still contained inside $\partial ^+M$.
%
%  \smallskip
%
%$(b)$ If $p \in \partial^+M$, $q \in {\cal J}_c^+$, and $p \leq _{\tilde{g}} q$, let $\alpha:[0,1] \rightarrow \tilde{M}$ be a past-directed causal curve from $q$ to $p$. Since $\partial^+M$ is achronal (cf. Corollary \ref{convexitygh3}), $\alpha$ must {\em initially} (i.e., near $q$) be, up to reparametrization, a piece of the past-directed null generator of ${\cal J}^+_c$ starting at $q$. Since this is past-inextendible in $I^+(M,\tilde{M})$, it cannot leave the connected component of ${\cal J}^+_c$ containing $q$, and hence $\alpha$ must be {\em entirely} contained in such a generator. Therefore $p \in {\cal J}^+_c$ as well.
%
%\smallskip
%
%\cambiosn{$(c)$ Consider again the homeomorphism $\Psi$ in Theorem \ref{thm:causaltoconformal} taking
%$\overline{M}$ to $\overline{M}_i^{*}$. It follows then that $\overline{M}$ is both, Hausdorff and simple (see \cite[Definition 2.4]{FloresGromovCauchycausal2013}), hence $\hat{M}$ is Hausdorff. In particular, $\widetilde{I^{+}(C)}$ is closed from previous lemma. }
%
%\cambiosn{In order to prove that $\mathcal{J}^{+}$ is regular, we need to show that it is both, ample and past-complete. The latter is quite straightforward once we recall that both $\overline{M}$ and $\overline{M}_{i}^{*}$ are chronologically isomorphic and assertion (b), so we will focus on the ample condition.}
%
%\cambiosn{Let $C$ be a compact set. Observe that we need to prove that
%\begin{equation}
%\mathcal{J}^{+}\cap \left( \overline{M}\setminus \widetilde{I^{+}(C)} \right)\neq \emptyset,\label{eq:5}
%\end{equation}
%as the openness follows from the fact that $\widetilde{I^{+}(C)}$ is closed. Assume by contradiction that \eqref{eq:5} is empty, so $\mathcal{J}\subset \widetilde{I^+(C)}$. Since $\mathcal{J}_{c}^{+}\subset \Psi(\mathcal{J}^{+})$ by Corollary \ref{voila}, from assertion (a) we can prove that
%\begin{equation}
%  \mathcal{J}^{+}\cap \left( \overline{M}\setminus I^{+}(K) \right)\neq \emptyset\label{eq:6}
%\end{equation}
%for any compact set $K$. Let $X$ be a past-directed timelike vector field defined over a neighbourhood $U$ around $C$, and denote by $\varphi_{X}:W\subset U\times \mathbb{R}\rightarrow M$ its associated flow. As $C$ is compact, there exists $t_{0}>0$ so $\varphi^{t_{0}}_{X}(x):=\varphi_{X}(x,t_{0})$ is defined for all $x\in C$.}
%
%\cambiosn{Let us then define $C^{t_{0}}=\varphi^{t_{0}}_{X}(C)$ which is a compact set satisfying, from the timelike character of $X$, that $\widetilde{I^{+}(C)}\subset I^{+}(C^{t_{0}})$. But then $\mathcal{J}^{+}\subset \widetilde{I^+(C)}\subset I^+(C^{t_0})$ and hence
%\[
%  \mathcal{J}^{+}\cap \left( \overline{M}\setminus I^{+}(C^{t_{0}}) \right)= \emptyset,
%  \]
%a contradiction with \eqref{eq:6}.}
%
%\end{proof}


% We are now ready to state and prove the main result in this section.

\begin{theorem}
\label{main}
Assume that ${\cal J}^+$ is regular and let $C \subset M$ be an {\em achronal} compact set. If $C$ is not entirely contained in $B^+$, then there exists a future-complete null $C$-ray $\eta: [0,+\infty) \rightarrow M$ \cambios{with endpoint} $(P,F) \in {\cal J}^+$.
\end{theorem}
\begin{proof}
First, assume that $C$ is not entirely contained in $\overline{B^+}$. Then, BH3) in Proposition \ref{bhprops} implies that ${\cal J}^+\cap I^+ (C)\neq \emptyset$. Let $x \in {\cal J}^+\cap I^+ (C)$, and pick a TIP $P_0$ such that $x \in P_0$ and $(P_0,F_0)\in {\cal J}^+$. Denote by ${\cal J}^+_0$ the connected component of ${\cal J}^+$ containing $(P_0,F_0)$.

\cambios{Since ${\cal J}^+$ is ample, we have
\[
{\cal J}_0^+\cap(\overline{M}\setminus I^+(C))\supset {\cal J}_0^+\cap(\overline{M}\setminus\widetilde{I^+(C)}) \neq \emptyset.
\]
So, taking into account that ${\cal J}^+_0$ is connected, and $(P_0,F_0)\in {\cal J}^+_0\cap I^+ (C)\neq\emptyset$. Thus,
\[
\partial_{{\cal J}^+_0}({\cal J}^+_0\cap I^+ (C))\neq\emptyset.
\]
Take some $(P,F)\in \partial_{{\cal J}^+_0}({\cal J}^+_0\cap I^+ (C))$. Since $I^+ (C)$ is open, it follows that $(P,F) \notin I^+ (C)$ and in particular $C \cap P = \emptyset$.}

We now claim that $I^{-}(x_0)\subset P$ for some $x_0\in C$. Suppose, by way of contradiction, that this is false. Then, it follows that $(P,F)\in U=\mathcal{J}_{0}^{+}\cap\left(\overline{M}\setminus\widetilde{I^{+}(C)}\right)$, which is an open set from the ample condition. As $I^{+}(C)\subset \widetilde{I^{+}(C)}$, then $I^{+}(C)\cap U=\emptyset$. Hence, the point $(P,F)$ belongs to an open set with empty intersection with $I^{+}(C)$, in contradiction with $(P,F)\in\partial_{{\cal J}^+_0}(\mathcal{J}^+_{0}\cap I^+ (C))$. This establishes the claim.

Therefore, $x_0 \in \overline{I^{-}(x_0)}\subset \overline{P}$, but $x_0\in C\setminus P$. We conclude that $x_0 \in\partial_{M} P$. Since $(P,F) \in \partial M$, in particular $P$ is a TIP. Its boundary in $M$ is thus a union of future-inextendible null geodesics, so that we can take $\eta$ the future-inextendible null geodesic generator of $\partial_{M} P$ starting at $x_0$. Note that since $I^-(\eta)$ is also a TIP, we conclude that $(P',F') \in {\cal J}^+$ with $P'=I^-(\eta)$ by clause (A2) in Definition \ref{ample}, and in particular $\eta$ must be future-complete by clause (ii) in Definition \ref{scri}.

We wish to show that $\eta$ is a $C$-ray. By construction, $\eta \subset \overline{I^+(C)}$, but since $\partial_{M} P \cap I^+(C) = \emptyset$, $\eta \subset \partial_{M} I^+(C)$. Moreover, since $C$ is achronal, $C \subset \partial_{M} I^+(C)$. Finally, since  $\partial_{M} I^+(C)$ is an achronal set, any causal curve segment connecting a point of $C$ with a point $x$ (say) along $\eta$ must be a reparametrization of a null geodesic, and in particular has zero Lorentzian arc-length. This means that the initial segment of $\eta$ between $C$ and $x$ is maximal. Hence, $\eta$ is a $C$-ray as claimed.

Now, assume that $C$ is contained in $\overline{B^+}$, but not in $B^+$. We can then pick a point $p \in H^+\cap C\cap J^-(V_{\infty})$. By the item BH2) of Proposition \ref{bhprops}, there exists a future-complete null geodesic ray, which we again denote by $\eta$, starting at $p$ and with future endpoint on ${\cal J}^+$. We only need to check it is again a $C$-ray. But if this were not the case, then there would exist some $q \in C$ and some point $r$ along $\eta$ with $q \ll_g r$. But since $r$ is visible from infinity (cf. Remark \ref{rmk5}), this would mean that $q \in I^-(V_{\infty}) \equiv M\setminus \overline{B^+}$ (cf. BH3) in Proposition \ref{bhprops}), a contradiction.
\end{proof}



%\begin{proof}
%First, assume that $C$ is not entirely contained in $\overline{B^+}$. Then, because of BH3) on Proposition \ref{bhprops}, ${\cal J}^+\cap I^+ (C)\neq \emptyset$. Let $x \in {\cal J}^+\cap I^+ (C)$, and pick a TIP $P_0$ such that $x \in P_0$ and $(P_0,F_0)\in {\cal J}^+$. Denote by ${\cal J}^+_0$ the connected component of ${\cal J}^+$ containing $(P_0,F_0)$.
%
%\cambios{Since ${\cal J}^+$ is ample, we have
%\[
%{\cal J}_0^+\cap(\overline{M}\setminus I^+(C))\supset {\cal J}_0^+\cap(\overline{M}\setminus\widetilde{I^+(C)}) \neq \emptyset.
%\]
%So, taking into account that ${\cal J}^+_0$ is connected, necessarily
%\[
%{\cal J}^+_0\cap \partial _{\overline{M}}I^+ (C)\neq\emptyset.
%\]
%Take some $(P,F)\in {\cal J}^+_0\cap \partial _{\overline{M}}I^+ (C)$. Since $I^+ (C)$ is open, it follows that $(P,F) \notin I^+ (C)$ and in particular $C \cap P = \emptyset$.}
%
%\cambios{Let us prove that $I^{-}(x_0)\subset P$ for some $x_0\in C$. Assume by contradiction that this does not hold. Then, from the definition of $\widetilde{I^+ (C)}$,
%\[
%(P,F)\in {\cal J}^+_0\cap (\overline{M} \setminus \widetilde{I^+ (C)}).
%\]
%But recall that ${\cal J}^+_0\cap (\overline{M} \setminus \widetilde{I^+ (C)})$ is assumed to be open. Moreover, it satisfies
%\[
%({\cal J}^+_0\cap (\overline{M} \setminus \widetilde{I^+ (C)}))\cap I^{+}(C)\subset ({\cal J}^+_0\cap (\overline{M} \setminus I^+ (C)))\cap I^{+}(C)=\emptyset.
%\]
%So, we have found an open set ${\cal J}^+_0\cap (\overline{M} \setminus \widetilde{I^+ (C)})$ that contains $(P,F)$ and does not intersects $I^{+}(C)$. This contradicts that $(P,F)\in\partial_{\overline{M}}I^+(C)$. In conclusion, some $x_{0}\in C\setminus P$ exists such that $I^-(x_0)\subset P$.}
%
%
%%\cambios{Now, since ${\cal J}^+$ is ample,  ${\cal J}^+_0\cap (\overline{M} \setminus \widetilde{I^+ (C)})\neq \emptyset$, and so ${\cal J}^+_0\cap \partial_{\overline{M}}I^+ (C)$\footnote{Along this proof, we will make use of a subindex $_{\overline{M}}$ to emphasize that such a topological object is considered on $\overline{M}$.} is non empty. Observe that, for any $(P,F) \in {\cal J}^+_0\cap \partial _{\overline{M}}I^+ (C)$, since $I^+ (C)$ is open, it follows that $(P,F) \notin I^+ (C)$ and in particular $C \cap P = \emptyset$.}
%%
%%\cambios{
%%Let us take a point $(P,F)\in {\cal J}^+_0\cap \partial _{\overline{M}}I^+ (C)$ and observe that, necessarily, $I^-(x_0)\subset P$. Otherwise,from definition of $\widetilde{I^+(C)}$ \eqref{eq:2},
%%\[
%%(P,F)\in {\cal J}^+_0\cap (\overline{M} \setminus \widetilde{I^+ (C)}).
%%\]
%%But recall that ${\cal J}^+_0\cap (\overline{M} \setminus \widetilde{I^+ (C)})$ is assumed to be open. Moreover, it satisfies
%%\[
%%({\cal J}^+_0\cap (\overline{M} \setminus \widetilde{I^+ (C)}))\cap I^{+}(C)\subset ({\cal J}^+_0\cap (\overline{M} \setminus I^+ (C)))\cap I^{+}(C)=\emptyset.
%%\]
%%So, we have found an open set ${\cal J}^+_0\cap (\overline{M} \setminus \widetilde{I^+ (C)})$ that contains $(P,F)$ and does not intersects $I^{+}(C)$, a contradiction with the assumption that $(P,F)\in\partial_{\overline{M}}I^+(C)$.}
%
%% As $\hat{M}$ is Hausdorff and $(P,F)$ has $P\neq \emptyset$, Prop. \ref{prop:existenceSequence} ensures the existence of a sequence $\{(P_k,F_k)\}_{k \in \mathbb{N}}\subset I^+(C)$ and such that $P\in L(\{P_k\}_k))$. %In particular,
%% %\[
%% %\overline{R} = \liminf R_k = \limsup R_k.
%% %\]
%% Pick $x_k \in P_k \cap C$ for each $k \in \mathbb{N}$. By compactness of $C$ we can assume the $x_k \rightarrow x_0$, up to passing to a subsequence, for some $x_0 \in C$ satisfying that $I^-(x_0)\in \liminf P_k$. Now observe that Zorn's Lemma allow us to ensure the existence of another IP $P'$ (maybe $P$) so $I^-(x_0)\subset P'$ and it is maximal on $\limsup P_k$, i.e., $P'\in \hat{L}(\{P_k\}_k)$. Hence both, $P$ and $P'$ are Hausdorff related and from the Hausdorff-compatibility of ${\cal J}^+$, it follows that $(P',F')\in {\cal J}^+$. Moreover, $(P',F')\in {\cal J}^+_0\cap \partial I^+ (C)$ (where this boundary is taken on $\overline{M}$), so $P'\cap C=\emptyset$ and hence $x_0\notin P'$.
%
%Therefore, we can take $\eta$ the future-inextendible null geodesic generator of $\partial_{\overline{M}} P$ starting at $x_0\in C\setminus P$. Note that since $I^-(\eta)$ is necessarily a TIP and $I^-(\eta) \subseteq P$,  we conclude that $(P',F') \in {\cal J}^+$ with $P'=I^-(\eta)$ by clause (A2) in Definition \ref{ample}, and in particular $\eta$ must be future-complete by clause (ii) in Definition \ref{scri}.
%
%By construction, $\eta \subset \overline{I^+(C)}_{\overline{M}}$, but since $\partial_{\overline{M}} P \cap I^+(C) = \emptyset$, $\eta \subset \partial_{\overline{M}} I^+(C)$. Moreover, since $C$ is achronal, $C \subset \partial_{\overline{M}} I^+(C)$; hence, $\eta$ is achronal, and thus, a $C$-ray.
%
%Now, assume that $C$ is contained in $\overline{B^+}$, but not in $B^+$, so we can pick a point $p \in H^+\cap C\cap J^-(V_{\infty})$. By BH2) on Proposition \ref{bhprops}, there exists a future-complete null geodesic ray $\eta$ starting at $p$ (and with future endpoint on ${\cal J}^+$). We only need to check it is indeed a $C$-ray. But if this were not the case, then there would exist some $q \in C$ and some point $r$ along $\eta$ with $q \ll_g r$. But since $r$ is visible from infinity (cf. Remark \ref{rmk5}), this would mean that $q \in I^-(V_{\infty}) \equiv M\setminus \overline{B^+}$ (cf. BH3) on Proposition \ref{bhprops}), a contradiction.
%\end{proof}


% \begin{remark}
%   The technical condition on $L(I^+(C))$ ensures that all the points on $\partial I^+(C)$ are $L$-limits of sequences contained in $I^+(C)$. It follows immediately if for instance the topology is AN1, which follows if the chronological topology is Hausdorff\footnote{Jony: Yo aqui haría una referencia al otro paper, donde si la topología es Hausdorff entonces se tiene que es metrizable...}
% \end{remark}

We can now use this theorem to prove a classic result in the theory of black holes \cite{BeemGlobalLorentzianGeometry1996,HawkingLargeScaleStructure1975,ONeillSemiRiemannianGeometryApplications1983}  for this extended context. Specifically, we wish to show that {\em any closed trapped surface stays inside the black hole region}, or, in other words, ``hidden from distant observers'' by the event horizon.

Recall that a {\em closed (future) trapped surface} in $(M,g)$ is a smooth, codimension 2, spacelike, achronal, compact submanifold $S \subset M$ without boundary whose mean curvature vector field is everywhere past-directed timelike \cite{ONeillSemiRiemannianGeometryApplications1983}. (The presence of this geometric object was introduced by Penrose \cite{PenroseGravitationalCollapse1965} as the mathematical surrogate for ``a point of no return'' in gravitational collapse.)  Also, recall that the {\em null convergence condition} holds in $(M,g)$ when $Ric(v,v) \geq 0$ for each null $v \in TM$. For solutions of the Einstein field equation of General Relativity (with or without a cosmological constant), this condition is implied by all the standard ``energy conditions'' on the stress-energy tensor, such as the dominant energy condition or the weak energy condition \cite{HawkingLargeScaleStructure1975,WaldGeneralRelativity1984}.

\begin{corollary}
\label{trappedcor}
Assume that ${\cal J}^+$ is regular and that the null convergence condition holds in $(M,g)$. If $S \subset M$ is a closed trapped surface, then $S \subset B^+$.
\end{corollary}
\begin{proof}
Suppose, to the contrary, that $S$ is not contained in $B^+$. Since $S$ is in particular achronal and compact by definition, Theorem \ref{main} implies the existence of some future-complete null $S$-ray $\eta: [0,+\infty) \rightarrow M$ with future endpoint $(P,F)\in {\cal J}^+$. But $\eta$ would be then a future-complete normal null geodesic ray starting at $S$ and without focal points, which contradicts standard results for closed trapped surfaces in spacetimes where the null converge condition holds (see, e.g. \cite{BeemGlobalLorentzianGeometry1996,HawkingLargeScaleStructure1975,ONeillSemiRiemannianGeometryApplications1983}).
\end{proof}

Finally, if ${\cal J}^+$ is regular, we can strengthen the item BH4) of Proposition \ref{bhprops} as follows.
\begin{corollary}
  Assume that ${\cal J}^+$ is regular. Then $p \in M \setminus B^+$ if and only if $p$ is visible from infinity.
\end{corollary}
\begin{proof}
Immediate from Theorem \ref{main} by taking $C= \{p\}$.
\end{proof}

%%% Local Variables:
%%% mode: latex
%%% TeX-master: "nullinfinityV5.tex"
%%% End:


\section{Application to generalized plane waves}\label{ppwaves}

We are going to study in this section the null infinity of, and prove the absence of black holes in, the class of {\em generalized plane waves}. This family of spacetimes has been intensely studied in the literature, and is especially relevant for us here because conformal boundaries are not always available therein, but $c$-boundaries do, since they are strongly causal under mild assumptions \cite[Section 3]{0264-9381-20-11-322}. Hence, definitions of null infinity and black holes based on the latter become a natural alternative. In that case, the classical notions of null infinity and/or black holes do not make sense. (For a different perspective on the same issue, see \cite{SenovillaNoBHinPPWaves2003}.) Let us begin by reviewing some generalities about these spacetimes.

A {\em generalized plane wave} is any spacetime $(M^{n+1},g)$ of the form
\begin{equation}\label{pfw}
M=M_0 \times \mathbb{R}^2,\qquad
g(\cdot,\cdot) = g_0(\cdot,\cdot) + 2dudv + H(x,u)du^2,
\end{equation}
where $g_0$ is a smooth Riemannian metric on the ($n-1$)-dimensional manifold $M_0$, the variables $(v,u)$ are the standard Cartesian coordinates of $\mathbb{R}^2$, and $H:M_0 \times \mathbb{R} \rightarrow \mathbb{R}$ is some smooth real function.

The vector field $\partial_v$ is parallel (i.e.,
covariantly constant) and null, and the time-orientation will
always be chosen which makes it past-directed. In particular, its integral curves are past-directed null geodesics.

Consider any future-directed
causal curve segment $t \in [a,b] \mapsto \gamma(s)=(x(s),v(s),u(s))$ in $(M,g)$. Since the gradient $\nabla \,u \equiv \partial_v$ is past-directed null, we have
\begin{equation}
\label{cute}
\dot{(u\circ \gamma)}(s) = g(\nabla \, u (\gamma(s)), \dot \gamma(s)) = g(\partial_v|_{\gamma(s)},\dot \gamma(s)) = \dot u(s) \geq 0,
\end{equation}
and the inequality is strict whenever $\dot \gamma(s)$ is timelike. Integrating Eq. (\ref{cute}), we get
\[
u(b) - u(a)\geq 0,
\]
with strict inequality unless $\gamma$ is a null pregeodesic without conjugate points contained in a $u\equiv\hbox{constant}$ hypersurface. If the latter is not the case, then $\gamma$ leaves any such hypersurface where it starts, and in particular $\gamma$ cannot be a closed curve.

This calculation reveals, in particular, that each $u\equiv\hbox{constant}$ hypersurface is null and achronal and $(M,g)$ does not admits any closed timelike curves, i.e., it is chronological. The null geodesic generators of any such hypersurface coincide with the maximal integral curves of $\partial_{v}$ therein, and achronality implies that any null geodesic inside these hypersurfaces coincide with such generators.

Finally, a closer analysis of the geodesic equations easily shows that any null geodesic inside a $u\equiv\hbox{constant}$ hypersurface is {\em injective} and {\em complete}. We conclude that {\em every generalized plane wave is causal}; moreover, the vector $\partial_v$ is complete and all its maximal integral curves are null geodesic lines.


Fixing some local coordinates $x_1, \dots, x_{n-1}$ for the Riemannian
part $M_0$,
%it is straightforward to compute the
%Christoffel symbols of $g$, and
%
%thus,
%to relate the Levi-Civita connections $\nabla, \nabla^0$ for $M$
%and $M_0$, respectively (see \cite{CFSgrg}). We remark the following
%facts:
%\begin{itemize}
%\item $M_0$ is totally geodesic, i.e., $\nabla_{\partial i}
%\partial_j = \nabla^0_{\partial_{i}} \partial_j $, $i,j=1, \dots, n-1$.
%\item The non-zero curvature coefficients are:
%\[
%R^i_{jkl} = (R_{g_0})^i_{jkl}, \quad R^i_{uuk} = \frac{1}{2}(g_0)^{il}(Hess_0H)_{lk}, \quad R^v_{iuj} = -\frac{1}{2}(Hess_0H)_{ij},
%\]
%where we have used $i,j,k,l, \ldots$ for the spatial coordinates, and $Hess_0H$ denotes the Hessian of $H = H( . , u)$ with respect to the metric $g_0$.
%
%\item The Ricci tensor
%$Ric$ of $(M,g)$ and ${\rm
%Ric}^{0}$ of $(M_0,g_0)$ satisfy
%\[
%Ric = \sum_{i,j=1}^{n-1} R^{0}_{i j} d x_i \otimes d x_j
%-\frac{1}{2}\Delta_{x}H du \otimes du .
%\]
%Thus, $Ric$ is zero if and only if both the Riemannian
%Ricci tensor Ric$^{0}$ and the transverse Laplacian $\Delta_{0}H$
%vanish.
%\end{itemize}
%
%From the direct computation of Christoffel symbols of a generalized plane wave, it is straightforward
%
%to write the geodesic equations
%in local coordinates. Remarkably,
the three geodesic equations for
a curve $\gamma(s)= (x(s), v(s), u(s))$, $s\in (a,b)$, can be solved in
the following three steps  \cite[Proposition 3.1]{CandelaGeneralPlaneFronted2003}:
\begin{enumerate}
\item[(a)] $u(s)$ is any affine function, $u(s) = u_0 + s \Delta
u$, for some constant $\Delta u\in {\mathbb R}$;

\item[(b)] $x = x(s)$ is a solution on $M_0$ of
\[
D_s\dot x = - {\rm grad}_x V_{\Delta}(x(s),s) \quad \mbox{for all
$s \in \ (a,b)$,}
\]
where $D_s$ denotes the covariant derivative and $V_{\Delta}$ is
defined as:
\[
V_{\Delta}(x,s) := -\ \frac{(\Delta u)^2}{2}\ H(x, u_0 + s \Delta
u);
\]

\item[(c)] finally, with a fixed $v_{0}$ and an $s_0\in (a,b)$,
$v(s)$ can be computed from
\[
v(s) = v_0 + \frac{1}{2 \Delta u} \int_{s_0}^s \left( E_{\gamma} -
g_0(\dot x(\sigma), \dot x(\sigma)) + 2
V_{\Delta}(x(\sigma), \sigma)\right) d\sigma,
\]
where $E_{\gamma}=g(\dot{\gamma}(s), \dot{\gamma}(s))$ is a
constant (if $\Delta u = 0$ then $v = v(s)$ is affine).
\end{enumerate}
In particular, if we fix some $(\overline{x},\overline{u})\in M_0\times {\mathbb R}$, then the curve
\[
\gamma_{\overline{x},\overline{u}}(s)=(\overline{x},-s,\overline{u})\quad\hbox{(resp. $\beta_{\overline{x},\overline{u}}(s)=(\overline{x},s,\overline{u}))$,}\quad s\in {\mathbb R},
\]
is a future-directed (resp. past-directed) null geodesic line.

The $c$-boundary for generalized plane waves has been systematically studied in \cite{Florescausalboundarywavetype2008}, where it was shown that its structure strongly depends on the growth of $H$ at infinity. Some of these asymptotic behaviours, which will be also relevant in our discussion, are the following:
\begin{definition} A function ${\cal F}:M_0\times{\mathbb R}\rightarrow {\mathbb R}$ is said to be:
\begin{itemize}
%\item[(i)] {\em superquadratic} if $M_0$ is unbounded and contains a sequence of points $\{x_m\}_m\in  M_0$ and $\hat{x}\in M_0$ such that
%$d(x_m,\hat{x})\rightarrow\infty$ and
%\[
%R_1 d(x_m,\hat{x})^{2+\epsilon}+R_0\leq F(x_m,u)\quad\forall u\in {\mathbb R},
%\]
%for some $\epsilon, R_1, R_0\in {\mathbb R}$ with $\epsilon, R_1 > 0$.
%
\item[(1)] {\em at most quadratic}, if there exist $\hat{x}\in M_0$ and positive continuous functions $R_0(u), R_1(u) > 0$ such that
\[
{\cal F}(x,u)\leq R_1(u)d(x,\hat{x})^2 + R_0(u),\quad\forall (x,u)\in M_0\times {\mathbb R};
\]
\item[(2)] {\em $\lambda$-asymptotically quadratic, with} $\lambda>0$, if $M_0$ is non-compact and there
exist $\hat{x}\in M_0$, continuous functions $R_0(u), R_1(u)>0$ and a constant $R_0^-\in {\mathbb R}$ such that:
\[
\frac{\lambda^2d(x,\hat{x})^2+R_0^-}{u^2+1}\leq {\cal F}(x,u)\leq R_1(u)d(x,\hat{x})^2 + R_0(u),\quad\forall (x,u)\in M_0\times {\mathbb R}.
\]
%\item[(ii)] {\em subquadratic} if there exist $\hat{x}\in M_0$ and continuous functions $R_0(u), R_1(u)(\geq 0)$,
%$p(u)<2$ such that:
%\[
%F(x,u)\leq R_1(u)d^{p(u)}(x, \hat{x}) + R_0(u),\quad\forall (x,u)\in M_0\times {\mathbb R}.
%\]
\end{itemize}
\end{definition}
Let us recall now some noteworthy statements about the structure of the $c$-boundary for generalized plane waves (see \cite[Theorems 7.9 and 8.2]{Florescausalboundarywavetype2008}; note that $F$ in that reference corresponds with $-H$ here). From now on, we will assume that $(M_0,g_0)$ {\em is geodesically complete}:
\begin{theorem}\label{t} Let $(M,g)$ be a generalized plane wave as in (\ref{pfw}). Then, the following assertions hold.
\begin{itemize}
%\item[(i)] If $-H$ is superquadratic and $H$ is at most quadratic then $(M,g)$ is neither future nor past-distinguishing. In particular, it does not admit a c-boundary.
\item[(i)] If $|H|$ is at most quadratic, then the future (resp. past) $c$-boundary of $(M,g)$ {\em contains} a
%locally lightlike\footnote{Ivan: Eso que es??? No creo que lo hayan definido....}
copy $L^+$ (resp. $L^-$) of ${\mathbb R}$ plus the ideal point $i^+$ (resp. $i^-$)\footnote{Here, by $i^{+}$ and $i^{-}$ we are denoting the pairs $(M,\emptyset)$ and $(\emptyset,M)$, and the entire manifold $M$ is a terminal set.}. Every ideal point $\overline{u}\in L^{+}$ (resp. $\overline{u}'\in L^{-}$) can be identified with the IP $I^{-}(\gamma_{\overline{x},\overline{u}})$ for any $\overline{x}\in M_0$ (resp. the IF $I^{+}(\gamma_{\overline{x}',\overline{u}'})$, for any $\overline{x}'\in M_0$).

    The (total) $c$-boundary of $(M,g)$ {\em contains} a subset which can be identified with the quotient space $$((L^{+}\cup \{i^{+}\})\cup (L^{-}\cup \{i^{-}\}))/R,$$ where $R$ is the equivalence relation obtained by symmetrizing the following relation:\footnote{From \cite[Remark 7.10]{Florescausalboundarywavetype2008}, a pair $(\overline{x},\overline{u})$ cannot be related with more than one pair $(\overline{x}',\overline{u}')$, and viceversa.}
    \[
    (\overline{x},\overline{u})R(\overline{x}',\overline{u}')\quad\Leftrightarrow\quad (I^{-}(\gamma_{\overline{x},\overline{u}}),I^{+}(\gamma_{\overline{x}',\overline{u}'}))\in\overline{M}.
    \]
    %with eventual identifications between their points originated by the eventual formation of pairs\footnote{???????????????????} of the form $(I^{-}(\gamma_{\overline{x},\overline{v}}),I^{+}(\gamma_{\overline{x}',\overline{v}'}))$, for some $\overline{v}<\overline{v}'$.

\item[(ii)] If $-H$ is $\lambda$-asymptotically quadratic for some
$\lambda>1/2$, then the future, past and total $c$-boundaries not only contains the structures described in (i), but necessarily {\em coincide} with them.

%\item[(iii)] In the limit case $\lambda=1/2$ there exists an explicit example with higher dimensional boundary, showing that the $1$-dimensional c-boundary can no longer be expected for $\lambda\leq 1/2$.

%\item[(ii)] If $-H$ subquadratic and $(M_0,g_0)$ is complete, then $(M,g)$ is globally hyperbolic. In this case, the structure of the spacetime suggests a c-boundary with two pieces which resemble in some sense the Cauchy hypersurfaces.
    %--notice that the Cauchy hypersurfaces are necessarily noncompact and, at least when $M$ is noncompact, one could expect that some portion of boundary were higher dimensional, even of dimension
%$(n + 1)$.

%\item[(v)] For plane waves with $-H$ having negative eigenvalues the spacetime is conformal to a region of
%${\mathbb L}^{n+2}$ bounded by two lightlike hyperplanes. Nevertheless, the causal and conformal boundaries differ in this case: the former has two connected
%pieces (a future boundary and a past one); the latter, which is necessarily compact, is connected and includes implicitly properties at spacelike infinity.
\end{itemize}
\end{theorem}

%\begin{remark} {\em In previous theorem, the set $L^{+}\cup \{i^{+}\}$ corresponds with (part of) the future c-boundary of $(M,g)$, and every ideal point $\overline{v}\in L^{+}$ is associated to the IP $P=I^{-}(\gamma_{\overline{x},\overline{v}})$, for any $\overline{x}\in M_0$. Analogously, the set $L^{-}\cup \{i^-\}$ corresponds with (part of) the past c-boundary, and every ideal point $\overline{v}'\in L^{-}$ is associated to the IF $F=I^{+}(\gamma_{\overline{x}',\overline{v}'})$, for any $\overline{x}'\in M_0$. The identifications cited above correspond to the eventual formation of pairs of the form $(I^{-}(\gamma_{\overline{x},\overline{v}}),I^{+}(\gamma_{\overline{x}',\overline{v}'}))$, for some $\overline{v}<\overline{v}'$.}
%\end{remark}


Now, we are ready to obtain the (future) null infinity for these spacetimes according to Definition \ref{scri}. To simplify the study, we will restrict our attention to {\em geodesically complete} generalized plane waves. This property is guaranteed, for instance, if $H(x,u)\equiv H(x)$ is at most quadratic (see \cite[Corollary 3.4]{CandelaGeneralPlaneFronted2003}), but there are pretty more situations were it holds. We will also assume that the null rays $\gamma_{\overline{x},\overline{u}}$ are future-regular.
%\footnote{\cambiosn{En realidad, solo necesitamos que se verifique la propiedad $\uparrow \gamma_{\overline{x},\overline{u}}=\uparrow I^{-}(\gamma_{\overline{x},\overline{u}})$. Creo que si nos restringimos a esta propiedad, podemos eliminar la condicion de "properly causal" de todo el paper, puesto que solo la necesitabamos para los resultados que vienen. Esta propiedad se verifica con bastante generalidad, y debe ser facil de comprobar en ejemplos concretos.}}
%\cambiosn{We will also assume that it is strongly properly causal, which again is guaranteed if $H(x,u)\equiv H(x)$.}
\begin{corollary} Let $(M,g)$ be a geodesically complete generalized plane wave whose null rays $\gamma_{\overline{x},\overline{u}}$ are future-regular. Then, the following assertions hold:
\begin{itemize}
\item[(i)] If $|H|$ is at most quadratic, then ${\cal J}^+$ {\em contains} all the pairs of the form $(P,F)$, with $P\neq\emptyset$, which appear in case (i) of Theorem \ref{t}.
%and ${\cal J}^-$ contain copies $L^+$ and $L^-$ of ${\mathbb R}$, resp.
%If $-H$ is $\lambda$-asymptotically quadratic for some $\lambda>1/2$, and $(M_0,g_0)$ is complete, then the null infinity ${\cal J}$ coincides with the c-boundary with the ideal points $i^{+}$, $i^{-}$, removed; that is, ${\cal J}$ is formed by two copies $L^+$, $L^-$ of ${\mathbb R}$, with eventual identifications between the points of the copies. Moreover, in this case ${\cal J}$ is ample.
\item[(ii)] If $-H$ is $\lambda$-asymptotically quadratic for some
$\lambda>1/2$, then ${\cal J}^+$ not only contains, but also {\em coincides} with the structure described in previous case (i).
%If $-H$ subquadratic and $(M_0,g_0)$ is complete, then ${\cal J}$ is the disjoint union of ${\cal J}^+$ and ${\cal J}^{-}$, and ${\cal J}^+$ and ${\cal J}^-$ contain copies $L^+$ and $L^-$ of ${\mathbb R}$, resp.
\end{itemize}
\end{corollary}

\begin{proof}
(i) Recall that $\gamma_{\overline{x},\overline{u}}=(\overline{x},-s,\overline{u})$, is a future-directed null geodesic line for every $(\overline{x},\overline{u})\in M_0\times {\mathbb R}$. Since $\uparrow \gamma_{\overline{x},\overline{u}}=\uparrow I^-(\gamma_{\overline{x},\overline{u}})$, the curve $\gamma_{\overline{x},\overline{u}}$ has a future endpoint of the form $(I^-(\gamma_{\overline{x},\overline{u}}),F)$, where either $F=I^+(\gamma_{\overline{x}',\overline{u}'})$ or $F=\emptyset$.
Moreover, since $(M,g)$ is assumed to be geodesically complete, any other inextendible future-directed null geodesic $\alpha$ with future endpoint $(I^-(\gamma_{\overline{x},\overline{u}}),F)$ is complete. Hence, $(I^{-}(\gamma_{\overline{x},\overline{u}}),F)$ belongs to ${\cal J}^+$ for every $(\overline{x},\overline{u})\in M_0\times {\mathbb R}$.
%For ${\cal J}^{-}$ we reason similarly. The ample character of ${\cal J}$ is direct from the definition.

The argument for the case (ii) is totally analogous.
\end{proof}

\noindent As a direct consequence we can now deduce the absence of BH for these spacetimes.
\begin{corollary}\label{cc}
 If $(M,g)$ is a geodesically complete generalized plane wave whose null rays $\gamma_{\overline{x},\overline{u}}$ are future-regular, then it does not contain black holes. In particular, causally continuous, geodesically complete generalized plane waves have no black holes.
 %the thesis follows for any causally continuous, geodesically complete generalized plane waves.}
\end{corollary}

\begin{proof}
Given an arbitrary event $p_0=(x_0,v_0,u_0)\in M$, it suffices to show that $p_0\in I^-({\cal J}^+)$. From the proof of the previous theorem, $(I^{-}(\gamma_{\overline{x},\overline{u}}),F)\in {\cal J}^+$ for all $(\overline{x},\overline{u})\in M_0\times {\mathbb R}$. Moreover, $p_0=(x_0,v_0,u_0)\in I^-((I^{-}(\gamma_{\overline{x},\overline{u}}),F))$ if and only if $u_0<\overline{u}$. Hence, $p_0\in I^-({\cal J}^+)$ whenever $u_0<\overline{u}$. In conclusion, $M\subset I^-({\cal J}^+)$, and thus, $B^+\subset M\setminus I^-({\cal J}^+)=\emptyset$.

For the last assertion, just recall that any causally continuous spacetime is strongly properly causal (Proposition \ref{prop:causalcontinuity}), and thus, the null rays $\gamma_{\overline{x},\overline{u}}$ are future-regular (Definition \ref{def:sproperlycausal}).
\end{proof}

\begin{remark} {\em In \cite[Thm. 6.9]{EHRLICH_1992} (see also \cite{minguzzi12:causal_kam}) the authors provide mild conditions under which a plane wave is causally continuous. So, according to Corollary \ref{cc}, these conditions joined to the previously cited ones for geodesic completeness ensure that a plane wave has no black holes.
}
\end{remark}



%\cambiosj{As a final comment on this section, we note that there exist (non trivial) causally continuous generalized plane waves. In fact, we recall the following result of Ehrlich and Emch (see \cite[Thm. 6.9]{EHRLICH_1992}; and also \cite{minguzzi12:causal_kam} for further conditions on causal continuity.)}
%
%\cambiosj{\begin{theorem}
%Let $(M,g)$ be a generalized plane wave as in \eqref{pfw} satisfying:
%\begin{enumerate}[label=(\alph*)]
%	\item $M=\mathbb{R}^4$ and $g_0$ is the $2$-dimensional Euclidean metric.
%	\item The function $H$ takes the form \[H(y,z,u)=f(u)(y^2-z^2)+2g(u)yz\] for $f,g\in \mathcal{C}^2(\mathbb{R})$ two real functions satisfying that
%	\[\int_{-\infty}^{\infty}\left(f^2(u)+g^2(u)\right)du<\infty.\]
%	\item $(M,g)$ is a vacuum solution.
%\end{enumerate}
%Then, $(M,g)$ is causally continuous.
% \end{theorem}}



%\cambiosj{OJO: TODO ESTO DEBE DESAPARECER AHORA.
%The properly causal hypothesis can be removed from previous result if, instead, we assume
%%strong causality (ensured if $-H(x,u)$ is at most quadratic; see [Flores, Sanchez, CQG03]),
%the timelike convergence condition (which is equivalent to $K(x)\geq 0$, being $K(x)$ the curvature of $(M_0,g_0)$ at $x$, and $\Delta_x H(x,u)$ for all $(x,u)\in M_0\times {\mathbb R}$; see [Flores, Sanchez, CQG03]), and ${\cal J}^+\neq\emptyset$. In fact, the following result is a direct consequence of Theorem \ref{completeness2}\footnote{\cambiosj{Jony: Cuidado aquí. Entiendo que es aquí donde habrá que imponer algo sobre strongly properly causal...}}:
%\begin{corollary}
%If $(M,g)$ is a geodesically complete generalized plane wave which is strongly causal\footnote{\cambiosn{Aqui la hipotesis strongly causal aparece explicitamente porque asi aparecia en el Theorem \ref{completeness2}. No obstante, se trata de una hipotesis que en realidad se supone en todo momento, ya que estamos trabajando con el borde causal.}} and satisfies the TCC and ${\cal J}^+\neq\emptyset$, then it does not contain black holes.
%\end{corollary}
%}

%\cambiosn{
%\noindent {\em Proof.} From Theorem \ref{completeness2} it suffices to show that ${\cal J}^+\neq\emptyset$, which is ensured by the proof of previous corollary. \qcd
%}

%On the other hand, we know that $\gamma_{\overline{x},\overline{u}}$ is a future-directed null geodesic line for every $(\overline{x},\overline{u})$. Moreover, since $(M,g)$ is assumed to be geodesically complete, any other inextendible future-directed null geodesic $\gamma$ with $I^{-}(\gamma)=I^{-}(\gamma_{x_0,u_0})$ is also complete. Hence, $I^{-}(\gamma_{\overline{x},\overline{u}})$ belons to ${\cal J}^+$, and the proof is over. \qcd



%
%First, observe that condition
%$I^{-}(\gamma)=I^{-}(\gamma_{x_0,u_0})$ implies $u(s)=u(0)+s \Delta u rightarrow u_0$. Assume by contradiction that
%
%
%
%So, if $\Delta u=0$, necessarily $u(s)\equiv u(0)=u_0$, $v(s)=v(0)-s$ and $x(s)\equiv x(0)$, and thus, $\gamma$ is complete. So, assume by contradiction that $\Delta u>0$.
%
%........
%
%
%Taking into account that $u(s)$ is affine, the domain of definition of $\gamma$ must be some interval $(s_0,s_*)\subset\R$ with $s_*<\infty$. Since $(M,g)$ is complete, the trajectories.... are also complete. Hence, $x(s)\rightarrow x_*$ as $s\nearrow s_*$. Hence, $\gamma(s)\rightarrow (x_*,-s_*,u_0)$, and thus, $I^{-}(\gamma)$ is a PIP, in contradiction with the identity $I^{-}(\gamma)=I^{-}(\gamma_{x_0,u_0})$. \qcd

\section*{Appendix}
\label{sec:some-examples}
In Definitions \ref{scri} and \ref{ample} we have included some clauses that, even though natural when interpreted from the classical conformal approach viewpoint, might conceivably be discarded in favor of less technical-looking ones. The following construction shows that this is not the case if the thesis of Theorem \ref{main} is to be preserved. In fact, we will display a globally hyperbolic spacetime which is not past-complete and where (unsurprisingly) Theorem \ref{main} fails. The example also shows that the Theorem \ref{main} is also false if past-completeness is assumed but condition (ii) on Definition \ref{scri} is removed. This will show in particular that, even if with strong causality requirements on the spacetime, one cannot expect to conserve Theorem \ref{main} if the past-completeness condition is removed from the notion of regularity in Definition \ref{ample}.
%In fact, by means of the \cambiosj{two} firsts examples we show that Thm. \ref{main} is no longer true if the past-complete condition\footnote{\cambiosn{¿Algun ejemplo que pruebe que tampoco podemos quitar la condicion ample?}} is removed from the notion of regularity for ${\cal J}^+$ (Definition \ref{ample}).


\begin{note}
\emph{The construction considered below is given by making simple modifications of Minkowski spacetime. It is not difficult to realize that the future c-boundaries of the resulting spacetimes are always Hausdorff. In particular, from Proposition \ref{lema:auxiliar}, the set $\widetilde{I^+(C)}$ will be closed for any compact set $C$ in the spacetime.}
\end{note}

%The examples are based on Minkowski spacetime, ensuring a quite regular main structure. For instance, it is a simple exercise to see that, given a compact set $K$, $\widetilde{I^{+}(K)}=J^{+}(K)$.

% \cambiosj{In Theorem \ref{main} the future null infinity is assumed to be regular in the sense of Definition \ref{ample}. The first, very simple, example shows that this result fails when the past-complete condition is removed from this notion of regularity.}

% \cambiosj{In the following example we construct a globally hyperbolic spacetime which is not past-complete and, not surprisingly, Theorem \ref{main} fails in this case. This shows that, even if we additionally assume some causality requirement on the spacetime, we cannot expect to conserve Theorem \ref{main} if we remove the past-completeness condition from the notion of regularity in Definition \ref{ample}.}

% \smallskip

Consider the $3$-dimensional Minkowski spacetime
\[
\mathbb{L}^3 = (\mathbb{R}^3, dx^2+dy^2-dt^{2}).
\]
Isometrically compactify its (cartesian) $x$-coordinate as a circle $\mathbb{S}^1$ in order to obtain a new (flat) Lorentzian manifold $\tilde{M}$ whose spatial sections are $2$-dimensional cylinders. Note that the future-complete null lines of $\tilde{M}$ consist of straight lines of $\tilde{M}$ with spatial component parallel to the $y$-axis. In fact, the spatial component of any other inextendible lightlike geodesic $\sigma$ in $\tilde{M}$ will ``waste time'' spinning around the spatial cylinder, and eventually, two points of $\sigma$ will become chronologically related by some timelike curve $c$ (see Figure \ref{fig:1}).

\begin{figure}
\centering

  \setlength{\unitlength}{1bp}%
  \begin{picture}(99.80, 230.32)(0,0)
  \put(-13,0){\includegraphics{fig1NI.pdf}}
  \put(76.52,124.51){\rotatebox{0.00}{\fontsize{8.54}{10.24}\selectfont \smash{\makebox[0pt][l]{$1$}}}}
   \put(96.52,109.76){\fontsize{8.54}{10.24}\selectfont $\mathbf{x}$}
  \put(8.37,124.28){\rotatebox{0.00}{\fontsize{8.54}{10.24}\selectfont \smash{\makebox[0pt][l]{$-1$}}}}
  \put(69.36,78.76){\fontsize{8.54}{10.24}\selectfont $\sigma$}
  \put(14.36,10.76){\fontsize{8.54}{10.24}\selectfont $\sigma(t_0)$}
  \put(26.36,67.76){\fontsize{8.54}{10.24}\selectfont $c$}
  \put(69.36,202.76){\fontsize{8.54}{10.24}\selectfont $\sigma(t_1)$}
   \put(42.36,218.76){\fontsize{8.54}{10.24}\selectfont $\mathbf{y}$}
  \end{picture}%


  % \begin{picture}(230.32, 99.80)(0,0)
  % \put(0,0){\includegraphics{Figuras/fig1NI.pdf}}
  % % \put(123.36,82.76){\fontsize{8.54}{10.24}\selectfont $1$}
  % % \put(64.36,78.76){\fontsize{8.54}{10.24}\selectfont $\sigma$}
  % % \put(14.36,12.76){\fontsize{8.54}{10.24}\selectfont $\sigma(t_0)$}
  % % \put(160.36,62.76){\fontsize{8.54}{10.24}\selectfont $c$}
  % %  \put(187.36,75.76){\fontsize{8.54}{10.24}\selectfont $\sigma(t_1)$}
  % %  \put(121.60,12.12){\fontsize{8.54}{10.24}\selectfont $-1$}
  % \end{picture}%
% \else
%   \setlength{\unitlength}{1bp}%
%   \begin{picture}(230.32, 99.80)(0,0)
%   \put(0,0){\includegraphics{Figuras/fig1NI}}
%   \put(123.36,84.76){\fontsize{8.54}{10.24}\selectfont $1$}
%   \put(121.60,10.12){\fontsize{8.54}{10.24}\selectfont $-1$}
%   \end{picture}%
%   \fi
  \caption{\label{fig:1} Representation of the spatial component of the spacetime $\tilde{M}$. If the spatial component of a lightlike geodesic $\sigma$ in $\tilde{M}$ is not parallel to the $y$-axis, then two points of $\sigma$ will eventually become chronologically related in $\tilde{M}$ by some timelike curve $c$.}


  %In fact. assume both that $\gamma$ is a null ray with $\gamma(t)=(t,\sigma(t))$, and that $\sigma$ behaves as in the figure. The null character of the curve ensures that $t_1-t_0={\rm length}(\sigma|_{[t_0,t_1]})$. Now observe that, from construction, ${\rm length}(\sigma|_{[t_0,t_1]})>{\rm length}(c)=d(\sigma(t_0),\sigma(t_1))$. Therefore, and due the characterization of the chronological relation in static spacetimes, $\sigma(t_0)\ll \sigma(t_1)$, i.e., $\gamma$ is no longer a ray in $\tilde{M}$.}
\end{figure}

Next, consider the compact set
\[
C=\{(0,y,0)\, : \, -1\leq y \leq 1\},
\]
and define

  \[
\Sigma:= \{(0,y,1)\, : \, y \in \mathbb{R}\}\cap J^+(C,\tilde{M}).
    \]
Consider the Lorentzian manifold $(M,g)$, where $M:=\tilde{M}\setminus J^+(\Sigma,\tilde{M})$ and $g$ is the induced metric on $M$ from $\tilde{M}$ (see figures  \ref{fig:3}  and \ref{fig:2} for illustrations of the projections of $M$ onto the $y=0$ and $x=0$ planes, resp.).

  The $c$-boundary of $(M,g)$ is the disjoint union of the future and past c-boundaries, each being formed by pairs with an empty $F$ or $P$ component, resp. In particular, $(M,g)$ is globally hyperbolic according to \cite[Theorem 3.29]{Floresfinaldefinitioncausal2011}. Moreover, the pairs $(P,\emptyset)$ of the future c-boundary can be separated in two classes: (a) those pairs defined by inextensible timelike curves with divergent component $y$, and (b) those pairs identifiable with the points of the (topological) boundary of $J^+(\Sigma,\tilde{M})$. It readily follows that the former points are in $\mathcal{J}^{+}$, since there exist complete null rays defining the corresponding TIPs (here we can proceed just as in Minkowski spacetime). However, the latter pairs belong to $\partial M\setminus \mathcal{J}^+$, since the null rays defining such TIPs are incomplete.  In any case, the following properties hold:

\begin{figure}
\centering
  \setlength{\unitlength}{1bp}%
  \begin{picture}(227.21, 140.19)(0,0)
    \put(0,0){\includegraphics{fig2NI.pdf}}
  \put(205.82,35.23){\fontsize{11.16}{13.00}\selectfont $\mathbf{y}$}
  \put(86.47,24.23){\fontsize{9.16}{11.00}\selectfont $C$}
  \put(134.95,63.63){\fontsize{9.16}{11.00}\selectfont $\Sigma$}
  \put(99.32,90.85){\fontsize{9.16}{11.00}\selectfont $J^+(\Sigma,\tilde{M})$}
  \put(115.03,123.78){\fontsize{11.75}{13.50}\selectfont $\mathbf{t}$}
  \end{picture}%
  \caption{\label{fig:3} Representation of the intersection of $(M,g)$ with the plane $x=0$. This is a standard plane with both the grey area and the set $\Sigma$ removed. All future-complete null rays in $M$ departing from points of the form $(t,0,y)$ are contained in this plane. In particular, any null ray departing from $C$ intersects $\Sigma$, and thus, there are no future-complete null $C$-rays.}
\end{figure}

\begin{figure}
\centering
  \setlength{\unitlength}{1bp}%
  \begin{picture}(216.58, 213.83)(0,0)
  \put(0,0){\includegraphics{fig3NI.pdf}}
  \put(104.74,199.52){\fontsize{11.07}{13.28}\selectfont $\mathbf{t}$}
  \put(191.92,99.81){\fontsize{11.07}{13.28}\selectfont $\mathbf{x}$}
  \put(161.92,99.81){\fontsize{11.07}{13.28}\selectfont $1$}
   \put(42.92,99.81){\fontsize{11.07}{13.28}\selectfont $-1$}
  \put(107.35,172.11){\fontsize{9.38}{9.86}\selectfont $J^+(\Sigma,\tilde{M})$}
  \put(107.71,135.42){\fontsize{9.38}{9.86}\selectfont $\Sigma$}
  \put(107.07,99.46){\fontsize{9.38}{9.86}\selectfont $C$}
  \end{picture}%
  \caption{\label{fig:2} Representation of the intersection of $(M,g)$ with the plane $y=0$. Here, the projections of $\Sigma$ and $C$ are points, and the lines $x=-1$, $x=1$ of this plane are identified.}
\end{figure}

\begin{itemize}
\item ${\cal J}^+$ is ample: Let $K\subset M$ be any compact set. In order to show the existence of points in ${\cal J}^+$ not contained in the closed set $\widetilde{I^+(K)}$, let $(x_0,y_0,t_0)$ be a point in $M$ such that both $K$ and $J^+(\Sigma)$ are contained in $I^+(x_0,y_0,t_0)$. For any $\epsilon>0$ the null line $\gamma(t)=(x_0,y_0+t,t_0-\epsilon+t)$ is contained in $M$ (since it does not intersect $J^+(\Sigma)$), and defines a point $(P,\emptyset)\in {\cal J}^+$ with $(x_0,y_0,t_{0})\not \in P=I^-(\gamma)$. Moreover, the pair $(P,\emptyset)$ is not contained in $\widetilde{I^{+}(K)}$. Indeed, otherwise there would exist
some $q\in K$ such that $I^{-}(q)\subset P$. But by construction,
$q\in I^{+}(x_0,y_0,t_{0})$, and thus, $(x_0,y_0,t_{0})\in P$, in contradiction with the properties of $\gamma$.

    %which is not contained in $J^+(K)$ (otherwise, the curve $\gamma$ would intersect $I^+(K)$, an absurd.)

  % For the Hausdorff condition, just observe that the c-completion on $M$ shares some properties of the well-known c-completion of the Minkowski spacetime, including its Hausdorff and AN1 character (in fact, the c-completion of the Minkowski spacetime coincides with its conformal completion.)

\item $\mathcal{J}^{+}$ is not past-complete: Consider the null line $\sigma$ in $M$ given by $\sigma(t)=(1/2,1+t,t)$, which defines a pair $(P,\emptyset)\in \mathcal{J}^+$ with $P=I^-(\sigma)$. By construction, $\partial P$ is a plane containing $\sigma$ and lying on the boundary of $J^+(\Sigma,\tilde{M})$. In particular, the future null line $\sigma'(t)=(0,1+t,t)$ is a null geodesic generator of $\partial P$. However, the pair $(P',\emptyset)\in \overline{M}$, $P'=I^-(\sigma')$ (which is associated with the point $(0,2,1)$) is not included in $\mathcal{J}^{+}$. Therefore $\mathcal{J}^{+}$ is not past-complete.
  % By abuse of notation, we will write $I^-((1,0,2))$ for the past of any inextendible future timelike curve on $M$ with endpoint $(1,0,2)$. By construction, two boundary points $(I^-((1,0,2)),\emptyset)\in \partial M\setminus {\cal J}^+$ (no future-complete null ray arrives at $(1,0,2)$) and $(I^-(\sigma),\emptyset)\in {\cal J}^+$ with $I^-((1,0,2))\subset I^-(\sigma)$. Therefore, ${\cal J}^+$ is not past-complete.
  % \item \cambiosj{$C$ is not fully covered by a black hole:\footnote{\cambiosj{OJO, AQUI JONY HA DETECTADO UN ERROR QUE DEBEMOS COMENTAR!!! NO ESTA CLARO QUE EMANE UN FUTURE COMPLETE NULL RAY DESDE Q.}} Let us show that} the intersection $I^+(C)\cap  \mathcal{J}^+$ is non-empty. In fact, for each point $p\in C$ there exists a point $q\in I^+(p)\cap M$ with time component equal to $1$, from which a future-complete null ray emanates. Intuitively, if we move slightly along the $x$-axis, we can avoid the set $J^+(\Sigma)$, showing the existence of future-complete null rays starting at $q$ (see Figure \ref{fig:2}).
  \item There are no future-complete null C-rays, since any such null C-ray in $\tilde{M}$ must intersect $J^{+}(\Sigma,\tilde{M})$ (see Figure \ref{fig:3}).
  % For the closeness of $L(I^+(C))$ just recall from previous point that the topology on $\overline{M}$ is AN1.


    \end{itemize}
Thus, in order to violate the thesis of Theorem \ref{main}, we need to show that $C$ is not entirely contained inside the black hole.
Our construction, however still does not satisfy this property, since all causal curves emanating from $C$ intersect $J^{+}(\Sigma,\tilde{M})$, whose boundary is not contained in $\mathcal{J}^+$.

To complete the construction, consider then a conformally rescaled metric $\mathfrak{g}=\Omega\, g$ such that: (i) all the boundary points in $J^+(\Sigma,\tilde{M})\setminus\Sigma$ are in $\mathcal{J}^{+}$ with $\mathfrak{g}$ and (ii) all the null $C$-rays are still incomplete. In order to obtain an appropriate conformal factor $\Omega$, recall a classic result due to Clarke \cite{Clarkegeodesiccompletenesscausal1971} ensuring the existence of a conformal factor $\tilde{\Omega}$ such that all lightlike geodesics on $(M,\tilde{\Omega} g)$ are complete. Now define $\Omega$ thus: (i) $\Omega\equiv 1$ in $I^{-}_{1/4}(\Sigma)$ and (ii) $\Omega\equiv \tilde{\Omega}$ in $M\setminus I^{-}(\Sigma)$, where $I^{-}_{a}(\cdot)$ denotes the chronological past computed with the metric $-adt^2+dx^2+dy^{2}$ (see Figure \ref{fig:4}).
This conformal factor ensures that all the future-directed lightlike geodesics with endpoint in $J^+(\Sigma,\tilde{M})\setminus \Sigma$ (and so, contained in a region where $\Omega\equiv \tilde{\Omega}$) are complete. In particular, $\left(J^+(\Sigma,\tilde{M})\setminus \Sigma\right)\subset \mathcal{J}^+$.

The situation with $\Sigma$ is quite peculiar: there exist future-complete null rays with endpoints on $\Sigma$ (so condition (i) on Defn. \ref{scri} holds) but not all null geodesics with endpoints on $\Sigma$ are complete (consider for instance the null rays emanating from $C$.) Hence, $\Sigma$ does not intersect $\mathcal{J}^+$ \textit{just because condition (ii) on Defn. \ref{scri} fails}. But now, since $\Sigma$ does not intersect $\mathcal{J}^+$, $(M,\Omega\,g)$ in turn fails to be past-complete, just by the same previous arguments. However, all the points in $C$ are now visible, as we can connect them with boundary points of $\left(J^+(\Sigma,\tilde{M})\setminus \Sigma\right)$ by means of a future-directed timelike curves.

\begin{figure}
\centering
\ifpdf
  \setlength{\unitlength}{1bp}%
  \begin{picture}(216.58, 213.83)(0,0)
  \put(0,0){\includegraphics{fig4NI.pdf}}
  % \put(104.74,199.52){\fontsize{11.07}{13.28}\selectfont $T$}
  % \put(191.92,97.81){\fontsize{11.07}{13.28}\selectfont $X$}
  % \put(111.35,172.11){\fontsize{7.38}{8.86}\selectfont $J^+(\Sigma)$}
  % \put(107.71,139.41){\fontsize{7.38}{8.86}\selectfont $\Sigma$}
  % \put(105.07,99.46){\fontsize{7.38}{8.86}\selectfont $C$}
  \put(90.52,48.36){\fontsize{13.38}{16.86}\selectfont $\Omega\equiv 1$}
  \put(190.52,100.36){\fontsize{11.38}{14.86}\selectfont $\mathbf{x}$}
     \put(104.52,205.36){\fontsize{11.38}{14.86}\selectfont $\mathbf{t}$}
  \put(35.76,133.07){\fontsize{13.38}{16.86}\selectfont $\Omega=\tilde{\Omega}$}
  \end{picture}%
\else
  \setlength{\unitlength}{1bp}%
  \begin{picture}(216.58, 213.83)(0,0)
  \put(0,0){\includegraphics{fig4NI}}
  % \put(104.74,199.52){\fontsize{11.07}{13.28}\selectfont $T$}
  % \put(191.92,97.81){\fontsize{11.07}{13.28}\selectfont $X$}
  % \put(111.35,172.11){\fontsize{7.38}{8.86}\selectfont $J^+(\Sigma)$}
  % \put(107.71,139.41){\fontsize{7.38}{8.86}\selectfont $\Sigma$}
  % \put(105.07,99.46){\fontsize{7.38}{8.86}\selectfont $C$}
  \put(94.52,52.36){\fontsize{7.38}{8.86}\selectfont $\Omega\equiv 1$}
  \put(36.76,133.07){\fontsize{7.38}{8.86}\selectfont $\Omega=\tilde{\Omega}$}
  \end{picture}%
  \fi

  \caption{\label{fig:4} Illustration of the behaviour of the conformal factor $\Omega$ in the section $y=0$ of the spacetime.}
\end{figure}


\begin{remark}
 \emph{Observe that the previous example also shows that condition (ii) on Definition \ref{scri} too necessary to obtain Theorem \ref{main}. In fact, as we have mentioned, if we remove such a condition then the points of $\Sigma$ will also belong to $\mathcal{J}^{+}$, and the spacetime is actually past-complete. However, it is still true that there are no future-complete null $C$-rays. }
\end{remark}

%     \vspace{3cm}

%  \noindent    However, there are no future-complete null $C$-rays, since any such null $C$-ray in $\tilde{M}$ intersects $J^+(\Sigma,\tilde{M})$ (see Figure \ref{fig:3}). \cambiosj{In conclusion, Theorem \ref{main} is false in this case.}}






% \cambiosj{In the last example we will show that Theorem \ref{main} does not hold if condition (ii) is removed from the notion of future null infinity $\mathcal{J}^+$ (Definition \ref{scri}), but still assuming regularity of $\mathcal{J}^+$.}
% % \begin{example}
% %   {\em  \cambiosj{The spacetime considered here is a slight conformal modification of the previous one. In particular, it will remain globally hyperbolic with the same c-boundary structure as before. However, the resulting null infinity $\mathcal{J}^+$ will be different, since it is not conformally invariant.}

% %     A classic result by Clarke (see \cite{Clarkegeodesiccompletenesscausal1971}) ensures the existence of a conformal factor $\Omega \in C^{\infty}(M)$ on $M$ such that $(M,\Omega^2 g)$ is null geodesically complete. \cambiosj{Let $\tilde{\Omega}:M\rightarrow [0,1]$ be another smooth function satisfying: (i) $\tilde{\Omega}\equiv 0$ in $I^{-}_{1/4}(\Sigma)$ and (ii) $\tilde{\Omega}\equiv 1$ in $M\setminus I^{-}_{1/2}(\Sigma)$, where $I^{-}_{a}(\cdot)$ denotes the chronological past computed with the metric $-adt^2+dx^2+dy^{2}$.}
% % \footnote{Esto esta incompleto, ¿no? \cambiosj{Nota de Jony: Ahora ya sí creo que no...}}
% % %   \[
% % %   \tilde{\Omega}(t,0,y)=0,\quad \tilde{\Omega}(t,x,y)=1\;\; \hbox{$\forall$ $t\geq 2x+1$ and $x<0$, and $\forall$ $t\geq -2x+1$ and $x>0$.}
% % % \]
% % Consider he metric $\hat{g}=\hat{\Omega}^{2}g=\left(\tilde{\Omega}\Omega+(1-\tilde{\Omega}) \right)^2 g$ on $M$, \cambiosj{This metric coincides with $g$ on the set $I^{-}_{1/4}(\Sigma)$, which includes the trapezium between $C$ and $\Sigma$ in Figure \ref{fig:3}}. Hence, the null rays departing from points in $C$ parallel to the $y$-axis are still future-incomplete.

% %  \cambiosj{However the set $\mathcal{J}^{+}$ contains all the pairs $(P,\emptyset)\in \partial M$, and so, it is necessarily past-complete. In fact, $\mathcal{J}^{+}$ includes the null infinity points of previous spacetime $(M,g)$, since its complete null lines are also complete in $(M,\hat{g})$. Moreover, $\mathcal{J}^{+}$ includes the pairs $(P,\emptyset)$ identifiable with points at the boundary of $J^+(\Sigma,\tilde{M})$. In fact, these points are reachable by future null rays whose intersection with $I^{-}_{1/2}(\Sigma)$ is empty, and so, they belong to a region on $M$ where the conformal factor $\hat{\Omega}$ coincides with $\Omega$; that is, these null rays in $(M,g)$ become complete null lines in $(M,\hat{g})$, since $\hat{g}=\Omega^2g$ around them (recall the definition of $\Omega$.)}

% % Summarizing, the future null infinity is ample (reasoning as in previous example) and past-complete. However, there are not complete null rays emanating from $C$, showing that Theorem \ref{main} is false in this case.

% %   }
% % \end{example}


%%% Local Variables:
%%% mode: latex
%%% TeX-master: "CostaFloresHerreraNullInfty"
%%% End:


\section*{Acknowledgments}

The authors are partially supported by the Spanish Grant MTM2016-78807-C2-2-P (MINECO and FEDER funds). They wish to acknowledge the IEMath-GR, and especially Miguel S\'anchez, for the kind hospitality while part of the work on this paper was being carried out.

\begin{thebibliography}{10}

\bibitem{AkeSpacetimecoveringscasual2017}
{\sc L.~A. Ak{\'e} and J.~Herrera}, {\em Spacetime coverings and the casual
  boundary}, Journal of High Energy Physics, 2017 (2017), p.~51.

\bibitem{BeemGlobalLorentzianGeometry1996}
{\sc J.~K. Beem, P.~E. Ehrlich, and K.~L. Easley}, {\em Global {{Lorentzian
  Geometry}}}, vol.~202 of Pure and Applied Mathematics, {Marcel Dekker}, New
  York, 1996.

\bibitem{CandelaGeneralPlaneFronted2003}
{\sc A.~M. Candela, J.~L. Flores, and M.~S{\'a}nchez}, {\em On general plane
  fronted waves. {{Geodesics}}}, General Relativity and Gravitation, 35 (2003),
  pp.~631--649.

\bibitem{ChruscielConformalboundaryextensions2010}
{\sc P.~T. Chru{\'s}ciel}, {\em Conformal boundary extensions of {{Lorentzian}}
  manifolds}, Journal of Differential Geometry, 84 (2010), pp.~19--44.

\bibitem{ChruschielDelayExistenceAsymptoticallySimple2002}
{\sc P.~T. Chru{\'s}ciel and E.~Delay}, {\em Existence of non-trivial, vacuum,
  asymptotically simple spacetimes}, Classical and Quantum Gravity, 19 (2002),
  pp.~L71--L79 Erratum: 3389.

\bibitem{Clarkegeodesiccompletenesscausal1971}
{\sc C.~J.~S. Clarke}, {\em On the geodesic completeness of causal
  space-times}, Mathematical Proceedings of the Cambridge Philosophical
  Society, 69 (1971), pp.~319--323.

\bibitem{CostaFloresHerrera2018-2}
{\sc I.~P. Costa~e Silva, J.~Herrera, and J.~L. Flores}, Preprint,  (2018).

\bibitem{EHRLICH_1992}
{\sc P.~E. Ehrlich and G.~G. Emch}, {\em Gravitational waves and causality},
  Reviews in Mathematical Physics, 04 (1992), p.~163–221.

\bibitem{Florescausalboundaryspacetimes2007}
{\sc J.~L. Flores}, {\em The causal boundary of spacetimes revisited},
  Communications in Mathematical Physics, 276 (2007), pp.~611--643.

\bibitem{Floresfinaldefinitioncausal2011}
{\sc J.~L. Flores, J.~Herrera, and M.~S{\'a}nchez}, {\em On the final
  definition of the causal boundary and its relation with the conformal
  boundary}, Advances in Theoretical and Mathematical Physics, 15 (2011),
  pp.~991--1057.

\bibitem{FloresGromovCauchycausal2013}
\leavevmode\vrule height 2pt depth -1.6pt width 23pt, {\em Gromov, {{Cauchy}}
  and causal boundaries for {{Riemannian}}, {{Finslerian}} and {{Lorentzian}}
  manifolds}, Memoirs of the American Mathematical Society, 226 (2013),
  pp.~vi+76.

\bibitem{FloresHausdorffseparabilityboundaries2016}
\leavevmode\vrule height 2pt depth -1.6pt width 23pt, {\em Hausdorff
  separability of the boundaries for spacetimes and sequential spaces}, Journal
  of Mathematical Physics, 57 (2016), pp.~022503, 25.

\bibitem{Florescausalboundarywavetype2008}
{\sc J.~L. Flores and M.~S{\'a}nchez}, {\em The causal boundary of wave-type
  spacetimes}, Journal of High Energy Physics. A SISSA Journal,  (2008),
  pp.~036, 43.

\bibitem{0264-9381-20-11-322}
{\sc J.~L. Flores and M.~Sánchez}, {\em Causality and conjugate points in
  general plane waves}, Classical and Quantum Gravity, 20 (2003), p.~2275.

\bibitem{Frauendiener2004}
{\sc J.~Frauendiener}, {\em Conformal infinity}, Living Reviews of Relativity,
  7 (2004).

\bibitem{Friedrich1}
{\sc H.~Friedrich}, {\em On the existence of n-geodesically complete or future
  complete solutions of einstein’s field equations with smooth asymptotic
  structures}, Communications in Mathematical Physics, 107 (1986),
  pp.~587--609.

\bibitem{Friedrich2}
\leavevmode\vrule height 2pt depth -1.6pt width 23pt, {\em On the global
  existence and the asymptotic behavior of solutions to the
  {{Einstein}}-{{Maxwell}}-{{Yang}}-{{Mills}} equations}, Journal of
  Differential Geometry, 34 (1991), pp.~275--345.

\bibitem{Friedrich3}
\leavevmode\vrule height 2pt depth -1.6pt width 23pt, {\em Smoothness at null
  infinity and the structure of initial data}, in The Einstein Equations and
  the Large Scale Behavior of Gravitational Fields, P.~T. Chru{\'s}ciel and
  H.~Frierdrich, eds., {Birkhauser}, Basel, 2004, pp.~243--275.

\bibitem{Garcia-ParradoCausalstructurescausal2005}
{\sc A.~Garc{\'\i}a-Parrado and J.~M.~M. Senovilla}, {\em Causal structures and
  causal boundaries}, Classical and Quantum Gravity, 22 (2005), pp.~R1--R84.

\bibitem{GerochIdealPointsSpaceTime1972}
{\sc R.~Geroch, E.~H. Kronheimer, and R.~Penrose}, {\em Ideal points in
  space-time}, Proceedings of the Royal Society A, 327 (1972), pp.~545--567.

\bibitem{HawkingLargeScaleStructure1975}
{\sc S.~W. Hawking and G.~F.~R. Ellis}, {\em The {{Large Scale Structure}} of
  {{Space}}-{{Time}} ({{Cambridge Monographs}} on {{Mathematical Physics}})},
  {Cambridge University Press}, 1975.

\bibitem{HubenyRagamaniNoHorizons}
{\sc V.~E. Hubeny and M.~Rangamani}, {\em No horizons in pp-waves}, Journal of
  High Energy Physics, 2002 (2002), p.~021.

\bibitem{Marolfnewrecipecausal2003}
{\sc D.~Marolf and S.~F. Ross}, {\em A new recipe for causal completions},
  Classical and Quantum Gravity, 20 (2003), pp.~4085--4117.

\bibitem{minguzzi12:causal_kam}
{\sc E.~Minguzzi}, {\em {Causality of spacetimes admitting a parallel null
  vector and weak KAM theory}},  (2012), arXiv:1211.2685v1.

\bibitem{Minguzzicausalhierarchyspacetimes2008}
{\sc E.~Minguzzi and M.~S{\'a}nchez}, {\em The causal hierarchy of spacetimes},
  in Recent Developments in Pseudo-{{Riemannian}} Geometry, ESI Lect. Math.
  Phys., {Eur. Math. Soc., Z{\"u}rich}, 2008, pp.~299--358.

\bibitem{ONeillSemiRiemannianGeometryApplications1983}
{\sc B.~O'Neill}, {\em Semi-{{Riemannian Geometry With Applications}} to
  {{Relativity}}, 103, {{Volume}} 103 ({{Pure}} and {{Applied Mathematics}})},
  {Academic Press}, 1983.

\bibitem{PenroseAsymptoticStructure1963}
{\sc R.~Penrose}, {\em Asymptotic properties of fields and space-times},
  Physical Review Letters, 10 (1963), pp.~66--68.

\bibitem{PenroseConformalInfinity1964}
\leavevmode\vrule height 2pt depth -1.6pt width 23pt, {\em Conformal treatment
  of infinity}, in Relativity, {{Groups}} and {{Topology}}, C.~M. de~Witt and
  B.~de~Witt, eds., {Gordon and Breach}, New York, NY, 1964, pp.~566--584.

\bibitem{PenroseGravitationalCollapse1965}
\leavevmode\vrule height 2pt depth -1.6pt width 23pt, {\em Gravitational
  collapse and space-time singularities}, Physical Review Letters, 14 (1965),
  pp.~57--59.

\bibitem{PenroseDifferentialTopology1972}
\leavevmode\vrule height 2pt depth -1.6pt width 23pt, {\em Techniques of
  {Differential} {Topology} in {{Relativity}}}, {{SIAM}}, {Conference Board of
  the Mathematical Sciences Regional Conference Series in Applied Mathematics},
  1972.

\bibitem{SachsPeeling1961}
{\sc R.~Sachs}, {\em Gravitational waves in {{General}} {{Relativity.}}{{ VI}}.
  {{The}} outgoing radiation condition}, Proceedings of the Royal Society A,
  264 (1961), pp.~339--354.

\bibitem{SenovillaNoBHinPPWaves2003}
{\sc J.M.M.~Senovilla}, {\em On the existence of horizons in spacetimes with
  vanishing curvature invariants}, Journal of High Energy Physics, 2003 (2003),
  pp.~046--051.

\bibitem{SzabadosCausalboundarystrongly1988}
{\sc L.~B. Szabados}, {\em Causal boundary for strongly causal spacetimes},
  Classical and Quantum Gravity, 5 (1988), pp.~121--134.

\bibitem{SzabadosCausalboundarystrongly1989}
{\sc L.~B. Szabados}, {\em Causal boundary for strongly causal spacetimes.
  {{II}}}, Classical and Quantum Gravity, 6 (1989), pp.~77--91.

\bibitem{WaldGeneralRelativity1984}
{\sc R.~Wald}, {\em General {{Relativity}}}, {University of Chicago Press},
  1984.

\end{thebibliography}
\vspace*{\fill}

%-----------------------------------------------------------------------------------------------------------

\vspace{.5cm}
\end{document}

\bean
\left\{
\begin{array}{l}
433434 \\
45345234
\end{array}\right.
\eean


%%% Local Variables:
%%% mode: latex
%%% TeX-master: t
%%% End:
