\section{Preliminaries: the c-boundary construction}\label{prelim}

Throughout this paper, $(M,g)$ will denote a {\it spacetime}, i.e. a connected, time-oriented $C^{\infty}$ Lorentzian manifold $(M,g)$. We shall assume that the reader is familiar with standard facts in Lorentzian geometry and causal theory as given in the basic references \cite{BeemGlobalLorentzianGeometry1996,HawkingLargeScaleStructure1975,ONeillSemiRiemannianGeometryApplications1983}.
%with signature
%$(-,+,\ldots,+)$.
%A  tangent vector $v\in T_{p}M$ on a point $p\in M$  is
%named {\em timelike} (resp. {\em lightlike}; {\em causal}) if
%$g(v,v)<0$ (resp. $g(v,v)=0$, $v\neq 0$; $v$ is either timelike or
%lightlike). Accordingly, a smooth curve $\gamma:I\rightarrow M$
%($I$ real interval) is called {\em timelike} (resp. {\em
%lightlike}; {\em causal}) if its tangent vector $\dot{\gamma}(s)$ is timelike (resp.
%lightlike; causal) for all $s$. We will assume along the paper that all spacetimes are  {\em
%time-oriented}, i.e. they are endowed with a continuous, globally
%defined, timelike vector field $X$. A {\em time-orientation}
%$X$ distributes the causal tangent vectors $v\in T_{p}M$ in two
%cones, each one containing future $g(v,X(p))<0$ or past-directed
%$g(v,X(p))>0$ causal vectors. So, a causal curve $\gamma(s)$ is
%said {\em future-directed} (resp. {\em past-directed}) if
%$g(\dot{\gamma}(s),X(\gamma(s)))<0$ (resp.
%$g(\dot{\gamma}(s),X(\gamma(s)))>0$) for all $s$.

%Timelike and causal curves define the so-called conformal structure of the spacetime $(M,g)$ given by the following two relations:
%given two points $p,q\in M$, we will say that they are {\em chronologically related}, denoted by $p\ll q$
%(resp. {\em causally related}, $p\leq q$) if there exists some
%future-directed timelike (resp. causal) curve from $p$ to $q$ (the
%case $p=q$ is allowed when $p\leq q$). The {\em
%chronological past} (resp. {\em future}) of $p$, $I^{-}(p)$ (resp. $I^{+}(p)$)
%is defined as:
%\[
%I^{-}(p)=\{q\in M: q\ll p\}\qquad(I^{+}(p)=\{q\in M: p\ll q\}).
%\]

%A spacetime is  {\em distinguishing} if any of its points is
%characterized by its past and future, {\em strongly causal} if it
%does not admit neither closed nor ``almost closed'' causal curves,
%and {\em globally hyperbolic} if it admits a {\em Cauchy
%hypersurface}, i.e. a topological hypersurface that is met exactly
%once by every inextendible timelike curve. Here, global
%hyperbolicity is the most restrictive causality condition, while
%distinguishing is the most general one.

The causal completion of $(M,g)$ is by now a standard construction put forth for the first time in a seminal paper by Geroch, Kronheimer and Penrose \cite{GerochIdealPointsSpaceTime1972}. The underlying idea is simple enough: to add \textit{ideal points} to the original spacetime, comprising the so called \textit{causal completion} (or \textit{$c$-completion} for short), in such a way that any inextendible timelike curve in $(M,g)$ has endpoints. The set of these ideal endpoints form the associated {\em causal boundary}, or {\em $c$-boundary} for short. This construction is in addition conformally invariant in the sense that two spacetimes in the same conformal class have identical $c$-boundaries. Finally, it is applicable to any strongly causal spacetime, independently of their asymptotic properties.

For the convenience of the reader and to set up the notation we shall use throughout, we briefly review the c-boundary construction here, referring to \cite{GerochIdealPointsSpaceTime1972} and \cite{Floresfinaldefinitioncausal2011} for further details and proofs. Let us start with some basic
notions.

A non-empty subset $P\subset M$ is called a {\em past set} if it
coincides with its chronological past, i.e. $P=I^{-}(P)$. (In particular, every past set is open.) A past set that cannot be written as the
union of two proper subsets, both of which are also past sets, is
said to be an {\em indecomposable past} set (IP). It can be shown that an IP either coincides with the past of some point of the spacetime, i.e., $P=I^{-}(p)$ for $p\in M$, or else $P=I^{-}(\gamma)$ for some inextendible future-directed
timelike curve $\gamma$. In the former case, $P$ is said to be a {\em proper indecomposable past
set} (PIP), and in the latter case $P$ is said to be a {\em
terminal indecomposable past set} (TIP). These two classes of IPs are disjoint.

Another useful technical definition is the following. The {\em common past} of a given set $S\subset
M$ is defined by \[\downarrow S:=I^{-}(\{p\in M:\;\; p\ll
q\;\;\forall q\in S\}).\]
The corresponding
definitions for {\em future sets}, IFs, TIFs,
PIFs, {\em common future}, etc., are obtained just by interchanging the roles of past and
future, and will always be understood.

The set of all IPs constitutes the so-called {\it future $c$-completion} of $(M,g)$, denoted by $\hat{M}$. If $(M,g)$ is strongly causal, then $M$ can naturally be viewed as a subset of $\hat{M}$ by identifying every point $p\in M$ with its respective PIP, namely $I^-(p)$. Indeed, it is well-known \cite{BeemGlobalLorentzianGeometry1996} that a strongly causal $(M,g)$ is {\em distinguishing}, i.e.,
\[
\forall p,q \in M, \, I^{\pm}(p)=I^{\pm}(q) \Longrightarrow p=q,
\]
so that the inclusion $i: p\in M \hookrightarrow I^-(p) \in \hat{M}$ is indeed one-to-one.

{\em Throughout this paper, unless otherwise explicitly stated, we shall assume that $(M,g)$ is strongly causal.}

The {\it future $c$-boundary} $\hat{\partial} M$ of $(M,g)$ is defined as the set of all its TIPs. Therefore, upon identifying $M$ with its image in $\hat{M}$ by the natural inclusion as outlined above,
\[
\hat{\partial} M \equiv \hat{M} \setminus M.
\]
The definitions of \textit{past $c$-completion} $\check{M}$ and {\it past $c$-boundary} $\check{\partial}M$ of $(M,g)$ are readily defined in a time-dual fashion using IFs.

Now, one would expect that the {\em (total) $c$-completion} of $(M,g)$ could be obtained by ``joining together'' in some suitable sense the past and future $c$-completions. However, naive attempts to do so meet with some surprisingly thorny technical issues (again, see \cite{Floresfinaldefinitioncausal2011} and references therein for a discussion). To circumvent these issues, certain clever manipulations are required which were first carried out by Szabados \cite{SzabadosCausalboundarystrongly1988, SzabadosCausalboundarystrongly1989}, with important improvements by Marolf and Ross \cite{Marolfnewrecipecausal2003}.

First, we introduce the so-called {\em Szabados relation} (or \textit{$S$-relation}) between IPs and IFs: an IP $P$ and an IF $F$ are {\em S-related}, $P\sim_{S}F$, if $P$ is a maximal IP inside
$\downarrow F$ and $F$ is a maximal IF inside $\uparrow P$. In particular, for any $p \in M$, it can be shown that $I^-(p) \sim_{S} I^+(p)$. Then, we have the following

\begin{definition}\label{d1} The {\em (total) c-completion} $\overline{M}$ is
formed by all
the pairs $(P,F)$ formed by $P\in \hat{M}\cup\{\emptyset\}$ and
$F\in \check{M}\cup\{\emptyset\}$ such that either
\begin{itemize}
\item[i)] both $P$ and $F$ are non-empty and $P\sim_{S}F$; or
\item[ii)] $P=\emptyset$, $F \neq \emptyset$ and there is no $P'\neq \emptyset$ such that $P'\sim_{S}F$; or
\item[iii)] $F=\emptyset$, $P \neq \emptyset$ and there is no $F'\neq\emptyset$ such that $P\sim_{S}F'$.
\end{itemize}
The original manifold $M$ is then identified with the set $\{(I^{-}(p),I^{+}(p)): p\in M\}$, and the {\em c-boundary} is defined as $\partial M\equiv\overline{M}\setminus M$.
\end{definition}
\begin{remark}\label{r0} \emph{We will systematically use the following fact, which is easy to check: given any IP $P \neq \emptyset$ (resp. any IF $F \neq \emptyset$), there always exists $F \in \check{M}\cup\{\emptyset\}$ (resp. $P \in \hat{M}\cup\{\emptyset\}$) such that $(P,F) \in \overline{M}$.
}
\end{remark}

Having defined the set structure of the $c$-completion, the next step is to extend the chronological relation in $(M,g)$ to the $c$-completion as follows:

\begin{equation}
(P,F)\ll
(P',F')\;\;\iff\;\; F\cap P'\neq\emptyset. \label{eq:7}
\end{equation}

Moreover, two pairs $(P,F)$,
$(P',F')$ are {\em causally related}, denoted by $(P,F)\leq (P',F')$, if
$F'\subset F$ and $P\subset P'$. Finally, these pairs said to be {\em
horismotically related} if they are causally, but not
chronologically related\footnote{\label{fn:1}These notions for causal and horismotical relations are not totally satifactory in general (see for instance \cite[Section 3.2]{Marolfnewrecipecausal2003} for more discussion on this issue). However, under the additional hypotheses we will assume later in this paper, they can be adopted without problems.}.

In concrete applications, it is also important to introduce a suitable topology on the $c$-completion $\overline{M}$.
%\cambiosn{Unfortunately, there is no consensus in the literature on how this should be done.}
The original topology considered in \cite{GerochIdealPointsSpaceTime1972}, although Hausdorff, was plagued by a number of technical problems and failed to yield sensible results even in simple cases. (An extensive discussion of these issues with further pertinent references on alternative proposed topologies on the $c$-completion can be found in Ref. \cite{Floresfinaldefinitioncausal2011}.) Our choice here, for reasons which are exhaustively discussed in \cite{Floresfinaldefinitioncausal2011}, is the so-called {\em chronological topology} on the $c$-completion.

The chronological topology is more conveniently defined by means of a so-called {\em limit operator} (see for instance \cite{FloresHausdorffseparabilityboundaries2016}). Let us briefly recall this notion here. Let $X$ be any set. Denote by ${\cal S}_X$ the set of all (infinite) sequences in $X$ (including their subsequences), and by $\mathbb{P}(X)$ the power set of all subsets of $X$. A {\em limit operator} on $X$ is any mapping $L:{\cal S}_X \rightarrow \mathbb{P}(X)$ such that if $\sigma \in {\cal S}_X$ is a sequence in $X$ and $\sigma '$ is a subsequence of $\sigma$, then $L(\sigma) \subset L(\sigma ')$. If $L$ is one such limit operator, the associated {\em derived topology} $\tau _{L}$ is defined via its closed sets: by definition, a subset $C \subset X$ is closed in $\tau _L$ if and only if $L(\sigma) \subset C$ for any sequence $\sigma$ of elements of $C$. The following facts about the topological space $(X,\tau _L)$ thus defined are readily verified.
\begin{itemize}
\item[LO1)] If $x \in L(\sigma)$ for a sequence $\sigma \in {\cal S}_X$, then $\sigma$ converges to $x$ with respect to $\tau_L$.
\item[LO2)] The topological space $(X,\tau _L)$ is {\em sequential}, i.e., a set $C \subset X$ is closed if and only if it is sequentially closed therein.
\end{itemize}
A limit operator $L$ is said of {\em first order} if the converse of (LO1) also holds, i.e. if the following equivalence is satisfied:
  \begin{equation}
x\in L(\sigma)\iff\sigma \hbox{ converges to $x$ with respect to $\tau_{L}$}.\label{eq:8}
\end{equation}


Recall also that with any sequence $\{A_n\}_{n \in \mathbb{N}}$ of subsets of $X$ we can associate the {\em Hausdorff inferior and superior limits} of sets as
\begin{eqnarray}
\mathrm{LI}(A_{n})\equiv
\liminf(A_{n})&:=&\cup_{n=1}^{\infty}\cap_{k=n}^{\infty}A_{k} \\
\mathrm{LS}(A_{n})\equiv
\limsup(A_{n})&:=&\cap_{n=1}^{\infty}\cup_{k=n}^{\infty}A_{k}.
\end{eqnarray}
Clearly one always has $\mathrm{LI}(A_{n}) \subset \mathrm{LS}(A_{n})$. Moreover, if $\{A_m\}$ is any subsequence,
\[
\mathrm{LI}(A_{n}) \subset \mathrm{LI}(A_{m}) \subset \mathrm{LS}(A_{m}) \subset \mathrm{LS}(A_{n}),
\]
Simple examples show that these inclusions are usually strict.

Next, we define the {\em future chronological limit operator} $\hat{L}$ on $\hat{M}$ as follows. Given a sequence $\sigma=\{P_{n}\}_{n}\subset \hat{M}$ of IPs and $P\in \hat{M}$, we set

\begin{equation}
\label{eq:3}
P\in \hat{L}(\sigma)\iff \left\{\begin{array}{l}
                                 P\subset \mathrm{LI}(\sigma)\\
                                 P \hbox{ is a maximal IP in }\mathrm{LS}(\sigma).
\end{array}\right.
\end{equation}
Again, by simply interchanging past and future sets we may analogously define the {\it past chronological limit operator} $\check{L}$ on $\check{M}$. Then, the {\em future (resp. past) chronological topology on $\hat{M}$ (resp. $\check{M}$)} is the derived topology associated to the limit operator $\hat{L}$ (resp. $\check{L}$).

\smallskip

We are now ready to define the chronological topology on the full $c$-boundary. To this end, define a limit operator $L$ on $\overline{M}$ as follows: given a sequence
$\sigma=\{(P_{n},F_{n})\}\subset\overline{M}$, put

\begin{equation}
\label{eq:4}
(P,F)\in L(\sigma)\iff \left\{
  \begin{array}{l}
    P\in \hat{L}(P_{n}) \hbox{ if $P\neq \emptyset$}\\
F\in\check{L}(F_n) \hbox{ if $F\neq \emptyset$}.
  \end{array}
\right.
\end{equation}
By definition, the {\em chronological topology on $\overline{M}$} is the derived topology $\tau_L$ associated to the limit operator $L$ defined in (\ref{eq:4}).

\begin{remark}\emph{
    In general, the limit operator defined in (\ref{eq:4}) is not of first order (see, for instance, \cite[Figure 7]{FloresHausdorffseparabilityboundaries2016}). However, this property can be proven to hold under some mild and general hypotheses \cite{FloresGromovCauchycausal2013, AkeSpacetimecoveringscasual2017}, valid in most cases of physical interest. Accordingly, throughout this paper we will implicitly assume that this property holds, and so, the equivalence (\ref{eq:8}) will be systematically used.
}
\end{remark}


The following result, whose proof is given in \cite{Floresfinaldefinitioncausal2011}, summarizes the key properties of the chronological topology.

\begin{theorem}
\label{thm:mainc-completion}
Let $(M,g)$ be a strongly causal spacetime and consider its associated $c$-completion $\overline{M}$ endowed with the chronological relations and chronological topology defined in (\ref{eq:7}) and (\ref{eq:4}), respectively. Then, the following statements hold.

\begin{enumerate}[label=(\roman*)]
\item \label{item:mismatopologia}The inclusion $M\hookrightarrow \overline{M}$ is continuous, with an open dense image. In particular, the topology induced on $M$ by the chronological topology on $\overline{M}$ coincides with the original, manifold topology.
\item The chronological topology is second-countable and $T_1$-separable, but not necessarily Hausdorff.
\item \label{item:chains} Let $\{x_n\}\subset M$ be a {\em future (resp. past) chain}, i.e., a sequence satisfying that $x_n\ll x_{n+1}$ (resp. $x_{n+1}\ll x_{n}$) for all $n$. Then,
\[
\begin{array}{c}
L(\left\{ x_{n} \right\})=\left\{ (P,F)\in \overline{M}: P=I^-(\left\{ x_n \right\}) \right\}
\\
\hbox{(resp. $L(\left\{ x_{n} \right\})=\left\{ (P,F)\in \overline{M}: F=I^+(\left\{ x_n \right\}) \right\})$.}
\end{array}
\]
%\footnote{\cambiosn{NOTA IVAN:FIJAOS QUE EL TERMINO ``LIMIT POINT'' YA TIENE UN SIGNIFICADO EN TOPOLOGI�A QUE NOS ES %EXACTAMENTE LO QUE QUEREMOS. HIZE UNOS CAMBIOS PARA ELIMINAR EL TERMINO Y A VER SI NO METI LA PATA.}}

\item \label{item:c-completioncompleta} The c-completion is {\em complete} in the following sense: given any (future or past) chain $\{x_n\}\subset M$, necessarily $L(\left\{ x_{n} \right\}) \neq \emptyset$, i.e. any (future or past) chain converges in $\overline{M}$ (cf. (\ref{eq:8})).
    %\cambiosn{In particular, any inextendible timelike curve $\gamma$ on $M$ has two endpoints in $\overline{M}$.}

\item The sets $I^{\pm}((P,F),\overline{M})$ are open for all $(P,F)\in \overline{M}$.
\end{enumerate}

 \end{theorem}

 Let $\gamma:[a,b) \rightarrow M$ be an arbitrary future(resp. past)-directed causal curve. We want to extend the usual notion of {\em future (resp. past) endpoint} of $\gamma$ to points on $\overline{M}$. Concretely, we say that a pair $(P,F) \in \overline{M}$ is a {\em future (resp. past) endpoint} of $\gamma$ if for any increasing sequence $\left\{ t_{n} \right\} \subset [a,b)$ with $t_{n}\nearrow b$, we have $(P,F) \in L(\left\{ \gamma (t_{n}) \right\})$.

 The following immediate consequence of Theorem \ref{thm:mainc-completion} describes when a point $(P,F)\in\overline{M}$ is an endpoint of a causal curve, and will be very important in later sections. We shall often deal only with future endpoints, since the corresponding statements for past endpoints are easy to obtain from time-duality and will always be understood.

\begin{corollary}
\label{cor:endpoints}
Suppose $(M,g)$ is strongly causal spacetime with $c$-completion $\overline{M}$ as above. Let $\gamma:[a,b) \rightarrow M$ be a future-directed causal curve in $M$ and $(P,F) \in \overline{M}$. Then, the following statements hold.
\begin{enumerate}[label=(\roman*)]
\item \label{item:endpoints1}If $P=I^-(p)$ and $F=I^+(p)$ for some $p \in M$, i.e., if $(P,F)$ is a point of $M$ via its natural inclusion, then $(P,F)$ is a future endpoint of $\gamma$ (in the extended sense) if, and only if, for any increasing sequence $\left\{ t_{n} \right\} \subset [a,b)$ with $t_{n}\nearrow b$, we have $\gamma (t_{n}) \rightarrow p$ in $M$. (In order words, $(P,F)$ is a future endpoint of $\gamma$ in the extended sense iff $p$ is an endpoint in the ordinary, spacetime sense.)
\item \label{item:endpoints2} If $\gamma$ is a future (resp. past) directed {\em timelike} curve, then
\begin{equation}
\label{keyeq1}
(P,F)\in\overline{M} \mbox{ is a future endpoint of $\gamma$} \Longleftrightarrow P=I^-(\gamma).
\end{equation}
Moreover, when $\gamma$ is timelike, then it always has some endpoint on $\overline{M}$.
\item \label{item:endpoint3}If the future-directed causal curve $\gamma$ has $(P,F)$ as future endpoint with $P\neq \emptyset$, then $P=I^{-}(\gamma)$.
\end{enumerate}
\end{corollary}

\begin{proof}
$(i)$ By definition, $(P,F)$ is a future endpoint of $\gamma$ (in the extended sense) if, and only if, for any increasing sequence $\left\{ t_{n} \right\} \subset [a,b)$ with $t_{n}\nearrow b$,
\[
(P,F) \in L(\left\{ \gamma (t_{n}) \right\}) \Longleftrightarrow (I^-(\gamma (t_{n})),I^+(\gamma (t_{n}))) \rightarrow (P,F) \mbox{ in $\overline{M}$}\Longleftrightarrow \gamma (t_{n}) \rightarrow p \mbox{ in $M$},
\]
where we have used the definition of endpoint in the extended sense for the first equivalence, the fact that the limit operator is first-order (cf. (\ref{eq:8})) to obtain the second equivalence, and the fact that the induced topology on $M$ is the manifold topology according to item \ref{item:mismatopologia} of Thm. \ref{thm:mainc-completion} to get the third equivalence.
\\
$(ii)$ If $\gamma$ is timelike, then for any increasing sequence $\left\{ t_{n} \right\} \subset [a,b)$ with $t_{n}\nearrow b$ the corresponding sequence $\left\{ \gamma (t_{n})\right\}$ is a future chain in $M$, (\ref{keyeq1}) follows immediately from items \ref{item:chains} and \ref{item:c-completioncompleta} of Thm. \ref{thm:mainc-completion}. \\
$(iii)$ Suppose $(P,F)$ is a future endpoint of the causal curve $\gamma$ with $P\neq \emptyset$, and take again any increasing sequence $\left\{ t_{n} \right\} \subset [a,b)$ with $t_{n}\nearrow b$, so that $(P,F) \in L(\left\{ \gamma (t_{n}) \right\})$. According to the definition of the limit operator $L$, Eq. (\ref{eq:4}), we have that
\[
P \in \hat{L}(\left\{ I^-(\gamma (t_{n}) \right\}),
\]
which in turn means, according to Eq. (\ref{eq:3}), that
\[
P\subset \mathrm{LI}(\left\{ I^-(\gamma (t_{n}) \right\}) \subset I^-(\gamma).
\]
Now, clearly $I^-(\gamma)\subset \mathrm{LS}(\left\{ I^-(\gamma (t_{n}) \right\})$, and the maximality of $P$ as an IP
therein, as required by (\ref{eq:3}), now yields that $P=I^-(\gamma)$, as desired.
\end{proof}

\begin{remark}\label{r}{\em The situation when $\gamma$ is causal but not timelike is more involved. The converse of item \ref{item:endpoint3} is no longer true (see \cite[Section 3.5]{Floresfinaldefinitioncausal2011}). Indeed, inextendible causal curves may not have endpoints in the c-completion. It is therefore natural to consider the following definition (compare with  \cite[Definition 3.33]{Floresfinaldefinitioncausal2011}).}
\end{remark}
%
\begin{definition}\label{def:properly-causal}
  A $c$-completion $\overline{M}$ is said to be {\em properly causal} if any future or past-inextendible causal curve in $M$ has an endpoint on $\overline{M}$. \end{definition}
  % \cambiosn{We will say that $\overline{M}$ is {\em strongly properly causal} if given $(P,F)\in\overline{M}$ and $\eta$ a future-directed (resp. past-directed) causal curve, we have:}\footnote{\cambiosn{No haría falta la nocion de "strongly properly causal" si suponemos la topologia de primer orden, verdad?.}}
%     \begin{equation}
% (P,F)\hbox{ is an endpoint of $\eta$}\iff P=I^-(\eta)\hbox{ (resp. $F=I^+(\eta)$)}.\label{eq:1}
% \end{equation}
% \end{definition}
% \noindent \cambiosn{Clearly, strongly properly causal implies properly causal.}
\noindent Some conditions ensuring the properly causal property were summarized in \cite[Theorem 3.35]{Floresfinaldefinitioncausal2011}. For instance, a spacetime is properly causal if it is globally hyperbolic. In order to obtain more general results, we will take a special look at \cite[Theorem 3.35 (iii)]{Floresfinaldefinitioncausal2011}. In fact, that theorem motivates the following definition.

\begin{definition} A future-inextendible (resp. past-inextendible) causal curve $\gamma:[a,b)\rightarrow M$ is {\em future-regular} (resp. {\em past-regular}) if
 \[
 \uparrow \gamma = \uparrow I^{-}(\gamma)\quad
    \left({\rm resp.} \downarrow \gamma = \downarrow I^{+}(\gamma) \right).
\]
 \end{definition}
\begin{remark}\label{r2}{\em
  Clearly, the inclusion $\uparrow \gamma \subset \uparrow I^{-}(\gamma)$ holds for any future-inextendible causal curve $\gamma$. Moreover, any future-inextendible timelike curve is future-regular (with time-dual analogous statements). In a globally hyperbolic spacetime the common future $\uparrow I^-(\gamma)$ of any future-directed causal curve $\gamma$ is empty. In particular, any future-inextendible causal curve is future-regular. Analogously, past-inextendible causal curves are always past-regular in a globally hyperbolic spacetime. On the other hand, there are very simple examples where future(past)-regularity fails. For instance, take the flat $2d$ Minkowski spacetime $(\mathbb{R}^2, -dt^2+dx^2)$ and delete the spacelike half-axis $t=0,x\geq 0$. Then, the null geodesic generator of the past of the origin $(0,0)$ on the side of positive $x$ is not future-regular.}
\end{remark}
\noindent The following result shows that only a few very specific causal curves can fail to be regular (see also \cite[Prop. 3.2]{Florescausalboundarywavetype2008}.

  \begin{proposition}
    Let $\gamma:[a,b)\rightarrow M$ be a future-inextendible (resp. past-inextendible) causal curve. If there is no $c\in [a,b)$ such that $\gamma|_{(c,b)}$ is a null geodesic ray, then $\gamma$ is future(resp. past)-regular.
  \end{proposition}
  \begin{proof}
     Let us focus on the future case (the past case is proved analogously). From the hypothesis, there exists a sequence $\left\{ t_{n} \right\}$ with $t_{n}\nearrow b$ such that $\gamma(t_{n})\ll \gamma(t_{n+1})$ for all $n$. Therefore, we can construct a timelike curve $\eta$ by concatenating timelike curves connecting consecutive points of the sequence. By construction, $\eta$ and $\gamma$ have the same past and the same common future, and the result follows from the fact that any timelike curve $\gamma$ is future-regular.
  \end{proof}

\noindent On the other hand, we shall often need to work with null geodesic rays; to make them treatable, we introduce the following definition:
  \begin{definition}
\label{def:sproperlycausal} \noindent A c-completion $\overline{M}$ is {\em strongly properly causal} if any future-inextendible (resp. past-inextendible) null geodesic ray is future(resp. past)-regular.
  \end{definition}
\noindent Clearly, strong proper causality implies proper causality (see \cite[Theorem 3.35 (iii)]{Floresfinaldefinitioncausal2011}). One might ask whether there is any natural causal condition on $(M,g)$ which implies strong proper causality. Note, however, that the flat $2d$ spacetime we construct in Remark \ref{r2} is {\em stably causal}, since the coordinate $t$ gives a time-function therein. Yet, it turns out that {\em causal continuity}, a causal condition immediately stronger than stable causality in the causal ladder (see Section 3.3 of Ch. 3 in \cite{BeemGlobalLorentzianGeometry1996}), holds in $M$:

\begin{proposition}\label{prop:causalcontinuity}
  The $c$-completion of any causally continuous spacetime $(M,g)$ is strongly properly causal.
\end{proposition}
\begin{proof}
Suppose, by way of contradiction, that $(M,g)$ is causality continuous but there exists a future-inextendible null geodesic ray $\gamma:[a,b) \rightarrow M$ which is not future-regular. Since we always have $\uparrow \gamma \subset \uparrow I^{-}(\gamma)$ (cf. Remark \ref{r2}), there exists $p \in \uparrow I^{-}(\gamma)\setminus \uparrow \gamma$. Let $q \in I^-(p)$ such that $I^{-}(\gamma) \subset I^-(q)$, and pick $q \ll q'\ll p$. Fix $t_0 \in [a,b)$. Since any sequence $\{x_n\}$ in $I^-(\gamma(t_0)) \subset I^-(q)$ such that $x_n \rightarrow \gamma(t_0)$ is in $I^-(q)$, we infer that $\gamma(t_0) \in \overline{I^-(q)}$. The causal continuity now implies (see Lemmas 3.42 and 3.43 of \cite{Minguzzicausalhierarchyspacetimes2008}) that $ q \in \overline{I^+(\gamma(t_0))}$, and hence $q' \in I^+(\gamma(t_0))$. Since $t_0$ is arbitrary, we conclude that $\gamma[a,b) \subset I^-(q')$. But then one concludes that $p \in \uparrow \gamma$, in contradiction.
\end{proof}

The importance of strong proper causality lies in that it provides the following characterization for the endpoints of any causal curve (which extends the equivalence (\ref{keyeq1}) for timelike curves):
\begin{proposition}\label{prop:strongfirst}
  Let $\overline{M}$  be the c-completion of a strongly causal spacetime $(M,g)$, and assume that $\overline{M}$ is strongly properly causal. A pair $(P,F)$ is endpoint of a future(resp. past)-directed, future(resp. past)-regular causal curve if, and only if, $P=I^{-}(\gamma)$ (resp. $F=I^{+}(\gamma)$).
\end{proposition}
\begin{proof}
Apply the same ideas as in the proofs of Theorem \ref{thm:mainc-completion} and \cite[Theor. 3.35 (iii)]{Floresfinaldefinitioncausal2011}.
\end{proof}



%%% Local Variables:
%%% mode: latex
%%% TeX-master: "nullinfinityV5"
%%% End:
