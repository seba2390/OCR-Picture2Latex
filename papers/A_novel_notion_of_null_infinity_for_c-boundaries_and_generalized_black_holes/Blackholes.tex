\section{Generalized black holes}\label{bh2}
We now use the new notion of null infinity to define a generalized notion of black hole. There is, of course, little choice for this particular definition once we have defined ${\cal J}^+$.

\begin{definition}
\label{defbh} Let $(M,g)$ be a spacetime. We shall say that a point $p \in M$ is {\em visible from (the future null)  infinity} ${\cal J}^+$ if there exists a future-inextendible null geodesic $\alpha$ starting at $p$ \cambios{and with future endpoint} some $(P,F) \in {\cal J}^+$. We denote the set of all points of $M$ visible from infinity by $V_{\infty}$. If ${\cal J}^+ \neq \emptyset$ (or equivalently, if $V_{\infty} \neq \emptyset$), then the {\em black hole (region)} of $(M,g)$ and the {\em (future) event horizon} are, respectively,
\[
B^+ := M\setminus J^-(V_{\infty}),
\]
and
\[
H^+:= \partial B^+.
\]
\end{definition}
\noindent (We require that ${\cal J}^+ \neq \emptyset$ in this definition to avoid that $M = B^+$.)

\begin{remark}
\label{rmk5}
{\em Given any point $p \in V_{\infty}$, and a future-complete null geodesic $\alpha:[0,+\infty) \rightarrow M$ starting
at $p$ with endpoint at ${\cal J}^+$, we have $I^-(\alpha) = I^-(\alpha|_{[t,+\infty)})$ for any $t \in [0,+\infty)$, and therefore any point along $\alpha$ is also visible from infinity.}
\end{remark}

Some of the more basic properties of black holes in this context are summarized in the following proposition. We particularly call the reader's attention to BH4), which precisely clinches the notion of a black hole as a region where causal communication with infinity (i.e., ``distant observers'') is forbidden.
\begin{proposition}
\label{bhprops}
The following properties of the black hole region $B^+$ and the event horizon $H^+$ hold.
\begin{itemize}
\item[BH1)] $I^+(B^+) \subset int(B^+)$. In particular, the interior $int(B^+)$ of the black hole is a future set. Moreover, the event horizon is the common boundary between $int(B^+)$ and the past set $I^-(V_{\infty})$, i.e., $M$ is the disjoint union
\begin{equation}
\label{equality}
M = int(B^+) \dot{\cup} H^+ \dot{\cup}I^-(V_{\infty}).
\end{equation}
In particular, $H^+$ is an {\em achronal boundary} (and hence \cite{ONeillSemiRiemannianGeometryApplications1983,PenroseDifferentialTopology1972} a closed $C^0$ hypersurface) in $M$.
\item[BH2)] For any  $p \in H^+\cap J^-(V_{\infty})$ there exists a future-complete null geodesic ray $\eta \subset H^+$ starting at $p$ with future endpoint at ${\cal J}^+$.
\item[BH3)] $\overline{B^+} = M \setminus I^-(V_{\infty}) = M \setminus I^-({\cal J}^+)$.
\item[BH4)] $p \in M \setminus B^+$ if and only if there exists a future-directed causal curve in $M$ starting at $p$ with a future endpoint on ${\cal J}^+$.
\end{itemize}
\end{proposition}
\begin{proof}
  \textit{BH1):} Let $q,p \in M$, with $q \in B^+$ and $q \ll_g p$. Then,
\[
q \notin J^-(V_{\infty}) \Longrightarrow p \notin  J^-(V_{\infty}) \Longrightarrow p \in B^+,
\]
but this actually proves that $I^+(q) \subset B^+$, and since $I^+(q)$ is open and contains $p$, that $p \in int(B^+)$. Now, $\overline{B^+} = int(B^+)\dot{\cup} H^+$, and
\[
p \notin \overline{B^+} \Longrightarrow \exists U \ni p \mbox{ open with }U \cap B^+ =\emptyset \Longrightarrow U \subset J^-(V_{\infty}) \Longrightarrow p \in I^-(V_{\infty}).
\]
That is, $M \setminus \overline{B^+} \subset I^-(V_{\infty})$. These implications are easily reversed, so also $I^-(V_{\infty}) \subset M \setminus \overline{B^+}$. Therefore, $M\setminus\overline{B}^+=I^{-}(V_{\infty})$, whence (\ref{equality}) follows.

\smallskip

\textit{BH2):} Let $p \in H^+\cap J^-(V_{\infty})$, and let $\alpha$ be a future-directed causal curve segment starting at $p$ and ending at some point $q \in V_{\infty}$. By the definition of visible from infinity, there exists a future-complete null geodesic $\gamma$ starting at $q$ with a future endpoint $(P,F) \in {\cal J}^+$. Let $\eta$ be the juxtaposition of these two curves. This is a future-inextendible causal curve starting at $p$. Now, we claim that its image has to be globally achronal. Otherwise, we could find some $r \in \gamma$ with $p \ll_g r$. But $H^+ \equiv \partial I^-(V_{\infty})$ by BH1), and $r \in J^-(V_{\infty})$ (cf. Remark \ref{rmk5}), so this would mean that $p \in I^-(V_{\infty})\cap \partial I^-(V_{\infty})$, which is impossible. But since $\eta$ is globally achronal, it can be affinely reparametrized as null geodesic ray (which we still call $\eta$). Since $\gamma$ is future-complete, so is $\eta$. Clearly, $(P,F)$ is also an endpoint for $\eta$ which thus has a future endpoint on ${\cal J}^+$, as desired.

\smallskip

\textit{BH3):} The equality $\overline{B^+} = M \setminus I^-(V_{\infty})$ follows immediately from (\ref{equality}). Let $p \in I^-(V_{\infty})$. Then for some $q \in I^+(p)$, there exist a future-directed  null geodesic $\gamma$ starting at $q$ and with an endpoint $(P,F) \in {\cal J}^+$. Moreover, taking into account that $P\neq \emptyset$ (recall Remark \ref{rem:1}), item \ref{item:endpoint3} of Corollary \ref{cor:endpoints} implies that $P=I^{-}(\gamma)$. But this means $p \in P=I^-(\gamma)$ and hence, from \eqref{eq:7}
\[
(I^-(p),I^+(p))\ll (P,F) \in {\cal J}^+,
\]
that is, $(I^-(p),I^+(p)) \in I^-({\cal J}^+)\cap M$. Therefore  we have
\[
I^-(V_{\infty}) \subset I^-({\cal J}^+) \cap M.
\]
Conversely, let $p\equiv (I^-(p),I^+(p)) \ll (P,F) \in {\cal J}^+$. This means that $p \in P=I^-(\gamma)$ for some future-complete  null ray $\gamma$, and so $p\ll_g q$ for some $q \in \gamma$. But then $q$ is visible from infinity (by Remark \ref{rmk5}), so $p \in I^-(V_{\infty})$. This now implies the opposite inclusion \[ I^-({\cal J}^+) \cap M\subset I^-(V_{\infty}), \] and hence $M \setminus I^-(V_{\infty}) = M \setminus I^-({\cal J}^+)$.

\smallskip

\textit{BH4):} Let $\gamma:[0,A) \rightarrow M$ ($A \leq +\infty$) be a future-inextendible causal curve starting at $p =\gamma(0)$ and with a future endpoint $(P,F) \in {\cal J}^+$. Either $\gamma$ is a (necessarily future-complete) null geodesic, in which case $p$ is visible from infinity, or else $p \in I^-(\gamma)$. But then $(I^-(p),I^+(p)) \in I^-({\cal J}^+)$, and hence $p \in I^-(V_{\infty})$ from the proof of \textit{BH3)}. In any case $p \in J^-(V_{\infty})$ and so $p \notin B^+$.

Conversely, if $p \notin B^+$ then there exists a future-complete null geodesic $\gamma:[0,+\infty) \rightarrow M$ with $p \in J^-(\gamma(0))$
such that some $(P,F)\in {\cal J}^+$ is endpoint of $\gamma$. We can therefore juxtapose a future-directed causal curve from $p$ to $\gamma(0)$ with $\gamma$ to obtain a future-inextendible causal curve $\beta$ starting at $p$. Clearly, $(P,F)$ is also endpoint of $\beta$.
\end{proof}

At least in some important cases, one can draw a direct connection between the absence of a black hole region and nonspacelike geodesic completeness.

\begin{proposition}
\label{completeness1}
Suppose that $(M,g)$ is globally hyperbolic and future null complete, with non-compact Cauchy hypersurfaces. Then ${\cal J}^+ \neq \emptyset$ but $B^+ = \emptyset$.
\end{proposition}
\begin{proof}
Let $p \in M$. Assume first that $\partial I^+(p)$ is compact. In this case, a standard argument (see, e.g., the proof of Theorem 61, Ch.14, p. 437 of \cite{ONeillSemiRiemannianGeometryApplications1983}) implies that $\partial I^+(p)$ is homeomorphic to a given Cauchy hypersurface $S\subset M$, which is absurd. Therefore, $\partial I^+(p)$ is non-compact, and again by simple arguments using limit curves (cf. the proof of Prop. 8.18, Ch. 8, p. 289 of \cite{BeemGlobalLorentzianGeometry1996}) we conclude that there exists an inextendible future-directed null geodesic ray $\eta$ starting at $p$. Now, consider the TIP $P=I^{-}(\eta)$. By Remark \ref{r0}, we can pick $F \in \check{M}\cup\{\emptyset\}$ such that $(P,F) \in \overline{M}$ (actually, $F\equiv \emptyset$ will do). Since $P$ is a TIP, $(P,F) \in \partial M$. By Remark \ref{r}, global hyperbolicity guarantees that $\eta$ is future-regular, and Proposition \ref{prop:strongfirst} now implies that $(P,F)$ is a future endpoint of $\eta$. Finally, null geodesic completeness and the future-regularity of $\eta$ now imply that $(P,F) \in {\cal J}^+$. We conclude that $p \notin B^+$ by Proposition \ref{bhprops} BH4).
\end{proof}

If we assume some extra natural geometric conditions on $(M,g)$, we can show that the general conclusion of the previous proposition can be extended to strongly causal spacetimes.

\begin{theorem}
\label{completeness2}
Suppose that the strongly causal spacetime $(M^{n+1},g)$, with $n\geq 2$, satisfies the following conditions:
\begin{itemize}
\item[(a)] $(M,g)$ is timelike and null geodesically complete;
\item[(b)] $(M,g)$ satisfies the {\em timelike convergence condition}, i.e., $Ric(v,v)\geq 0$ for any timelike $v\in TM$;
\item[(c)] ${\cal J}^+ \neq \emptyset$;
  \item[(d)] $\overline{M}$ is strongly properly causal (recall this holds, in particular, if $(M,g)$ is causally continuous - cf. Prop. \ref{prop:causalcontinuity}.)
\end{itemize}
Then $B^+ = \emptyset$.
\end{theorem}
\begin{proof}
Suppose, by way of contradiction, that $(a)-(d)$ do hold, but $B^+ \neq \emptyset$. In this case, Proposition \ref{bhprops} BH1) also implies that ${\rm int} (B^+) \neq \emptyset$; therefore, we can pick $p \in {\rm int} (B^+)$. Now, if there exists a future-directed null ray $\eta$ starting at $p$, from $(a)$ and $(d)$ it must be both future-complete and future-regular. Therefore, arguing exactly as in the end of the proof of Proposition \ref{completeness1} with strong proper causality in lieu of global hyperbolicity, we would conclude that $p \notin B^+$, a contradiction. But if such a ray does not exist, then $p$ is a {\em future trapped set}, i.e., its future horismos $E^+(p):= J^+(p) \setminus I^+(p)$ is compact, again by the proof of  of \cite[Prop. 8.18, Ch. 8, p. 289]{BeemGlobalLorentzianGeometry1996}. Therefore, since $(M,g)$ is strongly causal, the latter result together with \cite[Theor. 8.13]{BeemGlobalLorentzianGeometry1996} imply that $(M,g)$ admits a causal line intersecting $E^+(p)$, $\gamma$ say.

We claim that $\gamma$ is actually a {\em timelike} line. Without loss of generality we may assume that $\gamma$ is future-directed. Now, Proposition \ref{bhprops} BH1) means that $int(B^+)$ is a future set, and hence $E^+(p)$ is contained in ${\rm int}(B^+)$. Therefore, we can pick some $x \in {\rm int}(B^+) \cap \gamma$. If $\gamma$ were null, then its portion to the future of $x$ would be a future-directed null geodesic ray, that can be considered complete and future-regular, that is, (again arguing exactly as in the proof of Proposition \ref{completeness1}) with future endpoint on ${\cal J}^+$, which again contradicts Proposition \ref{bhprops} BH4). Thus, $\gamma$ has to be timelike.

However, conditions $(a)$ and $(b)$, together with the existence of a (complete) timelike geodesic line mean that we can apply the Lorentzian Splitting Theorem (see, e.g., Ch. 14 of \cite{BeemGlobalLorentzianGeometry1996} and references therein) to conclude that $(M,g)$ is actually {\em isometric} to a product spacetime $(\mathbb{R}\times S,-dt^2\oplus h)$, where $(S,h)$ is a complete Riemannian manifold.

Now, Theorem 3.67 in Ref. \cite{BeemGlobalLorentzianGeometry1996} implies that $(M,g)$ is actually globally hyperbolic, with Cauchy hypersurfaces homeomorphic to $S$. Hence $S$ cannot be non-compact, for in that case Proposition \ref{completeness1} would mean that $B^+ \equiv \emptyset$, contrary to our assumption. We conclude that $S$ is compact.

But then no future-inextendible null geodesic can be achronal, i.e., a geodesic ray; so that in this case we would have ${\cal J}^+ \equiv \emptyset$, contradicting $(c)$. This final contradiction ends the proof.
\end{proof}


In order to get deeper results out of our black hole definition, we need to refine it. We shall need some additonal conditions ensuring that future null infinity is ``good enough''. The following definition is meant to encode this.
\begin{definition}
\label{ample}
The future null infinity ${\cal J}^+$ is said to be:
\begin{itemize}
\item[(A1)] {\em Ample} if for any compact set $C \subset M$, and for any connected component ${\cal J}^+_0$ of ${\cal J}^+$,
%${\cal J}^+_0\cap (\overline{M} \setminus \overline{I^+ (C)})$ is non-empty.
${\cal J}^+_0\cap (\overline{M} \setminus \widetilde{I^+ (C)})$ is a non-empty open set, where
   \begin{equation}
\widetilde{I^+(C)}:=\{(P,F)\in \overline{M}: I^-(x)\subset P \mbox{ for some $x\in C$}\}.\label{eq:2}
\end{equation}
(Roughly speaking, no connected component of future null infinity can be entirely contained in the future of a compact set.)

% \item[(A2)] {\em Past-complete} if for any $(P,F) \in {\cal J}^+$ and $(P',F') \in \partial M$ with $\cambiosj{\emptyset\neq} P' \subset P$ \cambiosj{and $F'\cap P=\emptyset$,} we have $(P',F') \in {\cal J}^+$. \cambiosn{(Roughly speaking, any element of the $c$-boundary which is in the horismotic past of ${\cal J}^+$ must also be in ${\cal J}^+$.)}\footnote{\cambiosn{Igualmente, habria sido bonito poder sustituir $P' \subset P$ por $(P',F')\leq (P,F)$, pero, de nuevo, parece que no funciona.}}

\item[(A2)] \emph{Past-complete} if given $(P,F)\in \mathcal{J}^+$, any $(P',F')\in\partial M$ with $P'=I^-(\eta)$, being $\eta$ a future-directed inextendible null geodesic generator of $\partial P$, also belong to $\mathcal{J}^+$.

% \item[(A3)] \textit{Hausdorff-compatible:} if for any pairs $(P,F),(P',F')\in \partial M$ with $(P,F)\in {\cal J}^+$ and $P,P'$ Hausdorff-related, then $(P',F')$ also belong to ${\cal J}^+$. That is, if two boundary points are topologically \textit{close} and one is included in the null infinity, then the other is also included.\footnote{Jony: Esto creo que es fácil de construir. Si lo veis conveniente se puede meter el ejemplo.}

\end{itemize}
We will say that the future null infinity ${\cal J}^+$ is \emph{regular} if it is both ample and past-complete.
\end{definition}
% \begin{remark}
%  INCLUIR ALGUNAS MOTIVACIONES DE LAS DEFINICIONES ANTERIORES.
% \end{remark}
Condition (A1) in the previous definition means that the future of compact sets cannot encopass the whole future null infinity, and (A2) that ${\cal J}^+$ contains any point on the future boundary which may lie in its past. Together, they mean that the future null infinity is "big" in a precise sense. These assumptions are not really restrictive when some classical examples of physical interest are considered. In fact, on the one hand, the condition that the {\em conformal} $\mathcal{J}^+$ escapes from the future of any compact set holds, for instance, in many standard solutions of Einstein field equation, with vanishing cosmological constant, admitting a conformal completion for which conformal null infinity $\mathcal{J}^+$ is a null hypersurface in the extended spacetime having past-complete null generators, such as those in the Kerr-Newman family with suitable parameters. It also holds in some solutions to the Einstein fields equation with a negative cosmological constant, such as the Schwarzschild-Anti de Sitter spacetime, for example. On the other hand, it fails on, say, the Schwarzschild-de Sitter solutions, and will tend to fail, more generally, on globally hyperbolic spacetimes with compact Cauchy hypersurfaces. In any case, the seemingly technical assertion on the open character of the intersection considered in (A1) holds in very general situations, as becomes apparent from the following result:

% \cambiosn{The following result shows that the open condition for ${\cal J}^+_0\cap (\overline{M} \setminus \widetilde{I^+ (C)})$ in (A1) is not very restrictive, since it is guaranteed just if $\hat{M}$ is Hausdorff.}

%An important comparison with the analogous situation for conformal boundaries can be made. For this, we need first the following technical lemma:

\begin{proposition}
\label{lema:auxiliar}
If $\hat{M}$ is Hausdorff, then $\widetilde{I^+(C)}$ is a closed set.
\end{proposition}
\begin{proof}
Let $\left\{(P_n,F_n)  \right\}_{n}\subset \widetilde{I^+(C)}$ be a sequence and consider $(P,F)\in L(\left\{ (P_{n},F_n) \right\}_{n})$. Our aim is to prove that $(P,F)\in \widetilde{I^{+}(C)}$. For any $n$, let $x_{n}\in C$ be a point so $I^-(x_n)\subset P_{n}$. Observe that, as $C$ is compact, we can assume (up to a subsequence) that $\{x_{n}\}_{n}$ converges in $C$ to a point, say $x^{*}\in C$.

From such a convergence it follows (see \cite[Remark 3.17]{Floresfinaldefinitioncausal2011}) that $I^{-}(x^{*})\subset \mathrm{LI} (\left\{ I^-(x_{n}) \right\}_{n})$, and so, that $I^{-}(x^{*})\subset \mathrm{LI} (\left\{ P_n\right\}_{n})$. From standard arguments involving Zorn's Lemma, and up to a subsequence, we can ensure the existence of a IP $\overline{P}$ so $I^{-}(x^{*})\subset \overline{P}$ and $\overline{P}$ is maximal on $\mathrm{LS} (\left\{ P_{n} \right\}_{n})$, i.e., $\overline{P}\in \hat{L}(\left\{ P_{n} \right\}_{n})$. As $\hat{M}$ is Hausdorff and $P\in \hat{L}(\left\{ P_n \right\}_{n})$, $P=\overline{P}\supset I^{-}(x^{*})$, proving that $(P,F)\in \widetilde{I^+(C)}$.
\end{proof}

Condition (A2) in Definition \ref{ample} is comparatively more restrictive than (A1). As we will see on the Appendix, this condition seems unavoidable if we are to obtain the following theorem, insofar as imposing stronger requirements on the causality of the underlying spacetime will not help one to evade it.


%\begin{proposition}
%\label{voila2}
%Let $(\tilde{M},\tilde{g})$ be a future-nesting globally hyperbolic conformal extension of a globally hyperbolic spacetime $(M,g)$. Assume in addition that ${\cal J}^+_c$ is a smooth {\em null} hypersurface in $(\tilde{M},\tilde{g})$ whose null geodesic generators are past-inextendible in $I^+(M,\tilde{M})$. Then:
%\begin{itemize}
%\item[a)] For any compact set $C \subset M$, and for any connected component ${\cal I}^+_0$ of ${\cal J}_c^+$, ${\cal I}^+_0\cap [(\partial_i^+M \cup M) \setminus I^+ (C, \tilde{M})]\neq \emptyset$.
%\item[b)] If $p \in \partial^+M$, $q \in {\cal J}_c^+$, and $p \leq _{\tilde{g}} q$, then $p \in {\cal J}_c^+$.
%\item[c)] If the future null infinity ${\cal J}^+$ of $(M,g)$ is connected, then it is regular.
%\end{itemize}
%\end{proposition}
%
%\begin{proof}
%  $(a)$ Given any $p \in {\cal I}^+_0$, the past-inextendible null generator of ${\cal J}^+_c$ starting at $p$ must eventually leave the compact set $J^+(C,\tilde{M})\cap J^-(p,\tilde{M}) \subset I^+(M,\tilde{M})$ since $(\tilde{M},\tilde{g})$ is strongly causal, and will then leave $I^+ (C, \tilde{M})$ while still contained inside $\partial ^+M$.
%
%  \smallskip
%
%$(b)$ If $p \in \partial^+M$, $q \in {\cal J}_c^+$, and $p \leq _{\tilde{g}} q$, let $\alpha:[0,1] \rightarrow \tilde{M}$ be a past-directed causal curve from $q$ to $p$. Since $\partial^+M$ is achronal (cf. Corollary \ref{convexitygh3}), $\alpha$ must {\em initially} (i.e., near $q$) be, up to reparametrization, a piece of the past-directed null generator of ${\cal J}^+_c$ starting at $q$. Since this is past-inextendible in $I^+(M,\tilde{M})$, it cannot leave the connected component of ${\cal J}^+_c$ containing $q$, and hence $\alpha$ must be {\em entirely} contained in such a generator. Therefore $p \in {\cal J}^+_c$ as well.
%
%\smallskip
%
%\cambiosn{$(c)$ Consider again the homeomorphism $\Psi$ in Theorem \ref{thm:causaltoconformal} taking
%$\overline{M}$ to $\overline{M}_i^{*}$. It follows then that $\overline{M}$ is both, Hausdorff and simple (see \cite[Definition 2.4]{FloresGromovCauchycausal2013}), hence $\hat{M}$ is Hausdorff. In particular, $\widetilde{I^{+}(C)}$ is closed from previous lemma. }
%
%\cambiosn{In order to prove that $\mathcal{J}^{+}$ is regular, we need to show that it is both, ample and past-complete. The latter is quite straightforward once we recall that both $\overline{M}$ and $\overline{M}_{i}^{*}$ are chronologically isomorphic and assertion (b), so we will focus on the ample condition.}
%
%\cambiosn{Let $C$ be a compact set. Observe that we need to prove that
%\begin{equation}
%\mathcal{J}^{+}\cap \left( \overline{M}\setminus \widetilde{I^{+}(C)} \right)\neq \emptyset,\label{eq:5}
%\end{equation}
%as the openness follows from the fact that $\widetilde{I^{+}(C)}$ is closed. Assume by contradiction that \eqref{eq:5} is empty, so $\mathcal{J}\subset \widetilde{I^+(C)}$. Since $\mathcal{J}_{c}^{+}\subset \Psi(\mathcal{J}^{+})$ by Corollary \ref{voila}, from assertion (a) we can prove that
%\begin{equation}
%  \mathcal{J}^{+}\cap \left( \overline{M}\setminus I^{+}(K) \right)\neq \emptyset\label{eq:6}
%\end{equation}
%for any compact set $K$. Let $X$ be a past-directed timelike vector field defined over a neighbourhood $U$ around $C$, and denote by $\varphi_{X}:W\subset U\times \mathbb{R}\rightarrow M$ its associated flow. As $C$ is compact, there exists $t_{0}>0$ so $\varphi^{t_{0}}_{X}(x):=\varphi_{X}(x,t_{0})$ is defined for all $x\in C$.}
%
%\cambiosn{Let us then define $C^{t_{0}}=\varphi^{t_{0}}_{X}(C)$ which is a compact set satisfying, from the timelike character of $X$, that $\widetilde{I^{+}(C)}\subset I^{+}(C^{t_{0}})$. But then $\mathcal{J}^{+}\subset \widetilde{I^+(C)}\subset I^+(C^{t_0})$ and hence
%\[
%  \mathcal{J}^{+}\cap \left( \overline{M}\setminus I^{+}(C^{t_{0}}) \right)= \emptyset,
%  \]
%a contradiction with \eqref{eq:6}.}
%
%\end{proof}


% We are now ready to state and prove the main result in this section.

\begin{theorem}
\label{main}
Assume that ${\cal J}^+$ is regular and let $C \subset M$ be an {\em achronal} compact set. If $C$ is not entirely contained in $B^+$, then there exists a future-complete null $C$-ray $\eta: [0,+\infty) \rightarrow M$ \cambios{with endpoint} $(P,F) \in {\cal J}^+$.
\end{theorem}
\begin{proof}
First, assume that $C$ is not entirely contained in $\overline{B^+}$. Then, BH3) in Proposition \ref{bhprops} implies that ${\cal J}^+\cap I^+ (C)\neq \emptyset$. Let $x \in {\cal J}^+\cap I^+ (C)$, and pick a TIP $P_0$ such that $x \in P_0$ and $(P_0,F_0)\in {\cal J}^+$. Denote by ${\cal J}^+_0$ the connected component of ${\cal J}^+$ containing $(P_0,F_0)$.

\cambios{Since ${\cal J}^+$ is ample, we have
\[
{\cal J}_0^+\cap(\overline{M}\setminus I^+(C))\supset {\cal J}_0^+\cap(\overline{M}\setminus\widetilde{I^+(C)}) \neq \emptyset.
\]
So, taking into account that ${\cal J}^+_0$ is connected, and $(P_0,F_0)\in {\cal J}^+_0\cap I^+ (C)\neq\emptyset$. Thus,
\[
\partial_{{\cal J}^+_0}({\cal J}^+_0\cap I^+ (C))\neq\emptyset.
\]
Take some $(P,F)\in \partial_{{\cal J}^+_0}({\cal J}^+_0\cap I^+ (C))$. Since $I^+ (C)$ is open, it follows that $(P,F) \notin I^+ (C)$ and in particular $C \cap P = \emptyset$.}

We now claim that $I^{-}(x_0)\subset P$ for some $x_0\in C$. Suppose, by way of contradiction, that this is false. Then, it follows that $(P,F)\in U=\mathcal{J}_{0}^{+}\cap\left(\overline{M}\setminus\widetilde{I^{+}(C)}\right)$, which is an open set from the ample condition. As $I^{+}(C)\subset \widetilde{I^{+}(C)}$, then $I^{+}(C)\cap U=\emptyset$. Hence, the point $(P,F)$ belongs to an open set with empty intersection with $I^{+}(C)$, in contradiction with $(P,F)\in\partial_{{\cal J}^+_0}(\mathcal{J}^+_{0}\cap I^+ (C))$. This establishes the claim.

Therefore, $x_0 \in \overline{I^{-}(x_0)}\subset \overline{P}$, but $x_0\in C\setminus P$. We conclude that $x_0 \in\partial_{M} P$. Since $(P,F) \in \partial M$, in particular $P$ is a TIP. Its boundary in $M$ is thus a union of future-inextendible null geodesics, so that we can take $\eta$ the future-inextendible null geodesic generator of $\partial_{M} P$ starting at $x_0$. Note that since $I^-(\eta)$ is also a TIP, we conclude that $(P',F') \in {\cal J}^+$ with $P'=I^-(\eta)$ by clause (A2) in Definition \ref{ample}, and in particular $\eta$ must be future-complete by clause (ii) in Definition \ref{scri}.

We wish to show that $\eta$ is a $C$-ray. By construction, $\eta \subset \overline{I^+(C)}$, but since $\partial_{M} P \cap I^+(C) = \emptyset$, $\eta \subset \partial_{M} I^+(C)$. Moreover, since $C$ is achronal, $C \subset \partial_{M} I^+(C)$. Finally, since  $\partial_{M} I^+(C)$ is an achronal set, any causal curve segment connecting a point of $C$ with a point $x$ (say) along $\eta$ must be a reparametrization of a null geodesic, and in particular has zero Lorentzian arc-length. This means that the initial segment of $\eta$ between $C$ and $x$ is maximal. Hence, $\eta$ is a $C$-ray as claimed.

Now, assume that $C$ is contained in $\overline{B^+}$, but not in $B^+$. We can then pick a point $p \in H^+\cap C\cap J^-(V_{\infty})$. By the item BH2) of Proposition \ref{bhprops}, there exists a future-complete null geodesic ray, which we again denote by $\eta$, starting at $p$ and with future endpoint on ${\cal J}^+$. We only need to check it is again a $C$-ray. But if this were not the case, then there would exist some $q \in C$ and some point $r$ along $\eta$ with $q \ll_g r$. But since $r$ is visible from infinity (cf. Remark \ref{rmk5}), this would mean that $q \in I^-(V_{\infty}) \equiv M\setminus \overline{B^+}$ (cf. BH3) in Proposition \ref{bhprops}), a contradiction.
\end{proof}



%\begin{proof}
%First, assume that $C$ is not entirely contained in $\overline{B^+}$. Then, because of BH3) on Proposition \ref{bhprops}, ${\cal J}^+\cap I^+ (C)\neq \emptyset$. Let $x \in {\cal J}^+\cap I^+ (C)$, and pick a TIP $P_0$ such that $x \in P_0$ and $(P_0,F_0)\in {\cal J}^+$. Denote by ${\cal J}^+_0$ the connected component of ${\cal J}^+$ containing $(P_0,F_0)$.
%
%\cambios{Since ${\cal J}^+$ is ample, we have
%\[
%{\cal J}_0^+\cap(\overline{M}\setminus I^+(C))\supset {\cal J}_0^+\cap(\overline{M}\setminus\widetilde{I^+(C)}) \neq \emptyset.
%\]
%So, taking into account that ${\cal J}^+_0$ is connected, necessarily
%\[
%{\cal J}^+_0\cap \partial _{\overline{M}}I^+ (C)\neq\emptyset.
%\]
%Take some $(P,F)\in {\cal J}^+_0\cap \partial _{\overline{M}}I^+ (C)$. Since $I^+ (C)$ is open, it follows that $(P,F) \notin I^+ (C)$ and in particular $C \cap P = \emptyset$.}
%
%\cambios{Let us prove that $I^{-}(x_0)\subset P$ for some $x_0\in C$. Assume by contradiction that this does not hold. Then, from the definition of $\widetilde{I^+ (C)}$,
%\[
%(P,F)\in {\cal J}^+_0\cap (\overline{M} \setminus \widetilde{I^+ (C)}).
%\]
%But recall that ${\cal J}^+_0\cap (\overline{M} \setminus \widetilde{I^+ (C)})$ is assumed to be open. Moreover, it satisfies
%\[
%({\cal J}^+_0\cap (\overline{M} \setminus \widetilde{I^+ (C)}))\cap I^{+}(C)\subset ({\cal J}^+_0\cap (\overline{M} \setminus I^+ (C)))\cap I^{+}(C)=\emptyset.
%\]
%So, we have found an open set ${\cal J}^+_0\cap (\overline{M} \setminus \widetilde{I^+ (C)})$ that contains $(P,F)$ and does not intersects $I^{+}(C)$. This contradicts that $(P,F)\in\partial_{\overline{M}}I^+(C)$. In conclusion, some $x_{0}\in C\setminus P$ exists such that $I^-(x_0)\subset P$.}
%
%
%%\cambios{Now, since ${\cal J}^+$ is ample,  ${\cal J}^+_0\cap (\overline{M} \setminus \widetilde{I^+ (C)})\neq \emptyset$, and so ${\cal J}^+_0\cap \partial_{\overline{M}}I^+ (C)$\footnote{Along this proof, we will make use of a subindex $_{\overline{M}}$ to emphasize that such a topological object is considered on $\overline{M}$.} is non empty. Observe that, for any $(P,F) \in {\cal J}^+_0\cap \partial _{\overline{M}}I^+ (C)$, since $I^+ (C)$ is open, it follows that $(P,F) \notin I^+ (C)$ and in particular $C \cap P = \emptyset$.}
%%
%%\cambios{
%%Let us take a point $(P,F)\in {\cal J}^+_0\cap \partial _{\overline{M}}I^+ (C)$ and observe that, necessarily, $I^-(x_0)\subset P$. Otherwise,from definition of $\widetilde{I^+(C)}$ \eqref{eq:2},
%%\[
%%(P,F)\in {\cal J}^+_0\cap (\overline{M} \setminus \widetilde{I^+ (C)}).
%%\]
%%But recall that ${\cal J}^+_0\cap (\overline{M} \setminus \widetilde{I^+ (C)})$ is assumed to be open. Moreover, it satisfies
%%\[
%%({\cal J}^+_0\cap (\overline{M} \setminus \widetilde{I^+ (C)}))\cap I^{+}(C)\subset ({\cal J}^+_0\cap (\overline{M} \setminus I^+ (C)))\cap I^{+}(C)=\emptyset.
%%\]
%%So, we have found an open set ${\cal J}^+_0\cap (\overline{M} \setminus \widetilde{I^+ (C)})$ that contains $(P,F)$ and does not intersects $I^{+}(C)$, a contradiction with the assumption that $(P,F)\in\partial_{\overline{M}}I^+(C)$.}
%
%% As $\hat{M}$ is Hausdorff and $(P,F)$ has $P\neq \emptyset$, Prop. \ref{prop:existenceSequence} ensures the existence of a sequence $\{(P_k,F_k)\}_{k \in \mathbb{N}}\subset I^+(C)$ and such that $P\in L(\{P_k\}_k))$. %In particular,
%% %\[
%% %\overline{R} = \liminf R_k = \limsup R_k.
%% %\]
%% Pick $x_k \in P_k \cap C$ for each $k \in \mathbb{N}$. By compactness of $C$ we can assume the $x_k \rightarrow x_0$, up to passing to a subsequence, for some $x_0 \in C$ satisfying that $I^-(x_0)\in \liminf P_k$. Now observe that Zorn's Lemma allow us to ensure the existence of another IP $P'$ (maybe $P$) so $I^-(x_0)\subset P'$ and it is maximal on $\limsup P_k$, i.e., $P'\in \hat{L}(\{P_k\}_k)$. Hence both, $P$ and $P'$ are Hausdorff related and from the Hausdorff-compatibility of ${\cal J}^+$, it follows that $(P',F')\in {\cal J}^+$. Moreover, $(P',F')\in {\cal J}^+_0\cap \partial I^+ (C)$ (where this boundary is taken on $\overline{M}$), so $P'\cap C=\emptyset$ and hence $x_0\notin P'$.
%
%Therefore, we can take $\eta$ the future-inextendible null geodesic generator of $\partial_{\overline{M}} P$ starting at $x_0\in C\setminus P$. Note that since $I^-(\eta)$ is necessarily a TIP and $I^-(\eta) \subseteq P$,  we conclude that $(P',F') \in {\cal J}^+$ with $P'=I^-(\eta)$ by clause (A2) in Definition \ref{ample}, and in particular $\eta$ must be future-complete by clause (ii) in Definition \ref{scri}.
%
%By construction, $\eta \subset \overline{I^+(C)}_{\overline{M}}$, but since $\partial_{\overline{M}} P \cap I^+(C) = \emptyset$, $\eta \subset \partial_{\overline{M}} I^+(C)$. Moreover, since $C$ is achronal, $C \subset \partial_{\overline{M}} I^+(C)$; hence, $\eta$ is achronal, and thus, a $C$-ray.
%
%Now, assume that $C$ is contained in $\overline{B^+}$, but not in $B^+$, so we can pick a point $p \in H^+\cap C\cap J^-(V_{\infty})$. By BH2) on Proposition \ref{bhprops}, there exists a future-complete null geodesic ray $\eta$ starting at $p$ (and with future endpoint on ${\cal J}^+$). We only need to check it is indeed a $C$-ray. But if this were not the case, then there would exist some $q \in C$ and some point $r$ along $\eta$ with $q \ll_g r$. But since $r$ is visible from infinity (cf. Remark \ref{rmk5}), this would mean that $q \in I^-(V_{\infty}) \equiv M\setminus \overline{B^+}$ (cf. BH3) on Proposition \ref{bhprops}), a contradiction.
%\end{proof}


% \begin{remark}
%   The technical condition on $L(I^+(C))$ ensures that all the points on $\partial I^+(C)$ are $L$-limits of sequences contained in $I^+(C)$. It follows immediately if for instance the topology is AN1, which follows if the chronological topology is Hausdorff\footnote{Jony: Yo aqui haría una referencia al otro paper, donde si la topología es Hausdorff entonces se tiene que es metrizable...}
% \end{remark}

We can now use this theorem to prove a classic result in the theory of black holes \cite{BeemGlobalLorentzianGeometry1996,HawkingLargeScaleStructure1975,ONeillSemiRiemannianGeometryApplications1983}  for this extended context. Specifically, we wish to show that {\em any closed trapped surface stays inside the black hole region}, or, in other words, ``hidden from distant observers'' by the event horizon.

Recall that a {\em closed (future) trapped surface} in $(M,g)$ is a smooth, codimension 2, spacelike, achronal, compact submanifold $S \subset M$ without boundary whose mean curvature vector field is everywhere past-directed timelike \cite{ONeillSemiRiemannianGeometryApplications1983}. (The presence of this geometric object was introduced by Penrose \cite{PenroseGravitationalCollapse1965} as the mathematical surrogate for ``a point of no return'' in gravitational collapse.)  Also, recall that the {\em null convergence condition} holds in $(M,g)$ when $Ric(v,v) \geq 0$ for each null $v \in TM$. For solutions of the Einstein field equation of General Relativity (with or without a cosmological constant), this condition is implied by all the standard ``energy conditions'' on the stress-energy tensor, such as the dominant energy condition or the weak energy condition \cite{HawkingLargeScaleStructure1975,WaldGeneralRelativity1984}.

\begin{corollary}
\label{trappedcor}
Assume that ${\cal J}^+$ is regular and that the null convergence condition holds in $(M,g)$. If $S \subset M$ is a closed trapped surface, then $S \subset B^+$.
\end{corollary}
\begin{proof}
Suppose, to the contrary, that $S$ is not contained in $B^+$. Since $S$ is in particular achronal and compact by definition, Theorem \ref{main} implies the existence of some future-complete null $S$-ray $\eta: [0,+\infty) \rightarrow M$ with future endpoint $(P,F)\in {\cal J}^+$. But $\eta$ would be then a future-complete normal null geodesic ray starting at $S$ and without focal points, which contradicts standard results for closed trapped surfaces in spacetimes where the null converge condition holds (see, e.g. \cite{BeemGlobalLorentzianGeometry1996,HawkingLargeScaleStructure1975,ONeillSemiRiemannianGeometryApplications1983}).
\end{proof}

Finally, if ${\cal J}^+$ is regular, we can strengthen the item BH4) of Proposition \ref{bhprops} as follows.
\begin{corollary}
  Assume that ${\cal J}^+$ is regular. Then $p \in M \setminus B^+$ if and only if $p$ is visible from infinity.
\end{corollary}
\begin{proof}
Immediate from Theorem \ref{main} by taking $C= \{p\}$.
\end{proof}

%%% Local Variables:
%%% mode: latex
%%% TeX-master: "nullinfinityV5.tex"
%%% End:
