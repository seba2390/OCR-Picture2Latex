% belong to the arising field of argument mining, but existing work focuses on detecting argument components and their structure from the same documents
Our work is in line with argumentation mining, which has recently attracted significant research interest. Existing work focuses on argument extraction from news articles, legal documents, or online comments without given user-specified claim~\cite{moens2007automatic,palau2009argumentation,mochales2011argumentation,park2014identifying}.  
%\XY{The earliest work on argument scheme dates back to Ancient Greece, as in \cite{aristotle2006rhetoric}, where persuasion is categorized into \textit{logos, ethos}, and \textit{pathos}, based on different functions they perform in rhetoric structures.} 
Argument scheme classification is also widely studied~\cite{biran2011identifying,feng2011classifying,rooney2012applying,stab2014identifying,alkhatib-EtAl:2016:COLING}, which emphasizes on distinguishing different types of arguments. To the best of our knowledge, none of them studies the interaction between types of arguments and their usage to support a user-specified claim. This is the gap we aim to fill.

%[Other work lies in argument scheme classification]\XY{Later works include}~\cite{biran2011identifying,feng2011classifying,rooney2012applying,stab2014identifying}. \newcite{alkhatib-EtAl:2016:COLING} provides some insight on strategical perspectives by studying argumentative discourse units on news editorial with fine granularity. \XY{But there is no explicit study about combining arguments in a cross-document manner.}
%\cite{koreeda-EtAl:2016:ArgMining2016} proposed a neural attention model to determine polarity of sentence given two special kinds of claims. Their dataset is manually labeled on English Gigaword, and are limited to adjacent pairs due to the complexity for annotation. 
 

%Furthermore, online debates have been used for study on various levels. \newcite{aharoni2014benchmark} build a dataset based on topics and claims on idebate and evidence from Wikipedia. On top of that \newcite{rinott2015show} proposed an automatic evidence detection system conditioned on context information. \newcite{boltuzic2014back} identify comment-argument pairs and build several supervised model based on their annotated data. As is mentioned by \newcite{Fuhr:15y}, current information retrieval system suffers from bias due to their word-based nature, and lacks the ability to combine information from multiple documents for a single query. 


% relevant evidence detection under argument mining, previous work single type of supporting argument -- just evidence; we bring in opinions, reasoning, etc

%\cite{rinott2015show} have proposed an evidence detection framework based on the dataset provided by \cite{aharoni2014benchmark}. This dataset is manually labeled from 586 Wikipedia articles, covering 33 topics from Idebate.
%They hired 20 trained labelers for this task and let five labelers to independently crosscheck the joint list of debated candidates. Candidates are confirmed only when at least three labelers agree on that. 
%And they proposed a three-type scheme to classify all those evidence. Most of those evidence are objective due to the nature of Wikipedia articles. But in reality people extensively use opinionated information from the web to construct arguments. Therefore we try to construct a more inclusive scheme for argument types. We do not restrict our attention to Wikipedia articles, instead we use the original reference documents used by human debaters when they construct their arguments.

%\LW{add more description about this dataset, and why ours is different}
%Another dataset focus on editorial~\cite{alkhatib-EtAl:2016:COLING}

%\LW{add other work on idebate for argument mining}
%\cite{boltuzic2014back}