%%%%%%%%%%%%%%%%%%%%%%%%%%%%%%%%%%%%%%%%%%%%%%%%%%%%%%%%%%%%%%%%%%%%%%%%%%%%%%%%
%2345678901234567890123456789012345678901234567890123456789012345678901234567890
%        1         2         3         4         5         6         7         8

\documentclass[letterpaper, 10 pt, conference]{ieeeconf}  % Comment this line out if you need a4paper

%\documentclass[a4paper, 10pt, conference]{ieeeconf}      % Use this line for a4 paper
\usepackage{times}
\usepackage{epsfig}
\usepackage{graphicx}
\usepackage{amsmath}
\usepackage{amssymb}


\usepackage{tikz}
\usepackage{comment}
\usepackage[utf8]{inputenc} % allow utf-8 input
\usepackage[T1]{fontenc}    % use 8-bit T1 fonts
% \usepackage{hyperref}       % hyperlinks
\usepackage{url}            % simple URL typesetting
\usepackage{amsfonts}       % blackboard math symbols
\usepackage{nicefrac}       % compact symbols for 1/2, etc.
\usepackage{microtype}      % microtypography
% \usepackage[pdftex]{graphicx}
% \usepackage{floatrow}
\usepackage[bb=boondox]{mathalfa}
% \usepackage{wrapfig}
\usepackage{multirow}
%\newcommand{\theHalgorithm}{\arabic{algorithm}}
\usepackage{algorithmic}
\usepackage{algorithm}
% \usepackage{caption}
% \usepackage{xr}
% \externaldocument{appendix}
%\usepackage[figuresright]{rotating}
\usepackage{subfigure}


\usepackage{booktabs}       % professional-quality tables

\usepackage{float}
\usepackage{caption}
%\newcommand{\fullname}{\emph{<NAME>}}
%\newcommand{\name}{\emph{<NAME>}}
\newcommand{\name}{MANTM}
\newcommand{\planner}{{HTP}}
\usepackage[capitalize]{cleveref}
\crefname{section}{Sec.}{Secs.}
\Crefname{section}{Section}{Sections}
\Crefname{table}{Table}{Tables}
\crefname{table}{Tab.}{Tabs.}

\IEEEoverridecommandlockouts                              % This command is only needed if 
                                                          % you want to use the \thanks command

\overrideIEEEmargins                                      % Needed to meet printer requirements.

%In case you encounter the following error:
%Error 1010 The PDF file may be corrupt (unable to open PDF file) OR
%Error 1000 An error occurred while parsing a contents stream. Unable to analyze the PDF file.
%This is a known problem with pdfLaTeX conversion filter. The file cannot be opened with acrobat reader
%Please use one of the alternatives below to circumvent this error by uncommenting one or the other
%\pdfobjcompresslevel=0
%\pdfminorversion=4

% See the \addtolength command later in the file to balance the column lengths
% on the last page of the document

% The following packages can be found on http:\\www.ctan.org
%\usepackage{graphics} % for pdf, bitmapped graphics files
%\usepackage{epsfig} % for postscript graphics files
%\usepackage{mathptmx} % assumes new font selection scheme installed
%\usepackage{times} % assumes new font selection scheme installed
%\usepackage{amsmath} % assumes amsmath package installed
%\usepackage{amssymb}  % assumes amsmath package installed

\title{\LARGE \bf
Active Neural Topological Mapping for Multi-Agent Exploration
}


\author{Xinyi Yang$^{1*}$, Yuxiang Yang$^{1*}$, Chao Yu$^{1{\dag}}$, Jiayu Chen$^{1}$, Jingchen Yu$^{1}$, Haibing Ren$^{2}$, \\Huazhong Yang$^{1}$ and Yu Wang$^{1{\dag}}$  % <-this % stops a space
\thanks{*Equal contribution}% <-this % stops a space
\thanks{{\dag}Corresponding authors}
\thanks{$^{1}$Department of Electronic Engineering, Tsinghua University, Beijing, 100084, China.
        {\tt\small yang-xy20@mails.tsinghua.edu.cn; {yanghz, yu-wang}@tsinghua.edu.cn}}%
\thanks{$^{2}$Meituan, 7 Rongda Road, Chaoyang District, Beijing, 100012, China.
        }%
\thanks{Website: \protect\url{https://sites.google.com/view/mantm}}
\thanks{{This research was supported by National Natural Science Foundation of China (No.62203257, 62325405), Tsinghua University Initiative Scientific Research Program, Tsinghua-Meituan Joint Institute for Digital Life, Beijing National Research Center for Information Science, Technology (BNRist) and Beijing Innovation Center for Future Chips.}
}}




\begin{document}



\maketitle
\thispagestyle{empty}
\pagestyle{empty}


%%%%%%%%%%%%%%%%%%%%%%%%%%%%%%%%%%%%%%%%%%%%%%%%%%%%%%%%%%%%%%%%%%%%%%%%%%%%%%%%

\begin{abstract}
% explore是个问题,一个流行的解决范式是map-》explore,多采用传统方法,rl也有。但是他们大多基于metric map(not sure,传统方法是不是基于?if 不是,那就是rl比传统好,但rl基于。),而metric map有问题,topo map是个啥,他因为啥可以解决啥,所以我们propose啥。
This paper investigates the multi-agent cooperative exploration problem, which requires multiple agents to explore an unseen environment via sensory signals in a limited time. 
A popular approach to exploration tasks is to combine active mapping with planning. Metric maps capture the details of the spatial representation, but are with high communication traffic and may vary significantly between scenarios, resulting in inferior generalization. Topological maps are a promising alternative as they consist only of nodes and edges with abstract but essential information and are less influenced by the scene structures. However, most existing topology-based exploration tasks utilize classical methods for planning, which are time-consuming and sub-optimal due to their handcrafted design. Deep reinforcement learning (DRL) has shown great potential for learning (near) optimal policies through fast end-to-end inference. In this paper, we propose \underline{M}ulti-\underline{A}gent \underline{N}eural \underline{T}opological \underline{M}apping (\name) to improve exploration efficiency and generalization for multi-agent exploration tasks. {\name} mainly comprises a Topological Mapper and a novel RL-based \underline{H}ierarchical \underline{T}opological \underline{P}lanner (\planner). The Topological Mapper employs a visual encoder and distance-based heuristics to construct a graph containing main nodes and their corresponding ghost nodes. The {\planner} leverages graph neural networks to capture correlations between agents and graph nodes in a coarse-to-fine manner for effective global goal selection. Extensive experiments conducted in a physically-realistic simulator, Habitat, demonstrate that {\name} reduces the steps by at least 26.40\% over planning-based baselines and by at least 7.63\% over RL-based competitors in unseen scenarios.\looseness=-1

%However, combine RL with topological therefore, the existing RL-based solutions primarily rely on metric maps. 

%栅格地图不好,拓扑地图好,但是目前拓扑地图探索在用于传统方法,传统方法本身。最好的是与rl相结合,但是目前rl方法只和metric maps,所以我们这里。。。。Metric map xxx, topological map xxx,
% outperforms RL-based baselines with at least 6.96\% higher exploration efficiency. Compared with planning-based competitors, {\name} has at least 12.99\% higher exploration efficiency. 



\end{abstract}

%传统方法运用metric map
%对于不同结构的地图构建网格地图差异大,不具有泛化性,因为网格地图与空间的布局强相关。
% The {\planner} first selects a main node and then chooses the corresponding ghost node as the global goal for each agent.
\section{Introduction}

Many problems in econometrics, statistics, causal inference, and finance involve linear functionals of unknown functions:
\begin{equation}
\theta(g)=\E[m(Z; g)]
\end{equation}
where $Z$ denotes a random vector, and $g: \mcX\to \R$ is a function in some space $ \mcG$. A continuous linear functional that is mean square continuous with respect to $\ell_2$ norm can be written in a more benign and useful manner. Formally, for a given linear functional $\theta(\cdot)$, there exists a function $a_0$ such that for any $g\in \mcG$:\footnote{For simplicity of exposition, throughout the paper we consider scalar-valued functions $g$. All our results naturally extend to vector-valued functions $g$, and estimate a vector valued Riesz representer that satisfies that $\theta(g)=\E[a(X)'g(X)]$.}
\begin{equation}
    \theta(g) = \E[a_0(X)\, g(X)]
\end{equation}
This result is known as the Riesz representation theorem, and the function $a_0$ is the Riesz representer of the linear functional. Evaluation of a linear functional $\theta(g)$ can be achieved by simply taking the inner product between $a_0$ and $g$.

Knowing the Riesz representation of a linear functional is a critical building block in a variety of learning problems. For instance, in semi-parametric models, $g_0$ is an unknown regression function and $\theta(g_0)$ is a causal or structural parameter of interest. The Riesz representer $a_0$ of the functional $\theta(\cdot)$ can be used to debias the plug-in estimator and construct semi-parametrically efficient estimators of the parameter $\theta(g_0)$. In asset pricing applications, the Riesz representer corresponds to the stochastic discount factor, which is of primary interest when pricing financial derivatives.

Irrespective of the downstream application, the goal of this paper is to derive an estimator for the Riesz representer of any linear functional, when given access to $n$ samples of the random vector $Z$ and a target function space $\mcA$ that can well approximate the function $a_0$. We propose and analyze an estimator $\hat{a}$, with small mean-squared-error. Formally, with probability (w.p.) $1-\zeta$:
\begin{equation}
    \|\hat{a}-a_0\|_2 = \sqrt{\E\left[\left(\hat{a}(X) - a_0(X)\right)^2\right]} \leq \epsilon_{n,\zeta}
\end{equation}

We consider estimation of the Riesz representer within some function space $\mcA$ and propose an adversarial estimator based on regularized variants of the following min-max criterion:
\begin{equation}
    \hat{a} = \argmin_{a\in \mcA} \max_{f\in \mcF} \frac{1}{n}\sum_{i=1}^n \left(m(Z_i;f) - a(X_i)\cdot f(X_i) - f(X_i)^2\right)
\end{equation}
We derive oracle inequalities for this estimator as a function of the localized Rademacher complexity of the function space $\mcA$ and the approximation error $\epsilon = \min_{a\in \mcA} \|a-a_0\|_{2}$.

We show that as long as the function class $\mcF$ contains the star-hull of differences of functions in $\mcA$, i.e. $\mcF:= \{r(a-a'): a, a'\in \mcA, r\in [0, 1]\}$, then the estimation rate of the adversarial estimator achieves w.p. $1-\zeta$:
\begin{equation}
    \|\hat{a} - a_0\|_2 = O\left(\epsilon + \delta_n + \sqrt{\frac{\log(1/\zeta)}{n}}\right)
\end{equation}
where $\delta_n$ is the critical radius of the function classes $\mcF$ and $m\circ \mcF=\{Z\to m(Z; f): f\in \mcF\}$. The critical radius of a function class is a widely used quantity in statistical learning theory that allows one to argue fast estimation rates that are nearly optimal. For instance, for parametric function classes, the critical radius is of order $n^{-1/2}$, leading to fast parametric rates (as compared to $n^{-1/4}$ which would be achievable via looser uniform deviation bounds).

Moreover, the critical radius has been analyzed and derived for a variety of function spaces of interest, such as neural networks, high-dimensional linear functions, reproducing kernel Hilbert spaces, and VC-subgraph classes. Thus our general theorem allows us to appeal to these characterizations and provide oracle rates for a family of Riesz representer estimators. Prior work on estimating Riesz representers only considered particular high-dimensional parametric classes and derived specialized estimators for the function space of interest. Our adversarial estimator provides a single approach that tackles generic function spaces in a uniform manner.

We also examine the computational aspect of our estimator. We provide examples of how estimation can be achieved in a computationally efficient manner for several function spaces of interest.

Finally, we show how our estimator can be used in the context of estimating causal or structural parameters in semi-parametric models. Specifically, our mean square rate for the Riesz representer is sufficiently fast to achieve semi-parametric efficiency and asymptotic normality of the causal or structural parameter.

\subsection{Applications: Causal Inference and Asset Pricing}\label{sec:intro_examples}

This learning problem arises in two important domains for economic research: causal inference and asset pricing.

\paragraph{Automated De-biasing of Causal Estimates.} In causal inference, a variety of treatment effects and policy effects can be formulated as functionals--i.e., scalar summaries--of an underlying regression \cite{chernozhukov2016locally}. Formally, the causal parameter $\theta_0=\theta(g_0)=\mathbb{E}[m(Z;g_0)]$ is a functional $\theta(\cdot)$ of the nuisance parameter $g_0(x):=\mathbb{E}[Y|X=x]$. In this paper, we consider a variety of treatment and policy effects including
\begin{enumerate}
    \item Average treatment effect (ATE): $\theta_0=\mathbb{E}[g_0(1,W)-g_0(0,W)]$, where $X=(D,W)$ consists of treatment and covariates.
    \item Average policy effect: $\theta_0=\int g_0(x)d\mu(x)$ where $\mu(x)=F_1(x)-F_0(x)$
    \item Policy effect from transporting covariates: $\theta_0=\mathbb{E}[g_0(t(X))-g_0(X)]$
    \item Cross effect: $\theta_0=\mathbb{E}[Dg_0(0,W)]$, where $X=(D,W)$ consists of treatment and covariates.
    \item Regression decomposition: $\mathbb{E}[Y|D=1]-\mathbb{E}[Y|D=0]=\theta_0^{response}+\theta_0^{composition}$
    where
    \begin{align}
        \theta_0^{response}&=\mathbb{E}[g_0(1,W)|D=1]-\mathbb{E}[g_0(0,W)|D=1] \\
        \theta_0^{composition}&=\mathbb{E}[g_0(0,W)|D=1]-\mathbb{E}[g_0(0,W)|D=0]
    \end{align}
    \item Average treatment on the treated (ATT): $\theta_0=\mathbb{E}[g_0(1,W)|D=1]-\mathbb{E}[g_0(0,W)|D=1]$, where $X=(D,W)$ consists of treatment and covariates.
    \item Local average treatment effect (LATE): $\theta_0=\frac{\mathbb{E}[g_0(1,W)-g_0(0,W)]}{\mathbb{E}[h_0(1,W)-h_0(0,W)]}$, where $X=(V,W)$ consists of instrument and covariates and $h_0(x):=\mathbb{E}[D|X=x]$ is a second regression.
\end{enumerate}
More generally, our results extend to parameters defined implicitly by $0=\mathbb{E}[m(Z;g_0;\theta_0)]$, such as partially linear regression and partially linear instrumental variable regression.

    If the regression $g_0$ is learned by a regularized estimator $\hat{g}$, then estimation of the causal parameter $\theta_0$  by a plug-in estimator $\mathbb{E}_n[m(Z;\hat{g})]$ is badly biased. The solution is to use a de-biased formulation of the causal parameter instead: $\theta_0=\mathbb{E}[m(Z;g_0)+a_0(X)\{Y-g_0(X)\}]$. Observe that $a_0$ arises in the bias correction term. We re-visit this example in Section~\ref{sec:debiasing}.

%

\paragraph{Fundamental Asset Pricing Equation.} In asset pricing, a variety of financial models deliver the same fundamental asset pricing equation. This equation is of both theoretical and practical interest. Theoretically, it elucidates why asset prices or returns are what they are. Practically, it can be used to identify trading opportunities when assets are mis-priced. The asset pricing equation follows from two weak assumptions: free portfolio formation, and the law of one price.  In Appendix~\ref{sec:finance}, we review the derivation for a general audience.\footnote{The same asset pricing equation can be derived from either a model of complete markets for contingent claims, or a model of investor utility maximization. Free portfolio formation is a weaker assumption on market structure than the existence of complete markets for contingent claims. The law of one price is a weaker assumption on preference structure than investor utility maximization. We present these additional derivations in Appendix~\ref{sec:finance}.}

Formally, the fundamental asset pricing equation is $p_{t,i}=\mathbb{E}_t[m_{t+1}x_{t+1,i}]$ where $p_{t,i}$ is the price of asset $i$ at time $t$, $x_{t+1,i}$ is payoff of asset $i$ at time $t+1$, and $m_{t+1}$ is the market-wide stochastic discount factor (SDF) at time $t+1$.\footnote{The SDF has many additional names: marginal rate of substitution, state price density, and pricing kernel. Each name corresponds to a different derivation of the asset pricing equation, starting from different first principles.} The expectation is conditional on information $(I_t,I_{t,i})$ known at time $t$:  $I_t$ are macroeconomic conditioning variables that are not asset specific, e.g. inflation rates and market return; $I_{t,i}$ are asset-specific characteristics, e.g. the size or book-to-market ratio of firm $i$ at time $t$. The asset pricing equation encompasses stocks, bonds, and options. We clarify its many instantiations below, where $d_{t+1}$ is dividend, $C$ is the call price, $S_T$ is the stock price at expiration, $K$ is the strike price. 

\begin{table}[H]
       \centering
       \begin{tabular}{|c||c|c|}
        \hline 
            Asset & Price $p_t$ & Payoff $x_{t+1}$ \\
             \hline 
            \hline
            Stock &$p_t$& $p_{t+1}+d_{t+1}$ \\
              Bond &$p_t$&$1$\\
             Option &$C$&$\max\{S_T-K,0\}$ \\
             \hline 
            Return & $1$& $R_{t+1}$ \\
            Excess return &0&$R^e_{t+1}$ \\
            \hline 
       \end{tabular}
       \caption{Generality of asset pricing equation}
       \label{tab:my_label}
   \end{table}
 
 The fundamental asset pricing equation can also be parametrized in terms of returns. If an investor pays one dollar for an asset $i$ today, the gross rate of return $R_{t+1,i}$ is how many dollars the investor receives tomorrow; formally, the price is $p_{t,i}=1$ and the payoff is $x_{t+1,i}=R_{t+1,i}$ by definition. Next consider what happens when an investor borrows a dollar today at the interest rate $R_{t+1}^f$ and buys an asset $i$ that gives the gross rate of return $R_{t+1,i}$ tomorrow. From the perspective of the investor, who paid nothing out-of-pocket, the price is $p_{t,i}=0$ while the payoff is the excess rate of return $R_{t+1,i}^e:=R_{t+1,i}-R_{t+1}^f$, leading to the asset pricing equation: $0=\mathbb{E}_t[m_{t+1}R^e_{t+1,i}]$.
 
 
 Following \cite{chen2019deep}, we focus on the latter excess return parametrization of the asset pricing equation. Taking expectations yields the unconditional moment restriction
$$
0=\mathbb{E}[m_{t+1}R^e_{t+1,i}z(I_t,I_{t,i})]=\mathbb{E}[\mathbb{E}[m_{t+1}|R^e_{t+1,i},I_t,I_{t,i}]R^e_{t+1,i}z(I_t,I_{t,i})],\quad \forall z(\cdot)
$$
Our framework nests this final expression. Specifically,
$$
\theta(g)=0,\quad g(R^e_{t+1,i},I_t,I_{t,i})=R^e_{t+1,i}z(I_t,I_{t,i}),\quad a_0(R^e_{t+1,i},I_t,I_{t,i})=\mathbb{E}[m_{t+1}|R^e_{t+1,i},I_t,I_{t,i}]
$$
By estimating $a_0$, which is the projection of the SDF onto excess returns and other available information, one can pin down the price of any hypothetical asset. 

%
%
%
%

\subsection{Related Work}

\textbf{Classical Semi-parametric Statistics.} Classical semi-parametric statistical theory studies the asymptotic properties of statistical quantities that are functionals of a density or a regression over a low-dimensional domain \cite{levit1976efficiency,hasminskii1979nonparametric,ibragimov1981statistical,pfanzagl1982lecture,klaassen1987consistent,robinson1988root,van1991differentiable,bickel1993efficient,newey1994asymptotic,robins1995semiparametric,vaart,bickel1988estimating,newey1998undersmoothing,ai2003efficient,newey2004twicing,ai2007estimation,tsiatis2007semiparametric,kosorok2007introduction,ai2012semiparametric}. Any continuous linear functional has a Riesz representer. In this classical theory, the Riesz representer appears in the influence function and therefore in the asymptotic variance of semi-parametric estimators \cite{newey1994asymptotic}. We depart from classical theory by considering the high-dimensional setting.

\textbf{De-biased Machine Learning and Targeted Maximum Likelihood.} Because the Riesz representer appears in the asymptotic variance of semi-parametric estimators, it can be incorporated into estimation to ensure semi-parametric efficiency. In practice, this can be achieved by introducing a de-biasing term into the estimating equation \cite{hasminskii1979nonparametric,bickel1988estimating,zhang2014confidence,belloni2011inference,belloni2014inference,belloni2014uniform,belloni2014pivotal,javanmard2014confidence,javanmard2014hypothesis,javanmard2018debiasing,van2014asymptotically,ning2017general,chernozhukov2015valid,neykov2018unified,ren2015asymptotic,jankova2015confidence,jankova2016confidence,jankova2018semiparametric,bradic2017uniform,zhu2017breaking,zhu2018linear}. In doubly robust estimating equations for regression functionals, the de-biasing term is the product between the Riesz representer and the regression residual \cite{robins1995analysis,robins1995semiparametric,van2006targeted,van2011targeted,luedtke2016statistical,toth2016tmle}. The more general principle at play is Neyman orthogonality: the learning problem for the functional of interest becomes orthogonal to the learning problems for both the regression and the Riesz representer \cite{neyman1959,neyman1979c,vaart,robins2008higher,zheng2010asymptotic,belloni2014uniform,belloni2014pivotal,chernozhukov2016locally,belloni2017program,chernozhukov2018double,foster2019orthogonal}.

De-biased machine learning and targeted maximum likelihood combine the algorithmic insight of doubly-robust moment functions with the algorithmic insight of sample splitting \cite{bickel1982adaptive,schick1986asymptotically,klaassen1987consistent,vaart,robins2008higher}.  In doing so, these frameworks facilitate a general analysis of residuals such that the target functional is $\sqrt{n}$-consistent under minimal assumptions on the estimators used for the regression and Riesz representer \cite{scharfstein1999adjusting,rubin2005general,rubin2006extending,van2006targeted,zheng2010asymptotic,van2011targeted,diaz2013targeted,van2014targeted,kennedy2017nonparametric,kennedy2020optimal}. In particular, any machine learning estimators are permitted that satisfy $\sqrt{n}\|\hat{g}-g_0\|_2\cdot\|\hat{a}-a_0\|_2\rightarrow 0$ \cite{chernozhukov2018double,chernozhukov2016locally}.

The Riesz representer may be a difficult object to estimate. Even for simple regression functionals such as policy effects, its closed form involves ratios of densities. In restricted models, where the regression is known to belong to a certain function class, there is the further difficulty of projecting the Riesz representer accordingly. A recent literature explores the possibility of directly estimating the Riesz representer, without estimating its components or even knowing its functional form \cite{robins2007comment,newey2018cross,athey2018approximate,chernozhukov2018global,chernozhukov2018learning,hirshberg2018debiased,hirshberg2019augmented,singh2019biased,rothenhausler2019incremental}. A crucial insight, on which we build, is that the Riesz representer is directly identified from data. 

\cite{hirshberg2019augmented} observe that to debias an average moment, it is sufficient to estimate an empirical analogue of the Riesz representer that approximately satisfies the Riesz representer moment equation on the $n$ samples. They propose a parametric min-max criterion to estimate $n$ parameters corresponding to the $n$ evaluations of the empirical Riesz representer. Unlike \cite{hirshberg2019augmented}, we provide a guarantee on learning the true Riesz representer, we approximate the Riesz representer within non-parametric function spaces, and our result therefore has broader application beyond causal inference. Importantly, \cite{hirshberg2019augmented} require that the same sample used to estimate the $n$ parameters is used in final stage estimation of the causal parameter. As such, the analysis requires that the regression function $g$ lies in a Donsker class--a restriction that precludes many machine learning estimators. By contrast, our adversarial estimator provides fast estimation rates with respect to the true Reisz representer and hence can be used in combination with cross-fitting and sample splitting to eliminate the Donsker assumption.


\textbf{Adversarial Estimation.} Riesz representation theorem can be viewed as a continuum of unconditional moment restrictions. The non-parametric instrumental variable problem, based on a conditional moment restriction, also implies a continuum of unconditional moment restrictions \cite{newey2003instrumental,hall2005nonparametric,blundell2007semi,chen2009efficient,darolles2011nonparametric,chen2012estimation,chen2015sieve,chen2018optimal}. A central insight of this work is that the min-max approach for conditional moment models may be adapted to the problem of learning the Riesz representer. In a min-max approach, the continuum of unconditional moment restrictions is enforced adversarially over a set of test functions \cite{goodfellow2014generative,arjovsky2017wasserstein,dikkala2020minimax}. 

The fundamental advantage of the min-max approach is its unified analysis over arbitrary function classes. In particular, via local Rademacher analysis, one can derive an abstract bound that encompasses sparse linear models, neural networks, and RKHS methods \cite{koltchinskii2000rademacher,bartlett2005local}. As such, the min-max approach is actually a family of algorithms adaptive to a variety of data settings with a unified guarantee \cite{negahban2012,lecue2017regularization,Lecue2018}. 

\textbf{Machine Learning Approaches to Causal Inference and Asset Pricing.} By pursuing a min-max approach, our work relates to previous work that incorporates a variety of machine learning methods into causal inference. Much work on de-biased machine learning focuses on sparse and approximately sparse models \cite{chernozhukov2018global,chernozhukov2018learning,chernozhukov2018plug}. A neural network estimator with mean square rate has been successfully used to learn the nuisance regression in semiparametric estimation \cite{chen1999improved,farrell2018deep} and to learn the structural function in nonparametric instrumental variable regression \cite{deepiv,bennett2019deep,dikkala2020minimax}. A more recent literature incorporates RKHS methods into causal inference due to their convenient closed form solutions and strong performance on smooth designs \cite{nie2017quasi,singh2019kernel,muandet2019dual,singh2020kernel,muandet2020kernel}.

Finally, our works provides a theoretical foundation for a growing literature that incorporates machine learning into asset pricing. We follow the asset pricing literature in framing the problem of learning a stochastic discount factor as the problem of learning a Riesz representer \cite{hansen1997assessing}. Specifically, we propose a deep min-max approach based on free portfolio formation and the law of one price \cite{bansal1993no,chen2019deep}. This approach differs from deep learning approaches that predict asset prices via nonparametric regression \cite{messmer2017deep,feng2018deep,gu2020autoencoder,bianchi2020bond}. Unlike previous work, we prove mean square rates for the stochastic discount factor, and we prove $\sqrt{n}$-consistency and semiparametric efficiency for expected asset prices.
\section{Related Work}\label{related_work}
% \subsubsection{Multi-Turn Dialogue Understanding}
\textbf{Multi-Turn Dialogue Understanding}. Various tasks and corresponding benchmarks are proposed to evaluate the capacities of dialogue understanding models. Dialogue-based relation extraction (RE) is a classification task that assigns a pair of entities a relation label in a dialogue. Focusing on the word level,  \cite{xue2021gdpnet} constructed a multi-view graph with words in the dialogue as nodes and proposed Dynamic Time Warping Pooling to automatically select words in interest. SimpleRE \citep{SimpleRE} designed a novel input sequence format and utilized a Relation Refinement Gate to filter the semantic representation which is later fed into the classifier. TUCORE-GCN \citep{zahiri:18a} used a heterogeneous dialogue graph to encode the interaction between speakers, arguments, and turns across the dialogues.
% \citet{christopoulou-etal-2019-connecting} constructed an edge-oriented graph model to encode the dialogue as a graph with nodes and edges of different functions and applied an inference mechanism on the graph edges to recognize the internal relationship. By constructing a mention-level graph and an entity-level graph, \citet{zahiri:18a} reasoned the relation between entities by path inference.

Emotion Recognition in Conversation (ERC) has been extensively studied in the research community. It aims to attach an emotional label to every turn in a given dialogue. \cite{kratzwald2018deep} customized the recurrent neural network with bidirectional processing to solve the problem of emotion classification. \cite{majumder2019dialoguernn} leveraged the Recurrent Neural Network to extract the information of the party states and use it to predict the emotion in conversations with two speakers. On top of the recurrent neural network, COSMIC \cite{ghosal2020cosmic} models the commonsense knowledge, mental states, events, and actions to enhance emotion detection in dialogue. 
%Towards solving the context propagation problems in the recurrent neural network-based methods, \citet{ghosal-etal-2019-dialoguegcn} proposed a graph-based method that models the utterances as nodes and the speakers' dependency as edges. 

Deep learning-based methods have been extensively studied in recent works \citep{lee-dernoncourt-2016-sequential, chen2018dialogue, raheja2019dialogue} regarding Dialogue Act classification (DAC). \cite{chen2018dialogue} introduced a relation layer into the shared hierarchical encoder to model the interaction between the tasks of dialog act recognition and sentiment classification.
%Combining recurrent neural networks and convolutional neural networks, \citet{lee-dernoncourt-2016-sequential} incorporated the preceding texts while classifying the act.


% \subsubsection{Context-Aware Representation Learning}
\textbf{Context-Aware Representation Learning}. To address dynamics and semantic changes in multi-turn dialogue, previous works extend pre-trained large language models to learn context-aware representations for turns \citep{lee2021graph, DialogXL, DCM, chapuis2020hierarchical}. TUCORE-GCN \citep{lee2021graph} proposes the turn attention module, masking out distant turns to learn the contextual embeddings. Instead of adding extra modules, DialogXL \citep{DialogXL} targets the encoder and incorporates four self-attention mechanisms to different attention heads to capture diverse dialog-aware information. Similarly, such dialogue-oriented self-attention can also be found in MDFN \citep{MDFN} where it is defined as utterance-aware and speaker-aware channels. However, most of them involve an additional pre-training stage \citep{DialogXL, DCM, chapuis2020hierarchical}. 
%These efforts focus on the turn-level modeling but ignored the gap between the pre-training objective and dialogue understanding tasks. Also, though achieving preferable performance on limited datasets or tasks, these methods have not led to a unified solution to multi-turn dialogue understanding. 
\section{Task Setup}

%讲讲habita的设定
%我们multi-agent的设定
Multi-agent cooperative exploration requires agents to explore an unknown scene based on sensory signals.
At each time step, each agent receives a first-person RGB-D image and the estimated pose from
sensors. Agents then perform environmental actions in the physically realistic simulator, Habitat~\cite{habitat}. The horizon of the global decision-making step is 15 steps, and the available environmental actions include \emph{Turn Left}, \emph{Turn Right}, and \emph{Forward}. Following the settings in ANS~\cite{ans} and NRNS~\cite{norl}, we introduce Gaussian noise in the sensor readings and simulate real-world action noise. In the multi-agent scenario, we further consider the following settings. Firstly, we assume perfect communication, where relative spawn locations are shared between agents. This allows us to estimate the relative position of each agent at each timestep by using sensory pose readings and shared spawn locations. Besides, agents are randomly initialized within a 2-meter geodesic distance constraint. This spatially close initialization requires agents to expend more scanning effort for exploration, further increasing the difficulty of cooperation.\looseness=-1



%resulting in the bias between the estimated pose and the actual pose leading to the bias between the expected pose and the actual pose

%where the \emph{Turn} operation rotates the agent 10 degrees, and the \emph{Forward} operation enables the agent to go straight ahead for 0.25$m$.
% so that the behaviors of all agents tend to be the same with similar observations at the beginning. This constraint further increases the difficulty of cooperation.
% , where the \emph{Turn} operation rotates the agent 10 degrees, and the \emph{Forward} operation enables the agent to go straight ahead for 0.25$m$.  We consider a decision-making setting by assuming perfect communication between agents. The objective of the task is to maximize the accumulated explored area within a limited time horizon.






\begin{figure}[t]
    \centering
    \includegraphics[width=\textwidth]{flow.pdf}
    \caption{Overview of Event Fusion Photometric Stereo Network~(EFPS-Net). EFPS-Net consists of three parts: (i)~Event Interpolation Network~(EI-Net) which interpolates the sparse event observation maps~$O_{e}$,~(ii) Observation Fusion Module~(OFM) that fuses five observation maps, and~(iii) Surface Normal Estimation Network~(SNE-Net) which outputs surface normal vector given fused observation maps.}
    \label{fig:modelarchitecture}
\end{figure}
In this section, we describe our proposed pipeline, i.e., Event Fusion Photometric Stereo Network~(EFPS-Net) shown in Fig.~\ref{fig:modelarchitecture}, which utilizes fused information from RGB frames and event signals.

\subsection{Obseravation map}
An observation map is a feature map that consists of light intensities from each direction.
More specifically, To compute the observation map~$O_{c}$ of ${m\times m}$ size for channel~$c$, we project the $j$~th intensity~$i_{j_{c}}$ of light direction~$(l^{x}_{j},~l^{y}_{j},~l^{z}_{j})$ from the shaped unit hemisphere onto the 2D plane, as defined in Eq.~\ref{input:observationmap}.

\begin{equation}
O_{c}\left( \left \lfloor m\frac{l^{x}_{j}+1}{2}\right \rfloor,\left \lfloor m\frac{l^{y}_{j}+1}{2}\right \rfloor \right) = i_{j_{c}}
\label{input:observationmap}
\end{equation}
Since, at a time, this approach only deals with the necessary part of pixels in either RGB frames or event signals, it is computationally efficient. To fuse two different modalities effectively with observation maps, we converted RGB frames into observation maps~$O_{r,g,b,n}$ and transformed raw event signals into sparse event observation maps.  
In particular, we utilize the method from PX-Net~\citep{logothetis2021px}, which generates RGB observation maps~$O_{r,g,b}$ from RGB frames and $O_{n}$ which is a normalized map of RGB observation maps as defined in Eq.~\ref{input:normalize} to obtain $O_{r,g,b,n}$.


\begin{equation}
O_{n} = \frac{O_{r} + O_{g} + O_{b}}{\text{max}(O_{r} + O_{g} + O_{b})}
\label{input:normalize}
\end{equation}

\subsection{Event Interpolation Network}
To extract useful information from the event signals for EFPS-Net, we propose a novel event signal representation that can also be transformed into an observation map.
The proposed method is fundamentally different from previous representation methods, which simply accumulate the polarity along the time at each pixel coordinate.
We divided the polarities by $\lambda$ before accumulating generated event signals to ensure a wide distribution and applied a non-linear function such as a hyperbolic tangent.
Subsequently, the processed event signals were located in the voxel grid.
Moreover, we separated polarity into two channels.
One channel represents a positive polarity, which indicates that the light intensity is increased. 
The other channel represents negative polarity, which indicates decreased light intensity.
Consequently, we propose a new separate event voxel grid~$V_{s}$ containing the dimensions of polarity, as defined in Eq.~{\ref{input:separate_voxel_grid}}.
We generated $O_{e}$ with each polarity channel of $V_{s}$ after merging all the time slots.

We note that the event observation maps are sparse because event signals do not compulsorily occur even if the light direction is changed.
Sparse event observation maps are insufficient for estimating the surface normal.
Therefore, we used an EI-Net to interpolate the event observation maps into a dense normalized observation map that contains sufficient information.

$$
V\rightarrow \begin{cases}
V(x,y,\delta,0) = \sum^{k(\delta+1)}_{t_{n}=k\delta}{\frac{1}{\lambda}}, & \text{ if } p(x,y,t_{n})=1. \\
V(x,y,\delta,1) = \sum^{k(\delta+1)}_{t_{n}=k\delta}{\frac{1}{\lambda}}, & \text{ if } p(x,y,t_{n})=-1.
\end{cases}
$$
where $B$ is the channel of voxel grids generated during $\Delta T$, which is one period. One channel of voxel grid~$k$ indicates $\frac{\Delta T}{B}$.


\begin{equation}
V_{s} = \frac{\text{exp}^{V}-\text{exp}^{-V}}{\text{exp}^{V}+\text{exp}^{-V}}
\label{input:separate_voxel_grid}
\end{equation}

\begin{figure}[t]
    \centering
    \subfloat[]{
    \includegraphics[width=0.475\textwidth]{blocks_a.pdf}
    }
    \subfloat[]{
    \includegraphics[width=0.475\textwidth]{blocks_b.pdf}
    }
    \caption{Details of Event Interpolation Network~(EI-Net) blocks and Surface Normal Estimation Network~(SNE-Net) blocks. First, EI-Net blocks, depicted in (a), have two blocks which are Down Block encoding features and Up Block up-sampling observation map. Next, SNE-Net blocks consist of DB~(Dense Block) and TB~(Transition Block) as shown in (b). DB is same as \citep{logothetis2021px}, but TB contains an additional batch normalization layer.}
    \label{fig:blocks}
\end{figure}

EI-Net consists of a head part, two Down blocks, sixteen Residual blocks, two Up blocks, and follow by a pred part.
The head part is composed of a head convolution layer, batch normalization layer, and ReLU activation function.
This part amplifies the sparse event observation map to multiple channels, and the Down Block encodes them while reducing the size by half.
As shown in Fig.~\ref{fig:blocks}~(a), the Down Block comprises $5\times5$ kernel convolution layer, a batch normalization layer, and a leaky ReLU activation function.
After obtaining encoded feature maps from the last Down Block, sixteen residual blocks deal with the output feature maps to interpolate the sparse intensity features.
As expressed in Eq.~\ref{input:residual_block}, the residual block consists of a $3\times3$ kernel convolution layer, a batch normalization layer, and a ReLU activation function.
It derives more fluent-intensity output feature maps~$M_{out}$ from the sparse input feature maps~$M_{in}$.
To decode resized inputs, we used Up Blocks, shown in Fig.~\ref{fig:blocks}~(a), which upscale the last layer output of the residual block's $M_{out}$ to match the EI-Net's input resolution of EI-Net.
Finally, we transformed them into an interpolated event observation map using the pred part, which consists of a pred convolution layer and adjusts values between 0 and 1 using a sigmoid activation function to match the range of input values.

\begin{equation}
M_\text{out} = \text{ReLU}(\text{BN}(f^{3 \times 3}(M_\text{in}))) + M_\text{in}
\label{input:residual_block}
\end{equation}

\subsection{Observation Fusion Module}
In our model architecture, the SNE-Net utilizes fused observation maps~$\hat{O}$ as the input.
To acquire fused observation maps, we gathered five new observation maps~$\tilde{O}$ with point-wise convolution.
This operation compensates for each observation map by using other observation maps.
Subsequently, we performed element-wise multiplication for each channel on the OFM input~$O$ after setting it in the 0 to 1 range with the sigmoid activation function.
It injects interacted observation map features at the input observation maps that were concatenated with $O_{r,g,b,n}$ and $O_{\hat{e}}$.
Consequently, we acquire $\hat{O}$, complemented by each channel of the observation map, as input to SNE-Net.
Overall, the OFM implements the process illustrated in Fig.~\ref{fig:ofm}.

\begin{figure}[t]
    \centering
    \includegraphics[width=0.5\textwidth]{ofm.pdf}
    \caption{Overview of Observation Fusion Module~(OFM), which produces five refined observation maps using convolution layers.}
    \label{fig:ofm}
\end{figure}

\subsection{Surface Normal Estimation Network}
%We aim to confirm how beneficial event signals information is to existing photometric stereo methods.
SNE-Net is inspired by PX-Net~\citep{logothetis2021px}, which uses an observation map. SNE-Net uses a large batch size because the observation map size is small. When the model has a large batch size, appropriate batch normalization layers are especially effective. Thus, we attached the batch normalization layer~(BN) to our network between the convolution layer and the ReLU activation function in the Transition Block~(TB), as shown in Fig.~\ref{fig:blocks}~(b). 
Finally, SNE-Net estimates the surface normal from fused observation maps generated from the OFM.

\subsection{Loss function}
EFPS-Net estimates the surface normal per pixel on an object using numerous light directions and intensities represented by image frames and event signals according to each light direction. We used the scale-invariant loss~$\mathcal{L}_{e}$ Eq.~(\ref{input:scale-invariant})~\citep{eigen2014depth} to interpolate the sparse event observation maps~$O_{e}$ to the dense normalized observation map~$O_{n}$. Since event signals represent relative light information based on the absolute intensity of RGB frames, we interpolated them to $O_{n}$.

\begin{equation}
\mathcal{L}_{e} = \frac{1}{n}\sum_{i}R^{2}_{i}-\frac{1}{n^{2}}(\sum_{i}R_{i})^{2},
\label{input:scale-invariant}
\end{equation}
where $R$ is computed using $R_{i} = O_{\hat{e}(i)} - O_{n(i)}$.\\

Moreover, we applied the Mean Angular Error~(MAE) loss function defined in Eq.~\ref{input:mae}. This loss function optimizes the error between the ground truth surface normal~$\textbf{n}$ and the predicted surface normal~$\hat{\textbf{n}}$.

\begin{equation}
\mathcal{L}_{n} = \cos^{-1}(\textbf{n}\cdot \hat{\textbf{n}})
\label{input:mae}
\end{equation}

In summary, EFPS-Net is optimized by combining loss~$\mathcal{L}_{e}$ and loss~$\mathcal{L}_{n}$, defined as:
\begin{equation}
\mathcal{L}_{total} = \mathcal{L}_{e} + \mathcal{L}_{n}
\label{input:total}
\end{equation}
\section{Experiments}

\subsection{Experimental Details}
We conduct all experiments in the Habitat simulator~\cite{habitat}, using the Gibson Challenge dataset~\cite{dataset1} and the Habitat-Matterport 3D dataset (HM3D)~\cite{dataset2}. The scenes in these datasets are collected from real building-scale residential, commercial, and civic spaces using 3D scanning and reconstruction. We filter out some scenes that are inappropriate for our task, following \cite{RL_multi2}, which is one of the best RL-based approaches for cooperative exploration. This filtering process involved removing scenes with large disconnected regions or multiple floors where agents couldn't attain 90\% coverage of the entire house. 
Furthermore, we exclude scenes smaller than 70 $m^2$, as their topological maps would contain too few nodes to fully show the advantages of graphs.
To better demonstrate the robustness of {\name} on training scenes and its effective generalization to novel scenes, we follow \cite{RL_multi2,cvpr22} by dividing the remaining scenes into $10$ training scenes from the Gibson Challenge dataset and $28$ testing scenes from both datasets.
We perform RL training with $10^6$ timesteps over 3 random seeds. Each evaluation score has the format of ``mean (standard deviation)'', and is averaged over 300 testing episodes.\looseness=-1


% Due to space limitations, we only present the results with $N=2,3$ agents, while experimental results with other numbers of agents are provided in the appendix.

% such as scenes that have large disconnected regions or multiple floors so that the agents are not possible to achieve 90\% coverage of the entire house.

% We choose scenes in the size of over 70 $m^2$ with continuous regions and a single floor to better show the performance of multi-agent exploration. Besides, we only use 10 scenes for training and 25 scenes for evaluation in order to demonstrate the generalization ability.to illustrate the robustness and the generalization of {\name}, Then we categorize the remaining scenes into $23$ training scenes and $10$ testing scenes. where the agents are not possible to achieve 90\% coverage of the entire house. 

% \input{tables/communication}
% \usepackage{multirow}
% \usepackage{booktabs}

\begin{table*}
\vspace{2mm}
\centering
\footnotesize
\scalebox{1.}{
\footnotesize
\setlength\tabcolsep{1.5mm}{\begin{tabular}{ccrccc|ccc|c} 
\toprule
Agents     &Sce.                                                                                    & Metrics                    &CoScan  &NF& Voronoi & ANS-Merge &NCM& MAANS & {\name}  \\ 

\midrule
\multirow{6}{*}{N =3} & 
\multirow{3}{*}{\begin{tabular}[c]{@{}c@{}}Middle\\(\textgreater{}$70m^2$)\end{tabular}} & \textit{Mut. Over.} $\downarrow$   &  0.39\scriptsize{(0.01)}   &  0.56\scriptsize{(0.01)}  & \textbf{0.29\scriptsize{(0.01)}}       &   0.44\scriptsize{(0.01)} & 0.38\scriptsize{(0.01)} & 0.39\scriptsize{(0.01)}  & 0.36\scriptsize{(0.02) } \\ 
\cmidrule{3-10}
                       & & \textit{Steps} $\downarrow$    &     244.86\scriptsize{(7.25)} &    231.72\scriptsize{(5.43)}  &     226.35\scriptsize{(3.51)}  
                             &                    211.88\scriptsize{(4.23)} &202.20\scriptsize{(5.42)}  &  185.31\scriptsize{(5.81)} & \textbf{166.59\scriptsize{(4.19)} }   \\ 
\cmidrule{3-10}
                     & & \textit{Coverage} $\uparrow$   &0.89\scriptsize{(0.02) }     &  0.92\scriptsize{(0.01)}& 0.92\scriptsize{(0.01)}    & 
                     0.94\scriptsize{(0.01)} &  \textbf{0.97\scriptsize{(0.01)}}& 0.96\scriptsize{(0.01)} &   \textbf{0.97\scriptsize{(0.01)}}             \\ 
\cmidrule{2-10}
&\multirow{3}{*}{\begin{tabular}[c]{@{}c@{}}Large\\(\textgreater{}$100m^2$)\end{tabular}} & \textit{Mut. Over.}$\downarrow$ &  0.42\scriptsize{(0.01)}  & 0.65\scriptsize{(0.03)} & \textbf{0.33\scriptsize{(0.01)}}    &   0.52\scriptsize{(0.02)} &0.45\scriptsize{(0.01)} &   0.43\scriptsize{(0.01)} &      0.38\scriptsize{(0.01) }       \\ 
\cmidrule{3-10}
                     & & \textit{Steps} $\downarrow$      &485.74\scriptsize{(6.89)} &454.34\scriptsize{(6.89)}& 439.81\scriptsize{(8.47)}     & 
                  386.54\scriptsize{(11.96)} &381.84\scriptsize{(8.63)}& 379.45\scriptsize{(5.69)} & \textbf{323.72\scriptsize{(10.64)}}    \\ 
\cmidrule{3-10}
                  & & \textit{Coverage} $\uparrow$   
                  &0.90\scriptsize{(0.03) } & 0.92\scriptsize{(0.01)} & 0.88\scriptsize{(0.01)}  & 
                    0.94\scriptsize{(0.01)} &  0.93\scriptsize{(0.01)}  & 0.95\scriptsize{(0.01)} &    \textbf{0.96\scriptsize{(0.01)}}        \\
\midrule
\multirow{6}{*}{N = 4} & 
\multirow{3}{*}{\begin{tabular}[c]{@{}c@{}}Middle\\(\textgreater{}$70m^2$)\end{tabular}} & \textit{Mut. Over.} $\downarrow$  &0.36\scriptsize{(0.01)}  & 0.53\scriptsize{(0.03)} & \textbf{0.21\scriptsize{(0.01)}} & 0.42\scriptsize{(0.03)} &0.35\scriptsize{(0.01)} &  0.25\scriptsize{(0.01)}&  0.22\scriptsize{(0.01)}   \\ 
\cmidrule{3-10}
             & & \textit{Steps} $\downarrow$      &  236.81\scriptsize{(4.39)} &219.35\scriptsize{(3.68)}& 218.80\scriptsize{(2.46)}     & 173.30\scriptsize{(4.14)} &174.63\scriptsize{(4.57)}& 162.09\scriptsize{(3.94)}& \textbf{149.72\scriptsize{(1.94)}}  \\ 
\cmidrule{3-10}
                 & & \textit{Coverage} $\uparrow$   & 0.97\scriptsize{(0.03)}   &0.97\scriptsize{(0.01)}& 0.95\scriptsize{(0.01)}     &  \textbf{0.98\scriptsize{(0.01)}} &  0.97\scriptsize{(0.01)}& \textbf{ 0.98\scriptsize{(0.01)}} & \textbf{0.98\scriptsize{(0.01)} }           \\ 
\cmidrule{2-10}
&\multirow{3}{*}{\begin{tabular}[c]{@{}c@{}}Large\\(\textgreater{}$100m^2$)\end{tabular}} & \textit{Mut. Over.}$\downarrow$  & 0.36\scriptsize{(0.01)}  & 0.60\scriptsize{(0.01)}& \textbf{0.25\scriptsize{(0.01)}} & 0.45\scriptsize{(0.03)}& 0.38\scriptsize{(0.01) }&0.38\scriptsize{(0.01) }  &   0.28\scriptsize{(0.01) }    \\ 
\cmidrule{3-10}
                   & & \textit{Steps} $\downarrow$      &479.47\scriptsize{(7.35)} & 425.88\scriptsize{(6.15)}&
                   418.49\scriptsize{(5.23) }       &325.29\scriptsize{(5.48)} &322.27\scriptsize{(8.67)}
                   & 315.80\scriptsize{(3.55)} &\textbf{284.50\scriptsize{(3.66)}}   \\ 
\cmidrule{3-10}
                     & & \textit{Coverage} $\uparrow$     &  0.91\scriptsize{(0.03)} &\textbf{0.96\scriptsize{(0.01)}} 
                &\textbf{0.96\scriptsize{(0.01)}}         & 0.95\scriptsize{(0.01)} &0.95\scriptsize{(0.01)}&   0.95\scriptsize{(0.01)}& \textbf{0.96\scriptsize{(0.01)}}  \\
\bottomrule
\end{tabular}}}
\caption{Performance of {\name}, planning-based baselines, and RL-based baselines with $N=3,4$ agents on the Gibson dataset. Note that the horizon of middle and large maps is 300 steps and 600 steps, respectively.}
\label{tab: gibson_results}
\end{table*}



% \multirow{6}{*}{N =2} & 
% \multirow{3}{*}{\begin{tabular}[c]{@{}c@{}}Middle\\(\textgreater{}$70m^2$)\end{tabular}} & \textit{Mut. Over.} $\downarrow$ &0.45\scriptsize{(0.01)}      & 0.69\scriptsize{(0.01)}&     0.46\scriptsize{(0.01)}  &   0.53\scriptsize{(0.01)}   &0.45\scriptsize{(0.01)}  &      0.54\scriptsize{(0.01)}           &   \textbf{0.43\scriptsize{(0.01)}}    \\ 
% \cmidrule{3-10}
%                             & & \textit{Steps} $\downarrow$     &  250.58\scriptsize{(13.38)}       &       258.72\scriptsize{(2.42)}&
%                             242.05\scriptsize{(4.00)} &
%                            230.63\scriptsize{(3.53)}  &228.61\scriptsize{(4.19)} &   226.35\scriptsize{(2.78)}   &     
%                            \textbf{210.60\scriptsize{(3.82)}}   \\ 
% \cmidrule{3-10}
%                          & & \textit{Coverage} $\uparrow$   & 0.86\scriptsize{(0.06)}       & 0.90\scriptsize{(0.02)}&   0.92\scriptsize{(0.01)}  & 
%                      0.93\scriptsize{(0.02)}    &0.94\scriptsize{(0.01)}  &   0.94\scriptsize{(0.01)}          &   \textbf{0.95\scriptsize{(0.01)}}             \\ 
% \cmidrule{2-10}
% &\multirow{3}{*}{\begin{tabular}[c]{@{}c@{}}Large\\(\textgreater{}$100m^2$)\end{tabular}} & \textit{Mut. Over.}$\downarrow$  & 0.46\scriptsize{(0.02)}    & 0.71\scriptsize{(0.02)}& 
% 0.45\scriptsize{(0.02)}    &   0.51\scriptsize{(0.01)} &  0.41\scriptsize{(0.01)}  &       0.50\scriptsize{(0.01)}  &  \textbf{0.40\scriptsize{(0.02)}}      \\ 
% \cmidrule{3-10}
%                  & & \textit{Steps} $\downarrow$      &569.88\scriptsize{(16.06)}         & 530.30\scriptsize{(11.31)}  & 500.75\scriptsize{(10.43)} & 
%                   479.90\scriptsize{(6.58)} &478.17\scriptsize{(9.74)}
%                     &  465.84\scriptsize{(6.22)}  &     \textbf{427.43\scriptsize{(7.37)}}    \\ 
% \cmidrule{3-10}
%                      & & \textit{Coverage} $\uparrow$    &0.78\scriptsize{(0.05)}        & 0.88\scriptsize{(0.01)}&  0.86\scriptsize{(0.01)}  & 
%                      0.91\scriptsize{(0.02)}&0.92\scriptsize{(0.01)} &     0.92\scriptsize{(0.01)}  &  \textbf{0.93\scriptsize{(0.01)}}      \\
% \midrule
% \usepackage{multirow}
% \usepackage{booktabs}

\begin{table*}
\centering
\footnotesize
\scalebox{.95}{
\setlength\tabcolsep{1.5mm}{\begin{tabular}{ccrccc|ccc|c} 
\toprule
Agents     &Sce.                                                                                    & Metrics                  &CoScan  &NF &Voronoi  & ANS-Merge &NCM&MAANS & {\name}  \\ 
\midrule

\multirow{9}{*}{N =3}& 
\multirow{3}{*}{\begin{tabular}[c]{@{}c@{}}Middle\\(\textgreater{}$70m^2$)\end{tabular}} & \textit{Mut. Over.} $\downarrow$
&0.40\scriptsize{(0.01)} &  0.63\scriptsize{(0.04)}  & \textbf{0.32\scriptsize{(0.03)} }&0.48\scriptsize{(0.03)}   &0.41\scriptsize{(0.03)} &  0.46\scriptsize{(0.07)}   & 0.36\scriptsize{(0.01) }    \\
\cmidrule{3-10}
                  & & \textit{Steps} $\downarrow$   
                  &354.02\scriptsize{(5.40)} &333.09\scriptsize{(4.26)}& 256.78\scriptsize{(2.58)} &279.72\scriptsize{(5.38)}&268.03\scriptsize{(3.43)} & 267.23\scriptsize{(4.82)} & \textbf{238.10\scriptsize{(3.39)}} \\ 
\cmidrule{3-10}
                        & & \textit{Coverage} $\uparrow$    &0.91\scriptsize{(0.01)}&  0.95\scriptsize{(0.01)}&    0.96\scriptsize{(0.01)} & 0.96\scriptsize{(0.01) }&0.96\scriptsize{(0.01) } &   \textbf{0.97\scriptsize{(0.01) }}&    \textbf{0.97\scriptsize{(0.01) }}    \\ 
\cmidrule{2-10}
 &\multirow{3}{*}{\begin{tabular}[c]{@{}c@{}}Large\\(\textgreater{}$100m^2$)\end{tabular}} & \textit{Mut. Over.}$\downarrow$   &  0.40\scriptsize{(0.01)} & 0.63\scriptsize{(0.01)} & \textbf{0.28\scriptsize{(0.03)}}& 0.52\scriptsize{(0.05)}   &0.42\scriptsize{(0.03)}& 0.55\scriptsize{(0.07)} &   0.39\scriptsize{(0.04) }         \\ 
\cmidrule{3-10}
                         & & \textit{Steps} $\downarrow$  & 698.22\scriptsize{(8.16)} &649.60\scriptsize{(11.23)}& 509.24\scriptsize{(5.54)} & 497.41\scriptsize{(11.60)}&463.75\scriptsize{(12.10)} &458.69\scriptsize{(14.04)} & \textbf{419.88\scriptsize{(7.91)} }  \\ 
\cmidrule{3-10}
                         & & \textit{Coverage} $\uparrow$ 
                         &0.82\scriptsize{(0.01)} &0.92\scriptsize{(0.01)}&   0.94\scriptsize{(0.01)} &0.93\scriptsize{(0.01)} &0.95\scriptsize{(0.01)}&    0.95\scriptsize{(0.01)} &  \textbf{ 0.96\scriptsize{(0.01)}}    \\
\cmidrule{2-10}
&\multirow{3}{*}{\begin{tabular}[c]{@{}c@{}}Super Large\\(\textgreater{}$200m^2$)\end{tabular}} & \textit{Mut. Over.}$\downarrow$   &0.33\scriptsize{(0.01)}&0.63\scriptsize{(0.01)} & \textbf{0.28\scriptsize{(0.01)}} &  0.45\scriptsize{(0.04)}&0.41\scriptsize{(0.01)} &   0.40\scriptsize{(0.04)} &   0.35\scriptsize{(0.01)}   \\ 
\cmidrule{3-10}
                          & & \textit{Steps} $\downarrow$   
                          & 1710.88\scriptsize{(41.11)} &1456.80\scriptsize{(48.94)}& 1321.62\scriptsize{(44.73)} &1343.17\scriptsize{(50.68)}&1147.25\scriptsize{(57.26)}& 1135.24\scriptsize{(53.06)} & \textbf{982.54\scriptsize{(43.51)}}   \\ 
\cmidrule{3-10}
                          & & \textit{Coverage} $\uparrow$            &0.82\scriptsize{(0.02)} & 0.87\scriptsize{(0.01)}&    0.87\scriptsize{(0.01)} &0.92\scriptsize{(0.01)}&0.92\scriptsize{(0.01)}&     0.94\scriptsize{(0.02)} &  \textbf{ 0.97\scriptsize{(0.01) } }  \\
\midrule

\multirow{9}{*}{N = 4}& 
\multirow{3}{*}{\begin{tabular}[c]{@{}c@{}}Middle\\(\textgreater{}$70m^2$)\end{tabular}} & \textit{Mut. Over.} $\downarrow$ &0.35\scriptsize{(0.01)} &0.57\scriptsize{(0.02)}  &  \textbf{0.30\scriptsize{(0.02)}} & 0.34\scriptsize{(0.01)} &0.35\scriptsize{(0.03)}&   0.34\scriptsize{(0.01)}   & 0.35\scriptsize{(0.01)} \\ 
\cmidrule{3-10}
                     & & \textit{Steps} $\downarrow$   &280.38\scriptsize{(4.86)} & 314.12\scriptsize{(4.30)} & 233.51\scriptsize{(2.79)} &227.42\scriptsize{(5.81)} & 227.38\scriptsize{(2.10)}&219.68\scriptsize{(5.64)}&\textbf{191.31\scriptsize{(1.66)}}\\ 
\cmidrule{3-10}
                       & & \textit{Coverage} $\uparrow$    &0.95\scriptsize{(0.01)} & 0.96\scriptsize{(0.01)} &  \textbf{0.97\scriptsize{(0.01)}} & \textbf{0.97\scriptsize{(0.01)}}&
                       0.96\scriptsize{(0.02)}&  \textbf{0.97\scriptsize{(0.01)}}& \textbf{0.97\scriptsize{(0.01)} }\\ 
\cmidrule{2-10}
 &\multirow{3}{*}{\begin{tabular}[c]{@{}c@{}}Large\\(\textgreater{}$100m^2$)\end{tabular}} & \textit{Mut. Over.}$\downarrow$  &  0.37\scriptsize{(0.01)} & 0.58\scriptsize{(0.02)} &  \textbf{0.29\scriptsize{(0.01)}} & 0.44\scriptsize{(0.02)}  &0.42\scriptsize{(0.02)}& 0.41\scriptsize{(0.01)}      & 0.38\scriptsize{(0.01)}    \\ 
\cmidrule{3-10}
                 & & \textit{Steps} $\downarrow$  &  608.14\scriptsize{(7.24)} &601.15\scriptsize{(8.60)} & 407.31\scriptsize{(8.34)} &384.27\scriptsize{(9.63)}&389.12\scriptsize{(8.53)}& 363.75\scriptsize{(7.87)} &  \textbf{345.52\scriptsize{(7.85)}}\\ 
\cmidrule{3-10}
                   & & \textit{Coverage} $\uparrow$  &0.92\scriptsize{(0.01)} & 0.90\scriptsize{(0.02)}&   0.96\scriptsize{(0.01)}& 0.95\scriptsize{(0.01)}&
                   \textbf{0.97\scriptsize{(0.01)}} & \textbf{0.97\scriptsize{(0.02)}}& \textbf{0.97\scriptsize{(0.02)}} \\
\cmidrule{2-10}
&\multirow{3}{*}{\begin{tabular}[c]{@{}c@{}}Super Large\\(\textgreater{}$200m^2$)\end{tabular}} & \textit{Mut. Over.}$\downarrow$   & 0.29\scriptsize{(0.01)} &0.59\scriptsize{(0.01)} &   \textbf{0.22\scriptsize{(0.01)}} &0.35\scriptsize{(0.01)} & 0.34\scriptsize{(0.01)} &   0.31\scriptsize{(0.02)}& 0.33\scriptsize{(0.01)} \\ 
\cmidrule{3-10}
                          & & \textit{Steps} $\downarrow$   &1538.13\scriptsize{(22.13)} & 1338.95\scriptsize{(47.36)}&
                          1264.12\scriptsize{(39.34)} & 1156.49\scriptsize{(43.60)}&
                          1064.16\scriptsize{(42.85)}&
                          1012.52\scriptsize{(38.62)}  & \textbf{900.71\scriptsize{(29.24)}} \\ 
\cmidrule{3-10}
                       & & \textit{Coverage} $\uparrow$  &   0.89\scriptsize{(0.01)} & 0.90\scriptsize{(0.01)}&   0.91\scriptsize{(0.02)} &0.91\scriptsize{(0.01)} &0.94\scriptsize{(0.02)}&   0.93\scriptsize{(0.01)}& \textbf{0.97\scriptsize{(0.01)}}    \\
\bottomrule
\end{tabular}}}
\caption{Performance of {\name}, planning-based baselines and RL-based baselines with $N=3,4$ agents on the HM3D dataset. Note that the horizon of middle, large, and super large maps is 450 steps, 720 steps, and 1800 steps, respectively.}
\label{tab: hm3d_results}
\vspace{-6mm}
\end{table*}



% \multirow{9}{*}{N =2}& 
% \multirow{3}{*}{\begin{tabular}[c]{@{}c@{}}Middle\\(\textgreater{}$70m^2$)\end{tabular}} & \textit{Mut. Over.} $\downarrow$ & 0.43\scriptsize{(0.01)}         & 0.68\scriptsize{(0.01)} &0.41\scriptsize{(0.01)}     & 0.53\scriptsize{(0.01)}&\textbf{0.39\scriptsize{(0.02)}}&   0.47\scriptsize{(0.01)}         &   \textbf{0.39\scriptsize{(0.03)}}     \\ 
% \cmidrule{3-10}
%                      & & \textit{Steps} $\downarrow$   &428.43\scriptsize{(2.75)}    &    378.81\scriptsize{(4.29)}&      318.81\scriptsize{(3.30)}    &  344.37\scriptsize{(2.03)}  & 325.97\scriptsize{(3.45)} &      312.70\scriptsize{(2.16)}  &    \textbf{298.54\scriptsize{(4.14)}}    \\ 
% \cmidrule{3-10}
%                      & & \textit{Coverage} $\uparrow$    &0.80\scriptsize{(0.01)}&  0.91\scriptsize{(0.01)}&   0.93\scriptsize{(0.01)}   &  0.90\scriptsize{(0.01)}  &0.93\scriptsize{(0.01)}&  \textbf{0.95\scriptsize{(0.01)}}  &    \textbf{0.95\scriptsize{(0.01)}}    \\ 
% \cmidrule{2-10}
%  &\multirow{3}{*}{\begin{tabular}[c]{@{}c@{}}Large\\(\textgreater{}$100m^2$)\end{tabular}} & \textit{Mut. Over.}$\downarrow$  &  0.41\scriptsize{(0.01)}& 0.71\scriptsize{(0.01)}& 0.39\scriptsize{(0.02)}       &  0.45\scriptsize{(0.02)} &0.43\scriptsize{(0.01)} &0.41\scriptsize{(0.02)}    &   \textbf{0.32\scriptsize{(0.05)}}          \\ 
% \cmidrule{3-10}
%                           & & \textit{Steps} $\downarrow$   &689.86\scriptsize{(8.43)} & 712.75\scriptsize{(12.36)}&  604.30\scriptsize{(13.67)}    &    607.55\scriptsize{(8.48)}
%                 & 592.28\scriptsize{(10.31)} &  561.84\scriptsize{(5.72)} 
%                 & \textbf{523.19\scriptsize{(7.23)}}   \\ 
% \cmidrule{3-10}
%                   & & \textit{Coverage} $\uparrow$ &0.71\scriptsize{(0.01)}       & 0.80\scriptsize{(0.01)}&  0.89\scriptsize{(0.02)}        & 0.90\scriptsize{(0.01)}   &0.92\scriptsize{(0.01)}  & 0.92\scriptsize{(0.01)}      &    \textbf{ 0.94\scriptsize{(0.01)} } \\
% \cmidrule{2-10}
% &\multirow{3}{*}{\begin{tabular}[c]{@{}c@{}}Super Large\\(\textgreater{}$200m^2$)\end{tabular}} & \textit{Mut. Over.}$\downarrow$  & 0.31\scriptsize{(0.03)}    & 0.74\scriptsize{(0.01)}&    0.37\scriptsize{(0.01)}     &0.42\scriptsize{(0.01)}    & 0.35\scriptsize{(0.01)}  &    0.38\scriptsize{(0.01)}      &        \textbf{0.26\scriptsize{(0.01)} }\\ 
% \cmidrule{3-10}
%                          & & \textit{Steps} $\downarrow$          &1789.34\scriptsize{(59.42)} & 1667.61\scriptsize{(56.39)} &   1380.81\scriptsize{(63.82)}       & 1474.00\scriptsize{(53.54)}    & 1409.04\scriptsize{(37.62)} & 1323.48\scriptsize{(52.32)} &    \textbf{ 1228.07\scriptsize{(48.98)} }  \\ 
% \cmidrule{3-10}
%                         & & \textit{Coverage} $\uparrow$  &
%                        0.68\scriptsize{(0.03)}   &0.86\scriptsize{(0.01)}&      0.87\scriptsize{(0.01)}       &   0.83\scriptsize{(0.01)}   & 0.87\scriptsize{(0.02)}   &   0.91\scriptsize{(0.02)}   &     \textbf{0.94\scriptsize{(0.01)}}    \\
% \midrule
\subsection{Evaluation Metrics}
We consider 3 statistical metrics to capture different characteristics of a particular exploration strategy. These metrics are only for analysis, and we primarily focus on \emph{Steps} as our performance criterion.

% \vspace{-\topsep}
\begin{itemize}
\setlength{\parskip}{0pt} \setlength{\itemsep}{0pt plus 1pt}
    \item \textbf{Steps:  }This metric considers the timesteps required to achieve 90\% coverage within an episode. Fewer \emph{Steps} imply faster exploration.
    \item \textbf{Coverage:  }This metric denotes the final ratio of the explored area to total explorable area at the end of the episode. A higher \emph{Coverage} ratio reflects a more effective exploration.
    \item \textbf{Mutual Overlap: }This metric shows the ratio of the overlapped area to the currently explored area when the \emph{Coverage} ratio achieves 90\%. Lower \emph{Mutual Overlap} ratio indicates better collaboration.
\end{itemize}
\vspace{-\topsep}

%  We remark that there is an error in calculating the \emph{Coverage} ratio due to the inaccuracy of the explorable region and merged explored area. Therefore, the exploration can be considered thorough if the \emph{Coverage} ratio is over 95\%.  
%These metrics are only for analysis, and we primarily focus on \emph{Steps} as our performance criterion.

% \item \textbf{Communication Traffic: }This is the total bytes of message exchange when the coverage ratio achieves $90\%$. Lower \emph{Communication Traffic} infers lower bandwidth required.
\subsection{Baselines}
We challenge {\name} against three representative planning-based approaches (CoScan, Topological Frontier, Voronoi) and three prominent RL-based solutions (ANS-Merge, NeuralCoMapping, MAANS). Note that Topological Frontier and Voronoi are also graph-based approaches.

% Note that existing RL-based works with active topological mapping~\cite{vgm,norl,topo-map2} are designed for single-agent navigation, which is a downstream task of exploration~\cite{singleagent-RL1} and is difficult to be directly extended to multi-agent exploration without the pre-defined target goal.


% \vspace{-\topsep}
\begin{itemize}
\setlength{\parskip}{0pt} \setlength{\itemsep}{0pt plus 1pt}
 \item \textbf{CoScan}~\cite{CoScan}: This frontier-based method applies k-means clustering to all frontiers and assigns a frontier cluster to each agent. Afterward, each agent plans an optimal traverse path over the assigned frontiers.
 %based on the optimal mass transport problem
 \item \textbf{Topological Frontier (TF)}~\cite{normalized_frontier}: This graph-based approach calculates a normalized traveling cost for each ghost node built from the Topological Mapper and considers the node with the lowest cost as the global goal.\looseness=-1
%   which is a linear combination of the number of neighbor ghost nodes and the distance from the ghost node to the agent.  by considering the distance and the number of its neighbors 
% by considering the number of its neighbors and the distance from the agent to the node.
 \item \textbf{Voronoi}~\cite{Voronoi}: This graph-based solution divides the map into several parts and transforms it into a Voronoi graph. Each agent then only searches the unexplored region in its partition, reducing the overlapped area.
 \item \textbf{ANS-Merge}~\cite{ans}: ANS is exemplary in RL-based single-agent exploration. It takes in egocentric local and global metric maps and infers global goals for the agents. We extend ANS to multi-agent exploration by sending merged maps to the global planner and use the same reward function as ours.
 \item\textbf{NeuralCoMapping (NCM)}~\cite{cvpr22}: NeuralCoMapping introduces a multiplex graph neural network to predict the neural distance between frontier nodes and agents. It then assigns each agent a frontier node based on the neural distance in each global step.
 \item \textbf{MAANS}~\cite{RL_multi2}: MAANS is a variant of ANS for multi-agent exploration. This method leverages a transformer-based Spatial-TeamFormer to enhance cooperation. For a fair comparison, we conduct training on 10 maps without the policy distillation mentioned in \cite{RL_multi2}.
\end{itemize}
% \vspace{-\topsep}
% adopts a CNN-based architecture that
%  \item \textbf{WMA\_RRT}~\cite{WMA-RRT}: This is a multi-agent variant of RRT~\cite{RRT}, where agents cooperatively build a single tree and observe a locking-and-search scheme.
%\item\textbf{NeuralCoMapping (NCM)}~\cite{cvpr22}: This is a frontier-based multi-agent exploration method that constructs a multiplex graph neural network to choose a frontier node.
% \item \textbf{ILP}~\cite{ILP}: This strategy uses the information gain and the distance cost from frontiers to agents to calculate the utility function. The agents select a sequence of frontiers with the most utility value to make a planning using the Hungarian algorithm~\cite{Hungarian}.

We remark that {\name} and the baselines are under the same assumptions in our task. All the baselines only replace the Topological Mapper and the {\planner} with alternatives and keep the rest the same as {\name}, except for TF, which only substitutes the {\planner}. 


% We include more implementation details in the appendix.

\begin{figure*}[ht]
\captionsetup{justification=centering}
	\centering
    \subfigure[\label{fig:casea}Comparison of Map Construction]
        {\centering
        {\includegraphics[height=4cm]{figures/case_a.png}
            }
    }
    \subfigure[\label{fig:caseb}Comparison of Planning Strategy]
        {\centering
        {\includegraphics[height=4cm]{figures/case_b.png}
    	}
    }
     \vspace{-2mm}
    \caption{\raggedright{Case studies of Map Construction and Planning Strategy. (a) shows that the graph structures of different scenes seem congruous in general, marked in grey. (b) displays agents' trajectories in Voronoi and {\name}, respectively.  }}
    \label{fig:case}
    \vspace{-5mm}
\end{figure*} 

% Voronoi constrains 2 agents in the green and the black partition, respectively. {\name} distributes agents based on the graph nodes.

\begin{figure}[ht!]
	\centering
     \vspace{1mm}
    \includegraphics[width=1.0\linewidth]{figures/overall_scenes.png}
    \vspace{-4mm}
	\centering \caption{Comparison between {\name} (red) and its variants. {\name} has the lowest \emph{Steps}
and \emph{Mutual Overlap}}
\label{fig:ab}
\vspace{-6mm}
\end{figure}
\subsection{Main Results}
\subsubsection{Evaluation Results}
The results in \cref{tab: gibson_results} show that {\name} outperforms all baselines with $N=3,4$ agents on unseen scenes on Gibson.
In both middle and large maps, {\name} attains the fewest \emph{Steps} and the highest \emph{Coverage} ratio. More concretely, MAANS is the best RL baseline since its transformer-based Spatial-TeamFormer captures the spatial relationship and team representation, while {\name} has 10.10\% and 7.63\% fewer \emph{Steps} than MAANS with $N=3,4$ agents, respectively. This indicates that ghost nodes, which are always located in unexplored areas, motivate agents to explore unseen regions. {\name} also excels in the \emph{Mutual Overlap} ratio among the RL solutions, suggesting that it better assigns global goals to agents in different unexplored directions.\looseness=-1


Among the planning-based baselines, the best competitor, Voronoi, achieves the lowest \emph{Mutual Overlap} ratio in the 3-agent setting, demonstrating that the Voronoi separates agents to reduce the overlapped area. However, {\name} is superior to Voronoi in the \emph{Steps} and the \emph{Coverage} ratio, reducing \emph{Steps} by 26.40\% and 31.57\% for $N=3,4$ agents, respectively.
This implies that although the Voronoi partition separates agents, it may hinder exploration efficiency by confining agents to stay in their respective areas. 

% On the contrary, the RL-based solution has the potential to adapt to complex strategies in various situations.

% For example, the Voronoi agent cannot go through the corridor in the other agent's partition to further explore the unseen area. 

\subsubsection{Domain Generalization}
% We evaluate the model trained on the Gibson dataset  For domain generalization, \cref{tab: hm3d_results} summarizes the performance with $N=2,3$ agents on the HM3D dataset. 
We also report the domain generalization performance in \cref{tab: hm3d_results}, where
all models trained on the Gibson dataset are evaluated in the HM3D domain.
The results indicate that {\name} performs best. Compared to {MAANS}, the best competitor, {\name} achieves 13.45\% and 11.04\% fewer \emph{Steps} with $N=3,4$ in super large scenes. This implies that the trained MAANS may overfit to the spatial arrangements in the training maps, resulting in suboptimal performance on unseen scenes in other domains due to variations in different metric maps.
In contrast, {\name} exploits the graph structure with abstract but essential information to better adapt to scenarios in an unseen domain. \looseness=-1


The planning-based methods fail to achieve an average \emph{Coverage} ratio of 90\% on super large maps. 
This indicates that the factors affecting exploration success are more complex on unseen super large maps. It is difficult for planning-based agents to take all factors into account in manual parameter tuning. As a result, agents may get stuck in the corner and fail to reach 90\% \emph{Coverage}.



% The best planning-based method, Voronoi, has 11.06\% and 25.65\% more \emph{Steps} than {\name} with $N=2,3$ agents on super large maps.

% \subsubsection{Comparison in Communication Traffic}
% {\color{red}In a map with a size of $M\times M$, the communication traffic of the metric map can be compressed by Huffman encoding~\cite{Huffman}. At the same time, the communication traffic of each global step is $K\times b$ for the topological graph, where $K$ is the number of nodes, and $b$ is the feature dimension of a node. In {\name}, $K$ is around $40$ and $b$ is $4$ in a $480\times 480$ metric map. In~\cref{tab: communication}, the results of the communication traffic in an episode reveal that {\name} affords the best performance among all baselines, reducing the communication traffic by over 91.70\%. It is suggested that {\name} is more reliable in real scenarios with a restricted bandwidth.}
\iffalse
huffman 1bit*480*480 -> (5bit*4+4bit*6)/10 *480*480 -> 1bit*480*480
32 bit == 4 byte
\fi

\subsubsection{Case Study}
We present case studies of map construction in two different scenes to showcase the generalization of {\name}. Besides, to further demonstrate the cooperative exploration strategy of {\name}, we visualize the planning strategies of {\name} and the most competitive planning-based method, Voronoi.


%We visualize our MANTS and a planning-based algorithm Voronoi in three different scenes. The results show our capacity of generalization and coordination, which contributes to its great performance.

In Fig.~\ref{fig:casea}, the graph structures in two different scenes appear to be generally congruent (i.e., the area in grey). This suggests that topological maps, which contain abstract but essential information, are less influenced by scene structures, endowing them with significant generalization capabilities. Conversely, the shape of metric maps depends on the layout of the scene. As a result, finding a (near) optimal exploration strategy for various metric maps is challenging.
In Fig.~\ref{fig:caseb}, the agent trajectories show that {\name} successfully allocates agents to different unexplored areas via the selected ghost nodes. Moreover, {\name} agents can temporarily revisit previously explored areas to reach unexplored areas in different directions, thereby increasing the final coverage. 
On the contrary, when the only path to the unexplored area belongs to the partition of a particular Voronoi agent (i.e., the area in blue), the other agent can only be constrained to its own partition (i.e., the area in green), resulting in inefficient exploration.


%We hypothesize that Voronoi may fail in some complex environments due to its formally-designed cooperative strategy.

\iffalse
\subsubsection{Asynchronized Evaluation}
We consider an asynchronized situation where some agents loss their connection during exploration so that the message received by these agents is incomplete. In~\cref{tab: hm3d_results}



\fi

\iffalse

\begin{figure}[h]
	\centering
    \includegraphics[width=0.9\linewidth]{figures/real.png}
	\centering \caption{The deployment of real-world robot system. }
\label{fig:real}

\end{figure}
\fi




% and measure the \emph{Steps} and the \emph{Mutual Overlap} ratio  derived from the visual encoder~\cite{vgm}
\vspace{-1mm}
\subsection{Ablation Study}
We report the training performance of several RL variants to investigate the importance of the Mapper and the {\planner}.

% \vspace{-\topsep}
\begin{itemize}
\setlength{\parskip}{0pt} \setlength{\itemsep}{0pt plus 1pt}
 \item \textbf{Mapper w.o. Distance: }Without the help of the FMM distance, the Topological Mapper constructs the graph based solely on the similarity of the visual embeddings. 
  \item \textbf{{\planner} w.o. History: }We abandon the historical graphs, $\hat{G}$, and the Graph Fusion. The Main Node Selector only takes $G$ as input.
 \item \textbf{{\planner}-Single: }We abandon the Main Node Selector and only adopt the Ghost Node Selector to infer global goals.
 \item \textbf{{\planner}-Concat: }Before calculating $S_{g, re}$ in the Ghost Node Selector, we update ghost node features by concatenating them and the corresponding main node features. We then discard the multiplication in Equation~\ref{eq:multiply} and directly consider $S_{g, re}$ as the final matching score.
 
\end{itemize}
% \vspace{-\topsep}
\iffalse
\item \textbf{{\planner} w.o. GNN}: We replace GNN with Multi-Layer Perception to extract the relationship between different kinds of graphs in {\planner}.

  \item \textbf{{\planner}-Mean: }We replace GNN with mean operation to update the node features in the graphs and only implement GNN to calculate the weighted scores.
  \emph{{\planner}-Mean} results in the highest \emph{Mutual Overlap} ratio and \emph{Steps}, indicating that GNN infers probability distribution of different global goal candidates and can better capture the interactions among agents and topological map to encourage exploration efficiency and cooperation.
\fi

% an illogical map structure due to
As shown in Fig.~\ref{fig:ab}, {\name} has the lowest \emph{Steps} and \emph{Mutual Overlap} ratio in $N$= 2 agents on Gibson. {\name} is superior to \emph{Mapper w.o. Distance} with over 10\% lower \emph{Steps}, suggesting that distance-based heuristics reduce incorrect connections between nodes and provide a more accurate graph for effective global planning.
Among all {\planner} variants, \emph{{\planner}-Single} has the highest \emph{Mutual Overlap} ratio and \emph{Steps}. This indicates that directly selecting ghost nodes as global goals may lead to a sub-optimal solution due to the large number of node candidates. The performance of \emph{{\planner} w.o. History} is worse in \emph{Mutual Overlap} ratio, suggesting that the lack of historical memory affects cooperation.
\emph{{\planner}-Concat} shows the lowest training convergence. The result expresses that the concatenation of the main and ghost node features prevents the {\planner} from better perceiving the relationship between these two types of nodes.

% the hierarchical global selection can be easily optimized with much fewer node candidates, while 

% \subsection{Real-World Transfer}
% {\color{red}We deploy the {\name} for exploration on two Mecanum steering robots with noisy odometry sensors in the real world. However, the transfer from the visual-based simulator to the real world is challenging due to the perception gap and the imperfect motion. To reduce the uncertainty, we first test {\name} and baselines in Gazebo~\cite{Gazebo}, a simulator in support of ROS~(Robot Operating System)~\cite{ROS}, and then transfer the system into the real world.} 

% \input{tables/time}

% {\color{red}As shown in ~\cref{tab: time}, we compare {\name} with the best planning-based and RL-based competitors, Voronoi and MAANS, stating the \emph{Run-Time} when agents achieve 95\% \emph{Coverage} ratio in the environment. 
% In the Gazebo, {\name} shows the best performance, denoting that it prefers to assign distant targets to different agents resulting in cooperative and efficient exploration. However, {\name} is a bit better than other baselines in the real world. We hypothesize that the real-world test scene is relatively small (i.e., the size is 30 $m^2$), where {\name}'s strength of cooperative exploration is less obvious.
% More details can be found in Appendix 4.}

\iffalse
Besides, the running time of different methods in the real-world is presents in \yxy{tab:fig}, showing that {\name} outperforms the all the baselines with a large margin. Specifically, \yxy{add comment}, which suggests that {\name} effectively and cooperatively explores the environment.
\fi


\section{Conclusion and Limitations}
We propose \emph{Multi-Agent Topological Neural Mapping} (\name), a multi-agent topological exploration framework, to improve exploration efficiency and generalization. In {\name}, the Topological Mapper constructs graphs via a visual encoder and distance-based heuristics. The RL-based Hierarchical Topological Planner (\planner) captures the relationships between agents and graph nodes to infer global goals. Experiments in Habitat demonstrate that {\name} outperforms planning-based baselines and RL variants in unseen scenes. However, there is a huge room for improvement in {\name}. For example, our method leverages metric maps to prune graphs, thus the precision of the metric map may influence the quality of the topological maps. Besides, we assume fully synchronized decision-making, which may fail in environments with communication latency. 

% Learning communication protocols for limited bandwidth~\cite{NIPS2016_c7635bfd} is a future research direction.
% Moreover, our experiments focus on a small number of agents due to the low data efficiency of MAPPO~\cite{mappo}, which may be tackled by combining {\name} with curriculum learning~\cite{epciclr2020}.


% \begin{table}[h]
% \caption{An Example of a Table}
% \label{table_example}
% \begin{center}
% \begin{tabular}{|c||c|}
% \hline
% One & Two\\
% \hline
% Three & Four\\
% \hline
% \end{tabular}
% \end{center}
% \end{table}


%    \begin{figure}[thpb]
%       \centering
%       \framebox{\parbox{3in}{We suggest that you use a text box to insert a graphic (which is ideally a 300 dpi TIFF or EPS file, with all fonts embedded) because, in an document, this method is somewhat more stable than directly inserting a picture.
% }}
%       %\includegraphics[scale=1.0]{figurefile}
%       \caption{Inductance of oscillation winding on amorphous
%        magnetic core versus DC bias magnetic field}
%       \label{figurelabel}
%    \end{figure}
   


%%%%%%%%%%%%%%%%%%%%%%%%%%%%%%%%%%%%%%%%%%%%%%%%%%%%%%%%%%%%%%%%%%%%%%%%%%%%%%%%



%%%%%%%%%%%%%%%%%%%%%%%%%%%%%%%%%%%%%%%%%%%%%%%%%%%%%%%%%%%%%%%%%%%%%%%%%%%%%%%%



%%%%%%%%%%%%%%%%%%%%%%%%%%%%%%%%%%%%%%%%%%%%%%%%%%%%%%%%%%%%%%%%%%%%%%%%%%%%%%%%
% \section*{APPENDIX}

% Appendixes should appear before the acknowledgment.

% \section*{ACKNOWLEDGMENT}

% The preferred spelling of the word ÒacknowledgmentÓ in America is without an ÒeÓ after the ÒgÓ. Avoid the stilted expression, ÒOne of us (R. B. G.) thanks . . .Ó  Instead, try ÒR. B. G. thanksÓ. Put sponsor acknowledgments in the unnumbered footnote on the first page.



%%%%%%%%%%%%%%%%%%%%%%%%%%%%%%%%%%%%%%%%%%%%%%%%%%%%%%%%%%%%%%%%%%%%%%%%%%%%%%%%



% \begin{thebibliography}{99}
\bibliographystyle{IEEEtran}
\bibliography{IEEEfull}

% \bibitem{c1} G. O. Young, ÒSynthetic structure of industrial plastics (Book style with paper title and editor),Ó 	in Plastics, 2nd ed. vol. 3, J. Peters, Ed.  New York: McGraw-Hill, 1964, pp. 15Ð64.
% \bibitem{c2} W.-K. Chen, Linear Networks and Systems (Book style).	Belmont, CA: Wadsworth, 1993, pp. 123Ð135.
% \bibitem{c3} H. Poor, An Introduction to Signal Detection and Estimation.   New York: Springer-Verlag, 1985, ch. 4.
% \bibitem{c4} B. Smith, ÒAn approach to graphs of linear forms (Unpublished work style),Ó unpublished.
% \bibitem{c5} E. H. Miller, ÒA note on reflector arrays (Periodical styleÑAccepted for publication),Ó IEEE Trans. Antennas Propagat., to be publised.
% \bibitem{c6} J. Wang, ÒFundamentals of erbium-doped fiber amplifiers arrays (Periodical styleÑSubmitted for publication),Ó IEEE J. Quantum Electron., submitted for publication.
% \bibitem{c7} C. J. Kaufman, Rocky Mountain Research Lab., Boulder, CO, private communication, May 1995.
% \bibitem{c8} Y. Yorozu, M. Hirano, K. Oka, and Y. Tagawa, ÒElectron spectroscopy studies on magneto-optical media and plastic substrate interfaces(Translation Journals style),Ó IEEE Transl. J. Magn.Jpn., vol. 2, Aug. 1987, pp. 740Ð741 [Dig. 9th Annu. Conf. Magnetics Japan, 1982, p. 301].
% \bibitem{c9} M. Young, The Techincal Writers Handbook.  Mill Valley, CA: University Science, 1989.
% \bibitem{c10} J. U. Duncombe, ÒInfrared navigationÑPart I: An assessment of feasibility (Periodical style),Ó IEEE Trans. Electron Devices, vol. ED-11, pp. 34Ð39, Jan. 1959.
% \bibitem{c11} S. Chen, B. Mulgrew, and P. M. Grant, ÒA clustering technique for digital communications channel equalization using radial basis function networks,Ó IEEE Trans. Neural Networks, vol. 4, pp. 570Ð578, July 1993.
% \bibitem{c12} R. W. Lucky, ÒAutomatic equalization for digital communication,Ó Bell Syst. Tech. J., vol. 44, no. 4, pp. 547Ð588, Apr. 1965.
% \bibitem{c13} S. P. Bingulac, ÒOn the compatibility of adaptive controllers (Published Conference Proceedings style),Ó in Proc. 4th Annu. Allerton Conf. Circuits and Systems Theory, New York, 1994, pp. 8Ð16.
% \bibitem{c14} G. R. Faulhaber, ÒDesign of service systems with priority reservation,Ó in Conf. Rec. 1995 IEEE Int. Conf. Communications, pp. 3Ð8.
% \bibitem{c15} W. D. Doyle, ÒMagnetization reversal in films with biaxial anisotropy,Ó in 1987 Proc. INTERMAG Conf., pp. 2.2-1Ð2.2-6.
% \bibitem{c16} G. W. Juette and L. E. Zeffanella, ÒRadio noise currents n short sections on bundle conductors (Presented Conference Paper style),Ó presented at the IEEE Summer power Meeting, Dallas, TX, June 22Ð27, 1990, Paper 90 SM 690-0 PWRS.
% \bibitem{c17} J. G. Kreifeldt, ÒAn analysis of surface-detected EMG as an amplitude-modulated noise,Ó presented at the 1989 Int. Conf. Medicine and Biological Engineering, Chicago, IL.
% \bibitem{c18} J. Williams, ÒNarrow-band analyzer (Thesis or Dissertation style),Ó Ph.D. dissertation, Dept. Elect. Eng., Harvard Univ., Cambridge, MA, 1993. 
% \bibitem{c19} N. Kawasaki, ÒParametric study of thermal and chemical nonequilibrium nozzle flow,Ó M.S. thesis, Dept. Electron. Eng., Osaka Univ., Osaka, Japan, 1993.
% \bibitem{c20} J. P. Wilkinson, ÒNonlinear resonant circuit devices (Patent style),Ó U.S. Patent 3 624 12, July 16, 1990. 






% \end{thebibliography}




\end{document}
