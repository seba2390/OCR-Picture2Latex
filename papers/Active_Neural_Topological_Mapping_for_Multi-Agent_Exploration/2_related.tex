\section{Related Work}
%1、当前建图方法大多数是栅格图,各种方法,参考文献,优缺点
%2、近年来提出了拓扑地图+栅格地图或者纯碎拓扑地图的方法,各种方法,参考文献,优缺点
%3、here,为了解决拓扑图建图的不稳定和不精确和不具有特征等缺点,我们提出了xxx方法(名字待定),采用分层级建图的方法,自适应地解决了有效点和无效点的判断建图问题,提供更明确的对未知方向的指引。

\subsection{Multi-Agent Exploration} 
In classical visual exploration, agents first perform simultaneous localization and mapping (SLAM)~\cite{SLAM1} to locate their position and reconstruct 2D maps from sensory signals. They then utilize search-based planning algorithms to generate valid exploration trajectories. Representative works mainly include frontier-based methods~\cite{RRT,CoScan}, sampling-based methods~\cite{sample1,sample2}, and graph-based methods~\cite{normalized_frontier,Voronoi}. However, these solutions suffer from expensive computational costs and limited representation capabilities. Recently, deep reinforcement learning~\cite{cvpr22,RL_multi2, yang2023learning} has attracted significant attention due to its powerful expressiveness. 
NeuralCoMapping~\cite{cvpr22} utilizes a multiplex graph neural network to choose effective frontiers as global goals. MAANS~\cite{RL_multi2} uses a transformer-based architecture to infer spatial relationships and intra-agent interactions. However, all these approaches are based on metric maps, which are sensitive to different scene structures and result in subpar generalization. In this paper, we introduce an RL-based topological approach for efficient exploration and superior generalization in unseen scenarios.\looseness=-1



%EMAC~\cite{multiagent-RL} exploits the spatial inductive bias for effective exploration in an end-to-end fashion. autonomous topological map construction and a hierarchical goal selection. Many works leverage the spatial inductive bias for effective exploration via an end-to-end fashion~\cite{multiagent-RL}, spatial-teamformer~\cite{RL_multi2}, and bipartite graph~\cite{cvpr22}.
% Nevertheless, it is challenging to optimize multiple policies jointly with high search space dimensions in MARL, which results in training inefficiency and sub-optimal solutions in complex environments. 


\subsection{Spatial Representation}
Spatial representation usually includes two main types of maps: metric and topological maps. Metric maps are grid maps where each grid predicts its traversability~\cite{ans,save}. However, metric maps struggle with generalization due to significant structural variations across different scenes.
In contrast, topological maps~\cite{graph_generalization} abstractly preserve essential environmental features with nodes and edges, offering a potential solution for improved generalization.
Several works~\cite{topo_exploration,topo-map4} are based on pre-built graphs, focusing on graph refinement or finding optimal paths for coverage.
Recent literature~\cite{vgm,topo-map2,norl} utilizes active topological mapping for navigation tasks. For instance, \cite{vgm} employs the cosine similarity of visual embeddings to construct graphs. However, it requires expert trajectories which are difficult to acquire in the NP-hard multi-agent exploration problem.
Furthermore, \cite{norl} leverages depth images to predict explored/unexplored nodes, while \cite{topo-map2} utilizes semantics for approximate geometric reasoning in topological representations.
However, \cite{norl,topo-map2} requires a predefined goal to predict the geodesic distance from unexplored nodes and select a node with the shortest predicted distance. This is unsuitable for multi-agent exploration where there are no pre-defined goals. 
In this work, we introduce an RL-based Hierarchical Topological Planner to effectively apply active topological mapping in multi-agent exploration. \looseness=-1



% proposes an image-based topological graph construction method that applies a sampling-based map building approach
% However, the approaches above are designed for single-agent navigation, a downstream task of exploration~\cite{singleagent-RL1}, which has a pre-defined target to be achieved and does not require the graph to represent the entire environment. 
% \cite{topo-map5} proposes a topological visual navigation method with graph update strategies that improve lifelong navigation performance over time.
%via a collection of multi-agent-specific enhancements
% which is a common paradigm in exploration 
% tasks for both planning-based methods~\cite{normalized_frontier} and learning-based methods~\cite{norl}. 


% designs topological representations for space that
% It utilizes different types of nodes in the topological map to represent explored/unexplored areas, 

%1、目前的视觉导航在单机上的传统方法和rl上运用十分广泛,有各种方法+参考文献,传统优点,rl优点,但是拓展到多机上存在计算开销大,负载不均,多样性等问题(待定,参考eccv2022和cvpr2022),
%2、目前多机的传统和rl工作 存在的问题 泛化性和大地图通信受限的问题
%3、我们的方法xxx使用拓扑地图解决泛化性和大地图通信受限的问题,性能也有上升。
\begin{figure*}[t!]
	\centering
 \vspace{2mm}
    \includegraphics[width=0.96\linewidth]{figures/overview.png}
	\centering \caption{Overview of \emph{Multi-Agent Neural Topological Mapping} ({\name}). Here, we take Agent $k$ as an example.}
\label{fig:Overview}
\vspace{-5mm}
\end{figure*}
\subsection{Graph Neural Networks}
Graph neural networks (GNNs)~\cite{gnn} are widely utilized in multi-agent systems to model interactions between agents. 
\cite{gnn_hierarchical} proposes a hierarchical graph attention network that captures the underlying relationships at the agent and the group level, thereby improving generalization. \cite{vgm} considers GNN as an encoder for mapping tasks to extract node features from images. Additionally, \cite{cvpr22} formulates exploration tasks as bipartite graph matching. In this paper, we propose a hierarchical goal selector based on GNN to capture the correlations between agents and topological maps in a coarse-to-fine manner.\looseness=-1



%\cite{gnn_communication,gnn_communication2} utilizes GNN to learn communication between agents. improves the coordination among agents and the exploration efficiency.
% Additionally, SuperGlue~\cite{superglue} and LoFTR~\cite{loftr} are developed for the graph-matching problem. 
%1. 
%1、Graph Neural Network在视觉和nlp领域运用广泛,参考文献,也有用在multi-agent的参考我之前整理的paper list,参考cvpr2022。优点。
%2、这里我们使用来建立multi-agent和拓扑地图的联系,提取xxx特征