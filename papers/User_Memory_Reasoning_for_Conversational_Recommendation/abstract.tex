% for easy copy paste
Conversational recommendation system aims to collect users' up-to-date preferences through dialogue, instead of relying only on preferences learned offline.
However, most existing systems make an unnatural assumption that users' preferences can only be collected offline or online, and neglect the fact that the knowledge about a user is dynamic and cumulative in nature.
To this end, we propose a novel concept called user memory graph, which aims to maintain the knowledge about a user in a structured form for interpretability.
Each turn of dialogue is grounded onto this user memory graph for the reasoning of dialogue policy, and more importantly, further accumulation of user knowledge.
To the best of our knowledge, there is no existing dialogue dataset with such a grounded user memory graph.
As such, we first create a new Memory Graph <-> Conversational Recommendation parallel corpus called MGConvRex with 6K human-to-human role playing dialogues, grounded on a large-scale memory graphs bootstrapped from real-world user scenarios.
We then propose a graph convolutional network (GCN) based model for both reasoning and dialogue policy generation.
Experiments are conducted for both offline metrics and online simulation, showing promising results.
The dataset, framework library and models will be released for future research for conversational recommendation.