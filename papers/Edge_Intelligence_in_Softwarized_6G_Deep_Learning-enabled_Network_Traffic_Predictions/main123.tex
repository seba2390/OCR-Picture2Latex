\documentclass[10pt, conference]{IEEEtran}
%\documentclass[10pt, conference]{IEEEtran}
%\documentclass[draftclsnofoot,12pt,onecolumn]{IEEEtran}

\usepackage{cite}
\usepackage{amssymb}
\usepackage{mathtools}
\usepackage{amsmath}
% \DeclarePairedDelimiter\ceil{\lceil}{\rceil}
% \DeclarePairedDelimiter\floor{\lfloor}{\rfloor}
\usepackage{graphicx}
\graphicspath{ {./image/} }
%\usepackage{gensymb}
% \setlength{\columnsep}{0.2 in}
% \def\BibTeX{{\rm B\kern-.05em{\sc i\kern-.025em b}\kern-.08em T\kern-.1667em\lower.7ex\hbox{E}\kern-.125emX}}
%\usepackage{showframe}
% \usepackage{tikz}
% \usetikzlibrary{patterns}
% \usepackage{array}
%\usepackage{subfigure}
\usepackage{color}
\ifCLASSOPTIONcompsoc
    \usepackage[caption=false, font=normalsize, labelfont=sf, textfont=sf]{subfig}
\else
\usepackage[caption=false, font=footnotesize]{subfig}
\fi
\newcommand{\fakeparagraph}[1]{\vspace{.1mm}\noindent\textbf{#1}}
\newcommand{\fakepar}[1]{\fakeparagraph{#1}}
%\usepackage[sorting = none, backend = bibtex, style=numeric-comp]{biblatex}
% \newcommand{\rs}[1]{\mathstrut\mbox{\scriptsize\rm #1}}
% \newcommand{\rr}[1]{\mbox{\rm #1}}

% \usepackage[utf8]{inputenc}
%\usepackage[english]{babel}
% \usepackage{comment}
%\bibliography{IEEEabrv,mybib.bib}
% \usepackage[utf8]{inputenc}
%\usepackage{multirow}
%\usepackage{multicol}
% \usepackage{xcolor}
%\usepackage{subfigure}
% \usepackage[linesnumbered,ruled,vlined]{algorithm2e}
% \usepackage{algorithmic}
% \newcommand\mycommfont[1]{\footnotesize\ttfamily\textcolor{blue}{#1}}
% \SetCommentSty{mycommfont}
% \SetKwInput{KwInput}{Input}% Set the Input
% \SetKwInput{KwInitialize}{Initialize}% set the Output
% \SetKwInput{KwOutput}{Output} % set the Output
% \renewcommand\IEEEkeywordsname{Index Terms}

% On ToA-based Ranging over mmWave 5G for Indoor  Industrial IoT Networks
\usepackage[binary-units = true, range-phrase=--, per-mode=symbol]{siunitx}
\DeclareSIUnit{\decibelm}{dBm}
\DeclareSIUnit{\Joule}{Joule}
\DeclareSIUnit{\Sample}{S}
\usepackage{xcolor}

\begin{document}

%\title{Towards Deep Learning-enabled Edge Intelligence in Predicting Network Data Traffic for 6G Vision}
\title{Edge Intelligence in Softwarized 6G: Deep 
Learning-enabled Network Traffic Predictions}
\author{
    \IEEEauthorblockN{Shah~Zeb\IEEEauthorrefmark{1},
    Muhammad~Ahmad~Rathore\IEEEauthorrefmark{2},
    Aamir~Mahmood\IEEEauthorrefmark{3},  Syed~Ali~Hassan\IEEEauthorrefmark{1}, JongWon~Kim\IEEEauthorrefmark{2},\\
    and~Mikael~Gidlund\IEEEauthorrefmark{3}}
\IEEEauthorblockA{\IEEEauthorrefmark{1}School of Electrical Engineering \& Computer Science (SEECS),\\ National University of Sciences \& Technology (NUST), Pakistan.}    
\IEEEauthorblockA{\IEEEauthorrefmark{2}School of Electrical Engineering \& Computer Science,\\ Gwangju Institute of Science \& Technology (GIST), Gwangju 61005, South Korea}    
\IEEEauthorblockA{\IEEEauthorrefmark{3}Department of Information Systems \& Technology, Mid Sweden University, Sweden.}


Email: \IEEEauthorrefmark{1}\{szeb.dphd19seecs,ali.hassan\}@seecs.edu.pk,
\IEEEauthorrefmark{2}\{ahmadrathore,jongwon\}@gist.ac.kr,\IEEEauthorrefmark{3}\{firstname.lastname\}@miun.se. 
\vspace{-10pt}
}

\maketitle

\begin{abstract}
The 6G vision is envisaged to enable agile network expansion and rapid deployment of new on-demand microservices (e.g., visibility services for data traffic management, mobile edge computing services) closer to the network's edge IoT devices.
However, providing one of the critical features of network visibility services, i.e., data flow prediction in the network, is challenging at the edge devices within a dynamic cloud-native environment as the traffic flow characteristics are random and sporadic.
To provide the AI-native services for the 6G vision, we propose a novel edge-native framework to provide an intelligent prognosis technique for data traffic management in this paper.
The prognosis model uses long short-term memory (LSTM)-based encoder-decoder deep learning, which we train on real time-series multivariate data records collected from the edge $\mu$-boxes of a selected testbed network.
Our result accurately predicts the statistical characteristics of data traffic and verifies the trained model against the ground truth observations. Moreover, we validate our novel framework with two performance metrics for each feature of the multivariate data.

\end{abstract}


\begin{IEEEkeywords}
6G, network traffic flow, forecasting, deep learning, cloud-native deployments, edge computing
\end{IEEEkeywords}
\IEEEpeerreviewmaketitle

\section{Introduction}
The successful commercialization of 5G networks paves the way for discussion on the next evolution towards 6G networks and defines it's vision and requirements~\cite{saad2019vision}.
While 5G architecture is built on service-based architecture (SBA)~\cite{Aamir_IIoT}, the vision of 6G hyper-flexible architecture revolves around the state-of-the-art artificial intelligence (AI)-native design that brings the intelligent decision-making abilities in futuristic applications of digital society, such as digital twin-enabled self-evolving innovative industries~\cite{Roadmap,zeb2021industrial}. 
The Third Generation Partnership Project (3GPP) is reportedly shifting towards new AI-inspired models to monitor and enhance the SBA performance~\cite{ref}.

The critical functional component to the enhanced core network will be the seamless convergence of AI models, software-defined network (SDN) and network function virtualization (NFV)-enabled communication networks, and edge/hybrid cloud-native computing architecture~\cite{xiao2020toward,zeb2020}. Similarly, virtualization and containerization are an integral part of cloud-native computing infrastructure, which lies in the domain of software development and IT operations (DevOps)~\cite{waseem2020systematic}. By adopting DevOps-based design strategies, telecom companies can successfully provide and break down purpose-built hardware services (i.e., edge computation) and software-based solutions into real microservices (i.e., orchestrate application deployments)~\cite{kube5g}. These microservices then move towards the edge of a network to satisfy key performance indicators (KPIs), e.g., data rates, network latency, energy efficiency~\cite{SDNandCloudflow}. Collectively, these new features contribute towards the intelligence in visibility services (i.e., monitoring KPIs) to futuristic innovative network operations and management systems for accurate network traffic flows forecasting. Besides, during the network operations, the data traffic flow having a time-series (TS) data nature behaves non-linearly with aperiodic characteristics in an increasingly dynamic and complex network environment~\cite{timeseries}. Similarly, the integration of popular Internet-of-thing (IoT) technology with a de-facto cloud/edge computing model increases the importance of visibility services in inferring future traffic behavior from past traffic for providing enhanced quality-of-service (QoS) and quality-of-experience (QoE)~\cite{IoTQos}.

Despite significant potential usage of AI and other technologies, their widespread deployment is yet to be seen for predicting network traffic as there are many challenges to their comprehensive adoption in networked systems. 
% due to,

\fakepar{Insufficient resources for TS data}: Storage and computational resources needed for executing AI algorithms over network data flow near the edge network are limited and insufficient~\cite{Roadmap}. Meanwhile, as a high number of IoT devices connect to access networks, traffic volume unprecedentedly grows.
% rapidly. 

\fakepar{Need for large high-quality labeled data}: 
Machine learning (ML) algorithms need a large amount of labeled data for model training and learning, while most of the data from traffic flow at network points are unlabelled TS raw data that needs to be processed~\cite{timeseries}. Moreover, software and hardware network configurations, highly random sporadic geographic events, network infrastructure distribution, and other phenomena can lead to an abrupt change in traffic flows, affecting the quality of labeled data for the AI model construction.

\fakepar{Optimized network architecture for AI}: The design of existing network infrastructure lags behind the support for AI-inspired applications and services. Network resources can be drained off with the deployment of AI-based solutions~\cite{tang2021computing}. Therefore, the current networking infrastructure needs to evolve and adapt cloud-native AI deployment strategies to provide a balanced support for both the AI-based microservices functions and other diverse network functions.

One potential solution to approaching the preceding challenges is adopting \textit{edge intelligence or edge-native AI framework design} that integrates AI models, state-of-the-art communication networks, and cloud-native edge computing design to accurately predict traffic flows.
In this work, we propose a novel idea of an edge intelligence method for analyzing and predicting network data traffic at edge devices according to the emerging 6G vision of softwarized networks. For this purpose, we utilize the testbeds resources of cloud-native enabled \textit{OpenFlow over Trans-Eurasia Information Network (OF@TIEN++) Playground}, a multisite cloud connecting ten sites in nine Asian countries over TEIN network~\cite{rathore2020maintaining}. 
The main contributions of this paper are as follows.
\begin{itemize}
\item We present the novel method for predicting statistical characteristics of data traffic inflow at the edge devices of the cloud-native enabled network using a deep learning (DL) method.
\item For this purpose, we collect the time series raw traffic flow data at the edge of the network, which is sent to the visibility center for storage and processing.
\item We orchestrate the Kubeflow deployment at the orchestration center, which is used to develop and train the DL model for the prognosis.
\item We train the model on collected TS data and predict the statistical properties of data traffic. Moreover, we analyze the developed model's performance in terms of root-mean-square error (RMSE) and coefficient of determination (R\textsuperscript{2}) metrics.
\end{itemize}
The rest of this article is structured as follows. Section~\ref{sec:sysModel} presents the experimental model. Section~\ref{sec:performance} provides the discussion on designing Kubeflow-based deployment of a learning service. Section~\ref{sec:Results} presents the system validation and results. Finally, Section~\ref{sec:Conclusion} concludes this paper.
%%%%%%%%%%%%%%%%%%%%%%%%%%%%%%%%
\section{Experimental Model Description}
\label{sec:sysModel}
This section provides the details of designing the Kubernetes (K8s)-based edge cluster with GIST playground control center, and packet tracing and flows summarization for dataset collection.
% , Kube flow deployments, and DL framework architectures.
%%%%%%%%%%%
\subsection{OF@TEIN Playground Overview}
\label{sec:k8sedge}
The SDN-enabled multi-site clouds of OF@TEIN playground (OPG) interconnect multiple National Research and Education Network (NREN) of partner countries, enabling miniaturized academic experiments. 
Launched in 2020, it served as an \textit{Open Federated Playgrounds for AI-inspired SmartX Services} that has support for \textit{IoT-SDN/NFV-Cloud} functionalities.
Fig.~\ref{fig:PGInfratructure} shows the layered communication illustration of multi-site OPG infrastructure.
The playground (PG) within OPG supports the logical space in each centralized location/site called \textit{SmartX PG Tower} designated to develop, administer, and utilize the resources of distributed server-based hyper-converged cloud-native special boxes automatically. 
It maintains and dynamically distributes numerous physical and virtual resources for developers to execute research experiments and validate operational and development requirements in real-time.
Note that SmartX PG Tower leads the monitor and control functions of the network by installing and utilizing multiple operating centers, which are Provisioning and Orchestration (P+O), Visibility (V), and Intelligence (I).
\begin{figure}
\centering
\includegraphics[width=1\linewidth]{Mainfig1.png}
	\caption{Illustration of K8s-based edge cluster over OF@TEIN playground.}
	\label{fig:PGInfratructure}
\vspace{-10pt}	
\end{figure}
\subsection{K8s Edge Cluster over OF@TEIN Playground}
\label{sec:k8sedgecluster}
In this work, we selected the P+O center of PG Tower at the GIST site to orchestrate and deploy an AI-based learning microservice using Kubeflow to forecast traffic flows based on the accumulated multivariate processed data set. We extract processed data from the strings of measured visibility flows collected from the SmartX MicroBox ($\mu$-box) intended for the data lake storage in the visibility center. We placed these $\mu$-boxes at the multi-site edge locations of the OPG, having computing-storage-networking resources to allow IoT-Cloud-SDN/NFV functionalities-based experiments.
To provide tenacious multi-access networked connectivity in each $\mu$-box, we enabled three network interfaces, i.e., two wired network interfaces and one wireless connectivity interface. Two network interfaces are assigned public Internet Protocols (IP)-based addresses and configured as control interface and data interface. Moreover, data from the connected IoT devices or data lakes are offloaded over-the-air to $\mu$-boxes.  
Afterward, we prepared and configured each $\mu$-box as K8s-orchestrated worker nodes to support cloud-native containerized functionalities, which are provisioned and managed from K8s master in the P+O center of GIST PG. Each $\mu$-box has SDN-coordinated unique/dedicated connectivity with other boxes supporting mesh-style networking and forming K8s edge clusters over OPG networked environment. A private IP addressing scheme is employed inside each $\mu$-box, which the K8s master manages for orchestrated pods and container communication.
%%%%%%%%%%%%%%%%
\subsection{Network Traffic Data Set Collection}
\label{sec:dataset}
We use extended Berkeley Packet Filtering (eBPF)-enabled packet tracing tools such as IO Visor for measuring statistical summary of network data traffic. IO Visor-based packet tracing employs the eBPF core functionalities \cite{rathore2017comparing}, which enables in-kernel virtual machines (VMs) with byte-code tracing program execution. IO Visor has the main advantage to monitor and trace user and kernel events (through \textit{kprobe} and \textit{uprobe}), providing statistics in maps fetched on the points of interest \cite{gregg2016linux}. 
\textit{To collect network packet flows periodically}, we implement a data collection software that leverages eBPF and IO Visor to directly accumulate raw packets from the network interface of $\mu$-box and enables information compilation from each packet with a small number of CPU cycles (c.f~ Fig.~\ref{fig:AIservice}). At the same time, Apache-spark with Scala application program interface (API) is utilized to facilitate scalable, high-throughput, and fault-tolerant flows processing. The processed data flows generate five features containing a statistical summary of the network traffic packet. Later the generated multivariate data are pushed out using Spark stream and stored at MongoDB as a NoSQL database leveraging distributed messaging store of Apache Kafka.

%%%%%%%%%%%%%%%%%%%%%%%
\section{Kubeflow-based AI Service Design}
\label{sec:performance}
\begin{figure}
\centering
\includegraphics[width=1\linewidth]{Kubeflow2.png}
	\caption{Design of Kubeflow-based AI service orchestration in K8s cluster of GIST site's P+O center.}
	%Design of deploying Kubeflow-based AI service orchestration at GIST K8s cluster on P+O center.
	\label{fig:AIservice}
	\vspace{-10pt}
\end{figure}
Kubeflow consists of toolsets that enable and inscribe numerous critical stages of the ML/DL development cycle, including preparing data, model learning, experimentation and tuning, and feature extractions/transformation. Moreover, Kubeflow leverages the benefits of the K8s cluster HPC capabilities for container orchestration and auto-scaling computing resources for ML/DL jobs/pipelines. Therefore, we deploy and orchestrate the Kubeflow using K8s Master at the P+O center of GIST PG to perform the high-performance data analytics (HPDA) by leveraging K8s cluster capabilities (c.f. Fig.~\ref{fig:AIservice}). It enables the platform for accurate data traffic prediction by applying the DL algorithm on collected TS data flow, explained in the subsequent sections. 
%%%%%%%%%%%%%%%
\subsection{DL-based Data Prediction Model}
\label{sec:DLframeworks}
\begin{figure}
\centering
\includegraphics[width=0.9\linewidth]{RNN.png}
	\caption{Folded representation of RNN and an LSTM unit cell.}
	\label{fig:RNNInfratructure}
	\vspace{-10pt}
\end{figure}
Traditional neural networks (NN) cannot utilize the information learned in the previous steps (past observations) to make the spatio-temporal learning on TS data and accurately predict traffic features.
Numerous recurrent NN (RNN) algorithms are developed for the prediction problems based on the unit RNN cell architecture (c.f.~Fig.~\ref{fig:RNNInfratructure}) due to their natural interpretation property of TS data analysis. These RNN algorithms allow data information to persists by connecting the previous informational state to the present task as an input. However, their performance suffers from the constraints in learning long-term dependencies/correlations of the TS data because of the vanishing gradient problem. In this paper, we use a sequence-to-sequence (s2s) deep learning model to predict based on long short-term memory (LSTM) neuron cells. In the following subsections, we first explain the LSTM cell and then discuss the encoder-decoder architecture for prediction based on the LSTM cell. 
\subsubsection{LSTM Cell}
%\fakepar{LSTM Cell}:
An LSTM cell overcomes the RNNs vanishing gradient problem using back-propagation algorithms over time~\cite{LSTM}. The error derivatives for learning newly updated weights do not quickly vanish as they are distributed over sums and sent back in time, enabling LSTM units to learn and discover long-term correlated features over lengthy sequences in input multi-variate data.
As shown in Fig.~\ref{fig:RNNInfratructure}, an LSTM cell receives an input sequence vector $\mathbf{X}_t$ at current time $t$ which, together with the previous cell state $\mathbf{c}_t$ and hidden state $\mathbf{h}_t$, is used to trigger the different three gates by utilizing their activation processing units. Note that onward in the study, bold variable notation denotes a vector. A cell state of the LSTM unit can be considered as a memory unit, and its state can be read and modified through connected three gates. 
These LSTM unit gates are, 1) \textit{forget gate} ($\mathbf{f}_t$), which sets and decides what information to discard based on assigned condition, 2) \textit{input gate} ($\mathbf{i}_t$), which updates the memory cell state based on assigned conditions, and 3) \textit{output gate} ($\mathbf{o}_t$), which sets the output depending upon the input sequence and cell state with assigned conditions. 
The gates and cell updates of the LSTM unit at time $t$ can be formulated as 
\begin{equation*}\label{eq1}
{\mathbf{c}_t} = {\mathbf{f}_t} \odot {\mathbf{c}_{t - 1}} + {\mathbf{i}_t}\odot\tanh \left( {{\mathbf{w}_{hc}}\odot{\mathbf{h}_{t - 1}} + {\mathbf{w}_{xc}}\odot{\mathbf{X}_t} + {\mathbf{b}_c}} \right),
\end{equation*}
\begin{equation*}\label{eq2}
{\mathbf{i}_t} = \sigma \left( {{\mathbf{w}_{hi}}\odot{\mathbf{h}_{t - 1}} + {\mathbf{w}_{ci}}\odot{\mathbf{c}_{t - 1}} + {\mathbf{w}_{xi}}\odot{\mathbf{X}_t} + {\mathbf{b}_i}} \right),
\end{equation*}
\begin{equation*}\label{eq3}
{\mathbf{f}_t} = \sigma \left( {{\mathbf{w}_{hf}}\odot{\mathbf{h}_{t - 1}} + {\mathbf{w}_{cf}}\odot{\mathbf{c}_{t - 1}} + {\mathbf{w}_{xf}}\odot{\mathbf{X}_t} + {\mathbf{b}_f}} \right),
\end{equation*}
\begin{equation*}\label{eq4}
{\mathbf{o}_t} = \sigma \left( {{\mathbf{w}_{ho}}\odot{\mathbf{h}_{t - 1}} + {\mathbf{w}_{co}}\odot{\mathbf{c}_t} + {\mathbf{w}_{xo}}\odot{\mathbf{X}_t} + {\mathbf{b}_o}} \right),
\end{equation*}
\begin{equation*}\label{eq5}
{\mathbf{h}_t} = {\mathbf{o}_t}\tanh ({\mathbf{c}_t}),
\end{equation*}
where $\odot$ is the element-wise multiplication. Please note that each selected gate has a distinctly associated weight vector $\mathbf{w}$, and a bias vector $\mathbf{b}$, that are learned throughout the change of state and new information addition during processing in the training phase.
Moreover, each LSTM gate uses specific activation function (c.f.~Fig.~\ref{fig:RNNInfratructure}) for processing, e.g., sigmoid ($\sigma$) or hyperbolic tangent ($\mathrm{\tanh}$)~\cite[Sec.~3]{zhengtime}.
\begin{figure}
\centering
\includegraphics[width=0.82\linewidth]{lstm2.png}
	\caption{The schematic illustration shows the proposed DL-based prediction model with encoder-decoder architecture.}
	\label{fig:s2s}
	\vspace{-10pt}
\end{figure}
\subsubsection{LSTM cell-based Encoder-Decoder Model}
%\fakepar{LSTM based Encoder-Decoder Model}:
We used the single layer encoder-decoder architecture that employs the LSTM cells (c.f.~Fig.~\ref{fig:s2s}) in each layer to predict the vector set of output sequences of data traffic, $\mathcal{Y}_o=\{\mathbf{Y}_{t+1},\mathbf{Y}_{t+2},...,\mathbf{Y}_{t+f}\}$, based on the set of TS input sequences, $\mathcal{X}_i=\{\mathbf{X}_{t-T},\mathbf{X}_{t-T-1},...,\mathbf{X}_{t-1},\mathbf{X}_{t}\}$, which represents all the past observations of collected traffic. Note that $\mathbf{X}_{t}=\{x_{t,1}, x_{t,2},...,x_{t,m}\}$ represents the current time-dependent sequence vector containing the observation values for $m$ number of multivariate traffic features, $T$ is the lookback length of time, $f$ is the horizon window of future prediction and, $t,m,f,T \in \mathbb{N}$. 
The encoder produces the encoded temporal representation of current information sequences $\mathbf{X}_{t}$ through LSTM units in a single layer. The encoded output sequence vector is provided to the LSTM decoder through a repeat vector. Similarly, the encoder status of LSTM units is also simultaneously passed to the decoder units. Then, the decoder uses the cell state of the repeat vector as the initial temporal representation to reconstruct network data's target output, i.e., feature prediction. 
Now the final objective of the prediction model training problem is minimizing the output error $e_i$ for training samples in each $t$-th current time to find the optimized parameter space, $\Theta$, as,
\begin{equation}\label{eq6}
\mathop {\arg \min \: {e_i}}\limits_\Theta   = \sum\limits_{i = 1}^T {\sum\limits_{j = 1}^m {L_i^j} } \:,
\end{equation}
where $L_i^j$ is the selected Huber loss function for this study, and given as
\begin{equation}\label{eq7}
L_i^j\!\left( {X_i^j,Y_i^j} \right) \!\! = \!\! \left\{\!\! {\begin{array}{*{20}{c}}
{\frac{1}{2}{{\left( {X_i^j - Y_i^j} \right)}^2},}&{\forall \left| {X_i^j - Y_i^j} \right| \le \tau ,}\\
{\tau \left( {\left| {X_i^j - Y_i^j} \right| - \frac{1}{2}\tau } \right),}&{\mathrm{otherwise}.}
\end{array}} \right.
\end{equation}
In \eqref{eq6} and \eqref{eq7}, $\tau$ is the hyperparameter cut-off threshold to switch between two error functions (squared loss and absolute loss) which is 1 in our study,  $i,j \in \mathbb{N}$ represents the $j$-th feature observation value at each time-step $i$ in the input/predicted sequence of data, and $\Theta$ comprises of learned weights $\mathbf{w}$ and biases vectors $\mathbf{b}$ at each time-step. 
%%%%%%%%%%%%%%%%%%%%%%%%%%%%%%%%%%%%%%%%%%%%%%%%%%%%%%
\section{Experimental Results Analysis}
\label{sec:Results}
{\renewcommand{\arraystretch}{1.1}
\begin{table}[t!]
\centering
	\caption{Device Specifications of Control Tower and $\mu$-boxes} 
	\scalebox{0.92}{
\begin{tabular}{|c|c|}
\hline
 \textbf{Device Type} & \textbf{Specifications} \\ \hline
 \begin{tabular}[c]{@{}c@{}} Visibility\\ Center\end{tabular} &
 \begin{tabular}[c]{@{}c@{}}Ubuntu 16.04.4 LTS OS, Intel Xeon{\textsuperscript \textregistered}CPU E5-2690 \\ V2@3.00GHz, 12x8 GB DDR3 Memory, 5.5 TB \\HDD, 4 network interfaces of 1 Gbits/sec (Gbps)  \end{tabular}  \\ \hline
 
   \begin{tabular}[c]{@{}c@{}} Provisioning \& \\ Orchestration \\ Center\end{tabular}  &  \begin{tabular}[c]{@{}c@{}}Ubuntu 18.04.2 LTS OS,SYS-E200-8D SuperServer\\ with Intel Xeon{\textsuperscript \textregistered} D-1528 with 6 Cores @1.90 GHz,\\ 32 GB RAM, 480 GB Intel SSD DC S3500   \end{tabular} \\ \hline
   
   \begin{tabular}[c]{@{}c@{}} Intelligence \\ Center\end{tabular}  &  \begin{tabular}[c]{@{}c@{}}Ubuntu 18.04.2 LTS OS, Intel Xeon{\textsuperscript \textregistered} Scalable 5118 \\ 12x2 Cores @2.3GHz, Samsung 256 GB DDR4 RAM,\\ 512x2 GB and 1.6x4 TB SSD, Mellanox 100G SmartX \\ NIC (2 ports), 16x6 GB Tesla T4 Nvidia{\textsuperscript \textregistered} GPUs  \end{tabular} \\ \hline
 
   Edge $\mu$-box  & \begin{tabular}[c]{@{}c@{}}Ubuntu 18.04.2 LTS OS, Supermicro SuperServer \\E300-8D (Mini-1U Server) with 4 @2.2GHz Intel\\Cores, 32 GB Memory, 240 GB HDD, 2x10 Gbps  \\+ 6x1 Gbps network interfaces   \end{tabular}  \\ \hline

\end{tabular}}
\label{specification}
\vspace{-10pt}
\end{table}
}
In this section, we discuss the methodology and experimental results obtained on the real network traffic dataset collected from the edge $\mu$-boxes to analyze and evaluate the proposed prediction model and validate its effectiveness in terms of two performance metrics (RMSE and R\textsuperscript{2}).

We collected multivariate network flow data at the visibility center for one month with an interval gap of 5-minutes using eBPF-based packet tracing software (Sec.~\ref{sec:dataset}). The collected dataset comprised 43000 time-series records with five features representing statistical features of data flow, divided into training (65\%) and test/validation (35\%). To train the prediction model, we deployed the Kubeflow and trained the LSTM-based encoder-decoder model on Jupyter Notebook with DL libraries (Tensorflow \& Keras) and Scikit-learn library on the accessed dashboard of deployed Kubeflow (c.f.~Fig.~\ref{fig:AIservice}). This setup enabled us to run the DL workloads on a fully automated and scalable cloud-native environment.  GPU resources of Intelligence Centers are used to run the deep learning jobs. Table.~\ref{specification} shows the hardware specifications of used resources for this experiment. 
\begin{figure}
\centering
\includegraphics[width=0.85\linewidth]{lossfigure1.pdf}
\vspace{-7pt}
	\caption{Training loss and validation loss curves for our proposed DL-based prediction model.}
	\label{fig:loss}
	\vspace{-10pt}
\end{figure}
\begin{figure}
\centering
\includegraphics[width=0.85\linewidth]{averagedatabytes.pdf}
\vspace{-7pt}
	\caption{Average databytes in data traffic predicted against the collected groundtruth observations.}
	\label{fig:average}
	\vspace{-10pt}
\end{figure}
A single-layer encoder and decoder architecture with time distributed layer is implemented with 100 LSTM cells in each layer. We applied the min-max function in the Scikit-learn library to normalize the value of the observed features in the range of $[-1,1]$. We trained the model on the past 20 hours of observation to predict the next 10 hours of output samples, compared with the groundtruth observations for model validation. Hyperparameters like batch size and epoch time for learning are kept fixed at 32 and 40, respectively. We selected the Adam optimizer to learn the optimized parameters while minimizing the $L_i^j$ in \eqref{eq7} during the training process. We applied a callback utility function "Learning Rate Schedule" to obtain the updated learning rate value through training from the defined range of $[1e^{-3},0.90\,e^{\textrm{epoch}}]$. It uses the updated learning rate on the Adam optimizer with the current epoch and current learning rate. Fig.~\ref{fig:loss} shows the trend of loss function during the training and validation stage of prediction mode against the epoch time. It reveals that beyond the epoch interval of 20, the validation and training loss is comparatively low and converging to avoid overfitting or underfitting in the prediction model.  
\begin{figure}
\centering
\includegraphics[width=0.85\linewidth]{mindatabytes.pdf}
\vspace{-7pt}
	\caption{Predicted minimum databytes in data traffic against the collected groundtruth observations.}
	\label{fig:minimum}
	\vspace{-10pt}
\end{figure}
\begin{figure}
\centering
\includegraphics[width=0.85\linewidth]{stddatabytes.pdf}
\vspace{-7pt}
	\caption{Predicted standard deviation in databytes for data traffic predicted against the collected groundtruth observations.}
	\label{fig:std}
	\vspace{-10pt}
\end{figure}

The trained prediction model predicts the future samples of each statistical feature of network packet traffic in the selected future horizon window of 10 hours. Each predicted feature sample is verified against the ground truth observation. 
Fig.~\ref{fig:average} presents the statistics of average (mean) databytes recorded in the network flow at the edge $\mu$-boxes against the predicted average databytes by our learning model. Similarly, Fig.~\ref{fig:minimum} and Fig.~\ref{fig:std} show the predicted statistics of minimum (min) and standard deviation (std) in observed databytes of the network packet flow against the ground truth observations. 
These results show that our learning model can accurately learn and predict the future trend in mean, min, and std statistics of databytes at $\mu$-boxes over time and matches the recorded ground truth observations. 

Lastly, Fig.~\ref{fig:total} shows the predicted total traffic of the packet flows at the edge devices at the edge layer against the observed ground truth. It learns the total databytes statistics trend, which depicts the network traffic load over a specific period of time. Note that the statistics of avg, std, min, and maximum (max) databytes are recorded in kilobytes (KB) while total databytes are in megabytes (MB) units. Collectively predicting these five features characterizes the network traffic trend and accurately gives insight into the traffic statistics at the $\mu$-boxes of the edge layer. 
\begin{figure}
\centering
\includegraphics[width=0.85\linewidth]{totaldataytes.pdf}
\vspace{-10pt}
	\caption{Total databytes in data traffic predicted against the collected groundtruth observations.}
	\label{fig:total}
	\vspace{-12pt}
\end{figure}

To analyze and validate the deviations between the learned model prediction samples and ground truth observations, we evaluated two performance metrics, RMSE and R\textsuperscript{2}, of each feature of data traffic. RMSE can take values from a range of $[0,\infty]$, while R2 takes a value in the range of $[0,1]$.
Table.~\ref{performancemetrics} shows the performance metrics for each data feature, showing that most of the features RMSE is closer to zero while R2 values are closer to 1. Low values of RMSE and R2 value closer to 1 imply that the learned model can accurately predict the five multivariate statistical features of data traffic. It can be observed that the performance of our model in predicting avg, min, max, and std statistics is better than total bytes as RMSE and R\textsuperscript{2} score of four features is reasonably good compared to the total databytes. 
\vspace{-5pt}

%%%%%%%%%%%%%%%%%%%%%%%%%%%%%%%%%%%%%%%%%%%%%%%%%%%%%%
%\newpage
\section{Conclusion}
\label{sec:Conclusion}
Emerging computing techniques, AI, and state-of-the-art communication network enablers (SDN/NFV) are critical parts of the 6G vision, increasing the importance of intelligent network management. 
We proposed a DL-based novel intelligent prognosis technique for predicting statistical properties of data traffic incurred at the edge devices of the network. 
For this purpose, we captured, collected, and pre-processed the live TS traffic data from the edge $\mu$-boxes using the testbed network resources at GIST PG.
We orchestrated the Kubeflow deployment using K8s master at the orchestration center to train the LSTM-based seq2seq DL model on the collected TS data. We predicted various features of data traffic based on past observation into the future horizon window of 10 hours. We evaluated the predicted future observations with ground truth observation in terms of RMSE and R\textsuperscript{2}. Results showed that our model accurately predicts the future observations of all features.
For future work, network resource automation and scaling based on predicted traffic can be explored. 
%\vspace{-5pt}
%\section*{Acknowledgment}
%This publication has been produced with co-funding of the European Union (EU) for the Asi@Connect Project under Grant contract ACA 2016-376-562. The contents of this documents are the sole responsibility of GIST and can under no circumstances be regarded as reflecting the position of the EU.
%\vspace{-5pt}
{\renewcommand{\arraystretch}{1.3}
\begin{table}[t!]
\centering
	\caption{Prediction Performance for various features of Data Traffic} 
\scalebox{0.9}{	
	\begin{tabular}{|c|c|c|c|c|c|}
\hline
\textbf{\begin{tabular}[c]{@{}c@{}}Data \\ Feature\end{tabular}} & \textbf{\begin{tabular}[c]{@{}c@{}}Average\\ Databytes\end{tabular}} & \textbf{\begin{tabular}[c]{@{}c@{}}Min\\ Databytes\end{tabular}} & \textbf{\begin{tabular}[c]{@{}c@{}}Std\\ Databytes\end{tabular}} & \textbf{\begin{tabular}[c]{@{}c@{}}Total\\ Databytes\end{tabular}} & \textbf{\begin{tabular}[c]{@{}c@{}}Max\\ Databytes\end{tabular}}  \\ \hline
RMSE  &   5.33  & 8.63  &  6.03  & 231.64 & 33.12  \\ \hline
R\textsuperscript{2}   &  0.968    &   0.909   &   0.954  & 0.686 & 0.946   \\ \hline
\end{tabular}
}
  \label{performancemetrics}
  \vspace{-15pt}
\end{table} 
}
%\bibliographystyle{IEEEtran}
%\bibliography{biblio}
\documentclass[10pt,twocolumn]{article} 
\usepackage{simpleConference}
\usepackage{times}
\usepackage{graphicx}
\usepackage{amssymb}
\usepackage{url,hyperref}
\usepackage{caption}
\usepackage{subcaption}

\begin{document}

\title{EFFICIENTWORD-NET: AN OPEN SOURCE HOTWORD DETECTION ENGINE BASED ON ONE-SHOT LEARNING}

\author{Chidhambararajan R, Aman Rangapur, Dr. Sibi Chakkravarthy \\
\\
Vellore Institute of Technology-AP, India \\
%  \\
% \today
\\
%  \\
}


% }

\maketitle
\thispagestyle{empty}

\begin{abstract}
Voice assistants like Siri, Google Assistant, Alexa etc. are used widely across the globe for home automation, these require the use of special phrases also known as hotwords to wake it up and perform an action like "Hey Alexa!", "Ok, Google!", "Hey Siri!" etc. These hotwords are detected with lightweight real-time engines whose purpose is to detect the hotwords uttered by the user. This paper presents the design and implementation of a lightweight, easy-to-implement hotword detection engine based on one-shot learning which detects the hotword uttered by the user in real-time with just one or few training samples of the hotword. This approach is efficient compared to existing implementations because the process of adding a new hotword in the existing systems requires enormous amounts of positive and negative training samples and the model needs to retrain for every hotword. This makes the existing implementations inefficient in terms of computation and cost. The architecture proposed in this paper has achieved an accuracy of 94.51\%.
\end{abstract}


\Section{INTRODUCTION}
With the advent of the Internet of Things (IoT) and home automation, there is a growing need for voice automation in edge devices, but running a heavy Text To Speech (TTS) Engine is too computationally expensive in these edge devices \cite{amodei2015deep}. Instead, we can run engines that need to listen for specific activation phrases called "Hotwords" to perform certain actions since the detection of hotwords is computationally less expensive than full-blown TTS engines \cite{Yang_Jee_Leblanc_Weaver_Armand_2020}.

Lightweight models \cite{Yang_Jee_Leblanc_Weaver_Armand_2020} are trained for detecting these hotwords from audio streams. This is used to save resources from heavy models such as speech recognition from running all day. The Core application for hotword detection is shown in Fig. \ref{FIG:Illustration_of_the_proposed_architecture}.


\begin{figure}
\centering
\includegraphics[width=0.45\textwidth]{images/Fig1_Illustration_proposed_architecture}
\caption{Overview of the proposed model.}
\label{FIG:Illustration_of_the_proposed_architecture}
\end{figure}

Convolutional Neural Networks (CNNs) have proven to be the best in analysing image data \cite{Hershey_2017}. Audio files are converted into Log Mel spectrograms where various frequencies are distributed on the Mel scale and plotted as an image \cite{Atsavasirilert_2019,9413528}. This image data is further analysed by the CNNs to get maximum optimacy \cite{Hershey_2017}. The base network is further attached to a Siamese network which learns to output embedding vectors with less distance for similar hotwords and huge distance for dissimilar hotwords. This way a state of the art accuracy is achieved for hotword detection with fewer audio examples of the hotwords. To our knowledge, this is the first attempt to solve the problem of retraining hotwords with one shot / few-shot learning. This approach is highly inspired by Face-Net \cite{Schroff_2015}, one-shot learning deployed for face recognition, which allows us to add a new face to the system without retraining the model \cite{Schroff_2015}. Except for the well-known hotword detection engines with low accuracy, other engines require huge datasets with positive and negative samples for training new hotword and all of these are closed source \cite{Yang_Jee_Leblanc_Weaver_Armand_2020,kalith_2012}. 

EfficientNet \cite{tan2020efficientnet} is one of the most efficient CNN architecture to the date, and the first four blocks of EfficientNet (B0 variant) is chosen for the base model in the Siamese network. To train the Siamese network, positive and negative pairs of audios are given to the network and trained to output 1, 0 for positive pairs and negative pairs respectively to know how similar the pronunciations of words. In the edge device, raw audio is continuously read(with 1 sec time window), converted to Log Mel spectrogram, from which real-time vector embeddings are calculated, these embeddings are compared against a pre-calculated vector embedding of the desired hotword for similarity \cite{Vargas_2020}.


\Section{EXISTING RELATED WORKS}
\label{Sec:Related_Works}
The problem of detecting hotwords from audio streams started ever since the advent of voice-enabled IoT devices \cite{Ooi_2019,Michaely_2017, inproceedings2_Reis_2018,inproceedings3_Todisco_2019,Tom_2018,He_Kaiming_2016,He_2016,Lin_2018,becker2019interpreting,Uitdenbogerd_2004,Tang_2018}. Porcupine \cite{picovoice_alireza} is a closed source hotword detection framework which detects hotwords but requires a commercial licence. It has an accuracy of 94\%. A customizable hotword detection engine called Snowboy can be used to create your own hotwords \cite{Yang_Jee_Leblanc_Weaver_Armand_2020}. It is a closed source project and requires a huge amount of data samples. PocketSphinx \cite{kalith_2012} is a lightweight variation of the CMU-Sphinx \cite{kalith_2012}, an offline Speech-to-Text (STT) engine with low accuracy. 

There are other frameworks such as howl \cite{tang-etal-2020-howl}, these frameworks require large amounts of positive and negative samples of the hotword to train and recognize the new words. Snowboy \cite{chen_yao} and howl’s \cite{tang-etal-2020-howl} accuracy depends on the size of training dataset for each hotword. STT engines can also be used to detect hotwords \cite{amodei2015deep}, existing engines STT \cite{Kubota_patent_2014} have very high accuracy but require a constant internet connection and an expensive subscription to their cloud service to run 24/7. Offline STT open source engines like DeepSpeech (from Mozilla) \cite{amodei2015deep} and Silero \cite{Silero_Models} have good accuracy, yet require lots of on-device resources, hence cannot be run 24/7. Rhino \cite{kenarsari_2018}, an on-device STT engine achieved the best among existing engines, by giving better accuracy and low resource requirement but closed source and requires a commercial license.

Most existing audio processing neural networks employ the usage of Log Mel Spectrograms \cite{Atsavasirilert_2019} and Mel Frequency Cepstral Coefficient \cite{Atsavasirilert_2019} since it conveys a better picture of the audio than conventional audio stream bits, this is due to the representation of changes for various frequencies across the audio streams. The initial development of hotword detection was achieved without noise \cite{Huang_2019} and then deployed with active noise cancellation in real-time \cite{Huang_Yiteng_Wan_Li_2019}. Active noise cancellation is often achieved with hardware first or software first or hybrid approaches. The software first approaches require audio samples recorded from the ambient space. Out of these recorded sample metrics, the maximum amplitude for noise is calculated and used as a threshold for voice activity detection. Acoustic noise reduction is applied over real-time audio with the help of obtained noise only audio samples. This approach is not practical since noise only audio samples recorded from ambient space are required. To circumvent this issue, edge voice assistant’s like Google, Amazon’s Alexa often deploy multi-microphone array systems \cite{Huang_Yiteng_Wan_Li_2019} to gather audio from all directions and separate speech audio with ease. This is a relatively simpler task as devices are surrounded with uniform noise in all directions but speech audio is not uniform in all directions. This allows speech audio to be separated from noise audio. The separated speech audio is of high quality thereby helping in achieving low False Acceptance Rates (FAR).

In a hybrid approach, hardware-based noise cancellation is utilized and the audio processing neural networks are often trained with noise, this allows the system to achieve exceptionally low FAR \cite{Huang_Yiteng_Wan_Li_2019}.

Existing audio processing neural networks are designed as audio classification neural networks \cite{Hershey_2017}. The network needs a lot of initial layers to understand the audio fragment. Later, half a dozen of layers are required for the network to understand the logic of classification. These extra layers dedicated for classification requires additional computational time thereby crippling the model while running it on real-time edge devices. In this paper, this problem was resolved using EfficientNetB0 as a base network with one-shot learning. In one-shot learning, the network requires a lot of layers to understand audio samples, but the need for additional layers to understand classification is waived off, therefore, resulting in faster inference in the edge devices.

As mentioned, the existing lightweight models \cite{Yang_Jee_Leblanc_Weaver_Armand_2020,kalith_2012} are binary classifiers that need to be retrained with a huge number of negative and positive samples for a new hotword, this results in expensive and inefficient gathering of datasets \cite{kalith_2012}. These networks needs to be retrained for newer hotwords again. Also, these engines are closed source where there is no scope of development in the future and users need to spend lots of money for the useage. Hardware first and a hybrid approach are applicable in the scenarios where the edge device’s hardware specifications are under the control of developers like edge voice assistants like Alexa \cite{Huang_2019}. But this approach doesn’t work well when the edge device is not designed by developers \cite{Huang_Yiteng_Wan_Li_2019}. Hence, there is a need for audio processing neural networks to be trained with very high amounts of noise to work robustly without need of hardware interventions.

EfficientWord-Net is an open source engine that solves the process of retraining the model for new hotwords by eliminating the requirement of huge datasets. It works efficiently with the audio samples with decent noise added in the background with a great inference time on small devices like Raspberry Pi. Moreover, our system outperformed previous existing approaches in terms of accuracy and inference.

\Section{ONE-SHOT LEARNING/SIMILARITY LEARNING}
\label{Sec:oneshotlearning}
One of the demanding situations of face recognition/hotword recognition is to achieve performance with fewer samples of the target, which means, for maximum face recognition programs, model should recognize a person given with the aid of using one photo of the man or woman's face. Traditionally deep learning algorithms do not work well with the simplest one training example or one data point for a class. In one-shot learning, a model learns from one sample to apprehend or recognize the person, and the industry needs most face recognition models to use this due to the fact a company has one or few images of every of their personnel in the database. 

% One method is to enter the photo of that person, feed it to a CNN and feature an output label using a SoftMax unit with 5 outputs corresponding to 4 persons or none of them. But this method won’t be efficient as you have got only one image and having such a small training dataset is inadequate to train a robust neural network for this particular task. This causes more trouble when a new person joins your team, as an extra person to apprehend with six outputs. And the model needs to be retrained to achieve the confidence every time, which is not a great approach \cite{Yang_Jee_Leblanc_Weaver_Armand_2020,kalith_2012, inproceedings4_Platen_2020}.

Similarity learning is a type of supervised learning where a network is trained to identify the similarity between 2 data points of the same class instead of classification or regression. The network is also trained to learn the dissimilarity between 2 data points of different sources. This similarity is used to determine whether an unlabelled data point belongs to the same class or not.

When 2 images are fed to a neural network to learn the similarity between them (inputs two images and outputs the degree of difference between the two images), the output would be a small number if the two images are of the same person. And if they are of two different people, the output would be a large number. However, the use of different types of functions keeps the value between 0 and 1. A hyper parameter($\tau$) is used as a threshold if the degree of difference is less than the threshold value, these pictures belong to the same person and vice versa. Similarly, in this paper, a threshold value of 0.2 is defined, if the degree of difference between the two hotwords is less than 0.2, those two hotwords are the same, else they are different.
\\

d\emph{(hotword1, hotword2)} $=$ degree of difference between hotwords.

d\emph{(hotword1, hotword2)}  $ < \tau$, both hotwords are same.

d\emph{(hotword1, hotword2)}  $ > \tau$, hotwords are different. 
\\


This approach of feeding 2 different images to the same convolutional network and comparing the encodings of them is called a Siamese Network \cite{Vargas_2020}.

\begin{figure} 
\centering
	\includegraphics[width=0.5\textwidth]{images/Fig2_One-shot_learning_architecture-hotword_detection}
\caption{One-shot learning architecture for hotword detection}
\label{FIG:OneshotLearningArchitecture}
\end{figure}

In Figure. \ref{FIG:OneshotLearningArchitecture}, two raw audio segments are fed to the same neural network, output encodings of input audios are captured, Euclidean distance between these two vectors are calculated, if the Euclidean distance is less than the threshold value, the hotwords in the audio are same and vice versa.

\Section{PROPOSED METHOD}
\label{Sec:efficientword-net}
\subsection{Preparation Of Dataset}
The dataset used for training the network is homegrown artificially synthesized data that was made with naturally sounding neural voices from Azure Cloud Platform and Siri. Furthermore, the voices were selected to include all available voice accents with the respective countries to ensure better performance across accents and gender. Hotword detection does not rely on a word’s meaning but only on its pronunciation. So, to generate the audio, a pool of words is selected in which each word sounds unique compared to the other words in the pool. Google’s 10000-word list \cite{kaufman_bathman_myers_hingston} is used while training the model, and these words were converted to respective phoneme sequences. The sequences were checked for similarity, the words which shared $\ge$80\% of the same phoneme sequence were removed to ensure low similarity in pronunciation among the word pool. The word pool was converted to the audio pool with the above mentioned text-to-speech services. The generated audio is combined with noises such as traffic sounds, market place sounds, office and room ambient sounds to simulate a naturally collected dataset. For each word, 5 such audio samples were generated in which each sample had a randomly chosen voice and randomly chosen background noise to ensure variety. The randomly chosen noises are then combined with the original audio with a noise factor randomly chosen between 0.05 to 0.2 (noise factor is a fraction of noise’s volume in the resulting audio). A sampling rate of 16000 Hz was chosen since audio quality below 16000 Hz became very poor. Finally, the audios are converted to Log Mel Spectrograms.

Two audios generated from the same word are chosen to make the true pair and two audios generated from two different words are chosen to make the false pair. The total number of true and false pairs generated in this method were 2694. 80\% of the data was split for training, and remaining 20\% was used as testing data to eliminate over-fitting. The network is fed with the true pair and false pairs to output a higher similarity score for true pairs and lower similarity score for false pairs.

\subsection{Network Architecture}
The input for each base network is 98x64x1 (Refer Fig. \ref {FIG:Base_network_Block}. The base network of the model is made up of the first four blocks of EfficientNetB0 architecture \cite{tan2020efficientnet}. The output from the EfficientNet layers is processed further with Conv2D layer with 32 filters and 3 stride values, processed by batch normalization and max pool layer, this is fed to a similar stack of layers. This is done to reduce the number of feature points efficiently. Finally, the output from the convolutions is flattened and attached to the dense layer with 256 units followed by L2 regularization to give the reduced vector approximation of the input.


\begin{figure}
\centering
\includegraphics[width=0.5\textwidth]{images/Fig3-Base_network_arch}
\caption{Base network Block.}
\label{FIG:Base_network_Block}
\end{figure}


A true pair or a false pair in the dataset is fed to two parallel blocks of the base network, where these parallel blocks share the weights. Euclidean distance between corresponding output vector pairs is calculated. Table. \ref{TBL:Euclidean_Distance_and_Similarity_score} shows the Euclidean distance mapped to similarity score to the scale 0-1.0. % as shown in Fig. \label{FIG:graph_similarity_score}. 

\begin{table}
\centering
\caption{Euclidean Distance and Similarity score}
\label{TBL:Euclidean_Distance_and_Similarity_score}
\begin{tabular}{ p{2cm} p{2cm}   p{2cm}}
 \hline \hline
Euclidean Distance  &Similarity Score \\
 \hline
0   &1.0 \\
 $<\tau$  &1.0 - 0.5 \\
$\tau$ &0.5 \\
 $>\tau$	&0.5 - 0 \\
\hline
\end{tabular}
\end{table}%


\begin{figure}
\centering
\includegraphics[width=0.5\textwidth]{images/Fig5-Model_Scaling_in_EfficientNet}
\caption{Model Scaling in EfficientNet.}
\label{FIG:Model_Scaling_in_EfficientNet}
\end{figure}

\subsection{Training Parameters and Loss Function}
The problem of analysing raw audio to examine images is resolved by converting the audio into Log Mel spectrograms images, this helps the neural network by allowing it to directly analyse frequency distribution over time, learning first to identify different frequencies and analyse them.

These generated images were fed to a convolution neural network that follows the EfficientNet Architecture (Refer Fig. \ref{FIG:Model_Scaling_in_EfficientNet}). It is a convolutional neural network architecture and a scaling method that uniformly scales all dimensions of depth/width/resolution using a compound coefficient. 

This network architecture was chosen since the accuracy was similar to that of ResNet which held the previous state of the art top5 accuracy of 94.51\%. Moreover, the number of parameters in ResNet was 4.9x higher than EfficientNetB0’s parameter count, thereby making it computationally more efficient than ResNet. Only 4 blocks of EfficientNetB0 were taken for the base network, the output was further attached to a Conv2D block, which was later flattened and l2 normalized to give the output vector.

Conventional Siamese neural networks use triplet loss where a baseline (anchor) input is compared to a positive(true) input and a negative(false) input with vector distance calculation metrics such as Euclidean distance, cosine distance, etc.
\\


Triplet Loss Function =
\(\max\left( {\| f\left( x^{a} \right) - f\left( x^{p} \right) \|}^{2} - {\| f\left( x^{a} \right) - f\left( x^{n} \right) \|}^{2},0 \right)\)
\\

\(x^{a}\) is an~\emph{anchor}~example.

\(x^{p}\) is a~\emph{positive}~example that has the same identity as the anchor.

\(x^{n}\) is a~\emph{negative}~example that represents a different
entity.
\\


The triplet loss function makes the neural network minimize the distance from the baseline(anchor) input to the positive (true pair), and maximize the distance from the baseline(anchor) input to the negative (false pair). It is also equipped with a threshold, which forces the network to reduce the distance between true pairs below the threshold and false pairs beyond the threshold.

The calculated distance between the pair is sent through a function F(x) which outputs close to 1 when distance is low and 0 when distance is high, thereby making the function give a score close to 1 for similar pairs and close to 0 for dissimilar pairs. This was done so that the network will be able to tell similarities between a pair of samples in terms of percentage. 
\[F\left( x \right) = 1 - \frac{x^{4}}{({\tau}^{4}+x^{4})}\]


Here x is the calculated distance between the vectors, this function gives 1 when x is 0, gradually reduces to 0.5 when x = $\tau$ (threshold,t[in the above equation]) and eventually to zero. The function is symmetric making f(x) go to zero when x\textless0, with Euclidean distances $\ge$ 0.

For each true pair, the ground truth was set 1 and for each false pair the ground truth was set to 0, this allowed us to treat the problem as a binary classification problem and easily apply binary cross-entropy loss function while training the engine.
\\


\textbf{Binary cross-entropy loss is defined by}

\[loss = - \frac{1}{N}\sum_{i = 1}^{N}y_{i} \bullet log\left( p\left( y_{i} \right) \right) + \left( 1 - y_{i} \right) \bullet log\left( 1 - p\left( y_{i} \right) \right)\]


\subsection{Optimization Strategies}

\subsubsection{Log Mel Spectrogram:~}

Many audio processing neural networks directly process the audio with several Conv1D layers stacked on top of each other to make the network understand the audio and it has a drawback. The network would first have to understand the concept of frequency and check for the distribution of various frequencies across the audio. This forces the network to allocate its initial Conv1D layers to understand the concept of frequency and learn to check for the distribution of various frequencies across the audio, which would be further analysed by the next layers to make sense of the audio. These additional preprocessing layers can be skipped by going for a Log Mel Spectrogram (a heatmap distributing various frequencies across the audio).

Since the network is being directly fed with the distribution of various frequencies across the audio, the network can allocate all of its resources to make sense of the audio directly, thereby enhancing the accuracy of the system. This method is largely inspired by Google's TensorFlow magenta (a set of audio processing tools for TensorFlow).

\begin{figure}
\centering
\includegraphics[width=0.5\textwidth]{images/Fig7-Sample_audio_converted_to_Log_Mel_Spectrogram}
\caption{Sample audio converted to Log Mel Spectrogram.}
\label{FIG:SampleaudioconvertedtoLogMelSpectrogram}
\end{figure}

\subsubsection{L2 Regularization to an output vector of base network:}

Initially, the network was trained with no L2 regularization, the accuracy was never able to cross 89.9\%, we have tried various techniques out of which adding L2 regularization to the output vector of the base network gave the best performance. With that added, our network was able to reach 95\% accuracy with noise. This performance increase was due to the fact that L2 regularization confined the output vector to the surface of a 256-dimensional unit sphere at the origin; without this confinement, it was very easy for the network to give huge distances for false pairs. Hence, the network eventually became biased, worked well only for false pairs and didn't give small distances for true pairs resulting in the performance drop.

\Section{RESULTS}
\label{Sec:results}
The network was trained with noise using a batch size of 64, 75 training steps per epoch for 42 epochs beyond which the training loss started to oscillate. Adam optimization algorithm was used with an initial learning rate of 1e-3, this learning rate was reduced by a learning rate scheduler which checks for lack of decrease in training loss for 3 epochs and reduces the learning rate by a factor of 0.1. The reduction of the learning rate was stopped when the learning rate reached an absolute minimum of 1e-5. Early stopping was scheduled with patience of 6 which stops the training process when the training loss starts to oscillate in more than 6 epochs resulting in the maximum validation accuracy of 94.51\%. Fig. \ref{FIG:EpochvsAcc1} shows the training graph accuracy/loss vs epochs with noise.

\begin{figure*} [htbp]
\begin{subfigure}{0.5\textwidth}
  \centering
  % include first image
  \includegraphics[width=.7\linewidth]{images/Fig8-Model_training_Epochs_vs_Accuracy_with_noise}  
  \caption{Model training(Epochs vs Accuracy) with noise}
  \label{FIG:EpochvsAcc1}
\end{subfigure}
\begin{subfigure}{.5\textwidth}
  \centering
  % include second image
  \includegraphics[width=.7\linewidth]{images/Fig9-Model_training_Epochs_vs_Loss_with_noise}  
  \caption{Model training(Epochs vs Loss) with noise}
  \label{FIG:EpochvsLoss1}
\end{subfigure}\\
\begin{subfigure}{.5\textwidth}
  \centering
  % include third image
  \includegraphics[width=.7\linewidth]{images/Fig10-Model_training_Epochs_vs_Accuracy_without_noise}  
\caption{Model training(Epochs vs Accuracy) without noise}
 \label{FIG:EpochvsAcc2}
  \end{subfigure}
\begin{subfigure}{.5\textwidth}
  \centering
  % include four image
  \includegraphics[width=.7\linewidth]{images/Fig11-Model_training-Epochs_vs-Loss_without_noise}  
\caption{Model training(Epochs vs Loss) without noise}
  \label{FIG:EpochvsLoss2}
  \end{subfigure}
\caption{Model Accuracy and Loss}
\label{fig:model_accuracy_loss}
\end{figure*}

The maximum validation accuracy reached without noise is 96.8\%, this accuracy was achieved by retraining the previous model with the same hyperparameters and noise factor = 0 for 14 epochs. Fig. \ref{FIG:EpochvsAcc1} - Fig. \ref{FIG:EpochvsLoss2} shows the training graph for epoch/accuracy vs loss with and without noise.  Since the real-time audio is chunked into 1-sec windows with 0.25-sec hop length for inference, the base network was found to perform inference in 0.08 seconds in Raspberry Pi 4. Hence, the model will be able to perform inference from real-time audio streams in edge devices with no latency issues. Fig. \ref{FIG:EpochvsLoss1} shows the noise training graph for loss vs epochs.


\begin{figure*}[htbp]
\begin{subfigure}{.5\textwidth}
  \centering
  \includegraphics[width=.6\linewidth]{images/Fig12-FRP_vs_FAR_for_hotword_Alexa}  
  \caption{FRP vs FAR for hotword Alexa}
  \label{FIG:FRPvsFARalexa}
\end{subfigure}
\begin{subfigure}{.5\textwidth}
  \centering
  \includegraphics[width=.6\linewidth]{images/Fig13-FRP_vs_FAR_for_hotword_computer}  
  \caption{FRP vs FAR for hotword computer}
  \label{FIG:FRPvsFARcomputer}
\end{subfigure}\\
\begin{subfigure}{.5\textwidth}
  \centering
  \includegraphics[width=.6\linewidth]{images/Fig14-FRP_vs_FAR_for_hotword_people}  
\caption{FRP vs FAR for hotword people}
 \label{FIG:FRPvsFARpeople}
  \end{subfigure}
\begin{subfigure}{.5\textwidth}
  \centering
 \includegraphics[width=.6\linewidth]{images/Fig15-FRP_vs_FAR_for_hotword_restaurant}  
\caption{FRP vs FAR for hotword restaurant}
  \label{FIG:FRPvsFARrestaurant}
  \end{subfigure}
\caption{FRP vs FAR for different Hotwords}
\label{fig:frp_vs_far}
\end{figure*}


\begin{table}
\centering
\caption{Model benchmarks}
\label{TBL:Model_benchmarks}
\begin{tabular}{ p{2cm} p{2cm}   p{2cm}}
 \hline \hline
Model   &Accuracy &Inference \\
 \hline
Efficientword-Net (Current paper) &94.51\%   &0.071ms \\
Porcupine \cite{picovoice_alireza} &94.78\% &0.02ms \\
PocketSphinx \cite{kalith_2012} &54.23\% &0.076ms \\
Snowboy \cite{Yang_Jee_Leblanc_Weaver_Armand_2020} &88.43\% &0.091ms \\
\hline
\end{tabular}
\end{table}%

After performing significance test, the resulting trained model was benchmarked with other hotword detection systems on Raspberry Pi 3 clocked at 1.2GHz (4 core) and displayed in the Table. \ref{TBL:Model_benchmarks} and was found to outperform existing closed source models in terms of accuracy by a small level.

\begin{figure*}[htbp]
\begin{subfigure}{.5\textwidth}
  \centering
  \includegraphics[width=.5\linewidth]{images/Fig16-Bar_graph_for_FRP_for_hotword_alexa}  
  \caption{Bar graph for FRP for hotword Alexa}
  \label{FIG:BarGraphAlexa}
\end{subfigure}
\begin{subfigure}{.5\textwidth}
  \centering
  \includegraphics[width=.5\linewidth]{images/Fig17-Bar_graph_for_FRP_for_hotword_computer}
  \caption{Bar graph for FRP for hotword computer}
  \label{FIG:BarGraphComputer}
\end{subfigure}
\\
\begin{subfigure}{.5\textwidth}
  \centering
  \includegraphics[width=.5\linewidth]{images/Fig18-Bar_graph_for_FRP_for_hotword_people}  
\caption{Bar graph for FRP for hotword people}
\label{FIG:BarGraphPeople}
  \end{subfigure}
\begin{subfigure}{.5\textwidth}
  \centering
 \includegraphics[width=.5\linewidth]{images/Fig19-Bar_graph_for_FRP_for_hotword_restaurant}  
\caption{Bar graph for FRP for hotword restaurant}
 \label{FIG:BarGraphRestaurant}
  \end{subfigure}
\caption{Bar graphs for FRP of different Hotwords}
\label{fig:bar_graphs}
\end{figure*}

For a given sensitivity value, False Rejection Rate (FRP) – (True Negatives) is measured by playing a set of sample audio files which include the utterance of the hotword, and then calculate the ratio of rejections to the total number of samples. False Acceptance Rate (FAR-False Positives) is  measured by playing a background audio file which must not include any utterance of the hotword calculated by dividing the number of false acceptances by the length of the background audio in hours. Figures Fig. \ref{FIG:FRPvsFARalexa} - \ref{FIG:FRPvsFARrestaurant} and \ref{FIG:BarGraphAlexa} - \ref{FIG:BarGraphRestaurant} illustrate FRP vs FAR and model performance against existing implementations for various hotwords. All these hotwords are not included in the training dataset.


\Section{CONCLUSION}
\label{Sec:conclusion}

In this paper, we proposed a one-shot learning-based hotword detection engine to solve the problem of retraining and huge dataset requirements for each new hotword with good inference time on light-weight devices. To achieve the same we implemented Siamese neural network architecture with an image processing base network made with EfficientNet, which processes the Log Mel spectrograms of the respective input audio samples. Moreover, this network could also be repurposed for phrase detection's where a program needs to check for the occurrence of a specific sentence removing the requirement of heavy speech-to-text engines in edge devices. Such an engine can allow the end-users to set custom hotwords in their systems with minimal effort.

\Section{ACKNOWLEDGMENTS} 

This research is carried out at Artificial Intelligence and Robotics (AIR) Research Centre, VIT-AP University. We also thank the management for motivating and supporting AIR Research Centre, VIT-AP University in building this project.





\bibliographystyle{abbrv}
\bibliography{output}
% \graphicspath{{images/}}
\end{document}

\end{document} 
