\begin{table}
\begin{center}
\vspace{-1mm}
\caption{\textbf{JT-VAE - Gradient ascent results.} We obtain state-of-the-art performance in terms of penalized logP via gradient ascent. However, most generated molecules are of very low quality (only ~1\% pass the quality filters from \citet{Brown_2019}). Leveraging the uncertainty of the decoder (IS-MI) during optimization helps generating molecules with high penalized logP and high quality. NLLP constraints help maintain high quality but lead to suboptimal black-box objective values. Results with different hyperparameters and threshold values for each method are reported in Table~\ref{Appendix_E_Table_JTVAE_GA_results}.}
\resizebox{\textwidth}{!}{
\begin{tabular}{cccccc}
\toprule
\textbf{Decoder} &  \multicolumn{2}{c}{\textbf{Penalized logP - Before filters}} & \textbf{Quality top 10} & \multicolumn{2}{c}{\textbf{Penalized logP - Passing filters}}\\
\textbf{uncertainty} & \textbf{Top 1} $\uparrow$ & \textbf{Avg. top 10} $\uparrow$ & \textbf{(\%)} $\uparrow$ & \textbf{Top 1} $\uparrow$ & \textbf{Avg. top 10} $\uparrow$\\
\toprule
None & $\textbf{23.7} \pm \textbf{1.3}$& $\textbf{17.0} \pm \textbf{0.6}$& $ 1\% \pm 1\%$& $1.2 \pm 1.2$& $0.3 \pm 0.3$ \\
NLLP & $3.0 \pm 0.1$& $2.5 \pm 0.1$& $ 82\% \pm 6\%$& $3.0 \pm 0.1$& $2.0 \pm 0.2$ \\
IS-MI & $8.4 \pm 10.8$& $6.0 \pm 0.3$& $ \textbf{89\%} \pm 3\textbf{\%}$& $\textbf{7.7} \pm \textbf{0.7}$& $\textbf{5.3} \pm \textbf{0.3}$ \\
\bottomrule
\end{tabular}
}
\label{Sec5_Table_JTVAE_GA_results}
\end{center}
\vspace{-4mm}
\end{table}