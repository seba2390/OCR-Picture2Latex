\documentclass[11pt,letter]{article}
\usepackage{bbding}
\fussy
\usepackage[dvipsnames]{xcolor}
\usepackage{framed, xcolor}
\usepackage{tocvsec2}
\usepackage{datetime}
\usepackage{pifont}
\usepackage{subfigure}          % for subfigures in ACM
\usepackage{colortbl}
\usepackage{booktabs}
\usepackage{pdfsync}
\usepackage[pagebackref=false,citecolor=newblue,colorlinks=true,linkcolor=newblue,bookmarks=false]{hyperref}

\usepackage[font=scriptsize,bf]{caption}
\usepackage{tikz,subfigure}
\usepackage[active]{srcltx}
\definecolor{newblue}{rgb}{0.2,0.2,0.6} % Define the color BrickRed

\usepackage[margin=1in]{geometry}
\usepackage{algorithm}
\usepackage{algpseudocode}
\usepackage{datetime}
\usepackage{epsfig,amssymb,amsfonts,amsmath,amsthm}
\usepackage{multirow}
\usepackage[numbers,sort&compress,sectionbib]{natbib}
\newcommand{\tsum}{\textstyle\sum}
\newcommand{\tprod}{\textstyle\prod}
\newcommand{\Rset}{\mathbb{R}}
\newcommand{\PSD}{\mathsf{PSD}}
\newcommand{\ellip}{\mathsf{Ellip}}
\newcommand{\OSS}{\textsf{OSS}}

\newcommand{\oracle}{\mathtt{Oracle}}
\newcommand{\calM}{\mathcal{M}}


\newcommand{\rot}{\intercal}
\newcommand{\hfamily}{\mathcal{H}}
\newcommand{\calL}{\mathcal{L}}
\newcommand{\calA}{\mathcal{A}}
\newcommand{\expn}{\mathrm{e}}

\newcommand{\eps}{\varepsilon}
\newcommand{\vol}{\mu}
\newcommand{\sign}{\mathsf{sign}}
\newcommand{\de}{\operatorname{de}}
\newcommand{\even}{\mathsf{even}}
\newcommand{\odd}{\mathsf{odd}}
\newcommand{\COST}{\mathsf{COST}}
\newcommand{\OPT}{\mathsf{OPT}}
\newcommand{\SOL}{\mathsf{SOL}}
\newcommand{\ALG}{\textsf{ABC}}
\newcommand{\thmref}[1]{Theorem~\ref{thm:#1}}
\newcommand{\thmmref}[1]{Thm.~\ref{thm:#1}}
\newcommand{\thmrefs}[2]{Theorems~\ref{thm:#1} and~\ref{thm:#2}}
\newcommand{\proref}[1]{Proposition~\ref{pro:#1}}
\newcommand{\lemref}[1]{Lemma~\ref{lem:#1}}
\newcommand{\lemrefs}[2]{Lemmas~\ref{lem:#1} and~\ref{lem:#2}}
\newcommand{\lemrefss}[3]{Lemmas~\ref{lem:#1},~\ref{lem:#2}, and~\ref{lem:#3}}
\newcommand{\corref}[1]{Corollary~\ref{cor:#1}}
\newcommand{\obsref}[1]{Observation~\ref{obs:#1}}
\newcommand{\defref}[1]{Definition~\ref{def:#1}}
\newcommand{\defrefs}[2]{Definitions~\ref{def:#1} and~\ref{def:#2}}
\newcommand{\assref}[1]{Assumption~\eqref{ass:#1}}
\newcommand{\conref}[1]{Conjecture~\ref{con:#1}}
\newcommand{\figref}[1]{Figure~\ref{fig:#1}}
\newcommand{\figrefs}[2]{Figures~\ref{fig:#1} and~\ref{fig:#2}}
\newcommand{\tabref}[1]{Table~\ref{tab:#1}}
\newcommand{\secref}[1]{Section~\ref{sec:#1}}
\newcommand{\secrefs}[2]{Sections~\ref{sec:#1} and~\ref{sec:#2}}
\newcommand{\charef}[1]{Chapter~\ref{cha:#1}}
\newcommand{\eq}[1]{\eqref{eq:#1}}
\newcommand{\eqs}[2]{equations~\eqref{eq:#1} and~\eqref{eq:#2}}
\newcommand{\eqss}[3]{equations~\eqref{eq:#1},~\eqref{eq:#2}, and~\eqref{eq:#3}}
\newcommand{\Eqs}[2]{Equations~\eqref{eq:#1} and~\eqref{eq:#2}}
\newcommand{\propref}[1]{P\ref{prop:#1}}
\newcommand{\COOR}{\mathcal{PR}}
%\newcommand{\pro}[1]{\mathbf{Pr}[\,#1\,]}
\newcommand{\pr}[1]{\mathbb{P} [\,#1\,]}
\newcommand{\Pro}[1]{\mathbb{P} \left(,#1\,\right)}
\newcommand{\Prob}[2]{\mathbf{P}_{#1} \left[\,#2\,\right]}
\newcommand{\PRO}[1]{\widetilde{\mathbf{Pr}} \left[\,#1\,\right]}
\newcommand{\Proo}[1]{\mathbb{P}[\,#1\,]}
\newcommand{\e}{\mathbf{E}}
\newcommand{\ex}[1]{\mathbf{E}[\,#1\,]}
\newcommand{\TEx}[1]{\widetilde{\mathbb{E}} \left[\,#1\,\right]}
\newcommand{\Ex}[1]{\mathbb{E} \left[\,#1\,\right]}
\newcommand{\Tr}[1]{\mathrm{tr} \left(\,#1\,\right)}
\newcommand{\EX}[1]{\widetilde{\mathbb{E}} \left[\,#1\,\right]}
\newcommand{\EXX}[2]{\mathbb{E}_{#1} \left[\,#2\,\right]}
\newcommand{\Var}[1]{\mathbf{Var} \left[\,#1\,\right]}
\newcommand{\Cov}[1]{\mathbf{Cov} \left[\,#1\,\right]}
%\newcommand{\EX}[1]{\mathbb{E}\Bigl[\,#1\,\Bigr]}
\newcommand{\degree}{\operatorname{deg}}
\newcommand{\girth}{\operatorname{girth}}
\newcommand{\diam}{\operatorname{diam}}
\newcommand{\id}{\mathsf{id}}
\renewcommand{\tilde}{\widetilde}
\renewcommand{\epsilon}{\varepsilon}
\newcommand{\opt}{\mathrm{opt}}
\newcommand{\aseq}{\{A_i\}_{i=1}^k}
\newcommand{\sseq}{\{S_i\}_{i=1}^k}
\newcommand{\AlgoMean}{\mathsf{AlgoMean}}
\newcommand{\cald}{\mathcal{D}}

\renewcommand{\vec}[1]{#1}
\newcommand{\mat}[1]{#1}


\newtheorem{thm}{Theorem}[section]  %[chapter]
\newtheorem{fact}[thm]{Fact}
\newtheorem{claim}[thm]{Claim}
\newtheorem{lem}[thm]{Lemma}
\newtheorem{obs}[thm]{Observation}
\newtheorem{clm}[thm]{Claim}
\newtheorem{cor}[thm]{Corollary}
\newtheorem{fac}[thm]{Fact}
\newtheorem{rem}[thm]{Remark}
\newtheorem*{rem*}{Remark}
%\newtheorem{ass}[thm]{Assumption}
\newtheorem{pro}[thm]{Proposition}
\newtheorem{con}[thm]{Conjecture}
%\theoremstyle{definition}
\newtheorem{construction}{Construction}
\newtheorem{protocol}{Protocol}
\newtheorem{defi}[thm]{Definition}
\newtheorem{assumption}[thm]{Assumption}
\newtheorem{prob}{Problem}
\newcommand{\qinsays}[2][]{\textcolor{blue} {\textsc{Qinsays #1:} \emph{#2}}}

\newcommand{\he}[1]{{ \textbf{\color{red} He: #1}}}

\newcommand{\ONE}{{$\mathsf{ALLONE}_s$}}
\newcommand{\DISJN}{{$\mathsf{DISJ}_{s,n}$}}
\newcommand{\DISJL}{{$\mathsf{DISJ}_{s,\ell}$}}


\newcommand{\FOCS}[2]{#1 Annual IEEE Symposium on Foundations of Computer Science
(FOCS'#2)}
\newcommand{\STOC}[2]{#1 Annual ACM Symposium on Theory of Computing (STOC'#2)}
\newcommand{\SODA}[2]{#1 Annual ACM-SIAM Symposium on Discrete Algorithms (SODA'#2)}
\newcommand{\ICALP}[2]{#1 International Colloquium on Automata, Languages, and
Programming (ICALP'#2)}
\newcommand{\PODC}[2]{#1 Annual ACM-SIGOPT Principles of Distributed Computing
(PODS'#2)}
\newcommand{\STACS}[2]{#1 International Symposium on Theoretical Aspects of Computer
Science (STACS'#2)}
\newcommand{\SPAA}[2]{#1 Annual ACM Symposium on Parallel Algorithms and
Architectures (SPAA'#2)}
\newcommand{\MFCS}[2]{#1 International Symposium on Mathematical Foundations of
Computer
Science (MFCS'#2)}
\newcommand{\ISAAC}[2]{#1 International Symposium on Algorithms and Computation
(ISAAS'#2)}
\newcommand{\WG}[2]{#1 Workshop of Graph-Theoretic Concepts in Computer Science
(WG'#2)}
\newcommand{\SIROCCO}[2]{#1 International Colloquium on Structural Information and
Communication Complexity (SIROCCO'#2)}
\newcommand{\IPDPS}[2]{#1 IEEE International Parallel and Distributed Processing
Symposium (IPDPS'#2)}
\newcommand{\RANDOM}[2]{#1 International Workshop on Randomization and Computation
(RANDOM'#2)}
\newcommand{\CCC}[2]{#1 Conference on Computational Complexity
(CCC'#2)}
\newcommand{\ICML}[2]{#1 International Conference on Machine Learning
(ICML'#2)}

\newcommand{\ITCS}[2]{#1 Innovations in Theoretical Computer Science (ITCS'#2)}



\title{\textbf{Communication-Optimal Distributed Clustering}\footnote{A preliminary version of this paper appears at the 30th
Annual Conference on Neural Information Processing Systems (NIPS), 2016.}}

% The \author macro works with any number of authors. There are two
% commands used to separate the names and addresses of multiple
% authors: \And and \AND.
%
% Using \And between authors leaves it to LaTeX to determine where to
% break the lines. Using \AND forces a line break at that point. So,
% if LaTeX puts 3 of 4 authors names on the first line, and the last
% on the second line, try using \AND instead of \And before the third
% author name.

\author{
  Jiecao Chen\footnote{Department of Computer Science, Indiana University, Bloomington,  USA.  Work supported in part by NSF
CCF-1525024 and IIS-1633215.
  Email: \texttt{jiecchen@indiana.edu}}
 \and
  He Sun\footnote{Department of Computer Science, University of Bristol, Bristol,  UK.
 \texttt{h.sun@bristol.ac.uk}} 
\and
   David P. Woodruff\footnote{IBM Research Almaden, San Jose,  USA. \texttt{dpwoodru@us.ibm.com}}
   \and
   Qin Zhang\footnote{Department of Computer Science, Indiana University, Bloomington, USA. Work supported in part by NSF
CCF-1525024 and IIS-1633215.
 Email: \texttt{qzhangcs@indiana.edu}}
}

\date{}

\begin{document}
% \nipsfinalcopy is no longer used

\maketitle

\begin{abstract}
  %  Clustering large datasets is a fundamental problem with a number of applications in machine learning. Data is often collected on different sites and clustering needs to be performed in a distributed manner with low communication. We would like the quality of the clustering in the distributed setting to match that in the centralized setting for which all the data resides on a single site. In this work, we study both graph and geometric clustering problems in two distributed models: (1) a point-to-point model, and (2) a model with a broadcast channel. We give protocols in both models which we show are nearly optimal by proving almost matching communication lower bounds. Our work highlights the surprising power of a broadcast channel for clustering problems; roughly speaking, to  cluster $n$ points or $n$ vertices in a graph distributed across $s$ servers, for a worst-case partitioning the communication complexity in a point-to-point model is $n \cdot s$, while in the broadcast model it is $n + s$. Similar phenomenon holds for the geometric setting as well. We implement our algorithms and demonstrate this phenomenon on real life datasets, showing that our algorithms are also very efficient in practice.
Clustering large datasets is a fundamental problem with a number of applications in machine learning. Data is often collected on different sites and clustering needs to be performed in a distributed manner with low communication. We would like the quality of the clustering in the distributed setting to match that in the centralized setting for which all the data resides on a single site. In this work, we study both graph and geometric clustering problems in two distributed models: (1) a point-to-point model, and (2) a model with a broadcast channel. We give protocols in both models which we show are nearly optimal by proving almost matching communication lower bounds. Our work highlights the surprising power of a broadcast channel for clustering problems; roughly speaking, to spectrally cluster $n$ points or $n$ vertices in a graph distributed across $s$ servers, for a worst-case partitioning the communication complexity in a point-to-point model is $n \cdot s$, while in the broadcast model it is $n + s$. A similar phenomenon holds for the geometric setting as well. We implement our algorithms and demonstrate this phenomenon on real life datasets, showing that our algorithms are also very efficient in practice. 
\end{abstract}


\thispagestyle{empty}


\setcounter{page}{0}

\newpage

\tableofcontents

\thispagestyle{empty}

\setcounter{page}{0}

\newpage


% \leavevmode
% \\
% \\
% \\
% \\
% \\
\section{Introduction}
\label{introduction}

AutoML is the process by which machine learning models are built automatically for a new dataset. Given a dataset, AutoML systems perform a search over valid data transformations and learners, along with hyper-parameter optimization for each learner~\cite{VolcanoML}. Choosing the transformations and learners over which to search is our focus.
A significant number of systems mine from prior runs of pipelines over a set of datasets to choose transformers and learners that are effective with different types of datasets (e.g. \cite{NEURIPS2018_b59a51a3}, \cite{10.14778/3415478.3415542}, \cite{autosklearn}). Thus, they build a database by actually running different pipelines with a diverse set of datasets to estimate the accuracy of potential pipelines. Hence, they can be used to effectively reduce the search space. A new dataset, based on a set of features (meta-features) is then matched to this database to find the most plausible candidates for both learner selection and hyper-parameter tuning. This process of choosing starting points in the search space is called meta-learning for the cold start problem.  

Other meta-learning approaches include mining existing data science code and their associated datasets to learn from human expertise. The AL~\cite{al} system mined existing Kaggle notebooks using dynamic analysis, i.e., actually running the scripts, and showed that such a system has promise.  However, this meta-learning approach does not scale because it is onerous to execute a large number of pipeline scripts on datasets, preprocessing datasets is never trivial, and older scripts cease to run at all as software evolves. It is not surprising that AL therefore performed dynamic analysis on just nine datasets.

Our system, {\sysname}, provides a scalable meta-learning approach to leverage human expertise, using static analysis to mine pipelines from large repositories of scripts. Static analysis has the advantage of scaling to thousands or millions of scripts \cite{graph4code} easily, but lacks the performance data gathered by dynamic analysis. The {\sysname} meta-learning approach guides the learning process by a scalable dataset similarity search, based on dataset embeddings, to find the most similar datasets and the semantics of ML pipelines applied on them.  Many existing systems, such as Auto-Sklearn \cite{autosklearn} and AL \cite{al}, compute a set of meta-features for each dataset. We developed a deep neural network model to generate embeddings at the granularity of a dataset, e.g., a table or CSV file, to capture similarity at the level of an entire dataset rather than relying on a set of meta-features.
 
Because we use static analysis to capture the semantics of the meta-learning process, we have no mechanism to choose the \textbf{best} pipeline from many seen pipelines, unlike the dynamic execution case where one can rely on runtime to choose the best performing pipeline.  Observing that pipelines are basically workflow graphs, we use graph generator neural models to succinctly capture the statically-observed pipelines for a single dataset. In {\sysname}, we formulate learner selection as a graph generation problem to predict optimized pipelines based on pipelines seen in actual notebooks.

%. This formulation enables {\sysname} for effective pruning of the AutoML search space to predict optimized pipelines based on pipelines seen in actual notebooks.}
%We note that increasingly, state-of-the-art performance in AutoML systems is being generated by more complex pipelines such as Directed Acyclic Graphs (DAGs) \cite{piper} rather than the linear pipelines used in earlier systems.  
 
{\sysname} does learner and transformation selection, and hence is a component of an AutoML systems. To evaluate this component, we integrated it into two existing AutoML systems, FLAML \cite{flaml} and Auto-Sklearn \cite{autosklearn}.  
% We evaluate each system with and without {\sysname}.  
We chose FLAML because it does not yet have any meta-learning component for the cold start problem and instead allows user selection of learners and transformers. The authors of FLAML explicitly pointed to the fact that FLAML might benefit from a meta-learning component and pointed to it as a possibility for future work. For FLAML, if mining historical pipelines provides an advantage, we should improve its performance. We also picked Auto-Sklearn as it does have a learner selection component based on meta-features, as described earlier~\cite{autosklearn2}. For Auto-Sklearn, we should at least match performance if our static mining of pipelines can match their extensive database. For context, we also compared {\sysname} with the recent VolcanoML~\cite{VolcanoML}, which provides an efficient decomposition and execution strategy for the AutoML search space. In contrast, {\sysname} prunes the search space using our meta-learning model to perform hyperparameter optimization only for the most promising candidates. 

The contributions of this paper are the following:
\begin{itemize}
    \item Section ~\ref{sec:mining} defines a scalable meta-learning approach based on representation learning of mined ML pipeline semantics and datasets for over 100 datasets and ~11K Python scripts.  
    \newline
    \item Sections~\ref{sec:kgpipGen} formulates AutoML pipeline generation as a graph generation problem. {\sysname} predicts efficiently an optimized ML pipeline for an unseen dataset based on our meta-learning model.  To the best of our knowledge, {\sysname} is the first approach to formulate  AutoML pipeline generation in such a way.
    \newline
    \item Section~\ref{sec:eval} presents a comprehensive evaluation using a large collection of 121 datasets from major AutoML benchmarks and Kaggle. Our experimental results show that {\sysname} outperforms all existing AutoML systems and achieves state-of-the-art results on the majority of these datasets. {\sysname} significantly improves the performance of both FLAML and Auto-Sklearn in classification and regression tasks. We also outperformed AL in 75 out of 77 datasets and VolcanoML in 75  out of 121 datasets, including 44 datasets used only by VolcanoML~\cite{VolcanoML}.  On average, {\sysname} achieves scores that are statistically better than the means of all other systems. 
\end{itemize}


%This approach does not need to apply cleaning or transformation methods to handle different variances among datasets. Moreover, we do not need to deal with complex analysis, such as dynamic code analysis. Thus, our approach proved to be scalable, as discussed in Sections~\ref{sec:mining}.


%!TEX root = hopfwright.tex
%

In this section we systematically recast the Hopf bifurcation problem in Fourier space. 
We introduce appropriate scalings, sequence spaces of Fourier coefficients and convenient operators on these spaces. 
To study Equation~\eqref{eq:FourierSequenceEquation} we consider Fourier sequences $ \{a_k\}$ and fix a Banach space in which these sequences reside. It is indispensable for our analysis that this space have an algebraic structure. 
The Wiener algebra of absolutely summable Fourier series is a natural candidate, which we use with minor modifications. 
In numerical applications, weighted sequence spaces with algebraic and geometric decay have been used to great effect to study periodic solutions which are $C^k$ and analytic, respectively~\cite{lessard2010recent,hungria2016rigorous}. 
Although it follows from Lemma~\ref{l:analytic} that the Fourier coefficients of any solution decay exponentially, we choose to work in a space of less regularity. 
The reason is that by working in a space with less regularity, we are better able to connect our results with the global estimates in \cite{neumaier2014global}, see Theorem~\ref{thm:UniqunessNbd2}.


%
%
%\begin{remark}
%	Although it follows from Lemma~\ref{l:analytic} that the Fourier coefficients of any solution decay exponentially, we choose to work in a space of less regularity, namely summable Fourier coefficients. This allows us to draw SOME MORE INTERESTING CONCLUSION LATER.
%	EXPLAIN WHY WE CHOOSE A NORM WITH ALMOST NO DECAY!
%	% of s Periodic solutions to Wright's equation are known to be real analytic and so their  Fourier coefficients must decay geometrically [Nussbaum].
%	% We do not use such a strong result;  any periodic solution must be continuously differentiable, by which it follows that $ \sum | c_k| < \infty$.
%\end{remark}


\begin{remark}\label{r:a0}
There is considerable redundancy in Equation~\eqref{eq:FourierSequenceEquation}. First, since we are considering real-valued solutions $y$, we assume $\c_{-k}$ is the complex conjugate of $\c_k$. This symmetry implies it suffices to consider Equation~\eqref{eq:FourierSequenceEquation} for $k \geq 0$.
Second, we may effectively ignore the zeroth Fourier coefficient of any periodic solution \cite{jones1962existence}, since it is necessarily equal to $0$. 
%In \cite{jones1962existence}, it is shown that if $y \not\equiv -1$ is a periodic solution of~\eqref{eq:Wright} with frequency $\omega$, then $ \int_0^{2\pi/\omega} y(t) dt =0$. 
		The self contained argument is as follows. 
		As mentioned in the introduction, any periodic solution to Wright's equation must satisfy $ y(t) > -1$ for all $t$. 
	By dividing Equation~\eqref{eq:Wright} by $(1+y(t))$, which never vanishes, we obtain
	\[
	\frac{d}{dt} \log (1 + y(t)) = - \alpha y(t-1).
	\]  
	Integrating over one period $L$ we derive the condition 
	$0=\int_0^L y(t) dt $.
	Hence $a_0=0$ for any periodic solution. 
	It will be shown in Theorem~\ref{thm:FourierEquivalence1} that a related argument implies that we do not need to consider Equation~\eqref{eq:FourierSequenceEquation} for $k=0$.
\end{remark}

%%%
%%%
%%%\begin{remark}\label{r:c0} 
%%%In \cite{jones1962existence}, it is shown that if $y \not\equiv -1$ is a periodic solution of~\eqref{eq:Wright} with frequency $\omega$, then $ \int_0^{2\pi/\omega} y(t) dt =0$. 
%%%PERHAPS TOO MUCH DETAIL HERE. The self contained argument is as follows.
%%%If $y \not\equiv -1$ then $y(t) \neq -1$ for all $t$, since if $y(t_0)=-1$ for some $t_0 \in \R$ then $y'(t_0)=0$ by~\eqref{eq:Wright} and in fact by differentiating~\eqref{eq:Wright} repeatedly one obtains that all derivatives of $y$ vanish at $t_0$. Hence $y \equiv -1$ by Lemma~\ref{l:analytic}, a contradiction. Now divide~\eqref{eq:Wright} by $(1+y(t))$, which never vanishes, to obtain
%%%\[
%%%  \frac{d}{dt} \log |1 + y(t)| = - \alpha y(t-1).
%%%\]  
%%%Integrating over one period we obtain $\int_0^L y(t) dt =0$.
%%%\end{remark}



%Furthermore, the condition that $y(t)$ is real forces $\c_{-k} = \overline{\c}_{k}$.  
%
We define the spaces of absolutely summable Fourier series
\begin{alignat*}{1}
	\ell^1 &:= \left\{ \{ \c_k \}_{k \geq 1} : 
    \sum_{k \geq 1} | \c_k| < \infty  \right\} , \\
	\ell^1_\bi &:= \left\{ \{ \c_k \}_{k \in \Z} : 
    \sum_{k \in \Z} | \c_k| < \infty  \right\} .
\end{alignat*} 
We identify any semi-infinite sequence $ \{ \c_k \}_{k \geq 1} \in \ell^1$ with the bi-infinite sequence $ \{ \c_k \}_{k \in \Z} \in \ell^1_\bi$ via the conventions (see Remark~\ref{r:a0})
\begin{equation}
  \c_0=0 \qquad\text{ and }\qquad \c_{-k} = \c_{k}^*. 
\end{equation}
In other word, we identify $\ell^1$ with the set
\begin{equation*}
   \ell^1_\sym := \left\{ \c \in \ell^1_\bi : 
	\c_0=0,~\c_{-k}=\c_k^* \right\} .
\end{equation*}
On $\ell^1$ we introduce the norm
\begin{equation}\label{e:lnorm}
  \| \c \| = \| \c \|_{\ell^1} := 2 \sum_{k = 1}^\infty |\c_k|.
\end{equation}
The factor $2$ in this norm is chosen to have a Banach algebra estimate.
Indeed, for $\c, \tilde{\c} \in \ell^1 \cong \ell^1_\sym$ we define
the discrete convolution 
\[
\left[ \c * \tilde{\c} \right]_k = \sum_{\substack{k_1,k_2\in\Z\\ k_1 + k_2 = k}} \c_{k_1} \tilde{\c}_{k_2} .
\]
Although $[\c*\tilde{\c}]_0$ does not necessarily vanish, we have $\{\c*\tilde{\c}\}_{k \geq 1} \in \ell^1 $ and 
\begin{equation*}
	\| \c*\tilde{\c} \| \leq \| \c \| \cdot  \| \tilde{\c} \| 
	\qquad\text{for all } \c , \tilde{\c} \in \ell^1, 
\end{equation*}
hence $\ell^1$ with norm~\eqref{e:lnorm} is a Banach algebra.

By Lemma~\ref{l:analytic} it is clear that any periodic solution of~\eqref{eq:Wright} has a well-defined Fourier series $\c \in \ell^1_\bi$. 
The next theorem shows that in order to study periodic orbits to Wright's equation we only need to study Equation~\eqref{eq:FourierSequenceEquation} 
for $k \geq 1$. For convenience we introduce the notation 
\[
G(\alpha,\omega,\c)_k=
( i \omega k + \alpha e^{ - i \omega k}) \c_k + \alpha \sum_{k_1 + k_2 = k} e^{- i \omega k_1} \c_{k_1} \c_{k_2} \qquad \text{for } k \in \N.
\]
We note that we may interpret the trivial solution $y(t)\equiv 0$ as a periodic solution of arbitrary period.
\begin{theorem}
\label{thm:FourierEquivalence1}
Let $\alpha>0$ and $\omega>0$.
If $\c \in \ell^1 \cong \ell^1_{\sym}$ solves
$G(\alpha,\omega,\c)_k =0$  for all $k \geq 1$,
then $y(t)$ given by~\eqref{eq:FourierEquation} is a periodic solution of~\eqref{eq:Wright} with period~$2\pi/\omega$.
Vice versa, if $y(t)$ is a periodic solution of~\eqref{eq:Wright} with period~$2\pi/\omega$ then its Fourier coefficients $\c \in \ell^1_\bi$ lie in $\ell^1_\sym \cong \ell^1$ and solve $G(\alpha,\omega,\c)_k =0$ for all $k \geq 1$.
\end{theorem}

\begin{proof}	
	If $y(t)$ is a periodic solution of~\eqref{eq:Wright} then it is real analytic by Lemma~\ref{l:analytic}, hence its Fourier series $\c$ is well-defined and $\c \in \ell^1_{\sym}$ by Remark~\ref{r:a0}.
	Plugging the Fourier series~\eqref{eq:FourierEquation} into~\eqref{eq:Wright} one easily derives that $\c$ solves~\eqref{eq:FourierSequenceEquation} for all $k \geq 1$.

To prove the reverse implication, assume that $\c \in \ell^1_\sym$ solves
Equation~\eqref{eq:FourierSequenceEquation} for all $k \geq 1$. Since $\c_{-k}
= \c_k^*$, Equation \eqref{eq:FourierSequenceEquation} is also satisfied for
all $k \leq -1$. It follows from the Banach algebra property and
\eqref{eq:FourierSequenceEquation} that $\{k \c_k\}_{k \in \Z} \in \ell^1_\bi$,
hence $y$, given by~\eqref{eq:FourierEquation}, is continuously differentiable.
% (and by bootstrapping one infers that $\{k^m c_k \} \in \ell^1_\bi$, 
% hence $y \in C^m$ for any $m \geq 1$).
	Since~\eqref{eq:FourierSequenceEquation} is satisfied for all $k \in \Z \setminus \{0\}$ (but not necessarily for $k=0$) one may perform the inverse Fourier transform on~\eqref{eq:FourierSequenceEquation} to conclude that
	$y$ satisfies the delay equation 
\begin{equation}\label{eq:delaywithK}
   	y'(t) = - \alpha y(t-1) [ 1 + y(t)] + C
\end{equation}
	for some constant $C \in \R$. 
   Finally, to prove that $C=0$ we argue by contradiction.
   Suppose $C \neq 0$. Then $y(t) \neq -1$ for all $t$.
   Namely, at any point where $y(t_0) =-1$ one would have $y'(t_0) = C$
   which has fixed sign,   hence it would follow that $y$ is not periodic
   ($y$ would not be able to cross $-1$ in the opposite direction, 
   preventing $y$  from being periodic).  
  We may thus divide~\eqref{eq:delaywithK} through by $1 + y(t)$ and obtain 
\begin{equation*}
	\frac{d}{dt} \log | 1 + y(t) | = - \alpha y(t-1) + \frac{C}{1+y(t)} .
\end{equation*}
	By integrating both sides of the equation over one period $L$ and by using that $\c_0=0$, we 
	obtain
	\[
	 C \int_0^L \frac{1}{1+y(t)} dt =0.
	\]
	Since the integrand is either strictly negative or strictly positive, this implies that $C=0$. Hence~\eqref{eq:delaywithK} reduces to~\eqref{eq:Wright},
	and $y$ satisfies Wright's equation. 
\end{proof}






To efficiently study Equation~\eqref{eq:FourierSequenceEquation}, we introduce the following linear operators on $ \ell^1$:
\begin{alignat*}{1}
   [K \c ]_k &:= k^{-1} \c_k  , \\ 
   [ U_\omega \c ]_k &:= e^{-i k \omega} \c_k  .
\end{alignat*}
The map $K$ is a compact operator, and it has a densely defined inverse $K^{-1}$. The domain of $K^{-1}$ is denoted by
\[
  \ell^K := \{ \c \in \ell^1 : K^{-1} \c \in \ell^1 \}.  
\]
The map $U_{\omega}$ is a unitary operator on $\ell^1$, but
it is discontinuous in $\omega$. 
With this notation, Theorem~\ref{thm:FourierEquivalence1} implies that our problem of finding a SOPS to~\eqref{eq:Wright} is equivalent to finding an $\c \in \ell^1$ such that
\begin{equation}
\label{e:defG}
  G(\alpha,\omega,\c) :=
  \left( i \omega K^{-1} + \alpha U_\omega \right) \c + \alpha \left[U_\omega \, \c \right] * \c  = 0.
\end{equation}


%In order for the solutions of Equation \ref{eq:FHat} to be isolated we need to impose a phase condition. 
%If there is a sequence $ \{ c_k \} $ which satisfies  Equation \ref{eq:FHat}, then $ y( t + \tau) = \sum_{k \in \Z} c_k e^{ i k \omega (t + \tau)}$ satisfies Wright's equation at parameter $\alpha$. 
%Fix $ \tau = - Arg[c_1] / \omega$ so that $ c_1  e^{ i \omega \tau} $ is a nonnegative real number. 
%By Proposition \ref{thm:FourierEquivalence1} it follows that $\{ c'_k \} =  \{c_k e^{ i \omega k \tau }   \}$ is a solution to Equation \ref{eq:FHat}, and furthermore that $ c'_1 = \epsilon$ for some $ \epsilon \geq 0$. 


Periodic solutions are invariant under time translation: if $y(t)$ solves Wright's equation, then so does $ y(t+\tau)$ for any $\tau \in \R$. 
We remove this degeneracy by adding a phase condition. 
Without loss of generality, if $\c \in \ell^1$ solves Equation~\eqref{e:defG}, we may assume that $\c_1 = \epsilon$ for some 
\emph{real non-negative}~$\epsilon$:
\[
  \ell^1_{\epsilon} := \{\c \in \ell^1 : \c_1 = \epsilon \} 
  \qquad \text{where } \epsilon \in \R,  \epsilon \geq 0.
\]
In the rest of our analysis, we will split elements $\c \in \ell^1$ into two parts: $\c_1$ and $\{\c_{k}\}_{k \geq 2}$.  
We define the basis elements $\e_j \in \ell^1$ for $j=1,2,\dots$ as
\[
  [\e_j]_k = \begin{cases}
  1 & \text{if } k=j, \\
  0 & \text{if } k \neq j.
  \end{cases}
\]
We note that $\| \e_j \|=2$. 
Then we can decompose
% We define
% \[
%   \tilde{\epsilon} := (\epsilon,0,0,0,\dots) \in \ell^1
% \]
% and
% For clarity when referring to sequences $\{c_{k}\}_{k \geq 2}$, we make the following definition:
% \[
% \ell^1_0  := \{ \tc \in \ell^1 : \tc_1 = 0 \}.
% \]
% With the
any $\c \in \ell^1_\epsilon$ uniquely as
\begin{equation}\label{e:aepsc}
  \c= \epsilon \e_1 + \tc \qquad \text{with}\quad 
  \tc \in \ell^1_0 := \{ \tc \in \ell^1 : \tc_1 = 0 \}.
\end{equation}
We follow the classical approach in studying Hopf bifurcations and consider 
$\c_1 = \epsilon$ to be a parameter, and then find periodic solutions with Fourier modes in $\ell^1_{\epsilon}$.
This approach rewrites the function $G: \R^2 \times \ell^K \to \ell^1$ as a function $\tilde{F}_\epsilon : \R^2 \times \ell^K_0 \to \ell^1$, where 
we denote 
\[
\ell^K_0 := \ell^1_0 \cap \ell^K.
\]
% I AM ACTUALLY NOT SURE IF YOU WANT TO DEFINE THIS WITH RANGE IN $\ell^1$
% OR WITH DOMAIN IN $\ell^1_0$ ?? IT SEEMS TO DEPEND ON WHICH GLOBAL STATEMENT YOU WANT/NEED TO MAKE!?
\begin{definition}
We define the $\epsilon$-parameterized family of  functions $\tilde{F}_\epsilon: \R^2 \times \ell^K_0  \to \ell^1$ 
by 
\begin{equation}
\label{eq:fourieroperators}
\tilde{F}_{\epsilon}(\alpha,\omega, \tc) := 
\epsilon [i \omega + \alpha e^{-i \omega}] \e_1 + 
( i \omega K^{-1} + \alpha U_{\omega}) \tc + 
\epsilon^2 \alpha e^{-i \omega}  \e_2  +
\alpha \epsilon L_\omega \tc + 
\alpha  [ U_{\omega} \tc] * \tc ,
\end{equation}
where
$L_\omega : \ell^1_0 \to \ell^1$ is given by
\[
   L_{\omega} := \sigma^+( e^{- i \omega} I + U_{\omega}) + \sigma^-(e^{i \omega} I + U_{\omega}),
\]
with $I$ the identity and  $\sigma^\pm$ the shift operators on $\ell^1$:
\begin{alignat*}{2}
\left[ \sigma^- a \right]_k &:=  a_{k+1}  , \\
\left[ \sigma^+ a \right]_k &:=  a_{k-1}  &\qquad&\text{with the convention } \c_0=0.
\end{alignat*}
The operator $ L_\omega$ is discontinuous in $\omega$ and $ \| L_\omega \| \leq 4$. 
\end{definition} 

%The maps $ \sigma^{+}$ and $ \sigma^-$ are shift up and shift down operators respectively. 
We reformulate Theorem~\ref{thm:FourierEquivalence1}  in terms of the map  $\tilde{F}$. 
We note that it follows from Lemma~\ref{l:analytic} and 
%\marginpar{Reformulate}
%one's choice of  
Equation~\eqref{eq:FourierSequenceEquation}  
%or Equation ~\eqref{eq:fourieroperators},
that the Fourier coefficients of any periodic solution of~\eqref{eq:Wright} lie in $\ell^K$.
These observations are summarized in the following theorem.
\begin{theorem}
\label{thm:FourierEquivalence2}
	Let $ \epsilon \geq 0$,  $\tc \in \ell^K_0$, $\alpha>0$ and $ \omega >0$. 
	Define $y: \R\to \R$ as 
\begin{equation}\label{e:ytc}
	y(t) = 
	\epsilon \left( e^{i \omega t }  + e^{- i \omega t }\right) 
	+  \sum_{k = 2}^\infty   \tc_k e^{i \omega k t }  + \tc_k^* e^{- i \omega k t } .
\end{equation}
%	and suppose that $ y(t) > -1$. 
	Then $y(t)$ solves~\eqref{eq:Wright} if and only if $\tilde{F}_{\epsilon}( \alpha , \omega , \tc) = 0$. 
	Furthermore, up to time translation, any periodic solution of~\eqref{eq:Wright} with period $2\pi/\omega$ is described by a Fourier series of the form~\eqref{e:ytc} with $\epsilon \geq 0$ and $\tc \in \ell^K_0$.
\end{theorem}


%We note that for $\epsilon>0$ such solutions are truly periodic, while for $\epsilon=0$ a zero of $\tilde{F}_\epsilon$ may either correspond to a periodic solution or to the trivial solution $y(t) \equiv 0$. 



% \begin{proof}
%  By Proposition \ref{thm:FourierEquivalence1}, it suffices to show that $\tilde{F}(\alpha,\omega,c) =0$ is equivalent to Equation \ref{eq:FourierSequenceEquation} being satisfied for $k \geq 1$.
%  Since Equation \ref{eq:FourierSequenceEquation} is equivalent to Equation \ref{eq:FHat}, we expand  Equation \ref{eq:FHat} by writing $ \hat{c} = \hat{\epsilon } + c$  where $ \hat{\epsilon} := (\epsilon,0,0,\dots) \in \ell^1$ as below:
%  \begin{equation}
%  0=  \left( i \omega K^{-1} + \alpha U_\omega \right) (\hat{\epsilon}+ c) + \alpha \left[U_\omega \, (\hat{\epsilon}+ c) \right] * (\hat{\epsilon}+ c) \label{eq:Intial}
%  \end{equation}
%  The RHS of Equation \ref{eq:Intial} is $ \tilde{F}(\alpha,\omega,c)$, so the theorem is proved.
% \end{proof}



Since we want to analyze a Hopf bifurcation, we will want to solve $\tilde{F}_\epsilon = 0$ for small values of~$\epsilon$. 
However, at the bifurcation point, $ D \tilde{F}_0(\pp  ,\pp , 0)$ is not invertible.
In order for our asymptotic analysis to be non-degenerate,
we work with a rescaled version of the problem. To this end, for any $\epsilon >0$, we rescale both $\tc$ and $\tilde{F}$ as follows. Let $\tc = \epsilon c$ and 
\begin{equation}\label{e:changeofvariables}
  \tilde{F}_\epsilon (\alpha,\omega,\epsilon c) = \epsilon F_\epsilon (\alpha,\omega,c).
\end{equation}
For $\epsilon>0$ the problem then reduces to finding zeros of 
\begin{equation}
\label{eq:FDefinition}
	F_\epsilon(\alpha,\omega, c) := 
	[i \omega + \alpha e^{-i \omega}] \e_1 + 
	( i \omega K^{-1} + \alpha U_{\omega}) c + 
	\epsilon \alpha e^{-i \omega} \e_2  +
	\alpha \epsilon L_\omega c + 
	\alpha \epsilon [ U_{\omega} c] * c.
\end{equation}
We denote the triple $(\alpha,\omega,c) \in \R^2 \times \ell^1_0$ by $x$.
To pinpoint the components of $x$ we use the projection operators
\[
   \pi_\alpha x = \alpha, \quad \pi_\omega x = \omega, \quad 
  \pi_c x = c \qquad\text{for any } x=(\alpha,\omega,c).
\]

After the change of variables~\eqref{e:changeofvariables} we now have an invertible Jacobian $D F_0(\pp  ,\pp , 0)$ at the bifurcation point.
On the other hand, for $\epsilon=0$ the zero finding problems for $\tilde{F}_\epsilon$ and $F_\epsilon$ are not equivalent. 
However, it follows from the following lemma that any nontrivial periodic solution having $ \epsilon=0$ must have a relatively large size when $ \alpha $ and $ \omega $ are close to the bifurcation point. 

\begin{lemma}\label{lem:Cone}
	Fix $ \epsilon \geq 0$ and $\alpha,\omega >0$. 
	Let
	\[
	b_* :=  \frac{\omega}{\alpha} - \frac{1}{2} - \epsilon  \left(\frac{2}{3}+ \frac{1}{2}\sqrt{2 + 2 |\omega-\pp| } \right).
	\]
Assume that $b_*> \sqrt{2} \epsilon$. 
Define
% \begin{equation*}%\label{e:zstar}
% 	z^{\pm}_* :=b_* \pm \sqrt{(b_*)^2- \epsilon^2 } .
% \end{equation*}
% \note[J]{Proposed change to match Lemma E.4}
\begin{equation}\label{e:zstar}
z^{\pm}_* :=b_* \pm \sqrt{(b_*)^2- 2 \epsilon^2 } .
\end{equation}
If there exists a $\tc \in \ell^1_0$ such that $\tilde{F}_\epsilon(\alpha, \omega,\tc) = 0$, then \\
\mbox{}\quad\textup{(a)} either $ \|\tc\| \leq  z_*^-$ or $ \|\tc\| \geq z_*^+  $.\\
\mbox{}\quad\textup{(b)} 
$ \| K^{-1} \tc \| \leq (2\epsilon^2+ \|\tc\|^2) / b_*$. 
\end{lemma}
\begin{proof}
	The proof follows from Lemmas~\ref{lem:gamma} and~\ref{lem:thecone} in Appendix~\ref{appendix:aprioribounds}, combined with the observation that
$\frac{\omega}{\alpha} - \gamma \geq b_*$,
% \[
%   \frac{\omega}{\alpha} - \gamma \geq b_*
%  \qquad\text{for all }
% | \alpha - \pp| \leq r_\alpha \text{ and } 
%   | \omega - \pp| \leq r_\omega.
% \]
with $\gamma$ as defined in Lemma~\ref{lem:gamma}.
\end{proof}

\begin{remark}\label{r:smalleps}
We note that for $\alpha < 2\omega$
\begin{alignat*}{1}
z^+_* &\geq   \frac{2 \omega - \alpha}{\alpha} 
- \epsilon \left(4/3+\sqrt{2 + 2 |\omega-\pp| } \, \right) + \cO(\epsilon^2)
\\[1mm]
z^-_* & \leq   \cO(\epsilon^2)
\end{alignat*}
for small $\epsilon$. 
Hence Lemma~\ref{lem:Cone} implies that for values of $(\alpha,\omega)$ near $(\pp,\pp)$ any solution has either $\|\tc\|$ of order 1 or $\|\tc\| =  \cO(\epsilon^2)$. 
The asymptotically small term bounding $z_*^-$ is explicitly calculated in Lemma~\ref{lem:ZminusBound}. 
A related consequence is that for $\epsilon=0$ there are no nontrivial solutions 
of $\tilde{F}_0(\alpha,\omega,\tc)=0$ with 
$\| \tc \| < \frac{2 \omega - \alpha}{\alpha} $. 
\end{remark}

\begin{remark}\label{r:rhobound}
In Section~\ref{s:contraction} we will work on subsets of $\ell^K_0$ of the form
\[
  \ell_\rho := \{ c \in \ell^K_0 : \|K^{-1} c\| \leq \rho \} .
\]
Part (b) of Lemma~\ref{lem:Cone} will be used in Section~\ref{s:global} to guarantee that we are not missing any solutions by considering $\ell_\rho$ (for some specific choice of $\rho$) rather than the full space $\ell^K_0$.
In particular, we infer from Remark~\ref{r:smalleps} that  small solutions (meaning roughly that $\|\tc\| \to 0$ as $\epsilon \to 0$)
satisfy $\| K^{-1} \tc \| = \cO(\epsilon^2)$.
\end{remark}

The following theorem guarantees that near the bifurcation point the problem of finding all periodic solutions is equivalent to considering the rescaled problem $F_\epsilon(\alpha,\omega,c)=0$.
\begin{theorem}
\label{thm:FourierEquivalence3}
\textup{(a)} Let $ \epsilon > 0$,  $c \in \ell^K_0$, $\alpha>0$ and $ \omega >0$. 
	Define $y: \R\to \R$ as 
\begin{equation}\label{e:yc}
	y(t) = 
	\epsilon \left( e^{i \omega t }  + e^{- i \omega t }\right) 
	+ \epsilon  \sum_{k = 2}^\infty   c_k e^{i \omega k t }  + c_k^* e^{- i \omega k t } .
\end{equation}
%	and suppose that $ y(t) > -1$. 
	Then $y(t)$ solves~\eqref{eq:Wright} if and only if $F_{\epsilon}( \alpha , \omega , c) = 0$.\\
\textup{(b)}
Let $y(t) \not\equiv 0$ be a periodic solution of~\eqref{eq:Wright} of period $2\pi/\omega$
 with Fourier coefficients $\c$.
Suppose $\alpha < 2\omega$ and $\| \c \| < \frac{2 \omega - \alpha}{\alpha} $.
Then, up to time translation, $y(t)$ is described by a Fourier series of the form~\eqref{e:yc} with $\epsilon > 0$ and $c \in \ell^K_0$.
\end{theorem}

\begin{proof}
Part (a) follows directly from Theorem~\ref{thm:FourierEquivalence2} and the  change of variables~\eqref{e:changeofvariables}.
To prove part (b) we need to exclude the possibility that there is a nontrivial solution with $\epsilon=0$. The asserted bound on the ratio of $\alpha$ and $\omega$ guarantees, by Lemma~\ref{lem:Cone} (see also Remark~\ref{r:smalleps}), that indeed $\epsilon>0$ for any nontrivial solution. 
\end{proof}

We note that in practice (see Section~\ref{s:global}) a bound on $\| \c \|$ is derived from a bound on $y$ or $y'$ using Parseval's identity.

\begin{remark}\label{r:cone}
It follows from Theorem~\ref{thm:FourierEquivalence3} and Remark~\ref{r:smalleps} that for values of $(\alpha,\omega)$ near $(\pp,\pp)$ any reasonably bounded solution satisfies $\| c\| =  O(\epsilon)$ as well as $\|K^{-1} c \| = O(\epsilon)$ asymptotically (as $\epsilon \to 0$).
These bounds will be made explicit (and non-asymptotic) for specific choices of the parameters in Section~\ref{s:global}.
\end{remark}

% We are able to rule out such large amplitude solutions using global estimates such as those in \cite{neumaier2014global}.
% Hence, near the bifurcation point, the problem of describing periodic solutions of~\eqref{eq:Wright} reduces to studying the family of zeros finding problems $F_\epsilon=0$.





%Specifically, if a solution having $ \epsilon = 0$ does in fact correspond to a nontrivial periodic solution and $\alpha  < 2\omega $, then $ \| \tilde{c} \| > 2 \omega \alpha^{-1} -1$. 
%%PERHAPS THIS NEEDS A FORMULATION AS A THEOREM AS WELL?
%%IN OTHER WORDS: ARE WE SURE WE HAVE FOUND ALL ZEROS OF $\tilde{F}_0$, I.E. ALL SOLUTIONS WITH $\epsilon=0$ NEAR THE BIFURCATION POINT? AFTER RESCALING THESE ARE INVISIBLE?
%%THERE SHOULD BE A STATEMENT ABOUT THIS SOMEWHERE! EITHER HERE OR SOME





We finish this section by defining a curve of approximate zeros $\bx_\epsilon$ of $F_\epsilon$ 
(see \cite{chow1977integral,hassard1981theory}). 
%(see \cite{chow1977integral,morris1976perturbative,hassard1981theory}). 


\begin{definition}\label{def:xepsilon}
Let
\begin{alignat*}{1}
	\balpha_\epsilon &:= \pp + \tfrac{\epsilon^2}{5} ( \tfrac{3\pi}{2} -1)  \\
	\bomega_\epsilon &:= \pp -  \tfrac{\epsilon^2}{5} \\
	\bc_\epsilon 	 &:= \left(\tfrac{2 - i}{5}\right) \epsilon \,  \e_2 \,.
\end{alignat*}
We define the approximate solution 
$ \bx_\epsilon := \left( \balpha_\epsilon , \bomega_\epsilon  , \bc_\epsilon \right)$
for all $\epsilon \geq 0$.
\end{definition}

We leave it to the reader to verify that both 
 $F_\epsilon(\pp,\pp,\bc_{\epsilon})=\cO(\epsilon^2)$ and $F_\epsilon(\bx_\epsilon)=\cO(\epsilon^2)$.
%%%	
%%%	
%%%	}{Better like this?}
%%%\annote[J]{ $F_\epsilon(\bx_0)=\cO(\epsilon^2)$ and $F_\epsilon(\bx_\epsilon)=\cO(\epsilon^2)$.}{I think we'd still need the $ \bar{c}_\epsilon$ term in $\bar{x}_0$ to be of order $ \epsilon$.}
%%%\remove[JB]{We show in Proposition A.1
%%%%\ref{prop:ApproximateSolutionWorks} 
%%% that any $ x \in \R^2 \times \ell^1_0$ which is $ \cO(\epsilon^2)$ close to $ \bar{x}_\epsilon $ will yield the estimate $F_\epsilon(x) = \cO(\epsilon^2)$.
%%%Hence choosing $\{ \pp , \pp, \bar{c}_\epsilon\}$ as our approximate solution would also have been a natural choice for performing an $\cO(\epsilon^2)$ analysis and would have simplified several of our calculations.
%%%However,} 
%%%
We choose to use the more accurate approximation 
for the $ \alpha$ and $ \omega $ components to improve our final quantitative results. 














%
% Values for $ (\alpha, \omega,c)$ which approximately solve $\tilde{F}(\alpha,\omega,c) = 0$  are computed in  \cite{chow1977integral,morris1976perturbative,hassard1981theory} and are as follows:
%  \begin{eqnarray}
%  \tilde{\alpha}( \epsilon) &:=& \pi /2 + \tfrac{\epsilon^2}{5} ( \tfrac{3\pi}{2} -1) \nonumber \\
%  \tilde{\omega}( \epsilon) &:=& \pi /2 -  \tfrac{\epsilon^2}{5} \label{eq:ScaleApprox} \\
%  \tc(\epsilon) 	  &:=& \{ \left(\tfrac{2 - i}{5}\right)  \epsilon^2 , 0,0, \dots \} \nonumber
%  \end{eqnarray}
% In Appendix \ref{sec:OperatorNorms} we illustrate an alternative method for deriving this approximation.
%
%
%
%
% We want to solve $ \tilde{F}(\alpha , \omega, \hat{c}) =0$ for small values of $ \epsilon$.
% However $ D \tilde{F}(\alpha , \omega , c)$ is not invertible at $ ( \pp , \pp , 0)$ when $ \epsilon = 0$.
% In order for our asymptotic analysis to be non-degenerate, we need to make the change of variables $ c \mapsto \epsilon c$.
% Under this change of variables, we define the function $ F$ below so that $ \tilde{F}(\alpha , \omega , \epsilon c) =\epsilon  F( \alpha , \omega , c)$.
%
%
%
% \begin{definition}
% Construct an $\epsilon$-parameterized family of densely defined functions  $F : \R^2 \oplus \ell^1 / \C \to \ell^1$ by:
% \begin{equation}
% \label{eq:FDefinition}
% 	F(\alpha,\omega, c) :=
% 	[i \omega + \alpha e^{-i \omega}]_1 +
% 	( i \omega K^{-1} + \alpha U_{\omega}) c +
% 	[\epsilon \alpha e^{-i \omega}]_2  +
% 	\alpha \epsilon L_\omega c +
% 	\alpha \epsilon [ U_{\omega} c] * c.
% \end{equation}
% \end{definition}

%%
%%
%%\begin{corollary}
%%	\label{thm:FourierEquivalence3}
%%	Fix $ \epsilon > 0$, and $ c \in \ell^1 / \C $, and $ \omega >0$. Define $y: \R\to \R$ as 
%%	\[
%%	y(t) = 
%%	\epsilon \left( e^{i \omega t }  + e^{- i \omega t }\right) 
%%	+  \epsilon  \left( \sum_{k = 2}^\infty   c_k e^{i \omega k t }  + \overline{c}_k e^{- i \omega k t } \right) 
%%	\]
%%	and suppose that $ y(t) > -1$. 
%%	Then $y(t)$ solves Wright's equation at parameter $ \alpha > 0 $ if and only if $ F( \alpha , \omega , c) = 0$ at parameter $ \epsilon$. 
%%	
%%	
%%	
%%\end{corollary}
%%
%%
%%\begin{proof}
%%	Since $ \tilde{F}(\alpha,\omega, \epsilon c) = \epsilon F( \alpha , \omega , c)$, the result follows from Theorem \ref{thm:FourierEquivalence2}.
%%\end{proof}

% If we can find $(\alpha , \omega, c)$ for which $ F( \alpha , \omega,c)=0$ at parameter $\epsilon$, then $ \tilde{F}(\alpha ,\omega, c)=0$.
% By Theorem \ref{thm:FourierEquivalence2} this amounts to finding a periodic solution to Wright's equation.
% Lastly, because we have performed the change of variables $ c \mapsto \epsilon c$, we need to  apply this change of variables to our approximate solution as well.
%
% \begin{definition}
% 	Define the approximate solution $ x( \epsilon) = \left\{ \alpha(\epsilon ) , \omega ( \epsilon ) , c(\epsilon) \right\}$ as below,  where $c(\epsilon) = \{ c_2( \epsilon) , 0 ,0 , \dots\} $.
% 	We may also write $ x_\epsilon = x(\epsilon) $.
% 	\begin{eqnarray}
% 	\alpha( \epsilon) &:=& \pi /2 + \tfrac{\epsilon^2}{5} ( \tfrac{3\pi}{2} -1) \nonumber \\
% 	\omega( \epsilon) &:=& \pi /2 -  \tfrac{\epsilon^2}{5} \label{eq:Approx} \\
% 	c_2(\epsilon) 	  &:=& \left(\tfrac{2 - i}{5}\right) \epsilon \nonumber
% 	\end{eqnarray}
%
% \end{definition}





\section{Distributed graph  clustering\label{sec:graphmsg}}

In this section we study distributed graph clustering.   We assume that the vertex set of  the input graph $G=(V,E)$ can be partitioned into $k$ clusters, where vertices in each cluster $S$ are highly connected to each other, and there are fewer edges between $S$ and $V\setminus S$.  To formalize this notion, 
 we  define the \emph{$k$-way expansion constant} of graph $G$ by 
\[
\rho(k)\triangleq \min_{\mbox{\footnotesize partition $A_1,\ldots, A_k$}}\max_{1\leq i\leq k}\phi_G(A_i).
\] Notice that a graph $G$ has $k$ clusters if the value of $\rho(k)$ is small. It was shown in \cite{conf/stoc/LeeGT12} that $\rho(k)$ closely relates to  $\lambda_k(\mathcal{L}_G)$ by the following higher-order Cheeger inequality:
\begin{equation*}
\frac{\lambda_k(\mathcal{L}_G)}{2}\leq \rho(k)\leq O(k^2) \sqrt{\lambda_k(\mathcal{L}_G)}.
\end{equation*}
Hence, a large gap between $\lambda_{k+1}(\mathcal{L}_G)$ and $\rho(k)$ implies (i) the existence of 
a $k$-way partition $\{S_i\}_{i=1}^k$ such that every $S_i$ has small conductance
 $\phi_G(S_i)\leq \rho(k)$, and (ii) any $(k+1)$-way partition of $G$ contains a subset with  high conductance $\rho(k+1)\geq \lambda_{k+1}(\mathcal{L}_G)/2$. Therefore, a large gap between $\lambda_{k+1}(\mathcal{L}_G)$ and $\rho(k)$ ensures that $G$ has \emph{exactly} $k$ clusters. In the following, we assume that 
\[
\Upsilon\triangleq \lambda_{k+1}(\mathcal{L}_G)/\rho(k)=\Omega(k^3)
\]  
to ensure that the input graph $G$ has exactly $k$ clusters. The same assumption has been used 
 in the literature for studying graph clustering in the centralized setting~\cite{PSZ15}.

Both algorithms presented in the section are based on the following \emph{spectral clustering} algorithm: (i) compute the  $k$ eigenvectors $f_1,\ldots, f_k$ of $\mathcal{L}_G$ associated with $\lambda_1(\mathcal{L}_G),\ldots,\lambda_k(\mathcal{L}_G)$; (ii) embed every vertex $v$ to a point in $\mathbb{R}^k$ through the embedding 
\[
F(v)=\frac{1}{\sqrt{d_v}}\cdot(f_1(v),\ldots, f_k(v));
\]
 (iii) run $k$-means on the embedded points $\{F(v)\}_{v\in V}$, and group the vertices of $G$ into $k$ clusters according to the output of $k$-means.

\subsection{The message passing model\label{sec:gcmessage}}

We assume  the edges of the input graph $G=(V,E)$ are arbitrarily allocated among $s$ sites $\mathcal{P}_1,\cdots, \mathcal{P}_s$, and we use  $E_i$ to denote the edge set maintained by site $\mathcal{P}_i$. Our proposed algorithm consists of two steps: (i) every  $\mathcal{P}_i$ computes a linear-sized $(1+\eps)$-spectral sparsifier $H_i$ of $G_i\triangleq (V, E_i)$, 
for a small constant $\eps\leq 1/10$,
and sends the edge set of  $H_i$, denoted by $E_i'$, to the coordinator; (ii) the coordinator runs a spectral clustering algorithm on the union of received graphs $H\triangleq \left(V,\bigcup_{i=1}^k E_i' \right)$. The theorem  below summarizes the performance of this algorithm, and shows the approximation guarantee of this algorithm is as good as the provable guarantee of spectral clustering known in the centralized setting, which is shown in the lemma below.


\begin{lem}[\cite{PSZ15}]\label{lem:PSZ}
Let $G$ be a graph satisfying the condition $\Upsilon=\Omega(k^3)$, and $k\in \mathbb{N}$.
Then, a spectral clustering algorithm outputs sets $A_1,\ldots, A_k$ such that $\vol(A_i\triangle S_i)= O\left(k^3\cdot\Upsilon^{-1}\cdot\vol(S_i)\right)$ holds for any $1\leq i\leq k$, where $S_i$ is the optimal cluster corresponding to $A_i$.
\end{lem}



\begin{thm}\label{thm:gcss}
Let $G=(V,E)$ be an $n$-vertex graph with $\Upsilon=\Omega(k^3)$, and suppose the edges of $G$ are
arbitrarily allocated among $s$ sites.  Assume  $S_1,\cdots, S_k$ is an optimal partition that achieves $\rho(k)$.
Then, the algorithm above computes  a partition $A_1,\ldots, A_k$ satisfying $\vol(A_i\triangle S_i)= O\left( k^3\cdot\Upsilon^{-1}\cdot\vol(S_i)\right)$  for any $1\leq i\leq k$. The total communication cost of this algorithm is $\tilde{O}(ns)$ bits. 
\end{thm}


\begin{proof}
By the definition of the Laplacian matrix, we have that $L_G=\sum_{i=1}^s L_{G_i}$. Since every $H_i$ is a $(1+\eps)$-spectral sparsifier of graph $G_i$,   we have that 
$(1-\eps) L_{H_i}\preceq L_{G_i} \preceq (1+\eps) L_{H_i}$.
This implies that $
(1-\eps) L_{H}\preceq L_{G} \preceq (1+\eps) L_{H}$,
by the definition of $H_i$ and graph Laplacians. 
Now we show that our assumption on $\Upsilon$ is preserved in $H$. By \lemref{sssubset}, we have for any $1\leq i \leq k$ that 
$\phi_H(S_i)\in \left(\frac{1}{2},2\right) \phi_G(S_i)$,
which implies that $S_i$ has low conductance in $H$, and $\rho_H(k)\in \left(\frac{1}{2},2\right)\rho_G(k)$. To show that $\lambda_k(\mathcal{L}_H)$ is a constant approximation of $\lambda_k(\mathcal{L}_G)$, notice that \[(1-c)\cdot x^{\rot} L_Gx \leq x^{\rot} L_H x \leq (1+c)\cdot x^{\rot} L_G x\] holds for any $x\in\mathbb{R}^n$. Hence
it holds for any $x\in\mathbb{R}^n$ that 
\[
(1-\varepsilon)\cdot x^{\rot}D_G^{-1/2} L_GD_G^{-1/2}x \leq x^{\rot} D_G^{-1/2}L_H D_G^{-1/2}x \leq (1+\varepsilon)\cdot x^{\rot}D_G^{-1/2} L_GD_G^{-1/2} x.\]
Since $D_G^{-1/2} L_GD_G^{-1/2}=\mathcal{L}_G$ and  $\frac{1}{2} D_G^{-1}\preceq D_H^{-1}\preceq 2 D_G^{-1}$, we have that $\lambda_i\left(\mathcal{L}_H\right)=\Theta\left(\lambda_i\left(\mathcal{L}_G\right)\right)$, and the  assumption on $\Upsilon$ in $H$ is preserved from $G$ up to a constant factor.  By \lemref{PSZ}, the output of a spectral clustering algorithm on $H$ satisfies the claimed properties.
The total communication cost of $\tilde{O}(ns)$  bits follows from the fact that every $H_i$ has  $O(n)$ edges.
\end{proof}

Next we show that the communication cost of our proposed algorithm is optimal up to a logarithmic factor. 
 Our analysis is based on a reduction from graph clustering to the Multiparty Set-Disjointness problem~(\DISJN): for any $s$ sites  $\mathcal{P}_1,\ldots, \mathcal{P}_s$, where each  $\mathcal{P}_i$ has a set $S_i\subseteq [n]$, let $X_i = (X_i^1, \ldots, X_i^n)$ be the characteristic vector of $S_i$, and let $X = (X_1, \ldots, X_s)$ be the input matrix with $X_i$ being the $i$-th row.  Let $X^j = (X_1^j, \ldots, X_s^j)$ be the $j$-th column of the input matrix $X$. 
We define a function \ONE\ on an $s$-bit vector $Y = (Y_1, \ldots, Y_s)$ as
$
\text{\ONE} (Y)  =  \bigwedge_{i \in [s]} Y_i$, and $ \text{\DISJN}(X) = \bigvee_{j \in [n]} \text{\ONE} (X^j)$. Then the \DISJN\ problem asks the value of $\text{\DISJN}(X)$.
We  introduce two hard input distributions for \ONE\ and \DISJN\ respectively.  
\begin{itemize}
\item {\em Hard input distribution $\nu$ on $Y \in \{0,1\}^s$ for \ONE:} with probability $1/2$, we choose each $Y_i\ (i \in [s])$ to be $0$ or $1$ with equal probability; with probability $1/4$ we choose $Y$ to be an all-$1$ vector; and with the remaining probability $1/4$ we choose $Y$ to be a random vector with $n-1$ coordinates being $1$'s and a random coordinate being $0$.
\item {\em Hard input distribution $\mu_n$ on $X \in \{0,1\}^{s \times n}$ for \DISJN:}  For each $j \in [n]$, we choose $X^j \sim \nu$.  
\end{itemize}
%We will use the following results by Braverman et al.



\begin{thm}[\cite{beopv13}] \label{thm:DISJ} 
It holds that $\mathsf{IC}_{0.49, \nu}(\text{\ONE}) = \Omega(s)$, and 
  $\mathsf{IC}_{0.49, \nu}(\text{\DISJN}) = \Omega(sn)$.
\end{thm}



\begin{lem} \label{lem:DISJ}
In the message passing model, any randomized algorithm that computes \DISJN\ correctly with probability $0.9$ needs $\Omega(sn)$ bits of communication.
\end{lem}


\begin{proof}
The lemma follows from \thmref{DISJ} and Yao's minimax lemma.
\end{proof}

\begin{thm}\label{thm:lbmsg}
Let $G$ be an undirected graph with $n$ vertices, and suppose the edges of $G$ are distributed among $s$ sites. Then, any algorithm that correctly outputs a constant fraction of a cluster in $G$ requires $\Omega(ns)$ bits of communication. This holds even if each cluster has constant expansion.  
\end{thm}



\begin{proof}
Our proof is based on the reduction from graph clustering to the Multiparty Set-Disjointness problem~(\DISJN).
For any item $j$ and site $\mathcal{P}_i$, we set $X^j_i=0$ if item $j$ appears in site $\mathcal{P}_i$, and $X^j_i=1$ otherwise.
Then $\text{\DISJN}(X)=1$ if there is some item not appearing in any site.  Now we construct a graph $G$ based on the hard instance $X$ of 
$\mathsf{DISJ}_{n,s}$ as follows: initially,  graph $G$ consists of  $n$ isolated vertices $\ell_1,\ldots, \ell_n$, and $r$ isolated vertices $r_1,\ldots, r_s$. Then, we add an edge between $\ell_j$ and $r_i$ if item $j$ appears in site $\mathcal{P}_i$. With this construction, it is easy to see that $\text{\DISJN}(X)=0$ if every vertex $\ell_j$ is connected to some $r_i$, and $\text{\DISJN}(X)=1$ if there are some isolated vertices $\ell_j$. 


We will show that, when $\text{\DISJN}(X)=0$, our constructed  graph $G$ is a bipartite expander, i.e., $G$ has only $1$ cluster.   To prove this, notice that, from the hard input distribution $\mu$ on $Y\in\{0,1\}^s$  described above, with probability $1/2$ we choose each $Y_i(i\in [s])$ to be $0$ or $1$ with equal probability. This implies that, for any $\ell_i$ and $r_j$, there is an edge between $\ell_i$ and $r_j$ independently with probability at least $1/4$. By standard results on constructing expanders, this implies $G$ is a bipartite expander with constant expansion, and in particular is connected.   

On the other side, when $\text{\DISJN}(X)=1$, every isolated vertex $\ell_j$ itself forms a cluster with conductance $0$ and constant expansion, and the giant component of $G$ forms a cluster with conductance $0$ and constant expansion (since, as argued in the previous paragraph, it is a bipartite expander). Let $k$ be the number of connected  components in graph $G$. Then, $\rho(k)=0$, and our assumption on $\Upsilon=\lambda_{k+1}(\mathcal{L}_G)/\rho(k)=\Omega(k^3)$ holds trivially. Hence any clustering algorithm  that is able to find a constant fraction of each cluster in graph $G$ satisfying $\Upsilon=\Omega(k^3)$ can be used to solve \DISJN. The lower bound on the communication complexity of graph clustering follows from the lower bound for \DISJN.  
\end{proof}




As a  remark, it is easy to see that this  lower bound also holds for constructing spectral sparsifiers: for any $n\times n$ $\mathsf{PSD}$ matrix $A$ whose entries are arbitrarily distributed among $s$ sites, any distributed algorithm that constructs a $(1+\Theta(1))$-spectral sparsifier of $A$ requires $\Omega(ns)$ bits of communication. This follows since such a spectral sparsifier can be used to solve the spectral clustering problem. Spectral sparsification has played an important role in designing fast algorithms from different areas, e.g., machine learning, and  numerical linear algebra. Hence our lower bound result for constructing spectral sparsifiers may have applications to studying other distributed learning algorithms.


 






\subsection{The blackboard model\label{sec:bb}}
Next we present a graph clustering algorithm with $\tilde{O}(n+s)$ bits of communication cost in the blackboard model. 
Our result is based on the observation that a spectral sparsifier preserves the structure of clusters, which was used for proving \thmref{gcss}. So it suffices to design a  distributed algorithm for constructing a spectral sparsifier in the blackboard model. 


Our distributed algorithm  is based on constructing a chain of coarse sparsifiers~\cite{MP12}, which is described as follows: for any input \textsf{PSD} matrix $K$ with $\lambda_{\max}(K)\leq \lambda_u$ and all the non-zero eigenvalues of $K$  at least $\lambda_{\ell}$,   we define $d=\lceil \log_2(\lambda_u/\lambda_{\ell})\rceil$ and construct a chain of $d+1$ matrices
\begin{equation}\label{eq:chain}
[K(0),K(1),\ldots, K(d)],
\end{equation}
where $ 
\gamma(i)=\lambda_u/2^i$ and $K(i)= K+\gamma(i)I$.
Notice that in the chain above every $K(i-1)$
 is obtained by adding weights to the diagonal entries of $K(i)$, and 
 $K(i-1)$ approximates $K(i)$ as long as the weights added to the diagonal entries are  small.  
We will construct this chain recursively, so that $K(0)$ has heavy diagonal entries and can be approximated by a diagonal matrix. Moreover, since $K$ is the Laplacian matrix of a graph $G$, it is easy to see that $d=O(\log n)$ as long as the edge weights of $G$  are polynomially upper-bounded in $n$.


\begin{lem}[\cite{MP12}] \label{lem:MP12}
The chain~\eq{chain} satisfies the following relations: (1) 
 $K\preceq_r K(d)\preceq_r 2K$;
(2)  $K(\ell) \preceq K(\ell-1) \preceq 2K(\ell)$ for all $\ell\in\{1,\ldots, d\}$;
(3)  $K(0)\preceq 2\gamma(0)I\preceq 2K(0)$.
\end{lem}




%Such a chain can be constructed recursively so that $K(0)$ has heavy diagonal entries, and can be approximated by a diagonal matrix. Moreover, it is easy to prove that $d=O(\log n)$ if $K$ is the Laplacian matrix of a graph $G$, provided $G$'s edge weights are polynomially bounded.


Based on \lemref{MP12}, we will construct a chain of matrices  
\begin{equation}\label{eq:aptchain}
\left[ \tilde{K}(0), \tilde{K}(1),\ldots, \tilde{K}(d) \right]
\end{equation}
in the blackboard model, 
such that every $\tilde{K}(\ell)$ is a spectral sparsifier of $K(\ell)$, and 
every $\tilde{K}(\ell+1)$ can be constructed from $\tilde{K}(\ell)$.  The basic idea behind our construction is to use the relations among different $K(\ell)$ shown in \lemref{MP12} and the fact that, for any $K=B^{\rot}B$, sampling rows of $B$ with respect to their leverage scores can be used to obtain a matrix approximating $K$.


%It is well known that, for  $K(\ell)\triangleq B^{\rot}B$,  sampling rows $b_i$ of $B$ according to their leverage scores, denoted by$\tau_i\triangleq b_i^{\rot}K^{+}b_i$, will give a matrix approximating $K$. Formally, we assume that $\tilde{\tau}$ is the vector of leverage score overestimate for $B$'s rows such that $\tilde{\tau}_i\geq \tau_i$ for all $i\in[m]$, $0<\varepsilon<1$, and $c$ is a fixed constant. We sample every row $b_i$ with  probability $p_i=\min\left\{1, c\varepsilon^{-2}\tilde{\tau}_i\log n\right\}$, and define a diagonal matrix $W$ with $W_{i,i}=\frac{1}{p_i}$ with probability $p_i$, and $W(i,i)=0$ otherwise. Then, it holds with high probability that $(1-\varepsilon)K(\ell)\preceq \tilde{K}(\ell)=B^{\rot}WB \preceq (1+\varepsilon)K(\ell)$.Moreover, $W$ has $O\left( \|\tilde{\tau} \|_1\varepsilon^{-2}\log n \right)$ non-zeros with high probability. We prove in  Appendix~\ref{sec:b3} that the above sampling procedure can be implemented in the blackboard model, and this gives the following result:

\begin{thm}\label{thm:gcblackboard}
Let $G$ be an undirected graph on $n$ vertices, where the edges of $G$ are allocated among $s$ sites, and the edge weights are polynomially upper bounded in $n$. Then, a spectral sparsifier of $G$ can be constructed with $\tilde{O}(n+s)$ bits of communication in the blackboard model. That is, the chain \eq{aptchain} can be constructed with $\tilde{O}(n+s)$ bits of communication in the blackboard model.
\end{thm}



\begin{proof} Let $K= B^{\rot}B$  be the Laplacian matrix of the underlying graph $G$, where $B\in\mathbb{R}^{m\times n}$ is the  edge-vertex incidence matrix of $G$. We will  prove  that 
 every $\tilde{K}(i+1)$ can be constructed based on $\tilde{K}(i)$ with  $\tilde{O}(n+s)$ bits of communication. This implies that $\tilde{K}(d)$, a $(1+\varepsilon)$-spectral sparsifier of $K$, can be constructed with $\tilde{O}(n+s)$ bits of communication, as the length of the chain $d=O(\log n)$.  
 
 
 First of all, notice that  $\lambda_u\leq 2n$, and the value of $n$ can be obtained with communication cost $\tilde{O}(n+s)$~(different sites sequentially write the new IDs of the vertices on the blackboard). In the following  we assume that $\lambda_u$ is the upper bound of $\lambda_{\max}$ that we actually obtained in the blackboard.
 
\emph{Base case of $\ell=0$:} 
 By definition, $K(0)=K+\lambda_u\cdot  I$, and 
$
 \frac{1}{2}\cdot K(0)\preceq \gamma(0)\cdot I \preceq K(0)$, 
due to Statement~3 of  \lemref{MP12}.  Let $\oplus$ denote appending the rows of one matrix to another. We define $B_{\gamma(0)}=B\oplus \sqrt{\gamma(0)}\cdot I$, and write $K(0)=K+\gamma(0)\cdot I=B^{\rot}_{\gamma(0)}B_{\gamma(0)}$. By defining $\tau_i=b_i^{\rot}\left(K(0)\right)^{\rot}b_i$ for each row of $B_{\gamma(0)}$, we have
$
\tau_i\leq b_i^{\rot}\left(\gamma(0)\cdot I\right) b_i\leq 2\cdot \tau_i$.
Let $\tilde{\tau}_i=b_i^{\rot}\left( \gamma(0)\cdot I \right)^{+}b_i$ be the leverage score of $b_i$ approximated using $\gamma(0)\cdot I$, and let $\tilde{\tau}$ be the vector of approximate leverage scores, with the leverage scores of the $n$ rows corresponding to $\sqrt{\gamma(0)}\cdot I$ rounded up to 1. Then, with high probability sampling $O(\varepsilon^{-2}n\log n)$ rows of $B$ will give a matrix $\tilde{K}(0)$ such that $
(1-\varepsilon)K(0)\preceq \tilde{K}(0)\preceq (1+\varepsilon)K(0)$.
Notice that, as every row of $B$ corresponds to an edge of $G$, the approximate leverage scores $\tilde{\tau}_i$ for different edges can be computed locally by different sites maintaining the edges, and the sites only need to send the information of the sampled edges to the blackboard, hence the communication cost is $\tilde{O}(n+s)$ bits.

\emph{Induction step:} We assume that 
$
(1-\varepsilon)K(\ell)\preceq_r\tilde{K}(\ell) \preceq_r (1+\varepsilon)K(\ell)$,
and the blackboard maintains the matrix $\tilde{K}(\ell)$. This implies that
$
(1-\varepsilon)/(1+\varepsilon)\cdot K(\ell)\preceq_r 1/(1+\varepsilon)\cdot \tilde{K}(\ell) \preceq_r K(\ell).
$
Combining this with Statement~2 of \lemref{MP12}, we have that 
\[
\frac{1-\varepsilon}{2(1+\varepsilon)}K(\ell+1)\preceq_r \frac{1}{2(1+\varepsilon)}\tilde{K}(\ell) \preceq K(\ell+1).
\]
We apply the same sampling procedure as in the base case, and obtain a matrix $\tilde{K}(\ell+1)$ such that $
(1-\varepsilon) K(\ell+1) \preceq_r\tilde{K}(\ell+1) \preceq_r (1+\varepsilon) K(\ell+1)$.
Notice that, since $\tilde{K}(\ell)$ is written on  the blackboard, the probabilities used for sampling individual edges can be computed locally by different sites, and in each round only the sampled edges will be sent to the blackboard in order for the blackboard to obtain $\tilde{K}(\ell+1)$. Hence, the total communication cost in each iteration is $\tilde{O}(n+s)$ bits. Combining this with the fact that the chain length $d=O(\log n)$ proves the theorem. 
\end{proof}




Combining \thmref{gcblackboard} and the fact that a spectral sparsifier preserves the structure of clusters, 
we obtain a distributed  algorithm in the blackboard model with total communication cost $\tilde{O}(n+s)$ bits, and the performance of our algorithm is the same as in the statement of \thmref{gcss}. Notice that $\Omega(n+s)$ bits of communication are needed for graph clustering in the blackboard model, since the output of a clustering algorithm contains $\Omega(n)$ bits of information and each site needs to communicate at least one bit. Hence the communication cost of our proposed algorithm is optimal up to a poly-logarithmic factor.

 


%\input{median-etc-lb}

\section{Convexification}\label{sec:geometric}
%
We investigate how to transform the SGP~\eqref{eq:min_rand}--\eqref{eq:rewcomputation} into a convex program by relaxing equality constraints and a lifting of variables of the SGP. 
%We first describe a special convex form of an SGP and then show how to convexify the SGP to obtain a feasible solution to the original problem.
%\subsection{Geometric programming}
A certain subclass of SGPs called \emph{geometric programs} (GPs) can be transformed into convex programs~\cite[\S{}2.5]{boyd2007tutorial} and solved efficiently.
A GP is an SGP of the form~\eqref{eq:nl_obj}--\eqref{eq:nl_eq} where $f,g_i\in \posys[\Var]$ and $h_j\in\monos[\Var]$.
We will refer to a constraint with posynomial or monomial function as a posynomial or monomial constraint, respectively.
%
%Basically, the objective and inequality constraints are posynomial while the equality constraints are monomials. Note there is an implicit constraint that all variables are positive. 
%
%Let $\Var=\{x_1,\ldots,x_n\}$ be a set of real-valued \emph{positive} variables. 
%A GP is of the form
%\begin{align}
%	\text{minimize} 		&\quad \posy_0\label{eq:geo_objective}\\
%	\text{subject to} 		&\notag\\
%	\forall i.\, 1\leq i\leq m 	&\quad \posy_i\leq 1\label{eq:geo_posy}\\
%	\forall i.\, 1\leq i\leq p 	&\quad \mono_i= 1\label{eq:geo_mono}
%\end{align}
%where $\mono_i$ for $1\leq i\leq m$ are \emph{monomials} over $\Var$ and $\posy_i$ for $0\leq i\leq p$ a \emph{posynomials} over $\Var$.
% All posynomials are upper bounded by one, all monomials are equal to one, and the objective function is a posynomial. Note that the objective can also be maximized in case it is a monomial. 
% \nj{Say something about the log transformation and how GPs can be solved efficiently by which solver.}
%
\subsection{Transformation and relaxation of equality constraints}
As discussed before, the SGP~\eqref{eq:min_rand}--\eqref{eq:rewcomputation} is not convex because of the presence of non-affine equality constraints. 
First observe the following transformation~\cite{boyd2007tutorial}:
\begin{align}
	f\leq h\Longleftrightarrow \frac{f}{h}\leq 1,\label{eq:geo_trans}
\end{align}
for $f\in\posys[\Var]$ and $h\in\monos [\Var]$. Note that monomials are strictly positive (Def.~\ref{def:posy}). This \emph{(division-)transformation} of $f\leq h$ yields a \emph{posynomial inequality constraint}.

We \emph{relax} all equality constraints of SGP~\eqref{eq:min_rand}--\eqref{eq:rewcomputation} that are not monomials to inequalities, then we apply the division-transformation wherever possible. Constraints \eqref{eq:lambda}, \eqref{eq:kappa}, \eqref{eq:well-defined_sched_rand}, \eqref{eq:well-defined_probs_rand}, \eqref{eq:probcomputation_rand}, and \eqref{eq:rewcomputation} are transformed to
\begin{align}
%			\text{minimize } &\quad f\label{eq:min_rand}\\
%			\text{subject to}\notag \\
							 &\quad \frac{p_{\sinit}}{\lambda}\leq 1\label{eq:lambda_geo},\\
							 &\quad \frac{c_{\sinit}}{\kappa}\leq 1\label{eq:kappa_geo},\\
			\forall s\in S.	&\quad \sum_{\act\in\Act(s)}\sched^{s,\act}\leq 1\label{eq:welldefined_sched_rand_geo},\\			
%			\forall s\in T.	 &\quad p_s=1\label{eq:targetprob_rand}\\
			\forall s\in S\, \forall\act\in\Act(s).	 &\quad \sum_{s'\in S}\probmdp(s,\act,s')\leq 1\label{eq:well-defined_probs_rand_geo},\\
			\forall s\in S\setminus T.	&\quad \frac{\sum\limits_{\act\in\Act(s)}\sigma^{s,\act}\cdot\sum\limits_{s'\in S}	\probmdp(s,\act,s')\cdot p_{s'}}{p_s}\leq 1\label{eq:probcomputation_rand_geo},\\
%			\forall s\in G.	 &\quad r_s=0\label{eq:targetrew}\\
			\forall s\in S\setminus G.	&\quad \frac{\sum\limits_{\act\in\Act(s)} \sigma^{s,\act} \cdot \Bigl(c(s,\act) + \sum\limits_{s'\in S}	\probmdp (s,\act,s') \cdot c_{s'}\Bigr)}{c_s}\leq 1 \label{eq:rewcomputation_geo}.
		\end{align}
%
These constraints are not necessarily posynomial inequality constraints because (as in Def.~\ref{def:pmdp}) we allow signomial expressions in the transition probability function $\pmdp$. Therefore, replacing \eqref{eq:lambda}, \eqref{eq:kappa}, \eqref{eq:well-defined_sched_rand}, \eqref{eq:well-defined_probs_rand}, \eqref{eq:probcomputation_rand}, and \eqref{eq:rewcomputation} in the SGP with~\eqref{eq:lambda_geo}--\eqref{eq:rewcomputation_geo} does not by itself convert the SGP to a GP.

\subsection{Convexification by lifting}\label{sec:lifting}
%
\begin{figure}[t]\centering
	\subfigure[signomial transition functions]{
		\scalebox{0.7}{\begin{tikzpicture}[scale=1, die/.style={inner sep=0,outer sep=0},nodestyle/.style={draw,circle},baseline=(s0)]
%    \draw[white, use as bounding box] (-1.7,-4.5) rectangle (1.7,0.3); 
    \node [die,state] (s0) at (0,0) {\large$s$};    
    \node [die] (s1) [state, on grid, left=3cm of s0] {\large$s_1$};
    \node [die] (s2) [state, on grid, above left=3cm of s0] {\large$s_2$};
    \node [die] (s3) [state, on grid, above right=3cm of s0] {\large$s_{n-1}$};
    \node [die] (s4) [state, on grid, right=3cm of s0] {\large$s_n$};
    \draw (s0) edge[->] node[auto] {\large$f_1$} (s1);
    \draw (s0) edge[->] node[auto] {\large$f_2$} (s2);
    \draw (s0) edge[->] node[right] {\large$f_{n-1}$} (s3);
    \draw (s0) edge[->] node[below] {\large$1-\sum_{i=1}^{n-1} f_i$} (s4);
%    \node [] (s1dots) [on grid, above=.3cm of s1] {$\cdots$};
%    \node [] (s2dots) [on grid, right=.4cm of s2] {$\ldots$};
%    \node [] (s3dots) [on grid, below=.3cm of s3] {$\cdots$};
%    \node [] (s4dots) [on grid, left=.4cm of s4] {$\ldots$};
	\node [] (dummy1) [on grid, right=0.7cm of s2] {};
	\node [] (dummy2) [on grid, right=0.7cm of dummy1] {};
	\node [] (dummy3) [on grid, right=0.7cm of dummy2] {};
	\node [] (dummy4) [on grid, right=0.7cm of dummy3] {};
	\node [] (dummy5) [on grid, right=0.7cm of dummy4] {};
    \node [] (transitiondots) [on grid, above=.1cm of dummy3] {\large$\ldots$};
    \draw (s0) edge[->] (dummy1);
	\draw (s0) edge[->] (dummy2);	
	\draw (s0) edge[->] (dummy3);
	\draw (s0) edge[->] (dummy4);
	\draw (s0) edge[->] (dummy5);
%    \draw [loosely dotted, very thick] (s2) -- (s3);
    \node[draw=white, rectangle, fit=(current bounding box), inner sep=3pt, outer sep=0pt] {};    
\end{tikzpicture}
}
	\label{fig:pmc_signomials}
	}\hspace{1cm}
	\subfigure[posynomial transition functions]{
		\scalebox{0.7}{\begin{tikzpicture}[scale=1, die/.style={inner sep=0,outer sep=0},nodestyle/.style={draw,circle},baseline=(s0)]
%    \draw[white, use as bounding box] (-1.7,-4.5) rectangle (1.7,0.3); 
    \node [die,state] (s0) at (0,0) {\large$s$};    
    \node [die] (s1) [state, on grid, left=3cm of s0] {\large$s_1$};
    \node [die] (s2) [state, on grid, above left=3cm of s0] {\large$s_2$};
    \node [die] (s3) [state, on grid, above right=3cm of s0] {\large$s_{n-1}$};
    \node [die] (s4) [state, on grid, right=3cm of s0] {\large$s_n$};
    \draw (s0) edge[->] node[auto] {\large$f_1$} (s1);
    \draw (s0) edge[->] node[auto] {\large$f_2$} (s2);
    \draw (s0) edge[->] node[right] {\large$f_{n-1}$} (s3);
    \draw (s0) edge[->] node[below] {\large$\bar f$} (s4);
%    \node [] (s1dots) [on grid, above=.3cm of s1] {$\cdots$};
%    \node [] (s2dots) [on grid, right=.4cm of s2] {$\ldots$};
%    \node [] (s3dots) [on grid, below=.3cm of s3] {$\cdots$};
%    \node [] (s4dots) [on grid, left=.4cm of s4] {$\ldots$};
%    \node [] (transitiondots) [on grid, above=1cm of s0] {\large$\ldots$};
	\node [] (dummy1) [on grid, right=0.7cm of s2] {};
	\node [] (dummy2) [on grid, right=0.7cm of dummy1] {};
	\node [] (dummy3) [on grid, right=0.7cm of dummy2] {};
	\node [] (dummy4) [on grid, right=0.7cm of dummy3] {};
	\node [] (dummy5) [on grid, right=0.7cm of dummy4] {};
    \node [] (transitiondots) [on grid, above=.1cm of dummy3] {\large$\ldots$};
    \draw (s0) edge[->] (dummy1);
	\draw (s0) edge[->] (dummy2);	
	\draw (s0) edge[->] (dummy3);
	\draw (s0) edge[->] (dummy4);
	\draw (s0) edge[->] (dummy5);
%    \draw [loosely dotted, very thick] (s2) -- (s3);
    \node[draw=white, rectangle, fit=(current bounding box), inner sep=3pt, outer sep=0pt] {};    
\end{tikzpicture}
}
	\label{fig:pmc_monomials}
	}
%	\subfigure[Knuth--Yao die with monomial transition probability function]{
%		\scalebox{0.7}{\vspace{-5ex}%\begin{tikzpicture}[scale=1, die/.style={inner sep=0,outer sep=0},nodestyle/.style={draw,circle},baseline=(s0)]
\begin{tikzpicture}[scale=1]
%    \draw[white, use as bounding box] (-1.7,-4.5) rectangle (1.7,0.3); 
	\tikzstyle{nodestyle}=[draw,circle]
	\tikzstyle{die}=[inner sep=0,outer sep=0]
    
    \node [nodestyle] (s0) at (0,0) {$s_0$};
    \node [] (leftdummy)  [on grid, left=1.2cm of s0] {};
    \node [] (rightdummy) [on grid, right=1.2cm of s0] {};
    \node [nodestyle] (s1) [on grid, below=1.1cm of leftdummy] {$s_1$};
    \node [nodestyle] (s2) [on grid, below=1.1cm of rightdummy] {$s_2$};
    \node [nodestyle] (s3) [on grid, below=1.3cm of s1, xshift=-0.5cm] {$s_3$};
        \node [nodestyle] (s4) [on grid, below=1.3cm of s1, xshift=0.5cm] {$s_4$};
    \node [nodestyle] (s5) [on grid, below=1.3cm of s2, xshift=-0.5cm] {$s_5$};
    
    \node [nodestyle] (s6) [on grid, below=1.3cm of s2, xshift=0.5cm] {$s_6$};
    
    
    \node[die, scale=1.5, below=0.6cm of s3, xshift=-0.3cm] (X1) {\drawdie{1}};
    \node[die, scale=1.5, right=0.21cm of X1, inner sep=0pt] (X2) {\drawdie{2}};
    \node[die, scale=1.5, right=0.21cm of X2] (X3) {\drawdie{3}};
    \node[die, scale=1.5, right=0.21cm of X3] (X4) {\drawdie{4}};
    \node[die, scale=1.5, right=0.21cm of X4] (X5) {\drawdie{5}};
    \node[die, scale=1.5, right=0.21cm of X5] (X6) {\drawdie{6}};
    
    \draw ($(s0)-(0.7,0)$) edge[->] (s0);
    \draw (s0) edge[->] node[right] {\scriptsize$p$} (s1);
    \draw (s0) edge[->] node[right] {\scriptsize$\bar p$} (s2);
    \draw (s1) edge[bend left, ->] node[left] {\scriptsize$q$} (s3);
    \draw (s1) edge[->] node[right] {\scriptsize$\bar q$} (s4);
    \draw (s3) edge[bend left, ->] node[left] {\scriptsize$p$} (s1);
    \draw (s3) edge[->] node[left] {\scriptsize$\bar p$} (X1);
    \draw (s4) edge[->] node[left] {\scriptsize$\bar p$} (X2);
    \draw (s4) edge[->] node[right] {\scriptsize$p$} (X3);
    \draw (s2) edge[bend left, ->] node[left] {\scriptsize$q$} (s5);
    \draw (s2) edge[->] node[right] {\scriptsize$\bar q$} (s6);
    \draw (s5) edge[bend left, ->] node[left] {\scriptsize$p$} (s2);
    \draw (s5) edge[->] node[left] {\scriptsize$\bar p$} (X4);
    \draw (s6) edge[->] node[left] {\scriptsize$\bar p$} (X5);
    \draw (s6) edge[->] node[right] {\scriptsize$p$} (X6);
    
    \node[draw=white, rectangle, fit=(current bounding box), inner sep=3pt, outer sep=0pt] {};
    
\end{tikzpicture}
}
%	\label{fig:die_mon}
\caption{Lifting of signomial transition probability function.}
\end{figure}

%
The relaxed equality constraints~\eqref{eq:well-defined_probs_rand_geo}--\eqref{eq:rewcomputation_geo} involving $\probmdp$ are signomial, rather than posynomial, because the parameters enter Problem~\ref{prob:pmdpsyn} in signomial form. Specifically, consider the relaxed equality constraint~\eqref{eq:probcomputation_rand_geo} at $s_0$ in Example~\ref{ex:die},
\begin{align}
	\label{eq:kys0_relaxed}
	\frac{p\cdot p_{s_1} + (1-p)\cdot p_{s_2}}{p_{s_0}} \leq 1.
\end{align}
The term $(1-p)\cdot p_{s_2}$ is signomial in $p$ and $p_{s_2}$. We \emph{lift} by introducing a new variable $\bar p = 1-p$, and rewrite~\eqref{eq:kys0_relaxed} as a posynomial inequality constraint and an equality constraint in the lifted variables:
\begin{align}
	\frac{p\cdot p_{s_1} + \bar{p} \cdot p_{s_2}}{p_{s_0}} \leq 1,
	\quad
	\bar p = 1-p.
\end{align}
%
%
We relax the (non-monomial) equality constraint to $p + \bar p \leq 1$. 
%\ip{is this what we do, or do we approximate $p + \bar p = 1$ as a monomial equality constraint?}
More generally, we restrict the way parameters occur in $\probmdp$ as follows. Refer to Fig.~\ref{fig:pmc_signomials}. For every state $s\in S$ and every action $\act\in\Act(s)$ we require that there exists at most one state $\bar s\in S$ such that $\probmdp(s,\act,\bar s)\in\signomis[\Paramvar]$ and $\probmdp(s,\act,s')\in\posys[\Paramvar]$ for all $s'\in S\setminus\{\bar s\}$. In particular, we require that
	\begin{align*}
		\underbrace{\probmdp(s,\act,\bar s)}_{\in\signomis[\Paramvar]} = 1 - \sum_{s'\in S\setminus\{\bar s\}} \underbrace{\probmdp({s,\act,s'})}_{\in\posys[\Paramvar]}\ .
	\end{align*}
This requirement is met by all benchmarks available at the \tool{PARAM}--webpage~\cite{param_website}.
In general, we lift by introducing a new variable $\bar p_{s,\act,\bar s}=\probmdp(s,\act,\bar s)$ for each such state $s\in S$; refer to Fig.~\ref{fig:pmc_monomials}.
We denote this set of \emph{lifting variables} by $L$. Lifting as explained above then creates a new transition probability function $\bar \probmdp$ where for every $s,s'\in S$ and $\act\in \Act$ we have $\bar\probmdp(s,\act,s')\in\posys[\Paramvar\cup L]$. 

We call the set of constraints obtained through transformation, relaxation, and lifting of every constraint of the SGP~\eqref{eq:lambda}--\eqref{eq:rewcomputation} as shown above the \emph{convexified constraints}. 
Any posynomial objective subject to the convexified constraints forms by construction a GP over the pMDP parameters $\Paramvar$, the SGP additional variables $W$, and the lifting variables $L$. 

\subsection{Tightening the constraints}
A solution of the GP as obtained in the previous section does not have a direct relation to the original SGP~\eqref{eq:min_rand}--\eqref{eq:rewcomputation}.
In particular, a solution to the GP may not have the relaxed constraints satisfied with equality. 
For \eqref{eq:welldefined_sched_rand_geo} and~\eqref{eq:well-defined_probs_rand_geo}, the induced parameter valuation and the scheduler are not well-defined, \ie, the probabilities may not sum to one. \sj{In particular, this makes it trivial to fulfill the GP, just assign 0 to all schedulers}
We need to relate the relaxed and lifted GP to Problem~\ref{prob:pmdpsyn}. By defining a \emph{regularization function} $F$ over all parameter and scheduler variables, we ensure that the constraints are satisfied with equality; enforcing well-defined probability distributions.
%\begin{align*}
%	F = \sum_{p\in \Paramvar} \frac{1}{p} + \sum_{\bar p\in L} \frac{1}{\bar p} + \sum_{s\in S,\act\in\Act} \frac{1}{\sigma_{s,\act}}
%\end{align*}
%  that is used to enforce well-defined probability distributions:
% : the pMDP parameters, the lifting variables, and the scheduler variables:. The function is given by:
%
%Let $M$ denote the set of variables that are used to enforce well-defined probability distributions. Specifically, $M$ consists of the scheduler and transition probability variables,
%\begin{itemize}
%	\item $\{ \sched^{s,\act} \mid s \in S \}$
%	\item $\{ \probmdp(s,\act,s') \mid s \in S, \act\in\Act \}$.
%\end{itemize}
%Define a regularization function
\begin{align}
	F = \sum_{p\in \Paramvar} \frac{1}{p} + \sum_{\bar p\in L} \frac{1}{\bar p} + \sum_{s\in S,\act\in\Act(s)} \frac{1}{\sigma_{s,\act}}\label{eq:regularization} \ .
\end{align}
The function $F$ is monotone in all its variables. We discard the original objective $f$ in~\eqref{eq:min_rand} and form a GP with the regularization objective $F$~\eqref{eq:regularization}:
\begin{align}
			\text{minimize } &\quad F\label{eq:min_rand_geofull}\\
			\text{subject to}\notag \\
							 &\quad \frac{p_{\sinit}}{\lambda}\leq 1\label{eq:lambda_geofull},\\
							 &\quad \frac{c_{\sinit}}{\kappa}\leq 1\label{eq:kappa_geofull},\\
			\forall s\in S.	&\quad \sum_{\act\in\Act(s)}\sched^{s,\act}\leq 1\label{eq:welldefined_sched_rand_geofull},\\
			\forall s\in S\, \forall\act\in\Act(s). &\quad \sched^{s,\act} \leq 1			\label{eq:sched_is_dist_geofull},\\					%			\forall s\in T.	 &\quad p_s=1\label{eq:targetprob_rand}\\
			\forall s\in S\, \forall\act\in\Act(s).	 &\quad \sum_{s'\in S}\bar\probmdp(s,\act,s')\leq 1\label{eq:well-defined_probs_rand_geofull},\\
				\forall s, s'\in S\, \forall\act\in\Act(s).	 &\quad\bar{\probmdp}(s,\act,s') \leq 1,	\label{eq:probs_is_prob_geofull}\\
	\forall s\in T.	&\quad p_s=1,															\label{eq:targetprob_rand_geofull}\\
			\forall s\in S\setminus T.	&\quad \frac{\sum\limits_{\act\in\Act(s)}\sigma^{s,\act}\cdot\sum\limits_{s'\in S}	\bar\probmdp(s,\act,s')\cdot p_{s'}}{p_s}\leq 1\label{eq:probcomputation_rand_geofull},\\
%			\forall s\in G.	 &\quad r_s=0\label{eq:targetrew}\\
			\forall s\in S\setminus G.	&\quad \frac{\sum\limits_{\act\in\Act(s)} \sigma^{s,\act} \cdot \Bigl(c(s,\act) + \sum\limits_{s'\in S}	\bar\probmdp (s,\act,s') \cdot c_{s'}\Bigr)}{c_s}\leq 1 \label{eq:rewcomputation_geofull}.
\end{align}
%
%We make a few remarks about the GP~\eqref{eq:min_rand_geofull}--\eqref{eq:rewcomputation_geofull}. 
Since the objective $F$~\eqref{eq:regularization} and the inequality constraints~\eqref{eq:welldefined_sched_rand_geofull} and~\eqref{eq:well-defined_probs_rand_geofull} are monotone in $\Paramvar$, $L$, and the scheduler variables, each optimal solution for a feasible problem satisfies them with equality. We obtain a well-defined scheduler $\sched$ and a valuation $u$ as in Problem 1. Note that variables from~\eqref{eq:targetrew} are explicitly excluded from the GP by treating them as constants.

The reachability probability constraints~\eqref{eq:probcomputation_rand_geofull} and cost constraints \eqref{eq:rewcomputation_geofull} need not be satisfied with equality. However, \eqref{eq:probcomputation_rand_geofull} is equivalent to 
\begin{align*}
	{p_s} \geq {\sum\limits_{\act\in\Act(s)}\sigma^{s,\act}\cdot\sum\limits_{s'\in S} \bar\probmdp(s,\act,s')\cdot p_{s'}}
\end{align*}
for all $s\in S\setminus T$ and $\act\in \Act$.
The probability variables $p_s$ are assigned upper bounds on the actual probability to reach the target states $T$ under scheduler $\sched$ and valuation $u$. Put differently, the $p_s$ variables cannot be assigned values that are lower than the actual probability; ensuring that $\sched$ and $u$ induce satisfaction of the specification given by~\eqref{eq:lambda_geofull} if the problem is feasible and $\sched$ and $u$ are well-defined. An analogous reasoning applies to the expected cost computation~\eqref{eq:rewcomputation_geofull}. 
A solution consisting of a scheduler or valuation that are not well-defined occurs only if Problem~\ref{prob:pmdpsyn} itself is infeasible. Identifying that such a solution has been obtained is easy.
%Only if Problem~\ref{prob:pmdpsyn} itself is infeasible, a solution inducing not well-defined scheduler or valuation might be computed. This can easily be checked.
These facts allow us to state the main result of this section.
\begin{theorem}
	\label{thm:main}
	A solution of the GP~\eqref{eq:min_rand_geofull}--\eqref{eq:rewcomputation_geofull} inducing well-defined scheduler $\sched$ and valuation $u$ is a feasible solution to Problem 1.
\end{theorem}
Note that the actual probabilities induced by $\sched$ and $u$ for the given pMDP $\mdp$ are given by the MC $\mdp^\sched[u]$ induced by $\sched$ and instantiated by $u$.
Since all variables are implicitly positive in a GP, no transition probability function will be instantiated to probability zero. 
%Thereby, the topology of the pMDP is always preserved. 
The case of a scheduler variable being zero to induce the optimum can be excluded by a previous graph analysis.
%










\section{Experimental Evaluation}
\label{sec:experiment}
To demonstrate the viability of our modeling methodology, we show experimentally how through the deliberate combination and configuration of parallel FREEs, full control over 2DOF spacial forces can be achieved by using only the minimum combination of three FREEs.
To this end, we carefully chose the fiber angle $\Gamma$ of each of these actuators to achieve a well-balanced force zonotope (Fig.~\ref{fig:rigDiagram}).
We combined a contracting and counterclockwise twisting FREE with a fiber angle of $\Gamma = 48^\circ$, a contracting and clockwise twisting FREE with $\Gamma = -48^\circ$, and an extending FREE with $\Gamma = -85^\circ$.
All three FREEs were designed with a nominal radius of $R$ = \unit[5]{mm} and a length of $L$ = \unit[100]{mm}.
%
\begin{figure}
    \centering
    \includegraphics[width=0.75\linewidth]{figures/rigDiagram_wlabels10.pdf}
    \caption{In the experimental evaluation, we employed a parallel combination of three FREEs (top) to yield forces along and moments about the $z$-axis of an end effector.
    The FREEs were carefully selected to yield a well-balanced force zonotope (bottom) to gain full control authority over $F^{\hat{z}_e}$ and $M^{\hat{z}_e}$.
    To this end, we used one extending FREE, and two contracting FREEs which generate antagonistic moments about the end effector $z$-axis.}
    \label{fig:rigDiagram}
\end{figure}


\subsection{Experimental Setup}
To measure the forces generated by this actuator combination under a varying state $\vec{x}$ and pressure input $\vec{p}$, we developed a custom built test platform (Fig.~\ref{fig:rig}). 
%
\begin{figure}
    \centering
    \includegraphics[width=0.9\linewidth]{figures/photos/rig_labeled.pdf}
    \caption{\revcomment{1.3}{This experimental platform is used to generate a targeted displacement (extension and twist) of the end effector and to measure the forces and torques created by a parallel combination of three FREEs. A linear actuator and servomotor impose an extension and a twist, respectively, while the net force and moment generated by the FREEs is measured with a force load cell and moment load cell mounted in series.}}
    \label{fig:rig}
\end{figure}
%
In the test platform, a linear actuator (ServoCity HDA 6-50) and a rotational servomotor (Hitec HS-645mg) were used to impose a 2-dimensional displacement on the end effector. 
A force load cell (LoadStar  RAS1-25lb) and a moment load cell (LoadStar RST1-6Nm) measured the end-effector forces $F^{\hat{z_e}}$ and moments $M^{\hat{z_e}}$, respectively.
During the experiments, the pressures inside the FREEs were varied using pneumatic pressure regulators (Enfield TR-010-g10-s). 

The FREE attachment points (measured from the end effector origin) were measured to be:
\begin{align}
    \vec{d}_1 &= \bmx 0.013 & 0 & 0 \emx^T  \text{m}\\
    \vec{d}_2 &= \bmx -0.006 & 0.011 & 0 \emx^T  \text{m}\\
    \vec{d}_3 &= \bmx -0.006 & -0.011 & 0 \emx^T \text{m}
%    \vec{d}_i &= \bmx 0 & 0 & 0 \emx^T , && \text{for } i = 1,2,3
\end{align}
All three FREEs were oriented parallel to the end effector $z$-axis:
\begin{align}
    \hat{a}_i &= \bmx 0 & 0 & 1 \emx^T, \hspace{20pt} \text{for } i = 1,2,3
\end{align}
Based on this geometry, the transformation matrices $\bar{\mathcal{D}}_i$ were given by:
\begin{align}
    \bar{\mathcal{D}}_1 &= \bmx 0 & 0 & 1 & 0 & -0.013 & 0 \\ 0 & 0 & 0 & 0 & 0 & 1 \emx^T  \\
    \bar{\mathcal{D}}_2 &= \bmx 0 & 0 & 1 & 0.011 & 0.006 & 0 \\ 0 & 0 & 0 & 0 & 0 & 1 \emx^T  \\
    \bar{\mathcal{D}}_3 &= \bmx 0 & 0 & 1 & -0.011 & 0.006 & 0 \\ 0 & 0 & 0 & 0 & 0 & 1 \emx^T 
%    \bar{\mathcal{D}}_i &= \bmx 0 & 0 & 1 & 0 & 0 & 0 \\ 0 & 0 & 0 & 0 & 0 & 1 \emx^T , && \text{for } i = 1,2,3
\end{align}
These matrices were used in equation \eqref{eq:zeta} to yield the state-dependent fluid Jacobian $\bar{J}_x$ and to compute the resulting force zontopes.
%while using measured values of $\vec{\zeta}^{\,\text{meas}} (\vec{q}, \vec{P})$ and $\vec{\zeta}^{\,\text{meas}} (\vec{q}, 0)$ in \eqref{eq:fiberIso} yields the empirical measurements of the active force.



\subsection{Isolating the Active Force}
To compare our model force predictions (which focus only on the active forces induced by the fibers)
to those measured empirically on a physical system, we had to remove the elastic force components attributed to the elastomer. 
Under the assumption that the elastomer force is merely a function of the displacement $\vec{x}$ and independent of pressure $\vec{p}$ \cite{bruder2017model}, this force component can be approximated by the measured force at a pressure of $\vec{p}=0$. 
That is: 
\begin{align}
    \vec{f}_{\text{elast}} (\vec{x}) = \vec{f}_{\text{\,meas}} (\vec{x}, 0)
\end{align}
With this, the active generalized forces were measured indirectly by subtracting off the force generated at zero pressure:
\begin{align}
    \vec{f} (\vec{x}, \vec{p})  &= \vec{f}_{\text{meas}} (\vec{x}, \vec{p}) - \vec{f}_{\text{meas}} (\vec{x}, 0)     \label{eq:fiberIso}
\end{align}


%To validate our parallel force model, we compare its force predictions, $\vec{\zeta}_{\text{pred}}$, to those measured empirically on a physical system, $\vec{\zeta}_\text{meas}$. 
%From \eqref{eq:Z} and \eqref{eq:zeta}, the force at the end effector is given by:
%\begin{align}
%    \vec{\zeta}(\vec{q}, \vec{P}) &= \sum_{i=1}^n \bar{\mathcal{D}}_i \left( {\bar{J}_V}_i^T(\vec{q_i}) P_i + \vec{Z}_i^{\text{elast}} (\vec{q_i}) \right) \\
%    &= \underbrace{\sum_{i=1}^n \bar{\mathcal{D}}_i {\bar{J}_V}_i^T(\vec{q_i}) P_i}_{\vec{\zeta}^{\,\text{fiber}} (\vec{q}, \vec{P})} + \underbrace{\sum_{i=1}^n \bar{\mathcal{D}}_i \vec{Z}_i^{\text{elast}} (\vec{q_i})}_{\vec{\zeta}^{\text{elast}} (\vec{q})}   \label{eq:zetaSplit}
%     &= \vec{\zeta}^{\,\text{fiber}} (\vec{q}, \vec{P}) + \vec{\zeta}^{\text{elast}} (\vec{q})
%\end{align}
%\Dan{These will need to reflect changes made to previous section.}
%The model presented in this paper does not specify the elastomer forces, $\vec{\zeta}^{\text{elast}}$, therefore we only validate its predictions %of the fiber forces, $\vec{\zeta}^{\,\text{fiber}}$. 
%We isolate the fiber forces by noting that $\vec{\zeta}^{\text{elast}} (\vec{q}) = \vec{\zeta}(\vec{q}, 0)$ and rearranging \eqref{eq:zetaSplit}
%\begin{align}
%    \vec{\zeta}^{\,\text{fiber}} (\vec{q}, \vec{P})  &= \vec{\zeta}(\vec{q}, \vec{P}) - \vec{\zeta}(\vec{q}, 0)     \label{eq:fiberIso}
%%    \vec{\zeta}^{\,\text{fiber}}_{\text{emp}} (\vec{q}, \vec{P})  &= \vec{\zeta}_{\text{emp}}(\vec{q}, \vec{P}) - %\vec{\zeta}_{\text{emp}}(\vec{q}, 0)
%\end{align}
%Thus we measure the fiber forces indirectly by subtracting off the forces generated at zero pressure.  


\subsection{Experimental Protocol}
The force and moment generated by the parallel combination of FREEs about the end effector $z$-axis  was measured in four different geometric configurations: neutral, extended, twisted, and simultaneously extended and twisted (see Table \ref{table:RMSE} for the exact deformation amounts). 
At each of these configurations, the forces were measured at all pressure combinations in the set
\begin{align}
    \mathcal{P} &= \left\{ \bmx \alpha_1 & \alpha_2 & \alpha_3 \emx^T p^{\text{max}} \, : \, \alpha_i = \left\{ 0, \frac{1}{4}, \frac{1}{2}, \frac{3}{4}, 1 \right\} \right\}
\end{align}
with $p^{\text{max}}$ = \unit[103.4]{kPa}. 
\revcomment{3.2}{The experiment was performed twice using two different sets of FREEs to observe how fabrication variability might affect performance. The results from Trial 1 are displayed in Fig.~\ref{fig:results} and the error for both trials is given in Table \ref{table:RMSE}.}



\subsection{Results}

\begin{figure*}[ht]
\centering

\def\picScale{0.08}    % define variable for scaling all pictures evenly
\def\plotScale{0.2}    % define variable for scaling all plots evenly
\def\colWidth{0.22\linewidth}

\begin{tikzpicture} %[every node/.style={draw=black}]
% \draw[help lines] (0,0) grid (4,2);
\matrix [row sep=0cm, column sep=0cm, style={align=center}] (my matrix) at (0,0) %(2,1)
{
& \node (q1) {(a) $\Delta l = 0, \Delta \phi = 0$}; & \node (q2) {(b) $\Delta l = 5\text{mm}, \Delta \phi = 0$}; & \node (q3) {(c) $\Delta l = 0, \Delta \phi = 20^\circ$}; & \node (q4) {(d) $\Delta l = 5\text{mm}, \Delta \phi = 20^\circ$};

\\

&
\node[style={anchor=center}] {\includegraphics[width=\colWidth]{figures/photos/s0w0pic_colored.pdf}}; %\fill[blue] (0,0) circle (2pt);
&
\node[style={anchor=center}] {\includegraphics[width=\colWidth]{figures/photos/s5w0pic_colored.pdf}}; %\fill[blue] (0,0) circle (2pt);
&
\node[style={anchor=center}] {\includegraphics[width=\colWidth]{figures/photos/s0w20pic_colored.pdf}}; %\fill[blue] (0,0) circle (2pt);
&
\node[style={anchor=center}] {\includegraphics[width=\colWidth]{figures/photos/s5w20pic_colored.pdf}}; %\fill[blue] (0,0) circle (2pt);

\\

\node[rotate=90] (ylabel) {Moment, $M^{\hat{z}_e}$ (N-m)};
&
\node[style={anchor=center}] {\includegraphics[width=\colWidth]{figures/plots3/s0w0.pdf}}; %\fill[blue] (0,0) circle (2pt);
&
\node[style={anchor=center}] {\includegraphics[width=\colWidth]{figures/plots3/s5w0.pdf}}; %\fill[blue] (0,0) circle (2pt);
&
\node[style={anchor=center}] {\includegraphics[width=\colWidth]{figures/plots3/s0w20.pdf}}; %\fill[blue] (0,0) circle (2pt);
&
\node[style={anchor=center}] {\includegraphics[width=\colWidth]{figures/plots3/s5w20.pdf}}; %\fill[blue] (0,0) circle (2pt);

\\

& \node (xlabel1) {Force, $F^{\hat{z}_e}$ (N)}; & \node (xlabel2) {Force, $F^{\hat{z}_e}$ (N)}; & \node (xlabel3) {Force, $F^{\hat{z}_e}$ (N)}; & \node (xlabel4) {Force, $F^{\hat{z}_e}$ (N)};

\\
};
\end{tikzpicture}

\caption{For four different deformed configurations (top row), we compare the predicted and the measured forces for the parallel combination of three FREEs (bottom row). 
\revcomment{2.6}{Data points and predictions corresponding to the same input pressures are connected by a thin line, and the convex hull of the measured data points is outlined in black.}
The Trial 1 data is overlaid on top of the theoretical force zonotopes (grey areas) for each of the four configurations.
Identical colors indicate correspondence between a FREE and its resulting force/torque direction.}
\label{fig:results}
\end{figure*}






% & \node (a) {(a)}; & \node (b) {(b)}; & \node (c) {(c)}; & \node (d) {(d)};


For comparison, the measured forces are superimposed over the force zonotope generated by our model in Fig.~\ref{fig:results}a-~\ref{fig:results}d.
To quantify the accuracy of the model, we defined the error at the $j^{th}$ evaluation point as the difference between the modeled and measured forces
\begin{align}
%    \vec{e}_j &= \left( {\vec{\zeta}_{\,\text{mod}}} - {\vec{\zeta}_{\,\text{emp}}} \right)_j
%    e_j &= \left( F/M_{\,\text{mod}} - F/M_{\,\text{emp}} \right)_j
    e^F_j &= \left( F^{\hat{z}_e}_{\text{pred}, j} - F^{\hat{z}_e}_{\text{meas}, j} \right) \\
    e^M_j &= \left( M^{\hat{z}_e}_{\text{pred}, j} - M^{\hat{z}_e}_{\text{meas}, j} \right)
\end{align}
and evaluated the error across all $N = 125$ trials of a given end effector configuration.
% using the following metrics:
% \begin{align}
%     \text{RMSE} &= \sqrt{ \frac{\sum_{j=1}^{N} e_j^2}{N} } \\
%     \text{Max Error} &= \max \{ \left| e_j \right| : j = 1, ... , N \}
% \end{align}
As shown in Table \ref{table:RMSE}, the root-mean-square error (RMSE) is less than \unit[1.5]{N} (\unit[${8 \times 10^{-3}}$]{Nm}), and the maximum error is less than \unit[3]{N}  (\unit[${19 \times 10^{-3}}$]{Nm}) across all trials and configurations.

\begin{table}[H]
\centering
\caption{Root-mean-square error and maximum error}
\begin{tabular}{| c | c || c | c | c | c|}
    \hline
     & \rule{0pt}{2ex} \textbf{Disp.} & \multicolumn{2}{c |}{\textbf{RMSE}} & \multicolumn{2}{c |}{\textbf{Max Error}} \\ 
     \cline{2-6}
     & \rule{0pt}{2ex} (mm, $^\circ$) & F (N) & M (Nm) & F (N) & M (Nm) \\
     \hline
     \multirow{4}{*}{\rotatebox[origin=c]{90}{\textbf{Trial 1}}}
     & 0, 0 & 1.13 & $3.8 \times 10^{-3}$ & 2.96 & $7.8 \times 10^{-3}$ \\
     & 5, 0 & 0.74 & $3.2 \times 10^{-3}$ & 2.31 & $7.4 \times 10^{-3}$ \\
     & 0, 20 & 1.47 & $6.3 \times 10^{-3}$ & 2.52 & $15.6 \times 10^{-3}$\\
     & 5, 20 & 1.18 & $4.6 \times 10^{-3}$ & 2.85 & $12.4 \times 10^{-3}$ \\  
     \hline
     \multirow{4}{*}{\rotatebox[origin=c]{90}{\textbf{Trial 2}}}
     & 0, 0 & 0.93 & $6.0 \times 10^{-3}$ & 1.90 & $13.3 \times 10^{-3}$ \\
     & 5, 0 & 1.00 & $7.7 \times 10^{-3}$ & 2.97 & $19.0 \times 10^{-3}$ \\
     & 0, 20 & 0.77 & $6.9 \times 10^{-3}$ & 2.89 & $15.7 \times 10^{-3}$\\
     & 5, 20 & 0.95 & $5.3 \times 10^{-3}$ & 2.22 & $13.3 \times 10^{-3}$ \\  
     \hline
\end{tabular}
\label{table:RMSE}
\end{table}

\begin{figure}
    \centering
    \includegraphics[width=\linewidth]{figures/photos/buckling.pdf}
    \caption{At high fluid pressure the FREE with fiber angle of $-85^\circ$ started to buckle.  This effect was less pronounced when the system was extended along the $z$-axis.}
    \label{fig:buckling}
\end{figure}

%Experimental precision was limited by unmodeled material defects in the FREEs, as well as sensor inaccuracy. While the commercial force and moment sensors used have a quoted accuracy of 0.02\% for the force sensor and 0.2\% for the moment sensor (LoadStar Sensors, 2015), a drifting of up to 0.5 N away from zero was noticed on the force sensor during testing.

It should be noted, that throughout the experiments, the FREE with a fiber angle of $-85^\circ$ exhibited noticeable buckling behavior at pressures above $\approx$ \unit[50]{kPa} (Fig.~\ref{fig:buckling}). 
This behavior was more pronounced during testing in the non-extended configurations (Fig.~\ref{fig:results}a~and~\ref{fig:results}c). 
The buckling might explain the noticeable leftward offset of many of the points in Fig.~\ref{fig:results}a and Fig.~\ref{fig:results}c, since it is reasonable to assume that buckling reduces the efficacy of of the FREE to exert force in the direction normal to the force sensor. 

\begin{figure}
    \centering
    \includegraphics[width=\linewidth]{figures/zntp_vs_x4.pdf}
    \caption{A visualization of how the \emph{force zonotope} of the parallel combination of three FREEs (see Fig.~\ref{fig:rig}) changes as a function of the end effector state $x$. One can observe that the change in the zonotope ultimately limits the work-space of such a system.  In particular the zonotope will collapse for compressions of more than \unit[-10]{mm}.  For \revcomment{2.5}{scale and comparison, the convex hulls of the measured points from Fig.~\ref{fig:results}} are superimposed over their corresponding zonotope at the configurations that were evaluated experimentally.}
    % \marginnote{\#2.5}
    \label{fig:zntp_vs_x}
\end{figure}

%{\bf Acknowledgement: }D.W. thanks support from the
%     XDATA program of the Defense Advanced
%     Research Projects Agency (DARPA),
%     Air Force Research Laboratory
%     contract FA8750-12-C-0323.

\bibliography{reference,qin}
\bibliographystyle{plain}


%\appendix

%\chapter{Supplementary Material}
\label{appendix}

In this appendix, we present supplementary material for the techniques and
experiments presented in the main text.

\section{Baseline Results and Analysis for Informed Sampler}
\label{appendix:chap3}

Here, we give an in-depth
performance analysis of the various samplers and the effect of their
hyperparameters. We choose hyperparameters with the lowest PSRF value
after $10k$ iterations, for each sampler individually. If the
differences between PSRF are not significantly different among
multiple values, we choose the one that has the highest acceptance
rate.

\subsection{Experiment: Estimating Camera Extrinsics}
\label{appendix:chap3:room}

\subsubsection{Parameter Selection}
\paragraph{Metropolis Hastings (\MH)}

Figure~\ref{fig:exp1_MH} shows the median acceptance rates and PSRF
values corresponding to various proposal standard deviations of plain
\MH~sampling. Mixing gets better and the acceptance rate gets worse as
the standard deviation increases. The value $0.3$ is selected standard
deviation for this sampler.

\paragraph{Metropolis Hastings Within Gibbs (\MHWG)}

As mentioned in Section~\ref{sec:room}, the \MHWG~sampler with one-dimensional
updates did not converge for any value of proposal standard deviation.
This problem has high correlation of the camera parameters and is of
multi-modal nature, which this sampler has problems with.

\paragraph{Parallel Tempering (\PT)}

For \PT~sampling, we took the best performing \MH~sampler and used
different temperature chains to improve the mixing of the
sampler. Figure~\ref{fig:exp1_PT} shows the results corresponding to
different combination of temperature levels. The sampler with
temperature levels of $[1,3,27]$ performed best in terms of both
mixing and acceptance rate.

\paragraph{Effect of Mixture Coefficient in Informed Sampling (\MIXLMH)}

Figure~\ref{fig:exp1_alpha} shows the effect of mixture
coefficient ($\alpha$) on the informed sampling
\MIXLMH. Since there is no significant different in PSRF values for
$0 \le \alpha \le 0.7$, we chose $0.7$ due to its high acceptance
rate.


% \end{multicols}

\begin{figure}[h]
\centering
  \subfigure[MH]{%
    \includegraphics[width=.48\textwidth]{figures/supplementary/camPose_MH.pdf} \label{fig:exp1_MH}
  }
  \subfigure[PT]{%
    \includegraphics[width=.48\textwidth]{figures/supplementary/camPose_PT.pdf} \label{fig:exp1_PT}
  }
\\
  \subfigure[INF-MH]{%
    \includegraphics[width=.48\textwidth]{figures/supplementary/camPose_alpha.pdf} \label{fig:exp1_alpha}
  }
  \mycaption{Results of the `Estimating Camera Extrinsics' experiment}{PRSFs and Acceptance rates corresponding to (a) various standard deviations of \MH, (b) various temperature level combinations of \PT sampling and (c) various mixture coefficients of \MIXLMH sampling.}
\end{figure}



\begin{figure}[!t]
\centering
  \subfigure[\MH]{%
    \includegraphics[width=.48\textwidth]{figures/supplementary/occlusionExp_MH.pdf} \label{fig:exp2_MH}
  }
  \subfigure[\BMHWG]{%
    \includegraphics[width=.48\textwidth]{figures/supplementary/occlusionExp_BMHWG.pdf} \label{fig:exp2_BMHWG}
  }
\\
  \subfigure[\MHWG]{%
    \includegraphics[width=.48\textwidth]{figures/supplementary/occlusionExp_MHWG.pdf} \label{fig:exp2_MHWG}
  }
  \subfigure[\PT]{%
    \includegraphics[width=.48\textwidth]{figures/supplementary/occlusionExp_PT.pdf} \label{fig:exp2_PT}
  }
\\
  \subfigure[\INFBMHWG]{%
    \includegraphics[width=.5\textwidth]{figures/supplementary/occlusionExp_alpha.pdf} \label{fig:exp2_alpha}
  }
  \mycaption{Results of the `Occluding Tiles' experiment}{PRSF and
    Acceptance rates corresponding to various standard deviations of
    (a) \MH, (b) \BMHWG, (c) \MHWG, (d) various temperature level
    combinations of \PT~sampling and; (e) various mixture coefficients
    of our informed \INFBMHWG sampling.}
\end{figure}

%\onecolumn\newpage\twocolumn
\subsection{Experiment: Occluding Tiles}
\label{appendix:chap3:tiles}

\subsubsection{Parameter Selection}

\paragraph{Metropolis Hastings (\MH)}

Figure~\ref{fig:exp2_MH} shows the results of
\MH~sampling. Results show the poor convergence for all proposal
standard deviations and rapid decrease of AR with increasing standard
deviation. This is due to the high-dimensional nature of
the problem. We selected a standard deviation of $1.1$.

\paragraph{Blocked Metropolis Hastings Within Gibbs (\BMHWG)}

The results of \BMHWG are shown in Figure~\ref{fig:exp2_BMHWG}. In
this sampler we update only one block of tile variables (of dimension
four) in each sampling step. Results show much better performance
compared to plain \MH. The optimal proposal standard deviation for
this sampler is $0.7$.

\paragraph{Metropolis Hastings Within Gibbs (\MHWG)}

Figure~\ref{fig:exp2_MHWG} shows the result of \MHWG sampling. This
sampler is better than \BMHWG and converges much more quickly. Here
a standard deviation of $0.9$ is found to be best.

\paragraph{Parallel Tempering (\PT)}

Figure~\ref{fig:exp2_PT} shows the results of \PT sampling with various
temperature combinations. Results show no improvement in AR from plain
\MH sampling and again $[1,3,27]$ temperature levels are found to be optimal.

\paragraph{Effect of Mixture Coefficient in Informed Sampling (\INFBMHWG)}

Figure~\ref{fig:exp2_alpha} shows the effect of mixture
coefficient ($\alpha$) on the blocked informed sampling
\INFBMHWG. Since there is no significant different in PSRF values for
$0 \le \alpha \le 0.8$, we chose $0.8$ due to its high acceptance
rate.



\subsection{Experiment: Estimating Body Shape}
\label{appendix:chap3:body}

\subsubsection{Parameter Selection}
\paragraph{Metropolis Hastings (\MH)}

Figure~\ref{fig:exp3_MH} shows the result of \MH~sampling with various
proposal standard deviations. The value of $0.1$ is found to be
best.

\paragraph{Metropolis Hastings Within Gibbs (\MHWG)}

For \MHWG sampling we select $0.3$ proposal standard
deviation. Results are shown in Fig.~\ref{fig:exp3_MHWG}.


\paragraph{Parallel Tempering (\PT)}

As before, results in Fig.~\ref{fig:exp3_PT}, the temperature levels
were selected to be $[1,3,27]$ due its slightly higher AR.

\paragraph{Effect of Mixture Coefficient in Informed Sampling (\MIXLMH)}

Figure~\ref{fig:exp3_alpha} shows the effect of $\alpha$ on PSRF and
AR. Since there is no significant differences in PSRF values for $0 \le
\alpha \le 0.8$, we choose $0.8$.


\begin{figure}[t]
\centering
  \subfigure[\MH]{%
    \includegraphics[width=.48\textwidth]{figures/supplementary/bodyShape_MH.pdf} \label{fig:exp3_MH}
  }
  \subfigure[\MHWG]{%
    \includegraphics[width=.48\textwidth]{figures/supplementary/bodyShape_MHWG.pdf} \label{fig:exp3_MHWG}
  }
\\
  \subfigure[\PT]{%
    \includegraphics[width=.48\textwidth]{figures/supplementary/bodyShape_PT.pdf} \label{fig:exp3_PT}
  }
  \subfigure[\MIXLMH]{%
    \includegraphics[width=.48\textwidth]{figures/supplementary/bodyShape_alpha.pdf} \label{fig:exp3_alpha}
  }
\\
  \mycaption{Results of the `Body Shape Estimation' experiment}{PRSFs and
    Acceptance rates corresponding to various standard deviations of
    (a) \MH, (b) \MHWG; (c) various temperature level combinations
    of \PT sampling and; (d) various mixture coefficients of the
    informed \MIXLMH sampling.}
\end{figure}


\subsection{Results Overview}
Figure~\ref{fig:exp_summary} shows the summary results of the all the three
experimental studies related to informed sampler.
\begin{figure*}[h!]
\centering
  \subfigure[Results for: Estimating Camera Extrinsics]{%
    \includegraphics[width=0.9\textwidth]{figures/supplementary/camPose_ALL.pdf} \label{fig:exp1_all}
  }
  \subfigure[Results for: Occluding Tiles]{%
    \includegraphics[width=0.9\textwidth]{figures/supplementary/occlusionExp_ALL.pdf} \label{fig:exp2_all}
  }
  \subfigure[Results for: Estimating Body Shape]{%
    \includegraphics[width=0.9\textwidth]{figures/supplementary/bodyShape_ALL.pdf} \label{fig:exp3_all}
  }
  \label{fig:exp_summary}
  \mycaption{Summary of the statistics for the three experiments}{Shown are
    for several baseline methods and the informed samplers the
    acceptance rates (left), PSRFs (middle), and RMSE values
    (right). All results are median results over multiple test
    examples.}
\end{figure*}

\subsection{Additional Qualitative Results}

\subsubsection{Occluding Tiles}
In Figure~\ref{fig:exp2_visual_more} more qualitative results of the
occluding tiles experiment are shown. The informed sampling approach
(\INFBMHWG) is better than the best baseline (\MHWG). This still is a
very challenging problem since the parameters for occluded tiles are
flat over a large region. Some of the posterior variance of the
occluded tiles is already captured by the informed sampler.

\begin{figure*}[h!]
\begin{center}
\centerline{\includegraphics[width=0.95\textwidth]{figures/supplementary/occlusionExp_Visual.pdf}}
\mycaption{Additional qualitative results of the occluding tiles experiment}
  {From left to right: (a)
  Given image, (b) Ground truth tiles, (c) OpenCV heuristic and most probable estimates
  from 5000 samples obtained by (d) MHWG sampler (best baseline) and
  (e) our INF-BMHWG sampler. (f) Posterior expectation of the tiles
  boundaries obtained by INF-BMHWG sampling (First 2000 samples are
  discarded as burn-in).}
\label{fig:exp2_visual_more}
\end{center}
\end{figure*}

\subsubsection{Body Shape}
Figure~\ref{fig:exp3_bodyMeshes} shows some more results of 3D mesh
reconstruction using posterior samples obtained by our informed
sampling \MIXLMH.

\begin{figure*}[t]
\begin{center}
\centerline{\includegraphics[width=0.75\textwidth]{figures/supplementary/bodyMeshResults.pdf}}
\mycaption{Qualitative results for the body shape experiment}
  {Shown is the 3D mesh reconstruction results with first 1000 samples obtained
  using the \MIXLMH informed sampling method. (blue indicates small
  values and red indicates high values)}
\label{fig:exp3_bodyMeshes}
\end{center}
\end{figure*}

\clearpage



\section{Additional Results on the Face Problem with CMP}

Figure~\ref{fig:shading-qualitative-multiple-subjects-supp} shows inference results for reflectance maps, normal maps and lights for randomly chosen test images, and Fig.~\ref{fig:shading-qualitative-same-subject-supp} shows reflectance estimation results on multiple images of the same subject produced under different illumination conditions. CMP is able to produce estimates that are closer to the groundtruth across different subjects and illumination conditions.

\begin{figure*}[h]
  \begin{center}
  \centerline{\includegraphics[width=1.0\columnwidth]{figures/face_cmp_visual_results_supp.pdf}}
  \vspace{-1.2cm}
  \end{center}
	\mycaption{A visual comparison of inference results}{(a)~Observed images. (b)~Inferred reflectance maps. \textit{GT} is the photometric stereo groundtruth, \textit{BU} is the Biswas \etal (2009) reflectance estimate and \textit{Forest} is the consensus prediction. (c)~The variance of the inferred reflectance estimate produced by \MTD (normalized across rows).(d)~Visualization of inferred light directions. (e)~Inferred normal maps.}
	\label{fig:shading-qualitative-multiple-subjects-supp}
\end{figure*}


\begin{figure*}[h]
	\centering
	\setlength\fboxsep{0.2mm}
	\setlength\fboxrule{0pt}
	\begin{tikzpicture}

		\matrix at (0, 0) [matrix of nodes, nodes={anchor=east}, column sep=-0.05cm, row sep=-0.2cm]
		{
			\fbox{\includegraphics[width=1cm]{figures/sample_3_4_X.png}} &
			\fbox{\includegraphics[width=1cm]{figures/sample_3_4_GT.png}} &
			\fbox{\includegraphics[width=1cm]{figures/sample_3_4_BISWAS.png}}  &
			\fbox{\includegraphics[width=1cm]{figures/sample_3_4_VMP.png}}  &
			\fbox{\includegraphics[width=1cm]{figures/sample_3_4_FOREST.png}}  &
			\fbox{\includegraphics[width=1cm]{figures/sample_3_4_CMP.png}}  &
			\fbox{\includegraphics[width=1cm]{figures/sample_3_4_CMPVAR.png}}
			 \\

			\fbox{\includegraphics[width=1cm]{figures/sample_3_5_X.png}} &
			\fbox{\includegraphics[width=1cm]{figures/sample_3_5_GT.png}} &
			\fbox{\includegraphics[width=1cm]{figures/sample_3_5_BISWAS.png}}  &
			\fbox{\includegraphics[width=1cm]{figures/sample_3_5_VMP.png}}  &
			\fbox{\includegraphics[width=1cm]{figures/sample_3_5_FOREST.png}}  &
			\fbox{\includegraphics[width=1cm]{figures/sample_3_5_CMP.png}}  &
			\fbox{\includegraphics[width=1cm]{figures/sample_3_5_CMPVAR.png}}
			 \\

			\fbox{\includegraphics[width=1cm]{figures/sample_3_6_X.png}} &
			\fbox{\includegraphics[width=1cm]{figures/sample_3_6_GT.png}} &
			\fbox{\includegraphics[width=1cm]{figures/sample_3_6_BISWAS.png}}  &
			\fbox{\includegraphics[width=1cm]{figures/sample_3_6_VMP.png}}  &
			\fbox{\includegraphics[width=1cm]{figures/sample_3_6_FOREST.png}}  &
			\fbox{\includegraphics[width=1cm]{figures/sample_3_6_CMP.png}}  &
			\fbox{\includegraphics[width=1cm]{figures/sample_3_6_CMPVAR.png}}
			 \\
	     };

       \node at (-3.85, -2.0) {\small Observed};
       \node at (-2.55, -2.0) {\small `GT'};
       \node at (-1.27, -2.0) {\small BU};
       \node at (0.0, -2.0) {\small MP};
       \node at (1.27, -2.0) {\small Forest};
       \node at (2.55, -2.0) {\small \textbf{CMP}};
       \node at (3.85, -2.0) {\small Variance};

	\end{tikzpicture}
	\mycaption{Robustness to varying illumination}{Reflectance estimation on a subject images with varying illumination. Left to right: observed image, photometric stereo estimate (GT)
  which is used as a proxy for groundtruth, bottom-up estimate of \cite{Biswas2009}, VMP result, consensus forest estimate, CMP mean, and CMP variance.}
	\label{fig:shading-qualitative-same-subject-supp}
\end{figure*}

\clearpage

\section{Additional Material for Learning Sparse High Dimensional Filters}
\label{sec:appendix-bnn}

This part of supplementary material contains a more detailed overview of the permutohedral
lattice convolution in Section~\ref{sec:permconv}, more experiments in
Section~\ref{sec:addexps} and additional results with protocols for
the experiments presented in Chapter~\ref{chap:bnn} in Section~\ref{sec:addresults}.

\vspace{-0.2cm}
\subsection{General Permutohedral Convolutions}
\label{sec:permconv}

A core technical contribution of this work is the generalization of the Gaussian permutohedral lattice
convolution proposed in~\cite{adams2010fast} to the full non-separable case with the
ability to perform back-propagation. Although, conceptually, there are minor
differences between Gaussian and general parameterized filters, there are non-trivial practical
differences in terms of the algorithmic implementation. The Gauss filters belong to
the separable class and can thus be decomposed into multiple
sequential one dimensional convolutions. We are interested in the general filter
convolutions, which can not be decomposed. Thus, performing a general permutohedral
convolution at a lattice point requires the computation of the inner product with the
neighboring elements in all the directions in the high-dimensional space.

Here, we give more details of the implementation differences of separable
and non-separable filters. In the following, we will explain the scalar case first.
Recall, that the forward pass of general permutohedral convolution
involves 3 steps: \textit{splatting}, \textit{convolving} and \textit{slicing}.
We follow the same splatting and slicing strategies as in~\cite{adams2010fast}
since these operations do not depend on the filter kernel. The main difference
between our work and the existing implementation of~\cite{adams2010fast} is
the way that the convolution operation is executed. This proceeds by constructing
a \emph{blur neighbor} matrix $K$ that stores for every lattice point all
values of the lattice neighbors that are needed to compute the filter output.

\begin{figure}[t!]
  \centering
    \includegraphics[width=0.6\columnwidth]{figures/supplementary/lattice_construction}
  \mycaption{Illustration of 1D permutohedral lattice construction}
  {A $4\times 4$ $(x,y)$ grid lattice is projected onto the plane defined by the normal
  vector $(1,1)^{\top}$. This grid has $s+1=4$ and $d=2$ $(s+1)^{d}=4^2=16$ elements.
  In the projection, all points of the same color are projected onto the same points in the plane.
  The number of elements of the projected lattice is $t=(s+1)^d-s^d=4^2-3^2=7$, that is
  the $(4\times 4)$ grid minus the size of lattice that is $1$ smaller at each size, in this
  case a $(3\times 3)$ lattice (the upper right $(3\times 3)$ elements).
  }
\label{fig:latticeconstruction}
\end{figure}

The blur neighbor matrix is constructed by traversing through all the populated
lattice points and their neighboring elements.
% For efficiency, we do this matrix construction recursively with shared computations
% since $n^{th}$ neighbourhood elements are $1^{st}$ neighborhood elements of $n-1^{th}$ neighbourhood elements. \pg{do not understand}
This is done recursively to share computations. For any lattice point, the neighbors that are
$n$ hops away are the direct neighbors of the points that are $n-1$ hops away.
The size of a $d$ dimensional spatial filter with width $s+1$ is $(s+1)^{d}$ (\eg, a
$3\times 3$ filter, $s=2$ in $d=2$ has $3^2=9$ elements) and this size grows
exponentially in the number of dimensions $d$. The permutohedral lattice is constructed by
projecting a regular grid onto the plane spanned by the $d$ dimensional normal vector ${(1,\ldots,1)}^{\top}$. See
Fig.~\ref{fig:latticeconstruction} for an illustration of the 1D lattice construction.
Many corners of a grid filter are projected onto the same point, in total $t = {(s+1)}^{d} -
s^{d}$ elements remain in the permutohedral filter with $s$ neighborhood in $d-1$ dimensions.
If the lattice has $m$ populated elements, the
matrix $K$ has size $t\times m$. Note that, since the input signal is typically
sparse, only a few lattice corners are being populated in the \textit{slicing} step.
We use a hash-table to keep track of these points and traverse only through
the populated lattice points for this neighborhood matrix construction.

Once the blur neighbor matrix $K$ is constructed, we can perform the convolution
by the matrix vector multiplication
\begin{equation}
\ell' = BK,
\label{eq:conv}
\end{equation}
where $B$ is the $1 \times t$ filter kernel (whose values we will learn) and $\ell'\in\mathbb{R}^{1\times m}$
is the result of the filtering at the $m$ lattice points. In practice, we found that the
matrix $K$ is sometimes too large to fit into GPU memory and we divided the matrix $K$
into smaller pieces to compute Eq.~\ref{eq:conv} sequentially.

In the general multi-dimensional case, the signal $\ell$ is of $c$ dimensions. Then
the kernel $B$ is of size $c \times t$ and $K$ stores the $c$ dimensional vectors
accordingly. When the input and output points are different, we slice only the
input points and splat only at the output points.


\subsection{Additional Experiments}
\label{sec:addexps}
In this section, we discuss more use-cases for the learned bilateral filters, one
use-case of BNNs and two single filter applications for image and 3D mesh denoising.

\subsubsection{Recognition of subsampled MNIST}\label{sec:app_mnist}

One of the strengths of the proposed filter convolution is that it does not
require the input to lie on a regular grid. The only requirement is to define a distance
between features of the input signal.
We highlight this feature with the following experiment using the
classical MNIST ten class classification problem~\cite{lecun1998mnist}. We sample a
sparse set of $N$ points $(x,y)\in [0,1]\times [0,1]$
uniformly at random in the input image, use their interpolated values
as signal and the \emph{continuous} $(x,y)$ positions as features. This mimics
sub-sampling of a high-dimensional signal. To compare against a spatial convolution,
we interpolate the sparse set of values at the grid positions.

We take a reference implementation of LeNet~\cite{lecun1998gradient} that
is part of the Caffe project~\cite{jia2014caffe} and compare it
against the same architecture but replacing the first convolutional
layer with a bilateral convolution layer (BCL). The filter size
and numbers are adjusted to get a comparable number of parameters
($5\times 5$ for LeNet, $2$-neighborhood for BCL).

The results are shown in Table~\ref{tab:all-results}. We see that training
on the original MNIST data (column Original, LeNet vs. BNN) leads to a slight
decrease in performance of the BNN (99.03\%) compared to LeNet
(99.19\%). The BNN can be trained and evaluated on sparse
signals, and we resample the image as described above for $N=$ 100\%, 60\% and
20\% of the total number of pixels. The methods are also evaluated
on test images that are subsampled in the same way. Note that we can
train and test with different subsampling rates. We introduce an additional
bilinear interpolation layer for the LeNet architecture to train on the same
data. In essence, both models perform a spatial interpolation and thus we
expect them to yield a similar classification accuracy. Once the data is of
higher dimensions, the permutohedral convolution will be faster due to hashing
the sparse input points, as well as less memory demanding in comparison to
naive application of a spatial convolution with interpolated values.

\begin{table}[t]
  \begin{center}
    \footnotesize
    \centering
    \begin{tabular}[t]{lllll}
      \toprule
              &     & \multicolumn{3}{c}{Test Subsampling} \\
       Method  & Original & 100\% & 60\% & 20\%\\
      \midrule
       LeNet &  \textbf{0.9919} & 0.9660 & 0.9348 & \textbf{0.6434} \\
       BNN &  0.9903 & \textbf{0.9844} & \textbf{0.9534} & 0.5767 \\
      \hline
       LeNet 100\% & 0.9856 & 0.9809 & 0.9678 & \textbf{0.7386} \\
       BNN 100\% & \textbf{0.9900} & \textbf{0.9863} & \textbf{0.9699} & 0.6910 \\
      \hline
       LeNet 60\% & 0.9848 & 0.9821 & 0.9740 & 0.8151 \\
       BNN 60\% & \textbf{0.9885} & \textbf{0.9864} & \textbf{0.9771} & \textbf{0.8214}\\
      \hline
       LeNet 20\% & \textbf{0.9763} & \textbf{0.9754} & 0.9695 & 0.8928 \\
       BNN 20\% & 0.9728 & 0.9735 & \textbf{0.9701} & \textbf{0.9042}\\
      \bottomrule
    \end{tabular}
  \end{center}
\vspace{-.2cm}
\caption{Classification accuracy on MNIST. We compare the
    LeNet~\cite{lecun1998gradient} implementation that is part of
    Caffe~\cite{jia2014caffe} to the network with the first layer
    replaced by a bilateral convolution layer (BCL). Both are trained
    on the original image resolution (first two rows). Three more BNN
    and CNN models are trained with randomly subsampled images (100\%,
    60\% and 20\% of the pixels). An additional bilinear interpolation
    layer samples the input signal on a spatial grid for the CNN model.
  }
  \label{tab:all-results}
\vspace{-.5cm}
\end{table}

\subsubsection{Image Denoising}

The main application that inspired the development of the bilateral
filtering operation is image denoising~\cite{aurich1995non}, there
using a single Gaussian kernel. Our development allows to learn this
kernel function from data and we explore how to improve using a \emph{single}
but more general bilateral filter.

We use the Berkeley segmentation dataset
(BSDS500)~\cite{arbelaezi2011bsds500} as a test bed. The color
images in the dataset are converted to gray-scale,
and corrupted with Gaussian noise with a standard deviation of
$\frac {25} {255}$.

We compare the performance of four different filter models on a
denoising task.
The first baseline model (`Spatial' in Table \ref{tab:denoising}, $25$
weights) uses a single spatial filter with a kernel size of
$5$ and predicts the scalar gray-scale value at the center pixel. The next model
(`Gauss Bilateral') applies a bilateral \emph{Gaussian}
filter to the noisy input, using position and intensity features $\f=(x,y,v)^\top$.
The third setup (`Learned Bilateral', $65$ weights)
takes a Gauss kernel as initialization and
fits all filter weights on the train set to minimize the
mean squared error with respect to the clean images.
We run a combination
of spatial and permutohedral convolutions on spatial and bilateral
features (`Spatial + Bilateral (Learned)') to check for a complementary
performance of the two convolutions.

\label{sec:exp:denoising}
\begin{table}[!h]
\begin{center}
  \footnotesize
  \begin{tabular}[t]{lr}
    \toprule
    Method & PSNR \\
    \midrule
    Noisy Input & $20.17$ \\
    Spatial & $26.27$ \\
    Gauss Bilateral & $26.51$ \\
    Learned Bilateral & $26.58$ \\
    Spatial + Bilateral (Learned) & \textbf{$26.65$} \\
    \bottomrule
  \end{tabular}
\end{center}
\vspace{-0.5em}
\caption{PSNR results of a denoising task using the BSDS500
  dataset~\cite{arbelaezi2011bsds500}}
\vspace{-0.5em}
\label{tab:denoising}
\end{table}
\vspace{-0.2em}

The PSNR scores evaluated on full images of the test set are
shown in Table \ref{tab:denoising}. We find that an untrained bilateral
filter already performs better than a trained spatial convolution
($26.27$ to $26.51$). A learned convolution further improve the
performance slightly. We chose this simple one-kernel setup to
validate an advantage of the generalized bilateral filter. A competitive
denoising system would employ RGB color information and also
needs to be properly adjusted in network size. Multi-layer perceptrons
have obtained state-of-the-art denoising results~\cite{burger12cvpr}
and the permutohedral lattice layer can readily be used in such an
architecture, which is intended future work.

\subsection{Additional results}
\label{sec:addresults}

This section contains more qualitative results for the experiments presented in Chapter~\ref{chap:bnn}.

\begin{figure*}[th!]
  \centering
    \includegraphics[width=\columnwidth,trim={5cm 2.5cm 5cm 4.5cm},clip]{figures/supplementary/lattice_viz.pdf}
    \vspace{-0.7cm}
  \mycaption{Visualization of the Permutohedral Lattice}
  {Sample lattice visualizations for different feature spaces. All pixels falling in the same simplex cell are shown with
  the same color. $(x,y)$ features correspond to image pixel positions, and $(r,g,b) \in [0,255]$ correspond
  to the red, green and blue color values.}
\label{fig:latticeviz}
\end{figure*}

\subsubsection{Lattice Visualization}

Figure~\ref{fig:latticeviz} shows sample lattice visualizations for different feature spaces.

\newcolumntype{L}[1]{>{\raggedright\let\newline\\\arraybackslash\hspace{0pt}}b{#1}}
\newcolumntype{C}[1]{>{\centering\let\newline\\\arraybackslash\hspace{0pt}}b{#1}}
\newcolumntype{R}[1]{>{\raggedleft\let\newline\\\arraybackslash\hspace{0pt}}b{#1}}

\subsubsection{Color Upsampling}\label{sec:color_upsampling}
\label{sec:col_upsample_extra}

Some images of the upsampling for the Pascal
VOC12 dataset are shown in Fig.~\ref{fig:Colour_upsample_visuals}. It is
especially the low level image details that are better preserved with
a learned bilateral filter compared to the Gaussian case.

\begin{figure*}[t!]
  \centering
    \subfigure{%
   \raisebox{2.0em}{
    \includegraphics[width=.06\columnwidth]{figures/supplementary/2007_004969.jpg}
   }
  }
  \subfigure{%
    \includegraphics[width=.17\columnwidth]{figures/supplementary/2007_004969_gray.pdf}
  }
  \subfigure{%
    \includegraphics[width=.17\columnwidth]{figures/supplementary/2007_004969_gt.pdf}
  }
  \subfigure{%
    \includegraphics[width=.17\columnwidth]{figures/supplementary/2007_004969_bicubic.pdf}
  }
  \subfigure{%
    \includegraphics[width=.17\columnwidth]{figures/supplementary/2007_004969_gauss.pdf}
  }
  \subfigure{%
    \includegraphics[width=.17\columnwidth]{figures/supplementary/2007_004969_learnt.pdf}
  }\\
    \subfigure{%
   \raisebox{2.0em}{
    \includegraphics[width=.06\columnwidth]{figures/supplementary/2007_003106.jpg}
   }
  }
  \subfigure{%
    \includegraphics[width=.17\columnwidth]{figures/supplementary/2007_003106_gray.pdf}
  }
  \subfigure{%
    \includegraphics[width=.17\columnwidth]{figures/supplementary/2007_003106_gt.pdf}
  }
  \subfigure{%
    \includegraphics[width=.17\columnwidth]{figures/supplementary/2007_003106_bicubic.pdf}
  }
  \subfigure{%
    \includegraphics[width=.17\columnwidth]{figures/supplementary/2007_003106_gauss.pdf}
  }
  \subfigure{%
    \includegraphics[width=.17\columnwidth]{figures/supplementary/2007_003106_learnt.pdf}
  }\\
  \setcounter{subfigure}{0}
  \small{
  \subfigure[Inp.]{%
  \raisebox{2.0em}{
    \includegraphics[width=.06\columnwidth]{figures/supplementary/2007_006837.jpg}
   }
  }
  \subfigure[Guidance]{%
    \includegraphics[width=.17\columnwidth]{figures/supplementary/2007_006837_gray.pdf}
  }
   \subfigure[GT]{%
    \includegraphics[width=.17\columnwidth]{figures/supplementary/2007_006837_gt.pdf}
  }
  \subfigure[Bicubic]{%
    \includegraphics[width=.17\columnwidth]{figures/supplementary/2007_006837_bicubic.pdf}
  }
  \subfigure[Gauss-BF]{%
    \includegraphics[width=.17\columnwidth]{figures/supplementary/2007_006837_gauss.pdf}
  }
  \subfigure[Learned-BF]{%
    \includegraphics[width=.17\columnwidth]{figures/supplementary/2007_006837_learnt.pdf}
  }
  }
  \vspace{-0.5cm}
  \mycaption{Color Upsampling}{Color $8\times$ upsampling results
  using different methods, from left to right, (a)~Low-resolution input color image (Inp.),
  (b)~Gray scale guidance image, (c)~Ground-truth color image; Upsampled color images with
  (d)~Bicubic interpolation, (e) Gauss bilateral upsampling and, (f)~Learned bilateral
  updampgling (best viewed on screen).}

\label{fig:Colour_upsample_visuals}
\end{figure*}

\subsubsection{Depth Upsampling}
\label{sec:depth_upsample_extra}

Figure~\ref{fig:depth_upsample_visuals} presents some more qualitative results comparing bicubic interpolation, Gauss
bilateral and learned bilateral upsampling on NYU depth dataset image~\cite{silberman2012indoor}.

\subsubsection{Character Recognition}\label{sec:app_character}

 Figure~\ref{fig:nnrecognition} shows the schematic of different layers
 of the network architecture for LeNet-7~\cite{lecun1998mnist}
 and DeepCNet(5, 50)~\cite{ciresan2012multi,graham2014spatially}. For the BNN variants, the first layer filters are replaced
 with learned bilateral filters and are learned end-to-end.

\subsubsection{Semantic Segmentation}\label{sec:app_semantic_segmentation}
\label{sec:semantic_bnn_extra}

Some more visual results for semantic segmentation are shown in Figure~\ref{fig:semantic_visuals}.
These include the underlying DeepLab CNN\cite{chen2014semantic} result (DeepLab),
the 2 step mean-field result with Gaussian edge potentials (+2stepMF-GaussCRF)
and also corresponding results with learned edge potentials (+2stepMF-LearnedCRF).
In general, we observe that mean-field in learned CRF leads to slightly dilated
classification regions in comparison to using Gaussian CRF thereby filling-in the
false negative pixels and also correcting some mis-classified regions.

\begin{figure*}[t!]
  \centering
    \subfigure{%
   \raisebox{2.0em}{
    \includegraphics[width=.06\columnwidth]{figures/supplementary/2bicubic}
   }
  }
  \subfigure{%
    \includegraphics[width=.17\columnwidth]{figures/supplementary/2given_image}
  }
  \subfigure{%
    \includegraphics[width=.17\columnwidth]{figures/supplementary/2ground_truth}
  }
  \subfigure{%
    \includegraphics[width=.17\columnwidth]{figures/supplementary/2bicubic}
  }
  \subfigure{%
    \includegraphics[width=.17\columnwidth]{figures/supplementary/2gauss}
  }
  \subfigure{%
    \includegraphics[width=.17\columnwidth]{figures/supplementary/2learnt}
  }\\
    \subfigure{%
   \raisebox{2.0em}{
    \includegraphics[width=.06\columnwidth]{figures/supplementary/32bicubic}
   }
  }
  \subfigure{%
    \includegraphics[width=.17\columnwidth]{figures/supplementary/32given_image}
  }
  \subfigure{%
    \includegraphics[width=.17\columnwidth]{figures/supplementary/32ground_truth}
  }
  \subfigure{%
    \includegraphics[width=.17\columnwidth]{figures/supplementary/32bicubic}
  }
  \subfigure{%
    \includegraphics[width=.17\columnwidth]{figures/supplementary/32gauss}
  }
  \subfigure{%
    \includegraphics[width=.17\columnwidth]{figures/supplementary/32learnt}
  }\\
  \setcounter{subfigure}{0}
  \small{
  \subfigure[Inp.]{%
  \raisebox{2.0em}{
    \includegraphics[width=.06\columnwidth]{figures/supplementary/41bicubic}
   }
  }
  \subfigure[Guidance]{%
    \includegraphics[width=.17\columnwidth]{figures/supplementary/41given_image}
  }
   \subfigure[GT]{%
    \includegraphics[width=.17\columnwidth]{figures/supplementary/41ground_truth}
  }
  \subfigure[Bicubic]{%
    \includegraphics[width=.17\columnwidth]{figures/supplementary/41bicubic}
  }
  \subfigure[Gauss-BF]{%
    \includegraphics[width=.17\columnwidth]{figures/supplementary/41gauss}
  }
  \subfigure[Learned-BF]{%
    \includegraphics[width=.17\columnwidth]{figures/supplementary/41learnt}
  }
  }
  \mycaption{Depth Upsampling}{Depth $8\times$ upsampling results
  using different upsampling strategies, from left to right,
  (a)~Low-resolution input depth image (Inp.),
  (b)~High-resolution guidance image, (c)~Ground-truth depth; Upsampled depth images with
  (d)~Bicubic interpolation, (e) Gauss bilateral upsampling and, (f)~Learned bilateral
  updampgling (best viewed on screen).}

\label{fig:depth_upsample_visuals}
\end{figure*}

\subsubsection{Material Segmentation}\label{sec:app_material_segmentation}
\label{sec:material_bnn_extra}

In Fig.~\ref{fig:material_visuals-app2}, we present visual results comparing 2 step
mean-field inference with Gaussian and learned pairwise CRF potentials. In
general, we observe that the pixels belonging to dominant classes in the
training data are being more accurately classified with learned CRF. This leads to
a significant improvements in overall pixel accuracy. This also results
in a slight decrease of the accuracy from less frequent class pixels thereby
slightly reducing the average class accuracy with learning. We attribute this
to the type of annotation that is available for this dataset, which is not
for the entire image but for some segments in the image. We have very few
images of the infrequent classes to combat this behaviour during training.

\subsubsection{Experiment Protocols}
\label{sec:protocols}

Table~\ref{tbl:parameters} shows experiment protocols of different experiments.

 \begin{figure*}[t!]
  \centering
  \subfigure[LeNet-7]{
    \includegraphics[width=0.7\columnwidth]{figures/supplementary/lenet_cnn_network}
    }\\
    \subfigure[DeepCNet]{
    \includegraphics[width=\columnwidth]{figures/supplementary/deepcnet_cnn_network}
    }
  \mycaption{CNNs for Character Recognition}
  {Schematic of (top) LeNet-7~\cite{lecun1998mnist} and (bottom) DeepCNet(5,50)~\cite{ciresan2012multi,graham2014spatially} architectures used in Assamese
  character recognition experiments.}
\label{fig:nnrecognition}
\end{figure*}

\definecolor{voc_1}{RGB}{0, 0, 0}
\definecolor{voc_2}{RGB}{128, 0, 0}
\definecolor{voc_3}{RGB}{0, 128, 0}
\definecolor{voc_4}{RGB}{128, 128, 0}
\definecolor{voc_5}{RGB}{0, 0, 128}
\definecolor{voc_6}{RGB}{128, 0, 128}
\definecolor{voc_7}{RGB}{0, 128, 128}
\definecolor{voc_8}{RGB}{128, 128, 128}
\definecolor{voc_9}{RGB}{64, 0, 0}
\definecolor{voc_10}{RGB}{192, 0, 0}
\definecolor{voc_11}{RGB}{64, 128, 0}
\definecolor{voc_12}{RGB}{192, 128, 0}
\definecolor{voc_13}{RGB}{64, 0, 128}
\definecolor{voc_14}{RGB}{192, 0, 128}
\definecolor{voc_15}{RGB}{64, 128, 128}
\definecolor{voc_16}{RGB}{192, 128, 128}
\definecolor{voc_17}{RGB}{0, 64, 0}
\definecolor{voc_18}{RGB}{128, 64, 0}
\definecolor{voc_19}{RGB}{0, 192, 0}
\definecolor{voc_20}{RGB}{128, 192, 0}
\definecolor{voc_21}{RGB}{0, 64, 128}
\definecolor{voc_22}{RGB}{128, 64, 128}

\begin{figure*}[t]
  \centering
  \small{
  \fcolorbox{white}{voc_1}{\rule{0pt}{6pt}\rule{6pt}{0pt}} Background~~
  \fcolorbox{white}{voc_2}{\rule{0pt}{6pt}\rule{6pt}{0pt}} Aeroplane~~
  \fcolorbox{white}{voc_3}{\rule{0pt}{6pt}\rule{6pt}{0pt}} Bicycle~~
  \fcolorbox{white}{voc_4}{\rule{0pt}{6pt}\rule{6pt}{0pt}} Bird~~
  \fcolorbox{white}{voc_5}{\rule{0pt}{6pt}\rule{6pt}{0pt}} Boat~~
  \fcolorbox{white}{voc_6}{\rule{0pt}{6pt}\rule{6pt}{0pt}} Bottle~~
  \fcolorbox{white}{voc_7}{\rule{0pt}{6pt}\rule{6pt}{0pt}} Bus~~
  \fcolorbox{white}{voc_8}{\rule{0pt}{6pt}\rule{6pt}{0pt}} Car~~ \\
  \fcolorbox{white}{voc_9}{\rule{0pt}{6pt}\rule{6pt}{0pt}} Cat~~
  \fcolorbox{white}{voc_10}{\rule{0pt}{6pt}\rule{6pt}{0pt}} Chair~~
  \fcolorbox{white}{voc_11}{\rule{0pt}{6pt}\rule{6pt}{0pt}} Cow~~
  \fcolorbox{white}{voc_12}{\rule{0pt}{6pt}\rule{6pt}{0pt}} Dining Table~~
  \fcolorbox{white}{voc_13}{\rule{0pt}{6pt}\rule{6pt}{0pt}} Dog~~
  \fcolorbox{white}{voc_14}{\rule{0pt}{6pt}\rule{6pt}{0pt}} Horse~~
  \fcolorbox{white}{voc_15}{\rule{0pt}{6pt}\rule{6pt}{0pt}} Motorbike~~
  \fcolorbox{white}{voc_16}{\rule{0pt}{6pt}\rule{6pt}{0pt}} Person~~ \\
  \fcolorbox{white}{voc_17}{\rule{0pt}{6pt}\rule{6pt}{0pt}} Potted Plant~~
  \fcolorbox{white}{voc_18}{\rule{0pt}{6pt}\rule{6pt}{0pt}} Sheep~~
  \fcolorbox{white}{voc_19}{\rule{0pt}{6pt}\rule{6pt}{0pt}} Sofa~~
  \fcolorbox{white}{voc_20}{\rule{0pt}{6pt}\rule{6pt}{0pt}} Train~~
  \fcolorbox{white}{voc_21}{\rule{0pt}{6pt}\rule{6pt}{0pt}} TV monitor~~ \\
  }
  \subfigure{%
    \includegraphics[width=.18\columnwidth]{figures/supplementary/2007_001423_given.jpg}
  }
  \subfigure{%
    \includegraphics[width=.18\columnwidth]{figures/supplementary/2007_001423_gt.png}
  }
  \subfigure{%
    \includegraphics[width=.18\columnwidth]{figures/supplementary/2007_001423_cnn.png}
  }
  \subfigure{%
    \includegraphics[width=.18\columnwidth]{figures/supplementary/2007_001423_gauss.png}
  }
  \subfigure{%
    \includegraphics[width=.18\columnwidth]{figures/supplementary/2007_001423_learnt.png}
  }\\
  \subfigure{%
    \includegraphics[width=.18\columnwidth]{figures/supplementary/2007_001430_given.jpg}
  }
  \subfigure{%
    \includegraphics[width=.18\columnwidth]{figures/supplementary/2007_001430_gt.png}
  }
  \subfigure{%
    \includegraphics[width=.18\columnwidth]{figures/supplementary/2007_001430_cnn.png}
  }
  \subfigure{%
    \includegraphics[width=.18\columnwidth]{figures/supplementary/2007_001430_gauss.png}
  }
  \subfigure{%
    \includegraphics[width=.18\columnwidth]{figures/supplementary/2007_001430_learnt.png}
  }\\
    \subfigure{%
    \includegraphics[width=.18\columnwidth]{figures/supplementary/2007_007996_given.jpg}
  }
  \subfigure{%
    \includegraphics[width=.18\columnwidth]{figures/supplementary/2007_007996_gt.png}
  }
  \subfigure{%
    \includegraphics[width=.18\columnwidth]{figures/supplementary/2007_007996_cnn.png}
  }
  \subfigure{%
    \includegraphics[width=.18\columnwidth]{figures/supplementary/2007_007996_gauss.png}
  }
  \subfigure{%
    \includegraphics[width=.18\columnwidth]{figures/supplementary/2007_007996_learnt.png}
  }\\
   \subfigure{%
    \includegraphics[width=.18\columnwidth]{figures/supplementary/2010_002682_given.jpg}
  }
  \subfigure{%
    \includegraphics[width=.18\columnwidth]{figures/supplementary/2010_002682_gt.png}
  }
  \subfigure{%
    \includegraphics[width=.18\columnwidth]{figures/supplementary/2010_002682_cnn.png}
  }
  \subfigure{%
    \includegraphics[width=.18\columnwidth]{figures/supplementary/2010_002682_gauss.png}
  }
  \subfigure{%
    \includegraphics[width=.18\columnwidth]{figures/supplementary/2010_002682_learnt.png}
  }\\
     \subfigure{%
    \includegraphics[width=.18\columnwidth]{figures/supplementary/2010_004789_given.jpg}
  }
  \subfigure{%
    \includegraphics[width=.18\columnwidth]{figures/supplementary/2010_004789_gt.png}
  }
  \subfigure{%
    \includegraphics[width=.18\columnwidth]{figures/supplementary/2010_004789_cnn.png}
  }
  \subfigure{%
    \includegraphics[width=.18\columnwidth]{figures/supplementary/2010_004789_gauss.png}
  }
  \subfigure{%
    \includegraphics[width=.18\columnwidth]{figures/supplementary/2010_004789_learnt.png}
  }\\
       \subfigure{%
    \includegraphics[width=.18\columnwidth]{figures/supplementary/2007_001311_given.jpg}
  }
  \subfigure{%
    \includegraphics[width=.18\columnwidth]{figures/supplementary/2007_001311_gt.png}
  }
  \subfigure{%
    \includegraphics[width=.18\columnwidth]{figures/supplementary/2007_001311_cnn.png}
  }
  \subfigure{%
    \includegraphics[width=.18\columnwidth]{figures/supplementary/2007_001311_gauss.png}
  }
  \subfigure{%
    \includegraphics[width=.18\columnwidth]{figures/supplementary/2007_001311_learnt.png}
  }\\
  \setcounter{subfigure}{0}
  \subfigure[Input]{%
    \includegraphics[width=.18\columnwidth]{figures/supplementary/2010_003531_given.jpg}
  }
  \subfigure[Ground Truth]{%
    \includegraphics[width=.18\columnwidth]{figures/supplementary/2010_003531_gt.png}
  }
  \subfigure[DeepLab]{%
    \includegraphics[width=.18\columnwidth]{figures/supplementary/2010_003531_cnn.png}
  }
  \subfigure[+GaussCRF]{%
    \includegraphics[width=.18\columnwidth]{figures/supplementary/2010_003531_gauss.png}
  }
  \subfigure[+LearnedCRF]{%
    \includegraphics[width=.18\columnwidth]{figures/supplementary/2010_003531_learnt.png}
  }
  \vspace{-0.3cm}
  \mycaption{Semantic Segmentation}{Example results of semantic segmentation.
  (c)~depicts the unary results before application of MF, (d)~after two steps of MF with Gaussian edge CRF potentials, (e)~after
  two steps of MF with learned edge CRF potentials.}
    \label{fig:semantic_visuals}
\end{figure*}


\definecolor{minc_1}{HTML}{771111}
\definecolor{minc_2}{HTML}{CAC690}
\definecolor{minc_3}{HTML}{EEEEEE}
\definecolor{minc_4}{HTML}{7C8FA6}
\definecolor{minc_5}{HTML}{597D31}
\definecolor{minc_6}{HTML}{104410}
\definecolor{minc_7}{HTML}{BB819C}
\definecolor{minc_8}{HTML}{D0CE48}
\definecolor{minc_9}{HTML}{622745}
\definecolor{minc_10}{HTML}{666666}
\definecolor{minc_11}{HTML}{D54A31}
\definecolor{minc_12}{HTML}{101044}
\definecolor{minc_13}{HTML}{444126}
\definecolor{minc_14}{HTML}{75D646}
\definecolor{minc_15}{HTML}{DD4348}
\definecolor{minc_16}{HTML}{5C8577}
\definecolor{minc_17}{HTML}{C78472}
\definecolor{minc_18}{HTML}{75D6D0}
\definecolor{minc_19}{HTML}{5B4586}
\definecolor{minc_20}{HTML}{C04393}
\definecolor{minc_21}{HTML}{D69948}
\definecolor{minc_22}{HTML}{7370D8}
\definecolor{minc_23}{HTML}{7A3622}
\definecolor{minc_24}{HTML}{000000}

\begin{figure*}[t]
  \centering
  \small{
  \fcolorbox{white}{minc_1}{\rule{0pt}{6pt}\rule{6pt}{0pt}} Brick~~
  \fcolorbox{white}{minc_2}{\rule{0pt}{6pt}\rule{6pt}{0pt}} Carpet~~
  \fcolorbox{white}{minc_3}{\rule{0pt}{6pt}\rule{6pt}{0pt}} Ceramic~~
  \fcolorbox{white}{minc_4}{\rule{0pt}{6pt}\rule{6pt}{0pt}} Fabric~~
  \fcolorbox{white}{minc_5}{\rule{0pt}{6pt}\rule{6pt}{0pt}} Foliage~~
  \fcolorbox{white}{minc_6}{\rule{0pt}{6pt}\rule{6pt}{0pt}} Food~~
  \fcolorbox{white}{minc_7}{\rule{0pt}{6pt}\rule{6pt}{0pt}} Glass~~
  \fcolorbox{white}{minc_8}{\rule{0pt}{6pt}\rule{6pt}{0pt}} Hair~~ \\
  \fcolorbox{white}{minc_9}{\rule{0pt}{6pt}\rule{6pt}{0pt}} Leather~~
  \fcolorbox{white}{minc_10}{\rule{0pt}{6pt}\rule{6pt}{0pt}} Metal~~
  \fcolorbox{white}{minc_11}{\rule{0pt}{6pt}\rule{6pt}{0pt}} Mirror~~
  \fcolorbox{white}{minc_12}{\rule{0pt}{6pt}\rule{6pt}{0pt}} Other~~
  \fcolorbox{white}{minc_13}{\rule{0pt}{6pt}\rule{6pt}{0pt}} Painted~~
  \fcolorbox{white}{minc_14}{\rule{0pt}{6pt}\rule{6pt}{0pt}} Paper~~
  \fcolorbox{white}{minc_15}{\rule{0pt}{6pt}\rule{6pt}{0pt}} Plastic~~\\
  \fcolorbox{white}{minc_16}{\rule{0pt}{6pt}\rule{6pt}{0pt}} Polished Stone~~
  \fcolorbox{white}{minc_17}{\rule{0pt}{6pt}\rule{6pt}{0pt}} Skin~~
  \fcolorbox{white}{minc_18}{\rule{0pt}{6pt}\rule{6pt}{0pt}} Sky~~
  \fcolorbox{white}{minc_19}{\rule{0pt}{6pt}\rule{6pt}{0pt}} Stone~~
  \fcolorbox{white}{minc_20}{\rule{0pt}{6pt}\rule{6pt}{0pt}} Tile~~
  \fcolorbox{white}{minc_21}{\rule{0pt}{6pt}\rule{6pt}{0pt}} Wallpaper~~
  \fcolorbox{white}{minc_22}{\rule{0pt}{6pt}\rule{6pt}{0pt}} Water~~
  \fcolorbox{white}{minc_23}{\rule{0pt}{6pt}\rule{6pt}{0pt}} Wood~~ \\
  }
  \subfigure{%
    \includegraphics[width=.18\columnwidth]{figures/supplementary/000010868_given.jpg}
  }
  \subfigure{%
    \includegraphics[width=.18\columnwidth]{figures/supplementary/000010868_gt.png}
  }
  \subfigure{%
    \includegraphics[width=.18\columnwidth]{figures/supplementary/000010868_cnn.png}
  }
  \subfigure{%
    \includegraphics[width=.18\columnwidth]{figures/supplementary/000010868_gauss.png}
  }
  \subfigure{%
    \includegraphics[width=.18\columnwidth]{figures/supplementary/000010868_learnt.png}
  }\\[-2ex]
  \subfigure{%
    \includegraphics[width=.18\columnwidth]{figures/supplementary/000006011_given.jpg}
  }
  \subfigure{%
    \includegraphics[width=.18\columnwidth]{figures/supplementary/000006011_gt.png}
  }
  \subfigure{%
    \includegraphics[width=.18\columnwidth]{figures/supplementary/000006011_cnn.png}
  }
  \subfigure{%
    \includegraphics[width=.18\columnwidth]{figures/supplementary/000006011_gauss.png}
  }
  \subfigure{%
    \includegraphics[width=.18\columnwidth]{figures/supplementary/000006011_learnt.png}
  }\\[-2ex]
    \subfigure{%
    \includegraphics[width=.18\columnwidth]{figures/supplementary/000008553_given.jpg}
  }
  \subfigure{%
    \includegraphics[width=.18\columnwidth]{figures/supplementary/000008553_gt.png}
  }
  \subfigure{%
    \includegraphics[width=.18\columnwidth]{figures/supplementary/000008553_cnn.png}
  }
  \subfigure{%
    \includegraphics[width=.18\columnwidth]{figures/supplementary/000008553_gauss.png}
  }
  \subfigure{%
    \includegraphics[width=.18\columnwidth]{figures/supplementary/000008553_learnt.png}
  }\\[-2ex]
   \subfigure{%
    \includegraphics[width=.18\columnwidth]{figures/supplementary/000009188_given.jpg}
  }
  \subfigure{%
    \includegraphics[width=.18\columnwidth]{figures/supplementary/000009188_gt.png}
  }
  \subfigure{%
    \includegraphics[width=.18\columnwidth]{figures/supplementary/000009188_cnn.png}
  }
  \subfigure{%
    \includegraphics[width=.18\columnwidth]{figures/supplementary/000009188_gauss.png}
  }
  \subfigure{%
    \includegraphics[width=.18\columnwidth]{figures/supplementary/000009188_learnt.png}
  }\\[-2ex]
  \setcounter{subfigure}{0}
  \subfigure[Input]{%
    \includegraphics[width=.18\columnwidth]{figures/supplementary/000023570_given.jpg}
  }
  \subfigure[Ground Truth]{%
    \includegraphics[width=.18\columnwidth]{figures/supplementary/000023570_gt.png}
  }
  \subfigure[DeepLab]{%
    \includegraphics[width=.18\columnwidth]{figures/supplementary/000023570_cnn.png}
  }
  \subfigure[+GaussCRF]{%
    \includegraphics[width=.18\columnwidth]{figures/supplementary/000023570_gauss.png}
  }
  \subfigure[+LearnedCRF]{%
    \includegraphics[width=.18\columnwidth]{figures/supplementary/000023570_learnt.png}
  }
  \mycaption{Material Segmentation}{Example results of material segmentation.
  (c)~depicts the unary results before application of MF, (d)~after two steps of MF with Gaussian edge CRF potentials, (e)~after two steps of MF with learned edge CRF potentials.}
    \label{fig:material_visuals-app2}
\end{figure*}


\begin{table*}[h]
\tiny
  \centering
    \begin{tabular}{L{2.3cm} L{2.25cm} C{1.5cm} C{0.7cm} C{0.6cm} C{0.7cm} C{0.7cm} C{0.7cm} C{1.6cm} C{0.6cm} C{0.6cm} C{0.6cm}}
      \toprule
& & & & & \multicolumn{3}{c}{\textbf{Data Statistics}} & \multicolumn{4}{c}{\textbf{Training Protocol}} \\

\textbf{Experiment} & \textbf{Feature Types} & \textbf{Feature Scales} & \textbf{Filter Size} & \textbf{Filter Nbr.} & \textbf{Train}  & \textbf{Val.} & \textbf{Test} & \textbf{Loss Type} & \textbf{LR} & \textbf{Batch} & \textbf{Epochs} \\
      \midrule
      \multicolumn{2}{c}{\textbf{Single Bilateral Filter Applications}} & & & & & & & & & \\
      \textbf{2$\times$ Color Upsampling} & Position$_{1}$, Intensity (3D) & 0.13, 0.17 & 65 & 2 & 10581 & 1449 & 1456 & MSE & 1e-06 & 200 & 94.5\\
      \textbf{4$\times$ Color Upsampling} & Position$_{1}$, Intensity (3D) & 0.06, 0.17 & 65 & 2 & 10581 & 1449 & 1456 & MSE & 1e-06 & 200 & 94.5\\
      \textbf{8$\times$ Color Upsampling} & Position$_{1}$, Intensity (3D) & 0.03, 0.17 & 65 & 2 & 10581 & 1449 & 1456 & MSE & 1e-06 & 200 & 94.5\\
      \textbf{16$\times$ Color Upsampling} & Position$_{1}$, Intensity (3D) & 0.02, 0.17 & 65 & 2 & 10581 & 1449 & 1456 & MSE & 1e-06 & 200 & 94.5\\
      \textbf{Depth Upsampling} & Position$_{1}$, Color (5D) & 0.05, 0.02 & 665 & 2 & 795 & 100 & 654 & MSE & 1e-07 & 50 & 251.6\\
      \textbf{Mesh Denoising} & Isomap (4D) & 46.00 & 63 & 2 & 1000 & 200 & 500 & MSE & 100 & 10 & 100.0 \\
      \midrule
      \multicolumn{2}{c}{\textbf{DenseCRF Applications}} & & & & & & & & &\\
      \multicolumn{2}{l}{\textbf{Semantic Segmentation}} & & & & & & & & &\\
      \textbf{- 1step MF} & Position$_{1}$, Color (5D); Position$_{1}$ (2D) & 0.01, 0.34; 0.34  & 665; 19  & 2; 2 & 10581 & 1449 & 1456 & Logistic & 0.1 & 5 & 1.4 \\
      \textbf{- 2step MF} & Position$_{1}$, Color (5D); Position$_{1}$ (2D) & 0.01, 0.34; 0.34 & 665; 19 & 2; 2 & 10581 & 1449 & 1456 & Logistic & 0.1 & 5 & 1.4 \\
      \textbf{- \textit{loose} 2step MF} & Position$_{1}$, Color (5D); Position$_{1}$ (2D) & 0.01, 0.34; 0.34 & 665; 19 & 2; 2 &10581 & 1449 & 1456 & Logistic & 0.1 & 5 & +1.9  \\ \\
      \multicolumn{2}{l}{\textbf{Material Segmentation}} & & & & & & & & &\\
      \textbf{- 1step MF} & Position$_{2}$, Lab-Color (5D) & 5.00, 0.05, 0.30  & 665 & 2 & 928 & 150 & 1798 & Weighted Logistic & 1e-04 & 24 & 2.6 \\
      \textbf{- 2step MF} & Position$_{2}$, Lab-Color (5D) & 5.00, 0.05, 0.30 & 665 & 2 & 928 & 150 & 1798 & Weighted Logistic & 1e-04 & 12 & +0.7 \\
      \textbf{- \textit{loose} 2step MF} & Position$_{2}$, Lab-Color (5D) & 5.00, 0.05, 0.30 & 665 & 2 & 928 & 150 & 1798 & Weighted Logistic & 1e-04 & 12 & +0.2\\
      \midrule
      \multicolumn{2}{c}{\textbf{Neural Network Applications}} & & & & & & & & &\\
      \textbf{Tiles: CNN-9$\times$9} & - & - & 81 & 4 & 10000 & 1000 & 1000 & Logistic & 0.01 & 100 & 500.0 \\
      \textbf{Tiles: CNN-13$\times$13} & - & - & 169 & 6 & 10000 & 1000 & 1000 & Logistic & 0.01 & 100 & 500.0 \\
      \textbf{Tiles: CNN-17$\times$17} & - & - & 289 & 8 & 10000 & 1000 & 1000 & Logistic & 0.01 & 100 & 500.0 \\
      \textbf{Tiles: CNN-21$\times$21} & - & - & 441 & 10 & 10000 & 1000 & 1000 & Logistic & 0.01 & 100 & 500.0 \\
      \textbf{Tiles: BNN} & Position$_{1}$, Color (5D) & 0.05, 0.04 & 63 & 1 & 10000 & 1000 & 1000 & Logistic & 0.01 & 100 & 30.0 \\
      \textbf{LeNet} & - & - & 25 & 2 & 5490 & 1098 & 1647 & Logistic & 0.1 & 100 & 182.2 \\
      \textbf{Crop-LeNet} & - & - & 25 & 2 & 5490 & 1098 & 1647 & Logistic & 0.1 & 100 & 182.2 \\
      \textbf{BNN-LeNet} & Position$_{2}$ (2D) & 20.00 & 7 & 1 & 5490 & 1098 & 1647 & Logistic & 0.1 & 100 & 182.2 \\
      \textbf{DeepCNet} & - & - & 9 & 1 & 5490 & 1098 & 1647 & Logistic & 0.1 & 100 & 182.2 \\
      \textbf{Crop-DeepCNet} & - & - & 9 & 1 & 5490 & 1098 & 1647 & Logistic & 0.1 & 100 & 182.2 \\
      \textbf{BNN-DeepCNet} & Position$_{2}$ (2D) & 40.00  & 7 & 1 & 5490 & 1098 & 1647 & Logistic & 0.1 & 100 & 182.2 \\
      \bottomrule
      \\
    \end{tabular}
    \mycaption{Experiment Protocols} {Experiment protocols for the different experiments presented in this work. \textbf{Feature Types}:
    Feature spaces used for the bilateral convolutions. Position$_1$ corresponds to un-normalized pixel positions whereas Position$_2$ corresponds
    to pixel positions normalized to $[0,1]$ with respect to the given image. \textbf{Feature Scales}: Cross-validated scales for the features used.
     \textbf{Filter Size}: Number of elements in the filter that is being learned. \textbf{Filter Nbr.}: Half-width of the filter. \textbf{Train},
     \textbf{Val.} and \textbf{Test} corresponds to the number of train, validation and test images used in the experiment. \textbf{Loss Type}: Type
     of loss used for back-propagation. ``MSE'' corresponds to Euclidean mean squared error loss and ``Logistic'' corresponds to multinomial logistic
     loss. ``Weighted Logistic'' is the class-weighted multinomial logistic loss. We weighted the loss with inverse class probability for material
     segmentation task due to the small availability of training data with class imbalance. \textbf{LR}: Fixed learning rate used in stochastic gradient
     descent. \textbf{Batch}: Number of images used in one parameter update step. \textbf{Epochs}: Number of training epochs. In all the experiments,
     we used fixed momentum of 0.9 and weight decay of 0.0005 for stochastic gradient descent. ```Color Upsampling'' experiments in this Table corresponds
     to those performed on Pascal VOC12 dataset images. For all experiments using Pascal VOC12 images, we use extended
     training segmentation dataset available from~\cite{hariharan2011moredata}, and used standard validation and test splits
     from the main dataset~\cite{voc2012segmentation}.}
  \label{tbl:parameters}
\end{table*}

\clearpage

\section{Parameters and Additional Results for Video Propagation Networks}

In this Section, we present experiment protocols and additional qualitative results for experiments
on video object segmentation, semantic video segmentation and video color
propagation. Table~\ref{tbl:parameters_supp} shows the feature scales and other parameters used in different experiments.
Figures~\ref{fig:video_seg_pos_supp} show some qualitative results on video object segmentation
with some failure cases in Fig.~\ref{fig:video_seg_neg_supp}.
Figure~\ref{fig:semantic_visuals_supp} shows some qualitative results on semantic video segmentation and
Fig.~\ref{fig:color_visuals_supp} shows results on video color propagation.

\newcolumntype{L}[1]{>{\raggedright\let\newline\\\arraybackslash\hspace{0pt}}b{#1}}
\newcolumntype{C}[1]{>{\centering\let\newline\\\arraybackslash\hspace{0pt}}b{#1}}
\newcolumntype{R}[1]{>{\raggedleft\let\newline\\\arraybackslash\hspace{0pt}}b{#1}}

\begin{table*}[h]
\tiny
  \centering
    \begin{tabular}{L{3.0cm} L{2.4cm} L{2.8cm} L{2.8cm} C{0.5cm} C{1.0cm} L{1.2cm}}
      \toprule
\textbf{Experiment} & \textbf{Feature Type} & \textbf{Feature Scale-1, $\Lambda_a$} & \textbf{Feature Scale-2, $\Lambda_b$} & \textbf{$\alpha$} & \textbf{Input Frames} & \textbf{Loss Type} \\
      \midrule
      \textbf{Video Object Segmentation} & ($x,y,Y,Cb,Cr,t$) & (0.02,0.02,0.07,0.4,0.4,0.01) & (0.03,0.03,0.09,0.5,0.5,0.2) & 0.5 & 9 & Logistic\\
      \midrule
      \textbf{Semantic Video Segmentation} & & & & & \\
      \textbf{with CNN1~\cite{yu2015multi}-NoFlow} & ($x,y,R,G,B,t$) & (0.08,0.08,0.2,0.2,0.2,0.04) & (0.11,0.11,0.2,0.2,0.2,0.04) & 0.5 & 3 & Logistic \\
      \textbf{with CNN1~\cite{yu2015multi}-Flow} & ($x+u_x,y+u_y,R,G,B,t$) & (0.11,0.11,0.14,0.14,0.14,0.03) & (0.08,0.08,0.12,0.12,0.12,0.01) & 0.65 & 3 & Logistic\\
      \textbf{with CNN2~\cite{richter2016playing}-Flow} & ($x+u_x,y+u_y,R,G,B,t$) & (0.08,0.08,0.2,0.2,0.2,0.04) & (0.09,0.09,0.25,0.25,0.25,0.03) & 0.5 & 4 & Logistic\\
      \midrule
      \textbf{Video Color Propagation} & ($x,y,I,t$)  & (0.04,0.04,0.2,0.04) & No second kernel & 1 & 4 & MSE\\
      \bottomrule
      \\
    \end{tabular}
    \mycaption{Experiment Protocols} {Experiment protocols for the different experiments presented in this work. \textbf{Feature Types}:
    Feature spaces used for the bilateral convolutions, with position ($x,y$) and color
    ($R,G,B$ or $Y,Cb,Cr$) features $\in [0,255]$. $u_x$, $u_y$ denotes optical flow with respect
    to the present frame and $I$ denotes grayscale intensity.
    \textbf{Feature Scales ($\Lambda_a, \Lambda_b$)}: Cross-validated scales for the features used.
    \textbf{$\alpha$}: Exponential time decay for the input frames.
    \textbf{Input Frames}: Number of input frames for VPN.
    \textbf{Loss Type}: Type
     of loss used for back-propagation. ``MSE'' corresponds to Euclidean mean squared error loss and ``Logistic'' corresponds to multinomial logistic loss.}
  \label{tbl:parameters_supp}
\end{table*}

% \begin{figure}[th!]
% \begin{center}
%   \centerline{\includegraphics[width=\textwidth]{figures/video_seg_visuals_supp_small.pdf}}
%     \mycaption{Video Object Segmentation}
%     {Shown are the different frames in example videos with the corresponding
%     ground truth (GT) masks, predictions from BVS~\cite{marki2016bilateral},
%     OFL~\cite{tsaivideo}, VPN (VPN-Stage2) and VPN-DLab (VPN-DeepLab) models.}
%     \label{fig:video_seg_small_supp}
% \end{center}
% \vspace{-1.0cm}
% \end{figure}

\begin{figure}[th!]
\begin{center}
  \centerline{\includegraphics[width=0.7\textwidth]{figures/video_seg_visuals_supp_positive.pdf}}
    \mycaption{Video Object Segmentation}
    {Shown are the different frames in example videos with the corresponding
    ground truth (GT) masks, predictions from BVS~\cite{marki2016bilateral},
    OFL~\cite{tsaivideo}, VPN (VPN-Stage2) and VPN-DLab (VPN-DeepLab) models.}
    \label{fig:video_seg_pos_supp}
\end{center}
\vspace{-1.0cm}
\end{figure}

\begin{figure}[th!]
\begin{center}
  \centerline{\includegraphics[width=0.7\textwidth]{figures/video_seg_visuals_supp_negative.pdf}}
    \mycaption{Failure Cases for Video Object Segmentation}
    {Shown are the different frames in example videos with the corresponding
    ground truth (GT) masks, predictions from BVS~\cite{marki2016bilateral},
    OFL~\cite{tsaivideo}, VPN (VPN-Stage2) and VPN-DLab (VPN-DeepLab) models.}
    \label{fig:video_seg_neg_supp}
\end{center}
\vspace{-1.0cm}
\end{figure}

\begin{figure}[th!]
\begin{center}
  \centerline{\includegraphics[width=0.9\textwidth]{figures/supp_semantic_visual.pdf}}
    \mycaption{Semantic Video Segmentation}
    {Input video frames and the corresponding ground truth (GT)
    segmentation together with the predictions of CNN~\cite{yu2015multi} and with
    VPN-Flow.}
    \label{fig:semantic_visuals_supp}
\end{center}
\vspace{-0.7cm}
\end{figure}

\begin{figure}[th!]
\begin{center}
  \centerline{\includegraphics[width=\textwidth]{figures/colorization_visuals_supp.pdf}}
  \mycaption{Video Color Propagation}
  {Input grayscale video frames and corresponding ground-truth (GT) color images
  together with color predictions of Levin et al.~\cite{levin2004colorization} and VPN-Stage1 models.}
  \label{fig:color_visuals_supp}
\end{center}
\vspace{-0.7cm}
\end{figure}

\clearpage

\section{Additional Material for Bilateral Inception Networks}
\label{sec:binception-app}

In this section of the Appendix, we first discuss the use of approximate bilateral
filtering in BI modules (Sec.~\ref{sec:lattice}).
Later, we present some qualitative results using different models for the approach presented in
Chapter~\ref{chap:binception} (Sec.~\ref{sec:qualitative-app}).

\subsection{Approximate Bilateral Filtering}
\label{sec:lattice}

The bilateral inception module presented in Chapter~\ref{chap:binception} computes a matrix-vector
product between a Gaussian filter $K$ and a vector of activations $\bz_c$.
Bilateral filtering is an important operation and many algorithmic techniques have been
proposed to speed-up this operation~\cite{paris2006fast,adams2010fast,gastal2011domain}.
In the main paper we opted to implement what can be considered the
brute-force variant of explicitly constructing $K$ and then using BLAS to compute the
matrix-vector product. This resulted in a few millisecond operation.
The explicit way to compute is possible due to the
reduction to super-pixels, e.g., it would not work for DenseCRF variants
that operate on the full image resolution.

Here, we present experiments where we use the fast approximate bilateral filtering
algorithm of~\cite{adams2010fast}, which is also used in Chapter~\ref{chap:bnn}
for learning sparse high dimensional filters. This
choice allows for larger dimensions of matrix-vector multiplication. The reason for choosing
the explicit multiplication in Chapter~\ref{chap:binception} was that it was computationally faster.
For the small sizes of the involved matrices and vectors, the explicit computation is sufficient and we had no
GPU implementation of an approximate technique that matched this runtime. Also it
is conceptually easier and the gradient to the feature transformations ($\Lambda \mathbf{f}$) is
obtained using standard matrix calculus.

\subsubsection{Experiments}

We modified the existing segmentation architectures analogous to those in Chapter~\ref{chap:binception}.
The main difference is that, here, the inception modules use the lattice
approximation~\cite{adams2010fast} to compute the bilateral filtering.
Using the lattice approximation did not allow us to back-propagate through feature transformations ($\Lambda$)
and thus we used hand-specified feature scales as will be explained later.
Specifically, we take CNN architectures from the works
of~\cite{chen2014semantic,zheng2015conditional,bell2015minc} and insert the BI modules between
the spatial FC layers.
We use superpixels from~\cite{DollarICCV13edges}
for all the experiments with the lattice approximation. Experiments are
performed using Caffe neural network framework~\cite{jia2014caffe}.

\begin{table}
  \small
  \centering
  \begin{tabular}{p{5.5cm}>{\raggedright\arraybackslash}p{1.4cm}>{\centering\arraybackslash}p{2.2cm}}
    \toprule
		\textbf{Model} & \emph{IoU} & \emph{Runtime}(ms) \\
    \midrule

    %%%%%%%%%%%% Scores computed by us)%%%%%%%%%%%%
		\deeplablargefov & 68.9 & 145ms\\
    \midrule
    \bi{7}{2}-\bi{8}{10}& \textbf{73.8} & +600 \\
    \midrule
    \deeplablargefovcrf~\cite{chen2014semantic} & 72.7 & +830\\
    \deeplabmsclargefovcrf~\cite{chen2014semantic} & \textbf{73.6} & +880\\
    DeepLab-EdgeNet~\cite{chen2015semantic} & 71.7 & +30\\
    DeepLab-EdgeNet-CRF~\cite{chen2015semantic} & \textbf{73.6} & +860\\
  \bottomrule \\
  \end{tabular}
  \mycaption{Semantic Segmentation using the DeepLab model}
  {IoU scores on the Pascal VOC12 segmentation test dataset
  with different models and our modified inception model.
  Also shown are the corresponding runtimes in milliseconds. Runtimes
  also include superpixel computations (300 ms with Dollar superpixels~\cite{DollarICCV13edges})}
  \label{tab:largefovresults}
\end{table}

\paragraph{Semantic Segmentation}
The experiments in this section use the Pascal VOC12 segmentation dataset~\cite{voc2012segmentation} with 21 object classes and the images have a maximum resolution of 0.25 megapixels.
For all experiments on VOC12, we train using the extended training set of
10581 images collected by~\cite{hariharan2011moredata}.
We modified the \deeplab~network architecture of~\cite{chen2014semantic} and
the CRFasRNN architecture from~\cite{zheng2015conditional} which uses a CNN with
deconvolution layers followed by DenseCRF trained end-to-end.

\paragraph{DeepLab Model}\label{sec:deeplabmodel}
We experimented with the \bi{7}{2}-\bi{8}{10} inception model.
Results using the~\deeplab~model are summarized in Tab.~\ref{tab:largefovresults}.
Although we get similar improvements with inception modules as with the
explicit kernel computation, using lattice approximation is slower.

\begin{table}
  \small
  \centering
  \begin{tabular}{p{6.4cm}>{\raggedright\arraybackslash}p{1.8cm}>{\raggedright\arraybackslash}p{1.8cm}}
    \toprule
    \textbf{Model} & \emph{IoU (Val)} & \emph{IoU (Test)}\\
    \midrule
    %%%%%%%%%%%% Scores computed by us)%%%%%%%%%%%%
    CNN &  67.5 & - \\
    \deconv (CNN+Deconvolutions) & 69.8 & 72.0 \\
    \midrule
    \bi{3}{6}-\bi{4}{6}-\bi{7}{2}-\bi{8}{6}& 71.9 & - \\
    \bi{3}{6}-\bi{4}{6}-\bi{7}{2}-\bi{8}{6}-\gi{6}& 73.6 &  \href{http://host.robots.ox.ac.uk:8080/anonymous/VOTV5E.html}{\textbf{75.2}}\\
    \midrule
    \deconvcrf (CRF-RNN)~\cite{zheng2015conditional} & 73.0 & 74.7\\
    Context-CRF-RNN~\cite{yu2015multi} & ~~ - ~ & \textbf{75.3} \\
    \bottomrule \\
  \end{tabular}
  \mycaption{Semantic Segmentation using the CRFasRNN model}{IoU score corresponding to different models
  on Pascal VOC12 reduced validation / test segmentation dataset. The reduced validation set consists of 346 images
  as used in~\cite{zheng2015conditional} where we adapted the model from.}
  \label{tab:deconvresults-app}
\end{table}

\paragraph{CRFasRNN Model}\label{sec:deepinception}
We add BI modules after score-pool3, score-pool4, \fc{7} and \fc{8} $1\times1$ convolution layers
resulting in the \bi{3}{6}-\bi{4}{6}-\bi{7}{2}-\bi{8}{6}
model and also experimented with another variant where $BI_8$ is followed by another inception
module, G$(6)$, with 6 Gaussian kernels.
Note that here also we discarded both deconvolution and DenseCRF parts of the original model~\cite{zheng2015conditional}
and inserted the BI modules in the base CNN and found similar improvements compared to the inception modules with explicit
kernel computaion. See Tab.~\ref{tab:deconvresults-app} for results on the CRFasRNN model.

\paragraph{Material Segmentation}
Table~\ref{tab:mincresults-app} shows the results on the MINC dataset~\cite{bell2015minc}
obtained by modifying the AlexNet architecture with our inception modules. We observe
similar improvements as with explicit kernel construction.
For this model, we do not provide any learned setup due to very limited segment training
data. The weights to combine outputs in the bilateral inception layer are
found by validation on the validation set.

\begin{table}[t]
  \small
  \centering
  \begin{tabular}{p{3.5cm}>{\centering\arraybackslash}p{4.0cm}}
    \toprule
    \textbf{Model} & Class / Total accuracy\\
    \midrule

    %%%%%%%%%%%% Scores computed by us)%%%%%%%%%%%%
    AlexNet CNN & 55.3 / 58.9 \\
    \midrule
    \bi{7}{2}-\bi{8}{6}& 68.5 / 71.8 \\
    \bi{7}{2}-\bi{8}{6}-G$(6)$& 67.6 / 73.1 \\
    \midrule
    AlexNet-CRF & 65.5 / 71.0 \\
    \bottomrule \\
  \end{tabular}
  \mycaption{Material Segmentation using AlexNet}{Pixel accuracy of different models on
  the MINC material segmentation test dataset~\cite{bell2015minc}.}
  \label{tab:mincresults-app}
\end{table}

\paragraph{Scales of Bilateral Inception Modules}
\label{sec:scales}

Unlike the explicit kernel technique presented in the main text (Chapter~\ref{chap:binception}),
we didn't back-propagate through feature transformation ($\Lambda$)
using the approximate bilateral filter technique.
So, the feature scales are hand-specified and validated, which are as follows.
The optimal scale values for the \bi{7}{2}-\bi{8}{2} model are found by validation for the best performance which are
$\sigma_{xy}$ = (0.1, 0.1) for the spatial (XY) kernel and $\sigma_{rgbxy}$ = (0.1, 0.1, 0.1, 0.01, 0.01) for color and position (RGBXY)  kernel.
Next, as more kernels are added to \bi{8}{2}, we set scales to be $\alpha$*($\sigma_{xy}$, $\sigma_{rgbxy}$).
The value of $\alpha$ is chosen as  1, 0.5, 0.1, 0.05, 0.1, at uniform interval, for the \bi{8}{10} bilateral inception module.


\subsection{Qualitative Results}
\label{sec:qualitative-app}

In this section, we present more qualitative results obtained using the BI module with explicit
kernel computation technique presented in Chapter~\ref{chap:binception}. Results on the Pascal VOC12
dataset~\cite{voc2012segmentation} using the DeepLab-LargeFOV model are shown in Fig.~\ref{fig:semantic_visuals-app},
followed by the results on MINC dataset~\cite{bell2015minc}
in Fig.~\ref{fig:material_visuals-app} and on
Cityscapes dataset~\cite{Cordts2015Cvprw} in Fig.~\ref{fig:street_visuals-app}.


\definecolor{voc_1}{RGB}{0, 0, 0}
\definecolor{voc_2}{RGB}{128, 0, 0}
\definecolor{voc_3}{RGB}{0, 128, 0}
\definecolor{voc_4}{RGB}{128, 128, 0}
\definecolor{voc_5}{RGB}{0, 0, 128}
\definecolor{voc_6}{RGB}{128, 0, 128}
\definecolor{voc_7}{RGB}{0, 128, 128}
\definecolor{voc_8}{RGB}{128, 128, 128}
\definecolor{voc_9}{RGB}{64, 0, 0}
\definecolor{voc_10}{RGB}{192, 0, 0}
\definecolor{voc_11}{RGB}{64, 128, 0}
\definecolor{voc_12}{RGB}{192, 128, 0}
\definecolor{voc_13}{RGB}{64, 0, 128}
\definecolor{voc_14}{RGB}{192, 0, 128}
\definecolor{voc_15}{RGB}{64, 128, 128}
\definecolor{voc_16}{RGB}{192, 128, 128}
\definecolor{voc_17}{RGB}{0, 64, 0}
\definecolor{voc_18}{RGB}{128, 64, 0}
\definecolor{voc_19}{RGB}{0, 192, 0}
\definecolor{voc_20}{RGB}{128, 192, 0}
\definecolor{voc_21}{RGB}{0, 64, 128}
\definecolor{voc_22}{RGB}{128, 64, 128}

\begin{figure*}[!ht]
  \small
  \centering
  \fcolorbox{white}{voc_1}{\rule{0pt}{4pt}\rule{4pt}{0pt}} Background~~
  \fcolorbox{white}{voc_2}{\rule{0pt}{4pt}\rule{4pt}{0pt}} Aeroplane~~
  \fcolorbox{white}{voc_3}{\rule{0pt}{4pt}\rule{4pt}{0pt}} Bicycle~~
  \fcolorbox{white}{voc_4}{\rule{0pt}{4pt}\rule{4pt}{0pt}} Bird~~
  \fcolorbox{white}{voc_5}{\rule{0pt}{4pt}\rule{4pt}{0pt}} Boat~~
  \fcolorbox{white}{voc_6}{\rule{0pt}{4pt}\rule{4pt}{0pt}} Bottle~~
  \fcolorbox{white}{voc_7}{\rule{0pt}{4pt}\rule{4pt}{0pt}} Bus~~
  \fcolorbox{white}{voc_8}{\rule{0pt}{4pt}\rule{4pt}{0pt}} Car~~\\
  \fcolorbox{white}{voc_9}{\rule{0pt}{4pt}\rule{4pt}{0pt}} Cat~~
  \fcolorbox{white}{voc_10}{\rule{0pt}{4pt}\rule{4pt}{0pt}} Chair~~
  \fcolorbox{white}{voc_11}{\rule{0pt}{4pt}\rule{4pt}{0pt}} Cow~~
  \fcolorbox{white}{voc_12}{\rule{0pt}{4pt}\rule{4pt}{0pt}} Dining Table~~
  \fcolorbox{white}{voc_13}{\rule{0pt}{4pt}\rule{4pt}{0pt}} Dog~~
  \fcolorbox{white}{voc_14}{\rule{0pt}{4pt}\rule{4pt}{0pt}} Horse~~
  \fcolorbox{white}{voc_15}{\rule{0pt}{4pt}\rule{4pt}{0pt}} Motorbike~~
  \fcolorbox{white}{voc_16}{\rule{0pt}{4pt}\rule{4pt}{0pt}} Person~~\\
  \fcolorbox{white}{voc_17}{\rule{0pt}{4pt}\rule{4pt}{0pt}} Potted Plant~~
  \fcolorbox{white}{voc_18}{\rule{0pt}{4pt}\rule{4pt}{0pt}} Sheep~~
  \fcolorbox{white}{voc_19}{\rule{0pt}{4pt}\rule{4pt}{0pt}} Sofa~~
  \fcolorbox{white}{voc_20}{\rule{0pt}{4pt}\rule{4pt}{0pt}} Train~~
  \fcolorbox{white}{voc_21}{\rule{0pt}{4pt}\rule{4pt}{0pt}} TV monitor~~\\


  \subfigure{%
    \includegraphics[width=.15\columnwidth]{figures/supplementary/2008_001308_given.png}
  }
  \subfigure{%
    \includegraphics[width=.15\columnwidth]{figures/supplementary/2008_001308_sp.png}
  }
  \subfigure{%
    \includegraphics[width=.15\columnwidth]{figures/supplementary/2008_001308_gt.png}
  }
  \subfigure{%
    \includegraphics[width=.15\columnwidth]{figures/supplementary/2008_001308_cnn.png}
  }
  \subfigure{%
    \includegraphics[width=.15\columnwidth]{figures/supplementary/2008_001308_crf.png}
  }
  \subfigure{%
    \includegraphics[width=.15\columnwidth]{figures/supplementary/2008_001308_ours.png}
  }\\[-2ex]


  \subfigure{%
    \includegraphics[width=.15\columnwidth]{figures/supplementary/2008_001821_given.png}
  }
  \subfigure{%
    \includegraphics[width=.15\columnwidth]{figures/supplementary/2008_001821_sp.png}
  }
  \subfigure{%
    \includegraphics[width=.15\columnwidth]{figures/supplementary/2008_001821_gt.png}
  }
  \subfigure{%
    \includegraphics[width=.15\columnwidth]{figures/supplementary/2008_001821_cnn.png}
  }
  \subfigure{%
    \includegraphics[width=.15\columnwidth]{figures/supplementary/2008_001821_crf.png}
  }
  \subfigure{%
    \includegraphics[width=.15\columnwidth]{figures/supplementary/2008_001821_ours.png}
  }\\[-2ex]



  \subfigure{%
    \includegraphics[width=.15\columnwidth]{figures/supplementary/2008_004612_given.png}
  }
  \subfigure{%
    \includegraphics[width=.15\columnwidth]{figures/supplementary/2008_004612_sp.png}
  }
  \subfigure{%
    \includegraphics[width=.15\columnwidth]{figures/supplementary/2008_004612_gt.png}
  }
  \subfigure{%
    \includegraphics[width=.15\columnwidth]{figures/supplementary/2008_004612_cnn.png}
  }
  \subfigure{%
    \includegraphics[width=.15\columnwidth]{figures/supplementary/2008_004612_crf.png}
  }
  \subfigure{%
    \includegraphics[width=.15\columnwidth]{figures/supplementary/2008_004612_ours.png}
  }\\[-2ex]


  \subfigure{%
    \includegraphics[width=.15\columnwidth]{figures/supplementary/2009_001008_given.png}
  }
  \subfigure{%
    \includegraphics[width=.15\columnwidth]{figures/supplementary/2009_001008_sp.png}
  }
  \subfigure{%
    \includegraphics[width=.15\columnwidth]{figures/supplementary/2009_001008_gt.png}
  }
  \subfigure{%
    \includegraphics[width=.15\columnwidth]{figures/supplementary/2009_001008_cnn.png}
  }
  \subfigure{%
    \includegraphics[width=.15\columnwidth]{figures/supplementary/2009_001008_crf.png}
  }
  \subfigure{%
    \includegraphics[width=.15\columnwidth]{figures/supplementary/2009_001008_ours.png}
  }\\[-2ex]




  \subfigure{%
    \includegraphics[width=.15\columnwidth]{figures/supplementary/2009_004497_given.png}
  }
  \subfigure{%
    \includegraphics[width=.15\columnwidth]{figures/supplementary/2009_004497_sp.png}
  }
  \subfigure{%
    \includegraphics[width=.15\columnwidth]{figures/supplementary/2009_004497_gt.png}
  }
  \subfigure{%
    \includegraphics[width=.15\columnwidth]{figures/supplementary/2009_004497_cnn.png}
  }
  \subfigure{%
    \includegraphics[width=.15\columnwidth]{figures/supplementary/2009_004497_crf.png}
  }
  \subfigure{%
    \includegraphics[width=.15\columnwidth]{figures/supplementary/2009_004497_ours.png}
  }\\[-2ex]



  \setcounter{subfigure}{0}
  \subfigure[\scriptsize Input]{%
    \includegraphics[width=.15\columnwidth]{figures/supplementary/2010_001327_given.png}
  }
  \subfigure[\scriptsize Superpixels]{%
    \includegraphics[width=.15\columnwidth]{figures/supplementary/2010_001327_sp.png}
  }
  \subfigure[\scriptsize GT]{%
    \includegraphics[width=.15\columnwidth]{figures/supplementary/2010_001327_gt.png}
  }
  \subfigure[\scriptsize Deeplab]{%
    \includegraphics[width=.15\columnwidth]{figures/supplementary/2010_001327_cnn.png}
  }
  \subfigure[\scriptsize +DenseCRF]{%
    \includegraphics[width=.15\columnwidth]{figures/supplementary/2010_001327_crf.png}
  }
  \subfigure[\scriptsize Using BI]{%
    \includegraphics[width=.15\columnwidth]{figures/supplementary/2010_001327_ours.png}
  }
  \mycaption{Semantic Segmentation}{Example results of semantic segmentation
  on the Pascal VOC12 dataset.
  (d)~depicts the DeepLab CNN result, (e)~CNN + 10 steps of mean-field inference,
  (f~result obtained with bilateral inception (BI) modules (\bi{6}{2}+\bi{7}{6}) between \fc~layers.}
  \label{fig:semantic_visuals-app}
\end{figure*}


\definecolor{minc_1}{HTML}{771111}
\definecolor{minc_2}{HTML}{CAC690}
\definecolor{minc_3}{HTML}{EEEEEE}
\definecolor{minc_4}{HTML}{7C8FA6}
\definecolor{minc_5}{HTML}{597D31}
\definecolor{minc_6}{HTML}{104410}
\definecolor{minc_7}{HTML}{BB819C}
\definecolor{minc_8}{HTML}{D0CE48}
\definecolor{minc_9}{HTML}{622745}
\definecolor{minc_10}{HTML}{666666}
\definecolor{minc_11}{HTML}{D54A31}
\definecolor{minc_12}{HTML}{101044}
\definecolor{minc_13}{HTML}{444126}
\definecolor{minc_14}{HTML}{75D646}
\definecolor{minc_15}{HTML}{DD4348}
\definecolor{minc_16}{HTML}{5C8577}
\definecolor{minc_17}{HTML}{C78472}
\definecolor{minc_18}{HTML}{75D6D0}
\definecolor{minc_19}{HTML}{5B4586}
\definecolor{minc_20}{HTML}{C04393}
\definecolor{minc_21}{HTML}{D69948}
\definecolor{minc_22}{HTML}{7370D8}
\definecolor{minc_23}{HTML}{7A3622}
\definecolor{minc_24}{HTML}{000000}

\begin{figure*}[!ht]
  \small % scriptsize
  \centering
  \fcolorbox{white}{minc_1}{\rule{0pt}{4pt}\rule{4pt}{0pt}} Brick~~
  \fcolorbox{white}{minc_2}{\rule{0pt}{4pt}\rule{4pt}{0pt}} Carpet~~
  \fcolorbox{white}{minc_3}{\rule{0pt}{4pt}\rule{4pt}{0pt}} Ceramic~~
  \fcolorbox{white}{minc_4}{\rule{0pt}{4pt}\rule{4pt}{0pt}} Fabric~~
  \fcolorbox{white}{minc_5}{\rule{0pt}{4pt}\rule{4pt}{0pt}} Foliage~~
  \fcolorbox{white}{minc_6}{\rule{0pt}{4pt}\rule{4pt}{0pt}} Food~~
  \fcolorbox{white}{minc_7}{\rule{0pt}{4pt}\rule{4pt}{0pt}} Glass~~
  \fcolorbox{white}{minc_8}{\rule{0pt}{4pt}\rule{4pt}{0pt}} Hair~~\\
  \fcolorbox{white}{minc_9}{\rule{0pt}{4pt}\rule{4pt}{0pt}} Leather~~
  \fcolorbox{white}{minc_10}{\rule{0pt}{4pt}\rule{4pt}{0pt}} Metal~~
  \fcolorbox{white}{minc_11}{\rule{0pt}{4pt}\rule{4pt}{0pt}} Mirror~~
  \fcolorbox{white}{minc_12}{\rule{0pt}{4pt}\rule{4pt}{0pt}} Other~~
  \fcolorbox{white}{minc_13}{\rule{0pt}{4pt}\rule{4pt}{0pt}} Painted~~
  \fcolorbox{white}{minc_14}{\rule{0pt}{4pt}\rule{4pt}{0pt}} Paper~~
  \fcolorbox{white}{minc_15}{\rule{0pt}{4pt}\rule{4pt}{0pt}} Plastic~~\\
  \fcolorbox{white}{minc_16}{\rule{0pt}{4pt}\rule{4pt}{0pt}} Polished Stone~~
  \fcolorbox{white}{minc_17}{\rule{0pt}{4pt}\rule{4pt}{0pt}} Skin~~
  \fcolorbox{white}{minc_18}{\rule{0pt}{4pt}\rule{4pt}{0pt}} Sky~~
  \fcolorbox{white}{minc_19}{\rule{0pt}{4pt}\rule{4pt}{0pt}} Stone~~
  \fcolorbox{white}{minc_20}{\rule{0pt}{4pt}\rule{4pt}{0pt}} Tile~~
  \fcolorbox{white}{minc_21}{\rule{0pt}{4pt}\rule{4pt}{0pt}} Wallpaper~~
  \fcolorbox{white}{minc_22}{\rule{0pt}{4pt}\rule{4pt}{0pt}} Water~~
  \fcolorbox{white}{minc_23}{\rule{0pt}{4pt}\rule{4pt}{0pt}} Wood~~\\
  \subfigure{%
    \includegraphics[width=.15\columnwidth]{figures/supplementary/000008468_given.png}
  }
  \subfigure{%
    \includegraphics[width=.15\columnwidth]{figures/supplementary/000008468_sp.png}
  }
  \subfigure{%
    \includegraphics[width=.15\columnwidth]{figures/supplementary/000008468_gt.png}
  }
  \subfigure{%
    \includegraphics[width=.15\columnwidth]{figures/supplementary/000008468_cnn.png}
  }
  \subfigure{%
    \includegraphics[width=.15\columnwidth]{figures/supplementary/000008468_crf.png}
  }
  \subfigure{%
    \includegraphics[width=.15\columnwidth]{figures/supplementary/000008468_ours.png}
  }\\[-2ex]

  \subfigure{%
    \includegraphics[width=.15\columnwidth]{figures/supplementary/000009053_given.png}
  }
  \subfigure{%
    \includegraphics[width=.15\columnwidth]{figures/supplementary/000009053_sp.png}
  }
  \subfigure{%
    \includegraphics[width=.15\columnwidth]{figures/supplementary/000009053_gt.png}
  }
  \subfigure{%
    \includegraphics[width=.15\columnwidth]{figures/supplementary/000009053_cnn.png}
  }
  \subfigure{%
    \includegraphics[width=.15\columnwidth]{figures/supplementary/000009053_crf.png}
  }
  \subfigure{%
    \includegraphics[width=.15\columnwidth]{figures/supplementary/000009053_ours.png}
  }\\[-2ex]




  \subfigure{%
    \includegraphics[width=.15\columnwidth]{figures/supplementary/000014977_given.png}
  }
  \subfigure{%
    \includegraphics[width=.15\columnwidth]{figures/supplementary/000014977_sp.png}
  }
  \subfigure{%
    \includegraphics[width=.15\columnwidth]{figures/supplementary/000014977_gt.png}
  }
  \subfigure{%
    \includegraphics[width=.15\columnwidth]{figures/supplementary/000014977_cnn.png}
  }
  \subfigure{%
    \includegraphics[width=.15\columnwidth]{figures/supplementary/000014977_crf.png}
  }
  \subfigure{%
    \includegraphics[width=.15\columnwidth]{figures/supplementary/000014977_ours.png}
  }\\[-2ex]


  \subfigure{%
    \includegraphics[width=.15\columnwidth]{figures/supplementary/000022922_given.png}
  }
  \subfigure{%
    \includegraphics[width=.15\columnwidth]{figures/supplementary/000022922_sp.png}
  }
  \subfigure{%
    \includegraphics[width=.15\columnwidth]{figures/supplementary/000022922_gt.png}
  }
  \subfigure{%
    \includegraphics[width=.15\columnwidth]{figures/supplementary/000022922_cnn.png}
  }
  \subfigure{%
    \includegraphics[width=.15\columnwidth]{figures/supplementary/000022922_crf.png}
  }
  \subfigure{%
    \includegraphics[width=.15\columnwidth]{figures/supplementary/000022922_ours.png}
  }\\[-2ex]


  \subfigure{%
    \includegraphics[width=.15\columnwidth]{figures/supplementary/000025711_given.png}
  }
  \subfigure{%
    \includegraphics[width=.15\columnwidth]{figures/supplementary/000025711_sp.png}
  }
  \subfigure{%
    \includegraphics[width=.15\columnwidth]{figures/supplementary/000025711_gt.png}
  }
  \subfigure{%
    \includegraphics[width=.15\columnwidth]{figures/supplementary/000025711_cnn.png}
  }
  \subfigure{%
    \includegraphics[width=.15\columnwidth]{figures/supplementary/000025711_crf.png}
  }
  \subfigure{%
    \includegraphics[width=.15\columnwidth]{figures/supplementary/000025711_ours.png}
  }\\[-2ex]


  \subfigure{%
    \includegraphics[width=.15\columnwidth]{figures/supplementary/000034473_given.png}
  }
  \subfigure{%
    \includegraphics[width=.15\columnwidth]{figures/supplementary/000034473_sp.png}
  }
  \subfigure{%
    \includegraphics[width=.15\columnwidth]{figures/supplementary/000034473_gt.png}
  }
  \subfigure{%
    \includegraphics[width=.15\columnwidth]{figures/supplementary/000034473_cnn.png}
  }
  \subfigure{%
    \includegraphics[width=.15\columnwidth]{figures/supplementary/000034473_crf.png}
  }
  \subfigure{%
    \includegraphics[width=.15\columnwidth]{figures/supplementary/000034473_ours.png}
  }\\[-2ex]


  \subfigure{%
    \includegraphics[width=.15\columnwidth]{figures/supplementary/000035463_given.png}
  }
  \subfigure{%
    \includegraphics[width=.15\columnwidth]{figures/supplementary/000035463_sp.png}
  }
  \subfigure{%
    \includegraphics[width=.15\columnwidth]{figures/supplementary/000035463_gt.png}
  }
  \subfigure{%
    \includegraphics[width=.15\columnwidth]{figures/supplementary/000035463_cnn.png}
  }
  \subfigure{%
    \includegraphics[width=.15\columnwidth]{figures/supplementary/000035463_crf.png}
  }
  \subfigure{%
    \includegraphics[width=.15\columnwidth]{figures/supplementary/000035463_ours.png}
  }\\[-2ex]


  \setcounter{subfigure}{0}
  \subfigure[\scriptsize Input]{%
    \includegraphics[width=.15\columnwidth]{figures/supplementary/000035993_given.png}
  }
  \subfigure[\scriptsize Superpixels]{%
    \includegraphics[width=.15\columnwidth]{figures/supplementary/000035993_sp.png}
  }
  \subfigure[\scriptsize GT]{%
    \includegraphics[width=.15\columnwidth]{figures/supplementary/000035993_gt.png}
  }
  \subfigure[\scriptsize AlexNet]{%
    \includegraphics[width=.15\columnwidth]{figures/supplementary/000035993_cnn.png}
  }
  \subfigure[\scriptsize +DenseCRF]{%
    \includegraphics[width=.15\columnwidth]{figures/supplementary/000035993_crf.png}
  }
  \subfigure[\scriptsize Using BI]{%
    \includegraphics[width=.15\columnwidth]{figures/supplementary/000035993_ours.png}
  }
  \mycaption{Material Segmentation}{Example results of material segmentation.
  (d)~depicts the AlexNet CNN result, (e)~CNN + 10 steps of mean-field inference,
  (f)~result obtained with bilateral inception (BI) modules (\bi{7}{2}+\bi{8}{6}) between
  \fc~layers.}
\label{fig:material_visuals-app}
\end{figure*}


\definecolor{city_1}{RGB}{128, 64, 128}
\definecolor{city_2}{RGB}{244, 35, 232}
\definecolor{city_3}{RGB}{70, 70, 70}
\definecolor{city_4}{RGB}{102, 102, 156}
\definecolor{city_5}{RGB}{190, 153, 153}
\definecolor{city_6}{RGB}{153, 153, 153}
\definecolor{city_7}{RGB}{250, 170, 30}
\definecolor{city_8}{RGB}{220, 220, 0}
\definecolor{city_9}{RGB}{107, 142, 35}
\definecolor{city_10}{RGB}{152, 251, 152}
\definecolor{city_11}{RGB}{70, 130, 180}
\definecolor{city_12}{RGB}{220, 20, 60}
\definecolor{city_13}{RGB}{255, 0, 0}
\definecolor{city_14}{RGB}{0, 0, 142}
\definecolor{city_15}{RGB}{0, 0, 70}
\definecolor{city_16}{RGB}{0, 60, 100}
\definecolor{city_17}{RGB}{0, 80, 100}
\definecolor{city_18}{RGB}{0, 0, 230}
\definecolor{city_19}{RGB}{119, 11, 32}
\begin{figure*}[!ht]
  \small % scriptsize
  \centering


  \subfigure{%
    \includegraphics[width=.18\columnwidth]{figures/supplementary/frankfurt00000_016005_given.png}
  }
  \subfigure{%
    \includegraphics[width=.18\columnwidth]{figures/supplementary/frankfurt00000_016005_sp.png}
  }
  \subfigure{%
    \includegraphics[width=.18\columnwidth]{figures/supplementary/frankfurt00000_016005_gt.png}
  }
  \subfigure{%
    \includegraphics[width=.18\columnwidth]{figures/supplementary/frankfurt00000_016005_cnn.png}
  }
  \subfigure{%
    \includegraphics[width=.18\columnwidth]{figures/supplementary/frankfurt00000_016005_ours.png}
  }\\[-2ex]

  \subfigure{%
    \includegraphics[width=.18\columnwidth]{figures/supplementary/frankfurt00000_004617_given.png}
  }
  \subfigure{%
    \includegraphics[width=.18\columnwidth]{figures/supplementary/frankfurt00000_004617_sp.png}
  }
  \subfigure{%
    \includegraphics[width=.18\columnwidth]{figures/supplementary/frankfurt00000_004617_gt.png}
  }
  \subfigure{%
    \includegraphics[width=.18\columnwidth]{figures/supplementary/frankfurt00000_004617_cnn.png}
  }
  \subfigure{%
    \includegraphics[width=.18\columnwidth]{figures/supplementary/frankfurt00000_004617_ours.png}
  }\\[-2ex]

  \subfigure{%
    \includegraphics[width=.18\columnwidth]{figures/supplementary/frankfurt00000_020880_given.png}
  }
  \subfigure{%
    \includegraphics[width=.18\columnwidth]{figures/supplementary/frankfurt00000_020880_sp.png}
  }
  \subfigure{%
    \includegraphics[width=.18\columnwidth]{figures/supplementary/frankfurt00000_020880_gt.png}
  }
  \subfigure{%
    \includegraphics[width=.18\columnwidth]{figures/supplementary/frankfurt00000_020880_cnn.png}
  }
  \subfigure{%
    \includegraphics[width=.18\columnwidth]{figures/supplementary/frankfurt00000_020880_ours.png}
  }\\[-2ex]



  \subfigure{%
    \includegraphics[width=.18\columnwidth]{figures/supplementary/frankfurt00001_007285_given.png}
  }
  \subfigure{%
    \includegraphics[width=.18\columnwidth]{figures/supplementary/frankfurt00001_007285_sp.png}
  }
  \subfigure{%
    \includegraphics[width=.18\columnwidth]{figures/supplementary/frankfurt00001_007285_gt.png}
  }
  \subfigure{%
    \includegraphics[width=.18\columnwidth]{figures/supplementary/frankfurt00001_007285_cnn.png}
  }
  \subfigure{%
    \includegraphics[width=.18\columnwidth]{figures/supplementary/frankfurt00001_007285_ours.png}
  }\\[-2ex]


  \subfigure{%
    \includegraphics[width=.18\columnwidth]{figures/supplementary/frankfurt00001_059789_given.png}
  }
  \subfigure{%
    \includegraphics[width=.18\columnwidth]{figures/supplementary/frankfurt00001_059789_sp.png}
  }
  \subfigure{%
    \includegraphics[width=.18\columnwidth]{figures/supplementary/frankfurt00001_059789_gt.png}
  }
  \subfigure{%
    \includegraphics[width=.18\columnwidth]{figures/supplementary/frankfurt00001_059789_cnn.png}
  }
  \subfigure{%
    \includegraphics[width=.18\columnwidth]{figures/supplementary/frankfurt00001_059789_ours.png}
  }\\[-2ex]


  \subfigure{%
    \includegraphics[width=.18\columnwidth]{figures/supplementary/frankfurt00001_068208_given.png}
  }
  \subfigure{%
    \includegraphics[width=.18\columnwidth]{figures/supplementary/frankfurt00001_068208_sp.png}
  }
  \subfigure{%
    \includegraphics[width=.18\columnwidth]{figures/supplementary/frankfurt00001_068208_gt.png}
  }
  \subfigure{%
    \includegraphics[width=.18\columnwidth]{figures/supplementary/frankfurt00001_068208_cnn.png}
  }
  \subfigure{%
    \includegraphics[width=.18\columnwidth]{figures/supplementary/frankfurt00001_068208_ours.png}
  }\\[-2ex]

  \subfigure{%
    \includegraphics[width=.18\columnwidth]{figures/supplementary/frankfurt00001_082466_given.png}
  }
  \subfigure{%
    \includegraphics[width=.18\columnwidth]{figures/supplementary/frankfurt00001_082466_sp.png}
  }
  \subfigure{%
    \includegraphics[width=.18\columnwidth]{figures/supplementary/frankfurt00001_082466_gt.png}
  }
  \subfigure{%
    \includegraphics[width=.18\columnwidth]{figures/supplementary/frankfurt00001_082466_cnn.png}
  }
  \subfigure{%
    \includegraphics[width=.18\columnwidth]{figures/supplementary/frankfurt00001_082466_ours.png}
  }\\[-2ex]

  \subfigure{%
    \includegraphics[width=.18\columnwidth]{figures/supplementary/lindau00033_000019_given.png}
  }
  \subfigure{%
    \includegraphics[width=.18\columnwidth]{figures/supplementary/lindau00033_000019_sp.png}
  }
  \subfigure{%
    \includegraphics[width=.18\columnwidth]{figures/supplementary/lindau00033_000019_gt.png}
  }
  \subfigure{%
    \includegraphics[width=.18\columnwidth]{figures/supplementary/lindau00033_000019_cnn.png}
  }
  \subfigure{%
    \includegraphics[width=.18\columnwidth]{figures/supplementary/lindau00033_000019_ours.png}
  }\\[-2ex]

  \subfigure{%
    \includegraphics[width=.18\columnwidth]{figures/supplementary/lindau00052_000019_given.png}
  }
  \subfigure{%
    \includegraphics[width=.18\columnwidth]{figures/supplementary/lindau00052_000019_sp.png}
  }
  \subfigure{%
    \includegraphics[width=.18\columnwidth]{figures/supplementary/lindau00052_000019_gt.png}
  }
  \subfigure{%
    \includegraphics[width=.18\columnwidth]{figures/supplementary/lindau00052_000019_cnn.png}
  }
  \subfigure{%
    \includegraphics[width=.18\columnwidth]{figures/supplementary/lindau00052_000019_ours.png}
  }\\[-2ex]




  \subfigure{%
    \includegraphics[width=.18\columnwidth]{figures/supplementary/lindau00027_000019_given.png}
  }
  \subfigure{%
    \includegraphics[width=.18\columnwidth]{figures/supplementary/lindau00027_000019_sp.png}
  }
  \subfigure{%
    \includegraphics[width=.18\columnwidth]{figures/supplementary/lindau00027_000019_gt.png}
  }
  \subfigure{%
    \includegraphics[width=.18\columnwidth]{figures/supplementary/lindau00027_000019_cnn.png}
  }
  \subfigure{%
    \includegraphics[width=.18\columnwidth]{figures/supplementary/lindau00027_000019_ours.png}
  }\\[-2ex]



  \setcounter{subfigure}{0}
  \subfigure[\scriptsize Input]{%
    \includegraphics[width=.18\columnwidth]{figures/supplementary/lindau00029_000019_given.png}
  }
  \subfigure[\scriptsize Superpixels]{%
    \includegraphics[width=.18\columnwidth]{figures/supplementary/lindau00029_000019_sp.png}
  }
  \subfigure[\scriptsize GT]{%
    \includegraphics[width=.18\columnwidth]{figures/supplementary/lindau00029_000019_gt.png}
  }
  \subfigure[\scriptsize Deeplab]{%
    \includegraphics[width=.18\columnwidth]{figures/supplementary/lindau00029_000019_cnn.png}
  }
  \subfigure[\scriptsize Using BI]{%
    \includegraphics[width=.18\columnwidth]{figures/supplementary/lindau00029_000019_ours.png}
  }%\\[-2ex]

  \mycaption{Street Scene Segmentation}{Example results of street scene segmentation.
  (d)~depicts the DeepLab results, (e)~result obtained by adding bilateral inception (BI) modules (\bi{6}{2}+\bi{7}{6}) between \fc~layers.}
\label{fig:street_visuals-app}
\end{figure*}

  
  

\end{document}
