 

\begin{figure}[t]
(a){\includegraphics[width=0.8\textwidth]{schematic.pdf}}\hfil\\
(b)\includegraphics[width=0.8\textwidth, trim = 1cm 4cm 1cm 7cm, clip]{magneticforce.pdf}\hfil\\
\caption{(a)  (Figure 1 of \cite{YLAR2012})  The target is  at $(0,0,-R)$, 
in a cylindrical blood vessel of radius $R$.    A  magnetic point dipole  is defined by the dipole moment 
vector $\vec m$, directed from the magnet placed  below the target, and at an angle 
$\theta$ to the positive x-axis;  i.e.,  $\theta=0$ is upward.   The cross-sectional slide  
is from \cite{arterial}. 
(b) (Figure 6 of \cite{YLAR2012}) Schematic of the vertical slice through the cylindrical vessel at $y=0$. The magnetic force field
is shown when the dipole is at a
distance $d$ below the vessel wall. }
\label{schematic}
\end{figure}


In this section, we summarize the set-up and results of \cite{YLAR2012}.
One of the simplest  model problems for predicting the 
accuracy of drug targeting concerns superparamagnetic nanoparticles of at 
most several nanometers diameter, and flowing through a  venule of order 10$^{-4}$~m radius; see 
 figures \ref{schematic} (a - b). The  velocity is  $(u(y,z,t),0,0)$ where $u$ denotes the axial component.  The no-slip conditions at the
walls $y^2+z^2=R^2$ and the Stokes equation are applied.   For such a small radius, the effect of cardiac pumping on pressure 
fluctuations is negligible; therefore, $u=u_{max}(1-\frac{r^2}{R^2})$, where the flow rate $\frac{\pi}{2}u_{max}R^2$ is typically $10^{-10}$m$^3$s$^{-1}$ \cite{House86}. 
This 
gives $u_{max}\approx 0.006$m s$^{-1}$.  This 
flow generates a Stokes drag force on a solid sphere: ${\bf F}_v=-(6\pi\eta a)(\frac{d{\bf x}}{dt}-(u,0,0))$ where $\eta$ is the viscosity, $a$ the radius, ${\bf x}$ is the location of  the particle, and  $D$ is the friction coefficient. For a larger blood vessel, the oscillatory component of the flow can be estimated from the systolic and diastolic blood pressure data. 

 The external magnetic field ${\bf H}_e$  without a particle is known from experimental measurements and is used to calculate the field in the presence of the particle.  Under the  assumptions of  
   constant susceptibility $\chi$ (magnetic field not too high),  the  permeability is $\mu=\mu_0(1+\chi)$, where $\mu_0$ denotes the permeability of vacuum. Maxwell's 
equations (curl ${\bf H}_e={\bf 0}$, div ${\bf H}_e=0$) give  ${\bf H}_e=\nabla\phi$, where $\phi$ is a harmonic function. 
The external magnet is  modeled by a magnetic dipole. We find the actual field ${\bf H}$ with the particle by using the known formula $\phi=\frac{-1}{4\pi}
\frac{ {\bf m}\cdot  {\bf r}}{r^3}$, where the vector ${\bf r}$ points from the dipole to the 
particle.  Provided that the external magnet is small,  it is approximately a point dipole.
This magnetic field exerts a force on the  particle:  ${\bf F}_m=\int {  \mu_0 ({\bf M}\cdot\nabla){\bf H}_e  dV}$ where ${\bf M}=\chi{\bf H}$ denotes the magnetization corresponding to 
the magnetic field ${\bf H}$ with a particle of volume $V=\frac{4}{3}\pi a^3$, namely the Clausius-Mossotti formula 
\cite{Rosensweig,Cohen2007} \ 
 ${\bf M}=\frac{3\chi}{3+\chi}{\bf H}_e$. This is used to calculate ${\bf F}_m$. The governing equation  must satisfy ${\bf F}_m+{\bf F}_v={\bf 0}$. This is a system of nonlinear ordinary differential equations.


  The superparamagnetic particles and clusters addressed in \cite{YLAR2012} are small enough that the N\'eel relaxation time and the  magnetization are  instantaneous compared with the motion to be solved\cite{Bala2014}. However, we need  to keep in mind the that Brownian motion  should be included  if thermal energy $kT$ becomes  comparable to magnetic energy $\mu_0 MHV$, where $k$ denotes Boltzmann's constant, $T$ is in degrees Kelvin, $H$ is the field and $V$ is the volume of the particle.  This estimate gives the diameter to be smaller than  $(6kT/\pi\mu_0MH)^{1/3}$  ((2.2) of  \cite{Rosensweig}). In the examples  below, this is diameter of order $10^{-8}$m. 
 We incorporate Langevin's model for  Brownian motion 
through a stochastic forcing vector.  This leads to the system
\begin{equation}
\rho V\frac{d^2 {\bf x}}{dt^2} = -D (\frac{d{\bf  x}}{dt} - {\bf u_b}) + \frac{3 \chi V \mu_0  }
{ 3 + \chi} \nabla \left(\frac{1}{2}|\nabla\phi|^2\right) + \frac{\sqrt{2DkT}}{\sqrt{{\bf d}t}}{\bf N}(0,1), \label{brownian}
\end{equation}
where $\rho$ is the density of the particle/cluster, and ${\bf u_b}$ is
the base flow.  The left hand side term denotes acceleration which is estimated to be relatively small. $N_1(0,1)$, $N_2(0,1)$ and $N_3(0,1)$ denote independently generated, normally distributed,
random variables with zero mean and unit variance . The random variables $N_i(0,1)$, $i=1,2,3,$ are  constants over
very short time intervals $dt$, and change randomly with a Gaussian distribution. Finally, the stochastic ordinary differential equation for the state vector
$X (t)= {\vec x}^T =(x(t),y(t),z(t))^T$ is,
\begin{eqnarray}
dX && =f(X) dt + g dW, \\ 
f(X) && = \left[{\bf u_b}({\bf  x},t)) +  \frac{3 \chi V \mu_0  }{ (3 + \chi) D} \nabla \left( \frac{1}{2} |\nabla\phi(\vec x)|^2 
\right) \right]^T, \\
g && =  \sqrt{\frac{2kT}{D}} \mathbf{I};\label{system2}
\end{eqnarray}
 $\mathbf{I}$ is the $3\times 3$ identity matrix,
$f(X(t))$ is the $3\times 1$ drift-rate vector, $g$ is a  $3\times 3$ instantaneous diffusion-rate
matrix, and $dW(t)$ denotes the vector  $\sqrt{{\bf d}t} {\bf  N}(0,1)$.  

\subsection{Numerical algorithm}
The system (\ref{system2})  is integrated using the Euler-Maruyama method. At the $n$th time step, where $\Delta t$ denotes a fixed step size, $X_{n+1}=X_n+\Delta t\ f(X_n)+g(X_n)\Delta W_n$.  As detailed in \cite{Higham2001},  the advantage of this method is the strong convergence under broad assumptions. We use the implementation 
in the SDE Toolbox for Matlab  \cite{Picchini}. 
The capture rate for a particle is defined to be the probability of hitting the tumor, which is situated  at $|x|<R_{tumor}$ at the wall in figure \ref{schematic}. 
Figure \ref{fig:3dtraj}(a) shows sample trajectories, initially at  $x=-R_{tumor}$, where $R_{tumor}$ denotes the radius of the tumor at the wall. A trajectory can escape without hitting the wall ($-.-$), or  escape after hitting the wall ($-$),  or be  captured at the target.  Figures   \ref{fig:3dtraj}(b-d) use data from \cite{ARRRP,House86}.  The magnet is $d=0.03$ m outside the vessel wall;  vessel radius $R=10^{-4}$~m, $u_{max}=\frac{2}{\pi} 0.01$~ms$^{-1}$, particle radius $a=10^{-7}$~m, $\eta= 0.004$Pa~s,  $R_{tumor}=100R$, constant susceptibility $\chi=0.2$.  
(b) shows the capture rate versus dipole moment $m$. For larger $m$, 
   a non-constant Langevin fit approximates the magnetization data  \cite{ARRRP}. The capture rate scales with the square of the dipole moment even when  the velocity profile is slightly blunted at the centerline which may relate to blood flow.  To optimize the capture rate, it is found that the distance $d$ should match the tumor size.   (c)  shows trajectories for $a=10^{-7}$~m, and compared with this, (d) shows that  Brownian motion makes a significant difference for    $a=O(10^{-8})$~m. 

\begin{figure}[t]
\includegraphics[width=0.45\textwidth,angle=-90,origin=c]{traj3d.pdf}\hglue -0.3truein
\includegraphics[width=0.45\textwidth]{captd0p03.pdf}\hfil\\
\hfil (a)\hfil (b)\hfil\\
(c)\includegraphics[width=0.45\textwidth]{brwnmef_yz.pdf}
(d)\includegraphics[width=0.45\textwidth]{brwnmefam8_yz.pdf}\hfil\\
\caption{Reproduced with permission from \cite{YLAR2012}. (a) Typical  trajectories with sliding motion on the wall. The scale is  compressed in the $x$-direction.   $R=10^{-4}$ m, $u_{max}=\frac{2}{\pi}\times 10^{-2}$ m s$^{-1}$,  $a = 10^{-7}$ m, $\eta=0.004$ Pa s, 
$\chi=0.2$, $m=1$A~m$^2$, and $d=50 R$. The clusters are released from $x=-100R$, the target is at $x=0$. 
(b) Capture rate {\it vs.} dipole moment $m$ with $d=0.03$~m, $\chi=0.2$.
(c) Trajectories in the $y-z$ plane with $x=0$ (target). $m=1056$~A~m$^2$, $\chi=0.2$, $\Delta t=2\times 
 10^{-4}$~s. 100 trajectories.
(d)  With smaller particles,  $a=10^{-8}$~m. }\label{fig:3dtraj}
\end{figure}






The capture rate is optimal for a magnet that produces a strong field and a  focused effect, and therefore high field gradients. This is one of the technological challenges in magnetic drug targeting. Recent studies show that arrays of magnets, such as the Halbach array,  are an improvement over the single magnet. The balance of magnetic force and hydrodynamic drag for Stokes flow is calculated for high field gradient arrays and compared with experimental data in \cite{Barnsley2017}. This gives guidance on  how the particle trajectories depend on magnet shapes and arrays. 

Non-Newtonian effects in blood flow have been studied with  empirical formulas that replace the Stokes drag coefficient \cite{Sprenger2015}. For example,  in the case of  the larger arterial speeds,  an empirical viscosity term which incorporates shear-thinning at the local shear rate   is applied to  write approximate equations for the motion of  micron sized magnetic particles in Poiseuille flow \cite{Cherry2014,Mohren2017}.  The particles in \cite{Cherry2014} are of micron size, much larger than the nanometers considered in \cite{YLAR2012}, and also the diameter of the artery and flow speeds are much larger,  and therefore the Brownian effect for the particles is small.   The use of the local shear rate of the background  flow in the Stokes drag formula is an ad-hoc substitution.   For a particle moving in non-Newtonian flow, such as in a shear-thinning fluid,  the calculation of drag is tractable for quiescent flow,  not for non-quiescent flow such as shear flow. Unlike Stokes flow, the equations are nonlinear so that  solutions can not be superposed.  The real velocity of the particle is spatially and temporally varying in response to the magnetic field and flow.  
 A separate effect  is  the influence of  red blood cells on the capture rate.  \cite{Mohren2017} applies a  formula for shear-enhanced diffusion in colloidal suspensions (equation (1) of \cite{Griffiths2012})  to estimate that the red blood cells contribute an
   additive term proportional to the shear rate inside the square root in equation \ref{system2}.  Particle trajectories for the Cherry viscosity model and  including the diffusion due to interaction with red blood cells are investigated in \cite{Mohren2017}. They find that the additional diffusion deteriorates the capture rate; figure \ref{fig6} shows this. 
   

\begin{figure}[t]
\begin{center}
\includegraphics[width=0.45\textwidth]{zvx_yes_RBC.pdf}\hfil
\includegraphics[width=0.45\textwidth]{zvx_no_RBC.pdf}\\
\hfil (a) Z vs X with RBC \hfil (b) Z vs X without RBC\hfil \\
\end{center}
\caption{ (with permission from \cite{Mohren2017}, Figues 5(a-b)) 100 particle trajectories with point dipole magnetic field model and
Cherry viscosity model  with ${\bf m} =
[0, 0, 700Am^2]$. Without red blood cell collisions, the capture rate was 100\%. With red blood cell collisions, the capture rate was 91\%.}
\label{fig6}
\end{figure}

\FloatBarrier

