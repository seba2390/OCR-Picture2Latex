Ferrofluids are typically suspensions of magnetite nanoparticles, and behave as a homogeneous continuum.  The production of nanoparticles with a  narrow size distribution and the achievement of colloidal stability are important technological issues.  The ability of the ferrofluid to respond to an external magnetic field in a controllable manner has made it emerge as a smart material in a variety of applications, such as
seals, lubricants, electronics cooling, shock absorbers and adaptive optics. Magnetic nanoparticle suspensions have also gained 
attraction recently in a  range of  biomedical applications, such as cell separation, hyperthermia, MRI, drug targeting and cancer diagnosis.
 In this review, we provide an introduction  to mathematical modeling of three problems:  motion of  superparamagnetic nanoparticles in magnetic drug targeting, the motion of a ferrofluid drop consisting of chemically bound nanoparticles without a carrier fluid, and the breakage of a thin film of a ferrofluid. 
 
 