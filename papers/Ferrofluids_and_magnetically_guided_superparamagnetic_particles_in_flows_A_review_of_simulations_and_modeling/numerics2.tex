\subsection{One-fluid formulation and Volume-of-Fluid method}
\label{sec:vof}
We take advantage of the axial symmetry present in the 
motion and deformation of a ferrofluid droplet placed  in 
a viscous medium under an externally applied magnetic field 
and resort to the formulation in axisymmetric cylindrical coordinates $(r,z)$.
The  VOF method represents 
each liquid  with a color function as
\begin{eqnarray}
C(r,z,t) & = & \left\{
\begin{array}{ll}
0 & \mbox{in the surrounding medium}\cr
1 & \mbox{in the ferrofluid drop,}\cr
\end{array}
\right.
\end{eqnarray}
which is advected by the flow.  The position of the interface is 
reconstructed from 
$C(r,z,t)$.  The one-fluid formulation of the governing equations
(\ref{eq:motion}) and (\ref{eq:mass}) then reads
\begin{eqnarray}  
\frac{\partial C}{\partial t}+{\mathbf u}\cdot\nabla C&=&0, \label{eq:vof}\\
\rho \frac{\partial{\mathbf u}}{\partial t}+{\mathbf u}\cdot\nabla {\mathbf u}&=&-\nabla p + 
\nabla\cdot \bigl(2\eta {\mathbf D} \bigr) + \nabla\cdot{\mathbf \Pi}_M + {\mathbf F}_s + \rho {\mathbf g},
\label{eq:motion-vof}
\end{eqnarray}    
where
${\mathbf F}_s$ denotes the continuum body force due to interfacial tension \cite{Brackbill92}
\begin{equation}
{\mathbf F}_s=\sigma k  \delta_s {\bf n}, \label{forces}
\end{equation}
${\bf n} =\nabla C/|\nabla C|$, $\delta_s=|\nabla C|$ is the delta-function at the interface,   
$\rho = \rho(C)$, $\eta = \eta(C)$, and $\mu = \mu (C)$.
The magnetic potential $\psi$ in axisymmetric cylindrical coordinates satisfies equation
(\ref{eq:laplace}),
\begin{eqnarray}
\frac{1}{r}\frac{\partial}{\partial r}\left( \mu r\frac{\partial\psi}{\partial 
r}\right)+\frac 
{\partial}{\partial z} \left(\mu\frac{\partial \psi}{\partial z} \right)=0. 
\label{eq:phi}
\end{eqnarray}

The dimensionless variables, denoted by tildes, are defined as follows
\begin{align*}
  \tilde{\textbf{x}} = \frac{\textbf{x}}{L_c} 		&& \tilde{t} = \frac{t}{\tau}\\
  \tilde{\textbf{u}} = \frac{\tau}{L_c}\textbf{u}     &&\quad \tilde{\bf g} = \frac{{\bf g}}{g}\\
   \tilde{{\bf F}}_s = \frac{{\bf F}_s}{\sigma/L_c^2}            && \tilde{{\mathbf \Pi}}_M =  \frac{{\mathbf \Pi}_M}{H_o}\\ 
  \tilde{p} = \frac{p}{\sigma/L_c} && \tilde{\rho}=\frac{\rho}{\rho_d},\quad \tilde{\eta}=\frac{\eta}{\eta_d},\quad \tilde{\mu}=\frac{\mu}{\mu_0}
\end{align*}
where $L_c$, $\tau$, $H_o$ and $g$ are characteristic scales of length, time, magnetic field and gravitational acceleration, respectively. The ferrofluid density and viscosity are ${\rho_d}$ and ${\eta_d}$, respectively. For the choice of viscous time scale, 
$$
\tau = \frac{\eta_d L_c}{\sigma},
$$
corresponding to the viscous length scale
$$
L_{c} = \frac{\eta_d^2}{\rho_d\sigma},
$$
equation (\ref{eq:motion-vof}) becomes, dropping the tilde notation,
\begin{eqnarray}  
%\nabla \cdot {\bf u}&=&0\\
\mbox{Oh}^{-2}\left(\rho \frac{\partial{\mathbf u}}{\partial t}+{\mathbf u}\cdot\nabla {\mathbf u}\right)&=&-\nabla p + \nabla\cdot \bigl(2\eta {\bf D} \bigr) + \mbox{Bo}_{m}\nabla\cdot{\mathbf \Pi}_M + {\mathbf F}_s + \mbox{Bo} \rho{\mathbf g},
\label{eq:motion1}
\end{eqnarray}
where the Ohnesorge number $\mbox{Oh}=\eta_d/\sqrt{\rho_d \sigma L_c}$, the magnetic Bond number $\mbox{Bo}_{m}=\mu_0H^2_oL_c/\sigma$, and gravitational Bond number $\mbox{Bo}=\rho_d g L_c^2/\sigma$.


\subsection{Spatial discretization}\label{ap:discrete}
In axisymmetric coordinates $(r,z,\theta)$, the magnetic field
is given by
\begin{equation}
{\bf H}=(\frac{\partial\psi}{\partial r},\frac{\partial\psi}{\partial 
z},0),
\end{equation}
and the corresponding Maxwell stress is
\begin{equation}
{\mathbf \Pi}_M = \mu \left[\begin{array}{ccc}
\psi_r^2 - \frac{1}{2}\left(\psi_r^2 
+\psi_z^2\right)
&\psi_r\psi_z& 0 \\
\psi_r\psi_z& \psi_z^2 
-
\frac{1}{2}\left(\psi_r^2 +\psi_z^2\right) & 0 \\
0 & 0 & - \frac{1}{2}\left(\psi_r^2 + \psi_z^2\right) \end{array}
                \right]. \label{magnetic tensor}
\end{equation} 
We have
\begin{eqnarray}
\nabla\cdot{\mathbf \Pi}_M = \left[\frac{1}{r}\frac{\partial}{\partial r} [r  ({\Pi}_M)_{rr}]
\right. & + & \left. \frac{\partial}{\partial z} [ ({\Pi}_M)_{rz}] \right] \bf{e_r}  \\ \nonumber
& + &\left[\frac{1}{r}\frac{\partial}{\partial r} [r ({\Pi}_M)_{rz}] + \frac{\partial}{\partial z}
[ ({\Pi}_M)_{zz}] \right] \bf{e_z},
\end{eqnarray}
where $\bf{e_r}$ and $\bf{e_z}$ are unit vectors in $r$ and $z$ directions.
The magnetic potential field is discretized using second-order central differences.
In axisymmetric coordinates,
the discretization of  (\ref{eq:phi}) at cell $(i,j)$ yields, on a regular Cartesian grid of size $\Delta$, 
\begin{eqnarray}
\nabla\cdot(\mu\nabla\psi)_{i,j} &=&
\frac{1}{r_{i,j}}\frac{r_{i+1/2,j}\mu_{i+1/2,j}(\frac{\partial\psi}{\partial
r})_{i+1/2,j}
-r_{i-1/2,j}\mu_{i-1/2,j}(\frac{\partial\psi}{\partial r})_{i-1/2,j}}{\Delta}
\nonumber\\
&+&\frac{\mu_{i,j+1/2}(\frac{\partial\psi}{\partial
z})_{i,j+1/2}-\mu_{i,j-1/2}(\frac{\partial\psi}{\partial z})_{i,j-1/2}}{\Delta},
\end{eqnarray}
where, for instance for the cell face $(i+1/2,j)$,
\begin{eqnarray}
(\frac{\partial\psi}{\partial r})_{i+1/2,j} =
\frac{\psi_{i+1,j}-\psi_{i,j}}{\Delta}.
\end{eqnarray}

The spatial discretization of the velocity field is based on the
MAC grid in figure \ref{fig:figure22}.
\begin{figure}
\begin{center}
\scalebox{0.5}{\includegraphics[trim=20mm 40mm 0 20mm, clip=true]{figure22.pdf}}
\end{center}
 \caption{Location of the velocities and the magnetic
stress tensor components on a
MAC grid.} \label{fig:figure22}
\end{figure}
Therefore, the evaluation of the components of the magnetic stress tensor requires the evaluation
of gradients at faces. In axisymmetric coordinates,
the divergence of the magnetic stress tensor is discretized in the ${\bf e_r}$ and ${\bf e_z}$
directions as
\begin{eqnarray}
& & \frac{1}{r_{i+1/2,j}} \frac{r_{i+1,j}(({\Pi}_M)_{rr})_{i+1,j}
-r_{i,j}(({ \Pi}_M)_{rr})_{i,j}}{\Delta} + \nonumber\\
& &
\frac{(({\Pi}_M)_{rz})_{i+1/2,j+1/2}-(({ \Pi}_M)_{rz})_{i+1/2,j-1/2}}
{\Delta},\quad {\rm and} \nonumber\\
& & \frac{1}{r_{i,j+1/2}}
\frac{r_{i+1/2,j+1/2}(({ \Pi}_M)_{rz})_{i+1/2,j+1/2}
-r_{i-1/2,j+1/2}(({\Pi}_M)_{rz})_{i-1/2,j+1/2}}{\Delta} + \nonumber\\
& &
\frac{(({\Pi}_M)_{zz})_{i,j+1}-(({ \Pi}_M)_{zz})_{i,j}}{\Delta},
\end{eqnarray}
respectively, where, for example, components such as $(({\Pi}_M)_{rr})_{i,j}$ and
$(({ \Pi}_M)_{rz})_{i+1/2,j+1/2}$ are discretized as follows
\begin{eqnarray}
(({\Pi}_M)_{rr})_{i,j} &=& \mu_{i,j} \Bigl[\left(\frac{\psi_{i+1,j} - \psi_{i-1,j}}{2\Delta}\right)^2 \nonumber\\
&-& \frac{1}{2}\left(\left(\frac{\psi_{i+1,j} - \psi_{i-1,j}}{2\Delta}\right)^2 + \left(\frac{\psi_{i,j+1} -
\psi_{i,j-1}}{2\Delta}\right)^2\right) \Bigr], \nonumber
\end{eqnarray}
\begin{eqnarray}
(({\Pi}_M)_{rz})_{i+1/2,j+1/2} = \mu_{i+1/2,j+1/2}
\left(\frac{\psi_{i,j} - \psi_{i-1,j} + \psi_{i,j-1} - \psi_{i-1,j-1}}{2\Delta}\right)
\times \nonumber \\
\left(\frac{\psi_{i-1,j} - \psi_{i-1,j-1} + \psi_{i,j} - \psi_{i,j-1}}{2\Delta}\right),
\end{eqnarray}
respectively, where $\mu_{i+1/2,j+1/2}$ is computed using a simple averaging from cell
center values. $({\Pi}_M)_{rr}$ and $({ \Pi}_M)_{rz}$ at other grid locations
are discretized similarly. Analogous relationships can be written for the other components
of the magnetic stress tensor. 

\subsection{Drop deformation under a uniform magnetic field} 
When a ferrofluid drop is suspended in a non-magnetizable medium in an externally applied 
uniform magnetic field, it elongates in the direction of the applied magnetic field and assumes
a stable equilibrium configuration achieved via the competition  between the capillary and magnetic
forces.  Next, we show the numerical results of the VOF method for a ferrofluid drop suspended  in a non-magnetizable viscous medium by Afkhami et al.~\cite{ATRRWPR}. 
The initial configuration for the computational study of a ferrofluid drop  suspended 
in a non-magnetizable viscous medium is shown in figure~\ref{fig:figure4}.
A uniform magnetic field  ${\bf H}=(0,0,H_o)$, where $H_o$     
is the magnetic field intensity at infinity, is imposed at the top and  bottom boundaries
of the computational domain. In order to solve  Laplace's equation (\ref{eq:phi}) in the presence 
of an interface, the 
following boundary conditions are employed:
${\frac{\partial}{\partial z}\psi} = H_o$ at  $z=0, L_z$, and
${\frac{\partial}{\partial r}\psi} = 0$ at the side boundary $r=L_r$.
Note that a symmetry condition, ${\frac{\partial}{\partial z}\psi} = 0$, at $z = 0$,
and ${\frac{\partial}{\partial r}\psi} = 0$, at $r = 0$ can be applied. 
\begin{figure}
\begin{center}
\includegraphics[scale=0.4]{initial.pdf}
\caption{Schematic of the initial configuration. The computational domain is $0\leq z\leq 
L_z$,  $0\leq r \leq L_r$. Initially, a spherical superparamagnetic ferrofluid drop of 
radius $R$ is 
centered in the domain. The boundary conditions on the magnetic field are depicted at the 
boundaries. Reproduced from \cite{ATRRWPR}.} 
\label{fig:figure4}
\end{center}
\end{figure} 
The drop radius is $R_o = 1$~mm,  
interfacial tension is $\sigma = 1$ mN~m$^{-1}$, and 
$\mu_d$ is chosen to be constant. 

Figure \ref{fig:figure7}(a) shows the results in \cite{ATRRWPR} along with theoretical predictions.
The details of the theoretical analysis is given in \cite{ATRRWPR}. 
 From the computational results alone, it is observed that at sufficiently large 
 $\chi$, the drop equilibrium aspect ratio jumps to a higher value when the 
 magnetic Bond number reaches a critical value, while 
for small values of $\chi$, the drop equilibrium aspect ratio increases continuously as a function of the magnetic Bond number, in agreement with the theories. Specifically, at $\chi=20$, the jump in drop shape  occurs 
 when the magnetic Bond number changes from $\mbox{Bo}_{m} = 0.18$ to $0.19$.
they also find that for a fixed 
Figure \ref{fig:figure7}(b) shows the result for $b/a \approx 7$, $\mbox{Bo}_{m} = 0.2$, and $\chi = 20$
(see the line -.- in part a), where the shape exhibits 
conical ends. To provide further insight, 
the magnetic field lines inside the highly deformed drop as well as in the 
non-magnetizable surrounding medium, showing the effect of the magnetic field responsible
for the appearance of conical ends. 
\begin{figure}
\begin{center}
(a)\includegraphics[scale=0.4]{figure7.pdf}
(b)\includegraphics[height=59mm, trim = 25mm 1.5mm 72mm 2.5mm, clip=true]{figure10a.pdf}
\caption{From \cite{ATRRWPR}. (a) Comparison of the dependence of the drop aspect ratio $b/a$
(where $b$ is the semi-major axis and $a$ the
semi-minor axis) on the magnetic Bond number $\mbox{Bo}_{m}$.
Solid lines represent the theoretical analysis in \cite{ATRRWPR}. Numerical 
results are presented for $\chi = 2$ ({\large $\circ$}), $\chi = 5$ ($\blacktriangle$) and $\chi = 20$ ({\large $\bullet$}). 
(\small $\triangle$) denotes the prediction in \cite{Bacri}.
(b) The drop shape and contours of the magnetic field amplitude and magnetic field lines (right)
for the highly deformed drop suspended in a non-magnetizable media corresponding to $b/a\approx7$,
$\chi = 20$, and $\mbox{Bo}_{m} = 0.2$ data point in (a).}
\label{fig:figure7}
\end{center}
\end{figure}   

\subsubsection{Drop deformation under non-uniform magnetic fields} 
Here we present the motion of a hydrophobic ferrofluid droplet placed  in 
a viscous medium and driven by an externally applied non-uniform magnetic field is 
investigated numerically in an axisymmetric geometry. 
This numerical investigation is motivated by recent developments in the
synthesis and characterization of ferrofluids for possible use in
the treatment of 
retinal detachment \cite{Mefford}. 
Figure \ref{fig:retina} shows a cartoon of the application of 
a small ferrofluid drop injected into the vitreous cavity of the eye and guided by a permanent  magnet 
inserted outside the scleral wall of the eye.
The drop travels toward the side of  the eye, until it can seal a retinal hole. 
\begin{figure}
\begin{center}
\includegraphics[scale=0.45,trim = 50mm 0mm 50mm 0mm, clip=true,angle=-90]{retina.pdf}
 \caption{Schematic of a procedure for treating the retinal detachment. An external permanent magnet (a),
chosen based on its suitability for the given application to the eye surgery,
 is placed on the eye near the detachment site (b). \
 A ferrofluid droplet (c) is then injected into the eye. The external magnet guides the drop
to the site of the tear to seal it. Reproduced from \cite{Mefford}. 
} \label{fig:retina}
\end{center}
\end{figure}

To better understand the motion of the ferrofluid droplet
moving in the eye, Afkhami et al.~\cite{ARRRP} present 
a numerical investigation of a more straightforward scenario
where an initially spherical drop is placed at a height $L$ above 
the bottom of the cylindrical domain depicted in figure \ref{fig:figure4}. 
In addition, the boundary condition in the figure is changed to reflect 
the presence of a magnet at the bottom, which instantly magnetizes the drop. 
The boundary condition on the magnetic field is reconstructed from the experimental 
measurements in 
\cite{Mefford}, where 
the magnitude of $H(z)$ is measured as a function of distance from 
the magnet, $z$, in the absence of the drop.  
We then fit the data to a 5th degree polynomial, as shown in  
figure~\ref{fig:polynomial}.  
The scalar potential therefore is 
a 6th degree polynomial $\phi(0,z)=P_6(z)$ along the axis of the cylindrical domain.
In the absence of the drop, $\psi$ satisfies Laplace's equation 
$\frac{1}{r}\frac{\partial}{\partial r}(r\frac{\partial\psi}{\partial r})+\frac 
{\partial^2 \phi} {\partial z^2}=0$. If there is a solution, it is analytic and has 
$r^2$-symmetry.  The ansatz $\phi(r,z)=P_6(z)+r^2P_4(z)+r^4P_2(z)+r^6P_0(z)$ yields
\begin{eqnarray}
\phi(r,z)=P_6(z)-\frac{1}{4}r^2P_6''(z)+\frac{1}{64}r^4P_6^{(iv)}(z)-\frac{1}{(36)(64)}r^6P_6^{(vi)}(z). \label{phi}
\end{eqnarray}
This yields the boundary condition, and also approximates an initial condition 
when the drop is relatively small.  
%The lateral size of the computational domain is  
%chosen to be sufficiently large so that it is consistent with the %assumption 
%that results do not change if a larger size were used (see
%section \ref{subsection1}). These checks were done by calculating the solution for 
%double the lateral domain  size.
\begin{figure}
\begin{center}
\includegraphics[scale=0.5,trim = 0mm 80mm 40mm -20mm, clip=true]{polynomial.pdf}
 \caption{From \cite{ARRRP}. Measured data for the magnetic field from \cite{Mefford} 
($\blacklozenge$) and 
fifth degree  polynomial  fitted to the data (---) as functions of the distance from the 
magnet.}\label{fig:polynomial}
\end{center}
\end{figure}
The magnetic potential $\psi$ is calculated from equation (\ref{eq:phi}).
%In axisymmetric cylindrical coordinates,
%\begin{eqnarray}
%\frac{1}{r}\frac{\partial}{\partial r}\left( \mu r\frac{\partial\phi}{\partial 
%r}\right)+\frac 
%{\partial}{\partial z} \left(\mu\frac{\partial \phi}{\partial z} \right)=0 & \mbox{in 
%$\Omega$},
%\end{eqnarray}
%where $\Omega$ denotes the computational domain. 
The boundary conditions for $\psi$ 
on the domain boundaries $\partial \Omega$ are defined as
\begin{eqnarray}
\frac{\partial \psi}{\partial n_b} = \frac{\partial \phi}{\partial n_b}, %& \mbox{on $\partial \Omega$},
\end{eqnarray}
where $\partial /\partial n_b={\bf n}_b\cdot\nabla$, and ${\bf n}_b$  denotes the normal to the 
boundary $\partial \Omega$.
In order to impose the boundary condition in the numerical model,
a transformation of variables is performed to $\zeta$: $\phi=\psi+\zeta$, where
$\phi$ is the potential field without the magnetic medium.
One can then rewrite equation (\ref{eq:phi}) such that
\begin{equation}
\nabla\cdot(\mu\nabla\zeta)=-\nabla\cdot(\mu\nabla\phi),\label{magnet2}
\end{equation}
where $\nabla\cdot(\mu\nabla\phi)$ vanishes everywhere except on the surface
between the drop and the surrounding fluid $\partial \Omega_f$ and 
\begin{eqnarray}
\frac{\partial \zeta}{\partial n} = 0 & \mbox{on $\partial \Omega$}.\label{BCmagnet2}
\end{eqnarray}

Figure~\ref{fig:droplet_velocity} shows direct numerical simulation results 
for one of the series of experiments conducted in 
\cite{Mefford}: a droplet of 2~mm diameter placed 11~mm away
from the bottom of the domain where the permanent magnet is placed.
The figure shows the migration of the drop at times
t~=~120, 160, and 170~s  along with the velocity fields. 
The travel time compares well with the experimentally measured one in \cite{Mefford}.
Additionally, the results show that at an early stage, the flow occurs approximately downwards and 
only in the region close to the droplet. As the droplet approaches the magnet,
it elongates in the vertical direction and vortices induced in the viscous medium become stronger.
When the droplet reaches the bottom of the domain, the flow inside the droplet is pumped
outward from the center of the droplet, resulting in the flattening of the droplet
and consequently a decrease in the droplet height. These observations are consistent
with experiments in \cite{Mefford}.
\begin{figure}
\begin{center}
\includegraphics[trim = 70mm 25mm 70mm 40mm, clip, height=90mm]{1200_vel.pdf}
\includegraphics[trim = 70mm 25mm 70mm 40mm, clip, height=90mm]{1600_vel.pdf}
\includegraphics[trim = 70mm 25mm 70mm 40mm, clip, height=90mm]{1700_vel.pdf}
\includegraphics[trim = 70mm 25mm 70mm 40mm, clip, height=90mm]{1701_vel.pdf}
\end{center}
\caption{Drop shapes along with the velocity fields at times $t=120, 160$, and 170~s (from left to right) 
for a droplet of 2~mm diameter placed 11~mm away from the bottom of the domain,
where the magnet is placed. For these simulations, $\mbox{Bo}_m = 0.06$. Reproduced from \cite{ARRRP}.} \label{fig:droplet_velocity}
\end{figure}

%\subsubsection{Results}
%\label{sec:drop}
%\subsubsection{Magnetic Jetting Results}
%\label{sec:jetting}

\subsection{Magnetowetting of thin ferrofluid films}
\label{sec:lub}
\begin{figure}
\begin{center}
\includegraphics[scale=0.475,trim = 70mm 80mm 60mm 90mm]{science.pdf}
\includegraphics[scale=0.475,trim = 20mm 80mm 30mm 150mm]{science-wetting.pdf}
 \caption{Schematic side-view of the
experiments from \cite{Timonen2013}. State
of the droplet is: 1, near-zero field; 2,
weak field; 3, strong field; and
4, above critical field (drop splits to two daughter droplets). (B) Experimental photographs from \cite{Timonen2013} of a 20-ml ferrofluid droplet upon increasing the field
from 80 Oe (dH/dz 3.5 Oe/mm) to 680 Oe (dH/dz 66 Oe/mm).
}\label{fig:science}
\end{center}
\end{figure}

Figure \ref{fig:science} shows the phenomenon of magnetowetting
for an experiment with magnetic droplets on a superhydrophobic surface, below 
which is a permanent magnet \cite{Timonen2013}. By gradually increasing the strength of the magnetic
field and the vertical field gradient, the figure shows the transition of the the droplet from a 
spherical shape into a spiked cone and eventually splitting
into two smaller droplets at a critical field strength. 

In \cite{Seric2014}, a problem closely related to  magnetowetting is investigated:  
the application of a uniform magnetic field to induce dewetting of a thin ferrofluid film. 
Here we present the study in \cite{Seric2014}, where
thin film equations are derived using the long wave approximation of the 
coupled static Maxwell and Stokes equations and 
the contact angle is imposed via a disjoining/conjoining pressure model. 
%Numerical simulations show the patterning resulting from unstable perturbations and dewetting
%of the ferrofluid film. We find that the subtle competition between the applied field and the 
%van der Waals induced dewetting determines the appearance of satellite droplets. The results 
%suggest a new route for generating self-assembled ferrofluid droplets from a thin film using 
%an external magnetic field. An axisymmetric droplet on a surface is also studied and we demonstrate 
%the deformation of the droplet into a spiked cone, in agreement with experimental findings. 

%We use the long wave approximation to derive a two-dimensional model from Stokes equations coupled with 
%the static Maxwell equations. 
Figure \ref{fig:lub} shows the schematic of the system of two thin fluid
films in the region $ 0 < y < \beta h_c $, with the ferrofluid film occupying the region $ 0 < y < h_c $, 
and the nonmagnetic fluid occupying the rest of the domain.  We denote the permeability and viscosity of the 
fluids by $\mu_i$ and $\eta_i$, where $i=(1,2)$ denotes fluids $1$ and $2$, respectively. The interface between
two fluids is denoted by $y=h(x, t)$.   
\begin{figure}
  \centerline{\includegraphics{lub.pdf}}
  \caption{From \cite{Seric2014}. The schematic of the system of two thin fluids, where fluid $1$ is nonmagnetic and fluid $2$ is ferrofluid.}
\label{fig:lub}
\end{figure}
%The magnetic field satisfies the static Maxwell equations, $\nabla \cdot \mathbf{B} = 0$ and 
%$ \nabla \times \mathbf{H} = 0 $, where $\mathbf{B}$ is the magnetic flux and $\mathbf{H}$ 
%is the magnetic field, related by $\mathbf{B} = \mu_i \mathbf{H}$.  The permeabilities of the 
%nonmagnetic fluid  and the ferrofluid are $\mu_1 = \mu_0$, and $\mu_2 = \mu_0 \left( 1 + \chi_m \right)$, 
%respectively, where $\chi_m$ is the magnetic susceptibility and $\mu_0 = 4\pi \times 10^{-7} \mbox{ N}\, \mbox{A}^{-2}$ 
%is the permeability of the surrounding fluid, approximated by the permeability of vacuum. The magnetic scalar potential, 
%$\psi$, satisfies $\mathbf{H} = \nabla \psi $, which together with the Maxwell equations imply
%\begin{equation}
%\label{eq:LaplacePsi}
%\nabla^2\psi_i = 0 \, .
%\end{equation}
%The boundary conditions at the bottom and top boundaries are $\psi(0,t)=\psi_0$ and $ \psi(\beta h_c,t)=0$. 
%At the interface, $y=h\left(x, t\right)$, the normal component of $\mathbf{B}$ and the tangential component
%of $\mathbf{H}$ are continuous, $\left[ \mu_i \mathbf{H} \cdot \bf{n} \right] = 0$ and $\left[ \mathbf{H} \cdot \bf{t} \right] = 0,$
%where 
The unit normal and tangential vectors to the interface, $\textbf{n}$ and $\textbf{t}$, respectively, are given by
\begin{equation}
\textbf{n} = \frac{1}{\left(1+h_x^2\right)^{1/2}}\left( - h_x, 1\right) ,\,\,\,\,\,\,\,\,\,\, \textbf{t} = \frac{1}{\left(1+h_x^2\right)^{1/2}}\left(1, h_x\right).   \, \nonumber
\end{equation}
Given the small thickness of fluids, we ignore inertial effects and gravity. Hence, the equations governing the 
motion of the fluids are Stokes equations for continuity and momentum balance, $ \nabla \cdot {\bf{u}} = 0,\ \nabla \cdot \mathbf{T} = 0,$ respectively,
where $\mathbf{T} = -p \mathbf{I} + 2\eta {\mathbf D}  + \mathbf{\Pi}_M + \Pi_W\, \mathbf{I}$ 
is the total stress tensor and the disjoining pressure is specified by 
\begin{equation}
\label{eq:vdw}
\Pi_W \left( h \right)= \bar \kappa f\left( h \right), \,\,\,\,\mbox{where} \,\,\,\, f\left( h \right) =  \left( {h_*}/{h} \right)^n - \left( {h_*}/{h} \right)^m , \nonumber
\end{equation}
where $\bar \kappa = { \sigma \tan^2 \theta }/{\left(2 M {h}_* \right)}$ and $M = {\left(n - m\right)}/{\left[ \left( m - 1 \right)\left( n - 1 \right) \right]}.$
In this form, $\Pi_W$ includes the disjoining/conjoining intermolecular forces due to van der Waals interactions. 
The prefactor $\bar \kappa $, that can 
be related to the Hamaker constant,  measures the strength of van der Waals forces. Here,  $h_*$ is the short length scale introduced by the van der Waals potential.  We use $n=3$ and $m=2$.     
The contact angle, $\theta$, is the angle at which the fluid/fluid interface meets the substrate.
The addition of the van der Waals forces is crucial in the 
present context, since it allows us to study dewetting of thin films under a magnetic field. 

We nondimensionalizing the governing equations and boundary conditions using the following scales (dimensionless variables are denoted by tilde)
$$
x=x_c \tilde{x}, \,\,\,\,\,\, \left( y, h \right) = h_c \left( \tilde{y},\tilde{h} \right), \,\,\,\,\, \delta = h_c/x_c, \,\,\,\,\,\, u = u_c \tilde{u}, \,\,\,\,\,\,\,\, v=\left( \delta u_c \right) \tilde{v},  \nonumber
$$
$$
p = \left( \eta_2 u_c x_c/h_c^2 \right) \tilde{p}, \,\,\,\,\,\,\, t = \left( x_c/u_c \right) \tilde{t},\,\,\,\,\,\,\,\, \psi = \psi_0 \tilde{\psi},
$$
where $\delta \ll 1$. The initial thickness of the ferrofluid film is denoted $h_c$, and  $u_c$ and $x_c$ are 
the characteristic velocity and horizontal length scale, respectively, given by 
$$
 u_c = \frac{\mu_0\psi_0^2}{\eta_2 x_c}, \,\, x_c = \left( \frac{\sigma h_c^3}{\mu_0\psi_0^2} \right) ^{1/2}. \, \nonumber
$$
Dropping the tilde notation for simplicity, the evolution equation for $h(x,t)$ then becomes
\begin{equation}
\label{eq:axisymetric}
h_t + \frac{1}{3} \frac{1}{x^\alpha} \frac{\partial}{\partial x} \left[ \kappa f'\left(h\right)  x^\alpha h^3 h_{x} - \frac{ \mu_r \left( \mu_r - 1 \right)^{-1}}{\left( h - \frac{\beta \mu_r}{ \mu_r - 1 }\right)^3 } h^3 x^\alpha h_x + x^\alpha h^3 \frac{\partial}{\partial x}\left( \frac{1}{x^\alpha} \frac{\partial}{\partial x} \left(x^\alpha h_x  \right) \right) \right] = 0, 
\end{equation}
where $\kappa={h_c \sigma  \tan^2 \theta }/{\left(2 \mu_0\psi_0^2 M {h}_*\right)}$ is a nondimensional parameter representing 
the ratio of the van der Waals to the magnetic force, 
$\mu_r$ is the ratio of the ferrofluid film permeability to the 
vacuum permeability, $\mu_2/\mu_0$, and  $\alpha=0, 1$ for Cartesian and cylindrical coordinates, respectively.
The ratio $\beta \mu_r/(\mu_r - 1)$ is inversely proportional to the magnetic force (note that $\beta$ is the 
nondimensional distance between the plates with constant potential, 
so the gradient of the potential is inversely proportional to $\beta$). 


Here we show the thin film simulations of the steady state profiles 
obtained for a range of parameter values, in particular for nondimensional 
parameters $\beta$ and $\kappa$. We fix $\mu_r = 44.6$ and $h_* = 0.01$.
The initial condition is set to a flat film perturbed around a constant thickness $h_0$, i.e.
$h\left( x, 0 \right) = h_0 + \epsilon \cos{\left( k_m x \right)}\, ,$
with $\epsilon = 0.1$, and $h_0 = 1$.  Here, $k_m$ is the fastest growing mode computed from 
the linear stability analysis in \cite{Seric2014}. 
To put this in perspective,  $\kappa = 13.5$, $\beta = 8$,  and $\mu_r = 44.6$  correspond 
to values of $\sigma$ and $\mu_2$ for an oil-based ferrofluid ($\sigma = 0.034 \mbox{ N/m}$, 
$\mu_2 = \mu_0 (1 + \chi_m)$, where $\chi_m = 3.47 \times 4\pi$), and $\psi_0 = 1.2 \mbox{ A}$, and $\beta$ is chosen to produce a sufficiently strong magnetic field.
Making use of the symmetry of the problem, the computational domain is chosen to be equal to one half of the wavelength of the perturbation, i.e. $ L_x = \pi/k_m$.
No flux boundary conditions are imposed at the left and the right end boundaries as $ h'= h''' = 0 $.
Figure \ref{fig:Figure4}(a) shows the comparison of the steady state profiles for varying parameter $\kappa$ when keeping $\beta$ fixed. We note that as $\kappa$ decreases, i.e. the magnetic force increases compared to the van der Waals force, the satellite
droplets start to appear. Similar behavior is observed in figure~\ref{fig:Figure4}(b) for decreasing values of $\beta$; we observe the formation of satellite 
droplets for sufficiently small $\beta$, i.e. magnetic field dominates over the van der Waals interaction. 
It should also be noted that the satellite droplets are not present in the simulations where the effect of the magnetic field is ignored (i.e.~when $\mu_r = 1 $). 
We also note that for $\beta = 7.0$ shown in figure~\ref{fig:Figure4}(b), a static drop cannot be obtained, unlike when $\beta > 7.0$: here the height of the 
fluid approaches the top boundary and the assumptions of the model are not satisfied for the times later than the one at which this profile is shown.
\begin{figure}
\begin{center}
\includegraphics{Figure4.pdf}
\end{center}
\hspace{34mm} (a) \hspace{61mm} (b)% Images in 100% size
  \caption{From \cite{Seric2014}.The effect of varying $\beta$ and $\kappa$ on the film evolution 
                when (a) $\beta = 8$ is fixed, and (b) $\kappa = 13.5$ is fixed. 
                Note that when $\beta \le 7.0$, no steady-state drop profile can be achieved, 
                indicative of an interfacial instability.}
\label{fig:Figure4}
\end{figure}
The results of simulations suggest that  
the satellite droplets form when (i) the magnetic force is sufficiently strong, 
and (ii)  the van der Waals force is sufficiently 
weak, relative to the magnetic force. 







