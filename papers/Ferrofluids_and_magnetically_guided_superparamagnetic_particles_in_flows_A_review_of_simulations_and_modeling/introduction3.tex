  This review presents three aspects of the mathematical analysis of the motion of superparamagnetic particles and drops in a surrounding fluid, under the 
influence of an external 
magnetic field: drug targeting, ferrofluid drop deformation, and instabilities and dewetting of a thin ferrofluid film. 
 These problems are important for a number of biomedical applications.
  The motion of a magnetic particle is determined by a force balance of the magnetic force, drag, and stochastic forces. The motion of a ferrofluid drop is 
affected by its deformation. 
  Simulations  using the Volume-of-Fluidk method are presented. Once a drop reaches the target and coats it, a thin film approximation yields insight into 
instabilities and breakup.    We refer the
reader to  recent publications, which in turn contain more  comprehensive references and historical reviews.

  
An advantage of magnetic drug targeting is the delivery of drugs 
to  specific tissues via  the blood vessels as pathways under  externally  
applied magnetic fields. Progress towards  robust biomedical applications  
 relies on the synthesis of appropriately coated nanoparticles, clusters 
 and ferrofluids \cite{Liu2007,ATRRWPR,Bala2014,Dung2017}.  Pros and 
cons of 
directing the magnetic nanoparticles in blood vessels and tissues are 
described in 
\cite{Voltairas02,Neuberger2005,Buzea2007,Berry2009,Mishra2010,Nacev2011}. Recent reviews 
highlight the use of magnetic 
nanoparticles in many biomedical applications; for instance,  drug targeting, magnetic fluid hyperthermia, 
   design of devices to measure targeting efficiency, and tissue engineering
\cite{Ito2005,Puri2014,Maguire2014,aljamal2016,Mohammed2017,Radon2017}. 

In \cite{Suh2011}, the mathematical modeling of the transport of 
paramagnetic particles 
in viscous flow is  described in terms of dipole points in an external 
magnetic field, with particle interactions for  larger particles and 
higher concentrations\cite{Banerjee2012}. Numerical 
approaches using Brownian dynamics for particle interactions in 
aggregation and disaggregation are being developed \cite{vanReenen2014}.
Technological applications  include the sorting of not only cells but proteins and other biological components by biochemically funcionalized paramagnetic beads that 
are manufactured to be attracted to specific targets. The mixture flows in a fluid, and as they move through a channel, magnetic fields deflect the desired cells to 
migrate toward predetermined exits.
In \cite{Tsai2011}, paramagnetic beads floating in a liquid are subjected to a magnetic field normal to the flow. The deflection by the magnetic field steers the 
beads towards a target whose location depends on the field and on the properties of the bead.  An application to removing pathogens from the blood stream is shown in 
\cite{small}, reproduced in figure \ref{kangfig}.  Here the magnetic particles bind to pathogens such as bacteria. In addition to the motion of the 
particles, a model for the kinetics of this binding mechanism is formulated.
\begin{figure}
\includegraphics[width=1\textwidth]{kangfig.pdf}
\caption{With permission from \cite{small}. Summary of caption:  `(A) A schematic design of the biospleen that magnetically cleanses fluids contaminated with 
bacteria using magnetic nanoparticles coated with FcMBL. (B) The magnetic separation process dissociated into two sequential phases; magnetic beads binding to cells 
and magnetic separation of cell - magnetic bead complexes.  (C) A cross-sectional view of the biospleen channel through which magnetic bead-bound bacteria flow while 
they are deflected upward  by external magnets on the top. If bacteria that passed position A reach the other side of the channel before they pass position B, we can 
assume that they are completely removed.'
}
\label{kangfig}\end{figure}


The design of the superparamagnetic nanoparticles involves uniformity in the small size. Of particular interest is the formation of a ferrofluid from chemically bonding many of them. Unlike the colloidal suspension of magnetite particles, this ferrofluid avoids the  migration of the particles in the suspending fluid at high fields. This advantage is seen in the works of \cite{Mefford,Mefford2008a,Mefford2008b} for the potential treatment of retinal detachment by magnetically guiding a ferrofluid drop to seal the site. In particular, figure \ref{fig:fig1}(a) shows a sample transmission electron microscopy (TEM) 
image of the magnetite nanoparticles at $300000$ times magnification, and (b) shows the narrow size distribution. These particles are relevant to section~\ref{mdt}.  
The magnetic nanoparticles are coated with  biocompatible polydimethylsiloxane (PDMS) oligomers, and bonded to manufacture a ferrofluid. Figures \ref{fig:fig1}(c-d) show  experimentally observed states for the ferrofluid drop suspended in glycerol in uniform magnetic fields.  
The modeling in section \ref{sec:vof} concerns the numerical simulation of stable equilibrium configurations under applied uniform magnetic fields. A series of experiments conducted under low and high magnetic field strengths 
are presented in \cite{ATRRWPR} to understand the behavior of PDMS
ferrofluid drops in uniform magnetic fields. 
Figure~\ref{fig:figure12} shows the experiments in \cite{ATRRWPR} of the drop at equilibrium for
the magnetic field strengths  varying from $6.032$ to $162.33$ kA~m$^{-1}$.
These results are revisited in section~\ref{sec:vof}. For an updated review of literature
on ferrofluid drop deformation in uniform magnetic fields, see \cite{Rowghanian}.
%-------------------------------------------------------------------------------
\begin{figure}
\begin{center}
 \includegraphics[width=0.45\textwidth]{fig1a.pdf}
  \includegraphics[width=0.5\textwidth]{fig1b.pdf}\\
(c)\includegraphics[width=0.45\textwidth,angle=90]{IMG_0138.pdf}
(d)\includegraphics[width=0.45\textwidth,angle=90]{IMG_0085-new.pdf}
\end{center}
\caption{(a) Representative TEM image of PDMS--magnetite particles. 
(b) Histogram of the radii of the magnetite cores. From \cite{Mefford} and \cite{Mefford2008a}.
Experimental photographs of the deformation of a PDMS ferrofluid drop
in a viscous medium under applied uniform magnetic fields of $6.38$ kA~m$^{-1}$ (c) 
and $638.21$ kA~m$^{-1}$ (d). From \cite{ARRRP} with permission.} 
\label{fig:fig1}
\end{figure}
%-------------------------------------------------------------------------------
\begin{figure}
\begin{center}
\includegraphics[scale=3]{figure12.pdf}
\caption{Photographic images from \cite{ARRRP} of the drop equilibrium shape taken at
magnetic field strengths $6.032, 12.167, 23.947, 59.762, 162.33$ kA~m$^{-1}$ (from left to right).} 
\label{fig:figure12}
\end{center}
\end{figure}
%-------------------------------------------------------------------------------


The numerical investigation of the motion of a ferrofluid drop may be complicated by the highly distorted  material interface. An advantage of 
the boundary integral method for  three-dimensional (3D) drop deformation in Stokes flow is that the three-dimensionality is turned into a  two-dimensional surface 
computation; this together with a high order surface discretization has been able to track viscous drops with high curvature, up to the first pinch-off
\cite{CristiniBLSG,Janssen2008}. However,  this  method  does not extend easily to
non-Newtonian models such as a power-law model that is suggested for blood flow. Novel nonsingular boundary integral methods are improving the accuracy and numerical 
stability of the 3D drop deformation in viscous flow at low inertia, including the influence of a magnetic field for constant  magnetic susceptibility  
\cite{Bazhlekov06}. The effect of rotating magnetic field is investigated with the boundary integral formulation in \cite{Erdmanis2017}. 
The study of  rotating magnetic fields on ferrofluid drops is of importance in many fields including  astrophysics, and we refer the reader to recent reviews 
\cite{Lebedev2003,Fengchen2016}. A limitation of the boundary integral formulation is that it is no longer applicable if the magnetic behavior is nonlinear.

Numerical algorithms with a diffuse-interface method have been developed  
 for a droplet or layer in a uniform and rotating magnetic fields  \cite{Fengchen2016}. Intricate equilibrium shapes arise in the Rosensweig instability which 
originally refers to a ferrofluid layer under magnetic force, surface tension force, gravity and hydrodynamics \cite{Cowley,Rosensweig,Lange2007,Kadau2016}. In 
\cite{Lavrova2012}, the Rosensweig instability is investigated numerically for the diffusion of magnetic particles. The magnetic particle concentration and the free 
surface shape are part of the solution. A finite element method is used for Maxwell's equations and a finite difference method is used for a parametric 
representation of the free surface. 
 A finite element formulation is developed and used in \cite{Kang2013} to simulate the interaction of two or more particles in uniform and rotating magnetic fields. 
The interaction is  significant in  dense suspensions or larger sizes.   In section \ref{sec:vof}, we focus on  a 
Volume-of-Fluid formulation of \cite{ARRRP} which is 
capable of utilizing a non-constant susceptibility,  time-dependent evolution,  and  interface breakup and reconnection.

Finally, thin films driven by a magnetic field have been studied recently \cite{Seric2014,Conroy2015}. 
These models are derived for the flow of a thin ferrofluid film on a substrate in the same 
spirit as the previous work by Craster and  Matar \cite{Craster2005} for the case of a leaky 
dielectric model. The derivation for a nonconducting ferrofluid simplifies since there is no interfacial
charge involved. The work of Seric et al.~\cite{Seric2014} includes 
the van der Waals force to impose the contact angle using the disjoining/conjoining pressure model
and studies the film break up and the formation of satellite (secondary) droplets on 
the substrate under an external magnetic field. In section \ref{sec:lub}, we focus on  
a thin film approximation of ferrofluid on a substrate subjected
to an applied uniform magnetic field and present the results of 
the magnetic field induced dewetting of thin ferrofluid films and the consequent 
appearance of satellite droplets. 
\FloatBarrier
