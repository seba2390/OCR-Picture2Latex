To investigate the response of a ferrofluid droplet to an applied
magnetic field, extra effort is needed to solve the details of the 
magnetic field, which in turn requires the knowledge of the
shape of the interface. The non-local moving boundary problem
which couples the spatial variation of the magnetic field with the 
interfacial shape of the drop necessitates a direct numerical computation. 
Numerical algorithms that explicitly track the interface and use finite
element methods are developed to investigate equilibrium shapes of
ferrofluid drops and interfacial instabilities in \cite{Lavrova,Lavrova04,Bashtovoi,Matthies,Knieling}. 
For the case of constant magnetic susceptibility, a boundary-integral 
formulation is used to study equilibrium shapes for high-frequency 
rotating magnetic fields \cite{Erdmanis2017}.  Numerical algorithms that 
implicitly track the interface include the Volume-of-Fluid (VOF) methods, 
Level-Set methods and diffuse-interface (DI) methods \cite{Scar99,Sethian2003,AMW}. 
Each of these methods can be applied to subcases of the results we show below. 
We specifically describe the VOF method because of its simplicity, efficiency and 
robustness for tracking topologically complex evolving interfaces. 
However, this as well as other approaches has its limitations. 
An advantage is that it conserves mass and with the recent improvements 
in the discretization of the surface tension force, it remains a competitive 
method for modeling interfacial flows. We present these results in section \ref{sec:vof}.

A recent study has shown that a ferrofluid droplet on a hydrophobic substrate 
breaks up into two daughter droplets and a smaller satellite droplet under a sufficiently
strong magnetic field \cite{Timonen2013}. These experiments also show that if the magnetic
field is increased even further, the droplets break up again and rearrange to form an assembly
of droplets on the substrate. In the same spirit, we present a model derived in \cite{Seric2014}
for the flow of a thin ferrofluid film on a substrate. The results in  \cite{Seric2014} show 
the film break up and the deformation of sessile droplets on the substrate under an external magnetic field.
We present this study in section \ref{sec:lub} 

\subsection{Governing equations}
\label{sec:vof}
The equations governing the motion of an incompressible ferrofluid drop 
suspended in another fluid are the conservation of mass and the balance of momentum
\begin{eqnarray}  
\frac{\partial \rho}{\partial t}+{\mathbf u}\cdot\nabla \rho&=&0,\label{eq:mass}\\
\rho \frac{\partial{\mathbf u}}{\partial t}+{\mathbf u}\cdot\nabla {\mathbf u}&=&-\nabla p + 
\nabla\cdot \bigl(2\eta {\mathbf D} \bigr) + \nabla\cdot{\mathbf \Pi}_M + \rho {\mathbf g},
\label{eq:motion}
\end{eqnarray}      
where  ${\mathbf u}({\mathbf x},t)$ is the velocity field, $p({\mathbf x},t)$ the pressure, 
$\rho({\mathbf x},t)$  and $\eta({\mathbf x},t)$ are the phase dependent density
and viscosity, respectively,
${\mathbf D}=\frac{1}{2}\big( \nabla{\mathbf u} +(\nabla{\mathbf u})^T\big)$ 
is the rate of deformation tensor (where $T$ denotes the transpose),
$\rho\mathbf{g} = - \rho g{\hat z}$ the body force due to gravity
and ${\hat z}$ the unit vector in the $z$-direction (with gravitational constant $g$),
and ${\mathbf \Pi}_M({\mathbf x},t)$ is the magnetic stress tensor
\begin{eqnarray} 
{\mathbf \Pi}_M  =  - \frac{\mu_0}{2} ({\mathbf H}\cdot{\mathbf H}){\mathbf I} +  \mu{\mathbf H}{\mathbf H},
\end{eqnarray}
where ${\mathbf I}$ is the identity tensor.
The Maxwell equations
for the  magnetic induction ${\mathbf B} $, magnetic field  ${\mathbf H}$, and  magnetization
${\mathbf M}$ are the magnetostatic Maxwell equations for a non-conducting ferrofluid 
\begin{eqnarray}
\nabla\cdot{\mathbf B}&=&0,\quad \nabla\times{\mathbf H}=0,\label{eq:main_eq}\\
{\mathbf B}({\mathbf x},t) &=&  \left\{ 
\begin{array}{ll}
                \mu_{d} {\mathbf H }  &  \mbox{in ferrofluid drop} \\
                \mu_m {\mathbf H}   &  \mbox{outside ferrofluid drop,}
\end{array}        
\right. 
\label{classical}
\end{eqnarray}
where $\mu_d$ denotes the permeability of the drop, and 
$\mu_m$ is the permeability of the surrounding fluid.  
Here, we assume that the surrounding fluid has a permeability
very close to that for a  vacuum, $\mu_0$. Therefore, we shall set $\mu_m=\mu_0$ 
throughout. A magnetic scalar potential $\psi$ is defined by
${\mathbf H}=\nabla\psi$, and satisfies
\begin{equation}
\nabla\cdot(\mu\nabla\psi)=0.  \label{eq:laplace}
\end{equation}  

For linear magnetic materials, the magnetization is a linear function of the magnetic field given 
by  ${\mathbf M} = \chi {\mathbf H}$,
where $\chi= (\mu/\mu_0 - 1)$ is the magnetic susceptibility. 
The magnetic induction ${\mathbf B}$ is therefore  
${\mathbf B}=\mu_0({\mathbf H}+{\mathbf M})=\mu_0(1+\chi){\mathbf H}$. 
To describe the paramagnetic susceptibility quantitatively, the Langevin function
 $L(\alpha)$=coth~$\alpha-\alpha^{-1}$  is used to describe the magnetization 
versus the magnetic field as 
\begin{eqnarray}
{\mathbf M} ({\mathbf H}) = M_s L\left(\frac{\mu_0 m |{\mathbf H}|}{k_B T}\right)\frac{{\mathbf H}}{|{\mathbf H}|}, \label{langevin1}
\end{eqnarray} 
where the saturation magnetization, $M_s$, and the magnetic moment of the particle, $m$,
enter as parameters, $T$ denotes the temperature, and $k_B$ is the Boltzmann's constant.

Let $\lbrack\!\lbrack \cdot \rbrack\!\rbrack$ denote the difference, $\cdot_{\rm surrounding}-\cdot_{\rm drop}$,  at the 
interface between the ferrofluid drop and the external liquid. Let ${\mathbf n}$ denote the unit normal 
outwards from the interface, and ${\mathbf t_1}$ and ${\mathbf t_2}$ denote the orthonormal tangent vectors. We 
require 
\begin{enumerate}
\item The continuity of velocity, the normal component of magnetic induction, the tangential component of 
the magnetic field, the tangential component of surface stress, 
\begin{eqnarray}
\lbrack\!\lbrack {\mathbf u}\rbrack\!\rbrack ={\mathbf 0},\quad
{\mathbf n}\cdot \lbrack\!\lbrack {\mathbf B}\rbrack\!\rbrack =0,\quad
{\mathbf n}\times \lbrack\!\lbrack {\mathbf H} \rbrack\!\rbrack ={\mathbf 0},\quad
 \lbrack\!\lbrack {\mathbf t_1}^T \cdot {\mathbf \Sigma}  \cdot {\mathbf n}  \rbrack\!\rbrack = 0,\quad
\lbrack\!\lbrack {\mathbf t_2}^T  \cdot {\mathbf \Sigma}   \cdot {\mathbf n} \rbrack\!\rbrack = 0.\nonumber
\end{eqnarray}

\item The jump in the normal component of stress balanced by capillary effects,
\begin{eqnarray}
\lbrack\!\lbrack {\mathbf n}^T \cdot {\mathbf \Sigma}  \cdot {\mathbf n} \rbrack\!\rbrack = 
\sigma k ,\nonumber
\end{eqnarray}
where ${\mathbf \Sigma}  = p - \rho g z + \bigl(2\eta {\mathbf D} \bigr) + {\mathbf \Pi}_M$,  
$k = -\nabla \cdot {\mathbf n}$ is the interface curvatures, and $\sigma$ 
is the interfacial tension.
\end{enumerate}

%We will also consider two magnetic field types: uniform and nonuniform magnetic fields. 
%For a uniform magnetic field, we consider
%${\mathbf H}=(0,0,H_o)$, where $H_o$    
%is the magnetic field intensity at infinity. 
%In Sec.~\ref{sec:numerics}, we describe the boundary conditions 
%for simulating non-uniform magnetic fields.




