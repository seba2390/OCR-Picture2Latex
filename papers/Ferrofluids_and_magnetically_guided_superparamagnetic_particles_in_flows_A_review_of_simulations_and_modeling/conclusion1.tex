Biomedical technology is expanding the use of coated superparamagnetic nanoparticles and their suspensions in novel directions; for instance, the delivery of drugs to targeted cells, magnetic resonance imaging, early diagnosis of cancer, treatment such as hyperthermia,  and cell separation.  The forefront of research lies at the symbiotic development of nanomedicine, nanotechnology, engineering of strong and highly focused magnets, together with numerical algorithms. In this article, three simplified mathematical models have been presented with the aim of producing numerically generated solutions: stochastic differential equations linked to magnetic drug targeting, a volume-of-fluid computational scheme for the motion of a ferrofluid drop through a viscous medium under a magnetic field, and the evolution of a thin film of ferrofluid with numerical investigation of dewetting. An important future direction is the integration of open-source codes for the numerical simulations used with actual biomedical applications. A drawback of commercial software is that the  location of numerical  inaccuracy in any algorithm is difficult to pinpoint in a black box. The drive toward small scales and complex biomedical domains and materials necessitates a computational approach. 