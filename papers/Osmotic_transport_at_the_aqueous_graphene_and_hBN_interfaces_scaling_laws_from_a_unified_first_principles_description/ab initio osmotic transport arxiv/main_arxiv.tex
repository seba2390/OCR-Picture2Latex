\documentclass[journal=ancac3,manuscript=article,layout=twocolumn]{achemso}
% ,layout=twocolumn

\setkeys{acs}{maxauthors = 0}      % will list all authors

\setkeys{acs}{articletitle = true}

\usepackage{comment}
\setlength {\marginparwidth }{2cm}
\usepackage{todonotes}
\usepackage{amssymb}
\usepackage{amsfonts}
\usepackage{amsmath}

\usepackage{graphicx}% Include figure files
\usepackage{dcolumn}% Align table columns on decimal point
\usepackage{bm}% bold math

\usepackage[export]{adjustbox}

\usepackage{array}
\newcommand\Tstrut{\rule{0pt}{2.8ex}}
\newcommand\Bstrut{\rule[-1.ex]{0pt}{0pt}}   % = `bottom' strut
\newcolumntype{P}[1]{>{\centering\arraybackslash}p{#1}}


\usepackage[version=3]{mhchem}

\usepackage{graphicx}% Include figure files
\usepackage{dcolumn}% Align table columns on decimal point
\usepackage{bm}% bold math



 \makeatletter
% %\setlength\acs@tocentry@height{13.6cm}
% %\setlength\acs@tocentry@width{10.5cm}
% %\setlength\acs@tocentry@height{6.8cm}
% %\setlength\acs@tocentry@width{4.9cm}
\setlength\acs@tocentry@height{4.cm}
 \setlength\acs@tocentry@width{9.cm}
 \makeatother




\newcommand*\mycommand[1]{\texttt{\emph{#1}}}

\newcommand{\df}{\mathrm{d}}

\newcommand{\kt}{k_\text{B}T}
\newcommand{\eps}{\varepsilon}
\newcommand{\lB}{\ell_\text{B}}
\newcommand{\debye}{\lambda_\text{D}}
\newcommand{\lGC}{\ell_\text{GC}}
\newcommand{\rhoe}{\rho_\text{e}}
\newcommand{\Vs}{V_\text{s}}
\newcommand{\vs}{v_\text{s}}
\newcommand{\zs}{z_\text{s}}
\newcommand{\phis}{\phi_\text{s}}
\newcommand{\phim}{\phi_\text{m}}
\newcommand{\psim}{\psi_\text{m}}
\newcommand{\nm}{n_\text{m}}
\newcommand{\ns}{n_\text{s}}
\newcommand{\Rch}{R_\text{ch}}
\newcommand{\Zch}{Z_\text{ch}}
\newcommand{\Rl}{R_\text{L}}
\newcommand{\Sstr}{S_\text{str}}
\newcommand{\sgn}{\text{sgn}}
\newcommand{\tosm}{\text{to}}
\newcommand{\tel}{\text{te}}


\newcommand{\red}[1]{\textcolor{red}{#1}}
\newcommand{\blue}[1]{\textcolor{blue}{#1}}
\newcommand{\lj}[1]{\textcolor{teal}{#1}}
\newcommand{\rom}[1]{\textcolor{purple}{#1}}
\newcommand{\gt}[1]{\textcolor{orange}{#1}}

\newcommand{\uzh}{Department of Chemistry, Universit\"at Z\"urich, 8057 Z\"urich, Switzerland}
\newcommand{\ilm}{Univ Lyon, Univ Claude Bernard Lyon 1, CNRS, Institut Lumi\`ere Mati\`ere, F-69622, VILLEURBANNE, France}
\newcommand{\iuf}{Institut Universitaire de France (IUF), 1 rue Descartes, 75005 Paris, France}
\newcommand{\tuhh}{Hamburg University of Technology, Insitute of Polymers and Composites, Hamburg, 21073, Hamburg}
\newcommand{\helm}{Helmholtz-Zentrum Hereon, Institute of Surface Science, Geesthacht, 21502, Germany}


\author{Laurent Joly}
\affiliation[Universit\'e Lyon 1]{\ilm}
\alsoaffiliation[IUF]{\iuf}
\author{Robert H. Mei{\ss}ner}
\affiliation[Hamburg University of Technology]{\tuhh}
\alsoaffiliation[Helmholtz-Zentrum]{\helm}
\author{Marcella Iannuzzi}
\affiliation[Universit\"at Z\"urich]{\uzh}
\author{Gabriele Tocci}
\affiliation[Universit\"at Z\"urich]{\uzh}
\email{gabriele.tocci@chem.uzh.ch}
%%%%%%%%%%%%%%%%%%%%%%%%%%%%%%%%%%%%%%%%%%%%%%%%%%%%%%%%%%%%%%%%%%%%%
%% The document title should be given as usual. Some journals require
%% a running title from the author: this should be supplied as an
%% optional argument to \title.
%%%%%%%%%%%%%%%%%%%%%%%%%%%%%%%%%%%%%%%%%%%%%%%%%%%%%%%%%%%%%%%%%%%%%
\title{Osmotic transport at the aqueous graphene and hBN interfaces: scaling laws from a unified, first principles description.
}

\abbreviations{}
\keywords{osmotic transport, blue energy, nanofluidics, electrical double layer, \textit{ab initio} molecular dynamics, two-dimensional materials, graphene, hBN}

\begin{document}


\begin{tocentry}
\includegraphics[width=9.0cm,height=4cm
,clip]{TOC_figure.png}
\end{tocentry}
%

\begin{abstract}
Osmotic transport in nanoconfined aqueous electrolytes provides new venues for water desalination and ``blue energy'' harvesting; the osmotic response of nanofluidic systems is controlled by the interfacial structure of water and electrolyte solutions in the so-called electrical double layer (EDL), but
a molecular-level picture
of the EDL is to
a large extent still lacking.
Particularly, the role of the electronic
structure has not been considered
in the description of electrolyte/surface
interactions.
Here, we report enhanced sampling simulations
based on \textit{ab initio} molecular dynamics,
aiming at unravelling the free energy of prototypical
ions adsorbed at the aqueous graphene and hBN interfaces,
and  its consequences on nanofluidic osmotic transport.
Specifically, we predicted the zeta potential, the diffusio-osmotic
mobility and the diffusio-osmotic conductivity
for a wide range of salt
concentrations from the \textit{ab initio}
water and ion spatial distributions through an analytical
framework based on Stokes equation and a
modified Poisson-Boltzmann equation.
We observed concentration-dependent scaling laws,
together with
dramatic differences in osmotic transport
between the two interfaces, including
diffusio-osmotic flow and current
reversal on hBN,
but not on graphene.
We could rationalize the results for the three osmotic responses with a simple model based on characteristic length scales for ion and water adsorption at the surface, which are quite different on graphene and on hBN.
Our work provides first principles insights into the
structure and osmotic transport of aqueous electrolytes
on two-dimensional materials and
explores new pathways for efficient water
desalination and osmotic energy conversion.
\end{abstract}

\section{Introduction}
Universal access to drinkable water and
 widespread production of electricity
from renewable energy sources are two
of the most daring challenges faced by
modern society. Progress in the field of nanofluidics
offers alternative solutions to
water desalination and to energy conversion
through the mixing of salty and fresh water,
so-called ``blue'' energy harvesting.
Over the past decade, many novel osmotic
transport phenomena have been observed
in nanofluidic
systems.\cite{siria2017new,wang2017fundamental,Tong2021,macha20192d}
Among several noticeable examples is the osmotic
current generated across boron nitride nanotubes
and MoS$_2$ nanopores,
whose power densities exceed by several orders of magnitude
those produced by conventional
membranes\cite{Siria2013,feng2016single}.
A further example is the observation of qualitatively
different current-voltage characteristics
in the nonlinear transport of ions across graphene
and hBN angstrom-scale
slits, which interestingly hints at the critical role
of the crystal and electronic structure of the interface\cite{esfandiar2017size}.
The measurement of conductance oscillations and
Coulomb blockade in sub-nanometers MoS$_2$ pores
also indicates that  chemical nature,
 dimensions and geometry  of nanopores
are key factors to the observed nonlinear behaviours\cite{feng2016observation}.
Thus, it stands to reason that
obtaining a molecular-level picture of
aqueous interfaces is essential to predict and control
osmotic transport phenomena, and may lead to
fundamental advances in the field of nanofluidics.

A comprehensive picture of the structure of
water and electrolyte solutions
at electrified surfaces, in the
so-called electrical double layer (EDL),
has not been obtained so far.
Specifically, the structure of the EDL is not
accurately described using
the standard Gouy-Chapman theory
of the EDL based on the Poisson-Boltzmann (PB)
equation\cite{Hunter2001}.
For instance, in a thin
region of the EDL -- typically of the order of 1\,nm --
ions may interact specifically with solid
surfaces. Ion-specific
effects in this region, which we define ``adsorption layer'', are not captured
by the standard PB equation\cite{luo2006ion,huang2007ion}.
Additionally, the PB theory
of the EDL typically neglects the polar nature
of water and water layering at the liquid/solid
interface, and ignores
spatial and dynamic interface heterogeneities\cite{gonella2021water,Hartkamp2018,Markovich2016,Bonthuis2013,limmer2013hydration}.
Experimentally, a vast array of techniques have been used to probe
the structure of the EDL\cite{Hartkamp2018}. Recent examples include
second harmonic generation, which has
revealed structural and dynamical
heterogeneities in the water orientation
at the interface with silica\cite{macias2017optical},
and ambient pressure X-ray photoelectron spectroscopy,
which has probed the shape of the electrostatic potential profile of aqueous electrolytes
on gold electrodes at different concentrations\cite{favaro2016unravelling}.
Despite the tremendous advancement these types of work represent
for the field, achieving sub-nanometer resolution, which is required
to characterise the molecular structure of the EDL,
remains an open experimental challenge.


A further challenge is to link the  structure of aqueous  interfaces to osmotic transport properties. Although experiments have hinted at the predominant role of charged groups, pore geometry and pore chemistry\cite{Hartkamp2018}, a microscopic characterization of the interface under operating conditions has not been obtained.
Alternatively, atomistic simulations, and in particular molecular dynamics, can be used to explore the structure of the EDL at the molecular level.
Force-field molecular dynamics (FFMD), which is based on an empirical description of the interactions between the constituent atoms, has yielded invaluable insights into the structure of the EDL \cite{Siepmann1995InfluenceSystems, Scalfi2020ChargeEnsemble,scalfi2020a,Limmer2013ChargeCapacitors} and into the molecular mechanisms underlying liquid and solute transport in nanofluidics\cite{striolo2016carbon,phan2016confined,Faucher2019,Falk2010MolecularFriction,Ma2015WaterFriction,xie2018fast,huang2007ion,Ajdari2006,heiranian2015water,noh2020ion,liu2018pressure,simoncelli2018blue,Kalra2003OsmoticMembranes,Vasu2018ElectricallyMembranes}
%
Yet, it is challenging to determine accurate force fields that incorporate
electronic structure effects observed at complex liquid/solid interfaces.
In contrast,
\textit{ab initio} molecular dynamics (AIMD) simulations
based on density functional theory (DFT)
are instrumental to compare
the structure and dynamics of different aqueous
interfaces on equal footing.
Although AIMD is increasingly being used
to characterise the structure of
aqueous interfaces
\cite{le2020molecular,cheng2012alignment,gross2019modelling,lan2020ionization,seiler2018effect},
including OH$^{-}$ adsorption on two-dimensional
materials\cite{Grosjean2016},
H$_3$O$^+$ adsorption on TiO$_2$\cite{stecher2016first}
and ion adsorption at the liquid/vapour interface\cite{duignan2021toward,baer2011toward},
the impact of ion- and water-specific interactions
on osmotic transport at aqueous interfaces has not
been investigated with AIMD so far.


In this work we tackle two
challenges: we determine the structure of model aqueous
electrolyte interfaces from \textit{ab initio} methods and we compute the
mobilities underlying osmotic transport processes due to different applied external fields
(see Fig.~\ref{fig:schematic}). Our systems consist of a potassium iodide (KI) solution at the
aqueous graphene and  hBN interfaces.
We chose the aqueous graphene and hBN
interfaces because they are well-studied model systems in
nanofluidics and for their potential impact
as nano-osmotic power generators\cite{macha20192d}.
Also, we focus on KI
as a model electrolyte displaying ion-specific
effects\cite{schwierz2010reversed,huang2007ion,baer2011toward}.
The structure of the EDL and osmotic transport coefficients are computed
according to the following steps: First, we calculate the free
energy of adsorption
of K$^+$ and I$^-$ ions dissolved in a 2 nanometer-thick
water film using enhanced sampling
techniques based on AIMD  (a snapshot of a
representative system is shown in the inset of
Fig.~\ref{fig:free_energy}(a)); Second, we obtain the electrostatic potential profile and the
spatial distributions of the ions
at different salt concentrations by solving
a modified Poisson-Boltzmann (mPB) equation that
accounts for the ions' free energy of
adsorption on the sheets;
Finally, osmotic transport coefficients are obtained, within linear response theory,
by computing the relevant fluxes resulting from each external field, based on Stokes equation with a slip boundary condition at the wall.
We find remarkable ion- and surface-specific
adsorption of KI at the graphene and hBN interface. Such specific effects give rise
to concentration-dependent scaling laws of the
osmotic transport coefficients and result into strikingly
different osmotic transport behaviour at the graphene
and hBN interface. We rationalize the obtained scaling laws
with a theoretical model that describes
ion and water adsorption in terms of characteristic
length-scales that are limited to a few molecular diameters.



\section{Results}
\subsection{Unified framework of osmotic transport}
One of the main objectives of this work is to provide a theoretical
framework to calculate the osmotic transport coefficients
from the molecular structure of water and ions at liquid/solid interfaces.
We do so within linear response theory, where the
system of equations shown in Fig.~\ref{fig:schematic}
describes the thermodynamic fluxes resulting from applied external forces.
The desired transport coefficients
are the off-diagonal elements of the matrix
shown in the middle of Fig.~\ref{fig:schematic},
and obey Onsager's reciprocal relations\cite{Onsager1931a,Onsager1931b},
\textit{i.e.} the matrix is symmetric.
The osmotic transport coefficients
are computed from  hydrodynamics through
the Stokes equation, in which
the velocity profile is obtained
using a partial slip boundary condition at the wall $v(z=0) = b \partial _z v(z=0)$, with $b$
the slip length\cite{Bocquet2007}.
Throughout this work we thus assume
that continuum hydrodynamics
is valid at the nanoscale and
that the viscosity is homogeneous.
%%%UNCOMMENT FOR FINAL SUBMISSION TO ACS NANO
% {\color{red}While on hydrophilic and charged surfaces, osmotic flows are better described by assuming a larger viscosity in a subnanometric interfacial layer\cite{Bonthuis2013,rezaei2021interfacial},}
%%% CHANGE On hydrophobic TO on hydrophobic WHEN SUBMITTING THE FINAL VERSION TO ACS NANO
On hydrophobic, slipping surfaces such as the ones considered in this work, our assumptions have been shown to provide an accurate description of the velocity profiles even in the first molecular layers of the liquid.\cite{Bonthuis2013}

\begin{figure*}[thb!]
%\includegraphics[width=\textwidth,trim={2cm 7cm 2cm 7cm},clip]{Schematic_Final_version.png}
\includegraphics[width=\textwidth]{Schematic_Final_version.jpg}
\caption{\label{fig:schematic}
Schematic of the linear system of equations for osmotic transport.
The off-diagonal elements of Onsager
transport matrix represent the osmotic transport
coefficients, and are the central
quantities obtained in this work. They are computed
from the linear response of the fluxes,
(vector schematically shown on the left)
to an applied external force (vector on the right).
The matrix elements are color-coded according
to the color labeling of the respective flux, i.e., blue, violet and pink for
the elements due to a volumetric flow rate $Q$, an
excess solute flux $\delta J_s$ and an electrical current $I_e$, respectively.
From top to bottom, the external force vector on the right labels a pressure
gradient, a concentration gradient and an electrostatic potential gradient.
The sheets schematically depict a slit geometry, with cations and anions shown in
blue and red. Schematic inspired from Ref.~\citenum{marbach2019osmosis}. }
\end{figure*}


We start with electro-osmosis (EO), the flow generated by an electric field along a nanochannel, whose transport coefficient is illustrated in the top-right element of the matrix in Fig.~\ref{fig:schematic}.
%
The electro-osmotic response is commonly quantified by the so-called zeta potential $\zeta$, which relates the electro-osmotic velocity in the bulk liquid $v_\text{eo}$ to the electric field along the channel $E$ through the Helmholtz-Smoluchowski relation: $v_\text{eo} = -(\eps \zeta / \eta) E$, with $\eps$ and $\eta$ the permittivity and viscosity of the liquid in bulk, respectively.
%
According to Onsager's reciprocal relations, $\zeta$ also quantifies the streaming current density $j_e$ generated by a pressure gradient $-\nabla p$ along the channel: $j_e = -(\varepsilon \zeta / \eta) (-\nabla p)$.
From hydrodynamics equations,
one can relate the $\zeta$-potential to the charge density profile at the interface
(see \textit{e.g.} Refs.~\citenum{huang2007ion,marbach2019osmosis} and the supporting information (SI)):
\begin{equation}
\label{eq:zeta}
    \zeta = - \frac{1}{\eps} \int_0^{\infty}
    (z + b) \rhoe(z) \,\df z ,
\end{equation}
where the charge density
distribution is given by
$\rhoe = q_e\left(n_+ - n_-\right)$, with $n_+$ and
$n_-$ the cation and anion number
densities, respectively,
and $q_e$ the elementary charge.
Note that no assumption was made on the dielectric permittivity of the system: the bulk dielectric permittivity $\eps$ only appears in Eq.~\eqref{eq:zeta} through the Helmholtz-Smoluchowski definition of $\zeta$.

We move on to introduce diffusio-osmosis (DO), the flow generated by a gradient of salt concentration along the channel\cite{anderson1989colloid} (see Fig.~\ref{fig:schematic}). Diffusio-osmosis can be quantified by the so-called diffusio-osmotic mobility $D_\text{DO}$, which relates the diffusio-osmotic velocity in the bulk liquid $v_\text{do}$ to the gradient of salt concentration $\ns$: $v_\text{do} = D_\text{DO} (-\nabla \ns / \ns)$.
Employing again Onsager's reciprocal relations, $D_\text{DO}$ also quantifies the streaming excess solute flux density $\delta j_s$
generated by a pressure gradient
along the channel,
see details in Ref.~\citenum{Ajdari2006} and in the SI.


From hydrodynamics equations, one can relate $D_\text{DO}$ to the ionic density profiles $n_\pm(z)$ and to the water density profile $n_\text{w}(z)$ (normalized by its bulk value $n_\text{w}^\text{b}$) at the interface (see the SI):
\begin{multline}\label{eq:D_DO2}
    D_\text{DO} = \frac{\kt}{\eta} \int_0^{\infty} \left( z + b \right) \times \\ \left\{ n_+(z) + n_-(z) - 2 \ns \frac{n_\text{w}(z)}{n_\text{w}^\text{b}} \right\} \, \mathrm{d}z ,
\end{multline}
with $k_\text{B}T$ the thermal energy.
The integral expression for  $D_\mathrm{DO}$ in Eq.~\eqref{eq:D_DO2} is a first
central result of this work (see the derivation in the SI).
In particular, the contribution of the water density profile, which is usually ignored in the theoretical expressions of $D_\mathrm{DO}$ \cite{Mouterde2018,marbach2019osmosis}, makes a key difference, as ignoring it leads to a spurious negative contribution to the integral in the vacuum-like
region between the first water layer and the surface.
%
Note that an additional flow can be generated under a salt concentration gradient: for salt ions with an asymmetric diffusivity, a so-called diffusion electric field $E_0$ appears to avoid charge separation, creating an electro-osmotic flow, which adds to the intrinsic diffusio-osmotic flow\cite{anderson1989colloid,Lee2014b}. However, as discussed in the SI (see Fig.~S3), this
electro-osmotic component is negligible
in the systems considered here.


Finally, a gradient of salt concentration along the channel also generates an electric current, called diffusio-osmotic current (see Fig.~\ref{fig:schematic}), which is proportional to the perimeter of the channel cross section $P$. In order to quantify the intrinsic response of the liquid-solid interface, independently of the channel geometry, we therefore define the so-called diffusio-osmotic conductivity $K_\text{osm}$, which relates the diffusio-osmotic current generated per unit length of the channel circumference, $I_e/P$, to the gradient of salt concentration $\ns$: $I_e/P = K_\text{osm} (-\nabla \ns / \ns)$; note that slightly different definitions can be found in the literature\cite{Siria2013,marbach2019osmosis}.
According to Onsagers' reciprocal relations, $K_\text{osm}$ also quantifies the excess solute flux
generated by an electric field along the channel, see details in the SI.
%
From hydrodynamics equations, and assuming a homogeneous dielectric permittivity (this assumption will be discussed later in the article),
one can express $K_\text{osm}$ as:
\begin{multline}\label{eq:K_osm2}
K_\text{osm} = \frac{\kt q_e}{4 \pi \lB \eta} \int_0^{\infty}  \left[ \phi(z) - \phis -  \frac{2 \, \sgn(\Sigma) b}{\lGC} \right] \\ \times
\left\{ n_+(z) + n_-(z) - 2 \ns \frac{n_\text{w}(z)}{n_\text{w}^\text{b}} \right\} \,\mathrm{d}z,
\end{multline}
using the same notations as for Eq.~\eqref{eq:D_DO2}, and with $q_e$ the absolute ionic charge, $\phi(z) = q_e V(z) / (k_\text{B}T)$ the reduced electrostatic potential, $\phis$ its value at the surface, $\lB = q_e^2/(4\pi\eps\kt)$ the Bjerrum length (at which the
electrostatic interaction between two ions is comparable to the thermal energy),
and $\lGC = q_e/(2\pi \lB |\Sigma|)$ the Gouy-Chapman length\cite{Herrero2021}.
We remark that the expression for $K_\text{osm}$ shown in Eq.~\eqref{eq:K_osm2}
is the second important result of this work (see the SI for a complete
derivation).

The ions' density and the  electrostatic potential profiles
appearing in Eqs.~(\ref{eq:zeta}-\ref{eq:K_osm2})
are determined from the solution of the
Poisson-Boltzmann equation \cite{Hunter2001,Schoch2008},
which is used to describe the EDL near
electrified interfaces and is modified to include
the free energy of ion adsorption \cite{schwierz2010reversed,luo2006ion},
here computed from first principles simulations:
\begin{multline}\label{eq:mPB}
    \mathrm{d}^2_z \phi(z) = - 4 \pi \lB \left[ n_+(z) - n_-(z) \right] \\
    =- 4 \pi \lB  \ns \left[ e^{-\phi(z) - g_+ (z)} - e^{ \phi(z) - g_- (z) } \right],
\end{multline}
where the dimensionless free energies of ion adsorption
$g_{\pm}(z)= \Delta G_{\pm}(z) /(\kt)$ are the key terms that distinguish
Eq.~\eqref{eq:mPB} from the standard PB description of the EDL,
and which importantly enable the possibility for non-zero solutions of Eq.~\eqref{eq:mPB} even in the absence of charged surfaces,
as it is the case in our
aqueous graphene and hBN interfaces.
%%%UNCOMMENT FOR THE FINAL SUBMISSION TO ACS NANO
% Although the form of Eq.~\eqref{eq:mPB} assumes a constant value of the dielectric constant
% $\varepsilon$,  we have also considered
% its possible spatial
% dependence at the interface\cite{Bonthuis2013,rezaei2021interfacial} using a step model\cite{schwierz2010reversed,huang2007ion}
% and we have computed the transport coefficients
% within this model in the SI (see Fig.~S4).
%%%COMMENT THE SENTENCE BELOW FOR SUBMISSION TO ACS NANO. THE SENTENCE BELOW IS GOOD FOR ARXIV
Although
the form of Eq. (4) assumes a constant value
of the dielectric permittivity $\varepsilon$, we have also
considered a step model of
the dielectric constant\cite{schwierz2010reversed,huang2007ion} and we have computed the transport coefficients within this model in the SI (see Fig.~S4).
%
Whereas the electro-osmotic
and diffusio-osmotic coefficients are
not affected by the particular choice
made for $\varepsilon$, the
magnitude of the diffusio-osmotic
conductivity is altered depending
on the  model used for the dielectric constant;
still, the scaling as a function of concentration  and the change of sign remain the same, thus not
affecting the conclusion of our work (see the SI).

\subsection{Ion adsorption and water density oscillations
from first principles}
Thus, we start to discuss our simulation results by
presenting the free energy profiles
of I$^{-}$ and K$^{+}$ and the water density
profiles on graphene and hBN.
Figs.~\ref{fig:free_energy}(a) and (b)
display the free energy of ion adsorption
obtained from our \textit{ab initio} umbrella sampling
simulations. Significant ion- and surface-specific
adsorption can be observed,
which are limited to a
region of about 1\,nm from the sheets.
Further, the free energy of adsorption of the K$^+$
ion is essentially the same on
graphene and hBN, and whilst the signature of a local minimum appears
at a height of about 0.4\,nm from the sheets, K$^+$ is clearly more stable
in the bulk water region. On the other hand,
the free energy profile of I$^-$ exhibits pronounced differences
on graphene and hBN: a global minimum of about
$-0.08$ eV and $-0.06$\,eV is observed on graphene and hBN, respectively,
where I$^-$ is physisorbed at a height of about 0.4\,nm on both sheets.
Direct anion-substrate interactions due to van der Waals
dispersion forces are likely responsible for the
observed minimum, as reported in previous
FFMD simulations and second harmonic generation experiments
performed on closely related systems.\cite{mccaffrey2017mechanism}
Interestingly, an energy barrier of
about 0.07\,eV is observed for I$^-$
on hBN between the adsorption minimum at a height of 0.4\,nm
and the bulk water region at about 1\,nm from the sheet.
On graphene instead, a free energy barrier is not
observed. Analysis of the dipole orientation of water at the
two interfaces (reported in the SI in Fig.~S2) suggests that subtle
differences in the water dipole orientation
on the two sheets may be
responsible for the distinct features of the
free energy profile of I$^{-}$
on graphene and on hBN.
While some changes in the water dipole orientations on the
two sheets are noticed,
the water spatial density distribution
is remarkably similar on graphene and hBN and
presents strong oscillations
that are gradually suppressed above 1.0-1.2\,nm
from the surface, as highlighted in previous work\cite{tocci2014friction}.
In Eq.~\eqref{eq:mPB},
the ion density distributions, the
electrostatic potential profile $\phi(z)$
and its value at the sheets
$\phis$ depend non-linearly
on the bulk salt concentration $\ns$
through the free energy profiles shown in
Fig.~\ref{fig:free_energy}(a) and (b).
Both the free energy profiles of
the ions adsorbed on the sheets
and the water density profile $n_\text{w}(z)$
enter the transport integral expressions (Eqs.~(\ref{eq:zeta}-\ref{eq:K_osm2}))
and are therefore key to
understand osmotic transport at the interface.

\begin{figure}
\includegraphics[width=0.42\textwidth]{free_energy_vs_height.png}
\caption{\label{fig:free_energy}
Free energy of I$^-$ and K$^+$ adsorption
on (a) graphene and (b) hBN as a function of the height from the sheets,
and (c) water density profile on graphene and hBN.
The inset in (a) is a representative snapshot
of the water film simulation with a
iodide ion (blue) adsorbed on graphene (cyan).
The free energy profile of K$^+$
is similar on graphene and hBN and
does not show an adsorption minimum, in contrast with I$^-$ adsorption, which
reveals a minimum near the sheets
and a barrier on hBN, but not on graphene.}
\end{figure}



\subsection{Concentration dependence of osmotic transport}
The observed ion- and surface-specific
adsorption and the layering  of the water density
profiles have pronounced effects
on the osmotic transport coefficients.
Slippage can also enhance osmotic
transport dramatically\cite{Ajdari2006}, as it can be
evinced from Eqs.~(\ref{eq:zeta}--\ref{eq:K_osm2}).
In this work, we used slip length values
taken from previous AIMD results\cite{tocci2020nanoscale}.
Water flows significantly faster
on graphene ($b=19.6$\,nm) than on hBN ($b=4.0$\,nm),
and we wish to explore the consequences this
bears for osmotic transport.

The osmotic transport coefficients
are displayed in Fig.~\ref{fig:transport_coefficients}:
Each transport coefficient follows
a different asymptotic behaviour
as a function of salt concentration
and remarkable changes are noticed also
between graphene and hBN.
%
\begin{figure*}[thb!]
\includegraphics[width=1\textwidth]{Transport_coefficients.png}
\caption{\label{fig:transport_coefficients}
Molecular representation and concentration dependent
scaling of osmotic transport at the aqueous
graphene and hBN interfaces.
Schematic of the electro-osmotic velocity
profile $v_\mathrm{eo}(z)$ arising from
an electric field $E$ (a) and absolute
value of the $\zeta$ potential as a function of
salt concentration (b);
Schematic of the diffusio-osmotic velocity
profile $v_\mathrm{do}(z)$ in response to a concentration gradient $-k_\mathrm{B}T (\nabla\ns)/\ns$ (c) and absolute
value of the diffusio-osmotic coefficient $|D_\mathrm{DO}|$ (d);
Representation of the
diffusio-osmotic current $I_e$ arising from a concentration gradient
$-k_\mathrm{B}T (\nabla\ns)/\ns$ (e), and
absolute value of the diffusio-osmotic conductivity $|K_\mathrm{osm}|$ (f).
The transport coefficients display different
 scaling behaviours (see dashed lines).
The symbols are obtained from the numerical
integration of Eqs.~(\ref{eq:zeta}-\ref{eq:K_osm2}),
whereas the solid and dotted lines are from the effective surface charge model.
In (d) and (f) the arrows point to a
sign change, whereby $D_\text{DO}$
and $K_\text{osm}$
are negative on hBN at concentrations $\gtrsim 1$\,M, as indicated also
by the empty symbols and dotted line in (d).
In (a) $v_\mathrm{eo}(z)$ is not enhanced by slippage,
opposite to $v_\mathrm{do}(z)$ in (c) and (e), see text for details.}
\end{figure*}
%
Fig.~\ref{fig:transport_coefficients}(a)
is a schematic representation of electro-osmosis
for the systems considered here, and Fig.~\ref{fig:transport_coefficients}(b)
shows the absolute value of the
$\zeta$-potential as a function of
concentration. The absolute value of the $\zeta$ potential
increases from about $0.6$\,mV to
$40$\,mV on graphene and
from about $0.1$\,mV to
$20$\,mV on hBN.
A non-zero value of the $\zeta$
potential even in absence of a surface
charge highlights the
role of ion-specific adsorption on electro-osmosis,
as reported in the past by
electro-phoretic experiments and FFMD
simulations\cite{huang2007ion,petrache2006salt}.
It is also worth noting that
the slip contribution in Eq.~\eqref{eq:zeta} writes $-b/\eps \int_0^\infty \mathrm{d}z \rhoe = b \Sigma / \eps$, so that it vanishes for neutral surfaces\cite{huang2007ion}.
Thus, based on the form of Eq.~\eqref{eq:zeta},
it can be evinced that
both the magnitude and the different concentration dependence
of $\zeta$ on graphene and hBN solely arise from
differences in the free
energy profile of adsorption
of I$^{-}$ on the two sheets,
given that the free energy of K$^{+}$
is very similar on graphene and hBN.

A schematic representation of diffusio-osmotic
flow is illustrated in
Fig.~\ref{fig:transport_coefficients}(c) and
the concentration dependence of the
diffusio-osmotic coefficient $D_{\text{DO}}$
is presented in Fig.~\ref{fig:transport_coefficients}(d).
Whereas $D_\mathrm{DO}$ scales linearly
on graphene at all concentrations, deviations from
a linear asymptotics is noticed
on hBN already around $10^{-1}$\,M, and  above
 $1$\,M $D_\mathrm{DO}$ becomes negative (a positive value of $D_\mathrm{DO}$ indicates that the
diffusio-osmotic flow proceeds from high to low
concentration and vice-versa for negative $D_\mathrm{DO}$). Noting also that the K$^{+}$
free energy profiles
and the water spatial distribution
are very similar on graphene and hBN,
it is clear that $D_\mathrm{DO}$
changes sign on hBN because of the
different free energy of adsorption of I$^{-}$
from graphene.
We note that a liquid flow that proceeds
towards a larger concentration
has been observed before for neutral solutes,
which only interact specifically with the surfaces.\cite{Lee2014b,Lee2017}.
Opposite to electro-osmotic transport,
diffusio-osmotic transport is amplified
by slippage even in the absence of a surface
charge (see Eq.~\ref{eq:D_DO2}) and
a strong slip-induced enhancement of
the diffusio-osmotic flow
has been reported before\cite{Ajdari2006}.
 The larger slip-length, along with the
  absence of an adsorption barrier
of I$^{-}$ are the two main reasons
why $|D_\mathrm{DO}|$ is larger on graphene than on hBN.

Finally, a schematic
of the diffusio-osmotic
current mechanism is presented in
Fig.~\ref{fig:transport_coefficients}(e),
and the  scaling behaviour of the diffusio-osmotic
conductivity is shown in Fig.~\ref{fig:transport_coefficients}(e).
Below $10^{-1}$\,M, the diffusio-osmotic conductivity
exhibits a different scaling behaviour
with salt concentration on graphene
and hBN, in contrast to what was observed in the
case of the diffusio-osmotic coefficient.
On graphene, $K_\mathrm{osm}$
scales more slowly than on hBN, but it
remains positive at all concentrations considered.
On hBN on the other hand, a maximum in $K_\mathrm{osm}$ is observed
just below 1\,M, above which an abrupt sign change is
observed, similar to what observed for $D_\mathrm{DO}$.
Finally, in Eq.~\eqref{eq:K_osm2},  we note that in contrast to
$D_\mathrm{DO}$, slippage does not contribute to $K_\mathrm{osm}$
in the absence of a surface charge, for which $1/\lGC \propto \Sigma = 0$.
This is because slip shifts the DO velocity profile by a constant amount $v_\text{slip}$, so that the corresponding electrical current writes: $I_e \propto v_\text{slip} \int_0^\infty \mathrm{d}z \rhoe \propto v_\text{slip} (-\Sigma)$, and vanishes for neutral surfaces.

\subsection{Scaling laws of the osmotic transport coefficients}
To rationalize the different scaling properties,
we introduce an effective surface charge (ESC) model
of the EDL. The ESC model is discussed in
full detail in the SI, but here we
present the essential idea.
%
In this model, the structure of the EDL
is mapped to a standard PB description,
where the effects associated to ion-specific
adsorption are treated as surface terms by introducing
an effective surface charge $\Sigma_\mathrm{eff}$.
$\Sigma_\text{eff}$ is defined as
$\Sigma_\text{eff} = q_e \ns \left (e^{-\phi_\mathrm{s}} K_+ - e^{\phi_\mathrm{s}}K_- \right)$ and it
depends on the bulk salt concentration $\ns$,
on the surface potential $\phi_\mathrm{s}$ and on the characteristic
lengths $K_{\pm}$, which quantify the excess or
depletion of ions near the sheets (see the definition in Table~\ref{table:1}).
It will be shown that the
density oscillations of interfacial
water also enter this modified formulation
and are relevant to capture
diffusio-osmotic transport.
The key to this simplified description is that
the region where ions and water interact specifically
with the sheets is very thin compared to the EDL.
As such, the ESC model is strictly
valid when there is a net separation between the
water and ion adsorption length-scales
and those of the EDL, given by the Debye length and
the effective Gouy-Chapman length, which we define here in terms
of the effective surface charge
$\lGC^\text{eff}=q_e/\left(2\pi\lB |\Sigma_\mathrm{eff}|\right)$.
In the following, we will apply the ESC model to
Eqs.~(\ref{eq:zeta}-\ref{eq:K_osm2})
to rationalize the scaling behaviours observed in
Fig.~\ref{fig:transport_coefficients}.

Starting from electro-osmosis, the $\zeta$-potential
can be expressed within the ESC model, in the limit of small reduced surface potentials $\phis \ll 1$ (see the SI), as:
\begin{equation}
\label{eq:zeta_mGC}
     \zeta \approx \frac{\kt}{q_e} \phis \approx \frac{\kt}{q_e} \cdot \frac{   K_+ - K_- }{  2\lambda_D + K_+ + K_- }.
\end{equation}
The results of this equation are shown as solid lines in
Fig.~\ref{fig:transport_coefficients}(b).
To understand the two different limits observed in
the figure, it is instrumental to inspect
the values of the ion-specific length-scales
$K_+$ and $K_-$ entering the equation, which are
listed in Table~\ref{table:1}.
For both graphene and hBN, $K_+$ is negative
and indicates a net depletion of cations near the sheets,
whereas $K_-$ is positive and indicates a net accumulation
of anions. Noting also that for hBN,
$|K_+ + K_-| \ll \lambda_\mathrm D$ at all
concentrations,  Eq.~\eqref{eq:zeta_mGC}
simplifies to $\zeta \approx \kt (K_+ - K_-)/(2\lambda_D q_e)$, such that
$\zeta$ scales as the inverse of the Debye length, or equivalently
as  the square-root of the salt concentration, \textit{i.e.}, $\zeta \sim 1/\lambda_\mathrm D \sim \sqrt{\ns}$.
For graphene on the other hand,
departure from $\zeta\sim\sqrt{\ns}$ is visible
already at a concentration above $10^{-3}$\,M
because the term $|K_++K_-|$ becomes comparable to $\lambda_\mathrm D$.
Therefore, approximately above
$10^{-3}$\,M the scaling of $\zeta$ on graphene is better
captured by the full Eq.~\eqref{eq:zeta_mGC}.
Beyond $10^{-1}$\,M for graphene and beyond 1\,M for hBN,
the prediction from Eq.~\eqref{eq:zeta_mGC} deviates
from the numerically integrated results
because the surface potential
becomes of the order of the thermal voltage
(\textit{i.e.} $\phis \gtrsim 1$).

%
\begin{table*}[t!]
  \centering
\begin{tabular}{ P{3.5cm}P{5.0cm}P{3.5cm}P{3.5cm}}
\hline
\hline
  \centering
 lengths [nm]&definition&graphene&hBN\Tstrut\Bstrut\\
 \hline
 \Tstrut\Bstrut
   $\bm{b}$&  &$\bm{+19.6}$&$\bm{+4.0}$\Tstrut\Bstrut\\
   $K_+$&$\int_0^{\infty}  [e^{-g_+(z)}-1]\mathrm{d}z$&$-0.229$&$-0.207$\Tstrut\Bstrut\\
   $\bm{K_-}$&$\int_0^{\infty}  [e^{-g_-(z)}-1]\mathrm{d}z$&$\bm{+1.974}$&$\bm{+0.177}$\Tstrut\Bstrut\\
   $K_w$&$\int_0^{\infty}  [ n_w(z)/n_w^{b} -1]\mathrm{d}z$&$-0.179$&$-0.165$\Tstrut\Bstrut\\
   $L_+$&$K_+^{-1}\int_0^{\infty} z[e^{-g_+(z)}-1] \mathrm{d}z$&$-0.363$&$-0.442$\Tstrut\Bstrut\\
      $\bm{L_-}$&$K_-^{-1}\int_0^{\infty} z[e^{-g_-(z)}-1] \mathrm{d}z$&$\bm{+0.459}$&$\bm{-0.099}$\Tstrut\Bstrut\\
   $L_w$&$K_w^{-1}\int_0^{\infty}  z[ n_w(z)/n_w^{b} -1]\mathrm{d}z$&$-0.031$&$-0.027$\Tstrut\Bstrut\\
\hline
\hline
\end{tabular}
\caption{Slip length $(b)$ and length-scales characteristic
of cation-specific (with subscripts ``$+$''),
anion-specific adsorption (with subscripts ``$-$'') and water
density oscillations (with subscript ``$w$'')
at the aqueous graphene and hBN interfaces,
along with their definitions.
The slip length and the anion length-scales are
in bold to highlight the stark differences between graphene and hBN.}
\label{table:1}
\end{table*}
Application of the ESC model to diffusio-osmosis
(Eq.~\eqref{eq:D_DO2}) yields the following
relation for $D_\text{DO}$ (derived in the SI):
\begin{multline}
\label{eq:D_DO_mGC}
    D_\text{DO}
    \approx \frac{\kt}{2\pi\lB\eta} \left\{ -\ln\left(1-\gamma^2\right) + \frac{b|\gamma|}{\lGC^\text{eff}} + \right. \\
    \left. \frac{1}{4\debye^2} \left[ e^{-\phis} K_+ L_+ + e^{\phis} K_- L_- -2 K_\text{w} L_\text{w} + \right. \right. \\  \left. b \left( e^{-\phis} K_+ + e^{\phis} K_- -2 K_\text{w} \right) \right] \bigg \}.
\end{multline}
%
The first two terms depending on $\gamma$, with
$\gamma = \tanh (\phis/4)$,
result from a calculation of
the diffusio-osmotic flow according to the standard PB description of
the EDL\cite{anderson1989colloid,Mouterde2018}
and arise from the region in the EDL beyond the adsorption layer, whereas
the remaining terms involving the ion length-scales
($K_\pm$ and $L_\pm$) and the water length-scales
($K_w$ and $L_w$) arise from specific
interactions within the adsorption layer.
Along with $K_{\pm}$, the additional ion-specific
length-scales $L_{\pm}$ contributing to Eq.~\eqref{eq:D_DO_mGC}
(see definition in Table~\ref{table:1})
represent the characteristic thicknesses over which cations and anions adsorb.
The water-specific length-scale $K_w$ is characteristic of
a net accumulation or depletion of water near the
sheets, while $L_w$ represents
the characteristic size
over which water accumulates/depletes
at the aqueous interface (see definition in Table~\ref{table:1}).
By analysing the relevant terms in
Eq.~\eqref{eq:D_DO_mGC}, we can
interpret the scaling behaviour observed in
Fig.~\ref{fig:transport_coefficients}(c),
while pointing out that in the equation
$\lambda_\text{D}$, $l^{\text{eff}}_{\text{GC}}$,
$\gamma$ and $\phis$ all depend on concentration.
First, the
terms depending on $\gamma$ are
negligible compared to the adsorption
layer terms at small surface
potentials ($\phis <1$)
and at concentrations
below $\sim 10^{-1}$\,M
because to leading order
$\gamma \sim \phis$ whereas $e^{\pm\phis} \sim 1$.
Therefore, at low concentrations
a linear dependence of $D_\text{DO}$
can be readily understood based on the dependence
of the adsorption term proportional
to $\lambda_\mathrm D^{-2}$
 by realizing that $D_\mathrm{DO} \sim \lambda_\mathrm D^{-2} \sim \ns$.
%
Also, examining the characteristic
length-scales
listed in Table~\ref{table:1}, one notices  that
the terms involving $e^{\mp \phis }K_{\pm} L_{\pm}$ and
$2 K_{\text{w}} L_{\text{w}}$  can be neglected
compared to the factor proportional to $b$
 because the slip length is much larger than
 $|L_\pm|$ and $|L_{w}|$.
As such, $D_\mathrm{DO}$ is enhanced by slippage through
the slip length $b$ and the magnitude of $D_\mathrm{DO}$
is larger on graphene than on hBN partly
because of the larger slip length
in the former material.
As the concentration is
increased around 1\,M the PB term and in
particular the slip contribution $b|\gamma|/\lGC^{\text{eff}}$
can no longer be neglected.
Interestingly, competing effects between
the $b|\gamma|/\lGC^{\text{eff}}$ term, the ions' adsorption
term and the molecular-scale oscillations
 of water can give rise to a sign reversal
 in $D_\text{DO}$ at a critical
 concentration.
Reversal of diffusio-osmotic flow is
indeed observed on hBN,
but not on graphene, due to the
differences in the slip length and in the
ion adsorption length-scales between the two materials
(see Table \ref{table:1}).
On graphene,
the ion-specific adsorption terms
proportional to  $e^{-\phis}K_+ + e^{\phis}K_+$ and
the term proportional to $b|\gamma|/\lGC^{\text{eff}}$
dominate over the water contribution proportional to
$2K_w$ at all concentrations
and the diffusio-osmotic
flow always proceeds in the direction from high to low
concentrations. On hBN, instead, above a
critical concentration of about 1\,M
the water term becomes the dominant contribution
and a flow reversal is observed.
A more detailed analysis on the scaling behaviour
of $D_{\text{DO}}$ is provided in Fig.~S5
in the SI, where the diffusio-osmotic coefficient has been explicitly decomposed into the standard
PB contribution and the adsorption layer contribution.

Finally, an approximate expression for the diffusio-osmotic conductivity
$K_\mathrm{osm}$ is obtained to understand the scaling behaviour observed
in Fig.~\ref{fig:transport_coefficients}(c) as:
\begin{multline}
\label{eq:kosm_mGC}
    K_\mathrm{osm} \approx - \frac{\kt \Sigma_\text{eff}}{2 \pi \lB \eta} \times  \bigg ( 1  -\frac{\text{asinh} (\chi)} {\chi} \\
  + \frac{ e^{-\phis} \tilde{K}_+\tilde{L}_+   + e^{\phis} \tilde{K}_-\tilde{L}_-   -2 \tilde{K}_\text{w} \tilde{L}_\text{w}  }{4 \lambda_\mathrm D^2} \bigg ).
\end{multline}
The term $1-\text{asinh} (\chi)/\chi$
arises from the region in the EDL
beyond the adsorption layer,
with $\chi=\debye/\lGC^\text{eff}$\cite{Siria2013,Mouterde2018}.
The adsorption layer also contributes to
$K_{\text{osm}}$ through the ion-specific length-scales $\tilde{K}_{\pm}$ and $\tilde{L}_{\pm}$, as well as $\tilde{K}_\text{w}$ and $\tilde{L}_\text{w}$.
Their interpretation is
analogous to the characteristic lengths
reported in Table \ref{table:1},
but we refer the reader to the SI (Table~S1) for
their definition and numerical values.



At concentrations between $10^{-4}$\,M and $10^{-1}$\,M,
 Eq.~\eqref{eq:kosm_mGC} reproduces
 the asymptotic
 behaviour of the diffusio-osmotic conductivity,
 which scales as $K_\mathrm{osm} \sim n_\mathrm s ^{p}$,
 where $p\approx 2$ for graphene and
 $p\approx 2.5$ for hBN.
 We discuss each term separately to explain
 the scaling behaviour observed in
 Fig.~\ref{fig:transport_coefficients}(f).
First of all, to leading order, the effective surface charge
scales linearly with the salt concentration
(\textit{i.e.}, $\Sigma_\mathrm{eff}\sim n_\mathrm{s}$). Secondly,
noting that $\lGC^\text{eff} \propto \Sigma_\mathrm{eff}^{-1}$,
the contribution to the diffusio-osmotic
conductivity arising from the region beyond
 the adsorption layer
 also scales linearly with salt concentration
 (\textit{i.e.}, $1-\text{asinh} (\chi)/\chi \sim n_\mathrm{s}$).
Thirdly, the term arising from the
 adsorption layer exhibits
 a first obvious linear dependence on the
 concentration through the Debye length
 since $\lambda_\mathrm{D}^{-2} \propto n_\mathrm{s}$. This is indeed
 observed for graphene,
 such that overall the diffusio-osmotic
 conductivity scales as $K_\mathrm{osm} \sim n_\mathrm s ^{2}$.
 A second more subtle dependence on salt concentration however, is observed on hBN,
 where the adsorption layer term in  Eq.~\eqref{eq:kosm_mGC} changes sign
around $5\times 10^{-3}$ M (see Fig.~S6 in the SI).
The contribution  to  the diffusio-osmotic conductivity coming from the adsorption layer
term and the $1-\text{asinh} (\chi)/\chi$
term can be either  suppressed or enhanced
depending on the sign of the former, and ultimately results into a scaling of
$K_\mathrm{osm} \sim \ns^{2.5}$.
%
Further, similarly to what was observed in the case of the diffusio-osmotic
flow in Fig.~\ref{fig:transport_coefficients}(d),
at concentrations beyond 1\,M a reversal in the
diffusio-osmotic current is  observed in the
numerical calculations for hBN, because the water
contribution dominates over
its ion counterpart, whereas on graphene a sign change is
not observed because the ion-specific adsorption
contribution dominates at
all considered concentrations.
We note that the assumption of a linear
electrostatic potential difference $\phi(z) -\phis$
made while deriving Eq.~\eqref{eq:kosm_mGC} breaks down
at large concentrations, thus Eq.~\eqref{eq:kosm_mGC}
fails to predict the sign change on hBN at around 1 M.
\section{Discussion}
In this section, we
discuss the significance of our
work in connection with theoretical and experimental results
on the structure of the EDL and of osmotic transport in nanofluidics.
%
In our approach to calculate the
spatial distribution of ions in the EDL,
the electrostatic potential and the
free energy profile of ion adsorption are
determined, respectively, from the
solution of the mPB equation and from our
enhanced sampling simulations.
This framework has been introduced in
the past to study the structure of the EDL
at liquid/liquid interfaces\cite{luo2006ion} and to
shed light on the Hofmeister
series at hydrophobic and hydrophilic
surfaces for different surface
charges with FFMD simulations\cite{schwierz2010reversed}.
However, deviations from a modified PB
description of the electrostatic
potential appear at concentrations of the order of 1 M\cite{gonella2021water},
and it would be interesting
to incorporate
more advanced theories into our framework
to ameliorate the deficiencies
underlying the mPB theory
at such concentrations\cite{duignan2021toward,hartel2015fundamental,netz2001electrostatistics,kardar1999friction}.

Despite their substantial computational cost,
our \textit{ab initio} simulations
have also revealed that ion adsorption
can be surface-specific, as illustrated
by the differences in the free
energy of I$^-$ on the two sheets but not
of K$^+$. In particular, the presence of a
free energy barrier of I$^-$ on
hBN and not on graphene can be ascribed to
 differences in the water dipole orientation
 on the two sheets.
Deep UV second harmonic generation,
already used to investigate the free energy
of adsorption of SCN$^-$ on
graphene\cite{mccaffrey2017mechanism},
is an ideal technique to explore
ion adsorption on other substrates, including hBN.
Also, X-ray photoelectron spectroscopy
would be instrumental to probe the shape of the electrostatic potential profile
at aqueous electrified interfaces\cite{favaro2016unravelling}.

A central contribution of this paper is to have provided
a unified framework of osmotic transport that can be computed
from the structure of water and ions at aqueous interfaces,
and which transport coefficients can be probed experimentally. Measurements of osmotic transport of KI
solutions across graphene or hBN have not appeared yet, but
we think that they are already possible,
given that measurements of ion transport across
{\AA}-size slits have been reported,
and that diffusio-osmosis of several types of salts across
silica surfaces has also been probed.
It would be desirable that such experiments
be performed at the point of zero charge,
since differences in pH lead
to variations in the surface charge\cite{secchi2016scaling,Grosjean2019,Siria2013,Mouterde2018} and thus would likely alter the scaling behaviour
of the transport coefficients with concentration.
For instance,
a different scaling behaviour of the electrical
conductivity on salt concentration
has been observed in carbon nanotubes. This has been captured by
charge regulation models\cite{secchi2016scaling, biesheuvel2016analysis}
which are, however,
phenomenologically different
from the mechanisms described here, as we remark that
the sheets are not charged.

In absence of experiments on KI solutions,
we can connect our work to recent streaming-voltage experiments of KCl solutions across graphitic
\AA-size slits\cite{mouterde2019molecular}. Such experiments
did not present a clear dependence of the $\zeta$-potential
as a function of concentration
in a range between $ 10^{-3}$\,M and $10^{-1}$\,M
and extracted a value for $\zeta$ at least 10 times larger than that
shown in Fig.\ref{fig:transport_coefficients} (a).
Possible explanations are that, although ion-specific effects
of a KCl solution are less pronounced than a KI solution,
under the extreme confinement regime probed
in such experiments the Stokes equation of
hydrodynamics breaks down. Additionally,
ion exclusion at the entrance of the
membranes might play an important role in \AA-scale slits.

Concerning diffusio-osmosis,
we discuss our results in connection with
measurements performed on silica surfaces,
where the diffusio-osmotic coefficient has
been measured for several types of aqueous electrolytes, including KI\cite{Lee2014b}.
The reported diffusio-osmotic coefficient of KI on silica
is $D_\mathrm{DO} \approx 250 \mu\text{m}^2/s$,
independent of concentration.
Instead, we observe approximately a linear dependence
at all concentrations for graphene, and
above $10^{-1}$\,M $D_\mathrm{DO}$ reaches values beyond
$10^{4}\,\mu\text{m}^2/\text{s}$.
The observed enhancement of
the diffusio-osmotic flow
of KI on graphene compared to
silica can be largely attributed to the larger slip length of graphene.
Although the \textit{ab initio} values
for the slip length on graphene and
hBN\cite{tocci2020nanoscale}
compare well to recent flow  experiments performed on graphene slits\cite{xie2018fast},
as well as on hBN nanotubes and on carbon nanotubes
with a large radius $R$ of 50\,nm\cite{secchi2016massive},
the slip length for carbon nanotubes tubes
with a smaller radius $R=20$\,nm
is about one order of magnitude
larger than that of graphene, $b\sim 200$\,nm.
Provided that the structure of the EDL
would be only modestly affected
by such a large nanotube radius, one can extrapolate
the value of the diffusio-osmotic
coefficient that would be obtained for carbon
nanotubes with $R=20$\,nm from our results on graphene.
Doing so would result in approximately a $10$-fold
enhancement in the diffusio-osmotic
coefficient because of the much larger slip length.
Additionally, compared to silica,
where the diffusio-osmotic flow proceeds
from high to low concentrations,
we observe a flow reversal on hBN.


Measurements of the diffusio-osmotic current of KI solutions on graphene or hBN
have also not been reported. As such, we
connect our results for $K_\mathrm{osm}$
to osmotic energy conversion experiments of KCl solutions
flowing across single-pore BN
nanotubes\cite{Siria2013}.
On BN nanotubes, the diffusio-osmotic conductivity
$K^\mathrm{BNNT}_\mathrm{osm}$ has
been extracted from the ratio between the
diffusio-osmotic current and the difference in
the salt concentration at the reservoirs as
$K^\mathrm{BNNT}_\mathrm{osm} = I_\text{osm} /(\Delta \ns /\ns)$.
In order to compare directly to our results, we
normalize the current with respect to the perimeter
$P= 2\pi R$ and the length $L$ of the nanotube, \textit{i.e.}
$K^{\prime \mathrm{BNNT}}_\mathrm{osm} = \frac{I_\text{osm} /P}{\Delta \ns /( \ns L)}$. With an experimental value of the nanotube length   $L=1250$\,nm and of the radius $R=40$\,nm, the equivalent
experimental value for the diffusio-osmotic conductivity is
$K^{\prime \mathrm{BNNT}}_\mathrm{osm} = K^\mathrm{BNNT}_\mathrm{osm} L/(2\pi R) = 0.35 - 0.80$\,nA. These values for the diffusio-osmotic conductivity are much
larger than those computed here (at any salt concentration)
because of the very large surface charge $\Sigma$
that was reported in experiments ($\Sigma \approx 0.1 -1$\,C/m$^2$).
In BN nanotubes, the diffusio-osmotic
conductivity has been found to scale linearly on
the pH of the solution, and thus on the surface charge,
but to be independent of salt concentration\cite{Siria2013}.
Here instead, the diffusio-osmotic conductivity
scales roughly as $K_\mathrm{osm} \sim \ns^{p}$
with an exponent $p\approx 2$ and $p\approx2.5$
for graphene and hBN, respectively, over several decades
of salt concentration. The sign reversal in the
diffusio-osmotic current, as well as
in the diffusio-osmotic flux,
could be particularly relevant in biology.
Similar mechanisms might be at work in
membrane proteins, which could exploit changes
in concentration to regulate charge and solute
transport\cite{van2006claudins,van2003reversal}.
Although the focus here has been devoted to the understanding of
osmotic transport across a single-pore membrane or channel,
further challenges lie ahead of diffusio-osmosis in order to
establish itself as a viable source of renewable energy, in
particular, for what concerns the generation of electricity using
multi-pore systems.\cite{macha20192d,Tong2021,Wang2021}

In conclusion, we have provided a unified description of
osmotic transport by coupling first principles simulations with a
mean field description of the EDL and with the Stokes equation
of hydrodynamics, and have applied this framework to understand
the osmotic transport properties of a
prototypical salt that displays pronounced
ion-specific effects on two-dimensional materials.
Through the mPB equation of the EDL (see Eq.~\eqref{eq:mPB}),
the transport coefficients in
Eqs.~(\ref{eq:zeta}-\ref{eq:K_osm2}) can
be readily computed from the spatial
distribution of ions and water
at the interface.
We have reported on a concentration-dependent
scaling behaviour of the osmotic transport coefficients
at the aqueous graphene and hBN interfaces
and explained it with a model that
accounts for ion-specific adsorption and
water layering in a thin region of the EDL
 of the order of 1\,nm.
The observed scaling, along with the possibility of
diffusio-osmotic flow and
current reversal, may provide additional routes
to further improve osmotic energy conversion.
%
Moreover, it could foster the development of nanofluidic
diodes and sensors and may be instrumental to shed light
on the mechanisms underlying
charge and solute transport across membrane proteins.


%
\section*{Materials and Methods}
\subsection*{Electronic structure and \textit{ab initio} molecular dynamics}
The \textit{ab initio} umbrella sampling
simulations are performed with
the CP2K code \cite{kuhne2020cp2k}, and the
electronic structure problem is solved using DFT
with the optB88-vdW functional \cite{jiri_solids,jiri_molecules}.
The optB88-vdW functional has been applied to investigate slippage
on two-dimensional materials \cite{Joly2016,tocci2020nanoscale}
and it describes the  structure of graphite and bulk hBN
accurately \cite{graziano_vdw_gra_bn}.
Despite the limitations of most density functionals
in describing water, the optB88-vdW functional appears
to be one of the most satisfactory \cite{gillan2016perspective}.
Reference quantum monte carlo calculations
of water monomers adsorbed on graphene and hBN sheets
\cite{brandenburg2019physisorption,al2015communication}
also show that this functional captures the relative stability
of water on different adsorption sites.

The free energy of adsorption of K$^+$ and I$^-$ ions
and the spatial water distribution were calculated from umbrella sampling simulations\cite{torrie1977nonphysical}.
The systems consist of water films containing 400 molecules and approximately 2 nm-thick
placed above a $2.56 \times 2.46$ nm$^2$ graphene and a $2.61 \times 2.51$ hBN nm$^2$ sheet.
Separate umbrella sampling simulations were performed for potassium and iodide
adsorption by restraining each ion at different heights  $z_0$ above the sheets with
a harmonic bias potential $U_b(z,t) = k_b/2 (z(t)-z_0)^2$,  $z(t)$ being the
instantaneous height of the ion above the sheets,
and $k_b = 836.8$ kJ/mol/nm$^2$ the \textit{spring} constant.
A total of 23 umbrella sampling windows were used for each system
and for each window the dynamics were propagated for 40 ps
in the NVT ensemble at 300\,K within the Born-Oppenheimer
approximation, except for I$^{-}$ adsorption on graphene,
for which  dynamics were propagated for additional 30 ps to test
a possible dependence of our results on the length of the
simulations. The free energy was reconstructed using umbrella integration\cite{kastner2011umbrella}.
Further computational details on the  optimization
of the wave-function and on the umbrella sampling
simulations are reported in the SI.
The tools used to perform the analysis and the CP2K input files
to reproduce the main results of the manuscript have been deposited on
GitHub and are available at \url{https://github.com/gabriele16/osmotic_transport_scaling_laws}.


 \begin{acknowledgement}
 GT is supported by the SNSF project PZ00P2\_179964.
 LJ is supported by %the ANR through Project ANR-16-CE06-0004-01 NECtAR, and by
the Institut Universitaire de France.
RM and GT acknowledge funding by the Deutsche Forschungsgemeinschaft (DFG, German Research Foundation) -- 390794421.
 We also thank the Swiss National Supercomputer Centre (CSCS) under PRACE for awarding us access to
 Piz Daint, Switzerland, through projects pr66 and s826.

 \end{acknowledgement}

\begin{suppinfo}
Further computational details and tests
on the structure and dynamics
of water at the interface with
graphene and hBN  from
force field and \textit{ab initio} simulations;
derivation of integral expressions for the transport coefficients;
details of the modified Poisson-Boltzmann description;
details of the effective surface charge model.
\end{suppinfo}

%\bibliography{biblio_osmotic_transport,laurent,robeme}
\documentclass[number,preprint,3p]{elsarticle}

%\RequirePackage{fix-cm}
%
%\documentclass{svjour3}                     % onecolumn (standard format)
%\documentclass[smallcondensed]{svjour3}     % onecolumn (ditto)
%\documentclass[smallextended]{svjour3}       % onecolumn (second format)
%\documentclass[twocolumn]{svjour3}          % twocolumn
%
%\smartqed  % flush right qed marks, e.g. at end of proof
%
%\usepackage{graphicx}


\usepackage{color}
\usepackage{graphicx}
\usepackage{subcaption}
\usepackage{algorithm}
\usepackage{bm}
\usepackage[colorlinks]{hyperref}% Add hyper-ref for equations, figures, etc.
\makeatletter
% ead[url] with a hyperlink
\gdef\urlauthor#1#2{\g@addto@macro\@elsuads{\let\corref\@gobble%
     \def\@@tmp{#1}\raggedright\eadsep
     {\ttfamily\url{\expandafter\strip@prefix\meaning\@@tmp}}\space(#2)%
     \def\eadsep{\unskip,\space}}%
}
% ead with a mailto:
\gdef\emailauthor#1#2{\stepcounter{ead}%
     \g@addto@macro\@elseads{\raggedright%
      \let\corref\@gobble\def\@@tmp{#1}%
      \eadsep{\ttfamily\href{mailto:\expandafter\strip@prefix\meaning\@@tmp}{\expandafter\strip@prefix\meaning\@@tmp}}
      (#2)\def\eadsep{\unskip,\space}}%
}
\makeatother
\usepackage{amssymb}
\usepackage{amsthm}
\usepackage{amsmath}
\usepackage{amssymb}
\usepackage{mathtools}
\usepackage{dsfont}
\usepackage{booktabs}
\usepackage{url}
\usepackage{scrextend}
\usepackage{epstopdf}
\usepackage{float}
\usepackage{tcolorbox}
\usepackage{tablefootnote}
\usepackage[perpage]{footmisc}% Renew the footnote numbering for each page
\usepackage{bbm}
\usepackage{lineno}

\def\r{\mathbb{R}}
\def\rn{\mathbb{R}^n}
\def\defi{\coloneqq}
\def\tr{^\intercal}

\newcommand{\Prob}[1]{\mathbb{P}\left(#1\right)}
\newcommand{\E}[1]{\mathbb{E}\left[#1\right]}
\newcommand{\var}[1]{\mathbb{V}ar\left[#1\right]}
\newcommand{\cov}[2]{\mathbb{C}ov\left[#1\,,#2\right]}
\newcommand{\1}[2]{\mathbb{I}_{#1}\left(#2\right)}
\newcommand{\vect}[1]{\boldsymbol{#1}}

% Design a new environment for breakable algorithm, so that the algorithm can be automatically split to span two pages
\makeatletter
\newenvironment{breakablealgorithm}
{% \begin{breakablealgorithm}
	\begin{center}
		\refstepcounter{algorithm}% New algorithm
		\hrule height.8pt depth0pt \kern2pt% \@fs@pre for \@fs@ruled
		\renewcommand{\caption}[2][\relax]{% Make a new \caption
			{\raggedright\textbf{\ALG@name~\thealgorithm} ##2\par}%
			\ifx\relax##1\relax % #1 is \relax
			\addcontentsline{loa}{algorithm}{\protect\numberline{\thealgorithm}##2}%
			\else % #1 is not \relax
			\addcontentsline{loa}{algorithm}{\protect\numberline{\thealgorithm}##1}%
			\fi
			\kern2pt\hrule\kern2pt
		}
	}{% \end{breakablealgorithm}
		\kern2pt\hrule\relax% \@fs@post for \@fs@ruled
	\end{center}
}
\makeatother
%\biboptions{square}
\bibliographystyle{unsrt}
%\bibliographystyle{abbrv}
%\bibliographystyle{elsarticle-num}
\biboptions{numbers,sort&compress,square}
\journal{arXiv}
\linespread{1.25}


\begin{document}
%\linenumbers
	\begin{frontmatter}
		\renewcommand{\thefootnote}{\fnsymbol{footnote}}
		\title{A physics and data co-driven surrogate modeling method \\ for high-dimensional rare event simulation}
		\author[1]{Jianhua Xian}
		\author[1]{Ziqi Wang\corref{cor1}}
         \ead{ziqiwang@berkeley.edu}
         \cortext[cor1]{Corresponding author}
		\address[1]{Department of Civil and Environmental Engineering, University of California, Berkeley, United States}
		\begin{abstract}
			This paper presents a physics and data co-driven surrogate modeling method for efficient rare event simulation of civil and mechanical systems with high-dimensional input uncertainties. The method fuses interpretable low-fidelity physical models with data-driven error corrections. The hypothesis is that a well-designed and well-trained simplified physical model can preserve salient features of the original model, while data-fitting techniques can fill the remaining gaps between the surrogate and original model predictions. The coupled physics-data-driven surrogate model is adaptively trained using active learning, aiming to achieve a high correlation and small bias between the surrogate and original model responses in the critical parametric region of a rare event. A final importance sampling step is introduced to correct the surrogate model-based probability estimations. Static and dynamic problems with input uncertainties modeled by random field and stochastic process are studied to demonstrate the proposed method.		
		\end{abstract}
		
		\begin{keyword}
		surrogate modeling \sep active learning \sep importance sampling \sep high-dimensional \sep rare event simulation\sep uncertainty quantification
		\end{keyword}
		
	\end{frontmatter}
	
	\renewcommand{\thefootnote}{\fnsymbol{footnote}}
	
	%% main text
	\section{Introduction}
	
	\noindent Uncertainty quantification (UQ) aims at quantifying and understanding the influence of ubiquitous uncertainties arising in science and engineering. Rare event simulation is a challenging branch of UQ with numerous engineering applications, such as the reliability of aerospace systems \cite{morio2015estimation}, resilience of critical infrastructures \cite{zio2021risk}, and safety of nuclear power plants \cite{jiang2021triso}. Sampling and surrogate modeling-based methods are two general approaches for rare event simulation, although other more specialized techniques exist \cite{dang2020mixture,xu2022adaptive,li2009stochastic,chen2019direct,xian2021seismic,lyu2022unified}.  Monte Carlo simulation \cite{rubinstein1998modern,landau2015guide} is typically insensitive to the dimensionality and nonlinearity of computational models. However, for rare event simulation, even the advanced variance-reduction techniques \cite{au2001estimation,kurtz2013cross,wang2016cross,engel2023bayesian,grigoriu2020data,papaioannou2016sequential,wang2019hamiltonian,xian2023relaxation} would require thousands of samples, restricting their applications to expensive computational models. 
	
	Surrogate modeling seeks to replace the original expensive computational model with a cheap substitute. A non-intrusive data-fitting surrogate model is derived directly from training  samples of the input-output pairs of the  computational model. Examples of data-fitting surrogate models include the quadratic response surface \cite{rajashekhar1993new,allaix2011improvement}, polynomial chaos expansion \cite{sudret2002comparison,torre2019data}, support vector machine \cite{hurtado2004examination,roy2023support}, and Gaussian process (also called Kriging) \cite{kaymaz2005application}, among others. Unlike many regression models, a noise-free Gaussian process is an exact interpolation model, i.e., the predictions at the training points are exact. For non-training points, the Gaussian process offers estimations of prediction variability \cite{jones2001taxonomy,sudret2012meta}. This feature facilitates the application of active learning, such that the training set can be adaptively enriched to reduce prediction variability for specified quantities of interest, resulting in improved accuracy and efficiency for Gaussian process-based rare event simulation \cite{bichon2008efficient,echard2011ak}. The active learning-based metamodeling approach can be further combined with advanced sampling techniques \cite{echard2013combined,huang2016assessing}, or adapted to estimate probability distribution functions \cite{wang2020novel}. 
 
 Despite the remarkable success of Gaussian process-based methods in various UQ applications, the model training becomes increasingly challenging, ultimately infeasible, with a growing number of input variables--a phenomenon known as the curse of dimensionality \cite{lataniotis2020extending}. To alleviate this problem, specialized unsupervised \cite{giovanis2018uncertainty,giovanis2020data,dos2022grassmannian,kontolati2022survey} and supervised \cite{jiang2017high,zhou2021active,navaneeth2022surrogate,kim2023adaptive} dimensionality reduction techniques have been proposed to couple with Gaussian process modeling. However, the effectiveness of these techniques is problem-dependent. More critically, many real-world high-dimensional problems do not admit simple low-dimensional representations \cite{jiang2021recursive}. Qualitatively speaking, for a ``high-dimensional input--low-dimensional output" problem, an ``optimal" dimensionality reduction is the computational model itself. It follows that constructing an effective dimensionality reduction can be as challenging as finding an accurate high-dimensional surrogate model. A possible way out is to inject  domain/problem-specific prior knowledge into the surrogate modeling process, as evidenced by the emerging paradigm of scientific machine learning \cite{raissi2019physics,zhu2019physics,linka2022bayesian,meng2023pinn} and multi-fidelity UQ \cite{peherstorfer2018survey,peherstorfer2016multifidelity,kramer2019multifidelity}.
	
	In this paper, we leverage physics-based surrogate modeling to solve high-dimensional rare event estimation problems. This idea is under the broad umbrella of multi-fidelity UQ. %Data-fitting surrogate models are typically versatile but lack interpretability, while physics-based simplified models are readily interpretable but often inaccurate. 
 A physics-based surrogate model can be adapted from the original high-fidelity model in various ad hoc ways, such as domain-specific simplifications \cite{peterson2018overview,han2013improving,held2005gap} and relaxations of numerical solvers \cite{peherstorfer2018survey}. In this study, we construct physics-based surrogate models equipped with the properties of (i) {parsimonious}: the surrogate model is parameterized by a few tunable control parameters, and (ii) {compatible with high-dimensional input uncertainties}: the surrogate model can accept high-dimensional input uncertainties either directly as input of the model or indirectly through filtering and coarse-graining operations. For example, the classic equivalent linearization method \cite{crandall2006half,elishakoff2017sixty} for nonlinear random vibration analysis involves constructing linear physical models with random processes as input and output. Therefore, equivalent linearization method can be reformulated into a physics-based surrogate modeling approach for nonlinear random vibration problems \cite{wang2022optimized}. In other engineering applications such as computational fluid dynamics \cite{peterson2018overview}, aerodynamic design \cite{han2013improving}, and climate modeling \cite{held2005gap}, there exist various domain-specific approaches in constructing simplified physics-based models. %General approaches include coarse-graining of spatial and temporal grids and relaxations of tolerances involved in iterative solvers \cite{peherstorfer2018survey}. 
 %To ensure the accuracy of surrogate modeling for rare event simulation, it is necessary to introduce a final importance sampling step to couple the surrogate model simulations with a limited number of the original computational model simulations, known as multi-fidelity importance sampling \cite{peherstorfer2016multifidelity,kramer2019multifidelity}. However, the common practice of multi-fidelity importance sampling relies on the Gaussian mixture model as the importance density, which scales poorly with dimensionality \cite{wang2016cross}. In this work, we adopt a new importance sampling formulation  \cite{wang2022optimized} to address high-dimensional rare event simulations.
	
	Physics-based surrogate models may have inherent errors due to simplifications; therefore, it is promising to introduce a data-driven error correction to fill the gap between the surrogate and original model predictions. This idea has been investigated recently in \cite{dhulipala2022active,dhulipala2022reliability}, where an active learning-based Gaussian process is trained to correct the low-fidelity model predictions. Inspired by the success of existing works,
 %However, the data-driven surrogates of error corrections therein are constructed on the model inputs directly, and thus the methodology may be restricted to low- and medium-dimensional input uncertainties due to the poor scalability of Gaussian process metamodeling with dimensionality \cite{lataniotis2020extending}. 
 this study is devoted to low probability estimations with high-dimensional input uncertainties. Three critical ingredients--parametric optimization for physics-based surrogate models, heteroscedastic Gaussian process for error corrections, and active learning for effective training--are leveraged to construct and train coupled physics-data-driven surrogate models. Finally, one can apply an importance sampling to couple the surrogate model simulations with limited evaluations of the original model, known as multi-fidelity importance sampling \cite{peherstorfer2016multifidelity,kramer2019multifidelity}. The common practice of multi-fidelity importance sampling uses parametric distribution models such as the Gaussian mixture as the importance density, which scales poorly with dimensionality \cite{wang2016cross}. In this work, we adopt a different importance sampling formulation  \cite{wang2022optimized} to address high-dimensional rare event simulations. 
 
 %a critical feature of the present framework targeted for high-dimensionality is that the error correction surrogate is constructed on the 1-dimensional output of the low-fidelity physical model instead of the high-dimensional model input. In essence, this represents a dimensionality reduction for surrogate modeling of error corrections, i.e., the 1-dimensional output space of the low-fidelity physical model can be viewed as a reduced feature space for the high-dimensional input space, which will inevitably result in certain noises for the error corrections. Correspondingly, three main ingredients, i.e., parametric optimization, heteroscedastic Gaussian process, and active learning, are leveraged to fulfill the training of the coupled physics-data-driven surrogate model. Optimization of the parameterized low-fidelity physical model can significantly reduce the noises of error corrections; heteroscedastic Gaussian process can accommodate the noises of error corrections; and active learning can facilitate the adaptive training towards rare event simulation. Finally, an importance sampling adapted for high-dimensional problems \cite{wang2022optimized} is introduced to gurrantee the theoretical correctness of the present approach.
	
	This paper is organized as follows. Section \ref{Sec:surrogatemodel} introduces the general formulation of the coupled physics-data-driven surrogate model. Section \ref{Sec:surrogatemodeltraining} develops a training process for the surrogate model, including optimization of parametric physical models, error corrections using heteroscedastic Gaussian process, and adaptive training by active learning. Section \ref{Sec:ImportanceSampling} introduces an importance sampling scheme to correct the surrogate model-based probability estimations. Section \ref{Sec:Application} demonstrates the performance of the proposed method for rare event simulation of high-dimensional problems. Section \ref{Sec:conclude} provides concluding remarks. The implementation details of the proposed method are provided in \ref{Append:implementationdetails}.
	
	\section{Coupled physics-data-driven surrogate model}\label{Sec:surrogatemodel}
	\noindent Consider an end-to-end computational model $\mathcal{M}:\vect x\in\rn\mapsto y\in\r$ that maps a $n$-dimensional input $\vect x$ into a $1$-dimensional output $y$. The input $\vect x$ is an outcome of a random vector $\vect X$ defined in the probability space $(\rn,\mathcal{B}_n,\mathbb{P}_{\vect X})$, where $\mathcal{B}_n$ is the Borel $\sigma$-algebra on $\rn$, and $\mathbb{P}_{\vect X}$ is the probability measure of $\vect X$ with the distribution function $f_{\vect X}$. The output $y$ is a performance variable such that it defines the rare event of interest through $\{y\leq 0\}$, without loss of generality. Since the source of randomness is from $\vect X$, the rare event probability is typically formulated in the space of $\vect X$, expressed by $\mathbb{P}_{\vect X}(\mathcal{M}(\vect X)\leq 0)$. This problem is challenging in real-world applications, because (i) the dimensionality of $\vect x$ is high, (ii) the computational model $\mathcal{M}$ involves expensive physics-based simulations, and (iii) the probability to be estimated is small. 
%	\begin{equation}\label{HFmodel}
%		y_{H}=\mathcal{M}_{H}(\vect{x})\,,
%	\end{equation}
%	where $\mathcal{M}_{H}$ is the high-fidelity model that is computationally expensive, $\vect{x}$ is the outcome of a $D$-dimensional vector $\vect{X}$ of basic random variables, and $y_{H}$ is the output quantity of interest. 	
	To reduce the computational cost of the original physics-based model $\mathcal{M}$, we construct a simplified (end-to-end) computational model expressed by
	\begin{equation}\label{LFmodel}
 \begin{aligned}
     &y_{p}=({\mathcal{M}}_p\circ\psi)(\vect x;\vect\theta_p)\,,\\
     &\mathcal{M}_{p}:\vect x'\in\r^{n'}\mapsto y_p\in\r\,,\\
     & \psi:\vect x\in\rn\mapsto\vect x'\in\r^{n'}\,,n'\leq n\,,
 \end{aligned}		
	\end{equation}
	where $\mathcal{M}_{p}$ represents a cheaper physics-based model derived from the original model $\mathcal{M}$, $\psi$ is a filtering or coarse-graining function of the input $\vect x$, ``$\circ$" denotes function composition, and $\vect\theta_p$ are tunable parameters of $\mathcal{M}_{p}$ and/or $\psi$. Depending on the selection of $\mathcal{M}_{p}$, the filtering function $\psi$ can be a trivial identity mapping if $\mathcal{M}_{p}$ and $\mathcal{M}$ share the same input space, or it can be a dimensionality reduction mapping if $\mathcal{M}_{p}$ is defined on a coarser/different input space. However, it is important to notice that ${\mathcal{M}}_p\circ\psi$ and $\mathcal{M}$ share the same source of randomness from $\vect X$, enabling a tuning of ${\mathcal{M}}_p\circ\psi$ (via adjusting $\vect\theta_p$) to maximize the statistical correlation between $Y$ and $Y_p$. 

 To improve the accuracy of Eq.~\eqref{LFmodel}, a data-driven error correction can be introduced, leading to a coupled physics-data-driven surrogate model expressed by
% and $y_{L}$ is the low-fidelity counterpart of $y_{H}$. The construction of the physics-based surrogate model $\mathcal{M}_{L}$ relies on the domain-specific knowledges and in-depth implementation details \cite{peherstorfer2018survey}. For instance, in the context of nonlinear random vibration, $\mathcal{M}_{H}$ and $\mathcal{M}_{L}$ can be constructed from the nonlinear system and the equivalent linear system, respectively. In a more general setting, one can define $\mathcal{M}_{H}$ and $\mathcal{M}_{L}$ in terms of fine- and coarse-grid spatial and/or temporal discretizations, or high- and low-fidelity nonlinear iterative solvers.	
	%To fill the gap between $y_{H}$ and $y_{L}$, a data-driven surrogate of error corrections can be introduced and constructed on $y_{L}$, leading to a coupled physics-data-driven surrogate model, i.e., 
	\begin{equation}\label{CoupledModel}
		\hat{y}=y_{p}+{\epsilon}(y_{p};\vect\theta_{\epsilon})=(\mathcal{M}_p\circ\psi)(\vect{x};\vect\theta_{p})+({\epsilon}\circ\mathcal{M}_p\circ\psi)(\vect{x};\vect\theta_{p},\vect\theta_{\epsilon})\,,
	\end{equation}
	where ${\epsilon}:y_p\in\r\mapsto\varepsilon\in\r$ is the error correction function constructed using data-fitting methods, and $\vect\theta_{\epsilon}$ are parameters of the data-fitting error correction. It is worth highlighting that the input of the error correction function is the output of the physics-based surrogate model. This construction is different from existing works \cite{dhulipala2022active,dhulipala2022reliability}, where the error correction term was constructed on the product space of $\vect{x}$ and $\vect x'$. The methodology of \cite{dhulipala2022active,dhulipala2022reliability} may face difficulties for problems with high-dimensional input uncertainties, due to the weak scalability of conventional data-fitting methods such as Gaussian process and polynomial regression \cite{lataniotis2020extending}. In comparison, our construction of the surrogate model can mitigate the curse of dimensionality. However, it comes with a price of having inherent noises in the error correction term even at the training points of $y_p$. This is because the mapping from $y_p$ to $y$ can be one-to-many. An intuitive restatement is that the physics-based surrogate model cannot capture all details of the original model. To quantify and mitigate the impact of the inherent noises, we use the heteroscedastic Gaussian process \cite{lazaro2013retrieval,rogers2020probabilistic,kim2023estimation} to model ${\epsilon}(y_p)$. This differs from the noise-free and homoscedastic Gaussian process models because the heteroscedastic Gaussian process model no longer performs exact interpolations and the noise is input-dependent.

Figure \ref{Fig:Figure0} presents a schematic of the proposed surrogate modeling method for rare event simulation. The technical details for optimizing the physics-based surrogate and error correction models are introduced in Sections \ref{Sec:optimization} and \ref{Sec:ErrorCorrection}, respectively. The active learning technique to guide the surrogate model training is described in Section \ref{Sec:ActiveLearning}. Section \ref{Sec:ImportanceSampling} offers details on the importance sampling for rare event probability estimations. 
		
	\begin{figure}[H]
		\centering
		\includegraphics[scale=0.55]{Figure0.png}
		\caption{\textbf{Schematic of the proposed surrogate modeling method for rare event simulation.} \textit{The physics-based surrogate model $y_p=\mathcal{M}_p(\vect x';\vect\theta_p)$ contains tunable control parameters $\vect\theta_p$ to be optimized in the training process. The error correction function $\epsilon(y_p)$ is modeled in the space of $y_p$ by heteroscedastic Gaussian process to account for input-dependent noises. The active learning adaptively enriches the training set with a learning criterion to highlight contributions from the rare event region. The importance sampling estimates the correction factor $c_P$ to improve the probability estimation of the surrogate model}.}
		\label{Fig:Figure0}
	\end{figure}
	
	
	\section{Training of the coupled physics-data-driven surrogate model }\label{Sec:surrogatemodeltraining}
	\subsection{Optimization of parametric physical models}\label{Sec:optimization}
	\noindent In physics-based surrogate modeling, the response predictions of the surrogate model can be inaccurate but still be mildly/highly correlated with that of the original model. The statistical correlation between the surrogate and original model responses has a decisive impact on the noise of the error correction term (see Figure \ref{Fig:Figure1} for a simple illustration)--higher correlation indicates smaller noise. Pearson, Spearman's Rank, and Kendall's Tau correlation coefficients are popular indices for quantifying the correlation between two random variables \cite{hauke2011comparison}. The Pearson correlation coefficient can only reflect the linear correlation, while the Spearman's Rank and Kendall's Tau correlation coefficients can handle the nonlinear dependency. In our context, the physics-based surrogate model by construction is a simplification of the original model. Thus, a mildly nonlinear correlation is expected. Consequently, the Pearson correlation coefficient is sufficient. The Spearman's Rank and Kendall's Tau correlation coefficients can be readily implemented into the proposed surrogate modeling method, but we did not observe an improvement in performance for the  numerical examples we have studied. { It is worth mentioning that mutual information \cite{taverniers2021mutual,beneddine2023nonlinear} is a more general metric for measuring dependency between two random variables, but the sample estimate of mutual information involves additional assumptions on the joint distributions.}

 \begin{figure}[H]
		\centering
		\includegraphics[scale=0.5]{Figure1.png}
		\caption{\textbf{Impact of the correlation between surrogate and original model responses on the noise of the error correction function.} \textit{For illustration, $Y$ and $Y_{p}$ are assumed to be lognormal with a Pearson correlation coefficient varying among $\rho=0.895$, $0.946$, $0.987$, and $0.998$. It is seen that the noise in the error $\epsilon=y-y_p$ decreases with the increase of the correlation coefficient. It is also clear that an exact interpolation model should be avoided when $\epsilon$ is noisy.}}
		\label{Fig:Figure1}
	\end{figure}
 
 Because rare event probabilities are dominated by conditional distributions, the correlation between $Y$ and $Y_p$ generated by $\vect X\sim f_{\vect X}(\vect x)$ may not be an ideal objective for optimizing the surrogate model. A remedy is to consider the correlation between $\1{\leq0}{Y}$ and $\1{\leq0}{Y_p}$, where $\mathbb{I}_{\leq0}:\r\mapsto\{0,1\}$ is a binary indicator function for the rare event. The computational issue of this approach is that the samples generated from $f_{\vect X}(\vect x)$ are unlikely to fall into the rare event and thus the estimation of the Pearson correlation coefficient between $\1{\leq0}{Y}$ and $\1{\leq0}{Y_p}$ would be inaccurate. In this work, we adopt active learning to adaptively enrich the training set $\mathcal{D}=\{(\vect x^{i},\mathcal{M}(\vect x^{(i)}))\}$ toward the critical region around $\{\mathcal{M}(\vect x)=0\}$. Using the training set $\mathcal{D}$, the sample correlation coefficient emphasizes the contribution from the rare event. This is essentially equivalent to redefining the correlation between $\mathcal{M}(\vect X)$ and $(\mathcal{M}_p\circ\psi)(\vect X)$ using an active learning-controlled importance sampling distribution rather than the original $f_{\vect X}(\vect x)$. To summarize, the optimization for the physics-based surrogate model $(\mathcal{M}_{p}\circ\psi)(\vect x;\vect\theta_p)$ is formulated as
	\begin{equation}\label{OptimizationModel}
 \begin{aligned}
     \vect{\theta} ^{\ast }_{p}&=\mathop{\arg\max}\limits_{\vect{\theta}_{p}}\rho_{YY_p}(\vect{\theta}_{p}|\mathcal{D})\\
     &=\mathop{\arg\max}\limits_{\vect{\theta}_{p}}\frac{\left \langle (\mathcal{M}(\vect{X})- \langle \mathcal{M}(\vect{X})\rangle _{\mathcal{D}})\left((\mathcal{M}_{p}\circ\psi)(\vect{X};\vect{\theta}_{p})- \langle (\mathcal{M}_{p}\circ\psi)(\vect{X};\vect{\theta}_{p})\rangle _{\mathcal{D}}\right)\right \rangle _{\mathcal{D}}}{\sqrt{\left \langle (\mathcal{M}(\vect{X})- \langle \mathcal{M}(\vect{X})\rangle _{\mathcal{D}})^2\right \rangle _{\mathcal{D}} \left \langle ((\mathcal{M}_{p}\circ\psi)(\vect{X};\vect{\theta}_{p})- \left \langle (\mathcal{M}_{p}\circ\psi)(\vect{X};\vect{\theta}_{p}) \right \rangle _{\mathcal{D}})^2\right \rangle _{\mathcal{D}}}}\,,
 \end{aligned}
	\end{equation}
 where $\langle\cdot\rangle$ denotes sample mean. 
	%where $\rho(\vect{\theta}_{L})$ is the correlation coefficient between the high-fidelity model response $Y_{H}=\mathcal{M}_{H}(\vect{X})$ and the low-fidelity parameterized physical model response $Y_{L}(\vect{\theta}_{L})=\mathcal{M}_{L}(\vect{X};\vect{\theta}_{L})$, i.e.,
	%\begin{equation}\label{correlationcoefficient}
		%\rho(\vect{\theta}_{L})=\frac{\mathbb{C}ov_f[Y_{H},Y_{L}(\vect{\theta}_{L})]}{\sqrt{\mathbb{V}ar_f[Y_{H}]\mathbb{V}ar_f[Y_{L}(\vect{\theta}_{L})]}}\,,
	%\end{equation}
   % where $\mathbb{C}ov_f$ and $\mathbb{V}ar_f$ denote respectively the covariance and variance with respect to the probability density function of $\vect{X}$, i.e., $f_{\vect{X}}(\vect{x})$, and $\vect{\theta}_{L}$ denotes a vector containing the parameters of the low-fidelity physical model.
	
	%Notably, Eq.\eqref{OptimizationModel} can be interpreted as finding a low-fidelity physical model that is most positively correlated with the original high-fidelity model. In fact, it is also desirable to derive a low-fidelity physical model that is most negatively correlated with the original high-fidelity model (i.e., maximizing $-\rho$ instead in Eq.\eqref{OptimizationModel}). However, such a physical model generally does not have practical significance and therefore is not considered herein.
 
 Before solving Eq.~\eqref{OptimizationModel}, we need to design $\vect\theta_p$ for the physics-based surrogate model, i.e., determine which parameters are tunable. For specific applications,  engineering judgement can be used to design $\vect\theta_p$. A more objective approach is to adopt a parsimonious principle to select a minimum number of parameters that can achieve a relatively high $\max_{\vect\theta_p}\rho_{YY_p}(\vect{\theta}_{p}|\mathcal{D})$. This principle can be materialized as an incremental approach to start with a single tunable parameter and iteratively augment if $\max_{\vect\theta_p}\rho_{YY_p}(\vect{\theta}_{p}|\mathcal{D})$ can be significantly improved. In the simulation test cases of civil and mechanical systems, we have investigated the use of elastic modulus, damping ratio, stiffness, and yield displacement as tunable parameters, where $\max_{\vect\theta_p}\rho_{YY_p}(\vect{\theta}_{p}|\mathcal{D})$ can typically achieve $0.95$. 

 	Finally, the optimization in Eq.~\eqref{OptimizationModel} can be solved by gradient-free metaheuristic algorithms \cite{conn2009introduction}. Provided with a training set $\mathcal{D}$, solving Eq.~\eqref{OptimizationModel} does not involve additional simulations of the original model, but it requires simulations of the physics-based surrogate model. If $\mathcal{D}$ is enriched by active learning, Eq.~\eqref{OptimizationModel} needs to be re-solved; the solution from the previous training set can be used as a warm initial guess.

  	
 
 %In this sense, the parameters $\vect{\theta}_{L}$ should dominate the performance of the low-fidelity physical model and be selected based on certain domain-specific expertise, e.g., the elastic modulus of a linear elastic cantilever beam (refer to Example 1), the damping parameter of a linear oscillator (refer to Example 2), and the stiffness reduction ratio, initial stiffness, and yield displacement of a nonlinear hysteretic system (refer to Example 3).
	

	
	%In reality, the correlation coefficient $\rho(\vect{\theta}_{L})$ can not be calculated analytically using Eq.\eqref{correlationcoefficient}, and should be approximated using random samples. For rare event simulation, a sufficient number of random samples should fall in the low-probability region to ensure a high degree of correlation between the high- and low-fidelity model responses in this region. Obviously, if the random samples of $\vect{X}$ is directly generated from $f_{\vect{X}}(\vect{x})$, a large sample size may be required to guarantee sufficient random samples would occur in the low-probability region, resulting in a large number of evaluations of the high-fidelity model. To address this problem, a small number of random samples of $\vect{X}$ can be first generated from $f_{\vect{X}}(\vect{x})$, and the dataset can then be adaptively enriched by active learning techniques. Therefore, Eq.\eqref{correlationcoefficient} can be approximated as
	%\begin{equation}\label{correlationcoefficientappro}
	%	\rho(\vect{\theta}_{L})\approx \hat{\rho}(\vect{\theta}_{L}| \mathcal{D})=\frac{\left \langle (\mathcal{M}_{H}(\vect{x})- \left \langle \mathcal{M}_{H}(\vect{x}) \right \rangle _{\kappa})(\mathcal{M}_{L}(\vect{x};\vect{\theta}_{L})- \left \langle \mathcal{M}_{L}(\vect{x};\vect{\theta}_{L}) \right \rangle _{\kappa})\right \rangle _{\kappa}}{\sqrt{\left \langle (\mathcal{M}_{H}(\vect{x})- \left \langle \mathcal{M}_{H}(\vect{x}) \right \rangle _{\kappa})^2\right \rangle _{\kappa} \left \langle (\mathcal{M}_{L}(\vect{x};\vect{\theta}_{L})- \left \langle \mathcal{M}_{L}(\vect{x};\vect{\theta}_{L}) \right \rangle _{\kappa})^2\right \rangle _{\kappa}}}\,,
	%\end{equation}
%	where $\left \langle \cdot  \right \rangle _{\kappa}$ denotes the sample average with respect to a “learning kernel” $\kappa_{\vect{X}}(\vect{x})$, and $\mathcal{D}$ denotes a dataset containing the random samples of $\vect{X}$ generated from $\kappa_{\vect{X}}(\vect{x})$ and the corresponding $Y_{H}=\mathcal{M}_{H}(\vect{X})$. Note that it is unnecessary to specify the concrete form of the learning kernel $\kappa_{\vect{X}}(\vect{x})$, because the random samples can be generated in accordance with the learning scheme. In fact, the learning kernel $\kappa_{\vect{X}}(\vect{x})$ is initially set as $f_{\vect{X}}(\vect{x})$, and will evolve in the subsequent learning process.
	
%	Substitution of Eq.\eqref{correlationcoefficientappro} into Eq.\eqref{OptimizationModel} yields
	%\begin{equation}\label{OptimizationModelappro}
%		\vect{\theta} ^{\ast }_{L}\approx\mathop{\arg\max}\limits_{\vect{\theta}_{L}}\hat{\rho}(\vect{\theta}_{L}| \mathcal{D})\,.
%	\end{equation}
	
	
	\subsection{Error corrections using heteroscedastic Gaussian process}\label{Sec:ErrorCorrection}
	\noindent Given a training set $\mathcal{D}=\{(\vect x^{i},\mathcal{M}(\vect x^{(i)}))\}$ and the solution of Eq.~\eqref{OptimizationModel}, we obtain the training set for the error correction, $\mathcal{D}_\epsilon=\{(y_p^{(i)},\varepsilon^{(i)})\}$, where $y_p^{(i)}=(\mathcal{M}_p\circ\psi)(\vect x^{(i)};\vect\theta_p^*)$ and $\varepsilon^{(i)}=\mathcal{M}(\vect x^{(i)})-(\mathcal{M}_p\circ\psi)(\vect x^{(i)};\vect\theta_p^*)$. Using $\mathcal{D}_\epsilon$, we train a heteroscedastic Gaussian process \cite{lazaro2013retrieval,rogers2020probabilistic,kim2023estimation} to model $\epsilon(y_p)$, expressed by
% As it is almost impossible to seek a low-fidelity physical model that is completely correlated with the original high-fidelity model, certain noises inevitably exist in the error corrections and should be incorporated into the data-driven surrogate modeling. For this purpose, a heteroscedastic Gaussian process \cite{lazaro2013retrieval,rogers2020probabilistic,kim2023estimation} that can account for input-dependent noises is adopted herein to model the noisy error corrections. Suppose the error correction term $\hat{\epsilon}(y_{L})$ in Eq.\eqref{CoupledModel} can be considered as a realization of a heteroscedastic Gaussian process, i.e.,
	\begin{equation}\label{hGPerrorcorrection}
		{\epsilon}(y_{p};\vect\theta_\epsilon)=f(y_{p};\vect\theta_f)+\tau(y_{p};\vect\theta_\tau)\,,
	\end{equation}
	where $\vect\theta_\epsilon=\vect\theta_f\cup\vect\theta_\tau$, $f(y_{p}) \sim \mathcal{GP}(\mu_{f}(y_{p}),k _{f}(y_{p},y_{p}^{\prime});\vect{\theta}_{f})$ is a Gaussian process with hyperparameters $\vect{\theta}_{f}$ to model the mean $\mu_{f}(y_{p})$ and kernel $k _{f}(y_{p},y_{p}^{\prime})$, $\tau(y_{p}) \sim \mathcal{N}(0,\exp(g(y_{p})))$ is an input-dependent Gaussian noise with zero mean and variance $\exp(g(y_{p}))$, and $g(y_{p})\sim \mathcal{GP}(\mu_{g},k _{g}(y_{p},y_{p}^{\prime});\vect{\theta}_{g})$ is another Gaussian process with hyperparameters $\vect{\theta}_{g}$ to model the mean $\mu_{g}$ and kernel $k _{g}(y_{p},y_{p}^{\prime})$. %It is noted that a homoscedastic Gaussian process model can be readily derived if the Gaussian noise term in Eq.\eqref{hGPerrorcorrection} is independent of the input $y_{L}$, i.e., $\varepsilon(y_{L})=\varepsilon \sim \mathcal{N}(0,\sigma_{\varepsilon}^{2})$, where $\sigma_{\varepsilon}$ is a constant standard deviation. 
	
	The introduction of the heteroscedastic Gaussian noise is essential in this study because the mapping from $y_p$ to $y$ can be one-to-many. The use of heteroscedastic Gaussian process comes with a price of increasing the number of hyperparameters for the Gaussian process model, and the analytical marginal likelihood and prediction equations for the homoscedastic Gaussian process are no longer useful \cite{lazaro2013retrieval,rogers2020probabilistic,kim2023estimation}. The hyperparameters $\vect{\theta}_{f}$ and $\vect{\theta}_{g}$ can be approximated by maximizing the following lower bound of the exact marginal likelihood \cite{lazaro2013retrieval}:
	\begin{equation}\label{likelihoodlowerbound}
		F(\vect{\mu} ,\vect{\Sigma} )=\textrm{log}f_{\mathcal{N}}(\vect{\varepsilon};\vect{0},\vect{K}_{f}+\vect{R})-\frac{1}{4}\textrm{tr}(\vect{\Sigma})-\textrm{KL}(f_{\mathcal{N}}(\vect{g};\vect{\mu},\vect{\Sigma})|| f_{\mathcal{N}}(\vect{g};\mu_{g}\vect{1},\vect{K}_{g})) \,,
	\end{equation}
	where $\vect{\mu}$ and $\vect{\Sigma}$ are the variational mean vector and covariance matrix to be determined alongside with the hyperparameters $\vect{\theta}_{f}$ and $\vect{\theta}_{g}$, $\vect{\varepsilon}=[{\varepsilon}^{(1)},{\varepsilon}^{(2)},...]\tr$ are from the training set $\mathcal{D}_\epsilon$, $\vect{0}$ and $\vect{1}$ are vectors of zeros and ones, $\vect{K}_{f}$ and $\vect{K}_{g}$ are respectively the covariance matrices of the Gaussian processes $f(y_{p})$ and $g(y_{p})$, $\vect{R}$ is a diagonal matrix with entries $R_{i,i}=\exp(\mu_{i}-\Sigma_{i,i}/2)$, $i=1,2,...$, $f_{\mathcal{N}}$ denotes the probability density function of a multivariate Gaussian distribution, $\textrm{tr}$ denotes matrix trace, and $\textrm{KL}(\cdot||\cdot)$ denotes the Kullback–Leibler divergence between two probability density functions. 
	
	Once the hyperparameters $\vect{\theta}_{f}$ and $\vect{\theta}_{g}$ are estimated, the predictions for the mean and variance of $\epsilon(y_{p}^{\ast})$ at $y_{p}^{\ast}$ are obtained from \cite{lazaro2013retrieval}
	\begin{equation}\label{PredictionMean}
		\mu_{{\epsilon}}(y^{\ast}_{p})=\vect{k}_{f\ast}\tr(\vect{K}_{f}+\vect{R})^{-1}\vect{\varepsilon}\,,
	\end{equation}
	and
	\begin{equation}\label{PredictionSigma}
		\sigma_{{\epsilon}}^{2}(y_{p}^{\ast})=\exp\left(\vect{k}_{g\ast}\tr(\vect{\Lambda}-\tfrac{1}{2}\vect{I})\vect{1}+\mu_{g}+\frac{k_{g\ast\ast}-\vect{k}_{g\ast}\tr(\vect{K}_{g}+\vect{\Lambda}^{-1})^{-1}\vect{k}_{g\ast}}{2} \right)+k_{f\ast\ast}-\vect{k}_{f\ast}\tr(\vect{K}_{f}+\vect{R})^{-1}\vect{k}_{f\ast}\,,
	\end{equation}
	respectively, where $k_{f\ast\ast}=k_{f}(y_{p}^{\ast},y_{p}^{\ast})$, $k_{g\ast\ast}=k_{g}(y_{p}^{\ast},y_{p}^{\ast})$, $\vect{k}_{f\ast}=[k_{f}(y_{p}^{(1)},y_{p}^{\ast}),k_{f}(y_{p}^{(2)},y_{p}^{\ast}),...]\tr$ and $\vect{k}_{g\ast}=[k_{g}(y_{p}^{(1)},y_{p}^{\ast}),k_{g}(y_{p}^{(2)},y_{p}^{\ast}),...]{\tr}$, $\vect{I}$ is the identity matrix, and $\vect{\Lambda}$ is a positive semi-definite diagonal matrix introduced to re-parameterize $\vect{\mu}$ and $\vect{\Sigma}$ in a reduced order.
	
	The mean $\mu_{{\epsilon}}(y^{\ast}_{p})$ can be used as the prediction of the surrogate modeling error at $y_{p}^{\ast}$, and $\sigma_{{\epsilon}}^{2}(y_{p}^{\ast})$ quantifies the uncertainty of this prediction. The later quantity is desirable in active learning. It is worth noting that the roles of the heteroscedastic Gaussian process involve quantifying the noise and correcting the (mean) predictions of the physics-based surrogate model (see Figure \ref{Fig:Figure2}); it cannot reduce the noise. %or increase the degree of correlation between the high- and low-fidelity model responses.
	
	\begin{figure}[H]
		\centering
		\includegraphics[scale=0.5]{Figure2.png}
		\caption{\textbf{Heteroscedastic Gaussian process for error correction.} \textit{Following Figure \ref{Fig:Figure1}, for $\rho=0.946$, a heteroscedastic Gaussian process is trained to fit the error. It is seen that the heteroscedastic errors are well-captured. The heteroscedastic Gaussian process corrects the surrogate model predictions, however, the improvement on correlation coefficient is marginal: $0.949$ for $y_{}$ and $y_{p}+\mu_{{\epsilon}}(y_{p})$ compared with the original $\rho=0.946$.}}
		\label{Fig:Figure2}
	\end{figure}
	
	\subsection{Adaptive training by active learning}\label{Sec:ActiveLearning}
	\noindent Given an initial training set $\mathcal{D}=\{(\vect x^{(i)},\mathcal{M}(\vect x^{(i)}))\}_{i=1}^{N_0}\equiv\mathcal{D}_{\vect x}\times\mathcal{D}_y$, we sequentially train the physics-based surrogate model $y_p=(\mathcal{M}_p\circ\psi)(\vect x;\vect\theta_p)$ and the error correction function $\epsilon(y_p;\vect\theta_\epsilon)$ to obtain the initial surrogate model $\hat y=(\mathcal{M}_p\circ\psi)(\vect x;\vect\theta_p)+\epsilon(y_p;\vect\theta_\epsilon)$. Subsequently, we initiate an active learning process to iteratively enrich the training set $\mathcal{D}$ and update the surrogate model. Hereafter, we develop an active learning process tailored to the physics-data-driven surrogate modeling. 
 
 %The optimized low-fidelity physical model $\mathcal{M}_{L}(\vect{x};\vect{\theta}_{L}^{\ast})$ and the corresponding response samples $\mathcal{Y}_{L}=\{ y_{L,i}=\mathcal{M}_{L}(\vect{x}_{i};\vect{\theta}_{L}^{\ast}),i=1,2,...,n_0 \}$ can be obtained using the ingredients presented in Section \ref{Sec:optimization}. Thereafter, based on $\mathcal{Y}_{H}$ and $\mathcal{Y}_{L}$, the heteroscedastic Gaussian process model of error corrections $\hat{\epsilon}(y_{L})$, along with the prediction mean $\mu_{\hat{\epsilon}}(y_{L})$ and variance $\sigma_{\hat{\epsilon}}^{2}(y_{L})$, can be constructed using the ingredients presented in Section \ref{Sec:ErrorCorrection}. The above two steps can yield the coupled physics-data-driven surrogate model regarding the initial training points. However, such a coupled surrogate model is typically inadequate for rare event simulation associated with low-probability estimation, e.g., small failure probability estimation, because the initial training points may all fall into the high-probability region. In view of this, certain active learning techniques \cite{bichon2008efficient,echard2011ak} are desirable for adaptively identifying the training points around the vicinity of the limit state that considerably contributes to the failure probability. 
	
	%Let $\mathcal{X}=\{\vect{x}^{(i)}\}_{i=1}^N$, $N\gg N_0$, denote the candidate training set formed by sampling from $f_{\vect X}$. 
 
 Provided with the current surrogate model, we first define a critical learning region $\Omega_c$ as:
	\begin{equation}\label{LearningFun1}
		\Omega_c=\left \{ \vect{x}\in\rn:\frac{\left |y_{p}+\mu_{{\epsilon}}(y_{p})\right |}{\sigma_{{\epsilon}}(y_{p})} \leqslant \delta ,\,\,\, y_{p}=(\mathcal{M}_{p}\circ\psi)(\vect{x};\vect{\theta}_{p}) \right \}\,,
	\end{equation}
	where $\delta>0$ is a cutoff value for the learning region, and $\mu_\epsilon(y_p)$ and $\sigma_\epsilon(y_p)$ are respectively the mean and standard deviation of the Gaussian process error correction at $y_p$. 
 
 Next, we generate a candidate training set  ${\mathcal{X}}_c=\{\vect x^{(i)}\}_{i=1}^N$ through 
 \begin{equation}\label{LearningKernel}
     \vect X^{(i)}\sim\1{\Omega_c}{\vect x}f_{\vect X}(\vect x)\,,
 \end{equation}
 where $\mathbb{I}_{\Omega_c}:\rn\mapsto\{0,1\}$ is an indicator function for $\{\vect x\in\Omega_c\}$ and the normalizing constant for the density $\1{\Omega_c}{\vect x}f_{\vect X}(\vect x)$ is omitted for simplicity. To reduce surrogate model simulations and improve the scalability toward low probability estimations, the sequential Monte Carlo \cite{papaioannou2016sequential,wang2019hamiltonian,xian2023relaxation} is used to generate ${\mathcal{X}}_c$. 
 
 Given ${\mathcal{X}}_c$, the  next training point $\vect{x}^{\ast}$ is identified by:
	\begin{equation}\label{LearningFun2}
		\vect{x}^{\ast}=\mathop{\arg\max}\limits_{\vect{x}\in {\mathcal{X}}_c} \left ( \mathop{\min}\limits_{y_p^\prime \in{\mathcal{Y}_p}}\left\|{y}_p-{y}_p^\prime \right \| \right ),\,\,\, y_{p}=(\mathcal{M}_{p}\circ\psi)(\vect{x};\vect{\theta}_{p})\,,
	\end{equation}
 where $\mathcal{Y}_p=\{y_{p}=(\mathcal{M}_{p}\circ\psi)(\vect{x};\vect{\theta}_{p}):\vect x\in\mathcal{D}_{\vect x}\}$ is a set of $y_p$ predictions from existing training samples. This equation picks the point in $\mathcal{X}_c$ that differs (in terms of $y_p$) the most from the existing training points, aiming to achieve sparsely distributed training points in the space of $y_p$. This learning criterion is a relaxation of the classic U-function-based learning \cite{echard2011ak} through introducing additional, seemingly redundant, distance-based selection to enforce sparsity. This relaxation is necessary because the prediction uncertainty term $\sigma_\epsilon(y_p)$ of the U-function, which influences the ``sparsity" of the U-function-based training point selection, is not only affected by the distance to existing training points, but also contributed, sometimes dominantly, by the inherent noise of the surrogate model. In the conventional application \cite{echard2011ak,echard2013combined,huang2016assessing} of active learning-based Gaussian process modeling, the training data does not have inherent stochastic noise, and the distance-based selection in Eq.~\eqref{LearningFun2} can be unnecessary. Incidentally, the constructions of Eq.~\eqref{LearningFun1} and Eq.~\eqref{LearningKernel} are still meaningful even for noise-free scenarios, because they facilitate the use of efficient sampling methods to identify the low-probability critical region to propose training points. 
	
Provided with the next training point $\vect{x}^{\ast}$, the original model is simulated to obtain $y^{\ast}=\mathcal{M}(\vect{x}^{\ast})$, the training set $\mathcal{D}$ is augmented to include $(\vect{x}^{\ast},y^{\ast})$, and the surrogate model is updated. The learning process is repeated until a stopping criterion is reached. The simple stopping criterion adopted in this work is expressed by $\frac{\left | \hat{P}^{(m+1)}-\hat{P}^{(m)} \right |}{\hat{P}^{(m+1)}}\leqslant \eta$, $m=0,1,...$, i.e., the learning stops if the rare event probability estimation is stable. An alternative stopping criterion is to check the prediction uncertainty of the rare event probability \cite{dubourg2011reliability,moustapha2022active}; this approach typically requires more learning steps to stop but favors accuracy. 
	%\begin{equation}\label{Convergence}
	%	\eta=\frac{\left | \hat{P}_{f}^{(m+1)}-\hat{P}_{f}^{(m)} \right |}{\hat{P}_{f}^{(m+1)}}\leqslant \eta_c \,,
%	\end{equation}
	%where $\hat{P}_{f}^{(m)}$ and $\hat{P}_{f}^{(m+1)}$ are the failure probability estimations obtained by the Monte Carlo simulation of the coupled surrogate models at the $m$-th and ($m+$1)-th learning steps, respectively, and $\eta_c$ is the specified tolerance of convergence.
	
	The implementation details for the training of the coupled physics-data-driven surrogate model are presented in Appendix \ref{Fig:Appendix1}. Due to the presence of inherent noises in the surrogate model  (recall Figure \ref{Fig:Figure2}(b)), the  probability estimation has no guarantee of convergence to the correct probability. To address this issue, a final importance sampling step will be introduced in the following section.
	
	\section{Importance sampling using the coupled physics-data-driven surrogate model}\label{Sec:ImportanceSampling}
	\noindent The target rare event probability can be formulated as
	\begin{equation}\label{FailureProbability}
		P=\int _{\vect{x}\in \rn}\1{\leq0}{\mathcal{M}(\vect{x})}f_{\vect{X}}(\vect{x})\mathrm{d}\vect{x}\,.
	\end{equation}
	%where $\vect{X}$ is assumed to be a random vector of $D$ independent standard Gaussian variables, $f_{\vect{X}}(\vect{x})$ is the joint probability density function of $\vect{X}$, and $\mathbb{I}(\mathcal{M}_{H}(\vect{x})\geqslant  b)$ is the binary indicator function that gives 1 if $\mathcal{M}_{H}(\vect{x})\geqslant  b$ and 0 otherwise.
 The importance sampling rewrites the integral in Eq.~\eqref{FailureProbability} by introducing an importance density $h(\vect{x})$, i.e.,
	\begin{equation}\label{ImportanceSampling}
		P=\int _{\vect{x}\in \rn}\1{\leq0}{\mathcal{M}(\vect{x})}\frac{f_{\vect{X}}(\vect{x})}{h(\vect{x})}h(\vect{x})\mathrm{d}\vect{x}\,.
	\end{equation}
	The support of the importance density $h(\vect{x})$ should cover the rare event, and the optimal importance density is \cite{rubinstein1998modern}
	\begin{equation}\label{ImportanceDensityHF}
		h^{\ast }(\vect{x})=\frac{\1{\leq0}{\mathcal{M}(\vect{x})}f_{\vect{X}}(\vect{x})}{P}\,.
	\end{equation}
	Due to the high correlation between the original and surrogate model responses, it is tempting to use the following importance density from the surrogate model as an approximation for the optimal density. 
	\begin{equation}\label{ImportanceDensityLF}
		\hat{h}(\vect{x})=\frac{\1{\leq0}{(\mathcal{M}_{p}\circ\psi)(\vect{x};\vect{\theta}_{p})+\mu_{{\epsilon}}((\mathcal{M}_{p}\circ\psi)(\vect{x};\vect{\theta}_{p});\vect\theta_\epsilon)}f_{\vect{X}}(\vect{x})}{\hat{P}}\,,
	\end{equation}
	where $\hat{P}$ is the rare event probability estimated from the surrogate model. However, there is no guarantee that the target rare event is an improper subset of the support of $\hat h(\vect{x})$. Therefore, $\hat h(\vect{x})$ cannot be directly employed as an importance density. One remedy is to replace the binary indicator function in Eq.~\eqref{ImportanceDensityLF} by a smooth approximation \cite{papaioannou2016sequential,wang2022optimized}, so that $\hat h(\vect{x})$ becomes nonzero everywhere. This approach requires the tuning of a relaxation parameter associated with the smooth approximation of the indicator function. 
    
    An alternative remedy, which is adopted in this paper, is to use the following identity derived from properties of conditional probability \cite{wang2022optimized}:
	\begin{equation}\label{Identity}
		P=c_P\cdot\hat{P}\equiv\frac{\int _{\vect{x}\in\rn}\1{\leq0}{\mathcal{M}(\vect{x})}\hat h(\vect{x})\mathrm{d}\vect{x}}{\int _{\vect{x}\in \rn}\1{\leq0}{(\mathcal{M}_{p}\circ\psi)(\vect{x};\vect{\theta}_{p})+\mu_{{\epsilon}}((\mathcal{M}_{p}\circ\psi)(\vect{x};\vect{\theta}_{p});\vect\theta_\epsilon)}h^{\ast }(\vect{x})\mathrm{d}\vect{x}}\,\hat{P}\,,
	\end{equation}
	where $c_P$ is the correction factor for the surrogate model-based rare event probability estimation. The numerator represents the conditional probability that the original rare event given the surrogate rare event, while the denominator represents the conditional probability that the surrogate rare event given the original rare event. Ideally, the correction factor $c_P$ should be close to $1$; therefore, $c_P$ can be used as a performance measure of the surrogate model. The conditional probabilities in Eq.~\eqref{Identity} can be estimated by Monte Carlo simulation with Markov Chain Monte Carlo algorithms \cite{neal2011mcmc,papaioannou2015mcmc} to generate samples from $h^*(\vect x)$ and $\hat h(\vect x)$. For a well-trained surrogate model, the overlap between $h^*(\vect x)$ and $\hat h(\vect x)$ is expected to be significant; consequently, the computational cost to estimate $c_P$ would be small. However, theoretically speaking, Eq.~\eqref{Identity} implies that the surrogate model solution $\hat P$ can be highly accurate even when the surrogate rare event does not largely overlap with the actual rare event; this is illustrated by Figure \ref{Fig:Figure3}. Finally, the implementation details of the importance sampling using Eq.~\eqref{Identity} are presented in Appendix \ref{Fig:Appendix2}. 
		
	\begin{figure}[H]
		\centering
		\includegraphics[scale=0.5]{Figure3.png}
		\caption{\textbf{Importance sampling for the coupled physics-data-driven surrogate model.} \textit{The true rare event from the original model is $E_1\cup E_3$, and the ``surrogate" rare event from the surrogate model is $E_2\cup E_3$. Eq.~\eqref{Identity} is derived from the identity $\Prob{E_1\cup E_3}=\Prob{E_2\cup E_3}\frac{\Prob{E_3}/{\Prob{E_2\cup E_3}}}{\Prob{E_3}/{\Prob{E_1\cup E_3}}}$. It follows that if $\Prob{E_1}=\Prob{E_2}$, $P=\hat P$, i.e., $\hat P$ can be accurate even when the overlap $E_3$ is not significant. This condition can be met if $\hat Y$ is an unbiased estimator of $Y$ for the region $E_1\cup E_2\cup E_3$, supporting the proposed approach of training a Gaussian process to correct the bias in the critical region identified by active learning}.}
		\label{Fig:Figure3}
	\end{figure}
	
	\section{Numerical examples}\label{Sec:Application}
	\noindent In this section, we will investigate (1) a static problem of a linear elastic cantilever beam with material properties modeled by a Gaussian random field, (2) a dynamic problem of a nonlinear viscous damper under white noise excitation, and (3) a dynamic problem of a multi-degree-of-freedom hysteretic system under white noise excitation. Three schemes are investigated to construct physics-based surrogate models. Specifically, homogenization of material properties is considered in the first example, statistical linearization is used in the second example, and relaxation of the numerical solver is adopted in the third example.
	
	\subsection{Example 1: A linear elastic cantilever beam }\label{Sec:Applicationone}
	\noindent Consider a linear elastic cantilever beam of length $5$m and width $2$m subjected to 20 evenly spaced concentrated loads on the upper edge with the magnitudes of $Q=500\mathrm{kN}$, shown in Figure \ref{Fig:Example1_fig1}. The Poisson ratio of the beam is $\nu=0.3$. The reference/original computational model is represented by a discretization of the beam into $50\times20=1000$ four-node quadrilateral elements, and the elastic modulus of the beam is characterized by a homogeneous random field $E(x,y)$ with a discretization of $1000$ Gaussian random variables compatible with the spatial discretization. The autocorrelation function of the Gaussian random field is modeled by the isotropic exponential model $R_{E}(\Delta _{x},\Delta _{y})=\exp\left (-\sqrt{\Delta _{x}^2+\Delta _{y}^2}/l  \right )$ with the correlation length $l=10\mathrm{m}$. The mean value and standard deviation of the marginal Gaussian distribution are $\mu _{E}=200\mathrm{GPa}$ and $\sigma _{E}=30\mathrm{GPa}$, respectively. The physics-based surrogate model is constructed via a simple homogenization of the random field of the elastic modulus: 
 \begin{equation}
  E_{p}=a_{E}\bar{E}+b_{E}\,,   
 \end{equation}
 where $\{a_{E},b_{E}\}=\vect\theta_p$ are tunable parameters of the physics-based surrogate model, and $\bar{E}$ represents the average of the $1000$ Gaussian random variables. The initial values of $a_{E}$ and $b_{E}$ are set to be $1$ and $0$, respectively. The rare event of interest is defined as the vertical displacement of point A exceeds a prescribed threshold of $0.0032\mathrm{m}$, expressed by
 \begin{equation}
   \{y=0.0032-d_A\leq0\}\,,  
 \end{equation}
where $d_A$ is the vertical displacement of point A.  

\begin{figure}[H]
		\centering
		\includegraphics[scale=0.055]{Example1_Fig1.png}
		\caption{\textbf{Physics-based surrogate model of a linear elastic cantilever beam.} \textit{(a) original model: the spatial domain of the beam is discretized into $50\times 20=1000$ four-node quadrilateral elements. The elastic modulus of the beam is characterized by a homogeneous random field $E(x,y)$ with a discretization of $1000$ Gaussian random variables compatible with the spatial discretization; (b) physics-based surrogate model: the whole beam is represented by a single four-node quadrilateral element. The elastic modulus is assumed to be a homogenization of the random field, i.e., $E_{p}=a_{E}\bar{E}+b_{E}$, where $a_{E}$ and $b_{E}$ are tunable parameters, and $\bar{E}$ is the average of the $1000$ Gaussian random variables.}}
		\label{Fig:Example1_fig1}
	\end{figure}
	
The coupled physics-data-driven surrogate model is trained with $100$ random initial training points in conjunction with $57$ training points identified by active learning, leading to a total of $157$ runs of the original model. The heteroscedastic Gaussian process model for the error correction at the final learning stage is shown in Figure \ref{Fig:Example1_fig2}. It is seen that the heteroscedastic Gaussian process can model the noisy errors. The optimization histories of the correlation coefficient ${\rho}$, the parameter $b_{E}$, and the probability $\hat{P}$ are shown in Figure \ref{Fig:Example1_fig3}, and their final values are $0.9861$, $20.5$GPa, and $2.1110\times10^{-5}$, respectively. Notice that the correlation coefficient value of $0.9861$ is estimated using the training data, the ``global" correlation coefficient is around $0.9605$. The reference rare event probability obtained by the subset simulation with $4.6\times 10^{4}$ samples is $3.6130\times10^{-5}$. The scatter plots of $y_p$ and $y_p+\epsilon(y_p)$ against $y$ are shown in Figure \ref{Fig:Example1_fig4}. It is seen that the response predictions of the physics-based surrogate model are highly correlated with the responses of the original model, and the data-driven error correction can improve the bias of the surrogate predictions. Finally, the importance sampling is implemented to further improve the probability estimation. To achieve a coefficient of variation of $5\%$, 800 runs of the original model are required and the final estimate of probability is $3.4326\times10^{-5}$. This indicates that the correction factor $c_P$ (recall Eq.~\eqref{ImportanceSampling}) for the surrogate model solution is $1.63$.
	
	\begin{figure}[H]
		\centering
		\includegraphics[scale=0.5]{Example1_Fig2.png}
		\caption{\textbf{Heteroscedastic Gaussian process model of the error correction at the final learning stage for Example 1.} \textit{The Gaussian process model is obtained from $100(\text{initial})+57(\text{active learning})$ training data. It is seen that the heteroscedastic Gaussian process model can capture the noisy errors.}}
		\label{Fig:Example1_fig2}
	\end{figure}
	
	\begin{figure}[H]
		\centering
		\includegraphics[scale=0.4]{Example1_Fig4.png}
		\caption{\textbf{Optimization histories of the proposed method for Example 1.} \textit{(a) the correlation coefficient ${\rho}$ between $Y_p$ and $Y$ at the training points achieves $0.9861$; (b) the tunable parameter $b_{E}$ for the physics-based surrogate model converges to $20.5\mathrm{GPa}$; (c) the probability estimation $\hat{P}$ converges to $2.1110\times10^{-5}$. The reference solution from subset simulation with $4.6\times10^{4}$ samples is $3.6130\times10^{-5}$.}}
		\label{Fig:Example1_fig3}
	\end{figure}
	
	\begin{figure}[H]
		\centering
		\includegraphics[scale=0.5]{Example1_Fig3.png}
		\caption{\textbf{Response predictions of the physics-based surrogate model and the coupled physics-data-driven surrogate model for Example 1.} \textit{The response prediction of the physics-based surrogate model $Y_p$ is highly correlated with the true responses $Y$, and the data-driven error correction improves the bias. The estimated correlation coefficient between $Y$ and $Y_{p}$ as well as that between $Y$ and $Y_{p}+\mu_{\epsilon}(Y_{p})$ are both around $0.9605$. This is expected because from Figure \ref{Fig:Example1_fig2} we can observe that the error correction $\epsilon$ is linear with respect to $y_p$}.}
		\label{Fig:Example1_fig4}
	\end{figure}
	
	\subsection{Example 2: An oscillator with nonlinear viscous damper}\label{Sec:Applicationtwo}
	\noindent Consider an oscillator with nonlinear viscous damper under a Gaussian white noise excitation with the equation of motion expressed as
	\begin{equation}\label{viscousoscillator}
		m\ddot{u}(t)+c\dot{u}(t)+ku(t)+c_{d}\mathrm{sign}(\dot{u}(t))\left | \dot{u}(t) \right |^{\alpha_{d}}=-m\ddot{u}_g(t)\,,
	\end{equation}
	where $m=3000\mathrm{kg}$, $c=3000\mathrm{N/(m/s)}$, and $k=3\times10^5\mathrm{N/m}$ are the mass, damping, and stiffness of the oscillator, respectively, $u(t)$, $\dot{u}(t)$, and $\ddot{u}(t)$ are the displacement, velocity, and acceleration of the oscillator, respectively, $c_{d}=800\mathrm{N/(m/s)}^{\alpha_{d}}$ and $\alpha_{d}=0.3$ are the damping coefficient and velocity exponent of the nonlinear viscous damper, respectively, $\mathrm{sign}(\cdot )$ is the sign function, and $\ddot{u}_g(t)$ is the acceleration excitation.
	
	The excitation has a duration of $15$ seconds. Using the spectral representation method \cite{shinozuka1991simulation}, the white noise can be discretized into a finite set of Gaussian variables:  
	\begin{equation}\label{Seismicexcitationrandomprocess}
		\ddot{u}_g(\vect{X},t)=\sum_{i=1}^{D/2}\sqrt{2S_{0}\Delta \omega }\left ( X_i\textrm{cos}(\omega_it)+ \bar{X}_i\mathrm{sin}(\omega_it)\right )\,,
	\end{equation}
	where $X_i$ and $\bar{X}_i$, $i=1,2,...,D/2$, are mutually independent standard normal variables, $D=1000$, $\Delta \omega=2\omega_{\max}/D$ is the frequency increment with $\omega_{\max}=25\pi$ being the upper cutoff
	angular frequency, $\omega_i=(i-0.5)\Delta\omega$ is the discretized frequency points, and $S_0=5\times10^{-3} \mathrm{m^2/s^3}$ is the intensity of the white noise.
	
	To obtain the physics-based surrogate model, the nonlinear equation of motion shown in Eq.~\eqref{viscousoscillator} is linearized into
	\begin{equation}\label{viscousoscillatorlinearized}
		m\ddot{u}(t)+(c+c_{e})\dot{u}(t)+ku(t)=-m\ddot{u}_g(t)\,,
	\end{equation}
	where $\{c_{e}\}=\vect\theta_p$ is the equivalent damping coefficient used as the tuning parameter of the physics-based surrogate model. The rare event of interest is defined as the peak absolute displacement of the oscillator exceeding a threshold of $0.06$m, expressed by
 \begin{equation}
  \left\lbrace y=0.06-\sup_{t\in[0,15]}|u(t)|\leq0\right\rbrace\,. 
 \end{equation}
 The initial value of $c_{e}$ for optimization is determined by the traditional statistical linearization method \cite{xian2020stochastic}. 
 
 The coupled physics-data-driven surrogate model is trained using $100(\text{initial})+86(\text{active learning})$ samples, and the heteroscedastic Gaussian process model for error correction at the final learning stage is shown in Figure \ref{Fig:Example2_fig1}. The optimization histories of the correlation coefficient ${\rho}$, the equivalent damping coefficient $c_{e}$, and the probability $\hat{P}$ are shown in Figure \ref{Fig:Example2_fig2}; their final values are $0.9980$, $1973.3$N/(m/s), and $2.5310\times10^{-7}$, respectively. The probability predicted by the coupled surrogate model is $2.5310\times10^{-7}$, which is close to the reference solution $2.9540\times10^{-7}$ obtained from subset simulation using $6.49\times 10^{4}$ samples. The scatter plots of $y_p$ and $y_p+\epsilon(y_p)$ against $y$ are shown in Figure \ref{Fig:Example2_fig3}. The correction factor $c_P$ in the importance sampling step is estimated to be $1.25$ using an additional $800$ runs of the original model, leading to a final probability estimate of $3.1536\times10^{-7}$.
	
	\begin{figure}[H]
		\centering
		\includegraphics[scale=0.5]{Example2_Fig1.png}
		\caption{\textbf{Heteroscedastic Gaussian process model of the error correction at the final learning stage for Example 2.} \textit{The surrogate model is trained using $100(\text{initial})+86(\text{active learning})$ samples. The heteroscedastic Gaussian process model captures the noisy errors.}}
		\label{Fig:Example2_fig1}
	\end{figure}
	
	\begin{figure}[H]
		\centering
		\includegraphics[scale=0.4]{Example2_Fig3.png}
		\caption{\textbf{Optimization histories of the proposed method for Example 2.} \textit{(a) the correlation coefficient ${\rho}$ between $Y_p$ and $Y$ at the training points achieves $0.9980$; (b) the equivalent damping parameter $c_e$ for the physics-based surrogate model stops at $1973.3\mathrm{N/(m/s)}$; (c) the probability estimation $\hat{P}$ converges to $2.5310\times10^{-7}$. The reference solution of subset simulation with $6.49\times10^{4}$ samples is $2.9540\times10^{-7}$}.}
		\label{Fig:Example2_fig2}
	\end{figure}
	
	\begin{figure}[H]
		\centering
		\includegraphics[scale=0.5]{Example2_Fig2.png}
		\caption{\textbf{Response predictions of physics-based surrogate model and coupled physics-data-driven surrogate model for Example 2.} \textit{The response prediction of the physics-based surrogate model $Y_p$ is highly correlated with the true responses $Y$, and the data-driven error correction improves the bias. The estimated correlation coefficients between $Y$ and $Y_{p}$ and that between $Y$ and $Y_{p}+\mu_{\epsilon}(Y_{p})$ are $0.9783$ and $0.9791$, respectively. This tiny difference originates from the weak nonlinearity in Figure \ref{Fig:Example2_fig1}}.}
		\label{Fig:Example2_fig3}
	\end{figure}
	
	\subsection{Example 3: A multi-degree-of-freedom  hysteretic system}\label{Sec:Applicationthree}
	\noindent Consider a 6-degree-of-freedom shear-type  hysteretic system under ground motion excitation, shown in Figure \ref{Fig:Example3_fig1}. The mass and initial stiffness of each storey are $m_{i}=8000\mathrm{kg}$ and $k_{i}=1\times10^7\mathrm{N/m}$, $i=1,2,...,6$, respectively. Rayleigh damping is assumed for the hysteretic system with a damping ratio of $5\%$ for the 1st and 6th mode of the system. The restoring force of each storey is described by the Bouc-Wen model \cite{wen1980equivalent} as follows:	
	\begin{equation}\label{BoucWen1}
		f_{i}(t)=\alpha_{i} k_{i}u_{i}(t)+(1-\alpha_{i})k_{i}z_{i}(t)\,,
	\end{equation}
	\begin{equation}\label{BoucWen2}
		\dot{z_{i}}(t)=g_{i}(\dot{u_{i}}(t),z_{i}(t))=\phi_{i} \dot{u_{i}}(t)-\varphi_{i} \left |\dot{u_{i}}(t)  \right |z_{i}(t)\left |z_{i}(t)  \right |^{\gamma_{i} -1}-\psi_{i}  \dot{u_{i}}(t)\left |z_{i}(t)  \right |^{\gamma_{i} }\,,
	\end{equation}
	where $\alpha_{i}=0.1$ is the stiffness reduction ratio of the $i$-th storey; $u_{i}(t)$, $\dot{u_{i}}(t)$, and $z_{i}(t)$ are the relative displacement, relative velocity, and the hysteretic displacement of the $i$-th storey, respectively; $\phi_{i}=1$, $\varphi_{i}=\psi_{i}=1/(2x_{yi}^{\gamma_{i}})$, and $\gamma_{i}=1$ are the shape parameters of the hysteresis loop; and $x_{yi}=1.25$mm is the yield displacement of the $i$-th storey. The excitation is assumed to be the same as that in Example 2, but the intensity of the white noise is set to $S_0=8.5\times10^{-4} \mathrm{m^2/s^3}$.
	
	\begin{figure}[H]
		\centering
		\includegraphics[scale=0.03]{Example3_Fig1.png}
		\caption{\textbf{A 6-degree-of-freedom shear-type  hysteretic system under ground motion excitation.} \textit{}}
		\label{Fig:Example3_fig1}
	\end{figure}

 The rare event of interest is defined as the peak absolute deformation among the six storeys exceeding a threshold of $0.015$m, expressed by
\begin{equation}
 \left\lbrace y=0.015-\max_{i\in\{1,2,...,6\}}\left(\sup_{t\in[0,15]}|u_i(t)|\right)\leq0\right\rbrace\,.   
\end{equation} 
We define the original and physics-based surrogate models based on solvers of the equation of motion shown in Eq.~\eqref{BoucWen2}. For the original model, Eq.~\eqref{BoucWen2} is solved by the implicit Euler algorithm:
	\begin{equation}\label{ImplicitEuler}
		z_{i}(t+\Delta t)=z_{i}(t)+g_{i}(\dot{u_{i}}(t+\Delta t),z_{i}(t+\Delta t))\Delta t\,,
	\end{equation}
	and for the surrogate model, Eq.~\eqref{BoucWen2} is solved by the explicit Euler algorithm:
	\begin{equation}\label{ExplicitEuler}
		z_{i}(t+\Delta t)=z_{i}(t)+g_{i}(\dot{u_{i}}(t),z_{i}(t))\Delta t\,,
	\end{equation}
	where $\Delta t$ is the time step. The explicit Euler algorithm is highly efficient but less accurate, and thus is ideal in constructing the physics-based surrogate model. The stiffness reduction ratio $\alpha_{i}$, initial stiffness $k_{i}$, and yield displacement $x_{yi}$ are set as tunable parameters for the physics-based surrogate model, i.e., $\vect\theta_p=\{\alpha_{i},k_{i},x_{yi}\}^6_{i=1}$. The initial values of the those parameters are set to be the same as the original model. 
	
	The coupled physics-data-driven surrogate model is trained using $100(\text{initial})+17(\text{active learning})$ samples, and the heteroscedastic Gaussian process model for error correction at the final learning stage is shown in Figure \ref{Fig:Example3_fig2}. The optimization histories of the correlation coefficient ${\rho}$, the first-storey stiffness reduction ratio $\alpha_{1}$, and the probability estimation $\hat{P}$ are shown in Figure \ref{Fig:Example3_fig3}, and their final values are $0.9945$, $0.0844$, and $0.0015$, respectively. The reference solution is $0.0015$, estimated from the Monte Carlo simulation with $10^{5}$ samples. Figure \ref{Fig:Example3_fig4} confirms  the accuracy of the coupled surrogate model. 
	
	\begin{figure}[H]
		\centering
		\includegraphics[scale=0.5]{Example3_Fig2.png}
		\caption{\textbf{Heteroscedastic Gaussian process model of the error correction at the final learning stage for Example 3.} \textit{The surrogate model is trained using $100(\text{initial})+17(\text{active learning})$ samples. The heteroscedastic Gaussian process model captures the noisy errors}.}
		\label{Fig:Example3_fig2}
	\end{figure}
	
	\begin{figure}[H]
		\centering
		\includegraphics[scale=0.4]{Example3_Fig4.png}
		\caption{\textbf{Optimization histories of the proposed method for Example 3.} \textit{(a) the correlation coefficient ${\rho}$ between the responses of the physics-based original and surrogate models at the training points achieves $0.9945$; (b) the first-storey stiffness reduction ratio $\alpha_{1}$ for the physics-based surrogate model converges to $0.0844$; (c) the probability estimation $\hat{P}$ converges to $0.0015$. The reference solution obtained by Monte Carlo simulation with $10^{5}$ samples is $0.0015$, confirming the accuracy of the surrogate model prediction.}}
		\label{Fig:Example3_fig3}
	\end{figure}
	
	\begin{figure}[H]
		\centering
		\includegraphics[scale=0.5]{Example3_Fig3.png}
		\caption{\textbf{Response predictions of physics-based surrogate model and coupled physics-data-driven surrogate model for Example 3.} \textit{The response prediction of the physics-based surrogate model $Y_p$ is highly correlated with the true responses $Y$, and the data-driven error correction corrects the bias. The estimated correlation coefficient between $Y$ and $Y_{p}$ as well as that between $Y$ and $Y_{p}+\mu_{\epsilon}(Y_{p})$ are both around $0.9815$. This is expected because Figure \ref{Fig:Example3_fig2} is linear}.}
		\label{Fig:Example3_fig4}
	\end{figure}
	
	\section{Conclusions}\label{Sec:conclude}
	\noindent This paper develops an effective surrogate modeling method for rare event simulation with high-dimensional input uncertainties. The method circumvents the curse of dimensionality by using physics-based surrogate models parameterized by a few tunable parameters. A data-fitting error correction constructed in the output space of the physics-based surrogate model is leveraged to improve the bias of surrogate modeling. Due to inherent stochastic noise in the errors, the heteroscedastic Gaussian process is adopted to model the error correction. An active learning process is developed to effectively train the coupled physics-data-driven surrogate model to explore the critical region for the rare event. A final importance sampling step is designed to approximate the correction factor for the surrogate model-based probability estimation. Three numerical examples are studied to showcase the performance of the proposed method. The first example considers a static problem of a linear elastic cantilever beam with material properties represented by a Gaussian random field. The second example studies a dynamic problem of a nonlinear viscous damper under stochastic excitation. The third example investigates a multi-degree-of-freedom hysteretic system under stochastic excitation. Three schemes are investigated to construct physics-based surrogate models: a homogenization of material properties is considered in the first example, statistical linearization is used in the second example, and relaxation of the numerical solver is adopted in the third example. The numerical results are highly promising, suggesting that the proposed method can leverage limited calls of the original computational model to effectively estimate rare event probabilities with high-dimensional input uncertainties.  
	
	
	%\section*{Acknowledgement}
	%\noindent
	%Dr. Ziqi Wang was supported by the National Science and Technology Major Project of the Ministry of Science and Technology of China (Grant No. 2016YFB0200605), National Natural Science Foundation of China (Grant No.1808149), and the Natural Science Foundation of Guangdong province (Grant No.2018A030310067). Dr. Marco Broccardo was supported by DESTESS, a projected which has received funding from the European Union's Horizon 2020 research and innovation programme under Grant No.691728. This support is gratefully acknowledged. Any opinions, findings, and conclusions expressed in this paper are those of the authors, and do not necessarily reflect the views of the sponsors.
	
	\bibliography{ReferenceList}
	
	\appendix
	\section{Implementation details}\label{Append:implementationdetails}
	
	\begin{figure}[H]
		\centering
		\includegraphics[scale=0.75]{Appendix1.png}
		\caption{\textbf{Training of the coupled physics-data-driven surrogate model}.}
		\label{Fig:Appendix1}
	\end{figure}
	
	\begin{figure}[H]
		\centering
		\includegraphics[scale=0.82]{Appendix2.png}
		\caption{\textbf{Importance sampling for the coupled physics-data-driven surrogate model}.}
		\label{Fig:Appendix2}
	\end{figure}
	
\end{document} 
\end{document}
