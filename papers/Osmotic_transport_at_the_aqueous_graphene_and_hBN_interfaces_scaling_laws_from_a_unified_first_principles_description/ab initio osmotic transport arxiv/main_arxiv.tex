\documentclass[journal=ancac3,manuscript=article,layout=twocolumn]{achemso}
% ,layout=twocolumn

\setkeys{acs}{maxauthors = 0}      % will list all authors

\setkeys{acs}{articletitle = true}

\usepackage{comment}
\setlength {\marginparwidth }{2cm}
\usepackage{todonotes}
\usepackage{amssymb}
\usepackage{amsfonts}
\usepackage{amsmath}

\usepackage{graphicx}% Include figure files
\usepackage{dcolumn}% Align table columns on decimal point
\usepackage{bm}% bold math

\usepackage[export]{adjustbox}

\usepackage{array}
\newcommand\Tstrut{\rule{0pt}{2.8ex}}
\newcommand\Bstrut{\rule[-1.ex]{0pt}{0pt}}   % = `bottom' strut
\newcolumntype{P}[1]{>{\centering\arraybackslash}p{#1}}


\usepackage[version=3]{mhchem}

\usepackage{graphicx}% Include figure files
\usepackage{dcolumn}% Align table columns on decimal point
\usepackage{bm}% bold math



 \makeatletter
% %\setlength\acs@tocentry@height{13.6cm}
% %\setlength\acs@tocentry@width{10.5cm}
% %\setlength\acs@tocentry@height{6.8cm}
% %\setlength\acs@tocentry@width{4.9cm}
\setlength\acs@tocentry@height{4.cm}
 \setlength\acs@tocentry@width{9.cm}
 \makeatother




\newcommand*\mycommand[1]{\texttt{\emph{#1}}}

\newcommand{\df}{\mathrm{d}}

\newcommand{\kt}{k_\text{B}T}
\newcommand{\eps}{\varepsilon}
\newcommand{\lB}{\ell_\text{B}}
\newcommand{\debye}{\lambda_\text{D}}
\newcommand{\lGC}{\ell_\text{GC}}
\newcommand{\rhoe}{\rho_\text{e}}
\newcommand{\Vs}{V_\text{s}}
\newcommand{\vs}{v_\text{s}}
\newcommand{\zs}{z_\text{s}}
\newcommand{\phis}{\phi_\text{s}}
\newcommand{\phim}{\phi_\text{m}}
\newcommand{\psim}{\psi_\text{m}}
\newcommand{\nm}{n_\text{m}}
\newcommand{\ns}{n_\text{s}}
\newcommand{\Rch}{R_\text{ch}}
\newcommand{\Zch}{Z_\text{ch}}
\newcommand{\Rl}{R_\text{L}}
\newcommand{\Sstr}{S_\text{str}}
\newcommand{\sgn}{\text{sgn}}
\newcommand{\tosm}{\text{to}}
\newcommand{\tel}{\text{te}}


\newcommand{\red}[1]{\textcolor{red}{#1}}
\newcommand{\blue}[1]{\textcolor{blue}{#1}}
\newcommand{\lj}[1]{\textcolor{teal}{#1}}
\newcommand{\rom}[1]{\textcolor{purple}{#1}}
\newcommand{\gt}[1]{\textcolor{orange}{#1}}

\newcommand{\uzh}{Department of Chemistry, Universit\"at Z\"urich, 8057 Z\"urich, Switzerland}
\newcommand{\ilm}{Univ Lyon, Univ Claude Bernard Lyon 1, CNRS, Institut Lumi\`ere Mati\`ere, F-69622, VILLEURBANNE, France}
\newcommand{\iuf}{Institut Universitaire de France (IUF), 1 rue Descartes, 75005 Paris, France}
\newcommand{\tuhh}{Hamburg University of Technology, Insitute of Polymers and Composites, Hamburg, 21073, Hamburg}
\newcommand{\helm}{Helmholtz-Zentrum Hereon, Institute of Surface Science, Geesthacht, 21502, Germany}


\author{Laurent Joly}
\affiliation[Universit\'e Lyon 1]{\ilm}
\alsoaffiliation[IUF]{\iuf}
\author{Robert H. Mei{\ss}ner}
\affiliation[Hamburg University of Technology]{\tuhh}
\alsoaffiliation[Helmholtz-Zentrum]{\helm}
\author{Marcella Iannuzzi}
\affiliation[Universit\"at Z\"urich]{\uzh}
\author{Gabriele Tocci}
\affiliation[Universit\"at Z\"urich]{\uzh}
\email{gabriele.tocci@chem.uzh.ch}
%%%%%%%%%%%%%%%%%%%%%%%%%%%%%%%%%%%%%%%%%%%%%%%%%%%%%%%%%%%%%%%%%%%%%
%% The document title should be given as usual. Some journals require
%% a running title from the author: this should be supplied as an
%% optional argument to \title.
%%%%%%%%%%%%%%%%%%%%%%%%%%%%%%%%%%%%%%%%%%%%%%%%%%%%%%%%%%%%%%%%%%%%%
\title{Osmotic transport at the aqueous graphene and hBN interfaces: scaling laws from a unified, first principles description.
}

\abbreviations{}
\keywords{osmotic transport, blue energy, nanofluidics, electrical double layer, \textit{ab initio} molecular dynamics, two-dimensional materials, graphene, hBN}

\begin{document}


\begin{tocentry}
\includegraphics[width=9.0cm,height=4cm
,clip]{TOC_figure.png}
\end{tocentry}
%

\begin{abstract}
Osmotic transport in nanoconfined aqueous electrolytes provides new venues for water desalination and ``blue energy'' harvesting; the osmotic response of nanofluidic systems is controlled by the interfacial structure of water and electrolyte solutions in the so-called electrical double layer (EDL), but
a molecular-level picture
of the EDL is to
a large extent still lacking.
Particularly, the role of the electronic
structure has not been considered
in the description of electrolyte/surface
interactions.
Here, we report enhanced sampling simulations
based on \textit{ab initio} molecular dynamics,
aiming at unravelling the free energy of prototypical
ions adsorbed at the aqueous graphene and hBN interfaces,
and  its consequences on nanofluidic osmotic transport.
Specifically, we predicted the zeta potential, the diffusio-osmotic
mobility and the diffusio-osmotic conductivity
for a wide range of salt
concentrations from the \textit{ab initio}
water and ion spatial distributions through an analytical
framework based on Stokes equation and a
modified Poisson-Boltzmann equation.
We observed concentration-dependent scaling laws,
together with
dramatic differences in osmotic transport
between the two interfaces, including
diffusio-osmotic flow and current
reversal on hBN,
but not on graphene.
We could rationalize the results for the three osmotic responses with a simple model based on characteristic length scales for ion and water adsorption at the surface, which are quite different on graphene and on hBN.
Our work provides first principles insights into the
structure and osmotic transport of aqueous electrolytes
on two-dimensional materials and
explores new pathways for efficient water
desalination and osmotic energy conversion.
\end{abstract}

\section{Introduction}
Universal access to drinkable water and
 widespread production of electricity
from renewable energy sources are two
of the most daring challenges faced by
modern society. Progress in the field of nanofluidics
offers alternative solutions to
water desalination and to energy conversion
through the mixing of salty and fresh water,
so-called ``blue'' energy harvesting.
Over the past decade, many novel osmotic
transport phenomena have been observed
in nanofluidic
systems.\cite{siria2017new,wang2017fundamental,Tong2021,macha20192d}
Among several noticeable examples is the osmotic
current generated across boron nitride nanotubes
and MoS$_2$ nanopores,
whose power densities exceed by several orders of magnitude
those produced by conventional
membranes\cite{Siria2013,feng2016single}.
A further example is the observation of qualitatively
different current-voltage characteristics
in the nonlinear transport of ions across graphene
and hBN angstrom-scale
slits, which interestingly hints at the critical role
of the crystal and electronic structure of the interface\cite{esfandiar2017size}.
The measurement of conductance oscillations and
Coulomb blockade in sub-nanometers MoS$_2$ pores
also indicates that  chemical nature,
 dimensions and geometry  of nanopores
are key factors to the observed nonlinear behaviours\cite{feng2016observation}.
Thus, it stands to reason that
obtaining a molecular-level picture of
aqueous interfaces is essential to predict and control
osmotic transport phenomena, and may lead to
fundamental advances in the field of nanofluidics.

A comprehensive picture of the structure of
water and electrolyte solutions
at electrified surfaces, in the
so-called electrical double layer (EDL),
has not been obtained so far.
Specifically, the structure of the EDL is not
accurately described using
the standard Gouy-Chapman theory
of the EDL based on the Poisson-Boltzmann (PB)
equation\cite{Hunter2001}.
For instance, in a thin
region of the EDL -- typically of the order of 1\,nm --
ions may interact specifically with solid
surfaces. Ion-specific
effects in this region, which we define ``adsorption layer'', are not captured
by the standard PB equation\cite{luo2006ion,huang2007ion}.
Additionally, the PB theory
of the EDL typically neglects the polar nature
of water and water layering at the liquid/solid
interface, and ignores
spatial and dynamic interface heterogeneities\cite{gonella2021water,Hartkamp2018,Markovich2016,Bonthuis2013,limmer2013hydration}.
Experimentally, a vast array of techniques have been used to probe
the structure of the EDL\cite{Hartkamp2018}. Recent examples include
second harmonic generation, which has
revealed structural and dynamical
heterogeneities in the water orientation
at the interface with silica\cite{macias2017optical},
and ambient pressure X-ray photoelectron spectroscopy,
which has probed the shape of the electrostatic potential profile of aqueous electrolytes
on gold electrodes at different concentrations\cite{favaro2016unravelling}.
Despite the tremendous advancement these types of work represent
for the field, achieving sub-nanometer resolution, which is required
to characterise the molecular structure of the EDL,
remains an open experimental challenge.


A further challenge is to link the  structure of aqueous  interfaces to osmotic transport properties. Although experiments have hinted at the predominant role of charged groups, pore geometry and pore chemistry\cite{Hartkamp2018}, a microscopic characterization of the interface under operating conditions has not been obtained.
Alternatively, atomistic simulations, and in particular molecular dynamics, can be used to explore the structure of the EDL at the molecular level.
Force-field molecular dynamics (FFMD), which is based on an empirical description of the interactions between the constituent atoms, has yielded invaluable insights into the structure of the EDL \cite{Siepmann1995InfluenceSystems, Scalfi2020ChargeEnsemble,scalfi2020a,Limmer2013ChargeCapacitors} and into the molecular mechanisms underlying liquid and solute transport in nanofluidics\cite{striolo2016carbon,phan2016confined,Faucher2019,Falk2010MolecularFriction,Ma2015WaterFriction,xie2018fast,huang2007ion,Ajdari2006,heiranian2015water,noh2020ion,liu2018pressure,simoncelli2018blue,Kalra2003OsmoticMembranes,Vasu2018ElectricallyMembranes}
%
Yet, it is challenging to determine accurate force fields that incorporate
electronic structure effects observed at complex liquid/solid interfaces.
In contrast,
\textit{ab initio} molecular dynamics (AIMD) simulations
based on density functional theory (DFT)
are instrumental to compare
the structure and dynamics of different aqueous
interfaces on equal footing.
Although AIMD is increasingly being used
to characterise the structure of
aqueous interfaces
\cite{le2020molecular,cheng2012alignment,gross2019modelling,lan2020ionization,seiler2018effect},
including OH$^{-}$ adsorption on two-dimensional
materials\cite{Grosjean2016},
H$_3$O$^+$ adsorption on TiO$_2$\cite{stecher2016first}
and ion adsorption at the liquid/vapour interface\cite{duignan2021toward,baer2011toward},
the impact of ion- and water-specific interactions
on osmotic transport at aqueous interfaces has not
been investigated with AIMD so far.


In this work we tackle two
challenges: we determine the structure of model aqueous
electrolyte interfaces from \textit{ab initio} methods and we compute the
mobilities underlying osmotic transport processes due to different applied external fields
(see Fig.~\ref{fig:schematic}). Our systems consist of a potassium iodide (KI) solution at the
aqueous graphene and  hBN interfaces.
We chose the aqueous graphene and hBN
interfaces because they are well-studied model systems in
nanofluidics and for their potential impact
as nano-osmotic power generators\cite{macha20192d}.
Also, we focus on KI
as a model electrolyte displaying ion-specific
effects\cite{schwierz2010reversed,huang2007ion,baer2011toward}.
The structure of the EDL and osmotic transport coefficients are computed
according to the following steps: First, we calculate the free
energy of adsorption
of K$^+$ and I$^-$ ions dissolved in a 2 nanometer-thick
water film using enhanced sampling
techniques based on AIMD  (a snapshot of a
representative system is shown in the inset of
Fig.~\ref{fig:free_energy}(a)); Second, we obtain the electrostatic potential profile and the
spatial distributions of the ions
at different salt concentrations by solving
a modified Poisson-Boltzmann (mPB) equation that
accounts for the ions' free energy of
adsorption on the sheets;
Finally, osmotic transport coefficients are obtained, within linear response theory,
by computing the relevant fluxes resulting from each external field, based on Stokes equation with a slip boundary condition at the wall.
We find remarkable ion- and surface-specific
adsorption of KI at the graphene and hBN interface. Such specific effects give rise
to concentration-dependent scaling laws of the
osmotic transport coefficients and result into strikingly
different osmotic transport behaviour at the graphene
and hBN interface. We rationalize the obtained scaling laws
with a theoretical model that describes
ion and water adsorption in terms of characteristic
length-scales that are limited to a few molecular diameters.



\section{Results}
\subsection{Unified framework of osmotic transport}
One of the main objectives of this work is to provide a theoretical
framework to calculate the osmotic transport coefficients
from the molecular structure of water and ions at liquid/solid interfaces.
We do so within linear response theory, where the
system of equations shown in Fig.~\ref{fig:schematic}
describes the thermodynamic fluxes resulting from applied external forces.
The desired transport coefficients
are the off-diagonal elements of the matrix
shown in the middle of Fig.~\ref{fig:schematic},
and obey Onsager's reciprocal relations\cite{Onsager1931a,Onsager1931b},
\textit{i.e.} the matrix is symmetric.
The osmotic transport coefficients
are computed from  hydrodynamics through
the Stokes equation, in which
the velocity profile is obtained
using a partial slip boundary condition at the wall $v(z=0) = b \partial _z v(z=0)$, with $b$
the slip length\cite{Bocquet2007}.
Throughout this work we thus assume
that continuum hydrodynamics
is valid at the nanoscale and
that the viscosity is homogeneous.
%%%UNCOMMENT FOR FINAL SUBMISSION TO ACS NANO
% {\color{red}While on hydrophilic and charged surfaces, osmotic flows are better described by assuming a larger viscosity in a subnanometric interfacial layer\cite{Bonthuis2013,rezaei2021interfacial},}
%%% CHANGE On hydrophobic TO on hydrophobic WHEN SUBMITTING THE FINAL VERSION TO ACS NANO
On hydrophobic, slipping surfaces such as the ones considered in this work, our assumptions have been shown to provide an accurate description of the velocity profiles even in the first molecular layers of the liquid.\cite{Bonthuis2013}

\begin{figure*}[thb!]
%\includegraphics[width=\textwidth,trim={2cm 7cm 2cm 7cm},clip]{Schematic_Final_version.png}
\includegraphics[width=\textwidth]{Schematic_Final_version.jpg}
\caption{\label{fig:schematic}
Schematic of the linear system of equations for osmotic transport.
The off-diagonal elements of Onsager
transport matrix represent the osmotic transport
coefficients, and are the central
quantities obtained in this work. They are computed
from the linear response of the fluxes,
(vector schematically shown on the left)
to an applied external force (vector on the right).
The matrix elements are color-coded according
to the color labeling of the respective flux, i.e., blue, violet and pink for
the elements due to a volumetric flow rate $Q$, an
excess solute flux $\delta J_s$ and an electrical current $I_e$, respectively.
From top to bottom, the external force vector on the right labels a pressure
gradient, a concentration gradient and an electrostatic potential gradient.
The sheets schematically depict a slit geometry, with cations and anions shown in
blue and red. Schematic inspired from Ref.~\citenum{marbach2019osmosis}. }
\end{figure*}


We start with electro-osmosis (EO), the flow generated by an electric field along a nanochannel, whose transport coefficient is illustrated in the top-right element of the matrix in Fig.~\ref{fig:schematic}.
%
The electro-osmotic response is commonly quantified by the so-called zeta potential $\zeta$, which relates the electro-osmotic velocity in the bulk liquid $v_\text{eo}$ to the electric field along the channel $E$ through the Helmholtz-Smoluchowski relation: $v_\text{eo} = -(\eps \zeta / \eta) E$, with $\eps$ and $\eta$ the permittivity and viscosity of the liquid in bulk, respectively.
%
According to Onsager's reciprocal relations, $\zeta$ also quantifies the streaming current density $j_e$ generated by a pressure gradient $-\nabla p$ along the channel: $j_e = -(\varepsilon \zeta / \eta) (-\nabla p)$.
From hydrodynamics equations,
one can relate the $\zeta$-potential to the charge density profile at the interface
(see \textit{e.g.} Refs.~\citenum{huang2007ion,marbach2019osmosis} and the supporting information (SI)):
\begin{equation}
\label{eq:zeta}
    \zeta = - \frac{1}{\eps} \int_0^{\infty}
    (z + b) \rhoe(z) \,\df z ,
\end{equation}
where the charge density
distribution is given by
$\rhoe = q_e\left(n_+ - n_-\right)$, with $n_+$ and
$n_-$ the cation and anion number
densities, respectively,
and $q_e$ the elementary charge.
Note that no assumption was made on the dielectric permittivity of the system: the bulk dielectric permittivity $\eps$ only appears in Eq.~\eqref{eq:zeta} through the Helmholtz-Smoluchowski definition of $\zeta$.

We move on to introduce diffusio-osmosis (DO), the flow generated by a gradient of salt concentration along the channel\cite{anderson1989colloid} (see Fig.~\ref{fig:schematic}). Diffusio-osmosis can be quantified by the so-called diffusio-osmotic mobility $D_\text{DO}$, which relates the diffusio-osmotic velocity in the bulk liquid $v_\text{do}$ to the gradient of salt concentration $\ns$: $v_\text{do} = D_\text{DO} (-\nabla \ns / \ns)$.
Employing again Onsager's reciprocal relations, $D_\text{DO}$ also quantifies the streaming excess solute flux density $\delta j_s$
generated by a pressure gradient
along the channel,
see details in Ref.~\citenum{Ajdari2006} and in the SI.


From hydrodynamics equations, one can relate $D_\text{DO}$ to the ionic density profiles $n_\pm(z)$ and to the water density profile $n_\text{w}(z)$ (normalized by its bulk value $n_\text{w}^\text{b}$) at the interface (see the SI):
\begin{multline}\label{eq:D_DO2}
    D_\text{DO} = \frac{\kt}{\eta} \int_0^{\infty} \left( z + b \right) \times \\ \left\{ n_+(z) + n_-(z) - 2 \ns \frac{n_\text{w}(z)}{n_\text{w}^\text{b}} \right\} \, \mathrm{d}z ,
\end{multline}
with $k_\text{B}T$ the thermal energy.
The integral expression for  $D_\mathrm{DO}$ in Eq.~\eqref{eq:D_DO2} is a first
central result of this work (see the derivation in the SI).
In particular, the contribution of the water density profile, which is usually ignored in the theoretical expressions of $D_\mathrm{DO}$ \cite{Mouterde2018,marbach2019osmosis}, makes a key difference, as ignoring it leads to a spurious negative contribution to the integral in the vacuum-like
region between the first water layer and the surface.
%
Note that an additional flow can be generated under a salt concentration gradient: for salt ions with an asymmetric diffusivity, a so-called diffusion electric field $E_0$ appears to avoid charge separation, creating an electro-osmotic flow, which adds to the intrinsic diffusio-osmotic flow\cite{anderson1989colloid,Lee2014b}. However, as discussed in the SI (see Fig.~S3), this
electro-osmotic component is negligible
in the systems considered here.


Finally, a gradient of salt concentration along the channel also generates an electric current, called diffusio-osmotic current (see Fig.~\ref{fig:schematic}), which is proportional to the perimeter of the channel cross section $P$. In order to quantify the intrinsic response of the liquid-solid interface, independently of the channel geometry, we therefore define the so-called diffusio-osmotic conductivity $K_\text{osm}$, which relates the diffusio-osmotic current generated per unit length of the channel circumference, $I_e/P$, to the gradient of salt concentration $\ns$: $I_e/P = K_\text{osm} (-\nabla \ns / \ns)$; note that slightly different definitions can be found in the literature\cite{Siria2013,marbach2019osmosis}.
According to Onsagers' reciprocal relations, $K_\text{osm}$ also quantifies the excess solute flux
generated by an electric field along the channel, see details in the SI.
%
From hydrodynamics equations, and assuming a homogeneous dielectric permittivity (this assumption will be discussed later in the article),
one can express $K_\text{osm}$ as:
\begin{multline}\label{eq:K_osm2}
K_\text{osm} = \frac{\kt q_e}{4 \pi \lB \eta} \int_0^{\infty}  \left[ \phi(z) - \phis -  \frac{2 \, \sgn(\Sigma) b}{\lGC} \right] \\ \times
\left\{ n_+(z) + n_-(z) - 2 \ns \frac{n_\text{w}(z)}{n_\text{w}^\text{b}} \right\} \,\mathrm{d}z,
\end{multline}
using the same notations as for Eq.~\eqref{eq:D_DO2}, and with $q_e$ the absolute ionic charge, $\phi(z) = q_e V(z) / (k_\text{B}T)$ the reduced electrostatic potential, $\phis$ its value at the surface, $\lB = q_e^2/(4\pi\eps\kt)$ the Bjerrum length (at which the
electrostatic interaction between two ions is comparable to the thermal energy),
and $\lGC = q_e/(2\pi \lB |\Sigma|)$ the Gouy-Chapman length\cite{Herrero2021}.
We remark that the expression for $K_\text{osm}$ shown in Eq.~\eqref{eq:K_osm2}
is the second important result of this work (see the SI for a complete
derivation).

The ions' density and the  electrostatic potential profiles
appearing in Eqs.~(\ref{eq:zeta}-\ref{eq:K_osm2})
are determined from the solution of the
Poisson-Boltzmann equation \cite{Hunter2001,Schoch2008},
which is used to describe the EDL near
electrified interfaces and is modified to include
the free energy of ion adsorption \cite{schwierz2010reversed,luo2006ion},
here computed from first principles simulations:
\begin{multline}\label{eq:mPB}
    \mathrm{d}^2_z \phi(z) = - 4 \pi \lB \left[ n_+(z) - n_-(z) \right] \\
    =- 4 \pi \lB  \ns \left[ e^{-\phi(z) - g_+ (z)} - e^{ \phi(z) - g_- (z) } \right],
\end{multline}
where the dimensionless free energies of ion adsorption
$g_{\pm}(z)= \Delta G_{\pm}(z) /(\kt)$ are the key terms that distinguish
Eq.~\eqref{eq:mPB} from the standard PB description of the EDL,
and which importantly enable the possibility for non-zero solutions of Eq.~\eqref{eq:mPB} even in the absence of charged surfaces,
as it is the case in our
aqueous graphene and hBN interfaces.
%%%UNCOMMENT FOR THE FINAL SUBMISSION TO ACS NANO
% Although the form of Eq.~\eqref{eq:mPB} assumes a constant value of the dielectric constant
% $\varepsilon$,  we have also considered
% its possible spatial
% dependence at the interface\cite{Bonthuis2013,rezaei2021interfacial} using a step model\cite{schwierz2010reversed,huang2007ion}
% and we have computed the transport coefficients
% within this model in the SI (see Fig.~S4).
%%%COMMENT THE SENTENCE BELOW FOR SUBMISSION TO ACS NANO. THE SENTENCE BELOW IS GOOD FOR ARXIV
Although
the form of Eq. (4) assumes a constant value
of the dielectric permittivity $\varepsilon$, we have also
considered a step model of
the dielectric constant\cite{schwierz2010reversed,huang2007ion} and we have computed the transport coefficients within this model in the SI (see Fig.~S4).
%
Whereas the electro-osmotic
and diffusio-osmotic coefficients are
not affected by the particular choice
made for $\varepsilon$, the
magnitude of the diffusio-osmotic
conductivity is altered depending
on the  model used for the dielectric constant;
still, the scaling as a function of concentration  and the change of sign remain the same, thus not
affecting the conclusion of our work (see the SI).

\subsection{Ion adsorption and water density oscillations
from first principles}
Thus, we start to discuss our simulation results by
presenting the free energy profiles
of I$^{-}$ and K$^{+}$ and the water density
profiles on graphene and hBN.
Figs.~\ref{fig:free_energy}(a) and (b)
display the free energy of ion adsorption
obtained from our \textit{ab initio} umbrella sampling
simulations. Significant ion- and surface-specific
adsorption can be observed,
which are limited to a
region of about 1\,nm from the sheets.
Further, the free energy of adsorption of the K$^+$
ion is essentially the same on
graphene and hBN, and whilst the signature of a local minimum appears
at a height of about 0.4\,nm from the sheets, K$^+$ is clearly more stable
in the bulk water region. On the other hand,
the free energy profile of I$^-$ exhibits pronounced differences
on graphene and hBN: a global minimum of about
$-0.08$ eV and $-0.06$\,eV is observed on graphene and hBN, respectively,
where I$^-$ is physisorbed at a height of about 0.4\,nm on both sheets.
Direct anion-substrate interactions due to van der Waals
dispersion forces are likely responsible for the
observed minimum, as reported in previous
FFMD simulations and second harmonic generation experiments
performed on closely related systems.\cite{mccaffrey2017mechanism}
Interestingly, an energy barrier of
about 0.07\,eV is observed for I$^-$
on hBN between the adsorption minimum at a height of 0.4\,nm
and the bulk water region at about 1\,nm from the sheet.
On graphene instead, a free energy barrier is not
observed. Analysis of the dipole orientation of water at the
two interfaces (reported in the SI in Fig.~S2) suggests that subtle
differences in the water dipole orientation
on the two sheets may be
responsible for the distinct features of the
free energy profile of I$^{-}$
on graphene and on hBN.
While some changes in the water dipole orientations on the
two sheets are noticed,
the water spatial density distribution
is remarkably similar on graphene and hBN and
presents strong oscillations
that are gradually suppressed above 1.0-1.2\,nm
from the surface, as highlighted in previous work\cite{tocci2014friction}.
In Eq.~\eqref{eq:mPB},
the ion density distributions, the
electrostatic potential profile $\phi(z)$
and its value at the sheets
$\phis$ depend non-linearly
on the bulk salt concentration $\ns$
through the free energy profiles shown in
Fig.~\ref{fig:free_energy}(a) and (b).
Both the free energy profiles of
the ions adsorbed on the sheets
and the water density profile $n_\text{w}(z)$
enter the transport integral expressions (Eqs.~(\ref{eq:zeta}-\ref{eq:K_osm2}))
and are therefore key to
understand osmotic transport at the interface.

\begin{figure}
\includegraphics[width=0.42\textwidth]{free_energy_vs_height.png}
\caption{\label{fig:free_energy}
Free energy of I$^-$ and K$^+$ adsorption
on (a) graphene and (b) hBN as a function of the height from the sheets,
and (c) water density profile on graphene and hBN.
The inset in (a) is a representative snapshot
of the water film simulation with a
iodide ion (blue) adsorbed on graphene (cyan).
The free energy profile of K$^+$
is similar on graphene and hBN and
does not show an adsorption minimum, in contrast with I$^-$ adsorption, which
reveals a minimum near the sheets
and a barrier on hBN, but not on graphene.}
\end{figure}



\subsection{Concentration dependence of osmotic transport}
The observed ion- and surface-specific
adsorption and the layering  of the water density
profiles have pronounced effects
on the osmotic transport coefficients.
Slippage can also enhance osmotic
transport dramatically\cite{Ajdari2006}, as it can be
evinced from Eqs.~(\ref{eq:zeta}--\ref{eq:K_osm2}).
In this work, we used slip length values
taken from previous AIMD results\cite{tocci2020nanoscale}.
Water flows significantly faster
on graphene ($b=19.6$\,nm) than on hBN ($b=4.0$\,nm),
and we wish to explore the consequences this
bears for osmotic transport.

The osmotic transport coefficients
are displayed in Fig.~\ref{fig:transport_coefficients}:
Each transport coefficient follows
a different asymptotic behaviour
as a function of salt concentration
and remarkable changes are noticed also
between graphene and hBN.
%
\begin{figure*}[thb!]
\includegraphics[width=1\textwidth]{Transport_coefficients.png}
\caption{\label{fig:transport_coefficients}
Molecular representation and concentration dependent
scaling of osmotic transport at the aqueous
graphene and hBN interfaces.
Schematic of the electro-osmotic velocity
profile $v_\mathrm{eo}(z)$ arising from
an electric field $E$ (a) and absolute
value of the $\zeta$ potential as a function of
salt concentration (b);
Schematic of the diffusio-osmotic velocity
profile $v_\mathrm{do}(z)$ in response to a concentration gradient $-k_\mathrm{B}T (\nabla\ns)/\ns$ (c) and absolute
value of the diffusio-osmotic coefficient $|D_\mathrm{DO}|$ (d);
Representation of the
diffusio-osmotic current $I_e$ arising from a concentration gradient
$-k_\mathrm{B}T (\nabla\ns)/\ns$ (e), and
absolute value of the diffusio-osmotic conductivity $|K_\mathrm{osm}|$ (f).
The transport coefficients display different
 scaling behaviours (see dashed lines).
The symbols are obtained from the numerical
integration of Eqs.~(\ref{eq:zeta}-\ref{eq:K_osm2}),
whereas the solid and dotted lines are from the effective surface charge model.
In (d) and (f) the arrows point to a
sign change, whereby $D_\text{DO}$
and $K_\text{osm}$
are negative on hBN at concentrations $\gtrsim 1$\,M, as indicated also
by the empty symbols and dotted line in (d).
In (a) $v_\mathrm{eo}(z)$ is not enhanced by slippage,
opposite to $v_\mathrm{do}(z)$ in (c) and (e), see text for details.}
\end{figure*}
%
Fig.~\ref{fig:transport_coefficients}(a)
is a schematic representation of electro-osmosis
for the systems considered here, and Fig.~\ref{fig:transport_coefficients}(b)
shows the absolute value of the
$\zeta$-potential as a function of
concentration. The absolute value of the $\zeta$ potential
increases from about $0.6$\,mV to
$40$\,mV on graphene and
from about $0.1$\,mV to
$20$\,mV on hBN.
A non-zero value of the $\zeta$
potential even in absence of a surface
charge highlights the
role of ion-specific adsorption on electro-osmosis,
as reported in the past by
electro-phoretic experiments and FFMD
simulations\cite{huang2007ion,petrache2006salt}.
It is also worth noting that
the slip contribution in Eq.~\eqref{eq:zeta} writes $-b/\eps \int_0^\infty \mathrm{d}z \rhoe = b \Sigma / \eps$, so that it vanishes for neutral surfaces\cite{huang2007ion}.
Thus, based on the form of Eq.~\eqref{eq:zeta},
it can be evinced that
both the magnitude and the different concentration dependence
of $\zeta$ on graphene and hBN solely arise from
differences in the free
energy profile of adsorption
of I$^{-}$ on the two sheets,
given that the free energy of K$^{+}$
is very similar on graphene and hBN.

A schematic representation of diffusio-osmotic
flow is illustrated in
Fig.~\ref{fig:transport_coefficients}(c) and
the concentration dependence of the
diffusio-osmotic coefficient $D_{\text{DO}}$
is presented in Fig.~\ref{fig:transport_coefficients}(d).
Whereas $D_\mathrm{DO}$ scales linearly
on graphene at all concentrations, deviations from
a linear asymptotics is noticed
on hBN already around $10^{-1}$\,M, and  above
 $1$\,M $D_\mathrm{DO}$ becomes negative (a positive value of $D_\mathrm{DO}$ indicates that the
diffusio-osmotic flow proceeds from high to low
concentration and vice-versa for negative $D_\mathrm{DO}$). Noting also that the K$^{+}$
free energy profiles
and the water spatial distribution
are very similar on graphene and hBN,
it is clear that $D_\mathrm{DO}$
changes sign on hBN because of the
different free energy of adsorption of I$^{-}$
from graphene.
We note that a liquid flow that proceeds
towards a larger concentration
has been observed before for neutral solutes,
which only interact specifically with the surfaces.\cite{Lee2014b,Lee2017}.
Opposite to electro-osmotic transport,
diffusio-osmotic transport is amplified
by slippage even in the absence of a surface
charge (see Eq.~\ref{eq:D_DO2}) and
a strong slip-induced enhancement of
the diffusio-osmotic flow
has been reported before\cite{Ajdari2006}.
 The larger slip-length, along with the
  absence of an adsorption barrier
of I$^{-}$ are the two main reasons
why $|D_\mathrm{DO}|$ is larger on graphene than on hBN.

Finally, a schematic
of the diffusio-osmotic
current mechanism is presented in
Fig.~\ref{fig:transport_coefficients}(e),
and the  scaling behaviour of the diffusio-osmotic
conductivity is shown in Fig.~\ref{fig:transport_coefficients}(e).
Below $10^{-1}$\,M, the diffusio-osmotic conductivity
exhibits a different scaling behaviour
with salt concentration on graphene
and hBN, in contrast to what was observed in the
case of the diffusio-osmotic coefficient.
On graphene, $K_\mathrm{osm}$
scales more slowly than on hBN, but it
remains positive at all concentrations considered.
On hBN on the other hand, a maximum in $K_\mathrm{osm}$ is observed
just below 1\,M, above which an abrupt sign change is
observed, similar to what observed for $D_\mathrm{DO}$.
Finally, in Eq.~\eqref{eq:K_osm2},  we note that in contrast to
$D_\mathrm{DO}$, slippage does not contribute to $K_\mathrm{osm}$
in the absence of a surface charge, for which $1/\lGC \propto \Sigma = 0$.
This is because slip shifts the DO velocity profile by a constant amount $v_\text{slip}$, so that the corresponding electrical current writes: $I_e \propto v_\text{slip} \int_0^\infty \mathrm{d}z \rhoe \propto v_\text{slip} (-\Sigma)$, and vanishes for neutral surfaces.

\subsection{Scaling laws of the osmotic transport coefficients}
To rationalize the different scaling properties,
we introduce an effective surface charge (ESC) model
of the EDL. The ESC model is discussed in
full detail in the SI, but here we
present the essential idea.
%
In this model, the structure of the EDL
is mapped to a standard PB description,
where the effects associated to ion-specific
adsorption are treated as surface terms by introducing
an effective surface charge $\Sigma_\mathrm{eff}$.
$\Sigma_\text{eff}$ is defined as
$\Sigma_\text{eff} = q_e \ns \left (e^{-\phi_\mathrm{s}} K_+ - e^{\phi_\mathrm{s}}K_- \right)$ and it
depends on the bulk salt concentration $\ns$,
on the surface potential $\phi_\mathrm{s}$ and on the characteristic
lengths $K_{\pm}$, which quantify the excess or
depletion of ions near the sheets (see the definition in Table~\ref{table:1}).
It will be shown that the
density oscillations of interfacial
water also enter this modified formulation
and are relevant to capture
diffusio-osmotic transport.
The key to this simplified description is that
the region where ions and water interact specifically
with the sheets is very thin compared to the EDL.
As such, the ESC model is strictly
valid when there is a net separation between the
water and ion adsorption length-scales
and those of the EDL, given by the Debye length and
the effective Gouy-Chapman length, which we define here in terms
of the effective surface charge
$\lGC^\text{eff}=q_e/\left(2\pi\lB |\Sigma_\mathrm{eff}|\right)$.
In the following, we will apply the ESC model to
Eqs.~(\ref{eq:zeta}-\ref{eq:K_osm2})
to rationalize the scaling behaviours observed in
Fig.~\ref{fig:transport_coefficients}.

Starting from electro-osmosis, the $\zeta$-potential
can be expressed within the ESC model, in the limit of small reduced surface potentials $\phis \ll 1$ (see the SI), as:
\begin{equation}
\label{eq:zeta_mGC}
     \zeta \approx \frac{\kt}{q_e} \phis \approx \frac{\kt}{q_e} \cdot \frac{   K_+ - K_- }{  2\lambda_D + K_+ + K_- }.
\end{equation}
The results of this equation are shown as solid lines in
Fig.~\ref{fig:transport_coefficients}(b).
To understand the two different limits observed in
the figure, it is instrumental to inspect
the values of the ion-specific length-scales
$K_+$ and $K_-$ entering the equation, which are
listed in Table~\ref{table:1}.
For both graphene and hBN, $K_+$ is negative
and indicates a net depletion of cations near the sheets,
whereas $K_-$ is positive and indicates a net accumulation
of anions. Noting also that for hBN,
$|K_+ + K_-| \ll \lambda_\mathrm D$ at all
concentrations,  Eq.~\eqref{eq:zeta_mGC}
simplifies to $\zeta \approx \kt (K_+ - K_-)/(2\lambda_D q_e)$, such that
$\zeta$ scales as the inverse of the Debye length, or equivalently
as  the square-root of the salt concentration, \textit{i.e.}, $\zeta \sim 1/\lambda_\mathrm D \sim \sqrt{\ns}$.
For graphene on the other hand,
departure from $\zeta\sim\sqrt{\ns}$ is visible
already at a concentration above $10^{-3}$\,M
because the term $|K_++K_-|$ becomes comparable to $\lambda_\mathrm D$.
Therefore, approximately above
$10^{-3}$\,M the scaling of $\zeta$ on graphene is better
captured by the full Eq.~\eqref{eq:zeta_mGC}.
Beyond $10^{-1}$\,M for graphene and beyond 1\,M for hBN,
the prediction from Eq.~\eqref{eq:zeta_mGC} deviates
from the numerically integrated results
because the surface potential
becomes of the order of the thermal voltage
(\textit{i.e.} $\phis \gtrsim 1$).

%
\begin{table*}[t!]
  \centering
\begin{tabular}{ P{3.5cm}P{5.0cm}P{3.5cm}P{3.5cm}}
\hline
\hline
  \centering
 lengths [nm]&definition&graphene&hBN\Tstrut\Bstrut\\
 \hline
 \Tstrut\Bstrut
   $\bm{b}$&  &$\bm{+19.6}$&$\bm{+4.0}$\Tstrut\Bstrut\\
   $K_+$&$\int_0^{\infty}  [e^{-g_+(z)}-1]\mathrm{d}z$&$-0.229$&$-0.207$\Tstrut\Bstrut\\
   $\bm{K_-}$&$\int_0^{\infty}  [e^{-g_-(z)}-1]\mathrm{d}z$&$\bm{+1.974}$&$\bm{+0.177}$\Tstrut\Bstrut\\
   $K_w$&$\int_0^{\infty}  [ n_w(z)/n_w^{b} -1]\mathrm{d}z$&$-0.179$&$-0.165$\Tstrut\Bstrut\\
   $L_+$&$K_+^{-1}\int_0^{\infty} z[e^{-g_+(z)}-1] \mathrm{d}z$&$-0.363$&$-0.442$\Tstrut\Bstrut\\
      $\bm{L_-}$&$K_-^{-1}\int_0^{\infty} z[e^{-g_-(z)}-1] \mathrm{d}z$&$\bm{+0.459}$&$\bm{-0.099}$\Tstrut\Bstrut\\
   $L_w$&$K_w^{-1}\int_0^{\infty}  z[ n_w(z)/n_w^{b} -1]\mathrm{d}z$&$-0.031$&$-0.027$\Tstrut\Bstrut\\
\hline
\hline
\end{tabular}
\caption{Slip length $(b)$ and length-scales characteristic
of cation-specific (with subscripts ``$+$''),
anion-specific adsorption (with subscripts ``$-$'') and water
density oscillations (with subscript ``$w$'')
at the aqueous graphene and hBN interfaces,
along with their definitions.
The slip length and the anion length-scales are
in bold to highlight the stark differences between graphene and hBN.}
\label{table:1}
\end{table*}
Application of the ESC model to diffusio-osmosis
(Eq.~\eqref{eq:D_DO2}) yields the following
relation for $D_\text{DO}$ (derived in the SI):
\begin{multline}
\label{eq:D_DO_mGC}
    D_\text{DO}
    \approx \frac{\kt}{2\pi\lB\eta} \left\{ -\ln\left(1-\gamma^2\right) + \frac{b|\gamma|}{\lGC^\text{eff}} + \right. \\
    \left. \frac{1}{4\debye^2} \left[ e^{-\phis} K_+ L_+ + e^{\phis} K_- L_- -2 K_\text{w} L_\text{w} + \right. \right. \\  \left. b \left( e^{-\phis} K_+ + e^{\phis} K_- -2 K_\text{w} \right) \right] \bigg \}.
\end{multline}
%
The first two terms depending on $\gamma$, with
$\gamma = \tanh (\phis/4)$,
result from a calculation of
the diffusio-osmotic flow according to the standard PB description of
the EDL\cite{anderson1989colloid,Mouterde2018}
and arise from the region in the EDL beyond the adsorption layer, whereas
the remaining terms involving the ion length-scales
($K_\pm$ and $L_\pm$) and the water length-scales
($K_w$ and $L_w$) arise from specific
interactions within the adsorption layer.
Along with $K_{\pm}$, the additional ion-specific
length-scales $L_{\pm}$ contributing to Eq.~\eqref{eq:D_DO_mGC}
(see definition in Table~\ref{table:1})
represent the characteristic thicknesses over which cations and anions adsorb.
The water-specific length-scale $K_w$ is characteristic of
a net accumulation or depletion of water near the
sheets, while $L_w$ represents
the characteristic size
over which water accumulates/depletes
at the aqueous interface (see definition in Table~\ref{table:1}).
By analysing the relevant terms in
Eq.~\eqref{eq:D_DO_mGC}, we can
interpret the scaling behaviour observed in
Fig.~\ref{fig:transport_coefficients}(c),
while pointing out that in the equation
$\lambda_\text{D}$, $l^{\text{eff}}_{\text{GC}}$,
$\gamma$ and $\phis$ all depend on concentration.
First, the
terms depending on $\gamma$ are
negligible compared to the adsorption
layer terms at small surface
potentials ($\phis <1$)
and at concentrations
below $\sim 10^{-1}$\,M
because to leading order
$\gamma \sim \phis$ whereas $e^{\pm\phis} \sim 1$.
Therefore, at low concentrations
a linear dependence of $D_\text{DO}$
can be readily understood based on the dependence
of the adsorption term proportional
to $\lambda_\mathrm D^{-2}$
 by realizing that $D_\mathrm{DO} \sim \lambda_\mathrm D^{-2} \sim \ns$.
%
Also, examining the characteristic
length-scales
listed in Table~\ref{table:1}, one notices  that
the terms involving $e^{\mp \phis }K_{\pm} L_{\pm}$ and
$2 K_{\text{w}} L_{\text{w}}$  can be neglected
compared to the factor proportional to $b$
 because the slip length is much larger than
 $|L_\pm|$ and $|L_{w}|$.
As such, $D_\mathrm{DO}$ is enhanced by slippage through
the slip length $b$ and the magnitude of $D_\mathrm{DO}$
is larger on graphene than on hBN partly
because of the larger slip length
in the former material.
As the concentration is
increased around 1\,M the PB term and in
particular the slip contribution $b|\gamma|/\lGC^{\text{eff}}$
can no longer be neglected.
Interestingly, competing effects between
the $b|\gamma|/\lGC^{\text{eff}}$ term, the ions' adsorption
term and the molecular-scale oscillations
 of water can give rise to a sign reversal
 in $D_\text{DO}$ at a critical
 concentration.
Reversal of diffusio-osmotic flow is
indeed observed on hBN,
but not on graphene, due to the
differences in the slip length and in the
ion adsorption length-scales between the two materials
(see Table \ref{table:1}).
On graphene,
the ion-specific adsorption terms
proportional to  $e^{-\phis}K_+ + e^{\phis}K_+$ and
the term proportional to $b|\gamma|/\lGC^{\text{eff}}$
dominate over the water contribution proportional to
$2K_w$ at all concentrations
and the diffusio-osmotic
flow always proceeds in the direction from high to low
concentrations. On hBN, instead, above a
critical concentration of about 1\,M
the water term becomes the dominant contribution
and a flow reversal is observed.
A more detailed analysis on the scaling behaviour
of $D_{\text{DO}}$ is provided in Fig.~S5
in the SI, where the diffusio-osmotic coefficient has been explicitly decomposed into the standard
PB contribution and the adsorption layer contribution.

Finally, an approximate expression for the diffusio-osmotic conductivity
$K_\mathrm{osm}$ is obtained to understand the scaling behaviour observed
in Fig.~\ref{fig:transport_coefficients}(c) as:
\begin{multline}
\label{eq:kosm_mGC}
    K_\mathrm{osm} \approx - \frac{\kt \Sigma_\text{eff}}{2 \pi \lB \eta} \times  \bigg ( 1  -\frac{\text{asinh} (\chi)} {\chi} \\
  + \frac{ e^{-\phis} \tilde{K}_+\tilde{L}_+   + e^{\phis} \tilde{K}_-\tilde{L}_-   -2 \tilde{K}_\text{w} \tilde{L}_\text{w}  }{4 \lambda_\mathrm D^2} \bigg ).
\end{multline}
The term $1-\text{asinh} (\chi)/\chi$
arises from the region in the EDL
beyond the adsorption layer,
with $\chi=\debye/\lGC^\text{eff}$\cite{Siria2013,Mouterde2018}.
The adsorption layer also contributes to
$K_{\text{osm}}$ through the ion-specific length-scales $\tilde{K}_{\pm}$ and $\tilde{L}_{\pm}$, as well as $\tilde{K}_\text{w}$ and $\tilde{L}_\text{w}$.
Their interpretation is
analogous to the characteristic lengths
reported in Table \ref{table:1},
but we refer the reader to the SI (Table~S1) for
their definition and numerical values.



At concentrations between $10^{-4}$\,M and $10^{-1}$\,M,
 Eq.~\eqref{eq:kosm_mGC} reproduces
 the asymptotic
 behaviour of the diffusio-osmotic conductivity,
 which scales as $K_\mathrm{osm} \sim n_\mathrm s ^{p}$,
 where $p\approx 2$ for graphene and
 $p\approx 2.5$ for hBN.
 We discuss each term separately to explain
 the scaling behaviour observed in
 Fig.~\ref{fig:transport_coefficients}(f).
First of all, to leading order, the effective surface charge
scales linearly with the salt concentration
(\textit{i.e.}, $\Sigma_\mathrm{eff}\sim n_\mathrm{s}$). Secondly,
noting that $\lGC^\text{eff} \propto \Sigma_\mathrm{eff}^{-1}$,
the contribution to the diffusio-osmotic
conductivity arising from the region beyond
 the adsorption layer
 also scales linearly with salt concentration
 (\textit{i.e.}, $1-\text{asinh} (\chi)/\chi \sim n_\mathrm{s}$).
Thirdly, the term arising from the
 adsorption layer exhibits
 a first obvious linear dependence on the
 concentration through the Debye length
 since $\lambda_\mathrm{D}^{-2} \propto n_\mathrm{s}$. This is indeed
 observed for graphene,
 such that overall the diffusio-osmotic
 conductivity scales as $K_\mathrm{osm} \sim n_\mathrm s ^{2}$.
 A second more subtle dependence on salt concentration however, is observed on hBN,
 where the adsorption layer term in  Eq.~\eqref{eq:kosm_mGC} changes sign
around $5\times 10^{-3}$ M (see Fig.~S6 in the SI).
The contribution  to  the diffusio-osmotic conductivity coming from the adsorption layer
term and the $1-\text{asinh} (\chi)/\chi$
term can be either  suppressed or enhanced
depending on the sign of the former, and ultimately results into a scaling of
$K_\mathrm{osm} \sim \ns^{2.5}$.
%
Further, similarly to what was observed in the case of the diffusio-osmotic
flow in Fig.~\ref{fig:transport_coefficients}(d),
at concentrations beyond 1\,M a reversal in the
diffusio-osmotic current is  observed in the
numerical calculations for hBN, because the water
contribution dominates over
its ion counterpart, whereas on graphene a sign change is
not observed because the ion-specific adsorption
contribution dominates at
all considered concentrations.
We note that the assumption of a linear
electrostatic potential difference $\phi(z) -\phis$
made while deriving Eq.~\eqref{eq:kosm_mGC} breaks down
at large concentrations, thus Eq.~\eqref{eq:kosm_mGC}
fails to predict the sign change on hBN at around 1 M.
\section{Discussion}
In this section, we
discuss the significance of our
work in connection with theoretical and experimental results
on the structure of the EDL and of osmotic transport in nanofluidics.
%
In our approach to calculate the
spatial distribution of ions in the EDL,
the electrostatic potential and the
free energy profile of ion adsorption are
determined, respectively, from the
solution of the mPB equation and from our
enhanced sampling simulations.
This framework has been introduced in
the past to study the structure of the EDL
at liquid/liquid interfaces\cite{luo2006ion} and to
shed light on the Hofmeister
series at hydrophobic and hydrophilic
surfaces for different surface
charges with FFMD simulations\cite{schwierz2010reversed}.
However, deviations from a modified PB
description of the electrostatic
potential appear at concentrations of the order of 1 M\cite{gonella2021water},
and it would be interesting
to incorporate
more advanced theories into our framework
to ameliorate the deficiencies
underlying the mPB theory
at such concentrations\cite{duignan2021toward,hartel2015fundamental,netz2001electrostatistics,kardar1999friction}.

Despite their substantial computational cost,
our \textit{ab initio} simulations
have also revealed that ion adsorption
can be surface-specific, as illustrated
by the differences in the free
energy of I$^-$ on the two sheets but not
of K$^+$. In particular, the presence of a
free energy barrier of I$^-$ on
hBN and not on graphene can be ascribed to
 differences in the water dipole orientation
 on the two sheets.
Deep UV second harmonic generation,
already used to investigate the free energy
of adsorption of SCN$^-$ on
graphene\cite{mccaffrey2017mechanism},
is an ideal technique to explore
ion adsorption on other substrates, including hBN.
Also, X-ray photoelectron spectroscopy
would be instrumental to probe the shape of the electrostatic potential profile
at aqueous electrified interfaces\cite{favaro2016unravelling}.

A central contribution of this paper is to have provided
a unified framework of osmotic transport that can be computed
from the structure of water and ions at aqueous interfaces,
and which transport coefficients can be probed experimentally. Measurements of osmotic transport of KI
solutions across graphene or hBN have not appeared yet, but
we think that they are already possible,
given that measurements of ion transport across
{\AA}-size slits have been reported,
and that diffusio-osmosis of several types of salts across
silica surfaces has also been probed.
It would be desirable that such experiments
be performed at the point of zero charge,
since differences in pH lead
to variations in the surface charge\cite{secchi2016scaling,Grosjean2019,Siria2013,Mouterde2018} and thus would likely alter the scaling behaviour
of the transport coefficients with concentration.
For instance,
a different scaling behaviour of the electrical
conductivity on salt concentration
has been observed in carbon nanotubes. This has been captured by
charge regulation models\cite{secchi2016scaling, biesheuvel2016analysis}
which are, however,
phenomenologically different
from the mechanisms described here, as we remark that
the sheets are not charged.

In absence of experiments on KI solutions,
we can connect our work to recent streaming-voltage experiments of KCl solutions across graphitic
\AA-size slits\cite{mouterde2019molecular}. Such experiments
did not present a clear dependence of the $\zeta$-potential
as a function of concentration
in a range between $ 10^{-3}$\,M and $10^{-1}$\,M
and extracted a value for $\zeta$ at least 10 times larger than that
shown in Fig.\ref{fig:transport_coefficients} (a).
Possible explanations are that, although ion-specific effects
of a KCl solution are less pronounced than a KI solution,
under the extreme confinement regime probed
in such experiments the Stokes equation of
hydrodynamics breaks down. Additionally,
ion exclusion at the entrance of the
membranes might play an important role in \AA-scale slits.

Concerning diffusio-osmosis,
we discuss our results in connection with
measurements performed on silica surfaces,
where the diffusio-osmotic coefficient has
been measured for several types of aqueous electrolytes, including KI\cite{Lee2014b}.
The reported diffusio-osmotic coefficient of KI on silica
is $D_\mathrm{DO} \approx 250 \mu\text{m}^2/s$,
independent of concentration.
Instead, we observe approximately a linear dependence
at all concentrations for graphene, and
above $10^{-1}$\,M $D_\mathrm{DO}$ reaches values beyond
$10^{4}\,\mu\text{m}^2/\text{s}$.
The observed enhancement of
the diffusio-osmotic flow
of KI on graphene compared to
silica can be largely attributed to the larger slip length of graphene.
Although the \textit{ab initio} values
for the slip length on graphene and
hBN\cite{tocci2020nanoscale}
compare well to recent flow  experiments performed on graphene slits\cite{xie2018fast},
as well as on hBN nanotubes and on carbon nanotubes
with a large radius $R$ of 50\,nm\cite{secchi2016massive},
the slip length for carbon nanotubes tubes
with a smaller radius $R=20$\,nm
is about one order of magnitude
larger than that of graphene, $b\sim 200$\,nm.
Provided that the structure of the EDL
would be only modestly affected
by such a large nanotube radius, one can extrapolate
the value of the diffusio-osmotic
coefficient that would be obtained for carbon
nanotubes with $R=20$\,nm from our results on graphene.
Doing so would result in approximately a $10$-fold
enhancement in the diffusio-osmotic
coefficient because of the much larger slip length.
Additionally, compared to silica,
where the diffusio-osmotic flow proceeds
from high to low concentrations,
we observe a flow reversal on hBN.


Measurements of the diffusio-osmotic current of KI solutions on graphene or hBN
have also not been reported. As such, we
connect our results for $K_\mathrm{osm}$
to osmotic energy conversion experiments of KCl solutions
flowing across single-pore BN
nanotubes\cite{Siria2013}.
On BN nanotubes, the diffusio-osmotic conductivity
$K^\mathrm{BNNT}_\mathrm{osm}$ has
been extracted from the ratio between the
diffusio-osmotic current and the difference in
the salt concentration at the reservoirs as
$K^\mathrm{BNNT}_\mathrm{osm} = I_\text{osm} /(\Delta \ns /\ns)$.
In order to compare directly to our results, we
normalize the current with respect to the perimeter
$P= 2\pi R$ and the length $L$ of the nanotube, \textit{i.e.}
$K^{\prime \mathrm{BNNT}}_\mathrm{osm} = \frac{I_\text{osm} /P}{\Delta \ns /( \ns L)}$. With an experimental value of the nanotube length   $L=1250$\,nm and of the radius $R=40$\,nm, the equivalent
experimental value for the diffusio-osmotic conductivity is
$K^{\prime \mathrm{BNNT}}_\mathrm{osm} = K^\mathrm{BNNT}_\mathrm{osm} L/(2\pi R) = 0.35 - 0.80$\,nA. These values for the diffusio-osmotic conductivity are much
larger than those computed here (at any salt concentration)
because of the very large surface charge $\Sigma$
that was reported in experiments ($\Sigma \approx 0.1 -1$\,C/m$^2$).
In BN nanotubes, the diffusio-osmotic
conductivity has been found to scale linearly on
the pH of the solution, and thus on the surface charge,
but to be independent of salt concentration\cite{Siria2013}.
Here instead, the diffusio-osmotic conductivity
scales roughly as $K_\mathrm{osm} \sim \ns^{p}$
with an exponent $p\approx 2$ and $p\approx2.5$
for graphene and hBN, respectively, over several decades
of salt concentration. The sign reversal in the
diffusio-osmotic current, as well as
in the diffusio-osmotic flux,
could be particularly relevant in biology.
Similar mechanisms might be at work in
membrane proteins, which could exploit changes
in concentration to regulate charge and solute
transport\cite{van2006claudins,van2003reversal}.
Although the focus here has been devoted to the understanding of
osmotic transport across a single-pore membrane or channel,
further challenges lie ahead of diffusio-osmosis in order to
establish itself as a viable source of renewable energy, in
particular, for what concerns the generation of electricity using
multi-pore systems.\cite{macha20192d,Tong2021,Wang2021}

In conclusion, we have provided a unified description of
osmotic transport by coupling first principles simulations with a
mean field description of the EDL and with the Stokes equation
of hydrodynamics, and have applied this framework to understand
the osmotic transport properties of a
prototypical salt that displays pronounced
ion-specific effects on two-dimensional materials.
Through the mPB equation of the EDL (see Eq.~\eqref{eq:mPB}),
the transport coefficients in
Eqs.~(\ref{eq:zeta}-\ref{eq:K_osm2}) can
be readily computed from the spatial
distribution of ions and water
at the interface.
We have reported on a concentration-dependent
scaling behaviour of the osmotic transport coefficients
at the aqueous graphene and hBN interfaces
and explained it with a model that
accounts for ion-specific adsorption and
water layering in a thin region of the EDL
 of the order of 1\,nm.
The observed scaling, along with the possibility of
diffusio-osmotic flow and
current reversal, may provide additional routes
to further improve osmotic energy conversion.
%
Moreover, it could foster the development of nanofluidic
diodes and sensors and may be instrumental to shed light
on the mechanisms underlying
charge and solute transport across membrane proteins.


%
\section*{Materials and Methods}
\subsection*{Electronic structure and \textit{ab initio} molecular dynamics}
The \textit{ab initio} umbrella sampling
simulations are performed with
the CP2K code \cite{kuhne2020cp2k}, and the
electronic structure problem is solved using DFT
with the optB88-vdW functional \cite{jiri_solids,jiri_molecules}.
The optB88-vdW functional has been applied to investigate slippage
on two-dimensional materials \cite{Joly2016,tocci2020nanoscale}
and it describes the  structure of graphite and bulk hBN
accurately \cite{graziano_vdw_gra_bn}.
Despite the limitations of most density functionals
in describing water, the optB88-vdW functional appears
to be one of the most satisfactory \cite{gillan2016perspective}.
Reference quantum monte carlo calculations
of water monomers adsorbed on graphene and hBN sheets
\cite{brandenburg2019physisorption,al2015communication}
also show that this functional captures the relative stability
of water on different adsorption sites.

The free energy of adsorption of K$^+$ and I$^-$ ions
and the spatial water distribution were calculated from umbrella sampling simulations\cite{torrie1977nonphysical}.
The systems consist of water films containing 400 molecules and approximately 2 nm-thick
placed above a $2.56 \times 2.46$ nm$^2$ graphene and a $2.61 \times 2.51$ hBN nm$^2$ sheet.
Separate umbrella sampling simulations were performed for potassium and iodide
adsorption by restraining each ion at different heights  $z_0$ above the sheets with
a harmonic bias potential $U_b(z,t) = k_b/2 (z(t)-z_0)^2$,  $z(t)$ being the
instantaneous height of the ion above the sheets,
and $k_b = 836.8$ kJ/mol/nm$^2$ the \textit{spring} constant.
A total of 23 umbrella sampling windows were used for each system
and for each window the dynamics were propagated for 40 ps
in the NVT ensemble at 300\,K within the Born-Oppenheimer
approximation, except for I$^{-}$ adsorption on graphene,
for which  dynamics were propagated for additional 30 ps to test
a possible dependence of our results on the length of the
simulations. The free energy was reconstructed using umbrella integration\cite{kastner2011umbrella}.
Further computational details on the  optimization
of the wave-function and on the umbrella sampling
simulations are reported in the SI.
The tools used to perform the analysis and the CP2K input files
to reproduce the main results of the manuscript have been deposited on
GitHub and are available at \url{https://github.com/gabriele16/osmotic_transport_scaling_laws}.


 \begin{acknowledgement}
 GT is supported by the SNSF project PZ00P2\_179964.
 LJ is supported by %the ANR through Project ANR-16-CE06-0004-01 NECtAR, and by
the Institut Universitaire de France.
RM and GT acknowledge funding by the Deutsche Forschungsgemeinschaft (DFG, German Research Foundation) -- 390794421.
 We also thank the Swiss National Supercomputer Centre (CSCS) under PRACE for awarding us access to
 Piz Daint, Switzerland, through projects pr66 and s826.

 \end{acknowledgement}

\begin{suppinfo}
Further computational details and tests
on the structure and dynamics
of water at the interface with
graphene and hBN  from
force field and \textit{ab initio} simulations;
derivation of integral expressions for the transport coefficients;
details of the modified Poisson-Boltzmann description;
details of the effective surface charge model.
\end{suppinfo}

%\bibliography{biblio_osmotic_transport,laurent,robeme}
\documentclass[fleqn,10pt]{article}
\usepackage[utf8]{inputenc}
\usepackage[T1]{fontenc}
\usepackage{graphicx} 
\usepackage{hyperref}
\usepackage{lineno,color}
\usepackage{amsmath,amsfonts,amssymb}
\usepackage{authblk}
\usepackage{geometry}
%\linenumbers

\date{}
\geometry{hmargin=3cm,vmargin=3cm}

\title{Universal Database for Economic Complexity}

\author[1,2,*]{Aurelio Patelli}
\author[2,1]{Andrea Zaccaria}
\author[1,3]{Luciano Pietronero}
%\author[2,*]{Derek Author}
\affil[1]{Centro di Ricerca Enrico Fermi, Via Panisperna 89 A, I-00184 Rome, Italy}
\affil[2]{Istituto dei Sistemi Complessi (ISC) - CNR, UoS Sapienza,P.le A. Moro, 2, I-00185 Rome, Italy}
\affil[3]{Dipartimento di Fisica Universit\`a “Sapienza”, P.le A. Moro, 2, I-00185 Rome, Italy}


\affil[*]{corresponding author: Aurelio Patelli (aurelio.patelli@cref.it)}

%\affil[$\dag$]{these authors contributed equally to this work}



\begin{document}

\flushbottom
\maketitle
%  Click the title above to edit the author information and abstract

\begin{abstract}
	We present an integrated database suitable for the investigations of the Economic development of countries by using the Economic Fitness and Complexity framework.
	Firstly, we implement machine learning techniques to reconstruct the database of Trade of Services and we integrate it with the database of the Trade of the physical Goods, generating a complete view of the International Trade and denoted the Universal database. 
	Using this data, we derive a statistically significant network of interaction of the Economic activities, where preferred paths of development and clusters of High-Tech industries naturally emerge. 
	Finally, we compute the Economic Fitness, an algorithmic assessment of the competitiveness of countries, removing the unexpected misbehaviour of Economies under-represented by the sole consideration of the Trade of the physical Goods.
\end{abstract}




%%%%%%%%%%%%%%%%%%%%%%%%%%%%%%%%%%%%%%%%%%%%%%%%%%%%%%%%%%%%%%%%%%%%%%%%%%%%
\section*{Introduction}
%%%%%%%%%%%%%%%%%%%%%%%%%%%%%%%%%%%%%%%%%%%%%%%%%%%%%%%%%%%%%%%%%%%%%%%%%%%%
%%% generalities EFC
Economic Fitness and Complexity~\cite{Tacchella2012,Tacchella2013,Cristelli2013,Caldarelli2012,zaccaria2016case,Sbardella2018} (EFC) is a novel conceptual and practical framework for the estimation of the competitiveness of nations and the relatedness between sectors, borrowing concepts and methods from Statistical Physics and the Complex Systems Science~\cite{Pietronero2008}.
A key feature of the method resides in its bottom-up, data-driven approach, which relies on methods like complex networks and machine learning to reconstruct and investigate the ecosystem constituted by the economic actors and their activities; moreover, all scenarios are based on empirical observations and certifiable hypotheses.
This approach departs from the canonical econometric narrative, where the economic performances are usually gauged by monetary metrics, such as the Gross Domestic Production (GDP).
Instead, EFC aims to capture the competitiveness of a country, and not its wealth, by introducing a synthetic and non-monetary metric, the Fitness~\cite{Tacchella2012}, bringing to light new and relevant economic patterns.

%%% importance of the data in EFC
A key feature of Economic Fitness and Complexity is its reliance on homogeneous and high-quality data from which it effectively extracts information by aiming to achieve a maximal signal-to-noise ratio.
In the \emph{Big-Data} era, the novel and great size of available databases may generate a broad confidence on the new possibilities offered by large scale analyses, although issues related to the quality of the gathered data and the specific investigations are sometime underplayed~\cite{Hosni2018}.
It is rather intuitive that Big-Data may present Big-Noise~\cite{Silver2012}, and so a careful selection of the sources must be accomplished at the starting point of the research in any data-driven field. 
The reference database in the classical EFC analysis is the International Trade database reconciled and regularized by Tacchella and collaborators~\cite{Tacchella2018} starting from the UN-COMTRADE data. 
This database covers the external flows of physical Goods between the countries in the World with a very fine level of detail of the classification collecting about 5000 product's codes. 
EFC is based on the export database for various reasons, both practical, due to its homogeneity and standardization across different countries, and conceptual. 
The idea is to deduce the competitiveness of countries not explicitly looking at their capabilities, but by inferring the presence of such endowments in a indirect way, that is, by looking at the final outputs, the products that such countries are able to export~\cite{hausmann2007you}. 
Not trivially, the GDP predictions obtained by EFC and based on the Trades indicate a quantitatively better performance compared to the best conventional economic models~\cite{Tacchella2018} and with a much lower data requirement.

%%% Services and literature
Unfortunately, the Services are not included in the set of available features in the UN-COMTRADE and, albeit available from an IMF database (www.data.imf.org), neither in the usual analyses of EFC, despite a relevant and growing fraction of money flows through the channel of the Services~\cite{loungani2017world}.
A first attempt to complement the Services into the aforementioned analysis was followed by~\cite{Stojkoski2016}, finding that at a very aggregated scale the Services tend to be more complex with respect to the Goods.
However, the authors complain that the high level of aggregation may be the cause of a loss of information necessary to accurately separate the Complexity of the Activities and the Fitness of the Countries.
With a deeper level of aggregation, the authors in ~\cite{Zaccaria2018services} show a more heterogeneous scenario of the Complexity of the Activities, pointing out that complex Services cluster with complex Goods.
In the EFC narrative the clustering of a set of Activities is the signal indicating that there is a participation of common intangible capabilities~\cite{Cristelli2013}, necessary to be competitive in both the domains.
However, either manuscripts select a subset of the available economies in the World because the covering of the databases implemented in the analysis present many missing features.
Indeed, both the references consider only 116 countries, with important missing economies such as China and Great Britain. 
Recently, the work of Mishra and collaborators~\cite{Mishra2020} shows the importance to aggregate the Services and the Goods in a common database especially accounting properly the economic relevance of the developing nations. 
Saltarelli et al.~\cite{saltarelli2020export} use the World Input-Output Database (WIOD) to show that export mirrors remarkably well domestic production for manufacturing sectors, but this relation fades away for service related sectors. 
All these analyses, however, use a limited number of services sectors because of the high number of missing values present in the original IMF database; a limitation that the present work addresses.

%%% In this work
The analysis endowed by the previous works highlight the purposes of the Services in the correct estimation of the competitiveness of the nations.
However, the reference database for the Services, the \textit{Balance of Payments and International Investment} (BOP) data collected by the International Monetary Fund (IMF)~\cite{BOPS}, presents some weakness and important missing scores, especially in comparison with the quality of the Goods database. 
Strikingly, the actual IMF-BOP classification is unsuitable for the EFC analysis, presenting an overlapping and convoluted hierarchy of sector.
The first core result of the present paper is the reconciliation of the quality of two databases, the Goods and the Services, obtained by reclassifying the IMF services sectors and reconstructing the missing elements of BOP by using and comparing different machine learning techniques.
The best reconstruction method obtained is then used to create the so-called \textit{Universal} database of EFC, aggregating Services and Goods for 160 countries and a total of 124 sectors, providing the largest set of common nations and sectors available from both the BOP and the Trade datasets.
Finally, the Universal database is used in the Economic Fitness and Complexity analysis, obtaining two conceptual and practical advances: i) a statistically validated network of universal sectors, that will be of practical use to predict and recommend new sectors of development, and ii) the computation of the Universal fitness, the novel EFC indicator to assess the competitiveness of nations, now including also the services.



%%%%%%%%%%%%%%%%%%%%%%%%%%%%%%%%%%%%%%%%%%%%%%%%%%%%%%%%%%%%%%%%%%%%%%%%%%%%
\section*{Methods}
%%%%%%%%%%%%%%%%%%%%%%%%%%%%%%%%%%%%%%%%%%%%%%%%%%%%%%%%%%%%%%%%%%%%%%%%%%%%
This section introduces the database used in the later analysis, aggregating Goods and Services; in particular, we will focus on the motivations and the methodologies we adopted for the re-classification of services and the reconstruction of the missing values.



\subsection*{Goods: the International Trade}
Following the definition of the IMF~\cite{BOPS}, a Good is a physical item or commodity over which ownership can be passed via transaction. 
Consistently, it is recorded by the customs and available via UN-COMTRADE.
Following the standard approach of EFC, the reference database collecting the scores of the Goods is the reconciliation of the International Trade database contained in \textit{UN-COMTRADE}~\cite{COMTRADE}.
It collects the export flows of classes of physical products between nations.
The classification of the product follows the Harmonized System revised in 1992 (HS92), consisting of 5040 sectors labelled by 6-digits codes at the sub-heading level.
The temporal coverage of the database runs from 1996 to 2018, spanning slightly more than two decades, and it is made available for 169 countries corresponding to the principal economic players in the World.


The statistic of the codes at 6-digits does not present a level of details and a statistics comparable to the orders of magnitude of the data of the Services.
It is also worthy to note that the HS classification varies in time and in the available years a relevant modification happens in 2007.
Perhaps, such modification induced a re-balance in some sectors at 6-digits that create a non-uniform transition in few and low complexity codes~\cite{JRC_report}.
Therefore, we aggregate the classification at the closest comparison to the Services database, being at the 2-digits classification. Note that this choice also mirrors the respective weight in the international trade \cite{loungani2017world}.
Furthermore, the 2-digits classification has 97 codes, but we remove one sector (the code 99, the \emph{Commodities not specified according to kind}) because it does not represent a well defined sector and has a low impact on the overall exports.
%Finally, the raw structure of the database consists on a set of tables, each one reporting the volume in constant dollars in a year where exported by each nation in each sector at 6-digits.


%%% SITC
%For the seek of completeness, we also map the International Trade database into the Standard International Trade Classification (SITC), widely used in the academic literature because it bolster the comparability of the trade statistics.
%The map between the two classification is implemented from HS92 at 6-digits to SITC at 4-digits.
%Successively, we follow the same arguments proposed above and we aggregate the SITC classification at 2-digits such that the Goods have a compatible statistics with the Services.
%We anticipate here that the particular classification implemented does not introduce relevant changes at the level of the competitiveness between the nations, and we account the analysis using SITC in the Supplementary.
%Instead, HS92 displays a more heterogeneous behaviour than SITC in the analysis of the Complexity of the Goods, and it will be discussed below.


\subsection*{Services: description and classification issues}
%%% services: general info on the database
According the IMF~\cite{BOPS}, a Service is the result of a production activity, or facilitate the exchange of Goods, or is a financial asset.
Therefore, Services are usually non-separable items and cannot be separated from their production.
The Service database is based on the \textit{Balance of Payments and International Investment} (BOP) data collected by the International Monetary Fund (IMF)~\cite{BOPS}~\footnote{The data has been downloaded the 28th of January 202 from the website www.data.imf.org}.
The BOP database offers three measures of Services: the credit, the debit and the net values. In comparison with the measure available for the Goods, we select the credit indices, since they correspond to exported services. Two closely related classifications are present: the full BOP classification implemented by the IMF and an \emph{alternative} alpha-numerical classification reported in their metadata.
The full BOP classification is based on the composite nature of the codes of the Services but it presents a key drawback for the analytical purposes accounted below.
In fact, it is possible that the sum of the sons of a code have a total value larger that the assigned value of the parent.
This misleading situation happens because the classification may implement different methodologies in order to generate the hierarchy.


\begin{figure}[!t]
	\centering
	\includegraphics[scale=0.4]{figure/bp6_bx_graph_transports_name_crop.pdf}
	\caption{The network of the transport and its sons divided with the two possible dis-aggregation of the codes. Only the blue codes sum up to the value of the parent. The name of the nodes corresponds to the last part of the definition available from the BOP's metadata and the full definition correspond to the sequential aggregation of the definitions of the parent nodes.}
	\label{fig:network_transport}
\end{figure}
In order to explain this situation we discuss an example regarding the case of the  sons of the `\emph{Transport Services}' (`STR' codes) where the dis-aggregation considers both the kind of transportation implemented or the nature of the transported Goods.
For a visual inspection, figure~\ref{fig:network_transport} shows the network of the transport's codes, highlighting the two dis-aggregation paths in blue and red.
Remarkably, only the blue codes sum up to the value of their parents.
Instead, the red nodes do not sum to the available value of the parents, although it is possible that using the hierarchical tree obtained from the codes does not account properly for the red classification.


Contrarily, the alternative classification maintains only a single hierarchical structure removing the double methodologies, and the quality is corroborated by the fact that the sum of the values of the sons return the value of the parent codes.
We want to stress that the scenario of the two methodologies derives from a misleading interpretation of the indices offered by the IMF and not an error in their implementation of the database, since it is possible to reconstruct the correct structure.
\begin{figure}[!t]
	\centering
	\includegraphics[scale=0.5]{figure/bp6_bx_graph_alternative_short_crop.pdf}
	\caption{The tree graph of the relation between the BOP codes following the \emph{alternative} classification. The colour of the nodes represents the frequency of missing elements found in the raw database with red corresponding to 1 and white corresponding to 0. The green circle indicates the codes considered in the final construction of the Universal database (see text below).}
	\label{fig:net}
\end{figure}
We present the first three main layers of aggregations of the \emph{summable} classification in figure~\ref{fig:net}.
%A shallower structure since all the codes sons of `SO' (in layer 2 of BOP) are promoted to the layer above (here layer 2).
Among the layers, the first one accounts for the total sum each country produces in a year.
The second and third layers are more representative and diverse of the heterogeneous aggregation of the services.

\subsection*{Missing values in Services data}

In the construction of the Universal database we select the codes of table~\ref{tab:complete_set}, covering with the lower level of aggregation.
For the sake of clarity we highlight these codes in figure~\ref{fig:net}  with a green thick border, and we refer to the set as the \textit{complete set} of codes, since with its knowledge it is possible to obtain the full database.
\begin{table}[h!]
	\centering
	\small
	\begin{tabular}{| l | c | c |}
		\hline
		\textbf{code} & \textbf{layer} & \textbf{description}\\
		\hline
		BXSOGGS & 1 & Government Goods and Services \\
		BXSORL & 1 & Charges for the Intellectual Property \\
		BXSR & 1 & Maintenance and Repair \\
		\hline
		BXSMA & 2 & Manufacturing Services, Goods for Processing Abroad \\
		BXSMR & 2 & Manufacturing Services, Goods for Processing Inside \\
		BXSOTCMT & 2 & Telecommunications Services \\
		BXSOTCMM & 2 & Information Services \\
		BXSOTCMC & 2 & Computer Services \\
		BXSOPCRO & 2 & Cultural and Recreational \\
		BXSOPCRAU & 2 & Audiovisual \\
		BXSOOBTT & 2 & Technical, Trade-related, and Other Business \\
		BXSOOBRD & 2 & Research and Development \\
		BXSOOBPM & 2 & Consulting \\
		BXSOINRI & 2 & Reinsurance \\
		BXSOINPG & 2 & Pension \\
		BXSOIND & 2 & Direct Insurance \\
		BXSOINAI & 2 & Auxiliary Insurance Services \\
		BXSOFIFISM & 2 & FISIM \\
		BXSOFIEX & 2 & Explicitly Charged and Other Financial Services \\
		BXSOCNA & 2 & Construction Abroad \\
		BXSOCNAR & 2 & Construction Inside \\
		BXSTVB & 2 & Travel Business \\
		BXSTVP & 2 & Travel Personal \\
		BXSTRS & 2 & Sea Transport \\
		BXSTRPC & 2 & Postal and Courier \\
		BXSTROT & 2 & Other Passenger Transport \\
		BXSTRA & 2 & Air Transport \\
		\hline
	\end{tabular}
	\caption{The list of the 27 codes implemented in the Universal database with their description. The initial 2 letters `BX' indicate the credits.}
	\label{tab:complete_set}
\end{table}

The BOP data covers about 197 countries although we select the subset of 160 countries intersecting the set of countries available for the Goods, and covering basically all the relevant economies in the world in terms of economic impact.
Finally, in terms of temporal resolution, BOP goes back up to 1940 with very few sectors in the first years.
However, we reconstruct the series starting from 1990 because the Universal database is constraint by the presence of also the Goods database, which starts in 1996.

\begin{figure}
	\centering
	\includegraphics[scale=0.5]{figure/series_nan_year.png}	
	\caption{Proportion of missing elements in the complete set of indices of the BOP database as a function of the years in the layers of the raw database (160 countries).
		The lines with the shade of blue correspond to the different layers of the raw database while the red and dark red lines correspond to the complete set in the raw (red) and reconstructed (dark red) cases.}
	\label{fig:freq_nan}
\end{figure}
Focusing on the temporal distribution of the missing elements, it is clear that the raw database is not uniform, as shown in figure~\ref{fig:freq_nan}.
In the early years of the series there is a larger presence of missing values with a decreasing global trend approaching the recent times.
In the developed economies these behaviours are primarily induced by the complete absence of the first segments of the series for all the indices.
For example, in the case for Austria, no information is available before 2005; regarding Belgium before 2004, and Japan before 2000.
Instead, less developed countries present subsets of the indices with missing scores on short to medium temporal windows, while single NaNs are rarer in the datasets.
Hence, in between 2005 and 2015 the raw database offers its better quality, even if the raw version continues to have a high fraction of missing values.

Finally, also the geographical coverage of the raw database is not uniform, as shown in the top map of figure~\ref{fig:maps_nan}.
Many countries have a large fraction of missing values, even if some of them are developed economies.
Remarkably, Switzerland, China, Great Britain, and Spain are among the developed countries with a low quality representation in the raw data.
This situation of heavy lack of data and heterogeneity calls for a specific intervention to reconstruct the missing elements, which is the subject of the next section.


%%%%%%%%%%%%%%%%%%%%%%%%%%%%%%%%%%%%%%%%%%%%%%%%%%%%%%%%%%%%%%%%%%%%%%%%%%%%
\subsection*{Reconstruction of the missing Services}

%%% basic info
We implement a reconstruction of the BOP database using different machine learning techniques.
The use of the alternative classification discussed above allows the derivation of the full hierarchy reconstructing the complete subset of leafs by summing the indices (red-circled indices in figure~\ref{fig:net}).
However, the knowledge of the values of the upper layers of the classification are used in the reconstruction because we do not want to change abruptly the relative importance of the nations or of the daughter indices.
Moreover, in the following we assume that any not-missing value of the raw database is \textit{correct}, thus also the entries with zero value are labelled as correct.
%%% basic and references
A very basic and easily interpretable reconstruction technique is given by the linear interpolation of the temporal series in the forward direction, \textit{i.e.} generating the elements for which there exists some previous information.
Therefore, the linear interpolation is assumed as the reference method, giving relatively good results despite the simplicity of the method.
Successively, we consider two families of methodologies: the Random Forest class and the Nearest-Neighbour imputations~\cite{batista2002study} class.


%%% Random Forest
The Random Forest consists in an ensemble of decision trees~\cite{Breiman2001} trained to optimize the entropy of the reconstruction.
Rather, a decision tree is a single and non-overlapping subdivision of the space of the features, \textit{i.e.} the values of the series, into distinct regions following a hierarchical ordering of the feature-selection with a tree structure~\cite{Safavian1991}.
Since the splits of the tree are randomly chosen, it is possible to derive many different decision trees forming a statistical ensemble.
Using the ensemble of trees it is possible to estimate each missing feature of the input database (here the Services) based on the behaviour of the other entries for which those features are specified.
This method is applied on the complete set of series aggregating the entries in time and space (years and nations), the so-called Temporal Random Forest~\cite{Serafini}, such that the information from similar economies is taken into account.
In reference~\cite{Serafini}, this method is found to be superior in terms of the quality of reconstruction with respect to the separate reconstruction of each country series.

%%% kNN
The latter method we realize, the k-Nearest-Neighbour (kNN), consists in the generation of the missing features with a weighted network average from other features.
In detail, the network of similitude between the economies, from the prior knowledge of the indices in the upper layers, is built under the assumptions that the reconstruction does not strongly modify the relative importance of the nations.
Accordingly, the kNN imputer~\cite{Troyanskaya2001} links the first $K$ nearest-neighbours elements from the available information on the upper layer.
The resulting network is then applied for the selection of the nearest elements considered in the weighted reconstruction of the missing values.
Finally, the weight is defined as the Euclidean feature-distance in the original space between the selected entries.

%%% Tests
\begin{figure}[!h]
	\centering
	\includegraphics[scale=0.5]{figure/mae_reconstruction.png}	
	\caption{The Mean Absolute Error of layer of the proposed reconstruction techniques a s function of the number of Nearest-neighbours $K$. }
	\label{fig:mae}
\end{figure}
We test the quality of the reconstructions on a test-set of the database composed by countries and indices for which all the entries are specified.
On this dataset we generate random samples setting some feature as missing, and we generate 100 samples of this process.
For each replica we label as missing at random, and we compute the statistics on the reconstruction techniques described above.
The quality of the reconstructions are finally evaluated by the computation of the Mean Absolute Error (MAE) between the reconstructed and the truth values.
As shown by the left panel of figure~\ref{fig:mae}, the interpolation method down-perform the MAE of the Temporal Random Forest, 
but the best achievement is obtained by using the kNN network built with only a few of neighbours.

%%% hence!
\begin{figure}[!h]
	\centering
	\includegraphics[scale=0.125]{figure/map_nan_raw.png}
	\includegraphics[scale=0.125]{figure/map_nan_reconstructed.png}
	\caption{Maps of the World where the colour correspond to the density of missing elements in the complete set of indices of the database of the Services. 
		The top map shows the density in the raw data while the bottom map shows the density in the reconstructed data.}
	\label{fig:maps_nan}
\end{figure}
Therefore, in order to reconstruct the database we will use in the following analyses we implement the kNN reconstruction technique with $K=5$~\footnote{We may expect that size effects require a larger $K$ and the difference in MAE between 4 and 5 in not appreciable.} neighbours on the complete set of indices described in the previous section.
Figure~\ref{fig:maps_nan} shows World maps drawing the colours of the countries in terms of the fraction of missing elements in the raw (upper) and reconstructed (lower) databases and allowing a visual representation of the quality of the reconstruction.
Finally, the reconstructed set, truncated between 1996 and 2018, is then aggregated to the 2-digits UN-COMTRADE data in the so-called \textit{reconstructed Universal database}.

\subsection*{The Economic Fitness and Complexity input matrices}
The various Economic Complexity measures are computed starting from quantities derived from the export data.
The raw export volumes are not directly used essentially for two reasons: i) they trivially depend on both the sizes of the sector and the country and ii) they do not provide an assessment of the \textit{competitiveness} of the given country in exporting a product. 
In order to overcome these limitations, the Revealed Comparative Advantage (RCA), introduced by Balassa~\cite{Balassa1965}, is commonly used in the literature. 
In formula, the RCA of country $c$ in activity $a$ is computed as:
\begin{equation}
	\mathrm{RCA}_{c,a}=\frac {E _{c,a}/\sum_a E _{c,a}}{\sum_c E _{c,a}/\sum_{c,a} E _{c,a}}
\end{equation}
where $E_{c,a}$ is the value of the Product or Service sector $a$ (what we generically call an activity), in constant US dollars, exported by country $c$. 
Note that this formulation permits to identify a natural threshold of 1 to determine whether $c$ exports $a$ in a competitive way or not with respect to what the global set does. 
As a consequence, we can define the binary matrix $\textbf{M}$ whose element $M_{c,a}$ is equal to 1 if $\mathrm{RCA}_{c,a} \ge 1$, and it is equal to 0 otherwise. 

We also define the Marked Share (MS) matrix element as:
\begin{equation}
	\mathrm{MS}_{c,a}=E _{c,a}/\sum_c E _{c,a}
\end{equation}
that is an assessment of the importance of country $c$ in the global trade of $a$. 
Note that in this case no natural threshold is available, and a residual correlation with the size of the country is usually present.
Hence, we will call the Fitness based on $\textbf{MS}$ the Extensive Fitness, recalling the concept of extensive and intensive quantities in statistical physics, and we may indicate the \textit{standard} computations based on the binary RCA as intensive for comparison.

Both the series of the Market Share and of the binary RCA present their sources of noise and different techniques can be implemented in order to reduce their negative effect.
However, for the sake of simplicity and since the data at the aggregated level of 2-digits usually consider large volumes, in both the series of Market Share and RCA we apply a simple exponential smoothing with half-life of 3 years.
The use of the exponential smoothing allows the reduction of the noise has the positive side effect of maintaining a persistence of a few years in the temporal series, without the use of more complex Machine Learning Models.




%%%%%%%%%%%%%%%%%%%%%%%%%%%%%%%%%%%%%%%%%%%%%%%%%%%%%%%%%%%%%%%%%%%%%%%%%%%%
\section*{Results}
%%%%%%%%%%%%%%%%%%%%%%%%%%%%%%%%%%%%%%%%%%%%%%%%%%%%%%%%%%%%%%%%%%%%%%%%%%%%

The Economic Fitness and Complexity framework can be divided in two lines of research.
The first one aims at assessing the \textit{Relatedness} between sectors and sectors, and countries and sectors~\cite{hidalgo2018principle}. 
Usually, the output is a network of products, such as the Product Space~\cite{Hidalgo2009} or the Taxonomy Network~\cite{Zaccaria2014}. 
Actually, in this paper we will compute the Product Progression Network \cite{Zaccaria2018services}, based on the Assist Matrix validation framework introduced in~\cite{Pugliese2017}. 
With respect to other approaches, the one used here has two advantages: i) the time evolution is explicitly taken into account, permitting the construction of a \textit{directed} network of sectors, and ii) each link is statistically validated by comparing its weight against a distribution arising from a suitable null model.
The second line of research in Economic Fitness and Complexity consists in an assessment of the global competitiveness of a country, based on the idea that sophisticated capabilities are needed to export complex products.
One example is the Economic Complexity Index~\cite{Hidalgo2009}. 
Instead, in the present work we use the Fitness measure introduced in~\cite{Tacchella2012}, which has a number of advantages from both a practical and a theoretical points of view. 
For a broad discussion on these arguments we direct the reader to the relevant literature~\cite{Albeaik2017a,Gabrielli2017,Albeaik2017b,Pietronero2017}; herein we only emphasize the predictive power of the Fitness approach.
As shown in~\cite{Tacchella2018}, it outperforms the IMF in forecasting the GDP time evolution.


\subsection*{Goods and Services interaction}
%%% Assist matrix
A measure of the interaction between two activities can be obtained considering their co-occurrences, that is counting how often such couple of activities is present in the baskets of different nations. 
In order to avoid the size effect induced by the nested nature of the system, where developed countries are competitive in many activities, and some activities are by far more widespread than other, one has to properly normalize the simple co-occurrences. 
Following~\cite{Pugliese2017}, we compute the probability the information flows from one activity to the other along the shortest paths.
The resulting weighted network, projecting the original bipartite structure into the activity layer, is the Assist Matrix
\begin{equation}
	B_{a,a'}(t,\Delta) = \sum_{c} \frac{M_{ca}(t)}{d_a(t)}\frac{M_{ca'}(t+\Delta)}{u_c(t+\Delta)}.
	\label{eq:assist}
\end{equation}
Note that the time evolution of $\textbf{M}$ is explicitly taken into account, so one has a different network of activities for each $t$ and $\Delta$. 
Moreover, this network is almost fully connected, and the presence of spurious links prevents also simple tasks such as a clear visualization of the connectivity. 
In order to filter the links maintaining only the relevant ones, we have to statistically validate each link against a suitable null model.

%%% particular definition of the network and its reasoning
The implementation of the statistical validation of the network of interaction is pursued by considering an ensemble of random networks preserving few suitable macroscopic constraints~\cite{Saracco2017}. 
We generate the ensemble of randomized bipartite networks related to the empirical case, and we use them in the generation of the ensemble of the random Assist Matrix at fixed $(t,\Delta)$. 
Thus, we seize only those links in the empirical Assist Matrix that are above a percentile threshold, resulting in the statistically validated matrix at fixed $(t,\Delta)$.
Successively, we validate only those links that are statistically significant on all the available years $t$ at a fixed $\Delta$, aiming to reduce the false inference problem.
Therefore, we have the topological graph keeping only the links that are found relevant on all the temporal series analyzed.
Finally, we consider the presence of auto-correlation on the temporal series of the raw signal, and therefore we collapse a range the graphs into a single network where each link has a weight equal to the number of times it has been validated at different $\Delta$.


%%% the networks
In detail, in the present work the random ensemble considered for the validation of the network of interaction is the Bipartite Configuration Model~\cite{Squartini2011,Saracco2015} (BiCM).
The BiCM randomizes all the information available but the degrees of the nodes, \textit{a.k.a.} the ubiquity and the diversification, which are kept constant (on average).
Indeed, activities with a larger ubiquity have higher probability to be produced by a random country by chance and the BiCM contemplate this information.
For the link selection we apply a \textit{small} percentile threshold of $95\%$, compared to other cases in literature, because we require such level of statistical relevance along all the time series.

\begin{figure}[!h]
	\centering
	\includegraphics[scale=0.5]{figure/top_atleast_uni_rca_asset_dt1t10_100.pdf}
	\caption{\textbf{The statistically validated network of sectors}. Goods are in pink, while Services are in light blue. The width of the links is proportional to the number of statistically validated time intervals.}
	\label{fig:network_hs}
\end{figure}
The collected network spans 10 years of possible time delays ($\Delta\in[0,10]$) in order to keep track of the autocorrelation.
Figure~\ref{fig:network_hs} shows the network following the HS classification.
On the bottom right, a High-Tech cluster contains both high Complexity Goods and Services. 
Going counter-clockwise, immediately above we find all the Heavy Industries and then Textile related sectors.
Few other clusters collects the mineral Raw Materials and the Vegetable and the living Materials. 
These sectors show a lower degree of connectivity, indicating that countries specialized in these sectors rarely jump to more complex industries.
The Services are not strongly dispersed but are not forming a single block of nodes. 
Note that High-Tech related Services, those associated to royalties and R\&D, are strongly connected with the high Complexity sectors. 
Regarding only the Service, solely the group of transport nodes create a cluster \textit{per se}.
Finally, the statistical nature of the construction of the network allows that few nodes are not connected to any elements because their links are not found to be statistically relevant.
A lower percentile threshold may connect them at the price that the final network of interaction will be less statistically significant with more likely erroneous inferences.





\subsection*{Economic Fitness and Complexity}
The Fitness and Complexity algorithm~\cite{Tacchella2012,Cristelli2013,Pugliese2016} aims at assessing the competitiveness, or \textit{Fitness}, of a country, and the sophistication or \textit{Complexity} of products using a set of coupled equations. 
The idea is the following.
Each country is characterized by its endowments, or capabilities, which represent its social, cultural, and technological structure~\cite{Dosi2000}. 
These capabilities are expressed in what a country produces and exports, so sectors (physical Goods and Services) and their Complexities are linked to the Fitness of each country; in particular, the Complexity of a sector increases with the number and the quality of the capabilities needed in order to be competitive in it, and the Fitness is a measure of the Complexity and the number of the competitively exported sectors. 
In order to make this line of reasoning more quantitative, we start by considering the global structure of the matrix $\textbf{M}$ defined above.
Once countries and products are suitably arranged, the matrix $\textbf{M}$ is triangular, or nested~\cite{Mariani2019}, showing that developed countries have diversified exports, while less developed countries export fewer, lower Complexity products, and these products are actually the ones exported by all countries. 
In order to leverage on this structure to extract information about countries’ competitiveness and sectors’ Complexity, Tacchella et al.~\cite{Tacchella2012} proposed the following set of non-linear, coupled equations
\begin{eqnarray}\label{pilrs1}
	\begin{cases}
		\widetilde{F}_c^{(n)}=\sum_a M_{ca} Q_a^{(n-1)} & \\ \\
		\widetilde{Q}_a^{(n)}=\dfrac{1}{\sum_c M_{ca} \dfrac{1}{F_c^{(n)}}} \\
	\end{cases}
	\begin{cases}
		F_c^{(n)}=\dfrac{\widetilde{F}_c^{(n)}}{<\widetilde{F}_c^{(n)}>_c}  & \\ \\
		Q_a^{(n)}=\dfrac{\widetilde{Q}_a^{(n)}}{<\widetilde{Q}_a^{(n)}>_a}\\ 
	\end{cases}
	\label{eq:f-q}.
\end{eqnarray}
Here $<\cdot>_x$ denotes the arithmetic mean with respect to the possible values assumed by the variable dependent on $x$, with initial condition:
\begin{equation}\label{pilrs2}
	\sum_a Q_a^{(0)}=1   \hspace{6pt}\forall \, a.
\end{equation}

The iteration of the coupled equations leads to a single fixed point that does not depend on the initial conditions~\cite{Tacchella2012,Cristelli2013,Pugliese2016}.
The fixed point defines the non-monetary metrics quantifying the Fitness $F_c$ and the Complexity $Q_a$. 
The convergence properties of equations~\eqref{eq:f-q} are not trivial and have been extensively studied by Pugliese~\textit{et al.}~\cite{Pugliese2016}.
The coupled equations in~\eqref{eq:f-q} relies only on one input, the matrix $\textbf{M}$ stating which countries are competitive in which sectors. 
%This is what is called the \textit{intensive} Fitness, since the RCA is normalized with respect to both the total size of the countries and the sectors. 
Usually, the matrix is represented by the binary RCA, although also the Market Shares can be entered in equation~\eqref{eq:f-q}, return the Extensive Fitness~\cite{Tacchella2012,Zaccaria2018services}.
The immediate consequence of this choice is a higher correlation with the size of the country, as expressed for instance by its GDP.
\begin{figure}[!h]
	\centering
	\includegraphics[scale=0.33]{figure/scatter_extensive_fitness_6d_uni.png}
	\includegraphics[scale=0.33]{figure/scatter_fitness_6d_uni.png}
	\caption{\textbf{6d vs 2d and with/without services comparison}. One the left, the effect on the Extensive Fitness values of integrating services and aggregating into 2 digit sectors. On the right, the same analysis performed using the intensive Fitness. In both cases a good correlation is present, and the major deviations are given by services and good driven Economies. The dashed lines indicate the least square regression fit with a power law shape, while the colour scheme highlight the countries gaining ranking positions in ref and losing position in blue.}
	\label{fig:scatters}
\end{figure}
In Figure~\ref{fig:scatters} we compare the results of the Fitness algorithm when different input matrices are used. 
On the left we compare the Universal (i.e., considering Goods and Services) Extensive (i.e., the Market Shares matrix $\textbf{MS}$ is used) Fitness, on the x-axis, with the Extensive Fitness computed at 6-digits, that is, considering only the Goods and the lower available level of aggregation, containing about 4500 different codes. 
On the right, instead, we show the intensive counterpart, that uses as input the binary RCA. 
By visually inspecting the figure one can notice various properties. 
First, a rather good correlation is present, highlighting that the overall economical results based on the Fitness metrics are robust when Services are integrated and the sectors are aggregated from the 6 to 2 digits levels.
More importantly, the positive effect of specializing in high Complexity Services clearly emerges. 
Countries such as Great Britain (GBR) and Germany (DEU) provide two examples of Services or manufacturing driven countries, as correctly reflected by the Fitness indicator.
\begin{figure}[!h]
	\centering
	\includegraphics[scale=0.3]{figure/correlation_gdppc_uni_rca.png}
	\includegraphics[scale=0.3]{figure/correlation_gdp_uni_share.png}
	\caption{\textbf{Extensive fitness as GDP precursor}. Average correlations (Pearson coefficient) among the GDP and Fitness series with temporal lag. The left panel has the intensive measures while the right panel has the extensive ones. The filled regions is the region in between the $25\%$ and the $75\%$ quantile from the application of the bootstrap. The diagonal line is reported for the visualization of the variation of the strength relations.}
	\label{fig:correlation}
\end{figure}
As shown in references~\cite{Tacchella2018,Cristelli2017}, the Fitness metrics can be used to forecast the GDP variations with a relatively long time interval and high accuracy.
Hence, in this work we address a basic analysis in which we consider the time delayed cross-correlation between the Fitness measures and the respective GDP values with the aim to evaluate if the aggregation of the Services possibly introduce more signal or more noise in the forecasting.
In Figure~\ref{fig:correlation}, on the left, we plot the correlation between the intensive Fitness and the GDP per capita (GDP pc). 
One can easily see that lines are essentially flat, indicating that, albeit correlated, there is no clear time direction from Fitness to GDP pc or vice-versa for either measure.
Interestingly, this is the case of also the 6-digits Fitness usually implemented in the GDP analysis.
On the right, we show the extensive counterpart: the Fitness computed using the Market Shares and the total GDP, not normalized using the population.
The Extensive Fitness, and in particular the Universal Extensive Fitness, that contains both physical Goods and Services, is clearly able to predict the GDP with a time delay of up to 20 years, that however is of the order of the maximal extension of our data. 
Note that we are using the term prediction in the Economic sense, \textit{i.e.} this not an out of sample forecast but a time delayed correlation. 
In this sense, we can safety say that the Universal Extensive Fitness is able to statistically anticipate, or is a precursor of the total GDP. 
However, the converse is not completely true: although there is a signal of correlation, the GDP has a lower predictive power and the Fitness is somehow harder to be anticipated.\\ 
\begin{figure}[!h]
	\centering
	%	\includegraphics[scale=0.35]{figure/order_cpx_uni_share_smooth.png}
	\includegraphics[scale=0.35]{figure/order_cpx_uni_share_smooth_top.png}
	\includegraphics[scale=0.35]{figure/order_cpx_uni_share_smooth_bottom.png}
	\caption{The series of the ordering of the Universal Sectors. The red colors refer to the physical Goods while the blue colors indicates the Services.}
	\label{fig:complexity_orders}
\end{figure}
Finally, we show in figure~\ref{fig:complexity_orders} the time evolution of the Complexity rankings for both Goods (in red) and Services (in blue). 
Even if some noise is present, the rankings clearly follows the layman's intuition of which sector may be sophisticated and which is not. 
In the high Complexity rankings we find services such as R\&D and royalties, and manufacturing sectors such as nuclear reactors, optical instruments, and aircraft. 
Instead, low Complexity products correctly correspond to lower capabilities requirements such as Agrifood sectors.
Note that some top code such as 'Games, sport requisites' includes also accessories and sophisticated products necessary for any kind of sport and competition, while others codes like Umbrellas are boosted by the fact that more than $83\%$ of their Trade is performed by China.
%***che c'è in questi due digit?***
% 95 Toys, games and sports requisites; parts and accessories thereof
% 66 Umbrellas, sun umbrellas, walking-sticks, seat sticks, whips, riding crops; and parts thereof (this is not really diversified)
%

%%%%%%%%%%%%%%%%%%%%%%%%%%%%%%%%%%%%%%%%%%%%%%%%%%%%%%%%%%%%%%%%%%%%%%%%%%%%
\section*{Discussion}
All Economic Complexity measures are based on countries-activities bipartite networks, where the word \textit{activity} usually indicates a physical product. %(most importantly, assessments of the competitiveness of countries and the relatedness among products or among countries and products)
Nonetheless, a fast growing fraction of both the International Trades and local economies is based on Services.
Recent attempts tried to integrate the standard UN-COMTRADE database of physical Goods with the IMF BOP database collecting the Service data~\cite{Stojkoski2016,Zaccaria2018services,Mishra2020}.
However, two problems prevented the necessary achievements to reach a level of data control and sanitation comparable to the Goods-only data.
Firstly, the IMF classification is rather convoluted and, as it is, not suitable for a direct application into the EC framework, and the main issue is related to the structure of the classification tree.
Secondly, the significant presence of missing values at different scales of aggregation prevents the realization of a satisfactory level of detail of the classification, forcing the use of various gimmicks to arrange the data.
For instance, Stojkoski et al.~\cite{Stojkoski2016} modified the original Fitness and Complexity algorithm to make it converge. 
In this work we address and propose a solution to both these problems by selecting, on one hand, a self-consistent and not overlapping Service classification and, on the other hand, filling the missing values with a suitable machine learning approach. 
In particular, we compare and test different sanitation strategies and find that a kNN approach gives the lowest reconstruction error.

Once the reconstruction is performed, the \textit{Universal} dataset is available integrated, Goods and Services.
Hence, we applied the new dataset in the Economic Fitness and Complexity framework.
In particular, we build a statistically validated network of Universal sectors bringing to light the interaction among them. 
By construction the interaction is directed because the time evolution of the input matrices is explicitly taken into account.%, and displays the succession of the development signal. 
The resulting network is filtered by validating the interaction with the use of an ensemble of random matrices preserving the degree sequences of nations and activities. 
Thus, only the statistically significant connections are maintained.
Meaningful patterns indicate that there exist a deep interrelation between High-Tech manufacturing and complex Services, and there is a partial isolation of the agrifood sector.

Successively, we computed the Fitness of countries and the Complexity of universal sectors, both in the extensive and the intensive fashions. 
The importance of Services once again emerges, as previous measures of Fitness clearly underestimate the competitiveness of countries such as the United Kingdom. 
Finally, we considered the so-called Extensive Fitness that, being computed starting from the Market Shares, is more connected with the cumulative importance of the country at the International level. 
We show that the Extensive Universal Fitness is a precursor of the GDP, in the statistical sense that it is highly correlated with the GDP when a time delay is accounted. 
In particular, this correlation increases with time, showing that the Universal Extensive Fitness anticipates the GDP, significantly more than the contrary.
This work and most importantly the Universal database, opens up a number of possible lines of research. 
Firstly, even if the results of this paper are focused on the Economic Fitness and Complexity framework, the reconstruction of the Services database and its integration with the Goods data is of general interest and can be also used with other macroeconomic purposes. Secondly, the Universal Fitness can be used to build GDP and GDPpc forecasts using the Selective Predictability Scheme, as done in~\cite{Cristelli2017,Tacchella2018}. 
We expect that the inclusion of Services will improve the predictive performance of this approach. 
Finally, the Universal Progression Network, with or without the support of machine learning approaches, can be used to predict the future activation of sectors now absent in the export basket of countries.

%%%%%%%%%%%%%%%%%%%%%%%%%%%%%%%%%%%%%%%%%%%%%%%%%%%%%%%%%%%%%%%%%%%%%%%%%%%%


\section*{Data Availability}
The datasets generated and analysed during the current study are available in the \emph{Universal database} repository, \url{ https://efcdata.cref.it/}.

\section*{Usage Notes}
The raw database of the Services is the Balance of Payments, collected by the International Monetary Fund (IMF), which is free of charge, as described in the terms of the \emph{Copyright and Usage} (https://www.imf.org/external/terms.htm).
%The database of the physical Goods is the UN-COMTRADE
All the database and the files containing the results of the Fitness and Complexity algorithm are available in the \emph{csv} (comma-separated-values) format of Unicode (UTF-8) string format.
The format of the file requires that the first column contains the row indexes while the first row contains the column indexes.
The choice of the format of the database allows the implementation using various open-source codes and platform, as well as Office suites (and open or free versions).

\section*{Code availability}
The repository contains the Jupyter notebooks written in Python 3 necessary to reproduce the reconstruction and the Fitness and Complexity algorithm.


\section*{Acknowledgements}
We gratefully acknowledge funding received from the Joint Research Centre of Seville (grant number 938036-2019 IT).


\bibliographystyle{unsrt}
\bibliography{biblio}



\end{document}
\end{document}
