%\stackrel{\text{tree}}{=}
%****************************** Style **********************************%
\documentclass[11pt,a4paper]{article}
%***************************** Packages ********************************%
\usepackage{bm,jcappub,Macro,tabularx}
\usepackage{url}
\usepackage{hyperref}
\usepackage{url}
\usepackage{epstopdf}
\usepackage{enumitem}
\usepackage{colortbl}
\usepackage{color}
%\oddsidemargin 0in
%\evensidemargin 0in
%\textwidth 7.05in
%\textheight 8.75in
%\hoffset -.15in
%\voffset .4in
%
\def\neweq{\stackrel{\text{sq}}{\approx}}

\def\xt{({\bf x},\tau)}
\def\xti{({\bf x},\tau_i)}
\def\ti{\tau_i}
\def\cG{{\cal G}}

\def\la{\langle}
\def\ra{\rangle}
\def\n{\noindent}
\def\be{\begin{equation}}
\def\ee{\end{equation}}
\def\ben{\begin{eqnarray}}
\def\een{\end{eqnarray}}
\def\bes{\begin{subequations}}
\def\ees{\end{subequations}}
\def\nn{\nonumber}
\def\oh{\bf \hat\Omega}
\def\myC{{\cal C}}
\def\myB{{\cal B}}
\def\myf{\Theta}
\def\bk{{\bf k}}
\def\rtf{r,\theta,\varphi}
\def\br{{\bf r}}
\def\inc{{\int_0^{r_s}}}
\def\rH{\rm H}
\def\myc{{\cal C}}
\def\myC{{\cal C}}
\def\myB{{\cal B}}
\def\myT{{\cal T}}
\def\rD{{\rm D}}
\def\rW{{\rm W}}
\def\add{{_\delta}}
\def\ho{{\hat \Omega}}
\def\ad{{_{\rm{lin}}}}
\def\bk{{\bf k}}
\def\bp{{\bf p}}
\def\bK{{\bf K}}
\def\bl{{\bf l}}
\def\bx{{\bf x}}
\def\2p{{(2\pi)^2}}
\def\dl{d^2l}
\def\bl{{\bf l}}
\def\be{\begin{equation}}
\def\ee{\end{equation}}
\def\beq{\begin{equation}}
\def\eeq{\end{equation}}
\def\ben{\begin{eqnarray}}
\def\een{\end{eqnarray}}
\def\rpa{{r_{\parallel}}}
\def\rpe{{{\bf r}_{\perp}}}
\def\oh{{\hat\Omega}}
\def\nn{{\nonumber}}
\def\inte{\int_0^{\infty}}
\newcommand{\beqa}{\begin{eqnarray}}
\newcommand{\eeqa}{\end{eqnarray}}
%\newcommand{\edth}{\,\eth\,}
\newcommand{\edthbar}{\,\overline{\eth}\,}
\newcommand{\calX}{{\cal X}}
\newcommand{\calY}{{\cal Y}}
\newcommand{\calZ}{{\cal Z}}
\newcommand{\calU}{{\cal U}}
\newcommand{\calV}{{\cal V}}
\newcommand{\calW}{{\cal W}}
\newcommand{\rA}{{\rm A}}
\def\qtwo{\quad\quad}
\newcommand{\myL}{{\cal L}}
%\newcommand{\calY}{{\cal Y}}
%\newcommand{\calZ}{{\cal Z}}
%\newcommand{\calU}{{\cal U}}
%\newcommand{\calV}{{\cal V}}
%\newcommand{\calW}{{\cal W}}
\newcommand{\rP}{{P}}
\newcommand{\rSZ}{{\rm SZ}}
%\newcommand{\myf}{{\Theta}}
\def\ex{{\int {d^2 {\bf l} \over (2
\pi)^2}~ {\rm P}_{\delta} { \big ( {l\over d_A(r)} \big )} {\rm W}_{\rm TH}^2(l\theta_0)}}
\def\exnew{{\cal J}_{\theta_0}(r)}
\def\ikap0{{\cal J}_{\theta_0}(r)}
\def\ikapp{{\cal J}_{\theta_{12}}(r)}
\def\av{\langle \kappa_{(i)}^2(\theta_0) \rangle}
\def\aviold{\langle \kappa_{(i)}^2(\theta_0) \rangle}
\def\avi{\bar\xi_2^{(i)\kappa}(\theta_0)}
\def\kmin{{\int_0^{r_s}\; \omega_{(i)}(r)\; dr }}
\def\kminnew{{\kappa^{\rm min}_{(i)}}}
\def\one1{\langle \kappa_{(i)}\kappa_{(j)} \rangle}
\def\one{{[\bar \xi^{(ij)}]}}
\def\onei{{[\bar \xi_2^{(i)\kappa}]}}
\def\onej{{[\bar \xi_2^{(j)\kappa}]}}
%
\def\two{\langle \kappa_{\rm }(\oh_1) \kappa_{\rm }(\oh_2) \rangle_c}
\def\two{{[\xi^{(ij)\kappa}_{12}]}}
\def\corr{{{\cal J}_{\theta_{12}}(r)}}
\def\var{{{\cal J}_{\theta_0}(r)}}
\def\oldcorr{\Big [ \int  {d^2 {\bf l} \over (2
\pi)^2 )} {\rm  P}_{\delta}( { {\bf l} \over d_A(r) })  W_{\rm}^2(l\theta_0) \exp [ i l\cdot
\theta_{12}] \Big ]}
\def\oldvar{ \Big [  \int  {d^2 {\bf l} \over (2
\pi)^2 )} {\rm P}_{\delta} ( { {\bf l} \over d_A(r) })  W_{\rm TH}^2(l\theta_0) \Big ] } 
\def\qtwo{\quad\quad}
\def\rsz{{^{\rm SZ}}}
\def\fsky{{f^{-1}_{\rm sky}}}

\def\nfw{(\theta_b)}
\def\fw{\theta_b}
\def\ba{\begin{eqnarray}}
\def\ea{\end{eqnarray}}
\def\f{\frac}
\def\l{\left}
\def\r{\right}
\def\hub{{\cal H}}
\def\nmu{\nabla_{\mu}}
\def\nnu{\nabla_{\nu}}
\def\vp{\varphi}
\def\vt{\vartheta}
%
\def\bk{{\bf k}}
\def\bq{{\bf q}}
\def\rL{{\rm L}}
\def\rD{{\rm D}}
\def\rF{{\rm F}}
%
\def\bell{{\bm\ell}}

\def\pfact{{P_{\delta}(k)P_{\delta}(q_b)}}

%
%%%%%%%%%%%%%%%%%%%%%%%%%%%%%%%%%%%%%%%%%%%%%%%%%%%%%%%%%%%%%%%%%%%%%%%%%%%%%%%%%%%%%%%%%%%%%%%%%%
\title{Symmetries, Invariants and Generating Functions: Higher-order Statistics of Biased Tracers}
%%%%%%%%%%%%%%%%%%%%%%%%%%%%%%%%%%%%%%%%%%%%%%%%%%%%%%%%%%%%%%%%%%%%%%%%%%%%%%%%%%%%%%%%%%%%%%%%%%
\author{Dipak Munshi}
\affiliation{Astronomy Centre, School of Mathematical and Physical Sciences,\\ University of Sussex, Brighton BN1 9QH, U.K.}
\emailAdd{D.Munshi@sussex.ac.uk}
\abstract{Gravitationally collapsed objects are known to be biased tracers of an underlying density contrast.
Using symmetry arguments, generalised biasing schemes have recently been developed to
relate the halo density contrast $\delta_h$ with the underlying density contrast $\delta$, divergence of velocity $\theta$ 
and their higher-order derivatives. This is done by constructing invariants such as $s, t, \psi,\eta$.
We show how the generating function formalism in Eulerian standard perturbation theory (SPT) can be used to show that many of the
additional terms based on extended Galilean and Lifshitz symmetry actually do {\em not} 
make any contribution to the higher-order statistics of biased tracers.
Other terms can also be drastically simplified allowing us to write the vertices associated with
$\delta_h$ in terms of the vertices of $\delta$ and $\theta$, the higher-order derivatives and the bias coefficients.
We also compute the cumulant correlators (CCs) for two different tracer populations. 
These perturbative results are valid for {\em tree-level} contributions but at an {\em arbitrary order}. 
We also take into account the stochastic nature bias in our analysis.
Extending previous results of a local polynomial model of bias, we express the one-point cumulants ${\cal S}_N$ and their two-point counterparts,
the CCs i.e. ${\cal C}_{pq}$, of biased tracers in terms of that of their underlying density contrast counterparts.
As a by-product of our calculation we
also discuss the results using approximations based on Lagrangian perturbation theory (LPT).}
\keywords{Cosmology, Large-Scale Structure, Perturbation Theory} 
\begin{document}  
\maketitle
%
\bigskip
\bigskip
{\it ``Why do you want to know his bias? Form your own bias!''  R.P.Feynman}
%
\section{Introduction}


Tate introduced a $p$-adic analytic geometry 
so-called the rigid geometry. 
In the original definition by Tate, 
a rigid analytic space is not a topological space but 
a Grothendieck topological space (cf. \cite{BGR84}). 
To remedy this situation, 
Huber established the theory of adic spaces. 
He introduced a topological space $\Spa\,A$, 
called an affinoid spectrum, associated to an affinoid ring $A=(A^{\rhd}, A^+)$, 
where $A^{\rhd}$ is an f-adic ring and 
$A^+$ is a certain open subring of $A^{\rhd}$. 
An adic space is obtained by gluing affinoid spectra. 

Although Huber introduced a structure presheaf $\MO_A$ 
on $\Spa\,A$ for an arbitrary affinoid ring $A$, 
it is not a sheaf in general 
(cf. \cite{BV}, \cite[the example after Proposition 1.6]{Hub94}, \cite{Mih16}). 
The reason for this is that 
the definition of $\MO_A$ depends on the completion, 
which is a transcendental operation. 
Thus it is natural to ask whether 
the theory can be more well-behaved 
after replacing the completion by a more algebraic operation. 
For example, if $(A, I)$ is a pair consisting of a ring $A$ and an ideal $I$ of $A$, 
then we can associate the Zariskian ring $A^{\Zar}$ and 
the henselisation $A^h$ with respect to $I$. 
The henselisation $A^h$ is known as an algebraic approximation of the $I$-adic completion $\widehat{A}$, 
whilst the associated Zariskian ring $A^{\Zar}$ is closer to the original ring $A$ than $A^h$. 
The purpose of this paper is to establish the Zariskian version 
of Huber's theory. 
Therefore, the first step is to introduce a notion of Zariskian f-adic rings. 

\begin{dfn}[Definition~\ref{d-zar}, Remark \ref{r-zar1}, Definition~\ref{d-zar-top}]\label{intro-d-zar}
Let $A$ be an f-adic ring. 
\begin{enumerate}
\item 
We set $S_A^{\Zar}:=1+A^{\circ\circ}$. 
It is easy to show that $S_A^{\Zar}$ is a multiplicative subset of $A$. 
We set $A^{\Zar}:=(S_A^{\Zar})^{-1}A$. 
Both $A^{\Zar}$ and the natural ring homomorphism $\alpha:A \to A^{\Zar}$ 
are called the {\em Zariskisation} of $A$. 
We say that $A$ is {\em Zariskian} if $\alpha:A \to A^{\Zar}$ 
is bijective. 
\item 
For a ring of definition $A_0$ of $A$ and 
an ideal of definition $I_0$ of $A_0$, 
we equip $A^{\Zar}$ with the group topology 
defined by the images of $\{I_0^kA_0^{\Zar}\}_{k\in \Z_{>0}}$. 
We can show that 
this topology does not depend on the choice of $A_0$ and $I_0$ 
(cf. Lemma~\ref{l-top-indep}). 
\end{enumerate}
\end{dfn}

We will prove that $A^{\Zar}$ satisfies some reasonable properties. 
For instance, $A^{\Zar}$ is a Zariskian f-adic ring (Theorem~\ref{t-zar-zar}) and 
$A^{\Zar}$ has the same completion as the one of $A$ 
(Theorem~\ref{t-comp-factor}). 
However one might consider that 
the definition of the topology of $A^{\Zar}$ 
is somewhat artificial. 
The following theorem asserts that 
our definition given above can be characterised 
in a category-theoretic way, 
i.e. $A^{\Zar}$ is an initial object of 
the category of Zariskian f-adic $A$-algebras. 


\begin{thm}[Theorem~\ref{t-zar-univ}]\label{intro-t-zar-univ}
Let $A$ be an f-adic ring and 
let $\alpha:A \to A^{\Zar}$ be the Zariskisation of $A$. 
Then, for any continuous ring homomorphism $\varphi:A \to B$ 
to a Zariskian f-adic ring $B$, 
there exists a unique continuous ring homomorphism 
$\psi:A^{\Zar} \to B$ such that $\varphi=\alpha \circ \psi$. 
\end{thm}





For an affinoid ring $A=(A^{\rhd}, A^+)$, 
we introduce a presheaf $\MO_A^{\Zar}$ on $\Spa\,A$ 
in the same way as in Huber's theory. 
The presheaf $\MO_A$ introduced by Huber is not a sheaf in general, 
whilst the Zariskian version $\MO^{\Zar}_A$ 
is always a sheaf. 

\begin{thm}[Theorem~\ref{t-sheafy}]
For an affinoid ring $A=(A^{\rhd}, A^+)$, the presheaf $\MO_A^{\Zar}$ on $\Spa\,A$ is a sheaf. 
\end{thm}

Then one might be tempted to hope 
the Tate acyclicity in general. 
Unfortunately this is not the case. 


\begin{thm}[Theorem~\ref{t-non-TA}]
There exists an affinoid ring $A=(A^{\rhd}, A^+)$ such that 
$H^1(\Spa\,A, \MO_A^{\Zar}) \neq 0$. 
\end{thm}


Although the Zariskian structure sheaf $\MO_A^{\Zar}$ 
does not behave nicely 
to establish a theory of coherent sheaves, 
the Zariskian rings might be still useful. 
For instance, if $A$ is a noetherian ring equipped with an $\m$-adic topology for some maximal ideal $\m$, 
then the Zariskisation $A^{\Zar}$ is nothing but 
the local ring $A_{\m}$ at $\m$. 
Therefore, in this situation, 
$A$ is Zariskian if and only if its $\m$-adic completion 
$A \to \widehat{A}$ is faithfully flat. 
The flatness of completion is a thorny problem 
for non-noetherian rings, 
whilst we prove that the completion is actually faithfully flat, 
under the assumption that $A$ is Zariskian and the completion is flat. 
More generally, we obtain the following result. 


\begin{thm}[Corollary~\ref{c-ff-criterion}]\label{intro-t-ff}
Let $\varphi:A \to B$ be a continuous ring homomorphism 
of Zariskian f-adic rings. 
Assume that the induced map $\widehat{\varphi}:\widehat{A} \to \widehat{B}$ 
is an isomorphism of topological rings. 
Then the following hold. 
\begin{enumerate}
\item 
Any maximal ideal of $A$ is contained in the image of 
the induced map $\Spec\,B \to \Spec\,A$. 
\item 
If $\varphi$ is flat, then $\varphi$ is faithfully flat. 
\end{enumerate}
\end{thm}


Theorem~\ref{intro-t-ff} is a consequence of 
the following characterisation of Zariskian f-adic rings. 


\begin{thm}[Theorem~\ref{t-characterise}]
Let $(A, A^+)$ be an affinoid ring. 
Then $A$ is Zariskian if and only if 
any maximal ideal of $A$ is contained in the image of 
the natural map 
$$\theta:\Spa\,(A, A^+) \to \Spec\,A, \quad v \mapsto \Ker(v).$$
\end{thm}

For a Zariskian affinoid ring $(A, A^+)$, 
the above theorem claims that 
the image of $\theta$ contains all the maximal ideals, 
however the map $\theta$ 
is not surjective in general (Theorem~\ref{t-non-surje}). 


\medskip

\textbf{Acknowledgement:} 
The author was funded by EPSRC. 
He would like to thank the referee for reading the paper carefully and for giving many constructive comments. 
%\documentclass[preprint,12pt]{elsarticle}
%\if0
\usepackage{amssymb}
\usepackage{mathtools}
%\usepackage[dvipdfmx]{graphicx}
\usepackage{cite}
\usepackage{graphicx}
\usepackage{bm}
\usepackage{here}
\usepackage[subrefformat=parens]{subcaption}
\fi
%\usepackage{amssymb}
\usepackage{amsmath}
\usepackage[dvipdfmx]{}
\usepackage[dvipdfmx]{color}
%\usepackage{cite}
%\usepackage{upgreek}
\usepackage{url}
%\usepackage[dvipdfmx]{hyperref}
%\usepackage{pxjahyper}
%\usepackage {hyperref}
\usepackage{graphicx}
\usepackage{bm}
\usepackage{here}
\usepackage{caption}
\usepackage[subrefformat=parens]{subcaption}
\captionsetup{compatibility=false}

%% The amsthm package provides extended theorem environments
%% \usepackage{amsthm}

%% The lineno packages adds line numbers. Start line numbering with
%% \begin{linenumbers}, end it with \end{linenumbers}. Or switch it on
%% for the whole article with \linenumbers after \end{frontmatter}.
%% \usepackage{lineno}

%% natbib.sty is loaded by default. However, natbib options can be
%% provided with \biboptions{...} command. Following options are
%% valid:

%%   round  -  round parentheses are used (default)
%%   square -  square brackets are used   [option]
%%   curly  -  curly braces are used      {option}
%%   angle  -  angle brackets are used    <option>
%%   semicolon  -  multiple citations separated by semi-colon
%%   colon  - same as semicolon, an earlier confusion
%%   comma  -  separated by comma
%%   numbers-  selects numerical citations
%%   super  -  numerical citations as superscripts
%%   sort   -  sorts multiple citations according to order in ref. list
%%   sort&compress   -  like sort, but also compresses numerical citations
%%   compress - compresses without sorting
%%
%% \biboptions{comma,round}

% \biboptions{}

%% This list environment is used for the references in the
%% Program Summary
%%
\newcounter{bla}
\newenvironment{refnummer}{%
\list{[\arabic{bla}]}%
{\usecounter{bla}%
 \setlength{\itemindent}{0pt}%
 \setlength{\topsep}{0pt}%
 \setlength{\itemsep}{0pt}%
 \setlength{\labelsep}{2pt}%
 \setlength{\listparindent}{0pt}%
 \settowidth{\labelwidth}{[9]}%
 \setlength{\leftmargin}{\labelwidth}%
 \addtolength{\leftmargin}{\labelsep}%
 \setlength{\rightmargin}{0pt}}}
 {\endlist}
\begin{document}

\section{O-SUKI-N 3D code algorithm description}
\par

\subsection{O-SUKI-N 3D code structure}
     The O-SUKI-N 3D code system consists of three parts: The Lagrangian fluid code \cite{Schulz}, the data conversion code from the Lagrangian code to the Euler code, and Euler code. The fluid model is the three-temperature model in Ref. \cite{Tahir}. The Lagrangian fluid code, the data conversion code and the Euler code are described below in detail. 
     
     In the Lagrangian fluid code the spatial meshes move together with the fluid motion \cite{Schulz}. However, the Lagrange meshes can not follow the fluid large deformation. On the other hand, the Euler meshes are fixed to the space, and the fluid moves through the meshes. Therefore, just before the void closure time, that is, the stagnation phase, the Lagrangian code is used to simulate the DT fuel implosion. After the void closure time, the Euler code is employed to simulate the DT fuel further compression, ignition and burning. Between the Lagrangian code and the Euler code the data should be converted by the data conversion code. 

	All the simulation process is performed in its integrated way by using the script of "CodeO-SUKI-N-fusion-start.sh". The processes executed by this shell script are as follows: \\
1. Make the stack size infinite.\\
2. Remove all output data file and make the new output files.\\
3. Change the permission of shell scripts to executable. \\
4. Compile the main function of the Lagrangian code and execute it.\\
5. If any problems do not appear during the calculation of the Lagrangian code, compile the main function of the data conversion code and execute it.\\
6. If there is no problem during the data conversion, compile the main function of the Euler code and execute it.\\
     

\subsection{Steps in Lagrangian code}\par
     The Lagrangian code has the following steps: 

\begin{enumerate}
\item Initialize the variables and calculation of total input energy. \par
\item Calculation of time step size.\par
\item Calculation of coordinates.\par
\item Solve equation of motion. \par
\item Solve density by equation of continuity.\par
\item Calculation of artificial viscosity.\par
\item Transfer the data to the OK3. \par
\item Calculation of energy deposition distribution in code OK3. For details of the OK3, see the refs.\cite{ogoyski1,ogoyski2,ogoyski3}. \par
\item Solve energy equations\par
\item Calculation of heat conduction\par
\item Calculation of temperature relaxation among three temperatures.\par
\item Solve equation of state\par
\item Save the results.\par
\item End the Lagrangian calculation right before the void closure.\par
\item Transfer the data to converting code. \par
\end {enumerate}


\subsection{Data Conversion code from Lagrangian fluid code to Euler fluid code}

\begin {enumerate}
\item Read variables saved in Lagrangian code.\par
\item Generate the Eulerian mesh.\par
\item Calculate the interpolation of the physical quantity to them on the Eulerian mesh.\par
\item Write the converted data to the Eulerian code.\par
\end {enumerate}


\subsection{Steps in Eulerian code}

\begin {enumerate}
\item Read the mesh number from the converted data and define the each matrices.\par
\item Initialize the variables.\par
\item Calculation of time step size.\par
\item Solve equation of motion. \par
\item Track the material boundaries of DT, Al and Pb.\par
\item Linearly interpolate the boundary lines and transcribe them on the Eulerian code. \par
\item Discriminate the materials by using the transferred boundary line. \par
\item Solve density by equation of continuity.\par
\item Calculate artificial viscosity.\par
\item Solve energy equations\par
\item Calculation of fusion reaction.\par
\item Calculation of heat conduction\par
\item Calculation of temperature relaxation among three temperatures.\par
\item Solve equation of state.\par
\item Save the results.\par
\item End.
\end{enumerate}

%\end{document}
\section{Small energy regime}
In this section we prove two results under a small energy assumption. The first one is a clearing out result on the boundary. The second is the epsilon regularity result. We emphasize that these are proved for $\ue$, with estimates that are uniform in $\ve$. This is crucial for their application in the study of $\ue$ as $\ve \to 0$.

\subsection{Clearing out on the boundary} This is an important consequence of the monotonicity formula. It captures the intuition that if the energy is small enough then $\ue$ stays close to the potential wells uniformly with respect to $\ve$.
\begin{lemm} \label{lemmaclearing}
Let $\ue$ be a solution of (\ref{eqn:main}) for $R=1$ such that $|\ue| \leq 1$. There is a constant $\eta$ independent of $\ve$ such that such that $E_\ve(\ue,\ball 1)\leq \eta$ implies $|\ue| \geq \frac{1}{2}$ on $D_{1/2}$.
\end{lemm}
\begin{proof}


We first consider solution $u$ for $\ve =1$ and prove the claim by contradiction. Then we will prove it for any $\ve<1$ by rescaling. As $|u| \leq 1$, we have by (\ref{csmest}) that $\norm{u}_{C^{2,\alpha}(\ball {1/2})} \leq C_{\alpha}$. Let $\eta_1=2^{n-1}\eta$ such that $E_\ve(\ue,\ball 1)\leq \eta_1$. If the result is not true then we can find a sequence of solutions $\ui = u_{\ve_i}$ of decreasing energy and points $x_i \in D_{1/2}$ such that $|\ui(x_i)|\leq 1/2$ and $E_1(u_i,\ball 1) \to 0$. Due to the uniform estimate we have uniform convergence of $u_i$ in $\overline{\ball {1/2}}$. As $E_1(u_i,\ball 1) \to 0$, $u_i \to 1$ on $D_{1/2}$ which contradicts $|\ui(x_i)|\leq 1/2$. This proves the case $\ve=1$. 

Now for any $\epsilon<1$, for any solution $\ue$ and any point $x_0 \in D_{1/2}$,  consider the map $x \to x_0 + \ve x$ sending $\ball 1 \to \ball \ve (x_0)$. Then $v_{\Tilde{\ve}}(x)=\ue(x_0+\ve x)$ satisfies (\ref{eqn:main}) for $\ve=1$ and $R=1$. Further note that $E_1(v_{\Tilde{\ve}}, \ball 1)= I_{\ve}(\ve,x_0) \leq 2^{n-1} \eta=\eta_1$, the inequality is by (\ref{eqn:mflemma}). So we can apply the $\ve=1$ result which gives $|v_{\Tilde{\ve}}| \geq 1/2$ on $D_{1/2}$, but this is same as $|\ue| \geq 1/2$ on $D_{\frac{\ve}{2}}(x_0)$ for any $x_0 \in D_{1/2}$.
\end{proof}


\subsection{Epsilon regularity} We now prove the epsilon regularity result. It will become clear soon that it should be thought of as stating that if the energy bound is small enough, then the gradient of $\ue$ is bounded uniformly independent of $\ve$ \textit{up to the boundary.} 


\begin{theo}  There exist constants $\eta_0$ and $C_0$ independent of $\ve$ such that for $\ve <R$, and $u_\ve \in C^2(\ball R \cup D_R)$ satisfying $|u_\ve| \leq 1$ solving 

    \begin{equation} \label{eqn:Rscale}
        \begin{cases}
            \Delta \ue = 0 \qquad &\text{in } B_R^+ \\
            \pdv{\ue}{\nu} = -\frac{1}{\ve}W'(u_\ve) \qquad  &\text{on  } D_R
        \end{cases}
    \end{equation}
If we have $I_{\ve}(R,0) \leq \eta_0$, then 
    \begin{equation} \label{epsreg}
        \sup_{\ball {\frac{R}{4}}} |\nabla \ue|^2 + \sup_{D_{\frac{R}{4}}} \frac{W(\ue)}{\ve ^2} \leq \frac{C_0}{R^2}\eta_0
    \end{equation}
\end{theo}
\begin{rem}
We use an idea due to Schoen \cite{s} in harmonic maps setting, also used in similar geometric problems by several authors \cite{cs,cw,ms,t,w}. To obtain an estimate independent of $\ve$ we need a scale $r_{\ve}$ for which the gradients are uniformly bounded. Then the problem reduces to having a uniform bound on $r_{\ve}$. In \cite{s} this is done by using the mean value property to get a contradiction to the smallness of energy. However, due to the boundary we may only use the mean value property for $|\nabla \ue|^2$ only for points sufficiently away from the boundary. In the other case, we will rescale the problem to $r_{\ve}$-scale and then use the convergence lemma \ref{lem:conv} to get a contradiction to smallness of energy
\end{rem}

\begin{proof}
It is sufficient to prove the result for $R=1$. First observe that we have 
$$\frac{W(u)}{\ve^2}=\frac{1}{4u^2}\frac{W'(u)^2}{\ve^2} = \frac{1}{4u^2}\bigg|\pdv{\ue}{\nu}\bigg|^2 $$
We may assume that $I_\ve \leq \eta$, then by clearing out result (\ref{lemmaclearing}), $1/2 \leq|u_\ve|\leq 1$ on $D_{1/2}$ and therefore
\begin{equation*}
    \frac{W(u)}{\ve^2}= \frac{1}{4u^2}\bigg|\pdv{\ue}{\nu}\bigg|^2  \leq \bigg|\pdv{\ue}{\nu}\bigg|^2
\end{equation*}

Therefore to establish (\ref{epsreg}), it is enough estimate the gradient $|\nabla \ue|$ upto the boundary, i.e. we need to show
\begin{equation} \label{gradest}
    \sup_{\overline{\ball {\frac{1}{4}}}} |\nabla \ue| \leq C\sqrt{\eta_0} 
\end{equation}

Consider the distance weighted gradient on $\overline{\ball {\frac{1}{2}}}$,  $F(s)=(\frac{1}{2}-s)|\nabla \ue(x)|$. It attains its maximum for some $s_{\ve} \in [0,1/2]$ and we have 
$$\max _{s} F(s) = \max_{s} \bigg(\frac{1}{2}-s \bigg) \sup_{\overline{\ball s}}|\nabla \ue| =\bigg(\frac{1}{2}-s_{\ve}\bigg) \sup_{\overline{\ball {s_{\ve}}}} |\nabla \ue| $$

Let $x_\ve$ be such that $\sup_{\overline{\ball {s_{\ve}}}} |\nabla \ue|=|\nabla \ue(x_\ve)| =e_\ve$. Note that due to the definition of $F(s)$, $|x_{\ve}|=s_{\ve}$, so $\dist{(x_\ve, \partial^+ \ball \frac{1}{2})}= \frac{1}{2}-s_\ve$. We write $$ \frac{1}{2}-s_\ve = 2\rho_\ve \quad \text{and} \quad r_\ve = \rho_\ve e_\ve$$

Then we have $\max_{s} F(s) =2\rho_\ve  e_\ve = 2r_\ve $. Taking $s=\frac{1}{4}$ gives
\begin{equation} \label{eqn:grad1}
    \sup_{\overline{\ball {\frac{1}{4}}}} |\nabla \ue| \leq 8 r_\ve
\end{equation}
Therefore the gradient estimate would follow from a uniform bound on $r_\ve$. We collect another consequence of definition of $F$ that will be used later.
\begin{equation} \label{eqn:gs}
    |\nabla \ue|(x) \leq \frac{(\frac{1}{2}-s_\ve)}{(\frac{1}{2}+s_\ve)}2 e_\ve < 2e_\ve \quad \text{for all } x \in B_{\rho_\varepsilon}(x_\ve) \cap \overline{\ball {\frac{1}{2}}} 
\end{equation}

%Now we first deal with the case when $x_\ve$ is sufficiently away from $D_1$. 
Let $\overline{x_\ve}$ be projection of $x_\ve$ on $D_1$. Then denote by $z_\ve$ the height of $x_\ve$, i.e $z_\ve = x_\ve - \overline{x}_{\ve}$. Depending on how the height of $x_\ve$ compares to it is distance from the spherical boundary we get the above described two cases.


%\begin{enumerate}
    %\item 
      \textit{Case 1:}   $ \frac{z_\ve}{2\rho_\ve} > \frac{1}{4}$ i.e., away from  $D_1$\\
           The ball $B_{\frac{z_\ve}{2}}(x_\ve) \subset \ball {2z_\ve}(\overline{x}_\ve) \subset \ball 1 $, then by mean value property for $|\nabla u_\ve|^2$ we have 
        \begin{equation} \label{eqn:case1}
            e_\ve ^2 \leq \frac{1}{|B_{\frac{z_\ve}{2}}(x_\ve)|}\int_{B_{\frac{z_\ve}{2}}(x_\ve)} |\nabla u_\ve|^2 \,dx 
                    \leq \frac{1}{4z_{\ve}^2}\cdot C\eta_0  
        \end{equation}
        The second inequality follows from (\ref{eqn:mflemma}). Combining this with (\ref{eqn:grad1}) gives the desired estimate
        $$ \sup_{\overline{\ball {\frac{1}{4}}}} |\nabla \ue| \leq 8 r_\ve < 8\cdot  2z_\ve e_\ve \leq C\sqrt{\eta_0} $$   
   % \item 


\textit{Case 2: }$ \frac{z_\ve}{2\rho_\ve} \leq \frac{1}{4}$ i.e., close to $D_1$\\
            Given (\ref{eqn:grad1}), if $r_\ve \leq 1$, then we are done. So we assume otherwise, i.e $r_\ve > 1$ and arrive at a contradiction. For this, we will consider the problem at the $r_\ve$-scale. Consider $\ue$ solving (\ref{eqn:Rscale}) in the ball $\ball {\rho_{\ve}}(\overline{x}_\ve) \subset \ball {\frac{1}{2}} $. Then for $x \in \ball {r_\ve} \cup D_{r_\ve}$, with $\Tilde{\ve} = \ve e_\ve$, take
            $v_{\Tilde{\ve}}(x) = \ue (\overline{x}_\ve + x/e_\ve) $. With this rescaling $\ball {\rho_\ve}(\overline{x}_\ve)$ goes to $\ball {r_\ve}$. As $r_\ve > 1$, all $v_{\Tilde{\ve}}$ solve
            \begin{equation} \label{eqn:rscale}
        \begin{cases}
            \Delta v_{\Tilde{\ve}} = 0 \qquad &\text{in } \ball {1} \\
            \pdv{v_{\Tilde{\ve}}}{\nu} = -\frac{1}{\Tilde{\ve}}W'(v_{\Tilde{\ve}}) \qquad  &\text{on  } D_{1}
        \end{cases}
    \end{equation}
           Further due to our assumptions, and (\ref{eqn:gs}) we have
        \begin{equation} \label{eqn:convestimates}
            |\nabla v_{\Tilde{\ve}} (y_\ve)|=1 ,|\nabla v_{\Tilde{\ve}}| \leq 2,|v_{\Tilde{\ve}}| \leq 1 \quad \text{in } B_1^+ \cup D_1 \text{and } |v_{\Tilde{\ve}}|\geq 1/2 \quad \text{on } D_1 
        \end{equation}
        further using (\ref{eqn:mflemma}) we also get
        \begin{equation}\label{eqn:energyconv}
            E_{\Tilde{\ve}}(v_{\Tilde{\ve}}, B_1)<E_{\Tilde{\ve}}(v_{\Tilde{\ve}}, B_{r_\ve})=I_\ve(\rho_\ve,\overline{x}_\ve) \leq 2^{n-1} \eta_0
        \end{equation}
            Here $y_\ve = e_\ve(x_\ve-\overline{x}_\ve)$, so $|y_\ve|= z_\ve e_\ve$. Note that $z_\ve \leq \frac{1}{2}\rho_\ve$ is equivalent to $z_\ve e_\ve \leq r_\ve/2$ but we need $y_\ve \in \ball {1/2}$, indeed this is the case i.e.   $z_\ve e_\ve \leq 1/2$ as if we have $z_\ve > \frac{1}{2e_\ve}$ then exactly like (\ref{eqn:case1}) we have
            $$ e_\ve^2 \leq \frac{1}{4z_\ve^2}\cdot C\eta_0 \leq e_\ve^2\cdot C\eta_0$$ this gives $1 \leq C\eta_0 $
            which is false for $\eta_0$ small enough. Therefore, $y_\ve \in \ball {1/2}$. To simplify notation we write, $v_i=v_{\Tilde{\ve_i}}$ and $y_i=y_{\ve_{i}}$. 
            
            We will show that as $i \to \infty$, we have $v_i \to v_*$ $C^{1,\alpha}_{loc}(\ball 1 \cup D_1)$, then it will give us
            
            \begin{equation}\label{assump}
                \begin{cases}
                 \nabla v_i(y_i) \to \nabla v_*(y_*) \text{ thus } |\nabla v_*(y_*)|=1 \text{ as } |\nabla v_i(y_i)|=1  \\
               \text{There is a }\sigma<\frac{1}{10} \text{ such that }|\nabla v_*|  \geq 1/2 \text{ on }B_{\frac{1}{10}}(y_*)\cap \ball 1 
            \end{cases}
            \end{equation}
    This gives us the following. The last inequality is due to the estimate (\ref{eqn:energyconv}).
        \begin{equation}
            \frac{\sigma^{n+1}|\ball 1|}{2} \leq \int_{\ball 1}|\nabla v_*|^2\,dx \leq \liminf_{i\to \infty} \int_{\ball 1} |\nabla v_i|^2 \,dx \leq 2^{n-1} \eta_0
        \end{equation}    

This leads to a contradiction for small enough $\eta_0$. Therefore $r_\ve <1$ as desired. 



We now just need to show that $v_i \to v_*$ in $C^{1,\alpha}_{loc}(\ball 1 \cup D_1)$.  Recall that $\Tilde{\ve_i}=\ve_i e_{\ve_i}$. We have two cases: As $\ve_i \to 0$ we also have $\Tilde{\ve}_i \to 0$ Then because of the estimates (\ref{eqn:convestimates}), we can apply lemma (\ref{lem:conv}) to $\{v_i\}$.This gives uniform $C^{1,\alpha}_{loc}$ convergence $v_i$ to $v_*$ up to the boundary as required. However, if as $\ve_i \to 0$, $\Tilde{\ve}_i \nrightarrow 0$, that is $\Tilde{\ve}_i= e_{\ve_i} \ve_i \geq \beta >0$. Then by (\ref{csmest}) we have uniform $C^{2,\gamma}$ estimate up to the boundary, $\norm{v_i}_{C^{2,\gamma}_{loc}(\overline{\ball {3/4}}) } \leq C_{\gamma}$. By Holder interpolation, this gives uniform $C^{1,\alpha}_{loc}$ convergence $v_i$ to $v_*$ up to the boundary in this case as well. This finishes the proof.
\end{proof}

With the monotonicity formula, convergence lemma and the epsilon regularity result we can now study the behavior of $\ue$ and the associated energy as $\ve \to 0$. 



\section{Zariskian adic spaces}

In this section, we introduce Zariskian adic spaces. 
In Subsection~\ref{ss-affinoid}, 
we establish foundations for the affinoid case. 
In particular we define a presheaf $\MO_A^{\Zar}$ on $\Spa\,A$ 
for any affinoid ring $A$. 
In Subsection~\ref{ss-structure}, 
we prove that $\MO_A^{\Zar}$ is actually a sheaf. 
In Subsection~\ref{ss-zar-ad-sp}, 
we define Zariskian adic spaces. 
In Subsection~\ref{ss-cex-Tate}, 
we exhibit examples that violate the Tate acyclicity. 

\subsection{Affinoid case}\label{ss-affinoid}

In this subsection, we introduce a presheaf $\MO_A^{\Zar}$ on $\Spa\,A$ 
for any affinoid ring $A$. 
This is a Zariskian analogue of Huber's affinoid adic spaces 
introduced in \cite[Section 1]{Hub94}. 
Indeed, most of our arguments are quite similar 
to the ones in \cite[Section 1]{Hub94}. 

In (\ref{sss-rat-localise}), we consider 
what the definition of $\MO_A^{\Zar}(U)$ should be for rational subsets $U$. 
In (\ref{sss-zar-str}), we define $\MO_A^{\Zar}$  
based on results obtained in (\ref{sss-rat-localise}). 


\subsubsection{Rational localisation}\label{sss-rat-localise}



Let $A=(A^{\rhd}, A^+)$ be an affinoid ring and 
let $U$ be a rational subset of $\Spa\,A$. 
Then there exist elements $f_1, \cdots, f_r, g \in A^{\rhd}$ such that 
$(f_1, \cdots, f_r, g)$ is an open ideal of $A^{\rhd}$ and 
$$U=R\left(\frac{f_1, \cdots, f_r}{g}\right).$$
Take a ring of definition $A_0$ of $A^{\rhd}$ 
and an ideal of definition $I_0$ of $A_0$. 
We set $B^{\rhd}:=A^{\rhd}[1/g]=(A^{\rhd})_g$ and  
$B_0:=A_0\left[f_1/g, \cdots, f_r/g\right] \subset B^{\rhd}.$ 
We equip $B^{\rhd}$ with the group topology induced by $\{I^k_0 B_0\}_{k \in \Z_{>0}}$. 
We can check that this topology does not depend on the choice 
of $A_0$ and $I_0$. 
Set $B^+$ to be the integral closure of 
$B^+[f_1/g, \cdots, f_r/g]$ in $B^{\rhd}$. 
We set 
$$A\left(\frac{f_1, \cdots, f_r}{g}\right):=(B^{\rhd}, B^+).$$
It follows that $A\left(\frac{f_1, \cdots, f_r}{g}\right)$ is 
an affinoid ring and the induced ring homomorphism 
$A \to A\left(\frac{f_1, \cdots, f_r}{g}\right)$ 
is adic. 
Let $A\left(\frac{f_1, \cdots, f_r}{g}\right)^{\Zar}$ be 
the Zariskisation of $A\left(\frac{f_1, \cdots, f_r}{g}\right)$ and 
let 
$$\varphi=\varphi_U:A \xrightarrow{\alpha} A\left(\frac{f_1, \cdots, f_r}{g}\right) 
\xrightarrow{\theta} A\left(\frac{f_1, \cdots, f_r}{g}\right)^{\Zar}=:Z_A(U)$$
be the induced adic ring homomorphisms. 
Although it is not clear that $Z_A(U)$ does not depend on the choices 
of $f_1, \cdots, f_r, g$, 
we shall later see in Lemma~\ref{l-rat-universal1} that this is actually true. 
To this end, we first establish an auxiliary result: 
Lemma~\ref{l-zar-nonempty}. 





\begin{lem}\label{l-zar-nonempty}
Let $A=(A^{\rhd}, A^+)$ be a Zariskian affinoid ring. 
Then the following hold. 
\begin{enumerate}
\item $(A^{\rhd})^{\times}$ is an open subset of $A^{\rhd}$. 
\item If $\m$ is a maximal ideal of $A^{\rhd}$, 
then there exists a point $v \in \Spa\,A$ such that $\m=\Ker(v)$. 
\end{enumerate}
\end{lem}

\begin{proof}
First we show (1). 
Since $1+(A^{\rhd})^{\circ\circ}$ is an open subset of $A^{\rhd}$, 
so is 
$$(A^{\rhd})^{\times}=\bigcup_{x \in (A^{\rhd})^{\times}} x \cdot (1+(A^{\rhd})^{\circ\circ}).$$ 
Thus the assertion (1) holds. 

Second we show (2). 
Since $(A^{\rhd})^{\times}$ is an open subset of $A^{\rhd}$ by (1),  
the inclusion 
$\m \subset A^{\rhd} \setminus (A^{\rhd})^{\times}$ 
implies 
$\overline{\m} \subset A^{\rhd} \setminus (A^{\rhd})^{\times}$, 
where $\overline{\m}$ denotes the closure of $\m$ in $A^{\rhd}$. 
Since $\overline{\m}$ is again an ideal of $A^{\rhd}$ 
such that $\m \subset \overline{\m} \subsetneq A^{\rhd}$, 
we have that $\m=\overline{\m}$. 
We set $B^{\rhd}:=A^{\rhd}/\m$ to be the topological ring 
equipped with the quotient topology induced from $A^{\rhd}$. 
Since $\m=\overline{\m}$, it holds that $B^{\rhd}$ is Hausdorff. 
Let $B^+$ be the integral closure of the image of $A^+$. 
Then $B:= (B^{\rhd}, B^+)$ is an affinoid ring 
(cf. Subsection \ref{ss-quot-aff}). 
It follows from \cite[Proposition 3.6(i)]{Hub93} that 
$\Spa\,B \neq \emptyset$. 
Since we have a natural continuous ring homomorphism $A \to B$, 
any element of the image of $\Spa\,B \to \Spa\,A$ 
is an element $v \in \Spa\,A$ as in the statement. 
Thus (2) holds. 
\end{proof}


\begin{lem}\label{l-rat-universal1}
Let $A=(A^{\rhd}, A^+)$ be an affinoid ring. 
Let $U$ be a rational subset of $\Spa\,A$ 
and let $f_1, \cdots, f_r, g \in A^{\rhd}$ be elements 
such that $(f_1, \cdots, f_r, g)$ is an open ideal of $A^{\rhd}$ and 
$$U=R\left(\frac{f_1, \cdots, f_r}{g}\right).$$
If $\psi:A \to C$ is a continuous ring homomorphism 
to a Zariskian affinoid ring 
such that the image of the induced map $\Spa\,(\psi):\Spa\,C \to \Spa\,A$ 
is contained in $U$,  
then there exists a unique continuous ring homomorphism 
$$\psi':A\left(\frac{f_1, \cdots, f_r}{g}\right)^{\Zar} \to C$$ 
such that $\psi=\psi' \circ \varphi$.  
\end{lem}

\begin{proof}
Let $\psi:A \to C$ be a continuous ring homomorphism 
to a Zariskian affinoid ring $C=(C^{\rhd}, C^+)$ 
satisfying ${\rm Im}\,(\Spa\,(\psi)) \subset U$. 
We now show the following two assertions. 
\begin{enumerate}
\item[(i)] $\psi(g) \in (C^{\rhd})^{\times}$. 
\item[(ii)] $\psi(f_1)/\psi(g), \cdots, \psi(f_r)/\psi(g) \in C^+$. 
\end{enumerate}

First we prove (i). 
By $\Image(\Spa\,(\psi)) \subset U=R\left(\frac{f_1, \cdots, f_r}{g}\right)$, 
we have that $v(\psi(g)) \neq 0$ for any $v \in \Spa\,C$. 
It follows from Lemma~\ref{l-zar-nonempty}(2) that 
$\psi(g)$ is not contained in any maximal ideal of $C^{\rhd}$. 
Thus (i) holds. 

Second we show (ii). 
Again by $\Image(\Spa\,(\psi)) \subset U=R\left(\frac{f_1, \cdots, f_r}{g}\right)$, 
we have that $v(\psi(f_i)/\psi(g)) \leq 1$ 
for any $v \in \Spa\,C$ and any $i \in \{1, \cdots, r\}$. 
By \cite[Lemma 3.3(i)]{Hub93}, we get $\psi(f_i)/\psi(g) \in C^+$ 
for any $i \in \{1, \cdots, r\}$. 
Thus (ii) holds. 

\medskip

By (i) and (ii), 
there exists a unique ring homomorphism 
$\psi_1:A\left(\frac{f_1, \cdots, f_r}{g}\right) \to C$ 
inducing the following factorisation:  
$$\psi:A \xrightarrow{\alpha} A\left(\frac{f_1, \cdots, f_r}{g}\right) \xrightarrow{\psi_1} C.$$
It follows from \cite[(1.2)(ii)]{Hub94} that $\psi_1$ is continuous. 
Taking the Zariskisation of $\psi_1$, we get a unique factorisation of $\psi_1$: 
$$\psi:A \to A\left(\frac{f_1, \cdots, f_r}{g}\right) \xrightarrow{\theta} A\left(\frac{f_1, \cdots, f_r}{g}\right)^{\Zar} \xrightarrow{\psi'} C.$$
We are done. 
\end{proof}





\begin{lem}\label{l-rat-universal2}
Let $A$ be an affinoid ring and let $U$ be a rational subset of $\Spa\,A$. 
Then the following assertions hold. 
\begin{enumerate}
\item 
If $V$ is a rational subset of $\Spa\,A$ such that $V \subset U$, then 
there is a unique continuous ring homomorphism 
$Z_A(U) \to Z_A(V)$ that commutes with 
$\varphi_V:A \to Z_A(V)$ and $\varphi_U:A \to Z_A(U)$. 
\item 
The induced map $\Spa\,(\varphi_U):\Spa\,Z_A(U) \to \Spa\,A$ 
is an open injective map whose image is equal to $U$. 
If $W$ is a rational subset of $\Spa\,Z_A(U)$, then 
so is $\Spa\,(\varphi_U)(W)$. 
If $V$ is a rational subset of $\Spa\,A$ contained in $U$, then 
also $(\Spa\,(\varphi_U))^{-1}(V)$ is a rational subset. 
\item 
Set $B:=Z_A(U)$ and $g:=\Spa(\varphi_U):\Spa\,B \to \Spa\,A$. 
Let $V$ be a rational subset of $\Spa\,A$ contained in $U$. 
Then there exists a unique continuous ring homomorphism 
$r:Z_A(V) \to Z_B(g^{-1}(V))$ such that the following diagram is commutative: 
$$\begin{CD}
Z_B(g^{-1}(V)) @<<< Z_A(V)\\
@AAA @AAA\\
B @<<< A.
\end{CD}$$
Furthermore, $r$ is an isomorphism of topological rings. 
\end{enumerate}
\end{lem}

\begin{proof}
The assertion (1) holds by Lemma~\ref{l-rat-universal1}. 
The assertion (2) follows from \cite[Lemma 1.5(ii)]{Hub94} 
and Theorem \ref{t-comp-factor}. 
As for (3), 
we can apply the same argument as in \cite[Lemma 1.5(iii)]{Hub94} 
after replacing $F_A(-)$ and \cite[Lemma 1.3]{Hub94} by $Z_A(-)$ and 
Lemma~\ref{l-rat-universal1}, respectively. 
\end{proof}


\subsubsection{Zariskian structure presheaves}\label{sss-zar-str}

Let $A=(A^{\rhd}, A^+)$ be an affinoid ring. 
For any open subset $V$ of $\Spa\,A$, we set 
$$\Gamma(V, \MO_A^{\Zar}):=\MO_A^{\Zar}(V):=
\varprojlim_{\substack{U \subset V,\\ 
U:\text{ rational}\\ 
\text{subset}}} Z_A(U)^{\rhd},$$
where the inverse limit is taken in the category of $A^{\rhd}$-algebras. 
We equip $\MO_A^{\Zar}(V)$ with the inverse limit topology. 
If $V_1$ and $V_2$ are open subsets of $\Spa\,A$ satisfying 
$V_1 \supset V_2$, 
then the induced ring homomorphism  $\MO_A^{\Zar}(V_1) \to \MO_A^{\Zar}(V_2)$ is continuous. 
Thus $\MO^{\Zar}_A$ is a presheaf of topological rings. 

Fix $x \in \Spa\,A$. 
Set 
$$\MO^{\Zar}_{A, x}:=\varinjlim_{x \in U} \MO^{\Zar}_A(U),$$ 
where the direct limit is taken in the category of rings. 
We obtain a natural isomorphism of rings: 
$$\varinjlim_{\substack{x \in U,\\ 
U:\text{ rational}\\ 
\text{subset}}} \MO^{\Zar}_A(U) \xrightarrow{\simeq} 
\varinjlim_{x \in U} \MO^{\Zar}_A(U).$$
For any rational subset $U$ with $x \in U$, 
the valuation $x:A^{\rhd} \to \Gamma_x \cup \{0\}$ 
is uniquely extended to a valuation $v_U:\MO_A^{\Zar}(U) \to \Gamma_x \cup \{0\}$. 
Thus the set of valuations $\{v_U\}_{x \in U}$ define a valuation 
$$v_x:\MO_{A, x} \to \Gamma_x \cup \{0\}.$$
For any open subset $U$ of $\Spa\,A$, we set 
$$\MO_A^{\Zar, +}(U):=\{f \in \MO^{\Zar}_A(U)\,|\,v_x(f) \leq 1\text{ for any }x \in U\}.$$
Then $\MO_A^{\Zar, +}$ is a presheaf of rings on $\Spa\,A$. 
For any $x \in \Spa\,A$, 
we set $\MO_{A, x}^{\Zar, +}(U)$ to be the stalk of $\MO_A^{\Zar, +}$ at $x$. 


\begin{prop}\label{p-zar-LRS}
The following hold. 
\begin{enumerate}
\item 
For any $x \in \Spa\,A$, 
the stalk $\MO_{A, x}^{\Zar}$ is a local ring 
whose maximal ideal is equal to $\Ker(v_x)$. 
\item 
For any $x \in \Spa\,A$, 
the stalk $\MO_{A, x}^{\Zar, +}$ is a local ring. 
It holds that 
$\MO_{A, x}^{\Zar, +}=\{f \in \MO_{A, x}^{\Zar}\,|\, v_x(f) \leq 1\}$ and 
its maximal ideal is equal to $\{f \in \MO_{A, x}^{\Zar}\,|\, v_x(f) < 1\}$. 
\item 
For any open subset $U$ of $\Spa\,A$ and $f, g \in \MO_A^{\Zar}(U)$, 
the set $\{x \in U\,|\,v_x(f) \leq v_x(g) \neq 0\}$ is 
an open subset of $\Spa\,A$. 
\item 
 For any rational subset $U$ of $\Spa\,A$, 
 it holds that $\MO_A^{\Zar}(U)=Z_A(U)^{\rhd}$ and 
$\MO_A^{\Zar, +}(U)=Z_A(U)^+$. 
\end{enumerate}
\end{prop}

\begin{proof}
We can apply the same argument as in \cite[Proposition 1.6]{Hub94} 
after replacing \cite[Lemma 1.5]{Hub94} by Lemma~\ref{l-rat-universal2}. 
\end{proof}


Let $A=(A^{\rhd}, A^+)$ be an affinoid ring. 
Let $M$ be an $A^{\rhd}$-module. 
For any open subset $V$ of $\Spa\,A$, we set 
$$\Gamma(V, M \otimes \MO_A^{\Zar}):=
\varprojlim_{\substack{U \subset V,\\ 
U:\text{ rational}\\ 
\text{subset}}}M \otimes_A Z_A(U)^{\rhd},$$
where the inverse limit is taken in the category of $A^{\rhd}$-modules. 
We have that $M \otimes \MO_A^{\Zar}$ is a presheaf of $\MO_A$-modules. 



\subsection{Structure sheaves}\label{ss-structure}

The purpose of this subsection is to show that the structure presheaf $\MO_A^{\Zar}$ is actually a sheaf (Theorem~\ref{t-sheafy}). 
To this end, we start with an auxiliary result (Lemma \ref{l-rat-covering}) 
that assures the existence of refined rational coverings. 


\begin{lem}\label{l-rat-covering}
Let $A=(A^{\rhd}, A^+)$ be a Zariskian affinoid ring and 
let $\mathcal U$ be an open cover of $\Spa\,A$. 
Then there exist elements $f_0, \cdots, f_n \in A^{\rhd}$ 
such that $\sum_{i=0}^n A^{\rhd} f_i=A^{\rhd}$ and 
the induced open cover $\left\{R\left(\frac{f_0, \cdots, f_n}{f_i}\right)\right\}_{0 \leq i \leq n}$ of $\Spa\,A$ is a refinement of $\mathcal U$. 
\end{lem}




\begin{proof}
Let $\gamma:A^{\rhd} \to \widehat{A^{\rhd}}$ be the completion. 
Thanks to \cite[Lemma 2.6]{Hub94}, we can find a refinement 
$\mathcal U$ by a rational covering 
$\left\{R\left(\frac{f_0, \cdots, f_n}{f_i}\right)\right\}_{0 \leq i \leq n}$ 
for some elements $f_0, \cdots, f_n, a_0, \cdots, a_n \in \widehat{A^{\rhd}}$ 
such that $\sum_{i=0}^n a_if_i=1$. 
Fix a ring of definition $A_0$ of $A^{\rhd}$ and an ideal of definition $I_0$ of $A_0$. 
By \cite[Lemma 3.10]{Hub93}, 
we can find elements $f'_0, \cdots, f'_n \in A^{\rhd}$ 
whose images by $\gamma$ are sufficiently close to $f_0, \cdots, f_n$, 
so that 
$$-1+\sum_{i=0}^na_i\gamma(f'_i) \in I_0\widehat{A_0}$$ 
and 
$$R\left(\frac{f_0, \cdots, f_n}{f_i}\right)=
R\left(\frac{f'_0, \cdots, f'_n}{f'_i}\right)$$
for any $i \in \{0, 1, \cdots, n\}$. 
Take elements $a'_i \in A^{\rhd}$ whose images by $\gamma$ 
are sufficiently close to $a_i$, so that 
$$-1+\sum_{i=0}^n\gamma(a'_if'_i) \in I_0\widehat{A_0}.$$ 
Therefore, we get 
$$-1+\sum_{i=0}^na'_if'_i \in \gamma^{-1}(I_0\widehat{A_0})=I_0A_0.$$
Since $A^{\rhd}$ is Zariskian, we get $\sum_{i=0}^nA^{\rhd} f'_i=A^{\rhd}$ as desired. 
\end{proof}


\begin{thm}\label{t-sheafy}
Let $A=(A^{\rhd}, A^+)$ be an affinoid ring and let $M$ be an $A^{\rhd}$-module. 
Then the presheaf $M \otimes \MO_A^{\Zar}$ on $\Spa\,A$ is a sheaf. 
\end{thm}


\begin{proof}
Replacing $A$ by its Zariskisation, 
we may assume that $A$ is Zariskian. 
Set $X:=\Spa\,A$. 
Take elements $f_0, \cdots, f_n \in A^{\rhd}$ such that 
$\sum_{k=0}^n A^{\rhd} f_k=A^{\rhd}$. 
Set $U_k:=R\left(\frac{f_0, \cdots, f_n}{f_k}\right)$ and 
$U_{k_1k_2} :=U_{k_1} \cap U_{k_2}$ for any $0 \leq k, k_1, k_2 \leq n$. 
Thanks to Lemma~\ref{l-rat-covering}, 
it suffices to show that the sequence 
{\small
\begin{equation}\label{e-sheafy}
0 \to (M \otimes \MO_A^{\Zar})(X) \xrightarrow{\varphi} \prod_{0\leq k \leq n} (M \otimes \MO_A^{\Zar})(U_k) 
\xrightarrow{\psi} \prod_{0\leq k_1<k_2 \leq n} (M \otimes \MO_A^{\Zar})(U_{k_1k_2})
\end{equation}
}
is exact. 
Fix a ring of definition $A_0$ of $A^{\rhd}$ and 
an ideal of definition $I_0$ of $A_0$.
We have that 
$$(M \otimes \MO_A^{\Zar})(X)=M,$$
$$(M \otimes \MO_A^{\Zar})(U_k)
=M \otimes_{A^{\rhd}} \left(1+I_0A_0\middle[\frac{f_0}{f_k}, \cdots, \frac{f_n}{f_k}\middle]\right)^{-1}A^{\rhd}\left[\frac{1}{f_k}\right],$$
$$(M \otimes \MO_A^{\Zar})(U_{k_1k_2})=M \otimes_{A^{\rhd}}
\left(1+I_0A_0\middle[\middle\{\frac{f_{\ell_1}f_{\ell_2}}{f_{k_1}f_{k_2}}\middle\}_{0 \leq \ell_1, \ell_2 \leq n}\middle]\right)^{-1}A^{\rhd}\left[\frac{1}{f_{k_1}f_{k_2}}\right].$$
Let $J_0$ be the $A_0$-submodule of $A^{\rhd}$ defined by 
$$J_0:=\sum_{k=0}^n A_0f_k \subset A^{\rhd}.$$
For $J:=J_0A^{\rhd}=A^{\rhd}$, we get a commutative diagram of schemes: 
$$\begin{CD}
\Proj(\bigoplus_{d=0}^{\infty} J^d) @>\pi >> \Spec\,A^{\rhd}\\
@VV\beta V @VV\alpha V\\
\Proj(\bigoplus_{d=0}^{\infty} J_0^d) @>\pi_0>> \Spec\,A_0,
\end{CD}$$
where $J^0:=A^{\rhd}$ and $J_0^0:=A_0$. 
By $J=A^{\rhd}$, we have that $\pi$ is an isomorphism. 
We get a morphism 
$$\sigma:=\beta \circ \pi^{-1}:\Spec\,A^{\rhd} \to 
\Proj\left(\bigoplus_{d=0}^{\infty} J_0^d\right).$$
It holds that $\beta$ is an affine morphism satisfying $\beta^{-1}(D_+(f_k))=D_+(f_k)$ for any $k \in \{0, \cdots, n\}$. 
Since the equation $\pi^{-1}(D(f_k))=D_+(f_k)$ holds, 
we obtain $\sigma^{-1}(D_+(f_k))=D(f_k)$. 
Thus, for any non-empty subset $K$ of $\{0, \cdots, n\}$, 
if we set 
$$f_K:=\prod_{k \in K} f_i,$$ 
then it holds that  
\begin{equation}\label{e-sheafy2}
\Gamma(D_+\left(f_K\right), \sigma_*\widetilde M)=
\Gamma(D\left(f_K\right), \widetilde M)=M \otimes_{A^{\rhd}} A^{\rhd}\left[\frac{1}{f_K}\right].
\end{equation}



\begin{claim}
For any non-empty subset $K$ of $\{0, \cdots, n\}$, 
the equation
\begin{equation}\label{e-sheafy3}
\Gamma\left(D_+(f_K), 
\MO_{\Proj(\bigoplus_{d=0}^{\infty} J_0^d)}\right)
=A_0\left[\middle\{\frac{f_{i_1}\cdots f_{i_{|K|}}}{f_K}\middle\}_{0 \leq i_j \leq n}\right] 
\end{equation}
holds, where the right hand side is defined as 
the $A_0$-subalgebra of $A^{\rhd}[1/f_K]$ generated by 
the set 
$\{f_{i_1}\cdots f_{i_{|K|}}f_K^{-1}\,|\, 0 \leq i_j \leq n\}$. 
\end{claim}

\begin{proof}(of Claim) 
Consider the following $A_0$-algebra homomorphisms of 
graded $A_0$-algebras preserving degrees 
\begin{eqnarray*}
A_0[t_0, \cdots, t_n] &\xrightarrow{\rho}& \bigoplus_{d=0}^{\infty} J_0^d 
\hookrightarrow A^{\rhd}[s]\\
t_k &\mapsto & f_ks
\end{eqnarray*}
where $A_0[t_0, \cdots, t_n]$ is the polynomial ring over $A_0$ and 
$f_ks$ is the element of degree one 
which is the same element as $f_k$. We set 
$$t_K:=\prod_{k \in K} t_k.$$
Taking localisations, we obtain $A_0$-algebra homomorphisms of 
graded $A_0$-algebras preserving degrees: 
\begin{eqnarray*}
A_0[t_0, \cdots, t_n, t_K^{-1}] &\xrightarrow{\rho_K}& 
\left(\bigoplus_{d=0}^{\infty} J_0^d\right)\left[\frac{1}{f_Ks^{|K|}}\right] 
\overset{q}\hookrightarrow \left(A^{\rhd}\left[\frac{1}{f_K}\right]\right)[s, s^{-1}].
\end{eqnarray*}
Since $\rho$ is surjective, so is $\rho_K$. 
It follows from \cite[Ch. II, Proposition 2.5(b)]{Har77} that 
the left hand side 
$\Gamma\left(D_+(f_K), \MO_{\Proj(\bigoplus_{d=0}^{\infty} J_0^d)}\right)$ 
of (\ref{e-sheafy3}) is nothing but the degree zero part of 
the graded ring $(\bigoplus_{d=0}^{\infty} J_0^d)[1/f_Ks^{|K|}]$, 
which is equal to 
$\rho_K\left(A_0[\{\frac{t_{i_1} \cdots t_{i_{|K|}}}{t_K}\}_{0 \leq i_j \leq n}]\right)$ 
because $\rho_K$ is surjective and preserving degrees. 
Embedding this ring via $q$, we get 
\begin{eqnarray*}
\Gamma\left(D_+(f_K), \MO_{\Proj(\bigoplus_{d=0}^{\infty} J_0^d)}\right)
&=&\rho_K\left(A_0\middle[\middle\{\frac{t_{i_1} \cdots t_{i_{|K|}}}{t_K}\middle\}_{0 \leq i_j \leq n}\middle]\right)\\
 &\xrightarrow{q,\, \simeq}& A_0\left[\middle\{\frac{f_{i_1} \cdots f_{i_{|K|}}}{f_K}\middle\}_{0 \leq i_j \leq n}\right]
\end{eqnarray*}
where the right hand side is the $A_0$-subalgebra of $A^{\rhd}[1/f_K]$ 
generated by $\{f_{i_1} \cdots f_{i_{|K|}}f_K^{-1}\,|\,0 \leq i_j \leq n\}$. 
This completes the proof of Claim. 
\end{proof}





Let 
$\iota: W=\pi^{-1}_0(V(I_0)) \hookrightarrow \Proj(\bigoplus_{d=0}^{\infty} J_0^d)$ be the inclusion map and we equip $W$ the induced topology. 
We set $\mathcal D(g):=D(g) \cap W$ for any homogeneous element $g$ 
of positive degree. 
For $\mathcal M:=\iota^{-1}\sigma_*\widetilde{M},$ 
we have that 
\begin{eqnarray*}
&&\Gamma(\mathcal D_+(f_k), \mathcal M)\\
&=&M \otimes_{A^{\rhd}} A^{\rhd}\left[\frac{1}{f_k}\right] 
\otimes_{A_0\left[\frac{f_0}{f_k},\cdots, \frac{f_n}{f_k}\right]}
\left(A_0\left[\frac{f_0}{f_k},\cdots, \frac{f_n}{f_k}\right]\right)^{Z}\\
&=&M \otimes_{A^{\rhd}} \left(1+I_0A_0\left[\frac{f_0}{f_k},\cdots, \frac{f_n}{f_k}\right]\right)^{-1} A^{\rhd}\left[\frac{1}{f_k}\right]\\
&=& (M \otimes \MO_A^{\Zar})(U_k).
\end{eqnarray*}
where we set 
{\small 
$$\left(A_0\left[\frac{f_0}{f_k},\cdots, \frac{f_n}{f_k}\right]\right)^{Z}
:=\left(1+I_0A_0\left[\frac{f_0}{f_k},\cdots, \frac{f_n}{f_k}\right]\right)^{-1}
\left(A_0\left[\frac{f_0}{f_k},\cdots, \frac{f_n}{f_k}\right]\right)$$
}
and the first equation follows from 
(\ref{e-sheafy2}), (\ref{e-sheafy3}) and 
the equation (\ref{e-FK-sheafy}) in Subsection~\ref{ss-FK-global}. 
By the same argument, we get 
$$\Gamma( \mathcal D_+(f_{k_1}f_{k_2}), \mathcal M
)= (M \otimes \MO_A^{\Zar})(U_{k_1k_2}).$$ 
Therefore, the sequence (\ref{e-sheafy}) coincides 
with the following sequence 
$$0 \to \Gamma(W, \mathcal M) 
\to \prod_{0 \leq k\leq n}  \Gamma(\mathcal D_+(f_k), \mathcal M) 
\to \prod_{0\leq k_1<k_2\leq n}  \Gamma(\mathcal D_+(f_{k_1}f_{k_2}), \mathcal M).$$
This is an exact suquence, since $\mathcal M=\iota^{-1}\sigma_*\widetilde M$ is a sheaf. 
Thus also (\ref{e-sheafy}) is exact. 
\end{proof}



\subsection{Zariskian adic spaces}\label{ss-zar-ad-sp}

Let $\mathcal V$ be the category of 
the triples $(X, \MO_X, (v_x)_{x \in X})$, 
where $X$ is a topological space, $\MO_X$ is a sheaf of topological rings, 
and each $v_x$ is a valuation of the stalk $\MO_{X, x}$. 
An arrow $f:X \to Y$ is a morphism in $\mathcal V$ 
if $f$ is a continuous map such that 
$f^{\sharp}:\MO_Y \to f_*\MO_X$ 
is a continuous ring homomorphism and that 
the valuation $v_{f(x)}$ is equivalent to the composition of 
$v_x$ and $f^{\sharp}_x:\MO_{Y, f(x)} \to \MO_{X, x}$.  

\begin{dfn}\label{d-zar-adic-sp}
For an affinoid ring $A$, 
the object 
$$(\Spa\,A, \MO_A^{\Zar}, (v_x)_{x \in \Spa\,A})$$ 
of $\mathcal V$ is called the {\em Zariskian adic space associated with} $A$. 
An {\em affinoid Zariskian adic space} is an object of $\mathcal V$ 
which is isomorphic to the Zariskian adic space associated with an affinoid ring. 

A {\em Zariskian adic space} is an object $(X, \MO_X, (v_x)_{x \in X})$ 
such that any point $x \in X$ has an open neighbourhood $U$ of $X$ 
to which the restriction $(U, \MO_X|_U, (v_x)_{x \in U})$ is 
an affinoid Zariskian adic space. 
A {\em morphism} of adic spaces is a morphism in $\mathcal V$. 
\end{dfn}




\subsection{Examples violating the Tate acylicity}\label{ss-cex-Tate}



\begin{nota}\label{n-cex}
Fix a prime number $p$ such that $p \neq 2$. 
In the following, we equip $\Q$ and $\Z_{(p)}$ 
with the $p$-adic topologies. 
We equip $\Q[t]$ the group topology induced by $\{p^n\Z_{(p)}[t]\}_{n \in \Z_{>0}}$. 
Then the pair
$$A:=(\Q[t], \Z_{(p)}[t])$$
is an affinoid ring.  
Let 
$$U_1:=R\left(\frac{1}{t+1}\right), \quad U_2:=R\left(\frac{1}{t-1}\right)$$
be rational subsets of $\Spa\,A$. 
We get 
$$U_1 \cap U_2=R\left(\frac{1}{t^2-1}\right)$$
We consider the following map 
\begin{eqnarray*}
\rho: \MO_A^{\Zar}(U_1) \times \MO_A^{\Zar}(U_2) &\to & 
\MO_A^{\Zar}(U_1 \cap U_2)\\
(f, g)&\mapsto &f|_{U_1 \cap U_2}-g|_{U_1 \cap U_2}.
\end{eqnarray*}
\end{nota}

\begin{lem}\label{l-cex-cover}
We use Notation~\ref{n-cex}. 
Then 
$$\Spa\,A=R\left(\frac{1}{t+1}\right) 
\cup R\left(\frac{1}{t-1}\right).$$ 
\end{lem}

\begin{proof}
Take $v \in \Spa(\Q[t], \Z_{(p)}[t]) \setminus R(\frac{1}{t+1})$. 
Then we have that $v(t+1) < 1=v(2)$, which implies 
$$v(t-1)=v(t+1-2)=1.$$
Thus we get $v \in R(\frac{1}{t-1})$, as desired. 
\end{proof}


\begin{prop}\label{p-non-surje}
We use Notation~\ref{n-cex}. 
Then $\rho$ is not surjective.
\end{prop}


\begin{proof}
The map $\rho$ can be written as 
$$\rho:\Q\left[t, \frac{1}{t+1}\right]^{\Zar} \times \Q\left[t, \frac{1}{t-1}\right]^{\Zar} 
\to \Q\left[t, \frac{1}{t^2-1}\right]^{\Zar}.$$
We embed all the rings appearing above into $\Q(t)$. 
We have that 
$$\frac{1}{1+\frac{p}{t^2-1}} \in 
\left(1+p\Z_{(p)}\left[t, \frac{1}{t^2-1}\right]\right)^{-1}
\Q\left[t, \frac{1}{t^2-1}\right]=\Q\left[t, \frac{1}{t^2-1}\right]^{\Zar}.$$
Assume that this element is in the image of $\rho$. 
Let us derive a contradiction. 
We can write 
$$ \frac{1}{1+\frac{p}{t^2-1}}=\frac{g_1(t, \frac{1}{t+1})}{1+pf_1(t, \frac{1}{t+1})}+
\frac{g_2(t, \frac{1}{t-1})}{1+pf_2(t, \frac{1}{t-1})}$$
for some $f_i(X, Y) \in \Z_{(p)}[X, Y]$ and $g_i(X, Y) \in \Q[X, Y]$. 
Taking the multiplication with the product of the denominators, 
we get 
{\small 
\begin{equation}\label{e-cex1}
\left(1+pf_1\middle(t, \frac{1}{t+1}\middle)\middle)\middle(1+pf_2\middle(t, \frac{1}{t-1}\middle)\right)=\left(1+\frac{p}{t^2-1}\right)g\left(t, \frac{1}{t^2-1}\right)
\end{equation}
}
for some $g(t, \frac{1}{t^2-1}) \in \Q[t, \frac{1}{t^2-1}]$. 
We have that $g(t, \frac{1}{t^2-1}) \in \Z_{(p)}[t, \frac{1}{t^2-1}]$. 
Indeed, otherwise we can find a positive integer $\nu$ such that 
$p^{\nu}g(t, \frac{1}{t^2-1}) \in \Z_{(p)}[t, \frac{1}{t^2-1}]$ and its modulo $p$ reduction 
is not zero, which contradicts the fact that $\F_p[t, \frac{1}{t^2-1}]$ 
is an integral domain, where $\F_p:=\Z/p\Z$. 




\begin{claim}
$(1+\frac{p}{t^2-1})\Z_{(p)}[t, \frac{1}{t^2-1}]$ is a prime ideal of $\Z_{(p)}[t, \frac{1}{t^2-1}]$. 
\end{claim}

\begin{proof}(of Claim) 
We have that $t^2-1+p$ 
is an irreducible polynomial over $\Q$, which in turn implies that 
$$(t^2-1+p)\Z_{(p)}[t]$$
is a prime ideal of $\Z_{(p)}[t]$. 
In particular, we get $t^2-1 \not\in (t^2-1+p)\Z_{(p)}[t]$, 
since if $t^2-1 \in (t^2-1+p)\Z_{(p)}[t]$, then the residue ring 
$$\Z_{(p)}[t]/(t^2-1+p)\Z_{(p)}[t] \simeq \Z_{(p)}[t]/(t^2-1, p)\Z_{(p)}[t] \simeq \F_p[t]/(t^2-1)$$
is not an integral domain. 
For $S:=\{(t^2-1)^r\}_{r \geq 0}$, we have that 
{\small 
$$S^{-1}(\Z_{(p)}[t]/(t^2-1+p)\Z_{(p)}[t]) \simeq 
\Z_{(p)}\left[t, \frac{1}{t^2-1}\middle]\middle/(t^2-1+p)\Z_{(p)}\middle[t, \frac{1}{t^2-1}\right]$$
}
is an integral domain, as the image of $t^2-1$ in $\Z_{(p)}[t]/(t^2-1+p)\Z_{(p)}[t]$ is nonzero. 
Therefore, Claim holds. 
\end{proof}

Let us go back to the proof  of Proposition \ref{p-non-surje}. 
By Claim and (\ref{e-cex1}), one of 
$$1+pf_1\left(t, \frac{1}{t+1}\right)\quad \text{and}\quad 
1+pf_2\left(t, \frac{1}{t-1}\right)$$
is contained in $(1+\frac{p}{t^2-1})\Z_{(p)}[t, \frac{1}{t^2-1}]$. 
By symmetry, we may assume that the former case occurs. 
Then we can write 
\begin{equation}\label{e-cex2}
1+pf_1\left(t, \frac{1}{t+1}\right)=\left(1+\frac{p}{t^2-1}\right)
h\left(t, \frac{1}{t^2-1}\right)
\end{equation}
for some $h(X, Y) \in \Z_{(p)}[X, Y]$. 
For the additive $(t-1)$-adic valuation $v_{t-1}:\Q(t) \to \Z$ 
satisfying $v_{t-1}(t-1)=1$, 
it follows from (\ref{e-cex2}) that 
$v_{t-1}(h(t, \frac{1}{t^2-1}))\geq 1$, i.e. 
we can write 
\begin{equation}\label{e-cex3}
h\left(t, \frac{1}{t^2-1}\right)=(t-1)h_1\left(t, \frac{1}{t+1}\right)+...+(t-1)^kh_k\left(t, \frac{1}{t+1}\right)
\end{equation}
for some $h_1,\cdots, h_k \in \Z_{(p)}[t, \frac{1}{t+1}]$. 
Combining (\ref{e-cex2}) and (\ref{e-cex3}), we obtain 
\begin{equation}\label{e-cex4}
1+pf_1\left(t, \frac{1}{t+1}\right)=\left(t-1+\frac{p}{t+1}\right)
\widetilde{h}\left(t, \frac{1}{t+1}\right),
\end{equation}
for 
$$\widetilde{h}\left(t, \frac{1}{t+1}\right):=
h_1\left(t, \frac{1}{t+1}\right)+...+(t-1)^{k-1}h_k\left(t, \frac{1}{t+1}\right) 
\in \Z_{(p)}\left[t, \frac{1}{t+1}\right].$$
Substituting $t=1$ for (\ref{e-cex4}), 
we get an equation of rational numbers: 
\begin{equation}\label{e-cex5}
1+pf_1\left(1, \frac{1}{2}\right)=\frac{p}{2} \times 
\widetilde{h}\left(1, \frac{1}{2}\right).
\end{equation}
Since all of $\frac{1}{2}, f_1(1, \frac{1}{2})$ and 
$\widetilde{h}(1, \frac{1}{2})$ are contained in $\Z_{(p)}$, 
the $p$-adic valuations of the both hand sides of (\ref{e-cex5})
are different, which is absurd. 
This completes the proof of Proposition \ref{p-non-surje}.
\end{proof}
\begin{thm}\label{t-non-TA}
We use Notation~\ref{n-cex}. 
Then it holds that 
$$H^1(\Spa\,A, \MO_A^{\Zar}) \neq 0.$$ 
\end{thm}

\begin{proof}
Set $X:=\Spa\,A$. 
Let $j_1:U_1 \to X$, $j_2:U_2 \to X$ and $j_3:U_1 \cap U_2 \to X$ 
be the open immersions. 
Lemma~\ref{l-cex-cover} induces the following Mayer--Vietoris exact sequence: 
$$0 \to \MO^{\Zar}_A \to (j_1)_*(\MO^{\Zar}_A|_{U_1})
\times (j_2)_*(\MO^{\Zar}_A|_{U_2}) 
\to (j_3)_*(\MO^{\Zar}_A|_{U_1 \cap U_2}) \to 0.$$
It follows from Proposition~\ref{p-non-surje} that  
$H^1(X, \MO_A^{\Zar}) \neq 0$, as desired. 
\end{proof}






\section{Hadronic molecules in lattice QCD}
\label{sec:lattice}

Lattice QCD is, in principle, the tool to calculate the spectrum of QCD from
first principles. There has been a remarkable progress in the last years in this
field, see
e.g.~\cite{Durr:2008zz,Baron:2010bv,Edwards:2011jj,Liu:2012ze,Liu:2016kbb}.
Still, the extraction of the properties of resonances and, in particular, of
hadronic molecules, from finite volume {calculations}  poses severe
challenges.
When QCD is put on an Euclidean space-time lattice {with a finite
space-time volume, asymptotic states cannot be defined and right-hand cuts are replaced by
poles, thus preventing a direct calculation of scattering and resonance
properties.}
This obstacle was overcome by L\"uscher a long time ago. He derived a relation
between the energy shift in the finite volume and the scattering phase shift in
the continuum \cite{Luscher:1990ux,Luscher:1986pf}, see also
Refs.~\cite{Wiese:1988qy,DeGrand:1990ip}.
This approach has become  known and used as L\"uscher's method.
More precisely, in order to determine the mass and width from the measured
spectrum, one first extracts the scattering phase shift by using the L\"uscher
equation. In the next step, using some parameterization for the $K$-matrix
(e.g., the effective range expansion), a continuation into the complex energy
plane is performed. As noted in Sec.~\ref{sec:Sproperties}, resonances
correspond to poles of the scattering $T$-matrix on the second Riemann sheet,
and the real and imaginary parts of the pole position define the mass and the
width of a resonance. A nice example is given by the $\rho$-meson, that has been
considered using L\"uscher's method, e.g., in
Refs.~\cite{Feng:2010es,Lang:2011mn,Aoki:2011yj,Dudek:2012xn}.
In these papers it has already been shown that even for such realistic
calculations of a well isolated resonance,
the inclusion of {hadron-hadron type interpolating operators} is mandatory,
it is simply not sufficient to represent the decaying resonance by properly
chosen quark bilinears {(for mesons)}.
For the discussion of hadronic molecules (or most other hadron resonances), this
method needs to be extended in various directions, such as considering higher
partial waves and spins
\cite{Bernard:2008ax,Luu:2011ep,Konig:2011nz,Konig:2011ti,Briceno:2012yi,
Gockeler:2012yj}, moving frames
\cite{Rummukainen:1995vs,Bour:2011ef,Davoudi:2011md,Fu:2011xz,Leskovec:2012gb,
Gockeler:2012yj}, multi-channel scattering
\cite{Liu:2005kr,Lage:2009zv,Bernard:2010fp,Doring:2011ip,Li:2012bi,Guo:2012hv},
including the use of unitarized chiral perturbation theory (and related methods)
\cite{Doring:2011vk,MartinezTorres:2011pr,Doring:2011nd,Albaladejo:2012jr,
Wu:2014vma,Hu:2016shf}, and three-particle final states
\cite{Kreuzer:2010ti,Polejaeva:2012ut,Briceno:2012rv,Hansen:2014eka,
Meissner:2014dea,Hansen:2015zga,Hansen:2015zta,Hansen:2016ync,Hansen:2016fzj,Guo:2017ism}.


Here, we will not attempt to review the lattice QCD approach to the hadron
spectrum in any detail but just focus on the bits and pieces that are relevant
for the investigation of possible hadronic molecules.
In Sec.~\ref{sec:reso} we summarize the L\"uscher method and its extension to
the multi-channel space, followed by a discussion of the compositeness criterion
in a finite volume, see Sec.~\ref{sec:compo}.
In Sec.~\ref{sec:qmdep}, we discuss how the quark mass dependence of certain
observables can be used to differentiate hadronic molecules from more compact
multi-quark states and in Sec.~\ref{sec:latres}, we briefly summarize pertinent
lattice QCD calculations for the possible molecular states containing charm
quarks. A short final subsection is devoted to certain states made of light
quarks only.


\subsection{Resonances in a finite volume}
\label{sec:reso}

The essence of the L\"uscher approach can be understood in a simple
nonrelativistic model for the scattering of identical, spinless particles of
mass $m$ in 1+1 dimensions.
In the CM frame, the relative momentum is quantized according to $p =(2\pi/L)n$,
with $L$ the spatial lattice extension and $n$ an integer. In case of no
interactions between these particles, the energy of the two-particle system is
simply given by $E=2m+p^2/m$, which means that the free energy level-$n$ scales
as $n^2/L^2$ with the volume and thus levels with different $n$ do not
intersect. In the presence of interactions, this behaviour is modified. Let us
assume that this interaction leads to a narrow resonance at $\sqrt{s_R} = E_R -
i\Gamma_R/2$, that is $\Gamma_R \ll E_R$. In the infinite volume limit, this
interaction leads to a phase shift $\delta (p)$  in the asymptotic wave
function. Furthermore, in the presence of a resonance, the phase shift will
change by $\pi$ (known as Levinson's theorem~\cite{Levinson:1949zz}).
In a finite volume, this behavior translates into the boundary condition
\begin{equation}
p L + 2\delta(p) = 2\pi \, m~,~~~m\in {\mathbb Z}~ .
\end{equation}
This condition provides the link between the volume dependence of the
energy levels in the interacting system and the continuum phase shift.
If one follows an energy level inwards from the asymptotic region to
smaller lattice sizes, in the vicinity of a resonance, this boundary 
condition causes a visible distortion, the so-called {\em avoided level
crossing}, {\sl c.f.} Fig.~\ref{fig:avoided}. The plateau, where the energy of
the two-particle system is almost volume-independent, corresponds to the real
part of the pole $E_R$. The imaginary part of the pole is given by the slope
according to $\left.d \delta(p)/dE\right|_{E_R} = 2/\Gamma_R$. Clearly, this method can only
work when certain conditions are fulfilled. First, the method as described 
here is restricted to the elastic two-particle case. Second, one has to make
sure that the interaction range of the particles is much smaller than the
size of the box to make the notion of asymptotic states possible. Third,
to suppress polarization effects that arise from the interactions of the 
lightest particles in the theory with each other around the torus, one
has to choose $L$ such that $1/m \ll L$.

%-----------------------------------------------------------
\begin{figure}[t!]
\begin{center}
 \includegraphics[width=0.4\textwidth]{./figures/avoided_E.pdf}
\caption{Energy levels of an interacting two-particle system. In
case of a resonance in this system, the energy levels exhibit the
avoided level crossing (plateau) that allows to read off the 
resonance energy $E_R$ directly.} 
\label{fig:avoided}
\end{center}
\end{figure}
%-----------------------------------------------------------

We now consider the extension of the L\"uscher method to the multi-channel
case, as most hadronic molecules are located close to a two-particle threshold
or between two close-by thresholds. To achieve this extension, an 
appropriate tool is a particular version of an NREFT,
because up to the energies where multi-particle inelastic states become
significant, such a framework is completely equivalent to the relativistic
field theory, provided the couplings in the nonrelativistic framework
are determined from matching to the relativistic $S$-matrix elements, for
details and further references, see \cite{Bernard:2008ax,Colangelo:2006va,Gasser:2011ju}. For the one-channel case, it was already shown in 
Ref.~\cite{Beane:2003yx}
that using such an NREFT, one obtains at a very simple and transparent 
proof of L\"uscher's formula.

To keep the presentation simple, we first consider a two-channel LSE in NREFT in
the infinite volume.
Let us consider antikaon-nucleon scattering in the region of the $\Lambda(1405)$
resonance, $\bar KN\to\bar KN, \Sigma\pi$.
The channel number 1 refers to $\bar KN$ and 2 to $\Sigma\pi$ with total isospin
$I = 0$.
The resonance $\Lambda (1405)$ is located between two thresholds, on the second
Riemann sheet, close to the real axis.
The two  thresholds are given by  $s_t = (m_N + M_K)^2$ and
$s'_t=(m_\Sigma+M_\pi)^2$. We work in the isospin limit and neglect the fact
that there are really two poles --- see Refs.~\cite{Oller:2000fj,Jido:2003cb}
and Sec.~\ref{sec:1405th}.\footnote{Note that in this  two-channel formulation
one only has one pole corresponding to one $\Lambda (1405)$. To deal with the
two-pole scenario requires the inclusion of more channels and explicit isospin
breaking.}
For energies above the $\bar KN$  threshold, $s > (m_N + M_K)^2$, the
coupled-channel LSE for the $T$-matrix elements $T_{ij}(s)$ in
dimensionally regularized NREFT reads (we only consider $S$-waves here)
\begin{eqnarray}
\label{eq:LSinfini} 
T_{11}  &=& H_{11} + H_{11} \, iq_1 T_{11} + H_{12} \,
iq_2 T_{21}~,\nonumber\\
T_{21}  &=& H_{21} + H_{21} \, iq_1 T_{11} + H_{22} \, iq_2 T_{21} \,, 
\end{eqnarray}
with
$q_1 = \lambda^{1/2} (s,m_N^2,M_K^2)/(2\sqrt{s})$, $q_2 = \lambda^{1/2}
(s,m_\Sigma^2,M_\pi^2)/(2\sqrt{s})$ and $\lambda(x,y,z)=x^2+y^2+z^2-2xy-2yz-2zx$
is the K\"all\'en function.
Furthermore, the $H_{ij}(s)$ denote the driving potential in the corresponding
channel, i.e. the matrix element of the interaction Hamiltonian between the free
two-particle states.
Continuation of the CM momentum $q_1$ below threshold $(m_\Sigma + M_\pi)^2 <s<
(m_N + M_K)^2$ is obtained via \beq iq_1 \to -\kappa_1 =
-\frac{(-\lambda(s,M_K^2,m_N^2))^{1/2}}{2\sqrt{s}} \ .
\eeq The resonance corresponds to a pole on the second Riemann sheet in the
complex $s$-plane. Its position is given by the solution of the secular 
equation \beq\label{eq:secular} \Delta (s) = 1 + \kappa_1^R \, H_{11} -
\kappa_2^R \, H_{22} - \kappa_1^R \kappa_2^R \, \left( H_{11} H_{22} -
H_{12}^2\right) \eeq with $\kappa_1^R = -(-\lambda
(s_R,m_N^2,M_K^2))^{1/2}/(2\sqrt{s_R})$ and $\kappa_2^R = (-\lambda
(s_R,m_\Sigma^2,M_\pi^2)^{1/2})/(2\sqrt{s_R})$.
The energy and width of the resonance are then  given by $\sqrt{s_R} = E_R - i
\Gamma_R/2$.

Consider next the same problem in a finite volume. Only discrete
values of the three-momentum are allowed, given by $\bm{k} = 
(2\pi/L)\bm{n}$, with $\bm{n}$ a triplet of integer numbers. 
Thus, we replace the three-momentum integration in the loops by a discrete
sum (see  Ref.~\cite{Bernard:2008ax} for more details).
The rotational symmetry is broken to a cubic symmetry, so mixing of 
different partial waves occurs. Here,  however, we only consider $S$-waves, 
neglecting the small mixing to higher partial waves. 
If necessary, the mixing can be easily included at a later 
stage, see e.g.~\cite{Bernard:2008ax,Doring:2012eu}.
The finite-volume version of the LSE
Eq.~(\ref{eq:LSinfini}) then takes the form
\beqa
T_{11}  &=& H_{11} - \frac{2  Z_{00}(1;k_1^2)}{\sqrt{\pi}L}\,  H_{11}
T_{11} - \frac{2 Z_{00}(1;k_2^2)}{\sqrt{\pi}L}\,  H_{12} T_{21}~,\nonumber\\
T_{21}  &=& H_{21} - \frac{2 Z_{00}(1;k_1^2)}{\sqrt{\pi}L}\,  H_{21}
T_{11} - \frac{2  Z_{00}(1;k_2^2)}{\sqrt{\pi}L}\, H_{22} T_{21}~,\nonumber\\
\eeqa
with 
\beqa
k_{1/2}^2 &=& \left(\frac{L}{2\pi}\right)^2 \, 
\frac{\lambda(s,M_{K/\pi}^2,m_{N/\Sigma}^2)}{4s}~,\nonumber\\
Z_{00} (1;k^2) &=& \frac{1}{\sqrt{4\pi}} \,\lim_{r\to 1}\sum_{\bm{n} \in 
{\mathbb R}^3}
\frac{1}{({\bm{n}\,}^2 - k^2)^r}~.
\eeqa
Here, we have neglected the terms that vanish exponentially at large 
$L$.
The secular equation that determines the spectrum can be brought into the
form
\beqa\label{eq:pseudophase}
&&\qquad  1 - \frac{2}{\sqrt{\pi} L} \,  Z_{00}(1;k_2^2)\, F(s,L) = 0~, 
\nonumber\\
&& F(s,L) = \left[ H_{22} -   \frac{2 Z_{00}(1;k_1^2)}{\sqrt{\pi}L}\,   
(H_{11}H_{22} - H_{12}^2)\right] \nonumber\\ 
&& \qquad \qquad \times \left[1 - \frac{2 Z_{00}(1;k_1^2)}{\sqrt{\pi}L}\,  
H_{11}\right]^{-1}.
\eeqa
This can be rewritten as 
\beqa\label{eq:1channel}
\delta(s,L) &=& -\phi(k_2) + n\,\pi~, \quad n = 0,1,2, \ldots  \nonumber\\
\phi(k_2) &=& -\arctan \frac{\pi^{3/2} \, k_2}{ Z_{00} (1;k_2^2)}~,
\eeqa
with
\beq
\tan\delta(s,L) = q_2(s) \,  F(s, L)~.  
\eeq
$\delta (s,L)$ is called the {\em pseudophase}.
The dependence of the pseudophase on $s$ and $L$
is very different from that of the usual scattering 
phase.
Namely, the elastic phase extracted from the lattice data by using
L\"uscher's formula is independent of the volume modulo terms that 
exponentially vanish at a large $L$. Further, the energies where the
phase passes through $\pi/2$ lie close to the real resonance locations.
In contrast with this, the pseudophase contains the function
$Z_{00}(1;k_1^2)$, which does not vanish exponentially at
a large $L$ and a positive $k_1^2$.
Moreover, 
% it contains the tower of ``resonances'' 
the tangent of the pseudophase contains a tower of poles
at the energies given by the roots of the equation
$1-({2}/{\pi L})\,Z_{00}(1;k_1^2)\,H_{11}=0$. On the other hand,
in the infinite volume
this equation reduces to $1+\kappa_1^R\,H_{11}=0$, {\sl c.f.} with
Eq.~(\ref{eq:secular}), which has only one root below threshold very close to
the position of the $\Lambda(1405)$. Other roots in a finite volume 
% are not related to the dynamics of the system
% in the infinite volume and 
stem from oscillations of $Z_{00}(1,k_1^2)$ 
between $-\infty$ and $+\infty$ when the variable $k_1^2$ varies 
along the positive semi-axis. {Their locations depend on $H_{11}$ and thus
contain information of the infinite-volume interaction}.
This is an effect of discrete energy levels in the ``shielded'' channel, {which is the channel with the lower
threshold in the coupled-channel system}.
The pseudophase depends on the three real functions $H_{11},H_{12},H_{22}$. Based on 
synthetic data it was shown in 
Ref.~\cite{Lage:2009zv} that a measurement of the lowest two eigenvalues
at energies between 1.4 and 1.5~GeV allows one to reconstruct the
pseudophase and to extract in principle the pole position. It was further
pointed out in that work that two-particle
thresholds also lead to an avoided level crossing, so the extraction
of the resonance properties from the corresponding plateaus in the
energy dependence of certain levels is no longer possible.
In the case of real data, taking into account the
uncertainties of each measurements, one has to measure more levels
on a finer energy grid. To obtain a sufficient amount of data in a
given volume, twisting and asymmetric boxes can also be helpful.
First such {calculations} have become available
recently and will be discussed below.

An alternative formulation, that allows the use of  fully relativistic
two-particle propagators and can easily be matched to the representation
of a given scattering amplitude based on unitarized chiral perturbation
theory (UCHPT) was worked out in~\cite{Doring:2011vk}. 
The method is based on the observation that
in coupled-channel UCHPT, certain
resonances are dynamically generated, e.g. the light scalar mesons in the
coupled $\pi\pi/\bar KK$ system. The basic idea is to
consider this approach  in a finite volume to
produce the volume-dependent discrete energy spectrum.
Reversing the argument, one is then able to fit the parameters of the
chiral potential to the measured energy spectrum on the
lattice and, in the next step, determine the resonance locations
by solving the scattering equations in the infinite
volume. By construction, this method fulfills the constraints from
chiral symmetry such as the appearance of Adler zeros at certain
unphysical points.
%
For recent developments using a relativistic framework, we refer
to~\cite{Briceno:2015csa,Briceno:2015dca,Briceno:2015tza}.

\subsection{Quark mass dependence}
\label{sec:qmdep}


To reduce numerical noise as well as to speed up algorithms, 
lattice {calculations}
are often performed at unphysical values of the light quark masses. 
While this at first sight may appear as a disadvantage, it is indeed
a virtue as it enables a new handle on investigating the structure
of certain states. However, in the case of multiple coupled channels,
one also has to be aware that thresholds and poles can move very
strongly as a function of the quark masses. This intricate interplay
between $S$-wave thresholds and resonances needs to be accounted for 
when one tries to extract the resonance properties.

To address the first issue, we specifically consider the charm-strange
mesons $D_{s0}^*(2317)$ and $D_{s1}(2460)$. As shown in \cite{Cleven:2010aw},
in the molecular picture describing these as $DK$ and $D^*K$ bound states,
a particular pion and kaon mass dependence arises. Consider first the
dependence on the light quark masses, that can be mapped onto the
pion mass dependence utilizing the Gell-Mann--Oakes--Renner 
relation~\cite{GellMann:1968rz},
$M_{\pi^\pm}^2 = B(m_u+m_d)$, that naturally arises in QCD as the leading 
term in the chiral expansion of the Goldstone boson mass. Here, $B$ is
related to the vacuum expectation value of the quark condensate. In fact, this
relation is fulfilled to better than 94\% in QCD~\cite{Colangelo:2001sp}.
As shown in Ref.~\cite{Cleven:2010aw}, the pion mass dependence of such
a molecular state is much more pronounced than for a simple $c\bar s$ state.
Even more telling and unique is, however, the kaon mass dependence.
For that, consider the  $M_K$ dependence of the mass of a bound state
of a kaon and some other hadron. The mass of such a kaonic bound state
is given by
\begin{equation}
M = M_K+M_h-E_B,
\end{equation}
where $M_h$ is the mass of the other hadron, and $E_B$ denotes the binding
energy. Although both $M_h$ and $E_B$ have some kaon mass dependence, it is 
expected to be a lot weaker
than that of the kaon itself. Thus, the important implication of this simple 
formula is that the leading
kaon mass dependence of a kaon-hadron bound state is {\em linear, and the 
slope is unity}. The only
exception to this argument is if the other hadron
is also a kaon or anti-kaon. In this case, the
leading kaon mass dependence is still linear but with the 
slope being changed to two.
Hence, as for the $DK$ and $D^*K$ bound states, one expects that
their masses are linear in the kaon mass, and the slope is 
approximately one. This expectation is borne out by
the explicit calculations performed in~\cite{Cleven:2010aw}.
Early lattice QCD attempts to investigate this peculiar kaon mass dependence
have led to inconclusive results~\cite{mcneile}. Other papers that discuss 
methods 
to analyze the structure of states based on their quark mass dependence
or the behavior at large number of colors are 
e.g.~\cite{Hanhart:2008mx,Pelaez:2010fj,Bernard:2010fp,Albaladejo:2012te,
Guo:2011pa,Nebreda:2011cp,Pelaez:2006nj,Guo:2015dha}.

The second issue we want to address briefly is the intricate interplay
of $S$-wave thresholds and resonance pole positions with varying quark masses,
as detailed in Ref.~\cite{Doring:2013glu}. In that paper, pion-nucleon
scattering in the $J^P = 1/2^-$ sector in the finite volume and at
varying quark masses based on UCHPT was studied. In the infinite volume,
both the $N(1535)$ and the  $N(1650)$ are dynamically generated
from the coupled channel dynamics of the isospin $I=1/2$ and strangeness
$S=0$ $\pi N, \eta N, K\Lambda$  and $K\Sigma$ system. Having fixed the
corresponding LECs in the infinite volume, one can straightforwardly
calculate the spectrum in the finite volume provided one knows the
octet Goldstone boson masses, the masses of the ground-state octet 
baryons and the meson decay constants. Such sets of data at different
quark masses are given by ETMC and QCDSF. ETMC provides masses
and decay constants for $M_\pi=269$~MeV and the kaon mass 
close to its physical value~\cite{Alexandrou:2009qu,Ottnad:2012fv}.
Quite differently, the QCDSF Collaboration~\cite{Bietenholz:2011qq}
obtains the baryon and meson masses  from an alternative approach to 
tune the quark masses. Most importantly, while the lattice size and 
spacing are comparable to those of the ETMC, the strange quark mass 
differs significantly from the physical value. The latter results in a
different ordering of the masses of the ground-state octet mesons and, 
consequently, in a different ordering of meson-baryon thresholds.
For the ETMC parameters,  all thresholds are moved to higher energies. 
The cusp at the $\eta N$ threshold has become more pronounced, but
no clear resonance shapes are visible. Indeed, going to the complex
plane, one finds that the poles are hidden {in the Riemann sheets which
are not directly connected to the physical one by crossing the cut at the
energies corresponding to the real parts of the poles}.
Using the QCDSF parameters, the situation is very different. In contrast to the ETMC case, a 
clear resonance signal is visible below the $K\Lambda$ threshold, 
that is the first inelastic channel in this parameter setup. Indeed, one
finds a pole  on the corresponding Riemann sheet. Unlike in the 
ETMC case, it is not hidden behind a threshold. Between
the $K\Lambda$ and the $K\Sigma$ threshold, there is only a hidden pole. 
The $K\Sigma$ and $\eta N$ thresholds are almost
degenerate, and on sheets corresponding to these higher-lying 
thresholds one only finds hidden poles. For more details, the reader
is referred to Ref.~\cite{Doring:2013glu}. In that paper, strategies
to overcome such type of difficulties are also discussed.

{It is worthwhile to mention that the composition of a hadron in general
may vary when changing the quark masses. However, as long as the quark masses are
not very different from the physical values, the quark mass dependence is rather
suggestive towards revealing the internal structure as different structures
should result in different quark mass dependence. }

\subsection{Measuring compositeness on lattice}
\label{sec:compo}

As discussed in Sec.~\ref{sec:weinberg}, the Weinberg compositeness criterion
offers a possibility to disentangle compact bound states from loosely bound
hadronic molecules. {By measuring the low-energy scattering observables in
lattice using the L\"uscher formalism discussed before, one can extract the
compositeness by using Eqs.~\eqref{eq:arwein}. For related work, see, e.g.,
Refs.~\cite{Suganuma:2007uv,MartinezTorres:2011pr,Ozaki:2012ce,
Albaladejo:2013aka}. It is pointed out in~\cite{Agadjanov:2014ana} that the use
of partially twisted boundary conditions is cheaper than studying the volume dependence in lattice
for measuring the compositeness.} 
% The finite volume formulation of this approach
% was given in Ref.~\cite{Agadjanov:2014ana}. 
The basic object in that work is the
scattering amplitude in the finite volume, which can be obtained from the
corresponding loop function $\tilde G_L^{\bm{\theta}}(s) = G(s)+\Delta
G_L^{\bm{\theta}}(s)$~\cite{Doring:2011vk}, where $\Delta G_L^{\bm{\theta}}$ can
be related to the modified L\"uscher function $Z_{00}^{\bm{\theta}}$ via
\begin{equation}
  \label{deltaG-luscher1}
  \Delta G_L^{\bm{\theta}}(s)=\frac1{8\pi\sqrt{s}}\left(
    ik - \frac{2}{\sqrt{\pi}L} Z_{00}^{\bm{\theta}}(1,\hat k^2)
  \right)+\cdots,
\end{equation}
where $\hat k = kL/(2\pi)$ and ellipsis denote terms that are exponentially
suppressed with the lattice size $L$~\cite{Doring:2011vk}. Here, in case 
of twisted boundary conditions, the momenta also depend on the twist angle
$\bm{\theta}$ according to 
$\bm{q}_n=(2\pi/L)\bm{n}+(\bm{\theta}/L),~0\leq\theta_i<2\pi$. 
In case of a bound state with mass $M$ in the infinite volume, the 
scattering amplitude should have a pole at $s=M^2$, with the corresponding 
binding momentum $k_B\equiv i\kappa$, $\kappa>0$, given by the equation
\begin{equation}
  \label{inf-vol-pole-eq}
  \psi(k_B^2)+\kappa = -8\pi M\Big[V^{-1}(M^2) - G(M^2)\Big] = 0\,,
\end{equation}
with $\psi(k^2)$ the analytic continuation of $k\cot\delta(k)$ for
arbitrary complex values of $k^2$. From this, it is straightforward
to evaluate the pole position shift,
\begin{eqnarray}
  \label{mass-shift}
\kappa_L - \kappa &=& \frac1{1-2\kappa\psi'(k_B^2)} \left[ -8\pi M_L \Delta
G_L^{\bm{\theta}} (M_L^2)\right. \nonumber\\
&& \qquad\qquad\qquad ~~~ + \left. \psi'(k_B^2)(\kappa_L-\kappa)^2 \right],
\end{eqnarray}
where the prime denotes differentiation with respect to $k^2$. 
This equation gives the bound state pole position $\kappa_L$ (and thus
the finite volume mass $M_L$) as a function of the infinite-volume 
parameters $g^2$ and $\kappa$. Having determined these parameters
from the bound state levels $\kappa_L$, one is then able to determine
the wave function renormalization constant $Z$ in the infinite volume.
In Ref.~\cite{Agadjanov:2014ana}, this procedure is scrutinized using
synthetic lattice data, for a simple toy model and a molecular model
for the charm scalar meson  $D_{s0}^*(2317)$. An important finding of this
work is that the extraction of $Z$ is facilitated by using twisted
boundary conditions, measuring the  dependence of the spectrum on the 
twist angle. Also, the limitations of this approach are discussed in 
detail. It remains to be seen how useful this method is for real 
lattice data. For related papers, also making use of twisted boundary
conditions to explore the nature of states, see e.g.
Refs.~\cite{Ozaki:2012ce,Briceno:2013hya,Korber:2015rce}.
A different approach to quantify compositeness in a finite volume has
recently been given in Ref.~\cite{Tsuchida:2017gpb}. Using this 
method, the $\bar KN$ component of the $\Lambda(1405)$
was found to be 58\%,  and the $\Sigma\pi$ and other components
also contribute to its structure. This is interpreted as a reflection
of the two-pole scenario of the  $\Lambda(1405)$.

\subsection{Lattice QCD results on the charm-strange mesons
and \texorpdfstring{$\bm{XYZ}$~}~states}
\label{sec:latres}

There have been quite a few studies of the charm-strange mesons and
some of the $XYZ$ states in lattice QCD. However, there are very few 
conclusive results at present, so we expect that this section will be 
outdated most quickly.

Let us consider first the charm-strange mesons. A pioneering lattice study of
the low-energy interaction between a light pseudoscalar meson and a charmed
pseudoscalar meson was presented in Ref.~\cite{Liu:2012zya}.
The scattering lengths of the five channels $D\bar K(I=0)$, $D\bar K(I=1)$, $D_s
K$, $D\pi(I=3/2)$ and $D_s\pi$ were calculated based on four ensembles with pion
masses of 301, 364, 511 and 617~MeV. These channels are free of contributions
from disconnected diagrams.
SU(3) UCHPT as developed in Ref.~\cite{Guo:2009ct} was used to perform the
chiral extrapolation. The LECs of the chiral Lagrangian were determined from a
fit to the lattice results. With the same set of LECs and the masses of the
involved mesons set to their physical values,  predictions for the other
channels including $DK(I=0)$, $DK(I=1)$, $D\pi(I=1/2)$ and $D_s\bar K$ were
made. In particular, it was found that the attractive interaction in the
$DK(I=0)$ channel is strong enough so that a pole is generated in the unitarized
scattering amplitude.  Within $1\sigma$ uncertainties of the parameters, the
pole is at $2315^{+18}_{-28}$~MeV, and it is always below the $DK$ threshold.
From calculating the wave function normalization constant, it was found that
this pole is mainly an $S$-wave $DK$ bound state (the pertinent scattering
length being close to $-1$~fm as predicted in~\cite{Guo:2009ct} for such a
molecular state using Eq.~\eqref{eq:arwein}).
Further, a much sharper prediction of the isospin breaking  decay width of the
$D_{s0}^*(2317)\to D_s\pi$ could  be given
\begin{equation}
\label{Eq:DecayWidth}
 \Gamma(D_{s0}^*(2317)\to D_s \pi) = (133\pm22)~{\rm keV}~,
\end{equation}
to be contrasted with the molecular prediction without lattice
data, $\Gamma(D_{s0}^*(2317)\to D_s \pi)=(180\pm110)$~keV~\cite{Guo:2008gp},
and typical quark model predictions for a $c\bar s$ charm scalar meson of the 
order of 10~keV, see 
e.g. Refs.~\cite{Godfrey:2003kg,Faessler:2007gv}. For a similar study 
using a covariant UCHPT instead
of the heavy-baryon formalism, see Ref.~\cite{Altenbuchinger:2013vwa}.

A systematic study of the charm scalar and axial mesons at lighter
pion masses ($M_\pi = 156$ and $266$~MeV) was performed in 
Refs.~\cite{Mohler:2012na,Mohler:2013rwa,Lang:2014yfa}. These data
were later reanalyzed with the help of finite-volume 
UCHPT~\cite{Torres:2014vna}. Most notably, the $DK$ scattering with 
$J^P=0^+$ was investigated in \cite{Mohler:2013rwa}, using $DK$ as well
as $c\bar s$ interpolating fields. Clear evidence of a bound state below
the $DK$ threshold was found and the corresponding scattering length
was $a_0 = -1.33(20)\,$fm, consistent with the molecular scenario.
The analysis of Ref.~\cite{Torres:2014vna} found a 70\% $DK$ ($D^*K$) 
component in the $D_{s0}^*(2317)\,(D_{s1}(2460))$ state.

The most systematic study in the coupled-channel $D\pi, D\eta$ and $D_s\bar K$
system with isospin $1/2$ and $3/2$ was reported in Ref.~\cite{Moir:2016srx}.
Using a large basis of quark-antiquark and meson-meson basis states, the
finite volume energy spectrum could be calculated to high precision, allowing
for the extraction of the scattering amplitudes in the $S$-, $P$- and $D$-waves.
With the help of the coupled-channel L\"uscher formalism and various
parameterizations of the $T$-matrix, three poles were found in the complex 
plane: a 
$J^P=0^+$ near-threshold bound state, $M_S=(2275.9\pm 0.9)\,$MeV,
with a large coupling to $D\pi$,
a deeply bound $J^P=1^-$ state,   $M_P=(2009\pm 2)\,$MeV,
and evidence for a $J^P=2^+$ narrow 
resonance coupled predominantly to$D\pi$,  $M_D=(2527\pm 3)\,$MeV.
An interesting observation was made in Ref.~\cite{Albaladejo:2016lbb}.
Using UCHPT, it was shown that there are in fact two 
($I=1/2, J^P=0^+$) poles  in the region of the $D_0^*(2400)$ in 
the coupled-channel $D\pi, D\eta, 
D_s\bar K$ scattering amplitudes. {They couple differently to the involved
channels and thus should be understood as two states.} Having all the parameters
fixed from earlier studies in Ref.~\cite{Liu:2012zya}, the energy levels for the coupled-channel system in a 
finite volume were predicted. These agree remarkably well with the
lattice QCD results in~\cite{Moir:2016srx}. The intricate interplay of 
close-by thresholds and resonance poles already pointed out 
in~\cite{Doring:2013glu} is also found, and it is stressed that more
high-statistics data are needed to better determine the higher mass pole.

We now turn to the $XYZ$ states. Consider first the $X(3872)$. There
have been a number of studies using diquark-diquark or tetraquark
interpolating fields over the years, but none of these has been 
conclusive, see e.g.~\cite{Chiu:2006hd,Yang:2012mya}. 
Evidence for a bound state with $J^{PC}=1^{++}$
$(11\pm7)$~MeV below the $D\bar D^*$ threshold was reported in 
Ref.~\cite{Prelovsek:2013cra}. 
This establishes a candidate for the  $X(3872)$ in addition to the
near-by scattering states  $D\bar D^*$ and $J/\psi \rho$. This computation
was performed at $M_\pi =266\,$MeV but in a small volume $L\simeq 2\,$fm. 
This finding was validated using the Highly Improved Staggered Quark 
action~\cite{Lee:2014uta}. Finally, a refined study allowing also
for the mixing of tetraquark interpolators with $\bar c c$ components
was presented in~\cite{Padmanath:2015era}.
A candidate for the $X(3872)$ with $I = 0$ is observed very 
close to the  experimental state only if both $\bar cc$ and $D\bar D^*$ 
interpolators are included. However, the candidate is 
not found if diquark-antidiquark and $D\bar D^*$ are used in the 
absence of $\bar c c$. Note that in 
Refs.~\cite{Jansen:2013cba,Garzon:2013uwa,Jansen:2015lha,Baru:2015tfa}
strategies for extracting the properties of the $X(3872)$ from 
finite-volume data (at unphysical quark masses) have been worked out.

Consider next the $Z_c(3900)$. Various lattice calculations have
been performed, which, however, did not lead to  conclusive results,
see e.g. 
Refs.~\cite{Prelovsek:2013xba,Prelovsek:2014swa,Chen:2014afa,
Ikeda:2016zwx}. 
For example, in the most recent work \cite{Ikeda:2016zwx} it
was argued that this state is most probably a threshold cusp. Also,
a systematic analysis of most of these data using a finite volume version of 
the framework in Ref.~\cite{Albaladejo:2015lob} did not allow
for a definite conclusion
on the nature of the $Z_c(3900)$~\cite{Albaladejo:2016jsg}.

The Chinese Lattice QCD Collaboration has also studied $D^* \bar D_1$ 
\cite{Meng:2009qt,Chen:2016lkl} and $D^*\bar D$ scattering 
\cite{Chen:2015jwa}
with the aim of investigating the structure
of the $Z_c(4430)$ and $Z_c(4025)$, respectively.  These studies were
mostly exploratory and no definite statements could be drawn.

\subsection{Lattice QCD results on hadrons built from light quarks}
\label{sec:latres2}

Here we summarize briefly some very recent results on hadrons made 
entirely of light $u,d,s$ quarks, more precisely, the scalar mesons 
$f_0(500)$  and $a_0(980)$ as well as $\Lambda(1405)$.

The  first determination of the energy dependence of the isoscalar $\pi\pi$
elastic scattering phase shift and the extraction of the $f_0(500)$ based on
dynamical QCD using the methods described above was given by the Hadron Spectrum
Collaboration in Ref.~ \cite{Briceno:2016mjc}.
From the volume dependence of the spectrum the $S$-wave phase shift up to the
$K\bar K$ threshold could be extracted. The calculations were performed at pion
masses of 236 and 391~MeV. The resulting amplitudes are described in terms of a
scalar meson which evolves from a bound state below the $\pi\pi$ threshold at
the heavier quark mass to a broad resonance at the lighter quark mass.
{This is } in line with the prediction of Ref.~\cite{Hanhart:2008mx} based
on UCHPT to one loop.
 Earlier, the same collaboration had analyzed the coupled channel
$\pi\eta, K\bar K, \pi\eta'$ system with isospin $I=1$ and extracted properties
of the $a_0(980)$ meson~\cite{Dudek:2016cru}. {The model-independent
lattice data on energy levels were reanalyzed using UCHPT in
Refs.~\cite{Guo:2016zep,Doring:2016bdr}.} In particular, Ref.~\cite{Guo:2016zep}
pointed out some ambiguities in the $I=1$ solution.


There have been quite a few studies of the $\Lambda(1405)$ as a simple
three-quark baryon state by various lattice collaborations. In view of the
intricacies of the coupled channel $K^- p$ scattering discussed earlier,
we will not further consider these as coupled-channel effects must be 
considered.
An exception is the analysis of Ref.~\cite{Hall:2014uca}  based on 
the PCAS-CS ensembles~\cite{Aoki:2008sm} 
with  three-quark sources allowing for scalar and vector diquark configurations
that leads to the vanishing of the strange magnetic form factor of the 
$\Lambda(1405)$
at the physical pion mass. It is argued that this can only happen if the
$\Lambda(1405)$ is mostly an antikaon-nucleon molecule. This is further 
validated
by applying a finite-volume Hamiltonian approach to the measured energy 
levels~\cite{Wu:2014vma}. This lattice QCD result appears to be at odds with the
accepted two-pole scenario. However, as pointed out in the UCHPT analysis of 
Ref.~\cite{Molina:2015uqp}, these results exhibit some shortcomings. It is
argued in that work, that what is really discussed in \cite{Hall:2014uca} is 
the heavier of the two poles. In particular the complete absence of the 
$\pi \Sigma$ threshold in these data is discussed, as this threshold  would couple
stronger to the lighter pole. This effect is presumably due to the neglect of 
the baryon-meson interpolating fields in Ref.~\cite{Hall:2014uca}. The required
operators are also specified in~\cite{Molina:2015uqp}. It will be interesting to see
lattice QCD {calculations} including all the relevant channels and required
interpolating fields. {We also point out that
better methods to calculate the matrix elements of unstable states has been
given in~\cite{Bernard:2012bi,Briceno:2015tza}.}

%\documentclass[preprint,12pt]{elsarticle}
%\if0
\usepackage{amssymb}
\usepackage{mathtools}
%\usepackage[dvipdfmx]{graphicx}
\usepackage{cite}
\usepackage{graphicx}
\usepackage{bm}
\usepackage{here}
\usepackage[subrefformat=parens]{subcaption}
\fi
%\usepackage{amssymb}
\usepackage{amsmath}
\usepackage[dvipdfmx]{}
\usepackage[dvipdfmx]{color}
%\usepackage{cite}
%\usepackage{upgreek}
\usepackage{url}
%\usepackage[dvipdfmx]{hyperref}
%\usepackage{pxjahyper}
%\usepackage {hyperref}
\usepackage{graphicx}
\usepackage{bm}
\usepackage{here}
\usepackage{caption}
\usepackage[subrefformat=parens]{subcaption}
\captionsetup{compatibility=false}

%% The amsthm package provides extended theorem environments
%% \usepackage{amsthm}

%% The lineno packages adds line numbers. Start line numbering with
%% \begin{linenumbers}, end it with \end{linenumbers}. Or switch it on
%% for the whole article with \linenumbers after \end{frontmatter}.
%% \usepackage{lineno}

%% natbib.sty is loaded by default. However, natbib options can be
%% provided with \biboptions{...} command. Following options are
%% valid:

%%   round  -  round parentheses are used (default)
%%   square -  square brackets are used   [option]
%%   curly  -  curly braces are used      {option}
%%   angle  -  angle brackets are used    <option>
%%   semicolon  -  multiple citations separated by semi-colon
%%   colon  - same as semicolon, an earlier confusion
%%   comma  -  separated by comma
%%   numbers-  selects numerical citations
%%   super  -  numerical citations as superscripts
%%   sort   -  sorts multiple citations according to order in ref. list
%%   sort&compress   -  like sort, but also compresses numerical citations
%%   compress - compresses without sorting
%%
%% \biboptions{comma,round}

% \biboptions{}

%% This list environment is used for the references in the
%% Program Summary
%%
\newcounter{bla}
\newenvironment{refnummer}{%
\list{[\arabic{bla}]}%
{\usecounter{bla}%
 \setlength{\itemindent}{0pt}%
 \setlength{\topsep}{0pt}%
 \setlength{\itemsep}{0pt}%
 \setlength{\labelsep}{2pt}%
 \setlength{\listparindent}{0pt}%
 \settowidth{\labelwidth}{[9]}%
 \setlength{\leftmargin}{\labelwidth}%
 \addtolength{\leftmargin}{\labelsep}%
 \setlength{\rightmargin}{0pt}}}
 {\endlist}
\begin{document}

\section{Testing the program O-SUKI-N 3D}
The several tests are shown below to present the target fuel implosion dynamics. In the example cases, the HIBs and the target fuel have the following common parameters, which are the same values employed in Ref. \cite{CPC-O-SUKI}: the beam radius at the entrance of a reactor chamber $R_{en}$ = 35 mm, the beam particle density distribution is in the Gaussian profile and all projectile Pb ions have 8 GeV. The target is a multilayered pellet, in which the pellet outer radius is 4 mm, a Pb layer thickness is 0.029 mm, the Al thickness is 0.460 mm, and the DT thickness is 0.083 mm; the Pb, Al and DT layers have the radial mesh numbers of 4, 46 and 30 in these example cases, respectively, and the total mesh number in the theta direction is 90. The input beam pulse is shown in Fig. 12 in Ref. \cite{CPC-O-SUKI}. The beam radius is 3.8mm on the target surface. However, $R_b$ = 3.8mm changes at $\tau_{wb}$ to 3.7mm for the wobbling beam irradiation. Here $\tau_{wb}$ is the rotational period of the beam axis. The rotational frequency is 424MHz ($rotaionnumber$ = 11). 



%% INPUT PULSE  
%\begin{figure}[H]
%		\centering
%		\includegraphics[width=10cm]{images/pulse.eps}
%		\caption{An example for the input beam pulse.}\label{pulse}
%\end{figure}



First the 3D Langrange code was run without the OK3 illumination code. This is the case for $OK\_Switch=10$, and we added the artificial non-uniformity $Y_3^2$ (the spherical harmonics) with the amplitude of $30.0\%$. In Fig. \ref{NoOK3_23_Ti} the ion temperature distribution is shown at $t$=35ns, and in Fig. \ref{NoOK3_23_rho} the mass density distribution is presented at $t$=35ns. The target shape is largely distorted due to the non-uniformity of the HIBs deposition energy distribution.  


%% LAGRANGE CASE WITHOUT OK3
\begin{figure}[H]
		\centering
		\includegraphics[width=10cm]{images/NoOK3_Non23_30_35ns_Ti.eps}
		\caption{Ion temperature in the 3D Lagrange code without OK3 code at $t$=35ns. The non-uniformity distiribution is $Y_3^2$ with the amplitude of $30\%$.}\label{NoOK3_23_Ti}
\end{figure}
\begin{figure}[H]
		\centering
		\includegraphics[width=10cm]{images/NoOK3_Non23_30_35ns_rho.eps}
		\caption{Mass density in the 3D Lagrange code without OK3 code at $t$=35ns. The non-uniformity distriution is $Y_3^2$ with the amplitude of $30\%$.}\label{NoOK3_23_rho}
\end{figure}


We also performed run-through simulation tests. In the example cases, the OK3 code was coupled with the run-through simulations. The implosion data were obtained by the Lagrange code, and the data just before the void closure time were transferred to the Euler code through the data Conversion code. Two cases are computed for the target fuel implosion dynamics with the spiral wobbling or without the oscillating HIBs. These examples are the run-through simulations with the OK3 illumination code ($OK\_Switch = 1$). The input beam pulse, employed in the run-through tests, is shown in Fig. \ref {Beam}. This beam input energy is 5.4MJ. We show the $r-t$ diagram for the case without the HIBs wobbling in Fig. \ref{rt}. The Lagrange-code test results stored in the output directory are visualized in Figs. \ref {fusion_Ti} for the target ion temperature ($T_i$) distributions at $t$ = 0.0, 15.0, 20.0 and 22.5 ns for the case with the HIBs wobbling behavior.  The RMS non-uniformity results are shown in Figs. \ref{fusion_RMS} (a) for DT layer's Ion temperature($T_i$), (b) for DT layer's Mass density($\rho$), (c) for Al layer's Ion temperature($T_i$) and (d) for Al layer's Mass density($\rho$). 
%
When the HIBs have the wobbling motion during the implosion with the wobbling frequency of 424MHz, the radius acceleration distributions are shown in Figs. \ref{Vr_tp} (a) in the $\theta$ direction and (b) in the $\phi$ direction at $t=6.25t_w=11.2ns$ (solid lines) and at $t=6.75t_w=12.2ns$ (dotted lines). Here $t_w$ shows the one rotation time. Figures \ref{Vr_tp} present that the non-uniformity phase of the implosion acceleration is controlled externally by the HIBs wobbling behavior \cite{CPC-O-SUKI, RSato2}.  
%

\begin{figure}[H]
		\centering
		\includegraphics[width=7.5cm]{images/Beam.eps}
		\caption{Input beam pulse shape used in the example run-through tests.}\label{Beam}
\end{figure}
\begin{figure}[H]
		\centering
		\includegraphics[width=8cm]{images/YesWob_SLC.eps}
		\caption{The $r-t$ diagram for the implosion with the HIBs wobbling illumination. The black line area shows the Pb layer, the blue line area Al and the red line area is DT.}\label{rt}
\end{figure}
\begin{figure}[H]
		\centering
		\includegraphics[width=6.5cm]{images/YesWob_Ti_0ns.eps}
		\includegraphics[width=6.5cm]{images/YesWob_Ti_15ns.eps}\\
		\includegraphics[width=6.5cm]{images/YesWob_Ti_20ns.eps}
		\includegraphics[width=6.5cm]{images/YesWob_Ti_225ns.eps}\\
		\caption{Ion temperature distributions in the example run-through test with the HIBs wobbling illumination at (a) $t$=0.0ns, (b) 15.0ns, (c) 20.0ns and (d) 22.5ns.}\label{fusion_Ti}
\end{figure}
\begin{figure}[H]
		\centering
		\includegraphics[width=6.5cm]{images/FusionRMS_DTTi.eps}
		\includegraphics[width=6.5cm]{images/FusionRMS_DTrho.eps}\\
		\includegraphics[width=6.5cm]{images/FusionRMS_AlTi.eps}
		\includegraphics[width=6.5cm]{images/FusionRMS_Alrho.eps}\\
		\caption{RMS non-uniformity histories of (a) the DT ion temperature, (b) the DT mass density, (c) the Al ion temperature and (d) the Al mass density for the cases with the wobbling HIBs (solid lines) and without the wobbling HIBs (dotted lines).}\label{fusion_RMS}
\end{figure}
%
\begin{figure}[H]
		\centering
		\includegraphics[width=6.5cm]{images/theta-Vr.eps}
		\includegraphics[width=6.5cm]{images/phi-Vr.eps}\\
		\caption{Radial acceleration distributions in (a) $\theta$ and (b) $\phi$. The solid lines show the acceleration ditributions at $t=6.25t_w=11.3ns$, and the dotted lines at $t=6.75t_w=12.2ns$.}\label{Vr_tp}
\end{figure}
%

After the Lagrange code computation, the implosion data are converted and transferred to the Euler code. Figures \ref{Ti_EuWobblIgnited} show the ion temperature distributions by the Euler code. Figures \ref{Ti_EuWobblIgnited} show that the DT fuel is ignited and the gain obtained is about 17.5 in this example case. For a realistic HIF reactor design, the implosion parameters should be further optimized to obtain a sufficient gain, which should be larger than 30$\sim$40 in HIF \cite{CPC-O-SUKI, Kawata1, Kawata2, RSato2}. 

\begin{figure}[H]
		\centering
		\includegraphics[width=13cm]{images/EuWobblIgnited.eps}
		\caption{Ion temperature distributions (a) at $t=$24.88ns, (b) at 28.44ns and at 29.21ns.}\label{Ti_EuWobblIgnited}
\end{figure}

\if0
In Fig. \ref{NoOK3_03_Ti}, a non-uniform energy deposition of the HIBs illumination is introduced based on the spherical harmonics $Y_3^0$ with the amplitude of $3.0\%$ in the 3D Lagrange code. The implosion data was obtained by the Lagrange code, and the data just before the void closure time were transferred to the Euler code through the data Conversion code.  Figure \ref{Ti_Eu_Y03} shows the ion temperature distributions  by the Euler code at (a) at $t$=36.36ns, (b) 36.57ns, (c) 41.32ns and (d) 42.41ns. In this example case the DT fuel is not yet ignited due to the insufficient ion temperature. 

\begin{figure}[H]
		\centering
		\includegraphics[width=8.5cm]{images/NoOK3_Non03_03_35ns_Ti.eps}
		\caption{Ion temperature in the 3D Lagrange code without OK3 code at $t$=35ns. The non-uniformity distriution is $Y_3^0$ with the amplitude of $3\%$.}\label{NoOK3_03_Ti}
\end{figure}


%% TIME VS ION TEMPERATURE Euler Y03
\begin{figure}[H]
		\centering
		\includegraphics[width=6.5cm]{images/ion_Eu_Y03_36_36ns.eps}
		\includegraphics[width=6.5cm]{images/ion_Eu_Y03_36_57ns.eps} \\
		\includegraphics[width=6.5cm]{images/ion_Eu_Y03_41_32ns.eps}
		\includegraphics[width=6.5cm]{images/ion_Eu_Y03_42_41ns.eps} \\
		\caption{Ion temperature distributions under a non-uniform energy deposition based on the spherical harmonics $Y_0^3$ by the Euler code,  (a) at $t$=36.36ns, (b) 36.57ns, (c) 41.32ns and (d) 42.41ns.}\label{Ti_Eu_Y03}
\end{figure}
\fi


In order to check the accuracy of the 3D Euler code, we also performed the Euler code tests, using the initial conditions of the 2D Euler code. The initial conditions in the Euler code are the output of the Lagrangian code.  To this end, the 2D Euler initial conditions were converted into 3D. Therefore, the physical values are uniform in the $\phi$ direction. The Lagrangian test 2D results for the target ion temperature ($T_i$) and the mass density ($\rho$) distribution at $t$ = 29 ns are shown in Figs. 14 and 15 in Ref. \cite{CPC-O-SUKI} for the cases with and without the wobbling HIBs.  The 2D Eulerian test results for the fusion energy gain is shown in Fig. 16 in Ref. \cite{CPC-O-SUKI}.  In Fig. \ref{Ti_Eu_3d} we show the ion temperature distributions by the 3D Euler code. The wobbling HIBs are not used in this simulation. In this case the fuel is ignited at $t \sim $30.1ns. The histories of the fusion gain $G$ of the 2D code and the 3D code are shown in Fig. \ref{FusionGain_Eu}. The fusion gain was 52.5 by the 2D code and 57.6 by the 3D code. In addition, we also did another test for the wobbling HIBs (see Figs. 15 and 16 in Ref. \cite{CPC-O-SUKI}), and the fusion gain was 76.1 in 2D \cite{CPC-O-SUKI} and 67.4 in 3D. The results would confirm that the 3D Euler code reproduces the 2D results successfully for the ignition time and the fusion gain obtained. 


%% TIME VS ION TEMPERATURE Euler
\begin{figure}[H]
		\centering
		\includegraphics[width=6.5cm]{images/ion_Eu_30_42ns.eps}
		\includegraphics[width=6.5cm]{images/ion_Eu_30_53ns.eps} \\
		\includegraphics[width=6.5cm]{images/ion_Eu_32_35ns.eps}
		\includegraphics[width=6.5cm]{images/ion_Eu_32_58ns.eps} \\
		\caption{Ion temperature distributions by the 3D Euler code without the HIBs wobbling at (a) $t$=30.42ns, (b) 30.53ns, (c) 32.35ns and (d) 32.58ns}\label{Ti_Eu_3d}
\end{figure}


%% ENERGY GAIN Euler
\begin{figure}[H]
		\centering
		\includegraphics[width=11cm]{images/FusionGain_Eu.eps}
		\caption{Fusion energy gain curves for the cases with 3D code (a solid line) and with 2D code (a dotted line).}\label{FusionGain_Eu}
\end{figure}

We also simulated the double-cone ignition scheme\cite{Double-cone} using a 3D Euler code. The double-cone ignition scheme was proposed by Prof. Jie Zhang \cite{Double-cone}, and the two compressed DT clouds are created by the gold cones. The two DT spherical clouds collide each other like the impact fusion \cite{Winterberg}. In this example case, the compressed DT maximum density of the DT fuel is set to be $1.0\times 10^5$[kg/m$^3$] with the Gaussian spatial distribution. The DT ignition will be attained by an additional heating, which is not taken into consideration in this example. The ion, electron and radiation temperatures are 10[eV] initially in the Euler code. The radius of the fuel is 92[$\mu$m] and the mass was $0.1$[mg]. We set the colliding speed $w$ of the two DT fuel clouds to $3.0\times10^5$ [m/s]. The ion temperature distributions are shown in Fig. \ref{Double_cone_Ti}.


%% DOUBLE-CONE
\begin{figure}[H]
		\centering
		\includegraphics[width=6.5cm]{images/double_cone_0ns.eps}
		\includegraphics[width=6.5cm]{images/double_cone_15_06ns.eps} \\
		\includegraphics[width=6.5cm]{images/double_cone_29_80ns.eps}
		\includegraphics[width=6.5cm]{images/double_cone_46_78ns.eps} \\
		\caption{Ion temperature distributions for the Double-cone ignition scheme \cite{Double-cone} at (a) $t$=0.0ns, (b) 15.06ns, (c) 29.80ns and (d) 46.78ns.}\label{Double_cone_Ti}
\end{figure}

	
%\end{document}
%
%
%%%%%%%%%%%%%%%%%%%%%%%%%%%
\section*{Acknowledgments}
\label{sec:acknow}
%%%%%%%%%%%%%%%%%%%%%%%%%%%
The author acknowledges support from the Science and Technology
Facilities Council (grant numbers ST/L000652/1). 
It is a pleasure for the author to acknowledges useful discussion with members of the
University of Sussex cosmology group. The author would
like to thank Donough Regan for his help and suggestions to improve the draft.
%
\bibliography{bias.bbl}
%
%
%
\appendix
\chapter{Supplementary Material}
\label{appendix}

In this appendix, we present supplementary material for the techniques and
experiments presented in the main text.

\section{Baseline Results and Analysis for Informed Sampler}
\label{appendix:chap3}

Here, we give an in-depth
performance analysis of the various samplers and the effect of their
hyperparameters. We choose hyperparameters with the lowest PSRF value
after $10k$ iterations, for each sampler individually. If the
differences between PSRF are not significantly different among
multiple values, we choose the one that has the highest acceptance
rate.

\subsection{Experiment: Estimating Camera Extrinsics}
\label{appendix:chap3:room}

\subsubsection{Parameter Selection}
\paragraph{Metropolis Hastings (\MH)}

Figure~\ref{fig:exp1_MH} shows the median acceptance rates and PSRF
values corresponding to various proposal standard deviations of plain
\MH~sampling. Mixing gets better and the acceptance rate gets worse as
the standard deviation increases. The value $0.3$ is selected standard
deviation for this sampler.

\paragraph{Metropolis Hastings Within Gibbs (\MHWG)}

As mentioned in Section~\ref{sec:room}, the \MHWG~sampler with one-dimensional
updates did not converge for any value of proposal standard deviation.
This problem has high correlation of the camera parameters and is of
multi-modal nature, which this sampler has problems with.

\paragraph{Parallel Tempering (\PT)}

For \PT~sampling, we took the best performing \MH~sampler and used
different temperature chains to improve the mixing of the
sampler. Figure~\ref{fig:exp1_PT} shows the results corresponding to
different combination of temperature levels. The sampler with
temperature levels of $[1,3,27]$ performed best in terms of both
mixing and acceptance rate.

\paragraph{Effect of Mixture Coefficient in Informed Sampling (\MIXLMH)}

Figure~\ref{fig:exp1_alpha} shows the effect of mixture
coefficient ($\alpha$) on the informed sampling
\MIXLMH. Since there is no significant different in PSRF values for
$0 \le \alpha \le 0.7$, we chose $0.7$ due to its high acceptance
rate.


% \end{multicols}

\begin{figure}[h]
\centering
  \subfigure[MH]{%
    \includegraphics[width=.48\textwidth]{figures/supplementary/camPose_MH.pdf} \label{fig:exp1_MH}
  }
  \subfigure[PT]{%
    \includegraphics[width=.48\textwidth]{figures/supplementary/camPose_PT.pdf} \label{fig:exp1_PT}
  }
\\
  \subfigure[INF-MH]{%
    \includegraphics[width=.48\textwidth]{figures/supplementary/camPose_alpha.pdf} \label{fig:exp1_alpha}
  }
  \mycaption{Results of the `Estimating Camera Extrinsics' experiment}{PRSFs and Acceptance rates corresponding to (a) various standard deviations of \MH, (b) various temperature level combinations of \PT sampling and (c) various mixture coefficients of \MIXLMH sampling.}
\end{figure}



\begin{figure}[!t]
\centering
  \subfigure[\MH]{%
    \includegraphics[width=.48\textwidth]{figures/supplementary/occlusionExp_MH.pdf} \label{fig:exp2_MH}
  }
  \subfigure[\BMHWG]{%
    \includegraphics[width=.48\textwidth]{figures/supplementary/occlusionExp_BMHWG.pdf} \label{fig:exp2_BMHWG}
  }
\\
  \subfigure[\MHWG]{%
    \includegraphics[width=.48\textwidth]{figures/supplementary/occlusionExp_MHWG.pdf} \label{fig:exp2_MHWG}
  }
  \subfigure[\PT]{%
    \includegraphics[width=.48\textwidth]{figures/supplementary/occlusionExp_PT.pdf} \label{fig:exp2_PT}
  }
\\
  \subfigure[\INFBMHWG]{%
    \includegraphics[width=.5\textwidth]{figures/supplementary/occlusionExp_alpha.pdf} \label{fig:exp2_alpha}
  }
  \mycaption{Results of the `Occluding Tiles' experiment}{PRSF and
    Acceptance rates corresponding to various standard deviations of
    (a) \MH, (b) \BMHWG, (c) \MHWG, (d) various temperature level
    combinations of \PT~sampling and; (e) various mixture coefficients
    of our informed \INFBMHWG sampling.}
\end{figure}

%\onecolumn\newpage\twocolumn
\subsection{Experiment: Occluding Tiles}
\label{appendix:chap3:tiles}

\subsubsection{Parameter Selection}

\paragraph{Metropolis Hastings (\MH)}

Figure~\ref{fig:exp2_MH} shows the results of
\MH~sampling. Results show the poor convergence for all proposal
standard deviations and rapid decrease of AR with increasing standard
deviation. This is due to the high-dimensional nature of
the problem. We selected a standard deviation of $1.1$.

\paragraph{Blocked Metropolis Hastings Within Gibbs (\BMHWG)}

The results of \BMHWG are shown in Figure~\ref{fig:exp2_BMHWG}. In
this sampler we update only one block of tile variables (of dimension
four) in each sampling step. Results show much better performance
compared to plain \MH. The optimal proposal standard deviation for
this sampler is $0.7$.

\paragraph{Metropolis Hastings Within Gibbs (\MHWG)}

Figure~\ref{fig:exp2_MHWG} shows the result of \MHWG sampling. This
sampler is better than \BMHWG and converges much more quickly. Here
a standard deviation of $0.9$ is found to be best.

\paragraph{Parallel Tempering (\PT)}

Figure~\ref{fig:exp2_PT} shows the results of \PT sampling with various
temperature combinations. Results show no improvement in AR from plain
\MH sampling and again $[1,3,27]$ temperature levels are found to be optimal.

\paragraph{Effect of Mixture Coefficient in Informed Sampling (\INFBMHWG)}

Figure~\ref{fig:exp2_alpha} shows the effect of mixture
coefficient ($\alpha$) on the blocked informed sampling
\INFBMHWG. Since there is no significant different in PSRF values for
$0 \le \alpha \le 0.8$, we chose $0.8$ due to its high acceptance
rate.



\subsection{Experiment: Estimating Body Shape}
\label{appendix:chap3:body}

\subsubsection{Parameter Selection}
\paragraph{Metropolis Hastings (\MH)}

Figure~\ref{fig:exp3_MH} shows the result of \MH~sampling with various
proposal standard deviations. The value of $0.1$ is found to be
best.

\paragraph{Metropolis Hastings Within Gibbs (\MHWG)}

For \MHWG sampling we select $0.3$ proposal standard
deviation. Results are shown in Fig.~\ref{fig:exp3_MHWG}.


\paragraph{Parallel Tempering (\PT)}

As before, results in Fig.~\ref{fig:exp3_PT}, the temperature levels
were selected to be $[1,3,27]$ due its slightly higher AR.

\paragraph{Effect of Mixture Coefficient in Informed Sampling (\MIXLMH)}

Figure~\ref{fig:exp3_alpha} shows the effect of $\alpha$ on PSRF and
AR. Since there is no significant differences in PSRF values for $0 \le
\alpha \le 0.8$, we choose $0.8$.


\begin{figure}[t]
\centering
  \subfigure[\MH]{%
    \includegraphics[width=.48\textwidth]{figures/supplementary/bodyShape_MH.pdf} \label{fig:exp3_MH}
  }
  \subfigure[\MHWG]{%
    \includegraphics[width=.48\textwidth]{figures/supplementary/bodyShape_MHWG.pdf} \label{fig:exp3_MHWG}
  }
\\
  \subfigure[\PT]{%
    \includegraphics[width=.48\textwidth]{figures/supplementary/bodyShape_PT.pdf} \label{fig:exp3_PT}
  }
  \subfigure[\MIXLMH]{%
    \includegraphics[width=.48\textwidth]{figures/supplementary/bodyShape_alpha.pdf} \label{fig:exp3_alpha}
  }
\\
  \mycaption{Results of the `Body Shape Estimation' experiment}{PRSFs and
    Acceptance rates corresponding to various standard deviations of
    (a) \MH, (b) \MHWG; (c) various temperature level combinations
    of \PT sampling and; (d) various mixture coefficients of the
    informed \MIXLMH sampling.}
\end{figure}


\subsection{Results Overview}
Figure~\ref{fig:exp_summary} shows the summary results of the all the three
experimental studies related to informed sampler.
\begin{figure*}[h!]
\centering
  \subfigure[Results for: Estimating Camera Extrinsics]{%
    \includegraphics[width=0.9\textwidth]{figures/supplementary/camPose_ALL.pdf} \label{fig:exp1_all}
  }
  \subfigure[Results for: Occluding Tiles]{%
    \includegraphics[width=0.9\textwidth]{figures/supplementary/occlusionExp_ALL.pdf} \label{fig:exp2_all}
  }
  \subfigure[Results for: Estimating Body Shape]{%
    \includegraphics[width=0.9\textwidth]{figures/supplementary/bodyShape_ALL.pdf} \label{fig:exp3_all}
  }
  \label{fig:exp_summary}
  \mycaption{Summary of the statistics for the three experiments}{Shown are
    for several baseline methods and the informed samplers the
    acceptance rates (left), PSRFs (middle), and RMSE values
    (right). All results are median results over multiple test
    examples.}
\end{figure*}

\subsection{Additional Qualitative Results}

\subsubsection{Occluding Tiles}
In Figure~\ref{fig:exp2_visual_more} more qualitative results of the
occluding tiles experiment are shown. The informed sampling approach
(\INFBMHWG) is better than the best baseline (\MHWG). This still is a
very challenging problem since the parameters for occluded tiles are
flat over a large region. Some of the posterior variance of the
occluded tiles is already captured by the informed sampler.

\begin{figure*}[h!]
\begin{center}
\centerline{\includegraphics[width=0.95\textwidth]{figures/supplementary/occlusionExp_Visual.pdf}}
\mycaption{Additional qualitative results of the occluding tiles experiment}
  {From left to right: (a)
  Given image, (b) Ground truth tiles, (c) OpenCV heuristic and most probable estimates
  from 5000 samples obtained by (d) MHWG sampler (best baseline) and
  (e) our INF-BMHWG sampler. (f) Posterior expectation of the tiles
  boundaries obtained by INF-BMHWG sampling (First 2000 samples are
  discarded as burn-in).}
\label{fig:exp2_visual_more}
\end{center}
\end{figure*}

\subsubsection{Body Shape}
Figure~\ref{fig:exp3_bodyMeshes} shows some more results of 3D mesh
reconstruction using posterior samples obtained by our informed
sampling \MIXLMH.

\begin{figure*}[t]
\begin{center}
\centerline{\includegraphics[width=0.75\textwidth]{figures/supplementary/bodyMeshResults.pdf}}
\mycaption{Qualitative results for the body shape experiment}
  {Shown is the 3D mesh reconstruction results with first 1000 samples obtained
  using the \MIXLMH informed sampling method. (blue indicates small
  values and red indicates high values)}
\label{fig:exp3_bodyMeshes}
\end{center}
\end{figure*}

\clearpage



\section{Additional Results on the Face Problem with CMP}

Figure~\ref{fig:shading-qualitative-multiple-subjects-supp} shows inference results for reflectance maps, normal maps and lights for randomly chosen test images, and Fig.~\ref{fig:shading-qualitative-same-subject-supp} shows reflectance estimation results on multiple images of the same subject produced under different illumination conditions. CMP is able to produce estimates that are closer to the groundtruth across different subjects and illumination conditions.

\begin{figure*}[h]
  \begin{center}
  \centerline{\includegraphics[width=1.0\columnwidth]{figures/face_cmp_visual_results_supp.pdf}}
  \vspace{-1.2cm}
  \end{center}
	\mycaption{A visual comparison of inference results}{(a)~Observed images. (b)~Inferred reflectance maps. \textit{GT} is the photometric stereo groundtruth, \textit{BU} is the Biswas \etal (2009) reflectance estimate and \textit{Forest} is the consensus prediction. (c)~The variance of the inferred reflectance estimate produced by \MTD (normalized across rows).(d)~Visualization of inferred light directions. (e)~Inferred normal maps.}
	\label{fig:shading-qualitative-multiple-subjects-supp}
\end{figure*}


\begin{figure*}[h]
	\centering
	\setlength\fboxsep{0.2mm}
	\setlength\fboxrule{0pt}
	\begin{tikzpicture}

		\matrix at (0, 0) [matrix of nodes, nodes={anchor=east}, column sep=-0.05cm, row sep=-0.2cm]
		{
			\fbox{\includegraphics[width=1cm]{figures/sample_3_4_X.png}} &
			\fbox{\includegraphics[width=1cm]{figures/sample_3_4_GT.png}} &
			\fbox{\includegraphics[width=1cm]{figures/sample_3_4_BISWAS.png}}  &
			\fbox{\includegraphics[width=1cm]{figures/sample_3_4_VMP.png}}  &
			\fbox{\includegraphics[width=1cm]{figures/sample_3_4_FOREST.png}}  &
			\fbox{\includegraphics[width=1cm]{figures/sample_3_4_CMP.png}}  &
			\fbox{\includegraphics[width=1cm]{figures/sample_3_4_CMPVAR.png}}
			 \\

			\fbox{\includegraphics[width=1cm]{figures/sample_3_5_X.png}} &
			\fbox{\includegraphics[width=1cm]{figures/sample_3_5_GT.png}} &
			\fbox{\includegraphics[width=1cm]{figures/sample_3_5_BISWAS.png}}  &
			\fbox{\includegraphics[width=1cm]{figures/sample_3_5_VMP.png}}  &
			\fbox{\includegraphics[width=1cm]{figures/sample_3_5_FOREST.png}}  &
			\fbox{\includegraphics[width=1cm]{figures/sample_3_5_CMP.png}}  &
			\fbox{\includegraphics[width=1cm]{figures/sample_3_5_CMPVAR.png}}
			 \\

			\fbox{\includegraphics[width=1cm]{figures/sample_3_6_X.png}} &
			\fbox{\includegraphics[width=1cm]{figures/sample_3_6_GT.png}} &
			\fbox{\includegraphics[width=1cm]{figures/sample_3_6_BISWAS.png}}  &
			\fbox{\includegraphics[width=1cm]{figures/sample_3_6_VMP.png}}  &
			\fbox{\includegraphics[width=1cm]{figures/sample_3_6_FOREST.png}}  &
			\fbox{\includegraphics[width=1cm]{figures/sample_3_6_CMP.png}}  &
			\fbox{\includegraphics[width=1cm]{figures/sample_3_6_CMPVAR.png}}
			 \\
	     };

       \node at (-3.85, -2.0) {\small Observed};
       \node at (-2.55, -2.0) {\small `GT'};
       \node at (-1.27, -2.0) {\small BU};
       \node at (0.0, -2.0) {\small MP};
       \node at (1.27, -2.0) {\small Forest};
       \node at (2.55, -2.0) {\small \textbf{CMP}};
       \node at (3.85, -2.0) {\small Variance};

	\end{tikzpicture}
	\mycaption{Robustness to varying illumination}{Reflectance estimation on a subject images with varying illumination. Left to right: observed image, photometric stereo estimate (GT)
  which is used as a proxy for groundtruth, bottom-up estimate of \cite{Biswas2009}, VMP result, consensus forest estimate, CMP mean, and CMP variance.}
	\label{fig:shading-qualitative-same-subject-supp}
\end{figure*}

\clearpage

\section{Additional Material for Learning Sparse High Dimensional Filters}
\label{sec:appendix-bnn}

This part of supplementary material contains a more detailed overview of the permutohedral
lattice convolution in Section~\ref{sec:permconv}, more experiments in
Section~\ref{sec:addexps} and additional results with protocols for
the experiments presented in Chapter~\ref{chap:bnn} in Section~\ref{sec:addresults}.

\vspace{-0.2cm}
\subsection{General Permutohedral Convolutions}
\label{sec:permconv}

A core technical contribution of this work is the generalization of the Gaussian permutohedral lattice
convolution proposed in~\cite{adams2010fast} to the full non-separable case with the
ability to perform back-propagation. Although, conceptually, there are minor
differences between Gaussian and general parameterized filters, there are non-trivial practical
differences in terms of the algorithmic implementation. The Gauss filters belong to
the separable class and can thus be decomposed into multiple
sequential one dimensional convolutions. We are interested in the general filter
convolutions, which can not be decomposed. Thus, performing a general permutohedral
convolution at a lattice point requires the computation of the inner product with the
neighboring elements in all the directions in the high-dimensional space.

Here, we give more details of the implementation differences of separable
and non-separable filters. In the following, we will explain the scalar case first.
Recall, that the forward pass of general permutohedral convolution
involves 3 steps: \textit{splatting}, \textit{convolving} and \textit{slicing}.
We follow the same splatting and slicing strategies as in~\cite{adams2010fast}
since these operations do not depend on the filter kernel. The main difference
between our work and the existing implementation of~\cite{adams2010fast} is
the way that the convolution operation is executed. This proceeds by constructing
a \emph{blur neighbor} matrix $K$ that stores for every lattice point all
values of the lattice neighbors that are needed to compute the filter output.

\begin{figure}[t!]
  \centering
    \includegraphics[width=0.6\columnwidth]{figures/supplementary/lattice_construction}
  \mycaption{Illustration of 1D permutohedral lattice construction}
  {A $4\times 4$ $(x,y)$ grid lattice is projected onto the plane defined by the normal
  vector $(1,1)^{\top}$. This grid has $s+1=4$ and $d=2$ $(s+1)^{d}=4^2=16$ elements.
  In the projection, all points of the same color are projected onto the same points in the plane.
  The number of elements of the projected lattice is $t=(s+1)^d-s^d=4^2-3^2=7$, that is
  the $(4\times 4)$ grid minus the size of lattice that is $1$ smaller at each size, in this
  case a $(3\times 3)$ lattice (the upper right $(3\times 3)$ elements).
  }
\label{fig:latticeconstruction}
\end{figure}

The blur neighbor matrix is constructed by traversing through all the populated
lattice points and their neighboring elements.
% For efficiency, we do this matrix construction recursively with shared computations
% since $n^{th}$ neighbourhood elements are $1^{st}$ neighborhood elements of $n-1^{th}$ neighbourhood elements. \pg{do not understand}
This is done recursively to share computations. For any lattice point, the neighbors that are
$n$ hops away are the direct neighbors of the points that are $n-1$ hops away.
The size of a $d$ dimensional spatial filter with width $s+1$ is $(s+1)^{d}$ (\eg, a
$3\times 3$ filter, $s=2$ in $d=2$ has $3^2=9$ elements) and this size grows
exponentially in the number of dimensions $d$. The permutohedral lattice is constructed by
projecting a regular grid onto the plane spanned by the $d$ dimensional normal vector ${(1,\ldots,1)}^{\top}$. See
Fig.~\ref{fig:latticeconstruction} for an illustration of the 1D lattice construction.
Many corners of a grid filter are projected onto the same point, in total $t = {(s+1)}^{d} -
s^{d}$ elements remain in the permutohedral filter with $s$ neighborhood in $d-1$ dimensions.
If the lattice has $m$ populated elements, the
matrix $K$ has size $t\times m$. Note that, since the input signal is typically
sparse, only a few lattice corners are being populated in the \textit{slicing} step.
We use a hash-table to keep track of these points and traverse only through
the populated lattice points for this neighborhood matrix construction.

Once the blur neighbor matrix $K$ is constructed, we can perform the convolution
by the matrix vector multiplication
\begin{equation}
\ell' = BK,
\label{eq:conv}
\end{equation}
where $B$ is the $1 \times t$ filter kernel (whose values we will learn) and $\ell'\in\mathbb{R}^{1\times m}$
is the result of the filtering at the $m$ lattice points. In practice, we found that the
matrix $K$ is sometimes too large to fit into GPU memory and we divided the matrix $K$
into smaller pieces to compute Eq.~\ref{eq:conv} sequentially.

In the general multi-dimensional case, the signal $\ell$ is of $c$ dimensions. Then
the kernel $B$ is of size $c \times t$ and $K$ stores the $c$ dimensional vectors
accordingly. When the input and output points are different, we slice only the
input points and splat only at the output points.


\subsection{Additional Experiments}
\label{sec:addexps}
In this section, we discuss more use-cases for the learned bilateral filters, one
use-case of BNNs and two single filter applications for image and 3D mesh denoising.

\subsubsection{Recognition of subsampled MNIST}\label{sec:app_mnist}

One of the strengths of the proposed filter convolution is that it does not
require the input to lie on a regular grid. The only requirement is to define a distance
between features of the input signal.
We highlight this feature with the following experiment using the
classical MNIST ten class classification problem~\cite{lecun1998mnist}. We sample a
sparse set of $N$ points $(x,y)\in [0,1]\times [0,1]$
uniformly at random in the input image, use their interpolated values
as signal and the \emph{continuous} $(x,y)$ positions as features. This mimics
sub-sampling of a high-dimensional signal. To compare against a spatial convolution,
we interpolate the sparse set of values at the grid positions.

We take a reference implementation of LeNet~\cite{lecun1998gradient} that
is part of the Caffe project~\cite{jia2014caffe} and compare it
against the same architecture but replacing the first convolutional
layer with a bilateral convolution layer (BCL). The filter size
and numbers are adjusted to get a comparable number of parameters
($5\times 5$ for LeNet, $2$-neighborhood for BCL).

The results are shown in Table~\ref{tab:all-results}. We see that training
on the original MNIST data (column Original, LeNet vs. BNN) leads to a slight
decrease in performance of the BNN (99.03\%) compared to LeNet
(99.19\%). The BNN can be trained and evaluated on sparse
signals, and we resample the image as described above for $N=$ 100\%, 60\% and
20\% of the total number of pixels. The methods are also evaluated
on test images that are subsampled in the same way. Note that we can
train and test with different subsampling rates. We introduce an additional
bilinear interpolation layer for the LeNet architecture to train on the same
data. In essence, both models perform a spatial interpolation and thus we
expect them to yield a similar classification accuracy. Once the data is of
higher dimensions, the permutohedral convolution will be faster due to hashing
the sparse input points, as well as less memory demanding in comparison to
naive application of a spatial convolution with interpolated values.

\begin{table}[t]
  \begin{center}
    \footnotesize
    \centering
    \begin{tabular}[t]{lllll}
      \toprule
              &     & \multicolumn{3}{c}{Test Subsampling} \\
       Method  & Original & 100\% & 60\% & 20\%\\
      \midrule
       LeNet &  \textbf{0.9919} & 0.9660 & 0.9348 & \textbf{0.6434} \\
       BNN &  0.9903 & \textbf{0.9844} & \textbf{0.9534} & 0.5767 \\
      \hline
       LeNet 100\% & 0.9856 & 0.9809 & 0.9678 & \textbf{0.7386} \\
       BNN 100\% & \textbf{0.9900} & \textbf{0.9863} & \textbf{0.9699} & 0.6910 \\
      \hline
       LeNet 60\% & 0.9848 & 0.9821 & 0.9740 & 0.8151 \\
       BNN 60\% & \textbf{0.9885} & \textbf{0.9864} & \textbf{0.9771} & \textbf{0.8214}\\
      \hline
       LeNet 20\% & \textbf{0.9763} & \textbf{0.9754} & 0.9695 & 0.8928 \\
       BNN 20\% & 0.9728 & 0.9735 & \textbf{0.9701} & \textbf{0.9042}\\
      \bottomrule
    \end{tabular}
  \end{center}
\vspace{-.2cm}
\caption{Classification accuracy on MNIST. We compare the
    LeNet~\cite{lecun1998gradient} implementation that is part of
    Caffe~\cite{jia2014caffe} to the network with the first layer
    replaced by a bilateral convolution layer (BCL). Both are trained
    on the original image resolution (first two rows). Three more BNN
    and CNN models are trained with randomly subsampled images (100\%,
    60\% and 20\% of the pixels). An additional bilinear interpolation
    layer samples the input signal on a spatial grid for the CNN model.
  }
  \label{tab:all-results}
\vspace{-.5cm}
\end{table}

\subsubsection{Image Denoising}

The main application that inspired the development of the bilateral
filtering operation is image denoising~\cite{aurich1995non}, there
using a single Gaussian kernel. Our development allows to learn this
kernel function from data and we explore how to improve using a \emph{single}
but more general bilateral filter.

We use the Berkeley segmentation dataset
(BSDS500)~\cite{arbelaezi2011bsds500} as a test bed. The color
images in the dataset are converted to gray-scale,
and corrupted with Gaussian noise with a standard deviation of
$\frac {25} {255}$.

We compare the performance of four different filter models on a
denoising task.
The first baseline model (`Spatial' in Table \ref{tab:denoising}, $25$
weights) uses a single spatial filter with a kernel size of
$5$ and predicts the scalar gray-scale value at the center pixel. The next model
(`Gauss Bilateral') applies a bilateral \emph{Gaussian}
filter to the noisy input, using position and intensity features $\f=(x,y,v)^\top$.
The third setup (`Learned Bilateral', $65$ weights)
takes a Gauss kernel as initialization and
fits all filter weights on the train set to minimize the
mean squared error with respect to the clean images.
We run a combination
of spatial and permutohedral convolutions on spatial and bilateral
features (`Spatial + Bilateral (Learned)') to check for a complementary
performance of the two convolutions.

\label{sec:exp:denoising}
\begin{table}[!h]
\begin{center}
  \footnotesize
  \begin{tabular}[t]{lr}
    \toprule
    Method & PSNR \\
    \midrule
    Noisy Input & $20.17$ \\
    Spatial & $26.27$ \\
    Gauss Bilateral & $26.51$ \\
    Learned Bilateral & $26.58$ \\
    Spatial + Bilateral (Learned) & \textbf{$26.65$} \\
    \bottomrule
  \end{tabular}
\end{center}
\vspace{-0.5em}
\caption{PSNR results of a denoising task using the BSDS500
  dataset~\cite{arbelaezi2011bsds500}}
\vspace{-0.5em}
\label{tab:denoising}
\end{table}
\vspace{-0.2em}

The PSNR scores evaluated on full images of the test set are
shown in Table \ref{tab:denoising}. We find that an untrained bilateral
filter already performs better than a trained spatial convolution
($26.27$ to $26.51$). A learned convolution further improve the
performance slightly. We chose this simple one-kernel setup to
validate an advantage of the generalized bilateral filter. A competitive
denoising system would employ RGB color information and also
needs to be properly adjusted in network size. Multi-layer perceptrons
have obtained state-of-the-art denoising results~\cite{burger12cvpr}
and the permutohedral lattice layer can readily be used in such an
architecture, which is intended future work.

\subsection{Additional results}
\label{sec:addresults}

This section contains more qualitative results for the experiments presented in Chapter~\ref{chap:bnn}.

\begin{figure*}[th!]
  \centering
    \includegraphics[width=\columnwidth,trim={5cm 2.5cm 5cm 4.5cm},clip]{figures/supplementary/lattice_viz.pdf}
    \vspace{-0.7cm}
  \mycaption{Visualization of the Permutohedral Lattice}
  {Sample lattice visualizations for different feature spaces. All pixels falling in the same simplex cell are shown with
  the same color. $(x,y)$ features correspond to image pixel positions, and $(r,g,b) \in [0,255]$ correspond
  to the red, green and blue color values.}
\label{fig:latticeviz}
\end{figure*}

\subsubsection{Lattice Visualization}

Figure~\ref{fig:latticeviz} shows sample lattice visualizations for different feature spaces.

\newcolumntype{L}[1]{>{\raggedright\let\newline\\\arraybackslash\hspace{0pt}}b{#1}}
\newcolumntype{C}[1]{>{\centering\let\newline\\\arraybackslash\hspace{0pt}}b{#1}}
\newcolumntype{R}[1]{>{\raggedleft\let\newline\\\arraybackslash\hspace{0pt}}b{#1}}

\subsubsection{Color Upsampling}\label{sec:color_upsampling}
\label{sec:col_upsample_extra}

Some images of the upsampling for the Pascal
VOC12 dataset are shown in Fig.~\ref{fig:Colour_upsample_visuals}. It is
especially the low level image details that are better preserved with
a learned bilateral filter compared to the Gaussian case.

\begin{figure*}[t!]
  \centering
    \subfigure{%
   \raisebox{2.0em}{
    \includegraphics[width=.06\columnwidth]{figures/supplementary/2007_004969.jpg}
   }
  }
  \subfigure{%
    \includegraphics[width=.17\columnwidth]{figures/supplementary/2007_004969_gray.pdf}
  }
  \subfigure{%
    \includegraphics[width=.17\columnwidth]{figures/supplementary/2007_004969_gt.pdf}
  }
  \subfigure{%
    \includegraphics[width=.17\columnwidth]{figures/supplementary/2007_004969_bicubic.pdf}
  }
  \subfigure{%
    \includegraphics[width=.17\columnwidth]{figures/supplementary/2007_004969_gauss.pdf}
  }
  \subfigure{%
    \includegraphics[width=.17\columnwidth]{figures/supplementary/2007_004969_learnt.pdf}
  }\\
    \subfigure{%
   \raisebox{2.0em}{
    \includegraphics[width=.06\columnwidth]{figures/supplementary/2007_003106.jpg}
   }
  }
  \subfigure{%
    \includegraphics[width=.17\columnwidth]{figures/supplementary/2007_003106_gray.pdf}
  }
  \subfigure{%
    \includegraphics[width=.17\columnwidth]{figures/supplementary/2007_003106_gt.pdf}
  }
  \subfigure{%
    \includegraphics[width=.17\columnwidth]{figures/supplementary/2007_003106_bicubic.pdf}
  }
  \subfigure{%
    \includegraphics[width=.17\columnwidth]{figures/supplementary/2007_003106_gauss.pdf}
  }
  \subfigure{%
    \includegraphics[width=.17\columnwidth]{figures/supplementary/2007_003106_learnt.pdf}
  }\\
  \setcounter{subfigure}{0}
  \small{
  \subfigure[Inp.]{%
  \raisebox{2.0em}{
    \includegraphics[width=.06\columnwidth]{figures/supplementary/2007_006837.jpg}
   }
  }
  \subfigure[Guidance]{%
    \includegraphics[width=.17\columnwidth]{figures/supplementary/2007_006837_gray.pdf}
  }
   \subfigure[GT]{%
    \includegraphics[width=.17\columnwidth]{figures/supplementary/2007_006837_gt.pdf}
  }
  \subfigure[Bicubic]{%
    \includegraphics[width=.17\columnwidth]{figures/supplementary/2007_006837_bicubic.pdf}
  }
  \subfigure[Gauss-BF]{%
    \includegraphics[width=.17\columnwidth]{figures/supplementary/2007_006837_gauss.pdf}
  }
  \subfigure[Learned-BF]{%
    \includegraphics[width=.17\columnwidth]{figures/supplementary/2007_006837_learnt.pdf}
  }
  }
  \vspace{-0.5cm}
  \mycaption{Color Upsampling}{Color $8\times$ upsampling results
  using different methods, from left to right, (a)~Low-resolution input color image (Inp.),
  (b)~Gray scale guidance image, (c)~Ground-truth color image; Upsampled color images with
  (d)~Bicubic interpolation, (e) Gauss bilateral upsampling and, (f)~Learned bilateral
  updampgling (best viewed on screen).}

\label{fig:Colour_upsample_visuals}
\end{figure*}

\subsubsection{Depth Upsampling}
\label{sec:depth_upsample_extra}

Figure~\ref{fig:depth_upsample_visuals} presents some more qualitative results comparing bicubic interpolation, Gauss
bilateral and learned bilateral upsampling on NYU depth dataset image~\cite{silberman2012indoor}.

\subsubsection{Character Recognition}\label{sec:app_character}

 Figure~\ref{fig:nnrecognition} shows the schematic of different layers
 of the network architecture for LeNet-7~\cite{lecun1998mnist}
 and DeepCNet(5, 50)~\cite{ciresan2012multi,graham2014spatially}. For the BNN variants, the first layer filters are replaced
 with learned bilateral filters and are learned end-to-end.

\subsubsection{Semantic Segmentation}\label{sec:app_semantic_segmentation}
\label{sec:semantic_bnn_extra}

Some more visual results for semantic segmentation are shown in Figure~\ref{fig:semantic_visuals}.
These include the underlying DeepLab CNN\cite{chen2014semantic} result (DeepLab),
the 2 step mean-field result with Gaussian edge potentials (+2stepMF-GaussCRF)
and also corresponding results with learned edge potentials (+2stepMF-LearnedCRF).
In general, we observe that mean-field in learned CRF leads to slightly dilated
classification regions in comparison to using Gaussian CRF thereby filling-in the
false negative pixels and also correcting some mis-classified regions.

\begin{figure*}[t!]
  \centering
    \subfigure{%
   \raisebox{2.0em}{
    \includegraphics[width=.06\columnwidth]{figures/supplementary/2bicubic}
   }
  }
  \subfigure{%
    \includegraphics[width=.17\columnwidth]{figures/supplementary/2given_image}
  }
  \subfigure{%
    \includegraphics[width=.17\columnwidth]{figures/supplementary/2ground_truth}
  }
  \subfigure{%
    \includegraphics[width=.17\columnwidth]{figures/supplementary/2bicubic}
  }
  \subfigure{%
    \includegraphics[width=.17\columnwidth]{figures/supplementary/2gauss}
  }
  \subfigure{%
    \includegraphics[width=.17\columnwidth]{figures/supplementary/2learnt}
  }\\
    \subfigure{%
   \raisebox{2.0em}{
    \includegraphics[width=.06\columnwidth]{figures/supplementary/32bicubic}
   }
  }
  \subfigure{%
    \includegraphics[width=.17\columnwidth]{figures/supplementary/32given_image}
  }
  \subfigure{%
    \includegraphics[width=.17\columnwidth]{figures/supplementary/32ground_truth}
  }
  \subfigure{%
    \includegraphics[width=.17\columnwidth]{figures/supplementary/32bicubic}
  }
  \subfigure{%
    \includegraphics[width=.17\columnwidth]{figures/supplementary/32gauss}
  }
  \subfigure{%
    \includegraphics[width=.17\columnwidth]{figures/supplementary/32learnt}
  }\\
  \setcounter{subfigure}{0}
  \small{
  \subfigure[Inp.]{%
  \raisebox{2.0em}{
    \includegraphics[width=.06\columnwidth]{figures/supplementary/41bicubic}
   }
  }
  \subfigure[Guidance]{%
    \includegraphics[width=.17\columnwidth]{figures/supplementary/41given_image}
  }
   \subfigure[GT]{%
    \includegraphics[width=.17\columnwidth]{figures/supplementary/41ground_truth}
  }
  \subfigure[Bicubic]{%
    \includegraphics[width=.17\columnwidth]{figures/supplementary/41bicubic}
  }
  \subfigure[Gauss-BF]{%
    \includegraphics[width=.17\columnwidth]{figures/supplementary/41gauss}
  }
  \subfigure[Learned-BF]{%
    \includegraphics[width=.17\columnwidth]{figures/supplementary/41learnt}
  }
  }
  \mycaption{Depth Upsampling}{Depth $8\times$ upsampling results
  using different upsampling strategies, from left to right,
  (a)~Low-resolution input depth image (Inp.),
  (b)~High-resolution guidance image, (c)~Ground-truth depth; Upsampled depth images with
  (d)~Bicubic interpolation, (e) Gauss bilateral upsampling and, (f)~Learned bilateral
  updampgling (best viewed on screen).}

\label{fig:depth_upsample_visuals}
\end{figure*}

\subsubsection{Material Segmentation}\label{sec:app_material_segmentation}
\label{sec:material_bnn_extra}

In Fig.~\ref{fig:material_visuals-app2}, we present visual results comparing 2 step
mean-field inference with Gaussian and learned pairwise CRF potentials. In
general, we observe that the pixels belonging to dominant classes in the
training data are being more accurately classified with learned CRF. This leads to
a significant improvements in overall pixel accuracy. This also results
in a slight decrease of the accuracy from less frequent class pixels thereby
slightly reducing the average class accuracy with learning. We attribute this
to the type of annotation that is available for this dataset, which is not
for the entire image but for some segments in the image. We have very few
images of the infrequent classes to combat this behaviour during training.

\subsubsection{Experiment Protocols}
\label{sec:protocols}

Table~\ref{tbl:parameters} shows experiment protocols of different experiments.

 \begin{figure*}[t!]
  \centering
  \subfigure[LeNet-7]{
    \includegraphics[width=0.7\columnwidth]{figures/supplementary/lenet_cnn_network}
    }\\
    \subfigure[DeepCNet]{
    \includegraphics[width=\columnwidth]{figures/supplementary/deepcnet_cnn_network}
    }
  \mycaption{CNNs for Character Recognition}
  {Schematic of (top) LeNet-7~\cite{lecun1998mnist} and (bottom) DeepCNet(5,50)~\cite{ciresan2012multi,graham2014spatially} architectures used in Assamese
  character recognition experiments.}
\label{fig:nnrecognition}
\end{figure*}

\definecolor{voc_1}{RGB}{0, 0, 0}
\definecolor{voc_2}{RGB}{128, 0, 0}
\definecolor{voc_3}{RGB}{0, 128, 0}
\definecolor{voc_4}{RGB}{128, 128, 0}
\definecolor{voc_5}{RGB}{0, 0, 128}
\definecolor{voc_6}{RGB}{128, 0, 128}
\definecolor{voc_7}{RGB}{0, 128, 128}
\definecolor{voc_8}{RGB}{128, 128, 128}
\definecolor{voc_9}{RGB}{64, 0, 0}
\definecolor{voc_10}{RGB}{192, 0, 0}
\definecolor{voc_11}{RGB}{64, 128, 0}
\definecolor{voc_12}{RGB}{192, 128, 0}
\definecolor{voc_13}{RGB}{64, 0, 128}
\definecolor{voc_14}{RGB}{192, 0, 128}
\definecolor{voc_15}{RGB}{64, 128, 128}
\definecolor{voc_16}{RGB}{192, 128, 128}
\definecolor{voc_17}{RGB}{0, 64, 0}
\definecolor{voc_18}{RGB}{128, 64, 0}
\definecolor{voc_19}{RGB}{0, 192, 0}
\definecolor{voc_20}{RGB}{128, 192, 0}
\definecolor{voc_21}{RGB}{0, 64, 128}
\definecolor{voc_22}{RGB}{128, 64, 128}

\begin{figure*}[t]
  \centering
  \small{
  \fcolorbox{white}{voc_1}{\rule{0pt}{6pt}\rule{6pt}{0pt}} Background~~
  \fcolorbox{white}{voc_2}{\rule{0pt}{6pt}\rule{6pt}{0pt}} Aeroplane~~
  \fcolorbox{white}{voc_3}{\rule{0pt}{6pt}\rule{6pt}{0pt}} Bicycle~~
  \fcolorbox{white}{voc_4}{\rule{0pt}{6pt}\rule{6pt}{0pt}} Bird~~
  \fcolorbox{white}{voc_5}{\rule{0pt}{6pt}\rule{6pt}{0pt}} Boat~~
  \fcolorbox{white}{voc_6}{\rule{0pt}{6pt}\rule{6pt}{0pt}} Bottle~~
  \fcolorbox{white}{voc_7}{\rule{0pt}{6pt}\rule{6pt}{0pt}} Bus~~
  \fcolorbox{white}{voc_8}{\rule{0pt}{6pt}\rule{6pt}{0pt}} Car~~ \\
  \fcolorbox{white}{voc_9}{\rule{0pt}{6pt}\rule{6pt}{0pt}} Cat~~
  \fcolorbox{white}{voc_10}{\rule{0pt}{6pt}\rule{6pt}{0pt}} Chair~~
  \fcolorbox{white}{voc_11}{\rule{0pt}{6pt}\rule{6pt}{0pt}} Cow~~
  \fcolorbox{white}{voc_12}{\rule{0pt}{6pt}\rule{6pt}{0pt}} Dining Table~~
  \fcolorbox{white}{voc_13}{\rule{0pt}{6pt}\rule{6pt}{0pt}} Dog~~
  \fcolorbox{white}{voc_14}{\rule{0pt}{6pt}\rule{6pt}{0pt}} Horse~~
  \fcolorbox{white}{voc_15}{\rule{0pt}{6pt}\rule{6pt}{0pt}} Motorbike~~
  \fcolorbox{white}{voc_16}{\rule{0pt}{6pt}\rule{6pt}{0pt}} Person~~ \\
  \fcolorbox{white}{voc_17}{\rule{0pt}{6pt}\rule{6pt}{0pt}} Potted Plant~~
  \fcolorbox{white}{voc_18}{\rule{0pt}{6pt}\rule{6pt}{0pt}} Sheep~~
  \fcolorbox{white}{voc_19}{\rule{0pt}{6pt}\rule{6pt}{0pt}} Sofa~~
  \fcolorbox{white}{voc_20}{\rule{0pt}{6pt}\rule{6pt}{0pt}} Train~~
  \fcolorbox{white}{voc_21}{\rule{0pt}{6pt}\rule{6pt}{0pt}} TV monitor~~ \\
  }
  \subfigure{%
    \includegraphics[width=.18\columnwidth]{figures/supplementary/2007_001423_given.jpg}
  }
  \subfigure{%
    \includegraphics[width=.18\columnwidth]{figures/supplementary/2007_001423_gt.png}
  }
  \subfigure{%
    \includegraphics[width=.18\columnwidth]{figures/supplementary/2007_001423_cnn.png}
  }
  \subfigure{%
    \includegraphics[width=.18\columnwidth]{figures/supplementary/2007_001423_gauss.png}
  }
  \subfigure{%
    \includegraphics[width=.18\columnwidth]{figures/supplementary/2007_001423_learnt.png}
  }\\
  \subfigure{%
    \includegraphics[width=.18\columnwidth]{figures/supplementary/2007_001430_given.jpg}
  }
  \subfigure{%
    \includegraphics[width=.18\columnwidth]{figures/supplementary/2007_001430_gt.png}
  }
  \subfigure{%
    \includegraphics[width=.18\columnwidth]{figures/supplementary/2007_001430_cnn.png}
  }
  \subfigure{%
    \includegraphics[width=.18\columnwidth]{figures/supplementary/2007_001430_gauss.png}
  }
  \subfigure{%
    \includegraphics[width=.18\columnwidth]{figures/supplementary/2007_001430_learnt.png}
  }\\
    \subfigure{%
    \includegraphics[width=.18\columnwidth]{figures/supplementary/2007_007996_given.jpg}
  }
  \subfigure{%
    \includegraphics[width=.18\columnwidth]{figures/supplementary/2007_007996_gt.png}
  }
  \subfigure{%
    \includegraphics[width=.18\columnwidth]{figures/supplementary/2007_007996_cnn.png}
  }
  \subfigure{%
    \includegraphics[width=.18\columnwidth]{figures/supplementary/2007_007996_gauss.png}
  }
  \subfigure{%
    \includegraphics[width=.18\columnwidth]{figures/supplementary/2007_007996_learnt.png}
  }\\
   \subfigure{%
    \includegraphics[width=.18\columnwidth]{figures/supplementary/2010_002682_given.jpg}
  }
  \subfigure{%
    \includegraphics[width=.18\columnwidth]{figures/supplementary/2010_002682_gt.png}
  }
  \subfigure{%
    \includegraphics[width=.18\columnwidth]{figures/supplementary/2010_002682_cnn.png}
  }
  \subfigure{%
    \includegraphics[width=.18\columnwidth]{figures/supplementary/2010_002682_gauss.png}
  }
  \subfigure{%
    \includegraphics[width=.18\columnwidth]{figures/supplementary/2010_002682_learnt.png}
  }\\
     \subfigure{%
    \includegraphics[width=.18\columnwidth]{figures/supplementary/2010_004789_given.jpg}
  }
  \subfigure{%
    \includegraphics[width=.18\columnwidth]{figures/supplementary/2010_004789_gt.png}
  }
  \subfigure{%
    \includegraphics[width=.18\columnwidth]{figures/supplementary/2010_004789_cnn.png}
  }
  \subfigure{%
    \includegraphics[width=.18\columnwidth]{figures/supplementary/2010_004789_gauss.png}
  }
  \subfigure{%
    \includegraphics[width=.18\columnwidth]{figures/supplementary/2010_004789_learnt.png}
  }\\
       \subfigure{%
    \includegraphics[width=.18\columnwidth]{figures/supplementary/2007_001311_given.jpg}
  }
  \subfigure{%
    \includegraphics[width=.18\columnwidth]{figures/supplementary/2007_001311_gt.png}
  }
  \subfigure{%
    \includegraphics[width=.18\columnwidth]{figures/supplementary/2007_001311_cnn.png}
  }
  \subfigure{%
    \includegraphics[width=.18\columnwidth]{figures/supplementary/2007_001311_gauss.png}
  }
  \subfigure{%
    \includegraphics[width=.18\columnwidth]{figures/supplementary/2007_001311_learnt.png}
  }\\
  \setcounter{subfigure}{0}
  \subfigure[Input]{%
    \includegraphics[width=.18\columnwidth]{figures/supplementary/2010_003531_given.jpg}
  }
  \subfigure[Ground Truth]{%
    \includegraphics[width=.18\columnwidth]{figures/supplementary/2010_003531_gt.png}
  }
  \subfigure[DeepLab]{%
    \includegraphics[width=.18\columnwidth]{figures/supplementary/2010_003531_cnn.png}
  }
  \subfigure[+GaussCRF]{%
    \includegraphics[width=.18\columnwidth]{figures/supplementary/2010_003531_gauss.png}
  }
  \subfigure[+LearnedCRF]{%
    \includegraphics[width=.18\columnwidth]{figures/supplementary/2010_003531_learnt.png}
  }
  \vspace{-0.3cm}
  \mycaption{Semantic Segmentation}{Example results of semantic segmentation.
  (c)~depicts the unary results before application of MF, (d)~after two steps of MF with Gaussian edge CRF potentials, (e)~after
  two steps of MF with learned edge CRF potentials.}
    \label{fig:semantic_visuals}
\end{figure*}


\definecolor{minc_1}{HTML}{771111}
\definecolor{minc_2}{HTML}{CAC690}
\definecolor{minc_3}{HTML}{EEEEEE}
\definecolor{minc_4}{HTML}{7C8FA6}
\definecolor{minc_5}{HTML}{597D31}
\definecolor{minc_6}{HTML}{104410}
\definecolor{minc_7}{HTML}{BB819C}
\definecolor{minc_8}{HTML}{D0CE48}
\definecolor{minc_9}{HTML}{622745}
\definecolor{minc_10}{HTML}{666666}
\definecolor{minc_11}{HTML}{D54A31}
\definecolor{minc_12}{HTML}{101044}
\definecolor{minc_13}{HTML}{444126}
\definecolor{minc_14}{HTML}{75D646}
\definecolor{minc_15}{HTML}{DD4348}
\definecolor{minc_16}{HTML}{5C8577}
\definecolor{minc_17}{HTML}{C78472}
\definecolor{minc_18}{HTML}{75D6D0}
\definecolor{minc_19}{HTML}{5B4586}
\definecolor{minc_20}{HTML}{C04393}
\definecolor{minc_21}{HTML}{D69948}
\definecolor{minc_22}{HTML}{7370D8}
\definecolor{minc_23}{HTML}{7A3622}
\definecolor{minc_24}{HTML}{000000}

\begin{figure*}[t]
  \centering
  \small{
  \fcolorbox{white}{minc_1}{\rule{0pt}{6pt}\rule{6pt}{0pt}} Brick~~
  \fcolorbox{white}{minc_2}{\rule{0pt}{6pt}\rule{6pt}{0pt}} Carpet~~
  \fcolorbox{white}{minc_3}{\rule{0pt}{6pt}\rule{6pt}{0pt}} Ceramic~~
  \fcolorbox{white}{minc_4}{\rule{0pt}{6pt}\rule{6pt}{0pt}} Fabric~~
  \fcolorbox{white}{minc_5}{\rule{0pt}{6pt}\rule{6pt}{0pt}} Foliage~~
  \fcolorbox{white}{minc_6}{\rule{0pt}{6pt}\rule{6pt}{0pt}} Food~~
  \fcolorbox{white}{minc_7}{\rule{0pt}{6pt}\rule{6pt}{0pt}} Glass~~
  \fcolorbox{white}{minc_8}{\rule{0pt}{6pt}\rule{6pt}{0pt}} Hair~~ \\
  \fcolorbox{white}{minc_9}{\rule{0pt}{6pt}\rule{6pt}{0pt}} Leather~~
  \fcolorbox{white}{minc_10}{\rule{0pt}{6pt}\rule{6pt}{0pt}} Metal~~
  \fcolorbox{white}{minc_11}{\rule{0pt}{6pt}\rule{6pt}{0pt}} Mirror~~
  \fcolorbox{white}{minc_12}{\rule{0pt}{6pt}\rule{6pt}{0pt}} Other~~
  \fcolorbox{white}{minc_13}{\rule{0pt}{6pt}\rule{6pt}{0pt}} Painted~~
  \fcolorbox{white}{minc_14}{\rule{0pt}{6pt}\rule{6pt}{0pt}} Paper~~
  \fcolorbox{white}{minc_15}{\rule{0pt}{6pt}\rule{6pt}{0pt}} Plastic~~\\
  \fcolorbox{white}{minc_16}{\rule{0pt}{6pt}\rule{6pt}{0pt}} Polished Stone~~
  \fcolorbox{white}{minc_17}{\rule{0pt}{6pt}\rule{6pt}{0pt}} Skin~~
  \fcolorbox{white}{minc_18}{\rule{0pt}{6pt}\rule{6pt}{0pt}} Sky~~
  \fcolorbox{white}{minc_19}{\rule{0pt}{6pt}\rule{6pt}{0pt}} Stone~~
  \fcolorbox{white}{minc_20}{\rule{0pt}{6pt}\rule{6pt}{0pt}} Tile~~
  \fcolorbox{white}{minc_21}{\rule{0pt}{6pt}\rule{6pt}{0pt}} Wallpaper~~
  \fcolorbox{white}{minc_22}{\rule{0pt}{6pt}\rule{6pt}{0pt}} Water~~
  \fcolorbox{white}{minc_23}{\rule{0pt}{6pt}\rule{6pt}{0pt}} Wood~~ \\
  }
  \subfigure{%
    \includegraphics[width=.18\columnwidth]{figures/supplementary/000010868_given.jpg}
  }
  \subfigure{%
    \includegraphics[width=.18\columnwidth]{figures/supplementary/000010868_gt.png}
  }
  \subfigure{%
    \includegraphics[width=.18\columnwidth]{figures/supplementary/000010868_cnn.png}
  }
  \subfigure{%
    \includegraphics[width=.18\columnwidth]{figures/supplementary/000010868_gauss.png}
  }
  \subfigure{%
    \includegraphics[width=.18\columnwidth]{figures/supplementary/000010868_learnt.png}
  }\\[-2ex]
  \subfigure{%
    \includegraphics[width=.18\columnwidth]{figures/supplementary/000006011_given.jpg}
  }
  \subfigure{%
    \includegraphics[width=.18\columnwidth]{figures/supplementary/000006011_gt.png}
  }
  \subfigure{%
    \includegraphics[width=.18\columnwidth]{figures/supplementary/000006011_cnn.png}
  }
  \subfigure{%
    \includegraphics[width=.18\columnwidth]{figures/supplementary/000006011_gauss.png}
  }
  \subfigure{%
    \includegraphics[width=.18\columnwidth]{figures/supplementary/000006011_learnt.png}
  }\\[-2ex]
    \subfigure{%
    \includegraphics[width=.18\columnwidth]{figures/supplementary/000008553_given.jpg}
  }
  \subfigure{%
    \includegraphics[width=.18\columnwidth]{figures/supplementary/000008553_gt.png}
  }
  \subfigure{%
    \includegraphics[width=.18\columnwidth]{figures/supplementary/000008553_cnn.png}
  }
  \subfigure{%
    \includegraphics[width=.18\columnwidth]{figures/supplementary/000008553_gauss.png}
  }
  \subfigure{%
    \includegraphics[width=.18\columnwidth]{figures/supplementary/000008553_learnt.png}
  }\\[-2ex]
   \subfigure{%
    \includegraphics[width=.18\columnwidth]{figures/supplementary/000009188_given.jpg}
  }
  \subfigure{%
    \includegraphics[width=.18\columnwidth]{figures/supplementary/000009188_gt.png}
  }
  \subfigure{%
    \includegraphics[width=.18\columnwidth]{figures/supplementary/000009188_cnn.png}
  }
  \subfigure{%
    \includegraphics[width=.18\columnwidth]{figures/supplementary/000009188_gauss.png}
  }
  \subfigure{%
    \includegraphics[width=.18\columnwidth]{figures/supplementary/000009188_learnt.png}
  }\\[-2ex]
  \setcounter{subfigure}{0}
  \subfigure[Input]{%
    \includegraphics[width=.18\columnwidth]{figures/supplementary/000023570_given.jpg}
  }
  \subfigure[Ground Truth]{%
    \includegraphics[width=.18\columnwidth]{figures/supplementary/000023570_gt.png}
  }
  \subfigure[DeepLab]{%
    \includegraphics[width=.18\columnwidth]{figures/supplementary/000023570_cnn.png}
  }
  \subfigure[+GaussCRF]{%
    \includegraphics[width=.18\columnwidth]{figures/supplementary/000023570_gauss.png}
  }
  \subfigure[+LearnedCRF]{%
    \includegraphics[width=.18\columnwidth]{figures/supplementary/000023570_learnt.png}
  }
  \mycaption{Material Segmentation}{Example results of material segmentation.
  (c)~depicts the unary results before application of MF, (d)~after two steps of MF with Gaussian edge CRF potentials, (e)~after two steps of MF with learned edge CRF potentials.}
    \label{fig:material_visuals-app2}
\end{figure*}


\begin{table*}[h]
\tiny
  \centering
    \begin{tabular}{L{2.3cm} L{2.25cm} C{1.5cm} C{0.7cm} C{0.6cm} C{0.7cm} C{0.7cm} C{0.7cm} C{1.6cm} C{0.6cm} C{0.6cm} C{0.6cm}}
      \toprule
& & & & & \multicolumn{3}{c}{\textbf{Data Statistics}} & \multicolumn{4}{c}{\textbf{Training Protocol}} \\

\textbf{Experiment} & \textbf{Feature Types} & \textbf{Feature Scales} & \textbf{Filter Size} & \textbf{Filter Nbr.} & \textbf{Train}  & \textbf{Val.} & \textbf{Test} & \textbf{Loss Type} & \textbf{LR} & \textbf{Batch} & \textbf{Epochs} \\
      \midrule
      \multicolumn{2}{c}{\textbf{Single Bilateral Filter Applications}} & & & & & & & & & \\
      \textbf{2$\times$ Color Upsampling} & Position$_{1}$, Intensity (3D) & 0.13, 0.17 & 65 & 2 & 10581 & 1449 & 1456 & MSE & 1e-06 & 200 & 94.5\\
      \textbf{4$\times$ Color Upsampling} & Position$_{1}$, Intensity (3D) & 0.06, 0.17 & 65 & 2 & 10581 & 1449 & 1456 & MSE & 1e-06 & 200 & 94.5\\
      \textbf{8$\times$ Color Upsampling} & Position$_{1}$, Intensity (3D) & 0.03, 0.17 & 65 & 2 & 10581 & 1449 & 1456 & MSE & 1e-06 & 200 & 94.5\\
      \textbf{16$\times$ Color Upsampling} & Position$_{1}$, Intensity (3D) & 0.02, 0.17 & 65 & 2 & 10581 & 1449 & 1456 & MSE & 1e-06 & 200 & 94.5\\
      \textbf{Depth Upsampling} & Position$_{1}$, Color (5D) & 0.05, 0.02 & 665 & 2 & 795 & 100 & 654 & MSE & 1e-07 & 50 & 251.6\\
      \textbf{Mesh Denoising} & Isomap (4D) & 46.00 & 63 & 2 & 1000 & 200 & 500 & MSE & 100 & 10 & 100.0 \\
      \midrule
      \multicolumn{2}{c}{\textbf{DenseCRF Applications}} & & & & & & & & &\\
      \multicolumn{2}{l}{\textbf{Semantic Segmentation}} & & & & & & & & &\\
      \textbf{- 1step MF} & Position$_{1}$, Color (5D); Position$_{1}$ (2D) & 0.01, 0.34; 0.34  & 665; 19  & 2; 2 & 10581 & 1449 & 1456 & Logistic & 0.1 & 5 & 1.4 \\
      \textbf{- 2step MF} & Position$_{1}$, Color (5D); Position$_{1}$ (2D) & 0.01, 0.34; 0.34 & 665; 19 & 2; 2 & 10581 & 1449 & 1456 & Logistic & 0.1 & 5 & 1.4 \\
      \textbf{- \textit{loose} 2step MF} & Position$_{1}$, Color (5D); Position$_{1}$ (2D) & 0.01, 0.34; 0.34 & 665; 19 & 2; 2 &10581 & 1449 & 1456 & Logistic & 0.1 & 5 & +1.9  \\ \\
      \multicolumn{2}{l}{\textbf{Material Segmentation}} & & & & & & & & &\\
      \textbf{- 1step MF} & Position$_{2}$, Lab-Color (5D) & 5.00, 0.05, 0.30  & 665 & 2 & 928 & 150 & 1798 & Weighted Logistic & 1e-04 & 24 & 2.6 \\
      \textbf{- 2step MF} & Position$_{2}$, Lab-Color (5D) & 5.00, 0.05, 0.30 & 665 & 2 & 928 & 150 & 1798 & Weighted Logistic & 1e-04 & 12 & +0.7 \\
      \textbf{- \textit{loose} 2step MF} & Position$_{2}$, Lab-Color (5D) & 5.00, 0.05, 0.30 & 665 & 2 & 928 & 150 & 1798 & Weighted Logistic & 1e-04 & 12 & +0.2\\
      \midrule
      \multicolumn{2}{c}{\textbf{Neural Network Applications}} & & & & & & & & &\\
      \textbf{Tiles: CNN-9$\times$9} & - & - & 81 & 4 & 10000 & 1000 & 1000 & Logistic & 0.01 & 100 & 500.0 \\
      \textbf{Tiles: CNN-13$\times$13} & - & - & 169 & 6 & 10000 & 1000 & 1000 & Logistic & 0.01 & 100 & 500.0 \\
      \textbf{Tiles: CNN-17$\times$17} & - & - & 289 & 8 & 10000 & 1000 & 1000 & Logistic & 0.01 & 100 & 500.0 \\
      \textbf{Tiles: CNN-21$\times$21} & - & - & 441 & 10 & 10000 & 1000 & 1000 & Logistic & 0.01 & 100 & 500.0 \\
      \textbf{Tiles: BNN} & Position$_{1}$, Color (5D) & 0.05, 0.04 & 63 & 1 & 10000 & 1000 & 1000 & Logistic & 0.01 & 100 & 30.0 \\
      \textbf{LeNet} & - & - & 25 & 2 & 5490 & 1098 & 1647 & Logistic & 0.1 & 100 & 182.2 \\
      \textbf{Crop-LeNet} & - & - & 25 & 2 & 5490 & 1098 & 1647 & Logistic & 0.1 & 100 & 182.2 \\
      \textbf{BNN-LeNet} & Position$_{2}$ (2D) & 20.00 & 7 & 1 & 5490 & 1098 & 1647 & Logistic & 0.1 & 100 & 182.2 \\
      \textbf{DeepCNet} & - & - & 9 & 1 & 5490 & 1098 & 1647 & Logistic & 0.1 & 100 & 182.2 \\
      \textbf{Crop-DeepCNet} & - & - & 9 & 1 & 5490 & 1098 & 1647 & Logistic & 0.1 & 100 & 182.2 \\
      \textbf{BNN-DeepCNet} & Position$_{2}$ (2D) & 40.00  & 7 & 1 & 5490 & 1098 & 1647 & Logistic & 0.1 & 100 & 182.2 \\
      \bottomrule
      \\
    \end{tabular}
    \mycaption{Experiment Protocols} {Experiment protocols for the different experiments presented in this work. \textbf{Feature Types}:
    Feature spaces used for the bilateral convolutions. Position$_1$ corresponds to un-normalized pixel positions whereas Position$_2$ corresponds
    to pixel positions normalized to $[0,1]$ with respect to the given image. \textbf{Feature Scales}: Cross-validated scales for the features used.
     \textbf{Filter Size}: Number of elements in the filter that is being learned. \textbf{Filter Nbr.}: Half-width of the filter. \textbf{Train},
     \textbf{Val.} and \textbf{Test} corresponds to the number of train, validation and test images used in the experiment. \textbf{Loss Type}: Type
     of loss used for back-propagation. ``MSE'' corresponds to Euclidean mean squared error loss and ``Logistic'' corresponds to multinomial logistic
     loss. ``Weighted Logistic'' is the class-weighted multinomial logistic loss. We weighted the loss with inverse class probability for material
     segmentation task due to the small availability of training data with class imbalance. \textbf{LR}: Fixed learning rate used in stochastic gradient
     descent. \textbf{Batch}: Number of images used in one parameter update step. \textbf{Epochs}: Number of training epochs. In all the experiments,
     we used fixed momentum of 0.9 and weight decay of 0.0005 for stochastic gradient descent. ```Color Upsampling'' experiments in this Table corresponds
     to those performed on Pascal VOC12 dataset images. For all experiments using Pascal VOC12 images, we use extended
     training segmentation dataset available from~\cite{hariharan2011moredata}, and used standard validation and test splits
     from the main dataset~\cite{voc2012segmentation}.}
  \label{tbl:parameters}
\end{table*}

\clearpage

\section{Parameters and Additional Results for Video Propagation Networks}

In this Section, we present experiment protocols and additional qualitative results for experiments
on video object segmentation, semantic video segmentation and video color
propagation. Table~\ref{tbl:parameters_supp} shows the feature scales and other parameters used in different experiments.
Figures~\ref{fig:video_seg_pos_supp} show some qualitative results on video object segmentation
with some failure cases in Fig.~\ref{fig:video_seg_neg_supp}.
Figure~\ref{fig:semantic_visuals_supp} shows some qualitative results on semantic video segmentation and
Fig.~\ref{fig:color_visuals_supp} shows results on video color propagation.

\newcolumntype{L}[1]{>{\raggedright\let\newline\\\arraybackslash\hspace{0pt}}b{#1}}
\newcolumntype{C}[1]{>{\centering\let\newline\\\arraybackslash\hspace{0pt}}b{#1}}
\newcolumntype{R}[1]{>{\raggedleft\let\newline\\\arraybackslash\hspace{0pt}}b{#1}}

\begin{table*}[h]
\tiny
  \centering
    \begin{tabular}{L{3.0cm} L{2.4cm} L{2.8cm} L{2.8cm} C{0.5cm} C{1.0cm} L{1.2cm}}
      \toprule
\textbf{Experiment} & \textbf{Feature Type} & \textbf{Feature Scale-1, $\Lambda_a$} & \textbf{Feature Scale-2, $\Lambda_b$} & \textbf{$\alpha$} & \textbf{Input Frames} & \textbf{Loss Type} \\
      \midrule
      \textbf{Video Object Segmentation} & ($x,y,Y,Cb,Cr,t$) & (0.02,0.02,0.07,0.4,0.4,0.01) & (0.03,0.03,0.09,0.5,0.5,0.2) & 0.5 & 9 & Logistic\\
      \midrule
      \textbf{Semantic Video Segmentation} & & & & & \\
      \textbf{with CNN1~\cite{yu2015multi}-NoFlow} & ($x,y,R,G,B,t$) & (0.08,0.08,0.2,0.2,0.2,0.04) & (0.11,0.11,0.2,0.2,0.2,0.04) & 0.5 & 3 & Logistic \\
      \textbf{with CNN1~\cite{yu2015multi}-Flow} & ($x+u_x,y+u_y,R,G,B,t$) & (0.11,0.11,0.14,0.14,0.14,0.03) & (0.08,0.08,0.12,0.12,0.12,0.01) & 0.65 & 3 & Logistic\\
      \textbf{with CNN2~\cite{richter2016playing}-Flow} & ($x+u_x,y+u_y,R,G,B,t$) & (0.08,0.08,0.2,0.2,0.2,0.04) & (0.09,0.09,0.25,0.25,0.25,0.03) & 0.5 & 4 & Logistic\\
      \midrule
      \textbf{Video Color Propagation} & ($x,y,I,t$)  & (0.04,0.04,0.2,0.04) & No second kernel & 1 & 4 & MSE\\
      \bottomrule
      \\
    \end{tabular}
    \mycaption{Experiment Protocols} {Experiment protocols for the different experiments presented in this work. \textbf{Feature Types}:
    Feature spaces used for the bilateral convolutions, with position ($x,y$) and color
    ($R,G,B$ or $Y,Cb,Cr$) features $\in [0,255]$. $u_x$, $u_y$ denotes optical flow with respect
    to the present frame and $I$ denotes grayscale intensity.
    \textbf{Feature Scales ($\Lambda_a, \Lambda_b$)}: Cross-validated scales for the features used.
    \textbf{$\alpha$}: Exponential time decay for the input frames.
    \textbf{Input Frames}: Number of input frames for VPN.
    \textbf{Loss Type}: Type
     of loss used for back-propagation. ``MSE'' corresponds to Euclidean mean squared error loss and ``Logistic'' corresponds to multinomial logistic loss.}
  \label{tbl:parameters_supp}
\end{table*}

% \begin{figure}[th!]
% \begin{center}
%   \centerline{\includegraphics[width=\textwidth]{figures/video_seg_visuals_supp_small.pdf}}
%     \mycaption{Video Object Segmentation}
%     {Shown are the different frames in example videos with the corresponding
%     ground truth (GT) masks, predictions from BVS~\cite{marki2016bilateral},
%     OFL~\cite{tsaivideo}, VPN (VPN-Stage2) and VPN-DLab (VPN-DeepLab) models.}
%     \label{fig:video_seg_small_supp}
% \end{center}
% \vspace{-1.0cm}
% \end{figure}

\begin{figure}[th!]
\begin{center}
  \centerline{\includegraphics[width=0.7\textwidth]{figures/video_seg_visuals_supp_positive.pdf}}
    \mycaption{Video Object Segmentation}
    {Shown are the different frames in example videos with the corresponding
    ground truth (GT) masks, predictions from BVS~\cite{marki2016bilateral},
    OFL~\cite{tsaivideo}, VPN (VPN-Stage2) and VPN-DLab (VPN-DeepLab) models.}
    \label{fig:video_seg_pos_supp}
\end{center}
\vspace{-1.0cm}
\end{figure}

\begin{figure}[th!]
\begin{center}
  \centerline{\includegraphics[width=0.7\textwidth]{figures/video_seg_visuals_supp_negative.pdf}}
    \mycaption{Failure Cases for Video Object Segmentation}
    {Shown are the different frames in example videos with the corresponding
    ground truth (GT) masks, predictions from BVS~\cite{marki2016bilateral},
    OFL~\cite{tsaivideo}, VPN (VPN-Stage2) and VPN-DLab (VPN-DeepLab) models.}
    \label{fig:video_seg_neg_supp}
\end{center}
\vspace{-1.0cm}
\end{figure}

\begin{figure}[th!]
\begin{center}
  \centerline{\includegraphics[width=0.9\textwidth]{figures/supp_semantic_visual.pdf}}
    \mycaption{Semantic Video Segmentation}
    {Input video frames and the corresponding ground truth (GT)
    segmentation together with the predictions of CNN~\cite{yu2015multi} and with
    VPN-Flow.}
    \label{fig:semantic_visuals_supp}
\end{center}
\vspace{-0.7cm}
\end{figure}

\begin{figure}[th!]
\begin{center}
  \centerline{\includegraphics[width=\textwidth]{figures/colorization_visuals_supp.pdf}}
  \mycaption{Video Color Propagation}
  {Input grayscale video frames and corresponding ground-truth (GT) color images
  together with color predictions of Levin et al.~\cite{levin2004colorization} and VPN-Stage1 models.}
  \label{fig:color_visuals_supp}
\end{center}
\vspace{-0.7cm}
\end{figure}

\clearpage

\section{Additional Material for Bilateral Inception Networks}
\label{sec:binception-app}

In this section of the Appendix, we first discuss the use of approximate bilateral
filtering in BI modules (Sec.~\ref{sec:lattice}).
Later, we present some qualitative results using different models for the approach presented in
Chapter~\ref{chap:binception} (Sec.~\ref{sec:qualitative-app}).

\subsection{Approximate Bilateral Filtering}
\label{sec:lattice}

The bilateral inception module presented in Chapter~\ref{chap:binception} computes a matrix-vector
product between a Gaussian filter $K$ and a vector of activations $\bz_c$.
Bilateral filtering is an important operation and many algorithmic techniques have been
proposed to speed-up this operation~\cite{paris2006fast,adams2010fast,gastal2011domain}.
In the main paper we opted to implement what can be considered the
brute-force variant of explicitly constructing $K$ and then using BLAS to compute the
matrix-vector product. This resulted in a few millisecond operation.
The explicit way to compute is possible due to the
reduction to super-pixels, e.g., it would not work for DenseCRF variants
that operate on the full image resolution.

Here, we present experiments where we use the fast approximate bilateral filtering
algorithm of~\cite{adams2010fast}, which is also used in Chapter~\ref{chap:bnn}
for learning sparse high dimensional filters. This
choice allows for larger dimensions of matrix-vector multiplication. The reason for choosing
the explicit multiplication in Chapter~\ref{chap:binception} was that it was computationally faster.
For the small sizes of the involved matrices and vectors, the explicit computation is sufficient and we had no
GPU implementation of an approximate technique that matched this runtime. Also it
is conceptually easier and the gradient to the feature transformations ($\Lambda \mathbf{f}$) is
obtained using standard matrix calculus.

\subsubsection{Experiments}

We modified the existing segmentation architectures analogous to those in Chapter~\ref{chap:binception}.
The main difference is that, here, the inception modules use the lattice
approximation~\cite{adams2010fast} to compute the bilateral filtering.
Using the lattice approximation did not allow us to back-propagate through feature transformations ($\Lambda$)
and thus we used hand-specified feature scales as will be explained later.
Specifically, we take CNN architectures from the works
of~\cite{chen2014semantic,zheng2015conditional,bell2015minc} and insert the BI modules between
the spatial FC layers.
We use superpixels from~\cite{DollarICCV13edges}
for all the experiments with the lattice approximation. Experiments are
performed using Caffe neural network framework~\cite{jia2014caffe}.

\begin{table}
  \small
  \centering
  \begin{tabular}{p{5.5cm}>{\raggedright\arraybackslash}p{1.4cm}>{\centering\arraybackslash}p{2.2cm}}
    \toprule
		\textbf{Model} & \emph{IoU} & \emph{Runtime}(ms) \\
    \midrule

    %%%%%%%%%%%% Scores computed by us)%%%%%%%%%%%%
		\deeplablargefov & 68.9 & 145ms\\
    \midrule
    \bi{7}{2}-\bi{8}{10}& \textbf{73.8} & +600 \\
    \midrule
    \deeplablargefovcrf~\cite{chen2014semantic} & 72.7 & +830\\
    \deeplabmsclargefovcrf~\cite{chen2014semantic} & \textbf{73.6} & +880\\
    DeepLab-EdgeNet~\cite{chen2015semantic} & 71.7 & +30\\
    DeepLab-EdgeNet-CRF~\cite{chen2015semantic} & \textbf{73.6} & +860\\
  \bottomrule \\
  \end{tabular}
  \mycaption{Semantic Segmentation using the DeepLab model}
  {IoU scores on the Pascal VOC12 segmentation test dataset
  with different models and our modified inception model.
  Also shown are the corresponding runtimes in milliseconds. Runtimes
  also include superpixel computations (300 ms with Dollar superpixels~\cite{DollarICCV13edges})}
  \label{tab:largefovresults}
\end{table}

\paragraph{Semantic Segmentation}
The experiments in this section use the Pascal VOC12 segmentation dataset~\cite{voc2012segmentation} with 21 object classes and the images have a maximum resolution of 0.25 megapixels.
For all experiments on VOC12, we train using the extended training set of
10581 images collected by~\cite{hariharan2011moredata}.
We modified the \deeplab~network architecture of~\cite{chen2014semantic} and
the CRFasRNN architecture from~\cite{zheng2015conditional} which uses a CNN with
deconvolution layers followed by DenseCRF trained end-to-end.

\paragraph{DeepLab Model}\label{sec:deeplabmodel}
We experimented with the \bi{7}{2}-\bi{8}{10} inception model.
Results using the~\deeplab~model are summarized in Tab.~\ref{tab:largefovresults}.
Although we get similar improvements with inception modules as with the
explicit kernel computation, using lattice approximation is slower.

\begin{table}
  \small
  \centering
  \begin{tabular}{p{6.4cm}>{\raggedright\arraybackslash}p{1.8cm}>{\raggedright\arraybackslash}p{1.8cm}}
    \toprule
    \textbf{Model} & \emph{IoU (Val)} & \emph{IoU (Test)}\\
    \midrule
    %%%%%%%%%%%% Scores computed by us)%%%%%%%%%%%%
    CNN &  67.5 & - \\
    \deconv (CNN+Deconvolutions) & 69.8 & 72.0 \\
    \midrule
    \bi{3}{6}-\bi{4}{6}-\bi{7}{2}-\bi{8}{6}& 71.9 & - \\
    \bi{3}{6}-\bi{4}{6}-\bi{7}{2}-\bi{8}{6}-\gi{6}& 73.6 &  \href{http://host.robots.ox.ac.uk:8080/anonymous/VOTV5E.html}{\textbf{75.2}}\\
    \midrule
    \deconvcrf (CRF-RNN)~\cite{zheng2015conditional} & 73.0 & 74.7\\
    Context-CRF-RNN~\cite{yu2015multi} & ~~ - ~ & \textbf{75.3} \\
    \bottomrule \\
  \end{tabular}
  \mycaption{Semantic Segmentation using the CRFasRNN model}{IoU score corresponding to different models
  on Pascal VOC12 reduced validation / test segmentation dataset. The reduced validation set consists of 346 images
  as used in~\cite{zheng2015conditional} where we adapted the model from.}
  \label{tab:deconvresults-app}
\end{table}

\paragraph{CRFasRNN Model}\label{sec:deepinception}
We add BI modules after score-pool3, score-pool4, \fc{7} and \fc{8} $1\times1$ convolution layers
resulting in the \bi{3}{6}-\bi{4}{6}-\bi{7}{2}-\bi{8}{6}
model and also experimented with another variant where $BI_8$ is followed by another inception
module, G$(6)$, with 6 Gaussian kernels.
Note that here also we discarded both deconvolution and DenseCRF parts of the original model~\cite{zheng2015conditional}
and inserted the BI modules in the base CNN and found similar improvements compared to the inception modules with explicit
kernel computaion. See Tab.~\ref{tab:deconvresults-app} for results on the CRFasRNN model.

\paragraph{Material Segmentation}
Table~\ref{tab:mincresults-app} shows the results on the MINC dataset~\cite{bell2015minc}
obtained by modifying the AlexNet architecture with our inception modules. We observe
similar improvements as with explicit kernel construction.
For this model, we do not provide any learned setup due to very limited segment training
data. The weights to combine outputs in the bilateral inception layer are
found by validation on the validation set.

\begin{table}[t]
  \small
  \centering
  \begin{tabular}{p{3.5cm}>{\centering\arraybackslash}p{4.0cm}}
    \toprule
    \textbf{Model} & Class / Total accuracy\\
    \midrule

    %%%%%%%%%%%% Scores computed by us)%%%%%%%%%%%%
    AlexNet CNN & 55.3 / 58.9 \\
    \midrule
    \bi{7}{2}-\bi{8}{6}& 68.5 / 71.8 \\
    \bi{7}{2}-\bi{8}{6}-G$(6)$& 67.6 / 73.1 \\
    \midrule
    AlexNet-CRF & 65.5 / 71.0 \\
    \bottomrule \\
  \end{tabular}
  \mycaption{Material Segmentation using AlexNet}{Pixel accuracy of different models on
  the MINC material segmentation test dataset~\cite{bell2015minc}.}
  \label{tab:mincresults-app}
\end{table}

\paragraph{Scales of Bilateral Inception Modules}
\label{sec:scales}

Unlike the explicit kernel technique presented in the main text (Chapter~\ref{chap:binception}),
we didn't back-propagate through feature transformation ($\Lambda$)
using the approximate bilateral filter technique.
So, the feature scales are hand-specified and validated, which are as follows.
The optimal scale values for the \bi{7}{2}-\bi{8}{2} model are found by validation for the best performance which are
$\sigma_{xy}$ = (0.1, 0.1) for the spatial (XY) kernel and $\sigma_{rgbxy}$ = (0.1, 0.1, 0.1, 0.01, 0.01) for color and position (RGBXY)  kernel.
Next, as more kernels are added to \bi{8}{2}, we set scales to be $\alpha$*($\sigma_{xy}$, $\sigma_{rgbxy}$).
The value of $\alpha$ is chosen as  1, 0.5, 0.1, 0.05, 0.1, at uniform interval, for the \bi{8}{10} bilateral inception module.


\subsection{Qualitative Results}
\label{sec:qualitative-app}

In this section, we present more qualitative results obtained using the BI module with explicit
kernel computation technique presented in Chapter~\ref{chap:binception}. Results on the Pascal VOC12
dataset~\cite{voc2012segmentation} using the DeepLab-LargeFOV model are shown in Fig.~\ref{fig:semantic_visuals-app},
followed by the results on MINC dataset~\cite{bell2015minc}
in Fig.~\ref{fig:material_visuals-app} and on
Cityscapes dataset~\cite{Cordts2015Cvprw} in Fig.~\ref{fig:street_visuals-app}.


\definecolor{voc_1}{RGB}{0, 0, 0}
\definecolor{voc_2}{RGB}{128, 0, 0}
\definecolor{voc_3}{RGB}{0, 128, 0}
\definecolor{voc_4}{RGB}{128, 128, 0}
\definecolor{voc_5}{RGB}{0, 0, 128}
\definecolor{voc_6}{RGB}{128, 0, 128}
\definecolor{voc_7}{RGB}{0, 128, 128}
\definecolor{voc_8}{RGB}{128, 128, 128}
\definecolor{voc_9}{RGB}{64, 0, 0}
\definecolor{voc_10}{RGB}{192, 0, 0}
\definecolor{voc_11}{RGB}{64, 128, 0}
\definecolor{voc_12}{RGB}{192, 128, 0}
\definecolor{voc_13}{RGB}{64, 0, 128}
\definecolor{voc_14}{RGB}{192, 0, 128}
\definecolor{voc_15}{RGB}{64, 128, 128}
\definecolor{voc_16}{RGB}{192, 128, 128}
\definecolor{voc_17}{RGB}{0, 64, 0}
\definecolor{voc_18}{RGB}{128, 64, 0}
\definecolor{voc_19}{RGB}{0, 192, 0}
\definecolor{voc_20}{RGB}{128, 192, 0}
\definecolor{voc_21}{RGB}{0, 64, 128}
\definecolor{voc_22}{RGB}{128, 64, 128}

\begin{figure*}[!ht]
  \small
  \centering
  \fcolorbox{white}{voc_1}{\rule{0pt}{4pt}\rule{4pt}{0pt}} Background~~
  \fcolorbox{white}{voc_2}{\rule{0pt}{4pt}\rule{4pt}{0pt}} Aeroplane~~
  \fcolorbox{white}{voc_3}{\rule{0pt}{4pt}\rule{4pt}{0pt}} Bicycle~~
  \fcolorbox{white}{voc_4}{\rule{0pt}{4pt}\rule{4pt}{0pt}} Bird~~
  \fcolorbox{white}{voc_5}{\rule{0pt}{4pt}\rule{4pt}{0pt}} Boat~~
  \fcolorbox{white}{voc_6}{\rule{0pt}{4pt}\rule{4pt}{0pt}} Bottle~~
  \fcolorbox{white}{voc_7}{\rule{0pt}{4pt}\rule{4pt}{0pt}} Bus~~
  \fcolorbox{white}{voc_8}{\rule{0pt}{4pt}\rule{4pt}{0pt}} Car~~\\
  \fcolorbox{white}{voc_9}{\rule{0pt}{4pt}\rule{4pt}{0pt}} Cat~~
  \fcolorbox{white}{voc_10}{\rule{0pt}{4pt}\rule{4pt}{0pt}} Chair~~
  \fcolorbox{white}{voc_11}{\rule{0pt}{4pt}\rule{4pt}{0pt}} Cow~~
  \fcolorbox{white}{voc_12}{\rule{0pt}{4pt}\rule{4pt}{0pt}} Dining Table~~
  \fcolorbox{white}{voc_13}{\rule{0pt}{4pt}\rule{4pt}{0pt}} Dog~~
  \fcolorbox{white}{voc_14}{\rule{0pt}{4pt}\rule{4pt}{0pt}} Horse~~
  \fcolorbox{white}{voc_15}{\rule{0pt}{4pt}\rule{4pt}{0pt}} Motorbike~~
  \fcolorbox{white}{voc_16}{\rule{0pt}{4pt}\rule{4pt}{0pt}} Person~~\\
  \fcolorbox{white}{voc_17}{\rule{0pt}{4pt}\rule{4pt}{0pt}} Potted Plant~~
  \fcolorbox{white}{voc_18}{\rule{0pt}{4pt}\rule{4pt}{0pt}} Sheep~~
  \fcolorbox{white}{voc_19}{\rule{0pt}{4pt}\rule{4pt}{0pt}} Sofa~~
  \fcolorbox{white}{voc_20}{\rule{0pt}{4pt}\rule{4pt}{0pt}} Train~~
  \fcolorbox{white}{voc_21}{\rule{0pt}{4pt}\rule{4pt}{0pt}} TV monitor~~\\


  \subfigure{%
    \includegraphics[width=.15\columnwidth]{figures/supplementary/2008_001308_given.png}
  }
  \subfigure{%
    \includegraphics[width=.15\columnwidth]{figures/supplementary/2008_001308_sp.png}
  }
  \subfigure{%
    \includegraphics[width=.15\columnwidth]{figures/supplementary/2008_001308_gt.png}
  }
  \subfigure{%
    \includegraphics[width=.15\columnwidth]{figures/supplementary/2008_001308_cnn.png}
  }
  \subfigure{%
    \includegraphics[width=.15\columnwidth]{figures/supplementary/2008_001308_crf.png}
  }
  \subfigure{%
    \includegraphics[width=.15\columnwidth]{figures/supplementary/2008_001308_ours.png}
  }\\[-2ex]


  \subfigure{%
    \includegraphics[width=.15\columnwidth]{figures/supplementary/2008_001821_given.png}
  }
  \subfigure{%
    \includegraphics[width=.15\columnwidth]{figures/supplementary/2008_001821_sp.png}
  }
  \subfigure{%
    \includegraphics[width=.15\columnwidth]{figures/supplementary/2008_001821_gt.png}
  }
  \subfigure{%
    \includegraphics[width=.15\columnwidth]{figures/supplementary/2008_001821_cnn.png}
  }
  \subfigure{%
    \includegraphics[width=.15\columnwidth]{figures/supplementary/2008_001821_crf.png}
  }
  \subfigure{%
    \includegraphics[width=.15\columnwidth]{figures/supplementary/2008_001821_ours.png}
  }\\[-2ex]



  \subfigure{%
    \includegraphics[width=.15\columnwidth]{figures/supplementary/2008_004612_given.png}
  }
  \subfigure{%
    \includegraphics[width=.15\columnwidth]{figures/supplementary/2008_004612_sp.png}
  }
  \subfigure{%
    \includegraphics[width=.15\columnwidth]{figures/supplementary/2008_004612_gt.png}
  }
  \subfigure{%
    \includegraphics[width=.15\columnwidth]{figures/supplementary/2008_004612_cnn.png}
  }
  \subfigure{%
    \includegraphics[width=.15\columnwidth]{figures/supplementary/2008_004612_crf.png}
  }
  \subfigure{%
    \includegraphics[width=.15\columnwidth]{figures/supplementary/2008_004612_ours.png}
  }\\[-2ex]


  \subfigure{%
    \includegraphics[width=.15\columnwidth]{figures/supplementary/2009_001008_given.png}
  }
  \subfigure{%
    \includegraphics[width=.15\columnwidth]{figures/supplementary/2009_001008_sp.png}
  }
  \subfigure{%
    \includegraphics[width=.15\columnwidth]{figures/supplementary/2009_001008_gt.png}
  }
  \subfigure{%
    \includegraphics[width=.15\columnwidth]{figures/supplementary/2009_001008_cnn.png}
  }
  \subfigure{%
    \includegraphics[width=.15\columnwidth]{figures/supplementary/2009_001008_crf.png}
  }
  \subfigure{%
    \includegraphics[width=.15\columnwidth]{figures/supplementary/2009_001008_ours.png}
  }\\[-2ex]




  \subfigure{%
    \includegraphics[width=.15\columnwidth]{figures/supplementary/2009_004497_given.png}
  }
  \subfigure{%
    \includegraphics[width=.15\columnwidth]{figures/supplementary/2009_004497_sp.png}
  }
  \subfigure{%
    \includegraphics[width=.15\columnwidth]{figures/supplementary/2009_004497_gt.png}
  }
  \subfigure{%
    \includegraphics[width=.15\columnwidth]{figures/supplementary/2009_004497_cnn.png}
  }
  \subfigure{%
    \includegraphics[width=.15\columnwidth]{figures/supplementary/2009_004497_crf.png}
  }
  \subfigure{%
    \includegraphics[width=.15\columnwidth]{figures/supplementary/2009_004497_ours.png}
  }\\[-2ex]



  \setcounter{subfigure}{0}
  \subfigure[\scriptsize Input]{%
    \includegraphics[width=.15\columnwidth]{figures/supplementary/2010_001327_given.png}
  }
  \subfigure[\scriptsize Superpixels]{%
    \includegraphics[width=.15\columnwidth]{figures/supplementary/2010_001327_sp.png}
  }
  \subfigure[\scriptsize GT]{%
    \includegraphics[width=.15\columnwidth]{figures/supplementary/2010_001327_gt.png}
  }
  \subfigure[\scriptsize Deeplab]{%
    \includegraphics[width=.15\columnwidth]{figures/supplementary/2010_001327_cnn.png}
  }
  \subfigure[\scriptsize +DenseCRF]{%
    \includegraphics[width=.15\columnwidth]{figures/supplementary/2010_001327_crf.png}
  }
  \subfigure[\scriptsize Using BI]{%
    \includegraphics[width=.15\columnwidth]{figures/supplementary/2010_001327_ours.png}
  }
  \mycaption{Semantic Segmentation}{Example results of semantic segmentation
  on the Pascal VOC12 dataset.
  (d)~depicts the DeepLab CNN result, (e)~CNN + 10 steps of mean-field inference,
  (f~result obtained with bilateral inception (BI) modules (\bi{6}{2}+\bi{7}{6}) between \fc~layers.}
  \label{fig:semantic_visuals-app}
\end{figure*}


\definecolor{minc_1}{HTML}{771111}
\definecolor{minc_2}{HTML}{CAC690}
\definecolor{minc_3}{HTML}{EEEEEE}
\definecolor{minc_4}{HTML}{7C8FA6}
\definecolor{minc_5}{HTML}{597D31}
\definecolor{minc_6}{HTML}{104410}
\definecolor{minc_7}{HTML}{BB819C}
\definecolor{minc_8}{HTML}{D0CE48}
\definecolor{minc_9}{HTML}{622745}
\definecolor{minc_10}{HTML}{666666}
\definecolor{minc_11}{HTML}{D54A31}
\definecolor{minc_12}{HTML}{101044}
\definecolor{minc_13}{HTML}{444126}
\definecolor{minc_14}{HTML}{75D646}
\definecolor{minc_15}{HTML}{DD4348}
\definecolor{minc_16}{HTML}{5C8577}
\definecolor{minc_17}{HTML}{C78472}
\definecolor{minc_18}{HTML}{75D6D0}
\definecolor{minc_19}{HTML}{5B4586}
\definecolor{minc_20}{HTML}{C04393}
\definecolor{minc_21}{HTML}{D69948}
\definecolor{minc_22}{HTML}{7370D8}
\definecolor{minc_23}{HTML}{7A3622}
\definecolor{minc_24}{HTML}{000000}

\begin{figure*}[!ht]
  \small % scriptsize
  \centering
  \fcolorbox{white}{minc_1}{\rule{0pt}{4pt}\rule{4pt}{0pt}} Brick~~
  \fcolorbox{white}{minc_2}{\rule{0pt}{4pt}\rule{4pt}{0pt}} Carpet~~
  \fcolorbox{white}{minc_3}{\rule{0pt}{4pt}\rule{4pt}{0pt}} Ceramic~~
  \fcolorbox{white}{minc_4}{\rule{0pt}{4pt}\rule{4pt}{0pt}} Fabric~~
  \fcolorbox{white}{minc_5}{\rule{0pt}{4pt}\rule{4pt}{0pt}} Foliage~~
  \fcolorbox{white}{minc_6}{\rule{0pt}{4pt}\rule{4pt}{0pt}} Food~~
  \fcolorbox{white}{minc_7}{\rule{0pt}{4pt}\rule{4pt}{0pt}} Glass~~
  \fcolorbox{white}{minc_8}{\rule{0pt}{4pt}\rule{4pt}{0pt}} Hair~~\\
  \fcolorbox{white}{minc_9}{\rule{0pt}{4pt}\rule{4pt}{0pt}} Leather~~
  \fcolorbox{white}{minc_10}{\rule{0pt}{4pt}\rule{4pt}{0pt}} Metal~~
  \fcolorbox{white}{minc_11}{\rule{0pt}{4pt}\rule{4pt}{0pt}} Mirror~~
  \fcolorbox{white}{minc_12}{\rule{0pt}{4pt}\rule{4pt}{0pt}} Other~~
  \fcolorbox{white}{minc_13}{\rule{0pt}{4pt}\rule{4pt}{0pt}} Painted~~
  \fcolorbox{white}{minc_14}{\rule{0pt}{4pt}\rule{4pt}{0pt}} Paper~~
  \fcolorbox{white}{minc_15}{\rule{0pt}{4pt}\rule{4pt}{0pt}} Plastic~~\\
  \fcolorbox{white}{minc_16}{\rule{0pt}{4pt}\rule{4pt}{0pt}} Polished Stone~~
  \fcolorbox{white}{minc_17}{\rule{0pt}{4pt}\rule{4pt}{0pt}} Skin~~
  \fcolorbox{white}{minc_18}{\rule{0pt}{4pt}\rule{4pt}{0pt}} Sky~~
  \fcolorbox{white}{minc_19}{\rule{0pt}{4pt}\rule{4pt}{0pt}} Stone~~
  \fcolorbox{white}{minc_20}{\rule{0pt}{4pt}\rule{4pt}{0pt}} Tile~~
  \fcolorbox{white}{minc_21}{\rule{0pt}{4pt}\rule{4pt}{0pt}} Wallpaper~~
  \fcolorbox{white}{minc_22}{\rule{0pt}{4pt}\rule{4pt}{0pt}} Water~~
  \fcolorbox{white}{minc_23}{\rule{0pt}{4pt}\rule{4pt}{0pt}} Wood~~\\
  \subfigure{%
    \includegraphics[width=.15\columnwidth]{figures/supplementary/000008468_given.png}
  }
  \subfigure{%
    \includegraphics[width=.15\columnwidth]{figures/supplementary/000008468_sp.png}
  }
  \subfigure{%
    \includegraphics[width=.15\columnwidth]{figures/supplementary/000008468_gt.png}
  }
  \subfigure{%
    \includegraphics[width=.15\columnwidth]{figures/supplementary/000008468_cnn.png}
  }
  \subfigure{%
    \includegraphics[width=.15\columnwidth]{figures/supplementary/000008468_crf.png}
  }
  \subfigure{%
    \includegraphics[width=.15\columnwidth]{figures/supplementary/000008468_ours.png}
  }\\[-2ex]

  \subfigure{%
    \includegraphics[width=.15\columnwidth]{figures/supplementary/000009053_given.png}
  }
  \subfigure{%
    \includegraphics[width=.15\columnwidth]{figures/supplementary/000009053_sp.png}
  }
  \subfigure{%
    \includegraphics[width=.15\columnwidth]{figures/supplementary/000009053_gt.png}
  }
  \subfigure{%
    \includegraphics[width=.15\columnwidth]{figures/supplementary/000009053_cnn.png}
  }
  \subfigure{%
    \includegraphics[width=.15\columnwidth]{figures/supplementary/000009053_crf.png}
  }
  \subfigure{%
    \includegraphics[width=.15\columnwidth]{figures/supplementary/000009053_ours.png}
  }\\[-2ex]




  \subfigure{%
    \includegraphics[width=.15\columnwidth]{figures/supplementary/000014977_given.png}
  }
  \subfigure{%
    \includegraphics[width=.15\columnwidth]{figures/supplementary/000014977_sp.png}
  }
  \subfigure{%
    \includegraphics[width=.15\columnwidth]{figures/supplementary/000014977_gt.png}
  }
  \subfigure{%
    \includegraphics[width=.15\columnwidth]{figures/supplementary/000014977_cnn.png}
  }
  \subfigure{%
    \includegraphics[width=.15\columnwidth]{figures/supplementary/000014977_crf.png}
  }
  \subfigure{%
    \includegraphics[width=.15\columnwidth]{figures/supplementary/000014977_ours.png}
  }\\[-2ex]


  \subfigure{%
    \includegraphics[width=.15\columnwidth]{figures/supplementary/000022922_given.png}
  }
  \subfigure{%
    \includegraphics[width=.15\columnwidth]{figures/supplementary/000022922_sp.png}
  }
  \subfigure{%
    \includegraphics[width=.15\columnwidth]{figures/supplementary/000022922_gt.png}
  }
  \subfigure{%
    \includegraphics[width=.15\columnwidth]{figures/supplementary/000022922_cnn.png}
  }
  \subfigure{%
    \includegraphics[width=.15\columnwidth]{figures/supplementary/000022922_crf.png}
  }
  \subfigure{%
    \includegraphics[width=.15\columnwidth]{figures/supplementary/000022922_ours.png}
  }\\[-2ex]


  \subfigure{%
    \includegraphics[width=.15\columnwidth]{figures/supplementary/000025711_given.png}
  }
  \subfigure{%
    \includegraphics[width=.15\columnwidth]{figures/supplementary/000025711_sp.png}
  }
  \subfigure{%
    \includegraphics[width=.15\columnwidth]{figures/supplementary/000025711_gt.png}
  }
  \subfigure{%
    \includegraphics[width=.15\columnwidth]{figures/supplementary/000025711_cnn.png}
  }
  \subfigure{%
    \includegraphics[width=.15\columnwidth]{figures/supplementary/000025711_crf.png}
  }
  \subfigure{%
    \includegraphics[width=.15\columnwidth]{figures/supplementary/000025711_ours.png}
  }\\[-2ex]


  \subfigure{%
    \includegraphics[width=.15\columnwidth]{figures/supplementary/000034473_given.png}
  }
  \subfigure{%
    \includegraphics[width=.15\columnwidth]{figures/supplementary/000034473_sp.png}
  }
  \subfigure{%
    \includegraphics[width=.15\columnwidth]{figures/supplementary/000034473_gt.png}
  }
  \subfigure{%
    \includegraphics[width=.15\columnwidth]{figures/supplementary/000034473_cnn.png}
  }
  \subfigure{%
    \includegraphics[width=.15\columnwidth]{figures/supplementary/000034473_crf.png}
  }
  \subfigure{%
    \includegraphics[width=.15\columnwidth]{figures/supplementary/000034473_ours.png}
  }\\[-2ex]


  \subfigure{%
    \includegraphics[width=.15\columnwidth]{figures/supplementary/000035463_given.png}
  }
  \subfigure{%
    \includegraphics[width=.15\columnwidth]{figures/supplementary/000035463_sp.png}
  }
  \subfigure{%
    \includegraphics[width=.15\columnwidth]{figures/supplementary/000035463_gt.png}
  }
  \subfigure{%
    \includegraphics[width=.15\columnwidth]{figures/supplementary/000035463_cnn.png}
  }
  \subfigure{%
    \includegraphics[width=.15\columnwidth]{figures/supplementary/000035463_crf.png}
  }
  \subfigure{%
    \includegraphics[width=.15\columnwidth]{figures/supplementary/000035463_ours.png}
  }\\[-2ex]


  \setcounter{subfigure}{0}
  \subfigure[\scriptsize Input]{%
    \includegraphics[width=.15\columnwidth]{figures/supplementary/000035993_given.png}
  }
  \subfigure[\scriptsize Superpixels]{%
    \includegraphics[width=.15\columnwidth]{figures/supplementary/000035993_sp.png}
  }
  \subfigure[\scriptsize GT]{%
    \includegraphics[width=.15\columnwidth]{figures/supplementary/000035993_gt.png}
  }
  \subfigure[\scriptsize AlexNet]{%
    \includegraphics[width=.15\columnwidth]{figures/supplementary/000035993_cnn.png}
  }
  \subfigure[\scriptsize +DenseCRF]{%
    \includegraphics[width=.15\columnwidth]{figures/supplementary/000035993_crf.png}
  }
  \subfigure[\scriptsize Using BI]{%
    \includegraphics[width=.15\columnwidth]{figures/supplementary/000035993_ours.png}
  }
  \mycaption{Material Segmentation}{Example results of material segmentation.
  (d)~depicts the AlexNet CNN result, (e)~CNN + 10 steps of mean-field inference,
  (f)~result obtained with bilateral inception (BI) modules (\bi{7}{2}+\bi{8}{6}) between
  \fc~layers.}
\label{fig:material_visuals-app}
\end{figure*}


\definecolor{city_1}{RGB}{128, 64, 128}
\definecolor{city_2}{RGB}{244, 35, 232}
\definecolor{city_3}{RGB}{70, 70, 70}
\definecolor{city_4}{RGB}{102, 102, 156}
\definecolor{city_5}{RGB}{190, 153, 153}
\definecolor{city_6}{RGB}{153, 153, 153}
\definecolor{city_7}{RGB}{250, 170, 30}
\definecolor{city_8}{RGB}{220, 220, 0}
\definecolor{city_9}{RGB}{107, 142, 35}
\definecolor{city_10}{RGB}{152, 251, 152}
\definecolor{city_11}{RGB}{70, 130, 180}
\definecolor{city_12}{RGB}{220, 20, 60}
\definecolor{city_13}{RGB}{255, 0, 0}
\definecolor{city_14}{RGB}{0, 0, 142}
\definecolor{city_15}{RGB}{0, 0, 70}
\definecolor{city_16}{RGB}{0, 60, 100}
\definecolor{city_17}{RGB}{0, 80, 100}
\definecolor{city_18}{RGB}{0, 0, 230}
\definecolor{city_19}{RGB}{119, 11, 32}
\begin{figure*}[!ht]
  \small % scriptsize
  \centering


  \subfigure{%
    \includegraphics[width=.18\columnwidth]{figures/supplementary/frankfurt00000_016005_given.png}
  }
  \subfigure{%
    \includegraphics[width=.18\columnwidth]{figures/supplementary/frankfurt00000_016005_sp.png}
  }
  \subfigure{%
    \includegraphics[width=.18\columnwidth]{figures/supplementary/frankfurt00000_016005_gt.png}
  }
  \subfigure{%
    \includegraphics[width=.18\columnwidth]{figures/supplementary/frankfurt00000_016005_cnn.png}
  }
  \subfigure{%
    \includegraphics[width=.18\columnwidth]{figures/supplementary/frankfurt00000_016005_ours.png}
  }\\[-2ex]

  \subfigure{%
    \includegraphics[width=.18\columnwidth]{figures/supplementary/frankfurt00000_004617_given.png}
  }
  \subfigure{%
    \includegraphics[width=.18\columnwidth]{figures/supplementary/frankfurt00000_004617_sp.png}
  }
  \subfigure{%
    \includegraphics[width=.18\columnwidth]{figures/supplementary/frankfurt00000_004617_gt.png}
  }
  \subfigure{%
    \includegraphics[width=.18\columnwidth]{figures/supplementary/frankfurt00000_004617_cnn.png}
  }
  \subfigure{%
    \includegraphics[width=.18\columnwidth]{figures/supplementary/frankfurt00000_004617_ours.png}
  }\\[-2ex]

  \subfigure{%
    \includegraphics[width=.18\columnwidth]{figures/supplementary/frankfurt00000_020880_given.png}
  }
  \subfigure{%
    \includegraphics[width=.18\columnwidth]{figures/supplementary/frankfurt00000_020880_sp.png}
  }
  \subfigure{%
    \includegraphics[width=.18\columnwidth]{figures/supplementary/frankfurt00000_020880_gt.png}
  }
  \subfigure{%
    \includegraphics[width=.18\columnwidth]{figures/supplementary/frankfurt00000_020880_cnn.png}
  }
  \subfigure{%
    \includegraphics[width=.18\columnwidth]{figures/supplementary/frankfurt00000_020880_ours.png}
  }\\[-2ex]



  \subfigure{%
    \includegraphics[width=.18\columnwidth]{figures/supplementary/frankfurt00001_007285_given.png}
  }
  \subfigure{%
    \includegraphics[width=.18\columnwidth]{figures/supplementary/frankfurt00001_007285_sp.png}
  }
  \subfigure{%
    \includegraphics[width=.18\columnwidth]{figures/supplementary/frankfurt00001_007285_gt.png}
  }
  \subfigure{%
    \includegraphics[width=.18\columnwidth]{figures/supplementary/frankfurt00001_007285_cnn.png}
  }
  \subfigure{%
    \includegraphics[width=.18\columnwidth]{figures/supplementary/frankfurt00001_007285_ours.png}
  }\\[-2ex]


  \subfigure{%
    \includegraphics[width=.18\columnwidth]{figures/supplementary/frankfurt00001_059789_given.png}
  }
  \subfigure{%
    \includegraphics[width=.18\columnwidth]{figures/supplementary/frankfurt00001_059789_sp.png}
  }
  \subfigure{%
    \includegraphics[width=.18\columnwidth]{figures/supplementary/frankfurt00001_059789_gt.png}
  }
  \subfigure{%
    \includegraphics[width=.18\columnwidth]{figures/supplementary/frankfurt00001_059789_cnn.png}
  }
  \subfigure{%
    \includegraphics[width=.18\columnwidth]{figures/supplementary/frankfurt00001_059789_ours.png}
  }\\[-2ex]


  \subfigure{%
    \includegraphics[width=.18\columnwidth]{figures/supplementary/frankfurt00001_068208_given.png}
  }
  \subfigure{%
    \includegraphics[width=.18\columnwidth]{figures/supplementary/frankfurt00001_068208_sp.png}
  }
  \subfigure{%
    \includegraphics[width=.18\columnwidth]{figures/supplementary/frankfurt00001_068208_gt.png}
  }
  \subfigure{%
    \includegraphics[width=.18\columnwidth]{figures/supplementary/frankfurt00001_068208_cnn.png}
  }
  \subfigure{%
    \includegraphics[width=.18\columnwidth]{figures/supplementary/frankfurt00001_068208_ours.png}
  }\\[-2ex]

  \subfigure{%
    \includegraphics[width=.18\columnwidth]{figures/supplementary/frankfurt00001_082466_given.png}
  }
  \subfigure{%
    \includegraphics[width=.18\columnwidth]{figures/supplementary/frankfurt00001_082466_sp.png}
  }
  \subfigure{%
    \includegraphics[width=.18\columnwidth]{figures/supplementary/frankfurt00001_082466_gt.png}
  }
  \subfigure{%
    \includegraphics[width=.18\columnwidth]{figures/supplementary/frankfurt00001_082466_cnn.png}
  }
  \subfigure{%
    \includegraphics[width=.18\columnwidth]{figures/supplementary/frankfurt00001_082466_ours.png}
  }\\[-2ex]

  \subfigure{%
    \includegraphics[width=.18\columnwidth]{figures/supplementary/lindau00033_000019_given.png}
  }
  \subfigure{%
    \includegraphics[width=.18\columnwidth]{figures/supplementary/lindau00033_000019_sp.png}
  }
  \subfigure{%
    \includegraphics[width=.18\columnwidth]{figures/supplementary/lindau00033_000019_gt.png}
  }
  \subfigure{%
    \includegraphics[width=.18\columnwidth]{figures/supplementary/lindau00033_000019_cnn.png}
  }
  \subfigure{%
    \includegraphics[width=.18\columnwidth]{figures/supplementary/lindau00033_000019_ours.png}
  }\\[-2ex]

  \subfigure{%
    \includegraphics[width=.18\columnwidth]{figures/supplementary/lindau00052_000019_given.png}
  }
  \subfigure{%
    \includegraphics[width=.18\columnwidth]{figures/supplementary/lindau00052_000019_sp.png}
  }
  \subfigure{%
    \includegraphics[width=.18\columnwidth]{figures/supplementary/lindau00052_000019_gt.png}
  }
  \subfigure{%
    \includegraphics[width=.18\columnwidth]{figures/supplementary/lindau00052_000019_cnn.png}
  }
  \subfigure{%
    \includegraphics[width=.18\columnwidth]{figures/supplementary/lindau00052_000019_ours.png}
  }\\[-2ex]




  \subfigure{%
    \includegraphics[width=.18\columnwidth]{figures/supplementary/lindau00027_000019_given.png}
  }
  \subfigure{%
    \includegraphics[width=.18\columnwidth]{figures/supplementary/lindau00027_000019_sp.png}
  }
  \subfigure{%
    \includegraphics[width=.18\columnwidth]{figures/supplementary/lindau00027_000019_gt.png}
  }
  \subfigure{%
    \includegraphics[width=.18\columnwidth]{figures/supplementary/lindau00027_000019_cnn.png}
  }
  \subfigure{%
    \includegraphics[width=.18\columnwidth]{figures/supplementary/lindau00027_000019_ours.png}
  }\\[-2ex]



  \setcounter{subfigure}{0}
  \subfigure[\scriptsize Input]{%
    \includegraphics[width=.18\columnwidth]{figures/supplementary/lindau00029_000019_given.png}
  }
  \subfigure[\scriptsize Superpixels]{%
    \includegraphics[width=.18\columnwidth]{figures/supplementary/lindau00029_000019_sp.png}
  }
  \subfigure[\scriptsize GT]{%
    \includegraphics[width=.18\columnwidth]{figures/supplementary/lindau00029_000019_gt.png}
  }
  \subfigure[\scriptsize Deeplab]{%
    \includegraphics[width=.18\columnwidth]{figures/supplementary/lindau00029_000019_cnn.png}
  }
  \subfigure[\scriptsize Using BI]{%
    \includegraphics[width=.18\columnwidth]{figures/supplementary/lindau00029_000019_ours.png}
  }%\\[-2ex]

  \mycaption{Street Scene Segmentation}{Example results of street scene segmentation.
  (d)~depicts the DeepLab results, (e)~result obtained by adding bilateral inception (BI) modules (\bi{6}{2}+\bi{7}{6}) between \fc~layers.}
\label{fig:street_visuals-app}
\end{figure*}


\end{document}

