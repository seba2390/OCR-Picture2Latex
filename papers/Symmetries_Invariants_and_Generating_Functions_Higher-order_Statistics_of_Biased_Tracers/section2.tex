%%%%%%%%%%%%%%%%%%%%%%%%%%%%%%%%%%%%%%%%%%%%%%%%%%%%%%%%%%%%%%%
\section{Generating Functions}
\label{sec:gen}
%%%%%%%%%%%%%%%%%%%%%%%%%%%%%%%%%%%%%%%%%%%%%%%%%%%%%%%%%%%%%%
%
The generating function formalism is often used to compute the cumulants and the CCs of cosmological fields.
Our aim here is to provide a very brief review of the generating function formalism developed in \citep{Ber92} 
(also see \citep{MSS}) to 
construct the CCs of the biased tracers (halos or peaks) and express them in terms of the statistics
of underlying mass distribution. The results are relevant for the perturbative regime.
The $n$-th order of perturbative expansion of an arbitrary field $F$ defined as
$F^{(n)}$ with respect to $\delta$ is defined as follows:
\ben
\label{eq:gen}
\la {F}^{(n)}\ra_c = 
{{\int \la {\rm F}^{(n)}({\bf x}, a)\delta^{(1)}({\bf x}_1,a)\cdots \delta^{(1)}({\bf x}_n,a)\ra_c\; d^3{\bf x}\,d^3{\bf x}_1 \cdots d^3{\bf x}_n}
\over 
({\int \la\delta^{(1)}({\bf x},a)\delta^{(1)}({\bf x^{\prime}},a) \ra 
d^3{\bf x} d^3{\bf x}^{\prime}})^n}.
\een
Here, $\delta^{(1)}({\bf x},a)$ is the
linear approximation for $\delta({\bf x},a)$ at a comoving position ${\bf x}$ and $a(t)$ is the scale factor of the Universe. Only
connected diagrams are taken into
account, which explains the subscript $c$. Throughout, we will assume that the
initial density contrast $\delta$ is Gaussian, though it is possible to incorporate non-Gaussian initial condition.
The generating function ${\cal G}_{F}(\tau_s)$ for the vertices for any random field $F({\bf x},a)$ is given by:
\ben
{\cal G}_{F}(\tau_s) = \sum^{\infty}_{n=1} {\la {F}^{(n)} \ra_c \over n!}\tau_s^n.
\een


For two arbitrary fields $A(\bx,a)$ and $B(\bx,a)$, we have the following properties 
for the generating functions \citep{Ber92}:
\bes
\ben
\label{eq:gen_1}
&& {\cal G}_{A+B}(\tau_s) = {\cal G}_A(\tau_s)+{\cal G}_B(\tau_s); \quad\\
&& {\cal G}_{AB}(\tau_s) = {\cal G}_A(\tau_s){\cal G}_B(\tau_s); \quad\\
\label{eq:gen_2}
&&{\cal G}_{\nabla_i A \; \nabla_i B}(\tau_s) = 0; \quad\\
\label{eq:gen_3}
&& {\cal G}_{\nabla_i\nabla_j A \;\nabla_j\nabla_i B}(\tau_s) = {1 \over 3}{\cal G}_{\nabla^2 A\nabla^2 B}(\tau_s).
\label{eq:gen_n}
\een
\ees
We will denote the generating function of the density contrast $\delta$ by  ${\cal G}_{\delta}(\tau_s) = \sum^{\infty}_{n=1}\, {\nu_n/n!}\, \tau^n_s$
where $\nu_n \equiv \la \delta^{(n)} \ra_c$. We will also need the divergence of velocity $\theta$ (to be defined later),
for which, the generating function will be denoted as ${\cal G}_\theta(\tau_s)\equiv  \sum^{\infty}_{n=1}\, {\mu_n/n!}\, \tau^n_s$ 
with $\mu_n \equiv \la \theta^{(n)} \ra_c$.

The generating function formalism was  originally introduced in \citep{BaS89} and later
exploited in many publications including in \citep{Ber92,Ber94} to compute the lower order cumulants and cumulant correlators \citep{francis}
by linking the generating function of tree-level amplitudes directly with the dynamical
equations of a self-gravitating collisionless system \citep{review}. 
The resulting expressions will be applied to understand halo clustering in \textsection{\ref{sec:non}}.
Notice the bias of overdense regions has been studied using
the generating function formalism in  \citep{francis}.
%
