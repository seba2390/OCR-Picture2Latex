%%%%%%%%%%%%%%%%%%%%%%%%%%%%%%%%%%%%%%%%%%%%%%%%%%%%%%%%%%
\section{Symmetries and Generating Function}
\label{sec:append_sym} 
%%%%%%%%%%%%%%%%%%%%%%%%%%%%%%%%%%%%%%%%%%%%%%%%%%%%%%%%%%
In this appendix we provide a detailed derivation of of ${\cal G}_{s^2}=0$. These results are a direct 
consequence of the fact that $s_{ij}$ is a traceless tensor and the generating functions
which encode tree-level amplitudes of vertices in the perturbative 
analysis of Euler-Continuity-Poisson system satisfies spherical top-hat collapse equations;  in
a spherically symmetric setup the off-diagonal terms that are related to departure from spherical symmetry do not contribute.
We start with the definition of $s_{ij}$ in Eq.(\ref{eq:def_s}) which gives:
\ben
&& s_{ij}s^{ij} = {2 \over 3 {\cal H}^2} \nabla_{i}\nabla_j \phi\nabla_{i}\nabla_j \phi
-{2\over 3{\cal H}^2} \nabla_{i}\nabla_j \phi\, \delta^{K}_{ij} + {1 \over 9 }\,\delta^{K}_{ij}\,\delta^{K,ij}\delta^2.
\label{eq:sij_square}
\een
Summation over repeated indices is assumed. Next, we can simplify the first term using Eq.(\ref{eq:gen_3}) as:
%Poisson Eq.(\ref{eq:Poisson}) as:
\ben
{\cal G}_{\nabla_{i}\nabla_j \phi\nabla_{i}\nabla_j \phi} = {1\over 3}{\cal G}^2_{\delta}
\een
Using this expression in Eq.(\ref{eq:sij_square}) we arrive at the desired result.
A similar calculation can be used to prove ${\cal G}_{t^2}=0$ as well as ${\cal G}_{st}=0$.
These results are valid in the perturbative regime. Thus, it depends on
the assumption that the fluid flow is single stream and irrotational. 

Next we consider the terms involving the derivatives of $s_{ij}$ e.g. $\nabla^2[s_{ij}s^{ij}] $.% in Eq.(\ref{eq:def_bias}). 
We note that $\nabla^2[s_{ij}s^{ij}] = 2\nabla^2[s_{ij}]s^{ij}$. So, we can write:
\ben
\nabla^2[s_{ij}] = \nabla_{i}\nabla_j\delta -{1\over 3}\delta^{K}_{ij}\nabla^2\delta
\een
Using Eq.(\ref{eq:gen_3}) as before we can write:
\ben
{\cal G}_{\nabla^2{s_{ij}s^{ij}}} = 2{\cal G}_{s_{ij}\nabla^2{s_{ij}}} = 0.
\een
The result ${\cal G}_{\nabla^2{s_{ij}}\nabla^2{s^{ij}}}=0$ can be derived using similar steps.
Following similar arguments we can prove similar identities for the divergence of velocity
in case of potential flow. These terms are not included in our definition of bias.  
%
%
%%%%%%%%%%%%%%%%%%%%%%%%%%%%%%%%%%%%%%%%%%%%%%%%%%%%%%%%%%%%
\section{Bias and Lagrangian perturbation theory}
\label{sec:append_LPT}
%%%%%%%%%%%%%%%%%%%%%%%%%%%%%%%%%%%%%%%%%%%%%%%%%%%%%%%%%%%%
It is possible to consider the Lagrangian perturbation theory (LPT) to model the underlying dynamics.
The Zel'dovich approximation (ZA) is first order in LPT. We list below
the generating functions at various order \citep{MSS}:
\ben
&& 1+ {\cal G}^{\rm ZA}_{\delta}(\tau_s) = \sum^{\infty}_{n=1} {{\mu}^{\rm ZA}_n\over n!}\tau_s^n =
 \left ( 1- {\tau_s \over 3}\right )^{-3}; \nn \\
&& 1+ {\cal G}^{\rm PZA}_{\delta}(\tau_s) =\sum^{\infty}_{n=1} {{\mu}^{\rm PZA}_n\over n!}\tau_s^n = \left ( 1- {\tau_s \over 3}-{\tau_s^2 \over 21}\right )^{-3}.
%&& G^{\rm PZA}_{\delta}(\tau) = \sum^{\infty}_{n=1} {{\mu}^{\rm PZA}_n\over n!}\tau^n = \left ( 1- {\tau \over 3}-{\tau^2 \over 21}\right )^{-3}-1.
\een
Here $\rm PZA$ is the post Zel'dovich Approximation. A systematic development of higher order LPT in the 
context of generating function was developed in \citep{MSS}. For the Zel'dovich Approximation (1st order in LPT):
\ben
\{a_i\}^{\rm ZA} = \Big \{{2\over 3}, {10\over 27}, {5\over 27}, \cdots \Big \};  \quad 
\{a^I_i\}^{\rm ZA} = \Big \{-{2\over 3}, {14\over 27}, -{35\over 81}, \cdots \Big \}.
\een
%\ben
%&& a_2 = {2\over 3};\quad a_3 = {10\over 27};\quad a_4 = {5 \over 27};\\
%&& a^I = -{2\over 3};\quad  a^I_3 = {14\over 27};\quad a^I_4 = -{35\over 81}.
%\een
The corresponding relation between Lagrangian and Eulerian bias are:
\ben
&& b_2^E = {2 \over 3}b_1^L + b_2^L; \quad b_3^E = -{16\over 9}b_1^L -b^L_2 +b^L_3.
\label{eq:LE_ZA}
\een
For PZA:
\ben
\{a_i\}^{\rm PZA} = \Big \{{17\over 21}, {106\over 189}, {47\over 1323}, \cdots \Big \};  
\quad \{a^I_i\}^{\rm PZA} = \Big \{-{17\over 3}, {992\over 1323}, -{20558\over 27783}, \cdots \Big \}.
\een
%\ben
%&& a_2= {17\over 21}; \quad  a_3 = {106 \over 189}; \quad a_4 = {473 \over 1323}; \\
%&& a_2^I= -{17 \over 21}; \quad  a^I_3 = {992 \over 1323} ; a^I_4 = -{20558 \over 27783}. 
%\een
The corresponding relations between Lagrangian and Eulerian bias get modified to: 
\ben
&& b_2^E = {8 \over 21}b_1^L + b_2^L; \quad  b_3^E = -{158\over 441}b_1^L -{13 \over 7}b^L_2 +b^L_3.
\label{eq:LE_PZA}
\een
Eq.(\ref{eq:LE_ZA}) and Eq.(\ref{eq:LE_PZA}) are Lagrangian approximations to the exact expression in Eq.(\ref{eq:b3}).
These results can be trivially extended to expressions linking higher-order Lagrangian and Eulerian 
bias parameters
%as well to higher-order in LPT.
%\section{Lagrangian Perturbation Theory and Large deviation principle}
