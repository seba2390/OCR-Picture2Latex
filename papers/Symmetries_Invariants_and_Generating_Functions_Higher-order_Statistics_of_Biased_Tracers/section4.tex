%
%%%%%%%%%%%%%%%%%%%%%%%%%%%%%%%
\section{Scale-Dependent Bias}
\label{sec:bias}
%%%%%%%%%%%%%%%%%%%%%%%%%%%%%%%
%
%
The models discussed in \textsection\ref{sec:non} are not scale-dependent as the parameters $b^L_\ell$ are
independent of the wavenumbers $\bq_i$. 
The formalism of scale-dependent bias was developed in a series of papers: \citep{Matsubara,jack1,jack2,verde}
\ben
&& \delta_h(\bk) = \sum^{\infty}_{n=1} {1\over n!}\int{d^3\bq_1 \over (2\pi)^3}\cdots \int{d^3\bq_n \over (2\pi)^3} 
c^L_n(\bq_1,\dots,\bq_n)\delta_L(\bq_1)\cdots\delta_L(\bq_n)\delta_{\rm D}(\bk-\bq_{1\dots n}); \nn \\
&& \bq_{1\dots n} \equiv \bq_1+\cdots+\bq_n.
\een
Here $\delta_L$ is the linear density contrast in a perturbative expansion.  
In the Fourier domain the scale-dependent bias is implemented by 
replacing the scale-independent $b^L_{\ell}(\tau)$ 
parameters defined in previous section with the following 
functions $c^L_{\ell}(\bk,\tau)$ of wave numbers \citep{jacques2}:
\ben
\label{eq:c1}
&& c^L_1(\bq,\tau_i) = b^{\rm L}_{10}(\tau_i)+ b^{\rm L}_{01}(\tau_i) \bq^2\\
&& c^L_2(\bq_1,\bq_2,\tau_i) = {b^L_{20}}(\tau_i) + b^L_{11}(\tau_i)(\bq_1^2+ \bq_2^2)\nn \\
&&\quad + b^L_{02}(\tau_i)\bq_1^2\bq_2^2 -2 \chi^{L}_{10}(\bq_1\cdot\bq_2) + \chi^L_{01}(\tau_i)
\left [ 3(\bq_1\cdot\bq_2)^2 - \bq_1^2\bq_2^2\right ]
\label{eq:c2}
\een
An angular averaging of $c^L_n$ recovers the scale-dependent parameters $b^{L}_{\ell}$.
The functions $c^L_1$ and $c^L_2$ which depend on the parameters $b^L_{ij}$, $\chi^L_{ij}$ 
can be computed using peak-background split in \citep{vincent,jacques2}.
The form of the bias functions is based on rotationally symmetric invariants.
The peaks of the smoothed density fields are defined up to second order in derivatives
which explains the absence of terms with higher powers in $k$.
The Zel'dovich approximation (ZA) is used to map the Lagrangian positions to Eulerian position. 
In the real or configuration space \citep{jacques2}:
\ben
&& \delta_h\xti = b^L_{10}(\tau_i)\delta\xti-b^L_{01}(\tau_i)\Delta \delta\xti \nn \\
&& \quad\quad + {1\over 2!}b^L_{20}(\ti)[\delta\xti]^2 - b^L_{11}(\ti) \delta\xti\Delta\delta\xti
+{1\over 2} b^L_{02}(\ti)[\Delta\delta\xti]^2 \nn \\
&& \quad\quad + \chi^L_{10}(\tau_i)\, \nabla\delta\xt\cdot\nabla\delta\xt 
+ {1\over 2!}\chi^L_{01}(\tau_i)\left[ 3\nabla_i\nabla_j\delta -\delta^{(\rm K)}_{ij}\Delta\delta \right ]^2
+\cdots
\een
In terms of $\psi$, $\eta$ defined before \citep{jacques2}:
\ben
&& \delta_h\xt = b_{10}\delta\xt -b_{01}\Delta\delta\xt + {1\over 2!} b_{20} \delta^2\xt \nn\\
&& \quad\quad + {1\over 2!} b_{s^2} s^2\xt + b_{\psi}\psi\xt + b_{st} s\xt\cdot t\xt + \cdots
\een
The generating functions of $\delta_h$ and $\delta$ are related by the following expression:
\ben
&& \cG_{\delta_h} = b_{10}\cG_{\delta} -b_{01}\cG_{\Delta\delta} + {1\over 2!}b_{20}\cG_{\delta}^2 + {1\over 2!} b_{s^2}\cG_{s^2} \nn \\
&& \quad\quad + {1\over 2}b_{02}[{\cal G}_{\triangle \delta}]^2- b_{11}{\cal G}_{\delta}{\cal G}_{\triangle\delta} + b_{\psi}\cG_\psi + b_{st}\; \cG_{s\cdot t} + \cdots
\een
Thus the generating function $\cG_{\delta_h}$ at second order is determined by $\cG_{\delta}$ and $\cG_{\triangle\delta}$.
The following expressions relate the Eulerian bias coefficients $b_{ij}$ with their Lagrangian counterparts $b^L_{ij}$:
\ben
&& b_{10} = 1+ b_{10}^L;\quad b_{01} = -R_v^2+ b_{01}^L; \quad b_{20} = b_{20}^L + {8\over 21}b_{10}^L ; \nn \\
&& b_{s^2} =-{4\over 7}b_{10}^L;  \quad b_{\psi}=-{1\over 2}b_{10}^L; \quad b_{s t}=-{5\over 7}b_{10}^L;
\label{eq:bias_coeff}
\een
The expressions in Eq.(\ref{eq:zero}) gives statistics of $\delta_h$
in terms of the coefficients $b_{s^2}$, $b_{\psi}$ and $b_{st}$.
In addition to ${\cal G}_{\delta}$ it also depends on ${\cal G}_{\triangle \delta}$.
Scale dependent bias has also been used in the context of primordial non-Gaussianity \citep{Desjacques} which we have
ignored here. However, the results discussed here can trivially extended to include primordial non-Gaussianity.
%
