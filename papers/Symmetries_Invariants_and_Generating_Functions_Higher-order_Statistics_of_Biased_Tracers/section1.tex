%%%%%%%%%%%%%%%%%%%%%%%
\section{Introduction}
%%%%%%%%%%%%%%%%%%%%%%
\label{sec:intro}

%% Surveys
Recently completed cosmic microwave background (CMB) surveys e.g. the 
Planck surveyor\footnote{Planck: \href{http://www.cosmos.esa.int/web/planck/}
{\tt  http://www.cosmos.esa.int/web/planck/}}, have provided us with a robust cosmological framework 
that will allow us to investigate the physics beyond the {\em Standard Model} of cosmology.
Next generation surveys are mapping the entire CMB sky with higher resolution and accuracy
e.g. ACT\footnote{ACT: \href{http://www.physics.princeton.edu/act/}{\tt http://www.physics.princeton.edu/act/}} and
SPT\footnote{SPT: \href{http://pole.uchicago.edu/}{\tt http://pole.uchicago.edu}} to answer many of the
questions relevant to structure formation in the low redshift Universe.
In addition, the ongoing and future large scale surveys will map the sky with ever increasing precision
(BOSS\footnote{Baryon Oscillator Spectroscopic Survey: \href{http://www.sdss3.org/surveys/boss.php}{\tt http://www.sdss3.org/surveys/boss.php}}
\citep{EW},
WiggleZ\footnote{WiggleZ Survey : \href{http://wigglez.swin.edu.au/}{\tt http://wigglez.swin.edu.au/}}
\citep{DJA},
DES\footnote{Dark Energy Survey: \href{http://www.darkenergysurvey.org/}{\tt http://www.darkenergysurvey.org/}}
\citep{DES},
EUCLID\footnote{EUCLID: \href{http://www.euclid-ec.org/}{\tt http://www.euclid-ec.org/}}
\citep{LAA}). These surveys will provide a glimpse of physics beyond the standard model.
On one hand, they will check any departure from general relativity (GR) on cosmological scales,
on the other, they will also provide an estimate of the sum of the neutrino mass \citep{review2}. 
The galaxies, however, are known to be biased tracers of underlying dark matter distribution \citep{Desjacques}.
To achieve the full potential of the future surveys, parametrization and
understanding of bias is of utmost importance.
%%% Bias

Initial models of bias were linear and local relations
between the tracer's density contrast $\delta_h$ and the underlying matter density contrast $\delta$:  $\delta_h=b_1\delta$.
However, early simulations e.g. \citep{CenOstriker} pointed to a more complex nature of bias that can
be nonlinear and non-local. It was also realized that bias can also be stochastic. 
On the theoretical side, both perturbative and non-perturbative models started to emerge.
The non-perturbative theories based on peak approach were developed in \citep{BBKS};
these theories can mimic many of the properties of galaxy bias successfully. 
Halo based approaches were developed which remain an important tool for making
analytical predictions \citep{CooraySheth}.

Many recent approaches have also seen development of formal perturbative
approaches that extend and put known results on a solid foundation \citep{Matsubara,Carlson,Vlah}.   
The effective field theory based approaches were also developed \citep{Effective_bias,Angulo}.
Indeed, the polynomial model for bias developed in \citep{FryGaztanga} was valid for large smoothing scales 
and lacks the non-local terms that are generated due to gravitational clustering.
Using just second-order standard perturbation theory (SPT) it has been shown that gravitational
evolution is responsible for generating tidal interactions which are non-local in the density field \citep{Fry,Goroff,Bouchet}.
Similar non-local contribution is also expected in the clustering of halos and was studied in detail in several
publications \citep{Catelan1,Catelan2}. In \citep{Roy} a more generic scheme for bias was developed based on
symmetries inherent in the dynamical equations.  
In the nonlinear regime gravity-induced bias has been studied in the context of the
hierarchical ansatz (HA) \citep{bias_letter, BernardeauSchaeffer}.
It has recently been pointed out that non-local bias can mimic scale-dependent 
suppression of growth of perturbation in cosmologies with massive neutrinos  
\citep{Saito,Audren,Gong,Adreu,Beutler,Marlina}.
The non-local bias is relatively small compared to the linear bias
but future large-scale surveys will be sensitive to them.
Large-scale numerical simulations have been employed to investigate their
effect on the clustering of halos \citep{sim1,sim2,sim3, LoVerde}. It was shown that
scale-dependent bias in the power spectrum of dark matter halo is
degenerate with the signatures left by the various non-local bias terms.
The degeneracies present in the characterization of the bias can be broken by
using higher-order statistics of biased tracers. Going beyond the usual power spectrum analysis, 
in this paper we compute the higher-order statistics for a generic biasing scheme
which can be nonlinear, non-local and stochastic. The additional
terms in the perturbative description of bias stem from symmetry considerations.
The inherent symmetries in the dynamical
equations predict invariant quantities \citep{ChecnScocSheth,Baldauf,Kehagias} in the perturbative
expansion of bias. 

A generating function based approach in the perturbative regime was introduced in \citep{Ber92}.
It provides a powerful framework to analyse the higher-order statistics of cosmological fields.
We use this formalism along with the
functional relationship dictated by a biasing scheme of $\delta_h$ with $\delta$ and $\theta$ to compute the
tree-level vertices of $\delta_h$ as a function of those of $\delta$ and $\theta$.
These derivations are valid in the perturbative regime.
We derive the formal relations in the presence of nonlinear, non-local and higher-derivative terms.
We also include a stochastic noise that originates from our lack of 
knowledge of the fundamental physics related to the galaxy formation process.
We use these expressions to decide which terms in these expressions
do not contribute at any order. The results from the generating function formalism 
are next used to express the cumulants and cumulant-correlators of the biased tracers.
Our aim is to  extend the results presented in \citep{Munshi_IBIT} to more general biasing schemes. 

This paper is organised as follows. In 
\textsection{\ref{sec:gen}} we review the generating function approach.
In \textsection{\ref{sec:non}} we consider a family of generalized bias models and use the generating function to analyse them.
\textsection{\ref{sec:bias}} is devoted to discussion of a non-local bias.
\textsection{\ref{sec:cumu}} is devoted to the discussion of cumulants and cumulant correlators (CCs).
Finally, our conclusions are presented in \textsection{\ref{sec:conclu}}. Some of the details of our derivations
are relegated to the two appendices.
  