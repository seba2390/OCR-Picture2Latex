%%%%%%%%%%%%%%%%%%%%%%%%%%%%%%%%%%%%%%%%%%%
\section{Cumulants and Cumlant Correlators}
%%%%%%%%%%%%%%%%%%%%%%%%%%%%%%%%%%%%%%%%%%%
\label{sec:cumu}
In this section we use the results derived in previous
sections to compute the higher-order one-point and two-point
statistics of biased tracers.
The cumulants and their correlators for the halos and the underlying dark matter distribution is defined as follows:
\ben
{\cal S}_{\rm N}^{(h)} = {\la \delta_h^{\rm N}\ra_c \over \la \delta^2_h\ra_c^{\rm N-1}}; \quad 
{\cal C}_{\rm pq}^{(h)} = {\la \delta_{h1}^{p}\delta_{h2}^q\ra_c \over \la\delta_{h1}\delta_{h2} \ra_c \la \delta^2_h\ra_c^{\rm p+q-2}}; 
\quad \delta_{hi}\equiv \delta({\bf x}_i).
\label{eq:cc}
\een
A similar expression holds for the underlying dark matter distribution and will be denoted without the subscript ${}_h$ \citep{FryGaztanga}.
\ben
\label{eq:cc1}
\nu_1^{(h)} = b_1; \quad
\nu_2^{(h)} = (b_2 + b_1 \nu_2); \quad 
\nu_3^{(h)} = (b_3 + 3 b_2 \nu_2 + b_1 \nu_3).
\label{eq:cc4}
\een
For $b_1=1$ and $b_n=0$ we recover the unbiased result $\nu_n^{(h)} = \nu_n$.
In practice the $b_n$ are computed  using the Press-Schechter (PS) or Sheth-Tormen (ST) 
mass functions or using theories based on peak statistics.
The expressions of $b_n$ are given in Eq.(\ref{eq:zero}) and Eq.(\ref{eq:bias_coeff}).
%At this stage the bias parameters $b_n$ are left completely arbitrary.
%However, we will use the PS (ST) mass functions to compute these parameters.
%Results will be also valid for peaks, where the bias functions can be
%replaced with the bias functions for peaks beyond a threshold.
The cumulants ${\rm S}_n^{(h)}$ can be expressed in terms of the vertices $\nu^{(h)}_n$ \citep{Fry,Ber92}:
\ben
\label{eq:S_N}
{\cal S}^{(h)}_3= 3\nu^{(h)}_2; \quad
{\cal S}^{(h)}_4 = 4\nu^{(h)}_3 + 12[\nu^{(h)}_2]^2.
\label{eq:S_Np}
\een
In the perturbative regime the following relations hold \cite{Ber92,Ber94}:
\ben
{\cal S}_3 = {34\over 7} + \gamma_1;\quad
{\cal S}_4 = {60712 \over 1323} + {62 \over 3}\gamma_1 + {7 \over 3}\gamma_1^2 +{2 \over 3}\gamma_2.
\een
The terms involving $\gamma_p=[{d^p \log\,\sigma^2(R_0)/d(\log R_0)^p}]$ are results of smoothing using top-hat window
of radius $R_0$.
We will use the following notation to represent the variance of the smoothed field $\sigma^2(R_0) =\la\delta_s^2\ra_c$ and 
correlation function $\xi_{12}=\la\delta_s({\bf x}_1)\delta_s({\bf x}_2)\ra_c$. 
The CCs take the following form:
\ben
\label{eq:c21_DM}
{\cal C}^{(h)}_{21} = 2\nu^{(h)}_2; \quad {\cal C}^{(h)}_{31} = 3\nu^{(h)}_3 + 6\nu^{(h)}_2; 
%&& {\mathbb C}^{(h)}_{41} = 4\nu^{(h)}_4+36[\nu^{(h)}_2][\nu^{(h)}_3]+24[\nu^{(h)}_2]^3. 
\label{eq:c51}
\een
The CCs satisfy a factorization property in the large-separation limit [$\xi_{12}(|{\bf x}_1 -{\bf x}_2|) < \sigma^2(R_0)$]:
${\rm C}^{(h)}_{pq} = {\rm C}^{(h)}_{p1}{\rm C}^{(h)}_{q1}$. Here, $\sigma^2(R_0)$ is the variance of the
smoothed density field, and $\xi_2(|{\bf x}_1 -{\bf x}_2|$ represents the two-point correlation function.
A tophat smoothing window with a radius $R_0$ is assumed.
In the quasi-linear regime with a tophat smoothing window the CCs have the following expressions \citep{francis}: 
\ben
\label{eq:c21}
{\cal C}_{21} = {68\over 21}+ {\gamma_1 \over 3}; \quad
{\cal C}_{31} = {11710 \over 441} + {61 \over 7}\gamma_1 + {2 \over 3}\gamma_1^3 +{\gamma_2\over 3}.
\label{eq:c41}
\een

The cumulant correlators probe squeezed and collapsed configuration of the underlying multispectra.
The related statistics in the Fourier domain are the squeezed bispectrum and the squeezed and collapsed
trispectrum. The lowest order non-trivial ${\cal C}_{21}$ is independent of the contribution from $s_{ij}s^{ij}$.
In a similar manner the squeezed bispectrum do not take any contribution from $s_{ij}s^{ij}$ in Fourier
domain. It is expected that similar results will hold for the squeezed and collapsed
trispectrum, ${\cal C}_{31}$ and ${\cal C}_{22}$ respectively.
%
In \textsection\ref{sec:non} we have shown that all additional terms vanish and only contributions from $\psi$
need to be included. Taylor expanding ${\cal G}_{\psi}(\tau_s)$:
\ben
{\cal G}_{\psi}(\tau_s) =&& {4\over 21}+ {8\over 21}\tau_s +
{1\over 42}(8 + 21\mu_2  -13\nu_2)\tau_s^2 \nn \\
&& + {1\over 126} \left ( 21\mu_3 +24\nu_2 -13 \nu_3\right )\tau_s^3+ \cdots
\een
The expressions for $\nu^{(h)}_k$ defined in Eq.(\ref{eq:cc4}) now get modified and depend also on the bias coefficient $b_{\psi}=-b_L/2$ as:
\bes
\ben
\label{eq:cc1}
&& \nu_1^{(h)} = b_1 + {4\over 21}b_{\psi}; \quad\\
&& \nu_2^{(h)} = (b_2 + b_1 \nu_2) + {8\over 21}b_{\psi}; \quad \\
&& \nu_3^{(h)} = (b_3 + 3 b_2 \nu_2 + b_1 \nu_3) + {1\over 84}(8 + 21\mu_2  -13\nu_2)b_{\psi}.
\label{eq:cc4}
\een
\ees
Notice that ${\cal G}_{\eta}(\tau_s)$ takes contribution from both $\delta$ and $\theta$ vertices
thus making the $\delta_h$ statistics a function of both $\delta$ and $\theta$ Eq.(\ref{eq:cc1})-Eq.(\ref{eq:cc4}).
Indeed, Eq.(\ref{eq:noise})-Eq.(\ref{eq:noise1}) provide a framework for inclusion of arbitrary noise contribution.
In case of a Gaussian noise, the higher-order terms of Eq.(\ref{eq:noise1}) will not contribute and only the variance 
will be affected through the term $b_{\epsilon^2}\epsilon^2/2$. 

For two different populations of tracers $h$ and $h^{\prime}$ the CCs defined in Eq.(\ref{eq:cc}) can be generalized to:
\ben
{\cal C}_{pq}^{(hh^{\prime})} = 
{\la \delta_{h1}^{p}[\delta_{h^{\prime}2}]^q\ra_c \over \la\delta_{h1}\delta_{h\prime2} 
\ra_c \la \delta^2_{h1}\ra_c^{\rm p-1}\la [\delta_{h\prime2}]^2\ra_c^{\rm q-1}}; 
\een
We have used the following notations $\delta_{h1}\equiv\delta_h({\bf x}_1)$ and $\delta_{h^{\prime}2}\equiv\delta_{h^\prime}({\bf x}_2)$,
the respective CCs can be factorized and be expressed in terms of respective CCs i.e. 
${\cal C}^{hh\prime}_{pq}= {\cal C}^{h}_{p1} {\cal C}^{h\prime}_{q1}$.
The CCs  ${\cal C}^{h}_{p1}$ and ${\cal C}^{h\prime}_{q1}$ are constructed from the coefficients of the
series expansion of their respective density contrasts as in Eq.(\ref{eq:series}). 
%
%
%
%
%
%
 