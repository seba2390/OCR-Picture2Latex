%%%%%%%%%%%%%%%%%%%%%
\section{Summary and Outlook}
%%%%%%%%%%%%%%%%%%%%%
\label{sec:conclu}
%%
%
In a generic biasing scheme, based on symmetry arguments, many additional terms can be included.
Using a generating function formalism we test which of these terms actually contribute.
In \citep{Kehagias} symmetry arguments were used to determine the temporal dependence of the terms 
included in Eq.(\ref{eq:def_bias1}). Here, we use the symmetry to determine which terms will actually contribute.
Many other terms can be simplified drastically. We use the results to compute the higher-order one-point cumulants 
as well two-point CCs for collapsed objects.

%% Generating Functions
The degeneracies present in the characterization of bias at the level of power spectrum can only be broken by
using higher-order statistics of biased tracers. 
In this paper we compute higher-order statistics in terms of the recently introduced {\em invariant}
quantities. The inherent symmetries of the dynamic
equations predict invariant quantities \citep{ChecnScocSheth,Baldauf,Kehagias} in the perturbative
expansion of bias. Using a generating function approach we use the
functional relation of $\delta_h$ in terms of $\delta$ and $\theta$ to relate the
tree-level vertices of $\delta_h$ with that of $\delta$ and $\theta$.
We derive formal relations in the presence of higher-derivative terms.
We have used these expressions to show that certain terms in this expression
do not contribute at any given order. The results from the generating function formalism 
is used to express the cumulants and CCs of biased tracers.
The squeezed bispectrum as well as the squeezed and collapsed tri-spectra are related to their counterparts of same order, respectively
to ${\cal C}_{21}$ and ${\cal C}_{31}$, ${\cal C}_{22}$. The results
developed here generalize the expressions derived for CCs in ref.\citep{Munshi_IBIT} assuming a 
polynomial bias model. Similar generalizations are possible for squeezed and collapsed multispectra.

%% Stochastic Bias
classical physics at a fundamental level is deterministic. However, the detailed microscopic physics of 
galaxy formation is not well understood. A stochastic contribution $\epsilon$ is 
thus often included in the expression of bias to encode our lack of knowledge that relates $\delta_h$ with $\delta$ and $\theta$.
The terms in Eq.(\ref{eq:def_bias1}) correspond to  $\epsilon=0$ (no-noise). The terms in Eq.(\ref{eq:noise}) depicts the
first- and second-order terms in noise in a Taylor expansion of $\delta_h$ in terms of $\epsilon$.
In computing the higher-order statistics of any biased tracers these contributions
should be included. We have considered higher-order terms in $\epsilon$ and that represent its higher-order correlation with
$\delta$ and $\theta$. The generating function
approach is generalized to take in account presence of such stochastic contributions at an arbitrary order.
The generating function approach simplifies the order-by-order analysis.
Assuming a Gaussian stochastic noise can further simplify the expression as all terms beyond
second-order that characterize non-Gaussianity vanish. 

%% Large Deviation Principle
In recent years the large deviation principle (LDP) has been used to construct the one- and two-point 
PDFs of biased tracers \citep{Cora1,Cora2,Cora3} (also see \citep{valageas} for related approach based on steepest descent method).
This method is also related  to earlier generating function based approaches \citep{Ber92,Ber94,francis}.
Recent work based on LDP also has attempted to compute the PDF and bias of collapsed objects.,
These results were obtained assuming a polynomial biasing model.
Results presented here will help us to go beyond the polynomial model and include 
stochastic noise within the LDP formalism (Munshi 2017; in preparation).
%
  