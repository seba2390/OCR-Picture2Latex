%%%%%%%%%%%%%%%%%%%%%%%%%%%%%%%%%%%%%%%%%%
\section{Bias and Biased Tracers}
\label{sec:non}
%%%%%%%%%%%%%%%%%%%%%%%%%%%%%%%%%%%%%%%%%%
%
The idea of nonlocal bias has been investigated in great detail in the past by many author. Starting from ref.\citep{Roy} the idea was developed 
further in \citep{ChecnScocSheth,Baldauf}. More recently the results were extended to third order in perturbation theory in ref.\citep{halo}.
The idea behind these studies is to probe the statistics of proto-halos which preserve their identity and
their number density is conserved.  It is assumed that, though their shape and topology may change, the center of mass
of these proto-halos follow a well defined trajectory and their statistics can studied using perturbative techniques. 
We will consider the matter dominated case, ${\bf x}$ is the spatial comoving coordinate and $\tau$
the conformal time $\tau = \int {dt/a(t)}$. The associated Hubble parameter is ${\cal H} = {d\ln a(t)/d\tau}$.
We will also define the divergence of velocity as $\theta = \partial_i{v^i}$ where $v^i = dx^i/d\tau$.
The Euler, continuity and Poisson equations describe the gravitational clustering of a collisionless system in
the hydrodynamic limit: 
\bes
\ben
&& {\partial \delta \over \partial \tau} + \nabla_i[(1+\delta) v^i]=0; \\
&& {\partial v^i \over \partial \tau} + {\cal H} v^i + v^j\nabla_j v^i = -\nabla^i \phi;\\
&& \nabla^2\phi = {3 \over 2}\Omega_{\rm M}{\cal H}^2 \delta.
\een
\ees
In terms of generating functions for $\delta$ and $\theta$ denoted as 
${\cal G}_\delta = \sum ({\nu_n/n!})\tau_s^n$ and ${\cal G}_\theta = \sum ({\mu_n/n!})\tau_s^n$, these equations take the following forms \citep{MSS}:
\bes
\ben
\label{eq:gen1}
&& {\partial {\cal G}_\delta \over \partial \tau} + (1+{\cal G}_\theta){\cal G}_\delta=0; \\
&& {\partial {\cal G}_{\theta}\over \partial \tau} + {1\over 2}{\cal G}_{\theta}+{1\over 3}{\cal G}^2_{\theta}+ {\cal G}_{\nabla^2\Phi} = 0 ; \\
\label{eq:g2}
&& {\cal G}_{\nabla^2\phi} = {3 \over 2}{\cal G}_\delta.
\label{eq:g3}
\een
\ees
The solution to these equations are well known and ${\cal G}_\delta$ and ${\cal G}_\theta$ satisfy collapse of spherically
over-dense top-hat perturbation. However, Eq.(\ref{eq:gen1})-Eq.(\ref{eq:g3}) do not represent evolution of perturbations
they encode the statistical description of an ensemble of perturbations.

To relate the tracer density $\delta_h$ with $\delta$ many different simplifying assumption
are employed. It is typically assumed the number density of tracers (proto-halos) do not change and remains
conserved. Thus evolution of $\delta_h$ can be described by a continuity equation. In this picture
the halos can change shape or their topology but they follow a well-defined trajectories.
It is further assumed that halo velocities  ${\bf v}_h$ are unbiased estimators of underlying dark matter velocities
${\bf v}$ i.e. ${\bf v}_h={\bf v}$:
\ben
(\dot\delta_h-\dot\delta) + \nabla_i [(\delta_h-\delta){v^i}]=0.
\label{eq:master}
\een
The overdots represent derivative w.r.t. $\tau$.
The halo density contrast $\delta_h({\bf x},\tau_i)$ and the DM density contrast $\delta({\bf x},\tau_i)$ 
are related at some initial time $\tau_i$ as follows:
\ben
&& \delta_h({\bf x},\tau_i)\equiv b(\delta) = \sum_{\ell} {b^L_\ell(\tau_i) \over \ell!} [\delta({\bf x},\tau_i)]^{\ell}
 = \sum_{\ell} {b^L_\ell(\tau) \over \ell!} [\delta({\bf x},\tau)]^{\ell}.
\een
Thus the evolution of the Lagrangian bias $b^L_{\ell}(\tau)$ as a function of conformal time $\tau$ from  $b^L_{\ell}(\tau_i)$ takes the following form:
$b^L_{\ell}(\tau) = b^L_{\ell}(\tau_i)\left [ {a(\tau_i)/a(\tau)} \right ]^{\ell}$.
The above expression can be used to evaluate $b^L_{\ell}$ at a later time $\tau$
once specified at an initial epoch $\tau_i$. Notice that at this stage we have left the parameters $b^L_{\ell}$ arbitrary.
Perturbative analysis of Eq.(\ref{eq:master}) has been carried out in an order-by-order manner.
In \citep{ChecnScocSheth,Baldauf} an analysis was performed up to second order in the linear density contrast $\delta^{(1)}$,
more recently the result was extended to third order in \citep{halo}. 
These studies found that the
halo density contrast $\delta_h$ is related to the underlying $\delta$ through the
following expression:
\bes
\ben
&& \delta_h \equiv b(\delta,\theta, \nabla_i\nabla_j\phi,\nabla_i\theta_j,\cdots );\\
&&\delta_h= b_1\delta + {1\over 2!}b_2\delta^2 + {1\over 3!}b_3\delta^3 +
{1\over 2!}b_{s^2} s^2 + b_{\psi}\psi + b_{st} s\cdot t  \nn\\
&&\quad\quad  + b_{\nabla^2\delta} \nabla^2\delta + b_{\nabla^2} \nabla^2[s_{ij}s^{ij}] + b_{\nabla^4} \nabla^2[s_{ij}]\nabla^2[s^{ij}] \cdots.
\label{eq:def_bias1}
\een
\ees
The coefficients $b_{s^2}, b_{\psi} \cdots$ and the higher-order derivative operators 
$b_{\nabla^2\delta}, b_{\nabla^2}, \cdots$ appearing in Eq.(\ref{eq:def_bias1}) are left arbitrary at this stage.
The non-local operators $t,s,\eta$ ans $\psi$ above are defined as \citep{Roy}:
\bes
\ben
\label{eq:def_s}
&& s_{ij}= {2\over 3{\cal H}^2}\nabla_i\nabla_j \phi -{1\over 3}\delta^{\rm K}_{ij} \delta;  \\
&& t_{ij}= \partial_i v_j -{1\over 3}\delta_{ij} \theta - s_{ij};  \\
\label{eq:def_t}
&& \eta = \theta -\delta; \\
\label{eq:eta}
&& \psi = \eta -{2\over 7}s^2 +{4\over 21}\delta^2.
\label{eq:psi}
\een
\ees
we have introduced the following notations
\ben
s^2 \equiv s_{ij}s^{ij}; \quad\quad s\cdot t \equiv s^{ij} t_{ij}; \quad\quad t^2\equiv t_{ij}t^{ij}.
\een
We have also assumed a $\Omega=1$ universe. However, the higher-order statistics are known to be very weakly-dependent
on background cosmology. The traceless tidal tensor is denoted as $s^{ij}$.
Here, $t_{ij}$ is considered to be symmetric as vorticity is not generated at lower-order in perturbation theory
only needs to be accounted for at a very higher-order. The terms $\eta$ and $t$ start to contribute at second-order
while $\psi$ contributes at cubic order and beyond. 

These operators along with density $\delta$ are invariants under the extended Lifshitz and Galilean transformation.
The local bias expansion corresponds to the invariant $\delta$
and represents a Taylor expansion of $\delta_h$ with coefficients $b_{\ell}$ specifying the exact 
functional form of $b(\delta)$. However, this is incomplete as inherent 
extended Lifshitz and Galilean symmetry of the Euler-continuity-Poisson system also
allows the additional invariants $s^2,t^2$ and $s\cdot t$ involving $s^{ij}$ and $t^{ij}$ etc. 
It has been argued that even if these forms of bias are not present in the initial conditions
there is no guarantee that they will not be generated during the subsequent gravitational evolution
as they are permitted by the symmetry of the system. It is expected that on large-scales  the polynomial model will be more accurate.
The non-local derivative terms in Eq.(\ref{eq:def_bias1}) are an unavoidable consequence of symmetry and will contribute on smaller scales.
Indeed, modification of gravity doesn't necessarily respect the symmetry under these transformations,
and, hence, in addition to non-local terms, scale-dependent terms will also be generated.

We can use Eq.(\ref{eq:def_bias1}) to relate the generating function  ${\cG}_{\delta_h}$ of $\delta_h$
in terms of the generating function of other variables. 
The generating functions are defined in Eq.(\ref{eq:gen}).
Using Eq.(\ref{eq:gen_1}) we arrive at:
\ben
&&{\cG}_{\delta_h}\equiv  b(\cG_{\delta})= b_1\cG_{\delta} + {1\over 2!}b_2[\cG_{\delta}]^2 + {1\over 3!}b_3[\cG_\delta]^3 +
{1\over 2!}b_{s^2} {\cG_s^2} + b_{\psi}{\cG_\psi} + b_{st} \cG_{s\cdot t} + \nn \\
&&\quad\quad\quad  + b_{\nabla^2\delta} {\cal G}_{\nabla^2\delta} + b_{s^2\nabla^2} {\cal G}_{s^2\nabla^2}  +  
b_{s^2\nabla^4}{\cal G}_{s\nabla^4} \cdots.
\label{eq:series}
\een
This is one of the main result of this paper. Following the derivations outlined in Appendix-\ref{sec:append_sym}  it can be shown that many of the terms
involving the following generating functions vanish.
\ben
{\cal G}_{s^2} =0; \quad {\cal G}_{t^2}=0; \quad {\cal G}_{s\cdot t} =0; \quad {\cal G}_{s^2\nabla^2}=0; \quad {\cal G}_{s\nabla^4}=0.
\een
An important conclusion from this analysis is that the higher-order statistics of tracers are independent
of $b_{s^2}, b_{t^2}, b_{st}$ and other similar constructs to an {\em arbitrary order} though they do contribute to the variance.

%The coefficient $t_{ij}$ is traceless. It vanishes at first-order in perturbation. Also, $t^2$
%is a fourth-order parameter.
These quantities are the well known invariants that are the result of  
inherent extended Galilean and Lifshitz symmetries in the dynamic equations\citep{ChecnScocSheth,Baldauf,Kehagias}.
The corresponding expressions in terms of the generating functions take the following form:
\ben
\cG_{\eta} = \cG_{\theta}-\cG_{\delta}; \quad \cG_{\psi} = \cG_{\eta} -{2 \over 7}\cG_{s^2}
+{4 \over 21} [\cG_{\delta}]^2.
\een
It is possible to show using the properties of the generating functions in Eq.(\ref{eq:gen_1})-Eq.(\ref{eq:gen_n}) we have:
\ben
\cG_{\psi} = {\cal G}_{\eta} - {4\over 21}[\cG_{\delta}]^2.
\een
The Eulerian bias $b_{\ell}$ and the Lagrangian bias $b_{\ell}^L$
are related by the following expression \citep{halo}:  
\ben
b_1 = 1+ b_1^L;\quad b_2 = b_2^L + {8\over 21}b_1^L ;\quad b_{s^2} =-{4\over 7}b_1^L; 
\quad b_{\psi}=-{1\over 2}b_1^L; \quad b_{s t}=-{5\over 7}b_1^L; 
\label{eq:zero}
\een
It is recognized that galaxy formation is a stochastic process \citep{DekelLahav}.
A more general expression of Eq.(\ref{eq:def_bias1}) should include the stochasticity of galaxy formation with $\delta$ replaced by $\delta + n$ with
$n$ given by a more generic series expansion:
\bes
\ben
\label{eq:noise}
&& n= b_{\epsilon}\epsilon + b_{\delta\epsilon} \delta\epsilon + {1\over 2}b_{\delta^2\epsilon} \delta^2\epsilon + {1\over 2} b_{s^2\epsilon}s^2 \epsilon
+ {1\over 2} b_{\epsilon^2} \epsilon^2 + {1\over 3} b_{\delta\epsilon^2} \delta\epsilon^2 +{1\over 3}b_{\epsilon^3}\epsilon^3 + \dots; \\
&& {\cal G}_n = b_\epsilon {\cal G}_{\epsilon} + b_{\delta\epsilon} {\cal G}_{\delta}{\cal G}_{\epsilon} +  
{1\over 2}b_{\delta^2\epsilon} {\cal G}^2_\delta{\cal G}_\epsilon +{1\over 2}b_{s^2\epsilon} {\cal G}_{s^2}{\cal G}_\epsilon +
{1\over 2}b_{\delta\epsilon^2}{\cal G}_{\delta}{\cal G}^2_{\epsilon}+ {1\over 2}b_{\epsilon^3}{\cal G}^3_{\epsilon}.
\label{eq:noise1}
\een
\ees
The generating function ${\cal G}_\delta$ in Eq.(\ref{eq:series}) will be replaced by ${\cal G}_{\delta+n} = {\cal G}_{\delta}+{\cal G}_n$
Indeed, following the same arguments ${\cal G}_{s^2}=0$ and rest of the terms can be expressed in terms of ${\cal G}_{\delta}$ and ${\cal G}_{\epsilon}$.
If we assume $\epsilon$ to be Gaussian the expressions can be further simplified. 

It is also possible to consider a biasing model where the halo over-density at a given location
is assumed to be a function of dark matter fields and their higher-order derivatives
along the entire past trajectory. Such an expression would not only be 
non-local in space but also in time. It can however be argued that dominant perturbative expressions
can be factorized in spatial and temporal dependence. The integration of the temporal part
can be performed without distorting the spatial dependence. Thus only the parameters
defining the bias will get normalized.

The Eulerian bias in Eq.(\ref{eq:zero}) was expressed in terms of Lagrangian bias using order-by-order perturbative
calculation. The subset of polynomial bias coefficients $b_\ell$ can also be derived using the following mapping
that relates the Eulerian density contrast $\delta^{E}_{h}(\tau)$ for halos with the
Lagrangian density contrast $\delta^{L}_{h}(\tau)$:
\bes
\ben
\label{eq:spherical1}
&& 1+\delta^{E}_{h}(\tau)= (1+\delta_{})[1+\delta^{L}_h(\tau)];\\
\label{eq:spherical2}
&& 1+{\cal G}^{(h)}_E = (1+{\cal G}_\delta)(1+{\cal G}^L_{\delta});\\
&& \delta^{E}_h({\bf x},\tau) = \sum^{\infty}_{\ell=1} 
{b^E_\ell \over \ell!}[\delta({\bf x}, \tau)]^\ell; \quad 
\delta^{L}_h({\bf x},\tau) = \sum_{\ell=1}^{\infty} {b^L_\ell \over \ell!}[\delta^L({\bf x},\tau)]^\ell; \quad \\
&& {\cal G}^h_L(\tau) = \sum^{\infty}_{\ell=1} b^L_\ell {[{\cal G}_{\delta}(\tau)]^\ell\over \ell!}; \quad 
{\cal G}^h_E(\tau) = \sum^{\infty}_{\ell=1} b^E_\ell {[{\cal G}^L_\delta(\tau)]^\ell\over \ell!}; 
\quad {\cal G}_{\delta}(\tau) = \sum^{\infty}_{\ell=1} {\nu_\ell \over \ell!}\tau^{\ell}.
\label{eq:spherical3}
\een
\ees
We also express the Lagrangian and Eulerian generating functions as:
\ben
%\delta^L({\bf x},\tau) = \sum^{\infty}_{\ell=1} a_\ell^I [\delta^E({\bf x},\tau)]^\ell;
{\cal G}^L_{\delta}(\tau)=\tau =\sum^{\infty}_{\ell=1} a_\ell^I [{\cal G}_\delta(\tau)]^\ell.
\een
The above expansion is an inverse series of ${\cal G}_\delta$.
Using Eq.(\ref{eq:spherical2}) in Eq.(\ref{eq:spherical3}) we arrive at the following relations \citep{HJM1,HM}:
\bes
\ben
\label{eq:b1}
&& b^E_1(\tau) = 1+ b_1^L(\tau);\\
&& b^E_2(\tau) = 2(1+a^{I}_2)b_1^L(\tau) + b_2^L(\tau);\\
&& b^E_3(\tau) = 6(a^{I}_2+a^{I}_3)b_1^L(\tau) + 3(1+2a^{I}_2)b_2^L(\tau) + b_3^L(\tau).
\label{eq:b3}
\een
\ees
However, we would like to point out that in our derivation we have not assumed a spherical collapse model at any stage.
The $a_n$ parameters above are related to the $\nu_n$ parameters defined before $a_n=n!\nu_n$ which are determined by solving the dynamical equations 
Eq.(\ref{eq:gen1})-Eq.(\ref{eq:g3}).
\ben
%&& a_1= 1, \quad a_2 = {17 \over 21}, \quad a_3 = {341 \over 567}, \quad a_4 = {55805 \over 130977}, \cdots \\
%%&& a^I_1 = 1,\quad a^I_2 = -{17 \over 21}, \quad a^I_3 = {2815 \over 3969};  \quad a^I_4 = -{590725 \over 916839}, \cdots.
&& {a_n}= \left \{1, {17\over 21}, {341\over 567}, {55805\over 130977}, \cdots \right \}; \quad
{a^I_n}= \left \{1, -{17\over 21}, {2815\over 3969}, -{590725\over 916839}, \cdots \right \}.
\een
The coefficients ${a^I_n}$ are the coefficients of the inverse series.
Taylor expanding ${\cal G}_{\delta}^{\rm ZA}$ and  ${\cal G}_{\delta}^{\rm PZA}$ 
and replacing the $a_n$ coefficients in Eq.(\ref{eq:b1})-Eq.(\ref{eq:b3})
with the $\mu_n$ coefficients will produce the resulting $b_n(\tau)$ parameters for the Zel'dovich (ZA) or post Zel'dovich approximation (PZA)
(see Appendix-\ref{sec:append_LPT} for a detailed discussion).  
One important point is probably worth mentioning here. Unlike previous derivations, e.g. \citep{HJM1,HM},
the above derivation is directly derived from
of Euler, Continuity and Poisson given in Eq.(\ref{eq:gen1})-Eq.(\ref{eq:g3}).
%natuarally reproduces the generating function ${\cal G}_{\delta}$ which obeys the dynamics of spherical collpase.
%Use of genering function underlines the
%fact that the results are more fundamental than that and is a direct outcome
%of perturbative dynamics of a collisionless system.

Thus at this level, all the coefficients that describe the generating function of the so-called {\em proto-halos} 
defined in Eq.(\ref{eq:series}) can be expressed in terms of the coefficients 
$b_1^L$ and $b_2^L$. These coefficients can be derived using a halo model based approach \citep{Saito}:
%\ben
%&& b_1^L = {1\over \bar n}{\partial n \over \partial \delta_1} = -{1\over \bar n}{2\nu\over \delta_c}{\partial n \over \partial\nu}\\
%&& b_2^L = {1\over \bar n}{\partial^2 n \over \partial \delta_1^2} = {4 \over \bar n}{\nu^2 \over \delta^2_c}{\partial^2 n\over \partial \nu^2}
%+ {2\over \bar n}{\nu \over \delta_c^2}{\partial n \over \partial \nu}.\\
%&& {1\over \bar n}{\partial n \over \partial \nu}= -{q\nu-1 \over 2\nu} - {p \over \nu(1+(q\nu)^p)}\\
%&& {1\over \bar n}{\partial^2 n \over \partial \nu^2} = {p^2 +\nu p q\over \nu^2{1+(q\nu)^p}} + {(q\nu)^2 - 2pq -1 \over 4\nu^2}.
%\een

An important conclusion of this section would thus be that in a non-local bias model the
clustering of halos only depend on the local Lagrangian bias parameters $b^L_{\ell}$ and clustering
of density $\delta$  and the divergence velocity field $\theta$ 
that are characterized by the
vertices $\nu_n$ or $\mu_n$. 
%and $\mu_n$ equivalently $\cG_{\theta}$ respectively. 
This extends the result presented in ref.\citep{Munshi_IBIT}.
Next we will consider the case of scale-dependent bias.
%   
%
%
%
%
%
%
