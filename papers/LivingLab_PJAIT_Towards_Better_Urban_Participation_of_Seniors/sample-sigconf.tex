%%%% Proceedings format for most of ACM conferences (with the exceptions listed below) and all ICPS volumes.
\documentclass[sigconf]{acmart}
%%%% As of March 2017, [siggraph] is no longer used. Please use sigconf (above) for SIGGRAPH conferences.

%%%% Proceedings format for SIGPLAN conferences 
% \documentclass[sigplan, anonymous, review]{acmart}

%%%% Proceedings format for SIGCHI conferences
% \documentclass[sigchi, review]{acmart}

%%%% To use the SIGCHI extended abstract template, please visit
% https://www.overleaf.com/read/zzzfqvkmrfzn

\usepackage{booktabs} % For formal tables
%for todo
\usepackage[english]{babel} %for todo
\usepackage[utf8x]{inputenc} %for todo
\usepackage[colorinlistoftodos]{todonotes} %for todo

% Copyright
%\setcopyright{none}
%\setcopyright{acmcopyright}
%\setcopyright{acmlicensed}
\setcopyright{rightsretained}
%\setcopyright{usgov}
%\setcopyright{usgovmixed}
%\setcopyright{cagov}
%\setcopyright{cagovmixed}


% % DOI
% \acmDOI{10.475/123_4}

% % ISBN
% \acmISBN{123-4567-24-567/08/06}

% % Conference
% \acmConference[WI-IAT 2017]{IEEE/WIC/ACM International Conference on Web Intelligence 2017}{August 2017}{Leipzig, Germany} 
% % \acmYear{1997}
% % \copyrightyear{2016}

% % \acmPrice{15.00}


\begin{document}
\title{LivingLab PJAIT: Towards Better Urban Participation of Seniors}


% \titlenote{Produces the permission block, and
%   copyright information}
% \subtitle{Extended Abstract}
% \subtitlenote{The full version of the author's guide is available as
%   \texttt{acmart.pdf} document}


\author{Wies\l{}aw Kope\'{c}}
% \authornote{author footnote}
% \orcid{1234-5678-9012}
\affiliation{%
  \institution{Polish-Japanese Academy of Information Technology}
  \streetaddress{86 Koszykowa str.}
  \postcode{02-008}
  \city{Warsaw}
  \country{Poland}
}
\email{wieslaw.kopec@pja.edu.pl}

\author{Kinga Skorupska}
\affiliation{%
  \institution{Polish-Japanese Academy of Information Technology}
  \streetaddress{86 Koszykowa str.}
  \postcode{02-008}
  \city{Warsaw}
  \country{Poland}
}
\email{kinga.skorupska@pja.edu.pl}

\author{Anna Jaskulska}
\affiliation{%
  \institution{Polish-Japanese Academy of Information Technology}
  \streetaddress{86 Koszykowa str.}
  \postcode{02-008}
  \city{Warsaw}
  \country{Poland}
}
\email{anna.jaskulska@pja.edu.pl}

\author{Katarzyna Abramczuk}
\affiliation{%
  \institution{Warsaw University}
  \streetaddress{Krakowskie Przedmie\'{s}cie 26/28}
  \postcode{00-927}
  \city{Warsaw}
  \country{Poland}
}
\email{k.abramczuk@uw.edu.pl}


\author{Radoslaw Nielek}
\affiliation{%
  \institution{Polish-Japanese Academy of Information Technology}
  \streetaddress{86 Koszykowa str.}
  \postcode{02-008}
  \city{Warsaw}
  \country{Poland}
}
\email{nielek@pja.edu.pl}

\author{Adam Wierzbicki}
\affiliation{%
  \institution{Polish-Japanese Academy of Information Technology}
  \streetaddress{86 Koszykowa str.}
  \postcode{02-008}
  \city{Warsaw}
  \country{Poland}
}
\email{adamw@pja.edu.pl}

% The default list of authors is too long for headers}
\renewcommand{\shortauthors}{W. Kopec et al.}

\begin{abstract}
In this paper we provide a brief summary of development LivingLab PJAIT as an attempt to establish a comprehensive and sustainable ICT-based solution for empowerment of elderly communities towards better urban participation of seniors. We report on our various endeavors for better involvement and participation of older adults in urban life by lowering ICT barriers, encouraging social inclusion, intergenerational interaction, physical activity and engaging older adults in the process of development of ICT solutions. We report on a model and assumptions of the LivingLab PJAIT as well as a number of activities created and implemented for LivingLab participants: from ICT courses, both traditional and e-learning, through on-line crowdsourcing tasks, to blended activities of different forms and complexity. We also provide conclusions on the lessons learned in the process and some future plans, including solutions for better senior urban participation and citizen science.


%\footnote{This is an abstract footnote}
\end{abstract}


% The code below should be generated by the tool at
% http://dl.acm.org/ccs.cfm
% Please copy and paste the code instead of the example below. 

\copyrightyear{2017} 
\acmYear{2017} 
\setcopyright{acmlicensed}
\acmConference{WI '17}{August 23-26, 2017}{Leipzig, Germany}\acmPrice{15.00}\acmDOI{10.1145/3106426.3109040}
\acmISBN{978-1-4503-4951-2/17/08}

\begin{CCSXML}
<ccs2012>
<concept>
<concept_id>10003120.10003121.10011748</concept_id>
<concept_desc>Human-centered computing~Empirical studies in HCI</concept_desc>
<concept_significance>500</concept_significance>
</concept>
<concept>
<concept_id>10003120.10003123.10010860.10010859</concept_id>
<concept_desc>Human-centered computing~User centered design</concept_desc>
<concept_significance>500</concept_significance>
</concept>
<concept>
<concept_id>10003120.10003123.10010860.10010911</concept_id>
<concept_desc>Human-centered computing~Participatory design</concept_desc>
<concept_significance>500</concept_significance>
</concept>
<concept>
<concept_id>10003120.10003123.10011759</concept_id>
<concept_desc>Human-centered computing~Empirical studies in interaction design</concept_desc>
<concept_significance>500</concept_significance>
</concept>
<concept>
<concept_id>10003120.10003130.10003233.10003301</concept_id>
<concept_desc>Human-centered computing~Wikis</concept_desc>
<concept_significance>500</concept_significance>
</concept>
<concept>
<concept_id>10003120.10003121.10003122.10010854</concept_id>
<concept_desc>Human-centered computing~Usability testing</concept_desc>
<concept_significance>300</concept_significance>
</concept>
<concept>
<concept_id>10003120.10003121.10003124.10011751</concept_id>
<concept_desc>Human-centered computing~Collaborative interaction</concept_desc>
<concept_significance>300</concept_significance>
</concept>
<concept>
<concept_id>10003120.10003123.10010860.10010877</concept_id>
<concept_desc>Human-centered computing~Activity centered design</concept_desc>
<concept_significance>300</concept_significance>
</concept>
<concept>
<concept_id>10003120.10003130.10011762</concept_id>
<concept_desc>Human-centered computing~Empirical studies in collaborative and social computing</concept_desc>
<concept_significance>300</concept_significance>
</concept>
</ccs2012>
\end{CCSXML}

\ccsdesc[500]{Human-centered computing~Empirical studies in HCI}
\ccsdesc[500]{Human-centered computing~User centered design}
\ccsdesc[500]{Human-centered computing~Participatory design}
\ccsdesc[500]{Human-centered computing~Empirical studies in interaction design}
\ccsdesc[500]{Human-centered computing~Wikis}
\ccsdesc[300]{Human-centered computing~Usability testing}
\ccsdesc[300]{Human-centered computing~Collaborative interaction}
\ccsdesc[300]{Human-centered computing~Activity centered design}
\ccsdesc[300]{Human-centered computing~Empirical studies in collaborative and social computing}

% % We no longer use \terms command
%\terms{Theory}

\keywords{living lab, older adults, crowdsourcing, participatory design, social design, intergenerational interaction, social inclusion}

%% Used in some conference proceedings e.g. sigplan and sigchi
% \begin{teaserfigure}
%   \includegraphics[width=\textwidth]{sampleteaser}
%   \caption{This is a teaser}
%   \label{fig:teaser}
% \end{teaserfigure}

\maketitle

% \section{Introduction}
The interrelated fields of machine learning (ML), and artificial neural networks (ANN) have grown significantly in previous decades due to the availability of powerful computing systems to train and simulate large scale ANNs within reasonable time-scales, as well as the abundance of data available to train such networks in recent years. The resulting research has realized a bevy of ANN architectures that have performed incredible feats including a wide range of classification problems, and various recognition tasks.

Most ML techniques in-use today rely on supervised learning, where the systems are trained on patterns with a known desired output, or label. However, intelligent biological systems exhibit unsupervised learning whereby statistically correlated input modalities are associated within an internal model used for probabilistic inference and decision making \cite{buesing2011}. One interesting class of unsupervised learning approaches that has been extensively researched is the Restricted Boltzmann machine (RBM) \cite{hinton2006}. RBMs can be hierarchically organized to realize deep belief networks (DBNs) that have demonstrated unsupervised learning abilities, such as natural language understanding \cite{Sarikaya2014}. Most RBM and DBN research has focused on software implementations, which provides flexibility, but requires significant execution time and energy due to large matrix multiplications that are relatively inefficient when implemented on standard Von-Neumann architectures due to the memory-processor bandwidth bottleneck when compared to hardware-based in-memory computing approaches \cite{Merolla2014}. Thus, research into hardware-based RBM designs has recently sought to alleviate these constraints. 

Previous hardware-based RBM implementations have aimed to overcome software limitations by utilizing FPGAs \cite{Kim2010,Ly2010} and stochastic CMOS \cite{Ardakani2017}. In recent years, emerging technologies such as resistive RAM (RRAM) \cite{SHERI2015,Bojnordi2016} and phase change memory (PCM) \cite{Eryilmaz2016} are proposed to be leveraged within the DBN architecture as weighted connections interconnecting building blocks in RBMs. While most of the previous hybrid Memristor/CMOS designs focus on improving the synapse behaviors, the work presented herein overcomes many of the preceding challenges by utilizing a novel spintronic p-bit device that leverages intrinsic thermal noise within low energy barrier nanomagnets to provide a natural building block for RBMs within a compact and low-energy package. The contribution of this paper is go to beyond using low-energy barrier magnetic tunnel junctions (MTJs), as has been previously introduced for a neuron in spiking neuromorphic systems \cite{sengupta2016magnetic,Sengupta2016prob}. To the best of our knowledge this paper is the first effort to use MTJs with near-zero energy barriers as neurons within an RBM implementation. Additionally, various parameters of a hybrid CMOS/spin weight array structure are investigated for metrics of power dissipation, and error rate using the MNIST digit recognition benchmarks.
 


\section{Fundamentals of RBM}
Boltzmann Machines (BM) are a class of recurrent stochastic ANNs with binary nodes whereby each possible state of the network, \textit{v}, has an energy determined by the undirected connection weights between nodes and the node bias as described by (1), where $s_i^v$ is the state of node \textit{i} in \textit{v}, \textit{b\textsubscript{i}} is the bias, or intrinsic excitability of node \textit{i}, and \textit{w\textsubscript{ij}} is the connection weight between nodes \textit{i} and \textit{j} \cite{ackley1985}.
\begin{equation}
  E(v) = -\sum_{i} s_i^v b_i -\sum_{i<j} s_i^v s_j^v w_{ij} 
\end{equation}

\begin{equation}
  P(s_i = 1) = \sigma (b_i + \sum_{j} w_{ij} s_j)
\end{equation}

Each node in a BM has a probability to be in state 1 according to (2), where $\sigma$ is the logistic sigmoid function. BMs, when given enough time, will reach a Boltzmann distribution where the probability of the system being in state \textit{\textbf{v}} is found by $P(v) = \frac{e^{-E(v)}}{\sum_{u} e^{-E(u)}}$, where \textit{\textbf{u}} could be any possible state of the system. Thus, the system is most likely to be found in states that have the lowest associated energy.
%\begin{equation}
 % P(v) = \frac{e^{-E(v)}}{\sum_{u} e^{-E(u)}}
%\end{equation}
Restricted Boltzmann machines (RBMs) are BMs constrained to two fully-connected non-recurrent layers called the \textit{visible layer}, where salient inputs clamp nodes to output levels of either zero or one, and the \textit{hidden layer}, where associations between input vectors are learned. By enforcing the conditional independence of the visible and hidden layers, unbiased samples from the input can be obtained in one time-step, which enhances the learning process. 

The most widely used method for training RBMs is contrastive divergence (CD), which is an approximate gradient descent procedure using Gibbs sampling \cite{carreira2005}. CD operates in three phases: \textit{(1) Positive Phase:} A training input vector, \textbf{$v$}, is applied to the visible layer by clamping the nodes to either 1 or 0 levels, and the hidden layer is sampled, \textbf{$h$}. \textit{(2) Negative Phase:} by clamping the hidden layer to \textbf{$h$}, the reconstructed input layer is sampled, \textbf{$v'$}. Then, clamp the visible layer to v' and sample the hidden layer to obtain \textbf{$h'$}. \textit{(3)} Update the weights according to $\Delta W = \eta (vh^T-v'h'^T)$, where $\eta$ is the learning rate and \textbf{$W$} is the weight matrix. 

DBNs are realized when additional hidden layers are stacked on top of an RBM, and can be trained in a very similar way to RBMs. Essentially, training a DBN involves performing CD on the visible layer and the first hidden layer for as many steps as desired, then fixing those weights and moving up a hierarchy as follows. The first hidden layer is now viewed as a visible layer, while the second hidden layer acts as a hidden layer with respect to the CD procedure identified above. Next, another set of CD steps are performed, and then the process is repeated for each additional layer of the DBN.


\section{Spin-Based Building Block For RBM}
In this section, we provide a detailed description of the p-bit that provides the building block for our proposed spin-based BM architecture. Individual building blocks are interconnected by networks of memristive devices whose resistances can be programmed to provide the desired weights. For instance, in this paper, we will assume that the memristive devices are implemented using the three terminal spin-orbit torque (SOT)-driven domain wall motion (DWM) device proposed in \cite{Sengupta2016hybrid}.

\begin{figure}
\includegraphics[scale=0.12]{fig1.pdf}
\caption{Structure of a p-bit.}
\end{figure}

\begin{figure}
\includegraphics[scale=0.28]{fig2.pdf}
\caption{Time-averaged results over 100 ns for p-bit.}
\end{figure}

The activation function is achieved by a spintronic building block that has been used in the design of probabilistic spin logic devices (p-bits) for a wide variety of Boolean and non-Boolean problems \cite{Camsari2017,Faria2017,sutton2017,behin2016}. The basic functionality of the p-bit shown in Fig. 1 \cite{Camsari2017} is to produce a stochastic output whose steady-state probability is modulated by an input current to generate a sigmoidal activation function. For instance, a high positive input current produces a stochastic output with a high probability of ``0'', and vice versa. In the absence of any input current, the device generates either 0 or VDD outputs with roughly equal probability of 0.5, as shown in Fig. 2. This device consists of a 3-terminal, spin-Hall driven MTJ \cite{Liu2012} that uses a circular, unstable nanomagnet ($\Delta \ll 40kT$), whereby its output is amplified by CMOS inverters as shown in Fig. 1. This MTJ with an unstable free layer can be fabricated using standard technology such that the surface anisotropy to achieve perpendicular magnetic anisotropy (PMA) that is not strong enough to overcome the demagnetizing field. Thus, the magnetization stochastically rotates in the plane, due to the presence of thermal fluctuations. 
 
The charge current that is injected to the spin-Hall layer creates a spin-current flowing into the circular FM (in the +y direction), which does not have an axis with any preferential geometry. The spin-polarization of this spin-current is in the ($\pm z$) direction, and pins the magnetization in the (+z) or (-z) direction depending on the direction of the charge current, through the spin-torque mechanism \cite{sutton2017}. The inherent physics of the spin-current driven low-barrier nanomagnet provides a natural sigmoidal function when a long time average of magnetization is taken. Through the tunneling magnetoresistance effect, a charge current flowing through the MTJ with a stable fixed layer detects the modulated magnetization as a voltage change. To achieve this, a small read voltage $V_R$ is applied between the $V+$ and $V-$ terminals through a reference resistance $R_0$, adjusted to the average conductance of the MTJ $(R_0^-1=GP+GAP/2)$ where $GP$ and $GAP$ represented conductance in parallel (P) and anti-parallel (AP) states, respectively. This voltage becomes an input to the CMOS inverters that are biased at the middle point of their DC operating point, creating a stochastic output whose probability can be tuned by the input charge current.  

\begin{table}[]
\centering
\small
\caption{Parameters for p-bit Based Activation Function.}
\label{tab:parameter}
\begin{tabular}{ccc}
\hline
Parameter & Description & Value  \\ \hline
\multicolumn{3}{l}{\textbf{Circular FM}} \\ \hline
%$M_S$ & \begin{tabular}[c]{@{}l@{}}Saturation magnetization\end{tabular} & $300 emu/cc$ \\
$\phi$ & Diameter & $100 nm$ \\
$t$  & Thickness  & $2 nm$   \\
$\alpha$ & Damping coefficient & $0.01$  \\ \hline
%$T$  & Temperature & $300 K$ \\
%$\Delta t$  & Timestep for sLLG & $0.25 ps$ \\ \hline
\multicolumn{3}{l}{\textbf{MTJ}} \\ \hline
$G0$ & Conductance & $150e^{-6} S$ \\
$P$  & Spin Polarization & $0.52$ \\ \hline
\multicolumn{3}{l}{\textbf{\begin{tabular}[c]{@{}l@{}}Giant Spin Hall Layer(GSHE)\end{tabular}}}      \\ \hline
$\lambda$ & Spin-diffusion length & $2.1 nm$ \\
$\theta$      & Spin Hall Angle  & $0.5$ \\
$Volume$   & $l \times w \times t$ & $100 nm \times 100 nm \times 3.15 nm$ \\ \hline
\end{tabular}
\vspace{-0.3cm}
\end{table}

\begin{figure*}
\includegraphics[height=2.2in, width=5.2in]{fig3.jpg}
\caption{Proposed $32 \times 32$ hybrid CMOS/spin-based weighted array structure for RBM implementation.}
\end{figure*}

Each component of the device is represented by an independent spin-circuit based on experimentally-benchmarked models that have been established in \cite{camsari2015modular} and simulated as a spin-circuit in a SPICE-like platform. Here, we obtain an analytical approximation to the time-averaged behavior of the output characteristics. We start by relating the charge current flowing in the spin Hall layer to the spin-current absorbed by the magnet, assuming short-circuit conditions for simplicity, i.e. 100\% spin absorption by the FM: 

\begin{equation}
  I_s/I_c = \beta = \frac{L}{t} (\theta) (1 - sech(\frac{t}{\lambda}))
\end{equation}
where $I_s$ is the spin-current, $I_c$ is the charge current, $\theta$ is the spin-Hall angle, $L$, $t$, $\lambda$ are the length, thickness and spin diffusion lengths for the spin-Hall layer. The length and width of the GSHE layer are assumed to be the same as the circular nanomagnet. With a suitable choice of the L and t, the spin-current generated can be greater in magnitude than the charge current generating ``gain.'' For the parameters used in this paper, which are listed in Table~\ref{tab:parameter}, the gain factor $\beta$ is $\sim 10$. Next, we approximate the behavior of the magnetization as a function of an input spin-current, polarized in the ($\pm z$) direction. For a magnet with only a PMA in the $\pm z$ direction, a distribution function at steady state can be written analytically as below, as long as the spin-current is also fully in the $\pm z$ direction:

\begin{equation}
  \rho (m_z) = \frac{1}{Z} exp(\Delta m_z^2 + 2 i_s m_z)
\end{equation}
where $Z$ is a normalization constant, $m_z$ is the magnetization along $+z$,  is the thermal barrier of the nanomagnet, and $i_s$ is a normalization quantity for the spin-current such that $i_s= I_s/(4q/ \hbar \alpha kT)$, $\alpha$ being the damping coefficient of the magnet, $q$ the electron charge and $\hbar$ the reduced Planck constant. It is possible to use (4) to obtain an average magnetization $<m_z> = \int_{-1}^{+1} d m_z m_z \rho (m_z) / \int_{-1}^{+1} d m_z \rho (m_z)$. Assuming $\Delta \ll kT$, $<m_z>$ can be evaluated to give the Langevin function, $<m_z> = L(i_s)$ where $L(x) = \frac{1}{x} - coth \frac{1}{x}$, which is an exact description for the average magnetization in the presence of a z-directed spin-current for a low barrier PMA magnet. 




In the present case, however, the nanomagnet has a circular shape with a strong in-plane anisotropy and no simple analytical formula can be derived, thus We use the Langevin function with a fitting parameter that adjusts the normalization current by a factor $\eta$, so that the modified normalization constant becomes $(4 q/\hbar \alpha kT)(\eta)$. This factor increases with elevating the shape anisotropy $(H_d \sim 4 \pi M_s)$ and becomes exactly one when there is no shape anisotropy. Once the magnetization and charge currents are related, we can approximate the output probability of the CMOS inverters by a phenomenological equation along with fitting parameter $\chi$ as follows, $p = \frac{V_{OUT}}{VDD} \approx \frac{1}{2} [1-tanh(\chi <m_z>)]$,   
%\begin{equation}
 %p = \frac{V_{OUT}}{VDD} \approx \frac{1}{2} [1-tanh(\chi <m_z>)]  
%\end{equation}
which allows us to relate the input charge current to the output probability, with physical parameters.  Fig. 2 shows the comparison of the full SPICE-model with respect to aforementioned equations showing good agreement with two fitting parameters $\eta$ and $\chi$, which fit the magnetization and CMOS components, respectively. 




\section{Proposed Weighted Array Design}
Figure 3 shows the structure of the weighted array proposed herein to implement the RBM architecture including the SOT-DWM based weighted connections and biases, as well as the p-bit based activation functions. Transmission gates (TGs) are utilized in write circuits within the bit cells of the weighted connection to adjust weights by moving the DW position. As investigated in \cite{zand2017}, TGs can provide energy-efficient and symmetric switching operation for SOT-based devices, which are desirable during the training phase. Table~\ref{tab:signaling} lists the required signaling for controlling the training and read operations in the weighted array structure. Herein, a chain of inverters are considered to drive signal lines, in which each successive inverter is twice as large as the previous one.

During the read operation, write word line (WWL) is connected to ground (GND) and the source line (SL) is in high impedance (Hi-Z) state, which disconnects the write path. The read word line (RWL) for each row is connected to VDD, which turns ON the read transistors in the weighted connection bit cell. The bit line (BL) will be connected to the input signal (VIN), which results in producing a current that affects the output probability of the p-bit device. The direction of the generated current relies on the VIN signal. In particular, since V- is supplied by a voltage source equal to VDD/2, if VIN is connected to VDD the injected current to the p-bit based activation function will have positive value, and if VIN is zero the input current will be negative. The amplitude of the generated current depends on the resistance of the weighted connection which is defined by the position of the DW in the SOT-DWM device. 

During the training operation, the RWL is connected to GND, which turns OFF the read transistors and disconnects the read path. The WWL is connected to an input pulse (VPULSE) signal which activates the write path for a short period of time. The duration of the VPULSE should be designed in a manner such that it can provide the desired learning rate, $\eta$, to the training circuit. For instance, a high VPULSE duration results in a significant change in the DW position in each training iteration, which effectively reduces the number of different resistive states that can be realized by the SOT-DWM device. Resistance of the weighted connections can be adjusted by the BL and SL signals, as listed in Table~\ref{tab:signaling}. A higher resistance leads to a smaller current injected to the p-bit device. Therefore, the input signal connected to the weighted connection will have lower impact on the output probability of the p-bit device, which means the input signal exhibits a lower weight. The bias nodes can also be adjusted similar to the weighted connection.  

\begin{table}[]
\centering
\small
\caption{Signaling to Control The Array Operations.}
\vspace{-0.2cm}
\label{tab:signaling}
\begin{tabular}{ccccccc}
\hline
Operation & WWL & RWL & BL  & SL   & V+   & V-    \\ \hline
\begin{tabular}[c]{@{}c@{}}Increase Weight\end{tabular} & VPULSE & GND & VDD & GND  & Hi-Z & Hi-Z  \\
\begin{tabular}[c]{@{}c@{}}Decrease Weight\end{tabular} & VPULSE & GND & GND & GND  & Hi-Z & Hi-Z  \\
Read & GND    & VDD & VIN & Hi-Z & VDD  & VDD/2 \\ \hline
\end{tabular}
\vspace{-0.3cm}
\end{table}


\begin{table}[]
\centering
\small
\caption{Relation between the input currents of activation functions and array size for $R_P = 1 M \Omega$.}
\vspace{-0.2cm}
\label{tab:arraysize}
\begin{tabular}{ccccc}
\hline
\multirow{2}{*}{Features}       & \multicolumn{4}{c}{Array Size}                                  \\ \cline{2-5} 
                                & $8 \times 8$ & $16 \times 16$ & $32 \times 32$ & $64 \times 64$ \\ \hline
Max. Positive Current $(\mu A)$ & 2.71         & 5.14           & 10.79          & 21.46          \\
Max. Negative Current $(\mu A)$ & 3.57         & 7.14           & 14.23          & 28.28          \\
Max. output ``0'' Probability   & 0.77         & 0.88           & 0.95           & 0.97           \\
Min. output ``0'' Probability   & 0.175        & 0.08           & 0.038          & 0.026          \\ \hline
\end{tabular}
\end{table}




\section{Simulation Results And Discussion}
To analyze the RBM implementation using the proposed p-bit device and the weighted array structure, we have utilized a hierarchical simulation framework including circuit-level and application-level simulations. In circuit level simulation, the behavioral models of the p-bit and SOT-DWM devices were leveraged in SPICE circuit simulations using 20nm CMOS technology with 0.9V nominal voltage to validate the functionality of the designed weighted array circuit. In application-level simulation, the results obtained from device-level and circuit-level simulations are used to implement a DBN architecture and analyze its behavior in MATLAB.


\begin{figure}
\includegraphics[scale=0.22]{fig4.pdf}
\caption{Weighted array power consumption versus the resistance of the weighted connections and array size.}
\vspace{-0.3cm}
\end{figure}


\subsection{Circuit-level simulation}
The device-level simulations shown in Fig. 2 verified a sigmoidal relation between the input current of the p-bit based activation function and its output probability. The shape of the activation on function is one of the major factors affecting the performance of the RBM. Therefore, we have provided comprehensive analyses on the impacts of weighted connection resistance and weighted array dimensions on the input currents of the p-bit based activation functions, and the power consumption of the weighted array. 



Table~\ref{tab:arraysize} lists the range of the activation function input currents for various weighted array dimensions, while the resistance of the SOT-DWM device in parallel state (RP) is constant and equals $1 M\Omega$.  The experimental results provided in [19, 28] exhibit that an MTJ resistance in the $M\Omega$ range can be obtained by increasing the oxide thickness in an MTJ structure. The highest positive and negative currents can be achieved while the weighted connections are in parallel state, i.e. lowest resistance, and all of the input voltages (VIN) are equal to VDD and GND, respectively. The difference between the amplitude of positive and negative currents in a given array size with constant RP is caused by the different pull-down and pull-up strengths in NMOS read transistors. The maximum and minimum output-level ``0'' probabilities are listed in Table~\ref{tab:arraysize}, which can be obtained according to the measured input currents and the sigmoidal activation function shown in Fig. 2. 





\begin{table}[]
\centering
\small
\caption{Relation between the input currents of activation functions and $R_P$ in a $32 \times 32$ array.}
\vspace{-0.2cm}
\label{tab:rp}
\begin{tabular}{ccccc}
\hline
\multirow{2}{*}{Features}       & \multicolumn{4}{c}{$R_P (M \Omega)$} \\ \cline{2-5} 
                                & 0.25    & 0.5     & 0.75    & 1      \\ \hline
Max. Positive Current $(\mu A)$ & 36.56   & 20.02   & 13.97   & 10.79  \\
Max. Negative Current $(\mu A)$ & 54.95   & 28.12   & 18.9    & 14.23  \\
Max. output ``0'' Probability   & 0.98    & 0.965   & 0.96    & 0.95   \\
Min. output ``0'' Probability   & 0.01    & 0.026   & 0.032   & 0.038  \\ \hline
\vspace{-0.4cm}
\end{tabular}
\end{table}

\begin{table*}[]
\centering
\small
\caption{Comparison between various RBM implementations with an emphasis on activation function structure.} 
\label{tab:compare}
\begin{tabular}{cccccccc} \hline

\iffalse %%%%%%%%%%%%%%%%%%
\begin{tabular}[c]{@{}c@{}}Design\end{tabular}                    & Kim et al.{[}9{]}  & Ly et al. {[}10{]} & Ardakani et al. {[}11{]} & Sheri et al. {[}12{]} & Bojnordi et al. {[}13{]} & Eryilmaz et al. {[}14{]} & \begin{tabular}[c]{@{}c@{}}Proposed\\ Herein\end{tabular}\\  \hline
\fi    %%%%%%%%%%%%%%%%%%%%%

\begin{tabular}[c]{@{}c@{}}Design\end{tabular}                    & {\cite{Kim2010}}  & {\cite{Ly2010}} & {\cite{Ardakani2017}} & {\cite{SHERI2015}} & {\cite{Bojnordi2016}} & {\cite{Eryilmaz2016}} & \begin{tabular}[c]{@{}c@{}}Proposed\\ Herein\end{tabular}\\  \hline

\begin{tabular}[c]{@{}c@{}}Weighted\\ Connection\end{tabular}        & \begin{tabular}[c]{@{}c@{}}Embedded\\ multipliers \end{tabular}  & \begin{tabular}[c]{@{}c@{}}Embedded\\ multipliers \end{tabular} & \begin{tabular}[c]{@{}c@{}}- LFSR\\  - AND/OR gates\end{tabular} &  \begin{tabular}[c]{@{}c@{}}RRAM\\ memristor \end{tabular} & RRAM & PCM & SOT-DWM  \\ \hline

\begin{tabular}[c]{@{}c@{}}Activation\\ Function\end{tabular}        & \begin{tabular}[c]{@{}c@{}}CMOS-based \\ LUTs\end{tabular} & \begin{tabular}[c]{@{}c@{}}-2-kB BRAM\\ - Picewise Linear \\Interpolator\\ - Random number\\ Generator\end{tabular} & \begin{tabular}[c]{@{}c@{}} - LFSR\\ - Bit-wise AND\\ - tree adder\\ - FSM-based \\tanh unit\end{tabular} & Off-chip & \begin{tabular}[c]{@{}c@{}} - $64 \times 16$ LUTs\\ - Pseudo Random \\Number Generator\\ - Comparator\end{tabular} & Off-chip                 & \begin{tabular}[c]{@{}c@{}} - near-zero \\energy barrier \\probabilistic \\spin logic \\ device\end{tabular} \\ \hline
\begin{tabular}[c]{@{}c@{}}Energy per neuron\end{tabular}          & N/A & $\sim 10-100 nJ$ & $\sim 10-100 pJ$ & N/A & $\sim 1-10 nJ$ & N/A    & $\sim 1-10 fJ$  \\ \hline
\begin{tabular}[c]{@{}c@{}}Normalized area per neuron\end{tabular} & N/A  & $\sim 3000 \times$ & $\sim 90 \times$ & N/A & $\sim 1250 \times$ & N/A    & $\sim 1 \times$ \\                                                               
\hline
\end{tabular} 
                
\end{table*}


Moreover, Table~\ref{tab:rp} illustrates the relation between the $R_P$ values and input currents of the activation functions, and their corresponding output probabilities, for a given $32 \times 32$ weighted array. The lower RP resistance and higher array size provides a wider range of output probabilities which can increase the RBM performance. However, this is achieved at the cost of higher area and power consumption. The trade-offs between the array size, weighted connection resistance, and average power consumption in a single read operation is shown in Fig. 4. The lowest power consumption of 22.6 $\mu W$ is realized by an $8 \times 8$ array with $R_P = 1 M \Omega$. However, this array provides the narrowest range of the output probabilities, which significantly reduces the performance of the DBN.




\subsection{Application-level simulation}
In the application-level simulation, we have leveraged the obtained device and circuit behavioral models to simulate a DBN architecture for digit recognition. In particular, learning rate and the shape of the sigmoid activation function is extracted by the SOT-DWM and p-bit device-level simulations, respectively, while the circuit-level simulations defines the range of the output probabilities.  To evaluate the performance of the system, we have modified a MATLAB implementation of DBN by Tanaka and Okutomi \cite{Tanaka2014} and used the MNIST data set \cite{Lecun1998} including 60,000 and 10,000 sample images with $28 \times 28$ pixels for training and testing operations, respectively. We have used Error rate (ERR) metric to evaluate the performance of the DBN, as expressed by $ERR=N_F/N$, where, $N$ is the number of input data, $N_F$ is the number of false inference \cite{Tanaka2014}. 




The simplest model of the DBN that can be implemented for MNIST digit recognition consists 784 nodes in visible layer to handle $28 \times 28$ pixels of the input images, and 10 nodes in hidden layer representing the output classes. Fig. 5 shows the relation between the performance of various DBN topologies, and the number of input training samples ranging from 100 to 5,000, which is obtained using 1,000 test samples. The ERR and RMSE metrics can be improved by enlarging the DBN structure through increasing the number of hidden layers, as well as the number of nodes in each layer. This improvement is realized at the cost of larger area and power consumptions. Increasing the input training samples can improve the DBN performance as well, however it will quickly converge due to the limited weight values that can be provided by SOT-DWM based weighted connections. As shown in Fig. 5, some random behaviors are observed for networks with smaller sizes that are trained by lower number of training samples, which will be significantly reduced by increasing the number of training samples.

The simulation results exhibit the highest error rate of 36.8\% for a $784 \times 10$ DBN that is trained by 100 training samples. Meanwhile, the lowest error rate of 3.7\% was achieved using a $784 \times 800 \times 800 \times 10$ DBN trained by 5,000 input training samples. This illustrates that the recognition error rate can be decreased by increasing the number of hidden layers, and training samples, which is also realized at the cost of higher area and power overheads.


\begin{figure}
\includegraphics[scale=0.5]{fig5.jpg}
\vspace{-0.2cm}
\caption{ERR for various DBN topologies.}
\vspace{-0.4cm}
\end{figure}



\subsection{Disucussion}
Table~\ref{tab:compare} lists previous hardware-based RBM implementations, which have aimed to overcome software limitations by utilizing FPGAs \cite{Kim2010,Ly2010}, stochastic CMOS \cite{Ardakani2017}, and hybrid memristor-CMOS designs \cite{SHERI2015,Bojnordi2016,Eryilmaz2016}. FPGA implementations demonstrated RBM speedups of 25-145 over software implementations \cite{Kim2010,Ly2010}, but had significant constraints such as only realizing a single $128 \times 128$ RBM per FPGA chip, routing congestion, and clock frequencies limited to 100MHz \cite{Ly2010}. The stochastic CMOS-based RBM implementation proposed in \cite{Ardakani2017} leveraged the low-complexity of stochastic CMOS arithmetic to save area and power. However, the need for extremely long bit-stream lengths negate energy savings and lead to very long latencies. Additionally, a significant amount of Linear Feedback Shift Registers (LFSRs) were required to produce the uncorrelated input and weight bit-streams. In both the FPGA and stochastic CMOS designs, improvements were achieved by implementing parallel Boolean circuits such as multipliers and pseudo-random number generators for probabilistic behavior, which has significant area and energy overheads compared to leveraging the physical behaviors of emerging devices to perform the computation intrinsically. Bojnordi et al. \cite{Bojnordi2016} leveraged resistive RAM (RRAM) devices to implement efficient matrix multiplication for weighted products within Boltzmann machine applications, and demonstrated significant speedup of up to 100-fold over single-threaded cores and energy savings of over 10-fold. Similarly, Sheri et al. \cite{SHERI2015} and Eryilmaz et al. \cite{Eryilmaz2016} utilized RRAM and PCM devices to implement matrix multiplication, while the corresponding activation function circuitry is still based on the CMOS technology, which suffers from the aforementioned area and power consumption overheads.

While most of the previous hybrid Memristor/CMOS designs focus on improving the performance of weighted connections, the work presented herein overcomes many of the preceding challenges of generating sigmoidal probabilistic activation functions by utilizing a novel p-bit device that leverages intrinsic thermal noise within low energy barrier nanomagnets to provide a natural building block for RBMs within a compact and low-energy package. As listed in Table V, the proposed design can achieve approximately three orders of magnitude improvement in term of energy consumption compared to the most energy-efficient designs, while realizing at least ~90X device count reduction for considerable area savings. Note that these calculations do not take into account the weighted connections, since the main focus of this paper is on the activation function. While SOT-DWM devices are utilized herein for the weighted connections, any other memristive devices could be utilized without loss of generality.  

\section{Conclusion}
Herein, we developed a hybrid CMOS/spin-based DBN implementation using p-bit based activation functions modeled to produce a probabilistic output that can be modulated by an input current. The device-level simulations exhibited a sigmoid relation between the input currents and output probability. The SPICE model of the p-bit is used to design a weighted array structure to implement RBM. The circuit simulations showed that the performance of the array can be improved by enlarging the array size, as well as reducing the resistance of the weighted connections. However, these improvements are achieved at the cost of increased area and power consumption. For instance, the lowest power dissipation among the examined designs belongs to an $8 \times 8$ array with the maximum resistance of $1 M \Omega$ for weighted connections. However, this structure can only provide the output probabilities ranging from 0.175 to 0.77, which is the narrowest range among the examined designs resulting in a DBN implementation with lowest accuracy. 

Next, we simulated a DBN for digit recognition application in MATLAB using the device and circuit-level behavioral models. Trade-offs include the relations between the recognition accuracy of the DBN and the number of training samples, which are comparable to conventional hardware implementations. The recognition error rate decreased substantially for the first thousand training samples, regardless of the size of the array, while benefits continue through several thousand inputs. However, at least two hidden layers are desirable to achieve suitable error rates. Finally, we have provided a comparison between previous hardware-based RBM implementations and our design with an emphasis on the probabilistic activation function within the neuron structure. The results exhibited that the p-bit based activation function can achieve roughly three orders of magnitude energy improvement, while realizing at least 90X reduction in terms of device count, compared to the previous most energy-efficient designs. The research directions herein enable several intriguing possibilities for future work, including: (1) implementing the entire network in SPICE to obtain more robust results; 2) investigating the effect of process variation and noise on the accuracy of proposed architecture; 3) studying alternative devices with lower susceptibility to thermal noise; and 4) studying the scalability challenges of DBNs using larger datasets, e.g. CIFAR.  


\section{Acknowledgements}
This work was supported in part by the Center for Probabilistic Spin Logic for Low-Energy Boolean and Non-Boolean Computing  (CAPSL), one of the Nanoelectronic Computing Research (nCORE) Centers as task 2759.006, a Semiconductor Research Corporation (SRC) program sponsored by the NSF through CCF 1739635.

%This material is based upon work partially supported by the National Science Foundation under Grant No. 1739635. This work was also supported in part by the Center for Probabilistic Spin Logic for Low-Energy Boolean and Non-Boolean Computing  (CAPSL), one of the Nanoelectronic Computing Research (nCORE) Centers as task 2759.006, a Semiconductor Research Corporation (SRC) program sponsored by the NSF through CCF 1739635.


\iffalse    %%%%%%%%%%%%%%%%%%%%%%comments
We have already seen several typeface changes in this sample.  You can
indicate italicized words or phrases in your text with the command
\texttt{{\char'134}textit}; emboldening with the command
\texttt{{\char'134}textbf} and typewriter-style (for instance, for
computer code) with \texttt{{\char'134}texttt}.  But remember, you do
not have to indicate typestyle changes when such changes are part of
the \textit{structural} elements of your article; for instance, the
heading of this subsection will be in a sans serif\footnote{Another
  footnote here.  Let's make this a rather long one to see how it
  looks.} typeface, but that is handled by the document class file.
Take care with the use of\footnote{Another footnote.}  the
curly braces in typeface changes; they mark the beginning and end of
the text that is to be in the different typeface.

You can use whatever symbols, accented characters, or non-English
characters you need anywhere in your document; you can find a complete
list of what is available in the \textit{\LaTeX\ User's Guide}
\cite{Lamport:LaTeX}.
\fi %%%%%%%%%%%%%%%%%%%%%%%%%%%%%%%%%%%%%%%%%%%%

\iffalse %%%%%%%%%%%%%%%%%%%%%%%%%%%%%%%%%%%%%%%
\subsection{Math Equations}
You may want to display math equations in three distinct styles:
inline, numbered or non-numbered display.  Each of
the three are discussed in the next sections.

\subsubsection{Inline (In-text) Equations}
A formula that appears in the running text is called an
inline or in-text formula.  It is produced by the
\textbf{math} environment, which can be
invoked with the usual \texttt{{\char'134}begin\,\ldots{\char'134}end}
construction or with the short form \texttt{\$\,\ldots\$}. You
can use any of the symbols and structures,
from $\alpha$ to $\omega$, available in
\LaTeX~\cite{Lamport:LaTeX}; this section will simply show a
few examples of in-text equations in context. Notice how
this equation:
\begin{math}
  \lim_{n\rightarrow \infty}x=0
\end{math},
set here in in-line math style, looks slightly different when
set in display style.  (See next section).
\fi %%%%%%%%%%%%%%%%%%%%%%%%%%%%%%%%%%%%%

\iffalse%%%%%%%%%%%%%%%%%%%%%%%%%%%%%%%%%%%%%%%%%%%%%%%%%%%%%%%%%%%%%%%%%%%%%%%%%%%%%%%%%
\subsubsection{Display Equations}
A numbered display equation---one set off by vertical space from the
text and centered horizontally---is produced by the \textbf{equation}
environment. An unnumbered display equation is produced by the
\textbf{displaymath} environment.

Again, in either environment, you can use any of the symbols
and structures available in \LaTeX\@; this section will just
give a couple of examples of display equations in context.
First, consider the equation, shown as an inline equation above:
\begin{equation}
  \lim_{n\rightarrow \infty}x=0
\end{equation}
Notice how it is formatted somewhat differently in
the \textbf{displaymath}
environment.  Now, we'll enter an unnumbered equation:
\begin{displaymath}
  \sum_{i=0}^{\infty} x + 1
\end{displaymath}
and follow it with another numbered equation:
\begin{equation}
  \sum_{i=0}^{\infty}x_i=\int_{0}^{\pi+2} f
\end{equation}
just to demonstrate \LaTeX's able handling of numbering.

\subsection{Citations}
Citations to articles~\cite{bowman:reasoning,
clark:pct, braams:babel, herlihy:methodology},
conference proceedings~\cite{clark:pct} or maybe
books \cite{Lamport:LaTeX, salas:calculus} listed
in the Bibliography section of your
article will occur throughout the text of your article.
You should use BibTeX to automatically produce this bibliography;
you simply need to insert one of several citation commands with
a key of the item cited in the proper location in
the \texttt{.tex} file~\cite{Lamport:LaTeX}.
The key is a short reference you invent to uniquely
identify each work; in this sample document, the key is
the first author's surname and a
word from the title.  This identifying key is included
with each item in the \texttt{.bib} file for your article.

The details of the construction of the \texttt{.bib} file
are beyond the scope of this sample document, but more
information can be found in the \textit{Author's Guide},
and exhaustive details in the \textit{\LaTeX\ User's
Guide} by Lamport~\shortcite{Lamport:LaTeX}.

This article shows only the plainest form
of the citation command, using \texttt{{\char'134}cite}.

Some examples.  A paginated journal article \cite{Abril07}, an enumerated
journal article \cite{Cohen07}, a reference to an entire issue \cite{JCohen96},
a monograph (whole book) \cite{Kosiur01}, a monograph/whole book in a series (see 2a in spec. document)
\cite{Harel79}, a divisible-book such as an anthology or compilation \cite{Editor00}
followed by the same example, however we only output the series if the volume number is given
\cite{Editor00a} (so Editor00a's series should NOT be present since it has no vol. no.),
a chapter in a divisible book \cite{Spector90}, a chapter in a divisible book
in a series \cite{Douglass98}, a multi-volume work as book \cite{Knuth97},
an article in a proceedings (of a conference, symposium, workshop for example)
(paginated proceedings article) \cite{Andler79}, a proceedings article
with all possible elements \cite{Smith10}, an example of an enumerated
proceedings article \cite{VanGundy07},
an informally published work \cite{Harel78}, a doctoral dissertation \cite{Clarkson85},
a master's thesis: \cite{anisi03}, an online document / world wide web
resource \cite{Thornburg01, Ablamowicz07, Poker06}, a video game (Case 1) \cite{Obama08} and (Case 2) \cite{Novak03}
and \cite{Lee05} and (Case 3) a patent \cite{JoeScientist001},
work accepted for publication \cite{rous08}, 'YYYYb'-test for prolific author
\cite{SaeediMEJ10} and \cite{SaeediJETC10}. Other cites might contain
'duplicate' DOI and URLs (some SIAM articles) \cite{Kirschmer:2010:AEI:1958016.1958018}.
Boris / Barbara Beeton: multi-volume works as books
\cite{MR781536} and \cite{MR781537}.

A couple of citations with DOIs: \cite{2004:ITE:1009386.1010128,
  Kirschmer:2010:AEI:1958016.1958018}. 

Online citations: \cite{TUGInstmem, Thornburg01, CTANacmart}.  


\subsection{Tables}
Because tables cannot be split across pages, the best
placement for them is typically the top of the page
nearest their initial cite.  To
ensure this proper ``floating'' placement of tables, use the
environment \textbf{table} to enclose the table's contents and
the table caption.  The contents of the table itself must go
in the \textbf{tabular} environment, to
be aligned properly in rows and columns, with the desired
horizontal and vertical rules.  Again, detailed instructions
on \textbf{tabular} material
are found in the \textit{\LaTeX\ User's Guide}.

Immediately following this sentence is the point at which
Table~\ref{tab:freq} is included in the input file; compare the
placement of the table here with the table in the printed
output of this document.

\begin{table}
  \caption{Frequency of Special Characters}
  \label{tab:freq}
  \begin{tabular}{ccl}
    \toprule
    Non-English or Math&Frequency&Comments\\
    \midrule
    \O & 1 in 1,000& For Swedish names\\
    $\pi$ & 1 in 5& Common in math\\
    \$ & 4 in 5 & Used in business\\
    $\Psi^2_1$ & 1 in 40,000& Unexplained usage\\
  \bottomrule
\end{tabular}
\end{table}

To set a wider table, which takes up the whole width of the page's
live area, use the environment \textbf{table*} to enclose the table's
contents and the table caption.  As with a single-column table, this
wide table will ``float'' to a location deemed more desirable.
Immediately following this sentence is the point at which
Table~\ref{tab:commands} is included in the input file; again, it is
instructive to compare the placement of the table here with the table
in the printed output of this document.


\begin{table*}
  \caption{Some Typical Commands}
  \label{tab:commands}
  \begin{tabular}{ccl}
    \toprule
    Command &A Number & Comments\\
    \midrule
    \texttt{{\char'134}author} & 100& Author \\
    \texttt{{\char'134}table}& 300 & For tables\\
    \texttt{{\char'134}table*}& 400& For wider tables\\
    \bottomrule
  \end{tabular}
\end{table*}
% end the environment with {table*}, NOTE not {table}!

It is strongly recommended to use the package booktabs~\cite{Fear05}
and follow its main principles of typography with respect to tables:
\begin{enumerate}
\item Never, ever use vertical rules.
\item Never use double rules.
\end{enumerate}
It is also a good idea not to overuse horizontal rules.


\subsection{Figures}

Like tables, figures cannot be split across pages; the best placement
for them is typically the top or the bottom of the page nearest their
initial cite.  To ensure this proper ``floating'' placement of
figures, use the environment \textbf{figure} to enclose the figure and
its caption.

This sample document contains examples of \texttt{.eps} files to be
displayable with \LaTeX.  If you work with pdf\LaTeX, use files in the
\texttt{.pdf} format.  Note that most modern \TeX\ systems will convert
\texttt{.eps} to \texttt{.pdf} for you on the fly.  More details on
each of these are found in the \textit{Author's Guide}.

\begin{figure}
\includegraphics{fly}
\caption{A sample black and white graphic.}
\end{figure}

\begin{figure}
\includegraphics[height=1in, width=1in]{fly}
\caption{A sample black and white graphic
that has been resized with the \texttt{includegraphics} command.}
\end{figure}


As was the case with tables, you may want a figure that spans two
columns.  To do this, and still to ensure proper ``floating''
placement of tables, use the environment \textbf{figure*} to enclose
the figure and its caption.  And don't forget to end the environment
with \textbf{figure*}, not \textbf{figure}!

\begin{figure*}
\includegraphics{flies}
\caption{A sample black and white graphic
that needs to span two columns of text.}
\end{figure*}


\begin{figure}
\includegraphics[height=1in, width=1in]{rosette}
\caption{A sample black and white graphic that has
been resized with the \texttt{includegraphics} command.}
\end{figure}

\subsection{Theorem-like Constructs}

Other common constructs that may occur in your article are the forms
for logical constructs like theorems, axioms, corollaries and proofs.
ACM uses two types of these constructs:  theorem-like and
definition-like.

Here is a theorem:
\begin{theorem}
  Let $f$ be continuous on $[a,b]$.  If $G$ is
  an antiderivative for $f$ on $[a,b]$, then
  \begin{displaymath}
    \int^b_af(t)\,dt = G(b) - G(a).
  \end{displaymath}
\end{theorem}

Here is a definition:
\begin{definition}
  If $z$ is irrational, then by $e^z$ we mean the
  unique number that has
  logarithm $z$:
  \begin{displaymath}
    \log e^z = z.
  \end{displaymath}
\end{definition}

The pre-defined theorem-like constructs are \textbf{theorem},
\textbf{conjecture}, \textbf{proposition}, \textbf{lemma} and
\textbf{corollary}.  The pre-defined de\-fi\-ni\-ti\-on-like constructs are
\textbf{example} and \textbf{definition}.  You can add your own
constructs using the \textsl{amsthm} interface~\cite{Amsthm15}.  The
styles used in the \verb|\theoremstyle| command are \textbf{acmplain}
and \textbf{acmdefinition}.

Another construct is \textbf{proof}, for example,

\begin{proof}
  Suppose on the contrary there exists a real number $L$ such that
  \begin{displaymath}
    \lim_{x\rightarrow\infty} \frac{f(x)}{g(x)} = L.
  \end{displaymath}
  Then
  \begin{displaymath}
    l=\lim_{x\rightarrow c} f(x)
    = \lim_{x\rightarrow c}
    \left[ g{x} \cdot \frac{f(x)}{g(x)} \right ]
    = \lim_{x\rightarrow c} g(x) \cdot \lim_{x\rightarrow c}
    \frac{f(x)}{g(x)} = 0\cdot L = 0,
  \end{displaymath}
  which contradicts our assumption that $l\neq 0$.
\end{proof}

\section{Conclusions}
This paragraph will end the body of this sample document.
Remember that you might still have Acknowledgments or
Appendices; brief samples of these
follow.  There is still the Bibliography to deal with; and
we will make a disclaimer about that here: with the exception
of the reference to the \LaTeX\ book, the citations in
this paper are to articles which have nothing to
do with the present subject and are used as
examples only.
%\end{document}  % This is where a 'short' article might terminate



\appendix
%Appendix A
\section{Headings in Appendices}
The rules about hierarchical headings discussed above for
the body of the article are different in the appendices.
In the \textbf{appendix} environment, the command
\textbf{section} is used to
indicate the start of each Appendix, with alphabetic order
designation (i.e., the first is A, the second B, etc.) and
a title (if you include one).  So, if you need
hierarchical structure
\textit{within} an Appendix, start with \textbf{subsection} as the
highest level. Here is an outline of the body of this
document in Appendix-appropriate form:
\subsection{Introduction}
\subsection{The Body of the Paper}
\subsubsection{Type Changes and  Special Characters}
\subsubsection{Math Equations}
\paragraph{Inline (In-text) Equations}
\paragraph{Display Equations}
\subsubsection{Citations}
\subsubsection{Tables}
\subsubsection{Figures}
\subsubsection{Theorem-like Constructs}
\subsubsection*{A Caveat for the \TeX\ Expert}
\subsection{Conclusions}
\subsection{References}
Generated by bibtex from your \texttt{.bib} file.  Run latex,
then bibtex, then latex twice (to resolve references)
to create the \texttt{.bbl} file.  Insert that \texttt{.bbl}
file into the \texttt{.tex} source file and comment out
the command \texttt{{\char'134}thebibliography}.
% This next section command marks the start of
% Appendix B, and does not continue the present hierarchy
\section{More Help for the Hardy}

Of course, reading the source code is always useful.  The file
\path{acmart.pdf} contains both the user guide and the commented
code.

\begin{acks}
  The authors would like to thank Dr. Yuhua Li for providing the
  MATLAB code of the \textit{BEPS} method.

  The authors would also like to thank the anonymous referees for
  their valuable comments and helpful suggestions. The work is
  supported by the \grantsponsor{GS501100001809}{National Natural
    Science Foundation of
    China}{http://dx.doi.org/10.13039/501100001809} under Grant
  No.:~\grantnum{GS501100001809}{61273304}
  and~\grantnum[http://www.nnsf.cn/youngscientists]{GS501100001809}{Young
    Scientists' Support Program}.

\end{acks}
\fi%%%%%%%%%%%%%%%%%%%%%%%%%%%%%%%%%%%%%%%%%%%%%%%%%%%%%%%%%%%%%%%%%%%%%%%%%

% CURRENT VERSION

\section{Introduction}

As determined by Eurostat, in 2014 older people, aged 65+ constituted 18.5 \% of the entire EU-28 population, which is a 4\% increase when compared to the 1994 statistics. This is in line with trends observed in most developed countries, with Japan being the outlier as the most aged society by 2050\footnote{$http://www.un.org/en/development/desa/population/publications/pdf/ageing/WPA2015\\_Report.pdf$} . The United Nations estimate that in 2050 this figure will reach a global average of 20 \% and with the increasing lifespan and medical advancements the number of people over 80 will double.\cite{UNreport} What follows is that the current old-age dependency ratio, which for EU-28 stands at almost 29 \%, meaning that there are about four working adults for every senior, will further increase.

On one hand this presents a difficulty, as with the current tendency to exclude the elderly from some aspects of social participation, their voice is unheard and their needs are unmet, but on the other hand it is an opportunity to tap into their potential and resources, not only to tackle their own problems, but also to allow them to help solve the issues common to all groups in the society. 

This requires the development of an approach to allow the elderly to actively participate in the society - which increasingly relies on ICT skills - and to contribute their time and experience to solving the problems of aging societies, including a shortage of resources and overpopulation. This can be done by crowdsourcing insights from this age group that could be studied to develop better services and by involving them in the creation of the solutions themselves, which is facilitated by the Living Lab framework.

For this, seniors need to learn ICT skills and ways to collaborate in intergenerational groups. As we have observed, one of the most effective ways to achieve this is to offer a mix of online and offline activities and to teach ICT skills in reference to the real world, people and locations. Owing to the use of location based services and augmented-reality the interaction with mobile devices in urban settings can turn seniors into engaged users,  willing contributors, and eventually designers in the technological marketplace.

Because the design of public spaces is now of key importance, both due to the fast progressing urbanization and the high population density, at Living Lab at PJAIT, we propose to tap into the potential of senior crowdsourcing solutions in public space design. The elderly are the demographic that has the knowledge and resources to offer powerful insights into the changes needed to make public places serve their intended purpose as well as facilitate engagement and encourage physical activity outdoors. In this article we will report our activities within the Living Lab framework which, step by step, build our experience and potential towards greater social inclusion of older adults into the city life. This will be facilitated by gaining insights into the senior ICT-skill learning process, their motivation and by creating a senior community of learners, users, contributors, testers and designers.

The rest of the article is organized as follows. First we present related works, which is followed by a brief description of PJAIT LivingLab  framework project. Next, we present details on two fundamental parts of the LivingLab platform: e-courses and crowdsourcing tasks, followed by brief overview of different activites with participants of the PJAIT LivingLab . Finally, we discuss our findings and present plans for the future.

\section{Related Works}
As it was outlined above in the previous section facing the aging society problems should be strictly connected with the process of enhancing the participation of seniors in the urban life and their involvement in creation process of new solutions. There are many ways to achieve that goal, however, two major ones are closely related to the topic of the INTENSE special session of WI-IAT 2017 conference. 

The first topic is related to the interaction with smart urban objects and the second, to the development of devices and software connected to the crucial topics of safe and independent living in urban space. While the former is focused on current trends in smart cities, namely sensors and IoT, that are subject of investigation from many authors, including Koch \cite{kotteritzsch2016expand}, the latter is strictly connected to our approach of involving the older adults in the process of active participation in developing solutions which is connected with one of the fundamental concepts of contemporary design studies: participatory design.

\subsection{Human-centered approach}

The potential of participatory design, that is designing together with the end-users, is increasingly recognized as crucial in developing well-targeted solutions. This approach is also related in literature i.e. by Sanders \cite{sanders2008co} to another term known as \textit{co-design}. Although these terms have different origins, they refer to the same human-centered approach, which is widely used not only in software engineering, but also in fields ranging from architecture to the healthcare industry \cite{szebeko2010co}. These concepts put human beings at the center of the design process, although there is a small but significant distinction between traditional user-centered design and participatory design: the former refers to the process of designing for users, while the latter is related to design process with potential end-users \cite{sanders2002user,sanders2008co}. According to Sanders the next step in participatory design is connected with the process of designing by the users themselves. She also claims that participatory approach can be applied throughout the whole design cycle, even including the pre-design stage. On the other hand Ladner  \cite{ladner2015design} limits the definition of the scope of participatory design just to two of the four stages of standard iterative cycle of product and software development, namely to design and testing phase (excluding analysis and prototype stages) while user-centered design is limited by him only to the testing stage. Apart from some differences in definitions and scopes human-centered design is undoubtedly crucial to the development of sustainable solutions as an idea which, according to Ladner, underlies the universal design approach.

With seniors as the intended target group, it is even more important to practice co-design, since the questions of accessibility and user motivation are especially prominent. To this end Ladner \cite{ladner2015design} proposed another term: \textit{Design for Empowerment}, which involves the users in all stages of the design cycle, that is analysis, design, prototyping and testing. One difficulty with this approach when used for technological enabling solutions for the seniors is that most of them lack the ICT skills necessary to actively participate in these design stages.

\subsection{Living Labs framework concept}

Here, the concept of Living Labs comes into focus. The term \textit{Living Lab} was coined by William Mitchell from MIT \cite{niitamo2006state} and was used to refer to the real environment, like a home or a city, where routines and everyday life interactions of users and new technology can be observed and recorded to foster the process of designing new useful and acceptable products and services. Thus, the idea of LivingLab is inherently coupled with broad concept of user-centric research methodology known as human-centered design where users are no longer just subject of the functional tests but rather active participants  able to create value and contribute both content and technology. So, a LivingLab consists of an experimental environment devoted to developing and shaping new products or services in cooperation with users as co-producers or co-creators involved throughout the development process (\cite{ballon2005test,schumacher2007living}. In other words LivingLab is an open, sharing and collaborative framework rooted in the practice of participatory design which invites experimentation and facilitates the development of communities of stakeholders, enriching the product development process with surprising insights and hence, mitigating business risks.\cite{pallot2010living} These networks of partnerships ensure that Living Labs are a space where users are the focus and take part in developing solutions - therefore researching and educating the affiliated community become a viable investment.

Because the involvement of stakeholders is central to the process of co-design, the creation of technologically literate community able to, and willing to participate in the process of design is also must. To build this community Living Labs designing for seniors have to expand their focus from developing solutions, to preparing the users to participate in this process. As community training is resource-intensive it is necessary to not only develop learning solutions, but also means of allowing older adults to feel engaged and motivated, so that they stay as part of the Living Lab community of stakeholders for an extended period of time.

\subsection{Senior ICT learning}

Learning abilities are a combination of various aspects of human cognition. Unfortunately, many types of cognitive abilities are subject to age-related decline which reinforces the misconception of learning being primarily the domain of the young.

The most important factors that affect learning are an individual's information processing speed and the capacity of their working memory \cite{gamberini_cognition_????}. We have to notice that learning abilities are also influenced by a multitude of other factors which include both the learner's cognitive abilities and psychological variables such as motivation and self-confidence \cite{boulton-lewis_ageing_????}.

While the components of fluid intelligence such as working memory and acquisition of new information do indeed deteriorate with age, long-term semantic memory is generally preserved. Moreover, crystallized intelligence (associated with general knowledge, wisdom and expertise derived from accumulated experience) not only does not decline but may even benefit from aging \cite{naumanen_guiding_????}. An awareness of these age-related changes in human cognition may be the key to adapting learning programs and methods to better accommodate the needs of elderly learners. One example of such an accommodation could be the use of a learning format consisting of step-by-step instructions on performing the desired task. This has been shown to benefit senior learners, just as avoiding a lecture-based approach which has been proven to be far less effective in this age group \cite{haeggans_60s_2012}. 

\subsection{Senior motivation}

Older adults are generally considered to be more conscious of their needs than the younger generation and are interested in learning technology primarily to fulfill these needs \cite{djoub_ict_2013}. The most common uses mentioned by senior technology learners are related to utilizing ICT for communication with friends and family members (mainly via e-mail services), to supporting seniors' existing hobbies (e.g. through the use of online forums and web browsing) and dealing with everyday administrative activities such as operating a bank account \cite{boulton-lewis_ageing_????}, \cite{naumanen_practices_2008}.

Even though some seniors declare that one of their reasons for learning ICT-related skills is proving to themselves and others that they are capable of mastering technology, the majority of them are genuinely motivated by an appreciation of the benefits of using ICT in their everyday lives and consider learning it a necessity in the present times (e.g. to be able to contact and relate to their grandchildren) \cite{aula_learning_2004}.


This is why Living Lab at PJAIT first and foremost focused on developing teaching and learning activities and motivation research involving the elderly as well as the intergenerational cooperation of the, often younger, designers, application developers and seniors.

Well-designed teaching and learning activities not only enable the seniors to feel empowered and participate in the design process, but also present health benefits as staying mentally active can delay the onset of Alzheimer's and other age-related issues \cite{kotteritzsch2014adaptive}.

\section{LivingLab framework project}

LivingLab at the Polish-Japanese Academy of Information Technology is a long-term project, whose goals are to encourage social inclusion and active engagement of the elderly in social life by facilitating the development of ICT literacy among them and creating an active community of stakeholders who are both the beneficiaries and enablers of research into their problems. 
In order to expedite this and gain access to potential participants as well as a media outlet a long-term partnership with the City of Warsaw has been established, within the "Fall in love with mature Warsaw" action.

This has been initiated by offering the seniors online courses and interactive activities relevant to their needs and interests, to act as a gateway into the world of ICT solutions. Moreover, PJAIT Living Lab tests and implements crowdsourcing tools with the end goal being to encourage entrepreneurial interest and at the same time conduct research within the broad spectrum of issues connected to the problem of aging. 

To that end the e-Senior platform \footnote{$http://esenior.um.warszawa.pl/se_index.do$ site available in Polish} was created which consists of two pillars: e-learning courses and the crowdsourcing module. On this platform the registered seniors can take online courses and improve their ICT skills as well as participate in scientific research by answering surveys or signing up for offline activities. It is also the source of information about other Living Lab activities, such as location based games, Wikipedia content creation and participatory design workshops. 


\subsection{On the making of LivingLab}

\subsubsection{Teaching methodology}

The Nomad e-learning system formed the foundation of the e-Senior platform. The Nomad system a comprehensive e-learning solution developed and implemented by the Polish-Japanese Academy of Information Technology.  \footnote{$https://kursy.nomad.pja.edu.pl/$ site available in Polish and English; content for Polish students as for now}

\begin{figure}
\centering
\includegraphics[height=150px]{figures/nomad_stages.png}
\caption{PJAIT Nomad system -- e-learning process stages}
\label{fig:nomad}
\end{figure}

The teaching and learning methodology which underlies the design of the Nomad platform was informed by practices, such as predefined flow and learning process control. It incorporates interactive puzzles (teaser) recommended to increase engagement and make the effects of learning stick. The content creation process involves a few stages, with predefined roles for quality assurance (author, editor, reviewer and publisher) and is available with the use of an easy drag and drop interface to allow educators to create courses without knowledge of programming.

However, as often the case with online courses, the learning path appeared too rigid. The modification of adding a feature of a free-strucutred course without strict validation took many months. The learning process is oriented on progress and certification.

\subsubsection{Usability tests}

The e-Senior e-learning module and LivingLab crowdsourcing tasks module was tailored to the needs of the target group. We conducted series of usability tests with the users of the early prototype of LivingLab platform. The usability tests were performed along the predefined scheme with the use of the thinking aloud method to enhance the moderator's observation. In total six user took part in evaluating platform. 
Evaluation sessions were conducted along the same protocol and lasted about an hour. Each session consisted of a complete walktrough of typical users' activities on the platform. During the session participant's voice and face image were recorded alongside with the computer screen with the use of the Camtasia Studio software.


\subsection{E-courses module}

The e-learning courses available on the platform are designed using human-centered approach and answer older adults' needs in the area of ICT i.e. using the Internet, computer and electronic devices. They also make it easier for seniors to remain mentally stimulated by the process of learning new skills and performing interactive puzzle activities as well as solving progress tests. While the platform offers e-learning content, it lacks a usable community-building feature which, for senior citizens who are not social media users, is a drawback.

\begin{figure}
\centering
\includegraphics[height=150px]{figures/esenior_logon_screen.jpg}
\caption{e-Senior logon screen}
\label{fig:esenior_logon}
\end{figure}

\subsection{Crowdsourcing tasks module}

The crowdsourcing section lists research activities aimed towards the 60+ demographic in which they can take part. These can be both scientific and entrepreneurial and involve testing the platform, software designed for the elderly, filling out profiling surveys and participating in market research.

Crowdsourcing is a growing opportunity for both learning and productivity of older adults. There is a lot of proposed taxonomy of crowdsourcing, like in \cite{geiger2011managing}. There are many proposed dimensions in literature that can be related to process, task or stakeholders. In our LivingLab we refer to the approach presented in \cite{schenk2011towards}, where authors propose characteristics of three types of task: simple, complex and creative. We distinguish two kind of tasks: predefined and free flow tasks.

Predefined tasks are strictly described and can be taken or assigned to workers. The most important here is assignment, not some greater goal. There is usually strictly described criteria and end result of such task. Those are similar to simple and complex tasks from \cite{schenk2011towards}. Examples of such tasks are those in ReCaptcha, OpenStreetMap, Amazon mTurk.

Free flow tasks, where tasks can be not described, criteria and end results can be undefined and participation in such tasks is usually open for everyone and not "assigned". Here, more important is goal of collective work. This include "creative" tasks from \cite{schenk2011towards}, but in slightly different meaning described before. Those are often (but not only) community-oriented tasks \cite{rouse2010preliminary}, where participants are building some kind knowledge base, common value etc. Examples of such tasks are participating in question and answer portals (Yahoo Answers, Stack Overflow)
or various human-centered activities from user content creation like editing Wikipedia to complex and creative tasks related to software development process with the use of participatory approach.

\begin{figure}
\centering
\includegraphics[height=150px]{figures/esenior_main_screen.jpg}
\caption{e-Senior main panel}
\label{fig:esenior_main}
\end{figure}

\section{Activities}

The two pillars described above have been incorporated and are being tested and developed. In this section we present the data and conclusions concerning the performance of educational and crowdsourcing programs available on the platform.

\subsection{E-courses and quizzes}

The courses are based on the model developed for the Nomad platform but with adjusted controls and design.

The e-Senior platform users are recruited thanks to the long term strategic framework agreement between the city of Warsaw and PJAIT, which provides support in technologically literate community building until the year 2020. Moreover, PJAIT Living Lab cooperates with numerous NGOs. This ensures that there are more seniors willing to participate than there are available places in the courses.

The educational cycle offered to participating seniors involves stationary computer courses, which are concluded by a ICT skills test. Those who pass the test are introduced to the platform, while every participating senior can request private IT consultations. 

\begin{figure}
\centering
\includegraphics[height=140px]{figures/introductory_workshops.jpg}
\caption{Introductory workshops for seniors}
\label{fig:introductory}
\end{figure}

In 2016 we have started the process of modifying the offered courses to provide learning in smaller chunks and to enable non-linear education with learning pills. To enable this the learning platform had to be modified to allow for a less rigid creation of the learning path for the students based on profiling, complex event processing and incorporating a motivational system to follow the existing motivational frameworks.

\begin{figure}
\centering
\includegraphics[height=150px]{figures/consultations.jpg}
\caption{Consultations for seniors}
\label{fig:consultations}
\end{figure}

Since 2017 we have been researching motivational frameworks for older adults, which we intend to use in the next edition of the PJAIT Living Lab. In the LL 2.0 version we intend to improve the blended learning cycle by offering longer introductory courses, complimented by consultations. The content of the courses was found too difficult and scarce. Therefore in LL 2.0 we intend to address this issue and to provide ways for users to interact via the platform, which was a common request and a key consideration to help build a community.

\subsection{Crowdsourcing}

The crowdsourcing module was to offer surveys and studies, as well as translation tasks with research, commercial and business applications.

This integrated module for web-based survey and crowdsourcing tasks has allowed over 200 participants to take profiling surveys and also to conduct an anecdotal evidence study of over 160 older participants and corresponding over 160 younger participants. The results are currently being analized and are the subject of being reported in the future.

\section{Other activities}

\subsection{Location based game}

As part of Warsaw Senior's Week Living Lab organized a location based game entitled "Stroll around yesterday" informed by best practices in game making and engagement of the elderly in city activities. The elderly  and younger tech-minded participants were divided into 15 teams of two and had to solve a mystery of a mad scientist time-shifting buildings from the communist times into the contemporary Warsaw. The game was staged to require close cooperation of team members, with the seniors using tablets to perform various ICT tasks, such as scanning QR codes, taking panoramic photos or connecting to Wi-Fi  with the assistance of their temamates given on need-to basis, and the younger participants benefiting from the background knowledge about the city and its history of the elderly. 

\begin{figure}
\centering
\includegraphics[height=150px]{figures/game_main_screen.png}
\caption{Location-based game main screen}
\label{fig:game_main}
\end{figure}

\begin{figure}
\centering
\includegraphics[height=150px]{figures/game_app.png}
\caption{Location-based game mobile application}
\label{fig:game_app}
\end{figure}


Overall, the game has improved the perception of the usability of mobile devices, the ICT skills of the elderly and has challenged stereotypes connected to aging.

\begin{figure}
\centering
\includegraphics[height=150px]{figures/game_photo_01b.jpg}
\caption{Location-based game -- taking a panoramic photo}
\label{fig:game_panoramic}
\end{figure}

It has also provided insights into the dynamics of cooperation in intergenerational pairs and gamification as means of encouraging ICT literacy among the elderly. The results and findings from initial games was presented on GOWELL conference held in 2016 in Budapest and reported in conference precedings \cite{kopec2017location} . A quick gameplay overview is also available on-line{\footnote{$https://youtu.be/htNieG0FwfY$}}.

\subsection{Hackathons}

In order to explore the area of potential best practices in designing for and with the seniors we have organized a hackaton with older adults, which took place in March 2016. We choose this special form of case study to examine in concise form a variety of emerging strategies from user-centered design perspective to fully participatory approach. Over a hundred participants took part in the case study, including 15 older adults from our LivingLab. We used various research methods and techniques, including direct observations, surveys and in-depth interviews (both individual and group) before and after the hackathon.

\begin{figure}
\centering
\includegraphics[height=150px]{figures/supersenior.png}
\caption{Hackathon -- screenshot of the winning app}
\label{fig:hackathon_photo}
\end{figure}

The results of the hackathon inform the design of further studies which concern the seniors, taking into account their strengths and limitations, the best time scope and levels of engagement. 

\begin{figure}
\centering
\includegraphics[height=150px]{figures/hackathon_photo_1b.jpg}
\caption{Hackathon -- a team with seniors at work}
\label{fig:hackathon_screen}
\end{figure}

The next event is planned in 2017 with the use of findings enclosed in the extensive in-depth report from our hackathon case study submitted to Empirical Software Engineering (review in progress) and undergoing research on participatory interface design (see section below). The brief overview of the hackathon is available on-line. \footnote{$https://youtu.be/13aEBkrxe20$}

\subsection{Wikipedia workshops}

deTo further engage the seniors and to test the applicability of employing crowdsourcing solutions in this age group we organized a Wikipedia editing workshop for them. In this case the participatory process was focused on the content co-creation mechanism in connection with identifying barriers in the user interface and other limitation in better user involvement in the community driven content creation processes. 

\begin{figure}
\centering
\includegraphics[height=150px]{figures/wiki_01b.jpg}
\caption{Hackathon -- screenshot of the winning app}
\label{fig:wiki_photo}
\end{figure}

The setup of the workshop was experimental with pre-selected example tasks, which limits insight on the influence of motivation coming from creating original content and autonomous topic selection process. The workshop, however, produced insights into the senior learning process and the design of content creation tools for older adults. The detailed information of the Wikipedia study is the subject of the main WI 2017 conference call.


\subsection{User Interface Design Workshops}

In order to obtain a better and in-depth insight into the process of user involvement in participatory and co-creation processes we organized a series of workshops for our LivingLab seniors. The workshops were focused on user interface design and prototyping by the senior citizens themselves.

\begin{figure}
\centering
\includegraphics[height=130px]{figures/design_photo_1b.jpg}
\caption{Design workshops -- paper prototyping}
\label{fig:design_paper}
\end{figure}

Throughout the workshops we observed the older adults' prototyping learning process and identified key issues they encounter while learning abstract ideas related to human-computer interaction. We conclude the study by proposing guidelines for teaching older adults abstract technology related concepts. The outcomes of this study is the subject of another conference call. Futher in-depth research is in progress based on conclusions towards more formal guidelines for older adults effective participation in IT solutions development process.

\begin{figure}
\centering
\includegraphics[height=130px]{figures/design_photo_3b.jpg}
\caption{Design workshops -- digital prototypes}
\label{fig:design_computer}
\end{figure}



\section{Discussion: lessons learned}

The activities connected to the Living Lab and the Hackatons have shown the enormous potential of the elderly as not only the users of devices and apps, but also as co-creators, who influence direction of the design process at the earliest stages from idea generation to testing prototypes. Such early involvement of the target group prevents stereotypes about their age group from influencing the end product and ensures the design and functionalities are geared towards their specific needs. 

A key insight is that the contemporary seniors prefer blended solutions, which integrate offline and online activities to standard e-learning courses and given the choice, they are more willing to participate in stationary courses and workshops. The preparatory ICT skills workshops also have to be more extensive and last for a longer time to create sustainable improvements in the way seniors interact with technology.

Regarding the crowdsourcing module, based on our observations, we concluded that one of the key motivators for them is the need to be engaged in real life social interactions and rewarding experiences, such as seeing an outcome of a game or an output produced by their team at a hackaton - otherwise the seniors get bored and lose motivation. 

Because of this it is difficult to create crowdsourcing opportunities for seniors which would make them feel sufficiently engaged. They find most of the standard crowdsourcing tasks boring and only a sense of real purpose can overcome this. According to recent Japanese research seniors have a tendency to look at a broader picture, therefore they need to clearly understand their place in the context of the task being performed in order to feel that it is worthwhile. 

Better design of social inclusion programs and applications for the elderly stemming from a better understanding of their motivation, and encouraging their participation in the whole design process would increase the rate of participation of seniors in the urban life by allowing them to be more self-assured and independent in their use of technology. To achieve this it is necessary to create a an active community fluent in ICT skills and driven by a sense of purpose.

\section{Summary and future work}

Our research and experience have shown that with prior training in ICT skills and introduction into each topic the elderly are eager to use mobile devices in order to create original content, engage in location-based augmented-reality interactions in public spaces and work in intergenerational teams. This leads us to propose that the idea of crowdsourcing solutions for urban spaces aimed at this target group has significant potential. 

We intend to further research this potential and develop best practices and content to the end of increasing senior urban participation with further strategic partnerships, one of which is in the making, involving the Federation of the Digital Universities of the Third Age - allowing us to reach seniors based outside of Warsaw, who have slightly different profiles. This will allow us to fully explore the potential of seniors as contributors to \textbf{citizen science}.

As an example of a specific future action: we have conceptualized a public space design application using geotagging and crowdsourcing. This application could incorporate photos of places which are to be redesigned, or which the application users have marked earlier as unsatisfactory. Once a photo of a public space is uploaded and geotagged the users can see it on the map. When their GPS location matches the location of the photo and they are in the presented space the possibility to re-design the space unlocks with one side of the screen showing the photo, and the other displaying drag and drop elements they can add to the photo. These could be benches, lampposts, trees, trash bins in parks, while inside public facilities they could include information signs, chairs and vending machines. Prior to using this application the seniors could participate in an e-learning course on the e-Senior platform which would teach them the basics of urban design and interior design.

To improve the appeal of the application the spaces to redesign could be connected into guided city walks with a narration, or interactive adventures where the users gather points if their designs are chosen to be implemented. This narrative dimension has proven very successful in the location-based game as the participants were motivated by the need to discover the story. 

The proposed integrated solution is promising on many levels as it may increase the amount of outdoor activity for the elderly and keep them mentally fit thanks to the need to learn and continually use ICT skills. On another level it will provide them with a way to meaningfully interact with each other and the environment. Their involvement in the pilot program will help them remove barriers to participation in technological solutions and will combat old-age stereotypes concerning ICT literacy, which may further improve their levels of happiness. Moreover thanks to the dimension of participatory design it will lead to improvements in the design of the city geared towards the needs of the elderly. 

\section{Acknowledgments}
This project has received funding from the European Union's Horizon 2020 research and innovation programme under the Marie Sklodowska-Curie grant agreement No 690962

\bibliographystyle{ACM-Reference-Format}
\bibliography{sigproc} 

\end{document}
