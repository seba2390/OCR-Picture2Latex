\documentclass{article}
\pdfoutput=1

\usepackage{pdfpages}
\usepackage{lmodern}
\usepackage[utf8]{inputenc}
\usepackage[T1]{fontenc}


\begin{document}
\title{Letters to Alan Weinstein about Courant algebroids}
\author{Pavol \v Severa}
\date{}
\maketitle
\section*{Introduction}
Between October 1998 and March 2000 I wrote several letters to Alan Weinstein about Courant algebroids (CAs). They still seem to be of some interest, so I decided to post them on the arXiv to make them more easily available.

Courant algebroids were introduced by Liu, Weinstein, and Xu \cite{lwx}; their principal motivation was extending the notion of Manin triples from Lie bialgebras to Lie bialgebroids. Similar structures were appearing in Poisson-Lie T-duality \cite{ks}, which was my original motivation for studying CAs.

Here is a short summary of the letters and some context.
\begin{description}
\item[Letter 1] contains the definition  and classification of exact CAs. It also gives a motivation for exact CAs coming from 2-dimensional variational problems, suggests a link between exact CAs and gerbes (motivated by the appearance of gerbes in 2-dim QFTs \cite{bry,gaw}), and contains a now-standard definition of CAs in terms of a non-skew-symmetric bracket. Some of these ideas were later put to \cite{SW}.
\item[Letter 2] is somewhat less relevant (perhaps besides linking Dirac structures with D-branes) - I tried there to explain  Poisson-Lie T-duality as it was my motivation for studying CAs. The proper link between CAs and Poisson-Lie T-duality is given  in Letter 4.
\item[Letter 3] contains local classification of transitive CAs and thoughts about global classification.
\item[Letter 4] contains a global classification of transitive CAs in terms of transitive Lie algebroids with vanishing 1st Pontryagin class, introduces reduction of CAs and formulates Poisson-Lie T-duality in terms of this reduction. It also introduces  examples of exact CAs over groups and homogeneous spaces. (Some of these ideas were later rediscovered, notably reduction of CAs in \cite{bur}, classification of transitive CAs in \cite{xu}, a link between CAs and T-duality \cite{cg}.)
\item[Letter 5] tries to make the link between exact CAs and gerbes more precise.
\item[Letter 6] gives a construction of a natural generating Dirac operator for any CA. It was written after I learned from Alan Weinstein that his student Dmitry Roytenberg, following ideas of Yvette Kosmann-Schwarzbach and of Arkady Vaintrob, found a link between (some) CAs and dg symplectic manifolds \cite{Rth}, that this link was improved by Weinstein himself who gave a construction of a dg symplectic manifold for any CA, and that there was a work in progress by Anton Alekseev and Ping Xu on generating Dirac operators for Courant algebroids \cite{ax}, which seemed a lot like a deformation quantization of those dg symplectic manifolds. This letter explains a natural version of such a deformation quantization.
\item[Letter 7] makes the link between  CAs and dg symplectic manifolds more precise - namely it states that CAs are equivalent to non-negatively graded  manifolds with a symplectic form of degree $2$ and with a degree-3 function $\theta$ satisfying $\{\theta,\theta\}=0$. As a part of this correspondence it gives a simple construction of a dg symplectic manifold, starting with a vector bundle $A\to M$ with a symmetric pairing, as a submanifold of $T^*[2]A[1]$. (Details were later published by Roytenberg \cite{Rst}.) In the first part of the letter the link between exact CAs and gebres is discussed from the dg point of view.
\item[Letter 8] suggests how to integrate CAs to symplectic 2-groupoids using ideas from Sullivan's rational homotopy theory \cite{Sul} and from the AKSZ construction \cite{AKSZ}. As a warm-up it gives a construction of a groupoid integrating a Lie algebroid by mimicking the construction of the fundamental groupoid of a topological space, which was apparently new at the time of writing. Last but not least, it introduces the silly terminology ``NQ-manifold''. (This letter was later expanded to \cite{some}.)
\end{description}

\begin{thebibliography}{9}
\bibitem{AKSZ}M. Alexandrov, M. Kontsevich, A. Schwarz, O. Zaboronski, The Geometry
of the Master Equation and Topological Quantum Field Theory, Int.
J. Mod. Phys. 12 (1997) no.7.
\bibitem{ax}A. Alekseev and P. Xu, Derived brackets and Courant algebroids, unpublished, available at
http://www.math.psu.edu/ping/anton-final.pdf (2001).
\bibitem{bry}J.-L. Brylinski: Loop Spaces, Characteristic Classes and Geometric Quantization. Prog. Math. 107, Birkhäuser, Boston 1993
\bibitem{bur} H. Bursztyn, G. Cavalcanti, M. Gualtieri: Reduction of Courant algebroids and generalized complex structures, Adv. Math., 211 (2), 2007, 726--765
\bibitem{cg}G. Cavalcanti, M. Gualtieri: Generalized complex geometry and T-duality. In: A Celebration of the Mathematical Legacy of Raoul Bott (CRM Proceedings \& Lecture Notes), American Mathematical Society, 2010, pp. 341--366.
\bibitem{xu} Z. Chen, M. Stienon, P. Xu: On Regular Courant Algebroids. J. Symplectic Geom.
Volume 11, Number 1 (2013), 1--24.

\bibitem{gaw} K. Gawedzki, Topological actions in two-dimensional quantum field theories. In: Nonperturbative Quantum Field Theory
NATO ASI Series Volume 185, 1988, pp 101--141.
\bibitem{ks} C. Klimčík, P. Ševera: Dual non-Abelian T-duality and the Drinfeld double. Phys.Lett. B 351 (1995), 455--462.


\bibitem{lwx} Zh.-J. Liu, A. Weinstein, P. Xu: Manin triples for Lie bialgebroids. J. Differential Geom. 45 (1997), no. 3,
547–574
\bibitem{Rth} D. Roytenberg, Courant algebroids, derived brackets and even symplectic supermanifolds, PhD thesis, arXiv:math/9910078
\bibitem{Rst} D. Roytenberg, On the structure of graded symplectic supermanifolds and Courant algebroids, in: Quantization,
Poisson brackets and beyond (Manchester, 2001), Amer. Math. Soc., Providence, RI, 2002, 169--185
\bibitem{some} P. Ševera: Some title containing the words “homotopy” and
“symplectic”, e.g. this one. Travaux mathématiques, Volume 16 (2005), 121--137
\bibitem{SW}P. \v Severa, A. Weinstein: Poisson geometry with a 3-form background, Prog.Theor.Phys.Suppl. 144 (2001), 145--154.
\bibitem{Sul}D. Sullivan, Infinitesimal Computations in Topology, I.H.E.S. Publ.
Math. 47 (1977).


\end{thebibliography}

\includepdf[pages={1-}, picturecommand*={\put(50,670){\makebox{\bf\large \color{red}{Letter 1}}}}]{no1.pdf}
\includepdf[pages={1-}, picturecommand*={\put(50,670){\makebox{\bf\large \color{red}{Letter 2}}}}]{no2.pdf}
\includepdf[pages={1-}, picturecommand*={\put(50,670){\makebox{\bf\large \color{red}{Letter 3}}}}]{no3.pdf}
\includepdf[pages={1-}, picturecommand*={\put(50,670){\makebox{\bf\large \color{red}{Letter 4}}}}]{no4.pdf}
\includepdf[pages={1-}, picturecommand*={\put(50,670){\makebox{\bf\large \color{red}{Letter 5}}}}]{no5.pdf}
\includepdf[pages={1-}, picturecommand*={\put(50,670){\makebox{\bf\large \color{red}{Letter 6}}}}]{no6.pdf}
\includepdf[pages={1-}, picturecommand*={\put(50,670){\makebox{\bf\large \color{red}{Letter 7}}}}]{no7.pdf}
\includepdf[pages={1-}, picturecommand*={\put(50,670){\makebox{\bf\large \color{red}{Letter 8}}}}]{no8.pdf}
 
\end{document}

