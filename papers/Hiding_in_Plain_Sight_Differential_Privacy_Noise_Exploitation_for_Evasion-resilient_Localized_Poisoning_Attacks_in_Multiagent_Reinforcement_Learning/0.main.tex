\documentclass[conference]{IEEEtran}
\makeatletter
\def\ps@headings{%
\def\@oddhead{\mbox{}\scriptsize\rightmark \hfil \thepage}%
\def\@evenhead{\scriptsize\thepage \hfil \leftmark\mbox{}}%
\def\@oddfoot{}%
\def\@evenfoot{}}
\makeatother
\pagestyle{empty}
\usepackage{cite}
\usepackage{amsmath,amssymb,amsfonts}
\usepackage{graphicx}
\usepackage{textcomp}

\usepackage{subfigure}
\usepackage{algorithmic}

\usepackage{xcolor}
\usepackage{caption}
\usepackage{wrapfig,lipsum,booktabs}
\usepackage[export]{adjustbox}
\definecolor{red}{rgb}{2,0,0}
\usepackage{graphicx}
\ifCLASSOPTIONcompsoc
    \usepackage[caption=false, font=normalsize, labelfont=sf, textfont=sf]{subfig}
\else
\usepackage[caption=false, font=footnotesize]{subfig}
\fi
\def\BibTeX{{\rm B\kern-.05em{\sc i\kern-.025em b}\kern-.08em
    T\kern-.1667em\lower.7ex\hbox{E}\kern-.125emX}}
    
\renewcommand{\footnoterule}
    {\noindent\smash{\rule[4pt]{0.15\textwidth}{0.4pt}}}

\IEEEoverridecommandlockouts 


\DeclareMathAlphabet{\mathpzc}{OT1}{pzc}{m}{it}
\newtheorem{definition}{Definition}
\newtheorem{proposition}{Proposition}
\newtheorem{theorem}{Theorem}
\usepackage[ruled, vlined, linesnumbered, commentsnumbered, longend]{algorithm2e}
\newcommand\mycommfont[1]{\small\ttfamily\textcolor{blue}{#1}}
\SetCommentSty{mycommfont}
\usepackage[export]{adjustbox}
\usepackage{booktabs}
\begin{document}
    
\title{\Large \uppercase{Hiding in Plain Sight: Differential Privacy Noise Exploitation for Evasion-resilient Localized Poisoning Attacks in Multiagent Reinforcement Learning}}

\author{
\IEEEauthorblockN{\textbf{\uppercase{Md Tamjid Hossain}}\IEEEauthorrefmark{1}, \textbf{\uppercase{Hung La}}\IEEEauthorrefmark{2}}

\IEEEauthorblockA{Advanced Robotics and Automation (ARA) Laboratory\\ University of Nevada, Reno, NV, USA
}

\IEEEauthorblockA{
E-mail: 
\IEEEauthorrefmark{1}mdtamjidh@nevada.unr.edu, \IEEEauthorrefmark{2}hla@unr.edu}

\thanks{This work was partially funded by the U.S. National Science Foundation (NSF) under grant NSF-CAREER: 1846513.  The views, opinions, findings, and conclusions reflected in this publication are solely those of the authors and do not represent the official policy or position of the NSF.}
\thanks{*Source code of the task and the computational model behind the setup available at {\tt \small https://github.com/aralab-unr/PeLPA}}%

}

\IEEEaftertitletext{\vspace{-1.0\baselineskip}}

\maketitle




\begin{abstract}
Lately, differential privacy (DP) has been introduced in cooperative multiagent reinforcement learning (CMARL) to safeguard the agents' privacy against adversarial inference during knowledge sharing. Nevertheless, we argue that the noise introduced by DP mechanisms may inadvertently give rise to a novel poisoning threat, specifically in the context of private knowledge sharing during CMARL, which remains unexplored in the literature. To address this shortcoming, we present an adaptive, privacy-exploiting, and evasion-resilient localized poisoning attack (PeLPA) that capitalizes on the inherent DP-noise to circumvent anomaly detection systems and hinder the optimal convergence of the CMARL model. We rigorously evaluate our proposed PeLPA attack in diverse environments, encompassing both non-adversarial and multiple-adversarial contexts. Our findings reveal that, in a medium-scale environment, the PeLPA attack with attacker ratios of $20\%$ and $40\%$ can lead to an increase in average \textit{steps to goal} by $50.69\%$ and $64.41\%$, respectively. Furthermore, under similar conditions, PeLPA can result in a $1.4$x and $1.6$x computational time increase in optimal reward attainment and a $1.18$x and $1.38$x slower convergence for attacker ratios of $20\%$ and $40\%$, respectively.
\end{abstract}

\begin{IEEEkeywords}
Differential Privacy, Adversarial Learning, Poisoning Attacks, Cooperative Multiagent Reinforcement Learning
\end{IEEEkeywords}

\section{Introduction}
The need for protecting input privacy in Deep learning is growing rapidly in many areas. 
Many of the data holders are, however, not machine learning experts. Hence, data holders are relying on machine learning as a service (MLaaS) platforms~\citep{AzureML,Google,Amazon}. 
%Microsoft Azure ML~\citep{AzureML}, Google AI platform~\citep{Google}, Amazon ML~\citep{Amazon} are some examples. 
These services incorporate ML accelerators such as GPUs for high performance and provide easy to use ML runtimes to enable data holders to quickly set up their models and train. While these platforms lower the steep learning curve, they exacerbate the users' concern regarding data privacy. %Cloud servers are also vulnerable to adversaries who may compromise some of the MLaaS infrastructures, further raising computational integrity concerns. Hence, there is a pressing need to protect privacy and integrity while still enabling the use of untrusted cloud accelerators. 
%Techniques such as homomorphic encryption can be employed in some cases but comes with orders of magnitude slowdowns~\cite{}.   

%At the same time, training large-scale DNNs with a large volume of data needs a significant amount of computation power. Hence, the need for collaborative learning including distributed learning and federated learning has emerged in recent years. Stochastic Gradient Descent is the most well-known technique used for optimization. In distributed learning, the parameter server (\textbf{PS}) is responsible for gradient synchronization by receiving gradients from all the workers, aggregating the gradients, and updating the parameters. 



%There are various methods to protect data privacy, such as Homomorphic Encryption, Secure Multi-Party Computing, Differential Privacy. Each of these methods provides a different privacy guarantee and comes at different costs. ML designers need to be aware of their needs to make a comprehensive decision on which method to use. To evaluate a privacy-preserving method, in addition to privacy guarantee and utility cost, practicality and human efforts made for the method also play important roles. Therefore, although some of the aforementioned methods show promising results for privacy, they cannot be used in the real world because of their expensive costs and high design time.
This work proposes DarKnight, a framework for accelerating privacy and integrity preserving deep learning using untrusted accelerators. DarKnight is built on top of an MLaaS platform that uses unique collaborative computing between the \emph{TEE} and GPU accelerators to tackle both privacy and security challenges. The data holder places their data, whether for training or inference, within the TEE of a cloud server. TEE provides hardware-assisted security for any data and computing performed within the trusted code base. %%%No unauthorized entity can access the data placed in TEE memory or modify the code within the trusted code base. %TEEs use hardware encryption and virtual memory-based isolation mechanisms to protect data and code. 
DarKnight uses TEE to encode input data using a customized matrix masking technique and then uses GPUs to accelerate DNN's linear computations on the encoded data. %Computing solely within TEEs can provide data privacy, by blocking access to TEE memory from intruders. However, TEE-enabled CPUs have limited computation power and memory availability, which creates unacceptable performance hurdles to run an entire model within a TEE. 
Linear operations (convolution, matrix multiplication, etc) are significantly faster on a GPU compared to a TEE-enabled CPU. Therefore, DarKnight distributes these compute-intensive linear operations to GPUs. DarKnight's usage of TEEs is limited to protecting the privacy of data through a customized matrix masking and performing non-linear operations (ReLU, Maxpool). 


TEE-GPU collaboration is first used in~\citep{tramer2018slalom} for inference. However, the method cannot be used for training as elaborated in their paper. Several prior works on protecting privacy use cryptography techniques on Finite Fields to provide data privacy. Such approaches limit their usage to arithmetic on quantized models~\citep{mohassel2017secureml,gascon2017privacy, so2019codedprivateml,wagh2019securenn, juvekar2018gazelle}. Quantization for deep learning is a challenging task. DarKnight supports \emph{floating point} model training and control the information leakage by encoding parameters. DarKnight can also detect any malicious activities of untrusted GPUs by its computation integrity feature. Furthermore, DarKnight can protect privacy and integrity even in the presence of a subset of colluding GPUs that try to extract information or sabotage the computation. 
%%%Moreover, Floating-point models are routinely used in training due to convergence, accuracy, and faster implementation considerations~\citep{johnson2018rethinking, guo2020secure, imani2019floatpim,aliasgari2013secure}. Many DNN accelerators such as google TPUs, Intel Xenon processors, Intel FPGAs, and Tensorflow now support bfloat16 floating-point format for training~\citep{kalamkar2019study,FPGA1,googlcloud1,googletensor1}. 
% Quantization for deep learning needs a meticulous hyper-parameter tuning process for different applications and datasets, which is a challenging task. In practice, f

%\name allows users to train using floating-point (FP) representation for model parameters, while still providing rigorous bounds on information leakage. 
%DarKnight supports \emph{floating point} model training. %ML designers can choose a setting based on their training and privacy needs. For Finite Field, the privacy guarantee is as strong as a one-time-pad by making the encoded data look uniformly random to the adversaries~\citep{deng2004secure}, which is the most commonly provided guarantee in prior works~\citep{tramer2018slalom, yu2019lagrange}.
%For providing privacy guarantee in our floating-point model, the information leakage is bounded by the parameters of the encoding. %The strength of the random noise used in our matrix masking process trades off information leakage with computational stability. In our preferred implementation parameters, DarKnight's selection of random noise bounds information leakage to be no more than $~10^{-7}$ without compromising accuracy.
%In floating-point arithmetic, perfect privacy can be achieved when the information leakage is less than the round off error~\cite{guo2020secure}. Hence, by bounding the information leakage to $~10^{-7}$, DarKnight achieves perfect privacy in IEEE standard single-precision arithmetic~\cite{hough2019ieee}.% In the rest of the paper, we focus on floating-point models. For avid readers, the fixed point privacy scheme and results are shown in Appendix~\ref{ap:fixed}.
% DarKnight can also detect any malicious activities of untrusted GPUs by its computation integrity feature. Furthermore, DarKnight can protect privacy and integrity even in the presence of a subset of colluding GPUs the try to leak information. 
%We implemented DarKnight using an Intel SGX-enabled CPU to perform matrix masking and non-linear DNN operations while using an Nvidia GPUs to accelerate linear operations. 
%{\color{red}The strength of the random numbers in matrix masking were chosen to preserve the original training accuracy. Using these parameters DarKnight guarantees that no more than one bit of information is leaked from a ten-megapixel input image. Note that this will be an upper bound on the leaked information, assuming that the adversary has access to unlimited computation power to decode the encoded inputs.} 
%To the best of our knowledge, this is the first work that uses TEE-GPU collaboration for \textbf{\emph{training}} large DNNs on private data while providing computational integrity and theoretical bounds on information leakage.     

\section{Related Works} \label{s:litReview}
In this section, we address the SOTA poisoning techniques.
\subsection{Application of Differential Privacy for Knowledge Sharing}

DP, a prominent method for privacy preservation, has been extensively employed in private knowledge sharing within the realm of CMARL \cite{ye2022differential,wang2019privacy,li2022privacy, Wei2022, Abahussein2023,ye2022one}. The scope of its application includes DP-guided Q-learning models to maintain the privacy of reward data \cite{wang2019privacy}, privacy-centric multi-agent frameworks leveraging federated learning (FL) and DP to obstruct illegitimate access to data statistics \cite{li2022privacy}, and harnessing ($\beta, \phi$)-DP to counteract offloading preference inference attacks in vehicular ad-hoc networks (VANET) \cite{Wei2022}. Apart from protecting user information during private knowledge sharing, DP has also been proposed for differential advising. In particular, Ye et al. \cite{ye2022differential} propose a DP-based advising method for CMARL that enables agents to use the advice in a state even if the advice is created in a slightly different state. \textit{Nevertheless, they overlook the susceptibility of DP to poisoning attacks during knowledge sharing \cite{ye2022differential, Abahussein2023,ye2022one}.}

\subsection{Poisoning Attacks in Cooperative Multiagent Learning} The infiltration of poisoning attacks in CMARL, which can alter training datasets and consequently disrupt learning outcomes, is a pertinent research concern \cite{figura2021adversarial,fang2020local, Xie2022,mohammadi2023implicit}. Research has delved into scenarios where adversarial agents can manipulate network-wide policies \cite{figura2021adversarial}, scrutinized targeted poisoning attacks in dual-agent frameworks where one agent’s policy is modified \cite{mohammadi2023implicit}, and investigated the implications of soft actor-critic algorithms in CMARL for executing poisoning attacks \cite{Xie2022}. For instance, Figura et al. \cite{figura2021adversarial} demonstrate that an adversarial agent can persuade all other agents in the network to implement policies that optimize its desired objective. Another approach for performing poisoning attacks by any malicious advisor in multiagent Q-learning as demonstrated in \cite{hossain2023BRNES}, is to shuffle the Q-values for all actions corresponding to the requested state
and inject false noise that is similar to the maximum reward using reward poisoning method. \textit{However, the ramifications of these SOTA poisoning techniques against anomaly detection and privacy-preserving knowledge-sharing technologies remain largely unexplored. Our work endeavors to model a DP noise-exploiting poisoning attack that remains resilient to detection algorithms.}

\subsection{Differential Privacy Exploitation Techniques} Another domain of interest focuses on the possible exploitation of DP in classification challenges, even though it does not necessarily concentrate on adversarial onslaughts on CMARL algorithms \cite{giraldo2017security_2, giraldo2020adversarial, hossain2021privacy,cao2021data,cheu2021manipulation,hossain2022adversarial, Hossain2021Desmp}. This research trajectory involves the systemic degradation of utility by exploiting DP noise \cite{giraldo2020adversarial}, gauging the impact of DP manipulation in smart grid networks \cite{hossain2021privacy}, and designing stealthy model poisoning attacks on an FL model \cite{Hossain2021Desmp, hossain2022adversarial}. Similarly, \cite{hossain2021privacy} investigates the impact of DP exploitation in a smart grid network and introduces a correlation among DP parameters to enable the system designer to calibrate the privacy level and reduce the attack surface. To examine the effect of DP-exploiting attacks on an FL model, \cite{Hossain2021Desmp} proposes a stealthy model poisoning attack leveraging DP noise added to ensure privacy. They improve their attack technique in \cite{hossain2022adversarial}, investigating how the degree of model poisoning can be adjusted dynamically through episodic loss memorization in FL and demonstrating how their attack can evade some SOTA defense techniques, such as norm, accuracy, and mix detection. \textit{However, these attack models face constraints in multi-agent environments or decentralized CMARL platforms.} Contrarily, Cao et al. \cite{cao2021data} propose an attack on the Local Differential Privacy (LDP) protocol by introducing fraudulent users. \textit{Our research, however, targets legitimate yet compromised users infusing false noise into shared data, also aiming to dodge anomaly detectors - a critical objective for a successful attack.}
\section{Local Differentially Private Cooperative Multiagent Reinforcement Learning}
\label{s:DP_CMARL}

We present a local differentially private CMARL (LDP-CMARL) framework akin to the one adopted in \cite{hossain2023BRNES}. However, for demonstration simplicity, instead of a generalized randomized response (GRR) technique, we leverage a Bounded Laplace (BLP) mechanism \cite{Neera2023} to model our LDP framework that also achieves the same $\varepsilon$-LDP guarantee.

\subsection{Cooperative Multiagent Reinforcement Learning (CMARL)}
\label{subsec:CMARL} \textbf{Environment model.} Our research formalizes a cooperative reinforcement learning context with a Markov game $\mathcal{M} =(\mathpzc{N}, \mathpzc{S}, \mathpzc{A}, \Phi, \Gamma, \mathpzc{T})$ incorporating $\mathpzc{N}$ robots navigating an environment $\mathbb{E}$ of dimensions height ($\mathpzc{H}$) and width ($\mathpzc{W}$) towards a goal $\mathpzc{G}$. It introduces obstacles $\mathpzc{O}$ and freeway $\mathpzc{F}$ with corresponding reward penalties and incentives, $\phi_{\mathpzc{O}}$ and $\phi_{\mathpzc{F}}$. Dynamic obstacle positioning adds complexity to learning, which concludes when the first agent reaches $\mathpzc{G}$.

\textbf{Learning objectives.} Agent $p_i$'s objective is to take the fewest steps, $\Pi$ to reach $\mathpzc{G}$, collect $\phi_f$, avoid hitting $o_x \in \mathpzc{O}$, and earn as much as rewards, $\phi_{\mathpzc{F},\mathpzc{G}}$. In short, the objectives can be formalized as
\begin{equation}
\begin{alignedat}{2}
(a)&\;\Pi{p_i} = \underset{\mathcal{M}}{min}\;\Pi\\
(b)&\;\phi_{p_i} = \phi_{\mathpzc{F}} + \phi_{\mathpzc{G}} + \left[\phi_{\mathpzc{O}} = 0\right]\\
(c)&\;\lVert (x_{p_i}, y_{p_i})- (x_{\mathpzc{G}}, y_{\mathpzc{G}}) \rVert = 0
 \end{alignedat}
 \forall \phi_{\mathpzc{G}, \mathpzc{F}, \mathpzc{O}} \in \Phi \text{ and } \phi_{\mathpzc{G}} > \phi_{\mathpzc{F}}
 \label{eqn:objective1}
\end{equation}

where $(x_{p_i}, y_{p_i})$, and $(x_{\mathpzc{G}}, y_{\mathpzc{G}})$ are $p_i$'s and $\mathpzc{G}$'s positions.

\subsection{Integrating Local Differential Privacy (LDP) in CMARL}
LDP protocols encapsulate two main stages: perturbation and aggregation. The Q-values domain, denoted as $\mathbb{Q} = \left[q\right]$, undergoes local perturbation before being relayed to the advisee, $p_i$, ensuring $p_i$'s inability to infer the original Q-value of the advisor, $p_k$. The aggregation phase facilitates $p_i$'s estimation of optimal advice utilizing the perturbed values received from all $p_k$, with perturbation function for Q-values of all actions, $a$ in state $s$ represented as $P(Q(s))$. Following the definition of $\varepsilon$-LDP \cite{cao2021data}, a protocol achieving LDP must ensure the probabilistic resemblance between any pair of perturbed Q-values.

LDP offers plausible deniability to $p_k$, restraining $p_i$ from determining the origin of the output confidently. This ambiguity is regulated by the privacy budget, $\varepsilon$ \cite{dwork2006}. To actualize $(\varepsilon, 0)$-DP, the Laplace mechanism, a noise-addition technique, is applied as follows \cite{dwork2006}:

\begin{equation}
    \mathpzc{M}(D) = f(D) + \eta \sim \mathcal{N}(0, b)
\end{equation}

where the added
noise, $\eta$ is drawn from a zero-mean Laplace distribution with
scale parameter, $b \geq \frac{\Delta}{\varepsilon}$. Here, $\Delta$ denotes the sensitivity of the query function. Nonetheless, the same Laplace mechanism that satisfies $(\varepsilon,0)$-DP,  can be deployed in a distributed fashion for achieving $\varepsilon$-LDP \cite{wang2020comprehensive, Neera2023}, by integrating randomized Laplace noise into each state-action pair's Q-values of an advisor. We leverage the higher noise sensitivity offered by the Laplace mechanism to attain stronger privacy protection as compared to Gaussian or Exponential mechanism. The advisee, $p_i$ computes the average value from all the noisy Q-values \cite{wang2020comprehensive}. We utilize the following BLP technique for input perturbation \cite{Neera2023}: 

\begin{definition}[Bounded Laplace Mechanism (BLP)]
   Given an input $q \in \left[l, u\right] \subset \mathbb{R}$, and scale $b>0$, the BLP technique, $\mathpzc{M}: \Omega \rightarrow \left[l, u\right]$ over output $\bar{q}$ can be represented by the following conditional probability density function (pdf):
   
   \begin{equation}
       f\mathpzc{M}(\bar{q}) = 
       \begin{cases}
       0 & \text{ if } \bar{q} \notin \left[l,u\right]\\
       \frac{1}{C_q} \frac{1}{2b}e^{-\frac{\lvert \bar{q}- q\rvert}{b}} & \text{ if } \bar{q} \in \left[l,u\right]
       \end{cases}
   \end{equation}
\end{definition}
where $l$ and $u$ are the lower and upper range, and $C_q = \int_{l}^{u} \frac{1}{2b}e^{-\frac{|\bar{q}- q|}{b}} \,d\bar{q}$ is a normalization constant. The proof and further details can be found in \cite{Neera2023}. BLP constrains noise sampling within a predefined range, avoiding values that may detriment learning performance. Hence, the sensitivity of the combined LDP mechanism is $\Delta = \lvert u-l\rvert$. Similar to \cite{ye2022differential}, within our LDP-CMARL framework, the sensitivity $\Delta$ needs to be calculated carefully. The LDP-CMARL framework training stages utilizing the BLP mechanism are outlined in Algorithm~\ref{algo:DP_CMARL}. During advice request dispatch, $p_i$ specifies a neighbor zone, $\mathpzc{Z}$, and sends advice requests only to advisors within $\mathpzc{Z}$. Both $p_i$ and $p_k$ calculate their advice requesting ($\varrho_{p_i}$) and advice giving ($\varrho_{p_k}$) probabilities as per \cite{ye2022differential}. After receiving advice from the neighbors, $p_i$ aggregates all the advice following a weighted linear aggregation technique, controlled by a predefined weight parameter, $w$ \cite{hossain2023BRNES}. Then, $p_i$ selects and executes an optimal action followed by a final Q-table update.

\setlength{\textfloatsep}{0pt}% 
\begin{algorithm}[!t]
    \SetKwFunction{LDP}{LDP}
    \SetKwProg{Fn}{Function}{:}{}
    \SetKwInOut{KwIn}{Input}
    \SetKwInOut{KwOut}{Output}

    \KwIn{$\mathbb{E}, \mathpzc{N}, \mathpzc{A}, \mathpzc{S}, \Phi \rightarrow (l, u)$}
    \KwOut{Trained LDP-CMARL model}

    Initialize Q-table, set $\varepsilon, \alpha, \Gamma$, and compute $b =\frac{\alpha\lvert u-l\rvert}{\varepsilon}$

    \For{each agent, $p_i \in \mathpzc{N}$}{
        \For{each episode}{
            Initialize state, $s$
            \For{each state}{
                Send advice request to $p_k$ in $\mathpzc{Z}$ with $\varrho_{p_i}$\\
                Receive LDP-advice, $\LDP(s, \varepsilon, b)\rightarrow \bar{Q}_{p_i}(s) = \left[\bar{Q}_i(s)\right]_{i=1}^k$\\ 
                \For{each action $a\in \mathpzc{A}_i$ in state, $s$}{
                    Find weighted Q-value, $Q_{p_i}^*(s,a) = w\cdot Q_{p_i}(s, a) + (1-w)(\frac{1}{k}\sum_{i=1}^{k} \bar{Q}_i(s, a))$\\
                    Append $Q_{p_i}^*(s,a)$ to $Q_{p_i}^*(s)$
                }
            Update Q-table with $Q_{p_i}^*(s)$\\
            Choose $a^*\in \mathpzc{A}_i$ for $s$ using $\epsilon$-greedy policy\\
            Execute action, $a^*$, observe $\phi_{p_i}, s'$\\
            Perform $Q_{p_i}(s,a) \leftarrow (1-\alpha)Q_{p_i}(s,a) + \alpha \left[\phi_{p_i}+\Gamma \; \underset{a'}{max}\;Q(s', a')\right]$\\
            Set, $s\leftarrow s'$
            }
            \textbf{If }$\lVert(x_{p_i}, y_{p_i})-(x_{\mathpzc{G}, y_{\mathpzc{G}}})>0\rVert$ \textbf{then}
                continue\\
                \textbf{else} end episode and reset $\mathbb{E}$
        }
    }
    \KwRet{Trained LDP-CMARL model}\\
    \Fn{\LDP{$s, \varepsilon$, b}}{
        \For{$i = 1, 2, ..., k$ advisors}{
            Receive advice request for the state, $s$\\
            With $\varrho_{p_k}$, \For{each action $a\in \mathpzc{A}_i$}{
                find $Q_i(s, a)$ and generate $\eta_i\sim\mathcal{N}(0, b)$
                %\; \forall\; b \geq \frac{\alpha\lvert u-l\rvert}{\varepsilon}$
                \\
                Add LDP-noise,
                $\bar{Q}_i(s, a) = Q_i(s, a)+\eta_i$\\
                \eIf{$\bar{Q}_i(s, a) \notin (l, u)$}{
                    Repeat loop until $\bar{Q}_i(s, a) \in (l, u)$
                }{
                Append $\bar{Q}_i(s, a)$ to $\bar{Q}_i(s)$
                }
            }
            \KwRet{$\bar{Q}_i(s)$}
        }
        \KwRet{$\left[\bar{Q}_i(s) \right]_{i=1}^k$}  
        
    }
    \caption{LDP-CMARL Framework}
    \label{algo:DP_CMARL}
\end{algorithm}
\section{Privacy Exploited Localized Poisoning Attack}
\label{s:problemFormulation}
In this section, we dissect the DP noise exploitation mechanism, formulating adversarial noise profile challenges. We also articulate our threat model and proposed PeLPA algorithm.

\subsection{How can LDP-noise be Exploited for Poisoning Attacks?} \hskip1em \textbf{DP not included.} Considering a non-LDP advising scenario, the agents exchange Q-value knowledge, facilitating learning. We formulate the knowledge as Q-values instead of the recommended actions since the Q-value advising, unlike the action advising, does not impair the performance of the agent’s learning directly \cite{zhu2021q}. Let us assume an anomaly detector at $p_i$'s end that monitors Q-values sequences from advisor agents for all actions in a specific state, $s$. Generally, for a received Q-value, $Q_{p_k}(s)$, from advisor $p_k$, the condition $|Q_{p_{k}}(s)-Q_{0}(s)|\leq \tau$ is consistently maintained, where $\tau$ is a detection threshold and $Q_{0}(s)$, a historical standard Q-value. Any deviation raises an alarm, implying a potential malicious advisor $p_a \in \left[p_k\right]$ with biased Q-values. Nonetheless, to evade detection, the attacker can introduce a bias up to a maximum of $\tau$ relative to the standard value, i.e., $Q_{p_a}(s) \leq Q_{0}(s) + \tau$.

\textbf{DP included.} With an LDP mechanism safeguarding knowledge exchange, any received Q-value, $\bar{Q}{p_k}(s) = Q{p_k}(s) + \eta$, includes noise, $\eta$ following a zero-mean Laplace distribution, $\mathcal{N}(0, b)$, where $b$ is the distribution scale. To prevent false-positive alarms for benign differentially private Q-values, the detector adjusts the previous detection condition to $|\bar{Q}{p_k}(s)-Q_{0}(s)|\leq \tau'$ with $\tau' = \tau \times \kappa; \forall \kappa \in \mathbb{R}$, where $\kappa$ is the tolerance multiplier. This adjustment creates a poisoning window of $\lvert\tau(1-\kappa)\rvert$ that an attacker can exploit, enabling a larger bias in knowledge (i.e., Q-values) without detection. Formally, the attacker shares malicious knowledge, $\bar{Q}{p_a}(s) = Q{p_a}(s)+\eta_a; \forall \eta_a \in \lvert\tau(1-\kappa)\rvert$, where $\eta_a$ denotes the malicious noise drawn from an adversarial noise profile, $\mathcal{N}_a$. Hence, an increase in noise for privacy enhancement also expands the detection and the poisoning window.

\subsection{Challenges in Formulating Adversarial Noise Profile}
\label{noiseProfile} Crafting an adversarial noise profile, $\mathcal{\eta}_a$, that optimizes attack gain while evading anomaly detection poses a technical conundrum. A previous methodology \cite{fang2020local} attempted this by maximizing utility degradation, although this leads to a paradoxical situation in the face of an anomaly detector - more noise aids detection but less noise diminishes the attack gain. A sophisticated alternative, as proposed by \cite{giraldo2020adversarial}, models this as a multi-objective optimization problem, i.e., $\underset{\mathcal{A}}{max}\;\mathpzc{G}(\mathcal{A}, \mathcal{D}) \ni |\bar{Q}_{p_{a}}(s)-Q_{0}(s)|\leq \tau'$ where $\mathcal{A}, \mathcal{D}, \text{and } \mathpzc{G}$ denote the attack, the detect, and the gain function, respectively. The solution of this multi-criteria optimization problem is derived in \cite{giraldo2020adversarial}, where the authors presented an attack impact, $\mu^*_a$, and an optimal adversarial distribution, $\mathcal{N}_a^*(\mu_a^*, b)$ having the probability density function, $f^*_a$ as

\begin{equation}
    f_a^*(x) = \frac{k^2 - b^2}{2bc^2}e^{-\frac{|x-\theta|}{b} + \frac{(x-\theta)}{c}} \;\text{and}\; \mu_a^* = \frac{b^2(\theta-2c)-\theta c^2}{b^2 - c^2}
    \label{eqn:attackDist}
\end{equation}

where $\theta$ is the mean, $b^2$ is the variance, and $c$ is the Lagrange multiplier. $c$ can be solved numerically from \cite{giraldo2020adversarial}: 

\begin{equation}
    \frac{2b^2}{c^2 - b^2} + \ln{(1-\frac{b^2}{c^2})}=\gamma.
    \label{eqn:c}
\end{equation} 

Here, $\gamma$ is the degree of knowledge poisoning; a high $\gamma$ implies a large malicious noise injection (i.e., a higher attack gain) and vice versa. In particular, choosing a high $\gamma$ can lead to unrealistically large Q-values whereas choosing a minuscule $\gamma$ can result in negligible to almost zero attack gain. Consequently, tuning $\gamma$ for an optimal attack is non-trivial but challenging, which, unfortunately, overlooked by literature so far. We address this research gap in section~\ref{s:methodology}. Figure~\ref{fig:outlier_rmse}(a) demonstrates the influence of $\kappa$ and $\gamma$ on detected outliers and RMSE. By adding LDP-noise to $100$ uniform random values, non-DP Q-values detect a steady number of outliers for a fixed $\tau$, whereas LDP implementation significantly increases outlier detection due to benign DP Q-values flagged as false positives. This can be mitigated by setting $\tau' = \tau\times\kappa$. Moreover, an optimal attack approach as per (\ref{eqn:attackDist}) allows successful detection evasion, maintaining the baseline outlier count while inflating the system's RMSE, as shown in Fig.~\ref{fig:outlier_rmse}(b).
\begin{figure}[!ht]
\centerline{\includegraphics[width=1.0\linewidth]{Figures/Outlier_RMSE.pdf}}
\caption{\small (a) Impact of tolerance multiplier, $\kappa$ over detected outliers in both non-DP and DP settings, (b) Impact of degree of knowledge poisoning, $\gamma$ over attack evasion (difference in outlier count between non-attack and attack scenario) and attack gain (System's RMSE).}
\label{fig:outlier_rmse}
\end{figure}

\subsection{Attacker's Capability and Knowledge} We contemplate an attacker manipulating knowledge submissions to an advisee, either by exploiting susceptible agents (internal threats) or by compromising communication channels (external threats) (Fig.~\ref{fig:poisoning_framework}a, b). The attacker, in line with SOTA research \cite{dwork2019differential}, is presumed to know the publicly available $\varepsilon$-value and noise distribution.
\begin{figure}[!t]
\centerline{\includegraphics[width=\linewidth]
{Figures/poisoning_framework.pdf}}
\caption{\small (a) Internal poisoning: Attacker compromises advisors and replaces benign LDP process with adversarial LDP process, (b) External poisoning: Attacker compromises the communication path and injects additional malicious noise.} 
\label{fig:poisoning_framework}
\end{figure}
\begin{figure*}[!t]
\centerline{\includegraphics[width=\textwidth]{Figures/comparison_convergence.pdf}}
\caption{\small Average steps to goal ($\bar{\Pi}$) and obtained reward ($\bar{\Phi}$) analysis for (a) small ($\mathpzc{H}\times\mathpzc{W}=5\times 5, \mathpzc{N}=5, \mathpzc{O}=1$), (b) medium ($\mathpzc{H}\times\mathpzc{W}=10\times 10, \mathpzc{N}=10, \mathpzc{O}=3$), and (c) large-scale ($\mathpzc{H}\times\mathpzc{W}=15\times15, \mathpzc{N}=20, \mathpzc{O}=5$) environments. The number of steps is increased as well as the maximum reward achievement is delayed with more attacks (large attacker ratio). Also, (d) convergence is delayed for both $20\%$ and $40\%$ attacks compared to the no-attack baseline.} 
\label{fig:comparison_convergence}
\end{figure*}
\subsection{Proposed PeLPA Algorithm} \label{s:methodology}
A malevolent advisor, $p_a \in \mathpzc{N}$, could disrupt $p_i$'s convergence by transmitting erroneous information during the knowledge-sharing phase. Having knowledge of $\mathpzc{A, S}, \Phi, (x_{\mathpzc{G}},y_{\mathpzc{G}})$ and $p_i$'s state, $s$, $p_a$ might manipulate larger $Q$-values for a misleading action $a_m$ versus an ideal action $a_h$. This would steer $p_i$ towards a malicious point. Yet, anomalous Q-values could either invite detection or result in an insignificant attack impact. The optimal attack method in section~\ref{noiseProfile} addresses this trade-off. Our proposed PeLPA attack for LDP-CMARL is detailed in Algorithm~\ref{algo:Attack}. $p_a$ continually injects adversarial noises ($\eta_a$) to its Q-values ($Q_a(s,a)$) until    either the malicious Q-values drop below $p_i$'s maximum Q-value for an action $a$, or $\gamma$ exceeds a predetermined poisoning threshold ($\tau_{\gamma}$). Additionally, $p_a$ ensures malicious advice stays within the reward range $\bar{Q}_{p_a}(s, a) \in [l,u]$ to evade detection.

\begin{algorithm}[!ht]
    \SetKwInOut{KwIn}{Input}
    \SetKwInOut{KwOut}{Output}
    \SetKwRepeat{Do}{do}{while}
    \KwIn{$\varepsilon, b, \alpha, Q_{p_i}(s), Q_{p_a}(s)$}
    Initialize $\bar{Q}_{p_a}(s) = \text{[}\;\text{]}$ and set $\gamma\leftarrow 0, \Psi\leftarrow True, \theta\leftarrow 0$\\
    \While{$\Psi$ is True}{
        $\gamma= \gamma+1$\\
        With $b$ and $\gamma$, find $c$ numerically from (\ref{eqn:c})\\
        Then, with $c, \theta$ and $b$, find $\mu^*_{p_a}$ from (\ref{eqn:attackDist})\\
        \For{each $a\in\mathpzc{A}_i$ in state, $s$}{
            \While{$\bar{Q}_{p_a}(s, a) \notin (l, u)$}{
                $\bar{Q}_{p_a}(s,a) = Q_{p_a}(s,a) +\eta_a\sim\mathcal{N}(\mu^*_{p_a}, b)$}
            Append $\bar{Q}_{p_a}(s,a)$ to $\bar{Q}_{p_a}(s)$\\
        }
        $\bar{Q}^*_{p_a}(s) $ =
            $\begin{cases}
                \begin{aligned}
                    & \bar{Q}_{p_a}(s) \;\text{and} \\
                    & \hskip1em\Psi \leftarrow False,
                \end{aligned} & 
                \begin{array}{rr}
                    & \text{if }\bar{Q}_{p_a}(s,a)<Q_{p_i}(s,a) \text{ s.t. }\\
                    &  a \text{ for }{}_{max}Q_{p_i}(s) \text{ or } \gamma>\tau_\gamma
                \end{array}\\
                
                % \bar{Q}_{p_a}(s,a)<Q_{p_i}(s,a) | a \text{ for }{}_{max}Q_{p_i}(s)\\
            Continue\text{,} & \text{Otherwise until } \gamma\leq\tau_\gamma
        \end{cases}$\\

        Set $\bar{Q}_{p_a}(s) = \text{[ ]}$   
    }
    \KwRet{$\bar{Q}^*_{p_a}(s)$}
    
    \caption{Proposed PeLPA Algorithm}
    
    \label{algo:Attack}
    
\end{algorithm}






\section{Experimental Analysis} \label{s:experimentalAnalysis}
In this section, we implement our proposed PeLPA attack in a modified predator-prey domain, following the environmental specifications detailed in section~\ref{subsec:CMARL} \cite{le2017coordinated}. The environment consists of multiple predator agents and one prey. The environment is reset if the initial agent doesn't achieve the goal within a specified number of steps. Table~\ref{tab:parameter} presents the experimental parameters. For comparative insight, we investigate three environment scales: \textbf{small-scale (5x5)}, \textbf{medium-scale (10x10)}, and \textbf{large-scale (15x15)}, exploring $0\%$, $20\%$, and $40\%$ attacker percentages in each. Each experiment is repeated 10 times to average results. We use a privacy budget $\varepsilon = 1.0$ for all results presented, even though a smaller $\varepsilon$ would indicate stronger privacy protection, albeit with larger attack gains.

\textbf{Steps to Goal ($\Pi$) Analysis.} 
The $\bar{\Pi}$-values represent the average steps an agent takes to achieve the goal, with lower values indicating efficient learning. The top three charts of Fig. \ref{fig:comparison_convergence}(a-c) reveals an increase in the required step count to reach the goal as the attacker ratio rises and the environment expands. For example, after $5000$ episodes in a medium-scale environment, $\bar{\Pi} = \{7.52, 11.332, 12.364\}$ for $\{0\%, 20\%, 40\%\}$ attackers, leading to a $\frac{(11.332-7.52)\times 100}{7.52}\approx 50.69\%$ and $\frac{(12.364-7.52)\times 100}{7.52}\approx 64.41\%$ increase in average \textit{steps to goal} for $20\%$ and $40\%$ attackers, respectively.

\textbf{Reward ($\Phi$) Analysis.} Similarly, the $\bar{\Phi}$-values represent average rewards obtained by agents as shown in the bottom three charts of Fig. \ref{fig:comparison_convergence}(a-c). Our experiments exhibit a decrease in the speed of obtaining optimal rewards as the attacker ratio escalates. For instance, in a medium-scale environment, $\{2500, 3500, 4000\}$ episodes are requisite to attain the optimal $\bar{\Phi}$, for $\{0\%, 20\%, 40\%\}$ attackers, respectively. This leads to a $\frac{3500}{2500} \approx 1.4$x and $\frac{4000}{2500} \approx 1.6$x time increase in optimal $\bar{\Phi}$ acquisition for $20\%$ and $40\%$ attackers, respectively.
\setlength{\textfloatsep}{8pt}% 
\begin{table}[!ht]
\centering
\caption{\small Parameter value. $\alpha$: learning rate, $\epsilon$: exploration-exploitation probability, $\Gamma$: discount factor, $B$: communication budget, $w$: aggregation factor, $\tau, \tau', \tau_{\gamma}$: predefined threshold, $\phi$: reward, $\varepsilon$: privacy budget.}
\label{tab:parameter}
\begin{adjustbox}{max width=\linewidth}
\begin{tabular}{c|c|c|c|c|c|c|c}
\toprule
Parameter &
$\alpha$                                                      & $\epsilon$                                                    & $\Gamma$                                                      & \multicolumn{1}{c|}{$B^{tot}_{p_i}$}           & $B^{tot}_{p_a}$                          & $w$                                          & $\tau_{\gamma}$   \\ 
\midrule
Value &
0.10                                                                           & 0.08                                                                           & 0.80                                                                           & 100,000                           & 10,000                                              & 0.90                                           & 12                       \\
\midrule
\multicolumn{1}{c|}{Parameter} &
\multicolumn{1}{c|}{$\phi_{\mathpzc{G}}$} & \multicolumn{1}{c|}{$\phi_{\mathpzc{F}}$} & \multicolumn{1}{c|}{$\phi_{\mathpzc{O}}$} & $\phi_{\mathpzc{W}}$ & \multicolumn{1}{l|}{$\varepsilon$} & \multicolumn{1}{c|}{$\tau$} & $\tau'$  \\ 
\midrule
\multicolumn{1}{c|}{Value}  & 
\multicolumn{1}{c|}{10.0}                                                      & \multicolumn{1}{l|}{0.50}                                                      & \multicolumn{1}{c|}{-1.50}                                                     & -0.50                                                     & \multicolumn{1}{l|}{1.0}                            & \multicolumn{1}{l|}{100}                      & 100,000   \\
\bottomrule
\end{tabular}
\end{adjustbox}
\end{table}

\textbf{Convergence ($\Delta Q$) Analysis.} 
To gauge the effectiveness of our proposed attack, we conduct a convergence analysis based on $\overline{\Delta Q}$ values, i.e., the average of the deviation of Q-values from the optimal value ($Q^*$). An optimal learning process would have $\overline{\Delta Q}$ values tending to zero, and our analysis confirms this behavior is impeded as the attacker ratio increases. This delay in convergence correlates with the increase in attacker prevalence. Specifically, in a medium-scale environment, $\overline{\Delta Q}$ falls below $10e^{-6}$ following $\{2360, 2800, 3280\}$ episodes for $\{0\%, 20\%, 40\%\}$ attackers. Consequently, convergence is delayed by $\frac{2800}{2360}\approx 1.18$x and $\frac{3280}{2360}\approx 1.38$x for attacker ratios of $20\%$ and $40\%$, respectively.

\textbf{Adaptive Degree of Knowledge Poisoning ($\gamma$).}
\label{degreeOfPoisoning}
Finally, we consider the degree of knowledge poisoning, $\gamma$, demonstrating its distribution and symmetry in various scenarios as shown in Fig.~\ref{fig:degree}. This parameter is adjusted following line $10$ in Algorithm~\ref{algo:Attack}, showing varied instances of its manipulation across different episodes. We only present the episodes in which the attacker adjusted the $\gamma$ value more than $20$ times. For example, in episode $1146$, the attacker maintained the $\gamma$ value under $5$ for most of the steps but increased it to more than $10$ for a few steps. Contrarily, in episode $2027$, the attacker never sets $\gamma$ in the range of $\left[5,10\right]$.

\begin{figure}[!t]
\centerline{\includegraphics[width=\linewidth]{Figures/distributionOfPoisoning.pdf}}
\caption{\small Distribution of the degree of knowledge poisoning, ($\gamma$) in some example episodes. For instance, in episode $1146$, the attacker maintained the $\gamma$ value under $5$ for most of the steps but increased it to more than $10$ for a few steps.}
\label{fig:degree}
\end{figure}



%\vspace{-9mm}
\section{Conclusion}
\label{sec:con}
%\vspace{-3mm}
This work proposes DarKnight a unified inference and training platform that uses TEE to perform data obfuscation and uses GPU to perform linear operations on obfuscated data. DarKnight uses a novel matrix masking to prevent data exposure. We provide a rigorous proof that bounds DarKnight's information leakage using mutual information. We achieved the privacy of 1 bit leakage on a Megapixel image while using FP operations. 
We evaluated three different models and datasets to demonstrate considerable speedups with provably bounded data privacy leakage and also verifying the computational integrity from GPU. For large DNNs, we observe an average of $12X$ speedup for inference and $5.8X$ speedup for training without accuracy degradation over the baseline fully implemented inside TEE.



\bibliography{reference}
\nocite{*}
\bibliographystyle{IEEEtran}


\end{document}

