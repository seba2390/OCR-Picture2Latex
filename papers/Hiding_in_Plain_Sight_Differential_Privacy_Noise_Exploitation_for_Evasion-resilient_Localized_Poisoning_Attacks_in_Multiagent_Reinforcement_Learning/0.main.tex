\documentclass[conference]{IEEEtran}
\makeatletter
\def\ps@headings{%
\def\@oddhead{\mbox{}\scriptsize\rightmark \hfil \thepage}%
\def\@evenhead{\scriptsize\thepage \hfil \leftmark\mbox{}}%
\def\@oddfoot{}%
\def\@evenfoot{}}
\makeatother
\pagestyle{empty}
\usepackage{cite}
\usepackage{amsmath,amssymb,amsfonts}
\usepackage{graphicx}
\usepackage{textcomp}

\usepackage{subfigure}
\usepackage{algorithmic}

\usepackage{xcolor}
\usepackage{caption}
\usepackage{wrapfig,lipsum,booktabs}
\usepackage[export]{adjustbox}
\definecolor{red}{rgb}{2,0,0}
\usepackage{graphicx}
\ifCLASSOPTIONcompsoc
    \usepackage[caption=false, font=normalsize, labelfont=sf, textfont=sf]{subfig}
\else
\usepackage[caption=false, font=footnotesize]{subfig}
\fi
\def\BibTeX{{\rm B\kern-.05em{\sc i\kern-.025em b}\kern-.08em
    T\kern-.1667em\lower.7ex\hbox{E}\kern-.125emX}}
    
\renewcommand{\footnoterule}
    {\noindent\smash{\rule[4pt]{0.15\textwidth}{0.4pt}}}

\IEEEoverridecommandlockouts 


\DeclareMathAlphabet{\mathpzc}{OT1}{pzc}{m}{it}
\newtheorem{definition}{Definition}
\newtheorem{proposition}{Proposition}
\newtheorem{theorem}{Theorem}
\usepackage[ruled, vlined, linesnumbered, commentsnumbered, longend]{algorithm2e}
\newcommand\mycommfont[1]{\small\ttfamily\textcolor{blue}{#1}}
\SetCommentSty{mycommfont}
\usepackage[export]{adjustbox}
\usepackage{booktabs}
\begin{document}
    
\title{\Large \uppercase{Hiding in Plain Sight: Differential Privacy Noise Exploitation for Evasion-resilient Localized Poisoning Attacks in Multiagent Reinforcement Learning}}

\author{
\IEEEauthorblockN{\textbf{\uppercase{Md Tamjid Hossain}}\IEEEauthorrefmark{1}, \textbf{\uppercase{Hung La}}\IEEEauthorrefmark{2}}

\IEEEauthorblockA{Advanced Robotics and Automation (ARA) Laboratory\\ University of Nevada, Reno, NV, USA
}

\IEEEauthorblockA{
E-mail: 
\IEEEauthorrefmark{1}mdtamjidh@nevada.unr.edu, \IEEEauthorrefmark{2}hla@unr.edu}

\thanks{This work was partially funded by the U.S. National Science Foundation (NSF) under grant NSF-CAREER: 1846513.  The views, opinions, findings, and conclusions reflected in this publication are solely those of the authors and do not represent the official policy or position of the NSF.}
\thanks{*Source code of the task and the computational model behind the setup available at {\tt \small https://github.com/aralab-unr/PeLPA}}%

}

\IEEEaftertitletext{\vspace{-1.0\baselineskip}}

\maketitle




\begin{abstract}
Lately, differential privacy (DP) has been introduced in cooperative multiagent reinforcement learning (CMARL) to safeguard the agents' privacy against adversarial inference during knowledge sharing. Nevertheless, we argue that the noise introduced by DP mechanisms may inadvertently give rise to a novel poisoning threat, specifically in the context of private knowledge sharing during CMARL, which remains unexplored in the literature. To address this shortcoming, we present an adaptive, privacy-exploiting, and evasion-resilient localized poisoning attack (PeLPA) that capitalizes on the inherent DP-noise to circumvent anomaly detection systems and hinder the optimal convergence of the CMARL model. We rigorously evaluate our proposed PeLPA attack in diverse environments, encompassing both non-adversarial and multiple-adversarial contexts. Our findings reveal that, in a medium-scale environment, the PeLPA attack with attacker ratios of $20\%$ and $40\%$ can lead to an increase in average \textit{steps to goal} by $50.69\%$ and $64.41\%$, respectively. Furthermore, under similar conditions, PeLPA can result in a $1.4$x and $1.6$x computational time increase in optimal reward attainment and a $1.18$x and $1.38$x slower convergence for attacker ratios of $20\%$ and $40\%$, respectively.
\end{abstract}

\begin{IEEEkeywords}
Differential Privacy, Adversarial Learning, Poisoning Attacks, Cooperative Multiagent Reinforcement Learning
\end{IEEEkeywords}

Ensemble learning (EL)~\cite{Zhou2009Ensemble} is a well-established area of machine learning (ML) that strives for better performance by merging the predictions from various ML models.
%There are countless options when building ensembles because each combination of models can be considered as a distinct setup. 
%Three prominent methods that dominate the subfield of EL are~\cite{Sagi2018Ensemble}:
Three prominent methods for building ensembles are:~\cite{Sagi2018Ensemble}
%bagging (or bootstrap aggregation)~\cite{Breiman1996Stacked}, boosting~\cite{Freund1996Experiments,Schapire1990Strength}, and stacking (or stacked generalization)~\cite{Wolpert1992Stacked}. 
%
bagging,~\cite{Breiman1996Stacked} boosting,~\cite{Freund1996Experiments,Schapire1990Strength} and stacking.~\cite{Wolpert1992Stacked} 
%The first method 
Bagging requires training many decision trees on separate groups of instances and taking the average of their predictions.~\cite{Breiman1996Stacked}
%The second method 
Boosting attaches weak classifiers (e.g., decision stumps or shallow decision trees) sequentially, each improving the predictions made by the previous models.~\cite{Freund1996Experiments,Schapire1990Strength} 
%The last method 
Stacking involves fitting many base models from different algorithms on the same data set and using a metamodel to combine their results.~\cite{Wolpert1992Stacked} The common ground between bagging and boosting methods is that they incorporate ML algorithms that produce numerous decision trees,~\cite{Kingsford2008Decision} such as random forest (RF)~\cite{Breiman2001Random} and adaptive boosting/AdaBoost (AB),~\cite{Freund1999A} respectively. The decision paths stemming from bagged or boosted decision trees are the target of the visual analytics (VA) approach proposed in this paper.

The popularity of RF and AB is confirmed by their success in solving typical supervised classification problems, which constitute the majority of problems in the real world.~\cite{Opitz1999Popular,Wyner2017Explaining} An in-depth study~\cite{Delgado2014Do} that estimates the performances of 179 algorithms of various types~\cite{Dua2017} concludes that bagged decision trees of RF are better than other (types of) algorithms, such as deep learning approaches. 
%Moreover, in numerous Kaggle competitions~\cite{KaggleBlog}, boosted decision trees of AB (or equivalent algorithms~\cite{Chen2016XGBoost,Ke2017LightGBM}) won first place. Admittedly, EL methods have the same resonance as deep learning due to their recognition of being reliable, adjustable, robust, needing fewer computational resources, and their ability to work exceptionally well with structured tabular data~\cite{Ziv2021Tabular}.
%
Despite their remarkable predictive power, a crucial concern for algorithms that generate many decision trees is \emph{interpretability}. Brieman,~\cite{Breiman2001Statistical} for instance, indicates that RF models, while superb predictors, receive a low rating regarding their interpretability. As ML models can provide incorrect predictions,~\cite{Caruana2015Intelligible} ML experts have to check whether the model functions properly.~\cite{Tam2017An} Also, domain experts in critical fields need to understand how a specific prediction has been reached in order to trust in ML.~\cite{Zhou20182D} For example, in medicine, a physician might not rely on a model without explanations of how and why it forms a prediction, since patient lives are at risk.~\cite{Ribeiro2016Why,Hastie2001The,Lakkaraju2016Interpretable} Or, in the financial domain, declined decisions for loan applicants require additional transparency with the precise justification of the outcome.~\cite{Sachan2020An}
Although both algorithms follow the same concept of growing decision trees, their objectives differ: AB focuses on correcting misclassified training instances, while RF mainly reduces variance to achieve better generalizability. However, this fundamental goal of the former makes it susceptible to noisy cases,~\cite{Bauer1999Empirical} while the latter arguably remains intact.~\cite{Kotsiantis2007Combining}
%Governments and public institutions have also joined the discussion in the topic with, for example, the new European General Data Protection Regulation (GDPR)~\cite{GDPR}. 
Thus, one research question that remains open is: 
%
%\textbf{(RQ1)} What is the difference between rules obtained from bagged decision trees and rules derived from boosted decision trees, and is there any potential benefit in combining them, regarding interpretability enhancement for decision making?
%
%RMM: I propose the following new writing for RQ1:
%
%\textbf{(RQ1)} Given the differences between rules obtained from bagged decision trees and those derived from boosted decision trees, is there any potential benefit in combining them, regarding interpretability enhancement for decision making?
\hl{\textbf{(RQ1)} Given the differences between rules obtained from bagged decision trees and those derived from boosted decision trees, how does their combination lead to potential benefits, regarding interpretability enhancement for decision making?}
%
%\textbf{(RQ1)} How do bagged decision trees' learned rules differ from boosted decision trees, and is there any potential benefit in combining them, regarding interpretability and predictive performance?

The interpretation of ML models typically happens either at a global or a local level.~\cite{Kopitar2019Local}
%On the one hand, 
Global approaches intend to explain the ML model as a whole,~\cite{Lipton2018The} assisting domain experts in exploring the general impact of each decision and gaining confidence in the produced predictions. 
% 
On the other hand, local approaches aim to provide case-based reasoning,~\cite{Du2019Techniques,Carvalho2019Machine} allowing domain experts to review a prediction and trace its decision path in order to conclude if the decision rule, and consequently the prediction, is trustworthy.~\cite{Weller2019Transparency} 
%
Nevertheless, comparing numerous alternative decision paths without the support of an intelligent system is a time-consuming and resource-heavy procedure. For example, to scan the list of test instances rapidly and investigate specific instances of interest from multiple perspectives (e.g., outliers and borderline cases) can be crucial.~\cite{Kim2014The} Thus, the whole process can benefit from a fast approach for automatic generation and semi-automatic exploration of reliable decisions with ML experts' involvement.
%and semi-automatic exploration of decisions relevant to the given problem, 
It should also result in robust decisions, since domain experts are the most suitable for carefully examining and then manually picking sensible decisions according to their prior experience and understanding. One research question that arises from this (possibly under-researched) need for cooperation, starting from the selection of models to the extraction of insightful decisions, is: \textbf{(RQ2)} How can VA tools/systems support the collaboration between ML experts and domain experts?
%One research question that arises from the explanations---derived from Streeb et al.~\cite{Streeb2021Task}---is: \textbf{(RQ2)} How can visualizations and VA tools/systems facilitate the externalization of domain knowledge?

%The absence of VA tools for reviewing both bagged and boosted decision trees concurrently, and the lack of empowerment of ML experts' and domain experts' multidisciplinary collaboration led us to focus on the two aforementioned research questions. 
In this paper, we present \textsc{VisRuler},
%(see Figure~\ref{fig:teaser})
a VA tool that addresses the research questions described above by supporting the exploratory combination of decisions from two closely-related ML algorithms (i.e., RF and AB). \textsc{VisRuler} uses validation metrics for picking performant and diverse models and combines the decision paths from bagged and boosted trees to extract insightful and interpretable rules.
%
Our contributions consist of the following:

\begin{itemize}
\item a visual analytic workflow for defining a methodical way of evaluating decisions (cf. Figure~\ref{fig:workflow-diagram} described in Section~\nameref{sec:overview});
\item a prototype VA tool, called \textsc{VisRuler}, that applies the suggested workflow with coordinated views that support the joint effort between ML experts and domain experts for extracting rules and making decisions, respectively;
\item a use case and a usage scenario, applying real-world data, that validate the effectiveness of utilizing both bagged and boosted decision trees at the same time; and
\item a user study that showed promising results.
\end{itemize}      

\noindent The rest of this paper is organized as follows. In Section~\nameref{sec:relwo}, we discuss relevant techniques for visualizing bagging and boosting decision trees, along with tree- and rule-based models and a bulk of relevant works of visual analytics systems for multi-model comparison. Section~\nameref{sec:back} explains the core differences between bagging and boosting, and it further motivates why mixing decisions stemming from both algorithms could be beneficial for the users. In Section~\nameref{sec:goals}, we describe the design goals and analytical tasks for comparing alternative decision rules, and we present the target groups (i.e., our stakeholders). Section~\nameref{sec:overview} focuses on the functionalities of the tool and describes the first use case with the goal of identifying which countries have a higher happiness score index and why. Next, in Section~\nameref{sec:case}, we demonstrate the applicability and usefulness of \textsc{VisRuler} with a usage scenario comprising another real-world data set focusing on loan applications, followed by Section~\nameref{sec:eval} where we assess the effectiveness of \textsc{VisRuler} by reporting the results of a user study. Subsequently, in Section~\nameref{sec:lim}, we discuss several limitations of our system and opportunities for future work. Finally, Section~\nameref{sec:con} concludes our paper.
\section{Related Works} \label{s:litReview}
In this section, we address the SOTA poisoning techniques.
\subsection{Application of Differential Privacy for Knowledge Sharing}

DP, a prominent method for privacy preservation, has been extensively employed in private knowledge sharing within the realm of CMARL \cite{ye2022differential,wang2019privacy,li2022privacy, Wei2022, Abahussein2023,ye2022one}. The scope of its application includes DP-guided Q-learning models to maintain the privacy of reward data \cite{wang2019privacy}, privacy-centric multi-agent frameworks leveraging federated learning (FL) and DP to obstruct illegitimate access to data statistics \cite{li2022privacy}, and harnessing ($\beta, \phi$)-DP to counteract offloading preference inference attacks in vehicular ad-hoc networks (VANET) \cite{Wei2022}. Apart from protecting user information during private knowledge sharing, DP has also been proposed for differential advising. In particular, Ye et al. \cite{ye2022differential} propose a DP-based advising method for CMARL that enables agents to use the advice in a state even if the advice is created in a slightly different state. \textit{Nevertheless, they overlook the susceptibility of DP to poisoning attacks during knowledge sharing \cite{ye2022differential, Abahussein2023,ye2022one}.}

\subsection{Poisoning Attacks in Cooperative Multiagent Learning} The infiltration of poisoning attacks in CMARL, which can alter training datasets and consequently disrupt learning outcomes, is a pertinent research concern \cite{figura2021adversarial,fang2020local, Xie2022,mohammadi2023implicit}. Research has delved into scenarios where adversarial agents can manipulate network-wide policies \cite{figura2021adversarial}, scrutinized targeted poisoning attacks in dual-agent frameworks where one agent’s policy is modified \cite{mohammadi2023implicit}, and investigated the implications of soft actor-critic algorithms in CMARL for executing poisoning attacks \cite{Xie2022}. For instance, Figura et al. \cite{figura2021adversarial} demonstrate that an adversarial agent can persuade all other agents in the network to implement policies that optimize its desired objective. Another approach for performing poisoning attacks by any malicious advisor in multiagent Q-learning as demonstrated in \cite{hossain2023BRNES}, is to shuffle the Q-values for all actions corresponding to the requested state
and inject false noise that is similar to the maximum reward using reward poisoning method. \textit{However, the ramifications of these SOTA poisoning techniques against anomaly detection and privacy-preserving knowledge-sharing technologies remain largely unexplored. Our work endeavors to model a DP noise-exploiting poisoning attack that remains resilient to detection algorithms.}

\subsection{Differential Privacy Exploitation Techniques} Another domain of interest focuses on the possible exploitation of DP in classification challenges, even though it does not necessarily concentrate on adversarial onslaughts on CMARL algorithms \cite{giraldo2017security_2, giraldo2020adversarial, hossain2021privacy,cao2021data,cheu2021manipulation,hossain2022adversarial, Hossain2021Desmp}. This research trajectory involves the systemic degradation of utility by exploiting DP noise \cite{giraldo2020adversarial}, gauging the impact of DP manipulation in smart grid networks \cite{hossain2021privacy}, and designing stealthy model poisoning attacks on an FL model \cite{Hossain2021Desmp, hossain2022adversarial}. Similarly, \cite{hossain2021privacy} investigates the impact of DP exploitation in a smart grid network and introduces a correlation among DP parameters to enable the system designer to calibrate the privacy level and reduce the attack surface. To examine the effect of DP-exploiting attacks on an FL model, \cite{Hossain2021Desmp} proposes a stealthy model poisoning attack leveraging DP noise added to ensure privacy. They improve their attack technique in \cite{hossain2022adversarial}, investigating how the degree of model poisoning can be adjusted dynamically through episodic loss memorization in FL and demonstrating how their attack can evade some SOTA defense techniques, such as norm, accuracy, and mix detection. \textit{However, these attack models face constraints in multi-agent environments or decentralized CMARL platforms.} Contrarily, Cao et al. \cite{cao2021data} propose an attack on the Local Differential Privacy (LDP) protocol by introducing fraudulent users. \textit{Our research, however, targets legitimate yet compromised users infusing false noise into shared data, also aiming to dodge anomaly detectors - a critical objective for a successful attack.}
\section{Local Differentially Private Cooperative Multiagent Reinforcement Learning}
\label{s:DP_CMARL}

We present a local differentially private CMARL (LDP-CMARL) framework akin to the one adopted in \cite{hossain2023BRNES}. However, for demonstration simplicity, instead of a generalized randomized response (GRR) technique, we leverage a Bounded Laplace (BLP) mechanism \cite{Neera2023} to model our LDP framework that also achieves the same $\varepsilon$-LDP guarantee.

\subsection{Cooperative Multiagent Reinforcement Learning (CMARL)}
\label{subsec:CMARL} \textbf{Environment model.} Our research formalizes a cooperative reinforcement learning context with a Markov game $\mathcal{M} =(\mathpzc{N}, \mathpzc{S}, \mathpzc{A}, \Phi, \Gamma, \mathpzc{T})$ incorporating $\mathpzc{N}$ robots navigating an environment $\mathbb{E}$ of dimensions height ($\mathpzc{H}$) and width ($\mathpzc{W}$) towards a goal $\mathpzc{G}$. It introduces obstacles $\mathpzc{O}$ and freeway $\mathpzc{F}$ with corresponding reward penalties and incentives, $\phi_{\mathpzc{O}}$ and $\phi_{\mathpzc{F}}$. Dynamic obstacle positioning adds complexity to learning, which concludes when the first agent reaches $\mathpzc{G}$.

\textbf{Learning objectives.} Agent $p_i$'s objective is to take the fewest steps, $\Pi$ to reach $\mathpzc{G}$, collect $\phi_f$, avoid hitting $o_x \in \mathpzc{O}$, and earn as much as rewards, $\phi_{\mathpzc{F},\mathpzc{G}}$. In short, the objectives can be formalized as
\begin{equation}
\begin{alignedat}{2}
(a)&\;\Pi{p_i} = \underset{\mathcal{M}}{min}\;\Pi\\
(b)&\;\phi_{p_i} = \phi_{\mathpzc{F}} + \phi_{\mathpzc{G}} + \left[\phi_{\mathpzc{O}} = 0\right]\\
(c)&\;\lVert (x_{p_i}, y_{p_i})- (x_{\mathpzc{G}}, y_{\mathpzc{G}}) \rVert = 0
 \end{alignedat}
 \forall \phi_{\mathpzc{G}, \mathpzc{F}, \mathpzc{O}} \in \Phi \text{ and } \phi_{\mathpzc{G}} > \phi_{\mathpzc{F}}
 \label{eqn:objective1}
\end{equation}

where $(x_{p_i}, y_{p_i})$, and $(x_{\mathpzc{G}}, y_{\mathpzc{G}})$ are $p_i$'s and $\mathpzc{G}$'s positions.

\subsection{Integrating Local Differential Privacy (LDP) in CMARL}
LDP protocols encapsulate two main stages: perturbation and aggregation. The Q-values domain, denoted as $\mathbb{Q} = \left[q\right]$, undergoes local perturbation before being relayed to the advisee, $p_i$, ensuring $p_i$'s inability to infer the original Q-value of the advisor, $p_k$. The aggregation phase facilitates $p_i$'s estimation of optimal advice utilizing the perturbed values received from all $p_k$, with perturbation function for Q-values of all actions, $a$ in state $s$ represented as $P(Q(s))$. Following the definition of $\varepsilon$-LDP \cite{cao2021data}, a protocol achieving LDP must ensure the probabilistic resemblance between any pair of perturbed Q-values.

LDP offers plausible deniability to $p_k$, restraining $p_i$ from determining the origin of the output confidently. This ambiguity is regulated by the privacy budget, $\varepsilon$ \cite{dwork2006}. To actualize $(\varepsilon, 0)$-DP, the Laplace mechanism, a noise-addition technique, is applied as follows \cite{dwork2006}:

\begin{equation}
    \mathpzc{M}(D) = f(D) + \eta \sim \mathcal{N}(0, b)
\end{equation}

where the added
noise, $\eta$ is drawn from a zero-mean Laplace distribution with
scale parameter, $b \geq \frac{\Delta}{\varepsilon}$. Here, $\Delta$ denotes the sensitivity of the query function. Nonetheless, the same Laplace mechanism that satisfies $(\varepsilon,0)$-DP,  can be deployed in a distributed fashion for achieving $\varepsilon$-LDP \cite{wang2020comprehensive, Neera2023}, by integrating randomized Laplace noise into each state-action pair's Q-values of an advisor. We leverage the higher noise sensitivity offered by the Laplace mechanism to attain stronger privacy protection as compared to Gaussian or Exponential mechanism. The advisee, $p_i$ computes the average value from all the noisy Q-values \cite{wang2020comprehensive}. We utilize the following BLP technique for input perturbation \cite{Neera2023}: 

\begin{definition}[Bounded Laplace Mechanism (BLP)]
   Given an input $q \in \left[l, u\right] \subset \mathbb{R}$, and scale $b>0$, the BLP technique, $\mathpzc{M}: \Omega \rightarrow \left[l, u\right]$ over output $\bar{q}$ can be represented by the following conditional probability density function (pdf):
   
   \begin{equation}
       f\mathpzc{M}(\bar{q}) = 
       \begin{cases}
       0 & \text{ if } \bar{q} \notin \left[l,u\right]\\
       \frac{1}{C_q} \frac{1}{2b}e^{-\frac{\lvert \bar{q}- q\rvert}{b}} & \text{ if } \bar{q} \in \left[l,u\right]
       \end{cases}
   \end{equation}
\end{definition}
where $l$ and $u$ are the lower and upper range, and $C_q = \int_{l}^{u} \frac{1}{2b}e^{-\frac{|\bar{q}- q|}{b}} \,d\bar{q}$ is a normalization constant. The proof and further details can be found in \cite{Neera2023}. BLP constrains noise sampling within a predefined range, avoiding values that may detriment learning performance. Hence, the sensitivity of the combined LDP mechanism is $\Delta = \lvert u-l\rvert$. Similar to \cite{ye2022differential}, within our LDP-CMARL framework, the sensitivity $\Delta$ needs to be calculated carefully. The LDP-CMARL framework training stages utilizing the BLP mechanism are outlined in Algorithm~\ref{algo:DP_CMARL}. During advice request dispatch, $p_i$ specifies a neighbor zone, $\mathpzc{Z}$, and sends advice requests only to advisors within $\mathpzc{Z}$. Both $p_i$ and $p_k$ calculate their advice requesting ($\varrho_{p_i}$) and advice giving ($\varrho_{p_k}$) probabilities as per \cite{ye2022differential}. After receiving advice from the neighbors, $p_i$ aggregates all the advice following a weighted linear aggregation technique, controlled by a predefined weight parameter, $w$ \cite{hossain2023BRNES}. Then, $p_i$ selects and executes an optimal action followed by a final Q-table update.

\setlength{\textfloatsep}{0pt}% 
\begin{algorithm}[!t]
    \SetKwFunction{LDP}{LDP}
    \SetKwProg{Fn}{Function}{:}{}
    \SetKwInOut{KwIn}{Input}
    \SetKwInOut{KwOut}{Output}

    \KwIn{$\mathbb{E}, \mathpzc{N}, \mathpzc{A}, \mathpzc{S}, \Phi \rightarrow (l, u)$}
    \KwOut{Trained LDP-CMARL model}

    Initialize Q-table, set $\varepsilon, \alpha, \Gamma$, and compute $b =\frac{\alpha\lvert u-l\rvert}{\varepsilon}$

    \For{each agent, $p_i \in \mathpzc{N}$}{
        \For{each episode}{
            Initialize state, $s$
            \For{each state}{
                Send advice request to $p_k$ in $\mathpzc{Z}$ with $\varrho_{p_i}$\\
                Receive LDP-advice, $\LDP(s, \varepsilon, b)\rightarrow \bar{Q}_{p_i}(s) = \left[\bar{Q}_i(s)\right]_{i=1}^k$\\ 
                \For{each action $a\in \mathpzc{A}_i$ in state, $s$}{
                    Find weighted Q-value, $Q_{p_i}^*(s,a) = w\cdot Q_{p_i}(s, a) + (1-w)(\frac{1}{k}\sum_{i=1}^{k} \bar{Q}_i(s, a))$\\
                    Append $Q_{p_i}^*(s,a)$ to $Q_{p_i}^*(s)$
                }
            Update Q-table with $Q_{p_i}^*(s)$\\
            Choose $a^*\in \mathpzc{A}_i$ for $s$ using $\epsilon$-greedy policy\\
            Execute action, $a^*$, observe $\phi_{p_i}, s'$\\
            Perform $Q_{p_i}(s,a) \leftarrow (1-\alpha)Q_{p_i}(s,a) + \alpha \left[\phi_{p_i}+\Gamma \; \underset{a'}{max}\;Q(s', a')\right]$\\
            Set, $s\leftarrow s'$
            }
            \textbf{If }$\lVert(x_{p_i}, y_{p_i})-(x_{\mathpzc{G}, y_{\mathpzc{G}}})>0\rVert$ \textbf{then}
                continue\\
                \textbf{else} end episode and reset $\mathbb{E}$
        }
    }
    \KwRet{Trained LDP-CMARL model}\\
    \Fn{\LDP{$s, \varepsilon$, b}}{
        \For{$i = 1, 2, ..., k$ advisors}{
            Receive advice request for the state, $s$\\
            With $\varrho_{p_k}$, \For{each action $a\in \mathpzc{A}_i$}{
                find $Q_i(s, a)$ and generate $\eta_i\sim\mathcal{N}(0, b)$
                %\; \forall\; b \geq \frac{\alpha\lvert u-l\rvert}{\varepsilon}$
                \\
                Add LDP-noise,
                $\bar{Q}_i(s, a) = Q_i(s, a)+\eta_i$\\
                \eIf{$\bar{Q}_i(s, a) \notin (l, u)$}{
                    Repeat loop until $\bar{Q}_i(s, a) \in (l, u)$
                }{
                Append $\bar{Q}_i(s, a)$ to $\bar{Q}_i(s)$
                }
            }
            \KwRet{$\bar{Q}_i(s)$}
        }
        \KwRet{$\left[\bar{Q}_i(s) \right]_{i=1}^k$}  
        
    }
    \caption{LDP-CMARL Framework}
    \label{algo:DP_CMARL}
\end{algorithm}
\section{Privacy Exploited Localized Poisoning Attack}
\label{s:problemFormulation}
In this section, we dissect the DP noise exploitation mechanism, formulating adversarial noise profile challenges. We also articulate our threat model and proposed PeLPA algorithm.

\subsection{How can LDP-noise be Exploited for Poisoning Attacks?} \hskip1em \textbf{DP not included.} Considering a non-LDP advising scenario, the agents exchange Q-value knowledge, facilitating learning. We formulate the knowledge as Q-values instead of the recommended actions since the Q-value advising, unlike the action advising, does not impair the performance of the agent’s learning directly \cite{zhu2021q}. Let us assume an anomaly detector at $p_i$'s end that monitors Q-values sequences from advisor agents for all actions in a specific state, $s$. Generally, for a received Q-value, $Q_{p_k}(s)$, from advisor $p_k$, the condition $|Q_{p_{k}}(s)-Q_{0}(s)|\leq \tau$ is consistently maintained, where $\tau$ is a detection threshold and $Q_{0}(s)$, a historical standard Q-value. Any deviation raises an alarm, implying a potential malicious advisor $p_a \in \left[p_k\right]$ with biased Q-values. Nonetheless, to evade detection, the attacker can introduce a bias up to a maximum of $\tau$ relative to the standard value, i.e., $Q_{p_a}(s) \leq Q_{0}(s) + \tau$.

\textbf{DP included.} With an LDP mechanism safeguarding knowledge exchange, any received Q-value, $\bar{Q}{p_k}(s) = Q{p_k}(s) + \eta$, includes noise, $\eta$ following a zero-mean Laplace distribution, $\mathcal{N}(0, b)$, where $b$ is the distribution scale. To prevent false-positive alarms for benign differentially private Q-values, the detector adjusts the previous detection condition to $|\bar{Q}{p_k}(s)-Q_{0}(s)|\leq \tau'$ with $\tau' = \tau \times \kappa; \forall \kappa \in \mathbb{R}$, where $\kappa$ is the tolerance multiplier. This adjustment creates a poisoning window of $\lvert\tau(1-\kappa)\rvert$ that an attacker can exploit, enabling a larger bias in knowledge (i.e., Q-values) without detection. Formally, the attacker shares malicious knowledge, $\bar{Q}{p_a}(s) = Q{p_a}(s)+\eta_a; \forall \eta_a \in \lvert\tau(1-\kappa)\rvert$, where $\eta_a$ denotes the malicious noise drawn from an adversarial noise profile, $\mathcal{N}_a$. Hence, an increase in noise for privacy enhancement also expands the detection and the poisoning window.

\subsection{Challenges in Formulating Adversarial Noise Profile}
\label{noiseProfile} Crafting an adversarial noise profile, $\mathcal{\eta}_a$, that optimizes attack gain while evading anomaly detection poses a technical conundrum. A previous methodology \cite{fang2020local} attempted this by maximizing utility degradation, although this leads to a paradoxical situation in the face of an anomaly detector - more noise aids detection but less noise diminishes the attack gain. A sophisticated alternative, as proposed by \cite{giraldo2020adversarial}, models this as a multi-objective optimization problem, i.e., $\underset{\mathcal{A}}{max}\;\mathpzc{G}(\mathcal{A}, \mathcal{D}) \ni |\bar{Q}_{p_{a}}(s)-Q_{0}(s)|\leq \tau'$ where $\mathcal{A}, \mathcal{D}, \text{and } \mathpzc{G}$ denote the attack, the detect, and the gain function, respectively. The solution of this multi-criteria optimization problem is derived in \cite{giraldo2020adversarial}, where the authors presented an attack impact, $\mu^*_a$, and an optimal adversarial distribution, $\mathcal{N}_a^*(\mu_a^*, b)$ having the probability density function, $f^*_a$ as

\begin{equation}
    f_a^*(x) = \frac{k^2 - b^2}{2bc^2}e^{-\frac{|x-\theta|}{b} + \frac{(x-\theta)}{c}} \;\text{and}\; \mu_a^* = \frac{b^2(\theta-2c)-\theta c^2}{b^2 - c^2}
    \label{eqn:attackDist}
\end{equation}

where $\theta$ is the mean, $b^2$ is the variance, and $c$ is the Lagrange multiplier. $c$ can be solved numerically from \cite{giraldo2020adversarial}: 

\begin{equation}
    \frac{2b^2}{c^2 - b^2} + \ln{(1-\frac{b^2}{c^2})}=\gamma.
    \label{eqn:c}
\end{equation} 

Here, $\gamma$ is the degree of knowledge poisoning; a high $\gamma$ implies a large malicious noise injection (i.e., a higher attack gain) and vice versa. In particular, choosing a high $\gamma$ can lead to unrealistically large Q-values whereas choosing a minuscule $\gamma$ can result in negligible to almost zero attack gain. Consequently, tuning $\gamma$ for an optimal attack is non-trivial but challenging, which, unfortunately, overlooked by literature so far. We address this research gap in section~\ref{s:methodology}. Figure~\ref{fig:outlier_rmse}(a) demonstrates the influence of $\kappa$ and $\gamma$ on detected outliers and RMSE. By adding LDP-noise to $100$ uniform random values, non-DP Q-values detect a steady number of outliers for a fixed $\tau$, whereas LDP implementation significantly increases outlier detection due to benign DP Q-values flagged as false positives. This can be mitigated by setting $\tau' = \tau\times\kappa$. Moreover, an optimal attack approach as per (\ref{eqn:attackDist}) allows successful detection evasion, maintaining the baseline outlier count while inflating the system's RMSE, as shown in Fig.~\ref{fig:outlier_rmse}(b).
\begin{figure}[!ht]
\centerline{\includegraphics[width=1.0\linewidth]{Figures/Outlier_RMSE.pdf}}
\caption{\small (a) Impact of tolerance multiplier, $\kappa$ over detected outliers in both non-DP and DP settings, (b) Impact of degree of knowledge poisoning, $\gamma$ over attack evasion (difference in outlier count between non-attack and attack scenario) and attack gain (System's RMSE).}
\label{fig:outlier_rmse}
\end{figure}

\subsection{Attacker's Capability and Knowledge} We contemplate an attacker manipulating knowledge submissions to an advisee, either by exploiting susceptible agents (internal threats) or by compromising communication channels (external threats) (Fig.~\ref{fig:poisoning_framework}a, b). The attacker, in line with SOTA research \cite{dwork2019differential}, is presumed to know the publicly available $\varepsilon$-value and noise distribution.
\begin{figure}[!t]
\centerline{\includegraphics[width=\linewidth]
{Figures/poisoning_framework.pdf}}
\caption{\small (a) Internal poisoning: Attacker compromises advisors and replaces benign LDP process with adversarial LDP process, (b) External poisoning: Attacker compromises the communication path and injects additional malicious noise.} 
\label{fig:poisoning_framework}
\end{figure}
\begin{figure*}[!t]
\centerline{\includegraphics[width=\textwidth]{Figures/comparison_convergence.pdf}}
\caption{\small Average steps to goal ($\bar{\Pi}$) and obtained reward ($\bar{\Phi}$) analysis for (a) small ($\mathpzc{H}\times\mathpzc{W}=5\times 5, \mathpzc{N}=5, \mathpzc{O}=1$), (b) medium ($\mathpzc{H}\times\mathpzc{W}=10\times 10, \mathpzc{N}=10, \mathpzc{O}=3$), and (c) large-scale ($\mathpzc{H}\times\mathpzc{W}=15\times15, \mathpzc{N}=20, \mathpzc{O}=5$) environments. The number of steps is increased as well as the maximum reward achievement is delayed with more attacks (large attacker ratio). Also, (d) convergence is delayed for both $20\%$ and $40\%$ attacks compared to the no-attack baseline.} 
\label{fig:comparison_convergence}
\end{figure*}
\subsection{Proposed PeLPA Algorithm} \label{s:methodology}
A malevolent advisor, $p_a \in \mathpzc{N}$, could disrupt $p_i$'s convergence by transmitting erroneous information during the knowledge-sharing phase. Having knowledge of $\mathpzc{A, S}, \Phi, (x_{\mathpzc{G}},y_{\mathpzc{G}})$ and $p_i$'s state, $s$, $p_a$ might manipulate larger $Q$-values for a misleading action $a_m$ versus an ideal action $a_h$. This would steer $p_i$ towards a malicious point. Yet, anomalous Q-values could either invite detection or result in an insignificant attack impact. The optimal attack method in section~\ref{noiseProfile} addresses this trade-off. Our proposed PeLPA attack for LDP-CMARL is detailed in Algorithm~\ref{algo:Attack}. $p_a$ continually injects adversarial noises ($\eta_a$) to its Q-values ($Q_a(s,a)$) until    either the malicious Q-values drop below $p_i$'s maximum Q-value for an action $a$, or $\gamma$ exceeds a predetermined poisoning threshold ($\tau_{\gamma}$). Additionally, $p_a$ ensures malicious advice stays within the reward range $\bar{Q}_{p_a}(s, a) \in [l,u]$ to evade detection.

\begin{algorithm}[!ht]
    \SetKwInOut{KwIn}{Input}
    \SetKwInOut{KwOut}{Output}
    \SetKwRepeat{Do}{do}{while}
    \KwIn{$\varepsilon, b, \alpha, Q_{p_i}(s), Q_{p_a}(s)$}
    Initialize $\bar{Q}_{p_a}(s) = \text{[}\;\text{]}$ and set $\gamma\leftarrow 0, \Psi\leftarrow True, \theta\leftarrow 0$\\
    \While{$\Psi$ is True}{
        $\gamma= \gamma+1$\\
        With $b$ and $\gamma$, find $c$ numerically from (\ref{eqn:c})\\
        Then, with $c, \theta$ and $b$, find $\mu^*_{p_a}$ from (\ref{eqn:attackDist})\\
        \For{each $a\in\mathpzc{A}_i$ in state, $s$}{
            \While{$\bar{Q}_{p_a}(s, a) \notin (l, u)$}{
                $\bar{Q}_{p_a}(s,a) = Q_{p_a}(s,a) +\eta_a\sim\mathcal{N}(\mu^*_{p_a}, b)$}
            Append $\bar{Q}_{p_a}(s,a)$ to $\bar{Q}_{p_a}(s)$\\
        }
        $\bar{Q}^*_{p_a}(s) $ =
            $\begin{cases}
                \begin{aligned}
                    & \bar{Q}_{p_a}(s) \;\text{and} \\
                    & \hskip1em\Psi \leftarrow False,
                \end{aligned} & 
                \begin{array}{rr}
                    & \text{if }\bar{Q}_{p_a}(s,a)<Q_{p_i}(s,a) \text{ s.t. }\\
                    &  a \text{ for }{}_{max}Q_{p_i}(s) \text{ or } \gamma>\tau_\gamma
                \end{array}\\
                
                % \bar{Q}_{p_a}(s,a)<Q_{p_i}(s,a) | a \text{ for }{}_{max}Q_{p_i}(s)\\
            Continue\text{,} & \text{Otherwise until } \gamma\leq\tau_\gamma
        \end{cases}$\\

        Set $\bar{Q}_{p_a}(s) = \text{[ ]}$   
    }
    \KwRet{$\bar{Q}^*_{p_a}(s)$}
    
    \caption{Proposed PeLPA Algorithm}
    
    \label{algo:Attack}
    
\end{algorithm}






\section{Experimental Analysis} \label{s:experimentalAnalysis}
In this section, we implement our proposed PeLPA attack in a modified predator-prey domain, following the environmental specifications detailed in section~\ref{subsec:CMARL} \cite{le2017coordinated}. The environment consists of multiple predator agents and one prey. The environment is reset if the initial agent doesn't achieve the goal within a specified number of steps. Table~\ref{tab:parameter} presents the experimental parameters. For comparative insight, we investigate three environment scales: \textbf{small-scale (5x5)}, \textbf{medium-scale (10x10)}, and \textbf{large-scale (15x15)}, exploring $0\%$, $20\%$, and $40\%$ attacker percentages in each. Each experiment is repeated 10 times to average results. We use a privacy budget $\varepsilon = 1.0$ for all results presented, even though a smaller $\varepsilon$ would indicate stronger privacy protection, albeit with larger attack gains.

\textbf{Steps to Goal ($\Pi$) Analysis.} 
The $\bar{\Pi}$-values represent the average steps an agent takes to achieve the goal, with lower values indicating efficient learning. The top three charts of Fig. \ref{fig:comparison_convergence}(a-c) reveals an increase in the required step count to reach the goal as the attacker ratio rises and the environment expands. For example, after $5000$ episodes in a medium-scale environment, $\bar{\Pi} = \{7.52, 11.332, 12.364\}$ for $\{0\%, 20\%, 40\%\}$ attackers, leading to a $\frac{(11.332-7.52)\times 100}{7.52}\approx 50.69\%$ and $\frac{(12.364-7.52)\times 100}{7.52}\approx 64.41\%$ increase in average \textit{steps to goal} for $20\%$ and $40\%$ attackers, respectively.

\textbf{Reward ($\Phi$) Analysis.} Similarly, the $\bar{\Phi}$-values represent average rewards obtained by agents as shown in the bottom three charts of Fig. \ref{fig:comparison_convergence}(a-c). Our experiments exhibit a decrease in the speed of obtaining optimal rewards as the attacker ratio escalates. For instance, in a medium-scale environment, $\{2500, 3500, 4000\}$ episodes are requisite to attain the optimal $\bar{\Phi}$, for $\{0\%, 20\%, 40\%\}$ attackers, respectively. This leads to a $\frac{3500}{2500} \approx 1.4$x and $\frac{4000}{2500} \approx 1.6$x time increase in optimal $\bar{\Phi}$ acquisition for $20\%$ and $40\%$ attackers, respectively.
\setlength{\textfloatsep}{8pt}% 
\begin{table}[!ht]
\centering
\caption{\small Parameter value. $\alpha$: learning rate, $\epsilon$: exploration-exploitation probability, $\Gamma$: discount factor, $B$: communication budget, $w$: aggregation factor, $\tau, \tau', \tau_{\gamma}$: predefined threshold, $\phi$: reward, $\varepsilon$: privacy budget.}
\label{tab:parameter}
\begin{adjustbox}{max width=\linewidth}
\begin{tabular}{c|c|c|c|c|c|c|c}
\toprule
Parameter &
$\alpha$                                                      & $\epsilon$                                                    & $\Gamma$                                                      & \multicolumn{1}{c|}{$B^{tot}_{p_i}$}           & $B^{tot}_{p_a}$                          & $w$                                          & $\tau_{\gamma}$   \\ 
\midrule
Value &
0.10                                                                           & 0.08                                                                           & 0.80                                                                           & 100,000                           & 10,000                                              & 0.90                                           & 12                       \\
\midrule
\multicolumn{1}{c|}{Parameter} &
\multicolumn{1}{c|}{$\phi_{\mathpzc{G}}$} & \multicolumn{1}{c|}{$\phi_{\mathpzc{F}}$} & \multicolumn{1}{c|}{$\phi_{\mathpzc{O}}$} & $\phi_{\mathpzc{W}}$ & \multicolumn{1}{l|}{$\varepsilon$} & \multicolumn{1}{c|}{$\tau$} & $\tau'$  \\ 
\midrule
\multicolumn{1}{c|}{Value}  & 
\multicolumn{1}{c|}{10.0}                                                      & \multicolumn{1}{l|}{0.50}                                                      & \multicolumn{1}{c|}{-1.50}                                                     & -0.50                                                     & \multicolumn{1}{l|}{1.0}                            & \multicolumn{1}{l|}{100}                      & 100,000   \\
\bottomrule
\end{tabular}
\end{adjustbox}
\end{table}

\textbf{Convergence ($\Delta Q$) Analysis.} 
To gauge the effectiveness of our proposed attack, we conduct a convergence analysis based on $\overline{\Delta Q}$ values, i.e., the average of the deviation of Q-values from the optimal value ($Q^*$). An optimal learning process would have $\overline{\Delta Q}$ values tending to zero, and our analysis confirms this behavior is impeded as the attacker ratio increases. This delay in convergence correlates with the increase in attacker prevalence. Specifically, in a medium-scale environment, $\overline{\Delta Q}$ falls below $10e^{-6}$ following $\{2360, 2800, 3280\}$ episodes for $\{0\%, 20\%, 40\%\}$ attackers. Consequently, convergence is delayed by $\frac{2800}{2360}\approx 1.18$x and $\frac{3280}{2360}\approx 1.38$x for attacker ratios of $20\%$ and $40\%$, respectively.

\textbf{Adaptive Degree of Knowledge Poisoning ($\gamma$).}
\label{degreeOfPoisoning}
Finally, we consider the degree of knowledge poisoning, $\gamma$, demonstrating its distribution and symmetry in various scenarios as shown in Fig.~\ref{fig:degree}. This parameter is adjusted following line $10$ in Algorithm~\ref{algo:Attack}, showing varied instances of its manipulation across different episodes. We only present the episodes in which the attacker adjusted the $\gamma$ value more than $20$ times. For example, in episode $1146$, the attacker maintained the $\gamma$ value under $5$ for most of the steps but increased it to more than $10$ for a few steps. Contrarily, in episode $2027$, the attacker never sets $\gamma$ in the range of $\left[5,10\right]$.

\begin{figure}[!t]
\centerline{\includegraphics[width=\linewidth]{Figures/distributionOfPoisoning.pdf}}
\caption{\small Distribution of the degree of knowledge poisoning, ($\gamma$) in some example episodes. For instance, in episode $1146$, the attacker maintained the $\gamma$ value under $5$ for most of the steps but increased it to more than $10$ for a few steps.}
\label{fig:degree}
\end{figure}



%\vspace{-9mm}
\section{Conclusion}
\label{sec:con}
%\vspace{-3mm}
This work proposes DarKnight a unified inference and training platform that uses TEE to perform data obfuscation and uses GPU to perform linear operations on obfuscated data. DarKnight uses a novel matrix masking to prevent data exposure. We provide a rigorous proof that bounds DarKnight's information leakage using mutual information. We achieved the privacy of 1 bit leakage on a Megapixel image while using FP operations. 
We evaluated three different models and datasets to demonstrate considerable speedups with provably bounded data privacy leakage and also verifying the computational integrity from GPU. For large DNNs, we observe an average of $12X$ speedup for inference and $5.8X$ speedup for training without accuracy degradation over the baseline fully implemented inside TEE.



\bibliography{reference}
\nocite{*}
\bibliographystyle{IEEEtran}


\end{document}

