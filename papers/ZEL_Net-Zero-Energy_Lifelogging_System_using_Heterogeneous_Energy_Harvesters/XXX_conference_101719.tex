\documentclass[conference]{IEEEtran}
\IEEEoverridecommandlockouts
% The preceding line is only needed to identify funding in the first footnote. If that is unneeded, please comment it out.
\usepackage{cite}
\usepackage{amsmath,amssymb,amsfonts,pifont}
\usepackage{algorithmic}
\usepackage{graphicx}
\usepackage{textcomp}
\usepackage{multirow}
\usepackage{comment}
\usepackage{xcolor}
\usepackage[hang,small,bf]{caption}
\usepackage[subrefformat=parens]{subcaption}
\captionsetup{compatibility=false}
\newcommand{\cmark}{\ding{51}}%
\newcommand{\xmark}{\ding{55}}%
\def\BibTeX{{\rm B\kern-.05em{\sc i\kern-.025em b}\kern-.08em
    T\kern-.1667em\lower.7ex\hbox{E}\kern-.125emX}}
\begin{document}

%\title{Battery Charge-less Lifeloging System Using Energy Harvesting}
%\title{XXX-Tag: Self-powered Lifelog System Using Energy Harvesters}

%\title{XXX-Tag: Self-powered Lifelog System Using Multiple solar cells}

%\title{ZEL: Self-powered Lifelog System Using Multiple Energy Harvesters}

%\title{ZEL: Self-powered Lifelog System using Heterogeneous Energy Harvesters}

\title{ZEL: Net-Zero Energy Lifelogging System using Heterogeneous Energy Harvesters}

\author{
\IEEEauthorblockN{
Anonymized authors
}
\IEEEauthorblockA{
}
%\vspace{-5mm}
}


\maketitle

\begin{abstract}
\begin{comment}
本研究では、オフィスワーカーが敷地内で、いつどこでどんな活動をしていたかを半永久的に記録可能な最初のセルフパワー型のライフログシステムであるZELを提案します。
ZELは、特製の異なる複数のエナジーハーベスターを用いることにより、高精度なライフログを実現します。また、2つの電圧検出器を用いて、動作モードをシームレスに切り替えることで一次電池の利用を最小化し、半永久的なデータ収集を実現します。
我々は、このシステムをNグラムのタグ型ウェアラブルデバイスとして実装し,・・・・・・する実験を行いました。
実験の結果、・・・・・・として機能する事を確認しました
We presented ZEL, the first self-powered lifelogging system that allows semi-permanently record when, where and what activities office workers were doing on the premises.
ZEL achieves high accuracy lifelogging by using heterogeneous energy harvesters with different characteristics.
In addition, we use two voltage detectors to seamlessly switch operating modes, minimizing the use of primary batteries and enabling net-zero energy, semi-permanent data collection.
We implement 192 grams name-tag-shaped wearable device and conduct data collection experiments with 11 participants in a practical environment to evaluate its performance.
The evaluation results show that the Person-Dependent model achieves 8-places recognition accuracy of 87.2\%(weighted F-measure) and static/dynamic activities recognition accuracy of 93.1\%(weighted F-measure).
Furthermore, practical testing has guaranteed the long-term operation of the system and confirmed that the system achieve 99.6\% self-powered operation.
\end{comment}
We presented ZEL, the first Net-Zero Energy Lifelogging System that allows record when, where and what activity office workers were doing on the premises.
ZEL achieves high accuracy lifelogging by using heterogeneous energy harvesters with different characteristics.
In addition, we use two voltage detectors to seamlessly switch the system state, minimizing the battery usage and enabling net-zero energy data collection.
We implement ZEL on 192 grams name-tag-shaped wearable device and conduct data collection experiments with 11 participants in a practical environment to evaluate its performance.
The evaluation results show that the Person-Dependent model achieves 8-place recognition accuracy of 87.2\%(weighted F-measure) and static/dynamic activities recognition accuracy of 93.1\%(weighted F-measure).
Furthermore, practical testing has guaranteed the long-term operation of the system and confirmed that the system achieve 99.6\% zero energy i.e. net-zero energy operation.

\end{abstract}

\begin{IEEEkeywords}
Wearable, Solar, Lifelog, Place Recognition, Activity Recognition, Energy Harvesting, Sensors, Net-Zero Energy System
\end{IEEEkeywords}

\section{introduction} \label{sec:intro}
The number of patients suffering from lifestyle-related diseases is increasing and has become a social problem. In particular, office workers, tend to be get stressed and lack exercise, are at risk of suffering lifestyle-related diseases, and many studies have been conducted on the lifestyles and health of office workers\cite{daneshmandi2017adverse,shariat2017application}.
Since lifestyle-related diseases are caused by daily habits, reviewing lifestyle using lifelog is one way to prevent them. For example, the amount of daily exercise or the number of times you visit a smoking area can be objectively monitored, leading to the first step in improving your lifestyle.
Lifelog requires three types of information: when, where, and what activity.

GPS(Global Positioning System) is often used for place recognition.
It can achieve high accuracy outdoors, but not indoors because of the influence of obstructions.
For office workers, they spend most of their day in a single building, because it is important to know where they are indoors.
Therefore, indoor positioning methods using beacons with WiFi, BLE(Bluetooth Low Energy), etc.\ have been studied\cite{zou2017winips,robesaat2017improved,flab/ishida16:iiai_aai_eskm,flab/ishida18:iiai_aai_eskm}.
However, all of the methods using radio waves require an exclusive beacon and wireless communication device such as a smartphone, in order to receive radio waves.
This leads to problems with battery consumption and the cost of installing beacons.
On the other hand, in activity recognition, accelerometers are often used.
By attaching an accelerometer to a human, it is possible to recognize human activity such as walking, up or down stairs, but sufficient sampling rate and high power are required.
In the case of lifelogging, high-resolution positioning is not required as in the case of beacon-based methods, and a history of stay in a place such as a classroom or a restroom is sufficient.
Therefore, it is difficult to achieve zero energy lifelogging using inertial activity sensors that require high power consumption.
%Also, the battery capacity of small wearable devices is limited. Thus, it is difficult to use some sensors for lifelogging at all times, except for pedometers with low power consumption.

%%%%%%%%%%%%%%%%%%%%%%%%%%%%%%%%%%%%%%%%%%%%%%%%
\begin{figure}[t]
    \centering
    \includegraphics[width=1.0\hsize]{img/SP-Tag.jpg}
    \caption{Net-zero energy lifelogging system called ZEL that allows record when, where and what activity workers were doing on the office premises}
    \label{fig:ZEL}
\end{figure}
%%%%%%%%%%%%%%%%%%%%%%%%%%%%%%%%%%%%%%%%%%%%%%%%
%%%%%%%%%%%%%%%%%%%%%%%%%%%%%%%%%%%%%%%%%%%%%%%%
\begin{table*}[bt]
\centering
\caption{Summary of related work using energy harvesters as a context recognition sensor and our proposed system ZEL}
\begin{tabular}{lllcccc} \hline
\multirow{2}{*}{year} & \multirow{2}{*}{Reference} &
\multirow{2}{*}{Energy harvesters} & 
%\multicolumn{1}{l}{\multirow{2}{*}{\begin{tabular}[c]{@{}l@{}}Energy harvesters\\as a sensor\end{tabular}}} & 
\multicolumn{2}{c}{Recognition target} &
\multicolumn{1}{l}{\multirow{2}{*}{\begin{tabular}[c]{@{}l@{}}Net-zero energy\\Design\end{tabular}}} & 
\multicolumn{1}{l}{\multirow{2}{*}{\begin{tabular}[c]{@{}l@{}}Net-zero energy\\Implimentation\end{tabular}}} 
\\
& & & Place & Activity  & &\\ 
\hline \hline
2015 & \cite{lan2015estimating}                & piezoelectric             & -                & walking/running   & \color{red}\xmark  & \color{red}\xmark   \\
2017 & \cite{khalifa2017harke}                 & piezoelectric             & -                & 5-activity        & \color{red}\xmark  & \color{red}\xmark   \\
2017 & \cite{lan2017capsense, lan2020capacitor}& piezoelectric             & -                & 5-activity        & \color{red}\xmark  & \color{red}\xmark   \\
2018 & \cite{ma2018sehs, ma2020simultaneous}   & piezoelectric             & -                & walking detection & \color{red}\xmark  & \color{red}\xmark   \\
2019 & \cite{aziz2019battery}                  & radio frequency           & 3D position      & -                 & \color{red}\xmark  & \color{red}\xmark   \\
2019 & \cite{umetsu2019ehaas}                  & solar cell, piezoelectric & 9-place          & -                 & \color{red}\xmark  & \color{red}\xmark   \\
2019 & \cite{sugata2019battery}                & solar cell                & 8-place          & -                 & \color{red}\xmark  & \color{red}\xmark   \\
2020 & \cite{sandhu2020towards}                & piezoelectric             & 6-transport mode & -                 & \color{red}\xmark  & \color{red}\xmark   \\
2021 & \cite{Sandhu2021SolAREP}                & solar cell                & -                & 5-activity        & \color{green}\cmark& \color{red}\xmark   \\ 
2021 & ZEL                                     & solar cell, piezoelectric & 8-place          & static/dynamic    & \color{green}\cmark& \color{green}\cmark \\ 
\hline
\end{tabular}
\label{tab:related_work}
\end{table*}
%%%%%%%%%%%%%%%%%%%%%%%%%%%%%%%%%%%%%%%%%%%%%%%%

With the above background, there is a challenge to wearable devices that record context information in zero energy using energy harvesters which convert ambient energy into electrical energy as a power source as well as the context recognition sensor.
Energy harvesters can be used as context recognition sensors because their harvesting energy depends on the ambient.
For example, the generated power by a solar cell has information about the ambient light.
By recording the harvesting signal using the energy harvester itself, the battery lifetime of the device can be extended or even remove the battery from the device.
Umetsu et al.\cite{umetsu2019ehaas} worked on an effort to use Energy Harvesters As A Sensor(EHAAS) and showed that various energy harvesters: solar cells, piezoelectric, and peltier elements can be used for lifelogging sensor.
Furthermore, Sandhu et al.\cite{Sandhu2021SolAREP} showed that zero energy activity recognition is possible using solar cells as sensors.
However, neither research has solved the key challenges of implementing harvesting function on a small wearable device that achieving long-term lifelogging.

% In order to realize a self-powered lifelogging system, we propose and implement ZEL which uses energy harvesters as a power source as well as an activity and place sensor as depicted in Fig.~\ref{fig:ZEL}.
In order to solve the above key challenge, we propose ZEL, net-zero energy lifelogging system that record when, where and what activity office workers were doing on the office premises as shown in Fig.~\ref{fig:ZEL}.
Net-zero energy is a term used in Net-Zero Energy Building(ZEB)\cite{ZeroEner14:online}, Net-Zero Energy House(ZEH), etc.
It means that the energy consumed by the overall system is covered by harvesting energy to achieve net zero.
Specifically, we aim to design and implement a net-zero energy lifelogging system that uses continuous energy harvesting to cover 99\% of power consumption.
% The reason for net-zero instead of zero is that the use of batteries to keep time while the system is down is unavoidable, and complete zero energy is impossible.
ZEL uses two types of solar cells and a piezoelectric as a power source as well as sensor and realizes zero energy data recording by intermittent operation using a capacitor.
In addition, seamless system state switching using dual power switching mechanism and two voltage detectors guarantee a net-zero energy and semi-permanent battery lifetime without battery recharging.
The data recorded in onboard EEPROM is extracted via USB cable, and the application builds a lifelog that includes information: where, when, and what activity.
In order to evaluate ZEL, we conducted data collection experiments in various environments.
The evaluation results show that 8-place recognition accuracy of 87.2\%(weighted F-measure) and static/dynamic activities binary classification accuracy of 93.1\%(weighted F-measure).
In addition, practical testing has shown that 99.6\% zero energy i.e. net-zero energy operation has been achieved.

The contribution of this paper are summarized as follow:
\begin{itemize}
    \item We design and implement ZEL, net-zero energy lifelogging system and implement it in a name-tag-shaped wearable device.
    \item In order to evaluate the proposed system, we conducted data collection experiments in different weather conditions(sunny, cloudy, rainy) and on different dates(6 days), and collected a total of 11 hours of data for 11 participants.
    The evaluation results show that the Person-Dependent model achieves 8-place recognition accuracy of 87.2\%(weighted F-measure) and static/dynamic activities binary classification accuracy of 93.1\%(weighted F-measure).
    \item By scheduling the system state using dual power switching mechanism and two voltage detectors, our proposed system has guaranteed to work semi-permanently and we achieve 99.6\% zero energy operation in practical testing.
\end{itemize}

% FIX:評価のところの文章かえる
The rest of this paper is organized as follows. Section~\ref{sec:related_work} reviews related work using energy harvesters as a sensor.
Section~\ref{sec:proposed_system} presents our proposed system.
In Section~\ref{sec:implementation}, we describe the implementation of our proposed system.
In Section~\ref{sec:eval_accuracy} and Section~\ref{sec:investigation}, we describe the evaluation results and the limit of performance.
Finally, Section~\ref{sec:conclusion} concludes this paper.

\section{Related work} \label{sec:related_work}
In this section, we describe the application of energy harvesting and related work on context recognition using energy harvesters.
Ma et al.\cite{ma2019sensing} reported that energy harvesting materials are attracting attention for the large-scale deployment of wearable Internet of Things(IoT) devices due to issues such as battery life and disposal.
As reported by Ma et al, research, on various wearable devices such as backpack\cite{xie2014human}, footwear\cite{zhao2014shoe} and wristband\cite{maharjan2019high} or applications\cite{Winkel2020gameboy,fraternali2020ember} using energy harvesting is becoming more and more popular.
Besides the use of energy harvesters to extend battery lifetime, some studies are using energy harvesters as sensors for context recognition.
Using an energy harvester simultaneously as a sensor and a power source removes the need to install other sensors that require a power source for context recognition, and reduces the overall cost and power consumption of the device.
The following section describes some studies using energy harvester as a sensor.

Khalifa et al.\cite{khalifa2017harke} reported the performance evaluation of human activity recognition using Kinetic Energy Harvesting(KEH), which converts kinetic energy into electric power.
As a result, their proposed system consumes less power than the conventional sensor-based system.
Ma et al.\cite{ma2018sehs, ma2020simultaneous} proposed a mechanism to use the KEH as a power source as well as a sensor.
They consider the distortion of the harvesting energy's sensing signal during the energy harvesting process and proposed a filtering algorithm to compensate for it.
Using a device that implements their algorithm, they have demonstrated that it can detect walking with higher accuracy than the previous system.
Lan et al.\cite{lan2015estimating} used KEH to classify walking/running. The results showed that the classification accuracy was close to accelerometer's one.
However, the system did not have a harvesting function and required an external battery.
Lan et al. proposed CapSense\cite{lan2017capsense, lan2020capacitor}, which connects the KEH to a capacitor and recognizes human activities based on its charging rate.
Using a wearable device with CapSense implemented in shoes, they have demonstrated that the system can achieve 95\% accuracy in recognizing 5 human activities while consuming 57\% less power than conventional systems.
However, for some static activities, the system took a long time to charge the capacitor.
Aziz et al.\cite{aziz2019battery} proposed a zero energy 3D positioning system using the power of RF(Radio Frequency) signals.
64 antennas received the radio waves generated by a huge beacon, enabling highly accurate 3D positioning.
However, it is not net-zero energy because it requires a lot of power on the environment side.
Umetsu et al.\cite{umetsu2019ehaas} proposed a room-level place recognition system using Energy Harvesters As A Sensor(EHAAS) for lifelogging.
They concluded that solar cells are the best harvesters for room-level place recognition, and achieved highly accurate place recognition with only two types of solar cells.
Sugata et al.\cite{sugata2019battery} implemented EHAAS on a name-tag-shaped device and achieved highly accurate place recognition under limited conditions.
Sandhu et al.\cite{sandhu2020towards} used KEH to recognize the transportation system such as train, ferry, etc. from its vibration.
After large-scale data collection experiments, they reported that some transportation systems with large vibrations achieved zero energy, where the power generated by the energy harvester exceeded the power used to record sensor signals.
Sandhu et al.\cite{Sandhu2021SolAREP} used Solar Energy Harvesting(SEH), which converts solar energy into electricity, to recognize 5 human activities.
They implemented a human activity recognition pipeline in a wearable device, including everything from sensor signal acquisition to wireless communication.
And showed that the pipeline can be executed with net-zero energy.
However, the prototype used for the evaluation did not have a harvesting function and required an external battery.

Each related work is summarized in TABLE~\ref{tab:related_work}.
Many studies are aimed at reducing the power consumption of context recognition and allow the use of external power sources, thus they are not net-zero energy.
Some studies have achieved zero energy in limited environments, e.g., while walking, but not net-zero energy because of the reliance on batteries while not walking.
Also, some studies have shown that net-zero energy can be achieved by measuring the power consumption of harvesting energy and overall system, but none have been implemented.

In order to achieve high accuracy and operation in various environments, we employed both harvesters with stable and unstable power generation.
The stable one guarantees a wide operation environment, while the unstable one enables highly accurate context recognition; it is greatly affected by the surrounding ambient.
In addition, two voltage detectors minimize the use of a battery, resulting in a net-zero energy device.

\section{ZEL: net-zero energy lifelogging system using heterogeneous energy harvesters} \label{sec:proposed_system}
ZEL consists of an energy harvesting data collection block and a lifelog generation block as shown in Fig.~\ref{fig:system_configuration}.
In the data collection block, the device records the data for lifelogging intermittently in self-powered.
In the lifelog generation block, the data manually extracted from the device and recognize the place and activity, then build a lifelog which have three types of information:when, where and what activity.
In this section, we describe the operation of each block.

%%%%%%%%%%%%%%%%%%%%%%%%%%%%%%%%%%%%%%%%%%%%%%%%
\begin{figure}[bt]
    \centering
    \includegraphics[width=0.9\hsize]{img/system_configuration.png}
    \caption{Proposed ZEL model, consisting of data collection block and lifelog generation block}
    \label{fig:system_configuration}
\end{figure}
%%%%%%%%%%%%%%%%%%%%%%%%%%%%%%%%%%%%%%%%%%%%%%%%
%%%%%%%%%%%%%%%%%%%%%%%%%%%%%%%%%%%%%%%%%%%%%%%%
\begin{figure}[bt]
    \centering
    \includegraphics[width=0.9\hsize]{img/sensor_data.png}
    \caption{Transition of solar cell's harvesting energy(open-circuit voltage) when visiting various places(average sampling rate is 2.15Hz).}
    \label{fig:voltage_sensor_data}
\end{figure}
%%%%%%%%%%%%%%%%%%%%%%%%%%%%%%%%%%%%%%%%%%%%%%%%
%%%%%%%%%%%%%%%%%%%%%%%%%%%%%%%%%%%%%%%%%%%%%%%%
\begin{figure}[bt]
    \centering
    \includegraphics[width=0.9\hsize]{img/hysterisis.png}
    \caption{Hysteresis characteristics}
    \label{fig:hysterisis}
\end{figure}
%%%%%%%%%%%%%%%%%%%%%%%%%%%%%%%%%%%%%%%%%%%%%%%%
%%%%%%%%%%%%%%%%%%%%%%%%%%%%%%%%%%%%%%%%%%%%%%%%
\begin{figure*}[bt]
    \centering
    \includegraphics[width=1.0\hsize]{img/state_transition.png}
     \caption{System state scheduling by capacitor voltage using two voltage detectors(when user move from a bright place to a dark place)}
    \label{fig:device_flow}
\end{figure*}
%%%%%%%%%%%%%%%%%%%%%%%%%%%%%%%%%%%%%%%%%%%%%%%%
\subsection{Data collection block}
In the following sections, we first explain the idea of record the harvesting signal in self-powered.
Next, we describe the system states scheduling using two voltage detectors.

\subsubsection{Recording the time-series data using harvesting energy}
%中村コメント:このセクションで何を書きたいのか日本語の補足が欲しい。下記の文で、梅津の研究は含まれない?
Inspired by the ideas of Sugata et al.\cite{sugata2019battery} and Capsense\cite{lan2017capsense, lan2020capacitor}, we record the time-series data using harvesting energy without a battery.
% intermittently:間欠的に(周期的にマイコンがアクティベートする)
% どこなにの情報はハーベスタ、時間の情報はRTC
% RTCは時間保持のために電池がないといけない
% セルフパワーだけどPCにもあるからいいでしょ
% システムを停止中または動作不可能な場合、電源を切り替えることで時間を保持
% FIX
%中村コメント:レビュアーの中には、いきなり間欠動作って言われてもピンとこない人がいると思うので、もう少し前置きの情報を追加する。間欠動作すると何が嬉しいのか?そもそも、なぜ間欠動作が求められるのか?キャパシターを繋がないとどんな問題が起きてしまうのか?を追記する。
By connecting the energy harvesters to a capacitor, the microcontroller is activated intermittently.
Specifically, when the microcontroller is activated, it acquires the harvester signal(open-circuit voltage), and when the microcontroller is slept the capacitor is recharged.
In order to recognize place and activity, we employ heterogeneous solar cells and piezoelectric that have been confirmed to use as sensors for place and activity by previous studies\cite{umetsu2019ehaas, sugata2019battery,khalifa2017harke} as energy harvesters.
We combined one dye-sensitized solar cell, which has stable power generation characteristics even indoors, and one amorphous solar cell, which has unstable power generation characteristics to enable stable operation and high accuracy context recognition.
The information of where and what activity is obtained by these harvesters and the information of when is obtained by the Real-Time Clock(RTC).
Since RTCs need to keep time even when the system is down, the device needs a timekeeping battery.
A de-facto standard solution adopted by many mobile devices is the approach of integrating a small timekeeping battery into the circuit board.
In this paper, we follow this standard approach for our system.
By recording time data together with harvester signals, it becomes time-series data, which is then written to non-volatile memory.
On top of that, we aim to design and implement a wearable device that can operate net-zero energy, semi-permanently on its own energy harvesting system without battery recharging.

%In general, timekeeping batteries are initially installed in all mobile devices because their lifespan (approximately 3-4 years) is longer than the lifespan of a typical wearable device.
%The RTC needs to keep time even when the system is down, therefore the device has to have a battery.
%Batteries for time-keeping are generally used in various systems, such as personal computers, and do not need to be replaced for three to four years.


Fig.~\ref{fig:voltage_sensor_data} shows the generated power when wearing the device implementing ZEL and visiting various places.
% The voltage of sc1(dye-sensitized) is fluctuating greatly, while sc2(amorphous) is generating power stably at all places.
The voltage of sc1(dye-sensitized) is generating power stably at all places, while sc2(amorphous) is fluctuating greatly.
The average sampling rate in real-world operation tests was about 2.15Hz.

%%%%%%%%%%%%%%%%%%%%%%%%%%%%%%%%%%%%%%%%%%%%%%%%
\begin{figure}[bt]
    \centering
    \includegraphics[width=0.8\hsize]{img/intermittent.png}
    \caption{Period of intermittent operation dependent on brightness}
    \label{fig:intermittent}
\end{figure}
%%%%%%%%%%%%%%%%%%%%%%%%%%%%%%%%%%%%%%%%%%%%%%%%

%%%%%%%%%%%%%%%%%%%%%%%%%%%%%%%%%%%%%%%%%%%%%%%%
\begin{figure*}[t]
    \centering
    \includegraphics[width=1.0\hsize]{img/implementation.png}
    % TODO: check typo
    \caption{ZEL manufacturing details. The main components are: (A) solar cells connectors, (B) 47$\mu F\,$capacitor, (C) Renesas RY7011 with built-in RL78/G1D and antenna ultra-low power BLE module, (D) STMicroelectronics M95256-WMN6P SPI bus EEPROM, (E) Ricoh RP118N221B-TR-FE Low Drop Out(LDO) Voltage Regulator, (F) Texas Instruments TPS22860DBVR ultra-low leakage current load switch, (G) XC6134C21EMR-G and XC6134C22CMR-G voltage detectors, (H) piezoelectricity connector, (I) LED, (J) micro USB connector, (K) coin battery(CR2032)}
    \label{fig:circuit}
\end{figure*}
%%%%%%%%%%%%%%%%%%%%%%%%%%%%%%%%%%%%%%%%%%%%%%%%

\subsubsection{System state scheduling using dual power switching mechanism and two voltage detectors}
In order to guarantee semi-permanent, net-zero energy, two items are required as follows:
\begin{itemize}
    \item Can recover after visiting a dark place
    \item Minimize the use of a battery
\end{itemize}
We satisfied these requirements with system state scheduling using dual power switching mechanism and two voltage detectors.

% 間欠動作のためにdetectorを使う研究はすでにある
% ↑の説明する
% しかし動作保証外(今回の場合で言う暗い場所)環境では復旧不可能なためチャージ中は一次電池を借りる(パワースイッチングメカニズム)
% 一次電池を使うと半永久的動作ガー
% 閾値にオフセットをかけた2つ目のボルテージディテクターによって一次電池の使用を最小化

Intermittent operation using energy harvesting is usually implemented by voltage detectors or comparators with hysteresis characteristics\cite{electronics8121446,Chen2017BatteryfreePM}.
The input of the voltage detector is connected to a capacitor and the output is switched when the capacitor is charged by energy harvesters or discharged by the system power consumption.
The system becomes active when the voltage detector detects that the capacitor has been charged with enough power to operate the system and becomes sleep to recharge capacitor when the detector detects power consumption.
However, in case the system is not able to generate enough power (visiting a dark place in ZEL), a power switching circuit needs to switch the power source to the battery when power consumption is detected.
Since battery consumption needs to be minimized to achieve net zero energy we devise a mechanism of minimizing the battery uses with state scheduling using one additional voltage detector.

One of two voltage detectors is for switching the operation mode(sleep or active) of the microcontroller and the other is for power switching(capacitor or battery).
Each voltage detectors have hysteresis characteristics as shown in Fig.~\ref{fig:hysterisis}, and it has two threshold values.
The detector thresholds for the operating mode is a positive offset of the thresholds for power switching as shown in Fig.~\ref{fig:device_flow}.
State 1 to 4 shows the operation in a bright place, and states 5 and 6 show the operation in a dark place.
Each state is shown below.
\begin{enumerate}
    \renewcommand{\labelenumi}{\textcircled{\scriptsize \theenumi}}
    \item Supply from battery, microcontroller is in sleep.
    \item The power source switches to the capacitor.
    \item Transit to the recording sequence, and after recording, the power is quickly consumed by the LED.
    \item The microcontroller transit to sleep and the capacitor will be charged if there is enough power generation.
    \item The microcontroller transit to sleep and the capacitor voltage will decrease if there is not enough power generation.
    \item The power source switches to the battery.
\end{enumerate}
As shown in the above state scheduling, the use of two voltage detectors enable power supply from the capacitor even when the capacitor is being charged, thereby minimizing the use of the battery.
% In the previously proposed system, once visited a dark place and the capacitor voltage fell below the minimum operation voltage, the system could not recover even if the capacitor voltage was charged to the operation voltage again.
% To solve this problem, we use voltage detectors.
% This voltage detector has hysteresis characteristics as shown in Fig.~\ref{fig:hysterisis}, and it has 2 threshold values.
% When a capacitor is connected to the voltage detector and its output changes to "High", the microcontroller shifts to recording operation.
% And when the output changes to "Low", the microcontroller sleeps and shifts to capacitor charging.
% The above state transitions enable intermittent operations such as Fig.~\ref{fig:intermittent}.
% However, if the capacitor voltage falls below the minimum operation voltage, the microcontroller becomes unstable and the system cannot be recovered.
% Therefore, when the output of the voltage detector changes to low, the system switches to the primary battery for RTC.
% That makes a stable power supply at all times, and the system can be recovered.
% To realize a battery charge-less and maintenance-free, a mechanism that minimizes the use of the primary battery is required.
% Therefore, we designed a mechanism to minimize the use of the primary battery by using 2 voltage detectors, one for switching the power source and the other for triggering the microcontroller to sleep.
% Fig.~\ref{fig:device_flow} shows the state transition of the system.
\begin{comment}
\subsubsection{Implementation of proposed system}
In this study, we implemented the proposed method on a name-tag-shaped device for office workers.
In this paper, we refer to this device as a name-tag sensor.
In the name-tag sensor, 2 solar cells with different characteristics in TABLE~\ref{tab:solar_cell} and piezoelectric element are used as energy harvesters.
By combining a dye-sensitized solar cell(sc1) and an amorphous solar cell(sc2), we can realize highly accurate context recognition while ensuring a stable power supply.
Fig.~\ref{fig:voltage_sensor_data} shows the transition of the solar cell voltage when wearing the name-tag sensor and visiting various places.
The average sampling rate in real-world operation tests was about 2.15Hz.
Using the power generated from these energy harvesters, the device acquires its own harvested energy(open-circuit voltage) and records to the non-volatile memory in the device.
At this time, the device uses only the power from the solar cell and not the power from the piezoelectric element as the power source.
The reason for this is that we did not consider the use of the piezoelectric element as power when designing the circuit.
The appearance of the name-tag sensor and the name-tag sensor circuit are shown in Fig.~\ref{fig:device_appearance} and Fig.~\ref{fig:sensor_board}.

\end{comment}


\subsection{Lifelog generation block}
The lifelog generation block generates a lifelog from the time-series data recorded in the data collection block, with three information: when, where, and what activity.
Where and what activity are predicted using machine learning.
In the following section, we describe in detail each operation: prepossessing, feature extraction, and machine learning in the lifelog generation block.

\subsubsection{Prepossessing}
First, since the data immediately after power-on may contain outliers, we remove the data in the first 30 seconds.
Then add the sampling rate to the data calculated from the timestamp acquired by the RTC because it has also been confirmed that the rate of intermittent operation is important for context recognition\cite{lan2017capsense,lan2020capacitor}.
For example, as shown in Fig.~\ref{fig:intermittent}, the sampling rate varies greatly depending on the brightness of the room.
In order to apply time-related features, each data is divided using a fixed-length window. The window size is 1.24 seconds\cite{torigoe2020strike, nakamura2019waistonbelt}, which is commonly used in accelerometer activity recognition.
The window overlap rate is 50\%.

\subsubsection{Feature extraction}
In this study, since time series data is the target, we use the 17 features that have been validated in previous studies\cite{torigoe2020strike,nakamura2019waistonbelt} on activity recognition using accelerometers as following: mean, standard deviation, median absolute deviation, maximum, minimum, sum of squares, entropy, interquartile range, coefficients of the fourth-order Burg autoregressive model, range of minimum and maximum values, root mean square, skewness of the frequency signal, kurtosis of the frequency signal, maximum frequency component, weighted average of the frequency signal, spectral energy of the frequency band, and power spectral density.
%In this study, we will discuss 17 features: mean, standard deviation, median absolute deviation, maximum, minimum, sum of squares, entropy, interquartile range, coefficients of the fourth-order Burg autoregressive model, range of minimum and maximum values, root mean square, skewness of the frequency signal, kurtosis of the frequency signal, maximum frequency component, weighted average of the frequency signal, spectral energy of the frequency band, and power spectral density.

\subsubsection{Machine learning}
In this part, we construct a machine learning model that outputs place labels and activity labels.
In this study, nine popular machine learning algorithms: Support Vector Machine (SVM), Artificial Neural Network (ANN), Random Forest (RF), Decision Tree (DT), LightGBM Logistic Regression (LR), K-Nearest Neighbor (KNN), Naive Bayes (NB), and Extra-Trees (ET) is used for classification and determine the algorithm to be embedded in the final proposed system.
Since lifelogging does not require detailed position information, we perform majority voting for a certain number of samples for each place and activity predicted by the model.
In the next section, we will show how the accuracy of majority voting depends on the number of samples(1 sample: 1.24 seconds)

\section{ZEL implementation} \label{sec:implementation}
We implemented the design described in Section~\ref{sec:proposed_system} on a PCB and embedded in a name-tag-shaped wearable device considering the use of office workers.

\subsection{Implement circuit and wearable device}
We implement ZEL circuit shown in Fig.~\ref{fig:circuit}.
The size of the circuit is 50mm$\times$61mm and it weight 15 grams.
The circuit has three solar cell connectors and one piezoelectric connector.
The name-tag-shaped device that embeds this circuit shown in the Fig.~\ref{fig:ZEL} can be worn around the neck so that it does not interfere with desk work and its size is 123mm square on a side and it weights 192 grams.
Renesas RY7011 is used as the microcontroller to achieve ultra-low power consumption performance.
The LED quickly consumes power when the recording operation is finished in order to quickly move to the charging sequence.
Extract data stored in EEPROM via micro USB and pass it to the lifelog generation block.

\subsection{ZEL energy harvesters}
As described in Section~\ref{sec:proposed_system}, we employ one dye-sensitized solar cell and one amorphous solar cell and one piezoelectrical as energy harvesters.
Each specification is shown in TABLE~\ref{tab:specification_of_harvesters}.
The reason why we used different types of solar cells is for wide operating range and high accuracy recognition, as described in Section~\ref{sec:related_work}.
A weight made of a screw and nut is attached to the tip of the piezoelectric, and the sensor detects vibrations generated by user activity.
At this time, the device uses only the power from solar cells and not the power from the piezoelectric element as the power source.
The reason for this is that we did not consider the use of the piezoelectric element as power when designing the circuit.
Two solar cells are connected in parallel to connect to a capacitor and acquire the amount of harvesting energy by connecting those open-circuit voltages to the AD port of the microcontroller.

%%%%%%%%%%%%%%%%%%%%%%%%%%%%%%%%%%%%%%%%%%%%%%%%
\begin{table}[bt]
    \centering
    \caption{Specifications of the energy harvesters used for ZEL, one dye-sensitized solar cell(sc1), one amorphous solar cell(sc2), one piezoelectric}
    \begin{tabular}{cccc} \hline
        Type   & Dye-sensitized(sc1) & Amorphous(sc2) & Piezoelectric \\ \hline \hline
        Image   & 
        \begin{minipage}{20mm}
            \centering
            \includegraphics[width=0.6\hsize]{img/color.png}
        \end{minipage} &
        \begin{minipage}{20mm}
            \centering
            \includegraphics[width=0.6\hsize]{img/amo.png}
        \end{minipage} &
        \begin{minipage}{10mm}
            \centering
            \includegraphics[width=0.6\hsize]{img/piezo.png}
        \end{minipage} \\ \hline
        Power   & 252$\mu W$ & 332$\mu W$ & 400mV/g \\ \hline
        Size & 97mm$\times$57mm & 96mm$\times$47mm & 13mm$\times$25mm \\ \hline
        Weight & 20.0g & 16.2g & 2.5g \\ \hline
    \end{tabular}
    \label{tab:specification_of_harvesters}
\end{table}
%%%%%%%%%%%%%%%%%%%%%%%%%%%%%%%%%%%%%%%%%%%%%%%%

\subsection{Implementation of the system state scheduling}
The part of the system state scheduling is implemented using two voltage detectors, LDO, and load switch as shown in Fig.~\ref{fig:schematic}.
The output of the voltage detector for power switching is connected to the EN pin of the LDO, and the inverted output is connected to the ON pin of the load switch to which the coin cell battery is connected.
This is how the dual power switching mechanism is realized.
The output of the voltage detector for operation mode switching is connected to the interrupt pin of the microcontroller.
This pin is used as a trigger to switch between sleep and activate of the microcontroller.
The above enables system state scheduling as shown in Fig.~\ref{fig:device_flow}.

%%%%%%%%%%%%%%%%%%%%%%%%%%%%%%%%%%%%%%%%%%%%%%%%
\begin{figure}[t]
    \centering
    \includegraphics[width=1.0\hsize]{img/schematic.png}
    \caption{Schematic of the system state scheduling part(some ceramic capacitors are omitted for simplicity)}
    \label{fig:schematic}
\end{figure}
%%%%%%%%%%%%%%%%%%%%%%%%%%%%%%%%%%%%%%%%%%%%%%%%

%%%%%%%%%%%%%%%%%%%%%%%%%%%%%%%%%%%%%%%%%%%%%%%%
\begin{table}[b]
    \centering
    \caption{Experimental scenario(Move from the top of the table to the bottom)}
    \label{tab:scenario}
    \begin{tabular}{lll} \hline
        Place & Activity & Duration \\ \hline \hline
        Lab1 & sitting,standing,walking & 5 minutes \\
        Lab2 & sitting,standing,walking & 5 minutes \\
        Lab3 & sitting,standing,walking & 5 minutes \\
        Lab4 & sitting,standing,walking & 5 minutes \\
        Lab5 & sitting,standing,walking & 5 minutes \\
        9F Hall & standing,walking & time for moving \\
        Elevator & standing & time for moving \\
        4F Hall & standing,walking & time for moving \\
        Stairs  & downstairs & time for moving \\
        1F Hall & standing,walking & time for moving \\
        MTG room & sitting & 5 minutes \\
        1F Hall  & standing,walking & time for moving \\
        Convenience store & standing,walking & 5 minutes \\
        Outdoors & sitting,standing,walking& 5 minutes \\
        1F Hall & standing,walking & time for moving \\
        Stairs & upstairs & time for moving \\
        4F Hall & standing,walking & time for moving \\
        Elevator & standing & time for moving \\
        9F Hall & walking & time for moving \\
        Restroom & standing & 5 minutes \\
        9F Hall & walking & time for moving \\ \hline
    \end{tabular}
\end{table}
%%%%%%%%%%%%%%%%%%%%%%%%%%%%%%%%%%%%%%%%%%%%%%%%

%%%%%%%%%%%%%%%%%%%%%%%%%%%%%%%%%%%%%%%%%%%%%%%%
\begin{figure*}[t]
    \centering
    \includegraphics[width=1.0\hsize]{img/place.png}
    \caption{A look at the data collection experiment}
    \label{fig:experiment}
\end{figure*}
%%%%%%%%%%%%%%%%%%%%%%%%%%%%%%%%%%%%%%%%%%%%%%%%

\section{Evaluation of the recognition accuracy of ZEL} \label{sec:eval_accuracy}
In this section, we describe the data collection method and its experimental environment for evaluating the proposed system, and then we evaluate the following three items:
\begin{itemize}
    \item Scaling the number of the majority samples
    \item Comparison to previous studies and sensor-based method
    \item Evaluation of versatility by Person-Independent model
\end{itemize}
We used weighted F-measure as the metrics for evaluation.

\subsection{Data collection and experimental environment}
We conducted data collection experiments to evaluate ZEL.
Since solar cells are used as energy harvesters, the influence of external light should always be considered.
Therefore, we conducted data collection experiments on 11 participants in different weather conditions(sunny, cloudy, rainy) and on different dates(six days).
In order to ensure practicality, we made a list of possible activities and places to visit in a day in the university and designed a scenario that covers 14 places and 5 activities as shown in TABLE~\ref{tab:scenario}.
The participants stay and move from the top to the bottom of the table.
To compare with the conventional method, the participants wear a ZEL, accelerometer, and an illumination sensor and live according to the scenario.
In order to acquire data on various activities, there are some places where we specify the activities of the participants.
For example, in the laboratory, the participants sit and work.
In summary, we collected data on 14 places and 5 activities: sitting, standing, walking, upstairs, and downstairs for 11 participants for a total of 11 hours in various light environments.
A look at the data collection experiment is shown in Fig.~\ref{fig:experiment}.
We manually annotated the places and activities for the collected data.
Since the most important information for the lifelog is not the detailed position but the rough place, the labs 1 to 5 and the halls of each floor were grouped together.
As for activities, since it is important to know how much exercise they did during the day, we grouped the activities into two categories: static(sitting, standing) and dynamic(walking, upstairs, down the stairs).

\subsection{Scaling the number of the majority samples}
As described in Section~\ref{sec:proposed_system}, we apply majority voting to the labels output by the model.
Especially in terms of place, since the transition is slow, majority voting also plays a role in removing noise.
In order to determine the optimal number of majority samples, we recognize multiple numbers of majority samples and evaluate with the Person-Dependent model obtained by 10-fold Cross Validation(CV) for each user.
We chose the machine learning algorithm with the highest average accuracy, that is LightGBM for place recognition and SVM for activity recognition.
The user average accuracy of place and activity recognition for each sample size is shown in Fig.~\ref{fig:majority_sample_scaling}.
The best accuracy was acquired for the majority sample size of 20(13.02 seconds), which improved the accuracy by about 2\% compared to the case without majority voting.
Thus we use a sample size of 20 in the following evaluation.

%%%%%%%%%%%%%%%%%%%%%%%%%%%%%%%%%%%%%%%%%%%%%%%%
\begin{figure}[t]
    \centering
    \includegraphics[width=0.9\hsize]{img/majority_sample.png}
    \caption{Comparison of estimation accuracy when scaling the number of the majority samples}
    \label{fig:majority_sample_scaling}
\end{figure}
%%%%%%%%%%%%%%%%%%%%%%%%%%%%%%%%%%%%%%%%%%%%%%%%

\subsection{Comparison to previous studies and sensor-based system}
In order to compare previous studies and conventional sensor-based system using accelerometer(acc) and illuminance sensor(ill), we perform 8-place recognition summarizing the labels of laboratory and hall, and classify static/dynamic activities.
For evaluation, we use the Person-Dependent model as in the previous subsection.
For a fair comparison, up-sampling of 100Hz, the sampling rate of accelerometers, was applied to the data acquired by the ZEL and the illuminance sensor.
We used linear interpolation for up-sampling.
The results of choosing the most accurate model are shown in Fig.~\ref{fig:10fold_accuracy}.
The models with underscores represent the data used: \textit{sc1} means the amount of harvested energy by dye-sensitized solar cells, \textit{sc2} means the amount of harvested energy by amorphous solar cells, \textit{sr} means the sampling rate, \textit{acc} means the acceleration sensor, and \textit{ill} means the illumination sensor.
For example, \textit{ZEL\_sc1} is a model created only from the dye-sensitized solar cell power harvesting data and does not use other data such as piezoelectric elements.
\textit{sensor\_all} is a model created from the accelerometer and illumination sensor data.
The hyperparameters of each machine learning algorithm were not adjusted, and the default values were used.
For place recognition, ZEL achieved the second-highest accuracy after the model combining accelerometer and illuminance sensor.
This accuracy is comparable to that of previous studies\cite{umetsu2019ehaas,sugata2019battery}, indicating that ZEL is capable of place recognition even under practical use.
In most cases, the accuracy of LightGBM was the highest. This may be due to the overfitting of the Person-Dependent model.
For activity recognition, the model using only piezoelectric data achieved the highest accuracy of 93.2\%, followed by the model using the accelerometer.
It was also confirmed that the models using only the harvesting signal by the solar cells (ZEL\_sc1, ZEL\_sc2) could achieve sufficient accuracy in binary classification.
This is due to the fact that the harvesting signal changed because of shadows created during human activities.
These comparisons show that ZEL can recognize places and activities with an accuracy close to that of conventional sensors that consume a lot of power.

%%%%%%%%%%%%%%%%%%%%%%%%%%%%%%%%%%%%%%%%%%%%%%%%
\begin{figure}[bt]
    \centering
    \includegraphics[width=0.99\hsize]{img/accuracy.png}
    \caption{Accuracy of 8-place and static/dynamic activity recognition for each method}
    \label{fig:10fold_accuracy}
\end{figure}
%%%%%%%%%%%%%%%%%%%%%%%%%%%%%%%%%%%%%%%%%%%%%%%%

\subsection{Evaluation of versatility}
We compared the result from Person-Dependent model with that from Person-Independent model obtained by Leave-One-User-Out(LOUO) CV to evaluate the versatility of ZEL to participants.
The confusion matrix acquired by the Person-Dependent model and the Person-Independent model is shown in Fig.~\ref{fig:cv}.
While both models show high accuracy in classifying activities, the Person-Independent model has significantly lower accuracy in place recognition.
In particular, the recognition accuracy for restrooms was about 10\%.
This may be due to the fact that the number of the light source in the restroom is small and the amount of harvested energy varies greatly depending on the position of the user.
Other factors such as the participant's height and posture are also considered.

%%%%%%%%%%%%%%%%%%%%%%%%%%%%%%%%%%%%%%%%%%%%%%%%
%%%%%%%%%%%%%%%%%%%%%%%%%%%%%%%%%%%%%%%%%%%%%%%%
\begin{figure}[bt]
    \begin{tabular}{cc}
      \begin{minipage}[t]{0.45\hsize}
        \centering
        \includegraphics[width=1.0\hsize]{img/cm/PD_ehass_eh12pie_small_majority20_place_result_cm_LightGBM.png}
        \subcaption{8-place(Person-Dependent)}
        \label{fill}
      \end{minipage} &
      \begin{minipage}[t]{0.45\hsize}
        \centering
        \includegraphics[width=1.0\hsize]{img/cm/PD_ehass_eh12pie_small_majority20_action_result_cm_SVM.png}
        \subcaption{static/dynamic(Person-Dependent)}
        \label{transform}
      \end{minipage} \\
      
      \begin{minipage}[t]{0.45\hsize}
        \centering
        \includegraphics[width=1.0\hsize]{img/cm/PI_ehass_eh12pie_small_majority20_place_result_cm_LR.png}
        \subcaption{8-place(Person-Independent)}
        \label{LOUO_place}
      \end{minipage} &
      \begin{minipage}[t]{0.45\hsize}
        \centering
        \includegraphics[width=1.0\hsize]{img/cm/PI_ehass_eh12pie_small_majority20_action_result_cm_LightGBM.png}
        \subcaption{static/dynamic(Person-Independent)}
        \label{LOUO_action}
      \end{minipage}
    \end{tabular}
    \caption{Confusion matrix for each recognition obtained by Person-Dependent(10-fold CV) model and Person-Independent(LOUO CV) model}
    \label{fig:cv}
\end{figure}
%%%%%%%%%%%%%%%%%%%%%%%%%%%%%%%%%%%%%%%%%%%%%%%%

\section{Investigation of performance limits and the zero energy rate} \label{sec:investigation}
In this section, we describe the performance limits and the zero energy rate of ZEL in practical use.

\subsection{Performance limits}
To investigate the performance limits of ZEL, we performed 14-place recognition and 5-activity recognition with subdivided labels.
The result of each cross-validation is summarized in TABLE~\ref{tab:cv_for_each_target}.
The accuracy of the 14-place recognition in the PI model was reduced by about 32\% compared to the 8-place.
Fig.~\ref{fig:14places_cm} shows the confusion matrix of the 14-place recognition acquired by Person-Independent model.
The significantly lower accuracy between labs and halls indicates that ZEL is not suitable for the recognition of fine details of places within the same room or in a similar light environment, as the principle indicates.
As for the 5-activity recognition, there were many misrecognition for walking and up and downstairs.
It indicates that it is necessary to reconsider the position of the piezoelectric element for more detailed activity recognition because because there are many studies such as \cite{ma2018sehs} that achieve a fine level of action recognition using piezoelectricity.

%%%%%%%%%%%%%%%%%%%%%%%%%%%%%%%%%%%%%%%%%%%%%%%%
\begin{table}[bt]
    \centering
    \caption{Results of recognition accuracy for each classification target and each model(Person-Dependent and Person-Independent)}
    \label{tab:cv_for_each_target}
    \begin{tabular}{lll} \hline
        Target & Peson-Dependent & Person-Independent\\ \hline \hline
        8-place  & 0.873(LightGBM) & 0.728(LR) \\
        static/dynamic & 0.932(LightGBM) & 0.943(LightGBM) \\
        14-place & 0.832(LightGBM) & 0.497(LR) \\
        5-activity & 0.821(LightGBM) & 0.728(SVM) \\
        \hline
    \end{tabular}
\end{table}
%%%%%%%%%%%%%%%%%%%%%%%%%%%%%%%%%%%%%%%%%%%%%%%%
%%%%%%%%%%%%%%%%%%%%%%%%%%%%%%%%%%%%%%%%%%%%%%%%
\begin{figure}[bt]
     \centering
     \includegraphics[width=0.9\hsize]{img/cm/PI_cm.png}
     \caption{14-place recognition confusion matrix in Person-Independent model}
     \label{fig:14places_cm}
\end{figure}
%%%%%%%%%%%%%%%%%%%%%%%%%%%%%%%%%%%%%%%%%%%%%%%%

\subsection{Zero energy rate}
\begin{comment}
図で示すように提案システムは暗い場所では十分な発電量の確保ができないため、動作できずライフログ情報の記録ができない。そこで、実環境で提案システムを使用した場合にどれくらいの割合でシステムが動作をしているかの調査を行った。
調査のためにパワースイッチング用のボルテージデティクター(図のTh1)をデータロガーに接続した。
ボルテージディテクターの出力をロガー内のSDカードに記録することでシステムの動作割合がわかる。
なぜならharvesting energyが動作保証電圧を下回った場合ボルテージディテクターの出力がLowとなるためである。
このロガー付きの名札センサを着用して、午前10時から午後19時まで行動や場所の制限なく作業を行う実験を行った。実験は計4日間行い、計36時間分のデータを取得した。
実験結果を表に示す。
4日間を平均システムが動作保証外となった割合は1\%未満にとどまった。このことから提案システムは実適用した場合でも十分な適用範囲を持つことが可能であると示された。
\end{comment}
As shown in the Fig.~\ref{fig:device_flow}, ZEL cannot operate and record lifelog information in dark places because it cannot generate enough power.
Therefore, we investigated how much of the system activate i.e. zero energy rate when ZEL is used in a real environment.
For investigation, we connect the output of load switch (VOUT of TPS22860DBVR in Fig.~\ref{fig:schematic}) to the data logger(Adafruit Feather M0 Adalogger).
The output of this pin goes HIGH when the system is inoperable, i.e., not zero energy, and by recording this output to the SD card in the data logger, the zero energy rate can be determined.

We conducted an experiment in which one participant wore ZEL with data logger to work from 10:00 to 19:00 without any restrictions on their places or activities.
The experiment was conducted for a total of four days, and a total of 36 hours of data was acquired.
% To show that the proposed method minimizes the battery usage, we compare it with the zero energy rate of a version of the circuit that embeds only one voltage detector that switches operation mode of microcontroller and power source at same time.
To show that the proposed method minimizes the battery usage, we compare it with the zero energy rate of a version of the circuit that embeds only one voltage detector that not minimizing battery usage.
The results are shown in TABLE.~\ref{tab:activation_ratio}.
The proposed method improved the zero energy rate by n\% and the average not zero energy rate was less than 1\%.
From this result, ZEL can be said to have achieved net-zero energy.

%%%%%%%%%%%%%%%%%%%%%%%%%%%%%%%%%%%%%%%%%%%%%%%%
\begin{comment}
\begin{table}[bt]
\centering
\caption{Zero energy rate of the system in practical use}
\begin{tabular}{crrrr}
\hline
\multirow{2}{*}{Date} & \multicolumn{2}{c}{Activation}                              & \multicolumn{2}{c}{Deactivation}                            \\
                      & \multicolumn{1}{l}{time{[}s{]}} & \multicolumn{1}{l}{rate} & \multicolumn{1}{l}{time{[}s{]}} & \multicolumn{1}{l}{rate} \\ \hline \hline
Aug 28th              & 32021                           & 98.88\%                   & 380                             & 1.17\%                    \\
Aug 30th              & 32354                           & 99.85\%                   & 47                              & 0.15\%                    \\
Sep 1st               & 32327                           & 99.77\%                   & 74                              & 0.23\%                    \\
Sep 6th               & 32342                           & 99.82\%                   & 59                              & 0.18\%                    \\ \hline \hline
avg/total             & 32261                           & 99.57\%                   & 140                             & 0.43\%                    \\ \hline
\end{tabular}
\label{tab:activation_ratio}
\end{table}
\end{comment}

\begin{table}[bt]
\centering
\caption{Zero energy rate of the system in practical use(average of four days)}
\begin{tabular}{crrrr}
\hline
\multirow{2}{*}{\# of detectors} & \multicolumn{2}{c}{Zero energy} & \multicolumn{2}{c}{Not zero energy} \\
                      & \multicolumn{1}{l}{time{[}s{]}} & \multicolumn{1}{l}{rate} & \multicolumn{1}{l}{time{[}s{]}} & \multicolumn{1}{l}{rate} \\ \hline \hline
One & (temp) & (temp) & (temp) & (temp) \\
Two & 32261 & 99.57\% & 140 & 0.43\% \\ \hline
\end{tabular}
\label{tab:activation_ratio}
\end{table}
%%%%%%%%%%%%%%%%%%%%%%%%%%%%%%%%%%%%%%%%%%%%%%%%

\section{Conclusion and future work} \label{sec:conclusion}
\begin{comment}
本論文ではオフィスワーカのためのネットゼロエナジーライフロギングシステムであるZELを提案した
ZELはヘテロジニアスなハーベスタを採用し、それらの収穫エネルギーを使いライフログに必要ないつ、どこで、何の行動をしていたかの情報をゼロエナジーで記録する
さらにデュアルパワースイッチング回路とボルテージディタクターによるシステム状態スケジューリングによって電池使用を最小化する
ライフログ構成ブロックでは記録された時系列データから機械学習を用いて場所認識及び行動認識を行いユーザのライフログを構成する
我々はZELを名札型のウェアラブルデバイスに実装し大規模なデータ収集実験を11人の被験者を対象に行った
評価の結果Person-Dependentモデルで87.2\%の場所認識精度を93.1\%の行動認識精度を達成した
また電池使用最少化機構によって実環境で約99\%のゼロエナジーを達成し、ZELがネットゼロエナジーシステムであることを示した。

% Future work

ZELはユーザフレンドリーなデバイスのためにいくつかの課題がある
まずデバイスの大きさと重さに関する問題である
本研究で実装した名札型のデバイスは123mmの正方形とウェアラブルにしては大きすぎる
また首かけ式のため192グラムという重さはネックペインの原因となる
しかし、回路は図2に示すように十分小型であるため小型で軽量なハーベスタ例えばリコーのOPVを使うことでサイズと重さは抑えることができる
次に汎用性に関する問題である
3章で示したようにPerson-Independentモデルでは認識精度が著しく低下する
これはユーザの姿勢によってできる影や光源に対する方向などが原因と考えられるためシステムの実装場所を変えることで克服することができると考える
例えばシステムを帽子型のデバイスに実装すればユーザの姿勢などに影響されないため認識精度の向上が見込める
最後にデータの取得に関する問題がある
ユーザがライフログを確認したいと思った時デバイスにケーブルを接続しデータを抽出する必要がある
データ量が多い場合吸出しに時間がかかるため非常に煩わしい
しかしSolARで示されているend-to-endなコンテキスト認識に必要な電力である33mW以上を現在LEDで消費しているためZELもライフログ情報送信までを含めたend-to-endなコンテキスト認識を実装できる能力を秘めてい
今後の展望としてこれら課題を解決しユーザフレンドリーなデバイスを目指す
\end{comment}
In this paper, we propose ZEL, net-zero energy lifelogging system for office workers.
ZEL employs heterogeneous harvesters and uses their harvesting energy to zero energy record the lifelogging information: when, where, what activity.
In addition, dual power switching circuit and voltage detectors are used for system state scheduling to minimize battery usage.
In the lifelog generation block, the recorded time-series data is used to generate a user's lifelog by recognizing places and activities using machine learning.
We implemented ZEL on a name-tag-shaped wearable device and conducted a large-scale data collection experiment with 11 participants.
As a result, we achieved 87.2\% accuracy of 8-place recognition and 93.1\% accuracy of static/dynamic recognition in the Person-Dependent model.
In addition, we demonstrated that ZEL is a net-zero energy system by achieving about 99\% zero energy rate in the real environment through the battery minimization mechanism.

% future work
ZEL has a few challenges for a user-friendly device.
The first issue is the size and weight of the device.
Our name-tag-shaped device is 123mm square, which is too large for a wearable.
In addition, the weight of the device is 192 grams, which is too heavy for a neck wearable and causes neck pain.
However, the circuit is small enough as shown in Fig.~\ref{fig:ZEL}, and the size and weight can be reduced by using a small and lightweight harvester\cite{doi:10.1021/acsami.9b00018}.
Second, there is the issue of versatility.
As shown in Section~\ref{sec:investigation}, the recognition accuracy of the Person-Independent model is significantly reduced.
This could be caused by the shadows created by the user's posture or by the direction of the light source, which can be overcome by changing the implementation placement of the system.
For example, if the system is implemented in a hat-shaped device, the recognition accuracy can be improved because it is not affected by the user's posture or direction.
% Finally, there is the issue of data acquisition.
% When users want to check their lifelog, they need to connect a cable to the device and extract the data.
% If the data size is large, it takes time to extract the data, which is very annoying.
% However, since LEDs currently consume more than 33mW, which is the power required for the end-to-end context recognition shown in SolAR\cite{Sandhu2021SolAREP}, ZEL has the capability to implement end-to-end context recognition, including the transmission of lifelogging information.
Finally, there is the issue of data acquisition.
When users want to check their lifelog, they need to connect a cable to the device and extract the data.
In the future, we will simplify this data acquisition process by incorporating a program into the onboard device that performs context recognition processing and wirelessly sends the recognition results to a smartphone.
The current ZEL prototype is designed to consume more than 33mW which is the power required for the end-to-end context recognition shown in SolAR\cite{Sandhu2021SolAREP} by intentionally lighting up the LEDs. 
Therefore, the ZEL prototype has the surplus power to implement end-to-end context awareness, including the transmission of lifelog information.
The implementation of programs that can be embedded in devices will be addressed in future work. 
%In the future, we aim to solve these issues and create user-friendly devices.

\section*{Acknowledgment}
% 本研究は,JSPS科研費JP18H03233の支援を受けて実施したものである.
This work was partially supported by JSPS KAKENHI Grant Number JP18H03233 and JP19H05665.

\bibliographystyle{IEEEtran}

\bibliography{bib}

\end{document}
