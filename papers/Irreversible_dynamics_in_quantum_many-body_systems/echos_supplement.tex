\documentclass[%
 reprint,
%superscriptaddress,
%groupedaddress,
%unsortedaddress,
%runinaddress,
%frontmatterverbose, 
%preprint,
%showpacs,preprintnumbers,
%nofootinbib,
%nobibnotes,
%bibnotes,
 amsmath,amssymb,
 prl,
 %aps,
%pra,
%prb,
%rmp,
%prstab,
%prstper,
%floatfix,
]{revtex4-1}

%\usepackage{amsmath}
%\usepackage{mathtools}
%\usepackage{amssymb}
\usepackage{dsfont}
\usepackage{graphicx}
\usepackage{hyperref}
\usepackage{color}

\usepackage{mathrsfs}

\usepackage{braket}

\usepackage{overpic}
\usepackage{tikz}
\usetikzlibrary{decorations.pathreplacing}

\newcommand\emc{E=mc^{2}}
\newcommand{\im}{\mathrm{i}}
\newcommand{\idop}{\mathds{1}}
\newcommand{\argmin}[1]{\underset{#1}{\text{argmin}}}
\newcommand{\argmax}[1]{\underset{#1}{\text{argmax}}}
\newcommand{\drangle}{\rangle\rangle}
\newcommand{\dlangle}{\langle\langle}

\definecolor{mygreen}{rgb}{0,0.5,0}
\definecolor{myblue}{rgb}{0,0,0.75}
\definecolor{mymagenta}{cmyk}{0,1,0,0.12}
\newcommand{\btext}[1]{{\color{myblue} #1}}
\newcommand{\gtext}[1]{{\color{mygreen} #1 }}
\newcommand{\mtext}[1]{{\color{mymagenta} #1}}
\newcommand{\comment}[1]{\btext{[#1]}}
\usepackage{umoline}\newcommand{\removed}[1]{\Midline{#1}}
\newcommand{\newtext}[1]{\gtext{#1}}

\newcommand{\ch}{\text{ch}}

\newcommand{\mh}[1]{\btext{#1}}
\newcommand{\ms}[1]{\textcolor{red}{[MS: #1]}}


\begin{document}
\title{Supplemental material to\\\emph{Irreversible dynamics in quantum many-body systems}}


\author{Markus Schmitt}
%\email{mschmitt@pks.mpg.de}
\author{Stefan Kehrein}
\affiliation{%
 Institute for Theoretical Physics, 
	Georg-August-Universit\"at G\"ottingen, 
	Friedrich-Hund-Platz 1, 37077 G\"ottingen, Germany
}
\date{\today}

\maketitle

\section{Extraction of echo peak decay from numerical data}
The analysis of the echo peak decay as shown in Fig.\ 3 of the main text requires particular caution.
In the following, we describe the details of our analysis. 

Each point for $E_O(\tau)$ shown in Fig.\ 3 of the main text is obtained by conducting the following two steps:
\begin{enumerate}
\item Compute the full time evolution with the echo protocol for the given waiting time $\tau$. Since we use the Lanczos algorithm, which results in an iterative propagation of the wave function, we have to compute the full dynamics although we are only interested in the behavior at the echo time. The result is a curve showing an echo peak as depicted in Fig.\ 2 of the main text. Note that for a data point $E_O(\tau)$ time evolution up to $t=2\tau$ is required.
\item From our previous work, Ref.\ [26] of the main text, we know that depending on the choice of the echo protocol the echo peak does not necessarily occur at $t_E=2\tau$. Hence, we extract the echo peak height from the data as denoted in Eq.\ (7) of the main text: we search for the maximal deviation from the saturation value during the backwards propagation. The result gives the echo time $t_E$ and the corresponding echo peak height $E_O(\tau)$.
\end{enumerate}

In the numerical simulation of finite systems we remove the parallel component of the perturbed state, which will vanish in the thermodynamic limit, by hand, meaning that in practice the perturbation operator $\hat P_{\delta t}$ introduced in the main text involves a $\tau$-dependent projection:
\begin{align}
	\hat P_{\delta t}^\tau=\frac{\big(1-\ket{\psi(\tau)}\bra{\psi(\tau)}\big)e^{-\im\hat H_p\delta t}}{\sqrt{1-|\braket{\psi(\tau)|e^{-\im\hat H_p\delta t}|\psi(\tau)}|^2}}
\end{align}
Here $\ket{\psi(\tau)}=e^{-\im\hat H\tau}\ket{\psi_0}$ is the forward-evolved state.

Although the projection onto the orthogonal component of the perturbed state is included in the echo dynamics, the computed echo peak height does not necessarily decay to zero, because at finite size there are additional time-independent contributions in Eq.\ (6) of the main text leading to a non-vanishing long time limit. For the extraction of the decay law of the time-dependent part it is essential to subtract all of the time-independent contributions. In order to estimate the stationary value we compute an average over peak heights at late times. In practice, we computed echos up to time $J\tau=80$ for $\hat H_\text{loc}$ and $J\tau=60$ for $\hat H_\text{fc}$ and averaged over $J\tau\in[60,80]$ or $J\tau\in[40,60]$, respectively, to obtain an estimate of the stationary value that was then subtracted from the result. 

\section{Finite size analysis}
To demonstrate the existence of persisting echos in finite systems and their vanishing in the thermodynamic limit we consider the difference of the echo at $t=2\tau$ from the perfect echo,
\begin{align}
	\Delta M(\tau)=\braket{\psi_0|\hat M|\psi_0}
	-\braket{\psi_0|\hat U_E^\delta t(\tau)^\dagger\hat M\hat U_E^\delta t(\tau)|\psi_0}\ ,
\end{align}
where $\hat U_E^\delta t(\tau)=e^{\im\hat H_p\tau}\hat P_{\delta t}e^{-\im\hat H_p\tau}$. In this case do not project out the parallel component after the perturbation.

Fig. \ref{fig:finite_size} shows $\Delta M(\tau)$ obtained for $\hat H_\text{loc}$ and averaged over 40 realizations of the perturbation Hamiltonian $\hat H_p$. The saturation of the difference to the perfect echo at late times corresponds to the persistent echo. Clearly, however, the saturation value increases as the system size is increased, indicating that the persistent echo vanishes in the thermodynamic limit $N=\infty$.
\begin{figure}[h!]
\includegraphics{finite_size.pdf}
\caption{Divergence from the perfect echo for different system sizes and $\delta t=0.01/J$.}
\label{fig:finite_size}
\end{figure}
\end{document}

