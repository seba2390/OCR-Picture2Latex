% !TEX TS-program = pdflatexmk  
\documentclass[draftcls,onecolumn]{IEEEtran}



\usepackage{amsmath, amsfonts,amssymb, latexsym,epsfig,color,rotating,paralist,times,float,subfigure,algorithm,verbatim,amsthm}
% Natbib setup for author-year style

%!TEX root = ../book.tex
%%%%
%% section and chapter symbol
\newcommand{\secn}{\S}
\newcommand{\chp}{\S}

%%%% 
%% my environment for assumptions%%%
%%%%

\newenvironment{myassumptions}{%
   \begin{description}[style=multiline, leftmargin = 18pt, align=left]%
}{%
   \end{description}%
}
%%%%%
%%%%

%% ENUMERATED LABEL FOR ASSUMPTIONS
\makeatletter
\def\nl#1#2{\begingroup
    #2%
    \def\@currentlabel{#2}%
    \phantomsection\label{#1}\endgroup
}
\makeatother
%%%%%%
%%%%
%%

% to box multiple equations

\newenvironment{boxalign}[1]% 
{\ensuremath{
\begin{empheq}[box=\fbox]{align}
{#1}
\end{empheq}
}}


\newenvironment{advanced}{\small\fontfamily{ppl}\selectfont}{}
% ----- theorem-like environments                                                                                                   
\newtheorem{theorem}            {Theorem}[section]
\newtheorem{conjecture}            {Conjecture}[section]
\newtheorem{corollary}          [theorem]{Corollary}
\newtheorem{proposition}        [theorem]{Proposition}
\newtheorem{definition}         [theorem]{Definition}
\newtheorem{example}            [theorem]{Example}
\newtheorem{lemma}              [theorem]{Lemma}
\newtheorem{ass}                [theorem]{Assumption}
\newtheorem{remark}             [theorem]{Remark}
\newtheorem{result}             [theorem]{Result}



\newcommand{\normal}{\mathbfit{N}}  % this is the distribution
\newcommand{\uniform}{u}  % this is the rv

\newenvironment{boxresult}[1]% 
{
\begin{mdframed}
\par\noindent\textbf{#1:}\begin{rmfamily}\noindent}% 
{\end{rmfamily}
\end{mdframed}
} 

\newcommand{\pwe}{Complements and Sources}

%\newcommand{\Ep}{\E_\pdf}
\newcommand{\Ept}{\E_{\beliefm_\infty}}

\newcommand{\I}{\Pi}
\newcommand{\avg}{\phi}
\newcommand{\unavg}{L}
\newcommand{\dob}{\rho}  %dobrushin
\newcommand{\dist}{D}
\newcommand{\bias}{\operatorname{Bias}}
\newcommand{\var}{\operatorname{Var}}
\newcommand{\msd}{\operatorname{MSD}}

% prob, expectation

\newcommand{\rv}{x}
\newcommand{\rvY}{y}
\newcommand{\cdfb}{\bar{\cdf}}
\newcommand{\outcome}{\zeta}
\newcommand{\pdfX}{p}
\newcommand{\pdfY}{q}

\newcommand{\pdf}{p}
\newcommand{\cdf}{F}
\newcommand{\prob}{\mathbb{P}}
\newcommand{\E}                 {\Bbb{E}}
\newcommand{\bE}{\bar{\E}}
\renewcommand{\P}                 {\Bbb{P}}
\newcommand{\cov}{\operatorname{cov}}
\newcommand{\bre}{\mathbf{L}}
\newcommand{\pref}{q}  % reference prob method  density under \bar{P}

% state space
\newcommand{\grid}{\state}
\newcommand{\lowdim}{r}
\newcommand{\inp}{u}
\newcommand{\bstate}{\bar{\state}}
\newcommand{\onoisevar}{\sigma^2_\onoise}

\newcommand{\borelset}{S}
\newcommand{\tstate}{\tilde{\state}}

\newcommand{\obs}{y}
%\newcommand{\Obs}{\mathbf{y}}
\newcommand{\Obs}{\obs}

\newcommand{\snoise}{w}
\newcommand{\onoise}{v}
\newcommand{\statem}{A}
%\newcommand{\statem}{F}

\newcommand{\statemg}{\phi}
\newcommand{\obsmg}{\psi}
\newcommand{\snoisem}{\Gamma}

\newcommand{\obsm}{C}
%\newcommand{\obsm}{H}

%\newcommand{\obsmnew}{\psi}
\newcommand{\obsmnew}{H}
\newcommand{\level}{g}

%\newcommand{\onoisem}{D}
\newcommand{\onoisem}{D}
\newcommand{\inpm}{f}
\newcommand{\oinpm}{g}
\newcommand{\snoisecov}{Q}
\newcommand{\onoisecov}{R}
\newcommand{\onoisecovnew}{R}
\newcommand{\ar}{a} % AR coefficients
\newcommand{\arp}{s} % ar process

\newcommand{\state}{x}
\newcommand{\statespace}{\mathcal{X}}
\newcommand{\obspace}{\mathcal{Y}}
\newcommand{\statedim}{X}
\newcommand{\obsdim}{{Y}}

\newcommand{\mc}{r}  % markov chain of JMLS
\newcommand{\finals}{x} % reciprocal markov process

\newcommand{\fun}{\phi}
\newcommand{\funbar}{\psi}

\newcommand{\obsdummy}{y}

\newcommand{\identity}{I}



%% stoch convergence
\newcommand{\law}{\stackrel{\mathcal{L}}{=}}

% VARIATIONAL DISTANCE
\newcommand{\dvar}[2]{\|{#1}-{#2}\|_{\text{\tiny{TV}}}}
\newcommand{\dvarsq}[2]{\|{#1}-{#2}\|^2_{\text{\tiny{TV}}}}
\newcommand{\dvarn}{\| \cdot \|_{\text{\tiny{TV}}}}

% HMM parameters
\newcommand{\oprob}{B}
\newcommand{\tp}{P}
\newcommand{\utp}{\bar{\tp}}
\newcommand{\ltp}{\underline{\tp}}


\newcommand{\btp}{\bar{\tp}}
\newcommand{\jump}{J}
\newcommand{\duration}{D}
\newcommand{\sensm}{S^\model}

\newcommand{\eigenvec}{\nu}
%%

\newcommand{\finaltime}{N}
\newcommand{\btime}{\finaltime}


% models
\newcommand{\model}{\theta}
\newcommand{\modelpsi}{\psi}
\newcommand{\truemodel}{\model^o}
\newcommand{\modelem}{\model_\text{EM}}
\newcommand{\modelnr}{\model_\text{NR}}

\newcommand{\Model}{\Theta}
\newcommand{\lik}{L_\finaltime} % likelihood
\newcommand{\logl}{\mathcal{L}_\finaltime}  % log likeilhood
\newcommand{\aux}{\mathcal{Q}}  % EM algorithm Q aux likelihood

\newcommand{\beliefm}{\belief_\model}
\newcommand{\Mbelief}{\Pi_\infty}


\newcommand{\ubeliefm}{\ubelief^\model}
\newcommand{\likc}{\pdf(\obs_{1:\finaltime}| \model)}
\newcommand{\pdfm}{\pdf^\model}
\newcommand{\mle}{\model^*}
\newcommand{\ml}[1]{{#1}^*}
\newcommand{\oprobm}{\oprob^\model}
\newcommand{\tpm}{{\tp_\model}}
\newcommand{\nablam}{\nabla_\model}
\newcommand{\ardim}{M} % dimension of ar model
\newcommand{\sigmasnoise}{\sigma}

% radar
\newcommand{\range}{d}

% belief space
\newcommand{\belief}{\pi}
\newcommand{\beliefzero}{\belief}
\newcommand{\bbelief}{\bar{\pi}}
\newcommand{\ubelief}{{q}}
\newcommand{\upbelief}{\bar{\pi}}
\newcommand{\lbelief}{\underline{\pi}}

\newcommand{\lmean}{\underline{\state}}
\newcommand{\mean}{{\hat{\state}}}
\newcommand{\umean}{\bar{\state}}


\newcommand{\mat}{M}
\newcommand{\diff}{L^\epsilon}

%%%%%%%%%%

\newcommand{\Belief}{\Pi(\statedim)}
\newcommand{\back}{\beta}
\newcommand{\bay}[2]{\mathcal{B}[{#1},{#2}]}
\newcommand{\diagp}{P}


%social learning
\newcommand{\tbelief}{\pi^0}
\newcommand{\history}{\mathcal{H}}
\newcommand{\full}{\mathcal{F}}

\newcommand{\sigs}{\sigma}
\newcommand{\Bs}{R^\pi} 
\newcommand{\Bsl}{R^n}
\newcommand{\priv}{\eta}
\newcommand{\etaregion}{\kappa}

   \newcommand{\ca}{\cost_\action}
   \newcommand{\socialcost}{l}
   \newcommand{\ta}{\tilde{\action}}
   \newcommand{\sA}{\mathcal{A}}
   
   % incest 
   \newcommand{\quantized}{Q}
   
% Kalman filter variables
\newcommand{\kalmancov}{\Sigma}
\newcommand{\kalmangain}{K}
\newcommand{\lqgain}{L}
\newcommand{\kgain}{\bar{K}}  % this satisfies A L = kalmangain
\newcommand{\tr}{\normalfont{\text{trace}}}
\newcommand{\Sig}{S}   %  used for S_k in Kalman filter
\newcommand{\prederr}{\nu}
\newcommand{\ksgain}{G}

% delay variables
\newcommand{\delay}{\Delta}
\newcommand{\Delay}{\Delta}
\newcommand{\delaydim}{L}
\newcommand{\tpd}{\tp^{(\delay)}}
\newcommand{\dobs}{z}



\newcommand{\delaysetim}{\{1,2,\ldots,L-1\}}
\newcommand{\Delayset}{\{0 \text{ (announce change)} ,\D_1,\D_2,\ldots,\D_L\}}
\newcommand{\Delayseti}{\{0 \text{ (announce change)} ,1,2,\ldots, L\}}


% particle filter
\newcommand{\imp }{\pi}
\newcommand{\weight}{\omega}
\newcommand{\nw}{\tilde{\weight}}  % particle filter normalized weight
\newcommand{\ole}{\stackrel{\text{defn}}{=}}


% additive functional
\newcommand{\af}{S}  % ADDITIVE functional
\newcommand{\mf}{\mu}   % measure valued pdf of functional


\newcommand{\est}{\kappa}
\newcommand{\Est}{\boldsymbol{\kappa}}

\newcommand{\gtp}{\underset{\text{\tiny TP2}}{\geq}}
\newcommand{\gr}{\geq_r}
\newcommand{\lr}{\leq_r}
\newcommand{\gs}{\geq_s}
\newcommand{\ls}{\leq_s}






\newcommand{\filterd}{\sigma}
\newcommand{\filtern}{\alpha}
\newcommand{\filternum}{\gamma}
\newcommand{\filter}{T}
\newcommand{\bayes}{\mathcal{B}}

\renewcommand{\vec}{\operatorname{vec}}

\newcommand{\argmin}{\operatornamewithlimits{argmin}}
\newcommand{\argmax}{\operatornamewithlimits{argmax}}

\newcommand{\reals}{{\rm I\hspace{-.07cm}R}}
\newcommand{\R}{{\rm I\hspace{-.07cm}R}}
\newcommand{\N}{{\rm I\hspace{-.07cm}N}}
\newcommand{\continuous}{{\rm I\hspace{-.16cm}C}}

\newcommand{\F}{\mathcal{F}}   % sigma algebra

\newcommand{\beq}{\begin{equation}}
\newcommand{\eeq}{\end{equation}}
\newcommand{\nn}{\nonumber}

\renewcommand{\(}		{\left(}
\renewcommand{\)}		{\right)}


\newcommand{\Y}{\mathbf{Y}}

\renewcommand{\th}{\theta}
\newcommand{\Th}{\Theta}

\newcommand{\p}{\prime}

\newcommand{\one}{\mathbf{1}}
\newcommand{\ones}{\mathbf{1}}
\newcommand{\zero}{\mathbf{0}}


% tracking variables
\newcommand{\pos}{p}
\newcommand{\samt}{\Delta}
\newcommand{\jerk}{q}
\newcommand{\assoc}{a}

\newcommand{\f}{f} % false measurement

%%%%% continuous time


\newcommand{\bs}{\bar{\sigma}}
\newcommand{\osig}{\mathcal{Y}}
\renewcommand{\div} {{\operatorname{div}}}
\newcommand{\ftime}{T}
\newcommand{\gen}{Q}
\newcommand{\diag}{\textnormal{diag}}

\newcommand{\bop}{\mathcal L}
\newcommand{\fop}{\bop^*}

\newcommand{\pnoise}{\nu}
\newcommand{\rate}{\lambda}
\newcommand{\pois}{N}
\newcommand{\markpdf}{\pdf}
\newcommand{\marked}{z}
\newcommand{\event}{t}
\newcommand{\markspace}{\mathcal M}
\newcommand{\sigf}{\mathcal{F}}
\newcommand{\testf}{\phi}
\newcommand{\mart}{M}

\newcommand{\bq}{\bar{\ubelief}}
\newcommand{\be}{\bar{\epsilon}}
\newcommand{\bback}{\bar{\back}}

\newcommand{\forwardzakai}{q}
\def\la{\langle}
\def\ra{\rangle}


%structural filter
\newcommand{\gRm}{\succeq_M}
\newcommand{\lRm}{\preceq_M}
\newcommand{\uA}{\underline{A}}

\renewcommand{\i}{\mathbf{i}}
\renewcommand{\j}{\mathbf{j}}
\newcommand{\gtptwo}{\underset{\text{\tiny TP2}}{\geq}}
\newcommand{\ltptwo}{\underset{\text{\tiny TP2}}{\leq}}

\newcommand{\nm}{L}
\newcommand{\lR}{\preceq}
\newcommand{\gR}{\succeq}
\newcommand{\cons}{{\text{\bf Cons}}}
\newcommand{\consb}{\overline{\text{\bf Cons}}}
\newcommand{\conv}{{\text{conv}}}

\newcommand{\levels}{g}

\newcommand{\map}{\hat{\state}^{\text{MAP}}}
\newcommand{\lmap}{\underline{\state}^{\text{MAP}}}
\newcommand{\umap}{\bar{\state}^{\text{MAP}}}


% STOCHASTIC CONTROL
\newcommand{\Minimize}{\operatorname{Minimize}}
\newcommand{\marginal}{p}
\newcommand{\bmodel}{\bar{\model}}
\newcommand{\hQ}{\hat{Q}}

\newcommand{\ltwo}{\log_2}

%\newcommand {\brMet}{\ensuremath{\rho}} % the metric on the space \brSpc 
\newcommand{\sgain}{M}
\newcommand{\again}{N}
\newcommand{\fgain}{L}

\newcommand{\terminalcosta}{\cost_{\finaltime+1}}
\newcommand{\actions}{\action^\state}
\newcommand{\actionm}{\action^\obs}

\newcommand{\dpop}{L}
\newcommand{\cost}{c}
\newcommand{\covcost}{C}
\newcommand{\reward}{r}
\newcommand{\terminalcost}{\cost_\finaltime}
\newcommand{\Cost}{C}
\newcommand{\nlcost}{d}
\newcommand{\nlCost}{D}

\newcommand{\yi}{y^{(1)}}
\newcommand{\yii}{y^{(2)}}
\newcommand{\valueft}{\tilde{\valuef}}

\newcommand{\bmc}{\bar{m}}

\newcommand{\bQ}{\bar{Q}}
\newcommand{\bvalueb}{\bar{\valueb}}

\newcommand{\action}{u}
\newcommand{\baction}{\bar{\action}}
\newcommand{\actionspace}{\,\mathcal{U}}
\newcommand{\actiondim}{U}
\newcommand{\laspace}{\mathcal{A}}
%\newcommand{\cost}{c}
\newcommand{\totalcost}{J}
\newcommand{\exptotalcost}{J_{\bpolicy}}
\newcommand{\opttotalcost}{J_{\bpolicy^*}}
\newcommand{\discount}{\rho}

\newcommand{\region}{\mathcal{R}}
\def \con {{\beta}}
\def \rcon {{\gamma}}
\def \Con {{B}}
\def\param{{\alpha}}
\def \Z {{\cal Z}}
%%%% \def \a {{\alpha}}
\def \a {{\psi}}
\def\Param{{\boldsymbol{\alpha}}}
\def \Epi {{\E}_{\pi(\param)}}
 \def \G{{B}}
 \newcommand{\statpi}{\pi}
 \newcommand{\Ep}{\E_{\policy}}

\def\Policy{{\boldsymbol{\policy}}}
\newcommand{\policy}{\mu}
\newcommand{\optpolicyv}{\bpolicy^*}
\newcommand{\optpolicy}{\policy^*}
\newcommand{\bpolicy}{{\boldsymbol{\mu}}}
\newcommand{\optimaltotalcost}{J_{\optpolicy}}
\newcommand{\valuef}{V}
\newcommand{\valuefb}{\bar{\valuef}}
\newcommand{\valueb}{J}
\newcommand{\info}{\mathcal{I}}
\newcommand{\uV}{\underline{V}}

\newcommand{\bvalvec}{\bar{\gamma}}
\newcommand{\valvec}{\gamma}
\newcommand{\Valvec}{\Gamma}

\newcommand{\overlook}{\beta}
\newcommand{\blockprob}{q}
\newcommand{\augss}{\bar{\statespace}}
\newcommand{\terminal}{T}
\newcommand{\schRwd}{h}
\newcommand{\actionCost}{\cost}
\newcommand{\schHst}{\info}
\newcommand{\policySpc}{\mathcal{U}}
\newcommand{\schRwdFn}{J}

% stopping set
\newcommand{\stopset}{\mathcal{S}}

% POMDP (pom) quantities

\newcommand{\Jh}{J^{\text{h}}}
\newcommand{\Jo}{\bar{J}}
\newcommand{\thstate}{\state}
\newcommand{\thstatedim}{\statedim}

\newcommand{\pie}{\pi^{\epsilon_1,\epsilon_2,\ldots,\epsilon_{\statedim-1}}}
\newcommand{\pieone}{\pi^{\epsilon_1,\ldots\epsilon_j,\ldots, \epsilon_{\statedim-1}}}
\newcommand{\pietwo}{\pi^{\epsilon_1,\ldots\bar{\epsilon}_j,\ldots, \epsilon_{\statedim-1}}}


\newcommand{\Pimon}{\Pi_{\mathcal{M}}}
\newcommand{\beliefmon}{\belief_{\mathcal{M}}}

\newcommand{\ric}{\Gamma}
\newcommand{\iter}{n}
\newcommand{\iterfinal}{N}

\renewcommand{\time}{k}
\newcommand{\timet}{t}


\newcommand{\Hyperplane}{\mathcal{H}}


\newcommand{\Io}{{\Pi^o(X)}}
\newcommand{\Ib}{{\Pi^b(X)}}


\newcommand{\poly} {S}

\renewcommand{\l}{\mathcal{L}}


\newcommand{\unordered}{\sim}

\newcommand{\eye}{\textit{I}}
\newcommand{\bd}{\succeq_{\mathcal{B}}}
\newcommand{\aB}{R}
\newcommand{\tpone}{{\tp}}
\newcommand{\tptwo}{\bar{\tp}}

\newcommand{\tpe}{p}
\newcommand{\oprobe}{b}
\newcommand{\boprob}{\bar{\oprob}}

\newcommand{\noise}{n}
\newcommand{\asmp}{\textbf{A}}
\newcommand{\asmpg}{\bar{\text{A}}}

\newcommand{\statelvl} {h}
\newcommand{\sqg} {H}
\newcommand{\expcost} {C}

\newcommand{\cop}{C_o}
\newcommand{\thr}{\mathbf{\Gamma}}
\newcommand{\copomat}{\Gamma}

\newcommand{\uvalue}{\underline{V}}
\newcommand{\bvalue}{\overline{V}}
\newcommand{\valueaction}{Q}
\newcommand{\uvalueaction}{\underline{Q}}
\newcommand{\bvalueaction}{\overline{Q}}
\newcommand{\optvalue}{V}
\newcommand{\trid}{\Upsilon}
\newcommand{\filternorm}{\sigma}

\newcommand{\gc} {\succeq}
\newcommand{\lc} {\preceq}

%\newcommand{\ls}{_{s}{\le}}
\newcommand{\error}  {\eta}


\newcommand{\mus}{{\mu^*(\model)}}
\newcommand{\bmus}{{\mu^*(\bmodel)}}

\newcommand{\tcost}[1]{J_\mu^{(#1)}}


\newcommand{\rmm}{\rho_{\model,\bmodel}}

\newcommand{\lcost}{\underline{\textit{C}}}
\newcommand{\ucost}{\overline{\textit{C}}}
\newcommand {\uf} {\textit{f}}%{\underline{\textit{f}}\xspace}
\newcommand {\of} {\textit{g}}%{\overline{\textit{f}}\xspace}
\newcommand {\nooverlap} {\bar{\Belief}}
\newcommand {\setoverlap} {{\Pi_{O}}}
%\newcommand {\optpolicy} {\mu^*}
\newcommand {\policyu} {\overline{\mu}}
\newcommand {\policyl} {\underline{\mu}}


\newcommand{\yp}{y_{\pi;\model}^*}
\newcommand{\yps}{y_{\pi;\model,\bmodel}^*}
\newcommand{\ypsx}{y_{e_X;\model,\bmodel}^*}
\newcommand{\ytt}{y^*_{\model,\bmodel}}


\newcommand {\Timel} {\Gamma_l}
\newcommand {\Timev} {\Gamma_v}
%\newtheorem{mytheorem}{Theorem}
%\newcommand{\argmin}{\operatornamewithlimits{argmin}}
\newcommand{\basisvec} {\textit{e}}
\newcommand {\vect} {\textit{v}}
\newcommand {\percentloss} {\epsilon}
\newcommand {\approxpolicy} {\tilde{\mu}}
\newcommand{\bT}{\bar{T}}
\newcommand{\bsigma}{\bar{\sigma}}
\newcommand{\pomSt}{\ensuremath{s}}
\newcommand{\pomStSpc}{\ensuremath{S}}
\newcommand{\pomRwd}{\ensuremath{g}}
\newcommand{\pomTranMat}{\ensuremath{P}}
\newcommand{\pomObsMat}{\ensuremath{R}}
\newcommand{\pomAct}{\ensuremath{a}}
\newcommand{\pomActSpc}{\ensuremath{A}}
\newcommand{\pomObs}{\ensuremath{\theta}}
\newcommand{\pomObsSpc}{\ensuremath{\Theta}}
\newcommand{\pomHst}{\ensuremath{\Psi}}
\newcommand{\pomPol}{\ensuremath{\delta}}
\newcommand{\pomPolSpc}{\ensuremath{\Delta}}
\newcommand{\pomRwdFn}{\ensuremath{V}}
\newcommand{\pomHMM}{\ensuremath{\Phi}}
\newcommand{\pomPolVecSet}{\ensuremath{\Gamma}} % the PWLC representation of a policy
\newcommand{\pomPolVec}{\ensuremath{\gamma}}
\newcommand{\pomPolMapLB}{\ensuremath{\underline{\cal{H}}}} % lovejoy lower bound
\newcommand{\pomPolMapUB}{\ensuremath{\overline{\cal{H}}}} % lovejoy upper bound
\newcommand{\pomPolMap}{\ensuremath{\mathcal{H}}}

\newcommand{\comAct}{\pomAct} %
\newcommand{\comActSpc}{\pomActSpc}
\newcommand{\comObs}{\pomObs}
\newcommand{\comObsSpc}{\pomObsSpc}
\newcommand{\comPol}{\pomPol}
\newcommand{\comPolSpc}{\pomPolSpc}
\newcommand{\comRwdFn}{\pomRwdFn}
\newcommand{\obsMat}{R}



\newcommand{\gl}{\geq_{L_i}}
\newcommand{\glp}{\geq_{L_{i+1}}}

\newcommand{\glx}{\geq_{L_x}}
\newcommand{\glX}{\geq_{L_X}}
\newcommand{\glone}{\geq_{L_1}}


\newcommand{\bp}{{\bar{\pi}}}
%%
%grad
\def \La {{\cal L}}
\def \GF{{\widehat{\nabla C}^{\text{WD}}}}
\def \GS{{\widehat{\nabla C}^{\text{Score}}}}


%% SPSA
\newcommand{\wderiv}{g}
%\newcommand{\deriv}{L}
\newcommand{\hphi}{\hat{\phi}}
\newcommand{\nablat}{\widehat{\nabla}_{\phi}}
\newcommand{\direction}{\omega}

%%%%
%Sensor scheduling
\newcommand{\ac}{\tt{active}}
\newcommand{\co}{\tt passive}
\newcommand{\pr}{\tt predict}
\newcommand{\fa}{\rho^{\text{\ac}}}
\newcommand{\fc}{\rho^{\text{\co}}}
\newcommand{\fp}{\rho^{\text{\pr}}}
\newcommand{\ha}{r^{\text{\ac}}}
\newcommand{\hc}{r^{\text{\co}}}
\newcommand{\hp}{r^{\text{\pr}}}
\newcommand{\use}{n}
\newcommand{\tar}{z}

%%% quickest detection
%%%%%%
\newcommand{\epoch}{\tau}
\newcommand{\D}{D}
\newcommand{\delayset}{\{\D_1,\D_2,\ldots,\D_L\}}
\newcommand{\delayseti}{\{1,2,\ldots,L\}}
\newcommand{\kstar}{k^*}
\newcommand{\changetime}{\tau^0}
\newcommand{\falsealarm}{{f}}
\newcommand{\Vb}{\bar{V}}
\newcommand{\Pp}{\P_\mu}
\newcommand{\Cb}{\bar{C}}
\newcommand{\risk}{\epsilon}
%%%%
\newcommand{\stepsize}{\epsilon}

\newcommand{\A}{\mathbb{A}}
%%%%

%%%radar
\newcommand{\bkalmancov}{\bar{\Sigma}}
\newcommand{\price}{\nu} %{\mathbf{\nu}}
\newcommand{\Mu}{\boldsymbol{\mu}}
\newcommand{\grnew}{\succeq}
\newcommand{\gsnew}{\succ}
\newcommand{\lrnew}{\preceq}
\newcommand{\Q}{\mathcal{Q}}
\newcommand{\bt}{\underline{\theta}}
\newcommand{\lstar}{{l^*}}
\def\Lyapunov{{\cal L}}
\def\Ricatti{{\cal R}}
\def\pdfmat{\mathcal{M}}
%%%
%bandit
\newcommand{\numtarget}{L}
\newcommand{\target}{l}
\newcommand{\delt}{\delta}

%%%%%%%%

%%% disc opt %%
\newcommand{\bm}{w}
\newcommand{\globalopt}{\mathcal{G}}
\newcommand{\cd}{(\cdot)}
%\newcommand{\br}{\boldsymbol{b}^{\gamma}}
\newcommand{\br}{{b}^{\gamma}}
\newcommand{\bbr}{b^{\gamma}}
\newcommand{\boldf}{f}
\def\z{\mathbf{z}}
\def\lb{\left[}
\def\rb{\right]}
\def\lbr{\left\lbrace}
\def\rbr{\right\rbrace}
\def\RR{{\mathbb{R}}}
\newcommand{\pp}{{\mathbb P}}
\newcommand{\maxdiff}{D}

\newcommand{\score}{S}
\newcommand{\batchsize}{N}
\newcommand{\infodelta}{\info^\Delta}

%\newcommand{\degreelink}{\alpha}
\newcommand{\degreelink}{\th}
%%%%

%%%
%% primal dual
%%

\newcommand{\penaltyweight}{\Delta}

%%%%%
%%
%%% part 4  commands  %%%
%%%%
\def\ph{\varphi}
\newcommand{\M}{{\cal M}}
\newcommand{\lbar}{\overline}
\newcommand{\wdt}{\widetilde}

%\def\l{\Big|}

%\def\r{\Big|}
\newcommand{\ad}{&\!\!\!\disp}
\newcommand{\aad}{&\disp}
\newcommand{\barray}{\begin{array}{ll}}
\newcommand{\earray}{\end{array}}
\newcommand{\disp}{\displaystyle}

\newcommand{\h}{{\ell}}
\def\cd{(\cdot)}
\newcommand{\e}{\varepsilon}
\newcommand{\tth}{\tilde{\th}}
%%%
%%%
%% population dynamics and social network
\newcommand{\numagents}{M}

\newcommand{\nature}{\state}
\newcommand{\pop}{\theta}
\newcommand{\bpop}{\bar{\pop}}
\newcommand{\mpop}{\bar{\pop}}
\newcommand{\popspace}{\Theta}
\newcommand{\pa}{\alpha}

\newcommand{\network}{G}
\newcommand{\Vertexset}{V}
\newcommand{\vertexnum}{\numagents}
\newcommand{\vertexset}{\{1,2,\ldots,\vertexnum\}}
\newcommand{\edgeset}{E}
\newcommand{\degdist}{\rho}
\newcommand{\nodem}{m}
\newcommand{\degreediff}[1]{D^{(#1)}}
\newcommand{\atp}{\bar{\tp}}
\newcommand{\ed}{\bar{A}}

\newcommand{\wdh}{\widehat}


%%%% social learning chapter
 
\newcommand{\Req}{\mathcal{R}}
\newcommand{\poll}{\Req_{L}}
\newcommand{\additional}{\mathcal{E}}
\newcommand{\cbelief}{l^0}
\newcommand{\logoprob}{o}
\newcommand{\obseq}{\mathcal{Y}}
\newcommand{\omat}{\mathcal{O}}
\newcommand{\loprob}{\bar{R}}


%% afriat
   \newcommand{\probe}{\belief}
\newcommand{\response}{a}
\newcommand{\responseb}{\bar{x}}
\newcommand{\utility}{V}
\newcommand{\budget}{I}
\newcommand{\dataset}{\mathcal{D}}
    \newcommand{\norm}[1]{\lVert#1\rVert}
\newcommand{\tindx}{k}
\newcommand{\Tindxter}{\finaltime}


%%% mean field

\newcommand{\mtg}{\nu}
%\newcommand{\dev}{\Delta}
\newcommand{\dev}{\tilde{\theta}}


%%% search
\newcommand{\blocked}{b}

%% multivariate POMDP


\newcommand{\Sc}{\statespace}


\newcommand{\eprob}{\gamma}
\newcommand{\suffstat}{\eta}
\newcommand{\simplex}{\Pi}

\newcommand{\batch}{\iota}
\newcommand{\bsize}{N}
\newcommand{\bindex}{n}

\newcommand{\weightnl}{\alpha}

\newcommand{\Vt}{V(\filter(\pi,p,a))}

\newcommand{\sigfilter}{\filterd(\pi,p,a)}

\newcommand{\moncost}{c_o}
\newcommand{\monprice}{p}
\newcommand{\monreward}{c_\monprice}
\newcommand{\rewardv}[1]{c_{\monprice,#1}}
\newcommand{\rewv}{r_\monprice}


\newcommand{\riskf}{\mathbf{R}}
%\newcommand{\riskt}{\tilde{\E}}
\newcommand{\riskt}{\mathcal{R}}

\newcommand{\bfun}{\bar{\fun}}
%\newcommand{\det}{\operatorname{det}}


\newcommand{\valuer}{W}
\newcommand{\rbelief}{\alpha}

\begin{document}

%\title{Sequential Social Learning in Sensor Networks: From Herding to  Socially Optimal Behavior}
\title{POMDP Structural Results for Controlled Sensing}
\author{Vikram Krishnamurthy
\thanks{V. Krishnamurthy is
 with the Department of Electrical and Computer
Engineering, Cornell University, USA. 
(email:  vikramk@cornell.edu).}}


\maketitle

%%%%%%%%%%%%%%%%%%%%%%%%%%%%%%%%%%%%%%%%%%%%%%%%%%%%%%%%%%%%%%%%%%%%%%

% Samples of sectioning (and labeling) in OPRE
% NOTE: (1) \section and \subsection do NOT end with a period
%       (2) \subsubsection and lower need end punctuation
%       (3) capitalization is as shown (title style).
%
%\section{Introduction.}\label{intro} %%1.
%\subsection{Duality and the Classical EOQ Problem.}\label{class-EOQ} %% 1.1.
%\subsection{Outline.}\label{outline1} %% 1.2.
%\subsubsection{Cyclic Schedules for the General Deterministic SMDP.}
%  \label{cyclic-schedules} %% 1.2.1
%\section{Problem Description.}\label{problemdescription} %% 2.

% Text of your paper here

\section{Introduction}
Structural results for POMDPs are important since solving POMDPs numerically are typically intractable. Solving a classical POMDP is known to be PSPACE-complete
\cite{PT87}.
Moreover, in controlled sensing problems \cite{Kri02,KD09,EKN05}, it is often necessary to use POMDPs that are 
nonlinear in the belief state in order to model the uncertainty in the state estimate. (For example, the variance of the state estimate is a quadratic function of the belief.)
In such cases, there is no finite dimensional characterization of the optimal POMDP policy even for a finite horizon.


The seminal papers \cite{Lov87,Rie91,RZ94} give sufficient conditions on the costs, transition provabilities and observation probabilities so  that the value function of a POMDP is monotone with respect to 
the monotone likelihood ratio  (MLR) order (and more generally the multivariate TP2 order).
These papers then use this monotone  result  to show that the optimal policy
can be lower bounded by a myopic policy. 
Our recent works \cite{Kri16,KP15} relax the conditions on the transition matrix to construct myopic lower and upper bounds.  





\section {The Partially Observed Markov Decision Process} \label{sec:pomdp}
For notational convenience, we consider  a discrete time, infinite horizon discounted cost POMDP. A   discrete time Markov chain  evolves on the  state space $\statespace = \{1,2,\ldots, \statedim\}$. Denote the
action space  as $\actionspace = \{1,2,\ldots,\actiondim\}$ and observation space as $\obspace$. For discrete-valued observations $\obspace = \{1,2,\ldots,\obsdim\}$ and for continuous observations $\obspace \subset \reals$.

Let
$\Belief = \left\{\belief: \belief(i) \in [0,1], \sum_{i=1}^\statedim \belief(i) = 1 \right\}$ denote the belief space of $\statedim$-dimensional probability vectors.  For stationary policy  $\policy: \Belief \rightarrow \actionspace$,
 initial belief  $\belief_0\in \Belief$,  discount factor $\discount \in [0,1)$, define the  discounted cost:
\begin{align}\label{eq:discountedcost}
J_{\policy}(\belief_0) = \Ep\left\{\sum_{\time=0}^{\infty} \discount ^{\time} \cost_{\policy(\belief_\time)}^\p\belief_\time\right\}.
\end{align}
%that minimizes (\ref{eq:discountedcost}).
Here $\cost_\action = [\cost(1,\action),\ldots,\cost(\statedim,\action)]^\p$, $u\in \actionspace$ is the cost vector for each action, and the belief state evolves as
$\belief_{k} = \filter(\belief_{k-1},\obs_k,\action_k)$ where
\begin{align}  \filter\left(\belief,\obs,\action\right) = \cfrac{\oprob_{\obs} (\action)\, \tp^\p(\action)\belief}{\filterd\left(\belief,\obs,\action\right)} , \quad
\filterd\left(\belief,\obs,\action\right) = \one_{\statedim}'\oprob_{\obs}(\action) \tp^\p(\action)\belief, \quad
\oprob_{\obs}(\action) = \diag\{\oprob_{1,\obs}(\action),\cdots,\oprob_{\statedim,\obs}(\action)\}. \label{eq:information_state}
\end{align}
Here  $\one_{\statedim}$ represents a $\statedim$-dimensional vector of ones,
$ \tp(\action) = \left[\tp_{ij}(\action)\right]_{\statedim\times\statedim}$
$ \tp_{ij}(\action) = \prob(\state_{\time+1} = j | \state_\time = i, \action_\time = a )$ denote the transition probabilities,
 $\oprob_{\state\obs}(\action) = \prob(\obs_{\time+1} = \obs| \state_{\time+1} = \state, \action_{\time} = \action)$ when $\obspace$ is finite,
 or  $\oprob_{\state\obs}(\action)$ is the conditional probability density function when $\obspace \subset \reals$.


The aim is to compute the optimal  stationary policy $\optpolicy:\Belief \rightarrow \actionspace$ such that
$J_{\optpolicy}(\belief_0) \leq J_{\policy}(\belief_0)$ for all $\belief_0 \in \Belief$.
Obtaining the optimal policy  $\optpolicy$ is equivalent to solving
 Bellman's  dynamic programming equation:
$ \optpolicy(\belief) =  \underset{\action \in \actionspace}\argmin~ Q(\belief,\action)$, $J_{\optpolicy}(\belief_0) = \valuef(\belief_0)$, where
\begin{equation}
\valuef(\belief)  = \underset{\action \in \actionspace}\min ~Q(\belief,\action), \quad
  Q(\belief,\action) =  ~\cost_\action^\prime\belief + \discount\sum_{\obs \in \obsdim} \valuef\left(\filter\left(\belief,\obs,\action\right)\right)\filterd \left(\belief,\obs,\action\right). \label{eq:bellman}
\end{equation}

Since  $\Belief$ is continuum, Bellman's equation \eqref{eq:bellman} does not translate into practical solution methodologies as  the value function $\valuef(\belief)$ needs to be evaluated at each $\belief \in \Belief$. 


\subsection{POMDPs in Controlled Sensing}
In controlled sensing, to incorporate
 uncertainty of  the state estimate, we generalize the above POMDP to consider costs that are nonlinear in the belief.
Consider the following   instantaneous cost at each time $k$:
 $$ \cost(\state_k,\action_k) + \nlcost(\state_k,\belief_k,\action_k), \quad \action_k \in \actionspace = \{1,2,\ldots,\actiondim\}. $$
  (i) {\em Sensor Usage Cost}:
$\cost(\state_k,\action_k)$ denotes the instantaneous cost of using sensor  $\action_k$  at time $k$ when the
 Markov chain is in state $\state_k$. \\
 (ii) {\em Sensor Performance Loss}:  $\nlcost(\state_k,\belief_k,\action_k)$ models the
performance loss when using sensor $\action_k$.  This loss is modeled as an  explicit function of the belief state $\belief_k$  to capture the uncertainty in the state estimate. %Expressed
%in terms of the belief state,  it is shown below that this cost becomes nonlinear.

Typically there is trade off between the sensor usage cost and performance loss.
 Accurate sensors have high usage cost but small performance loss.

Then in terms of the belief state,  the instantaneous cost can be expressed as
\beq \label{eq:nlcoststart}
\begin{split} \Cost(\belief_k,\action_k) &= \E\{\cost(\state_k,\action_k)+ \nlcost(\state_k,\belief_k,\action_k)| \info_k\} \\
%&= \sum_i \cost(i,\action_k) \belief_k(i) +    \sum_{i} {\nlcost}(i,\belief_k,\action_k) \belief_k(i) \\
&= \cost_{\action_k}^\p \belief_k + \nlCost(\belief_k,\action_k), \\
\text{ where } & \cost_\action = (c(\action,1),\ldots,\cost(\action,\statedim))^\p,  \\ & \nlCost(\belief_k,\action_k)  \ole \E\{\nlcost(\state_k,\belief_k,\action_k)| \info_k\} = \sum_{i=1}^\statedim {\nlcost}(i,\belief_k,\action_k)\, \belief_k(i) .\end{split}
\eeq
Define the controlled sensing objective
\begin{align}\label{eq:costcst}
J_{\policy}(\belief_0) = \Ep\left\{\sum_{\time=0}^{\infty} \discount ^{\time} \nlCost(\belief_\time, \action_\time) \right\}.
\end{align}
In controlled sensing,
the aim is to compute the optimal  stationary policy $\optpolicy:\Belief \rightarrow \actionspace$ such that
$J_{\optpolicy}(\belief_0) \leq J_{\policy}(\belief_0)$ for all $\belief_0 \in \Belief$.
Obtaining the optimal controlled sensing policy  $\optpolicy$ is equivalent to solving
 Bellman's  dynamic programming equation:
$ \optpolicy(\belief) =  \underset{\action \in \actionspace}\argmin~ Q(\belief,\action)$, $J_{\optpolicy}(\belief_0) = \valuef(\belief_0)$, where
\begin{equation}
\valuef(\belief)  = \underset{\action \in \actionspace}\min ~Q(\belief,\action), \quad
  Q(\belief,\action) =   \Cost(\belief,\action)+ \discount\sum_{\obs \in \obsdim} \valuef\left(\filter\left(\belief,\obs,\action\right)\right)\filterd \left(\belief,\obs,\action\right). \label{eq:bellmancs}
\end{equation}
\subsection{Examples of Nonlinear Cost POMDP}
The non-standard feature of the objective (\ref{eq:costcst}) is  the nonlinear performance loss terms  $\nlCost(\belief,\action)$.
 These   costs\footnote{A linear function $ \cost_\action^\p \belief $ cannot attain its maximum at the centroid
of a simplex since a linear function achieves it maximum at a boundary point.}  should be 
chosen so that they are   zero at the vertices $e_i$ of the belief space $\Belief$  (reflecting perfect state estimation) 
and largest at the centroid of the belief space (most uncertain estimate). We now discuss examples of $\nlcost(\state,\belief,\action)$ and
its conditional expectation $\nlCost(\belief,\action)$.  \\
{\em (i). Piecewise Linear Cost}: Here we choose the performance loss as
\beq \nlcost(\state, \belief,\action) = \begin{cases}
   0  & \text{ if } \|\state - \belief\|_\infty \leq \epsilon \\
     \epsilon & \text{ if } \epsilon \leq \|\state- \belief\|_\infty \leq 1 - \epsilon \\
      1 & \text{ if } \|\state-\belief\|_\infty \geq 1 - \epsilon  \end{cases} , \quad \epsilon \in [0,0.5]. \label{eq:pwcost}\eeq
        Then $\nlCost(\belief,\action)$ is piecewise linear and concave.
      This cost is useful for subjective decision making. e.g., the distance of a target to a radar is quantized into three regions: close, medium
      and far.
          \\
{\em (ii). Mean Square, $l_1$ and  $l_\infty$ Performance Loss}: Suppose in (\ref{eq:costcst}) we choose
   \begin{equation}  \label{eq:l2u} \nlcost(\state, \pi,u) =
 \alpha(u) (\state - \belief)^\p M (\state - \belief) + \beta(u)
 %\alpha(u)|g^\p \state - g^\p \pi|^2 +\beta(u) 
 , \;\state \in \{e_1,\ldots,e_\statedim\}, \pi \in \Pi.
\end{equation}
Here $M$ is  a user defined positive semi-definite symmetric matrix, $\alpha(u)$ and $\beta(u)$, $u \in \actionspace$ are user defined positive scalar
weights  that
allow different sensors (sensing modes)
to be  weighed differently. So (\ref{eq:l2u})   is the 
squared error of the Bayesian estimator (weighted by $M$, scaled by $\alpha(u)$
and translated by $\beta(u)$). %This  is an intuitively appealing measure of sensor performance.
In terms of the belief state, the mean square performance loss (\ref{eq:l2u})  is %the conditional variance of the Markov chain estimate and is a quadratic function of the belief:
\beq  \nlCost(\belief_k, \action_k) = \E\{\nlcost(\state_k,\belief_k,\action_k)| \info_k\}=
 \alpha(\action_k) \bigl( \sum_{i=1}^\statedim M_{ii} \belief_k(i) - \belief_k^\p M \belief_k\bigr) + \beta(u_k)
% \alpha(\action_k)(h^\p \pi_k -\pi_k^{\p} g g^\p\pi_k) + \beta(u_k) \\
%\text {where } h \ole (g_1^2,\ldots,g_S^2)^\p.
\label{eq:quadcostconcave}
\eeq
because $\E\{ (x_k-\belief_k)^\p M (x_k-\belief_k) | \info_k\} = \sum_{i=1}^\statedim (e_i - \belief)^\p M (e_i - \belief) \belief(i)$.
The cost (\ref{eq:quadcostconcave}) is quadratic and concave in the belief. \\
Alternatively, if $ \nlcost(\state, \pi,u)  = \|\state - \belief\|_1$ then  $ \nlCost(\belief, \action) = 2 (1 - \belief^\p \belief)$ is also quadratic
in the belief. Also, choosing  $ \nlcost(\state, \pi,u)  = \|\state - \belief\|_\infty$ yields $ \nlCost(\belief, \action) =  (1 - \belief^\p \belief)$.

\noindent
{\em  (iii). Entropy based  Performance Loss}:
Here   we choose
%the sensor performance cost $\nlcost(\state,\pi,u)$ in (\ref{eq:costcst}) is the
% entropy of the information state: %(where $\alpha(u),\beta(u)$ are defined as in Case 1):
  \begin{equation}  %\nlcost( \state,\pi,u) =
   \nlCost(\belief,\action)
 =-\alpha(u) \sum_{i=1}^S \pi(i) \ltwo \pi(i) + \beta(u), \qquad  \pi \in \Pi .\label{eq:entropy}\end{equation}
The intuition 
is that an inaccurate sensor with cheap usage cost yields a Bayesian  estimate $\pi$ 
with a higher entropy compared to an accurate sensor.


\section{Structural Result 1 - Convexity of Value Function and Stopping Set}
Our  first result is that the value function $V(\pi)$  in (\ref{eq:bellmancs}) is concave in $\belief \in \Belief$.


\begin{theorem} \label{thm:concavevaluegencost} Consider a POMDP with possibly continuous-valued observations.
Assume that for each action $u$, the instantaneous cost $\Cost(\belief,\action)$
 are concave and continuous with respect to  $\belief \in \Belief$.
Then the value function  $\valuef(\belief)$ is concave in $\belief$.
\end{theorem}
The proof is given in \cite[Chapter 8]{Kri16}.


\subsection{Convexity of Stopping Set for Stopping Time POMDPs with nonlinear cost}
With the above concavity result we have the following important result for contolled sensing stopping time POMDPs.
A stopping time  POMDP has  action space $\actionspace = \{1 \text{ (stop)},2 \text{ (continue)}\}$.


The  stop action  $u=1$ incurs a terminal cost
of $\cost(\state,\action=1)$ and the problem terminates.


For continue action $\action= 2$, the  state $\state \in \statespace = \{1,2,\ldots,\statedim\}$  evolves with transition matrix $\tp$ and is observed
 via observations
$\obs$ with
observation probabilities $\oprob_{\state\obs} = \prob(\obs_k=\obs|\state_k=\state)$.   An instantaneous  cost $\cost(\state,\action=2)$ is incurred. Thus for $\action=2$, 
the belief state evolves according to the HMM filter
$\belief_{k} = \filter(\belief_{k-1},\obs_k)$. Since  action 1 is  a stop action and has no dynamics, to simplify notation,
we write $\filter(\belief,\obs,2)$ as $\filter(\belief,\obs)$ and $\filterd(\belief,\obs,2)$ as $\filterd(\belief,\obs)$  in this subsection.



For the  stopping time POMDP, $\optpolicy$ is the solution of
 Bellman's equation which is of the form  
\begin{align} \label{eq:bellmanstop}
 \optpolicy(\belief) &=  \underset{\action \in \actionspace}\argmin ~\valueaction(\belief,\action), \quad 
\optvalue(\belief) = \underset{\action \in \actionspace}\min ~\valueaction(\belief,\action),     \\
  \valueaction(\belief,1) &= \cost_1^\prime\belief , \quad
  \valueaction(\belief,2) =  \Cost(\belief,2) + \discount\sum_{\obs \in \obsdim} \optvalue\left(\filter\left(\belief,\obs\right)\right)\filternorm \left(\belief,\obs\right).
\nonumber 
\end{align}
where $\filter(\belief,\obs)$ and $\filterd(\belief,\obs)$ are the HMM filter and normalization (\ref{eq:information_state}).

We now present the first  structural result for stopping time POMDPs:  the stopping region for the optimal policy is convex.
Define the stopping set $\region_1$  as the set of belief states for which stopping ($\action=1$)  is the optimal action.
Define $\region_2$ as the set of belief states for which continuing ($\action=2$) is the optimal action. That is
\beq
\region_1 = \{\belief:  \optpolicy(\belief) = 1  \text{ (stop) }\} , \quad \region_2 =  \{\belief:  \optpolicy(\belief) = 2 \} = \Belief - \region_1.\eeq

  The theorem below shows that the stopping set $\region_1$  is convex (and therefore a connected set). 
Recall that the value function $\valuef(\belief)$ is concave on $\Belief$. 

\begin{theorem} 
\label{thm:pomdpconvex}
Consider the stopping-time POMDP with value function given by (\ref{eq:bellmanstop}). Suppose that the possibly nonlinear cost $C(\pi,2)$ is concave in $\pi$.
Then the stopping set $\region_1$ is a convex subset of the belief space $\Belief$. \index{convexity of stopping region}
\end{theorem}

\begin{proof}
Pick any two belief states $\belief_1,\belief_2 \in \region_1$. To demonstrate convexity of $\region_1$,
we need to show for any $\lambda \in [0,1]$,  $\lambda \belief_1 + (1-\lambda) \belief_2 \in \region_1$.
%This is shown as follows.
Since $V(\belief)$ is concave,
\begin{align*}
V(\lambda \belief_1 + (1-\lambda) \belief_2) &\geq \lambda V(\belief_1) + (1-\lambda) V(\belief_2) \nonumber\\
&= \lambda Q(\belief_1,1) + (1-\lambda) Q(\belief_2,1)  \text{ (since $\belief_1,\belief_2 \in \region_1$) } \nonumber\\
&= Q(\lambda \belief_1 + (1-\lambda) \belief_2,1 ) \text{ (since $Q_{1}(\belief,1)$ is linear in $\belief$) }\nonumber \\
& \hspace{-1cm} \geq V(\lambda \belief_1 + (1-\lambda) \belief_2) \text{ (since $V(\belief)$ is the optimal value function) }
\end{align*}
Thus all the inequalities above are equalities, and $\lambda \belief_1 + (1-\lambda) \belief_2 \in 
\region_1$.
\end{proof}

The above theorem is a small extension of \cite{Lov87a} which deals with case when the costs $C(\pi,2)$ are linear in $\pi$.
The proof is exactly the same as in \cite{Lov87a} -- all that is required is that $C(\pi,2)$ is concave




\subsection{Example.  Quickest Change Detection with Nonlinear Delay Cost}  \label{sec:classicalqd}  \index{quickest detection! classical|(}

Quickest  detection is a useful example of a stopping time POMDP that  has applications in
numerous areas   \cite{PH08,BN93}. 
The classical Bayesian quickest detection problem is as follows: 
An underlying discrete-time state process $\state$ jump changes at a geometrically distributed random time $\tau^0$.
Consider a sequence of random measurements $\{\obs_k,k \geq 1\}$, such that 
 conditioned on the event $\{\tau^0 = t\}$, $\obs_k$, $\{k \leq t\}$  are i.i.d. random variables with distribution 
$\oprob_{1\obs}$ and $\{\obs_k, k >t\}$ are i.i.d. random variables with distribution $\oprob_{2\obs}$.
The quickest  detection problem involves detecting the change time $\tau^0$ with minimal cost. That is,
at each time $k=1,2,\ldots$, a decision $u_k \in \{\text{continue}, \text{stop and announce change}\}$ needs to be made to optimize a tradeoff
between false alarm frequency and linear delay penalty.\footnote{There are two general formulations for quickest time
detection.  In the first
formulation, the change point $\tau^0$ is an unknown deterministic time,
and the goal is to determine a
stopping rule such that a  worst case delay penalty is
minimized subject to a constraint on the false alarm frequency
(see, e.g., \cite{Mou86,Poo98,YKP99,PH08}). 
The second formulation, which is the formulation considered in this book (this chapter and also Chapter \ref{chp:stopapply}),  is  the  Bayesian approach where
the change time $\tau^0$ is specified by a prior distribution.}

A geometrically distributed change time $\tau^0$ is realized by a  two state ($\statedim=2$) Markov chain  
with absorbing transition matrix $\tp$ and prior $\belief_0$ as follows:
\beq \tp = \begin{bmatrix} 1 & 0 \\ 1- \tp_{22} & \tp_{22}  \end{bmatrix} , \;  \belief_0 = \begin{bmatrix} 0 \\ 1 \end{bmatrix} , \quad
\tau^0 = \inf\{ k:  \state_k = 1\}. \label{eq:tpqdp} \eeq
The system starts  in state 2 and then jumps to  the absorbing state 1 at time $\tau^0$. Clearly $\tau^0$ is geometrically distributed
with mean $1/(1-\tp_{22})$.

The cost criterion in classical quickest detection is the {\em Kolmogorov--Shiryayev 
criterion} for detection of disorder \cite{Shi63}  \index{quickest detection! Kolmogorov--Shiryayev  criterion of disorder}
 \beq J_\policy(\belief) =   d\, \Ep\{(\tau - \tau^0)^+\} + \Pp(\tau < \tau^0) , \quad \belief_0 = \belief.
\label{eq:ksd} \eeq
where $\policy$ denotes the decision policy.
The first term is the delay penalty in making a decision at time $\tau > \tau^0$ and $d$ is a positive real number.
The second term is the false alarm penalty incurred in announcing a change at time $\tau< \tau^0$.

\noindent {\bf Stopping time POMDP}:
The quickest detection problem with penalty (\ref{eq:ksd}) is a stopping time POMDP with 
$\actionspace = \{1 \text{ (announce change and stop)},2  \text{ (continue)} \}$,  $\statespace=\{1,2\}$,
transition matrix in (\ref{eq:tpqdp}), arbitrary observation probabilities $\oprob_{\state\obs}$,
 cost vectors  $c_1 = [ 0 ,\; 1 ]^\p$, $c_2 = [d,\; 0 ]^\p$ and discount factor $ \discount = 1$.


In light of Theorem  \ref{thm:pomdpconvex}, we can generalize this to delay costs  $C(\pi,2)$ that are convex and nonlinear in the belief.
For example such a cost could be motivated by the square error or entropy of the belief reflecting an inaccurate state estimate.
We have the following  structural result.
 \begin{corollary} \label{cor:qdclassical}
 The optimal policy $\optpolicy$ for classical quickest detection has a {\em threshold} structure:
There exists a threshold point $\belief^* \in [0,1]$ such that  
\beq u_k = \optpolicy(\belief_k) = \begin{cases} 2 \text{ (continue) } & \text{ if }
\belief_k(2) \in [ \belief^*,1] \\   1 \text{ (stop and announce change)  } &  \text{ if } \belief_k(2) \in [0, \belief^*).
\end{cases} \label{eq:onedim}
\eeq  \end{corollary}
\begin{proof} Since $\statedim=2$, $\Belief$ is  the interval $[0,1]$, and   $\belief(2) \in [0,1]$ is the belief state.
Theorem \ref{thm:pomdpconvex} implies that the stopping set $\region_1$ is convex. In one dimension this implies  that 
$\region_1$ is an interval of the form $[a^*,\pi^*)$ for $0 \leq a< \pi^*\leq 1$. Since state 1 is absorbing,
 Bellman's equation (\ref{eq:bellmanstop}) with $\discount=1$ applied at $\belief = e_1$  implies
$$\optpolicy(e_1) = \argmin_u\{\underbrace{\cost(1,u=1)}_{0},\;\; d ( 1 - \belief(2)) + V(e_1)\} = 1.$$ % since $c(1,u=1) = 0$.
So $ e_1$ or equivalently $\belief(2) = 0$ belongs to $\region_1$. Therefore,
$\region_1$ is an interval of the form $[0,\belief^*)$.
Hence  the optimal policy is of the form  (\ref{eq:onedim}). \end{proof}


 Theorem \ref{thm:pomdpconvex} says that   for quickest
 detection of a multi-state Markov chain,
 the stopping set $\region_1$  is  convex for any concave  non-linear delay cost. This is different to the result in \cite{Kri11} which considered a nonlinear stopping cost (false alarm cost) - in \cite{Kri11} the stopping
 set was not necessarily convex. For additional results on controlled sampling with quickest detection see \cite{Kri13}. 
 
 
 
 \subsection*{Social Learning}
Social learning, or learning from the actions of others, is an integral part of human behavior and has been studied widely  in behavioral economics, sociology, electrical engineering and 
 computer science
  to model the  interaction of  decision makers \cite{BHW92,AO11,Cha04,EK10,Say14b,WD16,Kri12,KP13,KP14}. POMDPs with social learning result in interesting behaviour.

Social learning models  present unique challenges from a statistical signal processing point of view.
First,  agents interact with and influence each other.  For example, ratings posted on online reputation systems strongly influence the behavior of  individuals.
  This is usually not the case with physical sensors.  %\cite{IMS11,Luc11}.
%Such interactive sensing can result in  unusual information patterns due to correlations introduced by the learning process.
Second,  agents (humans) lack the capability to quickly absorb information  and translate it into decisions.
According to the paradigm of rational inattention theory, pioneered
by economics Nobel prize winner Sims \cite{Sim03},  attention is a time-limited resource that can be modelled in terms of an information-theoretic channel capacity. Therefore, while apparently mistaken decisions are ubiquitous, this does not imply that decision makers are irrational.\footnote{Limits on attention impact choice. For example, purchasers limit their attention to a relatively small
number of websites when buying over the internet; shoppers buy expensive products due to their failure to notice if sales tax is includes in the price  \cite{CD15}.}
  More recently for results in quickest detection POMDPs with social learning and risk averse agents
 please see \cite{Kri12,KB16}.


{\em Remark}: Of course, one of the best known examples of a stopping time problem is optimal search for a Markov target \cite{Eag84,MJ95,SK02,JK06}.
Another interesting example is a multiple stopping problem \cite{Nak95,KAB16}; this has applications in interactive advertising in social multimedia like YouTube. The problem has distinct parallels to scheduling in communication
systems \cite{NK09}.


\section{The value function is positively homogenous}

Define the positive $\statedim$-orthant as  $\reals_+^\statedim$.
On this positive orthant, define the relaxed belief state $\rbelief$. We can define the following Bellman's equation where $\valuer$ below denotes the value function with $\rbelief \in \reals_+^\statedim$.

\beq
\valuer(\rbelief)  = \underset{\action \in \actionspace}\min ~Q(\rbelief,\action), \quad
  Q(\rbelief,\action) =  ~\cost_\action^\prime\rbelief + \discount\sum_{\obs \in \obsdim} \valuer\left(\filter\left(\rbelief,\obs,\action\right)\right)\filterd \left(\rbelief,\obs,\action\right). \label{eq:rbellman}
\end{equation}

Clearly when $\rbelief $ is restricted to the belief space (unit simplex) $\Belief$, then $\valuer(\alpha) = \valuef(\alpha)$.  This can be established by mathematical induction (valued iteration) and the proof is omitted.
We now have the following result.

\begin{theorem} The relaxed value function $\valuer(\cdot)$ of a linear cost POMDP is positively homogenous. That is, for any constant $\kappa > 0$, 
$\valuer(\kappa \rbelief) =  \kappa \valuer(\rbelief)$. Therefore,  (\ref{eq:rbellman}) can be expressed as 
\beq  \label{eq:valuerep}
 \valuer(\rbelief)  = \underset{\action \in \actionspace}\min ~Q(\rbelief,\action), \quad Q(\rbelief,\action) =  ~\cost_\action^\prime\rbelief + \discount\sum_{\obs \in \obsdim} \valuer\left( \oprob_\obs(\action) \tp^\p(\action) \rbelief \right) 
\eeq

\end{theorem}
The proof is straightforward since the cost $c_u^\p \rbelief $  and $\sigma(\rbelief,\obs,\action)$ are linear in $\rbelief$ and 
$\filter(\kappa \rbelief, \obs,\mu) = \filter( \rbelief, \obs,\mu)$. 

It is this positive homogeneity property of the value function and especially the representation (\ref{eq:valuerep}) which allows for the finite horizon case to immediately show that the value function is piecewise linear
and concave.


\section{Monotone Value Function}

 \begin{definition}[Monotone Likelihood Ratio (MLR) order $\gr$]  \index{stochastic dominance! monotone likelihood ratio} \label{def:mlr}
Let $\belief_1, \belief_2 \in \Belief$ denote  two beliefs.
Then $\belief_1$ dominates $\belief_2$ with respect to the MLR order, denoted as
$\belief_1 \gr \belief_2$,
 if %(see Footnote \ref{foot:mlrdef} for a more general definition)
$ \belief_1(i) \belief_2(j) \leq \belief_2(i) \belief_1(j)$ $i < j$,  $i,j\in \{1,\ldots,\statedim\}$.  A function $\fun:\Belief\rightarrow \reals$ is said to be MLR increasing if $\belief_1 \gr \belief_2$ implies $\fun(\belief_1) \geq \fun(\belief_2)$. 

\end{definition}
\begin{enumerate}[\bf{(A}1)]

\item\label{it:decreasing_cost}  $\Cost(\belief,\action)$ is  first order stochastic  decreasing in $\belief$ for each $ \action \in \actionspace$.
\item\label{it:TP2_gen} $\tp(\action)$,  $\action \in \actionspace$ is  totally positive of order 2 (TP2):  all second-order minors are nonnegative.
\item\label{it:TP2_obs} $\oprob(\action)$,  $\action \in \actionspace$ is  totally positive of order 2 (TP2).
%\item \label{obsdom} $\oprob_{iy}(u+1) \, \oprob_{jy}(u) \leq \oprob_{iy} (u) \, \oprob_{jy}(u+1)$, $j > i$.
\end{enumerate}

\begin{theorem} \label{thm:valuedec}
Under A\ref{it:decreasing_cost},  A\ref{it:TP2_gen} and A\ref{it:TP2_obs}, the value function $V(\belief)$ in (\ref{eq:bellmancs}) is MLR decreasing.
\end{theorem}
The proof of the theorem is in \cite{Kri16,KD07}.

Note that for 2 states  ($\statedim = 2$),  one can  always  permute the observation labels so that A\ref{it:TP2_obs} holds. 
Moreover, A\ref{it:TP2_gen} then becomes the same as the first row being first order stochastic dominated by the second row.
Therefore for $\statedim=2$, the conditons for a monotone value function for a POMDP are identical to that for a fully observed MDP.

Based on extensive numerical experiments, we conjecture that assumption A\ref{it:TP2_obs} is not required for  Theorem \ref{thm:valuedec}.

\begin{conjecture} Under A\ref{it:decreasing_cost} and   A\ref{it:TP2_gen}, the value function $V(\belief)$ in (\ref{eq:bellmancs}) is MLR decreasing.
\end{conjecture}

This conjecture implies that monotone value functions for POMDPs require very similar conditions to monotone value functions for fully observed MDPs.
Of course, the TP2 condition  A\ref{it:TP2_gen} for the transition matrix is stronger than the first order dominance conditions on the transition matrix used for fully observed MDPs.

Finally we mention that one can also show that the value function involving controlled sensing with a Kalman filter is monotone \cite{KBGM12}.  In this case, the covariance matrices of the Kalman filters
are partially ordered with respect to positive definiteness.
Results for monotone HMM filters are given in 
\cite{KR14}. These monotone results can also be used for POMDP bandits as discussed in \cite{KW09}. One can also consider controlled sampling of an evolving duplication deletion graph; the dynamics of the belief
are given by the HMM filter as described in \cite{KBP16}.



\section{Blackwell Dominance and Optimality of Myopic Policies}


\subsection{Myopic Policy Bound to Optimal Decision Policy} \label{sec:myopic}

Motivated by active sensing applications, consider the following POMDPs where
based on the current
belief state $\belief_{k-1}$, agent $k$ chooses sensing mode $$u_k \in \{1 \text{ (low resolution sensor) }, 2 \text{ (high resolution sensor)}\}.$$

The assumption that mode $u=2$ yields more accurate observations than mode $u=1$ is modeled as follows: We say mode 2 {\em Blackwell
dominates} mode 1, denoted as \index{Blackwell dominance}
\beq \label{eq:bd}
  \oprob(2) \bd \oprob(1)\quad  \text{ if }  \quad  B(1)  = B(2)\, \aB . \eeq
   Here $\aB$ is a $Y^{(2)} \times Y^{(1)}$ stochastic matrix. $\aB$ can be viewed as a {\em confusion matrix} that maps $\obspace^{(2)}$ probabilistically to $\obspace^{(1)}$.
   (In a communications context, one can view $\aB$ as a noisy discrete memoryless channel with input $y^{(2)}$ and output $y^{(1)}$).
Intuitively (\ref{eq:bd})  means that
  $B^{(2)} $ is more accurate than   $B^{(1)} $.

The goal is to compute the optimal policy $\mu^*(\belief) \in \{1,2\}$ to minimize the  expected cumulative  cost incurred by  all the agents
\beq
J_\mu(\belief) = \Ep \{ \sum_{k=0}^\infty \discount^{k} C(\belief_{k},u_k) \} .
\eeq
where  $\discount \in [0,1)$ is  the discount factor.
%Here $\Ep$ and $\rho$ are defined in (\ref{eq:csdef}).
Even though solving the above POMDP 
is computationally intractable in general,
 using Blackwell dominance, we show below that a myopic policy forms a lower
 bound for the optimal policy.


The
value function $V(\belief)$ and optimal policy $\mu^*(\belief)$  satisfy Bellman's equation
\beq  \label{eq:dp_algmove}\begin{split}
V(\belief) &= \min_{u \in \actionspace} Q(\belief,u), \quad
\mu^*(\belief)= \arg\min_{u \in \actionspace} Q(\belief,u) ,\; J_{\mu^*}(\belief) = V(\belief) \\
 Q(\belief,u) &=  C(\belief,u) 
+ \discount \sum_{y^{(u)} \in \obspace^{(u)}}  V\left( T(\belief ,y, {u}) \right) \sigma(\belief,y, {u}), \\
T(\belief,y, {u}) &= \frac{B_{y^{(u)}} (\action) P^\p \belief}{\sigma(\belief,\obs,\action)},
\; \filterd(\belief,y, \action) = \mathbf{1}_X^\p B_{y^{(u)}}(\action) P^\p \belief .
\end{split} \eeq

We now present the structural result.
Let $\Pi^s \subset \I$ denote the set of belief states for which
$C(\belief,2) < C(\belief,1)$. %So from (\ref{eq:cp1})
%\beq \label{eq:pis}
%\Pi^s = \{\belief:   \sum_{y\in \Y} \min_a \cb^\p B_y \belief < \sum_{y\in \Y} \min_a \ca^\p B_y \belief\} .
%\eeq
Define the  myopic policy
$$\policyl(\belief) = \begin{cases} 2 & \belief \in \Pi^s \\
 								1 & \text{ otherwise } \end{cases}$$ 


\begin{theorem}\label{thm:compare2}  Assume that $\Cost(\belief,\action)$ is concave with respect to $\belief \in \Belief$
for each action $\action$.
Suppose  $\oprob(2) \bd \oprob(1)  $, i.e., $    B(1)  = B(2) \aB $ holds where $\aB$ is a stochastic matrix.
Then the myopic policy
 $\policyl(\belief)$ is a lower    bound to the optimal
policy $\mu^*(\belief)$, i.e.,  $\mu^*(\belief) \geq \policyl(\belief)$ for
all $\belief \in \I$. 
In particular, 
for $\belief \in \Pi^s$,  $\mu^*(\belief) = \policyl(\belief)$,
i.e., it is optimal to choose action 2 when the  belief is in $\Pi^s$.
\qed
\end{theorem}

{\em Remark}: If  $\oprob(1) \bd \oprob(2)  $, then the myopic policy constitutes an upper bound to the optimal policy.

Theorem \ref{thm:compare2} is proved below.
%Blackwell dominance for POMDPs is described in 
%\cite{WD80,Rie91}.
 The proof exploits the fact that the value function is concave  and uses Jensen's inequality.
The usefulness of Theorem \ref{thm:compare2} stems from the fact that $\policyl(\belief)$ is trivial to compute. It  forms a provable
lower bound to the computationally intractable optimal policy $\mu^*(\belief)$.  
Since $\policyl$ is sub-optimal, it incurs a higher   cumulative cost. This  cumulative cost can be evaluated via simulation and is
an upper bound to the achievable  optimal cost.


Theorem \ref{thm:compare2} is  non-trivial.
The instantaneous costs
satisfying $C(\belief,2) < C(\belief,1)$,  does not trivially  imply that the myopic policy 
$\policyl(\belief)$ coincides with the optimal policy $\mu^*(\belief)$, since the optimal policy applies to a cumulative cost function involving
an infinite horizon
 trajectory of the dynamical system.
%This is because the result uses Jensen's inequality
%applied to a convex function.

%Note also  that the action and observation spaces are of equal dimension.

\subsection{Example 1.  Optimal Filter  vs Predictor Scheduling} \index{sensor scheduling! filter vs predictor} 
\index{Blackwell dominance! filter vs predictor scheduling}
Suppose $\action=2$ is an active sensor  (filter) which obtains measurements of the underlying Markov chain and uses the optimal HMM  filter
on these measurements to compute the belief and therefore
the state estimate.
 So the usage cost of sensor 2 is high (since obtaining observations is expensive and can also result in increased threat of being discovered), but its performance cost is
low (performance quality is high). 

Suppose  sensor $\action=1$ is a predictor which needs no measurement. So its usage cost  is low (no measurement is required).
However its performance cost is high since it is more inaccurate compared to sensor 2.

Since the predictor has non-informative observation probabilities, its observation probability matrix is $\oprob(1) = 
\frac{1}{\obsdim}\ones_{\statedim \times \obsdim}$. So clearly
$\oprob(1) = \oprob(2)  \oprob(1) $
meaning that the filter (sensor 2)  Blackwell dominates the predictor (sensor 1)
Theorem \ref{thm:compare2} then says that if the current belief is $\belief_k$, then if  $\Cost(\belief_k,2) < \Cost(\belief_k,1)$, it is always
optimal to deploy the filter (sensor 2).


\subsection{Example 2.  Ultrametric Matrices  and Blackwell Dominance}
\index{Blackwell dominance! ultrametric matrix|(}
An $\statedim\times \statedim$ square matrix $\oprob$ is   a symmetric stochastic ultrametric matrix if \index{ultrametric matrix}
\index{root of stochastic matrix} \index{stochastic matrix! roots}
\index{Blackwell dominance! ultrametric matrix}
\begin{compactenum}
\item $\oprob$ is symmetric and stochastic. 
\item $\oprob_{ij} \geq \min \{\oprob_{ik}, \oprob_{kj}\}$ for all $i,j,k \in \{1,2,\ldots,\statedim\}$.
\item $\oprob_{ii} > \max \{\oprob_{ik}\}, k \in  \{1,2,\ldots,\statedim\}- \{i\}$  (diagonally dominant).
\end{compactenum}
 It is shown in \cite{HL11}  that if $\oprob$ is a symmetric  stochastic  ultrametric matrix, then the $\actiondim$-th root, namely $\oprob^{1/\actiondim}$, is also a stochastic matrix\footnote{Although we do not pursue it here, conditions that 
 ensure that  the $\actiondim$-th root of a transition matrix is a valid stochastic matrix  is important
 in interpolating Markov chains. For example, transition matrices for credit ratings on a yearly time scale
 can be obtained from rating agencies such as Standard \& Poor's. Determining the transition matrix for periods of six months involves the square root of the yearly transition matrix \cite{HL11}.}
for any positive  integer $\actiondim$. Then with $\bd$ denoting Blackwell dominance (\ref{eq:bd}),
clearly  $$\oprob^{1/\actiondim} \bd \oprob^{2/(\actiondim)} \bd \cdots \bd \oprob^{(\actiondim-1)/\actiondim} \bd \oprob . $$
Consider a  social network where
the reputations of agents are denoted as $\action \in \{1,2,\ldots, \actiondim\}$. An agent with reputation $\action$
has  observation probability matrix $\oprob^{(\actiondim-\action+1)/\actiondim}$.
So an agent with reputation 1 (lowest reputation) is $\actiondim$ degrees of separation from the source signal while an agent with reputation $\actiondim$ (highest reputation) is  \index{ultrametric matrix! Blackwell dominance}
1 degree of separation from the source signal. \index{social network! sampling}
The underlying source (state) could be a news event, sentiment or corporate policy that evolves
with time.  A marketing agency can sample these agents - it can sample high reputation agents that have accurate
observations but this costs more than sampling low reputation agents that have less accurate observations. 
Then Theorem \ref{thm:compare2}  gives a suboptimal policy that provably lower bounds the optimal sampling policy.
\index{Blackwell dominance! ultrametric matrix|)}

\subsection{Proof of Theorem  \ref{thm:compare2}}
% The proof uses Blackwell dominance of measures.
 

Recall from Theorem \ref{thm:concavevaluegencost} % in Chapter \ref{sec:nonlinearpomdpmotivation}
 that $\Cost(\belief,\action)$ concave implies that   $V(\belief)$ is concave on $\Belief$. 
We then use the Blackwell dominance condition (\ref{eq:bd}). In particular,
\begin{align*} \filter(\belief,\yi,1) &=   \sum_{\yii \in \obspace^{(2)}} \filter(\belief,\yii,2) \frac{\filterd(\belief,\yii,2)}{\sigs(\belief,\yi,1)} P(\yi|\yii)  \\
 \sigs(\belief,\yi,1) &= \sum_{\yii \in \obspace^{(2)}} \filterd(\belief,\yii,2) P(\yi|\yii). \end{align*}
Therefore $\frac{\filterd(\belief,\yii,2)}{\sigs(\belief,\yi,1)} P(\yi|\yii) $ is a probability measure w.r.t.\  $\yii$ (since the denominator is the sum
of the numerator over all $\yii$).
Since $V(\cdot)$ is concave, using Jensen's inequality it follows that
\begin{align}
&V(\filter(\belief,\yi,1) )  = V \left(\sum_{\yii \in \obspace^{(2)}} \filter(\belief,\yii,2) \frac{\filterd(\belief,\yii,2)}{\sigs(\belief,\yi,1)} P(\yi|\yii) \right)\nn  \\
&\geq \sum_{\yii \in \obspace^{(2)}}  V (\filter(\belief,\yii,2)) \frac{\filterd(\belief,\yii,2)}{\sigs(\belief,\yi,1)} P(\yi|\yii) \nn \\
&\implies   \sum_{\yi}  V(\filter(\belief,\yi,1) ) \sigs(\belief,\yi,1) \geq
\sum_{\yii} V(\filter(\belief,\yii,2)\filterd(\belief,\yii,2). \label{eq:bdproof1}
\end{align}
Therefore for $\belief \in \Pi^s$, 
$$ C(\belief,2) + \discount\sum_{\yii} V(\filter(\belief,\yii,2)\filterd(\belief,\yii,2) \leq 
C(\belief,1) + \discount \sum_{\yi}  V(\filter(\belief,\yi),1 ) \sigs(\belief,\yi,1) . $$
So for $\belief \in \Pi^s$, the optimal policy $\mu^*(\belief) = \arg\min_{u \in \actionspace}Q(\belief,u) = 2$.
So $\policyl(\belief) = \mu^*(\belief)=2$ for $\belief \in \Pi^s$ and $\bar{\mu}(\belief)=1$ otherwise, implying that
$\bar{\mu}(\belief)$ is a lower  bound for $\mu^*(\belief)$.

The above result is quite general and can be extended to  controlled sensing of  jump Markov linear systems \cite{DGK01,LK99,EKN05}.


\section{Inverse POMDPs and Revealed Preferences}
{\em How to develop data-centric non-parametric methods (algorithms and associated mathematical analysis) to identify utility functions of agents?}
Classical statistical decision theory  arising in electrical engineering (and statistics) is model based: given a model\footnote{In non-parametric detection theory, the set of decision rules can be considered to be the model.}, we wish to detect specific events in a dataset.  The  goal is the reverse:
given a dataset, we wish to determine  if 
the actions of  agents are consistent with  utility maximization behavior, or more generally, consistent with play from a  Nash equilibrium; and  we then wish to 
estimate the associated utility function. Such problems will be studied using  revealed preference methods arising in  micro-economics. Classical revealed preferences deals with
analyzing choices made by individuals.  The celebrated  ``Afriat's theorem" \cite{Var82,Blu05} provides a necessary and sufficient condition for a finite dataset  to have originated from a utility maximizer.
Specifically,  revealed preferences~\cite{KH15,HNK15,HKA16}, rational inattention, homophily~\cite{NHK16}, and social learning can be used to study 
multi-agent behavior in  social networks; particularly YouTube.

%\section{Conclusions}


% Appendix here
% Options are (1) APPENDIX (with or without general title) or
%             (2) APPENDICES (if it has more than one unrelated sections)
% Outcomment the appropriate case if necessary
%
% \begin{APPENDIX}{<Title of the Appendix>}
% \end{APPENDIX}
%
%   or
%
% \begin{APPENDICES}
% \section{<Title of Section A>}
% \section{<Title of Section B>}
% etc
% \end{APPENDICES}

%%
%\theendnotes

% Acknowledgments here
%\ACKNOWLEDGMENT{The authors gratefully acknowledge the existence of
%the Journal of Irreproducible Results and the support of the Society
%for the Preservation of Inane Research.}


% References here (outcomment the appropriate case)

% CASE 1: BiBTeX used to constantly update the references
%   (while the paper is being written).
%\bibliographystyle{informs2014} % outcomment this and next line in Case 1
%\bibliography{<your bib file(s)>} % if more than one, comma separated

% CASE 2: BiBTeX used to generate mypaper.bbl (to be further fine tuned)
%\documentclass[aps,prd,preprint,superscriptaddress,nofootinbib,tightenlines]{revtex4-1}
\usepackage{color}
\usepackage{amsmath}
\usepackage{graphicx}
\usepackage{latexsym}
\bibliographystyle{apsrev4-1}


\newcommand{\red}[1]{\textcolor{red}{#1}}


\begin{document}

\preprint{RESCEU-6/18, RUP-18-14}

\title{Self-anisotropizing inflationary universe in Horndeski theory and beyond}

\author{Hiroaki W. H. Tahara}
\affiliation{Research Center for the Early Universe (RESCEU), Graduate School of Science,
The University of Tokyo, Tokyo 113-0033, Japan}
\affiliation{Department of Physics, Graduate School of Science,
The University of Tokyo, Tokyo 113-0033, Japan}

\author{Sakine Nishi}
\affiliation{Department of Physics, Rikkyo University, Toshima, Tokyo 171-8501, Japan}

\author{Tsutomu Kobayashi}
\affiliation{Department of Physics, Rikkyo University, Toshima, Tokyo 171-8501, Japan}

\author{Jun'ichi Yokoyama}
\affiliation{Research Center for the Early Universe (RESCEU), Graduate School of Science,
The University of Tokyo, Tokyo 113-0033, Japan}
\affiliation{Department of Physics, Graduate School of Science,
The University of Tokyo, Tokyo 113-0033, Japan}
\affiliation{Kavli Institute for the Physics and Mathematics of the Universe (Kavli IPMU), WPI, UTIAS,
The University of Tokyo, Kashiwa, Chiba 277-8568, Japan}

\begin{abstract}
As opposed to Wald's cosmic no-hair theorem in general relativity,
it is shown that the Horndeski theory (and its generalization)
admits anisotropic inflationary attractors
if the Lagrangian
depends cubically on the second derivatives of the scalar field.
We dub such a solution as a self-anisotropizing inflationary universe
because anisotropic inflation can occur without introducing
any anisotropic matter fields such as a vector field.
As a concrete example of self-anisotropization
we present the dynamics of a Bianchi type-I universe
in the Horndeski theory.
\end{abstract}

\maketitle

\section{Introduction}
Inflation in the early universe \cite{Starobinsky:1980te,Sato:1980yn,Guth:1980zm,Sato:2015dga}
solves a number of fundamental problems in cosmology, such as horizon, flatness, monopole and the origin-of-structure problems.
Its basic predictions have been confirmed by a number of observations of the cosmic microwave background (CMB)
and large-scale structures.
The simplest single-field inflation paradigm fits, in a sense, observations too well, so that it is difficult to single out the correct field theory model of inflation.
In this context, much work has been done in search for anomalies in observations.
One possibility among them is statistical anisotropy of spectrum of primordial perturbations.
If established, we need an inflation model to realize anisotropic expansion.
In this regard it has been known that inflation driven by a scalar potential make the observable universe fully isotropic \cite{Jensen:1986vy,Turner:1986gj}.
Hence previous models of potential-driven anisotropic inflation inevitably include a vector field \cite{Ford:1989me,Watanabe:2009ct,Do:2017qyd,Adshead:2018emn}.

Here we wish to show that these observations are true only in the Einstein gravity, and that a class of modified gravity with a scalar field can
realize anisotropic (inflationary) solution without introducing any vector fields
nor higher-order curvature terms as an effective anisotropic stress source \cite{Barrow:2005qv,Barrow:2006xb,Barrow:2009gx}.
Specifically we consider Horndeski theory \cite{Horndeski:1974wa} or the generalized Galileon \cite{Deffayet:2011gz} to show that
the quintic galileon term $\mathcal{L}_5$ in generalized G-inflation \cite{Kobayashi:2011nu} plays an essential role to realize anisotropic inflation.
Such terms are known to emerge after Kaluza-Klein compactification of higher-dimensional Lovelock gravity \cite{VanAcoleyen:2011mj} on one hand
but its magnitude has been severely constrained \cite{Baker:2017hug,Creminelli:2017sry,Sakstein:2017xjx,Ezquiaga:2017ekz}
now on the other hand by the simultaneous discovery of the gravitational-wave event GW170817 \cite{TheLIGOScientific:2017qsa}
of binary neutron star coalescence and the associated gamma-ray burst GRB170817A \cite{Goldstein:2017mmi},
which shows that the relative deviation of the speed of gravitational waves from light velocity is at most $10^{-15}$.
Such an observational constraint on $\mathcal{L}_5$, however, applies only in the low-redshift universe,
and it may well evolve nontrivially in the high energy regime in the early universe.

The present paper is organized as follows.
In Section \ref{Horndeskitheory}, we review the covariant form and ADM form of Horndeski theory and its equation of evolution.
In Section \ref{newsolution}, by using the trace-free part of the equation of evolution, we show there are emerged solutions which describe
expanding universes with nonvanishing anisotropies.
In Section \ref{Bianchimodel}, we apply the solution to the Bianchi-I model without matter,
which is the simplest homogeneous anisotropic model.
In Section \ref{discussion}, we disscuss the nature of the solution and cosmological application,
and we conclude in Section \ref{conclusion}.



\section{Horndeski theory and beyond}\label{Horndeskitheory}
The
Horndeski theory \cite{Horndeski:1974wa} describes the most general couplings between a scalar field $\phi$
and the metric $g_{\mu\nu}$
which yield second-order field equations.
It was rediscovered in \cite{Deffayet:2011gz} in the context of generalized Galileons
and their equivalence was proved in \cite{Kobayashi:2011nu}.

This theory is characarized by four arbitrary functions, $G_2$, $G_3$, $G_4$ and $G_5$,
of $\phi$ and its canonical kinetic function $X\equiv-\partial_\mu\phi \partial^\mu\phi/2$ as
\begin{eqnarray}
S&=&\int \mathrm{d}^4x \sqrt{-g} \sum_{i=2}^{5} \tilde{\mathcal{L}}_i, \label{Horndeski}
\\
\tilde{\mathcal{L}}_2&=&G_2(\phi,X),\\
\tilde{\mathcal{L}}_3&=&-G_3(\phi,X)\Box\phi,\\
\tilde{\mathcal{L}}_4&=&G_4(\phi,X){\cal R}
+G_{4X}[(\Box\phi)^2-(\nabla_\mu \nabla_\nu \phi)^2],\\
\tilde{\mathcal{L}}_5&=&G_5(\phi,X){\cal G}_{\mu\nu}\nabla^\mu \nabla^\nu \phi
-\frac{1}{6}G_{5X}[(\Box\phi)^3-3(\Box\phi)(\nabla_\mu \nabla_\nu \phi)^2
+2(\nabla_\mu \nabla_\nu \phi)^3],
\end{eqnarray}
where ${\cal R}$
is the Ricci scalar, ${\cal G}_{\mu\nu}$ is the Einstein tensor,
$(\nabla_\mu\nabla_\nu\phi)^2=\nabla_\mu\nabla_\nu\phi\nabla^\mu\nabla^\nu\phi$,
$(\nabla_\mu\nabla_\nu\phi)^3=\nabla_\mu\nabla_\nu\phi\nabla^\nu\nabla^\lambda%
\phi\nabla_\lambda\nabla^\mu\phi$,
and $G_{iX}=\partial G_i/\partial X$.

The action descibed by the ADM variables
is more useful to study anisotropic cosmological solutions
than the covariant form \eqref{Horndeski}.
The metric is given by
\begin{eqnarray}
ds^2 = -N^2 dt^2 + g_{ij} (dx^i+N^i dt) (dx^j+N^j dt) .
\end{eqnarray}
We take the unitary gauge, $\phi=\phi(t)$, and
then $X$ is given by $X=\dot\phi^2/2N^2$ with $N$ being the lapse function.
If $\phi$ is a monotonic function of $t$,
this is a very convenient gauge and we can use $(t, N)$
instead of $(\phi,X)$ to express the action.
Then, the theory is described only in terms of
$t$ and
geometrical quantities as
\begin{eqnarray}
S&=&\int \mathrm{d}t \mathrm{d}^3x N \sqrt{g} \sum_{i=2}^{5} \mathcal{L}_i ,\label{ADM}\\
\mathcal{L}_2&=&A_2(t,N),\label{admL2}\\
\mathcal{L}_3&=&A_3(t,N){K},\\
\mathcal{L}_4&=&A_4(t,N)\left({K}^2-{K}^i_j{K}^j_i\right)+B_4(t,N){R},\\
\mathcal{L}_5&=&A_5(t,N)\left({K}^3-3{K}{K}^i_j{K}^j_i+2{K}^i_j{K}^j_k{K}^k_i\right)
+B_5(t,N)\left({R}_{ij}-\frac{1}{2}g_{ij}{R}\right){K}^{ij} .\label{admL5}
\end{eqnarray}
$K^i_j$ and $R_{ij}$ are the extrinsic and intrinsic curvature
of constant $t$ (constant $\phi$) hypersurfaces.
The functions $A_i, B_i$ and $G_i$
are related with each other as follows:
\begin{eqnarray}
A_2(t,N) &=& G_2(\phi,X) - \sqrt{X} \int \frac{G_{3\phi}(\phi,X)}{\sqrt{X}} dX, \\
A_3(t,N) &=& \int \sqrt{2X} G_{3X}(\phi,X)  dX - 2 \sqrt{2X} G_{4\phi}(\phi,X), \\
A_4(t,N) &=& - G_4(\phi,X)  + 2X G_{4X}(\phi,X)  - XG_{5\phi}(\phi,X) ,\label{defA4} \\
A_5(t,N) &=& \frac{1}{6} (2X)^{3/2} G_{5X}(\phi,X) , \\
B_4(t,N) &=& G_4(\phi,X)  - \frac{\sqrt{X}}{2} \int \frac{G_{5\phi}(\phi,X) }{\sqrt{X}} dX, \\
B_5(t,N) &=& - \int \sqrt{2X} G_{5X}(\phi,X)  dX\label{defB5},
\end{eqnarray}
where we identify $X=\dot\phi^2(t)/2N^2$.
As seen below, among those terms the most crucial ones in this paper are
the terms cubic in the extrinsic curvature.
In the covariant language they come from $\tilde{{\cal L}}_5$
which depends cubically on the second derivatives of the scalar field.




In the Horndeski theory, $(A_4,A_5)$
and $(B_4,B_5)$ are not independent,
as is clear from
Eqs. \eqref{defA4}--\eqref{defB5} and also from
the fact that we originally have four free functions in the action.
However, this point turns out to be not essential in the following discussion.
The most important ingredient here is the cubic (or higher) order terms
in the extrinsic curvature.
This allows us to start from the ADM Lagrangians \eqref{admL2}--\eqref{admL5}
and consider all $A_i$'s and $B_i$'s to be independent free functions,
which amounts to employing the so-called ``beyond Horndeski'' theory~\cite{Gleyzes:2014dya}, 
although $A_i$'s and $B_i$'s may have to satisfy degeneracy conditions to avoid an extra dangerous degree of freedom 
\cite{Langlois:2015cwa,Crisostomi:2016tcp} (see, however, \cite{DeFelice:2018ewo}).
The following discussion can thus apply not only to the Horndeski theory
but also to beyond Horndeski theory.



In addition to the action for
the gravitational sector described above,
we include the action for matter minimally coupled to gravity, $S_{\rm m}$.
By the use of the residual gauge degrees of freedom one can
further impose $N^i=0$.
Then, we obtain the evolution equations from \eqref{ADM} as
\begin{eqnarray}
T^i_j=
&&\frac{1}{ N\sqrt{g} } \partial_t
\left\{
\sqrt{g}
\left\{
A_3 \delta^i_j
+ 2 A_4(K\delta^i_j - K^i_j)
+ 3 A_5 [(K^2-K^k_l K^l_k)\delta^i_j - 2 (KK^i_j - K^i_k K^k_j)]
\right\}
\right\}
\nonumber\\
&&
- \delta^i_j \mathcal{L}_A
+ \left(2 B_4 + \frac{ \partial_t B_5}{N} \right)\left(
 R^i_j-\frac{1}{2}\delta_i^jR\right)
+\Phi^i_j ,\label{Einstein}
\end{eqnarray}
where $T_{ij}$ is the stress-energy tensor calculated from the matter action $S_\mathrm{m}$,
\begin{eqnarray}
T_{ij} = - \frac{2}{N\sqrt{g}} \frac{\delta S_\mathrm{m}}{\delta g^{ij}} ,
\end{eqnarray}
and
$\mathcal{L}_A$ is the kinetic part of
the Lagrangian,
\begin{eqnarray}
\mathcal{L}_A = A_2 + A_3 K + A_4 (K^2-K^i_j K^j_i) + A_5 (K^3 - 3 K K^i_j K^j_i + 2K^i_j K^j_k K^k_i) .
\end{eqnarray}
We have collected the terms that vanish
if the lapse function is homogeneous, $N(t,\Vec{x})=N(t)$,
and written
\begin{eqnarray}
\Phi_{ij}
&=&
\frac{2}{N}[\nabla^2(NB_4)g_{ij} - \nabla_i \nabla_j (NB_4)]
\nonumber\\
&&+
g_{ij} K^{lm} \nabla_l \nabla_m B_5
+ K \nabla_k \nabla_j B_5
- 2 K^l_{(i} \nabla_{j)} \nabla_l B_5
+ K_{ij} \nabla^2 B_5
- g_{ij} K\nabla^2B_5
\nonumber\\
&&
+ \frac{2}{N}  \left[
g_{ij}\nabla_l (NK^{lm})\nabla_m B_5
+ \nabla_{(i}(NK)\nabla_{j)}B_5
- \nabla_l (NK^l_{(i}) \nabla_{j)} B_5
\right.
\nonumber\\
&&~~~~~~~~~~~~
\left.
- \nabla_{(i} (NK^l_{j)}) \nabla_l B_5
+ \nabla_l (NK_{ij}) \nabla^l B_5
- g_{ij} \nabla_l (NK) \nabla^l B_5
\right] .
\end{eqnarray}
The Hamiltonian constraint is given by
\begin{eqnarray}
&&\partial_N (NA_2) + N \partial_N A_3 K + N^2 \partial_N (N^{-1} A_4)(K^2 - K^i_j K^j_i) + \partial_N (NB_4) R \nonumber\\
&&+ N^3 \partial_N (N^{-2}A_5)(K^3 - 3 K K^i_j K^j_i + 2K^i_j K^j_k K^k_i) + N \partial_N B_5 \left(R_{ij} K^{ij}-\frac{1}{2}RK\right)
 + \frac{1}{\sqrt{g}}\frac{\delta S_\mathrm{m} }{\delta N} = 0.
 \nonumber\\
\label{Hamiltonian}
\end{eqnarray}
In the following we will not use the momentum constraint equations.



\section{Self-anisotropizing inflationary solutions}\label{newsolution}

We now show that even without any anisotropic matter sources
the universe can exhibit anisotropic inflationary expansion as an atractor solution
in the Horndeski theory.

Since we consider Bianchi cosmology, we may set $N^i=0$.
Thanks to the homogeneity, $\Phi_{ij}$ in the evolution
equation \eqref{Einstein} vanishes.
To study anisotropic cosmological models it is convenient to
decompose the extrinsic curvature $K_{ij}$ into its trace $K$ and trace-free part $\Sigma_{ij}$
as
\begin{eqnarray}
K_{ij} = \frac{1}{3} K g_{ij} + \Sigma_{ij} ,
\end{eqnarray}
with $g^{ij} \Sigma_{ij}=0$.
The trace and trace-free parts of the evolution equation \eqref{Einstein} read,
respectively,
\begin{eqnarray}
\frac{1}{ N\sqrt{g} } \partial_t
[
\sqrt{g}
(
3 A_3 + 4 A_4 K + A_5 (2 K^2- 3\Sigma^i_j\Sigma^j_i )
)
]
- 3 \mathcal{L}_A
- \left( B_4 + \frac{\partial_t B_5}{2N} \right) R
=T^i_i . \label{tracepart}
\end{eqnarray}
and
\begin{eqnarray}
\frac{2}{N\sqrt{g}} \partial_t
\left[
\sqrt{g} (-A_4 \Sigma^i_j - A_5 K \Sigma^i_j + 3 A_5 \{ \Sigma^i_k \Sigma^k_j \}_\mathrm{TF})
\right]
+
\left(2 B_4 + \frac{\partial_t B_5}{N} \right) \{ R^i_j \}_\mathrm{TF}
=
\{ T^i_j \}_\mathrm{TF} ,
\nonumber\\
 \label{tracefree}
\end{eqnarray}
where $\{ X^i_j \}_\mathrm{TF}$ stands for the trace-free part of a tensor $X^i_j$,
\begin{eqnarray}
\{ X^i_j \}_\mathrm{TF} = X^i_j - \frac{1}{3} X^k_k \delta^i_j .
\end{eqnarray}


Let us look for slow-roll inflationary solutions
in which $\sqrt{g}$ exponentially increases,
while other functions remain either nearly constant or exponentially decrease.
First,
we focus on Eq. \eqref{tracefree}, assuming that
 the energy-momentum tensor consists of isotropic matter
and hence $\{ T^i_j \}_\mathrm{TF}$ vanishes.
If the spatial curvature
$R^i_j$ decreases exponentially,
the first term also decreases in the same way.
As a result, we find, asymptotically,
\begin{eqnarray}
-A_4 \Sigma^i_j - A_5 K \Sigma^i_j + 3 A_5 ( \Sigma^i_k \Sigma^k_j - \frac{1}{3}\Sigma^k_l \Sigma^l_k \delta^i_j) = 0 .
\label{attractor}
\end{eqnarray}
A trivial solution of Eq.~\eqref{attractor} is that all components of $\Sigma^i_j$ vanish.
This solution corresponds to the isotropic attractor
which we see in the conventional inflation models.
The presence of the quadratic terms
in $\Sigma^i_j$ due to nonvanishing $A_5$ yields nontrivial solutions with
$\Sigma^i_j\neq 0$ as well,
which represent an expanding universe retaining
finite anisotropies.
We dub this anisotropic attractors as {\em self-anisotropizing} inflationary solutions,
as this is {\em not} caused by an anisotropic energy-momentum tensor.
We will demonstrate in the next section that such solutions do exist
in the case of Bianchi type-I cosmology.


The self-anisotropizing attractors are distinct from the previous
anisotropic inflationary solutions, because the anisotropic expansion of
the previous scenarios are supported by some anisotropic energy-momentum source
such as a vector field coupled with an inflaton field \cite{Watanabe:2009ct}.
Such scenarios produce background anisotropies
$\Sigma^i_j/H \approx \{T^i_j \}_\mathrm{TF}/(6 A_4 H^2) = (8\pi G/3H^2) \{T^i_j \}_\mathrm{TF} $,
where $H$ is the Hubble parameter.
The trace-free part of the energy-momentum tensor, $\{T^i_j \}_\mathrm{TF}$,
just displaces the terminal point from the isotropic one.




By contrast, here
the self-anisotropizing inflationary solution
is realized by the terms quadratic in $\Sigma^i_j$ in Eq. \eqref{attractor},
which is a consequence of modification of gravity.
The magnitude of produced background anisotropies is
estimated from \eqref{attractor}
as $\Sigma^i_j/H\sim (A_4 + 3 H A_5)/3 H A_5$.
We require neither an anisotropic energy-momentum tensor
nor any fields other than the scalar $\phi$ built in the Horndeski theory.
In this sense, the emerged anisotropic terminal points should be distinguished from
those of previous anisotropic inflation models.


Let us evaluate the eigenvalues of the nontrivial solutions of $\Sigma^i_j$ for
given values of $A_4,~A_5$ and $K$.
We can prove that the root $\Sigma$ of matrix equation \eqref{attractor}
has two different eigenvalues at most as follows.
First we define a polynomial $p(x)$ by substituting a real variable $x$ for $\Sigma$ in the left side of \eqref{attractor} as
\begin{eqnarray}
p(x) = -A_4 x - A_5 K x + 3 A_5 \left( x^2 -\frac{1}{3} \mathrm{tr}\!\left(\Sigma^2\right) \right) , \label{minimal}
\end{eqnarray}
where the remaining $\Sigma$ in the trace is a root of \eqref{attractor}.
$p(\Sigma)=0$ obviously follows from \eqref{attractor} and \eqref{minimal},
and so $p(x)$ can be divided by the minimal polynomial $\phi_\Sigma(x)$ of $\Sigma$.
In linear algebra, it is well-known that if $\lambda$ is an eigenvalue of matrix $\Sigma$ then $\lambda$ is a root of $\phi_\Sigma(x)=0$.
Therefore, the eigenvalue $\lambda$ is also a root of $p(x)=0$.
Since $p(x)$ is a quadratic polynomial of $x$, the number of different roots is equal to or less than two.
This is the proof that $\Sigma$ has two different eigenvalues, $\lambda_1$ and $\lambda_2$ at most.
It induces that, \textit{e.g.}, anisotropic attractors in Bianchi type-I model has axial symmetry in the order of background,
which we show in Section \ref{Bianchimodel}.
As one can see from \eqref{minimal}, the different eigenvalues $\lambda_1$ and $\lambda_2$ satisfy
\begin{eqnarray}
\lambda_1 + \lambda_2
=
\frac{A_4+A_5 K}{3A_5}.
\end{eqnarray}
Being a three dimensional tensor, $\Sigma$ has three eigenvalues.
Without loss of generality, we set them as $\lambda_1$, $\lambda_1$ and $\lambda_2$, respectively. They also satisfy
\begin{eqnarray}
2 \lambda_1 + \lambda_2 = 0,
\end{eqnarray}
because $\Sigma$ is trace-free. Therefore we have
\begin{eqnarray}
\lambda_1 = -\frac{A_4+A_5K}{3A_5}
,~~~~
\lambda_2 = \frac{2(A_4+A_5K)}{3A_5}.
\label{eigenvalues}
\end{eqnarray}

So far we have focused on the evolution equation for $\Sigma^i_j$ \eqref{tracefree}
and its nontrivial solution under the assumption that
the spatial volume element $\sqrt{g}$ increases exponentially
and the spatial curvature $R^i_j$ decreases accordingly.
To determine all the components of the metric, we need to
solve the Hamiltonian constraint \eqref{Hamiltonian} and
the trace part of the evolution equations \eqref{tracepart} consistently.
On the anisotropic attractor where
$\Sigma^i_j$'s eigenvalues are given by \eqref{eigenvalues},
the rest of the field equations
\eqref{Hamiltonian} and \eqref{tracepart} reduce to
\begin{align}
\partial_N \left[ N \left( A_2 - \frac{2 A_4^3}{9 A_5^2} \right) \right]
+ N \partial_N \left( A_3 -\frac{2 A_4^2}{3 A_5} \right) K
&=
- \frac{1}{\sqrt{g}}\frac{\delta S_\mathrm{m} }{\delta N}
\label{eq:at1}
,
\\
\frac{1}{ N } \frac{d}{dt}
\left( A_3 -\frac{2 A_4^2}{3 A_5} \right)
- \left( A_2 - \frac{2 A_4^3}{9 A_5^2} \right)
&=\frac{1}{3} T^i_i,\label{eq:at2}
\end{align}
respectively.
These two equations can be used to determine $K=K(t)$ and $N=N(t)$.


Let us ignore the matter field $S_{\rm m}$ for the moment
and consider a theory with (approximate) shift symmetry.
In this case, $A_i$'s depend only on $\dot\phi/N$
and from Eq. \eqref{eq:at2}
one obtains a solution $N\simeq {\rm const}\times \dot\phi$
satisfying $F(\dot\phi/N)\equiv A_2- 2A_4^3/9A_5^2\simeq 0$. Equation \eqref{eq:at1} is then
solved to give $K\simeq -\partial_N(A_3-2A_4^2/3A_5)/\partial_N(A_2-2A_4^3/9A_5^2)%
\simeq\;$const. One thus obtains an inflating solution with nonvanishing anisotropies.



\section{Vacuum Bianchi type-I model}\label{Bianchimodel}
\subsection{Evolution toward attractors}
To be more explicit, let us consider
the Bianchi type-I model, which is the simplest homogeneous anisotropic model
and hence helps us to understand what nonvanishing $A_5$ causes.



Once we diagonalize the spatial metric and its time derivative, off-diagonal components are not generated in this model,
so that we can express the metric in the Kasner-type form as
\begin{eqnarray}
ds^2&=&-N^2(t)dt^2
+a^2(t)
\left[
e^{2(\beta_+(t)+\sqrt{3}\beta_-(t))}dx^2 +
e^{2(\beta_+(t)-\sqrt{3}\beta_-(t))}dy^2 +
e^{-4\beta_+(t)}dz^2
\right] ,
\label{metric}
\end{eqnarray}
where $a(t)$ is a scale factor and $\beta_\pm(t)$ show the
differences between the expantion rates in different directions.
Substituting the metric \eqref{metric} in the ADM form of the action \eqref{ADM},
we obtain
\begin{eqnarray}
 S &=&
 \int dt d^3x~
N a^{3}
 \left[
 A_2 + 3 H A_3 + 6 A_4 (H^2-\sigma_+^2-\sigma_-^2)
 \right.
 \left.
 + 6A_5(H^3-3H(\sigma_+^2+\sigma_-^2)+2(3\sigma_+\sigma_-^2 - \sigma_+^3))
 \right],
 \nonumber\\
 \label{action0}
\end{eqnarray}
where we defined the Hubble parameter $H$ and the shear $\sigma_\pm$ as
\begin{eqnarray}
H \equiv \frac{1}{N}\frac{d\ln a}{dt} , ~~~~
\sigma_\pm \equiv \frac{1}{N}\frac{d\beta_\pm}{dt}.
\end{eqnarray}
Using $\sigma_\pm$, the trace-free part of the extrinsic curvature is given by
\begin{eqnarray}
\Sigma^i_j = \mathrm{diag} \left( \sigma_+ + \sqrt{3} \sigma_- , \sigma_+ - \sqrt{3} \sigma_- , -2 \sigma_+ \right).
\label{Sigma}
\end{eqnarray}
Since the spatial Ricci tensor vanishes in the Bianchi type-I model
and consequently Eq. \eqref{action0}
depends on $\beta_\pm$ only through their time derivatives,
the conjugate momenta of $\beta_\pm$ are conserved in time.
The conserved momenta are given by
\begin{eqnarray}
P_{\beta_+}
&=&
a^3
\left[
(A_4+3HA_5)\sigma_+
+3A_5(\sigma_+^2-\sigma_-^2)
\right],\label{Pp}
 \\
P_{\beta_-}
&=&
a^3
\left[
(A_4+3HA_5)\sigma_-
-6A_5\sigma_+ \sigma_-
\right] .%,
\label{Pm}
\end{eqnarray}
Equivalently,
one can also obtain the same conserved momenta
from the trace-free part of the evolution equations \eqref{tracefree}
by substituting Eq. \eqref{metric}.
It is manifest that as the scale factor $a(t)$ increases,
the expressions inside the square brackets of Eqs. \eqref{Pp} and \eqref{Pm}
decay toward zero as $[\cdots] = P_{\beta_\pm}a^{-3} \to 0$, and
thus $\sigma_+$ and $\sigma_-$ evolve to one of the fixed points.
In the present case, there are four fixed points.
One is the isotropic solution $\sigma_\pm=0$,
whereas the other three are anisotropic attractors.


Let us look at the trajectories on the $(\sigma_+,\sigma_-)$ plane of
the phase space.
Given the initial data, the constants $P_{\beta_\pm}$ are fixed.
Then, $\sigma_\pm$ can be expressed in terms of $A_4$, $A_5$, $a$, $H$, and $P_{\beta_\pm}$
by
solving the algebraic equations \eqref{Pp} and \eqref{Pm}.
In order to show the dynamics of the anisotropies in a single figure,
we use the normalized shear ${\cal A}_\pm$ defined as
\begin{eqnarray}
\mathcal{A}_\pm \equiv \frac{ 3A_5 }{ A_4 + 3HA_5 }  \sigma_\pm ,
\end{eqnarray}
instead of $\sigma_+$ and $\sigma_-$. Here we
assumed that $A_4+3HA_5\ne 0$ and $A_5\ne 0$.
It is also convenient to introduce the new time coordinate
$\tau \equiv a^3 (A_4 + 3HA_5)^2 / 3A_5$.
In an expanding universe, $|\tau|$ is an increasing function of $t$
provided that $A_4$, $A_5$, and $H$ depend on $t$ only weakly,
which is a natural assumption during inflation.
With $\tau$ and $\mathcal{A}_\pm$, we can rewrite Eqs. \eqref{Pp} and \eqref{Pm}
simply as
\begin{eqnarray}
P_{\beta_+}
&=&
\tau
\left[
\mathcal{A}_+
+\mathcal{A}_+^2-\mathcal{A}_-^2
\right],\label{nPp}
 \\
P_{\beta_-}
&=&
\tau
\left[
\mathcal{A}_-
-2\mathcal{A}_+ \mathcal{A}_-
\right] . \label{nPm}
\end{eqnarray}
We show trajectories $({\cal A}_+(\tau), {\cal A}_-(\tau))$
for different values of $P_{\beta_\pm}$ in Figure \ref{attractors}.
As stated above,
there are four fixed points in the $(\mathcal{A}_+,\mathcal{A}_-)$ plane:
one isotropic solution, $(0,0)$, and three anisotropic solutions,
$(-1,0)$ and $(1/2,\pm \sqrt{3}/2)$. All of them
are attractors (as long as $|\tau|$ is an increasing function of $t$).




\begin{figure}[htbp]
  \begin{center}
       \includegraphics[width=10cm]{attractors.eps}
       \caption{
       Trajectries of the evolution of the normalized shear $({\cal A}_+,{\cal A}_-)$.
       If the initial conditions lie inside the circle given by
       $\mathcal{A}_+^2+\mathcal{A}_-^2=1/4$,
       the universe evolves toward the center, $(\mathcal{A}_+,\mathcal{A}_-)=(0,0)$,
      as $\tau$ increases.
      If the universe starts from outside of the circle,
       it goes to the closest one of the anisotropic fixed points on the vertices,
       $(\mathcal{A}_+,\mathcal{A}_-)=(-1,0),(1/2,\pm \sqrt{3}/2)$, of the triangle as $\tau$ increases.
       }
       \label{attractors}
  \end{center}
\end{figure}



The initial anisotropies determine which attractor the universe approaches.
To see this explcitly, we
differentiate Eqs. \eqref{nPp} and \eqref{nPm} and
get
\begin{eqnarray}
\tau \frac{d \mathcal{A}_+}{d\tau}
=
- \frac{  (2\mathcal{A}_+ - 1)
(\mathcal{A}_+^2+\mathcal{A}_-^2+\mathcal{A}_+)}{4\mathcal{A}_+^2 + 4\mathcal{A}_-^2 - 1 } ,\label{eq:dap}
\\
\tau \frac{d \mathcal{A}_-}{d\tau}
=
- \frac{ \mathcal{A}_- (2\mathcal{A}_+^2 + 2\mathcal{A}_-^2 - 2\mathcal{A}_+ - 1)}{ 4\mathcal{A}_+^2 + 4\mathcal{A}_-^2 - 1 } .\label{eq:dam}
\end{eqnarray}
Equivalently, one may introduce
the polar coordinates $(r(\tau),\theta(\tau))$ defined by ${\cal A_+}=r\cos\theta$
and ${\cal A}_-=r\sin\theta$ and write
\begin{align}
\tau \frac{dr}{d\tau}&=-\frac{r[2r^2+r\cos(3\theta)-1]}{4r^2-1},
\label{eq:dr}
\\
\tau\frac{d\theta}{d\tau}&=\frac{r\sin(3\theta)}{4r^2-1}.\label{eq:dtht}
\end{align}
The denominators vanish on a circle given by
$r^2={\cal A}_+^2+{\cal A}_-^2=(1/2)^2$
(the black circle in Fig.~\ref{attractors}).\footnote{The shear evolution
equations bocome singular on this circle. However,
if we consider the full phase space by
taking into account the trace part of the evolution equation,
we see that this singularity is only apparent.}
The fate of the universe depends on whether the
initial anisotropies are inside this circle or not:
the universe is attracted toward the isotropic solution
at the origin if the initial anisotropies lie inside the circle,
while it goes away from the circle to
the closest one of the anisotorpic attractors
if outside initially.
That is to say, if the universe is sufficiently anisotropic initially,
then it converges to the anisotropic attractor.


The exceptional case is the trajectories with $\theta=0, 2\pi/3, 4\pi/3$.
Those constant values of $\theta$ solve Eq. \eqref{eq:dtht},
while Eq. \eqref{eq:dr} leads to
$r(\tau)=(\sqrt{C/|\tau|+1}-1)/2$, where $C$ is an integration constant.
Therefore, for all initial conditions on $\theta=0, 2\pi/3, 4\pi/3$
the isotropic universe is the attractor.




The structure of Fig.~\ref{attractors} will be more transparent
in terms of the polar coordinates.
Equations \eqref{eq:dr} and \eqref{eq:dtht}
clearly show that there are discrete rotation symmetry $\theta\to \theta + 2\pi/3$
and reflection symmetry across $\theta=0, 2\pi/3$, and $4\pi/3$ axises.
Because of these symmetries only a sixth part of Fig.~\ref{attractors}
is physically independent.






Each of the anisotropic attractors corresponds to an axially symmetric space,
whose symmetry axis is the $x$, $y$ or $z$ axis.
This axial symmetry is closely related to the degeneracy of the eigenvalues of $\Sigma_i^j$
discussed in the previous section.
The discrete rotation symmetry in the $(\mathcal{A}_+,\mathcal{A}_-)$
plane is the manifestation of the fact that
one can always take, say, the $z$ axis as the symmetry axis
without loss of generality by a rotation of the spatial coordinates.


So far we have focused only on the shear evolution equations.
This is sufficient for the purpose of seeing that
the anisotropic fixed points do exist and
for initial anisotropies larger than a certain threshold
they are indeed the attractors.
To determine the precise dynamics of the universe
including the evolution of $H$ and $\phi$, one needs to
solve the full set of the field equations
(the trace and trace-free parts of the evolution equations
as well as the constraint equation) consistently.
In the next subsection we will show a
numerical example obtained by
solving all the equations consistently.




\subsection{Examples}

Let us present some examples which yield self-anisotropizing
Bianchi type-I solutions. The first one is simply given by
\begin{align}
G_2=-V_0,\quad G_3=0,
\quad G_4=\frac{M^2}{2}+g_4X,\quad G_5=g_5X,
\end{align}
where $V_0$, $M$, $g_4$, and $g_5$ are constants.
The corresponding ADM form in the unitary gauge is given by
\begin{align}
&A_2=-V_0,\quad A_3=0,\quad
A_4=-\frac{M^2}{2}+\frac{g_4}{2}\left(\frac{\dot\phi}{N}\right)^2,
\quad A_5=\frac{g_5}{6}\left(\frac{\dot\phi}{N}\right)^3,
\notag \\ &
\quad B_4=\frac{M^2}{2}+\frac{g_4}{2}\left(\frac{\dot\phi}{N}\right)^2,
\quad B_5=-\frac{g_5}{3}\left(\frac{\dot\phi}{N}\right)^3.
\end{align}



Figure~\ref{fig:ex1} shows the evolution of the Hubble parameter,
(the velocity of) the scalar field, and the shear
obtained by solving the dynamical and constraint equations numerically
with a certain initial condition away from the attractors at $a=1$.
The parameters in this toy example are given by
$V_0=0.1$, $M=1$, $g_4=-0.2$, and $g_5=1$.
It can be seen that the universe quicly converges to
the anisoropic inflationary attractor.


\begin{figure}[htbp]
  \begin{center}
       \includegraphics[width=10cm]{ex1.eps}
       \caption{
       Numerical example of a self-anisotrpizing Bianchi type-I universe:
       (a) $H$; (b) $\dot\phi/N$; (c) $\sigma_\pm/H$ as functions of $\ln a$.}
       \label{fig:ex1}
  \end{center}
\end{figure}



Another example with $A_5$ (or, equivalently, $G_{5X}$)
is the Gauss-Bonnet term coupled to a scalar field,
and the total Lagrangian is of the form
\begin{align}
{\cal L}=f(\phi){\cal R}+P(\phi, X)+\xi(\phi)
\left({\cal R}^2-4{\cal R}_{\mu\nu}{\cal R}^{\mu\nu}
+{\cal R}_{\mu\nu\rho\sigma}{\cal R}^{\mu\nu\rho\sigma}\right).
\label{eq:LagGB}
\end{align}
Aspects of
this theory has been studied extensively in the literature.
The Lagrangian can be reproduced by taking the following
Horndeski functions \cite{Kobayashi:2011nu}:
\begin{align}
&
G_2=P+8\xi^{(4)}X^2(3-\ln X),\quad G_3=4\xi^{(3)}X(7-3\ln X),
\notag \\
&G_4=f+4\xi^{(2)}X(2-\ln X),
\quad
G_5=-4\xi^{(1)}\ln X,
\end{align}
where $\xi^{(n)}=d^n\xi/d\phi^n$. Though this looks quite non-trivial,
the corresponding ADM form is very simple:
\begin{align}
A_2=P,\quad  A_3=-2\frac{\dot\phi}{N}\frac{d f}{d\phi},
\quad A_4=-f,\quad A_5=-\frac{4\xi^{(1)}}{3}\frac{\dot\phi}{N},
\quad B_4=f,\quad B_5=8\xi^{(1)}\frac{\dot\phi}{N}.
\end{align}
Even this familiar theory admits self-anisotropizing inflationary solutions.


The theory \eqref{eq:LagGB} possesses a shift symmetry
if $f=\;$const, $P=P(X)$, and $\xi\propto \phi$.
In this case it is easy to find an inflationary solution
with $H=\;$const, $\dot\phi/N=\;$const retaining
the nonvanishing shear
\begin{align}
\frac{\sigma_\pm}{H}\sim \frac{f+4H\xi^{(1)}\dot\phi/N}{H\xi^{(1)}\dot\phi/N}.
\end{align}




\section{Discussion}\label{discussion}

It has been pointed out by Wald that in general relativity, all vacuum Bianchi universes with a positive cosmological constant except type IX evolve toward the isotropic attractor,
which was proven by using the Hamiltonian constraint and the trace of
the Einstein equations \cite{Wald:1983ky}.
In our case, since the Horndeski action dramatically changes both of them,
it must be checked one by one whether a specific model under consideration
evolves toward the isotropic or anisotropic attractor.
We note that the magnitude of
the shear on the anisotropic attractors
diverges when we take the general relativity limit $A_5 \to 0$ keeping $A_4$ constant.
In this limit,
for all initial conditions the isotropic universe is an attractor
(as they are all inside the circle in Fig.~\ref{attractors}),
and thus the standard result of Wald in gerenal relativity is recovered.


Noting that
the background anisotropies of the Bianchi type-I universe can be regarded
as gravitational waves with infinitely long wavelengths,
we point out that
the emergence of anisotropic attractors
is closely related to
the three-point coupling of gravitational waves in the Horndeski theory.
From Eq. (15) of \cite{Gao:2011vs}, one sees that
there are two types of the three-point couplings
of the form $hh\partial^2h$ and $\dot h\dot h\dot h$,
giving rise to local and equilateral non-Gaussianity, respectively.
The former appears even in general relativity as well as in a generic scalar-tensor theory,
while the latter, which obviously comes from $K_{ij}^3$, emerges
only in the class with $A_5\ne 0$ (i.e., $G_{5X}\neq 0$).
The former has spatial derivatives and therefore vanishes in the long-wavelength limit,
whereas the latter has only time derivatives and hence does not vanish
even in the homogeneous limit.


Since ${\cal A_{\pm}}={\cal O}(1)$ on the anisotropic attractors,
the magnitude of the resultant anisotropy is given by
\begin{align}
\frac{\sigma_\pm}{H}\sim \frac{A_4+3HA_5}{3HA_5},
\end{align}
which is typically of ${\cal O}(1)$ or larger.
In theories with $G_{5X}\neq 0$, initial anisotropies must be
smaller than this value in order to realize an isotropic universe
through inflation. Otherwise, the resultant universe would be
unacceptably anisotropic.
Another possiblity is that one has
$A_4+3HA_5\ll 3HA_5$ via fine-tuning,
leaving an observationally viable universe with only tiny anisotropies
on the anisotropic attractor. This would be a very interesting scenario,
but one has to study reheating, cosmological perturbations, and
the stability in detail to see whether such a scenario is indeed viable or not,
which is beyond the scope of the present paper.




\section{Conclusions}\label{conclusion}
We have shown the quintic galileon term proportional $G_{5X}$ or $A_5$ in generalized G-inflation can realize
anisotropic inflationary solution.
On the anisotropic attractor, the Hamiltonian constraint becomes a linear equation for the Hubble parameter
which is strikingly different from the conventional Friedmann equation.

Although our solution generically produces anisotropy of the order of unity or larger,
it can also accommodate much smaller anisotropy
by partially canceling $A_4$ and $3HA_5$.
In order to see if observationally viable anisotropic inflation is possible,
we must calculate perturbations as well as discuss transition to the Friedmann regime with proper reheating,
which will be discussed in future publications.

It is also interesting to study higher dimentional models in this context to show a new compactification mechanism of
extra dimensions in the presence of the highest-order galileon terms in the dimension under consideration.
As we can show that the eigenvalues of the extrinsic curvature tensor take only two distinct values at most
even in higher dimensional models, this may provide a promissing mechanism of compactification or dimensional reduction,
which will also be discussed in a forthcoming paper.


\begin{acknowledgments}
HWHT was supported by the Advanced Leading Graduate Course for Photon Science (ALPS).
  The work of TK was supported by
  MEXT KAKENHI Grant Nos.~JP15H05888, JP16H01102, JP17H06359, JP16K17707,
  and MEXT-Supported Program for the Strategic Research Foundation at Private Universities,
  2014-2018 (S1411024).
The work of JY was supported by JSPS KAKENHI, Grant JP15H02082
and Grant on Innovative Areas JP15H05888.
\end{acknowledgments}

\bibliography{mypaper}

\end{document}
 % outcomment this line in Case 2

%If you don't use BiBTex, you can manually itemize references as shown below.


\bibliographystyle{plain}
\bibliography{$HOME/styles/bib/vkm}


\end{document}