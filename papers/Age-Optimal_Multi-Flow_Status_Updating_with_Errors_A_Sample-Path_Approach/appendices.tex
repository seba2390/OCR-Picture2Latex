% !TEX root = ./heterogeneous_servers.tex
\appendices
%\appendix

\section{Proof of Theorem \ref{thm1}}\label{app1}
Let the age vector $\bm\Delta_{\pi}(t)$ represent the \emph{system state} of policy $\pi$ at time $t$ and $\{\bm\Delta_{\pi}(t),t\in [0,\infty)\}$ be the \emph{state process} of policy $\pi$. For notational simplicity, let policy $P$ represent the P-MAF-LGFS policy, which is a work-conserving policy. We first establish two lemmas that are useful to prove Theorem \ref{thm1}. 
Using
the memoryless property of exponential distribution, we can
obtain the following coupling lemma:
\begin{lemma}\emph{(Coupling Lemma)}\label{coupling}
In continuous-time status up- dating systems, consider policy $P$ and any {work-conserving} policy $\pi\in \Pi$. For any given $\mathcal{I}$, if (i) the transmission errors are \emph{i.i.d.} with an error probability $q\in [0,1)$ and (ii)  the packet transmission times are exponentially distributed and \emph{i.i.d.} across  packets,
 then there exist policy $P_1$ and  policy $\pi_1$ in the same probability space which satisfy the same scheduling disciplines with policy $P$ and policy $\pi$, respectively,  such that 
\begin{itemize}
\itemsep0em 
\item[1.] the state process $\{\bm\Delta_{P_1}(t),t\in [0,\infty)\}$ of policy $P_1$ has the same distribution as the state process $\{\bm\Delta_{P}(t),t\in [0,\infty)\}$ of policy $P$,
\item[2.] the state process $\{\bm\Delta_{\pi_1}(t),t\in [0,\infty)\}$ of policy $\pi_1$ has the same distribution as the state process $\{\bm\Delta_{\pi}(t),t\in [0,\infty)\}$  of policy $\pi$,
\item[3.] if a packet from the flow with age $\Delta_{[i],P_1}(t)$ is successfully delivered at time $t$ in policy $P_1$, then almost surely, a packet from the flow with age $\Delta_{[i],\pi_1}(t)$ is successfully delivered at time $t$ in policy $\pi_1$; and vice versa.
%whenever there exist unassigned tasks in the queue,
\end{itemize} 
\end{lemma}
\ifreport
\begin{proof}
Notice that (i) all policies have identical packet generation and arrival times $\mathcal{I}$, (ii) the packet transmission times are  \emph{i.i.d.} memoryless, and (iii) policy $P$ and policy $\pi$ are both work-conserving. In addition, the packet generation/arrival times $\mathcal I$, the packet transmission times, and the transmission failures are governed by three mutually independent stochastic processes, none of which are influenced by the scheduling policy. Because of these facts, service preemption does not affect the distribution of packet delivery times. Following the inductive construction argument used in the proof of Theorem 6.B.3 in \cite{StochasticOrderBook}, one can construct the packet transmission success and failure events one by one in policy $P_1$ and policy $\pi_1$ to prove this lemma. 
In particular, since the transmission errors are \emph{i.i.d.} and they are not influenced by the scheduling policy, it is feasible to couple the packet transmission success and failure events in policy $P_1$ and policy $\pi_1$ in such a way that a packet from the flow with age $\Delta_{[i],P_1}(t)$ is successfully delivered at time $t$ in policy $P_1$ if, and only if, a packet from the flow with age $\Delta_{[i],\pi_1}(t)$ is successfully delivered at time $t$ in policy $\pi_1$. The details are omitted. 
\end{proof}
\else
\begin{proof}
See our technical report \cite{SunMultiFlow18}.
\end{proof}
\fi

We will compare policy $P_1$ and policy $\pi_1$ on a sample path by using the following lemma: 

\begin{lemma} \emph{(Inductive Comparison)}\label{lem2}
Suppose that a packet is delivered at time $t$ in policy $P_1$ and a packet is delivered at the same time $t$ in policy $\pi_1$. The system state  of policy $P_1$ is $\bm\Delta_{P_1}$ before the packet delivery, which becomes $\bm\Delta_{P_1}'$ after the packet delivery. The system state  of policy $\pi_1$ is $\bm\Delta_{\pi_1}$ before the packet delivery, which becomes $\bm\Delta_{\pi_1}'$ after the packet delivery. Under the conditions of Lemma \ref{coupling}, if (i) the packet generation and arrival times are {synchronized} across the $N$ flows and (ii) 
\begin{equation}\label{hyp1}
\Delta_{[i],P_1} \leq \Delta_{[i],\pi_1},~i=1,\ldots,N,
\end{equation}
then
\begin{equation}\label{law6}
\Delta_{[i],P_1}' \leq \Delta_{[i],\pi_1}',~i=1,\ldots,N.
\end{equation}  
\end{lemma}


\ifreport
\begin{proof}
For synchronized packet generations and arrivals, let $W(t) = \max_i\{S_i: A_i \leq t\}$ 
be the generation time of the freshest packet of each flow that has arrived at the queue by time $t$. At time $t$, because no packet that has arrived at the queue was generated later than $W(t)$, we can obtain
\begin{align}%\label{eq_proof_1}
%\Delta_{[i],P_1}'\geq t -W(t),~i=1,\ldots,N,\\
\Delta_{[i],\pi_1}' \geq t  -W(t),~i=1,\ldots,N.\label{eq_proof_2}
\end{align} 

Because (i) policy $P_1$ follows the same scheduling discipline with the P-MAF-LGFS policy and (ii) the packet generation and arrival times are {synchronized} across the $N$ flows, the delivered packet at time $t$ in policy $P_1$ must be the freshest packet generated at time $W(t)$. Hence, in policy $P_1$, the flow associated with the delivered packet must have the minimum age after the delivery, given by
\begin{align}
%\Delta_{[i],P_1}(t+T_s) &= \Delta_{[j_1],P_1},~i=1,\ldots,N-l,\\
\Delta_{[N],P_1}' &= t - W(t). \label{eq_proof_4}
\end{align}
Combining \eqref{eq_proof_2} and \eqref{eq_proof_4}, yields
\begin{align}
\Delta_{[N],P_1}' = t - W(t) \leq \Delta_{[N],\pi_1}'. \label{eq_proof_5}
\end{align}

Moreover, suppose that the packet delivered at time $t$ in policy $P_1$ is from the flow with age value $\Delta_{[j],P_1}$ before the packet delivery. This indicates 
\begin{align}
\Delta_{[i],P_1}' &= \Delta_{[i],P_1},~i=1,2,\ldots, j -1, \label{eq_proof_6}\\
\Delta_{[i],P_1}'& = \Delta_{[i+1],P_1},~i=j,2,\ldots, N-1. \label{eq_proof_7}
\end{align}
According to Lemma \ref{coupling}, the packet delivered at time $t$ in policy $\pi_1$ is from the flow with age value $\Delta_{[j],\pi_1}$ before the packet delivery. Hence, 
\begin{align}
\Delta_{[i],\pi_1}' &= \Delta_{[i],\pi_1},~i=1,2,\ldots, j -1, \label{eq_proof_8}\\
\Delta_{[i],\pi_1}'& \geq \Delta_{[i+1],\pi_1},~i=j,2,\ldots, N-1. \label{eq_proof_9}
\end{align}
Combining \eqref{hyp1}, \eqref{eq_proof_6}, and \eqref{eq_proof_8}, yields
\begin{align}
\Delta_{[i],P_1}' = \Delta_{[i],P_1} \leq \Delta_{[i],\pi_1} = \Delta_{[i],\pi_1}',~i=1,2,\ldots, j -1.\label{eq_proof_10}
\end{align}
Moreover, combining \eqref{hyp1}, \eqref{eq_proof_7}, and \eqref{eq_proof_9}, yields
\begin{align}
\Delta_{[i],P_1}' = \Delta_{[i+1],P_1} \leq \Delta_{[i+1],\pi_1} \leq \Delta_{[i],\pi_1}',\nonumber\\~i=j,2,\ldots, N-1.\label{eq_proof_11}
\end{align}
Finally, \eqref{law6} follows from \eqref{eq_proof_5}, \eqref{eq_proof_10}, and \eqref{eq_proof_11}. This completes the proof. 
\end{proof}
\else
\begin{proof}
See our technical report \cite{SunMultiFlow18}.
\end{proof}
\fi

%\begin{lemma}
%Consider two $N$-dimensional vectors $\bm{x}$ and $\bm{y}$. If $\bm{x}\leq\bm{y}$, then $x_{[i]} \leq y_{[i]}$ for all $i=1,\ldots,N$.
%\end{lemma}
%\begin{proof}
%For each $i=1,\ldots,N$, there exist $i$ elements $x_{[1]}, \ldots, x_{[i]}$ in $\bm{x}$ which are no smaller than $x_{[i]}
%$. This, together with $\bm{x}\leq\bm{y}$, tells us that at least $i$ elements in $\bm{y}$ are no smaller than $x_{[i]}
%$. Because $y_{[i]}$ is the $i$-th largest element in $\bm{y}$, $x_{[i]} \leq y_{[i]}$. This completes the proof.
%\end{proof}
Now we are ready to prove Theorem \ref{thm1}.
\begin{proof}[Proof of Theorem \ref{thm1}]
%See Appendix \ref{app1}.
Consider any work-conserving policy $\pi\in\Pi$. By Lemma \ref{coupling}, there exist policy $P_1$ and policy $\pi_1$
satisfying the same scheduling disciplines with policy $P$ and policy $\pi$, respectively, and the packet delivery times in policy $P_1$ and policy $\pi_1$ are synchronized almost surely.

For any given sample path of policy $P_1$ and policy $\pi_1$, $\bm\Delta_{P_1}(0^-) = \bm\Delta_{\pi_1}(0^-)$ at time $t=0^-$. We consider two cases:

\emph{Case 1:} When there is no packet delivery, the age of each flow grows linearly with a slope 1. 

\emph{Case 2:} When a packet is successfully delivered, the evolution of the system state is governed by  Lemma \ref{lem2}. 

By induction over time, we obtain
\begin{align}\label{eq_thm1_proof1}
\Delta_{[i],P_1} (t) \leq \Delta_{[i],\pi_1} (t),~i=1,\ldots,N,~t\geq 0.
\end{align}

For any symmetric and non-decreasing   function $p_t$, it holds from \eqref{eq_thm1_proof1} that for all sample paths and all $t\geq 0$
\begin{align}\label{eq_thm1_proof3}
&p_t\circ \bm \Delta_{P_1}(t) \nonumber\\
=& p_t(\Delta_{1,P_1} (t), \ldots, \Delta_{N,P_1} (t))\nonumber\\
=& p_t (\Delta_{[1],P_1} (t), \ldots, \Delta_{[N],P_1} (t))\nonumber\\
\leq & p_t (\Delta_{[1],\pi_1} (t), \ldots, \Delta_{[N],\pi_1} (t))\nonumber\\
=& p_t (\Delta_{1,\pi_1} (t), \ldots, \Delta_{N,\pi_1} (t))\nonumber\\
=& p_t\circ \bm \Delta_{\pi_1}(t).
\end{align}
By Lemma \ref{coupling}, the state process $\{\bm\Delta_{P_1}(t),t\in [0,\infty)\}$ of policy $P_1$ has the same distribution with the state process $\{\bm\Delta_{P}(t),t\in [0,\infty)\}$ of policy $P$;
the state process $\{\bm\Delta_{\pi_1}(t),t\in [0,\infty)\}$ of policy $\pi_1$ has the same distribution with the state process $\{\bm\Delta_{\pi}(t),t\in [0,\infty)\}$  of policy $\pi$. Hence, $\{p_t\circ\bm\Delta_{P_1}(t),t\in [0,\infty)\}$ has the same distribution with $\{p_t\circ\bm\Delta_{P}(t),t\in [0,\infty)\}$; $\{p_t\circ\bm\Delta_{\pi_1}(t),t\in [0,\infty)\}$ has the same distribution with $\{p_t\circ\bm\Delta_{\pi}(t),t\in [0,\infty)\}$. By substituting this and \eqref{eq_thm1_proof3} into Theorem 6.B.30 of \cite{StochasticOrderBook}, we can obtain that \eqref{thm1eq1} holds for all work-conserving policy $\pi\in\Pi$. 

For non-work-conserving policies $\pi$, because the service times are exponentially distributed (i.e., memoryless) and \emph{i.i.d.} across servers and time, server idling only postpones the delivery times of the packets. One can construct a coupling to show that for any non-work-conserving policy $\pi$, there exists a work-conserving policy $\pi'$ whose age process is smaller than that of policy $\pi$ in stochastic ordering; the details are omitted. 
%the age of 
%Therefore, the age under non-work-conserving policies will be greater. 
As a result, \eqref{thm1eq1} holds for all policies $\pi\in\Pi$. 


Finally, the equivalence between \eqref{thm1eq1} and \eqref{thm1eq2} follows from \eqref{eq_order}. This completes the proof.
\end{proof}

\ignore{

\section{Proof of Theorem \ref{coro2}}\label{appcoro2}
In order to establish Theorem \ref{coro2}, we make a modification to the proof of Theorem \ref{coro1}  as follows: Instead of using on Lemma \ref{coupling}, we introduce a new coupling lemma that is specifically designed to address transmission failures.

\begin{lemma}\emph{(Coupling Lemma)}\label{coupling_2}
In continuous-time status updating systems, consider policy $P$ and any {work-conserving} policy $\pi\in \Pi$. For  any given $\mathcal{I}$, if (i) there are \emph{i.i.d.} transmission failures with a failure probability $q\in (0,1)$,  and (ii)  the packet transmission times are exponentially distributed and \emph{i.i.d.} across  packets,
 then there exist policy $P_1$ and policy $\pi_1$ in the same probability space which satisfy the same scheduling disciplines with policy $P$ and policy $\pi$, respectively,  such that 
\begin{itemize}
\itemsep0em 
\item[1.] the state process $\{\bm\Delta_{P_1}(t),t\in [0,\infty)\}$ of policy $P_1$ has the same distribution as the state process $\{\bm\Delta_{P}(t),t\in [0,\infty)\}$ of policy $P$,
\item[2.] the state process $\{\bm\Delta_{\pi_1}(t),t\in [0,\infty)\}$ of policy $\pi_1$ has the same distribution as the state process $\{\bm\Delta_{\pi}(t),t\in [0,\infty)\}$  of policy $\pi$,
\item[3.] if a packet transmission fails at time $t'$ in policy $P_1$ as $\bm\Delta_{P_1}(t)$ evolves, then almost surely, a packet transmission fails at time $t'$ in policy $\pi_1$ as $\bm\Delta_{\pi_1}(t)$ evolves; and vice versa,
\item[4.] if a packet is successfully delivered  at time $t''$ in policy $P_1$ as $\bm\Delta_{P_1}(t)$ evolves, then almost surely, a packet is successfully delivered  at time $t''$ in policy $\pi_1$ as $\bm\Delta_{\pi_1}(t)$ evolves; and vice versa.
%whenever there exist unassigned tasks in the queue,
\end{itemize} 
\end{lemma}

\begin{proof}
According to Lemma \ref{coupling}, there exists two policy $P_1$ and policy $\pi_1$ in the same probability space which satisfy the same scheduling disciplines with policy $P$ and policy $\pi$, respectively,  such that 
\begin{itemize}
\itemsep0em 
\item[1.] the state process $\{\bm\Delta_{P_1}(t),t\in [0,\infty)\}$ of policy $P_1$ has the same distribution with the state process $\{\bm\Delta_{P}(t),t\in [0,\infty)\}$ of policy $P$,
\item[2.] the state process $\{\bm\Delta_{\pi_1}(t),t\in [0,\infty)\}$ of policy $\pi_1$ has the same distribution with the state process $\{\bm\Delta_{\pi}(t),t\in [0,\infty)\}$  of policy $\pi$,
\item[3.] if a packet completes transmission (regardless of success or failure) at time $t$ in policy $P_1$ as $\bm\Delta_{P_1}(t)$ evolves, then almost surely, a packet completes transmission (with a success or failure) at time $t$ in policy $\pi_1$ as $\bm\Delta_{\pi_1}(t)$ evolves; and vice versa. 
%whenever there exist unassigned tasks in the queue,
\end{itemize} 

Recall that the packet generation/arrival times $\mathcal I$, the packet transmission times, and the transmission failures are governed by three mutually independent stochastic processes, none of which are influenced by the scheduling policy. Following the inductive construction argument used in the proof of Theorem 6.B.3 in \cite{StochasticOrderBook}, one can construct packet transmission success and failure events one by one in policy $P_1$ and policy $\pi_1$ such that 
\begin{itemize}
\itemsep0em 
\item[1.] if a packet transmission fails at time $t'$ in policy $P_1$ as $\bm\Delta_{P_1}(t)$ evolves, then almost surely, a packet transmission fails at time $t'$ in policy $\pi_1$ as $\bm\Delta_{\pi_1}(t)$ evolves; and vice versa,
\item[2.] if a packet is delivered successfully at time $t''$ in policy $P_1$ as $\bm\Delta_{P_1}(t)$ evolves, then almost surely, a packet is delivered successfully at time $t''$ in policy $\pi_1$ as $\bm\Delta_{\pi_1}(t)$ evolves; and vice versa.
%whenever there exist unassigned tasks in the queue,
\end{itemize} 
By this, Lemma \ref{coupling_2} is proven. 
\end{proof}

By employing Lemma \ref{coupling_2} as a replacement for Lemma \ref{coupling} in the proof of Theorem \ref{thm1}, we obtain a proof of Theorem \ref{coro2}.
}

\ifreport

\section{Proof of Theorem \ref{thm3}}\label{app2}

 
Let $(\bm\Delta_{\pi}(t),\bm\Xi_{\pi}(t))$ represent  the \emph{system state} of policy $\pi$ at time $t$ and $\{(\bm\Delta_{\pi}(t),\bm\Xi_{\pi}(t)),t\in [0,\infty)\}$ be the \emph{state process} of policy $\pi$. For notational simplicity, let policy $P$ represent the NP-MASIF-LGFS policy, which is a non-preemptive, work-conserving policy. 

For single-server  systems, the following \emph{work conservation law} 
plays an important role in the scheduling literature (see, e.g., \cite{Leonard_Kleinrock_book,Jose2010,Gittins:11}):
At any time, the expected total amount of time for completing the packets in the queue is invariant across different work-conserving policies. However, the work conservation law does not hold in multi-server  systems: It often happens that some servers are busy while the rest servers are idle, which leads to inefficient utilization of the idle servers and sub-optimal scheduling performance. 
In the sequel, we use a \emph{weak work-efficiency ordering} \cite{sun2016delay,sun2017delay} to compare different non-preemptive policies for multi-server systems. 

% proof is motivated by the sample-path  method developed in \cite{sun2016delay,sun2017delay} for near delay-optimal scheduling  in multi-server  systems.
%\footnote{Two work-efficiency orderings were used in \cite{sun2016delay,sun2017delay} to study (near) delay-optimal online scheduling in multi-server queueing systems.}


\begin{definition} \label{def_order} \emph{Weak Work-efficiency Ordering \cite{sun2016delay,sun2017delay}:}
For any given $\mathcal{I}$ and a sample path realization of two non-preemptive policies $\pi_1,\pi_2\in\Pi_{np}$, policy $\pi_1$ is said to be \emph{weakly more work-efficient} than policy $\pi_2$, if the following assertion is true:
%there is a one-to-one correspondence between the packets executed in policy $P$ and policy $\pi$ such that, if
{For each packet $j$ executed in policy $\pi_2$, if
\begin{itemize}
\item[1.] in policy $\pi_2$, a packet $j$ starts service at time $\tau$ and completes service at time $\nu$ ($\tau\leq \nu$), 
\item[2.] in policy $\pi_1$, the queue is not empty during $[\tau,\nu]$, 
\end{itemize}
then  in policy $\pi_1$, there always exists one corresponding packet $j'$ that starts service during $[\tau,\nu]$. It is worth noting that the weak work-efficiency ordering does not require to specify which servers are used to process packets $j$ and $j'$.} \end{definition}

\begin{figure}
\centering 
\includegraphics[width=0.3\textwidth]{work_efficiency_ordering1.pdf} \caption{An illustration of the weak work-efficiency ordering, where the service duration of a packet is indicated by a rectangle, without specifying which servers are used to process the packets. 
%If a packet is replicated on multiple servers, then the service duration of this packet starts from the earliest time that one replica of this packet enters a server, until one replica of this packet is completed. 
Suppose that policy $\pi_1$ is {weakly more work-efficient than} policy $\pi_2$. If (i) a packet $j$ starts service at time $\tau$ and completes service at time $\nu$ in policy $\pi_2$, and (ii) the queue is not empty during $[\tau,\nu]$ in policy $\pi_1$, then in policy $\pi_1$ there exists one corresponding packet $j'$ that starts service at some time $t$ during $[\tau,\nu]$. 
}
% work--efficiency ordering holds for any priorities of the jobs.
\label{Work_Efficiency_Ordering} 
\end{figure} 

An illustration of the weak work-efficiency ordering is provided in Fig. \ref{Work_Efficiency_Ordering}. The weak work-efficiency ordering formalizes the following useful intuition for comparing two non-preemptive policies $\pi_1$ and $\pi_2$: \emph{If one packet $j$ is delivered at time $\nu$ in policy $\pi_2$, then  there exists one corresponding packet $j'$ that has started its transmission shortly before time $\nu$  in policy $\pi_1$, as long as the queue is not empty.} %Suppose that policy $\pi_1$ is {weakly more work-efficient than} policy $\pi_2$. If a packet $j$ starts service at time $\tau$ and completes service at time $\nu$ in policy $\pi_2$, then in policy $\pi_1$ either (i) there exists one corresponding packet $j'$ that starts service during $[\tau,\nu]$ or (ii) the queue is empty during $[\tau,\nu]$. 
The weak work-efficiency ordering  was originally introduced for near-optimal delay minimization in multi-server systems \cite{sun2016delay,sun2017delay}. In this paper, we use it for near-optimal age minimization in multi-server systems. 

%This is the key feature that enables us to establish a tight age lower bound. 
 

The following coupling lemma was established in \cite{sun2017delay} by using the property of NBU distributions:
\begin{lemma}\emph{(Coupling Lemma)} \cite[Lemma 2]{sun2017delay}\label{thm3lem_coupling}
In continuous-time status updating systems, consider two non-preemptive policies $P,\pi\in \Pi_{np}$. For any given $\mathcal{I}$, if (i) policy $P$ is work-conserving, and (ii) the packet service times are NBU, independent across the servers, and \emph{i.i.d.} across  the packets assigned to the same server, then there exist policy $P_1$ and  policy $\pi_1$ in the same probability space which satisfy the same scheduling disciplines with policy $P$ and policy $\pi$, respectively,  such that 
\begin{itemize}
\itemsep0em 
\item[1.] The state process $\{(\bm\Delta_{P_1}(t),\bm\Xi_{P_1}(t)),t\in [0,\infty)\}$ of policy $P_1$ has the same distribution as the state process $\{(\bm\Delta_{P}(t),\bm\Xi_{P}(t)),t\in [0,\infty)\}$ of policy $P$,
\item[2.] The state process $\{(\bm\Delta_{\pi_1}(t),\bm\Xi_{\pi_1}(t)),t\in [0,\infty)\}$ of policy $\pi_1$ has the same distribution as the state process $\{(\bm\Delta_{\pi}(t),\bm\Xi_{\pi}(t)),t\in [0,\infty)\}$  of policy $\pi$,
\item[3.] Policy $P_1$ is weakly more work-efficient than policy $\pi_1$ with probability one. 
%whenever there exist unassigned packets in the queue,
\end{itemize} 
\end{lemma}
The proof of Lemma \ref{thm3lem_coupling} is provided in \cite{sun2017delay}.



We will compare the age of service information of policy $P_1$ and the age of policy $\pi_1$ on a sample path by using the following lemma:

\begin{lemma} \emph{(Inductive Comparison)}\label{thm3lem2}
%Under the conditions of Lemma \ref{coupling}, 
Suppose that a packet starts service at time $t$ in policy $P_1$ and a packet completes service (i.e., delivered to the destination)  at the same time $t$ in policy $\pi_1$. The system state  of policy $P_1$ is $(\bm\Delta_{P_1},\bm\Xi_{P_1})$ before the service starts, which becomes $(\bm\Delta_{P_1}',\bm\Xi_{P_1}')$ after the service starts. The system state  of policy $\pi_1$ is $(\bm\Delta_{\pi_1},\bm\Xi_{\pi_1})$ before the service completes, which becomes $(\bm\Delta_{\pi_1}',\bm\Xi_{\pi_1}')$ after the service completes.
 If the packet generation and arrival times are {synchronized} across the $N$ flows and
\begin{equation}\label{thm3hyp1}
 \Xi_{[i],P_1} \leq \Delta_{[i],\pi_1},~i=1,\ldots,N,
\end{equation}
then
\begin{equation}\label{thm3law6}
\Xi_{[i],P_1}' \leq \Delta_{[i],\pi_1}',~i=1,\ldots,N.
\end{equation}  
\end{lemma}


\begin{proof}
For synchronized packet generations and arrivals,  let $W(t) = \max\{S_i: A_i \leq t\}$ 
be the generation time of the freshest packet of each flow that has arrived at the queue by time $t$. At time $t$, because no packet that has arrived at the queue was generated later than $W(t)$, we can obtain
\begin{align}%\label{eq_proof_1}
\Xi_{[i],P_1}' \geq t-W(t),~i=1,\ldots,N,\\
\Delta_{[i],\pi_1}' \geq t-W(t),~i=1,\ldots,N.\label{thm3eq_proof_2}
\end{align} 

Because policy $P_1$ follows the same scheduling discipline with the NP-MASIF-LGFS policy, each  packet starts service in policy $P_1$ must be from the flow with the maximum  age of served information $\Xi_{[1],P_1}$ (denoted as flow $n^*$), and the delivered packet must be the freshest packet that was generated at time $W(t)$. In other words, the age of served information of flow $n^*$ is reduced from the maximum age of served information $\Xi_{[1],P_1}$ to the minimum age of served information $\Xi_{[N],P_1}'=t-W(t)$, and the ages of served information of the other $(N-1)$ flows remain unchanged. Hence, 
\begin{align}\label{thm3eq_proof_3}
\Xi_{[i],P_1}' &= \Xi_{[i+1],P_1},~i=1,\ldots,N-1,\\
\Xi_{[N],P_1}' &= t - W(t). \label{thm3eq_proof_4}
\end{align}

In policy $\pi_1$, the delivered packet can be any packet from any flow. For all possible cases of policy $\pi_1$, it must hold that 
\begin{align}\label{thm3eq_proof_1}
\Delta_{[i],\pi_1}' \geq \Delta_{[i+1],\pi_1},~i=1,\ldots,N-1. 
\end{align}
By combining \eqref{thm3hyp1}, \eqref{thm3eq_proof_3}, and \eqref{thm3eq_proof_1}, we have
\begin{align}
\Delta_{[i],\pi_1}' \geq \Delta_{[i+1],\pi_1} \geq \Xi_{[i+1],P_1} = \Xi_{[i],P_1}',~i=1,\ldots,N-1.\nonumber
\end{align}
In addition, combining \eqref{thm3eq_proof_2} and \eqref{thm3eq_proof_4}, yields
\begin{align}
\Delta_{[N],\pi_1}' \geq  t-W(t) = \Xi_{[N],P_1}'.\nonumber
\end{align}
By this, \eqref{thm3law6} is proven.
\end{proof}
Now we are ready to prove Theorem \ref{thm3}.
\begin{proof}[Proof of Theorem \ref{thm3}]
Recall that policy $P$ is non-preemptive and work-conserving. 
 Consider any non-preemptive policy $\pi\in\Pi_{np}$.
By Lemma \ref{thm3lem_coupling}, there exist policy $P_1$ and policy $\pi_1$
satisfying the same scheduling disciplines with policy $P$ and policy $\pi$, respectively, and policy $P_1$ is weakly more work-efficient than policy $\pi_1$ with probability one.

Next, we construct another policy $\pi_1'$ in the same probability space with policy  $P_1$ and policy $\pi_1$: 
Because policy $P_1$ is weakly more work-efficient than policy $\pi_1$ with probability one, if
%After this modification, the following three properties are true:
\begin{itemize}
\item[1.] in policy $\pi_1$, a packet $j$ starts service at time $\tau$ and completes service at time $\nu$ ($\tau\leq \nu$),
\item[2.] in policy $P_1$, the queue is \emph{not empty} during $[\tau,\nu]$,
\end{itemize}
then  in policy $P_1$, there exists one corresponding packet $j'$ that starts service during $[\tau,\nu]$. Let  $t\in [\tau,\nu]$ be the service starting time of packet $j'$ in policy $P_1$, then  in policy $\pi_1'$, packet $j$ is constructed to start service at time $\tau$ and complete service at time $t$, as illustrated in Fig. \ref{Work_Efficiency_Ordering2}. On the other hand, if
%After this modification, the following three properties are true:
\begin{itemize}
\item[1.] in policy $\pi_1$, a packet $j$ starts service at time $\tau$ and completes service at time $\nu$ ($\tau\leq \nu$),
\item[2.] in policy $P_1$, the queue is \emph{empty} during $[\tau,\nu]$,
\end{itemize}
then in policy $\pi_1'$, packet $j$ is constructed to start service at time $\tau$ and complete service at time $\nu$. 
The initial age of policy $\pi_1'$ is chosen to be the same as that of other policies. Hence, $ \bm\Delta_{\pi_1'}(0^-) = \bm\Delta_{P_1}(0^-) = \bm\Delta_{\pi_1}(0^-)$. 


The  policy $\pi_1'$ constructed above satisfies the following two useful properties: 
\begin{figure}
\centering 
\includegraphics[width=0.3\textwidth]{work_efficiency_ordering2} \caption{An illustration of the construction of policy $\pi'_1$, where the queue is \emph{not empty} during $[\tau,\nu]$ in policy $P_1$. The service completion time $t$ of packet $j$ in policy $\pi_1'$ is smaller than the service completion time $\nu$ of packet $j$ in policy $\pi$, and is equal to the service starting time $t$ of packet $j'$ in policy $P_1$. }
% work--efficiency ordering holds for any priorities of the jobs.
\label{Work_Efficiency_Ordering2} 
\end{figure} 



\emph{Property (i):} The service completion time of each packet in policy $\pi_1'$ is equal to or earlier than that in policy $\pi$. 
Hence, 
\begin{align}\label{eq_thm3_proof2}
\bm\Delta_{\pi_1'}(t) \leq \bm\Delta_{\pi_1}(t), ~t\in[0,\infty)
\end{align}
holds with probably one. 

\emph{Property (ii):} If the queue is not empty at time $t$ in policy $P_1$ and a packet completes service  at time $t$ in policy $\pi_1'$, then a packet starts service at the same time $t$ in policy $P_1$. 


Next, we use \emph{Property (ii)} to  show that, almost surely, 
\begin{align}\label{eq_thm3_proof1}
\Xi_{[i],P_1} (t) \leq \Delta_{[i],\pi_1'} (t),~i=1,\ldots,N,~t\geq 0.
\end{align}

At time $t= 0^-$, because $\bm\Xi_{P_1}(0^-) \leq \bm\Delta_{P_1}(0^-)$ and $\bm\Delta_{P_1}(0^-) = \bm\Delta_{\pi_1'}(0^-)$, we can obtain $\bm\Xi_{P_1}(0^-) \leq \bm\Delta_{\pi_1'}(0^-)$. This further implies that 
\begin{align}
\Xi_{[i],P_1} (0^-) \leq \Delta_{[i],\pi_1'} (0^-),~i=1,\ldots,N. 
\end{align}
For any time $t> 0$, there could be three cases:

\emph{Case 1:} if the queue is empty at time $t$ in policy $P_1$, then \eqref{eq_thm3_proof1} holds naturally at time $t$ because all packets have started services in policy $P_1$ (otherwise, the queue is not empty). 

\emph{Case 2:} if the queue is not empty at time $t$ in policy $P_1$ and a packet completes service at time $t$ in policy $\pi_1'$, according to \emph{Property (ii)}, a packet starts service at time $t$ in policy $P_1$. Hence, the evolution of the system state before and after time $t$ is governed by  Lemma \ref{thm3lem2}. 

\emph{Case 3:} if the queue is not empty at time $t$ in policy $P_1$ and no packet completes service at time $t$ in policy $\pi_1'$, there may exist some packet that starts service at time $t$ in policy $P_1$. Therefore, the age of each flow in policy $\pi_1'$  grows linearly with a slope 1 at time $t$; the age of served information of each flow in policy $P_1$ may either grow linearly with a slope 1 or drop to a lower value. By comparison, the age of served information of each flow in policy $P_1$ grows at a speed no faster than the age of each flow in policy $\pi_1'$.  

By induction over time and considering the above three cases, \eqref{eq_thm3_proof1} is proven.

Furthermore, for any symmetric and non-decreasing   function $p_t$, it holds from \eqref{eq_thm3_proof2} and \eqref{eq_thm3_proof1} that for all sample paths and all $t\geq 0$
\begin{align}\label{eq_thm3_proof3}
&p_t\circ \bm \Xi_{P_1}(t) \nonumber\\
=& p_t(\Xi_{1,P_1} (t), \ldots, \Xi_{N,P_1} (t))\nonumber\\
=& p_t (\Xi_{[1],P_1} (t), \ldots, \Xi_{[N],P_1} (t))\nonumber\\
\leq & p_t (\Delta_{[1],\pi_1'} (t), \ldots, \Delta_{[N],\pi_1'} (t))\nonumber\\
=& p_t (\Delta_{1,\pi_1'} (t), \ldots, \Delta_{N,\pi_1'} (t))\nonumber\\
=& p_t\circ \bm \Delta_{\pi_1'}(t)\nonumber\\
\leq & p_t\circ \bm \Delta_{\pi_1}(t).
\end{align}
By Lemma \ref{thm3lem_coupling}, the state process $\{(\bm\Delta_{P_1}(t),\bm\Xi_{P_1}(t)),t\in [0,\infty)\}$ of policy $P_1$ has the same distribution with the state process $\{(\bm\Delta_{P}(t),\bm\Xi_{P}(t)),t\in [0,\infty)\}$ of policy $P$;
the state process $\{(\bm\Delta_{\pi_1}(t),\bm\Xi_{\pi_1}(t)),t\in [0,\infty)\}$ of policy $\pi_1$ has the same distribution with the state process $\{(\bm\Delta_{\pi}(t),\bm\Xi_{\pi}(t)),t\in [0,\infty)\}$  of policy $\pi$. Hence, $\{p_t\circ\bm\Xi_{P_1}(t),t\in [0,\infty)\}$ 
has the same distribution with $\{p_t\circ\bm\Xi_{P}(t),t\in [0,\infty)\}$; $\{p_t\circ\bm\Delta_{\pi_1}(t),t\in [0,\infty)\}$ 
has the same distribution with $\{p_t\circ\bm\Delta_{\pi}(t),t\in [0,\infty)\}$. By substituting this and \eqref{eq_thm3_proof3} into Theorem 6.B.30 of \cite{StochasticOrderBook}, we can obtain that \eqref{thm3eq1} holds for all policy $\pi\in\Pi_{np}$. According to  \eqref{eq_order}, the first inequality in \eqref{thm3eq2} is equivalent to   \eqref{thm3eq1}. The second inequality in \eqref{thm3eq2} holds naturally. This completes the proof.
\end{proof}

\section{Proof of Theorem \ref{thm4}}\label{app_thm4}


Let the age vector $\bm\Delta_{\pi}(t) = (\Delta_{1,\pi} (t),\ldots,\Delta_{N,\pi} (t))$ represent the \emph{system state} of policy $\pi$ at time $t$ and $\{\bm\Delta_{\pi}(t),t=0,T_s,2T_s,\ldots\}$ be the \emph{state process} of policy $\pi$. Recall that $\Delta_{[i],\pi}(t)$ is the $i$-th largest component of the age vector $\bm\Delta_{\pi}(t)$. For notational simplicity, let policy $P$ represent the DT-MAF-LGFS policy, which is a non-preemptive, work-conserving policy. We first present two lemmas that are useful to prove Theorem \ref{thm4}. 


\begin{lemma}\emph{(Coupling Lemma)}\label{coupling_4}
In discrete-time status updating systems, consider policy $P$ and any {non-preemptive, work-conserving} policy $\pi\in \Pi_{np}$. For any given $\mathcal{I}$, if (i) the transmission errors are \emph{i.i.d.} with an error probability $q\in [0,1)$ and (ii) the transmission time of each packet is equal to $T_s$,
 then there exist policy $P_1$ and policy $\pi_1$ in the same probability space which satisfy the same scheduling disciplines with policy $P$ and policy $\pi$, respectively,  such that 
\begin{itemize}
\itemsep0em 
\item[1.] the state process $\{\bm\Delta_{P_1}(t),t=0,T_s,2T_s,\ldots\}$ of policy $P_1$ has the same distribution as the state process $\{\bm\Delta_{P}(t),t=0,T_s,2T_s,\ldots\}$ of policy $P$,
\item[2.] the state process $\{\bm\Delta_{\pi_1}(t),t=0,T_s,2T_s,\ldots\}$ of policy $\pi_1$ has the same distribution as the state process $\{\bm\Delta_{\pi}(t),t=0,T_s,2T_s,\ldots\}$  of policy $\pi$,
\item[3.] if a packet from the flow with age $\Delta_{[i],P_1}(t)$ at time $t$ is successfully delivered  at time $(t + T_s)$ in policy $P_1$, then almost surely, a packet from the flow with age $\Delta_{[i],\pi_1}(t)$ at time $t$ is successfully delivered  at time $(t+ T_s)$ in policy $\pi_1$; and vice versa.
%whenever there exist unassigned tasks in the queue,
\end{itemize} 
\end{lemma}

\begin{proof}
By employing the inductive construction argument used in the proof of Theorem 6.B.3 in \cite{StochasticOrderBook}, one can construct the packet transmission success and failure events one by one in policy $P_1$ and policy $\pi_1$ to prove this lemma. 
In particular, since the transmission errors are \emph{i.i.d.} and they are not influenced by the scheduling policy, it is feasible to couple the packet transmission success and failure events in policy $P_1$ and policy $\pi_1$ in such a way that a packet from the flow with age $\Delta_{[i],P_1}(t)$ at time $t$ is successfully delivered at time $(t+ T_s)$ in policy $P_1$ if, and only if, a packet from the flow with age $\Delta_{[i],\pi_1}(t)$ at time $t$ is successfully delivered  at time $(t+ T_s)$ in policy $\pi_1$. \end{proof}

Notice that policy $P_1$ and policy $\pi_1$ are two distinct policies, so the flow with age $\Delta_{[i],P_1}(t)$ in policy $P_1$ and the flow with age $\Delta_{[i],\pi_1}(t)$ at time $t$ in policy $\pi_1$ are likely representing different flows. However, policy $P_1$ and policy $\pi_1$ are coupled in such a way  that  the packet deliveries for these two flows occur simultaneously at time $(t+ T_s)$.

We will compare policy $P_1$ and policy $\pi_1$ on a sample path by using the following lemma: 

\begin{lemma} \emph{(Inductive Comparison)}\label{thm4lem2}
Under the conditions of Lemma \ref{coupling_4}, if  (i) the packet generation and arrival times are {synchronized} across the $N$ flows and (ii) 
\begin{equation}\label{thm4lem2eq1}
\Delta_{[i],P_1}(t) \leq \Delta_{[i],\pi_1}(t),~i=1,\ldots,N,
\end{equation}
then
\begin{equation}\label{thm4lem2eq2}
\Delta_{[i],P_1}(t+T_s)\leq \Delta_{[i],\pi_1}(t+T_s),~i=1,\ldots,N.
\end{equation}  
\end{lemma}

\begin{proof}
For synchronized packet generations and arrivals, let $W(t) = \max_i\{S_i: A_i \leq t\}$ 
be the generation time of the freshest packet of each flow that has arrived at the queue by time $t$. Because (i) the packet transmission time is $T_s$ and (ii) no packet that has arrived at the queue by time $t$ was generated after time $W(t)$, we can obtain
\begin{align}%\label{eq_proof_1}
%\Delta_{[i],P_1}(t+T_s)\geq t + T_s-W(t),~i=1,\ldots,N,\\
\Delta_{[i],\pi_1}(t+T_s)\geq t + T_s -W(t),~i=1,\ldots,N.\label{thm4lem2eq3}
\end{align} 
Without loss of generality, suppose that there are $l$ transmission errors and $(N-l)$ successful packet deliveries at time $(t+T_s)$ in policy $P_1$. 
Because (i) policy $P_1$ follows the same scheduling discipline with the DT-MAF-LGFS policy and (ii) the packet generation and arrival times are {synchronized} across the $N$ flows, each delivered packet must be the freshest packet generated at time $W(t)$. Hence, the flows associated with these delivered packets must have the minimum age at time $(t+T_s)$, given by
\begin{align}
%\Delta_{[i],P_1}(t+T_s) &= \Delta_{[j_1],P_1},~i=1,\ldots,N-l,\\
\Delta_{[i],P_1}(t+T_s) &= t + T_s- W(t),~i=l+1,\ldots,N. \label{thm4lem2eq4}
\end{align}
Combining \eqref{thm4lem2eq3} and \eqref{thm4lem2eq4}, yields
\begin{align}
\Delta_{[i],P_1}(t+T_s) = t + T_s- W(t) \leq \Delta_{[i],\pi_1}(t+T_s),\nonumber\\ ~i=l+1,\ldots,N. \label{thm4lem2eq16}
\end{align}

Moreover, suppose that the transmission errors at time $(t+T_s)$ are from the flows with age values $(\Delta_{[j_1],P_1}(t),\Delta_{[j_2],P_1}(t),\ldots,\Delta_{[j_{l}],P_1}(t))$ at time $t$, which are sorted such that $j_1\geq j_2\geq \ldots \geq j_l$. Because $\Delta_{[i],P_1}(t)$ is the $i$-th largest component of the age vector $\bm\Delta_{P_1}(t)$, we have
\begin{align}
 \Delta_{[j_1],P_1}(t) \geq \Delta_{[j_2],P_1}(t)\geq\ldots\geq\Delta_{[j_{l}],P_1}(t). \label{thm4lem2eq8}
\end{align} 
If flow $n$ is one of the flows that encounter a transmission error at time $t+T_s$ in policy $P_1$, then
\begin{align}
\Delta_{n,P_1}(t+T_s) = \Delta_{n,P_1}(t) +T_s. \label{thm4lem2eq10}
\end{align}
%otherwise, it follows from \eqref{thm4lem2eq4} that flow $n$ has the minimum age
%\begin{align}
%\Delta_{n,P_1}(t+T_s) = t + T_s- W(t). \label{thm4lem2eq11}
%\end{align}
From \eqref{thm4lem2eq4}, \eqref{thm4lem2eq8}, and \eqref{thm4lem2eq10}, the components of vector $\bm\Delta_{P_1}(t+T_s)$
are $\Delta_{[j_1],P_1}(t)+T_s,\Delta_{[j_2],P_1}(t)+T_s,\ldots,\Delta_{[j_{l}],P_1}(t)+T_s$ and $(N-l)$ numbers with the same value $t + T_s- W(t)$. Hence,
\begin{align}
\Delta_{[i],P_1}(t+T_s) = \Delta_{[j_i],P_1}(t) +T_s, ~i=1,\ldots,l. \label{thm4lem2eq5}
\end{align}

According to Lemma \ref{coupling_4}, there are $l$ transmission errors at time $(t+T_s)$ in policy $\pi_1$, which are from the flows with age values $(\Delta_{[j_1],\pi_1}(t),\Delta_{[j_2],\pi_1}(t),\ldots,\Delta_{[j_{l}],\pi_1}(t))$ at time $t$. Because $j_1\geq j_2\geq \ldots \geq j_l$, we have
 \begin{align}
 \Delta_{[j_1],\pi_1}(t) \geq \Delta_{[j_2],\pi_1}(t)\geq\ldots\geq\Delta_{[j_{l}],\pi_1}(t). \label{thm4lem2eq12}
\end{align} 
If flow $n$ is one of the flows that encounter a transmission error at time $t+T_s$ in policy $\pi_1$, then
\begin{align}
\Delta_{n,\pi_1}(t+T_s) = \Delta_{n,\pi_1}(t) +T_s. \label{thm4lem2eq13}
\end{align}
From \eqref{thm4lem2eq12} and \eqref{thm4lem2eq13}, one can observe that $\Delta_{[j_1],\pi_1}(t)+T_s,\Delta_{[j_2],\pi_1}(t)+T_s,\ldots,\Delta_{[j_{l}],\pi_1}(t)+T_s$ are $l$ components of vector $\bm\Delta_{\pi_1}(t+T_s)$. Hence, 
\begin{align}
\Delta_{[i],\pi_1}(t+T_s) \geq \Delta_{[j_i],\pi_1}(t) +T_s, ~i=1,\ldots,l. \label{thm4lem2eq6}
\end{align}
Combining \eqref{thm4lem2eq1}, \eqref{thm4lem2eq5}, and \eqref{thm4lem2eq6}, yields
\begin{align}
&\Delta_{[i],P_1}(t+T_s) \nonumber\\
= &\Delta_{[j_i],P_1}(t) +T_s  \nonumber\\
\leq &\Delta_{[j_i],\pi_1}(t) +T_s \nonumber\\
\leq & \Delta_{[i],\pi_1}(t+T_s), ~i=1,\ldots,l. \label{thm4lem2eq7}
\end{align}
Finally, \eqref{thm4lem2eq2} follows from \eqref{thm4lem2eq16} and \eqref{thm4lem2eq7}. This completes the proof. 
\end{proof}


Now we  prove Theorem \ref{thm4}.
\begin{proof}[Proof of Theorem \ref{thm4}]
%See Appendix \ref{app1}.
Consider any non-preemptive, work-conserving policy $\pi\in\Pi_{np}$. By Lemma \ref{coupling_4}, there exist policy $P_1$ and policy $\pi_1$
satisfying the same scheduling disciplines with policy $P$ and policy $\pi$, respectively, such that if a packet from the flow with age $\Delta_{[i],P_1}(t)$ at time $t$ is successfully delivered  at time $(t + T_s)$ in policy $P_1$, then almost surely, a packet from the flow with age $\Delta_{[i],\pi_1}(t)$ at time $t$ is successfully delivered  at time $(t+ T_s)$ in policy $\pi_1$; and vice versa.

For any given sample path of policy $P_1$ and policy $\pi_1$, the initial system state is $\bm\Delta_{P_1}(0) = \bm\Delta_{\pi_1}(0)$ at time $t=0$. The evolution of the system state is governed by  Lemma \ref{thm4lem2}. By induction over time, we obtain
\begin{align}\label{eq_thm4_proof1}
\Delta_{[i],P_1} (t) \leq \Delta_{[i],\pi_1} (t),~i=1,\ldots,N,~t = 0, T_s, 2T_s,\ldots.
\end{align}
The rest of the proof is quite similar to that of Theorem \ref{thm1} and hence are omitted. 
\end{proof}



\fi