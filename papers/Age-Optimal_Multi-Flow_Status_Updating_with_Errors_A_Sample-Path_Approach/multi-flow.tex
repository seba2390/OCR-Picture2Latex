\newif\ifreport\reporttrue
\documentclass[journal]{IEEEtran}
%\documentclass{sig-alternate}

%\newcommand{\sublite}[1]{\par\noindent {\bf #1:}\ }
%\newcommand{\paperlink}[2]{}
%\newcommand{\confpaperlink}[2]{}
\usepackage{setspace}
\setstretch{1.0}
\usepackage{cite}
\usepackage{graphicx}
\usepackage{bm}
\usepackage[cmex10]{amsmath}
\usepackage{mathtools}
\usepackage{amsthm}
\usepackage{amsfonts}
\usepackage{amssymb}
\usepackage{hyperref}
%\usepackage{texdef2015}
%\usepackage{epstopdf}
\usepackage{algpseudocode}
\usepackage[lined, boxed, linesnumbered, ruled]{algorithm2e}
\usepackage{fixltx2e}%fixes latex float algorithm in 2 columns
\usepackage{tikz}
\usetikzlibrary{decorations.pathreplacing}
\theoremstyle{definition}
\newtheorem{definition}{Definition}
%\newtheorem{theorem}{Theorem}
%\newdef{definition}{Definition}
\newcommand{\nn}{\nonumber\\}
\newcommand{\Tcal}{\mathcal{T}}
\newcommand{\age}{\Delta}
%\newcommand{\Spre}{S^{\ast}}
\newcommand{\Spre}{Z}
%\margin
%\usepackage[margin=1in]{geometry}


\newcommand{\Yq}{Y^{(q)}}


%\DeclareMathOperator*{\argmax}{argmax}
%\DeclareMathOperator*{\argmin}{argmin}
\usepackage{color}
\def\orange{\color{orange}}
\def\blue{\color{black}}
\def\red{\color{red}}
\newcommand{\ignore}[1]{}

\renewcommand{\baselinestretch}{1}

%\theoremstyle{plain}
\newtheorem{assumption}{Assumption}
\newtheorem{example}{Example}
\newtheorem{lemma}{Lemma}
\newtheorem{proposition}{Proposition}
\newtheorem{Prop}{Proposition}
\newtheorem{theorem}{Theorem}
\newtheorem{corollary}{Corollary}
%\newtheorem{definition}{Definition}
%\theoremstyle{definition}
\newtheorem{Conjecture}{Conjecture}
%\newdef{definition}{Definition}

\newtheorem{propy}{Property}
\newtheorem{claim}{Claim}
\newtheorem{fact}{Fact}

%\theoremstyle{remark}
\newtheorem{remark}{Remark}
\newtheorem{Assumption}{Assumption}
\DeclareMathOperator*{\argmax}{argmax}
\DeclareMathOperator*{\argmin}{argmin}

\begin{document}
\IEEEoverridecommandlockouts
%\title{Maximizing Data Freshness for Information Updates}
%\title{Minimizing the Age of Information in Multi-Source Networks}
\title{Minimizing the Age of the Information to Multiple Sources}
\title{\huge	 Age-Optimal Updates of Multiple Information Flows}
\title{\huge	 Age-Optimal Updates of Multiple Information Flows: A Sample-path Approach}
\title{Age of Information Minimization for Multiple Information Flows: A Sample-path Approach}
\title{Age-Optimal Multi-Flow Status Updating with Errors: A Sample-Path Approach}

%\title{Multi-flow Age Minimization through Multiple Channels with Errors: A Sample-path Approach}

\author{Yin Sun and Sastry Kompella %, Elif Uysal, 
\thanks{This paper was presented in part at the IEEE INFOCOM Age of Information (AoI) Workshop in 2018 \cite{SunAoIWorkshop2018}.}
\thanks{Yin Sun's work is supported in part by the NSF grant CNS-2239677 and the ARO grant W911NF-21-1-0244.}
\thanks{Yin Sun is with the Department of ECE, Auburn University, Auburn, AL 36849 USA, email: yzs0078@auburn.edu.}
%\thanks{Elif Uysal is with the Department of EEE, Middle East Technical University, Ankara, Turkey, email: uelif@metu.edu.tr.}
\thanks{Sastry Kompella is with Nexcepta Inc., Gaithersburg, MD 20878 USA, e-mail: sk@ieee.org.}

}

\maketitle
\thispagestyle{plain}
\pagestyle{plain}
\begin{abstract}
\label{sec:abstract}

%% 1. what is the problem 
Scientific applications that run on leadership computing facilities often face the challenge 
of being unable to fit leading science cases onto accelerator devices due to memory constraints 
(memory-bound applications).
%
% 2. what is your solution 
In this work, the authors studied one such US Department of Energy mission-critical condensed matter 
physics application, Dynamical Cluster Approximation (DCA++), and this paper discusses how device memory-bound challenges were successfully reduced  by proposing an effective 
``all-to-all'' communication method---a ring communication algorithm. 
%
This implementation takes advantage of acceleration on GPUs and remote direct memory access (RDMA) for fast data exchange between GPUs. 
%
\\Additionally, the ring algorithm was optimized with sub-ring communicators
and multi-threaded support to further reduce communication overhead and 
expose more concurrency, respectively.
%
% 3. What's the cherry-picked evaluation result you want to mention
The computation and communication were also analyzed 
by using the Autonomic Performance Environment for Exascale 
(APEX) profiling tool,  and this paper further discusses the 
performance trade-off for the ring algorithm implementation. 
%
The memory analysis on the ring algorithm shows that the allocation size for the authors' most 
memory-intensive data structure per GPU is now reduced to $1/p$ of the original size, where $p$ is the number of GPUs in the ring communicator.
%
The communication analysis suggests that 
the distributed Quantum Monte Carlo execution time grows linearly as sub-ring size increases, and the cost of messages passing through the network interface connector could be a limiting factor.


%
% \todoRed{Ronnie: Next sentence needs rewrite, too much information about Green's function that no one knows in the abstract; recommend generalizing.} \emph {However, DCA++ is currently facing memory-bound challenge as 
% a larger device array $G_t$ is limited by device memory size, where
% $G_t$ is a two-particle Green's function that allows condensed matter
% scientists to explore larger and more complex (higher fidelity)
% physics cases.}

\end{abstract}

\keywords{DCA++, Quantum Monte Carlo, GPU Remote Direct Memory Access, memory-bound issue, exascale machines}

\section{Introduction}
Over the past decade, we have witnessed two parallel trends. On one side, the increasing popularity of the internet of things, i.e., intelligent networked things everywhere, a consequence of the growing capabilities of the embedded systems, enhanced with capable processing units working at always increasing frequencies and offering attractive low-power modes \cite{cortex-m, raspberry, friendlyelec}. 
On the other side, with the advent of deep learning techniques, machine learning algorithms' size grows exponentially, thanks to the improvements in processor speeds and the availability of large training data. However, %as illustrated by Cerutti et. al in \cite{Cerutti2019}, 
embedded systems cannot sustain the resource requirements of standard deep learning techniques, adequate for GP-GPUs \cite{Cerutti2019, Rusci2018, hao2021enabling}.

How then to compose the need for exploiting the opportunity to bring intelligence at the edge with the complexity of deep learning?
In this context, the junction point is TinyML \cite{wang2020convergence, xu2020edge}, a cutting-edge field that brings the machine learning (ML) transformative power to the performance- and power-constrained domain of tiny devices and embedded systems.

Among the several application domains in which exploring this novel trend, computer vision is one of the most popular. In this domain, object detection and tracking should be real-time, reliable, and accurate. Recent best-performing pipelines for multi-object detection and tracking imply using many computational resources, thus substantially limiting the application scenarios in which such techniques can be exploited. The current limitations mainly depend on the high computational cost of state-of-the-art techniques, which require GPUs for real-time inference and thus are not a good candidate for resource-constrained devices.

%In terms of computational complexity - which is the most substantial constraint -, the most efficient solution for tracking pipelines is a tracking-by-detection approach. 
The most efficient solutions for Multi-Object Tracking (MOT) follows the tracking-by-detection paradigm. The tracking algorithm consists of an association algorithm based on the detected bounding boxes. Many pipelines are available for object detection. In particular, the main difference for what concerns object detection is the number of stages required to detect objects in one frame. The detection pipeline can be one-stage (single-shot bounding box regression) \cite{bochkovskiy2020yolov4, liu2016ssd} or two-stage (region proposal, object identification) \cite{he2017mask, ren2015faster}. The one-stage detectors are the most efficient from the computational complexity perspective; thus, they are the go-to solution for lightweight, real-time object detection.

The most popular one-stage detectors are YOLO and SSD. The core of the detection pipeline is the convolutional backbone for latent space representation of the images, which accounts for most of the Multiply-Accumulate operations (MACC). After that, the detection head is applied to the output of the backbone and outputs the bounding box regression.
Since the most expensive part of the pipeline is the backbone, many works have proposed scalable architectures to improve the efficiency in image processing \cite{howard2017mobilenets, tan2021efficientnetv2}. The ability of the networks to change computational requirements is an asset for embedded inferencing since the embedded platforms are diverse in hardware constraints. For example, typical IoT-oriented MCUs, such as the STM32F7 MCU or STM32L4, only have 320kB SRAM/1MB Flash and 32KB SRAM/256KB Flash, respectively, thus they cannot run the same networks. Also, developing a CNN architecture to tackle vision problems efficiently is not a new problem. Networks using minimal resources, such as MobileNets \cite{howard2017mobilenets} and EfficientNets \cite{tan2019efficientnet}, have already been optimized to allow state-of-the-art performance with less than $1B$ Multiply and Accumulate operations (MACCs), but with low performance on MCU scale computational resources.

Our work contributes to the state-of-the-art by proposing a novel scalable backbone,  \textit{PhiNets}, for detection and multi-object tracking on resource-constrained platforms. 
We prove the efficiency of \textit{PhiNets} by comparing them with existing lightweight backbones within a YOLOv2 \cite{redmon2017yolo9000} detection head and Simple Online Real-time Tracking (SORT) tracker \cite{bewley2016simple}.
Moreover, we implemented the tracking inference on off-the-shelf MCUs with state-of-the-art consumption (1.3mJ per frame).
%Moreover, we demonstrate the tracking inference on a prototype based on H7... and... obtaining... quantità varie on this task...
Thus, our work has a meaningful impact in the fields of:
\begin{itemize}
\item embedded vision processing by proposing a new architecture family, \textit{PhiNets}, which pushes forward the state-of-the-art in object detection and  on tiny devices;
\item low-power image processing, since our pipeline requires only 1.3mJ per frame or 13mW at 10 fps;
\end{itemize}
\section{The \MakeLowercase{i}W\MakeLowercase{inr}NFL model}
\label{sec:model}

In this section we are going to present the data we used to develop our in-game probability model as well as the design details of {\method}. 

{\bf Data: }In order to perform our analysis we utilize a dataset collected from NFL's Game Center for all the regular season games between the seasons 2009 and 2016. 
We access the data using the Python {\tt nflgame} API \cite{nflgame}. 
The dataset includes detailed play-by-play information for every game that took place during these seasons. 
This information is used to obtain the state of the game that will drive the design of {\method}. 
In total, we collected information for 2,048 regular season games and a total of 338,294 snaps/plays. 

{\bf Model: }
{\method} is based on a logistic regression model that calculates the probability of the home team winning given the current status of the game as: 

\begin{equation}
\Pr(H=1| \mathbf{x})= \frac{\exp(\mathbf{\weight}^T\cdot\mathbf{x})}{1+\exp(\mathbf{\weight}^T\cdot\mathbf{x})}
\label{eq:reg}
\end{equation}
where $H$ is the dependent random variable of our model representing whether the home team wins or not, $\mathbf{x}$ is the vector with the independent variables, while the coefficient vector $\mathbf{\weight}$ includes the weights for each independent variable and is estimated using the corresponding data.  
For a game of infinite duration a linear model could be a very good approximation.  
However, the boundary effects from the finite duration of a game create several non-linearities \cite{winston2012mathletics}.  
For this reason, we enhance our model - using the same set of features - with a Support Vector Machine classifier with radial kernel for the last three minutes of regulation.  
In order to obtain a probability output from the SVM classifier, we further use Platt's scaling \cite{platt1999probabilistic}: 

\begin{equation}
\Pr(H=1| \mathbf{x})= \frac{1}{1+\exp{(Af(x)+B)}}
\label{eq:platt}
\end{equation}
where $f(x)$ is the uncalibrated value produced by the SVM classifier: 

\begin{equation}
f(x) = \sum_{i} (\alpha_i y_i k(\mathbf{x}_i\cdot\mathbf{x}))+ b
\label{eq:svm}
\end{equation}
where $k(\mathbf{x},\mathbf{x}')$ is the kernel used for the SVM.   
Figure \ref{fig:iwinrNFL} depicts the simple flow chart of {\method}. 


\begin{figure}[t]
\begin{center}
\includegraphics[scale=0.35]{plots/iwinrNFL.pdf}%\vspacecap
 \caption{{\method} includes a linear and a non-linear component.}
 \label{fig:iwinrNFL}
\end{center}
\end{figure}

In order to describe the status of the game we use the following variables:

\begin{enumerate}
\item {\bf Ball Possession Team:} This binary feature captures whether the home or the visiting team has the ball possession
\item {\bf Score Differential:} This feature captures the current score differential (home - visiting)
\item {\bf Timeouts Remaining:} This feature is represented by two independent variables - one for the home and one for the away team - and they capture the number of timeouts remaining for each of the teams
%\item {\bf Quarter:} This feature captures the current quarter of the game
%\item {\bf Time Remaining:} This feature captures the time (in seconds) remaining for the current quarter to end
\item {\bf Time Elapsed: } This feature captures the time elapsed since the beginning of the game
\item {\bf Down:} This feature represents the down of the team in possession
\item {\bf Field Position:} This feature captures the distance covered by the team in possession from their own yard line
\item {\bf Yards-to-go:} This variables represents the number of yards needed for a first down
\item {\bf Ball Possession Time: } This variable captures the time that the offensive unit of the home team is on the field 
\item {\bf Ranking Differential: } This variable represents the difference of the win percentage for the two team (home - visiting)
\end{enumerate}

The last independent variable is representative of the power ranking difference between the two teams. 
Most of the existing models that include such a variable are using the Vegas line spread for each game.  
We choose not to do so for the following reason.  
The objective of the Vegas line is not to predict game outcomes but rather distribute money across the different bets.  
Exactly because of this objective the line is changing during the week before the game.  
While this line can change due to new information for the competing teams (e.g., injury updates), the line is mainly changing when a particular team has accumulated the majority of the bets. 
In this case it will also be hard to choose which line to use (e.g., the opening, the closing or some average of them).  
Therefore, we choose to use the win percentage differential of the two teams as an indicator of their strength (even though this has its own issues given the uneven schedule in NFL).  
However, note that if one would like to use the point spread as a variable this can be easily incorporated in the model. 
Table \ref{tab:iwinrnfl} presents the coefficients of the logistic regression model of {\method} with standardized independent variables for better comparisons. 


\begin{table}[ht]
\begin{center}
\def\sym#1{\ifmmode^{#1}\else\(^{#1}\)\fi}
\begin{tabular}{l*{1}{c}}
\toprule
                    &\multicolumn{1}{c}{(1)}\\
                    &\multicolumn{1}{c}{Winner}\\
\midrule
Possession Team (H)         &      0.41\sym{***}\\
                    &     (49.19)         \\
\addlinespace
Score Differential           &      3.59\sym{***}\\
                    &    (247.34)         \\
\addlinespace
Home Timeouts           &     0.12\sym{***}\\
                    &      (8.74)         \\
\addlinespace
Away Timeouts           &     -0.11\sym{***}\\
                    &    (-12.47)         \\
\addlinespace
Ball Possession Time  &     -0.05.\\
                    &    (-1.66)         \\
\addlinespace
Time Lapsed       &   -0.05.\\
                    &      (-1.66)         \\
\addlinespace
Down                &   -0.01         \\
                    &      (0.04)         \\
\addlinespace
Field Position            &   0.02\sym{**} \\
                    &      (2.71)         \\
\addlinespace
Yards-to-go                &  -0.01         \\
                    &      (0.23)         \\
\addlinespace
Rating differential         &       0.75\sym{***}\\
                    &     (80.47)         \\
\addlinespace
Intercept            &       0.57\sym{*}\\
                    &    (2.09)         \\
\midrule
Observations        &      338,294         \\
\bottomrule
\multicolumn{2}{l}{\footnotesize \textit{t} statistics in parentheses}\\
\multicolumn{2}{l}{\footnotesize \sym{$_.$} \(p<0.1\), \sym{*} \(p<0.05\), \sym{**} \(p<0.01\), \sym{***} \(p<0.001\)}\\
\end{tabular}
\end{center}
\caption{Standardized logisitic regression coefficients for {\method}.}
\label{tab:iwinrnfl}
\end{table}


As we can see, as one might have expected the current scoring differential exhibits the strongest correlation with the in-game win probability.  
The only factors that do not appear to be statistically significant predictors of the dependent variable are the down and the yards-to-go. 
Even though the corresponding coefficients are negative as one might have expected (e.g., being at an earlier down gives you more chances to advance the ball), they are not significant in estimating the win probability. 
On the contrary, all else being equal timeouts appear to be quiet important since they can help a team stop the clock, while teams with better win percentage appear to have an advantage as well, since this can be a sign of a better team. 
In the following section we provide a detailed evaluation of {\method}.
%\input{sufficient_conditions}
% \vspace{-0.20in}
%\subsection{Analysis:}


\textbf{Use of Multiple Projection Heads:} The use of different projection heads for each view on OpenImages classification gives us a boost of $1.1$ mAP on Obj-Obj+Dilate crop. Pre-training on COCO and finetuning on VOC dataset for object-detection task gives a boost of $0.4$ mAP. Hence using multiple projection heads results in a consistent improvement. 

\textbf{Varying Dilation Parameter:} Table 3 (appendix) shows the effect of varying the dilation parameter. A sweet spot exists at a moderate dilation value of $\delta=0.1$ for COCO object detection. 

% \textbf{Computational Cost:} BING adds negligible time to the pre-training. Generating object proposals takes ~29 mins for the full OHMS dataset (one-time cost) and ~16 mins for COCO. Instead of pre-generating, adding the BING operator to the data loader pipeline has a trivial overhead (+$0.1\%$). %As an example, the wall-clock time taken for 1 epoch of training is 1'46'' for the Dense-CL baseline and 1'45'' for our method.
% \textbf{}



%Between two views, we measure the number of common pixels; and then measure the fraction of these common pixels that overlap with a ground truth bounding box (object). We find that this fraction for COCO is $99\%$ for object-scene crops and $92.1\%$ for the scene-scene crop. In the case of OpenImages-Subset, the numbers are, respectively, $99.1\%$ and $87.3\%$. This is another way of seeing that OpenImages-Subset can benefit more from object-scene crops, borne out by the numbers in Tables \ref{tab:ssl_comparison_classification} and \ref{tab:coco_detection}. 


% \as{Shlok: could you please make this description a little better and clear?}
% We find the overlapping pixels between two crops ($C_{int} = C_1 \cap C_2$). Next we calculate intersection of $C_{int}$  with the most overlapping ground truth object ($O$) and calculate the score $\frac{C_{int} \cap O}{C_{int}}$ for each image and average it. 
% To do this, we calculate the \% intersection of the most overlapping ground truth object with the inter
% Next we try to find the probability of an actual ground truth object co-occuring in between two crops. We find  object-overlap between both Scene-Object crops and Scene-Scene crops. To do this we firstly calculate the overlapping region between two crops. Overlapping region is the area of overlap between two crops before the resize operation. Then for all the ground truth objects present in the original image  we find the object with maximum overlap in the overlapping region. Intuitively for a object to have high overlap, the object should be present in both the crops. 

% Similarly instead of taking an crop with maximum overlap we calculate average of all the crops that are present in the image. We find this average probability to be 65.12 \% for Object-Scene crop and 73.47 \% for Scene-Scene crop. 
% This is consistent with the findings of the InfoMin \cite{tian2020contrastive} that there is a tradeoff between how much information views can share.  

% Similarly in the case of OpenImages we can see from Fig \ref{fig:radius_openimages} that as we increase the radius of the object-object crops the performance firstly increases and then decreases, suggesting there is a sweet point on mutual information on OpenImages dataset as well.
% \\

% \textbf{Performance on 5 classes per image images?}









%\input{sec_solution}
%\input{sec_examples}
\bibliographystyle{IEEEtran}
\bibliography{ref,ref1,sueh,trialout}

	\section{Proof of Proposition~\ref{prop-quasi-order}}
\begin{proof}
	We will prove the result for relation $\sqsubseteq$, the proof for $\preceq$ being similar. We need to prove that $\sqsubseteq$ is reflexive and transitive. For reflexivity, it is obvious that since $G\subseteq G$ for any goal $G$, we have $G\sqsubseteq_\theta G$ for the empty substitution $\theta$. For transitivity, suppose that for goals $G_1$, $G_2$ and $G_3$, it holds that $G_1 \sqsubseteq_{\theta_1} G_2$ and $G_2 \sqsubseteq_{\theta_2} G_3$. Then by Definition~\ref{def-generalization}, there exist sets of atoms $\Delta_1$ and $\Delta_2$ such that $G_1\theta_1 \cup \Delta_1 = G_2$ and $G_2\theta_2\cup\Delta_2 = G_3$. In other words it holds that $(G_1\theta_1\cup\Delta_1)\theta_2\cup\Delta_2 = G_3$ or equivalently, $(G_1\theta_1)\theta_2 \cup (\Delta_1\theta_2 \cup\Delta_2) = G_3$. As the composition of two substitutions is a substitution, by defining $\theta_3 = \theta_2\circ\theta_1$ and $\Delta_3 = \Delta_1\theta_2 \cup\Delta_2$, we have $G_1\theta_3 \cup \Delta_3 = G_3$, so $G_1\sqsubseteq_{\theta_3} G_3$, which concludes the proof.
\end{proof}
	

	\section{Proof of Proposition~\ref{prop-msg-lcg}}
	
		First, observe the following property that holds for both relations, essentially stating that a common generalization that is not a lcg has a direct extension obtained by the addition of one atom. 
		
		\begin{proposition}\label{prop-lcg-extensible}
			Let $G_1, \dots, G_n$ and $G$ be goals such that $G$ is a $\leqslant$-common generalization, but not a $\leqslant$-lcg, of $\{G_1, \dots, G_n\}$. Then there exists an atom $A\notin G$ such that $G\cup\{A\}$ is a $\leqslant$-common generalization of $\{G_1, \dots, G_n\}$.
		\end{proposition} 
	
	
		\begin{proof}
		
		Let us suppose the existence of some goal $G$, a $\leqslant$-common generalization that is not a $\leqslant$-lcg of $\{G_1, \dots, G_n\}$, and let us try and extend $G$ into a $\leqslant$-common generalization $G\cup\{A\}$ with $A\notin G$ an atom. As $G$ is not a lcg, there must exist another goal $G'$ being a $\leqslant$-lcg of $G_1$ and $G_2$ and obviously we have $|G'|>|G|$. As a consequence at this point there are three groups of atoms that can be identified: let us denote by $\hat{A}_1, \dots, \hat{A}_p$ the $p (\ge 0)$ atom(s) that are both in $G$ and in $G'$; by $A_1, \dots, A_m$ the $m (\ge 0)$ atom(s) that are part of $G$ but not of $G'$; and by $B_1, \dots, B_l$ the $l (\ge 1)$ atom(s) that are part of $G'$ but not of $G$. For an element $A$ of any of these sets, we denote by $A^1, \dots, A_n$ the atom in respectively $G_1, \dots, G_n$ whose anti-unification led to having $A$ as part of the generalizations.
		
		From the fact that $|G'|>|G|$ it follows that $l>m$. Now each $A_i (i \in 1..m)$ is such that $\exists  h\in 1..n : A_i^h\in \{B_i^h|i\in 1..l\}$: if not, it would be possible to add an atom generalizing $\{A_i^1, \dots, A_i^n\}$ (such as $A_i$) in $G'$ and get a larger generalization, which is impossible given that $G'$ is a lcg. Also note that for two atoms $B_i$ and $B_j (1\le i < j \le l)$, for $g, h \in 1..n : g\neq h$, if $B_i^g$ is anti-unifiable with an atom $B_j^h$ then $B_j^g$ is also anti-unifiable with $B_i^h$ (as it means that all four base atoms are a call to one and the same predicate (with relation $\sqsubseteq$) or have the exact same inner structure save for variables (with relation $\preceq$)), so that it is possible to switch the atoms $B_i^g$ and $B_j^h$, compute the anti-unification of $\{B_i^g, B_j^h)$ and $(B_j^g, B_i^h)$, and get an equally valid anti-unification. Thanks to this we can, where necessary, perform switches so as to rearrange the atoms $B_i (1\le i \le l)$ into $\{\hat{B}_i|i \in 1..l\}$ in such a way that $\{A_i|i\in 1..m\} \subset \{\hat{B}_i|i \in 1..l\}$ and for each atom $\hat{B}_i$, either $\hat{B}_i \in \{A_k|k\in 1..m\}$ or $\exists g, h \in 1..n : g\neq h \wedge \hat{B}_i^g\notin \{A_k^g|k\in 1..m\}\wedge \hat{B}_i^h\notin \{A_k^h|k\in 1..m\}$. We can now define a new generalization $\hat{G}$ defined as the union of these rearranged atoms and those that are common to $G$ and $G'$, i.e. $\hat{G} = \{\hat{A_i}|i\in 1..p\}\cup\{\hat{B}_i|i\in 1..l\}$. Since $|\hat{G}| = |G'|$ and $G\subset \hat{G}$, it suffices to add one of the atoms $A \in \hat{G}\setminus G$ to $G$ in order to obtain $G\cup\{A\}$, a $\leqslant$-common generalization of $\{G_1, \dots, G_n\}$ by construction.
	\end{proof}


		Next, we prove Proposition~\ref{prop-msg-lcg}.
		
	\begin{proof}
	We prove that any $\leqslant$-msg is a $\leqslant$-lcg by contradiction. Let us suppose that some goal $G$ is both a $\leqslant$-msg and not a $\leqslant$-lcg of the set of $\{G_1, \dots, G_n\}$. According to Proposition~\ref{prop-lcg-extensible} it must then be possible to select an atom $A \notin G$ such that $G\cup\{A\}$ is a $\leqslant$-common generalization of $\{G_1, \dots, G_n\}$. Since $A\notin G$ and any atom has a $\tau$-value of at least 1, it follows that $|\tau(G\cup\{A\})|>|\tau(G)|$. Consequently $G$ cannot be a $\leqslant$-most specific generalization of $G_1$ and $G_2$: a contradiction.
	
	As for the fact that any $\preceq$-lcg is a $\preceq$-msg, we prove this also by contradiction. Let $G$ represent a $\preceq$-lcg of the set of goals $\{G_1, \dots, G_n\}$ and let us suppose that $G$ is not a $\preceq$-msg. Then there must exist another goal that is a $\preceq$-msg of $\{G_1, \dots, G_n\}$, say $G'$, such that $|\tau(G')|>|\tau(G)|$ and, according to the first part of the proposition, $|G'|=|G|$. 
	Now, observe that for a set of atoms $\{A_1, \dots, A_n\}$ to be anti-unified with $\preceq$ into an atom $A$, necessarily all $A_i (1\le i\le n)$ must have the same $\terms$-value. Indeed relation $\preceq$ is defined upon renamings so that only variables (having a $\terms$-value of zero) are impacted by the generalization process. Therefore, the only possibility for the inequality $|\tau(G')|>|\tau(G)|$ to be true is that some atoms $B_1, \dots, B_n$ of respective goals $G_1,\dots,G_n$ appear in a generalized form (say $B$) in $G'$, while these atoms have not been generalized in $G$. This means that it is possible to add a (possibly renamed) version of $B$ in $G$ and obtain $G\cup\{B\}$, also a $\preceq$-common generalization and larger than $G$: a contradiction.
	\end{proof}

	
	\section{Detailed proof of Lemma~\ref{lemma-au-op}}
	\begin{proof}
The lemma will be shown correct by the definition of three anti-unification operators. A first anti-unification operator, based on $\sqsubseteq$ is the following. 

\begin{definition}%[Simple anti-unification operator]
	\label{def-atoms-au}
	Given a variabilization function $\Phi$, let $\au^\Phi_\sqsubseteq$ (or simply $\au_\sqsubseteq$ if $\Phi$ is clear from the context) denote the anti-unification operator such that for any two atoms $A = a(t^A_1, \dots, t^A_n)$ and $B = b(t^B_1, \dots, t^B_m)$, it holds that \[\au^{\Phi}_\sqsubseteq(A,B)=\left\{\begin{array}{l}
		a\big(\Phi(t^A_1, t^B_1), \dots, \Phi(t^A_n, t^B_n)\big) \\ \qquad  \mbox{if } a = b \mbox{ and } n = m\\
		\bot \\
		\qquad \mbox{otherwise}\\
	\end{array}\right. \]
\end{definition} 

\begin{example}\label{ex-au-sq}
	In Table~\ref{table:sqsubseteq}, we show three atomic anti-unification results obtained by the application of $\au_\sqsubseteq^\Phi$ with $\Phi$ a given variabilization function. Note how in the first example, the predicates used in $A_1$ and $A_2$ differ (resp. $p/2$ and $p/3$), leading to an impossible anti-unification.
\end{example}

\begin{table*}
	\caption{Example results for $au_\sqsubseteq^\Phi$}
	\label{table:sqsubseteq}
	\centering
	\begin{tabular}{l|l|l}
		%\hline 
		$\bm{A_1}$ & $\bm{A_2}$ & $\bm{\au_\sqsubseteq^\Phi(A_1, A_2)}$\\\hline 
		$p(X, 5, q(Y,4))$ & $p(W,t(Z))$ & $\bot$\\\hline 
		$p(r(X,3), t(5))$ & $p(W, t(Z))$ & $p(\Phi(r(X,3),W), \Phi(t(5), t(Z)))$\\\hline 
		$p(r(X,3), t(Y))$ & $p(r(W,3),t(Z))$ & $p(\Phi(r(X,3),r(W,3)), \Phi(t(Y),t(Z)))$ %\\\hline 
	\end{tabular} 
\end{table*}

Note that the anti-unification operator defined in Definition~\ref{def-atoms-au} differs from the traditional subsumption operator in the ordered case (i.e. when goals are ordered sequences of atoms). The difference comes from the fact that our goals being sets, all the possible couples of atoms have to be considered, whereas traditional subsumption must handle one atom at the time, making the anti-unification operator more straigtforward.     

Let us now introduce a second anti-unification operator that will allow to compute a $\preceq$-lcg. 
%and to prove Theorem~\ref{thm-preceq-lcg}. 
Since the result of this operator should be a $\preceq$-common generalization, the operator need only to anti-unify the \textit{variables} occurring at the corresponding positions in the atoms under investigation. 
The operator must thus go deeper into the term structure of the atoms than $\au_\sqsubseteq$ does, as it needs to only anti-unify those atoms that harbor the exact same structure at the level of their non-variable terms.
\begin{definition}%[Variable anti-unification operator]
	\label{def-term-au-through-variables}
	Given some variabilization function $\Phi$, let $\au^\Phi_\preceq$ (or simply $\au_\preceq$ if $\Phi$ is clear from the context) denote the function such that for any two terms $T = t(t_1, \dots, t_n)$ and $U = u(u_1, \dots, u_m)$ it holds that
	\[\au^\Phi_\preceq(T,U)=\left\{\begin{array}{l}
		\Phi(T,U) 
		\\ \qquad \mbox{if } T\in\mathcal{V}\mbox{ and } U\in\mathcal{V}
		\\t\big(\au^\Phi_\preceq(t_1,u_1), \dots, \au^\Phi_\preceq(t_n, u_n)\big) 
		\\ \qquad \mbox{if } t = u \mbox{ and } n = m 
		\\ \qquad \mbox{and } \forall i \in 1..n: \au^\Phi_\preceq(t_i,u_i)\neq\bot
		\\ \bot
		\\ \qquad  \mbox{otherwise}
	\end{array}\right.\]
	and for any two atoms $A = a(t^A_1, \dots, t^A_n)$ and $B = b(u^B_1, \dots, u^B_m)$, it holds that
	\[\au^\Phi_\preceq(A,B)=\left\{\begin{array}{l}
		a\big(\au^\Phi_\preceq(t^A_1, u^B_1),\dots, \au^\Phi_\preceq(t^A_n, u^B_n)\big) 
		\\ \qquad \mbox{if } a = b \mbox{ and } n = m 
		\\ \qquad \mbox{and } \forall i \in 1..n: \au^\Phi_\preceq(t^A_i, u^B_i) \neq\bot
		\\ \bot  
		\\ \qquad \mbox{otherwise}
	\end{array}\right.\]
\end{definition}

\begin{example}
	In Table~\ref{table:preceq}, we treat the anti-unification of the same atoms as above, this time with the use of $\au_\preceq^\Phi$ with $\Phi$ a given variabilization function. Note how $\au_\preceq$ behaves differently than $\au_\sqsubseteq$ on the second and third couple of atoms as it requires its arguments to exhibit a similar structure in order to be anti-unifiable.
\end{example}

\begin{table*}
	\caption{Example results for $au_\preceq^\Phi$}
	\label{table:preceq}
	\centering
	\begin{tabular}{l|l|l}
		%\hline
		$\bm{A_1}$ & $\bm{A_2}$ & $\bm{\au_\preceq^\Phi(A_1, A_2)}$\\\hline 
		$p(X, 5, q(Y,4))$ & $p(W,t(Z))$ & $\bot$\\\hline 
		$p(r(X,3), t(5))$ & $p(W, t(Z))$ & $\bot$\\\hline 
		$p(r(X,3), t(Y))$ & $p(r(W,3),t(Z))$ & $p(r(\Phi(X,W),3), t(\Phi(Y,Z)))$ %\\\hline
	\end{tabular}
\end{table*}

Now, in order to compute $\sqsubseteq$-msgs, we need a more precise anti-unification operator: one that goes deeper into detail when comparing atoms so as not to miss their maximal common structure. 
\begin{definition} %[Deep anti-unification operator]\label{def-deep-operator}
	Given some variabilization function $\Phi$, let $\dau^\Phi_\sqsubseteq$ (or simply $\dau_\sqsubseteq$ if $\Phi$ is clear from the context) denote the function such that for any two terms $T = t(t_1, \dots, t_n)$ and $U = u(u_1, \dots, u_m)$ it holds that 
	\[\dau^\Phi_\sqsubseteq(T,U)=\left\{\begin{array}{l}
		
		t\big(\dau^\Phi_\sqsubseteq(t_1,u_1), \dots, \dau^\Phi_\sqsubseteq(t_n, u_n)\big) 
		\\ \qquad \mbox{if } t = u \mbox{ and } n = m 
		\\ \qquad \mbox{and } T \notin \mathcal{V} \mbox{ and } U \notin \mathcal{V}
		\\ \Phi(T,U) 
		\\ \qquad \mbox{otherwise}
	\end{array}\right.\]
	and for any two atoms $A = a(t^A_1, \dots, t^A_n)$ and $B = b(u^B_1, \dots, u^B_m)$, it holds that
	\[\dau^\Phi_\sqsubseteq(A,B)=\left\{\begin{array}{l}
		a\big(\dau^\Phi_\sqsubseteq(t^A_1, u^B_1),\dots, \dau^\Phi_\sqsubseteq(t^A_n, u^B_n)\big) 
		\\ \qquad \mbox{if } a = b \mbox{ and } n = m 
		\\ \bot
		\\ \qquad \mbox{otherwise}
	\end{array}\right.\]
\end{definition}

When applied on atoms, it is easy to see that $\dau_\sqsubseteq$ is an anti-unification operator based on relation $\sqsubseteq$.

\begin{example}
	Let us once more consider the anti-unification of the atoms introduced in Example~\ref{ex-au-sq}. This time we make use of $\dau_\sqsubseteq^\Phi$ with $\Phi$ a given variabilization function, to anti-unify the three pairs of atoms. The result is shown in Table~\ref{table:dau}. Notice how the operator preserves as much non-variable atomic structure as possible in the process.
\end{example}
\begin{table*}
	\caption{Example results for $dau_\sqsubseteq^\Phi$}
	\label{table:dau}
	\centering
	\begin{tabular}{l|l|l}
		%\hline 
		$\bm{A_1}$ & $\bm{A_2}$ & $\bm{\dau_\sqsubseteq^\Phi(A_1, A_2)}$\\\hline 
		$p(X, 5, q(Y,4))$ & $p(W,t(Z))$ & $\bot$\\\hline 
		$p(r(X,3), t(5))$ & $p(W, t(Z))$ & $p(\Phi(r(X,3),W),t(\Phi(5,Z)))$\\\hline 
		$p(r(X,3), t(Y))$ & $p(r(W,3),t(Z))$ & $p(r(\Phi(X,W),3), t(\Phi(Y,Z)))$ %\\\hline 
	\end{tabular} 
\end{table*}
The existence of these operators proves Lemma~\ref{lemma-au-op}.
	\end{proof}
	
	\section{Proof of Theorem~\ref{thm-ausqsubseteq}}
	\begin{proof}
	Obviously the $\au_\sqsubseteq(A_1,A_2)$ operation can be achieved in a time linear with respect to the arity $n$ of $A_1$. In the worst case, the operation needs to be performed for each atom in $G_1$ with respect to each atom in $G_2$. Hence the first result.
	
	
	It is also easy to see that the $\au_\preceq(A_1,A_2)$ operation can be achieved in linear time with respect to the maximum number of function applications in the argument terms of the atom $A_1$ under scrutiny. In the worst case, the operation needs to be performed for each atom in $G_1$ with respect to each atom in $G_2$. Hence the second result.
	\end{proof}
		
%	
%	\section{Maximum Weight Matching of Example~\ref{example-mwm}}
%	See Fig.~\ref{fig:mwm}.
	
	
	\section{Proof of Theorem~\ref{thm-sqsubseteq-msg}}
\begin{proof}
	First note how the atomic anti-unifications and the weights of the associated bipartite graph's edges can be computed simultaneously, by working out $\dau_\sqsubseteq(A_1,A_2)$ for each possible couple $(A_1,A_2)$ in $G_1\times G_2$ and keeping account of the number of non-variable terms encountered during the operation (or $-1$). Given that $\dau_\sqsubseteq(A_1,A_2)$ can obviously operate linearly in the number of terms appearing in $A_1$ (denoted $N$), the computation of all weights is carried out in a time not exceeding $\mathcal{O}(|G_1|.|G_2|.N)$.
	
	Now the obtained assignment problem can be solved by existing algorithms (such as the Hungarian method~\cite{assignment}) that compute a MWM in $\mathcal{O}(n^3)$, where $n$ is the number of vertexes appearing on the side of the bipartite graph that has the most vertexes. In our case, there are $|G_1|$ left vertexes and $|G_2|$ right vertexes so that a MWM algorithm can be ran in $\mathcal{O}(max(|G_1|,|G_2|)^3)$.
\end{proof}
	
	\section{Proof of Theorem~\ref{thm-dataflow-np-complete}}
\begin{proof}
	First, let us consider MSG-MIN. It clearly belongs to NP. Indeed, given an arbitrary generalization $G$, we can verify in polynomial time whether it is a most specific generalization. The procedure is as follows. We can compute at least one $\leqslant$-msg, say $G'$, in polynomial time (see Theorem~\ref{thm-sqsubseteq-msg}). It suffices then to compare the $\tau$-value of $G'$ with that of $G$ in order to decide whether $G$ is a msg. Next, verifying whether the number of variables in $G$ is bounded by a constant is obviously achieved in polynomial time as well.
	
	In order to prove NP-hardness, we will construct a reduction from the well-known set cover problem (known to be NP-complete~\cite{karp}) to MSG-MIN. The set cover problem in its decision-problem version (denoted SCP), can be formulated as follows. Given a constant $p \in \mathbb{N}_0$, a universe $U$ of values and a collection $S$ composed of $n$ sets $\{S_1, \dots, S_n\}$ that cover $U$, i.e. $U = \underset{i=1}{\overset{n}{\cup}}S_i$, the problem is to decide whether there exists $p$ subsets from $S$ that still cover $U$.
	
	We can transform an arbitrary instance of SCP into MSG-MIN as follows. Let us consider without loss of generality a universe $U$ where the elements are lowercase strings and $p \in \mathbb{N}_0$ a constant. Given a collection of sets $S=\{S_1, \dots, S_n\}$ we construct an instance of MSG-MIN as follows. In our construction we use $n+1$ different variables, namely $V$ and $(W_i)_{i\in1..n}$. We use $x_j$ to denote some element of $U$; these elements being strings, we can easily use them as predicate names. The construction of goals $G_1$ and $G_2$ proceeds then as follows:
	
	\begin{algorithmic}
		\State $G_1 = \{\}$ 
		\State $G_2 = \{\}$ 
		\For {each ($S_i \in S$)}
		\For {each ($x_j \in S_i$)}
		\State $G_1 \gets G_1\cup \{x_j(V)\}$				
		\State $G_2 \gets G_2\cup \{x_j(W_i)\}$
		\EndFor
		\EndFor
	\end{algorithmic}
	Note that all the atoms in $G_1$ have the same argument (namely the variable $V$) and there are as many atoms in $G_1$ as there are distinct elements in $S$. In $G_2$, however, there is an atom of the form $x_j(W_i)$ for each element $x_j$ occurring in $S_i$.
	
	The construction is such that any $\leqslant$-msg of $G_1$ and $G_2$ will be a version of $G_1$ where each occurrence of a variable $V$ is replaced by $\Phi(V, W_k)$ for some $W_k\in\vars(G_2)$ (where $\Phi$ is a variabilization function). Now, introducing such a variable $\Phi(V, W_k)$ in the generalization will allow to reuse the same variable for all the atoms $x_j(V)$ in $G_1$ that have a corresponding $x_j(W_k)$ in $G_2$. In other words, choosing to have variable $\Phi(V,W_k)$ in the $\leqslant$-msg is the same as selecting the subset $S_k$ to be part of the solution of the set cover problem. Consequently, using this transformation MSG-MIN can be used to decide SCP. Since the transformation can clearly be done in polynomial time, and since SCP is known to be NP-complete, we conclude that MSG-MIN is NP-complete as well.
	
	Now let us prove the result for LCG-MIN. We know that a $\leqslant$-lcg can be computed in polynomial time, so that a positive instance of LCG-MIN can be verified just like it can be for MSG-MIN. Moreover, the absence of non-variable terms in the transformation from SCP to MSG-MIN above allows us to reuse said transformation as-is to prove that LCG-MIN is NP-hard. Indeed, since the obtained anti-unification problem doesn't harbor terms other than variables, it is both an instance of MSG-MIN and LCG-MIN. LCG-MIN is therefore also NP-complete.
\end{proof}
	
	\section{Proof of Theorem~\ref{thm-inj-np-complete}}
\begin{proof}
	INJ is in NP: given a relation $\leqslant^\iota$, goals $G_1$ and $G_2$ and a substitution (or renaming) $\theta$, it is possible to verify in polynomial time whether the application of $\theta$ on $G_1$ results on a subset of $G_2$ or not.
	As for the proof of NP-hardness, we refer to~\cite{gen} in which the problem ``is $G_1$ a $\preceq^\iota$-lcg of $G_1$ and $G_2$?'' has been proved to be NP-complete using a polynomial reduction from the Induced Subgraph Isomorphism Problem~\cite{SYSLO198291}. The same reduction can be used for the other cases, leading to the conclusion that INJ is NP-complete.
\end{proof}
	

%\begin{IEEEbiography}[{\includegraphics[width=1in,height=1.25in,clip,keepaspectratio]{YinSun.eps}}]{Yin Sun} received his B.Eng. and Ph.D. degrees in Electronic Engineering from Tsinghua University, in 2006 and 2011, respectively. He was a Postdoctoral Scholar and Research Associate at the Ohio State University from 2011-2017 and an Assistant Professor in the Department of Electrical and Computer Engineering at Auburn University from 2017-2023. He is currently the Godbold Associate Professor in the Department of Electrical and Computer Engineering at Auburn University, Alabama. His research interests include Wireless Networks, Machine Learning, Semantic Communications, Age of Information, Information Theory, and Robotic Control. He is also interested in applying AI and Machine Learning techniques in Agricultural, Food, and Nutrition Sciences. 
%
%He has been an Associate Editor of the \emph{IEEE Transactions on Network Science and Engineering}, an Editor of the \emph{Journal of Communications and Networks}, an Editor of the \emph{IEEE Transactions on Green Communications and Networking}, a Guest Editor of five special journal issues, and an Organizing Committee Member of several conferences. He founded the Age of Information (AoI) Workshop in 2018 and the Modeling and Optimization in Semantic Communications (MOSC) Workshop in 2023. His articles received the Best Student Paper Award of the IEEE/IFIP WiOpt 2013, Best Paper Award of the IEEE/IFIP WiOpt 2019, runner-up for the Best Paper Award of ACM MobiHoc 2020, and 2021 Journal of Communications and Networks (JCN) Best Paper Award. He co-authored a monograph \emph{Age of Information: A New Metric for Information Freshness}. He received the Auburn Author Award of 2020, the National Science Foundation (NSF) CAREER Award in 2023, and was named a Ginn Faculty Achievement Fellow in 2023. He is a Senior Member of the IEEE and a Member of the ACM. 
%\end{IEEEbiography}
%
%\begin{IEEEbiography}[{\includegraphics[width=1in,height=1.25in,clip,keepaspectratio]{SK.eps}}]{Sastry Kompella} earned his Ph.D. in electrical and computer engineering from the Virginia Polytechnic Institute and State University in 2006. Currently, he serves as the Chief Scientist for Nexcepta, Inc., an advanced R\&D company providing cutting-edge technical solutions and mission-critical capabilities to the Department of Defense (DoD). Prior to this role, he held the position of Section Head for the Wireless Network Research Section within the Information Technology Division at the U.S. Naval Research Laboratory in Washington, DC, USA. His research interests encompass various aspects of wireless networks, including cognitive radio, dynamic spectrum access, and age of information.
%\end{IEEEbiography}


\end{document}
