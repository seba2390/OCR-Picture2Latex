\section{Introduction}
%{\red ideas: where do we have strong need for multi-server multi-flow scheduling? IoT, social networks, online gaming, news, and notifications.} 
%In computer and communcation networks, 

%The  increased availability of network connected mobile devices has spurred a plethora of industrial and daily life applications involving real-time remote measurement, tracking, and control, which relies heavily on the availability of fresh information updates. 
%
%These applications, whether their end user is a person (e.g., a social network or news feed, driving directions) or device (e.g., industrial environmental monitoring, vehicle sensor status, automated driving), are characterized by a dependence on status updates, that is, information packets that contain recently sampled data. Status updates are desired to be sufficiently \emph{fresh}, or \emph{timely} for the application at hand. 

In many information-update and networked control systems, such as news updates, stock trading, autonomous driving, remote surgery, robotics control, and real-time surveillance, information usually has the greatest value when it is fresh. A metric for information freshness, called  \emph{age of information} or simply  \emph{age}, was introduced in \cite{Song1990,KaulYatesGruteser-Infocom2012}. Consider a flow of status update packets that are sent from a source to a destination through a channel. 
Let $U(t)$ be the time stamp (i.e., generation time) of the {newest update that the destination has received} by time $t$. {Age of information, as a function of time $t$, is defined as} 
$\Delta (t) = t - U(t)$, which is the time elapsed since the newest update was generated.

%how to reduce the age $\Delta (t)$ and keep the information fresh, 

In recent years, there have been a lot of research efforts on (i) analyzing the distributional  quantities of age $\Delta (t)$ for various network models and (ii) controlling $\Delta (t)$ to keep the destination's information as fresh as possible, e.g., 
\cite{SunAoIWorkshop2018,Song1990,KaulYatesGruteser-Infocom2012,2012CISS-KaulYatesGruteser,2012ISIT-YatesKaul,LiInfocom2015,6875100,CostaCodreanuEphremides_TIT, KamKompellaEphremidesTIT,Icc2015Pappas,2015ISITHuangModiano,Suninfocom2016,AgeOfInfo2016,Bedewy2016,BedewyJournal2017,Bedewy2017,BedewyMultihop2017,SunBook,Kosta2017Age,Yates2016, AliTCOM2022,IgorAllerton2016,HsuTWC2017,Vishrant2017,Arunabh2019,He2018,8822722,8812616,9137714,8406891,SunMutualInformation2018,SunNonlinear2019,shisher2021age,ShisherMobiHoc22, shisher2023learning0, shisher2023learning, pan2022age, pan2022optimal, bedewy2021low, ornee2021sampling, bedewy2021optimal, tang2022sampling, ornee2023whittle,yates2021AgeSurvey}. If there is  a single flow of status update packets, the Last Generated, First Served (LGFS) update transmission policy, in which the last generated packet is served the first, has been shown to be (nearly) optimal for minimizing the age process $\{\age(t),t\geq 0\}$ in a stochastic ordering sense for queueing networks with multiple servers or multiple hops \cite{Bedewy2016,BedewyJournal2017,Bedewy2017,BedewyMultihop2017,SunBook}. 
These results hold for arbitrary packet generation times at the information source (e.g., a sensor) and arbitrary packet arrival times at the transmitter's queueing buffer; they also hold for minimizing any non-decreasing functional %\footnote{A functional is a mapping from functions to real numbers.} 
$\phi(\{\age(t),t\geq 0\})$ of the age process $\{\age(t),t\geq 0\}$. If packets arrive at the queue in the order of their generation times, then the LGFS policy reduces to the Last Come, First Served (LCFS) policy, thus demonstrating the (near) age-optimality of the LCFS policy. These studies motivated us to delve deeper into the design of scheduling policies to minimize age of information in more complex networks involving \emph{multiple flows of status update packets} and \emph{transmission errors}, where each flow is from one source node to a destination node. In this scenario, the transmission scheduler must compare not only packets from the same flow, but also packets from different flows. Additionally, the presence of transmission errors adds an additional layer of complexity to the scheduling problem. As a result, addressing these challenges becomes crucial in achieving efficient age minimization in such systems. 

\begin{figure}
\centering 
\includegraphics[width=0.49\textwidth]{model_2.eps} 
\vspace{-0mm}
\caption{System model.}
% work--efficiency ordering holds for any priorities of the jobs.
\label{fig_model} 
\vspace{-5mm}
\end{figure} 

In this paper, we investigate age-optimal scheduling in \emph{continuous-time} and \emph{discrete-time} status updating systems that involve \emph{multiple flows}, \emph{multiple servers}, and \emph{transmission errors}, as illustrated in Figure \ref{fig_model}. Each server can transmit packets to their respective destinations, one packet at a time. Different servers are not allowed to simultaneously transmit packets from the same flow. {We assume that the packet generation and arrival times are \emph{synchronized} across the flows. 
In other words, when a packet from flow $n$ arrives at the queue at time $A_i$, with its generation time denoted as $S_i$ (where $S_i\leq A_i$), one corresponding packet from each flow simultaneously received at time $A_i$, and all of these packets were generated at the same time $S_i$.
%This assumption is a generalized version of  the model in \cite{IgorAllerton2016}.
In practice, synchronized packet generations and arrivals occur when there is a single source and multiple destinations (e.g.,  \cite{IgorAllerton2016}), or in periodic sampling where multiple sources are synchronized by the same clock, which is common in  monitoring and control systems \ifreport
(e.g.,  \cite{Phadke1994,Sivrikaya2004})\fi.}
We develop a unifying sample-path approach and use it to show that the proposed scheduling policies can achieve optimal or near-optimal age performance in a quite strong sense (i.e., in terms of stochastic ordering of age-penalty stochastic processes). 
The contributions of this paper are summarized as follows:
\begin{itemize}
\item Let $\bm{\Delta}(t)$ denote the age vector of multiple flows. We introduce an age penalty function $p_t (\bm\age(t))$ to represent the level of dissatisfaction for having aged information at the destinations at time $t$, where $p_t$ can be any \emph{time-dependent}, \emph{symmetric}, and \emph{non-decreasing} function of the age vector $\bm{\Delta}(t)$. 

\item For continuous-time status updating systems with one or multiple flows, one or multiple servers, and \emph{i.i.d.}~exponential transmission times, we propose a \emph{Preemptive, Maximum Age First, Last Generated First Served (P-MAF-LGFS) scheduling policy}.\footnote{
%We note that this P-MAF-LGFS policy is more general than that presented in \cite{SunAoIWorkshop2018} with the same name, which only applies to single-server systems. the P-MAF-LGFS policy discussed here is applicable to both single-server and multi-server systems. It broadens the scope of the original P-MAF-LGFS policy designed for single-server setups, as introduced in \cite{SunAoIWorkshop2018}, to encompass the more general multi-server scenario.
This new P-MAF-LGFS policy is suitable for both single-server and multi-server systems, whereas the original P-MAF-LGFS policy, as presented in \cite{SunAoIWorkshop2018}, was specifically tailored for single-server scenarios. 
} 
If the packet generation and arrival times are synchronized across the flows, then for any age penalty function $p_t$ defined above, any number of flows, any number of servers, any synchronized packet generation and arrival times, and regardless the presence of transmission errors or not, the P-MAF-LGFS policy is proven to minimize the continuous-time age penalty process $\{p_t (\bm\age(t)), t\geq 0\}$ among all causal policies in a stochastic ordering sense (see Theorem \ref{thm1} and Corollary \ref{coro1}). Theorem \ref{thm1} is more general than \cite[Theorem 1]{SunAoIWorkshop2018}, as the latter was established for the special case of single-server status updating systems without transmission errors. In addition, if packet replication is allowed, we show that a \emph{Preemptive, Maximum Age First, Last Generated First Served scheduling policy with packet Replications (P-MAF-LGFS-R)} is age-optimal for minimizing the age penalty process $\{p_t (\bm\age(t)), t\geq 0\}$ in terms of stochastic ordering (see Corollary \ref{corollary_new}). 


\item For continuous-time status updating systems with one or multiple flows, one or multiple servers, and \emph{i.i.d.}~New-Better-than-Used (NBU) transmission times (which include exponential transmission times as a special case),
 age-optimal multi-flow scheduling is quite difficult to achieve. In this case, 
we consider an age lower bound called the \emph{Age of Served Information} and propose a \emph{Non-Preemptive, Maximum Age of Served Information First, Last Generated First Served (NP-MASIF-LGFS) scheduling policy}. The NP-MASIF-LGFS policy is shown to be near age-optimal. Specifically, it is within an additive gap from the optimum for minimizing the expected time-average of the average age of the flows, where the gap is equal to the mean transmission time of one packet (see Theorem \ref{thm3} and Corollary \ref{coro4}). This additive sub-optimality gap is quite small. 
%Numerical evaluations are provided to verify our (near) age optimality results. %{\blue Some possible extensions are discussed at the end of the paper.}

\item For discrete-time status updating systems with one or multiple flows and one or multiple servers, we propose a \emph{Discrete Time, Maximum Age First, Last Generated First Served (DT-MASIF-LGFS) scheduling policy}. If the packet generation and arrival times are synchronized across the flows, then for any age penalty function $p_t$, any number of flows, any number of servers, any synchronized packet generation and arrival times, and regardless the presence of transmission errors or not, the DT-MAF-LGFS policy is proven to minimize the discrete-time age penalty process $\{p_t (\bm\age(t)), t= 0, T_s, 2 T_s, \ldots\}$ among all causal policies in a stochastic ordering sense, where $T_s$ is the fundamental time unit of the discrete-time systems (see Theorem \ref{thm4}). %Theorem \ref{thm4} is more general than \cite[Theorem 1]{IgorAllerton2016}, because the former allows for more general packet generation and arrival times, and a broader range of age penalty functions.


\end{itemize}

%A comparison with related work is presented in Section \ref{sec_related_work}. 
\ifreport
Our results can be potentially applied to: (i) cloud-hosted Web services where the servers in Figure \ref{fig_model} represent a pool of threads (each for a TCP connection) connecting a front-end proxy node to clients \cite{Fox:1997:CSN:269005.266662}, (ii) industrial robotics and factory automation systems where multiple sensor-output flows are sent to a wireless AP and then forwarded to a system monitor and/or controller \cite{Gungor2009}, and (iii) Multi-access Edge Computing (MEC) that can process fresh data (e.g., data for video analytics, location services, and IoT) locally at the very edge of the mobile network. % \cite{MEC}. 
\fi

