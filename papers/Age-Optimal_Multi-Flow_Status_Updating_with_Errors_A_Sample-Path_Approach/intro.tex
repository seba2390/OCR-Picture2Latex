% !TEX root = ./Age_of_Info_updates.tex


The increased availability of network connected mobile and portable devices has spurred a plethora of industrial and daily life applications involving remote measurement, tracking, or control. These applications, whether their end user is a person (\eg, a social network or news feed, driving directions) or device (\eg, industrial environmental monitoring, vehicle sensor status, automated driving), are characterized by a dependence on status updates, that is, information packets that contain recently sampled data. Status updates are desired to be sufficiently \emph{fresh}, or \emph{timely} for the application at hand. This freshness is typically constrained by network delay, bandwidth (or more precisely, data rate available in a medium access environment such as a shared wireless link), sampling rate limits imposed by the source (\eg, due to finite battery capacity, and/or intermittently available energy, as in an energy harvesting sensor).

Previous consideration of timeliness of data in networks exists in previous and loosely related threads of literature on database updating \cite{Xiong99, adelberg1995applying, cho2000synchronizing}, scheduling in data warehouses \cite{golab2009scheduling}, web caches \cite{Shenker99}, route caches in ad-hoc network routing \cite{Johnson2002}, though these works did not focus on creating an analytical framework. Relatively recently, \cite{Stratis2009} analyzed how a network service provider should allocate its bandwidth to multiple users so as to optimize a utility function related to freshness. More recently, a metric called \emph{Age of Information} that captures the timeliness of status updates was proposed \cite{2011KaulGruteserRaiKenney, 2011Globecom-KaulYatesGruteser} and has gathered a strong following particularly because of the analytical basis it provides for evaluation and design of update algorithms.  

Consider a flow of packets (status updates) sent by a source (S) to a destination (D). Let $u (t)$ be the time stamp (i.e. generation time) of the \textit{newest update that D has received} by time $t$. \textbf{The status age is defined as} 
\[
\Delta (t) = t - u(t)
\]
Age of Information (AoI) for a flow of status updates usually refers to the time average of $\Delta(t)$. 

At first glance it may appear that AoI is a latency metric closely related to time average delay. However, perhaps counter-intuitively, in many problem setups age behaves differently from delay and in some cases, optimizing for delay can be significantly sub-optimal in terms of AoI. The seminal work in~\cite{KaulYatesGruteser-Infocom2012} modeled the transmission of updates from S to D as a M/M/1 queue with FCFS service. They showed that minimizing status age is neither the same as ensuring small delay seen by the updates nor the same as maximizing the rate of updates (utilization of the server, or communication link.)  

While delay is a per-packet freshness metric, measuring freshness from the perspective of packets, in contrast, age measures freshness from the perspective of the destination.  A low AoI requires not only low delay, but also a sufficiently high rate of updating. On the other hand, in an FCFS discipline, as update rate increases, the AoI will start increasing again after some value of load, due to the queuing delay.  With the intrigue created by the seminal work in~\cite{KaulYatesGruteser-Infocom2012}, some of the initial works on AoI have considered the optimization of update generation rate for First-Come First-Served (FCFS) update policies \cite{Kaul2012,2012ISIT-YatesKaul,2015ISITHuangModiano}. 

In \cite{Icc2015Pappas, CostaCodreanuEphremides2014ISIT} it was found that age can be improved by discarding
old packets waiting in the queue when a new sample arrives. Clearly, 
among all status update packets that are currently available in a queue, the transmission of the youngest packet
will not only minimize the status age at the monitor, but also obviates the
need for transmission of the older outdated packets in the queue under an assumption of Markovian state. This observation makes it quite intuitive that the last-come first-served (LCFS) queueing discipline in which a new status update packet
will preempt any previously queued update packets, will lower age ~\cite{KaulCISS2012}.  In \cite{KaulCISS2012}, the time-average age was characterized for Last-Come First-Served (LCFS) update systems with and without preemption for exponential service times, extended to a gamma service time distribution in \cite{Gamma_dist}, the case with transmission errors was considered in \cite{age_with_delivery_error}.
 
In \cite{age_optimality_multi_server} which analyzed age in a multiserver system with exponential service times, the optimality of the preemptive Last-Come First-Served (LCFS) policy was proved under any arbitrary update arrival 
process and any buffer size. In particular, preemptive LCFS  achieves an age that is
stochastically smaller than that of any causally feasible policy. The result holds for many data freshness metrics, including
a large class of time-average age penalty functions and average peak age. Recently, \cite{AgeDistributionISIT2017} presented an analysis of the probability distribution of AoI and specifically closed form expressions of the stationary distiributions of average and peak AoI in the first-come first-served (FCFS) GI/GI/1 queue. 

A different thread of work focused on the design of service policies to {\emph{control} age: under update timing constraints due to an energy harvesting sender in \cite{BacinogCeranUysal_Biyikoglu2015, 2015ISITYates}; under network delay that is independent of the status update process in \cite{report_AgeOfInfo2016, 2015ISITYates}, and under random packet erasures at a wireless downlink in \cite{IgorAllerton2016}. Furthermore, age-optimal scheduling formulations have been studied in \cite{AtillaInfocom2015, EphremidesSchedulingICC2016, EphremidesSchedulingWiOpt2016, EytanSchedulingISIT2017}.

In continuation of the multiserver formulation analyzed in \cite{Bedewy2016}, a general multi-hop network was recently considered in \cite{Bedewy2017} where the update packets generated at an external source are dispersed throughout the network through a gateway node. It was proved that, if the packet transmission times over the network links are exponentially distributed, then for arbitrary arrival process, network topology, and buffer size at each link, the preemptive Last-Generated First-Served (LGFS) policy achieves smaller age processes at all nodes in the network than any causal policy in the sense of stochastic ordering, and further, minimizes any non-decreasing functional
of the age processes (\eg, time average age penalty, average peak age, etc.) Moreover, it is shown in \cite{Bedewy2017} that the non-preemptive LGFS policy minimizes the age processes at all nodes in the network among all non-preemptive work-conserving policies in the sense of stochastic ordering, for arbitrary general distributions of packet transmission times, which may differ across links.
 
This paper is closest in formulation and methods to \cite{Bedewy2016} and \cite{Bedewy2017}. We formulate and solve the age-optimal multi-flow multi-server scheduling problem. Our formulation is an extension of the formulation in \cite{bedewy}. We consider $n$ indepedent packet streams (flows) to be served by a system with $m$ servers and a single queue of size $m<\infty$ (see Figure \ref{fig:sysmodel}). A possible motivation for the model is a wireless access point serving multiple independent data streams. For this model, we formulate and solve the multiserver scheduling problem with respect to a general family of age-related metrics. To be concrete, we now define age and the family of age-related metrics that are within the scope of this paper.

\subsection{Age Metrics}
At any time $t$, the freshest packet delivered to the destination node $d_n$ is time-stamped with 
\begin{align}
U_{n} (t) = \max\{S_{n,i}: D_{n,i} \leq t\}. \nonumber
\end{align}
Hence, the \emph{age-of-information}, or simply the \emph{age}, of flow $n$ is defined as
\begin{align}%\label{eq_age}
\Delta_{n} (t) = t - U_{n} (t) ,\nonumber
\end{align}
which represents the staleness of the information available at node $d_n$.
%\begin{align}
%U_{n}(t) = \max\{S_{n,i}: D_{n,i} \leq t\}.
%\end{align}
%to denote the time-stamp of the freshest update packet received by  up to time $t$. 
%The \emph{age-of-information}, or simply the \emph{age}, of flow $n$ is defined as
%\begin{align}\label{eq_age}
%\Delta_{n} (t) = t - U_{n}(t),
%\end{align}
%which represents the staleness of the available information at  flow $n$. 
Let $\bm{\Delta}(t)=(\Delta_{1} (t),\ldots,\Delta_{n} (t))$ denote the age vector of all $N$ flows at time $t$. 
Some important age metrics are: %in $\mathcal{M}_{\text{Sch}}$ are:
\begin{itemize}
\item[1.] The \emph{average age} of the flows is
\begin{align}%\label{eq_avgage}
\Delta_{\text{avg}} (t) = \frac{1}{N}\sum_{n=1}^N \Delta_{n} (t).\nonumber
\end{align}

\item[2.] The \emph{maximum age} of the flows is
\begin{align}%\label{eq_maxage}
\Delta_{\max} (t) = \max_{n=1,\ldots,N} \Delta_{n} (t).\nonumber
\end{align}

\item[3.] The \emph{mean square age} of the flows is
\begin{align}%\label{eq_msage}
\Delta_{\text{ms}} (t) = \frac{1}{N}\sum_{n=1}^N (\Delta_{n} (t))^2.\nonumber
\end{align}

\item[4.] The \emph{$p$-norm of the age} of the flows is
\begin{align}%\label{eq_msage}
\Delta_{\text{$p$-norm}} (t) = \left[\sum_{n=1}^N (\Delta_{n} (t))^p\right]^{\frac{1}{p}}, ~p>0.\nonumber
\end{align}

%\item[5.] The \emph{proportional fair age utility} of the flows is
%\begin{align}%\label{eq_msage}
%\Delta_{\text{PF}} (t) = \sum_{n=1}^N \log[\Delta_{n} (t)+ \epsilon],\nonumber
%\end{align}
%where $\epsilon>0$ is any positive number.\footnote{We are }

\item[5.] The \emph{total age penalty} of the flows is
\begin{align}%\label{eq_msage}
\Delta_{\text{penalty}} (t) = \sum_{n=1}^N g(\Delta_{n} (t)),\nonumber
\end{align}
where $g: [0,\infty) \rightarrow \mathbb{R}$ is any non-decreasing function. For example, as pointed out in \cite{report_AgeOfInfo2016}, a stair-shape function $g(\Delta)=\lfloor a \Delta\rfloor$ with $a\geq 0$ can be used to characterize the dissatisfaction of data staleness when the information of interests is checked periodically, and an exponential function $g(\Delta) = e^{a \Delta}$ is appropriate for online learning and control applications where the desire for data refreshing grows quickly with respect to the age.
\end{itemize}

In general, each reasonable \emph{age metric} can be expressed as a \emph{non-decreasing} function $f: \mathbb{R}^N\rightarrow \mathbb{R}$ of the age vector  $\bm{\Delta}(t)$, i.e., $f (\bm{\Delta} (t)) = f \circ\bm{\Delta} (t)$. Let $\{f \circ\bm{\Delta}(t), t\in[0,\infty)\}$ denote  an age metric process. 
In this paper, we consider the following class of \emph{symmetric} age metrics:
\begin{align}%\label{eq_class}
\mathcal{M}_{\text{sym}}\!=\!\{f \circ\bm{\Delta} : f \text{ is non-decreasing and symmetric}\}.\nonumber
\end{align}
%Because every convex and symmetric function is Schur convex \cite[Proposition 3.C.2]{Marshall2011}, 
Notice that  the function $f$ in the definition of $\mathcal{M}_{\text{sym}}$ can be discontinuous, non-convex, or non-separable. Hence, the class  of age metrics $\mathcal{M}_{\text{sym}}$ is quite general.
It is easy to see 
\begin{align}
\{\Delta_{\text{avg}},\Delta_{\max}, \Delta_{\text{ms}},\Delta_{\text{$p$-norm}}, %\Delta_{\text{PF}}, 
\Delta_{\text{penalty}}\}\subseteq \mathcal{M}_{\text{sym}}.\nonumber
\end{align}
Each age metric is a function of the scheduling policy $\pi$ and time $t$, which is denoted as  $f \circ\bm{\Delta}_\pi(t)$ in the rest of this paper.
%

%Then, all these age metrics can be expressed as a function $f (\bm{\Delta}_{\pi} (t))$ of the age vector $\bm{\Delta}(t)$, where $f\in F_{\text{Sch}$ is an non-decreasing and Schur convex function.

A brief summary of related work is the following. In \cite{IgorAllerton}, age-optimal scheduling of a set of independent packet streams over a shared wireless access channel was formulated. The streams share one of several time slots in a frame, which may be considered as a special case of the multi-server setup considered in this paper. However, distinctly from our formulation, \cite{IgorAllerton} limited attention to periodically generated packets that expire at the end of each time frame.  
The problem of optimizing scheduling decisions in broadcast wireless networks with respect to
throughput and delivery times has been studied extensively in the literature. [to be improved] Throughput
maximization of traffic with strict packet delay constraints has been addressed in . Interdelivery
time is considered in as a measure of service regularity. Age of Information has been considered
recently in .
In recent works, the problem of minimizing the AoI has been explored using different approaches.
[to be improved] Queueing Theory is used in  for finding the optimal server utilization with
respect to AoI. The authors in consider the problem of optimizing the times in which packets are
generated at the source. Link scheduling optimization is considered in  where the complexity of the
problem is established and insights into the structure of the problem are provided. In contrast, our work
focuses on characterizing scheduling policies of interest, analyzing their performance in different network
configurations and discuss their main features.


The remainder of this paper is outlined as follows. 
