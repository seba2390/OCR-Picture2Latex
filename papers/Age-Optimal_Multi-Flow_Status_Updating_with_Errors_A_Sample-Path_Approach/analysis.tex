% !TEX root = ./Age_of_Info_multi_source.tex
\section{Multi-flow Status Update Scheduling: \\ The Continuous-time Case}\label{sec_analysis}
In this section, we investigate multi-flow scheduling in continuous-time status updating systems. We first consider a system setting with multiple servers and exponential transmission times, where an age-optimal scheduling result is established. Next, we study a more general system setting with multiple servers and NBU transmission times. In the second setting, age optimality is inherently difficult to achieve and we present a near age-optimal scheduling result. 

%When there is a single server with exponential service times, a preemptive \emph{Maximum Age first, Last Generated First Served (MAF-LGFS) policy} is shown to be age-optimal in a quite strong sense. When there are multiple servers with NBU service times, a non-preemptive \emph{Maximum Age of Served Information first, Last Generated First Served (MASIF-LGFS) policy} is shown to be within a small additive gap from the optimal age performance. 


%A server is in an idle state when it is not transmitting any update packet. Once an update packet is submitted to a server, the server becomes busy and will maintain a busy state until the update packet is successfully delivered to its destined receiver. When the server is busy, new updates cannot be admitted to the server. The source keeps tracking the servers' idle/busy states and can generate updates at any time by its own will. 


%\subsection{Proof Approach}
%The average age $\Delta (t)$ drops whenever a new update is delivered to the destination $f_d$.  

%Using \eqref{eq_age2}, we can obtain
%
%\begin{lemma}
%If for all $I$ and $\pi\in\Pi$, policy $P'$ satisfies
%\begin{itemize}
%\item[1.] $[(t_{1,P'},t_{2,P'},\ldots)|I]\leq_{\text{st}}[(t_{1,\pi},t_{2,\pi},\ldots)|I]$,
%\item[2.] $[(V_{1,P'},V_{2,P'},\ldots)|I]\leq_{\text{st}}[(V_{1,\pi},V_{2,\pi},\ldots)|I]$,
%\end{itemize}
%then 
%\begin{align}
%[\{\Delta_P (t), t\in [0,\infty)\}|I] \leq_{\text{st}} [\{\Delta_\pi (t), t\in [0,\infty)\}|I].
%\end{align}
%In addition, if policy $P$ satisfies 
%\begin{align}
%[\{\Delta_P (t\!+\!X(t)), t\!\in\![0,\infty)\}|I]\!\! \leq_{\text{st}}\!\! [\{\Delta_{P'}(t)\!+\!X(t), t\!\in\![0,\infty)\}|I],
%\end{align}
%then $[\{\Delta_P (t\!+\!X(t)), t\!\in\![0,\infty)\}|I]\!\! \leq_{\text{st}}\!\! [\{\Delta_\pi (t)\!+\!X(t), t\!\in\![0,\infty)\}|I]$.
%\end{lemma}
%\subsection{Scheduling Policy}
%We first provide an alternative expression of $\age_\pi(t)$.
%Let $f_{j,\pi}$, $i_{j,\pi}$, and $t_{j,\pi}$ denote the flow index, packet index, and delivery time of the $j$-th delivered update of all $M$ severs in policy $\pi$, respectively. 
%Then, $t_{j,\pi}\leq t_{j+1,\pi}$ and the age of flow $f$ is expressed as
%\begin{align}\label{eq_age1}
%\age_{f,\pi}(t)= t- \sum_{j=1}^\infty u(t-t_{j,\pi}) V_{j,\pi} 1_{\{f_{j,\pi}=f\}},
%\end{align}
%where 
%\begin{align}\label{eq_agereduce}
%V_{j,\pi}=\max[S_{f_{j,\pi}}(i_{j,\pi})-U_{f_{j,\pi}}^d(t_{j,\pi}),0] 
%\end{align}
%is the amount of age reduction induced by the $j$-th update delivery. By \eqref{eq_avgage} and \eqref{eq_age1}, the average age is determined by
%\begin{align}\label{eq_age2}
%\age_\pi(t)= t- \frac{1}{F}\sum_{j=1}^\infty u(t-t_{j,\pi}) V_{j,\pi}.
%\end{align}
%
%Equations \eqref{eq_agereduce} and \eqref{eq_age2} suggest that the system should submit the updates with the largest age reductions to minimize $\age_{\pi}(t)$. 
%Motivated by this, 


% which is described in Algorithm \ref{alg1}. 
%Define $V_f(t)=V_n(t) - U_n(t)$ as the age differential between the source $s_n$ and destination $f_d$. A flow $f$ is said to be \emph{informative} at time $t$ if $V_f(t)>0$. 
%In MAP, the servers are assigned to the latest generated updates of informative flows with the largest age differentials; if all informative flows are under service, the remaining severs are kept idle (see Steps \ref{max_differential}-\ref{max_differential_1}).

\ignore{
Maximum Age Difference first (MAP) 
Maximum Age Back-pressure first (MAR)
Maximum Age backPressure first (MAP)
}
%\subsection{Scheduling Policy}


\subsection{Multiple Flows, Multiple Servers, Exponential Service Times}
%When there is a single flow, the scheduler needs to  decide which packet to serve the first. In this case, it is known that the  \emph{Last Generated First Served (LGFS)}  scheduling discipline can achieve the minimum age process in a stochastic ordering sense \cite{Bedewy2016,BedewyJournal2017,Bedewy2017,BedewyMultihop2017}. 
%
%When there are multiple flows, the scheduler needs to compare not only the packets from the same flow, but also the packets from different flows, which makes the scheduling problem more complicated. 

To address the multi-flow scheduling problem, we consider a flow selection discipline called \emph{Maximum Age First  (MAF)}  \cite{LiInfocom2015,IgorAllerton2016,HsuTWC2017}, in which 
\emph{the flow with the maximum age is served first, with ties broken arbitrarily}. 
%(ii) The second discipline is called \emph{Maximum Age Reduction (MAR) first}: If the $i$-th packet of flow $n$ is fresher than any packet available at the destination $d_n$, the age reduction brought by the delivery of this packet is $S_{n,i}-U_n(t)$. In the MAR discipline,\emph{the packet with the maximum  age reduction is served first, with ties broken arbitrarily}.

For multi-flow single-server systems, a scheduling policy is defined by combining the Preemptive, MAF, and LGFS service disciplines as follows: 


%We use these disciplines to define two scheduling policies:

%\begin{definition} \emph{Maximum Age Reduction first, Maximum Age-first (MAR-MA) policy:} The scheduler first picks the packets with the maximum age reduction from all the flows and assign these packets to idle servers according to the MA discipline; if there exist idle servers after the first round, the scheduler picks the packets with the next maximum age reduction from all the flows and assign these packets to idle servers according to the MA discipline; this procedure continues until all servers are busy or all packets are under service. Hence, in the MAR-MA policy, the MAR discipline is adopted with a higher priority than the MA discipline.
%\end{definition}
\ignore{
\begin{figure}
\centering
\includegraphics[width=0.2\textwidth]{./figs/policies}   
\caption{{\red Remove periodic arrivals.}
Packet service priorities of the MAF-LGFS policy for periodic arrivals with a period $T$, where  $\triangle$ denote delivered packets,  $\bigcirc$ denote undelivered packets waiting to be served,
and the numbers in the circles $\bigcirc$ represent the service priorities of the undelivered packets. %(a) In the MAR-MA policy, the packets with priorities 1 and 2 have the maximum age reduction. The packets with priorities 3-5 have the third maximum age reduction, where the packets with priorities 3 and 4 are from the flows with the maximum age and hence will be served earlier than the packet with priority 5. 
The packets with priorities 1-3 are generated the last and  will be served first; further, the packets with priorities 1 and 2 are from the flows with the maximum age and hence will be served earlier than the packet with priority 3.}\vspace{-0.0cm}
\label{fig_policies}
\end{figure}  }
\ignore{
\begin{figure}
\centering
\includegraphics[width=0.25\textwidth]{./figs/policies1}   
\caption{%{\red Remove periodic arrivals.}
Packet service priorities of the MAF-LGFS policy for synchronized packet generations and arrivals, where  $\triangle$ denote delivered packets,  $\bigcirc$ denote undelivered packets waiting to be served, the packet generation times $S_{n,i}$ are marked on the left, and the numbers in the circles $\bigcirc$ represent the service priorities of the undelivered packets. %(a) In the MAR-MA policy, the packets with priorities 1 and 2 have the maximum age reduction. The packets with priorities 3-5 have the third maximum age reduction, where the packets with priorities 3 and 4 are from the flows with the maximum age and hence will be served earlier than the packet with priority 5. 
In particular, packets with priorities 1-4 are generated the last and  will be served first; further, the service priorities of these 4 last generated packets are determined by using the maximum age (MA) first discipline. Notice that the service priorities are not determined by the packet arrival times $A_{n,i}$.}\vspace{-0.0cm}
\label{fig_policies}
\end{figure}  }

\begin{definition} \emph{Preemptive, Maximum Age First, Last Generated First Served (P-MAF-LGFS) policy:} This is a work-conserving scheduling policy for multiple-server, continuous-time systems with synchronized packet generations and arrivals. It operates as follows:

\begin{itemize}
\item[1.] If the queue is not empty, a server is assigned to process the most recently generated packet from the flow with the maximum age, with ties broken arbitrarily. 

\item[2.] The next server is assigned to process the most recently generated packet from the flow with the second maximum age, with ties broken arbitrarily. 

\item[3.] This process continues until either (i) the most recently generated packet of every flow is under service or has been delivered, or (ii) all servers are busy. 

\item[4.] If the most recently generated packet of every flow is under service or has been delivered, the remaining servers can be arbitrarily assigned to send the remaining packets in the queue, until the queue becomes empty.

\item[5.] When fresher packets arrive, the scheduler can preempt the packets that are currently under service and assign the new packets to servers following Steps 1-4 above. The preempted packets are then returned to the queue, where they await their turn to be transmitted at a later time.
\end{itemize}

%each packet assigned to the server is the last generated packet from the flow with the maximum age, , with ties broken arbitrarily.
%The scheduler first picks the last generated packet from each flow and assign these packets to idle servers according to the MA discipline; if there exist idle servers after the first round, the scheduler picks the second last generated packet from each flow and assign these packets to idle servers according to the MA discipline; this procedure continues until all servers are busy or all packets are under service. 
\end{definition}

The following observation provides useful insights into the operations of the P-MAF-LGFS policy: Due to synchronized packet generations and arrivals,  when the most recently generated packet of flow $n$ is successfully delivered in the P-MAF-LGFS policy, flow $n$ must have the \emph{minimum} age among the $N$ flows. Conversely, if flow $n$ does not have the \emph{minimum} age among all the flows, its most recently generated packet must be undelivered. Hence, in the P-MAF-LGFS policy, the most recently generated packet from a flow that does not have the \emph{minimum} age  is always available to be scheduled. 

The above P-MAF-LGFS policy is suitable for use in both single-server and multiple-server systems. It extends the original single-server P-MAF-LGFS policy introduced in \cite{SunAoIWorkshop2018} to encompass the more general multi-server scenario. 

%whether the preempted packets are dropped or stored back to the queue does not affect the age performance of the P-MAF-LGFS policy
%In the special case that there is a single flow ($N=1$), the MAF-LGFS policy reduces to the LGFS policy studied in \cite{Bedewy2016,BedewyJournal2017,Bedewy2017,BedewyMultihop2017}. 
%Hence, in the MAF-LGFS policy, the  LGFS discipline is adopted with a higher priority than the MA discipline. 
%The packet service priorities of the MAF-LGFS policy are illustrated in Fig. \ref{fig_policies}. 
The age optimality of the P-MAF-LGFS policy is established in Theorem \ref{thm1} and Corollary \ref{coro1} below.
%Next, we will show that the preemptive and non-preemptive  are (near) age-optimal in several system settings.
%When there is a single server, the MAF-LGFS and  MAF-LGFS policies are equivalent to each other. In this case, the following theorem shows that age optimality can be established for a more general class of symmetric age penalty functions $\mathcal{P}_{\text{sym}}$.

 
%At any time $t$, the freshest packet of flow $n$ at the source node $s_n$ (also in the queue) is generated at time
%\begin{align}
%V_{n} (t) = \max\{S_{n,i}: S_{n,i} \leq t\}. \nonumber
%\end{align} 
%Hence, the age back-pressure between the source $s_n$ and the destination $d_n$ of flow $n$ is $V_{n} (t)-U_{n}(t)$.
%In the MAP policy, 
%\emph{the last generated packet from the flow with }  
% The P-MAF-LGFS policy is described in Algorithm \ref{alg1}.  

%%{\red single server} 
%\begin{algorithm}[h]
%\SetKwData{MAP}{MAP}
%\SetCommentSty{small}
%%$V_n:=-1$\tcp*[r]{Generation time of the freshest packet at  $s_n$}
%%$U_n:=-1$\tcp*[r]{Generation time of the freshest packet at  $d_n$}
%$\alpha :=0$\tcp*[r]{smallest time-stamp of  the packets under service} 
%%$A := \emptyset$\tcp*[r]{The set of flows that are under service~~~}
%\While{the system is ON} {
%\If{a new packet with time-stamp $s$ arrives}{
%%$I_f^s := I_f^s \cup \{i\}$\;
%\uIf{$s\leq\alpha$}{
%%Pick any idle \;
%Store the arrived packet in the queue\;}
%\Else{
%\If{all servers are busy}{
%Preempt the packet with time-stamp $\alpha$ and store it back to the queue\;
%}
%Assign the arrived packet to one idle server\;
%Set $\alpha$ as the smallest generation time of the packets under service\;
%%\If{$f^*\neq\NULL$}{
%%%$i^* = \argmax_{i\in I_{f^*}^s } S_{f^*}(i)$\;
%%
%%$A:=A\cup\{f^*\}$\label{max_differential_1}\;
%%}
%%\Else(~~~~\tcp*[h]{All informative flows are under service}){
%%\textbf{break}\;\label{idle}
%%}
%}
%
%}
%
%\If{the $j$-th packet of flow $l$ is delivered to  $d_l$}{
%%$I_f^d := I_f^d \cup \{i\}$\;
%$U_l:=\max\{S_{l,j},U_l\}$\;
%\uIf{the queue is not empty}{
%$n^*:=\argmax_{n=1,\ldots,N}\{V_n - U_n\}$\label{max_differential}\;
%Assign the freshest packet of flow $n^*$ to the server\;
%$ad := V_{n^*} - U_{n^*}$\;
%}
%\Else{
%$ad := 0$\;
%}
%}
%}
%\caption{The preemptive Maximum Age first with Last Generated First Served (MAF-LGFS) policy.}\label{alg1}
%\end{algorithm}

%\subsection{Age Optimality Analysis} % for Exponential Service Times}





%We consider a scheduling policy named \emph{Maximum Age first, Last Generated First Served (MAF-LGFS)}, where the scheduler first finds the flow $n^*$ with the maximum age and then picks the last generated packet of flow $n^*$.


%The age-optimality of the preemptive MAP policy is characterized in the following theorem:



\begin{theorem}(Continuous-time, multiple flows, multiple servers, exponential transmission times with transmission errors)\label{thm1}
In continuous-time status updating systems, if (i) the transmission errors are \emph{i.i.d.} with an error probability $q \in [0, 1)$, (ii)  the packet generation and arrival times are {synchronized} across the $N$ flows, and (iii)  the packet transmission times are exponentially distributed and \emph{i.i.d.} across packets, then it holds that for all $\mathcal{I}$, all $p_t \in\mathcal{P}_{\text{sym}}$, and all $\pi\in\Pi$ 
\begin{align}\label{thm1eq1}
&[\{p_t \circ\bm{\Delta}_{\text{P-MAF-LGFS}}(t), t\in [0,\infty)\}\vert\mathcal{I}] \nonumber\\
\leq_{\text{st}} & [\{p_t \circ\bm{\Delta}_\pi(t), t\in [0,\infty)\}\vert\mathcal{I}],
\end{align}
or equivalently, for all $\mathcal{I}$, all $p_t \in\mathcal{P}_{\text{sym}}$, and all non-decreasing functional $\phi$
 \begin{align}\label{thm1eq2}
&\mathbb{E}\left[\phi (\{p_t \circ\bm{\Delta}_{\text{P-MAF-LGFS}}(t),t\in [0,\infty)\})\vert\mathcal{I}\right] \nonumber\\
= & \min_{\pi\in\Pi} \mathbb{E}\left[\phi (\{p_t \circ\bm{\Delta}_\pi(t),t\in [0,\infty)\})\vert\mathcal{I}\right],
\end{align}
provided that the expectations in \eqref{thm1eq2} exist.
%, where $\mathbb{V}$ is the set of Lebesgue measurable functions defined in \eqref{eq_functions}.
\end{theorem}
\begin{proof}
%We develop a  sample-path method to prove Theorem \ref{thm1}.
%In the P-MAF-LGFS policy, if a packet from flow $n^*$ is delivered to its destination, then flow $n^*$ must be the flow with the \emph{maximum} age before the packet is delivered. Because the packet generation and arrival times are synchronized across the flows, flow $n^*$ is also the flow with the \emph{minimum} age after the packet is delivered.\footnote{If packet generations or arrivals are asynchronized, then this property is not guaranteed to hold.} Theorem \ref{thm1} is proven by employing this property to construct a sample-path argument. 
%The details are provided in 

See Appendix \ref{app1}.  \end{proof}



According to Theorem \ref{thm1}, for any age penalty function in $\mathcal{P}_{\text{sym}}$, any number of flows  $N$, any number of servers $M$,  any synchronized packet generation and arrival times in $\mathcal{I}$, and regardless the presence of \emph{i.i.d.} transmission errors or not,
the P-MAF-LGFS policy minimizes the stochastic process $[\{p_t \circ\bm{\Delta}_\pi(t), t\in [0,\infty)\}|\mathcal{I}]$ among all causal policies  in terms of stochastic ordering. Theorem \ref{thm1} is more general than \cite[Theorem 1]{SunAoIWorkshop2018}, as the latter was established for the special case of single-server systems without transmission errors. 




By considering a mixture over the different realizations of $\mathcal{I}$, it can be readily deduced from Theorem \ref{thm1} that 
\begin{corollary}\label{coro1}
Under the conditions of Theorem \ref{thm1}, it holds that for all $p_t \in\mathcal{P}_{\text{sym}}$ and all $\pi\in\Pi$ 
\begin{align}\label{coro1eq1}
\{p_t \circ\bm{\Delta}_{\text{P-MAF-LGFS}}(t), t\in [0,\infty)\} \!\leq_{\text{st}} \!\{p_t \circ\bm{\Delta}_\pi(t), t\in [0,\infty)\}, 
\end{align}
or equivalently, for all $p_t \in\mathcal{P}_{\text{sym}}$ and all non-decreasing functional $\phi$
 \begin{align}\label{coro1eq2}
&\mathbb{E}\left[\phi (\{p_t \circ\bm{\Delta}_{\text{P-MAF-LGFS}}(t),t\in [0,\infty)\})\right] \nonumber\\
= & \min_{\pi\in\Pi} \mathbb{E}\left[\phi (\{p_t \circ\bm{\Delta}_\pi(t),t\in [0,\infty)\})\right],
\end{align}
provided that the expectations in \eqref{coro1eq2} exist.
%, where $\mathbb{V}$ is the set of Lebesgue measurable functions defined in \eqref{eq_functions}.
\end{corollary}

Corollary \ref{coro1} states that the P-MAF-LGFS policy minimizes the stochastic process $\{p_t \circ\bm{\Delta}_\pi(t), t\in [0,\infty)\}$ in a stochastic ordering sense, outperforming all other causal policies.

\subsubsection{Status Update Scheduling with Packet Replications}

As discussed in Section \ref{sec:queuemodel}, our study has been centered on a scenario where different servers are not allowed to simultaneously transmit packets from the same flow. In this context, we have demonstrated the age-optimality of the P-MAF-LGFS policy in Theorem \ref{thm1}. However, in situations where multiple servers can transmit packets from the same flow, and packet replication is permitted, it becomes possible to create multiple copies of the same packet and transmit these copies concurrently across multiple servers. The packet is considered delivered once any one of these copies is successfully delivered; at that point, the other copies are canceled to release the servers. If the packet service times follow an \emph{i.i.d.} exponential distribution with a service rate of $\mu$, the $N$ servers can be equivalently viewed as a single, faster server with exponential service times and a higher service rate of $N\mu$. Additionally, this fast server exhibits \emph{i.i.d.} transmission errors with an error probability $q$. Our study also addresses this scenario. 

\begin{definition} \emph{Preemptive, Maximum Age First, Last Generated First Served policy with packet Replications  (P-MAF-LGFS-R):} In this policy, the last generated packet from the flow with the maximum age is served the first among all packets of all flows, with ties broken arbitrarily. This packet is replicated into $N$ copies, which are transmitted concurrently over the $N$ servers. The packet is considered delivered once any one of these $N$ copies is successfully delivered; at that point, the other copies are canceled to release the servers.
\end{definition}



By applying Theorem \ref{thm1} to this particular scenario with a single, faster server, we derive the following corollary.


\begin{corollary}\label{corollary_new}
Under the conditions of  Theorem \ref{thm1}, if packet replication is allowed, then it holds that for all $\mathcal{I}$, all $p_t \in\mathcal{P}_{\text{sym}}$, and all $\pi\in\Pi$ 
\begin{align}\label{corollary_neweq1}
&[\{p_t \circ\bm{\Delta}_{\text{P-MAF-LGFS-R}}(t), t\in [0,\infty)\}\vert\mathcal{I}] \nonumber\\
\leq_{\text{st}} & [\{p_t \circ\bm{\Delta}_\pi(t), t\in [0,\infty)\}\vert\mathcal{I}],
\end{align}
or equivalently, for all $\mathcal{I}$, all $p_t \in\mathcal{P}_{\text{sym}}$, and all non-decreasing functional $\phi$
 \begin{align}\label{corollary_neweq2}
&\mathbb{E}\left[\phi (\{p_t \circ\bm{\Delta}_{\text{P-MAF-LGFS-R}}(t),t\in [0,\infty)\})\vert\mathcal{I}\right] \nonumber\\
= & \min_{\pi\in\Pi} \mathbb{E}\left[\phi (\{p_t \circ\bm{\Delta}_\pi(t),t\in [0,\infty)\})\vert\mathcal{I}\right],
\end{align}
provided that the expectations in \eqref{corollary_neweq2} exist.
%, where $\mathbb{V}$ is the set of Lebesgue measurable functions defined in \eqref{eq_functions}.
\end{corollary}

\ignore{


The following theorem establishes the age-optimality of the P-MAF-LGFS policy in the presence of \emph{i.i.d.} transmission errors.

\begin{theorem}(Continuous-time, transmission errors, multiple flows, single server, exponential transmission times)\label{coro2}
If there are \emph{i.i.d.} transmission errors with an error probability $q\in (0,1)$ and the conditions (ii)-(iv) of Theorem \ref{thm1} holds, then the results of Theorem \ref{thm1} and Corollary \ref{coro1} remain true.  
\end{theorem}
\begin{proof}
In order to prove Theorem \ref{coro2}, we modify the sample-path argument in the proof of Theorem \ref{coro1} by adopting a new coupling lemma that can handle transmission errors. See Appendix \ref{appcoro2} for the details. 
\end{proof}
}

\subsection{Multiple Flows, Multiple Servers, NBU Service Times}
Next, we consider a more general system setting with multiple servers and a class of New-Better-than-Used (NBU)  transmission time distributions that include exponential distribution as a special case. 
%The next question we proceed to answer is whether for an important class of distributions that are more general than exponential, age-optimality or near age-optimality can be achieved. 

%NBU distributions are defined as follows.
\begin{definition}  \emph{New-Better-than-Used Distributions:} Consider a non-negative random variable $X$ with complementary cumulative distribution function (CCDF) $\bar{F}(x)=\Pr[X>x]$. Then, $X$ is said to be \emph{New-Better-than-Used (NBU)} if for all $t,\tau\geq0$
\begin{equation}\label{NBU_Inequality}
\bar{F}(\tau +t)\leq \bar{F}(\tau)\bar{F}(t).
\end{equation} 
Examples of NBU distributions include deterministic distribution, exponential distribution, shifted exponential distribution, geometric distribution, gamma distribution, and negative binomial distribution. 
\end{definition}


In the scheduling literature, optimal  scheduling results were successfully established for delay minimization in single-server queueing systems, e.g., \cite{Schrage68,Jackson55}, but it can become inherently difficult in the multi-server cases. In particular, minimizing the average delay in deterministic scheduling problems with more than one servers is NP-hard  \cite{Leonardi:1997}. Similarly, delay-optimal stochastic scheduling in multi-class, multi-server queueing systems is deemed to be quite difficult \cite{Weiss:1992,Weiss:1995,Dacre1999}. The key challenge in multi-class, multi-server scheduling is that \emph{one cannot combine the capacities of all the servers to jointly process the most important packet}. Due to the same reason, age-optimal scheduling in multi-flow, multi-server systems is quite challenging. In the sequel, we consider a  relaxed goal to seek for \emph{near} age-optimal scheduling of multiple information flows, where our proposed scheduling policy is shown to be within a small additive gap from the optimum age performance.   

%We first construct a lower bound of the age $\age_{n}(t)$: 

\begin{figure}
\centering 
\includegraphics[width=0.3\textwidth]{times1.pdf} \caption{An illustration of $S_{n,i}$, $A_{n,i}$, $V_{n,i}$, and $D_{n,i}$.}
% work--efficiency ordering holds for any priorities of the jobs.
\label{fig_times1} 
\vspace{-5mm}
\end{figure} 


%Notice that age $\age_{n}(t)$ in \eqref{eq_age} is determined by the packets that have been delivered to the destination $d_n$ by time $t$. 
To establish near age optimality, we introduce another age metric named \emph{age of served information}, denoted as $\Xi_{n} (t)$, which is a lower bound for age of information $\age_{n}(t)$: 

Let $V_{n,i}$ be the time that the $i$-th packet of flow $n$ starts its service by a server, i.e., the service starting time of the $i$-th packet of flow $n$. It holds that $S_{n,i}\leq A_{n,i}\leq V_{n,i}\leq D_{n,i}$, as illustrated in Fig. \ref{fig_times1}.
\emph{Age of served information} for flow $n$ is defined as
\begin{align}\label{eq_age_served}
\Xi_{n} (t) = t - \max_i\{S_{n,i}: V_{n,i} \leq t\},
\end{align}
which is the time difference between the current time $t$ and the generation time of the freshest packet that has started service by time $t$. Let $\bm{\Xi}(t)=(\Xi_{1} (t),\ldots,\Xi_{N} (t))$ be the age of served information vector at time $t$. Age of served information $\Xi_{n} (t)$ reflects the staleness of the packets that has begun service, whereas $\age_{n}(t)$ represents the staleness of the packets that has been successfully delivered to their destination. As depicted in Fig. \ref{fig_times2}, it is evident that $\Xi_{n} (t)\leq \age_{n}(t)$. In non-preemptive policies, the discrepancy between $\Xi_{n} (t)$ and $\age_{n}(t)$ solely arises from the \emph{i.i.d.} packet transmission times. 
Consequently, the age of served information $\Xi_{n} (t)$ closely approximates the age $\age_{n}(t)$.








We propose a new flow selection discipline called \emph{Maximum Age of Served Information First (MASIF)}, in which 
\emph{the flow with the maximum Age of Served Information is served first, with ties broken arbitrarily}. Using this discipline, we define another scheduling policy:


\begin{definition} \emph{Non-Preemptive, Maximum Age of Served Information first, Last Generated First Served (NP-MASIF-LGFS) policy:} This is a non-preemptive, work-conserving scheduling policy for multi-server systems. It operates as follows:
\begin{itemize}
\item[1.] When the queue is not empty and there are idle servers, an idle server is assigned to process the most recently generated packet from the flow with the maximum age of served information, with ties broken arbitrarily. 

\item[2.] After a packet from flow $n$ is assigned to an idle server, the server transitions into a busy state and will complete the transmission of the current packet from flow $n$ before serving any other packet. The age of served information $\Xi_{n} (t)$ of flow $n$ decreases. As a result, flow $n$ may no longer retain the maximum age of served information, allowing the remaining idle servers to be allocated to process other flows. The next idle server is assigned to process the most recently generated packet from the flow with the maximum age of served information, with ties broken arbitrarily. 

\item[3.] This procedure continues until either all servers are busy or the queue becomes empty. 

\end{itemize}

\end{definition}

%{\blue In some previous studies, e.g., \cite{IgorAllerton2016,HsuTWC2017,CostaCodreanuEphremides_TIT}, it was proposed to discard old packets and only store and send the freshest one. While this technique can reduce the age, in many applications such as social updates, news seeds, and stock trading, some old packets with earlier generation times are still quite useful and are needed to be sent to the destinations. 

Next, we will establish the near-age optimality of the NP-MASIF-LGFS policy. %Hence, the additional age reduction provided by discarding old packets in the NP-MASIF-LGFS policy is not large. 
The following theorem shows that the age of served information obtained by the NP-MASIF-LGFS policy serves as a lower bound (in terms of stochastic ordering) for the age of all other non-preemptive and causal policies. 

\begin{figure}
\centering 
\includegraphics[width=0.3\textwidth]{times2.pdf} \caption{The age of served information $\Xi_{n} (t)$ as a lower bound of  age $\age_{n}(t)$.}
\vspace{-5mm}
% work--efficiency ordering holds for any priorities of the jobs.
\label{fig_times2} 
\end{figure} 

\begin{theorem}(Continuous-time, multiple flows, multiple servers, NBU transmission times with no errors) \label{thm3}
In continuous-time status updating systems, if  (i) there is no  transmission errors (i.e., $q=0$),  (ii) the packet generation and arrival times are {synchronized} across the $N$ flows,  and (iii) the packet transmission times are NBU and \emph{i.i.d.} across both servers and packets, then it holds that for all $\mathcal{I}$, all $p_t \in\mathcal{P}_{\text{sym}}$, and all $\pi\in\Pi_{np}$\footnote{Recall that $\Pi_{np}$ is the set of non-preemptive and causal scheduling policies.} 
\begin{align}\label{thm3eq1}
&[\{p_t \circ\bm{\Xi}_{\text{NP-MASIF-LGFS}}(t), t\in [0,\infty)\}\vert\mathcal{I}] \nonumber\\
\leq_{\text{st}} & [\{p_t \circ\bm{\Delta}_\pi(t), t\in [0,\infty)\}\vert\mathcal{I}],
\end{align}
or equivalently, for all $\mathcal{I}$, all $p_t \in\mathcal{P}_{\text{sym}}$, and all non-decreasing functional $\phi$
 \begin{align}\label{thm3eq2}
&\mathbb{E}\left[\phi (\{p_t \circ\bm{\Xi}_{\text{NP-MASIF-LGFS}}(t),t\in [0,\infty)\})\vert\mathcal{I}\right] \nonumber\\
\leq & \min_{\pi\in\Pi_{np}} \mathbb{E}\left[\phi (\{p_t \circ\bm{\Delta}_\pi(t),t\in [0,\infty)\})\vert\mathcal{I}\right] \nonumber\\
\leq & \mathbb{E}\left[\phi (\{p_t \circ\bm{\age}_{\text{NP-MASIF-LGFS}}(t),t\in [0,\infty)\})\vert\mathcal{I}\right],
\end{align}
provided that the expectations in \eqref{thm3eq2} exist.
\end{theorem}

\begin{proof}[Proof idea]
In the NP-MASIF-LGFS policy, if a packet from flow $n^*$ begins service, it implies that flow $n^*$ possesses the \emph{maximum} age of served information before the service starts. If the packet generation and arrival times are synchronized across the flows, flow $n^*$ also exhibits the \emph{minimum} age of served information after the service starts. The proof of Theorem \ref{thm3} relies on this property and a sample-path argument that is developed for NBU service time distributions. %{\blue We note that the sample-path method in \cite{sun2016delay,sun2017delay} is the key for addressing the challenge in multi-flow, multi-server scheduling.} 
\ifreport
See Appendix \ref{app2} for the details.
\else
See our technical report \cite{SunMultiFlow18} and \cite{sun2016delay,sun2017delay} for the details.
\fi
\end{proof}

%According to Theorem \ref{thm3}, the NP-MASIF-LGFS policy is near age-optimal in the sense of \eqref{thm3eq1} and \eqref{thm3eq2}. 



Considering the close approximation between the age of served information $\bm{\Xi}_{\text{NP-MASIF-LGFS}}(t)$ and the age of information $\bm{\age}_{\text{NP-MASIF-LGFS}}(t)$ in \eqref{thm3eq2}, we can conclude that the NP-MASIF-LGFS policy is near age-optimal.
Furthermore, in the case of the average age metric as defined in \eqref{eq_avgage} (i.e., $p_t = p_{\text{avg}} $ for all $t$), we can derive the following corollary: 
\begin{corollary}\label{coro4}
Under the conditions of Theorem \ref{thm3}, it holds that for all $\mathcal{I}$
\begin{align}\label{eq_gap}
\min_{\pi\in\Pi_{np}}\! [\bar{\age}_{ \pi}|\mathcal{I}] \!\leq\! [\bar{\age}_{\text{NP-MASIF-LGFS}}|\mathcal{I}]\!\leq\! \min_{\pi\in\Pi_{np}}\! [\bar{\age}_{ \pi}|\mathcal{I}] \!+\! \frac{1}{\mu},
\end{align}
where 
\begin{align}\label{eq_gap11}
[\bar{\age}_{ \pi}|\mathcal{I}] = \lim\sup_{T\rightarrow \infty} \frac{1}{T} \mathbb{E}\left[\int_0^T p_{\text{avg}} \circ\bm{\Delta}_\pi (t) dt \Bigg| \mathcal{I}\right]
\end{align} is the expected time-average of the average age of the $N$ flows, and $1/\mu$ is the mean packet transmission time.
\end{corollary}


\begin{proof}
The proof of Corollary \ref{coro4} is the same as that of Theorem 12 in \cite{BedewyJournal2017} and hence is omitted here. 
\end{proof}

By Corollary \ref{coro4}, the  average age of the NP-MASIF-LGFS policy is within an additive gap from the optimum, where the gap $1/\mu$ is invariant of the packet arrival and generation times $\mathcal{I}$, the number of flows $N$, and the number of servers $M$. 

Similar to Corollary \ref{coro1}, by taking a mixture over the different realizations of $\mathcal{I}$, one can remove the condition $\mathcal{I}$ from \eqref{thm3eq1}, \eqref{thm3eq2}, \eqref{eq_gap}, and \eqref{eq_gap11}.


The sampling-path argument utilized in the proof of Theorem \ref{thm3} can effectively handle multiple flows,  multiple servers, and \emph{i.i.d.} NBU transmission time distributions. This is achieved by establishing a coupling between the start time of packet transmissions in the NP-MASIF-LGFS policy and the completion time of packet transmissions in any other work-conserving policy from $\Pi_{np}$. However, extending this sampling-path argument to encompass the scenario of \emph{i.i.d.} transmission errors poses additional challenges that are currently difficult to overcome. 


\section{Multi-flow Status Update Scheduling: \\ The Discrete-time Case}
In this section, we investigate age-optimal scheduling in discrete-time status updating systems, where the variables $S_{n,i}, A_{n,i}, D_{n,i}, t, U_{n} (t), \Delta_{n} (t)$ are all multiples of the period $T_s$, the transmission time of each packet is fixed as $T_s$, and the packet submissions are subject to \emph{i.i.d.} errors with an error probability $q\in[0,1)$. Service preemption is not allowed in discrete-time systems.

%In this discrete-time setting, the absence of packet arrivals during ongoing transmissions makes service preemption inconsequential and devoid of any advantages. As a result, we exclude service preemption from our design of the scheduling policy. 
For multiple-server, discrete-time systems, a scheduling policy is defined by combining the MAF and LGFS service disciplines as follows:


\begin{definition} \emph{Discrete Time, Maximum Age First, Last Generated First Served (DT-MAF-LGFS) policy:} This is a work-conserving scheduling policy for multiple-server, discrete-time systems with synchronized packet generations and arrivals. It operates as follows:

\begin{itemize}
\item[1.] When  time $t$ is a multiple of period $T_s$, if the queue is not empty, an idle server is assigned to process the most recently generated packet from the flow with the maximum age, with ties broken arbitrarily. 

\item[2.] The next  idle server is assigned to process the most recently generated packet from the flow with the second maximum age, with ties broken arbitrarily. 

\item[3.] This process continues until either (i) the most recently generated packet of each flow is under service or has been delivered, or (ii) all servers are busy. 

\item[4.] If the most recently generated packet of each flow is under service or has been delivered, and there are additional idle servers, then these servers can be arbitrarily assigned to send the remaining packets in the queue, until the queue becomes empty.
\end{itemize}
\end{definition}

One can observe that the DT-MAF-LGFS policy for discrete-time systems is similar to the P-MAF-LGFS policy designed for continuous-time systems. 

The age optimality of the DT-MAF-LGFS policy is obtained in the following theorem.







\begin{theorem}(Discrete-time, multiple flows, multiple servers, constant transmission times with transmission errors)\label{thm4}
In discrete-time status updating systems, if  (i) the transmission errors are \emph{i.i.d.} with an error probability $q\in [0,1)$, (ii)  the packet generation and arrival times are {synchronized} across the $N$ flows, and (iii) the packet transmission times are fixed as $T_s$, then it holds that for all $\mathcal{I}$, all $p_t \in\mathcal{P}_{\text{sym}}$, and all $\pi\in\Pi_{np}$ 
\begin{align}\label{thm4eq1}
&[\{p_t \circ\bm{\Delta}_{\text{DT-MAF-LGFS}}(t), t=0,T_s,2T_s,\ldots\}\vert\mathcal{I}] \nonumber\\
\leq_{\text{st}} & [\{p_t \circ\bm{\Delta}_\pi(t), t=0,T_s,2 T_s,\ldots\}\vert\mathcal{I}],
\end{align}
or equivalently, for all $\mathcal{I}$, all $p_t \in\mathcal{P}_{\text{sym}}$, and all non-decreasing functional $\phi$
 \begin{align}\label{thm4eq2}
&\mathbb{E}\left[\phi (\{p_t \circ\bm{\Delta}_{\text{DT-MAF-LGFS}}(t),t=0,T_s,2T_s,\ldots\})\vert\mathcal{I}\right] \nonumber\\
= & \min_{\pi\in\Pi_{np}} \mathbb{E}\left[\phi (\{p_t \circ\bm{\Delta}_\pi(t),t=0,T_s,2T_s,\ldots\})\vert\mathcal{I}\right],
\end{align}
provided that the expectations in \eqref{thm4eq2} exist.
%, where $\mathbb{V}$ is the set of Lebesgue measurable functions defined in \eqref{eq_functions}.
\end{theorem}

\begin{proof}See Appendix \ref{app_thm4}.  \end{proof}

\begin{figure}
\centering 
\includegraphics[width=0.45\textwidth]{figure_1_1.eps} 

\caption{Expected time-average of the maximum age of 3 flows in a system with a single server and \emph{i.i.d.} exponential transmission times.}
\vspace{-5mm}
% work--efficiency ordering holds for any priorities of the jobs.
\label{fig_simulation1} 
\end{figure} 

According to  Theorem \ref{thm4}, the DT-MAF-LGFS policy minimizes the stochastic process $[\{p_t \circ\bm{\Delta}_\pi(t), t=0,T_s,2T_s,\ldots)\}|\mathcal{I}]$ in terms of stochastic ordering within discrete-time status updating systems. This optimality result holds for any age penalty function in $\mathcal{P}_{\text{sym}}$, any number of flows $N$, any number of servers $M$, any synchronized packet generation and arrival times in $\mathcal{I}$, and regardless the existence of \emph{i.i.d.} transmission errors. 

Theorem \ref{thm4} generalizes \cite[Theorem 1]{IgorAllerton2016}, %as the latter only holds in single-server systems with $p_t = p_{\text{avg}}$ for all $t$.
by allowing for multiple servers and a broader range of age penalty functions.
Similar to Corollary \ref{coro1}, one can remove the condition $\mathcal{I}$ from \eqref{thm4eq1} and \eqref{thm4eq2}.





\ignore{
\subsection{Periodic Arrivals, Multiple Servers, Exponential Service Times}

\begin{theorem}\label{thm2}
If (i) the packet arrival times are \emph{periodic} and (ii) the packet transmission times are exponential distributed and \emph{i.i.d.} across servers and time, then for all $\mathcal{I}$, all $p \in\mathcal{P}_{\text{Sch}}$, and all $\pi\in\Pi$ 
\begin{align}\label{thm2eq1}
&[\{p \circ\bm{\Delta}_{\text{prmp, MAR-MA}}(t), t\in [0,\infty)\}\vert\mathcal{I}] \nonumber\\
\leq_{\text{st}} & [\{p \circ\bm{\Delta}_\pi(t), t\in [0,\infty)\}\vert\mathcal{I}],
\end{align}
or equivalently, for all $\mathcal{I}$, all $p \in\mathcal{P}_{\text{Sch}}$, and all non-decreasing functional $\phi:\mathbb{V}\mapsto\mathbb{R}$
 \begin{align}\label{thm2eq2}
&\mathbb{E}\left[\phi (\{p \circ\bm{\Delta}_{\text{prmp, MAR-MA}}(t),t\in [0,\infty)\})\vert\mathcal{I}\right] \nonumber\\
= & \min_{\pi\in\Pi} \mathbb{E}\left[\phi (\{p \circ\bm{\Delta}_\pi(t),t\in [0,\infty)\})\vert\mathcal{I}\right],
\end{align}
provided that the expectations in \eqref{thm2eq2} exist.
\end{theorem}

\section{Proof of Theorem \ref{thm2}}
We first establish two lemmas that are useful in the proof of Theorem \ref{thm2}. 
Let the age vector $\bm\Delta_{\pi}(t)$ denote the \emph{system state} of policy $\pi$ at time $t$ and $\{\bm\Delta_{\pi}(t),t\in [0,\infty)\}$ denote the \emph{state process} of policy $\pi$. Because the system starts to operate at time $0$, we assume that $\bm\Delta_{\pi}(0^-)=\bm 0$ at time $t=0^-$ for all $\pi\in\Pi$. For notational simplicity, let policy $P$ represent the preemptive MAR-MA policy. 

We define an \emph{MAR-MA ordering} that sorts packets according to the priority rule in the MAR-MA policy: As shown Fig. \ref{fig_policies}(a), the packets with larger age reduction have higher priorities; among the packets with the same age reduction, the packets with larger age have higher priorities. 
Using the memoryless property of exponential distribution, we can obtain the following coupling lemma:

\begin{lemma}\emph{(Coupling Lemma)}\label{thm2coupling}
For any given $\mathcal{I}$, consider policy $P$ and any \emph{work-conserving} policy $\pi\in \Pi$. If  the packet transmission times are exponential distributed and \emph{i.i.d.} across servers and time,   
 then there exist policy $P_1$ and  policy $\pi_1$ in the same probability space which satisfy the same scheduling disciplines with policy $P$ and policy $\pi$, respectively,  such that 
\begin{itemize}
\itemsep0em 
\item[1.] The state process $\{\bm\Delta_{P_1}(t),t\in [0,\infty)\}$ of policy $P_1$ has the same distribution with the state process $\{\bm\Delta_{P}(t),t\in [0,\infty)\}$ of policy $P$,
\item[2.] The state process $\{\bm\Delta_{\pi_1}(t),t\in [0,\infty)\}$ of policy $\pi_1$ has the same distribution with the state process $\{\bm\Delta_{\pi}(t),t\in [0,\infty)\}$  of policy $\pi$,
\item[3.] If a packet with the $j$-th highest MAR-MA order among all the packets under service is delivered at time $t$ in policy $P_1$ as $\bm\Delta_{P_1}(t)$ evolves, then almost surely, a packet  with the $j$-th highest MAR-MA order among all the packets under service is  delivered at time $t$ in policy $\pi_1$ as $\bm\Delta_{\pi_1}(t)$ evolves; and vice versa. 
%whenever there exist unassigned packets in the queue,
\end{itemize} 
\end{lemma}
\begin{proof}
Note that all policies have identical arrival processes, and the transmission times are  memoryless. Following the inductive sample-path construction in the proof of \cite[Theorem 6.B.3]{StochasticOrderBook}, one can construct the packet deliveries one by one in policy $P_1$ and policy $\pi_1$ to prove this lemma. The details are omitted. 
\end{proof}

\begin{lemma} \emph{(Inductive Comparison)}\label{thm2lem2}
Under the conditions of Lemma \ref{thm2coupling}, 
suppose that a packet is delivered in both policy $P_1$ and policy $\pi_1$  at the same time $t$. The system state  of policy $P_1$ is $\bm\Delta_{P_1}$ before the packet delivery, which becomes $\bm\Delta_{P_1}'$ after the packet delivery. The system state  of policy $\pi_1$ is $\bm\Delta_{\pi_1}$ before the packet delivery, which becomes $\bm\Delta_{\pi_1}'$ after the packet delivery. If  the packet arrival times are  \emph{periodic} and
\begin{equation}\label{thm2hyp1}
\bm\Delta_{P_1} \prec_{\text{w}} \bm\Delta_{\pi_1},
\end{equation}
then
\begin{equation}\label{thm2law6}
\bm\Delta_{P_1}' \prec_{\text{w}} \bm\Delta_{\pi_1}'.
\end{equation}  
\end{lemma}

\begin{proof}


For periodic arrivals with a period $T$, let $V(t) = \max\{iT: iT \leq t,i=1,2,\ldots\}$ 
be the time-stamp of the freshest packet of each flow that has been generated by time $t$. At time $t$, because no packets is generated later than $V(t)$, we can obtain
\begin{align}%\label{eq_proof_1}
\Delta_{i,P_1} \geq\Delta_{i,P_1}' \geq t-V(t),~i=1,\ldots,N,\nonumber\\
\Delta_{i,\pi_1} \geq\Delta_{i,\pi_1}' \geq t-V(t),~i=1,\ldots,N.\label{eq_proof_2}
\end{align} 

Policy $P_1$ follows the same scheduling discipline with the preemptive MAR-MA policy. 


Hence, the delivered packet in policy $P_1$ must be from the flow with the maximum age $\Delta_{[1],P_1}$ (denoted as flow $n^*$), and the delivery packet must be generated at time $V(t)$. In other words, the age of flow $n^*$ is reduced from the maximum age $\Delta_{[1],P_1}$ to the minimum age $\Delta_{[N],P_1}'=t-V(t)$, and the age of the other $(N-1)$ flows remain unchanged. Hence, 
\begin{align}\label{eq_proof_3}
\Delta_{[i],P_1}' &= \Delta_{[i+1],P_1},~i=1,\ldots,N-1,\\
\Delta_{[N],P_1}' &= t - V(t). \label{eq_proof_4}
\end{align}

In policy $\pi_1$, the delivered packet can be any packet from any flow. For all possible cases of policy $\pi_1$, it must hold that 
\begin{align}\label{eq_proof_1}
\Delta_{[i],\pi_1}' \geq \Delta_{[i+1],\pi_1},~i=1,\ldots,N-1. 
\end{align}
By combining \eqref{hyp1}, \eqref{eq_proof_3}, and \eqref{eq_proof_1}, we have
\begin{align}
\Delta_{[i],\pi_1}' \geq \Delta_{[i+1],\pi_1} \geq \Delta_{[i+1],P_1} = \Delta_{[i],P_1}',~i=1,\ldots,N-1.\nonumber
\end{align}
In addition, combining \eqref{eq_proof_2} and \eqref{eq_proof_4}, yields
\begin{align}
\Delta_{[N],\pi_1}' \geq  t-V(t) = \Delta_{[N],P_1}'.\nonumber
\end{align}
By this, \eqref{law6} is proven.
\end{proof}
}



%Hence, under the conditions of Theorem \ref{thm1}, for all  $\mathcal{I}$ and all symmetric age penalty functions in $\mathcal{P}_{\text{sym}}$, the preemptive MAP policy is \emph{age-optimal} in terms of \eqref{thm1eq1} and \eqref{thm1eq2} among all policies in $\Pi$. 
\section{Numerical Results}
In this section, we evaluate the age performance of several multi-flow scheduling policies. These scheduling policies are defined by combining the flow selection disciplines $\{$MAF, MASIF, RAND$\}$ and the packet selection disciplines $\{$FCFS, LGFS$\}$, where  RAND represents randomly choosing a flow among the flows with un-served packets. The packet generation times $S_i$ follow a Poisson process with rate $\lambda$, and the time difference $(A_i-S_i)$ between packet generation and arrival is equal to either 0 or $4/\lambda$ with equal probability. The mean transmission time of each server is set as $\mathbb{E}[X]=1/\mu=1$. Hence, the traffic intensity is $\rho =  \lambda N/M$, where $N$ is the number of flows and $M$ is the number of servers.

\begin{figure}
\centering 
\includegraphics[width=0.45\textwidth]{figure_2_1.eps} 
\caption{Expected time-average of the average age of 50 flows in a system with 3 servers and \emph{i.i.d.} NBU service times.}
% work--efficiency ordering holds for any priorities of the jobs.
\vspace{-5mm}

\label{fig_simulation2} 
\end{figure} 

Figure \ref{fig_simulation1} illustrates the expected time-average of the maximum age $p_{\max} (\bm\age(t))$ of 3 flows in a system with a single server and \emph{i.i.d.} exponential transmission times. One can see that the P-MAF-LGFS policy has the best age performance and its age is quite small even for $\rho>1$, in which case  the queue is actually unstable. On the other hand,  both the RAND and FCFS disciplines have much higher age. Note that there is no need for preemptions under the FCFS discipline.  Figure \ref{fig_simulation2} plots the expected time-average of the average age $p_{\text{avg}} (\bm\age(t))$ of 50 flows in a system with 3 servers and \emph{i.i.d.} NBU transmission times. In particular, the  transmission time $X$ follows the following shifted exponential distribution:
\begin{align}
\Pr[X>x] = \left\{\begin{array}{l l}1,&\text{if}~x<\frac{1}{3};\\
\exp[-\frac{3}{2}(x-\frac{1}{3})],&\text{if}~x\geq \frac{1}{3}.
\end{array}\right.
\end{align}
One can observe that the NP-MASIF-LGFS policy is better than the other policies, and is quite close to the age lower bound where the gap from the lower bound is no more than the mean transmission time $\mathbb{E}[X]=1$. {\blue One interesting observation is that the NP-MASIF-LGFS policy is better than the NP-MAF-LGFS policy for NBU transmission times. The reason behind this  is as follows: When multiple servers are idle,  the NP-MAF-LGFS policy will assign these servers to process multiple packets from the flow with the maximum age (say flow $n$). This reduces the age of flow $n$, but at a cost of postponing the service of the flows with the second and third maximum ages. On the other hand, in the NP-MASIF-LGFS policy, once a packet from the flow with the maximum age of served information  (say flow $m$) is assigned to a server, the age of served information of flow $m$ drops greatly, and the next server will be assigned to process the flow with the second maximum age of served information. 
As shown in \cite{sun2016delay,sun2017delay}, using multiple parallel servers to process different flows is beneficial for NBU service times. 

%The behavior of NP-MASIF-LGFS policy is similar to the maximum matching scheduling algorithms, e.g., \cite{Joo:2009,Ji2014} for time-slotted systems, where multiple servers are assigned to process different flows in each time-slot. One difference is that the NP-MASIF-LGFS policy can even operate in continuous-time systems, but the maximum matching scheduling algorithms cannot. 
%This phenomenon suggests that the MASIF discipline deserves further investigation, which will be a research task of our future studies.
 }
%These numerical results are in accordance with our theoretical analysis in Section \ref{sec_analysis}. 





\section{Conclusion}\label{sec_conclusion}
We have proposed causal scheduling policies and developed a unifying sample-path approach to prove that these scheduling policies are (near) optimal for minimizing age of information in continuous-time and discrete-time status updating systems with multiple flows, multiple servers, and transmission errors. 

\section*{Acknowledgement}

We appreciate Elif Uysal's support throughout this endeavor. Additionally, we thank the anonymous reviewers for their valuable comments.


%A policy $P\in\Pi$ is said to be \emph{age-optimal in stochastic ordering} for minimizing the age metric process $\{p \circ\bm{\Delta}_\pi(t), t\in [0,\infty)\}$ within the policy space $\Pi$, if for all $\pi\in\Pi$
%\begin{align}\label{eq_optimal}
%\{p \circ\bm{\Delta}_P(t), t\in [0,\infty)\} \leq_{\text{st}} \{p \circ\bm{\Delta}_\pi(t), t\in [0,\infty)\},
%\end{align}
%or equivalently, for all non-decreasing functional $\phi:\mathbf{V}\mapsto\mathbb{R}$
%\begin{align}\label{eq_optimal1}
%\mathbb{E}\left[\phi (p \circ\bm{\Delta}_P)\right] =  \min_{\pi\in\Pi} \mathbb{E}\left[\phi (p \circ\bm{\Delta}_\pi)\right]
%\end{align}
%provided the expectations in \eqref{eq_optimal1} exist, 