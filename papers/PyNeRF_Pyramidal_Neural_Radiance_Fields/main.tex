\documentclass{article}
% \pdfoutput=1

% if you need to pass options to natbib, use, e.g.:
%     \PassOptionsToPackage{numbers, compress}{natbib}
% before loading neurips_2023

\usepackage{xspace}
\newcommand{\method}{PyNeRF\xspace}

% ready for submission
\PassOptionsToPackage{numbers,sort&compress}{natbib}  % before loading neurips_2023
\usepackage[breaklinks,colorlinks,citecolor=green,urlcolor=blue,bookmarks=false]{hyperref}
\usepackage[final]{neurips_2023}

% to compile a preprint version, e.g., for submission to arXiv, add add the
% [preprint] option:
    % \usepackage[preprint]{neurips_2023}


% to compile a camera-ready version, add the [final] option, e.g.:
%     \usepackage[final]{neurips_2023}


% to avoid loading the natbib package, add option nonatbib:
%    \usepackage[nonatbib]{neurips_2023}


\usepackage[utf8]{inputenc} % allow utf-8 input
\usepackage[T1]{fontenc}    % use 8-bit T1 fonts
\usepackage{hyperref}       % hyperlinks
\usepackage{url}            % simple URL typesetting
\usepackage{booktabs}       % professional-quality tables
\usepackage{amsfonts}       % blackboard math symbols
\usepackage{nicefrac}       % compact symbols for 1/2, etc.
\usepackage{microtype}      % microtypography
\usepackage{xcolor}         % colors
\usepackage{amsmath}
\usepackage{graphicx}
\usepackage{mathtools}
\usepackage{caption}
\usepackage{subcaption}
\usepackage{makecell}
\usepackage{pifont}

\usepackage{algpseudocode}
\usepackage{algorithm}
\renewcommand{\algorithmicrequire}{\textbf{Input:}}
\renewcommand{\algorithmicensure}{\textbf{Output:}}

\usepackage[capitalise,noabbrev,nameinlink]{cleveref}
\crefformat{equation}{#2Equation~#1#3}

\newcommand{\abs}[1]{\left\lvert#1\right\rvert}
\newcommand{\norm}[1]{\left\lVert#1\right\rVert}
\DeclarePairedDelimiter{\ceil}{\lceil}{\rceil}
\DeclarePairedDelimiter{\floor}{\lfloor}{\rfloor}

\definecolor{CheckGreen}{rgb}{0, 0.55, 0}
\definecolor{XRed}{RGB}{180,0,0}
\newcommand{\xmark}{{\color{XRed}\ding{55}}}

\title{\method: Pyramidal Neural Radiance Fields}


% The \author macro works with any number of authors. There are two commands
% used to separate the names and addresses of multiple authors: \And and \AND.
%
% Using \And between authors leaves it to LaTeX to determine where to break the
% lines. Using \AND forces a line break at that point. So, if LaTeX puts 3 of 4
% authors names on the first line, and the last on the second line, try using
% \AND instead of \And before the third author name.


\author{
  Haithem Turki \\
  Carnegie Mellon University \\
  \texttt{hturki@cs.cmu.edu} \\
  \And
  Michael Zollh\"{o}fer \\
  Meta Reality Labs Research \\
  \texttt{zollhoefer@meta.com} \\
  \AND
  Christian Richardt \\
  Meta Reality Labs Research \\
  \texttt{crichardt@meta.com} \\
  \And
  Deva Ramanan \\
  Carnegie Mellon University \\
  \texttt{deva@cs.cmu.edu} \\
}


\begin{document}


\maketitle


\begin{abstract}
  Neural Radiance Fields (NeRFs) can be dramatically accelerated by spatial grid representations~\cite{yu_and_fridovichkeil2021plenoxels, mueller2022instant, Chen2022ECCV, kplanes_2023}. However, they do not explicitly reason about scale and so introduce aliasing artifacts when reconstructing scenes captured at different camera distances. Mip-NeRF and its extensions propose scale-aware renderers that project volumetric frustums rather than point samples but such approaches rely on positional encodings that are not readily compatible with grid methods. We propose a simple modification to grid-based models by training model heads at different spatial grid resolutions. At render time, we simply use coarser grids to render samples that cover larger volumes. Our method can be easily applied to existing accelerated NeRF methods and significantly improves rendering quality (reducing error rates by 20–90\% across synthetic and unbounded real-world scenes) while incurring minimal performance overhead (as each model head is quick to evaluate). Compared to Mip-NeRF, we reduce error rates by 20\% while training over 60× faster.
\end{abstract}

\section{Introduction}

Recent advances in neural volumetric rendering techniques, most notably around Neural Radiance Fields~\cite{mildenhall2020nerf} (NeRFs), have
lead to significant progress towards photo-realistic novel view synthesis. However, although NeRF provides state-of-the-art rendering quality, it is notoriously slow to train and render due in part to its internal MLP representation. It further assumes that scene content is equidistant from the camera and rendering quality degrades due to aliasing and excessive blurring when that assumption is violated. 

Recent methods~\cite{mueller2022instant, yu_and_fridovichkeil2021plenoxels, kplanes_2023, Chen2022ECCV} accelerate NeRF training and rendering significantly through the use of grid representations. Others, such as Mip-NeRF~\cite{barron2021mipnerf}, address aliasing by projecting camera frustum volumes instead of point-sampling rays. However, these anti-aliasing methods rely on the base NeRF MLP representation (and are thus slow) and are incompatible with grid representations due to their reliance on non-grid-based inputs.

\begin{figure*}[t!]
\includegraphics[width=\linewidth]{FIGS/aaarf-overview.pdf}
\caption{{\bf Comparison of methods.}
\textbf{(a)} NeRF traces a ray from the camera's center of projection through each pixel and samples points $\mathbf{x}$ along each ray. Sample locations are then encoded with a positional encoding to produce a feature $\gamma(\mathbf{x})$ that is fed into an MLP.
\textbf{(b)} Mip-NeRF instead reasons about \textit{volumes} by defining a 3D conical frustum per camera pixel. It splits the frustum into sampled volumes, approximates them as multivariate Gaussians, and computes the integral of the positional encodings of the coordinates contained within the Gaussians. Similar to NeRF, these features are then fed into an MLP.
\textbf{(c)} Accelerated grid methods, such as iNGP, sample points as in NeRF, but do not use positional encoding and instead featurize each point by interpolating between vertices in a feature grid. These features are then passed into a much smaller MLP, which greatly accelerates training and rendering.
\textbf{(d)} \method also uses feature grids, but reasons about volumes by training separate models at different scales (similar to a mipmap). It calculates the area covered by each sample in world coordinates, queries the models at the closest corresponding resolutions, and interpolates their outputs.
}
  \label{fig:architecture}
\end{figure*}

Inspired by divide-and-conquer NeRF extensions~\cite{derf, reiser2021kilonerf, Turki_2022_CVPR, tancik2022blocknerf} and classical approaches such as Gaussian pyramids~\cite{gaussian_pyramids} and mipmaps~\cite{mipmaps}, we propose a simple approach that can easily be applied to any existing accelerated NeRF implementation. We train a pyramid of models at different scales, sample along camera rays (as in the original NeRF), and simply query coarser levels of the pyramid for samples that cover larger volumes (similar to voxel cone tracing~\cite{voxel-cone-tracing}). Our method is simple to implement and significantly improves the rendering quality of fast rendering approaches with minimal performance overhead.

\textbf{Contribution:} Our primary contribution is a partitioning method that can be easily adapted to any existing grid-rendering approach. We present state-of-the-art reconstruction results against a wide range of datasets, including on novel scenes we designed that explicitly target common aliasing patterns. We evaluate different posssible architectures and demonstrate that our design choices provide a high level of visual fidelity while maintaining the rendering speed of fast NeRF approaches.

\section{Related Work}

The now-seminal Neural Radiance Fields (NeRF) paper~\cite{mildenhall2020nerf} inspired a vast corpus of follow-up work. We discuss a non-exhaustive list of such approaches along axes relevant to our work.

\textbf{Grid-based methods.} The original NeRF took 1--2 days to train, with extensions for unbounded scenes~\cite{zhang2020npp, barron2022mipnerf360} taking longer. Once trained, rendering takes seconds per frame and is far below interactive thresholds. NSVF~\cite{liu2020neural} combines NeRF's implicit representation with a voxel octree that allows for empty-space skipping and improves inference speeds by 10×. Follow-up works~\cite{yu2021plenoctrees, garbin2021fastnerf, hedman2021snerg} further improve rendering to interactive speeds by storing precomputed model outputs into auxiliary grid structures that bypass the need to query the original model altogether at render time. Plenoxels~\cite{yu_and_fridovichkeil2021plenoxels} and DVGO~\cite{SunSC22} accelerate both training and rendering by directly optimizing a voxel grid instead of an MLP to train in minutes or even seconds. TensoRF~\cite{Chen2022ECCV} and K-Planes~\cite{kplanes_2023} instead use the product of low-rank tensors to approximate the voxel grid and reduce memory usage, while Instant-NGP~\cite{mueller2022instant} (iNGP) encodes features into a multi-resolution hash table. The main goal of our work is to combine the speed benefits of grid-based methods with an approach that maintains quality across different rendering scales.

\begin{figure*}[t!]
\includegraphics[width=\linewidth]{FIGS/levels.pdf}
\caption{We visualize renderings from a pyramid of spatial grid-based NeRFs trained for different voxel resolutions. Models at finer pyramid levels tend to capture finer content.}
\label{fig:level-renderings}
\end{figure*}

\textbf{Divide-and-conquer.} Several works note the diminishing returns in using large networks to represent scene content, and instead render the area of interest with multiple smaller models. DeRF~\cite{derf} and KiloNeRF~\cite{reiser2021kilonerf} focus on inference speed while Mega-NeRF~\cite{Turki_2022_CVPR}, Block-NeRF~\cite{tancik2022blocknerf}, and SUDS~\cite{turki2023suds} use scene decomposition to efficiently train city-scale neural representations. Our method is similar in philosophy, although we partition across different resolutions instead of geographical area.

\textbf{Aliasing.} The original NeRF assumes that scene content is captured at roughly equidistant camera distances and emits blurry renderings when the assumption is violated. Mip-NeRF~\cite{barron2021mipnerf} reasons about the volume covered by each camera ray and proposes an integrated positional encoding that alleviates aliasing. Mip-NeRF 360~\cite{barron2022mipnerf360} extends the base method to unbounded scenes. Exact-NeRF~\cite{isaacmedina2023exactnerf} derives a more precise integration formula that better reconstructs far-away scene content. Bungee-NeRF~\cite{xiangli2022bungeenerf} leverages Mip-NeRF and further adopts a coarse-to-fine training approach with residual blocks to train on large-scale scenes with viewpoint variation. LIRF~\cite{huang2023lirf} proposes a multiscale image-based representation that can generalize across scenes. The methods all build upon the original NeRF MLP model and do not readily translate to accelerated grid-based methods.

\textbf{Concurrent work.} Several contemporary efforts explore the intersection of anti-aliasing and fast rendering. Zip-NeRF~\cite{barron2023zipnerf} combines a hash table representation with a multi-sampling method that approximates the true integral of features contained within each camera ray's view frustum. Although it trains faster than Mip-NeRF, it is explicitly not designed for fast rendering as the multi-sampling adds significant overhead.  Mip-VoG~\cite{hu2023multiscale} downsamples and blurs a voxel grid according to the volume of each sample in world coordinates. We compare their reported numbers to ours in \cref{sec:synthetic}. Tri-MipRF~\cite{hu2023Tri-MipRF} uses a similar prefiltering approach, but with triplanes instead of a 3D voxel grid.

\textbf{Classical methods.} Similar to \method, classic image processing methods, such as Gaussian~\cite{gaussian_pyramids} and Laplacian~\cite{laplacian} hierarchy, maintain a coarse-to-fine pyramid of different images at different resolutions. Compared to Mip-NeRF, which attempts to learn a single MLP model across all scales, one could argue that our work demonstrates that the classic pyramid approach can be efficiently adapted to neural volumetric models. In addition, our ray sampling method is similar to Crassin et al.'s approach~\cite{voxel-cone-tracing}, which approximates cone tracing by sampling along camera rays and querying different mipmap levels according the spatial footprint of each sample (stored as a voxel octree in their approach and as a NeRF model in ours).

\section{Approach}

\subsection{Preliminaries}

\textbf{NeRF.} NeRF~\cite{mildenhall2020nerf} represents a scene within a continuous volumetric radiance field that captures geometry and view-dependent appearance. It encodes the scene within the weights of a multilayer perceptron (MLP). At render time, NeRF casts a camera ray $\mathbf{r}$ for each image pixel. NeRF samples multiple positions $\mathbf{x}_i$ along each ray and queries the MLP at each position (along with the ray viewing direction $\mathbf{d}$) to obtain density and color values $\sigma_i$ and $\mathbf{c}_i$. To better capture high-frequency details, NeRF maps $\mathbf{x}_i$ and $\mathbf{d}$ through an $L$-dimensional positional encoding (PE) $\gamma(x) = [\sin(2^0 \pi x), \cos(2^0 \pi x), \ldots, \sin(2^L \pi x), \cos(2^L \pi x)]$ instead of directly using them as MLP inputs. It then composites a single color prediction $\hat{C}(\mathbf{r})$ per ray using numerical quadrature $\sum_{i=0}^{N-1} T_i (1 - \exp( -\sigma_{i} \delta_{i})) \, c_i$, where $T_i = \exp( -\sum_{j=0}^{i-1} \sigma_j \delta_j)$ and $\delta_i$ is the distance between samples. The training process optimizes the model by sampling batches $\mathcal{R}$ of image pixels and minimizing the loss $\sum_{\mathbf{r} \in \mathcal{R}} \norm{C(\mathbf{r}) - \hat{C}(\mathbf{r})}^2$. We refer the reader to \citet{mildenhall2020nerf} for details.

\textbf{Anti-aliasing.} The original NeRF suffers from aliasing artifacts when reconstructing scene content observed at different distances or resolutions due to its reliance on point-sampled features. As these features ignore the volume viewed by each ray, different cameras viewing the same position from different distances may produce the same ambiguous feature. Mip-NeRF~\cite{barron2021mipnerf} and variants instead reason about \textit{volumes} by defining a 3D conical frustum per camera pixel. It featurizes intervals within the frustum with a integrated positional encoding (IPE) that approximates each frustum as a multivariate Gaussian to estimate the integral $\mathbb{E}[\gamma(x)]$ over the PEs of the coordinates within it. 

\textbf{Grid-based acceleration.} Various methods \cite{yu_and_fridovichkeil2021plenoxels, mueller2022instant, SunSC22, Chen2022ECCV, kplanes_2023} eschew NeRF's positional encoding and instead store learned features into a grid-based structure, e.g.\ implemented as an explicit voxel grid, hash table, or a collection of low-rank tensors. The features are interpolated based on the position of each sample and then passed into a hard-coded function or much smaller MLP to produce density and color, thereby accelerating training and rendering by orders of magnitude. However, these approaches all use the same volume-insensitive point sampling of the original NeRF and do not have a straightforward analogy to Mip-NeRF's IPE as they no longer use positional encoding.

\begin{figure*}[t!]
  \centering
  \begin{subfigure}[ht]{0.55\textwidth}
    \centering
    \includegraphics[width=0.9\textwidth]{FIGS/approach.001.jpg}
    \vspace*{0.3mm}
    \caption{\textbf{Point Sampling}}
  \end{subfigure}
  \begin{subfigure}[ht]{0.2\textwidth}
  \vspace*{1.5mm}
    \begin{subfigure}{\textwidth}
      \centering
      \includegraphics[width=0.6\textwidth]{FIGS/approach.002.jpg}
    \caption*{$(\mathbf{c}_8, \sigma_8) = f_8(\mathbf{x}, \mathbf{d})$}
    \end{subfigure}
    \begin{subfigure}{\textwidth}
      \centering
      \includegraphics[width=0.6\textwidth]{FIGS/approach.003.jpg}
    \caption*{$(\mathbf{c}_9, \sigma_9) = f_9(\mathbf{x}, \mathbf{d})$}
    \end{subfigure}
    \caption{\textbf{Model Evaluation}}
    \label{fig:flat}
  \end{subfigure}
  \hspace*{3mm}
  \begin{subfigure}[ht]{0.2\textwidth}
    \vspace*{12.3mm}
    \centering
    $\mathbf{c} = 0.4\mathbf{c}_8 + 0.6\mathbf{c}_9$ \\
    $\sigma = 0.4\sigma_8 + 0.6\sigma_9$
    \vspace*{12.3mm}
    \caption{\textbf{Interpolation}}
    
  \end{subfigure}
\caption{{\bf Overview.} \textbf{(a)} We sample frustums along the camera ray corresponding to each pixel and derive the scale of each sample according to its width in world coordinates. (\textbf{b}) We query the model heads closest to the scale of each sample. (\textbf{c}) We derive a single color and density value for each sample by interpolating between model outputs according to scale.}
  \label{fig:overview}
\end{figure*}


\subsection{Multiscale sampling}
\label{sec:multiscale-sampling}
Assume that each sample $\mathbf{x}$ (where we drop the $i$ index to reduce notational clutter) is associated with an integration volume.
Intuitively, samples close to a camera correspond to small volumes, while samples far away from a camera correspond to large volumes (\cref{fig:overview}).
Our crucial insight for enabling multiscale sampling with grid-based approaches is remarkably simple: \emph{we train separate NeRFs at different voxel resolutions and simply use coarser NeRFs for samples covering larger volumes}.
Specifically, we define a hierarchy of $L$ resolutions that divide the world into voxels of length $1/N_0, ..., 1/N_{L-1}$, where $N_{l+1} = sN_l$ and $s$ is a constant scaling factor.
We also define a function $f_l(\mathbf{x}, \mathbf{d})$ at each level that maps from sample location $\mathbf{x}$ and viewing direction $\mathbf{d}$ to color $\mathbf{c}$ and density $\sigma$.
$f_l$ can be implemented by any grid-based NeRF; in our experiments, we use a hash table followed by small density and color MLPs, similar to iNGP.
We further define a mapping function $M$ that assigns the integration volume of sample $\mathbf{x}$ to the hierarchy level $l$.
We explore different alternatives, but find that selecting the level whose voxels project to the 2D pixel area $P(\mathbf{x})$ used to define the integration volume works well: 
\begin{align}
& M(P(\mathbf{x})) = \log_s(P({\bf x})/N_0) \label{eq:mapping-func} \\
& l = \min(L-1, \max(0, \ceil{M(P(\mathbf{x}))})) \label{eq:get-l} \\
& \sigma, \mathbf{c} = f_l(\mathbf{x}, \mathbf{d}) \text{,} \quad\quad\quad\quad\quad\quad\quad\quad\quad\quad\quad\quad\quad\quad\quad\quad\quad\quad\quad\quad\quad\quad\quad && \textbf{[GaussPyNeRF]}
\label{eq:gauss-eq}
\end{align}
where $\ceil{\cdot}$ is the ceiling function.  Such a model can be seen as a (Gaussian) pyramid of spatial grid-based NeRFs (Fig.~\ref{fig:level-renderings}). If the final density and color were obtained by {\em summing} across different pyramid levels, the resulting levels would learn to specialize to residual or ``band-pass” frequencies (as in a 3D Laplacian pyramid~\cite{laplacian}):
\begin{align}
      & \sigma, \mathbf{c} = \sum_{i=0}^{l} f_i(\mathbf{x,d}). \quad\quad\quad\quad\quad\quad\quad\quad\quad\quad\quad\quad\quad\quad\quad\quad\quad\quad\quad\quad && \textbf{[LaplacianPyNeRF]} \label{eq:residual}
\end{align}
Our experiments show that such a representation is performant, but expensive since it requires $l$ model evaluations per sample. Instead, we find a good tradeoff is to linearly interpolate between two model evaluations at the levels
just larger than and smaller than the target integration volume:
\begin{align}
      & \sigma, \mathbf{c} = w f_l(\mathbf{x,d}) + (1-w)f_{l-1} (\mathbf{x,d}) \text{,} \quad \text{where} \quad w = l - M(P(\mathbf{x})). && \textbf{(Default) [PyNeRF]} 
\label{eq:default-eq}
\end{align}
This adds the cost of only a \emph{single} additional evaluation (increasing the overall rendering time from 0.0045 to 0.005 ms per pixel) while maintaining rendering quality (see \cref{sec:diagnostics}). Our algorithm is summarized in \cref{alg:render-alg}.

\begin{algorithm}[t!]
\caption{\method\ rendering function}\label{alg:render-alg}
\begin{algorithmic}
\Require $m$ rays \textbf{r}, $L$ pyramid levels, hierarchy mapping function $M$, base resolution $N_0$, scaling factor $s$
\Ensure $m$ estimated colors \textbf{c}
\State $\textbf{x}, \textbf{d}, P(\textbf{x}) \gets sample(\textbf{r})$ \Comment{Sample points $\textbf{x}$ along each ray with direction $\textbf{d}$ and area $P(\textbf{x})$} 
\State $M(P(\mathbf{x})) \gets \log_s(P({\bf x})/N_0)$ \Comment{\cref{eq:mapping-func}}
\State $l \gets \min(L-1, \max(0, \ceil{M(P(\mathbf{x}))}))$ \Comment{\cref{eq:get-l}}
\State $w \gets l - M(P(\mathbf{x}))$ \Comment{\cref{eq:default-eq}}
\State $model\_out \gets zeros(len(\textbf{x}))$ \Comment{Zero-initialize model outputs for each sample \textbf{x}}
\For{$i$ in unique($l$)} \Comment{Iterate over sample levels}
    \State $model\_out[l = i] \mathrel{{+}{=}} w[l = i]f_i(\textbf{x}[l = i], \textbf{d}[l = i])$
    \State $model\_out[l = i] \mathrel{{+}{=}} (1 - w)[l = i]f_{i-1}(\textbf{x}[l = i], \textbf{d}[l = i])$ 
\EndFor
\State $\textbf{c} \gets composite(model\_out)$ \Comment{Composite model outputs into per-ray color \textbf{c}}
\State \Return \textbf{c}
\end{algorithmic}
\end{algorithm}

{\bf Matching areas vs volumes.} One might suspect it may be better to select the voxel level $l$ whose volume best matches the sample's 3D integration volume. We experimented with this, but found it more effective to match the projected 2D pixel area rather than volumes. Note that both approaches would produce identical results if the 3D volume was always a cube, but volumes may be elongated along the ray depending on the sampling pattern. Matching areas is preferable because most visible 3D scenes consist of empty space and surfaces, implying that when computing the composite color for a ray $r$, most of the contribution will come from a few
samples $\mathbf{x}$ lying near the surface of intersection. When considering the target 3D integration volume associated with $\mathbf{x}$, most of the contribution to the final composite color will come from integrating along the 2D surface (since the rest of the 3D volume is either empty or hidden). This loosely suggests we should select levels of the voxel hierarchy based on (projected) area rather than volume. 

{\bf Hierarchical grid structures.} Our method can be applied to any accelerated grid method irrespective of the underyling storage. However, a drawback of this approach is an increased on-disk serialization footprint due to training a hierarchy of spatial grid NeRFs. A possible solution is to exploit hierarchical grid structures that already exist {\em within} the base NeRF.  Note that multi-resolution grids such as those used by iNGP~\cite{mueller2022instant} or K-Planes~\cite{kplanes_2023} already define a scale hierarchy that is a natural fit for \method. Rather than learning a separate feature grid for each model in our pyramid, we can reuse the same multi-resolution features across levels (while still training different MLP heads).

{\bf Multi-resolution pixel input.} One added benefit of the above is that one can train with multiscale training data, which is particularly helpful for learning large, city-scale NeRFs~\cite{Turki_2022_CVPR, tancik2022blocknerf, xiangli2022bungeenerf, turki2023suds, xu2023gridguided}. For such scenarios, even storing high-resolution pixel imagery may be cumbersome. In our formulation, one can store low-resolution images and quickly train a coarse scene representation. The benefits are multiple. Firstly, divide-and-conquer approaches such as Mega-NeRF~\cite{turki2023suds} partition large scenes into smaller cells and train using different training pixel/ray subsets for each (to avoid training on irrelevant data). However, in the absence of depth sensors or a priori 3D scene knowledge, Mega-NeRF is limited in its ability to prune irrelevant pixels/rays (due to intervening occluders) which empirically bloat the size of each training partition by 2×~\cite{Turki_2022_CVPR}. With our approach, we can learn a coarse 3D knowledge of the scene on downsampled images and then filter higher-resolution data partitions more efficiently. Once trained, lower-resolution levels can also serve as an efficient initialization for finer layers. In addition, many contemporary NeRF methods use occupancy grids~\cite{mueller2022instant} or proposal networks~\cite{barron2022mipnerf360} to generate refined samples near surfaces. We can quickly train these along with our initial low-resolution model and then use them to train higher-resolution levels in a sample-efficient manner. We show in our experiments that such course-to-fine multiscale training can speed up convergence (\cref{sec:city-scale}).

{\bf Unsupervised levels.} A naive implementation of our method will degrade when zooming in and out of areas that have not been seen at training time. Our implementation mitigates this by maintaining an auxiliary data structure (similar to an occupancy grid~\cite{mueller2022instant}) that tracks the coarsest and finest levels queried in each region during training. We then use the structure at inference time to only query levels that were supervised during training.

\section{Experiments}

We first evaluate \method's performance by measuring its reconstruction quality on bounded synthetic (\cref{sec:synthetic}) and unbounded real-world (\cref{sec:real-world}) scenes. 
We demonstrate \method's generalizability by evaluating it on additional NeRF backbones (\cref{sec:more-backbones}) and then explore the convergence benefits of using multiscale training data in city-scale reconstruction scenarios (\cref{sec:city-scale}). We ablate our design decisions in \cref{sec:diagnostics}.


\subsection{Experimental Setup}

\begin{table*}
\caption{\textbf{Synthetic results.} \method outperforms all baselines and trains over 60× faster than Mip-NeRF. Both \method and Mip-NeRF properly reconstruct the brick wall in the Blender-A dataset, but Mip-NeRF fails to accurately reconstruct checkerboard patterns.}
\centering
\resizebox{\linewidth}{!}{
\begin{tabular}{l@{\hspace{1em}}c@{\hspace{1em}}c@{\hspace{1em}}c@{\hspace{1em}}c@{\hspace{1em}}c@{\hspace{2em}}c@{\hspace{1em}}c@{\hspace{1em}}c@{\hspace{1em}}c@{\hspace{1em}}c@{\hspace{1em}}}
\toprule 
& \multicolumn{5}{c}{Multiscale Blender~\cite{barron2021mipnerf}} & \multicolumn{5}{c}{Blender-A} \\ \cmidrule(lr){2-6}\cmidrule(lr){7-11}
& $\uparrow$PSNR & $\uparrow$SSIM & $\downarrow$LPIPS & $\downarrow$Avg Error & $\downarrow$Train Time (h) 
& $\uparrow$PSNR & $\uparrow$SSIM & $\downarrow$LPIPS & $\downarrow$Avg Error & $\downarrow$ Train Time (h) \\ \midrule
Plenoxels~\cite{yu_and_fridovichkeil2021plenoxels} & 24.98 & 0.843 & 0.161 & 0.080 & \phantom{0}0:28
& 18.13 & 0.511 & 0.523 & 0.190 & \phantom{0}\textbf{0:40} \\
K-Planes~\cite{kplanes_2023} & 29.88 & 0.946 & 0.058 & 0.022 & \phantom{0}0:32
& 21.17 & 0.593 & 0.641 & 0.405 & \phantom{0}1:22 \\
TensoRF~\cite{Chen2022ECCV} & 30.04 & 0.948 & 0.056 & 0.021 & \phantom{0}0:27
& 27.01 & 0.785 & 0.197 & 0.054 & \phantom{0}1:20 \\
iNGP~\cite{mueller2022instant} & 30.21 & 0.958 & 0.040 & 0.022 & \phantom{0}\textbf{0:20}
& 20.85 & 0.767 & 0.244 & 0.089 & \phantom{0}\underline{0:56} \\
Nerfacto~\cite{nerfstudio} & 29.56 & 0.947 & 0.051 & 0.022 & \phantom{0}\underline{0:25}
& 27.46 & 0.796 & 0.195 & \underline{0.053} & \phantom{0}1:07 \\
Mip-VoG~\cite{hu2023multiscale} & 30.42 & 0.954 & 0.053 & --- & \phantom{0}--- & ---
& --- & --- & ---	& \phantom{0}--- \\ 
Mip-NeRF~\cite{barron2021mipnerf} & \underline{34.50} & \underline{0.974} & \underline{0.017} & \underline{0.009} & 29:49
& \underline{31.33} & \underline{0.894}	& \underline{0.098}	& 0.063 & 30:12 \\
\midrule
\method & \textbf{34.78} & \textbf{0.976} & \textbf{0.015} & \textbf{0.008} & \phantom{0}\underline{0:25}
& \textbf{41.99} & \textbf{0.986} & \textbf{0.007} & \textbf{0.004} & \phantom{0}1:10 \\

\bottomrule
\end{tabular}
}
\label{table:synthetic-results}
\end{table*}

{\bf Training.}
We implement \method on top of the Nerfstudio library \cite{nerfstudio} and train on each scene with 8,192 rays per batch by default for 20,000 iterations on the Multiscale Blender and Mip-NeRF 360 datasets, and 50,000 iterations on the Boat dataset and Blender-A. We train a hierarchy of 8 PyNeRF levels backed by a single multi-resolution hash table similar to that used by iNGP~\cite{mueller2022instant} in \cref{sec:synthetic} and \cref{sec:real-world} before evaluating additional backbones in \cref{sec:more-backbones}.  We use 4 features per level with a hash table size of $2^{20}$ by default, which we found to give the best quality-performance trade-off on the A100 GPUs we use in our experiments. Each PyNeRF uses a 64-channel density MLP with one hidden layer followed by a 128-channel color MLP with two hidden layers. We use similar model capacities in our baselines for fairness. We sample rays using an occupancy grid~\cite{mueller2022instant} on the Multiscale Blender dataset, and with a proposal network~\cite{barron2022mipnerf360} on all others. We use gradient scaling~\cite{gradient_scaling} to improve training stability in scenes with that capture content at close distance (Blender-A and Boat). We parameterize unbounded scenes with Mip-NeRF 360's contraction method.

{\bf Metrics.}
We report quantitative results based on PSNR, SSIM~\cite{1284395}, and the AlexNet implementation of LPIPS~\cite{zhang2018perceptual}, along with the training time in hours as measured on a single A100 GPU.
For ease of comparison, we also report the ``average” error metric proposed by Mip-NeRF~\cite{barron2021mipnerf} composed of the geometric mean of $\mathrm{MSE}=10^{-\mathrm{PSNR}/10}$, $\sqrt{1-\mathrm{SSIM}}$, and LPIPS.



\subsection{Synthetic Reconstruction}
\label{sec:synthetic}

\begin{figure*}[t!]
\includegraphics[width=\linewidth]{FIGS/synthetic-results.pdf}
\caption{{\bf Synthetic results.} \method and Mip-NeRF provide comparable results on the first three scenes that are crisper than those of the other fast renderers. Mip-NeRF does not accurately render the tiles in the last row while \method recreates them near-perfectly.}
\label{fig:synthetic-results}
\end{figure*}

\textbf{Datasets.}
We evaluate \method on the Multiscale Blender dataset proposed by Mip-NeRF along with our own Blender scenes (which we name ``Blender-A'') intended to further probe the anti-aliasing ability of our approach (by reconstructing a slanted checkerboard and zooming into a brick wall).

\textbf{Baselines.}
We compare \method to several fast-rendering approaches, namely Instant-NGP~\cite{mueller2022instant} and Nerfacto~\cite{nerfstudio}, which store features within a multi-resolution hash table, Plenoxels~\cite{yu_and_fridovichkeil2021plenoxels} which optimizes an explicit voxel grid, and TensoRF~\cite{Chen2022ECCV} and K-Planes~\cite{kplanes_2023}, which rely on low-rank tensor decomposition.
We also compare our Multiscale Blender results to those reported by Mip-VoG~\cite{hu2023multiscale}, a contemporary fast anti-aliasing approach, and to Mip-NeRF~\cite{barron2021mipnerf} on both datasets.

\textbf{Results.}
We summarize our results in \cref{table:synthetic-results} and show qualitative examples in \cref{fig:synthetic-results}.
\method outperforms all fast rendering approaches as well as Mip-VoG by a wide margin and is slightly better than Mip-NeRF on Multiscale Blender while training over 60× faster.
Both \method and Mip-NeRF properly reconstruct the brick wall in the Blender-A dataset, but Mip-NeRF fails to accurately reconstruct checkerboard patterns.

\subsection{Real-World Reconstruction}
\label{sec:real-world}

\textbf{Datasets.}
We evaluate \method on the Boat scene of the ADOP~\cite{adop} dataset, which to our knowledge is one of the only publicly available unbounded real-world captures that captures its primary object of interest from different camera distances.
For further comparison, we construct a multiscale version of the outdoor scenes in the Mip-NeRF 360~\cite{barron2022mipnerf360} dataset using the same protocol as Multiscale Blender~\cite{barron2021mipnerf}.

\textbf{Baselines.}
We compare \method to the same fast-rendering approaches as in \cref{sec:synthetic}, along with two unbounded Mip-NeRF variants: Mip-NeRF 360~\cite{barron2022mipnerf360} and Exact-NeRF~\cite{isaacmedina2023exactnerf}.
We report numbers on each variant with and without generative latent optimization~\cite{martinbrualla2020nerfw} to account for lighting changes.

\textbf{Results.}
We summarize our results in \cref{table:real-world-results} along with qualitative results in \cref{fig:unbounded-results}.
Once again, \method outperforms all baselines, trains 40× faster than Mip-NeRF 360, and 100× faster than Exact-NeRF (the next best alternatives).

\subsection{Additional Backbones}
\label{sec:more-backbones}

\textbf{Methods.} We demonstrate how \method can be applied to any grid-based NeRF method by evaluating it with K-Planes~\cite{kplanes_2023} and TensoRF~\cite{Chen2022ECCV} in addition to our default iNGP-based implementatino. We take advantage of the inherent multi-resolution structure of iNGP and K-Planes by reusing the same feature grid across PyNeRF levels and train a separate feature grid per level in our TensoRF variant.

\textbf{Results.}
We train the \method\ variants along with their backbones across the datasets described in \cref{sec:synthetic} and \cref{sec:real-world}, and summarize the results in \cref{table:backbones}. All \method\ variants show clear improvements over their base methods.

\subsection{City-Scale Convergence}
\label{sec:city-scale}

\begin{figure}[t!]
\includegraphics[width=\linewidth]{FIGS/unbounded-results.pdf}
\caption{{\bf Real-world results.} \method reconstructs higher-fidelity details (such as the spokes on the bicycle and the lettering within the boat) than other methods.}
\label{fig:unbounded-results}
\end{figure}

\textbf{Dataset.}
We evaluate \method's convergence properties on the the Argoverse 2~\cite{Argoverse2} Sensor dataset (to our knowledge, the largest city-scale dataset publicly available).
We select the largest overlapping subset of logs and filter out moving objects through a pretrained segmentation model~\cite{cheng2020panoptic}.
The resulting training set contains 400 billion rays across 150K video frames.

\begin{table}
\caption{{\bf Real-world results.} \method outperforms all baselines in PSNR and average error, and trains 40× faster than Mip-NeRF 360 and 100× faster than Exact-NeRF (the next best methods).}
\centering
\resizebox{\linewidth}{!}{
\begin{tabular}{l@{\hspace{1em}}c@{\hspace{1em}}c@{\hspace{1em}}c@{\hspace{1em}}c@{\hspace{1em}}c@{\hspace{2em}}c@{\hspace{1em}}c@{\hspace{1em}}c@{\hspace{1em}}c@{\hspace{1em}}c@{\hspace{1em}}}
\toprule 
& \multicolumn{5}{c}{Boat~\cite{adop}} & \multicolumn{5}{c}{Mip-NeRF 360~\cite{barron2022mipnerf360}} \\ \cmidrule(lr){2-6}\cmidrule(lr){7-11}
& $\uparrow$PSNR & $\uparrow$SSIM & $\downarrow$LPIPS & $\downarrow$Avg Error & $\downarrow$Train Time (h) 
& $\uparrow$PSNR & $\uparrow$SSIM & $\downarrow$LPIPS & $\downarrow$Avg Error & $\downarrow$ Train Time (h) \\ \midrule
Plenoxels~\cite{yu_and_fridovichkeil2021plenoxels} & 17.05 & 0.505 & 0.617 & 0.185 & \phantom{00}2:14
& 21.88 & 0.606 & 0.524 & 0.117 & \phantom{00}\underline{1:00} \\
K-Planes~\cite{kplanes_2023} & 18.00 & 0.501 & 0.590 & 0.168 & \phantom{00}2:41
& 21.53 & 0.577 & 0.500 & 0.120 & \phantom{00}1:08 \\
TensoRF~\cite{Chen2022ECCV} & 14.75 & 0.398 & 0.630 & 0.234 & \phantom{00}2:30
& 18.07 & 0.439 & 0.677 & 0.181 & \phantom{00}1:07 \\
iNGP~\cite{mueller2022instant} & 15.34 & 0.433 & 0.646 & 0.222 & \phantom{00}\textbf{1:42}
& 21.14 & 0.568 & 0.521 & 0.126 & \phantom{00}\textbf{0:40} \\
Nerfacto~\cite{nerfstudio} & 19.27 & 0.570 & 0.425 & 0.135 & \phantom{00}\underline{2:12}
& 22.47 & 0.616 & 0.431 & 0.105 & \phantom{00}1:02 \\
Mip-NeRF 360 w/ GLO~\cite{barron2022mipnerf360} & 20.03 & \underline{0.595} & \textbf{0.416} & 0.124 & \phantom{0}37:28
& \underline{22.76} & \textbf{0.664} & \textbf{0.342} & \underline{0.095} & \phantom{0}37:35 \\
Mip-NeRF 360 w/o GLO~\cite{barron2022mipnerf360} & 15.92 & 0.480 & 0.501 & 0.194 & \phantom{0}37:10
& 22.70 & \textbf{0.664} & \textbf{0.342} & \underline{0.095} & \phantom{0}37:22 \\
Exact-NeRF w/ GLO~\cite{isaacmedina2023exactnerf} & \underline{20.21} & \textbf{0.601} & 0.425 & \underline{0.123} & 109:11
& 21.40 & 0.619 & 0.416 & 0.121 & 110:06 \\
Exact-NeRF w/o GLO~\cite{isaacmedina2023exactnerf} & 16.33 & 0.489 & 0.510 & 0.187 & 107:52
& 22.56 & 0.619 & 0.410 & 0.121 & 108:11 \\
\midrule
\method & \textbf{20.43} & \textbf{0.601} & \underline{0.422} & \textbf{0.121} & \phantom{00}\underline{2:12}
& \textbf{23.09} & \underline{0.654} & \underline{0.358} & \textbf{0.094} & \phantom{00}\underline{1:00} \\

\bottomrule
\end{tabular}
}
\label{table:real-world-results}
\end{table}

\textbf{Methods.}
We use SUDS~\cite{turki2023suds} as the backbone model in our experiments. We begin training our method on 8× downsampled images (containing 64× fewer rays) for 5,000 iterations and then on progressively higher resolutions (downsampled to 4×, 2×, and 1×) every 5,000 iterations hereafter. We compare to the original SUDS method as a baseline.

\textbf{Metrics.}
We report the evolution of the quality metrics used in \cref{sec:synthetic} and \cref{sec:real-world} over the first four hours of the training process.

\textbf{Results.} We summarize our results in \cref{table:city-scale}. \method\ converges more rapidly than the SUDS baseline, achieving the same rendering quality at 2 hours as SUDS after 4.

\begin{table}
\caption{\textbf{Additional backbones.} We train the \method\ variants along with their backbones across the datasets described in \cref{sec:synthetic} and \cref{sec:real-world} All \method\ variants outperform their baselines by a wide margin.}
\centering
\footnotesize

\begin{tabular}{l@{\hspace{1em}}c@{\hspace{1em}}c@{\hspace{1em}}c@{\hspace{1em}}c@{\hspace{2em}}c@{\hspace{1em}}c@{\hspace{1em}}c@{\hspace{1em}}c@{\hspace{1em}}}
\toprule 
& \multicolumn{4}{c}{Synthetic} & \multicolumn{4}{c}{Real-World} 
\\ \cmidrule(lr){2-5}\cmidrule(lr){6-9}
& $\uparrow$PSNR & $\uparrow$SSIM & $\downarrow$LPIPS & $\downarrow$Avg Error
& $\uparrow$PSNR & $\uparrow$SSIM & $\downarrow$LPIPS & $\downarrow$Avg Error \\ \midrule
iNGP~\cite{mueller2022instant} & 28.86 & 0.916 & 0.087 & 0.032
& 19.94 & 0.541 & 0.537 & 0.146 \\
K-Planes~\cite{kplanes_2023} & 27.90 & 0.865 & 0.131 & 0.047
& 20.54 & 0.553 & 0.520 & 0.136 \\
TensoRF~\cite{Chen2022ECCV} & 29.12 & 0.902 & 0.100 & 0.042
& 17.21 & 0.421 & 0.696 & 0.200 \\
\midrule
\method & \textbf{36.22} & \textbf{0.979} & \textbf{0.013} & \textbf{0.004}
& \textbf{22.65} & \textbf{0.645} & \textbf{0.369} & \textbf{0.098}  \\
\method-K-Planes & 35.42 & 0.975 & \underline{0.014} & \underline{0.005}
& \underline{22.00} & \underline{0.622} & \underline{0.405} & \underline{0.108}  \\
\method-TensoRF & \underline{35.67} & \underline{0.976} & 0.015 & \underline{0.005}
& 21.35 & 0.568 & 0.482 & 0.122  \\

\bottomrule
\end{tabular}
\label{table:backbones}
\end{table}

\begin{table}
\caption{\textbf{City-scale convergence.} We track rendering quality over the first four hours of training. \method\ achieves the same rendering quality as SUDS 2× faster.}
\centering
\footnotesize
\subcaptionbox*{\textbf{$\uparrow$ PSNR}}{
\begin{tabular}{lcccc}
\toprule
Time (h) & 1:00 & 2:00 & 3:00 & 4:00 \\ \midrule
SUDS~\cite{turki2023suds} & 16.01 & 17.41 & 18.08 & 18.53 \\
\method & \textbf{17.17} & \textbf{18.44} & \textbf{18.59} & \textbf{18.73} \\
\end{tabular}
}
\subcaptionbox*{\textbf{$\uparrow$ SSIM}}{
\begin{tabular}{lcccc}
\toprule
Time (h) & 1:00 & 2:00 & 3:00 & 4:00 \\ \midrule
SUDS~\cite{turki2023suds} & 0.570 & 0.600 & 0.602 & 0.606 \\
\method & \textbf{0.614} & \textbf{0.618} & \textbf{0.619} & \textbf{0.621} \\
\end{tabular}
}
\subcaptionbox*{\textbf{$\downarrow$ LPIPS}}{
\begin{tabular}{lcccc}
\toprule
Time (h) & 1:00 & 2:00 & 3:00 & 4:00 \\ \midrule
SUDS~\cite{turki2023suds} & 0.531 & 0.496 & 0.470 & 0.466 \\
\method & \textbf{0.521} & \textbf{0.485} & \textbf{0.469} & \textbf{0.465} \\
\end{tabular}
}
\subcaptionbox*{\textbf{$\downarrow$ Avg Error}}{
\begin{tabular}{lcccc}
\toprule
Time (h) & 1:00 & 2:00 & 3:00 & 4:00 \\ \midrule
SUDS~\cite{turki2023suds}  & 0.182 & 0.160 & 0.150 & 0.145 \\
\method & \textbf{0.165} & \textbf{0.146} & \textbf{0.144} & \textbf{0.142} \\
\end{tabular}
}
\label{table:city-scale}
\end{table}

\subsection{Diagnostics}
\label{sec:diagnostics}

\textbf{Methods.}
We validate our design decisions by testing several variants.
We ablate our MLP-level interpolation described in \cref{eq:default-eq} and compare it to the Gausss\method and Laplacian\method variants described in \cref{sec:multiscale-sampling} along with another that instead interpolates the learned grid feature vectors (which avoids the need for an additional MLP evaluation per sample). As increased storage footprint is a potential drawback method, we compare our default strategy of sharing the same multi-resolution feature grid across \method\ levels to the naive implementation that trains a separate grid per level. We also explore using 3D sample volumes instead of projected 2D pixel areas to determine voxel levels $l$.

\textbf{Results.}
We train our method and variants as described in \cref{sec:synthetic} and \cref{sec:real-world}, and summarize the results (averaged across datasets) in \cref{table:diagnostics}.
Our proposed interpolation method strikes a good balance --- its performance is near-identical to the full Laplacian\method approach while training 3× faster (and is significantly better than the other interpolation methods). Our strategy of reusing the same feature grid across levels performs comparably to the naive implementation while training faster due to fewer feature grid lookups. Using 2D pixel areas instead of 3D volumes to determine voxel level $l$ provides an improvement.

\begin{table}
\caption{\textbf{Diagnostics.} The rendering quality of our interpolation method is near-identical to the full residual approach while training 3× faster, and is significantly better than other alternatives. Reusing the same feature grid across levels performs comparably to storing separate hash tables per level while training faster.}
\centering
\resizebox{\linewidth}{!}{
\begin{tabular}{l||ccc|ccccc}
\toprule
Method & 
\makecell{Our\\Interp.} &
\makecell{Shared\\Features} &
\makecell{2D\\Area} &
$\uparrow$PSNR &
$\uparrow$SSIM & 
$\downarrow$LPIPS &
\makecell{$\downarrow$ Avg\\Error} &
\makecell{$\downarrow$ Train \\Time (h)} \\ \midrule

GaussPyNeRF (Eq.~\ref{eq:gauss-eq}) & \xmark & \textcolor{CheckGreen}{\checkmark} & \textcolor{CheckGreen}{\checkmark} & 28.72 & 0.803 & 0.201 & 0.056 & \textbf{0:43} \\
LaplacianPyNeRF (Eq.~\ref{eq:residual}) & \xmark & \textcolor{CheckGreen}{\checkmark} & \textcolor{CheckGreen}{\checkmark} 
& \textbf{29.48} & \textbf{0.813} & \textbf{0.190} & \textbf{0.052} & 2:44 \\
Feature grid interpolation & \xmark & \xmark & \textcolor{CheckGreen}{\checkmark} 
& 28.45 & 0.767 & 0.244 & 0.070	& \underline{0:46} \\
Separate hash tables & \textcolor{CheckGreen}{\checkmark} & \xmark & \textcolor{CheckGreen}{\checkmark} 
& 29.41 & \textbf{0.813} & 0.196 & 0.054	& 0:52 \\
Levels w/ 3D Volumes & \textcolor{CheckGreen}{\checkmark} & \textcolor{CheckGreen}{\checkmark} & \xmark
& 29.19 & 0.811 & 0.184 & 0.054 & 0:48 \\
\midrule
\method & \textcolor{CheckGreen}{\checkmark} & \textcolor{CheckGreen}{\checkmark} & \textcolor{CheckGreen}{\checkmark} 
& \underline{29.44} & \underline{0.812} & \underline{0.191} & \underline{0.053} & 0:48 \\
\end{tabular}
}
\label{table:diagnostics}
\end{table}

\section{Limitations}
\label{sec:limitations}

Although our method generalizes to any grid-based method (\cref{sec:more-backbones}), it requires a larger on-disk serialization footprint due to training a hierarchy of spatial grid NeRFs. This can be mitigated by reusing the same feature grid when the underlying backbone uses a multi-resolution feature grid~\cite{mueller2022instant, kplanes_2023}, but this is not true of all methods~\cite{Chen2022ECCV, yu_and_fridovichkeil2021plenoxels}.

\section{Societal Impact}

Our method facilitates the rapid construction of high-quality neural representations in a resource efficient manner. As such, the risks inherent to our work is similar to those of other neural rendering papers, namely privacy and security concerns related to the intentional or inadvertent capture or privacy-sensitive information such as human faces and vehicle license plate numbers. While we did not apply our approach to data with privacy or security concerns, there is a risk, similar to other neural rendering approaches, that such data could end up in the trained model if the employed datasets are not properly filtered before use. Many recent approaches~\cite{Zhi:etal:ICCV2021, kobayashi2022distilledfeaturefields, tschernezki22neural, turki2023suds, lerf2023} distill semantics into NeRF's representation, which may be used to filter out sensitive information at render time. However this information would still reside in the model itself. This could in turn be mitigated by preprocessing the input data used to train the model~\cite{10.1145/3083187.3083192}.

\section{Conclusion}

We propose a method that significantly improves the anti-aliasing properties of fast volumetric renderers.
Our approach can be easily applied to any existing grid-based NeRF, and although simple, provides state-of-the-art reconstruction results against a wide variety of datasets (while training 60--100× faster than existing anti-aliasing methods).
We propose several synthetic scenes that model common aliasing patterns as few existing NeRF datasets cover these scenarios in practice. Creating and sharing additional real-world captures would likely facilitate further research.

{\bf Acknowledgements.} HT and DR were supported in part by the Intelligence Advanced Research Projects Activity (IARPA) via Department of Interior/ Interior Business Center (DOI/IBC) contract number 140D0423C0074. The U.S. Government is authorized to reproduce and distribute reprints for Governmental purposes notwithstanding any copyright annotation thereon. Disclaimer: The views and conclusions contained herein are those of the authors and should not be interpreted as necessarily representing the official policies or endorsements, either expressed or implied, of IARPA, DOI/IBC, or the U.S. Government.

\clearpage

{\small
\bibliographystyle{abbrvnat}
\bibliography{main}
}

\appendix


\section{Notation}

Matrices are denoted with capital letters such as $A$, and vectors with lower case $a$.  In situations where we partition a matrix into pieces, the partitions will be referred to as $A_{ij}$.  Individual entries in a matrix will be referred to as lower case letters with two subscripts, $a_{ij}$.  
$\sigma_k(A)$ denotes the $k$-th leading singular value of $A$, and $\kappa(A)$ the condition number.
For a matrix $A\in\R^{n\times m}$ we let $A_{\mathcal{J},\I}$ denote a sub-selection of the matrix $A$ using sets $\mathcal{J}\subset [n]$ to denote the selected rows and $\I\subset [m]$ to denote the selected columns; $:$ denotes a selection of all rows or columns.

\section{Fixed-rank interpolative decompositions}
%\paragraph{Further applications: choosing important sub-samples}
%Aside from pruning, interpolative decompositions can be used to select an important sub-sample of the %training
%data which can act as a surrogate for the whole set.
%Following our notation in Section~\ref{sec:pruneID} for a one hidden layer network, let $Z \in \R^{m \times n}$ be the first-layer output, i.e. $Z = g(W^\top X)$.
%Now we wish to preserve the network outputs with fewer data points by computing a rank-$k$ interpolative decomposition $Z^\top \approx (X^\top)_{:,\I} H$ where $H$ is an interpolation matrix.
%In particular, we compute a rank-$k$ interpolative decomposition of 
%%$\gamma(W^\top X)$ (i.e., 
%$Z^\top$ denoted $\gamma(W^\top X) \approx \gamma(W^\top X_{:,\mathcal{J}})H$, where $|\mathcal{J}|=k$ is the subset of the data and $H$ the associated interpolation matrix. 

As stated in Section~\ref{sec:ID}, a formal algorithmic statement is given for computing fixed-rank interpolative decompositions.
    
\begin{algorithm}[ht]%t
\begin{algorithmic}
\label{alg:genericID}
%\INPUT
\REQUIRE
%\KwIn{
matrix $A \in \R^{n \times m}$, rank-$k$
%}
%\OUTPUT
\ENSURE
%\KwOut{
interpolative decomposition $A_{:,\I} T$
%}
\STATE Compute column-pivoted QR factorization 
\[
    A
    \begin{bmatrix}
    \Pi_1 & \Pi_2
    \end{bmatrix}
    =
    \begin{bmatrix}
    Q_1 & Q_2
    \end{bmatrix}
    \begin{bmatrix}
    R_{11} & R_{12} \\
    & R_{22}
    \end{bmatrix}.
\]
where
$\Pi_1 \in \R^{m \times k}$, 
%$\Pi_2 \in \R^{m \times (m-k)}$, 
%$Q_1 \in \R^{n \times k}$, 
%$Q_2 \in \R^{n \times (\ell-k)}$, 
$R_{11} \in \R^{k \times k}$, 
$R_{12} \in \R^{k \times (m-k)}$, 
%and $R_{22} \in \R^{(\ell-k)\times(\ell-n)}$
and remaining dimensions as required.
\;
\STATE $A_{:,\I} \gets A \Pi_1$ \;
\STATE $T \gets 
\begin{bmatrix}
I_k & R_{11}^{-1} R_{12}
\end{bmatrix}
\Pi^\top$\;
\end{algorithmic}
%\SetAlgoLined
%\DontPrintSemicolon
\caption{Interpolative Decomposition}
\end{algorithm}

\section{Proofs}

\begin{customthm}{\ref{thm:generalization}}
%\label{thm:generalization}
Consider a model $h_{FC}=u^\top g(W^\top x)$ with m hidden neurons and a pruned model $\widehat{h}_{FC}=\widehat{u}^\top g(\widehat{W}^\top x)$ constructed using an $\epsilon$ accurate ID with $n$ data points drawn i.i.d\ from $\cD.$ The risk of the pruned model $\mathcal{R}_p$ on a data set $(x,y) \sim D$ assuming $\cD$ is compactly supported on $\Omega_x\times\Omega$ is bounded by  
\begin{equation*}
    \mathcal{R}_p \leq \mathcal{R}_{ID} + \mathcal{R}_0+ 2  \sqrt{ \mathcal{R}_{ID}  \mathcal{R}_0},
\end{equation*}
where $\mathcal{R}_{ID}$ is the risk associated with approximating the full model by a pruned one and with probability $1-\delta$ satisfies
\begin{equation*}
    {\mathcal{R}}_{ID} \leq \epsilon^2M+M(1+\|T\|_2)^2n^{-\frac{1}{2}} \left( \sqrt{2\zeta dm \log (dm)\log\frac{en}{\zeta dm \log (dm)}}+ \sqrt{\frac{\log (1/\delta)}{2}}\right).
\end{equation*} 
Here, $M = \sup_{x\in\Omega_x} \|u\|_2^2 \| g(W^T x)\|_2^2$ and $\zeta$ is a universal constant that depends on $g$. %the activation function.  


\end{customthm}


\begin{proof}
We can write the risk for this network as
\begin{equation*}
    \mathcal{R}_p=\mathbb{E}(\|\widehat{u}^\top g(\widehat{W}^\top x)- y\|^2),
\end{equation*} and adding and subtract the original network yields

\begin{equation*}
\begin{aligned}
    \mathbb{E}(\|\widehat{u}^\top g(\widehat{W}^\top x)- y\|^2) &= \mathbb{E}(\|(\widehat{u}^\top g(\widehat{W}^\top x)-u^\top g (w^\top x))+(u^\top g (w^\top x)- y)\|^2)\\
    &\leq \mathbb{E}((\|(\widehat{u}^\top g(\widehat{W}^\top x)-u^\top g (w^\top x))\|+\|(u^\top g (w^\top x)- y)\|)^2)\\
   &\leq \mathbb{E}(\|(\widehat{u}^\top g(\widehat{W}^\top x)-u^\top g (w^\top x))\|^2)\\
   &\phantom{\leq}+ \mathbb{E}(2\|(\widehat{u}^\top g(\widehat{W}^\top x)-u^\top g (w^\top x))\|\|(u^\top g (w^\top x)- y)\|)\\
   &\phantom{\leq}+\mathbb{E}(\|(u^\top g (w^\top x)- y)\|^2)\\
   &\leq  \mathcal{R}_{ID} + 2  \sqrt{ \mathcal{R}_{ID}  \mathcal{R}_0}+ \mathcal{R}_0.
\end{aligned}
\end{equation*}

Now, we bound $\mathcal{R}_{ID}$ by 
considering the interpolative decomposition to be a learning algorithm learning the function $u^\top g(W^\top X)$.
Specifically, we use Lemma \ref{lem:prunedRisk} to bound $\mathcal{R}_{ID}$ as  
\begin{equation*}
    \mathcal{R}_{ID} \leq \widehat{\cR}_{ID} + M(1+\|T\|_2)^2n^{-\frac{1}{2}}\left( \sqrt{2p\log(en/p)}+ 2^{-\frac{1}{2}}\sqrt{\log (1/\delta)}\right).
\end{equation*}
where p is the pseudo-dimension. We can then use Lemma \ref{lem:IDEmperical} to bound the empirical risk of the interpolative decomposition as  
\begin{equation*}
    \widehat{\cR}_{ID} \leq  \epsilon^2 \|u\|_2^2 \| g(W^T X)\|_2^2 / n,
\end{equation*}
and it follows that
\begin{equation*}
    \widehat{\cR}_{ID} \leq \epsilon^2 \sup_{x\in\Omega_x} \|u\|_2^2 \| g(W^T x)\|_2^2.
\end{equation*}

\end{proof}


\begin{customlemma}{\ref{lem:prunedRisk}}
Under the assumptions of Theorem~\ref{thm:generalization}, for any $\delta\in(0,1)$, $\cR_{ID}$ satisfies  
\begin{equation*}
    \mathcal{R}_{ID} \leq \widehat{\cR}_{ID} + M(1+\|T\|_2)^2n^{-\frac{1}{2}}\left( \sqrt{2p\log(en/p)}+ 2^{-\frac{1}{2}}\sqrt{\log (1/\delta)}\right)
\end{equation*}
with probability $1-\delta,$ where $M = \sup_{x\in\Omega_x} \|u\| ^2 \| g(W^T x)\|^2$ and $p=\zeta dm \log (dm)$ for some universal constant $\zeta$ that depends only on the activation function.
\end{customlemma}
\begin{proof} 
Considering the interpolative decomposition as a learning algorithm to learn $u^\top g(W^\top X)$, we can use Theorem 11.8 in~\cite{foundationsML} to bound the risk on the data distribution. Given a maximum on the loss function $\eta$, and the ReLU activation function, 


\begin{equation*}
    \mathcal{R}_{ID} \leq \widehat{\mathcal{R}}_{ID} + \frac{\eta}{n^{1/2}}( \sqrt{2p\log en/p}+ 2^{-1/2} \sqrt{\log (1/\delta)})
\end{equation*}
with probability $(1-\delta)$.\footnote{e is the base of the natural log.} Here, the constant $\eta$ is bounded by Lemma~\ref{lem:eta}. Bartlett et al.~\cite{pmlr-v65-harvey17a}  show that the p-dimension for a ReLU network is $O(\mathcal{W}Llog(\mathcal{W}))$ where $\mathcal{W}$ is the number of weights and L is the number of layers.   Here, that translates to $p=\zeta dm \log(dm)$ for some constant $\zeta$ that depends only on the choice of activation function.  
\end{proof}




\begin{customlemma}{\ref{lem:IDEmperical}}

Following the notation of Theorem~\ref{thm:generalization}, an ID pruning to accuracy $\epsilon$ yields a compressed network that satisfies
\begin{equation*}
    \widehat{\cR}_{ID} \leq  \epsilon^2 \|u\|_2^2 \| g(W^T X)\|_2^2 / n,
\end{equation*}
where $X\in\R^{d\times n}$ is a matrix whose columns are the pruning data.
\end{customlemma}

\begin{proof}
\begin{equation}
    \hat{\mathcal{R}}_{ID}=\frac{1}{n}\sum_{i=1}^n |u^\top g(W^\top x_i) - \widehat{u}^\top g(\widehat{W}^\top x_i) |^2
\end{equation}
Here, we can appeal to our definition of the ID to bound each term in the sum.    

\begin{equation}
    |u^\top g(W^\top x_i) - \widehat{u}^\top g(\widehat{W}^\top x_i) | =|u^\top g(W^\top x_i) - {u}^\top  T^\top g(P^\top {W}^\top x_i) |
\end{equation}
By our definition of an $\epsilon$-accurate interpolative decomposition, 

\begin{equation}
    |u^\top g(W^\top x_i) - \widehat{u}^\top g(\widehat{W}^\top x_i) | \leq  \epsilon \|u\| \| g(W^T x_i)\|
\end{equation}

Therefore, 

\begin{equation}
    \hat{\mathcal{R}}_{ID} \leq \frac{1}{n} \epsilon^2 \|u\| ^2 \| g(W^T X)\|^2
\end{equation}
\end{proof}

\begin{lemma}
\label{lem:eta}
The maximum $\eta$ of the loss function associated with approximating the full network with the pruned on is bounded as   
\[
    \eta \leq \sup_{x\in\Omega_x} \|u\|_2^2 \| g(W^T x)\|_2^2 \|(1+ \|T\|))^2
\]

\end{lemma}
\begin{proof} 
\begin{equation*}
    \eta=\max_{x, W, u} \| u^\top g(W^\top x ) - \widehat{ u}^\top g (\widehat{W}^\top x) \|^2
\end{equation*}
For any $x\in\Omega_x$ we have the bound 
\begin{equation*}
\begin{aligned}
    \| u^\top g(W^\top x ) - \widehat{ u}^\top g (\widehat{W}^\top x) \|^2 &\leq (\|u^\top g(W^\top x) \| + \|\widehat{ u}^\top g (\widehat{W}^\top x) \|)^2\\
     &\leq (\|u^\top g(W^\top x) \| + \|{u}^\top  T^\top g(P^\top {W}^\top x)\|)^2\\
     &\leq (\|u^\top g(W^\top x)\|(1+ \|T\|))^2.\\
\end{aligned}
\end{equation*}
Therefore, 
\begin{equation*}
    \eta \leq \sup_{x\in\Omega_x} \|u\|_2^2 \| g(W^T x)\|_2^2 \|(1+ \|T\|))^2
\end{equation*}

%Here, we appeal to the existence of a matrix T which interpolates $g(W^\top x)$ such that each $t_{ij} \leq 2$ \megan{cite}.  This bounds $\|T\| $ \megan{check what the tightest bound on this is is it 2m?  }
\end{proof}

\begin{remarks}
We can explicitly measure the norm of the interpolation matrix $T$ that appears in the upper bound of Lemma~\ref{lem:prunedRisk}.  Moreover, we expect this to be small because there exists an interpolation matrix such that  $t_{ij} \leq 2 \forall \{i,j\}$~\cite{liberty2007randomized}. The better the interpolation matrix, the better the bound.    
\end{remarks}
%\subsection{A note about sparse regression}
%\jerry{Keep?}
%Instead of a rank-revealing QR factorization, one could also use a sparse regression approach.
%Ultimately we did not proceed with this approach because it empirically did not perform as well.





\section{Correlation between random trials}
\label{app:sec:correlation}
We introduce a metric that we call "model correlation" in order to evaluate different pruning methods.  We define the correlation between two models on a data set as the percent of labels that the two models agree on, irrespective of if those labels agree with the ground truth.  There are situations in which the user of a network may care about more than just the simple accuracy --- it may matter which items a network is most likely to get wrong, and how.  It is also possible for two networks to have the same accuracy but perform very differently on subsets of the dataset.  We include this metric after fine tuning to demonstrate the efficacy of our compression method relative to methods that necessitate extensive fine tuning and, therefore, may not correlate well with the original model.  Here we provide some baseline measurements of what affects model correlation, in order to better understand this metric.  

Our baseline measurement of model correlation is the model correlation between two same-sized but randomly initialized networks trained using the same hyper parameters but with different (random) data orders.  We first test the effects for a fully connected one hidden layer network on FashionMNIST. 

\begin{table}[h!]
\begin{center}
 \begin{tabular}{||c c c c c||} 
 \hline
Size & Same Initialization & Same Data Order & Vary Both & Accuracy\\ [0.5ex] 
 \hline\hline
 500 & 94.6 & 94.0 & 93.3 & 89.21 \\ 
 \hline
 2000 & 96.7 & 94.7 & 94.5& 89.63\\
 \hline
 4000 & 97.3 & 95.4 & 95.1 &89.74\\
 \hline
\end{tabular}
\caption{Correlation data for a single hidden layer fully connected network on the FashionMNIST data set. We keep the same initialization but vary the data order (Same Initialization), use the same data order but vary the initialization (Same Data Order) or vary both the data order and initialization (Vary both). This data is averaged over 9 trials.  We see that starting at the same initialization increases the correlation at the end of training, and, interestingly, that using the same data order during training can increase the correlation as well. }
\end{center}
\end{table}
We continue our experiments on the CIFAR10 dataset using the VGG-16 architecture.  We find that two differently randomly initialized models trained using different data orders to  the same state-of-the-art accuracy (93.6\%) agree on classifications 93.0\% of the time.  The effect of data order is also seen on CIFAR10 VGG-16 networks.  When we prune to 50\% FLOPS and then re-train a VGG-16 network using magnitude pruning, if the data order is the same as the original network, then the correlation is 94.01\%.  However, using a different data order the correlation is 93.15\%. The model correlation breaks down quickly when we use a large learning rate (.1 for 200 epochs).  

\section{Comparison methods}
\label{app:sec:comparison_methods}
Here we provide a key for the various methods we compare to in the main text, along with their classifications within our taxonomy. %(Table~\ref{tab:taxonomy}).  
We give both the paper citation and implementation citation.

\begin{table}[h!]
\label{tab:citations}
%\small
\begin{center}
 %\begin{tabular}{||c c c c c||} 
 \begin{tabular}{c c c c c} 
 %\hline 
 \toprule
 \textbf{Dense} & \textbf{Structure Preserving} & \textbf{Corrects Next Layer} & \textbf{No Local FT} & \\ 
 %\textbf{Dense}&\textbf{Structure Preserving}& \textbf{Corrects Next Layer}&\textbf{No Local FT}&\\ 
 \midrule
 %\textbf{Name} & \textbf{Citation} &\textbf{Implementation} & \textbf{Table} & \textbf{Figures}\\ 
 Name & Citation & Implementation & Table & Figures \\ 
 \midrule
 ID & (Ours) &(Ours) & All & All\\
 PFP & \citet{liebenwein2020provable} &\citet{liebenwein2020provable}& - & \ref{fig:vgg16preft}\\ 
 AMC & \citet{he2018amc} & \citet{he2018amc} & \ref{tab:vggImgNet} & -\\
 \midrule
 \\\\
 \toprule
 \textbf{Dense}&\textbf{Structure Preserving}& \textbf{Corrects Next Layer}&\textbf{Local FT}&\\ 
 \midrule
 Name & Citation & Implementation & Table & Figures \\ 
 \midrule
 Thi & \citet{luo2017thinet} & \citet{liebenwein2020provable} & \ref{tab:vggImgNet} & \ref{fig:vgg16preft}\\ 
 CP & \citet{he2017feat} & \citet{he2017feat} &  \ref{tab:vggImgNet} & -\\ 
 NS & \citet{liu2017netslim} &\citet{zhuang2020polar} & \ref{tab:cifarVgg} & -\\ 
 \midrule
 \\\\
 \toprule
 \textbf{Dense}&\textbf{Structure Preserving}&\textbf{No Correction}&&\\
 \midrule
 Name & Citation & Implementation & Table & Figures \\ 
 \midrule
 Uni & Uniform Random Filter Pruning &\citet{liebenwein2020provable}&  - & \ref{fig:vgg16preft}\\ 
 Soft & \citet{softnetHe} &\citet{liebenwein2020provable}&  - & \ref{fig:vgg16preft}\\ 
 StructMag &\citet{li2017l1} & \citet{liebenwein2020provable} & \ref{tab:cifarVgg} & \ref{fig:vgg16preft}\\ 
 FPGM & \citet{he2019fpgm} &(Ours)&  \ref{tab:cifarVgg} & -\\
 {HRank} & ~\citet{lin2020hrank}  &~\citet{lin2020hrank} &  \ref{tab:cifarVgg} & -\\ 
 \midrule
 \\\\
 \toprule
 \textbf{Dense}&\textbf{Extra Layers}&&&\\
 \midrule
 Name & Citation & Implementation & Table & Figures \\ 
 \midrule
 ALDS & \citet{liebenwein2021alds} &\citet{liebenwein2021alds}&  - & \ref{fig:mobilenet_atom3d},\ref{fig:combiningID} \\ 
 Messi & \citet{Maalouf2021DeepLM} &\citet{liebenwein2021alds}&  - & \ref{fig:mobilenet_atom3d},\ref{fig:vgg16preft}\\ 
 PCA & \citet{zhang20153dfilter} &\citet{liebenwein2021alds}&  - &\ref{fig:mobilenet_atom3d}, \ref{fig:combiningID}\\ 
 SVD & \citet{denten2014svd} &\citet{liebenwein2021alds}&  - & \ref{fig:mobilenet_atom3d}, \ref{fig:vgg16preft}\\ 
 LRank & \citet{idel2020lrank} &\citet{liebenwein2021alds} &  \ref{tab:cifarVgg},\ref{tab:vggImgNet}& \ref{fig:combiningID}\\ 
 Polar & \citet{zhuang2020polar}&\citet{zhuang2020polar}&  \ref{tab:cifarVgg} & -\\
 \midrule
 \\\\
 \toprule
 \textbf{Sparse}&&&&\\
 \midrule
 Name & Citation & Implementation & Table & Figures \\ 
 \midrule
 SiPP & \citet{sippBayal}&\citet{sippBayal} &  - & \ref{fig:vgg16preft}\\ 
 Snip & \citet{lee2019snip} &\citet{sippBayal}& - & \ref{fig:vgg16preft}\\
 Thres & \citet{li2017l1} &\citet{liebenwein2020provable}&  - & \ref{fig:vgg16preft}\\ 

\hline
 \end{tabular}

 \end{center}
       \caption{
    Taxonomy of pruning and methods and look-up table for references.  As you go further down the list, methods tend to become more different from our own.  
    }
 \end{table}
 
% \begin{table}[h!]
%%\small
%\begin{center}
% \begin{tabular}{||c c c c c c c||} \hline
% Name & Full Name & Citation &Impl. & Cat.& Tab. & Figs.\\ \hline
% ID & Interpolative Decomposition & (Ours) &(Ours)& E & All & All\\ 
% ALDS & Automatic Layer-wise Decomposition Selector & \citet{liebenwein2021alds} &\citet{liebenwein2021alds}& I& - & \ref{fig:mobilenetCifar},\ref{fig:CombineCifar}, \ref{fig:compose_vggImgNet} \\ 
% Messi & Multiple Estimated SVDs for Smaller Intralayers & \citet{Maalouf2021DeepLM} &\cite{liebenwein2021alds}& I& - & \ref{fig:mobilenetCifar},\ref{fig:vggCifar}\\ 
% PCA & Accel Very Deep ConvNets for Class. \& Detect. & \citet{zhang20153dfilter} &\citet{liebenwein2021alds}& I& - & \ref{fig:mobilenetCifar},\ref{fig:CombineCifar}, \ref{fig:compose_vggImgNet}\\ 
% SVD & Exploiting Linear Struct. w/in ConvNets & \citet{denten2014svd} &\citet{liebenwein2021alds}& F& - & \ref{fig:mobilenetCifar}, \ref{fig:mobilenetImgNet}, \ref{fig:compressonly_vggImgNet}\\ 
% Uni & Full Name & Citation &\citet{liebenwein2020provable}& Cat.& - & \ref{fig:vggCifar}\\ 
% PFP & Provable Filter Pruning & \citet{liebenwein2020provable} &\citet{liebenwein2020provable}& Cat& - & \ref{fig:vggCifar}\\ 
% SiPP & Sensitivity-informed Provable Pruning & \citet{sippBayal}&\cite{sippBayal} & A& - & \ref{fig:vggCifar}\\ 
% Snip & Single-shot Network Pruning & \citet{} &\citet{sippBayal}&G& - & \ref{fig:vggCifar}\\ 
% Thi & ThiNet: A Filter Level Pruning Method & \citet{luo2017thinet} & \citet{liebenwein2020provable} & Cat& \ref{tab:vggImgNet} & \ref{fig:vggCifar}, \ref{fig:compressonly_vggImgNet}\\ 
% Soft & Soft filter pruning & \citet{softnetHe} &\citet{liebenwein2020provable}& Cat& - & \ref{fig:vggCifar}\\ 
% Thres &  Pruning Filters for Efficient ConvNets &\citet{li2017l1} &\citet{liebenwein2020provable}& A& - & \ref{fig:vggCifar}\\ 
% StructMag & Pruning Filters for Efficient ConvNets & \citet{li2017l1} & Impl. & B& \ref{tab:cifarVgg} & \ref{fig:compressonly_vggImgNet}\\ 
%  LRank & Low-Rank Compressino of NNs & \citet{idel2020lrank} &\citet{liebenwein2021alds} & I& \ref{tab:cifarVgg},\ref{tab:vggImgNet}& \ref{fig:CombineCifar}, \ref{fig:compose_vggImgNet}\\ 
%   CP & Channel Pruning for Accel Very Deep NN & \citet{he2017feat} & \citet{he2017feat} & cat& \ref{tab:vggImgNet} & -\\ 
%   AMC & AUtoML for Model Compression & \citet{he2018amc} & \citet{he2018amc} & cat& \ref{tab:vggImgNet} & -\\ 
%   Polar &Structured Pruning using Polarization Regularizer & \citet{zhuang2020polar}&\cite{zhuang2020polar}& cat& \ref{tab:cifarVgg} & -\\ 
%  NS &Network Slimming & \citet{liu2017netslim} &\citet{zhuang2020polar}& cat& \ref{tab:cifarVgg} & -\\ 
%  FPGM &Filter Pruning via Geometric Median & \citet{he2019fpgm} &(Ours)& cat& \ref{tab:cifarVgg} & -\\ 
%
%\hline
%
% \end{tabular}
% \end{center}
%% \label{tab:citations}
% \end{table}
 
\section{Setting $k(\epsilon)$ per layer in deep networks with iterative pruning}
\label{app:sec:iterativeID}
By Definition~\ref{def:ID}, an $\epsilon$-accurate interpolative decomposition is associated with a number $k$ of selected columns.
We observe that for deep networks it is not straightforward to apply a single accuracy $\epsilon$ to the entire network.
Figure \ref{fig:vggMet} illustrates the representative variety in the layer-wise singular value decay for a trained VGG-16 model.
For our method a sharper singular value decay indicates greater prunability. 
However, not all layers contribute equally to the number of FLOPS and the number of parameters.  Typically convolutional layers contribute disproportionately to the number of FLOPs compared to the number of parameters they contain.
When we prune neurons or channels in a layer, that has an effect on the number of FLOPS performed by that layer, and also in the next layer.  Therefore, we iteratively prune the network by finding the layer that will allow us to prune the most FLOPs compared to the error that we expect from pruning that layer, and pruning that layer.  This is given in Algorithm \ref{alg:deepIDIter}.  

\begin{algorithm}[t]{\small
\caption{Pruning a multilayer network with iterative interpolative decompositions}
\begin{algorithmic}
\label{alg:deepIDIter}
%\INPUT
\REQUIRE
Neural net $h(x; W^{(1)},\ldots,W^{(L)})$,
pruning set $X$,
step size $\lambda$,
FLOPs ratio $\rho$
%\OUTPUT
\ENSURE
Pruned network $h(x; \widehat{W}^{(1)},\ldots,\widehat{W}^{(L)})$
\vspace{0.5em}
\FOR{$l \in \{1, \dots, L\}$}
\STATE $\widehat{W}^{(l)} \gets W^{(l)}$
\ENDFOR
\STATE $F \gets \operatorname{Compute\_FLOPs}(h(x; \widehat{W}^{(1)},\ldots,\widehat{W}^{(L)})$
\STATE $F_\rho \gets F * \rho$
\WHILE{$F > F_\rho$}
\STATE $S, K \gets \{\}$
\FOR{$l \in \{1, \dots, L\}$}
\STATE $Z \gets h_{1:l}(X; W^{(1)}, \dots, W^{(l)})$
\COMMENT{layer l activations}
\STATE $R \gets \operatorname{Pivot\_QR}(\operatorname{Reshape}(Z))$
\COMMENT{reshape if Conv layer}
\STATE $k \gets \operatorname{num\_channel}(\widehat{W}^{(l)} \times \lambda$
\COMMENT{calculates proportion of channels or neurons to potentially remove}
\STATE $Err \gets |R[k+1,k+1]/R[1,1]|$
\COMMENT{error from ID approximation}
\STATE $F_l \gets \operatorname{Compute\_FLOPs}(\widehat{W}^{(l)}, \widehat{W}^{(l+1)})$
\COMMENT{compute FLOPs of current and next layer}
\STATE $S.\operatorname{append}(Err / F_l)$
\COMMENT{weighted layer prunability score}
\STATE $K.\operatorname{append}(k)$
\ENDFOR
\STATE $l \gets \operatorname{argmin}(S)$
\STATE $\widehat{W}^{(l)} \gets \operatorname{ID}$ prune layer $l$ to $K[l]$ neurons or channels
\STATE $F \gets \operatorname{Compute\_FLOPs}(h(x; \widehat{W}^{(1)},\ldots,\widehat{W}^{(L)})$
\ENDWHILE
%\STATE $T^{(0)} \gets I$ \;
%\FOR{$l \in \{1 \dots L\}$}
%\STATE $Z \gets h_{1:l}(X; W^{(1)}, \dots, W^{(l)})$
%\COMMENT{layer l activations}
%%\COMMENT{compute activations of layer l}
%\IF{layer $l$ is a FC layer}
%\STATE $(\I, T^{(l)}) \gets \operatorname{ID}(Z^T; \alpha) \textbf{ if } l \notin S \textbf{ else } (:, I)$ \;
%%prune $\alpha\%$ of neurons with ID of $Z^\top$: $\I, T$\;
%%compute rank-$k$ ID of $Z^\top$: $\I, T$\;
%\STATE $\widehat{W}^{(l)} \gets T^{(l-1)} W^{(l)}_{:,\I}$
%\COMMENT{sub-select neurons, multiply T of prev layer's ID}
%%\tcp{select neurons in current layer}
%%$\widehat{W}^{(l+1)} \gets T \widehat{W}^{(l+1)}$
%%\;
%%\tcp{propagate T to next layer}
%\ELSIF{layer l is a Conv layer (or Conv+Pool)}
%\STATE $(\I, T^{(l)}) \gets \operatorname{ID}(\Reshape(Z); \alpha) \textbf{ if } l \notin S \textbf{ else } (:, I)$ \;
%% prune $\alpha\%$ of channels with ID of $\Reshape(Z)$: $\I, T$\;
%%compute rank-$k$ ID of $\Reshape(Z)$: $\I, T$\;
%\STATE $\widehat{W}^{(l)} \gets \Matmul(T^{(l-1)}, W^{(l)}_{\I,\ldots})$
%%\;
%\COMMENT{select channels; multiply T} %in current layer
%% $\widehat{W}^{(l+1)} \gets \Matmul(T, \widehat{W}^{(l+1)})$
%%\tcp{propagate T to next layer}
%%\tcp{depends if next layer is FC or Conv}
%\ELSIF{layer l is a Flatten layer}
%\STATE $T^{(l)} \gets T^{(l-1)} \otimes I \,\,$ 
%\COMMENT{expand T to have the expected size}
%\ENDIF
%\ENDFOR
\end{algorithmic}
%%Specify direction\;
%%How to compute Z\;
%\caption{ID pruning a multi-layer neural network}
}\end{algorithm}






\begin{figure}[!tbp]
\centering

  \includegraphics[width=.25\linewidth]{figures/layer_1.pdf}
  \includegraphics[width=.25\linewidth]{figures/layer_8.pdf}
  \includegraphics[width=.25\linewidth]{figures/layer_13.pdf}

    \caption{Metrics for different layers in VGG-16 for Cifar-10.  The leftmost layer is the first convolutional layer.  The center figure is one of the middle convolutional layers, and the rightmost figure is the last convolutional layer.  As we can see, the singular value decay varies throughout the network. }
\label{fig:vggMet}
\end{figure}
\section{Sensitivity of parameters}
\label{sec:sens}
When we prune using Iterative ID, we have a choice of hyperparameter in how what percent of channels to cut per iteration (pruning fraction $\alpha$ in Algorithm~\ref{alg:deepIDIter}). In Figure~\ref{fig:sensitivity}, we demonstrate that the choice of pruning fraction does not greatly affect the accuracy of the network after iterative pruning.  However, using a smaller $\alpha$ results in the network taking significantly more time to prune.  

\begin{figure}[h!]
    \centering
    \includegraphics[width=\linewidth]{figures/sensitivity.pdf}
    \caption{Iterative pruning performed at different amounts of channels/neurons cut per iteration in the chosen layer.  We see that the choice of number of channels per iteration does not have a large impact on the outcome, though smaller percents take a longer time.}
    \label{fig:sensitivity}
\end{figure}

%One benefit of maintaining properties of the original model may include maintaining the fairness of an already fair model.  In order to accomplish this, we want our method to be robust to the contents of the pruning set, including when classes are underrepresented.  While we do not recommend 


\section{Additional experimental details and results}
%FLOPs reduction = 1 - pruned FLOPs / orinal FLOPs.

\subsection{Illustrative example}


We created a simple synthetic data set for which we know the form of a relatively minimal model representation.

Draw $n$ points iid. from the unit circle in 2 dimensions.
Next select 2 (normalized) random vectors $v_1$ and $v_2$.
All labels are initialized to zero, and add or subtract 1 to the label for each time it produces a positive inner product with one of the random vectors.
With the ReLU activation function it is possible to correctly label all points with a one hidden layer fully connected network of width 4.
Construct pairs of neurons, with each pair aligned with the center of one of the random vectors.  We can think of the pair as creating a flat function by using one neuron to "cut the top" off of the round part of the function created by the other.

We parameterize this with an angle $\phi$ away from the pair. If each neuron in the pair has the same weight magnitude $w$, coefficient $\pm u$ and a bias $b \pm \delta/2$, then given a particular $w$, $b$ and $\delta$, we can write:

\begin{equation*}
       u ({\ReLU( w \cos \phi -b +\delta/2)- \ReLU(w \cos \phi -(b-\delta/2))})
\end{equation*}

As long as $w>b$, this looks like a step function, where the sides get steeper as $w$ approaches infinity.  This allows us to perfectly label all of the points given twice the number of neurons as we have random vectors $v_1$ and $v_2$, which is much smaller than the number of data points $n$.

We train an over-parameterized single hidden layer network to perform fairly well on this task, using an initialization scale that is common in some machine learning platforms such as TensorFlow~\cite{tensorflow2015-whitepaper}. This is shown in figure \ref{fig:patchesBonus}.  
However, magnitude pruning will not necessarily recognize the structure of the minimum representative network because both of the neurons in the pair construction may not have a large magnitude.
On this example we see that the interpolative decomposition is able to select neurons which resemble a close to minimal representative network.  

\begin{figure}[!tbp]
\centering
  \includegraphics[width=.4\linewidth]{figures/patchesAcc.pdf}
  \includegraphics[width=.4\linewidth]{figures/patchesFuncEval.pdf}
\caption{A plot of loss v.s. number of neurons kept (left) and a plot of the function evaluation for the full network, ID pruning, and magnitude pruned model (right).  The data is drawn from the unit circle, and we parameterize X in terms of an angle $\theta$. To see more detail in the magnitude pruned function, we draw it with a scale of 10x applied uniformly.  We see that it takes very few (12) neurons to completely represent the function approximated by the network. In fact, the ID pruned function is visually indistinguishable from the full model in this case.  }
\label{fig:patchesBonus}
\end{figure}


In Figure~\ref{fig:patches} we see that the ID well represents the original network by ignoring duplicate neurons and taking into account differences in the bias.
The ID even achieves slightly better test loss than the original model.
Magnitude pruning does not approximate the original model well; it keeps duplicate neurons and fails to find important information with the same number of neurons.

\begin{figure}[!tbp]
\centering
  \includegraphics[width=.3\linewidth]{figures/fullPatch.pdf}
  \includegraphics[width=.3\linewidth]{figures/IDpatch.pdf}
  \includegraphics[width=.3\linewidth]{figures/magPatch.pdf}
    \caption{Neuron visualization for a simple 2-d regression task. The axes are the weights of the neurons, and the color is the bias.  We have the weights trained for a full model(left) with 5000 hidden nodes.  The 12 neurons kept by the ID (center) represent the function well. These are re-scaled by the magnitude of their coefficients in the second layer to display the effect of the ID.  Magnitude pruning (right) keeps duplicate neurons that do not represent the function.  The test loss is 0.037786 for the full model, 0.037781 for the model pruned with ID, and 0.33 magnitude-pruned model. }
\label{fig:patches}
\end{figure}
  
%We created a synthetic dataset, with points drawn iid. from the unit sphere in 3 dimensions.  We select 2 random vectors and normalize.  All of the labels are initialized to zero, and we add 1 to the labels of each of the points with a positive inner product with each of the vectors.  If we use the ReLU activation function, it is possible to correctly label all of the points using a one hidden layer neural network with 2n neurons in its hidden layer.   One arrangement that accomplishes this is by pairing the the neurons, with each pair aligned with each of the centers of the patches.  

%\subsection{Data set and code licenses}
%Fashion MNIST~\cite{datafashionmnist} is licensed under the MIT License.
%We are unaware of a license for the CIFAR-10 data set~\cite{datacifar10}.
%We adapted code and hyper-parameters from \textcite{liu2019rethink} (MIT License), \textcite{li2017l1} (license unaware), and \textcite{he2019fpgm} (license unaware).

%\subsection{Estimating compute time}
%The experiments on the illustrative example and Fashion MNIST were performed on a 2019 iMac running an Intel I9. The illustrative example computes in minutes. We trained 15 different model configurations, 8 one hidden layer, and 7 two hidden layer network sizes.  For 5 random seeds, we used $5*(50+10+10)=350$ epochs per model size, and trained 15 different model configurations.   We used 10,000 images for the pruning set, however, the runtime for computing the ID is cheap compared to training, using the scipy QR decomposition function which calls a LAPACK subroutine~\cite{lapack}.

%The experiments on CIFAR-10 were performed on a NVIDIA GeForce GTX 1080Ti. 
%For 5 random seeds and two models a total of $5*2*160=1600$ epochs was used for training the %ResNet-56 and
%VGG-16 models.
%The Table~\ref{tab:cifar10mag} our results required $5*3*40=1200$ epochs of fine-tuning for 5 random seeds and 2 pruning configurations on ResNet-56, 1 configuration on VGG-16.
%Computing the interpolative decomposition were comparatively cheap, only requiring 1000 data points. 
%The Table~\ref{tab:cifar10modern} and~\ref{tab:cifar10beforeFT} ID results required $5 * 2 * 200 = 2000$ total epochs of fine-tuning for 5 random seeds and 2 different models.






%\paragraph{Fashion MNIST}
\subsection{Fashion MNIST}
\label{sec:fashionmnist}

For one and two hidden layer networks on Fashion MNIST~\cite{datafashionmnist} we compare the performance of ID and magnitude pruning to networks of the same size trained from scratch.
Each pruning method is used to prune to half the number of neurons of the original model, and is then fine-tuned for 10 epochs.
%For our first experiment, we train simple fully connected one and two hidden layer networks on the Fashion MNIST~\cite{datafashionmnist} data set and prune each network to one half of the neurons in each layer. 
%This consists of 60000 grey-scale training images and 10000 test images with 10 classes.  
We use a stochastic gradient descent optimizer with a learning rate of 0.3 which decays by a factor of 0.9 for each epoch to train the initial networks.  Each was trained for 50 epochs.  Our pruning set was 10000 images, which did not need to be held out from training for the simple data set. The fine tuning ran for 10 epochs with an initial learning rate of 0.1 with a decay rate of 0.6 for two layer networks, and 0.2 with a learning rate decay of 0.7 for one layer.  Larger ID-pruned models (greater than 512 neurons) can be fine-tuned with a much smaller learning rate of 0.002. These learning rates and number of epochs may not be optimal but were determined through brief empirical tests. We used 5 random seeds for each size of model.  The error bars are reported as the uncertainty in the mean, defined in terms of the standard deviation $\sigma$, and number of independent trials $N$,  $\sigma_{mean}=\sigma/\sqrt{N}$.
%We compare the performance of the networks trained from scratch, and pruned using magnitude pruning and ID pruning.  
Results are shown in Figure~\ref{fig:fmnist}; before fine-tuning the ID achieves a significantly higher test accuracy than magnitude pruning.
ID pruning with fine-tuning outperforms training from scratch. %, and all methods begin to converge at large network sizes (except just magnitude pruning).
%We see that the pruning and then fine tuning technique out performs training from scratch for moderate network sizes, and that all of the methods except magnitude pruning without fine tuning begin to converge at large network sizes,
At sufficiently large network sizes, ID pruning alone performs similarly to training a network from scratch. 
When we start to see diminishing returns from adding more neurons, the performance of the various methods begin to converge.  
%In addition by about $2^9$ hidden neurons we've surpassed the minimal network size and see diminishing returns from more neurons.  


\begin{figure}
\centering
  \centering
  \includegraphics[width=.47\linewidth]{figures/OHL.pdf}
  \hspace{2mm}
  \includegraphics[width=.47\linewidth]{figures/THL.pdf}
  
    \caption{Accuracy for a one hidden layer (left) and two hidden layer (right) fully connected neural networks of varying size on Fashion-MNIST for ID and magnitude pruning, as well as training directly.
    These curves are averaged over 5 trials; error bars report uncertainty in the mean. 
    }
\label{fig:fmnist}
\end{figure}

\begin{figure}[ht]
\centering
  \includegraphics[width=.45\linewidth]{figures/metrics.pdf}
  %\caption{Normalized singular value decay, $||R_{22}||/||R||$, and the magnitude of$r_{kk}/r_{00}$ for a matrix Z from a one hidden layer neural network trained on Fashion MNIST.  We see that the three metrics generally correlate well in a practical setting.}
  \includegraphics[width=.45\linewidth]{figures/accuracyvsrkk.pdf}
  %\caption{Accuracy of a single hidden layer fully connected neural network pruned from 4096 to k neurons.  The horizontal lines are the accuracy of the full sized network  }

    \caption{Left: Normalized singular value decay, $\|R_{22}\|_2/\|R\|_2$, and $\lvert r_{k+1,k+1}/r_{1,1}\rvert$ for a matrix from~\eqref{eq:1hiddenfc}) from a one hidden layer network (i.e., $g(W^\top X)$) trained on Fashion MNIST. We see that the metrics generally correlate well in this setting. Right: Accuracy of a single hidden layer network pruned from width 4096 to $k.$  The horizontal lines are the accuracy of the full sized network. The difference  between pruning and test accuracy is due to a slight class imbalance in the canonical test set.}%Note that the slight difference between the pruning and test set exists for the full network and is due to a slight imbalance in classes for the canonical testing set.}
\label{fig:chosingk}
\end{figure}


%\subsection{Ablation studies for design of Algorithm~\ref{alg:deepID}}
%We found that using a held-out pruning set was important to prevent the interpolative decomposition from over-fitting to the training data, and improve generalization performance to the test set.
%Table~\ref{tab:prunesetAblation} reports our results.
%
%\begin{table}[]
%    \centering
%    \caption{Ablation study on the effect of using a held-out (or not) pruning set for the interpolative decomposition.
%    A ResNet-56 model on CIFAR-10 was pruned to 51\% FLOPs reduction with a pruning set of 1000.
%    The pruning set was either held-out from the test set, or randomly sampled from the training set.
%    Accuracies are reported as mean and standard deviation over 5 independent trials.
%    \\}
%    \label{tab:prunesetAblation}
%    \begin{tabular}{ccc}
%    \toprule
%    Interpolative Decomposition &
%    Baseline  &
%    Pruning \\
%    Pruning Set & 
%    Acc. (\%) &
%    Acc. (\%)\\
%    \midrule
%    Held Out &
%    \multirow{2}{*}{93.04 ($\pm$ 0.25)} & 
%    69.64 ($\pm$ 2.14) \\
%    Not Held Out & &
%    66.34 ($\pm$ 1.30)  \\
%    \bottomrule
%    \end{tabular}
%\end{table}
%
%
%In Algorithm~\ref{alg:deepID} we apply the ID from the beginning to the end of the multi-layer network.
%Doing so one could use the interpolative decomposition to approximate the activation outputs of either the original model or the model.
%Table~\ref{tab:Zablation} gives an ablation study which compares the pruning accuracy for both such cases.
%We observe that using the ID to approximate the original model achieves higher pruning accuracy (before fine-tuning).
%We believe that in deep networks there is greater concern for the ID to propagate errors forward through the matrix.
%Thus by approximating the original model's activation outputs, we can mitigate some of this error propagation.
%
%
%\begin{table}[]
%    \centering
%    \caption{Ablation study on the effect of approximating the activation outputs of the original or pruned model using the interpolative decomposition.
%    A ResNet-56 model on CIFAR-10 was pruned to 51\% FLOPs reduction with a held-out pruning set of size 1000.
%    Accuracies are reported as mean and standard deviation over 5 independent trials.
%    \\}
%    \label{tab:Zablation}
%    \begin{tabular}{ccc}
%    \toprule
%    Interpolative Decomposition &
%    Baseline  &
%    Pruning \\
%    Approximation Target & 
%    Acc. (\%) &
%    Acc. (\%)\\
%    \midrule
%    Original Model &
%    \multirow{2}{*}{93.04 ($\pm$ 0.25)} & 
%    69.64 ($\pm$ 2.14) \\
%    Pruned Model & &
%    63.19 ($\pm$ 4.66)  \\
%    \bottomrule
%    \end{tabular}
%\end{table}


%\subsection{Learning rate and prunability}
%\outline{LR and SVD}

\subsection{Additional ImageNet experiments on MobileNet V1}
\label{app:sec:imgnet_extra}
Here we give additional pruning experiments on Mobilenet V1 for ImageNet.
We see that the ID is competitive against compression methods which do not do any local fine tuning (Figure~\ref{fig:mobilenetImgNet_nolocFT}.
Figure~\ref{fig:mobilenetImgNet_yeslocFT} includes methods which do use local fine tuning; ID performs better than two out of the three.
Interestingly, we see that although the PCA and LRank methods performed well on an overparameterized network such as VGG-16, they do not work well on a much more efficient network.

\begin{figure}
\centering
\begin{subfigure}{.45\textwidth}
\centering
\includegraphics[width=.95\linewidth]{figures/mob_nolocFT_ImgNet.pdf}
\caption{Compare with methods that do not use any fine tuning.
% We see that the ID is superior in this setting.
}
\label{fig:mobilenetImgNet_nolocFT}
\end{subfigure}
\begin{subfigure}{0.45\textwidth}
\centering
\includegraphics[width=.95\linewidth]{figures/mob_yeslocFT_ImgNet.pdf}
\caption{Comparing with methods that use a local fine tuning correction.  
%We see that the ID is superior in this setting.
}
\label{fig:mobilenetImgNet_yeslocFT}
\end{subfigure}
\centering
\caption{MobileNet V1 compression results on ImageNet. Note that the matrix methods often work by changing the depth of convolution in the network, and given that MobileNet uses depth-wise separable convolutions, it is not surprising that matrix methods would perform poorly.  }
\end{figure}


\subsection{Imagenet pre fine tuning correlation}
\label{app:sec:imgnet_preft_corr}

\begin{figure}
    \centering
    \includegraphics[width=0.47\linewidth]{figures/mobCorr_ImgNet.pdf}
    \caption{MobileNet correlation results compared against various compression methods. Note that several other compression methods work by doing decompositions on the weight matrix and changing the number of groups in the convolution.  Due to the MobileNet architecture, this may result in a very low accuracy for methods that work well on other architectures.}
    \label{fig:movCorr_ImgNet}
\end{figure}

\begin{figure}
\centering
\begin{subfigure}{.47\textwidth}
\centering
\includegraphics[width=.95\linewidth]{figures/vggCorr_nolocFT_ImgNet.pdf}
\caption{
Comparing with methods that do not use a local fine tuning correction.
}
\label{fig:vggCorrImgNet_nolocFT}
\end{subfigure}
\begin{subfigure}{0.47\textwidth}
\centering
\includegraphics[width=.95\linewidth]{figures/vggCorr_yeslocFT_ImgNet.pdf}
\caption{
Comparing with methods that use a fine tuning correction.
}
\label{fig:vggCorrImgNet_yeslocFT}
\end{subfigure}
\centering
\caption{VGG-16 correlation results on ImageNet}
\end{figure}

Here we report correlation results om ImageNet with the MobileNet V1 and VGG-16 models.
For methods which do not incorporate a fine tuning correction (Figure~\ref{fig:vggCorrImgNet_nolocFT}), we see that the ID proves as good as any other method we compare to.
Figure~\ref{fig:vggCorrImgNet_yeslocFT} compares to methods which do incorporate a local fine tuning correction. 
Similar to accuracy, we see that composing ID with another compression method improves upon either method.





\subsection{Hyper-parameter details}
\paragraph{CIFAR-10}
The VGG-16 models are trained with 5 random seeds and with hyper-parameter specifications and code provided by \citet{liu2019rethink}.
The test set is randomly partitioned into a prune set and new test set.
The Iterative ID used a held out set of size 1000.
%The held-out pruning set is randomly sampled from the test set.
%This shrinks the test set slightly, from 10000 to 9000.
%For magnitude pruning we use the hyper-parameters specified by \textcite{liu2019rethink} to fine-tune for 40 epochs and learning rate 0.001.
%The interpolative decomposition uses a held out pruning set of 1000 data points and the same fine-tuning hyper-parameters, except the ResNet-56 is retrained with initial learning rate 0.01 and decreased to 0.001 after 10 epochs.
%The VGG-16 uses the same hyper-parameters as~\cite{liu2019rethink}.
%For Table~\ref{tab:cifarVgg}, we implement the method by \textcite{he2019fpgm} with their provided code and hyper-parameter settings on our trained models to fine-tune for 200 epochs.
%The interpolative decomposition uses a held out pruning set of 1000 data points with a hyper-parameter configuration from~\textcite{he2019fpgm}: use a starting learning rate 0.1 decaying to 0.02, 0.004, 0.0008 at epochs 60, 120, 160 to train for 200 epochs total with batch size 128 and weight decay 5e-4.
For the iterative ID on VGG-16, we prune 10\% of a layer per iteration.  %This results in pruning the 
For VGG-16 fine tuning we use SGD with initial learning rate of 5e-3, reduced to 2.5e-3 after 20 epochs and again reduced to 1e-3 after a another 20 epochs have passed.
We use a batch size of 128, momentum of 0.9, and weight decay of 5e-4.

%\paragraph{Mobilenet V1}
The full-size Mobilenet V1 network for Cifar-10 was trained using the ADAM optimizer for 120 epochs, using the default parameters of lr=.001, betas= (.9, .999).  The test set is randomly partitioned into a prune set and new test set.  

\paragraph{ImageNet}
For VGG-16 the iterative ID algorithm uses a randomly held out set of 5000 images with a stepsize parameter of 5\%.
For fine-tuning, we use SGD with a learning rate of 1e-7, batch size of 256, momentum of 0.9, and weight decay of 1e-4.
For ID+PCA, we switched to learning rate 1e-8 at 75 epochs.

For Mobilenet V1 we prune to a constant fraction using the ID with a randomly sampled held out set of 1000 images.


\subsection{Estimating compute}
The experiments on the illustrative example and Fashion MNIST were performed on a 2019 iMac running an Intel I9. The illustrative example computes in minutes. We trained 15 different model configurations, 8 one hidden layer, and 7 two hidden layer network sizes.  For 5 random seeds, we used $5*(50+10+10)=350$ epochs per model size, and trained 15 different model configurations.   We used 10,000 images for the pruning set, however, the runtime for computing the ID is cheap compared to training, using the scipy QR decomposition function which calls a LAPACK subroutine~\cite{lapack}.

The experiments on CIFAR-10 were performed on a NVIDIA GeForce GTX 1080Ti on a university compute cluster. 
Epochs of fine tuning are specified in the tables of the main paper.
Computing the interpolative decomposition were comparatively cheap, only requiring 1000 data points. 
Computing costs for any compression method were negligible compared to fine tuning.

ImageNet experiments were conducted on a university compute cluster.
For fine tuning NVIDIA GeForce RTX 3090, RTX A6000, or TITAN RTX were used.
The compression portion was conducted on a NIVIDIA GeForce GTX 1080Ti or FTX 2080Ti.
Epochs of fine tuning are specified in the tables of the main paper.
The compute cost of the compression methods before fine tuning (including interpolative decomposition) were trivial compared to any amount of global fine tuning.



\section{Sensitivity to pruning set}
\label{sec:sensitivity}
\subsection{Pruning set size}
We compare the pre-fine-tuning accuracy as a function of the pruning set size. 
Here we show that the accuracy is not particularly sensitive to the pruning set size.
Table~\ref{tab:id_sensitivity_imagenet} shows on ImageNet that our ID-based pruning method is robust to the prune set size, and in fact quite efficient.
Figure~\ref{fig:id_sensitivity_cifar10} shows on CIFAR-10 that this trend persists across a wide range of compression levels.
Note that the number of pruning examples must be at least the number of neurons or channels that we prune to for each layer.
%Pruning using the interpolative decomposition only takes a modes pruning set size.  At a minimum, it requires at least as many pruning examples as the number of neurons or channels that we wish to prune to. Here we show that the outcome of pruning is not particularly sensitive to the pruning set size. 

\begin{table}[]
    \centering
    \begin{tabular}{cc}
        Prune set size & Pre-fine-tuning accuracy \\
        \hline
        5k & 68.03 \\
        10k & 68.06 
    \end{tabular}
    \caption{Pre-fine-tuning accuracy compared to prune set size for VGG16 model pruned to 25\% FLOPs reduction on ImageNet.}
    \label{tab:id_sensitivity_imagenet}
\end{table}

\begin{figure}
\centering
%\begin{subfigure}{.87\textwidth}
\centering
\includegraphics[width=.5\linewidth]{figures/setsize.pdf}
\caption{
Comparing the pre-fine-tuning accuracy and FLOP reduction for different pruning set sizes on VGG-16 Cifar-10 model.  We see that above the minimum threshold, pruning set size has a minimial impact on the accuracy of the pruned model. 
}
\label{fig:id_sensitivity_cifar10}
%\end{subfigure}
\end{figure}

\subsection{Pruning set contents}

We want our method to be robust to the data selection method used to generate our pruning set. 
We test our method to this specific sensitivity by removing an entire class from the pruning set.
%This may include situations where an entire class is missing from the pruning set. 

We begin with a full-sized VGG-16 CIFAR-10 model that was trained
%to perform reasonably well
on all 10 classes.  Then we draw a pruning set from only 9 of the classes, completely leaving out images from one of the classes. We prune to 50\% of the original FLOPs, using Iterative ID, and compare the per-class test accuracies for each of the 10 classes. Figure~\ref{fig:id_sensitivity_class} shows that the ID maintains good accuracy even on images from the class that was excluded from the pruning set.  

We expect that other methods which preserve the model's decision boundaries (and therefore correlation) will likely show similar results.  However, methods which require extensive fine tuning and effectively re-train the network will likely not recover accuracy on the missing class.  To demonstrate this, we prune the same full-sized network using magnitude pruning using the same pruning set to 50\% FLOPs reduction, and then fine tune with only 9 classes in the fine tuning set.  As shown in Figure~\ref{fig:mag_sensitivity_class}, the accuracy for the other 9 classes recovers, however, the accuracy for the class that was removed does not.  

This experiment
%is a contrived example, but it 
suggests that pruning methods which maintain properties of the original model may potentially be able to maintain fairness (i.e. per-class accuracy) even when some classes of data are under-represented.  
Our compression method was able to reasonably preserve per-class accuracies (our measure of fairness), even without access to data from one of those classes.

%We began with a ``fair model'' which performed well on all 10 classes of CIFAR-10, and prune it to 50\% of the original FLOPs while maintaining accuracy on all classes, even without access to data from one of those classes.  

\begin{figure}
\centering
\begin{subfigure}{.47\textwidth}
\includegraphics[width=.95\linewidth]{figures/subclass.pdf}
\caption{Per class accuracies while pruning a VGG-16 model using only data from 9 classes.}
\label{fig:id_sensitivity_class}
\end{subfigure}
\begin{subfigure}{.47\textwidth}
\includegraphics[width=.95\linewidth]{figures/IDcorr.pdf}
\caption{Model Correlation when ID pruning using data from only 9 classes}
\label{fig:id_sensitivity_corr}
\end{subfigure}
\caption{
Per-class accuracies as we prune a VGG-16 model on CIFAR-10 with only access to data from 9/10 classes.  No fine tuning was done.  We see that ID maintains reasonable accuracy on all (including the unrepresented) classes, and stays correlated with the original model.  
}

\end{figure}

\begin{figure}
\centering
\begin{subfigure}{.47\textwidth}
\includegraphics[width=.95\linewidth]{figures/magFT.pdf}
\caption{Per class accuracies while fine tuning a magnitude-pruned model with only 9 classes}
\label{fig:mag_sensitivity_class}
\end{subfigure}
\begin{subfigure}{.47\textwidth}
\includegraphics[width=.95\linewidth]{figures/magFTcorr.pdf}
\caption{Model Correlation while fine-tuning a magnitude pruned model using data from only 9 classes.}
\label{fig:mag_sensitivity_corr}
\end{subfigure}
\caption{
Comparing class accuracies for a 50\% FLOPS VGG-16 model pruned using magnitude pruning, and then fine-tuned using only 9 out of 10 classes of CIFAR-10.  We see that the model recovers on the represented classes, but not on the unrepresented class. 
}

\end{figure}


\end{document}