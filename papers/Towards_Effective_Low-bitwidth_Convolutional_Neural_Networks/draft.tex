

\documentclass[10pt,twocolumn,letterpaper]{article}

\usepackage{cvpr}
\usepackage{epsfig}
\usepackage{graphicx}
\usepackage{subfigure}
\usepackage{amsmath}
\usepackage{amssymb}
\usepackage{amsthm}
\usepackage{amsfonts}
\usepackage{mathrsfs}
\usepackage{booktabs}
\usepackage{multirow, eucal}

\usepackage{helvet}
\usepackage{courier}
\usepackage{bm}
%
%
%
\usepackage{graphicx}
\usepackage{color}
\usepackage{epstopdf}
\usepackage{wrapfig}
\usepackage{picinpar}
\usepackage{url}

\usepackage[vlined,ruled,linesnumbered]{algorithm2e}
\usepackage{cite}
\newtheorem{theorem}{Theorem}


\usepackage{caption}
\captionsetup{margin=5pt,font=small,labelfont=bf}

\cvprfinalcopy





\def\kuired{\textcolor{red}}
%
%
%
%
%

%
%
%
%
%
%
%

%

%
%
%

\def\cD{ {\cal D } }
\def\cW{ {\cal W } }


\def\diag{\mbox{diag}}
\def\rank{\mbox{rank}}
\def\grad{\mbox{\text{grad}}}
\def\dist{\mbox{dist}}
\def\sgn{\mbox{sgn}}
\def\tr{\mbox{tr}}
\def\etal{{\em et al.\/}\, }
\def\card{{\mbox{Card}}}

%
\def\balpha{\mbox{{\boldmath $\alpha$}}}
\def\bbeta{\mbox{{\boldmath $\beta$}}}
\def\bzeta{\mbox{{\boldmath $\zeta$}}}
\def\bgamma{\mbox{{\boldmath $\gamma$}}}
\def\bdelta{\mbox{{\boldmath $\delta$}}}
\def\bmu{\mbox{{\boldmath $\mu$}}}
\def\beps{\mbox{{\boldmath $\epsilon$}}}
\def\blambda{\mbox{{\boldmath $\lambda$}}}
\def\bnu{\mbox{{\boldmath $\nu$}}}
\def\bomega{\mbox{{\boldmath $\omega$}}}
\def\bfeta{\mbox{{\boldmath $\eta$}}}
\def\bsigma{\mbox{{\boldmath $\sigma$}}}
\def\bzeta{\mbox{{\boldmath $\zeta$}}}
\def\bphi{\mbox{{\boldmath $\phi$}}}
\def\bxi{\mbox{{\boldmath $\xi$}}}
\def\bvphi{\mbox{{\boldmath $\phi$}}}
\def\bdelta{\mbox{{\boldmath $\delta$}}}
\def\bvarsigma{\mbox{{\boldmath $\varsigma$}}}
\def\bXi{\mbox{{\boldmath $\Xi$}}}


\def\bPi{\mbox{{\boldmath $\Pi$}}}

\def\bDelta{\mbox{{\boldmath $\Delta$}}}
\def\bPi{\mbox{{\boldmath $\Pi$}}}
\def\bPsi{\mbox{{\boldmath $\Psi$}}}
\def\bSigma{\mbox{{\boldmath $\Sigma$}}}

%
\def\mA{{\mathcal A}}
\def\mB{{\mathcal B}}
\def\mC{{\mathcal C}}
\def\mD{{\mathcal D}}
\def\mE{{\mathcal E}}
\def\mF{{\mathcal F}}
\def\mG{{\mathcal G}}
\def\mH{{\mathcal H}}
\def\mI{{\mathcal I}}
\def\mJ{{\mathcal J}}
\def\mK{{\mathcal K}}
\def\mL{{\mathcal L}}
\def\mM{{\mathcal M}}
\def\mN{{\mathcal N}}
\def\mO{{\mathcal O}}
\def\mP{{\mathcal P}}
\def\mQ{{\mathcal Q}}
\def\mR{{\mathcal R}}
\def\mS{{\mathcal S}}
\def\mT{{\mathcal T}}
\def\mU{{\mathcal U}}
\def\mV{{\mathcal V}}
\def\mW{{\mathcal W}}
\def\mX{{\mathcal X}}
\def\mY{{\mathcal Y}}
\def\mZ{{\mathcal{Z}}}




%
\def\0{{\bf 0}}
\def\1{{\bf 1}}

%
\def\bA{{\bf A}}
\def\bB{{\bf B}}
\def\bC{{\bf C}}
\def\bD{{\bf D}}
\def\bE{{\bf E}}
\def\bF{{\bf F}}
\def\bG{{\bf G}}
\def\bH{{\bf H}}
\def\bI{{\bf I}}
\def\bJ{{\bf J}}
\def\bK{{\bf K}}
\def\bL{{\bf L}}
\def\bM{{\bf M}}
\def\bN{{\bf N}}
\def\bO{{\bf O}}
\def\bP{{\bf P}}
\def\bQ{{\bf Q}}
\def\bR{{\bf R}}
\def\bS{{\bf S}}
\def\bT{{\bf T}}
\def\bU{{\bf U}}
\def\bV{{\bf V}}
\def\bW{{\bf W}}
\def\bX{{\bf X}}
\def\bY{{\bf Y}}
\def\bZ{{\bf{Z}}}


%
\def\ba{{\bf a}}
\def\bb{{\bf b}}
\def\bc{{\bf c}}
\def\bd{{\bf d}}
\def\be{{\bf e}}
\def\bff{{\bf f}}
\def\bg{{\bf g}}
\def\bh{{\bf h}}
\def\bi{{\bf i}}
\def\bj{{\bf j}}
\def\bk{{\bf k}}
\def\bl{{\bf l}}
\def\bm{{\bf m}}
\def\bn{{\bf n}}
\def\bo{{\bf o}}
\def\bp{{\bf p}}
\def\bq{{\bf q}}
\def\br{{\bf r}}
\def\bs{{\bf s}}
\def\bt{{\bf t}}
\def\bu{{\bf u}}
\def\bv{{\bf v}}
\def\bw{{\bf w}}
\def\bx{{\bf x}}
\def\by{{\bf y}}
\def\bz{{\bf z}}

%
\def\hy{\hat{y}}
\def\hby{\hat{{\bf y}}}


%
%
%
%
%
%
%
%
%
%
%
%
%
%
%
%
%
%
%
%
%

%
\def\pd{{\succ\0}}
\def\psd{{\succeq\0}}
\def\vphi{\varphi}
\def\trsp{{\sf T}}


\def\mRMD{{\mathrm{D}}}
\def\mRMD{{\mathrm{D}}}
%

\def\kui{\textcolor{black}}



\def\citep{\cite}
\def\citet{\cite}
\newtheorem{coll}{Corollary}
\newtheorem{deftn}{Definition}
\newtheorem{thm}{Theorem}
\newtheorem{prop}{Proposition}
\newtheorem{lemma}{Lemma}
\newtheorem{remark}{Remark}
\newtheorem{ass}{Assumption}


%
\def\diag{\mbox{diag}}
\def\rank{\mbox{rank}}
\def\grad{\mbox{\text{grad}}}
\def\dist{\mbox{dist}}
\def\sgn{\mbox{sgn}}
\def\tr{\mbox{tr}}
\def\etal{{\em et al.\/}\, }
\def\card{{\mbox{Card}}}
\def\st{\mbox{s.t. }}

\def\kui{\textcolor{black}}
\def\young{\textcolor{black}}
\def\mark{\textcolor{red}}

%

%
\def\httilde{\mbox{\tt\raisebox{-.5ex}{\symbol{126}}}}

%


\begin{document}

%
\title{Towards Effective Low-bitwidth Convolutional Neural Networks\thanks{B. Zhuang, C. Shen, L. Liu and I. Reid are with The University of Adelaide,
  Australia. M. Tan is with South China University of Technology, China.}
  \thanks{Correspondence to C. Shen (e-mail: chhshen@gmail.com).}
%
}

\author{
Bohan Zhuang, Chunhua Shen,
Mingkui Tan, Lingqiao Liu,  Ian Reid
}




\maketitle


\begin{abstract}
	This paper tackles the problem of training a deep convolutional neural network with both low-precision weights and low-bitwidth activations. Optimizing a low-precision network is very challenging since the training process can easily get trapped in a poor local minima, which results in substantial accuracy loss. To mitigate this problem, we propose three simple-yet-effective approaches to improve the network training.
	First, we propose to use a two-stage optimization strategy to progressively find good local minima. Specifically, we propose to first optimize a net with quantized weights and then quantized activations. This is in contrast to the traditional methods which optimize them simultaneously.
	Second, following a similar spirit of the first method, we propose another progressive optimization approach which progressively decreases the bit-width from high-precision to low-precision during the course of training.
	Third, we adopt a novel learning scheme to jointly train a full-precision model alongside the low-precision one. By doing so, the full-precision model provides hints to guide the low-precision model training.
	Extensive experiments on various datasets (\ie, CIFAR-100 and ImageNet) show the effectiveness of the proposed methods. To highlight, using our methods to train a 4-bit precision network leads to no performance decrease in comparison with its full-precision counterpart with standard network architectures (\ie, AlexNet and ResNet-50).


\end{abstract}


\tableofcontents
\clearpage





	\section{Introduction}  \label{sec:introduction}

\newcommand\inexpIntro[3]{#1?(#2,#3).}
\newcommand\rinexpIntro[3]{*#1?(#2,#3).}
\newcommand\outexpIntro[3]{#1!(#2,#3).}
\newcommand\outatomIntro[3]{#1!(#2,#3)}

We propose a fully automated method for proving termination of \(\pi\)-calculus processes.
Although there have been a lot of studies on termination analysis for the \(\pi\)-calculus
and related calculi~\cite{Deng06IC,Demangeon07,SangiorgiTermination,KobayashiHybrid,Yoshida04IC,DBLP:journals/jlp/DemangeonHS10,Venet98SAS}, most of them have been rather theoretical,
and there have been surprisingly little efforts in developing  fully automated termination
verification methods and tools based on them. To our knowledge,
Kobayashi's \typical{}~\cite{TyPiCal,KobayashiHybrid} is the only exception that
can prove termination of \(\pi\)-calculus processes (extended with natural numbers)
fully automatically, but its termination analysis is quite limited (see Section~\ref{sec:relatedwork}).

Our method is based on a reduction to termination analysis for sequential programs:
we translate a \(\pi\)-calculus process \(P\) to a sequential program \(S_P\), so that
if \(S_P\) is terminating, so is \(P\). The reduction allows us to use
powerful, mature methods and tools
for termination analysis of sequential programs~\cite{heizmann2016ultimate,freqterm,DBLP:conf/lics/PodelskiR04,Kuwahara2014Termination,DBLP:journals/cacm/CookPR11}.

The idea of the translation is to convert a chain of communications on replicated input
channels to a chain of recursive function calls of the target sequential program.
Let us consider the following Fibonacci process:
\begin{align*}
    & \rinexpIntro{\fib}{n}{r}
        \ifexp{n<2}{ \soutatom{r}{1} \\ &\quad}
                   { \nuexp{s_1} \nuexp{s_2} (\outatomIntro{\fib}{n-1}{s_1} \PAR \outatomIntro{\fib}{n-2}{s_2} \PAR \sinexp{s_1}{x}\sinexp{s_2}{y}\soutatom{r}{x+y}) \\}
    & \PAR \outatomIntro{\fib}{m}{r}
\end{align*}
Here, the process
$\rinexpIntro{\fib}{n}{r} \ldots$ is a function server that computes the \(n\)-th Fibonacci number
in parallel and returns the result to \(r\),
and $\outatom{\fib}{m}{r}$ sends a request for computing the \(m\)-th Fibonacci number;
those who are not familiar with the syntax of the \(\pi\)-calculus may wish to consult
Section~\ref{sec:targetlanguage} first.
To prove that the process above is terminating for any integer \(m\),
it suffices to show that there is no infinite chain of communications on $\fib$:
\[
    \fib(m,r) \to \fib(m_1,r_1) \to \fib(m_2,r_2) \to \cdots.
\]
We convert the process above to the following program:\footnote{The actual translation
  given later is a little more complex.}
\begin{verbatim}
 let rec fib(n) = if n<2 then () else (fib(n-1) [] fib(n-2)) in
 fib(m)
\end{verbatim}
Here, \texttt{[]} represents the non-deterministic choice.
Note that, although the calculation of Fibonacci numbers is not preserved,
for each chain of communications on \texttt{fib}, there is a corresponding
sequence of recursive calls:
\[
\mathtt{fib}(m) \to \mathtt{fib}(m_1) \to \mathtt{fib}(m_2) \to \cdots.
\]
Thus, the termination of the sequential program above implies the termination of
the original process.
As shown in the example above, (i) each communication on a replicated input channel
is converted to a function call, (ii) each communication on a non-replicated input
channel is just removed (or, in the actual translation, replaced by a call of
a trivial function defined by \(f(\seq{x})=(\,)\)), and (iii) parallel composition
is replaced by a non-deterministic choice.
We formalize the translation outlined above and prove its correctness.

The basic translation sketched above sometimes loses too much information.
For example, consider the following process:
\begin{align*}
    & \rinexpIntro{\pre}{n}{r} \soutatom{r}{n-1} \\
    & \PAR \rinexpIntro{f}{n}{r} \ifexp{n<0}{ \soutatom{r}{1} }
                                       { \nuexp{s} (\outatomIntro{\pre}{n}{s} \PAR \sinexp{s}{x}\outatomIntro{f}{x}{r}) } \\
    & \PAR \outatomIntro{f}{m}{r}
\end{align*}
The translation sketched above would yield:
\begin{verbatim}
  let pred(n) = n-1 in
  let rec f(n) = if n<0 then () else (pred(n) [] f(*)) in
  f(m)
\end{verbatim}
Here, \texttt{*} represents a non-deterministic integer: since we have removed
the input $\sinatom{s}{x}$, we do not have information about the value of \( x \).
As a result, the sequential program above is non-terminating, although the original
process is terminating.
To remedy this problem, we also refine the basic translation above by using a refinement
type system for the \(\pi\)-calculus. Using the refinement type system,
we can infer that the value of \(x\) in the original process is less than \(n\),
so that we can refine the definition of \texttt{f} to:
\begin{verbatim}
 let rec f(n) = ... else (pred(n) [] let x=* in assume(x<n);f(x))
\end{verbatim}
The target program is now terminating, from which
we can deduce that the original process is also terminating.
We have implemented an automated tool based on the refined translation above.

The contributions of this paper are summarized as follows.
\begin{itemize}
\item The formalization of the basic translation from the \(\pi\)-calculus
  (extended with integers) to sequential programs, and a proof of its correctness.
\item The formalization of a refined translation based on a refinement type system.
\item An implementation of the refined translation, including automated refinement type
  inference based on CHC solving, and experiments to evaluate the effectiveness of
  our method.
\end{itemize}

The rest of this paper is structured as follows.
Section~\ref{sec:targetlanguage} introduces the source and target languages
of our translation.
Section~\ref{sec:approach} 
formalizes the basic translation, and proves its correctness.
Section~\ref{sec:refinement} refines the basic translation by using a refinement type system.
Section~\ref{sec:implementation} reports an implementation and experiments.
Section~\ref{sec:relatedwork} discusses related work,
and Section~\ref{sec:conclusion} concludes the paper.

	The industry standard for pose edition is to create rigs, a collection of pieces of software designed to manipulate a character's skeleton. The rig describes the skeleton's bones, how they relate to each other, are constrained in their possible motion and are deformed. These rules are loosely specified and creating a good rig requires a detailed understanding of physics and anatomy, as well as technical and artistic skills. Rigging is thus a time consuming task even for experienced animators, and even more so in large scale productions which often require a different in-depth rig for each character in the cast.
Previous work has helped alleviate this difficulty by providing efficient tools to speed up/and or ease the rigging process, relying on inverse kinematics or data-driven methods.
\subsection{Character pose design}
\subsubsection{Inverse Kinematics (IK)}
IK solvers are a family of methods commonly used in robotics, engineering and computer graphics, in which the parameterization of a kinematic chain is determined from the position of its end effector.
They are a staple tool in pose design software, ensuring the respect of elementary constraints during pose edition. Their de-facto role is to guarantee the length of the limbs, and in some cases to enforce the orientation angle range of a joint.
Many IK solutions have been studied over the years \cite{aristidou_inverse_2018}; usually revolving around approximated linearizations or heuristics. 

Numerical methods require a set of iterations to achieve a satisfactory solution formulated by a cost function to be minimized.
IK solutions can generally be divided into three sub-categories: Jacobian \cite{Siciliano_Handbook_Robot_2007}, Newtonians \cite{cohen_ik_1996} and Heuristics. Most software implement heuristic methods such as Cyclic Coordinate Descent (CCD) \cite{wang_ccd_1991} or 
Forward-Backward Reaching IK (FABRIK) \cite{aristidou_fabrik:_2011} due to their simplicity and extensibility. 

The main drawback of 
these solvers is that they manipulate kinematic chains without taking into account many morphological aspects that make a pose more or less plausible. They offer a first level of help to users but are not sufficient to guarantee a realistic pose. Many joints constraints are dependent on each other and require subjective, human-made approximations.

\subsubsection{Data-driven pose edition}
Data-driven methods offer promising opportunities to solve these approximations. Using real-life data can help in modelling the complex inter-dependencies of skeletons and providing users with smarter edition tools.
While it is still an early field of research, some solutions have been studied. Wu \etal \cite{wu_posing_2009} propose a method for natural character posing from a large motion database. It employs adaptive KD-clustering to select a representative frame from a database and sparse approximations to accelerate training and posing. 
Huang \etal in \cite{Huang_IK_MGDM_2017} present a method based on the formulation of multi-variate Gaussian distribution models (MGDMs), which learn the joint constraints of a kinematic skeleton from motion capture data. 

Some work has also been dedicated to finding new editing interfaces. \modify{}{Instead of the usual setup manipulating joints directly, Guay \etal \cite{guay_line_2013} articulate a framework based on the conceptual "line of action" which describes the overall pose dynamics. They provide a mathematical definition of the line of action, and a interface in which the software modifies the pose to follow a user-provided line. In the same line of though} Garcia \etal \cite{garcia_sketching_2019} propose \modify{a method transforming doodle of trajectories (position and orientation over time) }{a virtual reality-based interface where the user's hands motion (position and orientation over time) are transformed} into sequences of actions and then into detailed character animations using a dataset of parametrized motion clips automatically fitted to the trajectory. 

% ==> DL et Latent Space. 
\subsection{Neural modelling of human motion}
Neural networks have received a great amount of attention over the last decade and shown impressive result in modelling complex data. Human motion has not been spared and deep learning methods have proven their capability of generating realistic motion in a number of difficult cases. 

The literature in neural-based animation include example in user-controlled character navigation \cite{Holden2017} and interactions with the environment \cite{starke_neural_2019}. 
Holden \etal \cite{Holden2020} also show that neural networks can be used to replace parts of existing data-driven methods, improving their scalability potential.
More recently, some work has also focused on improving smaller parts of the animation pipeline rather than replacing it completely. Berson et al. \cite{berson_intuitive_2020} leverage neural networks to provide an interactive system to edit facial animation. 

% Wrap up
Data-driven IK and pose editing can relieve animators from time-consuming, back-and-forth pose adjustments by applying constraints extracted from real-world data. Recently, neural-network-based approaches have demonstrated their ability to model the intricacies of human motion while scaling to large amount of data and retaining a fast inference time. In this paper we seek to take advantage of these properties to create an efficient posing tool, intuitively usable even by a inexperienced user.
\section{Methods}
\begin{figure*}[!t]
	\centering
	\resizebox{0.9\linewidth}{!}
	{
		\begin{tabular}{c}
			\includegraphics{pdf/1.pdf}
		\end{tabular}
	}
	\caption{Demonstration of the guided training strategy. We use the residual network structure for illustration.}
	\label{fig:knowledge_transfer}
\end{figure*}
In this section, we will first revisit the quantization function in the neural network and the way to train it. Then we will elaborate our three methods in the subsequent sections.
\subsection{Quantization function revisited} \label{sec:baseline}
%

A common practise in training a neural network with low-precision weights and activations is to introduce a quantization function. Considering the general case of $k$-bit quantization as in~\cite{zhou2016dorefa}, we define the quantization function $Q(\cdot)$ to be
\begin{equation}
	{z_q} = Q({z_r}) = \frac{1}{{{2^k} - 1}}round(({2^k} - 1){z_r})
\end{equation}
where ${z_r} \in [0,1]$ denotes the full-precision value and ${z_q} \in [0,1]$ denotes the quantized value. With this quantization function, we can define the weight quantization process and the activation quantization process as follows:


\noindent \textbf{Quantization on weights}:
\begin{equation}\label{eq:quan-weigtht}
	{w_q} = Q(\frac{{\tanh (w)}}{{2\max (\left| {\tanh (w)} \right|)}} + \frac{1}{2}).
\end{equation}In other words, we first use $\frac{{\tanh (w)}}{{2\max (\left| {\tanh (w)} \right|)}} + \frac{1}{2}$ to obtain a normalized version of $w$ and then perform the quantization, where $\tanh(\cdot)$ is adopted to reduce the impact of large values.

\noindent \textbf{Quantization on activations}:

 Same as \cite{zhou2016dorefa}, we first use a clip function $f(x) = clip(x,\,0,1)$ to bound the activations to $[0, 1]$. After that, we conduct quantize the activation by applying the quantization function $Q(\cdot)$ on $f(x)$.
\begin{equation} \label{eq:quan-activations}
	{x_q} = Q(f(x)).
\end{equation}

\noindent \textbf{Back-propagation with quantization function}: In general, the quantization function is non-differentiable and thus it is impossible to directly apply the back-propagation to train the network. To overcome this issue, we adopt the straight-through estimator \cite{zhou2016dorefa, hubara2016binarized, bengio2013estimating} to approximate the gradients calculation. Formally, we approximate the partial gradient $\frac{{\partial {z_q}}}{{\partial {z_r}}}$ with an identity mapping, namely $\frac{{\partial {z_q}}}{{\partial {z_r}}} \approx 1$.  Accordingly, $\frac{{\partial l}}{{\partial {z_r}}}$ can be approximated by
\begin{equation}
	\frac{{\partial l}}{{\partial {z_r}}} = \frac{{\partial l}}{{\partial {z_q}}}\frac{{\partial {z_q}}}{{\partial {z_r}}} \approx \frac{{\partial l}}{{\partial {z_q}}}.
\end{equation}


%
%
%
%
%
%
%
\subsection{Two-stage optimization}\label{sec:two-stage}
With the straight-through estimator, it is possible to directly optimize the low-precision network. However, the gradient approximation of the quantization function inevitably introduces noisy signal for updating network parameters. Strictly speaking, the approximated gradient may not be the right updating direction. Thus, the training process will be more likely to get trapped at a poor local minima than training a full precision model. Applying the quantization function to both weights and activations further worsens the situation.

To reduce the difficulty of training, we devise a two-stage optimization procedure: at the first stage, we only quanitze the weights of the network while setting the activations to be full precision. After the converge (or after certain number of iterations) of this model, we further apply the quantization function on the activations as well and retrain the network. Essentially, the first stage of this method is a related subproblem of the target one. Compared to the target problem, it is easier to optimize since it only introduces quantization function on weights. Thus, we are more likely to arrive at a good solution for this sub-problem. Then, using it to initialize the target problem may help the network avoid poor local minima which will be encountered if we train the network from scratch.
Let $M_{low}^{K}$ be the high-precision model with $K$-bit. We propose to learn a low-precision model $M_{low}^{k}$ in a two-stage manner with $M_{low}^{K}$ serving as the initial point, where $k<K$.
 The detailed algorithm is shown in Algorithm \ref{algo:two-stage}.
\begin{algorithm}[]
	\KwIn{Training data $\{ ({{\bf{x}}_i},y_i)\}_{i=1}^N$; A $K$-bit precision model $M_{low}^K$.}
	\KwOut{A low-precision deep model $M^k_{low}$ with weights ${{\bf{W}}_{low}}$ and activations being quantized into $k$-bit.}

	\textbf{Stage 1}: Quantize ${{\bf{W}}_{low}}$:\\
	\For{ $\mathrm{epoch} = 1,...,L$}
	{
		\For{ $t = 1,...T$}
		{
			Randomly sample a mini-batch data;\\
			Quantize the weights ${{\bf{W}}_{low}}$ into $k$-bit by calling some quantization methods with $K$-bit activations\;
		}
	}
	\textbf{Stage 2}: Quantize activations:\\
	Initialize ${{\bf{W}}_{low}}$ using the converged $k$-bit weights from \textbf{Stage 1} as the starting point; \\
	\For{ $\mathrm{epoch} = 1,...,L$}
	{
		\For{ $t = 1,...T$}
		{
			Randomly sample a mini-batch data;\\
			Quantize the activations into $k$-bit  by calling some quantization methods while keeping the weights to $k$-bit;
		}
	}
	\caption{Two-stage optimization for $k$-bit quantization}
	\label{algo:two-stage}
\end{algorithm}


\subsection{Progressive quantization} \label{sec:progressive}

%

The aforementioned two-stage optimization approach suggests the benefits of using a related easy optimized problem to find a good initialization. However, separating the quantization of weights and activations is not the only solution to implement the above idea. In this paper, we also propose another solution which progressively lower the bitwidth of the quantization during the course of network training.
Specifically, we progressively conduct the quantization from higher precisions to lower precisions (\eg, 32-bit $\to$ 16-bit $\to$ 4-bit $\to$ 2-bit). The model of higher precision will be used the the starting point of the relatively lower precision, in analogy with annealing.


Let $\{{b_1},...,{b_n}\}$ be a  sequence precisions, where  $b_n<b_{n-1}, ..., b_2<{b_1}$, $b_n$ is the target precision and $b_1$ is set to 32 by default. The whole progressive optimization procedure  is summarized in as Algorithm~\ref{algo:progressive optimization}.
%
 Let $M_{low}^{k}$ be the low-precision model with $k$-bit and $M_{full}$ be the full precision model. In each step, we propose to learn $M_{low}^{k}$, with the solution in the $(i-1)$-th step, denoted by $M_{low}^{K}$, serving as the initial point, where $k<K$.
%

%
%
%



%

%


%

%

%

%
\begin{algorithm}[]
	\KwIn{Training data $\{ ({{\bf{x}}_j},y_j)\}_{j=1}^N$; A pre-trained 32-bit full-precision  model ${M_{full}}$ as baseline; the precision sequence $\{{b_1},...,{b_n}\}$ where $b_n<b_{n-1}, ..., b_2<{b_1} = 32$.}
	\KwOut{A low-precision deep model $M_{low}^{b_n}$.}
	Let $M_{low}^{b_1}=M_{full}$, where $b_1 = 32$\;
	\For{ $i = 2,...n$}
	{
		Let $k = b_i$ and $K=b_{i-1}$\;
		Obtain $M_{low}^{k}$ by calling some quantization methods with $M_{low}^{K}$  being the input\;
		%
		%
		%
	}
	\caption{Progressive quantization for accurate CNNs with low-precision weights and activations}
	\label{algo:progressive optimization}
\end{algorithm}


%


%


%

\subsection{Guided training with a full-precision network}\label{sec:mutual}
The third method proposed in this paper is inspired by the success of using information distillation ~\cite{romero2014fitnets, hinton2015distilling, parisotto2016actor, zagoruyko2016paying, ba2014deep} to train a relatively shallow network. Specifically, these methods usually use a teacher model (usually a pretrained deeper network) to provide guided signal for the shallower network. Following this spirit, we propose to train the low-precision network alongside another guidance network. Unlike the work in \cite{romero2014fitnets, hinton2015distilling, parisotto2016actor, zagoruyko2016paying, ba2014deep}, the guidance network shares the same architecture as the target network but is pretrained with full-precision weights and activations.

However, a pre-trained model may not be necessarily optimal or may not be suitable for quantization. As a result, directly using a fixed pretrained model to guide the target network may not produce the best guidance signals. To mitigate this problem, we do not fix the parameters of a pretrained full precision network as in the previous work \cite{zhang2017deep}.


By using the guidance training strategy, we assume that there exist some full-precision models with good generalization performance, and an accurate low-precision model can be obtained by directly performing the quantization on those full-precision models. In this sense, the feature maps of the learned low-precision model should be close to that obtained by directly doing quantization on the full-precision model. To achieve this, essentially, in our learning scheme, we can jointly train the full-precision and low-precision models. This allows these two models adapt to each other. We even find by doing so the performance of the full-precision model can be slightly improved in some cases.

%

Formally, let ${{\bf{W}}_{full}}$ and ${{\bf{W}}_{low}}$ be the full-precision model and low-precision model, respectively. Let $\mu ({\bf{x}};{{\bf{W}}_{{full}}})$ and $\nu ({\bf{x}};{{\bf{W}}_{{low}}})$ be the nested feature maps (e.g., activations) of the full-precision model and low-precision model, respectively. To create the guidance signal, we may require that the nested feature maps from the two models should be similar. However,  $\mu ({\bf{x}};{{\bf{W}}_{{full}}})$ and $\nu ({\bf{x}};{{\bf{W}}_{{low}}})$  is usually not directly comparable since one is full precision and the other is low-precision.

%

%
%




%




%



%


%

%
%

%
%
%



%

To link these two models,  we can directly quantize the weights and activations of the full-precision model by equations (\ref{eq:quan-weigtht}) and (\ref{eq:quan-activations}). For simplicity, we denote the quantized feature maps by  $Q(\mu ({\bf{x}};{{\bf{W}}_{{full}}}))$. Thus, $Q(\mu ({\bf{x}};{{\bf{W}}_{{full}}}))$ and  $\nu ({\bf{x}};{{\bf{W}}_{{low}}})$ will become comparable. Then we can define the guidance loss as:
\begin{equation}
	R({{\bf{W}}_{full}},{{\bf{W}}_{low}}) = \frac{1}{2}\parallel Q(\mu ({\bf{x}};{{\bf{W}}_{{full}}})) - \nu ({\bf{x}};{{\bf{W}}_{{low}}}){\parallel^2},
\end{equation}
where $\parallel\cdot\parallel$ denotes some proper norms.
%


Let ${L_{{\theta _1}}}$ and ${L_{{\theta _2}}}$ be the cross-entropy classification losses for the full-precision and low-precision model, respectively. The guidance loss will be added to ${L_{{\theta _1}}}$ and ${L_{{\theta _2}}}$, respectively, resulting in two new objectives for the two networks, namely
\begin{equation} \label{eq:objective1}
	L_1({{\bf{W}}_{full}})  = {L_{{\theta _1}}} + \lambda R({{\bf{W}}_{full}},{{\bf{W}}_{low}}).
\end{equation}
and
\begin{equation} \label{eq:objective2}
	L_2({{\bf{W}}_{low}})  = {L_{{\theta _2}}} +  \lambda R({{\bf{W}}_{full}},{{\bf{W}}_{low}}).
\end{equation}
where $\lambda$ is a balancing parameter. Here, the guidance loss $R$ can be considered as some regularization on ${L_{{\theta _1}}}$ and ${L_{{\theta _2}}}$.


%
In the learning procedure, both ${{\bf{W}}_{full}}$ and ${{\bf{W}}_{low}}$ will be updated by minimizing $L_1({{\bf{W}}_{full}})$ and $L_2({{\bf{W}}_{low}})$ separately, using a mini-batch stochastic gradient descent method. The detailed algorithm is shown in Algorithm \ref{algo:one-mutual learning}. A high-bit precision model $M_{low}^K$ is used as an initialization of $M_{low}^k$, where $K>k$. Specifically, for the full-precision model, we have $K=32$. Relying on $M_{full}$, the weights and activations of $M_{low}^k$ can be initialized by equations (\ref{eq:quan-weigtht}) and (\ref{eq:quan-activations}), respectively.


Note that the training process of the two networks are different.
When updating ${{\bf{W}}_{low}}$ by minimizing $L_2({{\bf{W}}_{low}})$, we use full-precision model as the initialization and apply the forward-backward propagation rule in Section \ref{sec:baseline}  to fine-tune the model. When updating ${{\bf{W}}_{full}}$ by minimizing $L_1({{\bf{W}}_{full}})$, we use conventional forward-backward propagation to fine-tune the model.


\begin{algorithm}[]
	\KwIn{Training data $\{ ({{\bf{x}}_i},y_i)\}_{i=1}^N$; A pre-trained 32-bit full-precision model $M_{full}$; A $k$-bit precision model $M_{low}^k$.}
	\KwOut{A low-precision deep model $M^k_{low}$ with weights and activations being quantized into $k$ bits.}
	Initialize $M_{low}^k$ based on $M_{full}$;\\
	\For{ $\mathrm{epoch} = 1,...,L$}
	{
		\For{ $t = 1,...T$}
		{
			Randomly sample a mini-batch data;\\
			Quantize the weights ${{\bf{W}}_{low}}$  and activations into $k$-bit by minimizing $L_2({{\bf{W}}_{low}})$\;
			Update $M_{full}$ by minimizing $L_1({{\bf{W}}_{full}})$\;
		}

	}
	\caption{Guided training with a full-precision network for $k$-bit quantization}
	\label{algo:one-mutual learning}
\end{algorithm}


%
%
%
%
%
%
%
%
%
%

\subsection{Remark on the proposed methods}
The proposed three approaches tackle the difficulty in training a low-precision model with different strategies. They can be applied independently. However, it is also possible to combine them together. For example, we can apply the progressive quantization to any of the steps in the two-stage approach; we can also apply the guided training to any sub-step in the progressive training. Detailed analysis on possible combinations will be experimentally evaluated in the experiment section.


\subsection{Implementation details} \label{sec:implementation}

In all the three methods, we quantize the weights and activations of all layers except that the input data are kept to 8-bit. Furthermore, to promote convergence, we propose to add a scalar layer after the last fully-connected layer before feeding the low-bit activations into the softmax function for classification. The scalar layer has only one trainable small scalar parameter and is initialized to 0.01 in our approach.


During training, we randomly crop 224x224 patches from an image or its horizontal flip, with the per-pixel mean subtracted. We don't use any further data augmentation in our implementation. We adopt batch normalization (BN)~\cite{ioffe2015batch} after each convolution before activation. For pretraining the full-precision baseline model, we use Nesterov SGD and batch size is set to 256. The learning rate starts from 0.01 and is divided by 10 every 30 epochs. We use a weight decay 0.0001 and a momentum 0.9. For weights and activations quantization, the initial learning rate is set to 0.001 and is divided by 10 every 10 epochs. We use a simple single-crop testing for standard evaluation. Following~\cite{zagoruyko2016paying}, for ResNet-50, we add only two guidance losses in the 2 last groups of residual blocks. And for AlexNet, we add two guidance losses in the last two fully-connected layers.



%
	 \newcommand{\twomoons}{{\tt Twomoons}}
\newcommand{\gauss}{{\tt Gauss}}
\newcommand{\sculpture}{{\tt Sculpture}}
\newcommand{\baseline}{{\tt Baseline}}
\newcommand{\MM}{{\tt MsgPassing}}
\newcommand{\blackboard}{{\tt Blackboard}}
\newcommand{\ncut}{\text{ncut}}
\newcommand{\chensays}[2][]{\textcolor{blue} {\textsc{Jiecao #1:} \emph{#2}}}

\section{Experiments}
In this section we present experimental results for  graph clustering in the message passing and blackboard models. We will compare the following three algorithms. (1) \baseline: each site sends all the data to the coordinator directly; (2) \MM: our algorithm in the message passing model (Section~\ref{sec:gcmessage}); (3) 
\blackboard: our algorithm in  the blackboard model (Section~\ref{sec:bb}).


%Since both of our algorithms are crucially based on the use of spectral scarification, our main focus in the experiments is to investigate to what extend the quality of the spectral clustering algorithms will be affected by using spectral sparsification, the saving of communication costs by using spectral sparsificaion, ...
%
%
%The goal of this experiment is not to demonstrate the effectiveness of the spectral clustering algorithm. We mainly want to investigate the following, 
%\begin{itemize}
%\item to what extend the quality of clustered results will be affected by using spectral sparsification.
%\item saving of communication costs by using spectral sparsifier.
%\item the affect of constants in algorithms of the message passing/blackboard model.
%\end{itemize}
%
%
%\subsection{The Setup}
%\paragraph{Reference Algorithms}
%We compare different algorithms in our experiment.

%Note that we can also run \MM~ in the blackboard model.

Besides giving the visualized results of these algorithms on various datasets, we also measure the qualities of the results via the {\em normalized cut}, defined as 
\[
\ncut(A_1, \ldots, A_{k}) = \frac{1}{2}\sum_{i\in[k]}\frac{w(A_i, V\backslash A_i)}{\vol(A_i)},
\]
 which is a standard objective function to be minimized for spectral clustering algorithms. 
%We will compare the communication costs of these algorithms in different settings.

%We also compare the total communication costs of different algorithms/models. As the unit does not matter in our case, we normalize all communication costs by the cost of \baseline.  Whenever possible, we will visualize the clustered results.

We implemented the algorithms using multiple languages, including Matlab, Python and C++. Our experiments were conducted on an IBM NeXtScale nx360 M4 server, which is equipped with 2 Intel Xeon E5-2652 v2 8-core processors, 32GB RAM and 250GB local storage.


\subsection{Datasets.}
We test the algorithms in the following real and synthetic datasets, which is visualized in \figref{visualization}.


\begin{figure}[h]
     \centering
     \subfigure[\twomoons]{\includegraphics[width=0.23\textwidth]{twomoons-14000-original.png}\label{fig:twomoons}}
     ~~
     \subfigure[\gauss]{\includegraphics[width=0.23\textwidth]{gauss-10000-original.png}\label{fig:gauss}}
     ~~
     \subfigure[\sculpture]{\includegraphics[width=0.13\textwidth,height=0.16\textwidth]{sculpture-11680-original.jpg}\label{fig:sculpture}}
     \caption{Visualization of the datasets for our experiments.}
     \label{fig:visualization}
\end{figure}



\vspace{-1mm}
\begin{itemize}
\item \twomoons : this dataset contains $n=14,000$ coordinates in $\mathbb{R}^2$. We consider each point to be a vertex. For any two vertices $u, v$, we add an edge with weight $w(u,v) = \exp\{-\|u-v\|_2^2/\sigma^2\}$ with $\sigma = 0.1$ when one vertex is among the $7000$-nearest points of the other.  This construction results in a graph with about $110,000,000$ edges.

\item  \gauss : this dataset contains $n = 10,000$ points in $\mathbb{R}^2$. There are $4$ clusters in this dataset, each generated using a Gaussian distribution. We construct a complete graph as the similarity graph.  For any two vertices $u, v$, we define the weight $w(u,v) = \exp\{-\|u-v\|_2^2/\sigma^2\}$ with $\sigma = 1$. The resulting graph has about $100,000,000$ edges.

\item \sculpture : a photo of \textit{The Greek Slave}~\footnote{Available in e.g., \url{http://artgallery.yale.edu/collections/objects/14794}}. We use an $80\times 150$ version of this photo where each pixel is viewed as a vertex. To construct a similarity graph, we map each pixel to a point in $\mathbb{R}^5$, i.e., $(x, y, r, g, b)$, where the latter three coordinates are the RGB values. For any two vertices $u, v$, we  put an edge between $u, v$ with weight $w(u,v) = \exp\{-\|u-v\|_2^2/\sigma^2\}$ with $\sigma = 0.5$ if one of $u, v$ is among the $5000$-nearest points of the other. This results in a graph with about $70,000,000$ edges.
\end{itemize}
\vspace{-1mm}
In the distributed model edges are randomly partitioned across $s$ sites. 

%\vspace{-1.5mm}



\subsection{Results on clustering quality}
%{\em Quality.} \
\begin{figure*}[ht]
     \centering
     \subfigure[\baseline]{\includegraphics[width=0.2\textwidth]{twomoons-14000-original-clustered.png}\label{fig:twomoons-clustered-original}}
     \subfigure[\MM]{\includegraphics[width=0.2\textwidth]{twomoons-14000-sparsify-clustered-15.png}\label{fig:twomoons-clustered-sparsify}}
     \subfigure[\blackboard]{\includegraphics[width=0.2\textwidth]{twomoons-14000-chain-clustered.png}\label{fig:twomoons-clustered-chain}}
     \caption*{\twomoons, $k = 2$;}

\subfigure[\baseline]{\includegraphics[width=0.2\textwidth]{gauss-10000-original-clustered.png}\label{fig:gauss-clustered-original}}
     \subfigure[\MM]{\includegraphics[width=0.2\textwidth]{gauss-10000-sparsify-clustered-15.png}\label{fig:gauss-clustered-sparsify}}
     \subfigure[\blackboard]{\includegraphics[width=0.2\textwidth]{gauss-10000-chain-clustered.png}\label{fig:gauss-clustered-chain}}
     \caption*{\gauss, $k = 4$}


     \subfigure[\baseline]{\includegraphics[width=0.2\textwidth,height=0.2\textwidth]{sculpture-11680-original-clustered.png}\label{fig:sculpture-clustered-original}}  
     \subfigure[\MM]{\includegraphics[width=0.2\textwidth,height=0.2\textwidth]{sculpture-11680-sparsify-clustered-15.png}\label{fig:sculpture-clustered-sparsify}}
     \subfigure[\blackboard]{\includegraphics[width=0.2\textwidth,height=0.2\textwidth]{sculpture-11680-chain-clustered.png}\label{fig:sculpture-clustered-chain}}
     \caption*{\sculpture, $k = 3$. }


     
     \caption{Visualization of the results on \twomoons, \gauss\ and \sculpture. In the message passing model each site samples $5 n$ edges; in the blackboard model all sites jointly sample $10n$ edges (in \twomoons~ and \gauss) or $20n$ edges (in \sculpture) and the chain has length $18$. $s = 15$.}
     \label{fig:quality-1}
\end{figure*}

We visualize the clustered results for 
the \twomoons, \gauss\ and \sculpture\ in Figure~\ref{fig:quality-1}.
% and visualize the clustered results for \gauss\ and \sculpture in Figure~\ref{fig:quality-2}.
It can be seen that \baseline, \MM\ and \blackboard\ give results of very similar qualities.  For simplicity, here we only present the visualization for $s=15$. Similar results were observed when we varied the values of $s$.  
%\he{To Qin: Do you plan to have two titles (Results \& Quality)?}


% \begin{figure*}[h]
%      \centering
% \subfigure[\baseline]{\includegraphics[width=0.3\textwidth]{gauss-10000-original-clustered.png}\label{fig:gauss-clustered-original}}
%      \subfigure[\MM]{\includegraphics[width=0.3\textwidth]{gauss-10000-sparsify-clustered-15.png}\label{fig:gauss-clustered-sparsify}}
%      \subfigure[\blackboard]{\includegraphics[width=0.3\textwidth]{gauss-10000-chain-clustered.png}\label{fig:gauss-clustered-chain}}
%      \caption*{\gauss, $k = 4$}


%      \subfigure[\baseline]{\includegraphics[width=0.2\textwidth]{sculpture-11680-original-clustered.png}\label{fig:sculpture-clustered-original}}  
%      \subfigure[\MM]{\includegraphics[width=0.2\textwidth]{sculpture-11680-sparsify-clustered-15.png}\label{fig:sculpture-clustered-sparsify}}
%      \subfigure[\blackboard]{\includegraphics[width=0.2\textwidth]{sculpture-11680-chain-clustered.png}\label{fig:sculpture-clustered-chain}}
%      \caption*{\sculpture, $k = 3$. }

%      \caption{Visualization of results on \gauss\ and \sculpture; in the message passing model each site samples $5 n$ edges; in the blackboard model all sites jointly sample $10n$ (in \gauss) or $20n$ (in \sculpture) edges and the chain has length $18$.}
%      \label{fig:quality-2}
% \end{figure*}


We also compare the normalized cut (ncut) values of the clustering results of different algorithms.  The results are presented in Figure \ref{fig:quality}. In all datasets, the ncut values of different algorithms are very close. The ncut value of \MM\ slightly decreases when we increase the value of $s$, while the ncut value of \blackboard\ is independent of $s$.
%We comment that in general, it is difficult to compare \MM\ and \blackboard\ directly because they are affected by different parameters.


\begin{figure*}[!ht]
  \centering
  \subfigure[\twomoons]{\includegraphics[width=0.33\textwidth]{twomoons-14000-ncut.png}\label{fig:twomoons-quality}}\hspace*{-1.1em}
  \subfigure[\gauss]{\includegraphics[width=0.31\textwidth]{gauss-10000-ncut.png}\label{fig:gauss-quality}}\hspace*{-1.1em}
  \subfigure[\sculpture]{\includegraphics[width=0.31\textwidth]{sculpture-11680-ncut.png}\label{fig:sculpture-quality}}\hspace*{-1.1em}
  \subfigure{\includegraphics[width=0.14\textwidth]{legend.png}}
     \caption{Comparisons on normalized cuts. In the message passing model, each site samples $5n$ edges; in each round of the algorithm in the blackboard model, all sites jointly sample $10n$ edges (in \twomoons~and \gauss) or $20n$ edges (in \sculpture) edges and the chain has length $18$.}
     \label{fig:quality}
\end{figure*}

%\textcolor{red}{To Jiecao: Can you put the color lines indicating baseline, message passing, and blackboard within one row in Pic 2? Withthis we can save some space.}

%\vspace{-1.5mm}

\subsection{Results on communication costs} 
\begin{figure*}[!ht]
     \centering
     \subfigure[\twomoons]{\includegraphics[width=0.3\textwidth]{twomoons-14000-communication.png}\label{fig:twomoons-communication}}
     \subfigure[\gauss]{\includegraphics[width=0.3\textwidth]{gauss-10000-communication.png}\label{fig:gauss-communication}}
     \subfigure[\sculpture]{\includegraphics[width=0.3\textwidth]{sculpture-11680-communication.png}\label{fig:sculpture-communication}}


     \subfigure[\twomoons]{\includegraphics[width=0.32\textwidth]{twomoons-14000-communication-2.png}\label{fig:twomoons-communication-2}}
     \subfigure[\gauss]{\includegraphics[width=0.32\textwidth]{gauss-10000-communication-2.png}\label{fig:gauss-communication-2}}
     \subfigure[\sculpture]{\includegraphics[width=0.32\textwidth]{sculpture-11680-communication-2.png}\label{fig:sculpture-communication-2}}
     \caption{Comparisons on communication costs. In the message passing model, each site samples $5n$ edges; in each round of the algorithm in the blackboard model, all sites jointly sample $10n$ (in \twomoons~and \gauss) or $20n$ (in \sculpture) edges and the chain has length $18$. }
     \label{fig:communication}
\end{figure*}

We compare the communication costs of different algorithms in Figure \ref{fig:communication}. We observe that while achieving similar clustering qualities as \baseline, both \MM\ and \blackboard\ are significantly more communication-efficient (by one or two orders of magnitudes in our experiments). We also notice that the value of $s$ does not affect the communication cost of \blackboard, while the communication cost of \MM\ grows almost linearly with $s$; when $s$ is large, \MM\ uses significantly more communication than \blackboard. These confirm our theory.  %In Figure~\ref{fig:mm-const} and Figure~\ref{fig:blackboard-const}   in Appendix~\ref{sec:parameters} we present how the performance of \MM\ and \blackboard\ are affected by their parameters.

%
%
%\vspace{-1.5mm}
%\paragraph{Summary.}  From our experimental results we conclude that \MM\ and \blackboard\ achieve similar clustering quality as the native algorithm \baseline, while significantly reduce the communication cost.  When the number of sites is large, \blackboard\ is more communication efficient than \MM, as predicted by our theory.



\subsection{Parameters in \MM\ and \blackboard}
\label{sec:parameters}

Figure \ref{fig:mm-const} shows in \MM how the value of ncut is affected by the number of sites and the number of edges sampled in each site. 
Here, each site samples $cn$ edges. 
When $c=3$ and $s=1$, the ncut value diverges in all datasets. This is because with such a small $c$, the algorithm does not generate a valid sparsifier. In general, increasing $c$ or $s$ will slightly decrease the ncut value. But once they are above some thresholds, the ncut values of \MM\ and \baseline\ become very close.

Figure \ref{fig:blackboard-const} shows in \blackboard  how the ncut value is affected by the number of iterations and the number of edges sampled. When the number of iterations is set to be $5$, ncut values diverge in all datasets. This is because we cannot expect to generate a valid sparsifier by using such few iterations. It can be seen from \ref{fig:bb-gauss-constant} that for a fixed $c$, performing more iterations will help to reduce ncut values. From the same figure, one can also conclude that for fixed iterations, increasing $c$ also helps to reduce the ncut values.



\begin{figure*}[h!t]
     \centering
     \subfigure[\twomoons]{\includegraphics[width=0.3\textwidth]{twomoons-c.png}\label{fig:mm-twomoons-constant}}
     \subfigure[\gauss~dataset]{\includegraphics[width=0.3\textwidth]{gauss-c.png}\label{fig:mm-gauss-constant}}
     \subfigure[\sculpture]{\includegraphics[width=0.3\textwidth]{sculpture-c.png}\label{fig:mm-sculpture-constant}}
     \caption{The pictures above show the $\ncut$ values with respect to the values of $c$ and $s$ for the \MM\ algorithm. Here  
 each site samples $c n$ edges.}
     \label{fig:mm-const}
\end{figure*}


\begin{figure*}[h!t]
     \centering
     \subfigure[\twomoons]{\includegraphics[width=0.3\textwidth]{twomoons-iter.png}\label{fig:bb-twomoons-constant}}
     \subfigure[\gauss]{\includegraphics[width=0.3\textwidth]{gauss-iter.png}\label{fig:bb-gauss-constant}}
     \subfigure[\sculpture]{\includegraphics[width=0.3\textwidth]{sculpture-iter.png}\label{fig:bb-sculpture-constant}}
     \caption{The pictures above show how the $\ncut$ values are affected by the number of iterations and the value of $c$ for the \blackboard\ algorithm. Here 
all sites jointly sample $c n$ edges. }
     \label{fig:blackboard-const}
\end{figure*}







	 
\begin{comment}
\begin{figure}
\includegraphics[width=\linewidth]{figs/beyond_tss_lesion.pdf}
\caption[]{End-to-End runtime lesion study of the entire MNIST dataset and the FMA featurized music dataset. Each of DROP's contributions provides a runtime improvement.}
\label{fig:beyond_lesion}
\end{figure}
\end{comment}



\section{Conclusion}
\label{sec:conclusion}

Advanced data analytics techniques must scale to rising data volumes. 
DR techniques offer a powerful toolkit when processing these datasets, with PCA frequently outperforming popular techniques in exchange for high computational cost. 
In response, we propose DROP, a new dimensionality reduction optimizer. 
DROP combines progressive sampling, progress estimation, and online aggregation to identify high quality low dimensional bases via PCA without processing the entire dataset by balancing the runtime of downstream tasks and achieved dimensionality. 
Thus, DROP provides a first step in bridging the gap between quality and efficiency in end-to-end DR for downstream \red{analytics}. 

%We revisit canonical operators for time series dimensionality reduction and the measurement study of~\cite{keogh-study}, and show that PCA is more effective than popular alternatives in the data mining literature often by a margin of over $2\times$ on average on gold-standard time series benchmark data sets with respect to output data dimension. More surprisingly, we empirically demonstrate that a small number of samples are sufficient to accurately characterize directions of maximum variance and obtain a high-quality low-dimensional transformation.




\small
\bibliographystyle{ieee}
\bibliography{reference}


\end{document}
