

\documentclass[10pt,twocolumn,letterpaper]{article}

\usepackage{cvpr}
\usepackage{epsfig}
\usepackage{graphicx}
\usepackage{subfigure}
\usepackage{amsmath}
\usepackage{amssymb}
\usepackage{amsthm}
\usepackage{amsfonts}
\usepackage{mathrsfs}
\usepackage{booktabs}
\usepackage{multirow, eucal}

\usepackage{helvet}
\usepackage{courier}
\usepackage{bm}
%
%
%
\usepackage{graphicx}
\usepackage{color}
\usepackage{epstopdf}
\usepackage{wrapfig}
\usepackage{picinpar}
\usepackage{url}

\usepackage[vlined,ruled,linesnumbered]{algorithm2e}
\usepackage{cite}
\newtheorem{theorem}{Theorem}


\usepackage{caption}
\captionsetup{margin=5pt,font=small,labelfont=bf}

\cvprfinalcopy





\def\kuired{\textcolor{red}}
%
%
%
%
%

%
%
%
%
%
%
%

%

%
%
%

\def\cD{ {\cal D } }
\def\cW{ {\cal W } }


\def\diag{\mbox{diag}}
\def\rank{\mbox{rank}}
\def\grad{\mbox{\text{grad}}}
\def\dist{\mbox{dist}}
\def\sgn{\mbox{sgn}}
\def\tr{\mbox{tr}}
\def\etal{{\em et al.\/}\, }
\def\card{{\mbox{Card}}}

%
\def\balpha{\mbox{{\boldmath $\alpha$}}}
\def\bbeta{\mbox{{\boldmath $\beta$}}}
\def\bzeta{\mbox{{\boldmath $\zeta$}}}
\def\bgamma{\mbox{{\boldmath $\gamma$}}}
\def\bdelta{\mbox{{\boldmath $\delta$}}}
\def\bmu{\mbox{{\boldmath $\mu$}}}
\def\beps{\mbox{{\boldmath $\epsilon$}}}
\def\blambda{\mbox{{\boldmath $\lambda$}}}
\def\bnu{\mbox{{\boldmath $\nu$}}}
\def\bomega{\mbox{{\boldmath $\omega$}}}
\def\bfeta{\mbox{{\boldmath $\eta$}}}
\def\bsigma{\mbox{{\boldmath $\sigma$}}}
\def\bzeta{\mbox{{\boldmath $\zeta$}}}
\def\bphi{\mbox{{\boldmath $\phi$}}}
\def\bxi{\mbox{{\boldmath $\xi$}}}
\def\bvphi{\mbox{{\boldmath $\phi$}}}
\def\bdelta{\mbox{{\boldmath $\delta$}}}
\def\bvarsigma{\mbox{{\boldmath $\varsigma$}}}
\def\bXi{\mbox{{\boldmath $\Xi$}}}


\def\bPi{\mbox{{\boldmath $\Pi$}}}

\def\bDelta{\mbox{{\boldmath $\Delta$}}}
\def\bPi{\mbox{{\boldmath $\Pi$}}}
\def\bPsi{\mbox{{\boldmath $\Psi$}}}
\def\bSigma{\mbox{{\boldmath $\Sigma$}}}

%
\def\mA{{\mathcal A}}
\def\mB{{\mathcal B}}
\def\mC{{\mathcal C}}
\def\mD{{\mathcal D}}
\def\mE{{\mathcal E}}
\def\mF{{\mathcal F}}
\def\mG{{\mathcal G}}
\def\mH{{\mathcal H}}
\def\mI{{\mathcal I}}
\def\mJ{{\mathcal J}}
\def\mK{{\mathcal K}}
\def\mL{{\mathcal L}}
\def\mM{{\mathcal M}}
\def\mN{{\mathcal N}}
\def\mO{{\mathcal O}}
\def\mP{{\mathcal P}}
\def\mQ{{\mathcal Q}}
\def\mR{{\mathcal R}}
\def\mS{{\mathcal S}}
\def\mT{{\mathcal T}}
\def\mU{{\mathcal U}}
\def\mV{{\mathcal V}}
\def\mW{{\mathcal W}}
\def\mX{{\mathcal X}}
\def\mY{{\mathcal Y}}
\def\mZ{{\mathcal{Z}}}




%
\def\0{{\bf 0}}
\def\1{{\bf 1}}

%
\def\bA{{\bf A}}
\def\bB{{\bf B}}
\def\bC{{\bf C}}
\def\bD{{\bf D}}
\def\bE{{\bf E}}
\def\bF{{\bf F}}
\def\bG{{\bf G}}
\def\bH{{\bf H}}
\def\bI{{\bf I}}
\def\bJ{{\bf J}}
\def\bK{{\bf K}}
\def\bL{{\bf L}}
\def\bM{{\bf M}}
\def\bN{{\bf N}}
\def\bO{{\bf O}}
\def\bP{{\bf P}}
\def\bQ{{\bf Q}}
\def\bR{{\bf R}}
\def\bS{{\bf S}}
\def\bT{{\bf T}}
\def\bU{{\bf U}}
\def\bV{{\bf V}}
\def\bW{{\bf W}}
\def\bX{{\bf X}}
\def\bY{{\bf Y}}
\def\bZ{{\bf{Z}}}


%
\def\ba{{\bf a}}
\def\bb{{\bf b}}
\def\bc{{\bf c}}
\def\bd{{\bf d}}
\def\be{{\bf e}}
\def\bff{{\bf f}}
\def\bg{{\bf g}}
\def\bh{{\bf h}}
\def\bi{{\bf i}}
\def\bj{{\bf j}}
\def\bk{{\bf k}}
\def\bl{{\bf l}}
\def\bm{{\bf m}}
\def\bn{{\bf n}}
\def\bo{{\bf o}}
\def\bp{{\bf p}}
\def\bq{{\bf q}}
\def\br{{\bf r}}
\def\bs{{\bf s}}
\def\bt{{\bf t}}
\def\bu{{\bf u}}
\def\bv{{\bf v}}
\def\bw{{\bf w}}
\def\bx{{\bf x}}
\def\by{{\bf y}}
\def\bz{{\bf z}}

%
\def\hy{\hat{y}}
\def\hby{\hat{{\bf y}}}


%
%
%
%
%
%
%
%
%
%
%
%
%
%
%
%
%
%
%
%
%

%
\def\pd{{\succ\0}}
\def\psd{{\succeq\0}}
\def\vphi{\varphi}
\def\trsp{{\sf T}}


\def\mRMD{{\mathrm{D}}}
\def\mRMD{{\mathrm{D}}}
%

\def\kui{\textcolor{black}}



\def\citep{\cite}
\def\citet{\cite}
\newtheorem{coll}{Corollary}
\newtheorem{deftn}{Definition}
\newtheorem{thm}{Theorem}
\newtheorem{prop}{Proposition}
\newtheorem{lemma}{Lemma}
\newtheorem{remark}{Remark}
\newtheorem{ass}{Assumption}


%
\def\diag{\mbox{diag}}
\def\rank{\mbox{rank}}
\def\grad{\mbox{\text{grad}}}
\def\dist{\mbox{dist}}
\def\sgn{\mbox{sgn}}
\def\tr{\mbox{tr}}
\def\etal{{\em et al.\/}\, }
\def\card{{\mbox{Card}}}
\def\st{\mbox{s.t. }}

\def\kui{\textcolor{black}}
\def\young{\textcolor{black}}
\def\mark{\textcolor{red}}

%

%
\def\httilde{\mbox{\tt\raisebox{-.5ex}{\symbol{126}}}}

%


\begin{document}

%
\title{Towards Effective Low-bitwidth Convolutional Neural Networks\thanks{B. Zhuang, C. Shen, L. Liu and I. Reid are with The University of Adelaide,
  Australia. M. Tan is with South China University of Technology, China.}
  \thanks{Correspondence to C. Shen (e-mail: chhshen@gmail.com).}
%
}

\author{
Bohan Zhuang, Chunhua Shen,
Mingkui Tan, Lingqiao Liu,  Ian Reid
}




\maketitle


\begin{abstract}
	This paper tackles the problem of training a deep convolutional neural network with both low-precision weights and low-bitwidth activations. Optimizing a low-precision network is very challenging since the training process can easily get trapped in a poor local minima, which results in substantial accuracy loss. To mitigate this problem, we propose three simple-yet-effective approaches to improve the network training.
	First, we propose to use a two-stage optimization strategy to progressively find good local minima. Specifically, we propose to first optimize a net with quantized weights and then quantized activations. This is in contrast to the traditional methods which optimize them simultaneously.
	Second, following a similar spirit of the first method, we propose another progressive optimization approach which progressively decreases the bit-width from high-precision to low-precision during the course of training.
	Third, we adopt a novel learning scheme to jointly train a full-precision model alongside the low-precision one. By doing so, the full-precision model provides hints to guide the low-precision model training.
	Extensive experiments on various datasets (\ie, CIFAR-100 and ImageNet) show the effectiveness of the proposed methods. To highlight, using our methods to train a 4-bit precision network leads to no performance decrease in comparison with its full-precision counterpart with standard network architectures (\ie, AlexNet and ResNet-50).


\end{abstract}


\tableofcontents
\clearpage





	% \leavevmode
% \\
% \\
% \\
% \\
% \\
\section{Introduction}
\label{introduction}

AutoML is the process by which machine learning models are built automatically for a new dataset. Given a dataset, AutoML systems perform a search over valid data transformations and learners, along with hyper-parameter optimization for each learner~\cite{VolcanoML}. Choosing the transformations and learners over which to search is our focus.
A significant number of systems mine from prior runs of pipelines over a set of datasets to choose transformers and learners that are effective with different types of datasets (e.g. \cite{NEURIPS2018_b59a51a3}, \cite{10.14778/3415478.3415542}, \cite{autosklearn}). Thus, they build a database by actually running different pipelines with a diverse set of datasets to estimate the accuracy of potential pipelines. Hence, they can be used to effectively reduce the search space. A new dataset, based on a set of features (meta-features) is then matched to this database to find the most plausible candidates for both learner selection and hyper-parameter tuning. This process of choosing starting points in the search space is called meta-learning for the cold start problem.  

Other meta-learning approaches include mining existing data science code and their associated datasets to learn from human expertise. The AL~\cite{al} system mined existing Kaggle notebooks using dynamic analysis, i.e., actually running the scripts, and showed that such a system has promise.  However, this meta-learning approach does not scale because it is onerous to execute a large number of pipeline scripts on datasets, preprocessing datasets is never trivial, and older scripts cease to run at all as software evolves. It is not surprising that AL therefore performed dynamic analysis on just nine datasets.

Our system, {\sysname}, provides a scalable meta-learning approach to leverage human expertise, using static analysis to mine pipelines from large repositories of scripts. Static analysis has the advantage of scaling to thousands or millions of scripts \cite{graph4code} easily, but lacks the performance data gathered by dynamic analysis. The {\sysname} meta-learning approach guides the learning process by a scalable dataset similarity search, based on dataset embeddings, to find the most similar datasets and the semantics of ML pipelines applied on them.  Many existing systems, such as Auto-Sklearn \cite{autosklearn} and AL \cite{al}, compute a set of meta-features for each dataset. We developed a deep neural network model to generate embeddings at the granularity of a dataset, e.g., a table or CSV file, to capture similarity at the level of an entire dataset rather than relying on a set of meta-features.
 
Because we use static analysis to capture the semantics of the meta-learning process, we have no mechanism to choose the \textbf{best} pipeline from many seen pipelines, unlike the dynamic execution case where one can rely on runtime to choose the best performing pipeline.  Observing that pipelines are basically workflow graphs, we use graph generator neural models to succinctly capture the statically-observed pipelines for a single dataset. In {\sysname}, we formulate learner selection as a graph generation problem to predict optimized pipelines based on pipelines seen in actual notebooks.

%. This formulation enables {\sysname} for effective pruning of the AutoML search space to predict optimized pipelines based on pipelines seen in actual notebooks.}
%We note that increasingly, state-of-the-art performance in AutoML systems is being generated by more complex pipelines such as Directed Acyclic Graphs (DAGs) \cite{piper} rather than the linear pipelines used in earlier systems.  
 
{\sysname} does learner and transformation selection, and hence is a component of an AutoML systems. To evaluate this component, we integrated it into two existing AutoML systems, FLAML \cite{flaml} and Auto-Sklearn \cite{autosklearn}.  
% We evaluate each system with and without {\sysname}.  
We chose FLAML because it does not yet have any meta-learning component for the cold start problem and instead allows user selection of learners and transformers. The authors of FLAML explicitly pointed to the fact that FLAML might benefit from a meta-learning component and pointed to it as a possibility for future work. For FLAML, if mining historical pipelines provides an advantage, we should improve its performance. We also picked Auto-Sklearn as it does have a learner selection component based on meta-features, as described earlier~\cite{autosklearn2}. For Auto-Sklearn, we should at least match performance if our static mining of pipelines can match their extensive database. For context, we also compared {\sysname} with the recent VolcanoML~\cite{VolcanoML}, which provides an efficient decomposition and execution strategy for the AutoML search space. In contrast, {\sysname} prunes the search space using our meta-learning model to perform hyperparameter optimization only for the most promising candidates. 

The contributions of this paper are the following:
\begin{itemize}
    \item Section ~\ref{sec:mining} defines a scalable meta-learning approach based on representation learning of mined ML pipeline semantics and datasets for over 100 datasets and ~11K Python scripts.  
    \newline
    \item Sections~\ref{sec:kgpipGen} formulates AutoML pipeline generation as a graph generation problem. {\sysname} predicts efficiently an optimized ML pipeline for an unseen dataset based on our meta-learning model.  To the best of our knowledge, {\sysname} is the first approach to formulate  AutoML pipeline generation in such a way.
    \newline
    \item Section~\ref{sec:eval} presents a comprehensive evaluation using a large collection of 121 datasets from major AutoML benchmarks and Kaggle. Our experimental results show that {\sysname} outperforms all existing AutoML systems and achieves state-of-the-art results on the majority of these datasets. {\sysname} significantly improves the performance of both FLAML and Auto-Sklearn in classification and regression tasks. We also outperformed AL in 75 out of 77 datasets and VolcanoML in 75  out of 121 datasets, including 44 datasets used only by VolcanoML~\cite{VolcanoML}.  On average, {\sysname} achieves scores that are statistically better than the means of all other systems. 
\end{itemize}


%This approach does not need to apply cleaning or transformation methods to handle different variances among datasets. Moreover, we do not need to deal with complex analysis, such as dynamic code analysis. Thus, our approach proved to be scalable, as discussed in Sections~\ref{sec:mining}.
	\section{Related Work}
%\mz{We lack a comparison to this paper: https://arxiv.org/abs/2305.14877}
%\anirudh{refine to be more on-topic?}
\iffalse
\paragraph{In-Context Learning} As language models have scaled, the ability to learn in-context, without any weight updates, has emerged. \cite{brown2020language}. While other families of large language models have emerged, in-context learning remains ubiquitous \cite{llama, bloom, gptneo, opt}. Although such as HELM \cite{helm} have arisen for systematic evaluation of \emph{models}, there is no systematic framework to our knowledge for evaluating \emph{prompting methods}, and validating prompt engineering heuristics. The test-suite we propose will ensure that progress in the field of prompt-engineering is structured and objectively evaluated. 

\paragraph{Prompt Engineering Methods} Researchers are interested in the automatic design of high performing instructions for downstream tasks. Some focus on simple heuristics, such as selecting instructions that have the lowest perplexity \cite{lowperplexityprompts}. Other methods try to use large language models to induce an instruction when provided with a few input-output pairs \cite{ape}. Researchers have also used RL objectives to create discrete token sequences that can serve as instructions \cite{rlprompt}. Since the datasets and models used in these works have very little intersection, it is impossible to compare these methods objectively and glean insights. In our work, we evaluate these three methods on a diverse set of tasks and models, and analyze their relative performance. Additionally, we recognize that there are many other interesting angles of prompting that are not covered by instruction engineering \cite{weichain, react, selfconsistency}, but we leave these to future work.

\paragraph{Analysis of Prompting Methods} While most prompt engineering methods focus on accuracy, there are many other interesting dimensions of performance as well. For instance, researchers have found that for most tasks, the selection of demonstrations plays a large role in few-shot accuracy \cite{whatmakesgoodicexamples, selectionmachinetranslation, knnprompting}. Additionally, many researchers have found that even permuting the ordering of a fixed set of demonstrations has a significant effect on downstream accuracy \cite{fantasticallyorderedprompts}. Prompts that are sensitive to the permutation of demonstrations have been shown to also have lower accuracies \cite{relationsensitivityaccuracy}. Especially in low-resource domains, which includes the large public usage of in-context learning, these large swings in accuracy make prompting less dependable. In our test-suite we include sensitivity metrics that go beyond accuracy and allow us to find methods that are not only performant but reliable.

\paragraph{Existing Benchmarks} We recognize that other holistic in-context learning benchmarks exist. BigBench is a large benchmark of 204 tasks that are beyond the capabilities of current LLMs. BigBench seeks to evaluate the few-shot abilities of state of the art large language models, focusing on performance metrics such as accuracy \cite{bigbench}. Similarly, HELM is another benchmark for language model in-context learning ability. Rather than only focusing on performance, HELM branches out and considers many other metrics such as robustness and bias \cite{helm}. Both BigBench and HELM focus on ranking different language model, while fix a generic instruction and prompt format. We instead choose to evaluate instruction induction / selection methods over a fixed set of models. We are the first ever evaluation script that compares different prompt-engineering methods head to head. 
\fi

\paragraph{In-Context Learning and Existing Benchmarks} As language models have scaled, in-context learning has emerged as a popular paradigm and remains ubiquitous among several autoregressive LLM families \cite{brown2020language, llama, bloom, gptneo, opt}. Benchmarks like BigBench \cite{bigbench} and HELM \cite{helm} have been created for the holistic evaluation of these models. BigBench focuses on few-shot abilities of state-of-the-art large language models, while HELM extends to consider metrics like robustness and bias. However, these benchmarks focus on evaluating and ranking \emph{language models}, and do not address the systematic evaluation of \emph{prompting methods}. Although contemporary work by \citet{yang2023improving} also aims to perform a similar systematic analysis of prompting methods, they focus on simple probability-based prompt selection while we evaluate a broader range of methods including trivial instruction baselines, curated manually selected instructions, and sophisticated automated instruction selection.

\paragraph{Automated Prompt Engineering Methods} There has been interest in performing automated prompt-engineering for target downstream tasks within ICL. This has led to the exploration of various prompting methods, ranging from simple heuristics such as selecting instructions with the lowest perplexity \cite{lowperplexityprompts}, inducing instructions from large language models using a few annotated input-output pairs \cite{ape}, to utilizing RL objectives to create discrete token sequences as prompts \cite{rlprompt}. However, these works restrict their evaluation to small sets of models and tasks with little intersection, hindering their objective comparison. %\mz{For paragraphs that only have one work in the last line, try to shorten the paragraph to squeeze in context.}

\paragraph{Understanding in-context learning} There has been much recent work attempting to understand the mechanisms that drive in-context learning. Studies have found that the selection of demonstrations included in prompts significantly impacts few-shot accuracy across most tasks \cite{whatmakesgoodicexamples, selectionmachinetranslation, knnprompting}. Works like \cite{fantasticallyorderedprompts} also show that altering the ordering of a fixed set of demonstrations can affect downstream accuracy. Prompts sensitive to demonstration permutation often exhibit lower accuracies \cite{relationsensitivityaccuracy}, making them less reliable, particularly in low-resource domains.

Our work aims to bridge these gaps by systematically evaluating the efficacy of popular instruction selection approaches over a diverse set of tasks and models, facilitating objective comparison. We evaluate these methods not only on accuracy metrics, but also on sensitivity metrics to glean additional insights. We recognize that other facets of prompting not covered by instruction engineering exist \cite{weichain, react, selfconsistency}, and defer these explorations to future work. 
\section{Methods}
\begin{figure*}[!t]
	\centering
	\resizebox{0.9\linewidth}{!}
	{
		\begin{tabular}{c}
			\includegraphics{pdf/1.pdf}
		\end{tabular}
	}
	\caption{Demonstration of the guided training strategy. We use the residual network structure for illustration.}
	\label{fig:knowledge_transfer}
\end{figure*}
In this section, we will first revisit the quantization function in the neural network and the way to train it. Then we will elaborate our three methods in the subsequent sections.
\subsection{Quantization function revisited} \label{sec:baseline}
%

A common practise in training a neural network with low-precision weights and activations is to introduce a quantization function. Considering the general case of $k$-bit quantization as in~\cite{zhou2016dorefa}, we define the quantization function $Q(\cdot)$ to be
\begin{equation}
	{z_q} = Q({z_r}) = \frac{1}{{{2^k} - 1}}round(({2^k} - 1){z_r})
\end{equation}
where ${z_r} \in [0,1]$ denotes the full-precision value and ${z_q} \in [0,1]$ denotes the quantized value. With this quantization function, we can define the weight quantization process and the activation quantization process as follows:


\noindent \textbf{Quantization on weights}:
\begin{equation}\label{eq:quan-weigtht}
	{w_q} = Q(\frac{{\tanh (w)}}{{2\max (\left| {\tanh (w)} \right|)}} + \frac{1}{2}).
\end{equation}In other words, we first use $\frac{{\tanh (w)}}{{2\max (\left| {\tanh (w)} \right|)}} + \frac{1}{2}$ to obtain a normalized version of $w$ and then perform the quantization, where $\tanh(\cdot)$ is adopted to reduce the impact of large values.

\noindent \textbf{Quantization on activations}:

 Same as \cite{zhou2016dorefa}, we first use a clip function $f(x) = clip(x,\,0,1)$ to bound the activations to $[0, 1]$. After that, we conduct quantize the activation by applying the quantization function $Q(\cdot)$ on $f(x)$.
\begin{equation} \label{eq:quan-activations}
	{x_q} = Q(f(x)).
\end{equation}

\noindent \textbf{Back-propagation with quantization function}: In general, the quantization function is non-differentiable and thus it is impossible to directly apply the back-propagation to train the network. To overcome this issue, we adopt the straight-through estimator \cite{zhou2016dorefa, hubara2016binarized, bengio2013estimating} to approximate the gradients calculation. Formally, we approximate the partial gradient $\frac{{\partial {z_q}}}{{\partial {z_r}}}$ with an identity mapping, namely $\frac{{\partial {z_q}}}{{\partial {z_r}}} \approx 1$.  Accordingly, $\frac{{\partial l}}{{\partial {z_r}}}$ can be approximated by
\begin{equation}
	\frac{{\partial l}}{{\partial {z_r}}} = \frac{{\partial l}}{{\partial {z_q}}}\frac{{\partial {z_q}}}{{\partial {z_r}}} \approx \frac{{\partial l}}{{\partial {z_q}}}.
\end{equation}


%
%
%
%
%
%
%
\subsection{Two-stage optimization}\label{sec:two-stage}
With the straight-through estimator, it is possible to directly optimize the low-precision network. However, the gradient approximation of the quantization function inevitably introduces noisy signal for updating network parameters. Strictly speaking, the approximated gradient may not be the right updating direction. Thus, the training process will be more likely to get trapped at a poor local minima than training a full precision model. Applying the quantization function to both weights and activations further worsens the situation.

To reduce the difficulty of training, we devise a two-stage optimization procedure: at the first stage, we only quanitze the weights of the network while setting the activations to be full precision. After the converge (or after certain number of iterations) of this model, we further apply the quantization function on the activations as well and retrain the network. Essentially, the first stage of this method is a related subproblem of the target one. Compared to the target problem, it is easier to optimize since it only introduces quantization function on weights. Thus, we are more likely to arrive at a good solution for this sub-problem. Then, using it to initialize the target problem may help the network avoid poor local minima which will be encountered if we train the network from scratch.
Let $M_{low}^{K}$ be the high-precision model with $K$-bit. We propose to learn a low-precision model $M_{low}^{k}$ in a two-stage manner with $M_{low}^{K}$ serving as the initial point, where $k<K$.
 The detailed algorithm is shown in Algorithm \ref{algo:two-stage}.
\begin{algorithm}[]
	\KwIn{Training data $\{ ({{\bf{x}}_i},y_i)\}_{i=1}^N$; A $K$-bit precision model $M_{low}^K$.}
	\KwOut{A low-precision deep model $M^k_{low}$ with weights ${{\bf{W}}_{low}}$ and activations being quantized into $k$-bit.}

	\textbf{Stage 1}: Quantize ${{\bf{W}}_{low}}$:\\
	\For{ $\mathrm{epoch} = 1,...,L$}
	{
		\For{ $t = 1,...T$}
		{
			Randomly sample a mini-batch data;\\
			Quantize the weights ${{\bf{W}}_{low}}$ into $k$-bit by calling some quantization methods with $K$-bit activations\;
		}
	}
	\textbf{Stage 2}: Quantize activations:\\
	Initialize ${{\bf{W}}_{low}}$ using the converged $k$-bit weights from \textbf{Stage 1} as the starting point; \\
	\For{ $\mathrm{epoch} = 1,...,L$}
	{
		\For{ $t = 1,...T$}
		{
			Randomly sample a mini-batch data;\\
			Quantize the activations into $k$-bit  by calling some quantization methods while keeping the weights to $k$-bit;
		}
	}
	\caption{Two-stage optimization for $k$-bit quantization}
	\label{algo:two-stage}
\end{algorithm}


\subsection{Progressive quantization} \label{sec:progressive}

%

The aforementioned two-stage optimization approach suggests the benefits of using a related easy optimized problem to find a good initialization. However, separating the quantization of weights and activations is not the only solution to implement the above idea. In this paper, we also propose another solution which progressively lower the bitwidth of the quantization during the course of network training.
Specifically, we progressively conduct the quantization from higher precisions to lower precisions (\eg, 32-bit $\to$ 16-bit $\to$ 4-bit $\to$ 2-bit). The model of higher precision will be used the the starting point of the relatively lower precision, in analogy with annealing.


Let $\{{b_1},...,{b_n}\}$ be a  sequence precisions, where  $b_n<b_{n-1}, ..., b_2<{b_1}$, $b_n$ is the target precision and $b_1$ is set to 32 by default. The whole progressive optimization procedure  is summarized in as Algorithm~\ref{algo:progressive optimization}.
%
 Let $M_{low}^{k}$ be the low-precision model with $k$-bit and $M_{full}$ be the full precision model. In each step, we propose to learn $M_{low}^{k}$, with the solution in the $(i-1)$-th step, denoted by $M_{low}^{K}$, serving as the initial point, where $k<K$.
%

%
%
%



%

%


%

%

%

%
\begin{algorithm}[]
	\KwIn{Training data $\{ ({{\bf{x}}_j},y_j)\}_{j=1}^N$; A pre-trained 32-bit full-precision  model ${M_{full}}$ as baseline; the precision sequence $\{{b_1},...,{b_n}\}$ where $b_n<b_{n-1}, ..., b_2<{b_1} = 32$.}
	\KwOut{A low-precision deep model $M_{low}^{b_n}$.}
	Let $M_{low}^{b_1}=M_{full}$, where $b_1 = 32$\;
	\For{ $i = 2,...n$}
	{
		Let $k = b_i$ and $K=b_{i-1}$\;
		Obtain $M_{low}^{k}$ by calling some quantization methods with $M_{low}^{K}$  being the input\;
		%
		%
		%
	}
	\caption{Progressive quantization for accurate CNNs with low-precision weights and activations}
	\label{algo:progressive optimization}
\end{algorithm}


%


%


%

\subsection{Guided training with a full-precision network}\label{sec:mutual}
The third method proposed in this paper is inspired by the success of using information distillation ~\cite{romero2014fitnets, hinton2015distilling, parisotto2016actor, zagoruyko2016paying, ba2014deep} to train a relatively shallow network. Specifically, these methods usually use a teacher model (usually a pretrained deeper network) to provide guided signal for the shallower network. Following this spirit, we propose to train the low-precision network alongside another guidance network. Unlike the work in \cite{romero2014fitnets, hinton2015distilling, parisotto2016actor, zagoruyko2016paying, ba2014deep}, the guidance network shares the same architecture as the target network but is pretrained with full-precision weights and activations.

However, a pre-trained model may not be necessarily optimal or may not be suitable for quantization. As a result, directly using a fixed pretrained model to guide the target network may not produce the best guidance signals. To mitigate this problem, we do not fix the parameters of a pretrained full precision network as in the previous work \cite{zhang2017deep}.


By using the guidance training strategy, we assume that there exist some full-precision models with good generalization performance, and an accurate low-precision model can be obtained by directly performing the quantization on those full-precision models. In this sense, the feature maps of the learned low-precision model should be close to that obtained by directly doing quantization on the full-precision model. To achieve this, essentially, in our learning scheme, we can jointly train the full-precision and low-precision models. This allows these two models adapt to each other. We even find by doing so the performance of the full-precision model can be slightly improved in some cases.

%

Formally, let ${{\bf{W}}_{full}}$ and ${{\bf{W}}_{low}}$ be the full-precision model and low-precision model, respectively. Let $\mu ({\bf{x}};{{\bf{W}}_{{full}}})$ and $\nu ({\bf{x}};{{\bf{W}}_{{low}}})$ be the nested feature maps (e.g., activations) of the full-precision model and low-precision model, respectively. To create the guidance signal, we may require that the nested feature maps from the two models should be similar. However,  $\mu ({\bf{x}};{{\bf{W}}_{{full}}})$ and $\nu ({\bf{x}};{{\bf{W}}_{{low}}})$  is usually not directly comparable since one is full precision and the other is low-precision.

%

%
%




%




%



%


%

%
%

%
%
%



%

To link these two models,  we can directly quantize the weights and activations of the full-precision model by equations (\ref{eq:quan-weigtht}) and (\ref{eq:quan-activations}). For simplicity, we denote the quantized feature maps by  $Q(\mu ({\bf{x}};{{\bf{W}}_{{full}}}))$. Thus, $Q(\mu ({\bf{x}};{{\bf{W}}_{{full}}}))$ and  $\nu ({\bf{x}};{{\bf{W}}_{{low}}})$ will become comparable. Then we can define the guidance loss as:
\begin{equation}
	R({{\bf{W}}_{full}},{{\bf{W}}_{low}}) = \frac{1}{2}\parallel Q(\mu ({\bf{x}};{{\bf{W}}_{{full}}})) - \nu ({\bf{x}};{{\bf{W}}_{{low}}}){\parallel^2},
\end{equation}
where $\parallel\cdot\parallel$ denotes some proper norms.
%


Let ${L_{{\theta _1}}}$ and ${L_{{\theta _2}}}$ be the cross-entropy classification losses for the full-precision and low-precision model, respectively. The guidance loss will be added to ${L_{{\theta _1}}}$ and ${L_{{\theta _2}}}$, respectively, resulting in two new objectives for the two networks, namely
\begin{equation} \label{eq:objective1}
	L_1({{\bf{W}}_{full}})  = {L_{{\theta _1}}} + \lambda R({{\bf{W}}_{full}},{{\bf{W}}_{low}}).
\end{equation}
and
\begin{equation} \label{eq:objective2}
	L_2({{\bf{W}}_{low}})  = {L_{{\theta _2}}} +  \lambda R({{\bf{W}}_{full}},{{\bf{W}}_{low}}).
\end{equation}
where $\lambda$ is a balancing parameter. Here, the guidance loss $R$ can be considered as some regularization on ${L_{{\theta _1}}}$ and ${L_{{\theta _2}}}$.


%
In the learning procedure, both ${{\bf{W}}_{full}}$ and ${{\bf{W}}_{low}}$ will be updated by minimizing $L_1({{\bf{W}}_{full}})$ and $L_2({{\bf{W}}_{low}})$ separately, using a mini-batch stochastic gradient descent method. The detailed algorithm is shown in Algorithm \ref{algo:one-mutual learning}. A high-bit precision model $M_{low}^K$ is used as an initialization of $M_{low}^k$, where $K>k$. Specifically, for the full-precision model, we have $K=32$. Relying on $M_{full}$, the weights and activations of $M_{low}^k$ can be initialized by equations (\ref{eq:quan-weigtht}) and (\ref{eq:quan-activations}), respectively.


Note that the training process of the two networks are different.
When updating ${{\bf{W}}_{low}}$ by minimizing $L_2({{\bf{W}}_{low}})$, we use full-precision model as the initialization and apply the forward-backward propagation rule in Section \ref{sec:baseline}  to fine-tune the model. When updating ${{\bf{W}}_{full}}$ by minimizing $L_1({{\bf{W}}_{full}})$, we use conventional forward-backward propagation to fine-tune the model.


\begin{algorithm}[]
	\KwIn{Training data $\{ ({{\bf{x}}_i},y_i)\}_{i=1}^N$; A pre-trained 32-bit full-precision model $M_{full}$; A $k$-bit precision model $M_{low}^k$.}
	\KwOut{A low-precision deep model $M^k_{low}$ with weights and activations being quantized into $k$ bits.}
	Initialize $M_{low}^k$ based on $M_{full}$;\\
	\For{ $\mathrm{epoch} = 1,...,L$}
	{
		\For{ $t = 1,...T$}
		{
			Randomly sample a mini-batch data;\\
			Quantize the weights ${{\bf{W}}_{low}}$  and activations into $k$-bit by minimizing $L_2({{\bf{W}}_{low}})$\;
			Update $M_{full}$ by minimizing $L_1({{\bf{W}}_{full}})$\;
		}

	}
	\caption{Guided training with a full-precision network for $k$-bit quantization}
	\label{algo:one-mutual learning}
\end{algorithm}


%
%
%
%
%
%
%
%
%
%

\subsection{Remark on the proposed methods}
The proposed three approaches tackle the difficulty in training a low-precision model with different strategies. They can be applied independently. However, it is also possible to combine them together. For example, we can apply the progressive quantization to any of the steps in the two-stage approach; we can also apply the guided training to any sub-step in the progressive training. Detailed analysis on possible combinations will be experimentally evaluated in the experiment section.


\subsection{Implementation details} \label{sec:implementation}

In all the three methods, we quantize the weights and activations of all layers except that the input data are kept to 8-bit. Furthermore, to promote convergence, we propose to add a scalar layer after the last fully-connected layer before feeding the low-bit activations into the softmax function for classification. The scalar layer has only one trainable small scalar parameter and is initialized to 0.01 in our approach.


During training, we randomly crop 224x224 patches from an image or its horizontal flip, with the per-pixel mean subtracted. We don't use any further data augmentation in our implementation. We adopt batch normalization (BN)~\cite{ioffe2015batch} after each convolution before activation. For pretraining the full-precision baseline model, we use Nesterov SGD and batch size is set to 256. The learning rate starts from 0.01 and is divided by 10 every 30 epochs. We use a weight decay 0.0001 and a momentum 0.9. For weights and activations quantization, the initial learning rate is set to 0.001 and is divided by 10 every 10 epochs. We use a simple single-crop testing for standard evaluation. Following~\cite{zagoruyko2016paying}, for ResNet-50, we add only two guidance losses in the 2 last groups of residual blocks. And for AlexNet, we add two guidance losses in the last two fully-connected layers.



%
	 \section{Experimental Evaluation}
\label{sec:experiment}
To demonstrate the viability of our modeling methodology, we show experimentally how through the deliberate combination and configuration of parallel FREEs, full control over 2DOF spacial forces can be achieved by using only the minimum combination of three FREEs.
To this end, we carefully chose the fiber angle $\Gamma$ of each of these actuators to achieve a well-balanced force zonotope (Fig.~\ref{fig:rigDiagram}).
We combined a contracting and counterclockwise twisting FREE with a fiber angle of $\Gamma = 48^\circ$, a contracting and clockwise twisting FREE with $\Gamma = -48^\circ$, and an extending FREE with $\Gamma = -85^\circ$.
All three FREEs were designed with a nominal radius of $R$ = \unit[5]{mm} and a length of $L$ = \unit[100]{mm}.
%
\begin{figure}
    \centering
    \includegraphics[width=0.75\linewidth]{figures/rigDiagram_wlabels10.pdf}
    \caption{In the experimental evaluation, we employed a parallel combination of three FREEs (top) to yield forces along and moments about the $z$-axis of an end effector.
    The FREEs were carefully selected to yield a well-balanced force zonotope (bottom) to gain full control authority over $F^{\hat{z}_e}$ and $M^{\hat{z}_e}$.
    To this end, we used one extending FREE, and two contracting FREEs which generate antagonistic moments about the end effector $z$-axis.}
    \label{fig:rigDiagram}
\end{figure}


\subsection{Experimental Setup}
To measure the forces generated by this actuator combination under a varying state $\vec{x}$ and pressure input $\vec{p}$, we developed a custom built test platform (Fig.~\ref{fig:rig}). 
%
\begin{figure}
    \centering
    \includegraphics[width=0.9\linewidth]{figures/photos/rig_labeled.pdf}
    \caption{\revcomment{1.3}{This experimental platform is used to generate a targeted displacement (extension and twist) of the end effector and to measure the forces and torques created by a parallel combination of three FREEs. A linear actuator and servomotor impose an extension and a twist, respectively, while the net force and moment generated by the FREEs is measured with a force load cell and moment load cell mounted in series.}}
    \label{fig:rig}
\end{figure}
%
In the test platform, a linear actuator (ServoCity HDA 6-50) and a rotational servomotor (Hitec HS-645mg) were used to impose a 2-dimensional displacement on the end effector. 
A force load cell (LoadStar  RAS1-25lb) and a moment load cell (LoadStar RST1-6Nm) measured the end-effector forces $F^{\hat{z_e}}$ and moments $M^{\hat{z_e}}$, respectively.
During the experiments, the pressures inside the FREEs were varied using pneumatic pressure regulators (Enfield TR-010-g10-s). 

The FREE attachment points (measured from the end effector origin) were measured to be:
\begin{align}
    \vec{d}_1 &= \bmx 0.013 & 0 & 0 \emx^T  \text{m}\\
    \vec{d}_2 &= \bmx -0.006 & 0.011 & 0 \emx^T  \text{m}\\
    \vec{d}_3 &= \bmx -0.006 & -0.011 & 0 \emx^T \text{m}
%    \vec{d}_i &= \bmx 0 & 0 & 0 \emx^T , && \text{for } i = 1,2,3
\end{align}
All three FREEs were oriented parallel to the end effector $z$-axis:
\begin{align}
    \hat{a}_i &= \bmx 0 & 0 & 1 \emx^T, \hspace{20pt} \text{for } i = 1,2,3
\end{align}
Based on this geometry, the transformation matrices $\bar{\mathcal{D}}_i$ were given by:
\begin{align}
    \bar{\mathcal{D}}_1 &= \bmx 0 & 0 & 1 & 0 & -0.013 & 0 \\ 0 & 0 & 0 & 0 & 0 & 1 \emx^T  \\
    \bar{\mathcal{D}}_2 &= \bmx 0 & 0 & 1 & 0.011 & 0.006 & 0 \\ 0 & 0 & 0 & 0 & 0 & 1 \emx^T  \\
    \bar{\mathcal{D}}_3 &= \bmx 0 & 0 & 1 & -0.011 & 0.006 & 0 \\ 0 & 0 & 0 & 0 & 0 & 1 \emx^T 
%    \bar{\mathcal{D}}_i &= \bmx 0 & 0 & 1 & 0 & 0 & 0 \\ 0 & 0 & 0 & 0 & 0 & 1 \emx^T , && \text{for } i = 1,2,3
\end{align}
These matrices were used in equation \eqref{eq:zeta} to yield the state-dependent fluid Jacobian $\bar{J}_x$ and to compute the resulting force zontopes.
%while using measured values of $\vec{\zeta}^{\,\text{meas}} (\vec{q}, \vec{P})$ and $\vec{\zeta}^{\,\text{meas}} (\vec{q}, 0)$ in \eqref{eq:fiberIso} yields the empirical measurements of the active force.



\subsection{Isolating the Active Force}
To compare our model force predictions (which focus only on the active forces induced by the fibers)
to those measured empirically on a physical system, we had to remove the elastic force components attributed to the elastomer. 
Under the assumption that the elastomer force is merely a function of the displacement $\vec{x}$ and independent of pressure $\vec{p}$ \cite{bruder2017model}, this force component can be approximated by the measured force at a pressure of $\vec{p}=0$. 
That is: 
\begin{align}
    \vec{f}_{\text{elast}} (\vec{x}) = \vec{f}_{\text{\,meas}} (\vec{x}, 0)
\end{align}
With this, the active generalized forces were measured indirectly by subtracting off the force generated at zero pressure:
\begin{align}
    \vec{f} (\vec{x}, \vec{p})  &= \vec{f}_{\text{meas}} (\vec{x}, \vec{p}) - \vec{f}_{\text{meas}} (\vec{x}, 0)     \label{eq:fiberIso}
\end{align}


%To validate our parallel force model, we compare its force predictions, $\vec{\zeta}_{\text{pred}}$, to those measured empirically on a physical system, $\vec{\zeta}_\text{meas}$. 
%From \eqref{eq:Z} and \eqref{eq:zeta}, the force at the end effector is given by:
%\begin{align}
%    \vec{\zeta}(\vec{q}, \vec{P}) &= \sum_{i=1}^n \bar{\mathcal{D}}_i \left( {\bar{J}_V}_i^T(\vec{q_i}) P_i + \vec{Z}_i^{\text{elast}} (\vec{q_i}) \right) \\
%    &= \underbrace{\sum_{i=1}^n \bar{\mathcal{D}}_i {\bar{J}_V}_i^T(\vec{q_i}) P_i}_{\vec{\zeta}^{\,\text{fiber}} (\vec{q}, \vec{P})} + \underbrace{\sum_{i=1}^n \bar{\mathcal{D}}_i \vec{Z}_i^{\text{elast}} (\vec{q_i})}_{\vec{\zeta}^{\text{elast}} (\vec{q})}   \label{eq:zetaSplit}
%     &= \vec{\zeta}^{\,\text{fiber}} (\vec{q}, \vec{P}) + \vec{\zeta}^{\text{elast}} (\vec{q})
%\end{align}
%\Dan{These will need to reflect changes made to previous section.}
%The model presented in this paper does not specify the elastomer forces, $\vec{\zeta}^{\text{elast}}$, therefore we only validate its predictions %of the fiber forces, $\vec{\zeta}^{\,\text{fiber}}$. 
%We isolate the fiber forces by noting that $\vec{\zeta}^{\text{elast}} (\vec{q}) = \vec{\zeta}(\vec{q}, 0)$ and rearranging \eqref{eq:zetaSplit}
%\begin{align}
%    \vec{\zeta}^{\,\text{fiber}} (\vec{q}, \vec{P})  &= \vec{\zeta}(\vec{q}, \vec{P}) - \vec{\zeta}(\vec{q}, 0)     \label{eq:fiberIso}
%%    \vec{\zeta}^{\,\text{fiber}}_{\text{emp}} (\vec{q}, \vec{P})  &= \vec{\zeta}_{\text{emp}}(\vec{q}, \vec{P}) - %\vec{\zeta}_{\text{emp}}(\vec{q}, 0)
%\end{align}
%Thus we measure the fiber forces indirectly by subtracting off the forces generated at zero pressure.  


\subsection{Experimental Protocol}
The force and moment generated by the parallel combination of FREEs about the end effector $z$-axis  was measured in four different geometric configurations: neutral, extended, twisted, and simultaneously extended and twisted (see Table \ref{table:RMSE} for the exact deformation amounts). 
At each of these configurations, the forces were measured at all pressure combinations in the set
\begin{align}
    \mathcal{P} &= \left\{ \bmx \alpha_1 & \alpha_2 & \alpha_3 \emx^T p^{\text{max}} \, : \, \alpha_i = \left\{ 0, \frac{1}{4}, \frac{1}{2}, \frac{3}{4}, 1 \right\} \right\}
\end{align}
with $p^{\text{max}}$ = \unit[103.4]{kPa}. 
\revcomment{3.2}{The experiment was performed twice using two different sets of FREEs to observe how fabrication variability might affect performance. The results from Trial 1 are displayed in Fig.~\ref{fig:results} and the error for both trials is given in Table \ref{table:RMSE}.}



\subsection{Results}

\begin{figure*}[ht]
\centering

\def\picScale{0.08}    % define variable for scaling all pictures evenly
\def\plotScale{0.2}    % define variable for scaling all plots evenly
\def\colWidth{0.22\linewidth}

\begin{tikzpicture} %[every node/.style={draw=black}]
% \draw[help lines] (0,0) grid (4,2);
\matrix [row sep=0cm, column sep=0cm, style={align=center}] (my matrix) at (0,0) %(2,1)
{
& \node (q1) {(a) $\Delta l = 0, \Delta \phi = 0$}; & \node (q2) {(b) $\Delta l = 5\text{mm}, \Delta \phi = 0$}; & \node (q3) {(c) $\Delta l = 0, \Delta \phi = 20^\circ$}; & \node (q4) {(d) $\Delta l = 5\text{mm}, \Delta \phi = 20^\circ$};

\\

&
\node[style={anchor=center}] {\includegraphics[width=\colWidth]{figures/photos/s0w0pic_colored.pdf}}; %\fill[blue] (0,0) circle (2pt);
&
\node[style={anchor=center}] {\includegraphics[width=\colWidth]{figures/photos/s5w0pic_colored.pdf}}; %\fill[blue] (0,0) circle (2pt);
&
\node[style={anchor=center}] {\includegraphics[width=\colWidth]{figures/photos/s0w20pic_colored.pdf}}; %\fill[blue] (0,0) circle (2pt);
&
\node[style={anchor=center}] {\includegraphics[width=\colWidth]{figures/photos/s5w20pic_colored.pdf}}; %\fill[blue] (0,0) circle (2pt);

\\

\node[rotate=90] (ylabel) {Moment, $M^{\hat{z}_e}$ (N-m)};
&
\node[style={anchor=center}] {\includegraphics[width=\colWidth]{figures/plots3/s0w0.pdf}}; %\fill[blue] (0,0) circle (2pt);
&
\node[style={anchor=center}] {\includegraphics[width=\colWidth]{figures/plots3/s5w0.pdf}}; %\fill[blue] (0,0) circle (2pt);
&
\node[style={anchor=center}] {\includegraphics[width=\colWidth]{figures/plots3/s0w20.pdf}}; %\fill[blue] (0,0) circle (2pt);
&
\node[style={anchor=center}] {\includegraphics[width=\colWidth]{figures/plots3/s5w20.pdf}}; %\fill[blue] (0,0) circle (2pt);

\\

& \node (xlabel1) {Force, $F^{\hat{z}_e}$ (N)}; & \node (xlabel2) {Force, $F^{\hat{z}_e}$ (N)}; & \node (xlabel3) {Force, $F^{\hat{z}_e}$ (N)}; & \node (xlabel4) {Force, $F^{\hat{z}_e}$ (N)};

\\
};
\end{tikzpicture}

\caption{For four different deformed configurations (top row), we compare the predicted and the measured forces for the parallel combination of three FREEs (bottom row). 
\revcomment{2.6}{Data points and predictions corresponding to the same input pressures are connected by a thin line, and the convex hull of the measured data points is outlined in black.}
The Trial 1 data is overlaid on top of the theoretical force zonotopes (grey areas) for each of the four configurations.
Identical colors indicate correspondence between a FREE and its resulting force/torque direction.}
\label{fig:results}
\end{figure*}






% & \node (a) {(a)}; & \node (b) {(b)}; & \node (c) {(c)}; & \node (d) {(d)};


For comparison, the measured forces are superimposed over the force zonotope generated by our model in Fig.~\ref{fig:results}a-~\ref{fig:results}d.
To quantify the accuracy of the model, we defined the error at the $j^{th}$ evaluation point as the difference between the modeled and measured forces
\begin{align}
%    \vec{e}_j &= \left( {\vec{\zeta}_{\,\text{mod}}} - {\vec{\zeta}_{\,\text{emp}}} \right)_j
%    e_j &= \left( F/M_{\,\text{mod}} - F/M_{\,\text{emp}} \right)_j
    e^F_j &= \left( F^{\hat{z}_e}_{\text{pred}, j} - F^{\hat{z}_e}_{\text{meas}, j} \right) \\
    e^M_j &= \left( M^{\hat{z}_e}_{\text{pred}, j} - M^{\hat{z}_e}_{\text{meas}, j} \right)
\end{align}
and evaluated the error across all $N = 125$ trials of a given end effector configuration.
% using the following metrics:
% \begin{align}
%     \text{RMSE} &= \sqrt{ \frac{\sum_{j=1}^{N} e_j^2}{N} } \\
%     \text{Max Error} &= \max \{ \left| e_j \right| : j = 1, ... , N \}
% \end{align}
As shown in Table \ref{table:RMSE}, the root-mean-square error (RMSE) is less than \unit[1.5]{N} (\unit[${8 \times 10^{-3}}$]{Nm}), and the maximum error is less than \unit[3]{N}  (\unit[${19 \times 10^{-3}}$]{Nm}) across all trials and configurations.

\begin{table}[H]
\centering
\caption{Root-mean-square error and maximum error}
\begin{tabular}{| c | c || c | c | c | c|}
    \hline
     & \rule{0pt}{2ex} \textbf{Disp.} & \multicolumn{2}{c |}{\textbf{RMSE}} & \multicolumn{2}{c |}{\textbf{Max Error}} \\ 
     \cline{2-6}
     & \rule{0pt}{2ex} (mm, $^\circ$) & F (N) & M (Nm) & F (N) & M (Nm) \\
     \hline
     \multirow{4}{*}{\rotatebox[origin=c]{90}{\textbf{Trial 1}}}
     & 0, 0 & 1.13 & $3.8 \times 10^{-3}$ & 2.96 & $7.8 \times 10^{-3}$ \\
     & 5, 0 & 0.74 & $3.2 \times 10^{-3}$ & 2.31 & $7.4 \times 10^{-3}$ \\
     & 0, 20 & 1.47 & $6.3 \times 10^{-3}$ & 2.52 & $15.6 \times 10^{-3}$\\
     & 5, 20 & 1.18 & $4.6 \times 10^{-3}$ & 2.85 & $12.4 \times 10^{-3}$ \\  
     \hline
     \multirow{4}{*}{\rotatebox[origin=c]{90}{\textbf{Trial 2}}}
     & 0, 0 & 0.93 & $6.0 \times 10^{-3}$ & 1.90 & $13.3 \times 10^{-3}$ \\
     & 5, 0 & 1.00 & $7.7 \times 10^{-3}$ & 2.97 & $19.0 \times 10^{-3}$ \\
     & 0, 20 & 0.77 & $6.9 \times 10^{-3}$ & 2.89 & $15.7 \times 10^{-3}$\\
     & 5, 20 & 0.95 & $5.3 \times 10^{-3}$ & 2.22 & $13.3 \times 10^{-3}$ \\  
     \hline
\end{tabular}
\label{table:RMSE}
\end{table}

\begin{figure}
    \centering
    \includegraphics[width=\linewidth]{figures/photos/buckling.pdf}
    \caption{At high fluid pressure the FREE with fiber angle of $-85^\circ$ started to buckle.  This effect was less pronounced when the system was extended along the $z$-axis.}
    \label{fig:buckling}
\end{figure}

%Experimental precision was limited by unmodeled material defects in the FREEs, as well as sensor inaccuracy. While the commercial force and moment sensors used have a quoted accuracy of 0.02\% for the force sensor and 0.2\% for the moment sensor (LoadStar Sensors, 2015), a drifting of up to 0.5 N away from zero was noticed on the force sensor during testing.

It should be noted, that throughout the experiments, the FREE with a fiber angle of $-85^\circ$ exhibited noticeable buckling behavior at pressures above $\approx$ \unit[50]{kPa} (Fig.~\ref{fig:buckling}). 
This behavior was more pronounced during testing in the non-extended configurations (Fig.~\ref{fig:results}a~and~\ref{fig:results}c). 
The buckling might explain the noticeable leftward offset of many of the points in Fig.~\ref{fig:results}a and Fig.~\ref{fig:results}c, since it is reasonable to assume that buckling reduces the efficacy of of the FREE to exert force in the direction normal to the force sensor. 

\begin{figure}
    \centering
    \includegraphics[width=\linewidth]{figures/zntp_vs_x4.pdf}
    \caption{A visualization of how the \emph{force zonotope} of the parallel combination of three FREEs (see Fig.~\ref{fig:rig}) changes as a function of the end effector state $x$. One can observe that the change in the zonotope ultimately limits the work-space of such a system.  In particular the zonotope will collapse for compressions of more than \unit[-10]{mm}.  For \revcomment{2.5}{scale and comparison, the convex hulls of the measured points from Fig.~\ref{fig:results}} are superimposed over their corresponding zonotope at the configurations that were evaluated experimentally.}
    % \marginnote{\#2.5}
    \label{fig:zntp_vs_x}
\end{figure}

	 % \vspace{-0.5em}
\section{Conclusion}
% \vspace{-0.5em}
Recent advances in multimodal single-cell technology have enabled the simultaneous profiling of the transcriptome alongside other cellular modalities, leading to an increase in the availability of multimodal single-cell data. In this paper, we present \method{}, a multimodal transformer model for single-cell surface protein abundance from gene expression measurements. We combined the data with prior biological interaction knowledge from the STRING database into a richly connected heterogeneous graph and leveraged the transformer architectures to learn an accurate mapping between gene expression and surface protein abundance. Remarkably, \method{} achieves superior and more stable performance than other baselines on both 2021 and 2022 NeurIPS single-cell datasets.

\noindent\textbf{Future Work.}
% Our work is an extension of the model we implemented in the NeurIPS 2022 competition. 
Our framework of multimodal transformers with the cross-modality heterogeneous graph goes far beyond the specific downstream task of modality prediction, and there are lots of potentials to be further explored. Our graph contains three types of nodes. While the cell embeddings are used for predictions, the remaining protein embeddings and gene embeddings may be further interpreted for other tasks. The similarities between proteins may show data-specific protein-protein relationships, while the attention matrix of the gene transformer may help to identify marker genes of each cell type. Additionally, we may achieve gene interaction prediction using the attention mechanism.
% under adequate regulations. 
% We expect \method{} to be capable of much more than just modality prediction. Note that currently, we fuse information from different transformers with message-passing GNNs. 
To extend more on transformers, a potential next step is implementing cross-attention cross-modalities. Ideally, all three types of nodes, namely genes, proteins, and cells, would be jointly modeled using a large transformer that includes specific regulations for each modality. 

% insight of protein and gene embedding (diff task)

% all in one transformer

% \noindent\textbf{Limitations and future work}
% Despite the noticeable performance improvement by utilizing transformers with the cross-modality heterogeneous graph, there are still bottlenecks in the current settings. To begin with, we noticed that the performance variations of all methods are consistently higher in the ``CITE'' dataset compared to the ``GEX2ADT'' dataset. We hypothesized that the increased variability in ``CITE'' was due to both less number of training samples (43k vs. 66k cells) and a significantly more number of testing samples used (28k vs. 1k cells). One straightforward solution to alleviate the high variation issue is to include more training samples, which is not always possible given the training data availability. Nevertheless, publicly available single-cell datasets have been accumulated over the past decades and are still being collected on an ever-increasing scale. Taking advantage of these large-scale atlases is the key to a more stable and well-performing model, as some of the intra-cell variations could be common across different datasets. For example, reference-based methods are commonly used to identify the cell identity of a single cell, or cell-type compositions of a mixture of cells. (other examples for pretrained, e.g., scbert)


%\noindent\textbf{Future work.}
% Our work is an extension of the model we implemented in the NeurIPS 2022 competition. Now our framework of multimodal transformers with the cross-modality heterogeneous graph goes far beyond the specific downstream task of modality prediction, and there are lots of potentials to be further explored. Our graph contains three types of nodes. while the cell embeddings are used for predictions, the remaining protein embeddings and gene embeddings may be further interpreted for other tasks. The similarities between proteins may show data-specific protein-protein relationships, while the attention matrix of the gene transformer may help to identify marker genes of each cell type. Additionally, we may achieve gene interaction prediction using the attention mechanism under adequate regulations. We expect \method{} to be capable of much more than just modality prediction. Note that currently, we fuse information from different transformers with message-passing GNNs. To extend more on transformers, a potential next step is implementing cross-attention cross-modalities. Ideally, all three types of nodes, namely genes, proteins, and cells, would be jointly modeled using a large transformer that includes specific regulations for each modality. The self-attention within each modality would reconstruct the prior interaction network, while the cross-attention between modalities would be supervised by the data observations. Then, The attention matrix will provide insights into all the internal interactions and cross-relationships. With the linearized transformer, this idea would be both practical and versatile.

% \begin{acks}
% This research is supported by the National Science Foundation (NSF) and Johnson \& Johnson.
% \end{acks}

\small
\bibliographystyle{ieee}
\bibliography{reference}


\end{document}
