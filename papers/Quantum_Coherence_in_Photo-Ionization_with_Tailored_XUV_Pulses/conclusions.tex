\section{Conclusion}
\label{sec:org67cfded}
In summary, we have shown that it is possible to induce coherence
between two ionic substates using pulses of duration longer than the
quantum beat time of their superposition. This is possible, provided a
resonance condition is fulfilled, namely that the driving field has at
least two frequency components spaced by precisely the energy
difference of the levels of interest. This result shows that when the
electron wave packets arising from different pathways have the same
kinetic energy, we cannot know which way the ionization occurred. This
situation is reminiscent of a Young's double slit experiment [see for
instance \textcite{Arndt2005}], with the two harmonic orders playing
the roles of the two slits. It has to be noted, however, that no
interference can be detected in the photo-electron signal, unless the
ions are brought to the same final state via some mechanism [\eg
\textcite{Goulielmakis2010} did this by further exciting from the fine
structure superposition using an XUV pulse]. Otherwise, it would be
possible to detect the ions and the photo-electrons in coincidence
mode and establishing the ionization pathway.