\section{Introduction}
\label{sec:introduction}
\begin{figure}[tb]
  \centering
  \input{figures/comp-systems.tikz}
  \caption{\label{fig:comp-systems}~Different ways of preparing
    coherent superpositions using light; (a) excitation of a two-level
    system, such as those used for qubits in quantum information; (b)
    coherent excitation from the ground state to two excited bound
    states; (c) single-photon ionization, with the ion left in a
    superposition of substates; (d) strong-field ionization, also
    leaving the ion in a superposition of substates.}
\end{figure}
The wave nature of matter is central to the quantum mechanical
description of the microcosmos; therefore coherence --- a measure of the
ability to produce stationary interference patterns --- is an important
property of any quantum system. An example of a coherent system is the
superposition of two pure states,
\(\ket{\psi} = a\ket{1} + b\ket{2}\); such superpositions form the basis
for the field of quantum information, where they are used to represent
qubits. The manipulation of qubits for quantum computing necessarily
requires that the coherence of the system is retained; if not, the
information contained within the qubit is lost. In quantum optics,
superpositions between two states may be created via a transition
between the two states with an appropriately tailored pulse (\eg a
\(\cpi/2\)-pulse) [figure~\ref{fig:comp-systems}~(a)].

\begin{figure}[tb]
  \begin{floatrow}
    \floatbox{table}[][][t]{\caption{\label{tab:quantum-beat-periods}
        Some properties of heavy noble gases.
        \(\Ddiff E_{\textrm{s--o}}\) is the spin--orbit splitting of the
        ionic ground state \(n\conf{p}^5\;\conf{^2P^o}\). The quantum
        beat period, \(T (=2\cpi\Ddiff E_{\textrm{s--o}}^{-1}\)), is
        the intrinsic atomic clock associated to two states separated
        by an energy difference of \(\Ddiff E_{\textrm{s--o}}\).}}{
      \footnotesize
      \begin{tabular}{rrrlr}
        Element & \(Z\) & \(n\)
        & \(\Ddiff E_{\textrm{s--o}}\) [\si{\electronvolt}]
        & \(T\) [\si{\femto\second}]\\
        \hline
        Ne & 10 & 2 & 0.09676024 & 42.8\\
        Ar & 18 & 3 & 0.17749368  & 23.3\\
        Kr & 36 & 4 & 0.665808  & 6.2\\
        Xe & 54 & 5 & 1.306423 & 3.2\\
      \end{tabular}
    }
    \floatbox{figure}[][][t]{\caption{\label{fig:quantum-path-overlap}
        Overlap (shaded area) between two quantum paths (1 and 2)
        separated by \(\Ddiff E\). The spectral bandwidth, \(\Omega\), of
        the ionizing pulse is inversely proportional to the duration
        of the pulse; increasing the pulse duration thus leads to
        decreased overlap between the quantum
        paths. %For pulse durations on the
        %Heisenberg limit (\(\Ddiff t\) in
        %table~\ref{tab:quantum-beat-periods}), there is still
        %appreciable spectral overlap between quantum paths; for pulse
        %durations approaching the quantum beat period (\(T\) in the
        %same table), the spectral overlap is negligible, which leads
        %to dephasing and decoherence.
        }} {\input{figures/coherence-uncertainty.tikz}}
  \end{floatrow}
\end{figure}
Superpositions of states can also be achieved by direct excitation
using short light pulses [figure~\ref{fig:comp-systems}~(b)], provided
the bandwidth of the pulse is larger than the energy difference
(\(\omega_{21}\)) between the two states.  This requires a pulse duration
short enough, \(\tau \leq {2\cpi}/{\omega_{21}}\). If the superposition is
successfully created, it may be observed through quantum beats
\parencite{Teets1977,Salour1977,Mauritsson2010PRL,Tzallas2011NP} which
usually last substantially longer than the pulse duration. The
characteristic decay time is termed the \emph{coherence time}. In the
cases depicted in \ref{fig:comp-systems}~(a) and (b), the light
couples the bound states and enables coherent population transfer.

Another way to produce a superposition of states is via short-pulse
ionization, when the ion is left in a coherent superposition of final
states, \eg due to spin--orbit interaction. This can be done using
either high-frequency [figure~\ref{fig:comp-systems}~(c)] or
high-intensity short-pulse [figure~\ref{fig:comp-systems}~(d)]
radiation. As previously, the bandwidth of the ionizing pulse has to
exceed the energy splitting between the ion
states. \textcite{Kurka2009} investigated case \ref{fig:comp-systems}
(c) by photo-ionizing neon using short XUV pulses from a free-electron
laser. A coherent superposition of the ionic fine-structure substates
was prepared and probed by subsequent ionization. Using a strong laser
field [figure~\ref{fig:comp-systems}~(d)], \textcite{Goulielmakis2010}
photo-ionized krypton, leaving the residual Kr\(^+\) ion in a coherent
superposition of the ionic substates. The quantum beat was observed by
probing with a delayed attosecond (\si{\atto\second}) XUV pulse, as
long as the duration of the ionizing pulse was shorter than
the quantum beat period of the Kr\(^+\) ion.
This experimental activity stimulated an important theoretical effort
to investigate the coherence of superpositions of states produced
either directly by photo-excitation [figure~\ref{fig:comp-systems}
(b); \cite{Tzallas2011NP,Klunder2013PRA}], single-photon ionization
[figure \ref{fig:comp-systems}~(c); \cite{Nikolopoulos2013PRL}], or
strong-field ionization [figure~\ref{fig:comp-systems}~(d);
\cite{Pabst2011PRL,Pabst2016}].

As mentioned above, the creation of a coherent superposition in the
cases depicted in figures~\ref{fig:comp-systems}~(b--d), requires
sufficient bandwidth of the exciting/ionizing radiation. As the pulse
duration increases, or the energy separation between electronic states
increases, states become spectrally resolvable. Excitation/ionization
to one state or the other can then be seen as distinguishable quantum
paths taken by the system (see
figure~\ref{fig:quantum-path-overlap}). When the spectral overlap
between these quantum paths decreases, the coherence between states
diminishes. While a spectral representation provides meaningful
physical insight, it does not allow understanding how coherence is
built-up in real time. Therefore, a temporal representation, based on
an atomic clock constituted by the quantum beat period, is very useful
to determine whether decoherence will occur or not during light--matter
interaction. As long as the pulse duration is shorter than a quantum
beat period, the atomic clock will not dephase, since there is still
appreciable overlap between the quantum paths (see
figure~\ref{fig:quantum-path-overlap}). For ionization of noble gases,
the available quantum beat periods span an order of magnitude (see
table~\ref{tab:quantum-beat-periods}). However, ultrashort pulses are
still necessary to manipulate the coherence.

In this article, we present a theoretical study of single-photon
ionization of xenon atoms using XUV pulses, tailored in such a way
that the quantum paths \emph{always} overlap spectrally, providing a
new ionization scheme to control and maximize the coherence between
states, independent of the pulse duration. This is achieved by
employing a bichromatic or multi-colour ionizing field, consisting of
phase-locked harmonics such as those resulting from high-order
harmonic generation (HHG), provided the spectral components fulfill a
certain resonance condition. We investigate the tolerance of this
resonance condition, \ie how strict the requirements on the driving
field are, with respect to the excitation frequencies, pulse duration,
and temporal structure, for maintaining a certain level of
coherence. We find the existence of resonant conditions, which
correspond to a situation where multiple quantum paths lead to the
same photo-electron state.

The paper is organized as follows; in the following section, we
present our numerical calculations based on the fully correlated
time-dependent Schrödinger equation (TDSE) in the case of weak-field
ionization, as well as the theoretical tools that are used to
calculate the evolution of the superposition of states in the presence
of the driving field. We also introduce the density matrix formalism
used to analyze the coherence of the quantum system. Atomic units are
used throughout, unless otherwise stated. Using the numerical
calculations, in section~\ref{sec:bichromatic}, we study ionization
using two harmonic components, and investigate the relation between
their energy separation and the spin--orbit splitting using a temporal
model of the ionization dynamics. From this, we extract a generalized
quantum beat condition. Finally in section~\ref{sec:multi-colour}, we
derive a spectral model for the general case of ionization by
multi-colour fields. This enables us to capture the essential physics
of the observed phenomena. We conclude with a short discussion about
the foundation of the work in relation to quantum mechanics,
suggesting that coherence between states should exist as long as
ionization pathways are indistinguishable by the measurement.
