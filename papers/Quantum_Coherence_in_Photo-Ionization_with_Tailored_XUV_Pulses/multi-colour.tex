\section{The multi-colour case --- Ionization by an attosecond pulse
  train}
\label{sec:multi-colour}
We consider now the effect of ionization with an attosecond pulse
train, by including additional harmonic components. To focus on this
aspect of the problem, we use a simplified model, where the dipole
matrix elements for ionization are replaced with Heaviside functions
(this is an approximation of a flat continuum, \ie no resonances
present):
\begin{equation}
  z(\varepsilon) = \theta(\varepsilon),
\end{equation}
where \(\varepsilon\), as before, is the energy of the continuum electron. We
still use the \emph{Ansatz} \eqref{eqn:expansion}, albeit with a
compact notation only considering different channels \(n\):
\begin{equation}
  \label{eqn:expansion-simplified}
  \ket{\Psi(t)}
  =
  c_0(t)\ket{\Psi_0}
  +\sum_n \int \diff{\varepsilon}
  c_n(t;\varepsilon)\ket{n\varepsilon}.
\end{equation}
Inserting this in the Schrödinger equation and applying first-order
time-dependent perturbation theory (\ie the ground state is unaffected
by the weak-field ionization; \(c_0(t)=1\)), the solution reads
\begin{equation}
  \label{eqn:spectral-model-solution}
  c_n(t)
  =-\im\theta(\varepsilon)
  \int_{-\infty}^t\diff{t'}
  \mathcal{E}(t')
  \exp(\im E_n t').
\end{equation}
Evaluating at the time of measurement (\(t=+\infty\)), we see that the
coefficient becomes the Fourier transform of the driving field,
evaluated at \(-E_n - \varepsilon\). From this, we get the coherence between two
channels \((m,n)\) as
\begin{equation}
  \label{eqn:simplified-coherence}
  \rho_{mn}
  =-\int\diff{\varepsilon}\theta(\varepsilon)
  \conj{\hat{\mathcal{E}}}(E_m+\varepsilon)
  \hat{\mathcal{E}}(-E_n-\varepsilon),
\end{equation}
where \(\hat{\mathcal{E}}(\omega)\) designates the Fourier transform of
\(\mathcal{E}(t)\). This expression shows that the coherence primarily
arises from a correlation of the field with itself shifted by the
energy difference \(\Ddiff E \equiv E_m-E_n\).

\begin{figure}[htb]
  \centering
  \includegraphics{figures/five-harms-wide.pdf}
  \caption{The effect of using five harmonics: At \(d=1\), 4 harmonics
    out of 5 will yield photo-electron peaks that overlap with those
    shifted by the spin--orbit splitting. The peaks appearing at
    \(d=\sfrac{1}{4},\sfrac{1}{3},\sfrac{1}{2}\) correspond to 1, 2,
    and 3 harmonics resulting in overlapping photo-electron peaks,
    respectively. In the time domain, this corresponds to ionizing
    pulses arriving at every \(r=4,3,2\) quantum beat periods.}
  \label{fig:multiple-harms}
\end{figure}
In the long-pulse limit, the spectral components of
\(\hat{\op{E}}(\omega)\) will become very narrow. If the ionizing field is
an attosecond pulse train, consisting of the \(n_q = q_2-q_1+1\)
successive harmonic orders \(q\in\{q_1..q_2\}\), the shift has to be
precisely an integer amount of photons (\(\Delta E=N\omega_0\)) for the
integrand of \eqref{eqn:simplified-coherence} to be non-zero. Assuming
\(q_1\omega_0, q_2\omega_0>E_m,E_n\), \ie all constituent harmonic orders reach
above both ionization thresholds, we have in total \(2n_q\) pathways
into the continuum. In the case of an integer photon shift,
\(2(n_q-N)\) of these pathways will overlap, which means
\begin{equation}
  \label{eqn:multi-colour-coherence}
  \tilde{\rho}_{mn}=\frac{n_q-N}{n_q}.
\end{equation}
From \eqref{eqn:multi-colour-coherence}, we see that by adding more
and more colours, we can increase the degree of coherence towards
unity. This is illustrated in figure \ref{fig:multiple-harms} for the
case of five harmonics; the maximum degree of coherence is indeed
\sfrac{4}{5}, which occurs at \(d=1\).

Figure~\ref{fig:multiple-harms} also serves the purpose of
illustrating the generalization of the quantum beat condition
\eqref{eqn:quantum-beat-condition}; \eg the case \(r=2\) corresponds
in the time domain to ionizing pulses arriving every other quantum
beat period. Such a field can be realized through red--blue HHG
(resulting in odd \emph{and} even harmonic orders) from a fundamental
frequency \(\Ddiff E_{\textrm{s--o}}/2\) (for Xe with
\(\Ddiff E_{\textrm{s--o}}\approx\SI{1.3}{\electronvolt}\), a fundamental
driving wavelength \(\lambda\approx\SI{1.9}{\micro\meter}\) would thus be
necessary). From this we see that in the spectral domain, the
generalized quantum beat condition \eqref{eqn:quantum-beat-condition}
is simply \(d=\sfrac{1}{r}\), \(r\in\Integer\).

Finally, we have also checked that in the long-pulse limit, the case
of \(d=\sfrac{1}{2}\), using only odd-order harmonics is no different
from the case of \(d=1\), using even- and odd-order harmonics. Thus
the change of parity (phase difference) of consecutive pulses in the
attosecond pulse train has no impact on the coherence, and we conclude
that the degree of coherence is essentially dictated by the amount of
indiscernible pathways.