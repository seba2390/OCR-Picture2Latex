\section{Theoretical framework}
\label{sec:theoretical-framework}
\noindent We are interested in studying the coherence of different
ionic states produced by photo-ionization with tailored XUV
pulses. To this end, we use as a model system noble gas ions, which
have a spin--orbit splitting of the ground state
(\(n\conf{p^5\;^2P}^\conf{o}_{j_i}\),
\(j_i=\sfrac{3}{2},\sfrac{1}{2}\)), in particular xenon (\(n=5\)).

\begin{figure}[tb]
  \begin{floatrow}
    \floatbox{table}[\FBwidth][][t]{\caption{\label{tab:ion-channels}
        Ionization channels accessible via one-photon ionization from
        the valence shell of a noble gas (final \(J=1\)), in the case
        of \(jK\) coupling.}}{
      \begin{tabular}{r|rl}
        \# & Channel configuration\\
        \hline
        1 & \(n\conf{p}^5(\conf{^2P^o_{\sfrac{3}{2}}})k\conf{d}\;^2[\sfrac{1}{2}]_{1}\)\Tstrut\\
        2 & \(n\conf{p}^5(\conf{^2P^o_{\sfrac{3}{2}}})k\conf{s}\;^2[\sfrac{3}{2}]_{1}\)\\
        3 & \(n\conf{p}^5(\conf{^2P^o_{\sfrac{3}{2}}})k\conf{d}\;^2[\sfrac{3}{2}]_{1}\)\\
        \hline
        4 & \(n\conf{p}^5(\conf{^2P^o_{\sfrac{1}{2}}})k\conf{s}\;^2[\sfrac{1}{2}]_{1}\)\Tstrut\\
        5 & \(n\conf{p}^5(\conf{^2P^o_{\sfrac{1}{2}}})k\conf{d}\;^2[\sfrac{3}{2}]_{1}\)\\
        \hline
        6 & \(n\conf{s}n\conf{p}^6(\conf{^2S_{\sfrac{1}{2}}})k\conf{p}\;^2[\sfrac{1}{2}]_{1}\)\Tstrut\\
        7 & \(n\conf{s}n\conf{p}^6(\conf{^2S_{\sfrac{1}{2}}})k\conf{p}\;^2[\sfrac{3}{2}]_{1}\)\\
      \end{tabular}}
    \floatbox{figure}[\Xhsize][][t]{\caption{\label{fig:sketch} Schematic
        energy diagram of a noble gas (heavier than He, \ie with a
        spin--orbit splitting of the first ionic ground state
        \(n\conf{p^5\;^2P}_{j_i}\), \(j_i=\sfrac{3}{2},\sfrac{1}{2}\))
        photo-ionized with a tailored XUV pulse consisting of two
        frequencies with \(\Omega_>-\Omega_<=\omega_0\). The energy scale is that of
        the photo-electron kinetic energy, which depends on the final
        ion state, \(\conf{^2P_{\sfrac{3}{2}}}\) or
        \(\conf{^2P_{\sfrac{1}{2}}}\).  It can be seen from the
        diagram that there are four pathways to the continuum. If
        \(\omega_0\approx\Ddiff E_{\textrm{s--o}}\), two of the quantum paths
        (absorption of \(\Omega_<\) and the ion in
        \(\conf{^2P_{\sfrac{3}{2}}}\); absorption of \(\Omega_>\) and the
        ion in \(\conf{^2P_{\sfrac{1}{2}}}\)) lead to the same
        photo-electron energy.}}{\input{figures/sketch.tikz}}
  \end{floatrow}
\end{figure}
Figure \ref{fig:sketch} shows a simplified diagram of photo-ionization
of a \(n\conf{p}\) electron. The ionic ground state has a spin--orbit
splitting, which in xenon is \SI{1.3}{\electronvolt}. We ionize with a
weak XUV pulse with two frequency components, whose difference is
\(\omega_0\). Absorption of the two frequency components, \(\Omega_>\) and
\(\Omega_<\), leads to an ion in either
\(\conf{^2P^{o}_{\sfrac{3}{2}}}\) or
\(\conf{^2P^{o}_{\sfrac{1}{2}}}\), resulting in four different
pathways. If the frequency difference is equal to the spin--orbit
spacing, there will be two (indistinguishable) pathways to the same
final photo-electron energy; we call this the \emph{resonant case}. We
introduce the \emph{detuning ratio}
\(d\equiv\omega_0/\Ddiff E_{\textrm{s--o}}\), and study photoionization in the
vicinity of this resonance (\(d\approx1\)).

The calculations are performed by solving the time-dependent
Schrödinger equation (TDSE) in a limited subspace,
\begin{equation}
  \label{eqn:tdse}
  \im\partial_t\ket{\Psi(t)} = \Ham(t)\ket{\Psi(t)},
\end{equation}
where the Hamiltonian in the dipole approximation is
\begin{equation}
  \label{eqn:hamiltonian}
  \Ham(t) = \Ham_0 + \op{E}(t)z.
\end{equation}
\(\Ham_0\) is the atomic Hamiltonian, \(\op{E}(t)\) the electric
field, and \(z\) is the dipole operator for linear polarization
along the \(z\) axis. The solution is found by propagating the
initial state (the neutral ground state) to time \(t\)
\begin{equation}
  \ket{\Psi(t)} = \op{U}(t,0)\ket{\Psi_0},
\end{equation}
where the short-time propagator \(\op{U}(t+\Ddiff t,t)\) is
approximated by a \textcite{Magnus1954} propagator of fourth order
\parencite{Saad1992SJoNA,Alvermann2012}.

The time-dependent wavefunction is expanded as
\begin{equation}
  \label{eqn:expansion}
  \ket{\Psi(t)} =
  c_0(t)\ket{\Psi_0} +
  \sum_{i}\sum_{\ell}\int\diff{\varepsilon}
  c_{i}^{\ell}(t;\varepsilon)
  \ket{i \varepsilon \ell},
\end{equation}
where \(\ket{\Psi_0}\) is the ground state
\(n\conf{s^2}n\conf{p^6\;^1S_0}\) with energy \(-\Ip\), \(c_0(t)\) the
complex, time-dependent amplitude, \(i\) denotes the final state of
the ion, and \(\varepsilon \ell\) the quantum state of the photo-electron with
angular momentum \(\ell\) and energy \(\varepsilon\) (related to the momentum
\(k\) by \(\varepsilon=k^2/2\)).  The ionization channels formed by different
possible combinations of \(i\) and \(\ell\), are listed in table
\ref{tab:ion-channels}, in the case of \(jK\) coupling. \(jK\) (or
pair) coupling is defined as \parencite{Cowan1981}
\(\vec{j}_i + \Bell = \vec{K}\) and \(\vec{K} + \vec{s} = \vec{J}\),
where \(\vec{j}_i\) is the total angular momentum of the parent ion,
which couples to the angular momentum of the electron \(\Bell\) to
form an intermediate \(\vec{K}\). The levels are then written as
\(\gamma_i(^{2S+1}L_{j_i})k\ell\;^{2S+1}[K]_J\), where \(\gamma_i\) is the electron
configuration of the ion.

The \emph{Ansatz} \eqref{eqn:expansion} turns the
TDSE~\eqref{eqn:tdse} into a set of coupled ordinary differential
equations (ODE):
\begin{equation}
  \label{eqn:tdse-expansion}
  \im\partial_t\vec{c}(t) = \mat{H}(t)\vec{c}(t),
  \tag{\ref{eqn:tdse}*}
\end{equation}
where the vector \(\vec{c}(t)\) consists of the expansion coefficients
in \eqref{eqn:expansion}, and the \emph{Hamiltonian matrix} is given
by
\begin{equation}
  \begin{aligned}
    \mat{H}(t) =&
    -\ketbra[\Ip]{\Psi_0}{\Psi_0}
    + \sum_{i\ell}\int\diff{\varepsilon}
    \ketbra[\varepsilon]{i\varepsilon\ell}{i\varepsilon\ell}\\
    &+
    \op{E}(t)\left[
      \sum_{i\ell}\int\diff{\varepsilon}
      \ket{i\varepsilon\ell}\matrixel{i\varepsilon\ell}{z}{\Psi_0}\bra{\Psi_0}
      +
      \sum_{i'\ell'}\int\diff{\varepsilon'}
      \ket{i\varepsilon\ell}\matrixel{i\varepsilon\ell}{z}{i'\varepsilon'\ell'}\bra{i'\varepsilon'\ell'}
      +\cc
    \right].
  \end{aligned}
  \label{eqn:hamiltonian-expansion}
  \tag{\ref{eqn:hamiltonian}*}
\end{equation}
In the field-free basis, \(\mat{H}_0\) is simply a diagonal matrix,
with the energies of the photo-electron with respect to the lowest
ionization threshold as matrix elements. The interaction term couples
the ground state to the continua and the continua to each other. In
the weak-field limit, however, the partial-wave expansion is
restricted to total angular momentum \(J\leq1\), \ie no multi-photon
processes are considered. Furthermore, ionization is only allowed from
the outer \(n\conf{p}\) shell (photo-electron energies in the range
\SIrange{0}{11}{eV} in the case of xenon), to avoid autoionization of
embedded Rydberg states in the vicinity of the
\(n\conf{s}n\conf{p^6\;^2S_{\sfrac{1}{2}}}\) threshold (that is,
channels 6 and 7 in table~\ref{tab:ion-channels} need not be
considered). We also neglect mixing of singlet and triplet
terms. Thus, the only non-zero matrix elements of the dipole operator
\(z\) are \(\matrixel{i \varepsilon \ell}{z}{\Psi_0}\) (and the complex
conjugate). The basis functions (\(\ket{\Psi_0}\) and
\(\ket{i\varepsilon\ell}\)) and the dipole matrix elements
(\(\matrixel{i \varepsilon \ell}{z}{\Psi_0}\)) are determined using \program{ATSP2K}
(multi-configurational Hartree--Fock; \cite{FroeseFischer2007CPC}) and
\program{BSR} (close-coupling \(R\)-matrix approach;
\cite{Zatsarinny2006,Zatsarinny2009}). The dipole matrix elements are
spin-averaged by \program{BSR}.

The analysis of the coherence is made using the density matrix
formalism [\cite[§14]{Landau1977quant}], where the full \emph{density
  matrix} operator is formed from the wavefunction \(\ket{\Psi(t)}\)
obtained by solving \eqref{eqn:tdse} (time dependence \(t\) omitted
below, for brevity)
\begin{equation}
  \label{eqn:density-matrix-operator}
  \rho_T = \ketbra{\Psi}{\Psi},
\end{equation}
with matrix elements of the continuum block
\begin{equation}
  \label{eqn:continuum-densities}
  \rho_{i_1i_2}^{\ell_1\ell_2}(\varepsilon_1,\varepsilon_2) \equiv
  c_{i_1}^{\ell_1}(\varepsilon_1)
  c_{i_2}^{\ell_2*}(\varepsilon_2).
\end{equation}
We reduce this density matrix\index{reduced density matrix} to an
ion--channel density matrix by first taking the trace over the
photo-electron energy \(\varepsilon\):
\begin{equation}
  \label{eqn:ion-channel-density-matrix}
  \begin{aligned}
    \rho_{i_1i_2}^{\ell_1\ell_2}&\equiv
    \int\diff{\varepsilon}
    \braket{\varepsilon}{\Psi}\braket{\Psi}{\varepsilon}
    =\int\diff{\varepsilon}
    \sum_{i_1i_2}\sum_{\ell_1\ell_2}
    \int\diff{\varepsilon_1}\diff{\varepsilon_2}
    \braket{\varepsilon}{i_1\varepsilon_1\ell_1}
    c_{i_1}^{\ell_1}(\varepsilon_1)c_{i_2}^{\ell_2*}(\varepsilon_2)
    \braket{i_2\varepsilon_2\ell_2}{\varepsilon}\\
    &=
    \sum_{i_1i_2}\sum_{\ell_1\ell_2}
    \int\diff{\varepsilon}
    \ket{i_1\ell_1}
    \rho_{i_1i_2}^{\ell_1\ell_2}(\varepsilon,\varepsilon)
    \bra{i_2\ell_2}.
  \end{aligned}
\end{equation}
Finally, we construct the ion density matrix by tracing over the
photo-electron angular momenta:
\begin{equation}
  \label{eqn:ion-density-matrix}
  \begin{aligned}
    \rho_{i_1i_2}&\equiv
    \sum_\ell
    \sum_{i_1i_2}\sum_{\ell_1\ell_2}
    \braket{\ell}{i_1\ell_1}
    \rho_{i_1i_2}^{\ell_1\ell_2}
    \braket{i_2\ell_2}{\ell}
    =
    \sum_\ell
    \sum_{i_1i_2}
    \ket{i_1}
    \rho_{i_1i_2}^{\ell\ell}
    \bra{i_2}.
  \end{aligned}
\end{equation}
The diagonal elements (\(\rho_{mm}\)) of this matrix are the populations
in each of the ionic states, while the off-diagonal elements
(\(\rho_{mn}\)) contain the coherences between the ionic states. The only
non-zero off-diagonal elements are those corresponding to channels for
which all quantum numbers are the same (except for the angular
momentum of the ion); \ie only \(\rho_{35} = \conj{\rho}_{53}\neq0\).
Decoherence due to decay (through dipole-forbidden interaction) from
the upper ionic state to the lower, is neglected.