\section{The bichromatic case}
\label{sec:bichromatic}
\begin{figure}[tb]
  \centering
  \includegraphics{figures/real_time_resonant.pdf}
  \caption{\label{fig:real-time-coherence} Real-time coherence
    build-up, for the case of ionization with two harmonics, 13 and 14
    of a fundamental frequency \(d\Ddiff E_{\textrm{s--o}}\), in the
    non-resonant case (\(d=1.3\)) left and the resonant case (\(d=1\))
    right. The upper panels show the driving fields used; the red
    curve corresponds to a pulse duration (FWHM of the temporal
    intensity profile) of \SI{500}{\atto\second} while the black curve
    corresponds to a pulse duration of \SI{15}{\femto\second}. The
    middle panels show the populations in the residual ionic substates
    (solid: contribution from channel 3 to
    \(\conf{^2P^o_{\sfrac{3}{2}}}\); dashed: contribution from channel
    5 to \(\conf{^2P^o_{\sfrac{1}{2}}}\)), which increase with
    time. The lower panels show the induced coherence between the
    ionic substates, which is built up over time. For the short-pulse
    case, there is always coherence left at the end of the pulse,
    while for the longer pulse duration, the resonance criterion has
    to be fulfilled \(d\approx1\) for this to happen. The lower population
    in the non-resonant case is explained by the decrease in
    photoionization cross-section with increasing photon energy.}
\end{figure}
We investigate the real-time build-up of coherence with a XUV pulse of
short or long duration in the non-resonant case
(figure~\ref{fig:real-time-coherence}, left) and the resonant case
(figure~\ref{fig:real-time-coherence}, right). The electromagnetic
fields are presented in the upper panels. In both cases, they consist
of harmonics 13 \& 14 of a fundamental driving field, and a peak
intensity of \SI{e8}{\watt\per\centi\meter\squared}. The short-pulse
duration is \SI{500}{\atto\second}, while the long-pulse duration is
\SI{15}{\femto\second} resulting in the formation of a periodic
beating of the XUV pulse. Regardless of the pulse duration, the
population in the ionic substates (middle shown in solid and dashed
lines) increases as the pulse ionizes the atom. The population is
proportional to the integral of the pulse intensity, hence the
appearance of steps in the population.

\subsection{The non-resonant case.}
\label{sec:bichromatic-non-resonant}
We first consider the non-resonant case (left panels of figure
\ref{fig:real-time-coherence}), where the fundamental driving
frequency is \(\omega_0=1.3\Ddiff E_{\textrm{s--o}}\). For a short pulse
duration, the coherence increases during the interaction and stays
constant after the pulse has passed. In contrast, for long pulse
duration, the coherence first builds up, and then vanishes at the end
of the pulse. The decoherence time, \ie the time from the onset of the
pulse to the decrease of the coherence (see lower panel of figure
\ref{fig:real-time-coherence}), is approximately equal to the quantum
beat period \(T=2\cpi/\Ddiff E_{\textrm{s--o}}\) of the ionic
substates; for xenon with a spin--orbit splitting of
\SI{1.3}{\electronvolt}, it is \SI{3.2}{\femto\second}. This is
similar to what was observed by \textcite{Goulielmakis2010}, where the
interference pattern, present in the transient absorption signal,
disappeared when the ionizing infrared pulse exceeded the quantum beat
period of \SI{6.2}{\femto\second} of krypton.

\subsection{The resonant case. A generalized quantum beat condition.}
\label{sec:orgd61bd89}
In the resonant case (right panels of figure
\ref{fig:real-time-coherence}), the situation is completely
different. The coherence is built up during all the interaction time,
and remains after the end of the pulse. We can understand the
existence of this resonance condition, using a simple
two-excited-state model. The two excited states (labelled \(\ket{1}\)
and \(\ket{2}\), for simplicity) correspond to the ionic substates,
the populations of which result from a periodic sequence of ionization
events (occurring at times \(t_k\), \(k\in\Natural\)) from the ground
state (labelled \(\ket{0}\)). We express this model using an
inhomogeneous TDSE for the excited superposition
\(\ket{\Psi(t)}=c_1(t)\ket{1}+c_2(t)\ket{2}\), where the ground state
constitutes a source term:
\begin{equation}
  \im\partial_t\ket{\Psi(t)} = \mat{H}_0\ket{\Psi(t)} +
  \mathcal{E}(t)(z_{10}\ket{1}+z_{20}\ket{2}),
  \quad
  \mat{H}_0\equiv
  E_1\ketbra{1}{1}+E_2\ketbra{2}{2},
  \quad
  z_{i0}\equiv\matrixel{i}{z}{0}.
  \label{eqn:two-level-system}
\end{equation}
Assuming no initial excited population, the solution of
\eqref{eqn:two-level-system} is given by
\begin{equation}
  \begin{aligned}
    \ket{\Psi(t)}
    &=
    \exp(\im\mat{H}_0t)
    \left\{
      \int^t\diff{t'}
      \op{E}(t')
      [\ce^{-\im E_1t'}z_{10}\ket{1}
      +\ce^{-\im E_2t'}z_{20}\ket{2}]
    \right\}
  \end{aligned}
  \label{eqn:two-level-solution}
\end{equation}
If we assume that \(\op{E}(t)\) is a train of pulses, separated by
\(\Ddiff t\equiv t_k - t_{k-1}\), we can write the field as
\[
  \op{E}(t) =
  \tilde{\op{E}}(t)\left[
    \sum_k\delta(t_k)\ce^{\im\phi_{\textrm{XUV}}(t_k)}
  \right],
\]
where \(\tilde{\op{E}}(t)\) is a slowly varying envelope. The solution
to the two-excited-state system in this case is
\begin{equation}
  \begin{aligned}
    \ket{\Psi(t)}
    &=
    \exp(\im\mat{H}_0t)
    \sum_k
    \tilde{\op{E}}(t_k)
    \ce^{\im\phi_{\textrm{XUV}}(t_k)}
    \left[
      \ce^{-\im E_1t_k}z_{10}\ket{1}
      +\ce^{-\im E_2t_k}z_{20}\ket{2}
    \right]\\
    &=
    \sum_k
    \tilde{\op{E}}(t_k)
    \ce^{\im\phi_{\textrm{XUV}}(t_k)}
    \left[
      \ce^{-\im E_1(t_k-t)}z_{10}\ket{1}
      +\ce^{-\im E_2(t_k-t)}z_{20}\ket{2}
    \right]\\
    &=
    \sum_k
    \tilde{\op{E}}(t_k)
    \ce^{\im[\phi_{\textrm{XUV}}(t_k) - E_1(t_k-t)]}z_{10}
    \left[
      \ket{1}
      +\tilde{z}\ce^{-\im \Ddiff E (t_k-t)}\ket{2}
    \right],
  \end{aligned}
  \label{eqn:two-level-solution-train}
  \tag{\ref{eqn:two-level-solution}*}
\end{equation}
where \(\tilde{z}\equiv\matrixel{2}{z}{0}/\matrixel{1}{z}{0}\) is the
relative dipole matrix element independent of the instant of
ionization, and --- assuming ionization solely into an unstructured
continuum --- independent of final electron energy and
\(\Ddiff E \equiv E_2 - E_1\), as well.

For the two substates to remain coherent, we require that no dephasing
is introduced by pulses in the train. Since subsequent pulses are
separated in time by \(\Ddiff t\), this is equivalent to requiring
that the phase argument in \eqref{eqn:two-level-solution-train}
fulfills \(\Ddiff E(t_k-t_a)=\Ddiff E\Ddiff t(k-a)=2\cpi\), \(k,a\in\Natural\), or
\begin{equation}
  \label{eqn:quantum-beat-condition}
  \Ddiff t = \frac{2\cpi r}{\Ddiff E}=rT,
\end{equation}
is fulfilled, for any integer \(r\). \(\Ddiff t\) is a multiple of the
quantum beat period for an energy separation \(\Ddiff E\), which does
not depend on the duration of the XUV pulse. In the spectral domain,
this corresponds to requiring the final electron kinetic energy to be
the same.

\subsection{Maximum coherence achievable. Degree of coherence.}
\label{sec:maximum-coherence}
\begin{figure}[tb]
  \centering
  \includegraphics{figures/Xe-duration-detuning-coh-map.pdf}
  \caption{\label{fig:duration-detuning-map} Degree of coherence as a
    function of pulse duration (FWHM of temporal intensity profile)
    and detuning ratio \(d\). The hyperbolas in the main panel
    indicate the `coherence bandwidth' within which
    \(\tilde{\rho}_{35}>(2\sqrt{\ce})^{-1}\). The black
    horizontal/vertical lines mark lineouts at constant
    duration/detuning ratio, which are shown in the lower/left panels.
    As the pulse duration exceeds the quantum beat period, the peaks
    become spectrally resolvable, and the total overlap at the central
    frequency becomes half of that at vanishing pulse duration. The
    peaks appearing below \(d\approx0.9\) occur when the lower harmonic
    excites autoionizing Rydberg states between
    \(\conf{^2P_{\sfrac{3}{2}}}\) and \(\conf{^2P_{\sfrac{1}{2}}}\)
    thresholds.}
\end{figure}
As seen in figure~\ref{fig:real-time-coherence}, given that the
resonance condition \(d=1\) is fulfilled, the coherence between the
ionization channels seem to increase as long as the pulse is of
appreciable amplitude. What is the maximum coherence achievable using
this scheme? The theoretical maximum coherence
\(\abs{\rho_{mn}}=\abs{c_m\conj{c_n}}\), is bounded by the
Cauchy--Schwartz inequality:
\begin{equation}
  \label{eqn:cauchy--schwartz}
  \abs{\rho_{mn}}^2 \leq \rho_{mm}^2\rho_{nn}^2,
\end{equation}
where \(\rho_{ii}\) is the probability of finding the system in state
\(\ket{i}\), which obviously cannot exceed unity. From
\eqref{eqn:cauchy--schwartz}, it is natural to introduce the
\emph{degree of coherence}:
\begin{equation}
  \label{eqn:degree-of-coherence}
  \tilde{\rho}_{mn} \equiv \frac{\abs{\rho_{mn}}}{\sqrt{{\rho_{mm}}{\rho_{nn}}}},
\end{equation}
which normalizes the coherence between two ions to their respective
populations. This quantity is useful since, even though the
populations in two states are minuscule (as is the case in
figure~\ref{fig:real-time-coherence}) and thereby also the absolute
coherence, they may be fully coherent \emph{with respect to each
  other}. If this is the case, the degree of coherence will be
unity.

Figure~\ref{fig:duration-detuning-map} shows the degree of coherence,
as a function of XUV pulse duration and detuning ratio. For short
pulse durations, this quantity is larger than \sfrac{1}{2} (left panel
of figure \ref{fig:duration-detuning-map}). In this regime, the
interaction with the XUV pulse occurs within one quantum beat period
(\SI{3.2}{\femto\second}), and the four pathways into the continuum
(indicated in figure \ref{fig:sketch}) have a partial spectral
overlap. For larger pulse durations, two of the pathways become
distinguishable, and do not contribute to the coherence between the
ionic substates. The two remaining pathways, namely via \(\Omega_<\)
leaving the ion in \(\conf{^2P^{o}_{\sfrac{3}{2}}}\), and via
\(\Omega_>\) leaving the ion in \(\conf{^2P^{o}_{\sfrac{1}{2}}}\), cannot
be distinguished when measuring the photo-electron. Provided the
resonance condition \(d=1\) is met, the maximum degree of coherence is
\sfrac{1}{2}. Complete decoherence always occurs in the long-pulse
limit if \(d\neq1\) since the quantum pathways are distinguishable. As we
will see below, the maximum degree of coherence can be increased by
adding more colours to the ionizing field.
