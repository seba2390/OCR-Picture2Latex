%% 
%% Copyright 2019-2020 Elsevier Ltd
%% 
%% This file is part of the 'CAS Bundle'.
%% --------------------------------------
%% 
%% It may be distributed under the conditions of the LaTeX Project Public
%% License, either version 1.2 of this license or (at your option) any
%% later version.  The latest version of this license is in
%%    http://www.latex-project.org/lppl.txt
%% and version 1.2 or later is part of all distributions of LaTeX
%% version 1999/12/01 or later.
%% 
%% The list of all files belonging to the 'CAS Bundle' is
%% given in the file `manifest.txt'.
%% 
%% Template article for cas-dc documentclass for 
%% double column output.

%\documentclass[a4paper,fleqn,longmktitle]{cas-dc}
\documentclass[a4paper,fleqn]{cas-dc}

%\usepackage[authoryear,longnamesfirst]{natbib}
%\usepackage[authoryear]{natbib}
\usepackage[numbers]{natbib}
\usepackage{graphics} % for pdf, bitmapped graphics files
%\usepackage{epsfig} % for postscript graphics files
%\usepackage{mathptmx} % assumes new font selection scheme installed
%\usepackage{times} % assumes new font selection scheme installed
\usepackage{amsmath} % assumes amsmath package installed
\usepackage{amssymb}  % assumes amsmath package installed
\usepackage{amsfonts}
\usepackage{xcolor}
\usepackage{epsfig}
\usepackage{graphicx}
\usepackage{subfigure}
\usepackage{epstopdf} % for eps figure using pdflatex
% \usepackage[linesnumbered,ruled,vlined]{algorithm2e}
\usepackage{algorithm} 
\usepackage{algpseudocode} 
\usepackage{balance}
\usepackage{stfloats}

\newcommand{\red}{\textcolor{red}}
\newcommand{\blue}{\textcolor{blue}}

%%%Author definitions
\def\tsc#1{\csdef{#1}{\textsc{\lowercase{#1}}\xspace}}
\tsc{WGM}
\tsc{QE}
\tsc{EP}
\tsc{PMS}
\tsc{BEC}
\tsc{DE}
%%%

\begin{document}
\let\WriteBookmarks\relax
\def\floatpagepagefraction{1}
\def\textpagefraction{.001}
\shorttitle{Biomimetic Intelligence and Robotics}
%\shortauthors{CV Radhakrishnan et~al.}

\title [mode = title]{Relevant Region Sampling Strategy with Adaptive Heuristic for Asymptotically Optimal Path Planning
}             

\address[1]{Department of Electronic Engineering, The Chinese University of Hong Kong, Shatin N.T., Hong Kong SAR, China}
\address[2]{Department of Electronic and Electrical Engineering, Southern University of Science and Technology, Shenzhen, China}
\address[3]{Shenzhen Research Institute of the Chinese University of Hong Kong, Shenzhen, China}

\author[1]{Chenming Li}[orcid=0000-0001-6322-0834]
% \credit{Conceptualization, Data curation, Formal analysis, Investigation, Methodology, Software, Visualization, Writing - original draft}
\ead{licmjy@link.cuhk.edu.hk}

\author[1]{Fei Meng}
% \cormark[1]
% \credit{Conceptualization, Investigation, Methodology, Validation, Writing - review \& editing}
\ead{feimeng@link.cuhk.edu.hk}

\author[1]{Han Ma}
% \cormark[1]
% \credit{Conceptualization, Investigation, Methodology, Validation, Writing - review \& editing}
\ead{hanma@link.cuhk.edu.hk}

\author[2]{Jiankun Wang}
% [orcid=0000-0001-9139-0291]
% \cormark[1]
% \credit{Conceptualization, Data curation, Formal analysis, Investigation, Writing - review \& editing}
\ead{wangjk@sustech.edu.cn}

\author[1,2,3]{Max Q.-H. Meng}
% [orcid=0000-0002-5255-5898]
% \credit{Funding acquisition, Project administration, Resources, Supervision, Writing - review \& editing}
\ead{max.meng@ieee.org}

%\cortext[cor1]{Corresponding author. 1155067752@link.cuhk.edu.hk}
% \cortext[cor1]{Jiankun Wang, Weinan Chen, Xiao Xiao and Yangxin Xu contributed eqaully to this work.}
\cortext[cor2]{This project is supported by Shenzhen Key Laboratory of Robotics Perception and Intelligence (ZDSYS20200810171800001) and the Hong Kong RGC GRF grants \# 14200618 awarded to Max Q.-H. Meng. \textit{(Corresponding authors: Jiankun Wang, Max Q.-H. Meng.)}}
  

\begin{abstract}
Sampling-based planning algorithm is a powerful tool for solving planning problems in high-dimensional state spaces. 
In this article, we present a novel approach to sampling in the most promising regions, which significantly reduces planning time-consumption.
The RRT\# algorithm defines the Relevant Region based on the cost-to-come provided by the optimal forward-searching tree. 
However, it uses the cumulative cost of a direct connection between the current state and the goal state as the cost-to-go.
To improve the path planning efficiency, we propose a batch sampling method that samples in a refined Relevant Region with a direct sampling strategy, which is defined according to the optimal cost-to-come and the adaptive cost-to-go, taking advantage of various sources of heuristic information.
% With our proposed method, the algorithm can generate new samples in the Relevant Region with a direct sampling method that takes advantage of various sources of heuristic information.
The proposed sampling approach allows the algorithm to build the search tree in the direction of the most promising area, resulting in a superior initial solution quality and reducing the overall computation time compared to related work.
To validate the effectiveness of our method, we conducted several simulations in both $SE(2)$ and $SE(3)$ state spaces.
And the simulation results demonstrate the superiorities of proposed algorithm.
\end{abstract}
%%%%%%%%%%%%%%%%%%%%%%%%%%%%%%%%%%%%%%%%%%%%%%%%%%%%%%%%%%%%%%%%%
%%%%%%%%%%%%%%%%%%%%%%%%%%%%%%%%%%%%%%%%%%%%%%%%%%%%%%%%%%%%%%%%%
%%%%%%%%%%%%%%%%%%%%%%%%%%%%%%%%%%%%%%%%%%%%%%%%%%%%%%%%%%%%%%%%%
%\begin{graphicalabstract}
%\includegraphics{figs/grabs.pdf}
%\end{graphicalabstract}

%\begin{highlights}
%\item Research highlights item 1
%\item Research highlights item 2
%\item Research highlights item 3
%\end{highlights}
%%%%%%%%%%%%%%%%%%%%%%%%%%%%%%%%%%%%%%%%%%%%%%%%%%%%%%%%%%%%%%%%%
%%%%%%%%%%%%%%%%%%%%%%%%%%%%%%%%%%%%%%%%%%%%%%%%%%%%%%%%%%%%%%%%%
%%%%%%%%%%%%%%%%%%%%%%%%%%%%%%%%%%%%%%%%%%%%%%%%%%%%%%%%%%%%%%%%%
\begin{keywords}
Path Planning \sep Asymptotical Optimality \sep Relevant Region \sep Adaptive heuristic
\end{keywords}

\maketitle

\IEEEraisesectionheading{\section{Introduction}}

\IEEEPARstart{V}{ision} system is studied in orthogonal disciplines spanning from neurophysiology and psychophysics to computer science all with uniform objective: understand the vision system and develop it into an integrated theory of vision. In general, vision or visual perception is the ability of information acquisition from environment, and it's interpretation. According to Gestalt theory, visual elements are perceived as patterns of wholes rather than the sum of constituent parts~\cite{koffka2013principles}. The Gestalt theory through \textit{emergence}, \textit{invariance}, \textit{multistability}, and \textit{reification} properties (aka Gestalt principles), describes how vision recognizes an object as a \textit{whole} from constituent parts. There is an increasing interested to model the cognitive aptitude of visual perception; however, the process is challenging. In the following, a challenge (as an example) per object and motion perception is discussed. 



\subsection{Why do things look as they do?}
In addition to Gestalt principles, an object is characterized with its spatial parameters and material properties. Despite of the novel approaches proposed for material recognition (e.g.,~\cite{sharan2013recognizing}), objects tend to get the attention. Leveraging on an object's spatial properties, material, illumination, and background; the mapping from real world 3D patterns (distal stimulus) to 2D patterns onto retina (proximal stimulus) is many-to-one non-uniquely-invertible mapping~\cite{dicarlo2007untangling,horn1986robot}. There have been novel biology-driven studies for constructing computational models to emulate anatomy and physiology of the brain for real world object recognition (e.g.,~\cite{lowe2004distinctive,serre2007robust,zhang2006svm}), and some studies lead to impressive accuracy. For instance, testing such computational models on gold standard controlled shape sets such as Caltech101 and Caltech256, some methods resulted $<$60\% true-positives~\cite{zhang2006svm,lazebnik2006beyond,mutch2006multiclass,wang2006using}. However, Pinto et al.~\cite{pinto2008real} raised a caution against the pervasiveness of such shape sets by highlighting the unsystematic variations in objects features such as spatial aspects, both between and within object categories. For instance, using a V1-like model (a neuroscientist's null model) with two categories of systematically variant objects, a rapid derogate of performance to 50\% (chance level) is observed~\cite{zhang2006svm}. This observation accentuates the challenges that the infinite number of 2D shapes casted on retina from 3D objects introduces to object recognition. 

Material recognition of an object requires in-depth features to be determined. A mineralogist may describe the luster (i.e., optical quality of the surface) with a vocabulary like greasy, pearly, vitreous, resinous or submetallic; he may describe rocks and minerals with their typical forms such as acicular, dendritic, porous, nodular, or oolitic. We perceive materials from early age even though many of us lack such a rich visual vocabulary as formalized as the mineralogists~\cite{adelson2001seeing}. However, methodizing material perception can be far from trivial. For instance, consider a chrome sphere with every pixel having a correspondence in the environment; hence, the material of the sphere is hidden and shall be inferred implicitly~\cite{shafer2000color,adelson2001seeing}. Therefore, considering object material, object recognition requires surface reflectance, various light sources, and observer's point-of-view to be taken into consideration.


\subsection{What went where?}
Motion is an important aspect in interpreting the interaction with subjects, making the visual perception of movement a critical cognitive ability that helps us with complex tasks such as discriminating moving objects from background, or depth perception by motion parallax. Cognitive susceptibility enables the inference of 2D/3D motion from a sequence of 2D shapes (e.g., movies~\cite{niyogi1994analyzing,little1998recognizing,hayfron2003automatic}), or from a single image frame (e.g., the pose of an athlete runner~\cite{wang2013learning,ramanan2006learning}). However, its challenging to model the susceptibility because of many-to-one relation between distal and proximal stimulus, which makes the local measurements of proximal stimulus inadequate to reason the proper global interpretation. One of the various challenges is called \textit{motion correspondence problem}~\cite{attneave1974apparent,ullman1979interpretation,ramachandran1986perception,dawson1991and}, which refers to recognition of any individual component of proximal stimulus in frame-1 and another component in frame-2 as constituting different glimpses of the same moving component. If one-to-one mapping is intended, $n!$ correspondence matches between $n$ components of two frames exist, which is increased to $2^n$  for one-to-any mappings. To address the challenge, Ullman~\cite{ullman1979interpretation} proposed a method based on nearest neighbor principle, and Dawson~\cite{dawson1991and} introduced an auto associative network model. Dawson's network model~\cite{dawson1991and} iteratively modifies the activation pattern of local measurements to achieve a stable global interpretation. In general, his model applies three constraints as it follows
\begin{inlinelist}
	\item \textit{nearest neighbor principle} (shorter motion correspondence matches are assigned lower costs)
	\item \textit{relative velocity principle} (differences between two motion correspondence matches)
	\item \textit{element integrity principle} (physical coherence of surfaces)
\end{inlinelist}.
According to experimental evaluations (e.g.,~\cite{ullman1979interpretation,ramachandran1986perception,cutting1982minimum}), these three constraints are the aspects of how human visual system solves the motion correspondence problem. Eom et al.~\cite{eom2012heuristic} tackled the motion correspondence problem by considering the relative velocity and the element integrity principles. They studied one-to-any mapping between elements of corresponding fuzzy clusters of two consecutive frames. They have obtained a ranked list of all possible mappings by performing a state-space search. 



\subsection{How a stimuli is recognized in the environment?}

Human subjects are often able to recognize a 3D object from its 2D projections in different orientations~\cite{bartoshuk1960mental}. A common hypothesis for this \textit{spatial ability} is that, an object is represented in memory in its canonical orientation, and a \textit{mental rotation} transformation is applied on the input image, and the transformed image is compared with the object in its canonical orientation~\cite{bartoshuk1960mental}. The time to determine whether two projections portray the same 3D object
\begin{inlinelist}
	\item increase linearly with respect to the angular disparity~\cite{bartoshuk1960mental,cooperau1973time,cooper1976demonstration}
	\item is independent from the complexity of the 3D object~\cite{cooper1973chronometric}
\end{inlinelist}.
Shepard and Metzler~\cite{shepard1971mental} interpreted this finding as it follows: \textit{human subjects mentally rotate one portray at a constant speed until it is aligned with the other portray.}



\subsection{State of the Art}

The linear mapping transformation determination between two objects is generalized as determining optimal linear transformation matrix for a set of observed vectors, which is first proposed by Grace Wahba in 1965~\cite{wahba1965least} as it follows. 
\textit{Given two sets of $n$ points $\{v_1, v_2, \dots v_n\}$, and $\{v_1^*, v_2^* \dots v_n^*\}$, where $n \geq 2$, find the rotation matrix $M$ (i.e., the orthogonal matrix with determinant +1) which brings the first set into the best least squares coincidence with the second. That is, find $M$ matrix which minimizes}
\begin{equation}
	\sum_{j=1}^{n} \vert v_j^* - Mv_j \vert^2
\end{equation}

Multiple solutions for the \textit{Wahba's problem} have been published, such as Paul Davenport's q-method. Some notable algorithms after Davenport's q-method were published; of that QUaternion ESTimator (QU\-EST)~\cite{shuster2012three}, Fast Optimal Attitude Matrix \-(FOAM)~\cite{markley1993attitude} and Slower Optimal Matrix Algorithm (SOMA)~\cite{markley1993attitude}, and singular value decomposition (SVD) based algorithms, such as Markley’s SVD-based method~\cite{markley1988attitude}. 

In statistical shape analysis, the linear mapping transformation determination challenge is studied as Procrustes problem. Procrustes analysis finds a transformation matrix that maps two input shapes closest possible on each other. Solutions for Procrustes problem are reviewed in~\cite{gower2004procrustes,viklands2006algorithms}. For orthogonal Procrustes problem, Wolfgang Kabsch proposed a SVD-based method~\cite{kabsch1976solution} by minimizing the root mean squared deviation of two input sets when the determinant of rotation matrix is $1$. In addition to Kabsch’s partial Procrustes superimposition (covers translation and rotation), other full Procrustes superimpositions (covers translation, uniform scaling, rotation/reflection) have been proposed~\cite{gower2004procrustes,viklands2006algorithms}. The determination of optimal linear mapping transformation matrix using different approaches of Procrustes analysis has wide range of applications, spanning from forging human hand mimics in anthropomorphic robotic hand~\cite{xu2012design}, to the assessment of two-dimensional perimeter spread models such as fire~\cite{duff2012procrustes}, and the analysis of MRI scans in brain morphology studies~\cite{martin2013correlation}.

\subsection{Our Contribution}

The present study methodizes the aforementioned mentioned cognitive susceptibilities into a cognitive-driven linear mapping transformation determination algorithm. The method leverages on mental rotation cognitive stages~\cite{johnson1990speed} which are defined as it follows
\begin{inlinelist}
	\item a mental image of the object is created
	\item object is mentally rotated until a comparison is made
	\item objects are assessed whether they are the same
	\item the decision is reported
\end{inlinelist}.
Accordingly, the proposed method creates hierarchical abstractions of shapes~\cite{greene2009briefest} with increasing level of details~\cite{konkle2010scene}. The abstractions are presented in a vector space. A graph of linear transformations is created by circular-shift permutations (i.e., rotation superimposition) of vectors. The graph is then hierarchically traversed for closest mapping linear transformation determination. 

Despite of numerous novel algorithms to calculate linear mapping transformation, such as those proposed for Procrustes analysis, the novelty of the presented method is being a cognitive-driven approach. This method augments promising discoveries on motion/object perception into a linear mapping transformation determination algorithm.




\section{Related Work}


\subsection{Sampling-based Motion Planning Method}

Plenty of modifications are proposed to enhance the performance of the RRT algorithm \cite{lavalle1998rapidly} such as the RRT* algorithm \cite{karaman2011sampling}.
The rewiring stage of the RRT* only rewires locally, which means the global optimization of the current tree is ignored. 
The RRT\# \cite{arslan2013use} proposes to find the global optimality in each rewiring stage with dynamic programming.
Dynamic programming is also used in the Fast Marching Tree (FMT*) method \cite{janson2015fast} to grow the searching tree.
% And the FMT* introduces the thought of batch sampling into the robot motion planning field.
The Informed sampling strategy \cite{gammell2014informed, gammell2018informed} is proposed to overcome the drawback of uniform sampling.
% It can accelerate the convergence speed significantly with very little computation consumption. 
% The Informed sampling strategy uses a direct sampling method to generate samples in the $L_2$-Informed set.
% An advanced version of the Informed sampling strategy is proposed in \cite{gammell2018informed}, which includes the graph pruning stage to keep a relatively constricted tree. 
% A tree with fewer vertices means that the cost reduction in finding the nearest tree vertex. 
Using the neural network to reinforce the sampling stage to enhance the sampling efficiency \cite{wang2020neural, li2021efficient, qureshi2019motion} is proved as a promising technique.

% In the human-robot coexisting environment, the planning problem become more complex than that in the static environment \cite{wang2020eb}.
% The objective of optimization needs to consider safety, efficiency, and human feelings.


% The Informed sampling strategy can constraint the whole planning procedure in a subset of the whole state space, the $L_2$-Informed set, and the Lebesgue measure of the $L_2$-Informed set decreases as the solution improves.
% \\ \textcolor{red}{*SWIRRT*: maybe not including the SWIRRT is better}




\subsection{Batch Sampling Technique}

% A batch sampling method is described in the FMT* method \cite{janson2015fast}.
The FMT* \cite{janson2015fast} introduces the thought of batch sampling into the robot motion planning field.
The FMT* samples a batch of points and constructs the searching tree according to this batch of samples.
% The asymptotic optimal of the FMT* is guaranteed when the size of the batch goes to infinity. 
The Batch Informed Trees (BIT*) \cite{gammell2015batch, gammell2020batch} method is developed based on the Informed RRT*, besides, the BIT* absorbs the thoughts in the FMT* method \cite{janson2015fast} and the Lifelong Planning A* (LPA*) algorithm \cite{koenig2004lifelong}.
The Regionally Accelerated Batch Informed Trees (RABIT*) \cite{choudhury2016regionally} aims to solve the difficult-to-sample planning problem, like the narrow passage problem.
% The RABIT* uses the Covariant Hamiltonian Optimization for Motion Planning (CHOMP) method as its local optimizer, and the local optimizer will exploit the local information.
The Fast-BIT* \cite{holston2017fast} modifies the edge queue and searches the initial solution more aggressively. 
The Greedy BIT* \cite{chen2021greedy} uses the greedy searching method to generate the initial solution faster and accelerate the convergence speed.
But these greedy-based methods often fail to assist the searching procedure without an accurate heuristic estimation method. 
The Adaptively Informed Trees (AIT*) \cite{strub2020adaptively} and the Advanced BIT* (ABIT*) \cite{strub2020advanced} proposed by Strub and Gammell are developed based on the BIT* as well. 
The AIT* calculate a relatively accurate heuristic estimation with a lazy reverse-searching tree.
The ABIT* proposes to utilize inflation and truncation to balance the exploitation and exploration in the increasingly complex Random Geometric Graph (RGG) \cite{penrose2003random}.
Though the AIT* and the ABIT* achieve significant improvements, their sampling regions are not compact enough, and the sampling efficiency will be critically low in the complex environment.


\subsection{Relevant Region Sampling Strategy}

The concept of 'relevant' is first proposed in the searching-based robot path planning method like the A* \cite{hart1968formal}. 
In the A* algorithm, the set of expanded vertices is relevant to the query, such that the A* algorithm could expand a smaller set of vertices than the Dijkstra's algorithm \cite{dijkstra1959note}.  
% And it is also not a new idea in the sampling-based planning field. 
% The Relevant Region is formally defined in \cite{arslan2013use}. 
% The Relevant Region related vertices are the vertices of which the sum of the optimal cost-to-come and the heuristic is less than the cost of the current optimal solution.
% Since the Relevant Region is the most promising region that could help to improve the solution, so a straightforward modification is to reduce the chance of sampling outside the  Relevant Region.
The concept of the Relevant Region is formally introduced in \cite{arslan2013use}, whose sum of the optimal cost-to-come and cost-to-go heuristic is less than the cost of the current optimal solution. 
Since the Relevant Region is the most promising area for improving the solution, a straightforward modification would be reducing the likelihood of sampling outside of it.
Three different metrics are used to achieve this in \cite{arslan2015dynamic}, the modified versions achieve better performance in the convergence speed than the RRT\# method.
The methods described in \cite{arslan2013use} and \cite{arslan2015dynamic} use the rejection method for sampling, which is not efficient since the Relevant Region is a small subset of the whole state space in most scenarios.
The direct sampling method is illustrated to overcome this drawback, and the details are described in \cite{joshi2020relevant}.
However, they all use the cumulative cost along the direct connection between the current state and the goal state as the cost-to-go. 
This approach results in inaccurate estimated cost-to-go in most scenarios.
% Their ordered priority queues are also far from the ground truth.

% The optimal cost-to-come value can be defined as the vertex which has the lowest optimal cost-to-come value in the destination region. 


\subsection{Bi-directional Searching Method}

The RRT and RRT* methods may not always discover a solution within the allotted time, particularly when dealing with narrow passages
The RRT-Connect \cite{kuffner2000rrt} is proposed to find the initial solution faster. 
% It grows two trees from the source point and the goal region simultaneously.
% It is proved that the RRT-Connect can achieve better performance than the RRT.
However, the approach described in \cite{kuffner2000rrt} is not asymptotically optimal. Therefore, its successor, RRT-Connect, is also not asymptotically optimal. 
To overcome this, an enhanced version of the bidirectional searching RRT is introduced in \cite{klemm2015rrt} to guarantee the asymptotical optimality.
To take advantage of the benefits of bi-directional search, the kinematic constraints are taken into consideration in the bi-directional search method to generate executable trajectories efficiently \cite{wang2021kinematic}.
% The method described in \cite{klemm2015rrt} is an asymptotically optimal single-query version of the RRT-Connect, called the RRT*-Connect. 
% The RRT*-Connect provides asymptotically optimal guarantee like the RRT*, and its efficiency and robustness are proofed in real-world experiments.
% In addition, the bi-directional searching method can be used to combine with the kinematic constraints , which is essential in generating executable trajectory.

% In the RRT-Connect, one tree is extended in each iteration and tries to connect itself to the other tree; this manner will attempt to grow the trees towards each other.

One drawback of the Informed RRT* \cite{gammell2014informed} \cite{gammell2018informed} is that it uses the RRT* to search the whole state space before finding the initial solution.
Therefore, the Informed RRT* often fails to find the solution in the required period, same as the RRT*.
By combining the advantages of both the Informed and the RRT*-Connect, the Informed RRT*-Connect \cite{2020Informed} proposes to use the RRT*-Connect to generate the initial solution and use the Informed sampling strategy to constrain the sampling region after the initial solution is found.
% It combines the advantages of both the Informed RRT* and the RRT*-Connect.
% The Informed RRT*-Connect can achieve a much higher success rate in its simulations than the Informed RRT*.
Besides, the AIT* \cite{strub2020adaptively} can also be viewed as a bi-directional searching method.





\section{Problem Definition}

Consider the state space $\mathcal{X}$, which is the subset of $\Re^d$.
$\Re^d$ is the whole $d$-dimensional space, and $d$ is a positive integer.
$\mathcal{X}_{obs}$ shows the space occupied by the obstacles, the free space is defined as $\mathcal{X}_{free} = \mathcal{X} \setminus \mathcal{X}_{obs} $.
The $x \in \mathcal{X}$ represents any state in the state space.
The source point $x_{start}$ is the initial state of the robot.
The destination is a region represented by $\mathcal{X}_{goal}$.
The source point and the destination in a valid planning problem must be defined within the free space $x_{start} \in \mathcal{X}_{free} \ \& \ \mathcal{X}_{goal} \subseteq \mathcal{X}_{free}$.
The motion planning problem is defined as:

% The robot motion planning problem refers to find a trajectory that could move the robot from the source point to the destination, and the trajectory should satisfy the constraints. 


\begin{equation}
\begin{aligned}
& \pi \in [0, 1] \to \mathcal{X}_{free}, \\  
& s.t. \ \pi(0) = x_{start}, \ \pi(1) \in \mathcal{X}_{goal}, \ 
\pi(s) \in \mathcal{X}_{free},
\forall s \in [0, 1] .
\label{MotionPlanningDefinition}
\end{aligned}
\end{equation}


The optimization objective can be minimizing the trajectory length, maximizing the minimum clearance, or any objects that could be mathematically defined. 
It can also be set as the sum of individual optimization objectives and form a hybrid optimization objective.
For simplicity, the optimization objective in the proposed method is set as minimizing the trajectory length.
Note that it could be extended to meet the specific planning requirement.
The $v \in \mathcal{T}$ denotes any vertex in the tree.
Assume there are two vertices $v_1, \ v_2 \in \mathcal{T}$, where $v_2$ is the descendant of $v_1$.
The cost between any two vertices $v_1, \ v_2$ is calculated with the intergal cost along the tree from $v_1$ to $v_2$, denotes as $d_{\mathcal{T}}(v_1, \ v_2)$.
Let $\Pi$ denote the set of all feasible solutions.
With the definition of trajectory cost, the optimization objective in our method is written as:


\begin{equation}
    \begin{aligned}
    & \pi* = \operatorname*{\arg \min}_{\pi \in \Pi} d_{\mathcal{T}}(v_{start}, v_{goal}) \\
    & s.t. \ \pi(0) = v_{start}, \ \pi(1) \in \mathcal{X}_{goal}, \pi(s) \in \mathcal{X}_{free},  \forall s \in [0, 1] .
\label{TrajectoryLengthOptimizationObjective}
\end{aligned}
\end{equation}
    

The cost estimation between any two states $x_1, x_2 \in \mathcal{X}$ can be set as the cumulative cost along the direct connection. 
The cost for the unit distance is usually deemed as $1$, the heuristic can be calculated with the Euclidean metric: $ h(x_1, x_2) = \left\| x_2 - x_1 \right\|_2$.
However, using this metric as the heuristic estimation method between any two states is not promising.
The direct connection is highly likely to collide with the obstacle, which can mislead the searching procedure, especially in complex environments.


The RRT algorithm contains two stages: the sampling and the searching stage. The sampling stage can be viewed as the abstracting process of the state space $\mathcal{X}$, and the searching stage will construct the searching tree $\mathcal{T}$ based on this abstracted state space. 
These two phases are performed alternatively in each iteration. 
We use the $c_{cur}$ to represent the cost of the current optimal solution. 
Before finding the initial feasible solution, the $c_{cur}$ is set to an infinitely large number $+\infty$.



\subsection{Adaptive Heuristic Estimation}

The adaptive heuristic estimation method was first proposed by the AIT* algorithm \cite{strub2020adaptively}, which is able to provide a problem-specific heuristic without overburdening computation resources.
We borrowed this idea to provide a fairly accurate estimation of the cost-to-go, and used it to guide the algorithm to take samples in relevant regions.

Since calculating the heuristic along the direct connection is inaccurate, the planner uses a lazy reverse-searching tree $\mathcal{T}_{\mathcal{R}}$ to provide the cost-to-go estimation $h_{\mathcal{T}_{\mathcal{R}}}(x)$ of any state $x$ in the state space, which is analogous to the AIT* algorithm \cite{strub2020adaptively}.
The $\mathcal{T}_{\mathcal{R}}$ is constructed without edge evaluation because edge evaluation is the most time-consuming procedure in the majority of motion planning scenarios.
To construct the $\mathcal{T}_{\mathcal{R}}$, the planner separates the sampling stage and searching stage into two separate modules instead of performing them alternatively in each iteration.
In the sampling stage, the planner uses our sampling strategy to generate a batch of sampling points in $\mathcal{X}_{free}$. 
Then, in the searching stage, the planner constructs the $\mathcal{T}_{\mathcal{R}}$ and $\mathcal{T}_{\mathcal{F}}$ in terms of the current RGG.
% A conceptual illustration is provided in Fig. \ref{schematicIllustration}.
Since the $\mathcal{T}_{\mathcal{R}}$ is constructed without edge evaluation, the edges in the $\mathcal{T}_{\mathcal{R}}$ may collide with the obstacles or not satisfy the constraints.
To deal with this, the $\mathcal{T}_{\mathcal{R}}$ will be updated incrementally upon collision, besides, AIT* uses a black and white list approach to prevent re-checking of the collided connections.
%  will be incrementally updated when the collision happens, and a black and white list technique is utilized in AIT* to avoid re-checking the collided connection.
In our proposed method, since we do not need edges in the $\mathcal{T}_{\mathcal{R}}$ to be executable, we borrow this idea to use the $\mathcal{T}_{\mathcal{R}}$ to provide a relatively accurate cost-to-go estimation and guidance for sampling.



\section{Methodology}
\label{sec:Methodology}

We begin by formally defining our problem.  
\subsection{Problem Statement}
\label{subsec:ProblemStatement}

\noindent \textbf{Input:}
\begin{enumerate}
    \item $\Sigma$: a finite alphabet. $\Sigma^+$ denotes the set of all non-empty strings over $\Sigma$. In this paper, we focus on strings that are job descriptions expressed as free form text.
    \item $\mathcal{Y}$: a finite set of labels. In this paper, SOC codes are treated as labels.
    \item $\mathcal{D} = \{(x_i, y_i): 1 \leq i \leq n \}$: a labeled dataset of size $n \in \mathbb{N}$, where $x_i \in \Sigma^+$ is a job description, and $y_i \in \mathcal{Y}$ is its corresponding SOC code.
\end{enumerate}

\noindent \textbf{Output:}
A function $f: \Sigma^+ \rightarrow \mathcal{Y}$ which maps a job description $x$ to an SOC code $y = f(x)$ such that $f$ minimizes the expected error with respect to some loss function.

From a pragmatic standpoint, we want such a function $f$ to be available as a web service (i.e., web API) which accepts a request containing description $x$ to produce a response containing the predicted SOC code $y = f(x)$.

\subsection{Approach}
\label{subsec:Approach}

Our approach may be described as a sequence of steps as follows.

\subsubsection{Text Vectorization}
Since a majority of machine learning algorithms assume inputs to be real valued vectors, predictive modeling based on text often requires vectorizing the text, i.e., computing real valued vector representation of text. We consider two different vectorization techniques, which are as follows.
\paragraph{TF-IDF $n$-grams} An $n$-gram ($n \in \mathbb{N}$) is a sequence of $n$ tokens. Given $n_{\mathrm{min}}, n_{\mathrm{max}} \in \mathbb{N}$ ($n_{\mathrm{min}} \leq n_{\mathrm{max}}$), a corpus of text in $\Sigma^+$ can be used to compute the vocabulary of all $n$-grams where $n_{\mathrm{min}} \leq n \leq  n_{\mathrm{max}}$. Subsequently, any string $x \in \Sigma^+$ may be represented as a vector of counts, i.e., term frequencies (TF) of $n$-grams present in $x$. Such a vector representation of a string is typically sparse, i.e., most of its components are zero, since most $n$-grams in the vocabulary are typically absent in it. To offset the effect of highly frequent $n$-grams with little semantic value, the vectors are weighted by inverse document frequencies (IDF), resulting in TF-IDF $n$-gram representations.
While TF-IDF representations have been found to achieve high accuracy in text categorization \cite{DBLP:conf/ecml/Joachims98}, the high dimensionality of the sparse vectors generally entails high computational costs for training predictive models.
\paragraph{Doc2vec} An alternative approach that addresses the issue of dimensionality consists of using neural architectures for vectorizing words \cite{mikolov2013efficient} and strings \cite{DBLP:conf/icml/LeM14}, using contextual similarity to predict semantic similarity. The resulting representations are known as word embeddings and document embeddings, respectively, and the above neural architectures are referred to as word2vec and doc2vec, respectively. Embeddings computed by word2vec and doc2vec are typically of lower dimensionality compared to TF-IDF $n$-gram representations. Therefore, such embeddings are considered dense vector representations. Since job descriptions are strings of arbitrary length, we use doc2vec to compute dense vector representations of such descriptions.

\subsubsection{Predictive Modeling}
For each type of vectorization, we train a set of standard classifiers for predicting SOC code, namely, $k$-nearest neighbors (KNN), Gaussian na\"ive Bayes (GNB), logistic regression (LR), linear support vector machine (LinearSVC), support vector machine with radial basis function (SVC-RBF), decision tree (DT), and random forest (RF).

\subsubsection{Evaluation and Model Selection}
To evaluate the models, we use $n$-fold cross validation. The dataset is first divided into $n$ slices (or folds) of (roughly) equal size. In each round of cross validation, a different slice is held out for testing while the remaining $n - 1$ slices are used for training. Several metrics are recorded in each round. At the end of $n$ rounds of training and testing, these metrics are averaged and reported. These scores help identify models best suited to the problem.

\subsubsection{Deployment}
Once a model has been selected, we deploy it as a web service which can accept a \texttt{POST} request whose body contains a job description in free form text and produce a response containing the predicted SOC code.

The next section presents our empirical evaluation.



\section{Simulations}



The simulations are all carried through the benchmark platform of the OMPL \cite{sucan2012open} \cite{moll2015benchmarking}.
To validate the generalization ability of our method, we solve the planning problem in both the $SE(2)$ and $SE(3)$ state spaces with our method and several state-of-art algorithms.
In simulation environments, the state spaces are continuous. 
The algorithms that have been tested are all based on random sampling and take samples from these continuous state spaces without discretizing the space.
The robot is represented by a collection of convex polyhedrons and occupies a certain volume.


\subsection{Qualitative Analysis}

\begin{figure*}[t]
    \centering
    \begin{minipage}[t]{1\linewidth}
        \subfigure[]{
            \begin{minipage}[t]{1\linewidth}
                \centering
                \includegraphics[width=1.0\textwidth]{./img/BenchmarkImgs/3D_Apartment_Result.png}
            \end{minipage}%
        }
    \end{minipage}%

    \centering
    \begin{minipage}[t]{1\linewidth}
        \subfigure[]{
            \begin{minipage}[t]{1\linewidth}
                \centering
                \includegraphics[width=1.0\textwidth]{./img/BenchmarkImgs/3D_CustomEasy_Result.png}
            \end{minipage}%
        }
    \end{minipage}%
    \caption{(a) and (b) show the 2D simulation result in `BugTrap', the `Maze', and the `RandomPolygons' environments, where the left pictures are the time each planner spent to meet the optimization objective 
    and the right pictures are the cost variations over time.
    Planners try to meet the optimization objective, dashed lines in the right pictures show the cost value of the optimization objective.
    }
\label{SimulationResults_3D}
\end{figure*}

\begin{figure}[t]
    \centering
    \includegraphics[width=0.46\textwidth]{./img/envs_3D.png}
    \caption{The 3D simulation environments.}
    \label{SimulationEnvironments_3D}
\end{figure}


% To provide a further explanation of our method, we use the RRT\# \cite{arslan2013use}, the AIT* \cite{strub2020adaptively}, and our method to solve the motion planning problem in an OMPL benchmark environment called the `BugTrap'.
To give more detail explanation about our approach, we employ the RRT\# \cite{arslan2013use}, AIT* \cite{strub2020adaptively}, and our own method to address the path planning challenge within the `BugTrap' OMPL benchmark environment.
The state space of the `BugTrap' environment is the $SE(2)$ state space, which is composed of the position $x$, $y$ and the orientation $w$.
The planning procedures of the RRT\# \cite{arslan2013use}, the AIT* \cite{strub2020adaptively}, and our method are illustrated in Fig. \ref{PlanningProcedure}, where obstacles, the free space, the start state, the goal region, and vertices are indicated with black, ivory white, pale blue, wine, and orange color, respectively.
We use the dark green lines and violet lines to show the forward tree and the current optimal solution, respectively.
The reverse trees are shown as the grey lines in the figure of the AIT* \cite{strub2020adaptively} and our method.


In the simulation shown in Fig. \ref{PlanningProcedure}, the planning problem contains two optimal solutions, one is to pass through the region upper the obstacle, and the other one is to pass through the lower part.
Fig. \ref{PlanningProcedure} shows that all the methods in Fig. \ref{PlanningProcedure} can acquire the global asymptotical optimality.
Both the AIT* and our method use the lazy reverse-searching tree to guide the sampling and have the graph pruning method to constraint the samples and the trees.
From the (j)-(m) in Fig. \ref{PlanningProcedure}, it can find that both the forward and reverse trees of our method are optimal under current state space abstraction.
In addition, our method concentrates on taking samples in the region with a higher potential to improve the current solution, which can be seen in (m) of Fig. \ref{PlanningProcedure}, our method pays more attention to the turning corners with our direct sampling method.


\subsection{Simulations in $SE(2)$ State Space}



We choose the $SE(2)$ environments shown in Fig. \ref{SimulationEnvironments_2D} to verify our method, and they are called the `BugTrap', the `Maze', and the `RandomPolygons' in the OMPL benchmark platform. 
To give the reader an intuitional understanding of our 2D planning simulations, we show the trajectories found by our method in Fig. \ref{SimulationPath_2D}.
The trajectories are interpolated in terms of time.


In our 2D simulations, the state space definition contains the position $x, y$ and orientation $w$.
To manifest the superiority of our method, we compared with seven different state-of-art algorithms, they are the RRT*, the BIT*, the AIT*, the ABIT*, the Informed RRT*, the RRT\#, and the Informed $+$ Relevant sampling method proposed in \cite{joshi2020relevant}.
In these simulations, we use the trajectory length as the cost metric.
The optimization objective is set as $\beta \times c_{opt}$, where the $c_{opt}$ is the cost of the optimal solution and $\beta$ is a number close to $100\%$.
% The $c_{opt}$ is the solution cost of the RRT* method after $300$ seconds' execution, which is nearly the optimal solution cost, and we choose to use this number to represent the optimal cost.
The $c_{opt}$ is the solution cost of the RRT* method after $300$ seconds of execution, which is nearly optimal. 
We choose to use this number to represent the optimal cost.
To reduce the randomness, each planner runs $100$ times in each environment.



The simulation results in the $SE(2)$ state spaces are shown in Fig. \ref{SimulationResults_2D}.
On the left side of the pictures, the charts shows the amount of time each planner took to generate the required path.
On the right side, the cost distribution is presented in terms of time. We begin plotting the line charts once 50\% of all runs have found a solution, and stop once 95\% have completed the problem-solving process. 
Hence, the speed of obtaining the initial solution can also be displayed in the same chart.
Additionally, error bars are provided for all bar charts and line charts.


The 2D simulation results show that our method acquired significant improvements and achieved better performance.
Both the initial solution quality and the convergence rate of our method are better than the others.
The only drawback of our method is we generate the initial solution slower than the others, but we acquired the best initial solution quality.
And our initial solution is better than the others' optimized solutions at the same time point.


\subsection{Simulations in $SE(3)$ State Space}



Besides the 2D simulation introduced previously, we also carried on the simulation in the $SE(3)$ state space. 
In the 3D simulations, we include the `3D\_Apartment' planning problem from the OMPL benchmark platform \cite{moll2015benchmarking}, which is a `piano movers' problem, as the left environment in Fig. \ref{SimulationEnvironments_3D} shows.
The other simulation is set as a planning problem in 3D narrow passage environment.
In the 3D simulation, planners and their parameter sets are the same as the planners we choose in 2D simulations.
Each planner will solve each planning problem 100 times to avoid the randomness.
The 3D simulation results are shown in Fig. \ref{SimulationResults_3D}.




\section{Conclusions}
In this paper, we suggest improving planning efficiency with a novel sampling strategy. This strategy relies on the optimal cost-to-come and problem-specific cost-to-go heuristics, acquired through the global replanning function and the lazy-reverse searching method, respectively. This sampling strategy generates samples in the Relevant Region to guide the planner to search the most promising regions, improving the quality of the initial solution path and the convergence rate. Simulation results in $SE(2)$ and $SE(3)$ spaces demonstrate that the proposed method achieves better performance when solving the planning problem. In the future, we aim to take advantage of the reverse searching method to further accelerate planning efficiency.


%%%%%%%%%%%%%%%%%%%%%%%%%%%%%%%%%%%%%%%%%%%%%%%%%%%%%%%%%%%%%%%%%
%%%%%%%%%%%%%%%%%%%%%%%%%%%%%%%%%%%%%%%%%%%%%%%%%%%%%%%%%%%%%%%%%
%%%%%%%%%%%%%%%%%%%%%%%%%%%%%%%%%%%%%%%%%%%%%%%%%%%%%%%%%%%%%%%%%

\printcredits

%% Loading bibliography style file
%\bibliographystyle{model1a-num-names}
\bibliographystyle{cas-model2-names}

% Loading bibliography database
\bibliography{references.bib}

\balance
%\vskip3pt

\end{document}

