\section{Introduction}

Robot motion planning algorithms are designed to find a trajectory for the robot from its starting point to its destination while avoiding obstacles. 
The robot navigation framework usually simplifies the problem by treating it as a geometric path planning problem to reduce computation complexity. 
While the geometric path planning problem is easier to solve than the motion planning problem, it can still be computationally challenging in some scenarios, such as continuous or high-dimensional spaces.
There are path planning algorithms that use random sampling to solve high-dimensional continuous space planning problems, such as the Rapidly-exploring Random Trees (RRT) \cite{lavalle1998rapidly} and the Probabilistic Roadmap \cite{kavraki1996probabilistic}.
However, these algorithms do not optimize the solution they find. 
To improve the planning efficiency, asymptotically optimal planning algorithms continuously refine the current solution path until it meets the desired quality or the allotted time has expired. 
Although asymptotically optimal planning methods guarantee finding a satisfactory solution path given sufficient computation time, the speed of solving the robot planning problem is crucial for real-time robot operation. 
% The speed depends on several key factors, including the speed of finding the initial feasible solution, the quality of the initial solution, and the convergence rate.
This speed depends on several key factors, including the speed of finding the initial feasible solution, the quality of the initial solution, and the convergence rate. 
Therefore, it is essential to optimize these factors to ensure efficient path planning for real-time robot operation.
% Some path planning algorithms can not optimize the found solution, such as the Rapidly-exploring Random Trees (RRT) \cite{lavalle1998rapidly} and the Probabilistic Roadmap \cite{kavraki1996probabilistic} algorithm.
% The asymptotically optimal planning algorithms can continuously optimize the current solution until the solution can meet the requirement or the time is up.
% The optimization metric can be defined as the length of the path, the measurement of the human comfort level in the human-robot coexisting environment, or any metric defined mathematically.
% The speed of solving the robot planning problem is the key to real-time robot operation.
% As a critical factor for application, the planning speed is subject to the speed of finding the initial feasible solution, the initial solution quality, and the speed of convergence. 


% The planning algorithm mainly contains two categories, the searching-based algorithm and the sampling-based algorithm.
% The searching-based method has a long history of evolution. 
% However, it naturally suffers from the curse of dimensionality. 
% Performing the deterministic searching is impractical in the high-dimensional planning problem. 

% Therefore, this article aims to improve the sampling strategy to reduce unnecessary edge evaluation without affecting the asymptotic optimality of the sampling-based algorithm.

% The sampling-based motion planning algorithm is a widely used approach when solving the high-dimensional robot motion planning problem. 
% Typically, the sampling-based motion planning algorithm can be divided into two stages, the sampling stage and the searching stage \cite{khaksar2016application}. 
% In the sampling stage, the algorithm will sample randomly in the free space.
% Then, it uses the sample to construct a searching tree or a roadmap in the searching stage. 
% Edge evaluation will ensure that each edge is executable in the searching stage.
% But the edge evaluation is not cheap in most scenarios, and it is the bottleneck of the computation speed.
% Therefore, one of the objectives of this article is to make the sampling more efficient and avoid explicitly evaluating every edge.
% We only check the valuable candidate edges that have the potential to improve the current solution by the adaptive heuristic estimation.
% The priority queue is used to achieve this aim.

As widely used approaches when solving the planning problems in high-dimensional continuous environments, sampling-based path planning algorithms typically separate the planning process into two stages: sampling and searching \cite{khaksar2016application}.
They samples randomly in the sampling stage, and then use these samples to construct a searching tree or a roadmap in the searching stage. 
Edge evaluation in the searching stage is necessary to ensure that each edge can be executed by the robot.
However, edge evaluation can be expensive and a bottleneck for computation speed in most cases. 
Using inaccurate heuristics or inappropriate strategies can cause the planner to search irrelevant regions, which wastes computational resources and affects real-time performance.
Representative sampling-based path planning methods, such as the RRT \cite{lavalle1998rapidly}, typically use uniform sampling to generate samples and utilize a tree structure to search the space.
The RRT* algorithm \cite{karaman2011sampling}, an asymptotical version of RRT, selects parent vertices based on the cost-to-come value and rewires the tree structure. 
However, RRT* only refines the current tree structure locally.
In contrast, the RRT\# \cite{arslan2013use} proposes to use a replanning function to propagate changes in the relevant part of the graph, rather than rewiring the tree locally.
However, both the RRT* and RRT\# use uniform sampling to produce samples across the entire state space, leading to low sampling efficiency.
Uniform sampling can provide a topology abstraction of the entire state space, however, planners do not need to know information about the entire state space in most planning problems.
The uniform sampling method causes the planner to sample in redundant regions that do not contribute to the solution. 
As a result, this lead to unnecessary edge evaluations and wasted computations.
To address the aforementioned problems, we present a direct sampling technique within a subset of the entire state space, enhancing the efficiency of the sampling stage and allowing the algorithm to concentrate on abstracting the most promising region and optimizing the overall performance.

% A representative sampling-based motion planning method called the RRT \cite{lavalle1998rapidly} uses the uniformly sampling strategy to generate the sample and utilizes a tree structure to perform the space searching.
% The RRT method takes a sample in the sampling stage and connects it to the tree in the searching stage.
% But the RRT is not an asymptotically optimal planner, and the RRT* is proposed to solve this problem \cite{karaman2011sampling}. 
% The new vertex of the RRT* will choose its parents in terms of the cost. 
% And the rewire function can refine the current tree structure locally.
% Instead of rewiring the tree locally, the RRT\# \cite{arslan2013use} proposes to use a replanning function to propagate changes in the relevant part of the graph.
% However, the RRT* and RRT\# use the uniformly sampling strategy to produce samples in the whole state space, and it suffers from the low sampling efficiency.
% In their sampling stage, uniformly sampling can provide a topology abstraction of the entire state space, though most planning problems do not need to learn the information of the entire state space.
% This article proposes to sample in a subset of the whole state space with a direct sampling method. 
% Our method can make the sampling stage more efficient, and the algorithm will concentrate on abstracting the most promising region.


% Using the reject sampling method to generate samples in the Informed space will be inefficient as the dimensionality goes up, therefore, a direct sampling method in the Informed space is described in \cite{gammell2014informed}.
% In this article, only the points within the Informed space are token into consideration. 
% Besides, we absort the idea in our previous work, \textcolor{red}{SWIRRT}, using the current optimal path to guide the sampling procedure which could accelerate the convergence speed.

The cost-to-come and cost-to-go are fairly easy to acquire, and therefore widely used to guide the sampling process.
Conventional methods typically estimate the cost-to-come using non-optimal search trees and estimate the cost-to-go using Euclidean distance to the goal. 
Unfortunately, these estimates are often significantly different from the true cost-to-come and cost-to-go. 
In light of this, we propose utilizing optimal cost-to-come and adaptive cost-to-go estimation methods to guide the sampling process. 
To calculate the optimal cost-to-come, a replanning method is used that propagates changes along the search tree. 
The adaptive cost-to-go estimation employs similar approach as the Adaptively Informed Trees (AIT*) algorithm, which is obtained through a lazy-reverse tree technique.
As a result of the optimal cost-to-come and the adaptive cost-to-go estimation, our method focuses on taking samples from the areas that have the highest probability of improving the solution paths, which is closely relevant to the current query and current tree. 
Therefore, we have named this sampling technique the Relevant Region sampling strategy.

% The Informed sampling strategy constraints the sampling stage in a subset of the whole state space, which is a $d$-dimensional prolate hyper-spheroid in $d$-dimensional space \cite{gammell2014informed} \cite{gammell2018informed}.
% The Informed sampling strategy is implemented based on the RRT* algorithm.
% However, it defines the prolate hyper-spheroid in terms of the estimated cost-to-come and cost-to-go, which are both calculated with the Euclidean metric. 
% Therefore, the Informed sampling strategy is not a promising sampling method in complex environments, such as the maze-like environment. 
% To overcome this, we propose to utilize the optimal cost-to-come and the adaptive cost-to-go estimation methods to guide the sampling.
% Our method takes samples in the region with the highest chance of improving the current solution (both the approximate solution and the exact solution \cite{sucan2012open}), which is closely relevant to the current query and current tree.
% That is why we call this sampling method as the Relevant Region sampling strategy.
% In this work, we follow-up the concept of the Relevant Region defined in \cite{arslan2013use} \cite{arslan2015dynamic}, and \cite{joshi2020relevant}, but we refined the cost-to-go estimation method, the definition of the priority queue, and the new vertex generation process.

\begin{figure*}[ht]
    \centering
        \begin{minipage}[t]{1\linewidth}
            \subfigure[]{
                \begin{minipage}[t]{0.24\linewidth}
                    \centering
                    \includegraphics[width=0.9\linewidth]{./img/NewIllustration_1.png}
                \end{minipage}%
            }
            \subfigure[]{
                \begin{minipage}[t]{0.24\linewidth}
                    \centering
                    \includegraphics[width=0.9\linewidth]{./img/NewIllustration_2.png}
                    %\caption{fig2}
                    \label{collisionFigB}
                \end{minipage}%
            }
            \subfigure[]{
                \begin{minipage}[t]{0.24\linewidth}
                    \centering
                    \includegraphics[width=0.9\linewidth]{./img/NewIllustration_3.png}
                    %\caption{fig2}
                    \label{collisionFigC}
                \end{minipage}
            }%
            \subfigure[]{
                \begin{minipage}[t]{0.24\linewidth}
                    \centering
                    \includegraphics[width=0.9\linewidth]{./img/NewIllustration_4.png}
                \end{minipage}
            }%
        \end{minipage}%
    \centering
    \caption{ The schematic for the proposed method in a simple 2D environment, where the black blocks, the orange point, solid green lines, and the dashed grey lines demonstrate the obstacles, the sampling points, edges in the forward-searching tree, and edges in the reverse-searching trees, respectively.
              Fig. (a) shows that our method sample a batch of points in the current iteration.
              Fig. (b) shows the lazy reverse-searching stage.
              Fig. (c) shows the forward-searching stage.
              And Fig. (d) shows the current lazy reverse-searching tree is discarded, and a new batch of samples is added in the next iteration, where the newly added samples are more likely to be concentrated in areas that are `difficult to navigate'.
              }
    \label{schematicIllustration}
\end{figure*}

% The RRT\# \cite{arslan2013use} utilizes the uniform sampling through its entire process, and the Informed sampling strategy only works after the initial feasible solution is found.
% Three modified versions of RRT\# are proposed in \cite{arslan2015dynamic} with different inclusion or rejection criteria.
% But they do not include any effective heuristic for generating the initial solution.

% The speed of finding the initial feasible solution is one of the critical points for speeding up the entire planning procedure. 
% The bi-directional search method is an effective strategy to accelerate the initial feasible solution generation speed \cite{kuffner2000rrt}. 
% Instead of explicitly constructing both the forward-searching and reverse-searching trees, we implicitly construct the reverse-searching tree lazily. 
% We store the samples in a Geometric Near-neighbor Access Tree (GNAT) \cite{brin1995near}, a data structure for nearest neighbor search, instead of a linear data structure. 
% The lazy reverse-searching tree will provide the adaptive cost-to-go estimation for vertices in our priority queue. 
% In addition, the searching direction for a specific vertex is guided by the lazy reverse-searching tree. 
% The schematic of solving a simple path planning problem in a 2D environment is used to describe the workflow of our method since it is easily visualized, as shown in Fig. \ref{schematicIllustration}. 
% For illustration purposes, the schematic of solving a simple path planning problem in a 2D environment is used to describe the workflow of our method since it is easily visualized, as shown in Fig. \ref{schematicIllustration}. 


To describe the workflow of our method, we use a simple path planning schematic in a 2D environment illustrate it.
This diagram makes it easy to visualize the process, as shown in Fig. \ref{schematicIllustration}.
In the very first, the planner draws random samples in the free space and builds a lazy-reverse tree to provide a problem-specific cost-to-go estimation. 
Based on this estimation, the planner builds a forward search tree to explore the space in terms of the cost-to-go estimation.
When the forward search tree is updated, a replanning technique propagates the update to find a better global connection and the optimal cost-to-come for the newly added vertex. 
% A refined rewire function runs globally to achieve this.
After each batch of samples is fully explored, we take a new batch of samples based on the optimal cost-to-come and the last cost-to-go estimation.
Then a new lazy-reverse tree is constructed to guide the searching of the forward tree, until a termination condition is met.


Simulations are all carried through the benchmark platform of the Open Motion Planning Library (OMPL) \cite{sucan2012open} \cite{moll2015benchmarking}. 
To demonstrate the generality of the proposed method, we evaluate it in both $SE(2)$ and $SE(3)$ simulation environments.
We show that by combining forward tree replanning with the proposed sampling technique, the planning performance is improved.
% The related algorithms are configured with their default parameter settings in the OMPL.
% The parameters of our planner are chosen as the optimal setting for each simulation through parameter sweeping.
% In our simulations, we set up the other planners with their default parameter sets in the OMPL.


In this article, we introduce the related work in Section II. 
The problem definition is described in Section III, and the problem-specific heuristic is discussed as well.
Then, we present the mathematical details and the proposed algorithm in Section IV. 
Simulations are carried out to validate the effectiveness of the proposed method, and the results are shown in Section V.  
Finally, we draw our conclusions in Section VI.
% The Section. V is about the real world experiments. 
