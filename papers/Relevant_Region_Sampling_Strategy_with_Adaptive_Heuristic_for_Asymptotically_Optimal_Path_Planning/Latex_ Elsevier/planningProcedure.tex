\begin{figure*}[t]
    \centering
        \begin{minipage}[t]{1\linewidth}
            \subfigure[]{
                \begin{minipage}[t]{0.235\linewidth}
                    \centering
                    \includegraphics[width=0.9\linewidth]{./img/planningProcedureImg/RRTsharp/a.png}
                \end{minipage}%
            }
            \subfigure[]{
                \begin{minipage}[t]{0.235\linewidth}
                    \centering
                    \includegraphics[width=0.9\linewidth]{./img/planningProcedureImg/RRTsharp/b.png}
                    %\caption{fig2}
                \end{minipage}%
            }
            \subfigure[]{
                \begin{minipage}[t]{0.235\linewidth}
                    \centering
                    \includegraphics[width=0.9\linewidth]{./img/planningProcedureImg/RRTsharp/c.png}
                    %\caption{fig2}
                \end{minipage}
            }%
            \subfigure[]{
                \begin{minipage}[t]{0.235\linewidth}
                    \centering
                    \includegraphics[width=0.9\linewidth]{./img/planningProcedureImg/RRTsharp/d.png}
                    %\caption{fig2}
                \end{minipage}
            }%
        \end{minipage}%
    \centering
    \\
    The RRT\# algorithm.

        \begin{minipage}[t]{1\linewidth}
            \subfigure[]{
                \begin{minipage}[t]{0.235\linewidth}
                    \centering
                    \includegraphics[width=0.9\linewidth]{./img/planningProcedureImg/AITstar/a.png}
                    %\caption{fig1}
                \end{minipage}%
            }
            \subfigure[]{
                \begin{minipage}[t]{0.235\linewidth}
                    \centering
                    \includegraphics[width=0.9\linewidth]{./img/planningProcedureImg/AITstar/b.png}
                    %\caption{fig2}
                \end{minipage}%
            }
            \subfigure[]{
                \begin{minipage}[t]{0.235\linewidth}
                    \centering
                    \includegraphics[width=0.9\linewidth]{./img/planningProcedureImg/AITstar/c.png}
                    %\caption{fig2}
                \end{minipage}
            }%
            \subfigure[]{
                \begin{minipage}[t]{0.235\linewidth}
                    \centering
                    \includegraphics[width=0.9\linewidth]{./img/planningProcedureImg/AITstar/d.png}
                    %\caption{fig2}
                \end{minipage}
            }%
        \end{minipage}%
    \centering
    \\
    The AIT* algorithm.

    \subfigure{
        \begin{minipage}[t]{1\linewidth}
            \subfigure[]{
                \begin{minipage}[t]{0.23\linewidth}
                    \centering
                    \includegraphics[width=0.9\linewidth]{./img/planningProcedureImg/OurMethod/a.png}
                    %\caption{fig1}
                \end{minipage}%
            }
            \subfigure[]{
                \begin{minipage}[t]{0.23\linewidth}
                    \centering
                    \includegraphics[width=0.9\linewidth]{./img/planningProcedureImg/OurMethod/b.png}
                    %\caption{fig2}
                \end{minipage}%
            }
            \subfigure[]{
                \begin{minipage}[t]{0.23\linewidth}
                    \centering
                    \includegraphics[width=0.9\linewidth]{./img/planningProcedureImg/OurMethod/c.png}
                    %\caption{fig2}
                \end{minipage}
            }%
            \subfigure[]{
                \begin{minipage}[t]{0.23\linewidth}
                    \centering
                    \includegraphics[width=0.9\linewidth]{./img/planningProcedureImg/OurMethod/d.png}
                    %\caption{fig2}
                \end{minipage}
            }%
        \end{minipage}%
    }
    \centering
    \\
    Our method.

    \centering
    \caption{ Planning procedures of the RRT\# algorithm, the AIT* algorithm, and our method.
              The planning problem is set in the 'BugTrap' environment provided by the OMPL benchmark platform.
              We choose its start state and goal region to make there exists two different optimal solutions, which pass through different zones of the space and have the same solution cost. 
              The optimization objective is set as minimizing the path length.
              It can be seen that our method output a better path than the other two algorithms.
            }
    \label{PlanningProcedure}
\end{figure*}