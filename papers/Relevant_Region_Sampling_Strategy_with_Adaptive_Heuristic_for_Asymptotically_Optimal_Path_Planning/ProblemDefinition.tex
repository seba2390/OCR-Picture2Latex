\section{Problem Definition}

Consider the state space $\mathcal{X}$, which is the subset of $\Re^d$.
$\Re^d$ is the whole $d$-dimensional space, and $d$ is a positive integer.
$\mathcal{X}_{obs}$ shows the space occupied by the obstacles, the free space is defined as $\mathcal{X}_{free} = \mathcal{X} \setminus \mathcal{X}_{obs} $.
The $x \in \mathcal{X}$ represents any state in the state space.
The source point $x_{start}$ is the initial state of the robot.
The destination is a region represented by $\mathcal{X}_{goal}$.
The source point and the destination in a valid planning problem must be defined within the free space $x_{start} \in \mathcal{X}_{free} \ \& \ \mathcal{X}_{goal} \subseteq \mathcal{X}_{free}$.
The motion planning problem is defined as:

% The robot motion planning problem refers to find a trajectory that could move the robot from the source point to the destination, and the trajectory should satisfy the constraints. 


\begin{equation}
\begin{aligned}
& \pi \in [0, 1] \to \mathcal{X}_{free}, \\  
& s.t. \ \pi(0) = x_{start}, \ \pi(1) \in \mathcal{X}_{goal}, \ 
\pi(s) \in \mathcal{X}_{free},
\forall s \in [0, 1] .
\label{MotionPlanningDefinition}
\end{aligned}
\end{equation}


The optimization objective can be minimizing the trajectory length, maximizing the minimum clearance, or any objects that could be mathematically defined. 
It can also be set as the sum of individual optimization objectives and form a hybrid optimization objective.
For simplicity, the optimization objective in the proposed method is set as minimizing the trajectory length.
Note that it could be extended to meet the specific planning requirement.
The $v \in \mathcal{T}$ denotes any vertex in the tree.
Assume there are two vertices $v_1, \ v_2 \in \mathcal{T}$, where $v_2$ is the descendant of $v_1$.
The cost between any two vertices $v_1, \ v_2$ is calculated with the intergal cost along the tree from $v_1$ to $v_2$, denotes as $d_{\mathcal{T}}(v_1, \ v_2)$.
Let $\Pi$ denote the set of all feasible solutions.
With the definition of trajectory cost, the optimization objective in our method is written as:


\begin{equation}
    \begin{aligned}
    & \pi* = \operatorname*{\arg \min}_{\pi \in \Pi} d_{\mathcal{T}}(v_{start}, v_{goal}) \\
    & s.t. \ \pi(0) = v_{start}, \ \pi(1) \in \mathcal{X}_{goal}, \pi(s) \in \mathcal{X}_{free},  \forall s \in [0, 1] .
\label{TrajectoryLengthOptimizationObjective}
\end{aligned}
\end{equation}
    

The cost estimation between any two states $x_1, x_2 \in \mathcal{X}$ can be set as the cumulative cost along the direct connection. 
The cost for the unit distance is usually deemed as $1$, the heuristic can be calculated with the Euclidean metric: $ h(x_1, x_2) = \left\| x_2 - x_1 \right\|_2$.
However, using this metric as the heuristic estimation method between any two states is not promising.
The direct connection is highly likely to collide with the obstacle, which can mislead the searching procedure, especially in complex environments.


The RRT algorithm contains two stages: the sampling and the searching stage. The sampling stage can be viewed as the abstracting process of the state space $\mathcal{X}$, and the searching stage will construct the searching tree $\mathcal{T}$ based on this abstracted state space. 
These two phases are performed alternatively in each iteration. 
We use the $c_{cur}$ to represent the cost of the current optimal solution. 
Before finding the initial feasible solution, the $c_{cur}$ is set to an infinitely large number $+\infty$.



\subsection{Adaptive Heuristic Estimation}

The adaptive heuristic estimation method was first proposed by the AIT* algorithm \cite{strub2020adaptively}, which is able to provide a problem-specific heuristic without overburdening computation resources.
We borrowed this idea to provide a fairly accurate estimation of the cost-to-go, and used it to guide the algorithm to take samples in relevant regions.

Since calculating the heuristic along the direct connection is inaccurate, the planner uses a lazy reverse-searching tree $\mathcal{T}_{\mathcal{R}}$ to provide the cost-to-go estimation $h_{\mathcal{T}_{\mathcal{R}}}(x)$ of any state $x$ in the state space, which is analogous to the AIT* algorithm \cite{strub2020adaptively}.
The $\mathcal{T}_{\mathcal{R}}$ is constructed without edge evaluation because edge evaluation is the most time-consuming procedure in the majority of motion planning scenarios.
To construct the $\mathcal{T}_{\mathcal{R}}$, the planner separates the sampling stage and searching stage into two separate modules instead of performing them alternatively in each iteration.
In the sampling stage, the planner uses our sampling strategy to generate a batch of sampling points in $\mathcal{X}_{free}$. 
Then, in the searching stage, the planner constructs the $\mathcal{T}_{\mathcal{R}}$ and $\mathcal{T}_{\mathcal{F}}$ in terms of the current RGG.
% A conceptual illustration is provided in Fig. \ref{schematicIllustration}.
Since the $\mathcal{T}_{\mathcal{R}}$ is constructed without edge evaluation, the edges in the $\mathcal{T}_{\mathcal{R}}$ may collide with the obstacles or not satisfy the constraints.
To deal with this, the $\mathcal{T}_{\mathcal{R}}$ will be updated incrementally upon collision, besides, AIT* uses a black and white list approach to prevent re-checking of the collided connections.
%  will be incrementally updated when the collision happens, and a black and white list technique is utilized in AIT* to avoid re-checking the collided connection.
In our proposed method, since we do not need edges in the $\mathcal{T}_{\mathcal{R}}$ to be executable, we borrow this idea to use the $\mathcal{T}_{\mathcal{R}}$ to provide a relatively accurate cost-to-go estimation and guidance for sampling.

