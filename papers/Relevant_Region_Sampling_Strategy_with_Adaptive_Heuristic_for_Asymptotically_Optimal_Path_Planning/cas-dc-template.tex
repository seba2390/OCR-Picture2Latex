%% 
%% Copyright 2019-2020 Elsevier Ltd
%% 
%% This file is part of the 'CAS Bundle'.
%% --------------------------------------
%% 
%% It may be distributed under the conditions of the LaTeX Project Public
%% License, either version 1.2 of this license or (at your option) any
%% later version.  The latest version of this license is in
%%    http://www.latex-project.org/lppl.txt
%% and version 1.2 or later is part of all distributions of LaTeX
%% version 1999/12/01 or later.
%% 
%% The list of all files belonging to the 'CAS Bundle' is
%% given in the file `manifest.txt'.
%% 
%% Template article for cas-dc documentclass for 
%% double column output.

%\documentclass[a4paper,fleqn,longmktitle]{cas-dc}
\documentclass[a4paper,fleqn]{cas-dc}

%\usepackage[authoryear,longnamesfirst]{natbib}
%\usepackage[authoryear]{natbib}
\usepackage[numbers]{natbib}
\usepackage{graphics} % for pdf, bitmapped graphics files
%\usepackage{epsfig} % for postscript graphics files
%\usepackage{mathptmx} % assumes new font selection scheme installed
%\usepackage{times} % assumes new font selection scheme installed
\usepackage{amsmath} % assumes amsmath package installed
\usepackage{amssymb}  % assumes amsmath package installed
\usepackage{amsfonts}
\usepackage{xcolor}
\usepackage{epsfig}
\usepackage{graphicx}
\usepackage{subfigure}
\usepackage{epstopdf} % for eps figure using pdflatex
% \usepackage[linesnumbered,ruled,vlined]{algorithm2e}
\usepackage{algorithm} 
\usepackage{algpseudocode} 
\usepackage{balance}
\usepackage{stfloats}

\newcommand{\red}{\textcolor{red}}
\newcommand{\blue}{\textcolor{blue}}

%%%Author definitions
\def\tsc#1{\csdef{#1}{\textsc{\lowercase{#1}}\xspace}}
\tsc{WGM}
\tsc{QE}
\tsc{EP}
\tsc{PMS}
\tsc{BEC}
\tsc{DE}
%%%

\begin{document}
\let\WriteBookmarks\relax
\def\floatpagepagefraction{1}
\def\textpagefraction{.001}
\shorttitle{Biomimetic Intelligence and Robotics}
%\shortauthors{CV Radhakrishnan et~al.}

\title [mode = title]{Relevant Region Sampling Strategy with Adaptive Heuristic for Asymptotically Optimal Path Planning
}             

\address[1]{Department of Electronic Engineering, The Chinese University of Hong Kong, Shatin N.T., Hong Kong SAR, China}
\address[2]{Department of Electronic and Electrical Engineering, Southern University of Science and Technology, Shenzhen, China}
\address[3]{Shenzhen Research Institute of the Chinese University of Hong Kong, Shenzhen, China}

\author[1]{Chenming Li}[orcid=0000-0001-6322-0834]
% \credit{Conceptualization, Data curation, Formal analysis, Investigation, Methodology, Software, Visualization, Writing - original draft}
\ead{licmjy@link.cuhk.edu.hk}

\author[1]{Fei Meng}
% \cormark[1]
% \credit{Conceptualization, Investigation, Methodology, Validation, Writing - review \& editing}
\ead{feimeng@link.cuhk.edu.hk}

\author[1]{Han Ma}
% \cormark[1]
% \credit{Conceptualization, Investigation, Methodology, Validation, Writing - review \& editing}
\ead{hanma@link.cuhk.edu.hk}

\author[2]{Jiankun Wang}
% [orcid=0000-0001-9139-0291]
% \cormark[1]
% \credit{Conceptualization, Data curation, Formal analysis, Investigation, Writing - review \& editing}
\ead{wangjk@sustech.edu.cn}

\author[1,2,3]{Max Q.-H. Meng}
% [orcid=0000-0002-5255-5898]
% \credit{Funding acquisition, Project administration, Resources, Supervision, Writing - review \& editing}
\ead{max.meng@ieee.org}

%\cortext[cor1]{Corresponding author. 1155067752@link.cuhk.edu.hk}
% \cortext[cor1]{Jiankun Wang, Weinan Chen, Xiao Xiao and Yangxin Xu contributed eqaully to this work.}
\cortext[cor2]{This project is supported by Shenzhen Key Laboratory of Robotics Perception and Intelligence (ZDSYS20200810171800001) and the Hong Kong RGC GRF grants \# 14200618 awarded to Max Q.-H. Meng. \textit{(Corresponding authors: Jiankun Wang, Max Q.-H. Meng.)}}
  

\begin{abstract}
Sampling-based planning algorithm is a powerful tool for solving planning problems in high-dimensional state spaces. 
In this article, we present a novel approach to sampling in the most promising regions, which significantly reduces planning time-consumption.
The RRT\# algorithm defines the Relevant Region based on the cost-to-come provided by the optimal forward-searching tree. 
However, it uses the cumulative cost of a direct connection between the current state and the goal state as the cost-to-go.
To improve the path planning efficiency, we propose a batch sampling method that samples in a refined Relevant Region with a direct sampling strategy, which is defined according to the optimal cost-to-come and the adaptive cost-to-go, taking advantage of various sources of heuristic information.
% With our proposed method, the algorithm can generate new samples in the Relevant Region with a direct sampling method that takes advantage of various sources of heuristic information.
The proposed sampling approach allows the algorithm to build the search tree in the direction of the most promising area, resulting in a superior initial solution quality and reducing the overall computation time compared to related work.
To validate the effectiveness of our method, we conducted several simulations in both $SE(2)$ and $SE(3)$ state spaces.
And the simulation results demonstrate the superiorities of proposed algorithm.
\end{abstract}
%%%%%%%%%%%%%%%%%%%%%%%%%%%%%%%%%%%%%%%%%%%%%%%%%%%%%%%%%%%%%%%%%
%%%%%%%%%%%%%%%%%%%%%%%%%%%%%%%%%%%%%%%%%%%%%%%%%%%%%%%%%%%%%%%%%
%%%%%%%%%%%%%%%%%%%%%%%%%%%%%%%%%%%%%%%%%%%%%%%%%%%%%%%%%%%%%%%%%
%\begin{graphicalabstract}
%\includegraphics{figs/grabs.pdf}
%\end{graphicalabstract}

%\begin{highlights}
%\item Research highlights item 1
%\item Research highlights item 2
%\item Research highlights item 3
%\end{highlights}
%%%%%%%%%%%%%%%%%%%%%%%%%%%%%%%%%%%%%%%%%%%%%%%%%%%%%%%%%%%%%%%%%
%%%%%%%%%%%%%%%%%%%%%%%%%%%%%%%%%%%%%%%%%%%%%%%%%%%%%%%%%%%%%%%%%
%%%%%%%%%%%%%%%%%%%%%%%%%%%%%%%%%%%%%%%%%%%%%%%%%%%%%%%%%%%%%%%%%
\begin{keywords}
Path Planning \sep Asymptotical Optimality \sep Relevant Region \sep Adaptive heuristic
\end{keywords}

\maketitle

\section{Introduction}
\label{sec:Introduction}


The goal in top-$\size$ recommendation is to recommend to each
consumer a small set of $\size$ items from a large collection of
items~\cite{cremonesi2010performance}.  For example, Netflix may want
to recommend $\size$ appealing movies to each consumer.  Collaborative
Filtering (CF)~\cite{herlocker2002empirical,lee2012comparative} is a
common top-$\size$ recommendation method.  CF infers user interests by
analyzing partially observed user-item interaction data, such as user
ratings on movies or historical purchase
logs~\cite{kanagal2012supercharging}. The main assumption in CF is that
users with similar interaction patterns have similar interests.


Standard CF methods for top-$\size$ recommendation focus on making  suggestions  that accurately reflect the user's preference history. However, as  observed in previous work,  CF recommendations are generally biased toward  popular items, leading to a rich get richer effect~\cite{vargas2014improving,steck2011item}.  The major reasons for this are \textit{popularity bias} and \textit{sparsity} of CF interaction data (detailed in Section~\ref{sec:related-work}). In a nutshell, to maintain  accuracy, recommendations are generated from the dense regions of the data,  where the popular items lie.  

However,  accurately suggesting popular items, may not be satisfactory for the consumers. For example, in Netflix, an accuracy-focused movie recommender may recommend ``Star Wars: The Force Awakens'' to users who have seen ``Star Wars: Rogue One''.  But, those users are probably already aware of ``The Force Awakens''. Considering additional factors, such as novelty of recommendations,  can lead to more effective suggestions~\cite{cremonesi2010performance,Castells2015,zhang2008avoiding,ziegler2005improving,zhang2012auralist}. 
%Second, accuracy-focused models typically achieve a   overall item-space coverage across their recommendations,  whereas high item-space coverage helps providers of the items increase revenue
%, users satisfaction since they are  likely already aware of or can find these items on their own.  

Focusing on popular items also adversely affects the satisfaction of  the providers of the items. This is because  accuracy-focused models typically achieve a  low overall item space coverage across their recommendations, whereas   high item space coverage helps providers of the items increase their revenue~\cite{vargas2014improving,Castells2015,adomavicius2011maximizing,anderson2006thelongtail, yin2012challenging,adomavicius2012improving}.
%accuracy-focused models typically achieve a

In contrast to the relatively small number of popular items, there are copious  {\it long-tail\/} items that have fewer observations (e.g., ratings) available. More precisely,  using the Pareto  principle (i.e.,~the $80/20$ rule),  long-tail items can be defined as items that generate the lower $20\%$ of observations~\cite{yin2012challenging}. Experimentally we found that these items correspond to almost $85\%$ of the items in several datasets (Sections~\ref{sec:Notation} and \ref{sec:Experiments}). %Table~\ref{tab:DatasetStatsticsSmall})


As previously shown, one way to improve the novelty of top-$\size$ sets is to recommend interesting long-tail items~\cite{cremonesi2010performance,ge2010beyond}.  The intuition  is that since they have fewer observations available,  they are more likely to be unseen~\cite{Kaminskas:2016:DSN:3028254.2926720}.  
 %For example, in online commerce,  newly added items are long-tail items that are yet to be discovered.  
Moreover, long-tail item promotion also results in higher overall coverage of the item space%, which increases profits for providers of the items
~\cite{vargas2014improving,Castells2015,zhang2008avoiding,zhang2012auralist,adomavicius2011maximizing,anderson2006thelongtail,yin2012challenging,jambor2010optimizing}. Because long-tail promotion reduces accuracy~\cite{steck2011item}, there are trade-offs to be explored.


%original submitted to ICDE
%This work studies three aspects of top-$\size$ recommendation: accuracy, novelty, and item-space coverage, and examines their trade-offs. In most previous work, predictions of a base recommendation system are re-ranked to handle their trade-offs~\cite{adomavicius2012improving,jambor2010optimizing,zhang2013personalize,wang2009portfolio}. Due to performance considerations, however, these techniques are not customized per user. For example,  parameters that balance the trade-off between novelty and accuracy are cross-validated at a global level.  This can be detrimental since users have varying preferences for  objectives such as long-tail novelty. We explore how to  automatically infer  user  preference for long-tail novelty, and how to leverage  it to correct  the popularity bias in standard recommender models. Our work does not rely on any additional contextual data, although such data, if available, can help promote newly-added long-tail items~\cite{agarwal2009regression,Saveski:2014:ICR:2645710.2645751}.

This work studies three aspects of top-$\size$ recommendation: accuracy, novelty, and item space coverage, and examines their trade-offs. In most previous work, predictions of a base recommendation algorithm are \textit{re-ranked} to handle these trade-offs~\cite{adomavicius2012improving,jambor2010optimizing,zhang2013personalize,wang2009portfolio}. The re-ranking models are computationally efficient but suffer from two drawbacks. First, due to performance considerations,  parameters that balance the trade-off between novelty and accuracy  are not customized per user. Instead they are cross-validated at a global level.  This can be detrimental since users have varying preferences for  objectives such as long-tail novelty. Second,  the re-ranking methods are often limited to a specific base recommender  that may be sensitive to dataset density. 
As a result, the datasets are pruned and the problem is studied in dense settings~\cite{adomavicius2012improving,ho2014likes}; but real world  scenarios are often sparse~\cite{kanagal2012supercharging,liu2017experimental}.   
% Because  dataset density can impact the performance of most base recommenders (like R-SVD), which in turn affects the performance of the re-ranking model, 

\iffalse
We address these limitations by directly inferring  user  preference for long-tail novelty  from interaction data.  This  allows us to customize the re-ranking  per user, and design a \textit{generic} framework, which resolves the second problem. In particular, since the long-tail novelty preferences are estimated independently of any base  recommender model, we can  plug-in an appropriate base recommender w.r.t. the dataset sparsity.% including ones that are more suitable for sparse settings.  

Modelling  user  preference for  long-tail novelty using only item popularity statistics, e.g., the average popularity of rated items as in~\cite{jugovac2017efficient}, disregards additional information like whether the user found the item interesting and the long-tail preferences of other users  of the items. \iffalse To incorporate them, we introduce the notion of  \emph{item long-tail importance}. Both  user long-tail preferences and item long-tail importance are dependent:  a user has high preference for discovering long-tail items if she is interested in important long-tail items, and an item that is associated with many of these kinds of users is likely to be more important.  We propose a joint optimization framework to directly learn,  from interaction data, both the users' long-tail preferences and the  items' long-tail importance. \fi
We propose an optimization approach that  incorporates  this information and  directly learns,  from interaction data, the users' long-tail novelty preferences.

Next, we use these learned preferences  to design a  top-$\size$ recommendation framework thats is generic, and provides customized balance between accuracy, novelty, and coverage. We refer to it as framework as GANC.  Using GANC, we design a novel algorithm, {\it Ordered Sampling-based Locally Greedy (OSLG)\/}, that relies on the learned long-tail novelty preferences  to scalably correct for popularity bias. Our work does not rely on any additional contextual data, although such data, if available, can help promote newly-added long-tail items~\cite{agarwal2009regression,Saveski:2014:ICR:2645710.2645751}. In summary:
\fi

We address the first limitation by directly inferring  user  preference for long-tail novelty  from interaction data.   Estimating these  preferences  using only item popularity statistics, e.g., the average popularity of rated items as in~\cite{jugovac2017efficient}, disregards additional information, like whether the user found the item interesting or the long-tail preferences of other users  of the items. We propose an approach that  incorporates  this information and  learns the users' long-tail novelty preferences from interaction data.

This approach allows us to customize the re-ranking  per user, and  design a \textit{generic} re-ranking framework, which resolves the second limitation of prior work. In particular, since the long-tail novelty preferences are estimated independently of any base recommender, we can  plug-in an appropriate one w.r.t. different factors, such as the dataset sparsity.

Our top-$\size$ recommendation framework, \textbf{GANC}, is \textbf{G}eneric, and provides customized balance between \textbf{A}ccuracy, \textbf{N}ovelty, and \textbf{C}overage. % Moreover, based on the learned long-tail novelty preferences, we also design a novel algorithm, {\it Ordered Sampling-based Locally Greedy (OSLG)\/}, that relies on the learned long-tail novelty preferences  to scalably correct for popularity bias. 
Our work does not rely on any additional contextual data, although such data, if available, can help promote newly-added long-tail items~\cite{agarwal2009regression,Saveski:2014:ICR:2645710.2645751}. In summary:

%Consider  the following toy example:
\vspace{-0.2cm}
\begin{table}[htb]
\centering
\scriptsize
%\small
\begin{tabular}{ccccccc} 
%\toprule
%&\multirow{2}{*}{}&\multicolumn{7}{c}{Ratings}\\
& & \cellcolor{blue!35}$w_1$ &\cellcolor{blue!18} $w_2$ & $\dots$ &\cellcolor{blue!8} $w_{89}$  &\cellcolor{blue!8} $w_{99}$   
\\
&   &$i_1$&$i_2$&$\dots$&$i_{89}$&$i_{90}$\\ 
\cmidrule(r){3-7} 	 
%\midrule
\cellcolor{red!35}$\theta_1$  &$u_1 $   &5 &   & $\dots$ &  &   \\
\cellcolor{red!28}$\theta_2$  &$u_2$     &5 &    & $\dots$ &  &  \\
 $\theta_3=?$  &$\bf u_3$  &5 &  &   $\dots$ &  &  \\
\cellcolor{red!10}$\theta_4$ & $u_4$  &  &5   & $\dots$ & &\\ 
\cellcolor{red!10}$\theta_5$ & $u_5$  &  & 5  & $\dots$ & &\\ 
$\theta_6=?$  & $\bf u_6$ & &5  &      $\dots$& &  \\ 
 & & $\hdots$  &$\hdots$   &$\hdots$   &$\hdots$   &$\hdots$  \\
%\midrule 
\cmidrule(r){3-7} 	 
\multicolumn{2}{c}{item pop.}  & 3  & 3  & $\dots$ &50&60\\  
%\bottomrule
%$ f_i$    &3  &3  &1  &3  &1  &2  \\  \hline
\end{tabular}
%#.
\caption{Simplified user-item interaction data. The user long-tail novelty preference ($\theta_u$), item long-tail importance weight ($w_i$) are highlighted. Darker colors indicate larger values. } \label{tab:example}
\end{table} 
\vspace{-0.2cm}
\begin{example}  
In Table~\ref{tab:example}, we are interested in estimating $\theta_3$ and $\theta_6$,  the long-tail preference of users $u_3$ and $u_6$ who have each rated a single movie. Additional ratings for other users  are not included here.  Considering only rating information, we observe $i_1$ and $i_2$ are  equally popular $|\mathcal{U}_{i_1}^{\trainset}| = |\mathcal{U}_{i_2}^{\trainset}|=3$, and $r_{31}=5$ and $r_{62}=5$. Using Eq.~\ref{eq:tfidf-risk}  we have $\theta_3 = \theta_6$. However, if we were given the long-tail preferences of the each item's user set, specifically that $u_1$ and $u_2$ have high long-tail preference (darker red), while $u_4$ and $u_5$ have lower long-tail preference (lighter red), we could conclude $i_1$ is a more important long-tail item compared to $i_2$ (indicated by a darker blue shade for $w_1$), and we expect  $\theta_3 \geq \theta_6$.

% On the other hand, if we knew that $u_4$ and $u_5$ have lower long-tail preference, we could conclude $i_2$ is a  less significant long-tail item. Therefore, However, if we  consider the long-tail preferences of other users, we may reason differently.    We need another variable $w_i$ which captures this information. 
%we would conclude that $u_3$ has higher long-tail preference compared to $u_6$, since the users $i_1$ is a more prominent long-tail item. 

% Relying only  on item popularity information, we would  conclude   $u_3$ and $u_6$ have equal long-tail preference, since $i_1$ and $i_2$ are  equally popular. However, considering  the second column,  long-tail preference of users,  long-tail importance for each item,  which captures the long-tail preference of its users. Since  that  both users of $i_1$ have high long-tail preference while  the users of $i_2$ have lower preference,  we may conclude $i_1$ is a more important long-tail item compared to $i_2$. Therefore, $u_3$'s long-tail preference should be at least as large as $u_6$'s preference. Specifically, consider two  items $i_1$ and $i_2$, with the following rating data: $i_1=\{u_1:5, u_2:5, u_3:5 \}$, $i_2=\{u_4:5, u_5:5, u_6:5\}$.  

%Table~\ref{tab:example} shows  simplified rating data. We want an estimate of the long-tail preference of $u_3$ and $u_6$, who have each  rated a single movie.  Relying only  on movie popularity information, we would  conclude   $u_3$ and $u_6$ have similar long-tail preference, since $m_1$ and $m_2$ are  equally popular. However, considering the long-tail preferences of other users of those movies, we may reason differently: since $u_1$ and $u_2$ have high long-tail preference, and $u_4$ and $u_5$ have low long-tail preference, $m_1$ is a more prominent long-tail item compared to $m_2$. Therefore, it is likely that $u_3$ has higher long-tail preference compared to $u_6$.considering the long-tail preferences of other users of those movies, we may reason differently.  For example, 
\label{ex:running}
\end{example}



%------------------------------

\iffalse
\begin{example}
Table~\ref{tab:example} shows rating data for a simplified system. %Note the user-item interaction matrix is sparse.
For this example, we define popular movies as those that have received  three or more ratings; $\{m_1, m_2, m_4\}$ are popular and  $\{m_3, m_5, m_6\}$ are niche movies. We observe $u_1$ and $u_3$  have rated relatively popular movies (risk-averse) while $u_2$ and $u_4$ have rated niche movies (risk-loving). 
\label{ex:running}
\end{example}

\begin{table}[htb]
\centering
\scriptsize
\begin{tabular}{ccccccc} 
\toprule
			&$m_1$ &$m_2$   &$m_3$    &$m_4$   &$m_5$ &$m_6$  \\ \hline 
$u_1 $ &5  &4  & - &-  &-  &-   \\
$u_2$  &-  &-  &-  &-  &5  &5   \\
$u_3$  &-  &4  &-  &5  &-  &-   \\
$u_4$  &-  &-  &3  &-  &-  &4   \\ 
$u_5$  &5  &-  &-  &3  &-  &-   \\ 
$u_6$  &4  &2  &-  &4  &-  &-   \\ 
\bottomrule
%$ f_i$    &3  &3  &1  &3  &1  &2  \\  \hline
\end{tabular}
\caption{User-Movie rating data} \label{tab:example}
\end{table}

It is essential to consider consumer characteristics in designing recommender systems so that they promote long-tail items to the right group of users and spread demand evenly between hit and niche items.  

\fi





%------------------------------
\iffalse
\begin{table}[htb]
\centering
\scriptsize
\begin{tabular}{ccccccc} 
\toprule
			&$m_1$ &$m_2$   &$m_3$    &$m_4$   &$m_5$ &$m_6$  \\ \hline 
$u_1 $ &\textbf{5}  & \textbf{4}  &\textcolor{gray}{ 1.2} &-  &-  &-   \\
$u_2$  &-  &-  &-  &-  & \textbf{5}  &\textbf{5}   \\
$u_3$  &-  &\textbf{4}  &-  &\textbf{5}  &-  &-   \\
$u_4$  &-  &-  &\textbf{3}  &-  &-  &\textbf{4}   \\ 
$u_5$  &\textbf{5}  &-  &-  &\textbf{3}  &-  &-   \\ 
$u_6$  &\textbf{4}  &\textbf{2}  &-  &\textbf{4}  &-  &-   \\ 
\bottomrule
%$ f_i$    &3  &3  &1  &3  &1  &2  \\  \hline
\end{tabular}
\caption{User-Movie rating data} \label{tab:example}
\end{table}
% $\mathcal{P}^1= \{ \mathcal{P}_1^1 \{i_1,i_2,i_3\}, \mathcal{P}_2^1:\{i_2,i_3,i_5\}  \}$
 %$\mathcal{P}^2= \{ \mathcal{P}_1^2: \{i_1,i_2,i_3\}, \mathcal{P}_2^2:\{i_2,i_5,i_6\}  \}$
 %$\mathcal{P}^3= \{ \mathcal{P}_1^3: \{i_7,i_8,i_9\}, \mathcal{P}_2^3:\{i_{10},i_{11},i_{12}\}  \}$
\begin{table}[htb]
\centering
\tiny
\begin{tabular}{ccc} 
\toprule
		&$u_1$&$u_2$  \\ \hline 
$\mathcal{P}^1 $ & $\{i_1,i_2,i_3\}$ & $\{i_2,i_3,i_5\} $ \\
$\mathcal{P}^2$ & $\{i_1,i_2,i_3\}$ & $\{i_2,i_5,i_6\} $ \\
$\mathcal{P}^3$ & $\{i_7,i_8,i_9\}$ & $\{i_{10},i_{11},i_{12} \}$ \\
\bottomrule
%$ f_i$    &3  &3  &1  &3  &1  &2  \\  \hline
\end{tabular}
\caption{Top-$\size$ allocations to users.} \label{tab:paretoExamples}
\end{table}
\fi


\iffalse
When considering long-tail items, it is important to consider consumers' willingness  to explore niche or unpopular items and their propensity towards similar items. In particular, they can be characterized by their  {\it risk degree\/} and {\it focusing degree\/}, respectively.  We compute these estimates  based on historical rating information. The following example further describes these notions in the context of movie rating data. 

\begin{example}  
Table~\ref{tab:example} shows rating data for a simplified system with $6$ users, $6$ movies, and $3$ genres. $m_i^{j}$ implies that movie $m_i$ belongs to genre $j$. Note the user-item interaction matrix is sparse. 
  For this setting, we define popular movies as those that have received  three or more ratings; $\{m_1, m_2, m_4\}$ are popular and  $\{m_3, m_5, m_6\}$ are niche movies. We now profile the users according to their risk and focusing degree. E.g., $u_1$ has rated relatively popular movies belonging to the same genre (risk-averse, high focusing degree); $u_2$ has rated niches movies in the same genre (risk-loving, high focusing degree); $u_3$ has rated popular movies in two different genres (risk-averse, low focusing degree), and $u_4$ has rated niches movies in two different genres (risk-loving, low focusing degree). 
\label{ex:running}
\end{example}
\begin{table}[htb]
\centering
\tiny
\begin{tabular}{ccccccc} 
\toprule
			&$m_1^{1}$ &$m_2^{1}$   &$m_3^{2}$    &$m_4^{3}$   &$m_5^{3}$ &$m_6^{3}$  \\ \hline 
$u_1 $ &5  &4  &-  &-  &-  &-   \\
$u_2$  &-  &-  &-  &-  &5  &5   \\
$u_3$  &-  &4  &-  &5  &-  &-   \\
$u_4$  &-  &-  &3  &-  &-  &4   \\ 
$u_5$  &5  &-  &-  &3  &-  &-   \\ 
$u_6$  &4  &2  &-  &4  &-  &-   \\ 
\bottomrule
%$ f_i$    &3  &3  &1  &3  &1  &2  \\  \hline
\end{tabular}
\caption{User-Movie rating data} \label{tab:example}
\end{table}
It is essential to consider these consumer characteristics in designing recommender systems so that they promote long-tail items to the right group of users and spread demand evenly between the hit and niche items.  
\fi
\iffalse
\begin{center}
\begin{figure*}[tp]
%\scalebox{0.5}{%
\resizebox{1\textwidth}{!}{%
%\small%\addtolength{\tabcolsep}{5pt}% below sums to 8
\begin{tabularx}{1.5\textwidth}{>{\hsize=2.5\hsize}X>{\hsize=2.5\hsize}X>{\hsize=0.5\hsize}X>{\hsize=0.5\hsize}X>{\hsize=0.5\hsize}X>{\hsize=0.5\hsize}X>{\hsize=0.5\hsize}X>{\hsize=0.5\hsize}X}
    \multirow{12}{*}{\includegraphics[scale=0.3]{codeForExample/popularity-movie.png}} & \multirow{12}{*}{\includegraphics[scale=0.3]{codeForExample/scatterplot.png}} & & & & & & \\
%   & &               &       &       &       &       &       \\
    & &\multicolumn{1}{l|}{}               &$m_1^{g1}$   	&$m_2^{g1}$    	&$m_3^{g2}$    &$m_4^{g2}$      &$m_5^{g3}$    \\ \cline{3-8}%\hline
    & &\multicolumn{1}{l|}{u1}          &5  &5  &-  &-   &-  \\
    & &\multicolumn{1}{l|}{u2}    		&-  &-  &4  &4  &5  \\
    & &\multicolumn{1}{l|}{u3}   			&1  &2  &1  &-  &-   \\
    & &\multicolumn{1}{l|}{u4}     		&1  &-  &-  &-  &-  \\
    & &               &       &       &       &       &       \\
    & &               &       &       &       &       &       \\
    & &               &       &       &       &       &       \\
    & &               &       &       &       &       &	\\
    \\
\end{tabularx}}
\caption{User-Movie interaction data a) Popularity-Movie histogram b)Movie genres/clusters c) User-Movie rating data} \label{fig:example}
\end{figure*}
\end{center}
\fi



%We propose a novel approach that allows us to  promote long-tail items in a targeted manner, thereby improving the novelty of top-$\size$ sets, the overall item-space coverage across recommendations, while maintaining reasonable levels of accuracy.

%Next, we integrate these learned preferences  in a generic  top-$\size$ recommendation framework to provide customized balance between accuracy and coverage.

%sequentially make recommendations, while adjusting its parameters with regard to the set of top-$\size$ recommendations made so far. However, since  sequential parameter updates  cause  scalability issues, we propose a sampling based algorithm. This variant of our framework, called {\it Ordered Sampling-based Locally Greedy (OSLG)\/},  allows us to  correct for the popularity bias in recommendations with regard to individual user long-tail preferences. 

%ICDE submission
%Our framework differs with  prior work in the following aspects:  unlike~\cite{adomavicius2011maximizing,adomavicius2012improving,zhang2013personalize,ho2014likes},  the long-tail preference personalization in our framework is learned rather than optimized using cross-validation or parameter tuning. In other words, our personalization method is independent of the underlying base  recommendation models.  Moreover, our framework is  generic. This enables us to  plug-in several base recommenders, and evaluate their  effectiveness without requiring  extensive tuning for the accuracy and coverage trade-off. 


%\vspace{-2.8pt}
\begin{itemize}

\item  We examine various measures for estimating user long-tail novelty preference in Section~\ref{sec:lt-pref} and formulate an optimization problem  to directly learn users' preferences for long-tail  items from interaction data in Section~\ref{sec:learning-lt-pref}. %In addition, we introduce several heuristics for measuring the user preference for less common items from historical rating data.% 

\item  We integrate the user preference estimates into GANC %, a generic re-ranking framework that provides customized balance between accuracy, novelty, and coverage 
(Section~\ref{sec:RiskbasedReranking}), and  introduce {\it Ordered Sampling-based Locally Greedy (OSLG)\/}, a scalable algorithm that relies  on user long-tail preferences to correct the popularity bias (Section~\ref{sec:optimizationAlgorithm}).
%We introduce OSLG, a scalable algorithm that relies  on user long-tail preferences to  maximize item space coverage \textcolor{red}{while maintaining acceptable levels of accuracy} (Section~\ref{sec:optimizationAlgorithm}).

\item   We conduct an extensive empirical study and evaluate performance from  accuracy, novelty, and coverage perspectives (Section~\ref{sec:Experiments}).  We use five  datasets with varying density and difficulty levels. %:  Netflix, MovieTweetings, and MovieLens (100K, 1M, 10M). 
  In contrast to most related work,  our evaluation considers realistic settings that include a large number of infrequent  items and users. %This enables us to study the impact of  data density on the performance trade-offs of several  state of the art top-$\size$ recommendation algorithms. %   %,  and use the all-items ranking protocol~\cite{steck2013evaluation,vargas2014improving}, where performance is measured using all items with train data. to evaluate the performance of several  state of the art top-$\size$ recommendation algorithms 
 
\item Our empirical results confirm that the performance of re-ranking models is impacted by the underlying   base recommender and the dataset density. Our generic approach enables us to easily incorporate a suitable base recommender to devise an effective solution for both dense and sparse settings. In dense settings, we use the same base recommender as existing re-ranking approaches, and we outperform them in accuracy and coverage metrics. For sparse settings, we plug-in a more suitable base recommender, and devise an effective solution that is competitive with existing top-$\size$ recommendation methods in accuracy and novelty. 

%Directly estimating the long-tail novelty preferences allows us to customize re-ranking per user, and  devise a generic framework.   
 
\end{itemize}

Section~\ref{sec:related-work} describes related work. Section~\ref{sec:conclusion} concludes.


\section{Related Work}


\subsection{Sampling-based Motion Planning Method}

Plenty of modifications are proposed to enhance the performance of the RRT algorithm \cite{lavalle1998rapidly} such as the RRT* algorithm \cite{karaman2011sampling}.
The rewiring stage of the RRT* only rewires locally, which means the global optimization of the current tree is ignored. 
The RRT\# \cite{arslan2013use} proposes to find the global optimality in each rewiring stage with dynamic programming.
Dynamic programming is also used in the Fast Marching Tree (FMT*) method \cite{janson2015fast} to grow the searching tree.
% And the FMT* introduces the thought of batch sampling into the robot motion planning field.
The Informed sampling strategy \cite{gammell2014informed, gammell2018informed} is proposed to overcome the drawback of uniform sampling.
% It can accelerate the convergence speed significantly with very little computation consumption. 
% The Informed sampling strategy uses a direct sampling method to generate samples in the $L_2$-Informed set.
% An advanced version of the Informed sampling strategy is proposed in \cite{gammell2018informed}, which includes the graph pruning stage to keep a relatively constricted tree. 
% A tree with fewer vertices means that the cost reduction in finding the nearest tree vertex. 
Using the neural network to reinforce the sampling stage to enhance the sampling efficiency \cite{wang2020neural, li2021efficient, qureshi2019motion} is proved as a promising technique.

% In the human-robot coexisting environment, the planning problem become more complex than that in the static environment \cite{wang2020eb}.
% The objective of optimization needs to consider safety, efficiency, and human feelings.


% The Informed sampling strategy can constraint the whole planning procedure in a subset of the whole state space, the $L_2$-Informed set, and the Lebesgue measure of the $L_2$-Informed set decreases as the solution improves.
% \\ \textcolor{red}{*SWIRRT*: maybe not including the SWIRRT is better}




\subsection{Batch Sampling Technique}

% A batch sampling method is described in the FMT* method \cite{janson2015fast}.
The FMT* \cite{janson2015fast} introduces the thought of batch sampling into the robot motion planning field.
The FMT* samples a batch of points and constructs the searching tree according to this batch of samples.
% The asymptotic optimal of the FMT* is guaranteed when the size of the batch goes to infinity. 
The Batch Informed Trees (BIT*) \cite{gammell2015batch, gammell2020batch} method is developed based on the Informed RRT*, besides, the BIT* absorbs the thoughts in the FMT* method \cite{janson2015fast} and the Lifelong Planning A* (LPA*) algorithm \cite{koenig2004lifelong}.
The Regionally Accelerated Batch Informed Trees (RABIT*) \cite{choudhury2016regionally} aims to solve the difficult-to-sample planning problem, like the narrow passage problem.
% The RABIT* uses the Covariant Hamiltonian Optimization for Motion Planning (CHOMP) method as its local optimizer, and the local optimizer will exploit the local information.
The Fast-BIT* \cite{holston2017fast} modifies the edge queue and searches the initial solution more aggressively. 
The Greedy BIT* \cite{chen2021greedy} uses the greedy searching method to generate the initial solution faster and accelerate the convergence speed.
But these greedy-based methods often fail to assist the searching procedure without an accurate heuristic estimation method. 
The Adaptively Informed Trees (AIT*) \cite{strub2020adaptively} and the Advanced BIT* (ABIT*) \cite{strub2020advanced} proposed by Strub and Gammell are developed based on the BIT* as well. 
The AIT* calculate a relatively accurate heuristic estimation with a lazy reverse-searching tree.
The ABIT* proposes to utilize inflation and truncation to balance the exploitation and exploration in the increasingly complex Random Geometric Graph (RGG) \cite{penrose2003random}.
Though the AIT* and the ABIT* achieve significant improvements, their sampling regions are not compact enough, and the sampling efficiency will be critically low in the complex environment.


\subsection{Relevant Region Sampling Strategy}

The concept of 'relevant' is first proposed in the searching-based robot path planning method like the A* \cite{hart1968formal}. 
In the A* algorithm, the set of expanded vertices is relevant to the query, such that the A* algorithm could expand a smaller set of vertices than the Dijkstra's algorithm \cite{dijkstra1959note}.  
% And it is also not a new idea in the sampling-based planning field. 
% The Relevant Region is formally defined in \cite{arslan2013use}. 
% The Relevant Region related vertices are the vertices of which the sum of the optimal cost-to-come and the heuristic is less than the cost of the current optimal solution.
% Since the Relevant Region is the most promising region that could help to improve the solution, so a straightforward modification is to reduce the chance of sampling outside the  Relevant Region.
The concept of the Relevant Region is formally introduced in \cite{arslan2013use}, whose sum of the optimal cost-to-come and cost-to-go heuristic is less than the cost of the current optimal solution. 
Since the Relevant Region is the most promising area for improving the solution, a straightforward modification would be reducing the likelihood of sampling outside of it.
Three different metrics are used to achieve this in \cite{arslan2015dynamic}, the modified versions achieve better performance in the convergence speed than the RRT\# method.
The methods described in \cite{arslan2013use} and \cite{arslan2015dynamic} use the rejection method for sampling, which is not efficient since the Relevant Region is a small subset of the whole state space in most scenarios.
The direct sampling method is illustrated to overcome this drawback, and the details are described in \cite{joshi2020relevant}.
However, they all use the cumulative cost along the direct connection between the current state and the goal state as the cost-to-go. 
This approach results in inaccurate estimated cost-to-go in most scenarios.
% Their ordered priority queues are also far from the ground truth.

% The optimal cost-to-come value can be defined as the vertex which has the lowest optimal cost-to-come value in the destination region. 


\subsection{Bi-directional Searching Method}

The RRT and RRT* methods may not always discover a solution within the allotted time, particularly when dealing with narrow passages
The RRT-Connect \cite{kuffner2000rrt} is proposed to find the initial solution faster. 
% It grows two trees from the source point and the goal region simultaneously.
% It is proved that the RRT-Connect can achieve better performance than the RRT.
However, the approach described in \cite{kuffner2000rrt} is not asymptotically optimal. Therefore, its successor, RRT-Connect, is also not asymptotically optimal. 
To overcome this, an enhanced version of the bidirectional searching RRT is introduced in \cite{klemm2015rrt} to guarantee the asymptotical optimality.
To take advantage of the benefits of bi-directional search, the kinematic constraints are taken into consideration in the bi-directional search method to generate executable trajectories efficiently \cite{wang2021kinematic}.
% The method described in \cite{klemm2015rrt} is an asymptotically optimal single-query version of the RRT-Connect, called the RRT*-Connect. 
% The RRT*-Connect provides asymptotically optimal guarantee like the RRT*, and its efficiency and robustness are proofed in real-world experiments.
% In addition, the bi-directional searching method can be used to combine with the kinematic constraints , which is essential in generating executable trajectory.

% In the RRT-Connect, one tree is extended in each iteration and tries to connect itself to the other tree; this manner will attempt to grow the trees towards each other.

One drawback of the Informed RRT* \cite{gammell2014informed} \cite{gammell2018informed} is that it uses the RRT* to search the whole state space before finding the initial solution.
Therefore, the Informed RRT* often fails to find the solution in the required period, same as the RRT*.
By combining the advantages of both the Informed and the RRT*-Connect, the Informed RRT*-Connect \cite{2020Informed} proposes to use the RRT*-Connect to generate the initial solution and use the Informed sampling strategy to constrain the sampling region after the initial solution is found.
% It combines the advantages of both the Informed RRT* and the RRT*-Connect.
% The Informed RRT*-Connect can achieve a much higher success rate in its simulations than the Informed RRT*.
Besides, the AIT* \cite{strub2020adaptively} can also be viewed as a bi-directional searching method.





\section{Problem Definition}

Consider the state space $\mathcal{X}$, which is the subset of $\Re^d$.
$\Re^d$ is the whole $d$-dimensional space, and $d$ is a positive integer.
$\mathcal{X}_{obs}$ shows the space occupied by the obstacles, the free space is defined as $\mathcal{X}_{free} = \mathcal{X} \setminus \mathcal{X}_{obs} $.
The $x \in \mathcal{X}$ represents any state in the state space.
The source point $x_{start}$ is the initial state of the robot.
The destination is a region represented by $\mathcal{X}_{goal}$.
The source point and the destination in a valid planning problem must be defined within the free space $x_{start} \in \mathcal{X}_{free} \ \& \ \mathcal{X}_{goal} \subseteq \mathcal{X}_{free}$.
The motion planning problem is defined as:

% The robot motion planning problem refers to find a trajectory that could move the robot from the source point to the destination, and the trajectory should satisfy the constraints. 


\begin{equation}
\begin{aligned}
& \pi \in [0, 1] \to \mathcal{X}_{free}, \\  
& s.t. \ \pi(0) = x_{start}, \ \pi(1) \in \mathcal{X}_{goal}, \ 
\pi(s) \in \mathcal{X}_{free},
\forall s \in [0, 1] .
\label{MotionPlanningDefinition}
\end{aligned}
\end{equation}


The optimization objective can be minimizing the trajectory length, maximizing the minimum clearance, or any objects that could be mathematically defined. 
It can also be set as the sum of individual optimization objectives and form a hybrid optimization objective.
For simplicity, the optimization objective in the proposed method is set as minimizing the trajectory length.
Note that it could be extended to meet the specific planning requirement.
The $v \in \mathcal{T}$ denotes any vertex in the tree.
Assume there are two vertices $v_1, \ v_2 \in \mathcal{T}$, where $v_2$ is the descendant of $v_1$.
The cost between any two vertices $v_1, \ v_2$ is calculated with the intergal cost along the tree from $v_1$ to $v_2$, denotes as $d_{\mathcal{T}}(v_1, \ v_2)$.
Let $\Pi$ denote the set of all feasible solutions.
With the definition of trajectory cost, the optimization objective in our method is written as:


\begin{equation}
    \begin{aligned}
    & \pi* = \operatorname*{\arg \min}_{\pi \in \Pi} d_{\mathcal{T}}(v_{start}, v_{goal}) \\
    & s.t. \ \pi(0) = v_{start}, \ \pi(1) \in \mathcal{X}_{goal}, \pi(s) \in \mathcal{X}_{free},  \forall s \in [0, 1] .
\label{TrajectoryLengthOptimizationObjective}
\end{aligned}
\end{equation}
    

The cost estimation between any two states $x_1, x_2 \in \mathcal{X}$ can be set as the cumulative cost along the direct connection. 
The cost for the unit distance is usually deemed as $1$, the heuristic can be calculated with the Euclidean metric: $ h(x_1, x_2) = \left\| x_2 - x_1 \right\|_2$.
However, using this metric as the heuristic estimation method between any two states is not promising.
The direct connection is highly likely to collide with the obstacle, which can mislead the searching procedure, especially in complex environments.


The RRT algorithm contains two stages: the sampling and the searching stage. The sampling stage can be viewed as the abstracting process of the state space $\mathcal{X}$, and the searching stage will construct the searching tree $\mathcal{T}$ based on this abstracted state space. 
These two phases are performed alternatively in each iteration. 
We use the $c_{cur}$ to represent the cost of the current optimal solution. 
Before finding the initial feasible solution, the $c_{cur}$ is set to an infinitely large number $+\infty$.



\subsection{Adaptive Heuristic Estimation}

The adaptive heuristic estimation method was first proposed by the AIT* algorithm \cite{strub2020adaptively}, which is able to provide a problem-specific heuristic without overburdening computation resources.
We borrowed this idea to provide a fairly accurate estimation of the cost-to-go, and used it to guide the algorithm to take samples in relevant regions.

Since calculating the heuristic along the direct connection is inaccurate, the planner uses a lazy reverse-searching tree $\mathcal{T}_{\mathcal{R}}$ to provide the cost-to-go estimation $h_{\mathcal{T}_{\mathcal{R}}}(x)$ of any state $x$ in the state space, which is analogous to the AIT* algorithm \cite{strub2020adaptively}.
The $\mathcal{T}_{\mathcal{R}}$ is constructed without edge evaluation because edge evaluation is the most time-consuming procedure in the majority of motion planning scenarios.
To construct the $\mathcal{T}_{\mathcal{R}}$, the planner separates the sampling stage and searching stage into two separate modules instead of performing them alternatively in each iteration.
In the sampling stage, the planner uses our sampling strategy to generate a batch of sampling points in $\mathcal{X}_{free}$. 
Then, in the searching stage, the planner constructs the $\mathcal{T}_{\mathcal{R}}$ and $\mathcal{T}_{\mathcal{F}}$ in terms of the current RGG.
% A conceptual illustration is provided in Fig. \ref{schematicIllustration}.
Since the $\mathcal{T}_{\mathcal{R}}$ is constructed without edge evaluation, the edges in the $\mathcal{T}_{\mathcal{R}}$ may collide with the obstacles or not satisfy the constraints.
To deal with this, the $\mathcal{T}_{\mathcal{R}}$ will be updated incrementally upon collision, besides, AIT* uses a black and white list approach to prevent re-checking of the collided connections.
%  will be incrementally updated when the collision happens, and a black and white list technique is utilized in AIT* to avoid re-checking the collided connection.
In our proposed method, since we do not need edges in the $\mathcal{T}_{\mathcal{R}}$ to be executable, we borrow this idea to use the $\mathcal{T}_{\mathcal{R}}$ to provide a relatively accurate cost-to-go estimation and guidance for sampling.



\section{Methodology}

% \begin{figure}[!t]
% \centering
% \includegraphics[width=\linewidth]{fig/FLDv3.pdf}
% \caption{Procedure of FLD}
% \label{Overview}
% \end{figure}

\begin{figure*}[!t]
\centering
\includegraphics[width=.8\linewidth]{fig/FLDv1.pdf}
\caption{Procedure of FLD}
\label{Overview}
\end{figure*}



\subsection{Motivation}
According to the attack analysis in constrain-and-scale~\cite{howtobackdoor}, the federated backdoor attacks are divided into two attack scenarios: single-shot attacks (\textit{Attack A-S}) and multi-shot attacks (\textit{Attack A-M})~\cite{xie2020dba,am1}.
\setlist[itemize]{leftmargin=*}
%\begin{itemize}

\noindent\textbf{\textit{Attack A-S}}: the attacker successfully embeds its backdoor trigger in only one round. The attacker performs parameter scaling on the compromised clients' updates to substitute the global model $G^{t}$ with a backdoor model $X$ in Equation~\ref{eq:backdooreq1}: 
\begin{equation}
\label{eq:backdooreq1}
X = {\sum_{i=1}^{n}} \frac{1}{n} w_{i}^{t}.
\end{equation}
To achieve this goal, the attacker can scale the model parameters as follows:
\begin{equation}
\begin{split}\label{eq:backdooreq2}
\tilde{w} _{n}^{t} 
& = nX-\sum_{i=1}^{n-1} w_{i}^{t}\approx nX-\sum_{i=1}^{n-1} G ^{t-1}\\
& = n\left ( X-G ^{t-1} \right )+G ^{t-1}.
\end{split}
\end{equation}
As the global model converges, $w_{i}^{t}\approx  G ^{t-1}$. In other words, the attacker scales up the model weights $X$ by $n$ to prevent the malicious updates from being mitigated by the aggregation.


\noindent\textbf{\textit{Attack A-M}} lets compromised clients accumulate malicious updates over multiple rounds, instead of directly scaling the uploaded parameters, to avoid being detected by the defense algorithm. 
%the malicious client does not scale the update, thus avoiding detection by the defence.The malicious clients are selected in multiple rounds thus accumulating malicious updates. 
%\end{itemize}
%After reviewing the backdoor defenses methods in FL, we found that the past defenses basically defaulted to the attacker performing \textit{Attack A-S} and ignoring \textit{Attack A-M}. However, after experiments, we found that the attacker can successfully embed the back model in the global model by using Attack A-M, and in this case, many defenses will fail, such as clipping, Euclidean distance, etc. 
We have thoroughly reviewed the SOTA backdoor defense works in federated learning~\cite{foolgold,krum,Trimmed_Mean,Bulyan,RFA,canyou} and found that existing methods focus on defending against \textit{Attack A-S} while overlooking \textit{Attack A-M}. Unsurprisingly, we found through empirical experiments that SOTA defense algorithms fail at defending against {\textit{Attack A-M}.
%\subsection{Overview}

To address the challenges mentioned above, we propose Federated Layer Detection~(FLD), an innovative defense method for effectively detecting backdoor attacks in federated learning that overcomes the deficiencies of previous works. As depicted in Fig.~\ref{Overview}, FLD consists of two components, namely Layer Scoring and Anomaly Detection. Layer Scoring assigns scores to the local models uploaded by the clients according to the concept of isolation. Anomaly Detection checks the outlier scores given by Layer Scoring to determine if an uploaded model is compromised. The overall process is as follows: 1) The server receives the local models from the clients participating in the current round. 2) The server assigns each layer of each model an outlier score using Layer Scoring. 3) The server labels each layer as abnormal or not according to its outlier score, and excludes from the aggregation the anomalous models that contain more anomalous layers than the predefined threshold, as summarized in Algorithm~\ref{algorithm1}.
\begin{algorithm}
    \caption{Overview}\label{algorithm1}
    \begin{algorithmic} [1]
        \algrenewcommand\algorithmicrequire{\textbf{Input:}}
        \Require Set of clients $ C = \left \{ C_{1},C_{2},C_{3},\cdots,C_{N}  \right \} $, local datasets $ D = \left \{ D_{1},D_{2},D_{3},\cdots,D_{N} \right \}$, the number of training iterations $T$, the percentage of participating clients per round $K$.
        \algrenewcommand\algorithmicrequire{\textbf{Output:}} 
        \Require Global model $G^T $
        \State Initialize the global model $G^0 $
        \For { $ t\ in \left [ 1,T \right ] $ }
        \State $ n\gets \max \left ( K\cdot N,1  \right ) $ 
        \State $ C^{t} \gets $ (random set of $n$ clients )
        \For {each client $ i\in C^{t} $ in parallel} 
        \State The server sends $G^{t-1}$ to client $i$
        \State $w_{i}^{t} \gets $ ClientUpdate$\left ( D_i, G^{t-1}  \right) $ 
        \State Client $i$ sends $w_{i}^{t}$ back to the server
        \EndFor
        \State $\left(S_{1} ,\cdots,S_{n}\right)\gets Layer Scoring\left(w_{1}^{t},\cdots,w_{n}^{t}\right)$
        \State $C_{b}^{t}\gets Anomaly Detection \left(S_{1} ,\cdots,S_{n}\right)$
        \State  $m\gets len\left ( C_b^t \right ) $
        \State  $  G^{t}=  \sum_{i\in  C_b}^{} \frac{1 }{m} w_{i}^{t}$
        \EndFor
    \end{algorithmic}
\end{algorithm}

%\subsection{Federated Layer Detection Design}

%
\subsection{Layer Scoring}
In round $t$, the parameter server sends the global model $G^{t-1}$ to the selected clients $i\in C^t$, each of which trains $G^{t-1} $ using its local data $D_{i}$ and sends the model parameters $w_{i}^{t}$ back to the server after local training is completed.
%Our aim is to eliminate the threat of potentially malicious local models.

Existing backdoor defense methods assess uploaded models based on either similarity or distance metrics. They usually flatten the parameters of each model layer and then stitch them into a vector to perform the assessment. However, different layers of the neural network have heterogeneous parameter value distribution spaces due to their different functions. For example, in a CNN, the lower layers learn to detect simple features such as stripes, the middle layers learn to detect a part of an object, and the higher layers learn to detect a concept (e.g., a person)~\cite{distribution}. As a result, directly flattening and splicing the parameters of each layer easily leads to the loss of important information and hence the escape of malicious models. Therefore, finer-grained detection is demanded.

To address this issue, we have devised a hierarchical detection method, Layer Scoring, to measure fine-grained model differences, as shown in Algorithm~\ref{algorithm2}. Layer Scoring examines and assigns an outlier score to each layer of the uploaded models in turn. To provide accurate scores, the outlier scoring method is crucial and faces the following challenges:
%The method , and for layer $j$, we use an outlier detection \textcolor{blue}{method} to detect layer $j$ of all uploaded models, assigning an anomaly score to each $j$ layer of each model. \textcolor{blue}{This outlier detection method faces two challenges:}

\noindent\textbf{C1}: \textbf{The proportion of compromised clients is unknown.} Many existing works on outlier detection reply to the impractical assumption of knowing the proportion of compromised clients in advance, which severely limits their applicability in reality. To address this limitation, in this work, we propose an algorithm without requiring such prior knowledge. As such, conventional outlier detection methods such as K-Nearest Neighbors (KNN) and One-Class SVM, which rely on prior knowledge of the number of neighbors, are not feasible. 

%Knowing the proportion of compromised clients is a common assumption in existing works which has a huge impact on their applicability in reality. In this work, we do not assume such knowledge. As a result, popular outlier detection methods such as KNN and ABOD which require the prior knowledge of the number of neighbors are not feasible.
%for the outlier detection method. For example, it is difficult to use commonly used algorithms such as KNN, LOF, etc., which require the number of neighbors to be specified in advance, or the choice of metric matrix, such as Euclidean distance or cosine distance, as we do not know the proportion of malicious attackers in advance.

\noindent\textbf{C2}: \textbf{Identifying backdoored models in dynamic scenarios.} In each round, the number of injected backdoors is unknown and may vary. Hence, it is important to have a stable backdoored model identification method that can effectively handle dynamic attacks. Otherwise, many false positives may be generated, failing the backdoor defenses and impacting the main task's accuracy.

% \textbf{C2}: \textbf{\textcolor{blue}{The detection algorithm is computationally expensive.}} The neural networks of complex outlier detection algorithms are too large and thus computation-intensive. For example, \textcolor{blue}{For example, the ResNet network has a parameter size in the tens of millions.}

To address the above challenges, we chose Connectivity-based Outlier Factor~(COF)~\cite{cof} as our outlier detection algorithm. COF is a density-based outlier detection algorithm that measures the degree of connectivity of a data point to its neighboring points. COF calculates the outlier score of each data point by comparing its average reachability distance to that of its neighbors. COF is advantageous over other distance-based outlier detection algorithms as it is less sensitive to the number of dimensions of the data and can effectively detect outliers in high-dimensional data. It is also able to detect outliers in non-uniform density data sets and is less affected by the presence of noise in the data. Additionally, COF does not require any assumptions about the underlying data distribution, making it more robust to different types of data. Therefore, COF is a better choice for identifying backdoored models in a dynamic federated learning setting where the proportion of compromised clients is unknown and may vary over time.
%是否需要列出cof的计算公式 tbd


% iForest is an unsupervised, non-parametric (no assumptions about the sample distribution) outlier detection algorithm that does not require distance or density calculations. 
% It has low computational complexity and time complexity of $\mathcal{O}(n)$. As such, iForest can effectively overcome the aforementioned challenges.

% 挑战1:outlier detection method的参数选择。例如KNN、LOF等算法需要指定计算邻居的个数,但由于我们实现不知道恶意攻击者的比例大小,因此难以选择。亦或是度量矩阵的选择,很多算法需要实现指定度量矩阵,例如欧氏距离或余弦距离,该选择哪一个度量距离也会对outlier detection method的效果影响
% 挑战2:模型权重的维度很大。在RenNet网络中,权重的维度已经达到十几w,使得复杂的异常检测算法难以计算。
% 为了解决上述挑战,我们选择了iForest作为我们的异常点检测算法。iForest是一种无监督、非参数(不对样本的整体分布进行假设)的outlier detection method,不需要计算距离和密度,计算复杂度低,时间复杂度为O(n)。因此,iForest可以有效地克服这两个挑战。


        
%介绍异常值得分的过程
\begin{algorithm}
    \caption{Layer Scoring}\label{algorithm2}
    \begin{algorithmic} [1]
        \algrenewcommand\algorithmicrequire{\textbf{Input:}}
        \Require The local model $w_{i}$ uploaded by each client $ i\in C^{t} $
        \algrenewcommand\algorithmicrequire{\textbf{Output:}} 
        \Require  The set of Layer Scoring $S_{i}$ for each client $ i\in C^{t}$ 
        \State \textbf{initialize }  $n\gets len\left ( C^{t} \right ) $
        \For {$layer\ j\ in \left [ 1,total \right ]$ } \Comment{$total$ is the number of layers of the model }
        \State  $\left ( s_{1}^{j} ,\cdots,s_{n}^{j} \right ) \gets  COF\left ( w_{1}^{j} ,\cdots,w_{n}^{j} \right ) $
        \For {$  i \in \left [ n \right ]$ } 
        \State Add $s_{i}^{j}$ to the set of Layer Scoring $S_{i}$
        \EndFor
        \EndFor
        \State return $S_{1} ,\cdots,S_{n}$
    \end{algorithmic}
\end{algorithm}



\subsection{Anomaly Detection}
Layer Scoring assigns each layer of each local model an outlier score. Then, Anomaly Detection uses the scores to identify the anomalous clients to safeguard the model from backdoor attacks. Anomaly Detection checks the scores \textit{layer by layer} and increments a model's flag count by one upon finding an abnormal layer score. In the end, the clients with the higher flag counts are marked as anomalies. In this paper, we mark clients with more than 50\% of the layer count as anomalies. The algorithm for determining layer anomalies needs to be carefully designed to achieve the three \textit{defense goals} as mentioned in Section~\ref{sec:problem}.

We follow the common assumption that less than 50\% of clients are compromised. We argue that the commonly employed Three Sigma Rule~\cite{ThreeSigmaRule} and Z-score~\cite{Zscore}} can not identify anomalous clients well, because these algorithms assess clients using the mean value, which can be strongly influenced and shifted towards the location of the outliers in the presence of extreme outliers, resulting in failed outlier identifications.
%Because these algorithms assess clients using the mean value which is often shifted to extreme large outlier values and thus fail to identify anomalies. 
%\textcolor{blue}{For example, suppose we have the following set of data: [1, 2, 3, 30, 60]. In this set of data, most of the values are relatively normal, but there is one outlier (60), which will greatly affect the result of the mean calculation. The mean of outlier scores is 19.2, which is shifted to the outlier value 60.If we use Three Sigma Rule or Z-score algorithms to determine the outliers, it may lead to misclassified 30 as benign. This is because these algorithms focus only on the mean value and ignore other factors.}
% \textcolor{blue}{ For example, if the outlier scores are $[1,2,3,30,60]$, then the mean of outlier scores is 19.2, which is shifted to the outlier value 60 and leads to failure of detection}. 
To solve this problem, we use MAD for anomaly detection because: 
%which can effectively avoid the mean value shifting caused by extreme values. 
\begin{enumerate*}
  \item \textbf{it tolerates extreme values} since MAD uses the median which is not affected by extreme values, and 
  \item \textbf{it can be applied to any data distribution}, unlike Three Sigma Rule and Z-score which are only applicable to normally distributed data.% MAD, on the other hand, is applicable to any type of data distribution.
\end{enumerate*}
The Anomaly Detection processes include: 
\begin{enumerate*}
  \item Calculate the median of all the outlier scores.
  \item Calculate the absolute deviation value of the outlier scores from the median.
  \item Assign the median of all the absolute deviation values to MAD.
  \item A layer whose outlier score deviates from the median by larger than $\mu$ MAD is classified as anomalous and the model's flag ($Outlierflag$) is incremented by one. $\mu$ is the hyperparameter which we set to 3 by default in the experiments.
\end{enumerate*}
In each round, the server gets all the uploaded models and checks all their layers to get $ Outlierflag_{i},\forall i\in \left [ n \right ] $. FLD classifies the models of which at least half of the layers are marked as anomalies as anomalous models and aggregates only the other models that are classified as benign models. 
%介绍异常层判断的过程
\begin{algorithm}
    \caption{Anomaly Detection}\label{algorithm3}
    \begin{algorithmic} [1]
        \algrenewcommand\algorithmicrequire{\textbf{Input:}}
        \Require: The set of Layer Scoring from each client $ i\in C_{t}$ are regarded as $S_{i}$
        \algrenewcommand\algorithmicrequire{\textbf{Output:}} 
        \Require  The benign clients set $C_b^t$
        \State \textbf{initialize }  $n\gets len\left ( C^{t} \right ) $
        \State \textbf{initialize }  $Outlierflag_{i}  \gets 0, \forall i\in \left [ n \right ] $
        
        \For {$layer\ j\ in \left [ 1,total \right ]$ } \Comment{$total$ is the number of layers of the model }
        \State  $ Me \gets  MEDIAN\left ( S_{1}^{j} ,\cdots,S_{n}^{j} \right ) $
        \State  $MAD\gets MEDIAN(\left | S_{1}^{j}-Me \right |, \cdots, \left | S_{n}^{j}-Me \right |)$
        \For {$  i \in \left [ n \right ]$ } 
        \State $ flag1\gets \left ( S_{i}^{j} >= Me + \mu *MAD\right  ) ?1:0$
        \State $ flag2\gets \left ( S_{i}^{j} <= Me - \mu *MAD\right  ) ?1:0$
        \State $ Outlierflag_{i}\gets Outlierflag_{i}+flag1+flag2$
        \EndFor
        \EndFor
        \For {$  i \in \left [ n \right ]$ }
        \If{$Outlierflag_{i}<total/2$}
        \State Add $i$ to the benign clients set $C_b^t$
        \EndIf
        \EndFor
        \State return $C_b^t$
    \end{algorithmic}
\end{algorithm}


\subsection{Private FLD}
Many attacks on federated learning have been proposed besides backdoor attacks, such as membership inference attack and attribute inference attack. These attacks all demonstrate the necessity of enhancing the privacy protection of federated learning to prohibit access to local model plaintext updates. In general, there are two approaches to protect the privacy of customer data: differential privacy and encryption techniques such as homomorphic encryption~\cite{hom} or multi-party secure computation~\cite{mpc}. Differential privacy is a statistical and simple-to-implement method, but with impacts on the model performance, while encryption provides strong privacy guarantees and protection, but at the cost of reduced efficiency.
%%再介绍Paillier同态加密
Specifically, homomorphic encryption is a cryptographic primitive that allows computations to be performed on encrypted data without revealing the underlying plaintext. The basic idea is to encrypt the plaintext first to obtain the ciphertext and continue the calculation operation on the ciphertext, decrypt the final ciphertext result to obtain the plaintext, to keep the result consistent with the calculation on the plaintext. For example, Paillier cryptosystem is a representative additive homomorphic encryption that has been commonly used in federated learning. It has the following two homomorphic properties:
\setlist[itemize]{leftmargin=*}
\begin{itemize}
\item \textbf{Homomorphic addition of plaintexts}: $\llbracket{ x_{1}\rrbracket}\cdot \llbracket{ x_{2}\rrbracket}= \llbracket{x_{1}+x_{2}\rrbracket}$, where $ x_{1}$ and $ x_{2}$ represent  plaintexts, $\llbracket{~\rrbracket}$ 
is an encryption operation.
\item \textbf{Homomorphic multiplication of plaintexts}: $ \llbracket{ x\rrbracket}^{r}= \llbracket{ r\cdot x\rrbracket} $, where  $ x$  represents  plaintext, $\llbracket{~\rrbracket}$ 
is an encryption operation, $r$ is  a constant.
\end{itemize}
Next we illustrate the applicability of FLD in federated learning homomorphic encryption scenarios.
First, we follow the federated setup of~\cite{privacyfl}:
\setlist[itemize]{leftmargin=*}
\begin{itemize}
\item \textbf{Server} is responsible for receiving the gradients submitted by all participants and conducting aggregation to obtain a new global model.
\item \textbf{Cloud Platform (CP)} performs homomorphic encryption calculations together with the server. The CP holds a \textit{(private-key, public-key)} pair generated by a trusted authority for encryption and decryption.
\end{itemize}
Our algorithm is summarized in Algorithm~\ref{algorithm4}.\\
\begin{algorithm}[t!]
    \caption{Private FLD}\label{algorithm4}
    \begin{algorithmic} [1]
        \algrenewcommand\algorithmicrequire{\textbf{Input:}}
        \Require: The local model $\llbracket{w_{i}}\rrbracket$ uploaded by each client $ i\in C^{t} $
        \algrenewcommand\algorithmicrequire{\textbf{Output:}} 
        \Require  The set of Layer Scoring $S_{i}$ for each client $ i\in C^{t}$ 
        \algrenewcommand\algorithmicrequire{\textbf{Server:}}
        \Require
        \State  Randomly select m nonzero integer $r_i $ for j in [1,m]
        \Comment{m is the length of $w_{i}$ }
        \For {$  j \ in \left [ 1,m \right ]$ } \Comment{$n$ is  }
        \State $ c_{ij} \gets \llbracket{\omega_{ij} }\rrbracket\cdot \llbracket{r_{j} }\rrbracket  $
        \EndFor
        \State send $\left \{  c_{ij}  \right \} _{j=1}^{j=n} $ to CP
    \end{algorithmic}
    \begin{algorithmic}[1]
    \algrenewcommand\algorithmicrequire{\textbf{CP:}}
    \Require:
    \For {$  j \ in \left [ 1,m \right ]$ } \Comment{$n$ is  }
        \State  $ \omega _{ij}^{'} \gets Dec(sk_{c},c_{ij} ) $
    \EndFor
    \State $\left(S_{1} ,\cdots,S_{n}\right)\gets Layer Scoring\left(w_{1}^{'},\cdots,w_{n}^{'}\right)$
    \State Send $\left(S_{1} ,\cdots,S_{n}\right)$ to PS
    \end{algorithmic}
\end{algorithm}
%我们遵循Privacy-Enhanced Federated Learning Against Poisoning Adversaries的设置,算法描述如下,
%参考文献的描述
%我们需要可信的密钥生成中心(KGC)生成一对非对称密钥(pk c, sk c)云平台 (CP) 的 LHE,其中私钥sk c 仅由 CP 保存。同时,所有授权用户,持有LHE的同一对非对称密钥( pk x , sk x )由 KGC 生成。此外,在年初协议中,服务提供商(SP)随机初始化全局模型参数 ωini t 。
%介绍cof异常算法为什么可以
Correctness: To ensure that FLD can effectively identify malicious gradients, we need to prove that homomorphic encryption does not affect the calculation of COF anomaly detection.
According to the properties of homomorphic encryption, we have 
\begin{equation}
    \begin{split}\label{hm}
	  c_{ij}
	& =  \llbracket{\omega_{ij} }\rrbracket\cdot \llbracket{r_{j} }\rrbracket \\
	& =  \llbracket{\omega_{ij} } + {r_{j} }\rrbracket .
    \end{split}
\end{equation}
so $\omega _{ij}^{'} = \omega_{ij}  + r_{j} $, for $\omega _{x}^{'} $ and $\omega _{y}^{'} $ the Euclidean distance is
    \begin{equation}
    \begin{split}\label{hm}
	  \left \| \omega _{x}^{'}-\omega _{y}^{'} \right \| 
        & = \sqrt{\sum_{j=1}^{n}{\left( \omega _{xj}^{'}-\omega _{yj}^{'} \right)^{2} }}\\
	& =  \sqrt{\sum_{j=1}^{n}{\left( \omega_{xj}  + r_{j}- (\omega_{yj}  + r_{j}) \right)^{2} }} \\
        & = \sqrt{\sum_{j=1}^{n}{\left( \omega _{xj}-\omega _{yj} \right)^{2} }}\\
        & = \left \| \omega _{x}-\omega _{y} \right \|.
    \end{split}
\end{equation}

When the distance metric is Euclidean distance, the COF anomaly detection algorithm can still function in the homomorphic encryption scenario and the results are consistent with the plaintext.

\subsection{Convergence Analysis}
To analyze the convergence of FLD, we propose the theorem of convergence and prove it.

% \subsection{Notation and Assumptions}
% Let $F_i$ denotes the local model of the $i$-th client, $i=1,2,\cdots,N$. Let $F$ denotes the global model in the central parameter server. Suppose our models satisfy Lipschitz continuous gradient, we make Assumptions \ref{assumption1} and \ref{assumption2}. 


% \begin{assumption}\label{assumption1}
% ($L$-smooth). $F_1,\cdots,F_N$ are all $L$-smooth: $\forall x,y, F_i(x)\leq F_i(y)+(x-y)^{\mathsf{T}}\nabla F_i(y)+\frac{L}{2}||x-y||_2^2$.
% \end{assumption}
 
% \begin{assumption}\label{assumption2}
% ($\mu$-strongly convex). $F_1,\cdots,F_N$ are all $\mu$-strongly convex:  $\forall x,y, F_i(x)\geq F_i(y)+(x-y)^{\mathsf{T}}\nabla F_i(y)+\frac{\mu}{2}||x-y||_2^2$.
% \end{assumption}
% We also follow the assumption made by~\cite{stich2018sparsified,yu2019parallel,li2019convergence} as follows.
% \begin{assumption}\label{assumption3}
% The expected squared norm of stochastic gradients is uniformly bounded, i.e., $\exists U>0$, $\mathbb{E}||\nabla F_i(\cdot)||^2 \leq U^2$ for all $i=1,\cdots,N$.
% \end{assumption}
% We make Assumption \ref{assumption4} to bound the expectation of $||w_i^t||^2$, where $w_i^t$ denotes the parameters of $F_i$ in $t$-round.
% \begin{assumption}\label{assumption4}
% (Bounding the expectation of $|| w_i^t ||^2$). The expected squared norm of $i$-th client's local model parameters is bounded: $\exists M>0$, $\mathbb{E}||w_i^t||^2 \leq M^2$ for all $i=1,\cdots,N$ and $t=1,\cdots,T$.
% \end{assumption}

% \subsection{Theorem and Proof}

%提出收敛性理论, FLD收敛
\newtheorem{thm}{Theorem}
\begin{thm}\label{thm1}
Let Assumptions \ref{assumption1} to \ref{assumption4} hold and $L$, $\mu$, $U$, $M$ be defined therein. Choose the learning rate $\eta^t=\frac{\theta}{t+\epsilon}$, $ \epsilon>0$, $\theta > \frac{1}{\mu}$, we define $\lambda=\max\{\frac{\theta A}{\theta \mu -1}, (\epsilon+1)Z_1\}$. Then FLD satisfies 
\begin{equation}
    \begin{split}
        \mathbb{E}[F(G^t)]-F^*
	\leq \frac{L}{2} Z_t 
	\leq \frac{L}{2}\frac{\lambda}{(t+\epsilon )^{\frac{1}{2}}}
	\stackrel{t \to \infty}{\longrightarrow}0,
    \end{split}
\end{equation}
where 
\begin{equation}
    \begin{split}
        & A=4U^2+M^2+2\Gamma, \\
        & Z_t=\mathbb{E}||G^t-G^*||^2.
    \end{split}
\end{equation}
\end{thm} 


\begin{proof}
%以下是证明过程
%第一部分证明\mathbb{E}||G^{t+1}-G^*||^2\mathbb{E}||G^{t+1}-G^*||^2有上界,其中G^{t+1}是第t+1轮全局模型的权重,G^*是全局模型最优权重。
Let $G^{t+1}$ denote the global model's parameters in the central server in $(t+1)$-round and $G^*$ be the optimal parameters in the central server. Additionally, $g^t=\sum\limits_{i\in C_b^t}p_i\nabla F_i(w_i^t,\xi_i^t)$, where $g^t$ denotes the gradient updates uploaded by the clients in $t$-round and $p_i$ denotes the weight of the $i$-client's gradient during aggregation. $\bar{g^t}=\sum\limits_{i\in C_b^t}p_i\nabla F_i(w_i^t)$ and $G^{t+1}=G^t-\eta^t g^t$, where $C_b^t$ denotes the collection of benign clients chosen by FLD in $t$-round. Then, we have 
\begin{equation}
    \begin{split}\label{ineq1}
	||G^{t+1}-G^*||^2
	& = ||G^t-\eta^t g^t-G^*-\eta^t \bar{g^t} + \eta^t \bar{g^t}|| \\
	& = \underbrace{||G^t-G^*-\eta^t \bar{g^t}||^2}_{P_1} \\
        & +\underbrace{2\eta^t<G^t-G^*-\eta^t \bar{g^t},\bar{g^t}-g^t>}_{P_2} \\
	& +(\eta^t)^2||\bar{g^t}-g^t||^2.
    \end{split}
\end{equation}

Since $\mathbb{E}g^t=\bar{g^t}$, we see $\mathbb{E}P_2=0$. Now we split $P_1$ into three terms:   
\begin{equation}
    \begin{split}\label{ineq2}
	P_1
	& = ||G^t-G^*-\eta^t \bar{g^t}||^2 \\
	& = ||G^t-G^*||^2\underbrace{-2\eta^t<G^t-G^*,\bar{g^t}>}_{P_3}+\underbrace{(\eta^t)^2||\bar{g^t}||^2}_{P_4}.
    \end{split}
\end{equation}

Focusing on the last term in the above equation, according to Assumption \ref{assumption3}, we have 
\begin{equation*}
\begin{split}
    \mathbb{E}P_4
 & =\mathbb{E}[(\eta^t)^2||\bar{g^t}||^2] \\
 & \leq (\eta^t)^2\sum\limits_{i\in C_b^t}p_i^2\mathbb{E}||\nabla F_i(w_i^t)||^2 \\
 & \leq (\eta^t)^2 U^2 .  
\end{split}
\end{equation*}

Consider $P_3$, it follows:
\begin{equation}
    \begin{split}\label{ineq3}
	P_3
	& = -2\eta^t<G^t-G^*,\bar{g^t}> \\
	& = -2\eta^t\sum\limits_{i\in C_b^t}p_i<G^t-w_i^t,\nabla F_i(w_i^t)>\\
	& -2\eta^t\sum\limits_{i\in C_b^t}p_i<w_i^t-G^*,\nabla F_i(w_i^t)>.
    \end{split} 
\end{equation}

It is well known that $-2ab\leq a^2+b^2$, so
\begin{equation}
    \begin{split}\label{ineq4}
    & -2<G^t-w_i^t,\nabla F_i(w_i^t)>\\ 
    & \leq ||G^t-w_i^t||^2+||\nabla F_i(w_i^t)||^2 .
    \end{split}
\end{equation}

According to Assumption \ref{assumption2}, it follows:
\begin{equation}
    \begin{split}\label{ineq5}
    & -<w_i^t-G^*,\nabla F_i(w_i^t)> \\
    & \leq -(F_i(w_i^t)-F_i(G^*)) -\frac{\mu}{2}||w_i^t-G^*||^2 .
    \end{split}
\end{equation}

%由 (5),(6),(7),(8),可得
Use Equation~\ref{ineq2} and Inequalities~\ref{ineq3},~\ref{ineq4},~\ref{ineq5}, we obtain the following formula
\begin{equation*}
    \begin{split}
	   P_1  
        & = ||G^t-G^*-\eta^t \bar{g^t}||^2 \\
	& \leq ||G^t-G^*||^2+(\eta^t)^2||\nabla F_i(w_i^t)||^2 \\
        & +\eta^t\sum\limits_{i \in C_b^t}p_i(||G^t-w_i^t||^2
         +||\nabla F_i(w_i^t)||^2) \\
	& - 2\eta^t\sum\limits_{i \in C_b^t}p_i(F_i(w_i^t)-F_i(G^*)+\frac{\mu}{2}||w_i^t-G^*||^2)\\
	& \leq (1-\eta^t\mu)||G^t-G^*||^2+((\eta^t)^2+\eta^t)||\nabla F_i(w_i^t)||^2\\
	& + \eta^t\sum\limits_{i \in C_b^t}p_i||G^t-w_i^t||^2 \\
        & \underbrace{- 2\eta^t\sum\limits_{i \in C_b^t}p_i(F_i(w_i^t)-F_i(G^*))}_{P_5}.
    \end{split} 
\end{equation*}
Motivated by ~\cite{li2019convergence}, we define $\Gamma=F^*-\sum\limits_{i\in C_b^t}p_iF_i^*$. $\Gamma$ is used to measure the degree of heterogeneity between the local models and the global model, in i.i.d data distributions, $\mathbb{E}\Gamma=0$. We have 
%计算p5
\begin{equation*}
    \begin{split}
	P_5 
	& = - 2\eta^t\sum\limits_{i \in C_b^t}p_i(F_i(w_i^t)-F_i(G^*))\\
	& = - 2\eta^t\sum\limits_{i \in C_b^t}p_i(F_i(w_i^t)-F_i^*+F_i^*-F_i(G^*))\\
	& \leq  2\eta^t\sum\limits_{i \in C_b^t}p_i(F^*-F_i^*)=2\eta^t \Gamma, 
    \end{split} 
\end{equation*}
Hence,
\begin{equation*}
    \begin{split}
	P_1
	& \leq (1-\eta^t\mu)||G^t-G^*||^2+((\eta^t)^2+\eta^t)||\nabla F_i(w_i^t)||^2\\
	& + \eta^t\sum\limits_{i \in C_b^t}p_i||G^t-w_i^t||^2 +2\eta^t \Gamma.
    \end{split} 
\end{equation*}

Utilize the above results, we have
%从而可得
\begin{equation}
    \begin{split}\label{ineq6}
	\mathbb{E}|G^{t+1}-G^*||^2
	& \leq (1-\eta^t\mu)\mathbb{E}||G^t-G^*||^2 \\
        & + ((\eta^t)^2+\eta^t)\mathbb{E}||\nabla F_i(w_i^t)||^2\\
	& + \eta^t\sum\limits_{i \in C_b^t}p_i\mathbb{E}||G^t-w_i^t||^2 +2\eta^t \Gamma \\
        & + (\eta^t)^2\mathbb{E}||\bar{g^t}-g^t||^2.
    \end{split}
\end{equation}
%至此,Part1 证明完毕

%Part 2
%利用Assumption 3证明有界
According to Assumption \ref{assumption3}, it follows:
\begin{equation}
    \begin{split}\label{ineq7}
	\mathbb{E}||g^t-\bar{g^t}||^2
	& = \mathbb{E}||\sum\limits_{i \in C_b^t}p_i\nabla F_i(w_i^t,\xi_i^t)-\nabla F_i(w_i^t)||^2 \\
	& \leq \sum\limits_{i \in C_b^t}p_i^2 (\mathbb{E}||\nabla F_i(w_i^t,\xi_i^t)||^2 \\
        & +\mathbb{E}||\nabla F_i(w_i^t)||^2) \\
	&\leq 2\sum\limits_{i \in C_b^t}p_i^2 U^2.
    \end{split} 
\end{equation}
%至此,Part2 证明完毕


%Part 3
%利用Assumption 4证明有界
According to Assumption \ref{assumption4}, it follows:
\begin{equation}\label{ineq8}
    \begin{split}
	\sum\limits_{i \in C_b^t}p_i||G^t-w_i^t||^2
	& = \sum\limits_{i \in C_b^t}p_i||\sum\limits_{i \in C_b^t}p_i w_i^t-w_i^t||^2 \\
	& \leq \sum\limits_{i \in C_b^t}p_i || w_i^t||^2 \\
	& \leq M^2.
    \end{split}
\end{equation}
%至此part 3证明完毕

%Part 4
%现在由前三部分的结论来证明最终结论。
So far, we have all the preparations ready to prove the final conclusion. Let  $Z_t=\mathbb{E}||G^t-G^*||^2$, $\eta^t=\frac{\theta}{t+\epsilon}$, $\epsilon>0$, $\theta > \frac{1}{\mu}$, $\lambda=\max\{\frac{\theta A}{\theta \mu -1}, (\epsilon+1)Z_1\}$, our goal of proving $Z_t \leq \frac{\lambda}{(t+\epsilon)^{\frac{1}{2}}}$ can be achieved as follows.
\newline
For $t=1$, it holds. Suppose that the conclusion establishes for some t and use Inequalities~\ref{ineq6},~\ref{ineq7},~\ref{ineq8}, we have $Z_{t+1}$ as follows: %Note $A=4U^2+M^2+2\Gamma$, use (\ref{ineq6}), (\ref{ineq7}), (\ref{ineq8}), it follows
\begin{equation}
    \begin{split}
        Z_{t+1}
	& \leq (1-\eta^t\mu) Z_{t}+((\eta^t)^2+\eta^t)U^2 + \eta^t M^2 \\
        &  + 2 (\eta^t)^2 \sum\limits_{i \in C_b^t}p_i^2 U^2+2\eta^t \Gamma \\
	& \leq (1-\eta^t\mu)Z_{t} + \eta^t A \\
	& = \frac{(t+\epsilon)^{\frac{1}{2}}-1}{(t+\epsilon)}\lambda+(\frac{\theta A}{t+\epsilon}-\frac{\theta \mu -1}{t+\epsilon}\lambda) \\
	& \leq \frac{\lambda}{(t+\epsilon +1)^{\frac{1}{2}}},
    \end{split}
\end{equation}
where $A=4U^2+M^2+2\Gamma$.
%t+1t+1时刻成立,于是Z_t \leq \frac{\lambda}{(t+\epsilon)^{\frac{1}{2}}}Z_t \leq \frac{\lambda}{(t+\epsilon)^{\frac{1}{2}}}成立。
%利用Assumption **1**,从而有
Then, from Assumption \ref{assumption1}, we get
\begin{equation}
    \begin{split}
        \mathbb{E}[F(G^t)]-F^*
	\leq \frac{L}{2} Z_t 
	\leq \frac{L}{2}\frac{\lambda}{(t+\epsilon )^{\frac{1}{2}}}
	\stackrel{t \to \infty}{\longrightarrow}0.
    \end{split}
\end{equation}
\end{proof}














\section{Results and Discussion} \label{sec:results}
\vspace{-1mm}
In this section, we conduct several experiments to evaluate our approach using synthetic and real network datasets. We aim to examine our approach in regards to: \textit{(i)} the behaviour during the training process, \textit{(ii)} the accuracy and solution distributions to the optimality with different penalty coefficient and search strategy settings, \textit{(iii)} the impact of routing costs and traffic loads on the optimality performance and total network cost, and \textit{(iv)} the computational time.




\vspace{-2mm}
\subsection{Environment \& Experiment Setup}
%\vspace{-1mm}
We use synthetic (R1) and real (R2) network datasets to evaluate our approach. We generate R1 \thirdrev{with} stricter constraints and a larger scale environment than R2. R1 is generated using the Waxman algorithm \cite{waxman} with parameters such as link probability ($\alpha$) and edge length control ($\beta$). These respective parameters $(\alpha,\beta)$ are set to $(0.5, 0.1)$. R1 has 1 CU and 99 DUs. In the case of R2, we utilize a real network dataset from \cite{network_sndb}, which has 1 CU and 63 DUs. \secrev{We assume that the routers are co-located with the DUs.} R1 and R2 differ in parameters, e.g., location, link capacity, weighted link, delay. We use  a standard store-and-forward model to calculate the delay. It is from $12000/c_{ij}$, $4 \mu\text{secs}$/Km and $5 \mu\text{secs}$ for transmission, propagation and processing delay, respectively; see \cite{vranmec_andres}. The link capacity varies to $100$ Gbps (R1) and $252$ Gbps (R2). The path delay reaches to $3658 \ \mu s$ (R1) and $42 \ \mu s$ (R2). In R1, the routing cost per path is calculated from the total cost per link (randomly generated) which belongs to the selected path. A link with a routing cost of 1 monetary unit per Mbps means having the same cost as a DU computing cost. We consider the routing cost within a range of $0.001 - 0.01$ times of DU computing cost (for the same network load) for each link in R1. In R2, we calculate the distance between nodes based on its geolocation dataset from \cite{network_sndb} and charge the cost of $0.01$ monetary units per Mbps/km. Fig. \ref{fig:ran_params} depicts the parameter distributions of our RANs with eCDF.

%%%%%%%%%%%%%%%%%%%%%%%%%%%%%%%%%%%%%%%%%%%%%%%%%%%%%%%%%%

\begin{figure}[t]
	\centering
	\begin{subfigure}[t]{.23\textwidth}
		\centering
		\includegraphics[width=\textwidth]{./images/cdf_weight1.pdf}
		%\vspace{-3mm}	
		\small\caption{\small}
	\end{subfigure}
	%
	\begin{subfigure}[t]{.235\textwidth}
		\centering
		\includegraphics[width=\textwidth]{./images/cdf_bw1.pdf}
		%\vspace{-3mm}	
		\small\caption{\small }
	\end{subfigure}
	\begin{subfigure}[t]{.235\textwidth}
		\centering
		\includegraphics[width=\textwidth]{./images/cdf_lat1.pdf}
		%\vspace{-3mm}	
		\small\caption{\small }
	\end{subfigure}
	\caption{\small \textbf{RANs dist.} eCDF of (a) per-path routing cost, (b) per-link capacity, (c) per-path latency for R1 and R2.}
	\label{fig:ran}	
	\label{fig:ran_params}
	%\vspace{-3mm}	
	\vspace{-3mm}	
\end{figure}

%%%%%%%%%%%%%%%%%%%%%%%%%%%%%%%%%%%%%%%%%%%%%%


In this experiment, all system parameters correspond to testbed measurements of previous studies \cite{crancomplexity, vranmec_andres, vran_murti2,cost_vm}. We assume a high load scenario $\lambda_{n} = 150$ Mbps for every DU. This setting is based on 1 user/TTI, $2 \times 2$ MIMO, 20 Mhz (100 PRB), 2 TBs of 75376 bits/subframe and IP MTU 1500B. We use an Intel Haswell i7-4770 3.40GHz CPU as the \textit{reference core}, and set the maximum computing capacity to 75 RCs for CU and 7.5 RCs for each DU. Each split $o \in \{ 0,1,2,3 \}$ inccurs computational load $\rho_{o}^{{d}} = \{ 0.05, 0.04, 0.00325, 0\}$ RCs per Mbps at each DU and $\rho_{o}^{{c}} = \{0, 0.001, 0.00175, 0.05 \} $ RCs per Mbps at the CU. The VM instantiation cost at the CU is half of the DU $(\alpha_0 = \alpha_n/2)$ and the processing cost is set to $\beta_0 = 0.017 \beta_n$. 

Our learning rate is initially set to $\eta_a = 0.0001$ (Agent) and $\eta_b = 0.005$ (Baseline) with the batch size: 128. Our neural network has the number of layers, hidden dimension and embedding size with $1, 32$ and $ 32$, respectively. The temperature hyperparameter is set to $T=1$ by default, so the model computes the softmax function directly. We scale all the original values of weighted paths and traffic loads randomly with uniform distribution $[0,1]$ as in \cite{neural_bello}. Then, we generate three models (RL-pretaining) as outputs of our training with 50000 (in R1) and 15000 (in R2) epochs each. CDRS-Fixed uses a fixed penalty coefficient with $\mu_i =1, \forall i$ for all epochs while CDRS-Ada is set with initial penalty coefficient $\mu_i (0) =1, \forall i$ and step-size $\eta_d = 0.001$. The training is performed with Tensorflow 1.15.3 and Python 3.7.4. In the test, the temperature sampling method uses $16$ samples and $T = 15 $ (softmax temperature). 
%Finally, we summarize our default parameters in this experiment in Table xxx. 

\vspace{-2mm}
\subsection{Training Analysis}
%\vspace{-1mm}

\begin{figure*}[t] 
	\centering
	\begin{subfigure}[t]{.49\textwidth} %\label{fig:res_traina}
		\centering
		\includegraphics[width=\textwidth]{./images/train_r1.pdf}
		\small\caption{\small R1}
	\end{subfigure}
	%
	\begin{subfigure}[t]{.49\textwidth} %\label{fig:res_trainb}
		\centering
		\includegraphics[width=\textwidth]{./images/train_r2.pdf}
		\small\caption{\small R2}
	\end{subfigure}		
	\caption{\small \textbf{Training results of CDRS in (a) R1 and (b) R2.} CDRS-Fixed uses a fixed value of penalty coefficient (reward shaping) with $\mu_i = 1, \forall i$. CDRS-Ada utilizes an adaptive update of penalty coefficient.} 
	\label{fig:res_train} 
	\vspace{-3mm}
\end{figure*}

We aim to examine the behaviour of CDRS-Fixed and CDRS-Ada during the training process in R1 and R2. We focus on the mini-batch loss, reward (total network cost), Lagrangian cost and penalization.  

Fig. \ref{fig:res_train} visualizes the training of CDRS-Fixed and CDRS-Ada in R1 and R2. We found additional costs because of penalization at the beginning of the training for both settings in R1 and R2. It occurs because CDRS-Fixed and CDRS-Ada try to find the solution, but violate the constraint sets (e.g., latency, bandwidth, computation). Fig. \ref{fig:res_train} also shows a significant difference in the cost of penalization in R1 compared to R2. The main reason is that R1 has stricter constraints, e.g., larger path delays, smaller link capacity than R2. We can also see that CDRS-Fixed and CDRS-Ada improve their policy by focusing on constraint satisfaction and then correcting the weights via stochastic gradient descent. It is proven from our agent's behaviour in R1 and R2, where each penalization cost keeps decreasing and turns to zero as soon as the training goes. CDRS-Ada sets the penalty coefficient increasing in the ascent direction, causing a higher penalization value than CDRS-Fixed. However, it can help speed up the policy toward constraint satisfaction, i.e., CDRS-Ada penalization downs faster than CDRS-Fixed.


We also found that the policy of CDRS-Ada converges faster than CDRS-Fixed from the behaviour of mini-batch loss in R1. Despite the mini-batch loss decreases to near zero after several epochs, the mini-batch loss of CDRS-Ada diminishes faster than CDRS-Fixed. However, CDRS-Ada suffers from more severe sub-optimality. It is shown by the total vRAN cost of CDRS-Ada that converges to a fixed value but has a higher cost compared to CDRS-Fixed.  Then, we have the Lagrangian cost from the sum of vRAN cost and penalization cost. It describes how our agent tries to minimize the primal problem $\mathbb{P}_{\text{1P}}$ through the dual problem $\mathbb{P}_{\text{1D}}$.  When our agent finally dismisses the penalization cost, it means that all constraints are satisfied. As a result, the Lagrangian cost becomes equal to the vRAN cost, and the penalty coefficient of CDRS-Ada converges to a fixed value.  Although having different behaviours, CDRS-Ada and CDRS-Fixed can learn the solution and converge to the local minima or saddle point in R1 and R2.

\textbf{Findings:} 1) R1 has stricter constraint requirements than R2; hence, it produces a higher additional cost for penalization to CDRS-Fixed and CDRS-Ada. 2) CDRS-Fixed and CDRS-Ada improve the policy by focusing on the penalization; then, it adjusts the weights as the training goes. 3) CDRS-Ada receives higher penalization compared to CDRS-Fixed as a result of increasing the penalty coefficient in the ascent direction; however, it also helps speed up the policy to constraint satisfaction. 4) CDRS-Ada converges faster but has a higher cost than CDRS-Fixed in R1. 5) When all constraints are satisfied,  the Lagrangian cost becomes equal to the total vRAN cost, and the penalty coefficient of CDRS-Ada converges to a fixed value.

%

\vspace{-2mm}
\subsection{Accuracy of Solutions}
%\vspace{-1mm}
%
%
\begin{figure*}[t]
	\centering
	\begin{subfigure}[t]{.47\textwidth}
		\centering
		\includegraphics[width=\textwidth]{./images/acc_r1.pdf}
		\small\caption{\small R1}
	\end{subfigure}
	%
	\begin{subfigure}[t]{.47\textwidth}
		\centering
		\includegraphics[width=\textwidth]{./images/acc_r2.pdf}
		\small\caption{\small R2}
	\end{subfigure}	
	\caption{\small \textbf{Histogram of CDRS accuracy in (a) R1 and (b) R2.} The accuracy is calculated over 128 tests. CDRS-Ada-T and CDRS-Fixed-T are set with $T=15$ and $16$ samples.} 	\label{fig:_accmain}
	\vspace{-3mm}	
\end{figure*}


In this part, we study the accuracy of CDRS over different penalty coefficient and search strategy settings: CDRS-Fixed-G, CDRS-Fixed-T, CDRS-Ada-G and CDRS-Ada-T. We conduct 128 tests with a distinct sequence order of the BSs in R1 and R2 to assess how accurate these four CDRS settings find the solution of the vRAN split problem. We utilize three pretraining models \secrev{from} our CDRS training.

Fig. \ref{fig:_accmain} shows the distribution of \secrev{the solutions from} CDRS-Fixed-G, CDRS-Fixed-T, CDRS-Ada-G and CDRS-Ada-T in R1 and R2. \secrev{Each bar counts the number of offered solutions resulting in some suboptimality, represented using the optimality gap (error). It shows that the distribution varies between four settings, especially in a stricter environment (R1). Still,} all of these settings can guarantee less than $0.6 \%$ (R1) and $0.1 \%$ (R2) of the optimality gap. In R1, CDRS-Fixed-G and CDRS-Fixed-T perform better by offering lower solution errors ($ \leq 0.05 \%$ and $\leq 0.05 \%$ of optimality gap) than CDRS-Ada-G and CDRS-Ada-T ($\leq 0.6 \%$). It means that a fixed penalty coefficient setting can lead to a better optimality performance during the test than the adaptive one. However, CDRS-Fixed-G, CDRS-Ada-G and CDRS-Ada-T have a similar performance in R2. Regardless of R1 or R2, using a sampling method with a temperature hyperparameter can improve (or at least at same) the optimality performance than greedy decoding. It is shown from the higher total number of solutions (counts) for a sampling method that having a lower error. The combination of a fixed penalty in the training and temperature sampling method (CDRS-Fixed-T) can improve the solution performance significantly both in R1 and R2. It can achieve an optimal value (R2) and less than $0.05 \%$ of error for a more complex environment (R1). It is also shown that CDRS-Fixed-T is less affected to the stricter environment than any other settings where all of the distribution solutions are in less than $0.05 \%$.


\textbf{Findings:} 1) CDRS-Fixed-G, CDRS-Fixed-T, CDRS-Ada-G and CDRS-Ada-T can guarantee the solution with very close to the optimal value offering less than $0.6 \%$ (R1) and $0.1 \%$ (R2) of the optimality gap over 128 tests. 2) CDRS-Fixed-T can significantly improve the optimality performance (offers $\leq 0.05 \%$ of optimality gap) and outperforms the other settings.  



\vspace{-2mm}
\subsection{Impact of Routing Cost}
\vspace{-1mm}


This part studies the impact of altering the routing cost to CDRS-Fixed-G, CDRS-Fixed-T, CDRS-Ada-G and CDRS-Ada-T. We aim to examine how the routing cost affects optimality performance and the total network cost. Hence, the default routing cost is changed within a range of $\gamma=0.1$ to $\gamma=1$. This change can arise due to increasing or decreasing the leasing agreement's price, maintenance, etc. The traffic load is fixed with $\lambda_{n}=150$ Mbps. We utilize three \secrev{pretraining} models, conduct 128 tests for each routing cost scale, and analyze the offered solutions' distribution. We also consider benchmarking with two extremes of RAN setups, fully D-RAN and C-RAN\footnote{We practically can not implement C-RAN because our RANs do not meet the constraint requirements of delay, bandwidth and CU capacity to deploy C-RAN. The presented C-RAN in this experiment is just for benchmarking; hence we also do not consider the penalization cost (constrains violation) for this case.} to assess how significant the routing cost affects the total network cost over various RAN setups.


\begin{figure*}[t] 
	\centering
	\begin{subfigure}[t]{.99\textwidth}
		\centering
		\includegraphics[width=\textwidth]{./images/routing_acc_r1.pdf}
		\small\caption{\small R1}
	\end{subfigure}
	%
	\begin{subfigure}[t]{.99\textwidth}
		\centering
		\includegraphics[width=\textwidth]{./images/routing_acc_r2.pdf}
		\small\caption{\small R2}
	\end{subfigure}		
	\caption{\small \textbf{Impact of the routing cost to the accuracy in (a) R1 and (b) R2.} Study of altering the routing cost to the optimality performance with $\lambda_{n}=150$ Mbps, $\forall n \in \mathcal{N}$. There are 128 tests for each routing scale $[0.1,1]$.} \label{fig:routing_acc}
	\vspace{-3mm}
\end{figure*}


\begin{figure*}[t] 
	\centering
	\begin{subfigure}[t]{.375\textwidth}
		\centering
		\includegraphics[width=\textwidth]{./images/routing_cost_r1.pdf}
		\small\caption{\small R1}
	\end{subfigure}
	%
	\begin{subfigure}[t]{.375\textwidth}
		\centering
		\includegraphics[width=\textwidth]{./images/routing_cost_r2.pdf}
		\small\caption{\small R2}
	\end{subfigure}		
	\caption{\small \textbf{Impact of routing cost to the total cost in (a) R1 and (b) R2.} We also compare our approach (e.g., CDRS-Fixed-T) to two extreme cases: fully D-RAN and C-RAN, and the optimal value with the routing cost scaling from 0.1 to 1 of default R1 and R2. The presented cost above is normalized toward fully C-RAN cost.} \label{fig:routing_cost}
	\vspace{-3mm}
\end{figure*}


Fig. \ref{fig:routing_acc} depicts how the routing cost affects the optimality performance of CDRS-Fixed-G, CDRS-Fixed-T, CDRS-Ada-G and CDRS-Ada-T. It shows that the \secrev{overall} optimality gap (error) diminishes as the routing cost increases; then, it converges to a specific value. In R1, we see a performance improvement as the errors decrease for CDRS-Fixed-G ($\approx75\%$), CDRS-Fixed-T ($\approx75\%$), CDRS-Ada-G ($\approx78\%$) and CDRS-Ada-T ($\approx75\%$) \secrev{by the increase of routing cost}. It also shows that CDRS-Ada-G gets the most impact while CDRS-Fixed-T is the least affected. In R2, all CDRS settings \secrev{also} have a similar trend in terms of error \secrev{performance}. Although we have changed the routing cost from the default parameter, we found that \secrev{altering the} routing cost gives relatively less effect to these settings where the errors are maintained under $1.8 \%$. CDRS-Fixed-T even can guarantee the solution under $0.08 \%$ ($\gamma  = 0.1$) of the optimality gap.



Fig. \ref{fig:routing_cost} shows the routing cost's effect on the total network cost of CDRS-Fixed-T and D-RAN, normalized to the C-RAN cost in R1 and R2. \secrev{It shows that CDRS-Fixed-T can obtain a larger cost-saving than the D-RAN cost at a cheaper routing cost by up to $59.06\%$ of cost-saving at $\gamma = 0.1$ while only $25.49\%$ of cost-saving at $\gamma = 1$ in R1. Compared to C-RAN, CDRS-Fixed-T can save the cost by up to $92\%$ at $\gamma = 1$ in R1. However, this gain diminishes as the routing cost decreases and eventually CDRS-Fixed-T will reach near the C-RAN cost if all constraint requirements are eligible. A similar trend also appears for R2. Moreover, CDRS-Fixed-T can offer the solution extremely close to the optimal solution by $\leq 0.09\%$ (R1) and $\leq 0.5\%$  (R2). }
%

%It shows that the increase of routing cost gives CDRS-Fixed-T and D-RAN cost relatively decrease (around $320\%$ and  $420\%$ from $\gamma = 0.1$ to $\gamma = 1$ in R1) to the C-RAN cost. Hence, we can conclude that the routing cost gives more impact to C-RAN than other setups. Additionally, CDRS-Fixed-T is the most cost-efficient with around $500\%$ and $200\%$ cost-saving of C-RAN and D-RAN at low routing cost ($ \gamma = 0.1$). It also can save to 20 times and two-fold compared to the respective RAN setups at high routing cost ($ \gamma = 1$) in R1. In R2, CDRS-Fixed-T can save to around five times and two times of C-RAN and D-RAN cost at low routing cost ($ \gamma = 0.1$). It has cost-saving to 20 times and two-fold compared to the respective RAN setups at high routing cost ($ \gamma = 1$). CDRS-Fixed-T offers the solution very close to the optimal solution ($\leq 0.09\%$ in R1 and $\leq 0.5\%$ in R2) and efficiently adapts to the change of the routing cost. 



\textbf{Findings:} 1) The increase of routing cost \secrev{reduces} the optimality gap \secrev{(error)}; then, \secrev{it} converges to a fixed value. 2) CDRS-Fixed-T is the least affected by the routing cost changes, while CDRS-Ada-G is the most affected. 3) \secrev{Scaling} the routing cost from $\gamma = 0.1$ to $\gamma = 1$ does not significantly degrade the optimality performance. 4) CDRS-Fixed-T has the lowest optimality gap  \secrev{than other CDRS settings}, and becomes the most cost-effective setup in R1 and R2. \secrev{5) CDRS-Fixed-T can reach near the D-RAN cost at a high routing cost, while it can be near the C-RAN cost at a cheap routing cost if all constraint requirements are eligible.}
%

\vspace{-2mm}
\subsection{Impact of Traffic Load}
\vspace{-1mm}

\begin{figure*}[t] 
	\centering
	\begin{subfigure}[t]{.99\textwidth}
		\centering
		\includegraphics[width=\textwidth]{./images/traffic_acc_r1.pdf}
		\small\caption{\small R1}
	\end{subfigure}
	%
	\begin{subfigure}[t]{.99\textwidth}
		\centering
		\includegraphics[width=\textwidth]{./images/traffic_acc_r2.pdf}
		\small\caption{\small R2}
	\end{subfigure}		
	\caption{\small \textbf{Impact of the traffic load to the accuracy in (a) R1 and (b) R2.} Study of traffic load to the optimality performance. There are 128 tests for each traffic load. } \label{fig:traffic_acc}
	\vspace{-3mm}
\end{figure*}


\begin{figure*}[t] 
	\centering
	\begin{subfigure}[t]{.375\textwidth}
		\centering
		\includegraphics[width=\textwidth]{./images/traffic_cost_r1.pdf}
		\small\caption{\small R1}
	\end{subfigure}
	%
	\begin{subfigure}[t]{.375\textwidth}
		\centering
		\includegraphics[width=\textwidth]{./images/traffic_cost_r2.pdf}
		\small\caption{\small R2}
	\end{subfigure}		
	\caption{\small \textbf{Impact of traffic load to total vRAN cost in (a) R1 and (b) R2.} On the comparison of our approach (e.g., CDRS-Fixed-T) to fully D-RAN. The presented cost above is normalized toward fully C-RAN cost.} \label{fig:traffic_cost}
	\vspace{-3mm}
\end{figure*}

In this part, we assess how the traffic load affects the optimality performance and the total network cost. We change the traffic load from 10 Mbps to 150 Mbps. This evaluation is conducted using three \secrev{pretraining} models and examined over 128 tests.   

Fig \ref{fig:traffic_acc} shows the impact of altering the traffic load to the optimality performance of CDRS-Fixed-G, CDRS-Fixed-T, CDRS-Ada-G and CDRS-Ada-T. In R1, it shows that the increase of traffic load in line with the rise of the error to CDRS-Ada-G and CDRS-Ada-T, but it then diminishes to a fixed value, i.e., around $0.4 \%$ (CDRS-Ada-G) and $0.18\%$ (CDRS-Ada-T). However, the traffic load does not significantly affect CDRS-Fixed-G and CDRS-Fixed-T, where they stay at around $0.04\%$ and $0.02\%$ of errors, respectively, in R1. In R2, CDRS-Fixed-G, CDRS-Fixed-T, CDRS-Ada-G and CDRS-Ada-T have the same trend where the optimality gap increases with the traffic load; then, it diminishes at around  $0.05\%$. We also found that CDRS-Fixed-T \secrev{has a} better optimality performance and a more stable solution. 


Fig \ref{fig:traffic_cost} examines the impact of traffic load on CDRS-Fixed-T and D-RAN cost normalized to the C-RAN cost. Despite an increase in CDRS-Fixed-T cost as the traffic load rises, it shows that CDRS-Fixed-T is still the most cost-effective compared to D-RAN and C-RAN \secrev{in} R1 and R2. \secrev{CDRS-Fixed-T almost has the same cost as D-RAN at the low traffic load with only $12.33\%$ cost-saving. This cost-saving then increases for the higher traffic load settings by up to $25.5\%$ at 150 Mbps in R1. This trend also happens in R2. Compared to C-RAN, CDRS-Fixed-T significantly outperforms at the low traffic load, but this gain then diminishes as the increase of the load. CDRS-Fixed-T can reach near the C-RAN cost when all constraint requirements are satisfied, and the traffic load is high, but the routing cost is significantly low.}

%
%In R1, CDRS-Fixed-T can save $114\%$ at low traffic load (10 Mbps) and $134\%$ at high traffic load (150 Mbps) of D-RAN cost, while $124\%$ and $177\%$ for the respective load in R2. The cost-saving gap is also more prominent with the increase of traffic load. We also found that D-RAN is the most affected by the increase in traffic load.

\textbf{Findings:} 1) CDRS-Fixed-T can offer to better optimality performance and more stable solution \secrev{than other CDRS settings}. 2) In R2, all CDRS settings have similar trends where the increase of traffic load can also increase the optimality gap, but it then diminishes and stays at around $0.05\%$ for CDRS-Fixed-T and $0.06\%$ for the others. 3)  CDRS-Fixed-T is the most cost-efficient compared to C-RAN and D-RAN. 4) \secrev{CDRS-Fixed-T can eventually almost have the same C-RAN cost when all constraint requirements are satisfied, and the traffic load is high, but the routing cost is significantly low.}

\vspace{-2mm}
\subsection{Computational Time}
\vspace{-1mm}

Finally, we examine the computational time to solve a single instance of the vRAN split problem. We use a small laptop with an Intel Core i5-7300U CPU@2.60GHz and 8GB memory. The computational time for each CDRS setting is a result of averaging 128 executions with a trained model. We report this evaluation in Table \ref{table:computationaltime}.  Overall, our proposed CDRS settings: CDRS-Fixed-G, CDRS-Fixed-T, CDRS-Ada-G and CDRS-Ada-T, have a faster computational time than the MIP solver. CDRS-Ada-G is the fastest with $0.0120$ secs and $0.0077$ secs in R1 and R2 reaching to $22.82$ times faster than the MIP solver. We also found that any CDRS settings with greedy decoding for the inference process, e.g., CDRS-Fixed-G, CDRS-Ada-G, is more time-efficient than a temperature sampling method with around 10-20 times faster. It is also shown that CDRS-Ada-G/T has a slightly faster computational time than CDRS-Fixed-G/T. Finally, we can sort from the fastest computational time as 1) CDRS-Ada-G, 2) CDRS-Fixed-G, 3) CDRS-Ada-T, 4) CDRS-Fixed-T, 5) the MIP solver.
%Besides, an adaptive penalty coefficient can speed up the policy to find the solution than a fixed penalty coefficient, especially in the highly constrained environment (R1).

\begin{table*}[t!] \centering
	%\ra{1.3}
	\begin{small}
		\begin{tabular}{@{}cccccc@{}}\toprule
			\textbf{Topology}& \textbf{MIP solver} &\textbf{CDRS-Fixed-T} & \textbf{CDRS-Fixed-G}  & \textbf{CDRS-Ada-T} & \textbf{CDRS-Ada-G}
			\\ \midrule
			\textbf{R1} &      0.2527   & 0.2026 & 0.0155& 0.1985 & 0.0120          
			\\ \hdashline
			{\textbf{R2}} &  0.1756 &  0.1240 & 0.0098 & 0.1207 &0.0077
			\\ \hdashline
			\bottomrule
		\end{tabular}
	\end{small}
	\caption{\small \textbf{Computational time.} Study of computational time for solving a single problem instance in seconds. The presented computational time is a result of averaging 128 executions.}
	\label{table:computationaltime}
	\vspace{-3mm}
\end{table*}

\textbf{Findings:} 1) CDRS-Ada-G, CDRS-Fixed-G, CDRS-Ada-T, and CDRS-Fixed-T can reach up to $22.82, 17.99, 1.45$ and $1.41$ times faster than the MIP solver. 2) Greedy decoding is more time-efficient than a temperature sampling method for the inference process.


\section{Conclusions}
In this paper, we suggest improving planning efficiency with a novel sampling strategy. This strategy relies on the optimal cost-to-come and problem-specific cost-to-go heuristics, acquired through the global replanning function and the lazy-reverse searching method, respectively. This sampling strategy generates samples in the Relevant Region to guide the planner to search the most promising regions, improving the quality of the initial solution path and the convergence rate. Simulation results in $SE(2)$ and $SE(3)$ spaces demonstrate that the proposed method achieves better performance when solving the planning problem. In the future, we aim to take advantage of the reverse searching method to further accelerate planning efficiency.


%%%%%%%%%%%%%%%%%%%%%%%%%%%%%%%%%%%%%%%%%%%%%%%%%%%%%%%%%%%%%%%%%
%%%%%%%%%%%%%%%%%%%%%%%%%%%%%%%%%%%%%%%%%%%%%%%%%%%%%%%%%%%%%%%%%
%%%%%%%%%%%%%%%%%%%%%%%%%%%%%%%%%%%%%%%%%%%%%%%%%%%%%%%%%%%%%%%%%

\printcredits

%% Loading bibliography style file
%\bibliographystyle{model1a-num-names}
\bibliographystyle{cas-model2-names}

% Loading bibliography database
\bibliography{references.bib}

\balance
%\vskip3pt

\end{document}

