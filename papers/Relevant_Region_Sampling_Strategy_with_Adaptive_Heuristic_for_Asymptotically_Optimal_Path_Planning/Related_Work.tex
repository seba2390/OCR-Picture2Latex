\section{Related Work}


\subsection{Sampling-based Motion Planning Method}

Plenty of modifications are proposed to enhance the performance of the RRT algorithm \cite{lavalle1998rapidly} such as the RRT* algorithm \cite{karaman2011sampling}.
The rewiring stage of the RRT* only rewires locally, which means the global optimization of the current tree is ignored. 
The RRT\# \cite{arslan2013use} proposes to find the global optimality in each rewiring stage with dynamic programming.
Dynamic programming is also used in the Fast Marching Tree (FMT*) method \cite{janson2015fast} to grow the searching tree.
% And the FMT* introduces the thought of batch sampling into the robot motion planning field.
The Informed sampling strategy \cite{gammell2014informed, gammell2018informed} is proposed to overcome the drawback of uniform sampling.
% It can accelerate the convergence speed significantly with very little computation consumption. 
% The Informed sampling strategy uses a direct sampling method to generate samples in the $L_2$-Informed set.
% An advanced version of the Informed sampling strategy is proposed in \cite{gammell2018informed}, which includes the graph pruning stage to keep a relatively constricted tree. 
% A tree with fewer vertices means that the cost reduction in finding the nearest tree vertex. 
Using the neural network to reinforce the sampling stage to enhance the sampling efficiency \cite{wang2020neural, li2021efficient, qureshi2019motion} is proved as a promising technique.

% In the human-robot coexisting environment, the planning problem become more complex than that in the static environment \cite{wang2020eb}.
% The objective of optimization needs to consider safety, efficiency, and human feelings.


% The Informed sampling strategy can constraint the whole planning procedure in a subset of the whole state space, the $L_2$-Informed set, and the Lebesgue measure of the $L_2$-Informed set decreases as the solution improves.
% \\ \textcolor{red}{*SWIRRT*: maybe not including the SWIRRT is better}




\subsection{Batch Sampling Technique}

% A batch sampling method is described in the FMT* method \cite{janson2015fast}.
The FMT* \cite{janson2015fast} introduces the thought of batch sampling into the robot motion planning field.
The FMT* samples a batch of points and constructs the searching tree according to this batch of samples.
% The asymptotic optimal of the FMT* is guaranteed when the size of the batch goes to infinity. 
The Batch Informed Trees (BIT*) \cite{gammell2015batch, gammell2020batch} method is developed based on the Informed RRT*, besides, the BIT* absorbs the thoughts in the FMT* method \cite{janson2015fast} and the Lifelong Planning A* (LPA*) algorithm \cite{koenig2004lifelong}.
The Regionally Accelerated Batch Informed Trees (RABIT*) \cite{choudhury2016regionally} aims to solve the difficult-to-sample planning problem, like the narrow passage problem.
% The RABIT* uses the Covariant Hamiltonian Optimization for Motion Planning (CHOMP) method as its local optimizer, and the local optimizer will exploit the local information.
The Fast-BIT* \cite{holston2017fast} modifies the edge queue and searches the initial solution more aggressively. 
The Greedy BIT* \cite{chen2021greedy} uses the greedy searching method to generate the initial solution faster and accelerate the convergence speed.
But these greedy-based methods often fail to assist the searching procedure without an accurate heuristic estimation method. 
The Adaptively Informed Trees (AIT*) \cite{strub2020adaptively} and the Advanced BIT* (ABIT*) \cite{strub2020advanced} proposed by Strub and Gammell are developed based on the BIT* as well. 
The AIT* calculate a relatively accurate heuristic estimation with a lazy reverse-searching tree.
The ABIT* proposes to utilize inflation and truncation to balance the exploitation and exploration in the increasingly complex Random Geometric Graph (RGG) \cite{penrose2003random}.
Though the AIT* and the ABIT* achieve significant improvements, their sampling regions are not compact enough, and the sampling efficiency will be critically low in the complex environment.


\subsection{Relevant Region Sampling Strategy}

The concept of 'relevant' is first proposed in the searching-based robot path planning method like the A* \cite{hart1968formal}. 
In the A* algorithm, the set of expanded vertices is relevant to the query, such that the A* algorithm could expand a smaller set of vertices than the Dijkstra's algorithm \cite{dijkstra1959note}.  
% And it is also not a new idea in the sampling-based planning field. 
% The Relevant Region is formally defined in \cite{arslan2013use}. 
% The Relevant Region related vertices are the vertices of which the sum of the optimal cost-to-come and the heuristic is less than the cost of the current optimal solution.
% Since the Relevant Region is the most promising region that could help to improve the solution, so a straightforward modification is to reduce the chance of sampling outside the  Relevant Region.
The concept of the Relevant Region is formally introduced in \cite{arslan2013use}, whose sum of the optimal cost-to-come and cost-to-go heuristic is less than the cost of the current optimal solution. 
Since the Relevant Region is the most promising area for improving the solution, a straightforward modification would be reducing the likelihood of sampling outside of it.
Three different metrics are used to achieve this in \cite{arslan2015dynamic}, the modified versions achieve better performance in the convergence speed than the RRT\# method.
The methods described in \cite{arslan2013use} and \cite{arslan2015dynamic} use the rejection method for sampling, which is not efficient since the Relevant Region is a small subset of the whole state space in most scenarios.
The direct sampling method is illustrated to overcome this drawback, and the details are described in \cite{joshi2020relevant}.
However, they all use the cumulative cost along the direct connection between the current state and the goal state as the cost-to-go. 
This approach results in inaccurate estimated cost-to-go in most scenarios.
% Their ordered priority queues are also far from the ground truth.

% The optimal cost-to-come value can be defined as the vertex which has the lowest optimal cost-to-come value in the destination region. 


\subsection{Bi-directional Searching Method}

The RRT and RRT* methods may not always discover a solution within the allotted time, particularly when dealing with narrow passages
The RRT-Connect \cite{kuffner2000rrt} is proposed to find the initial solution faster. 
% It grows two trees from the source point and the goal region simultaneously.
% It is proved that the RRT-Connect can achieve better performance than the RRT.
However, the approach described in \cite{kuffner2000rrt} is not asymptotically optimal. Therefore, its successor, RRT-Connect, is also not asymptotically optimal. 
To overcome this, an enhanced version of the bidirectional searching RRT is introduced in \cite{klemm2015rrt} to guarantee the asymptotical optimality.
To take advantage of the benefits of bi-directional search, the kinematic constraints are taken into consideration in the bi-directional search method to generate executable trajectories efficiently \cite{wang2021kinematic}.
% The method described in \cite{klemm2015rrt} is an asymptotically optimal single-query version of the RRT-Connect, called the RRT*-Connect. 
% The RRT*-Connect provides asymptotically optimal guarantee like the RRT*, and its efficiency and robustness are proofed in real-world experiments.
% In addition, the bi-directional searching method can be used to combine with the kinematic constraints , which is essential in generating executable trajectory.

% In the RRT-Connect, one tree is extended in each iteration and tries to connect itself to the other tree; this manner will attempt to grow the trees towards each other.

One drawback of the Informed RRT* \cite{gammell2014informed} \cite{gammell2018informed} is that it uses the RRT* to search the whole state space before finding the initial solution.
Therefore, the Informed RRT* often fails to find the solution in the required period, same as the RRT*.
By combining the advantages of both the Informed and the RRT*-Connect, the Informed RRT*-Connect \cite{2020Informed} proposes to use the RRT*-Connect to generate the initial solution and use the Informed sampling strategy to constrain the sampling region after the initial solution is found.
% It combines the advantages of both the Informed RRT* and the RRT*-Connect.
% The Informed RRT*-Connect can achieve a much higher success rate in its simulations than the Informed RRT*.
Besides, the AIT* \cite{strub2020adaptively} can also be viewed as a bi-directional searching method.



