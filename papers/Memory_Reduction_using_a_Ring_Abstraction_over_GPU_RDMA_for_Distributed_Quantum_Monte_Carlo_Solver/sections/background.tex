\section{Background}
\label{sec:background}

QMC solver applications are widely used and are mission-critical across the US Department
of Energy's (DOE's) application landscape. For the purpose of this paper, the authors chose to use one of the
primary QMC applications,
the DCA++ code.
A production-scale scientific
problem runs on DOE's fastest supercomputer, Summit, at the Oak Ridge Leadership Computing Facility on all 4,600
nodes; each node contains six NVIDIA Volta V100 GPUs, attaining a peak performance of 73.5 PFLOPS with a mixed
precision implementation~\cite{dca2019}.

Monte Carlo simulations are embarrassingly parallel, and the authors exploited this on distributed systems with a two-level
(MPI + threading) parallelization scheme (Figure~\ref{fig:dca_overview}). Although DCA++ has 
been highly optimized and is scalable on existing hardware, this is the first effort to focus on 
solving the memory-bound issue described in Section \ref{sec:intro} and further take advantage 
of Summit's GPU RDMA capability. 

\begin{figure}[h]
	\centering
	\includegraphics[width=\columnwidth]{images/dca_overview.pdf}
	\caption{Workflow of the QMC DCA++ solver.}%
	\label{fig:dca_overview}
\end{figure}


Figure~\ref{fig:dca_overview} shows the parallelism
hierarchy in one iteration of the QMC solver (MPI
distribution + on-node threading parallelism). 
%
For example, each rank $\{R0,\ldots,RN\}$ 
is assigned a Markov Chain and the initial input (two particle
Green's function, $G_{t,i}$, where \textit{t} means ``two-particle,'' 
and \textit{i} is rank index).  Each rank spawns multiple independent
worker threads (i.e., walkers and accumulators). 
%
Most work and computation are performed on the GPU. 
%
Each walker thread generates a measurement
result ($G_{\sigma, i}$ array, where \textit{i} is thread ID) by performing 
nonuniform Fourier transform implemented by matrix-matrix multiplication.
%
Each walker passes its $G_{\sigma, i}$ to its corresponding accumulator
thread. 
%
In other words, each thread has its own $G_{\sigma, i}$ array,
and each rank will have $k$ different $G_{\sigma, i}$ arrays, where $k$
is the number of walker threads per rank. 
%
Each accumulator thread then updates $G_{t,i}$ via the formula in Eq.~(\ref{eq:G4})
to compute and update rank-local
$G_{t,i}$ to $G^{\prime}_{t,i}$.  
%
The updated partial $G^{\prime}_{t,i}$ is then fed
into the coarse-graining step for the next measurement. At the end
of all measurements, an \texttt{MPI\_Reduce} operation will be
performed on $G_t$ across all ranks to produce a final and complete
$G_{t}$ in the root rank. 
%
$G_t$ is allocated before all measurements start and has a life that spans
until the end of the DCA++ program.
%

%%% Note: if the Fourier transform is on a uniform grid, then the classical Fast Fourier Transform (FFT) can be used. However, if the data is not on a uniform grid, then this can be more complicated

\subsection{Memory-Bound Issue in DCA++}
The results from Balduzzi et al. \cite{dca2019} show that although storing a $G_t$ on the accelerator device allows condensed matter scientists
to explore larger and more complex (i.e., higher fidelity) physics
cases, the problem size is limited to the device memory size.
Updating the device array $G_t$ is the most time-consuming and memory-intensive process throughout DCA++ computation. A distributed $G_t$ approach is needed to reduce memory allocation and operation in the device.

In the original DCA++ algorithm, $G_t$ is 
updated by a product of two smaller
matrices (single-particle Green's function, or $G_{\sigma}$). This computation update is 
in the particle-particle channel and is accumulated according to Eq.~(\ref{eq:G4}).
\begin{equation}
\label{eq:G4}
    G_t (K_1, K_2, K_3)  \pluseq 
      \sum_{\sigma} G_\sigma (K_3 - K_2, K_3 - K_1) \, G_{-\sigma} (K_2, K_1)\,, 
\end{equation}
where $K_i$ is a combined index that represents a particular point in the momentum and frequency space, 
and $\sigma= {+1} \mbox{ or } {-1} $ specifies the electron spin value. $G_{\sigma}$ is the single-particle Green's function that
describes the configuration of single electrons.



% Generally, larger $G_t$ means that we can access larger problem
% sizes, which allows us to significantly increase the accuracy/fidelity
% of a simulation of a certain material. In terms of physics, this
% means that we can solve more realistic models of materials that
% include more details of the actual materials. For example, with the
% previous design, we have been able to calculate a single-orbital
% Hubbard model which is the simplest and a very generic model of a
% correlated quantum material. It is a coarse-grained model that
% allows us to study generic behavior arising from electronic
% correlations in materials, but it cannot distinguish between different
% materials. Materials specific modeling requires more complex models
% that include more orbital (and other) degrees of freedom, and this
% in turn requires much larger $G_t$. Having larger $G_t$ enables us
% to study these more complex and materials specific models, and thus
% address questions that are specific to certain materials.
% 
% Another benefit is that we can now compute the full $G_t$ tensor
% in a single computation. With the previous design, we were limited
% to computing certain slices of $G_t$, and this required physical
% insight or prior knowledge to decide which slices contain the
% important physics. This approach worked well for the simple
% single-orbital model, for which we have a good idea which slices
% to look at. For the more complex multi-orbital models, we don’t
% necessarily know which slices are important, so being able to compute
% the full $G_t$ means that we won't miss important physics.

The ability to handle a larger $G_t$  allows simulations of complex materials to significantly increase the details, accuracy, and fidelity.
%
In the previous design that kept $G_t$ within one GPU, only a
sub-slice of $G_t$ could be computed in a single computation.
%
For the simple single-orbital coarse-grained Hubbard model, physics insights or prior
knowledge can be used to decide which sub-slices in $G_t$
contain the important physics and thus avoid the generation of full $G_t$.
%
This simple model allows the generic behavior that comes
from electronic corrections in materials to be studied, but it cannot distinguish between
different specific materials.
%
Material-specific modeling requires more complex models
that include more orbital---and other---degrees of freedom, and
this requires a much larger $G_t$. 
%
This new distributed ring implementation enables the full large $G_t$
array to be computed in a single computation, even for more
complex multi-orbital models, to ensure that no important 
physics cases are overlooked.
%



\subsection{GPU RDMA Technology}
GPU RDMA allows direct peer access to multi-GPU memory through a high-speed network. 
%
For NVIDIA GPUs, GPUDirect is a technology that allows for the direct transfer of data in GPU device memory to other GPUs on the same node by using the NVLINK2 interconnect and/or between GPUs on different nodes by using RDMA support that can bypass buffers on host memory. 

A CUDA-aware MPI\footnote{\url{https://developer.nvidia.com/blog/introduction-cuda-aware-mpi/}} implementation can directly pass the GPU buffer pointer to MPI calls. Acceleration support, such as GPUDirect, can be used by the MPI library 
and allows buffers being sent from the kernel memory to a network without 
staging through host memory. There are various CUDA-aware MPI implementations, such as OpenMPI, MVAPICH2, and 
IBM Spectrum MPI\footnote{IBM Spectrum MPI is supported on
the Summit supercomputer, and is also the CUDA-aware MPI implementation used by the authors in this paper.}.
% the need to stage data through CPU memory.  
% how do we verify the H2D and D2H?  
% \url{https://docs.olcf.ornl.gov/systems/summit_user_guide.html#cuda-aware-mpi}
