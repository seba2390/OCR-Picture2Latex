\begin{abstract}
\label{sec:abstract}

%% 1. what is the problem 
Scientific applications that run on leadership computing facilities often face the challenge 
of being unable to fit leading science cases onto accelerator devices due to memory constraints 
(memory-bound applications).
%
% 2. what is your solution 
In this work, the authors studied one such US Department of Energy mission-critical condensed matter 
physics application, Dynamical Cluster Approximation (DCA++), and this paper discusses how device memory-bound challenges were successfully reduced  by proposing an effective 
``all-to-all'' communication method---a ring communication algorithm. 
%
This implementation takes advantage of acceleration on GPUs and remote direct memory access (RDMA) for fast data exchange between GPUs. 
%
\\Additionally, the ring algorithm was optimized with sub-ring communicators
and multi-threaded support to further reduce communication overhead and 
expose more concurrency, respectively.
%
% 3. What's the cherry-picked evaluation result you want to mention
The computation and communication were also analyzed 
by using the Autonomic Performance Environment for Exascale 
(APEX) profiling tool,  and this paper further discusses the 
performance trade-off for the ring algorithm implementation. 
%
The memory analysis on the ring algorithm shows that the allocation size for the authors' most 
memory-intensive data structure per GPU is now reduced to $1/p$ of the original size, where $p$ is the number of GPUs in the ring communicator.
%
The communication analysis suggests that 
the distributed Quantum Monte Carlo execution time grows linearly as sub-ring size increases, and the cost of messages passing through the network interface connector could be a limiting factor.


%
% \todoRed{Ronnie: Next sentence needs rewrite, too much information about Green's function that no one knows in the abstract; recommend generalizing.} \emph {However, DCA++ is currently facing memory-bound challenge as 
% a larger device array $G_t$ is limited by device memory size, where
% $G_t$ is a two-particle Green's function that allows condensed matter
% scientists to explore larger and more complex (higher fidelity)
% physics cases.}

\end{abstract}

\keywords{DCA++, Quantum Monte Carlo, GPU Remote Direct Memory Access, memory-bound issue, exascale machines}
