\documentclass[hidelinks]{siamart0216}
\usepackage[T1]{fontenc}
\usepackage{mystyle}
\usepackage{todonotes}
\usepackage{algorithmicx}
\usepackage{algpseudocode}
\usepackage{subcaption}

\newcommand{\maxx}{x^+}
\newcommand{\minn}{x^-}
\newcommand{\conv}[1]{\mathrm{conv}\ens{#1}}
\DeclareMathOperator{\aff}{aff}
\newcommand{\X}{\mathbf{X}}
\newcommand{\MAX}{\mathbf{S}}
\newcommand{\Xl}{\X_{\lambda}}
\newcommand{\Xo}{\X_{0}}
\newcommand{\Sl}{\MAX_{\lambda}}

\DeclareMathOperator{\dom}{dom}
\DeclareMathOperator{\card}{card}

\newcommand{\TheTitle}{Maximal Solutions of Sparse Analysis Regularization}
\newcommand{\TheAuthors}{A. Barbara, A. Jourani and S. Vaiter}


\ifpdf
\hypersetup{
  pdftitle={\TheTitle},
  pdfauthor={\TheAuthors}
}
\fi

\headers{\TheTitle}{\TheAuthors}

\title{{\TheTitle}}

\author{A. Barbara%
  \thanks{Universit\'e de Bourgogne Franche-Comt\'e, Institut de Math\'ematiques de Bourgogne, UMR 5584 CNRS, (\email{\{barbara,jourani\}@u-bourgogne.fr})}%
  \and
  A. Jourani%
  \footnotemark[1]%
  \and
  S. Vaiter%
  \thanks{CNRS \& Universit\'e de Bourgogne Franche-Comt\'e, Institut de Math\'ematiques de Bourgogne, UMR 5584 CNRS, (\email{vaiter@u-bourgogne.fr})}
}


\begin{document}

\maketitle

\begin{abstract}
  This paper deals with the non-uniqueness of the solutions of an analysis-Lasso regularization.
  Most of previous works in this area is concerned with the case where the solution set is a singleton, or to derive guarantees to enforce uniqueness.
  Our main contribution consists in providing a geometrical interpretation of a solution with a maximal $D$-support, namely the fact that such a solution lives in the relative interior of the solution set.
  With this result in hand, we also provide a way to exhibit a maximal solution using a primal-dual interior point algorithm.
\end{abstract}

\begin{keywords}
  Lasso, analysis sparsity,  inverse problem, support identification, barrier penalization
\end{keywords}

\begin{AMS}
  90C25, % Convex programming
  49J52  % Nonsmooth analysis
\end{AMS}

Reinforcement learning has achieved great success in areas such as Game-playing \citep{silver2018general,vinyals2019grandmaster}, robotics \cite{kober2013reinforcement}, large language models \citep{ouyang2022training}, etc.
However, due to safety concerns or physical limitations, in some real-world reinforcement learning problems, we must consider additional constraints that may influence the optimal policy and the learning process \citep{garcia2015comprehensive}.
% For example, a robotic arm must not take actions that may cause harm to itself or the environments.
A standard framework to handle such cases is the constrained Markov Decision Process (CMDP) \citep{altman1999constrained}.
Within the CMDP framework, the agent has to maximize
the expected cumulative reward while
obeying a finite number of constraints, which are usually in the form of expected cumulative cost criteria.

However, we are sometimes concerned with the problem with a continuum of constraints.
For example,
the constraints we meet might be time-evolving or subject to uncertain parameters, which
cannot be formulated as an ordinary CMDP
(see Examples \ref{Example_Time_Evolving} and  \ref{Example_Uncertain}).
In this paper we would study a generalized CMDP  
to address the above problem.  Because the constraints are not only infinite-number but also lie
in a continuous set,
the generalization is not trivial. Fortunately, we find that we can borrow the idea behind semi-infinite programming (SIP) \citep{remez1934determination, hettich1993semi} to deal with the semi-infinite constraints.
Accordingly, we propose \emph{semi-infinitely constrained Markov decision processes} (SICMDPs)
as a novel complement to the ordinary CMDP framework.
%More specifically,  an SICMDP model %, we consider 
%contains a continuum of constraints whereas an ordinary CMDP contains a finite number of constraints. 

%This generalization is natural but not trivial. However, we can brows the idea  
%The idea is quite natural and can be backtracked
%to the practice of extending linear programming to linear semi-infinite programming (LSIP) %\cite{remez1934determination, GobernaLSIO1998}.
%In addition, 
%As a complementary approach to the ordinary CMDP framework, 
%SICMDP can be used to model these problems  which cannot be described by a finite number of constraints
%that are not covered by .
%For example,
%the restrictions we consider can be time-evolving or subject to uncertain parameters
%, thus
%cannot be described by a finite number of constraints but a continuum of constraints 
%(see Examples \ref{Example_Time_Evolving} and  \ref{Example_Uncertain}).

We also present two reinforcement learning algorithms to solve SICMDPs called SI-CRL and SI-CPO, respectively.
SI-CRL is a model-based reinforcement learning algorithm designed for tabular cases, and SI-CPO is a policy optimization algorithm for non-tabular cases.
% and analyze its performance both theoretically and empirically.
The main challenge is that we need to deal with a continuum of constraints, thus reinforcement learning algorithms for ordinary CMDPs do not work anymore.
In SI-CRL, we tackle this difficulty by first transforming the reinforcement learning problem to an equivalent LSIP problem, which can then be solved using methods in the LSIP literature like the dual exchange methods \citep{Hu1990,reemtsen1998numerical}.
In SI-CPO, we resort to the idea of cooperative stochastic approximation developed in \cite{lan2020algorithms, wei2020comirror}.
As far as we know, we are the first to introduce tools from semi-infinitely programming (SIP) into the reinforcement learning community for solving constrained reinforcement learning problems.

% To the best of our knowledge, we are the first to apply tools from semi-infinitely programming (SIP) to solve reinforcement learning problems.
Furthermore, we give theoretical analysis for both SI-CRL and SI-CPO.
We decompose the error of SI-CRL into two parts: the statistical error from approximating the true SICMDP with an offline dataset and the optimization error due to the fact that the solution of the LSIP problem obtained by the dual exchange method is inexact.
On the optimization side, we show that the iteration complexity of SI-CRL is $O\left(\left\{\mathrm{diam}(Y)L\sqrt{|\gS|^2|\gA|m}/\left[(1-\gamma)\epsilon\right]\right\}^m\right)$.
On the statistical side, we show that the sample complexity of SI-CRL is $\widetilde O\left(\frac{|S|^2|A|^2}{\epsilon^2(1-\gamma)^3}\right)$ if the offline dataset is generated by a generative model, and $\widetilde O\left(\frac{|S||A|}{\nu_{\min} \epsilon^2(1-\gamma)^3}\right)$ if the dataset is generated by a probability measure $\nu$ as considered in \cite{chen2019information}.
Here $\widetilde O$ means that all logarithm terms are discarded.
For SI-CPO, things become a little more complicated because other than the statistical error and the optimization error, we also need to consider the function approximation error, which comes from imperfect policy parametrizations.
It is shown if the function approximation error can be controlled to $O(\epsilon)$ order, the iteration complexity of SI-CPO is $\widetilde{O}\left(\frac{1}{\epsilon^2(1-\gamma)^6}\right)$ and the sample complexity of SI-CPO is $\widetilde{O}(\frac{1}{\epsilon^4(1-\gamma)^{10}})$.
Here our iteration complexity bound is equivalent to a typical $\widetilde O(1/\sqrt{T})$ global convergence rate.

We perform a set of numerical experiments to illustrate the SICMDP model and validate our proposed algorithms.
Specifically, we examine two numerical examples, namely the discharge of sewage and ship route planning.
Through the discharge of sewage example, we show the advantage of the SICMDP framework over the CMDP baseline obtained by naive discretization in modeling realistic sequential decision-making problems.
Moreover, we demonstrate the effectiveness of the SI-CRL and SI-CPO algorithms in such tabular environments. 
In the ship route planning example, we illustrate the benefits of the SICMDP framework and the ability of the SI-CPO algorithm to address complex continuous control tasks involving continuous state spaces with modern deep reinforcement learning techniques.

% In summary, our contributions are listed as follows.
% First, we present the SICMDP model, which can be viewed as a generalization of the ordinary CMDP model.
% Second, we propose an algorithm to perform reinforcement learning for SICMDPs, which is called SI-CRL, and we believe that we are the first to apply tools from SIP
% to solve reinforcement learning problems.
% Third, we give a theoretical analysis of SI-CRL and identify both its sample complexity and iteration complexity.
% In addition, we perform numerical experiments to illustrate the SICMDP model and validate the SI-CRL algorithm.
% \{This paragraph can be removed!!! \}





\section{Contributions}
\label{sec:contrib}

In~\cref{sec:sol}, we review some properties of the solution set.
In all this paper, \textbf{we consider the following hypothesis of restricted injectivity}
\begin{equation}\label{eq:hyp-inv}
  \Ker D^* \cap \Ker \Phi = \ens{0} ,
\end{equation}
in order to ensure that $\Xl$ is well-defined and bounded.
We prove in particular that $\Xl$ is a polytope, i.e. a bounded polyhedron.

Our main contribution is proved in~\cref{sec:max}.
It consist in providing a geometrical interpretation of a solution with a maximal $D$-support, namely the fact that such a solution lives in the relative interior of the solution set.
More precisely, we are concerned with the characterization of a vector of maximal $D$-support, i.e. a solution of~\eqref{eq:p} such that for every $x \in \Xl, \normz{D^* x} \leq \normz{D^* \maxx}$.
\begin{definition}
  A vector $\maxx \in \RR^n$ is \emph{a solution of maximal $D$-support} if $\maxx$ is a solution, i.e. $\maxx \in \Xl$ such that for every $x \in \Xl, \normz{D^* x} \leq \normz{D^* \maxx}$.
\end{definition}
We denote by $\Sl$ the set of solution of~\eqref{eq:p} which have maximal $D$-support.
Clearly this set is well-defined and contained in $\Xl$.
Our result is the following.
\begin{theorem}\label{thm:maximal-characterization}
  Let $\bar{x} \in \Xl$. Then $\bar{x}$ is a maximally $D$-supported solution if, and only if, $\bar{x} \in \rint \Xl$ (or equivalently if $\bar{x} \in \rint \Sl$).
  In other words, 
  \begin{equation*}
    \Sl = \rint \Sl = \rint \Xl .
  \end{equation*}
\end{theorem}
We recall that for any set $S$, the relative interior $\rint S$ of $S$ is
defined as its interior with respecto to the topology of the affine hull of $S$.

With this result in hand, we provide a way to construct such maximal solutions.
In \cref{sec:finding}, we show that with the help of a technical penalization using the so-called concave gauge ~\cite{barbara2015strict}, we can construct a path which converges to a point in the relative interior of $\Xl$, and more specifically, to the analytic center with respect to the chosen gauge.
We defer the precise statement to~\cref{sec:finding}.
\section{Solutions to exercises}
\label{sec:solutions}

\subsection{The optimal discriminator strategy}
\label{sec:opt_d_soln}

Our goal is to minimize
\begin{equation}
  J^{(D)}(\vtheta^{(D)}, \vtheta^{(G)}) = -\frac{1}{2} \E_{\vx \sim \pdata} \log D(\vx) - \frac{1}{2} \E_{\vz} \log \left(1 - D\left( G(z) \right) \right)
\end{equation}
in function space, specifying $D(\vx)$ directly.

We begin by assuming that both $\pdata$ and $\pmodel$ are nonzero everywhere.
If we do not make this assumption, then some points are never visited during training,
and have undefined behavior.

To minimize $J^{(D)}$ with respect to $D$, we can write down the functional derivatives with
respect to a single entry $D(\vx)$, and set them equal to zero:
\[
\frac{\delta} {\delta D(\vx)} J^{(D)} = 0.
\]
By solving this equation, we obtain
\[
D^*(\vx) = \frac{ \pdata(\vx) } {\pdata(\vx) + \pmodel(\vx) }.
\]

Estimating this ratio is the key approximation mechanism used by GANs.

The process is illustrated in \figref{fig:ratio}.

\begin{figure}
\centering
\includegraphics[width=\figwidth]{ratio}
\caption{
An illustration of how the discriminator estimates a ratio of
densities.
In this example, we assume that both $z$ and $x$ are one dimensional
for simplicity.
The mapping from $z$ to $x$ (shown by the black arrows) is non-uniform so that $\pmodel(x)$
(shown by the green curve) is
greater in places where $z$ values are brought together more densely.
The discriminator (dashed blue line) estimates the ratio between the data density (black dots)
and the sum of the data and model densities.
Wherever the output of the discriminator is large, the model density is too low, and wherever
the output of the discriminator is small, the model density is too high.
The generator can learn to produce a better model density by following the discriminator uphill;
each $G(z)$ value should move slightly in the direction that increases $D(G(z))$.
Figure reproduced from \citet{Goodfellow-et-al-NIPS2014-small}.
}
\label{fig:ratio}
\end{figure}


\subsection{Gradient descent for games}
\label{sec:xy_soln}

The value function
\[ V(x, y) = x y \]
is the simplest possible example of a continuous function with a saddle point.
It is easiest to understand this game by visualizing the value function in three
dimensions, as shown in \figref{fig:xy}.

\begin{figure}
\centering
\includegraphics[width=\figwidth]{xy}
\caption{A three-dimensional visualization of the value function $V(x,y) = xy$.
  This is the canonical example of a function with a saddle point, at $x=y=0$.
}
\label{fig:xy}
\end{figure}

The three dimensional visualization shows us clearly that there is a saddle point
at $x=y=0$. This is an equilibrium of the game. We could also have found this point
by solving for where the derivatives are zero.

Not every saddle point is an equilibrium; we require that an infinitesimal perturbation
of one player's parameters cannot reduce that player's cost.
The saddle point for this game satisfies that requirement.
It is something of a pathological equilibrium because the value function is constant
as a function of each player's parameter when holding the other player's parameter
fixed.

To solve for the trajectory taken by gradient descent, we take the derivatives, and find that
\begin{align}
  \frac{\partial x}{\partial t} = - y(t) \\
  \frac{\partial y}{\partial t} = x(t). \label{eq:dy}
\end{align}
Diffentiating \eqref{eq:dy}, we obtain
\[
  \frac{\partial^2 y}{\partial t^2} = \frac{\partial x}{\partial t} = -y(t).
\]
Differential equations of this form have sinusoids as their set of basis functions
of solutions.
Solving for the coefficients that respect the boundary conditions, we obtain
\begin{align}
  x(t) = x(0) \cos(t) - y(0) \sin(t) \\
  y(t) = x(0) \sin(t) + y(0) \cos(t).
\end{align}

These dynamics form a circular orbit, as shown in \figref{fig:orbit}.
In other words, simultaneous gradient descent with an infinitesimal learning rate
will orbit the equilibrium forever, at the same radius that it was initialized.
With a larger learning rate, it is possible for simultaneous gradient descent to
spiral outward forever.
Simultaneous gradient descent will never approach the equilibrium.

\begin{figure}
  \center
  \includegraphics[width=\figwidth]{orbit}
  \caption{Simultaneous gradient descent with infinitesimal learning rate
    will orbit indefinitely at constant radius when applied to $V(x,y) = xy$,
    rather than approaching the equilibrium solution at $x=y=0$.
  }
  \label{fig:orbit}
\end{figure}

For some games, simultaneous gradient descent does converge, and for others,
such as the one in this exercise, it does not.
For GANs, there is no theoretical prediction as to whether simultaneous
gradient descent should converge or not.
Settling this theoretical question, and developing algorithms guaranteed to
converge, remain important open research problems.

\subsection{Maximum likelihood in the GAN framework}
\label{sec:mle_soln}

We wish to find a function $f$ such that the expected gradient of 
\begin{equation}
  J^{(G)} = \E_{\vx \sim p_g} f(\vx)
  \label{eq:cost_per_sample}
\end{equation}
is equal to the expected gradient of 
$\KL(\pdata \Vert p_g)$.

First we take the derivative of the KL divergence with respect to a parameter $\theta$:
\begin{equation}
  \frac{\partial}{\partial \theta} \KL(\pdata \Vert p_g) = - \E_{\vx \sim \pdata} \frac{\partial}{\partial \theta} \log p_g(\vx) .
\label{eq:mle_gradient}
\end{equation}

We now want to find the $f$ that will make the derivatives of \eqref{eq:cost_per_sample} match \eqref{eq:mle_gradient}.
We begin by taking the derivatives of \eqref{eq:cost_per_sample}:
\[
  \frac{\partial}{\partial \theta} J^{(G)} = \E_{\vx \sim p_g} f(x) \frac{\partial}{\partial \theta} \log p_g(\vx).
\]
To obtain this result, we made two assumptions:
\begin{enumerate}
  \item We assumed that $p_g(\vx) \geq 0$ everywhere so that we were able to use the identity $p_g(\vx) = \exp( \log p_g(\vx) ).$
  \item We assumed that we can use Leibniz's rule to exhange the order of differentiation and integration (specifically, that both the function and its derivative are continuous, and that the function vanishes for infinite values of $\vx$).
\end{enumerate}

We see that the derivatives of $J^{(G)}$ come very near to giving us what we want; the only problem is that
the expectation is computed by drawing samples from $p_g$ when we would like it to be computed by drawing
samples from $\pdata$.
We can fix this problem using an importance sampling trick; by setting $f(x) = \frac{\pdata(\vx)}{p_g(\vx)}$
we can reweight the contribution to the gradient from each generator sample to compensate for it having
been drawn from the generator rather than the data.

Note that when constructing $J^{(G)}$ we must {\em copy} $p_g$ into $f(x)$ so that $f(x)$ has a derivative of
zero with respect to the parameters of $p_g$.
Fortunately, this happens naturally if we obtain the value of $\frac{\pdata(\vx)}{p_g(\vx)}$.

From \secref{sec:opt_d_soln}, we already know that the discriminator estimates the desired ratio.
Using some algebra, we can obtain a numerically stable implementation of $f(\vx)$.
If the discriminator is defined to apply a logistic sigmoid function at the output layer,
with $D(\vx) = \sigma( a(\vx) )$, then $f(x) = - \exp(a(\vx))$.

This exercise is taken from a result shown by \citet{Goodfellow-ICLR2015}.
From this exercise, we see that the discriminator estimates a ratio of densities
that can be used to calculate a variety of divergences.

\section{Maximal support and proof of \cref{thm:maximal-characterization}}
\label{sec:max}

We recall that a vector $\maxx \in \RR^n$ is a solution of maximal $D$-support if $\maxx$ is a solution, i.e., $\maxx \in \Xl$ such that for every $x \in \Xl, \normz{D^* x} \leq \normz{D^* \maxx}$.
The following proposition proves that \emph{the} $D$-maximal support is indeed
uniquely defined.
\begin{proposition}
  Let $x \in \Xl$.
  Then the two following propositions are equivalent.
  \begin{enumerate}
  \item $x$ is a solution of maximal $D$-support, i.e. $x \in \Sl$.
  \item For any $\bar{x} \in \Xl$, $\supp(D^* \bar{x}) \subseteq \supp(D^* x)$.
  \end{enumerate}
\end{proposition}
\begin{proof}
  The two directions are proved separately.\\
  $(1) \Rightarrow (2)$.
  Suppose there exists $i_0 \in \ens{1,\dots,p}$ such that $i_0 \in \supp(D^* \bar{x})$ and $i_0 \not\in \supp(D^* x)$.
  Observe that $\tilde x = \frac{1}{2}(\bar{x} + x)$ is also an element of $\Xl$ by convexity of $\Xl$.
  Using Proposition~\ref{prop:sign}, we get that $\supp(D^* \tilde x) \supseteq \supp(D^* \bar{x}) \cup \supp(D^* x)$.
  In particular, $\supp(D^* \tilde x) \supseteq \supp(D^* x) \cup \ens{i_0} \supsetneq \supp(D^* x)$.
  Hence, $\abs{\supp(D^* \tilde x)} > \abs{\supp(D^* x)}$ which contradicts the fact that $x$ has maximal $D$-support.\\
  $(2) \Rightarrow (1)$.
  Taking the cardinal in the property $\forall \bar{x} \in \Xl$, $\supp(D^* \bar{x}) \subseteq \supp(D^* x)$ is sufficient.
\end{proof}
In particular, two solutions of maximal support share the same $D$-support. Notice that in this case, the sign vectors are also the same.

We start by a technical Corollary of Proposition~\ref{prop:sign} which will be convenient in the following.
\begin{corollary}\label{cor:diagpos}
  There exists an integer $m \in \NN$, a matrix $\Lambda = \diag(\lambda_i)_{i=1,\dots,p}$ with $\lambda_i \in \ens{-1,1}$ for $i \in \ens{1,\dots,m}$ and $\lambda_i = 0$ for $i \in \ens{m+1,\dots,p}$, and a permutation matrix $\Sigma$ such that for $\Gamma = \Lambda \Sigma$, one has
  \begin{equation*}
    \Gamma D^* \Xl \subset (\RR_+)^m \times \ens{0}^{p-m} .
  \end{equation*}
  Moreover, for all $x \in \Xl$, $\normu{\Gamma D^* x} = \normu{D^* x}$.
\end{corollary}
\begin{proof}
  Let $\maxx$ an element of $\Sl$.
  Consider $I = \supp(D^* \maxx)$, $J = I^c$ and $m = \abs{I}$.
  Let $\Sigma$ be the permutation matrix associated to any permutation $\sigma$ which sends $I$ to $\ens{1,\dots,m}$.
  Define the matrix $\Lambda$ by its diagonal as
  \begin{equation*}
    \lambda_{\sigma(i)} = 
    \begin{cases}
      1 & \text{if } (D^* \maxx)_{\sigma(i)} > 0 \\
      -1 & \text{if } (D^* \maxx)_{\sigma(i)} < 0 \\
      0 & \text{if } (D^* \maxx)_{\sigma(i)} = 0 .
    \end{cases}
  \end{equation*}

  Now take any solution $x \in \Xl$ and consider the vector $u = \Gamma D^* x$.
  Let $i \in \ens{1,\dots,m}$, then
  \begin{equation*}
    u_i = \dotp{e_i}{\Lambda \Sigma D^* x} .
  \end{equation*}
  Since $\Lambda$ is self-adjoint, one has
  \begin{equation*}
    u_i = \dotp{\Lambda e_i}{\Sigma D^* x} .
  \end{equation*}
  Since $\Lambda$ is a diagonal matrix, we get that
  \begin{equation*}
    u_i = \lambda_i \dotp{e_i}{\Sigma D^* x} .
  \end{equation*}
  Now, since $\Sigma$ is a permutation matrix, we have that $\Sigma^* = \Sigma^{-1}$, i.e.
  \begin{equation*}
    u_i = \lambda_i \dotp{\Sigma^{-1} e_i}{D^* x} .
  \end{equation*}
  Using the permutation $\sigma$ associated to $\Sigma$, we have that
  \begin{equation*}
    u_i = \lambda_i \dotp{e_{\sigma^{-1}(i)}}{D^* x} ,
  \end{equation*}
  which can be rewritten as
  \begin{equation*}
    u_i = \lambda_i \dotp{d_{\sigma^{-1}(i)}}{x} .
  \end{equation*}  
  According to Proposition~\ref{prop:sign}, one have $(D^* x)_{\sigma^{-1}(i)} (D^* \maxx)_{\sigma^{-1}(i)} \geq 0$.
  Moreover, $\lambda_i = \lambda_{\sigma(\sigma^{-1}(i))}$ has the same sign than $(D^* \maxx)_{\sigma^{-1}(i)}$.
  Thus, $u_i = \lambda_i \dotp{d_{\sigma^{-1}(i)}}{x} \geq 0$.

  For $i \in \ens{m+1,\dots,p}$, we have that
  \begin{equation*}
    u_i = \lambda_i \dotp{e_i}{\Sigma D^* x} = 0,
  \end{equation*}
  since $\lambda_i = 0$.
\end{proof}
Note that the matrix $\Lambda$ and $\Sigma$ are not uniquely defined. Corollary~\ref{cor:diagpos} allows us to work only on positive vectors in dimension $m$.

We will also need to exclude at some point the case where a solution $x$ lives in the kernel of $D^*$.
The following lemma shows that if this is the case, then the solution set is reduced to a singleton $\Xl = \ens{x}$.
\begin{lemma}\label{lem:kernel-one-image}
  If there exists $x \in \Ker D^* \cap \Xl$, then $\Xl = \ens{x}$ .
\end{lemma}
\begin{proof}
  We recall that $\Xl \subset x + \Ker \Phi$.
  Let $\bar x \in \Xl$, and rewrite it as $\bar x = x + h$ where $h \in \Ker \Phi$.
  Then, according to Proposition~\ref{lem:same-image}, one has $\normu{D^* \bar x} = \normu{D^* x} = 0$.
  In particular, $\normu{D^* \bar x} = \normu{D^* x + D^* h} =  \normu{D^* h} = 0$.
  Using hypothesis~\eqref{eq:hyp-inv}, we get that $h = 0$.
\end{proof}

We can now provide the proof of Theorem~\ref{thm:maximal-characterization}.
\begin{proof}[Proof of Theorem~\ref{thm:maximal-characterization}]
  We exclude here the case where $\Xl$ is reduced to a singleton, since the result is then trivially verified.
  Let us prove both direction separately.

  $(\Leftarrow: \rint \Xl \subseteq \Sl)$.
  First, we recall that $\rint \Xl = \rint (A \Delta_k) = A \rint \Delta_k$.
  Let $\bar{x} \in \rint \Xl$.
  We have
  \begin{equation*}
    \bar{x} = A \bar{z} \qwithq \sum_{i=1}^k \bar{z}_i = 1 \qandq \bar{z}_i > 0.
  \end{equation*}
  For $i \in \ens{1,\dots,m}$, one has
  \begin{equation*}
    (\Gamma D^* \bar{x})_i = (\Gamma D^* A \bar{z})_i = \dotp{e_i}{\Gamma D^* A \bar{z}} =  \dotp{e_i}{\Lambda \Sigma D^* A \bar{z}}.
  \end{equation*}
  Using the fact that $\Lambda$ is a diagonal matrix and $\Sigma$ is a permutation matrix, we have that
  \begin{equation*}
    (\Gamma D^* \bar{x})_i = \lambda_i \dotp{D \Sigma^{-1} e_i}{A \bar{z}} ,
  \end{equation*}
  which can be rewritten, using the fact that $\Sigma^{-1} e_i = e_{\sigma^{-1}(i)}$ where $\sigma$ is the permutation associated to $\Sigma$, as
  \begin{equation*}
    (\Gamma D^* \bar{x})_i = \lambda_i \dotp{d_{\sigma^{-1}(i)}}{A \bar{z}} .
  \end{equation*}
  Now, one can rewrite it as
  \begin{equation*}
    (\Gamma D^* \bar{x})_i = \lambda_i \dotp{(D^* A)^* e_{\sigma^{-1}(i)}}{\bar{z}} .
  \end{equation*}
  Since for any $i$, $\bar{z}_i > 0$ and, according to Proposition~\ref{prop:sign}, there exists $j_0$ such that $((D^* A)^* e_{\sigma^{-1}(i)})_{j_0} > 0$, one concludes that $(\Gamma D^* \bar{x})_i \neq 0$.

  $(\Rightarrow: \Sl \subseteq \rint \Xl)$.
  We are going to prove that $\Sl = \rint \Sl$.
  Indeed, according to $(\Leftarrow)$, $\rint \Xl \subseteq \Sl$.
  Moreover, since every element of $\Sl$ is also an element of $\Xl$, we have $\rint \Xl \subseteq \Sl \subseteq \Xl$.
  In particular, $\aff \Xl = \aff \Sl$.
  Let
  \begin{equation*}
    \alpha = \min_{i \in \supp(D^* \maxx)} \abs{(D^* \maxx)_i} = \min_{i \in \ens{1,\dots,m}} (\Gamma D^* \maxx)_i
  \end{equation*}
  where $\maxx$ is an element of $\Sl$.
  Note that according to Lemma~\ref{lem:kernel-one-image}, since $\Xl$ is not reduced to a singleton, then $\supp(D^* \maxx)$ has cardinal greater than 1, hence $\alpha > 0$.

  Now take any $u \in B_\infty(\maxx, r) \cap \aff \Xl$ where
  \begin{equation*}
    r = \frac{\alpha - \epsilon}{\norm{\Gamma D^*}_{\infty,\infty}},
  \end{equation*}
  and $0 < \epsilon < \alpha$.

  Let's prove first that $\Gamma D^* u \in (\RR_+^*)^m \times \ens{0}^{p-m}$.
  From the definition of $u$, we get that
  \begin{equation*}
    \normi{\Gamma D^* u - \Gamma D^* x} \leq \norm{\Gamma D^*}_{\infty,\infty} \normi{u - x} \leq \alpha - \epsilon .
  \end{equation*}
  For $i \in \ens{1,\dots,m}$, one has $\abs{(\Gamma D^* u)_i - (\Gamma D^* x)_i} \leq \alpha - \epsilon$. In particular one has
  \begin{equation*}
    (\Gamma D^* u)_i - (\Gamma D^* x)_i \geq -\alpha + \epsilon \Leftrightarrow (\Gamma D^* u)_i \geq (\Gamma D^* x)_i - \alpha + \epsilon .
  \end{equation*}
  Since $(\Gamma D^* x)_i - \alpha \geq 0$ and $\epsilon > 0$, we conclude that $(\Gamma D^* u)_i > 0$.
  Thus, $(\Gamma D^* u)_i > 0$ for $i \in \ens{1,\dots,m}$ and $(\Gamma D^* u)_i = 0$ for $i \not\in \ens{1,\dots,m}$.

  It remains to prove that $u$ is a solution of~\eqref{eq:p}, i.e. $u \in \Xl$.
  Since $u \in \aff \Xl$, there exists $t \in \RR$ and $x \in \Xl$ such that
  \begin{equation*}
    u = \maxx + t (x - \maxx) .
  \end{equation*}
  From this equality, we get that
  \begin{align*}
    \normu{D^* u} &= \normu{\Gamma D^* u} = \sum_{i=1}^p (\Gamma D^* u)_i && \text{according to Corollary~\ref{cor:diagpos}}  \\
                  &= \sum_{i=1}^p (1-t) (\Gamma D^* \maxx)_i + t (\Gamma D^* x)_i && \\
                  &= (1-t) \normu{\Gamma D^* \maxx} + t \normu{\Gamma D^* x} && \\
                  &= \normu{D^* \maxx} && \text{since } \normu{D^* \maxx} = \normu{D^* x} .
  \end{align*}
  Moreover, $\Phi u = \Phi \maxx + t(\Phi x - \Phi \maxx) = \Phi \maxx$.
  Thus, $u$ is a solution which concludes our proof.
\end{proof}

\section{Finding a Maximal Solution}
\label{sec:finding}

Using the classical barrier function, in this section we show how to get a path that converges to a relative interior point of $\Xl$, which turns out to be the analytic center of $\Xl$.

Setting $Q = \Phi^* \Phi$ is the Gram matrix and $c = \Phi^* y$, we start by rewriting our initial problem~\cref{eq:p} as an augmented quadratic program under constraints, i.e.
\begin{equation*}
  \umin{x \in \RR^n, t \in \RR^p}
  \frac{1}{2} \dotp{Qx}{x} - \dotp{c}{x} + \lambda \sum_{i=1}^p t_i
  \qsubjq
  \begin{cases}
    -t \leq D^* x \leq t &\\
    t_i \geq 0 &\\
  \end{cases} ,
\end{equation*}
witch also can be rewritten as
\begin{equation*}
  \umin{x \in \RR^n, t \in \RR^p}
  \frac{1}{2} \dotp{Qx}{x} - \dotp{c}{x} + \lambda \sum_{i=1}^p t_i
  \qsubjq
  \begin{cases}
  -t+s=D^*x&\\
  t-s'=D^*x&\\
    t_i \geq 0,\ s_i\geq0,\ s'_i\geq0 &\\
  \end{cases}.
\end{equation*}
Now observe that $t=\displaystyle{1\over 2}(s+s')$. Then setting $z=\displaystyle{1\over 2}\left(\begin{array}{l}s\cr s'\end{array}\right)$, $I_p$ the $p$ by $p$ identity matrix, ${\tilde I}=\left(\begin{array}{lr}I_p& -I_p\end{array}\right)$ and $e=(1, \cdots,1)\in\RR^{2p}$, we come to the following equivalent formulation of the problem
\begin{equation}
\label{eq:paug}
\umin{x\in\RR^n, z\in\RR^{2p}}
f(x,z)
\qsubjq z\in[0,+\infty)^{2p}
\end{equation}
where $$
f(x,z)=\left\{\begin{array}{ll}\displaystyle{1\over2}\dotp{Qx}{x} - \dotp{c}{x} +\lambda\dotp{e}{z}&\mbox{ if }D^*x+{\tilde I}z=0\\
+\infty&\mbox{ elsewhere,}\end{array}\right.$$ 
or equivalently 
$$f(x,z)=\left\{\begin{array}{ll}\displaystyle{1\over2}\|\Phi x-y\|^2-\displaystyle{1\over2}\|y\|^2 +\lambda\dotp{e}{z}&\mbox{ if }D^*x+{\tilde I}z=0\\
+\infty&\mbox{ elsewhere.}\end{array}\right.$$ 


Its classical dual is
\begin{equation}
\label{eq:daug}
\umax{x\in\RR^n, s\in\RR^{2p}, u\in\RR^p}
g(x,s,u)
\qsubjq s\in[0,+\infty)^{2p}
\end{equation}
where 
$$g(x,s,u)=\left\{\begin{array}{ll}
-\displaystyle{1\over 2}\langle Qx,x\rangle&\mbox{if }{D} u+c-Qx=0,\ s=\lambda e-{\tilde I}^*u\cr 
-\infty&\mbox{elsewhere.}\end{array}\right.$$ 
We set $S_{(P)}$ (resp. $S_{(D)}$) the optimal solutions' set of
problem~\cref{eq:paug} (resp. problem~\cref{eq:daug}).
We know that $\Xl$ is non-empty and so $S_{(P)}$.
Since, in addition~\cref{eq:paug} is a convex problem with polyedral constraints, $S_{(D)}$ is non empty and there is no duality gap. We denote by $\alpha$ the optimal value of the two problems. 

\begin{proposition}\label{dcompacity}{$ $}

\begin{itemize}
\item[1.]The optimal solution $S_{(P)}$ of the problem (\ref{eq:paug}) is bounded or equivalently the set $\{(d_x,d_z):\ f_\infty(d_x,d_z)\leq0,\ d_z\geq0\}=\{0\}$,
\item[2.]$S(.,(D))=\{(s,u):\ \exists x\in\RR^n\mbox{ such that }(x,s,u)\in S_{(D)}\}$ is bounded, in other words, the dual feasible solutions' set is bounded in $(s,u)$.
\end{itemize}
\end{proposition}
\begin{proof}
1. Because of relation (\ref{eq:hyp-inv}) it is not difficult to show that the optimal solution $S_{(P)}$ of the problem (\ref{eq:paug}) is bounded.


2. Let $(x^k,s^k,u^k)$ be a sequence of the dual feasible solutions' set. We have $s^k=\lambda e-{\tilde I}^*u=\left(\begin{array}{l}\lambda e^p\cr\lambda e^p\end{array}\right)-\left(\begin{array}{l}u^k\cr-u^k\end{array}\right)\geq0$, where $e^p=(1,\cdots 1)\in\RR^p$. It follows that $-\lambda e^p\leq u^k\leq \lambda e^p$. Hence $(u^k)$ and then $(s^k)$, is bounded.
\end{proof}




Using the classical logarithmic barrier function introduced by Frish~\cite{frisch}, we deal with the family of problems $(P_\mu)_{\mu>0}$ given by
\begin{equation*}
\theta(\mu)=\umin{x\in\RR^n, z\in\RR^{2p}}
F_{\mu}(x,z)=f(x,z)+\zeta(z,\mu)
\end{equation*}
where $$\zeta(z,\mu)=\left\{\begin{array}{ll}
\mu \xi\left(z/\mu\right)&\mbox{if }\mu>0,\cr\xi_\infty(z)&\mbox{if }\mu=0,\cr+\infty&\mbox{elsewhere,}
\end{array}\right.$$ $$ \xi(z)=\left\{\begin{array}{ll}-\ln \varphi(z)&\mbox{if }\varphi(z)>0,\cr+\infty&\mbox{elsewhere,}\end{array}\right. \mbox{ and }\varphi(z)=\left\{\begin{array}{ll}\left(\prod\limits_{i=1}^{2p}z_i\right)^{1\over2p}&\mbox{if }z\geq0,\cr-\infty&\mbox{elsewhere.}\end{array}\right.$$


Note that the function $\varphi$ is strictly quasiconcave and then according to Lemma 1 of \cite{barbara2015strict}, for every $\mu>0$, the function $\zeta_\mu:z\mapsto\zeta(z,\mu)$ is strictly convex on $(0,+\infty)^{2p}$. 
\begin{proposition}\label{strict_convexity}
For every $\mu>0$, the function $F_{\mu}$ is inf-compact on $\RR^n\times\RR^{2p}$ and strictly convex on $\RR^n\times(0,+\infty)^{2p}$.
\end{proposition}
\begin{proof} Let us show that  
\begin{eqnarray}\label{xiinfty}
\xi_\infty(d)=\left\{\begin{array}{ll}0&\mbox{if }d\geq0,\cr+\infty&\mbox{elsewhere.}\end{array}
\right.\end{eqnarray}
Let $(z,d)\in \dom(\xi)\times\RR^{2p}$. We have necessarily $z>0$. First we observe that when $d\not\in[0,+\infty)^{2p}$, $z+\lambda d\not\in[0,+\infty)^{2p}$ for $\lambda$ large enough and then $\xi_\infty(d)=+\infty$. Now consider the case $d\geq0$. Since $z>0$ we have necessarily $z+d>0$. The concave gauge function $\varphi$ is monotone with respect to its domaine the positive orthant. Then by Proposition 2.1 of \cite{barbara_crouzeix},
$$0<\varphi(z+d)\leq\varphi(z+\lambda d)\leq\varphi(\lambda z+\lambda d)=\lambda\varphi(z+d)$$
for $\lambda$ large enough. It follows that
$$\begin{array}{ll}
0=\lim\limits_{\lambda\uparrow+\infty}\displaystyle{\ln\varphi(z+d)-\ln\varphi(z)\over\lambda}&\leq\lim\limits_{\lambda\uparrow+ \infty}\displaystyle{\ln\varphi(z+\lambda d)-\ln\varphi(z)\over\lambda}\cr&\leq\lim\limits_{\lambda\uparrow+\infty} \displaystyle{\ln\lambda\varphi(z+d)-\ln\varphi(z)\over\lambda}=0\end{array}$$ and hence $\lim\limits_{\lambda\uparrow+ \infty}\displaystyle{\ln\varphi(z+\lambda d)-\ln\varphi(z)\over\lambda}=0$. Consequently $\xi_\infty(d)=0$.

By Proposition \ref{dcompacity}, we have $\{(d_x,d_z):\ f_\infty(d_x,d_z)\leq0,\ d_z\geq0\}=\{(0,0)\}$. Thus
$\{(d_x,d_z):\ {F_\mu}_\infty(d_x,d_z)\leq0,\ d_z\geq0\}=\{(0,0)\}$, or equivalently, $F_\mu$ is inf-compact. 

Now let us proceed to prove the strict convexity of $F_\mu$.  Take  $(x,z)\not=(x',z')$ in $\RR^n\times(0,+\infty)^{2p}$ and $t\in(0,1)$. In the case where $z\not= z'$, by strict-convexity of $\zeta_\mu$ on $(0,+\infty)^{2p}$ we have necessarily $F_\mu(t(x,z)+(1-t)(x',z'))<tF_\mu(x,z)+(1-t)F_\mu(x',z').$ Assume that $z=z'$. Using (\ref{eq:hyp-inv}) and the definition of $f$ we obtain $\Phi x\not=\Phi x'$ and the result follows by using the strict convexity of $\|.\|_2^2$.
\end{proof}
Propositions \ref{strict_convexity} and \ref{dcompacity} assert that for every $\mu>0$ there is a unique optimal solution  $(x(\mu),z(\mu))$ to $(P_\mu)$. Moreover using the fact that $F_\mu(x,\cdot)$ is a barrier function for every $x\in\RR^n$, $z(\mu)>0$. Consider the function $\gamma:\RR^n\times[0,+\infty)^{2p}\times[0,+\infty)\to \RR\cup\{+\infty\}$ defined by
$$\gamma(x,z,\mu)=F_{\mu}(x,z).$$
Then we have the following proposition.

\begin{proposition}
\label{coercivity}
The function $\gamma$ is convex and lsc on $\RR^n\times\RR^{2p}\times[0,+\infty)$. It is inf-compact on $\RR^n\times\RR^{2p}\times[0,{\overline \mu}]$, $\forall\overline{\mu}>0$ being fixed. Moreover $\theta$ is convex and continuous on $[0,+\infty)$, $\theta(0)=\alpha$ and $f(x,z)=\gamma(x,z,0)$, $\forall (x,z)\in\RR^n\times(0,+\infty)^{2p}$.
\end{proposition}
\begin{proof}
It is known that the function $\zeta$ is convex on $\RR^{2p}\times[0,+\infty)$
and so is $\gamma$. The function $\theta$ is then convex on $[0,+\infty)$ as the
infimum over $(x,z)$ of a convex function in $(x,z,\mu)$. Now the function
$\zeta(z,.)$ is continuous on $[0,+\infty)$ and, because of (\ref{xiinfty}),
$\zeta(z,0)=0$ for all $z\in(0,+\infty)^{2p}$. Thus $f(x,z)=\gamma(x,z,0)$ for
all $(x,z)\in\RR^n\times(0,+\infty)^{2p}$ and therefore $\theta(0)=\alpha$ (the
optimal value of the problem (\ref{eq:paug})). Set
$\tilde{\gamma}=\gamma_{|\RR^n\times\RR^{2p}\times[0,\overline{\mu}]}$ the
restriction of $\gamma$ to the set
$\RR^n\times\RR^{2p}\times[0,\overline{\mu}]$. Then $\{(d_x,d_z,\mu):\
\tilde{\gamma}_\infty(d_x,d_z,\mu)\leq0,\ d_z\geq0,\ \mu=0\}=\{(d_x,d_z,0):\
f_\infty(d_x,d_z)\leq0,\ d_z\geq0\}=\{(0,0,0)\}$ (see Proposition
\ref{dcompacity}). The function $\gamma$ is then inf-compact on
$\RR^n\times\RR^{2p}\times[0,{\overline \mu}]$. Consequently, there is a compact
$\tilde{S}$ such that $(x(\mu),z(\mu))\in\tilde{S}$,
$\forall\mu\in(0,\overline{\mu}]$, i.e.,
$(x(\mu),z(\mu))_{\mu\in(0,\overline{\mu})}$ is bounded. We established that $\theta$ is convex on $[0,+\infty)$. It is then continuous on $(0,+\infty)$. Let us show now that $\lim\limits_{\mu\downarrow 0}\theta(\mu)=\theta(0)=\alpha$. In this respect we shall prove that 
$\lim\limits_{\mu\downarrow0}\mu\ln\left(\displaystyle{\varphi(z(\mu)) \over\mu}\right)= 0$. Let $(\mu^k)_{k\in\NN}$ be a positive sequence such that $\lim\limits_{k\uparrow+\infty}\mu^k=0.$ We established that $(x(\mu),z(\mu))_{\mu\in(0,\overline{\mu}]}$ is bounded. It follows that the set $\{(x(\mu^k),z(\mu^k))\}$ contains a subsequence converging to a point  $(\tilde{x},\tilde{z})$.
In the case where $\tilde{z}>0$ the result is obvious. Assume that $\varphi(\tilde{z})=0$. Then for $k$ sufficiently large one has
$$\begin{array}{ll}\alpha-\mu^k\ln\left(\displaystyle{\varphi(z)\over\mu^k}\right)
\leq \theta(\mu^k)&=f(x(\mu^k),z(\mu^k))-\mu^k\ln\left(\displaystyle{\varphi(z(\mu^k)) \over \mu^k}\right)\cr&
\leq f(x,z)-\mu^k\ln\left(\displaystyle{\varphi(z)\over\mu^k}\right)
\end{array}$$
for every $(x,z)$ satisfying $z>0$. Since $\lim\limits_{k\uparrow 0}\mu^k\ln\left(\displaystyle{\varphi(z)\over\mu^k}\right)=0$, we have
$$\alpha\leq\lim\inf\limits_{k\uparrow+\infty}\theta(\mu^k)\leq f(x,z)$$
and then
$$\alpha\leq\lim\sup\limits_{k\uparrow+\infty}\theta(\mu^k)\leq \inf\limits_{x,z}\{f(x,z):\ z>0\}=\inf\limits_{x,z}\{f(x,z):\ z\geq0\}=\alpha.$$
Consequently $\lim\limits_{k\uparrow+\infty}\theta(\mu^k)=\alpha$.
\end{proof}
\bigskip
\bigskip


Given $\mu>0$, the KKT optimalty conditions for the problem $(P_\mu)$ can be formulated, for some $u\in\RR^p$, as
$$\left\{\begin{array}{ll}Qx(\mu)-c-Du=0,\\ \lambda e-\displaystyle{ \mu\over 2p}(Z(\mu))^{-1}e-{\tilde I}^*u=0,\\ D^*x(\mu)+{\tilde I}z(\mu)=0,\end{array}\right.$$
where $Z(\mu)=diag(z(\mu))$.
Observe that $u$ is necessarily unique. Put 
$$u=u(\mu)\mbox{ and }s(\mu)=\displaystyle {\mu\over 2p}Z^{-1}(\mu)e.$$ We rewrite the KKT conditions as
$$\left\{\begin{array}{lr}Qx(\mu)-c-Du(\mu)=0&(E1)\\ \lambda e-s(\mu)-{\tilde I}^*u(\mu)=0&(E2)\\ Z(\mu)s(\mu)=\displaystyle {\mu\over 2p}e&(E3)\\D^*x(\mu)+{\tilde I}z(\mu)=0&(E4)\end{array}\right.$$

\begin{proposition}
For every $\mu>0$, $(s(\mu),u(\mu))$ is a feasible solution to (\ref{eq:daug}) and $\big((s(\mu),u(\mu)\big)_{\mu\in(0,\overline{\mu}]}$ is bounded.
\end{proposition}

\begin{proof}
By $(E1)$, $(E2)$ and the fact that $s(\mu)=\displaystyle {\mu\over 2p}(Z(\mu))^{-1}e>0$, $(u(\mu),s(\mu))$ is a feasible solution to (\ref{eq:daug}). The boundedness of $(s(\mu),u(\mu))_{\mu\in(0,{\overline\mu}]}$ is due to Proposition \ref{dcompacity}.
\end{proof} 


Set ${\overline I}=\displaystyle\bigcup_{\atop z\in S(.,(P))}I(z)$ and ${\overline J}=\displaystyle\bigcup_{\atop s\in S(.,(D))}J(s)$, where 
$$S(.,(P))=\left\{z:\ \exists x\in\RR^n\mbox{ such that } (x,z)\in S_{(P)}\right\},$$ 
$$S(.,(D))=\left\{s:\ \exists u\in\RR^p\mbox{ such that } (s,u)\in S_{(D)}\right\},$$ $$I(z)=\{i:\ z_i>0\}\mbox{ the support of }z\mbox{ and }J(s)=\{i:\ s_i>0\}\mbox{ the support of }s.$$  

\begin{lemma}\label{complementarity}

There is at least one $({\hat z},\hat{s})\in S(.,(P))\times S(.,(D))$ such that ${\overline I}=I({\hat z})$ and $\overline{J}=J(\hat{s})$.
\end{lemma}

\begin{proof}
We have ${\overline I}$ a subset of a finite set $\{1,\cdots,2p\}$. Let then $(z^1,\ z^2,\cdots,z^k)\in S(.,(P))^k$, for some $k\in\{1,2,\cdots,2p\}$
satisfying ${\overline I}=I\left(z^1\right)\cup I\left(z^2\right)\cup\cdots\cup I\left(z^k\right)$. Set ${\hat z}=\displaystyle{1\over k}\left(z^1+z^2+\cdots+z^k\right)$. Since $S(.,(P))$ is convex  ${\hat z}\in S(.,(P))$. So it is easy to see that $I(z^i)\subset I({\hat z})$, $\forall i\in\{1,2,\cdots,k\}$. The result then follows. A vector $\hat{s}$ is constructed in a similar way.
\end{proof} 
Observe that every optimal solution $(x,z)$ of the problem (\ref{eq:paug}) satisfying $I(z)=\overline{I}$ is in the relative interior of $S_{(P)}$. Similarily every optimal solution $(x,s,u)$ of the problem (\ref{eq:daug}) satisfying $J(s)=\overline{J}$ is in the relative interior of $S_{(D)}$.

\bigskip

Set 
$$({\overline x},{\overline z})=\arg\max\left\{\varphi_{\overline I}(z_{\overline I}):\ \displaystyle{1\over 2}\langle Qx,x\rangle-\langle c,x\rangle+\lambda\langle e,z\rangle=\alpha,\ D^*x+{\tilde I}z=0,\ z_{\overline{J}}=0\right\},$$
where 
$$\varphi_{\overline I}(z_{\overline I})=\left\{\begin{array}{ll}
\left(\prod\limits_{i\in {\overline I}}z_i\right)^{1\over \card({\overline I})}&\mbox{if }z_J\in(0,+\infty)^{\card(J)}\cr
-\infty&\mbox{elsewhere.}
\end{array}\right.
$$
Symmetrically we set
$$({\overline s},{\overline u})=\arg\max\left\{\varphi_{\overline J}(s_{\overline J}):\ 
s=\lambda e-\tilde{I}^*u,\ D u+c-Q{\overline x}=0, s_{\overline I}=0\right\},$$
where
$$\varphi_{\overline{J}}(s_{\overline{J}})=\left\{\begin{array}{ll}\left(\prod\limits_{i\in \overline{J}}s_i\right)^{1\over \card(\overline{J})}&\mbox{if }s_{\overline{J}}\in(0,+\infty)^{\card(\overline{J})}\cr-\infty&\mbox{elsewhere.}\end{array}\right.
$$
$(\overline{x},\overline{z})$ is called the analytic center\footnotemark[1]\footnotetext[1]{A generalization of the central path and the analytic center is proposed in \cite{barbara2015strict} by using the so called concave gauge functions.} of (\ref{eq:paug}) and $(\overline{x},\overline{s},\overline{u})$ the analytic center of (\ref{eq:daug}). The uniqueness is ensured by the strict quasiconcavity of functions $\varphi_{\overline{I}}$ and  $\varphi_{\overline{J}}$ on the interior of their respective domain and the assumption (\ref{eq:hyp-inv}). We now give an important result. 



Its proof is inspired in part by those of Theorems I.7 and I.9 in \cite{RoTevi}.
\begin{theorem}\label{thm:convergence}
Under assumption \ref{eq:hyp-inv}, we have $$\lim\limits_{\mu\downarrow0}(x(\mu),z(\mu),s(\mu),u(\mu))=({\overline x},{\overline z},{\overline s},{\overline u}).$$ Moreover, $({\overline x},{\overline z})$  and $({\overline x},{\overline s},{\overline u})$ belong to the relative interior of $S_{(P)}$ and $S_{(D)}$, respectively. 
\end{theorem}

\begin{proof}
We proved that $\big((x(\mu),z(\mu)\big)_{\mu\in(0,\overline{\mu}]}$ and $\big((s(\mu),u(\mu)\big)_{\mu\in(0,\overline{\mu}]}$ are bounded. Let $(\mu^k)_{k\in \NN}$ a positive increasing sequence satisfying $$\lim\limits_{k\uparrow +\infty}\mu^k=0\mbox{ and }\lim\limits_{k\uparrow+\infty}(x(\mu^k),z(\mu^k),s(\mu^k),u(\mu^k))=(\tilde{x},\tilde{z},\tilde{s},\tilde{u}).$$ Then replacing $\mu$ by $\mu^k$ in $(E1)-(E4)$ and letting $k$ tend to $+\infty$, we observe that the pair $\{(\tilde{x},\tilde{z}),(\tilde{x},\tilde{s},\tilde{u})\}$ satisfies the KKT optimality conditions of (\ref{eq:paug}) and then it is a primal-dual optimal solution pair of (\ref{eq:paug}). Let us show now that $I(\tilde{z})=\overline{I}$ and $J(\tilde{s})=\overline{J}$. 
Now by $(E1)$, $(E2)$ and $(E4)$ we have  
$$\left(\begin{array}{l}
x(\mu^k)-\overline{x}\cr
z(\mu^k)-\overline{z}\end{array}\right)\in\Ker{\left(\begin{array}{ll}D^*&{\tilde I}\end{array}\right)}\mbox{ and }
\left(\begin{array}{l}
Q(x(\mu^k)-\overline{x})\cr
-(s(\mu^k)-\overline{s})\end{array}\right)\in\Im{\left(\begin{array}{l}D \\ {\tilde I}^*\end{array}\right)}.
$$
Then using the following orthogonality property 
\begin{equation}
\label{orthogonality}
\Ker{\left(\begin{array}{ll}{D^*}&{\tilde I}\end{array}\right)}=\left[\Im{\left(\begin{array}{l}D \\ {\tilde I}^*\end{array}\right)}\right]^\bot,
\end{equation} 
$(E3)$ and the fact that $\langle\overline{z},\overline{s}\rangle=\langle\tilde{z},\tilde{s}\rangle=0$ we have 
$$\langle\overline{z},s(\mu^k)\rangle+\langle\overline{s},z(\mu^k)\rangle=\mu^k-\langle Q(x(\mu^k)-\overline{x}),x(\mu^k)-\overline{x}\rangle.$$ 
Since in addition $I(\overline{z})=\overline{I}$, $J(\overline{s})=\overline{J}$ and $Q$ is positive semi-definite we get
$$\sum\limits_{i\in \overline{I}}\overline{z}_is(\mu^k)_i+\sum\limits_{i\in \overline{J}}\overline{s}_iz(\mu^k)_i=\mu^k-\langle Q(x(\mu^k)-\tilde{x}),x(\mu^k)-\tilde{x}\rangle\leq\mu^k.$$
But from $(E3)$, $z(\mu^k)_is(\mu^k)_i=\displaystyle{\mu^k\over 2p},\ \forall i$. it follows that
$$\displaystyle\sum\limits_{i\in \overline{J}}{\overline{s}_i\over s(\mu^k)_i}+\sum\limits_{i\in \overline{I}}{\overline{z}_i\over z(\mu^k)_i}\leq 2p.$$
Now letting $k$ tend to $+\infty$, we get on the one hand
$$0<\displaystyle\sum\limits_{i\in \overline{J}}{\overline{s}_i\over \tilde{s}_i}+\sum\limits_{i\in \overline{I}}{\overline{z}_i\over \tilde{z}_i}\leq 2p<+\infty$$
and then, by construction of $\overline{I}$ and $\overline{J}$, we have necessarily $I(\tilde{z})=\overline{I}$ and $J(\tilde{s})=\overline{J}$. On the other hand,
using the arithmetic-geometric mean inequality we get
$$\left(\prod\limits_{i\in \overline{J}}\displaystyle{\overline{s}\over \tilde{s}_i}\prod\limits_{i\in \overline{I}}\displaystyle{\overline{z}\over \tilde{z}_i}\right)^{1\over 2p}\leq{1\over 2p}\left(\displaystyle\sum\limits_{i\in \overline{J}}{\overline{s}\over \tilde{s}_i}+\sum\limits_{i\in \overline{I}}{\overline{z}\over \tilde{z}_i}\right)\leq 1$$
and then 
$$\varphi_{\overline{J}}( \overline{s}_{\overline{J}})\varphi_{\overline{I}}( \overline{z}_{\overline{I}})\leq\varphi_{\overline{J}}( \tilde{s}_{\overline{J}})\varphi_{\overline{I}}( \tilde{z}_{\overline{I}}).$$
But, by definition of $(\overline{x},\overline{z},\overline{s},\overline{u})$, $\varphi_{\overline{J}}( \tilde{s}_{\overline{J}})\leq \varphi_{\overline{J}}( \overline{s}_{\overline{J}})$ and $\varphi_{\overline{I}}( \tilde{z}_{\overline{I}})\leq \varphi_{\overline{I}}( \overline{z}_{\overline{I}})$.  The result then follows. 
\end{proof}

Consequently, the following corollary holds
\begin{corollary}
  Under assumption (\ref{eq:hyp-inv}), we have $\lim\limits_{\mu \downarrow 0} x(\mu) = \bar x \in \rint \Xl$.
\end{corollary}

\begin{proof}
By  Theorem \ref{thm:convergence} $(\overline{x},\overline{z})$ belongs to  the relative interior of $S_{(P)}$ and hence $\overline{x}$ belongs to the linear projection of the relative interior of $S_{(P)}$ which is equal to $\rint \Xl$.
\end{proof}
\section{Efficiently Recognizing {\em Cyclic Hyper Degrees}}
This section consists of two parts. In the first part will culminate with Theorem~\ref{thm:shiftrange} which provides an efficiently computable closed form formula for the range of values taken by contiguous sum of $N$ elements in a list $c_{i,n}$, for any $i$. In the second part we will show how to use Theorem~\ref{thm:shiftrange} to decide if a given degree sequence is a {\em cyclic hyper degree}.

The elements in the columns of $T_n$ do not change their relative position after application of a cyclic permutation when seen as a cyclic list. We shall use this property to efficiently search for possible bit subsets which may sum up to a given input degree sequence.

\subsection{Contiguous Sum of Bit Lists}

\begin{definition}[Contiguous Sum]
\label{def:csum}
 Given a list $L$ of length $m$, the contiguous sum of $N$ elements in $L$ starting at the index $i\in [m]$ is defined to be
 $$\mathcal{S}(L,i,N):=\sum_{j=0}^{N-1} L(1+ ((i+j-1)\mod m)).$$
\end{definition}

The summation above treats the list $L$ as a cyclic list. Next, we prove that the contiguous sum function is a `continuous' function, this property will allow us to specify the range of sum by stating the minimum and the maximum value taken by it.
 Note that if $L$ is a $0$-$1$ list, for any index $\ell \in [m]$, 
 we have $\vert \mathcal{S}(L,\ell,N)-\mathcal{S}(L,\ell+1,N) \vert \in \{0,1\}$. This fact gives us the following property.

\begin{observation}
\label{obs:continuity}
 Let $L$ be a size $m$ list having $0$-$1$ entries and $N\in \mathbb{Z}_+$.
 If $v_i=\mathcal{S}(L,i,N)$ and $v_j=\mathcal{S}(L,j,N)$, for some $i,j\in[m]$, then 
 for every $v\in \mathbb{Z}_+$ contained between  $v_i$ and $v_j$
 there exists a $k\in [m]$  such that $\mathcal{S}(L,k,N)=v$.
\end{observation}

As the lists $c_{i,n}$ are over $0$-$1$ we get an easy relation between the maximum and minimum values taken by the contiguous sum as follows.

\begin{lemma}
\label{lem:sumcomplement}
  Let  $j\in\{0,\dots,n\}$, $i \in [n]$ and $N\in[2^n]$. The minimum of the sum of $N$ contiguous bits in a bit list $c_{i,n}$ is $m$ if and only if its maximum is $N-m$.
\end{lemma}
\begin{proof}
 Let $\overline{c}_{i,n}$ be the bit list obtained from the list $c_{i,n}$ by flipping each zero to one and vice versa. Let $\sigma_{2^{i-1}}$ be an order $2^{i-1}$ cyclic permutation, observe that $\overline{c}_{i,n}$ is equal to $\sigma_{2^{i-1}}(c_{i,n})$.
 If the minimum value is obtained at the contiguous segment which starts at the index $j$ in $c_{i,n}$, then the value $N-m$ can be obtained by the contiguous sum starting at index $\sigma_{2^{i-1}}(j)$. Finally, note that $m$ is the minimum value if and only if $N-m$ is the maximum value.
\end{proof}

Combining Observation~\ref{obs:continuity} and Lemma~\ref{lem:sumcomplement} we get the following.

\begin{lemma}
\label{lem:minmax}
 Let $N\in[2^n]$ and $m=\min_{j \in [2^n]} \mathcal{S}(c_{i,n}, j, N)$. For every value $v$ in the range $\{m,\dots, N-m\}$ there exists a $j\in [2^n]$ such that $\mathcal{S}(c_{i,n}, j, N)=v$.
\end{lemma}

The lemma above allows us to find the range of values taken by the contiguous sum by just finding the minimum value taken by it.
Next we prove a simpler lemma about the range of values taken. Using that, in Theorem~\ref{thm:shiftrange}, we will find the range of values taken by the contiguous sum of $N$ elements in any list $c_{i,n}$.

\begin{lemma}
\label{lem:shiftpower}
 For  $j\in\{0,\dots,n\}$ and $i \in [n]$, the sum of $2^j$ contiguous bits in a bit list $c_{i,n}$ takes the following values.
\begin{enumerate}
\item
\label{enum:shiftpower-one}
If $j \leq (i-1)$, then the range is $\{0, \dots,2^j \}$, and
\item
\label{enum:shiftpower-two}
If $j\geq i$, then the sum is exactly $2^{j-1}$.
\end{enumerate}
\end{lemma}
\begin{proof}
 By Lemma~\ref{lem:minmax}, it suffices to find the minimum value of contiguous sum function. Notice that we have, $c_{i,n}= (0_{\times 2^{i-1}} \cdot 1_{\times 2^{i-1}})_{\times 2^{n-i}}$, by Observation~\ref{obs:zero-one-power}.
 \begin{enumerate}
 \item
 When $j\leq (i-1)$, we can pick a block of $2^j$ zeros giving a total of zero, which is the minimum possible value.
 \item
 When $j\geq i$, let $L_k$ be a list of $2^j$ contiguous bits of $c_{i,n}$ starting at the index $k$ in $c_{i,n}$. To prove that $\sum L_k = \sum L_{k+1} $, it suffices to show that $c_{i,n}(k)=c_{i,n}(k+2^j)$. Rewriting
 $c_{i,n}=((0_{\times 2^{i-1}}\cdot 1_{\times 2^{i-1}})_{\times 2^{j-i}})_{\times 2^{n-j}}$ shows that any two indices with difference equal to $2^j$ store the same value. As the choice of $k$ was arbitrary, the contiguous sum is equal to $2^{j-1}$.
 \end{enumerate} 
\end{proof}

\begin{theorem}
\label{thm:shiftrange}
 For $i \in [n]$, $N\in[2^{n}]$ and $p=2^i$, the sum of $N$ contiguous bits in a bit list $c_{i,n}$ takes values in the range, ${\rm range}(i,N) \triangleq$
 $$ 
 \left\{
	\floor[\Big]{\frac{N}{p}} \frac{p}{2} + \max \left( (N\mod p) - \frac{p}{2}, 0 \right),
  	\cdots,
   	\floor[\Big]{\frac{N}{p}} \frac{p}{2} + \min \left( N\mod p, \frac{p}{2} \right)
 \right\}.
 $$
\end{theorem}
\begin{proof}
 For a fixed $i\in[n]$ consider the list $c_{i,n}$. Assuming that the minimum value of the range is as claimed, by Lemma~\ref{lem:minmax}, the maximum value is
\begin{align*}
   \max_{j \in [2^n]} \mathcal{S}(c_{i,n}, j, N)
   &= N-\min_{j \in [2^n]} \mathcal{S}(c_{i,n}, j, N)\\
   &=N- \left( \floor[\Big]{\frac{N}{p}} \frac{p}{2} + \max \left( (N\mod p) - \frac{p}{2}, 0 \right)\right)\\
 &= \floor[\Big]{\frac{N}{p}}p + (N\mod p) - 
 \left( 
 \floor [\Big] {\frac{N}{p}} \frac{p}{2} + \max \left( (N\mod p) - \frac{p}{2}, 0 \right)
 \right)	\\
 &= \floor[\Big]{\frac{N}{p}} \frac{p}{2} +(N\mod p) - 
 \max	\left(		(N \mod p) - \frac{p}{2}, 	0 	\right)	\\
 &= \floor[\Big]{\frac{N}{p}} \frac{p}{2} + \min \left( (N\mod p)-(N \mod p) + \frac{p}{2}, N\mod p\right)\\
 &=  \floor[\Big]{\frac{N}{p}} \frac{p}{2} + \min \left(\frac{p}{2}, N\mod p\right).
\end{align*}
 
 As proved in case~\ref{enum:shiftpower-two} of Lemma~\ref{lem:shiftpower}, the sum of $\floor{\frac{N}{p}}p$ contiguous bits is equal to $\floor{\frac{N}{p}}\frac{p}{2}$  irrespective of the starting index.
 Therefore, it suffices to find the minimum sum of $R=(N\mod 2^i)$ contiguous bits. Let $L_k$ be a list of $R$ bits occurring contiguously in $c_{i,n}$ starting at index $k$. If the first bit of $L_k$ is $1$, then $\sum L_{k+1}\leq \sum L_k$. Therefore, we can keep on increasing the value of $k$ until the first bit is zero,  without increasing the value of the contiguous sum. On the other hand, if $c_{i,n}(k-1)=0$, then $\sum L_{k-1}\leq \sum L_k$. Therefore, we can keep on decreasing the value of $k$ one at a time until $c_{i,n}(k-1)=1$, without increasing the value of the contiguous sum. Thus the minimum value of the contiguous sum is achieved when the index $k$ points to the start of any block $0_{\times 2^{i-1}}$ contained in $c_{i,n}$. The value of minimum is  $\max(R-\frac{p}{2},0)$ as the ones start appearing after $\frac{p}{2}$ indices from the start of a list $0_{\times 2^{i-1}}\cdot 1_{\times 2^{i-1}}$.  Adding it to $\floor{\frac{N}{p}}\frac{p}{2}$ gives the required minimum value.
\end{proof}

\subsection{Algorithm}
We next state a theorem which gives an equivalent definition of {\em cyclic hyper degrees}.
\begin{theorem}
\label{thm:chd}
A list $w=\{w_1,\dots,w_n\}\in \mathbb{Z}_+^n$ is a {\em cyclic hyper degree} if and only if there exist $N\in[2^n]$ and a permutation $\pi$, such that for each $i\in [n]$, $w_{\pi(i)} \in {\rm range}(i,N)$.% (see Theorem~\ref{thm:shiftrange}).
\end{theorem}

\begin{proof}
 Forward direction is a direct consequence of the definition.
 
 Using Definitions~\ref{def:chd} and~\ref{def:csum}, we get that there exist numbers $s_1,\dots,s_n \in [2^n]$ such that for each $i\in [n]$, we have $w_{\pi(i)}=\mathcal{S}(c_{i,n},s_i,N)$. Let $\Pi^{-1}=(\sigma_{s_1}^{-1},\dots,\sigma_{s_n}^{-1})$ be the list of cyclic permutations, where for each $i\in [n]$, $\sigma_{s_i}^{-1}$ is the inverse of the cyclic permutation of order $s_i$. Consider the table $\Pi^{-1}(T_n)$, by
 Theorem~\ref{thm:rotatebits}, all its rows are distinct. In particular, the first $N$ rows are distinct and their sum is $\pi(w)$. Finally, $w\in H_n$ if and only if $\pi(w)\in H_n$.
\end{proof}

Theorem~\ref{thm:shiftrange} gives us a way to efficiently find the number of bits in a contiguous sum of $N$ bits. If we know the number of distinct bit sequences that can sum up to a given vector $w\in \mathbb{Z}_+^n$, then using Theorem~\ref{thm:shiftrange} we can generate all the possible ranges of values which can be taken by each coordinate of the sum. Finally, we need to check if each coordinate of $w$ is contained in different ranges, this corresponds to finding the permutation $\pi$ in Theorem~\ref{thm:chd}. In the next lemma, we will find the number of possible  distinct bit-sequences which can sum up to a given $w$ using cyclic shifts, this corresponds to finding $N$ in Theorem~\ref{thm:chd}.
\begin{lemma}
\label{lem:set-size}
If $w=\{w_1,\dots,w_n\}\in \mathbb{Z}_+^n$ is a {\em cyclic hyper degree}, then the number of bit sequences which sum up to $w$ is an element of the set
$$\mathcal{N}_w\triangleq \{2w_i+j~:~i\in[n], j\in \{-1,0,1\}\}.$$
\end{lemma}
\begin{proof}
 As one of the coordinates of $w$, say $w_k$, is the contiguous sum of $c_{1,n}$, we need to find the number of bits which sum up to $w_k$. From the structure of $c_{1,n}$, it is easily seen that there are just three values viz. $2w_k-1,2w_k,2w_k+1$ which contain $w_k$ in their range of sums. Conversely, for any number $x$ not contained in $\{2w_i+j~:~i\in[n], j\in \{-1,0,1\}\}$, the sum of $x$ contiguous bits $c_{1,n}$ will not contain any of $w_i$, for $i\in [n]$.
\end{proof}

\begin{lemma}
\label{lem:embed}
 Given $w\in\mathbb{Z}_+^n$ and a list of integer intervals $R_1,\dots,R_n \subset \mathbb{Z}_+^2$. There exists an algorithm running in time polynomial in $n$ which correctly answers if there exists a permutation $\pi$ such that  for each $i\in [n]$, $w_{\pi(i)}\in R_i$. 
\end{lemma}
\begin{proof}
 Construct a bipartite graph $G=(A,B,E)$ on $2n$ vertices. Let $A=B=[n]$ and $(i,j)\in E$ if and only if $w_i \in R_j$. Using a polynomial time algorithm one can find if there exists a perfect matching in $G$. If there is a perfect matching then the answer is \YES, otherwise it is \NO.
\end{proof}

\begin{theorem}
 \label{thm:chd-poly}
 There is a polynomial time algorithm in $n$ which decides if a given $w\in\mathbb{Z}_+^n$ is a {\em cyclic hyper degree}.
\end{theorem}

\begin{proof}
 For each $N\in\mathcal{N}_w$, given by Lemma~\ref{lem:set-size}, and $i\in[n]$ compute ${\rm range}(i,N)$ as given by Theorem~\ref{thm:shiftrange}. Now, use Lemma~\ref{lem:embed} on these ranges of numbers and decide if $w$ is a {\em cyclic hyper degree}, if it is not then try the next number from the set $\mathcal{N}_w$. If it succeeds for at least one element of $\mathcal{N}_w$, we answer \YES, otherwise we answer \NO. Finally, note that $\vert N_w \vert\leq 3n$ and all the other steps can be performed in time which is a polynomial function of $n$. 
\end{proof}

\bibliographystyle{siamplain}
\bibliography{biblio}

\end{document}