\documentclass[hidelinks]{siamart0216}
\usepackage[T1]{fontenc}
\usepackage{mystyle}
\usepackage{todonotes}
\usepackage{algorithmicx}
\usepackage{algpseudocode}
\usepackage{subcaption}

\newcommand{\maxx}{x^+}
\newcommand{\minn}{x^-}
\newcommand{\conv}[1]{\mathrm{conv}\ens{#1}}
\DeclareMathOperator{\aff}{aff}
\newcommand{\X}{\mathbf{X}}
\newcommand{\MAX}{\mathbf{S}}
\newcommand{\Xl}{\X_{\lambda}}
\newcommand{\Xo}{\X_{0}}
\newcommand{\Sl}{\MAX_{\lambda}}

\DeclareMathOperator{\dom}{dom}
\DeclareMathOperator{\card}{card}

\newcommand{\TheTitle}{Maximal Solutions of Sparse Analysis Regularization}
\newcommand{\TheAuthors}{A. Barbara, A. Jourani and S. Vaiter}


\ifpdf
\hypersetup{
  pdftitle={\TheTitle},
  pdfauthor={\TheAuthors}
}
\fi

\headers{\TheTitle}{\TheAuthors}

\title{{\TheTitle}}

\author{A. Barbara%
  \thanks{Universit\'e de Bourgogne Franche-Comt\'e, Institut de Math\'ematiques de Bourgogne, UMR 5584 CNRS, (\email{\{barbara,jourani\}@u-bourgogne.fr})}%
  \and
  A. Jourani%
  \footnotemark[1]%
  \and
  S. Vaiter%
  \thanks{CNRS \& Universit\'e de Bourgogne Franche-Comt\'e, Institut de Math\'ematiques de Bourgogne, UMR 5584 CNRS, (\email{vaiter@u-bourgogne.fr})}
}


\begin{document}

\maketitle

\begin{abstract}
  This paper deals with the non-uniqueness of the solutions of an analysis-Lasso regularization.
  Most of previous works in this area is concerned with the case where the solution set is a singleton, or to derive guarantees to enforce uniqueness.
  Our main contribution consists in providing a geometrical interpretation of a solution with a maximal $D$-support, namely the fact that such a solution lives in the relative interior of the solution set.
  With this result in hand, we also provide a way to exhibit a maximal solution using a primal-dual interior point algorithm.
\end{abstract}

\begin{keywords}
  Lasso, analysis sparsity,  inverse problem, support identification, barrier penalization
\end{keywords}

\begin{AMS}
  90C25, % Convex programming
  49J52  % Nonsmooth analysis
\end{AMS}

% !TEX root = ../arxiv.tex

Unsupervised domain adaptation (UDA) is a variant of semi-supervised learning \cite{blum1998combining}, where the available unlabelled data comes from a different distribution than the annotated dataset \cite{Ben-DavidBCP06}.
A case in point is to exploit synthetic data, where annotation is more accessible compared to the costly labelling of real-world images \cite{RichterVRK16,RosSMVL16}.
Along with some success in addressing UDA for semantic segmentation \cite{TsaiHSS0C18,VuJBCP19,0001S20,ZouYKW18}, the developed methods are growing increasingly sophisticated and often combine style transfer networks, adversarial training or network ensembles \cite{KimB20a,LiYV19,TsaiSSC19,Yang_2020_ECCV}.
This increase in model complexity impedes reproducibility, potentially slowing further progress.

In this work, we propose a UDA framework reaching state-of-the-art segmentation accuracy (measured by the Intersection-over-Union, IoU) without incurring substantial training efforts.
Toward this goal, we adopt a simple semi-supervised approach, \emph{self-training} \cite{ChenWB11,lee2013pseudo,ZouYKW18}, used in recent works only in conjunction with adversarial training or network ensembles \cite{ChoiKK19,KimB20a,Mei_2020_ECCV,Wang_2020_ECCV,0001S20,Zheng_2020_IJCV,ZhengY20}.
By contrast, we use self-training \emph{standalone}.
Compared to previous self-training methods \cite{ChenLCCCZAS20,Li_2020_ECCV,subhani2020learning,ZouYKW18,ZouYLKW19}, our approach also sidesteps the inconvenience of multiple training rounds, as they often require expert intervention between consecutive rounds.
We train our model using co-evolving pseudo labels end-to-end without such need.

\begin{figure}[t]%
    \centering
    \def\svgwidth{\linewidth}
    \input{figures/preview/bars.pdf_tex}
    \caption{\textbf{Results preview.} Unlike much recent work that combines multiple training paradigms, such as adversarial training and style transfer, our approach retains the modest single-round training complexity of self-training, yet improves the state of the art for adapting semantic segmentation by a significant margin.}
    \label{fig:preview}
\end{figure}

Our method leverages the ubiquitous \emph{data augmentation} techniques from fully supervised learning \cite{deeplabv3plus2018,ZhaoSQWJ17}: photometric jitter, flipping and multi-scale cropping.
We enforce \emph{consistency} of the semantic maps produced by the model across these image perturbations.
The following assumption formalises the key premise:

\myparagraph{Assumption 1.}
Let $f: \mathcal{I} \rightarrow \mathcal{M}$ represent a pixelwise mapping from images $\mathcal{I}$ to semantic output $\mathcal{M}$.
Denote $\rho_{\bm{\epsilon}}: \mathcal{I} \rightarrow \mathcal{I}$ a photometric image transform and, similarly, $\tau_{\bm{\epsilon}'}: \mathcal{I} \rightarrow \mathcal{I}$ a spatial similarity transformation, where $\bm{\epsilon},\bm{\epsilon}'\sim p(\cdot)$ are control variables following some pre-defined density (\eg, $p \equiv \mathcal{N}(0, 1)$).
Then, for any image $I \in \mathcal{I}$, $f$ is \emph{invariant} under $\rho_{\bm{\epsilon}}$ and \emph{equivariant} under $\tau_{\bm{\epsilon}'}$, \ie~$f(\rho_{\bm{\epsilon}}(I)) = f(I)$ and $f(\tau_{\bm{\epsilon}'}(I)) = \tau_{\bm{\epsilon}'}(f(I))$.

\smallskip
\noindent Next, we introduce a training framework using a \emph{momentum network} -- a slowly advancing copy of the original model.
The momentum network provides stable, yet recent targets for model updates, as opposed to the fixed supervision in model distillation \cite{Chen0G18,Zheng_2020_IJCV,ZhengY20}.
We also re-visit the problem of long-tail recognition in the context of generating pseudo labels for self-supervision.
In particular, we maintain an \emph{exponentially moving class prior} used to discount the confidence thresholds for those classes with few samples and increase their relative contribution to the training loss.
Our framework is simple to train, adds moderate computational overhead compared to a fully supervised setup, yet sets a new state of the art on established benchmarks (\cf \cref{fig:preview}).

\section{Contributions}
\label{sec:contrib}

In~\cref{sec:sol}, we review some properties of the solution set.
In all this paper, \textbf{we consider the following hypothesis of restricted injectivity}
\begin{equation}\label{eq:hyp-inv}
  \Ker D^* \cap \Ker \Phi = \ens{0} ,
\end{equation}
in order to ensure that $\Xl$ is well-defined and bounded.
We prove in particular that $\Xl$ is a polytope, i.e. a bounded polyhedron.

Our main contribution is proved in~\cref{sec:max}.
It consist in providing a geometrical interpretation of a solution with a maximal $D$-support, namely the fact that such a solution lives in the relative interior of the solution set.
More precisely, we are concerned with the characterization of a vector of maximal $D$-support, i.e. a solution of~\eqref{eq:p} such that for every $x \in \Xl, \normz{D^* x} \leq \normz{D^* \maxx}$.
\begin{definition}
  A vector $\maxx \in \RR^n$ is \emph{a solution of maximal $D$-support} if $\maxx$ is a solution, i.e. $\maxx \in \Xl$ such that for every $x \in \Xl, \normz{D^* x} \leq \normz{D^* \maxx}$.
\end{definition}
We denote by $\Sl$ the set of solution of~\eqref{eq:p} which have maximal $D$-support.
Clearly this set is well-defined and contained in $\Xl$.
Our result is the following.
\begin{theorem}\label{thm:maximal-characterization}
  Let $\bar{x} \in \Xl$. Then $\bar{x}$ is a maximally $D$-supported solution if, and only if, $\bar{x} \in \rint \Xl$ (or equivalently if $\bar{x} \in \rint \Sl$).
  In other words, 
  \begin{equation*}
    \Sl = \rint \Sl = \rint \Xl .
  \end{equation*}
\end{theorem}
We recall that for any set $S$, the relative interior $\rint S$ of $S$ is
defined as its interior with respecto to the topology of the affine hull of $S$.

With this result in hand, we provide a way to construct such maximal solutions.
In \cref{sec:finding}, we show that with the help of a technical penalization using the so-called concave gauge ~\cite{barbara2015strict}, we can construct a path which converges to a point in the relative interior of $\Xl$, and more specifically, to the analytic center with respect to the chosen gauge.
We defer the precise statement to~\cref{sec:finding}.
\section{The Solution Set}
\label{sec:sol}

This section deals reviews some properties of the solution set $\Xl$.
The following proposition shows that even if $\Xl$ is not reduced to a singleton, its image by $\Phi$ or the analysis-$\lun$-norm is single-valued.
\begin{proposition}[Unique image]\label{lem:same-image}
  Let $x^1, x^2 \in \Xl$.
  Then,
  \begin{enumerate}
  \item they share the same image by $\Phi$, i.e., $\Phi x^1 = \Phi x^2$ ;
  \item they have the same analysis-$\lun$-norm, i.e., $\normu{D^* x^1} = \normu{D^* x^2}$.
  \end{enumerate}
\end{proposition}
A proof of this statement can be found for instance in~\cite{vaiter2011robust}.

It is known that standard $\ldeux$-regularization suffers from sign inconsistencies, i.e. two differents solutions can be of opposite signs at some indice.
The following proposition gives another important information: the cosign of two solutions cannot be opposite.
\begin{proposition}[Consistency of the sign]\label{prop:sign}
  Let $x^1, x^2 \in \Xl$.
  Then,
  \begin{equation*}
    \forall i \in \ens{1,\dots,p}, \quad u_i^1 u_i^2 \geq 0 ,
  \end{equation*}
  where $u^k = D^* x^k$ for $k=1,2$.
\end{proposition}
\begin{proof}
  The proof of this statement follows closely the proof found in~\cite{attouch1999p} for $\lun$.
  Suppose there exists $i$ such that $u_i^1$ and $u_i^2$ have opposite signs.
  Then, one has
  \begin{equation}\label{eq:sign-strict-ineq}
    \frac{\abs{u_i^1 + u_i^2}}{2} < \frac{\abs{u_i^1} + \abs{u_i^2}}{2} .
  \end{equation}
  Let $z = u^1 + u^2$.
  Using the convexity of $x \mapsto \norm{y - \Phi x}_2^2$ and inequality~\eqref{eq:sign-strict-ineq}, we get that
  \begin{align*}
    \frac{1}{2} \norm{y - \Phi z}_2^2 + \normu{D^* z} 
    & \!<\! \frac{1}{2} \left( \left( \frac{1}{2} \norm{y - \Phi x^1}_2^2 + \normu{D^* x^1}  \right) \!+\! \left( \frac{1}{2} \norm{y - \Phi x^2}_2^2 + \normu{D^* x^2} \right) \right) \\
    & \!=\! \umin{x \in \RR^n} \frac{1}{2} \norm{y - \Phi x}_2^2 + \normu{D^* x} ,
  \end{align*}
  which is a contradiction.
\end{proof}

Condition~\cref{eq:hyp-inv} (we recall that all through this paper, we suppose this condition holds) ensures that $\Xl$ is a non-empty, convex and compact set.
 Recall for all the following that given a lower semicontinuous real-valued extended convex function $h$ on $\RR^l$, its recession function can be defined by (Theorem 8.5 of \cite{rockafellar})
$$h_\infty(d)=\lim\limits_{\lambda\uparrow+\infty}\displaystyle{h(z+\lambda d)-h(z)\over\lambda},\ \forall (z,d)\in \dom(h)\times\RR^{l}.$$
In fact, as stated by the following proposition, the solution set $\Xl$ is a polytope.
\begin{proposition}\label{prop:polysol}
  $\Xl$ is a polytope (i.e. a bounded polyhedron).
\end{proposition}
\begin{proof}
  Let us first prove that $\Xl$ is a non-empty, convex and compact set. 
It follows with the help of hypothesis~\eqref{eq:hyp-inv} that $\{d:\ h_\infty(d)\leq0\}=\{0\}$.
  Hence, $\Xl$ is bounded.

  We shall now prove that $\Xl$ is a polytope.
  Let $\bar{x} \in \Xl$.
  According to Proposition~\ref{lem:same-image}, we have
  \begin{equation*}
    \Xl \subseteq
    \enscond{x \in \RR^n}{\normu{D^* x} = \normu{D^* \bar{x}}} \cap
    \enscond{x \in \RR^n}{\Phi x = \Phi \bar{x}} .
  \end{equation*}
  The reverse inclusion came from the fact that if $x$ shares the same image by $\Phi$ as $\bar{x}$ and the same analysis-$\lun$-norm, then the objective function at $x$ is equal to the one at $\bar{x}$, hence is also a solution.
  Thus,
  \begin{equation*}
    \Xl =
    \enscond{x \in \RR^n}{\normu{D^* x} = \normu{D^* \bar{x}}} \cap
    \enscond{x \in \RR^n}{\Phi x = \Phi \bar{x}} .
  \end{equation*}
  Hence, $\Xl$ is a polyhedron. Since $\Xl$ is a bounded set, it is also a polytope.
\end{proof}

Owing to Proposition~\ref{prop:polysol}, we can rewrite the set $\Xl$ as the convex hull of $k$ points in $\RR^n$ as
\begin{equation*}
  \Xl = \conv{a_1, \dots, a_k} ,
\end{equation*}
where $a_i$ are the extremal points of $\Xl$.
Observe that each $a_i$ lives on the boundary of the analysis-$\lun$-ball of radius $\normu{D^* \bar{x}}$.
Naturally, we can even rewrite the solution as
\begin{equation*}
  \Xl = A \Delta_k = \enscond{A z}{z \in \Delta_k} ,
\end{equation*}
where $A$ is a matrix $n \times k$ such that its columns are the vectors
$a_i$ and the $n$-simplex $\Delta_n$ of $\RR^n$ is defined as
\begin{equation*}
  \Delta_n = \enscond{x \in \RR^n}{\sum_{i=1}^n x_i = 1 \qandq \forall i, x_i \geq 0} = \conv{e_1,\dots,e_n} ,
\end{equation*}
where $(e_1,\dots,e_n)$ is the canonical basis of $\RR^n$.
Since $a_i$ are the extremal points of $\Xl$, notice that $A$ has maximal rank.
Observe in particular that the lines of the matrix $D^* A$ have same signs according to Proposition~\ref{prop:sign}.
\section{Maximal support and proof of \cref{thm:maximal-characterization}}
\label{sec:max}

We recall that a vector $\maxx \in \RR^n$ is a solution of maximal $D$-support if $\maxx$ is a solution, i.e., $\maxx \in \Xl$ such that for every $x \in \Xl, \normz{D^* x} \leq \normz{D^* \maxx}$.
The following proposition proves that \emph{the} $D$-maximal support is indeed
uniquely defined.
\begin{proposition}
  Let $x \in \Xl$.
  Then the two following propositions are equivalent.
  \begin{enumerate}
  \item $x$ is a solution of maximal $D$-support, i.e. $x \in \Sl$.
  \item For any $\bar{x} \in \Xl$, $\supp(D^* \bar{x}) \subseteq \supp(D^* x)$.
  \end{enumerate}
\end{proposition}
\begin{proof}
  The two directions are proved separately.\\
  $(1) \Rightarrow (2)$.
  Suppose there exists $i_0 \in \ens{1,\dots,p}$ such that $i_0 \in \supp(D^* \bar{x})$ and $i_0 \not\in \supp(D^* x)$.
  Observe that $\tilde x = \frac{1}{2}(\bar{x} + x)$ is also an element of $\Xl$ by convexity of $\Xl$.
  Using Proposition~\ref{prop:sign}, we get that $\supp(D^* \tilde x) \supseteq \supp(D^* \bar{x}) \cup \supp(D^* x)$.
  In particular, $\supp(D^* \tilde x) \supseteq \supp(D^* x) \cup \ens{i_0} \supsetneq \supp(D^* x)$.
  Hence, $\abs{\supp(D^* \tilde x)} > \abs{\supp(D^* x)}$ which contradicts the fact that $x$ has maximal $D$-support.\\
  $(2) \Rightarrow (1)$.
  Taking the cardinal in the property $\forall \bar{x} \in \Xl$, $\supp(D^* \bar{x}) \subseteq \supp(D^* x)$ is sufficient.
\end{proof}
In particular, two solutions of maximal support share the same $D$-support. Notice that in this case, the sign vectors are also the same.

We start by a technical Corollary of Proposition~\ref{prop:sign} which will be convenient in the following.
\begin{corollary}\label{cor:diagpos}
  There exists an integer $m \in \NN$, a matrix $\Lambda = \diag(\lambda_i)_{i=1,\dots,p}$ with $\lambda_i \in \ens{-1,1}$ for $i \in \ens{1,\dots,m}$ and $\lambda_i = 0$ for $i \in \ens{m+1,\dots,p}$, and a permutation matrix $\Sigma$ such that for $\Gamma = \Lambda \Sigma$, one has
  \begin{equation*}
    \Gamma D^* \Xl \subset (\RR_+)^m \times \ens{0}^{p-m} .
  \end{equation*}
  Moreover, for all $x \in \Xl$, $\normu{\Gamma D^* x} = \normu{D^* x}$.
\end{corollary}
\begin{proof}
  Let $\maxx$ an element of $\Sl$.
  Consider $I = \supp(D^* \maxx)$, $J = I^c$ and $m = \abs{I}$.
  Let $\Sigma$ be the permutation matrix associated to any permutation $\sigma$ which sends $I$ to $\ens{1,\dots,m}$.
  Define the matrix $\Lambda$ by its diagonal as
  \begin{equation*}
    \lambda_{\sigma(i)} = 
    \begin{cases}
      1 & \text{if } (D^* \maxx)_{\sigma(i)} > 0 \\
      -1 & \text{if } (D^* \maxx)_{\sigma(i)} < 0 \\
      0 & \text{if } (D^* \maxx)_{\sigma(i)} = 0 .
    \end{cases}
  \end{equation*}

  Now take any solution $x \in \Xl$ and consider the vector $u = \Gamma D^* x$.
  Let $i \in \ens{1,\dots,m}$, then
  \begin{equation*}
    u_i = \dotp{e_i}{\Lambda \Sigma D^* x} .
  \end{equation*}
  Since $\Lambda$ is self-adjoint, one has
  \begin{equation*}
    u_i = \dotp{\Lambda e_i}{\Sigma D^* x} .
  \end{equation*}
  Since $\Lambda$ is a diagonal matrix, we get that
  \begin{equation*}
    u_i = \lambda_i \dotp{e_i}{\Sigma D^* x} .
  \end{equation*}
  Now, since $\Sigma$ is a permutation matrix, we have that $\Sigma^* = \Sigma^{-1}$, i.e.
  \begin{equation*}
    u_i = \lambda_i \dotp{\Sigma^{-1} e_i}{D^* x} .
  \end{equation*}
  Using the permutation $\sigma$ associated to $\Sigma$, we have that
  \begin{equation*}
    u_i = \lambda_i \dotp{e_{\sigma^{-1}(i)}}{D^* x} ,
  \end{equation*}
  which can be rewritten as
  \begin{equation*}
    u_i = \lambda_i \dotp{d_{\sigma^{-1}(i)}}{x} .
  \end{equation*}  
  According to Proposition~\ref{prop:sign}, one have $(D^* x)_{\sigma^{-1}(i)} (D^* \maxx)_{\sigma^{-1}(i)} \geq 0$.
  Moreover, $\lambda_i = \lambda_{\sigma(\sigma^{-1}(i))}$ has the same sign than $(D^* \maxx)_{\sigma^{-1}(i)}$.
  Thus, $u_i = \lambda_i \dotp{d_{\sigma^{-1}(i)}}{x} \geq 0$.

  For $i \in \ens{m+1,\dots,p}$, we have that
  \begin{equation*}
    u_i = \lambda_i \dotp{e_i}{\Sigma D^* x} = 0,
  \end{equation*}
  since $\lambda_i = 0$.
\end{proof}
Note that the matrix $\Lambda$ and $\Sigma$ are not uniquely defined. Corollary~\ref{cor:diagpos} allows us to work only on positive vectors in dimension $m$.

We will also need to exclude at some point the case where a solution $x$ lives in the kernel of $D^*$.
The following lemma shows that if this is the case, then the solution set is reduced to a singleton $\Xl = \ens{x}$.
\begin{lemma}\label{lem:kernel-one-image}
  If there exists $x \in \Ker D^* \cap \Xl$, then $\Xl = \ens{x}$ .
\end{lemma}
\begin{proof}
  We recall that $\Xl \subset x + \Ker \Phi$.
  Let $\bar x \in \Xl$, and rewrite it as $\bar x = x + h$ where $h \in \Ker \Phi$.
  Then, according to Proposition~\ref{lem:same-image}, one has $\normu{D^* \bar x} = \normu{D^* x} = 0$.
  In particular, $\normu{D^* \bar x} = \normu{D^* x + D^* h} =  \normu{D^* h} = 0$.
  Using hypothesis~\eqref{eq:hyp-inv}, we get that $h = 0$.
\end{proof}

We can now provide the proof of Theorem~\ref{thm:maximal-characterization}.
\begin{proof}[Proof of Theorem~\ref{thm:maximal-characterization}]
  We exclude here the case where $\Xl$ is reduced to a singleton, since the result is then trivially verified.
  Let us prove both direction separately.

  $(\Leftarrow: \rint \Xl \subseteq \Sl)$.
  First, we recall that $\rint \Xl = \rint (A \Delta_k) = A \rint \Delta_k$.
  Let $\bar{x} \in \rint \Xl$.
  We have
  \begin{equation*}
    \bar{x} = A \bar{z} \qwithq \sum_{i=1}^k \bar{z}_i = 1 \qandq \bar{z}_i > 0.
  \end{equation*}
  For $i \in \ens{1,\dots,m}$, one has
  \begin{equation*}
    (\Gamma D^* \bar{x})_i = (\Gamma D^* A \bar{z})_i = \dotp{e_i}{\Gamma D^* A \bar{z}} =  \dotp{e_i}{\Lambda \Sigma D^* A \bar{z}}.
  \end{equation*}
  Using the fact that $\Lambda$ is a diagonal matrix and $\Sigma$ is a permutation matrix, we have that
  \begin{equation*}
    (\Gamma D^* \bar{x})_i = \lambda_i \dotp{D \Sigma^{-1} e_i}{A \bar{z}} ,
  \end{equation*}
  which can be rewritten, using the fact that $\Sigma^{-1} e_i = e_{\sigma^{-1}(i)}$ where $\sigma$ is the permutation associated to $\Sigma$, as
  \begin{equation*}
    (\Gamma D^* \bar{x})_i = \lambda_i \dotp{d_{\sigma^{-1}(i)}}{A \bar{z}} .
  \end{equation*}
  Now, one can rewrite it as
  \begin{equation*}
    (\Gamma D^* \bar{x})_i = \lambda_i \dotp{(D^* A)^* e_{\sigma^{-1}(i)}}{\bar{z}} .
  \end{equation*}
  Since for any $i$, $\bar{z}_i > 0$ and, according to Proposition~\ref{prop:sign}, there exists $j_0$ such that $((D^* A)^* e_{\sigma^{-1}(i)})_{j_0} > 0$, one concludes that $(\Gamma D^* \bar{x})_i \neq 0$.

  $(\Rightarrow: \Sl \subseteq \rint \Xl)$.
  We are going to prove that $\Sl = \rint \Sl$.
  Indeed, according to $(\Leftarrow)$, $\rint \Xl \subseteq \Sl$.
  Moreover, since every element of $\Sl$ is also an element of $\Xl$, we have $\rint \Xl \subseteq \Sl \subseteq \Xl$.
  In particular, $\aff \Xl = \aff \Sl$.
  Let
  \begin{equation*}
    \alpha = \min_{i \in \supp(D^* \maxx)} \abs{(D^* \maxx)_i} = \min_{i \in \ens{1,\dots,m}} (\Gamma D^* \maxx)_i
  \end{equation*}
  where $\maxx$ is an element of $\Sl$.
  Note that according to Lemma~\ref{lem:kernel-one-image}, since $\Xl$ is not reduced to a singleton, then $\supp(D^* \maxx)$ has cardinal greater than 1, hence $\alpha > 0$.

  Now take any $u \in B_\infty(\maxx, r) \cap \aff \Xl$ where
  \begin{equation*}
    r = \frac{\alpha - \epsilon}{\norm{\Gamma D^*}_{\infty,\infty}},
  \end{equation*}
  and $0 < \epsilon < \alpha$.

  Let's prove first that $\Gamma D^* u \in (\RR_+^*)^m \times \ens{0}^{p-m}$.
  From the definition of $u$, we get that
  \begin{equation*}
    \normi{\Gamma D^* u - \Gamma D^* x} \leq \norm{\Gamma D^*}_{\infty,\infty} \normi{u - x} \leq \alpha - \epsilon .
  \end{equation*}
  For $i \in \ens{1,\dots,m}$, one has $\abs{(\Gamma D^* u)_i - (\Gamma D^* x)_i} \leq \alpha - \epsilon$. In particular one has
  \begin{equation*}
    (\Gamma D^* u)_i - (\Gamma D^* x)_i \geq -\alpha + \epsilon \Leftrightarrow (\Gamma D^* u)_i \geq (\Gamma D^* x)_i - \alpha + \epsilon .
  \end{equation*}
  Since $(\Gamma D^* x)_i - \alpha \geq 0$ and $\epsilon > 0$, we conclude that $(\Gamma D^* u)_i > 0$.
  Thus, $(\Gamma D^* u)_i > 0$ for $i \in \ens{1,\dots,m}$ and $(\Gamma D^* u)_i = 0$ for $i \not\in \ens{1,\dots,m}$.

  It remains to prove that $u$ is a solution of~\eqref{eq:p}, i.e. $u \in \Xl$.
  Since $u \in \aff \Xl$, there exists $t \in \RR$ and $x \in \Xl$ such that
  \begin{equation*}
    u = \maxx + t (x - \maxx) .
  \end{equation*}
  From this equality, we get that
  \begin{align*}
    \normu{D^* u} &= \normu{\Gamma D^* u} = \sum_{i=1}^p (\Gamma D^* u)_i && \text{according to Corollary~\ref{cor:diagpos}}  \\
                  &= \sum_{i=1}^p (1-t) (\Gamma D^* \maxx)_i + t (\Gamma D^* x)_i && \\
                  &= (1-t) \normu{\Gamma D^* \maxx} + t \normu{\Gamma D^* x} && \\
                  &= \normu{D^* \maxx} && \text{since } \normu{D^* \maxx} = \normu{D^* x} .
  \end{align*}
  Moreover, $\Phi u = \Phi \maxx + t(\Phi x - \Phi \maxx) = \Phi \maxx$.
  Thus, $u$ is a solution which concludes our proof.
\end{proof}

\section{Finding a Maximal Solution}
\label{sec:finding}

Using the classical barrier function, in this section we show how to get a path that converges to a relative interior point of $\Xl$, which turns out to be the analytic center of $\Xl$.

Setting $Q = \Phi^* \Phi$ is the Gram matrix and $c = \Phi^* y$, we start by rewriting our initial problem~\cref{eq:p} as an augmented quadratic program under constraints, i.e.
\begin{equation*}
  \umin{x \in \RR^n, t \in \RR^p}
  \frac{1}{2} \dotp{Qx}{x} - \dotp{c}{x} + \lambda \sum_{i=1}^p t_i
  \qsubjq
  \begin{cases}
    -t \leq D^* x \leq t &\\
    t_i \geq 0 &\\
  \end{cases} ,
\end{equation*}
witch also can be rewritten as
\begin{equation*}
  \umin{x \in \RR^n, t \in \RR^p}
  \frac{1}{2} \dotp{Qx}{x} - \dotp{c}{x} + \lambda \sum_{i=1}^p t_i
  \qsubjq
  \begin{cases}
  -t+s=D^*x&\\
  t-s'=D^*x&\\
    t_i \geq 0,\ s_i\geq0,\ s'_i\geq0 &\\
  \end{cases}.
\end{equation*}
Now observe that $t=\displaystyle{1\over 2}(s+s')$. Then setting $z=\displaystyle{1\over 2}\left(\begin{array}{l}s\cr s'\end{array}\right)$, $I_p$ the $p$ by $p$ identity matrix, ${\tilde I}=\left(\begin{array}{lr}I_p& -I_p\end{array}\right)$ and $e=(1, \cdots,1)\in\RR^{2p}$, we come to the following equivalent formulation of the problem
\begin{equation}
\label{eq:paug}
\umin{x\in\RR^n, z\in\RR^{2p}}
f(x,z)
\qsubjq z\in[0,+\infty)^{2p}
\end{equation}
where $$
f(x,z)=\left\{\begin{array}{ll}\displaystyle{1\over2}\dotp{Qx}{x} - \dotp{c}{x} +\lambda\dotp{e}{z}&\mbox{ if }D^*x+{\tilde I}z=0\\
+\infty&\mbox{ elsewhere,}\end{array}\right.$$ 
or equivalently 
$$f(x,z)=\left\{\begin{array}{ll}\displaystyle{1\over2}\|\Phi x-y\|^2-\displaystyle{1\over2}\|y\|^2 +\lambda\dotp{e}{z}&\mbox{ if }D^*x+{\tilde I}z=0\\
+\infty&\mbox{ elsewhere.}\end{array}\right.$$ 


Its classical dual is
\begin{equation}
\label{eq:daug}
\umax{x\in\RR^n, s\in\RR^{2p}, u\in\RR^p}
g(x,s,u)
\qsubjq s\in[0,+\infty)^{2p}
\end{equation}
where 
$$g(x,s,u)=\left\{\begin{array}{ll}
-\displaystyle{1\over 2}\langle Qx,x\rangle&\mbox{if }{D} u+c-Qx=0,\ s=\lambda e-{\tilde I}^*u\cr 
-\infty&\mbox{elsewhere.}\end{array}\right.$$ 
We set $S_{(P)}$ (resp. $S_{(D)}$) the optimal solutions' set of
problem~\cref{eq:paug} (resp. problem~\cref{eq:daug}).
We know that $\Xl$ is non-empty and so $S_{(P)}$.
Since, in addition~\cref{eq:paug} is a convex problem with polyedral constraints, $S_{(D)}$ is non empty and there is no duality gap. We denote by $\alpha$ the optimal value of the two problems. 

\begin{proposition}\label{dcompacity}{$ $}

\begin{itemize}
\item[1.]The optimal solution $S_{(P)}$ of the problem (\ref{eq:paug}) is bounded or equivalently the set $\{(d_x,d_z):\ f_\infty(d_x,d_z)\leq0,\ d_z\geq0\}=\{0\}$,
\item[2.]$S(.,(D))=\{(s,u):\ \exists x\in\RR^n\mbox{ such that }(x,s,u)\in S_{(D)}\}$ is bounded, in other words, the dual feasible solutions' set is bounded in $(s,u)$.
\end{itemize}
\end{proposition}
\begin{proof}
1. Because of relation (\ref{eq:hyp-inv}) it is not difficult to show that the optimal solution $S_{(P)}$ of the problem (\ref{eq:paug}) is bounded.


2. Let $(x^k,s^k,u^k)$ be a sequence of the dual feasible solutions' set. We have $s^k=\lambda e-{\tilde I}^*u=\left(\begin{array}{l}\lambda e^p\cr\lambda e^p\end{array}\right)-\left(\begin{array}{l}u^k\cr-u^k\end{array}\right)\geq0$, where $e^p=(1,\cdots 1)\in\RR^p$. It follows that $-\lambda e^p\leq u^k\leq \lambda e^p$. Hence $(u^k)$ and then $(s^k)$, is bounded.
\end{proof}




Using the classical logarithmic barrier function introduced by Frish~\cite{frisch}, we deal with the family of problems $(P_\mu)_{\mu>0}$ given by
\begin{equation*}
\theta(\mu)=\umin{x\in\RR^n, z\in\RR^{2p}}
F_{\mu}(x,z)=f(x,z)+\zeta(z,\mu)
\end{equation*}
where $$\zeta(z,\mu)=\left\{\begin{array}{ll}
\mu \xi\left(z/\mu\right)&\mbox{if }\mu>0,\cr\xi_\infty(z)&\mbox{if }\mu=0,\cr+\infty&\mbox{elsewhere,}
\end{array}\right.$$ $$ \xi(z)=\left\{\begin{array}{ll}-\ln \varphi(z)&\mbox{if }\varphi(z)>0,\cr+\infty&\mbox{elsewhere,}\end{array}\right. \mbox{ and }\varphi(z)=\left\{\begin{array}{ll}\left(\prod\limits_{i=1}^{2p}z_i\right)^{1\over2p}&\mbox{if }z\geq0,\cr-\infty&\mbox{elsewhere.}\end{array}\right.$$


Note that the function $\varphi$ is strictly quasiconcave and then according to Lemma 1 of \cite{barbara2015strict}, for every $\mu>0$, the function $\zeta_\mu:z\mapsto\zeta(z,\mu)$ is strictly convex on $(0,+\infty)^{2p}$. 
\begin{proposition}\label{strict_convexity}
For every $\mu>0$, the function $F_{\mu}$ is inf-compact on $\RR^n\times\RR^{2p}$ and strictly convex on $\RR^n\times(0,+\infty)^{2p}$.
\end{proposition}
\begin{proof} Let us show that  
\begin{eqnarray}\label{xiinfty}
\xi_\infty(d)=\left\{\begin{array}{ll}0&\mbox{if }d\geq0,\cr+\infty&\mbox{elsewhere.}\end{array}
\right.\end{eqnarray}
Let $(z,d)\in \dom(\xi)\times\RR^{2p}$. We have necessarily $z>0$. First we observe that when $d\not\in[0,+\infty)^{2p}$, $z+\lambda d\not\in[0,+\infty)^{2p}$ for $\lambda$ large enough and then $\xi_\infty(d)=+\infty$. Now consider the case $d\geq0$. Since $z>0$ we have necessarily $z+d>0$. The concave gauge function $\varphi$ is monotone with respect to its domaine the positive orthant. Then by Proposition 2.1 of \cite{barbara_crouzeix},
$$0<\varphi(z+d)\leq\varphi(z+\lambda d)\leq\varphi(\lambda z+\lambda d)=\lambda\varphi(z+d)$$
for $\lambda$ large enough. It follows that
$$\begin{array}{ll}
0=\lim\limits_{\lambda\uparrow+\infty}\displaystyle{\ln\varphi(z+d)-\ln\varphi(z)\over\lambda}&\leq\lim\limits_{\lambda\uparrow+ \infty}\displaystyle{\ln\varphi(z+\lambda d)-\ln\varphi(z)\over\lambda}\cr&\leq\lim\limits_{\lambda\uparrow+\infty} \displaystyle{\ln\lambda\varphi(z+d)-\ln\varphi(z)\over\lambda}=0\end{array}$$ and hence $\lim\limits_{\lambda\uparrow+ \infty}\displaystyle{\ln\varphi(z+\lambda d)-\ln\varphi(z)\over\lambda}=0$. Consequently $\xi_\infty(d)=0$.

By Proposition \ref{dcompacity}, we have $\{(d_x,d_z):\ f_\infty(d_x,d_z)\leq0,\ d_z\geq0\}=\{(0,0)\}$. Thus
$\{(d_x,d_z):\ {F_\mu}_\infty(d_x,d_z)\leq0,\ d_z\geq0\}=\{(0,0)\}$, or equivalently, $F_\mu$ is inf-compact. 

Now let us proceed to prove the strict convexity of $F_\mu$.  Take  $(x,z)\not=(x',z')$ in $\RR^n\times(0,+\infty)^{2p}$ and $t\in(0,1)$. In the case where $z\not= z'$, by strict-convexity of $\zeta_\mu$ on $(0,+\infty)^{2p}$ we have necessarily $F_\mu(t(x,z)+(1-t)(x',z'))<tF_\mu(x,z)+(1-t)F_\mu(x',z').$ Assume that $z=z'$. Using (\ref{eq:hyp-inv}) and the definition of $f$ we obtain $\Phi x\not=\Phi x'$ and the result follows by using the strict convexity of $\|.\|_2^2$.
\end{proof}
Propositions \ref{strict_convexity} and \ref{dcompacity} assert that for every $\mu>0$ there is a unique optimal solution  $(x(\mu),z(\mu))$ to $(P_\mu)$. Moreover using the fact that $F_\mu(x,\cdot)$ is a barrier function for every $x\in\RR^n$, $z(\mu)>0$. Consider the function $\gamma:\RR^n\times[0,+\infty)^{2p}\times[0,+\infty)\to \RR\cup\{+\infty\}$ defined by
$$\gamma(x,z,\mu)=F_{\mu}(x,z).$$
Then we have the following proposition.

\begin{proposition}
\label{coercivity}
The function $\gamma$ is convex and lsc on $\RR^n\times\RR^{2p}\times[0,+\infty)$. It is inf-compact on $\RR^n\times\RR^{2p}\times[0,{\overline \mu}]$, $\forall\overline{\mu}>0$ being fixed. Moreover $\theta$ is convex and continuous on $[0,+\infty)$, $\theta(0)=\alpha$ and $f(x,z)=\gamma(x,z,0)$, $\forall (x,z)\in\RR^n\times(0,+\infty)^{2p}$.
\end{proposition}
\begin{proof}
It is known that the function $\zeta$ is convex on $\RR^{2p}\times[0,+\infty)$
and so is $\gamma$. The function $\theta$ is then convex on $[0,+\infty)$ as the
infimum over $(x,z)$ of a convex function in $(x,z,\mu)$. Now the function
$\zeta(z,.)$ is continuous on $[0,+\infty)$ and, because of (\ref{xiinfty}),
$\zeta(z,0)=0$ for all $z\in(0,+\infty)^{2p}$. Thus $f(x,z)=\gamma(x,z,0)$ for
all $(x,z)\in\RR^n\times(0,+\infty)^{2p}$ and therefore $\theta(0)=\alpha$ (the
optimal value of the problem (\ref{eq:paug})). Set
$\tilde{\gamma}=\gamma_{|\RR^n\times\RR^{2p}\times[0,\overline{\mu}]}$ the
restriction of $\gamma$ to the set
$\RR^n\times\RR^{2p}\times[0,\overline{\mu}]$. Then $\{(d_x,d_z,\mu):\
\tilde{\gamma}_\infty(d_x,d_z,\mu)\leq0,\ d_z\geq0,\ \mu=0\}=\{(d_x,d_z,0):\
f_\infty(d_x,d_z)\leq0,\ d_z\geq0\}=\{(0,0,0)\}$ (see Proposition
\ref{dcompacity}). The function $\gamma$ is then inf-compact on
$\RR^n\times\RR^{2p}\times[0,{\overline \mu}]$. Consequently, there is a compact
$\tilde{S}$ such that $(x(\mu),z(\mu))\in\tilde{S}$,
$\forall\mu\in(0,\overline{\mu}]$, i.e.,
$(x(\mu),z(\mu))_{\mu\in(0,\overline{\mu})}$ is bounded. We established that $\theta$ is convex on $[0,+\infty)$. It is then continuous on $(0,+\infty)$. Let us show now that $\lim\limits_{\mu\downarrow 0}\theta(\mu)=\theta(0)=\alpha$. In this respect we shall prove that 
$\lim\limits_{\mu\downarrow0}\mu\ln\left(\displaystyle{\varphi(z(\mu)) \over\mu}\right)= 0$. Let $(\mu^k)_{k\in\NN}$ be a positive sequence such that $\lim\limits_{k\uparrow+\infty}\mu^k=0.$ We established that $(x(\mu),z(\mu))_{\mu\in(0,\overline{\mu}]}$ is bounded. It follows that the set $\{(x(\mu^k),z(\mu^k))\}$ contains a subsequence converging to a point  $(\tilde{x},\tilde{z})$.
In the case where $\tilde{z}>0$ the result is obvious. Assume that $\varphi(\tilde{z})=0$. Then for $k$ sufficiently large one has
$$\begin{array}{ll}\alpha-\mu^k\ln\left(\displaystyle{\varphi(z)\over\mu^k}\right)
\leq \theta(\mu^k)&=f(x(\mu^k),z(\mu^k))-\mu^k\ln\left(\displaystyle{\varphi(z(\mu^k)) \over \mu^k}\right)\cr&
\leq f(x,z)-\mu^k\ln\left(\displaystyle{\varphi(z)\over\mu^k}\right)
\end{array}$$
for every $(x,z)$ satisfying $z>0$. Since $\lim\limits_{k\uparrow 0}\mu^k\ln\left(\displaystyle{\varphi(z)\over\mu^k}\right)=0$, we have
$$\alpha\leq\lim\inf\limits_{k\uparrow+\infty}\theta(\mu^k)\leq f(x,z)$$
and then
$$\alpha\leq\lim\sup\limits_{k\uparrow+\infty}\theta(\mu^k)\leq \inf\limits_{x,z}\{f(x,z):\ z>0\}=\inf\limits_{x,z}\{f(x,z):\ z\geq0\}=\alpha.$$
Consequently $\lim\limits_{k\uparrow+\infty}\theta(\mu^k)=\alpha$.
\end{proof}
\bigskip
\bigskip


Given $\mu>0$, the KKT optimalty conditions for the problem $(P_\mu)$ can be formulated, for some $u\in\RR^p$, as
$$\left\{\begin{array}{ll}Qx(\mu)-c-Du=0,\\ \lambda e-\displaystyle{ \mu\over 2p}(Z(\mu))^{-1}e-{\tilde I}^*u=0,\\ D^*x(\mu)+{\tilde I}z(\mu)=0,\end{array}\right.$$
where $Z(\mu)=diag(z(\mu))$.
Observe that $u$ is necessarily unique. Put 
$$u=u(\mu)\mbox{ and }s(\mu)=\displaystyle {\mu\over 2p}Z^{-1}(\mu)e.$$ We rewrite the KKT conditions as
$$\left\{\begin{array}{lr}Qx(\mu)-c-Du(\mu)=0&(E1)\\ \lambda e-s(\mu)-{\tilde I}^*u(\mu)=0&(E2)\\ Z(\mu)s(\mu)=\displaystyle {\mu\over 2p}e&(E3)\\D^*x(\mu)+{\tilde I}z(\mu)=0&(E4)\end{array}\right.$$

\begin{proposition}
For every $\mu>0$, $(s(\mu),u(\mu))$ is a feasible solution to (\ref{eq:daug}) and $\big((s(\mu),u(\mu)\big)_{\mu\in(0,\overline{\mu}]}$ is bounded.
\end{proposition}

\begin{proof}
By $(E1)$, $(E2)$ and the fact that $s(\mu)=\displaystyle {\mu\over 2p}(Z(\mu))^{-1}e>0$, $(u(\mu),s(\mu))$ is a feasible solution to (\ref{eq:daug}). The boundedness of $(s(\mu),u(\mu))_{\mu\in(0,{\overline\mu}]}$ is due to Proposition \ref{dcompacity}.
\end{proof} 


Set ${\overline I}=\displaystyle\bigcup_{\atop z\in S(.,(P))}I(z)$ and ${\overline J}=\displaystyle\bigcup_{\atop s\in S(.,(D))}J(s)$, where 
$$S(.,(P))=\left\{z:\ \exists x\in\RR^n\mbox{ such that } (x,z)\in S_{(P)}\right\},$$ 
$$S(.,(D))=\left\{s:\ \exists u\in\RR^p\mbox{ such that } (s,u)\in S_{(D)}\right\},$$ $$I(z)=\{i:\ z_i>0\}\mbox{ the support of }z\mbox{ and }J(s)=\{i:\ s_i>0\}\mbox{ the support of }s.$$  

\begin{lemma}\label{complementarity}

There is at least one $({\hat z},\hat{s})\in S(.,(P))\times S(.,(D))$ such that ${\overline I}=I({\hat z})$ and $\overline{J}=J(\hat{s})$.
\end{lemma}

\begin{proof}
We have ${\overline I}$ a subset of a finite set $\{1,\cdots,2p\}$. Let then $(z^1,\ z^2,\cdots,z^k)\in S(.,(P))^k$, for some $k\in\{1,2,\cdots,2p\}$
satisfying ${\overline I}=I\left(z^1\right)\cup I\left(z^2\right)\cup\cdots\cup I\left(z^k\right)$. Set ${\hat z}=\displaystyle{1\over k}\left(z^1+z^2+\cdots+z^k\right)$. Since $S(.,(P))$ is convex  ${\hat z}\in S(.,(P))$. So it is easy to see that $I(z^i)\subset I({\hat z})$, $\forall i\in\{1,2,\cdots,k\}$. The result then follows. A vector $\hat{s}$ is constructed in a similar way.
\end{proof} 
Observe that every optimal solution $(x,z)$ of the problem (\ref{eq:paug}) satisfying $I(z)=\overline{I}$ is in the relative interior of $S_{(P)}$. Similarily every optimal solution $(x,s,u)$ of the problem (\ref{eq:daug}) satisfying $J(s)=\overline{J}$ is in the relative interior of $S_{(D)}$.

\bigskip

Set 
$$({\overline x},{\overline z})=\arg\max\left\{\varphi_{\overline I}(z_{\overline I}):\ \displaystyle{1\over 2}\langle Qx,x\rangle-\langle c,x\rangle+\lambda\langle e,z\rangle=\alpha,\ D^*x+{\tilde I}z=0,\ z_{\overline{J}}=0\right\},$$
where 
$$\varphi_{\overline I}(z_{\overline I})=\left\{\begin{array}{ll}
\left(\prod\limits_{i\in {\overline I}}z_i\right)^{1\over \card({\overline I})}&\mbox{if }z_J\in(0,+\infty)^{\card(J)}\cr
-\infty&\mbox{elsewhere.}
\end{array}\right.
$$
Symmetrically we set
$$({\overline s},{\overline u})=\arg\max\left\{\varphi_{\overline J}(s_{\overline J}):\ 
s=\lambda e-\tilde{I}^*u,\ D u+c-Q{\overline x}=0, s_{\overline I}=0\right\},$$
where
$$\varphi_{\overline{J}}(s_{\overline{J}})=\left\{\begin{array}{ll}\left(\prod\limits_{i\in \overline{J}}s_i\right)^{1\over \card(\overline{J})}&\mbox{if }s_{\overline{J}}\in(0,+\infty)^{\card(\overline{J})}\cr-\infty&\mbox{elsewhere.}\end{array}\right.
$$
$(\overline{x},\overline{z})$ is called the analytic center\footnotemark[1]\footnotetext[1]{A generalization of the central path and the analytic center is proposed in \cite{barbara2015strict} by using the so called concave gauge functions.} of (\ref{eq:paug}) and $(\overline{x},\overline{s},\overline{u})$ the analytic center of (\ref{eq:daug}). The uniqueness is ensured by the strict quasiconcavity of functions $\varphi_{\overline{I}}$ and  $\varphi_{\overline{J}}$ on the interior of their respective domain and the assumption (\ref{eq:hyp-inv}). We now give an important result. 



Its proof is inspired in part by those of Theorems I.7 and I.9 in \cite{RoTevi}.
\begin{theorem}\label{thm:convergence}
Under assumption \ref{eq:hyp-inv}, we have $$\lim\limits_{\mu\downarrow0}(x(\mu),z(\mu),s(\mu),u(\mu))=({\overline x},{\overline z},{\overline s},{\overline u}).$$ Moreover, $({\overline x},{\overline z})$  and $({\overline x},{\overline s},{\overline u})$ belong to the relative interior of $S_{(P)}$ and $S_{(D)}$, respectively. 
\end{theorem}

\begin{proof}
We proved that $\big((x(\mu),z(\mu)\big)_{\mu\in(0,\overline{\mu}]}$ and $\big((s(\mu),u(\mu)\big)_{\mu\in(0,\overline{\mu}]}$ are bounded. Let $(\mu^k)_{k\in \NN}$ a positive increasing sequence satisfying $$\lim\limits_{k\uparrow +\infty}\mu^k=0\mbox{ and }\lim\limits_{k\uparrow+\infty}(x(\mu^k),z(\mu^k),s(\mu^k),u(\mu^k))=(\tilde{x},\tilde{z},\tilde{s},\tilde{u}).$$ Then replacing $\mu$ by $\mu^k$ in $(E1)-(E4)$ and letting $k$ tend to $+\infty$, we observe that the pair $\{(\tilde{x},\tilde{z}),(\tilde{x},\tilde{s},\tilde{u})\}$ satisfies the KKT optimality conditions of (\ref{eq:paug}) and then it is a primal-dual optimal solution pair of (\ref{eq:paug}). Let us show now that $I(\tilde{z})=\overline{I}$ and $J(\tilde{s})=\overline{J}$. 
Now by $(E1)$, $(E2)$ and $(E4)$ we have  
$$\left(\begin{array}{l}
x(\mu^k)-\overline{x}\cr
z(\mu^k)-\overline{z}\end{array}\right)\in\Ker{\left(\begin{array}{ll}D^*&{\tilde I}\end{array}\right)}\mbox{ and }
\left(\begin{array}{l}
Q(x(\mu^k)-\overline{x})\cr
-(s(\mu^k)-\overline{s})\end{array}\right)\in\Im{\left(\begin{array}{l}D \\ {\tilde I}^*\end{array}\right)}.
$$
Then using the following orthogonality property 
\begin{equation}
\label{orthogonality}
\Ker{\left(\begin{array}{ll}{D^*}&{\tilde I}\end{array}\right)}=\left[\Im{\left(\begin{array}{l}D \\ {\tilde I}^*\end{array}\right)}\right]^\bot,
\end{equation} 
$(E3)$ and the fact that $\langle\overline{z},\overline{s}\rangle=\langle\tilde{z},\tilde{s}\rangle=0$ we have 
$$\langle\overline{z},s(\mu^k)\rangle+\langle\overline{s},z(\mu^k)\rangle=\mu^k-\langle Q(x(\mu^k)-\overline{x}),x(\mu^k)-\overline{x}\rangle.$$ 
Since in addition $I(\overline{z})=\overline{I}$, $J(\overline{s})=\overline{J}$ and $Q$ is positive semi-definite we get
$$\sum\limits_{i\in \overline{I}}\overline{z}_is(\mu^k)_i+\sum\limits_{i\in \overline{J}}\overline{s}_iz(\mu^k)_i=\mu^k-\langle Q(x(\mu^k)-\tilde{x}),x(\mu^k)-\tilde{x}\rangle\leq\mu^k.$$
But from $(E3)$, $z(\mu^k)_is(\mu^k)_i=\displaystyle{\mu^k\over 2p},\ \forall i$. it follows that
$$\displaystyle\sum\limits_{i\in \overline{J}}{\overline{s}_i\over s(\mu^k)_i}+\sum\limits_{i\in \overline{I}}{\overline{z}_i\over z(\mu^k)_i}\leq 2p.$$
Now letting $k$ tend to $+\infty$, we get on the one hand
$$0<\displaystyle\sum\limits_{i\in \overline{J}}{\overline{s}_i\over \tilde{s}_i}+\sum\limits_{i\in \overline{I}}{\overline{z}_i\over \tilde{z}_i}\leq 2p<+\infty$$
and then, by construction of $\overline{I}$ and $\overline{J}$, we have necessarily $I(\tilde{z})=\overline{I}$ and $J(\tilde{s})=\overline{J}$. On the other hand,
using the arithmetic-geometric mean inequality we get
$$\left(\prod\limits_{i\in \overline{J}}\displaystyle{\overline{s}\over \tilde{s}_i}\prod\limits_{i\in \overline{I}}\displaystyle{\overline{z}\over \tilde{z}_i}\right)^{1\over 2p}\leq{1\over 2p}\left(\displaystyle\sum\limits_{i\in \overline{J}}{\overline{s}\over \tilde{s}_i}+\sum\limits_{i\in \overline{I}}{\overline{z}\over \tilde{z}_i}\right)\leq 1$$
and then 
$$\varphi_{\overline{J}}( \overline{s}_{\overline{J}})\varphi_{\overline{I}}( \overline{z}_{\overline{I}})\leq\varphi_{\overline{J}}( \tilde{s}_{\overline{J}})\varphi_{\overline{I}}( \tilde{z}_{\overline{I}}).$$
But, by definition of $(\overline{x},\overline{z},\overline{s},\overline{u})$, $\varphi_{\overline{J}}( \tilde{s}_{\overline{J}})\leq \varphi_{\overline{J}}( \overline{s}_{\overline{J}})$ and $\varphi_{\overline{I}}( \tilde{z}_{\overline{I}})\leq \varphi_{\overline{I}}( \overline{z}_{\overline{I}})$.  The result then follows. 
\end{proof}

Consequently, the following corollary holds
\begin{corollary}
  Under assumption (\ref{eq:hyp-inv}), we have $\lim\limits_{\mu \downarrow 0} x(\mu) = \bar x \in \rint \Xl$.
\end{corollary}

\begin{proof}
By  Theorem \ref{thm:convergence} $(\overline{x},\overline{z})$ belongs to  the relative interior of $S_{(P)}$ and hence $\overline{x}$ belongs to the linear projection of the relative interior of $S_{(P)}$ which is equal to $\rint \Xl$.
\end{proof}
\section{DROP: Workload Optimization}
\label{sec:algo}

In this section, we introduce DROP, a system that performs workload-aware DR via progressive sampling and online progress estimation.
DROP takes as input a target dataset, metric to preserve (default, target $TLB$), and an optional downstream runtime model.
DROP then uses sample-based PCA to identify and return a low-dimensional representation of the input that preserves the specified property while minimizing estimated workload runtime (Figure 2, Alg.~\ref{alg:DROP}).

%DROP answers a crucial question that stochastic PCA techniques have traditionally ignored: how long should these methods run? 

\begin{comment}
Notation used is in Table~\ref{table:inputs}.

\begin{table}
\centering
\small
\caption{\label{table:inputs} 
 DROP algorithm notation and defaults}
{\renewcommand{\arraystretch}{1.2}
\begin{tabular}{|c|l l|}
\hline 
Symbol & Description (\emph{Default}) & Type\tabularnewline
\hline
$X$  & Input dataset                          & $\mathbb{R}^{\mvar \times \dvar}$ \tabularnewline
$\mvar$  & Number of input data points            & $\mathbb{Z}_{+}$\tabularnewline
$\dvar$  & Input data dimension                   & $\mathbb{Z}_{+}$ \tabularnewline
$B$  & Target $TLB$ preservation      		 & $0 < \mathbb{R} \leq 1 $ \tabularnewline
$\mathcal{C}_\mvar(\dvar)$  & Downstream runtime function (\textit{k-NN runtime})       & $\mathbb{Z}_{+} \to \mathbb{R}_{+}$\tabularnewline
$R$  & Total DROP runtime       & $\mathbb{R}_{+}$ \tabularnewline
$c$ & Confidence level for $TLB$ preservation (\textit{$95 \%$})          & $\mathbb{R}$  \tabularnewline
$T_k$  & DROP output $k$-dimensional transformation &$\mathbb{R}^{\dvar \times k}$ \tabularnewline
$i $ & Current DROP iteration        & $\mathbb{Z}_+$  \tabularnewline

\hline 
\end{tabular}
}
\end{table}
\end{comment}

\begin{figure}
\includegraphics[width=\linewidth]{figs/progressive.pdf}
\caption[]{ Reduction in dimensionality for  $TLB = 0.80$ with progressive sampling. Dimensionality decreases until reaching a state equivalent to running PCA over the full dataset ("convergence").}
\label{fig:progressive}
\end{figure}

\subsection{DROP Algorithm}
\label{subsec:arch}
%DROP is a system that performs workload-aware dimensionality reduction, optimizing the combined runtime of downstream tasks and DR as defined in Problem~\ref{def:opt}.
DROP operates over a series of data samples, and determines when to terminate via \red{a} four-step procedure at each iteration: %progressive sampling, transformation evaluation, progress estimation, and cost-based optimization:

%To power this pipeline, DROP combines database and machine learning techniques spanning online aggregation (\S\ref{subsec:teval}), progress estimation (\S\ref{subsec:pest}), progressive sampling (\S\ref{subsec:psample}), and PCA approximation (\S\S\ref{subsec:pcaroutine},\ref{subsec:reuse}).

%We now provide a brief overview of DROP's sample-based iterative architecture before detailing each.

\begin{comment}
\item Progressive Sampling (\S\ref{subsec:psample}): DROP draws a data sample, performs PCA over it, and uses of a novel reuse mechanism across iterations (\S\ref{subsec:reuse}).

\item Transform Evaluation (\S\ref{subsec:teval}): DROP evaluates the above by identifying the size of the smallest metric-preserving transformation that can be extracted. 

\item Progress Estimation (\S\ref{subsec:pest}): Given the size of the smallest metric-preserving transform and the time required to obtain this transform, DROP estimates the size and computation time of continued iteration.

\item Cost-Based Optimization (\S\ref{subsec:opt}): DROP optimizes over DR and downstream task runtime to determine if it should terminate.
\end{comment}

\minihead{Step 1: Progressive Sampling (\S\ref{subsec:psample})}

\noindent DROP draws a data sample, performs PCA over it, and uses a novel reuse mechanism across iterations (\S\ref{subsec:reuse}).

\minihead{Step 2: Transform Evaluation (\S\ref{subsec:teval})} 

\noindent DROP evaluates the above by identifying the size of the smallest metric-preserving transformation that can be extracted. 

\minihead{Step 3: Progress Estimation (\S\ref{subsec:pest})} 

\noindent Given the size of the smallest metric-preserving transform and the time required to obtain this transform, DROP estimates the size and computation time of continued iteration.

\minihead{Step 4: Cost-Based Optimization (\S\ref{subsec:opt})} 

\noindent DROP optimizes over DR and downstream task runtime to determine if it should terminate.

\subsection{Progressive Sampling}
\label{subsec:psample}

Inspired by stochastic PCA methods (\S\ref{sec:relatedwork}), DROP uses sampling to tackle workload-aware DR. 
Many real-world \red{datasets} are intrinsically low-dimensional; a small data sample is sufficient to characterize dataset behavior. 
To verify, we extend our case study (\S\ref{sec:RQW}) by computing how many uniformly selected data samples are required to obtain a $TLB$-preserving transform with $k$ equal to input dimension $\dvar$.
On average, a sample of under $0.64\%$ $(\text{up to } 5.5\%)$ of the input is sufficient for $TLB = 0.75$, and under $4.2\%$ $(\text{up to } 38.6\%)$ is sufficient for $TLB=0.99$.  
If this sample rate is known a priori, we obtain up to \red{$91\times$ speedup} over PCA via SVD.%---with no algorithmic improvement. 

However, this benefit is dataset-dependent, and unknown a priori.
We thus turn to progressive sampling (gradually increasing the sample size) to identify how large a sample suffices.
Figure~\ref{fig:progressive} shows how the dimensionality required to attain a given $TLB$ changes when we vary dataset and proportion of data sampled.
Increasing the number of samples (which increases PCA runtime) provides lower $k$ for the same $TLB$.
However, this decrease in dimension plateaus as the number of samples increases.
Thus, while progressive sampling allows DROP to tune the amount of time spent on DR, DROP must determine when the downstream value of decreased dimension is overpowered by the cost of DR---that is, whether to sample to convergence or terminate early (e.g., at $0.3$ proportion of data sampled for SmallKitchenAppliances). 


Concretely, DROP first repeatedly chooses a subset of data and computes a $\dvar$-dimensional transformation via PCA on the subsample, and then proceeds to determine if continued sampling is beneficial to end-to-end runtime.
We consider a simple uniform sampling strategy: each iteration, DROP samples a fixed percentage of the data.
 
 
 
 
 
%Exploring data-dependent and weighted sampling schemes that are dependent on the current basis is an exciting area for future work. 
%While we considered a range of alternative sampling strategies, uniform sampling strikes a balance between computational and statistical efficiency. 
%Data-dependent and weighted sampling schemes that are dependent on the current basis may decrease the total number of iterations required by DROP, but may require expensive reshuffling of data at each iteration~\cite{coresets}. 

%DROP provides configurable strategies for both base number of samples and the per-iteration increment, in our experimental evaluation in \S\ref{sec:experiments}, we consider a sampling rate of $1\%$ per iteration.
%We discuss more sophisticated additions to this base sampling schedule in the extended manuscript.

\begin{algorithm}[t!]
\begin{algorithmic}[1]
\small
\Statex \textbf{Input:}  $X$: data; $B$: target metric preservation level; $\mathcal{C}_\mvar$: cost of downstream operations
\Statex \textbf{Output:} $T_k$: $k$-dimensional transformation matrix
\Statex
\Statex \hrule
\Function{drop}{$X,  B, \mathcal{C}_\mvar$}:
	\State Initialize: $i = 0; k_0 = \infty$ 
		\Comment{iteration and current basis size}
	\Do
		\State i$\texttt{++}$, \textsc{clock.restart}
		\State $X_i$ = \textsc{sample}($X, \textsc{sample-schedule}(i)$) \label{eq:sample}
			\Comment{\S~\ref{subsec:psample}}
		\State $T_{k_i}$ = \textsc{compute-transform}($X, X_i,  B$) \label{eq:evaluate}
			\Comment{\S~\ref{subsec:teval}}
		\State $r_i = \textsc{clock.elapsed}$	
			\Comment{$R = \sum_i r_i$}
		\State $\hat{k}_{i+1}, \hat{r}_{i+1} $ = \textsc{estimate}($k_i, r_i$) \label{eq:estimate}
			\Comment{\S~\ref{subsec:pest}}
	\doWhile{\textsc{optimize}($\mathcal{C}_\mvar,k_i,r_i,\hat{k}_{i+1}, \hat{r}_{i+1}$)} \label{eq:optimize}
		\Comment{\S~\ref{subsec:opt}}
	\\\Return{$T_{k_i}$}
\EndFunction
\end{algorithmic}
\caption{DROP Algorithm}
\label{alg:DROP}
\end{algorithm}



\subsection{Transform Evaluation}
\label{subsec:teval}
DROP must accurately and efficiently evaluate this iteration's performance with respect to the metric of interest \red{over the entire dataset}. 
%To do so, DROP adapts an approach for deterministic queries in online aggregation: treating quality metrics as aggregation functions and using confidence intervals for fast estimation. 
%We first discuss this approach in the context of $TLB$, then discuss how to extend this approach to alternative metrics at the end of this section.
We define this iteration's performance as the size of the lowest dimensional $TLB$-preserving transform ($k_i$) that it can return. 
There are two challenges in performance evaluation.
First, the lowest $TLB$-achieving $k_i$ is unknown a priori. 
Second, brute-force $TLB$ computation would dominate the runtime of computing PCA over a sample. 
We now describe how to solve these challenges.

\subsubsection{Computing the Lowest Dimensional Transformation}

Given the $\dvar$-dimensional transformation from step 1, to reduce dimensionality, DROP must determine if a smaller dimensional $TLB$-preserving transformation can be obtained and return the smallest such transform. 
Ideally, the smallest $k_i$ would be known a priori, but in practice, this is not true---thus, DROP uses the $TLB$ constraint and two properties of PCA to automatically identify it.
%A na\"ive strategy would evaluate the $TLB$ for every combination of the $\dvar$ basis vectors for every transformation size, requiring $O(2^\dvar)$ evaluations. 
%Instead, DROP exploits two key properties of PCA to avoid this.

First, PCA via SVD produces an orthogonal linear transformation where the principal components  are returned in order of decreasing dataset variance explained.
As a result, once DROP has computed the transformation matrix for dimension $\dvar$, DROP obtains the transformations for all dimensions $k$ less than $\dvar$ by truncating the matrix to $\dvar \times k$ .
%PCA via SVD produces an orthogonal linear transformation where the first principal component explains the most variance in the dataset, the second explains the second most---subject to being orthogonal to the first---and so on.  

Second, with respect to $TLB$ preservation, the more principal components that are retained, the better the lower-dimensional representation in terms of $TLB$.  
This is because orthogonal transformations such as PCA preserve inner products. 
Therefore, an $\dvar$-dimensional PCA perfectly preserves $\ell_2$-distance between data points. 
As $\ell_2$-distance is a sum of squared (positive) terms, the more principal components retained, the better the representation preserves $\ell_2$-distance.

Using the first property, DROP obtains all low-dimensional transformations for the sample from the $\dvar$-dimensional basis.  
Using the second property, DROP runs binary search over these transformations to return the lowest-dimensional basis that attains $B$ (Alg.~\ref{alg:candidate}, l\ref{eq:basis}).
If $B$ cannot be realized with this sample, DROP omits further optimization steps and continues the next iteration by drawing a larger sample.

Additionally, computing the full $\dvar$-dimensional basis at every iteration may be wasteful. 
Thus, if DROP has found a candidate $TLB$-preserving basis of size $\dvar' < \dvar$ in prior iterations, then DROP only computes $\dvar'$ components at the start of the next iteration.
This allows for more efficient PCA computation for future iterations, as advanced PCA routines can exploit the $\dvar'$-th eigengap to converge faster (\S\ref{sec:relatedwork}).
% \red{This is because similar to a hold-out or validation set, $TLB$ evaluation is representative of the entire dataset, not just the current sample (see Alg.~\ref{alg:candidate} L5). 
%Thus, sampling additional training datapoints enables DROP to better learn global data structure and perform at least as well as over a smaller sample.}


% stop here!

\subsubsection{Efficient $TLB$ Computation}

Given a transformation, DROP must determine if it preserves the desired $TLB$.
Computing pairwise $TLB$ for all data points requires $O(\mvar^2\dvar)$ time, which dominates the runtime of computing PCA on a sample.
However, as the $TLB$ is an average of random variables bounded from 0 to 1, DROP can use sampling and confidence intervals to compute the $TLB$ to arbitrary confidences.

Given a transformation, DROP iteratively refines an estimate of its $TLB$ (Alg.~\ref{alg:candidate}, l\ref{eq:eval}) by \red{incrementally sampling an increasing number of} pairs from the input data (Alg.~\ref{alg:candidate}, l\ref{eq:paircheck}), transforming each pair into the new basis, then measuring the distortion of $\ell_2$-distance between the pairs, providing a $TLB$ estimate to confidence level $c$ (Alg.~\ref{alg:candidate}, l\ref{eq:tlbeval}). 
If the confidence interval's lower bound is greater than the target $TLB$, the basis is a sufficiently good fit; if its upper bound is less than the target $TLB$, the basis is not a sufficiently good fit. 
If the confidence interval contains the target $TLB$,  \red{ DROP cannot determine if the target $TLB$ is achieved. 
Thus, DROP automatically samples additional pairs to refine its estimate.
%in practice, and especially for our initial target time series datasets, DROP rarely uses more than 500 pairs on average in its $TLB$ estimates (often using far fewer)
}

To estimate the $TLB$ to confidence $c$, DROP uses the Central Limit Theorem: computing the standard deviation of a set of sampled pairs' $TLB$ measures and applying a confidence interval to the sample according to the $c$.
%For low variance data, DROP evaluates a candidate basis with few samples from the dataset \red{as the confidence intervals shrink rapidly}. 

The techniques in this section are presented in the context of $TLB$, but can be applied to any downstream task and metric for which we can compute confidence intervals and are monotonic in number of principal components retained.

\begin{comment}
\red{For instance, DROP can operate while using all of its optimizations when using any $L^p$-norm.}
\red{Euclidean similarity search} is simply one such domain that is a good fit for PCA: when performing DR via PCA, as we increase the number of principal components, a clear positive correlation exists between the percent of variance explained and the $TLB$ regardless of data spectrum.
We demonstrate this correlation in the experiment below, where we generate three synthetic datasets with predefined spectrum (right), representing varying levels of structure present in real-world datasets. 
The positive correlation is evident (left) despite the fact that the two do not directly correspond ($x=y$ provided as reference). 
This holds true for all of the evaluated real world datasets.

\vspace{.2cm}
\includegraphics[width= .9\linewidth]{figs/tlb-pca.pdf}

For alternative preservation metrics, we can utilize closed-form confidence intervals~\cite{stats-book,ci1,onlineagg}, or bootstrap-based methods~\cite{bootstrap1,bootstrap2}, which incur higher overhead but can be more generally applied.
\end{comment}

\begin{algorithm}
\begin{algorithmic}[1]
\small
\Statex \textbf{Input:}  
\Statex $X$: sampled data matrix
\Statex $B$: target metric preservation level; default $TLB = 0.98$
\Statex  \hrule 
\Function{compute-transform}{$X, X_i B$}: \label{eq:basis}
	\State \textsc{pca.fit}$(X_i)$
			\Comment{fit PCA on the sample}
	\State Initialize: high $= k_{i-1}$; low $=0$; $k_i= \frac{1}{2}$(low + high); $B_i = 0$
	\While{(low $!=$ high)}
		\State $T_{k_i}, B_i  = \textsc{evaluate-tlb}( X, B, k_i)$
		\If{$B_i \leq B$}  low $= k_i + 1$ 
		\Else  \hspace{0pt} high $= k_i $
		\EndIf
		\State $k_i = \frac{1}{2}$(low + high)
	\EndWhile
	\State $T_{k_i} = $ cached $k_i$-dimensional PCA transform\\
	\Return $T_{k_i}$
\EndFunction
\Statex 
\Function{evaluate-tlb}{$X, B, k$}: \label{eq:eval}
	\State numPairs $= \frac{1}{2}\mvar(\mvar-1)$
	\State $p = 100$
		\Comment{number of pairs to check metric preservation}
	\While{($p < $ numPairs)}
		\State $B_i, B_{lo}, B_{hi} = $ \textsc{tlb}($ X, p, k$)
			 \label{eq:paircheck}
		\If{($B_{lo} > B$ or $B_{hi} < B$)}   \textbf{break}
		\Else \hspace{0pt} pairs $\times$= $ 2$
		\EndIf
	\EndWhile
	\\\Return $B_i$	
\EndFunction
\Statex 
\Function{tlb}{$X, p, k$}: \label{eq:tlbeval}
	\State \textbf{return } mean and 95\%-CI of the $TLB$ after transforming $p$ $d$-dimensional pairs of points from $X$ to dimension $k$. The highest transformation computed thus far is cached to avoid recomputation of the transformation matrix.
\EndFunction

\end{algorithmic}
\caption{Basis Evaluation and Search}
\label{alg:candidate}
\end{algorithm}


\subsection{Progress Estimation}
\label{subsec:pest}
%Given a low dimensional $TLB$-achieving transformation from the evaluation step, DROP must identify the dimensionality $k_i$ and runtime ($r_i$) of the transformation that would be obtained from an additional DROP iteration.
%We refer to this as the $progress estimation$ step.

Recall that the goal of workload-aware DR is to minimize $R + \mathcal{C}_\mvar(k)$ such that $TLB(XT_k) \geq B$, with $R$ denoting total DR (i.e., DROP's) runtime, $T_k$ the $k$-dimensional $TLB$-preserving transformation of data $X$ returned by DROP, and $\mathcal{C}_\mvar(k)$ the workload cost function. 
Therefore, given a $k_i$-dimensional transformation $T_{k_i}$ returned by the evaluation step of DROP's $i^{\text{th}}$ iteration, DROP can compute the value of this objective function by substituting its elapsed runtime for $R$ and $T_{k_i}$ for $T_k$.  
We denote the value of the objective at the end of iteration $i$ as $obj_i$. 

To decide whether to continue iterating to find a lower dimensional transform, we show in  \S\ref{subsec:opt} that DROP must estimate $obj_{i+1}$. To do so, DROP must estimate the runtime required for iteration $i+1$ (which we denote as $r_{i+1}$, where $R=\sum_i r_i$ after $i$ iterations) and the dimensionality of the $TLB$-preserving transformation produced by iteration $i+1$, $k_{i+1}$. 
DROP cannot directly measure $r_{i+1}$ or $k_{i+1}$ without performing iteration $i+1$, thus performs online progress estimation. Specifically, DROP performs online parametric fitting to compute future values based on prior values for $r_{i}$ and $k_i$ (Alg.~\ref{alg:DROP}, l\ref{eq:estimate}). 
By default, given a sample of size $m_i$ in iteration $i$, DROP performs linear extrapolation to estimate $k_{i+1}$ and $r_{i+1}$. The estimate of $r_{i+1}$, for instance, is:

\vspace{-.4cm}
\begin{equation*}
\hat{r}_{i+1} = r_i + \frac{r_i - r_{i-1}}{m_i - m_{i-1}} (m_{i+1} -  m_i).
\end{equation*}

\begin{comment}
\red{
DROP's use of a basic first-order approximation is motivated by the fact that when adding a small number of data samples each iteration, both runtime and resulting lower dimension do not change drastically (i.e., see Fig.~\ref{fig:progressive} after a feasible point is achieved). 
While linear extrapolation acts as a proof-of-concept for progress estimation, the architecture can incorporate more sophisticated functions as needed (\S\ref{sec:relwork}).
}
\end{comment}

\subsection{Cost-Based Optimization}
\label{subsec:opt}

DROP must determine if continued PCA on additional samples will improve overall runtime. 
%We refer to this as the $cost-based optimization$ step. 
Given predictions of the next iteration's runtime ($\hat{r}_{i+1}$) and dimensionality ($\hat{k}_{i+1}$), DROP uses a greedy heuristic to estimate the optimal stopping point.
If the estimated objective value is greater than its current value ($obj_i < \widehat{obj}_{i+1}$), DROP will terminate. 
If DROP's runtime is convex in the number of iterations, we can prove that this condition is the optimal stopping criterion via convexity of composition of convex functions. 
This stopping criterion leads to the following check at each iteration (Alg.\ref{alg:DROP}, l\ref{eq:optimize}): 

\vspace{-.4cm}
\begin{align}
  obj_i &< \widehat{obj}_{i+1} \nonumber \\
  \mathcal{C}_\mvar(k_i) + \sum_{j=0}^i r_j &< \mathcal{C}_\mvar(\hat{k}_{i+1}) + \sum_{j=0}^{i} r_j + \hat{r}_{i+1} \nonumber \\
  % \mathcal{C}_\mvar(k_i)  &< \mathcal{C}_\mvar(\hat{k}_{i+1}) + \hat{r}_{i+1}  \nonumber \\
  \mathcal{C}_\mvar(k_i) - \mathcal{C}_\mvar(\hat{k}_{i+1}) &< \hat{r}_{i+1}  \label{eq:check}
\end{align}

DROP terminates when the projected time of the next iteration exceeds the estimated downstream runtime benefit. 
%Absent $\mathcal{C}_d$, we default to execution until convergence (i.e, $k$ plateaus), and show the cost of doing so in \S\ref{sec:experiments}.


\begin{comment}
\red{In the general case as the rate of decrease in dimension ($k_i$) is data dependent, thus convexity is not guaranteed. 
Should $k_i$ plateau before continued decrease, DROP will terminate prematurely. 
This occurs during DROP's first iterations if sufficient data to meet the $TLB$ threshold at a dimension lower than $\dvar$ has not been sampled (SmallKitchenAppliances in Fig.~\ref{fig:progressive}).
Thus, optimization is only enabled once a feasible point is attained, as we prioritize accuracy over runtime (i.e., $0.3$ for SmallKitchenAppliances).
We show the implications of this decision in DROP in \S\ref{subsec:arch}.%, and in the streaming setting in the extended manuscript.
}
\end{comment}

\subsection{Choice of PCA Subroutine}
\label{subsec:pcaroutine}

The most straightforward means of implementing PCA via SVD in DROP is computationally inefficient compared to DR alternatives (\S\ref{sec:background}).  
DROP computes PCA via a randomized SVD algorithm from~\cite{tropp} (SVD-Halko).
Alternative efficient methods for PCA exist (i.e., PPCA, which we also provide), but we found that SVD-Halko is asymptotically of the same running time as techniques used in practice, is straightforward to implement, is $2.5-28\times$ faster than our baseline implementations of SVD-based PCA, PPCA, and Oja's method, and does not require hyperparameter tuning for batch size, learning rate, or convergence criteria.  
%While SVD-Halko is not as efficient as other techniques with respect to communication complexity as in~\cite{ppca-sigmod}, or convergence rate as in~\cite{re-new}, these techniques can be easily substituted for SVD-Halko in DROP's architecture.
%%%%We demonstrate this by implementing multiple alternatives in \S\ref{subsec:pcaexp}.
%%%%\red{Further, we also demonstrate that this implementation is competitive with the widely used SciPy Python library~\cite{scipy}}.

\begin{comment}
\begin{algorithm}[t]
\begin{algorithmic}
\State \textbf{Input:}  \\
$H$: concatenation of previous transformation matrices \\
$T$: new sample's transformation \\
 points to sample per iteration; default 5\% \\
 
\\ \hrule

\Function{distill}{$H, T$}:
	\State $H \gets [H | T]$
		\Comment{Horizontal concatenation to update history}
	\State $U, \Sigma, V^\intercal \gets \textsc{SVD}(H)$ 
				\Comment{$U$ is a basis for the range of $T$}
	\State $T \gets U[:,\textsc{num-columns(T)}]$
	\\\Return{$T$}
\EndFunction
\end{algorithmic}
\caption{Work Reuse}
\label{alg:reuse}
\end{algorithm} 
\end{comment}

\subsection{Work Reuse}
\label{subsec:reuse}

A natural question arises due to DROP's iterative architecture: can we combine information across each sample's transformations without computing PCA over the union of the data samples? 
Stochastic PCA methods enable work reuse across samples as they iteratively refine a single transformation matrix, but other methods do not.
%We propose an algorithm that allows reuse of previous work when utilizing arbitrary PCA routines with DROP.
DROP uses two insights to enable work reuse over any PCA routine.

First, given PCA transformation matrices $T_1$ and $T_2$, their horizontal concatenation $H = [T_1 | T_2]$ is a transformation into the union of their range spaces.
Second, principal components returned from running PCA on repeated data samples generally concentrate to the true top principal components for datasets with rapid spectrum drop off.
Work reuse thus proceeds as follows:
DROP maintains a transformation history consisting of the horizontal concatenation of all transformations to this point, computes the SVD of this matrix, and returns the first $k$ columns as the transformation matrix. 

Although this requires an SVD computation, computational overhead is dependent on the size of the history matrix, not the dataset size.
This size is proportional to the original dimensionality $\dvar$ and size of lower dimensional transformations, which are in turn proportional to the data's intrinsic dimensionality and the $TLB$ constraint.
As preserving \emph{all history} can be expensive in practice, 
DROP periodically shrinks the history matrix using DR via PCA. 
We validate the benefit of using work reuse---up to \red{15\%} on real-world data---in \S\ref{sec:experiments}.



\bibliographystyle{siamplain}
\bibliography{biblio}

\end{document}