\section{Finding a Maximal Solution}
\label{sec:finding}

Using the classical barrier function, in this section we show how to get a path that converges to a relative interior point of $\Xl$, which turns out to be the analytic center of $\Xl$.

Setting $Q = \Phi^* \Phi$ is the Gram matrix and $c = \Phi^* y$, we start by rewriting our initial problem~\cref{eq:p} as an augmented quadratic program under constraints, i.e.
\begin{equation*}
  \umin{x \in \RR^n, t \in \RR^p}
  \frac{1}{2} \dotp{Qx}{x} - \dotp{c}{x} + \lambda \sum_{i=1}^p t_i
  \qsubjq
  \begin{cases}
    -t \leq D^* x \leq t &\\
    t_i \geq 0 &\\
  \end{cases} ,
\end{equation*}
witch also can be rewritten as
\begin{equation*}
  \umin{x \in \RR^n, t \in \RR^p}
  \frac{1}{2} \dotp{Qx}{x} - \dotp{c}{x} + \lambda \sum_{i=1}^p t_i
  \qsubjq
  \begin{cases}
  -t+s=D^*x&\\
  t-s'=D^*x&\\
    t_i \geq 0,\ s_i\geq0,\ s'_i\geq0 &\\
  \end{cases}.
\end{equation*}
Now observe that $t=\displaystyle{1\over 2}(s+s')$. Then setting $z=\displaystyle{1\over 2}\left(\begin{array}{l}s\cr s'\end{array}\right)$, $I_p$ the $p$ by $p$ identity matrix, ${\tilde I}=\left(\begin{array}{lr}I_p& -I_p\end{array}\right)$ and $e=(1, \cdots,1)\in\RR^{2p}$, we come to the following equivalent formulation of the problem
\begin{equation}
\label{eq:paug}
\umin{x\in\RR^n, z\in\RR^{2p}}
f(x,z)
\qsubjq z\in[0,+\infty)^{2p}
\end{equation}
where $$
f(x,z)=\left\{\begin{array}{ll}\displaystyle{1\over2}\dotp{Qx}{x} - \dotp{c}{x} +\lambda\dotp{e}{z}&\mbox{ if }D^*x+{\tilde I}z=0\\
+\infty&\mbox{ elsewhere,}\end{array}\right.$$ 
or equivalently 
$$f(x,z)=\left\{\begin{array}{ll}\displaystyle{1\over2}\|\Phi x-y\|^2-\displaystyle{1\over2}\|y\|^2 +\lambda\dotp{e}{z}&\mbox{ if }D^*x+{\tilde I}z=0\\
+\infty&\mbox{ elsewhere.}\end{array}\right.$$ 


Its classical dual is
\begin{equation}
\label{eq:daug}
\umax{x\in\RR^n, s\in\RR^{2p}, u\in\RR^p}
g(x,s,u)
\qsubjq s\in[0,+\infty)^{2p}
\end{equation}
where 
$$g(x,s,u)=\left\{\begin{array}{ll}
-\displaystyle{1\over 2}\langle Qx,x\rangle&\mbox{if }{D} u+c-Qx=0,\ s=\lambda e-{\tilde I}^*u\cr 
-\infty&\mbox{elsewhere.}\end{array}\right.$$ 
We set $S_{(P)}$ (resp. $S_{(D)}$) the optimal solutions' set of
problem~\cref{eq:paug} (resp. problem~\cref{eq:daug}).
We know that $\Xl$ is non-empty and so $S_{(P)}$.
Since, in addition~\cref{eq:paug} is a convex problem with polyedral constraints, $S_{(D)}$ is non empty and there is no duality gap. We denote by $\alpha$ the optimal value of the two problems. 

\begin{proposition}\label{dcompacity}{$ $}

\begin{itemize}
\item[1.]The optimal solution $S_{(P)}$ of the problem (\ref{eq:paug}) is bounded or equivalently the set $\{(d_x,d_z):\ f_\infty(d_x,d_z)\leq0,\ d_z\geq0\}=\{0\}$,
\item[2.]$S(.,(D))=\{(s,u):\ \exists x\in\RR^n\mbox{ such that }(x,s,u)\in S_{(D)}\}$ is bounded, in other words, the dual feasible solutions' set is bounded in $(s,u)$.
\end{itemize}
\end{proposition}
\begin{proof}
1. Because of relation (\ref{eq:hyp-inv}) it is not difficult to show that the optimal solution $S_{(P)}$ of the problem (\ref{eq:paug}) is bounded.


2. Let $(x^k,s^k,u^k)$ be a sequence of the dual feasible solutions' set. We have $s^k=\lambda e-{\tilde I}^*u=\left(\begin{array}{l}\lambda e^p\cr\lambda e^p\end{array}\right)-\left(\begin{array}{l}u^k\cr-u^k\end{array}\right)\geq0$, where $e^p=(1,\cdots 1)\in\RR^p$. It follows that $-\lambda e^p\leq u^k\leq \lambda e^p$. Hence $(u^k)$ and then $(s^k)$, is bounded.
\end{proof}




Using the classical logarithmic barrier function introduced by Frish~\cite{frisch}, we deal with the family of problems $(P_\mu)_{\mu>0}$ given by
\begin{equation*}
\theta(\mu)=\umin{x\in\RR^n, z\in\RR^{2p}}
F_{\mu}(x,z)=f(x,z)+\zeta(z,\mu)
\end{equation*}
where $$\zeta(z,\mu)=\left\{\begin{array}{ll}
\mu \xi\left(z/\mu\right)&\mbox{if }\mu>0,\cr\xi_\infty(z)&\mbox{if }\mu=0,\cr+\infty&\mbox{elsewhere,}
\end{array}\right.$$ $$ \xi(z)=\left\{\begin{array}{ll}-\ln \varphi(z)&\mbox{if }\varphi(z)>0,\cr+\infty&\mbox{elsewhere,}\end{array}\right. \mbox{ and }\varphi(z)=\left\{\begin{array}{ll}\left(\prod\limits_{i=1}^{2p}z_i\right)^{1\over2p}&\mbox{if }z\geq0,\cr-\infty&\mbox{elsewhere.}\end{array}\right.$$


Note that the function $\varphi$ is strictly quasiconcave and then according to Lemma 1 of \cite{barbara2015strict}, for every $\mu>0$, the function $\zeta_\mu:z\mapsto\zeta(z,\mu)$ is strictly convex on $(0,+\infty)^{2p}$. 
\begin{proposition}\label{strict_convexity}
For every $\mu>0$, the function $F_{\mu}$ is inf-compact on $\RR^n\times\RR^{2p}$ and strictly convex on $\RR^n\times(0,+\infty)^{2p}$.
\end{proposition}
\begin{proof} Let us show that  
\begin{eqnarray}\label{xiinfty}
\xi_\infty(d)=\left\{\begin{array}{ll}0&\mbox{if }d\geq0,\cr+\infty&\mbox{elsewhere.}\end{array}
\right.\end{eqnarray}
Let $(z,d)\in \dom(\xi)\times\RR^{2p}$. We have necessarily $z>0$. First we observe that when $d\not\in[0,+\infty)^{2p}$, $z+\lambda d\not\in[0,+\infty)^{2p}$ for $\lambda$ large enough and then $\xi_\infty(d)=+\infty$. Now consider the case $d\geq0$. Since $z>0$ we have necessarily $z+d>0$. The concave gauge function $\varphi$ is monotone with respect to its domaine the positive orthant. Then by Proposition 2.1 of \cite{barbara_crouzeix},
$$0<\varphi(z+d)\leq\varphi(z+\lambda d)\leq\varphi(\lambda z+\lambda d)=\lambda\varphi(z+d)$$
for $\lambda$ large enough. It follows that
$$\begin{array}{ll}
0=\lim\limits_{\lambda\uparrow+\infty}\displaystyle{\ln\varphi(z+d)-\ln\varphi(z)\over\lambda}&\leq\lim\limits_{\lambda\uparrow+ \infty}\displaystyle{\ln\varphi(z+\lambda d)-\ln\varphi(z)\over\lambda}\cr&\leq\lim\limits_{\lambda\uparrow+\infty} \displaystyle{\ln\lambda\varphi(z+d)-\ln\varphi(z)\over\lambda}=0\end{array}$$ and hence $\lim\limits_{\lambda\uparrow+ \infty}\displaystyle{\ln\varphi(z+\lambda d)-\ln\varphi(z)\over\lambda}=0$. Consequently $\xi_\infty(d)=0$.

By Proposition \ref{dcompacity}, we have $\{(d_x,d_z):\ f_\infty(d_x,d_z)\leq0,\ d_z\geq0\}=\{(0,0)\}$. Thus
$\{(d_x,d_z):\ {F_\mu}_\infty(d_x,d_z)\leq0,\ d_z\geq0\}=\{(0,0)\}$, or equivalently, $F_\mu$ is inf-compact. 

Now let us proceed to prove the strict convexity of $F_\mu$.  Take  $(x,z)\not=(x',z')$ in $\RR^n\times(0,+\infty)^{2p}$ and $t\in(0,1)$. In the case where $z\not= z'$, by strict-convexity of $\zeta_\mu$ on $(0,+\infty)^{2p}$ we have necessarily $F_\mu(t(x,z)+(1-t)(x',z'))<tF_\mu(x,z)+(1-t)F_\mu(x',z').$ Assume that $z=z'$. Using (\ref{eq:hyp-inv}) and the definition of $f$ we obtain $\Phi x\not=\Phi x'$ and the result follows by using the strict convexity of $\|.\|_2^2$.
\end{proof}
Propositions \ref{strict_convexity} and \ref{dcompacity} assert that for every $\mu>0$ there is a unique optimal solution  $(x(\mu),z(\mu))$ to $(P_\mu)$. Moreover using the fact that $F_\mu(x,\cdot)$ is a barrier function for every $x\in\RR^n$, $z(\mu)>0$. Consider the function $\gamma:\RR^n\times[0,+\infty)^{2p}\times[0,+\infty)\to \RR\cup\{+\infty\}$ defined by
$$\gamma(x,z,\mu)=F_{\mu}(x,z).$$
Then we have the following proposition.

\begin{proposition}
\label{coercivity}
The function $\gamma$ is convex and lsc on $\RR^n\times\RR^{2p}\times[0,+\infty)$. It is inf-compact on $\RR^n\times\RR^{2p}\times[0,{\overline \mu}]$, $\forall\overline{\mu}>0$ being fixed. Moreover $\theta$ is convex and continuous on $[0,+\infty)$, $\theta(0)=\alpha$ and $f(x,z)=\gamma(x,z,0)$, $\forall (x,z)\in\RR^n\times(0,+\infty)^{2p}$.
\end{proposition}
\begin{proof}
It is known that the function $\zeta$ is convex on $\RR^{2p}\times[0,+\infty)$
and so is $\gamma$. The function $\theta$ is then convex on $[0,+\infty)$ as the
infimum over $(x,z)$ of a convex function in $(x,z,\mu)$. Now the function
$\zeta(z,.)$ is continuous on $[0,+\infty)$ and, because of (\ref{xiinfty}),
$\zeta(z,0)=0$ for all $z\in(0,+\infty)^{2p}$. Thus $f(x,z)=\gamma(x,z,0)$ for
all $(x,z)\in\RR^n\times(0,+\infty)^{2p}$ and therefore $\theta(0)=\alpha$ (the
optimal value of the problem (\ref{eq:paug})). Set
$\tilde{\gamma}=\gamma_{|\RR^n\times\RR^{2p}\times[0,\overline{\mu}]}$ the
restriction of $\gamma$ to the set
$\RR^n\times\RR^{2p}\times[0,\overline{\mu}]$. Then $\{(d_x,d_z,\mu):\
\tilde{\gamma}_\infty(d_x,d_z,\mu)\leq0,\ d_z\geq0,\ \mu=0\}=\{(d_x,d_z,0):\
f_\infty(d_x,d_z)\leq0,\ d_z\geq0\}=\{(0,0,0)\}$ (see Proposition
\ref{dcompacity}). The function $\gamma$ is then inf-compact on
$\RR^n\times\RR^{2p}\times[0,{\overline \mu}]$. Consequently, there is a compact
$\tilde{S}$ such that $(x(\mu),z(\mu))\in\tilde{S}$,
$\forall\mu\in(0,\overline{\mu}]$, i.e.,
$(x(\mu),z(\mu))_{\mu\in(0,\overline{\mu})}$ is bounded. We established that $\theta$ is convex on $[0,+\infty)$. It is then continuous on $(0,+\infty)$. Let us show now that $\lim\limits_{\mu\downarrow 0}\theta(\mu)=\theta(0)=\alpha$. In this respect we shall prove that 
$\lim\limits_{\mu\downarrow0}\mu\ln\left(\displaystyle{\varphi(z(\mu)) \over\mu}\right)= 0$. Let $(\mu^k)_{k\in\NN}$ be a positive sequence such that $\lim\limits_{k\uparrow+\infty}\mu^k=0.$ We established that $(x(\mu),z(\mu))_{\mu\in(0,\overline{\mu}]}$ is bounded. It follows that the set $\{(x(\mu^k),z(\mu^k))\}$ contains a subsequence converging to a point  $(\tilde{x},\tilde{z})$.
In the case where $\tilde{z}>0$ the result is obvious. Assume that $\varphi(\tilde{z})=0$. Then for $k$ sufficiently large one has
$$\begin{array}{ll}\alpha-\mu^k\ln\left(\displaystyle{\varphi(z)\over\mu^k}\right)
\leq \theta(\mu^k)&=f(x(\mu^k),z(\mu^k))-\mu^k\ln\left(\displaystyle{\varphi(z(\mu^k)) \over \mu^k}\right)\cr&
\leq f(x,z)-\mu^k\ln\left(\displaystyle{\varphi(z)\over\mu^k}\right)
\end{array}$$
for every $(x,z)$ satisfying $z>0$. Since $\lim\limits_{k\uparrow 0}\mu^k\ln\left(\displaystyle{\varphi(z)\over\mu^k}\right)=0$, we have
$$\alpha\leq\lim\inf\limits_{k\uparrow+\infty}\theta(\mu^k)\leq f(x,z)$$
and then
$$\alpha\leq\lim\sup\limits_{k\uparrow+\infty}\theta(\mu^k)\leq \inf\limits_{x,z}\{f(x,z):\ z>0\}=\inf\limits_{x,z}\{f(x,z):\ z\geq0\}=\alpha.$$
Consequently $\lim\limits_{k\uparrow+\infty}\theta(\mu^k)=\alpha$.
\end{proof}
\bigskip
\bigskip


Given $\mu>0$, the KKT optimalty conditions for the problem $(P_\mu)$ can be formulated, for some $u\in\RR^p$, as
$$\left\{\begin{array}{ll}Qx(\mu)-c-Du=0,\\ \lambda e-\displaystyle{ \mu\over 2p}(Z(\mu))^{-1}e-{\tilde I}^*u=0,\\ D^*x(\mu)+{\tilde I}z(\mu)=0,\end{array}\right.$$
where $Z(\mu)=diag(z(\mu))$.
Observe that $u$ is necessarily unique. Put 
$$u=u(\mu)\mbox{ and }s(\mu)=\displaystyle {\mu\over 2p}Z^{-1}(\mu)e.$$ We rewrite the KKT conditions as
$$\left\{\begin{array}{lr}Qx(\mu)-c-Du(\mu)=0&(E1)\\ \lambda e-s(\mu)-{\tilde I}^*u(\mu)=0&(E2)\\ Z(\mu)s(\mu)=\displaystyle {\mu\over 2p}e&(E3)\\D^*x(\mu)+{\tilde I}z(\mu)=0&(E4)\end{array}\right.$$

\begin{proposition}
For every $\mu>0$, $(s(\mu),u(\mu))$ is a feasible solution to (\ref{eq:daug}) and $\big((s(\mu),u(\mu)\big)_{\mu\in(0,\overline{\mu}]}$ is bounded.
\end{proposition}

\begin{proof}
By $(E1)$, $(E2)$ and the fact that $s(\mu)=\displaystyle {\mu\over 2p}(Z(\mu))^{-1}e>0$, $(u(\mu),s(\mu))$ is a feasible solution to (\ref{eq:daug}). The boundedness of $(s(\mu),u(\mu))_{\mu\in(0,{\overline\mu}]}$ is due to Proposition \ref{dcompacity}.
\end{proof} 


Set ${\overline I}=\displaystyle\bigcup_{\atop z\in S(.,(P))}I(z)$ and ${\overline J}=\displaystyle\bigcup_{\atop s\in S(.,(D))}J(s)$, where 
$$S(.,(P))=\left\{z:\ \exists x\in\RR^n\mbox{ such that } (x,z)\in S_{(P)}\right\},$$ 
$$S(.,(D))=\left\{s:\ \exists u\in\RR^p\mbox{ such that } (s,u)\in S_{(D)}\right\},$$ $$I(z)=\{i:\ z_i>0\}\mbox{ the support of }z\mbox{ and }J(s)=\{i:\ s_i>0\}\mbox{ the support of }s.$$  

\begin{lemma}\label{complementarity}

There is at least one $({\hat z},\hat{s})\in S(.,(P))\times S(.,(D))$ such that ${\overline I}=I({\hat z})$ and $\overline{J}=J(\hat{s})$.
\end{lemma}

\begin{proof}
We have ${\overline I}$ a subset of a finite set $\{1,\cdots,2p\}$. Let then $(z^1,\ z^2,\cdots,z^k)\in S(.,(P))^k$, for some $k\in\{1,2,\cdots,2p\}$
satisfying ${\overline I}=I\left(z^1\right)\cup I\left(z^2\right)\cup\cdots\cup I\left(z^k\right)$. Set ${\hat z}=\displaystyle{1\over k}\left(z^1+z^2+\cdots+z^k\right)$. Since $S(.,(P))$ is convex  ${\hat z}\in S(.,(P))$. So it is easy to see that $I(z^i)\subset I({\hat z})$, $\forall i\in\{1,2,\cdots,k\}$. The result then follows. A vector $\hat{s}$ is constructed in a similar way.
\end{proof} 
Observe that every optimal solution $(x,z)$ of the problem (\ref{eq:paug}) satisfying $I(z)=\overline{I}$ is in the relative interior of $S_{(P)}$. Similarily every optimal solution $(x,s,u)$ of the problem (\ref{eq:daug}) satisfying $J(s)=\overline{J}$ is in the relative interior of $S_{(D)}$.

\bigskip

Set 
$$({\overline x},{\overline z})=\arg\max\left\{\varphi_{\overline I}(z_{\overline I}):\ \displaystyle{1\over 2}\langle Qx,x\rangle-\langle c,x\rangle+\lambda\langle e,z\rangle=\alpha,\ D^*x+{\tilde I}z=0,\ z_{\overline{J}}=0\right\},$$
where 
$$\varphi_{\overline I}(z_{\overline I})=\left\{\begin{array}{ll}
\left(\prod\limits_{i\in {\overline I}}z_i\right)^{1\over \card({\overline I})}&\mbox{if }z_J\in(0,+\infty)^{\card(J)}\cr
-\infty&\mbox{elsewhere.}
\end{array}\right.
$$
Symmetrically we set
$$({\overline s},{\overline u})=\arg\max\left\{\varphi_{\overline J}(s_{\overline J}):\ 
s=\lambda e-\tilde{I}^*u,\ D u+c-Q{\overline x}=0, s_{\overline I}=0\right\},$$
where
$$\varphi_{\overline{J}}(s_{\overline{J}})=\left\{\begin{array}{ll}\left(\prod\limits_{i\in \overline{J}}s_i\right)^{1\over \card(\overline{J})}&\mbox{if }s_{\overline{J}}\in(0,+\infty)^{\card(\overline{J})}\cr-\infty&\mbox{elsewhere.}\end{array}\right.
$$
$(\overline{x},\overline{z})$ is called the analytic center\footnotemark[1]\footnotetext[1]{A generalization of the central path and the analytic center is proposed in \cite{barbara2015strict} by using the so called concave gauge functions.} of (\ref{eq:paug}) and $(\overline{x},\overline{s},\overline{u})$ the analytic center of (\ref{eq:daug}). The uniqueness is ensured by the strict quasiconcavity of functions $\varphi_{\overline{I}}$ and  $\varphi_{\overline{J}}$ on the interior of their respective domain and the assumption (\ref{eq:hyp-inv}). We now give an important result. 



Its proof is inspired in part by those of Theorems I.7 and I.9 in \cite{RoTevi}.
\begin{theorem}\label{thm:convergence}
Under assumption \ref{eq:hyp-inv}, we have $$\lim\limits_{\mu\downarrow0}(x(\mu),z(\mu),s(\mu),u(\mu))=({\overline x},{\overline z},{\overline s},{\overline u}).$$ Moreover, $({\overline x},{\overline z})$  and $({\overline x},{\overline s},{\overline u})$ belong to the relative interior of $S_{(P)}$ and $S_{(D)}$, respectively. 
\end{theorem}

\begin{proof}
We proved that $\big((x(\mu),z(\mu)\big)_{\mu\in(0,\overline{\mu}]}$ and $\big((s(\mu),u(\mu)\big)_{\mu\in(0,\overline{\mu}]}$ are bounded. Let $(\mu^k)_{k\in \NN}$ a positive increasing sequence satisfying $$\lim\limits_{k\uparrow +\infty}\mu^k=0\mbox{ and }\lim\limits_{k\uparrow+\infty}(x(\mu^k),z(\mu^k),s(\mu^k),u(\mu^k))=(\tilde{x},\tilde{z},\tilde{s},\tilde{u}).$$ Then replacing $\mu$ by $\mu^k$ in $(E1)-(E4)$ and letting $k$ tend to $+\infty$, we observe that the pair $\{(\tilde{x},\tilde{z}),(\tilde{x},\tilde{s},\tilde{u})\}$ satisfies the KKT optimality conditions of (\ref{eq:paug}) and then it is a primal-dual optimal solution pair of (\ref{eq:paug}). Let us show now that $I(\tilde{z})=\overline{I}$ and $J(\tilde{s})=\overline{J}$. 
Now by $(E1)$, $(E2)$ and $(E4)$ we have  
$$\left(\begin{array}{l}
x(\mu^k)-\overline{x}\cr
z(\mu^k)-\overline{z}\end{array}\right)\in\Ker{\left(\begin{array}{ll}D^*&{\tilde I}\end{array}\right)}\mbox{ and }
\left(\begin{array}{l}
Q(x(\mu^k)-\overline{x})\cr
-(s(\mu^k)-\overline{s})\end{array}\right)\in\Im{\left(\begin{array}{l}D \\ {\tilde I}^*\end{array}\right)}.
$$
Then using the following orthogonality property 
\begin{equation}
\label{orthogonality}
\Ker{\left(\begin{array}{ll}{D^*}&{\tilde I}\end{array}\right)}=\left[\Im{\left(\begin{array}{l}D \\ {\tilde I}^*\end{array}\right)}\right]^\bot,
\end{equation} 
$(E3)$ and the fact that $\langle\overline{z},\overline{s}\rangle=\langle\tilde{z},\tilde{s}\rangle=0$ we have 
$$\langle\overline{z},s(\mu^k)\rangle+\langle\overline{s},z(\mu^k)\rangle=\mu^k-\langle Q(x(\mu^k)-\overline{x}),x(\mu^k)-\overline{x}\rangle.$$ 
Since in addition $I(\overline{z})=\overline{I}$, $J(\overline{s})=\overline{J}$ and $Q$ is positive semi-definite we get
$$\sum\limits_{i\in \overline{I}}\overline{z}_is(\mu^k)_i+\sum\limits_{i\in \overline{J}}\overline{s}_iz(\mu^k)_i=\mu^k-\langle Q(x(\mu^k)-\tilde{x}),x(\mu^k)-\tilde{x}\rangle\leq\mu^k.$$
But from $(E3)$, $z(\mu^k)_is(\mu^k)_i=\displaystyle{\mu^k\over 2p},\ \forall i$. it follows that
$$\displaystyle\sum\limits_{i\in \overline{J}}{\overline{s}_i\over s(\mu^k)_i}+\sum\limits_{i\in \overline{I}}{\overline{z}_i\over z(\mu^k)_i}\leq 2p.$$
Now letting $k$ tend to $+\infty$, we get on the one hand
$$0<\displaystyle\sum\limits_{i\in \overline{J}}{\overline{s}_i\over \tilde{s}_i}+\sum\limits_{i\in \overline{I}}{\overline{z}_i\over \tilde{z}_i}\leq 2p<+\infty$$
and then, by construction of $\overline{I}$ and $\overline{J}$, we have necessarily $I(\tilde{z})=\overline{I}$ and $J(\tilde{s})=\overline{J}$. On the other hand,
using the arithmetic-geometric mean inequality we get
$$\left(\prod\limits_{i\in \overline{J}}\displaystyle{\overline{s}\over \tilde{s}_i}\prod\limits_{i\in \overline{I}}\displaystyle{\overline{z}\over \tilde{z}_i}\right)^{1\over 2p}\leq{1\over 2p}\left(\displaystyle\sum\limits_{i\in \overline{J}}{\overline{s}\over \tilde{s}_i}+\sum\limits_{i\in \overline{I}}{\overline{z}\over \tilde{z}_i}\right)\leq 1$$
and then 
$$\varphi_{\overline{J}}( \overline{s}_{\overline{J}})\varphi_{\overline{I}}( \overline{z}_{\overline{I}})\leq\varphi_{\overline{J}}( \tilde{s}_{\overline{J}})\varphi_{\overline{I}}( \tilde{z}_{\overline{I}}).$$
But, by definition of $(\overline{x},\overline{z},\overline{s},\overline{u})$, $\varphi_{\overline{J}}( \tilde{s}_{\overline{J}})\leq \varphi_{\overline{J}}( \overline{s}_{\overline{J}})$ and $\varphi_{\overline{I}}( \tilde{z}_{\overline{I}})\leq \varphi_{\overline{I}}( \overline{z}_{\overline{I}})$.  The result then follows. 
\end{proof}

Consequently, the following corollary holds
\begin{corollary}
  Under assumption (\ref{eq:hyp-inv}), we have $\lim\limits_{\mu \downarrow 0} x(\mu) = \bar x \in \rint \Xl$.
\end{corollary}

\begin{proof}
By  Theorem \ref{thm:convergence} $(\overline{x},\overline{z})$ belongs to  the relative interior of $S_{(P)}$ and hence $\overline{x}$ belongs to the linear projection of the relative interior of $S_{(P)}$ which is equal to $\rint \Xl$.
\end{proof}