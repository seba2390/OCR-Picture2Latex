Using the analysis, we propose an algorithm directly adapted from the Predictor-corrector Mehrotra's algorithm~\cite{Mehrotra}.
The pseudo-code is given in Algorithm~\ref{alg:main}.
The user is expected to give a primal-dual starting point $(x^0,z^0,u^0,s^0)$ satisfying $z^0 > 0$ and $s^0 > 0$, the scenario $\Phi$, $D^*$, $y$, a stopping criterion $\epsilon > 0$, and a relaxation parameter $\eta \in (0,1)$.

To illustrate our theoretical results, we consider a very simple scenario in $\RR^2$ to $\RR$.
Let $D = \Id_2$, $\Phi = (1 \quad 1)$, $y = 1$ and $\lambda = \frac{1}{2}$.
The first order conditions reads
\begin{align*}
  2 x_1 + 2 x_2 - 2 + s_1 &= 0 \\
  2 x_1 + 2 x_2 - 2 + s_2 &= 0 ,
\end{align*}
where $s \in \partial \normu{\cdot}(x)$.
One can check that $x^\star = (\frac{1}{2} \quad 0)^*$ is a solution.
Using the fact that $\Xl \subseteq x^\star + \Ker \Phi$ and that every solution share the same $\lun$-norm, we have that $\Xl = \conv{(\frac{1}{2} \quad 0)^*, (0 \quad \frac{1}{2})^*}$.
Figure~\ref{fig:path} represents the evolution of the primal iterate on the plane $\RR^2$.
\begin{figure}[h]
  \centering
  \begin{subfigure}[t]{0.45\textwidth}
    \centering
    \includegraphics[height=2in]{img/path.pdf}
    \caption{Path}
  \end{subfigure}%
  ~ 
  \begin{subfigure}[t]{0.45\textwidth}
    \centering
    \includegraphics[height=2in]{img/zoom.pdf}
    \caption{Zoom around the analytical center}
  \end{subfigure}
  \caption{Algorithm path. The red line corresponds to the solution set $\Xl$, the blue line is the algorithm path for $x^0 = (0.7 \, 0)^*$ and the green line for $x^0$ obtained by a least square.}
  \label{fig:path}
\end{figure}

\renewcommand{\algorithmicrequire}{\textbf{Input:}}

\begin{algorithm}
  \caption{Adapted predictor-corrector Mehrotra's algorithm}\label{alg:main}
  \begin{algorithmic}
    \Require $(x^0,z^0,u^0,s^0)$, $\Phi$, $D^*$, $y$, $\epsilon > 0$, $\eta \in (0,1)$
    \State $Q \gets \Phi^* \Phi$, $c \gets \Phi^* y$
    \State Set complementarity measure
    \[
          r_1 \gets Qx-c-D^*u,
          r_2 \gets \lambda e-s-\tilde{I}^*u,
          r_3 \gets Zs,
          r_4 \gets D^*x+\tilde{I}z.
    \] 
    \State $\mu \gets \displaystyle{\langle z,s\rangle\over 2p}$

    \While{$\max\{\|r_1\|_2,\|r_2\|_2,\|r_3\|_2,\|r_4\|_2\}>\epsilon$}
    \State Compute the affine scaling direction $(d_x^a,d_z^a,d_u^a,d_s^a)$ by solving the system
    \[
      \left\{
        \begin{array}{ll}
          Qd_x^a-D^*d_u^a&=-r_1 \\
          -d_s^a-\tilde{I}^*d_u^a&=-r_2 \\
          Sd_z^a+Zd_s^a&=-r_3 \\
          D^*d_x^a+\tilde{I}d_z^a&=-r_4,
        \end{array}
      \right.
    \]
    \State $t^a_{\max} \gets \max\{ t\geq 0:\ z+td_z^a\geq 0,\ s+d_s^a\geq 0\}$
    \State $\mu^a \gets \displaystyle{\langle z+t_{\max}^ad_z^a,s+t_{\max}^ad_s\rangle\over 2p}$
    \State $\sigma \gets \displaystyle{\left(\mu^a\over\mu\right)^3}$ \Comment{centering parameter}
    \State Compute corrector and centering direction $(d_x^c,d_z^c,d_u^c,d_s^c)$ by solving
    \[
      \left\{
        \begin{array}{ll}
          Qd_x^c-{D^*}^*d_u^c&=0\\
          -d_s^c-\tilde{I}^*d_u^c&=0\\
          Sd_z^c+Zd_s^c&=-D_z^ad_s^a+\sigma\mu e\\
          D^*d_x^a+\tilde{I}d_z^a&=0,
        \end{array}
      \right. \qwhereq D_z^a=\diag(d_z^a)
    \]
    \State $(d_x,d_z,d_u,d_s) \gets (d_x^a,d_z^a,d_u^a,d_s^a)+(d_x^c,d_z^c,d_u^c,d_s^c)$ \Comment{predictor direction}
    \State $t_{\max} \gets \max\{ t\geq 0:\ z+td_z\geq 0,\ s+d_s\geq 0\}$
    \State $(x,z,u,s) \gets (x,z,u,s)+\eta t_{\max}(d_x,d_z,d_u,d_s)$
    \State Update complementarity measure
    \[
          r_1 \gets Qx-c-D^*u,
          r_2 \gets \lambda e-s-\tilde{I}^*u,
          r_3 \gets Zs,
          r_4 \gets D^*x+\tilde{I}z.
    \]
    \EndWhile
  \end{algorithmic}
\end{algorithm}