\section{The Solution Set}
\label{sec:sol}

This section deals reviews some properties of the solution set $\Xl$.
The following proposition shows that even if $\Xl$ is not reduced to a singleton, its image by $\Phi$ or the analysis-$\lun$-norm is single-valued.
\begin{proposition}[Unique image]\label{lem:same-image}
  Let $x^1, x^2 \in \Xl$.
  Then,
  \begin{enumerate}
  \item they share the same image by $\Phi$, i.e., $\Phi x^1 = \Phi x^2$ ;
  \item they have the same analysis-$\lun$-norm, i.e., $\normu{D^* x^1} = \normu{D^* x^2}$.
  \end{enumerate}
\end{proposition}
A proof of this statement can be found for instance in~\cite{vaiter2011robust}.

It is known that standard $\ldeux$-regularization suffers from sign inconsistencies, i.e. two differents solutions can be of opposite signs at some indice.
The following proposition gives another important information: the cosign of two solutions cannot be opposite.
\begin{proposition}[Consistency of the sign]\label{prop:sign}
  Let $x^1, x^2 \in \Xl$.
  Then,
  \begin{equation*}
    \forall i \in \ens{1,\dots,p}, \quad u_i^1 u_i^2 \geq 0 ,
  \end{equation*}
  where $u^k = D^* x^k$ for $k=1,2$.
\end{proposition}
\begin{proof}
  The proof of this statement follows closely the proof found in~\cite{attouch1999p} for $\lun$.
  Suppose there exists $i$ such that $u_i^1$ and $u_i^2$ have opposite signs.
  Then, one has
  \begin{equation}\label{eq:sign-strict-ineq}
    \frac{\abs{u_i^1 + u_i^2}}{2} < \frac{\abs{u_i^1} + \abs{u_i^2}}{2} .
  \end{equation}
  Let $z = u^1 + u^2$.
  Using the convexity of $x \mapsto \norm{y - \Phi x}_2^2$ and inequality~\eqref{eq:sign-strict-ineq}, we get that
  \begin{align*}
    \frac{1}{2} \norm{y - \Phi z}_2^2 + \normu{D^* z} 
    & \!<\! \frac{1}{2} \left( \left( \frac{1}{2} \norm{y - \Phi x^1}_2^2 + \normu{D^* x^1}  \right) \!+\! \left( \frac{1}{2} \norm{y - \Phi x^2}_2^2 + \normu{D^* x^2} \right) \right) \\
    & \!=\! \umin{x \in \RR^n} \frac{1}{2} \norm{y - \Phi x}_2^2 + \normu{D^* x} ,
  \end{align*}
  which is a contradiction.
\end{proof}

Condition~\cref{eq:hyp-inv} (we recall that all through this paper, we suppose this condition holds) ensures that $\Xl$ is a non-empty, convex and compact set.
 Recall for all the following that given a lower semicontinuous real-valued extended convex function $h$ on $\RR^l$, its recession function can be defined by (Theorem 8.5 of \cite{rockafellar})
$$h_\infty(d)=\lim\limits_{\lambda\uparrow+\infty}\displaystyle{h(z+\lambda d)-h(z)\over\lambda},\ \forall (z,d)\in \dom(h)\times\RR^{l}.$$
In fact, as stated by the following proposition, the solution set $\Xl$ is a polytope.
\begin{proposition}\label{prop:polysol}
  $\Xl$ is a polytope (i.e. a bounded polyhedron).
\end{proposition}
\begin{proof}
  Let us first prove that $\Xl$ is a non-empty, convex and compact set. 
It follows with the help of hypothesis~\eqref{eq:hyp-inv} that $\{d:\ h_\infty(d)\leq0\}=\{0\}$.
  Hence, $\Xl$ is bounded.

  We shall now prove that $\Xl$ is a polytope.
  Let $\bar{x} \in \Xl$.
  According to Proposition~\ref{lem:same-image}, we have
  \begin{equation*}
    \Xl \subseteq
    \enscond{x \in \RR^n}{\normu{D^* x} = \normu{D^* \bar{x}}} \cap
    \enscond{x \in \RR^n}{\Phi x = \Phi \bar{x}} .
  \end{equation*}
  The reverse inclusion came from the fact that if $x$ shares the same image by $\Phi$ as $\bar{x}$ and the same analysis-$\lun$-norm, then the objective function at $x$ is equal to the one at $\bar{x}$, hence is also a solution.
  Thus,
  \begin{equation*}
    \Xl =
    \enscond{x \in \RR^n}{\normu{D^* x} = \normu{D^* \bar{x}}} \cap
    \enscond{x \in \RR^n}{\Phi x = \Phi \bar{x}} .
  \end{equation*}
  Hence, $\Xl$ is a polyhedron. Since $\Xl$ is a bounded set, it is also a polytope.
\end{proof}

Owing to Proposition~\ref{prop:polysol}, we can rewrite the set $\Xl$ as the convex hull of $k$ points in $\RR^n$ as
\begin{equation*}
  \Xl = \conv{a_1, \dots, a_k} ,
\end{equation*}
where $a_i$ are the extremal points of $\Xl$.
Observe that each $a_i$ lives on the boundary of the analysis-$\lun$-ball of radius $\normu{D^* \bar{x}}$.
Naturally, we can even rewrite the solution as
\begin{equation*}
  \Xl = A \Delta_k = \enscond{A z}{z \in \Delta_k} ,
\end{equation*}
where $A$ is a matrix $n \times k$ such that its columns are the vectors
$a_i$ and the $n$-simplex $\Delta_n$ of $\RR^n$ is defined as
\begin{equation*}
  \Delta_n = \enscond{x \in \RR^n}{\sum_{i=1}^n x_i = 1 \qandq \forall i, x_i \geq 0} = \conv{e_1,\dots,e_n} ,
\end{equation*}
where $(e_1,\dots,e_n)$ is the canonical basis of $\RR^n$.
Since $a_i$ are the extremal points of $\Xl$, notice that $A$ has maximal rank.
Observe in particular that the lines of the matrix $D^* A$ have same signs according to Proposition~\ref{prop:sign}.