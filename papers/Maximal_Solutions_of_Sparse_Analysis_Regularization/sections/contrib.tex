\section{Contributions}
\label{sec:contrib}

In~\cref{sec:sol}, we review some properties of the solution set.
In all this paper, \textbf{we consider the following hypothesis of restricted injectivity}
\begin{equation}\label{eq:hyp-inv}
  \Ker D^* \cap \Ker \Phi = \ens{0} ,
\end{equation}
in order to ensure that $\Xl$ is well-defined and bounded.
We prove in particular that $\Xl$ is a polytope, i.e. a bounded polyhedron.

Our main contribution is proved in~\cref{sec:max}.
It consist in providing a geometrical interpretation of a solution with a maximal $D$-support, namely the fact that such a solution lives in the relative interior of the solution set.
More precisely, we are concerned with the characterization of a vector of maximal $D$-support, i.e. a solution of~\eqref{eq:p} such that for every $x \in \Xl, \normz{D^* x} \leq \normz{D^* \maxx}$.
\begin{definition}
  A vector $\maxx \in \RR^n$ is \emph{a solution of maximal $D$-support} if $\maxx$ is a solution, i.e. $\maxx \in \Xl$ such that for every $x \in \Xl, \normz{D^* x} \leq \normz{D^* \maxx}$.
\end{definition}
We denote by $\Sl$ the set of solution of~\eqref{eq:p} which have maximal $D$-support.
Clearly this set is well-defined and contained in $\Xl$.
Our result is the following.
\begin{theorem}\label{thm:maximal-characterization}
  Let $\bar{x} \in \Xl$. Then $\bar{x}$ is a maximally $D$-supported solution if, and only if, $\bar{x} \in \rint \Xl$ (or equivalently if $\bar{x} \in \rint \Sl$).
  In other words, 
  \begin{equation*}
    \Sl = \rint \Sl = \rint \Xl .
  \end{equation*}
\end{theorem}
We recall that for any set $S$, the relative interior $\rint S$ of $S$ is
defined as its interior with respecto to the topology of the affine hull of $S$.

With this result in hand, we provide a way to construct such maximal solutions.
In \cref{sec:finding}, we show that with the help of a technical penalization using the so-called concave gauge ~\cite{barbara2015strict}, we can construct a path which converges to a point in the relative interior of $\Xl$, and more specifically, to the analytic center with respect to the chosen gauge.
We defer the precise statement to~\cref{sec:finding}.