\section{Maximal support and proof of \cref{thm:maximal-characterization}}
\label{sec:max}

We recall that a vector $\maxx \in \RR^n$ is a solution of maximal $D$-support if $\maxx$ is a solution, i.e., $\maxx \in \Xl$ such that for every $x \in \Xl, \normz{D^* x} \leq \normz{D^* \maxx}$.
The following proposition proves that \emph{the} $D$-maximal support is indeed
uniquely defined.
\begin{proposition}
  Let $x \in \Xl$.
  Then the two following propositions are equivalent.
  \begin{enumerate}
  \item $x$ is a solution of maximal $D$-support, i.e. $x \in \Sl$.
  \item For any $\bar{x} \in \Xl$, $\supp(D^* \bar{x}) \subseteq \supp(D^* x)$.
  \end{enumerate}
\end{proposition}
\begin{proof}
  The two directions are proved separately.\\
  $(1) \Rightarrow (2)$.
  Suppose there exists $i_0 \in \ens{1,\dots,p}$ such that $i_0 \in \supp(D^* \bar{x})$ and $i_0 \not\in \supp(D^* x)$.
  Observe that $\tilde x = \frac{1}{2}(\bar{x} + x)$ is also an element of $\Xl$ by convexity of $\Xl$.
  Using Proposition~\ref{prop:sign}, we get that $\supp(D^* \tilde x) \supseteq \supp(D^* \bar{x}) \cup \supp(D^* x)$.
  In particular, $\supp(D^* \tilde x) \supseteq \supp(D^* x) \cup \ens{i_0} \supsetneq \supp(D^* x)$.
  Hence, $\abs{\supp(D^* \tilde x)} > \abs{\supp(D^* x)}$ which contradicts the fact that $x$ has maximal $D$-support.\\
  $(2) \Rightarrow (1)$.
  Taking the cardinal in the property $\forall \bar{x} \in \Xl$, $\supp(D^* \bar{x}) \subseteq \supp(D^* x)$ is sufficient.
\end{proof}
In particular, two solutions of maximal support share the same $D$-support. Notice that in this case, the sign vectors are also the same.

We start by a technical Corollary of Proposition~\ref{prop:sign} which will be convenient in the following.
\begin{corollary}\label{cor:diagpos}
  There exists an integer $m \in \NN$, a matrix $\Lambda = \diag(\lambda_i)_{i=1,\dots,p}$ with $\lambda_i \in \ens{-1,1}$ for $i \in \ens{1,\dots,m}$ and $\lambda_i = 0$ for $i \in \ens{m+1,\dots,p}$, and a permutation matrix $\Sigma$ such that for $\Gamma = \Lambda \Sigma$, one has
  \begin{equation*}
    \Gamma D^* \Xl \subset (\RR_+)^m \times \ens{0}^{p-m} .
  \end{equation*}
  Moreover, for all $x \in \Xl$, $\normu{\Gamma D^* x} = \normu{D^* x}$.
\end{corollary}
\begin{proof}
  Let $\maxx$ an element of $\Sl$.
  Consider $I = \supp(D^* \maxx)$, $J = I^c$ and $m = \abs{I}$.
  Let $\Sigma$ be the permutation matrix associated to any permutation $\sigma$ which sends $I$ to $\ens{1,\dots,m}$.
  Define the matrix $\Lambda$ by its diagonal as
  \begin{equation*}
    \lambda_{\sigma(i)} = 
    \begin{cases}
      1 & \text{if } (D^* \maxx)_{\sigma(i)} > 0 \\
      -1 & \text{if } (D^* \maxx)_{\sigma(i)} < 0 \\
      0 & \text{if } (D^* \maxx)_{\sigma(i)} = 0 .
    \end{cases}
  \end{equation*}

  Now take any solution $x \in \Xl$ and consider the vector $u = \Gamma D^* x$.
  Let $i \in \ens{1,\dots,m}$, then
  \begin{equation*}
    u_i = \dotp{e_i}{\Lambda \Sigma D^* x} .
  \end{equation*}
  Since $\Lambda$ is self-adjoint, one has
  \begin{equation*}
    u_i = \dotp{\Lambda e_i}{\Sigma D^* x} .
  \end{equation*}
  Since $\Lambda$ is a diagonal matrix, we get that
  \begin{equation*}
    u_i = \lambda_i \dotp{e_i}{\Sigma D^* x} .
  \end{equation*}
  Now, since $\Sigma$ is a permutation matrix, we have that $\Sigma^* = \Sigma^{-1}$, i.e.
  \begin{equation*}
    u_i = \lambda_i \dotp{\Sigma^{-1} e_i}{D^* x} .
  \end{equation*}
  Using the permutation $\sigma$ associated to $\Sigma$, we have that
  \begin{equation*}
    u_i = \lambda_i \dotp{e_{\sigma^{-1}(i)}}{D^* x} ,
  \end{equation*}
  which can be rewritten as
  \begin{equation*}
    u_i = \lambda_i \dotp{d_{\sigma^{-1}(i)}}{x} .
  \end{equation*}  
  According to Proposition~\ref{prop:sign}, one have $(D^* x)_{\sigma^{-1}(i)} (D^* \maxx)_{\sigma^{-1}(i)} \geq 0$.
  Moreover, $\lambda_i = \lambda_{\sigma(\sigma^{-1}(i))}$ has the same sign than $(D^* \maxx)_{\sigma^{-1}(i)}$.
  Thus, $u_i = \lambda_i \dotp{d_{\sigma^{-1}(i)}}{x} \geq 0$.

  For $i \in \ens{m+1,\dots,p}$, we have that
  \begin{equation*}
    u_i = \lambda_i \dotp{e_i}{\Sigma D^* x} = 0,
  \end{equation*}
  since $\lambda_i = 0$.
\end{proof}
Note that the matrix $\Lambda$ and $\Sigma$ are not uniquely defined. Corollary~\ref{cor:diagpos} allows us to work only on positive vectors in dimension $m$.

We will also need to exclude at some point the case where a solution $x$ lives in the kernel of $D^*$.
The following lemma shows that if this is the case, then the solution set is reduced to a singleton $\Xl = \ens{x}$.
\begin{lemma}\label{lem:kernel-one-image}
  If there exists $x \in \Ker D^* \cap \Xl$, then $\Xl = \ens{x}$ .
\end{lemma}
\begin{proof}
  We recall that $\Xl \subset x + \Ker \Phi$.
  Let $\bar x \in \Xl$, and rewrite it as $\bar x = x + h$ where $h \in \Ker \Phi$.
  Then, according to Proposition~\ref{lem:same-image}, one has $\normu{D^* \bar x} = \normu{D^* x} = 0$.
  In particular, $\normu{D^* \bar x} = \normu{D^* x + D^* h} =  \normu{D^* h} = 0$.
  Using hypothesis~\eqref{eq:hyp-inv}, we get that $h = 0$.
\end{proof}

We can now provide the proof of Theorem~\ref{thm:maximal-characterization}.
\begin{proof}[Proof of Theorem~\ref{thm:maximal-characterization}]
  We exclude here the case where $\Xl$ is reduced to a singleton, since the result is then trivially verified.
  Let us prove both direction separately.

  $(\Leftarrow: \rint \Xl \subseteq \Sl)$.
  First, we recall that $\rint \Xl = \rint (A \Delta_k) = A \rint \Delta_k$.
  Let $\bar{x} \in \rint \Xl$.
  We have
  \begin{equation*}
    \bar{x} = A \bar{z} \qwithq \sum_{i=1}^k \bar{z}_i = 1 \qandq \bar{z}_i > 0.
  \end{equation*}
  For $i \in \ens{1,\dots,m}$, one has
  \begin{equation*}
    (\Gamma D^* \bar{x})_i = (\Gamma D^* A \bar{z})_i = \dotp{e_i}{\Gamma D^* A \bar{z}} =  \dotp{e_i}{\Lambda \Sigma D^* A \bar{z}}.
  \end{equation*}
  Using the fact that $\Lambda$ is a diagonal matrix and $\Sigma$ is a permutation matrix, we have that
  \begin{equation*}
    (\Gamma D^* \bar{x})_i = \lambda_i \dotp{D \Sigma^{-1} e_i}{A \bar{z}} ,
  \end{equation*}
  which can be rewritten, using the fact that $\Sigma^{-1} e_i = e_{\sigma^{-1}(i)}$ where $\sigma$ is the permutation associated to $\Sigma$, as
  \begin{equation*}
    (\Gamma D^* \bar{x})_i = \lambda_i \dotp{d_{\sigma^{-1}(i)}}{A \bar{z}} .
  \end{equation*}
  Now, one can rewrite it as
  \begin{equation*}
    (\Gamma D^* \bar{x})_i = \lambda_i \dotp{(D^* A)^* e_{\sigma^{-1}(i)}}{\bar{z}} .
  \end{equation*}
  Since for any $i$, $\bar{z}_i > 0$ and, according to Proposition~\ref{prop:sign}, there exists $j_0$ such that $((D^* A)^* e_{\sigma^{-1}(i)})_{j_0} > 0$, one concludes that $(\Gamma D^* \bar{x})_i \neq 0$.

  $(\Rightarrow: \Sl \subseteq \rint \Xl)$.
  We are going to prove that $\Sl = \rint \Sl$.
  Indeed, according to $(\Leftarrow)$, $\rint \Xl \subseteq \Sl$.
  Moreover, since every element of $\Sl$ is also an element of $\Xl$, we have $\rint \Xl \subseteq \Sl \subseteq \Xl$.
  In particular, $\aff \Xl = \aff \Sl$.
  Let
  \begin{equation*}
    \alpha = \min_{i \in \supp(D^* \maxx)} \abs{(D^* \maxx)_i} = \min_{i \in \ens{1,\dots,m}} (\Gamma D^* \maxx)_i
  \end{equation*}
  where $\maxx$ is an element of $\Sl$.
  Note that according to Lemma~\ref{lem:kernel-one-image}, since $\Xl$ is not reduced to a singleton, then $\supp(D^* \maxx)$ has cardinal greater than 1, hence $\alpha > 0$.

  Now take any $u \in B_\infty(\maxx, r) \cap \aff \Xl$ where
  \begin{equation*}
    r = \frac{\alpha - \epsilon}{\norm{\Gamma D^*}_{\infty,\infty}},
  \end{equation*}
  and $0 < \epsilon < \alpha$.

  Let's prove first that $\Gamma D^* u \in (\RR_+^*)^m \times \ens{0}^{p-m}$.
  From the definition of $u$, we get that
  \begin{equation*}
    \normi{\Gamma D^* u - \Gamma D^* x} \leq \norm{\Gamma D^*}_{\infty,\infty} \normi{u - x} \leq \alpha - \epsilon .
  \end{equation*}
  For $i \in \ens{1,\dots,m}$, one has $\abs{(\Gamma D^* u)_i - (\Gamma D^* x)_i} \leq \alpha - \epsilon$. In particular one has
  \begin{equation*}
    (\Gamma D^* u)_i - (\Gamma D^* x)_i \geq -\alpha + \epsilon \Leftrightarrow (\Gamma D^* u)_i \geq (\Gamma D^* x)_i - \alpha + \epsilon .
  \end{equation*}
  Since $(\Gamma D^* x)_i - \alpha \geq 0$ and $\epsilon > 0$, we conclude that $(\Gamma D^* u)_i > 0$.
  Thus, $(\Gamma D^* u)_i > 0$ for $i \in \ens{1,\dots,m}$ and $(\Gamma D^* u)_i = 0$ for $i \not\in \ens{1,\dots,m}$.

  It remains to prove that $u$ is a solution of~\eqref{eq:p}, i.e. $u \in \Xl$.
  Since $u \in \aff \Xl$, there exists $t \in \RR$ and $x \in \Xl$ such that
  \begin{equation*}
    u = \maxx + t (x - \maxx) .
  \end{equation*}
  From this equality, we get that
  \begin{align*}
    \normu{D^* u} &= \normu{\Gamma D^* u} = \sum_{i=1}^p (\Gamma D^* u)_i && \text{according to Corollary~\ref{cor:diagpos}}  \\
                  &= \sum_{i=1}^p (1-t) (\Gamma D^* \maxx)_i + t (\Gamma D^* x)_i && \\
                  &= (1-t) \normu{\Gamma D^* \maxx} + t \normu{\Gamma D^* x} && \\
                  &= \normu{D^* \maxx} && \text{since } \normu{D^* \maxx} = \normu{D^* x} .
  \end{align*}
  Moreover, $\Phi u = \Phi \maxx + t(\Phi x - \Phi \maxx) = \Phi \maxx$.
  Thus, $u$ is a solution which concludes our proof.
\end{proof}
