%24. April 2008
%\documentclass{article}
%\documentclass[twoside]{article}
\documentclass[3p,times,twocolumn]{elsarticle}
 \biboptions{comma,sort&compress}
 
\usepackage{graphicx}
%\usepackage{amsmath}
\usepackage{here}
%% The `ecrc' package must be called to make the CRC functionality available
\usepackage{ecrc}

%% The ecrc package defines commands needed for running heads and logos.
%% For running heads, you can set the journal name, the volume, the starting page and the authors
\def\be{\begin{equation}}
\def\ee{\end{equation}}
\def\ba{\begin{eqnarray}}
\def\ea{\end{eqnarray}}
\def\bea{\begin{eqnarray}}
\def\eea{\end{eqnarray}}
\def\bes{\begin{subequations}}
\def\ees{\end{subequations}}
\newcommand{\A}{{\mathcal{A}}}
%% set the volume if you know. Otherwise `00'
\volume{00}

%% set the starting page if not 1
\firstpage{1}

%% Give the name of the journal
\journalname{Nuclear and Particle Physics Proceedings}

%% Give the author list to appear in the running head
%% Example \runauth{C.V. Radhakrishnan et al.}
\runauth{}

%% The choice of journal logo is determined by the \jid and \jnltitlelogo commands.
%% A user-supplied logo with the name <\jid>logo.pdf will be inserted if present.
%% e.g. if \jid{yspmi} the system will look for a file yspmilogo.pdf
%% Otherwise the content of \jnltitlelogo will be set between horizontal lines as a default logo

%% Give the abbreviation of the Journal.
\jid{nppp}

%% Give a short journal name for the dummy logo (if needed)
\jnltitlelogo{Nuclear and Particle Physics Proceedings}

%% Hereafter the template follows `elsarticle'.
%% For more details see the existing template files elsarticle-template-harv.tex and elsarticle-template-num.tex.

%% Elsevier CRC generally uses a numbered reference style
%% For this, the conventions of elsarticle-template-num.tex should be followed (included below)
%% If using BibTeX, use the style file elsarticle-num.bst

%% End of ecrc-specific commands
%%%%%%%%%%%%%%%%%%%%%%%%%%%%%%%%%%%%%%%%%%%%%%%%%%%%%%%%%%%%%%%%%%%%%%%%%%

%% The amssymb package provides various useful mathematical symbols
\usepackage{amssymb}
%% The amsthm package provides extended theorem environments
%% \usepackage{amsthm}

%% The lineno packages adds line numbers. Start line numbering with
%% \begin{linenumbers}, end it with \end{linenumbers}. Or switch it on
%% for the whole article with \linenumbers after \end{frontmatter}.
%% \usepackage{lineno}

%% natbib.sty is loaded by default. However, natbib options can be
%% provided with \biboptions{...} command. Following options are
%% valid:

%%   round  -  round parentheses are used (default)
%%   square -  square brackets are used   [option]
%%   curly  -  curly braces are used      {option}
%%   angle  -  angle brackets are used    <option>
%%   semicolon  -  multiple citations separated by semi-colon
%%   colon  - same as semicolon, an earlier confusion
%%   comma  -  separated by comma
%%   numbers-  selects numerical citations
%%   super  -  numerical citations as superscripts
%%   sort   -  sorts multiple citations according to order in ref. list
%%   sort&compress   -  like sort, but also compresses numerical citations
%%   compress - compresses without sorting
%%
%% \biboptions{comma,round}

% \biboptions{}

% if you have landscape tables
\usepackage[figuresright]{rotating}

% put your own definitions here:
%   \newcommand{\cZ}{\cal{Z}}
%   \newtheorem{def}{Definition}[section]
%   ...

% add words to TeX's hyphenation exception list
%\hyphenation{author another created financial paper re-commend-ed Post-Script}

% declarations for front matter

\begin{document}

\begin{frontmatter}

%%
%%%%%%%%%%%%%%%%%%%%%%%%%%%%%%%%%%%%%%%%%%%%%%%%%
\title{QCD coupling which respects lattice restrictions at low energies
 $^*$}
 % \corref{cor0}}
 \cortext[cor0]{Talk given at 18th International Conference in Quantum Chromodynamics (QCD 17,  20th anniversary),  3 - 7 july 2017, Montpellier - FR}
 \author[label1]{C\'esar Ayala\fnref{fn1}}
   \fntext[fn1]{Speaker, Corresponding author.}
%  \cortext[cor0]{FAPESP CNPq-Brasil PhD student fellow.}
\ead{cesar.ayala@usm.cl}
\address[label1]{Department of Physics, Universidad T{\'e}cnica Federico Santa Mar{\'\i}a, Casilla 110-V, Valpara{\'\i}so, Chile\\ }

\pagestyle{myheadings}
\markright{ }
\begin{abstract}
We consider a phenomenologycal parametrization of the QCD running coupling which 
arises from the dispersion relation respecting the holomorphic properties of the physical 
QCD observables in the complex momentum plane. The parameters are fixed by the following requirements: 1) at enough high energies, it reproduces the underlying perturbative coupling, 2) at intermediate energy momenta, it reproduces the experimental semihadronic tau decay ratio, 
and 3) in the deep IR regime, it satisfies the qualitative properties coming from recent lattice results. 
Finally, we apply this new coupling to low-energy available experimental data. In particular, to Borel sum rules for $\tau$-decay, extracting the values of the dimension 4 and 6 condensates, to the V-channel Adler function, and to polarized Bjorken Sum Rule.
 
\end{abstract}
% \begin{document}
\begin{keyword}  
%% keywords here, in the form: keyword \sep keyword
Perturbative QCD \sep Lattice QCD \sep QCD Phenomenology \sep Resummation
%% MSC codes here, in the form: \MSC code \sep code
%% or \MSC[2008] code \sep code (2000 is the default)

\end{keyword}

\end{frontmatter}
%%%%%%%%%%%%
%\vspace*{-1.5cm}
\section{The method: Constructing the Holomorphic Coupling}

We present a generalization/extension of the perturbative QCD running coupling 
under the assumption that it has a physical branch on the negative semiaxes of the 
$Q^2$-complex momenta plane, elsewhere it is a holomorphic function of $Q^2$. 
On the other hand, this coupling should satisfy the asymptotic freedom. These assumptions can be implemented via dispersion relation with the application of the Cauchy theorem to the integrand  $\A(Q'^2)/(Q'^2 - Q^2)$, i. e. 
\be
\A(Q^2) = \frac{1}{\pi} \int_{\sigma=M^2_{\rm thr}-\eta}^{\infty} \frac{d \sigma \rho_{\A}(\sigma)}{(\sigma + Q^2)} 
\qquad (\eta \to +0),
\label{Adisp}
\ee
where $\rho_{\A}(\sigma) \equiv {\rm Im} \mathcal{A}(-\sigma - i \varepsilon)$ is the discontinuity function (spectral function) of $\A$ along the cut. 

For different choices of $M_{\rm thr}^2>0$, we recover different known approaches. Between the most known are Fractional Analytic Perturbation Theory (FAPT) \cite{APT,BMS}; Massive Perturbation Theory (MPT) \cite{ShMPT} and $M\delta$ analytic QCD ($M\delta$anQCD) \cite{2danQCD,3danQCD1,3danQCD2}. In these  models, the threshold squared mass is 
\begin{equation}
M_{\rm thr}^2 = \left\{ \begin{array}{cl} 
-\Lambda_{\rm QCD}^2& ,{\rm pQCD}\\ 0& ,{\rm (F)APT}\\ 
m_{gl}^2-\Lambda_{\rm QCD}^2&, {\rm MPT} \\ \sim m_\pi^2&, M\delta {\rm anQCD}  \end{array}\right.
\end{equation}
Note that in pQCD and (F)APT $\rho_{\A}(\sigma) = \rho_{a}(\sigma)$, and MPT is defined from $\A_{\rm MPT}\equiv a(Q^2+m_{gl}^2)$. Here $a(Q^2)=\alpha_s(Q^2)/\pi$
In this work we will present and use a new coupling for the last case, i.e., 3$\delta$anQCD \cite{3danQCD1,3danQCD2}.

We define the unknown low-energy part of the integral (\ref{Adisp}) in the range $M_{\rm thr}^2<\sigma<M_0^2$ ($\sim1$ GeV$^2$) as $\Delta \A_{\rm IR}(Q^2)$. This integral has the same structure as a Stieltjes function \cite{Baker}. Then we can use a theorem that guarantees the convergence of a sequence of Pad\'es $[M-1/M]$ to $\Delta \A_{\rm IR}(Q^2)$ as $M\to\infty$. $[M-1/M]$ is a polynomial in $Q^2$ of power $M-1$ divided by a polynomial of power $M$. Therefore,

%The integral in Eq.~(\ref{Adisp}) has the same structure as a Stieltjes function \cite{Baker}, 
%and then we can use a theorem that guarantees the convergence of unknown low energy region (let's take from $M_{\rm thr}^2$ up to certain $M_0^2$ ($\sim1$ GeV$^2$) scale to be fixed by phenomenology) parametrizing with 
%Pad\'e of the type $[M-1/M]$, i.e., as a ratio of a polynomial in $Q^2$ of power $M-1$ divided by a polynomial of power $M$: 
\bea
\Delta \A_{\rm IR}(Q^2) &\equiv&  \frac{1}{\pi} \int_{\sigma=M_{\rm thr}^2}^{M_0^2} \frac{d \sigma \rho_{\A}(\sigma)}{(\sigma + Q^2)} 
\label{M1M}
\nonumber\\
&=&  \sum_{j=1}^{M} \frac{{\cal F}_j}{Q^2 + M_j^2} \ .
\label{PFM1M}
\eea
And for $\sigma$ from $M_0^2$ to infinity, we recover the perturbative discontinuity $\rho_a(\sigma)$. That corresponds to 
\be
\rho_{\A}(\sigma) =  \pi \sum_{j=1}^{M} {\cal F}_j \; \delta(\sigma - M_j^2)  + \Theta(\sigma - M_0^2) \rho_a(\sigma) \ ,
\label{rhoA}
\ee
where $\Theta$ is the Heaviside step function. Then, the considered coupling $\A(Q^2)$ is parametrized as 
\be
\A(Q^2)=\sum_{j=1}^M \frac{{\cal F}_j}{(Q^2 + M_j^2)} + \frac{1}{\pi} \int_{M_0^2}^{\infty} d \sigma \frac{ \rho_a(\sigma) }{(Q^2 + \sigma)} \ .
\label{AQ2}
\ee
The coupling (\ref{AQ2}) has $2M+1$ free parameters ${\cal F}_j$, $M^2_j$ ($j=1,2,\ldots ,M$) and $M_0^2$.

In order to have a good estimation of the running coupling, the question is: how many delta functions are appropriate (sufficient) for reproduce the physics at $Q^2\lesssim 1$GeV$^2$?.

Before we answer this, let us show the main properties that a possible candidate for a new universal coupling should have:
\begin{enumerate}[(i)]
\item \label{HE} Reproduce the high-energy QCD phenomenology as obtained from perturbation theory. This requirement can be written as 
\be
\A(Q^2) - a(Q^2)   \sim \left( \frac{\Lambda^2_{\rm QCD}}{Q^2} \right)^{N_{\rm max}} \ ,
\label{Aadiff1}
\ee
for $|Q^2| > \Lambda^2_{\rm QCD}$, $N_{\rm max}>1$ sufficiently large, and 
\be
\A(M_Z^2)=\frac{\alpha_s(M_Z^2)}{\pi}=``{\rm world\ average}''
\label{AsymFree}
\ee
This is obtained from the world average value in the $\overline{\rm MS}$-scheme, e. g. 
$\alpha_s(M_Z^2,\overline{\rm MS})\approx0.1185$ \cite{PDG}. In Eq.~(\ref{AsymFree}) we should change the scheme according to our needs. 
\item \label{IE} Reproduce the intermediate-energy QCD phenomenology, by requiring that the main features of the semihadronic $\tau$-lepton decay physics be respected. Stated otherwise, we will require that the approach with the coupling $\A(Q^2)$ reproduce the experimentally suggested value of the V+A semihadronic $\tau$-decay ratio parameter $r^{(D=0)}_{\tau} \approx 0.20$ \cite{ALEPH2,DDHMZ}. This is the QCD part of the V+A $\tau$-decay ratio $R_{\tau} = \Gamma(\tau^- \to \nu_{\tau}{\rm hadrons}(\gamma))/\Gamma(\tau^- \to \nu_{\tau} e^- {\bar \nu}_e (\gamma))$, where the hadrons are strangeless ($\Delta S=0$) and the quark mass effects and other (small) higher-twist effects are subtracted, i.e., it is the dimension $D=0$ strangeless and massless part.
\item \label{LE} Satisfy some qualitative and/or quantitative properties of the coupling in the deep-IR region when $Q^2\to0$. In general, we have three different possibilities inspired by different physical/mathematical evidence. These are: IR-finite coupling (freezing); infinite effective coupling that reproduces confinement already at one loop level, and vanishing coupling inspired by lattice simulations. In this report, we will consider the last case, where the coupling should behave as $\A(Q^2)\sim Q^2$ at $Q^2\to0$.
\end{enumerate}


\section{Phenomenology: Fixing Parameters}

Now, we should take some decisions. The first (high-energy) condition (\ref{HE}) implies fixing the precision with respect to the underlying pQCD coupling, i.e., while $N_{\rm max}$ increases our precision increases too. We will use $N_{\rm max}=5$, which imply the following four equations (for the elimination of four free parameters). 
\be
\frac{1}{\pi} \int_{-\Lambda_{\rm QCD}^2}^{M_0^2} d \sigma \sigma^k \rho_a(\sigma)=\sum_{j=1}^3 {\cal F}_j M_j^{2 k}\ ,
\label{1u}
\ee
with $k=0,1,2,3$. The world average value will fix our $\Lambda_{\rm QCD}$ scale or equivalently, the underlying pQCD coupling $a(Q^2)$ and thus $\rho_a(\sigma)$.

The second (intermediate-energy) condition (\ref{IE}) will fix us one free parameter by the  semihadronic $\tau$-lepton decay physics.
The considered quantity $r^{(D=0)}_{\tau}$ is timelike, but it can be expressed theoretically, 
by using the Cauchy integral formula, by means of a spacelike quantity called (leading-twist and massless) Adler function $d(Q^2;D=0)$ \cite{Braaten,PichPra}:
\be
r^{(D=0)}_{\tau, {\rm th}} = \frac{1}{2 \pi} \int_{-\pi}^{+ \pi}
d \phi \ (1 + e^{i \phi})^3 (1 - e^{i \phi}) \
d(m_{\tau}^2 e^{i \phi};0) \ .
\label{rtaucont}
\ee
The Adler function $d(Q^2;D=0)$ is a derivative of the quark current correlator $\Pi$: $d(Q^2;D=0) = - 2 \pi^2 d \Pi(Q^2; D=0)/d \ln Q^2 - 1$, in the massless limit. Its perturbation expansion is known up to $\sim a^4$ \cite{d3} and rewritten in terms of the new coupling. The expansion in terms of the holomorphic coupling is different from the perturbative one due to nonperturbative nature of the theory, i. e., the analogs of the pQCD powers $a(Q^2)^n$ are specific functions $\A_n(Q^2)$ [$\not= \A(Q^2)^n$]
\bea
\lefteqn{d(Q^2;D=0)\equiv d(Q^2,\mu^2;D=0)_{\rm an}^{[4]} + {\cal O}(\A_5)}
\nonumber\\
& & = \A(Q^2)+d_1 \A_{2}(Q^2)+d_2 \A_{3}(Q^2)
\nonumber\\
&& + \ d_3 \A_{4}(Q^2) + {\cal O}(\A_5).
\label{dan}
\eea
The power analogs $\A_n(Q^2)$ from $\A(Q^2)$($=\A_1(Q^2)$) were constructed in general holomorphic theories from $\A(Q^2)$ by using renormalization group equations (RGE) Ref.~\cite{CV12} for integer $n$ and in Ref.~\cite{GCAK} for general real $n$.

The third (low-energy) condition (\ref{LE}) depends on what approach we will consider. We will take in this regime the information from the lattice simulations~\cite{LattcoupNf0} of the Landau gauge gluon $Z_{\rm gl}(Q^2)$ and ghost $Z_{\rm gh}(Q^2)$ dressing functions. These simulations were performed with large physical volume and high statistics, giving presumably reliable results in the low-momentum regime $0<Q^2 < 1 \ {\rm GeV}^2$. Then, we can obtain the lattice version of the coupling as 
\be
\A_{\rm latt.}(Q^2)  =  \A_{\rm latt.}(\Lambda^2)  \frac{Z_{\rm gl}^{(\Lambda)}(Q^2) Z_{\rm gh}^{(\Lambda)}(Q^2)^2}{{\widetilde Z}_1 ^{(\Lambda)}(Q^2)^2} \ ,
\label{Alatt}
\ee
where the value of the gluon-ghost-ghost vertex function is ${\widetilde Z}_1 ^{(\Lambda)}(Q^2)^2=1$ in the Landau gauge, and the UV cutoff squared $\Lambda$ is determined by the lattice spacing. The resulting lattice coupling (\ref{Alatt}) has two interesting features: it goes to zero as $\A_{\rm latt.}(Q^2) \sim Q^2$ when $Q^2 \to 0$ and has a a maximum at $Q_{\rm max}^2 \approx 0.135 \ {\rm GeV}^2$. These two properties will fix us two parameters.

Altogether we can adjust seven parameters of our coupling  (\ref{AQ2}), where four come from high energy, one from  intermediate and two from low-energy regime. This is equivalent to taking three delta functions in (\ref{rhoA}).

%Before to continue, we are going to tell some words about the scheme in which our coupling is working. 
For practical implementation, we need the underlying pQCD coupling $a(Q^2)$, and thus $\rho_a(\sigma)$ in an explicit form if we want evaluate the integral in Eq.~(\ref{AQ2}). It is given by solving the $\beta$-function for a specific Pad\'e form \cite{GCIK} whose expansion gives the known MiniMOM coefficients $c_2({\rm MM},N_f=3)=9.2970$ and $c_3({\rm MM},N_f=3)=71.4538$ [the expansion of this Pad\'e $\beta$-function up to $\sim a(Q^2)^5$ reproduces the four-loop polynomial $\beta$-function]. This coupling involves Lambert function which can be easily implemented in Mathematica software. 
%will have the usual $\Lambda_{\rm QCD}$-dependence but in this new ``Lambert form'' $\Lambda_L$ and we can relate both via numerical solution of the exact $\beta$-function and this Lambert form. 
When comparing with lattice results, we must take into account the following relation between lattice MiniMOM (MM), \cite{MiniMOM} and $\overline{\rm MS}$-scheme scale convention
\bea
\frac{\Lambda_{\rm MM}}{\Lambda_{\overline{\rm MS}}} &=& 1.8968 \; ({\rm for} \; N_f=0); 
\nonumber\\
&=&1.8171 \; ({\rm for} \; N_f=3);
\label{LambdaMMMSbar}
\eea

\begin{table*}[hbt]
% space before first and after last column: 1.5pc
% space between columns: 3.0pc (twice the above)
\setlength{\tabcolsep}{1.5pc}
% -----------------------------------------------------
% adapted from TeX book, p. 241
\newlength{\digitwidth} \settowidth{\digitwidth}{\rm 0}
\catcode`?=\active \def?{\kern\digitwidth}
% -----------------------------------------------------
\caption{The seven parameters of the coupling $\A(Q^2)$: $M_j^2$ ($j=0,1,2,3$); ${\cal F}_j$ ($j=1,2,3$), both in [GeV$^2$]. These values are given for the representative case: $r^{(D=0)}_{\tau, {\rm th}}=0.201$ and $0.201 \pm 0.002$;  with $\alpha_s(M_Z^2;\overline{\rm MS}) = 0.1185 \pm 0.004$.}
\label{tab:effluents}
\begin{tabular*}{\textwidth}{llllllll}
\hline
$r^{(D=0)}_{\tau, {\rm th}}$ & $M_0^2$ & $M_1^2$  & $M_2^2$ & $M_3^2$ & ${\cal F}_1$ & ${\cal F}_2$  & ${\cal F}_3$\\
  \hline
$0.201$ & $8.719$ & $0.05$ & $0.247$ & $6.34$ & $-0.038$ & $0.158$  & $0.070$\\
$0.203$ & $9.254$ & $0.04$ & $0.329$ & $6.75$ & $-0.020$ & $0.143$  & $0.073$\\
$0.199$ & $8.211$ & $0.10$ & $0.143$ & $5.96$ & $-0.245$ & $0.361$  & $0.068$\\
  \hline
   \hline

\end{tabular*}
\end{table*}
\addtocounter{table}{-1}

In Table~\ref{tab:effluents} we present our results for the free parameters that fulfills the  three conditions (\ref{HE}), (\ref{IE}) and (\ref{LE}).

\begin{figure}[htb]
\vspace{9pt}
\centering\includegraphics[width=70mm]{FigComb4lrt0201al01185.pdf}
%\framebox[55mm]{\rule[-21mm]{0mm}{43mm}}
\caption{The points represent the data obtained for the quenched lattice coupling from Ref.~\cite{LattcoupNf0} with their corresponding uncertainties. And the solid line our new coupling (\ref{AQ2}) with parameters given in Table \ref{tab:effluents}. We relate momenta in the MiniMOM (MM) lattice scheme to the usual ${\overline{\rm MS}}$-like scale \cite{3danQCD1,3danQCD2}.}
\label{fig:largenenough}
\end{figure}
%

In Fig.~\ref{fig:largenenough} we show the obtained ($N_f=3$) running coupling $\pi \A(Q^2)$ by solid the line, and the lattice $N_f=0$ calculations by points. In general our coupling agrees well with $\pi \A_{\rm latt.}(Q^2)$ at very low $Q^2$ ($Q \lesssim 0.01 \ {\rm GeV}^2$), and is lower than the lattice coupling near the maximum ($Q^2 \sim 0.1 \ {\rm GeV}^2$). We recall that we do not expect to have a good agreement between the theoretical and lattice coupling at $Q^2 \lesssim 0.1 \ {\rm GeV}^2$, but only a qualitative agreement. Even at higher $Q^2$ ($Q^2 > 1 \ {\rm GeV}^2$), there is a difference between $\pi \A(Q^2)$ and $\pi \A_{\rm latt.}(Q^2)$ of the higher-twist form $\sim\Lambda_{\rm QCD}^2$, and because we are working with $N_f=3$ while the lattice results \cite{LattcoupNf0} are for $N_f=0$. In fact, increasing $N_f$ in general decreases $\A_{\rm latt.}(Q^2)$, cf.~Fig.~5 of Ref.~\cite{LattcoupNf2}. Further, the lattice results concentrate on the deep IR regime, i.e., they had large lattice volume ($L \sim 10$ fm), but not small lattice spacing, which makes the lattice results \cite{LattcoupNf0,LattcoupNf2} unreliable at $Q^2\gtrsim1$GeV$^2$
 
\section{Applications}

In the present Section we will apply our coupling (\ref{AQ2}) with the corresponding parameters given in Table~\ref{tab:effluents} to some low-energy processes. 
Due to the condition (\ref{HE}), we can use OPE with $\A$-coupling in a way analogous to the OPE with pQCD $a$-coupling. In particular, due to Eq.(\ref{Aadiff1}) for $N_{\rm max}=5$, we can include in OPE with $\A$-coupling unambiguously the terms of dimensionality $D < 10$. The relevant programs in the implementation of this machinery are available and described online in Refs.~\cite{prgs,mathprg}.

\subsection{Borel Sum Rules to $\tau$-decay}

The application of dispersion relation to the polarization (current correlation) function $\Pi(Q^2)$ of the strangeless vector (V) and axial (A) currents gives us a holomorphic (analytic) function in the complex $Q^2$-plane, for $Q^2 \in \mathbb{C} \backslash (-\infty, -M_{\rm thr}^2]$ where the hadron production threshold mass is $M_{\rm thr}=M_1 \sim 0.1$ GeV. This quantity is then multiplied by any function $g(Q^2)$ [$\exp(Q^2/M^2)$ in the case of Borel Sum Rules] analytic in the entire complex $Q^2$-plane, 
and the Cauchy integral formula can be applied to the integral of $g(Q^2) \Pi_{V+A}(Q^2)$. With this, we arrive to the following relation
\be
{\rm Re} B_{\rm exp}(M^2) =  {\rm Re} B_{\rm th}(M^2) \ ,
\label{sr3o0}
\ee
where
\bea
\!\!\!B_{\rm exp}(M^2) &\equiv& \int_0^{\sigma_{\rm max}} 
\frac{d \sigma}{M^2} \; \exp( - \sigma/M^2) \omega_{\rm exp}(\sigma)_{V+A} \ ,
\label{sr3a}
\nonumber\\
\!B_{\rm th}(M^2) &\equiv&  \left( 1 - \exp(-\sigma_{\rm max}/M^2) \right)
+ B_{\rm th}(M^2;D\!=\!0)
\nonumber\\
&&+ 2 \pi^2 \sum_{n \geq 2}
 \frac{ \langle O_{2n} \rangle_{V+A}}{ (n-1)! \; (M^2)^n} \ ,
\label{sr3b}
\eea
and where the leading-twist contributions ($D=0$) is
\bea
\lefteqn{B_{\rm th}(M^2;D\!=\!0)=\frac{1}{2 \pi}\int_{-\pi}^{\pi}
d \phi \; d(\sigma_{\rm max} e^{i \phi};0) }
\nonumber\\
&&\times\left[ 
\exp \left( \frac{\sigma_{\rm max} e^{i \phi}}{M^2} \right) -
\exp \left( - \frac{\sigma_{\rm max}}{M^2} \right) \right] \ .
\label{BD0}
\eea
The total $D (\equiv 2 n) =2$ contribution in the OPE (\ref{sr3b}) is negligible, and we will include there the $D=4$ and $D=6$ terms. The advantage of the use of the Borel sum rules approach is that it is dominant in the low-$\sigma$ (IR) regime, and we can extract the gluon ($D=4$) and quark ($D=6$) condensates separately, depending on the choice of the complex argument [when $M^2=|M^2| \exp(i \pi/6)$, $D=6$ term in ${\rm Re} B_{\rm th}(M^2)$ is zero; and when  $M^2=|M^2| \exp(i \pi/4)$, the corresponding $D=4$ term is zero]. 

The experimental data used here are given by OPAL  \cite{OPAL} and ALEPH Collaborations \cite{ALEPH2,ALEPHfin}. Our combined fitting values of the condensates are \cite{3danQCD2}
\bea
\langle a GG \rangle&=&-0.0046\pm 0.0038\ [{\rm GeV}^4] ,
\label{aGGOPAL}
\\
\langle O_6 \rangle_{V+A}&=&0.00135\pm0.00039\ [{\rm GeV}^6].
\label{O6OPAL}
\eea

In Fig.~\ref{FigPsi0}, the curves for ${\rm Arg} M^2=0$ are presented, with the corresponding central values of the condensates obtained from OPAL Collaborations data (close to (\ref{aGGOPAL})-(\ref{O6OPAL}) values). We observe that our model applied in this case ($\A$QCD$+$OPE approach) agrees well with the (OPAL) experimental band in the entire presented $M^2$-interval, in contrast to the pQCD approach which agrees for the range $|M^2|\gtrsim 0.8$GeV$^2$. 


\begin{figure}[htb]
\vspace{9pt}
\centering\includegraphics[width=70mm]{BexpBinPsi0o98full4lrt0201al01185.pdf}
%\framebox[55mm]{\rule[-21mm]{0mm}{43mm}}
\caption{Borel transforms ${\rm Re} B(M^2)$ for real $M^2 > 0$. The grey band represents the experimental results. For comparison, we show the fitted theoretical curve of $\overline{\rm MS}$ pQCD approach (dotted line). The theoretical curve given by our coupling ($\A$QCD) almost agrees with the central experimental (OPAL) curve.}
\label{FigPsi0}
\end{figure}
%

We note that in the Borel sum rules we used $\sigma_{\rm max}=3.136 \ {\rm GeV}^2$ in the OPAL case, and $\sigma_{\rm max}=2.80 \ {\rm GeV}^2$ in the ALEPH case \cite{3danQCD1,3danQCD2}. We are interested in what happens when we decrease the value of $\sigma_{\rm max}$ while keeping the obtained original values of the condensates. In Fig.~\ref{FigPsi0o26} for instance, we show the case when $\sigma_{\rm max}=0.832 \ {\rm GeV}^2$; the $\A$QCD$+$OPE approach is significatively better than pQCD, the latter is located well outside the narrow experimental uncertainty band, and our approach remains inside in the whole presented range of $M^2$.
Similar results and conclusions given in this Section are obtained when using ALEPH Collaboration data \cite{3danQCD1,3danQCD2}.
\begin{figure}[htb]
\vspace{9pt}
\centering\includegraphics[width=70mm]{BexpBinPsi0o26full4lrt0201al01185.pdf}
%\framebox[55mm]{\rule[-21mm]{0mm}{43mm}}
\caption{As in Fig.~\ref{FigPsi0}, but now for a lower scale $\sigma_{\rm max}=0.832 \ {\rm GeV}^2$. The $\A$QCD curve (dashed) is inside the experimental band. Again, we show the $\overline{\rm MS}$ pQCD approach (dotted line) for comparison.}
\label{FigPsi0o26}
\end{figure}
%

\subsection{V-channel Adler function $\mathcal{D}_V(Q^2)$}

The V-channel Adler function ${\cal D}_V(Q^2)$ is related via dispersion relation with the production ratio $R(\sigma)$ for $e^+ e^- \to$ hadrons at the center-of-mass squared energy $\sigma$. The V-channel Adler function is
\bea
{\cal D}_V(Q^2) &\equiv&  - 4 \pi^2 \frac{d \Pi_V(Q^2)}{d \ln Q^2} 
 =  1 + d(Q^2;D=0) 
\nonumber\\
&&\qquad + 2 \pi^2 \sum_{n \geq 2}
 \frac{ n 2 \langle O_{2n} \rangle_V}{(Q^2)^n}  \ ,
\label{DV}
\eea
where $d(Q^2;D=0) $ is given by (\ref{dan}), and we estimate the values of the V-channel condensates from the values of the V+A channel condensates obtained in the previous Subsection. With these condensates, we can apply it using the relation $\langle O_4 \rangle_{V+A}=2 \langle O_4 \rangle_V$ ($=2 \langle O_4 \rangle_A$) for the $D=4$ condensates \cite{Braaten,PichPra}, and vacuum saturation approximation $\langle O_6 \rangle_{V+A}\approx-\frac{4}{7}\langle O_6 \rangle_V$ for the $D=6$ condensates \cite{3danQCD1,3danQCD2,Ioffe}.

\begin{figure}[htb]
\vspace{9pt}
\centering\includegraphics[width=70mm]{figCombAdlVvsQOPoTheoOPE4l3d.pdf}
%\framebox[55mm]{\rule[-21mm]{0mm}{43mm}}
\caption{The V-channel Adler function at $Q^2>0$. Experimental data are denoted by the grey band taken from \cite{NestBook}. The solid lines are our coupling (\ref{AQ2}) (or $\A$QCD), and the dash-dotted lines are in the $\overline{\rm MS}$ pQCD approach. The dashed line is the leading twist (LT) contribution in $\A$QCD, and the dotted line in $\overline{\rm MS}$ pQCD.}
\label{FigDVOPE}
\end{figure}
%

In Fig.~\ref{FigDVOPE}, the $\A$QCD+OPE approach gives results within the experimental band for all $Q^2$ down to  $Q^2 \approx 1 \ {\rm GeV}^2$, while the $\overline{\rm MS}$ pQCD+OPE only down to $Q^2 \approx 2.5 \ {\rm GeV}^2$. We stress that the incorporation of the lattice-motivated behavior for $\A(Q^2)$ at $|Q^2| \lesssim 0.1 \ {\rm GeV}^2$ influences significantly the behavior of $\A(Q^2)$ in the entire complex $Q^2$-plane, including in the regime of our principal interest, $|Q^2| \sim 1 \ {\rm GeV}^2$. The OPE series (\ref{DV}) is expected to fail always at $|Q^2| < 1 \ {\rm GeV}^2$

\subsection{Bjorken Sum Rule (BSR)} 

The polarized Bjorken sum rule (BSR) is defined as integral over the $x$-Bjorken of the nonsinglet combination of the proton and neutron polarized structure functions, i. e., 
\be
\Gamma_1^{p-n}(Q^2)=\int_0^1 dx \left[g_1^p(x,Q^2)-g_1^n(x,Q^2) \right]\ .
\label{BSRdef}
\ee 
BSR can be written in terms of a sum of two series, one coming from pQCD and the other from the higher-twist (HT) contributions dictated by the 
OPE \cite{BjorkenSR}
\be
\label{BSR}
\Gamma_1^{p-n}(Q^2)=\frac{g_A}{6}E_{\rm {NS}}(Q^2)+\sum_{i=2}^\infty
\frac{\mu_{2i}^{p-n}(Q^2)}{Q^{2i-2}}\ ,
\ee
 where the nucleon axial charge is $g_A=1.2723$ \cite{PDG2016}.We will include only the first HT term $\sim \mu_4^{p-n}$.

In our analysis it is convenient to exclude the elastic contribution, because the $Q^2$-dependence of the nonsinglet inelastic BSR in low-$Q^2$ regime is constrained by the Gerasimov-Drell-Hearn (GDH) sum rule \cite{GDHlow}, as was pointed out in \cite{PSTSK10}. 
%Therefore, we will investigate the behavior of the pure inelastic contribution as a continuation to low-energy scale \cite{GDHlow}.

The leading-twist (LT) contribution $E_{\rm {NS}}(Q^2)$ was calculated up to N$^3$LO contribution in \cite{nnnloBSR}.  

In analytic QCD approaches, the powers $a^{\nu}$ (where $\nu$ is not necessarily integer) get transformed to $\A_{\nu}$ (which is in general different from $\A^\nu$), according to the general formalism of Ref.~\cite{GCAK}. We apply it to the twist-4 term \cite{anBSR}
\be
\mu_{4,j}^{p-n}(Q^2)=\mu_{4,j}^{p-n}(Q_{\rm in}^2) \frac{\A_{\nu}^{(j)}(Q^2)}
{\A_{\nu}^{(j)}(Q_{\rm in}^2)}\ .
\label{HTQ2an}
\ee
where $\nu=1/8\beta_0$.
With the corresponding analytization of the HT term (\ref{HTQ2an}) and the implementation in the LT part, i.e., $a(Q^2)^n\mapsto\A_n(Q^2)$ in (\ref{BSR}), we can find a fit for $\mu_{4,j}^{p-n}(Q_{\rm in}^2)$ \cite{anBSR}. The resulting value at $Q_{\rm in}^2=1$GeV$^2$ is $\mu_{4,3\delta{\rm anQCD}}^{p-n}(1$GeV$^2)=-0.019$ and the corresponding plot is given in Fig.~\ref{FigFitNf3}.  We observe that the resulting fit describes almost the whole available experimental data (for $Q^2\gtrsim0.2$GeV$^2$) in contrast to pQCD which describes the data only for $Q^2\gtrsim0.7$GeV$^2$.

\begin{figure}[htb]
\vspace{9pt}
\centering\includegraphics[width=70mm]{BSRfig.pdf}
%\framebox[55mm]{\rule[-21mm]{0mm}{43mm}}
\caption{Fits of JLAB and SLAC combined data \cite{dataBSR} on BSR $\Gamma_1^{p-n}(Q^2)$ as a function of $Q^2$, using (four-loop) $\overline{\rm MS}$ pQCD and our coupling $\A$ given by Eq.(\ref{AQ2}).}
\label{FigFitNf3}
\end{figure}

\section{Conclusions}

In this work we have presented a new QCD running coupling from their 
dispersive representation. Here we parametrize the IR regime of the spectral function with three delta functions. This allowed us to fulfill various physically motivated conditions, at high, intermediate and low momenta generating a holomorphic running coupling. The main feature is that in the deep-IR it behaves as $\A(Q^2)\sim Q^2$ as motivated by lattice calculations, and it reproduces the pQCD coupling at high-momenta.
Then we applied it to three different low-energy observables and we found that at $Q^2\sim1$GeV$^2$ scales our coupling is significantly better than pQCD+OPE approach. 

\section*{Acknowledgements}
This work was supported by FONDECYT (Chile) Postdoctoral Grant No.~3170116. 


\begin{thebibliography}{999}
  %\cite{Shirkov:2006gv}
\bibitem{APT} 
  D.~V.~Shirkov and I.~L.~Solovtsov,
  %``Ten years of the Analytic Perturbation Theory in QCD,''
  Theor.\ Math.\ Phys.\  {\bf 150}, 132 (2007)
  %doi:10.1007/s11232-007-0010-7
  [hep-ph/0611229] and references therein. 
  %%CITATION = doi:10.1007/s11232-007-0010-7;%%
  %112 citations counted in INSPIRE as of 07 Sep 2017

%\cite{Bakulev:2005gw}
\bibitem{BMS} 
  A.~P.~Bakulev, S.~V.~Mikhailov and N.~G.~Stefanis,
  %``QCD analytic perturbation theory: From integer powers to any power of the running coupling,''
  Phys.\ Rev.\ D {\bf 72}, 074014 (2005)
  [Phys.\ Rev.\ D {\bf 72}, 119908 (2005)]
  %doi:10.1103/PhysRevD.72.074014, 10.1103/PhysRevD.72.119908
  [hep-ph/0506311];
  %%CITATION = doi:10.1103/PhysRevD.72.074014, 10.1103/PhysRevD.72.119908;%%
  %84 citations counted in INSPIRE as of 07 Sep 2017
%\cite{Bakulev:2006ex}
%\bibitem{BMS} 
  %A.~P.~Bakulev, S.~V.~Mikhailov and N.~G.~Stefanis,
  %``Fractional Analytic Perturbation Theory in Minkowski space and application to Higgs boson decay into a b anti-b pair,''
  Phys.\ Rev.\ D {\bf 75}, 056005 (2007)
  Erratum: [Phys.\ Rev.\ D {\bf 77}, 079901 (2008)]
  %doi:10.1103/PhysRevD.75.056005, 10.1103/PhysRevD.77.079901
  [hep-ph/0607040];
  %%CITATION = doi:10.1103/PhysRevD.75.056005, 10.1103/PhysRevD.77.079901;%%
  %69 citations counted in INSPIRE as of 07 Sep 2017
  %\cite{Bakulev:2010gm}
%\bibitem{Bakulev:2010gm} 
  %A.~P.~Bakulev, S.~V.~Mikhailov and N.~G.~Stefanis,
  %``Higher-order QCD perturbation theory in different schemes: From FOPT to CIPT to FAPT,''
  JHEP {\bf 1006}, 085 (2010)
  %doi:10.1007/JHEP06(2010)085
  [arXiv:1004.4125 [hep-ph]].
  %%CITATION = doi:10.1007/JHEP06(2010)085;%%
  %35 citations counted in INSPIRE as of 07 Sep 2017
 
%\cite{Shirkov:2012ux}
\bibitem{ShMPT} 
  D.~V.~Shirkov,
  %``'Massive' Perturbative QCD, regular in the IR limit,''
  Phys.\ Part.\ Nucl.\ Lett.\  {\bf 10}, 186 (2013)
  %doi:10.1134/S1547477113030138
  [arXiv:1208.2103 [hep-th]].
  %%CITATION = doi:10.1134/S1547477113030138;%%
  %25 citations counted in INSPIRE as of 13 Sep 2017

%\cite{Ayala:2012xf}
\bibitem{2danQCD} 
  C.~Ayala, C.~Contreras and G.~Cvetic,
  %``Extended analytic QCD model with perturbative QCD behavior at high momenta,''
  Phys.\ Rev.\ D {\bf 85}, 114043 (2012)
  %doi:10.1103/PhysRevD.85.114043
  [arXiv:1203.6897 [hep-ph]].
  %%CITATION = doi:10.1103/PhysRevD.85.114043;%%
  %24 citations counted in INSPIRE as of 13 Sep 2017

%\cite{Ayala:2016zrz}
\bibitem{3danQCD1} 
  C.~Ayala, G.~Cvetic and R.~Kogerler,
  %``Lattice-motivated holomorphic nearly perturbative QCD,''
  J.\ Phys.\ G {\bf 44}, no. 7, 075001 (2017)
  %doi:10.1088/1361-6471/aa6fdf
  [arXiv:1608.08240 [hep-ph]].
  %%CITATION = doi:10.1088/1361-6471/aa6fdf;%%
  %5 citations counted in INSPIRE as of 06 Sep 2017
 
 %\cite{Ayala:2017tco}
\bibitem{3danQCD2} 
  C.~Ayala, G.~Cvetic, R.~Kogerler and I.~Kondrashuk,
  %``Nearly perturbative lattice-motivated QCD coupling with zero IR limit,''
  arXiv:1703.01321 [hep-ph].
  %%CITATION = ARXIV:1703.01321;%%
  %4 citations counted in INSPIRE as of 06 Sep 2017
  
\bibitem{Baker}
G.A.~Baker and P.~Graves-Morris, Pad\'e Approximants, Encyclopedia of Mathematics and its Applications, Cambridge Univ. Press 1996. Section 5.4, Theorem 5.4.2.
%%CITATION = NONE%%

  \bibitem{PDG}
  K.~A.~Olive {\it et al.} [Particle Data Group Collaboration],
  ``Review of Particle Physics,''
  Chin.\ Phys.\ C {\bf 38}, 090001 (2014).
  %doi:10.1088/1674-1137/38/9/090001.
  %%CITATION = doi:10.1088/1674-1137/38/9/090001;%%
  %4393 citations counted in INSPIRE as of 23 Jul 2016
  
  \bibitem{ALEPH2}
%\bibitem{Schael:2005am}
  S.~Schael {\it et al.}  [ALEPH Collaboration],
%``Branching ratios and spectral functions of tau decays: final ALEPH measurements and physics implications,''
Phys.\ Rept.\  {\bf 421}, 191 (2005)
  %doi:10.1016/j.physrep.2005.06.007
  [hep-ex/0506072];
  %%CITATION = doi:10.1016/j.physrep.2005.06.007;%%
  %275 citations counted in INSPIRE as of 25 Jul 2016
  %\bibitem{Davier:2005xq} 
  M.~Davier, A.~H\"ocker and Z.~Zhang,
 % ``The Physics of hadronic tau decays,''
 Rev.\ Mod.\ Phys.\  {\bf 78}, 1043 (2006)
  %doi:10.1103/RevModPhys.78.1043
  [hep-ph/0507078].
  %%CITATION = doi:10.1103/RevModPhys.78.1043;%%
  %232 citations counted in INSPIRE as of 25 Jul 2016
  
\bibitem{DDHMZ}
  M.~Davier, S.~Descotes-Genon, A.~H\"ocker, B.~Malaescu and Z.~Zhang,
 % ``The Determination of $\alpha_s$ from $\tau$ decays revisited,''
 Eur.\ Phys.\ J.\ C {\bf 56}, 305 (2008)
  %doi:10.1140/epjc/s10052-008-0666-7
  [arXiv:0803.0979 [hep-ph]].
  %%CITATION = doi:10.1140/epjc/s10052-008-0666-7;%%
  %177 citations counted in INSPIRE as of 25 Jul 2016
  
  %\bibitem{Braaten:1988hc}
\bibitem{Braaten}
  E.~Braaten,
  %``QCD Predictions for the decay of the tau lepton,''
 Phys.\ Rev.\ Lett.\  {\bf 60}, 1606 (1988);
  %doi:10.1103/PhysRevLett.60.1606;
  %%CITATION = doi:10.1103/PhysRevLett.60.1606;%%
  %261 citations counted in INSPIRE as of 24 Jul 2016
%\bibitem{Braaten:1992qm}
E.~Braaten, S.~Narison, and A.~Pich,
%``QCD analysis of the tau hadronic width,''
 Nucl.\ Phys.\ B {\bf 373}, 581 (1992).
  %doi:10.1016/0550-3213(92)90267-F.
  %%CITATION = doi:10.1016/0550-3213(92)90267-F;%%
  %612 citations counted in INSPIRE as of 24 Jul 2016

\bibitem{PichPra}
%\bibitem{Narison:1988ni}
  S.~Narison and A.~Pich,
  %``QCD formulation of the tau decay and determination of $\Lambda_{{\overline {\rm MS}}}$,''
Phys.\ Lett.\ B {\bf 211}, 183 (1988);
  %doi:10.1016/0370-2693(88)90830-1;
  %%CITATION = doi:10.1016/0370-2693(88)90830-1;%%
  %261 citations counted in INSPIRE as of 24 Jul 2016
%\bibitem{Pich:1998yn}
A.~Pich and J.~Prades,
%``Perturbative quark mass corrections to the tau hadronic width,''
 JHEP {\bf 9806}, 013 (1998)
  %doi:10.1088/1126-6708/1998/06/013
  [hep-ph/9804462].
  %%CITATION = doi:10.1088/1126-6708/1998/06/013;%%
  %95 citations counted in INSPIRE as of 24 Jul 2016


\bibitem{d3}
  P.~A.~Baikov, K.~G.~Chetyrkin and J.~H.~K\"uhn,
  %``Order $\alpha^4_s$ QCD Corrections to $Z$ and $\tau$ Decays,''
 Phys.\ Rev.\ Lett.\  {\bf 101}, 012002 (2008)
  %doi:10.1103/PhysRevLett.101.012002
  [arXiv:0801.1821 [hep-ph]].
  %%CITATION = doi:10.1103/PhysRevLett.101.012002;%%
  %296 citations counted in INSPIRE as of 24 Jul 2016

\bibitem{CV12}
%\bibitem{Cvetic:2006mk}
  G.~Cveti\v{c} and C.~Valenzuela,
%  ``An approach for evaluation of observables in analytic versions of QCD,''
 J.\ Phys.\ G {\bf 32}, L27 (2006)
  %doi:10.1088/0954-3899/32/6/L01
  [hep-ph/0601050];
  %%CITATION = doi:10.1088/0954-3899/32/6/L01;%%
  %36 citations counted in INSPIRE as of 24 Jul 2016
%\bibitem{Cvetic:2006gc}
%G.~Cveti\v c and C.~Valenzuela,
  %``Various versions of analytic QCD and skeleton-motivated evaluation of
 %observables,''
 Phys.\ Rev.\ D {\bf 74}, 114030 (2006)
  Erratum: [Phys.\ Rev.\ D {\bf 84}, 019902 (2011)]
  %doi:10.1103/PhysRevD.74.114030, 10.1103/PhysRevD.84.019902
  [hep-ph/0608256].
  %%CITATION = doi:10.1103/PhysRevD.74.114030, 10.1103/PhysRevD.84.019902;%%
  %39 citations counted in INSPIRE as of 24 Jul 2016  

\bibitem{GCAK}
%\bibitem{Cvetic:2011ym} 
  G.~Cveti\v{c} and A.~V.~Kotikov,
  %``Analogs of Noninteger Powers in General Analytic QCD,''
  J.\ Phys.\ G {\bf 39}, 065005 (2012)
  %doi:10.1088/0954-3899/39/6/065005
  [arXiv:1106.4275 [hep-ph]].
  %%CITATION = doi:10.1088/0954-3899/39/6/065005;%%
  %27 citations counted in INSPIRE as of 24 Jul 2016
  
 \bibitem{LattcoupNf0}
%\bibitem{Bogolubsky:2009dc} 
  I.~L.~Bogolubsky, E.~M.~Ilgenfritz, M.~M\"uller-Preussker and A.~Sternbeck, 
%``Lattice gluodynamics computation of Landau gauge Green's functions in the deep infrared,''
  Phys.\ Lett.\ B {\bf 676}, 69 (2009)
  %doi:10.1016/j.physletb.2009.04.076
  [arXiv:0901.0736 [hep-lat]].
  %%CITATION = doi:10.1016/j.physletb.2009.04.076;%%
  %309 citations counted in INSPIRE as of 20 Jul 2016
 
 \bibitem{GCIK}
%\bibitem{Cvetic:2011vy} 
  G.~Cveti\v{c} and I.~Kondrashuk,
 % ``Explicit solutions for effective four- and five-loop QCD running coupling,''
  JHEP {\bf 1112}, 019 (2011)
  %doi:10.1007/JHEP12(2011)019
  [arXiv:1110.2545 [hep-ph]].
  %%CITATION = doi:10.1007/JHEP12(2011)019;%%
  %10 citations counted in INSPIRE as of 03 Sep 2016
  
  \bibitem{MiniMOM}
%\bibitem{vonSmekal:2009ae} 
  L.~von Smekal, K.~Maltman and A.~Sternbeck,
  %``The Strong coupling and its running to four loops in a minimal MOM scheme,''
  Phys.\ Lett.\ B {\bf 681}, 336 (2009)
  %doi:10.1016/j.physletb.2009.10.030
  [arXiv:0903.1696 [hep-ph]].
  %%CITATION = doi:10.1016/j.physletb.2009.10.030;%%
  %56 citations counted in INSPIRE as of 21 Jul 2016
  
   \bibitem{LattcoupNf2}
%\bibitem{Ilgenfritz:2006gp} 
  E.-M.~Ilgenfritz, M.~M\"uller-Preussker, A.~Sternbeck and A.~Schiller,  
%``Gauge-variant propagators and the running coupling from lattice QCD,''
  hep-lat/0601027.
  %%CITATION = HEP-LAT/0601027;%%
  %34 citations counted in INSPIRE as of 20 Jul 2016

\bibitem{prgs}
On www page http://gcvetic.usm.cl 
%%CITATION = NONE;%%

%\cite{Ayala:2014pha}
\bibitem{mathprg} 
  C.~Ayala and G.~Cveti\v{c},
  %``anQCD: a Mathematica package for calculations in general analytic QCD models,''
  Comput.\ Phys.\ Commun.\  {\bf 190}, 182 (2015)
  [arXiv:1408.6868 [hep-ph]];
  %%CITATION = ARXIV:1408.6868;%%
%\bibitem{mathprgb} 
%\cite{Ayala:2014qea}
%\bibitem{Ayala:2014qea} 
  %C.~Ayala and G.~Cveti\v{c},
  %``Mathematica and Fortran programs for various analytic QCD couplings,''
  J.\ Phys.\ Conf.\ Ser.\  {\bf 608}, no. 1, 012064 (2015)
  [arXiv:1411.1581 [hep-ph]];
  %%CITATION = ARXIV:1411.1581;%%
  %3 citations counted in INSPIRE as of 09 juil. 2015
%\cite{Ayala:2015sca}
%\bibitem{Ayala:2015sca} 
  %C.~Ayala and G.~Cvetic,
  %``anQCD: Fortran programs for couplings at complex momenta in various analytic QCD models,''
Comput.\ Phys.\ Commun.\  {\bf 199}, 114 (2016)
  %doi:10.1016/j.cpc.2015.10.004
  [arXiv:1506.07201 [hep-ph]].
  %%CITATION = doi:10.1016/j.cpc.2015.10.004;%%

\bibitem{OPAL}
 %\bibitem{Ackerstaff:1998yj} 
  K.~Ackerstaff {\it et al.} [OPAL Collaboration],
 % ``Measurement of the strong coupling constant $\alpha_s$ and the vector and axial vector spectral functions in hadronic tau decays,''
  Eur.\ Phys.\ J.\ C {\bf 7}, 571 (1999)
  %doi:10.1007/s100520050430, 10.1007/s100529901061
  [hep-ex/9808019].
  %%CITATION = doi:10.1007/s100520050430, 10.1007/s100529901061;%%
  %364 citations counted in INSPIRE as of 25 Jul 2016
 
%\bibitem{Davier:2013sfa}
  \bibitem{ALEPHfin}
  M.~Davier, A.~H\"ocker, B.~Malaescu, C.~Z.~Yuan and Z.~Zhang,
  %``Update of the ALEPH non-strange spectral functions from hadronic $\tau$ decays,''
  Eur.\ Phys.\ J.\ C {\bf 74}, no. 3, 2803 (2014)
  %doi:10.1140/epjc/s10052-014-2803-9
  [arXiv:1312.1501 [hep-ex]].
  %%CITATION = doi:10.1140/epjc/s10052-014-2803-9;%%
  %49 citations counted in INSPIRE as of 15 Dec 2016
           
\bibitem{Ioffe}
  B.~L.~Ioffe,
 % ``QCD at low energies,''
  Prog.\ Part.\ Nucl.\ Phys.\  {\bf 56}, 232 (2006)
  [arXiv:hep-ph/0502148].
  %%CITATION = PPNPD,56,232;%%

\bibitem{NestBook}
  %\bibitem{Nest3a}
%\bibitem{Nesterenko:2005wh}
A.~V.~Nesterenko and J.~Papavassiliou,
 %``A novel integral representation for the Adler function,''
  J.\ Phys.\ G {\bf 32}, 1025 (2006)
  %doi:10.1088/0954-3899/32/7/011
  [hep-ph/0511215];
  %%CITATION = doi:10.1088/0954-3899/32/7/011;%%
  %41 citations counted in INSPIRE as of 24 Jul 2016
%\bibitem{Nesterenko:2008fb} 
  A.~V.~Nesterenko,
  %``On the low-energy behavior of the Adler function,''
  Nucl.\ Phys.\ Proc.\ Suppl.\  {\bf 186}, 207 (2009)
  %doi:10.1016/j.nuclphysbps.2008.12.048
  [arXiv:0808.2043 [hep-ph]];
  %%CITATION = doi:10.1016/j.nuclphysbps.2008.12.048;%%
  %21 citations counted in INSPIRE as of 17 Apr 2017
  %\bibitem{Nesterenko:2016pmx} 
 % A.~V.~Nesterenko,
  ``Strong interactions in spacelike and timelike domains: dispersive approach,'' Elsevier, Amsterdam, 2016, eBook ISBN: 9780128034484.
  %%CITATION = INSPIRE-1514185;%%

%\cite{Bjorken:1966jh}
\bibitem{BjorkenSR} 
  J.~D.~Bjorken,
  %``Applications of the Chiral U(6) x (6) Algebra of Current Densities,''
  Phys.\ Rev.\  {\bf 148}, 1467 (1966);
  %%CITATION = PHRVA,148,1467;%%
  %1346 citations counted in INSPIRE as of 24 Apr 2015
%\cite{Bjorken:1969mm}
%\bibitem{Bjorken:1969mm} 
  %J.~D.~Bjorken,
  %``Inelastic Scattering of Polarized Leptons from Polarized Nucleons,''
  Phys.\ Rev.\ D {\bf 1}, 1376 (1970).
  %%CITATION = PHRVA,D1,1376;%%
  %582 citations counted in INSPIRE as of 24 Apr 2015

\bibitem{PDG2016}
%\bibitem{Olive:2016xmw} 
  C.~Patrignani {\it et al.} [Particle Data Group Collaboration],
  %``Review of Particle Physics,''
  Chin.\ Phys.\ C {\bf 40}, no. 10, 100001 (2016).
  %doi:10.1088/1674-1137/40/10/100001
  %%CITATION = doi:10.1088/1674-1137/40/10/100001;%%

%\cite{Soffer:1992ck}
\bibitem{GDHlow} 
  J.~Soffer and O.~Teryaev,
  %``The Role of g-2 in relating the Schwinger and Gerasimov-Drell-Hearn sum rules,''
  Phys.\ Rev.\ Lett.\  {\bf 70}, 3373 (1993);
  %%CITATION = PRLTA,70,3373;%%
  %55 citations counted in INSPIRE as of 05 Mar 2015
  %\cite{Soffer:2004ip}
%\bibitem{Soffer:2004ip} 
  %J.~Soffer and O.~Teryaev,
  %``QCD radiative and power corrections and generalized GDH sum rules,''
  Phys.\ Rev.\ D {\bf 70}, 116004 (2004)
  [hep-ph/0410228].
  %%CITATION = HEP-PH/0410228;%%
  %28 citations counted in INSPIRE as of 05 Mar 2015
  
 %\cite{Pasechnik:2009yc}
\bibitem{PSTSK10} 
 %\cite{Pasechnik:2008th}
%\bibitem{Pasechnik:2008th} 
  R.~S.~Pasechnik, D.~V.~Shirkov and O.~V.~Teryaev,
  %``Bjorken Sum Rule and pQCD frontier on the move,''
  Phys.\ Rev.\ D {\bf 78}, 071902 (2008); 
  %doi:10.1103/PhysRevD.78.071902
  [arXiv:0808.0066 [hep-ph]];
  %%CITATION = doi:10.1103/PhysRevD.78.071902;%%
  %55 citations counted in INSPIRE as of 12 Aug 2017
  R.~S.~Pasechnik, D.~V.~Shirkov, O.~V.~Teryaev, O.~P.~Solovtsova and V.~L.~Khandramai,
  %``Nucleon spin structure and pQCD frontier on the move,''
  Phys.\ Rev.\ D {\bf 81}, 016010 (2010)
  [arXiv:0911.3297 [hep-ph]];
  %%CITATION = ARXIV:0911.3297;%%
  %30 citations counted in INSPIRE as of 05 Mar 2015
%\cite{Khandramai:2011zd}
%\bibitem{KPSST} 
  %V.~L.~Khandramai, R.~S.~Pasechnik, D.~V.~Shirkov, O.~P.~Solovtsova and O.~V.~Teryaev,
  %``Four-loop QCD analysis of the Bjorken sum rule vs data,''
  Phys.\ Lett.\ B {\bf 706}, 340 (2012)
  [arXiv:1106.6352 [hep-ph]];
  %%CITATION = ARXIV:1106.6352;%%
  %18 citations counted in INSPIRE as of 16 Mar 2015
  %\cite{Khandramai:2013haz}
%\bibitem{Khandramai:2013haz} 
  V.~L.~Khandramai, O.~P.~Solovtsova and O.~V.~Teryaev,
  %``Polarized Bjorken Sum Rule Analysis: Revised,''
  Nonlin.\ Phenom.\ Complex Syst.\  {\bf 16}, 93 (2013)
  [arXiv:1302.3952 [hep-ph]].
  %%CITATION = ARXIV:1302.3952;%%
  %2 citations counted in INSPIRE as of 16 Mar 2015

%\cite{Baikov:2010je}
\bibitem{nnnloBSR} 
  P.~A.~Baikov, K.~G.~Chetyrkin and J.~H.~K\"uhn,
  %``Adler Function, Bjorken Sum Rule, and the Crewther Relation to Order alpha_s^4 in a General Gauge Theory,''
  Phys.\ Rev.\ Lett.\  {\bf 104}, 132004 (2010)
  [arXiv:1001.3606 [hep-ph]].
  %%CITATION = ARXIV:1001.3606;%%
  %61 citations counted in INSPIRE as of 05 mar 2015

%\cite{Ayala:2017uzx}
\bibitem{anBSR} 
  C.~Ayala, G.~Cvetic, A.~V.~Kotikov and B.~G.~Shaikhatdenov,
  %``Bjorken sum rule in QCD frameworks with analytic (holomorphic) coupling,''
  arXiv:1708.06284 [hep-ph].
  %%CITATION = ARXIV:1708.06284;%%

%\cite{Deur:2014vea}
\bibitem{dataBSR} 
  A.~Deur {\it et al.},
  %``High precision determination of the $Q^2$ evolution of the Bjorken Sum,''
  Phys.\ Rev.\ D {\bf 90}, no. 1, 012009 (2014)
  %doi:10.1103/PhysRevD.90.012009
  [arXiv:1405.7854 [nucl-ex]] and references therein.
  %%CITATION = doi:10.1103/PhysRevD.90.012009;%%
  %14 citations counted in INSPIRE as of 14 Sep 2017

\end{thebibliography}
\end{document}
