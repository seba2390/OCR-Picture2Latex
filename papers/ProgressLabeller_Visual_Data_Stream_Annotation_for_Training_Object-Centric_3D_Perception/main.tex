%%%%%%%%%%%%%%%%%%%%%%%%%%%%%%%%%%%%%%%%%%%%%%%%%%%%%%%%%%%%%%%%%%%%%%%%%%%%%%%%
%2345678901234567890123456789012345678901234567890123456789012345678901234567890
%        1         2         3         4         5         6         7         8

\documentclass[letterpaper, 10 pt, conference]{ieeeconf}
\usepackage[letterpaper, left=0.75in, right=0.75in, bottom=0.77in, top=0.78in]{geometry}
%\documentclass[a4paper, 10pt, conference]{ieeeconf}      % Use this line for a4 paper

\IEEEoverridecommandlockouts                              % This command is only needed if 
                                                         % you want to use the \thanks command

%\overrideIEEEmargins                                      % Needed to meet printer requirements.

% See the \addtolength command later in the file to balance the column lengths
% on the last page of the document

% The following packages can be found on http:\\www.ctan.org
% \usepackage{graphics} % for pdf, bitmapped graphics files
\usepackage{graphicx}
\usepackage{amsmath}
\usepackage{xcolor}
\usepackage{floatrow}
\usepackage{epstopdf}
\usepackage[colorlinks = true,
            linkcolor = black,
            urlcolor  = black,
            citecolor = black,
            anchorcolor = black]{hyperref}
\newtheorem{theorem}{Lemma}
\usepackage{algorithm,algpseudocode}
\usepackage[font=small]{caption}
\usepackage{accents}
\usepackage{amsfonts}
\usepackage{wrapfig}
% \usepackage{subcaption}
\usepackage{booktabs}
\usepackage{balance}
\usepackage{scalerel}
\usepackage{subfigure}
\usepackage{multirow}
\usepackage{url}
\usepackage{svg}
\usepackage{hhline}
\usepackage{hyperref}
% \usepackage[numbers,sort&compress]{natbib}
\def\bibfont{\footnotesize}

\newcommand*{\Comb}[2]{{}^{#1}C_{#2}}
\newcommand{\norm}[1]{\left\lVert#1\right\rVert}
   
\long\def\ocj#1{\textcolor{red}{#1}}
\newcommand{\zmchange}[1]{{\color{blue}  #1} } 
\newcommand{\zm}[1]{{\color{blue} \bf **#1**} }
\long\def\chad#1{\textcolor{red}{\bf \small CJ: **#1**}}
\long\def\zz#1{\textcolor{orange}{\bf \small ZZ: **#1**}}
\long\def\kz#1{\textcolor{cyan}{\bf \small kz: **#1**}}
\long\def\cxt#1{\textcolor{magenta}{\bf \small cxt: **#1**}}
\long\def\jp#1{\textcolor{teal}{\bf \small JP: **#1**}}
\long\def\ignore#1{}
\textfloatsep=3pt
\floatsep=3pt
\intextsep=3pt
\dblfloatsep=3pt
\dbltextfloatsep=3pt

\DeclareMathOperator*{\argmax}{argmax}
% \abovecaptionskip=3pt
% \belowcaptionskip=3pt
\begin{document}
\title{\LARGE \bf
ProgressLabeller: Visual Data Stream Annotation for Training Object-Centric 3D Perception
}

\author{Xiaotong Chen\hspace{0.5cm} 
Huijie Zhang\hspace{0.5cm}  Zeren Yu\hspace{0.5cm}  Stanley Lewis\hspace{0.5cm} Odest Chadwicke Jenkins
% \thanks{Manuscript received: February 24, 2020; Accepted May 12, 2020.}%Use only for final RAL version
% \thanks{This paper was recommended for publication by Editor Hong Liu upon evaluation of the Associate Editor and Reviewers' comments.} %Use only for final RAL version
\thanks{\authorrefmark{1}X. Chen, H. Zhang, Z. Yu, S. Lewis and O. C. Jenkins are with the Department of Electrical Engineering and Computer Science, and Robotics Institute at the University of Michigan, Ann
Arbor, MI 48109 USA {\tt\footnotesize [cxt|huijiezh|yuzeren|stanlew|ocj] @umich.edu}}%
\thanks{Digital Object Identifier (DOI): see top of this page.}
}


% \markboth{IEEE Robotics and Automation Letters. Preprint Version. Accepted May, 2020}
% {Zhou \MakeLowercase{\textit{et al.}}: LIT: Light-field Inference of Transparency for Refractive Object Localization} 

\maketitle
% \thispagestyle{empty}
% \pagestyle{empty}
% \graphicspath{{images/}}
%%%%%%%%%%%%%%%%%%%%%%%%%%%%%%%%%%%%%%%%%%%%%%%%%%%%%%%%%%%%%%%%%%%%%%%%%%%%%%%%
\begin{abstract}
%\medskip
%\centering \textcolor{red}{Write the abstract last}
Silicon-compatible short- and mid-wave infrared emitters are highly sought-after for on-chip monolithic integration of electronic and photonic circuits to serve a myriad of applications in sensing and communication. To address this longstanding challenge, GeSn semiconductors have been proposed as versatile building blocks for silicon-integrated optoelectronic devices. In this regard, this work demonstrates light-emitting diodes (LEDs) consisting of a vertical PIN double heterostructure  p-Ge$_{0.94}$Sn$_{0.06}$/i-Ge$_{0.91}$Sn$_{0.09}$/n-Ge$_{0.95}$Sn$_{0.05}$ grown epitaxially on a silicon wafer using germanium interlayer and multiple GeSn buffer layers. The emission from these GeSn LEDs at variable diameters in the 40-120 $\mu$m range is investigated under both DC and AC operation modes. The fabricated LEDs exhibit a room temperature emission in the extended short-wave range centered around 2.5 $\mu$m under an injected current density as low as 45 A/cm$^2$.  By comparing the photoluminescence and electroluminescence signals, it is demonstrated that the LED emission wavelength is not affected by the device fabrication process or heating during the LED operation. Moreover, the measured optical power was found to increase monotonically as the duty cycle increases indicating that the DC operation yields the highest achievable optical power. The LED emission profile and bandwidth are also presented and discussed. 
\end{abstract}

% \begin{IEEEkeywords}
% Perception for Grasping and Manipulation, Grasping
% \end{IEEEkeywords}

\section{Introduction}
\section{Introduction}

Many problems in econometrics, statistics, causal inference, and finance involve linear functionals of unknown functions:
\begin{equation}
\theta(g)=\E[m(Z; g)]
\end{equation}
where $Z$ denotes a random vector, and $g: \mcX\to \R$ is a function in some space $ \mcG$. A continuous linear functional that is mean square continuous with respect to $\ell_2$ norm can be written in a more benign and useful manner. Formally, for a given linear functional $\theta(\cdot)$, there exists a function $a_0$ such that for any $g\in \mcG$:\footnote{For simplicity of exposition, throughout the paper we consider scalar-valued functions $g$. All our results naturally extend to vector-valued functions $g$, and estimate a vector valued Riesz representer that satisfies that $\theta(g)=\E[a(X)'g(X)]$.}
\begin{equation}
    \theta(g) = \E[a_0(X)\, g(X)]
\end{equation}
This result is known as the Riesz representation theorem, and the function $a_0$ is the Riesz representer of the linear functional. Evaluation of a linear functional $\theta(g)$ can be achieved by simply taking the inner product between $a_0$ and $g$.

Knowing the Riesz representation of a linear functional is a critical building block in a variety of learning problems. For instance, in semi-parametric models, $g_0$ is an unknown regression function and $\theta(g_0)$ is a causal or structural parameter of interest. The Riesz representer $a_0$ of the functional $\theta(\cdot)$ can be used to debias the plug-in estimator and construct semi-parametrically efficient estimators of the parameter $\theta(g_0)$. In asset pricing applications, the Riesz representer corresponds to the stochastic discount factor, which is of primary interest when pricing financial derivatives.

Irrespective of the downstream application, the goal of this paper is to derive an estimator for the Riesz representer of any linear functional, when given access to $n$ samples of the random vector $Z$ and a target function space $\mcA$ that can well approximate the function $a_0$. We propose and analyze an estimator $\hat{a}$, with small mean-squared-error. Formally, with probability (w.p.) $1-\zeta$:
\begin{equation}
    \|\hat{a}-a_0\|_2 = \sqrt{\E\left[\left(\hat{a}(X) - a_0(X)\right)^2\right]} \leq \epsilon_{n,\zeta}
\end{equation}

We consider estimation of the Riesz representer within some function space $\mcA$ and propose an adversarial estimator based on regularized variants of the following min-max criterion:
\begin{equation}
    \hat{a} = \argmin_{a\in \mcA} \max_{f\in \mcF} \frac{1}{n}\sum_{i=1}^n \left(m(Z_i;f) - a(X_i)\cdot f(X_i) - f(X_i)^2\right)
\end{equation}
We derive oracle inequalities for this estimator as a function of the localized Rademacher complexity of the function space $\mcA$ and the approximation error $\epsilon = \min_{a\in \mcA} \|a-a_0\|_{2}$.

We show that as long as the function class $\mcF$ contains the star-hull of differences of functions in $\mcA$, i.e. $\mcF:= \{r(a-a'): a, a'\in \mcA, r\in [0, 1]\}$, then the estimation rate of the adversarial estimator achieves w.p. $1-\zeta$:
\begin{equation}
    \|\hat{a} - a_0\|_2 = O\left(\epsilon + \delta_n + \sqrt{\frac{\log(1/\zeta)}{n}}\right)
\end{equation}
where $\delta_n$ is the critical radius of the function classes $\mcF$ and $m\circ \mcF=\{Z\to m(Z; f): f\in \mcF\}$. The critical radius of a function class is a widely used quantity in statistical learning theory that allows one to argue fast estimation rates that are nearly optimal. For instance, for parametric function classes, the critical radius is of order $n^{-1/2}$, leading to fast parametric rates (as compared to $n^{-1/4}$ which would be achievable via looser uniform deviation bounds).

Moreover, the critical radius has been analyzed and derived for a variety of function spaces of interest, such as neural networks, high-dimensional linear functions, reproducing kernel Hilbert spaces, and VC-subgraph classes. Thus our general theorem allows us to appeal to these characterizations and provide oracle rates for a family of Riesz representer estimators. Prior work on estimating Riesz representers only considered particular high-dimensional parametric classes and derived specialized estimators for the function space of interest. Our adversarial estimator provides a single approach that tackles generic function spaces in a uniform manner.

We also examine the computational aspect of our estimator. We provide examples of how estimation can be achieved in a computationally efficient manner for several function spaces of interest.

Finally, we show how our estimator can be used in the context of estimating causal or structural parameters in semi-parametric models. Specifically, our mean square rate for the Riesz representer is sufficiently fast to achieve semi-parametric efficiency and asymptotic normality of the causal or structural parameter.

\subsection{Applications: Causal Inference and Asset Pricing}\label{sec:intro_examples}

This learning problem arises in two important domains for economic research: causal inference and asset pricing.

\paragraph{Automated De-biasing of Causal Estimates.} In causal inference, a variety of treatment effects and policy effects can be formulated as functionals--i.e., scalar summaries--of an underlying regression \cite{chernozhukov2016locally}. Formally, the causal parameter $\theta_0=\theta(g_0)=\mathbb{E}[m(Z;g_0)]$ is a functional $\theta(\cdot)$ of the nuisance parameter $g_0(x):=\mathbb{E}[Y|X=x]$. In this paper, we consider a variety of treatment and policy effects including
\begin{enumerate}
    \item Average treatment effect (ATE): $\theta_0=\mathbb{E}[g_0(1,W)-g_0(0,W)]$, where $X=(D,W)$ consists of treatment and covariates.
    \item Average policy effect: $\theta_0=\int g_0(x)d\mu(x)$ where $\mu(x)=F_1(x)-F_0(x)$
    \item Policy effect from transporting covariates: $\theta_0=\mathbb{E}[g_0(t(X))-g_0(X)]$
    \item Cross effect: $\theta_0=\mathbb{E}[Dg_0(0,W)]$, where $X=(D,W)$ consists of treatment and covariates.
    \item Regression decomposition: $\mathbb{E}[Y|D=1]-\mathbb{E}[Y|D=0]=\theta_0^{response}+\theta_0^{composition}$
    where
    \begin{align}
        \theta_0^{response}&=\mathbb{E}[g_0(1,W)|D=1]-\mathbb{E}[g_0(0,W)|D=1] \\
        \theta_0^{composition}&=\mathbb{E}[g_0(0,W)|D=1]-\mathbb{E}[g_0(0,W)|D=0]
    \end{align}
    \item Average treatment on the treated (ATT): $\theta_0=\mathbb{E}[g_0(1,W)|D=1]-\mathbb{E}[g_0(0,W)|D=1]$, where $X=(D,W)$ consists of treatment and covariates.
    \item Local average treatment effect (LATE): $\theta_0=\frac{\mathbb{E}[g_0(1,W)-g_0(0,W)]}{\mathbb{E}[h_0(1,W)-h_0(0,W)]}$, where $X=(V,W)$ consists of instrument and covariates and $h_0(x):=\mathbb{E}[D|X=x]$ is a second regression.
\end{enumerate}
More generally, our results extend to parameters defined implicitly by $0=\mathbb{E}[m(Z;g_0;\theta_0)]$, such as partially linear regression and partially linear instrumental variable regression.

    If the regression $g_0$ is learned by a regularized estimator $\hat{g}$, then estimation of the causal parameter $\theta_0$  by a plug-in estimator $\mathbb{E}_n[m(Z;\hat{g})]$ is badly biased. The solution is to use a de-biased formulation of the causal parameter instead: $\theta_0=\mathbb{E}[m(Z;g_0)+a_0(X)\{Y-g_0(X)\}]$. Observe that $a_0$ arises in the bias correction term. We re-visit this example in Section~\ref{sec:debiasing}.

%

\paragraph{Fundamental Asset Pricing Equation.} In asset pricing, a variety of financial models deliver the same fundamental asset pricing equation. This equation is of both theoretical and practical interest. Theoretically, it elucidates why asset prices or returns are what they are. Practically, it can be used to identify trading opportunities when assets are mis-priced. The asset pricing equation follows from two weak assumptions: free portfolio formation, and the law of one price.  In Appendix~\ref{sec:finance}, we review the derivation for a general audience.\footnote{The same asset pricing equation can be derived from either a model of complete markets for contingent claims, or a model of investor utility maximization. Free portfolio formation is a weaker assumption on market structure than the existence of complete markets for contingent claims. The law of one price is a weaker assumption on preference structure than investor utility maximization. We present these additional derivations in Appendix~\ref{sec:finance}.}

Formally, the fundamental asset pricing equation is $p_{t,i}=\mathbb{E}_t[m_{t+1}x_{t+1,i}]$ where $p_{t,i}$ is the price of asset $i$ at time $t$, $x_{t+1,i}$ is payoff of asset $i$ at time $t+1$, and $m_{t+1}$ is the market-wide stochastic discount factor (SDF) at time $t+1$.\footnote{The SDF has many additional names: marginal rate of substitution, state price density, and pricing kernel. Each name corresponds to a different derivation of the asset pricing equation, starting from different first principles.} The expectation is conditional on information $(I_t,I_{t,i})$ known at time $t$:  $I_t$ are macroeconomic conditioning variables that are not asset specific, e.g. inflation rates and market return; $I_{t,i}$ are asset-specific characteristics, e.g. the size or book-to-market ratio of firm $i$ at time $t$. The asset pricing equation encompasses stocks, bonds, and options. We clarify its many instantiations below, where $d_{t+1}$ is dividend, $C$ is the call price, $S_T$ is the stock price at expiration, $K$ is the strike price. 

\begin{table}[H]
       \centering
       \begin{tabular}{|c||c|c|}
        \hline 
            Asset & Price $p_t$ & Payoff $x_{t+1}$ \\
             \hline 
            \hline
            Stock &$p_t$& $p_{t+1}+d_{t+1}$ \\
              Bond &$p_t$&$1$\\
             Option &$C$&$\max\{S_T-K,0\}$ \\
             \hline 
            Return & $1$& $R_{t+1}$ \\
            Excess return &0&$R^e_{t+1}$ \\
            \hline 
       \end{tabular}
       \caption{Generality of asset pricing equation}
       \label{tab:my_label}
   \end{table}
 
 The fundamental asset pricing equation can also be parametrized in terms of returns. If an investor pays one dollar for an asset $i$ today, the gross rate of return $R_{t+1,i}$ is how many dollars the investor receives tomorrow; formally, the price is $p_{t,i}=1$ and the payoff is $x_{t+1,i}=R_{t+1,i}$ by definition. Next consider what happens when an investor borrows a dollar today at the interest rate $R_{t+1}^f$ and buys an asset $i$ that gives the gross rate of return $R_{t+1,i}$ tomorrow. From the perspective of the investor, who paid nothing out-of-pocket, the price is $p_{t,i}=0$ while the payoff is the excess rate of return $R_{t+1,i}^e:=R_{t+1,i}-R_{t+1}^f$, leading to the asset pricing equation: $0=\mathbb{E}_t[m_{t+1}R^e_{t+1,i}]$.
 
 
 Following \cite{chen2019deep}, we focus on the latter excess return parametrization of the asset pricing equation. Taking expectations yields the unconditional moment restriction
$$
0=\mathbb{E}[m_{t+1}R^e_{t+1,i}z(I_t,I_{t,i})]=\mathbb{E}[\mathbb{E}[m_{t+1}|R^e_{t+1,i},I_t,I_{t,i}]R^e_{t+1,i}z(I_t,I_{t,i})],\quad \forall z(\cdot)
$$
Our framework nests this final expression. Specifically,
$$
\theta(g)=0,\quad g(R^e_{t+1,i},I_t,I_{t,i})=R^e_{t+1,i}z(I_t,I_{t,i}),\quad a_0(R^e_{t+1,i},I_t,I_{t,i})=\mathbb{E}[m_{t+1}|R^e_{t+1,i},I_t,I_{t,i}]
$$
By estimating $a_0$, which is the projection of the SDF onto excess returns and other available information, one can pin down the price of any hypothetical asset. 

%
%
%
%

\subsection{Related Work}

\textbf{Classical Semi-parametric Statistics.} Classical semi-parametric statistical theory studies the asymptotic properties of statistical quantities that are functionals of a density or a regression over a low-dimensional domain \cite{levit1976efficiency,hasminskii1979nonparametric,ibragimov1981statistical,pfanzagl1982lecture,klaassen1987consistent,robinson1988root,van1991differentiable,bickel1993efficient,newey1994asymptotic,robins1995semiparametric,vaart,bickel1988estimating,newey1998undersmoothing,ai2003efficient,newey2004twicing,ai2007estimation,tsiatis2007semiparametric,kosorok2007introduction,ai2012semiparametric}. Any continuous linear functional has a Riesz representer. In this classical theory, the Riesz representer appears in the influence function and therefore in the asymptotic variance of semi-parametric estimators \cite{newey1994asymptotic}. We depart from classical theory by considering the high-dimensional setting.

\textbf{De-biased Machine Learning and Targeted Maximum Likelihood.} Because the Riesz representer appears in the asymptotic variance of semi-parametric estimators, it can be incorporated into estimation to ensure semi-parametric efficiency. In practice, this can be achieved by introducing a de-biasing term into the estimating equation \cite{hasminskii1979nonparametric,bickel1988estimating,zhang2014confidence,belloni2011inference,belloni2014inference,belloni2014uniform,belloni2014pivotal,javanmard2014confidence,javanmard2014hypothesis,javanmard2018debiasing,van2014asymptotically,ning2017general,chernozhukov2015valid,neykov2018unified,ren2015asymptotic,jankova2015confidence,jankova2016confidence,jankova2018semiparametric,bradic2017uniform,zhu2017breaking,zhu2018linear}. In doubly robust estimating equations for regression functionals, the de-biasing term is the product between the Riesz representer and the regression residual \cite{robins1995analysis,robins1995semiparametric,van2006targeted,van2011targeted,luedtke2016statistical,toth2016tmle}. The more general principle at play is Neyman orthogonality: the learning problem for the functional of interest becomes orthogonal to the learning problems for both the regression and the Riesz representer \cite{neyman1959,neyman1979c,vaart,robins2008higher,zheng2010asymptotic,belloni2014uniform,belloni2014pivotal,chernozhukov2016locally,belloni2017program,chernozhukov2018double,foster2019orthogonal}.

De-biased machine learning and targeted maximum likelihood combine the algorithmic insight of doubly-robust moment functions with the algorithmic insight of sample splitting \cite{bickel1982adaptive,schick1986asymptotically,klaassen1987consistent,vaart,robins2008higher}.  In doing so, these frameworks facilitate a general analysis of residuals such that the target functional is $\sqrt{n}$-consistent under minimal assumptions on the estimators used for the regression and Riesz representer \cite{scharfstein1999adjusting,rubin2005general,rubin2006extending,van2006targeted,zheng2010asymptotic,van2011targeted,diaz2013targeted,van2014targeted,kennedy2017nonparametric,kennedy2020optimal}. In particular, any machine learning estimators are permitted that satisfy $\sqrt{n}\|\hat{g}-g_0\|_2\cdot\|\hat{a}-a_0\|_2\rightarrow 0$ \cite{chernozhukov2018double,chernozhukov2016locally}.

The Riesz representer may be a difficult object to estimate. Even for simple regression functionals such as policy effects, its closed form involves ratios of densities. In restricted models, where the regression is known to belong to a certain function class, there is the further difficulty of projecting the Riesz representer accordingly. A recent literature explores the possibility of directly estimating the Riesz representer, without estimating its components or even knowing its functional form \cite{robins2007comment,newey2018cross,athey2018approximate,chernozhukov2018global,chernozhukov2018learning,hirshberg2018debiased,hirshberg2019augmented,singh2019biased,rothenhausler2019incremental}. A crucial insight, on which we build, is that the Riesz representer is directly identified from data. 

\cite{hirshberg2019augmented} observe that to debias an average moment, it is sufficient to estimate an empirical analogue of the Riesz representer that approximately satisfies the Riesz representer moment equation on the $n$ samples. They propose a parametric min-max criterion to estimate $n$ parameters corresponding to the $n$ evaluations of the empirical Riesz representer. Unlike \cite{hirshberg2019augmented}, we provide a guarantee on learning the true Riesz representer, we approximate the Riesz representer within non-parametric function spaces, and our result therefore has broader application beyond causal inference. Importantly, \cite{hirshberg2019augmented} require that the same sample used to estimate the $n$ parameters is used in final stage estimation of the causal parameter. As such, the analysis requires that the regression function $g$ lies in a Donsker class--a restriction that precludes many machine learning estimators. By contrast, our adversarial estimator provides fast estimation rates with respect to the true Reisz representer and hence can be used in combination with cross-fitting and sample splitting to eliminate the Donsker assumption.


\textbf{Adversarial Estimation.} Riesz representation theorem can be viewed as a continuum of unconditional moment restrictions. The non-parametric instrumental variable problem, based on a conditional moment restriction, also implies a continuum of unconditional moment restrictions \cite{newey2003instrumental,hall2005nonparametric,blundell2007semi,chen2009efficient,darolles2011nonparametric,chen2012estimation,chen2015sieve,chen2018optimal}. A central insight of this work is that the min-max approach for conditional moment models may be adapted to the problem of learning the Riesz representer. In a min-max approach, the continuum of unconditional moment restrictions is enforced adversarially over a set of test functions \cite{goodfellow2014generative,arjovsky2017wasserstein,dikkala2020minimax}. 

The fundamental advantage of the min-max approach is its unified analysis over arbitrary function classes. In particular, via local Rademacher analysis, one can derive an abstract bound that encompasses sparse linear models, neural networks, and RKHS methods \cite{koltchinskii2000rademacher,bartlett2005local}. As such, the min-max approach is actually a family of algorithms adaptive to a variety of data settings with a unified guarantee \cite{negahban2012,lecue2017regularization,Lecue2018}. 

\textbf{Machine Learning Approaches to Causal Inference and Asset Pricing.} By pursuing a min-max approach, our work relates to previous work that incorporates a variety of machine learning methods into causal inference. Much work on de-biased machine learning focuses on sparse and approximately sparse models \cite{chernozhukov2018global,chernozhukov2018learning,chernozhukov2018plug}. A neural network estimator with mean square rate has been successfully used to learn the nuisance regression in semiparametric estimation \cite{chen1999improved,farrell2018deep} and to learn the structural function in nonparametric instrumental variable regression \cite{deepiv,bennett2019deep,dikkala2020minimax}. A more recent literature incorporates RKHS methods into causal inference due to their convenient closed form solutions and strong performance on smooth designs \cite{nie2017quasi,singh2019kernel,muandet2019dual,singh2020kernel,muandet2020kernel}.

Finally, our works provides a theoretical foundation for a growing literature that incorporates machine learning into asset pricing. We follow the asset pricing literature in framing the problem of learning a stochastic discount factor as the problem of learning a Riesz representer \cite{hansen1997assessing}. Specifically, we propose a deep min-max approach based on free portfolio formation and the law of one price \cite{bansal1993no,chen2019deep}. This approach differs from deep learning approaches that predict asset prices via nonparametric regression \cite{messmer2017deep,feng2018deep,gu2020autoencoder,bianchi2020bond}. Unlike previous work, we prove mean square rates for the stochastic discount factor, and we prove $\sqrt{n}$-consistency and semiparametric efficiency for expected asset prices.

\section{Related Work}
With ProgressLabeller, a user can scalably label new datasets with camera world pose, scene object poses and scene object segmentations. This process is enabled by fusing streaming RGB (or RGB-D) inputs into a single scene-wide representation, and then allowing a human user to input relevant 6-DoF information via 3D modelling interfaces (such as those provided by Blender \cite{blender}). This process demonstrates label stability even over long input video streams, and due to its functionality with direct RGB inputs, can label even difficult objects such as transparent cups.  We discuss below methods related to ProgressLabeller.

\subsection{Direct \& Human-in-the-loop labelling}
The creation of 2D segmentation data is analogous to the object detection, keypoint detection, or semantic segmentation tasks (depending on desired output labels). Tools such as LabelMe \cite{russell2008labelme} required users to directly interact with the underlying data to be labelled. This manual process was improved by model-assisted approaches such as Deep Extreme Cut \cite{maninis2018deep} which decreases the amount of user effort necessary to label images.
Shared autonomy and mixed-initiative methods have also been used in this approach, in which the user provides coarse pose or other estimations which are fine-tuned via a model-informed approach \cite{ye2021human}.

%%%%%%%%%%%%%%%%%%%%%%%%%%%%%%%
 
\subsection{End-to-End Labellers}
Previous tools have been created to enable this style of learning process. LabelFusion \cite{marion2018label} is perhaps the most commonly utilized example. LabelFusion utilizes streaming RGB-D inputs to create a dense reconstruction of the scene, which is then labelled semi-manually by aligning 3D object models to the 3D reconstruction. While this approach is typically robust, it relies on RGB-D input for reconstruction, and experiences difficulties under certain regimes. In particular, transparent objects cause problems for commonly employed depth sensor technologies, and long-running input streams typically result in large amounts of 'drift'. 

Some methods have been introduced to eliminate the need for CAD models in the labelling process. Singh et al. \cite{singh2021rapid} proposed a method which utilizes user labelled keypoints and bounding boxes to generate pose and segmentation labels. This frees the system from dependency on CAD models, but requires user interaction directly with the images. SALT \cite{stumpf2021salt} proposed utilizing GrabCut to generate 3D bounding boxes and image segmentation labels for relevant scenes. This allows removing the dependency on object masks while also allows the labelling of dynamic scenes such as human gait videos. 
% This method requires point cloud inputs however, which precludes the labelling of RGB only image streams.

Other works sought to improve the labelling procedure itself. EasyLabel \cite{suchi2019easylabel} allows for semi-automatic labelling of scenes via sequentially added objects.  This process is scalable, and generates high quality labels. However, it requires tight physical control over the scene to be labelled, which is not always feasible to obtain. Objectron \cite{ahmadyan2021objectron} utilized modern smartphone's AR capabilities combined with human-labelled 3D bounding boxes to scalably create a large scale dataset. This method however is susceptible to label drift during long-duration input videos. KeyPose \cite{liu2020keypose} specifically sought to generate labelled datasets for transparent objects. This method utilized stereoscopic images taken from a robot armature in order to avoid the problems of typical depth cameras have with transparent objects.
% StereOBJ-1M \cite{liu2021stereobj} improves the data collection efficiency in a setting with two more static cameras and more fiducial markers in the scene.

% \section{Method}
\section{ProgressLabeller}
\section{Method}
Fig.~\ref{fig:framework} presents the illustration of the proposed \frameworkName.
In this section,  
we start by providing the problem definition of online CIL. Then, we describe the definition of the online prototype, the proposed online prototype equilibrium, and the proposed adaptive prototypical feedback. Finally, we propose an online prototype learning framework.

\subsection{Problem Definition}
Formally, online CIL considers a continuous sequence of tasks from a single-pass data stream $\mathfrak{D}=\left\{\mathcal{D}_1, \ldots, \mathcal{D}_T \right\} $, where $\mathcal{D}_t = \left\{ x_{i}, y_{i} \right\} ^{N_t}_{i=1} $ is the dataset of task $t$, and $T$ is the total number of tasks. Dataset $\mathcal{D}_t$ contains $N_t$ labeled samples, $y_{i}$ is the class label of sample $x_{i}$ and $y_{i} \in \mathcal{C}_t$, where $\mathcal{C}_t$ is the class set of task $t$ and the class sets of different tasks are disjoint. 
For replay-based methods, a memory bank is used to store a small subset of seen data, and we also maintain a memory bank $\mathcal{M}$ in our method.
At each time step of task $t$, the model receives a mini-batch data $X \cup X^\mathrm{b}$ for training, where $X$ and $X^\mathrm{b}$ are drawn from the i.i.d distribution $\mathcal{D}_t$ and the memory bank $\mathcal{M}$, respectively. 
Moreover, we adopt the single-head evaluation setup~\cite{EWC}, where a unified classifier must choose labels from all seen classes at inference due to unavailable task identifiers. 
The goal of online CIL is to train a unified model on data seen only once while predicting well on both new and old classes.

\subsection{Online Prototype Definition}
Prior to introducing the online prototypes, we first present the network architecture of our \frameworkName. Suppose that the model consists of three components: an encoder network $f$, a projection head $g$, and a classifier $\varphi$. Each sample $x$ in incoming data $X$ (a mini-batch data from new classes) is mapped to a projected vectorial embedding (representation) $\mathbf{z}$ by encoder $f$ and projector $g$:
\begin{align}
\label{eq:cal_z}
    \mathbf{z} = g(f(\operatorname{aug}(x);\theta_f);\theta_g),
\end{align}
where $\operatorname{aug}$ represents the data augmentation operation, $\theta_f$ and $\theta_g$ represent the parameters of $f$ and $g$, respectively, and $\mathbf{z}$ is $\ell_2$-normalized. 
Similar to Eq.~\eqref{eq:cal_z}, we use $\mathbf{z}^\mathrm{b}$ to denote the representation of replay data $X^\mathrm{b}$ (a mini-batch data from seen classes in the memory bank). 

At each time step of task $t$, the online prototype of each class is defined as the mean representation in a mini-batch:
\begin{align}
\label{eq:cal_p}
    \mathbf{p}_i = \frac{1}{n_i}\sum\nolimits_j\mathbf{z}_j\cdot \mathbbm{1}\{y_j = i\},
\end{align}
where $n_i$ is the number of samples for class $i$ in a mini-batch, and $\mathbbm{1}$ is the indicator function. 
We can get a set of $K$ online prototypes  in $X$, $\mathcal{P} = \left\{ \mathbf{p}_{i} \right\} ^{K}_{i=1}$, and a set of $K^\mathrm{b}$ online prototypes in $X^\mathrm{b}$, $\mathcal{P}^\mathrm{b} = \left\{ \mathbf{p}_i^\mathrm{b} \right\} ^{K^\mathrm{b}}_{i=1}$.
Note that $K = |\mathcal{P}| \leq |\mathcal{C}_t|$ and $K^\mathrm{b} = |\mathcal{P}^\mathrm{b}| \leq \sum_{i=1}^{t}|\mathcal{C}_i| $, where $|\cdot|$ denotes the cardinal number.



\subsection{Online Prototype Equilibrium}
The introduced online prototypes can provide representative features and avoid class-unrelated information.  
These characteristics are exactly the key to counteracting shortcut learning in online CL.
Besides, maintaining the discrimination among seen classes is also essential to mitigate catastrophic forgetting.
Based on these, we attempt to learn representative features of each class by pulling online prototypes $\mathcal{P}$ and their augmented views $\widehat{\mathcal{P}}$ closer in the embedding space, and learn discriminative features between classes by pushing online prototypes of different classes away, formally defined as a contrastive loss:
\begin{align}
\label{eq:proto_infoNCE}
    \ell(\mathcal{P},\widehat{\mathcal{P}})\!=\!
    % \frac{-1}{K}
    \frac{-1}{|\mathcal{P}|}\sum_{i=1}^{|\mathcal{P}|}\!\log\! 
    \tfrac
    {\exp \big(\tfrac{{\mathbf{p}_i^\mathrm{T} \widehat{\mathbf{p}}_i}}{\tau}\big)}
    {
    \sum\limits_{j} \exp \big(\tfrac{{\mathbf{p}_i^\mathrm{T} \widehat{\mathbf{p}}_j}}{\tau}\big)
    +\!
    \sum\limits_{\substack{j \neq i}} \exp \big(\tfrac{{\mathbf{p}_i^\mathrm{T} \mathbf{p}_j}}{\tau}\big) 
    },
\end{align}
where 
$\tau$ is the temperature hyper-parameter, 
$\mathcal{P}$ and $\widehat{\mathcal{P}}$ are $\ell_2$-normalized. To compute the contrastive loss across all positive pairs in both $(\mathcal{P}, \widehat{\mathcal{P}})$ and $(\widehat{\mathcal{P}}, \mathcal{P})$, we define $\mathcal{L}_{\mathrm{pro}}$ as the final contrastive loss over online prototypes:
\begin{align}
    \mathcal{L}_{\mathrm{pro}}(\mathcal{P},\widehat{\mathcal{P}}) = 
    \frac{1}{2}
    \left[\ell(\mathcal{P}, \widehat{\mathcal{P}}) + \ell(\widehat{\mathcal{P}}, \mathcal{P})\right].
\end{align}



Considering the learning of new classes and the consolidation of learned knowledge simultaneously in online CL, we propose Online Prototype Equilibrium (\methodname) to 
learn representative and discriminative features on both new and seen classes by employing $\mathcal{L}_{\mathrm{pro}}$:
\begin{equation}
    \begin{aligned}
    \mathcal{L}_{\mathrm{\methodname}}
    &=
    \mathcal{L}^{\mathrm{new}}_{\mathrm{pro}}(\mathcal{P},\widehat{\mathcal{P}}) + \mathcal{L}^{\mathrm{seen}}_{\mathrm{pro}}(\mathcal{P}^\mathrm{b},\widehat{\mathcal{P}}^\mathrm{b}),
    \end{aligned}
\end{equation}
where
$\mathcal{L}^{\mathrm{new}}_{\mathrm{pro}}$ focuses on learning knowledge from \emph{new} classes, and $\mathcal{L}^{\mathrm{seen}}_{\mathrm{pro}}$ is dedicated to preserving learned knowledge of all \emph{seen} classes.
\textit{This process is similar to a zero-sum game, 
and \methodname aims to achieve the equilibrium to play a win-win game.}
Concretely,
as the model learns, the knowledge of new classes is gained and added to the prototypes over the memory bank $\mathcal{M}$, causing $\mathcal{L}^{\mathrm{seen}}_{\mathrm{pro}}$ gradually changes to the equilibrium that separates all seen classes well, including new ones. 
This variation is crucial to mitigate forgetting and is consistent with the goal of CIL.



\subsection{Adaptive Prototypical Feedback} 
Although \methodname can bring an overall equilibrium, it tends to treat each class \emph{equally}. 
In fact, the degree of confusion varies among classes, 
and the model should focus purposefully on confused classes to consolidate learned knowledge. 
To this end, we propose Adaptive Prototypical Feedback (\dataaugname) with the feedback of online prototypes to sense the classes that are prone to be misclassified and then enhance their decision boundaries.
 
For each class pair in the memory bank $\mathcal{M}$, \dataaugname calculates the distances between online prototypes of all seen classes from the previous time step, showing the class confusion status by these distances. The closer the two prototypes are, the easier to be misclassified. 
Based on this analysis, 
our idea is to enhance the boundaries for those classes. Therefore, we convert the prototype distance matrix to a probability distribution $P$ over the classes via a symmetric Gaussian kernel, defined as follows:
\begin{align}
\label{eq:cal_d}
    P_{i, j} \propto \exp (-\| \mathbf{p}_i^\mathrm{b} - \mathbf{p}_j^\mathrm{b} \|_2^2),
\end{align}
where $i,j \in \{ 1, \ldots, |\mathcal{P}^\mathrm{b}| \}$ and $i \neq j$. 
Then, 
all probabilities are normalized to a probability mass function that sums to one.
\dataaugname returns probabilities to $\mathcal{M}$ for guiding the next sampling process and enhancing decision boundaries of easily misclassified classes. 


Our adaptive prototypical feedback is implemented as a sampling-based mixup. Specifically, 
\dataaugname adaptively selects more samples from easily misclassified classes in $\mathcal{M}$ for mixup~\cite{Mixup} according to the probability distribution $P$. 
Considering not over-penalizing the equilibrium of current online prototypes, we introduce a two-stage sampling strategy for replay data $X^\mathrm{b}$ of size $m$. 
First, we select $n_{\mathrm{\dataaugname}}$ samples  
with $P$, and a larger $P_{a,b}$ means more sampling from classes $a$ and $b$. Here, $n_{\mathrm{\dataaugname}} = \alpha \cdot m$, and $\alpha$ is the ratio of \dataaugname.
Second, the remaining $m-n_{\mathrm{\dataaugname}}$ samples are uniformly randomly selected from the entire memory bank to avoid the model only focusing on easily misclassified classes and disrupting the established equilibrium. 




\subsection{Overall Framework of \frameworkName}
The overall structure of \frameworkName is shown in Fig.~\ref{fig:framework}. \frameworkName comprises two key components based on proposed online prototypes: Online Prototype Equilibrium (\methodname) and Adaptive Prototypical Feedback (\dataaugname). 
With the two components, 
the model can learn representative features against shortcut learning, and 
all seen classes maintain discriminative when learning new classes. 
However, classes may not be compact, because the online prototypes cannot cover full instance-level information.
To further achieve intra-class compactness, 
we employ supervised contrastive learning~\cite{SupCL} to learn instance-wise representations:
\begin{equation}
\begin{aligned}
    \mathcal{L}_{\mathrm{INS}}
    &=
    \sum_{i=1}^{2N} \frac{-1}{\left|I_i\right|} \sum_{j \in I_i} \log \frac{\exp \left(\mathrm{sim}(\mathbf{z}_i, \mathbf{z}_j) / \tau^{\prime}\right)}{\sum\limits_{k \neq i} \exp \left(\mathrm{sim}(\mathbf{z}_i, \mathbf{z}_k) / \tau^{\prime}\right)}
    \\
    &+
    \sum_{i=1}^{2m} \frac{-1}{\left|I_i^{\mathrm{b}}\right|} \sum_{j \in I_i^{\mathrm{b}}} \log \frac{\exp (\mathrm{sim}(\mathbf{z}_i^{\mathrm{b}}, \mathbf{z}_j^{\mathrm{b}}) / \tau^{\prime})}{\sum\limits_{k \neq i} \exp \left(\mathrm{sim}(\mathbf{z}_i^{\mathrm{b}}, \mathbf{z}_k^{\mathrm{b}}) / \tau^{\prime}\right)},
\end{aligned}
\end{equation}
where $I_i=\left\{j \in\{1, \ldots, 2 N\} \mid j \neq i, y_j=y_i\right\}$ and $I_i^\mathrm{{b}}=\left\{j \in\{1, \ldots, 2m\} \mid j \neq i, y_j^\mathrm{b}=y_i^\mathrm{b}\right\}$ are the set of positive samples indexes to $\mathbf{z}_i$ and $\mathbf{z}_i^\mathrm{{b}}$, respectively. $y_i^\mathrm{b}$ is the class label of input $x_i^\mathrm{b}$ from $X^\mathrm{b}$. $N$ is the batch size of $X$. $\tau^{\prime}$ is the temperature hyperparameter.
The similarity function $\mathrm{sim}$ is computed in the same way as Eq.~(9) in OCM~\cite{OCM}.

Thus, the total loss of our \frameworkName framework is given as:
\begin{align}
    \mathcal{L}_{\mathrm{\frameworkName}}=\mathcal{L}_{\mathrm{\methodname}} + \mathcal{L}_{\mathrm{INS}} + \mathcal{L}_{\mathrm{CE}},
\end{align}
where $\mathcal{L}_{\mathrm{CE}} = \mathrm{CE}(y^\mathrm{b}, \varphi(f(\operatorname{aug}(x^\mathrm{b}))))$ is the cross-entropy loss; see Appendix~\ref{appendix:algorithm} for detailed training algorithms.

Following other replay-based methods~\cite{ER, SCR, OCM}, we update the memory bank in each time step by uniformly randomly selecting samples from $X$ to push into $\mathcal{M}$ and, if $\mathcal{M}$ is full, pulling an equal number of samples out of $\mathcal{M}$.

\subsection{Annotation accuracy estimation from simulation}

We verify the accuracy of annotations throughout this multi-view silhouette matching process by simulating an iterative object pose update process. In each iteration, given a certain camera frame, 
we assume the object will be translated in a plane parallel with its x-y plane, or rotated about z-axis (for better control), towards a pose that maximizes the Intersection-over-Union (IoU) between the rendered silhouette at current pose and ground truth.
% we assume the object will be translated or rotated to maximize the IoU between the object silhouette rendered at current pose and ground truth from that camera view.

\subsubsection{Problem Definition} Given set of $N$ images $I^{\{i\}}$ with their corresponding camera pose $T^{\{i\}}$ in the world frame, $i \in \{1, 2, 3, \ldots, N\}$. Our goal is to find ground truth object pose $T^{\text{obj}\{j\}}_{gt}$ in the world frame for all the objects $j \in \{1, 2, 3, \ldots, M\}$ in the scene. We define the projection operator as $S^{\{i, j\}} = \text{Proj}(T^{\{i\}}, T^{\text{obj}\{j\}})$, which render object $j$ given its CAD model, camera pose $T^{\{i\}}$ and object pose $T^{\text{obj}\{j\}}$ into an object texture/silouette $S^{\{i, j\}}$. Also defined the IoU operator as $\text{IoU}_{\text{obj}\{j\}}(I^{\{i\}}, \text{Proj}(T^{\{i\}}, T^{\text{obj}\{j\}}))$ to calculate the IoU for pixels in object $j$ between real image $I^{\{i\}}$ and synthetic texture/silouette $S^{\{i, j\}}$. 

% \begin{equation}
%     T^{\text{obj}\{j\}}_{gt} = \argmax_{T^{\text{obj}\{j\}}}\text{IoU}_{\text{obj}\{j\}}(I^{\{i\}}, \text{Proj}(T^{\{i\}}, T^{\text{obj}\{j\}}))
%     \label{eq:fact}
% \end{equation}

The multi-view texture/silhouette matching iterative update is proceeded with a goal to maximize the IoU. Given the pose for object $j$ in the $k$th iteration $T^{\text{obj}\{j\}}_{(k)}$, in ($k + 1$)th iteration:

\begin{equation}
    T^{\text{obj}\{j\}}_{(k+1)} = \argmax_{f[T^{\text{obj}\{j\}}_{(k)}]}\text{IoU}_{\text{obj}\{j\}}(I^{\{i\}}, \text{Proj}(T^{\{i\}}, f[T^{\text{obj}\{j\}}_{(k)}] )) 
    \label{eq:iterate}
\end{equation}
where $f[T^{\text{obj}\{j\}}_{(k)}]$ describes all possible translation start from the  $T^{\text{obj}\{j\}}_{(k)}$ 
% To limit the annotation freedom for a better human control, when viewing perpendicular to one view, only the translation 
that is within the plane $p$ or the rotation along the axis $\omega$ as shown in Figure \ref{fig:annotation_limitation}. So:

\begin{equation}
    f[T^{\text{obj}\{j\}}_{(k)}] = \exp^{\widehat{\xi}_1 \theta_1}  \exp^{\widehat{\xi}_2 \theta_2} T^{\text{obj}\{j\}}_{(k)}
    \label{eq:range}
\end{equation}
where ${\xi}_1 = \begin{bmatrix}-\omega \times v_o \\ \omega \end{bmatrix}$, ${\xi}_2 = \begin{bmatrix} v \\ 0 \end{bmatrix}$ are the twist coordinates for twist $\widehat{\xi}_1, \widehat{\xi}_2$.

% After plug Equation \eqref{eq:range} to Equation \eqref{eq:iterate}, we could get:

% \begin{align}
%     &T^{\text{obj}\{j\}}_{(k+1)} = \argmax_{\theta_1, \theta_2, v}\text{IoU}_{\text{obj}\{j\}}(I^{\{i\}}, \text{Proj}(T^{\{i\}}, T(\theta_1, \theta_2, v) )) \\
%     &T(\theta_1, \theta_2, v) = \exp^{\widehat{\xi}_1 \theta_1}  \exp^{\widehat{\xi}_2 \theta_2} T^{\text{obj}\{j\}}_{(k)}
%     \label{eq:iterate}
% \end{align}

% A more intuitively understanding is that during each iteration, we rotate the object along the axis $\omega$ and translate it within the plane $p$ to maximize the IoU between the projection silhouette $S^{\{i, j\}}$ and real image $I^{\{i\}}$. Simulation experiments in the coming section will demonstrate the convergence of this problem.


 \begin{figure}[htbp]
     \centering
     \includegraphics[width=0.8\columnwidth]{figure/annotation_limitation_resize.pdf}
     \caption{Diagram for an object shown under a camera. $c$ denote the location of camera and $c_x, c_y, c_z$ are its x, y, z axis. $p$ is a plane parallel to camera plane and passing through object's center $v_o$. $\omega$ the rotation direction parallel to $c_x$ and passing through $v_o$. $\theta_1$ is the magnitude of rotation radius. $v$ is the translation direction within the plane $p$ and $\theta_2$ is the translation magnitude. On the right hand side is the projection image, the object in the transparent color is the object with ground truth pose.}
     \label{fig:annotation_limitation}
 \end{figure}

\subsubsection{Simulation results}

We generate a CAD model sets with 44 different CAD models. For each run, we generate $T^{\text{obj}}_{gt}$ with a random rotation matrix and location at the origin. 40 cameras are created with their z axis pointing towards the origin and a random location at a sphere around the object. The initial pose $T^{\text{obj}}_{0}$ is generated by adding a random position noise from Gaussian distribution with variance of 10cm to origin and with a random 3D orientation. During each iteration, $v, \theta_1, \theta_2$ in Equation \ref{eq:range} are discretized for simulation. The result shows that it takes around 10.36 iterations for the algorithm to converge within 1mm location error and dot product larger than 0.99 between ground truth and converged rotation axes.

\section{Experiments}

\begin{figure*}
    \centering
    \includegraphics[width=1.0\linewidth]{Figures/imgs/tsne_motivation.pdf}
    \caption{$t$-SNE~\cite{tsne} visualizations of features learned from ER and \frameworkName on the test set of CIFAR-10.
    When learning new classes, ER suffers serious class confusion probably because shortcut learning. In contrast, \frameworkName significantly mitigates the forgetting.
    }
    \label{fig:tsne_motivation}
\end{figure*}
\begin{table*}[ht]
\small
\begin{center}
\resizebox{\linewidth}{!}{
\begin{tabular}{rrrrrrrrrrrr}
\shline
\multirow{2}{*}{Method}  & \multicolumn{3}{c}{CIFAR-10}   && \multicolumn{3}{c}{CIFAR-100}  && \multicolumn{3}{c}{TinyImageNet} \\ \cline{2-4}\cline{6-8}\cline{10-12}
       & $M=0.1k$   & $M=0.2k$   & $M=0.5k$     && $M=0.5k$     & $M=1k$     & $M=2k$     && $M=1k$      & $M=2k$ & $M=4k$   \\ \midrule
iCaRL~\cite{iCaRL}    & 31.0\std{$\pm$1.2} & 33.9\std{$\pm$0.9} & 42.0\std{$\pm$0.9} && 12.8\std{$\pm$0.4}  & 16.5\std{$\pm$0.4}  & 17.6\std{$\pm$0.5} && 5.0\std{$\pm$0.3}   & 6.6\std{$\pm$0.4} & 7.8\std{$\pm$0.4} \\ 
DER++~\cite{DER++}   & 31.5\std{$\pm$2.9} & 39.7\std{$\pm$2.7} & 50.9\std{$\pm$1.8} && 16.0\std{$\pm$0.6}  & 21.4\std{$\pm$0.9}  & 23.9\std{$\pm$1.0} && 3.7\std{$\pm$0.4} & 5.1\std{$\pm$0.8} & 6.8\std{$\pm$0.6} \\ 
PASS~\cite{protoAug}    & 33.7\std{$\pm$2.2} & 33.7\std{$\pm$2.2} & 33.7\std{$\pm$2.2} && 7.5\std{$\pm$0.7}  & 7.5\std{$\pm$0.7}  & 7.5\std{$\pm$0.7} && 0.5\std{$\pm$0.1}   & 0.5\std{$\pm$0.1} & 0.5\std{$\pm$0.1} \\ 
\hline
AGEM~\cite{AGEM}    & 17.7\std{$\pm$0.3} & 17.5\std{$\pm$0.3} & 17.5\std{$\pm$0.2} && 5.8\std{$\pm$0.1}  & 5.9\std{$\pm$0.1}  & 5.8\std{$\pm$0.1} && 0.8\std{$\pm$0.1}   & 0.8\std{$\pm$0.1} & 0.8\std{$\pm$0.1} \\ 
GSS~\cite{GSS}     & 18.4\std{$\pm$0.2} & 19.4\std{$\pm$0.7} & 25.2\std{$\pm$0.9} && 8.1\std{$\pm$0.2}  & 9.4\std{$\pm$0.5}  & 10.1\std{$\pm$0.8} && 1.1\std{$\pm$0.1}   & 1.5\std{$\pm$0.1} & 2.4\std{$\pm$0.4} \\ 
ER~\cite{ER}      & 19.4\std{$\pm$0.6} & 20.9\std{$\pm$0.9} & 26.0\std{$\pm$1.2} && 8.7\std{$\pm$0.3}  & 9.9\std{$\pm$0.5}  & 10.7\std{$\pm$0.8} && 1.2\std{$\pm$0.1}   & 1.5\std{$\pm$0.2} & 2.0\std{$\pm$0.2} \\ 
MIR~\cite{MIR}     & 20.7\std{$\pm$0.7} & 23.5\std{$\pm$0.8} & 29.9\std{$\pm$1.2} && 9.7\std{$\pm$0.3}  & 11.2\std{$\pm$0.4}  & 13.0\std{$\pm$0.7} && 1.4\std{$\pm$0.1}   & 1.9\std{$\pm$0.2} & 2.9\std{$\pm$0.3} \\ 
GDumb~\cite{GDumb}   & 23.3\std{$\pm$1.3} & 27.1\std{$\pm$0.7} & 34.0\std{$\pm$0.8} && 8.2\std{$\pm$0.2}  & 11.0\std{$\pm$0.4}  & 15.3\std{$\pm$0.3} && 4.6\std{$\pm$0.3}   & 6.6\std{$\pm$0.2} & 10.0\std{$\pm$0.3} \\ 
ASER~\cite{ASER}   & 20.0\std{$\pm$1.0} & 22.8\std{$\pm$0.6} & 31.6\std{$\pm$1.1} && 11.0\std{$\pm$0.3}  & 13.5\std{$\pm$0.3}  & 17.6\std{$\pm$0.4} && 2.2\std{$\pm$0.1}   & 4.2\std{$\pm$0.6} & 8.4\std{$\pm$0.7} \\ 
SCR~\cite{SCR}     & 40.2\std{$\pm$1.3} & 48.5\std{$\pm$1.5} & 59.1\std{$\pm$1.3} && 19.3\std{$\pm$0.6}  & 26.5\std{$\pm$0.5}  & 32.7\std{$\pm$0.3} && 8.9\std{$\pm$0.3}   & 14.7\std{$\pm$0.3} & 19.5\std{$\pm$0.3} \\ 
CoPE~\cite{online_pro_ema}  & 33.5\std{$\pm$3.2} & 37.3\std{$\pm$2.2} & 42.9\std{$\pm$3.5} && 11.6\std{$\pm$0.7}  & 14.6\std{$\pm$1.3}  & 16.8\std{$\pm$0.9} && 2.1\std{$\pm$0.3}   & 2.3\std{$\pm$0.4} & 2.5\std{$\pm$0.3} \\
DVC~\cite{DVC} & 35.2\std{$\pm$1.7}  & 41.6\std{$\pm$2.7} & 53.8\std{$\pm$2.2} &&  15.4\std{$\pm$0.7} & 20.3\std{$\pm$1.0} & 25.2\std{$\pm$1.6} && 4.9\std{$\pm$0.6} &  7.5\std{$\pm$0.5} & 10.9\std{$\pm$1.1} \\ 
OCM~\cite{OCM} & 47.5\std{$\pm$1.7}  & 59.6\std{$\pm$0.4} & 70.1\std{$\pm$1.5} && 19.7\std{$\pm$0.5} & 27.4\std{$\pm$0.3} & 34.4\std{$\pm$0.5} && 10.8\std{$\pm$0.4} & 15.4\std{$\pm$0.4} & 20.9\std{$\pm$0.7} \\ 
\hline
\frameworkName (\textbf{ours}) & \textbf{57.8}\std{$\pm$1.1} & \textbf{65.5}\std{$\pm$1.0} & \textbf{72.6}\std{$\pm$0.8} && \textbf{22.7}\std{$\pm$0.7} & \textbf{30.0}\std{$\pm$0.4} & \textbf{35.9}\std{$\pm$0.6} && \textbf{11.9}\std{$\pm$0.3} & \textbf{16.9}\std{$\pm$0.4} &  \textbf{22.1}\std{$\pm$0.4}
\\ 
\shline
\end{tabular}
}
\end{center}
\caption{Average Accuracy~(higher is better) on three benckmark datasets with different memory bank sizes $M$. All results are the average and standard deviation of 15 runs.}
\label{tab:acc}
\end{table*}

\section{Experiments}
\subsection{Experimental Setup}
\paragraph{Datasets.}
We use three image classification benchmark datasets, including \textbf{CIFAR-10}~\cite{cifar10_100}, \textbf{CIFAR-100}~\cite{cifar10_100}, and \textbf{TinyImageNet}~\cite{tinyImageNet}, to evaluate the performance of online CIL methods. 
Following~\cite{ASER, SCR, DVC}, we split CIFAR-10 into 5 disjoint tasks, where each task has 2 disjoint classes, 10,000 samples for training, and 2,000 samples for testing, and split CIFAR-100 into 10 disjoint tasks, where each task has 10 disjoint classes, 5,000 samples for training, and 1,000 samples for testing.
Following~\cite{OCM}, we split TinyImageNet into 100 disjoint tasks, where each task has 2 disjoint classes, 1,000 samples for training, and 100 samples for testing.
Note that the order of tasks is fixed in all experimental settings.

\paragraph{Baselines.}
We compare our \frameworkName with 13 baselines, including 10 replay-based online CL baselines: {AGEM}~\cite{AGEM}, {MIR}~\cite{MIR}, {GSS}~\cite{GSS}, {ER}~\cite{ER}, {GDumb}~\cite{GDumb}, {ASER}~\cite{ASER}, {SCR}~\cite{SCR}, {CoPE}~\cite{online_pro_ema}, {DVC}~\cite{DVC}, and {OCM}~\cite{OCM}; 3 offline CL baselines that use knowledge distillation by running them in one epoch: {iCaRL}~\cite{iCaRL}, {DER++}~\cite{DER++}, and PASS~\cite{protoAug}. Note that PASS is a non-exemplar method.

\paragraph{Evaluation metrics.}
We use Average Accuracy and Average Forgetting~\cite{ASER, DVC} to measure the performance of our framework in online CIL. Average Accuracy evaluates the accuracy of the test sets from all seen tasks, defined as $\text {Average Accuracy} =\frac{1}{T} \sum_{j=1}^T a_{T, j},$
where $a_{i, j}$ is the accuracy on task $j$ after the model is trained from task $1$ to $i$.
Average Forgetting represents how much the model forgets about each task after being trained on the final task, defined as
$\text { Average Forgetting } =\frac{1}{T-1} \sum_{j=1}^{T-1} f_{T, j}, 
\text { where } f_{i, j}=\max _{k \in\{1, \ldots, i-1\}} a_{k, j}-a_{i, j}.$

\paragraph{Implementation details.}
We use ResNet18~\cite{ResNet} as the backbone $f$ and a linear layer as the projection head $g$ like~\cite{SCR, OCM, Co2L}; the hidden dim in $g$ is set to 128 as~\cite{SimCLR}. We also employ a linear layer as the classifier $\varphi$. We train the model from scratch with Adam optimizer and an initial learning rate of $5\times10^{-4}$ for all datasets. The weight decay is set to $1.0\times10^{-4}$. Following~\cite{ASER, DVC}, we set the batch size $N$ as 10, and following~\cite{OCM} the replay batch size $m$ is set to 64. 
For CIFAR-10, we set the ratio of \dataaugname $\alpha = 0.25$. For CIFAR-100 and TinyImageNet, $\alpha$ is set to $0.1$. The temperature $\tau = 0.5$ and $\tau^{\prime} = 0.07$.
For baselines, we also use ResNet18 as their backbone and set the same batch size and replay batch size for fair comparisons.
We reproduce all baselines in the same environment with their source code and default settings; see Appendix~\ref{appendix:baselines} for implementation details about all baselines.
We report the average results across 15 runs for all experiments.



\paragraph{Data augmentation.}
Similar to data augmentations used in SimCLR~\cite{SimCLR}, we use resized-crop, horizontal-flip, and gray-scale as our data augmentations. For all baselines, we also use these augmentations. In addition, for DER++\cite{DER++}, SCR~\cite{SCR}, and DVC~\cite{DVC}, we follow their default settings and use their own extra data augmentations. OCM~\cite{OCM} uses extra rotation augmentations, which are also used in \frameworkName.


\subsection{Motivation Justification}
\label{pre_exp}
\paragraph{Shortcut learning in online CL.}
Shortcut learning is severe in online CL since the model cannot learn sufficient representative features due to the single-pass data stream. To intuitively demonstrate this issue,  
we conduct GradCAM++~\cite{Grad-cam++} on the training set of CIFAR-10 ($M=0.2k$) after the model is trained incrementally, as shown in Fig.~\ref{fig:heatmap}.
Each row in Fig.~\ref{fig:heatmap} represents a task with two classes.
We can observe that although ER and DVC predict the correct class, the models actually take shortcuts and focus on some object-unrelated features. 
An interesting phenomenon is that ER tends to take shortcuts in each task. For example, ER learns the sky on both the airplane class in task 1 (the first row) and the bird class in task 2 (the second row) . Thus, ER forgets almost all the knowledge of the old classes.  
DVC maximizes the mutual information between instances like contrastive learning~\cite{SimCLR, MoCo}, which only partially alleviates shortcut learning in online CL. 
In contrast, \frameworkName focuses on the representative features of the objects themselves. The results confirm that learning representative features is crucial against shortcut learning; see Appendix~\ref{appendix:more_visual} for more visual explanations.


\begin{table*}[htbp]
\small
\begin{center}
\resizebox{\linewidth}{!}{
\begin{tabular}{rrrrrrrrrrrr}
\shline
\multirow{2}{*}{Method}  & \multicolumn{3}{c}{CIFAR-10}   && \multicolumn{3}{c}{CIFAR-100}  && \multicolumn{3}{c}{TinyImageNet} \\ \cline{2-4}\cline{6-8}\cline{10-12}
       &  $M=0.1k$   &  $M=0.2k$   &  $M=0.5k$     &&  $M=0.5k$     &  $M=1k$     &  $M=2k$    &&  $M=1k$      &  $M=2k$ &  $M=4k$    \\ \midrule
iCaRL~\cite{iCaRL}    & 52.7\std{$\pm$1.0} & 49.3\std{$\pm$0.8} & 38.3\std{$\pm$0.9} && 16.5\std{$\pm$1.0}  & 11.2\std{$\pm$0.4}  & 10.4\std{$\pm$0.4} && 9.9\std{$\pm$0.5}   & 10.1\std{$\pm$0.5} & 9.7\std{$\pm$0.6} \\ 
DER++~\cite{DER++}   & 57.8\std{$\pm$4.1} & 46.7\std{$\pm$3.6} & 33.6\std{$\pm$3.5} && 41.0\std{$\pm$1.1} & 34.8\std{$\pm$1.1} & 33.2\std{$\pm$1.2} && 77.8\std{$\pm$1.0} & 74.9\std{$\pm$0.6} & 73.2\std{$\pm$0.8}  \\ 
PASS~\cite{protoAug}    & 21.2\std{$\pm$2.2} & 21.2\std{$\pm$2.2} & 21.2\std{$\pm$2.2} && 10.6\std{$\pm$0.9}  & 10.6\std{$\pm$0.9}  & 10.6\std{$\pm$0.9} && 27.0\std{$\pm$2.4}   & 27.0\std{$\pm$2.4} & 27.0\std{$\pm$2.4} \\ 
\hline
AGEM~\cite{AGEM}    & 64.8\std{$\pm$0.7} & 64.8\std{$\pm$0.7} & 64.5\std{$\pm$0.5} && 41.7\std{$\pm$0.8} & 41.8\std{$\pm$0.7} & 41.7\std{$\pm$0.6} && 73.9\std{$\pm$0.7} & 73.1\std{$\pm$0.7} & 72.9\std{$\pm$0.5} \\ 
GSS~\cite{GSS}     & 67.1\std{$\pm$0.6} & 65.8\std{$\pm$0.6} & 61.2\std{$\pm$1.2} && 48.7\std{$\pm$0.8} & 46.7\std{$\pm$1.3} & 44.7\std{$\pm$1.1} && 78.9\std{$\pm$0.7} & 77.0\std{$\pm$0.5} & 75.2\std{$\pm$0.7} \\ 
ER~\cite{ER}      & 64.7\std{$\pm$1.1} & 62.9\std{$\pm$1.0} & 57.5\std{$\pm$1.8} && 47.0\std{$\pm$1.0} & 46.4\std{$\pm$0.8} & 44.7\std{$\pm$1.5} && 79.1\std{$\pm$0.6} & 77.7\std{$\pm$0.6} & 76.3\std{$\pm$0.5} \\ 
MIR~\cite{MIR}     & 62.6\std{$\pm$1.0} & 58.5\std{$\pm$1.4} & 51.1\std{$\pm$1.1} && 45.7\std{$\pm$0.9} & 44.2\std{$\pm$1.3} & 42.3\std{$\pm$1.0} && 75.3\std{$\pm$0.9} & 71.5\std{$\pm$1.0} & 66.8\std{$\pm$0.8} \\ 
GDumb~\cite{GDumb}   & 28.5\std{$\pm$1.4} & 28.4\std{$\pm$1.0} & 28.1\std{$\pm$1.0} && 25.0\std{$\pm$0.4} & 23.2\std{$\pm$0.4} & 20.7\std{$\pm$0.3}  && 22.7\std{$\pm$0.3} & 18.4\std{$\pm$0.2} & 17.0\std{$\pm$0.2} \\
ASER~\cite{ASER}    & 64.8\std{$\pm$1.0} & 62.6\std{$\pm$1.1} & 53.2\std{$\pm$1.5} && 52.8\std{$\pm$0.8} & 50.4\std{$\pm$0.9} & 46.8\std{$\pm$0.7} && 78.9\std{$\pm$0.5} & 75.4\std{$\pm$0.7} & 68.2\std{$\pm$1.1} \\ 
SCR~\cite{SCR}     & 43.2\std{$\pm$1.5} & 35.5\std{$\pm$1.8} & 24.1\std{$\pm$1.0} && 29.3\std{$\pm$0.9} & 20.4\std{$\pm$0.6} & 11.5\std{$\pm$0.6} && 44.8\std{$\pm$0.6} & 26.8\std{$\pm$0.5} & 20.1\std{$\pm$0.4} \\ 
CoPE~\cite{online_pro_ema}  & 49.7\std{$\pm$1.6} & 45.7\std{$\pm$1.5} & 39.4\std{$\pm$1.8} && 25.6\std{$\pm$0.9}  & 17.8\std{$\pm$1.3}  & 14.4\std{$\pm$0.8} && 11.9\std{$\pm$0.6}   & 10.9\std{$\pm$0.4} & 9.7\std{$\pm$0.4} \\
DVC~\cite{DVC} & 40.2\std{$\pm$2.6} & 31.4\std{$\pm$4.1} & 21.2\std{$\pm$2.8} && 32.0\std{$\pm$0.9} & 32.7\std{$\pm$2.0} & 28.0\std{$\pm$2.2} && 59.8\std{$\pm$2.2} & 52.9\std{$\pm$1.3} & 45.1\std{$\pm$1.9} \\
OCM~\cite{OCM} & 35.5\std{$\pm$2.4} & 23.9\std{$\pm$1.4} & 13.5\std{$\pm$1.5} && 18.3\std{$\pm$0.9} & 15.2\std{$\pm$1.0} & 10.8\std{$\pm$0.6} && 23.6\std{$\pm$0.5} & 26.2\std{$\pm$0.5}  & 23.8\std{$\pm$1.0} \\ 
\hline
{\frameworkName} (\textbf{ours})   & 23.2\std{$\pm$1.3} & 17.6\std{$\pm$1.4} & 12.5\std{$\pm$0.7} && 
15.0\std{$\pm$0.8} & 10.4\std{$\pm$0.5} & 6.1\std{$\pm$0.6} && 21.3\std{$\pm$0.5} & 17.4\std{$\pm$0.4} & 16.8\std{$\pm$0.4} \\
\shline
\end{tabular}
}
\end{center}
\caption{Average Forgetting~(lower is better) on three benckmark datasets. All results are the average and standard deviation of 15 runs.}
\label{tab:forget}
\end{table*}

\begin{figure*}[htp]
  \centering
  \subfloat[Average incremental performance]{
    \includegraphics[width=0.55\linewidth]{Figures/imgs/incremental_step_acc.pdf}
    \label{fig:incrementalAcc}
  }
  \subfloat[Confusion matrix of OCM and \frameworkName]{
    \includegraphics[width=0.42\linewidth]{Figures/imgs/confusion_matrix.pdf}
    \label{fig:confusionMatrix}
  }
  \caption{Incremental accuracy on tasks observed so far and confusion matrix of accuracy (\%) in the {test set} of CIFAR-10.}
  \label{fig:incrementalAcc_confusionMatrix}
\end{figure*}



\paragraph{Class confusion in online CL.}
Fig.~\ref{fig:tsne_motivation} provides the $t$-SNE~\cite{tsne} visualization results for ER and \frameworkName on the test set of CIFAR-10 ($M=0.2k$). 
We can draw intuitive observations as follows. 
(1) There is serious class confusion in ER.
When the new task (task 2) arrives, features learned in task 1 are not discriminative for task 2, leading to class confusion and decreased performance in old classes.
(2) Shortcut learning may cause class confusion. For example, the performance of ER decreases more on airplanes compared to automobiles, probably because birds in the new task have more similar backgrounds to airplanes, as shown in Fig.~\ref{fig:heatmap}.
(3) \frameworkName achieves better discrimination both on task 1 and task 2. The results demonstrate that \frameworkName can maintain discrimination of all seen classes and significantly mitigate forgetting by 
combining the proposed \methodname and \dataaugname.






\subsection{Results and Analysis}
\label{result}
\paragraph{Performance of average accuracy.}
Table~\ref{tab:acc} presents the results of average accuracy with different memory bank sizes ($M$) on three benchmark datasets. Our \frameworkName consistently outperforms all baselines on three datasets.
Remarkably, the performance improvement of \frameworkName is more significant when the memory bank size is relatively small; this is critical for online CL with limited resources. For example, compared to the second-best method OCM, \frameworkName achieves about 10$\%$ and 6$\%$ improvement on CIFAR-10 when $M$ is 100 and 200, respectively. 
The results show that our \frameworkName can learn more representative and discriminative features with a limited memory bank.
Compared to baselines that use knowledge distillation (iCaRL, DER++, PASS, OCM), our \frameworkName achieves better performance by leveraging the feedback of online prototypes.  
Besides, \frameworkName significantly outperforms PASS and CoPE that also use prototypes, showing that online prototypes are more suitable for online CL. 


We find that the performance improvement tends to be gentle when $M$ increases.
The reason is that as $M$ increases, the samples in the memory bank become more diverse, and the model can extract sufficient information from massive samples to distinguish seen classes. 
In addition, many baselines perform poorly on CIFAR-100 and TinyImageNet due to a dramatic increase in the number of tasks. In contrast, \frameworkName still performs well and improves accuracy over the second best.



\paragraph{Performance of average forgetting.}
We report the Average Forgetting results of our \frameworkName and all baselines on three benchmark datasets in Table~\ref{tab:forget}. The results confirm that \frameworkName can effectively mitigate catastrophic forgetting. 
For CIFAR-10 and CIFAR-100, \frameworkName achieves the lowest average forgetting compared to all replay-based baselines. 
For TinyImageNet, our result is a little higher than iCaRL and CoPE but better than the latest methods DVC and OCM. 
The reason is that iCaRL uses a nearest class mean classifier, but we use softmax and FC layer during the test phase, and CoPE slowly updates prototypes with a high momentum.
However, as shown in Table~\ref{tab:acc}, \frameworkName provides more accurate classification results than iCaRL and CoPE. 
It is a fact that when the maximum accuracy of a task is small, the forgetting on this task is naturally rare, even if the model completely forgets what it learned.





\paragraph{Performance of each incremental step.}
We evaluate the average incremental performance~\cite{DER++, DVC} on CIFAR-10 ($M=0.1k$) and CIFAR-100 ($M=0.5k$), which indicates the accuracy over all seen tasks at each incremental step. 
Fig.~\ref{fig:incrementalAcc} shows that \frameworkName achieves better accuracy and effectively mitigates forgetting while the performance of most baselines degrades rapidly with the arrival of new classes.

\paragraph{Confusion matrices at the end of learning.}
We report the confusion matrices of our \frameworkName and the second-best method OCM, as shown in Fig.~\ref{fig:confusionMatrix}. 
After learning the last task (\ie, the last two classes), OCM forgets the knowledge of early tasks (classes 0 to 3). 
In contrast, \frameworkName performs relatively well in all classes, especially in the first task (classes 0 and 1), outperforming OCM by 27.8\% average improvements.
The results show that learning representative and discriminative features is crucial to mitigate catastrophic forgetting; see Appendix~\ref{appendix:extra_exp} for extra experimental results.  




\subsection{Ablation Studies}
\label{ablation}

\begin{table}[t]
\small
\begin{center}
\begin{tabular}{ccccc}
\shline
\multirow{2}{*}{{Method}} & {CIFAR-10}&{CIFAR-100} \\
& Acc $\uparrow$(Forget $\downarrow$) & Acc $\uparrow$(Forget $\downarrow$) \\ 
\midrule
baseline & 46.4\std{$\pm$1.2}(36.0\std{$\pm$}2.1) & 18.8\std{$\pm$0.8}(18.5\std{$\pm$}0.7) \\
w/o \methodname & 53.1\std{$\pm$1.4}(24.7\std{$\pm$2.0}) & 19.3\std{$\pm$0.7}(15.9\std{$\pm$0.9}) \\
w/o \dataaugname & 52.0\std{$\pm$1.5}(34.6\std{$\pm$2.4}) & 21.5\std{$\pm$0.5}(16.3\std{$\pm$0.8}) \\ 
\hline
w/o $\mathcal{L}^{\mathrm{new}}_{\mathrm{pro}}$ & 54.8\std{$\pm$1.2}(\textbf{22.1}\std{$\pm$3.0}) & 19.6\std{$\pm$0.8}(19.9\std{$\pm$0.7}) \\
w/o $\mathcal{L}^{\mathrm{seen}}_{\mathrm{pro}}$ & 55.7\std{$\pm$1.4}(25.5\std{$\pm$1.5}) & 20.1\std{$\pm$0.4}(16.2\std{$\pm$0.6}) \\ 
$\mathcal{L}^{\mathrm{seen}}_{\mathrm{pro}}$ w/o $\mathcal{C}^\mathrm{new}$ & 56.2\std{$\pm$1.2}(26.4\std{$\pm$2.3}) & 20.8\std{$\pm$0.6}(17.9\std{$\pm$0.7}) \\ 
\hline
{\frameworkName} (\textbf{ours}) & \textbf{57.8}\std{$\pm$1.1}(23.2\std{$\pm$1.3}) & \textbf{22.7}\std{$\pm$0.7}(\textbf{15.0}\std{$\pm$0.8}) \\ 
\shline 
\end{tabular}
\end{center}
\caption{Ablation studies on CIFAR-10 ($M=0.1k$) and CIFAR-100 ($M=0.5k$). 
``baseline'' means $\mathcal{L}_\mathrm{INS}+\mathcal{L}_\mathrm{CE}$.
``$\mathcal{L}^{\mathrm{seen}}_{\mathrm{pro}}$ w/o $\mathcal{C}^\mathrm{new}$'' means $\mathcal{L}^{\mathrm{seen}}_{\mathrm{pro}}$ do not consider new classes in current task.
}
\label{tab:ablation}
\end{table}

\paragraph{Effects of each component.} Table~\ref{tab:ablation} presents the ablation results of each component. Obviously, \methodname and \dataaugname can consistently improve the average accuracy of classification. 
We can observe that the effect of \methodname is more significant on more tasks while \dataaugname plays a crucial role when the memory bank size is limited. Moreover, when combining \methodname and \dataaugname, the performance is further improved, which indicates that both can benefit from each other. For example, \dataaugname boosts \methodname by about 6$\%$ improvements on CIFAR-10 ($M=0.1k$), and the performance of \dataaugname is improved by about 3$\%$ on CIFAR-100 ($M=0.5k$) by combining \methodname.


\paragraph{Equilibrium in \methodname.}
When learning new classes, the data of new classes is involved in both $\mathcal{L}^{\mathrm{new}}_{\mathrm{pro}}$ and $\mathcal{L}^{\mathrm{seen}}_{\mathrm{pro}}$ of \methodname, where $\mathcal{L}^{\mathrm{new}}_{\mathrm{pro}}$ only focuses on learning new knowledge while $\mathcal{L}^{\mathrm{seen}}_{\mathrm{pro}}$ tends to alleviate forgetting on seen classes.
To explore the best way of learning new classes, we consider three scenarios for \methodname in Table~\ref{tab:ablation}.
The results show that only learning new knowledge (w/o $\mathcal{L}^{\mathrm{seen}}_{\mathrm{pro}}$) or only consolidating the previous knowledge (w/o $\mathcal{L}^{\mathrm{new}}_{\mathrm{pro}}$) can significantly degrade the performance, which indicates that both are indispensable for online CL.
Furthermore, when $\mathcal{L}^{\mathrm{seen}}_{\mathrm{pro}}$ only considers old classes and ignores new classes ($\mathcal{L}^{\mathrm{seen}}_{\mathrm{pro}}$ w/o $\mathcal{C}^\mathrm{new}$), the performance also decreases. These results show that the equilibrium of all seen classes (\methodname) can achieve the best performance and is crucial for online CL.


\paragraph{Effects of \dataaugname.} 
To verify the advantage of \dataaugname, we compare it with the completely random mixup
in Table~\ref{tab:ablation_mixup}.
\begin{table}
\small
\begin{center}
\begin{tabular}{c|rrr}
\shline
\multicolumn{1}{c|}{Method}       & ${M=0.1k}$   & ${M=0.2k}$   & ${M=0.5k}$     \\ \hline
Random & 53.5\std{$\pm$2.7} & 62.9\std{$\pm$2.5} & 70.8\std{$\pm$2.2} \\
\dataaugname (\textbf{ours})  & \textbf{57.8}\std{$\pm$1.1} & \textbf{65.5}\std{$\pm$1.0} & \textbf{72.6}\std{$\pm$0.8} \\ 
\shline
\end{tabular}
\end{center}
\caption{Comparison of Random Mixup and \dataaugname on CIFAR-10. 
}
\label{tab:ablation_mixup}
\end{table}
\dataaugname outperforms random mixup in all three scenarios. Notably, \dataaugname works significantly when the memory bank size is small, which shows that the feedback can prevent class confusion due to a restricted memory bank; see Appendix~\ref{appendix:ablations} for extra ablation studies.



\subsection{Validation of Online Prototypes}
\label{prove_onlinePrototypes}
\begin{figure}
    \centering
    \includegraphics[width=1.0\linewidth]{Figures/imgs/cosine_similarity.pdf}
    \caption{The cosine similarity between online prototypes and prototypes of the entire memory bank.}
    \label{fig:cosine_similarity}
\end{figure}
Fig.~\ref{fig:cosine_similarity} shows the cosine similarity between online prototypes and global prototypes (prototypes of the entire memory bank) at each time step.
For the first mini-batch of each task, online prototypes are equal to global prototypes (similarity is 1, omitted in Fig.~\ref{fig:cosine_similarity}).
In the first task, online and global prototypes are updated synchronously with the model updates, resulting in high similarity. 
In subsequent tasks, the model initially learns inadequate features of new classes, causing online prototypes to be inconsistent with global prototypes and low similarity, which shows that accumulating early features as prototypes may be harmful to new tasks. However, the similarity will improve as the model learns, because the model gradually learns representative features of new classes.
Furthermore, the similarity on old classes is only slightly lower, showing that online prototypes are resistant to forgetting. 


\section{Conclusion}
In this work, we demonstrate that it's possible to distill huge models trained on large datasets to obtain much smaller models that perform well on paralinguistic speech tasks.
The distillation uses only \textbf{7\% of the training data} and is entirely from public sources. The models we obtain are between 22MB and 314MB, and achieve between \textbf{90\% and 96\% of the larger CAP12 accuracy on 6 of 7 tasks}. These models are between \textbf{1\% and 15\% the size} of the original model. We release the model to allow the research community to benefit from the practical applications of self-supervised representations for paralinguistic speech.

\balance

%\addtolength{\textheight}{-12cm}   % This command serves to balance the column lengths
                                  % on the last page of the docume+nt manually. It shortens
                                  % the textheight of the last page by a suitable amount.
                                  % This command does not take effect until the next page
                                  % so it should come on the page before the last. Make
                                  % sure that you do not shorten the textheight too much.



\bibliographystyle{IEEEtran}
\bibliography{ref}




\end{document}