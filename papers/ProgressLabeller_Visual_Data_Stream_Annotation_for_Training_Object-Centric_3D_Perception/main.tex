%%%%%%%%%%%%%%%%%%%%%%%%%%%%%%%%%%%%%%%%%%%%%%%%%%%%%%%%%%%%%%%%%%%%%%%%%%%%%%%%
%2345678901234567890123456789012345678901234567890123456789012345678901234567890
%        1         2         3         4         5         6         7         8

\documentclass[letterpaper, 10 pt, conference]{ieeeconf}
\usepackage[letterpaper, left=0.75in, right=0.75in, bottom=0.77in, top=0.78in]{geometry}
%\documentclass[a4paper, 10pt, conference]{ieeeconf}      % Use this line for a4 paper

\IEEEoverridecommandlockouts                              % This command is only needed if 
                                                         % you want to use the \thanks command

%\overrideIEEEmargins                                      % Needed to meet printer requirements.

% See the \addtolength command later in the file to balance the column lengths
% on the last page of the document

% The following packages can be found on http:\\www.ctan.org
% \usepackage{graphics} % for pdf, bitmapped graphics files
\usepackage{graphicx}
\usepackage{amsmath}
\usepackage{xcolor}
\usepackage{floatrow}
\usepackage{epstopdf}
\usepackage[colorlinks = true,
            linkcolor = black,
            urlcolor  = black,
            citecolor = black,
            anchorcolor = black]{hyperref}
\newtheorem{theorem}{Lemma}
\usepackage{algorithm,algpseudocode}
\usepackage[font=small]{caption}
\usepackage{accents}
\usepackage{amsfonts}
\usepackage{wrapfig}
% \usepackage{subcaption}
\usepackage{booktabs}
\usepackage{balance}
\usepackage{scalerel}
\usepackage{subfigure}
\usepackage{multirow}
\usepackage{url}
\usepackage{svg}
\usepackage{hhline}
\usepackage{hyperref}
% \usepackage[numbers,sort&compress]{natbib}
\def\bibfont{\footnotesize}

\newcommand*{\Comb}[2]{{}^{#1}C_{#2}}
\newcommand{\norm}[1]{\left\lVert#1\right\rVert}
   
\long\def\ocj#1{\textcolor{red}{#1}}
\newcommand{\zmchange}[1]{{\color{blue}  #1} } 
\newcommand{\zm}[1]{{\color{blue} \bf **#1**} }
\long\def\chad#1{\textcolor{red}{\bf \small CJ: **#1**}}
\long\def\zz#1{\textcolor{orange}{\bf \small ZZ: **#1**}}
\long\def\kz#1{\textcolor{cyan}{\bf \small kz: **#1**}}
\long\def\cxt#1{\textcolor{magenta}{\bf \small cxt: **#1**}}
\long\def\jp#1{\textcolor{teal}{\bf \small JP: **#1**}}
\long\def\ignore#1{}
\textfloatsep=3pt
\floatsep=3pt
\intextsep=3pt
\dblfloatsep=3pt
\dbltextfloatsep=3pt

\DeclareMathOperator*{\argmax}{argmax}
% \abovecaptionskip=3pt
% \belowcaptionskip=3pt
\begin{document}
\title{\LARGE \bf
ProgressLabeller: Visual Data Stream Annotation for Training Object-Centric 3D Perception
}

\author{Xiaotong Chen\hspace{0.5cm} 
Huijie Zhang\hspace{0.5cm}  Zeren Yu\hspace{0.5cm}  Stanley Lewis\hspace{0.5cm} Odest Chadwicke Jenkins
% \thanks{Manuscript received: February 24, 2020; Accepted May 12, 2020.}%Use only for final RAL version
% \thanks{This paper was recommended for publication by Editor Hong Liu upon evaluation of the Associate Editor and Reviewers' comments.} %Use only for final RAL version
\thanks{\authorrefmark{1}X. Chen, H. Zhang, Z. Yu, S. Lewis and O. C. Jenkins are with the Department of Electrical Engineering and Computer Science, and Robotics Institute at the University of Michigan, Ann
Arbor, MI 48109 USA {\tt\footnotesize [cxt|huijiezh|yuzeren|stanlew|ocj] @umich.edu}}%
\thanks{Digital Object Identifier (DOI): see top of this page.}
}


% \markboth{IEEE Robotics and Automation Letters. Preprint Version. Accepted May, 2020}
% {Zhou \MakeLowercase{\textit{et al.}}: LIT: Light-field Inference of Transparency for Refractive Object Localization} 

\maketitle
% \thispagestyle{empty}
% \pagestyle{empty}
% \graphicspath{{images/}}
%%%%%%%%%%%%%%%%%%%%%%%%%%%%%%%%%%%%%%%%%%%%%%%%%%%%%%%%%%%%%%%%%%%%%%%%%%%%%%%%
\begin{abstract}

Visual perception tasks often require vast amounts of labelled data, including 3D poses and image space segmentation masks. The process of creating such training data sets can prove difficult or time-intensive to scale up to efficacy for general use. Consider the task of pose estimation for rigid objects. Deep neural network based approaches have shown good performance when trained on large, public datasets. However, adapting these networks for other novel objects, or fine-tuning existing models for different environments, requires significant time investment to generate newly labelled instances. Towards this end, we propose ProgressLabeller as a method for more efficiently generating large amounts of 6D pose training data from color images sequences for custom scenes in a scalable manner. ProgressLabeller is intended to also support transparent or translucent objects, for which the previous methods based on depth dense reconstruction will fail.
We demonstrate the effectiveness of ProgressLabeller by rapidly create a dataset of over 1M samples with which we fine-tune a state-of-the-art pose estimation network in order to markedly improve the downstream robotic grasp success rates. Progresslabeller is open-source at \href{https://github.com/huijieZH/ProgressLabeller}{https://github.com/huijieZH/ProgressLabeller}

\end{abstract}

% \begin{IEEEkeywords}
% Perception for Grasping and Manipulation, Grasping
% \end{IEEEkeywords}

\section{Introduction}
\epigraph{\normalsize ``\textit{ \textbf{The essence of a riddle is to express true facts under impossible combinations.}}"}{\normalsize--- \textit{Aristotle}, \textit{Poetics} (350 BCE)\vspace{0pt}}

\noindent
A \textit{riddle} is a puzzling question about {concepts} in our everyday life.
% , and we which one needs common sense to reason about.
For example, a riddle might ask ``\textit{My life can be measured in hours. I serve by being devoured. Thin, I am quick. Fat, I am slow. Wind is my foe. What am I?}''~
The correct answer ``\textit{candle},'' is reached by considering a collection of \textit{commonsense knowledge}:
{a candle can be lit and burns for a few hours; a candle's life depends upon its diameter; wind can extinguish candles, etc.}
\begin{figure}[t]
	\centering 
	\includegraphics[width=1\linewidth]{riddle_intro_final.pdf}
	\caption{ 
    The top example is a trivial commonsense question from the CommonsenseQA~\cite{Talmor2018CommonsenseQAAQ} dataset. 
    The two bottom examples are from our proposed \textsc{RiddleSense} challenge.
    The right-bottom question is a descriptive riddle that implies multiple commonsense facts about \textit{candle}, and it needs understanding of figurative language such as metaphor;
    The left-bottom one additionally needs counterfactual reasoning ability to address the \textit{`but-no'} cues. 
    These riddle-style commonsense questions  require NLU systems to have higher-order reasoning skills with the understanding of creative language use.
	}
	\label{fig:intro} 
\end{figure}

It is believed that the \textit{riddle} is one of the earliest forms of oral literature,
which can be seen as a formulation of thoughts about common sense, a mode of association between everyday concepts, and a metaphor as higher-order use of natural language~\cite{hirsch2014poet}.
Aristotle stated in his \textit{Rhetoric} (335-330 BCE) that good riddles generally provide satisfactory metaphors for rethinking common concepts in our daily life.
He also pointed out in the \textit{Poetics} (350 BCE): ``the essence of a riddle is to express true facts under impossible combinations,'' which suggests that solving riddles is a nontrivial  reasoning task.

Answering riddles is indeed a challenging cognitive process as it requires \textit{complex} {commonsense reasoning skills}.
% which we refer to \textit{higher-order commonsense reasoning}. 
% A successful riddle-solving model should be able to reason with \textit{multiple pieces} of commonsense facts, as 
A riddle can describe \textit{multiple pieces} of commonsense knowledge with \textit{figurative} devices such as metaphor and personification (e.g., ``wind is my \underline{foe} $\xrightarrow[]{}$ \textit{extinguish}'').
% , as shown by the examples in Figure~\ref{fig:intro}.
%%%
Moreover, \textit{counterfactual thinking} is also necessary for answering many riddles such as ``\textit{what can you hold in your left hand \underline{but not} in your right hand? $\xrightarrow[]{}$ your right elbow.}''
These riddles with \textit{`but-no'} cues require that models use counterfactual reasoning ability to consider possible solutions for situations or objects that are seemingly impossible at face value.
This \textit{reporting bias}~\cite{gordon2013reporting} makes riddles a more difficult type of commonsense question for pretrained language models to learn and reason.
% In addition, the model needs to associate commonsense knowledge with the creative use of language in descriptions, which may have figurative devices such as metaphor and personification (e.g., ``wind is my \underline{foe} $\xrightarrow[]{}$ \textit{extinguish}''). 
%For instance, one needs to know that devour
% Thus, a riddle here can be seen as a complex commonsense question that tests higher-order reasoning ability with creativity.
In contrast, \textit{superficial} commonsense questions such as ``\textit{What home entertainment equipment requires cable?}'' in  CommonsenseQA~\cite{Talmor2018CommonsenseQAAQ} are more straightforward and explicitly stated.
We illustrate this comparison in Figure~\ref{fig:intro}.


In this paper,
we introduce the \textsc{RiddleSense} challenge 
to study the task of answering riddle-style commonsense questions\footnote{We use ``riddle'' and ``riddle-style commonsense question'' interchangeably in this paper.} requiring \textit{creativity}, \textit{counterfactual thinking} and \textit{complex commonsense reasoning}.
\textsc{RiddleSense} is presented as a \textit{multiple-choice question answering} task where a model selects one of five answer choices to a given riddle question as its predicted answer, as shown in Fig.~\ref{fig:intro}.
We construct the dataset by first crawling from several free websites featuring large collections of human-written riddles and then aggregating, verifying, and correcting these examples using a combination of human rating and NLP tools to create a dataset consisting of 5.7k high-quality examples.
Finally, we use \textit{Amazon Mechanical Turk} to crowdsource quality distractors to create a challenging benchmark.
We show that our riddle questions are more challenging than {CommonsenseQA} by analyzing graph-based statistics over ConceptNet~\cite{Speer2017ConceptNet5A}, a large knowledge graph for common sense reasoning.

% The distractors for the training data are automatically generated from ConceptNet and language models while the distractors for the dev and the test sets are crowd-sourced from Amazon Mechanical Turk (AMT).
% Through data analysis based on graph connectivity, 




Recent studies have demonstrated that
 fine-tuning large pretrained language models, such as {BERT}~\cite{Devlin2019}, RoBERTa, and ALBERT~\cite{Lan2020ALBERT}, can achieve strong results on current commonsense reasoning benchmarks.
Developed on top of these language models, graph-based language reasoning models such as KagNet~\cite{kagnet-emnlp19} and MHGRN~\cite{feng2020scalable} show superior performance. 
Most recently, UnifiedQA~\cite{khashabi2020unifiedqa} proposes to unify different QA tasks and train a text-to-text model for learning from all of them, which achieves state-of-the-art performance on many commonsense benchmarks.

To provide a comprehensive benchmarking analysis, we systematically compare the above methods.
Our experiments reveal that while humans achieve 91.33\% accuracy on \textsc{riddlesense}, the best language models can only achieve 68.80\% accuracy, suggesting that there is still much room for improvement in the field of solutions to complex commonsense reasoning questions with language models.
% We also provide error analysis to better understand the limitation of current methods.
We believe the proposed \textsc{RiddleSense} challenge suggests productive future directions for machine commonsense reasoning as well as the understanding of higher-order and creative use of natural language.


% (previous state-of-the-art on \texttt{CommonsenseQA} (56.7\%)).
% However, there still exists a large gap between performance of said baselines and human performance.
% we show that the questions in RiddleSense is significantly more challenging, in terms of the length of the paths from question concepts and answer concepts.


%Apart from that, current pre-trained language models (e.g., BERT~\cite{}, RoBERTa~\cite{}, etc.) and commonsense-reasoning models (e.g., KagNet~\cite{}), can be easily adapted to work for this format with minimal modifications. 


%Note that these auto-generated distractors may be still easy for , which could diminish the testing ability of the dataset.
%We design an ader filtering method to get rid of the false negative   and control the task difficulty. 
% To strengthen the task, we propose an adversarial cross-filtering method to remove the distractors that ineffectively mislead the selected base models.
% Finally, we use human efforts to inspect the distractors and remove false negative ones, to make sure that all distractors either does not make sense or much less plausible than the correct answers.
%Introducing these fine-tuned models is inspired by the adversarial filtering algorithms~\cite{}, which can effectively reduce the  bias inside datasets for creating a more reliable benchmark.  



%Those distractors are explicitly annotated by human experts such that they are close to the meaning of 
%The main idea is to use multiple trainable generative models for learning to generate answers in a cross-validation style. 
%The wrong predictions
%Simply put, for every step, we use a large subset of the riddles and their current options ot learn multiple models for answering the remaining riddles via generation.
%After each step, we consolidate the 


% In the distantly supervised learning, we use the definition of concepts (i.e., glossary) of \textit{Wiktionary}\footnote{\url{https://www.wiktionary.org/}} to create riddles with answers as training data. 
% In the transfer learning setting, we aim to test the transferability of models across relevant datasets, such as CommonsenseQA~\cite{Talmor2018CommonsenseQAAQ}.

% We believe the \textsc{RiddleSense} task can benefit multiple communities in natural language processing. 
% First, the commonsense reasoning community can use \textsc{RiddleSense} as a new space to evaluate their reasoning models. The \textsc{RiddleSense} focuses on more complex and creative commonsense questions, which will encourage them to propose more higher-order commonsense reasoning models. 
% Second, \textsc{RiddleSense} is an NLU 
% task similar to those in the GLUE~\cite{wang2018glue} and SuperGLUE~\cite{wang2019superglue} leaderboard that can serve as a benchmark for testing various pre-trained language models.
% Last but not the least, as our task shares the similar format with many open-domain question answering tasks like \textit{Natural Questions}~\cite{kwiatkowski2019natural}, researchers in QA area may be also interested in \textsc{RiddleSense}. 






\section{Related Work}
With ProgressLabeller, a user can scalably label new datasets with camera world pose, scene object poses and scene object segmentations. This process is enabled by fusing streaming RGB (or RGB-D) inputs into a single scene-wide representation, and then allowing a human user to input relevant 6-DoF information via 3D modelling interfaces (such as those provided by Blender \cite{blender}). This process demonstrates label stability even over long input video streams, and due to its functionality with direct RGB inputs, can label even difficult objects such as transparent cups.  We discuss below methods related to ProgressLabeller.

\subsection{Direct \& Human-in-the-loop labelling}
The creation of 2D segmentation data is analogous to the object detection, keypoint detection, or semantic segmentation tasks (depending on desired output labels). Tools such as LabelMe \cite{russell2008labelme} required users to directly interact with the underlying data to be labelled. This manual process was improved by model-assisted approaches such as Deep Extreme Cut \cite{maninis2018deep} which decreases the amount of user effort necessary to label images.
Shared autonomy and mixed-initiative methods have also been used in this approach, in which the user provides coarse pose or other estimations which are fine-tuned via a model-informed approach \cite{ye2021human}.

%%%%%%%%%%%%%%%%%%%%%%%%%%%%%%%
 
\subsection{End-to-End Labellers}
Previous tools have been created to enable this style of learning process. LabelFusion \cite{marion2018label} is perhaps the most commonly utilized example. LabelFusion utilizes streaming RGB-D inputs to create a dense reconstruction of the scene, which is then labelled semi-manually by aligning 3D object models to the 3D reconstruction. While this approach is typically robust, it relies on RGB-D input for reconstruction, and experiences difficulties under certain regimes. In particular, transparent objects cause problems for commonly employed depth sensor technologies, and long-running input streams typically result in large amounts of 'drift'. 

Some methods have been introduced to eliminate the need for CAD models in the labelling process. Singh et al. \cite{singh2021rapid} proposed a method which utilizes user labelled keypoints and bounding boxes to generate pose and segmentation labels. This frees the system from dependency on CAD models, but requires user interaction directly with the images. SALT \cite{stumpf2021salt} proposed utilizing GrabCut to generate 3D bounding boxes and image segmentation labels for relevant scenes. This allows removing the dependency on object masks while also allows the labelling of dynamic scenes such as human gait videos. 
% This method requires point cloud inputs however, which precludes the labelling of RGB only image streams.

Other works sought to improve the labelling procedure itself. EasyLabel \cite{suchi2019easylabel} allows for semi-automatic labelling of scenes via sequentially added objects.  This process is scalable, and generates high quality labels. However, it requires tight physical control over the scene to be labelled, which is not always feasible to obtain. Objectron \cite{ahmadyan2021objectron} utilized modern smartphone's AR capabilities combined with human-labelled 3D bounding boxes to scalably create a large scale dataset. This method however is susceptible to label drift during long-duration input videos. KeyPose \cite{liu2020keypose} specifically sought to generate labelled datasets for transparent objects. This method utilized stereoscopic images taken from a robot armature in order to avoid the problems of typical depth cameras have with transparent objects.
% StereOBJ-1M \cite{liu2021stereobj} improves the data collection efficiency in a setting with two more static cameras and more fiducial markers in the scene.

% \section{Method}
\section{ProgressLabeller}
\section{Proposed Method: SyMFM6D}

We propose a deep multi-directional fusion approach called SyMFM6D that estimates the 6D object poses of all objects in a cluttered scene based on multiple RGB-D images while considering object symmetries. 
In this section, we define the task of multi-view 6D object pose estimation and present our multi-view deep fusion architecture.

\begin{figure*}[tbh]
  \vspace{2mm}
  \centering
  \includegraphics[page=1, trim = 5mm 40mm 5mm 42mm, clip,  width=1.0\linewidth]{figures/SyMFM6D_architecture4_2.pdf}
   \caption{Network architecture of SyMFM6D which fuses $N$ RGB-D input images. Our method converts the $N$ depth images to a single point cloud which is processed by an encoder-decoder point cloud network. The $N$ RGB images are processed by an encoder-decoder CNN. Every hierarchy contains a point-to-pixel fusion module and a pixel-to-point fusion module for deep multi-directional multi-view fusion. We utilize three MLPs with four layers each to regress 3D keypoint offsets, center point offsets, and semantic labels based on the final features. The 6D object poses are computed as in \cite{pvn3d} based on mean shift clustering and least-squares fitting. We train our network by minimizing our proposed symmetry-aware multi-task loss function using precomputed object symmetries. $N_p$ is the number of points in the point cloud. $H$ and $W$ are height and width of the RGB images.}
   \label{fig_architecture}
   \vspace{-2mm}
\end{figure*}


6D object pose estimation describes the task of predicting a rigid transformation $\boldsymbol p = [\boldsymbol R |  \boldsymbol t] \in SE(3)$ which transforms the coordinates of an observed object from the object coordinate system into the camera coordinate system. This transformation is called 6D object pose because it is composed of a 3D rotation $\boldsymbol R \in SO(3)$ and a 3D translation $\boldsymbol t \in \mathbb{R}^3$. 
The designated aim of our approach is to jointly estimate the 6D poses of all objects in a given cluttered scene using multiple RGB-D images which depict the scene from multiple perspectives. We assume the 3D models of the objects and the camera poses to be known as proposed by \cite{mv6d}.



\subsection{Network Overview}

Our symmetry-aware multi-view network consists of three stages which are visualized in \cref{fig_architecture}. 
The first stage receives one or multiple RGB-D images and extracts visual features as well as geometric features which are fused to a joint representation of the scene. 
The second stage performs a detection of predefined 3D keypoints and an instance semantic segmentation.
Based on the keypoints and the information to which object the keypoints belong, we compute the 6D object poses with a least-squares fitting algorithm \cite{leastSquares} in the third stage.



\subsection{Multi-View Feature Extraction}

To efficiently predict keypoints and semantic labels, the first stage of our approach learns a compact representation of the given scene by extracting and merging features from all available RGB-D images in a deep multi-directional fusion manner. For that, we first separate the set of RGB images $\text{RGB}_1, ..., \text{RGB}_N$ from their corresponding depth images $\text{Dpt}_1$, ..., $\text{Dpt}_N$. The $N$ depth images are converted into point clouds, transformed into the coordinate system of the first camera, and merged to a single point cloud using the known camera poses as in \cite{mv6d}. 
Unlike \cite{mv6d}, we employ a point cloud network based on RandLA-Net \cite{hu2020randla} with an encoder-decoder architecture using skip connections.
The point cloud network learns geometric features from the fused point cloud and considers visual features from the multi-directional point-to-pixel fusion modules as described in \cref{sec_multi_view_fusion}.

The $N$ RGB images are independently processed by a CNN with encoder-decoder architecture using the same weights for all $N$ views. The CNN learns visual features while considering geometric features from the multi-directional pixel-to-point fusion modules. We followed \cite{ffb6d} and build the encoder upon a ResNet-34 \cite{resnet} pretrained on ImageNet~\cite{imagenet} and the decoder upon a PSPNet \cite{pspnet}. 

After the encoding and decoding procedures including several multi-view feature fusions, we collect the visual features from each view corresponding to the final geometric feature map and concatenate them. The output is a compact feature tensor containing the relevant information about the entire scene which is used for keypoint detection and instance semantic segmentation as described in \cref{sec_keypoint_detection_and_segmentation}.


\begin{figure*}[tbh]
  \vspace{2mm} 
  \centering  
\begin{subfigure}[b]{0.48\textwidth}
  \includegraphics[page=1, trim = 1mm 6mm 6mm 6mm, clip,  width=1.0\linewidth]{figures/p2r_8.pdf}
   \caption{Point-to-pixel fusion module.~~~~}
   \label{fig_pt2px_fusion}
\end{subfigure}
\begin{subfigure}[b]{0.48\textwidth}
  \centering  
  \includegraphics[page=1, trim = 1mm 6mm 6mm 6mm, clip,  width=1.0\linewidth]{figures/r2p_8.pdf}
   \caption{Pixel-to-point fusion module.~~~~~}
   \label{fig_px2pt_fusion}
   \end{subfigure}
      \caption{Overview of our proposed multi-directional multi-view fusion modules. They combine pixel-wise visual features and point-wise geometric features by exploiting the correspondence between pixels and points using the nearest neighbor algorithm. We compute the resulting features using multiple shared MLPs with a single layer and max-pooling.
      For simplification, we depict an example with $N=2$ views and $K_\text{i}=K_\text{p}=3$ nearest neighbors. The points of ellipsis (...) illustrate the generalization for an arbitrary number of views $N$. Please refer to \cite{ffb6d} for better understanding the basic operations.
      }
   \label{fig_fusion_modules}
   \vspace{-1mm}
\end{figure*}



\subsection{Multi-View Feature Fusion}
\label{sec_multi_view_fusion}
In order to efficiently fuse the visual and geometric features from multiple views, we extend the fusion modules of FFB6D~\cite{ffb6d} from bi-directional fusion to \emph{multi-directional fusion}. We present two types of multi-directional fusion modules which are illustrated in \cref{fig_fusion_modules}.
Both types of fusion modules take the pixel-wise visual feature maps and the point-wise geometric feature maps from each view, combine them, and compute a new feature map.
This process requires a correspondence between pixel-wise and point-wise features which we obtain by computing an XYZ map for each RGB feature map based on the depth data of each pixel using the camera intrinsic matrix as in \cite{ffb6d}. To deal with the changing dimensions at different layers, we use the centers of the convolutional kernels as new coordinates of the feature maps and resize the XYZ map to the same size using nearest interpolation as proposed in \cite{ffb6d}.

The \emph{point-to-pixel} fusion module in \cref{fig_pt2px_fusion} computes a 
fused feature map $\bb F_\text{f}$ based on the image features $\bb F_{\text{i}}(v)$ of all views $v \in \{1, \ldots, N\}$.
We collect the $K_\text{p}$ nearest point features $\bb F_{\text{p}_k}(v)$ with $k \in \{1, \ldots, K_\text{p}\}$ from the point cloud for each pixel-wise feature and each view independently by computing the nearest neighbors according to the Euclidean distance in the XYZ map. Subsequently, we process them by a shared MLP before aggregating them by max-pooling, i.e.,
\begin{align} 
    \widetilde{\bb F}_{\text{p}}(v) = \max_{k \in \{1, \ldots, K_\text{p}\}} 
    \Big( \text{MLP}_\text{p}(\bb F_{\text{p}_k}(v)) \Big).
    \label{eq_p2r}
\end{align}
Finally, we apply a second shared MLP to fuse all features $\bb F_\text{i}$ and 
$\widetilde{\bb F}_{\text{p}}$ as 
$\bb F_{\text{f}} = \text{MLP}_\text{fp}(\widetilde{\bb F}_{\text{p}} \oplus \bb F_\text{i})$ where $\oplus$ denotes the concatenate operation.


The \emph{pixel-to-point} fusion module in \cref{fig_px2pt_fusion} collects the $K_\text{i}$ nearest image features $\bb F_{\text{i}_k}(\textrm{i2v}(i_k))$ with $k\in\{1, ..., K_\text{i}\}$. $\textrm{i2v}(i_k)$ is a mapping that maps the index of an image feature to its corresponding view. This procedure is performed for each point feature vector $\bb F_\text{p}(n)$.
We aggregate the collected image features by max-pooling and apply a shared MLP, i.e.,
\begin{align}
    \widetilde{\bb F}_{\text{i}} = \text{MLP}_\text{i} 
    \left( \max_{k \in \{1, \ldots, K_\text{i}\}} 
    \Big( \bb F_{\text{i}_k}(\textrm{i2v}(i_k)) \Big)  
    \right).
    \label{eq_r2p}
\end{align}
One more shared MLP fuses the resulting image features $\widetilde{\bb F}_{\text{i}}$ with the point features $\bb F_\text{p}$ as 
$\bb F_{\text{f}} = \text{MLP}_\text{fi}(\widetilde{\bb F}_{\text{i}} \oplus \bb F_\text{p})$.




\subsection{Keypoint Detection and Segmentation}
\label{sec_keypoint_detection_and_segmentation}
The second stage of our SyMFM6D network contains modules for 3D keypoint detection and instance semantic segmentation following \cite{mv6d}. However, unlike \cite{mv6d}, we use the SIFT-FPS algorithm \cite{lowe1999sift} as proposed by FFB6D \cite{ffb6d} to define eight target keypoints for each object class. SIFT-FPS yields keypoints with salient features which are easier to detect.
Based on the extracted features, we apply two shared MLPs to estimate the translation offsets from each point of the fused point cloud to each target keypoint and to each object center.
We obtain the actual point proposals by adding the translation offsets to the respective points of the fused point cloud. 
Applying the mean shift clustering algorithm \cite{cheng1995meanshift} results in predictions for the keypoints and the object centers.
We employ one more shared MLP 
for estimating the object class of each point in the fused point cloud as in \cite{pvn3d}.



\subsection{6D Pose Computation via Least-Squares Fitting}

Following \cite{pvn3d}, we use the least-squares fitting algorithm \cite{leastSquares} to compute the 6D poses of all objects based on the estimated keypoints. As the $M$ estimated keypoints $\boldsymbol{\widehat{k}}_1, ..., \boldsymbol{\widehat{k}}_M$ are in the coordinate system of the first camera and the target keypoints $\boldsymbol k_1, ..., \boldsymbol k_M$ are in the object coordinate system, least-squares fitting calculates the rotation matrix $\boldsymbol R$ and the translation vector $\boldsymbol t$ of the 6D pose by minimizing the squared loss
\begin{equation}
    L_\text{Least-squares} = \sum_{i=1}^M \norm{\boldsymbol{\widehat{k}_i} - (\boldsymbol R \boldsymbol k_i + \boldsymbol t)}_2^2.
\end{equation}



\subsection{Symmetry-aware Keypoint Detection}

Most related work, including \cite{pvn3d, ffb6d}, and \cite{mv6d} does not specifically consider object symmetries. 
However, symmetries lead to ambiguities in the predicted keypoints as multiple 6D poses can have the same visual and geometric appearance. 
Therefore, we introduce a novel symmetry-aware training procedure for the 3D keypoint detection including a novel symmetry-aware objective function to make the network predicting either the original set of target keypoints for an object or a rotated version of the set corresponding to one object symmetry. Either way, we can still apply the least-squares fitting which efficiently computes an estimate of the target 6D pose or a rotated version corresponding to an object symmetry. To do so, we precompute the set $\boldsymbol{S}_I$ of all rotational symmetric transformations for the given object instance $I$ with a stochastic gradient
descent algorithm \cite{sgdr}.
Given the known mesh of an object and an initial estimate for the symmetry axis, we transform the object mesh along the symmetry axis estimate and optimize the symmetry axis iteratively by minimizing the ADD-S metric \cite{hinterstoisser2012model}.
Reflectional symmetries which can be represented as rotational symmetries are handled as rotational symmetries. 
Other reflectional symmetries are ignored, since the reflection cannot be expressed as an Euclidean transformation.
To consider continuous rotational symmetries, we discretize them into 16 discrete rotational symmetry transformations.

We extend the keypoints loss function of \cite{pvn3d} to become symmetry-aware such that it predicts the keypoints of the closest symmetric transformation, i.e. 
\begin{equation}
    L_\text{kp}(\mathcal{I}) = \frac{1}{N_I} 
    \min_{\boldsymbol{S} \in \boldsymbol{S}_I} 
    \sum_{i \in \mathcal{I}} \sum_{j=1}^M 
    \norm{\boldsymbol{x}_{ij} - \boldsymbol{S}\boldsymbol{\widehat{x}}_{ij}}_2, 
\label{eq_keypoint_loss}
\end{equation}
where $N_I$ is the number of points in the point cloud for object instance $I$, $M$ is the number of target keypoints per object, and $\mathcal{I}$ is the set of all point indices that belong to object instance $I$.  
The vector $\boldsymbol{\widehat{x}}_{ij}$ is the predicted keypoint offset for the $i$-th point and the $j$-th keypoint while $\boldsymbol{x}_{ij}$ is the corresponding ground truth. 



\subsection{Objective Function}

We train our network by minimizing the multi-task loss function
\begin{equation}
 \label{eq_total_loss}
    L_\text{multi-task} = \lambda_1 L_\text{kp} 
    + \lambda_2 L_\text{semantic}  
    +  \lambda_3 L_\text{cp},
\end{equation}
where $L_\text{kp}$ is our symmetry-aware keypoint loss from \cref{eq_keypoint_loss}.
$L_\text{cp}$ is an L1 loss for the center point prediction, $L_\text{semantic}$ is a Focal loss \cite{focalLoss} for the instance semantic segmentation, and $\lambda_1=2$, $\lambda_2=1$, and $\lambda_3=1$ are the weights for the individual loss functions as in \cite{ffb6d}.

\subsection{Annotation accuracy estimation from simulation}

We verify the accuracy of annotations throughout this multi-view silhouette matching process by simulating an iterative object pose update process. In each iteration, given a certain camera frame, 
we assume the object will be translated in a plane parallel with its x-y plane, or rotated about z-axis (for better control), towards a pose that maximizes the Intersection-over-Union (IoU) between the rendered silhouette at current pose and ground truth.
% we assume the object will be translated or rotated to maximize the IoU between the object silhouette rendered at current pose and ground truth from that camera view.

\subsubsection{Problem Definition} Given set of $N$ images $I^{\{i\}}$ with their corresponding camera pose $T^{\{i\}}$ in the world frame, $i \in \{1, 2, 3, \ldots, N\}$. Our goal is to find ground truth object pose $T^{\text{obj}\{j\}}_{gt}$ in the world frame for all the objects $j \in \{1, 2, 3, \ldots, M\}$ in the scene. We define the projection operator as $S^{\{i, j\}} = \text{Proj}(T^{\{i\}}, T^{\text{obj}\{j\}})$, which render object $j$ given its CAD model, camera pose $T^{\{i\}}$ and object pose $T^{\text{obj}\{j\}}$ into an object texture/silouette $S^{\{i, j\}}$. Also defined the IoU operator as $\text{IoU}_{\text{obj}\{j\}}(I^{\{i\}}, \text{Proj}(T^{\{i\}}, T^{\text{obj}\{j\}}))$ to calculate the IoU for pixels in object $j$ between real image $I^{\{i\}}$ and synthetic texture/silouette $S^{\{i, j\}}$. 

% \begin{equation}
%     T^{\text{obj}\{j\}}_{gt} = \argmax_{T^{\text{obj}\{j\}}}\text{IoU}_{\text{obj}\{j\}}(I^{\{i\}}, \text{Proj}(T^{\{i\}}, T^{\text{obj}\{j\}}))
%     \label{eq:fact}
% \end{equation}

The multi-view texture/silhouette matching iterative update is proceeded with a goal to maximize the IoU. Given the pose for object $j$ in the $k$th iteration $T^{\text{obj}\{j\}}_{(k)}$, in ($k + 1$)th iteration:

\begin{equation}
    T^{\text{obj}\{j\}}_{(k+1)} = \argmax_{f[T^{\text{obj}\{j\}}_{(k)}]}\text{IoU}_{\text{obj}\{j\}}(I^{\{i\}}, \text{Proj}(T^{\{i\}}, f[T^{\text{obj}\{j\}}_{(k)}] )) 
    \label{eq:iterate}
\end{equation}
where $f[T^{\text{obj}\{j\}}_{(k)}]$ describes all possible translation start from the  $T^{\text{obj}\{j\}}_{(k)}$ 
% To limit the annotation freedom for a better human control, when viewing perpendicular to one view, only the translation 
that is within the plane $p$ or the rotation along the axis $\omega$ as shown in Figure \ref{fig:annotation_limitation}. So:

\begin{equation}
    f[T^{\text{obj}\{j\}}_{(k)}] = \exp^{\widehat{\xi}_1 \theta_1}  \exp^{\widehat{\xi}_2 \theta_2} T^{\text{obj}\{j\}}_{(k)}
    \label{eq:range}
\end{equation}
where ${\xi}_1 = \begin{bmatrix}-\omega \times v_o \\ \omega \end{bmatrix}$, ${\xi}_2 = \begin{bmatrix} v \\ 0 \end{bmatrix}$ are the twist coordinates for twist $\widehat{\xi}_1, \widehat{\xi}_2$.

% After plug Equation \eqref{eq:range} to Equation \eqref{eq:iterate}, we could get:

% \begin{align}
%     &T^{\text{obj}\{j\}}_{(k+1)} = \argmax_{\theta_1, \theta_2, v}\text{IoU}_{\text{obj}\{j\}}(I^{\{i\}}, \text{Proj}(T^{\{i\}}, T(\theta_1, \theta_2, v) )) \\
%     &T(\theta_1, \theta_2, v) = \exp^{\widehat{\xi}_1 \theta_1}  \exp^{\widehat{\xi}_2 \theta_2} T^{\text{obj}\{j\}}_{(k)}
%     \label{eq:iterate}
% \end{align}

% A more intuitively understanding is that during each iteration, we rotate the object along the axis $\omega$ and translate it within the plane $p$ to maximize the IoU between the projection silhouette $S^{\{i, j\}}$ and real image $I^{\{i\}}$. Simulation experiments in the coming section will demonstrate the convergence of this problem.


 \begin{figure}[htbp]
     \centering
     \includegraphics[width=0.8\columnwidth]{figure/annotation_limitation_resize.pdf}
     \caption{Diagram for an object shown under a camera. $c$ denote the location of camera and $c_x, c_y, c_z$ are its x, y, z axis. $p$ is a plane parallel to camera plane and passing through object's center $v_o$. $\omega$ the rotation direction parallel to $c_x$ and passing through $v_o$. $\theta_1$ is the magnitude of rotation radius. $v$ is the translation direction within the plane $p$ and $\theta_2$ is the translation magnitude. On the right hand side is the projection image, the object in the transparent color is the object with ground truth pose.}
     \label{fig:annotation_limitation}
 \end{figure}

\subsubsection{Simulation results}

We generate a CAD model sets with 44 different CAD models. For each run, we generate $T^{\text{obj}}_{gt}$ with a random rotation matrix and location at the origin. 40 cameras are created with their z axis pointing towards the origin and a random location at a sphere around the object. The initial pose $T^{\text{obj}}_{0}$ is generated by adding a random position noise from Gaussian distribution with variance of 10cm to origin and with a random 3D orientation. During each iteration, $v, \theta_1, \theta_2$ in Equation \ref{eq:range} are discretized for simulation. The result shows that it takes around 10.36 iterations for the algorithm to converge within 1mm location error and dot product larger than 0.99 between ground truth and converged rotation axes.

\section{Experiments}

\begin{figure*}
    \centering
    \includegraphics[width=1.0\linewidth]{Figures/imgs/tsne_motivation.pdf}
    \caption{$t$-SNE~\cite{tsne} visualizations of features learned from ER and \frameworkName on the test set of CIFAR-10.
    When learning new classes, ER suffers serious class confusion probably because shortcut learning. In contrast, \frameworkName significantly mitigates the forgetting.
    }
    \label{fig:tsne_motivation}
\end{figure*}
\begin{table*}[ht]
\small
\begin{center}
\resizebox{\linewidth}{!}{
\begin{tabular}{rrrrrrrrrrrr}
\shline
\multirow{2}{*}{Method}  & \multicolumn{3}{c}{CIFAR-10}   && \multicolumn{3}{c}{CIFAR-100}  && \multicolumn{3}{c}{TinyImageNet} \\ \cline{2-4}\cline{6-8}\cline{10-12}
       & $M=0.1k$   & $M=0.2k$   & $M=0.5k$     && $M=0.5k$     & $M=1k$     & $M=2k$     && $M=1k$      & $M=2k$ & $M=4k$   \\ \midrule
iCaRL~\cite{iCaRL}    & 31.0\std{$\pm$1.2} & 33.9\std{$\pm$0.9} & 42.0\std{$\pm$0.9} && 12.8\std{$\pm$0.4}  & 16.5\std{$\pm$0.4}  & 17.6\std{$\pm$0.5} && 5.0\std{$\pm$0.3}   & 6.6\std{$\pm$0.4} & 7.8\std{$\pm$0.4} \\ 
DER++~\cite{DER++}   & 31.5\std{$\pm$2.9} & 39.7\std{$\pm$2.7} & 50.9\std{$\pm$1.8} && 16.0\std{$\pm$0.6}  & 21.4\std{$\pm$0.9}  & 23.9\std{$\pm$1.0} && 3.7\std{$\pm$0.4} & 5.1\std{$\pm$0.8} & 6.8\std{$\pm$0.6} \\ 
PASS~\cite{protoAug}    & 33.7\std{$\pm$2.2} & 33.7\std{$\pm$2.2} & 33.7\std{$\pm$2.2} && 7.5\std{$\pm$0.7}  & 7.5\std{$\pm$0.7}  & 7.5\std{$\pm$0.7} && 0.5\std{$\pm$0.1}   & 0.5\std{$\pm$0.1} & 0.5\std{$\pm$0.1} \\ 
\hline
AGEM~\cite{AGEM}    & 17.7\std{$\pm$0.3} & 17.5\std{$\pm$0.3} & 17.5\std{$\pm$0.2} && 5.8\std{$\pm$0.1}  & 5.9\std{$\pm$0.1}  & 5.8\std{$\pm$0.1} && 0.8\std{$\pm$0.1}   & 0.8\std{$\pm$0.1} & 0.8\std{$\pm$0.1} \\ 
GSS~\cite{GSS}     & 18.4\std{$\pm$0.2} & 19.4\std{$\pm$0.7} & 25.2\std{$\pm$0.9} && 8.1\std{$\pm$0.2}  & 9.4\std{$\pm$0.5}  & 10.1\std{$\pm$0.8} && 1.1\std{$\pm$0.1}   & 1.5\std{$\pm$0.1} & 2.4\std{$\pm$0.4} \\ 
ER~\cite{ER}      & 19.4\std{$\pm$0.6} & 20.9\std{$\pm$0.9} & 26.0\std{$\pm$1.2} && 8.7\std{$\pm$0.3}  & 9.9\std{$\pm$0.5}  & 10.7\std{$\pm$0.8} && 1.2\std{$\pm$0.1}   & 1.5\std{$\pm$0.2} & 2.0\std{$\pm$0.2} \\ 
MIR~\cite{MIR}     & 20.7\std{$\pm$0.7} & 23.5\std{$\pm$0.8} & 29.9\std{$\pm$1.2} && 9.7\std{$\pm$0.3}  & 11.2\std{$\pm$0.4}  & 13.0\std{$\pm$0.7} && 1.4\std{$\pm$0.1}   & 1.9\std{$\pm$0.2} & 2.9\std{$\pm$0.3} \\ 
GDumb~\cite{GDumb}   & 23.3\std{$\pm$1.3} & 27.1\std{$\pm$0.7} & 34.0\std{$\pm$0.8} && 8.2\std{$\pm$0.2}  & 11.0\std{$\pm$0.4}  & 15.3\std{$\pm$0.3} && 4.6\std{$\pm$0.3}   & 6.6\std{$\pm$0.2} & 10.0\std{$\pm$0.3} \\ 
ASER~\cite{ASER}   & 20.0\std{$\pm$1.0} & 22.8\std{$\pm$0.6} & 31.6\std{$\pm$1.1} && 11.0\std{$\pm$0.3}  & 13.5\std{$\pm$0.3}  & 17.6\std{$\pm$0.4} && 2.2\std{$\pm$0.1}   & 4.2\std{$\pm$0.6} & 8.4\std{$\pm$0.7} \\ 
SCR~\cite{SCR}     & 40.2\std{$\pm$1.3} & 48.5\std{$\pm$1.5} & 59.1\std{$\pm$1.3} && 19.3\std{$\pm$0.6}  & 26.5\std{$\pm$0.5}  & 32.7\std{$\pm$0.3} && 8.9\std{$\pm$0.3}   & 14.7\std{$\pm$0.3} & 19.5\std{$\pm$0.3} \\ 
CoPE~\cite{online_pro_ema}  & 33.5\std{$\pm$3.2} & 37.3\std{$\pm$2.2} & 42.9\std{$\pm$3.5} && 11.6\std{$\pm$0.7}  & 14.6\std{$\pm$1.3}  & 16.8\std{$\pm$0.9} && 2.1\std{$\pm$0.3}   & 2.3\std{$\pm$0.4} & 2.5\std{$\pm$0.3} \\
DVC~\cite{DVC} & 35.2\std{$\pm$1.7}  & 41.6\std{$\pm$2.7} & 53.8\std{$\pm$2.2} &&  15.4\std{$\pm$0.7} & 20.3\std{$\pm$1.0} & 25.2\std{$\pm$1.6} && 4.9\std{$\pm$0.6} &  7.5\std{$\pm$0.5} & 10.9\std{$\pm$1.1} \\ 
OCM~\cite{OCM} & 47.5\std{$\pm$1.7}  & 59.6\std{$\pm$0.4} & 70.1\std{$\pm$1.5} && 19.7\std{$\pm$0.5} & 27.4\std{$\pm$0.3} & 34.4\std{$\pm$0.5} && 10.8\std{$\pm$0.4} & 15.4\std{$\pm$0.4} & 20.9\std{$\pm$0.7} \\ 
\hline
\frameworkName (\textbf{ours}) & \textbf{57.8}\std{$\pm$1.1} & \textbf{65.5}\std{$\pm$1.0} & \textbf{72.6}\std{$\pm$0.8} && \textbf{22.7}\std{$\pm$0.7} & \textbf{30.0}\std{$\pm$0.4} & \textbf{35.9}\std{$\pm$0.6} && \textbf{11.9}\std{$\pm$0.3} & \textbf{16.9}\std{$\pm$0.4} &  \textbf{22.1}\std{$\pm$0.4}
\\ 
\shline
\end{tabular}
}
\end{center}
\caption{Average Accuracy~(higher is better) on three benckmark datasets with different memory bank sizes $M$. All results are the average and standard deviation of 15 runs.}
\label{tab:acc}
\end{table*}

\section{Experiments}
\subsection{Experimental Setup}
\paragraph{Datasets.}
We use three image classification benchmark datasets, including \textbf{CIFAR-10}~\cite{cifar10_100}, \textbf{CIFAR-100}~\cite{cifar10_100}, and \textbf{TinyImageNet}~\cite{tinyImageNet}, to evaluate the performance of online CIL methods. 
Following~\cite{ASER, SCR, DVC}, we split CIFAR-10 into 5 disjoint tasks, where each task has 2 disjoint classes, 10,000 samples for training, and 2,000 samples for testing, and split CIFAR-100 into 10 disjoint tasks, where each task has 10 disjoint classes, 5,000 samples for training, and 1,000 samples for testing.
Following~\cite{OCM}, we split TinyImageNet into 100 disjoint tasks, where each task has 2 disjoint classes, 1,000 samples for training, and 100 samples for testing.
Note that the order of tasks is fixed in all experimental settings.

\paragraph{Baselines.}
We compare our \frameworkName with 13 baselines, including 10 replay-based online CL baselines: {AGEM}~\cite{AGEM}, {MIR}~\cite{MIR}, {GSS}~\cite{GSS}, {ER}~\cite{ER}, {GDumb}~\cite{GDumb}, {ASER}~\cite{ASER}, {SCR}~\cite{SCR}, {CoPE}~\cite{online_pro_ema}, {DVC}~\cite{DVC}, and {OCM}~\cite{OCM}; 3 offline CL baselines that use knowledge distillation by running them in one epoch: {iCaRL}~\cite{iCaRL}, {DER++}~\cite{DER++}, and PASS~\cite{protoAug}. Note that PASS is a non-exemplar method.

\paragraph{Evaluation metrics.}
We use Average Accuracy and Average Forgetting~\cite{ASER, DVC} to measure the performance of our framework in online CIL. Average Accuracy evaluates the accuracy of the test sets from all seen tasks, defined as $\text {Average Accuracy} =\frac{1}{T} \sum_{j=1}^T a_{T, j},$
where $a_{i, j}$ is the accuracy on task $j$ after the model is trained from task $1$ to $i$.
Average Forgetting represents how much the model forgets about each task after being trained on the final task, defined as
$\text { Average Forgetting } =\frac{1}{T-1} \sum_{j=1}^{T-1} f_{T, j}, 
\text { where } f_{i, j}=\max _{k \in\{1, \ldots, i-1\}} a_{k, j}-a_{i, j}.$

\paragraph{Implementation details.}
We use ResNet18~\cite{ResNet} as the backbone $f$ and a linear layer as the projection head $g$ like~\cite{SCR, OCM, Co2L}; the hidden dim in $g$ is set to 128 as~\cite{SimCLR}. We also employ a linear layer as the classifier $\varphi$. We train the model from scratch with Adam optimizer and an initial learning rate of $5\times10^{-4}$ for all datasets. The weight decay is set to $1.0\times10^{-4}$. Following~\cite{ASER, DVC}, we set the batch size $N$ as 10, and following~\cite{OCM} the replay batch size $m$ is set to 64. 
For CIFAR-10, we set the ratio of \dataaugname $\alpha = 0.25$. For CIFAR-100 and TinyImageNet, $\alpha$ is set to $0.1$. The temperature $\tau = 0.5$ and $\tau^{\prime} = 0.07$.
For baselines, we also use ResNet18 as their backbone and set the same batch size and replay batch size for fair comparisons.
We reproduce all baselines in the same environment with their source code and default settings; see Appendix~\ref{appendix:baselines} for implementation details about all baselines.
We report the average results across 15 runs for all experiments.



\paragraph{Data augmentation.}
Similar to data augmentations used in SimCLR~\cite{SimCLR}, we use resized-crop, horizontal-flip, and gray-scale as our data augmentations. For all baselines, we also use these augmentations. In addition, for DER++\cite{DER++}, SCR~\cite{SCR}, and DVC~\cite{DVC}, we follow their default settings and use their own extra data augmentations. OCM~\cite{OCM} uses extra rotation augmentations, which are also used in \frameworkName.


\subsection{Motivation Justification}
\label{pre_exp}
\paragraph{Shortcut learning in online CL.}
Shortcut learning is severe in online CL since the model cannot learn sufficient representative features due to the single-pass data stream. To intuitively demonstrate this issue,  
we conduct GradCAM++~\cite{Grad-cam++} on the training set of CIFAR-10 ($M=0.2k$) after the model is trained incrementally, as shown in Fig.~\ref{fig:heatmap}.
Each row in Fig.~\ref{fig:heatmap} represents a task with two classes.
We can observe that although ER and DVC predict the correct class, the models actually take shortcuts and focus on some object-unrelated features. 
An interesting phenomenon is that ER tends to take shortcuts in each task. For example, ER learns the sky on both the airplane class in task 1 (the first row) and the bird class in task 2 (the second row) . Thus, ER forgets almost all the knowledge of the old classes.  
DVC maximizes the mutual information between instances like contrastive learning~\cite{SimCLR, MoCo}, which only partially alleviates shortcut learning in online CL. 
In contrast, \frameworkName focuses on the representative features of the objects themselves. The results confirm that learning representative features is crucial against shortcut learning; see Appendix~\ref{appendix:more_visual} for more visual explanations.


\begin{table*}[htbp]
\small
\begin{center}
\resizebox{\linewidth}{!}{
\begin{tabular}{rrrrrrrrrrrr}
\shline
\multirow{2}{*}{Method}  & \multicolumn{3}{c}{CIFAR-10}   && \multicolumn{3}{c}{CIFAR-100}  && \multicolumn{3}{c}{TinyImageNet} \\ \cline{2-4}\cline{6-8}\cline{10-12}
       &  $M=0.1k$   &  $M=0.2k$   &  $M=0.5k$     &&  $M=0.5k$     &  $M=1k$     &  $M=2k$    &&  $M=1k$      &  $M=2k$ &  $M=4k$    \\ \midrule
iCaRL~\cite{iCaRL}    & 52.7\std{$\pm$1.0} & 49.3\std{$\pm$0.8} & 38.3\std{$\pm$0.9} && 16.5\std{$\pm$1.0}  & 11.2\std{$\pm$0.4}  & 10.4\std{$\pm$0.4} && 9.9\std{$\pm$0.5}   & 10.1\std{$\pm$0.5} & 9.7\std{$\pm$0.6} \\ 
DER++~\cite{DER++}   & 57.8\std{$\pm$4.1} & 46.7\std{$\pm$3.6} & 33.6\std{$\pm$3.5} && 41.0\std{$\pm$1.1} & 34.8\std{$\pm$1.1} & 33.2\std{$\pm$1.2} && 77.8\std{$\pm$1.0} & 74.9\std{$\pm$0.6} & 73.2\std{$\pm$0.8}  \\ 
PASS~\cite{protoAug}    & 21.2\std{$\pm$2.2} & 21.2\std{$\pm$2.2} & 21.2\std{$\pm$2.2} && 10.6\std{$\pm$0.9}  & 10.6\std{$\pm$0.9}  & 10.6\std{$\pm$0.9} && 27.0\std{$\pm$2.4}   & 27.0\std{$\pm$2.4} & 27.0\std{$\pm$2.4} \\ 
\hline
AGEM~\cite{AGEM}    & 64.8\std{$\pm$0.7} & 64.8\std{$\pm$0.7} & 64.5\std{$\pm$0.5} && 41.7\std{$\pm$0.8} & 41.8\std{$\pm$0.7} & 41.7\std{$\pm$0.6} && 73.9\std{$\pm$0.7} & 73.1\std{$\pm$0.7} & 72.9\std{$\pm$0.5} \\ 
GSS~\cite{GSS}     & 67.1\std{$\pm$0.6} & 65.8\std{$\pm$0.6} & 61.2\std{$\pm$1.2} && 48.7\std{$\pm$0.8} & 46.7\std{$\pm$1.3} & 44.7\std{$\pm$1.1} && 78.9\std{$\pm$0.7} & 77.0\std{$\pm$0.5} & 75.2\std{$\pm$0.7} \\ 
ER~\cite{ER}      & 64.7\std{$\pm$1.1} & 62.9\std{$\pm$1.0} & 57.5\std{$\pm$1.8} && 47.0\std{$\pm$1.0} & 46.4\std{$\pm$0.8} & 44.7\std{$\pm$1.5} && 79.1\std{$\pm$0.6} & 77.7\std{$\pm$0.6} & 76.3\std{$\pm$0.5} \\ 
MIR~\cite{MIR}     & 62.6\std{$\pm$1.0} & 58.5\std{$\pm$1.4} & 51.1\std{$\pm$1.1} && 45.7\std{$\pm$0.9} & 44.2\std{$\pm$1.3} & 42.3\std{$\pm$1.0} && 75.3\std{$\pm$0.9} & 71.5\std{$\pm$1.0} & 66.8\std{$\pm$0.8} \\ 
GDumb~\cite{GDumb}   & 28.5\std{$\pm$1.4} & 28.4\std{$\pm$1.0} & 28.1\std{$\pm$1.0} && 25.0\std{$\pm$0.4} & 23.2\std{$\pm$0.4} & 20.7\std{$\pm$0.3}  && 22.7\std{$\pm$0.3} & 18.4\std{$\pm$0.2} & 17.0\std{$\pm$0.2} \\
ASER~\cite{ASER}    & 64.8\std{$\pm$1.0} & 62.6\std{$\pm$1.1} & 53.2\std{$\pm$1.5} && 52.8\std{$\pm$0.8} & 50.4\std{$\pm$0.9} & 46.8\std{$\pm$0.7} && 78.9\std{$\pm$0.5} & 75.4\std{$\pm$0.7} & 68.2\std{$\pm$1.1} \\ 
SCR~\cite{SCR}     & 43.2\std{$\pm$1.5} & 35.5\std{$\pm$1.8} & 24.1\std{$\pm$1.0} && 29.3\std{$\pm$0.9} & 20.4\std{$\pm$0.6} & 11.5\std{$\pm$0.6} && 44.8\std{$\pm$0.6} & 26.8\std{$\pm$0.5} & 20.1\std{$\pm$0.4} \\ 
CoPE~\cite{online_pro_ema}  & 49.7\std{$\pm$1.6} & 45.7\std{$\pm$1.5} & 39.4\std{$\pm$1.8} && 25.6\std{$\pm$0.9}  & 17.8\std{$\pm$1.3}  & 14.4\std{$\pm$0.8} && 11.9\std{$\pm$0.6}   & 10.9\std{$\pm$0.4} & 9.7\std{$\pm$0.4} \\
DVC~\cite{DVC} & 40.2\std{$\pm$2.6} & 31.4\std{$\pm$4.1} & 21.2\std{$\pm$2.8} && 32.0\std{$\pm$0.9} & 32.7\std{$\pm$2.0} & 28.0\std{$\pm$2.2} && 59.8\std{$\pm$2.2} & 52.9\std{$\pm$1.3} & 45.1\std{$\pm$1.9} \\
OCM~\cite{OCM} & 35.5\std{$\pm$2.4} & 23.9\std{$\pm$1.4} & 13.5\std{$\pm$1.5} && 18.3\std{$\pm$0.9} & 15.2\std{$\pm$1.0} & 10.8\std{$\pm$0.6} && 23.6\std{$\pm$0.5} & 26.2\std{$\pm$0.5}  & 23.8\std{$\pm$1.0} \\ 
\hline
{\frameworkName} (\textbf{ours})   & 23.2\std{$\pm$1.3} & 17.6\std{$\pm$1.4} & 12.5\std{$\pm$0.7} && 
15.0\std{$\pm$0.8} & 10.4\std{$\pm$0.5} & 6.1\std{$\pm$0.6} && 21.3\std{$\pm$0.5} & 17.4\std{$\pm$0.4} & 16.8\std{$\pm$0.4} \\
\shline
\end{tabular}
}
\end{center}
\caption{Average Forgetting~(lower is better) on three benckmark datasets. All results are the average and standard deviation of 15 runs.}
\label{tab:forget}
\end{table*}

\begin{figure*}[htp]
  \centering
  \subfloat[Average incremental performance]{
    \includegraphics[width=0.55\linewidth]{Figures/imgs/incremental_step_acc.pdf}
    \label{fig:incrementalAcc}
  }
  \subfloat[Confusion matrix of OCM and \frameworkName]{
    \includegraphics[width=0.42\linewidth]{Figures/imgs/confusion_matrix.pdf}
    \label{fig:confusionMatrix}
  }
  \caption{Incremental accuracy on tasks observed so far and confusion matrix of accuracy (\%) in the {test set} of CIFAR-10.}
  \label{fig:incrementalAcc_confusionMatrix}
\end{figure*}



\paragraph{Class confusion in online CL.}
Fig.~\ref{fig:tsne_motivation} provides the $t$-SNE~\cite{tsne} visualization results for ER and \frameworkName on the test set of CIFAR-10 ($M=0.2k$). 
We can draw intuitive observations as follows. 
(1) There is serious class confusion in ER.
When the new task (task 2) arrives, features learned in task 1 are not discriminative for task 2, leading to class confusion and decreased performance in old classes.
(2) Shortcut learning may cause class confusion. For example, the performance of ER decreases more on airplanes compared to automobiles, probably because birds in the new task have more similar backgrounds to airplanes, as shown in Fig.~\ref{fig:heatmap}.
(3) \frameworkName achieves better discrimination both on task 1 and task 2. The results demonstrate that \frameworkName can maintain discrimination of all seen classes and significantly mitigate forgetting by 
combining the proposed \methodname and \dataaugname.






\subsection{Results and Analysis}
\label{result}
\paragraph{Performance of average accuracy.}
Table~\ref{tab:acc} presents the results of average accuracy with different memory bank sizes ($M$) on three benchmark datasets. Our \frameworkName consistently outperforms all baselines on three datasets.
Remarkably, the performance improvement of \frameworkName is more significant when the memory bank size is relatively small; this is critical for online CL with limited resources. For example, compared to the second-best method OCM, \frameworkName achieves about 10$\%$ and 6$\%$ improvement on CIFAR-10 when $M$ is 100 and 200, respectively. 
The results show that our \frameworkName can learn more representative and discriminative features with a limited memory bank.
Compared to baselines that use knowledge distillation (iCaRL, DER++, PASS, OCM), our \frameworkName achieves better performance by leveraging the feedback of online prototypes.  
Besides, \frameworkName significantly outperforms PASS and CoPE that also use prototypes, showing that online prototypes are more suitable for online CL. 


We find that the performance improvement tends to be gentle when $M$ increases.
The reason is that as $M$ increases, the samples in the memory bank become more diverse, and the model can extract sufficient information from massive samples to distinguish seen classes. 
In addition, many baselines perform poorly on CIFAR-100 and TinyImageNet due to a dramatic increase in the number of tasks. In contrast, \frameworkName still performs well and improves accuracy over the second best.



\paragraph{Performance of average forgetting.}
We report the Average Forgetting results of our \frameworkName and all baselines on three benchmark datasets in Table~\ref{tab:forget}. The results confirm that \frameworkName can effectively mitigate catastrophic forgetting. 
For CIFAR-10 and CIFAR-100, \frameworkName achieves the lowest average forgetting compared to all replay-based baselines. 
For TinyImageNet, our result is a little higher than iCaRL and CoPE but better than the latest methods DVC and OCM. 
The reason is that iCaRL uses a nearest class mean classifier, but we use softmax and FC layer during the test phase, and CoPE slowly updates prototypes with a high momentum.
However, as shown in Table~\ref{tab:acc}, \frameworkName provides more accurate classification results than iCaRL and CoPE. 
It is a fact that when the maximum accuracy of a task is small, the forgetting on this task is naturally rare, even if the model completely forgets what it learned.





\paragraph{Performance of each incremental step.}
We evaluate the average incremental performance~\cite{DER++, DVC} on CIFAR-10 ($M=0.1k$) and CIFAR-100 ($M=0.5k$), which indicates the accuracy over all seen tasks at each incremental step. 
Fig.~\ref{fig:incrementalAcc} shows that \frameworkName achieves better accuracy and effectively mitigates forgetting while the performance of most baselines degrades rapidly with the arrival of new classes.

\paragraph{Confusion matrices at the end of learning.}
We report the confusion matrices of our \frameworkName and the second-best method OCM, as shown in Fig.~\ref{fig:confusionMatrix}. 
After learning the last task (\ie, the last two classes), OCM forgets the knowledge of early tasks (classes 0 to 3). 
In contrast, \frameworkName performs relatively well in all classes, especially in the first task (classes 0 and 1), outperforming OCM by 27.8\% average improvements.
The results show that learning representative and discriminative features is crucial to mitigate catastrophic forgetting; see Appendix~\ref{appendix:extra_exp} for extra experimental results.  




\subsection{Ablation Studies}
\label{ablation}

\begin{table}[t]
\small
\begin{center}
\begin{tabular}{ccccc}
\shline
\multirow{2}{*}{{Method}} & {CIFAR-10}&{CIFAR-100} \\
& Acc $\uparrow$(Forget $\downarrow$) & Acc $\uparrow$(Forget $\downarrow$) \\ 
\midrule
baseline & 46.4\std{$\pm$1.2}(36.0\std{$\pm$}2.1) & 18.8\std{$\pm$0.8}(18.5\std{$\pm$}0.7) \\
w/o \methodname & 53.1\std{$\pm$1.4}(24.7\std{$\pm$2.0}) & 19.3\std{$\pm$0.7}(15.9\std{$\pm$0.9}) \\
w/o \dataaugname & 52.0\std{$\pm$1.5}(34.6\std{$\pm$2.4}) & 21.5\std{$\pm$0.5}(16.3\std{$\pm$0.8}) \\ 
\hline
w/o $\mathcal{L}^{\mathrm{new}}_{\mathrm{pro}}$ & 54.8\std{$\pm$1.2}(\textbf{22.1}\std{$\pm$3.0}) & 19.6\std{$\pm$0.8}(19.9\std{$\pm$0.7}) \\
w/o $\mathcal{L}^{\mathrm{seen}}_{\mathrm{pro}}$ & 55.7\std{$\pm$1.4}(25.5\std{$\pm$1.5}) & 20.1\std{$\pm$0.4}(16.2\std{$\pm$0.6}) \\ 
$\mathcal{L}^{\mathrm{seen}}_{\mathrm{pro}}$ w/o $\mathcal{C}^\mathrm{new}$ & 56.2\std{$\pm$1.2}(26.4\std{$\pm$2.3}) & 20.8\std{$\pm$0.6}(17.9\std{$\pm$0.7}) \\ 
\hline
{\frameworkName} (\textbf{ours}) & \textbf{57.8}\std{$\pm$1.1}(23.2\std{$\pm$1.3}) & \textbf{22.7}\std{$\pm$0.7}(\textbf{15.0}\std{$\pm$0.8}) \\ 
\shline 
\end{tabular}
\end{center}
\caption{Ablation studies on CIFAR-10 ($M=0.1k$) and CIFAR-100 ($M=0.5k$). 
``baseline'' means $\mathcal{L}_\mathrm{INS}+\mathcal{L}_\mathrm{CE}$.
``$\mathcal{L}^{\mathrm{seen}}_{\mathrm{pro}}$ w/o $\mathcal{C}^\mathrm{new}$'' means $\mathcal{L}^{\mathrm{seen}}_{\mathrm{pro}}$ do not consider new classes in current task.
}
\label{tab:ablation}
\end{table}

\paragraph{Effects of each component.} Table~\ref{tab:ablation} presents the ablation results of each component. Obviously, \methodname and \dataaugname can consistently improve the average accuracy of classification. 
We can observe that the effect of \methodname is more significant on more tasks while \dataaugname plays a crucial role when the memory bank size is limited. Moreover, when combining \methodname and \dataaugname, the performance is further improved, which indicates that both can benefit from each other. For example, \dataaugname boosts \methodname by about 6$\%$ improvements on CIFAR-10 ($M=0.1k$), and the performance of \dataaugname is improved by about 3$\%$ on CIFAR-100 ($M=0.5k$) by combining \methodname.


\paragraph{Equilibrium in \methodname.}
When learning new classes, the data of new classes is involved in both $\mathcal{L}^{\mathrm{new}}_{\mathrm{pro}}$ and $\mathcal{L}^{\mathrm{seen}}_{\mathrm{pro}}$ of \methodname, where $\mathcal{L}^{\mathrm{new}}_{\mathrm{pro}}$ only focuses on learning new knowledge while $\mathcal{L}^{\mathrm{seen}}_{\mathrm{pro}}$ tends to alleviate forgetting on seen classes.
To explore the best way of learning new classes, we consider three scenarios for \methodname in Table~\ref{tab:ablation}.
The results show that only learning new knowledge (w/o $\mathcal{L}^{\mathrm{seen}}_{\mathrm{pro}}$) or only consolidating the previous knowledge (w/o $\mathcal{L}^{\mathrm{new}}_{\mathrm{pro}}$) can significantly degrade the performance, which indicates that both are indispensable for online CL.
Furthermore, when $\mathcal{L}^{\mathrm{seen}}_{\mathrm{pro}}$ only considers old classes and ignores new classes ($\mathcal{L}^{\mathrm{seen}}_{\mathrm{pro}}$ w/o $\mathcal{C}^\mathrm{new}$), the performance also decreases. These results show that the equilibrium of all seen classes (\methodname) can achieve the best performance and is crucial for online CL.


\paragraph{Effects of \dataaugname.} 
To verify the advantage of \dataaugname, we compare it with the completely random mixup
in Table~\ref{tab:ablation_mixup}.
\begin{table}
\small
\begin{center}
\begin{tabular}{c|rrr}
\shline
\multicolumn{1}{c|}{Method}       & ${M=0.1k}$   & ${M=0.2k}$   & ${M=0.5k}$     \\ \hline
Random & 53.5\std{$\pm$2.7} & 62.9\std{$\pm$2.5} & 70.8\std{$\pm$2.2} \\
\dataaugname (\textbf{ours})  & \textbf{57.8}\std{$\pm$1.1} & \textbf{65.5}\std{$\pm$1.0} & \textbf{72.6}\std{$\pm$0.8} \\ 
\shline
\end{tabular}
\end{center}
\caption{Comparison of Random Mixup and \dataaugname on CIFAR-10. 
}
\label{tab:ablation_mixup}
\end{table}
\dataaugname outperforms random mixup in all three scenarios. Notably, \dataaugname works significantly when the memory bank size is small, which shows that the feedback can prevent class confusion due to a restricted memory bank; see Appendix~\ref{appendix:ablations} for extra ablation studies.



\subsection{Validation of Online Prototypes}
\label{prove_onlinePrototypes}
\begin{figure}
    \centering
    \includegraphics[width=1.0\linewidth]{Figures/imgs/cosine_similarity.pdf}
    \caption{The cosine similarity between online prototypes and prototypes of the entire memory bank.}
    \label{fig:cosine_similarity}
\end{figure}
Fig.~\ref{fig:cosine_similarity} shows the cosine similarity between online prototypes and global prototypes (prototypes of the entire memory bank) at each time step.
For the first mini-batch of each task, online prototypes are equal to global prototypes (similarity is 1, omitted in Fig.~\ref{fig:cosine_similarity}).
In the first task, online and global prototypes are updated synchronously with the model updates, resulting in high similarity. 
In subsequent tasks, the model initially learns inadequate features of new classes, causing online prototypes to be inconsistent with global prototypes and low similarity, which shows that accumulating early features as prototypes may be harmful to new tasks. However, the similarity will improve as the model learns, because the model gradually learns representative features of new classes.
Furthermore, the similarity on old classes is only slightly lower, showing that online prototypes are resistant to forgetting. 


\section{Conclusion}
\section{Conclusion}
In this work, we present a novel strategy for addressing few-shot open-set recognition. We frame the few-shot open-set classification task as a meta-learning problem similar to \cite{peeler}, but unlike their strategy, we do not solely rely on thresholding softmax scores to indicate the openness of a sample. We argue that existing thresholding type FSOSR methods \cite{peeler,snatcher} rely heavily on the choice of a carefully tuned threshold to achieve good performance. Additionally, the proclivity of softmax to overfit to unseen classes makes it an unreliable choice as an open-set indicator, especially when there is a dearth of samples. Instead, we propose to use a reconstruction of exemplar images as a key signal to detect out-of-distribution samples. 
The learned embedding which is used to classify the sample is further modulated to ensure a proficient gap between the seen and unseen class clusters in the feature space. Finally, the modulated embedding, the softmax score, and the quality reconstructed exemplar are jointly utilized to cognize if the sample is in-distribution or out-of-distribution. 
The enhanced performance of our framework is verified empirically over a wide variety of few-shot tasks and the results establish it as the new state-of-the-art. In the future, we would like to extend this approach to more cross-domain few-shot tasks, including videos.
\vspace{-2em}
\section{Acknowledgement}
This work was partially supported by US National Science Foundation grant 2008020 and US Office of Naval Research grants N00014-19-1-2264 and N00014-18-1-2252.
\vspace{-1em}

\balance

%\addtolength{\textheight}{-12cm}   % This command serves to balance the column lengths
                                  % on the last page of the docume+nt manually. It shortens
                                  % the textheight of the last page by a suitable amount.
                                  % This command does not take effect until the next page
                                  % so it should come on the page before the last. Make
                                  % sure that you do not shorten the textheight too much.



\bibliographystyle{IEEEtran}
\bibliography{ref}




\end{document}