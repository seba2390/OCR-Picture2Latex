\section{Conclusions}
\label{sec.conclusion}

% A key contribution of D2A is
% the use of a formal software/hardware interface that specifies
% an accelerator’s operations and their semantics. Specifically, we
% leverage the Instruction-Level Abstraction (ILA) formal specifi-
% cation for accelerators that has been successfully used thus far for
% accelerator implementation verification. We show how the ILA
% for accelerators serves as a software/hardware interface, similar
% to the Instruction-Set Architecture (ISA) for processors, that can
% be used for automated development of compilers and instruction-
% level simulators to significantly reduce accelerator development
% costs. Another key contribution of this work is to show how
% ILA-based accelerator semantics enables extending recent work
% on equality saturation to auto-generate basic compiler support
% for prototype accelerators in a technique we term “flexible
% matching.”


%\hl{[Sharad]}
In this work we address the key gaps %preventing
%that make it challenging to perform
hindering
application-level evaluation of accelerator designs, especially during early design stages.
%(i) the lack of compiler support for automatically identifying acceleration opportunities without modifying the applications
%%/compiler infrastructure with accelerator specific information
%or developing bespoke compiler infrastructure,
%and (ii) the lack of support for automated high-level simulator generation for application-level co-simulation. We address these gaps 
We propose the \TLA methodology that contributes 
\begin{inlinelist}
\item the use of a formal software/hardware interface for specifying accelerator operations,
%, where we use the recently developed ILA specification %for this purpose 
which enables identifying acceleration opportunities and automatically generating correct high-level simulators, and 
\item compiler rewrites and equality saturation for flexible matching, which 
%finds matches beyond exact syntactic matches without manually modifying the application. 
facilitates automatically searching through a large space of equivalent programs to find %more 
acceleration opportunities. %than can be found by exact syntactic matching.
\end{inlinelist}
%
We provide a \TLA prototype implementation for DL applications using the TVM 
and ILAng frameworks
and evaluate it %its capabilities
%. We evaluate the capabilities of this prototype 
through automated compilation of \AppNum applications on three different accelerator platforms. 
% These evaluations show the power of equality saturation--based flexible matching 
%   % by identifying more acceleration opportunities than exact syntactic matching could,
%   thanks to rewrite rules
%   that can be reused across applications.
% %through mapping of a larger set of accelerator-supported operations in DSL applications than is possible using exact matching. 
% %Further, we discuss case studies that highlight automated application-level simulation-based validation to check the acceptability of application-level results even with deviations at the operation level due to numerics. One of the case studies exposed precision issues which required modification of the accelerator numerics, demonstrating its use in software/hardware co-design. 
% An additional case study highlighted 
%   the utility of {\TLA}'s automated application-level co-simulation
% %  for validating accelerator designs,  particularly in the face of custom numerics:
%   by exposing numerical precision issues
%   in an accelerator that required modifications to the design,
%   demonstrating our approach's strength in supporting
%   rapid exploration in software/hardware co-design.
% \emph{We believe this is the first mostly automated compilation and simulation feedback loop that addresses application-level accelerator evaluation---no longer limiting it to
% large enterprises that can afford teams of hardware, software, and system experts.} 
% Further, this methodology also provides a foundation for full-fledged integration of support for specialized accelerators in traditional compiler pipelines.
% %and not limit this to

