\section{Background}
\label{sec.background}

%Our compiler and simulation infrastructure builds on two foundational components: the ILA software/hardware interface and term rewriting using equality saturation. Here, we briefly review them highlighting aspects that enable our methodology.

\subsection{ILA Software/Hardware Interface Specification}
\label{sec:ila}
%A principal component in the {\TLA} methodology is an ILA specification of the target accelerator~\cite{huang2018instruction}.
%  e.g., FlexASR and HLSCNN in this case.
  %which is, in this case, FlexASR.
%The ILA models accelerator operations in terms of instructions with associated state transition functions, similarly to an ISA for a programmable processor.
The ILA is an ISA-like formal model 
  for specifying the functional behavior of accelerators.
It generalizes the ISA to accelerators, 
%and provides a modular functional specification as a set of instructions. Each 
  where each 
  instruction of an accelerator ILA corresponds 
  to a command at the accelerator interface, i.e., an MMIO load or store from a host processor.
Like processor ISAs, 
  the ILA captures a formal semantics of the accelerator behavior, by specifying how each instruction 
  reads or updates software-visible (viz., architectural) state variables in the accelerator,
  while abstracting out implementation details. 
  
Fig.~\ref{fig.ila-example} shows an example
  ILA specification for one of the instructions of FlexASR~\cite{tambe20219} (one of the three accelerators used in our evaluation studies). The ILA models are written in ILAng, a DSL embedded in C++. The figure caption points out the per-instruction modular specification, where each instruction is specified by defining its decode condition (i.e., when the instruction is triggered) and state update functions (i.e., how it updates
  the architectural state variables). 
  %updates are specified.

Thus far, the ILA has been used only for accelerator implementation verification and co-verification of firmware~\cite{huang2018formal,huang2018instruction}.
In this work, we use ILA as a formal SW/HW interface that drives the key tasks in compilation and application-level testing for accelerators.

% \begin{figure*}[!htb]
% changed BITWIDTH to BITS because of column margins
\begin{figure}[!ht]
  \begin{minipage}[h]{0.31\textwidth}
    \vspace{-3\fboxsep}
    \caption{
    \textbf{ILA model (snippet) for FlexASR accelerator.}
    Lines 3-11 define the inputs and architectural state variables.
    Lines 14-22 show an example ILA instruction 
    ``\texttt{\small pe\_0\_cfg\_mngr}.'' 
    Its decode condition (lines 16-17) specifies that this instruction is triggered when there is a write command to the %address of the 
    processing element's (PE) management configuration register.
    Its state update functions (lines 19-22) specify that this instruction stores arguments from the interface inputs into the corresponding configuration registers. The state update functions of instructions such as for linear layer (elided here) %(not shown here as they are much longer) 
    encode their operational semantics.
    This snippet highlights:
    (1) similarity of ILA with ISA, and
    (2) the formal semantics based on how each instruction reads/writes the architectural state variables.
    }
    \label{fig.ila-example}
  \end{minipage}\hfill
  %\hspace{15pt}
  \begin{minipage}[h]{0.66\textwidth}
    \begin{lstlisting}[language=c]
auto m = ilang::Ila("flexasr-ila");
// declare inputs at the interface
auto wr = m.NewBvInput("top_if_wr", TOP_IF_WR_BITS);
auto rd = m.NewBvInput("top_if_rd", TOP_IF_RD_BITS);
auto addr = m.NewBvInput("top_addr_in", TOP_ADDR_IN_BITS);
auto data = m.NewBvInput("top_data_in", TOP_DATA_IN_BITS);
// declare architectural states
m.NewBvState("pe_0_is_valid", PE_VALID_BITS);
m.NewBvState("pe_0_is_bias", PE_IS_BIAS_BITS);
m.NewMemState("gb_large_buffer", TOP_ADDR_IN_BITS, TOP_DATA_IN_BITS);
// ... (some code)
// ILA instruction for configuring pe_cfg_mngr
auto instr = m.NewInstr("pe_0_cfg_mngr");
// define decode condition for this instruction
auto is_write = (wr == 1) & (rd == 0);
instr.SetDecode(is_write & (addr == PE_0_CFG_MNGR_ADDR));
// define state update functions for this instruction
auto is_valid = ilang::SelectBit(data, PE_IS_VALID_BIT_IDX);
instr.SetUpdate(m.state("pe_0_is_valid"), is_valid);
auto is_bias = ilang::SelectBit(data, PE_IS_BIAS_BIT_IDX);
instr.SetUpdate(m.state("pe_0_is_bias"), is_bias);
// ... (more code)
    \end{lstlisting}
  \end{minipage}%\hfill
\Description{}
\end{figure}


% the ILA provides several benefits for validating an accelerator design in the context of the {\TLA} methodology.
%The specification can increase the level of confidence in the design's correctness because ILA specifications can be verified against RTL implementations, as has been done successfully in prior work~\cite{huang2018instruction}.
%The formally defined semantics of the ILA also make it possible to automatically generate an executable software model for ILA instructions, which is implemented in the ILAng platform~\cite{huang2019ilang}.
%Additionally, a revision to the accelerator design can easily be tested simply by making changes to the comparatively high-level ILA specification, which is much simpler than directly changing the hardware design; indeed, it is possible to write an ILA specification for a device whose design has not yet been finished.
%Also, since ILA instructions correspond one-to-one with the individual MMIO commands that operate the accelerator,
 %that are used to operate accelerators,
 %which means that ILA instructions can be easily lowered
 %ILA instructions can be directly lowered to MMIO commands for invoking accelerator operations from the application code running on a host processor.
 %as we also perform in our evaluation in Section~\ref{sec.eval}.

%Figure~\ref{fig.ila-example} in Appendix~\ref{app.fragments} provides an example of the ILA specification for one of FlexASR's operations.
% We'll move to section 5
%The ILA specifications for two of the accelerators used in our evaluations, FlexASR~\cite{tambe20219} and HLSCNN~\cite{whatmough201916nm}
  %are about 5600 and 1600 lines of code, respectively;
  %their High-Level Synthesis (HLS) C-language implementations
  %are about 9300 and 5100 lines of code, respectively. 
%Thus, the specifications are of modest size, 
  %even compared to the relatively compact HLS implementations.
%and much smaller than even the C-language accelerator implementations.

\subsection{Term Rewriting with Equality Saturation}
\label{sec.rewrite}

Term rewriting is a well-known technique 
  for program transformations,
  with some compiler optimizations being implemented
  as term-rewriting systems~\cite{
    dershowitz1993taste,
    baader1999term,
    blindell2016instruction,
    regis-pact22}.
%to transform program fragments to equivalent program fragments. 
Given a set of syntactic rewrite rules ($\ell \rightarrow r$) that also preserve semantic equality, 
  a term-rewriting system 
  rewrites instances of pattern $\ell$ 
  in the input program with semantically equivalent pattern $r$ where applicable.
%If all the rules preserve semantic equality,
%  then the application of multiple rules
%  also preserves equality,
%  allowing for modular correctness checking 
%  by verifying the individual rewrite rules.
%It has a variety of applications in compilers including replacing a program with an equivalent program that is optimal for some cost function. 
  
In traditional term rewriting,
  % the ordering of rewrites is important,since 
  applying one rewrite rule
  may prevent
  using other, potentially profitable, rewrite rules;
  this is referred to as the phase-ordering problem~\cite{whitfield1997approach}.
  %and it may impact the quality of the final results
Equality saturation avoids
  phase-ordering issues % when applying rewrites
  by searching over many equivalent rewritings of the same program~\cite{tate2011equality,joshi2002denali}.
  %including different choices of accelerator operations, which ultimately 
  %reduces the need for manual program restructuring.
%  and improves application portability.
Given an input program $p$, 
  equality saturation repeatedly applies 
  the given rewrite rules 
  to explore all equivalent ways to express $p$
%  (with respect to the rewrite rules and resource limits)
%This is accomplished 
using an \textit{e-graph} data structure
  to efficiently represent an exponentially large set of equivalent program expressions~\cite{nelson1980fast,nieuwenhuis2005proof}.
Upon reaching a fixed point, 
  i.e., when no application of any rewrite rule can introduce a new program expression,
  or upon hitting a predetermined resource limit,
  the optimal rewritten program
  can be extracted from an e-graph
  according to a given cost function.
  % thus providing for searching over many candidate rewritings 
  % without sophisticated ordering considerations. 

In {\TLA}, we extend equality saturation to support accelerators. 
Specifically, we
%it enables exploration of \emph{all} possible equivalent expressions. When 
 create custom rewrite rules for accelerators (\S\ref{sec:fragmentmapping}), 
%it enables us to automatically transform applications to identify semantically-equivalent opportunities for accelerator offloads. Second, we can 
and specialize the cost function to maximize the number of accelerator offloads (\S\ref{sec.method.flexible}) or 
consider cost of data movement (\S\ref{sec:optimization}).
% equality saturation enables us to automatically transform applications to identify acceleration opportunities.
  %find matches in application code that are equivalent to, but not necessarily exact matches of accelerator operations.
  Our prototype uses
  the \egg library~\cite{willsey2021egg} for its
  efficient implementation of equality saturation.