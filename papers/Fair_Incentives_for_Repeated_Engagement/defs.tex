
\usepackage{bm,bbm}
\usepackage{amsmath,amsthm,amssymb,amsfonts,cancel}
\usepackage{thmtools,enumitem}
\usepackage{algorithm}
\usepackage{algpseudocode}
% \usepackage{algorithmicx}[ruled]{algorithm2e}
% \usepackage[noend]{algpseudocode}

\usepackage{mathtools,dsfont}

% \renewcommand{\algorithmcfname}{ALGORITHM}
% \SetAlFnt{\small}
% \SetAlCapFnt{\small}
% \SetAlCapNameFnt{\small}
% \SetAlCapHSkip{0pt}
% \IncMargin{-\parindent}
\newtheorem{theorem}{Theorem}

\newcommand{\problem}{$\mathrm{SRMNS}$}

% \newcommand{\problem}{$\mathrm{SRMNS}$}
\newtheorem{lemma}{Lemma}
\newtheorem{assumption}{Assumption}
\newtheorem{claim}{Claim}
\newtheorem{definition}{Definition}
\newtheorem{proposition}{Proposition}
\newtheorem{corollary}{Corollary}

% \newtheorem{theorem}{Theorem}
% \newtheorem{definition}{Definition}
% \newtheorem{proposition}{Proposition}
\usepackage{mathtools}
\newcommand{\detPi}{\widetilde{\Pi}}
 \newcommand{\N}[2]{N^{(#2)}_#1}
\newcommand{\bx}{\mathbf{x}}
\newcommand{\by}{\mathbf{y}}
\newcommand{\bz}{\mathbf{z}}
\newcommand{\rhplus}{r_{h+}}
\newcommand{\rhminus}{r_{h-}}
\newcommand{\rh}{r_h}
\newcommand{\rl}{r_l}
\newcommand{\rhigh}{r_1}
\newcommand{\rlow}{r_2}
\newcommand{\runique}{r_u}
\newcommand{\rhat}{\widehat{r}}
\newcommand{\lhat}{\widehat{\ell}}
\newcommand{\randomstatevector}{{\mathbf{N}}}
\newcommand{\scaledrandomstatevector}{\mathbf{N}^\theta}
\newcommand{\randomstate}{{N}}
\newcommand{\scaledrandomstate}{N^\theta}
\newcommand{\statevector}{{\textbf{n}}}
%\newcommand{\state}{{n}}
\newcommand{\rewardset}{\Xi}
\newcommand{\randomendstate}{\widetilde{N}}
\newcommand{\rev}{R}
\newcommand{\workerid}{w}
\newcommand{\xvec}{\mathbf{x}}
\newcommand{\simplex}{\boldsymbol{\Delta}}
\newcommand{\EE}{\mathbb{E}}
\newcommand{\fluidopt}{\textsc{Fluid-Opt}}
\newcommand{\supplyopt}{\textsc{Supply-Opt}}
\newcommand{\scaledlambda}{\lambda^\theta}
\newcommand{\scaledv}{v_{\theta}}
\newcommand{\PP}{\mathbb{P}}
\newcommand{\EEn}{\EE\left[\randomstate^\star\right]}
\newcommand{\Lambdaub}{\overline{\Lambda}}
\newcommand{\Lambdalb}{\underline{\Lambda}}
\newcommand{\supplyoptsol}{\bx^S}
\newcommand{\fluidn}{\widetilde{N}}
\newcommand{\avgreward}{\widehat{r}}
\newcommand{\supp}{\text{supp}}
\newcommand{\nf}{\normalfont}
\newcommand{\rmax}{r_{\max}}
\newcommand{\rmin}{r_{\min}}
\newcommand{\avgloss}{\hat{\ell}}
\newcommand{\algxvec}{\widetilde{\bx}^S}
\newcommand{\algx}{\widetilde{x}^S}
\newcommand{\scaledrevenue}{\widehat{R}^\theta}
\newcommand{\scaledprofit}{\widehat{\Pi}^\theta}
\newcommand{\fluidprofit}{\widetilde{\Pi}^\theta}
\newcommand{\optfluidprofit}{\widetilde{\Pi}^*}
\newcommand{\badset}{\mathcal{N}^{\mathcal{S}}}
\newcommand{\revmax}{\bar{\rev}^\mathcal{S}}
\newcommand{\unitvec}{\mathbf{e}}
\newcommand{\ttau}{\widetilde{\tau}}
% \newcommand{\Halmos}{\qed}
%\renewcommand{\vec{x}}{\ensuremath{\underline{x}}}


%\renewcommand{\vec{x}}{\ensuremath{\underline{x}}}


\mathchardef\mhyphen="2D % Define a "math hyphen"
%\DeclareMathOperator*{\argmin}{\arg\!\min}
%\DeclareMathOperator*{\argmax}{\arg\!\max}
%\DeclareMathOperator*{\arginf}{arg\,inf} 
%\DeclareMathOperator*{\argsup}{arg\,sup} 
% Places subscript directly under operator
\DeclarePairedDelimiter{\norm}{\lVert}{\rVert}
\DeclarePairedDelimiter{\abs}{\lvert}{\rvert}
\DeclarePairedDelimiter{\ceil}{\lceil}{\rceil}
\newcommand{\frall}{\ensuremath{\,\forall\,}}
%\newcommand{\frall}{\ensuremath{\text{ for all }}}



% Math delimiters
\DeclarePairedDelimiter{\brk}{[}{]}
\DeclarePairedDelimiter{\crl}{\{}{\}}
\DeclarePairedDelimiter{\prn}{(}{)}
\DeclarePairedDelimiter{\nrm}{\|}{\|}
\DeclarePairedDelimiter{\tri}{\langle}{\rangle}
\DeclarePairedDelimiter{\dtri}{\llangle}{\rrangle}
\DeclarePairedDelimiter{\floor}{\lfloor}{\rfloor}

\DeclareMathOperator*{\argmax}{arg\,max}
\DeclareMathOperator*{\argmin}{arg\,min}

\newtheorem*{definition*}{Definition}
\newtheorem*{lemma*}{Lemma}
\newtheorem*{theorem*}{Theorem}
\newtheorem*{claim*}{Claim}
