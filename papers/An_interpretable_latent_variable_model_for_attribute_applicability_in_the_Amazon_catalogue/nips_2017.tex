\documentclass{article}

% if you need to pass options to natbib, use, e.g.:
% \PassOptionsToPackage{numbers, compress}{natbib}
% before loading nips_2017
%
% to avoid loading the natbib package, add option nonatbib:
%c\usepackage[nonatbib]{nips_2017}

\usepackage[nonatbib,final]{nips_2017}
\usepackage[numbers]{natbib}

% to compile a camera-ready version, add the [final] option, e.g.:

\usepackage[utf8]{inputenc} % allow utf-8 input
\usepackage[T1]{fontenc}    % use 8-bit T1 fonts
\usepackage{hyperref}       % hyperlinks
\usepackage{url}            % simple URL typesetting
\usepackage{booktabs}       % professional-quality tables
\usepackage{amsfonts}       % blackboard math symbols
% \usepackage{nicefrac}       % compact symbols for 1/2, etc.
\usepackage{microtype}      % microtypography
\usepackage{graphicx}

\title{An interpretable latent variable model for attribute applicability in the Amazon catalogue}

% The \author macro works with any number of authors. There are two
% commands used to separate the names and addresses of multiple
% authors: \And and \AND.
%
% Using \And between authors leaves it to LaTeX to determine where to
% break the lines. Using \AND forces a line break at that point. So,
% if LaTeX puts 3 of 4 authors names on the first line, and the last
% on the second line, try using \AND instead of \And before the third
% author name.

\author{
  Tammo Rukat\thanks{Work was done at Amazon Berlin}\\
  Department of Statistics\\ University of Oxford\\
  \texttt{tammo.rukat@stats.ox.ac.uk}
  \And
  Dustin Lange\\
  Amazon\\ Berlin, Germany\\
  \texttt{cedrica@amazon.com}  
  \And
  C\'edric Archambeau\\
  Amazon\\ Berlin, Germany\\
  \texttt{langed@amazon.com}  
  % David S.~Hippocampus\thanks{Use footnote for providing further
  %   information about author (webpage, alternative
  %   address)---\emph{not} for acknowledging funding agencies.} \\
  % Department of Computer Science\\
  % Cranberry-Lemon University\\
  % Pittsburgh, PA 15213 \\
  % \texttt{hippo@cs.cranberry-lemon.edu} \\
  %% examples of more authors
  %% \And
  %% Coauthor \\
  %% Affiliation \\
  %% Address \\
  %% \texttt{email} \\
  %% \AND
  %% Coauthor \\
  %% Affiliation \\
  %% Address \\
  %% \texttt{email} \\
  %% \And
  %% Coauthor \\
  %% Affiliation \\
  %% Address 
  %% \texttt{email} \\
  %% \And
  %% Coauthor \\
  %% Affiliation \\
  %% Address \\
  %% \texttt{email} \\
}

\begin{document}
% \nipsfinalcopy is no longer used

\maketitle

\begin{abstract}
  Learning attribute applicability of products in the Amazon catalog (e.g., predicting that a \texttt{shoe}
  should have a value for \texttt{size}, but not for
  \texttt{battery-type}) at scale is a challenge. The need for an interpretable model is contingent on (1) the lack of ground truth training data, (2) the need to utilise prior information about the underlying latent space and (3) the ability to understand the quality of predictions on new, unseen data.
  To this end, we develop the MaxMachine, a probabilistic latent
  variable model that learns distributed binary representations,
  associated to sets of features that are likely to co-occur in the
  data. Layers of MaxMachines can be stacked such that higher layers
  encode more abstract information. Any set of variables can be
  clamped to encode prior information. We develop fast sampling based posterior inference.
  Preliminary results show that the model improves over the baseline in 17 out of 19 product groups and provides qualitatively reasonable predictions.
\end{abstract}


\section{Attribute Applicability}
\label{sec:org75c4eff}
Many real-world datasets can be viewed as object-by-attribute matrices. A prominent example is the Amazon catalogue which contains over 100 million products (objects) and hundreds of attributes, of which only a small subset is assigned to each product. Thus, product-attribute-assignment can be viewed as a sparse binary matrix, shown for a small subsample of the German Amazon marketplace in Fig.\space{}\ref{fig:data}.
\begin{figure}[h]
  \centering
\includegraphics[width=.65\textwidth]{./data_random4.png}
\caption{
  Attribute applicability as binary matrix for a random subset of the German marketplace. Attributes are filtered for being applicable to at least 1\% of the sampled products. Black dots denote ones, indicating applied attributes. White dots denote zeros, indicating the absence of any attribute value.
  \label{fig:data}}
\end{figure}
Being able to distinguish between attributes that are truly
non-applicable (e.g., \texttt{battery-type} for a shoe), attributes that 
could reasonably be applied (e.g., \texttt{weight} for a book), and
attributes that are clearly applicable (e.g., \texttt{size} for a T-shirt) is
crucial for applications such as attribute imputation models, data
quality management, template generation, product comparison and
virtually all customer-facing downstream applications.

We can cast the task of predicting attribute applicability as a
multi-label classification problem, 
where each attribute constitutes a label and an arbitrary number 
of labels is assigned to each product. While there is recent progress
in such extreme multi-label classification problems
\cite{Bhatia2015,Jain2017}, we face a particular challenge:
The absence of reliable training labels makes it difficult to
define a training metric. Therefore, we approach attribute
applicability as an unsupervised problem and develop a probabilistic
latent variable model
that describes the generative process by which the binary
product/applicability matrix is generated from a set of latent
features. We aim to retain a simple,
interpretable model, resembling the process of a marketplace
seller who is filling in attributes for their product.
The rationale behind the model is that each latent
feature corresponds to a set of attributes that are likely to appear
together such as (\texttt{title}, \texttt{pages}, \texttt{language}, \texttt{release date}) or
(\texttt{width}, \texttt{height}, \texttt{length}).
Each of these sets is represented by a latent dimension and the
generative process for any product-attribute-pair is a noisy
disjunction of these feature sets. 
The model design is further motivated by keeping sampling-based
posterior inference scalable.
We also assume that additional product or attribute specific prior
information is available. Here, this is exemplified by the
\texttt{product type}, a string attribute that is assigned to a majority of products. Product types are curated such that there exist only
around 500 distinct values. They provide useful information that we need
to leverage for optimal performance. Moreover, attribute assignments
have a strong product type specific pattern which lets us use the
product type specific attribute frequency as a baseline probability estimate.

\section{MaxMachine Model}
\label{sec:orga18e9af}

A first candidate model is Boolean Matrix 
Factorisation \cite{Rukat2017},
where, both, the features and their allocations are
binary. The data generating process is a noisy matrix product between these
binary matrices whose result is thresholded at 1. With \(N\) products and \(D\)
attributes \(x_{nd} \in \{0,1\}\) denotes the application status for
a product-attribute-pair. We have $L$ latent dimensions and the factor matrices
are denoted as \(U \in \{0,1\}^{L\times D}\) and \(Z \in
  \{0,1\}^{N\times L}\), such that \(u_{l,d=1,\ldots, D}\) encodes a set
of co-occuring attributes and \(z_{n, l=1,\ldots,L}\) denotes the
compressed representation of observation \(x_n\). The likelihood for
Boolean Matrix Factorisation then takes the form $ p(x_{nd}|.) = \sigma[\lambda \tilde{x}_{nd} (1-2 \prod_{l}(1-z_{nl}u_{ld}) ) ]\;, $
where $\sigma$ is the logistics sigmoid and $\tilde{x} = 2x{-}1$. Note that the product over \(l\) is the Boolean disjunction. This model has a global noise
parameter, $\lambda \in \mathbb{R}^+$, that governs the random
flipping of bits. However, due to the heterogeneity in the data, we require a more expressive model
that can capture heteroscedastic noise. To this end, we propose the
\emph{MaxMachine}. Here, the noise for each latent dimension is
governed by a separate parameter and each data point is generated from
the least noisy associated latent dimension. Hence the model retains
composability and is easily interpretable.
The likelihood takes the form
$  p(x_{nd}|.) = \sigma\left[\tilde{x}_{nd} \max_l (\lambda_l z_{nl} u_{ld})\right],$
where $\lambda_l \in \mathbb{R}$.
Using the max-operation, the latent dimensions compete for
explaining the observations. The \emph{winner} is 
the most accurate predictor and gets to fully explain the
observation. In order for the model to be well defined we have an
additional, clamped latent dimension with \(u_{l,d}{=}z_{nl}{=}1 \;\forall\;(n,d)\).
We propose a beta prior on each $\sigma(\lambda_l)$ and binomial priors on the cardinality of the rows of $U$. The latter can encode our prior belief that the number of co-occuring attributes in each set is much smaller than the total number of attributes.\\
Now, we include the product-type information by adding another layer of
matrix factorisation in the spirit of a \textit{Bayesian hierarchical model}.
This means that the prior on the matrix of latent representation is
factorised according to another MaxMachine moandel. Here, the
higher-order object specific factor matrix is fixed to an encoding of the product type
in a one-hot fashion.


\subsection{Inference}
\label{sec:orge79dcef}
The inference task amounts to estimating, both, the attribute sets
and their assignments and is combinatorially challenging.
We develop sampling based posterior inference, an approach that
has been shown to outperform competing methods in Boolean Matrix
Factorisation\space{}\cite{Rukat2017}. Computation of the full
conditional probabilities of each variable \(u_{ld}\) or \(z_{nl}\)
generally depends on the variables full Markov blanket. We make use
of several
algorithmic tricks and leverage the purely binary states of all
variables as well as the lack of interaction between dimensions that is
induced by the max operation. This enables efficient updates, such that
the algorithm converges for hundreds of thousands of data points
within few minutes on a laptop. After every sweep through all entries of $U$ and $Z$, we set
all $\lambda_l$ to their MAP estimate which is analytically available.
Following posterior inference, we can compute Monte Carlo estimates
of the posterior predictive and thus predict applicability.

%However, as shown in Fig.\space{}\ref{fig:orgb9da665}

% We find empirically that the predictive performance does not benefit from more than $O(1e3)$ training samples, which can be explained by their redundancy.

% \begin{figure}[htbp]
% \centering
% \includegraphics[width=10cm]{./figs/subsampling.png}
% \caption{\label{fig:orgb9da665}
% Predictive performs under variation of the size of the training data.}
% \end{figure}



\section{Experiments and Results}
We consider a snapshot of the Amazon German marketplace that have been found to be particularly important to customers.
In order to simplify analysis and evaluation we stratify the data into
19 different clusters corresponding to related types of products, such
as \texttt{clothing}, \texttt{computers} or \texttt{jewellery}. Each
cluster consists of products from a variety of product types (1-50). In particular products from different clusters share only very few attributes and therefore share no co-occurrence patterns of mutual relevance. We train on only 500 randomly sampled products from each product type, since a further increase in training data has no effect on the quality of the results. This is due to the redundancy in the binary data.\\
We measure test-set performance by treating randomly selected
product-attribute pairs as unobserved during training and evaluate the area under the ROC curve for the posterior predictive on these test data points.
For the following, experiments we choose \texttt{binomial(0.1)}
priors on the cardinality of each latent set of 
attributes, reflecting our prior belief that the number of
attributes in each co-occuring set is relatively small. For the 
noise parameters (mapped to $[0,1]$), we use a \texttt{beta(10,1)} prior, encoding our believe 
that attribute sets that are applied to a product are relatively
likely to be actually present in the data.
Based on random search, we choose
the remaining hyperparameters.
% as follows.
% The prior on latent representation for each asins is iid Bernoulli
% 0.5. and the initialisation is set means cluster
% centroids along the corresponding dimension of the data. 
% The latent features are initialised uniformly iid random with
% \(p(u=1)=10\%\). 
% Empirically, we find that the variation for using different priors is on the same scale as for using different random initialisations.


\begin{figure}[htbp]
\centering
\includegraphics[width=12cm]{./experiment2_codes.png}
\caption{\label{fig:orgca0a434}
Patterns of attribute co-occurrence. Shown are posterior means of the
inferred codes \(U\) (black: 1, white: 0). Each row denotes a set of
frequently co-occurring attributes, each column denote an attribute; $\nu$ denotes the expected
percentage of ones in the data that are explained by the corresponding
dimension, $\hat\lambda$ is the average posterior MAP of the noise parameter.}
\end{figure}

We show the inferred sets of co-occurring attributes for clothing
products in Fig.\space{}\ref{fig:orgca0a434}. They indicate reasonable co-occurrences such as (\texttt{waist-style}, \texttt{waist-size}, \texttt{inseam-length},~\ldots).
The corresponding posterior predictive achieves a ROC-AUC of 94\% on held out data, while the product type mean reaches close to 93\%. The
moderate improvements in ROC-AUC can partly be attributed to the
fact that 
the product-type mean predictor has a higher certainty for
attributes that are present in almost all products.
Qualitative evaluation shows that the model makes reasonable
predictions, as for instance to add the attribute \texttt{material} to most
products of type \texttt{bra} or the attribute \texttt{manufacturer} to products of type
\texttt{shirt}, more anecdotal evidence is provided in Table
% \ref{tab:org80ba429} and
\ref{tab:orgdc754d7}.
We find a
non-zero, but rather low probability of adding the attribute
\texttt{cup-size} to products of type \texttt{bra}. This can be understood by
noting that \texttt{cup-size} occurs for no other product type and,
therefore, cannot be inferred from correlation. While the probability
is low, it is notable evidence if considered in relative terms:
compared to any other product types, products of the type \emph{bra} have 
by far the highest probability of the attribute \emph{cup-size} being
applicable. This suggests the exploration of attribute-specific
thresholds for practical applications.\\
We repeat this experiment across all product types in the catalogues and find that
the  MaxMachine outperforms the baseline
by margins between 1\% and 15\% in 17 out of 19 clusters. 
The slightly weaker performance in the remaining cluster can largely be explained by extremely homogeneous attribute distributions for each of the contained product types.

% \begin{table}[htbp]
%   \small
% \caption{\label{tab:org80ba429}\small
% Anecdotal evidence. We choose three typical asins from different product types and list all atributes that are added with a probability \(>25\%\) or removed with a probability \(>99\%\).}
% \centering
% \begin{tabular}{l|ll|ll|ll}
%  & type: shorts &  & type: bra &  & type: software & \\
%  & attribute & p(apply) & attribute & p(apply) & attribute & p(apply)\\
% \hline
% \textbf{add} & department & \(82\%\) & collection & \(36\%\) & edition & \(65\%\)\\
%  & care-indstruction & \(44\%\) & style & \(31\%\) & bullet point & \(58\%\)\\
%  & age-gender category & \(42\%\) & lifesstyle & \(27\%\) & width, height & \(48\%\)\\
%  & waist-style & \(32\%\) & pattern-type & \(25\%\) & length & \\
%  & inseam-length & \(32\%\) & band-size & \(25\%\) & genre & \(33\%\)\\
% \hline
% \textbf{remove} & item-weight & \(<1\%\) & item weight & \(<1\%\) & binding & \(<1\%\)\\
%  & special-size-type & \(<1\%\) & type keyword & \(<1\%\) & model year & \(<1\%\)\\
%  & sport-type & \(<1\%\) &  &  &  & \\
% \end{tabular}
% \end{table}


\begin{table}[htbp]
  \small
\caption{\label{tab:orgdc754d7}\small
Anecdotal evidence -- for three attributes we list the product types that they are most likely applied to by the model. The percentage is the mean probability of being applied for all products in the product type. In brackets we give the mean probability only for those products that do not have corresponding attribute assigned.}
\centering
\begin{tabular}{l|lll|}
Attribute & cup-size & closure-type & leather-type\\
\hline
Product types & Bra \(22 (10) \%\) & Shoes \(48 (18) \%\) & Shoes \(48 (15)\%\)\\
with largest & Swimwear \(3 (2) \%\) & Pants \(24 (10) \%\) & Outerwear \(3 (3)\%\)\\
p(apply) & Underwear \(3 (2) \%\) & Shorts \(6 (3) \%\) & Shorts \(2(2)\%\)\\
 & Shoes \(2 (2) \%\) & Outerwear \(4 (2) \%\) & (< \(1\%\))\\
 & Suit \(2 (2) \%\) & Bra \(2 (2)\%\) & \\
\end{tabular}
\end{table}



% \begin{figure}[htbp]
% \centering
% \includegraphics[width=14cm]{./figs/experiment_cmrps.png}
% \caption{\label{fig:orgb0d10f7}
% Performance comparison across attribute clusters. The hierarchical MaxMachine achieves the best predictive performs in all clusters, except for \emph{media} and \emph{movie downloads}. The shallow MaxMachine does not make use of any product-type information and still outperforms the baseline in 11/19 cases.}
% \end{figure}


% \begin{figure}[htbp]
% \centering
% \includegraphics[width=17cm]{./figs/rocs.png}
% \caption{\label{fig:org2a670c3}
% Receiver-operator characteristics for three product type clusters. The hierachical MaxMachine model achieves the best area under the curve for all examples. For \emph{clothing} products (right) and high thresholds (top region of the subplot), the mean predictor achives higher true positive rate. However, the regions of the curve that are relevant for our application are low false positive rates.}
% \end{figure}


\section{Conclusion}
\label{sec:org5d08bba}
We have described the problem of attribute applicability in the
Amazon catalogue and developed a latent variable model for the
denoising of product/applicability matrices.
Due to the lack of ground truth data we have optimised
reconstruction accuracy conditional on a model that describes an
intuitive generative process, resembling the real-world
procedure of a seller, assigning attributes to products.
As a baseline we have used product-type specific patterns of
applicability and improved in the area under the ROC curve on
hold-out data for 17 out of 19 product clusters.
Anecdotal evidence confirms that our model makes reasonable
predictions.\\
In a practical scenario, more prior expert knowledge
might be available and of high importance. Many types of such
information can be flexibly integrated in the proposed
model. For instance the presence of certain important attributes could be clamped
for certain types of products, while it is inferred for others.
For future work, it would be desirable to avoid the subjective
choice of hyperparameters. In particular the use of non-parametric
priors could lead to a more principled choice of the latent
dimensionality and help model convergence.

\section{Acknowledgements}
We thank Felix Biessmann and David Salinas for helpful discussions, their insights and their expertise that greatly assisted this research. 

\small
% %\documentclass{article}

\pdfoutput = 1

% if you need to pass options to natbib, use, e.g.:
% \PassOptionsToPackage{numbers, compress}{natbib}
% before loading nips_2017
%
% to avoid loading the natbib package, add option nonatbib:
% \usepackage[nonatbib]{nips_2017}

\usepackage[final]{nips_2017}

% to compile a camera-ready version, add the [final] option, e.g.:
% \usepackage[final]{nips_2017}

\usepackage[utf8]{inputenc} % allow utf-8 input
\usepackage[T1]{fontenc}    % use 8-bit T1 fonts
\usepackage{hyperref}       % hyperlinks
\usepackage{url}            % simple URL typesetting
\usepackage{booktabs}       % professional-quality tables
\usepackage{amsfonts}       % blackboard math symbols
\usepackage{nicefrac}       % compact symbols for 1/2, etc.
\usepackage{microtype}      % microtypography
\usepackage{amsmath}
\usepackage{svg}
\usepackage{graphicx}
\usepackage{bbm}
\usepackage{array}
\usepackage{subfig}

\title{Label Efficient Learning of Transferable Representations across Domains and Tasks}

% \newcommand{\jh}[1]{\textcolor{green}{JH: #1}}
% \newcommand{\yl}[1]{\textcolor{red}{YL: #1}}
% \newcommand{\alan}[1]{\textcolor{blue}{Alan: #1}}

% The \author macro works with any number of authors. There are two
% commands used to separate the names and addresses of multiple
% authors: \And and \AND.
%
% Using \And between authors leaves it to LaTeX to determine where to
% break the lines. Using \AND forces a line break at that point. So,
% if LaTeX puts 3 of 4 authors names on the first line, and the last
% on the second line, try using \AND instead of \And before the third
% author name.

\author{
  Zelun Luo \\
  Stanford University \\
  \texttt{zelunluo@stanford.edu} \\
  \And
  Yuliang Zou \\
  Virginia Tech \\
  \texttt{ylzou@vt.edu} \\
  \AND
  \hspace{0.8cm} Judy Hoffman \\
  \hspace{0.5cm} University of California, Berkeley \\
  \hspace{0.5cm} \texttt{jhoffman@eecs.berkeley.edu} \\
  \And
  \hspace{-0.4cm} Li Fei-Fei \\
  \hspace{-0.4cm} Stanford University \\
  \hspace{-0.4cm} \texttt{feifeili@cs.stanford.edu} \\
}

\begin{document}
% \nipsfinalcopy is no longer used

\maketitle

\begin{abstract}
We propose a framework that learns a representation transferable across different domains and tasks in a label efficient manner. Our approach battles domain shift with a domain adversarial loss, and generalizes the embedding to novel task using a metric learning-based approach. Our model is simultaneously optimized on labeled source data and unlabeled or sparsely labeled data in the target domain. Our method shows compelling results on novel classes within a new domain even when only a few labeled examples per class are available, outperforming the prevalent fine-tuning approach. In addition, we demonstrate the effectiveness of our framework on the transfer learning task from image object recognition to video action recognition. 
\end{abstract}

% Introduction
\section{Introduction}
Reinforcement learning has achieved great success in areas such as Game-playing \citep{silver2018general,vinyals2019grandmaster}, robotics \cite{kober2013reinforcement}, large language models \citep{ouyang2022training}, etc.
However, due to safety concerns or physical limitations, in some real-world reinforcement learning problems, we must consider additional constraints that may influence the optimal policy and the learning process \citep{garcia2015comprehensive}.
% For example, a robotic arm must not take actions that may cause harm to itself or the environments.
A standard framework to handle such cases is the constrained Markov Decision Process (CMDP) \citep{altman1999constrained}.
Within the CMDP framework, the agent has to maximize
the expected cumulative reward while
obeying a finite number of constraints, which are usually in the form of expected cumulative cost criteria.

However, we are sometimes concerned with the problem with a continuum of constraints.
For example,
the constraints we meet might be time-evolving or subject to uncertain parameters, which
cannot be formulated as an ordinary CMDP
(see Examples \ref{Example_Time_Evolving} and  \ref{Example_Uncertain}).
In this paper we would study a generalized CMDP  
to address the above problem.  Because the constraints are not only infinite-number but also lie
in a continuous set,
the generalization is not trivial. Fortunately, we find that we can borrow the idea behind semi-infinite programming (SIP) \citep{remez1934determination, hettich1993semi} to deal with the semi-infinite constraints.
Accordingly, we propose \emph{semi-infinitely constrained Markov decision processes} (SICMDPs)
as a novel complement to the ordinary CMDP framework.
%More specifically,  an SICMDP model %, we consider 
%contains a continuum of constraints whereas an ordinary CMDP contains a finite number of constraints. 

%This generalization is natural but not trivial. However, we can brows the idea  
%The idea is quite natural and can be backtracked
%to the practice of extending linear programming to linear semi-infinite programming (LSIP) %\cite{remez1934determination, GobernaLSIO1998}.
%In addition, 
%As a complementary approach to the ordinary CMDP framework, 
%SICMDP can be used to model these problems  which cannot be described by a finite number of constraints
%that are not covered by .
%For example,
%the restrictions we consider can be time-evolving or subject to uncertain parameters
%, thus
%cannot be described by a finite number of constraints but a continuum of constraints 
%(see Examples \ref{Example_Time_Evolving} and  \ref{Example_Uncertain}).

We also present two reinforcement learning algorithms to solve SICMDPs called SI-CRL and SI-CPO, respectively.
SI-CRL is a model-based reinforcement learning algorithm designed for tabular cases, and SI-CPO is a policy optimization algorithm for non-tabular cases.
% and analyze its performance both theoretically and empirically.
The main challenge is that we need to deal with a continuum of constraints, thus reinforcement learning algorithms for ordinary CMDPs do not work anymore.
In SI-CRL, we tackle this difficulty by first transforming the reinforcement learning problem to an equivalent LSIP problem, which can then be solved using methods in the LSIP literature like the dual exchange methods \citep{Hu1990,reemtsen1998numerical}.
In SI-CPO, we resort to the idea of cooperative stochastic approximation developed in \cite{lan2020algorithms, wei2020comirror}.
As far as we know, we are the first to introduce tools from semi-infinitely programming (SIP) into the reinforcement learning community for solving constrained reinforcement learning problems.

% To the best of our knowledge, we are the first to apply tools from semi-infinitely programming (SIP) to solve reinforcement learning problems.
Furthermore, we give theoretical analysis for both SI-CRL and SI-CPO.
We decompose the error of SI-CRL into two parts: the statistical error from approximating the true SICMDP with an offline dataset and the optimization error due to the fact that the solution of the LSIP problem obtained by the dual exchange method is inexact.
On the optimization side, we show that the iteration complexity of SI-CRL is $O\left(\left\{\mathrm{diam}(Y)L\sqrt{|\gS|^2|\gA|m}/\left[(1-\gamma)\epsilon\right]\right\}^m\right)$.
On the statistical side, we show that the sample complexity of SI-CRL is $\widetilde O\left(\frac{|S|^2|A|^2}{\epsilon^2(1-\gamma)^3}\right)$ if the offline dataset is generated by a generative model, and $\widetilde O\left(\frac{|S||A|}{\nu_{\min} \epsilon^2(1-\gamma)^3}\right)$ if the dataset is generated by a probability measure $\nu$ as considered in \cite{chen2019information}.
Here $\widetilde O$ means that all logarithm terms are discarded.
For SI-CPO, things become a little more complicated because other than the statistical error and the optimization error, we also need to consider the function approximation error, which comes from imperfect policy parametrizations.
It is shown if the function approximation error can be controlled to $O(\epsilon)$ order, the iteration complexity of SI-CPO is $\widetilde{O}\left(\frac{1}{\epsilon^2(1-\gamma)^6}\right)$ and the sample complexity of SI-CPO is $\widetilde{O}(\frac{1}{\epsilon^4(1-\gamma)^{10}})$.
Here our iteration complexity bound is equivalent to a typical $\widetilde O(1/\sqrt{T})$ global convergence rate.

We perform a set of numerical experiments to illustrate the SICMDP model and validate our proposed algorithms.
Specifically, we examine two numerical examples, namely the discharge of sewage and ship route planning.
Through the discharge of sewage example, we show the advantage of the SICMDP framework over the CMDP baseline obtained by naive discretization in modeling realistic sequential decision-making problems.
Moreover, we demonstrate the effectiveness of the SI-CRL and SI-CPO algorithms in such tabular environments. 
In the ship route planning example, we illustrate the benefits of the SICMDP framework and the ability of the SI-CPO algorithm to address complex continuous control tasks involving continuous state spaces with modern deep reinforcement learning techniques.

% In summary, our contributions are listed as follows.
% First, we present the SICMDP model, which can be viewed as a generalization of the ordinary CMDP model.
% Second, we propose an algorithm to perform reinforcement learning for SICMDPs, which is called SI-CRL, and we believe that we are the first to apply tools from SIP
% to solve reinforcement learning problems.
% Third, we give a theoretical analysis of SI-CRL and identify both its sample complexity and iteration complexity.
% In addition, we perform numerical experiments to illustrate the SICMDP model and validate the SI-CRL algorithm.
% \{This paragraph can be removed!!! \}







% Related Work
\section{Related work}
\textbf{Related work}:
% Object detection related datasets/algo in non-medical domain
% Locally labeled CXR dataset
A few CXR datasets have localized abnormality annotations \cite{shih2019augmenting,filice2020crowdsourcing,jaeger2014two} that are curated manually. These are high quality gold standard ground truth datasets but tend to be smaller in scale (< 30,000 images) and have a narrow coverage, with typically only 1-2 labels. In addition, since most labeling efforts only have abnormality semantics attached, no direct relationships with the affected anatomical locations are available. 

%MEHDI: repeated concepts from above. I am removing the following: 

%The lack of anatomic semantics in the annotation is a limitation for complex multi-modal clinical reasoning work, e.g., differential diagnosis, since clinicians often integrate information along anatomical lines, and for downstream report generation tasks, which often requires describing not only the abnormality but also correctly communicate the location of the abnormalities (and medical devices) to the receiving clinicians. 

Two recent CXR datasets have labels for anatomies described in the reports. In \cite{datta2020dataset}, a small manually annotated dataset (2000 reports) included 10 abnormalities that are individually associated with 29 unique spatial locations (anatomies) at the report level. Another CXR dataset has automatically extracted abnormality and anatomy labels as disconnected concepts that are only correlated at the study level from  160,000 reports using a supervised NLP algorithm \cite{bustos2020padchest}. This was trained on a smaller set of manually annotated data. Neither datasets contain localized annotations for the associated CXR images, nor any comparison relation annotations between sequential exams, both of which are available in the Chest ImaGenome dataset. In Table \ref{tab:related}, we present a comparison of our Chest ImagGenome dataset with other datasets available in the literature.

% Table -- Kashyap

% MEdical imaging datasets to go here: Discussed that we will only focus on cxr datasets that are available for this paper. 
% \caption{\color{red} Kashyap, feel free to continue with the table. We should remove the questionmarks and add a line for our dataset (since all others are not graph). For longer text, using abbreviations and explaining them in the caption often works better. If fill in the values is not possible, it is better to remove the table altogether.}


\begin{table}[t!]
\caption{Summary of existing chest X-ray datasets}
\resizebox{\textwidth}{!}{%
\begin{tabular}{@{}lllllllll@{}}
\toprule
\textbf{Dataset} & \textbf{Annotation Level} & \textbf{Annotation Method} & \textbf{Num Labels} & \textbf{Anatomy Labeled} & \textbf{Graph} & \textbf{Dataset Size} & \textbf{Temporal Labels} & \textbf{Reports} \\ \midrule
SIIM-ACR Pneumothorax Segmentation \cite{filice2020crowdsourcing} & Segmentation & Manual + augmented & 1 & No & No & 12,047 & No & No \\
RSNA Pneumonia Detection Challenge   \cite{shih2019augmenting} & Bounding Boxes & Manual & 1 & No & No & 30,000 & No & No \\
Indiana University Chest X-ray collection \cite{demner2016preparing} & Global & Automated & 10 & No & No & 3,813 & No & Yes \\
NIH CXR dataset \cite{wang2017chestx} & Global & Automated & 14 & No & No & 112,120 & No & No \\
PLCO \cite{team2000prostate} & Global & Automated & 24 & Yes & No & 236,000 & Yes & No \\
Stanford CheXpert \cite{irvin2019chexpert} & Global & Automated & 14 & No & No & 224,316 & No & No \\
MIMIC-CXR \cite{johnson2019mimic} & Global & Automated & 14 & No & No & 377,110 & No & Yes \\
Dutta \cite{datta2020dataset} & Global & Manual & 10 & Yes & Yes & 2,000 & No & Yes \\
PadChest \cite{bustos2020padchest} & Global & Manual + automated & 297 & Yes & No & 160,868 & No & Yes \\
Montgomery County Chest X-ray   \cite{jaeger2014two} & Segmentation & Manual & 1 & Yes & No & 138 & No & No \\
Shenzen Hospital Chest X-ray   \cite{jaeger2014two} & Segmentation & Manual & 1 & Yes & No & 662 & No & No \\  \hline \hline
\textbf{Chest ImaGenome} & Bounding Boxes & Automated & 131 & Yes & Yes & 242,072 & Yes & Yes \\
\bottomrule
\end{tabular}%
}
\label{tab:related}
\vspace{-0.4cm}
\end{table}
% removed (Derived from MIMIC-CXR \cite{johnson2019mimic}) % makes table really small


% Model
\section{Method}
The proposed segmentation-by-detection framework, as depicted in Figure \ref{fig:framework}, consists of a detection module and a segmentation module.
In detection stage, 2D slices (layered box) from the input volume are fed to the RPN. Based on the region proposals obtained from RPN, an attention model (block in orange) is formed. The input volume as well as the attention model are further processed in segmentation stage to get the refined anatomical segmentation. 
\vspace{1em} 

\begin{figure}[t]
\centering
\includegraphics[width=0.95\linewidth]{fig/framework.pdf}
\caption{Schematic representation of the segmentation-by-detection framework. The left part is the detection module while the segmentation module is followed on the right. The blue block denotes the input volume which is 3D ultrasound scan of femoral head. The output segmentation is in red.}
\label{fig:framework}
\end{figure}
% dana could you improve the figure. we can try to think together of better ways 

\noindent\textbf{Detection Module:} 
% dana : here you have to make the clarification that you have ground truth on the boxes (in implementation part)
The detection module follows an RPN architecture, a fully convolutional network which takes image slice as input and outputs object region candidates. 
We use the VGG-16 model as the backbone \cite{simonyan2014very} to learn convolutional features and an $3 \times 3$ spatial window to generate region proposals. At each sliding-window location, 9 anchors are predicted associated with different scales and aspect ratios. The last layer consists of a box-regression (reg) layer and a box-classification (cls) layer in parallel. The reg layer outputs 4 regression offsets, $ t = (t_x,t_y,t_w,t_h)$, denoting a scale-invariant translation as well as log-space height and width shift, where $x,y,w$ and $h$ specify two coordinates of the box center, width and height. The cls layer outputs two scores by softmax, related to probabilities of object and background for each proposal. We assign a positive label (of being object) to candidate which has an Intersection-over-Union (IoU) ratio higher than 0.7 with ground truth box. Note that an image slice may contain multiple object regions or none. 

The loss function of RPN follows the multi-task loss \cite{ren2015faster} which is defined as $L = L_{reg} + L_{cls}$. The regression loss, $L_{reg} = -\log p_{obj}$ is log loss and the classification loss,
\begin{equation} \label{eq:loss}
L_{cls} = \sum_{i \in \{x,y,w,h\}} smooth_{L_1} (t_i - t_i^*)
\end{equation}
is smooth $L_1$ loss where $t_i^*$ denotes the ground truth box for the target object. 
\vspace{1em}

\noindent\textbf{Segmentation Module:}
3D U-Net \cite{cciccek20163d} is utilized in the segmentation module as its outstanding performance in medical image segmentation. The u-shaped architecture consists of two paths: a contracting path, where each layer contains two $3\times3\times3$ convolutions followed by a rectified linear unit (ReLU) and then a max pooling, provides high resolution features. While, the symmetric expanding path for semantically richer features replaces max pooling with a upconvolution $2\times2\times2$ with stride of 2 in each dimension, and then two $3\times3\times3$ convolutions each followed by a ReLU. Skip connections between layers of equal resolution in the contracting path and the expanding path enables context information as well as precise localization.

Different from 3D U-Net, to incorporate the attention model detected by the RPN, our architecture takes as input both the volumetric image data and the candidate RoIs proposed by the RPN, concatenated as 3D volume. 
% dana not sure what you like to say below
% densely annotated
The attention model makes the network to focus on the potential RoIs and can reduce the interference of the surrounding noise.
The anatomical segmentation is then generated from a $1\times1\times1$ convolution which reduces the number of feature maps to the number of labels.  The energy function is computed by a pixel-wise softmax combined with the cross entropy loss.
% dana equation ??

\subsection{System and implementation Details}
The segmentation-by-detection approach adopts a cascade structure with two stages: detection and segmentation. The two networks are trained separately in an end-to-end manner. All the new layers are randomly initialized from zero-mean Gaussian distribution with standard deviations 0.01. Biases are initialized to 0. We use Caffe \cite{jia2014caffe} for the implementation and an NVIDIA Titan X GPU for training.

In the detection stage, we initialize the VGG-16 model by the pre-trained model for ImageNet classification \cite{russakovsky2015imagenet} and further fine-tune the model for our detection task. The input fed to the network are image slices with a fixed size of $184\times96$ and the corresponding ground truth boxes are generated from the annotation in the format of tight bounding boxes surrounding the segmentation contour (as illustrated in Figure \ref{fig:hip} (b), the boundary of white area). To optimize the energy function, stochastic gradient descent (SGD) is used. The global learning rate is set to 0.001, while a momentum of 0.9 and a weight decay of 0.0005 are used. The batch size is set to 256 and each mini-batch only contains the positive anchors for training. The region proposals are obtained from the reg path for each image slice. The attention model is then formed by concatenating all the detected regions, as binary masks, into a volume.

In the segmentation stage, we use the Adam optimizer \cite{kingma2014adam} to learn the network parameters. A global learning rate is set to 0.001 while the two momentum coefficients are set to 0.9 and 0.999 respectively. A batch size of 1 is used due to the memory constraints of the GPU. The network takes the volume data as well as the attention model as input. We train the network for a maximum of 30K iterations and reserve the learned weights with the best performance from every 1K iterations. 
\vspace{1em}

\noindent\textbf{Inference:}
At test time, the 2D slices from an input volume are first fed to the detection module. The attention model is obtained based on the output. Then the volume data as well as the attention model are fed to the segmentation module to get the pixel-wise prediction.




% Experiment
\section{Experiment}
\newcommand{\twomoons}{{\tt Twomoons}}
\newcommand{\gauss}{{\tt Gauss}}
\newcommand{\sculpture}{{\tt Sculpture}}
\newcommand{\baseline}{{\tt Baseline}}
\newcommand{\MM}{{\tt MsgPassing}}
\newcommand{\blackboard}{{\tt Blackboard}}
\newcommand{\ncut}{\text{ncut}}
\newcommand{\chensays}[2][]{\textcolor{blue} {\textsc{Jiecao #1:} \emph{#2}}}

\section{Experiments}
In this section we present experimental results for  graph clustering in the message passing and blackboard models. We will compare the following three algorithms. (1) \baseline: each site sends all the data to the coordinator directly; (2) \MM: our algorithm in the message passing model (Section~\ref{sec:gcmessage}); (3) 
\blackboard: our algorithm in  the blackboard model (Section~\ref{sec:bb}).


%Since both of our algorithms are crucially based on the use of spectral scarification, our main focus in the experiments is to investigate to what extend the quality of the spectral clustering algorithms will be affected by using spectral sparsification, the saving of communication costs by using spectral sparsificaion, ...
%
%
%The goal of this experiment is not to demonstrate the effectiveness of the spectral clustering algorithm. We mainly want to investigate the following, 
%\begin{itemize}
%\item to what extend the quality of clustered results will be affected by using spectral sparsification.
%\item saving of communication costs by using spectral sparsifier.
%\item the affect of constants in algorithms of the message passing/blackboard model.
%\end{itemize}
%
%
%\subsection{The Setup}
%\paragraph{Reference Algorithms}
%We compare different algorithms in our experiment.

%Note that we can also run \MM~ in the blackboard model.

Besides giving the visualized results of these algorithms on various datasets, we also measure the qualities of the results via the {\em normalized cut}, defined as 
\[
\ncut(A_1, \ldots, A_{k}) = \frac{1}{2}\sum_{i\in[k]}\frac{w(A_i, V\backslash A_i)}{\vol(A_i)},
\]
 which is a standard objective function to be minimized for spectral clustering algorithms. 
%We will compare the communication costs of these algorithms in different settings.

%We also compare the total communication costs of different algorithms/models. As the unit does not matter in our case, we normalize all communication costs by the cost of \baseline.  Whenever possible, we will visualize the clustered results.

We implemented the algorithms using multiple languages, including Matlab, Python and C++. Our experiments were conducted on an IBM NeXtScale nx360 M4 server, which is equipped with 2 Intel Xeon E5-2652 v2 8-core processors, 32GB RAM and 250GB local storage.


\subsection{Datasets.}
We test the algorithms in the following real and synthetic datasets, which is visualized in \figref{visualization}.


\begin{figure}[h]
     \centering
     \subfigure[\twomoons]{\includegraphics[width=0.23\textwidth]{twomoons-14000-original.png}\label{fig:twomoons}}
     ~~
     \subfigure[\gauss]{\includegraphics[width=0.23\textwidth]{gauss-10000-original.png}\label{fig:gauss}}
     ~~
     \subfigure[\sculpture]{\includegraphics[width=0.13\textwidth,height=0.16\textwidth]{sculpture-11680-original.jpg}\label{fig:sculpture}}
     \caption{Visualization of the datasets for our experiments.}
     \label{fig:visualization}
\end{figure}



\vspace{-1mm}
\begin{itemize}
\item \twomoons : this dataset contains $n=14,000$ coordinates in $\mathbb{R}^2$. We consider each point to be a vertex. For any two vertices $u, v$, we add an edge with weight $w(u,v) = \exp\{-\|u-v\|_2^2/\sigma^2\}$ with $\sigma = 0.1$ when one vertex is among the $7000$-nearest points of the other.  This construction results in a graph with about $110,000,000$ edges.

\item  \gauss : this dataset contains $n = 10,000$ points in $\mathbb{R}^2$. There are $4$ clusters in this dataset, each generated using a Gaussian distribution. We construct a complete graph as the similarity graph.  For any two vertices $u, v$, we define the weight $w(u,v) = \exp\{-\|u-v\|_2^2/\sigma^2\}$ with $\sigma = 1$. The resulting graph has about $100,000,000$ edges.

\item \sculpture : a photo of \textit{The Greek Slave}~\footnote{Available in e.g., \url{http://artgallery.yale.edu/collections/objects/14794}}. We use an $80\times 150$ version of this photo where each pixel is viewed as a vertex. To construct a similarity graph, we map each pixel to a point in $\mathbb{R}^5$, i.e., $(x, y, r, g, b)$, where the latter three coordinates are the RGB values. For any two vertices $u, v$, we  put an edge between $u, v$ with weight $w(u,v) = \exp\{-\|u-v\|_2^2/\sigma^2\}$ with $\sigma = 0.5$ if one of $u, v$ is among the $5000$-nearest points of the other. This results in a graph with about $70,000,000$ edges.
\end{itemize}
\vspace{-1mm}
In the distributed model edges are randomly partitioned across $s$ sites. 

%\vspace{-1.5mm}



\subsection{Results on clustering quality}
%{\em Quality.} \
\begin{figure*}[ht]
     \centering
     \subfigure[\baseline]{\includegraphics[width=0.2\textwidth]{twomoons-14000-original-clustered.png}\label{fig:twomoons-clustered-original}}
     \subfigure[\MM]{\includegraphics[width=0.2\textwidth]{twomoons-14000-sparsify-clustered-15.png}\label{fig:twomoons-clustered-sparsify}}
     \subfigure[\blackboard]{\includegraphics[width=0.2\textwidth]{twomoons-14000-chain-clustered.png}\label{fig:twomoons-clustered-chain}}
     \caption*{\twomoons, $k = 2$;}

\subfigure[\baseline]{\includegraphics[width=0.2\textwidth]{gauss-10000-original-clustered.png}\label{fig:gauss-clustered-original}}
     \subfigure[\MM]{\includegraphics[width=0.2\textwidth]{gauss-10000-sparsify-clustered-15.png}\label{fig:gauss-clustered-sparsify}}
     \subfigure[\blackboard]{\includegraphics[width=0.2\textwidth]{gauss-10000-chain-clustered.png}\label{fig:gauss-clustered-chain}}
     \caption*{\gauss, $k = 4$}


     \subfigure[\baseline]{\includegraphics[width=0.2\textwidth,height=0.2\textwidth]{sculpture-11680-original-clustered.png}\label{fig:sculpture-clustered-original}}  
     \subfigure[\MM]{\includegraphics[width=0.2\textwidth,height=0.2\textwidth]{sculpture-11680-sparsify-clustered-15.png}\label{fig:sculpture-clustered-sparsify}}
     \subfigure[\blackboard]{\includegraphics[width=0.2\textwidth,height=0.2\textwidth]{sculpture-11680-chain-clustered.png}\label{fig:sculpture-clustered-chain}}
     \caption*{\sculpture, $k = 3$. }


     
     \caption{Visualization of the results on \twomoons, \gauss\ and \sculpture. In the message passing model each site samples $5 n$ edges; in the blackboard model all sites jointly sample $10n$ edges (in \twomoons~ and \gauss) or $20n$ edges (in \sculpture) and the chain has length $18$. $s = 15$.}
     \label{fig:quality-1}
\end{figure*}

We visualize the clustered results for 
the \twomoons, \gauss\ and \sculpture\ in Figure~\ref{fig:quality-1}.
% and visualize the clustered results for \gauss\ and \sculpture in Figure~\ref{fig:quality-2}.
It can be seen that \baseline, \MM\ and \blackboard\ give results of very similar qualities.  For simplicity, here we only present the visualization for $s=15$. Similar results were observed when we varied the values of $s$.  
%\he{To Qin: Do you plan to have two titles (Results \& Quality)?}


% \begin{figure*}[h]
%      \centering
% \subfigure[\baseline]{\includegraphics[width=0.3\textwidth]{gauss-10000-original-clustered.png}\label{fig:gauss-clustered-original}}
%      \subfigure[\MM]{\includegraphics[width=0.3\textwidth]{gauss-10000-sparsify-clustered-15.png}\label{fig:gauss-clustered-sparsify}}
%      \subfigure[\blackboard]{\includegraphics[width=0.3\textwidth]{gauss-10000-chain-clustered.png}\label{fig:gauss-clustered-chain}}
%      \caption*{\gauss, $k = 4$}


%      \subfigure[\baseline]{\includegraphics[width=0.2\textwidth]{sculpture-11680-original-clustered.png}\label{fig:sculpture-clustered-original}}  
%      \subfigure[\MM]{\includegraphics[width=0.2\textwidth]{sculpture-11680-sparsify-clustered-15.png}\label{fig:sculpture-clustered-sparsify}}
%      \subfigure[\blackboard]{\includegraphics[width=0.2\textwidth]{sculpture-11680-chain-clustered.png}\label{fig:sculpture-clustered-chain}}
%      \caption*{\sculpture, $k = 3$. }

%      \caption{Visualization of results on \gauss\ and \sculpture; in the message passing model each site samples $5 n$ edges; in the blackboard model all sites jointly sample $10n$ (in \gauss) or $20n$ (in \sculpture) edges and the chain has length $18$.}
%      \label{fig:quality-2}
% \end{figure*}


We also compare the normalized cut (ncut) values of the clustering results of different algorithms.  The results are presented in Figure \ref{fig:quality}. In all datasets, the ncut values of different algorithms are very close. The ncut value of \MM\ slightly decreases when we increase the value of $s$, while the ncut value of \blackboard\ is independent of $s$.
%We comment that in general, it is difficult to compare \MM\ and \blackboard\ directly because they are affected by different parameters.


\begin{figure*}[!ht]
  \centering
  \subfigure[\twomoons]{\includegraphics[width=0.33\textwidth]{twomoons-14000-ncut.png}\label{fig:twomoons-quality}}\hspace*{-1.1em}
  \subfigure[\gauss]{\includegraphics[width=0.31\textwidth]{gauss-10000-ncut.png}\label{fig:gauss-quality}}\hspace*{-1.1em}
  \subfigure[\sculpture]{\includegraphics[width=0.31\textwidth]{sculpture-11680-ncut.png}\label{fig:sculpture-quality}}\hspace*{-1.1em}
  \subfigure{\includegraphics[width=0.14\textwidth]{legend.png}}
     \caption{Comparisons on normalized cuts. In the message passing model, each site samples $5n$ edges; in each round of the algorithm in the blackboard model, all sites jointly sample $10n$ edges (in \twomoons~and \gauss) or $20n$ edges (in \sculpture) edges and the chain has length $18$.}
     \label{fig:quality}
\end{figure*}

%\textcolor{red}{To Jiecao: Can you put the color lines indicating baseline, message passing, and blackboard within one row in Pic 2? Withthis we can save some space.}

%\vspace{-1.5mm}

\subsection{Results on communication costs} 
\begin{figure*}[!ht]
     \centering
     \subfigure[\twomoons]{\includegraphics[width=0.3\textwidth]{twomoons-14000-communication.png}\label{fig:twomoons-communication}}
     \subfigure[\gauss]{\includegraphics[width=0.3\textwidth]{gauss-10000-communication.png}\label{fig:gauss-communication}}
     \subfigure[\sculpture]{\includegraphics[width=0.3\textwidth]{sculpture-11680-communication.png}\label{fig:sculpture-communication}}


     \subfigure[\twomoons]{\includegraphics[width=0.32\textwidth]{twomoons-14000-communication-2.png}\label{fig:twomoons-communication-2}}
     \subfigure[\gauss]{\includegraphics[width=0.32\textwidth]{gauss-10000-communication-2.png}\label{fig:gauss-communication-2}}
     \subfigure[\sculpture]{\includegraphics[width=0.32\textwidth]{sculpture-11680-communication-2.png}\label{fig:sculpture-communication-2}}
     \caption{Comparisons on communication costs. In the message passing model, each site samples $5n$ edges; in each round of the algorithm in the blackboard model, all sites jointly sample $10n$ (in \twomoons~and \gauss) or $20n$ (in \sculpture) edges and the chain has length $18$. }
     \label{fig:communication}
\end{figure*}

We compare the communication costs of different algorithms in Figure \ref{fig:communication}. We observe that while achieving similar clustering qualities as \baseline, both \MM\ and \blackboard\ are significantly more communication-efficient (by one or two orders of magnitudes in our experiments). We also notice that the value of $s$ does not affect the communication cost of \blackboard, while the communication cost of \MM\ grows almost linearly with $s$; when $s$ is large, \MM\ uses significantly more communication than \blackboard. These confirm our theory.  %In Figure~\ref{fig:mm-const} and Figure~\ref{fig:blackboard-const}   in Appendix~\ref{sec:parameters} we present how the performance of \MM\ and \blackboard\ are affected by their parameters.

%
%
%\vspace{-1.5mm}
%\paragraph{Summary.}  From our experimental results we conclude that \MM\ and \blackboard\ achieve similar clustering quality as the native algorithm \baseline, while significantly reduce the communication cost.  When the number of sites is large, \blackboard\ is more communication efficient than \MM, as predicted by our theory.



\subsection{Parameters in \MM\ and \blackboard}
\label{sec:parameters}

Figure \ref{fig:mm-const} shows in \MM how the value of ncut is affected by the number of sites and the number of edges sampled in each site. 
Here, each site samples $cn$ edges. 
When $c=3$ and $s=1$, the ncut value diverges in all datasets. This is because with such a small $c$, the algorithm does not generate a valid sparsifier. In general, increasing $c$ or $s$ will slightly decrease the ncut value. But once they are above some thresholds, the ncut values of \MM\ and \baseline\ become very close.

Figure \ref{fig:blackboard-const} shows in \blackboard  how the ncut value is affected by the number of iterations and the number of edges sampled. When the number of iterations is set to be $5$, ncut values diverge in all datasets. This is because we cannot expect to generate a valid sparsifier by using such few iterations. It can be seen from \ref{fig:bb-gauss-constant} that for a fixed $c$, performing more iterations will help to reduce ncut values. From the same figure, one can also conclude that for fixed iterations, increasing $c$ also helps to reduce the ncut values.



\begin{figure*}[h!t]
     \centering
     \subfigure[\twomoons]{\includegraphics[width=0.3\textwidth]{twomoons-c.png}\label{fig:mm-twomoons-constant}}
     \subfigure[\gauss~dataset]{\includegraphics[width=0.3\textwidth]{gauss-c.png}\label{fig:mm-gauss-constant}}
     \subfigure[\sculpture]{\includegraphics[width=0.3\textwidth]{sculpture-c.png}\label{fig:mm-sculpture-constant}}
     \caption{The pictures above show the $\ncut$ values with respect to the values of $c$ and $s$ for the \MM\ algorithm. Here  
 each site samples $c n$ edges.}
     \label{fig:mm-const}
\end{figure*}


\begin{figure*}[h!t]
     \centering
     \subfigure[\twomoons]{\includegraphics[width=0.3\textwidth]{twomoons-iter.png}\label{fig:bb-twomoons-constant}}
     \subfigure[\gauss]{\includegraphics[width=0.3\textwidth]{gauss-iter.png}\label{fig:bb-gauss-constant}}
     \subfigure[\sculpture]{\includegraphics[width=0.3\textwidth]{sculpture-iter.png}\label{fig:bb-sculpture-constant}}
     \caption{The pictures above show how the $\ncut$ values are affected by the number of iterations and the value of $c$ for the \blackboard\ algorithm. Here 
all sites jointly sample $c n$ edges. }
     \label{fig:blackboard-const}
\end{figure*}







% Conclusion
\section{Conclusion}

\begin{comment}
\begin{figure}
\includegraphics[width=\linewidth]{figs/beyond_tss_lesion.pdf}
\caption[]{End-to-End runtime lesion study of the entire MNIST dataset and the FMA featurized music dataset. Each of DROP's contributions provides a runtime improvement.}
\label{fig:beyond_lesion}
\end{figure}
\end{comment}



\section{Conclusion}
\label{sec:conclusion}

Advanced data analytics techniques must scale to rising data volumes. 
DR techniques offer a powerful toolkit when processing these datasets, with PCA frequently outperforming popular techniques in exchange for high computational cost. 
In response, we propose DROP, a new dimensionality reduction optimizer. 
DROP combines progressive sampling, progress estimation, and online aggregation to identify high quality low dimensional bases via PCA without processing the entire dataset by balancing the runtime of downstream tasks and achieved dimensionality. 
Thus, DROP provides a first step in bridging the gap between quality and efficiency in end-to-end DR for downstream \red{analytics}. 

%We revisit canonical operators for time series dimensionality reduction and the measurement study of~\cite{keogh-study}, and show that PCA is more effective than popular alternatives in the data mining literature often by a margin of over $2\times$ on average on gold-standard time series benchmark data sets with respect to output data dimension. More surprisingly, we empirically demonstrate that a small number of samples are sufficient to accurately characterize directions of maximum variance and obtain a high-quality low-dimensional transformation.




% Ack
\section*{Acknowledgement}
We would like to start by thanking
our sponsors: Stanford Computer Science Department and Stanford Program in AI-assisted Care
(PAC). Next, we specially thank De-An Huang, Kenji Hata, Serena Yeung, Ozan Sener and all the members of Stanford Vision and Learning Lab for their insightful discussion and feedback. Lastly, we thank all the anonymous reviewers for their valuable comments.

\clearpage
\bibliography{nips_2017.bib}{}
\bibliographystyle{plain}

% Supp
\clearpage
%%% Network Architecture
\section*{Network Architecture}
%% SVHN -> MNIST
\subsection*{(a) SVHN 0-4 $\rightarrow$ MNIST 5-9}
% Encoder
\begin{table}[htbp]
\centering
\caption{Embedding network structure}
\begin{tabular}{|c|c|c|c|c|}
\hline
name & conv1 & pool1 & conv2 & pool2\\
\hline
\hline
layer type & conv-batchnorm-relu & max pool & conv-batchnorm-relu & max pool\\
kernel & 3$\times$3$\times$64 & 2$\times$2 & 3$\times$3$\times$64 & 2$\times$2\\
stride & 1 & 2 & 1 & 2\\
padding & 1 & 0 & 1 & 0\\
\hline
\hline
name & conv3 & pool3 & conv4 & pool4\\
\hline
\hline
layer type & conv-batchnorm-relu & max pool & conv-batchnorm-relu & max pool\\
kernel & 3$\times$3$\times$64 & 2$\times$2 & 3$\times$3$\times$64 & 2$\times$2\\
stride & 1 & 2 & 1 & 2\\
padding & 1 & 0 & 1 & 0\\
\hline
\hline
name & fc1 & fc2 & {} & {}\\
\hline
\hline
layer type & fc-relu & fc & {} & {}\\
kernel & 64$\times$64 & 64$\times$5 & {} & {}\\
\hline
\end{tabular}
\end{table}

% Discriminator
\begin{table}[htbp]
\centering
\caption{Discriminator structure}
\begin{tabular}{|c|c|c|c|c|c|c|}
\hline
name & fc1 & fc2 & fc3 & fc4 & fc5 & fc6\\
\hline
\hline
layer type & fc-relu & fc-relu & fc-relu & fc-relu & fc-relu & fc\\
kernel & 64$\times$64 & 64$\times$5 & 5$\times$500 & 500$\times$500 & 500$\times$500 & 500$\times$1\\
\hline
\end{tabular}
\end{table}
\clearpage

%% Image -> Video
\subsection*{(b) Image object recognition $\rightarrow$ video action recognition}
% Encoder
\ \ \ \ \ \ Embedding network structure: ResNet-18~\footnote{We refer readers to the PyTorch implementation: \url{https://github.com/pytorch/vision/blob/master/torchvision/models/resnet.py}}.

% Decoder
\begin{table}[htbp]
\centering
\caption{Discriminator structure}
\begin{tabular}{|c|c|c|c|c|}
\hline
name & conv1 &  conv2 & conv3\\
\hline
layer type & conv-batchnorm-leaky relu & conv-batchnorm-leaky relu & conv\\
kernel & 3$\times$3$\times$512  & 3$\times$3$\times$512 & 1$\times$1$\times$1\\
stride & 2 & 1 & 1\\
padding & 1 & 1 & 0\\
\hline
\end{tabular}
\end{table}

\subsection*{(c) Ablation: unsupervised domain adaptation}
% Encoder
\begin{table}[htbp]
\centering
\caption{Embedding network structure}
\begin{tabular}{|c|c|c|c|c|}
\hline
name & conv1 & pool1 & conv2 & pool2\\
\hline
\hline
layer type & conv-relu & max pool & conv-relu & max pool\\
kernel & 5$\times$5$\times$20 & 2$\times$2 & 5$\times$5$\times$20 & 2$\times$2\\
stride & 1 & 2 & 1 & 2\\
padding & 0 & 0 & 0 & 0\\
\hline
\hline
name & fc1 & fc2 & {} & {}\\
\hline
\hline
layer type & fc-relu & fc & {} & {}\\
kernel & 800$\times$500 & 500$\times$10 & {} & {}\\
\hline
\end{tabular}
\end{table}

% Discriminator
\begin{table}[htbp]
\centering
\caption{Discriminator structure}
\begin{tabular}{|c|c|c|c|c|c|}
\hline
name & fc1 & fc2 & fc3 & fc4 & fc5\\
\hline
\hline
layer type & fc-relu & fc-relu & fc-relu & fc-relu & fc\\
kernel & 800$\times$500 & 500$\times$10 & 10$\times$500 & 500$\times$500 & 500$\times$1 \\
\hline
\end{tabular}
\end{table}

\end{document}
% \bibliographystyle{plain}
% \bibliography{/Users/trukat/Dropbox/library/library}

\begin{thebibliography}{1}

\bibitem{Bhatia2015}
Kush Bhatia, Himanshu Jain, Purushottam Kar, Manik Varma, and Prateek Jain.
\newblock {Sparse local embeddings for extreme multi-label classification}.
\newblock {\em Advances in Neural Information Processing Systems (NIPS '15)},
  pages 730--738, 2015.

\bibitem{Jain2017}
Vikas Jain, Nirbhay Modhe, and Piyush Rai.
\newblock Scalable generative models for multi-label learning with missing
  labels.
\newblock In Doina Precup and Yee~Whye Teh, editors, {\em Proceedings of the
  34th International Conference on Machine Learning}, volume~70 of {\em
  Proceedings of Machine Learning Research}, pages 1636--1644, International
  Convention Centre, Sydney, Australia, 2017. PMLR.

\bibitem{Rukat2017}
Tammo Rukat, Chris~C. Holmes, Michalis~K. Titsias, and Christopher Yau.
\newblock {Bayesian Boolean Matrix Factorisation}.
\newblock {\em Proceedings of the 34th Annual International Conference on
  Machine Learning}, pages 2969--2978, jul 2017.

\end{thebibliography}



% References follow the acknowledgments. Use unnumbered first-level
% heading for the references. Any choice of citation style is acceptable
% as long as you are consistent. It is permissible to reduce the font
% size to \verb+small+ (9 point) when listing the references. {\bf
%   Remember that you can go over 8 pages as long as the subsequent ones contain
%   \emph{only} cited references.}
% \medskip

% \small

\end{document}
