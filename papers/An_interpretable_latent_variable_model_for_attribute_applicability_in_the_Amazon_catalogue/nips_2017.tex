\documentclass{article}

% if you need to pass options to natbib, use, e.g.:
% \PassOptionsToPackage{numbers, compress}{natbib}
% before loading nips_2017
%
% to avoid loading the natbib package, add option nonatbib:
%c\usepackage[nonatbib]{nips_2017}

\usepackage[nonatbib,final]{nips_2017}
\usepackage[numbers]{natbib}

% to compile a camera-ready version, add the [final] option, e.g.:

\usepackage[utf8]{inputenc} % allow utf-8 input
\usepackage[T1]{fontenc}    % use 8-bit T1 fonts
\usepackage{hyperref}       % hyperlinks
\usepackage{url}            % simple URL typesetting
\usepackage{booktabs}       % professional-quality tables
\usepackage{amsfonts}       % blackboard math symbols
% \usepackage{nicefrac}       % compact symbols for 1/2, etc.
\usepackage{microtype}      % microtypography
\usepackage{graphicx}

\title{An interpretable latent variable model for attribute applicability in the Amazon catalogue}

% The \author macro works with any number of authors. There are two
% commands used to separate the names and addresses of multiple
% authors: \And and \AND.
%
% Using \And between authors leaves it to LaTeX to determine where to
% break the lines. Using \AND forces a line break at that point. So,
% if LaTeX puts 3 of 4 authors names on the first line, and the last
% on the second line, try using \AND instead of \And before the third
% author name.

\author{
  Tammo Rukat\thanks{Work was done at Amazon Berlin}\\
  Department of Statistics\\ University of Oxford\\
  \texttt{tammo.rukat@stats.ox.ac.uk}
  \And
  Dustin Lange\\
  Amazon\\ Berlin, Germany\\
  \texttt{cedrica@amazon.com}  
  \And
  C\'edric Archambeau\\
  Amazon\\ Berlin, Germany\\
  \texttt{langed@amazon.com}  
  % David S.~Hippocampus\thanks{Use footnote for providing further
  %   information about author (webpage, alternative
  %   address)---\emph{not} for acknowledging funding agencies.} \\
  % Department of Computer Science\\
  % Cranberry-Lemon University\\
  % Pittsburgh, PA 15213 \\
  % \texttt{hippo@cs.cranberry-lemon.edu} \\
  %% examples of more authors
  %% \And
  %% Coauthor \\
  %% Affiliation \\
  %% Address \\
  %% \texttt{email} \\
  %% \AND
  %% Coauthor \\
  %% Affiliation \\
  %% Address \\
  %% \texttt{email} \\
  %% \And
  %% Coauthor \\
  %% Affiliation \\
  %% Address 
  %% \texttt{email} \\
  %% \And
  %% Coauthor \\
  %% Affiliation \\
  %% Address \\
  %% \texttt{email} \\
}

\begin{document}
% \nipsfinalcopy is no longer used

\maketitle

\begin{abstract}
  Learning attribute applicability of products in the Amazon catalog (e.g., predicting that a \texttt{shoe}
  should have a value for \texttt{size}, but not for
  \texttt{battery-type}) at scale is a challenge. The need for an interpretable model is contingent on (1) the lack of ground truth training data, (2) the need to utilise prior information about the underlying latent space and (3) the ability to understand the quality of predictions on new, unseen data.
  To this end, we develop the MaxMachine, a probabilistic latent
  variable model that learns distributed binary representations,
  associated to sets of features that are likely to co-occur in the
  data. Layers of MaxMachines can be stacked such that higher layers
  encode more abstract information. Any set of variables can be
  clamped to encode prior information. We develop fast sampling based posterior inference.
  Preliminary results show that the model improves over the baseline in 17 out of 19 product groups and provides qualitatively reasonable predictions.
\end{abstract}


\section{Attribute Applicability}
\label{sec:org75c4eff}
Many real-world datasets can be viewed as object-by-attribute matrices. A prominent example is the Amazon catalogue which contains over 100 million products (objects) and hundreds of attributes, of which only a small subset is assigned to each product. Thus, product-attribute-assignment can be viewed as a sparse binary matrix, shown for a small subsample of the German Amazon marketplace in Fig.\space{}\ref{fig:data}.
\begin{figure}[h]
  \centering
\includegraphics[width=.65\textwidth]{./data_random4.png}
\caption{
  Attribute applicability as binary matrix for a random subset of the German marketplace. Attributes are filtered for being applicable to at least 1\% of the sampled products. Black dots denote ones, indicating applied attributes. White dots denote zeros, indicating the absence of any attribute value.
  \label{fig:data}}
\end{figure}
Being able to distinguish between attributes that are truly
non-applicable (e.g., \texttt{battery-type} for a shoe), attributes that 
could reasonably be applied (e.g., \texttt{weight} for a book), and
attributes that are clearly applicable (e.g., \texttt{size} for a T-shirt) is
crucial for applications such as attribute imputation models, data
quality management, template generation, product comparison and
virtually all customer-facing downstream applications.

We can cast the task of predicting attribute applicability as a
multi-label classification problem, 
where each attribute constitutes a label and an arbitrary number 
of labels is assigned to each product. While there is recent progress
in such extreme multi-label classification problems
\cite{Bhatia2015,Jain2017}, we face a particular challenge:
The absence of reliable training labels makes it difficult to
define a training metric. Therefore, we approach attribute
applicability as an unsupervised problem and develop a probabilistic
latent variable model
that describes the generative process by which the binary
product/applicability matrix is generated from a set of latent
features. We aim to retain a simple,
interpretable model, resembling the process of a marketplace
seller who is filling in attributes for their product.
The rationale behind the model is that each latent
feature corresponds to a set of attributes that are likely to appear
together such as (\texttt{title}, \texttt{pages}, \texttt{language}, \texttt{release date}) or
(\texttt{width}, \texttt{height}, \texttt{length}).
Each of these sets is represented by a latent dimension and the
generative process for any product-attribute-pair is a noisy
disjunction of these feature sets. 
The model design is further motivated by keeping sampling-based
posterior inference scalable.
We also assume that additional product or attribute specific prior
information is available. Here, this is exemplified by the
\texttt{product type}, a string attribute that is assigned to a majority of products. Product types are curated such that there exist only
around 500 distinct values. They provide useful information that we need
to leverage for optimal performance. Moreover, attribute assignments
have a strong product type specific pattern which lets us use the
product type specific attribute frequency as a baseline probability estimate.

\section{MaxMachine Model}
\label{sec:orga18e9af}

A first candidate model is Boolean Matrix 
Factorisation \cite{Rukat2017},
where, both, the features and their allocations are
binary. The data generating process is a noisy matrix product between these
binary matrices whose result is thresholded at 1. With \(N\) products and \(D\)
attributes \(x_{nd} \in \{0,1\}\) denotes the application status for
a product-attribute-pair. We have $L$ latent dimensions and the factor matrices
are denoted as \(U \in \{0,1\}^{L\times D}\) and \(Z \in
  \{0,1\}^{N\times L}\), such that \(u_{l,d=1,\ldots, D}\) encodes a set
of co-occuring attributes and \(z_{n, l=1,\ldots,L}\) denotes the
compressed representation of observation \(x_n\). The likelihood for
Boolean Matrix Factorisation then takes the form $ p(x_{nd}|.) = \sigma[\lambda \tilde{x}_{nd} (1-2 \prod_{l}(1-z_{nl}u_{ld}) ) ]\;, $
where $\sigma$ is the logistics sigmoid and $\tilde{x} = 2x{-}1$. Note that the product over \(l\) is the Boolean disjunction. This model has a global noise
parameter, $\lambda \in \mathbb{R}^+$, that governs the random
flipping of bits. However, due to the heterogeneity in the data, we require a more expressive model
that can capture heteroscedastic noise. To this end, we propose the
\emph{MaxMachine}. Here, the noise for each latent dimension is
governed by a separate parameter and each data point is generated from
the least noisy associated latent dimension. Hence the model retains
composability and is easily interpretable.
The likelihood takes the form
$  p(x_{nd}|.) = \sigma\left[\tilde{x}_{nd} \max_l (\lambda_l z_{nl} u_{ld})\right],$
where $\lambda_l \in \mathbb{R}$.
Using the max-operation, the latent dimensions compete for
explaining the observations. The \emph{winner} is 
the most accurate predictor and gets to fully explain the
observation. In order for the model to be well defined we have an
additional, clamped latent dimension with \(u_{l,d}{=}z_{nl}{=}1 \;\forall\;(n,d)\).
We propose a beta prior on each $\sigma(\lambda_l)$ and binomial priors on the cardinality of the rows of $U$. The latter can encode our prior belief that the number of co-occuring attributes in each set is much smaller than the total number of attributes.\\
Now, we include the product-type information by adding another layer of
matrix factorisation in the spirit of a \textit{Bayesian hierarchical model}.
This means that the prior on the matrix of latent representation is
factorised according to another MaxMachine moandel. Here, the
higher-order object specific factor matrix is fixed to an encoding of the product type
in a one-hot fashion.


\subsection{Inference}
\label{sec:orge79dcef}
The inference task amounts to estimating, both, the attribute sets
and their assignments and is combinatorially challenging.
We develop sampling based posterior inference, an approach that
has been shown to outperform competing methods in Boolean Matrix
Factorisation\space{}\cite{Rukat2017}. Computation of the full
conditional probabilities of each variable \(u_{ld}\) or \(z_{nl}\)
generally depends on the variables full Markov blanket. We make use
of several
algorithmic tricks and leverage the purely binary states of all
variables as well as the lack of interaction between dimensions that is
induced by the max operation. This enables efficient updates, such that
the algorithm converges for hundreds of thousands of data points
within few minutes on a laptop. After every sweep through all entries of $U$ and $Z$, we set
all $\lambda_l$ to their MAP estimate which is analytically available.
Following posterior inference, we can compute Monte Carlo estimates
of the posterior predictive and thus predict applicability.

%However, as shown in Fig.\space{}\ref{fig:orgb9da665}

% We find empirically that the predictive performance does not benefit from more than $O(1e3)$ training samples, which can be explained by their redundancy.

% \begin{figure}[htbp]
% \centering
% \includegraphics[width=10cm]{./figs/subsampling.png}
% \caption{\label{fig:orgb9da665}
% Predictive performs under variation of the size of the training data.}
% \end{figure}



\section{Experiments and Results}
We consider a snapshot of the Amazon German marketplace that have been found to be particularly important to customers.
In order to simplify analysis and evaluation we stratify the data into
19 different clusters corresponding to related types of products, such
as \texttt{clothing}, \texttt{computers} or \texttt{jewellery}. Each
cluster consists of products from a variety of product types (1-50). In particular products from different clusters share only very few attributes and therefore share no co-occurrence patterns of mutual relevance. We train on only 500 randomly sampled products from each product type, since a further increase in training data has no effect on the quality of the results. This is due to the redundancy in the binary data.\\
We measure test-set performance by treating randomly selected
product-attribute pairs as unobserved during training and evaluate the area under the ROC curve for the posterior predictive on these test data points.
For the following, experiments we choose \texttt{binomial(0.1)}
priors on the cardinality of each latent set of 
attributes, reflecting our prior belief that the number of
attributes in each co-occuring set is relatively small. For the 
noise parameters (mapped to $[0,1]$), we use a \texttt{beta(10,1)} prior, encoding our believe 
that attribute sets that are applied to a product are relatively
likely to be actually present in the data.
Based on random search, we choose
the remaining hyperparameters.
% as follows.
% The prior on latent representation for each asins is iid Bernoulli
% 0.5. and the initialisation is set means cluster
% centroids along the corresponding dimension of the data. 
% The latent features are initialised uniformly iid random with
% \(p(u=1)=10\%\). 
% Empirically, we find that the variation for using different priors is on the same scale as for using different random initialisations.


\begin{figure}[htbp]
\centering
\includegraphics[width=12cm]{./experiment2_codes.png}
\caption{\label{fig:orgca0a434}
Patterns of attribute co-occurrence. Shown are posterior means of the
inferred codes \(U\) (black: 1, white: 0). Each row denotes a set of
frequently co-occurring attributes, each column denote an attribute; $\nu$ denotes the expected
percentage of ones in the data that are explained by the corresponding
dimension, $\hat\lambda$ is the average posterior MAP of the noise parameter.}
\end{figure}

We show the inferred sets of co-occurring attributes for clothing
products in Fig.\space{}\ref{fig:orgca0a434}. They indicate reasonable co-occurrences such as (\texttt{waist-style}, \texttt{waist-size}, \texttt{inseam-length},~\ldots).
The corresponding posterior predictive achieves a ROC-AUC of 94\% on held out data, while the product type mean reaches close to 93\%. The
moderate improvements in ROC-AUC can partly be attributed to the
fact that 
the product-type mean predictor has a higher certainty for
attributes that are present in almost all products.
Qualitative evaluation shows that the model makes reasonable
predictions, as for instance to add the attribute \texttt{material} to most
products of type \texttt{bra} or the attribute \texttt{manufacturer} to products of type
\texttt{shirt}, more anecdotal evidence is provided in Table
% \ref{tab:org80ba429} and
\ref{tab:orgdc754d7}.
We find a
non-zero, but rather low probability of adding the attribute
\texttt{cup-size} to products of type \texttt{bra}. This can be understood by
noting that \texttt{cup-size} occurs for no other product type and,
therefore, cannot be inferred from correlation. While the probability
is low, it is notable evidence if considered in relative terms:
compared to any other product types, products of the type \emph{bra} have 
by far the highest probability of the attribute \emph{cup-size} being
applicable. This suggests the exploration of attribute-specific
thresholds for practical applications.\\
We repeat this experiment across all product types in the catalogues and find that
the  MaxMachine outperforms the baseline
by margins between 1\% and 15\% in 17 out of 19 clusters. 
The slightly weaker performance in the remaining cluster can largely be explained by extremely homogeneous attribute distributions for each of the contained product types.

% \begin{table}[htbp]
%   \small
% \caption{\label{tab:org80ba429}\small
% Anecdotal evidence. We choose three typical asins from different product types and list all atributes that are added with a probability \(>25\%\) or removed with a probability \(>99\%\).}
% \centering
% \begin{tabular}{l|ll|ll|ll}
%  & type: shorts &  & type: bra &  & type: software & \\
%  & attribute & p(apply) & attribute & p(apply) & attribute & p(apply)\\
% \hline
% \textbf{add} & department & \(82\%\) & collection & \(36\%\) & edition & \(65\%\)\\
%  & care-indstruction & \(44\%\) & style & \(31\%\) & bullet point & \(58\%\)\\
%  & age-gender category & \(42\%\) & lifesstyle & \(27\%\) & width, height & \(48\%\)\\
%  & waist-style & \(32\%\) & pattern-type & \(25\%\) & length & \\
%  & inseam-length & \(32\%\) & band-size & \(25\%\) & genre & \(33\%\)\\
% \hline
% \textbf{remove} & item-weight & \(<1\%\) & item weight & \(<1\%\) & binding & \(<1\%\)\\
%  & special-size-type & \(<1\%\) & type keyword & \(<1\%\) & model year & \(<1\%\)\\
%  & sport-type & \(<1\%\) &  &  &  & \\
% \end{tabular}
% \end{table}


\begin{table}[htbp]
  \small
\caption{\label{tab:orgdc754d7}\small
Anecdotal evidence -- for three attributes we list the product types that they are most likely applied to by the model. The percentage is the mean probability of being applied for all products in the product type. In brackets we give the mean probability only for those products that do not have corresponding attribute assigned.}
\centering
\begin{tabular}{l|lll|}
Attribute & cup-size & closure-type & leather-type\\
\hline
Product types & Bra \(22 (10) \%\) & Shoes \(48 (18) \%\) & Shoes \(48 (15)\%\)\\
with largest & Swimwear \(3 (2) \%\) & Pants \(24 (10) \%\) & Outerwear \(3 (3)\%\)\\
p(apply) & Underwear \(3 (2) \%\) & Shorts \(6 (3) \%\) & Shorts \(2(2)\%\)\\
 & Shoes \(2 (2) \%\) & Outerwear \(4 (2) \%\) & (< \(1\%\))\\
 & Suit \(2 (2) \%\) & Bra \(2 (2)\%\) & \\
\end{tabular}
\end{table}



% \begin{figure}[htbp]
% \centering
% \includegraphics[width=14cm]{./figs/experiment_cmrps.png}
% \caption{\label{fig:orgb0d10f7}
% Performance comparison across attribute clusters. The hierarchical MaxMachine achieves the best predictive performs in all clusters, except for \emph{media} and \emph{movie downloads}. The shallow MaxMachine does not make use of any product-type information and still outperforms the baseline in 11/19 cases.}
% \end{figure}


% \begin{figure}[htbp]
% \centering
% \includegraphics[width=17cm]{./figs/rocs.png}
% \caption{\label{fig:org2a670c3}
% Receiver-operator characteristics for three product type clusters. The hierachical MaxMachine model achieves the best area under the curve for all examples. For \emph{clothing} products (right) and high thresholds (top region of the subplot), the mean predictor achives higher true positive rate. However, the regions of the curve that are relevant for our application are low false positive rates.}
% \end{figure}


\section{Conclusion}
\label{sec:org5d08bba}
We have described the problem of attribute applicability in the
Amazon catalogue and developed a latent variable model for the
denoising of product/applicability matrices.
Due to the lack of ground truth data we have optimised
reconstruction accuracy conditional on a model that describes an
intuitive generative process, resembling the real-world
procedure of a seller, assigning attributes to products.
As a baseline we have used product-type specific patterns of
applicability and improved in the area under the ROC curve on
hold-out data for 17 out of 19 product clusters.
Anecdotal evidence confirms that our model makes reasonable
predictions.\\
In a practical scenario, more prior expert knowledge
might be available and of high importance. Many types of such
information can be flexibly integrated in the proposed
model. For instance the presence of certain important attributes could be clamped
for certain types of products, while it is inferred for others.
For future work, it would be desirable to avoid the subjective
choice of hyperparameters. In particular the use of non-parametric
priors could lead to a more principled choice of the latent
dimensionality and help model convergence.

\section{Acknowledgements}
We thank Felix Biessmann and David Salinas for helpful discussions, their insights and their expertise that greatly assisted this research. 

\small
% %\documentclass{article}

% if you need to pass options to natbib, use, e.g.:
% \PassOptionsToPackage{numbers, compress}{natbib}
% before loading nips_2017
%
% to avoid loading the natbib package, add option nonatbib:
% \usepackage[nonatbib]{nips_2017}

\usepackage{nips_2017}

% to compile a camera-ready version, add the [final] option, e.g.:
% \usepackage[final]{nips_2017}

\usepackage[utf8]{inputenc} % allow utf-8 input
\usepackage[T1]{fontenc}    % use 8-bit T1 fonts
\usepackage{hyperref}       % hyperlinks
\usepackage{url}            % simple URL typesetting
\usepackage{booktabs}       % professional-quality tables
\usepackage{amsfonts}       % blackboard math symbols
\usepackage{nicefrac}       % compact symbols for 1/2, etc.
\usepackage{microtype}      % microtypography
\usepackage[utf8]{inputenc}
\usepackage{graphicx}
\usepackage{subcaption}
\usepackage{amsmath}

\title{Rapid point-of-care Hemoglobin measurement through low-cost optics and Convolutional Neural Network based validation}

% The \author macro works with any number of authors. There are two
% commands used to separate the names and addresses of multiple
% authors: \And and \AND.
%
% Using \And between authors leaves it to LaTeX to determine where to
% break the lines. Using \AND forces a line break at that point. So,
% if LaTeX puts 3 of 4 authors names on the first line, and the last
% on the second line, try using \AND instead of \And before the third
% author name.

  \author{
  Chris Wu\\
  Athelas Inc. \\
  \texttt{chris@athelas.com}
  \And
  Tanay Tandon\\
  Athelas Inc. \\
  \texttt{tanay@athelas.com}
}
\begin{document}
% \nipsfinalcopy is no longer used
\maketitle
\begin{abstract}
  A low-cost, robust, and simple mechanism to measure hemoglobin would play a critical role in the modern health infrastructure. Consistent sample acquisition has been a long-standing technical hurdle for photometer-based portable hemoglobin detectors which rely on micro cuvettes and dry chemistry. Any particulates (e.g. intact red blood cells (RBCs), microbubbles, etc.) in a cuvette's sensing area drastically impact optical absorption profile, and commercial hemoglobinometers lack the ability to automatically detect faulty samples. We present the ground-up development of a portable, low-cost and open platform with equivalent accuracy to medical-grade devices, with the addition of CNN-based image processing for rapid sample viability prechecks. The developed platform has demonstrated precision to the nearest $0.18[g/dL]$ of hemoglobin, an \(R^{2} = 0.945\) correlation to hemoglobin absorption curves reported in literature, and a 97\% detection accuracy of poorly-prepared samples. We see the developed hemoglobin device/ML platform having massive implications in rural medicine, and consider it an excellent springboard for robust deep learning optical spectroscopy: a currently untapped source of data for detection of countless analytes.
\end{abstract}
\section{Introduction}
Hemoglobin is one of the most common blood tests requested by clinics, and can be used in conjunction with other metrics to diagnose a host of diseases and conditions \cite{mayoclinic}\cite{biomed}, quantify the effects of pharmaceutical drugs \cite{medlineplus2}, and provide a holistic health benchmark \cite{biomed}. As roughly a quarter of the world’s population suffers from a form of hemoglobin deficiency \cite{who}, there are many niches where accessible hemoglobin measurement would fulfill significant unmet need \cite{cdc}\cite{drugsaging}\cite{biomed}.

The widely-adopted approach to point-of-care hemoglobin measurement involves hemolysing whole blood and converting hemoglobin derivatives for single-wavelength absorption measurement, followed by a simple Beer Law calculation \cite{hemocue}\cite{vanzetti}\cite{oshiro}. In present hemoglobin monitors the single biggest risk of inaccurate counting and as a result, incorrect clinical decision making, is incomplete hemolysis due to variable reagent performance. Consequently, light scattering interference from intact cells (lipid membranes) have a drastic effect on spectrophotometric output \cite{lipemia}.

Our intention in developing the presented device platform was twofold: to create an effective and inexpensive hemoglobin solution without the need for hazardous dry reagents such as cyanide (Drabkin's method) or sodium azide \cite{hemocue}\cite{vanzetti}\cite{azide}, and to demonstrate the efficacy of convolutional neural network based validation for elimination of a long-standing error source in hemoglobin measurement. 

\section{Detection platform summary}

The current version of the device has demonstrated: 1) precision to the nearest $0.18[g/dL]$ of Hb, 2) stability of measurement over time, with no fluctuation from a single sample over 10 minutes of continuous output, and 3) \(R^2 = 0.945\) correlation to optical transmission-Hb curve reported in literature (Fig.\ref{fig:HbCorr} image A).

\subsection{Device platform}

\begin{figure*}[!ht]
\centering
    \includegraphics[width=0.8\textwidth]{Images/design.png}
    \caption{Image A) Design concept rendering. Image B) Completed device with blank strip and power adapter inserted.}
    \label{fig:Chassis}
\end{figure*}

\subsubsection{Sensor configuration}
For Hb measurement, a paired $540[nm]$ emitter-detector setup is used for sample transmission calculation. The LED and photodiode were selected to maximize light emission, photocurrent, and Hb absorption.

Image pre-processing for sample viability assurance are captured with a Pi Camera attached to a low-cost optical microscopy setup (separate from device at this point in development). See figure 4 for optical microscopy imaging examples. Both setups have a test strip insertion slot and a plane for allowing optical analysis.  
 
\subsubsection{Signal processing}
An $LTC1050$ Chopper-stabilized op amp is used in a standard transimpedance amplifier configuration for signal processing with a reverse-biased photodiode for signal linearity. Additional components protect sensitive analog signals from various forms of electromagnetic interference, including high frequency noise and LC-tank oscillation due to inherent component properties. As absorption is taken at steady-state, signal bandwidth is not a design concern.

Hb concentration is linearly proportional to optical absorbance ($A$), which is calculated from the negative log of fractional transmission, $Tf/Ti$. From Beer’s Law:
\begin{equation}
C = -Klog(\frac{Tf}{Ti})
\end{equation}
for $K = 1/(el)$, where $e$ is the molar extinction coefficient of hemoglobin, $l$ is the transmission path length ($100[\mu m]$), and $C$ is the concentration of hemoglobin in the sample (in $[g/dL]$).

The device's microcontroller contains a 10-bit ADC with a minimum voltage increment of $5[mV]$. With a swing of $1.5[V]$ the resolution (i.e. the smallest detectable change in sample percent transmission) is 0.333\%. Readjusting $R1$ to maximize full output swing ($5[V]$) will improve resolution in the next device iteration to 0.1[\%T]. For clinically relevant Hb levels and 0.333[\%T] increment size, the resolution is $0.18 [g/dL]$ Hb  (i.e. the difference between consecutive Hb values over a 0.333[\%T] step). This exceeds the precision of predicate devices such as Hemocue, which has an overall bias $\pm$ stdev of -$0.1 \pm 1.6[g/dL]$ compared to lab-grade hematology analyzers \cite{shah}. Of course, this level of precision from the presented device assumes no variability in capillary strip performance: a hefty assumption. 

\begin{figure*}[!ht]
\centering
    \includegraphics[width=0.8\textwidth]{Images/fig1.png}
    \caption{Image A) Plot of literature-reported Hb values for observed \%T measurements (extrapolated from $540[nm]$ molar extinction coefficient \cite{coeff1}\cite{coeff2}\cite{coeff3}) vs. the developed device readout for those \%T values. Strong correlation demonstrates hardware precision and accuracy over the full clinically-relevant range of hemoglobin levels. Image B) PIN photodiode and first amplifier stage used for signal acquisition.}
    \label{fig:HbCorr}
\end{figure*}
\subsection{Capillary test strip}

The test strip consists of a 100uM microchannel coated with sodium lauryl sulfate (SLS), a non-toxic surfactant which serves the dual purpose of lysing RBCs to eliminate turbidity and converting hemoglobin derivatives into a color-stable complex, SLS-Hb.
\begin{figure*}[!ht]
\centering
    \includegraphics[width=0.80\textwidth]{Images/blood.png}
    \caption{Image A is a diagram of the capillary strip design used. Strip with blood sample shown in B. Close-up of a channel demonstrating common issues with dry-reagent capillary strip methods in image C: turbidity at the bottom of the channel due to intact RBCs and air bubbles (white specks) at top. Image D shows the mechanical frame of the sensor input slot with the sample transmission measurement window.}
    \label{fig:blood}
\end{figure*}

In standard practice with SLS, $20[uL]$ of whole blood is added and mixed with $5[mL]$ of $2.08[mmol/L]$ SLS solution for a 0.52 moles SLS to blood sample volume ratio \cite{oshiro}. To adapt for dry-chemistry, $17[\mu mols]$ of SLS in aqueous solution were deposited and spread evenly onto each coverslip, for a total of $34[\mu mols]$ per $66[\mu L]$ available sample volume, maintaining the ratio.

However, since all fluid entering a capillary strip generally shares the same path, depending on viscosity and RBC density, the first units of blood entering the channel may use up reagent at the entrance, leaving none for subsequent units (hence incomplete hemolysis). This issue poses a critical error-source common to all dry capillary strip detection methods, which can be minimized through clever strip design.

A funnel-shaped channel (Fig.\ref{fig:blood} image A) was used to help mitigate the “use up” issue described. Blood deposited at the entrance of the funnel spreads out to cover a larger area before entering the narrow measurement area, having been hemolyzed by reagent in the funnel. While this design was generally effective in eliminating turbidity in the measurement window (Fig.\ref{fig:blood} panel D), variation in RBC density and sample viscosity prevented proper mixing/reaction on occasion, precluding this dry-chemistry approach from large-scale, frequent use with minimal user training. Only an automated viability check to auto detect such errors when present can completely eliminate this risk. 

\subsubsection{CNN-based image processing for sample viability check}

A core component of the presented system is the trained machine learning model for sample validation.

A separate imaging module was developed to analyze intact cells, air bubbles, and ensure appropriate sample prep. A filled strip is inserted into the imaging module, several images are taken across the strip at a scaled magnification, and a trained convolutional neural network determines the viability of the testing strip. 
\begin{figure*}[!ht]
\centering
    \includegraphics[width=1\textwidth]{Images/cnn.jpg}
    \caption{Images A,B,D) Image captures from the optical microscopy setup identified by the model as inadequate for spectrophotometry. Sample A shows debris and color instability. Sample B has image blur and particulates. Sample D contains intact cells which act as light scatterers. Additionally, green boxes identify white blood cells from the sample, demonstrating expandability of model for a vast number of use cases. Image C) Clear view with color stability and no scatterers: ideal for photometric hemoglobin measurement. Data was collected using a labeling interface of all run test strips (bottom of E). 200 strip images were labeled by a human identifying sections of debris and cell boundaries, and the overall image labeled "good" vs. "bad".}
    \label{fig:CNN}
\end{figure*}
The model was adapted from LeNet structure, and trained on 200 human labeled images of good/bad test strips. The images were augmented to expand the training set by 8x (rotational, translational, brightness, and hue transformations). These results were then cross validated on a test set of 100 strip images. 

The binary classification model performed with 97\% accuracy on the task across the non-augmented test set - this included filled strips by clinicians and general users. Similarly, cell segmentation modules were developed to threshold cell boundaries with each candidate cell being trained in a separate Convolutional Neural Net for type classification (White Blood Cell, Red Blood Cell, etc.). 
\section{Conclusion}
Our platform builds off decades of hemoglobin spectroscopy research, while providing unprecedented error detection and reliability thanks to convolutional neural networks. We believe this deep learning approach to on-board quality control can be rapidly scaled to other applications, boosting clinical performance, end diagnoses, and patient safety. 

Given the low-cost, open microcontroller nature of the presented device, we see the platform and model as a powerful combination for deploying in decentralized or rural areas. Untrained test operators have been a major reason point-of-care, self-administered hematology tests have remained unfeasible. However, with robust error correction, the risk of misdiagnosis is greatly reduced and this radical model of personalized medicine is brought within the realm of possibility.

With high-precision analog sensing and CNN based image processing, we have developed the foundation for multi-wavelength spectrophotometry, enabling both structural and molecular sample analyses. We are convinced that the confluence of these distinct, rich sources of data with the speed and versatility of AI will present numerous opportunities to advance the future of healthcare.

\section{Acknowledgements}
We are sincerely thankful to Dhruv Parthasarathy, Deepika Bodapati, Louis Virey, Steve Moffatt, and Sreevaths Kasireddy for funding acquisition and technical input in various aspects of this project.

\medskip

\begin{thebibliography}{9}

\bibitem{mayoclinic}
“Low Hemoglobin Count Causes.” \textit{Mayo Clinic}, Mayo Foundation for Medical Education and Research, 8 May 2015, www.mayoclinic.org/symptoms/low-hemoglobin/basics/causes/sym-20050760.

\bibitem{medlineplus2}
Gersten, Todd. “Drug-Induced Immune Hemolytic Anemia.” \textit{MedlinePlus Medical Encyclopedia}, 1 Feb. 2017, medlineplus.gov/ency/article/000578.htm.

\bibitem{shah}
Shah, N, E A Osea, and G J Martinez. “Accuracy of Noninvasive Hemoglobin and Invasive Point-of-Care Hemoglobin Testing Compared with a Laboratory Analyzer.” \textit{International Journal of Laboratory Hematology}, 2014, pp. 56–61.

\bibitem{who}
WHO. The global prevalence of anaemia in 2011. Geneva: World Health Organization;
2015.

\bibitem{cdc}
“CDC Newsroom: Falls Are Leading Cause of Injury and Death in Older Americans.” \textit{Centers for Disease Control and Prevention}, Centers for Disease Control and Prevention, www.cdc.gov/media/releases/2016/p0922-older-adult-falls.html.

\bibitem{drugsaging}
Duh, MS. “Anaemia and the Risk of Injurious Falls in a Community-Dwelling Elderly Population.” \textit{Drugs Aging}, vol. 4, 2008, pp. 325–334. US National Library of Medicine

\bibitem{biomed}
Yang, Wenfang. “Anemia, Malnutrition and Their Correlations with Socio-Demographic Characteristics and Feeding Practices among Infants Aged 0–18 Months in Rural Areas of Shaanxi Province in Northwestern China: a Cross-Sectional Study.” \textit{BMC Public Health}, 29 Dec. 2012.

\bibitem{hemocue}
Williamsson, Anders. \textit{Capillary Microcuvette}. 7 Oct. 1997.

\bibitem{vanzetti}
Vanzetti G. "An azide-methemoglobin method for hemoglobin determination in blood." \textit{Journal of Laboratory and Clinical Medicine}, 1966, pp. 116-26

\bibitem{azide}
Chang, Soju and Lamm H Steven. "Human Health Effects of Sodium Azide Exposure: A Literature Review and Analysis" \textit{International Journal of Toxicology}, vol. 22, issue 3, 2016, pp. 175 - 186

\bibitem{oshiro}
Oshiro, Iwao. “New Method for Hemoglobin Determination by Using Sodium Lauryl Sulfate (SLS).” \textit{Clinical Biochemistry}, vol. 15, Apr. 1982, pp. 83–88.

\bibitem{coeff1}
N. Kollias, Wellman Laboratories, Harvard Medical School, Boston

\bibitem{coeff2}
W. B. Gratzer, Med. Res. Council Labs, Holly Hill, London

\bibitem{coeff3}
Benesch, Ruth E., et al. “Equations for the Spectrophotometric Analysis of Hemoglobin Mixtures.” \textit{Analytical Biochemistry}, vol. 55, no. 1, 16 May 1973, pp. 245–248., doi:10.1016/0003-2697(73)90309-6.

\bibitem{lipemia}
Creer, Michael H, and Jack Ladenson. “Analytical Errors Due to Lipemia.” \textit{Laboratory Medicine}, vol. 14, no. 6, June 1983, pp. 351–355.

\end{thebibliography}

\end{document}




% \bibliographystyle{plain}
% \bibliography{/Users/trukat/Dropbox/library/library}

\begin{thebibliography}{1}

\bibitem{Bhatia2015}
Kush Bhatia, Himanshu Jain, Purushottam Kar, Manik Varma, and Prateek Jain.
\newblock {Sparse local embeddings for extreme multi-label classification}.
\newblock {\em Advances in Neural Information Processing Systems (NIPS '15)},
  pages 730--738, 2015.

\bibitem{Jain2017}
Vikas Jain, Nirbhay Modhe, and Piyush Rai.
\newblock Scalable generative models for multi-label learning with missing
  labels.
\newblock In Doina Precup and Yee~Whye Teh, editors, {\em Proceedings of the
  34th International Conference on Machine Learning}, volume~70 of {\em
  Proceedings of Machine Learning Research}, pages 1636--1644, International
  Convention Centre, Sydney, Australia, 2017. PMLR.

\bibitem{Rukat2017}
Tammo Rukat, Chris~C. Holmes, Michalis~K. Titsias, and Christopher Yau.
\newblock {Bayesian Boolean Matrix Factorisation}.
\newblock {\em Proceedings of the 34th Annual International Conference on
  Machine Learning}, pages 2969--2978, jul 2017.

\end{thebibliography}



% References follow the acknowledgments. Use unnumbered first-level
% heading for the references. Any choice of citation style is acceptable
% as long as you are consistent. It is permissible to reduce the font
% size to \verb+small+ (9 point) when listing the references. {\bf
%   Remember that you can go over 8 pages as long as the subsequent ones contain
%   \emph{only} cited references.}
% \medskip

% \small

\end{document}
