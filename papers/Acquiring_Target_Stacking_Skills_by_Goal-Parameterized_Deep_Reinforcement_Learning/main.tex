\documentclass{article} % For LaTeX2e
\usepackage{iclr2018_conference,times}
\usepackage{hyperref}
\usepackage{url}
\usepackage{amsmath}
\usepackage{graphicx}
\usepackage{subcaption}
\usepackage{booktabs}
\usepackage{multirow}
\usepackage{array}
\newcolumntype{P}[1]{>{\centering\arraybackslash}p{#1}}
\newcommand{\ra}[1]{\renewcommand{\arraystretch}{#1}}

\title{Acquiring Target Stacking Skills by Goal-Parameterized Deep Reinforcement Learning}

% Authors must not appear in the submitted version. They should be hidden
% as long as the \iclrfinalcopy macro remains commented out below.
% Non-anonymous submissions will be rejected without review.

\author{Wenbin Li \\
Max Planck Institute for Informatics\\
Saarland Informatics Campus\\
Germany \\
\texttt{wenbinli@mpi-inf.mpg.de} \\
\And
Jeannette Bohg \\
Department of Computer Science \\
Stanford University \\
USA \\
\texttt{bohg@stanford.edu} \\
\AND
Mario Fritz \\
Max Planck Institute for Informatics \\
Saarland Informatics Campus\\
Germany \\
\texttt{mfritz@mpi-inf.mpg.de} \\
}

% The \author macro works with any number of authors. There are two commands
% used to separate the names and addresses of multiple authors: \And and \AND.
%
% Using \And between authors leaves it to \LaTeX{} to determine where to break
% the lines. Using \AND forces a linebreak at that point. So, if \LaTeX{}
% puts 3 of 4 authors names on the first line, and the last on the second
% line, try using \AND instead of \And before the third author name.

\newcommand{\fix}{\marginpar{FIX}}
\newcommand{\new}{\marginpar{NEW}}

\iclrfinalcopy % Uncomment for camera-ready version, but NOT for submission.

\begin{document}


\maketitle

\begin{abstract}
Understanding physical phenomena is a key component of human intelligence and enables physical interaction with previously unseen environments. In this paper, we study how an artificial agent can autonomously acquire this intuition through interaction with the environment. We created a synthetic block stacking environment with physics simulation in which the agent can learn a policy end-to-end through trial and error. Thereby, we bypass to explicitly model physical knowledge within the policy. We are specifically interested in tasks that require the agent to reach a given goal state that may be different for every new trial. To this end, we propose a deep reinforcement learning framework that learns policies which are parametrized by a goal. We validated the model on a toy example navigating in a grid world with different target positions and in a block stacking task with different target structures of the final tower. In contrast to prior work, our policies show better generalization across different goals.
\end{abstract}

% !TEX root = ../arxiv.tex

Unsupervised domain adaptation (UDA) is a variant of semi-supervised learning \cite{blum1998combining}, where the available unlabelled data comes from a different distribution than the annotated dataset \cite{Ben-DavidBCP06}.
A case in point is to exploit synthetic data, where annotation is more accessible compared to the costly labelling of real-world images \cite{RichterVRK16,RosSMVL16}.
Along with some success in addressing UDA for semantic segmentation \cite{TsaiHSS0C18,VuJBCP19,0001S20,ZouYKW18}, the developed methods are growing increasingly sophisticated and often combine style transfer networks, adversarial training or network ensembles \cite{KimB20a,LiYV19,TsaiSSC19,Yang_2020_ECCV}.
This increase in model complexity impedes reproducibility, potentially slowing further progress.

In this work, we propose a UDA framework reaching state-of-the-art segmentation accuracy (measured by the Intersection-over-Union, IoU) without incurring substantial training efforts.
Toward this goal, we adopt a simple semi-supervised approach, \emph{self-training} \cite{ChenWB11,lee2013pseudo,ZouYKW18}, used in recent works only in conjunction with adversarial training or network ensembles \cite{ChoiKK19,KimB20a,Mei_2020_ECCV,Wang_2020_ECCV,0001S20,Zheng_2020_IJCV,ZhengY20}.
By contrast, we use self-training \emph{standalone}.
Compared to previous self-training methods \cite{ChenLCCCZAS20,Li_2020_ECCV,subhani2020learning,ZouYKW18,ZouYLKW19}, our approach also sidesteps the inconvenience of multiple training rounds, as they often require expert intervention between consecutive rounds.
We train our model using co-evolving pseudo labels end-to-end without such need.

\begin{figure}[t]%
    \centering
    \def\svgwidth{\linewidth}
    \input{figures/preview/bars.pdf_tex}
    \caption{\textbf{Results preview.} Unlike much recent work that combines multiple training paradigms, such as adversarial training and style transfer, our approach retains the modest single-round training complexity of self-training, yet improves the state of the art for adapting semantic segmentation by a significant margin.}
    \label{fig:preview}
\end{figure}

Our method leverages the ubiquitous \emph{data augmentation} techniques from fully supervised learning \cite{deeplabv3plus2018,ZhaoSQWJ17}: photometric jitter, flipping and multi-scale cropping.
We enforce \emph{consistency} of the semantic maps produced by the model across these image perturbations.
The following assumption formalises the key premise:

\myparagraph{Assumption 1.}
Let $f: \mathcal{I} \rightarrow \mathcal{M}$ represent a pixelwise mapping from images $\mathcal{I}$ to semantic output $\mathcal{M}$.
Denote $\rho_{\bm{\epsilon}}: \mathcal{I} \rightarrow \mathcal{I}$ a photometric image transform and, similarly, $\tau_{\bm{\epsilon}'}: \mathcal{I} \rightarrow \mathcal{I}$ a spatial similarity transformation, where $\bm{\epsilon},\bm{\epsilon}'\sim p(\cdot)$ are control variables following some pre-defined density (\eg, $p \equiv \mathcal{N}(0, 1)$).
Then, for any image $I \in \mathcal{I}$, $f$ is \emph{invariant} under $\rho_{\bm{\epsilon}}$ and \emph{equivariant} under $\tau_{\bm{\epsilon}'}$, \ie~$f(\rho_{\bm{\epsilon}}(I)) = f(I)$ and $f(\tau_{\bm{\epsilon}'}(I)) = \tau_{\bm{\epsilon}'}(f(I))$.

\smallskip
\noindent Next, we introduce a training framework using a \emph{momentum network} -- a slowly advancing copy of the original model.
The momentum network provides stable, yet recent targets for model updates, as opposed to the fixed supervision in model distillation \cite{Chen0G18,Zheng_2020_IJCV,ZhengY20}.
We also re-visit the problem of long-tail recognition in the context of generating pseudo labels for self-supervision.
In particular, we maintain an \emph{exponentially moving class prior} used to discount the confidence thresholds for those classes with few samples and increase their relative contribution to the training loss.
Our framework is simple to train, adds moderate computational overhead compared to a fully supervised setup, yet sets a new state of the art on established benchmarks (\cf \cref{fig:preview}).

\section{Related Work}\label{sec:related}
 
The authors in \cite{humphreys2007noncontact} showed that it is possible to extract the PPG signal from the video using a complementary metal-oxide semiconductor camera by illuminating a region of tissue using through external light-emitting diodes at dual-wavelength (760nm and 880nm).  Further, the authors of  \cite{verkruysse2008remote} demonstrated that the PPG signal can be estimated by just using ambient light as a source of illumination along with a simple digital camera.  Further in \cite{poh2011advancements}, the PPG waveform was estimated from the videos recorded using a low-cost webcam. The red, green, and blue channels of the images were decomposed into independent sources using independent component analysis. One of the independent sources was selected to estimate PPG and further calculate HR, and HRV. All these works showed the possibility of extracting PPG signals from the videos and proved the similarity of this signal with the one obtained using a contact device. Further, the authors in \cite{10.1109/CVPR.2013.440} showed that heart rate can be extracted from features from the head as well by capturing the subtle head movements that happen due to blood flow.

%
The authors of \cite{kumar2015distanceppg} proposed a methodology that overcomes a challenge in extracting PPG for people with darker skin tones. The challenge due to slight movement and low lighting conditions during recording a video was also addressed. They implemented the method where PPG signal is extracted from different regions of the face and signal from each region is combined using their weighted average making weights different for different people depending on their skin color. 
%

There are other attempts where authors of \cite{6523142,6909939, 7410772, 7412627} have introduced different methodologies to make algorithms for estimating pulse rate robust to illumination variation and motion of the subjects. The paper \cite{6523142} introduces a chrominance-based method to reduce the effect of motion in estimating pulse rate. The authors of \cite{6909939} used a technique in which face tracking and normalized least square adaptive filtering is used to counter the effects of variations due to illumination and subject movement. 
The paper \cite{7410772} resolves the issue of subject movement by choosing the rectangular ROI's on the face relative to the facial landmarks and facial landmarks are tracked in the video using pose-free facial landmark fitting tracker discussed in \cite{yu2016face} followed by the removal of noise due to illumination to extract noise-free PPG signal for estimating pulse rate. 

Recently, the use of machine learning in the prediction of health parameters have gained attention. The paper \cite{osman2015supervised} used a supervised learning methodology to predict the pulse rate from the videos taken from any off-the-shelf camera. Their model showed the possibility of using machine learning methods to estimate the pulse rate. However, our method outperforms their results when the root mean squared error of the predicted pulse rate is compared. The authors in \cite{hsu2017deep} proposed a deep learning methodology to predict the pulse rate from the facial videos. The researchers trained a convolutional neural network (CNN) on the images generated using Short-Time Fourier Transform (STFT) applied on the R, G, \& B channels from the facial region of interests.
The authors of \cite{osman2015supervised, hsu2017deep} only predicted pulse rate, and we extended our work in predicting variance in the pulse rate measurements as well.

All the related work discussed above utilizes filtering and digital signal processing to extract PPG signals from the video which is further used to estimate the PR and PRV.  %
The method proposed in \cite{kumar2015distanceppg} is person dependent since the weights will be different for people with different skin tone. In contrast, we propose a deep learning model to predict the PR which is independent of the person who is being trained. Thus, the model would work even if there is no prior training model built for that individual and hence, making our model robust. 

%

\section{Learning from LDR panoramas}
\label{sec:learning}

Now that we have the tools to extract accurate lighting information from LDR panoramas, we detail our approach for learning the relationship between a single photo and its lighting conditions. 


\begin{table}
\caption[]{The proposed CNN architecture. After a series of 7 convolutional layers (conv), some with residual connections (res), a fully-connected layer (FC) segues to two heads. The heads aim at reconstructing the light mask $\mathbf{y}_\text{mask}$ (left) and the RGB panorama $\mathbf{y}_\text{RGB}$ (right) through a series of deconvolutional layers (deconv). The ELU activation function~\cite{clevert-iclr-16} and batch normalization are used on all layers except the outputs, which are sigmoids for light mask and tangent hyperbolic for panorama prediction. The stride at each layer is indicated between parentheses. The ``res'' identifiers indicate residual layers~\cite{he-cvpr-16}. }
\centering
\begin{tabular}{c}
\toprule
\textbf{Layer (stride)} \\
\midrule
Input \\
\midrule
conv9-64 (2) \\
conv4-96 (2) \\
res3-96  (1) \\
res4-128 (2) \\
res4-192 (2) \\
res4-256 (2) \\
\midrule
FC-1024 \\
\end{tabular}
\\
\begin{tabular}{cc}
\midrule
FC-8192 & FC-6144 \\
deconv4-256 (2) & deconv4-192 (2) \\
deconv4-128 (2) & deconv4-128 (2) \\
deconv4-96 (2) & deconv4-64 (2) \\
deconv4-64 (2) & deconv4-32 (2) \\
deconv4-32 (2) & deconv4-24 (2) \\
conv5-1 (1) & conv5-3 (1) \\
Sigmoid & Tanh \\*[-.5em]
\noindent\rule{3.2cm}{.8pt} &
\noindent\rule{3.2cm}{.8pt} \\
Output: light mask $\mathbf{y}_\text{mask}$ &
Output: RGB panorama $\mathbf{y}_\text{RGB}$ \\
\end{tabular}
\label{t:learning-architecture}
\end{table}

\subsection{Training data preparation}
\label{sec:ldr-data-prep}

For each SUN360 indoor panorama, we compute the light mask to represent ground truth during the learning process (sec.~\ref{sec:lightdetection}). We then take 8 crops from each panorama at random elevation between $-30^\circ$ and $30^\circ$ and make a projection of them as rectilinear photos. Using our recentering warp (sec.~\ref{sec:warping}), we generate a corresponding warped panorama (and light mask) for each rectilinear photo. We also rotate the warped panorama and corresponding light mask so that the crop region always sits at center azimuth (fig.~\ref{f:overview}).  At the end of this process, we have 96,000 input-output pairs, where the input is a photo, and the output is a pair of a warped panorama and its corresponding light mask.

\subsection{Network architecture} 

As shown in table~\ref{t:learning-architecture}, we use a convolutional neural network that takes the photo as input, produces a low-dimensional encoding of the input through a series of convolutions downstream and splits into two upstream expansions, with two distinct tasks: (1) intensity estimation / binary light mask prediction, and (2) RGB panorama prediction. The encoder is split into two standard convolution layers, followed by four residual layers~\cite{he-cvpr-16}. The two individual decoders are exclusively composed of deconvolution layers. The input photo is of size $256\times192$, whereas the panorama and light mask outputs are of size $256\times128$. Each time a stride of 2 is encountered with a convolution (deconvolution) layer, the resolution of its output feature map is divided (multiplied) by two. The output light mask $\mathbf{x}_\text{mask}$ represents the probability of light for each pixel in the environment map. The RGB panorama $\mathbf{x}_\text{mask}$ serves as a high level colored texture in the final environment map. Please see sec.~\ref{sec:experiments} for several examples of estimated light masks and RGB panoramas.


\begin{figure}
\centering
\footnotesize
\setlength{\tabcolsep}{1pt}
\begin{tabular}{ccc}
\includegraphics[width=0.32\linewidth]{images/cosfilter/base.png} &
\includegraphics[width=0.32\linewidth]{images/cosfilter/cos_1.png} &
\includegraphics[width=0.32\linewidth]{images/cosfilter/cos_5.png} \\
(a) Original & (b) $\alpha e=1$ & (c) $\alpha e=5$ \\
\includegraphics[width=0.32\linewidth]{images/cosfilter/cos_10.png} &
\includegraphics[width=0.32\linewidth]{images/cosfilter/cos_20.png} &
\includegraphics[width=0.32\linewidth]{images/cosfilter/cos_80.png} \\
(d) $\alpha e=10$ & (e) $\alpha e=20$ & (f) $\alpha e=80$ \\
\end{tabular}
\caption{Effect of the cosine blurring from eq.~(\ref{e:filter}) on the light mask at various blurring levels. Note how this simple, differentiable scheme allows a smooth progression towards higher frequency content over time, but without the ringing artifacts of spherical harmonics.}
\label{f:learning-filter}
\end{figure}

\subsection{Loss function} 

For the RGB panorama prediction task, we use an L2 distance on the pixel output:
\begin{equation}
    \mathcal{L}_\text{L2}(\mathbf{y}, \mathbf{t}) = \frac{1}{N}\sum_{i=1}^{N} \mathbf{s}_i (\mathbf{y}_i - \mathbf{t}_i)^2 \,,
\label{e:rgbloss}
\end{equation}
where $N=\mathtt{width}\times\mathtt{height}\times 3$ is the total number of elements in the image, $\mathbf{y}$ is the network prediction, $\mathbf{t}$ the ground truth panorama and $\mathbf{s}_i$ the solid angle for pixel $i$.

Designing the loss function for the light mask $\mathbf{y}_\text{mask}$ is not as straightforward. Take, for example, the standard L2 or binary cross entropy losses computed on the light mask directly. If a small bright spotlight is estimated to be located slightly off its ground truth location, a huge penalty will incur. Since pinpointing the exact location of all the light sources from a single photo is not necessary, we instead blur the target light mask with a filter and compute the L2 loss on the blurred version. The filter starts with a coarse, low-frequency representation of the target light mask and progressively sharpens it over training time. To this end, we design a filter based on the cosine distance, followed by an L2 loss for the light mask: 
\begin{equation}
    \mathcal{L}_\text{cos}(\mathbf{y}, \mathbf{t}, e) = \frac{1}{N}\sum_{i=1}^{N} (\mathcal{F}(\mathbf{y}, i, e) - \mathcal{F}(\mathbf{t}, i, e))^2 \,,
    \label{e:maskloss}
\end{equation}
where $e$ is a real value corresponding to the current epoch (formally, $e=\textrm{\#epochs}+\textrm{\#current-mini-batch}/\textrm{\#total-mini-batches}$.). The filter $\mathcal{F}$ is defined as:
\begin{equation}
    \mathcal{F}(\mathbf{p}, i, e) = \frac{1}{K_i} \sum_{\omega \in \Omega_i} \mathbf{p}(\omega) s(\omega) \left( \omega \cdot n_i \right)^{\alpha e},
    \label{e:filter}
\end{equation}
where $\Omega_i$ is the hemisphere centered at pixel $i$ on the panorama $\mathbf{p}$,  $n_i$ the unit normal at pixel $i$, and $K_i$ the sum of solid angles on $\Omega_i$. $\omega$ is a unit vector in a specific direction on $\Omega_i$ and $s(\omega)$ the solid angle for the pixel in the direction $\omega$. As seen before, we define $e\in\mathbb{R}$ as the real valued number of training samples collectively seen, normalized by the total number of training samples. 

\begin{figure}
\centering
\footnotesize
\setlength{\tabcolsep}{1pt}
\begin{tabular}{cc}
\includegraphics[width=0.493\linewidth]{images/lossLDR.pdf} &
\includegraphics[width=0.493\linewidth]{images/lossHDR.pdf} \\
(a) LDR network & (b) HDR network
\end{tabular}
\caption{Evolution of training and test loss over the number of epochs for the (a) LDR and (b) HDR training.}
\label{f:learning-curves}
\end{figure}

Since eq.~\ref{e:filter} is differentiable, back-propagation can be used to efficiently train our CNN. Fig.~\ref{f:learning-filter} shows a visual example of the effect of the cosine distance filter on a binary light mask. Note how the target light mask becomes progressively sharper over time. 

\begin{figure*}[!t]
\centering
\footnotesize
\setlength{\tabcolsep}{1pt}
\begin{tabular}{cccc}
\includegraphics[height=2.5cm]{{images/bunnyRenders/pano_ajwmyarsmgyaxz-others-90-1.25792-0.96443_render}.jpg} & 
\includegraphics[height=2.5cm]{{images/bunnyRenders/pano_ajwmyarsmgyaxz-others-90-1.25792-0.96443_mask}.jpg} & 
\hspace{.5em}
%\includegraphics[height=2.5cm]{{images/bunnyRenders/pano_abumqtqhptujdn-kitchen-135-1.06244-0.98464_render}.jpg} & 
%\includegraphics[height=2.5cm]{{images/bunnyRenders/pano_abumqtqhptujdn-kitchen-135-1.06244-0.98464_mask}.jpg} \\
\includegraphics[height=2.5cm]{{images/bunnyRenders/pano_agrayivbwqkxds-others-270-1.63912-0.96887_render}.jpg} & 
\includegraphics[height=2.5cm]{{images/bunnyRenders/pano_agrayivbwqkxds-others-270-1.63912-0.96887_mask}.jpg} \\
%
\includegraphics[height=2.5cm]{{images/bunnyRenders/pano_adghmppfkzisfi-others-135-1.50414-0.96319_render}.jpg} & 
\includegraphics[height=2.5cm]{{images/bunnyRenders/pano_adghmppfkzisfi-others-135-1.50414-0.96319_mask}.jpg} & 
\hspace{.5em}
\includegraphics[height=2.5cm]{{images/bunnyRenders/pano_ajxsprezaqhacq-restaurant-315-1.87811-1.08249_render}.jpg} & 
\includegraphics[height=2.5cm]{{images/bunnyRenders/pano_ajxsprezaqhacq-restaurant-315-1.87811-1.08249_mask}.jpg} \\
%\includegraphics[height=2.5cm]{{images/bunnyRenders/pano_aedlvoixsqeuog-others-315-1.05417-1.13734_render}.jpg} & 
%\includegraphics[height=2.5cm]{{images/bunnyRenders/pano_aedlvoixsqeuog-others-315-1.05417-1.13734_mask}.jpg} \\
%
\includegraphics[height=2.5cm]{{images/bunnyRenders/pano_akpwzsghfylcvp-museum-270-1.56694-0.96451_render}.jpg} & 
\includegraphics[height=2.5cm]{{images/bunnyRenders/pano_akpwzsghfylcvp-museum-270-1.56694-0.96451_mask}.jpg} & 
\hspace{.5em}
%\includegraphics[height=2.5cm]{{images/bunnyRenders/pano_akelvanxoywmql-workshop-180-1.44527-1.07197_render}.jpg} & 
%\includegraphics[height=2.5cm]{{images/bunnyRenders/pano_akelvanxoywmql-workshop-180-1.44527-1.07197_mask}.jpg} \\
\includegraphics[height=2.5cm]{{images/bunnyRenders/pano_ajzjecrfajjfdl-church-45-1.93004-1.12643_render}.jpg} & 
\includegraphics[height=2.5cm]{{images/bunnyRenders/pano_ajzjecrfajjfdl-church-45-1.93004-1.12643_mask}.jpg} \\
%
\includegraphics[height=2.5cm]{{images/bunnyRenders/pano_ahffjewqynvufc-others-180-1.62782-1.18096_render}.jpg} & 
\includegraphics[height=2.5cm]{{images/bunnyRenders/pano_ahffjewqynvufc-others-180-1.62782-1.18096_mask}.jpg} & 
\hspace{.5em}
\includegraphics[height=2.5cm]{{images/bunnyRenders/pano_alauchiodctyya-others-45-1.81179-1.15641_render}.jpg} & 
\includegraphics[height=2.5cm]{{images/bunnyRenders/pano_alauchiodctyya-others-45-1.81179-1.15641_mask}.jpg} \\
%
(a) Relit with estimate & (b) Predicted light probability & 
\hspace{.5em}
(c) Relit with estimate & (d) Predicted light probability
\end{tabular}
\caption{Evaluation of the LDR network at predicting light source positions. For each example, we show a virtual bunny model inserted in a background image and relit with the LDR network estimate for that image ((a) and (c)), and the predicted lighting probabilities overlaid on the panorama ((b) and (d)). As can been seen, our method generalizes to a wide range of indoor scenes and illumination conditions. Many more examples are available in the supplementary material.}
\label{f:relighting-bunnies}
\end{figure*}

The global loss function is then defined as:
%
\begin{equation}
    \mathcal{L}(\mathbf{y}, \mathbf{t}, e) = w_1 \mathcal{L}_\text{L2}(\mathbf{y}_\text{RGB}, \mathbf{t}_\text{RGB})  
    + w_2 \mathcal{L}_\text{cos}(\mathbf{y}_\text{mask}, \mathbf{t}_\text{mask}, e) \,.
\label{e:globloss}
\end{equation}
%
In our experiments, we use $w_1=100$, $w_2=1$, and $\alpha=3$.

Our filtering scheme also has a rendering-based interpretation. It is well known that surface reflection for Lambertian objects can be modeled as low-pass filtering~\cite{ramamoorthi-sig-01}, while specular objects preserve more high frequencies of the illumination. In this sense, our loss function can be thought of evaluating the inferred illumination in terms of the resulting \emph{appearance} of spheres with increasingly glossy surface reflectance. In this vein, we experimented with directly representing the illumination with spherical harmonics (gradually increasing the number of coefficients to represent higher frequencies of illumination), but found that the network had a tendency to overfit to the ringing artifacts caused by high frequencies in the binary light mask.

\subsection{Training details} 
\label{sec:training-details}

We use $85\%$ of the panoramas as training data, and $15\%$ as test data. Note that we generate the train-test split such that no crop of the test panoramas exist in the training set. Hence, all tests are performed for scenes and lighting conditions that have not been seen by the network before. We use the ADAM optimizer~\cite{kingma2014adam} with a minibatch size of $64$, learning rate of $0.005$, and momentum parameters of $\beta_1=0.9$, $\beta_2=0.999$. Fig.~\ref{f:learning-curves}-(a) shows the loss (from eq.~(\ref{e:globloss})) curves on the training and test set during training. Training takes roughly 40 hours on an Nvidia Titan X Pascal GPU. At test time, lighting inference (both mask and RGB) from a photo takes approximately 10ms. The batch size was selected so it fills the 12GB memory of the GPU.



\section{Experimental Evaluation}
\label{sec:experiment}
To demonstrate the viability of our modeling methodology, we show experimentally how through the deliberate combination and configuration of parallel FREEs, full control over 2DOF spacial forces can be achieved by using only the minimum combination of three FREEs.
To this end, we carefully chose the fiber angle $\Gamma$ of each of these actuators to achieve a well-balanced force zonotope (Fig.~\ref{fig:rigDiagram}).
We combined a contracting and counterclockwise twisting FREE with a fiber angle of $\Gamma = 48^\circ$, a contracting and clockwise twisting FREE with $\Gamma = -48^\circ$, and an extending FREE with $\Gamma = -85^\circ$.
All three FREEs were designed with a nominal radius of $R$ = \unit[5]{mm} and a length of $L$ = \unit[100]{mm}.
%
\begin{figure}
    \centering
    \includegraphics[width=0.75\linewidth]{figures/rigDiagram_wlabels10.pdf}
    \caption{In the experimental evaluation, we employed a parallel combination of three FREEs (top) to yield forces along and moments about the $z$-axis of an end effector.
    The FREEs were carefully selected to yield a well-balanced force zonotope (bottom) to gain full control authority over $F^{\hat{z}_e}$ and $M^{\hat{z}_e}$.
    To this end, we used one extending FREE, and two contracting FREEs which generate antagonistic moments about the end effector $z$-axis.}
    \label{fig:rigDiagram}
\end{figure}


\subsection{Experimental Setup}
To measure the forces generated by this actuator combination under a varying state $\vec{x}$ and pressure input $\vec{p}$, we developed a custom built test platform (Fig.~\ref{fig:rig}). 
%
\begin{figure}
    \centering
    \includegraphics[width=0.9\linewidth]{figures/photos/rig_labeled.pdf}
    \caption{\revcomment{1.3}{This experimental platform is used to generate a targeted displacement (extension and twist) of the end effector and to measure the forces and torques created by a parallel combination of three FREEs. A linear actuator and servomotor impose an extension and a twist, respectively, while the net force and moment generated by the FREEs is measured with a force load cell and moment load cell mounted in series.}}
    \label{fig:rig}
\end{figure}
%
In the test platform, a linear actuator (ServoCity HDA 6-50) and a rotational servomotor (Hitec HS-645mg) were used to impose a 2-dimensional displacement on the end effector. 
A force load cell (LoadStar  RAS1-25lb) and a moment load cell (LoadStar RST1-6Nm) measured the end-effector forces $F^{\hat{z_e}}$ and moments $M^{\hat{z_e}}$, respectively.
During the experiments, the pressures inside the FREEs were varied using pneumatic pressure regulators (Enfield TR-010-g10-s). 

The FREE attachment points (measured from the end effector origin) were measured to be:
\begin{align}
    \vec{d}_1 &= \bmx 0.013 & 0 & 0 \emx^T  \text{m}\\
    \vec{d}_2 &= \bmx -0.006 & 0.011 & 0 \emx^T  \text{m}\\
    \vec{d}_3 &= \bmx -0.006 & -0.011 & 0 \emx^T \text{m}
%    \vec{d}_i &= \bmx 0 & 0 & 0 \emx^T , && \text{for } i = 1,2,3
\end{align}
All three FREEs were oriented parallel to the end effector $z$-axis:
\begin{align}
    \hat{a}_i &= \bmx 0 & 0 & 1 \emx^T, \hspace{20pt} \text{for } i = 1,2,3
\end{align}
Based on this geometry, the transformation matrices $\bar{\mathcal{D}}_i$ were given by:
\begin{align}
    \bar{\mathcal{D}}_1 &= \bmx 0 & 0 & 1 & 0 & -0.013 & 0 \\ 0 & 0 & 0 & 0 & 0 & 1 \emx^T  \\
    \bar{\mathcal{D}}_2 &= \bmx 0 & 0 & 1 & 0.011 & 0.006 & 0 \\ 0 & 0 & 0 & 0 & 0 & 1 \emx^T  \\
    \bar{\mathcal{D}}_3 &= \bmx 0 & 0 & 1 & -0.011 & 0.006 & 0 \\ 0 & 0 & 0 & 0 & 0 & 1 \emx^T 
%    \bar{\mathcal{D}}_i &= \bmx 0 & 0 & 1 & 0 & 0 & 0 \\ 0 & 0 & 0 & 0 & 0 & 1 \emx^T , && \text{for } i = 1,2,3
\end{align}
These matrices were used in equation \eqref{eq:zeta} to yield the state-dependent fluid Jacobian $\bar{J}_x$ and to compute the resulting force zontopes.
%while using measured values of $\vec{\zeta}^{\,\text{meas}} (\vec{q}, \vec{P})$ and $\vec{\zeta}^{\,\text{meas}} (\vec{q}, 0)$ in \eqref{eq:fiberIso} yields the empirical measurements of the active force.



\subsection{Isolating the Active Force}
To compare our model force predictions (which focus only on the active forces induced by the fibers)
to those measured empirically on a physical system, we had to remove the elastic force components attributed to the elastomer. 
Under the assumption that the elastomer force is merely a function of the displacement $\vec{x}$ and independent of pressure $\vec{p}$ \cite{bruder2017model}, this force component can be approximated by the measured force at a pressure of $\vec{p}=0$. 
That is: 
\begin{align}
    \vec{f}_{\text{elast}} (\vec{x}) = \vec{f}_{\text{\,meas}} (\vec{x}, 0)
\end{align}
With this, the active generalized forces were measured indirectly by subtracting off the force generated at zero pressure:
\begin{align}
    \vec{f} (\vec{x}, \vec{p})  &= \vec{f}_{\text{meas}} (\vec{x}, \vec{p}) - \vec{f}_{\text{meas}} (\vec{x}, 0)     \label{eq:fiberIso}
\end{align}


%To validate our parallel force model, we compare its force predictions, $\vec{\zeta}_{\text{pred}}$, to those measured empirically on a physical system, $\vec{\zeta}_\text{meas}$. 
%From \eqref{eq:Z} and \eqref{eq:zeta}, the force at the end effector is given by:
%\begin{align}
%    \vec{\zeta}(\vec{q}, \vec{P}) &= \sum_{i=1}^n \bar{\mathcal{D}}_i \left( {\bar{J}_V}_i^T(\vec{q_i}) P_i + \vec{Z}_i^{\text{elast}} (\vec{q_i}) \right) \\
%    &= \underbrace{\sum_{i=1}^n \bar{\mathcal{D}}_i {\bar{J}_V}_i^T(\vec{q_i}) P_i}_{\vec{\zeta}^{\,\text{fiber}} (\vec{q}, \vec{P})} + \underbrace{\sum_{i=1}^n \bar{\mathcal{D}}_i \vec{Z}_i^{\text{elast}} (\vec{q_i})}_{\vec{\zeta}^{\text{elast}} (\vec{q})}   \label{eq:zetaSplit}
%     &= \vec{\zeta}^{\,\text{fiber}} (\vec{q}, \vec{P}) + \vec{\zeta}^{\text{elast}} (\vec{q})
%\end{align}
%\Dan{These will need to reflect changes made to previous section.}
%The model presented in this paper does not specify the elastomer forces, $\vec{\zeta}^{\text{elast}}$, therefore we only validate its predictions %of the fiber forces, $\vec{\zeta}^{\,\text{fiber}}$. 
%We isolate the fiber forces by noting that $\vec{\zeta}^{\text{elast}} (\vec{q}) = \vec{\zeta}(\vec{q}, 0)$ and rearranging \eqref{eq:zetaSplit}
%\begin{align}
%    \vec{\zeta}^{\,\text{fiber}} (\vec{q}, \vec{P})  &= \vec{\zeta}(\vec{q}, \vec{P}) - \vec{\zeta}(\vec{q}, 0)     \label{eq:fiberIso}
%%    \vec{\zeta}^{\,\text{fiber}}_{\text{emp}} (\vec{q}, \vec{P})  &= \vec{\zeta}_{\text{emp}}(\vec{q}, \vec{P}) - %\vec{\zeta}_{\text{emp}}(\vec{q}, 0)
%\end{align}
%Thus we measure the fiber forces indirectly by subtracting off the forces generated at zero pressure.  


\subsection{Experimental Protocol}
The force and moment generated by the parallel combination of FREEs about the end effector $z$-axis  was measured in four different geometric configurations: neutral, extended, twisted, and simultaneously extended and twisted (see Table \ref{table:RMSE} for the exact deformation amounts). 
At each of these configurations, the forces were measured at all pressure combinations in the set
\begin{align}
    \mathcal{P} &= \left\{ \bmx \alpha_1 & \alpha_2 & \alpha_3 \emx^T p^{\text{max}} \, : \, \alpha_i = \left\{ 0, \frac{1}{4}, \frac{1}{2}, \frac{3}{4}, 1 \right\} \right\}
\end{align}
with $p^{\text{max}}$ = \unit[103.4]{kPa}. 
\revcomment{3.2}{The experiment was performed twice using two different sets of FREEs to observe how fabrication variability might affect performance. The results from Trial 1 are displayed in Fig.~\ref{fig:results} and the error for both trials is given in Table \ref{table:RMSE}.}



\subsection{Results}

\begin{figure*}[ht]
\centering

\def\picScale{0.08}    % define variable for scaling all pictures evenly
\def\plotScale{0.2}    % define variable for scaling all plots evenly
\def\colWidth{0.22\linewidth}

\begin{tikzpicture} %[every node/.style={draw=black}]
% \draw[help lines] (0,0) grid (4,2);
\matrix [row sep=0cm, column sep=0cm, style={align=center}] (my matrix) at (0,0) %(2,1)
{
& \node (q1) {(a) $\Delta l = 0, \Delta \phi = 0$}; & \node (q2) {(b) $\Delta l = 5\text{mm}, \Delta \phi = 0$}; & \node (q3) {(c) $\Delta l = 0, \Delta \phi = 20^\circ$}; & \node (q4) {(d) $\Delta l = 5\text{mm}, \Delta \phi = 20^\circ$};

\\

&
\node[style={anchor=center}] {\includegraphics[width=\colWidth]{figures/photos/s0w0pic_colored.pdf}}; %\fill[blue] (0,0) circle (2pt);
&
\node[style={anchor=center}] {\includegraphics[width=\colWidth]{figures/photos/s5w0pic_colored.pdf}}; %\fill[blue] (0,0) circle (2pt);
&
\node[style={anchor=center}] {\includegraphics[width=\colWidth]{figures/photos/s0w20pic_colored.pdf}}; %\fill[blue] (0,0) circle (2pt);
&
\node[style={anchor=center}] {\includegraphics[width=\colWidth]{figures/photos/s5w20pic_colored.pdf}}; %\fill[blue] (0,0) circle (2pt);

\\

\node[rotate=90] (ylabel) {Moment, $M^{\hat{z}_e}$ (N-m)};
&
\node[style={anchor=center}] {\includegraphics[width=\colWidth]{figures/plots3/s0w0.pdf}}; %\fill[blue] (0,0) circle (2pt);
&
\node[style={anchor=center}] {\includegraphics[width=\colWidth]{figures/plots3/s5w0.pdf}}; %\fill[blue] (0,0) circle (2pt);
&
\node[style={anchor=center}] {\includegraphics[width=\colWidth]{figures/plots3/s0w20.pdf}}; %\fill[blue] (0,0) circle (2pt);
&
\node[style={anchor=center}] {\includegraphics[width=\colWidth]{figures/plots3/s5w20.pdf}}; %\fill[blue] (0,0) circle (2pt);

\\

& \node (xlabel1) {Force, $F^{\hat{z}_e}$ (N)}; & \node (xlabel2) {Force, $F^{\hat{z}_e}$ (N)}; & \node (xlabel3) {Force, $F^{\hat{z}_e}$ (N)}; & \node (xlabel4) {Force, $F^{\hat{z}_e}$ (N)};

\\
};
\end{tikzpicture}

\caption{For four different deformed configurations (top row), we compare the predicted and the measured forces for the parallel combination of three FREEs (bottom row). 
\revcomment{2.6}{Data points and predictions corresponding to the same input pressures are connected by a thin line, and the convex hull of the measured data points is outlined in black.}
The Trial 1 data is overlaid on top of the theoretical force zonotopes (grey areas) for each of the four configurations.
Identical colors indicate correspondence between a FREE and its resulting force/torque direction.}
\label{fig:results}
\end{figure*}






% & \node (a) {(a)}; & \node (b) {(b)}; & \node (c) {(c)}; & \node (d) {(d)};


For comparison, the measured forces are superimposed over the force zonotope generated by our model in Fig.~\ref{fig:results}a-~\ref{fig:results}d.
To quantify the accuracy of the model, we defined the error at the $j^{th}$ evaluation point as the difference between the modeled and measured forces
\begin{align}
%    \vec{e}_j &= \left( {\vec{\zeta}_{\,\text{mod}}} - {\vec{\zeta}_{\,\text{emp}}} \right)_j
%    e_j &= \left( F/M_{\,\text{mod}} - F/M_{\,\text{emp}} \right)_j
    e^F_j &= \left( F^{\hat{z}_e}_{\text{pred}, j} - F^{\hat{z}_e}_{\text{meas}, j} \right) \\
    e^M_j &= \left( M^{\hat{z}_e}_{\text{pred}, j} - M^{\hat{z}_e}_{\text{meas}, j} \right)
\end{align}
and evaluated the error across all $N = 125$ trials of a given end effector configuration.
% using the following metrics:
% \begin{align}
%     \text{RMSE} &= \sqrt{ \frac{\sum_{j=1}^{N} e_j^2}{N} } \\
%     \text{Max Error} &= \max \{ \left| e_j \right| : j = 1, ... , N \}
% \end{align}
As shown in Table \ref{table:RMSE}, the root-mean-square error (RMSE) is less than \unit[1.5]{N} (\unit[${8 \times 10^{-3}}$]{Nm}), and the maximum error is less than \unit[3]{N}  (\unit[${19 \times 10^{-3}}$]{Nm}) across all trials and configurations.

\begin{table}[H]
\centering
\caption{Root-mean-square error and maximum error}
\begin{tabular}{| c | c || c | c | c | c|}
    \hline
     & \rule{0pt}{2ex} \textbf{Disp.} & \multicolumn{2}{c |}{\textbf{RMSE}} & \multicolumn{2}{c |}{\textbf{Max Error}} \\ 
     \cline{2-6}
     & \rule{0pt}{2ex} (mm, $^\circ$) & F (N) & M (Nm) & F (N) & M (Nm) \\
     \hline
     \multirow{4}{*}{\rotatebox[origin=c]{90}{\textbf{Trial 1}}}
     & 0, 0 & 1.13 & $3.8 \times 10^{-3}$ & 2.96 & $7.8 \times 10^{-3}$ \\
     & 5, 0 & 0.74 & $3.2 \times 10^{-3}$ & 2.31 & $7.4 \times 10^{-3}$ \\
     & 0, 20 & 1.47 & $6.3 \times 10^{-3}$ & 2.52 & $15.6 \times 10^{-3}$\\
     & 5, 20 & 1.18 & $4.6 \times 10^{-3}$ & 2.85 & $12.4 \times 10^{-3}$ \\  
     \hline
     \multirow{4}{*}{\rotatebox[origin=c]{90}{\textbf{Trial 2}}}
     & 0, 0 & 0.93 & $6.0 \times 10^{-3}$ & 1.90 & $13.3 \times 10^{-3}$ \\
     & 5, 0 & 1.00 & $7.7 \times 10^{-3}$ & 2.97 & $19.0 \times 10^{-3}$ \\
     & 0, 20 & 0.77 & $6.9 \times 10^{-3}$ & 2.89 & $15.7 \times 10^{-3}$\\
     & 5, 20 & 0.95 & $5.3 \times 10^{-3}$ & 2.22 & $13.3 \times 10^{-3}$ \\  
     \hline
\end{tabular}
\label{table:RMSE}
\end{table}

\begin{figure}
    \centering
    \includegraphics[width=\linewidth]{figures/photos/buckling.pdf}
    \caption{At high fluid pressure the FREE with fiber angle of $-85^\circ$ started to buckle.  This effect was less pronounced when the system was extended along the $z$-axis.}
    \label{fig:buckling}
\end{figure}

%Experimental precision was limited by unmodeled material defects in the FREEs, as well as sensor inaccuracy. While the commercial force and moment sensors used have a quoted accuracy of 0.02\% for the force sensor and 0.2\% for the moment sensor (LoadStar Sensors, 2015), a drifting of up to 0.5 N away from zero was noticed on the force sensor during testing.

It should be noted, that throughout the experiments, the FREE with a fiber angle of $-85^\circ$ exhibited noticeable buckling behavior at pressures above $\approx$ \unit[50]{kPa} (Fig.~\ref{fig:buckling}). 
This behavior was more pronounced during testing in the non-extended configurations (Fig.~\ref{fig:results}a~and~\ref{fig:results}c). 
The buckling might explain the noticeable leftward offset of many of the points in Fig.~\ref{fig:results}a and Fig.~\ref{fig:results}c, since it is reasonable to assume that buckling reduces the efficacy of of the FREE to exert force in the direction normal to the force sensor. 

\begin{figure}
    \centering
    \includegraphics[width=\linewidth]{figures/zntp_vs_x4.pdf}
    \caption{A visualization of how the \emph{force zonotope} of the parallel combination of three FREEs (see Fig.~\ref{fig:rig}) changes as a function of the end effector state $x$. One can observe that the change in the zonotope ultimately limits the work-space of such a system.  In particular the zonotope will collapse for compressions of more than \unit[-10]{mm}.  For \revcomment{2.5}{scale and comparison, the convex hulls of the measured points from Fig.~\ref{fig:results}} are superimposed over their corresponding zonotope at the configurations that were evaluated experimentally.}
    % \marginnote{\#2.5}
    \label{fig:zntp_vs_x}
\end{figure}
% \vspace{-0.5em}
\section{Conclusion}
% \vspace{-0.5em}
Recent advances in multimodal single-cell technology have enabled the simultaneous profiling of the transcriptome alongside other cellular modalities, leading to an increase in the availability of multimodal single-cell data. In this paper, we present \method{}, a multimodal transformer model for single-cell surface protein abundance from gene expression measurements. We combined the data with prior biological interaction knowledge from the STRING database into a richly connected heterogeneous graph and leveraged the transformer architectures to learn an accurate mapping between gene expression and surface protein abundance. Remarkably, \method{} achieves superior and more stable performance than other baselines on both 2021 and 2022 NeurIPS single-cell datasets.

\noindent\textbf{Future Work.}
% Our work is an extension of the model we implemented in the NeurIPS 2022 competition. 
Our framework of multimodal transformers with the cross-modality heterogeneous graph goes far beyond the specific downstream task of modality prediction, and there are lots of potentials to be further explored. Our graph contains three types of nodes. While the cell embeddings are used for predictions, the remaining protein embeddings and gene embeddings may be further interpreted for other tasks. The similarities between proteins may show data-specific protein-protein relationships, while the attention matrix of the gene transformer may help to identify marker genes of each cell type. Additionally, we may achieve gene interaction prediction using the attention mechanism.
% under adequate regulations. 
% We expect \method{} to be capable of much more than just modality prediction. Note that currently, we fuse information from different transformers with message-passing GNNs. 
To extend more on transformers, a potential next step is implementing cross-attention cross-modalities. Ideally, all three types of nodes, namely genes, proteins, and cells, would be jointly modeled using a large transformer that includes specific regulations for each modality. 

% insight of protein and gene embedding (diff task)

% all in one transformer

% \noindent\textbf{Limitations and future work}
% Despite the noticeable performance improvement by utilizing transformers with the cross-modality heterogeneous graph, there are still bottlenecks in the current settings. To begin with, we noticed that the performance variations of all methods are consistently higher in the ``CITE'' dataset compared to the ``GEX2ADT'' dataset. We hypothesized that the increased variability in ``CITE'' was due to both less number of training samples (43k vs. 66k cells) and a significantly more number of testing samples used (28k vs. 1k cells). One straightforward solution to alleviate the high variation issue is to include more training samples, which is not always possible given the training data availability. Nevertheless, publicly available single-cell datasets have been accumulated over the past decades and are still being collected on an ever-increasing scale. Taking advantage of these large-scale atlases is the key to a more stable and well-performing model, as some of the intra-cell variations could be common across different datasets. For example, reference-based methods are commonly used to identify the cell identity of a single cell, or cell-type compositions of a mixture of cells. (other examples for pretrained, e.g., scbert)


%\noindent\textbf{Future work.}
% Our work is an extension of the model we implemented in the NeurIPS 2022 competition. Now our framework of multimodal transformers with the cross-modality heterogeneous graph goes far beyond the specific downstream task of modality prediction, and there are lots of potentials to be further explored. Our graph contains three types of nodes. while the cell embeddings are used for predictions, the remaining protein embeddings and gene embeddings may be further interpreted for other tasks. The similarities between proteins may show data-specific protein-protein relationships, while the attention matrix of the gene transformer may help to identify marker genes of each cell type. Additionally, we may achieve gene interaction prediction using the attention mechanism under adequate regulations. We expect \method{} to be capable of much more than just modality prediction. Note that currently, we fuse information from different transformers with message-passing GNNs. To extend more on transformers, a potential next step is implementing cross-attention cross-modalities. Ideally, all three types of nodes, namely genes, proteins, and cells, would be jointly modeled using a large transformer that includes specific regulations for each modality. The self-attention within each modality would reconstruct the prior interaction network, while the cross-attention between modalities would be supervised by the data observations. Then, The attention matrix will provide insights into all the internal interactions and cross-relationships. With the linearized transformer, this idea would be both practical and versatile.

% \begin{acks}
% This research is supported by the National Science Foundation (NSF) and Johnson \& Johnson.
% \end{acks}

%\subsubsection*{Acknowledgments}
%
%Use unnumbered third level headings for the acknowledgments. All
%acknowledgments, including those to funding agencies, go at the end of the paper.

\bibliography{mybib}
\bibliographystyle{iclr2018_conference}

\end{document}
