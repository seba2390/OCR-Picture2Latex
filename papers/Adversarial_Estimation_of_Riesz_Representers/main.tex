\documentclass{article}

\usepackage[T1]{fontenc}
\usepackage{inputenc}
\usepackage{geometry}
%
\setlength{\parskip}{\medskipamount}
\setlength{\parindent}{0pt}
\usepackage{float}
\usepackage{ifthen}
\usepackage{amsbsy}
\usepackage{amssymb}
\usepackage{amsmath}
\usepackage{amsthm}
\usepackage{graphicx}
\usepackage{setspace}
\usepackage{esint}
\usepackage{comment}
%
\usepackage{mathtools}
\usepackage{xcolor}
%




%
\DeclarePairedDelimiter{\abs}{\lvert}{\rvert} %
\DeclarePairedDelimiter{\brk}{[}{]}
\DeclarePairedDelimiter{\crl}{\{}{\}}
\DeclarePairedDelimiter{\prn}{(}{)}
\DeclarePairedDelimiter{\nrm}{\|}{\|}
\DeclarePairedDelimiter{\tri}{\langle}{\rangle}
\DeclarePairedDelimiter{\dtri}{\llangle}{\rrangle}

\DeclarePairedDelimiter{\ceil}{\lceil}{\rceil}
\DeclarePairedDelimiter{\floor}{\lfloor}{\rfloor}


%
\let\Pr\undefined
\let\P\undefined
\DeclareMathOperator{\En}{\mathbb{E}}
\DeclareMathOperator*{\Eh}{\widehat{\mathbb{E}}}
\DeclareMathOperator{\P}{P}
\DeclareMathOperator{\Pr}{Pr}

%
\DeclareMathOperator*{\argmin}{arg\,min} %
\DeclareMathOperator*{\argmax}{arg\,max}             
\DeclareMathOperator*{\arginf}{arg\,inf} 
\DeclareMathOperator*{\argsup}{arg\,sup} 

\DeclareMathOperator*{\smax}{smax_{\eta}}
\DeclareMathOperator*{\smin}{smin_{\eta}}

%
\newcommand{\ls}{\ell}
\newcommand{\ind}{\mathbbm{1}}    %
\newcommand{\pmo}{\crl*{\pm{}1}}
\newcommand{\eps}{\epsilon}
\newcommand{\veps}{\varepsilon}

\newcommand{\ldef}{\vcentcolon=}
\newcommand{\rdef}{=\vcentcolon}

%
\newcommand{\mc}[1]{\mathcal{#1}}
\newcommand{\wt}[1]{\widetilde{#1}}
\newcommand{\wh}[1]{\hat{#1}}


%
%
\def\ddefloop#1{\ifx\ddefloop#1\else\ddef{#1}\expandafter\ddefloop\fi}
\def\ddef#1{\expandafter\def\csname bb#1\endcsname{\ensuremath{\mathbb{#1}}}}
\ddefloop ABCDEFGHIJKLMNOPQRSTUVWXYZ\ddefloop
\def\ddefloop#1{\ifx\ddefloop#1\else\ddef{#1}\expandafter\ddefloop\fi}
\def\ddef#1{\expandafter\def\csname b#1\endcsname{\ensuremath{\mathbf{#1}}}}
\ddefloop ABCDEFGHIJKLMNOPQRSTUVWXYZ\ddefloop
\def\ddef#1{\expandafter\def\csname c#1\endcsname{\ensuremath{\mathcal{#1}}}}
\ddefloop ABCDEFGHIJKLMNOPQRSTUVWXYZ\ddefloop
\def\ddef#1{\expandafter\def\csname h#1\endcsname{\ensuremath{\widehat{#1}}}}
\ddefloop ABCDEFGHIJKLMNOPQRSTUVWXYZ\ddefloop
\def\ddef#1{\expandafter\def\csname hc#1\endcsname{\ensuremath{\widehat{\mathcal{#1}}}}}
\ddefloop ABCDEFGHIJKLMNOPQRSTUVWXYZ\ddefloop
\def\ddef#1{\expandafter\def\csname t#1\endcsname{\ensuremath{\widetilde{#1}}}}
\ddefloop ABCDEFGHIJKLMNOPQRSTUVWXYZ\ddefloop
\def\ddef#1{\expandafter\def\csname tc#1\endcsname{\ensuremath{\widetilde{\mathcal{#1}}}}}
\ddefloop ABCDEFGHIJKLMNOPQRSTUVWXYZ\ddefloop

%
\newcommand{\Holder}{H{\"o}lder}

%

\newcommand{\smooth}[1][\beta]{\mathrm{smooth}_{#1}}

\newcommand{\Ot}{\wt{O}}

\newcommand{\kl}{\mathrm{KL}}

\newcommand{\high}{\mb{h}}
\newcommand{\low}{\mb{l}}
\newcommand{\normalized}{\mb{z}}
\newcommand{\zeros}{\mb{0}}
\usepackage{prettyref}
\newcommand{\pref}[1]{\prettyref{#1}}
\newcommand{\vphi}{\varphi}

\newcommand{\scores}{\cS}
\newcommand{\bls}{\mb{\ls}}

\newcommand{\bz}{\mb{z}}
\newcommand{\bx}{\mb{x}}
\newcommand{\by}{\mb{y}}
\newcommand{\bv}{\mb{v}}
\newcommand{\bw}{\mb{w}}
\newcommand{\bu}{\mb{u}}

\newcommand{\scoredist}{\mathbf{p}}

\newcommand{\radius}{\mathrm{rad}}

\newcommand{\midsem}{\,;\,}

\newcommand{\emprisk}{\wh{L}_n}
\newcommand{\poprisk}{L_{\cD}}
\usepackage{nicefrac}
%
\usepackage{subcaption}


\newcommand{\shull}{\ensuremath{\text{star}}}

\makeatletter
\newtheorem*{rep@theorem}{\rep@title}
\newcommand{\newreptheorem}[2]{%
\newenvironment{rep#1}[1]{%
 \def\rep@title{#2 \ref{##1}}%
 \begin{rep@theorem}}%
 {\end{rep@theorem}}}
\makeatother


\usepackage{url}
\usepackage[breaklinks]{hyperref}
\usepackage{cleveref}
\newtheorem{theorem}{Theorem}
\newreptheorem{theorem}{Theorem}
\newtheorem{corollary}[theorem]{Corollary}
\newreptheorem{corollary}{Corollary}
\newtheorem{lemma}[theorem]{Lemma}
\newtheorem{proposition}[theorem]{Proposition}
\newtheorem{definition}{Definition}
\newtheorem{example}{Example}
\newtheorem{remark}{Remark}
\newcommand{\creflastconjunction}{, and\nobreakspace}
\crefname{equation}{}{}
\crefname{proposition}{Proposition}{Propositions}
\crefname{appendix}{Appendix}{Appendices}
\usepackage{autonum}

%
%
%
%
%
%


\newcommand{\kibitz}[2]{\ifnum\Comments=1{\color{#1}{#2}}\fi}
\newcommand{\zf}[1]{\kibitz{amber}{[ZF: #1]}}
\newcommand{\cp}[1]{\kibitz{red}{[CP: #1]}}
\newcommand{\vs}[1]{\kibitz{blue}{[VS: #1]}}
\newcommand{\todo}[1]{\kibitz{blue}{[TODO: #1]}}
\newcommand{\E}{\mathbb{E}}
%
%
%
%
\newcommand{\1}{\mathbbm{1}}
\newcommand{\st}{\sum{t=1}^{T}}
%
\newcommand{\G}{\mathcal{G}^\eps}
\newcommand{\kk}{\textbf{k}}
\newcommand{\R}{\mathbb{R}}
\newcommand{\B}{\mathcal{B}}
\newcommand{\D}{\mathcal D}
\newcommand{\mbf}{\mathbf}
\newcommand{\p}{\mathbf{p}}
\newcommand{\s}{\mathbf{s}}
\newcommand{\g}{\textbf{g}}
\newcommand{\softmax}{\textbf{softmax}}
\newcommand{\w}{\mbf{w}}
\newcommand{\winexp}{\textsc{WIN-EXP}}
\newcommand{\winexpG}{\textsc{WIN-EXP-G}}
\renewcommand{\Pr}{\ensuremath{\mathrm{Pr}}}
\newcommand{\opt}{\ensuremath{\textsc{OPT}}}
%


\newcommand{\tto}{,\ldots,}
\newcommand{\eg}{e.g., \xspace}
\newcommand{\ie}{i.e.,\xspace}
\newcommand{\etc}{etc.\@\xspace}
\newcommand{\mcR}{{\mathcal R}}
\newcommand{\mcD}{{\mathcal D}}
\newcommand{\mcA}{{\mathcal A}}
\newcommand{\mcH}{{\mathcal H}}
\newcommand{\Hnorm}[1]{\norm{#1}_{\mcH}}
\newcommand{\mcM}{{\mathcal M}}
\newcommand{\mcN}{{\mathcal N}}
\newcommand{\mcP}{{\mathcal P}}
\newcommand{\mbP}{{\mathbb P}}
\newcommand{\mbG}{{\mathbb G}}
\newcommand{\mcQ}{{\mathcal Q}}
\newcommand{\mcX}{{\mathcal X}}
\newcommand{\iid}{\emph{i.i.d.}\ }
\newcommand{\mcG}{{\mathcal G}}
\newcommand{\mcF}{{\mathcal F}}
\newcommand{\Fnorm}[1]{\norm{#1}_{\mcF}}
\newcommand{\mcT}{{\mathcal T}}
\newcommand{\mcY}{{\mathcal Y}}
\newcommand{\mcV}{{\mathcal V}}
\newcommand{\mcW}{{\mathcal W}}
\newcommand{\etal}{et.\ al.\ }
\newcommand{\natur}{{\mathbb N}}
\newcommand{\ba}{\begin{array}}
\newcommand{\ea}{\end{array}}
\newcommand{\bs}{\begin{align}\begin{split}\nonumber}
\newcommand{\bsnumber}{\begin{align}\begin{split}}
\newcommand{\es}{\end{split}\end{align}}
%
%
\newcommand{\fns}{\singlespace\small}
\newcommand{\expect}{\mathcal{E}}
\newcommand{\cooldate}{\begin{flushright}\footnotesize\today\normalsize\end{flushright}}
\renewcommand{\(}{\left(}
\newcommand{\real}{{\mathbb R}}
\newcommand{\mcZ}{{\mathcal Z}}
\newcommand{\mcL}{{\mathcal L}}
\newcommand{\M}{{\bf M}}
\newcommand{\bps}{{\bf \psi}}
\newcommand{\brho}{{\bf \rho}}
\renewcommand{\)}{\right)}
\renewcommand{\[}{\left[}
\renewcommand{\]}{\right]}
\newtheorem{prop}{Proposition}
\newtheorem{assumption}{ASSUMPTION}
\newcommand{\x}{\mathbf{x}}
\newcommand{\xstar}{\mathbf{x}^*}
\newcommand{\bff}{\mathbf}
\newcommand{\aaa}{{\bf a}}

\newcommand{\spanF}{\ensuremath{\text{span}}}

\newcommand{\breg}{{\mcD}}
\newcommand{\Lp}{L_{1,p}^{\epsilon_n}}

\newcommand{\ldot}[2]{\langle #1, #2 \rangle}

\newcommand{\Var}{\ensuremath{\text{Var}}}
\newcommand{\MSE}{\ensuremath{\text{MSE}}}
\newcommand{\sign}{\ensuremath{\mathtt{sign}}}

\newcommand{\textfrac}[2]{{\textstyle\frac{#1}{#2}}}
\def\half{\frac{1}{2}}
\def\texthalf{{\textstyle\frac{1}{2}}}
\newcommand{\norm}[1]{\|{#1}\|} %
\newcommand{\Knorm}[1]{\norm{#1}_K} %
\newcommand{\KH}{K_\mcH}
\newcommand{\KHn}{K_{\mcH,n}}
\newcommand{\KF}{K_{\mcF}}
\newcommand{\KFn}{K_{\mcH,n}}
\newcommand{\KHnorm}[1]{\norm{#1}_{\KH}} %
\newcommand{\KFnorm}[1]{\norm{#1}_{\KF}} %
\newcommand{\onenorm}[1]{\norm{#1}_1} %
\newcommand{\twonorm}[1]{\norm{#1}_2} %
\newcommand{\infnorm}[1]{\norm{#1}_{\infty}} %
\newcommand{\opnorm}[1]{\norm{#1}_{op}} %
\newcommand{\fronorm}[1]{\norm{#1}_{F}} %
\newcommand{\inner}[2]{\langle{#1},{#2}\rangle} %
\newcommand{\KHinner}[2]{\inner{#1}{#2}_{\KH}} %
\newcommand{\KFinner}[2]{\inner{#1}{#2}_{\KF}} %
\newcommand{\binner}[2]{\left\langle{#1},{#2}\right\rangle} %
\def\supp#1{\mathrm{supp}({#1})}
\def\defeq{\triangleq} %
\newcommand{\qtext}[1]{\quad\text{#1}\quad} 
\def\balign#1\ealign{\begin{align}#1\end{align}}
\def\balignat#1\ealign{\begin{alignat}#1\end{alignat}}
\def\bitemize#1\eitemize{\begin{itemize}#1\end{itemize}}
\def\benumerate#1\eenumerate{\begin{enumerate}#1\end{enumerate}}
\newenvironment{talign}
 {\let\displaystyle\textstyle\csname align\endcsname}
 {\endalign}
\def\balignt#1\ealignt{\begin{talign}#1\end{talign}}%
\def\Cov{\mrm{Cov}} %
\newcommand{\cone}{\mathbb{C}}
\newcommand{\deltarow}[1]{g^{(#1)}} %
\DeclareMathOperator{\trace}{tr} %

\newcommand{\Rad}{\ensuremath{\mathcal{R}}}

\newcommand{\conv}{\ensuremath{\mathtt{conv}}}
\newcommand{\splin}{\ensuremath{\text{splin}}}
\newcommand{\nnet}{\ensuremath{\text{nnet}}}
\newcommand{\rkhs}{\ensuremath{\text{rkhs}}}

\title{Adversarial Estimation of Riesz Representers}
\author{Victor Chernozhukov \and Whitney Newey \and Rahul Singh \and Vasilis Syrgkanis}
\date{December 2020}

\begin{document}


\maketitle


\begin{abstract}
    We provide an adversarial approach to estimating Riesz representers of  linear functionals within arbitrary function spaces. We prove oracle inequalities based on the localized Rademacher complexity of the function space used to approximate the Riesz representer and the approximation error. These inequalities imply fast finite sample mean-squared-error rates for many function spaces of interest, such as high-dimensional sparse linear functions, neural networks and reproducing kernel Hilbert spaces. Our approach offers a new way of estimating Riesz representers with a plethora of recently introduced machine learning techniques. We show how our estimator can be used in the context of de-biasing structural/causal parameters in semi-parametric models, for automated orthogonalization of moment equations and for estimating the stochastic discount factor in the context of asset pricing.
\end{abstract}

\tableofcontents

\epigraph{\normalsize ``\textit{ \textbf{The essence of a riddle is to express true facts under impossible combinations.}}"}{\normalsize--- \textit{Aristotle}, \textit{Poetics} (350 BCE)\vspace{0pt}}

\noindent
A \textit{riddle} is a puzzling question about {concepts} in our everyday life.
% , and we which one needs common sense to reason about.
For example, a riddle might ask ``\textit{My life can be measured in hours. I serve by being devoured. Thin, I am quick. Fat, I am slow. Wind is my foe. What am I?}''~
The correct answer ``\textit{candle},'' is reached by considering a collection of \textit{commonsense knowledge}:
{a candle can be lit and burns for a few hours; a candle's life depends upon its diameter; wind can extinguish candles, etc.}
\begin{figure}[t]
	\centering 
	\includegraphics[width=1\linewidth]{riddle_intro_final.pdf}
	\caption{ 
    The top example is a trivial commonsense question from the CommonsenseQA~\cite{Talmor2018CommonsenseQAAQ} dataset. 
    The two bottom examples are from our proposed \textsc{RiddleSense} challenge.
    The right-bottom question is a descriptive riddle that implies multiple commonsense facts about \textit{candle}, and it needs understanding of figurative language such as metaphor;
    The left-bottom one additionally needs counterfactual reasoning ability to address the \textit{`but-no'} cues. 
    These riddle-style commonsense questions  require NLU systems to have higher-order reasoning skills with the understanding of creative language use.
	}
	\label{fig:intro} 
\end{figure}

It is believed that the \textit{riddle} is one of the earliest forms of oral literature,
which can be seen as a formulation of thoughts about common sense, a mode of association between everyday concepts, and a metaphor as higher-order use of natural language~\cite{hirsch2014poet}.
Aristotle stated in his \textit{Rhetoric} (335-330 BCE) that good riddles generally provide satisfactory metaphors for rethinking common concepts in our daily life.
He also pointed out in the \textit{Poetics} (350 BCE): ``the essence of a riddle is to express true facts under impossible combinations,'' which suggests that solving riddles is a nontrivial  reasoning task.

Answering riddles is indeed a challenging cognitive process as it requires \textit{complex} {commonsense reasoning skills}.
% which we refer to \textit{higher-order commonsense reasoning}. 
% A successful riddle-solving model should be able to reason with \textit{multiple pieces} of commonsense facts, as 
A riddle can describe \textit{multiple pieces} of commonsense knowledge with \textit{figurative} devices such as metaphor and personification (e.g., ``wind is my \underline{foe} $\xrightarrow[]{}$ \textit{extinguish}'').
% , as shown by the examples in Figure~\ref{fig:intro}.
%%%
Moreover, \textit{counterfactual thinking} is also necessary for answering many riddles such as ``\textit{what can you hold in your left hand \underline{but not} in your right hand? $\xrightarrow[]{}$ your right elbow.}''
These riddles with \textit{`but-no'} cues require that models use counterfactual reasoning ability to consider possible solutions for situations or objects that are seemingly impossible at face value.
This \textit{reporting bias}~\cite{gordon2013reporting} makes riddles a more difficult type of commonsense question for pretrained language models to learn and reason.
% In addition, the model needs to associate commonsense knowledge with the creative use of language in descriptions, which may have figurative devices such as metaphor and personification (e.g., ``wind is my \underline{foe} $\xrightarrow[]{}$ \textit{extinguish}''). 
%For instance, one needs to know that devour
% Thus, a riddle here can be seen as a complex commonsense question that tests higher-order reasoning ability with creativity.
In contrast, \textit{superficial} commonsense questions such as ``\textit{What home entertainment equipment requires cable?}'' in  CommonsenseQA~\cite{Talmor2018CommonsenseQAAQ} are more straightforward and explicitly stated.
We illustrate this comparison in Figure~\ref{fig:intro}.


In this paper,
we introduce the \textsc{RiddleSense} challenge 
to study the task of answering riddle-style commonsense questions\footnote{We use ``riddle'' and ``riddle-style commonsense question'' interchangeably in this paper.} requiring \textit{creativity}, \textit{counterfactual thinking} and \textit{complex commonsense reasoning}.
\textsc{RiddleSense} is presented as a \textit{multiple-choice question answering} task where a model selects one of five answer choices to a given riddle question as its predicted answer, as shown in Fig.~\ref{fig:intro}.
We construct the dataset by first crawling from several free websites featuring large collections of human-written riddles and then aggregating, verifying, and correcting these examples using a combination of human rating and NLP tools to create a dataset consisting of 5.7k high-quality examples.
Finally, we use \textit{Amazon Mechanical Turk} to crowdsource quality distractors to create a challenging benchmark.
We show that our riddle questions are more challenging than {CommonsenseQA} by analyzing graph-based statistics over ConceptNet~\cite{Speer2017ConceptNet5A}, a large knowledge graph for common sense reasoning.

% The distractors for the training data are automatically generated from ConceptNet and language models while the distractors for the dev and the test sets are crowd-sourced from Amazon Mechanical Turk (AMT).
% Through data analysis based on graph connectivity, 




Recent studies have demonstrated that
 fine-tuning large pretrained language models, such as {BERT}~\cite{Devlin2019}, RoBERTa, and ALBERT~\cite{Lan2020ALBERT}, can achieve strong results on current commonsense reasoning benchmarks.
Developed on top of these language models, graph-based language reasoning models such as KagNet~\cite{kagnet-emnlp19} and MHGRN~\cite{feng2020scalable} show superior performance. 
Most recently, UnifiedQA~\cite{khashabi2020unifiedqa} proposes to unify different QA tasks and train a text-to-text model for learning from all of them, which achieves state-of-the-art performance on many commonsense benchmarks.

To provide a comprehensive benchmarking analysis, we systematically compare the above methods.
Our experiments reveal that while humans achieve 91.33\% accuracy on \textsc{riddlesense}, the best language models can only achieve 68.80\% accuracy, suggesting that there is still much room for improvement in the field of solutions to complex commonsense reasoning questions with language models.
% We also provide error analysis to better understand the limitation of current methods.
We believe the proposed \textsc{RiddleSense} challenge suggests productive future directions for machine commonsense reasoning as well as the understanding of higher-order and creative use of natural language.


% (previous state-of-the-art on \texttt{CommonsenseQA} (56.7\%)).
% However, there still exists a large gap between performance of said baselines and human performance.
% we show that the questions in RiddleSense is significantly more challenging, in terms of the length of the paths from question concepts and answer concepts.


%Apart from that, current pre-trained language models (e.g., BERT~\cite{}, RoBERTa~\cite{}, etc.) and commonsense-reasoning models (e.g., KagNet~\cite{}), can be easily adapted to work for this format with minimal modifications. 


%Note that these auto-generated distractors may be still easy for , which could diminish the testing ability of the dataset.
%We design an ader filtering method to get rid of the false negative   and control the task difficulty. 
% To strengthen the task, we propose an adversarial cross-filtering method to remove the distractors that ineffectively mislead the selected base models.
% Finally, we use human efforts to inspect the distractors and remove false negative ones, to make sure that all distractors either does not make sense or much less plausible than the correct answers.
%Introducing these fine-tuned models is inspired by the adversarial filtering algorithms~\cite{}, which can effectively reduce the  bias inside datasets for creating a more reliable benchmark.  



%Those distractors are explicitly annotated by human experts such that they are close to the meaning of 
%The main idea is to use multiple trainable generative models for learning to generate answers in a cross-validation style. 
%The wrong predictions
%Simply put, for every step, we use a large subset of the riddles and their current options ot learn multiple models for answering the remaining riddles via generation.
%After each step, we consolidate the 


% In the distantly supervised learning, we use the definition of concepts (i.e., glossary) of \textit{Wiktionary}\footnote{\url{https://www.wiktionary.org/}} to create riddles with answers as training data. 
% In the transfer learning setting, we aim to test the transferability of models across relevant datasets, such as CommonsenseQA~\cite{Talmor2018CommonsenseQAAQ}.

% We believe the \textsc{RiddleSense} task can benefit multiple communities in natural language processing. 
% First, the commonsense reasoning community can use \textsc{RiddleSense} as a new space to evaluate their reasoning models. The \textsc{RiddleSense} focuses on more complex and creative commonsense questions, which will encourage them to propose more higher-order commonsense reasoning models. 
% Second, \textsc{RiddleSense} is an NLU 
% task similar to those in the GLUE~\cite{wang2018glue} and SuperGLUE~\cite{wang2019superglue} leaderboard that can serve as a benchmark for testing various pre-trained language models.
% Last but not the least, as our task shares the similar format with many open-domain question answering tasks like \textit{Natural Questions}~\cite{kwiatkowski2019natural}, researchers in QA area may be also interested in \textsc{RiddleSense}. 





\section{Adversarial Estimator}

For any function space $\mcG$, let $\text{star}(\mcG):=\{r\, g: g\in \mcG, r \in [0, 1]\}$, denote the star hull. Let $\partial \mcG:= \{g-g': g, g'\in \mcG\}$ denote the space of differences. We will consider estimators that estimate Riesz representers within some function space $\mcA$, equipped with some norm $\|\cdot\|_{\mcA}$. Moreover, let $\ldot{\cdot}{\cdot}_2$ be the inner product associated with the $\ell_2$ norm, i.e. $\ldot{a}{a'}_2 := \E_{X}[a(X)\, a'(X)]$.\footnote{In Appendix~\ref{sec:restricted}, we examine the relationship between $\mcG$ and the Riesz representer space $\mcA$.} Given this notation, we define the class:
\begin{equation}
\mcF:=\text{star}(\partial\mcA):=\{r\, (a-a'): a, a'\in \mcA, r\in [0,1]\}    
\end{equation}
and assume that the norm $\|\cdot\|_{\mcA}$ extends naturally to the larger space $\mcF$. Moreover, let $\E_n[\cdot]$ denote the empirical average and $\|\cdot\|_{2,n}$ the empirical $\ell_2$ norm, i.e. 
$$\|g\|_{2,n} :=\sqrt{\E_n[g(X)^2]}.$$
Consider the following adversarial estimator:
\begin{equation}\label{eqn:reg-estimator}
    \hat{a} = \argmin_{a\in \mcA} \max_{f\in \mcF} \E_n[m(Z; f) - a(X)\cdot f(X)] - \|f\|_{2,n}^2 - \lambda \|f\|_{\mcA}^2 + \mu \|a\|_{\mcA}^2
\end{equation}

\begin{remark}[Population limit]
Consider the population limit of our criterion where $n\to \infty$ and $\lambda,\mu \to 0$. Then our criterion is:
\begin{equation}
    \max_{f\in \mcF} \E[m(Z; f) - a(X)\cdot f(X)] - \|f\|_{2}^2
\end{equation}
By the definition of the Riesz representer we thus have:
\begin{align}
    \max_{f\in \mcF} \E\left[m(Z; f) - a(X)\cdot f(X)\right] - \|f\|_{2}^2 =~& \max_{f\in \mcF} \E\left[(a_0(X) - a(X))\cdot f(X) -  f(X)^2\right]\\
    =~& \frac{1}{4}\E\left[(a_0(X) - a(X))^2\right] =: \frac{1}{4} \|\hat{a}-a_0\|_2^2
\end{align}
Thus our empirical criterion converges to the mean-squared-error criterion in the population limit, even though we don't have access to unbiased samples from $a_0(X)$.
\end{remark}

\begin{remark}[Norm-Based Regularization] 
The extra vanishing norm-based regularization can be avoided if one knows a bound on the norm of the true $a_0$. In that case, one can impose a hard norm constraint on the hypothesis space $\mcA$ and $\bar{\mcA}$ and optimize over these norm-constrained sub-spaces. However, regularization allows the estimator to be adaptive to the true norm of $a_0$, without knowledge of it.
\end{remark}

\begin{remark}[Mis-specification]
We in fact allow for $a_0\notin \mcA$, and incur an extra bias part in our estimation error of the form of: $\min_{a\in \mcA} \|a-a_0\|_{2}$. Thus $\mcA$ need only be an $\ell_2$-norm approximating sequence of function spaces.
\end{remark}

\section{Fast Convergence Rate}\label{sec:estimation}

We now provide fast convergence rates of our regularized minimax estimator, parameterized by the critical radii of the function classes:
\begin{align}
    \mcF_B:=~&\{f\in \mcF: \|f\|_{\mcA}^2\leq B\}\\
    m\circ \mcF_B:=~&\{m(\cdot; f): f\in \mcF_B\}
\end{align}
for some appropriately defined constant $B$. The critical radius of a function class $\mcF$ with range in $[-1, 1]$ is defined as any solution $\delta_n$ to the inequality:
\begin{align}
    {\cal R}(\delta; \mcF)\leq~& \delta^2 &
    \text{with: } {\cal R}(\delta; \mcF) =~& \E\left[\sup_{f\in \mcF: \|f\|_2\leq \delta} \frac{1}{n} \sum_{i=1}^n \epsilon_i f(X_i)\right]
\end{align}
with $\epsilon_{1:n}$ are independent Rademacher random variables drawn equiprobably in $\{-1, 1\}$. For VC-subgraph function classes with constant VC dimension the critical radius is of the order of $\sqrt{\log(n)/n}$. The critical radius has been characterized by many other function classes such as reproducing kernel Hilbert spaces, neural networks and high-dimensional linear functions (c.f. \cite{wainwright2019high} and Section~\ref{sec:examples}).

We will also require the following norm-dominance condition:
\begin{assumption}[Mean-Squared Continuity]\label{ass:strong-smooth}
For some constant $M\geq 0$, the following property holds:
\begin{equation}
    \forall f\in \mcF: \sqrt{\E\left[m(Z;f)^2\right]} \leq M\, \|f\|_2 
\end{equation}
\end{assumption}
Observe that the fact that the operator $\theta(g)$ is bounded, implies that $|\E[m(Z;g)]| \leq M\, \|g\|_2$. Mean-squared continuity is a stronger condition than boundedness, since: $|\E[m(Z;g)]|\leq \E[|m(Z;g)|] \leq \sqrt{\E[m(Z;g)^2]}$. In Appendix~\ref{sec:continuity}, we verify this condition for a variety of popular functionals.

\begin{example}[Mean-Squared Continuity for ATE] Let $X=(D,W)$ consist of treatment and covariates. In the case of treatment effect estimation, the above is implied by a non-parametric overlap condition, i.e. $\Pr[D=1\mid w] \in (1/M, 1-1/M)$ for some $M\in (1, \infty)$. Then observe that:
\begin{align}
\E[(g(1, W) - g(0, W))^2]\leq~& 2\E[g(1,W)^2 + g(0, W)^2]\\
\leq~& 2M \E\left[\Pr[D=1\mid W]\, g(1,W)^2 + \Pr[D=0\mid W]\, g(0, W)^2\right] = 2M \|g\|_2^2
\end{align}
\end{example}

\begin{theorem}\label{thm:reg-main-error}
Assume that mean-squared continuity holds for some constant $M\geq 1$ and that for some $B\geq 0$, the functions in $\mcF_B$ and $m\circ \mcF_B$ have uniformly bounded ranges in $[-1, 1]$. Let:
\[
\delta:=\delta_n + \epsilon_n + c_0 \sqrt{\frac{\log(c_1/\zeta)}{n}},
\]
for universal constants $c_0, c_1$, where $\delta_n$ upper bounds the critical radii of $\mcF_{B}, m\circ \mcF_B$
and $\epsilon_n$ upper bounds the bias $\min_{a\in \mcA} \|a-a_0\|_2$. 
Let $a_*=\argmin_{a\in \mcA} \|a-a_0\|_2$. Then the estimator in Equation~\cref{eqn:reg-estimator}, with $\mu \geq 6\lambda \geq 12\delta^2/B$, satisfies w.p. $1-\zeta$:
\begin{equation}
    \|\hat{a} - a_0\|_2 \leq O\left(M^2 \delta + \frac{\mu}{\delta} \|a_*\|_{\mcA}^2\right)
\end{equation}
For $\mu\leq C \delta^2/B$, for some constant $C$, the latter is: $O\left(\delta\,\max\left\{M^2, \frac{\|a_*\|_{\mcA}^2}{B}\right\}\right)$.
\end{theorem}

\begin{remark}
Suppose we only want to approximate the Riesz representer with respect to the weaker distance metric $\|\cdot\|_{\mcF}$ defined as:\footnote{The metric $\|\cdot\|_{\mcF}$ satisfies the triangle inequality:
\begin{equation}
    \|a + b\|_{\mcF} \leq \sqrt{\sup_{f\in \mcF} \ldot{a}{f} - \frac{1}{4} \|f\|_2^2 + \sup_{f\in \mcF} \ldot{b}{f} - \frac{1}{4} \|f\|_2^2} \leq \|a\|_{\mcF} + \|b\|_{\mcF}
\end{equation}
and is positive definite i.e. $\|0\|_{\mcF}=0$, but not necessarily homogeneous, i.e. $\|\lambda a\|_{\mcF}=? |\lambda| \|a\|_{\mcF}$ for $\lambda\in \R$.}
\begin{equation}
    \|a\|_{\mcF}^2 = \sup_{f\in \mcF}\, \ldot{a}{f}_2 - \frac{1}{4}\|f\|_2^2 \leq \|a\|_2^2
\end{equation}
Then \Cref{thm:reg-main-error} can be adapted to show that: $\|\hat{a}-a_0\|_{\mcF} \leq \delta \max\left\{M^2, \|a_*\|_{\mcA}^2/B\right\}$, where now the approximation rate is $\epsilon_n = \inf_{a\in \mcA}\|a-a_0\|_{\mcF}$. Observe that $\|\cdot\|_{\mcF}$ satisfies:
\begin{equation}
    \|a\|_{\mcF}^2 = \inf_{f\in \mcF} \frac{1}{4} \|f\|_2^2 - \ldot{a}{f}_2 + \|a\|_2^2 - \|a\|_2^2 = \inf_{f\in \mcF} \left\|a - f/2\right\|_2^2 - \|a\|_2^2 \leq \inf_{f \in \mcF} \|a - f\|_2^2 - \|a\|_2^2
\end{equation}
where in the last inequality we used the fact that $\mcF$ is star-convex. Thus it is at most the projection of $a$ on $\mcF$. Hence, it is sufficient that $\mcA$ approximates $a_0$ in this weak sense that for some $a_*\in \mcA$ the projection of $a_*-a_0$ on $\mcF$ is at most $\epsilon_n$. Thus any component of $a_0$ that is orthogonal to $\mcF$ can be ignored, since if we denote with $a_0 = a_0^{\perp} + a_0^{\parallel}$ with $\sup_{f\in \mcF} \ldot{a_0^{\perp}}{f}_2=0$, then $\|a_0 - a_*\|_{\mcF} = \|a_0^{\parallel} - a_*\|_{\mcF}$.
\end{remark}

Our proof uses similar ideas as in the proof of Theorem~1 of \cite{dikkala2020minimax}, where an adversarial estimator was considered for the case of non-parametric instrumental variable regression. Theorem~1 of \cite{dikkala2020minimax} provides bounds on a weaker metric than the mean-squared-error metric and requires bounds on the critical radius of more complicated function spaces.

As a corollary of \Cref{thm:reg-main-error}, we can obtain a bound for the un-regularized estimator with $\lambda=\mu=0$, where the function classes $\mcF$ and $\mcG$ are already norm constrained, e.g. $\|f\|_{\mcA}\leq U$ for all $f\in \mcF$, which also implies that $\|a\|_{\mcA} \leq U$ for all $a\in \mcA$, such that functions in $\mcF$ and $\mcG$ have uniformly bounded range. This can be achieved by using the above norm-constrained definitions of $\mcF$ and $\mcG$ and taking the limit of \Cref{thm:reg-main-error-2} when $B\to \infty$. In that case, $\mcF_B \to \mcF$, $m\circ \mcF_B\to m\circ \mcF$ and $\lambda, \mu$ are allowed to take zero value. This leads to the corollary:

\begin{corollary}\label{cor:main-error}
Assume that mean-squared continuity holds for some constant $M\geq 1$ and that the functions in $\mcF$ and $m\circ \mcF$ have uniformly bounded ranges in $[-1, 1]$. Let:
\[
\delta:=\delta_n + \epsilon_n + c_0 \sqrt{\frac{\log(c_1/\zeta)}{n}},
\]
for universal constants $c_0, c_1$, where $\delta_n$ upper bounds the critical radii of $\mcF, m\circ \mcF$
and $\epsilon_n$ upper bounds the bias $\min_{a\in \mcA} \|a-a_0\|_2$. 
The estimator in Equation~\cref{eqn:reg-estimator}, with $\lambda=\mu=0$, satisfies:
\begin{equation}
    \|\hat{a} - a_0\|_2 \leq O\left(M^2 \delta\right)
\end{equation}
\end{corollary}

\subsection{Fast Rates without $\ell_2$-Penalty}

We will use the following notation:
\begin{equation}
\spanF_{\kappa}(\mcF) := \left\{\sum_{i=1}^p w_i f_i: f_i\in \mcF, \|w\|_1\leq \kappa, p\leq\infty\right\}
\end{equation}

\begin{theorem}\label{thm:reg-main-error-2}
Consider a set of test functions $\mcF:=\cup_{i=1}^d \mcF^i$, that is de-composable as a union of $d$ symmetric test function spaces $\mcF^i$ and suppose that $\mcA$ is star-convex. Consider the adversarial estimator:
\begin{equation}\label{eqn:reg-estimator-2}
    \hat{a} = 
    \argmin_{a\in \mcA} \,\,\,\,\sup_{f\in \mcF}\,\, \E_n[m(Z; f) - a(X)\cdot f(X)] + \lambda \|a\|_{\mcA}
\end{equation}
Let $m\circ \mcF^i = \{m(\cdot; f): f\in \mcF^i\}$ and
\[
\delta_{n,\zeta}:=2\max_{i=1}^d \left(\mcR(\mcF^i) + \mcR(m\circ \mcF^i)\right) + c_0 \sqrt{\frac{\log(c_1\, d/\zeta)}{n}},
\]
for some universal constants $c_0, c_1$ and $B_{n,\lambda,\zeta}:= \left(\|a_0\|_{\mcA} + \delta_{n,\zeta}/\lambda\right)^2$. Suppose that $\lambda\geq \delta_{n,\zeta}$ and:
\begin{equation}\label{cond:normalized-span}
\textstyle{\forall a\in \mcA_{B_{n,\lambda,\zeta}} \text{ with } \|a-a_0\|_2\geq \delta_{n,\zeta}: \frac{a - a_0}{\|a-a_0\|_2} \in \spanF_{\kappa}(\mcF)}
\end{equation}
Then $\hat{a}$ satisfies that w.p. $1-\zeta$:
\begin{equation}
    \|\hat{a}-a_0\|_2 \leq \kappa \left( 2\left(\|a_0\|_{\mcA}+1\right) \mcR(\mcA_1) + \delta_{n,\zeta} + \lambda \left(\|a_0\|_{\mcA}-\|\hat{a}\|_{\mcA}\right)\right)
\end{equation}
\end{theorem}





\section{Example Function Spaces}\label{sec:examples}

We now instantiate our two main theorems for several function classes of interest. Throughout this section we will use the following convenient characterization of the critical radius of a function class. Corollary~14.3 and Proposition~14.25 of \cite{wainwright2019high} imply that the critical radius of any function class $\mcF$, uniformly bounded in $[-b,b]$, is of the same order as any solution to the inequality:
\begin{equation}\label{eqn:metric-entropy-critical}
    \frac{64}{\sqrt{n}} \int_{\frac{\delta^2}{2b}}^{\delta} \sqrt{\log\left(N_n(\epsilon; B_n(\delta; \mcF)\right)} d\epsilon \leq \frac{\delta^2}{b}
\end{equation}
where $B_n(\delta; \mcF)=\{f\in \mcF: \|f\|_{2,n}\leq \delta\}$ and $N_n(\epsilon; \mcF)$ is the empirical $\ell_2$-covering number at approximation level $\epsilon$, i.e. the size of the smallest $\epsilon$-cover of $\mcF$, with respect to the empirical $\ell_2$ metric.

\subsection{Sparse Linear Functions}
Consider the class of $s$-sparse linear function classes in $p$ dimensions, with bounded coefficients, i.e., 
\begin{equation}
\mcA_{\splin} :=\{x \to \ldot{\theta}{x}: \|\theta\|_{0} \leq s, \|\theta\|_{\infty}\leq b\},    
\end{equation}
then observe that $\mcF$ is also the class of $s$-sparse linear functions, with bounded coefficients in $[-2b,2b]$. Moreover, suppose that the $\ell_1$-norm of the covariates $x$ is bounded. The critical radius $\delta_n$ is of order $O\left(\sqrt{\frac{s\log(p\,n)}{n}}\right)$. It is easy to see that the $\epsilon$-covering number of such a function class is of order $N_n(\epsilon; \mcF)=O\left(\binom{p}{s} \left(\frac{b}{\epsilon}\right)^{s}\right)\leq O\left(\left(\frac{p\,b}{\epsilon}\right)^s\right)$, since it suffices to choose the support of the coefficients and then place a uniform $\epsilon$-grid on the support. Thus we get that Equation~\eqref{eqn:metric-entropy-critical} is satisfied for $\delta=O\left(\sqrt{\frac{s\log(p\, b)\,\log(n)}{n}}\right)$. Moreover, observe that if $m(Z;f)$ is $L$-Lipschitz in $f$ with respect to the $\ell_{\infty}$ norm, then the covering number of $m\circ \mcF$ is also of the same order. Thus we can apply Corollary~\ref{cor:main-error} to get:
\begin{corollary}[Sparse Linear Riesz Representer]
    The estimator presented in Corollary~\ref{cor:main-error}, with $\mcA=\mcA_{\splin}$, satisfies w.p. $1-\zeta$:
    \begin{equation}
        \|\hat{a}-a_0\|_{2} \leq O\left(\min_{a\in A_{\splin}} \|a-a_0\|_2 + \sqrt{\frac{s\log(p\, b)\,\log(n)}{n}} + \sqrt{\frac{\log(1/\zeta)}{n}}\right)
    \end{equation}
\end{corollary}

The latter Theorem required a hard sparsity constraint. However, our second main theorem, Theorem~\ref{thm:reg-main-error-2}, allows us to prove a similar guarantee for the relaxed version of $\ell_1$-bounded high-dimensional linear function classes. For this corollary we require a restricted eigenvalue condition which is typical for such relaxations. 

\begin{corollary}[Sparse Linear Riesz Representer with Restricted Eigenvalue]\label{cor:sparse-linear-reg-ell1}
Suppose that $a_0(x)=\ldot{\theta_0}{x}$ with $\|\theta_0\|_0\leq s$ and $\|\theta_0\|_1\leq B$ and $\|\theta_0\|_{\infty}\leq 1$. Moreover, suppose that the covariance matrix $V=\E[xx']$ satisfies the restricted eigenvalue condition:
\begin{equation}
    \forall \nu\in \R^p \text{ s.t. } \|\nu_{S^c}\|_1 \leq \|\nu_S\|_1 + \delta_{n,\zeta}/\lambda: \nu^\top  V\nu \geq \gamma \|\nu\|_2^2
\end{equation}
Let $\mcA = \{x\to \ldot{\theta}{x}: \theta \in \R^p\}$, $\|\ldot{\theta}{\cdot}\|_{\mcA}=\|\theta\|_1$, and $\mcF=\{x \to  \xi x_i: i\in [p], \xi\in \{-1, 1\}\}$. Then the estimator presented in Equation~\eqref{eqn:reg-estimator-2} with $\lambda\leq \frac{\gamma}{8s}$, satisfies that w.p. $1-\zeta$:
\begin{equation}
    \textstyle{\|\hat{a}-a_0\|_2 \leq O\left( \max\left\{1, \frac{1}{\lambda}\frac{\gamma}{s}\right\} \sqrt{\frac{s}{\gamma}} \left((\|\theta_0\|_1 + 1)\sqrt{\frac{\log(p)}{n}} + \sqrt{\frac{\log(p/\zeta)}{n}}\right)\right)}
\end{equation}
\end{corollary}


\begin{remark}[Restricted Eigenvalue]
We note that if we have that the unrestricted minimum eigenvalue of $V$ is at least $\gamma$, then the restricted eigenvalue condition always holds. Moreover, observe that we only require a condition on the population covariance matrix $V$ and not on the empirical covariance matrix.
\end{remark}

\subsection{Neural Networks}

Suppose that the function class $\mcA$ can be represented as a RELU activation neural network with depth $L$ and width $W$, denoted as $\mcA_{\nnet(L,W)}$. Then observe that functions in $\mcF$ can be represented as neural networks with depth $L+1$ and width $2W$. Moreover, we assume that functions in $m\circ \mcF$ are also representable by neural networks of depth $O(L)$ and width $O(W)$. Finally, suppose that the covariates are distributed in a way that the outputs of $\mcF$ and $m\circ\mcF$ are uniformly bounded in $[-b,b]$. 

Then by the $L_1$ covering number for VC classes of \cite{haussler1995sphere}, the bounds of theorem 14.1 of \cite{anthony2009neural} and Theorem~6 of \cite{bartlett2019nearly}, one can show that the critical radius of $\mcF$ and $m\circ \mcF$ is of the order of $\delta_n=O\left(\sqrt{\frac{L\, W\, \log(W)\,\log(b)\, \log(n)}{n}}\right)$ (c.f. Proof of Example~3 of \cite{foster2019orthogonal} for a detailed derivation). Thus we can apply Corollary~\ref{cor:main-error} to get:
\begin{corollary}[Neural Network Riesz Representer]
    Suppose that $\mcA=\mcA_{\nnet(L,W)}$, and that $m\circ \mcF$ is representable as a neural network with depth $O(L)$ and width $O(W)$. Moreover, the input covariates are such that functions in $\mcF$ and $m\circ \mcF$ are uniformly bounded in $[-b,b]$. Then the estimator presented in Corollary~\ref{cor:main-error}, satisfies w.p. $1-\zeta$:
    \begin{equation}
        \|\hat{a}-a_0\|_{2} \leq O\left(\min_{a\in A_{\nnet(L,W)}} \|a-a_0\|_2 + \sqrt{\frac{L\, W\, \log(W)\,\log(b)\, \log(n)}{n}} + \sqrt{\frac{\log(1/\zeta)}{n}}\right)
    \end{equation}
\end{corollary}

If the true Riesz representer $a_0$ is representable as a RELU neural network, then the first term vanishes and we achieve an almost parametric rate. For non-parametric Holder function classes, one can easily combine the latter corollary with approximation results for RELU activation neural networks presented in \cite{yarotsky2017error,yarotsky2018optimal}. These approximation results typically require that the depth and the width of the neural network grow as some function of the approximation error $\epsilon$, leading to errors of the form: $O\left(\epsilon + \sqrt{\frac{L(\epsilon)\, W(\epsilon)\, \log(W(\epsilon))\,\log(b)\, \log(n)}{n}} + \sqrt{\frac{\log(1/\zeta)}{n}}\right)$. Optimally balancing $\epsilon$ then typically leads to almost tight non-parametric rates, of the same order as those presented in Theorem~1 of \cite{farrell2018DeepNeural}.

\subsection{Reproducing Kernel Hilbert Spaces}

Suppose that $a_0$ lies in a Reproducing Kernel Hilbert Space (RKHS) with kernel $K$, denoted as $\mcA_{\rkhs(K)}$ and with the norm $\|\cdot\|_{\mcA}$ being the RKHS norm. Then observe that $\mcF$ is the same function space. Moreover, we assume that $m\circ \mcF$ also lies in an RKHS with a potentially different kernel $\tilde{K}$. Finally, suppose that the input covariates are such that for some constant $B$, functions in $\mcF_B$ and $m\circ \mcF_B$ are bounded in $[-1, 1]$. 

Let $\{\hat{\lambda}_j\}_{j=1}^n$ be the eigenvalues of the $n\times n$ empirical kernel matrix, with $K_{ij}=K(x_i, x_j)/n$. Similarly, let $\{\hat{\mu}_j\}_{j=1}^n$ be the eigenvalues of the empirical kernel matrix $\tilde{K}$. Then by Corollary~13.18 of \cite{wainwright2019high}, we can derive the following corollary of Theorem~\ref{thm:reg-main-error}:
\begin{corollary}[RKHS Riesz Representer]
    Suppose that $\mcA=\mcA_{\rkhs}$, $a_0\in \mcA_{\rkhs}$, and that $m\circ \mcF\in \mcA_{\rkhs(\tilde{K})}$. Let $\{\hat{\lambda}_j\}_{j=1}^n$ and $\{\hat{\mu}_j\}_{j=1}^n$ be  the egienvalues of the empirical kernel matrices of $K$ and $\tilde{K}$, correspondingly. Let $\delta_n$ be any solution to the inequalities:
    \begin{align}
        B\sqrt{\frac{2}{n}}\sqrt{\sum_{j=1}^\infty \max\{\hat{\lambda}_j, \delta^2\}}\leq~& \delta^2 &
        B\sqrt{\frac{2}{n}}\sqrt{\sum_{j=1}^\infty \max\{\hat{\mu}_j, \delta^2\}}\leq~& \delta^2
    \end{align}
    Moreover, the input covariates are such that functions in $\mcF_B$ and $m\circ \mcF_B$ are uniformly bounded in $[-1, 1]$. Then the estimator presented in Theorem~\ref{thm:reg-main-error}, satisfies w.p. $1-\zeta$:
    \begin{equation}
        \|\hat{a}-a_0\|_{2} \leq O\left(\|a_0\|_{\mcA} \left( \delta_n + \sqrt{\frac{\log(1/\zeta)}{n}}\right)\right)
    \end{equation}
\end{corollary}

We note that the latter estimator does not need to know the RKHS norm of the true function $a_0$. Instead it automatically adapts to the unknown RKHS norm. Moreover, note that the bound $\delta_n$ is solely based on empirically observable quantities, as it is a function of the empirical eigenvalues. Thus these empirical quantities can be used as a data-adaptive diagnostic of the error.

Finally, we note that for particular kernels a more explicit bound can be derived as a function of the eigendecay. For instance, for the Gaussian kernel, which has an exponential eigendecay, Example~13.21 of \cite{wainwright2019high} derives that the solution to the eigenvalue inequality scales as $O\left(\sqrt{\frac{\log(n)}{n}}\right)$, thus leading to almost parametric rates: $\|\hat{a}-a_0\|_2 \leq O\left(\|a_0\|_{\mcA} \sqrt{\frac{\log(n)}{n}}\right)$.

\section{Computation}\label{sec:computation}

In this section we discuss computational aspects of the optimization problem implied by our adversarial estimator. We show how in many cases, the min-max optimization problem can be solved computationally efficiently and also discuss practical heuristics for cases where the problem is non-convex (e.g. in the case of neural networks).

\subsection{Sparse Linear Function Spaces}

For the case of sparse linear functions, the estimator in \Cref{thm:reg-main-error-2} requires solving the following optimization problem:
\begin{equation}\label{eqn:minimax-ell1}
    \min_{\theta\in \R^p: \|\theta\|_1\leq B}\,\, \max_{i\in [2p]} \E_n\left[m(Z; f_i) - f_i(X)\, \ldot{\theta}{X} \right] + \lambda \|\theta\|_1
\end{equation}
where $f_i(X) = X_i$ for $i\in \{1, \ldots, p\}$ and $f_i(X)=-X_i$ for $i\in \{p+1, \ldots, 2p\}$.
This can be solved via sub-gradient descent, which would yield an $\epsilon$-approximate solution after $O\left(p/\epsilon^2\right)$ steps. This can be improved to $O\left(\log(p)/\epsilon\right)$ steps if one views it as a zero-sum game and uses simultaneous gradient descent, where the $\theta$-player uses Optimistic-Follow-the-Regularized-Leader with an entropic regularizer and the $f$-player uses Optimistic Hedge over probability distributions on the finite set of test functions (analogous to Proposition~13 of \cite{dikkala2020minimax}).

To present the algorithm it will be convenient to re-write the problem where the maximizing player optimizes over distributions in the $2p$-dimensional simplex, i.e.:
\begin{equation}
    \min_{\theta\in \R^p: \|\theta\|_1\leq B}\,\, \max_{w\in \R_{\geq 0}^{2p}:\|w\|_1=1} \E_n\left[m(Z; \ldot{w}{f}) - \ldot{w}{f}(X)\, \ldot{\theta}{X} \right] + \lambda \|\theta\|_1
\end{equation}
 where $f=(f_1, \ldots, f_{2p})$, denote the vector of the $2p$ functions. Moreover, to avoid the non-smoothness of the $\ell_1$ penalty it will be convenient to introduce the augmented vector $V=(X; -X)$ and for the minimizing player to optimize over the positive orthant of a $2p$-dimensional vector $\rho=(\rho^+; \rho^-)$, with an $\ell_1$ bounded norm, such that in the end: $\theta=\rho^+ - \rho^-$. Then we can re-write the problem as:
\begin{equation}
    \min_{\rho\in \R_{\geq 0}^{2p}: \|\rho\|_1\leq B}\,\, \max_{w\in \R_{\geq 0}^{2p}:\|w\|_1=1} \E_n\left[m(Z; \ldot{w}{f}) - \ldot{w}{V}\, \ldot{\rho}{V} \right] + \lambda \sum_{i=1}^{2p} \rho_i
\end{equation}
where we also noted that $\ldot{w}{f}(X) = \ldot{w}{V}$.
\begin{proposition}\label{prop:sparse-optimization-ell1}
Consider the algorithm that for $t=1, \ldots, T$, sets:
\begin{align}
    \tilde{\rho}_{i, t+1} =~& \tilde{\rho}_{i, t} e^{- 2\frac{\eta}{B}\, \left(- \E_n[V_i\, \ldot{V}{w_t}] + \lambda\right) + \frac{\eta}{B}\, \left(- \E_n[V_i\, \ldot{V}{w_{t-1}}] + \lambda\right)} &
    \rho_{t+1} =~& \tilde{\rho}_{t+1}\, \min\left\{1, \frac{B}{\|\tilde{\rho}_{t+1}\|_1}\right\}\\
    \tilde{w}_{i, t+1} =~& w_{i, t} e^{2\, \eta\, \E_n[m(Z; f_i) - V_i \ldot{V}{\rho_t}] - \eta\, \E_n[m(Z; f_i) - V_i \ldot{V}{\rho_{t-1}}]} & w_{t+1} =~& \frac{\tilde{w}_{t+1}}{\|\tilde{w}_{t+1}\|_1}
\end{align}
with $\tilde{\rho}_{i,-1}=\tilde{\rho}_{i,0}=1/e$ and $\tilde{w}_{i,-1}=\tilde{w}_{i,0}=1/(2p)$ and returns $\bar{\rho}=\frac{1}{T} \sum_{t=1}^T \rho_t$. Then for $\eta=\frac{1}{4\|\E_n[VV^\top]\|_{\infty}}$,\footnote{For a matrix $A$, we denote with $\|A\|_{\infty}=\max_{i, j} |A_{ij}|$} after
\begin{equation}
T=16\|\E_n[VV^\top]\|_{\infty} \frac{4B^2 \log(B\vee 1) + (B+1) \log(2p)}{\epsilon}
\end{equation}
iterations, the parameter $\bar{\theta}=\bar{\rho}^{+} - \bar{\rho}^-$ is an $\epsilon$-approximate solution to the minimax problem in Equation~\eqref{eqn:minimax-ell1}.
\end{proposition}


\subsection{Neural Nets with Simultaneous Stochastic Gradient Descent}

When the function space $\mcA$ and $\mcF$ is represented as a deep neural network then the optimization problem is highly non-convex. This is the case even if we were just solving a square loss minimization problem. On top of this we also need to deal with the non-convexity and non-smoothness introduced by the min-max structure of our estimator. 

Luckily, the optimization problem that we are facing is similar to the optimization problem that is encountered in training Generative Adversarial Networks, i.e. we need to solve a non-convex, non-concave zero-sum game, where the strategy of each of the two players are the parameters of a neural net. Luckily, there has been a surge of recent work proposing iterative optimization algorithms inspired by the convex-concave zero-sum game theory (see, e.g. the Optimistic Adam algorithm of \cite{Daskalakis2017}, also utilized in the recent work of \cite{bennett2019deep,dikkala2020minimax} in the context of solving moment equations, or the work of \cite{Hsieh2019,Mishchenko2019} on the extra-gradient or stochastic extra-gradient algorithm). All these new algorithms for solving differentiable non-convex/non-concave zero-sum games can be deployed for our problem.

Recent work of \cite{liao2020provably} contributes to a literature on over-parameterized neural network training for square losses \cite{AllenZhu2018,du2018gradient,Soltanolkotabi2019}. The authors show that even for min-max losses that are very similar to the loss of our estimator, neural nets that are sufficiently wide and appropriately randomly initialized essentially behave like linear functions in an appropriate reproducing kernel Hilbert space, typically referred to as the neural tangent kernel space. Given this intuition, the authors show that a simple simultaneous gradient descent/ascent algorithm and subsequent averaging of the parameters converges to the solution of the min-max problem. In this regime neural networks behave like linear functions, so one can invoke analysis similar to the analysis we invoke for sparse linear function spaces, and then carefully account for the approximation error. The intuition and results of the work of \cite{liao2020provably} can be appropriately adapted for our loss function too so as to show that the average path of the simultaneous gradient descent/ascent algorithm also converges in our setting. One caveat is that growing the width of the neural net to facilitate optimization deteriorates the statistical guarantee, since the critical radius grows as a function of the width.

\subsection{Reproducing Kernel Hilbert Space}\label{sec:RKHS}


%

Recall the estimator is
$$
\hat{a} = \argmin_{a\in \mcA} \max_{f\in \mcF} \E_n[m(Z; f) - a(X)\cdot f(X)] - \|f\|_{2,n}^2 - \lambda \|f\|_{\mcA}^2 + \mu \|a\|_{\mcA}^2
$$
In this section, we derive a closed form solution for $\hat{a}$ that can be computed from matrix operations.

Towards this end, we impose additional structure on the problem. If $\mathcal{G}=\mathcal{F}=\mathcal{H}$ is a reproducing kernel Hilbert space (RKHS), then the projection $a_0^{\min}$ of any RR $a_0$ into $\mathcal{G}$ is clearly an element of $\mathcal{H}$ as well, so we can take $\mathcal{A}=\mathcal{H}$. Also assume that the functional $m$ satisfies $m(z;f)=m(x;f)$. Moreover, let the functional be such that it evaluates the function in some arguments. For example, in ATE, $m(z;f)=f(1,w)-f(0,w)$ where $z=x=(d,w)$. This property holds for treatment effects and policy effects, and it ensures that $m(\cdot;f)\in\mathcal{H}$.

Denote the kernel $k:\mathcal{X}\times\mathcal{X}\rightarrow\mathbb{R}$, and denote the feature map $\phi:x\mapsto k(x,\cdot)$. Denote the kernel matrix $K_{XX}$ with $(i,j)$-th entry $k(x_i,x_j)$. Denote the feature matrix $\Phi$ with $i$-th row $\phi(x_i)'$. Hence $K_{XX}=\Phi\Phi'$.

By the reproducing property
$
f(x)=\langle f,\phi(x) \rangle_{\mathcal{H}}
$.  Moreover, since $m$ is a linear functional, we can define the linear operator
$
M:\mathcal{H}\rightarrow \mathcal{H},\; f(\cdot)\mapsto m(\cdot;f)
$
whereby
$$
m(x;f)=[Mf](x)=\langle Mf,\phi(x) \rangle_{\mathcal{H}} = \langle f, M^*\phi(x) \rangle_{\mathcal{H}}
$$
where $M^*$ is the adjoint of $M$. Define the matrix $\Phi^{(m)}:=\Phi M$ with $i$-th row $\phi(x_i)'M$. Finally define $\Psi$ as the matrix with $2n$ rows that is constructed by concatenating $\Phi$ and $\Phi^{(m)}$. We denote the induced kernel matrix by $K:=\Psi\Psi'$. Formally,
$$
\Psi:=\begin{bmatrix} \Phi\\ \Phi^{(m)}\end{bmatrix},\quad K:=\begin{bmatrix} K^{(1)} & K^{(2)} \\ K^{(3)}  & K^{(4)} \end{bmatrix}:=\begin{bmatrix} \Phi \Phi' & \Phi (\Phi^{(m)})' \\ \Phi^{(m)}\Phi' & \Phi^{(m)}  (\Phi^{(m)})'  \end{bmatrix} 
$$
Note that $\{K^{(j)}\}_{j\in[4]}\in \mathbb{R}^{n\times n}$ and hence $K\in\mathbb{R}^{2n\times 2n}$ can be computed from data, though they depend on the choice of moment. 
\begin{proposition}[Computing kernel matrices]\label{prop:kernel_matrices}
For example, for ATE
\begin{align}
[K^{(1)}]_{ij}&=k((d_i,w_i),(d_j,w_j)) \\
[K^{(2)}]_{ij}&=k((1,w_i),(d_j,w_j))-k((0,w_i),(d_j,w_j)) \\
    [K^{(3)}]_{ij}&=k((d_i,w_i),(1,w_j))-k((d_i,w_i),(0,w_j)) \\
    [K^{(4)}]_{ij} &=k((1,w_i),(1,w_j))-k((1,w_i),(0,w_j))-k((0,w_i),(1,w_j))+k((0,w_i),(0,w_j))
\end{align}
\end{proposition}

We proceed in steps. First we prove the existence of a closed form for the maximizer $\hat{f}=\argmax_{f\in\mathcal{H}} \E_n[m(X;f) - a(X)\cdot f(X)] - \|f\|_{2,n}^2 - \lambda \|f\|_{\mcH}^2$ by extending the classic representation theorem of \cite{kimeldorf1971some,scholkopf2001generalized}.

%

\begin{proposition}[Representation of maximizer]\label{prop:rep1}
$\hat{f}=\Psi'\hat{\gamma}$ for some $\hat{\gamma}\in\mathbb{R}^{2n}$
\end{proposition}

Appealing to this abstract result, we derive the closed form expression for the maximizer in terms of kernel matrices.

\begin{proposition}[Closed form of maximizer]\label{prop:closed1}
$\hat{\gamma}=\frac{1}{2}\Delta^{-1}\left[n\Psi M' \hat{\mu} -\begin{bmatrix}K^{(1)} \\ K^{(3)} \end{bmatrix}\Phi a\right]$ 
where
$$ \Delta:=\begin{bmatrix} K^{(1)}K^{(1)} & K^{(1)} K^{(2)} \\ K^{(3)} K^{(1)} & K^{(3)}K^{(2)}\end{bmatrix}+n\lambda K\in\mathbb{R}^{2n\times 2n},\quad \hat{\mu}:=\frac{1}{n}\sum_{i=1}^n \phi(x_i)$$
\end{proposition}

%

Next we prove the existence of a closed form for the minimizer $\hat{a}=\argmin_{a\in\mathcal{H}} \E_n[m(X;\hat{f}) - a(X)\cdot \hat{f}(X)] - \|\hat{f}\|_{2,n}^2 - \lambda \|\hat{f}\|_{\mcH}^2 + \mu \|a\|_{\mcH}^2$ by appealing to the classic representation theorem of \cite{kimeldorf1971some,scholkopf2001generalized}.

\begin{proposition}[Representation of minimizer]\label{prop:rep2}
$\hat{a}=\Phi'\hat{\beta}$ for some $\hat{\beta}\in\mathbb{R}^n$
\end{proposition}

Again, with this abstract result in hand, we derive the closed form expression for the minimizer in terms of kernel matrices.

\begin{proposition}[Closed form of minimizer]\label{prop:closed2}
$\hat{\beta}=\bigg\{
       \frac{1}{n}\Omega  \Delta^{-1}
        \begin{bmatrix} K^{(1)} K^{(1)} \\ K^{(3)} K^{(1)}\end{bmatrix}
    +2\mu\cdot K^{(1)}\bigg\}^{-1}\Omega\Delta^{-1}\Psi M'\hat{\mu}$
    where
    $$
 \Omega:=\begin{bmatrix}K^{(1)}K^{(1)} \\ K^{(3)}K^{(1)} \end{bmatrix}'
%
-\frac{1}{2}\begin{bmatrix}K^{(1)}K^{(1)} \\ K^{(3)}K^{(1)} \end{bmatrix}'    
\Delta^{-1}
       \begin{bmatrix} K^{(1)}K^{(1)} & K^{(1)} K^{(2)} \\ K^{(3)} K^{(1)} & K^{(3)}K^{(2)}\end{bmatrix}
        %
        -\frac{n\lambda}{2}\begin{bmatrix}K^{(1)}K^{(1)} \\ K^{(3)}K^{(1)} \end{bmatrix}' \Delta^{-1}K \in\mathbb{R}^{n\times 2n}
    $$
\end{proposition}

For practical use, we require a way to evaluate the minimizer using only kernel operations. Evaluation directly follows from the closed form expression.

\begin{corollary}[Evaluation of minimizer]\label{cor:RKHS}
$$
\hat{a}(x)=K_{xX}\bigg\{
       \frac{1}{n}\Omega  \Delta^{-1}
        \begin{bmatrix} K^{(1)} K^{(1)} \\ K^{(3)} K^{(1)}\end{bmatrix}
    +2\mu\cdot K^{(1)}\bigg\}^{-1}\Omega\Delta^{-1}V
$$
where $V\in\mathbb{R}^{2n}$ is defined such that
$$
v_j=
\begin{cases} 
\frac{1}{n}\sum_{i=1}^n [K^{(2)}]_{ji} &\text{ if }j\in[n] \\
\frac{1}{n}\sum_{i=1}^n  [K^{(4)}]_{ji} &\text{ if }j\in\{n+1,...,2n\}\end{cases}
$$
\end{corollary}

\subsection{Oracle Based Training}



Consider the estimator with $\lambda=\mu=0$:
\begin{equation}\label{eqn:unreg-minimax}
    \min_{a\in \mcA} \max_{f\in \mcF} \E_n\left[m(Z;f) - a(X)\cdot f(X) - f(X)^2\right] =: \ell(a, f)
\end{equation}
We can solve this optimization problem by treating it as a zero-sum game, where one player controls $a$ and the other player controls $f$. Observe that the game is convex $a$ (in fact linear) and concave in $f$. Thus, we can solve this zero-sum game by having the $f$-player run a no-regret algorithm at each period $t\in \{1,\ldots, T\}$ and the $a$-player best responding to the current choice of the $f$ player.

Observe that for any fixed $f$, the best-response of the $a$-player is the solution to:
\begin{equation}
    a_t = \argmin_{a\in \mcA} -\E_n[a(X) \cdot f(X)] = \argmax_{a\in \mcA} \E_n[a(X) \cdot f(X)]
\end{equation}
In other words, the $a$-player wants to match the sign of the function $f$. Thus the best-response of the $a$-player is equivalent to a weighted classification oracle, where the label is $Y_i = \sign(f(X_i))$ and the weight is $w_i = |f(X_i)|$.

Finally, we need to solve the no-regret problem for the $f$ player. If the function space $\mcF$ is a convex space,\footnote{i.e. if $f,f'\in \mcF$, then $\gamma f+(1-\gamma)f'\in \mcF$ for any $\gamma\in [0, 1]$} then we can simply run the follow the leader (FTL) algorithm, where at every period the algorithm maximizes the empirical past reward:
\begin{equation}
    f_t = \argmax_{f\in \mcF} \E_n\left[m(Z;f) - \bar{a}_{< t}(X)\cdot f(X) - f(X)^2\right] = \argmin_{f\in \mcF} -\ell(\bar{a}_{<t}, f)
\end{equation}
where $\bar{a}_{<t} = \frac{1}{t-1} \sum_{\tau<t} a_{\tau}$.

\begin{proposition}\label{prop:oracle}
Suppose that the empirical operator $\E_n[m(Z; \cdot)]$ is bounded with operator norm upper bounded by $M_n\geq 1$ and that the function class $\mcF$ is convex. Consider the algorithm where at each period $t\in \{1, \ldots, T\}$:
\begin{align}
    f_t =~& \argmax_{f\in \mcF} \ell(\bar{a}_{<t}, f) & a_t =~& \argmin_{a\in \mcA} \ell(a, f_t)
\end{align}
Then for $T=\Theta\left(\frac{M_n\, \log(1/\epsilon)}{\epsilon}\right)$, the function $a_*=\frac{1}{T}\sum_{t=1}^T a_t$ is an $\epsilon$-approximate solution to the empirical minimax problem in Equation~\eqref{eqn:unreg-minimax}.
\end{proposition}

The above algorithm requires a weighted classification oracle for the $a$-player and an oracle for the $f$-player that solves the problem $\max_{f\in \mcF} \ell(a, f)$, for any $a$.

\begin{example}[$f$-player oracle for ATE]
For the case of ATE this problem is:
\begin{align}
    f_* =~& \argmin_{f\in \mcF} \E_n\left[f(T, X)^2 + a(T, X) f(T, X) - f(1, X) + f(0, X)\right]
\end{align}
\end{example}


\section{Debiasing Average Moment}\label{sec:debiasing}

Suppose our goal is to estimate $\theta_0 = \theta(g_0)$, where $g_0=\E[Y\mid X]$. We have access to an estimate $\hat{g}$ of $g_0$. We consider the de-biased moment:
\begin{equation}
    m_{a}(Z; g) = m(Z; g) + a(X)'\, (Y - g(X))
\end{equation}
For simplicity of exposition, we present the remainder of the section for the case of a single-valued regression function.

\subsection{Asymptotic Normality with Sample Splitting}

Consider the following cross-fitted estimate:
\begin{itemize}
    \item Partition $n$ samples into $K$ folds $P_1, \ldots, P_K$
    \item For each partition, estimate $\hat{a}_k, \hat{g}_k$ based on all out-of-fold data.
    \item Construct estimate:
    \begin{equation}
        \hat{\theta} = \frac{1}{n} \sum_{k=1}^K \sum_{i\in P_k} m_{\hat{a}_k}(Z_i; \hat{g}_k)
    \end{equation}
\end{itemize}

\begin{lemma}\label{lem:debias}
Suppose that $K=\Theta(1)$ and that:
\begin{align}\label{eqn:main-cond}
    \forall k \in [K]: \sqrt{n}\, \E[(a_0(X) - \hat{a}_k(X))\, (\hat{g}_k(X) - g_0(X))] \rightarrow_p 0
\end{align}
and that for some $a_*$ and $g_*$ (not necessarily equal to $a_0$ and $g_0$), we have that for all $k\in [K]$: $\|\hat{a}_k-a_*\|_2 \stackrel{L^2}{\to} 0$ and $\|\hat{g}_k-g_*\|_2\stackrel{L^2}{\to} 0$. Assume that Condition~\ref{ass:strong-smooth} is satisfied and the variables $Y, g(X), a(X)$ are bounded a.s. for all $g\in \mcG$ and $a\in \mcA$.\footnote{This condition can be relaxed to simply assuming bounded fourth moments of $Y, g(X), a(X)$, as long as we strengthen the requirement to assume $4$-th moment convergence to $a_*, g_*$, i.e. that $\|\hat{a}_k-a_*\|_4, \|\hat{g}_k-g_*\|_4 \to_p 0$.} Then if we let $\sigma_*^2 := \Var(m_{a_*}(Z; g_*))$
\begin{equation}
    \sqrt{n}\left(\hat{\theta} - \theta_0\right) \to_d N\left(0, \sigma_*^2\right)
\end{equation}
\end{lemma}

A sufficient condition for Condition~\ref{eqn:main-cond} is that $\sqrt{n} \|\hat{a}-a_0\|_2 \|\hat{g} - g_0\|_2 \to_p 0$, which is a condition on the product of the two RMSE rates. However, observe that Condition~\eqref{eqn:main-cond} is much weaker as it implies that our Riesz estimate $\hat{a}$ only needs to approximately satisfy the representer moment for test functions of the form: $\hat{g}-g_0$. Thus, if we assume that $\hat{g}$ satisfies an RMSE consistency rate that $\|\hat{g}-g_0\|_2\leq r_n$, then it suffices that it satisfies the moment for any $g\in {\cal G}$, with $\|g-\hat{g}\|_2\leq r_n$, i.e. it suffices that it is a local Riesz representer around $\hat{g}$. This can potentially make the Riesz estimation task much simpler than estimating a global Riesz representer. We formalize this observation in Appendix~\ref{sec:local_RR}.

Moreover, observe that the theorem does not require consistency of both nuisance functions. Only one of the two nuisance functions needs to be consistent, while the other must simply converge to some limit function. For instance, as long as $\sqrt{n} \|\hat{a}-a_0\|_2\to 0$ or $\sqrt{n}\|\hat{g}-g_0\|_2 \to 0$, then the result holds. Inconsistency will only impact the limit variance, which will not be equal to the efficient variance; nonetheless, confidence intervals will be asymptotically valid. The required rate for the latter scenario is implausible as it asks for faster than root-$n$ rate for either $a$ or $g$. However, we can still show that the de-biased moment satisfies a double robustness property: if one nuisance is inconsistent, as long as the other is root-$n$ consistent, then asymptotic normality of the causal parameter still holds. This result is presented in \Cref{sec:inconsistent}. The result is analogous to the one provided in \cite{Benkeser2017}, where an estimator with such a property was presented within the targeted maximum likelihood framework.


\subsection{Asymptotic Normality without Sample Splitting}

Consider the algorithm where no cross-fitting or sample splitting is employed:
\begin{itemize}
    \item Estimate $\hat{a}, \hat{g}$ on all the samples
    \item Construct estimate:
    \begin{equation}
        \hat{\theta} = \E_n\left[m_{\hat{a}}(Z; \hat{g})\right]
    \end{equation}
\end{itemize}

\begin{lemma}[Normality via Localized Complexities]\label{lem:debias-nocross}
Suppose that:
\begin{align}\label{eqn:main-cond-reuse}
    \forall k \in [K]: \sqrt{n}\, \E[(a_0(X) - \hat{a}_k(X))\, (\hat{g}_k(X) - g_0(X))] \rightarrow_p 0
\end{align}
and that for some $a_*$ and $g_*$ (not necessarily equal to $a_0$ and $g_0$), we have that: $\|\hat{a}_k-a_*\|_2, \|\hat{g}_k-g_*\|_2 = o_p(r_n)$. Assume that Condition~\ref{ass:strong-smooth} is satisfied and the variables $Y, g(X), a(X)$ are bounded a.s. for all $g\in \mcG$ and $a\in \mcA$. Moreover, assume that with high probability $\|\hat{g}\|_{\mcG} \leq B_1$ and $\|\hat{a}\|_{\mcA} \leq B_2$. Let $\delta_{n,*} = \delta_{n} + c_0 \sqrt{\frac{\log(c_1\, n)}{n}}$ for some appropriately defined universal constants $c_0, c_1$, where $\delta_n$ is a bound on the critical radius of $\mcG_{B_1}$, $m\circ \mcG_{B_1}$ and $\mcA_{B_2}$ and also at least $\sqrt{\frac{\log\log(n)}{n}}$. If \begin{equation}
\sqrt{n}\left(\delta_{n,*}\, r_n + \delta_{n,*}^2\right)\to 0
\end{equation}
then if we let $\sigma_*^2 := \Var(m_{a_*}(Z; g_*))$
\begin{equation}
    \sqrt{n}\left(\hat{\theta} - \theta_0\right) \to_d N\left(0, \sigma_*^2\right)
\end{equation}
\end{lemma}


Suppose we use for both $\hat{a}$ and $\hat{g}$ an $\ell_1$ constrained linear function class in $p$ dimensions and that $a_*, g_*$ are sparse linear functions with support size $s$. Moreover if $B_1=\|a_*\|_1 + o(1)$ and $B_2=\|g_*\|_1 + o(1)$, and the covariates satisfy a restricted eigenvalue condition, then we could show that $\delta_{n, *}=O\left(\sqrt{\frac{s\log(p)}{n}}\right)$ (a simplification by assuming $s\log(p)>\log(n)$). Then as long as $r_n \to 0$, the condition is satisfied. Moreover, for such function classes, we will typically have that $r_n = O\left(\sqrt{\frac{s\log(p)}{n}}\right)$. Therefore, the required condition is that: $\frac{s\log(p)}{\sqrt{n}}=o(1)$ or equivalently $s=o\left(\sqrt{n}/\log(p)\right)$. 

Of theoretical interest, it seems that without sample splitting, the analysis essentially goes through for general function classes that are not Donsker. With sample splitting, we would require from Condition~\eqref{eqn:main-cond-reuse} that $\frac{\sqrt{s_{a} s_{g}} \log(p)}{n} = o(n^{-1/2})$, where $s_a, s_g$ are the sparsity bounds on $a$ and $g$, respectively. Simplifying, with sample splitting we require $\sqrt{s_a\, s_g} = o\left(\sqrt{n}/\log(p)\right)$. By contrast, without sample splitting, we require this condition for both $s_a$ and $s_g$. Beyond this difference, the conditions on the sparsity of the function classes seem comparable.

We also provide a proof of asymptotic normality without sample splitting for uniformly stable estimators. This proof technique handles cases beyond Donsker classes or classes with small critical radius, since stability is not only a property of the function class but also of the estimation algorithm. Thus, it could be potentially apply to large neural net classes trained via few iterations of stochastic gradient descent \cite{Hardt2016} or sub-bagged ensembles of overfitting estimators \cite{elisseeff2003leave}.


\begin{lemma}[Normality via Uniform Stability]\label{lem:debias-nocross-stability}
Suppose that:
\begin{align}\label{eqn:main-cond-reuse}
    \forall k \in [K]: \sqrt{n}\, \E[(a_0(X) - \hat{a}_k(X))\, (\hat{g}_k(X) - g_0(X))] \rightarrow_p 0
\end{align}
and that for some $a_*$ and $g_*$ (not necessarily equal to $a_0$ and $g_0$), we have that:
\begin{equation}
\E\left[\|\hat{a}_k-a_*\|_2^2\right], \E\left[\|\hat{g}_k-g_*\|_2^2\right] = O(r_n^2)    
\end{equation}
Assume that Condition~\ref{ass:strong-smooth} is satisfied and the variables $Y, g(X), a(X)$ are bounded a.s. for all $g\in \mcG$ and $a\in \mcA$. Suppose that the algorithm for estimating $\hat{h}:=(\hat{a}, \hat{g})$ is symmetric across samples and satisfies $\beta_n$-mean-squared stability, i.e.:\footnote{The notion was originally defined in \cite{Kale11cross-validationand} and used to derive imporved bounds on $k$-fold cross-validation. It is weaker than the well-studied uniform stability \cite{Bousquet2002StabilityAG}. See \cite{elisseeff2003leave,Celisse2016,pmlr-v98-abou-moustafa19a} for more discussion.}
\begin{equation}
    \E_Z\left[\left\|\hat{h}(Z) - \hat{h}^{-i}(Z)\right\|_{\infty}^2\right] \leq \beta_n
\end{equation}
where $\hat{h}^{-i}$ is the function that the estimation algorithm would produce if sample $i$ was removed from the training set. If
\begin{equation}
r_{n-1}^2 + n\,\beta_{n-1} r_{n-2}\to 0
\end{equation}
then if we let $\sigma_*^2 := \Var(m_{a_*}(Z; g_*))$
\begin{equation}
    \sqrt{n}\left(\hat{\theta} - \theta_0\right) \to_d N\left(0, \sigma_*^2\right)
\end{equation}
\end{lemma}


\paragraph{Uniform stability of sub-bagged ensemble estimators.} If we use sub-bagging and return as an estimate the average of a base estimator over subsamples of size $s<n$, then the sub-bagged estimate is $\beta_n:=\frac{s}{n}$-uniformly stable (see e.g. \cite{elisseeff2003leave}). If the bias of the base estimator decays as some function $\textsc{bias}(s)$, then typically sub-bagged estimators will achieve $r_n = \sqrt{\frac{s}{n}} + \textsc{bias}(s)$ (see e.g. \cite{Athey2016,Khosravi2019,Syrgkanis2020}). Thus we need that $n\beta_n r_n = \sqrt{\frac{s^3}{n}} + s\, \textsc{bias}(s) \to 0$. As long as $s=o(n^{1/3})$ and $\textsc{bias}(s)=o(1/s)$, then the conditions of the latter theorem hold. The recent work of \cite{Syrgkanis2020} shows that in a high-dimensional regression setting, with $p\gg n$ and only $r\ll p,n$ of the variables being $\mu$-\emph{strictly} relevant variables, i.e. leading to a decrease in explained variance of at least $\mu$, (for some constant $\mu>0$), the bias of a deep Breiman tree trained on $s$ data points decays as $\exp(-s)$. Moreover, a deep Breiman forest where each tree is trained on $s=O\left(\frac{2^r \log(p)}{\mu}\right)=o(n^{1/3})$ samples, drawn without replacement, will achieve $r_n = O\left(\sqrt{\frac{s 2^r}{n}}\right)$. Thus sub-bagged deep Breiman random forests satisfy the conditions of the theorem in the case of sparse high-dimensional non-parametric regression.


\section{Orthogonalizing Non-Linear Moment}

Suppose our goal is to estimate the solution $\theta_0$ to a non-linear moment problem that depends on a regression function $g_0$, i.e.:
\begin{equation}
    \E[m(Z; \theta_0, g_0)] := 0
\end{equation}
One way to construct a Neyman orthogonal moment that is robust to first-stage errors of the regression is to introduce a bias correction term that involves the Riesz representer of the functional derivative of the moment with respect to $g$, i.e.:
\begin{equation}
    m_{a}(Z; \theta, g) = m(Z; \theta, g) + a(X)'\, (Y - g(X))
\end{equation}
where $a_0(X)$ is the Riesz representer of the functional derivative of $m$ with respect to $g$, i.e.:
\begin{equation}
    f(g) := \frac{\partial}{\partial \tau} \E\left[m(Z; \theta, g_0 + \tau\, (g - g_{0}))\right] \bigg|_{\tau=0} = \E[a(X)' g(X)]
\end{equation}

The Riesz representer $a_0$ can be estimated in a first stage as follows:
\begin{itemize}
    \item Estimate the regression function $\hat{g}$
    \item Estimate a preliminary $\tilde{\theta}$ using the non-orthogonal moment condition
    \item Calculate algebraically, or through automatic differentiation, the Gateaux derivative function:
    \begin{equation}
        \hat{f}(g) = \frac{\partial}{\partial \tau}\E\left[m(Z; \tilde{\theta}, \hat{g} + \tau\, (g - \hat{g}))\right]\bigg|_{\tau=0}
    \end{equation}
    \item Apply the adversarial Riesz representer estimator for functional $\hat{f}(g)$, to estimate $a$
\end{itemize}

Following similar analysis as in Section~5 of \cite{chernozhukov2018learning}, one can show that the moment $m_{a}$ satisfies Neyman orthogonality. Moreover, assuming that the moment function is sufficiently smooth, the estimator outlined above will achieve faster than $n^{-1/4}$ rates. These two properties are sufficient to show that the estimator for $\theta$, based on the orthogonal moment and using cross-fitting, will be root-$n$ asymptotically normal.

One caveat of the approach outlined above is the burden of either calculating the Gateaux derivative algebraically or auto-differentiating the moment. One can bypass this difficult, and reduce to evaluation oracles of the moment, by taking arbitrarily small approximations of the Gateaux derivative. In particular, the third step could be replaced by defining:
\begin{equation}
    \hat{f}_{\epsilon}(g) = \frac{1}{\epsilon}\left(\E\left[m(Z;\tilde{\theta}, \hat{g} + \epsilon (g - g_0)) - m(Z;\tilde{\theta}, \hat{g})\right]\right)
\end{equation}
For sufficiently small $\epsilon$, the approximation error $\|\hat{f}_{\epsilon} - \hat{f}\|$ is negligible. Moreover, $\hat{f}_{\epsilon}$ only requires black-box access to evaluations of the moment function to be computed.

\bibliographystyle{plain}
\bibliography{refs}

\appendix 

\section{Unrestricted and Restricted Models}\label{sec:restricted}

In the context of semi-parametric statistics, recall that the causal parameter $\theta_0=\theta(g_0)=\mathbb{E}[m(Z;g_0)]$ is a functional $m$ of the underlying regression $g_0(x):=\mathbb{E}[Y|X=x]$. In an unrestricted model, we assume $g_0\in L^2(\mathbb{P})$, the space of square integrable functions. In a restricted model, additional information about $g_0$ can be encoded by the restriction $g_0\in\mathcal{G}_0\subset L^2(\mathbb{P})$, where $\mathcal{G}_0$ is some convex function space. In this section, we give an account of Riesz representation in restricted models, following the notation and technical lemmas of \cite{chernozhukov2018global}. 

Denote $\mathcal{G}:=span(\mathcal{G}_0)$ and $\bar{\mathcal{G}}:=closure(\mathcal{G})$. Define the modulus of continuity of $g\mapsto \theta(g)$ by
$$
L:=\sup_{g\in \mathcal{G}\backslash \{0\}} \frac{|\theta(g)|}{\|g\|_2}
$$
\begin{definition}[RR and minimal RR]
A RR of the functional $\theta(g)$ is $a_0\in L^2(\mathbb{P})$ s.t.
$$
\theta(g)=\mathbb{E}[g(X)a_0(X)],\quad \forall g\in \mathcal{G}
$$
If $a_0\in \bar{\mathcal{G}}$, then it is the minimal RR and we denote it by $a_0^{\min}$. Any RR can be reduced to the minimal RR by projecting it onto $\bar{\mathcal{G}}$.
\end{definition}

\begin{lemma}[Lemma 1 of \cite{chernozhukov2018global}]
We have the following results
\begin{enumerate}
    \item If $L<\infty$ then there exists a unique minimal RR $a_0^{\min}$ and $L=\|a_0^{\min}\|_2$
    \item If there exists a RR $a_0$ with $\|a_0\|_2<\infty$ then $L=\|a_0^{\min}\|_2\leq \|a_0\|_2<\infty$, where $a_0^{\min}$ is the unique minimal RR obtained by projecting $a_0$ onto $\bar{\mathcal{G}}$
\end{enumerate}
In both cases, $g\mapsto\theta(g)$ can be extended to $\bar{\mathcal{G}}$ or to all of $L^2(\mathbb{P})$ with modulus of continuity $L$
\end{lemma}

To interpret these results, consider a toy example of vectors in $\mathbb{R}^3$ rather than functions in $L^2(\mathbb{P})$. Suppose the functional of interest is 
$$
\theta:\mathbb{R}^3 \rightarrow\mathbb{R},\quad (x,y,z)\mapsto x+2y+3z
$$
Moreover, assume $g_0\in \mathcal{G}$ where $\mathcal{G}$ is the $(x,y)$-plane, though the ambient space is $\mathbb{R}^3$. Then any vector of the form $a_0=(1,2,c)$ with $c\in\mathbb{R}$ is a valid RR. The unique minimal RR is $a_0^{\min}=(1,2,0)$. As an aside, the vector $a_0=(1,2,3)$ is a universal RR; it holds for any choice of $\mathcal{G}\subset \mathbb{R}^3$, not just the $(x,y)$-plane. From any RR, we can obtain the minimal RR by projection onto the $(x,y)$-plane.

In \cite[Theorem 2]{chernozhukov2018global}, we see that it is better to use $a_0^{\min}$ rather than any $a_0$ to attain full semi-parametric efficiency (unless of course $\mathcal{G}=L^2(\mathbb{P})$ so there is no difference). By the stated lemma, we know how to obtain $a_0^{\min}$ from any $a_0$: projection onto $\bar{\mathcal{G}}$.

When do these technical issues arise? In the semi-parametric literature, a popular restricted model is the additive model. It is an important setting where $\mathcal{G}$ is not dense in $L^2(\mathbb{P})$. We present a definition of the additive model, then a technical lemma about the minimal RR in an additive model.

\begin{definition}[Additive model]
Suppose that 
\begin{enumerate}
    \item the regression $g_0$ is additive in components $x=(x^{(1)},x^{(2)})$:
$
g_0(x)=g^{(1)}_0(x^{(1)})+g^{(2)}_0(x^{(2)})
$
    \item $g_0^{(1)} \in \mathcal{G}_0^{(1)}$, a dense subset of $L^2(\mathbb{P}^{(1)})$, where $\mathbb{P}^{(1)}$ is the distribution of $X^{(1)}$
    \item the functional depends on only the first component: $m(z;g)=m(z;g^{(1)})$
\end{enumerate}

\end{definition}

\begin{lemma}[Lemma 6 of \cite{chernozhukov2018global}]
Assume an additive model. Consider any RR $a_0\in L^2(\mathbb{P})$. Then $\forall g\in\mathcal{G}$
$$
\theta(g)=\theta(g^{(1)})=\int a_0^{\min}(x^{(1)})g^{(1)}(x^{(1)})d\mathbb{P}^{(1)},\quad a_0^{\min}(x^{(1)})=\mathbb{E}[a_0(X)|X^{(1)}=x^{(1)}]
$$
and
$$
\|a_0^{\min}\|_q\leq \|a_0\|_q,\quad\forall q\in[1,\infty]
$$
\end{lemma}
This preservation of order and contraction of norm is helpful in analysis.

Finally, we quote some projection geometry for sumspaces from \cite[Appendix A.4]{bickel1993efficient}. Suppose $\mathcal{H}_1$ and $\mathcal{H}_2$ are closed subspaces of a Hilbert space $\mathcal{H}$. 

\begin{lemma}
If $\mathcal{H}_1 \perp \mathcal{H}_2$ then the projection onto the sumspace $\mathcal{H}_1 + \mathcal{H}_2$ is the sum of the projections onto $\mathcal{H}_1$ and $\mathcal{H}_2$
\end{lemma}

More generally, $\mathcal{H}_1$ may not be orthogonal to $\mathcal{H}_2$. Denote by $P_i$ the orthogonal projection onto $\mathcal{H}_i$, and denote by $Q_i:=I-P_i$ the projection onto $\mathcal{H}_i^{\perp}$. Denote by $\Pi$ the projection onto the closure of $\mathcal{H}_1+\mathcal{H}_2$

\begin{lemma}[Corollary 1 of \cite{bickel1993efficient}]
For any $h\in\mathcal{H}$
$$
[I-(Q_1Q_2)^m]h\rightarrow \Pi h,\quad m\rightarrow\infty
$$
\end{lemma}

Stronger versions of this result are available that provide quantitative rates of convergence and that allow for $r\geq 2$ subspaces.
\section{Examples}

\subsection{Causal Inference}\label{sec:continuity}

Recall the definition of mean-squared continuity: $\exists M\geq 0$ s.t.
$$
\forall f\in \mcF: \sqrt{\E\left[m(Z;f)^2\right]} \leq M\, \|f\|_2 
$$

We verify mean-square continuity for several important functionals.
\begin{enumerate}
    \item Average treatment effect (ATE): $\theta_0=\mathbb{E}[g_0(1,W)-g_0(0,W)]$
    
    To lighten notation, let $\pi_0(w):=\mathbb{P}(D=1|W=w)$ be the propensity score.
    Assume $\pi_0(w)\in \left(\frac{1}{M},1-\frac{1}{M}\right)$ for $M\in(1,\infty)$. Then
    \begin{align}
        \E[g(1, W) - g(0, W)]^2&\leq~ 2\E[g(1,W)^2 + g(0, W)^2]\\
&\leq~ 2M \E\left[\pi_0(W)\, g(1,W)^2 + [1-\pi_0(W)]\, g(0, W)^2\right] \\
&= 2M \mathbb{E}[g(X)]^2
    \end{align}
    
    \item Average policy effect: $\theta_0=\int g_0(x)d\mu(x)$ where $\mu(x)=F_1(x)-F_0(x)$
    
    Denote the densities corresponding to distributions $(F,F_1,F_0)$ by $(f,f_1,f_0)$. Assume $\frac{f_1(x)}{f(x)}\leq \sqrt{M}$ and $\frac{f_0(x)}{f(x)}\leq \sqrt{M}$ for $M\in[0,\infty)$. In this example, $m(Z;g)=m(g)$.
    \begin{align}
        \mathbb{E}[m(Z;g)]^2&=\{m(g)\}^2 \\
        &=\left\{\int g(x)d\mu(x)\right\}^2 \\
        &=\left\{\mathbb{E}\left[ g(X)\left\{\frac{f_1(X)}{f(X)}-\frac{f_0(X)}{f(X)}\right\} \right]\right\}^2 \\
        &\leq \left\{2 \sqrt{M} \mathbb{E}|g(X)|\right\}^2 \\
        &\leq 4M \mathbb{E}[g(X)]^2
    \end{align}
    
    \item Policy effect from transporting covariates: $\theta_0=\mathbb{E}[g_0(t(X))-g_0(X)]$
    
    Denote the density of $t(X)$ by $f_t(x)$. Assume $\frac{f_t(x)}{f(x)}\leq M$ for $M\in[0,\infty)$. Then
    \begin{align}
        \mathbb{E}[g(t(X))-g(X)]^2 
        &\leq 2\mathbb{E}[g(t(X))^2+g(X)^2] \\
        &= 2\mathbb{E}\left[g(X)^2\left\{\frac{f_t(X)}{f(X)}-1\right\}\right] \\
        &\leq 2 (M+1) \mathbb{E}[g(X)]^2
    \end{align}
    \item Cross effect: $\theta_0=\mathbb{E}[Dg_0(0,W)]$
    
    Assume $\pi_0(w)<1-\frac{1}{M}$ for some $M\in(1,\infty)$. Then
    \begin{align}
        \mathbb{E}[Dg(0,W)]^2&\leq \mathbb{E}[g(0,W)]^2 \\
        &\leq M \mathbb{E}[\{1-\pi_0(W)\}g(0,W)^2] \\
        &\leq M \mathbb{E}[g(X)]^2
    \end{align}
    
    \item Regression decomposition: $\mathbb{E}[Y|D=1]-\mathbb{E}[Y|D=0]=\theta_0^{response}+\theta_0^{composition}$
    where
    \begin{align}
        \theta_0^{response}&=\mathbb{E}[g_0(1,W)|D=1]-\mathbb{E}[g_0(0,W)|D=1] \\
        \theta_0^{composition}&=\mathbb{E}[g_0(0,W)|D=1]-\mathbb{E}[g_0(0,W)|D=0]
    \end{align}
    
    Assume $\pi_0(w)<1-\frac{1}{M}$ for some $M\in(1,\infty)$. Then re-write the target parameters in terms of the cross effect.
      \begin{align}
        \theta_0^{response}&=\frac{\mathbb{E}[DY]-\mathbb{E}[Dg_0(0,W)]}{\mathbb{E}[D]} \\
        \theta_0^{composition}&=
        \frac{\mathbb{E}[D\gamma_0(0,W)]}{\mathbb{E}[D]}-\frac{\mathbb{E}[(1-D)Y]}{\mathbb{E}[1-D]}
    \end{align}
    We implement DML for the cross effect, empirical means for the population means, then delta method.
    
    \item Average treatment on the treated (ATT): $\theta_0=\mathbb{E}[g_0(1,W)|D=1]-\mathbb{E}[g_0(0,W)|D=1]$
    
    Assume $\pi_0(w)<1-\frac{1}{M}$ for some $M\in(1,\infty)$. Then re-write the target parameters in terms of the cross effect.
      \begin{align}
        \theta_0&=\frac{\mathbb{E}[DY]-\mathbb{E}[Dg_0(0,W)]}{\mathbb{E}[D]}
    \end{align}
    We implement DML for the cross effect, empirical means for the population means, then delta method.
    
    \item Local average treatment effect (LATE): $\theta_0=\frac{\mathbb{E}[g_0(1,W)-g_0(0,W)]}{\mathbb{E}[h_0(1,W)-h_0(0,W)]}$
    
    The result follows from the view of LATE as a ratio of two ATEs.
\end{enumerate}

\subsection{Asset Pricing}\label{sec:finance}

We present three proofs of the existence of the stochastic discount factor. These arguments are quoted from the excellent exposition of \cite{cochrane2009asset}.

\begin{enumerate}
   \item Marginal rate of substitution in a consumption model.
        
        Consider an investor with utility function $U(c_t,c_{t+1})=u(c_t)+\beta \mathbb{E}_t[u(c_{t+1})]$, where $u$ is period utility, $c_t$ is consumption at time $t$, and $\beta$ is a subjective discount factor. Denote by $e_t$ the original consumption level, and $\xi$ the amount of the asset the consumer buys. The consumer solves the optimization problem
        $$
        \max_{\xi} u(c_t)+\beta\mathbb{E}_t [u(c_{t+1})]\quad \text{s.t.}\quad c_t=e_t-p_t\xi,\quad c_{t+1}=e_{t+1}+x_{t+1}\xi
        $$
        Substituting constraints into the objective, the FOC yields
        $$
        p_t=\mathbb{E}_t\left[\beta \frac{u'(c_{t+1})}{u'(c_t)}x_{t+1}\right],\quad m_{t+1}=\beta \frac{u'(c_{t+1})}{u'(c_t)}
        $$
        The same FOC arises in the longer-term objective $\mathbb{E}_t \left[\sum_{j=0}^{\infty} \beta^j u(c_{t+j})\right]$.
     \item State price density in a contingent claim model with complete markets.
        
        For simplicity, consider a two-period model with $S$ possible states of nature tomorrow. A contingent claim is a security that pays one dollar in one state $s$ only tomorrow. $pc_t(s)$ is the price today of the contingent claim. In a \textit{complete market}, investors can buy any contingent claims. If there are complete contingent claims, the state price density exists, and it is equal to the contingent claim price divided by probabilities. Let $x_{t+1}(s)$ denote an asset's payoff in state of nature $s$. The asset's price must equal the value of the contingent claims of which it is a bundle. Let $\pi_{t+1}(s)$ be the probability that state $s$ occurs conditional on information available today. Then
        $$
        p_t=\sum_s pc_t(s) x_{t+1}(s)=\sum_s \pi_{t+1}(s) \frac{pc_t(s)}{\pi_{t+1}(s)} x_{t+1}(s),\quad m_{t+1}(s)=\frac{pc_t(s)}{\pi_{t+1}(s)}
        $$
    \item Pricing kernel from the law of one price.
    
    Let $\mathcal{X}$ be the set of all payoffs that investors can purchase (or the subset of tradeable payoffs used in a particular study). For example, if there are complete contingent claims to $S$ states of nature then $\mathcal{X}=\mathbb{R}^S$. More generally, markets are incomplete, so $\mathcal{X}\subset \mathbb{R}^S$. 
    
    Free portfolio formation means $x,x'\in\mathcal{X}$ implies $ax+bx'\in\mathcal{X}$ for any $a,b\in\mathbb{R}$. This assumption rules out short sales constraints, bid-ask spreads, and leverage limitations. Let $p_t(x)$ denote the price at time $t$ of the asset that delivers payoff $x$ at time $t+1$. The law of one price means $p_t(ax+bx')=ap_t(x)+bp_t(x')$. In other words, asset pricing is a linear functional over a vector space. This assumption says that investors cannot make instantaneous profits by repackaging portfolios. It would be satisfied in a market that has already reached equilibrium. 
        
Given free portfolio formation and the law of one price, there exists a unique payoff $m_{t+1}^*\in\mathcal{X}$ such that $p_t(x)=\mathbb{E}_t[m_{t+1}^* x]$ for all $x\in\mathcal{X}$. $m_{t+1}^*$ is called the \textit{mimicking portfolio}. Unless markets are complete, there are infinitely many SDFs that satisfy $p_t(x)=\mathbb{E}_t[m_{t+1} x]$ of the form $m_{t+1}=m_{t+1}^*+\epsilon$ where $\epsilon\in \mathcal{X}^{\perp}$. An incomplete market can be interpreted as a restricted model, and the mimicking portfolio can be interpreted as a minimal Riesz representer in the discussion of Section~\ref{sec:restricted}.
    
\end{enumerate}


\section{Local Riesz Representer Convergence Rate}\label{sec:local_RR}

Suppose that we use the constraint the test functions to lie in:
\begin{equation}
    \mcF(r_n) = \{f \in \text{star}\left(\partial(\mcG - \hat{g})\right): \|f\|_2\leq r_n \}
\end{equation}
And consider the estimator:
\begin{equation}
    \inf_{a \in \mcA}\,\, \sup_{f \in } \Psi_n(a, f)
\end{equation}
Then by a localized concentration bound we have:
\begin{align}
    \forall a\in \mcA, f\in \mcF(r_n): \left| \Psi_n(a, f) - \Psi(a, f) \right| \leq~& O\left(\delta_{n,\zeta} \|m(\cdot; f) - a\, f\|_2 + \delta_{n,\zeta}^2 \right)\\
    \leq~& O\left((M+1)\delta_{n,\zeta} \|f\|_2 + \delta_{n,\zeta}^2\right) \\
    \leq~& O\left((M+1)\delta_{n,\zeta} r_n + \delta_{n,\zeta}^2\right) =: \epsilon_{n}
\end{align}
where $\delta_{n,\zeta}=\delta_n + c_0\sqrt{\frac{\log(c_1/\zeta)}{n}}$ and $\delta_n$ bounds the critical radius of the function class:
\[
\{Z \to m(Z; f) - a(X)\, f(X): f\in \mcF(r_n), a\in \mcA\}.
\]
Thus we have that:
\begin{align}
   \sup_{f \in \mcF(r_n)} \Psi(\hat{a}, f) - \epsilon_n \leq~& \sup_{f \in\mcF(r_n)} \Psi_n(\hat{a}, f) \leq \sup_{f \in \mcF(r_n)} \Psi_n(a_*, f) \\
   \leq~& \sup_{f \in \mcF(r_n)} \Psi(a_*, f) + \epsilon_n\\
    =~& \inf_{a \in \mcA} \sup_{f \in \mcF(r_n)} \Psi(a, f) + \epsilon_n
\end{align}
Concluding that:
\begin{equation}
    \sup_{f \in \mcF(r_n)} \Psi(\hat{a}, f) \leq \inf_{a \in \mcA} \sup_{f \in \mcF(r_n)} \Psi(a, f) + 2\,\epsilon_n
\end{equation}
Moreover, if $a_0$ is a local Riesz representer, i.e. it satisfies the Riesz equation for differences with $\hat{g}$ of any function in $\mcG$ within a ball $r_n$ around $\hat{g}$, then:
\begin{equation}
    \inf_{a \in \mcA} \sup_{f \in \mcF(r_n)} \Psi(a, f)  = \inf_{a \in \mcA} \sup_{f \in \mcF(r_n)} \ldot{a_0 - a}{f} \leq r_n \inf_{a \in \mcA} \sup_{f\in \mcF(1)}  \ldot{a_0 - a}{f} \leq r_n \inf_{a\in \mcA} \|a_0 - a\|_2
\end{equation}

Thus if $g_0$ lies within a ball $r_n$ of $\hat{g}$, we conclude that:
\begin{equation}
    \E[(a_0(X) - \hat{a}(X))\, \left(\hat{g}(X) - g_0(X)\right)] \leq O\left(M\, r_n \delta_{n,\zeta} + \delta_{n,\zeta}^2 + r_n\, \inf_{a\in \mcA}\|a_0 - a\|_2\right)
\end{equation}

If for instance $r_n\, \delta_{n,\zeta} = o(n^{-1/2})$ and $\delta_{n,\zeta} = o(n^{-1/4})$ and $r_n\inf_{a\in \mcA}\|a_0 - a\|_2 = o(n^{-1/2})$, then we can conclude that:
\begin{equation}
    \sqrt{n}\, \E[(a_0(X) - \hat{a}(X))\, \left(\hat{g}(X) - g_0(X)\right)] \to_p 0
\end{equation}

If $a_0 \in \mcA$ and both $\mcA$ and $\mcG$ are VC-subgraph classes with constant VC dimension, then it can be shown that $\delta_{n,\zeta}=O\left(\sqrt{\frac{\log(n/\zeta)}{n}}\right)$. Thus for the above conditions to hold, it suffices that: $r_n = o(1)$ (i.e. that $\hat{g}$ is RMSE-consistent).

Finally observe that we need $\mcA$ to have a small approximation error to $a_0$, not with respect to the $\|\cdot\|_2$ norm, but rather with the weaker norm:
\begin{equation}
    \|a_0 - a\|_{\mcF} = \sup_{f\in \mcF(1)}  \ldot{a_0 - a}{f}
\end{equation}
Thus $a$ does not need to match the component of $a_0$ that is orthogonal to the subspace $\mcF$. If for instance, we assume that $\mcF$ lies in the space spanned by top $K$ eigenfunctions of a reproducing kernel hilbert space, then it suffices to consider $\mcA$ the space spanned by those functions too. Then $\inf_{a\in \mcA} \|a_0 - a\|_{\mcF}=0$. For instance, if $\mcG$ is a finite dimensional linear function space and $g_0\in \mcG$, then it suffices to consider $\mcA$ that is also finite dimensional linear, even if the true $a_0$ does not lie in that sub-space. Then all the conditions of \Cref{lem:debias} will be satisfied, even if $\hat{a}$ will never be consistent with respect to $a_0$.
\section{Proofs from \Cref{sec:estimation}}
For convenience, throughout this section we will use the notation:
\begin{align}
    \Psi(a, f) :=~& \E\brk*{m(Z; f) - a(X)\cdot f(X)} = \E[(a_0(X)-a(X))\, f(X)] \tag{by Riesz definition}\\
    \Psi_n(a, f) :=~& \frac{1}{n}\sum_{i=1}^{n} \left(m(Z_i; f) - a(X_i)\cdot f(X_i)\right)
\end{align}

\subsection{Proof of \Cref{thm:reg-main-error}}

\begin{proof} 
Let:
\begin{align}
    \Psi_n^\lambda(a, f) =~& \Psi_n(a, f) - \|f\|_{2,n}^2 - \lambda \|f\|_{\mcA}^2\\
    \Psi^{\lambda}(a, f) =~& \Psi(a, f) - \frac{1}{4} \|f\|_2^2 - \frac{\lambda}{2}\|f\|_{\mcA}^2
\end{align}
Thus our estimate can be written as:
\begin{equation}
    \hat{a} := \argmin_{a\in \mcA} \sup_{f\in \mcF} \Psi_n^{\lambda}(a, f) + \mu \|a\|_{\mcA}^2
\end{equation}


\paragraph{Relating empirical and population regularization.} 
As a preliminary observation, we have that by Theorem 14.1 of \cite{wainwright2019high}, w.p. $1-\zeta$:
\begin{equation}
    \forall f\in \mcF_{B}: \left|\|f\|_{n,2}^2 - \|f\|_2^2 \right| \leq \frac{1}{2} \|f\|_2^2 + \delta^2 
\end{equation}
for our choice of $\delta:=\delta_n + c_0 \sqrt{\frac{\log(c_1/\zeta)}{n}}$, where $\delta_n$ upper bounds the critical radius of $\mcF_{B}$ and $c_0, c_1$ are universal constants. Moreover, for any $f$, with $\|f\|_{\mcA}^2\geq B$, we can consider the function $f \sqrt{B}/\|f\|_{\mcA}$, which also belongs to $\mcF_{B}$, since $\mcF$ is star-convex. Thus we can apply the above lemma to this re-scaled function and multiply both sides by $\|f\|_{\mcA}^2/B$, leading to:
\begin{equation}
    \forall f\in \mcF \text{ s.t. } \|f\|_{\mcA}^2 \geq B: \left|\|f\|_{n,2}^2 - \|f\|_2^2 \right| \leq \frac{1}{2} \|f\|_2^2 + \delta^2\frac{\|f\|_{\mcA}^2}{B}
\end{equation}
Thus overall, we have:
\begin{equation}\label{eqn:reg-pop-emp}
    \forall f\in \mcF: \left|\|f\|_{n,2}^2 - \|f\|_2^2 \right| \leq \frac{1}{2} \|f\|_2^2 + \delta^2 \max\left\{1, \frac{\|f\|_{\mcA}^2}{B}\right\}
\end{equation}
Thus we have that w.p. $1-\zeta$: 
\begin{align}
    \forall f\in \mcF: \lambda \|f\|_{\mcA}^2 + \|f\|_{2,n}^2 \geq~& \lambda \|f\|_{\mcA}^2 + \frac{1}{2}\|f\|_{2}^2 - \delta^2 \max\left\{1, \frac{\|f\|_{\mcA}^2}{B}\right\}\\
    \geq~& \left(\lambda - \frac{\delta^2}{B}\right)\|f\|_{\mcA}^2 + \frac{1}{2}\|f\|_{2}^2 - \delta^2
\end{align}
Assuming that $\lambda \geq \frac{2\delta^2}{B}$, we have that, the latter is at least:
\begin{equation}
    \forall f\in \mcF: \lambda \|f\|_{\mcA}^2 + \|f\|_{2,n}^2 \geq \frac{\lambda}{2} \|f\|_{\mcA}^2 + \frac{1}{2}\|f\|_{2}^2 - \delta^2
\end{equation}


\paragraph{Upper bounding centered empirical sup-loss.} We now argue that the centered empirical sup-loss: 
$$\sup_{f \in \mcF} (\Psi_n(\hat{a}, f) - \Psi_n(a_*, f)) = \sup_{f \in \mcF} \E_n[(a_*(X)-\hat{a}(X))\, f(X)]$$ 
is small. By the definition of $\hat{a}$:
\begin{equation}\label{eqn:reg-optimality}
\sup_{f\in \mcF} \Psi_n^{\lambda}(\hat{a}, f) \leq \sup_{f\in \mcF} \Psi_n^{\lambda}(a_*, f) + \mu \left(\|a_*\|_{\mcA}^2 - \|\hat{a}\|_{\mcA}^2\right)
\end{equation}

By Lemma~7 of \cite{Foster2019}, the fact that $m(Z; f) - a_*(X) f(X)$ is $2$-Lipschitz with respect to the vector $(m(Z;f), f(z))$ (since $a_*(X)\in [-1, 1]$) and by our choice of $\delta:=\delta_n + c_0 \sqrt{\frac{\log(c_1/\zeta)}{n}}$, where $\delta_n$ is an upper bound on the critical radius of $\mcF_{B}$ and $m \circ \mcF_B$, w.p. $1-\zeta$:
\begin{align}
 \forall f\in \mcF_{B}: \left|\Psi_n(a_*, f) - \Psi(a_*, f)\right| \leq O\left(\delta \left(\|f\|_2 + \sqrt{\E[m(Z;f)^2]}\right) + \delta^2\right)
 =O\left(\delta\, M\, \|f\|_2 + \delta^2\right)
\end{align}
where we have invoked Assumption~\ref{ass:strong-smooth}. Thus, if $\|f\|_{\mcA} \geq \sqrt{B}$, we can apply the latter inequality for the function $f \sqrt{B}/\|f\|_{\mcA}$, which falls in $\mcF_{B}$, and then multiply both sides by $\|f\|_{\mcA}/\sqrt{B}$ (invoking the linearity of the operator $\Psi_n(a, f)$ with respect to $f$) to get:
\begin{align}\label{eqn:reg-concentration}
    \forall f \in \mcF: \left|\Psi_n(a_*, f) - \Psi(a_*, f)\right| \leq O\left(\delta\, M\, \|f\|_2 + \delta^2 \max\left\{1, \frac{\|f\|_{\mcA}}{\sqrt{B}}\right\}\right) 
\end{align}

By Equations~\eqref{eqn:reg-optimality} and \eqref{eqn:reg-concentration}, we have that w.p. $1-2\zeta$, for some universal constant $C$:
\begin{align}
   \sup_{f\in \mcF} \Psi_n^{\lambda}(a_*, f)
    =~& \sup_{f\in \mcF} \left(\Psi_n(a_*, f) - \|f\|_{2,n}^2 - \lambda \|f\|_{\mcA}^2\right)\\
    \leq~& \sup_{f\in \mcF} \left(\Psi(a_*, f) + C \delta^2 + \frac{C\delta^2}{\sqrt{B}}\|f\|_{\mcA} + C M \delta \|f\|_2 - \|f\|_{2,n}^2 - \lambda \|f\|_{\mcA}^2\right)\\
    \leq~& \sup_{f\in \mcF} \left(\Psi(a_*, f) + C \delta^2 + \frac{C\delta^2}{\sqrt{B}}\|f\|_{\mcA}  + C M\delta \|f\|_2 - \frac{1}{2} \|f\|_2^2 - \frac{\lambda}{2} \|f\|_{\mcA}^2 + \delta^2\right)\\
    \leq~& \sup_{f\in \mcF} \Psi^{\lambda/2}(a_*, f) + O\left(\delta^2\right)
    + \sup_{f\in \mcF} \left( \frac{C\delta^2}{\sqrt{B}}\|f\|_{\mcA} - \frac{\lambda}{4}\|f\|_{\mcA}^2\right) + \sup_{f\in \mcF} \left(C M\delta \|f\|_2 - \frac{1}{4}\|f\|_{2}^2\right)
\end{align}
Moreover, observe that for any norm $\|\cdot\|$ and any constants $a,b>0$:
\begin{align}
    \sup_{f\in \mcF} \left(a\|f\| - b \|f\|^2\right) \leq \frac{a^2}{4b}
\end{align}
Thus if we assume that $\lambda\geq 2\delta^2/B$, we have:
\begin{align}
\sup_{f\in \mcF} \left( \frac{C\delta^2}{\sqrt{B}}\|f\|_{\mcA} - \frac{\lambda}{4}\|f\|_{\mcA}^2\right) \leq~& \frac{C^2 \delta^4}{B \lambda} \leq \frac{C^2}{2} \delta^2\\
    \sup_{f\in \mcF} \left(C M\delta \|f\|_2 - \frac{1}{4}\|f\|_{2}^2\right) \leq~& C^2 M^2 \delta^2
\end{align}
Thus we have:
\begin{align}
    \sup_{f\in \mcF} \Psi_n^{\lambda}(a_*, f)\leq \sup_{f\in \mcF} \Psi^{\lambda/2}(a_*, f) + O\left(M^2\, \delta^2\right)
\end{align}
Moreover:
\begin{align}
\sup_{f\in \mcF} \Psi_n^{\lambda}(\hat{a}, f) =~& \sup_{f \in \mcF} \left(\Psi_n(\hat{a}, f) - \Psi_n(a_*, f) + \Psi_n(a_*, f) - \|f\|_{2,n}^2 - \lambda \|f\|_{\mcA}^2\right)\\ 
\geq~& \sup_{f \in \mcF} \left(\Psi_n(\hat{a}, f) - \Psi_n(a_*, f) - 2 \|f\|_{2,n}^2 - 2\, \lambda \|f\|_{\mcA}^2\right)  + \inf_{f\in \mcF} \left(\Psi_n(a_*, f) + \|f\|_{2,n}^2 + \lambda \|f\|_{\mcA}^2\right)
\end{align}
Observe that since $\Psi_n(a, f)$ is a linear operator of $f$ and $\mcF$ is a symmetric class, we have:
\begin{align}
    \inf_{f\in \mcF} \left(\Psi_n(a_*, f) + \|f\|_{2,n}^2 + \lambda \|f\|_{\mcA}^2\right) =~& \inf_{f\in \mcF} \left(\Psi_n(a_*, -f) + \|f\|_{2,n}^2 + \lambda \|f\|_{\mcA}^2\right)\\
    =~& \inf_{f\in \mcF} \left(- \Psi_n(a_*, f) + \|f\|_{2,n}^2 + \lambda \|f\|_{\mcA}^2\right)\\
    =~& - \sup_{f\in \mcF} \left(\Psi_n(a_*, f) - \|f\|_{2,n}^2 - \lambda \|f\|_{\mcA}^2\right) = -\sup_{f\in \mcF} \Psi_n^{\lambda}(a_*, f)
\end{align}
Combining this with Equation~\eqref{eqn:reg-optimality} yields:
\begin{align}
    \sup_{f \in \mcF} \left(\Psi_n(\hat{a}, f) - \Psi_n(a_*, f) - \|f\|_{2,n}^2 - \lambda \|f\|_{\mcA}^2\right) \leq~& 2\, \sup_{f\in \mcF} \Psi_n^{\lambda}(a_*, f) + \mu \left(\|a_*\|_{\mcA}^2 - \|\hat{a}\|_{\mcA}^2\right)\\
    \leq~& 2\,\sup_{f\in \mcF} \Psi^{\lambda/2}(a_*, f) + \mu \left(\|a_*\|_{\mcA}^2 - \|\hat{a}\|_{\mcA}^2\right) + O\left(M^2\, \delta^2\right)
\end{align}

\paragraph{Lower bounding centered empirical sup-loss.} First observe that:
\begin{equation}
    \Psi_n(a, f) - \Psi_n(a_*, f) = \E_n[(a_*(X)-a(X)) f(X)] 
\end{equation}
Let $\Delta=a_* - \hat{a}$. Suppose that $\|\Delta\|_2\geq \delta$ and let $r = \frac{\delta}{2\|\Delta\|_2}\in [0, 1/2]$. Then observe that since $\Delta\in \mcF$ and $\mcF$ is star-convex, we also have that $r \Delta\in \mcF$. Thus
\begin{align}
    \sup_{f \in \mcF} \left(\Psi_n(\hat{a}, f) - \Psi_n(a_*, f) - \|f\|_{2,n}^2 - \lambda \|f\|_{\mcA}^2\right) \geq~& 
    \Psi_n(\hat{a}, r\Delta) - \Psi_n(a_*, r\Delta) - r^2 \|\Delta\|_{2,n}^2 - \lambda r^2 \|\Delta\|_{\mcA}^2\\
    =~& r \E_n\left[(a_*(X) - \hat{a}(X))^2\right] - r^2 \|\Delta\|_{2,n}^2 - \lambda r^2 \|\Delta\|_{\mcA}^2\\
    =~& r \|\Delta\|_{2,n}^2 - r^2 \|\Delta\|_{2,n}^2 - \lambda r^2 \|\Delta\|_{\mcA}^2\\
    \geq~& r \|\Delta\|_{2,n}^2 - r^2 \|\Delta\|_{2,n}^2 - \lambda \|\Delta\|_{\mcA}^2
\end{align}

Moreover, since $\delta_n$ upper bounds the critical radius of $\mcF_{B}$ and by Equation~\eqref{eqn:reg-pop-emp}:
\begin{align}
    r^2 \|\Delta\|_{2,n}^2
    \leq~& r^2 \left(2\|\Delta\|_2^2 + \delta^2 + \delta^2 \frac{\|\Delta\|_{\mcA}^2}{B}\right)\\
    \leq~& 2\delta^2 + \delta^2 \frac{\|\Delta\|_{\mcA}^2}{B} \leq 2\delta^2 + \lambda \|\Delta\|_{\mcA}^2
\end{align}
Thus we get:
\begin{align}
    \sup_{f \in \mcF} \left(\Psi_n(\hat{a}, f) - \Psi_n(a_*, f) - \|f\|_{2,n}^2 - \lambda \|f\|_{\mcA}^2\right) \geq~& r \|\Delta\|_{2,n}^2 - 2\delta^2 - 2\lambda \|\Delta\|_{\mcA}^2
\end{align}
Furthermore, again, since $\delta_n$ upper bounds the critical radius of $\mcF_{B}$ and by Equation~\eqref{eqn:reg-pop-emp}:
\begin{align}
    \|\Delta\|_{2,n}^2 \geq \frac{1}{2} \|\Delta\|_2^2 - \frac{\delta^2}{2B}\|\Delta\|_{\mcA}^2 - \delta^2 \geq \frac{1}{2} \|\Delta\|_2^2 - \lambda\|\Delta\|_{\mcA}^2 - \delta^2
\end{align}
Thus we have:
\begin{align}
    \sup_{f \in \mcF} \left(\Psi_n(\hat{a}, f) - \Psi_n(a_*, f) - \|f\|_{2,n}^2 - \lambda \|f\|_{\mcA}^2\right) \geq~& \frac{r}{2} \|\Delta\|_{2}^2 - 3\delta^2 - 3\lambda \|\Delta\|_{\mcA}^2\\
    \geq~& \frac{\delta}{4} \|\Delta\|_2  - 3\delta^2 - 3\lambda \|\Delta\|_{\mcA}^2
\end{align}


\paragraph{Combining upper and lower bound.} Combining the upper and lower bound on the centered population sup-loss we get that w.p. $1-3\zeta$: either $\|\Delta\|_2\leq \delta$ or:
\begin{align}
    \frac{\delta}{4} \|\Delta\|_2 \leq~& O\left(M^2\, \delta^2\right) + 2\,\sup_{f\in \mcF} \Psi^{\lambda/2}(a_*, f) + 3\lambda \|\Delta\|_{\mcA}^2 + \mu \left(\|a_*\|_{\mcA}^2 - \|\hat{a}\|_{\mcA}^2\right)
\end{align}
We now control the last part. Since $\mu \geq 6\lambda$:
\begin{align}
    3\lambda \|\Delta\|_{\mcA}^2 + \mu \left(\|a_*\|_{\mcA}^2 - \|\hat{a}\|_{\mcA}^2\right) \leq~&  6\lambda \left(\|a_*\|_{\mcA}^2 + \|\hat{a}\|_{\mcA}^2\right) + \mu \left(\|a_*\|_{\mcA}^2 - \|\hat{a}\|_{\mcA}^2\right) \leq 2 \mu \|a_*\|_{\mcA}^2
\end{align}
We can then conclude that:
\begin{equation}
    \frac{\delta}{4} \|\Delta\|_2 \leq O\left(M^2\, \delta^2\right) + 2\,\sup_{f\in \mcF} \Psi^{\lambda/2}(a_*, f) + 2 \mu \|a_*\|_{\mcA}^2
\end{equation}
Dividing over by $\delta/4$, we get:
\begin{equation}
     \|\Delta\|_2 \leq O\left(M^2\, \delta\right) + \frac{8}{\delta}\,\sup_{f\in \mcF} \Psi^{\lambda/2}(a_*, f) + 8 \frac{\mu}{\delta} \|a_*\|_{\mcA}^2
\end{equation}
Thus either $\|\Delta\|_2\leq \delta$ or the latter inequality holds. Thus in any case the latter inequality holds.

\paragraph{Upper bounding population sup-loss at minimum.} Observe that by the definition of the Riesz representer:
\begin{align}
    \sup_{f\in \mcF} \Psi^{\lambda/2}(a_*, f) =~& \sup_{f\in \mcF} \E[ (a_0(X)-a_*(X))\, f(z) ] - \frac{1}{4} \|f\|_2^2 - \frac{\lambda}{4}\|f\|_{\mcA}^2\\
    \leq~& \sup_{f\in \mcF} \E[ (a_0(X) - a_*(X))\, f(z) ] - \frac{1}{4} \|f\|_{2}^2
    = \|a_0 - a_*\|_2^2
\end{align}

\paragraph{Concluding.} Concluding we get that w.p. $1-3\zeta$:
\begin{equation}
    \|\hat{a}-a_*\|_2 \leq O\left(M^2\, \delta\right) + \frac{8}{\delta}\, \|a_*-a_0\|_2^2 + 8 \frac{\mu}{\delta} \|a_*\|_{\mcA}^2
\end{equation}
By a trinagle inequality we get:
\begin{equation}
    \|\hat{a}-a_0\|_2 \leq O\left(M^2\, \delta\right) + \frac{8}{\delta}\, \|a_*-a_0\|_2^2 + \|a_* - a_0\|_2 + 8 \frac{\mu}{\delta} \|a_*\|_{\mcA}^2
\end{equation}

Choosing $a_* = \argmin_{a\in \mcA} \|a-a_0\|_2$ and using the fact that $\delta \geq \epsilon_n$, we get:
\begin{equation}
    \|\hat{a}-a_0\|_2 \leq O\left(M^2 \delta + \|a_*-a_0\|_2 + \frac{\mu}{\delta} \|a_*\|_{\mcA}^2\right) \leq O\left(M^2 \delta + \frac{\mu}{\delta} \|a_*\|_{\mcA}^2\right)
\end{equation}
\end{proof}


\subsection{Proof of \Cref{thm:reg-main-error-2}}

\begin{proof}
By the definition of $\hat{a}$:
\begin{equation}
    0\leq \sup_{f} \Psi_n(\hat{a}, f) \leq \sup_{f} \Psi_n(a_0, f) + \lambda \left(\|a_0\|_{\mcA} - \|\hat{a}\|_{\mcA}\right)
\end{equation}
Let
\[
\delta_{n, \zeta}=\max_{i}\left(\mcR(\mcF^i) + \mcR(m\circ \mcF^i)\right) + c_0 \sqrt{\frac{\log(c_1/\zeta)}{n}}
\]
for some universal constants $c_0,c_1$. By Theorem~26.5 and 26.9 of \cite{shalev2014understanding}, and since $\mcF^i$ is a symmetric class and since $\|a_0\|_{\infty} \leq 1$, w.p. $1-\zeta$:
\begin{equation}
    \forall f\in \mcF^i: \left|\Psi_n(a_0, f) - \Psi(a_0, f)\right| \leq \delta_{n,\zeta}
\end{equation}
Since $\Psi(a_0, f)=0$ for all $f\in \mcF$, we have that, w.p. $1-\zeta$:
\begin{equation}
    \|\hat{a}\|_{\mcA} \leq \|a_0\|_{\mcA} + \delta_{n,\zeta}/\lambda 
\end{equation}
Let $B_{n,\lambda,\zeta} = (\|a_0\|_{\mcH} + \delta_{n,\zeta}/\lambda)^2$, $\mcA_B\cdot \mcF^i := \{a\cdot f: a\in \mcA_B, f\in \mcF^i\}$ and
\[
\epsilon_{n,\lambda, \zeta}=\max_{i}\left(\mcR(\mcA_{B_{n,\lambda,\zeta}}\cdot \mcF^i) + \mcR(m\circ \mcF^i)\right) + c_0 \sqrt{\frac{\log(c_1/\zeta)}{n}}
\]
for some universal constants $c_0,c_1$, then again by Theorem~26.5 and 26.9 of \cite{shalev2014understanding},
\begin{equation}
    \forall a\in \mcA_{B_{n,\lambda,\zeta}}, f\in \mcF_U^i \left|\Psi_n(a, f) - \Psi(a, f)\right| \leq \epsilon_{n,\lambda, \zeta}
\end{equation}
By a union bound over the $d$ function classes composing $\mcF$, we have that w.p. $1-2\zeta$:
\begin{equation}
    \sup_{f\in \mcF} \Psi_n(a_0, f) \leq \sup_{f\in \mcF} \Psi(a_0, f) + \delta_{n,\zeta/d} = \delta_{n,\zeta/d}
\end{equation}
and
\begin{equation}
    \sup_{f\in \mcF} \Psi_n(\hat{a}, f) \geq \sup_{f\in \mcF} \Psi(\hat{a}, f) - \epsilon_{n,\lambda, \zeta/d}
\end{equation}
If $\|\hat{a}-a_0\|_2\leq \delta_{n,\zeta}$, then the theorem follows immediately. Thus we consider the case when $\|\hat{a}-a_0\|_2\geq \delta_{n,\zeta}$. Since, by assumption, for any $a\in \mcA_{B}$ with $\|a-a_0\|\geq \delta_{n,\zeta}$ it holds that $\frac{a_0-a}{\|a_0-a\|_2}\in \spanF_{\kappa}(\mcF)$, we have $\frac{a_0 -\hat{a}}{\|a_0 - \hat{a}\|_2}=\sum_{i=1}^p w_i f_i$, with $p<\infty$, $\|w\|_1\leq \kappa$ and $f_i\in \mcF$. Thus:
\begin{align}
    \sup_{f\in \mcF} \Psi(\hat{a}, f) \geq~& \frac{1}{\kappa} \sum_{i=1}^p w_i \Psi(\hat{a}, f_i) = \frac{1}{\kappa} \Psi\left(\hat{a}, \sum_i w_i f_i\right)\\
    =~& \frac{1}{\kappa} \frac{1}{\|\hat{a}-a_0\|_2}\Psi(\hat{a}, a_0 - \hat{a})\\
    =~& \frac{1}{\kappa} \frac{1}{\|\hat{a}-a_0\|_2}\E[(a_0(X) - \hat{a}(X))^2]\\
    =~& \frac{1}{\kappa} \|\hat{a}-a_0\|_2
\end{align}
Combining all the above we have, w.p. $1-2\zeta$:
\begin{equation}
    \|\hat{a}-a_0\|_2 \leq \kappa\, \left(\epsilon_{n, \lambda, \zeta/d} + \delta_{n,\zeta/d} + \lambda \left(\|a_0\|_{\mcA} - \|\hat{a}\|_{\mcA}\right)\right)
\end{equation}
Moreover, since functions in $\mcA$ and $\mcF$ are bounded in $[-1,1]$, we have that the function $a\cdot f$ is $1$-Lipschitz with respect to the vector of functions $(a, f)$. Thus we can apply a vector version of the contraction inequality \cite{maurer2016vector} to get that:
\begin{equation}
\mcR(\mcA_{B_{n,\lambda, z}}\cdot \mcF^i) \leq 2\, \left(\mcR(\mcA_{B_{n,\lambda, z}}) + \mcR(\mcF^i)\right)
\end{equation}
Finally, we have that since $\mcA$ is star-convex:
\begin{equation}
    \mcR(\mcA_{B_{n,\lambda, z}}) \leq \sqrt{B_{n,\lambda, z}}\,\mcR(\mcA_1)
\end{equation}
Leading to the final bound of:
\begin{multline}
    \|\hat{a}-a_0\|_2 \leq \kappa \left( 2\left(\|a_0\|_{\mcA} + \delta_{n,\zeta}/\lambda\right) \mcR(\mcA_1) + 2\, \max_{i=1}^d \left(\mcR(\mcF^i) + \mcR(m\circ \mcF^i)\right) \right)\\ + \kappa \left(c_0\sqrt{\frac{\log(c_1\, d/\zeta)}{n}} + \lambda \left(\|a_0\|_{\mcA}-\|\hat{a}\|_{\mcA}\right)\right)
\end{multline}
Since $\lambda \geq \delta_{n,\zeta}$, we get the result.
\end{proof}

\subsection{Proof of \Cref{cor:sparse-linear-reg-ell1}}

\begin{proof}
Consider any $\hat{a}=\ldot{\hat{\theta}}{\cdot}\in \mcA_{B_{n,\lambda,\zeta}}$ and let $\nu=\hat{\theta} - \theta_0$, then:
\begin{align}
    \delta_{n,\zeta}/\lambda + \|\theta_0\|_1 \geq \|\hat{\theta}\|_1 =\|\theta_0 + \nu\|_1 = \|\theta_0 + \nu_S\|_1+\|\nu_{S^c}\|_1 \geq \|\theta_0\|_1 - \|\nu_S\|_1 + \|\nu_{S^c}\|_1
\end{align}
Thus:
\begin{equation}
    \|\nu_{S^c}\|_1\leq \|\nu_S\|_1 + \delta_{n,\zeta}/\lambda
\end{equation}
and $\nu$ lies in the restricted cone for which the restricted eigenvalue of $V$ holds. Moreover, since $|S|=s$:
\begin{equation}
    \|\nu\|_1 \leq 2 \|\nu_S\|_1 + \delta_{n,\zeta}/\lambda \leq 2\sqrt{s} \|\nu_S\|_2  + \delta_{n,\zeta}/\lambda \leq 2\sqrt{s}\|\nu\|_2  + \delta_{n,\zeta}/\lambda \leq 2 \sqrt{\frac{s}{\gamma} \nu^\top  V \nu}  + \delta_{n,\zeta}/\lambda 
\end{equation}
Moreover, observe that: 
\begin{equation}
    \|\hat{a}-a_0\|_2 = \sqrt{\E[ \ldot{\nu}{x}^2 ]} = \sqrt{\nu^\top  V \nu} 
\end{equation}
Thus we have:
\begin{equation}
    \frac{\hat{a}(x)-a_0(x)}{\|\hat{a}-a_0\|_2} = \sum_{i=1}^p \frac{\nu_i}{\sqrt{\nu^\top V\nu}} x_i 
\end{equation}
Thus for any $\hat{a}\in \mcA_{B_{n,\lambda,\zeta}}$, we can write $\frac{\hat{a}-a_0}{\|\hat{a}-a_0\|_2}$ as $\sum_{i=1}^p w_i f_i$, with $f_i\in \mcF$ and:
\begin{equation}
    \|w\|_1 = \frac{\|\nu\|_1}{\sqrt{\nu^\top V \nu}} \leq 2\sqrt{\frac{s}{\gamma}} + \frac{\delta_{n,\zeta}}{\lambda} \frac{1}{\|\hat{a}-a_0\|_2}.
\end{equation}
Thus: $\frac{\hat{a}-a_0}{\|\hat{a}-a_0\|_2} \in \spanF_{\kappa}(\mcF)$ for $\kappa=2\sqrt{\frac{s}{\gamma}} + \frac{\delta_{n,\zeta}}{\lambda} \frac{1}{\|\hat{a}-a_0\|_2}$.

Moreover, observe that by the triangle inequality:
\begin{equation}
    \|a_0\|_{\mcA} - \|\hat{a}\|_{\mcA} = \|\theta_0\|_1 - \|\hat{\theta}\|_1 \leq \|\theta_0-\hat{\theta}\|_1 = \|\nu\|_1 \leq 2 \sqrt{\frac{s}{\gamma} \nu^\top  V \nu}  + \delta_{n,\zeta}/\lambda 
\end{equation}
Moreover, by standard results on the Rademacher complexity of linear function classes (see e.g. Lemma~26.11 of \cite{shalev2014understanding}), we have $\mcR(\mcA_B)\leq B\sqrt{\frac{2\log(2\, p)}{n}}\max_{x\in \mcX} \|x\|_{\infty}$ and $\mcR(\mcF^i), \mcR(m\circ \mcF^i)\leq \sqrt{\frac{2\log(2)}{n}}\max_{x\in \mcX} \|x\|_{\infty}$ (the latter via the fact that each $\mcF^i$; and therefore also $m\circ \mcF^i$; contains only two elements and invoking Masart's lemma). Thus invoking \Cref{thm:reg-main-error-2}:
\begin{align}
    \|\hat{a}-a_0\|_2 \leq~& \left(2\sqrt{\frac{s}{\gamma}} + \frac{\delta_{n,\zeta}}{\lambda} \frac{1}{\|\hat{a}-a_0\|_2}\right)\cdot \left(2 (\|\theta_0\|_{1}+1) \sqrt{\frac{\log(2p)}{n}} + \delta_{n,\zeta} + \lambda \sqrt{\frac{s}{\gamma}} \|\hat{a}-a_0\|_2\right)
\end{align} 
The right hand side is upper bounded by the sum of the following four terms:
\begin{align}
    Q_1 :=~& 2\sqrt{\frac{s}{\gamma}} \left(2(\|\theta_0\|_1+1) \sqrt{\frac{\log(2p)}{n}} + \delta_{n,\zeta}\right)\\
    Q_2 :=~& \left(\frac{\delta_{n,\zeta}}{\lambda} \frac{1}{\|\hat{a}-a_0\|_2}\right)\left(2 (\|\theta_0\|_1+1) \sqrt{\frac{\log(2p)}{n}} + \delta_{n,\zeta} \right)\\
    Q_3 :=~& 2 \lambda \frac{s}{\gamma} \|\hat{a}-a_0\|_2\\
    Q_4 :=~& \delta_{n,\zeta} \sqrt{\frac{s}{\gamma}}
\end{align}
If $\|\hat{a}-a_0\|_2 \geq \sqrt{\frac{s}{\gamma}} \delta_{n,\zeta}$ and setting $\lambda \leq \frac{\gamma}{8s}$, yields:
\begin{align}
    Q_2 \leq~& 8 \frac{1}{\lambda}\sqrt{\frac{\gamma}{s}}\left(2 (\|\theta_0\|_1+1) \sqrt{\frac{\log(2p)}{n}} + \delta_{n,\zeta} \right)\\
    Q_3 \leq~& \frac{1}{4} \|\hat{a}-a_0\|_2
\end{align}
Thus bringing $Q_3$ on the left-hand-side and dividing by $3/4$, we have:
\begin{equation}
    \|\hat{a}-a_0\|_2 \leq \frac{4}{3} (Q_1 + Q_2 + Q_4) = \frac{4}{3}\max\left\{\sqrt{\frac{s}{\gamma}}, \frac{1}{\lambda} \sqrt{\frac{\gamma}{s}}\right\} \left(20\, (\|\theta_0\|_1+1) \sqrt{\frac{\log(2p)}{n}} + 11 \delta_{n,\zeta}\right)
\end{equation}
On the other hand if $\|\hat{a}-a_0\|_2\leq \sqrt{\frac{s}{\gamma}} \delta_{n,\zeta}$, then the latter inequality trivially holds. Thus it always holds.
\end{proof}
\section{Proofs from \Cref{sec:computation}}

\subsection{Proof of \Cref{prop:sparse-optimization-ell1}}

\begin{proposition}
Consider an online linear optimization algorithm over a convex strategy space $S$ and consider the OFTRL algorithm with a $1$-strongly convex regularizer with respect to some norm $\|\cdot\|$ on space $S$:
\begin{equation}
    f_t = \argmin_{f \in S} f^\top \left(\sum_{\tau\leq t} \ell_{\tau} + \ell_t\right) + \frac{1}{\eta} R(f)
\end{equation}
Let $\|\cdot\|_*$ denote the dual norm of $\|\cdot\|$ and $R=\sup_{f\in S} R(f) - \inf_{f\in S} R(f)$. Then for any $f^*\in S$:
\begin{equation}
    \sum_{t=1}^T (f_t-f^*)^\top \ell_t \leq \frac{R}{\eta} + \eta \sum_{t=1}^T \|\ell_t - \ell_{t-1}\|_* - \frac{1}{4\eta} \sum_{t=1}^T \|f_t - f_{t-1}\|^2
\end{equation}
\end{proposition}
\begin{proof}
The proof follows by observing that Proposition~7 in \cite{syrgkanis2015fast} holds verbatim for any convex strategy space $S$ and not necessarily the simplex.
\end{proof}

\begin{proposition}\label{prop:appendix-minimax}
Consider a minimax objective: $\min_{\theta\in \Theta} \max_{w\in W} \ell(\theta, w)$. Suppose that $\Theta, W$ are convex sets and that $\ell(\theta, w)$ is convex in $\theta$ for every $w$ and concave in $\theta$ for any $w$. Let $\|\cdot\|_\Theta$ and $\|\cdot\|_W$ be arbitrary norms in the corresponding spaces. Moreover, suppose that the following Lipschitzness properties are satisfied:
\begin{align}
    \forall \theta\in \Theta, w, w'\in W: \left\|\nabla_{\theta}\ell(\theta, w)  - \nabla_{\theta}\ell(\theta, w')\right\|_{\Theta, *} \leq L \|w-w'\|_W\\
    \forall w\in W, \theta, \theta'\in \Theta: \left\|\nabla_{w}\ell(\theta, w)  - \nabla_{w}\ell(\theta', w)\right\|_{W, *} \leq L \|\theta-\theta'\|_\Theta
\end{align}
where $\|\cdot\|_{\Theta, *}$ and $\|\cdot\|_{W, *}$ correspond to the dual norms of $\|\cdot\|_{\Theta}, \|\cdot\|_W$. Consider the algorithm where at each iteration each player updates their strategy based on:
\begin{align}
    \theta_{t+1} =~& \argmin_{\theta\in \Theta} \theta^\top \left(\sum_{\tau\leq t} \nabla_{\theta}\ell(\theta_\tau, w_\tau) + \nabla_{\theta} \ell(\theta_t, w_t)\right) + \frac{1}{\eta} R_{\min}(\theta)\\
    w_{t+1} =~& \argmax_{w\in W} w^T \left(\sum_{\tau \leq t} \nabla_{w} \ell(\theta_\tau, w_\tau) + \nabla_w \ell(\theta_t, w_t)\right) - \frac{1}{\eta} R_{\max}(w)
\end{align}
such that $R_{\min}$ is $1$-strongly convex in the set $\Theta$ with respect to norm $\|\cdot\|_\Theta$ and $R_{\max}$ is $1$-strongly convex in the set $W$ with respect to norm $\|\cdot\|_W$ and with any step-size $\eta \leq \frac{1}{4L}$. Then the parameters $\bar{\theta} = \frac{1}{T} \sum_{t=1}^T \theta_t$ and $\bar{w}=\frac{1}{T}\sum_{t=1}^T w_t$ correspond to an $\frac{2 R_*}{\eta \cdot T}$-approximate equilibrium and hence $\bar{\theta}$ is a $\frac{4 R_*}{\eta T}$-approximate solution to the minimax objective, where $R$ is defined as:
\begin{equation}
    R_* := \max\left\{ \sup_{\theta\in \Theta} R_{\min}(\theta) - \inf_{\theta\in \Theta} R_{\min}(\theta), \sup_{w\in W} R_{\max}(w)-\inf_{w\in W} R_{\max}(w)\right\}
\end{equation}
\end{proposition}
\begin{proof}
The proposition is essentially a re-statement of Theorem~25 of \cite{syrgkanis2015fast} (which in turn is an adaptation of Lemma~4 of \cite{Rakhlin2013}), specialized to the case of the OFTRL algorithm and to the case of a two-player convex-concave zero-sum game, which implies that the if the sum of regrets of players is at most $\epsilon$, then the pair of average solutions corresponds to an $\epsilon$-equilibrium (see e.g. \cite{FREUND199979} and Lemma~4 of \cite{Rakhlin2013}).
\end{proof}

\paragraph{Proof of \Cref{prop:sparse-optimization-ell1}} Let $R_E(x)=\sum_{i=1}^{2p} x_i \log(x_i)$. For the space $\Theta:=\{\rho\in \R^{2p}: \rho \geq 0, \|\rho\|_1\leq B\}$, the entropic regularizer is $\frac{1}{B}$-strongly convex with respect to the $\ell_1$ norm and hence we can set $R_{\min}(\rho)=B\, R_{E}(\rho)$. Similarly, for the space $W:=\{w\in \R^{2p}: w\geq 0, \|w\|_1=1\}$, the entropic regularizer is $1$-strongly convex with respect to the $\ell_1$ norm and thus we can set $R_{\max}(w)=R_E(w)$. For this choice of regularizers, the update rules can be easily verified to have a closed form solution provided in \Cref{prop:sparse-optimization-ell1}, by writing the Lagrangian of each OFTRL optimization problem and invoking strong duality. Further, we can verify the lipschitzness conditions. Since the dual of the $\ell_1$ norm is the $\ell_{\infty}$ norm,  $\nabla_{\rho}\ell(\rho, w) = \E_n[VV^\top] w + \lambda$ and thus:
\begin{align}
    \left\|\nabla_{\rho}\ell(\rho, w) - \nabla_{\rho}\ell(\rho, w')\right\|_{\infty} =\|\E_n[VV^\top] (w-w')\|_{\infty} \leq \|\E_n[VV^\top]\|_{\infty} \|w-w'\|_1\\
    \left\|\nabla_{w}\ell(\rho, w) - \nabla_{w}\ell(\rho', w)\right\|_{\infty} =\|\E_n[VV^\top] (\rho-\rho')\|_{\infty} \leq \|\E_n[VV^\top]\|_{\infty} \|\rho-\rho'\|_1\\
\end{align}
Thus we have $L=\|\E_n[VV^\top]\|_{\infty}$. Finally, observe that:
\begin{align}
    \sup_{\rho\in \Theta} B\, R_{E}(\rho) - \inf_{\rho\in \Theta} B\, R_E(\rho) =~& B^2 \log(B\vee 1) + B \log(2p)\\
    \sup_{w\in W} R_{E}(w) - \inf_{w\in W} R_E(w) =~& \log(2p)
\end{align}
Thus we can take $R_*=B^2 \log(B\vee 1) + (B+1) \log(2p)$. Thus if we set $\eta = \frac{1}{4\|\E_n[VV^\top]\|_{\infty}}$, then we have that after $T$ iterations, $\bar{\theta}=\bar{\rho}^+-\bar{\rho}^-$ is an $\epsilon(T)$-approximate solution to the minimax problem, with \begin{equation}
\epsilon(T)=16\|\E_n[VV^\top]\|_{\infty} \frac{4B^2 \log(B\vee 1) + (B+1) \log(2p)}{T}.
\end{equation}
Combining all the above with \Cref{prop:appendix-minimax} yields the proof of \Cref{prop:sparse-optimization-ell1}.


\subsection{Proof of \Cref{prop:oracle}}


Observe that the loss function $-\ell(a, \cdot)$ is strongly convex in $f$ with respect to the $\|\cdot\|_{2,n}$ norm, i.e.:
\begin{equation}
    -\frac{1}{2} D_{ff} \ell(a, f)[\nu, \nu] \geq \E_n[\nu(X)^2]
\end{equation}
and that the difference:
\begin{equation}
    \ell(a, f) - \ell(a', f) = \E_n[(a(X)-a'(X))\cdot f(X)]
\end{equation}
is an $\|a-a'\|_{2,n}$-Lipschitz function with respect to the $\ell_{2,n}$ norm (via a Cauchy-Schwarz inequality). Thus we can conclude that (see Lemma~1 in \cite{syrgkanis}):
\begin{equation}
    \|f_t - f_{t+1}\|_{2,n} \leq \|\bar{a}_{<t} - \bar{a}_{<t+1}\|_{2, n}
\end{equation}
Moreover, we know that the cumulative regret of the FTL algorithm is at most (see proof of Theorem~1 in \cite{syrgkanis}):
\begin{equation}
    R(T) \leq \sum_{t=1}^T \left|\ell(a_t, f_t) - \ell(a_t, f_{t+1})\right|
\end{equation}
Since $\|a_t\|_{\infty}, \|f_t\|_{\infty} \leq 1$, each summand of the latter is upper bounded by:
\begin{equation}
    \left|\E_n[m(Z; f_t - f_{t+1})]\right| + 3\|f_t - f_{t+1}\|_{1,n}
\end{equation}
We will assume that the empirical operator $E_n[m(Z; f)]$ is also a bounded linear operator, with a bound of $M_n$. Thus we have: 
\begin{equation}
    \left|\E_n[m(Z; f_t - f_{t+1})]\right| \leq M_n \|f_t - f_{t+1}\|_{2,n}
\end{equation}
Thus overall we get:
\begin{equation}
     \left|\ell(a_t, f_t) - \ell(a_t, f_{t+1})\right| \leq (M_n+3) \|f_t - f_{t+1}\|_{2,n} \leq (M_n+3) \|\bar{a}_{<t} - \bar{a}_{<t+1}\|_{2,n} \leq \frac{2\,(M_n+3)}{t}
\end{equation}
where we used the fact that $\left|\bar{a}_{<t}(X) - \bar{a}_{<t+1}(X)\right|\leq \frac{2}{t}$, since $\|a\|_{\infty}\leq 1$. Thus we conclude that:
\begin{equation}
    R(T) \leq 2\,(M_n+3)\sum_{t=1}^T \frac{1}{t} = O(M_n\, \log(T))
\end{equation}
Thus after $T=\Theta\left(\frac{M_n\, \log(1/\epsilon)}{\epsilon}\right)$ iterations, of the algorithm, the $f$-player has regret of at most $\epsilon$. By standard results in solving convex-concave zero-sum games, this then implies that the average solutions: $f_*=\frac{1}{T}\sum_{t=1}^T f_t$ and $a_* = \frac{1}{T} \sum_{t=1}^T a_t$ are an $\epsilon$-equilibrium and therefore also that $a_*$ is an $\epsilon$-approximate solution to the minimax problem. This concludes the proof of the proposition.

\subsection{Proof of Proposition~\ref{prop:kernel_matrices}}

\begin{proof}
For example for ATE
\begin{align}
    [K^{(3)}]_{ij}&=[\Phi^{(m)}\Phi']_{ij} \\
    &=\langle M^*\phi(x_i),\phi(x_j)\rangle\\
    &=\langle \phi(x_i),M\phi(x_j)\rangle \\
    &= \langle \phi(d_i,w_i),\phi(1,w_j)-\phi(0,w_j)\rangle \\
    &=k((d_i,w_i),(1,w_j))-k((d_i,w_i),(0,w_j))
\end{align}
Likewise
\begin{align}
    [K^{(4)}]_{ij}&=[\Phi^{(m)}(\Phi^{(m)})']_{ij}\\
    &=\langle M^*\phi(x_i),M^* \phi(x_j)\rangle\\
    &=\langle \phi(x_i),M M^* \phi(x_j)\rangle\\
    &=\langle \phi(x_i),M^* \phi(1,w_j)-M^*\phi(0,w_j)\rangle \\
    &=\langle M\phi(x_i),\phi(1,w_j)-\phi(0,w_j) \rangle \\
    &=\langle \phi(1,w_i)-\phi(0,w_i),\phi(1,w_j)-\phi(0,w_j) \rangle \\
    &=k((1,w_i),(1,w_j))-k((1,w_i),(0,w_j))-k((0,w_i),(1,w_j))+k((0,w_i),(0,w_j))
\end{align}
\end{proof}

\subsection{Proof of Proposition~\ref{prop:rep1}}

\begin{proof}
Write the objective as 
$$
\mathcal{E}_{1}(f):=\frac{1}{n}\sum_{i=1}^n \langle f,M^*\phi(x_i)\rangle_{\mathcal{H}}-a(x_i)\langle f,\phi(x_i)\rangle_{\mathcal{H}}-\langle f,\phi(x_i)\rangle_{\mathcal{H}}^2-\lambda\|f\|^2_{\mathcal{H}}
$$
Recall that for an RKHS, evaluation is a continuous functional represented as the inner product with the feature map. Due to the ridge penalty, the stated objective is coercive and strongly convex w.r.t $f$. Hence it has a unique maximizer $\hat{f}$ that obtains the maximum.

Write $\hat{f}=\hat{f}_n+\hat{f}^{\perp}_n$ where $\hat{f}_n\in row(\Psi)$ and $\hat{f}_n^{\perp}\in null(\Psi)$. Substituting this decomposition of $\hat{f}$ into the objective, we see that
$$
\mathcal{E}_{1}(\hat{f})=\mathcal{E}_{1}(\hat{f}_n)-\lambda \|\hat{f}_n^{\perp}\|^2_{\mathcal{H}}
$$
Therefore
$$
\mathcal{E}_{1}(\hat{f})\leq \mathcal{E}_{1}(\hat{f}_n)
$$
Since $\hat{f}$ is the unique maximizer, $\hat{f}=\hat{f}_n$.
\end{proof}

\subsection{Proof of Proposition~\ref{prop:closed1}}

\begin{proof}
Write the objective as
\begin{align}
    \mathcal{E}_1(f)&= \frac{1}{n}\sum_{i=1}^n \langle Mf,\phi(x_i) \rangle -\langle a,\phi(x_i)\rangle \langle f,\phi(x_i)\rangle -\langle f,\phi(x_i)\rangle^2-\lambda \langle f,f \rangle   \\
    &= f' M' \hat{\mu} -f'\hat{T} a- f' \hat{T} f-\lambda f' f
\end{align}
where $\hat{\mu}:=\frac{1}{n}\sum_{i=1}^n \phi(x_i)$ and $\hat{T}:=\frac{1}{n}\sum_{i=1}^n \phi(x_i)\otimes \phi(x_i)$. Appealing to the representer theorem
\begin{align}
    \mathcal{E}_1(\gamma)&= \gamma'\Psi M' \hat{\mu} -\gamma'\Psi\hat{T} a- \gamma'\Psi \hat{T} \Psi'\gamma-\lambda \gamma'\Psi \Psi'\gamma \\
    &= \gamma'\Psi M' \hat{\mu} -\frac{1}{n}\gamma'\begin{bmatrix}K^{(1)} \\ K^{(3)} \end{bmatrix}\Phi a- \frac{1}{n}\gamma'\begin{bmatrix} K^{(1)}K^{(1)} & K^{(1)} K^{(2)} \\ K^{(3)} K^{(1)} & K^{(3)}K^{(2)}\end{bmatrix}\gamma-\lambda \gamma'K\gamma
\end{align}
The FOC yields
$$
\Psi M' \hat{\mu} -\frac{1}{n}\begin{bmatrix}K^{(1)} \\ K^{(3)} \end{bmatrix}\Phi a- \frac{2}{n}\begin{bmatrix} K^{(1)}K^{(1)} & K^{(1)} K^{(2)} \\ K^{(3)} K^{(1)} & K^{(3)}K^{(2)}\end{bmatrix}\hat{\gamma}-2\lambda K\hat{\gamma}=0
$$
Hence
\begin{align}
    \hat{\gamma}&=\frac{1}{2}\left[\frac{1}{n}\begin{bmatrix} K^{(1)}K^{(1)} & K^{(1)} K^{(2)} \\ K^{(3)} K^{(1)} & K^{(3)}K^{(2)}\end{bmatrix}+\lambda K\right]^{-1}\left[\Psi M' \hat{\mu} -\frac{1}{n}\begin{bmatrix}K^{(1)} \\ K^{(3)} \end{bmatrix}\Phi a\right] \\
    &=
\frac{1}{2}\left[\begin{bmatrix} K^{(1)}K^{(1)} & K^{(1)} K^{(2)} \\ K^{(3)} K^{(1)} & K^{(3)}K^{(2)}\end{bmatrix}+n\lambda K\right]^{-1}\left[n\Psi M' \hat{\mu} -\begin{bmatrix}K^{(1)} \\ K^{(3)} \end{bmatrix}\Phi a\right]
\end{align}

\end{proof}

\subsection{Proof of Proposition~\ref{prop:rep2}}

\begin{proof}
Observe that
\begin{align}
    \hat{f}(x)&=\langle \hat{f} ,\phi(x) \rangle=\phi(x)'\Psi' \hat{\gamma}=\frac{1}{2}\phi(x)'\Psi'\Delta^{-1}\left[n\Psi M' \hat{\mu} -\begin{bmatrix}K^{(1)} \\ K^{(3)} \end{bmatrix}\Phi a\right] \\
    m(x;\hat{f})&=\langle M\hat{f},\phi(x) \rangle =\frac{1}{2} \phi(x)'M\Psi' \Delta^{-1}\left[n\Psi M' \hat{\mu} -\begin{bmatrix}K^{(1)} \\ K^{(3)} \end{bmatrix}\Phi a\right] \\
   \|\hat{f}\|_{\mathcal{H}}^2 &=\hat{\gamma}'\Psi \Psi'\hat{\gamma}
   = \frac{1}{4}  \Delta^{-1}\left[n\Psi M' \hat{\mu} -\begin{bmatrix}K^{(1)} \\ K^{(3)} \end{bmatrix}\Phi a\right]'
   \Delta^{-1}
   K  
   \Delta^{-1}\left[n\Psi M' \hat{\mu} -\begin{bmatrix}K^{(1)} \\ K^{(3)} \end{bmatrix}\Phi a\right]
\end{align}

Write the objective as 
\begin{align}
    \mathcal{E}_{2}(a)&=\frac{1}{n}\sum_{i=1}^n m(x_i;\hat{f})-\langle a,\phi(x_i)\rangle \hat{f}(x_i)- \hat{f}(x_i)^2-\lambda\|\hat{f}\|^2_{\mathcal{H}}+\mu\|a\|^2_{\mathcal{H}}
\end{align}
where the various terms involving $\hat{f}$ \textit{only} depend on $a$ in the form $\Phi a$. Due to the ridge penalty, the stated objective is coercive and strongly convex w.r.t $a$. Hence it has a unique maximizer $\hat{a}$ that obtains the maximum.

Write $\hat{a}=\hat{a}_n+\hat{a}^{\perp}_n$ where $\hat{a}_n\in row(\Phi)$ and $\hat{a}_n^{\perp}\in null(\Phi)$. Substituting this decomposition of $\hat{a}$ into the objective, we see that
$$
\mathcal{E}_{2}(\hat{a})=\mathcal{E}_{2}(\hat{a}_n)+\mu \|\hat{a}_n^{\perp}\|^2_{\mathcal{H}}
$$
Therefore
$$
\mathcal{E}_{2}(\hat{a})\geq \mathcal{E}_{2}(\hat{a}_n)
$$
Since $\hat{a}$ is the unique minimizer, $\hat{a}=\hat{a}_n$.
\end{proof}

\subsection{Proof of Proposition~\ref{prop:closed2}}

\begin{proof}
Write the objective as 
\begin{align}
    \mathcal{E}_2(a)&=\hat{f}' M' \hat{\mu} -\hat{f}'\hat{T} a- \hat{f}' \hat{T} \hat{f}-\lambda \hat{f}' \hat{f}+\mu a'a \\
    %
    \mathcal{E}_2(\beta)
    %
    &=\hat{\gamma}'\Psi M'\hat{\mu}-\hat{\gamma}'\Psi\hat{T}\Phi'\beta- \hat{\gamma}'\Psi\hat{T}\Psi'\hat{\gamma}-\lambda \hat{\gamma}'\Psi\Psi'\hat{\gamma}+\mu \beta'\Phi\Phi'\beta \\
    %
    &=\hat{\gamma}'\Psi M'\hat{\mu}
    -\frac{1}{n}\hat{\gamma}'\begin{bmatrix} K^{(1)} K^{(1)} \\ K^{(3)} K^{(1)}\end{bmatrix}\beta
    - \frac{1}{n}\hat{\gamma}'\begin{bmatrix} K^{(1)}K^{(1)} & K^{(1)} K^{(2)} \\ K^{(3)} K^{(1)} & K^{(3)}K^{(2)}\end{bmatrix}\hat{\gamma}
    -\lambda \hat{\gamma}'K\hat{\gamma}
    +\mu \beta'K^{(1)}\beta  \\
    &=\sum_{j=1}^5 E_j
\end{align}
where 
\begin{align}
    E_1&=\hat{\gamma}'\Psi M'\hat{\mu} \\
    E_2&= -\frac{1}{n}\hat{\gamma}'\begin{bmatrix} K^{(1)} K^{(1)} \\ K^{(3)} K^{(1)}\end{bmatrix}\beta \\
    E_3&=- \frac{1}{n}\hat{\gamma}'\begin{bmatrix} K^{(1)}K^{(1)} & K^{(1)} K^{(2)} \\ K^{(3)} K^{(1)} & K^{(3)}K^{(2)}\end{bmatrix}\hat{\gamma} \\
    E_4&= -\lambda \hat{\gamma}'K\hat{\gamma} \\
    E_5&=\mu \beta'K^{(1)}\beta
\end{align}

Recall that
$$
\hat{\gamma}
=\frac{1}{2}\Delta^{-1}\left[n\Psi M' \hat{\mu} -\begin{bmatrix}K^{(1)} \\ K^{(3)} \end{bmatrix}\Phi a\right]
=\frac{1}{2}\Delta^{-1}
\left[n\Psi M' \hat{\mu} -\begin{bmatrix}K^{(1)}K^{(1)} \\ K^{(3)}K^{(1)} \end{bmatrix} \beta\right]
$$
Hence
$$
\hat{\gamma}'=\frac{1}{2} \left[n \hat{\mu}' M \Psi' -\beta'\begin{bmatrix}K^{(1)}K^{(1)} \\ K^{(3)}K^{(1)} \end{bmatrix}' \right] \Delta^{-1}
$$
We analyze each term
\begin{enumerate}
    \item $E_1$
    
    $$
    E_1=\frac{1}{2} \left[n \hat{\mu}' M \Psi' -\beta'\begin{bmatrix}K^{(1)}K^{(1)} \\ K^{(3)}K^{(1)} \end{bmatrix}'\right] \Delta^{-1}\Psi M'\hat{\mu}
    $$
    Hence
    $$
    \frac{\partial E_1}{\partial \beta}=-\frac{1}{2} \begin{bmatrix}K^{(1)}K^{(1)} \\ K^{(3)}K^{(1)} \end{bmatrix}'\Delta^{-1}\Psi M'\hat{\mu}
    $$
    
    \item $E_2$
    
   \begin{align}
        E_2&=\frac{1}{2n} \left[\beta' \begin{bmatrix}K^{(1)}K^{(1)} \\ K^{(3)}K^{(1)} \end{bmatrix}'-n \hat{\mu}' M \Psi'\right] \Delta^{-1}\begin{bmatrix} K^{(1)} K^{(1)} \\ K^{(3)} K^{(1)}\end{bmatrix}\beta \\
        &= \frac{1}{2n} \beta'\begin{bmatrix}K^{(1)}K^{(1)} \\ K^{(3)}K^{(1)} \end{bmatrix}'\Delta^{-1}\begin{bmatrix} K^{(1)} K^{(1)} \\ K^{(3)} K^{(1)}\end{bmatrix}\beta-\frac{1}{2} \hat{\mu}' M \Psi'\Delta^{-1}\begin{bmatrix} K^{(1)} K^{(1)} \\ K^{(3)} K^{(1)}\end{bmatrix}\beta
   \end{align}
 Hence
  $$
    \frac{\partial E_2}{\partial \beta}=\frac{1}{n} \begin{bmatrix} K^{(1)} K^{(1)} \\ K^{(3)} K^{(1)}\end{bmatrix}'\Delta^{-1}\begin{bmatrix} K^{(1)} K^{(1)} \\ K^{(3)} K^{(1)}\end{bmatrix}\beta-\frac{1}{2}\begin{bmatrix} K^{(1)} K^{(1)} \\ K^{(3)} K^{(1)}\end{bmatrix}'\Delta^{-1} \Psi M'\hat{\mu}
    $$
    \item $E_3$
    
    \begin{align}
        E_3&=-\frac{1}{4n} 
        \left[n\Psi M' \hat{\mu} -\begin{bmatrix}K^{(1)}K^{(1)} \\ K^{(3)}K^{(1)} \end{bmatrix} \beta\right]'
        \Delta^{-1}
       \begin{bmatrix} K^{(1)}K^{(1)} & K^{(1)} K^{(2)} \\ K^{(3)} K^{(1)} & K^{(3)}K^{(2)}\end{bmatrix}
        \Delta^{-1}
\left[n\Psi M' \hat{\mu} -\begin{bmatrix}K^{(1)}K^{(1)} \\ K^{(3)}K^{(1)} \end{bmatrix} \beta\right]
    \end{align}
Note that
$$
\frac{\partial }{\partial s} [x-As]'W[x-As]=-2A'W(x-As)
$$
Therefore
$$
\frac{\partial E_3}{\partial \beta}=\frac{1}{2n}\begin{bmatrix}K^{(1)}K^{(1)} \\ K^{(3)}K^{(1)} \end{bmatrix}'    
\Delta^{-1}
       \begin{bmatrix} K^{(1)}K^{(1)} & K^{(1)} K^{(2)} \\ K^{(3)} K^{(1)} & K^{(3)}K^{(2)}\end{bmatrix}
        \Delta^{-1}
        \left[n\Psi M' \hat{\mu} -\begin{bmatrix}K^{(1)}K^{(1)} \\ K^{(3)}K^{(1)} \end{bmatrix} \beta\right]
$$
    
    \item $E_4$
    
     \begin{align}
        E_4&=-\frac{\lambda}{4} 
        \left[n\Psi M' \hat{\mu} -\begin{bmatrix}K^{(1)}K^{(1)} \\ K^{(3)}K^{(1)} \end{bmatrix} \beta\right]'
        \Delta^{-1}
        K
        \Delta^{-1}
\left[n\Psi M' \hat{\mu} -\begin{bmatrix}K^{(1)}K^{(1)} \\ K^{(3)}K^{(1)} \end{bmatrix} \beta\right]
    \end{align}
Note that
$$
\frac{\partial }{\partial s} [x-As]'W[x-As]=-2A'W(x-As)
$$
Therefore
$$
\frac{\partial E_4}{\partial \beta}=\frac{\lambda}{2}\begin{bmatrix}K^{(1)}K^{(1)} \\ K^{(3)}K^{(1)} \end{bmatrix}'   
\Delta^{-1}
       K
        \Delta^{-1}
        \left[n\Psi M' \hat{\mu} -\begin{bmatrix}K^{(1)}K^{(1)} \\ K^{(3)}K^{(1)} \end{bmatrix} \beta\right]
$$
    
    \item  $E_5$
    
    $$
    \frac{\partial E_1}{\partial \beta_1}=2\mu\cdot K^{(1)}\beta
    $$
    
\end{enumerate}

Collecting these results gives the FOC
\begin{align}
    0&= -\frac{1}{2} \begin{bmatrix}K^{(1)}K^{(1)} \\ K^{(3)}K^{(1)} \end{bmatrix}'\Delta^{-1}\Psi M'\hat{\mu} \\
    %
    &+\frac{1}{n} \begin{bmatrix} K^{(1)} K^{(1)} \\ K^{(3)} K^{(1)}\end{bmatrix}'\Delta^{-1}\begin{bmatrix} K^{(1)} K^{(1)} \\ K^{(3)} K^{(1)}\end{bmatrix}\beta-\frac{1}{2}\begin{bmatrix} K^{(1)} K^{(1)} \\ K^{(3)} K^{(1)}\end{bmatrix}'\Delta^{-1} \Psi M'\hat{\mu} \\
    %
    &+\frac{1}{2n}\begin{bmatrix}K^{(1)}K^{(1)} \\ K^{(3)}K^{(1)} \end{bmatrix}'    
\Delta^{-1}
       \begin{bmatrix} K^{(1)}K^{(1)} & K^{(1)} K^{(2)} \\ K^{(3)} K^{(1)} & K^{(3)}K^{(2)}\end{bmatrix}
        \Delta^{-1}
        \left[n\Psi M' \hat{\mu} -\begin{bmatrix}K^{(1)}K^{(1)} \\ K^{(3)}K^{(1)} \end{bmatrix} \beta\right] \\
        %
        &+\frac{\lambda}{2}\begin{bmatrix}K^{(1)}K^{(1)} \\ K^{(3)}K^{(1)} \end{bmatrix}'   
\Delta^{-1}
       K
        \Delta^{-1}
        \left[n\Psi M' \hat{\mu} -\begin{bmatrix}K^{(1)}K^{(1)} \\ K^{(3)}K^{(1)} \end{bmatrix} \beta\right] \\
        %
        &+2\mu\cdot K_{XX}\hat{\beta}
\end{align}

Grouping terms
\begin{align}
    &\begin{bmatrix}K^{(1)}K^{(1)} \\ K^{(3)}K^{(1)} \end{bmatrix}'\Delta^{-1}\Psi M'\hat{\mu} \\
    &-\frac{1}{2n}\begin{bmatrix}K^{(1)}K^{(1)} \\ K^{(3)}K^{(1)} \end{bmatrix}'    
\Delta^{-1}
       \begin{bmatrix} K^{(1)}K^{(1)} & K^{(1)} K^{(2)} \\ K^{(3)} K^{(1)} & K^{(3)}K^{(2)}\end{bmatrix}
        \Delta^{-1}
        n\Psi M' \hat{\mu}  \\
    &-\frac{\lambda}{2}\begin{bmatrix}K^{(1)}K^{(1)} \\ K^{(3)}K^{(1)} \end{bmatrix}'   
\Delta^{-1}
       K
        \Delta^{-1}
        n\Psi M' \hat{\mu} \\
    &= 
    \frac{1}{n} \begin{bmatrix} K^{(1)} K^{(1)} \\ K^{(3)} K^{(1)}\end{bmatrix}'\Delta^{-1}\begin{bmatrix} K^{(1)} K^{(1)} \\ K^{(3)} K^{(1)}\end{bmatrix}\beta
    \\
    &-\frac{1}{2n}\begin{bmatrix}K^{(1)}K^{(1)} \\ K^{(3)}K^{(1)} \end{bmatrix}'    
\Delta^{-1}
       \begin{bmatrix} K^{(1)}K^{(1)} & K^{(1)} K^{(2)} \\ K^{(3)} K^{(1)} & K^{(3)}K^{(2)}\end{bmatrix}
        \Delta^{-1}
        \begin{bmatrix}K^{(1)}K^{(1)} \\ K^{(3)}K^{(1)} \end{bmatrix} \beta  \\
    &-\frac{\lambda}{2}\begin{bmatrix}K^{(1)}K^{(1)} \\ K^{(3)}K^{(1)} \end{bmatrix}'   
\Delta^{-1}
       K
        \Delta^{-1}
        \begin{bmatrix}K^{(1)}K^{(1)} \\ K^{(3)}K^{(1)} \end{bmatrix} \beta \\
    &+2\mu\cdot K^{(1)}\beta
\end{align}

Define
$$
\Omega:=\begin{bmatrix}K^{(1)}K^{(1)} \\ K^{(3)}K^{(1)} \end{bmatrix}'
%
-\frac{1}{2}\begin{bmatrix}K^{(1)}K^{(1)} \\ K^{(3)}K^{(1)} \end{bmatrix}'    
\Delta^{-1}
       \begin{bmatrix} K^{(1)}K^{(1)} & K^{(1)} K^{(2)} \\ K^{(3)} K^{(1)} & K^{(3)}K^{(2)}\end{bmatrix}
        %
        -\frac{n\lambda}{2}\begin{bmatrix}K^{(1)}K^{(1)} \\ K^{(3)}K^{(1)} \end{bmatrix}' \Delta^{-1}K 
$$
We simplify each side of the equation
\begin{enumerate}
    \item LHS
 \begin{align}
&\left\{
\begin{bmatrix}K^{(1)}K^{(1)} \\ K^{(3)}K^{(1)} \end{bmatrix}'
%
-\frac{1}{2}\begin{bmatrix}K^{(1)}K^{(1)} \\ K^{(3)}K^{(1)} \end{bmatrix}'    
\Delta^{-1}
       \begin{bmatrix} K^{(1)}K^{(1)} & K^{(1)} K^{(2)} \\ K^{(3)} K^{(1)} & K^{(3)}K^{(2)}\end{bmatrix}
        %
        -\frac{n\lambda}{2}\begin{bmatrix}K^{(1)}K^{(1)} \\ K^{(3)}K^{(1)} \end{bmatrix}' \Delta^{-1}K 
        \right\}\Delta^{-1}\Psi M'\hat{\mu}\\
        &=\Omega\Delta^{-1}\Psi M'\hat{\mu}
 \end{align}
    \item RHS 
 \begin{align}
&\bigg\{
       \left(\frac{1}{n} \begin{bmatrix} K^{(1)} K^{(1)} \\ K^{(3)} K^{(1)}\end{bmatrix}'
       -\frac{1}{2n}\begin{bmatrix}K^{(1)}K^{(1)} \\ K^{(3)}K^{(1)} \end{bmatrix}'    
\Delta^{-1}
       \begin{bmatrix} K^{(1)}K^{(1)} & K^{(1)} K^{(2)} \\ K^{(3)} K^{(1)} & K^{(3)}K^{(2)}\end{bmatrix}-\frac{\lambda}{2}\begin{bmatrix}K^{(1)}K^{(1)} \\ K^{(3)}K^{(1)} \end{bmatrix}'   
\Delta^{-1}
       K\right) \Delta^{-1}
        \begin{bmatrix} K^{(1)} K^{(1)} \\ K^{(3)} K^{(1)}\end{bmatrix} \\
    &\quad +2\mu\cdot K_{XX}\bigg\}\hat{\beta}\\
    &=\bigg\{
       \frac{1}{n}\Omega  \Delta^{-1}
        \begin{bmatrix} K^{(1)} K^{(1)} \\ K^{(3)} K^{(1)}\end{bmatrix}
    +2\mu\cdot K^{(1)}\bigg\}\hat{\beta}
 \end{align}
\end{enumerate}
\end{proof}

\subsection{Proof of Corollary~\ref{cor:RKHS}}

\begin{proof}
\begin{align}
    \hat{a}(x)&=\langle \hat{a}, \phi(x)\rangle  \\
    &=\phi(x)'\Phi'\hat{\beta} \\
    &=K_{xX}\bigg\{
       \frac{1}{n}\Omega  \Delta^{-1}
        \begin{bmatrix} K^{(1)} K^{(1)} \\ K^{(3)} K^{(1)}\end{bmatrix}
    +2\mu\cdot K^{(1)}\bigg\}^{-1}\Omega\Delta^{-1}\Psi M'\hat{\mu}
\end{align}

What remains is an account of how to evaluate $V:=\Psi M' \hat{\mu}\in\mathbb{R}^{2n}$.

There are two cases
\begin{enumerate}
    \item $j\in [n]$
    
    Observe that the $j$-th element of $V$ is
$$
v_j=\phi(x_j)' M' \hat{\mu}=\frac{1}{n}\sum_{i=1}^n\phi(x_j)' M' \phi(x_i)
$$
Moreover
$$
\phi(x_j)' M' \phi(x_i)= \langle \phi(x_j) ,M^*\phi(x_i) \rangle=[K^{(2)}]_{ji}
$$
Therefore 
$$
v_j=\frac{1}{n}\sum_{i=1}^n   [K^{(2)}]_{ji}
$$
    
    \item $j\in\{n+1,...,2n\}$
    
     Observe that the $j$-th element of $V$ is
$$
v_j=\phi(x_j)' M M' \hat{\mu}=\frac{1}{n}\sum_{i=1}^n\phi(x_j)'M M' \phi(x_i)
$$
Moreover
$$
\phi(x_j)' M M' \phi(x_i)= \langle M^*\phi(x_j) ,M^*\phi(x_i) \rangle= [K^{(4)}]_{ji}
$$
Therefore 
$$
v_j=\frac{1}{n}\sum_{i=1}^n [K^{(4)}]_{ji}
$$
\end{enumerate}
\end{proof}

\section{Proofs from \Cref{sec:debiasing}}

\subsection{Proof of \Cref{lem:debias}}
\begin{proof}
Observe that $\theta_0 = \E[m_{a}(Z; g_0)]$ for all $a$. Moreover, we can decompose:
\begin{align}
    \hat{\theta}-\theta_0 =~& \frac{1}{n} \sum_{k=1}^K \sum_{i\in P_k} \left(m_{\hat{a}_k}(Z_i; \hat{g}) - \E_Z[m_{\hat{a}_k}(Z; \hat{g}_k)]\right) + \frac{1}{K} \sum_{k=1}^K \left(\E_Z[m_{\hat{a}_k}(Z; \hat{g}_k)] - \E_Z[m_{\hat{a}_k}(Z; g_0)]\right)\\
    =~& \frac{1}{n} \sum_{k=1}^K \sum_{i\in P_k} \left(m_{\hat{a}_k}(Z_i; \hat{g}_k) - \E_Z[m_{\hat{a}_k}(Z; \hat{g}_k)]\right) + \frac{1}{K} \sum_{k=1}^K \E_X[(a_0(X) - \hat{a}_k(X))\, (\hat{g}_k(X) - g_0(X))]
\end{align}

Thus as long as $K=\Theta(1)$ and:
\begin{equation}
    \sqrt{n} \E_X[(a_0(X) - \hat{a}_k(X))\, (\hat{g}_k(X) - g_0(X))] \rightarrow_p 0
\end{equation}
we have that:
\begin{align}
    \sqrt{n}\left(\hat{\theta}-\theta_0\right) =~& \sqrt{n} \underbrace{\frac{1}{n} \sum_{k=1}^K \sum_{i\in P_k} \left(m_{\hat{a}_k}(Z_i; \hat{g}_k) - \E_Z[m_{\hat{a}_k}(Z; \hat{g}_k)]\right)}_{A} + o_p(1)
\end{align}
Suppose that for some $a_*$ and $g_*$ (not necessarily equal to $a_0$ and $g_0$), we have that: $\|\hat{a}_k-a_*\|_2 \to_p 0$ and $\|\hat{g}_k-g_*\|_2\to_p 0$. Then we can further decompose $A$ as:
\begin{align}
    A =~& \E_n[m_{a_*}(Z; g_*)] - \E_Z[m_{a_*}(Z; g_*)] + \frac{1}{n} \sum_{k=1}^K \sum_{i\in P_k} \underbrace{m_{\hat{a}_k}(Z_i; \hat{g}_k) - m_{a_*}(Z_i; g_*) - \E_Z[m_{\hat{a}_k}(Z; \hat{g}_k) - m_{a_*}(Z; g_*)]}_{V_i}
\end{align}
Denote with: 
\begin{equation}
    B := \frac{1}{n}\sum_{k=1}^K \sum_{i\in P_k} V_i =: \frac{1}{n} \sum_{k=1}^{K} B_k
\end{equation}
As long as $n\,\E[B^2]\to 0$, then we have that $\sqrt{n} B \to_p 0$. The second moment of each $B_k$ is:
\begin{equation}
    \E\left[B_k^2\right] = \sum_{i, j\in P_k} \E[V_i V_j] = \sum_{i, j\in P_k} \E[\E[V_i V_j \mid \hat{g}_k]] = \sum_{i\in P_k} \E\left[V_i^2\right] 
\end{equation}
where in the last equality we used the fact that due to cross-fitting, for any $i\neq j$, $V_i$ is independent of $V_j$ and mean-zero, conditional on the nuisance $\hat{g}_k$ estimated on the out-of-fold data for fold $k$. 
Moreover, by Jensen's inequality with respect to $\frac{1}{K}\sum_{k=1}^K B_k$
\begin{equation}
    \E[B^2] = \E\left[\left(\frac{1}{n} \sum_{k=1}^K B_k\right)^2\right] = \frac{K^2}{n^2} \E\left[\left(\frac{1}{K} \sum_{k=1}^K B_k\right)^2\right] \leq \frac{K}{n^2} \sum_{k=1}^K \E[B_k^2] = \frac{K}{n^2} \sum_{k=1}^K \sum_{i\in P_k} \E[V_i^2] = \frac{K}{n^2} \sum_{i=1}^n \E[V_i^2]
\end{equation}
Finally, observe that $\E[V_i^2] \to_p 0$, by mean-squared-continuity of the moment and by boundedness of the Riesz representer function class, the function class $\mcG$ and the variable $Y$. More elaborately:
\begin{align}
    \E[V_i^2] \leq~& \E\left[\left(m_{\hat{a}}(Z_i; \hat{g}_k) - m_{a_*}(Z_i; g_*)\right)^2\right] \\
    \leq~& 2\,\E\left[\left(m(Z_i; \hat{g}_k) - m(Z_i; g_*)\right)^2\right] + 2\, \E[\left(\hat{a}_k(X)\,(Y - \hat{g}_k(X)) - a_*(X)\,(Y-g_*(X))\right)^2]
\end{align}
The latter can further be bounded as:
\begin{align}
    4\E[\left(a_k(X) - a_*(X)\right)^2 (Y-g_k(X))^2] + 4\E[a_*(X)^2 (g_*(X) - g_k(X))^2] \leq 4 C\, \left(\E\left[\|\hat{a}_k - a_*\|_2^2 + \|\hat{g} - g_*\|_2^2\right]\right)
\end{align}
assuming that $(Y-\hat{g}_k(X))^2 \leq C$ and $a_*(X)^2 \leq C$ a.s..
Finally, by linearity of the operator and mean-squared continuity, we have:
\begin{equation}
    \E[\left(m(Z_i; \hat{g}_k) - m(Z_i; g_*)\right)^2] = \E[\left(m(Z_i; \hat{g}_k - g_*)\right)^2] \leq M\, \E\left[\|\hat{g}_k - g_*\|_2^2\right]
\end{equation}
Thus we have:
\begin{equation}
    \E[V_i^2] \leq (2M + 4C)\, \left(\E\left[\|\hat{a}_k - a_*\|_2^2 + \|\hat{g} - g_*\|_2^2\right]\right) \to 0
\end{equation}
Thus as long as $K = \Theta(1)$, we have that:
\begin{equation}\label{eqn:crucial-normality}
    n\, \E[B^2] = \frac{K}{n} \sum_{i=1}^n \E[V_i^2] \leq (2M + 4C)\, K\, \E\left[\|\hat{g}-g_*\|_2^2 + \|\hat{a}-a_*\|_2^2\right] \to 0
\end{equation}
and we can conclude the result that:
\begin{align}
    \sqrt{n}\left(\hat{\theta} - \theta_0\right) = \sqrt{n} \left(\E_n[m_{a_*}(Z; g_*)] - \E_Z[m_{a_*}(Z; g_*)]\right) + o_p(1)
\end{align}
where the latter term can be easily argued, invoking the Central Limit Theorem, to be asymptotically normal $N(0, \sigma_*^2)$ with $\sigma_*^2 =\Var(m_{a_*}(Z; g_*))$.
\end{proof}


\subsection{Proof of Normality without Consistency}\label{sec:inconsistent}

\begin{lemma}\label{lem:debias-inconsistent}
Suppose that $K=\Theta(1)$ and that for some $a_*$ and $g_*$ (not necessarily equal to $a_0$ and $g_0$), we have that for all $k\in [K]$: $\|\hat{a}_k-a_*\|_2 \stackrel{L^2}{\to} 0$ and $\|\hat{g}_k-g_*\|_2\stackrel{L^2}{\to} 0$.
Assume that:
\begin{align}
    \forall k \in [K]: \sqrt{n}\, \E[(a_*(X) - \hat{a}_k(X))\, (\hat{g}_k(X) - g_*(X))] \rightarrow_p 0
\end{align}
and that $\hat{g}_k$ admits an asymptotically linear representation around the truth $g_0$, i.e.:
\begin{equation}
    \sqrt{|P_k|}\left(\hat{g}_k(X) - g_0(X)\right) = \frac{1}{\sqrt{|P_k|}} \sum_{i\in P_k} \psi(X, Z_i; g_0) + o_p(1)
\end{equation}
with $\E[\psi(X, Z_i; g_0)\mid X]=0$ and let:
\begin{equation}
    \sigma_*^2 :=\Var_{Z_i}(m_{a_*}(Z_i; g_*) + \E_X\left[(a_0(X) - a_*(X))\,\psi(X, Z_i; g_0)\right])
\end{equation}
 Assume that Condition~\ref{ass:strong-smooth} is satisfied and the variables $Y, g(X), a(X)$ are bounded a.s. for all $g\in \mcG$ and $a\in \mcA$. Then:
\begin{equation}
    \sqrt{n}\left(\hat{\theta} - \theta_0\right) \to_d N\left(0, \sigma_*^2\right)
\end{equation}
Similarly, if $\hat{a}_k$ has an asymptotically linear representation around the truth, then the statement above holds with:
\begin{equation}
    \sigma_*^2 :=\Var_{Z_i}(m_{a_*}(Z_i; g_*) + \E_X\left[\psi(X, Z_i; a_0)\, (g_0(X) - g_*(X))\right])
\end{equation}
\end{lemma}
\begin{proof}
Observe that $\theta_0 = \E[m_{a}(Z; g_0)]$ for all $a$. Moreover, we can decompose:
\begin{align}
    \hat{\theta}-\theta_0 =~& \frac{1}{n} \sum_{k=1}^K \sum_{i\in P_k} \left(m_{\hat{a}_k}(Z_i; \hat{g}) - \E[m_{\hat{a}_k}(Z; \hat{g}_k)]\right) + \frac{1}{K} \sum_{k=1}^K \left(\E[m_{\hat{a}_k}(Z; \hat{g}_k)] - \E[m_{\hat{a}_k}(Z; g_0)]\right)\\
    =~& \underbrace{\frac{1}{n} \sum_{k=1}^K \sum_{i\in P_k} \left(m_{\hat{a}_k}(Z_i; \hat{g}_k) - \E[m_{\hat{a}_k}(Z; \hat{g}_k)]\right)}_{A} + \underbrace{\frac{1}{K} \sum_{k=1}^K \E[(a_0(X) - \hat{a}_k(X))\, (\hat{g}_k(X) - g_0(X))]}_{C}
\end{align}
Suppose that for some $a_*$ and $g_*$ (not necessarily equal to $a_0$ and $g_0$), we have that: $\|\hat{a}_k-a_*\|_2 \to_p 0$ and $\|\hat{g}_k-g_*\|_2\to_p 0$. Then we can further decompose $A$ as:
\begin{align}
    A =~& \E_n[m_{a_*}(Z; g_*)] - \E[m_{a_*}(Z; g_*)] + \frac{1}{n} \sum_{k=1}^K \sum_{i\in P_k} \underbrace{m_{\hat{a}_k}(Z_i; \hat{g}_k) - m_{a_*}(Z_i; g_*) - \E[m_{\hat{a}_k}(Z; \hat{g}_k) - m_{a_*}(Z; g_*)]}_{V_i}
\end{align}
Denote with: 
\begin{equation}
    B := \frac{1}{n}\sum_{k=1}^K \sum_{i\in P_k} V_i =: \frac{1}{n} \sum_{k=1}^{K} B_k
\end{equation}
As long as $n\,\E[B^2]\to 0$, then we have that $\sqrt{n} B \to_p 0$. The second moment of each $B_k$ is:
\begin{equation}
    \E[B_k^2] = \sum_{i, j\in P_k} \E[V_i V_j] = \sum_{i, j\in P_k} \E[\E[V_i V_j \mid \hat{g}_k]] = \sum_{i\in P_k} \E[V_i^2] 
\end{equation}
where in the last equality we used the fact that due to cross-fitting, for any $i\neq j$, $V_i$ is independent of $V_j$ and mean-zero, conditional on the nuisance $\hat{g}_k$ estimated on the out-of-fold data for fold $k$. 
Moreover, by Jensen's inequality with respect to $\frac{1}{K}\sum_{k=1}^K B_k$
\begin{equation}
    \E[B^2] = \E\left[\left(\frac{1}{n} \sum_{k=1}^K B_k\right)^2\right] = \frac{K^2}{n^2} \E\left[\left(\frac{1}{K} \sum_{k=1}^K B_k\right)^2\right] \leq \frac{K}{n^2} \sum_{k=1}^K \E[B_k^2] = \frac{K}{n^2} \sum_{k=1}^K \sum_{i\in P_k} \E[V_i^2] = \frac{K}{n^2} \sum_{i=1}^n \E[V_i^2]
\end{equation}
Finally, observe that $\E[V_i^2] \to_p 0$, by mean-squared-continuity of the moment and by boundedness of the Riesz representer function class, the function class $\mcG$ and the variable $Y$. More elaborately:
\begin{align}
    \E[V_i^2] \leq~& \E\left[\left(m_{\hat{a}}(Z_i; \hat{g}_k) - m_{a_*}(Z_i; g_*)\right)^2\right] \\
    \leq~& 2\,\E\left[\left(m(Z_i; \hat{g}_k) - m(Z_i; g_*)\right)^2\right] + 2\, \E[\left(\hat{a}_k(X)\,(Y - \hat{g}_k(X)) - a_*(X)\,(Y-g_*(X))\right)^2]
\end{align}
The latter can further be bounded as:
\begin{align}
    4\E[\left(a_k(X) - a_*(X)\right)^2 (Y-g_k(X))^2] + 4\E[a_*(X)^2 (g_*(X) - g_k(X))^2] \leq 4 C\, \E\left[\|\hat{a}_k - a_*\|_2^2 + \|\hat{g} - g_*\|_2^2\right]
\end{align}
assuming that $(Y-\hat{g}_k(X))^2 \leq C$ and $a_*(X)^2 \leq C$ a.s..
Finally, by linearity of the operator and mean-squared continuity, we have:
\begin{equation}
    \E[\left(m(Z_i; \hat{g}_k) - m(Z_i; g_*)\right)^2] = \E[\left(m(Z_i; \hat{g}_k - g_*)\right)^2] \leq M\, \E\left[\|\hat{g}_k - g_*\|_2^2\right]
\end{equation}
Thus we have:
\begin{equation}
    \E[V_i^2] \leq (2M + 4C)\, \E\left[\|\hat{a}_k - a_*\|_2^2 + \|\hat{g} - g_*\|_2^2\right] \to 0
\end{equation}
Thus as long as $K = \Theta(1)$, we have that:
\begin{equation}\label{eqn:crucial-normality}
    n\, \E[B^2] = \frac{K}{n} \sum_{i=1}^n \E[V_i^2] \leq (2M + 4C)\, K\, \left(\|\hat{g}-g_*\|_2^2 + \|\hat{a}-a_*\|_2^2\right) \to_p 0
\end{equation}
and we can conclude the result that:
\begin{align}
    \sqrt{n}\, A = \sqrt{n} \left(\E_n[m_{a_*}(Z; g_*)] - \E[m_{a_*}(Z; g_*)]\right) + o_p(1)
\end{align}


Now we analyze term $C$. We will prove one of the two conditions in the ``or'' statement, when $\hat{g}_k$ has an asymptotically linear representation, i.e. \begin{equation}
\sqrt{|P_k|}\left(\hat{g}_k(X) - g_0(X)\right) = \frac{1}{\sqrt{|P_k|}} \sum_{i\in P_k} \psi(X, Z_i; g_0) + o_p(1)
\end{equation}
with $\E[\psi(X, Z_i; g_0)\mid X]=0$. The case when $\hat{a}_k$ is asymptotically linear can be proved analogously.

Let:
\begin{equation}
    C_k := \E[(a_0(X) - \hat{a}_k(X))\, (\hat{g}_k(X) - g_0(X))] 
\end{equation}
We can then write:
\begin{align}
    C_k =  \E[(a_*(X) - \hat{a}_k(X))\, (\hat{g}_k(X) - g_0(X))] + \E[(a_0(X) - a_*(X))\, (\hat{g}_k(X) - g_0(X))]
\end{align}
Since:
\begin{equation}
\sqrt{|P_k|}\E[(a_*(X) - \hat{a}_k(X))\, (\hat{g}_k(X) - g_0(X))]\leq \sqrt{|P_k|} \|a_* - \hat{a}_k\|_2 \, \|\hat{g}_k - g_0\|_2 = \|a_* - \hat{a}_k\|_2\, O_p(1) = o_p(1)
\end{equation}
we have that:
\begin{align}
    \sqrt{|P_k|} C_k =~& \sqrt{|P_k|}\E[(a_0(X) - a_*(X))\, (\hat{g}_k(X) - g_0(X))] + o_p(1)\\
    =~& \frac{1}{\sqrt{|P_k|}} \sum_{i\in P_k} \E_X[(a_0(X) - a_*(X))\,\psi(X, Z_i; g_0)] + o_p(1)
\end{align}
Since $K=\Theta(1)$ and $n/|P_k| \to K$, we then also have that:
\begin{align}
    \sqrt{n} C =~& \frac{\sqrt{n}}{K} \sum_{k=1}^K C_k =  \frac{\sqrt{K}}{K} \sum_{k=1}^K \sqrt{|P_k|} C_k + o(1)\\
    =~& \frac{1}{\sqrt{K}} \sum_{k=1}^K \frac{1}{\sqrt{|P_k|}} \sum_{i\in P_k} \E_X[(a_0(X) - a_*(X))\,\psi(X, Z_i; g_0)] + o_p(1)\\
    =~& \frac{1}{\sqrt{n}} \sum_{k=1}^K \sum_{i\in P_k} \E_X[(a_0(X) - a_*(X))\,\psi(X, Z_i; g_0)] + o_p(1)\\
    =~& \frac{1}{\sqrt{n}} \sum_{i\in [n]} \E_X[(a_0(X) - a_*(X))\,\psi(X, Z_i; g_0)] + o_p(1)\\
    =~& \sqrt{n} \E_n\left[\E_X[(a_0(X) - a_*(X))\,\psi(X,Z_i; g_0)]\right] + o_p(1)
\end{align}
\begin{align}
    \sqrt{n}\left(\hat{\theta}-\theta_0\right) =~& \sqrt{n} \left(\E_n\left[m_{a_*}(Z; g_*) + \E_X\left[(a_0(X) - a_*(X))\,\psi(X, Z_i; g_0)\right]\right] - \E[m_{a_*}(Z; g_*)]\right)  + o_p(1)
\end{align}
where the latter term can be easily argued, invoking the Central Limit Theorem, to be asymptotically normal $N(0, \sigma_*^2)$ with $\sigma_*^2 =\Var_{Z_i}(m_{a_*}(Z_i; g_*) + \E_X\left[(a_0(X) - a_*(X))\,\psi(X, Z_i; g_0)\right])$.
\end{proof}


\subsection{Proof of \Cref{lem:debias-nocross}}

\begin{proof}
Observe that $\theta_0 = \E[m_{a}(Z; g_0)]$ for all $a$. Moreover, we can decompose:
\begin{align}
    \hat{\theta}-\theta_0 =~& \E_n[m_{\hat{a}}(Z; \hat{g})] - \E[m_{\hat{a}}(Z; \hat{g})] + \E[m_{\hat{a}}(Z; \hat{g})] - \E[m_{\hat{a}}(Z; g_0)]\\
    =~& \E_n[m_{\hat{a}}(Z; \hat{g})] - \E[m_{\hat{a}}(Z; \hat{g})]  + \E[(a_0(X) - \hat{a}(X))\, (\hat{g}(X) - g_0(X))]
\end{align}

Thus as long as $K=\Theta(1)$ and:
\begin{equation}
    \sqrt{n} \E[(a_0(X) - \hat{a}(X))\, (\hat{g}(X) - g_0(X))] \rightarrow_p 0
\end{equation}
we have that:
\begin{align}
    \sqrt{n}\left(\hat{\theta}-\theta_0\right) =~& \sqrt{n} \underbrace{\E_n[m_{\hat{a}}(Z; \hat{g})] - \E[m_{\hat{a}}(Z; \hat{g})]}_{A} + o_p(1)
\end{align}
Suppose that for some $a_*$ and $g_*$ (not necessarily equal to $a_0$ and $g_0$), we have that: $\|\hat{a}_k-a_*\|_2 \to_p 0$ and $\|\hat{g}_k-g_*\|_2\to_p 0$. Then we can further decompose $A$ as:
\begin{align}
    A =~& \E_n[m_{a_*}(Z; g_*)] - \E[m_{a_*}(Z; g_*)] + \E_n\left[m_{\hat{a}}(Z; \hat{g}) - m_{a_*}(Z; g_*)\right] - \E[m_{\hat{a}}(Z; \hat{g}) - m_{a_*}(Z; g_*)]
\end{align}
Let $\delta_{n,\zeta}=\delta_n + c_0 \sqrt{\frac{\log(c_1/\zeta)}{n}}$, where $\delta_n$ upper bounds the critical radius of function classes $\mcG_B$ and $m \circ \mcG_B$ and $\mcA_B$, where $B$ is set such that these sets contain functions that are bounded in $[-1, 1]$. By a concentration inequality, almost identical to that of Equation~\eqref{eqn:reg-concentration}, we have that w.p. $1-\zeta$: $\forall f\in \mcF, a\in \mcA$
\begin{multline}
     \left|\E_n\left[m_{a}(Z; g) - m_{a_*}(Z; g_*)\right] - \E[m_{a}(Z; a) - m_{a_*}(Z; g_*)]\right| \\
     \leq O\left(\delta_{n,\zeta} \left(\|m\circ (g - g_*)\|_2 \| \|a\|_{\mcA} + \|a-a_*\|_2 \|g\|_{\mcG} + \|g-g_*\|_2 \|a\|_{\mcA}\right) + \delta_{n,\zeta}^2 \|a\|_{\mcA} \|g\|_{\mcG}\right)
\end{multline}
Applying the latter for $\hat{g}, \hat{a}$ and invoking the MSE continuity, w.p. $1-\zeta$:
\begin{multline}
     \left|\E_n\left[m_{\hat{a}}(Z; \hat{g}) - m_{a_*}(Z; g_*)\right] - \E[m_{\hat{a}}(Z; \hat{g}) - m_{a_*}(Z; g_*)]\right| \\
     \leq O\left(\delta_{n,\zeta} M \left(\|\hat{a}-a_*\|_2\, \|\hat{g}\|_{\mcG} + \|\hat{g}-g_*\|_2\, \|\hat{a}\|_{\mcA}\right) + \delta_{n,\zeta}^2\, \|\hat{g}\|_{\mcG} \, \|\hat{a}\|_{\mcA}\right)
\end{multline}
If we let $\delta_{n,*} = \delta_n + c_0 \sqrt{\frac{c_1 n}{n}}$, then we have that:
\begin{multline}
    \left|\E_n\left[m_{\hat{a}}(Z; \hat{g}) - m_{a_*}(Z; g_*)\right] - \E[m_{\hat{a}}(Z; \hat{g}) - m_{a_*}(Z; g_*)]\right| \\
     = O_p\left(\delta_{n,*} M \left(\|\hat{a}-a_*\|_2\, \|\hat{g}\|_{\mcG} + \|\hat{g}-g_*\|_2\, \|\hat{a}\|_{\mcA}\right) + \delta_{n,*}^2\, \|\hat{g}\|_{\mcG} \, \|\hat{a}\|_{\mcA}\right)
\end{multline}
If $\|\hat{a} - a_*\|_2, \|\hat{g}-g_*\|_2= O_p(r_n)$ and $\|\hat{a}\|_{\mcA}, \|\hat{g}\|_{\mcG} = O_p(1)$, we have that:
\begin{equation}
    \left|\E_n\left[m_{\hat{a}}(Z; \hat{g}) - m_{a_*}(Z; g_*)\right] - \E[m_{\hat{a}}(Z; \hat{g}) - m_{a_*}(Z; g_*)]\right| 
     = O_p\left(M\, \delta_{n,*} r_n  + \delta_{n,*}^2\right)
\end{equation}
Thus as long as: $\sqrt{n}\left(\delta_{n, *} r_n + \delta_{n,*}^2\right) \to 0$, we have that:
\begin{equation}
    \sqrt{n}\left|\E_n\left[m_{\hat{a}}(Z; \hat{g}) - m_{a_*}(Z; g_*)\right] - \E[m_{\hat{a}}(Z; \hat{g}) - m_{a_*}(Z; g_*)]\right| 
     = o_p(1)
\end{equation}
Thus we conclude that:
\begin{align}
    \sqrt{n}\left(\hat{\theta} - \theta_0\right) = \sqrt{n} \left(\E_n[m_{a_*}(Z; g_*)] - \E[m_{a_*}(Z; g_*)]\right) + o_p(1)
\end{align}
where the latter term can be easily argued, invoking the Central Limit Theorem, to be asymptotically normal $N(0, \sigma_*^2)$ with $\sigma_*^2 =\Var(m_{a_*}(Z; g_*))$.
\end{proof}


\subsection{Proof of \Cref{lem:debias-nocross-stability}}

\begin{proof}
Let $h=(a, g)$ and $V(Z; h)=m_{a}(Z; g) - m_{a_*}(Z; g_*) - \E[m_{a}(Z; g) - m_{a_*}(Z; g_*)]$. We argue that:
\begin{equation}
    \sqrt{n}\, \E_n\left[V(Z; \hat{h})\right] = o_p(1)
\end{equation}
The remainder of the proof is identical to the proof of \Cref{lem:debias-nocross}. For the above property it suffices to show that $n\, \E\left[\E_n\left[V(Z; \hat{h})\right]^2\right] \to 0$.

First we re-write the differences $V(Z; h) - V(Z;h')$: 
\begin{align}
    V(Z; h) - V(Z;h') 
    =~& m(Z; g-g') + (a(X) - a'(X))\, Y - a(X) g(X) + a'(X) g'(X)\\
    & - \left(\ldot{a_0}{g-g'}_2 - \ldot{a}{g}_2 + \ldot{a'}{g'}_2 + \ldot{a-a'}{g_0}_2\right)
\end{align}
By MSE continuity of the the moment and boundedness of the functions we have that:
\begin{equation}
\E\left[\left(V(Z; h) - V(Z; h')\right)^2\right] \leq c_0 \E\left[\|h(X) - h'(X)\|_{\infty}^2\right]
\end{equation}
for some constant $c_0$. Moreover, since, for every $x, y$: $x^2 \leq y^2 + |x|\, |x-y| + |y|\, |x-y|$:
\begin{align}
    \E\left[\E_n[V(Z; \hat{h})]^2\right] =~& \frac{1}{n^2} \sum_{i, j}\E\left[V(Z_i; \hat{h})  V(Z_j;\hat{h})\right]\\
    \leq~& \frac{1}{n^2}\sum_{i, j} \left(\E\left[V(Z_i;\hat{h}^{-i,j})  V(Z_j;\hat{h}^{-i,j})\right] + 2\, \E\left[\left|V(Z_i;\hat{h}^{-i,j})\right|\, \left|V(Z_j;\hat{h}^{-i,j}) - V(Z_j; \hat{h})\right|\right]\right)\\
    \leq~& \frac{1}{n^2}\sum_{i, j} \left(\E\left[V(Z_i;\hat{h}^{-i,j})  V(Z_j;\hat{h}^{-i,j})\right] + 2\, \sqrt{\E\left[V(Z_i;\hat{h}^{-i,j})^2\right]}\, \sqrt{\E\left[\left(V(Z_j;\hat{h}^{-i,j}) - V(Z_j; \hat{h})\right)^2\right]}\right)\\
    \leq~& \frac{1}{n^2}\sum_{i, j} \left(\E\left[V(Z_i;\hat{h}^{-i,j})  V(Z_j;\hat{h}^{-i,j})\right] + 2\, c_0\,  \sqrt{\E\left[V(Z_i;\hat{h}^{-i,j})^2\right]}\, \sqrt{\E\left[\|\hat{h}^{-i,j}(X_j) - \hat{h}(X_j)\|_{\infty}^2\right]}\right)\\
    \leq~& \frac{1}{n^2}\sum_{i, j} \left(\E\left[V(Z_i;\hat{h}^{-i,j})  V(Z_j;\hat{h}^{-i,j})\right] + 8\, c_0\, \beta_{n-1} \sqrt{\E\left[V(Z_i;\hat{h}^{-i,j})^2\right]}\right)
\end{align}
For every $i\neq j$ we have:
\begin{align}
    \E[V(Z_i; \hat{h}^{-i,j})  V(Z_j; \hat{h}^{-i,j})] =~& \E\left[\E\left[V(Z_i; \hat{h}^{-i})  V(Z_j; \hat{h}^{-j}) \mid \hat{h}^{-i,j}\right]\right]\\ 
    =~& \E\left[\E\left[V(Z_i;\hat{h}^{-i,j}) \mid \hat{h}^{-i,j}\right]  \E\left[V(Z_j; \hat{h}^{-i,j}) \mid \hat{h}^{-i,j}\right]\right] = 0
\end{align}
and
\begin{align}
\sqrt{\E[V(Z; \hat{h}^{-i,j})^2]} \leq~& O\left(\sqrt{\E\left[\|\hat{a}^{-i,j} - a_*\|_2^2 + \|\hat{g}^{-i,j} - g_*\|_2^2\right]}\right) = O(r_{n-2})\\
\E[V(Z; \hat{h}^{-i})^2] \leq~&  O\left(\E\left[\|\hat{a}^{-i} - a_*\|_2^2 + \|\hat{g}^{-i} - g_*\|_2^2\right]\right) = O(r_{n-1}^2)
\end{align}
Thus we get that:
\begin{equation}
    n\, \E\left[\E_n[V(Z; \hat{h})]^2\right] = \frac{1}{n} \sum_{i=1}^n \E[V(Z_i; \hat{h}^{-i})^2]  + O\left(\beta_{n-1} r_{n-2}\right)= O\left(r_{n-1}^2 + n\,\beta_{n-1} r_{n-2}\right)
\end{equation}
Thus it suffices to assume that
\begin{equation}
r_{n-1}^2 + n\,\beta_{n-1} r_{n-2}\to 0
\end{equation}
\end{proof}

\end{document}
