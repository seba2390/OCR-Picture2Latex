\section{Examples}

\subsection{Causal Inference}\label{sec:continuity}

Recall the definition of mean-squared continuity: $\exists M\geq 0$ s.t.
$$
\forall f\in \mcF: \sqrt{\E\left[m(Z;f)^2\right]} \leq M\, \|f\|_2 
$$

We verify mean-square continuity for several important functionals.
\begin{enumerate}
    \item Average treatment effect (ATE): $\theta_0=\mathbb{E}[g_0(1,W)-g_0(0,W)]$
    
    To lighten notation, let $\pi_0(w):=\mathbb{P}(D=1|W=w)$ be the propensity score.
    Assume $\pi_0(w)\in \left(\frac{1}{M},1-\frac{1}{M}\right)$ for $M\in(1,\infty)$. Then
    \begin{align}
        \E[g(1, W) - g(0, W)]^2&\leq~ 2\E[g(1,W)^2 + g(0, W)^2]\\
&\leq~ 2M \E\left[\pi_0(W)\, g(1,W)^2 + [1-\pi_0(W)]\, g(0, W)^2\right] \\
&= 2M \mathbb{E}[g(X)]^2
    \end{align}
    
    \item Average policy effect: $\theta_0=\int g_0(x)d\mu(x)$ where $\mu(x)=F_1(x)-F_0(x)$
    
    Denote the densities corresponding to distributions $(F,F_1,F_0)$ by $(f,f_1,f_0)$. Assume $\frac{f_1(x)}{f(x)}\leq \sqrt{M}$ and $\frac{f_0(x)}{f(x)}\leq \sqrt{M}$ for $M\in[0,\infty)$. In this example, $m(Z;g)=m(g)$.
    \begin{align}
        \mathbb{E}[m(Z;g)]^2&=\{m(g)\}^2 \\
        &=\left\{\int g(x)d\mu(x)\right\}^2 \\
        &=\left\{\mathbb{E}\left[ g(X)\left\{\frac{f_1(X)}{f(X)}-\frac{f_0(X)}{f(X)}\right\} \right]\right\}^2 \\
        &\leq \left\{2 \sqrt{M} \mathbb{E}|g(X)|\right\}^2 \\
        &\leq 4M \mathbb{E}[g(X)]^2
    \end{align}
    
    \item Policy effect from transporting covariates: $\theta_0=\mathbb{E}[g_0(t(X))-g_0(X)]$
    
    Denote the density of $t(X)$ by $f_t(x)$. Assume $\frac{f_t(x)}{f(x)}\leq M$ for $M\in[0,\infty)$. Then
    \begin{align}
        \mathbb{E}[g(t(X))-g(X)]^2 
        &\leq 2\mathbb{E}[g(t(X))^2+g(X)^2] \\
        &= 2\mathbb{E}\left[g(X)^2\left\{\frac{f_t(X)}{f(X)}-1\right\}\right] \\
        &\leq 2 (M+1) \mathbb{E}[g(X)]^2
    \end{align}
    \item Cross effect: $\theta_0=\mathbb{E}[Dg_0(0,W)]$
    
    Assume $\pi_0(w)<1-\frac{1}{M}$ for some $M\in(1,\infty)$. Then
    \begin{align}
        \mathbb{E}[Dg(0,W)]^2&\leq \mathbb{E}[g(0,W)]^2 \\
        &\leq M \mathbb{E}[\{1-\pi_0(W)\}g(0,W)^2] \\
        &\leq M \mathbb{E}[g(X)]^2
    \end{align}
    
    \item Regression decomposition: $\mathbb{E}[Y|D=1]-\mathbb{E}[Y|D=0]=\theta_0^{response}+\theta_0^{composition}$
    where
    \begin{align}
        \theta_0^{response}&=\mathbb{E}[g_0(1,W)|D=1]-\mathbb{E}[g_0(0,W)|D=1] \\
        \theta_0^{composition}&=\mathbb{E}[g_0(0,W)|D=1]-\mathbb{E}[g_0(0,W)|D=0]
    \end{align}
    
    Assume $\pi_0(w)<1-\frac{1}{M}$ for some $M\in(1,\infty)$. Then re-write the target parameters in terms of the cross effect.
      \begin{align}
        \theta_0^{response}&=\frac{\mathbb{E}[DY]-\mathbb{E}[Dg_0(0,W)]}{\mathbb{E}[D]} \\
        \theta_0^{composition}&=
        \frac{\mathbb{E}[D\gamma_0(0,W)]}{\mathbb{E}[D]}-\frac{\mathbb{E}[(1-D)Y]}{\mathbb{E}[1-D]}
    \end{align}
    We implement DML for the cross effect, empirical means for the population means, then delta method.
    
    \item Average treatment on the treated (ATT): $\theta_0=\mathbb{E}[g_0(1,W)|D=1]-\mathbb{E}[g_0(0,W)|D=1]$
    
    Assume $\pi_0(w)<1-\frac{1}{M}$ for some $M\in(1,\infty)$. Then re-write the target parameters in terms of the cross effect.
      \begin{align}
        \theta_0&=\frac{\mathbb{E}[DY]-\mathbb{E}[Dg_0(0,W)]}{\mathbb{E}[D]}
    \end{align}
    We implement DML for the cross effect, empirical means for the population means, then delta method.
    
    \item Local average treatment effect (LATE): $\theta_0=\frac{\mathbb{E}[g_0(1,W)-g_0(0,W)]}{\mathbb{E}[h_0(1,W)-h_0(0,W)]}$
    
    The result follows from the view of LATE as a ratio of two ATEs.
\end{enumerate}

\subsection{Asset Pricing}\label{sec:finance}

We present three proofs of the existence of the stochastic discount factor. These arguments are quoted from the excellent exposition of \cite{cochrane2009asset}.

\begin{enumerate}
   \item Marginal rate of substitution in a consumption model.
        
        Consider an investor with utility function $U(c_t,c_{t+1})=u(c_t)+\beta \mathbb{E}_t[u(c_{t+1})]$, where $u$ is period utility, $c_t$ is consumption at time $t$, and $\beta$ is a subjective discount factor. Denote by $e_t$ the original consumption level, and $\xi$ the amount of the asset the consumer buys. The consumer solves the optimization problem
        $$
        \max_{\xi} u(c_t)+\beta\mathbb{E}_t [u(c_{t+1})]\quad \text{s.t.}\quad c_t=e_t-p_t\xi,\quad c_{t+1}=e_{t+1}+x_{t+1}\xi
        $$
        Substituting constraints into the objective, the FOC yields
        $$
        p_t=\mathbb{E}_t\left[\beta \frac{u'(c_{t+1})}{u'(c_t)}x_{t+1}\right],\quad m_{t+1}=\beta \frac{u'(c_{t+1})}{u'(c_t)}
        $$
        The same FOC arises in the longer-term objective $\mathbb{E}_t \left[\sum_{j=0}^{\infty} \beta^j u(c_{t+j})\right]$.
     \item State price density in a contingent claim model with complete markets.
        
        For simplicity, consider a two-period model with $S$ possible states of nature tomorrow. A contingent claim is a security that pays one dollar in one state $s$ only tomorrow. $pc_t(s)$ is the price today of the contingent claim. In a \textit{complete market}, investors can buy any contingent claims. If there are complete contingent claims, the state price density exists, and it is equal to the contingent claim price divided by probabilities. Let $x_{t+1}(s)$ denote an asset's payoff in state of nature $s$. The asset's price must equal the value of the contingent claims of which it is a bundle. Let $\pi_{t+1}(s)$ be the probability that state $s$ occurs conditional on information available today. Then
        $$
        p_t=\sum_s pc_t(s) x_{t+1}(s)=\sum_s \pi_{t+1}(s) \frac{pc_t(s)}{\pi_{t+1}(s)} x_{t+1}(s),\quad m_{t+1}(s)=\frac{pc_t(s)}{\pi_{t+1}(s)}
        $$
    \item Pricing kernel from the law of one price.
    
    Let $\mathcal{X}$ be the set of all payoffs that investors can purchase (or the subset of tradeable payoffs used in a particular study). For example, if there are complete contingent claims to $S$ states of nature then $\mathcal{X}=\mathbb{R}^S$. More generally, markets are incomplete, so $\mathcal{X}\subset \mathbb{R}^S$. 
    
    Free portfolio formation means $x,x'\in\mathcal{X}$ implies $ax+bx'\in\mathcal{X}$ for any $a,b\in\mathbb{R}$. This assumption rules out short sales constraints, bid-ask spreads, and leverage limitations. Let $p_t(x)$ denote the price at time $t$ of the asset that delivers payoff $x$ at time $t+1$. The law of one price means $p_t(ax+bx')=ap_t(x)+bp_t(x')$. In other words, asset pricing is a linear functional over a vector space. This assumption says that investors cannot make instantaneous profits by repackaging portfolios. It would be satisfied in a market that has already reached equilibrium. 
        
Given free portfolio formation and the law of one price, there exists a unique payoff $m_{t+1}^*\in\mathcal{X}$ such that $p_t(x)=\mathbb{E}_t[m_{t+1}^* x]$ for all $x\in\mathcal{X}$. $m_{t+1}^*$ is called the \textit{mimicking portfolio}. Unless markets are complete, there are infinitely many SDFs that satisfy $p_t(x)=\mathbb{E}_t[m_{t+1} x]$ of the form $m_{t+1}=m_{t+1}^*+\epsilon$ where $\epsilon\in \mathcal{X}^{\perp}$. An incomplete market can be interpreted as a restricted model, and the mimicking portfolio can be interpreted as a minimal Riesz representer in the discussion of Section~\ref{sec:restricted}.
    
\end{enumerate}

