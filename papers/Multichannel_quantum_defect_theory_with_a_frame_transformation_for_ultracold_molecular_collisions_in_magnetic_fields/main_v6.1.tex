\documentclass[reprint,amssymb,noeprint,twocolumn,longbibliography]{revtex4-2}
%\documentclass[preprint,amssymb,noeprint,onecolumn,longbibliography]{revtex4-2}
%\def\baselinestretch{2.0}

\usepackage{graphicx}
\usepackage{amsfonts}
\usepackage{amsmath}

\usepackage[usenames,dvipsnames]{color}
\usepackage{bm}
\usepackage{amssymb}
\usepackage{braket}



\usepackage{multirow}
\usepackage{epstopdf}
\usepackage[normalem]{ulem}

\usepackage{hyperref}
\hypersetup{
colorlinks=true,
linkcolor=blue,
citecolor=blue,
filecolor=green,
urlcolor=blue,
}
\usepackage[nameinlink,capitalise]{cleveref}


\begin{document}

\title{Multichannel quantum defect theory with a frame transformation \\ for ultracold molecular collisions in magnetic fields}

%  nuclear spin entanglement via long-range electric dipolar interactions of ultracold polar molecules}
%\title{Entangling nuclear spins via electric dipole-dipole interactions of ultracold polar molecules}


\author{Masato Morita$^{1}$, Paul Brumer$^{1}$, and Timur V. Tscherbul$^{2}$}

\affiliation{$^{1}$Chemical Physics Theory Group, Department of Chemistry, and Center for Quantum Information and Quantum Control, University of Toronto, Toronto, Ontario, M5S 3H6, Canada}
\affiliation{$^{2}$Department of Physics, University of Nevada, Reno, Nevada, 89557, USA}


\date{\today}

\begin{abstract}
We extend the powerful formalism of multichannel quantum defect theory combined with a frame transformation (MQDT-FT) to ultracold molecular collisions in magnetic fields. By solving the coupled-channel equations with hyperfine and Zeeman interactions omitted at short range, MQDT-FT enables a drastically simplified description of the intricate quantum dynamics of ultracold molecular collisions in terms of a small number of short-range parameters. We apply the formalism to ultracold Mg + NH collisions in a magnetic field, achieving a 10$^4$-fold reduction in computational effort.
\end{abstract}
\maketitle

\newpage


Recent advances in cooling and trapping diatomic and polyatomic molecules have established ultracold molecular gases as an emerging platform for quantum information science, ultracold chemistry, and precision searches for new physics beyond Standard Model \cite{Carr_09,Balakrishnan:16,Bohn_17,Demille_17}.
The exquisite control over molecular degrees of freedom achieved in these experiments enables the exploration of novel regimes of ultracold chemical dynamics tunable by external electromagnetic fields \cite{Lemeshko_13,Krems_09,Dulieu_17}. 
However, the use of molecules for these applications has been hindered by rapid losses observed in ultracold molecular gases due to the formation of intermediate complexes in two-body collisions \cite{Mayle_12,Mayle_13,Christianen_19,Gregory_19,Liu_20,Gersema_21,Nichols_22,Bause_23}.
Therefore, understanding and controlling ultracold molecular collisions have long been recognized as major goals in the field \cite{Carr_09,Bohn_17}.
This, however, has proven to be an elusive task due to the enormous number of molecular states (rotational, vibrational, fine, and hyperfine) strongly coupled by highly anisotropic interactions and external fields, which results in skyrocketing computational costs of rigorous coupled channel (CC) calculations on ultracold molecular collisions \cite{Morita_19,Morita_20}. 
%The rigorous coupled channel (CC) method for solving the time-independent Schr\"{o}dinger equation is a powerful tool to study ultracold molecular collisions \cite{Morita_19,Morita_20}. However, CC calculations struggle to obtain converged results owing to the need of including all of these states, which results in skyrocketing computational costs.
While efficient basis sets have been developed to mitigate this problem \cite{Tscherbul_10,Suleimanov_12,Morita_18,Morita_20,Tscherbul_23}, fully converged CC calculations including all degrees of freedom are yet to be reported on ultracold K~+~NaK \cite{Yang_19,Wang_21},  Na~+~NaLi \cite{Son_22,Park:23b}, and Rb~+~KRb \cite{Nichols_22} collisions explored in recent experiments.  

%, which involve strongly anisotropic interaction potentials, hyperfine interactions, and external fields. 



%molecular gasses allow us to explore low energy regimes with unprecedented high resolution as well as to control the collision outcomes with remarkable efficiency by external electromagnetic fields. In turn, it makes ultracold molecules promising platforms for quantum simulation, quantum information, ultracold chemistry and new physics beyond the Standard Model \cite{Carr_09,Bohn_17,Demille_17}.  
%Deepening the understanding of ultracold chemistry provides immediate feedback to experimental technologies required in the research of ultracold chemistry itself and other applications. 
%Importance of ultracold molecular collisions in the context of recent experiments. Atom-molecule Feshbach resonances, Na-NaLi work \cite{}, Rb-CaF work \cite{Jurgilas_21,Bird_23,Frye_23}, K-NaK \cite{Wang_21} experiments.  
%CaF-CaF\cite{Mukherjee_23}
%Deepening the understanding of ultracold chemistry provides beneficial insights into the further development of experimental techniques and new applications. In this decade, experimental interest on polar alkali dimers \cite{Hu_19,Liu_21,Nichols_22,Liu_22,Gregory_21,Son_22} and laser-cooled $^2\Sigma$ radicals and their variants \cite{Shuman_10,Fitch_21,Kozyryev_17,Mitra_20,Cheuk_20,Jurgilas_21} have been exerting the rapid progress of ultracold chemistry. However, while the tuning of Feashbach resonances to control the collision outcomes is a widespread concept and observed in experiments, the mechanism in ultracold molecular collisions is largely unexplored yet \cite{Yang_19,Margulis_20,Son_22,Park_23}. Also, the problem of fast loss of molecules via forming collision complexes is an elusive \cite{Mayle_12,Mayle_13,Christianen_19,Gregory_19,Liu_20,Gersema_21,Nichols_22,Bause_23}.
%Rigorous quantum scattering calculation is one of the indispensable tools to shed light on these problems {\cite{Croft_17,Wallis_09,Kendrick_18,Morita_20,Morita_23}}. However, solving the converged coupled channel (CC) equations to rigorously obtain the scattering observables is extremely challenging for relevant collision systems because it is indispensable to treat the huge number of channels due to the electron and nuclear spin degrees of freedom and the channel coupling coming from the anisotropy of the potential and external fields \cite{Morita_19}. While the total angular momentum basis set \cite{Tscherbul_10,Tscherbul_11,Suleimanov_12,Morita_17,Morita_18,Morita_20} and TRAM basis set \cite{Tscherbul_23} have a significant advantage to reduce the computational cost in the presence of external fields, the calculation including both hyperfine interaction and external fields are yet to be reported.


A promising avenue toward resolving these difficulties could be based on multichannel quantum defect theory (MQDT), an elegant technique for solving the time-independent Schr\"{o}dinger equation based on the separation of distance and energy scales in ultracold collisions \cite{Greene_82,Mies_84,Gao_96,Gao:09,Gao:11,Mies_00,Raoult_04,Croft_11,Croft_12,Croft_13,Jisha_14,Jisha_14b}.
MQDT allows one to avoid the costly numerical procedure of solving CC equations over extended ranges of the radial coordinate $R$, collision energy, and magnetic field, leading to a substantial reduction of computational cost \cite{Croft_11,Croft_12, Croft_13,Jisha_14,Jisha_14b,Burke_98,Burke_99,Gao_05}.
However, in conventional MQDT, it is still necessary to solve CC equations in the short-range collision complex region, which is forbiddingly difficult for the atom-molecule systems interacting via deep and strongly anisotropic potentials, such as those studied in recent experiments \cite{Yang_19,Wang_21,Son_22,Park:23b,Nichols_22}.


MQDT becomes especially powerful when combined with a frame transformation (FT) approach, in which the hyperfine-Zeeman structure of colliding atoms is neglected at short range.  This results in a large reduction of computational effort and provides a physically meaningful description of ultracold atomic collisions in terms of a few short-range parameters \cite{Burke_98,Burke_99,Gao_05, Hanna_09,Idziaszek_2011,Perez-Rios_15,Li_15}. An extension of MQDT-FT to the much more computationally intensive (and less well understood) ultracold molecular collisions would thus be highly desirable. Previous theoretical work has shown that  MQDT can be successfully applied to describe ultracold atom-molecule collisions \cite{Croft_11,Croft_12,Croft_13,Jisha_14} and chemical reactions \cite{Jisha_14b}, but these calculations did not consider the essential FT aspect of MQDT, which makes it such an indispensable tool in modern ultracold atomic collision theory  \cite{Burke_98,Burke_99,Gao_05, Hanna_09,Idziaszek_2011,Perez-Rios_15,Li_15}.



Here, we show that the MQDT-FT approach can be extended to ultracold molecular collisions in a magnetic field, enabling one to drastically simplify their rigorous theoretical description. This is achieved by using compact CC basis sets at short range, which exclude the hyperfine and Zeeman interactions. These interactions are incorporated at long range via MQDT-FT boundary conditions, resulting in a complete description of ultracold atom-molecule collision dynamics across large collision energy and magnetic field ranges in terms of a small number of short-range parameters. %``universal'' parameters. 
These parameters can be used to, e.g., fit experimental observations and to obtain insight into complex molecular collision dynamics without performing expensive CC calculations. 
Our results show the potential of MQDT-FT to significantly extend the scope of ultracold atom-molecule collisions and chemical reactions amenable to rigorous quantum dynamical studies, which would facilitate the accurate characterization of atom-molecule Feshbach resonances observed in recent pioneering experiments \cite{Yang_19,Wang_21,Son_22,Park:23b}, as well as new insights into quantum chaotic behavior and microscopic interactions in atom-molecule collision complexes \cite{Croft_17}. 





{\it Theory.} We begin by briefly reviewing the key concepts of MQDT as they apply to ultracold atom-molecule collisions in a magnetic field.
The Hamiltonian of the atom-molecule collision complex is (in atomic units) \cite{Krems_04}
\begin{equation}
{
\hat{H} = - \frac{1}{2\mu R} \frac{\partial^2}{\partial R^2}R + \frac{\hat{\mathbf{l}}^2}{2\mu R^2} +\hat{H}_\text{as} + \hat{V}(R,\theta)
}.
\label{eq:Heff}
\end{equation}
Below we will use ultracold Mg($^1$S)~+~NH($\tilde{X}^3\Sigma^-$) collisions as a representative example, which served as a testbed for applying MQDT to ultracold atom-molecule collisions \cite{Croft_11}. We note, however, that our treatment can be readily generalized to collisions of atoms and molecules with a more complex internal structure, and to molecule-molecule collisions. 
In Eq.~\eqref{eq:Heff}, $\mu$ and $\hat{\mathbf{l}}$ are the reduced mass and the orbital angular momentum for the collision, $R$ is the atom-molecule distance, and  $\hat{H}_\mathrm{as}$ is the asymptotic Hamiltonian
%($\mu=9.60046$ amu), 
\begin{equation}
\hat{H}_\text{as} = 
B_{e}\,\hat{\mathbf{N}}^2+ \hat{H}_\text{fs} + \hat{H}_\text{hfs}  + \hat{H}_\text{Z},
\label{eq:Has}
\end{equation}
%where $B_\mathrm{rot}$ is the rotational constant ($B_\mathrm{rot}=16.2712$ cm$^{-1}$), $\gamma_\mathrm{sr}$ is the spin-rotation constant ( $\gamma_\mathrm{sr}=-0.0546$ cm$^{-1}$), and $\lambda_{ss}$ is the spin-spin constant ($\lambda_{ss}=0.9199$ cm$^{-1}$) \cite{Bizzocchi_2018}. 
where $B_{e}$ is the rotational constant,  $\hat{\mathbf{N}}$ is the rotational angular momentum of the molecule, $\hat{H}_\text{fs} = \gamma_\mathrm{sr}\,\hat{\mathbf{N}}\cdot\hat{\mathbf{S}}+\frac{2}{3}\lambda_\mathrm{ss}\sqrt{\frac{24\pi}{5}}\sum^{}_{q}(-1)^q\,Y_{2\,-q}(\hat{\bm{r}})[\hat{\mathbf{S}} \otimes \hat{\mathbf{S}}]_q^{(2)}$  is the intramolecular fine-structure Hamiltonian, which depends on the electron spin $\hat{\mathbf{S}}$ 
%($S=|\hat{\mathbf{S}}|=1$ for $^3\Sigma$ molecules) 
and the orientation of the molecular axis $\bm{r}$, and $\gamma_\text{sr}$ and $\lambda_\text{ss}$ are the spin-rotation and spin-spin interaction constants \cite{Wallis_09,Maykel_11}.  
% are the intramolecular spin-rotation and spin-spin constants, respectively. 
%For these constants, we employ the same values used in previous studies \cite{Maykel_11,Croft_11} (see also supplemental material \cite{SM}). 
The Zeeman interaction $\hat{H}_\text{Z} = g_S \mu_\mathrm{B} \mathbf{B}\cdot \hat{\mathbf{S}}$, where $g_S \simeq 2.002$ is the electron $g$-factor, $\mu_\mathrm{B}$ is the Bohr magneton, the magnetic field vector $\mathbf{B}$ defines the space-fixed quantization axis $z$, and the hyperfine interaction $ \hat{H}_\text{hfs}  $ is considered below. 
We use the accurate {\it ab initio} interaction potential $V(R,\theta)$ for Mg-NH \cite{MOLSCAT}, which depends on $R$ and the atom-molecule Jacobi angle $\theta$  (see the Supplemental Material for details of CC and MQDT-FT calculations \cite{SM}).

%Detailed information on the interaction potential between Mg and NH is given in the references \cite{Soldan_09,MOLSCAT} and Supplemental Material \cite{SM}. 

%atom-molecule interaction potential $V_\text{int}$. For concreteness, we use the potential Mg+NH, we use the parameter values which were used in previous studies 
%a moderately anisotropic

%\label{eq:Hzeeman}
%\begin{equation}
%\hat{\mathcal{H}}_\mathrm{Z} = 
%g_\mathrm{e} \mu_\mathrm{B} \bm{B}\cdot \hat{\bm{s}},
%\label{eq:Hzeeman}
%\end{equation}


In MQDT, the matrix solution $\bm{\Psi}$ of the Schr\"odinger equation $\hat{H}\bm{\Psi}=E\bm{\Psi}$ in the basis of eigenvectors of the asymptotic Hamiltonian \eqref{eq:Has} is given by \cite{Mies_00,Croft_11} 
%the Hamiltonian \eqref{eq:Heff}
% $\hat{H}\Psi = E\Psi$ 
\begin{equation}
\label{eq:MQDT}
\bm{\Psi}= R_{m}^{-1}\,[\bm{f}(R_{m})+\bm{g}(R_{m})\,\bm{K}^{\text{sr}}\,],
\end{equation}
%where the $N\times N$ matrix $\bm{\Psi}$ of the total wave function for the Schr\"{o}dinger equation with $\hat{\mathcal{H}}_\mathrm{0}$, and reference functions obtained by solving a set 
where the matching radius $R=R_m$ marks the boundary between the short-range and long-range regions, $E$ is the total energy, 
 $\bm{K}^{\text{sr}}$ is the short-range $K$-matrix, which includes open and weakly closed channels, and $\bm{f}(R)$ and $\bm{g}(R)$ are diagonal matrices of regular and irregular 
solutions of the one-dimensional radial Schr\"odinger equations
\begin{equation}
\label{eq:1Deq}
\left[\, - \frac{1}{2\mu} \frac{\partial^2}{\partial R^2} + U_i^\text{ref}(R)\, \right]\, \psi_i(R)\,=\, E\,\psi_i(R).
\end{equation}
Here, 
$U_i^\text{ref}(R)\, =\, V_0(R) + {l_i(l_i+1)}/{(2\mu R^2)} + E_i^{\infty}$ is the MQDT reference potential \cite{Croft_11,SM}, for the $i$-th reference channel with threshold energy $E_i^{\infty}$ and orbital angular momentum $l_i$, and $V_0(R)$  is the isotropic part of the interaction potential. 
%Here, the weakly closed channels are defined as the closed channels whose diabatic energies in the asymptotic channel basis are smaller than the total energy $E$ at $\simeq R_m$ \cite{Croft_11,Croft_12  \textcolor{magenta}{Masato, please check}. 

% We note that the basis set for the vectors and matrices in \cref{eq:shortBoundary} is a set of eigenvectors of the asymptotic Hamiltonian  $\hat{\mathcal{H}}_\mathrm{0}^\text{Asym}$ for $\hat{\mathcal{H}}_\mathrm{0}$. 
%$\bm{\Psi}_\mathrm{0}
% one defines the reference functions obtained by solving a set of one-dimensional Schr\"{o}dinger equations for the reference channels \cite{Croft_11} (see also Supplemental Material \cite{SM}). 
%Here, the reference potentials in the Schr\"{o}dinger equations are given as, 
%\begin{equation}
%U_i^\text{ref}(R)\, =\, V_\text{int}^{\lambda=0}(R) + \frac{l_i(l_i+1)}{2\mu R^2} + E_i^{\infty},
%\label{eq:refpot}
%\end{equation}
%The reference channels are classified into open and weakly closed channels \cite{Croft_11,Croft_12}.
%Once we obtain the values of regular and irregular reference functions,  $f_i(R)$ and $g_i(R)$, at the matching distance $R_\mathrm{match}$, the condition defining $\bm{K}_\mathrm{0}^{\text{SR}}$ is expressed as,


Having defined the short-range $K$-matrix, we take into account the coupling between the open and weakly closed channels by forming the matrix
%, $\bar{\bm{K}}^\text{sr}$,
%extract the open-open block of the short-range $K$-matrix by taking
%The $N_\mathrm{o} \times N_\mathrm{o}$ ($N_\mathrm{o}$: Number of open channels) matrix $\bar{\bm{K}}^\text{SR}$ 
\begin{equation}
\bar{\bm{K}}^\text{sr}=\bm{K}_\mathrm{o,o}^\text{sr}-\bm{K}_\mathrm{o,wc}^\text{sr}\,[\,\tan{\bm{\nu}}+\bm{K}_\mathrm{wc,wc}^\text{sr} \,]^{-1}\,\bm{K}_\mathrm{wc,o}^\text{sr},
\label{eq:Ksr_exact}
\end{equation}
%whose diagonal elements are
where $\tan{\bm{\nu}}$ is a diagonal matrix with elements $\tan \nu_i$ for weakly closed channels \cite{Mies_84,Mies_00,Croft_11}.
The zeros of the bound-state phase ($\tan \nu_i=0$) locate the bound state energy in the $i$-th reference potential and $\bm{K}_\mathrm{wc,wc}^\text{sr}$ introduces shifts in the resonance positions \cite{Mies_00}.
We neglect the strongly closed channels, whose coupling with open and weakly closed channels has no effect on scattering dynamics \cite{Croft_11,Croft_12,SM}. 
Here, we define strongly closed channels as those, for which $U_i^\text{ref}(R)>E$ for all $R$ \cite{SM}.

%in the range $R \geq R_m$.  
Finally, we take into account the threshold effects  using the diagonal matrices of MQDT parameters $\bm{C}$ and $\tan{\bm{\lambda}}$ for open channels to obtain the $\bar{\bm{R}}$ matrix 
\begin{equation}
\bar{\bm{R}}=\bm{C}^{-1}\,[\,({\bar{\bm{K}}^\text{sr}})^{-1}-\tan{\bm{\lambda}} \,]^{-1}\,\bm{C}^{-1},
\label{eq:Rbar}
\end{equation}
where $C_i$ determines the amplitude of the reference functions and $\lambda_i$ is the phase difference between the long-range and short-range irregular solutions \cite{Raoult_04}.
%(the deviation of $\lambda_i$ from 0 is a manifestation of the breakdown of the WKB approximation
Finally, the  transition $T$-matrix may be written as 
\begin{equation}
\bm{T}=\bm{I}-e^{i\bm{\xi}}\,[\,\bm{I}+i\bar{\bm{R}} \,]\,[\,\bm{I} - i\bar{\bm{R}} \,]^{-1}\,e^{i\bm{\xi}},
\label{eq:Smatrix}
\end{equation}
where $\bm{I}$ is the unit matrix and $\bm{\xi}$ is the diagonal matrix of phase shifts $\xi_i$ associated with the energy-normalized reference functions  \cite{Mies_84,Mies_00,Raoult_04,Croft_11,Croft_12}. The state-to-state integral cross sections are 
obtained from the transition probability $|T_{i,f}|^2$ as
$\sigma_{i \to f}=(\pi/k_i^2)|T_{i,f}|^2$, where $i$ and $f$ label the initial and final internal molecular states, $k_i=\sqrt{2\mu E_\mathrm{c}}$ is the incident wavevector, and $E_\mathrm{c}$ is the collision energy.


In the spirit of MQDT-FT for atomic collisions  \cite{Burke_98,Burke_99,Gao_05, Hanna_09,Idziaszek_2011,Perez-Rios_15,Li_15}, we seek to describe short-range quantum dynamics with a simplified Hamiltonian $\hat{H}_\mathrm{0}$ obtained by omitting the terms, which are small compared to $\hat{V}$, from $\hat{H}_\text{as}$ in \cref{eq:Has}.
% (below we will discuss several choices for $\hat{\mathcal{H}}_\mathrm{0}$).
% from  $\hat{H}_\text{as}$
%\color{magenta}
To this end, %we approximate 
the exact short-range $K$-matrix 
($\bm{K}^\text{sr}$) corresponding to the original Hamiltonian $\hat{H}$ is obtained from the approximate $K$-matrix  ($\bm{K}_\mathrm{0}^{\text{sr}}$), which  
 corresponds to the simplified Hamiltonian $\hat{H}_{0}$:
\begin{equation}\label{eq:FT}
\bm{K}^\text{sr} \simeq \bm{U}^{\dagger}\bm{K}_\mathrm{0}^{\text{sr}}\bm{U},
\end{equation}
where $\bm{U}$ is the unitary FT matrix composed of the eigenvectors of $\hat{H}_\text{as}$ in \cref{eq:Has} (see the Supplemental Material \cite{SM} for details). Note that $\bm{K}_\mathrm{0}^{\text{sr}}$ is much easier to compute than the exact $\bm{K}^\text{sr}$ % in Eq.~\eqref{eq:Ksr_exact} 
because $\hat{H}_0$ can be chosen in such a way as to exclude the small hyperfine and Zeeman interactions (see below), thus eliminating the need to use enormous CC basis sets including the electron and/or nuclear spin basis functions of the collision partners \cite{Tscherbul_23}. In addition, since $\hat{H}_0$ is independent of external fields, $\bm{K}_\mathrm{0}^{\text{sr}}$ can be efficiently computed using the total angular momentum (TAM) basis as shown below, and it is independent of $M$, the $z$ projection of $J$.

 

%$\bm{K}_\mathrm{0}^{\text{SR}}$ is the direct sum of $\bm{K}_\mathrm{0}^{\text{SR},J}$ as $\bm{K}_\mathrm{0}^\text{SR}=\bm{K}_\mathrm{0}^{\text{SR}, J=0} \oplus \bm{K}_\mathrm{0}^{\text{SR},J=1} \oplus \cdots \oplus \bm{K}_\mathrm{0}^{\text{SR},J=n} \oplus \cdots$, and 
% This is motivated by a major reduction in computational effort, which would result from  

Once the FT approximation for $\bm{K}^\text{sr}$ is implemented, the standard MQDT procedure \cite{Mies_84,Mies_00,Raoult_04,Croft_11} 
can be applied %without changes 
to obtain scattering observables as a function of collision energy and magnetic field. This procedure is computationally efficient, scaling as $O(N)$ with the total number of channels $N$. The only additional modification is that the  threshold energies $E_i^{\infty}$ should be replaced by exact threshold energies [eigenvalues of $\hat{H}_\text{as}$ in \cref{eq:Has}].
%which depend on external fields and hyperfine structure. 
%The details of our implementation of MQDT-FT are described in the Supplemental Material \cite{SM}.

%. The CC equations are solved using the log-derivative propagation \cite{Johnson_73} at short range, then we obtain a short range K-matrix, $\bm{K}_\mathrm{0}^\text{SR}$, by matching to the reference functions \cite{Mies_84, Mies_00,Croft_11} associated with the asymptotic channel states of $\hat{\mathcal{H}}_\mathrm{0}$.

% , We note that  which is computationally efficient, scaling as $O(N)$ with the total number of channels   \cite{Croft_11,Mies_00,Raoult_04}. with one additional modification that the values of threshold energies $E_i^{\infty}$ should be replaced by the accurate threshold energies [eigenvalues of $\hat{\mathcal{H}}_\text{as}$] which depend on external fields and hyperfine structure. The to the Zeeman interactions.

%The reference functions and associated QD parameters are calculated based on the reference potentials given in the form of \cref{eq:refpot} 

%MOVE BELOW: An added benefit is that we can now exploiting the conservation of the total angular momentum of the collision complex  $J$  in the absence of external fields, and the independence of collision dynamics on $M$, the $z$ projection of $J$. 
%The key idea of the MQDT-FT is 

% Fig. caption: collisions with NH molecules initially in the $M_S=1$ Zeeman sublevel of the ground rotational state. 

% then compare the results with those obtained by the rigorous CC calculation with the converged basis set (see Ref.~\cite{SM} for the detail of the CC calculation). This inelastic spin relaxation collision is a relevant process for sympathetic cooling, thus was explored previously with the CC calculation \cite{Wallis_09}. Also, the applicability of the MQDT method was reported by Croft {\it et al}. \cite{Croft_11,Croft_12}

We now apply MQDT-FT to describe the quantum dynamics of ultracold  Mg~+~NH collisions in a magnetic field, focusing on the inelastic transitions between the different Zeeman sublevels of NH($v=0, N=0$) labeled by the $z$-projection of the molecule's electron spin $M_S=-1,0,1$. These transitions cause trap loss and thus limit the efficiency of sympathetic cooling of NH molecules by Mg atoms \cite{Wallis_09}. 
For now, we neglect the hyperfine interaction ($\hat{H}_\text{as} = B_e\hat{\bm{\mathrm{N}}}^2 + \hat{H}_\text{fs}+ \hat{H}_\text{Z}$) and choose the simplified short-range Hamiltonian $\hat{H_0}$ 
%to compute $\bm{K}_\mathrm{0}^{\text{sr}}$
by neglecting the Zeeman interaction as  
%$\hat{\mathcal{H}}_{0} = \hat{\mathcal{H}} - \hat{\mathcal{H}}_\text{Z}$.
$\hat{H}_{0} = \hat{H} -\hat{H}_\text{Z}$.
Since $\hat{H}_{0}$ is field-independent, we obtain $\bm{K}_\mathrm{0}^{\text{sr}}$ by solving CC equations in the TAM 
basis  $|(NS)jlJM\rangle$, where  $\hat{\mathbf{J}}=\hat{\mathbf{l}} + \hat{\mathbf{j}}$  is the total angular momentum of the atom-molecule system and $\hat{\mathbf{j}}=\hat{\mathbf{N}} + \hat{\mathbf{S}}$ is that of the diatomic molecule.
Because $\hat{H}_{0}$ is block diagonal in $J$ and independent of $M$, 
%their number is much smaller (XXX) than that used in previous MQDT calculations on Mg~+~NH, providing an $\simeq XXX$ reduction in computational cost.
the computational cost of MQDT-FT is much smaller than that of  conventional MQDT \cite{Croft_11}. 

%On the other hand, the CC equation for MQDT is composed of 534 channels for $M=1$ with the same basis set. We note that no additional short-range CC calculations are required for MQDT-FT even for the calculations with different $M$ values as long as $J_\text{max}$ is sufficiently large. 




%%%%%%%%%%%%%%%%%%%%%%%%%%%%%%%%%%%%%%%%%%%%%%%%%%%%%%
%\begin{figure}[bh!]
\begin{figure}[t]
\begin{center}
\includegraphics[height=0.162\textheight, trim = 10 0 0 0]{Fig_1.pdf}
\end{center}
\vspace{-0.5cm}
\caption{
Probabilities for the spin relaxation transition $M_S=1$ ($l=0$) $\to$ $M^\prime_S=0$ ($l^\prime=2$) in cold  Mg~+$^{14}$NH($N=0$) collisions plotted as a function of collision energy (a) and magnetic field (b).
Solid lines -- MQDT-FT calculations, open circles -- exact CC results. }
\label{Fig_1}
\end{figure}
%%%%%%%%%%%%%%%%%%%%%%%%%%%%%%%%%%%%%%%%%%%%%%%%%%%%%%


%square of the modulus of the T-matrix element, $|T|^2$, for the
%partial wave-resolved 
\Cref{Fig_1} compares the probabilities $|T_{i\to f}|^2$ for the spin relaxation transition $M_S=1 \to 0$ between the Zeeman levels of NH ($N=0$) in cold Mg+NH collisions calculated using MQDT-FT with exact CC results.
Encouragingly, MQDT-FT provides an essentially exact description of the transition probability as a function of collision energy {\it and} external magnetic field with only 139 channels as compared to 954 channels required in standard MQDT \cite{Croft_12}, a 10$^3$-fold increase in computational efficiency.
The excellent performance of MQDT-FT illustrated in \cref{Fig_1} validates the assumption that $\hat{H}_\text{Z}$ plays a negligible role in short-range dynamics. 
%The effect of $\hat{\mathcal{H}}_\text{Z}$ is included indirectly via the long-range part.
The strong magnetic field dependence of the inelastic probability at 1~$\mu$K  thus arises entirely from long-range physics. Specifically, inelastic scattering is suppressed by the centrifugal barrier in the outgoing $d$-wave channel at low magnetic fields \cite{Volpi_02,Campbell_09}.
This is an example of physically meaningful insight into ultracold atom-molecule collisions gained from MQDT-FT simulations. 


%Previous MQDT calculations on Mg~+~NH collisions used a fully uncoupled basis set $|NM_N\rangle|SM_S\rangle|lm_l\rangle$ \cite{Wallis_09,Croft_12}, which leads to 954 coupled channels for $M=1$ with $N_\text{max}=6$. The maximum number of channels in our MQDT-FT calculation is 139 using the same rotational basis set with $N_\text{max}=6$ and $J=5$. This nearly 10-fold reduction in the number of coupled channels translates to a 1000-fold increase in computational efficiency of MQDT-FT over standard MQDT.


%We note that because collisional transitions between the $N=0$ Zeeman states in $^3\Sigma$ molecules are driven by the intramolecular spin-spin interaction \cite{Krems_03,Krems_04,Wallis_09,Campbell_09}, it is essential to retain this interaction (and hence the electron spin degrees of freedom) in $\hat{H}_0$ to describe these transitions. 
%Below we will show that this requirement can be relaxed if one is interested in the description of much stronger rotationally inelastic transitions.



%rotationally inelastic transitions.
% can go one step further and exclude even  will show that   this is no longer necessary for can be r

% because the spin-relaxation in the rotational ground state is driven by these intra-molecular spin-dependent interactions.
%due to the change of the threshold energy and the wavevector of the outgoing wave. The observed suppression of $|T|^2$ at low magnetic fields and collision energies is, 
% ($l'=2$).

%In the studies with MQDT and MQDT-FT, it has been emphasized that short-range dynamics in the collision is governed by the interaction potential $V$.  Indeed, this is the reason that we can disregard $\hat{\mathcal{H}}_\text{Z}$ in the short-range for the MQDT-FT calculations. However,




%%%%%%%%%%%%%%%%%%%%%%%%%%%%%%%%%%%%%%%%%%%%%%%%%%%%%%
%\begin{figure}[b!]
\begin{figure}[t]
\begin{center}
%\includegraphics[width=\linewidth]{Fig_2}
\includegraphics[height=0.25\textheight,keepaspectratio]{Fig_2.pdf}
\end{center}
\caption{Hyperfine energy levels of $^{15}$NH ($N=0$) at zero field (a) and as a function of magnetic field (b). 
%Some numbers displayed in the panel show the indexes of the energy levels in the ascending order of their energies at each magnetic field.
State-to-state transition probabilities $|T_{il, fl'}|^2$  plotted as a function of collision energy at $B=100$\,G (c) and $B=1000$\,G (d) for  Mg~+~NH($N=0$) collisions for $l=0 $ and $l^\prime=2$. 
Solid lines -- MQDT-FT results, symbols -- exact CC calculations. 
The initial and final states of NH are indicated next to each curve as $i \to f$ [see panel (b)].
% Transitions from the 12th state (the highest energy state in the $N=0$ manifold) are shown as $12 \to 5$ (purple), $12 \to 6$ (blue), $12 \to 7$ (green), and $12 \to 8$ (red). 
}
\label{Fig_2}
\end{figure}
%The results calculated with the coupled-channel (CC) calculation and the MQDT-FT method are shown 
%%%%%%%%%%%%%%%%%%%%%%%%%%%%%%%%%%%%%%%%%%%%%%%%%%%%%%

%Before considering rotationally inelastic scattering, 
We next address the question of whether MQDT-FT can be used to describe hyperfine interactions, which play a crucial role in ultracold molecular collisions \cite{Balakrishnan:16,Bause_23}, yet are notoriously difficult to describe using standard CC methods, requiring enormous basis sets \cite{Tscherbul_23}. 
As depicted in ~\cref{Fig_2}(a) the hyperfine structure of $^{15}$NH arises from the nuclear spins of $^{15}$N ($I_1=1/2$) and H ($I_2=1/2$) which couple to $\hat{\mathbf{j}}$ 
(see above) to produce the total angular momentum $\hat{\mathbf{F}}$ of NH, which is a good quantum number at low $B$-field. The hyperfine structure is described by the Hamiltonian  \cite{Bailleux_12,Bizzocchi_18,SM}
%  In this case, the nuclear spin quantum numbers for N ($I_\text{N}$) and H ($I_\text{H}$) in the NH molecule are both 1/2, and the Hamiltonian for the collision complex is given as the sum of $\hat{\mathcal{H}}$ in \cref{eq:Heff} and the hyperfine interactions $\hat{\mathcal{H}}_\text{hf}$ given as \cite{Tscherbul_07}
\begin{equation}
%\hat{\mathcal{H}}_\text{hf} & = \sum_{i} \{ (b_i+\frac{c_i}{3}) \hat{\bm{I}_i} \cdot \hat{\bm{s}} \\
%&+\frac{c_i\sqrt{6}}{3}\sqrt{\frac{4\pi}{5}} \sum_{q=-2}^{2} (-)^q Y_{2\, -q}(\theta_r,\phi_r)[\hat{\bm{I_i}} \otimes \hat{\bm{s}} ]_q^{(2)} \},
\hat{H}_\text{hf} = \sum_{i=1,2}  a_i \hat{\mathbf{I}}_i \cdot \hat{\mathbf{S}} + \hat{H}_\text{ahf}, 
\label{eq:hf}
\end{equation}
where $a_i$ and $c_i$ are the isotropic and anisotropic hyperfine  constants for the $i$-th nucleus, and $\hat{H}_\text{ahf}=\sum_i \frac{c_i\sqrt{6}}{3}\sqrt{\frac{4\pi}{5}} \sum_{q=-2}^{2} (-1)^q Y_{2\, -q}(\bm{r})  [\hat{\mathbf{I}}_i \otimes \hat{\mathbf{S}} ]_q^{(2)}$ is the anisotropic hyperfine interaction.
%the Fermi contact and anisotropic (tensor) hyperfine constant
%\textcolor{red}{
%Frosch and Foley parameters \cite{Frosch_52} for the $i$-th nucleus, and $b_i+{c_i/3}$ is the constant for the isotropic Fermi contact interaction and 2nd term is anisotropic tensor 
%}
%where $i=1$ and $2$ for N and H, respectively.
 %We use the constants, including the hyperfine constants $b_i$ and $c_i$ from ~\cite{Bizzocchi_18} (see also \cite{SM}).  
In the high $B$-field limit, the 12 hyperfine levels of $^{15}$NH are arranged in three groups, with 4 states per group, according to the value of $M_S = -1,0,$ and $1$, as shown in \cref{Fig_2}(b).  



To account for the hyperfine structure, we perform MQDT-FT calculations using the same simplified short-range Hamiltonian $\hat{H}_\mathrm{0}$ and the TAM basis set as described above without extra computational costs.
It is only necessary to augment $\bm{K}_\mathrm{0}^\text{sr}$ by the nuclear spin basis states $\ket{I_1M_{I_1}} \ket{I_2 M_{I_2}}$ before applying
the FT \cite{SM}.
%which involve the nuclear spin basis functions
% $\bm{I}$ is a 4x4 unit matrix corresponding to the space composed of 4 possible linear independent nucleus spin states such as $M_{I_\text{N}} =M_{I_\text{H}} =1/2$, $M_{I_\text{N}} =1/2$ and $M_{I_\text{H}} =-1/2$, $M_{I_\text{N}}=-1/2$ and $M_{I_\text{H}} =1/2$, and $M_{I_\text{N}}=M_{I_\text{H}} =-1/2$. 
 

%and the basis set for the short-range CC calculations are the same as the above calculations for \cref{Fig_1}, thus we can start from the same $\bm{K}_\mathrm{0}^\text{SR}$ without any additional calculations.
% On the other hand, $\bm{K}_\mathrm{0}^\text{SR}$ does not have a dimension for the nuclear spin degrees of freedom, thus we need to redefine $\bm{K}_\mathrm{0}^\text{SR}$ by extending as $\bm{K}_\mathrm{0}^\text{SR}\otimes\bm{I}$, where $\bm{I}$ is a 4x4 unit matrix corresponding to the space composed of 4 possible linear independent nucleus spin states such as $M_{I_\text{N}} =M_{I_\text{H}} =1/2$, $M_{I_\text{N}} =1/2$ and $M_{I_\text{H}} =-1/2$, $M_{I_\text{N}}=-1/2$ and $M_{I_\text{H}} =1/2$, and $M_{I_\text{N}}=M_{I_\text{H}} =-1/2$. 


Figures \ref{Fig_2}(c) and (d) show transition probabilities between the different hyperfine levels of NH in ultracold collisions with Mg atoms. 
%as functions of collision energy for the partial wave resolved ($l=0 \to l^\prime=2$) state-to-state hyperfine relaxations from the energetically highest hyperfine state (12th state) to the final states (5-8th states) in the rotational ground state ($N=0$) with the magnetic field of $B=100$G and $B=1000$G. 
We observe very good agreement between MQDT-FT and exact CC results, which indicates that the effects of hyperfine structure on ultracold atom-molecule collisions can be accurately described by MQDT-FT. 
Importantly, only 147 channels %with $J_\text{max}=7$ 
need to be coupled in our MQDT-FT calculations, whereas full CC computations involve as many as 3854 channels \cite{Maykel_11}. %with the converged fully uncoupled basis set  
Given the $O(N^3)$ scaling of the computational effort with the number of channels $N$, MQDT-FT affords a massive reduction of computational cost by a factor of 10$^4$.
%\textcolor{red}{
%The required computational time for the short-range CC calculations can be more than 10,000 times faster than previous MQDT studies.
%}
% The computational cost of short-range This further efficiently yet precisely without using the information of $\hat{\mathcal{H}}_\text{Z}+\hat{\mathcal{H}}_\text{hf}$ in the short-range. 


 
%%%%%%%%%%%%%%%%%%%%%%%%%%%%%%%%%%%%%%%%%%%%%%%%%%%%%%
%\begin{figure}[b!]
\begin{figure}[t]
\begin{center}
\includegraphics[height=0.22\textheight,keepaspectratio]{Fig_3.pdf}
\end{center}
\vspace{-0.15cm}
\caption{Probabilities for the transition $M_S=1$, $l=0$ $\to$ 
$M^\prime_S=0$,
$l^\prime=2$ in cold Mg~+~$^{14}$NH$(N=0)$ collisions as a function of collision energy. Solid lines -- MQDT-FT results based on a $2\times2$ short-range K-matrix ($K^\text{sr}$:$2\times2$), open circles -- exact CC calculations. 
}
\label{Fig_3}
\end{figure}
%%%%%%%%%%%%%%%%%%%%%%%%%%%%%%%%%%%%%%%%%%%%%%%%%%%%%%

%While the above calculations are efficient enough to address unexplored relevant molecular collisions, we explicitly treat multiple open channels simultaneously. 

An attractive feature of the MQDT-FT approach is its capability to describe the complex quantum dynamics of ultracold collisions in terms of a small number of ''universal'' parameters \cite{Burke_99,Gao_05}.  While this capability has provided invaluable physical insight into ultracold atomic collisions  \cite{Burke_99,Gao_05}, it has been unclear whether it can be extended to more complex molecular collisions. 
A minimal MQDT-FT model of ultracold Mg~+~NH collision is based on a $2\times 2$ short-range $K$-matrix obtained 
using only the elements of $\bm{K}^\text{sr}$ corresponding to the initial and final channels ($K_{ii}^\text{sr}$, $K_{if}^\text{sr}=K_{fi}^\text{sr}$, and $K_{ff}^\text{sr}$).
%only the four elements associated with the initial $i$ and final $f$ channels in $\bm{K}^\text{SR}$, namely $K_{ii}^\text{SR}$, $K_{if}^\text{SR}=K_{fi}^\text{SR}$, and $K_{ff}^\text{SR}$, which was computed in the calculation for \cref{Fig_1}.
 

As shown in \cref{Fig_3}, the probability for the $M_S=1$ $\to$ $M^\prime_S=0$ transition obtained with the two-channel model is in excellent agreement with the full 954-channel results. {\it Remarkably, the quantum dynamics of inelastic spin relaxation in ultracold Mg~+~NH collisions in a magnetic field can be described nearly exactly using just three short-range parameters ($K_{ii}^\text{sr}$, $K_{if}^\text{sr}$, and $K_{ff}^\text{sr}$) over a wide range of collision energies.} 
This result highlights the universal nature of complex ultracold atom-molecule collisions in a dc magnetic field, and suggests MQDT-FT as a possible alternative to the approaches based on a single-channel universal model (UM) \cite{Idziaszek_10,Frye_15}.
While the phenomenological $y$ parameter of the UM can be related to the results of CC calculations only in some special cases \cite{Frye_15}, the MQDT-FT parameters 
(${K}^\text{sr}_{ij}$)
are readily available from modest CC calculations, which only need to be performed in the short-range region, and do not need to include the Zeeman and hyperfine interactions. 


 %possibility of universal parameters determined from small CC calculations. be used  of the MQDT-FT formalism to reduce a complex atom-molecule scattering problem to a few parameters, 
%In other words, we make the 2x2 $\bm{K}^\text{SR}$ that includes the initial, $M_s=1 (l=0)$, and final, $M_s=0 (l=2)$, components. 
%We observe only a tiny deviation between the reduced dimensional MQDT-FT and CC, indicating that the essential short-range information for the transition is described with the 2x2 symmetric matrix or three independent parameters. 



%minimum number of short-range parameters required
%, such as the singlet and triplet phase shifts  for ultracold collisions of two alkali-metal atoms
%due to the numerous rotational degres of freedom,
% short-range $K$-matrix, we can now use to expresses  

%Another advantage of the
%Having reduced the description of 
%Because our reduced short-range $K$-matrix  depends on a such a small number of parameters, we can now address the question of how many short-range parameters are necessary to accurately characterize ultracold atom-molecule collision dynamics.







%%%%%%%%%%%%%%%%%%%%%%%%%%%%%%%%%%%%%%%%%%%%%%%%%%%%%%
%\begin{figure}[b!]
\begin{figure}[t!]
\begin{center}
\includegraphics[height=0.22\textheight, trim = 35 0 0 0]{Fig_4.pdf}
%\includegraphics[height=0.17\textheight, trim = 35 0 0 0]{Fig_4_twofig.pdf}
\end{center}
\vspace{-0.2cm}
\caption{Collision energy dependence of transition probabilities between the different fine-structure levels of $^{14}$NH in ultracold collisions with Mg atoms. Symbols --  exact CC calculations, lines -- MQDT-FT results obtained using the recoupling FT. The quantum numbers for the initial and final states involved in each transition are indicated next to the corresponding curves. All results are for a single incident $s$-wave component ($l=0$) and summed over all final $l'$. 
%(a) The incoming partial $s$-wave component ($l=0$) for the fine state transition from $N=1, j=1$ to $N^\prime=0, j^\prime=1$ is shown by the color of black, and the transitions from $N=2,j=2$ to $N^\prime=1,j^\prime=2$, $N^\prime=1,j^\prime=1$, and $N^\prime=0, j^\prime=1$ are shown by the color of red, green, blue. (b) The incoming partial $s$-wave component ($l=0$) for the fine state transition from $N=3, j=3$ to $N^\prime=3,j^\prime=4$ (magenta), $N^\prime=2,j^\prime=2$ (brown), and $N^\prime=1,j^\prime=2$ (light blue).
}
\label{Fig_4}
\end{figure}
%%%%%%%%%%%%%%%%%%%%%%%%%%%%%%%%%%%%%%%%%%%%%%%%%%%%%%


Finally, we consider the possibility of MQDT-FT based on an even more drastically simplified short-range Hamiltonian $\hat{H}_0$,  {\it which includes only rovibrational degrees of freedom}. Because $\hat{H}_0$ excludes all spin-dependent interactions, it cannot describe inelastic transitions between the different hyperfine-Zeeman sublevels of NH($N=0$), which, for collisions with structureless atoms such as Mg, are mediated by intramolecular spin-dependent interactions \cite{Krems_04,Campbell_09}. We thus focus on inelastic transitions between the different fine-structure components of excited rotational states of NH.
We use a modified recoupling technique, which has been successfully applied to such transitions at high collision energies   \cite{Corey_83,Alexander_85,Corey_85,Lique_11,Lique_19}.  
To extend this technique to the ultracold domain, we apply it to the short-range $K$-matrix rather than the $T$-matrix, as was done previously \cite{Corey_83,Alexander_85,Corey_85,Lique_11,Lique_19}.


%without explicitly treating the spin-dependent part in the Hamiltonian
%Since most of the inelastic scattering occurs accompanied by rotational and rovibrational transitions at higher energies, the spin of electrons and nucleus plays a spectator role \cite{Corey_83,Alexander_85}, thus the recoupling technique for the scattering properties, such as T-matrix, provides an efficient yet reliable method.   

%The above MQDT-FT calculations are applicable to the transitions in the rotationally excited states. 
%However, it may be possible to perform simpler calculations for the transitions accompanied by rotational transitions.
%From the viewpoint of MQDT-FT, the application of the recoupling technique to the short-range K-matrix implies our assumption that the asymptotic channel states are described well with a set of analytical basis set like atomic collisions \cite{Burke_98,Gao_05,Hanna_09,Idziaszek_2011,Perez-Rios_15}.   


Specifically, we obtain $\bm{K}^\mathrm{sr}_0$ by solving CC equations for the simplified Hamiltonian  $\hat{H}_0 = \hat{H} - \hat{H}_\text{fs} - \hat{H}_\text{hfs}-\hat{H}_\text{Z}$ (see \cref{eq:Heff,eq:Has})   
using the basis set $|(Nl)J_rM_{r}\rangle$, where $\hat{\mathbf{J}}_r=\hat{\mathbf{N}} + \hat{\mathbf{l}}$  is the total rotational angular momentum of the collision complex \cite{Tscherbul_23}.
%the procedures described in Ref.~\cite{Corey_83}. Here, $\hat{\mathcal{H}}$ is equal to the right-hand-side in \cref{eq:Has}, and $\hat{\mathcal{H}}_\mathrm{0}$
%as described above
%and solving solving the CC equations for the resulting spinless atom-molecule Hamiltonian  obtained from  defined by neglecting the spin degrees of freedom from $\hat{\mathcal{H}}$ (see \cref{eq:Hmol}) as
%\begin{equation}
%\hat{\mathcal{H}}_\mathrm{0} = - \frac{1}{2\mu R} \frac{d^2}{dR^2}R + \frac{\hat{\bm{l}}^2}{2\mu R^2} +B_\mathrm{rot}\,\hat{\bm{N}}^2+ V_\text{int}(R,\theta)
%\label{eq:Hsf}
%\end{equation}
%using the basis set $|J_rM_{J_r}(Nl)\rangle$, where $\bm{J}_r=\bm{N}+\bm{l}$ is solved to obtain $\bm{K}_\text{0}^\text{SR}$. 
%RESUME HERE
The basis functions $|(Nl)J_rM_{r}\rangle$ are the eigenstates of $\hat{H}_\mathrm{0}$ in the limit $R\to \infty$. The eigenstates   
%(the asymptotic channel states).  
of $\hat{H}_\mathrm{0}+\hat{H}_\text{fs}$ are well approximated by the TAM basis functions $|(NS)jlJM\rangle$ for $^{2S+1}\Sigma$ molecules \cite{Corey_83} provided the weak spin-spin interaction between the different $N$ states can be neglected.
% in the absence of external fields \cite{Corey_83}. 
%described well in terms of a single basis function
 To obtain the desired short-range $K$-matrix in
  the $|(NS)jlJM\rangle$ basis from that in the  $|(Nl)J_rM_r\rangle$ basis, we use an analytical recoupling transformation \cite{Corey_83,SM}.
%\begin{equation}
%\begin{split}
%K^{\text{sr},J}_{N'Sj'l',NSjl} =  \sum_{J_r=|N-l|}^{N+l} [J_r][j' j]^{1/2} (-1)^{-l'+l+j'-j}   
%begin{Bmatrix} S & N' & j' \\ l' & J & J_r \end{Bmatrix} 
%\begin{Bmatrix} S & N  & j  \\ l & J & J_r \end{Bmatrix}
%K^{\text{sr},J_r}_{0,N'l',Nl},
%\label{eq:7}
%\end{split}
%\end{equation}
%where $\{\}$ denotes a 6-$j$ symbol and $[X]=(2X+1)$.  


Figure~\ref{Fig_4} shows that the probabilities for rotationally inelastic fine-structure transitions in ultracold Mg~+~NH collisions are well described by MQDT-FT. %based on the analytic FT matrix  (see also \cite{SM}). 
Thus, using a spin-independent $\hat{H}_0$ in short-range CC calculations in the framework of MQDT-FT enables one to describe a wide range of transitions between rotational and fine-structure levels of open-shell molecules in ultracold collisions with atoms.

%it is sufficient to use describe  ultracold atom-molecule collision dynamics at short range.

%approximated with recoupling technique
%. The fine state transitions accompanied by the rotational transitions
% squares of the modulus of the T-matrix elements for the final state resolved fine state transitions for the incoming s-wave scattering
%In particular, we observe the correct resonance feature in the transitions from the $N=2,j=2$ state in (a). We find an excellent agreement for a fine state transition without changing the rotational state $N=3$ (magenta in (b)). However, it is not necessarily possible to obtain such good agreement for all the transitions without being accompanied by the rotational transitions \cite{SM}.    
%open-shell molecular collisions


In summary, we have generalized the powerful MQDT-FT approach to ultracold atom-molecule collisions in external magnetic fields, providing a robust conceptual and numerical framework for their theoretical description.
We have applied the approach to a realistic atom-molecule collision system (Mg~+~NH) using a variety of short-range Hamiltonians, obtaining encouraging agreement with exact CC calculations in all cases. 
Our calculations show that MQDT-FT can provide a dramatic ($10^4$-fold) reduction of computational effort of CC calculations of atom-molecule collisions compared to standard MQDT \cite{Croft_12} due the fact that there is no need to explicitly account for the hyperfine-Zeeman structure of the collision partners at short range. 
%\textcolor{red}{Efficiency of MQDT-FT becomes exponentially prominent with increasing the number of channels. }
In many cases of practical interest, it will only be necessary to solve CC equations in the strong interaction region employing a rovibrational (TAM) basis in the absence of external fields,  which can be efficiently accomplished using currently available computational techniques \cite{ABC,Pack_87,Kendrick_18,Croft_17,Morita_23,Jisha_14}. This opens up the possibility of performing rigorous quantum scattering calculations on a wide array of previously intractable ultracold atom-molecule collisions and chemical reactions in external fields, including those probed in recent experiments \cite{Yang_19,Wang_21,Son_22,Park:23b,Nichols_22}.
%which result from not having to include the spin degrees of freedom and the associated fine, hyperfine, and Zeeman interactions of the colliding molecules at short range.  

From a conceptual viewpoint, our results show that it is possible to describe the intricate quantum dynamics of ultracold atom-molecule collisions over wide ranges of collision energies and magnetic field in terms of a small number of short-range parameters (three for Mg+NH). This provides a solid basis for the development of few-parameter MQDT-FT models of ultracold atom-molecule (and possibly even molecule-molecule \cite{Park_23}) collisions. Such models, which have been so fruitful in the field of ultracold atomic collisions \cite{Burke_99,Gao_05}, may allow for novel insights into fascinating quantum dynamics of ultracold molecular collisions. 



We thank Chris Greene, John Bohn, James Croft, and Brian Kendrick for stimulating discussions. 
This work was supported by the NSF CAREER program (PHY-2045681) and by the U.S. AFOSR under Contract FA9550-22-1-0361 to P.B. and T.V.T.
Computations were performed on the Niagara supercomputer at the SciNet HPC Consortium.
%and the Digital Research Alliance of Canada.
%SciNet is funded by: the Canada Foundation for Innovation; the Government of Ontario; Ontario Research Fund - Research Excellence; and the University of Toronto. We also thank the Digital Research Alliance of Canada.

\newpage

\widetext
\begin{center}
\textbf{\large Supplemental Material}


\end{center}
%%%%%%%%%% Merge with supplemental materials %%%%%%%%%%
%%%%%%%%%% Prefix a "S" to all equations, figures, tables and reset the counter %%%%%%%%%%
\setcounter{equation}{0}
\setcounter{figure}{0}
\setcounter{table}{0}
%\setcounter{page}{1}


\title{ \bf{Supplemental Material for \\  ``Multichannel quantum defect theory with a frame transformation \\ for ultracold molecular collisions in magnetic fields''}}

\author{ Masato Morita$^{1}$, Paul Brumer$^{1}$ and Timur V. Tscherbul$^{2}$ \\ }

\vspace{0.5cm}
\affiliation{ \vspace{0.3cm}
$^{1}$Chemical Physics Theory Group, Department of Chemistry, and Center for Quantum Information and Quantum Control, University of Toronto, Toronto, Ontario, M5S 3H6, Canada\\
$^{2}$Department of Physics, University of Nevada, Reno, NV, 89557, USA 
}

%\date\today
\maketitle



\tableofcontents 


\vspace{0.8cm}
In this Supplemental Material (SM), we provide technical details of our MQDT-FT calculations as well as some additional results, which complement the figures and discussions in the main text. We describe the computational details in \cref{sec:SM_Computation}, and present the cross sections for Zeeman and hyperfine transitions in cold Mg+NH($N=0$) collisions   in \cref{sec:SM_SR,sec:SM_hf}. 
Results for the short-range K-matrix as a function of collision energy are presented in \cref{sec:SM_Y}. 
In \cref{sec:SM_fine}, the capabilities of the recoupling FT approach to calculate the short-range K-matrix are illustrated by MQDT-FT calculations of fine structure transitions from the $N=3$, $j=3$  initial state of NH. 
%to all possible final states.  



\setcounter{equation}{0}
\setcounter{figure}{0}



\vspace{0.5cm}
\section{\label{sec:SM_Computation} Computational details}

\subsection{Coupled channel (CC) calculations}

We performed coupled-channel (CC) calculations to obtain the reference results for cold Mg~+~NH collisions  in a magnetic field, against which we compare our MQDT-FT calculations in the main text.  
The Hamiltonian of the atom-molecule collision complex is given by Eq.~(1) of the main text.
% for the calculations of Fig.~2, by $\hat{H} - \hat{H}_\text{Z}$ for those of  Figs.~1 and 3, and by $\hat{H} - \hat{H}_\text{Z}-\hat{H}_\text{hfs}$ for those of Fig.~4.
We use converged uncoupled basis sets, $|NM_N\rangle|SM_S\rangle|I_{1} M_{I_1}\rangle|I_{2}M_{I_2}\rangle|lm_l\rangle$ for Fig.~2 and $|NM_N\rangle|SM_S\rangle|lm_l\rangle$ for Figs.~1 and 3, with $N_\text{max}=6$ and $l_\text{max}=8$ \cite{Wallis_09}. 
On the other hand, for Fig.~4, we employ a coupled total angular momentum (TAM) basis set $|(NS)jlJM\rangle$ with $N_\text{max}=6$ and $J_\text{max}=5$.
The CC equations are solved with the log-derivative propagation method \cite{Johnson_73} between $R_\text{min}= 4\,a_0$ and $R_\text{max}= 500\,a_0$. We use the propagation interval of $\Delta\,R= 0.05\,a_0$ for $R\leq30\,a_0$ and $\Delta\,R=0.1\,a_0$ for $R>30\,a_0$. 

\subsection{Physical constants}
We use the same reduced mass of Mg+$^{14}$NH ($\mu=9.232679959$ amu), and spectroscopic constants of $^{14}$NH  ($B_e=16.343$ cm$^{-1}$, $\gamma_\mathrm{sr}=-0.055$ cm$^{-1}$, and $\lambda_\mathrm{ss}=0.92$ cm$^{-1}$) as in Refs.~\cite{Wallis_09,MOLSCAT}. For the calculations  shown in Fig.~2 of the main text, we use the reduced mass of $\mu=9.60046$ amu for Mg+$^{15}$NH, the spectroscopic constants of $^{15}$NH ($B_e=16.2712$ cm$^{-1}$, $\gamma_\mathrm{sr}=-0.0546$ cm$^{-1}$, and $\lambda_\mathrm{ss}=0.9199$ cm$^{-1}$), and 
%in Ref.~\cite{Bizzocchi_18}. 
the hyperfine constants of $^{15}$NH ($b_1=-1.9406\times 10^{-3}$ cm$^{-1}$, $c_1=3.1783\times 10^{-3}$ cm$^{-1}$, $b_2=-3.2108\times 10^{-3}$ cm$^{-1}$ and $c_2=3.0203\times 10^{-3}$ cm$^{-1}$) ~\cite{Bizzocchi_18}. 
%Here, the constants $b_i$ ($i=1,2$) are obtained from the values of $b_{\text{F},i}$ and $c_i$ given in Ref.~\cite{Bizzocchi_18} with the relation $b_{\text{F},i}=b_i+c_i/3$, where $b_{\text{F},i}$ is the constant for the Fermi contact interactions. 


\subsection{Interaction potential}
The interaction potential between Mg ($^1$S) and NH ($\tilde{X}\, ^3\Sigma^-$) used in our calculations is published as part of the MOLSCAT program suite \cite{MOLSCAT} as pot-Mg\_NH.data and vstar-Mg\_NH.f. 
This potential is slightly different from that used in Refs.~\cite{Wallis_09,Maykel_11,Croft_11}.




\subsection{Multi-channel quantum defect theory (MQDT)}

%\textcolor{red}{Except for the frame transformation step to obtain the short-range K-matrix ($\bm{K}^\text{sr}$),} 
Except for the frame transformation step to obtain the short-range K-matrix ($\bm{K}^\text{sr}$), the required calculation procedures in MQDT-FT are essentially the same as those of conventional atom-molecule MQDT \cite{Croft_11,Croft_12}. 

The total number of reference channels is $N_\mathrm{ref} = N_\mathrm{o} +N_\mathrm{wc}$ \cite{Croft_11}, where $N_\mathrm{o}$ and $N_\mathrm{wc}$ are the numbers of open and weakly closed channels, respectively.
As stated in the main text,  we use only  weakly closed channels when calculating the short-range $K$-matrix (i.e., we exclude strongly closed channels as defined in the main text).
%the weakly closed channels as the closed channels whose diabatic energies, the diagonal elements of the interaction potential in the asymptotic basis set, become lower than the total energy in the range of $R\geq R_m$ where $R_m$ is the matching distance. 
The value of the matching radius $R_m$ should be determined carefully by exploring the $ R_m$ dependence of the results.

We use $R_m=30\, a_0$ for all MQDT-FT calculations presented  here and in the main text. No sensitivity to $R_m$ was observed in the range  $15\, a_0 \leq R_m \leq 40\, a_0$ for the results shown in Fig.~1 of the main text and in the range $25\, a_0 \leq R_m \leq 35\, a_0$  for those shown in Fig.~2 of the main text.

%\textcolor{cyan}{
%\it (Actually I observed that the results are not sensitive to the change of $R_m$ in the range we examined around ($15\, a_0 \leq R_m \leq 40\, a_0$) for the calculations without the hyperfine interactions (Fig.~1 in the main text). For Fig.~2, including hyeprfine interactions, I explored the stability of the results in ($25\, a_0 \leq R_m \leq 35\, a_0$). For Fig.~4, I did not examine the sensitivity of the results with respec to $R_m$.)  }

%We note that in MQDT-FT the value of $R_m$ should be determined carefully by exploring the $R_m$-dependence of the results because the approximation level at short-range depends also on the value of $R_m$. 
%On the contrary, the short-range part is described rigorously in MQDT based on the CC methodology with the full Hamiltonian for the collision complex.


The reference potentials, $U_i^\text{ref}(R)\, (i=1,..,N_\text{ref})$, describe the long-range part of the scattering problem and dissociate into accurate channel energies including Zeeman and/or hyperfine interactions.
%which should behave appropriately in the long-range reflecting the channel energies as well as the long-range forces. 
We employed the isotropic part, $V_0(R)$, of the interaction potential to construct the reference potentials. The reference potentials contain a hard wall at $R=R_\text{wall}$ in the short range region ($U_i^\text{ref}(R_\text{wall})=+\infty$) restricting the amplitude of the regular reference function at $R=R_\text{wall}$.
We employ a common wall position $R_\text{wall}=5\,a_\text{0}$ for all reference channels although the final results are not affected by the choice of $R_\mathrm{wall}$. 
We obtain the reference functions for the $i$-th reference channel by numerically solving the following 1D radial equation \cite{Mies_84,Croft_11}
\begin{equation}
\label{eq:1Deq}
\left[\, \frac{d^2}{dR^2}+K_i^2(R)\, \right]\, \psi_i(R)\, =\, 0,
\end{equation}
where $K_i(R)=\sqrt{2\mu[\,E-U_i^\mathrm{ref}(R)\,]}$ is the local wavevector and $E$ is the total energy.
In the following, we refer to the regular solution as $f_i(R)$ ($f_i(R) \to 0$ as $R \to 0$) and to the irregular solution as $g_i(R)$. 
Here, we use the reference potentials with minima at a finite $R$ in the range $0<R<R_m$ (in Eq.~(5) in the main text) and the WKB normalizations are taken at the minimum of the reference potentials following Mies et al. \cite{Mies_84}. 
%These are normalized to have WKB form, with amplitude $1/\sqrt{K_i(R)}$, at a point of $R$ in the classically allowed region. Here, we use the reference potentials with minima at a finite $R$ in the range $0<R<R_m$ (in Eq.~(5) in the main text) following Mies \etal \cite{Mies_84} and the WKB normalizations are taken at the minimum of the reference potentials. 


Once we obtain the values of the regular and irregular reference functions at the matching point $(R=R_m)$, 
we calculate the $N_\mathrm{ref} \times N_\mathrm{ref}$ short-range K matrix $\bm{K}_\mathrm{0}^\text{sr}$ following the matching boundary condition at $R=R_m$:
\begin{equation}
\label{eq:shortBoundary}
\bm{\Psi}(R=R_m)= R_m^{-1}[\bm{f}(R_m)+\bm{g}(R_m)\bm{K}_\mathrm{0}^\text{sr}],
\end{equation}
where $\bm{f}(R_m)$ and $\bm{g}(R_m)$ are the diagonal matrices in the asymptotic basis with their diagonal elements given by the values of the reference functions $f_i(R_m)$ and $g_i(R_m)$ \cite{Croft_13}. 


\section{\label{sec:SM_SR} Frame transformation (FT)}
%\textcolor{red}{
%The selection of $H_0$ is not straightforward for ultracold molecular collisions. Previous applications of MQDT-FT to the collisions of ultracold alkali atoms relied on the assumption that spin-changing processes are primarily driven by spin-exchange interactions characterized by the difference of the singlet and triplet interaction potentials at short-range. However, recent intriguing applications of ultracold molecular collisions, such as sympathetic cooling and evaporative cooling, predominantly proceed on a single potential energy surface. Consequently, intramolecular interactions, such as spin-rotation and spin-spin interactions, become the primary mechanisms responsible for the spin-changing processes at short-range. Furthermore, the impact of intramolecular interactions varies in terms of the values of the spectroscopic constants, which may result in the non-negligible coupling between different rotational states (e.g. spin-spin interaction: $N=0 \leftrightarrow N=2$). Thus, unlike alkali atomic collisions, it is not possible to provide a general and analytical form of frame transformation even when neglecting the magnetic field dependence and specifying the quantum numbers of spins in the system. 
%}

{
Here, as in the main text, $\hat{H}$ denotes the Hamiltonian of the atom-molecule collision complex, and $\hat{H}_0$ denotes the simplified Hamiltonian used in short-range CC calculations of $\bm{K}_\mathrm{0}$ in MQDT-FT. 
We approximate $\bm{K}^\text{sr}$ by the matrix obtained via the frame transformation (FT) as $\bm{K}^\text{sr} \simeq \bm{U}^{\dagger}\bm{K}_\mathrm{0}^{\text{sr}}\bm{U}$, where %$\bm{K}_\mathrm{0}^{\text{sr}}$ is the short-range K-matrix with $\hat{H}_0$ and 
$\bm{U}$ is a unitary matrix composed of the eigenstates of the asymptotic Hamiltonian $\hat{H}_\text{as}=\lim_{R\to \infty} \hat{H}$.
Since $\bm{K}_\mathrm{0}^{\text{sr}}$ is initially expressed in the basis of eigenstates of the simplified asymptotic Hamiltonian $\hat{H}_{0,\text{as}} = \lim_{R\to \infty}\hat{H}_\mathrm{0}$, it is necessary to establish a relationship between the eigenstates of $\hat{H}_\text{as}$ and $\hat{H}_{0,\text{as}}$ to represent $\bm{K}^\text{sr}$ in the basis of eigenstates of  $\hat{H}_\text{as}$.
}

%represented in the eigenstates of $\hat{H}_\text{as}$ for the succeeding steps in the MQDT-FT calculation, it is necessary to know the relation between the eigenstates of $\hat{H}^\text{as}$ and $\hat{H}_0^\text{as}$. \\
%$\bm{U}$ (and $\hat{H}^\text{as}$) with the eigenstates of $\hat{H}_0^\text{as}$. Depending on the technical convenience and available basis sets, we can perform FT using different representations. For example, when we obtain $\bm{U}$, eigenvectors of $\hat{H}^\text{as}$, by a basis set, it is a way to represent $\bm{K}_\mathrm{0}^{\text{sr}}$ in the basis set to perform FT ($\bm{K}^\text{sr} \simeq \bm{U}^{\dagger}\bm{K}_\mathrm{0}^{\text{sr}}\bm{U}$).  
%If $\hat{H}_0^\text{as}$ does not include some (spin) degrees of freedom compared to $\hat{H}^\text{as}$, like the nuclear spins in the calculations of Fig.~2 and \cref{SM_Fig_2}, we need to augment the missing degrees of freedom to the basis set or eigenstates of $\hat{H}_0^\text{as}$. 
%Such transformation is possible only when the contained degrees of freedom are equal.  %into the representation of the 
%represent $\bm{U}$ with eigenstates of $\hat{H}_0^\text{as}$ or represent $\bm{K}_\mathrm{0}^{\text{sr}}$ with the basis set used in the representation of $\bm{U}$.
%\vspace{0.2cm}

%While the above procedure is general, we can consider efficient variants of FT based on the character of the relevant channels in the collisions. 
% as a possible technical difficulty in molecular collisions, 
% and thus not all channel basis states need to be c in MQDT-FT calculations.  

{
Once we represent $\hat{H}_0$ and  $\hat{H}_\text{as}$  in the same basis set, it is straightforward to apply the FT. However, it can be challenging to obtain the eigenvectors of $\hat{H}_\text{as}$ in the basis set used in short-range CC calculations. The latter are best performed using the TAM basis set, and it is not straightforward to derive and/or implement the matrix elements of $\hat{H}_\text{as}$ in this basis. 
To avoid these difficulties, % we pre-transform $\bm{K}_\mathrm{0}^{\text{sr}}$ to 
we propose to use the fully uncoupled basis set, in which the matrix elements of $\hat{H}_\text{as}$ are readily available \cite{Krems_04}. 
}
% in the fully uncoupled basis. 
%We note that strongly closed channels are excluded from the calculation of  $\bm{K}_\mathrm{0}^{\text{sr}}$ (see above).


For example, to obtain the results  shown in Fig.~1 of  the main text, the following sequence of steps is used

\begin{enumerate}

\item
{
The matrix of $\hat{H}_\text{as}$ is constructed and diagonalized in the fully uncoupled basis  $|NM_N\rangle|SM_S\rangle|lm_l\rangle$. 
The resulting eigenstates are expressed as linear combinations of the fully uncoupled basis functions
\begin{equation}
\label{eq:eigenstate}
|\, \alpha\, \rangle\, =\,  \sum_{i}^{}\, c^\alpha_i |\, i\, \rangle,
\end{equation}
where $|\,i\,\rangle$ is a short-hand notation for the uncoupled basis functions $|NM_N\rangle|SM_S\rangle|lm_l\rangle$, which includes the quantum numbers $N$, $M_N$, $M_S$, $l$, and $m_l$ (note that $S=1$ is conserved), and the expansion coefficients \{$c_i^\alpha$\} correspond to the $\alpha$-th eigenvector of $\hat{H}_\text{as}$.  
The calculated eigenvectors are stored for subsequent use in step 4. 
}

\item
{
The CC equations for the simplified Hamiltonian $\hat{H}_0$ expressed in the primitive TAM basis  $|(NS)jlJM\rangle$ are integrated numerically out to $R=R_m$.  
The log-derivative matrix is then transformed from the primitive TAM basis set to the basis of eigenvectors of $\hat{H}_{0,\text{as}}$ to obtain
the short-range $K$ matrix $\bm{K}_\mathrm{0}^{\text{sr}, J}$ using the matching condition given by \cref{eq:shortBoundary} for each $J$ ($J_\text{max}=5$). 
%We note that components associated with strongly closed channels are excluded from $\bm{K}_\mathrm{0}^{\text{sr},J}$ (see above).
Thus, $\bm{K}_\mathrm{0}^\text{sr}=\bm{K}_\mathrm{0}^{\text{sr}, J=0} \oplus \bm{K}_\mathrm{0}^{\text{sr},J=1} \oplus \cdots \oplus \bm{K}_\mathrm{0}^{\text{sr},J=5}$.
}

%\textcolor{magenta}{
%We obtain 
%$\bm{K}_\mathrm{0}^{\text{sr}}$ as $\bm{K}_\mathrm{0}^\text{sr}=\bm{K}_\mathrm{0}^{\text{sr}, J=0} \oplus \bm{K}_\mathrm{0}^{\text{sr},J=1} \oplus \cdots \oplus \bm{K}_\mathrm{0}^{\text{sr},J=6}$.  
%}

\item
{
For Mg~+~NH collisions, the eigenstates of $\hat{H}_{0,\text{as}}$ are well approximated by the individual TAM basis functions due to the small spin-spin interaction in NH (in other words, the matrix of $\hat{H}_{0,\text{as}}$ is nearly diagonal in the TAM basis). Therefore, in the following, we assume that $\bm{K}_\mathrm{0}^{\text{sr}}$ is expressed in the TAM basis. If $\hat{H}_{0,\text{as}}$ is not diagonal in this basis,  $\bm{K}^\text{sr}$ would need to be transformed to the basis, in which  $\hat{H}_{0,\text{as}}$ is diagonal (see the discussion after Step 4).
}


\vspace{0.2cm}
{
The matrix elements of $\bm{K}^\text{sr}$ in the basis of eigenstates $\alpha$ and $\beta$ of $\hat{H}_{\text{as}}$ is obtained by applying the FT
\begin{equation}
\label{eq:FT}
\langle\, \alpha\, |\, \bm{K}^\text{sr}\, |\, \beta\, \rangle =  \sum_{p,q}^{}\, {d_p^\alpha}^\ast d_q^\beta \langle\, p\, |\, \bm{K}_0^\text{sr}\, |\, q\, \rangle,
\end{equation}
where $|p\rangle$ and $|q\rangle$ are the TAM basis functions, \{$d_p^\alpha$\} and \{$d_q^\beta$\} are the eigenvectors of $\hat{H}_{\text{as}}$ in the TAM basis. Instead of calculating these eigenvectors directly, we express them via the eigenvectors of $\hat{H}_{\text{as}}$ in the fully uncoupled basis (see Step 1) as described below. 
%an alternative method based on the eigenvectors of step 1. 
}

\item

As noted above, we wish to calculate $\bm{K}^\text{sr}$ in the eigenbasis of $\hat{H}_{\text{as}}$ with the latter operator is expressed in the fully uncoupled basis. This is motivated by the ease of evaluating  the matrix elements of $\hat{H}_{\text{as}}$ in the fully uncoupled basis \cite{Krems_04}.   
It follows from Eq.~\eqref{eq:eigenstate} that 
\begin{equation}
\label{eq:uncFT}
\langle\, \alpha\, |\, \bm{K}^\text{sr}\, |\, \beta\, \rangle =  \sum_{i,j}^{}\, {c_i^\alpha}^\ast c_j^\beta \langle\, i\, |\, \bm{K}_0^\text{sr}\, |\, j\, \rangle.
\end{equation}

To express the matrix elements on the right-hand side in terms of the known quantities $\langle\, p\, |\bm{K}_0^\text{sr}\,|\, q\, \rangle$, we use the completeness relation $\hat{1}=\sum_p |p\rangle \langle p|$:
%However, the transformation of the basis of $\bm{K}_0^\text{sr}$ is simple, thus we evaluate the following relation to obtain $\bm{K}^\text{sr}$
\begin{equation}
\label{eq:finalFT}
\langle\, \alpha\, |\, \bm{K}^\text{sr}\, |\, \beta\, \rangle =  \sum_{i,j}^{}\,{c_i^\alpha}^\ast c_j^\beta \langle\, i\, |\,  \left[\, \sum_{p,q}^{} |\,p\,\rangle\,\langle\, p\, |\bm{K}_0^\text{sr}\,|\, q\, \rangle\,\langle\, q\, |\, \right]\ |\, j\, \rangle \\
 =  \sum_{i,j,p,q}^{}\, {c_i^\alpha}^\ast c_j^\beta \langle\, i\, |\,p\,\rangle\, \langle\, q\, |\,j\,\rangle\,  \langle\, p\, |\bm{K}_0^\text{sr}\,|\, q\, \rangle\, ,
\end{equation}
where the overlaps between the fully uncoupled basis state $|\,i\,\rangle$ and the TAM basis state $|\,p\,\rangle$ are given analytically by Clebsh-Gordan coefficients. We note that the expression in \cref{eq:finalFT} is more generally  applicable to the cases, where the  overlaps $\langle\, i\, |\,p\,\rangle$ are only available from a numerical computation (which could rely on approximations). Additionally, it is not necessary to explicitly construct the matrix $\bm{K}_0^\text{sr}$ in the fully uncoupled basis set in this calculation.   

\end{enumerate}


\vspace{0.3cm}
For completeness, we also consider the situation, in which the eigenstates of $\hat{H}_{0,\text{as}}$ are strongly different from the individual TAM basis functions (this is not the case in the present work). In this situation, the assumption discussed in Step 3 above does not hold, and additional summations over the TAM basis functions and the associated overlap coefficients (eigenvectors) appear in \cref{eq:finalFT}. Let $|\phi\rangle$ be an eigenstate of $\hat{H}_{0,\text{as}}$, which is expanded in the TAM basis set as 
\begin{equation}
\label{eq:eigenstateH0as}
|\, \phi\, \rangle\, =\,  \sum_{p}^{}\, h_p^\phi |\, p \, \rangle.
\end{equation}
From \cref{eq:finalFT}, we obtain 
\begin{equation}
\label{eq:finalFT2}
\langle\, \alpha\, |\, \bm{K}^\text{sr}\, |\, \beta\, \rangle 
 =  \sum_{i,j,\phi,\chi}^{}\, {c_i^\alpha}^\ast c_j^\beta \langle\, i\, |\,\phi \,\rangle\, \langle\, \chi \, |\,j\,\rangle\,  \langle\, \phi \, |\bm{K}_0^\text{sr}\,|\, \chi \, \rangle\,.
\end{equation}
The matrix elements $\langle\, \phi \, |\bm{K}_0^\text{sr}\,|\, \chi \, \rangle$ and the expansion coefficients \{$h_p$\}  \cref{eq:eigenstateH0as} are available from short-range CC calculations in the TAM basis  subject to the matching condition \cref{eq:shortBoundary}. 


The most general FT is then obtained using \cref{eq:eigenstateH0as} as  
\begin{equation}
\label{eq:finalFT3}
\langle\, \alpha\, |\, \bm{K}^\text{sr}\, |\, \beta\, \rangle 
 =  \sum_{i,j,\phi,\chi}^{}\, {c_i^\alpha}^\ast c_j^\beta 
 \langle\, \phi \, |\bm{K}_0^\text{sr}\,|\, \chi \, \rangle\
 \sum_{p,q}^{}\, 
 {h_p^\phi} {h_q^\chi}^\ast  \langle\, i\, |\,p \,\rangle\, \langle\, q \, |\,j\,\rangle\,.
\end{equation}


%\textcolor{red}{
%Because the mixing of basis functions with $N=0$ and $N=2$ due to the spin-spin interaction in %$\hat{H}_0^\text{as}$ is very small in NH, we  exclude the $N=2$ strongly closed channel.
%The eigenstates of $\hat{H}_0^\text{as}$ are well approximated by each basis function of the total angular momentum (TAM) basis set $|(NS)jlJM\rangle$ used in the short-range CC equations with $\hat{H}_0$. The mixing of basis functions with $N=0$ and $N=2$ due to the spin-spin interaction in $\hat{H}_0^\text{as}$ is very small in NH, thus excluding $N=2$ as a strongly closed channel does not cause a problem. 
%In such case, we can approximate the basis, eigenvectors of $\hat{H}_0^\text{as}$, representing $\bm{K}_\mathrm{0}^{\text{sr}}$ as $|(NS)jlJM>$.
%We just need to transform  $\bm{K}_\mathrm{0}^{\text{sr}}$ 
%as $\bm{U}^{\dagger}\bm{K}_\mathrm{0}^{\text{sr}}\bm{U}$, where $\bm{U}$ is composed of the eigenvectors of $\hat{H}^\text{as}$ in TAM basis. However, as mentioned above, we employ a procedure via the representation in a fully uncoupled basis set instead of representing the eigenvectors of $\hat{H}^\text{as}$ in TAM basis. Firstly we obtain $\bm{U}$ in the fully uncoupled basis set by solving the eigenproblem of $\hat{H}^\text{as}$. Then, we carry out the short-range CC calculations with TAM basis and obtain $\bm{K}_\mathrm{0}^{\text{sr}}$. 
%After we change the basis of  $\bm{K}_\mathrm{0}^{\text{sr}}$ into the fully uncoupled basis set, we perform FT as $\bm{U}^{\dagger}\bm{K}_\mathrm{0}^{\text{sr}}\bm{U}$, where all matrices are represented in the fully uncoupled basis set. }


%When we change the basis of $\bm{K}_\mathrm{0}^{\text{sr}}$ into the fully uncoupled basis set, we use the above assumption that eigenvectors of $\hat{H}_0^\text{as}$ are described well with each function in TAM basis.  
%Thus, we just need to consider Clebsh Gordan coefficients to transform TAM to fully uncoupled basis.  
%We use the same assumptions also for the calculations of Fig~2 in the main text because $\hat{H}_0$ is common in these calculations. In this case, 


\vspace{0.5cm}
To produce the results shown in Fig.~2 of the main text, we implemented the following sequence of MQDT-FT steps

\begin{enumerate}

\item
{
Build and diagonalize the matrix of $\hat{H}_\text{as}$ (including the hyperfine interactions) in the fully uncoupled basis set $|NM_N\rangle|SM_S\rangle|lm_l\rangle|I_1M_{I_1}\rangle|I_2M_{I_2}\rangle$. 
}


\item
{
This step is the same as Step 2 above. The only difference is that the number of $J$-blocks is increased to $J_\text{max}=7$. 
}

\item
{
As in Step 3 above, we assume that the individual TAM basis functions $|(NS)jlJM\rangle$ are the eigenstates of $\hat{H}_{0,\text{as}}$. In this case, $\hat{H}_\text{as}$ includes the nuclear spin degrees of freedom for the two nuclei of NH but $\hat{H}_{0,\text{as}}$ is identical to that used above (see the steps for Fig.~1).

Because the short-range $K$-matrix is computed with the nuclear spin degrees of freedom omitted, we need to reintroduce them  before applying MQDT boundary conditions (which do include the nuclear spin degrees of freedom). To this end, we augment $\bm{K}_\mathrm{0}^\text{sr}\to \bm{K}_\mathrm{0}^\text{sr}\otimes\bm{I}$, where $\bm{I}$ is a $4\times4$ unit matrix in the basis of 4 possible nuclear  spin states of $^{15}$NH, $|M_{I_1} =\pm1/2,M_{I_2} =\pm1/2\rangle$. 
% $M_{I_1} =1/2$ and $M_{I_2} =-1/2$,\, $M_{I_1}=-1/2$ and $M_{I_2} =1/2$, and $M_{I_1}=M_{I_2} =-1/2$.
The new basis for $\bm{K}_\mathrm{0}^\text{sr}\otimes\bm{I}$ is given by the direct product of the TAM basis and $|I_1M_{I_1}\rangle|I_2M_{I_2}\rangle$, namely $|(NS)jlJM\rangle|I_1M_{I_1}\rangle|I_2M_{I_2}\rangle$. 
}

\item
{
Finally, we identify $|i\rangle \leftrightarrow |NM_N\rangle|SM_S\rangle|lm_l\rangle|I_1M_{I_1}\rangle|I_2M_{I_2}\rangle$ and $|p\rangle \leftrightarrow |(NS)jlJM\rangle|I_1M_{I_1}\rangle|I_2M_{I_2}\rangle$ and use \cref{eq:finalFT} to obtain the desired matrix elements of $\bm{K}^{\text{sr}}$ in the basis of eigenstates of $\hat{H}_\text{as}$.
}


\end{enumerate}



%\textcolor{blue}{Also, the ground rotational state of the NH molecule is well represented with Hunt case (b) type $\hat{H}_0^\text{as}$ and the eigenstates of  Above stat Once we determine the basis sets for representing $\hat{H}_0$ ($\hat{H}_0^\text{as}$) and $\hat{H}$ ($\hat{H}^\text{as}$),$\bm{K}_\mathrm{0}^{\text{sr}}$ with the eigenstate of $\hat{H}^\text{as}$ $\bm{\hat{H}}^\text{as} \bm{\Phi}_i=E_i \bm{\Phi}_i$, where $\bm{\Phi}_i$ is an eigenvector of $\bm{\hat{H}}^\text{as}$. We need to represent $\bm{U}$, eigenvectors of  by the eigand the basis set for the short-range CC calculations are the same as the above calculations for \cref{Fig_1}, thus we can start from the same $\bm{K}_\mathrm{0}^\text{SR}$ without any additional calculations. On the other hand, $\bm{K}_\mathrm{0}^\text{SR}$ does not have a dimension for the nuclear spin degrees of freedom, thus we need to redefine $\bm{K}_\mathrm{0}^\text{SR}$ by extending as $\bm{K}_\mathrm{0}^\text{SR}\otimes\bm{I}$, where $\bm{I}$ is a 4x4 unit matrix corresponding to the space composed of 4 possible linear independent nucleus spin states such as $M_{I_\text{N}} =M_{I_\text{H}} =1/2$, $M_{I_\text{N}} =1/2$ and $M_{I_\text{H}} =-1/2$, $M_{I_\text{N}}=-1/2$ and $M_{I_\text{H}} =1/2$, and $M_{I_\text{N}}=M_{I_\text{H}} =-1/2$. However, The required information is the relation between the asymptotic (channel) basis of $H_0$ and the full Hamiltonian of the system. Here, we denote the sets of the asymptotic (channel) basis of $H_0$ as $\{\Phi^i_0\}$ and that of the full Hamiltonian as $\{\Phi^j\}$, and the primitive basis set used in solving the short-range CC equations as $\{\phi^k\}$. If we know the expressions of the $\{\Phi^i_0\}$ and $\{\Phi^j\}$ in a common basis set, say $\{\phi^k\}$, the frame transformation. For the calculations of Fig.~1 in the main text,  }

\vspace{0.2cm}
\section{\label{sec:SM_SR} Electron spin transitions}

%%%%%%%%%%%%%%%%%%%%%%%%%%%%%%%%%%%%%%%%%%%%%%%%%%%%%%
\begin{figure}[t!]
\begin{center}
\includegraphics[height=0.32
\textheight,keepaspectratio]{SM_Fig_1.pdf}
\end{center}
\caption{Probabilities for the spin relaxation transitions (a) $M_S=1$ ($l=1$) $\to$ $M^\prime_S=0$ ($l^\prime=1$) and (b) $M_S=1$ ($l=0$) $\to$ $M^\prime_S=-1$ ($l^\prime=2$) in Mg~+~$^{14}$NH ($N=0$) collisions. Open circles --  exact CC calculations, solid lines -- MQDT-FT results. The magnitude of the external magnetic field is indicated next to each curve. 
%shown by the color of purple ($B=1000$\,G), blue ($B=100$\,G), green ($B=10$\,G), and red ($B=1$\,G).
}
\label{SM_Fig_1}
\end{figure}
%%%%%%%%%%%%%%%%%%%%%%%%%%%%%%%%%%%%%%%%%%%%%%%%%%%%%%


%%%%%%%%%%%%%%%%%%%%%%%%%%%%%%%%%%%%%%%%%%%%%%%%%%%%%
\begin{figure}[b!]
\begin{center}
%\includegraphics[width=\linewidth]{SM_Fig_2.pdf}
\includegraphics[height=0.33\textheight,keepaspectratio]{SM_Fig_2.pdf}
\end{center}
\caption{
Probabilities for the state-to-state transitions ($l=0$ $\to$ $l^\prime=2$) at the magnetic field of (a) $B=100$\,G (b) and $B=1000$\ 
in Mg+$^{15}$NH ($N=0$) collisions.
Open circles --  exact CC calculations, solid lines -- MQDT-FT results. The initial and final states of $^{15}$NH ($N=0$) are indicated next to each curve.
%as $12 \to 1$ (purple), $12 \to 2$ (blue), $12 \to 3$ (green), and $12 \to 4$ (red). 
}
\label{SM_Fig_2}
\end{figure}
%%%%%%%%%%%%%%%%%%%%%%%%%%%%%%%%%%%%%%%%%%%%%%%%%%%%%%


To further illustrate the good agreement between our MQDT-FT results and exact CC calculations (see Fig.~1 of the main text), we show in \cref{SM_Fig_1} a sampling of state-to-state transitions in ultracold Mg~+~NH collisions. 
While our main interest here is in ultracold $s$-wave collisions, we also show the results for the $M_S=1$ ($l=1$) $\to$ $M^\prime_S=0$ ($l^\prime=1$) transition in panel (a) to demonstrate the capability of MQDT-FT to describe $p$-wave collisions. In panel (b), we show the results for the $M_S=1$ ($l=0$) $\to$ $M^\prime_S=-1$ ($l^\prime=2$) transition, which is similar to the  $M_S=1$ ($l=0$) $\to$ $M^\prime_S=0$ ($l^\prime=2$) transition (see Fig.~1 of the main text).  
%We show the MQDT-FT results for the $M_S=1$ ($l=0$) $\to$ $M^\prime_S=0$ ($l^\prime=2$) transition in Fig.~1 in the main text. Here we show other relevant partial-wave resolved state-to-state transitions in \cref{SM_Fig_1}. While our main concern in this paper is the ultracold (incoming) $s$-wave collisions, we show the results for $M_S=1$ ($l=1$) $\to$ $M^\prime_S=0$ ($l^\prime=1$) in (a) to demonstrate the capability of the MQDT-FT method for $p$-wave. In (b), we show $M_S=1$ ($l=0$) $\to$ $M^\prime_S=-1$ ($l^\prime=2$), which is similar to the $M_S=1$ ($l=0$) $\to$ $M^\prime_S=0$ ($l^\prime=2$) in Fig.~1 of the main text. 








%\vspace{0.5cm}
\section{\label{sec:SM_hf} Hyperfine transitions}

In \cref{SM_Fig_2}, we show the cross sections for the hyperfine transitions to the final states of $^{15}$NH ($N=0$) corresponding to $M^\prime_S=-1$ (see the main text) in $s$-wave collisions of Mg atoms with  $^{15}$NH molecules initially in the highest energy state (12). We observe excellent agreement between MQDT-FT and exact CC results for most transitions. The agreement improves at higher magnetic fields.
%initially in the highest state (12th state) of the rotational ground state of NH. We observe excellent agreement with the rigorous CC results despite of neglecting the nuclear spin-dependent Hamiltonian in the short-range for the MQDT-FT calculation, leading to a significant reduction of computational time and the amount of memory use. 
%On the other hand, we observe a slight discrepancy for the $12 \to 2$ transition.    




 %%%%%%%%%%%%%%%%%%%%%%%%%%%%%%%%%%%%%%%%%%%%%%%%%%%%%%
\begin{figure}[t!]
\begin{center}
\includegraphics[height=0.28\textheight,keepaspectratio]{SM_Fig_3.pdf}
\end{center}
\caption{Elements of the $2\times2$ short-range K-matrix plotted as functions of collision energy at  $B=1000$\,G for the spin relaxation transition $M_S=1$ ($l=0$) $\to$ $M^\prime_S=0$ ($l^\prime=2$) in Mg+$^{14}$NH ($N=0$) collisions. 
}
\label{SM_Fig_3}
\end{figure}
%%%%%%%%%%%%%%%%%%%%%%%%%%%%%%%%%%%%%%%%%%%%%%%%%%%%%%

\begin{center}
\section{\label{sec:SM_Y} Short-range K-matrix}
\end{center}

In Fig.~3 of the main text, we show MQDT-FT results for  spin relaxation cross sections in Mg~+~NH collisions obtained using a reduced $2\times2$ short-range $K$-matrix. \Cref{SM_Fig_3} shows the variation of the matrix elements $K_{ij}^\text{sr}$ with collision energy. Both diagonal (blue and red) and off-diagonal (green) elements are constant throughout the low energy region ($E_\text{c} < 10$mK), indicating that 3 parameters are sufficient to describe spin relaxation in cold Mg+NH collisions over a wide range of collision energies spanning 4 orders of magnitude.

We also observed no magnetic field dependence of $K_{ij}^\text{sr}$. In particular, the off-diagonal matrix element $K_{M_S=1(l=0),M_S=0(l=2)}^\text{sr}$ is conserved to within 6 significant digits in the $B=1-1000$\,G range at $E_\text{c}=10^{-6}$K.
%The stability of the matrix elements of the matrix about energy is displayed in \cref{SM_Fig_3}. Both diagonal (blue and red) and off-diagonal (green) elements behave as constant values through low energy region ($E_\text{c} < 10$mK), indicating the 3 parameters are sufficient to describe the short-range information for this transition in cold and ultracold energy regimes. The stability against the magnetic field is also observed. We observe the conservation of the 6-digits for the off-diagonal matrix element in the $B=1-1000$\,G region with $E_\text{c}=10^{-6}$K. 
\newline






\vspace{0.8cm}
\section{\label{sec:SM_fine} Fine structure transitions}

%%%%%%%%%%%%%%%%%%%%%%%%%%%%%%%%%%%%%%%%%%%%%%%%%%%%%%
\begin{figure}[h!]
\begin{center}
\includegraphics[height=0.27\textheight,keepaspectratio]{SM_Fig_4.pdf}
\end{center}
\caption{
Fine structure transition probabilities in ultracold 
 Mg~+~$^{14}$NH($N=3$, $j=3$) collisions plotted as functions of collision energy for all possible final states and $l=0$. 
}
\label{SM_Fig_4}
\end{figure}
%%%%%%%%%%%%%%%%%%%%%%%%%%%%%%%%%%%%%%%%%%%%%%%%%%%%%%

To calculate the cross sections for fine-structure transitions  via the MQDT-FT procedure outlined in the main text, it is necessary to express the short-range $K$-matrix in
  the $|(NS)jlJM\rangle$ basis. However, our short-range $K$-matrix  is defined in the spin-free  $|(Nl)J_rM_r\rangle$ basis. These basis sets are related by an analytical recoupling transformation given by Eq.~(2.27) of Ref.~\cite{Corey_83}
\begin{equation}
\begin{split}
K^{\text{sr},J}_{N'Sj'l',NSjl} =  \sum_{J_r=|N-l|}^{N+l} [J_r][j' j]^{1/2} (-1)^{-l'+l+j'-j}   
\begin{Bmatrix} S & N' & j' \\ l' & J & J_r \end{Bmatrix} 
\begin{Bmatrix} S & N  & j  \\ l & J & J_r \end{Bmatrix}
K^{\text{sr},J_r}_{0,N'l',Nl},
\label{eq:7}
\end{split}
\end{equation}
where $\{\}$ denotes a 6-$j$ symbol, $[J_r]=(2J_r+1)$, and  $[j,j']=(2j+1)(2j'+1)$.



In Fig.~4(b) of the main text, we show the cross sections for fine structure  transitions from the $N=3$, $j=3$ ($l=0$) initial state of NH in cold collisions with Mg atoms.
\Cref{SM_Fig_4} shows an extended sample of the results for all possible final states of NH using the recoupling FT technique (see the main text). All the transitions accompanied by rotational de-excitation ($N=3$ $\to$ $N^\prime=0,1,$ and $2$) are seen to be accurately described by MQDT-FT. For the $N$-conserving, $j$-changing transitions within the $N=3$ manifold, we observe a slight discrepancy between the CC and MQDT results for $j^\prime=2$. The overall collision energy dependence of this transition is well reproduced by MQDT-FT. 
%On the other hand, we observe excellent agreement for $j^\prime=4$. 
This can be explained by noting that the recoupling transformation (Eq.~(10) of the main text) relies on the electron spin playing a spectator role and neglects the intramolecular spin-spin interaction in NH, which plays an important role in $N$-conserving fine structure transitions. 

%Here, we show the results of the transitions to all possible final states in \cref{SM_Fig_4} using the recoupling technique for the frame transformation of the K-matrix. All the transitions accompanied by the rotational relaxations ($N=3$ $\to$ $N^\prime=0,1,$ and $2$) are calculated precisely with the MQDT-FT using the recoupling technique. For the pure $j$-state transitions in $N=3$, we see a slight discrepancy between the CC and MQDT results for $N^\prime=3$, $j^\prime=2$ despite the overall collision energy dependence is reproduced well. On the other hand, we observe excellent agreement for $N^\prime=3$, $j^\prime=4$ as written shown in the main text. It is not surprising if the MQDT-FT does not work for the $N$-conserving transition as long as we use the recouping technique because it relies on the assumption that the spectator role of spin in the rotational (rovibrational) transitions. However, further investigation and analysis may help us to extend the recoupling technique or find an effective method for all the $N$-conserving fine state transitions.  



\bibliography{Master.bib,MQDTFT.bib,extra.bib,MQDTFT.bib}
\end{document}


