While sampling from the pathwise smoothing distribution of a LGSSM is rather easy, it naturally has a computational complexity of $\bigO(T)$ in the number of time steps. However, this can be improved by parallelisation. In this section, we present two different methods to sample from a LGSSM in logarithmic $\bigO(\log(T))$ time. The methods are based on similar ideas as the parallel smoothing methods presented in \citet{Sarkka2021temporal}. One of the method uses a computational primitive called prefix-sum or associative scan~\citep{blelloch1989scans}, which generalises cumulative sums to other (associative) operators than addition, while the second one relies on a divide-and-conquer mechanism. It is worth noting that we expect the former to perform better, in particular due to its lower memory requirements, but the second one can be used more easily in distributed settings.

Putting aside the notations of the rest of the article, in this section we consider a generic LGSSM given by its joint distribution $q(x_{0:T}, y_{0:T})$ over the states and observations such that
\begin{equation}
    \label{eq:general-LGSSM}
    \begin{split}
        X_0 &\sim \mathcal{N}(m_0, P_0) \\
        X_t &= F_{t-1} X_{t-1} + b_{t-1} + e_{t-1}, \quad t > 0\\
        Y_t &= H_{t} z_{t} + c_{t} + r_{t}, \quad t \geq 0,
    \end{split}
\end{equation}
with $e_t \sim \mathcal{N}(0, Q_t)$ and $r_t \sim \mathcal{N}(0, R_t)$ for all $t \geq 0$.