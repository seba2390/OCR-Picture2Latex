\documentclass[letterpaper, 10 pt, conference]{ieeeconf}  % Comment this line out if you need a4paper
  \usepackage{pgfplots}
  \pgfplotsset{compat=newest}
  %% the following commands are needed for some matlab2tikz features
  \usetikzlibrary{plotmarks}
  \usetikzlibrary{arrows.meta}
  \usepgfplotslibrary{patchplots}
  \usepackage{grffile}
  \usepackage{amsmath}
  \usepackage{blindtext}
  \usepackage[bottom]{footmisc}
  \usepackage{mathrsfs}
  \usepackage{lipsum}
 \usepackage{url}
 % table
 \usepackage{array}
\usepackage{booktabs}
\usepackage{float}
\usepackage{multirow}
\usepackage{cite}

\usepackage{svg}

\newlength\fwidth
 \usepackage[linesnumbered,ruled]{algorithm2e}

\usepackage{graphicx}
\usepackage{framed}
% \usepackage{subcaption}
\usepackage{indentfirst}
\usepackage{multirow}
\usepackage{makecell}
\let\proof\relax
\let\endproof\relax
\usepackage{amsmath, amssymb,amsthm}
\usepackage{color}
\usepackage{booktabs}
\usepackage{tabularx}
\usepackage{hyperref}
\usepackage[capitalise]{cleveref}
\usepackage{comment}

\usepackage{makecell}
\let\labelindent\relax
\usepackage{enumitem}

\setcellgapes{1pt}
\newcommand{\shrink}{\vspace{-10pt}}
\newcommand{\shrinktop}{\vspace{-16pt}}
\newcommand{\shrinkbottom}{\vspace{-7pt}}

% \theoremstyle{exampstyle}
% \newtheorem{hypothesis}{H}
\newtheoremstyle{exampstyle}
{2pt} % Space above
{2pt} % Space below
{\itshape} % Body font
{} % Indent amount
{\bfseries} % Theorem head font
{.} % Punctuation after theorem head
{.5em} % Space after theorem head
{\thmname{#1}\thmnumber{#2}\thmnote{(#3)}} % Theorem head spec (can be left empty, meaning `normal')   
\theoremstyle{exampstyle}
\newtheorem{hypothesis}{H}

\usepackage{hyperref}

\newcommand{\ww}{w}                     % weight

\newcommand{\eref}[1]{Eq.~\eqref{#1}}  % Eqn 
\newcommand{\sref}[1]{Sec.~\ref{#1}}    % Section 
\newcommand{\figref}[1]{Fig.~\ref{#1}}  % Figure
\newcommand{\tabref}[1]{Table~\ref{#1}} % Table
\newcommand{\algoref}[1]{Alg.~\ref{#1}}  % Algorithm
\newcommand{\prg}[1]{\noindent\textbf{#1}} %Paragraph
%\documentclass[a4paper, 10pt, conference]{ieeeconf}      % Use this line for a4 paper

\IEEEoverridecommandlockouts                              % This command is only needed if 
                                                          % you want to use the \thanks command

\overrideIEEEmargins                                      % Needed to meet printer requirements.

%In case you encounter the following error:
%Error 1010 The PDF file may be corrupt (unable to open PDF file) OR
%Error 1000 An error occurred while parsing a contents stream. Unable to analyze the PDF file.
%This is a known problem with pdfLaTeX conversion filter. The file cannot be opened with acrobat reader
%Please use one of the alternatives below to circumvent this error by uncommenting one or the other
%\pdfobjcompresslevel=0
%\pdfminorversion=4

% See the \addtolength command later in the file to balance the column lengths
% on the last page of the document

% The following packages can be found on http:\\www.ctan.org
%\usepackage{graphics} % for pdf, bitmapped graphics files
%\usepackage{epsfig} % for postscript graphics files
%\usepackage{mathptmx} % assumes new font selection scheme installed
%\usepackage{times} % assumes new font selection scheme installed
%\usepackage{amsmath} % assumes amsmath package installed
%\usepackage{amssymb}  % assumes amsmath package installed
\usepackage[colorinlistoftodos]{todonotes}
% \usepackage[disable]{todonotes}
\newcommand{\mengxi}[1]{\todo[inline,color=blue!20]{M: #1}}
\newcommand{\dylan}[1]{\todo[inline,color=teal!20]{Dy: #1}}
\newcommand{\jean}[1]{\todo[inline,color=red!40]{J: #1}}
\newcommand{\dorsa}[1]{\todo[inline,color=orange!20]{Do: #1}}
\newcommand{\mx}[1]{\textcolor{black}{#1}}
\newcommand{\mxdel}[1]{\textcolor{blue}{#1}}
\usepackage[font=footnotesize]{caption}
\usepackage{balance}
\usepackage{hyperref}
 \hypersetup{
     colorlinks=true,
     linkcolor=orange,
     filecolor=orange,
     citecolor=orange,      
     urlcolor=orange,
     }
% scriptsize
% footnotesize
% small
% normalsize
% large
% Large

% \title{\LARGE \bf
% Learning Robot Objectives from pHRI \\by Reasoning over Sequences of Corrections
% }
\title{\LARGE \bf
Learning Human Objectives from Sequences of Physical Corrections
}


\author{Mengxi Li$^{1}$, Alper Canberk$^{1}$, Dylan P. Losey$^{2}$, Dorsa Sadigh$^{1}$
\thanks{$^{1}$Intelligent and Interactive Autonomous Systems Group (\href{https://iliad.stanford.edu/}{ILIAD}), Dept of Computer Science, Stanford University, Stanford, CA 94305. \newline
$^{2}$ Collaborative Robotics Lab (\href{https://collab.me.vt.edu/}{Collab}), Virginia Tech.
\newline
{(e-mail: mengxili@stanford.edu)}}%
}


\begin{document}



\maketitle
\thispagestyle{empty}
\pagestyle{empty}


%%%%%%%%%%%%%%%%%%%%%%%%%%%%%%%%%%%%%%%%%%%%%%%%%%%%%%%%%%%%%%%%%%%%%%%%%%%%%%%%
\begin{abstract}

When personal, assistive, and interactive robots make mistakes, humans naturally and intuitively correct those mistakes through physical interaction. In simple situations, one correction is sufficient to convey what the human wants. But when humans are working with multiple robots or the robot is performing an intricate task often the human must make \textit{several} corrections to fix the robot's behavior. Prior research assumes each of these physical corrections are \textit{independent} events, and learns from them one-at-a-time. However, this misses out on crucial information: each of these interactions are \textit{interconnected}, and may only make sense if viewed together. Alternatively, other work reasons over the \textit{final} trajectory produced by all of the human's corrections. But this method must wait until the end of the task to learn from corrections, as opposed to inferring from the corrections in an online fashion. In this paper we formalize an approach for learning from sequences of physical corrections during the current task. To do this we introduce an auxiliary reward that captures the human's trade-off between making corrections which improve the robot's immediate reward and long-term performance. We evaluate the resulting algorithm in remote and in-person human-robot experiments, and compare to both \textit{independent} and \textit{final} baselines. Our results indicate that users are best able to convey their objective when the robot reasons over their sequence of corrections.

\end{abstract}


%%%%%%%%%%%%%%%%%%%%%%%%%%%%%%%%%%%%%%%%%%%%%%%%%%%%%%%%%%%%%%%%%%%%%%%%%%%%%%%%
Reinforcement learning has achieved great success in areas such as Game-playing \citep{silver2018general,vinyals2019grandmaster}, robotics \cite{kober2013reinforcement}, large language models \citep{ouyang2022training}, etc.
However, due to safety concerns or physical limitations, in some real-world reinforcement learning problems, we must consider additional constraints that may influence the optimal policy and the learning process \citep{garcia2015comprehensive}.
% For example, a robotic arm must not take actions that may cause harm to itself or the environments.
A standard framework to handle such cases is the constrained Markov Decision Process (CMDP) \citep{altman1999constrained}.
Within the CMDP framework, the agent has to maximize
the expected cumulative reward while
obeying a finite number of constraints, which are usually in the form of expected cumulative cost criteria.

However, we are sometimes concerned with the problem with a continuum of constraints.
For example,
the constraints we meet might be time-evolving or subject to uncertain parameters, which
cannot be formulated as an ordinary CMDP
(see Examples \ref{Example_Time_Evolving} and  \ref{Example_Uncertain}).
In this paper we would study a generalized CMDP  
to address the above problem.  Because the constraints are not only infinite-number but also lie
in a continuous set,
the generalization is not trivial. Fortunately, we find that we can borrow the idea behind semi-infinite programming (SIP) \citep{remez1934determination, hettich1993semi} to deal with the semi-infinite constraints.
Accordingly, we propose \emph{semi-infinitely constrained Markov decision processes} (SICMDPs)
as a novel complement to the ordinary CMDP framework.
%More specifically,  an SICMDP model %, we consider 
%contains a continuum of constraints whereas an ordinary CMDP contains a finite number of constraints. 

%This generalization is natural but not trivial. However, we can brows the idea  
%The idea is quite natural and can be backtracked
%to the practice of extending linear programming to linear semi-infinite programming (LSIP) %\cite{remez1934determination, GobernaLSIO1998}.
%In addition, 
%As a complementary approach to the ordinary CMDP framework, 
%SICMDP can be used to model these problems  which cannot be described by a finite number of constraints
%that are not covered by .
%For example,
%the restrictions we consider can be time-evolving or subject to uncertain parameters
%, thus
%cannot be described by a finite number of constraints but a continuum of constraints 
%(see Examples \ref{Example_Time_Evolving} and  \ref{Example_Uncertain}).

We also present two reinforcement learning algorithms to solve SICMDPs called SI-CRL and SI-CPO, respectively.
SI-CRL is a model-based reinforcement learning algorithm designed for tabular cases, and SI-CPO is a policy optimization algorithm for non-tabular cases.
% and analyze its performance both theoretically and empirically.
The main challenge is that we need to deal with a continuum of constraints, thus reinforcement learning algorithms for ordinary CMDPs do not work anymore.
In SI-CRL, we tackle this difficulty by first transforming the reinforcement learning problem to an equivalent LSIP problem, which can then be solved using methods in the LSIP literature like the dual exchange methods \citep{Hu1990,reemtsen1998numerical}.
In SI-CPO, we resort to the idea of cooperative stochastic approximation developed in \cite{lan2020algorithms, wei2020comirror}.
As far as we know, we are the first to introduce tools from semi-infinitely programming (SIP) into the reinforcement learning community for solving constrained reinforcement learning problems.

% To the best of our knowledge, we are the first to apply tools from semi-infinitely programming (SIP) to solve reinforcement learning problems.
Furthermore, we give theoretical analysis for both SI-CRL and SI-CPO.
We decompose the error of SI-CRL into two parts: the statistical error from approximating the true SICMDP with an offline dataset and the optimization error due to the fact that the solution of the LSIP problem obtained by the dual exchange method is inexact.
On the optimization side, we show that the iteration complexity of SI-CRL is $O\left(\left\{\mathrm{diam}(Y)L\sqrt{|\gS|^2|\gA|m}/\left[(1-\gamma)\epsilon\right]\right\}^m\right)$.
On the statistical side, we show that the sample complexity of SI-CRL is $\widetilde O\left(\frac{|S|^2|A|^2}{\epsilon^2(1-\gamma)^3}\right)$ if the offline dataset is generated by a generative model, and $\widetilde O\left(\frac{|S||A|}{\nu_{\min} \epsilon^2(1-\gamma)^3}\right)$ if the dataset is generated by a probability measure $\nu$ as considered in \cite{chen2019information}.
Here $\widetilde O$ means that all logarithm terms are discarded.
For SI-CPO, things become a little more complicated because other than the statistical error and the optimization error, we also need to consider the function approximation error, which comes from imperfect policy parametrizations.
It is shown if the function approximation error can be controlled to $O(\epsilon)$ order, the iteration complexity of SI-CPO is $\widetilde{O}\left(\frac{1}{\epsilon^2(1-\gamma)^6}\right)$ and the sample complexity of SI-CPO is $\widetilde{O}(\frac{1}{\epsilon^4(1-\gamma)^{10}})$.
Here our iteration complexity bound is equivalent to a typical $\widetilde O(1/\sqrt{T})$ global convergence rate.

We perform a set of numerical experiments to illustrate the SICMDP model and validate our proposed algorithms.
Specifically, we examine two numerical examples, namely the discharge of sewage and ship route planning.
Through the discharge of sewage example, we show the advantage of the SICMDP framework over the CMDP baseline obtained by naive discretization in modeling realistic sequential decision-making problems.
Moreover, we demonstrate the effectiveness of the SI-CRL and SI-CPO algorithms in such tabular environments. 
In the ship route planning example, we illustrate the benefits of the SICMDP framework and the ability of the SI-CPO algorithm to address complex continuous control tasks involving continuous state spaces with modern deep reinforcement learning techniques.

% In summary, our contributions are listed as follows.
% First, we present the SICMDP model, which can be viewed as a generalization of the ordinary CMDP model.
% Second, we propose an algorithm to perform reinforcement learning for SICMDPs, which is called SI-CRL, and we believe that we are the first to apply tools from SIP
% to solve reinforcement learning problems.
% Third, we give a theoretical analysis of SI-CRL and identify both its sample complexity and iteration complexity.
% In addition, we perform numerical experiments to illustrate the SICMDP model and validate the SI-CRL algorithm.
% \{This paragraph can be removed!!! \}





The industry standard for pose edition is to create rigs, a collection of pieces of software designed to manipulate a character's skeleton. The rig describes the skeleton's bones, how they relate to each other, are constrained in their possible motion and are deformed. These rules are loosely specified and creating a good rig requires a detailed understanding of physics and anatomy, as well as technical and artistic skills. Rigging is thus a time consuming task even for experienced animators, and even more so in large scale productions which often require a different in-depth rig for each character in the cast.
Previous work has helped alleviate this difficulty by providing efficient tools to speed up/and or ease the rigging process, relying on inverse kinematics or data-driven methods.
\subsection{Character pose design}
\subsubsection{Inverse Kinematics (IK)}
IK solvers are a family of methods commonly used in robotics, engineering and computer graphics, in which the parameterization of a kinematic chain is determined from the position of its end effector.
They are a staple tool in pose design software, ensuring the respect of elementary constraints during pose edition. Their de-facto role is to guarantee the length of the limbs, and in some cases to enforce the orientation angle range of a joint.
Many IK solutions have been studied over the years \cite{aristidou_inverse_2018}; usually revolving around approximated linearizations or heuristics. 

Numerical methods require a set of iterations to achieve a satisfactory solution formulated by a cost function to be minimized.
IK solutions can generally be divided into three sub-categories: Jacobian \cite{Siciliano_Handbook_Robot_2007}, Newtonians \cite{cohen_ik_1996} and Heuristics. Most software implement heuristic methods such as Cyclic Coordinate Descent (CCD) \cite{wang_ccd_1991} or 
Forward-Backward Reaching IK (FABRIK) \cite{aristidou_fabrik:_2011} due to their simplicity and extensibility. 

The main drawback of 
these solvers is that they manipulate kinematic chains without taking into account many morphological aspects that make a pose more or less plausible. They offer a first level of help to users but are not sufficient to guarantee a realistic pose. Many joints constraints are dependent on each other and require subjective, human-made approximations.

\subsubsection{Data-driven pose edition}
Data-driven methods offer promising opportunities to solve these approximations. Using real-life data can help in modelling the complex inter-dependencies of skeletons and providing users with smarter edition tools.
While it is still an early field of research, some solutions have been studied. Wu \etal \cite{wu_posing_2009} propose a method for natural character posing from a large motion database. It employs adaptive KD-clustering to select a representative frame from a database and sparse approximations to accelerate training and posing. 
Huang \etal in \cite{Huang_IK_MGDM_2017} present a method based on the formulation of multi-variate Gaussian distribution models (MGDMs), which learn the joint constraints of a kinematic skeleton from motion capture data. 

Some work has also been dedicated to finding new editing interfaces. \modify{}{Instead of the usual setup manipulating joints directly, Guay \etal \cite{guay_line_2013} articulate a framework based on the conceptual "line of action" which describes the overall pose dynamics. They provide a mathematical definition of the line of action, and a interface in which the software modifies the pose to follow a user-provided line. In the same line of though} Garcia \etal \cite{garcia_sketching_2019} propose \modify{a method transforming doodle of trajectories (position and orientation over time) }{a virtual reality-based interface where the user's hands motion (position and orientation over time) are transformed} into sequences of actions and then into detailed character animations using a dataset of parametrized motion clips automatically fitted to the trajectory. 

% ==> DL et Latent Space. 
\subsection{Neural modelling of human motion}
Neural networks have received a great amount of attention over the last decade and shown impressive result in modelling complex data. Human motion has not been spared and deep learning methods have proven their capability of generating realistic motion in a number of difficult cases. 

The literature in neural-based animation include example in user-controlled character navigation \cite{Holden2017} and interactions with the environment \cite{starke_neural_2019}. 
Holden \etal \cite{Holden2020} also show that neural networks can be used to replace parts of existing data-driven methods, improving their scalability potential.
More recently, some work has also focused on improving smaller parts of the animation pipeline rather than replacing it completely. Berson et al. \cite{berson_intuitive_2020} leverage neural networks to provide an interactive system to edit facial animation. 

% Wrap up
Data-driven IK and pose editing can relieve animators from time-consuming, back-and-forth pose adjustments by applying constraints extracted from real-world data. Recently, neural-network-based approaches have demonstrated their ability to model the intricacies of human motion while scaling to large amount of data and retaining a fast inference time. In this paper we seek to take advantage of these properties to create an efficient posing tool, intuitively usable even by a inexperienced user.
\section{Formalizing Sequences of Physical Corrections}
\label{sec:formalism}

In this section we formalize a physical human-robot interaction setting where one or more robots are performing a task incorrectly. The human expert knows how these robots should behave, and physically corrects the robots to convey the true objective. But the human doesn't interact just once: the human may need to interact \textit{multiple times} in order to correct the robots. Our goal is for these robots to learn the human's objective from this sequence of physical corrections.


\subsection{Task Formulation} 
\label{sec:task_formulation}

We formulate our problem as a discrete-time Markov Decision Process (MDP) $\mathcal{M}=\left(\mathcal{S}, \mathcal{A}, \mathcal{T}, r, \gamma, \rho_{0}\right)$. Here $\mathcal{S} \subseteq \mathbb{R}^{n}$ is the \mx{robot} state space, $\mathcal{A} \subseteq \mathbb{R}^{m}$ is the robot action space, $\mathcal{T}(s, a)$ is the transition probability function, $r$ is the reward, $\gamma$ is the discount factor, and $\rho_{0}$ is the initial distribution.

\smallskip
\prg{Reward.}
Let the robot start from a state $s^0$ at time $t=0$. As the robot completes the task it follows a trajectory of states: $\xi = \{s^0, s^1, \dots, s^T\} \in \Xi$. The human has in mind a trajectory that they prefer for the robot to follow. Recall our motivating example --- here the human wants the robot arms to follow a trajectory that avoids the cabinets without squashing the bag. Similar to prior work \cite{ziebart2008maximum, abbeel2004apprenticeship, osa2018algorithmic, jeon2020reward}, we capture this objective through our reward function:
$R(\xi; \theta) = \theta \cdot \Phi(\xi)$.
Here $\Phi$ denotes the feature counts over trajectory $\xi$, and $\theta$ captures how important each feature is to the human. We let $\xi_R^t$ denote the robot's trajectory at timestep $t$, and we let $\theta^t$ denote the robot's current reward weights.

\smallskip
\prg{Suboptimal Initial Trajectory.} The system of one or more robots starts off with an initial reward function $R(\xi; \theta^0)$, and optimizes this reward function to produce its initial trajectory.
$$\xi_R^{0}=\arg \min _{\xi \in \Xi} \theta^{0} \cdot \Phi(\xi)$$
But this initial trajectory $\xi_R^0$ misses out on what the human really wants --- going back to our example, the robot does not realize that the blue region is wet and it needs to place the bag on the green region.
More formally, the robot's estimated reward function (which is parameterized by $\theta^0$) does not match the human's preferred reward function (parameterized by the true weights $\theta^*$).

\smallskip
\prg{Human Corrections.} The robot learns about the human's reward --- i.e., the true reward weights --- from physical corrections. Intuitively, these corrections are applied forces and torques which push, twist, and guide the robots. To formulate these interactions we must revise our problem definition: let $a_R$ be the robot's action and let $a_H$ be the human's \textit{correction}. In practice, both $a_R$ and $a_H$ could be applied joint torques \cite{bajcsy2017learning, bajcsy2018learning}. Now the overall system transitions based on both human and robot actions: $s^{t+1} = \mathcal{T}\left(s^t, a_{R}+a_{H}\right)$. We use $A_H = \{(t_i, a_H^i), i=1,\ldots, K\}$ to denote a \textit{sequence} of $K$ ordered human corrections $a_H^i$ at time step $t_i$, where $i$ keeps track of order of the corrections.
Our goal is to learn the human's true reward weights from the sequence of corrections $A_H$.

% We denote the robot(s) action with $u_R$. At every time step, the user might physically correct the robot team with $u_H$, and thus the robot transitions from state $x$ to next state $x\prime$ with the dynamics $x\prime = \mathcal{T}\left(x, u_{R}+u_{H}\right)$ instead of $x\prime = \mathcal{T}\left(x, u_{R}\right)$. We use $U_H = \{(t_i, u_R^i), i=1\ldots K\}$ to denote an assembly of all human correction $u_H^i$ at time step $t_i$. And our goal is to learn the ground truth weight from the human corrections $U_H$.


% \prg{Learning from Physical Corrections.}
% Consider two robots are holding a grocery bag to the table as in \figref{toaddfig}. if the robots do not realize the grocery bag is full, it might squeeze the bag to get through the way without touching nearby housewares. However, realizing that the groceries are going to fall out if the robots keep squeezing, the human would intervene and correct its behaviors by physically pushing the robot arms apart. In this work, we explore how to enable robots to better learn from these physical human corrections.


\subsection{Physical Corrections as Observations}
\label{sec:observation}

When robots are performing a task suboptimally, the human expert intervenes to correct those robots towards the right behavior. Going back to our example from Fig.~\ref{fig:front}. The user sees that the robot is making a mistake (moving towards the wet blue region), and physically intervenes. In the process of fixing this first issue, the human is forced to create another problem: by moving the first robot arm away from the blue region, they also move both arms closer together, and start to squash the bag. We note two important characteristics of these corrections: i) each human correction is intentional, and conveys information about the human's objective, but ii) the corrections viewed together may provide more information than isolating each interaction.

Leveraging these corrections, our goal is to find a better estimate of the reward parameters $P(\theta \mid A_H, \xi_R^0)$. We start by applying Bayes' rule:
\begin{equation} \label{eq:bayes}
   P(\theta \mid A_H, \xi_R^0) \propto P(\theta) P(A_H \mid \xi_R^0,\theta)
\end{equation}
In line with prior work \cite{bajcsy2017learning, bajcsy2018learning, losey2018including, bobu2020quantifying}, we will model $P(A_H|\xi_R^0, \theta)$ by mapping each human correction to a \textit{preferred trajectory}. Given the human's correction $(t_i, a_H^i)$, we deform the robot's trajectory to reach $\xi_H^i$. 
% \mxdel{One simple example of this is to let $\xi_H^i = \xi_R^i$ everywhere except the current waypoint, where we shift the trajectory in the direction of the human's applied correction.}
\mx{One simple example of this is to let the robot execute $a_R^{t_i} + a_H^{t_i}$ at this time step $t_i$, and stick to its original action plan $a_R^t$ for future time steps $t > t_i$.} More generally, we propagate the human's applied correction along the robot's current trajectory \cite{losey2017trajectory}:
\begin{eqnarray}
\centering
\begin{aligned}
    \xi_H^1 &= \xi_R^0 + \mu A^{-1} a_H^1\\
    \xi_H^i &= \xi_H^{i-1} + \mu A^{-1} a_H^i ,~~ i \in \{2,\ldots, K\}
\end{aligned}
\label{eq:traj_prop}
\end{eqnarray}
Consistent with \cite{losey2017trajectory} and \cite{dragan2015movement}, $\mu$ and $A$ are hyperparameters that determine the deformation shape and size. We emphasize that here the robot is not yet learning --- instead, it is locally modifying its trajectory in the direction of the applied correction. Within our motivating example, let the human apply a force pushing the first robot arm away from the blue region. Equation (\ref{eq:traj_prop}) maps this correction to $\xi_H$, a trajectory that moves the robot arm farther from the blue region than $\xi_R$. In Fig.~\ref{fig:propogation_traj}, we demonstrate how a sequence of corrections lead to a sequence of trajectories that enable the robot to correct its path and reach the preferred goals.


% \subsection{pHRI as Observations of Reward Function}
% \label{sec:observation}
% When robots perform a task suboptimally, the human intervenes to correct the robots for higher rewards. These interventions are intentional and informative, and thus we model these human's physical corrections as observations towards the ground truth reward parameter. However, directly modeling $P(U_H|\xi_R^0; \theta)$ is difficult as it requires computing $Q$-value function for that $\theta$ \cite{bajcsy2017learning}. Therefore, instead of directly relating $U_H$ to $\theta$, we interpret each human correction $(t_i, u_H^i)$ with a intermediate corresponding trajectory $\xi_H^i$ as in \cite{bajcsy2017learning}. Formally, for each correction $(t_i, u_H^i)$, we propagate current correction based on previous trajectory $\xi_H^{i-1}$ to get deformed trajectory \cite{losey2017trajectory}:
% \begin{eqnarray}
% \centering
% \begin{aligned}
%     \xi_H^1 &= \xi_R^0 + \mu A^{-1} u_H^1\\
%     \xi_H^i &= \xi_H^{i-1} + \mu A^{-1} u_H^i ,~~ i \geq 2
% \end{aligned}
% \label{eq:traj_prop}
% \end{eqnarray}
% where $\mu$ controls the scale for deformation, $A$ is the differencing matrix
% Returning to our running example. 

Now that we have this tool for mapping corrections to preferred trajectories, we can rewrite Equation~(\ref{eq:bayes}):
\begin{eqnarray}
\begin{aligned}
    P(\theta \mid A_H, \xi_R^0) &\propto P(\theta) P(A_H \mid \xi_R^0, \theta) \\
    &= P(\theta)P\Big((t_1, a_H^1), \ldots, (t_K, a_H^K) \mid \xi_R^0, \theta\Big) \\
    &\approx P(\theta)P(\xi_H^1, \ldots, \xi_H^K \mid \xi_R^0, \theta)
\end{aligned}
\label{eq:bayesian}
\end{eqnarray}
Here $P(\theta)$ is the robot's prior over the human's objective, and $P(\xi_H^1, \ldots, \xi_H^K \mid \xi_R^0, \theta)$ is the likelihood that the human provides a specific \textit{sequence} of preferred trajectories given the robot's initial behavior $\xi_R^0$ and the reward weights $\theta$.


\begin{figure*}[t]
	\begin{center}
		\includegraphics[width=1.8\columnwidth]{figs/propagation_traj.png}

		\caption{An example of a sequence of human corrections along with her corresponding correction trajectories $\xi_H^1, \xi_H^2, \xi_H^3, \xi_H^4$ to guide the robot to place the grocery bag on the green region while avoiding any stretching or squeezing of the bag.}
		\label{fig:propogation_traj}
	\end{center}
	
 	\vspace{-2em}
\end{figure*}


\section{Approach}
\begin{figure}[t]
\centering
\resizebox{0.48\textwidth}{!}{ 
  \includegraphics[width=\textwidth]{figures/workflow.PNG}
}
  \caption{Workflow of \system}
  \label{fig:workflow}
\end{figure}

Figure ~\ref{fig:workflow} shows the overall workflow of \system. The triggers for using \system are usually alert(s) from automated anomaly detection, or sometimes an SRE engineer's suspicion. There are three major steps: constructing the service  dependency graph, constructing the event causality graph,  and root cause ranking. The outputs are the root causes ranked by the likelihood. To support fast human investigation experience, we build an interactive UI as shown in  Figure~\ref{fig:UI}: the service dependency, events with causal links and additional details such as raw metrics or the developer contact (of a code deployment event) are presented to the user for next steps. As an  offline part of human investigation, we label/collect a data set, perform validation, and summarize the knowledge for further improvement on all incidents on a daily basis. %as validations and heterogeneous graph learning (HGL)~\cite{qiao2020heterogeneous} to synthesize the knowledge from existing cases in order to further improve the system.

\subsection{Constructing Service Dependency Graph}
\label{sec:appgraph}

The construction of the service dependency graph starts with the initial alerted or suspicious service(s), denoted as $I$. For example, in Figure ~\ref{fig:ex1_dep}, $I=\{\textit{Checkout}\}$. $I$ can contain multiple services based on the range of the trigger alerts or suspicions. We maintain domain service lists where domain-level alerts can be triggered because there is no clear service-level indication.

At the back end, \system maintains a global service dependency graph $G_{global}$ via distributed tracing and log analysis. The directed edge from nodes $A$ to $B$ (two services or system components) in the dependency graph indicates a service invocation or other forms of dependency. In Figure~\ref{fig:ex1_dep}, the black arrows indicate such edges. Bi-directional edges and cycles between the services can be possible and exist. In this work, the global dependency graph is updated daily.%by extracting from one day's total site traffic.

The service dependency (sub)graph $G$ is constructed using $G_{global}$ and $I$. An extended service list $L$ is first constructed by traversing each service in $I$ over $G_{global}$ for a radius range $r$. Each service $u \in L$ can be traversed by at least one service $v \in I$ within $r$ steps: $L=\{u|\exists v\in I, \ dist(u,v)\le r\ or\ dist(v,u)\le r\}$. Then, the service dependency subgraph $G$ is constructed by the nodes in $L$ and the edges between them in $G_{global}$. In our current implementation, $r$ is set to $2$, since this dependency graph may be dynamically extended in the next steps based on events' detail for longer issue chains or additional dependencies.

\subsection{Constructing Event Causality Graph}
\label{sec:causality}

In the second step, \system collects all supported events for each service in $G$ and constructs the causal links between events. 

\subsubsection{Collecting Events}

Table~\ref{tab:events} presents some example event types and detection techniques for \system's production implementation. For detection techniques, ``De Facto'' indicates that the event can be directly collected via a specific API or storage. %The detection can be done passively at the back end continuously then store anomaly events in different databases; or done actively by pulling data and run detection on the fly to save compute resources. 
The detection either runs passively in the back end to reduce delay and improve accuracy, or runs actively for only the services within the dependency graph range to save resources. %For example, low-level error signals or logs are detected actively since they are too many to scale. 

There are three major categories of events: performance metrics, status logs, and developer activities:
\begin{itemize}
    \item \emph{Performance metrics} represent an anomaly of monitored time series metrics. For example, high CPU usage indicates that the service is causing high CPU usage on a certain machine. In this category, most events are continuously and passively detected and stored. %For high CPU usage, threshold indicates the event is created when CPU usage is higher than certain predefined value. TPS spike indicates a spike in transaction per second, since TPS is a moving average value, we use some statistical model learned from historical data to detect such events.
    \item \emph{Status logs} are caused by abnormal system status, such as spike of HTTP error code metrics while accessing other services' endpoints. Different types of error metrics are important and supported in \system, including third-party APIs. For example, Bad Host indicates abnormal patterns on some machines running the service, and can be detected by a  clustering-based ML approach.%Markdown indicates that the whole service is down. 
    \item \emph{Developer activities} are the events generated when a certain activity of developers is triggered, such as code deployment and config change.
\end{itemize}

\begin{table}[t]
\centering
\caption{List of example event types used in \system}
\resizebox{0.4\textwidth}{!}{ 
\begin{tabular}{|c|c|c|}
\hline
Type                                & Event Type                  & Detection Technique  \\ \hline
\multirow{6}{*}{Performance Metrics} & High GC (Overhead)      & Rule-based        \\ \cline{2-3} 
                                    & High CPU Usage          & Rule-based        \\ \cline{2-3} 
%                                    & Out of Memory           & Rule-based        \\ \cline{2-3} 
%                                    & LB Connection Stacking  & Statistical Model \\ \cline{2-3} 
                                    & Latency Spike           & Statistical Model \\ \cline{2-3} 
                                    & TPS Spike               & Statistical Model \\ \cline{2-3} 
                                    & Database Anomaly        & ML Model          \\ \cline{2-3} 
                                    & Business Metric Anomaly & ML Model          \\ \hline
\multirow{4}{*}{Status Logs}        & WebAPI Error            & Statistical Model \\ \cline{2-3} 
                                    & Internal Error          & Statistical Model \\ \cline{2-3} 
                                    & ServiceClient Error     & Statistical Model \\ \cline{2-3} 
                                    & Bad Host                & ML Model          \\ \hline %\cline{2-3} 
%                                    & Hystrix Circuit Break   & De Facto          \\ \hline
\multirow{3}{*}{Developer Activities} & Code Deployment         & De Facto          \\ \cline{2-3} 
                                    & Configuration Change    & De Facto          \\ \cline{2-3} 
                                    & Execute URL             & De Facto          \\ \hline
\end{tabular}
}
\label{tab:events}
\end{table}

In Groot, there are more than a dozen event types such as \emph{Latency Spike} as listed in the column 2 of Table~\ref{tab:events}. 
Each event type is characterized by three aspects: $Name$ indicates the name of this event type; $Lookback Period$ %\footnote{In Figure~\ref{fig:ex2_n1}, there are two periods, 1 day indicates the look-back range if the model has already finished deployment, 4 days indicates the range if the model deployment is still ongoing(incremental deployment).} 
indicates the time range to look back (from the time when the use of \system is triggered) for collecting events of this event type;  $PropertyType$ indicates the types of the properties that an event of this event type should hold. 
$PropertType$  is characterized by a vector of pairs, each of which indicates the string type for a property's name and the primitive type for the property's value such as string, integer, and float. 
Formally, an event type is defined as a tuple: 
$ET = <Name, Lookback Period, PropertyType>$ 
where 
$PropertyType = <(string, \textit{type}_1), ..., (string, \textit{type}_{n})>$ ($n$ is the number of properties that an event of this event type holds). 
%

Each event of a certain event type $ET$ is characterized by four aspects:
$\textit{Service}$ indicates the service name that the event belongs to; $\textit{Type}$ indicates $ET$'s $\textit{Name}$;  $\textit{StartTime}$ indicates the time when the event happens; $\textit{Properties}$ indicates the properties that the event  holds.
Formally, an event is defined as a tuple: 
$e = <Service, Type, StartTime, Properties>$ 
where $Properties$ is an instantiation of $ET$'s  $PropertyType$. 


%and each event is defined as $e = \{<\textit{Property}_i, \textit{value}_i>\}$. Each event type serves as a template for the event instantiation. such as a string, an integer, a float or a set of primitive types while $\textit{value}$ is limited to primitive data types. 
%
%Each event is defined as a sequence of property-value pairs where the set size is $n$.

For example, in Figure~\ref{fig:example1}, the generated event for \emph{Latency Spike in DataCenter-A} in \emph{Service-C} would be $<``\textit{Service-C}'', ``\textit{Latency\ Spike}'', \textit{2021/08/01-12:36:04}, <(``\textit{DataCenter}'',``\textit{DC-1}''),  ...>>$. %So for each service in $G$, we detect/collect and filter the events within specified time range of the alert.

\subsubsection{Constructing Causal Link}

After collecting all events on all services in $G$, in this step, causal links between these events are constructed for RCA ranking. The causal links (red arrows) in Figure~\ref{fig:ex1_cas} are such examples. A causal link represents that the source event can possibly be caused by the target event. SRE knowledge is engineered into rules and used to create causal links between the pairs of events. %As shown in Figure~\ref{fig:example2}, there are two categories of rules: basic rules and conditional rules. 

A rule for constructing a causal link is defined as a tuple:  $Rule = <Target\mbox{-}Type,  Source\mbox{-}Events, Target\mbox{-}Events, Direction,\\ Target\mbox{-}Service,  Condition>$  ($Condition$ can be optionally specified). $Target\mbox{-}Type$ indicates the type of the rule, being either $Static$ or $Dynamic$ (explained further later). $Source\mbox{-}Events$ indicates the type of the causal link's source event ($Source\mbox{-}Events$ are listed in the names of the rules shown in Figures~\ref{fig:ex2_n1},~\ref{fig:ex2_n2} and~\ref{fig:dynamic_example}).   $Target\mbox{-}Events$ indicates the type of the causal link's target event. $Direction$ indicates the direction of the casual link between the target event and source event. $Target\mbox{-}Service$ indicates the service that the target event should belong to. Note that $Target\mbox{-}Service$ in $Static$ rules can be  $Self$, which indicates that the target event would be within the same service as the source event, or $Outgoing$/$Incoming$, which indicates that the target event would belong to the downstream/upstream services of the service that the source event belongs to in $G$.

\begin{figure}[t]
\centering
\includegraphics[width=0.56\columnwidth]{figures/example3.png}
\caption{Example of dynamic rule}
\label{fig:dynamic_example}
\end{figure}

There are two categories of special rules. The first category is \emph{dynamic} rules (i.e., rules whose $Target\mbox{-}Type$  is set to $Dynamic$) to support dynamic dependencies. Here $Target\mbox{-}Service$ does not indicate any of the three possible options listed earlier but indicates the name of the target service that \system would need to create. For example, live DB dependencies are not available due to different tech stacks and high volume. In Figure~\ref{fig:dynamic_example}, a DB issue (DB Markdown) is shown in \emph{Service-A}. Based on the listed \emph{dynamic} rule, \system creates a new ``service'' \emph{DB-1} in $G$, a new event ``Issues'' that belongs to \emph{DB-1}, and a causal link between the two events.  In practice, the SRE teams use dynamic rules to cover a lot of third-party services and database issues since the live dependencies are not easy to maintain.  %However through the internal error messages and dynamic rules, \system is still able to handle these dependencies. %we can still support external inferences. 

The second category of special rules is \emph{conditional} rules. \emph{Conditional} rules are used when some prerequisite conditions should be satisfied before a certain causal link is created. In these rules, $Condition$ is specified with a boolean predicate. As shown in Figure~\ref{fig:ex2_n2}, the SRE teams believe \emph{Latency Spike} events from different services are related only when both events happen within the same data center. Based on this observation, \system would first evaluate the predicate in $Condition$ and build only the causal link when the predicate is true. A conditional rule overwrites the basic rule on the same source-target event pair.

When constructing causal links, \system first applies the \emph{dynamic} rules so that dynamic dependencies and events are first created at once. Then for every event in the initial services (denoted as $I$), if the rule conditions are satisfied, one or many causal links are created from this event to other events from the same or upstream/downstream services. When a causal link is created, the step is repeated recursively for the target event (as a new origin) to create new causal links. After no new causal links are created, the construction of the event causality graph is finished.

% When \system constructs the causal links, \system first processes all dynamic rules as they may create new event nodes in the graph. %\system enumerates the dynamic rules on each existing event node and also on the newly added nodes (There could also be rules applicable to the newly added nodes) until no new event nodes can be created. 


%Each rule is defined as a predicate containing both events' property-value pair. If the predicate evaluates to be true between two events, then we would add the edge in the causality graph. For example, in Figure~\ref{fig:example1}, the rule used to establish the edge between \emph{GC overhead in RNO} and \emph{Latency increase in LVS, RNO, SLC} would be like this: Suppose we are now determining whether there should be a link from event $u$ to event $v$, then this rule would be $u.\text{pool} = v.\text{pool}\ and\ u.\text{type} = ``\text{High GC Overhead}"\ and\ v.\text{type} = ``\text{Latency increase}"\ and\ u.\text{center} \cap v.\text{center} \ne \emptyset$ which holds true for these two events. Each causality link is also associated with a weight which represents the likelihood of causality - we set all initial values as $1.0$. Overtime these value are updated by the statistical analysis result of the collected data set.


\subsection{Root Cause Ranking}
Finally, \system ranks and recommends the most probable root causes from the event causality graph. Similar to how search engines infer the importance of pages by page links, we customize the PageRank \cite{manning2010introduction} algorithm to calculate the root cause ranking; the customized algorithm is named as GrootRank. The input is the event causality graph from the previous step. Each edge is associated with a weighted score for weighted propagation. The default value is set as $1$, and is set lower for alerts with high false-positive rates. 

Based on the observation that dangling nodes are more likely to be the root cause, we customize the personalization vector as $P_n = f_n $ or $P_d = 1$, where $P_d$ is the personalization score for dangling nodes, and $P_n$ is for the remaining nodes; and $f_n$ is a value smaller than 1 to enhance the propagation between dangling nodes. In our work, the parameter setting is $f_n = 0.5$, $\alpha = 0.85$, $max_{iter} = 100$ (which are parameters for the PageRank algorithm). Figure \ref{fig:person} illustrates an example. The grey circles are the events collected from three services and one database. The grey arrows are the dependency links and the red ones are the causal links with the weight of $1$. Both of the PageRank and GrootRank algorithms detect $event 5$ (DB issue) as the root cause, which is expected and correct. However, the PageRank algorithm ranks $event 4$ higher than $event 3$. But $event 3$ of $\textit{Service-C}$ is more likely to be the second most possible root cause (besides $event 5$), because the scores on dangling nodes are propagated to all others equally in each iteration. We can see that $event 3$ is correctly ranked as second using the GrootRank algorithm.

The second step of GrootRank is to break the tied results from the previous step. The tied results are due to the fact that the event graph can contain multiple disconnected sub-graphs with the same shape. We design two techniques to untie the ranking: 
\begin{figure}[t]
\centering
  \includegraphics[width=0.8\columnwidth]{figures/personalvector.png}
  \caption{Example of personalization vector customization}
  \label{fig:person}
\end{figure}

\begin{figure}[t]
\centering
  \includegraphics[width=0.8\columnwidth]{figures/accessdistance.png}
  \caption{Example of using access distance to untie the ranking results}
  \label{fig:untie}
\end{figure}
\begin{enumerate}
\item For each joint event, the access distance (sum) is calculated from the initial anomaly service(s) to the service where the event belongs to. If any ``access'' is not reachable, the distance is set as $d_m+1$ where $d_m$ is the maximum possible distance. The one with shorter access distance (sum) would be ranked higher and vice versa. Figure \ref{fig:untie} presents an example, where \emph{Service-A} and \emph{Service-B} are both initial anomaly services. Since \system suspects that $event 2$ is caused by either $event 3$ or $event 1$ with the same weight. The scores of $event 3$ and $event 1$ are tied. Then, $event 3$ has a score of $1$ (i.e., $0+1$) and $event 1$ has a score of 2 (i.e., $0+2$), since it is not reachable by \emph{Service-B}). Therefore, $event 3$ is ranked first and logical. 
\item For the remaining joint results with the same access distances, \system continues to untie by using the historical root cause frequency of the event types under the same trigger conditions (e.g., checkout domain alerts). This frequency information is generated from the manually labeled dataset. A more frequently occurred root cause type is ranked higher.% than the less frequent ones.
\end{enumerate}


\subsection{Rule Customization Management}

While \system users create or update the rules,  there could be overlaps, inconsistencies, or even conflicts being introduced such as the example in Figure~\ref{fig:ex2_n2}. \system uses two graphs to manage the rule relationships and avoid conflicts for users. One graph is to represent the link rules between events in the same service (\emph{Same-Graph}) while the other is to represent links between different services (\emph{Diff-Graph}). The nodes in these two graphs are the event types defined in Section~\ref{sec:causality}. There are three statuses between each (directional) pair of event types: (1) no rule, (2) only basic rule, and (3) conditional rule (since it overwrites the basic rule). In \emph{Same-Graph}, \system does not allow self-loop as it does not build links between an event and itself.
% but it is possible that we build links between different services with the same event type.

When rule change happens, existing rules are enumerated to build edges in \emph{Same-Graph} and \emph{Diff-Graph} based on $Target\mbox{-}Events$ and $Target\mbox{-}Service$. Based on the users' operation of 
% \begin{itemize}
%     \item 
    (1) ``remove a rule'',  \system removes the corresponding edge on the graphs;
    % \item 
    (2) ``add/update a rule'',  \system checks whether there are existing edges between the given event types, and then warns the users for possible overwrites. 
    % The users can also combine the conditional rules.   % while users are adding basic rules between event types if there are existing conditional rules between them.
    If there are no conflicts, \system just adds/updates edges between the event types.
    % \item Add conditional rules. We would first alert the possible overwrite. Then if users are about to add new conditional rules on the top of existing conditional rules, we would ask the users to combine these two conditions to add a new one. We then build or change all corresponding edges to status 3.
% \end{itemize} 

After all changes, \system extracts the rules from the graphs by converting each edge to a single rule. These rules are automatically implemented, and then tested against our labeled data set. The \system users need to review the changes with validation reports before the changes go online.

% Note that currently we don't check the consistencies between dynamic rules as we cannot process the dynamic event types, but this could be solved in the future by using nodes with symbolic values to represent such event types. 

\section{Experiments}\label{sec:experiments}
We validate our approach using multiple datasets containing real-life data from the fields of criminal risk assessment, credit, lending, and college admissions. In each of the datasets we select a binary feature and treat it as the protected attribute (e.g., race or gender), which is the feature we require our trained classifier to behave fairly upon. Our proposed method performs well on all of these datasets, succeeding in removing unfairness almost entirely, at a very modest price in terms of accuracy.


\begin{table*}[h]
\centering
\resizebox{\textwidth}{!}{
\def\arraystretch{1.2}

\begin{tabular}{c c c | c | c | c || c | c | c || c | c | c |}

\cline{4-12}
&&&
\multicolumn{9}{ c| }{\textbf{COMPAS Dataset}}
\\ \cline{4-12}
&&&
\multicolumn{3}{ c|| }{\textbf{FPR Considerations}}&
\multicolumn{3}{ c|| }{\textbf{FNR Considerations}}&
\multicolumn{3}{ c| }{\textbf{Both Considerations}}
\\ \cline{4-12}
&&&
 $\mathbf{Acc.}$ &  $\mathbf{D_{FPR}}$ &  $\mathbf{D_{FNR}}$ &  $\mathbf{Acc.}$ &  $\mathbf{D_{FPR}}$ &  $\mathbf{D_{FNR}}$ &  $\mathbf{Acc.}$ &  $\mathbf{D_{FPR}}$ &  $\mathbf{D_{FNR}}$
\\  \cline{4-12}
\vspace*{-0.5ex}
\\ \cline{1-2} \cline{4-12}
\multicolumn{1}{ |c  }{} &
\multicolumn{1}{ c|  }{  \textbf{Our Method (AVD Penalizers)}}  &&
$\mathbf{0.660}$    &  $\mathbf{0.01}$  &  $0.04$ &
$\mathbf{0.653}$    &  $0.02$   &  $\mathbf{0.04}$ &
$\mathbf{0.654}$    &  $\mathbf{0.02}$  &  $\mathbf{0.04}$
\\ \cline{1-2} \cline{4-12}
\multicolumn{1}{ |c  }{} &
\multicolumn{1}{ c|  }{  \textbf{Our Method (SD Penalizers)}}  &&
$\mathbf{0.664}$    &  $\mathbf{0.02}$  &  $0.09$ &
$\mathbf{0.661}$    &  $0.05$   &  $\mathbf{0.03}$ &
$\mathbf{0.661}$    &  $\mathbf{0.02}$  &  $\mathbf{0.03}$
\\ \cline{1-2} \cline{4-12}
\multicolumn{1}{ |c  }{} &
\multicolumn{1}{ c|  }{  Zafar et al.~(\citeyear{disparatemistreatment})}  &&
$0.660$    &   $0.06$    &   $0.14$  &
$0.662$    &   $0.03$    &   $0.10$  &
$0.661$    &   $0.03$    &   $0.11$
\\ \cline{1-2} \cline{4-12}
\multicolumn{1}{ |c  }{} &
\multicolumn{1}{ c|  }{  Zafar et al. Baseline~(\citeyear{disparatemistreatment})}  &&
$0.643$    &   $0.03$    &   $0.11$  &
$0.660$    &   $0.00$    &   $0.07$  &
$0.660$    &   $0.01$    &   $0.09$
\\ \cline{1-2} \cline{4-12}
\multicolumn{1}{ |c  }{} &
\multicolumn{1}{ c|  }{  Hardt et al.~(\citeyear{hardt})}  &&
$0.659$    &  $0.02$    &   $0.08$  &
$0.653$    &  $0.06$   &    $0.01$  &
$0.645$    &  $0.01$   &    $0.01$
\\ \cline{1-2} \cline{4-12}
\multicolumn{1}{ |c  }{} &
\multicolumn{1}{ c|  }{  \textbf{Vanilla Regularized Logistic Regression}}  &&
$\mathbf{0.672}$    &   $\mathbf{0.20}$    &   $\mathbf{0.30}$  &
$\mathbf{0.672}$    &   $\mathbf{0.20}$    &   $\mathbf{0.30}$  &
$\mathbf{0.672}$    &   $\mathbf{0.20}$    &   $\mathbf{0.30}$
\\ \cline{1-2} \cline{4-12}
\end{tabular}
}
\vspace{3mm}
\caption{Performance comparison on the COMPAS dataset. For the approaches in bold -- Accuracy, FPR difference and FNR difference are evaluated on the test set, averaging over five runs and using a 70-30 training/test split. The performance of the remaining three approaches is stated as reported in Zafar et al.~(\citeyear{disparatemistreatment}).} \label{table:comparison_results}
\end{table*}



\begin{figure*}[b]
  \includegraphics[scale=0.6]{compas0-400.png}
  \caption{COMPAS Dataset. Accuracy, FPR difference ($\mathbf{D_{FPR}}$), and FNR difference ($\mathbf{D_{FNR}}$) (all evaluated on the test set) of the learned classifier, as a function of the weight $c=c_1 = c_2 \geq 0$ placed on the fairness penalizer terms. On the left we use the Absolute Value Difference (AVD) penalizer, and the Squared Difference (SD) penalizer on the right, both as presented in Section~\ref{regularization}. ``Relaxed FPR/FNR Diff.'' plots the value of the relevant penalization term.} %In this particular run, parameters chosen for the absolute value relaxation were: $c=80, q_c=60$, and for the squared relaxation: $c=220, q_c=30$.}
  \label{fig:compas}
\end{figure*}


\subsection{Implementation}
\textbf{Our method} 
%We instantiate our method in the following way: Given dataset $Q$, we split it randomly into a training set $S$ (which we will use for learning) and a test set $T$ (which we will only use for reporting performance). 
For the purpose of comparison with  Zafar et al.~(\citeyear{disparatemistreatment}) and Hardt et al.~\cite{hardt} on the COMPAS data, we use a parameter $c$ to induce three possible combinations of weights on the FPR and FNR penalization terms: $c = c_1$ and $c_2 = 0$; $c_1 = 0$ and $c = c_2$; and $c = c_1 = c_2$. For the other three datasets, we consider only $c = c_1 = c_2$.\footnote{The reason for varying the values of $c$ in the training phase is since we shifted to a proxy problem, in which we rely on the distance from the decision boundary rather the actual classifications. 
%Our hope is that there is no need for a worst-case cross validation between all of the combinations of $c_1, c_2, c_3$, and that the training scheme we propose is sufficient. 
It is possible, of course, that even better results are attainable using our scheme with other combinations of $c_1, c_2$, and $q$.} To explore the accuracy/fairness trade-off curve for the relaxed optimization problem~(\ref{eq:2}), we train for different values of $c$, starting at $c=0$ (which is just standard logistic regression), and growing gradually.



Given a dataset $Q$ and fixing a $d_1, d_2 \in \{0, 1\}$ of interest, we use the following training scheme:
\begin{enumerate}
\item Split $Q$ at random into training set $S$ and test set $T$.
\item For each $c$, perform cross-validation on $S$ to select the corresponding best value $q_c$ for the regularization parameter.
\item For each $(c,q_c)$, let $\theta_c = \argmin\limits_{\theta} \text{Proxy}(\theta;S,c,c,q_c)$.
\item Select $\theta^* \in \argmin\limits_{\theta_c} \text{Objective}(\theta_c;S,d_1,d_2)$.
\item Evaluate performance using $\theta^*$ on test set $T$.
\end{enumerate}
We report the average of five such runs, each with a fresh training-test split.




%We instantiate our method by solving the relaxed optimization problem~(\ref{eq:2}), in place of the original, non-convex problem~(\ref{eq:1}).  
%We test our approach with three different combinations of weights on the penalization terms:
%\katrina{What are the $d$, and how are they related to the $c$s?}
%\begin{enumerate}
%\item FPR considerations only: $d_1 = 1, d_2 = 0$.
%\item FNR considerations only: $d_1 = 0, d_2 = 1$.
%\item Both FPR, FNR considerations, assigned similar significance: $d_1 = 1, d_2 = 1$.
%\end{enumerate}
%One could, of course, pick any other combination of the FPR and FNR penalty weights.

%\katrina{I don't understand how the below is distinct from the list above}
%Learning is done by training the parameters of a logistic regressor to solve~\ref{eq:2}, while picking the value of $c_1, %c_2$ as the following:
%\begin{enumerate}
%\item FPR considerations only: $c_1 = c \geq 0$, $c_2 = 0$.
%\item FNR considerations only: $c_1 = 0$, $c_2 = c \geq 0$.
%\item Both FPR, FNR considerations, assigned similar significance: $c_1 = c_2 = c \geq 0$
%\end{enumerate}



% We then cross-validate to pick the best $c_3$ (the weight on the standard $\ell_2$-regularization term) given $c$.\footnote{The reason for varying the values of $c$ in the training phase is since we shifted to a proxy problem, in which we rely on the distance from the decision boundary rather the actual classifications. 
%Our hope is that there is no need for a worst-case cross validation between all of the combinations of $c_1, c_2, c_3$, and that the training scheme we propose is sufficient. 
%It is possible, of course, that even better results are attainable using our scheme with other combinations of $c_1, c_2, c_3$.} For each such combination, we report results as the averages of multiple \katrina{how many?} different runs, each time splitting data randomly into training and test sets.
%\yahav{We need to shorten this description.}

We solve the relaxed convex optimization problem using the CVXPY solver. Due to stability issues with large training sets, we use a train/test split of 30-70 on the larger datasets, rather than 70-30 as on the COMPAS dataset\footnote{The code implementing our method can be found at https://github.com/jjgold012/lab-project-fairness}.

%
%
%We then report the results (as evaluated on the test set) attained by a regressor $\theta \in \mathbb{R}^d$ that minimizes (on the training set $S$) a weighted combination of the $0$-$1$ loss and the differences in FPR and FNR across populations:
%\begin{equation*}
%\begin{aligned}
%&\underset{\theta}{\text{argmin}}
%& & L_{S}^{0\text{-}1}(\theta) \\
%&&& + d_1|FPR_{A=0}(\theta;S)-FPR_{A=1}(\theta;S)| \\
%&&& + d_2|FNR_{A=0}(\theta;S)-FNR_{A=1}(\theta;S)|
%\end{aligned}
%\end{equation*}
%
%\katrina{What is $d_1$ vs. $c_1$ etc.?}



%For classification, we decided use a standard cut-off threshold of $c=0.5$. There are of course, further possible interactions between the FPR, FNR considerations, and picking a certain cut-off level. These are not straightforward, since  these interactions are data-specific. 



%allows for flexibility in picking the values of $c_1, c_2$, which reflect the significance we wish to place on the objectives of achieving accuracy, equal FPR, and equal FNR. As for $c_3$, we will want to find the value of it that achieves the best results, for any combined objective of accuracy and fairness defined by a specific selection of $c_1,c_2$. Therefore, given a specific selection of $c_1, c_2$, we apply cross-validation to select the value of $c_3$. 




We briefly describe the other algorithmic approaches to which we compare:\\
\textbf{Zafar et al.}~(\citeyear{disparatemistreatment}) performs optimization by considering a proxy for the bias: the covariance between the samples' sensitive attributes and the signed distance between the feature vectors of misclassified users and the classifier decision boundary.\\
\textbf{Zafar et al. Baseline}~(\citeyear{disparatemistreatment}) tries to enforce equal FP/FN rates on the different groups by introducing different penalties for misclassified data points with different sensitive attribute values during the training phase.\\
\textbf{Hardt et al.}~(\citeyear{hardt}) performs post-processing on a standard trained (unfair) logistic regressor, picking different decision thresholds for different groups, and possibly adding randomization.


\subsection{Experimental Results}

In what follows, we use the following notation, given a trained classifier $\hat{Y}$:
\begin{align*}
\mathbf{D_{FPR}}&=\left|FPR_{A=0}(\hat{Y})-FPR_{A=1}(\hat{Y})\right| \\ 
\mathbf{D_{FNR}}&=\left|FNR_{A=0}(\hat{Y})-FNR_{A=1}(\hat{Y})\right|
\end{align*}
The values $FPR_{A=0}(\hat{Y})$, $FPR_{A=1}(\hat{Y})$, $FNR_{A=0}(\hat{Y})$, $FNR_{A=1}(\hat{Y})$ are reported as evaluated on the test set.

\paragraph{The COMPAS Dataset\footnote{https://github.com/propublica/compas-analysis}} The Correctional Offender Management Profiling for Alternative Sanctions (COMPAS) records from Broward County, Florida 2013-2014, made available online by ProPublica, are perhaps the best-studied data in the context of fairness.  The goal in this scenario is to successfully predict recidivism within two years, based on features such as age, gender, race, number of prior offenses, and charge degree. The dataset contains 5,278 samples. The protected attribute in this scenario is race, where $A$ indicates black or white. We filtered the dataset using the same features as Zafar et al.~(\citeyear{disparatemistreatment}), to allow for comparison.

%\begin{table}[h]
%\centering
%\begin{tabularx}{\columnwidth}{c|c|c|c}
%\hline
%  &  Recid. ($y = 1$)        & No Recid.  ($y = 0$)       & Total \\ \hline
%Black &  $ 1661   $ & $ 1514 $ &  $ 3175 $ \\ \hline
%White &  $ 822   $  & $1281  $ &  $ 2103 $ \\ \hline
%Total &  $ 2483  $  & $2795 $ &  $ 5278 $ \\\hline
%\end{tabularx}
%\caption{Statistics of the ProPublica COMPAS data.} \label{table:compas-stats}
%\label{tab:stats}
%\end{table}
%\vspace{-1em}

%\begin{table}[h]
%\centering
%\begin{tabularx}{\columnwidth}{c|c}
%\hline
%Feature  &  Description \\ \hline
%Age Category &  $<25$, between $25$ and $45$, $>45$ \\
%Gender &  Male or Female \\
%Race &  White or Black \\
%Priors Count &  0--37 \\
%Charge Degree &  Misconduct or Felony \\
%\hline
%2-year-recid. & Whether or not the  \\
%(target feature)  & defendant recidivated within two years
%\end{tabularx}
%\caption{Description of features used from ProPublica COMPAS data.} \label{table:compas-features}
%\label{tab:features}
%\end{table}




\begin{table*}[t]
\centering
\caption{A description of the datasets used, along with parameters of the training procedure used for each.}
\label{table:datasets_description}
\begin{adjustbox}{max width=\textwidth}
\begin{tabular}{|l|l|l|l|l|l|l|l|}
\hline
\textbf{Dataset} & \textbf{No. Samples} & \textbf{No. Features} & \textbf{Train/Test Split} & \textbf{No. Repetitions} & \textbf{No. Folds in CV} & \textbf{Protected Feature} & \textbf{Target Variable} \\ \hline
COMPAS           & 5,278                     & 5                          & 70-30                     & 5                        & 5                                 & Race                       & 2-Year-Recidivism        \\ \hline
Adult            & 30,162                    & 10                         & 30-70                     & 5                        & 5                                 & Gender                     & Income Over/Under 50K    \\ \hline
Default          & 30,000                    & 23                         & 30-70                     & 5                        & 3                                 & Gender                     & Defaulting On Payments   \\ \hline
Admissions       & 20,839                    & 17                         & 30-70                     & 5                        & 3                                 & Race                       & Passing Bar Exam         \\ \hline
\end{tabular}
\end{adjustbox}
\end{table*}


\begin{table*}[t]
\centering
\resizebox{\textwidth}{!}{
\def\arraystretch{1.2}

\begin{tabular}{c c c | c | c | c || c | c | c || c | c | c |}

\cline{4-12}
&&&
\multicolumn{3}{ c|| }{\textbf{Adult Dataset}}&
\multicolumn{3}{ c|| }{\textbf{Default Dataset}}&
\multicolumn{3}{ c| }{\textbf{Admissions Dataset}}
\\ \cline{4-12}
%&&&
%\multicolumn{3}{ c|| }{\textbf{Both Considerations}}&
%\multicolumn{3}{ c|| }{\textbf{Both Considerations}}&
%\multicolumn{3}{ c| }{\textbf{Both Considerations}}
%\\ \cline{4-12}
&&&
 $\mathbf{Acc.}$ &  $\mathbf{D_{FPR}}$ &  $\mathbf{D_{FNR}}$ &  $\mathbf{Acc.}$ &  $\mathbf{D_{FPR}}$ &  $\mathbf{D_{FNR}}$ &  $\mathbf{Acc.}$ &  $\mathbf{D_{FPR}}$ &  $\mathbf{D_{FNR}}$
\\  \cline{4-12}
\vspace*{-0.5ex}
\\ \cline{1-2} \cline{4-12}
\multicolumn{1}{ |c  }{} &
\multicolumn{1}{ c|  }{  \textbf{Our Method (AVD Penalizers)}}  &&
$\mathbf{0.776}$    &  $\mathbf{0.00}$  &  $\mathbf{0.04}$ &
$\mathbf{0.807}$    &  $\mathbf{0.00}$   &  $\mathbf{0.01}$ &
$\mathbf{0.950}$    &  $\mathbf{0.01}$  &  $\mathbf{0.00}$
\\ \cline{1-2} \cline{4-12}
\multicolumn{1}{ |c  }{} &
\multicolumn{1}{ c|  }{  \textbf{Our Method (SD Penalizers)}}  &&
$\mathbf{0.783}$    &  $\mathbf{0.00}$  &  $\mathbf{0.09}$ &
$\mathbf{0.806}$    &  $\mathbf{0.01}$   &  $\mathbf{0.02}$ &
$\mathbf{0.950}$    &  $\mathbf{0.00}$  &  $\mathbf{0.00}$
\\ \cline{1-2} \cline{4-12}
\multicolumn{1}{ |c  }{} &
\multicolumn{1}{ c|  }{  \textbf{Vanilla Regularized Logistic Regression}}  &&
$\mathbf{0.800}$    &   $\mathbf{0.08}$    &   $\mathbf{0.39}$  &
$\mathbf{0.807}$    &   $\mathbf{0.01}$    &   $\mathbf{0.05}$  &
$\mathbf{0.951}$    &   $\mathbf{0.16}$    &   $\mathbf{0.02}$
\\ \cline{1-2} \cline{4-12}
\end{tabular}
}
\vspace{3mm}
\caption{Performance on the Adult, Loan Default, and Admissions datasets, penalizing for both FPR and FNR difference. Accuracy, FPR difference and FNR difference are evaluated on the test set, averaging over five runs and using a 30-70 training/test split.} \label{table:comparison_results_rest}
\end{table*}


In Table~\ref{table:comparison_results}, we compare the performance of our approach with that of three other techniques from the literature. Each method was trained based on logistic regression.  As a basis for comparison, we also present the performance of vanilla logistic regression, absent fairness considerations, with the regularization parameter selected via cross-validation.\footnote{Zafar et al.~(\citeyear{disparatemistreatment}) do not incorporate regularization in any of the approaches they report.}
%Results are reported as the averages of 5 different runs \katrina{Is that still correct?}, each time splitting data evenly and randomly into training and test sets. 
Results for Zafar et al., Zafar et al. baseline, and Hardt et al. appear here as reported in Zafar et al.~(\citeyear{disparatemistreatment}).\footnote{Our method selects the classifier based on the training set only and reports its performance over the test set. Results for the three other approaches, reported by Zafar et al.~(\citeyear{disparatemistreatment}), are based on tuning parameters after seeing the trade-off curve over the test set, and reporting according to the best selection of these parameters.}
%\katrina{Perhaps here is the right place for a footnote about the discrepancy with the Zafar baseline}

We find that the vanilla logistic regressor (absent fairness considerations) results in significant unfairness, as $\mathbf{D_{FPR}}=0.20$, and $\mathbf{D_{FNR}}=0.30$. The overall accuracy of this classifier measured on the test set was $0.672$.\footnote{Zafar et al.~(\citeyear{disparatemistreatment}) report a slightly different baseline of: Accuracy = 0.668, $\mathbf{D_{FPR}}=0.18$, $\mathbf{D_{FNR}}=0.30$.} Our SD penalization approach empirically achieves approximately the same accuracy as the Zafar et al.~(\citeyear{disparatemistreatment}) approach, with significantly better fairness. It is difficult to compare fairness-accuracy tradeoffs with the Hardt et al.~(\citeyear{hardt}) approach, since their accuracy is significantly lower than ours. A more direct comparison is possible by noting that our learned classifier can be post-processed to improve its fairness at a direct cost to accuracy. Hence, we can achieve accuracy of $0.659$ with $\mathbf{D_{FPR}} = \mathbf{D_{FNR}} = 0.01$, which compares very favorably with the Hardt et al. accuracy rate of 0.645 given the same FPR and FNR rates.\footnote{For completeness, we note that using a 50-50 training-test split (again not using the test set for parameter selection), our method (SD, both considerations) produces a classifier that provides: Accuracy = 0.659, $\mathbf{D_{FPR}} = 0.01, \mathbf{D_{FNR}} = 0.05$. This classifier can be post-processed to achieve rates of: Accuracy = 0.655, $\mathbf{D_{FPR}} = \mathbf{D_{FNR}} = 0.01$.}

Figure \ref{fig:compas} illustrates the accuracy/fairness trade-offs achievable using our scheme. Increasing the weight $c$ on the proxy fairness penalizers results in reducing their magnitude. The figure also illustrates how our relaxed penalizers succeed in tracking the real FPR and FNR differences. 
%
%
%\katrina{Must rewrite the following paragraph}
%We observe that our method succeeds in eliminating unfairness almost completely on the COMPAS dataset, while retaining most of the accuracy, when compared to the vanilla logistic regression. We achieve very low difference rates when penalizing for achieving each of the FPR and FNR criteria individually, and also for both. We achieve preferable results comparing to Zafar et al. and Zafar et al. baseline in all 3 scenarios, and also comparing to Hardt et al. in the settings of false positive/false negative considerations only. In the setting of both considerations - The Hardt et al. method removes a larger portion of the unfairness, however it results in major accuracy loss as it achieves accuracy rate of 0.645 in comparison to our method which results in accuracy of 0.665, retaining most of the original accuracy rate while removing most of the unfairness.




%The Hardt et al.~\cite{hardt} approach as reported removes a smaller portion of the bias in the different scenarios, however for FP/FN constraints alone, it provides higher accuracy rates. The Zafar et al.~(\citeyear{disparatemistreatment}) approach as reported retains significant bias (in most cases), but in some cases  achieves slightly superior accuracy rates to the methods above. 

%These performance comparisons are incomplete in the sense that each of the compared techniques has the potential to trade off between accuracy and fairness, using some degree of parameter tuning; what we report here is only one point on the achievable trade-off frontier for each algorithm. The ``correct'' trade-off, and, in particular, the best manner in which to weigh unfairness in the FPR against unfairness in the FNR, are matters of opinion. We have chosen to report our method's performance under parameters designed to very aggressively mitigate unfairness, at some cost to the accuracy.

%It would certainly be desirable to evaluate these and other approaches to fair learning on other datasets and on different tasks, particularly on larger datasets, which might afford both greater accuracy and better bias-reduction. The present empirical evaluations, however, suggest that our regularization-based approach provides a new tool worthy of consideration---we succeed in almost entirely eliminating bias on the hold-out set, at a modest price in terms of accuracy.

%Due to the fact that our true objective includes the original non-convex penalization terms, our approach does not carry any formal guarantees. However, the ease of implementation, generality, and empirical results are encouraging. Figure~\ref{fig:test1} illustrates the rate of convergence to a fair, accurate classifier on this dataset.
%In terms of computation costs, given that at each iteration we must calculate the gradient according to the FPR and FNR regularizers, we are required to predict the labels for the entire training set at each step. 
%However, this does not pose a computational burden, as it is already required by the (classic) gradient descent algorithm in our logistic regressor fitting scheme. Furthermore, when given a sufficiently large dataset (one or two orders of magnitude larger than the one currently available for the COMPAS scores data), this could be relaxed to sampling only a mini-batch of samples from the training data set at each iteration (much as is done in stochastic gradient descent).






\subsection{Additional Datasets}


Table~\ref{table:datasets_description} provides summary statistics on each of the datasets on which we tested our approach. We also briefly describe the datasets below. 


{\bf The Adult Dataset}\footnote{http://archive.ics.uci.edu/ml/datasets/Adult} is based on 1994 US Census data. The task we consider is to predict whether the income of each individual is over or under 50K dollars per year, based on features such as occupation, marital status, and education. The protected attribute selected in this task is gender. 

{\bf The Loan Default Dataset}\footnote{{\scriptsize https://archive.ics.uci.edu/ml/datasets/default+of+credit+card+clients}}
contains data regrading Taiwanese credit card users. The task we consider is to predict whether an individual will default on payments, based on features such as history of past payments, age, and the amount of given credit. The protected attribute is gender.

{\bf The Admissions Dataset}\footnote{http://www2.law.ucla.edu/sander/Systemic/Data.htm}
contains records of law school students who went on to take the bar exam. The task we consider is to predict whether a student will pass the exam based on features such as LSAT score, undergraduate GPA, and family income. The protected attribute is set to race.

Table~\ref{table:comparison_results_rest} describes the performance of our approach on these datasets, and Figures~\ref{fig:adult},~\ref{fig:default}, and~\ref{fig:lawschool} illustrate the fairness-accuracy trade-offs we achieve in each context. Overall, we see that unfairness is nearly eliminated while accuracy remains quite high. The dataset on which accuracy suffers most under our approach is the Adult dataset, which is also the dataset on which the vanilla regression is the most unfair.


\begin{figure*}[]
  \includegraphics[scale=0.6]{adult0-800.png}
  \caption{Adult Dataset. Fairness-Accuracy tradeoffs, as in Figure~\ref{fig:compas}.}
  \label{fig:adult}  
\end{figure*}



\begin{figure*}[]
  \includegraphics[scale=0.6]{default0-50.png}
  \caption{Loan Default Dataset. Fairness-Accuracy tradeoffs, as in Figure~\ref{fig:compas}.}
  \label{fig:default}
\end{figure*}



\begin{figure*}[]
  \includegraphics[scale=0.6]{admissions0-400.png}
  \caption{Admissions Dataset. Fairness-Accuracy tradeoffs, as in Figure~\ref{fig:compas}.}
  \label{fig:lawschool}
\end{figure*}



In this paper, 2D and 3D CNN models were used to generate pelvic sCTs from T1-weighted MR images. Our sCT generation methods were fully automated, requiring no deformable registration or manual segmentation of bone tissues. As shown in Figure~\ref{fig3}, the 2D and 3D CNN models generated high quality sCTs. MAE curves shown in Figure~\ref{fig4} indicated that both models could precisely estimate soft-tissue HU values but had difficulty in reproducing air and high-density bone tissues. 

The MAEs within the body contour across all patients were 40.5 $\pm$ 5.4 HU and 37.6 $\pm$ 5.1 HU for the 2D and 3D models, respectively. The time required for generating a pelvic sCT using our CNN models was about 5.5 s. Our MAE results are comparable to previous studies. Kim $et \ al.$\cite{RN41} presented a voxel-based weighted summation method that produced an MAE of 74.3 $\pm$ 3.9 HU. However, manual contouring of bone tissues required for this method can be tedious and time-consuming. An MAE of 40.5 $\pm$ 8.2 HU was achieved by Dowling $et \ al.$\cite{RN11} using an average MRI-CT atlas from 38 patients. Andreasen $et \ al.$\cite{RN42} reported an MAE of 54 $\pm$ 8 HU using an atlas-based method with pattern recognition, and its prediction time was about 20.8 min. Another random forest model proposed by Andreasen $et \ al.$\cite{RN43} generated sCTs with an MAE of 58 $pm$ 9 HU. A hybrid method suggested by Siversson $et \ al.$ \cite{RN45} obtained an MAE of 36.5 $\pm$ 4.1 HU when ignoring errors introduced by gas cavities. This hybrid method was implemented in the cloud-based commercial software MriPlanner (Spectronic Medical AB, Helsingborg, Sweden), which required 50 to 80 min to generate a sCT.\cite{RN45} The patch-based 3D context-aware generative adversarial network presented by Nie $et \ al.$\cite{RN26} achieved an MAE of 39.0 $\pm$ 4.6 HU. 

Our CNN models reproduced low-density bone as shown in Figure ~\ref{fig4}. The bone-region DSCs were 0.81 $\pm$ 0.04 and 0.82 $\pm$ 0.04 from the 2D and 3D models, respectively. These results are comparable to reported DSC results of 0.79 $\pm$ 0.12\cite{RN10} and 0.91$\pm$0.03{\cite{RN11}}, where the authors compared bone contours manually drawn on the sCT and CT.

It was feasible to train the proposed 3D model with 16 image volumes from scratch. Results of the Wilcoxon signed-rank tests shown in Table~\ref{tab1} demonstrated a statistically significant improvement in overall MAE, bone DSC, and bone precision of the 3D model compared to the 2D model. However, as shown in Figure~\ref{fig4}, the 2D model seemed to perform better in estimating the high-density bone HU values. It should be noted that smaller overall MAEs do not guarantee improved sCT dose calculation and patient positioning performance. While the models performed well, we will continue to acquire more patient data to potentially improve model accuracy and further test model differences.

As this was a retrospective study, the MR image voxel sizes were not matched, resulting in different voxel intensities between images. This may have affected the sCT generation accuracy although we applied intensity normalization. A potential study could examine how voxel size variations affects sCT estimation. 

The proposed 3D model can be implemented on a 12 GB GPU to process volumetric images with dimensions of 256 $\times$ 256 $\times$ 30. More GPU memory would be required to process higher resolution 3D images. Considering the limited access to multi-GPU systems, a 3D architecture with fewer convolutional layers could be considered to deal with higher resolutions. However, the performance could be affected by the reduced parameters and smaller receptive fields of the less complex model. Another approach would be to extract 30-slice sub-volumes from CT and MR images for training the 3D model. The sCT could then be generated by averaging 30-slice sCT sub-volumes produced by the model. 

A number of techniques could be investigated for improving model performance.  Nie $et \ al.$\cite{RN26} showed that introducing an additional adversarial discriminator improved overall sCT quality. The same approach could be adapted in our proposed 2D and 3D CNN models.  Non-rigid deformation\cite{RN44} could also be applied to both CT and MR images in the process of the on-the-fly data augmentation to produce more training pairs. Multiple MR images acquired with different sequences could be fed into models to provide more information for distinguishing different tissues. Multi-GPU systems with more memory would enable the exploration of larger batch sizes for training CNN models, which could reduce variances in gradient estimation and accelerate the training. 





%%%%%%%%%%%%%%%%%%%%%%%%%%%%%%%%%%%%%%%%%%%%%%%%%%%%%%%%%%%%%%%%%%%%%%%%%%%%%%%%



%%%%%%%%%%%%%%%%%%%%%%%%%%%%%%%%%%%%%%%%%%%%%%%%%%%%%%%%%%%%%%%%%%%%%%%%%%%%%%%%



%%%%%%%%%%%%%%%%%%%%%%%%%%%%%%%%%%%%%%%%%%%%%%%%%%%%%%%%%%%%%%%%%%%%%%%%%%%%%%%%
% \section*{APPENDIX}

% Appendixes should appear before the acknowledgment.

% \section*{Acknowledgements}
% We acknowledge funding from NSF Award 1941722, Fanuc Fellowship, and Qualcomm Innovation Fellowship for their support, and would like to thank Prof. Jeannette Bohg for insightful early discussion. 


%%%%%%%%%%%%%%%%%%%%%%%%%%%%%%%%%%%%%%%%%%%%%%%%%%%%%%%%%%%%%%%%%%%%%%%%%%%%%%%%

\clearpage
\section*{Acknowledgement}
We acknowledge funding from the NSF Award \#1941722 and \#2006388, Future of Life Institute, and DARPA Hicon-Learn project.

\balance
\bibliographystyle{IEEEtran}
\bibliography{main}


\end{document}
