\newpage

\section{Equation Notes}
\begin{enumerate}
\item The overall reasoning we are doing.
    
\begin{eqnarray}
\begin{aligned}
p(\theta|\xi_R^0,\xi_H^0, \xi_H^1 \ldots, \xi_H^{t}) &\propto p(\xi_R^0,\xi_H^0, \xi_H^1 \ldots, \xi_H^{t}| \theta) p(\theta) \\
&= \int_{\Tilde{\xi}} p(\xi_R^0,\xi_H^0, \xi_H^1 \ldots, \xi_H^{t}|\Tilde{\xi}, \theta)p(\Tilde{\xi} | \theta) p(\theta) ~d \Tilde{\xi}
\end{aligned}
\end{eqnarray}

Interpretations:\\
Probability for intended trajectory,
\begin{equation}
P(\Tilde{\xi} | \theta) \propto e^{\theta^\top\Phi(\Tilde{\xi})}
\end{equation}

Accumulated Evidence for the intended trajectory.
\begin{equation}
    p(\xi_R^0,\xi_H^0, \xi_H^1 \ldots, \xi_H^{t}|\Tilde{\xi}, \theta)
\end{equation}

\item How we model 'sequential accumulated evidence'
\begin{equation}
    P(\xi_H^0, \xi_H^0, \ldots, \xi_H^{K} ~|~ \Tilde{\xi}, \theta) = \frac{\exp\Big(E\big(\xi_R^0,\xi_H^0, \ldots, \xi_H^{k}, \Tilde{\xi}, \theta\big)\Big)}{\int_{ \xi_H^0, \ldots, \xi_H^{K}}\exp\Big(E\big(\xi_R^0,\xi_H^0, \ldots, \xi_H^{k}, \Tilde{\xi}, \theta\big)\Big)d\xi_H^0 \ldots d\xi_H^{K}}
\end{equation}

% \begin{eqnarray}
% \begin{aligned}
% D\big(\xi_R^0,\xi_H^0, \ldots, \xi_H^{k}, \Tilde{\xi}, \theta\big) &=  \sum_{t = 0}^{K} R(\xi_H^t, \theta) - \lambda_1 \Bigg(\sum_{i = 0}^K\|u_H^i|^2\Bigg) - \lambda_2 \Big(R(\Tilde{\xi}, \theta) - R(\xi_H^k, \theta)\Big) 
% \end{aligned}
% \label{eq:correction_cost}
% \end{eqnarray}

\begin{eqnarray}
\begin{aligned}
E\big(\xi_R^0,\xi_H^0, \ldots, \xi_H^{k}, \Tilde{\xi}, \theta\big) &=   \lambda \sum_{t = 0}^{K} \alpha ^{K-t}R(\xi_H^t, \theta) - \gamma \Bigg(\sum_{t = 0}^K\|u_H^t|^2\Bigg) - \Big(R(\Tilde{\xi}, \theta) - R(\xi_H^k, \theta)\Big) 
\end{aligned}
\label{eq:correction_cost}
\end{eqnarray}
K is the number of corrections.\\
The three terms are: accumulated decaying trajectory sequence rewards $\sum_{t = 0}^{K} \alpha ^{K-t}R(\xi_H^t, \theta)$, human interventions $\Bigg(\sum_{t = 0}^K\|u_H^t|^2\Bigg)$, penalty for gap to intended trajectory $- \Big(R(\Tilde{\xi}, \theta) - R(\xi_H^k, \theta)\Big)$. $\lambda$ and $\gamma$ are weights for balancing between the terms.\\

\item If we go through some mathematical manipulations, we would find the terms $- R(\Tilde{\xi}, \theta)$ get canceled out in both numerator/denominator. We denote $D = E + R(\Tilde{\xi}, \theta)$

\begin{eqnarray}
\begin{aligned}
D\big(\xi_R^0,\xi_H^0, \ldots, \xi_H^{k},  \theta\big) &=   R(\xi_H^k, \theta) + \lambda \sum_{t = 0}^{K} \alpha ^{K-t}R(\xi_H^t, \theta) - \gamma \Bigg(\sum_{t = 0}^K\|u_H^t|^2\Bigg) 
\end{aligned}
\label{eq:correction_cost}
\end{eqnarray}
Finally, we have
\begin{equation}
    P(\xi_H^0, \xi_H^0, \ldots, \xi_H^{K} ~|~ , \theta) = \frac{\exp\Big(D\big(\xi_R^0,\xi_H^0, \ldots, \xi_H^{k}, \Tilde{\xi}, \theta\big)\Big)}{\int_{ \xi_H^0, \ldots, \xi_H^{K}}\exp\Big(D\big(\xi_R^0,\xi_H^0, \ldots, \xi_H^{k}, \Tilde{\xi}, \theta\big)\Big)d\xi_H^0 \ldots d\xi_H^{K}}
\end{equation}

\item So far, no approximation. Here is where. approximation comes in. We use Laplace Approximation here.
\begin{equation}
     P(\xi_H^0, \xi_H^0, \ldots, \xi_H^{K} ~|~ \Tilde{\xi}, \theta) \sim \frac{\exp\Big(D\big(\xi_R^0,\xi_H^0, \ldots, \xi_H^{k}, \Tilde{\xi}, \theta\big)\Big)}{\exp\Big(\max_{\xi_H^0 \ldots \xi_H^k} D\big(\xi_R^0,\xi_H^0, \ldots, \xi_H^{k}, \Tilde{\xi}, \theta\big)\Big)}
\label{eq:seq_prob}
\end{equation}
\end{enumerate}

\clearpage

