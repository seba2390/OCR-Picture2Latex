\documentclass[10pt,twocolumn,letterpaper]{article}
\usepackage{makecell}
%\usepackage{cvpr}
\usepackage{times}
\usepackage{epsfig}
\usepackage{graphicx}
\usepackage{graphics}
\usepackage{amsmath}
\usepackage{amssymb}
\usepackage{booktabs, multirow}
\usepackage{amsmath,amssymb,amsthm}
\usepackage{algorithm}
\usepackage{algorithmic}
\usepackage{bm}
\usepackage{enumerate}
\usepackage{standalone}
\newtheorem{thm}{Theorem}
\newtheorem{lemma}{Lemma}
\newtheorem{cor}{Corollary}
\newtheorem{prop}{Proposition}
\theoremstyle{definition}
\newtheorem{definition}{Definition}
\newtheorem{remark}{Remark}
\newtheorem{example}{Example}
\newtheorem{assume}{Assumption}
\newtheorem{obs}{Observation}

% Include other packages here, before hyperref.


\newcommand*{\affaddr}[1]{#1} % No op here. Customize it for different styles.
\newcommand*{\affmark}[1][*]{\textsuperscript{#1}}
\newcommand*{\email}[1]{\texttt{#1}}

% If you comment hyperref and then uncomment it, you should delete
% egpaper.aux before re-running latex.  (Or just hit 'q' on the first latex
% run, let it finish, and you should be clear).
\usepackage[breaklinks=true,bookmarks=false]{hyperref}

%\cvprfinalcopy % *** Uncomment this line for the final submission

%\def\cvprPaperID{8323} % *** Enter the CVPR Paper ID here
\def\httilde{\mbox{\tt\raisebox{-.5ex}{\symbol{126}}}}

% Pages are numbered in submission mode, and unnumbered in camera-ready
%\ifcvprfinal\pagestyle{empty}\fi
%\setcounter{page}{4321}
\begin{document}

%%%%%%%%% TITLE
\title{Synthetic Learning: Learn From Distributed Asynchronized Discriminator GAN Without Sharing Medical Image Data}

\author{%
Qi Chang\affmark[1]\thanks{equal contribution}, Hui Qu\affmark[1]\footnotemark[1], Yikai Zhang\affmark[1]\footnotemark[1], Mert Sabuncu\affmark[2], \\Chao Chen\affmark[3], Tong Zhang\affmark[4] and Dimitris Metaxas\affmark[2]\\
\affaddr{\affmark[1]Rutgers University}
\affaddr{\affmark[2]Cornell University}\\
\affaddr{\affmark[3]Stony Brook University}
\affaddr{\affmark[4]Hong Kong University of Science and Technology}\\
\email{\{qc58,hq43,yz422,dnm\}@cs.rutgers.edu} , \email{msabuncu@cornell.edu},\\ \email{chao.chen.cchen@gmail.com}, \email{tongzhang@tongzhang-ml.org}%
}
%\author{%
%Qi Chang\affmark^1, Hui Qu\affmark^1, Yikai Zhang\affmark^1, Mert Sabuncu\affmark^2,Chao Chen\affmark^3,Tong Zhang\affmark^4 and Dimitris Metaxas\affmark^1\\
%\affaddr{\affmark^1 Rutgers University, New Brunswick NJ}\\
%\affaddr{\affmark[2]Department of Mechanical Engineering}\\
%\email{\{A,B,C,D,E\}@university.edu}\\
%\affaddr{\LaTeX\ University}%
%}

%\author{First Author\\
%Institution1\\
%Institution1 address\\
%{\tt\small firstauthor@i1.org}
%% For a paper whose authors are all at the same institution,
%% omit the following lines up until the closing ``}''.
%% Additional authors and addresses can be added with ``\and'',
%% just like the second author.
%% To save space, use either the email address or home page, not both
%\and
%Second Author\\
%Institution2\\
%First line of institution2 address\\
%{\tt\small secondauthor@i2.org}
%}

\maketitle
%\thispagestyle{empty}

%%%%%%%%% ABSTRACT
\begin{abstract}
In this paper, we propose a data privacy-preserving and communication efficient distributed GAN learning framework named Distributed Asynchronized Discriminator GAN (AsynDGAN). 
Our proposed framework aims to train a central generator learns from distributed discriminator, and use the generated synthetic image solely to train the segmentation model.
We validate the proposed framework on the application of \emph{health entities learning problem} which is known to be privacy sensitive. 
Our experiments show that our approach: 1) could learn the real image's distribution from multiple datasets without sharing the patient's raw data. 2) is more efficient and requires lower bandwidth than other distributed deep learning methods. 3) achieves higher performance compared to the model trained by one real dataset, and almost the same performance compared to the model trained by all real datasets. 4) has provable guarantees  that the generator could learn the distributed distribution in an \emph{all important} fashion thus is unbiased.We release our AsynDGAN source code at: https://github.com/tommy-qichang/AsynDGAN
	
	
%Can we train a deep learning model from different health entities to achieve higher performance without compromising the patient data privacy?
%We propose a Distributed Asynchronized Discriminator GAN(AsynDGAN) which learns from the real images in different health entities' dataset, and generates the synthetic images to train a deep learning model. 
%The synthetic image from Dad-GAN share the similar distribution with the real image, and let the task specific model achieve higher performance than just learn the real images from only one dataset and almost the same performance to the model learns the real images from all datasets together but without compromising the patient data privacy. 
%Our experiments show that our approach: 1) could learn the real image's distribution without sharing patient's raw data. 2) provide an architecture which could study through multiple private dataset 3) could provide future scalability for the task specific model. 4) more efficient and requires lower bandwidth than other distributed deep learning method.
%We will release our AsynDGAN source code in Github. 
% We validate the proposed algorithm on two different segmentation tasks including split 10-fold BraTS HGG(same distribution )segmentation and BraTs HGG and LGG segmentation(different distribution). The result demonstrate the synthesis image could be used for other machine learning tasks and achieve reasonable performance without loss of privacy. 
% 
% the effectiveness of the distributed architecture for improving without loss of privacy
%
%
%We demonstrate the synthetic images have the similar distribution with the real image, and could achieve higher performance than any of the dataset, and 
%
%Can we create a large medical dataset from different health entities without compromising the data privacy? Instead of directly train a task specified model like what federated learning or split learning did, we trained a Distributed Asynchronized Discriminator GAN(Dad-GAN) from the real image and then store the generator as a medical dataset for the future use. We proved that Dad-GAN could (provide a better dataset) than just learn the real images from one entities. Specifically, we introduce the architecture of Dad-GAN and also the perception loss as one of the generator loss function. Our result show that Dad-GAN could generate realistic image and by using such synthetic image, the segmentation result could achieve higher performance than the real image in any of the subset.
%
% The AsynDGAN could be treated as a dataset and any model 
%
%
%We proposed a novel method to build a Distributed Asynchronized Discriminator GAN(Dad-GAN) as a medical repository which learned medical images from different sites. Our experiments shows the segmentation tasks learnt from our Dad-GAN could be better than 
%
%One of the major challenge of the medical intelligent is the discrepancy between maintaining the confidentiality of the subject's privacy data and the acquirement of the  outstandingly large quantity of data that the deep neural network are learning from. New methods like federated learning or split learning focus on distributed architecture but ignore the  

%\footnote{*these authors contribute equally}
\end{abstract}

%%%%%%%%% BODY TEXT

\vspace{-1em}
\section{Introduction}
\label{sec:intro}
\section{Introduction}
\label{sec:intro}

Computer graphics has long been concerned with generating photorealistic images at high resolution that allow for direct control over semantic attributes. Until recently, the primary paradigm was to create carefully designed 3D models which are then rendered using realistic camera and illumination models. A parallel line of research approaches the problem from a data-centric perspective. 
In particular, probabilistic generative models ~\cite{Goodfellow2014NEURIPS,Oord2017NEURIPS,Song2021ICLR} have shifted the paradigm from designing assets to designing training procedures and datasets. Style-based GANs (StyleGANs) are a specific instance of these models, and they exhibit many desirable properties. They achieve high image fidelity~\cite{Karras2019CVPR, Karras2020CVPR}, fine-grained semantic control~\cite{Haerkoenen2020NEURIPS, WU2021CVPRa,Ling2021ARXIV}, and recently alias-free generation enabling realistic animation~\cite{Karras2021NEURIPS}. Moreover, they reach impressive photorealism on carefully curated datasets, especially of human faces. However, when trained on large and unstructured datasets like ImageNet~\cite{Deng2009CVPR}, StyleGANs do not achieve satisfactory results yet. One other problem plaguing data-centric methods, in general, is that they become prohibitively more expensive when scaling to higher resolutions as bigger models are required.

Initially, StyleGAN~\cite{Karras2019CVPR} was proposed to explicitly disentangle factors of variations, allowing for better control and interpolation quality. 
However, its architecture is more restrictive than a standard generator network~\cite{Radford2016ICLR, Karras2018ICLR} which seems to come at a price when training on complex and diverse datasets such as ImageNet. Previous attempts at scaling StyleGAN and StyleGAN2 to ImageNet led to sub-par results~\cite{Gwern2020MISC, Grigoryev2022ICLR},  giving reason to believe it might be fundamentally limited for highly diverse datasets~\cite{Gwern2020MISC}.

BigGAN~\cite{Brock2019ICLR} is the state-of-the-art GAN model for image synthesis on ImageNet. The main factors for BigGANs success are larger batch and model sizes.
However, BigGAN has not reached a similar standing as StyleGAN as its performance varies significantly between training runs~\cite{Karras2020NeurIPS} and as it does not employ an intermediate latent space which is essential for GAN-based image editing~\cite{Abdal2021TOG, Patashnik2021ICCV, Collins2020CVPR, WU2021CVPRa}. Recently, BigGAN has been superseded in performance by diffusion models~\cite{Dhariwal2021NEURIPS}. Diffusion models achieve more diverse image synthesis than GANs but are significantly slower during inference and prior work on GAN-based editing is not directly applicable. Following these arguments, successfully training StyleGAN on ImageNet has several advantages over existing methods.

The previously failed attempts at scaling StyleGAN raise the question of whether architectural constraints fundamentally limit style-based generators or if the missing piece is the right training strategy.
Recent work by \cite{Sauer2021NEURIPS} introduced \textit{Projected GANs} which project generated and real samples into a fixed, pretrained feature space. Rephrasing the GAN setup this way leads to significant improvements in training stability, training time, and data efficiency.
Leveraging the benefits of Projected GAN training might enable scaling StyleGAN to ImageNet.
However, as observed by~\cite{Sauer2021NEURIPS}, the advantages of Projected GANs only partially extend to StyleGAN on the unimodal datasets they investigated.
We study this issue and propose architectural changes to address it.
We then design a progressive growing strategy tailored to the latest StyleGAN3. 
These changes in conjunction with Projected GAN already allow surpassing prior attempts of training StyleGAN on ImageNet. 
To further improve results, we analyze the pretrained feature network used for Projected GANs and find that the two standard neural architectures for computer vision, CNNs and ViTs~\cite{Dosovitskiy2021ICLR}, significantly improve performance when used jointly. Lastly, we leverage \textit{classifier guidance}, a technique originally introduced for diffusion models to inject additional class-information~\cite{Dhariwal2021NEURIPS}.

Our contributions culminate in a new state-of-the-art on large-scale image synthesis, pushing the performance beyond existing GAN and diffusion models. 
We showcase inversion and editing for ImageNet classes and find that Pivotal Tuning Inversion (PTI)~\cite{Roich2021ARXIV}, a powerful new inversion paradigm, combines well with our model and even embeds out-of-domain images smoothly into our learned latent space. Our efficient training strategy allows us to triple the parameters of the standard StyleGAN3 while reaching prior state-of-the-art performance of diffusion models~\cite{Dhariwal2021NEURIPS} in a fraction of their training time. It further enables us to be the first to demonstrate image synthesis on ImageNet-scale at a resolution of $1024^2$ pixels. We will open-source our code and models upon publication. 


\section{Related Work}
\label{sec:related-work}
\section{Related Work}
\label{sec:related}
%\begin{itemize}
%	\item The concept of event-based vision
%	\item Description of DVS and the AER protocol
%	\item Related sensors, e.g. ATIS, DAVIS, meDVS.
%\end{itemize}

This section introduces the fundamental concepts and previous contributions relevant for this work. First, \cref{sec:related_landing} discusses bio-inspired landing strategies involving optical flow and associated research involving MAVs. Second, the concept of event-based cameras is described in \cref{sec:related_event_cams}. Third, an overview of existing approaches to optical flow estimation is provided in \cref{sec:related_optical_flow}, including both frame-based camera applications and recently developed event-based techniques.

\subsection{Landing Using Optical Flow} 
\label{sec:related_landing}
Although optical flow does not by itself provide metric scale to motion, information from optical flow fields is useful for several navigation tasks, including landing. Simple bio-inspired strategies were proposed in the past decades that utilize the visual observables in the optical flow field perceived from the ground. Such strategies form a lightweight alternative for visual estimation of three-dimensional structure, ego-motion, and relative pose, which can be performed through visual Simultaneous Localization And Mapping (SLAM) \cite{Davison2007a} or visual odometry \cite{Nister2004}. These techniques have become increasingly efficient over the last few years \cite{Forster2014,Engel2014}, yet still require processing and maintaining large amounts of measurement data. This strongly contrasts with optical flow based techniques, in which all information required for navigation is contained in a small number of visual observables.

The visual observables related to horizontal motion above ground are the ventral flows $\omega_x$, $\omega_y$, referring to the average flows along the $x$ and $y$ image axes. In several experiments with a tethered MAV \cite{Ruffier2014,Expert2015}, the authors mimicked navigation strategies seen in insects, which have been observed to follow terrain and land using ventral flow \cite{Srinivasan1996}. By maintaining a constant ratio between forward motion and height and at the same time slowing down, they perform smooth landings. Ventral flows may also be used for hover stabilization to augment visual vertical control \cite{Alkowatly2015}.

In recent aerial robotic applications, mainly visual observables based on vertical motion were applied, allowing control of vertical dynamics independent of horizontal motion. One of these observables is flow field divergence $D$, i.e. the ratio of vertical velocity to height. Its reciprocal is the time-to-contact to the ground $\tau$. Similar to ventral flows, divergence was seen to guide docking and landing motion in biology. In \cite{Baird2013} honeybees were seen to keep $D$ constant. Hence, velocity is decreased exponentially, ensuring a smooth touchdown. Other strategies for vertical landing exist based on $\tau$, such as the constantly decreasing $\tau$ strategy observed in braking human drivers \cite{Lee1976}. This strategy provides more control over the landing trajectory \cite{Alkowatly2015}, though it involves more parameters.

In practical control systems, a constant divergence approach suffers from instability as height decreases, due to self-induced oscillations. In \citet{DeCroon2016} it was shown that a relation exists between the employed divergence controller gain and the height at which oscillations occur. A main insight then was that a drone could detect its own oscillations, and in this way determine its height. This strategy can be employed to trigger a separate final touchdown phase, or to continuously measure height by landing at a near-unstable control gain. The finding in \citet{DeCroon2016} also contains an important key to high-performance optical flow divergence landings. This was used in \citet{Ho2016a} to develop an adaptive gain controller, which detects the height at the start of a landing maneuver, sets initial controller gains based on the height, and lands while exponentially reducing the gains. Although the landings performed were quite fast compared to landings in the literature ($D=0.3$ compared to a typical $D=0.05$ \cite{Herisse2012}), the speed of the landings in \citet{Ho2016a} are still quite limited by the standard cameras available on the used AR drone 2.0.

\subsection{Event-Based Cameras}
\label{sec:related_event_cams}
Inspired by the workings of biological retinas, event-based cameras rely on a sensing mechanism that fundamentally differs from their frame-based counterparts. In frame-based cameras the pixel values are measured at fixed time intervals to produce a sequence of images. In event-based cameras, on the other hand, pixel activity is driven by light intensity changes. Whenever a pixel measures a local change, it produces a \emph{event}. Specifically, this occurs when the pixel's logarithmic intensity measurement $I(x,y,t)$ (at pixel location $(x,y)$ and timestamp $t$) increases or decreases beyond a threshold $C$:

\begin{equation}
\label{eq:event_threshold}
\lvert\Delta\left(\log I\left(x,y,t\right)\right)\rvert > C
\end{equation}

Events are encoded according to an Address-Event Representation (AER) \cite{Lichtsteiner2008}, which consists of event information encoded by an address and the timestamp of detection. Typically, an event encodes the pixel position $(x,y)$, the timestamp $t$, and the polarity $P\in\lbrace-1,1\rbrace$, which indicates the sign of the intensity change. A visualization of a basic stream of events, in comparison to an equivalent set of frames, is shown in \cref{fig:events_frames}.

\begin{figure}[!htpb]
	\centering
	\includegraphics[width=.65\linewidth]{images/events_frames.png}
	\caption{Frame-based and event-based visual output generated from a simple synthetic scene, in which a black horizontal bar moves upward. The events are visualized as points in space-time, hence showing the trajectory of the leading and trailing edges of the black bar. Events with positive polarity are highlighted in green; those with negative polarity are marked in red.}
	\label{fig:events_frames}
\end{figure}

The sensor used in this work is the Dynamic Vision Sensor (DVS) - specifically, the DVS128 - which is the first commercially available event-based camera \cite{Cho2015}. It features a 128x128 pixel grid operating at an intrascene dynamic range of 120 dB, measuring events at 1 $\upmu$s timing resolution with a latency of 15 $\upmu$s \cite{Lichtsteiner2008}. A picture of the DVS is shown in \cref{fig:dvs}. Since the availability of the DVS, other event-based cameras have been developed. Most notable are the Asynchronous Time-based Image Sensor (ATIS) \cite{Posch2011}, which measures absolute intensity as well as polarity for each event, and the Dynamic and Active pixel Vision Sensor (DAVIS) \cite{Brandli2014}, whose pixels record events as well as full frames. Interesting in the context of this work is the 2.2 gram micro embedded DVS (meDVS) \cite{Conradt2015}, which is highly suitable for on-board MAV applications.

\begin{figure}[!t]
	\centering
	\includegraphics[width=0.3\linewidth]{images/dvs.png}
	\caption{Picture of the event-based camera employed in this work, the DVS.}
	\label{fig:dvs}
\end{figure}

Event-based cameras have several interesting applications for robotic navigation. Initial work has been performed on visual SLAM with event-based cameras \cite{Weikersdorfer2013,Weikersdorfer2014}. In \citet{Mueggler2014} a pose estimation algorithm based on line tracking is applied to a quadrotor, enabling it to perform aggressive maneuvers . Some studies demonstrate the ability to simultaneously reconstruct intensity maps and relative pose \cite{Kim2014} and, more recently, three-dimensional structure \cite{Kim2016}. Others aim at combining the benefits of event-based and frame-based vision using the DAVIS. For example, the method presented in \citet{Kueng2016} uses frames to identify visual features and events to track their position in high-speed motion, in order to perform visual odometry.

\subsection{Optical Flow Estimation}
\label{sec:related_optical_flow}
In the following, we discuss available techniques for estimating optical flow. Many recent visual navigation experiments, in particular those with commercially available quadrotors, employ standard frame-based cameras, in combination with follow-up processing algorithms. Others employ off-the-shelf optical mouse sensors or specialized neuromorphic optical flow sensors, which directly yield translational optical flow output. For event-based cameras, several techniques have recently been developed, yet these have seen limited applications in robotic navigation.

\subsubsection{Estimation from Frame Sequences}
At present, a wide range of optical flow estimation techniques is available for frame-based cameras. Most of these algorithms derive from the brightness constancy assumption, which states that, when a pixel flows from one frame to another, its intensity $I$ is conserved \cite{Baker2011}. This assumption leads to the well-known optical flow constraint:

\begin{equation}
\label{eq:optical_flow_constraint}
I_x(x,y) u + I_y(x,y) v = -I_t(x,y)
\end{equation}

where $x$ and $y$ are the pixel position and $u$ and $v$ denote the unknown optical flow components in pixels per second. The partial derivatives $I_x$, $I_y$, and $I_t$ are obtained from two sequential frames. Since this equation provides two unknown components, a second constraint is necessary to obtain optical flow. Many recent methods aim at providing a dense optical flow field estimate, where optical flow is estimated for any pixel for the frame. In this case, a global cost function minimization is performed, in which a second constraint is provided by prior knowledge. An example of such a constraint is the requirement of smoothness in the flow field in the well-known Horn-Schunck technique \cite{Horn1981}. Recent dense optical flow algorithms provide accurate results for complex scenes, but at the cost of high computation times \cite{Baker2011}.

In recent real-time robotic applications, the most popular frame-based algorithm is the Lucas-Kanade algorithm \cite{Lucas1981}. This algorithm is originally developed for estimating optical flow in the local neighborhood of a pixel. In order to solve \eqref{eq:optical_flow_constraint}, the second assumption is that $u$ and $v$ are constant across neighboring pixels. Therefore, a least-squares system can be composed based on \eqref{eq:optical_flow_constraint}, using $I_x$, $I_y$, and $I_t$ from neighboring pixels. This system can be solved for $u$ and $v$. This technique is mainly applied to sparse estimation, where motion is only computed locally at visual features of interest.

Local optical flow estimation techniques are subjected to the aperture problem, which occurs when motion ambiguity is present due to a limited field of view \cite{Beauchemin1995}. This occurs along object contours which lack clearly distinguishable corner points. The result is that only \emph{normal flow} can be estimated, which is the motion component normal to the contour's orientation. At corner locations, this ambiguity is not present. Only at these points, optical flow is estimated using Lucas-Kanade. Therefore, a corner detection algorithm (e.g. \cite{Rosten2008}) can be applied to first obtain points of interest in the frame. This strategy is applied in many recent optical flow based landing experiments \citet{DeCroon2013, Alkowatly2015, Ho2016, DeCroon2016}.

Alternatively, in \citet{Herisse2012} a 'pyramidal' variant of Lucas-Kanade is applied \cite{Bouguet2000} to account for large displacements. This is a coarse-to-fine approach: optical flow is first computed for a highly downsampled frame. Then, the frame is iteratively refined, computing more detailed optical flow at each refinement level, using the estimate at the previous level to initialize the estimate.

In all these approaches, it is necessary to process full frames, either to find features of interest such as corners, or to obtain sufficiently detailed dense optical flow. While it is possible to use low resolution frames for faster processing, this comes at the cost of reduced detail and hence lower accuracy. Event-based cameras are much less subject to this trade-off, since their output directly highlights locations of interest for estimating optical flow.

\subsubsection{Optical Flow Sensors}
Hardware-based solutions for estimating optical flow have also been applied. Some researchers employ off-the-shelf optical mouse sensors for measuring translational optical flow e.g. \cite{Zufferey2010}. In addition, the visual motion processing in insects inspired researchers to develop highly simplified optical flow sensors, such as the 2-photodetector elementary motion detector was developed for the tethered MAV research in \cite{Ruffier2005,Ruffier2014}. 
Optical flow sensors achieve relatively high sampling rates due to their simplicity. However, their operating principle is generally limited to measuring translational flow. For measuring patterns of optical flow, such as divergence, multiple separate sensors need to be applied and integrated.

\subsubsection{Event-Based Methods}
Since the introduction of the DVS and subsequently developed sensors, several different approaches to event-based optical flow estimation have been developed. Most of these techniques operate on each newly detected event and its spatiotemporal neighborhood, providing sparse optical flow estimates. However, in most cases, the algorithms do not distinguish between corner points and other visual features. Thus, they primarily estimate normal flow. In the following, a brief review of recent approaches is presented.

An adaptation of the frame-based Lucas-Kanade tracker is introduced in \citet{Benosman2012}. Similar to the original algorithm, it solves the optical flow constraint by including the local neighborhood of a pixel. Since absolute measurements of $I$ are not available, the authors replaced the intensity $I$ by the sum of event polarities at a pixel location, obtained over a fixed time window. The reconstructed 'relative intensity' is used to numerically estimate $I_x$, $I_y$, and $I_t$. However, the number of events is generally too low for this approach to provide accurate gradient estimates, in particular for the temporal gradient $I_t$. 

In \citet{Benosman2014} an algorithm is presented that operates on the spatiotemporal representation of events as a point cloud (as shown in \cref{fig:events_frames}). When representing a sequence of events by three-dimensional points of $(x,y,t)$, they form surface-like structures. The gradient of such a surface relates to the motion of the object that triggered the events. By computing a local tangent plane to an event and its neighbor events, normal flow for that event is estimated. A follow-up study employs this algorithm for detecting and tracking corners from neighboring normal flow vectors, hence obtaining fully observable optical flow \cite{Clady2015}. However, real-time results are not yet demonstrated with a non-parallelized implementation.
%\cite{Tschechne2014} investigated an alternative approach based on 
%
%Very recently, a GPU-based technique for simultaneous estimation of optical flow and absolute intensity
%
%However, in particular \cite{Benosman2014}

In \cite{Barranco2014} a technique is introduced that estimates optical flow on object contours, based on both events and absolute intensity measurements. Input events are used to locate motion boundaries on contours. Along each boundary, motion is estimated using the width of the contour, which is computed from the local event distribution and the absolute intensity. The latter can be reconstructed from events, but having separate intensity measurements (e.g. from a DAVIS or ATIS sensor) simplifies the process.

A bio-inspired approach is proposed in \citet{Brosch2015}. In this approach, optical flow is estimated using direction- and speed-selective filters based on the first stages of visual processing in humans. A bank of spatiotemporal filters is employed, each of which is maximally selective for a certain direction and speed of optical flow. For each new event, the neighboring event cloud is convolved with the filters to obtain a confidence measure for each filter. Optical flow for that event is then obtained from the sum of the confidence measures weighted by direction.

More complex event-based algorithms have also been developed, which have not demonstrated real-time performance, but show promising results. In \citet{Barranco2015} a phase-based optical flow method is discussed, which is developed for high-frequency textures. The algorithm is compared to other event-based methods \cite{Benosman2012,Benosman2014,Barranco2014}, indeed showing significant accuracy improvements. Also, an approach was presented for simultaneous estimation of dense optical flow and absolute intensity \cite{Bardow2016}. This is the only available approach aimed towards dense optical flow estimation. Visual results of this method are encouraging, yet a quantitative evaluation is not performed.

Recently, several datasets for event-based visual navigation have been published. The set in \citet{Barranco2016} provides both frame and event measurements from a DAVIS sensor accompanied by odometry measurements. This facilitates comparison between frame-based and event-based techniques for optical flow estimation or visual odometry. However, to the best of the authors' knowledge, an actual comparison of existing techniques has not yet been published for this set. In this respect, the work in \citet{Ruckauer2016} is more relevant for this work, as it features both an event-based dataset and a comparison of various optical flow algorithms. These are variants of the techniques in \cite{Benosman2012} and \cite{Benosman2014}, as well as a basic direction selective algorithm.

We select the local plane fitting algorithm in \citet{Benosman2014} as the basis of the approach in our work. It has shown the most promising results in \cite{Ruckauer2016} and has recently been incorporated into follow-up experiments \cite{Clady2014,Clady2015}. In addition, its implementations yielded real-time operation for high event measurement rates.

%\subsection{Estimation of Visual Observables}
%Optical flow
%The second class of methods is based on the structure of the observed optical flow field, such as the planar flow field structure in \eqref{eq:planar_flow_field1}. In this case, it is possible to simultaneously estimate the flow field parameters $\vartheta_x$, $\vartheta_y$, and $\vartheta_z$, for example through least-squares estimation. This technique is seen in recent vertical landing approaches \cite{DeCroon2013,Alkowatly2015,DeCroon2015}, and forms the basis of our approach.

%Limited work has been performed towards obtaining visual observables from event-based optical flow. In \citet{Clady2014} an event-based method for estimating time-to-contact $\tau$ is described. However, the this approach is limited to motion towards a point which lies within the camera's field of view. Hence, it is not robust to fast rotational motion and translational motion, which are inevitable in on-board applications in MAVs.


\section{Method}
\label{sec:method}
\subsection{Overview}
\begin{figure*}[h]
\begin{center}
\includegraphics[width=0.9\linewidth,height=5.5cm]{imgs/arch1_1.png}
\end{center}
%\caption{The overall structure of AsynDGAN. It is composed of two parts, a central generator $G$ and multiple distributed discriminators $D^1, D^2, \cdots, D^n$ in each medical entity. $G$ takes the mask as input and learns to fool the discriminators. Each discriminators learn to differentiate between the real images of current medical entities and synthetic images from $G$. Notice that each real image $y$ is only used in a specific discriminator located at one hospital/medical center, and only fake images, masks, and generator loss will be transferred between the central generator and hospitals/medical centers. The well-trained central generator then is used as an image provider to train the segmentation network as we expect the synthetic image to share the same or similar distribution with the real image.}
\caption{The overall structure of AsynDGAN. It contains two parts, a central generator $G$ and multiple distributed discriminators $D^1, D^2, \cdots, D^n$ in each medical entity. $G$ takes a task-specific input (segmentation masks in our experiments) and output synthetic images. Each discriminator learns to differentiate between the real images of current medical entity and synthetic images from $G$. The well-trained $G$ is then used as an image provider to train a task-specific model (segmentation in our experiments).}
\label{arch1}
\end{figure*}

Our proposed AsynDGAN is comprised of one central generator and multiple distributed discriminators located in different medical entities. 
In the following, we present the network architecture, object function and then analysis the procedure of the distributed asynchronized optimization.

\subsection{Network architecture}

%An overview of the proposed architecture is shown in Figure \ref{arch1}.
%The central generator, denoted as $G$, takes the segmentation masks as input and generates synthetic images to fool the discriminators. The local discriminators, denote as $D^1$ to $D^n$, learn to differentiate between the real images and the synthetic images. 
%Due to the sensitivity of patient images, the real images may not be accessed from the outside. Our architecture is naturally capable of such limitation since only the specific discriminator which located in the same medical entity could learn the real image. The synthetic images, masks, and the gradients will be transferred between the central generator and the medical entities while keeping the real image privately. 
%The generator will learn the joint distribution from different discriminators that belongs to different medical entities or diseases. Both generator and discriminator follow the convolution-Batchnorm-ReLu \cite{ioffe2015batch} basic block.
%The well-trained central generator will then be used as an image provider to train the segmentation network as we expect the synthetic image to share the same or similar distribution with the real image. We adopt U-Net \cite{ronneberger2015u} as the segmentation network to verify the efficiency of our AsynDGAN architecture. 
% The detailed architecture is described below.

An overview of the proposed architecture is shown in Figure~\ref{arch1}.
The central generator, denoted as $G$, takes task-specific inputs (segmentation masks in our experiments) and generates synthetic images to fool the discriminators. The local discriminators, denote as $D^1$ to $D^n$, learn to differentiate between the local real images and the synthetic images from $G$. 
Due to the sensitivity of patients' images, the real images in each medical center may not be accessed from outside. Our architecture is naturally capable of avoiding such limitation because only the specific discriminator in the same medical entity needs to access the real images. In this way, the real images in local medical entities will be kept privately. Only synthetic images, masks, and gradients are needed to be transferred between the central generator and the medical entities. 

The generator will learn the joint distribution from different datasets that belong to different medical entities. Then it can be used as an image provider to train a specific task, because we expect the synthetic images to share the same or similar distribution as the real images. In the experiments, we apply the AsynDGAN framework to segmentation tasks to illustrate its effectiveness. The U-Net~\cite{ronneberger2015u} is used as the segmentation model, and details about $G$ and $Ds$ designed for segmentation tasks are described below.


\subsubsection{Central generator}
For segmentation tasks, the central generator is an encoder-decoder network that consists of two stride-2 convolutions (for downsampling), nine residual blocks~\cite{he2016resnet}, and two transposed convolutions. All non-residual convolutional layers are followed by batch normalization~\cite{ioffe2015batch} and the ReLU activation. All convolutional layers use $3\times3$ kernels except the first and last layers that use $7\times7$ kernels.

\subsubsection{Distributed discriminators}
In the AsynDGAN framework, the discriminators are distributed over $N$ nodes (hospitals, mobile devices). Each discriminator $D_j$ only has access to data stored in the $j$-th node thus discriminators are trained in an asynchronized fashion. For segmentation, each discriminator has the same structure as that in PatchGAN~\cite{isola2016pix2pix}. The discriminator individually quantifies the fake or real value of different small patches in the image. Such architecture assumes patch-wise independence of pixels in a Markov random field fashion \cite{li2016precomputed,isola2017image}, and can capture the difference in geometrical structures such as background and tumors. 


\subsection{Objective of AsynDGAN}
The AsynDGAN is based on the conditional GAN~\cite{mirza2014conditiongan}. The objective of a classical conditional GAN is:
\begin{equation}
\begin{aligned}
\min\limits_{G}\max\limits_{D}V(D,G) &= \mathbb{E}_{x\sim s(x)}\mathbb{E}_{y\sim p_{data}(y|x)} [\log D(y|x)]\\
&+\mathbb{E}_{\hat{y}\sim p_{\hat{y}}(\hat{y}|x)} [\log(1-D(\hat{y}|x))]
\end{aligned}
\end{equation}
where $D$ represents the discriminator and $G$ is the generator. $G$ aims to approximate the conditional distribution $p_{data}(y|x)$ so that $D$ can not tell if the data is `fake' or not. The hidden variable $x$ is an auxiliary variable to control the mode of generated data~\cite{mirza2014conditiongan}. In reality, $x$ is usually a class label or a mask that can provide information about the data to be generated. Following previous works~(\cite{mathieu2015deep,isola2016pix2pix}), instead of providing Gaussian noise $z$ as an input to the generator, we provide the noise only in the form of dropout, which applied to several layers of the generator of AsynDGAN at both training and test time.

In the AsynDGAN framework, the generator is supervised by $N$ different discriminators. Each discriminator is associated with a subset of datasets. It is natural to quantify such a setting using a mixture distribution on auxiliary variable $x$. In another word, instead of given a naive $s(x)$, the distributions of $x$ becomes $s(x)=\sum\limits_{j\in[N]} \pi_js_j(x)$. For each sub-distribution, there is a corresponding discriminator $D_j$ which only receives data generated from prior $s_j(x)$. Therefore, the loss function of our AsynDGAN becomes:
\begin{equation}
\begin{aligned}
&\min\limits_{G}\max\limits_{D_1:D_N}V(D_{1:N},G) \\
&= \sum\limits_{j\in [N]} \pi_j \{\mathbb{E}_{x\sim s_j(x)}\mathbb{E}_{y\sim p_{data}(y|x)} [\log D_j(y|x)] \\
&+\mathbb{E}_{\hat{y}\sim p_{\hat{y}}(\hat{y}|x)} [\log(1-D_j(\hat{y}|x))]\}
%	&=\sum \pi_j\int\limits_{y} s_j(x)\int\limits_{x} p(y|x)log D_j(y,x)+q(y|x)log(1-D_j(y,x)) dxdy
\end{aligned}
\end{equation}
%	\begin{equation*}
%		L(q)=    \mathbb{E}_{x\sim p}[\pi_jp_j(x) log D_j(x)] + \mathbb{E}_{z\sim q} [q(z)log (1-D_j(z))]
%	\end{equation*}


\subsection{Optimization process}

\begin{figure}[t]
	\vspace{-2em}
	\begin{center}
		\includegraphics[width=5.5cm,height=5.2cm]{imgs/workflow.png}
	\end{center}
	%\caption{The optimization process: the solid arrows show the forward pass, the dotted arrows show gradient flow during the backward pass of our iterative update procedure. Solid blocks indicate that it is being updated while the dotted blocks indicate that the block is frozen during that update step. Red denotes source mask, and Blue denotes target real images}
	%\label{workflow}
	%\end{figure}
	%
	%
	%The optimization process is shown in Figure \ref{workflow}. In each iteration, a randomly sampled tuple $(x, y)$ is provided to the system. Then the network blocks are updated iteratively in the following order:
	%
	%\begin{enumerate}
	%	\item D-update: For each of the discriminator $i$ ($i<N$, N is the number of discriminators), it will calculate adversarial loss $ \mathcal{L}^i_{adv,D}$. The overall discriminator loss is given as: $\mathcal{L}_D=\sum^i_N \mathcal{L}^i_{adv,D}$
	%	\item G-update: After update all discriminators, the central generator will be updated using combination of adversarial loss $\mathcal{L}_{adv,G}$, L1 loss which encourages less blurring $\mathcal{L}_{L1,G}$ and a perception loss $\mathcal{L}_{perception,G}$. The overall generator loss is given as: $\mathcal{L}_G=\mathcal{L}_{adv,G}+\mathcal{L}_{L1,G}+\mathcal{L}_{perception,G}$
	%\end{enumerate}
	%
	%We will discuss the loss function in full detail in the next section.
	\caption{The optimization process of AsynDGAN. The solid arrows show the forward pass, and the dotted arrows show gradient flow during the backward pass of our iterative update procedure. The solid block indicate that it is being updated while the dotted blocks mean that they are frozen during that update step. Red and blue rectangles are source mask and target real image, respectively.}
	\label{workflow}
\end{figure}


The optimization process of the AsynDGAN is shown in Figure~\ref{workflow}. In each iteration, a randomly sampled tuple $(x, y)$ is provided to the system. Here, $x$ denotes the input label which observed by the generator, and $y$ is the real image only accessible by medical entities. Then the network blocks are updated iteratively in the following order:

%\begin{enumerate}[1)]
%	\item D-update: For each of the discriminators $j$ ( $j<N$, $N$ is the number of discriminators), it will calculate adversarial loss for $j$-th discriminator $D_j$. 
%	\item G-update: After update all discriminators, the central generator will then be updated using combination of adversarial loss $\sum_{j \in N} loss(D_j)$.
%\end{enumerate}

\begin{enumerate}[1)]
	\item D-update: Calculating the adversarial loss for $j$-th discriminator $D_j$ and update $D_j$, where $j=1, 2, \cdots, N$.
	\item G-update: After updating all discriminators, $G$ will be updated using the adversarial loss $\sum_{j=1}^N loss(D_j)$.
\end{enumerate}

This process is formulated as Algorithm~\ref{algo1}. We apply the cross entropy loss and in the algorithm and further analyze the AsynDGAN framework in this setting. We stress that the framework is general and can be collaborated with variants of GAN loss including Wasserstein distance and classical regression loss~\cite{arjovsky2017wasserstein,mao2017least}.


\begin{algorithm}[] 
	\caption{\small Training algorithm of AsynDGAN.
	}
	\begin{algorithmic}\label{algo1}
%		\label{alg:AGF}
		\FOR{number of total training iterations}
		\FOR{number of interations to train discriminator}
		\FOR{each node $j \in [N]$}
		\STATE{-- Sample  minibatch of of $m$ auxiliary variables $\{x^j_1,...,x^j_m\}$ from $s_j(x)$ and send to generator $G$.}
		\STATE{-- Generate $m$ fake data from generator $G$, $\{\hat{y}^j_1,...,\hat{y}^j_m\}\sim q(\hat{y}|x)$ and send to node $j$.}
		\STATE{-- Update the discriminator by ascending its stochastic gradient:
			\vspace{-0.5em}
			\[\nabla_{\theta_{D_j}} \frac{1}{m} \sum_{i=1}^m \left[
			\log D_j(y_i^j)
			+ \log (1-D_j(G(\hat{y}_i^j)))
			\right].
			\]}
			\vspace{-1.5em}
		\ENDFOR
		\ENDFOR
		\FOR{each node $j \in [N]$}
		\STATE{-- Sample  minibatch of $m$ auxiliary variables $\{x^j_1,...,x^j_m\}$ from $s_j(x)$ and send to generator $G$.}
		\vspace{-0.2em}		
		\STATE{-- Generate corresponding $m$ fake data from generator $G$, $\{\hat{y}^j_1,...,\hat{y}^j_m\}\sim q(\hat{y}|x)$ and send to node $j$.}
		\vspace{-0.2em}
		\STATE{-- Discriminator $D_j$ passes error to generator $G$.}
		\ENDFOR
		\STATE{-- Update $G$ by descending its stochastic gradient:
			\vspace{-0.5em}
			\[	\nabla_{\theta_G} \frac{1}{Nm} \sum_{j=1}^N\sum_{i=1}^m
			\log (1-D_j(G(\hat{y}^j_i))).\]}
			\vspace{-2em}
		\ENDFOR
		\\The gradient-based updates can use any standard gradient-based learning rule. We used momentum in our experiments.
	\end{algorithmic}
	\vspace{-0.3em}
\end{algorithm}



\subsection{Analysis: AsynDGAN learns the correct distribution}

In this section, we present a theoretical analysis of AsynDGAN and discuss the implications of the results. We first begin with a technical lemma describing the optimal strategy of the discriminator.

\begin{lemma}\label{lem1}
	When generator $G$ is fixed,  the optimal discriminator $D_j(y|x)$ is :\\
	\vspace{-0.5em}
	\begin{equation}
	D_j(y|x)=\frac{p(y|x)}{p(y|x)+q(y|x)}
	\end{equation}
\end{lemma}


Suppose in each training step the discriminator achieves its maxima criterion in Lemma \ref{lem1}, the loss function for the generator becomes:\\
\begin{equation*}
\begin{aligned}
\min\limits_{G}V(G)&= \mathbb{E}_{y}\mathbb{E}_{x\sim p_{data}(y|x)} [\log D(y|x)] \\
&+\mathbb{E}_{\hat{y}\sim p_{\hat{y}}(\hat{y}|x)} [\log(1-D(\hat{y}|x))]\\
&=\sum_{j\in[N]} \pi_j\int\limits_{y} s_j(x)\int\limits_{x} p(y|x)\log\frac{p(y|x)}{p(y|x)+q(y|x)}\\
&+q(y|x)\log\frac{q(y|x)}{p(y|x)+q(y|x)} dxdy\\
\end{aligned}
\end{equation*}
Assuming in each step, the discriminator always performs optimally, we show indeed the generative distribution $G$ seeks to minimize the loss by approximating the underlying distribution of data.
\begin{thm}
	Suppose the discriminators $D_{1\sim N}$ always behave optimally (denoted as $D^*_{1 \sim N}$), the loss function of generator is global optimal iff $q(y,x)=p(y,x)$ where the optimal value of $V(G,D^*_{1\sim N})$ is $-\log 4$. 
\end{thm}

\begin{remark}
	While analysis of AsynDGAN loss shares similar spirit with~\cite{goodfellow2014generative}, it has different implications. In the distributed learning setting, data from different nodes are often dissimilar. Consider the case where $\Omega(s_j(x)) \cap \Omega(s_k(y)) =\emptyset, \text{for } k \neq j$, the information for $p(y|x), y\in \Omega(s_j(x))$ will be missing if we lose the $j$-th node. The behavior of trained generative model is unpredictable when receiving auxiliary variables from unobserved  distribution $s_j(x)$.
	The AsynDGAN framework provides a solution for unifying different datasets by collaborating multiple discriminators.
\end{remark}

\section{Experiments}
\label{sec:exp}
In this section, we first perform experiments on a synthetic dataset to illustrate how AsynDGAN learns a mixed Gaussian distribution from different subsets, and then apply AsynDGAN to the brain tumor segmentation task on BraTS2018 dataset~\cite{bakas2018identifying} and nuclei segmentation task on Multi-Organ dataset~\cite{kumar2017dataset}.


\subsection{Datasets and evaluation metrics}
\subsubsection{Datasets}
\paragraph{Synthetic dataset}
The  synthetic dataset is generated by mixing $3$ one-dimensional Gaussian. In another word, we generate $x\in \{1,2,3\}$ with equal probabilities. Given $x$, the random variable $y$ is generated from  $y =y_1{\textbf{1}_{x=1}}+y_2{\textbf{1}_{x=2}}+y_3{\textbf{1}_{x=3}} $ where $\textbf{1}_{event}$ is the indicator function and $y_1\sim \mathcal{N}(-3,2),y_2\sim \mathcal{N}(1,1), y_3\sim\mathcal{N}(3,0.5)$. Suppose the generator learns the conditional distribution of $y$: $p(y|x)$ perfectly, the histogram should behave similarly to the shape of the histogram of mixture gaussian. 


\paragraph{BraTS2018}

This dataset comes from the Multimodal Brain Tumor Segmentation Challenge 2018~\cite{bakas2017advancing,bakas2018identifying,menze2014multimodal} and contains multi-parametric magnetic resonance imaging (mpMRI) scans of low-grade glioma (LGG) and high-grade glioma (HGG) patients. There are 210 HGG and 75 LGG cases in the training data, and each case has four types of MRI scans and three types of tumor subregion labels. In our experiments, we perform 2D segmentation on T2 images of the HGG cases to extract the whole tumor regions. The 2D slices with tumor areas smaller than 10 pixels are excluded for both GAN training and segmentation phases. In the GAN synthesis phase, all three labels are utilized to generate fake images. For segmentation, we focus on the whole tumor (regions with any of the three labels).

\paragraph{Multi-Organ}
This dataset is proposed by Kumar et al.~\cite{kumar2017dataset} for nuclei segmentation. There are 30 histopathology images of size $1000\times1000$ from 7 different organs. The train set contains 16 images of breast, liver, kidney and prostate (4 images per organ). The same organ test set contains 8 images of the above four organs (2 images per organ) while the different organ test set has 6 images from bladder, colon and stomach. In our experiments, we focus on the four organs that exist both in the train and test sets, and perform color normalization~\cite{reinhard2001color} for all images. Two training images of each organ is treated as a subset that belongs to a medical entity.

\subsubsection{Evaluation metrics}
We adopt the same metrics in the BraTS2018 Challenge~\cite{bakas2018identifying} to evaluate the segmentation performance of brain tumor: Dice score (Dice), sensitivity (Sens), specificity (Spec), and 95\% quantile of Hausdorff distance (HD95). The Dice score, sensitivity (true positive rate) and specificity (true negative rate) measure the overlap between ground-truth mask $G$ and segmented result $S$. They are defined as
\begin{equation}
Dice(G, S) = \frac{2|G \cap S|}{|G| + |S|}
\end{equation}
\begin{equation}
Sens(G, S)=\frac{|G \cap S|}{|G|}
\end{equation}
\begin{equation}
Spec(G, S)=\frac{|(1-G) \cap (1-S)|}{|1-G|}
\end{equation}
The Hausdorff distance evaluates the distance between boundaries of ground-truth and segmented masks:
\begin{equation}
HD(G, S) = \max\{\sup_{x\in\partial G}\inf_{y\in\partial S}d(x, y), \sup_{y\in\partial S}\inf_{x\in\partial G}d(x, y)\}
\end{equation}
where $\partial$ means the boundary operation, and $d$ is Euclidean distance. Because the Hausdorff distance is sensitive to small outlying subregions, we use the 95\% quantile of the distances instead of the maximum as in~\cite{bakas2018identifying}. To simplify the problem while fairly compare each experiment, we choose 2D rather than 3D segmentation task for the BraTS2018 Challenge and compute these metrics on each 2D slices and take an average on all 2D slices in the test set.

For nuclei segmentation, we utilize the Dice score and the Aggregated Jaccard Index (AJI)~\cite{kumar2017dataset}:
\begin{equation}
AJI = \frac{\sum_{i=1}^{n_\mathcal{G}} |G_i \cap S(G_i)|}{\sum_{i=1}^{n_\mathcal{G}}|G_i \cup S(G_i)| + \sum_{k\in K}|S_k|}
\end{equation}
where $S(G_i)$ is the segmented object that has maximum overlap with $G_i$ with regard to Jaccard index, $K$ is the set containing segmentation objects that have not been assigned to any ground-truth object.

\subsection{Implementation details}
In the synthetic learning phase, we use 9-blocks ResNet~\cite{he2016deep} architecture for the generator, and multiple discriminators which have the same structure as that in PatchGAN~\cite{isola2016pix2pix} with patch size $70\times70$. We resize the input image as $286\times286$ and then randomly crop the image to $256\times256$. In addition to the GAN loss and the L1 loss, we also used perceptual loss as described in \cite{Johnson2016Perceptual}. We use minibatch SGD and apply the Adam solver \cite{kingma2014adam}, with a learning rate of 0.0002, and momentum parameters $\beta_1 = 0.5$, $\beta_2 = 0.999$. The batch size we used in AsynDGAN depends on the number of discriminators. We use batch size 3 and 1 for BraTS2018 dataset and Multi-Organ dataset, respectively.

In the segmentation phase, we randomly crop images of 224$\times$224 with a batch size of 16 as input. The model is trained with Adam optimizer using a learning rate of 0.001 for 50 epochs in brain tumor segmentation and 100 epochs in nuclei segmentation. To improve performance, we use data augmentation in all experiments, including random horizontal flip and rotation in tumor segmentation and additional random scale and affine transformation in nuclei segmentation.


\subsection{Experiment on synthetic dataset}
In this subsection, we show that the proposed synthetic learning framework can learn a mixture of Gaussian distribution from different subsets. We compare the quality of learning distribution in $3$ settings: (1) \textbf{Syn-All.} Training a regular GAN using all samples in the dataset.
(2) \textbf{Syn-Subset-n.} Training a regular GAN using only samples in local subset $n$, where $n\in\{1,2,3\}$. (3) \textbf{AsynDGAN.} Training our AsynDGAN using samples in all subsets in a distributed fashion.

%1) The model only has access to local-distribution which represents the locally trained model. 2) The model is trained by an aggregated dataset which represents the globally trained model. 3) The model is trained by the proposed AsynDGAN framework. 



\begin{figure}
\centering
\begin{minipage}{0.325\linewidth}
	\centering\includegraphics[width=\linewidth]{imgs-gaussian/syn-all} \\ (a) Syn-All
\end{minipage}
\begin{minipage}{0.325\linewidth}
\centering\includegraphics[width=\linewidth]{imgs-gaussian/syn-subset} \\ (b) Syn-Subset-n
\end{minipage}
\begin{minipage}{0.325\linewidth}
\centering\includegraphics[width=\linewidth]{imgs-gaussian/syn-AsynDGAN} \\ (c) AsynDGAN
\end{minipage}
\caption{Generated distributions of different methods.}
%\vspace{-0.1in}
\label{fig:gaussian}
\vspace{-1em}
\end{figure}

The learned distributions are shown in Figure~\ref{fig:gaussian}. In particular, any local learning (indicated in Figure~\ref{fig:gaussian}(b)) can only fit one mode Gaussian due to the restriction of local information while AsynDGAN is able to capture global information thus has a comparable performance with the regular GAN using the union of separated datasets (\textbf{Syn-All}).

\subsection{Brain tumor segmentation}\label{sec:exp:hgg}
In this subsection, we show that our AsynDGAN can work well when there are patients' data of the same disease in different medical entities.
\subsubsection{Settings}
There are 210 HGG cases in the training data. Because we have no access to the test data of the BraTS2018 Challenge, we split the 210 cases into train (170 cases) and test (40 cases) sets. The train set is then sorted according to the tumor size and divided into 10 subsets equally, which are treated as data in 10 distributed medical entities. There are 11,057 images in the train set and 2,616 images in the test set. We conduct the following segmentation experiments:
(1) \textbf{Real-All.} Training using real images from the whole train set (170 cases).
(2) \textbf{Real-Subset-n.} Training using real images from the $n$-th subset (medical entity), where $n=1,2,\cdots,10$. There are 10 different experiments in this category.
(3) \textbf{Syn-All.} Training using synthetic images generated from a regular GAN. The GAN is trained directly using all real images from the 170 cases.
(4) \textbf{AsynDGAN.} Training using synthetic images from our proposed AsynDGAN.
The AsynDGAN is trained using images from the 10 subsets (medical entities) in a distributed fashion.

In all experiments, the test set remains the same for fair comparison. It should be noted that in the \textbf{Syn-All} and \textbf{AsynDGAN} experiments, the number of synthetic images are the same as that of real images in \textbf{Real-All}. The regular GAN has the same generator and discriminator structures as AsynDGAN, as well as the hyper-parameters. The only difference is that AsynDGAN has 10 different discriminators, and each of them is located in a medical entity and only has access to the real images in one subset.

\begin{figure*}[t]
	\vspace{-2em}
	\begin{center}
		\includegraphics[width=0.15\linewidth]{imgs-seg-hgg/img-1}
		\includegraphics[width=0.15\linewidth]{imgs-seg-hgg/label-1}
		\includegraphics[width=0.15\linewidth]{imgs-seg-hgg/real-1}
		\includegraphics[width=0.15\linewidth]{imgs-seg-hgg/fake-1}
		\includegraphics[width=0.15\linewidth]{imgs-seg-hgg/subset6-1}
		\includegraphics[width=0.15\linewidth]{imgs-seg-hgg/dadgan-1} \\ \vspace{0.01in}
		\begin{minipage}{0.15\linewidth}
			\centering\includegraphics[width=\linewidth]{imgs-seg-hgg/img-2} \\ (a) Image
		\end{minipage}
		\begin{minipage}{0.15\linewidth}
			\centering\includegraphics[width=\linewidth]{imgs-seg-hgg/label-2} \\ (b) Label
		\end{minipage}
		\begin{minipage}{0.15\linewidth}
			\centering\includegraphics[width=\linewidth]{imgs-seg-hgg/real-2}  \\  (c) Real-All
		\end{minipage}
		\begin{minipage}{0.15\linewidth}
			\centering\includegraphics[width=\linewidth]{imgs-seg-hgg/fake-2} \\ (d) Syn-All
		\end{minipage}
		\begin{minipage}{0.15\linewidth}
			\centering\includegraphics[width=\linewidth]{imgs-seg-hgg/subset6-2} \\ (e) Real-Subset-6
		\end{minipage}
		\begin{minipage}{0.15\linewidth}
			\centering\includegraphics[width=\linewidth]{imgs-seg-hgg/dadgan-2} \\ (f) AsynDGAN
		\end{minipage}
	\end{center}
	\caption{Typical brain tumor segmentation results. (a) Test images. (b) Ground-truth labels of tumor region. (c)-(f) are results of models trained on all real images, synthetic images of regular GAN, real images from subset-6, synthetic images of AsynDGAN, respectively.}
	\label{fig:seg:hgg}
\end{figure*}

\begin{figure}[t]
	\begin{center}
		\includegraphics[width=0.3\linewidth]{imgs-vis-gan/120_skull_mask}
		\includegraphics[width=0.3\linewidth]{imgs-vis-gan/120_syn}
		\includegraphics[width=0.3\linewidth]{imgs-vis-gan/120_real}\\ \vspace{0.01in}
		\begin{minipage}{0.3\linewidth}
			\centering\includegraphics[width=\linewidth]{imgs-vis-gan/1153_skull_mask} \\ (a) Input
		\end{minipage}
		\begin{minipage}{0.3\linewidth}
			\centering\includegraphics[width=\linewidth]{imgs-vis-gan/1153_syn} \\ (b) AsynDGAN
		\end{minipage}
		\begin{minipage}{0.3\linewidth}
			\centering\includegraphics[width=\linewidth]{imgs-vis-gan/1153_real}  \\  (c) Real
		\end{minipage}
	\end{center}
	\caption{The examples of synthetic brain tumor images from the AsynDGAN. (a) The input of the AsynDGAN network. (b) Synthetic images of AsynDGAN based on the input. (c) Real images.}
	\label{fig:syn:hgg}
	\vspace{-1em}
\end{figure}

\begin{table}[t]
	\begin{center}
		\begin{tabular}{lcccc}
			\toprule
			Method & Dice $\uparrow$ & Sens $\uparrow$  & Spec $\uparrow$  & HD95 $\downarrow$ \\
			\midrule
			Real-All & 0.7485 & 0.7983	& 0.9955 & 12.85 \\ \midrule
			Real-Subset-1 & 0.5647 &	0.5766 &	0.9945 &	26.90 \\
			Real-Subset-2 & 0.6158 &	0.6333 &	0.9941 &	21.87 \\
			Real-Subset-3 & 0.6660 &	0.7008 &	0.9950 &	21.90 \\
			Real-Subset-4 & 0.6539 &	0.6600 &	0.9962 &	21.07 \\
			Real-Subset-5 & 0.6352 &	0.6437 &	0.9956 &	19.27 \\
			Real-Subset-6 & 0.6844 &	0.7249 &	0.9935 &	21.10 \\
			Real-Subset-7 & 0.6463 &	0.6252 &	0.9972 &	15.60 \\
			Real-Subset-8 & 0.6661 &	0.6876 &	0.9957 &	18.16 \\
			Real-Subset-9 & 0.6844 &	0.7088 &	0.9953 &	18.56 \\
			Real-Subset-10 & 0.6507 &	0.6596 &	0.9957 &	17.33 \\ \midrule
			Syn-All & 0.7114 &	0.7099 &	0.9969 &	16.22 \\ \midrule
			\textbf{AsynDGAN}  & 0.7043 &	0.7295 &	0.9957 &	14.94 \\
			\bottomrule
		\end{tabular}
	\end{center}
	\caption{Brain tumor segmentation results.}	
%	\vspace{-0.1in}
	\label{tab:hgg}
	\vspace{-1em}
\end{table}

\subsubsection{Results}
The quantitative brain tumor segmentation results are shown in Table~\ref{tab:hgg}. The model trained using all real images (\textbf{Real-All}) is the ideal case that we can access all data. It is our baseline and achieves the best performance. Compared with the ideal baseline, the performance of models trained using data in each medical entity (\textbf{Real-Subset-1$\sim$10}) degrades a lot, because the information in each subset is limited and the number of training images is much smaller.

Our AsynDGAN can learn from the information of all data during training, although the generator doesn't ``see'' the real images. And we can generate as many synthetic images as we want to train the segmentation model. Therefore, the model (\textbf{AsynDGAN}) outperforms all models using single subset. For reference, we also report the results using synthetic images from regular GAN (\textbf{Syn-All}), which is trained directly using all real images. The AsynDGAN has the same performance as the regular GAN, but has no privacy issue because it doesn't collect real image data from medical entities. The examples of synthetic images from AysnDGAN are shown in Figure~\ref{fig:syn:hgg}. Several qualitative segmentation results of each method are shown in Figure~\ref{fig:seg:hgg}.


%\begin{figure*}[t]
%	\begin{center}
%	\includegraphics[width=0.16\linewidth]{imgs-vis-gan/7_skull_mask}
%	\includegraphics[width=0.16\linewidth]{imgs-vis-gan/7_syn}
%	\includegraphics[width=0.16\linewidth]{imgs-vis-gan/7_real}
%	\includegraphics[width=0.16\linewidth]{imgs-vis-gan/80_skull_mask}
%	\includegraphics[width=0.16\linewidth]{imgs-vis-gan/80_syn}
%	\includegraphics[width=0.16\linewidth]{imgs-vis-gan/80_real} \\ \vspace{0.01in}
%	\includegraphics[width=0.16\linewidth]{imgs-vis-gan/120_skull_mask}
%	\includegraphics[width=0.16\linewidth]{imgs-vis-gan/120_syn}
%	\includegraphics[width=0.16\linewidth]{imgs-vis-gan/120_real} 
%	\includegraphics[width=0.16\linewidth]{imgs-vis-gan/1093_skull_mask}
%	\includegraphics[width=0.16\linewidth]{imgs-vis-gan/1093_syn}
%	\includegraphics[width=0.16\linewidth]{imgs-vis-gan/1093_real} \\ \vspace{0.01in}
%	\begin{minipage}{0.16\linewidth}
%		\centering\includegraphics[width=\linewidth]{imgs-vis-gan/1153_skull_mask} \\ (a) Input
%	\end{minipage}
%	\begin{minipage}{0.16\linewidth}
%		\centering\includegraphics[width=\linewidth]{imgs-vis-gan/1153_syn} \\ (b) Synthetic image
%	\end{minipage}
%	\begin{minipage}{0.16\linewidth}
%		\centering\includegraphics[width=\linewidth]{imgs-vis-gan/1153_real}  \\  (c) Real image
%	\end{minipage}
%	\begin{minipage}{0.16\linewidth}
%		\centering\includegraphics[width=\linewidth]{imgs-vis-gan/1353_skull_mask} \\ (a) Input
%	\end{minipage}
%	\begin{minipage}{0.16\linewidth}
%		\centering\includegraphics[width=\linewidth]{imgs-vis-gan/1353_syn} \\ (b) Synthetic image
%	\end{minipage}
%	\begin{minipage}{0.16\linewidth}
%		\centering\includegraphics[width=\linewidth]{imgs-vis-gan/1353_real} \\ (c) Real image
%	\end{minipage}
%	\end{center}
%	\caption{The examples of synthetic images from the AsynDGAN. In every three columns, the left images are the input of the AsynDGAN network, and the middle, right images are synthetic and real image based on the mask input.}
%	\label{fig:seg:hgg}
%\end{figure*}

\begin{figure*}[t]
	\vspace{-2em}
	\begin{center}
		\includegraphics[width=0.15\linewidth]{imgs-seg-nuclei/img-1}
		\includegraphics[width=0.15\linewidth]{imgs-seg-nuclei/label-1}
		\includegraphics[width=0.15\linewidth]{imgs-seg-nuclei/real-1}
		\includegraphics[width=0.15\linewidth]{imgs-seg-nuclei/fake-1}
		\includegraphics[width=0.15\linewidth]{imgs-seg-nuclei/prostate-1}
		\includegraphics[width=0.15\linewidth]{imgs-seg-nuclei/asyndgan-1} \\ \vspace{0.01in}
		\begin{minipage}{0.15\linewidth}
			\centering\includegraphics[width=\linewidth]{imgs-seg-nuclei/img-2} \\ (a) Image
		\end{minipage}
		\begin{minipage}{0.15\linewidth}
			\centering\includegraphics[width=\linewidth]{imgs-seg-nuclei/label-2} \\ (b) Label
		\end{minipage}
		\begin{minipage}{0.15\linewidth}
			\centering\includegraphics[width=\linewidth]{imgs-seg-nuclei/real-2}  \\  (c) Real-All
		\end{minipage}
		\begin{minipage}{0.15\linewidth}
			\centering\includegraphics[width=\linewidth]{imgs-seg-nuclei/fake-2} \\ (d) Syn-All
		\end{minipage}
		\begin{minipage}{0.15\linewidth}
			\centering\includegraphics[width=\linewidth]{imgs-seg-nuclei/prostate-2} \\ (e) subset-prostate
		\end{minipage}
		\begin{minipage}{0.15\linewidth}
			\centering\includegraphics[width=\linewidth]{imgs-seg-nuclei/asyndgan-2} \\ (f) AsynDGAN
		\end{minipage}
	\end{center}
	\caption{Typical nuclei segmentation results. (a) Test images. (b) Ground-truth labels of nuclei. (c)-(f) are results of models trained on all real images, synthetic images of regular GAN, real images from prostate, synthetic images of AsynDGAN, respectively. Distinct colors indicate different nuclei.}
	\label{fig:seg:nuclei}
\end{figure*}


\subsection{Nuclei segmentation}
In this subsection, we apply the AsynDGAN to multiple organ nuclei segmentation and show that our method is effective to learn the nuclear features of different organs. 
%Therefore, the synthetic images can be used to train a good nuclei segmentation model.
\vspace{-1em}
\subsubsection{Settings}
We assume that the training images belong to four different medical entities and each entity has four images of one organ. Similar to Section~\ref{sec:exp:hgg}, we conduct the following experiments:
(1) \textbf{Real-All.} Training using the 16 real images of the train set.
(2) \textbf{Real-Subset-n.} Training using 4 real images from each subset (medical entity), where $n\in\{\text{breast, liver, kidney, prostate}\}$.
(3) \textbf{Syn-All.} Training using synthetic images from regular GAN, which is trained using all 16 real images.
(4) \textbf{AsynDGAN.} Training using synthetic images from the AsynDGAN, which is trained using images from the 4 subsets distributively.
In all above experiments, we use the same organ test set for evaluation.

\begin{table}[t]
	\vspace{-0.7em}
	\begin{center}
		\begin{tabular}{lcc}
			\toprule
			Method & Dice $\uparrow$ & AJI $\uparrow$\\
			\midrule
			Real-All & 0.7833 &	0.5608  \\ \midrule
			Real-Subset-breast & 0.7340  &	0.4942 \\
			Real-Subset-liver & 0.7639 & 0.5191 \\
			Real-Subset-kidney & 0.7416 & 0.4848 \\
			Real-Subset-prostate &0.7704 & 0.5370 \\ \midrule
			Syn-All & 0.7856 &	0.5561 \\ \midrule
			\textbf{AsynDGAN}  & 0.7930 & 0.5608 \\
			\bottomrule
		\end{tabular}
	\end{center}
	\vspace{-0.3em}
	\caption{Nuclei segmentation results.}
	\label{tab:nuclei}
	\vspace{-1em}
\end{table}

\begin{figure}[t]
%	\vspace{-1.5em}
	\begin{center}
		\includegraphics[width=0.3\linewidth]{nuclei-vis-gan/4-label}
		\includegraphics[width=0.3\linewidth]{nuclei-vis-gan/4-dadgan}
		\includegraphics[width=0.3\linewidth]{nuclei-vis-gan/4-img}\\ \vspace{0.01in}
		%\includegraphics[width=0.3\linewidth]{nuclei-vis-gan/20-label}
		%\includegraphics[width=0.3\linewidth]{nuclei-vis-gan/20-dadgan}
		%\includegraphics[width=0.3\linewidth]{nuclei-vis-gan/20-img}\\ \vspace{0.01in}
		\begin{minipage}{0.3\linewidth}
			\centering\includegraphics[width=\linewidth]{nuclei-vis-gan/13-label} \\ (a) Input
		\end{minipage}
		\begin{minipage}{0.3\linewidth}
			\centering\includegraphics[width=\linewidth]{nuclei-vis-gan/13-dadgan} \\ (b) AsynDGAN
		\end{minipage}
		\begin{minipage}{0.3\linewidth}
			\centering\includegraphics[width=\linewidth]{nuclei-vis-gan/13-img}  \\  (c) Real
		\end{minipage}
	\end{center}
	\vspace{-0.5em}
	\caption{The examples of synthetic nuclei images from the AsynDGAN. (a) The input of the AsynDGAN network. (b) Synthetic images of AsynDGAN based on the input. (c) Real images.}
	\label{fig:syn:nuclei}
	\vspace{-0.5em}
\end{figure}
\vspace{-0.5em}
\subsubsection{Results}
The quantitative nuclei segmentation results are presented in Table~\ref{tab:nuclei}. Compared with models using single organ data, our method achieves the best performance. The reason is that local models cannot learn the nuclear features of other organs. Compared with the model using all real images, the AsynDGAN has the same performance, which proves the effectiveness of our method in this type of tasks. The result using regular GAN (\textbf{Syn-All}) is slightly worse than ours, probably because one discriminator is not good enough to capture different distributions of nuclear features in multiple organs. In AsynDGAN, each discriminator is responsible for one type of nuclei, which may be better for the generator to learn the overall distribution. We present several examples of synthetic images from AsynDGAN in Figure~\ref{fig:syn:nuclei}, and typical qualitative segmentation results in Figure~\ref{fig:seg:nuclei}.







\section{Conclusion}
\label{sec:conclude}
In this work, we proposed a distributed GAN learning framework as a solution to the privacy restriction problem in multiple health entities. Our proposed framework applies GAN to aggregate and learns the overall distribution of datasets in different health entities without direct access to patients' data. The well-trained generator can be used as an image provider for training task-specific models, without accessing or storing private patients' data. Our evaluation on different datasets shows that our training framework could learn the real image's distribution from distributed datasets without sharing the patient's raw data. In addition, the task-specific model trained solely by synthetic data has a competitive performance with the model trained by all real data, and outperforms models trained by local data in each medical entity.


\section*{Acknowledgements}
We thank anonymous reviewers for helpful comments. This work was partially supported by ARO-MURI-68985NSMUR, NSF-1909038, NSF-1855759, NSF-1855760, NSF-1733843, NSF-1763523, NSF-1747778 and NSF-1703883.

{\small
\bibliographystyle{ieee_fullname}
\bibliography{main}
}

\documentclass{article}

\usepackage{times}
\usepackage{epsfig}
\usepackage{graphicx}
\usepackage{amsmath}
\usepackage{amssymb}
\usepackage{booktabs}
\usepackage{amsmath,amssymb,amsthm}
\usepackage{algorithm}
\usepackage{algorithmic}
\usepackage{bm}
\newtheorem{thm}{Theorem}
\newtheorem{lemma}{Lemma}
\newtheorem{cor}{Corollary}
\newtheorem{prop}{Proposition}
\theoremstyle{definition}
\newtheorem{definition}{Definition}
\newtheorem{remark}{Remark}
\newtheorem{example}{Example}
\newtheorem{assume}{Assumption}
\newtheorem{obs}{Observation}

\title{The AsynDGAN}
\begin{document}
	
\clearpage
\onecolumn
	\section{Appendix: AsynDGAN learns the correct distribution}
	In this section we present an analysis of AsynDGAN and discuss the implications of the results. We show that the AsynDGAN is able to aggregates multiple separated data set and learn generative distribution in an \emph{all important} fashion.
	We first begin with a technical lemma describing the optimal strategy of the discriminator.
	\begin{lemma}\label{lem1}
		When generator $G$ is fixed,  the optimal discriminator $D_j(y,x)$ is :\\
		\begin{equation}
		D_j(y,x)=\frac{p(y|x)}{p(y|x)+q(y|x)}
		\end{equation}
	\end{lemma}
	%	The proof could be found in \cite{goodfellow}, we include here for completeness.\\
	\textbf{Proof}:\\
	%%	 By taking derivative w.r.t $D_j(y,x)$ on $V(D,G)y$ we have $\sum \pi_j\int\limits_{y} s_j(x)\int\limits_{x} \frac{p(y|x)}{D_j(y,x)}  -q(y|x)log(1-D_j(y,x)) dydx$
	\begin{equation*}
	\begin{aligned}
	&\max\limits_{D}V(D)=\max\limits_{D_1\sim D_N}\sum \pi_j\int\limits_{x} s_j(x)\int\limits_{y} p(y|x)log D_j(y,x)+q(y|x)log(1-D_j(y,x)) dydx\\
	&\leq \sum \pi_j\int\limits_{x} s_j(x)\int\limits_{y} \max\limits_{D_j} \{p(y|x)log D_j(y,x)+q(y|x)log(1-D_j(y,x)) \}dydx
	\end{aligned}
	\end{equation*}
	by setting $D_j(y,x)=\frac{p(y|x)}{p(y|x)+q(y|x)}$ we can maximize each component in the integral thus make the inequality hold with equality.\qed
	
	Suppose in each training step the discriminator achieves its maxima criterion in Lemma \ref{lem1}, the loss function for the generator becomes:\\
	\begin{equation*}
	\begin{aligned}
	&\min\limits_{G}V(G)= \mathbb{E}_{y}\mathbb{E}_{x\sim p_{data}(x|y) [logD(y,x)]} +\mathbb{E}_{z\sim p_{\hat{y}}\sim(\hat{y}|x)} [log(1-D(z,y))]\\
	&=\sum_{j\in[N]} \pi_j\int\limits_{x} s_j(x) \underbrace{\int\limits_{y} p(y|x)log\frac{p(y|x)}{p(y|x)+q(y|x)}+q(y|x)log\frac{q(y|x)}{p(y|x)+q(y|x)} dydx}_{\text {To be analyzed in Lemma \ref{lem2}}}
	\end{aligned}
	\end{equation*}
	
	\begin{lemma} \label{lem2}
		Let $a(y)$ and $b(y)$ be two probability distributions s.t. support  $\Omega(a)\subset \Omega(b)$, the loss function $L(a)=\int\limits_{y} a(y)log\frac{a(y)}{a(y)+b(y)}+b(y)log\frac{b(y)}{a(y)+b(y)} dy \geq  -2 log2$. The inequality holds iff $b(y)=a(y)$.
	\end{lemma}
	
	\textbf{Proof}:\\
	Let $\lambda$ be the Lagrangian multiplier. 
	\begin{equation}
	\begin{aligned}
	L(a,\lambda)=\int\limits_{y} a(y)log\frac{a(y)}{a(y)+b(y)}+b(y)log\frac{b(y)}{a(y)+a(y)} + \lambda a(y) \;\;dy -\lambda
	\end{aligned}
	\end{equation}
	By setting $\frac{\partial L}{\partial a} =0$ we have $log\frac{a(y)}{a(y)+b(y)}=\lambda$ holds for all $x$. The fact that $log\frac{a(y)}{a(y)+b(y)}$ is a constant enforces $a(y)=b(y)$. Plugging $a(y)=b(y)$ into loss function we have $L(a)_{|a(y)=b(y)}=-2log(2)$.  \\
	% 	\begin{lemma}
	%		$L(q)\geq 2log(\frac{1}{2})$
	%	\end{lemma}
	Now we are ready to prove our main result that AsynDGAN learns the correct distribution. Assuming in each step, the discriminator always perform optimally, we show indeed the generative distribution $G$ seeks to minimize the loss by approximating underlying generative distribution of data.
	\begin{thm}
		Suppose the discriminators $D_{1\sim N}$ always behaves optimally (denoted as $D^*_{1 \sim N}$), the loss function of generator is global optimal iff $q(y,x)=p(y,x)$ where the optimal value of $V(G,D^*_{1\sim N})$ is $-log 4$. 
	\end{thm}
	
	\textbf{Proof}:\\
	\begin{equation*}
	\begin{aligned}
	&\min\limits_{\substack{ q(y,x)>0,\\\int\limits_{y} q(y,x)=s(x)}}\sum\limits_{j\in[N]} \pi_j \int\limits_{x}s_j(x)\int\limits_{y} p(y|x)log\frac{p(y|x)}{p(y|x)+q(y|x)}+q(y|x)log\frac{q(y|x)}{p(y|x)+q(y|x)} dydx\\
	&\geq\sum\limits_{j\in[N]} \pi_j \int\limits_{x}s_j(x) \min\limits_{\substack{ q(y|x)>0,\\\int\limits_{y} q(y|x)=1}}\int\limits_{y} p(y|x)log\frac{p(y|x)}{p(y|x)+q(y|x)}+q(y|x)log\frac{q(y|x)}{p(y|x)+q(y|x)} dydx\\
	%	&\geq \sum\limits_{j\in[N]} \pi_j \int\limits_{y}s_j(x) \int\limits_{x} \min\limits_{\substack{ q(y|x),\\\int\limits_{x} q(y|x)=1}} p(y|x)log\frac{p(y|x)}{p(y|x)+q(y|x)}+q(y|x)log\frac{q(y|x)}{p(y|x)+q(y|x)} dydx\\
	%	&\geq 
	\end{aligned}\\
	\end{equation*}
	By Lemma \ref{lem2}, the optimal condition for minimizing:\\
	\begin{equation*}
	\min\limits_{\substack{ q(y|x)>0,\\\int\limits_{y} q(y|x)=1}}\int\limits_{y} p(y|x)log\frac{p(y|x)}{p(y|x)+q(y|x)}+q(y|x)log\frac{q(y|x)}{p(y|x)+q(y|x)} dydx
	\end{equation*}
	is by setting $q(y|x)=p(y|x), \forall x$. Such choice of $q(y|x)$ makes the inequality holds as an equality. Meanwhile $q(y|x)=p(y|x)$ implies $q(y,x)=p(y,x)$ given the fact that $p,q$ has the same marginal distribution on $y$. By plugging in $q(y|x)=p(y|x)$ we can derive the optimal value of $V(G,D^*_{1\sim N})$ to be $-log4$.
	
	\begin{remark}
		While analysis of AsynDGAN loss shares similar spirit with \cite{goodfellow2014generative}, it has different implications. In the distributed learning setting, data from different nodes are often dissimilar. Consider the case where $\Omega(s_j(x)) \cap \Omega(s_k(x)) =\emptyset, for k \neq j$, the information for $p(y|x), x\in \Omega(s_j(x))$ will be missing if we lose the $j$-th node. The behavior of trained generative model is unpredictable when receiving auxiliary variables from unobserved  distribution $s_j(x)$.
		The AsynDGAN framework provides a solution for unifying different datasets by collaborating multiple discriminators thus can aggregate separated datasets in an \emph{all important} fashion.
	\end{remark}
	

	
\end{document}


\end{document}
