\subsection{The privacy policies and challenges in medical intelligence}
The privacy issue, while important in every domain, is enforced vigorously when it comes to medical data. Multiple level of regulations such as HIPAA\cite{annas2003hipaa,centers2003hipaa,mercuri2004hipaa,gostin2009beyond} and the approval process for the Institutional Review Board(IRB) \cite{bankert2006institutional} protect the patients' sensitive data from malicious copy or even tamper evidence of medical conditions. Like a double-edge sword, these regulations objectively cause insufficient collaborations in health records.
For instance, America, European Union and many other countries do not allow patient data leave their country \cite{kerikmae2017challenges,seddon2013cloud}, as a result, many hospitals and research institutions are wary of cloud platforms and prefer to use their own server. 



\subsection{The restriction of Medical data accessibility}
It's widely known that sufficient data volume is necessary for training a successful machine learning algorithm \cite{domingos2012few} for medical image analysis. 
However, due to the policies and challenges mentioned above, it is hard to acquire medical scans for training a machine learning model. In 2016, there were approximately 38 million MRI scans and 79 million CT scans performed in the United States \cite{papanicolas2018health}. Even so, the available datasets for machine learning research still very limited: the largest set of medical image data available to public is 32 thousand \cite{yan2018deeplesion} CT Images, only 0.02\% of the annual acquired images in the United States.
In contrast, the ImageNet \cite{deng2009imagenet} project, which is the large visual dataset designed for use in visual object recognition research, has more than 14 million images have been annotated in more than 20,000 categories.
\subsection{Adaptivity to the architecture updates}
The machine learning architecture evolves rapidly to achieve a better performance by novel loss functions \cite{sudre2017generalised,hochberg1964depth}, neural network modules \cite{hoffman2016fcn, ronneberger2015u,milletari2016v} or optimizers \cite{ruder2016overview, zeiler2012adadelta, mason2000boosting} . To accelerate this process, the modern machine learning frameworks like Pytorch \cite{ketkar2017introduction} and Tensorflow \cite{abadi2016tensorflow} provide more efficient ways to build and update networks and training configurations. We could reasonably infer that the recently well-trained model may outdated or underperformed in the future as new architectures invented. Since the private-sensitive data may not always accessible, even if we once trained the model based on these datasets, we couldn't embrace the new architectures to achieve higher performance. Instead of train a task specific model, our proposed method will train a generator learns from the distributed discriminators. Specifically, We learn the distribution of private datasets as a generator to generate synthetic images for future use without worrying about the lost of the proprietary dataset.

Our proposed approach
Briefly, our contributions lie in two folds: (1) A distributed asynchronized discriminator GAN is proposed to learn the real image's distribution without sharing patient's raw data from different dataset. (2) AsynDGAN achieves higher performance than just learn the real images from only one dataset and almost the same performance to the model learns the real images from all datasets.



%regulation...
%
%
%Especially when Adversarial Generative Network(GAN) attract everyone's attention, the privacy of medical data face a more serious challenge. Though we could apply GAN to achieve many goals like artifact reduction[adversarial sparse-view CBCT], domain adaption for different disease or modality[task driven...], data augmentation[], the GAN, like a double-edged sword, could also hurt us by tampering the medical images, ie., add or remove critical medical findings.


%Since patient data in European countries is typically not allowed to leave Europe, many hospitals and research institutions are wary of cloud platforms and prefer to use their own servers.

