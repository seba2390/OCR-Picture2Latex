
%% bare_jrnl_compsoc.tex
%% V1.4b
%% 2015/08/26
%% by Michael Shell
%% See:
%% http://www.michaelshell.org/
%% for current contact information.
%%
%% This is a skeleton file demonstrating the use of IEEEtran.cls
%% (requires IEEEtran.cls version 1.8b or later) with an IEEE
%% Computer Society journal paper.
%%
%% Support sites:
%% http://www.michaelshell.org/tex/ieeetran/
%% http://www.ctan.org/pkg/ieeetran
%% and
%% http://www.ieee.org/

%%*************************************************************************
%% Legal Notice:
%% This code is offered as-is without any warranty either expressed or
%% implied; without even the implied warranty of MERCHANTABILITY or
%% FITNESS FOR A PARTICULAR PURPOSE!
%% User assumes all risk.
%% In no event shall the IEEE or any contributor to this code be liable for
%% any damages or losses, including, but not limited to, incidental,
%% consequential, or any other damages, resulting from the use or misuse
%% of any information contained here.
%%
%% All comments are the opinions of their respective authors and are not
%% necessarily endorsed by the IEEE.
%%
%% This work is distributed under the LaTeX Project Public License (LPPL)
%% ( http://www.latex-project.org/ ) version 1.3, and may be freely used,
%% distributed and modified. A copy of the LPPL, version 1.3, is included
%% in the base LaTeX documentation of all distributions of LaTeX released
%% 2003/12/01 or later.
%% Retain all contribution notices and credits.
%% ** Modified files should be clearly indicated as such, including  **
%% ** renaming them and changing author support contact information. **
%%*************************************************************************


% *** Authors should verify (and, if needed, correct) their LaTeX system  ***
% *** with the testflow diagnostic prior to trusting their LaTeX platform ***
% *** with production work. The IEEE's font choices and paper sizes can   ***
% *** trigger bugs that do not appear when using other class files.       ***                          ***
% The testflow support page is at:
% http://www.michaelshell.org/tex/testflow/


\documentclass[10pt,journal,compsoc]{IEEEtran}
%
% If IEEEtran.cls has not been installed into the LaTeX system files,
% manually specify the path to it like:
% \documentclass[10pt,journal,compsoc]{../sty/IEEEtran}





% Some very useful LaTeX packages include:
% (uncomment the ones you want to load)


% *** MISC UTILITY PACKAGES ***
%
%\usepackage{ifpdf}
% Heiko Oberdiek's ifpdf.sty is very useful if you need conditional
% compilation based on whether the output is pdf or dvi.
% usage:
% \ifpdf
%   % pdf code
% \else
%   % dvi code
% \fi
% The latest version of ifpdf.sty can be obtained from:
% http://www.ctan.org/pkg/ifpdf
% Also, note that IEEEtran.cls V1.7 and later provides a builtin
% \ifCLASSINFOpdf conditional that works the same way.
% When switching from latex to pdflatex and vice-versa, the compiler may
% have to be run twice to clear warning/error messages.






% *** CITATION PACKAGES ***
%
\ifCLASSOPTIONcompsoc
  % IEEE Computer Society needs nocompress option
  % requires cite.sty v4.0 or later (November 2003)
  \usepackage[nocompress]{cite}
\else
  % normal IEEE
  \usepackage{cite}
\fi
% cite.sty was written by Donald Arseneau
% V1.6 and later of IEEEtran pre-defines the format of the cite.sty package
% \cite{} output to follow that of the IEEE. Loading the cite package will
% result in citation numbers being automatically sorted and properly
% "compressed/ranged". e.g., [1], [9], [2], [7], [5], [6] without using
% cite.sty will become [1], [2], [5]--[7], [9] using cite.sty. cite.sty's
% \cite will automatically add leading space, if needed. Use cite.sty's
% noadjust option (cite.sty V3.8 and later) if you want to turn this off
% such as if a citation ever needs to be enclosed in parenthesis.
% cite.sty is already installed on most LaTeX systems. Be sure and use
% version 5.0 (2009-03-20) and later if using hyperref.sty.
% The latest version can be obtained at:
% http://www.ctan.org/pkg/cite
% The documentation is contained in the cite.sty file itself.
%
% Note that some packages require special options to format as the Computer
% Society requires. In particular, Computer Society  papers do not use
% compressed citation ranges as is done in typical IEEE papers
% (e.g., [1]-[4]). Instead, they list every citation separately in order
% (e.g., [1], [2], [3], [4]). To get the latter we need to load the cite
% package with the nocompress option which is supported by cite.sty v4.0
% and later. Note also the use of a CLASSOPTION conditional provided by
% IEEEtran.cls V1.7 and later.





% *** GRAPHICS RELATED PACKAGES ***
%
\ifCLASSINFOpdf
  % \usepackage[pdftex]{graphicx}
  % declare the path(s) where your graphic files are
  % \graphicspath{{../pdf/}{../jpeg/}}
  % and their extensions so you won't have to specify these with
  % every instance of \includegraphics
  % \DeclareGraphicsExtensions{.pdf,.jpeg,.png}
\else
  % or other class option (dvipsone, dvipdf, if not using dvips). graphicx
  % will default to the driver specified in the system graphics.cfg if no
  % driver is specified.
  % \usepackage[dvips]{graphicx}
  % declare the path(s) where your graphic files are
  % \graphicspath{{../eps/}}
  % and their extensions so you won't have to specify these with
  % every instance of \includegraphics
  % \DeclareGraphicsExtensions{.eps}
\fi
% graphicx was written by David Carlisle and Sebastian Rahtz. It is
% required if you want graphics, photos, etc. graphicx.sty is already
% installed on most LaTeX systems. The latest version and documentation
% can be obtained at:
% http://www.ctan.org/pkg/graphicx
% Another good source of documentation is "Using Imported Graphics in
% LaTeX2e" by Keith Reckdahl which can be found at:
% http://www.ctan.org/pkg/epslatex
%
% latex, and pdflatex in dvi mode, support graphics in encapsulated
% postscript (.eps) format. pdflatex in pdf mode supports graphics
% in .pdf, .jpeg, .png and .mps (metapost) formats. Users should ensure
% that all non-photo figures use a vector format (.eps, .pdf, .mps) and
% not a bitmapped formats (.jpeg, .png). The IEEE frowns on bitmapped formats
% which can result in "jaggedy"/blurry rendering of lines and letters as
% well as large increases in file sizes.
%
% You can find documentation about the pdfTeX application at:
% http://www.tug.org/applications/pdftex






% *** MATH PACKAGES ***
%
%\usepackage{amsmath}
% A popular package from the American Mathematical Society that provides
% many useful and powerful commands for dealing with mathematics.
%
% Note that the amsmath package sets \interdisplaylinepenalty to 10000
% thus preventing page breaks from occurring within multiline equations. Use:
%\interdisplaylinepenalty=2500
% after loading amsmath to restore such page breaks as IEEEtran.cls normally
% does. amsmath.sty is already installed on most LaTeX systems. The latest
% version and documentation can be obtained at:
% http://www.ctan.org/pkg/amsmath





% *** SPECIALIZED LIST PACKAGES ***
%
%\usepackage{algorithmic}
% algorithmic.sty was written by Peter Williams and Rogerio Brito.
% This package provides an algorithmic environment fo describing algorithms.
% You can use the algorithmic environment in-text or within a figure
% environment to provide for a floating algorithm. Do NOT use the algorithm
% floating environment provided by algorithm.sty (by the same authors) or
% algorithm2e.sty (by Christophe Fiorio) as the IEEE does not use dedicated
% algorithm float types and packages that provide these will not provide
% correct IEEE style captions. The latest version and documentation of
% algorithmic.sty can be obtained at:
% http://www.ctan.org/pkg/algorithms
% Also of interest may be the (relatively newer and more customizable)
% algorithmicx.sty package by Szasz Janos:
% http://www.ctan.org/pkg/algorithmicx




% *** ALIGNMENT PACKAGES ***
%
%\usepackage{array}
% Frank Mittelbach's and David Carlisle's array.sty patches and improves
% the standard LaTeX2e array and tabular environments to provide better
% appearance and additional user controls. As the default LaTeX2e table
% generation code is lacking to the point of almost being broken with
% respect to the quality of the end results, all users are strongly
% advised to use an enhanced (at the very least that provided by array.sty)
% set of table tools. array.sty is already installed on most systems. The
% latest version and documentation can be obtained at:
% http://www.ctan.org/pkg/array


% IEEEtran contains the IEEEeqnarray family of commands that can be used to
% generate multiline equations as well as matrices, tables, etc., of high
% quality.




% *** SUBFIGURE PACKAGES ***
%\ifCLASSOPTIONcompsoc
%  \usepackage[caption=false,font=footnotesize,labelfont=sf,textfont=sf]{subfig}
%\else
%  \usepackage[caption=false,font=footnotesize]{subfig}
%\fi
% subfig.sty, written by Steven Douglas Cochran, is the modern replacement
% for subfigure.sty, the latter of which is no longer maintained and is
% incompatible with some LaTeX packages including fixltx2e. However,
% subfig.sty requires and automatically loads Axel Sommerfeldt's caption.sty
% which will override IEEEtran.cls' handling of captions and this will result
% in non-IEEE style figure/table captions. To prevent this problem, be sure
% and invoke subfig.sty's "caption=false" package option (available since
% subfig.sty version 1.3, 2005/06/28) as this is will preserve IEEEtran.cls
% handling of captions.
% Note that the Computer Society format requires a sans serif font rather
% than the serif font used in traditional IEEE formatting and thus the need
% to invoke different subfig.sty package options depending on whether
% compsoc mode has been enabled.
%
% The latest version and documentation of subfig.sty can be obtained at:
% http://www.ctan.org/pkg/subfig




% *** FLOAT PACKAGES ***
%
%\usepackage{fixltx2e}
% fixltx2e, the successor to the earlier fix2col.sty, was written by
% Frank Mittelbach and David Carlisle. This package corrects a few problems
% in the LaTeX2e kernel, the most notable of which is that in current
% LaTeX2e releases, the ordering of single and double column floats is not
% guaranteed to be preserved. Thus, an unpatched LaTeX2e can allow a
% single column figure to be placed prior to an earlier double column
% figure.
% Be aware that LaTeX2e kernels dated 2015 and later have fixltx2e.sty's
% corrections already built into the system in which case a warning will
% be issued if an attempt is made to load fixltx2e.sty as it is no longer
% needed.
% The latest version and documentation can be found at:
% http://www.ctan.org/pkg/fixltx2e


%\usepackage{stfloats}
% stfloats.sty was written by Sigitas Tolusis. This package gives LaTeX2e
% the ability to do double column floats at the bottom of the page as well
% as the top. (e.g., "\begin{figure*}[!b]" is not normally possible in
% LaTeX2e). It also provides a command:
%\fnbelowfloat
% to enable the placement of footnotes below bottom floats (the standard
% LaTeX2e kernel puts them above bottom floats). This is an invasive package
% which rewrites many portions of the LaTeX2e float routines. It may not work
% with other packages that modify the LaTeX2e float routines. The latest
% version and documentation can be obtained at:
% http://www.ctan.org/pkg/stfloats
% Do not use the stfloats baselinefloat ability as the IEEE does not allow
% \baselineskip to stretch. Authors submitting work to the IEEE should note
% that the IEEE rarely uses double column equations and that authors should try
% to avoid such use. Do not be tempted to use the cuted.sty or midfloat.sty
% packages (also by Sigitas Tolusis) as the IEEE does not format its papers in
% such ways.
% Do not attempt to use stfloats with fixltx2e as they are incompatible.
% Instead, use Morten Hogholm'a dblfloatfix which combines the features
% of both fixltx2e and stfloats:
%
% \usepackage{dblfloatfix}
% The latest version can be found at:
% http://www.ctan.org/pkg/dblfloatfix




%\ifCLASSOPTIONcaptionsoff
%  \usepackage[nomarkers]{endfloat}
% \let\MYoriglatexcaption\caption
% \renewcommand{\caption}[2][\relax]{\MYoriglatexcaption[#2]{#2}}
%\fi
% endfloat.sty was written by James Darrell McCauley, Jeff Goldberg and
% Axel Sommerfeldt. This package may be useful when used in conjunction with
% IEEEtran.cls'  captionsoff option. Some IEEE journals/societies require that
% submissions have lists of tables at the end of the paper and that
% tables without any captions are placed on a page by themselves at
% the end of the document. If needed, the draftcls IEEEtran class option or
% \CLASSINPUTbaselinestretch interface can be used to increase the line
% spacing as well. Be sure and use the nomarkers option of endfloat to
% prevent endfloat from "marking" where the figures would have been placed
% in the text. The two hack lines of code above are a slight modification of
% that suggested by in the endfloat docs (section 8.4.1) to ensure that
% the full captions always appear in the list of tables - even if
% the user used the short optional argument of \caption[]{}.
% IEEE papers do not typically make use of \caption[]'s optional argument,
% so this should not be an issue. A similar trick can be used to disable
% captions of packages such as subfig.sty that lack options to turn off
% the subcaptions:
% For subfig.sty:
% \let\MYorigsubfloat\subfloat
% \renewcommand{\subfloat}[2][\relax]{\MYorigsubfloat[]{#2}}
% However, the above trick will not work if both optional arguments of
% the \subfloat command are used. Furthermore, there needs to be a
% description of each subfigure *somewhere* and endfloat does not add
% subfigure captions to its list of figures. Thus, the best approach is to
% avoid the use of subfigure captions (many IEEE journals avoid them anyway)
% and instead reference/explain all the subfigures within the main caption.
% The latest version of endfloat.sty and its documentation can obtained at:
% http://www.ctan.org/pkg/endfloat
%
% The IEEEtran \ifCLASSOPTIONcaptionsoff conditional can also be used
% later in the document, say, to conditionally put the References on a
% page by themselves.




% *** PDF, URL AND HYPERLINK PACKAGES ***
%
%\usepackage{url}
% url.sty was written by Donald Arseneau. It provides better support for
% handling and breaking URLs. url.sty is already installed on most LaTeX
% systems. The latest version and documentation can be obtained at:
% http://www.ctan.org/pkg/url
% Basically, \url{my_url_here}.

\usepackage{booktabs} % For formal tables
\usepackage{xparse}
%\smartqed

%\typeout{FONTSPEC LOADING}
%\usepackage{savesym}
%\savesymbol{liningnums}
%\usepackage{fontspec}
%\restoresymbol{fontspec}{liningnums}

%\usepackage{lineno,hyperref}
\usepackage{hyperref}
\usepackage{enumitem}
\usepackage{epsfig}
\usepackage[lofdepth,lotdepth]{subfig}

\usepackage[numbers,square]{natbib}

\usepackage{url}
\usepackage[noend]{algorithmic}
\usepackage{xcolor}
\usepackage{amsmath, amssymb}

\usepackage[ruled]{algorithm}
\usepackage[export]{adjustbox}

\usepackage{pstricks}
\usepackage{pst-all}
\usepackage{pst-plot}
\usepackage{pst-func,pst-math}
\usepackage{pgfplots}
\usepackage{tikz}
%\usepackage{xltxtra}
\usepackage{etex}
\usepackage{bbm}

\newgray{shadecolor}{0.85}
\definecolor{shadecolor}{gray}{0.85}

\newgray{graylight}{0.75}
\newgray{grayplus}{0.35}
\newgray{graydark}{0.1}

\def\algorithmicrequire{\textbf{Input:}}
\def\algorithmicensure{\textbf{Output:}}
\def\algorithmicif{\textbf{if}}
\def\algorithmicthen{\textbf{then}}
\def\algorithmicelse{\textbf{else}}
\def\algorithmicelsif{\textbf{else if}}
\def\algorithmicfor{\textbf{for}}
\def\algorithmicforall{\textbf{for all}}
\def\algorithmicdo{}
\def\algorithmicwhile{\textbf{while}}
\def\algorithmicrepeat{\textbf{repeat}}
\def\algorithmicuntil{\textbf{until}}
\def\algorithmicloop{\textbf{loop}}

\newcommand{\MovieL}{\textsc{MovieLens}}
\newcommand{\NetF}{\textsc{Netflix}}
\newcommand{\RecS}{\textsc{RecSys 2016}}
\newcommand{\Out}{\textsc{Outbrain}}
\newcommand{\ML}{\textsc{ML}}

\newcommand{\userS}{\mathcal{U}}
\newcommand{\itemS}{\mathcal{I}}
\newcommand{\vecU}{\mathbf{U}}
\newcommand{\vecI}{\mathbf{V}}
\newcommand{\MAP}{\texttt{MAP}}
\newcommand{\NDCG}{\texttt{NDCG}}

\newcommand{\RecNet}{\texttt{RecNet}}
\newcommand{\RecNetE}{{\RecNet}$_{E\!\!\backslash}$}
\newcommand{\MostPop}{\texttt{MostPop}}
\newcommand{\BPR}{\texttt{BPR-MF}}
\newcommand{\CoFactor}{\texttt{Co-Factor}}
\newcommand{\LightFM}{\texttt{LightFM}}


\newcommand{\Loss}{\mathcal{L}}
\newcommand{\Trn}{\mathcal{S}}


\newcommand{\D}{\mathcal D}
\newcommand{\EE}{\mathbb E}
\newcommand{\Ind}{\mathbbm{1}}

\newcommand{\N}{\mathbb N}
\newcommand{\Input}{\mathcal X}
\newcommand{\R}{\mathbb R}
\newcommand{\prefu}{\renewcommand\arraystretch{.2} \begin{array}{c}
   {\succ} \\  \mbox{{\tiny {\it u}}}
  \end{array}\renewcommand\arraystretch{1ex}}


\newcommand{\graph}{\Omega}
\newcommand{\graphH}{\mathcal H}

\newcommand{\vertices}{\mathcal V}
\newcommand{\edges}{\mathcal E}
\newcommand{\Cset}{\mathcal M}
\newcommand{\Weight}{W}
\newcommand{\cover}{\mathcal C}
\newcommand{\Xset}{\mathcal X}
\newcommand{\covers}{{\mathcal K}}
\newcommand{\bfZ}{\mathbf{z}}
\newcommand{\rademacher}{\mathfrak{R}}
\newcommand{\DA}{^\downarrow}
\newcommand{\kasandr}{\textsc{Kasandr}}
\newcommand{\cmmnt}[1]{}

\newtheorem{theorem}{Theorem}
\newtheorem{definition}{Definition}

\usepackage{mathtools}


% *** Do not adjust lengths that control margins, column widths, etc. ***
% *** Do not use packages that alter fonts (such as pslatex).         ***
% There should be no need to do such things with IEEEtran.cls V1.6 and later.
% (Unless specifically asked to do so by the journal or conference you plan
% to submit to, of course. )


% correct bad hyphenation here
\hyphenation{op-tical net-works semi-conduc-tor}
\sloppy
\begin{document}
\begin{sloppypar}
%
% paper title
% Titles are generally capitalized except for words such as a, an, and, as,
% at, but, by, for, in, nor, of, on, or, the, to and up, which are usually
% not capitalized unless they are the first or last word of the title.
% Linebreaks \\ can be used within to get better formatting as desired.
% Do not put math or special symbols in the title.
\title{Representation Learning and Pairwise Ranking for Implicit Feedback in Recommendation Systems}
%
%
% author names and IEEE memberships
% note positions of commas and nonbreaking spaces ( ~ ) LaTeX will not break
% a structure at a ~ so this keeps an author's name from being broken across
% two lines.
% use \thanks{} to gain access to the first footnote area
% a separate \thanks must be used for each paragraph as LaTeX2e's \thanks
% was not built to handle multiple paragraphs
%
%
%\IEEEcompsocitemizethanks is a special \thanks that produces the bulleted
% lists the Computer Society journals use for "first footnote" author
% affiliations. Use \IEEEcompsocthanksitem which works much like \item
% for each affiliation group. When not in compsoc mode,
% \IEEEcompsocitemizethanks becomes like \thanks and
% \IEEEcompsocthanksitem becomes a line break with idention. This
% facilitates dual compilation, although admittedly the differences in the
% desired content of \author between the different types of papers makes a
% one-size-fits-all approach a daunting prospect. For instance, compsoc
% journal papers have the author affiliations above the "Manuscript
% received ..."  text while in non-compsoc journals this is reversed. Sigh.

\author{Sumit Sidana, Mikhail Trofimov,  Oleg Horodnitskii,  Charlotte Laclau, Yury Maximov, Massih-Reza Amini% <-this % stops a space
\IEEEcompsocitemizethanks{\IEEEcompsocthanksitem Univ. Grenoble Alpes/CNRS.\protect\\
		% note need leading \protect in front of \\ to get a newline within \thanks as
		% \\ is fragile and will error, could use \hfil\break instead.
		E-mail: \{fname.lname\}@univ-grenoble-alpes.fr
		\IEEEcompsocthanksitem Federal Research Center "Computer Science and Control" of Russian Academy of Sciences . \protect\\
		E-mail: mikhail.trofimov@phystech.edu
			\IEEEcompsocthanksitem Center for Energy Systems, Skolkovo Institute of Science . \protect\\
		E-mail: Oleg.Gorodnitskii@skoltech.ru
			\IEEEcompsocthanksitem T-4 and CNLS, Los Alamos National Laboratory, and Center for Energy Systems, Skolkovo Institute of Science and Technology . \protect\\
		E-mail: yury@lanl.gov}% <-this % stops an unwanted space
	\thanks{Manuscript received April 19, 2005; revised August 26, 2015.}}

%\author{Michael~Shell,~\IEEEmembership{Member,~IEEE,}
%        John~Doe,~\IEEEmembership{Fellow,~OSA,}
%        and~Jane~Doe,~\IEEEmembership{Life~Fellow,~IEEE}% <-this % stops a space
%\IEEEcompsocitemizethanks{\IEEEcompsocthanksitem M. Shell was with the Department
%of Electrical and Computer Engineering, Georgia Institute of Technology, Atlanta,
%GA, 30332.\protect\\
%% note need leading \protect in front of \\ to get a newline within \thanks as
%% \\ is fragile and will error, could use \hfil\break instead.
%E-mail: see http://www.michaelshell.org/contact.html
%\IEEEcompsocthanksitem J. Doe and J. Doe are with Anonymous University.}% <-this % stops an unwanted space
%\thanks{Manuscript received April 19, 2005; revised August 26, 2015.}}

% note the % following the last \IEEEmembership and also \thanks -
% these prevent an unwanted space from occurring between the last author name
% and the end of the author line. i.e., if you had this:
%
% \author{....lastname \thanks{...} \thanks{...} }
%                     ^------------^------------^----Do not want these spaces!
%
% a space would be appended to the last name and could cause every name on that
% line to be shifted left slightly. This is one of those "LaTeX things". For
% instance, "\textbf{A} \textbf{B}" will typeset as "A B" not "AB". To get
% "AB" then you have to do: "\textbf{A}\textbf{B}"
% \thanks is no different in this regard, so shield the last } of each \thanks
% that ends a line with a % and do not let a space in before the next \thanks.
% Spaces after \IEEEmembership other than the last one are OK (and needed) as
% you are supposed to have spaces between the names. For what it is worth,
% this is a minor point as most people would not even notice if the said evil
% space somehow managed to creep in.



% The paper headers
\markboth{Journal of \LaTeX\ Class Files,~Vol.~14, No.~8, August~2015}%
{Shell \MakeLowercase{\textit{et al.}}: Bare Demo of IEEEtran.cls for Computer Society Journals}
% The only time the second header will appear is for the odd numbered pages
% after the title page when using the twoside option.
%
% *** Note that you probably will NOT want to include the author's ***
% *** name in the headers of peer review papers.                   ***
% You can use \ifCLASSOPTIONpeerreview for conditional compilation here if
% you desire.



% The publisher's ID mark at the bottom of the page is less important with
% Computer Society journal papers as those publications place the marks
% outside of the main text columns and, therefore, unlike regular IEEE
% journals, the available text space is not reduced by their presence.
% If you want to put a publisher's ID mark on the page you can do it like
% this:
%\IEEEpubid{0000--0000/00\$00.00~\copyright~2015 IEEE}
% or like this to get the Computer Society new two part style.
%\IEEEpubid{\makebox[\columnwidth]{\hfill 0000--0000/00/\$00.00~\copyright~2015 IEEE}%
%\hspace{\columnsep}\makebox[\columnwidth]{Published by the IEEE Computer Society\hfill}}
% Remember, if you use this you must call \IEEEpubidadjcol in the second
% column for its text to clear the IEEEpubid mark (Computer Society jorunal
% papers don't need this extra clearance.)



% use for special paper notices
%\IEEEspecialpapernotice{(Invited Paper)}



% for Computer Society papers, we must declare the abstract and index terms
% PRIOR to the title within the \IEEEtitleabstractindextext IEEEtran
% command as these need to go into the title area created by \maketitle.
% As a general rule, do not put math, special symbols or citations
% in the abstract or keywords.
\IEEEtitleabstractindextext{%
\begin{abstract}
In this paper, we propose a novel ranking framework for collaborative filtering with the overall aim of learning user preferences over items by minimizing a pairwise ranking loss. We show the minimization problem involves dependent random variables and provide a theoretical analysis by proving the consistency of the empirical risk minimization in the worst case where all users choose a minimal number of positive and negative items. We further derive a Neural-Network model that jointly learns a new representation of users and items in an embedded space as well as the preference relation of users over the pairs of items. The learning objective is based on three scenarios of ranking losses that control the ability of the model to maintain the ordering over the items induced from the users' preferences, as well as, the capacity of the dot-product defined in the learned embedded space to produce the ordering. The proposed model is by nature suitable for implicit feedback and involves the estimation of only very few parameters. Through extensive experiments on several real-world benchmarks on implicit data, we show the interest of learning the preference and the embedding simultaneously when compared to learning those separately. We also demonstrate that our approach is very competitive with the best state-of-the-art collaborative filtering techniques proposed for implicit feedback.
\end{abstract}

% Note that keywords are not normally used for peerreview papers.
\begin{IEEEkeywords}
Recommender Systems; Learning-to-rank; Neural Networks; Collaborative Filtering
\end{IEEEkeywords}}


% make the title area
\maketitle


% To allow for easy dual compilation without having to reenter the
% abstract/keywords data, the \IEEEtitleabstractindextext text will
% not be used in maketitle, but will appear (i.e., to be "transported")
% here as \IEEEdisplaynontitleabstractindextext when the compsoc
% or transmag modes are not selected <OR> if conference mode is selected
% - because all conference papers position the abstract like regular
% papers do.
\IEEEdisplaynontitleabstractindextext
% \IEEEdisplaynontitleabstractindextext has no effect when using
% compsoc or transmag under a non-conference mode.



% For peer review papers, you can put extra information on the cover
% page as needed:
% \ifCLASSOPTIONpeerreview
% \begin{center} \bfseries EDICS Category: 3-BBND \end{center}
% \fi
%
% For peerreview papers, this IEEEtran command inserts a page break and
% creates the second title. It will be ignored for other modes.
\IEEEpeerreviewmaketitle



%\IEEEraisesectionheading{\section{Introduction}\label{sec:introduction}}
% Computer Society journal (but not conference!) papers do something unusual
% with the very first section heading (almost always called "Introduction").
% They place it ABOVE the main text! IEEEtran.cls does not automatically do
% this for you, but you can achieve this effect with the provided
% \IEEEraisesectionheading{} command. Note the need to keep any \label that
% is to refer to the section immediately after \section in the above as
% \IEEEraisesectionheading puts \section within a raised box.




% The very first letter is a 2 line initial drop letter followed
% by the rest of the first word in caps (small caps for compsoc).
%
% form to use if the first word consists of a single letter:
% \IEEEPARstart{A}{demo} file is ....
%
% form to use if you need the single drop letter followed by
% normal text (unknown if ever used by the IEEE):
% \IEEEPARstart{A}{}demo file is ....
%
% Some journals put the first two words in caps:
% \IEEEPARstart{T}{his demo} file is ....
%
% Here we have the typical use of a "T" for an initial drop letter
% and "HIS" in caps to complete the first word.
%\IEEEPARstart{T}{his} demo file is intended to serve as a ``starter file''
%for IEEE Computer Society journal papers produced under \LaTeX\ using
%IEEEtran.cls version 1.8b and later.
%% You must have at least 2 lines in the paragraph with the drop letter
%% (should never be an issue)
%I wish you the best of success.
%
%\hfill mds
%
%\hfill August 26, 2015
%
%\subsection{Subsection Heading Here}
%Subsection text here.
%
%% needed in second column of first page if using \IEEEpubid
%%\IEEEpubidadjcol
%
%\subsubsection{Subsubsection Heading Here}
%Subsubsection text here.


% An example of a floating figure using the graphicx package.
% Note that \label must occur AFTER (or within) \caption.
% For figures, \caption should occur after the \includegraphics.
% Note that IEEEtran v1.7 and later has special internal code that
% is designed to preserve the operation of \label within \caption
% even when the captionsoff option is in effect. However, because
% of issues like this, it may be the safest practice to put all your
% \label just after \caption rather than within \caption{}.
%
% Reminder: the "draftcls" or "draftclsnofoot", not "draft", class
% option should be used if it is desired that the figures are to be
% displayed while in draft mode.
%
%\begin{figure}[!t]
%\centering
%\includegraphics[width=2.5in]{myfigure}
% where an .eps filename suffix will be assumed under latex,
% and a .pdf suffix will be assumed for pdflatex; or what has been declared
% via \DeclareGraphicsExtensions.
%\caption{Simulation results for the network.}
%\label{fig_sim}
%\end{figure}

% Note that the IEEE typically puts floats only at the top, even when this
% results in a large percentage of a column being occupied by floats.
% However, the Computer Society has been known to put floats at the bottom.


% An example of a double column floating figure using two subfigures.
% (The subfig.sty package must be loaded for this to work.)
% The subfigure \label commands are set within each subfloat command,
% and the \label for the overall figure must come after \caption.
% \hfil is used as a separator to get equal spacing.
% Watch out that the combined width of all the subfigures on a
% line do not exceed the text width or a line break will occur.
%
%\begin{figure*}[!t]
%\centering
%\subfloat[Case I]{\includegraphics[width=2.5in]{box}%
%\label{fig_first_case}}
%\hfil
%\subfloat[Case II]{\includegraphics[width=2.5in]{box}%
%\label{fig_second_case}}
%\caption{Simulation results for the network.}
%\label{fig_sim}
%\end{figure*}
%
% Note that often IEEE papers with subfigures do not employ subfigure
% captions (using the optional argument to \subfloat[]), but instead will
% reference/describe all of them (a), (b), etc., within the main caption.
% Be aware that for subfig.sty to generate the (a), (b), etc., subfigure
% labels, the optional argument to \subfloat must be present. If a
% subcaption is not desired, just leave its contents blank,
% e.g., \subfloat[].


% An example of a floating table. Note that, for IEEE style tables, the
% \caption command should come BEFORE the table and, given that table
% captions serve much like titles, are usually capitalized except for words
% such as a, an, and, as, at, but, by, for, in, nor, of, on, or, the, to
% and up, which are usually not capitalized unless they are the first or
% last word of the caption. Table text will default to \footnotesize as
% the IEEE normally uses this smaller font for tables.
% The \label must come after \caption as always.
%
%\begin{table}[!t]
%% increase table row spacing, adjust to taste
%\renewcommand{\arraystretch}{1.3}
% if using array.sty, it might be a good idea to tweak the value of
% \extrarowheight as needed to properly center the text within the cells
%\caption{An Example of a Table}
%\label{table_example}
%\centering
%% Some packages, such as MDW tools, offer better commands for making tables
%% than the plain LaTeX2e tabular which is used here.
%\begin{tabular}{|c||c|}
%\hline
%One & Two\\
%\hline
%Three & Four\\
%\hline
%\end{tabular}
%\end{table}


% Note that the IEEE does not put floats in the very first column
% - or typically anywhere on the first page for that matter. Also,
% in-text middle ("here") positioning is typically not used, but it
% is allowed and encouraged for Computer Society conferences (but
% not Computer Society journals). Most IEEE journals/conferences use
% top floats exclusively.
% Note that, LaTeX2e, unlike IEEE journals/conferences, places
% footnotes above bottom floats. This can be corrected via the
% \fnbelowfloat command of the stfloats package.




%\section{Conclusion}
%The conclusion goes here.





% if have a single appendix:
%\appendix[Proof of the Zonklar Equations]
% or
%\appendix  % for no appendix heading
% do not use \section anymore after \appendix, only \section*
% is possibly needed

% use appendices with more than one appendix
% then use \section to start each appendix
% you must declare a \section before using any
% \subsection or using \label (\appendices by itself
% starts a section numbered zero.)
%


%\appendices
%\section{Proof of the First Zonklar Equation}
%Appendix one text goes here.
%
%% you can choose not to have a title for an appendix
%% if you want by leaving the argument blank
%\section{}
%Appendix two text goes here.
%
%
%% use section* for acknowledgment
%\ifCLASSOPTIONcompsoc
%  % The Computer Society usually uses the plural form
%  \section*{Acknowledgments}
%\else
%  % regular IEEE prefers the singular form
%  \section*{Acknowledgment}
%\fi
%
%
%The authors would like to thank...


% Can use something like this to put references on a page
% by themselves when using endfloat and the captionsoff option.
\ifCLASSOPTIONcaptionsoff
  \newpage
\fi



% trigger a \newpage just before the given reference
% number - used to balance the columns on the last page
% adjust value as needed - may need to be readjusted if
% the document is modified later
%\IEEEtriggeratref{8}
% The "triggered" command can be changed if desired:
%\IEEEtriggercmd{\enlargethispage{-5in}}





% end of style class




\section{Introduction}

In the recent years, recommender systems (RS) have attracted a lot of interest in both industry and academic research communities, mainly due to new challenges that the design of a decisive and efficient RS presents. Given a set of customers (or users), the goal of RS is to provide a personalized recommendation of products to users which would likely to be of their interest. Common examples of applications include the recommendation of movies (Netflix, Amazon Prime Video), music (Pandora), videos (YouTube), news content (Outbrain) or advertisements (Google). The development of an efficient RS is critical from both the company and the consumer perspective. On one hand, users usually face a very large number of options: for instance, Amazon proposes over 20,000 movies in its selection, and it is therefore important to help them to take the best possible decision by narrowing down the choices they have to make. On the other hand, major companies report significant increase of their traffic and sales coming from personalized recommendations: Amazon declares that $35\%$ of its sales is generated by recommendations, two-thirds of the movies watched on Netflix are recommended and $28\%$ of ChoiceStream users said that they would buy more music, provided the fact that they meet their tastes and interests.\footnote{Talk of Xavier Amatriain - Recommender Systems - Machine Learning Summer School 2014 @ CMU.}

\smallskip

%Pazzani:2007:CRS:1768197.1768209

Two main approaches have been proposed to tackle this problem \cite{Ricci:2010:RSH:1941884}. The first
one, referred to as Content-Based recommendation technique \cite{reference/rsh/LopsGS11} makes use of existing contextual information about the users (e.g. demographic information) or items (e.g. textual description) for recommendation. The second approach, referred to as collaborative filtering (CF) and undoubtedly the most popular one, relies on the past interactions and recommends items to users based on the feedback provided by other similar users. Feedback can be {\it explicit}, in the form of ratings; or {\it implicit}, which includes clicks, browsing over an item or listening to a song. Such implicit feedback is readily available in abundance but is more challenging to take into account as it does not clearly depict the preference of a user for an item. Explicit feedback, on the other hand, is very hard to get in abundance.

\smallskip

The adaptation of CF systems designed for another type of feedback has been shown to be sub-optimal as the basic hypothesis of these systems inherently depends on the nature of the feedback \cite{ir2004010}. Further, learning a suitable representation of users and items has been shown to be the bottleneck of these systems \cite{DBLP:conf/kdd/WangWY15}, mostly in the cases where contextual information over users and items which allow to have a richer representation is unavailable.

\bigskip

In this paper we are interested in the learning of user preferences mostly provided in the form of implicit feedback in RS. Our aim is twofold and concerns:
\begin{enumerate}
\item the development of a theoretical framework for learning user preference in recommender systems and its analysis in the worst case where all users provide a minimum of positive/negative feedback;
\item the design of a new neural-network model based on this framework that learns the preference of users over pairs of items and their representations in an embedded space simultaneously without requiring any contextual information.
\end{enumerate}

We extensively validate our proposed approach over standard benchmarks with implicit feedback by comparing it to state of the art models.

The remainder of this paper is organized as follows. In Section \ref{sec:model}, we define the notations and the proposed framework, and analyze its theoretical properties. Then, Section \ref{sec:sim} provides an overview of existing related methods.
Section \ref{sec:experiment} is devoted to numerical experiments on four real-world benchmark data sets including binarized versions of MovieLens and Netflix, and one real data set on online advertising. We compare different versions of our model with state-of-the-art methods showing the appropriateness of our contribution. Finally, we summarize the study and give possible future research perspectives in Section \ref{sec:conclusion}.


\section{User preference and embedding learning with Neural nets }\label{sec:model}

We denote by $\userS\subseteq \N$ (resp. $\itemS\subseteq \N$) the set of indexes over users (resp. the set of indexes over items). %Further, we suppose that  users and items are represented in the same  input space of dimension $k$, $\Input\subseteq \R^k$. We propose to learn this representation space using an embedded ranking algorithm (Section \ref{sec:RecNetModel}). For each user (resp. item) index $u\in\userS$ (resp. $i\in\itemS$), we then denote by $\vecU_u\in\Input$ (resp. $\vecI_i\in\Input$) its corresponding vector representation in the input space.
Further, for each user $u\in\userS$, we consider two subsets of items $\itemS^-_u\subset \itemS$ and $\itemS^+_u\subset \itemS$ such that;
\begin{itemize}
\item[$i)$]  $\itemS^-_u\neq \varnothing$ and $\itemS^+_u \neq \varnothing$,
\item[$ii)$] for any pair of items $(i,i')\in\itemS^+_u\times \itemS^-_u$; $u$ has a preference, symbolized by \!\!$\prefu$\!\!. Hence $i\!\prefu\! i'$ implies that, user $u$ prefers item $i$ over item $i'$.
\end{itemize}
From this preference relation, a desired output $y_{i,u,i'}\in\{-1,+1\}$ is defined over each triplet $(i,u,i')\in\itemS^+_u\times\userS\times\itemS^-_u$ as:

\begin{equation}
y_{i,u,i'}= \left\{
    \begin{array}{ll}
        1 & \mbox{if } i\!\prefu\! i', \\
        -1 & \mbox{otherwise.}
    \end{array}
\right.
\label{eq:Preference}
\end{equation}


\subsection{Learning objective}

The learning task we address is to find a scoring function $f$ from the class of functions $\mathcal F=\{f\mid f: \itemS\times\userS\times \itemS\rightarrow \R\}$ that minimizes the ranking loss:
\begin{equation}
\label{eq:PrefObj}
\Loss(f)=\EE\left[\frac{1}{|\itemS^+_u||\itemS^-_u|}\sum_{i\in\itemS^+_u}\sum_{i'\in\itemS^-_u}\Ind_{y_{i,u,i'}f(i,u,i')<0}\right],
\end{equation}
%
where $|.|$ measures the cardinality of sets and $\Ind_{\pi}$ is the indicator function which is equal to $1$, if the predicate $\pi$ is true, and $0$ otherwise. Here we suppose that there exists a mapping function $\Phi:\userS\times\itemS\rightarrow \Input\subseteq \mathbb{R}^k$ that projects a pair of user and item indices into a feature space of dimension $k$, and a function $g:\mathcal X\times \mathcal X\rightarrow \R$ such that each function $f\in\mathcal F$ can be decomposed as:
\begin{equation}
\label{eq:deff}
\forall u\in\userS, (i,i')\in\itemS^+_u\times \itemS^-_u,~ f(i,u,i')=g(\Phi(u,i))-g(\Phi(u,i')).
\end{equation}
In the next section we will present a Neural-Network model that learns the mapping function $\Phi$ and outputs the function $f$ based on a non-linear transformation of the user-item feature representation, defining the function $g$.

The previous loss \eqref{eq:PrefObj} is a pairwise ranking loss and it is related to the Area under the ROC curve  \cite{Usunier:1121}. The learning objective is, hence, to find a function $f$ from the class of functions $\mathcal F$ with a small expected risk, by minimizing the empirical error over a training set
\[
S=\{(\bfZ_{i,u,i'}\doteq(i,u,i'),y_{i,u,i'})\mid u\in\userS, (i,i')\in\itemS^+_u\times \itemS^-_u\},
\]
constituted over $N$ users, $\userS=\{1,\ldots,N\}$, and their respective preferences over $M$ items, $\itemS=\{1,\ldots,M\}$ and is given by:
\begin{align}
&\hat\Loss(f,S)=\frac{1}{N}\sum_{u\in\userS}\frac{1}{|\itemS^+_u||\itemS^-_u|}\sum_{i\in\itemS^+_u}\sum_{i'\in\itemS^-_u} \Ind_{y_{i,u,i'}\left(f(i,u,i')\right)<0} \nonumber\\
&=\frac{1}{N}\sum_{u\in\userS}\frac{1}{|\itemS^+_u||\itemS^-_u|}\sum_{i\in\itemS^+_u}\sum_{i'\in\itemS^-_u} \Ind_{y_{i,u,i'}\left(g(\Phi(u,i))-g(\Phi(u,i'))\right)<0}.\label{eq:EmpRisk}
\end{align}
However this minimization problem involves dependent random variables as for each user $u$ and item $i$; all comparisons $g(\Phi(u,i))-g(\Phi(u,i')); i'\in~\itemS^-_u$ involved in the empirical error \eqref{eq:EmpRisk} share the same observation $\Phi(u,i)$. Different studies proposed generalization error bounds for learning with interdependent data \cite{Amini:15}. Among the prominent works that
address this problem are a series of contributions based on the idea of graph coloring introduced in \cite{Janson04RSA}, and which consists in dividing a graph $\graph=(\vertices,\edges)$ that links dependent variables represented by its nodes $\vertices$ into $J$ sets
of {\em independent} variables, called the exact proper fractional cover of $\graph$ and defined as:
\begin{definition}[Exact proper fractional cover of $\graph$, \cite{Janson04RSA}]
\label{def:chromatic}
Let $\graph=(\vertices,\edges)$ be a
graph. $\cover=\{(\Cset_j,\omega_j)\}_{j\in\{1,\ldots,J\}}$, for some positive integer $J$, with
$\Cset_j\subseteq\vertices$ and $\omega_j\in [0,1]$ is an exact proper
fractional cover of $\graph$, if:
%\begin{enumerate}
%\item
i) it is {\em proper:} $\forall j,$ $\Cset_j$ is an {\em independent set}, i.e., there is no connections between vertices in~$\Cset_j$;
%\item
ii) it is an {\em exact fractional cover} of $\graph$: $\forall
  v\in\vertices,\;\sum_{j:v\in\Cset_j}\omega_j= 1$.
%\end{enumerate}
\end{definition}
The weight $\Weight(\cover)$ of $\cover$ is given by: $\Weight(\cover)\doteq\sum_{j=1}^J\omega_j$ and the
minimum weight $\chi^*(\graph)=\min_{\cover\in\covers(\graph)} \Weight(\cover)$ over the set $\covers(\graph)$ of all exact proper fractional covers of $\graph$ is the {\em fractional chromatic number} of $\graph$.


Figure \ref{fig:ProperCover} depicts an exact proper fractional cover corresponding to the problem we consider for a toy problem with $M=1$ user $u$, and $|\itemS^+_u|=2$ items preferred over $|\itemS^-_u|=3$ other ones. In this case, the nodes of the dependency graph correspond to $6$ pairs constituted by; pairs of the user and each of the preferred items, with the pairs constituted by the user and each of the no preferred items, involved in the empirical loss \eqref{eq:EmpRisk}. Among all the sets containing independent pairs of examples, the one shown in Figure \ref{fig:ProperCover},$(c)$ is the exact proper fractional cover of the $\graph$ and the fractional chromatic number is in this case $\chi^*(\graph)=|\itemS^-_u|=3$.
\begin{figure*}[t!]
\begin{center}
\includegraphics[width=.6\textwidth]{FncTrnsf3.pdf}
\end{center}\vspace{-5mm}
\caption{A toy problem with 1 user who prefers $|\itemS_u^+|=2$ items over $|\itemS_u^-|=3$ other ones (top). The dyadic representation of pairs constituted with the representation of the user and each of the representations of preferred and non-preferred items (middle). Different covering of the dependent set, $(a)$ and $(b)$; as well as the exact proper fractional cover, $(c)$, corresponding to the smallest disjoint sets containing independent pairs.}
\label{fig:ProperCover}
\end{figure*}

By mixing the idea of graph coloring with the Laplace transform, Hoeffding like concentration inequalities for the sum of dependent random variables are proposed by~\cite{Janson04RSA}. In \cite{UsunierAG05} this result is extended to provide a generalization of the bounded differences inequality of \cite{mcdiarmid89method} to the case of interdependent random variables. This extension then paved the way for the definition of the {\em fractional Rademacher complexity} that generalizes the idea of Rademacher complexity and allows one to derive generalization bounds for scenarios where the training
data are made of dependent data.

In the worst case scenario where all users provide the lowest interactions over the items, which constitutes the bottleneck of all recommendation systems:
\[
\forall u\in S, |\itemS^-_u|=n_*^-=\mathop{\min}_{u'\in S} |\itemS^-_{u'}| \text{,~and~~} |\itemS^+_u|=n_*^+=\mathop{\min}_{u'\in S} |\itemS^+_{u'}|,
\]

the empirical loss \eqref{eq:EmpRisk} is upper-bounded by:
\begin{equation}
\begin{split}
\label{eq:EmpRisk2}
\hat\Loss(f,S)\le \hat\Loss_*(f,S)\\= \frac{1}{N}\frac{1}{n_*^- n_*^+}\sum_{u\in\userS}\sum_{i\in\itemS^+_u}\sum_{i'\in\itemS^-_u} \Ind_{y_{i,u,i'}f(i,u,i')<0}.
\end{split}
\end{equation}


Following \cite[Proposition 4]{RalaiAmin15}, a generalization error bound can be derived for the second term of the inequality above based on local Rademacher Complexities that implies second-order (i.e. variance) information inducing faster convergence rates.

For sake of presentation and in order to be in line with the learning representations of users and items in an embedded space introduced in Section \ref{sec:RecNetModel}, let us consider kernel-based hypotheses with $\kappa:\Input\times\Input\rightarrow\mathbb{R}$ a {\em positive semi-definite} (PSD) kernel and $\Phi:\userS\times\itemS \rightarrow \Input$ its associated feature mapping function. Further we consider linear functions in the feature space with bounded norm:
\begin{equation}
\mathcal G_B=\{g_{\boldsymbol{w}}\circ \Phi: (u,i)\in \userS\times\itemS \mapsto \langle \boldsymbol{w},\Phi(u,i)\rangle \mid ||\boldsymbol{w}|| \leq B\}
\end{equation}
where $\boldsymbol{w}$ is the weight vector defining the kernel-based hypotheses and $\langle \cdot,\cdot\rangle$ denotes the dot product. We further define the following associated function class:
\begin{equation*}
\resizebox{0.5 \textwidth}{!}{$
\mathcal{F}_B=\{\bfZ_{i,u,i'}\doteq(i,u,i')\mapsto g_{\boldsymbol{w}}(\Phi(u,i))-g_{\boldsymbol{w}}(\Phi(u,i'))\mid g_{\boldsymbol{w}}\in \mathcal G_B\},
$}
\end{equation*}

and the parameterized family $\mathcal{F}_{B,r}$ which, for $r>0$, is defined as:
\[
    \mathcal{F}_{B,r} =
    \{f:f\in\mathcal{F}_B,\mathbb{V}[f]\doteq\mathbb{V}_{\bfZ,y}[\Ind_{y f(\bfZ)}]\leq r\},
\]
where $\mathbb{V}[.]$ denotes the variance.
%
The fractional Rademacher complexity introduced in \cite{UsunierAG05} entails our analysis:
%	\begin{equation}
%	    \label{eq:Rado2}
	    \[
	    \rademacher_{S}(\mathcal{F})=\frac{2}{m}\mathbb{E}_{\xi}\sum_{j=1}^{n_*^-}\mathbb{E}_{\Cset_j}\sup_{f\in\mathcal{F}}\sum_{\alpha\in\Cset_j \atop \bfZ_\alpha \in S}\xi_\alpha f(\bfZ_\alpha),
	\]
	%\end{equation}
	where $m=N\times n_*^+\times n_*^-$ is the total number of triplets $\bfZ$ in the training set and $(\xi_i)_{i=1}^m$ is a sequence of %$N$
	independent Rademacher variables verifying
	$\mathbb{P}(\xi_i=1)=\mathbb{P}(\xi_i=-1)=\frac{1}{2}$.

\bigskip

\begin{theorem}
	\label{thm:WorseCaseRecNet}
        Let $\userS$ be a set of $M$ independent users, such that each user $u \in \userS $ prefers $n_*^+$ items over $n_*^-$ ones in a predefined set of $\itemS$ items. Let $S=\{(\bfZ_{i,u,i'}\doteq(i,u,i'),y_{i,u,i'})\mid u\in\userS, (i,i')\in\itemS^+_u\times \itemS^-_u\}$ be the associated training set, then for any $1>\delta>0$ the following generalization bound holds for all $f\in  \mathcal{F}_{B,r}$ with probability at least $1-\delta$:
        \begin{align*}
     \Loss(f)\le ~~&\hat\Loss_*(f,S) + \frac{2B\mathfrak{C}(S)}{Nn^+_*}+\\ &\frac{5}{2}\left(\sqrt{\frac{2B\mathfrak{C}(S)}{Nn^+_*}}+\sqrt{\frac{r}{2}}\right)\sqrt{\frac{\log\frac{1}{\delta}}{n_*^+}}+\frac{25}{48}\frac{\log\frac{1}{\delta}}{n_*^+},
     \end{align*}
    where $\mathfrak{C}(S)=\sqrt{ \frac{1}{n^-_*}\sum_{j=1}^{n_*^-}\mathbb{E}_{\Cset_j}\left[  \sum_{\alpha\in\Cset_j \atop \bfZ_\alpha \in S}d(\bfZ_\alpha,\bfZ_{\alpha}))\right]}$, $\bfZ_\alpha=(i_\alpha,u_\alpha,i'_\alpha)$ and \\ \begin{align*}
    d(\bfZ_\alpha,\bfZ_{\alpha})=~&\kappa(\Phi(u_\alpha,i_\alpha),\Phi(u_\alpha,i_\alpha))\\+\kappa(\Phi(u_\alpha,i'_\alpha),\Phi(u_\alpha,i'_\alpha))-
    &2\kappa(\Phi(u_\alpha,i_\alpha),\Phi(u_\alpha,i'_\alpha)).
    \end{align*}
\end{theorem}

The proof is given in Appendix. This result suggests that~:
\begin{itemize}
\item even though the training set $S$ contains interdependent observations; following \cite[theorem 2.1, p. 38]{vapnik2000nature}, theorem \ref{thm:WorseCaseRecNet} gives insights on the consistency of the empirical risk minimization principle with respect to \eqref{eq:EmpRisk2},
\item in the case where the feature space $\Input\subseteq \mathbb{R}^k$ is of finite dimension; lower values of $k$ involves lower kernel estimation and hence lower complexity term $\mathfrak{C}(S)$ which implies a tighter generalization bound.
\end{itemize}


\subsection{A Neural Network model to learn user preference}
\label{sec:RecNetModel}
Some studies proposed to find the dyadic representation of users and items in an embedded space, using neighborhood similarity information \cite{Volkovs:2015} or the Bayesian Personalized Ranking (BPR) \cite{rendle_09}. In this section we propose a feed-forward Neural Network, denoted as {\RecNet}, to learn jointly the embedding representation, $\Phi(.)$, as well as the scoring function, $f(.)$, defined previously. The input of the network is a triplet $(i,u,i')$ composed by the indexes of an item $i$, a user $u$ and a second item $i'$; such that the user $u$ has a preference over the pair of items $(i, i')$ expressed by the desired output $y_{i,u,i'}$, defined with respect to the preference relation $\!\prefu\!$ (Eq. \ref{eq:Preference}). Each index in the triplet is then transformed to a corresponding binary indicator vector $\mathbf{i}, \mathbf{u},$ and $\mathbf{i}'$ having all its characteristics equal to $0$ except the one that indicates the position of the user or the items in its respective set, which is equal to $1$. Hence, the following one-hot vector corresponds to the binary vector representation of user $u\in\userS$:
\begin{center}
\includegraphics[width=0.3\textwidth]{u.pdf}
\end{center}
\bigskip



The network entails then three successive layers, namely {\it Embedding} (SG), {\it Mapping} and {\it Dense} hidden layers depicted in Figure  \ref{fig:recnet}.

\begin{figure*}[!ht]
\begin{center}
\includegraphics[width=.7\textwidth]{RecNet.pdf}
%\input{RecNet.tex}
\end{center}\vspace{-5mm}
\caption{The architecture of {\RecNet} trained to reflect the preference of a user $u$ over a pair of items $i$ and $i'$. }
\label{fig:recnet}
\end{figure*}

\begin{itemize}
   \item The {\it Embedding} layer transforms the sparse binary representations of the user and each of the items to a denser real-valued vectors. We denote by $\vecU_u$ and $\vecI_i$ the transformed vectors of user $u$ and item $i$; and $\mathbf{\mathbb{U}}=(\vecU_u)_{u\in\userS}$ and $\mathbf{\mathbb{V}}=(\vecI_i)_{i\in\itemS}$ the corresponding matrices. Note that as the binary indicator vectors of users and items contain one single non-null characteristic, each entry of the corresponding dense vector in the SG layer is connected by only one weight to that characteristic.
   \item The {\it Mapping} layer is composed of two groups of units each being obtained from the element-wise product between the user representation vector $\vecU_u$ of a user $u$ and a corresponding item representation vector $\vecI_i$ of an item $i$ inducing the feature representation of the pair $(u,i); \Phi(u,i)$.  %The latent representations for the three sparse vectors are learned by the skip-gram model.
 %   \item The {\it Element-wise product} layer is composed of two groups of units. Each group is fully connected to the units in the Embedding layer associated to the pair constituted of the user and one of the items.
    \item Each of these units are also fully connected to the units of a {\it Dense} layer composed of successive hidden layers (see Section \ref{sec:experiment} for more details related to the number of hidden units and the activation function used in this layer) .
\end{itemize}

The model is trained such that the output of each of the dense layers reflects the relationship between the corresponding item and the user and is mathematically defined by a multivariate real-valued function $g(.)$.
%\[
%g:\{0,1\}^N\times\{0,1\}^M\rightarrow [0,1],
%\]
Hence, for an input $(i,u,i')$, the output of each of the dense layers is a real-value score that reflects a preference associated to the corresponding pair $(u,i)$ or $(u,i')$ (i.e. $g(\Phi(u,i))$ or $g(\Phi(u,i'))$). Finally the prediction given by {\RecNet} for an input $(i,u,i')$ is:
\begin{equation}
\label{eq:defF}
f(i,u,i')=g(\Phi(u,i))-g(\Phi(u,i')).
\end{equation}


%such that for a given triplet $(i,u,i');  f(\mathbf{i},\mathbf{u},\mathbf{i'})$ should be positive, if and only if the user $u$ prefers item $i$ over item $i'$, and negative otherwise. Thereby, {\RecNet} naturally defines a pairwise ranking problem presented in the next section.


\subsection{Algorithmic implementation}

We decompose the ranking loss as a linear combination of two logistic surrogates:
   \begin{equation}
   \label{eq:rankingLoss}
   \mathnormal
   \Loss_{c,p}(f,\mathbb{U},\mathbb{V},\Trn)=\Loss_c(f,\Trn)+\Loss_p(\mathbb{U},\mathbb{V},\Trn),
    \end{equation}
   where the first term reflects the ability of the non-linear transformation of user and item feature representations, $g(\Phi(.,.))$, to respect the relative ordering of items with respect to users' preferences:
     \begin{equation}
     \resizebox{0.5\textwidth}{!}{$
    \label{eq:ranking_loss}
    \Loss_c(f,\Trn)=\frac{1}{|\Trn|}\sum_{(\bfZ_{i,u,i'},y_{i,u,i'})\in \Trn} \log(1+e^{y_{i,u,i'}(g(\Phi(u,i'))-g(\Phi(u,i))}).
    $}
    \end{equation}
    The second term focuses on the quality of the compact dense vector representations of items and users that have to be found, as measured by the ability of the dot-product in the resulting embedded vector space to respect the relative ordering of preferred items by users:

    \begin{equation}
    \resizebox{0.5\textwidth}{!}{$
    \label{eq:embedding_loss}
    \Loss_p(\mathbb{U},\mathbb{V},\Trn)=\frac{1}{|\Trn|}\!\!\sum_{(\bfZ_{i,u,i'},y_{i,u,i'})\in\Trn}\!\!\left[\log(1+e^{y_{i,u,i'}\vecU_u^\top(\vecI_{i'}-\vecI_{i})})+\lambda(\|\vecU_u\|+\|\vecI_{i'}\|+\|\vecI_{i}\|)\right],
    $}
    \end{equation}

    where $\lambda$ is a regularization parameter for the user and items norms.
Finally, one can also consider a version in which both losses are assigned different weights:
    \begin{equation}\label{eq:rankingLoss_alpha}
   \mathnormal
   \Loss_{c,p}(f,\mathbb{U},\mathbb{V},\Trn)=\alpha\Loss_c(f,\Trn)+(1-\alpha)\Loss_p(\mathbb{U},\mathbb{V},\Trn),
   \end{equation}
where $\alpha\in [0,1]$ is a real-valued parameter to balance between ranking prediction ability and expressiveness of the learned item and user representations. Both options will be discussed in the experimental section.

\subsubsection*{Training phase}

The training of the {\RecNet} is done by back-propagating \cite{Leon2012} the error-gradients from the output to both the deep and embedding parts of the model using mini-batch stochastic optimization (Algorithm 1).

%In the experiments, we used the skip-gram model for learning the latent representations of users and items, and Adam \cite{DBLP:journals/corr/KingmaB14} with $L2$ regularization for the optimization.
%In ${\RecNet}$, we used the back-propagation technique \cite{Leon2012} to compute the error gradients for the weights over different mini-batches. .

During training, the input layer takes a random set $\tilde \Trn_n$ of size $n$ of interactions by building triplets $(i,u,i')$ based on this set, and generating a sparse representation from id's vector corresponding to the picked user and the pair of items. The binary vectors of the examples in  $\tilde \Trn_n$ are then propagated throughout the network, and the ranking error (Eq. \ref{eq:rankingLoss}) is back-propagated.
%Beginning with a set of randomly chosen weights, the algorithm iterates until a maximum number of epochs $T$ is reached.\footnote{Another stopping criteria, based on the relative errors of $\Loss_c$ between two epochs could be chosen as well.}  At each epoch, a mini-batch $\tilde \Trn_n$ is randomly chosen from the original user-item matrix. The concatenated binary vectors of examples in  $\tilde \Trn_n$ are then propagated throughout the network, and the chosen ranking error (Eq. \ref{eq:rankingLoss}) is retro-propagated using the descent gradient algorithm


\begin{algorithm}[!ht]
\caption{{\RecNet$_.$}: Learning phase}
\begin{algorithmic}[J]
%\caption{\textbf{{\RecNet}: Learning phase}}
\REQUIRE \STATE $T$: maximal number of epochs
\STATE A set of users $\userS=\{1,\ldots,N\}$
\STATE A set of items $\itemS=\{1,\ldots,M\}$

\FOR{$ep=1,\dots,T$}
\STATE Randomly sample a mini-batch $\tilde \Trn_n\subseteq \Trn$ of size $n$ from the original user-item matrix
\FORALL{$((i,u,i'),y_{i,u,i'})\in \tilde \Trn_n$}
    	\STATE \textbf{Propagate} $(i,u,i')$ from the input to the output.
\ENDFOR
    	\STATE \textbf{Retro-propagate} the pairwise ranking error (Eq. \ref{eq:rankingLoss}) estimated over $\tilde \Trn_n$.
\ENDFOR
\ENSURE Users and items latent feature matrices $\mathbb{U}, \mathbb{V}$ and the model weights.
\end{algorithmic}
\end{algorithm}

\subsubsection*{Model Testing}
As for the prediction phase, shown in Algorithm 2, a ranked list $\mathfrak N_{u,k}$ of the $k\ll M$ preferred items for each user in the test set is maintained while retrieving the set $\mathcal I$. Given the latent representations of the triplets, and the weights learned; the two first items in $\mathcal I$ are placed in $\mathfrak N_{u,k}$ in a way which ensures that preferred one, $i^*$, is in the first position. Then, the algorithm retrieves the next item, $i\in \mathcal I$ by comparing it to $i^*$. This step is simply carried out by comparing the model's output over the concatenated binary indicator vectors of $(i^*, u, i)$ and $(i, u, i^*)$.

\medskip

\begin{algorithm}[b!]
\caption{{\RecNet$_.$}: Testing phase}
\begin{algorithmic}[J]
%\caption{\textbf{{\RecNet}: Test phase}}
\REQUIRE \STATE A user $u\in\userS$; A set of items $\itemS=\{1,\ldots,M\}$; \\
A set containing the $k$ preferred items in $\itemS$ by $u$;\\
$\mathfrak N_{u,k} \leftarrow \varnothing$;
\STATE The output of {\RecNet}$_{.}$ learned over a training set: $f$
\STATE Apply $f$ to the first two items of $\mathcal I$ and, note the preferred one $i^*$ and place it at the top of $\mathfrak N_{u,k}$;
\FOR{$i=3,\dots,M$}
		\IF {$g(\Phi(u,i))>g(\Phi(u,i^*))$}
			\STATE Add $i$ to $\mathfrak N_{u,k}$ at rank 1
		\ELSE
		    \STATE $j\leftarrow 1$
			\WHILE {$j\le k$ AND $g(\Phi(u,i))<g(\Phi(u,i_g))$) {\color{gray} // where $i_g=\mathfrak N_{u,k}(j)$}}
			    \STATE $j\leftarrow j+1$
			\ENDWHILE
            \IF {$j\le k$}
              \STATE Insert $i$ in $\mathfrak N_{u,k}$ at rank $j$
            \ENDIF
        \ENDIF
\ENDFOR
\ENSURE $\mathfrak N_{u,k}$;
\end{algorithmic}
\end{algorithm}

Hence, if $f(i,u,i^*)>f(i^*,u,i)$, which from Equation \ref{eq:defF} is equivalent to $g(\Phi(u,i))~>~g(\Phi(u,i^*))$, then $i$ is predicted to be preferred over $i^*$; $i \prefu i^*$; and it is put at the first place instead of $i^*$ in $\mathfrak N_{u,k}$. Here we assume that the predicted preference relation \!\!$\prefu$\!\! is transitive, which then ensures that the predicted order in the list is respected. Otherwise, if $i^*$ is predicted to be preferred over $i$, then $i$ is compared to the second preferred item in the list, using the model' prediction as before, and so on. The new item, $i$, is inserted in $\mathfrak N_{u,k}$ in the case if it is found to be preferred over another item in $\mathfrak N_{u,k}$.

\medskip

By repeating the process until the end of $\mathcal I$, we obtain a ranked list of the $k$ most preferred items for the user $u$. Algorithm 2 does not require an ordering of the whole set of items, as also in most cases we are just interested in the relevancy of the top ranked items for assessing the quality of a model. Further, its complexity is at most $O(k\times M)$ which is convenient in the case where $M>\!\!>\! 1$. The merits of a similar algorithm have been discussed by \cite{Ailon08anefficient} but, as pointed out above, the basic assumption for inserting a new item in the ranked list $\mathfrak N_{u,k}$ is that the predicted preference relation induced by the model should be transitive, which may not hold in general.

\smallskip

In our experiments, we also tested a more conventional inference algorithm, which for a given user $u$, consists in the ordering of items in $\mathcal I$ with respect to the output given by the function $g$, and we did not find any substantial difference in the performance of {\RecNet}$_.$, as presented in the following section.

\section{(Un)-related work}\label{sec:sim}

This section provides an overview of the state-of-the-art approaches that are the most similar to ours.

\subsection{Neural Language Models}
Neural language models have proven themselves to be successful in many natural language processing tasks including speech recognition, information retrieval and sentiment analysis. These models are based on a distributional hypothesis stating that words, occurring in the same context with the same frequency, are similar. In order to capture such similarities, these approaches propose to embed the word distribution into a low-dimensional continuous space using Neural Networks, leading to the development of several powerful and highly scalable language models such as the word2Vec Skip-Gram (SG) model \cite{word_emb,mikolov_13}.

%shazeer2016swivel

\medskip

The recent work of \cite{levy_14} has shown new opportunities to extend the word representation learning to characterize more complicated pieces of information. In fact, this paper established the equivalence between SG model with negative sampling, and implicitly factorizing a point-wise mutual information (PMI) matrix. Further, they demonstrated that word embedding can be applied to different types of data, provided that it is possible to design an appropriate context matrix for them. This idea has been successfully applied to recommendation systems where different approaches attempted to learn representations of items and users in an embedded space in order to meet the problem of recommendation more efficiently \cite{guardia_15, liang_16, DBLP:conf/kdd/GrbovicRDBSBS15}.

% Covington:2016:DNN:2959100.2959190, He:2017:NCF:3038912.3052569

\medskip

In \cite{ He:2017:NCF:3038912.3052569}, the authors
used a bag-of-word vector representation of items and users, from which the latent representations of latter are learned through word-2-vec.
\cite{liang_16} proposed a model that relies on the intuitive idea that the pairs of items which are scored in the same way by different users are similar. The approach reduces to finding both the latent representations of users and items, with the traditional Matrix Factorization (MF) approach, and simultaneously learning item embeddings using a co-occurrence shifted positive PMI (SPPMI) matrix defined by items and their context. The latter is used as a regularization term in the traditional objective function of MF. Similarly, in \cite{DBLP:conf/kdd/GrbovicRDBSBS15} the authors proposed Prod2Vec, which embeds items using a Neural-Network language model applied to a time series of user purchases. This model was further extended in \cite{vasile_16} who, by defining appropriate context matrices, proposed a new model called Meta-Prod2Vec. Their approach learns a representation for both items and side information available in the system. The embedding of additional information is further used to regularize the item embedding.
Inspired by the concept of sequence of words; the approach proposed by \cite{guardia_15} defined the consumption of items by users as trajectories. Then, the embedding of items is learned using the SG model and the users' embeddings are further used to predict the next item in the trajectory. In these approaches, the learning of item and user representations are employed to make prediction with predefined or fixed similarity functions (such as dot-products) in the embedded space.

\subsection{Learning-to-Rank with Neural Networks}
Motivated by automatically tuning the parameters involved in the combination of different scoring functions, Learning-to-Rank approaches were originally developed for Information Retrieval (IR) tasks and are grouped into three main categories: pointwise, listwise and pairwise \cite{Liu:2009}.

\medskip

Pointwise approaches \cite{Crammer01, Li08} assume that each queried document pair has an ordinal score. Ranking is then formulated as a regression problem, in which the rank value of each document is estimated as an absolute quantity. In the case
where relevance judgments are given as pairwise preferences
(rather than relevance degrees), it is usually not straightforward to apply these algorithms for learning. Moreover, pointwise techniques do not consider the inter-dependency among documents, so that the position of documents in the final ranked list is missing in the regression-like loss functions used for parameter tuning. On the other hand, listwise approaches \cite{Shi:2010, Xu07, Xu08} take the entire ranked
list of documents for each query as a training instance. As
a direct consequence, these approaches are able to differentiate documents from different queries, and consider their position in the output ranked list at the training stage. Listwise techniques aim to directly optimize a ranking
measure, so they generally face a complex optimization problem dealing with non-convex, non-differentiable and discontinuous functions. Finally, in pairwise approaches \cite{Cohen99, Freund03, Joachims02, PessiotTUAG07} the ranked list is decomposed into a set of document pairs. Ranking is therefore considered as the classification of pairs of documents, such that a classifier is trained by minimizing the number of misorderings in ranking. In the test phase, the classifier assigns a positive or negative class label to a document pair that indicates which of the documents in the pair should be better ranked than the other one.

\smallskip

Perhaps the first Neural Network model for ranking is RankProp, originally proposed by \cite{Caruana:1995}. RankProp is a pointwise approach that alternates between two phases of learning the desired real outputs by minimizing a Mean Squared Error (MSE) objective, and a modification of the desired values themselves to reflect the current ranking given by the net. Later on \cite{DBLP:conf/icml/BurgesSRLDHH05} proposed RankNet, a pairwise approach, that learns a preference function by minimizing a cross entropy cost over the pairs of relevant and irrelevant examples. SortNet proposed by \cite{DBLP:conf/icann/RigutiniPMB08, DBLP:journals/tnn/RigutiniPMS11} also learns a preference function by minimizing a ranking loss over the pairs of examples that are selected iteratively with the overall aim of maximizing the quality of the ranking. The three approaches above consider the problem of Learning-to-Rank for IR and without learning an embedding.

%\subsection{Deep Learning for Recommendation Systems}
%In this work, we propose to conduct learning-to-rank (L2R) and representation learning (RL) in a unified framework, where the embedded representations of users, items and the preferences of users over pairs of items are learned simultaneously.


%%%%%%%%%%%%%%%%%%%%%%%%%%%%%%%%%%%%%%%%%%
\section{Experimental Results}\label{sec:experiment}

We conducted a number of experiments aimed at evaluating how the simultaneous learning of user and item representations, as well as the preferences of users over items can be efficiently handled with {\RecNet}$_.$. To this end, we considered four real-world benchmarks commonly used for collaborative filtering. We validated our approach with respect to different hyper-parameters that impact the accuracy of the model and compare it with competitive state-of-the-art approaches.

\medskip

We run all experiments on a cluster of five {32 core Intel Xeon @ 2.6Ghz CPU (with 20MB cache per core)} systems with {256 Giga} RAM running {Debian GNU/Linux 8.6 (wheezy)} operating system.\cmmnt{Finally, since \NetF\ and \kasandr\ data sets are quite large, we run experiments on these data sets on 2 GRID-GPU(s) each having 8 GPU(s) of their own with {4 Giga} RAM.}
All subsequently discussed components were implemented in Python3 using the TensorFlow library with version 1.4.0.\footnote{\url{https://www.tensorflow.org/}. }$^,$\footnote{For research purpose we will make available all the codes implementing Algorithms 1 and 2 that we used in our experiments and all the pre-processed datasets.}
\begin{figure*}[!t]
    \centering
       \subfloat[{\ML-100K}]{
    \includegraphics[width=0.4\textwidth]{PlotMap_ML100K}
    }
    \subfloat[{\ML-1M}]{
    \includegraphics[width=0.4\textwidth]{PlotMap_ML1M}
    }\vspace{-2mm}\\
        \subfloat[{\kasandr}]{
    \includegraphics[width=0.4\textwidth]{PlotMap_KASANDR}
    }
            \subfloat[{\NetF{}}]{
    \includegraphics[width=0.4\textwidth]{PlotMap_NETFLIX}
    }\vspace{-1mm}
    \caption{MAP@1 as a function of the dimension of the embedding for {\ML}-100K, {\ML}-1M and {\kasandr}.}
    \label{fig:emb_dimension}
\end{figure*}

\subsection{Datasets}
\label{sec:Data}
We report results obtained on three publicly available movie data\-sets, for the
task of personalized top-N recommendation: {\MovieL}\footnote{\url{https://movielens.org/}} 100K (\ML-100K), {\MovieL} 1M (\ML-1M) \cite{Harper:2015:MDH:2866565.2827872}, {\NetF}\footnote{\url{http://academictorrents.com/details/9b13183dc4d60676b773c9e2cd6de5e5542cee9a}}, and one clicks dataset, {\kasandr}-Germany \footnote{\url{https://archive.ics.uci.edu/ml/datasets/KASANDR}} \cite{DBLP:conf/sigir/sidana17}, a recently released data set for on-line advertising.
\begin{itemize}
\item \ML-100K, \ML-1M and {\NetF} consists of user-movie ratings, on a scale of one to five, collected from a movie recommendation service and the Netflix company. The latter was released to support the Netlfix Prize competition\footnote{B. James and L. Stan, The Netflix Prize (2007).}. \cmmnt{\ML-100K dataset gathers 100,000 ratings from 943 users on 1682 movies, \ML-1M dataset comprises of 1,000,000 ratings from 6040 users and 3900 movies and {\NetF} consists of 100 million ratings from 480,000 users and 17,000 movies.} For all three datasets, we only keep users who have rated at least five movies and remove users who gave the same rating for all movies. In addition, for {\NetF}, we take a subset of the original data and randomly sample $20\%$ of the users and $20\%$ of the items. In the following experiments, as we only compare with approaches developed for the ranking purposes and our model is designed to handle implicit feedback, these three data sets are made binary such that a rating higher or equal to 4 is set to 1 and to 0 otherwise.

%\paragraph{{\RecS} clicks} \mbox{} \\
%We created one dataset out of the last {\RecS} Challenge . For {\RecS}, the initial task given by the challenge was to predict those job postings that a user will positively interact with (e.g. click, bookmark). For {\Out}, the task given by the challenge is to recommend content (e.g. news) that will most likely interest a user. For both datasets, we limited ourselves to the task of click prediction. In addition, we only keep users who clicked at least 5 times and did not click on at least 5 of the offers that were shown to them. For each user, thus, we keep at least 5 offers with a positive feedback (which they click) and at least 5 offers with a negative feedback (which they were shown and chose to ignore).

%\paragraph{{\Out} clicks}\mbox{} \\
%We created a dataset out of the last {\Out} Challenge. The initial . In this case as well, we limited ourselves to the task of click prediction. In this case also, we only keep users who clicked at least 5 times and did not click on at least 5 of the job offers that were shown to them. This way for each user we keep at least 5 job offers with a positive feedback (which they clicked) and at least 5 job offers with a negative feedback (which they were shown and chose to ignore)
\vspace{1mm}\item The original {\kasandr} dataset contains the interactions and clicks done by the users of Kelkoo, an online advertising platform, across twenty Europeans countries. In this article, we used a subset of {\kasandr} that only considers interactions from Germany. It gathers 17,764,280 interactions from 521,685 users on 2,299,713 offers belonging to 272 categories and spanning across 801 merchants. For \kasandr, we remove users who gave the same rating for all offers. This implies that all the users who never clicked\cmmnt{ (and always had negative rating on all offers) } or always clicked on each and every offer shown to them\cmmnt{ (and always had positive rating on all offers)} were removed.
\end{itemize}

Table \ref{tab:dataset-description} provides the basic statistics on these collections after pre-processing, as discussed above.

\begin{table}[!htpb]
\tiny
\centering
\caption{Statistics of various collections used in our experiments after preprocessing.}\vspace{-2mm}
\label{tab:dataset-description}
    \resizebox{0.5\textwidth}{!}{
    \begin{tabular}{lllll}
    \hline
         & \# of users &\# of items & \# of interactions & Sparsity\\
         \hline
        \ML-100K & 943 & 1,682 & 100,000& 93.685\%\\
        \ML-1M   & 6,040 & 3,706 & 1,000,209 & 95.530\% \\
        \NetF & 90,137 & 3,560 & 4,188,098 & 98.700\% \\
        \kasandr&25,848&1,513,038&9,489,273&99.976\%\\
        \hline
    \end{tabular}
    }
\end{table}

\subsection{Experimental set-up}
\begin{figure*}[!htpb]
    \centering
    \subfloat[{\ML-100K}]{
    \includegraphics[width=0.4\textwidth]{PlotMap_Alpha_ML100K}
    }
        \subfloat[{\ML-1M}]{
    \includegraphics[width=0.4\textwidth]{PlotMap_Alpha_ML1M}
    }\\
       \subfloat[{\kasandr}]{
    \includegraphics[width=0.4\textwidth]{PlotMap_Alpha_KASANDR}
    }
     \subfloat[{\NetF}]{
    \includegraphics[width=0.4\textwidth]{PlotMap_Alpha_NETFLIX}
    }

    \vspace{-1mm}

    \caption{MAP@1, MAP@5, MAP@10 as a function of the value of $\alpha$ for {\ML}-1M, {\ML}-100K and {\kasandr}.}

    \label{fig:alpha_impact}
\end{figure*}
\subsubsection*{Compared baselines}
In order to validate the framework defined in the previous section, we propose to compare the following approaches.

\begin{itemize}
%\item {\MostPop} recommends most popular items to all users i.e., items clicked or highly rated by all other users.
\item {\BPR} \cite{rendle_09} provides an optimization criterion based on implicit feedback; which is the maximum posterior estimator derived from a Bayesian analysis of the pairwise ranking problem, and proposes an algorithm based on Stochastic Gradient Descent to optimize it. The model can further be extended to the explicit feedback case.
\item {\CoFactor} \cite{liang_16}, developed for implicit feedback, constraints the objective of matrix factorization to use jointly item representations with a factorized shifted positive pointwise mutual information matrix of item
co-occurrence counts. The model was found to outperform WMF \cite{Hu:2008} also proposed for implicit feedback.
\item {\LightFM} \cite{kula_15} was first proposed to deal with the problem of cold-start using meta information. As with our approach, it relies on learning the embedding of users and items with the Skip-gram model and optimizes the cross-entropy loss.
\item {\RecNet}$_p$ focuses on the quality of the latent representation of users and items by learning the preference and the representation through the ranking loss $\Loss_p$ (Eq. \ref{eq:embedding_loss}).
\item {\RecNet}$_c$ focuses on the accuracy of the score obtained at the output of the framework and therefore learns the preference and the representation through the ranking loss $\Loss_c$ (Eq. \ref{eq:ranking_loss}).
\item {\RecNet}$_{c,p}$ uses a linear combination of $\Loss_p$ and $\Loss_c$ as the objective function, with $\alpha\in ]0,1[$. We study the two situations presented before (w.r.t. the presence/absence of a supplementary weighting hyper-parameter).
\end{itemize}
%\paragraph*{\textbf{Evaluation protocol}}\mbox{}\\


\begin{table*}[!htpb]
\centering
\caption{Best parameters for {\RecNet}$_p$, {\RecNet}$_c$ and {\RecNet}$_{c,p}$ when prediction is done on only shown offers; $k$ denotes the dimension of embeddings, $\lambda$ the regularization parameter. We also report the number of hidden units per layer. } \vspace{-2mm}
\label{tab:param}
\resizebox{\textwidth}{!}{\begin{tabular}{|c|ccc|ccc|ccc|ccc|}
\hline
                 & \multicolumn{3}{c|}{\ML-100K} & \multicolumn{3}{c|}{\ML-1M} & \multicolumn{3}{c|}{\NetF}& \multicolumn{3}{c|}{\kasandr}\\ \hline
                 &  {\RecNet}$_c$  & {\RecNet}$_p$&{\RecNet}$_{c,p}$ &{\RecNet}$_c$   &{\RecNet}$_p$ &{\RecNet}$_{c,p}$ &{\RecNet}$_c$   &{\RecNet}$_p$ &{\RecNet}$_{c,p}$&{\RecNet}$_c$   &{\RecNet}$_p$ &{\RecNet}$_{c,p}$  \\ \hline
$k$            &$1$&$2$&$2$&$16$&$1$&$1$&$9$&$2$&$6$ &$19$&$1$&$18$    \\
$\lambda$      &$0.05$&$0.005$&$0.005$&$0.05$&$0.0001$&$0.001$&$0.05$&$0.01$&$0.05$&$0.0001$&$0.05$&$0.005$   \\
\# units &$32$&$64$&$16$&$32$&$16$&$32$&$64$&$16$& $16$ &$64$&$16$&$64$      \\ \hline
%\multicolumn{1}{|c|}{\# hidden layers}      &$1$&$1$&$1$&$1$&$1$&$1$&$1$&$1$&$1$     \\ \hline
%\multicolumn{1}{|c|}{$\eta$}          &$1e-3$&$1e-3$&$1e-3$&$1e-3$&$1e-3$&$1e-3$&$1e-3$&$1e-3$&$1e-3$    \\ \hline
\end{tabular}
}
\end{table*}

\begin{table*}[!htpb]
\centering
\caption{Best parameters for {\RecNet}$_p$, {\RecNet}$_c$ and {\RecNet}$_{c,p}$ when prediction is done on all offers; $k$ denotes the dimension of embeddings, $\lambda$ the regularization parameter. We also report the number of hidden units per layer.}\vspace{-2mm}
\label{tab:param_all}
\resizebox{\textwidth}{!}{\begin{tabular}{|c|ccc|ccc|ccc|ccc|}
\hline
                 & \multicolumn{3}{c|}{\ML-100K} & \multicolumn{3}{c|}{\ML-1M} & \multicolumn{3}{c|}{\NetF}& \multicolumn{3}{c|}{\kasandr}\\ \hline
                 &  {\RecNet}$_c$  & {\RecNet}$_p$&{\RecNet}$_{c,p}$ &{\RecNet}$_c$   &{\RecNet}$_p$ &{\RecNet}$_{c,p}$ &{\RecNet}$_c$   &{\RecNet}$_p$ &{\RecNet}$_{c,p}$&{\RecNet}$_c$   &{\RecNet}$_p$ &{\RecNet}$_{c,p}$  \\ \hline
$k$            &$15$&$5$&$8$&$2$&$11$&$2$&$3$&$13$&$1$ &$4$&$16$&$14$    \\
$\lambda$      &$0.001$&$0.001$&$0.001$&$0.05$&$0.0001$&$0.001$&$0.0001$&$0.001$&$0.001$&$0.001$&$0.0001$&$0.05$   \\
\# units &$32$&$16$&$16$&$32$&$64$&$32$&$32$&$64$& $64$ &$32$&$64$&$64$      \\ \hline
%\multicolumn{1}{|c|}{\# hidden layers}      &$1$&$1$&$1$&$1$&$1$&$1$&$1$&$1$&$1$     \\ \hline
%\multicolumn{1}{|c|}{$\eta$}          &$1e-3$&$1e-3$&$1e-3$&$1e-3$&$1e-3$&$1e-3$&$1e-3$&$1e-3$&$1e-3$    \\ \hline
\end{tabular}
}
\end{table*}

\subsubsection*{Evaluation protocol}
For each dataset, we sort the interactions according to time, and take 80\% for training the model and the remaining 20\% for testing it. In addition, we remove all users and offers which do not occur during the training phase.
We study two different scenarios for the prediction phase: (1) for a given user, the prediction is done only on the items that were shown to him or her; (2) the prediction is done over the set of all items, regardless of any knowledge about previous interactions. In the context of movie recommendation, a shown item is defined as a movie for which the given user provided a rating. For {\kasandr}, the definition is quite straight-forward as the data were collected from an on-line advertising platform, where the items are displayed to the users, who can either click or ignore them.

\smallskip

The first setting is arguably the most common in academic research, but is abstracted from the real-world problem as at the time of making the recommendation, the notion of shown items is not available, therefore forcing the RS to  consider the set of all items as potential candidates. As a result, in this setting, for \ML-100K, \ML-1M,  \kasandr\ and  \NetF,\ we only consider in average 25, 72, 6 and 8 items for prediction per user. The goal of the second setting is to reflect this real-world scenario, and we can expect lower results than in the first setting as the size of the search space of items increases considerably. To summarize, predicting only among the items that were shown to user evaluates the model's capability of retrieving highly rated items among the shown ones, while predicting among all items measures the performance of the model on the basis of its ability to recommend offers which user would like to engage in. %For \ML-100K, \ML-1M,  \kasandr\ and  \NetF\  we consider 25.343, 72.002, 6.006 and 8.734 items respectively on an average for prediction in the setting of shown offers.

\smallskip

All comparisons are done based on a common ranking metric, namely the Mean Average Precision (MAP). First, let us recall that the Average Precision (AP$@\ell$) is defined over the precision, $Pr$ (fraction of recommended items clicked by the user), at rank $\ell$.
$$
\text{AP}@\ell=\frac{1}{\ell}\sum_{j=1}^\ell r_{j} Pr(j),
$$
where the relevance judgments, $r_j$, are binary (i.e. equal to $1$ when the item is clicked or preferred, and 0 otherwise). Then, the mean of these AP's across all users is the MAP. In the following results, we report MAP at different rank $\ell= 1$ and $10$.

\subsubsection*{Hyper-parameters tuning}

\iffalse
\begin{figure*}[!h]
    \centering
        \subfloat[{\ML-100K}]{
    \includegraphics[width=0.325\textwidth]{PlotMap_Batches_ML100K}
    }
    \subfloat[{\ML-1M}]{
    \includegraphics[width=0.325\textwidth]{PlotMap_Batches_ML1M}
    }
        \subfloat[{\kasandr}]{
    \includegraphics[width=0.325\textwidth]{PlotMap_Batches_Kasandr}
    }

    \caption{MAP@1 as a function of the number of batches for {\ML}-1M, {\ML}-100K and {\kasandr}.}

    \label{fig:map_batch}
\end{figure*}
\fi


First, we provide a detailed study of the impact of the different hyper-parameters involved in the proposed framework $\RecNet_.$. For all datasets, hyper-parameters tuning is done on a separate validation set.
\begin{itemize}
    \item The size of the embedding is chosen among $k \in \{1,\ldots,20\}$. The impact of $k$ on the performance is presented in Figure \ref{fig:emb_dimension}.
    \item We use $\ell_2$ regularization on the embeddings and choose $\lambda\in\{0.0001,0.001,0.005,0.01,0.05\}$.
    \item We run {\RecNet} with 1 hidden layer with relu activation functions, where the number of hidden units is chosen in $\{16,32,64\}$.
   % Number of hidden layers is set to 1 and the number of hidden units varied in the set of $\{16,32,64\}$. We use the $relu$ activation functions for the hidden layer. At the output layer, we use linear activation.
    \item In order to train $\RecNet$, we use ADAM \cite{KingmaB14} and found the learning rate $\eta=1e-3$ to be more efficient for all our settings.
    For other parameters involved in Adam, i.e., the exponential decay rates for the moment estimates, we keep the default values ($\beta_1=0.9$, $\beta_{2}=0.999$ and $\epsilon=10^{-8}$).
    %Training method: We set training method to be $ADAM$ \cite{KingmaB14} as it has been found to be more efficient than Stochastic Gradient Descent, Momentum and RMSProp.  For the optimization of the different ranking losses, we use Adam \cite{DBLP:journals/corr/KingmaB14}. We tried $learning\_rate \in \{0.05, 1e^{-2}, 1e^{-2}, 1e^{-4}\}$. Finally, we found $learning\_rate = 1e-3$ to be working best for our settings. In this work, we do not explore the methods which employ decaying $\eta$ (methods such as $adagrad$) for training. For other parameters involved in Adam, i.e., the exponential decay rates for the moment estimates, we keep the default values ($\beta_1=0.9$, $\beta_{2}=0.999$ and $\epsilon=10^{-8}$).
    %\item \# of Epochs: We fix the number of epochs to be $T=10,000$
    \item Finally, we fix the number of epochs to be $T=10,000$ in advance and the size of mini-batches to $n=512$.
    %\item Size of minibatches: The size of the mini-batches is set to $n=512$
    %\item Regularization: We use L2-regularization \cmmnt{and early stopping}to avoid over-fitting (as shown in Figure \ref{fig:map_batch} (a)).
    %\item Effect of $\alpha$ on training: \cmmnt{Earlier,  Taking this into consideration, we conducted following experiments on ML-100K. In particular, we had computed embedding, ranking, and target (vs. number of epochs) losses. In particular, starting with $\alpha$ = 0 and gradually increasing it in increments in 0.1. We also studied the effect of decreasing $\alpha$. }Performance of \RecNet gets affected when run with decaying or boosting $\alpha$ in terms of MAP when compared to fixed $\alpha$, which was contrary to our expectations. We expected embedding loss to converge first after few epochs and ranking loss to converge after considerably more number of epochs.
%    \item For {\RecNet}$_{c,p}$, we set $\alpha$ (Eq. \ref{eq:rankingLoss}) to $0.5$ as mentioned above.
   \item One can see that all three versions of $\RecNet$ perform the best with a quite small number of hidden units, only one hidden layer and a low dimension for the representation. As a consequence, they involve a few number of parameters to tune while training.\cmmnt{ and present an interesting computational complexity compared to other state-of-the-art approaches.}
%This is due to the fact that for each user, we have 1 weight associated/activated in the embeddings.
   \item In terms of the ability to recover a relevant ranked list of items for each user, we also tune the hyper-parameter $\alpha$ (Eq. \ref{eq:rankingLoss_alpha}) which balances the weight given to the two terms in $\RecNet_{c,p}$. These results are shown in Figure \ref{fig:alpha_impact}, where the values of $\alpha$ are taken in the interval $[0,1]$. While it seems to play a significant role on {\ML}-100K and {\kasandr}, we can see that for {\ML}-1M the results in terms of MAP are stable, regardless the value of $\alpha$.
\end{itemize}

From Figure \ref{fig:emb_dimension}, when prediction is done on the interacted offers, it is clear that best MAP@1 results are generally obtained with small sizes of item and user embedded vector spaces  $k$.\cmmnt{, which are the same as the size of the feature vector space.} These empirical results support our theoretical analysis where we found that small $k$ induces smaller generalization bounds. This observation on the dimension of embedding is also in agreement with the conclusion of \cite{kula_15}, which uses the same technique for representation learning. For instance, one can see that on {\ML}-1M, the highest MAP is achieved with a dimension of embedding equals to $1$.  Since in the interacted offers setting, the prediction is done among the very few shown offers, \RecNet\ makes non-personalized recommendations. This is due to the fact that \cmmnt{In the case of, {\RecNet}$_p$}having $k=1$ means that the recommendations for a given user with a positive (negative) value is done by sorting the positive (negative) items according to their learned embeddings, and in some sense, can therefore be seen as a bi-polar popularity model. This means that in such cases popularity and non-personalized based approaches are perhaps the best way to make recommendations.  For reproducibility purpose, we report the best combination of parameters for each variant of {\RecNet} in Table \ref{tab:param} and Table \ref{tab:param_all}.



%\textcolor{red}{Charlotte: TBC (check popularity bias on datasets where k=1 is chosen)} \textcolor{green}{Added in Table\ref{tab: popBiasInteracted}
%Results are lower than RecNet. Perhaps, RecNet gives non-personalized but still better results}
\iffalse
\begin{table}[]
\centering
\caption{popularity bias on datasets where k=1 is chosen}
\label{tab: popBiasInteracted}
\begin{tabular}{l|c|l|c|l|}
\cline{2-5}
                                 & \multicolumn{2}{c|}{ML-100K}                            & \multicolumn{2}{c|}{ML-1M}                              \\ \cline{2-5}
                                 & \multicolumn{1}{l|}{MAP@1} & MAP@10                     & \multicolumn{1}{l|}{MAP@1} & MAP@10                     \\ \hline
\multicolumn{1}{|l|}{Popularity} & 0.594                      & \multicolumn{1}{c|}{0.659} & 0.646                      & \multicolumn{1}{c|}{0.657} \\ \hline
\end{tabular}
\end{table}
\fi
%This is a bi-polar popularity model, and therefore unlikely to outperform personalized models.
%While this could be surprising, we can see two plausible explanations to the phenomenon.

%it is mainly due to the fact that we used a one-hot code input representation of both users and items, meaning that only one value is equal to one, and therefore, it seems quite reasonable to associate one weight to each user and each item in this context. This conclusion is also in line with the generalization bound that we have presented previously.


\iffalse
Lastly, to train {\RecNet$_.$}, we fix the number of epochs to $T=10,000$ and the size of the mini-batches to $n=512$. To further avoid over-fitting (as shown in Figure \ref{fig:map_batch} (a)), we also use early-stopping. For the optimization of the different ranking losses, we use Adam \cite{DBLP:journals/corr/KingmaB14} and the learning rate $\eta$ is set to 1e-3 using a validation set. For other parameters involved in Adam, i.e., the exponential decay rates for the moment estimates, we keep the default values ($\beta_1=0.9$, $\beta_{2}=0.999$ and $\epsilon=10^{-8}$).
\fi

\begin{table*}[!htpb]
\centering
\caption{Results of all state-of-the-art approaches for implicit feedback when prediction is done only on offers shown to users. The best result is in bold, and a $\DA$ indicates a result that is statistically significantly worse than the best, according to a Wilcoxon rank sum test with $p < .01$.}\vspace{-2mm}
\label{tab:results_warm_interacted}
\resizebox{0.9\textwidth}{!}{\begin{tabular}{|c|cc|cc|cc|cc|}
\hline
                   & \multicolumn{2}{c|}{\ML-100K} & \multicolumn{2}{c|}{\ML-1M} & \multicolumn{2}{c|}{\NetF}& \multicolumn{2}{c|}{\kasandr}\\ \hline
            & MAP@1    & MAP@10 &MAP@1   & MAP@10&MAP@1   &MAP@10 & MAP@1   &  MAP@10 \\ \hline
%MostPop     & $0.633\DA$  &   $0.553\DA$  &$0.571\DA$&$0.553\DA$&$0.471\DA$&$0.480\DA$ &$0.037\DA$&$0.0383\DA$ \\
BPR-MF      & $0.613\DA$  &  $0.608\DA$   &$0.788\DA$&$0.748\DA$&$\textbf{0.909}$&$0.842\DA$ &$0.857\DA$&$0.857\DA$     \\
LightFM        &     $0.772\DA$	    &    $0.770\DA$   &$0.832\DA$&$0.795\DA$&$0.800\DA$& $0.793\DA$ &$0.937\DA$&$0.936\DA$  \\
CoFactor          &      $0.718\DA$   &  $0.716\DA$        &$0.783\DA$&$0.741\DA$&$0.693\DA$&$0.705\DA$ &$0.925\DA$&$0.918\DA$\\
{\RecNet}$_c$      & $\textbf{0.894}$    & $\textbf{0.848}$  &$0.877\DA$&$0.835$&$0.880\DA$&$\textbf{0.847}$&$0.958\DA$&$0.963\DA$  \\
{\RecNet}$_p$      &   $0.881\DA$  &   $0.846$  &$0.876\DA$&$\textbf{0.839}$&$0.875\DA$&$0.844$ &$0.915\DA$&$0.923\DA$      \\
{\RecNet}$_{c,p}$     &  $0.888\DA$   & $0.842$       &$\textbf{0.884}$&$\textbf{0.839}$&$0.879\DA$&$\textbf{0.847}$ &$\mathbf{0.970}$&$\textbf{0.973}$   \\ \hline
\end{tabular}
}
\end{table*}
\begin{table*}[!htpb]
\centering
\caption{Results of all state-of-the-art approaches for recommendation on all implicit feedback data sets when prediction is done on all offers. The best result is in bold, and a $\DA$ indicates a result that is statistically significantly worse than the best, according to a Wilcoxon rank sum test with $p < .01$}
\label{tab:results_warm_all}
\resizebox{0.9\textwidth}{!}{\begin{tabular}{|c|cc|cc|cc|cc|}
\hline
                   & \multicolumn{2}{c|}{\ML-100K} & \multicolumn{2}{c|}{\ML-1M} & \multicolumn{2}{c|}{\NetF}& \multicolumn{2}{c|}{\kasandr}\\ \hline
                & MAP@1     & MAP@10 &MAP@1   &  MAP@10&MAP@1   &MAP@10 & MAP@1  & MAP@10 \\ \hline
BPR-MF          &  $0.140\DA$      &  $\textbf{0.261}$      &$0.048\DA$&$0.097\DA$&$0.035\DA$& $0.072\DA$&$0.016\DA$&$0.024\DA$     \\
LightFM         &  $0.144\DA$  & $0.173\DA$     &$0.028\DA$&$0.096\DA$&$0.006\DA$& $0.032\DA$ &$0.002\DA$&$0.003\DA$  \\
CoFactor          &   $ 0.056\DA$     &      $0.031\DA$      &$0.089\DA$&$0.033\DA$&$0.049\DA$&$0.030\DA$&$0.002\DA$&$0.001\DA$\\
{\RecNet}$_c$      &$0.106\DA$       & $0.137 \DA$  &$0.067\DA$&$0.093\DA$&$0.032\DA$&$0.048\DA$ &$0.049\DA$&$0.059\DA$  \\
{\RecNet}$_p$      &  $\textbf{0.239}$   &$0.249$     &$\textbf{0.209}$&$\textbf{0.220}$&$\textbf{0.080}$&$\textbf{0.089}$ &$0.100\DA$&$0.100\DA$     \\
{\RecNet}$_{c,p}$      &   $0.111\DA$ &      $0.134\DA$    &$0.098\DA$&$0.119\DA$&$0.066\DA$&$0.087$ &$\textbf{0.269}$&$\textbf{0.284}$  \\ \hline

%\multicolumn{1}{|c|}{Popularity}        &         &      &          &&&&&& &&& \\ \hline
\end{tabular}
}
\end{table*}
\subsection{Results}

Hereafter, we compare and summarize the performance of {\RecNet}$_.$ with the baseline methods on various data sets. Empirically, we observed that the version of $\RecNet_{c,p}$ where both $\Loss_c$ and $\Loss_p$ have an equal weight while training gives better results on average, and we decided to only report these results  later.

\medskip
%\subsubsection*{Test-Bench}

Tables \ref{tab:results_warm_interacted} and \ref{tab:results_warm_all} report all results. In addition, in each case, we statistically compare the performance of each algorithm, and we use bold face to indicate the highest performance, and the symbol $\DA$ indicates that performance is significantly worst than the best result, according to a Wilcoxon rank sum test used at a p-value threshold of $0.01$ \cite{lehmann_06}.

%Empirically, we observed that the version of $\RecNet_{c,p}$ where both $\Loss_c$ and $\Loss_p$ have an equal weight while training gives better results in average, and we decided to only report this latter in Table \ref{tab:results_warm_interacted} and \ref{tab:results_warm_all}. %This is the setting where both $\Loss_c$ and $\Loss_p$ have an equal weight while training $\RecNet_{c,p}$
%A $\uparrow$ indicates that the corresponding model is statistically significantly worse than the first ranked function. Whereas a ↓ indicates the opposite.

\subsubsection*{Setting 1 : interacted items}
When the prediction is done over offers which user interacted with (Table \ref{tab:results_warm_interacted}), the {\RecNet} architecture, regardless the weight given to $\alpha$, beats all the other algorithms on {\kasandr}, {\ML}-100K and {\ML}-1M. However, on {\NetF}, BPR-MF outperforms our approach in terms of MAP@1. This may be owing to the fact that the binarized \NetF\ movie data set is strongly biased towards the popular movies and usually, the majority of users have watched one or the other popular movies in such data sets and rated them well. In {\NetF}, around $75\%$ of the users have given ratings greater to 4 to the top-10 movies. We believe that this phenomenon adversely affects the performance of {\RecNet}. However, on {\kasandr}, which is the only true implicit dataset {\RecNet} significantly outperforms all other approaches.

\subsubsection*{Setting 2 : all items}
When the prediction is done over all offers (Table \ref{tab:results_warm_all}), we can make two observations. First, all the algorithms encounters an extreme drop of their performance in terms of MAP. Second, {\RecNet} framework significantly outperforms all other algorithms on all datasets, and this difference is all the more important on {\kasandr}, where for instance {\RecNet$_{c,p}$} is in average 15 times more efficient. We believe, that our model is a fresh departure from the models which learn pairwise ranking function without the knowledge of embeddings or which learn embeddings without learning any pairwise ranking function. While learning pairwise ranking function, our model is aware of the learned embeddings so far and vice-versa. We demonstrate that the simultaneous learning of two ranking functions helps in learning hidden features of implicit data and improves the performance of {\RecNet}.

\subsubsection*{Comparison between {\RecNet} versions}
One can note that while optimizing ranking losses by Eq. \ref{eq:rankingLoss} or Eq. \ref{eq:ranking_loss} or Eq. \ref{eq:embedding_loss}, we simultaneously learn representation and preference function; the main difference is the amount of emphasis we put in learning one or another. The results presented in both tables tend to demonstrate that, in almost all cases, optimizing the linear combination of the pairwise-ranking loss and the embedding loss ({\RecNet}$_{c,p}$) indeed increases the quality of overall recommendations than optimizing standalone losses to learn embeddings and pairwise preference function. For instance, when the prediction is done over offers which user interacted with (Table \ref{tab:results_warm_interacted}), ({\RecNet}$_{c,p}$) outperforms ({\RecNet}$_{p}$) and ({\RecNet}$_{c}$) on \ML-1M, \kasandr\ and \NetF. When prediction is done on all offers (Table \ref{tab:results_warm_all}), ({\RecNet}$_{c,p}$) outperforms ({\RecNet}$_{p}$) and ({\RecNet}$_{c}$) on \kasandr. Thus, in case of interacted offers setting, optimizing ranking and embedding loss simultaneously boosts performance on all datasets. However, in the setting of  all offers, optimizing both losses simultaneously is beneficial in case of true implicit feedback datasets such as \kasandr (recall that all other datasets were synthetically made implicit). \cmmnt{In the setting of predicting over all offers, learning a good representation/embeddings alone and ignoring preference loss function seems to be doing the trick for a good performance for such synthetically modified data sets.}
%\begin{itemize}
%\item As {\RecNet} is meant to handle implicit feedback, we compare with the baselines which were developed for the same purpose. When the prediction is done over all offers (Table \ref{tab:results_warm_all}), {\RecNet} outperforms all other algorithms on all datasets. When the prediction is done over offers which user interacted with (Table \ref{tab:results_warm_interacted}), {\RecNet} beats all the other algorithms on {\kasandr}, {\ML}-100K and {\ML}-1M. Our model is a fresh departure from the models which learn pairwise ranking function without the knowledge of embeddings or which learn embeddings without learning any pairwise ranking function. While learning pairwise ranking function, our model is aware of the learned embeddings so far and vice-versa. This simultaneous learning of two ranking functions helps in learning hidden features of implicit data and improves the performance of {\RecNet}. We also found out that \BPR\ manages to outperform the performance of {\RecNet} when the prediction is done on interacted movies (Table \ref{tab:results_warm_interacted}). This may be owing to the fact that binarized \NetF\ movie data set is strongly biased towards the popular movies and usually, the majority of users have watched one or the other popular movies in such data sets and rated them well. In {\NetF}, around $75\%$ of the users have given ratings greater to 4 to the top-10 movies.
%For example, in {\ML}-100K, more than $90\%$ of the users gave ratings greater than 4 to the top-10 movies. The same phenomenon occurs with Netflix, in around $75\%$ of the users, also gave ratings greater than rated top-10  popular movies.

%But, results on {\kasandr}, when prediction is done on all items and all datasets, when prediction is done only on interacted items exhibits the capability of {\RecNet} to perform better than other existing models for implicit feedback.
%\item One can note that while optimizing ranking losses by Eq. \ref{eq:rankingLoss} or Eq. \ref{eq:ranking_loss} or Eq. \ref{eq:embedding_loss}, we simultaneously learn representation and preference function; the main difference is the amount of emphasis we put in learning one or another. The results presented in both tables tend to demonstrate that, in almost all cases, optimizing the linear combination of the pairwise-ranking loss and the embedding loss ({\RecNet}$_{c,p}$) indeed increases the quality of overall recommendations than optimizing standalone losses to learn embeddings and pairwise preference function. For instance, when the prediction is done over offers which user interacted with (Table \ref{tab:results_warm_interacted}), ({\RecNet}$_{c,p}$) outperforms ({\RecNet}$_{p}$) and ({\RecNet}$_{c}$) on \ML-1M, \kasandr and \NetF. When prediction is done on all offers (Table \ref{tab:results_warm_all}), ({\RecNet}$_{c,p}$) outperforms ({\RecNet}$_{p}$) and ({\RecNet}$_{c}$) on \kasandr. Thus, in case of interacted offers setting, optimizing ranking and embedding loss simultaneously boosts performance on all datasets. However, in the setting of  all offers, optimizing both losses simultaneously is beneficial in case of true implicit feedback datasets such as \kasandr (recall that all other datasets were synthetically made implicit). In the setting predicting over all offers, learning a good representation/embeddings alone and ignoring preference loss function seems to be doing the trick for a good performance for such synthetically modified data sets.
%Either, we put more importance on the embedding or on preference function or on both.
%In order to show that optimizing the linear combination of pairwise-ranking loss and embeddings loss ({\RecNet}$_{c,p}$) indeed increases the quality of overall recommendations than when either ({\RecNet}$_c$ and {\RecNet}$_p$) is learned without the knowledge of the other, we test the performance of ({\RecNet}) by testing with values of $\alpha=0,0.5,1$. $\alpha=0$ corresponds to the case of ({\RecNet}$_c$), while $\alpha=1$ corresponds to the case of ({\RecNet}$_p$) and finally $\alpha=0.5$ corresponds to the case of ({\RecNet}$_{c,p}$). Barring the case of {\ML}-1M on all offers, in all cases ({\RecNet}$_{c,p}$) performs better than both {\RecNet}$_c$ and {\RecNet}$_p$. This re-instates our hypothesis that simultaneous optimization of embeddings and pair-wise ranking losses is better than standalone losses to learn embeddings and pairwise preference function. On {\ML}-1M, when testing the model on all offers, taking into account pair-wise ranking loss can actually hurt the performance of learned embeddings. We also found out that just putting more weight on pair-wise ranking alone in our framework performs better than putting more weight on embedding loss as also shown in Figure \ref{fig:emb_dimension}.
%\end{itemize}


\section{Conclusion}\label{sec:conclusion}

We presented and analyzed a learning to rank framework for recommender systems which consists of learning user preferences over items. We showed that the minimization of pairwise ranking loss over user preferences involves dependent random variables and provided a theoretical analysis by proving the consistency of the empirical risk minimization in the worst case where all users choose a minimal number of positive and negative items. From this analysis we then proposed {\RecNet}, a new neural-network based model for learning the user preference, where both the user's and item's representations and the function modeling the user's preference over pairs of items are learned simultaneously. The learning phase is guided using a ranking objective that can capture the ranking ability of the prediction function as well as the expressiveness of the learned embedded space, where the preference of users over items is respected by the dot product function defined over that space. The training of {\RecNet} is carried out using the back-propagation algorithm in mini-batches defined over a user-item matrix containing implicit information in the form of subsets of preferred and non-preferred items. The learning capability of the model over both prediction and representation problems show their interconnection and also that the proposed double ranking objective allows to conjugate them well.  We assessed and validated the proposed approach through extensive experiments, using four popular collections proposed for the task of recommendation. Furthermore, we propose to study two different settings for the prediction phase and demonstrate that the performance of each approach is strongly impacted by the set of items considered for making the prediction.

\medskip

For future work, we would like to extend {\RecNet} in order to take into account additional contextual information regarding users and/or items. More specifically, we are interested in the integration of data of different natures, such as text or demographic information. We believe that this information can be taken into account without much effort and by doing so, it is possible to improve the performance of our approach and tackle the problem of providing recommendation for new users/items at the same time, also known as the cold-start problem. The second important extension will be the development of an on-line version of the proposed algorithm in order to make the approach suitable for real-time applications and on-line advertising. Finally, we have shown that choosing a suitable $\alpha$, which controls the the trade-off between ranking and embedding loss, greatly impact the performance of the proposed framework, and we believe that an interesting extension will be to learn automatically this hyper-parameter, and to make it adaptive during the training phase.

%we would first like to first learn representations of user and item and then user's preference over items. In other words, we would like to optimize and vary $\alpha$ (the trade-off between ranking and embedding loss) with number of epochs.

\iffalse

\section*{Appendix}
\begin{theorem}
	\label{thm:WorseCaseRecNet}
        Let $\userS$ be a set of $M$ independent users, such that each user $u \in \userS $ prefers $n_*^+$ items over $n_*^-$ ones in a predefined set of $\itemS$ items. Let $S=\{(\bfZ_{i,u,i'}\doteq(i,u,i'),y_{i,u,i'})\mid u\in\userS, (i,i')\in\itemS^+_u\times \itemS^-_u\}$ be the associated training set, then for any $1>\delta>0$ the following generalization bound holds for all $f\in  \mathcal{F}_{B,r}$ with probability at least $1-\delta$:
        \begin{align*}
     \Loss(f)\le ~~&\hat\Loss_*(f,S) + \frac{2B\mathfrak{C}(S)}{Nn^+_*}+\\ &\frac{5}{2}\left(\sqrt{\frac{2B\mathfrak{C}(S)}{Nn^+_*}}+\sqrt{\frac{r}{2}}\right)\sqrt{\frac{\log\frac{1}{\delta}}{n_*^+}}+\frac{25}{48}\frac{\log\frac{1}{\delta}}{n_*^+},
     \end{align*}
    where $\mathfrak{C}(S)=\sqrt{ \frac{1}{n^-_*}\sum_{j=1}^{n_*^-}\mathbb{E}_{\Cset_j}\left[  \sum_{\alpha\in\Cset_j \atop \bfZ_\alpha \in S}d(\bfZ_\alpha,\bfZ_{\alpha}))\right]}$, $\bfZ_\alpha=(i_\alpha,u_\alpha,i'_\alpha)$ and \\ \begin{align*}
    d(\bfZ_\alpha,\bfZ_{\alpha})=~&\kappa(\Phi(u_\alpha,i_\alpha),\Phi(u_\alpha,i_\alpha))\\+\kappa(\Phi(u_\alpha,i'_\alpha),\Phi(u_\alpha,i'_\alpha))-
    &2\kappa(\Phi(u_\alpha,i_\alpha),\Phi(u_\alpha,i'_\alpha)).
    \end{align*}
\end{theorem}

\begin{IEEEproof}
As the set of users $\userS$ is supposed to be independent, the exact fractional cover of the dependency graph corresponding to the training set $S$ will be the union of the exact fractional cover associated to each user such that cover sets which do not contain any items in common are joined together.

Following \cite[Proposition 4]{RalaiAmin15}, for any $1>\delta>0$ we have with probability at least $1-\delta$:

\begin{equation*}
\small
\begin{split}
	&\mathbb{E}_S[\hat\Loss_*(f,S)]-\hat\Loss_*(f,S)\\
	&\leq \inf_{\beta>0}\left( (1+\beta)\rademacher_{S}(\mathcal{F}_{B,r})+\frac{5}{4}\sqrt{\frac{2r\log\frac{1}{\delta}}{n_*^+}}+\frac{25}{16}\left(\frac{1}{3}+\frac{1}{\beta}\right)\frac{\log \frac{1}{\delta}}{n_*^+}\right)
	\end{split}
\end{equation*}

The infimum is reached for $\beta^*=\sqrt{\frac{25}{16}\frac{\log \frac{1}{\delta}}{n_*^+\times \rademacher_{S}(\mathcal{F}_{B,r})}}$ which by plugging it back into the upper-bound, and from equation \eqref{eq:EmpRisk2}, gives:
        \begin{equation}
        \resizebox{0.5 \textwidth}{!}{$
            \label{eq:UpperBound}
     \Loss(f) \le \hat\Loss_*(f,S) + \rademacher_{S}(\mathcal{F}_{B,r})+\frac{5}{2}\left(\sqrt{\rademacher_{S}(\mathcal{F}_{B,r})}+\sqrt{\frac{r}{2}}\right)\sqrt{\frac{\log\frac{1}{\delta}}{n_*^+}}+\frac{25}{48}\frac{\log\frac{1}{\delta}}{n_*^+}.
     $}
        \end{equation}

Now, for all $j\in\{1,\ldots,J\}$ and $\alpha \in \mathcal M_j$, let $(u_\alpha,i_\alpha)$ and $(u_\alpha,i'_\alpha)$ be the first and the second pair constructed from $\bfZ_\alpha$, then from the bilinearity of dot product and the Cauchy-Schwartz inequality, $\rademacher_{S}(\mathcal{F}_{B,r})$ is upper-bounded by:
\begin{align}
&\frac{2}{m}\mathbb{E}_{\xi}\sum_{j=1}^{n_*^-}\mathbb{E}_{\Cset_j}\sup_{f\in\mathcal{F}_{B,r}} \left\langle \boldsymbol{w},\sum_{\alpha\in\Cset_j \atop \bfZ_\alpha \in S}\xi_\alpha\left( \Phi(u_\alpha,i_\alpha)-\Phi(u_\alpha,i'_\alpha)\right)\right\rangle \nonumber\\
     & \le  \frac{2B}{m}\sum_{j=1}^{n_*^-}\mathbb{E}_{\Cset_j} \mathbb{E}_{\xi}\left\|  \sum_{\alpha\in\Cset_j \atop \bfZ_\alpha \in S}\xi_\alpha(\Phi(u_\alpha,i_\alpha)-\Phi(u_\alpha,i'_\alpha))\right\| \nonumber \\
     & \le \frac{2B}{m}\sum_{j=1}^{n_*^-}\left(\mathbb{E}_{\Cset_j \xi}\left[  \sum_{\alpha,\alpha'\in\Cset_j \atop \bfZ_\alpha,\bfZ_{\alpha'} \in S}\xi_\alpha\xi_{\alpha'}d(\bfZ_\alpha,\bfZ_{\alpha'}))\right]\right)^{1/2},
\end{align}
where the last inequality follows from Jensen's inequality and the concavity of the square root, and
\[
d(\bfZ_\alpha,\bfZ_{\alpha'})=\left\langle \Phi(u_\alpha,i_\alpha)-\Phi(u_\alpha,i'_\alpha),\Phi(u_\alpha,i_\alpha)-\Phi(u_\alpha,i'_\alpha)\right\rangle.
\]
Further, for all $j\in\{1,\ldots,n^-_*\}, \alpha,\alpha' \in \mathcal M_j, \alpha\neq \alpha'; $ we have $\mathbb{E}_\xi[\xi_\alpha \xi_{\alpha'}]=0$, \cite[p. 91]{Shawe-Taylor:2004:KMP:975545} so:
\begin{align*}
\rademacher_{S}(\mathcal{F}_{B,r})\le & ~\frac{2B}{m}\sum_{j=1}^{n_*^-}\left(\mathbb{E}_{\Cset_j}\left[  \sum_{\alpha\in\Cset_j \atop \bfZ_\alpha \in S}d(\bfZ_\alpha,\bfZ_{\alpha}))\right]\right)^{1/2}\\
= &~ \frac{2Bn^-_*}{m}\sum_{j=1}^{n_*^-}\frac{1}{n^-_*}\left(\mathbb{E}_{\Cset_j}\left[  \sum_{\alpha\in\Cset_j \atop \bfZ_\alpha \in S}d(\bfZ_\alpha,\bfZ_{\alpha}))\right]\right)^{1/2}.
\end{align*}
By using Jensen's inequality and the concavity of the square root once again, we finally get
\begin{equation}
\label{eq:FracChromatic}
\rademacher_{S}(\mathcal{F}_{B,r})\le
\frac{2B}{Nn^+_*}\sqrt{\sum_{j=1}^{n_*^-}\frac{1}{n^-_*}\mathbb{E}_{\Cset_j}\left[  \sum_{\alpha\in\Cset_j \atop \bfZ_\alpha \in S}d(\bfZ_\alpha,\bfZ_{\alpha}))\right]}.
\end{equation}
The result follows from equations \eqref{eq:UpperBound} and \eqref{eq:FracChromatic}.
\end{IEEEproof}

\fi

\section*{Acknowledgements}
This work was partly done under the Calypso project supported by the FEDER program from the R\'egion Auvergne-Rh\^one-Alpes. The research of Yury Maximov at LANL was supported by Center of Non Linear Studies (CNLS). 


%\bibliographystyle{IEEEtran}
%\bibliography{_tkde}
% Generated by IEEEtran.bst, version: 1.13 (2008/09/30)
\begin{thebibliography}{10}
\providecommand{\url}[1]{#1}
\csname url@samestyle\endcsname
\providecommand{\newblock}{\relax}
\providecommand{\bibinfo}[2]{#2}
\providecommand{\BIBentrySTDinterwordspacing}{\spaceskip=0pt\relax}
\providecommand{\BIBentryALTinterwordstretchfactor}{4}
\providecommand{\BIBentryALTinterwordspacing}{\spaceskip=\fontdimen2\font plus
\BIBentryALTinterwordstretchfactor\fontdimen3\font minus
  \fontdimen4\font\relax}
\providecommand{\BIBforeignlanguage}[2]{{%
\expandafter\ifx\csname l@#1\endcsname\relax
\typeout{** WARNING: IEEEtran.bst: No hyphenation pattern has been}%
\typeout{** loaded for the language `#1'. Using the pattern for}%
\typeout{** the default language instead.}%
\else
\language=\csname l@#1\endcsname
\fi
#2}}
\providecommand{\BIBdecl}{\relax}
\BIBdecl

\bibitem{Ricci:2010:RSH:1941884}
F.~Ricci, L.~Rokach, B.~Shapira, and P.~B. Kantor, \emph{Recommender Systems
  Handbook}, 1st~ed.\hskip 1em plus 0.5em minus 0.4em\relax New York, NY, USA:
  Springer-Verlag New York, Inc., 2010.

\bibitem{reference/rsh/LopsGS11}
P.~Lops, M.~de~Gemmis, and G.~Semeraro, ``Content-based recommender systems:
  State of the art and trends.'' in \emph{Recommender Systems Handbook},
  F.~Ricci, L.~Rokach, B.~Shapira, and P.~B. Kantor, Eds.\hskip 1em plus 0.5em
  minus 0.4em\relax Springer, 2011, pp. 73--105.

\bibitem{ir2004010}
R.~White, J.~M. Jose, and I.~Ruthven, ``Comparing explicit and implicit
  feedback techniques for web retrieval: {TREC-10} interactive track report,''
  in \emph{Proceedings of {TREC}}, 2001.

\bibitem{DBLP:conf/kdd/WangWY15}
H.~Wang, N.~Wang, and D.~Yeung, ``Collaborative deep learning for recommender
  systems,'' in \emph{Proceedings of {SIGKDD}}, 2015, pp. 1235--1244.

\bibitem{Usunier:1121}
N.~Usunier, M.~Amini, and P.~Gallinari, ``A data-dependent generalisation error
  bound for the {AUC},'' in \emph{ICML'05 workshop on ROC Analysis in Machine
  Learning}, Bonn, Germany, 2005.

\bibitem{Amini:15}
M.-R. Amini and N.~Usunier, \emph{Learning with Partially Labeled and
  Interdependent Data}.\hskip 1em plus 0.5em minus 0.4em\relax New York, NY,
  USA: Springer, 2015.

\bibitem{Janson04RSA}
S.~Janson, ``{Large Deviations for Sums of Partly Dependent Random
  Variables},'' \emph{Random Structures and Algorithms}, vol.~24, no.~3, pp.
  234--248, 2004.

\bibitem{UsunierAG05}
N.~Usunier, M.-R. Amini, and P.~Gallinari, ``Generalization error bounds for
  classifiers trained with interdependent data,'' in \emph{Proceedings of
  NIPS}, 2006, pp. 1369--1376.

\bibitem{mcdiarmid89method}
C.~McDiarmid, ``On the method of bounded differences,'' \emph{Survey in
  Combinatorics}, pp. 148--188, 1989.

\bibitem{RalaiAmin15}
L.~Ralaivola and M.~Amini, ``Entropy-based concentration inequalities for
  dependent variables,'' 2015, pp. 2436--2444.

\bibitem{vapnik2000nature}
V.~Vapnik, \emph{The nature of statistical learning theory}.\hskip 1em plus
  0.5em minus 0.4em\relax Springer Science \& Business Media, 2000.

\bibitem{Volkovs:2015}
M.~Volkovs and G.~W. Yu, ``Effective latent models for binary feedback in
  recommender systems,'' in \emph{Proceedings of SIGIR}, 2015, pp. 313--322.

\bibitem{rendle_09}
S.~Rendle, C.~Freudenthaler, Z.~Gantner, and L.~Schmidt{-}Thieme, ``{BPR:}
  bayesian personalized ranking from implicit feedback,'' in \emph{Proceedings
  of {UAI}}, 2009, pp. 452--461.

\bibitem{Leon2012}
L.~Bottou, ``Stochastic gradient descent tricks,'' in \emph{Neural Networks:
  Tricks of the Trade - Second Edition}, 2012, pp. 421--436.

\bibitem{Ailon08anefficient}
N.~Ailon and M.~Mohri, ``An efficient reduction of ranking to classification,''
  in \emph{Proceedings of {COLT}}, (2008), pp. 87--98.

\bibitem{word_emb}
T.~Mikolov, K.~Chen, G.~Corrado, and J.~Dean, ``Efficient estimation of word
  representations in vector space,'' \emph{CoRR}, vol. abs/1301.3781, 2013.

\bibitem{mikolov_13}
T.~Mikolov, I.~Sutskever, K.~Chen, G.~S. Corrado, and J.~Dean, ``Distributed
  representations of words and phrases and their compositionality,'' in
  \emph{Proceedings of NIPS}, 2013, pp. 3111--3119.

\bibitem{levy_14}
O.~Levy and Y.~Goldberg, ``Neural word embedding as implicit matrix
  factorization,'' in \emph{Proceedings of NIPS}, 2014, pp. 2177--2185.

\bibitem{guardia_15}
{\'{E}}.~Gu{\`{a}}rdia{-}Sebaoun, V.~Guigue, and P.~Gallinari, ``Latent
  trajectory modeling: {A} light and efficient way to introduce time in
  recommender systems,'' in \emph{Proceedings of RecSys}, 2015, pp. 281--284.

\bibitem{liang_16}
D.~Liang, J.~Altosaar, L.~Charlin, and D.~M. Blei, ``Factorization meets the
  item embedding: Regularizing matrix factorization with item co-occurrence,''
  in \emph{Proceedings of RecSys}, 2016, pp. 59--66.

\bibitem{DBLP:conf/kdd/GrbovicRDBSBS15}
M.~Grbovic, V.~Radosavljevic, N.~Djuric, N.~Bhamidipati, J.~Savla, V.~Bhagwan,
  and D.~Sharp, ``E-commerce in your inbox: Product recommendations at scale,''
  in \emph{Proceedings of {SIGKDD}}, 2015, pp. 1809--1818.

\bibitem{He:2017:NCF:3038912.3052569}
X.~He, L.~Liao, H.~Zhang, L.~Nie, X.~Hu, and T.~Chua, ``Neural collaborative
  filtering,'' in \emph{Proceedings of {WWW}}, 2017, pp. 173--182.

\bibitem{vasile_16}
F.~Vasile, E.~Smirnova, and A.~Conneau, ``Meta-prod2vec: Product embeddings
  using side-information for recommendation,'' in \emph{Proceedings of RecSys},
  2016, pp. 225--232.

\bibitem{Liu:2009}
T.~Liu, ``Learning to rank for information retrieval,'' \emph{Foundations and
  Trends in Information Retrieval}, vol.~3, no.~3, pp. 225--331, 2009.

\bibitem{Crammer01}
K.~Crammer and Y.~Singer, ``Pranking with ranking,'' in \emph{Proceedings of
  NIPS}, 2001, pp. 641--647.

\bibitem{Li08}
P.~Li, C.~J.~C. Burges, and Q.~Wu, ``Mcrank: Learning to rank using multiple
  classification and gradient boosting,'' in \emph{Proceedings of NIPS}, 2007,
  pp. 897--904.

\bibitem{Shi:2010}
Y.~Shi, M.~Larson, and A.~Hanjalic, ``List-wise learning to rank with matrix
  factorization for collaborative filtering,'' in \emph{Proceedings of RecSys},
  2010, pp. 269--272.

\bibitem{Xu07}
J.~Xu and H.~Li, ``Adarank: a boosting algorithm for information retrieval,''
  in \emph{Proceedings of {SIGIR}}, 2007, pp. 391--398.

\bibitem{Xu08}
J.~Xu, T.~Liu, M.~Lu, H.~Li, and W.~Ma, ``Directly optimizing evaluation
  measures in learning to rank,'' in \emph{Proceedings of {SIGIR}}, 2008, pp.
  107--114.

\bibitem{Cohen99}
W.~W. Cohen, R.~E. Schapire, and Y.~Singer, ``Learning to order things,''
  \emph{J. Artif. Intell. Res. {(JAIR)}}, vol.~10, pp. 243--270, 1999.

\bibitem{Freund03}
Y.~Freund, R.~D. Iyer, R.~E. Schapire, and Y.~Singer, ``An efficient boosting
  algorithm for combining preferences,'' \emph{Journal of Machine Learning
  Research}, vol.~4, pp. 933--969, 2003.

\bibitem{Joachims02}
T.~Joachims, ``Optimizing search engines using clickthrough data,'' in
  \emph{Proceedings of {SIGKDD}}, 2002, pp. 133--142.

\bibitem{PessiotTUAG07}
J.~Pessiot, T.~Truong, N.~Usunier, M.~Amini, and P.~Gallinari, ``Learning to
  rank for collaborative filtering,'' in \emph{Proceedings of {ICEIS}}, 2007,
  pp. 145--151.

\bibitem{Caruana:1995}
R.~Caruana, S.~Baluja, and T.~M. Mitchell, ``Using the future to sort out the
  present: Rankprop and multitask learning for medical risk evaluation,'' in
  \emph{Proceedings of NIPS}, 1995, pp. 959--965.

\bibitem{DBLP:conf/icml/BurgesSRLDHH05}
C.~J.~C. Burges, T.~Shaked, E.~Renshaw, A.~Lazier, M.~Deeds, N.~Hamilton, and
  G.~N. Hullender, ``Learning to rank using gradient descent,'' in
  \emph{Proceedings of ICML}, 2005, pp. 89--96.

\bibitem{DBLP:conf/icann/RigutiniPMB08}
L.~Rigutini, T.~Papini, M.~Maggini, and M.~Bianchini, ``A neural network
  approach for learning object ranking,'' in \emph{Proceedings of {ICANN}},
  2008, pp. 899--908.

\bibitem{DBLP:journals/tnn/RigutiniPMS11}
L.~Rigutini, T.~Papini, M.~Maggini, and F.~Scarselli, ``Sortnet: Learning to
  rank by a neural preference function,'' \emph{{IEEE} Trans. Neural Networks},
  vol.~22, no.~9, pp. 1368--1380, 2011.

\bibitem{Harper:2015:MDH:2866565.2827872}
F.~M. Harper and J.~A. Konstan, ``The movielens datasets: History and
  context,'' \emph{ACM Trans. Interact. Intell. Syst.}, vol.~5, no.~4, pp.
  19:1--19:19, Dec. 2015.

\bibitem{DBLP:conf/sigir/sidana17}
S.~Sidana, C.~Laclau, M.-R. Amini, G.~Vandelle, and A.~Bois-Crettez, ``Kasandr:
  A large-scale dataset with implicit feedback for recommendation,'' in
  \emph{Proceedings of {SIGIR}}, 2017.

\bibitem{Hu:2008}
Y.~Hu, Y.~Koren, and C.~Volinsky, ``Collaborative filtering for implicit
  feedback datasets,'' in \emph{Proceedings of {ICDM}}, 2008, pp. 263--272.

\bibitem{kula_15}
M.~Kula, ``Metadata embeddings for user and item cold-start recommendations,''
  in \emph{Proceedings of the 2nd Workshop on New Trends on Content-Based
  Recommender Systems co-located with RecSys.}, 2015, pp. 14--21.

\bibitem{KingmaB14}
D.~P. Kingma and J.~Ba, ``Adam: {A} method for stochastic optimization,''
  \emph{CoRR}, vol. abs/1412.6980, 2014.

\bibitem{lehmann_06}
E.~Lehmann and H.~D'Abrera, \emph{Nonparametrics: statistical methods based on
  ranks}.\hskip 1em plus 0.5em minus 0.4em\relax Springer, 2006.

\end{thebibliography}
%\input{_tkde.bbl}
%\received{Month Year}{Month Year}{Month Year}

% references section

% can use a bibliography generated by BibTeX as a .bbl file
% BibTeX documentation can be easily obtained at:
% http://mirror.ctan.org/biblio/bibtex/contrib/doc/
% The IEEEtran BibTeX style support page is at:
% http://www.michaelshell.org/tex/ieeetran/bibtex/
%\bibliographystyle{IEEEtran}
% argument is your BibTeX string definitions and bibliography database(s)
%\bibliography{IEEEabrv,../bib/paper}
%
% <OR> manually copy in the resultant .bbl file
% set second argument of \begin to the number of references
% (used to reserve space for the reference number labels box)
%\begin{thebibliography}{1}
%
%\bibitem{IEEEhowto:kopka}
%H.~Kopka and P.~W. Daly, \emph{A Guide to \LaTeX}, 3rd~ed.\hskip 1em plus
%  0.5em minus 0.4em\relax Harlow, England: Addison-Wesley, 1999.

%\end{thebibliography}

% biography section
%
% If you have an EPS/PDF photo (graphicx package needed) extra braces are
% needed around the contents of the optional argument to biography to prevent
% the LaTeX parser from getting confused when it sees the complicated
% \includegraphics command within an optional argument. (You could create
% your own custom macro containing the \includegraphics command to make things
% simpler here.)
%\begin{IEEEbiography}[{\includegraphics[width=1in,height=1.25in,clip,keepaspectratio]{mshell}}]{Michael Shell}
% or if you just want to reserve a space for a photo:

%\begin{IEEEbiography}{Michael Shell}
%Biography text here.
%\end{IEEEbiography}
%
%% if you will not have a photo at all:

\begin{IEEEbiographynophoto}{Sumit Sidana}
is a PhD student at Grenoble Alpes University, Grenoble. He received a bachelor’s degree in Information Technology from SVIET, India and a master’s degree in computer science from IIIT, Hyderabad. His research interests include machine learning, recommender systems, probabilistic graphical models and deep learning.
\end{IEEEbiographynophoto}
\begin{IEEEbiographynophoto}{Mikhail Trofimov}
is a PhD student at Federal Research Center «Computer Science and Control» of Russian Academy of Sciences, Moscow. He received a bachelor's degree in applied math and physics and master's degrees in intelligent data analysis from Moscow Institute Of Physics and Technology. His research interests include machine learning on sparse data, tensor approximation methods and counterfactual learning.
\end{IEEEbiographynophoto}
\begin{IEEEbiographynophoto}{Oleg Gorodnitskii}
is a MSc student at Skoltech Institute of Science and Technology. He received a bachelor's degree in applied math and physics from Moscow Institute Of Physics and Technology. His research interests include machine learning, recommender systems, optimization methods and deep learning.
\end{IEEEbiographynophoto}
\begin{IEEEbiographynophoto}{Charlotte Laclau} received the PhD degree from the University of Paris Descartes in 2016. She is a a postdoctoral researcher in the Machine Learning group in the University of Grenoble Alpes. Her research interests include statistical machine learning, data mining and particularly unsupervised learning in information retrieval.
\end{IEEEbiographynophoto}
\begin{IEEEbiographynophoto}{Yury Maximov}
is a postdoc at CNLS and the Theoretical division of Los Alamos National Laboratory, and Assistant Professor of the Center of Energy Systems at Skolkovo Institute of Science and Technology. His research is in optimization methods and machine learning theory.
\end{IEEEbiographynophoto}
\begin{IEEEbiographynophoto} {Massih-R\'eza Amini}
 is a Professor in the University of Grenoble Alpes and head of the Machine Learning group. His  research is in statistical machine learning and he has contributed in developing machine learning techniques for information retrieval and text mining.
\end{IEEEbiographynophoto}

%% insert where needed to balance the two columns on the last page with
%% biographies
%%\newpage
%
%\begin{IEEEbiography}{Jane Doe}
%Biography text here.
%\end{IEEEbiography}

% You can push biographies down or up by placing
% a \vfill before or after them. The appropriate
% use of \vfill depends on what kind of text is
% on the last page and whether or not the columns
% are being equalized.

%\vfill

% Can be used to pull up biographies so that the bottom of the last one
% is flush with the other column.
%\enlargethispage{-5in}



% that's all folks
%\end{document}
\end{sloppypar}
\end{document}
